All of the computations in this thesis were performed with the following computational setup, unless otherwise stated.

\paragraph{PDE/FEM:}
The PDE we solve is the Interior Impedance Problem, i.e., \cref{prob:vtedp} with $\Dm = \emptyset,$ posed on the 2-d unit square; $D = [0,1]^2.$ We use first-order continuous finite elements, with $h = k^{-3/2}$. We use regular grids, see \cref{fig:grid} for an example grid. Where we needed to calculate a preconditioner $\AmatoI,$ we calculated the exact $LU$ decomposition of $\Amato$.

\begin{figure}[h]
  \centering
  %% Creator: Matplotlib, PGF backend
%%
%% To include the figure in your LaTeX document, write
%%   \input{<filename>.pgf}
%%
%% Make sure the required packages are loaded in your preamble
%%   \usepackage{pgf}
%%
%% Figures using additional raster images can only be included by \input if
%% they are in the same directory as the main LaTeX file. For loading figures
%% from other directories you can use the `import` package
%%   \usepackage{import}
%% and then include the figures with
%%   \import{<path to file>}{<filename>.pgf}
%%
%% Matplotlib used the following preamble
%%   \usepackage{fontspec}
%%   \setmainfont{DejaVuSerif.ttf}[Path=/home/owen/progs/firedrake-complex/firedrake/lib/python3.5/site-packages/matplotlib/mpl-data/fonts/ttf/]
%%   \setsansfont{DejaVuSans.ttf}[Path=/home/owen/progs/firedrake-complex/firedrake/lib/python3.5/site-packages/matplotlib/mpl-data/fonts/ttf/]
%%   \setmonofont{DejaVuSansMono.ttf}[Path=/home/owen/progs/firedrake-complex/firedrake/lib/python3.5/site-packages/matplotlib/mpl-data/fonts/ttf/]
%%
\begingroup%
\makeatletter%
\begin{pgfpicture}%
\pgfpathrectangle{\pgfpointorigin}{\pgfqpoint{6.400000in}{4.800000in}}%
\pgfusepath{use as bounding box, clip}%
\begin{pgfscope}%
\pgfsetbuttcap%
\pgfsetmiterjoin%
\definecolor{currentfill}{rgb}{1.000000,1.000000,1.000000}%
\pgfsetfillcolor{currentfill}%
\pgfsetlinewidth{0.000000pt}%
\definecolor{currentstroke}{rgb}{1.000000,1.000000,1.000000}%
\pgfsetstrokecolor{currentstroke}%
\pgfsetdash{}{0pt}%
\pgfpathmoveto{\pgfqpoint{0.000000in}{0.000000in}}%
\pgfpathlineto{\pgfqpoint{6.400000in}{0.000000in}}%
\pgfpathlineto{\pgfqpoint{6.400000in}{4.800000in}}%
\pgfpathlineto{\pgfqpoint{0.000000in}{4.800000in}}%
\pgfpathclose%
\pgfusepath{fill}%
\end{pgfscope}%
\begin{pgfscope}%
\pgfsetbuttcap%
\pgfsetmiterjoin%
\definecolor{currentfill}{rgb}{1.000000,1.000000,1.000000}%
\pgfsetfillcolor{currentfill}%
\pgfsetlinewidth{0.000000pt}%
\definecolor{currentstroke}{rgb}{0.000000,0.000000,0.000000}%
\pgfsetstrokecolor{currentstroke}%
\pgfsetstrokeopacity{0.000000}%
\pgfsetdash{}{0pt}%
\pgfpathmoveto{\pgfqpoint{1.432000in}{0.528000in}}%
\pgfpathlineto{\pgfqpoint{5.128000in}{0.528000in}}%
\pgfpathlineto{\pgfqpoint{5.128000in}{4.224000in}}%
\pgfpathlineto{\pgfqpoint{1.432000in}{4.224000in}}%
\pgfpathclose%
\pgfusepath{fill}%
\end{pgfscope}%
\begin{pgfscope}%
\pgfpathrectangle{\pgfqpoint{1.432000in}{0.528000in}}{\pgfqpoint{3.696000in}{3.696000in}}%
\pgfusepath{clip}%
\pgfsetbuttcap%
\pgfsetroundjoin%
\definecolor{currentfill}{rgb}{0.000000,0.000000,0.000000}%
\pgfsetfillcolor{currentfill}%
\pgfsetlinewidth{1.003750pt}%
\definecolor{currentstroke}{rgb}{0.000000,0.000000,0.000000}%
\pgfsetstrokecolor{currentstroke}%
\pgfsetdash{}{0pt}%
\pgfpathmoveto{\pgfqpoint{1.600000in}{0.671220in}}%
\pgfpathcurveto{\pgfqpoint{1.606572in}{0.671220in}}{\pgfqpoint{1.612875in}{0.673831in}}{\pgfqpoint{1.617522in}{0.678478in}}%
\pgfpathcurveto{\pgfqpoint{1.622169in}{0.683125in}}{\pgfqpoint{1.624780in}{0.689428in}}{\pgfqpoint{1.624780in}{0.696000in}}%
\pgfpathcurveto{\pgfqpoint{1.624780in}{0.702572in}}{\pgfqpoint{1.622169in}{0.708875in}}{\pgfqpoint{1.617522in}{0.713522in}}%
\pgfpathcurveto{\pgfqpoint{1.612875in}{0.718169in}}{\pgfqpoint{1.606572in}{0.720780in}}{\pgfqpoint{1.600000in}{0.720780in}}%
\pgfpathcurveto{\pgfqpoint{1.593428in}{0.720780in}}{\pgfqpoint{1.587125in}{0.718169in}}{\pgfqpoint{1.582478in}{0.713522in}}%
\pgfpathcurveto{\pgfqpoint{1.577831in}{0.708875in}}{\pgfqpoint{1.575220in}{0.702572in}}{\pgfqpoint{1.575220in}{0.696000in}}%
\pgfpathcurveto{\pgfqpoint{1.575220in}{0.689428in}}{\pgfqpoint{1.577831in}{0.683125in}}{\pgfqpoint{1.582478in}{0.678478in}}%
\pgfpathcurveto{\pgfqpoint{1.587125in}{0.673831in}}{\pgfqpoint{1.593428in}{0.671220in}}{\pgfqpoint{1.600000in}{0.671220in}}%
\pgfpathclose%
\pgfusepath{stroke,fill}%
\end{pgfscope}%
\begin{pgfscope}%
\pgfpathrectangle{\pgfqpoint{1.432000in}{0.528000in}}{\pgfqpoint{3.696000in}{3.696000in}}%
\pgfusepath{clip}%
\pgfsetbuttcap%
\pgfsetroundjoin%
\definecolor{currentfill}{rgb}{0.000000,0.000000,0.000000}%
\pgfsetfillcolor{currentfill}%
\pgfsetlinewidth{1.003750pt}%
\definecolor{currentstroke}{rgb}{0.000000,0.000000,0.000000}%
\pgfsetstrokecolor{currentstroke}%
\pgfsetdash{}{0pt}%
\pgfpathmoveto{\pgfqpoint{1.600000in}{1.007220in}}%
\pgfpathcurveto{\pgfqpoint{1.606572in}{1.007220in}}{\pgfqpoint{1.612875in}{1.009831in}}{\pgfqpoint{1.617522in}{1.014478in}}%
\pgfpathcurveto{\pgfqpoint{1.622169in}{1.019125in}}{\pgfqpoint{1.624780in}{1.025428in}}{\pgfqpoint{1.624780in}{1.032000in}}%
\pgfpathcurveto{\pgfqpoint{1.624780in}{1.038572in}}{\pgfqpoint{1.622169in}{1.044875in}}{\pgfqpoint{1.617522in}{1.049522in}}%
\pgfpathcurveto{\pgfqpoint{1.612875in}{1.054169in}}{\pgfqpoint{1.606572in}{1.056780in}}{\pgfqpoint{1.600000in}{1.056780in}}%
\pgfpathcurveto{\pgfqpoint{1.593428in}{1.056780in}}{\pgfqpoint{1.587125in}{1.054169in}}{\pgfqpoint{1.582478in}{1.049522in}}%
\pgfpathcurveto{\pgfqpoint{1.577831in}{1.044875in}}{\pgfqpoint{1.575220in}{1.038572in}}{\pgfqpoint{1.575220in}{1.032000in}}%
\pgfpathcurveto{\pgfqpoint{1.575220in}{1.025428in}}{\pgfqpoint{1.577831in}{1.019125in}}{\pgfqpoint{1.582478in}{1.014478in}}%
\pgfpathcurveto{\pgfqpoint{1.587125in}{1.009831in}}{\pgfqpoint{1.593428in}{1.007220in}}{\pgfqpoint{1.600000in}{1.007220in}}%
\pgfpathclose%
\pgfusepath{stroke,fill}%
\end{pgfscope}%
\begin{pgfscope}%
\pgfpathrectangle{\pgfqpoint{1.432000in}{0.528000in}}{\pgfqpoint{3.696000in}{3.696000in}}%
\pgfusepath{clip}%
\pgfsetbuttcap%
\pgfsetroundjoin%
\definecolor{currentfill}{rgb}{0.000000,0.000000,0.000000}%
\pgfsetfillcolor{currentfill}%
\pgfsetlinewidth{1.003750pt}%
\definecolor{currentstroke}{rgb}{0.000000,0.000000,0.000000}%
\pgfsetstrokecolor{currentstroke}%
\pgfsetdash{}{0pt}%
\pgfpathmoveto{\pgfqpoint{1.600000in}{1.343220in}}%
\pgfpathcurveto{\pgfqpoint{1.606572in}{1.343220in}}{\pgfqpoint{1.612875in}{1.345831in}}{\pgfqpoint{1.617522in}{1.350478in}}%
\pgfpathcurveto{\pgfqpoint{1.622169in}{1.355125in}}{\pgfqpoint{1.624780in}{1.361428in}}{\pgfqpoint{1.624780in}{1.368000in}}%
\pgfpathcurveto{\pgfqpoint{1.624780in}{1.374572in}}{\pgfqpoint{1.622169in}{1.380875in}}{\pgfqpoint{1.617522in}{1.385522in}}%
\pgfpathcurveto{\pgfqpoint{1.612875in}{1.390169in}}{\pgfqpoint{1.606572in}{1.392780in}}{\pgfqpoint{1.600000in}{1.392780in}}%
\pgfpathcurveto{\pgfqpoint{1.593428in}{1.392780in}}{\pgfqpoint{1.587125in}{1.390169in}}{\pgfqpoint{1.582478in}{1.385522in}}%
\pgfpathcurveto{\pgfqpoint{1.577831in}{1.380875in}}{\pgfqpoint{1.575220in}{1.374572in}}{\pgfqpoint{1.575220in}{1.368000in}}%
\pgfpathcurveto{\pgfqpoint{1.575220in}{1.361428in}}{\pgfqpoint{1.577831in}{1.355125in}}{\pgfqpoint{1.582478in}{1.350478in}}%
\pgfpathcurveto{\pgfqpoint{1.587125in}{1.345831in}}{\pgfqpoint{1.593428in}{1.343220in}}{\pgfqpoint{1.600000in}{1.343220in}}%
\pgfpathclose%
\pgfusepath{stroke,fill}%
\end{pgfscope}%
\begin{pgfscope}%
\pgfpathrectangle{\pgfqpoint{1.432000in}{0.528000in}}{\pgfqpoint{3.696000in}{3.696000in}}%
\pgfusepath{clip}%
\pgfsetbuttcap%
\pgfsetroundjoin%
\definecolor{currentfill}{rgb}{0.000000,0.000000,0.000000}%
\pgfsetfillcolor{currentfill}%
\pgfsetlinewidth{1.003750pt}%
\definecolor{currentstroke}{rgb}{0.000000,0.000000,0.000000}%
\pgfsetstrokecolor{currentstroke}%
\pgfsetdash{}{0pt}%
\pgfpathmoveto{\pgfqpoint{1.600000in}{1.679220in}}%
\pgfpathcurveto{\pgfqpoint{1.606572in}{1.679220in}}{\pgfqpoint{1.612875in}{1.681831in}}{\pgfqpoint{1.617522in}{1.686478in}}%
\pgfpathcurveto{\pgfqpoint{1.622169in}{1.691125in}}{\pgfqpoint{1.624780in}{1.697428in}}{\pgfqpoint{1.624780in}{1.704000in}}%
\pgfpathcurveto{\pgfqpoint{1.624780in}{1.710572in}}{\pgfqpoint{1.622169in}{1.716875in}}{\pgfqpoint{1.617522in}{1.721522in}}%
\pgfpathcurveto{\pgfqpoint{1.612875in}{1.726169in}}{\pgfqpoint{1.606572in}{1.728780in}}{\pgfqpoint{1.600000in}{1.728780in}}%
\pgfpathcurveto{\pgfqpoint{1.593428in}{1.728780in}}{\pgfqpoint{1.587125in}{1.726169in}}{\pgfqpoint{1.582478in}{1.721522in}}%
\pgfpathcurveto{\pgfqpoint{1.577831in}{1.716875in}}{\pgfqpoint{1.575220in}{1.710572in}}{\pgfqpoint{1.575220in}{1.704000in}}%
\pgfpathcurveto{\pgfqpoint{1.575220in}{1.697428in}}{\pgfqpoint{1.577831in}{1.691125in}}{\pgfqpoint{1.582478in}{1.686478in}}%
\pgfpathcurveto{\pgfqpoint{1.587125in}{1.681831in}}{\pgfqpoint{1.593428in}{1.679220in}}{\pgfqpoint{1.600000in}{1.679220in}}%
\pgfpathclose%
\pgfusepath{stroke,fill}%
\end{pgfscope}%
\begin{pgfscope}%
\pgfpathrectangle{\pgfqpoint{1.432000in}{0.528000in}}{\pgfqpoint{3.696000in}{3.696000in}}%
\pgfusepath{clip}%
\pgfsetbuttcap%
\pgfsetroundjoin%
\definecolor{currentfill}{rgb}{0.000000,0.000000,0.000000}%
\pgfsetfillcolor{currentfill}%
\pgfsetlinewidth{1.003750pt}%
\definecolor{currentstroke}{rgb}{0.000000,0.000000,0.000000}%
\pgfsetstrokecolor{currentstroke}%
\pgfsetdash{}{0pt}%
\pgfpathmoveto{\pgfqpoint{1.600000in}{2.015220in}}%
\pgfpathcurveto{\pgfqpoint{1.606572in}{2.015220in}}{\pgfqpoint{1.612875in}{2.017831in}}{\pgfqpoint{1.617522in}{2.022478in}}%
\pgfpathcurveto{\pgfqpoint{1.622169in}{2.027125in}}{\pgfqpoint{1.624780in}{2.033428in}}{\pgfqpoint{1.624780in}{2.040000in}}%
\pgfpathcurveto{\pgfqpoint{1.624780in}{2.046572in}}{\pgfqpoint{1.622169in}{2.052875in}}{\pgfqpoint{1.617522in}{2.057522in}}%
\pgfpathcurveto{\pgfqpoint{1.612875in}{2.062169in}}{\pgfqpoint{1.606572in}{2.064780in}}{\pgfqpoint{1.600000in}{2.064780in}}%
\pgfpathcurveto{\pgfqpoint{1.593428in}{2.064780in}}{\pgfqpoint{1.587125in}{2.062169in}}{\pgfqpoint{1.582478in}{2.057522in}}%
\pgfpathcurveto{\pgfqpoint{1.577831in}{2.052875in}}{\pgfqpoint{1.575220in}{2.046572in}}{\pgfqpoint{1.575220in}{2.040000in}}%
\pgfpathcurveto{\pgfqpoint{1.575220in}{2.033428in}}{\pgfqpoint{1.577831in}{2.027125in}}{\pgfqpoint{1.582478in}{2.022478in}}%
\pgfpathcurveto{\pgfqpoint{1.587125in}{2.017831in}}{\pgfqpoint{1.593428in}{2.015220in}}{\pgfqpoint{1.600000in}{2.015220in}}%
\pgfpathclose%
\pgfusepath{stroke,fill}%
\end{pgfscope}%
\begin{pgfscope}%
\pgfpathrectangle{\pgfqpoint{1.432000in}{0.528000in}}{\pgfqpoint{3.696000in}{3.696000in}}%
\pgfusepath{clip}%
\pgfsetbuttcap%
\pgfsetroundjoin%
\definecolor{currentfill}{rgb}{0.000000,0.000000,0.000000}%
\pgfsetfillcolor{currentfill}%
\pgfsetlinewidth{1.003750pt}%
\definecolor{currentstroke}{rgb}{0.000000,0.000000,0.000000}%
\pgfsetstrokecolor{currentstroke}%
\pgfsetdash{}{0pt}%
\pgfpathmoveto{\pgfqpoint{1.600000in}{2.351220in}}%
\pgfpathcurveto{\pgfqpoint{1.606572in}{2.351220in}}{\pgfqpoint{1.612875in}{2.353831in}}{\pgfqpoint{1.617522in}{2.358478in}}%
\pgfpathcurveto{\pgfqpoint{1.622169in}{2.363125in}}{\pgfqpoint{1.624780in}{2.369428in}}{\pgfqpoint{1.624780in}{2.376000in}}%
\pgfpathcurveto{\pgfqpoint{1.624780in}{2.382572in}}{\pgfqpoint{1.622169in}{2.388875in}}{\pgfqpoint{1.617522in}{2.393522in}}%
\pgfpathcurveto{\pgfqpoint{1.612875in}{2.398169in}}{\pgfqpoint{1.606572in}{2.400780in}}{\pgfqpoint{1.600000in}{2.400780in}}%
\pgfpathcurveto{\pgfqpoint{1.593428in}{2.400780in}}{\pgfqpoint{1.587125in}{2.398169in}}{\pgfqpoint{1.582478in}{2.393522in}}%
\pgfpathcurveto{\pgfqpoint{1.577831in}{2.388875in}}{\pgfqpoint{1.575220in}{2.382572in}}{\pgfqpoint{1.575220in}{2.376000in}}%
\pgfpathcurveto{\pgfqpoint{1.575220in}{2.369428in}}{\pgfqpoint{1.577831in}{2.363125in}}{\pgfqpoint{1.582478in}{2.358478in}}%
\pgfpathcurveto{\pgfqpoint{1.587125in}{2.353831in}}{\pgfqpoint{1.593428in}{2.351220in}}{\pgfqpoint{1.600000in}{2.351220in}}%
\pgfpathclose%
\pgfusepath{stroke,fill}%
\end{pgfscope}%
\begin{pgfscope}%
\pgfpathrectangle{\pgfqpoint{1.432000in}{0.528000in}}{\pgfqpoint{3.696000in}{3.696000in}}%
\pgfusepath{clip}%
\pgfsetbuttcap%
\pgfsetroundjoin%
\definecolor{currentfill}{rgb}{0.000000,0.000000,0.000000}%
\pgfsetfillcolor{currentfill}%
\pgfsetlinewidth{1.003750pt}%
\definecolor{currentstroke}{rgb}{0.000000,0.000000,0.000000}%
\pgfsetstrokecolor{currentstroke}%
\pgfsetdash{}{0pt}%
\pgfpathmoveto{\pgfqpoint{1.600000in}{2.687220in}}%
\pgfpathcurveto{\pgfqpoint{1.606572in}{2.687220in}}{\pgfqpoint{1.612875in}{2.689831in}}{\pgfqpoint{1.617522in}{2.694478in}}%
\pgfpathcurveto{\pgfqpoint{1.622169in}{2.699125in}}{\pgfqpoint{1.624780in}{2.705428in}}{\pgfqpoint{1.624780in}{2.712000in}}%
\pgfpathcurveto{\pgfqpoint{1.624780in}{2.718572in}}{\pgfqpoint{1.622169in}{2.724875in}}{\pgfqpoint{1.617522in}{2.729522in}}%
\pgfpathcurveto{\pgfqpoint{1.612875in}{2.734169in}}{\pgfqpoint{1.606572in}{2.736780in}}{\pgfqpoint{1.600000in}{2.736780in}}%
\pgfpathcurveto{\pgfqpoint{1.593428in}{2.736780in}}{\pgfqpoint{1.587125in}{2.734169in}}{\pgfqpoint{1.582478in}{2.729522in}}%
\pgfpathcurveto{\pgfqpoint{1.577831in}{2.724875in}}{\pgfqpoint{1.575220in}{2.718572in}}{\pgfqpoint{1.575220in}{2.712000in}}%
\pgfpathcurveto{\pgfqpoint{1.575220in}{2.705428in}}{\pgfqpoint{1.577831in}{2.699125in}}{\pgfqpoint{1.582478in}{2.694478in}}%
\pgfpathcurveto{\pgfqpoint{1.587125in}{2.689831in}}{\pgfqpoint{1.593428in}{2.687220in}}{\pgfqpoint{1.600000in}{2.687220in}}%
\pgfpathclose%
\pgfusepath{stroke,fill}%
\end{pgfscope}%
\begin{pgfscope}%
\pgfpathrectangle{\pgfqpoint{1.432000in}{0.528000in}}{\pgfqpoint{3.696000in}{3.696000in}}%
\pgfusepath{clip}%
\pgfsetbuttcap%
\pgfsetroundjoin%
\definecolor{currentfill}{rgb}{0.000000,0.000000,0.000000}%
\pgfsetfillcolor{currentfill}%
\pgfsetlinewidth{1.003750pt}%
\definecolor{currentstroke}{rgb}{0.000000,0.000000,0.000000}%
\pgfsetstrokecolor{currentstroke}%
\pgfsetdash{}{0pt}%
\pgfpathmoveto{\pgfqpoint{1.600000in}{3.023220in}}%
\pgfpathcurveto{\pgfqpoint{1.606572in}{3.023220in}}{\pgfqpoint{1.612875in}{3.025831in}}{\pgfqpoint{1.617522in}{3.030478in}}%
\pgfpathcurveto{\pgfqpoint{1.622169in}{3.035125in}}{\pgfqpoint{1.624780in}{3.041428in}}{\pgfqpoint{1.624780in}{3.048000in}}%
\pgfpathcurveto{\pgfqpoint{1.624780in}{3.054572in}}{\pgfqpoint{1.622169in}{3.060875in}}{\pgfqpoint{1.617522in}{3.065522in}}%
\pgfpathcurveto{\pgfqpoint{1.612875in}{3.070169in}}{\pgfqpoint{1.606572in}{3.072780in}}{\pgfqpoint{1.600000in}{3.072780in}}%
\pgfpathcurveto{\pgfqpoint{1.593428in}{3.072780in}}{\pgfqpoint{1.587125in}{3.070169in}}{\pgfqpoint{1.582478in}{3.065522in}}%
\pgfpathcurveto{\pgfqpoint{1.577831in}{3.060875in}}{\pgfqpoint{1.575220in}{3.054572in}}{\pgfqpoint{1.575220in}{3.048000in}}%
\pgfpathcurveto{\pgfqpoint{1.575220in}{3.041428in}}{\pgfqpoint{1.577831in}{3.035125in}}{\pgfqpoint{1.582478in}{3.030478in}}%
\pgfpathcurveto{\pgfqpoint{1.587125in}{3.025831in}}{\pgfqpoint{1.593428in}{3.023220in}}{\pgfqpoint{1.600000in}{3.023220in}}%
\pgfpathclose%
\pgfusepath{stroke,fill}%
\end{pgfscope}%
\begin{pgfscope}%
\pgfpathrectangle{\pgfqpoint{1.432000in}{0.528000in}}{\pgfqpoint{3.696000in}{3.696000in}}%
\pgfusepath{clip}%
\pgfsetbuttcap%
\pgfsetroundjoin%
\definecolor{currentfill}{rgb}{0.000000,0.000000,0.000000}%
\pgfsetfillcolor{currentfill}%
\pgfsetlinewidth{1.003750pt}%
\definecolor{currentstroke}{rgb}{0.000000,0.000000,0.000000}%
\pgfsetstrokecolor{currentstroke}%
\pgfsetdash{}{0pt}%
\pgfpathmoveto{\pgfqpoint{1.600000in}{3.359220in}}%
\pgfpathcurveto{\pgfqpoint{1.606572in}{3.359220in}}{\pgfqpoint{1.612875in}{3.361831in}}{\pgfqpoint{1.617522in}{3.366478in}}%
\pgfpathcurveto{\pgfqpoint{1.622169in}{3.371125in}}{\pgfqpoint{1.624780in}{3.377428in}}{\pgfqpoint{1.624780in}{3.384000in}}%
\pgfpathcurveto{\pgfqpoint{1.624780in}{3.390572in}}{\pgfqpoint{1.622169in}{3.396875in}}{\pgfqpoint{1.617522in}{3.401522in}}%
\pgfpathcurveto{\pgfqpoint{1.612875in}{3.406169in}}{\pgfqpoint{1.606572in}{3.408780in}}{\pgfqpoint{1.600000in}{3.408780in}}%
\pgfpathcurveto{\pgfqpoint{1.593428in}{3.408780in}}{\pgfqpoint{1.587125in}{3.406169in}}{\pgfqpoint{1.582478in}{3.401522in}}%
\pgfpathcurveto{\pgfqpoint{1.577831in}{3.396875in}}{\pgfqpoint{1.575220in}{3.390572in}}{\pgfqpoint{1.575220in}{3.384000in}}%
\pgfpathcurveto{\pgfqpoint{1.575220in}{3.377428in}}{\pgfqpoint{1.577831in}{3.371125in}}{\pgfqpoint{1.582478in}{3.366478in}}%
\pgfpathcurveto{\pgfqpoint{1.587125in}{3.361831in}}{\pgfqpoint{1.593428in}{3.359220in}}{\pgfqpoint{1.600000in}{3.359220in}}%
\pgfpathclose%
\pgfusepath{stroke,fill}%
\end{pgfscope}%
\begin{pgfscope}%
\pgfpathrectangle{\pgfqpoint{1.432000in}{0.528000in}}{\pgfqpoint{3.696000in}{3.696000in}}%
\pgfusepath{clip}%
\pgfsetbuttcap%
\pgfsetroundjoin%
\definecolor{currentfill}{rgb}{0.000000,0.000000,0.000000}%
\pgfsetfillcolor{currentfill}%
\pgfsetlinewidth{1.003750pt}%
\definecolor{currentstroke}{rgb}{0.000000,0.000000,0.000000}%
\pgfsetstrokecolor{currentstroke}%
\pgfsetdash{}{0pt}%
\pgfpathmoveto{\pgfqpoint{1.600000in}{3.695220in}}%
\pgfpathcurveto{\pgfqpoint{1.606572in}{3.695220in}}{\pgfqpoint{1.612875in}{3.697831in}}{\pgfqpoint{1.617522in}{3.702478in}}%
\pgfpathcurveto{\pgfqpoint{1.622169in}{3.707125in}}{\pgfqpoint{1.624780in}{3.713428in}}{\pgfqpoint{1.624780in}{3.720000in}}%
\pgfpathcurveto{\pgfqpoint{1.624780in}{3.726572in}}{\pgfqpoint{1.622169in}{3.732875in}}{\pgfqpoint{1.617522in}{3.737522in}}%
\pgfpathcurveto{\pgfqpoint{1.612875in}{3.742169in}}{\pgfqpoint{1.606572in}{3.744780in}}{\pgfqpoint{1.600000in}{3.744780in}}%
\pgfpathcurveto{\pgfqpoint{1.593428in}{3.744780in}}{\pgfqpoint{1.587125in}{3.742169in}}{\pgfqpoint{1.582478in}{3.737522in}}%
\pgfpathcurveto{\pgfqpoint{1.577831in}{3.732875in}}{\pgfqpoint{1.575220in}{3.726572in}}{\pgfqpoint{1.575220in}{3.720000in}}%
\pgfpathcurveto{\pgfqpoint{1.575220in}{3.713428in}}{\pgfqpoint{1.577831in}{3.707125in}}{\pgfqpoint{1.582478in}{3.702478in}}%
\pgfpathcurveto{\pgfqpoint{1.587125in}{3.697831in}}{\pgfqpoint{1.593428in}{3.695220in}}{\pgfqpoint{1.600000in}{3.695220in}}%
\pgfpathclose%
\pgfusepath{stroke,fill}%
\end{pgfscope}%
\begin{pgfscope}%
\pgfpathrectangle{\pgfqpoint{1.432000in}{0.528000in}}{\pgfqpoint{3.696000in}{3.696000in}}%
\pgfusepath{clip}%
\pgfsetbuttcap%
\pgfsetroundjoin%
\definecolor{currentfill}{rgb}{0.000000,0.000000,0.000000}%
\pgfsetfillcolor{currentfill}%
\pgfsetlinewidth{1.003750pt}%
\definecolor{currentstroke}{rgb}{0.000000,0.000000,0.000000}%
\pgfsetstrokecolor{currentstroke}%
\pgfsetdash{}{0pt}%
\pgfpathmoveto{\pgfqpoint{1.600000in}{4.031220in}}%
\pgfpathcurveto{\pgfqpoint{1.606572in}{4.031220in}}{\pgfqpoint{1.612875in}{4.033831in}}{\pgfqpoint{1.617522in}{4.038478in}}%
\pgfpathcurveto{\pgfqpoint{1.622169in}{4.043125in}}{\pgfqpoint{1.624780in}{4.049428in}}{\pgfqpoint{1.624780in}{4.056000in}}%
\pgfpathcurveto{\pgfqpoint{1.624780in}{4.062572in}}{\pgfqpoint{1.622169in}{4.068875in}}{\pgfqpoint{1.617522in}{4.073522in}}%
\pgfpathcurveto{\pgfqpoint{1.612875in}{4.078169in}}{\pgfqpoint{1.606572in}{4.080780in}}{\pgfqpoint{1.600000in}{4.080780in}}%
\pgfpathcurveto{\pgfqpoint{1.593428in}{4.080780in}}{\pgfqpoint{1.587125in}{4.078169in}}{\pgfqpoint{1.582478in}{4.073522in}}%
\pgfpathcurveto{\pgfqpoint{1.577831in}{4.068875in}}{\pgfqpoint{1.575220in}{4.062572in}}{\pgfqpoint{1.575220in}{4.056000in}}%
\pgfpathcurveto{\pgfqpoint{1.575220in}{4.049428in}}{\pgfqpoint{1.577831in}{4.043125in}}{\pgfqpoint{1.582478in}{4.038478in}}%
\pgfpathcurveto{\pgfqpoint{1.587125in}{4.033831in}}{\pgfqpoint{1.593428in}{4.031220in}}{\pgfqpoint{1.600000in}{4.031220in}}%
\pgfpathclose%
\pgfusepath{stroke,fill}%
\end{pgfscope}%
\begin{pgfscope}%
\pgfpathrectangle{\pgfqpoint{1.432000in}{0.528000in}}{\pgfqpoint{3.696000in}{3.696000in}}%
\pgfusepath{clip}%
\pgfsetbuttcap%
\pgfsetroundjoin%
\definecolor{currentfill}{rgb}{0.000000,0.000000,0.000000}%
\pgfsetfillcolor{currentfill}%
\pgfsetlinewidth{1.003750pt}%
\definecolor{currentstroke}{rgb}{0.000000,0.000000,0.000000}%
\pgfsetstrokecolor{currentstroke}%
\pgfsetdash{}{0pt}%
\pgfpathmoveto{\pgfqpoint{1.936000in}{0.671220in}}%
\pgfpathcurveto{\pgfqpoint{1.942572in}{0.671220in}}{\pgfqpoint{1.948875in}{0.673831in}}{\pgfqpoint{1.953522in}{0.678478in}}%
\pgfpathcurveto{\pgfqpoint{1.958169in}{0.683125in}}{\pgfqpoint{1.960780in}{0.689428in}}{\pgfqpoint{1.960780in}{0.696000in}}%
\pgfpathcurveto{\pgfqpoint{1.960780in}{0.702572in}}{\pgfqpoint{1.958169in}{0.708875in}}{\pgfqpoint{1.953522in}{0.713522in}}%
\pgfpathcurveto{\pgfqpoint{1.948875in}{0.718169in}}{\pgfqpoint{1.942572in}{0.720780in}}{\pgfqpoint{1.936000in}{0.720780in}}%
\pgfpathcurveto{\pgfqpoint{1.929428in}{0.720780in}}{\pgfqpoint{1.923125in}{0.718169in}}{\pgfqpoint{1.918478in}{0.713522in}}%
\pgfpathcurveto{\pgfqpoint{1.913831in}{0.708875in}}{\pgfqpoint{1.911220in}{0.702572in}}{\pgfqpoint{1.911220in}{0.696000in}}%
\pgfpathcurveto{\pgfqpoint{1.911220in}{0.689428in}}{\pgfqpoint{1.913831in}{0.683125in}}{\pgfqpoint{1.918478in}{0.678478in}}%
\pgfpathcurveto{\pgfqpoint{1.923125in}{0.673831in}}{\pgfqpoint{1.929428in}{0.671220in}}{\pgfqpoint{1.936000in}{0.671220in}}%
\pgfpathclose%
\pgfusepath{stroke,fill}%
\end{pgfscope}%
\begin{pgfscope}%
\pgfpathrectangle{\pgfqpoint{1.432000in}{0.528000in}}{\pgfqpoint{3.696000in}{3.696000in}}%
\pgfusepath{clip}%
\pgfsetbuttcap%
\pgfsetroundjoin%
\definecolor{currentfill}{rgb}{0.000000,0.000000,0.000000}%
\pgfsetfillcolor{currentfill}%
\pgfsetlinewidth{1.003750pt}%
\definecolor{currentstroke}{rgb}{0.000000,0.000000,0.000000}%
\pgfsetstrokecolor{currentstroke}%
\pgfsetdash{}{0pt}%
\pgfpathmoveto{\pgfqpoint{1.936000in}{1.007220in}}%
\pgfpathcurveto{\pgfqpoint{1.942572in}{1.007220in}}{\pgfqpoint{1.948875in}{1.009831in}}{\pgfqpoint{1.953522in}{1.014478in}}%
\pgfpathcurveto{\pgfqpoint{1.958169in}{1.019125in}}{\pgfqpoint{1.960780in}{1.025428in}}{\pgfqpoint{1.960780in}{1.032000in}}%
\pgfpathcurveto{\pgfqpoint{1.960780in}{1.038572in}}{\pgfqpoint{1.958169in}{1.044875in}}{\pgfqpoint{1.953522in}{1.049522in}}%
\pgfpathcurveto{\pgfqpoint{1.948875in}{1.054169in}}{\pgfqpoint{1.942572in}{1.056780in}}{\pgfqpoint{1.936000in}{1.056780in}}%
\pgfpathcurveto{\pgfqpoint{1.929428in}{1.056780in}}{\pgfqpoint{1.923125in}{1.054169in}}{\pgfqpoint{1.918478in}{1.049522in}}%
\pgfpathcurveto{\pgfqpoint{1.913831in}{1.044875in}}{\pgfqpoint{1.911220in}{1.038572in}}{\pgfqpoint{1.911220in}{1.032000in}}%
\pgfpathcurveto{\pgfqpoint{1.911220in}{1.025428in}}{\pgfqpoint{1.913831in}{1.019125in}}{\pgfqpoint{1.918478in}{1.014478in}}%
\pgfpathcurveto{\pgfqpoint{1.923125in}{1.009831in}}{\pgfqpoint{1.929428in}{1.007220in}}{\pgfqpoint{1.936000in}{1.007220in}}%
\pgfpathclose%
\pgfusepath{stroke,fill}%
\end{pgfscope}%
\begin{pgfscope}%
\pgfpathrectangle{\pgfqpoint{1.432000in}{0.528000in}}{\pgfqpoint{3.696000in}{3.696000in}}%
\pgfusepath{clip}%
\pgfsetbuttcap%
\pgfsetroundjoin%
\definecolor{currentfill}{rgb}{0.000000,0.000000,0.000000}%
\pgfsetfillcolor{currentfill}%
\pgfsetlinewidth{1.003750pt}%
\definecolor{currentstroke}{rgb}{0.000000,0.000000,0.000000}%
\pgfsetstrokecolor{currentstroke}%
\pgfsetdash{}{0pt}%
\pgfpathmoveto{\pgfqpoint{1.936000in}{1.343220in}}%
\pgfpathcurveto{\pgfqpoint{1.942572in}{1.343220in}}{\pgfqpoint{1.948875in}{1.345831in}}{\pgfqpoint{1.953522in}{1.350478in}}%
\pgfpathcurveto{\pgfqpoint{1.958169in}{1.355125in}}{\pgfqpoint{1.960780in}{1.361428in}}{\pgfqpoint{1.960780in}{1.368000in}}%
\pgfpathcurveto{\pgfqpoint{1.960780in}{1.374572in}}{\pgfqpoint{1.958169in}{1.380875in}}{\pgfqpoint{1.953522in}{1.385522in}}%
\pgfpathcurveto{\pgfqpoint{1.948875in}{1.390169in}}{\pgfqpoint{1.942572in}{1.392780in}}{\pgfqpoint{1.936000in}{1.392780in}}%
\pgfpathcurveto{\pgfqpoint{1.929428in}{1.392780in}}{\pgfqpoint{1.923125in}{1.390169in}}{\pgfqpoint{1.918478in}{1.385522in}}%
\pgfpathcurveto{\pgfqpoint{1.913831in}{1.380875in}}{\pgfqpoint{1.911220in}{1.374572in}}{\pgfqpoint{1.911220in}{1.368000in}}%
\pgfpathcurveto{\pgfqpoint{1.911220in}{1.361428in}}{\pgfqpoint{1.913831in}{1.355125in}}{\pgfqpoint{1.918478in}{1.350478in}}%
\pgfpathcurveto{\pgfqpoint{1.923125in}{1.345831in}}{\pgfqpoint{1.929428in}{1.343220in}}{\pgfqpoint{1.936000in}{1.343220in}}%
\pgfpathclose%
\pgfusepath{stroke,fill}%
\end{pgfscope}%
\begin{pgfscope}%
\pgfpathrectangle{\pgfqpoint{1.432000in}{0.528000in}}{\pgfqpoint{3.696000in}{3.696000in}}%
\pgfusepath{clip}%
\pgfsetbuttcap%
\pgfsetroundjoin%
\definecolor{currentfill}{rgb}{0.000000,0.000000,0.000000}%
\pgfsetfillcolor{currentfill}%
\pgfsetlinewidth{1.003750pt}%
\definecolor{currentstroke}{rgb}{0.000000,0.000000,0.000000}%
\pgfsetstrokecolor{currentstroke}%
\pgfsetdash{}{0pt}%
\pgfpathmoveto{\pgfqpoint{1.936000in}{1.679220in}}%
\pgfpathcurveto{\pgfqpoint{1.942572in}{1.679220in}}{\pgfqpoint{1.948875in}{1.681831in}}{\pgfqpoint{1.953522in}{1.686478in}}%
\pgfpathcurveto{\pgfqpoint{1.958169in}{1.691125in}}{\pgfqpoint{1.960780in}{1.697428in}}{\pgfqpoint{1.960780in}{1.704000in}}%
\pgfpathcurveto{\pgfqpoint{1.960780in}{1.710572in}}{\pgfqpoint{1.958169in}{1.716875in}}{\pgfqpoint{1.953522in}{1.721522in}}%
\pgfpathcurveto{\pgfqpoint{1.948875in}{1.726169in}}{\pgfqpoint{1.942572in}{1.728780in}}{\pgfqpoint{1.936000in}{1.728780in}}%
\pgfpathcurveto{\pgfqpoint{1.929428in}{1.728780in}}{\pgfqpoint{1.923125in}{1.726169in}}{\pgfqpoint{1.918478in}{1.721522in}}%
\pgfpathcurveto{\pgfqpoint{1.913831in}{1.716875in}}{\pgfqpoint{1.911220in}{1.710572in}}{\pgfqpoint{1.911220in}{1.704000in}}%
\pgfpathcurveto{\pgfqpoint{1.911220in}{1.697428in}}{\pgfqpoint{1.913831in}{1.691125in}}{\pgfqpoint{1.918478in}{1.686478in}}%
\pgfpathcurveto{\pgfqpoint{1.923125in}{1.681831in}}{\pgfqpoint{1.929428in}{1.679220in}}{\pgfqpoint{1.936000in}{1.679220in}}%
\pgfpathclose%
\pgfusepath{stroke,fill}%
\end{pgfscope}%
\begin{pgfscope}%
\pgfpathrectangle{\pgfqpoint{1.432000in}{0.528000in}}{\pgfqpoint{3.696000in}{3.696000in}}%
\pgfusepath{clip}%
\pgfsetbuttcap%
\pgfsetroundjoin%
\definecolor{currentfill}{rgb}{0.000000,0.000000,0.000000}%
\pgfsetfillcolor{currentfill}%
\pgfsetlinewidth{1.003750pt}%
\definecolor{currentstroke}{rgb}{0.000000,0.000000,0.000000}%
\pgfsetstrokecolor{currentstroke}%
\pgfsetdash{}{0pt}%
\pgfpathmoveto{\pgfqpoint{1.936000in}{2.015220in}}%
\pgfpathcurveto{\pgfqpoint{1.942572in}{2.015220in}}{\pgfqpoint{1.948875in}{2.017831in}}{\pgfqpoint{1.953522in}{2.022478in}}%
\pgfpathcurveto{\pgfqpoint{1.958169in}{2.027125in}}{\pgfqpoint{1.960780in}{2.033428in}}{\pgfqpoint{1.960780in}{2.040000in}}%
\pgfpathcurveto{\pgfqpoint{1.960780in}{2.046572in}}{\pgfqpoint{1.958169in}{2.052875in}}{\pgfqpoint{1.953522in}{2.057522in}}%
\pgfpathcurveto{\pgfqpoint{1.948875in}{2.062169in}}{\pgfqpoint{1.942572in}{2.064780in}}{\pgfqpoint{1.936000in}{2.064780in}}%
\pgfpathcurveto{\pgfqpoint{1.929428in}{2.064780in}}{\pgfqpoint{1.923125in}{2.062169in}}{\pgfqpoint{1.918478in}{2.057522in}}%
\pgfpathcurveto{\pgfqpoint{1.913831in}{2.052875in}}{\pgfqpoint{1.911220in}{2.046572in}}{\pgfqpoint{1.911220in}{2.040000in}}%
\pgfpathcurveto{\pgfqpoint{1.911220in}{2.033428in}}{\pgfqpoint{1.913831in}{2.027125in}}{\pgfqpoint{1.918478in}{2.022478in}}%
\pgfpathcurveto{\pgfqpoint{1.923125in}{2.017831in}}{\pgfqpoint{1.929428in}{2.015220in}}{\pgfqpoint{1.936000in}{2.015220in}}%
\pgfpathclose%
\pgfusepath{stroke,fill}%
\end{pgfscope}%
\begin{pgfscope}%
\pgfpathrectangle{\pgfqpoint{1.432000in}{0.528000in}}{\pgfqpoint{3.696000in}{3.696000in}}%
\pgfusepath{clip}%
\pgfsetbuttcap%
\pgfsetroundjoin%
\definecolor{currentfill}{rgb}{0.000000,0.000000,0.000000}%
\pgfsetfillcolor{currentfill}%
\pgfsetlinewidth{1.003750pt}%
\definecolor{currentstroke}{rgb}{0.000000,0.000000,0.000000}%
\pgfsetstrokecolor{currentstroke}%
\pgfsetdash{}{0pt}%
\pgfpathmoveto{\pgfqpoint{1.936000in}{2.351220in}}%
\pgfpathcurveto{\pgfqpoint{1.942572in}{2.351220in}}{\pgfqpoint{1.948875in}{2.353831in}}{\pgfqpoint{1.953522in}{2.358478in}}%
\pgfpathcurveto{\pgfqpoint{1.958169in}{2.363125in}}{\pgfqpoint{1.960780in}{2.369428in}}{\pgfqpoint{1.960780in}{2.376000in}}%
\pgfpathcurveto{\pgfqpoint{1.960780in}{2.382572in}}{\pgfqpoint{1.958169in}{2.388875in}}{\pgfqpoint{1.953522in}{2.393522in}}%
\pgfpathcurveto{\pgfqpoint{1.948875in}{2.398169in}}{\pgfqpoint{1.942572in}{2.400780in}}{\pgfqpoint{1.936000in}{2.400780in}}%
\pgfpathcurveto{\pgfqpoint{1.929428in}{2.400780in}}{\pgfqpoint{1.923125in}{2.398169in}}{\pgfqpoint{1.918478in}{2.393522in}}%
\pgfpathcurveto{\pgfqpoint{1.913831in}{2.388875in}}{\pgfqpoint{1.911220in}{2.382572in}}{\pgfqpoint{1.911220in}{2.376000in}}%
\pgfpathcurveto{\pgfqpoint{1.911220in}{2.369428in}}{\pgfqpoint{1.913831in}{2.363125in}}{\pgfqpoint{1.918478in}{2.358478in}}%
\pgfpathcurveto{\pgfqpoint{1.923125in}{2.353831in}}{\pgfqpoint{1.929428in}{2.351220in}}{\pgfqpoint{1.936000in}{2.351220in}}%
\pgfpathclose%
\pgfusepath{stroke,fill}%
\end{pgfscope}%
\begin{pgfscope}%
\pgfpathrectangle{\pgfqpoint{1.432000in}{0.528000in}}{\pgfqpoint{3.696000in}{3.696000in}}%
\pgfusepath{clip}%
\pgfsetbuttcap%
\pgfsetroundjoin%
\definecolor{currentfill}{rgb}{0.000000,0.000000,0.000000}%
\pgfsetfillcolor{currentfill}%
\pgfsetlinewidth{1.003750pt}%
\definecolor{currentstroke}{rgb}{0.000000,0.000000,0.000000}%
\pgfsetstrokecolor{currentstroke}%
\pgfsetdash{}{0pt}%
\pgfpathmoveto{\pgfqpoint{1.936000in}{2.687220in}}%
\pgfpathcurveto{\pgfqpoint{1.942572in}{2.687220in}}{\pgfqpoint{1.948875in}{2.689831in}}{\pgfqpoint{1.953522in}{2.694478in}}%
\pgfpathcurveto{\pgfqpoint{1.958169in}{2.699125in}}{\pgfqpoint{1.960780in}{2.705428in}}{\pgfqpoint{1.960780in}{2.712000in}}%
\pgfpathcurveto{\pgfqpoint{1.960780in}{2.718572in}}{\pgfqpoint{1.958169in}{2.724875in}}{\pgfqpoint{1.953522in}{2.729522in}}%
\pgfpathcurveto{\pgfqpoint{1.948875in}{2.734169in}}{\pgfqpoint{1.942572in}{2.736780in}}{\pgfqpoint{1.936000in}{2.736780in}}%
\pgfpathcurveto{\pgfqpoint{1.929428in}{2.736780in}}{\pgfqpoint{1.923125in}{2.734169in}}{\pgfqpoint{1.918478in}{2.729522in}}%
\pgfpathcurveto{\pgfqpoint{1.913831in}{2.724875in}}{\pgfqpoint{1.911220in}{2.718572in}}{\pgfqpoint{1.911220in}{2.712000in}}%
\pgfpathcurveto{\pgfqpoint{1.911220in}{2.705428in}}{\pgfqpoint{1.913831in}{2.699125in}}{\pgfqpoint{1.918478in}{2.694478in}}%
\pgfpathcurveto{\pgfqpoint{1.923125in}{2.689831in}}{\pgfqpoint{1.929428in}{2.687220in}}{\pgfqpoint{1.936000in}{2.687220in}}%
\pgfpathclose%
\pgfusepath{stroke,fill}%
\end{pgfscope}%
\begin{pgfscope}%
\pgfpathrectangle{\pgfqpoint{1.432000in}{0.528000in}}{\pgfqpoint{3.696000in}{3.696000in}}%
\pgfusepath{clip}%
\pgfsetbuttcap%
\pgfsetroundjoin%
\definecolor{currentfill}{rgb}{0.000000,0.000000,0.000000}%
\pgfsetfillcolor{currentfill}%
\pgfsetlinewidth{1.003750pt}%
\definecolor{currentstroke}{rgb}{0.000000,0.000000,0.000000}%
\pgfsetstrokecolor{currentstroke}%
\pgfsetdash{}{0pt}%
\pgfpathmoveto{\pgfqpoint{1.936000in}{3.023220in}}%
\pgfpathcurveto{\pgfqpoint{1.942572in}{3.023220in}}{\pgfqpoint{1.948875in}{3.025831in}}{\pgfqpoint{1.953522in}{3.030478in}}%
\pgfpathcurveto{\pgfqpoint{1.958169in}{3.035125in}}{\pgfqpoint{1.960780in}{3.041428in}}{\pgfqpoint{1.960780in}{3.048000in}}%
\pgfpathcurveto{\pgfqpoint{1.960780in}{3.054572in}}{\pgfqpoint{1.958169in}{3.060875in}}{\pgfqpoint{1.953522in}{3.065522in}}%
\pgfpathcurveto{\pgfqpoint{1.948875in}{3.070169in}}{\pgfqpoint{1.942572in}{3.072780in}}{\pgfqpoint{1.936000in}{3.072780in}}%
\pgfpathcurveto{\pgfqpoint{1.929428in}{3.072780in}}{\pgfqpoint{1.923125in}{3.070169in}}{\pgfqpoint{1.918478in}{3.065522in}}%
\pgfpathcurveto{\pgfqpoint{1.913831in}{3.060875in}}{\pgfqpoint{1.911220in}{3.054572in}}{\pgfqpoint{1.911220in}{3.048000in}}%
\pgfpathcurveto{\pgfqpoint{1.911220in}{3.041428in}}{\pgfqpoint{1.913831in}{3.035125in}}{\pgfqpoint{1.918478in}{3.030478in}}%
\pgfpathcurveto{\pgfqpoint{1.923125in}{3.025831in}}{\pgfqpoint{1.929428in}{3.023220in}}{\pgfqpoint{1.936000in}{3.023220in}}%
\pgfpathclose%
\pgfusepath{stroke,fill}%
\end{pgfscope}%
\begin{pgfscope}%
\pgfpathrectangle{\pgfqpoint{1.432000in}{0.528000in}}{\pgfqpoint{3.696000in}{3.696000in}}%
\pgfusepath{clip}%
\pgfsetbuttcap%
\pgfsetroundjoin%
\definecolor{currentfill}{rgb}{0.000000,0.000000,0.000000}%
\pgfsetfillcolor{currentfill}%
\pgfsetlinewidth{1.003750pt}%
\definecolor{currentstroke}{rgb}{0.000000,0.000000,0.000000}%
\pgfsetstrokecolor{currentstroke}%
\pgfsetdash{}{0pt}%
\pgfpathmoveto{\pgfqpoint{1.936000in}{3.359220in}}%
\pgfpathcurveto{\pgfqpoint{1.942572in}{3.359220in}}{\pgfqpoint{1.948875in}{3.361831in}}{\pgfqpoint{1.953522in}{3.366478in}}%
\pgfpathcurveto{\pgfqpoint{1.958169in}{3.371125in}}{\pgfqpoint{1.960780in}{3.377428in}}{\pgfqpoint{1.960780in}{3.384000in}}%
\pgfpathcurveto{\pgfqpoint{1.960780in}{3.390572in}}{\pgfqpoint{1.958169in}{3.396875in}}{\pgfqpoint{1.953522in}{3.401522in}}%
\pgfpathcurveto{\pgfqpoint{1.948875in}{3.406169in}}{\pgfqpoint{1.942572in}{3.408780in}}{\pgfqpoint{1.936000in}{3.408780in}}%
\pgfpathcurveto{\pgfqpoint{1.929428in}{3.408780in}}{\pgfqpoint{1.923125in}{3.406169in}}{\pgfqpoint{1.918478in}{3.401522in}}%
\pgfpathcurveto{\pgfqpoint{1.913831in}{3.396875in}}{\pgfqpoint{1.911220in}{3.390572in}}{\pgfqpoint{1.911220in}{3.384000in}}%
\pgfpathcurveto{\pgfqpoint{1.911220in}{3.377428in}}{\pgfqpoint{1.913831in}{3.371125in}}{\pgfqpoint{1.918478in}{3.366478in}}%
\pgfpathcurveto{\pgfqpoint{1.923125in}{3.361831in}}{\pgfqpoint{1.929428in}{3.359220in}}{\pgfqpoint{1.936000in}{3.359220in}}%
\pgfpathclose%
\pgfusepath{stroke,fill}%
\end{pgfscope}%
\begin{pgfscope}%
\pgfpathrectangle{\pgfqpoint{1.432000in}{0.528000in}}{\pgfqpoint{3.696000in}{3.696000in}}%
\pgfusepath{clip}%
\pgfsetbuttcap%
\pgfsetroundjoin%
\definecolor{currentfill}{rgb}{0.000000,0.000000,0.000000}%
\pgfsetfillcolor{currentfill}%
\pgfsetlinewidth{1.003750pt}%
\definecolor{currentstroke}{rgb}{0.000000,0.000000,0.000000}%
\pgfsetstrokecolor{currentstroke}%
\pgfsetdash{}{0pt}%
\pgfpathmoveto{\pgfqpoint{1.936000in}{3.695220in}}%
\pgfpathcurveto{\pgfqpoint{1.942572in}{3.695220in}}{\pgfqpoint{1.948875in}{3.697831in}}{\pgfqpoint{1.953522in}{3.702478in}}%
\pgfpathcurveto{\pgfqpoint{1.958169in}{3.707125in}}{\pgfqpoint{1.960780in}{3.713428in}}{\pgfqpoint{1.960780in}{3.720000in}}%
\pgfpathcurveto{\pgfqpoint{1.960780in}{3.726572in}}{\pgfqpoint{1.958169in}{3.732875in}}{\pgfqpoint{1.953522in}{3.737522in}}%
\pgfpathcurveto{\pgfqpoint{1.948875in}{3.742169in}}{\pgfqpoint{1.942572in}{3.744780in}}{\pgfqpoint{1.936000in}{3.744780in}}%
\pgfpathcurveto{\pgfqpoint{1.929428in}{3.744780in}}{\pgfqpoint{1.923125in}{3.742169in}}{\pgfqpoint{1.918478in}{3.737522in}}%
\pgfpathcurveto{\pgfqpoint{1.913831in}{3.732875in}}{\pgfqpoint{1.911220in}{3.726572in}}{\pgfqpoint{1.911220in}{3.720000in}}%
\pgfpathcurveto{\pgfqpoint{1.911220in}{3.713428in}}{\pgfqpoint{1.913831in}{3.707125in}}{\pgfqpoint{1.918478in}{3.702478in}}%
\pgfpathcurveto{\pgfqpoint{1.923125in}{3.697831in}}{\pgfqpoint{1.929428in}{3.695220in}}{\pgfqpoint{1.936000in}{3.695220in}}%
\pgfpathclose%
\pgfusepath{stroke,fill}%
\end{pgfscope}%
\begin{pgfscope}%
\pgfpathrectangle{\pgfqpoint{1.432000in}{0.528000in}}{\pgfqpoint{3.696000in}{3.696000in}}%
\pgfusepath{clip}%
\pgfsetbuttcap%
\pgfsetroundjoin%
\definecolor{currentfill}{rgb}{0.000000,0.000000,0.000000}%
\pgfsetfillcolor{currentfill}%
\pgfsetlinewidth{1.003750pt}%
\definecolor{currentstroke}{rgb}{0.000000,0.000000,0.000000}%
\pgfsetstrokecolor{currentstroke}%
\pgfsetdash{}{0pt}%
\pgfpathmoveto{\pgfqpoint{1.936000in}{4.031220in}}%
\pgfpathcurveto{\pgfqpoint{1.942572in}{4.031220in}}{\pgfqpoint{1.948875in}{4.033831in}}{\pgfqpoint{1.953522in}{4.038478in}}%
\pgfpathcurveto{\pgfqpoint{1.958169in}{4.043125in}}{\pgfqpoint{1.960780in}{4.049428in}}{\pgfqpoint{1.960780in}{4.056000in}}%
\pgfpathcurveto{\pgfqpoint{1.960780in}{4.062572in}}{\pgfqpoint{1.958169in}{4.068875in}}{\pgfqpoint{1.953522in}{4.073522in}}%
\pgfpathcurveto{\pgfqpoint{1.948875in}{4.078169in}}{\pgfqpoint{1.942572in}{4.080780in}}{\pgfqpoint{1.936000in}{4.080780in}}%
\pgfpathcurveto{\pgfqpoint{1.929428in}{4.080780in}}{\pgfqpoint{1.923125in}{4.078169in}}{\pgfqpoint{1.918478in}{4.073522in}}%
\pgfpathcurveto{\pgfqpoint{1.913831in}{4.068875in}}{\pgfqpoint{1.911220in}{4.062572in}}{\pgfqpoint{1.911220in}{4.056000in}}%
\pgfpathcurveto{\pgfqpoint{1.911220in}{4.049428in}}{\pgfqpoint{1.913831in}{4.043125in}}{\pgfqpoint{1.918478in}{4.038478in}}%
\pgfpathcurveto{\pgfqpoint{1.923125in}{4.033831in}}{\pgfqpoint{1.929428in}{4.031220in}}{\pgfqpoint{1.936000in}{4.031220in}}%
\pgfpathclose%
\pgfusepath{stroke,fill}%
\end{pgfscope}%
\begin{pgfscope}%
\pgfpathrectangle{\pgfqpoint{1.432000in}{0.528000in}}{\pgfqpoint{3.696000in}{3.696000in}}%
\pgfusepath{clip}%
\pgfsetbuttcap%
\pgfsetroundjoin%
\definecolor{currentfill}{rgb}{0.000000,0.000000,0.000000}%
\pgfsetfillcolor{currentfill}%
\pgfsetlinewidth{1.003750pt}%
\definecolor{currentstroke}{rgb}{0.000000,0.000000,0.000000}%
\pgfsetstrokecolor{currentstroke}%
\pgfsetdash{}{0pt}%
\pgfpathmoveto{\pgfqpoint{2.272000in}{0.671220in}}%
\pgfpathcurveto{\pgfqpoint{2.278572in}{0.671220in}}{\pgfqpoint{2.284875in}{0.673831in}}{\pgfqpoint{2.289522in}{0.678478in}}%
\pgfpathcurveto{\pgfqpoint{2.294169in}{0.683125in}}{\pgfqpoint{2.296780in}{0.689428in}}{\pgfqpoint{2.296780in}{0.696000in}}%
\pgfpathcurveto{\pgfqpoint{2.296780in}{0.702572in}}{\pgfqpoint{2.294169in}{0.708875in}}{\pgfqpoint{2.289522in}{0.713522in}}%
\pgfpathcurveto{\pgfqpoint{2.284875in}{0.718169in}}{\pgfqpoint{2.278572in}{0.720780in}}{\pgfqpoint{2.272000in}{0.720780in}}%
\pgfpathcurveto{\pgfqpoint{2.265428in}{0.720780in}}{\pgfqpoint{2.259125in}{0.718169in}}{\pgfqpoint{2.254478in}{0.713522in}}%
\pgfpathcurveto{\pgfqpoint{2.249831in}{0.708875in}}{\pgfqpoint{2.247220in}{0.702572in}}{\pgfqpoint{2.247220in}{0.696000in}}%
\pgfpathcurveto{\pgfqpoint{2.247220in}{0.689428in}}{\pgfqpoint{2.249831in}{0.683125in}}{\pgfqpoint{2.254478in}{0.678478in}}%
\pgfpathcurveto{\pgfqpoint{2.259125in}{0.673831in}}{\pgfqpoint{2.265428in}{0.671220in}}{\pgfqpoint{2.272000in}{0.671220in}}%
\pgfpathclose%
\pgfusepath{stroke,fill}%
\end{pgfscope}%
\begin{pgfscope}%
\pgfpathrectangle{\pgfqpoint{1.432000in}{0.528000in}}{\pgfqpoint{3.696000in}{3.696000in}}%
\pgfusepath{clip}%
\pgfsetbuttcap%
\pgfsetroundjoin%
\definecolor{currentfill}{rgb}{0.000000,0.000000,0.000000}%
\pgfsetfillcolor{currentfill}%
\pgfsetlinewidth{1.003750pt}%
\definecolor{currentstroke}{rgb}{0.000000,0.000000,0.000000}%
\pgfsetstrokecolor{currentstroke}%
\pgfsetdash{}{0pt}%
\pgfpathmoveto{\pgfqpoint{2.272000in}{1.007220in}}%
\pgfpathcurveto{\pgfqpoint{2.278572in}{1.007220in}}{\pgfqpoint{2.284875in}{1.009831in}}{\pgfqpoint{2.289522in}{1.014478in}}%
\pgfpathcurveto{\pgfqpoint{2.294169in}{1.019125in}}{\pgfqpoint{2.296780in}{1.025428in}}{\pgfqpoint{2.296780in}{1.032000in}}%
\pgfpathcurveto{\pgfqpoint{2.296780in}{1.038572in}}{\pgfqpoint{2.294169in}{1.044875in}}{\pgfqpoint{2.289522in}{1.049522in}}%
\pgfpathcurveto{\pgfqpoint{2.284875in}{1.054169in}}{\pgfqpoint{2.278572in}{1.056780in}}{\pgfqpoint{2.272000in}{1.056780in}}%
\pgfpathcurveto{\pgfqpoint{2.265428in}{1.056780in}}{\pgfqpoint{2.259125in}{1.054169in}}{\pgfqpoint{2.254478in}{1.049522in}}%
\pgfpathcurveto{\pgfqpoint{2.249831in}{1.044875in}}{\pgfqpoint{2.247220in}{1.038572in}}{\pgfqpoint{2.247220in}{1.032000in}}%
\pgfpathcurveto{\pgfqpoint{2.247220in}{1.025428in}}{\pgfqpoint{2.249831in}{1.019125in}}{\pgfqpoint{2.254478in}{1.014478in}}%
\pgfpathcurveto{\pgfqpoint{2.259125in}{1.009831in}}{\pgfqpoint{2.265428in}{1.007220in}}{\pgfqpoint{2.272000in}{1.007220in}}%
\pgfpathclose%
\pgfusepath{stroke,fill}%
\end{pgfscope}%
\begin{pgfscope}%
\pgfpathrectangle{\pgfqpoint{1.432000in}{0.528000in}}{\pgfqpoint{3.696000in}{3.696000in}}%
\pgfusepath{clip}%
\pgfsetbuttcap%
\pgfsetroundjoin%
\definecolor{currentfill}{rgb}{0.000000,0.000000,0.000000}%
\pgfsetfillcolor{currentfill}%
\pgfsetlinewidth{1.003750pt}%
\definecolor{currentstroke}{rgb}{0.000000,0.000000,0.000000}%
\pgfsetstrokecolor{currentstroke}%
\pgfsetdash{}{0pt}%
\pgfpathmoveto{\pgfqpoint{2.272000in}{1.343220in}}%
\pgfpathcurveto{\pgfqpoint{2.278572in}{1.343220in}}{\pgfqpoint{2.284875in}{1.345831in}}{\pgfqpoint{2.289522in}{1.350478in}}%
\pgfpathcurveto{\pgfqpoint{2.294169in}{1.355125in}}{\pgfqpoint{2.296780in}{1.361428in}}{\pgfqpoint{2.296780in}{1.368000in}}%
\pgfpathcurveto{\pgfqpoint{2.296780in}{1.374572in}}{\pgfqpoint{2.294169in}{1.380875in}}{\pgfqpoint{2.289522in}{1.385522in}}%
\pgfpathcurveto{\pgfqpoint{2.284875in}{1.390169in}}{\pgfqpoint{2.278572in}{1.392780in}}{\pgfqpoint{2.272000in}{1.392780in}}%
\pgfpathcurveto{\pgfqpoint{2.265428in}{1.392780in}}{\pgfqpoint{2.259125in}{1.390169in}}{\pgfqpoint{2.254478in}{1.385522in}}%
\pgfpathcurveto{\pgfqpoint{2.249831in}{1.380875in}}{\pgfqpoint{2.247220in}{1.374572in}}{\pgfqpoint{2.247220in}{1.368000in}}%
\pgfpathcurveto{\pgfqpoint{2.247220in}{1.361428in}}{\pgfqpoint{2.249831in}{1.355125in}}{\pgfqpoint{2.254478in}{1.350478in}}%
\pgfpathcurveto{\pgfqpoint{2.259125in}{1.345831in}}{\pgfqpoint{2.265428in}{1.343220in}}{\pgfqpoint{2.272000in}{1.343220in}}%
\pgfpathclose%
\pgfusepath{stroke,fill}%
\end{pgfscope}%
\begin{pgfscope}%
\pgfpathrectangle{\pgfqpoint{1.432000in}{0.528000in}}{\pgfqpoint{3.696000in}{3.696000in}}%
\pgfusepath{clip}%
\pgfsetbuttcap%
\pgfsetroundjoin%
\definecolor{currentfill}{rgb}{0.000000,0.000000,0.000000}%
\pgfsetfillcolor{currentfill}%
\pgfsetlinewidth{1.003750pt}%
\definecolor{currentstroke}{rgb}{0.000000,0.000000,0.000000}%
\pgfsetstrokecolor{currentstroke}%
\pgfsetdash{}{0pt}%
\pgfpathmoveto{\pgfqpoint{2.272000in}{1.679220in}}%
\pgfpathcurveto{\pgfqpoint{2.278572in}{1.679220in}}{\pgfqpoint{2.284875in}{1.681831in}}{\pgfqpoint{2.289522in}{1.686478in}}%
\pgfpathcurveto{\pgfqpoint{2.294169in}{1.691125in}}{\pgfqpoint{2.296780in}{1.697428in}}{\pgfqpoint{2.296780in}{1.704000in}}%
\pgfpathcurveto{\pgfqpoint{2.296780in}{1.710572in}}{\pgfqpoint{2.294169in}{1.716875in}}{\pgfqpoint{2.289522in}{1.721522in}}%
\pgfpathcurveto{\pgfqpoint{2.284875in}{1.726169in}}{\pgfqpoint{2.278572in}{1.728780in}}{\pgfqpoint{2.272000in}{1.728780in}}%
\pgfpathcurveto{\pgfqpoint{2.265428in}{1.728780in}}{\pgfqpoint{2.259125in}{1.726169in}}{\pgfqpoint{2.254478in}{1.721522in}}%
\pgfpathcurveto{\pgfqpoint{2.249831in}{1.716875in}}{\pgfqpoint{2.247220in}{1.710572in}}{\pgfqpoint{2.247220in}{1.704000in}}%
\pgfpathcurveto{\pgfqpoint{2.247220in}{1.697428in}}{\pgfqpoint{2.249831in}{1.691125in}}{\pgfqpoint{2.254478in}{1.686478in}}%
\pgfpathcurveto{\pgfqpoint{2.259125in}{1.681831in}}{\pgfqpoint{2.265428in}{1.679220in}}{\pgfqpoint{2.272000in}{1.679220in}}%
\pgfpathclose%
\pgfusepath{stroke,fill}%
\end{pgfscope}%
\begin{pgfscope}%
\pgfpathrectangle{\pgfqpoint{1.432000in}{0.528000in}}{\pgfqpoint{3.696000in}{3.696000in}}%
\pgfusepath{clip}%
\pgfsetbuttcap%
\pgfsetroundjoin%
\definecolor{currentfill}{rgb}{0.000000,0.000000,0.000000}%
\pgfsetfillcolor{currentfill}%
\pgfsetlinewidth{1.003750pt}%
\definecolor{currentstroke}{rgb}{0.000000,0.000000,0.000000}%
\pgfsetstrokecolor{currentstroke}%
\pgfsetdash{}{0pt}%
\pgfpathmoveto{\pgfqpoint{2.272000in}{2.015220in}}%
\pgfpathcurveto{\pgfqpoint{2.278572in}{2.015220in}}{\pgfqpoint{2.284875in}{2.017831in}}{\pgfqpoint{2.289522in}{2.022478in}}%
\pgfpathcurveto{\pgfqpoint{2.294169in}{2.027125in}}{\pgfqpoint{2.296780in}{2.033428in}}{\pgfqpoint{2.296780in}{2.040000in}}%
\pgfpathcurveto{\pgfqpoint{2.296780in}{2.046572in}}{\pgfqpoint{2.294169in}{2.052875in}}{\pgfqpoint{2.289522in}{2.057522in}}%
\pgfpathcurveto{\pgfqpoint{2.284875in}{2.062169in}}{\pgfqpoint{2.278572in}{2.064780in}}{\pgfqpoint{2.272000in}{2.064780in}}%
\pgfpathcurveto{\pgfqpoint{2.265428in}{2.064780in}}{\pgfqpoint{2.259125in}{2.062169in}}{\pgfqpoint{2.254478in}{2.057522in}}%
\pgfpathcurveto{\pgfqpoint{2.249831in}{2.052875in}}{\pgfqpoint{2.247220in}{2.046572in}}{\pgfqpoint{2.247220in}{2.040000in}}%
\pgfpathcurveto{\pgfqpoint{2.247220in}{2.033428in}}{\pgfqpoint{2.249831in}{2.027125in}}{\pgfqpoint{2.254478in}{2.022478in}}%
\pgfpathcurveto{\pgfqpoint{2.259125in}{2.017831in}}{\pgfqpoint{2.265428in}{2.015220in}}{\pgfqpoint{2.272000in}{2.015220in}}%
\pgfpathclose%
\pgfusepath{stroke,fill}%
\end{pgfscope}%
\begin{pgfscope}%
\pgfpathrectangle{\pgfqpoint{1.432000in}{0.528000in}}{\pgfqpoint{3.696000in}{3.696000in}}%
\pgfusepath{clip}%
\pgfsetbuttcap%
\pgfsetroundjoin%
\definecolor{currentfill}{rgb}{0.000000,0.000000,0.000000}%
\pgfsetfillcolor{currentfill}%
\pgfsetlinewidth{1.003750pt}%
\definecolor{currentstroke}{rgb}{0.000000,0.000000,0.000000}%
\pgfsetstrokecolor{currentstroke}%
\pgfsetdash{}{0pt}%
\pgfpathmoveto{\pgfqpoint{2.272000in}{2.351220in}}%
\pgfpathcurveto{\pgfqpoint{2.278572in}{2.351220in}}{\pgfqpoint{2.284875in}{2.353831in}}{\pgfqpoint{2.289522in}{2.358478in}}%
\pgfpathcurveto{\pgfqpoint{2.294169in}{2.363125in}}{\pgfqpoint{2.296780in}{2.369428in}}{\pgfqpoint{2.296780in}{2.376000in}}%
\pgfpathcurveto{\pgfqpoint{2.296780in}{2.382572in}}{\pgfqpoint{2.294169in}{2.388875in}}{\pgfqpoint{2.289522in}{2.393522in}}%
\pgfpathcurveto{\pgfqpoint{2.284875in}{2.398169in}}{\pgfqpoint{2.278572in}{2.400780in}}{\pgfqpoint{2.272000in}{2.400780in}}%
\pgfpathcurveto{\pgfqpoint{2.265428in}{2.400780in}}{\pgfqpoint{2.259125in}{2.398169in}}{\pgfqpoint{2.254478in}{2.393522in}}%
\pgfpathcurveto{\pgfqpoint{2.249831in}{2.388875in}}{\pgfqpoint{2.247220in}{2.382572in}}{\pgfqpoint{2.247220in}{2.376000in}}%
\pgfpathcurveto{\pgfqpoint{2.247220in}{2.369428in}}{\pgfqpoint{2.249831in}{2.363125in}}{\pgfqpoint{2.254478in}{2.358478in}}%
\pgfpathcurveto{\pgfqpoint{2.259125in}{2.353831in}}{\pgfqpoint{2.265428in}{2.351220in}}{\pgfqpoint{2.272000in}{2.351220in}}%
\pgfpathclose%
\pgfusepath{stroke,fill}%
\end{pgfscope}%
\begin{pgfscope}%
\pgfpathrectangle{\pgfqpoint{1.432000in}{0.528000in}}{\pgfqpoint{3.696000in}{3.696000in}}%
\pgfusepath{clip}%
\pgfsetbuttcap%
\pgfsetroundjoin%
\definecolor{currentfill}{rgb}{0.000000,0.000000,0.000000}%
\pgfsetfillcolor{currentfill}%
\pgfsetlinewidth{1.003750pt}%
\definecolor{currentstroke}{rgb}{0.000000,0.000000,0.000000}%
\pgfsetstrokecolor{currentstroke}%
\pgfsetdash{}{0pt}%
\pgfpathmoveto{\pgfqpoint{2.272000in}{2.687220in}}%
\pgfpathcurveto{\pgfqpoint{2.278572in}{2.687220in}}{\pgfqpoint{2.284875in}{2.689831in}}{\pgfqpoint{2.289522in}{2.694478in}}%
\pgfpathcurveto{\pgfqpoint{2.294169in}{2.699125in}}{\pgfqpoint{2.296780in}{2.705428in}}{\pgfqpoint{2.296780in}{2.712000in}}%
\pgfpathcurveto{\pgfqpoint{2.296780in}{2.718572in}}{\pgfqpoint{2.294169in}{2.724875in}}{\pgfqpoint{2.289522in}{2.729522in}}%
\pgfpathcurveto{\pgfqpoint{2.284875in}{2.734169in}}{\pgfqpoint{2.278572in}{2.736780in}}{\pgfqpoint{2.272000in}{2.736780in}}%
\pgfpathcurveto{\pgfqpoint{2.265428in}{2.736780in}}{\pgfqpoint{2.259125in}{2.734169in}}{\pgfqpoint{2.254478in}{2.729522in}}%
\pgfpathcurveto{\pgfqpoint{2.249831in}{2.724875in}}{\pgfqpoint{2.247220in}{2.718572in}}{\pgfqpoint{2.247220in}{2.712000in}}%
\pgfpathcurveto{\pgfqpoint{2.247220in}{2.705428in}}{\pgfqpoint{2.249831in}{2.699125in}}{\pgfqpoint{2.254478in}{2.694478in}}%
\pgfpathcurveto{\pgfqpoint{2.259125in}{2.689831in}}{\pgfqpoint{2.265428in}{2.687220in}}{\pgfqpoint{2.272000in}{2.687220in}}%
\pgfpathclose%
\pgfusepath{stroke,fill}%
\end{pgfscope}%
\begin{pgfscope}%
\pgfpathrectangle{\pgfqpoint{1.432000in}{0.528000in}}{\pgfqpoint{3.696000in}{3.696000in}}%
\pgfusepath{clip}%
\pgfsetbuttcap%
\pgfsetroundjoin%
\definecolor{currentfill}{rgb}{0.000000,0.000000,0.000000}%
\pgfsetfillcolor{currentfill}%
\pgfsetlinewidth{1.003750pt}%
\definecolor{currentstroke}{rgb}{0.000000,0.000000,0.000000}%
\pgfsetstrokecolor{currentstroke}%
\pgfsetdash{}{0pt}%
\pgfpathmoveto{\pgfqpoint{2.272000in}{3.023220in}}%
\pgfpathcurveto{\pgfqpoint{2.278572in}{3.023220in}}{\pgfqpoint{2.284875in}{3.025831in}}{\pgfqpoint{2.289522in}{3.030478in}}%
\pgfpathcurveto{\pgfqpoint{2.294169in}{3.035125in}}{\pgfqpoint{2.296780in}{3.041428in}}{\pgfqpoint{2.296780in}{3.048000in}}%
\pgfpathcurveto{\pgfqpoint{2.296780in}{3.054572in}}{\pgfqpoint{2.294169in}{3.060875in}}{\pgfqpoint{2.289522in}{3.065522in}}%
\pgfpathcurveto{\pgfqpoint{2.284875in}{3.070169in}}{\pgfqpoint{2.278572in}{3.072780in}}{\pgfqpoint{2.272000in}{3.072780in}}%
\pgfpathcurveto{\pgfqpoint{2.265428in}{3.072780in}}{\pgfqpoint{2.259125in}{3.070169in}}{\pgfqpoint{2.254478in}{3.065522in}}%
\pgfpathcurveto{\pgfqpoint{2.249831in}{3.060875in}}{\pgfqpoint{2.247220in}{3.054572in}}{\pgfqpoint{2.247220in}{3.048000in}}%
\pgfpathcurveto{\pgfqpoint{2.247220in}{3.041428in}}{\pgfqpoint{2.249831in}{3.035125in}}{\pgfqpoint{2.254478in}{3.030478in}}%
\pgfpathcurveto{\pgfqpoint{2.259125in}{3.025831in}}{\pgfqpoint{2.265428in}{3.023220in}}{\pgfqpoint{2.272000in}{3.023220in}}%
\pgfpathclose%
\pgfusepath{stroke,fill}%
\end{pgfscope}%
\begin{pgfscope}%
\pgfpathrectangle{\pgfqpoint{1.432000in}{0.528000in}}{\pgfqpoint{3.696000in}{3.696000in}}%
\pgfusepath{clip}%
\pgfsetbuttcap%
\pgfsetroundjoin%
\definecolor{currentfill}{rgb}{0.000000,0.000000,0.000000}%
\pgfsetfillcolor{currentfill}%
\pgfsetlinewidth{1.003750pt}%
\definecolor{currentstroke}{rgb}{0.000000,0.000000,0.000000}%
\pgfsetstrokecolor{currentstroke}%
\pgfsetdash{}{0pt}%
\pgfpathmoveto{\pgfqpoint{2.272000in}{3.359220in}}%
\pgfpathcurveto{\pgfqpoint{2.278572in}{3.359220in}}{\pgfqpoint{2.284875in}{3.361831in}}{\pgfqpoint{2.289522in}{3.366478in}}%
\pgfpathcurveto{\pgfqpoint{2.294169in}{3.371125in}}{\pgfqpoint{2.296780in}{3.377428in}}{\pgfqpoint{2.296780in}{3.384000in}}%
\pgfpathcurveto{\pgfqpoint{2.296780in}{3.390572in}}{\pgfqpoint{2.294169in}{3.396875in}}{\pgfqpoint{2.289522in}{3.401522in}}%
\pgfpathcurveto{\pgfqpoint{2.284875in}{3.406169in}}{\pgfqpoint{2.278572in}{3.408780in}}{\pgfqpoint{2.272000in}{3.408780in}}%
\pgfpathcurveto{\pgfqpoint{2.265428in}{3.408780in}}{\pgfqpoint{2.259125in}{3.406169in}}{\pgfqpoint{2.254478in}{3.401522in}}%
\pgfpathcurveto{\pgfqpoint{2.249831in}{3.396875in}}{\pgfqpoint{2.247220in}{3.390572in}}{\pgfqpoint{2.247220in}{3.384000in}}%
\pgfpathcurveto{\pgfqpoint{2.247220in}{3.377428in}}{\pgfqpoint{2.249831in}{3.371125in}}{\pgfqpoint{2.254478in}{3.366478in}}%
\pgfpathcurveto{\pgfqpoint{2.259125in}{3.361831in}}{\pgfqpoint{2.265428in}{3.359220in}}{\pgfqpoint{2.272000in}{3.359220in}}%
\pgfpathclose%
\pgfusepath{stroke,fill}%
\end{pgfscope}%
\begin{pgfscope}%
\pgfpathrectangle{\pgfqpoint{1.432000in}{0.528000in}}{\pgfqpoint{3.696000in}{3.696000in}}%
\pgfusepath{clip}%
\pgfsetbuttcap%
\pgfsetroundjoin%
\definecolor{currentfill}{rgb}{0.000000,0.000000,0.000000}%
\pgfsetfillcolor{currentfill}%
\pgfsetlinewidth{1.003750pt}%
\definecolor{currentstroke}{rgb}{0.000000,0.000000,0.000000}%
\pgfsetstrokecolor{currentstroke}%
\pgfsetdash{}{0pt}%
\pgfpathmoveto{\pgfqpoint{2.272000in}{3.695220in}}%
\pgfpathcurveto{\pgfqpoint{2.278572in}{3.695220in}}{\pgfqpoint{2.284875in}{3.697831in}}{\pgfqpoint{2.289522in}{3.702478in}}%
\pgfpathcurveto{\pgfqpoint{2.294169in}{3.707125in}}{\pgfqpoint{2.296780in}{3.713428in}}{\pgfqpoint{2.296780in}{3.720000in}}%
\pgfpathcurveto{\pgfqpoint{2.296780in}{3.726572in}}{\pgfqpoint{2.294169in}{3.732875in}}{\pgfqpoint{2.289522in}{3.737522in}}%
\pgfpathcurveto{\pgfqpoint{2.284875in}{3.742169in}}{\pgfqpoint{2.278572in}{3.744780in}}{\pgfqpoint{2.272000in}{3.744780in}}%
\pgfpathcurveto{\pgfqpoint{2.265428in}{3.744780in}}{\pgfqpoint{2.259125in}{3.742169in}}{\pgfqpoint{2.254478in}{3.737522in}}%
\pgfpathcurveto{\pgfqpoint{2.249831in}{3.732875in}}{\pgfqpoint{2.247220in}{3.726572in}}{\pgfqpoint{2.247220in}{3.720000in}}%
\pgfpathcurveto{\pgfqpoint{2.247220in}{3.713428in}}{\pgfqpoint{2.249831in}{3.707125in}}{\pgfqpoint{2.254478in}{3.702478in}}%
\pgfpathcurveto{\pgfqpoint{2.259125in}{3.697831in}}{\pgfqpoint{2.265428in}{3.695220in}}{\pgfqpoint{2.272000in}{3.695220in}}%
\pgfpathclose%
\pgfusepath{stroke,fill}%
\end{pgfscope}%
\begin{pgfscope}%
\pgfpathrectangle{\pgfqpoint{1.432000in}{0.528000in}}{\pgfqpoint{3.696000in}{3.696000in}}%
\pgfusepath{clip}%
\pgfsetbuttcap%
\pgfsetroundjoin%
\definecolor{currentfill}{rgb}{0.000000,0.000000,0.000000}%
\pgfsetfillcolor{currentfill}%
\pgfsetlinewidth{1.003750pt}%
\definecolor{currentstroke}{rgb}{0.000000,0.000000,0.000000}%
\pgfsetstrokecolor{currentstroke}%
\pgfsetdash{}{0pt}%
\pgfpathmoveto{\pgfqpoint{2.272000in}{4.031220in}}%
\pgfpathcurveto{\pgfqpoint{2.278572in}{4.031220in}}{\pgfqpoint{2.284875in}{4.033831in}}{\pgfqpoint{2.289522in}{4.038478in}}%
\pgfpathcurveto{\pgfqpoint{2.294169in}{4.043125in}}{\pgfqpoint{2.296780in}{4.049428in}}{\pgfqpoint{2.296780in}{4.056000in}}%
\pgfpathcurveto{\pgfqpoint{2.296780in}{4.062572in}}{\pgfqpoint{2.294169in}{4.068875in}}{\pgfqpoint{2.289522in}{4.073522in}}%
\pgfpathcurveto{\pgfqpoint{2.284875in}{4.078169in}}{\pgfqpoint{2.278572in}{4.080780in}}{\pgfqpoint{2.272000in}{4.080780in}}%
\pgfpathcurveto{\pgfqpoint{2.265428in}{4.080780in}}{\pgfqpoint{2.259125in}{4.078169in}}{\pgfqpoint{2.254478in}{4.073522in}}%
\pgfpathcurveto{\pgfqpoint{2.249831in}{4.068875in}}{\pgfqpoint{2.247220in}{4.062572in}}{\pgfqpoint{2.247220in}{4.056000in}}%
\pgfpathcurveto{\pgfqpoint{2.247220in}{4.049428in}}{\pgfqpoint{2.249831in}{4.043125in}}{\pgfqpoint{2.254478in}{4.038478in}}%
\pgfpathcurveto{\pgfqpoint{2.259125in}{4.033831in}}{\pgfqpoint{2.265428in}{4.031220in}}{\pgfqpoint{2.272000in}{4.031220in}}%
\pgfpathclose%
\pgfusepath{stroke,fill}%
\end{pgfscope}%
\begin{pgfscope}%
\pgfpathrectangle{\pgfqpoint{1.432000in}{0.528000in}}{\pgfqpoint{3.696000in}{3.696000in}}%
\pgfusepath{clip}%
\pgfsetbuttcap%
\pgfsetroundjoin%
\definecolor{currentfill}{rgb}{0.000000,0.000000,0.000000}%
\pgfsetfillcolor{currentfill}%
\pgfsetlinewidth{1.003750pt}%
\definecolor{currentstroke}{rgb}{0.000000,0.000000,0.000000}%
\pgfsetstrokecolor{currentstroke}%
\pgfsetdash{}{0pt}%
\pgfpathmoveto{\pgfqpoint{2.608000in}{0.671220in}}%
\pgfpathcurveto{\pgfqpoint{2.614572in}{0.671220in}}{\pgfqpoint{2.620875in}{0.673831in}}{\pgfqpoint{2.625522in}{0.678478in}}%
\pgfpathcurveto{\pgfqpoint{2.630169in}{0.683125in}}{\pgfqpoint{2.632780in}{0.689428in}}{\pgfqpoint{2.632780in}{0.696000in}}%
\pgfpathcurveto{\pgfqpoint{2.632780in}{0.702572in}}{\pgfqpoint{2.630169in}{0.708875in}}{\pgfqpoint{2.625522in}{0.713522in}}%
\pgfpathcurveto{\pgfqpoint{2.620875in}{0.718169in}}{\pgfqpoint{2.614572in}{0.720780in}}{\pgfqpoint{2.608000in}{0.720780in}}%
\pgfpathcurveto{\pgfqpoint{2.601428in}{0.720780in}}{\pgfqpoint{2.595125in}{0.718169in}}{\pgfqpoint{2.590478in}{0.713522in}}%
\pgfpathcurveto{\pgfqpoint{2.585831in}{0.708875in}}{\pgfqpoint{2.583220in}{0.702572in}}{\pgfqpoint{2.583220in}{0.696000in}}%
\pgfpathcurveto{\pgfqpoint{2.583220in}{0.689428in}}{\pgfqpoint{2.585831in}{0.683125in}}{\pgfqpoint{2.590478in}{0.678478in}}%
\pgfpathcurveto{\pgfqpoint{2.595125in}{0.673831in}}{\pgfqpoint{2.601428in}{0.671220in}}{\pgfqpoint{2.608000in}{0.671220in}}%
\pgfpathclose%
\pgfusepath{stroke,fill}%
\end{pgfscope}%
\begin{pgfscope}%
\pgfpathrectangle{\pgfqpoint{1.432000in}{0.528000in}}{\pgfqpoint{3.696000in}{3.696000in}}%
\pgfusepath{clip}%
\pgfsetbuttcap%
\pgfsetroundjoin%
\definecolor{currentfill}{rgb}{0.000000,0.000000,0.000000}%
\pgfsetfillcolor{currentfill}%
\pgfsetlinewidth{1.003750pt}%
\definecolor{currentstroke}{rgb}{0.000000,0.000000,0.000000}%
\pgfsetstrokecolor{currentstroke}%
\pgfsetdash{}{0pt}%
\pgfpathmoveto{\pgfqpoint{2.608000in}{1.007220in}}%
\pgfpathcurveto{\pgfqpoint{2.614572in}{1.007220in}}{\pgfqpoint{2.620875in}{1.009831in}}{\pgfqpoint{2.625522in}{1.014478in}}%
\pgfpathcurveto{\pgfqpoint{2.630169in}{1.019125in}}{\pgfqpoint{2.632780in}{1.025428in}}{\pgfqpoint{2.632780in}{1.032000in}}%
\pgfpathcurveto{\pgfqpoint{2.632780in}{1.038572in}}{\pgfqpoint{2.630169in}{1.044875in}}{\pgfqpoint{2.625522in}{1.049522in}}%
\pgfpathcurveto{\pgfqpoint{2.620875in}{1.054169in}}{\pgfqpoint{2.614572in}{1.056780in}}{\pgfqpoint{2.608000in}{1.056780in}}%
\pgfpathcurveto{\pgfqpoint{2.601428in}{1.056780in}}{\pgfqpoint{2.595125in}{1.054169in}}{\pgfqpoint{2.590478in}{1.049522in}}%
\pgfpathcurveto{\pgfqpoint{2.585831in}{1.044875in}}{\pgfqpoint{2.583220in}{1.038572in}}{\pgfqpoint{2.583220in}{1.032000in}}%
\pgfpathcurveto{\pgfqpoint{2.583220in}{1.025428in}}{\pgfqpoint{2.585831in}{1.019125in}}{\pgfqpoint{2.590478in}{1.014478in}}%
\pgfpathcurveto{\pgfqpoint{2.595125in}{1.009831in}}{\pgfqpoint{2.601428in}{1.007220in}}{\pgfqpoint{2.608000in}{1.007220in}}%
\pgfpathclose%
\pgfusepath{stroke,fill}%
\end{pgfscope}%
\begin{pgfscope}%
\pgfpathrectangle{\pgfqpoint{1.432000in}{0.528000in}}{\pgfqpoint{3.696000in}{3.696000in}}%
\pgfusepath{clip}%
\pgfsetbuttcap%
\pgfsetroundjoin%
\definecolor{currentfill}{rgb}{0.000000,0.000000,0.000000}%
\pgfsetfillcolor{currentfill}%
\pgfsetlinewidth{1.003750pt}%
\definecolor{currentstroke}{rgb}{0.000000,0.000000,0.000000}%
\pgfsetstrokecolor{currentstroke}%
\pgfsetdash{}{0pt}%
\pgfpathmoveto{\pgfqpoint{2.608000in}{1.343220in}}%
\pgfpathcurveto{\pgfqpoint{2.614572in}{1.343220in}}{\pgfqpoint{2.620875in}{1.345831in}}{\pgfqpoint{2.625522in}{1.350478in}}%
\pgfpathcurveto{\pgfqpoint{2.630169in}{1.355125in}}{\pgfqpoint{2.632780in}{1.361428in}}{\pgfqpoint{2.632780in}{1.368000in}}%
\pgfpathcurveto{\pgfqpoint{2.632780in}{1.374572in}}{\pgfqpoint{2.630169in}{1.380875in}}{\pgfqpoint{2.625522in}{1.385522in}}%
\pgfpathcurveto{\pgfqpoint{2.620875in}{1.390169in}}{\pgfqpoint{2.614572in}{1.392780in}}{\pgfqpoint{2.608000in}{1.392780in}}%
\pgfpathcurveto{\pgfqpoint{2.601428in}{1.392780in}}{\pgfqpoint{2.595125in}{1.390169in}}{\pgfqpoint{2.590478in}{1.385522in}}%
\pgfpathcurveto{\pgfqpoint{2.585831in}{1.380875in}}{\pgfqpoint{2.583220in}{1.374572in}}{\pgfqpoint{2.583220in}{1.368000in}}%
\pgfpathcurveto{\pgfqpoint{2.583220in}{1.361428in}}{\pgfqpoint{2.585831in}{1.355125in}}{\pgfqpoint{2.590478in}{1.350478in}}%
\pgfpathcurveto{\pgfqpoint{2.595125in}{1.345831in}}{\pgfqpoint{2.601428in}{1.343220in}}{\pgfqpoint{2.608000in}{1.343220in}}%
\pgfpathclose%
\pgfusepath{stroke,fill}%
\end{pgfscope}%
\begin{pgfscope}%
\pgfpathrectangle{\pgfqpoint{1.432000in}{0.528000in}}{\pgfqpoint{3.696000in}{3.696000in}}%
\pgfusepath{clip}%
\pgfsetbuttcap%
\pgfsetroundjoin%
\definecolor{currentfill}{rgb}{0.000000,0.000000,0.000000}%
\pgfsetfillcolor{currentfill}%
\pgfsetlinewidth{1.003750pt}%
\definecolor{currentstroke}{rgb}{0.000000,0.000000,0.000000}%
\pgfsetstrokecolor{currentstroke}%
\pgfsetdash{}{0pt}%
\pgfpathmoveto{\pgfqpoint{2.608000in}{1.679220in}}%
\pgfpathcurveto{\pgfqpoint{2.614572in}{1.679220in}}{\pgfqpoint{2.620875in}{1.681831in}}{\pgfqpoint{2.625522in}{1.686478in}}%
\pgfpathcurveto{\pgfqpoint{2.630169in}{1.691125in}}{\pgfqpoint{2.632780in}{1.697428in}}{\pgfqpoint{2.632780in}{1.704000in}}%
\pgfpathcurveto{\pgfqpoint{2.632780in}{1.710572in}}{\pgfqpoint{2.630169in}{1.716875in}}{\pgfqpoint{2.625522in}{1.721522in}}%
\pgfpathcurveto{\pgfqpoint{2.620875in}{1.726169in}}{\pgfqpoint{2.614572in}{1.728780in}}{\pgfqpoint{2.608000in}{1.728780in}}%
\pgfpathcurveto{\pgfqpoint{2.601428in}{1.728780in}}{\pgfqpoint{2.595125in}{1.726169in}}{\pgfqpoint{2.590478in}{1.721522in}}%
\pgfpathcurveto{\pgfqpoint{2.585831in}{1.716875in}}{\pgfqpoint{2.583220in}{1.710572in}}{\pgfqpoint{2.583220in}{1.704000in}}%
\pgfpathcurveto{\pgfqpoint{2.583220in}{1.697428in}}{\pgfqpoint{2.585831in}{1.691125in}}{\pgfqpoint{2.590478in}{1.686478in}}%
\pgfpathcurveto{\pgfqpoint{2.595125in}{1.681831in}}{\pgfqpoint{2.601428in}{1.679220in}}{\pgfqpoint{2.608000in}{1.679220in}}%
\pgfpathclose%
\pgfusepath{stroke,fill}%
\end{pgfscope}%
\begin{pgfscope}%
\pgfpathrectangle{\pgfqpoint{1.432000in}{0.528000in}}{\pgfqpoint{3.696000in}{3.696000in}}%
\pgfusepath{clip}%
\pgfsetbuttcap%
\pgfsetroundjoin%
\definecolor{currentfill}{rgb}{0.000000,0.000000,0.000000}%
\pgfsetfillcolor{currentfill}%
\pgfsetlinewidth{1.003750pt}%
\definecolor{currentstroke}{rgb}{0.000000,0.000000,0.000000}%
\pgfsetstrokecolor{currentstroke}%
\pgfsetdash{}{0pt}%
\pgfpathmoveto{\pgfqpoint{2.608000in}{2.015220in}}%
\pgfpathcurveto{\pgfqpoint{2.614572in}{2.015220in}}{\pgfqpoint{2.620875in}{2.017831in}}{\pgfqpoint{2.625522in}{2.022478in}}%
\pgfpathcurveto{\pgfqpoint{2.630169in}{2.027125in}}{\pgfqpoint{2.632780in}{2.033428in}}{\pgfqpoint{2.632780in}{2.040000in}}%
\pgfpathcurveto{\pgfqpoint{2.632780in}{2.046572in}}{\pgfqpoint{2.630169in}{2.052875in}}{\pgfqpoint{2.625522in}{2.057522in}}%
\pgfpathcurveto{\pgfqpoint{2.620875in}{2.062169in}}{\pgfqpoint{2.614572in}{2.064780in}}{\pgfqpoint{2.608000in}{2.064780in}}%
\pgfpathcurveto{\pgfqpoint{2.601428in}{2.064780in}}{\pgfqpoint{2.595125in}{2.062169in}}{\pgfqpoint{2.590478in}{2.057522in}}%
\pgfpathcurveto{\pgfqpoint{2.585831in}{2.052875in}}{\pgfqpoint{2.583220in}{2.046572in}}{\pgfqpoint{2.583220in}{2.040000in}}%
\pgfpathcurveto{\pgfqpoint{2.583220in}{2.033428in}}{\pgfqpoint{2.585831in}{2.027125in}}{\pgfqpoint{2.590478in}{2.022478in}}%
\pgfpathcurveto{\pgfqpoint{2.595125in}{2.017831in}}{\pgfqpoint{2.601428in}{2.015220in}}{\pgfqpoint{2.608000in}{2.015220in}}%
\pgfpathclose%
\pgfusepath{stroke,fill}%
\end{pgfscope}%
\begin{pgfscope}%
\pgfpathrectangle{\pgfqpoint{1.432000in}{0.528000in}}{\pgfqpoint{3.696000in}{3.696000in}}%
\pgfusepath{clip}%
\pgfsetbuttcap%
\pgfsetroundjoin%
\definecolor{currentfill}{rgb}{0.000000,0.000000,0.000000}%
\pgfsetfillcolor{currentfill}%
\pgfsetlinewidth{1.003750pt}%
\definecolor{currentstroke}{rgb}{0.000000,0.000000,0.000000}%
\pgfsetstrokecolor{currentstroke}%
\pgfsetdash{}{0pt}%
\pgfpathmoveto{\pgfqpoint{2.608000in}{2.351220in}}%
\pgfpathcurveto{\pgfqpoint{2.614572in}{2.351220in}}{\pgfqpoint{2.620875in}{2.353831in}}{\pgfqpoint{2.625522in}{2.358478in}}%
\pgfpathcurveto{\pgfqpoint{2.630169in}{2.363125in}}{\pgfqpoint{2.632780in}{2.369428in}}{\pgfqpoint{2.632780in}{2.376000in}}%
\pgfpathcurveto{\pgfqpoint{2.632780in}{2.382572in}}{\pgfqpoint{2.630169in}{2.388875in}}{\pgfqpoint{2.625522in}{2.393522in}}%
\pgfpathcurveto{\pgfqpoint{2.620875in}{2.398169in}}{\pgfqpoint{2.614572in}{2.400780in}}{\pgfqpoint{2.608000in}{2.400780in}}%
\pgfpathcurveto{\pgfqpoint{2.601428in}{2.400780in}}{\pgfqpoint{2.595125in}{2.398169in}}{\pgfqpoint{2.590478in}{2.393522in}}%
\pgfpathcurveto{\pgfqpoint{2.585831in}{2.388875in}}{\pgfqpoint{2.583220in}{2.382572in}}{\pgfqpoint{2.583220in}{2.376000in}}%
\pgfpathcurveto{\pgfqpoint{2.583220in}{2.369428in}}{\pgfqpoint{2.585831in}{2.363125in}}{\pgfqpoint{2.590478in}{2.358478in}}%
\pgfpathcurveto{\pgfqpoint{2.595125in}{2.353831in}}{\pgfqpoint{2.601428in}{2.351220in}}{\pgfqpoint{2.608000in}{2.351220in}}%
\pgfpathclose%
\pgfusepath{stroke,fill}%
\end{pgfscope}%
\begin{pgfscope}%
\pgfpathrectangle{\pgfqpoint{1.432000in}{0.528000in}}{\pgfqpoint{3.696000in}{3.696000in}}%
\pgfusepath{clip}%
\pgfsetbuttcap%
\pgfsetroundjoin%
\definecolor{currentfill}{rgb}{0.000000,0.000000,0.000000}%
\pgfsetfillcolor{currentfill}%
\pgfsetlinewidth{1.003750pt}%
\definecolor{currentstroke}{rgb}{0.000000,0.000000,0.000000}%
\pgfsetstrokecolor{currentstroke}%
\pgfsetdash{}{0pt}%
\pgfpathmoveto{\pgfqpoint{2.608000in}{2.687220in}}%
\pgfpathcurveto{\pgfqpoint{2.614572in}{2.687220in}}{\pgfqpoint{2.620875in}{2.689831in}}{\pgfqpoint{2.625522in}{2.694478in}}%
\pgfpathcurveto{\pgfqpoint{2.630169in}{2.699125in}}{\pgfqpoint{2.632780in}{2.705428in}}{\pgfqpoint{2.632780in}{2.712000in}}%
\pgfpathcurveto{\pgfqpoint{2.632780in}{2.718572in}}{\pgfqpoint{2.630169in}{2.724875in}}{\pgfqpoint{2.625522in}{2.729522in}}%
\pgfpathcurveto{\pgfqpoint{2.620875in}{2.734169in}}{\pgfqpoint{2.614572in}{2.736780in}}{\pgfqpoint{2.608000in}{2.736780in}}%
\pgfpathcurveto{\pgfqpoint{2.601428in}{2.736780in}}{\pgfqpoint{2.595125in}{2.734169in}}{\pgfqpoint{2.590478in}{2.729522in}}%
\pgfpathcurveto{\pgfqpoint{2.585831in}{2.724875in}}{\pgfqpoint{2.583220in}{2.718572in}}{\pgfqpoint{2.583220in}{2.712000in}}%
\pgfpathcurveto{\pgfqpoint{2.583220in}{2.705428in}}{\pgfqpoint{2.585831in}{2.699125in}}{\pgfqpoint{2.590478in}{2.694478in}}%
\pgfpathcurveto{\pgfqpoint{2.595125in}{2.689831in}}{\pgfqpoint{2.601428in}{2.687220in}}{\pgfqpoint{2.608000in}{2.687220in}}%
\pgfpathclose%
\pgfusepath{stroke,fill}%
\end{pgfscope}%
\begin{pgfscope}%
\pgfpathrectangle{\pgfqpoint{1.432000in}{0.528000in}}{\pgfqpoint{3.696000in}{3.696000in}}%
\pgfusepath{clip}%
\pgfsetbuttcap%
\pgfsetroundjoin%
\definecolor{currentfill}{rgb}{0.000000,0.000000,0.000000}%
\pgfsetfillcolor{currentfill}%
\pgfsetlinewidth{1.003750pt}%
\definecolor{currentstroke}{rgb}{0.000000,0.000000,0.000000}%
\pgfsetstrokecolor{currentstroke}%
\pgfsetdash{}{0pt}%
\pgfpathmoveto{\pgfqpoint{2.608000in}{3.023220in}}%
\pgfpathcurveto{\pgfqpoint{2.614572in}{3.023220in}}{\pgfqpoint{2.620875in}{3.025831in}}{\pgfqpoint{2.625522in}{3.030478in}}%
\pgfpathcurveto{\pgfqpoint{2.630169in}{3.035125in}}{\pgfqpoint{2.632780in}{3.041428in}}{\pgfqpoint{2.632780in}{3.048000in}}%
\pgfpathcurveto{\pgfqpoint{2.632780in}{3.054572in}}{\pgfqpoint{2.630169in}{3.060875in}}{\pgfqpoint{2.625522in}{3.065522in}}%
\pgfpathcurveto{\pgfqpoint{2.620875in}{3.070169in}}{\pgfqpoint{2.614572in}{3.072780in}}{\pgfqpoint{2.608000in}{3.072780in}}%
\pgfpathcurveto{\pgfqpoint{2.601428in}{3.072780in}}{\pgfqpoint{2.595125in}{3.070169in}}{\pgfqpoint{2.590478in}{3.065522in}}%
\pgfpathcurveto{\pgfqpoint{2.585831in}{3.060875in}}{\pgfqpoint{2.583220in}{3.054572in}}{\pgfqpoint{2.583220in}{3.048000in}}%
\pgfpathcurveto{\pgfqpoint{2.583220in}{3.041428in}}{\pgfqpoint{2.585831in}{3.035125in}}{\pgfqpoint{2.590478in}{3.030478in}}%
\pgfpathcurveto{\pgfqpoint{2.595125in}{3.025831in}}{\pgfqpoint{2.601428in}{3.023220in}}{\pgfqpoint{2.608000in}{3.023220in}}%
\pgfpathclose%
\pgfusepath{stroke,fill}%
\end{pgfscope}%
\begin{pgfscope}%
\pgfpathrectangle{\pgfqpoint{1.432000in}{0.528000in}}{\pgfqpoint{3.696000in}{3.696000in}}%
\pgfusepath{clip}%
\pgfsetbuttcap%
\pgfsetroundjoin%
\definecolor{currentfill}{rgb}{0.000000,0.000000,0.000000}%
\pgfsetfillcolor{currentfill}%
\pgfsetlinewidth{1.003750pt}%
\definecolor{currentstroke}{rgb}{0.000000,0.000000,0.000000}%
\pgfsetstrokecolor{currentstroke}%
\pgfsetdash{}{0pt}%
\pgfpathmoveto{\pgfqpoint{2.608000in}{3.359220in}}%
\pgfpathcurveto{\pgfqpoint{2.614572in}{3.359220in}}{\pgfqpoint{2.620875in}{3.361831in}}{\pgfqpoint{2.625522in}{3.366478in}}%
\pgfpathcurveto{\pgfqpoint{2.630169in}{3.371125in}}{\pgfqpoint{2.632780in}{3.377428in}}{\pgfqpoint{2.632780in}{3.384000in}}%
\pgfpathcurveto{\pgfqpoint{2.632780in}{3.390572in}}{\pgfqpoint{2.630169in}{3.396875in}}{\pgfqpoint{2.625522in}{3.401522in}}%
\pgfpathcurveto{\pgfqpoint{2.620875in}{3.406169in}}{\pgfqpoint{2.614572in}{3.408780in}}{\pgfqpoint{2.608000in}{3.408780in}}%
\pgfpathcurveto{\pgfqpoint{2.601428in}{3.408780in}}{\pgfqpoint{2.595125in}{3.406169in}}{\pgfqpoint{2.590478in}{3.401522in}}%
\pgfpathcurveto{\pgfqpoint{2.585831in}{3.396875in}}{\pgfqpoint{2.583220in}{3.390572in}}{\pgfqpoint{2.583220in}{3.384000in}}%
\pgfpathcurveto{\pgfqpoint{2.583220in}{3.377428in}}{\pgfqpoint{2.585831in}{3.371125in}}{\pgfqpoint{2.590478in}{3.366478in}}%
\pgfpathcurveto{\pgfqpoint{2.595125in}{3.361831in}}{\pgfqpoint{2.601428in}{3.359220in}}{\pgfqpoint{2.608000in}{3.359220in}}%
\pgfpathclose%
\pgfusepath{stroke,fill}%
\end{pgfscope}%
\begin{pgfscope}%
\pgfpathrectangle{\pgfqpoint{1.432000in}{0.528000in}}{\pgfqpoint{3.696000in}{3.696000in}}%
\pgfusepath{clip}%
\pgfsetbuttcap%
\pgfsetroundjoin%
\definecolor{currentfill}{rgb}{0.000000,0.000000,0.000000}%
\pgfsetfillcolor{currentfill}%
\pgfsetlinewidth{1.003750pt}%
\definecolor{currentstroke}{rgb}{0.000000,0.000000,0.000000}%
\pgfsetstrokecolor{currentstroke}%
\pgfsetdash{}{0pt}%
\pgfpathmoveto{\pgfqpoint{2.608000in}{3.695220in}}%
\pgfpathcurveto{\pgfqpoint{2.614572in}{3.695220in}}{\pgfqpoint{2.620875in}{3.697831in}}{\pgfqpoint{2.625522in}{3.702478in}}%
\pgfpathcurveto{\pgfqpoint{2.630169in}{3.707125in}}{\pgfqpoint{2.632780in}{3.713428in}}{\pgfqpoint{2.632780in}{3.720000in}}%
\pgfpathcurveto{\pgfqpoint{2.632780in}{3.726572in}}{\pgfqpoint{2.630169in}{3.732875in}}{\pgfqpoint{2.625522in}{3.737522in}}%
\pgfpathcurveto{\pgfqpoint{2.620875in}{3.742169in}}{\pgfqpoint{2.614572in}{3.744780in}}{\pgfqpoint{2.608000in}{3.744780in}}%
\pgfpathcurveto{\pgfqpoint{2.601428in}{3.744780in}}{\pgfqpoint{2.595125in}{3.742169in}}{\pgfqpoint{2.590478in}{3.737522in}}%
\pgfpathcurveto{\pgfqpoint{2.585831in}{3.732875in}}{\pgfqpoint{2.583220in}{3.726572in}}{\pgfqpoint{2.583220in}{3.720000in}}%
\pgfpathcurveto{\pgfqpoint{2.583220in}{3.713428in}}{\pgfqpoint{2.585831in}{3.707125in}}{\pgfqpoint{2.590478in}{3.702478in}}%
\pgfpathcurveto{\pgfqpoint{2.595125in}{3.697831in}}{\pgfqpoint{2.601428in}{3.695220in}}{\pgfqpoint{2.608000in}{3.695220in}}%
\pgfpathclose%
\pgfusepath{stroke,fill}%
\end{pgfscope}%
\begin{pgfscope}%
\pgfpathrectangle{\pgfqpoint{1.432000in}{0.528000in}}{\pgfqpoint{3.696000in}{3.696000in}}%
\pgfusepath{clip}%
\pgfsetbuttcap%
\pgfsetroundjoin%
\definecolor{currentfill}{rgb}{0.000000,0.000000,0.000000}%
\pgfsetfillcolor{currentfill}%
\pgfsetlinewidth{1.003750pt}%
\definecolor{currentstroke}{rgb}{0.000000,0.000000,0.000000}%
\pgfsetstrokecolor{currentstroke}%
\pgfsetdash{}{0pt}%
\pgfpathmoveto{\pgfqpoint{2.608000in}{4.031220in}}%
\pgfpathcurveto{\pgfqpoint{2.614572in}{4.031220in}}{\pgfqpoint{2.620875in}{4.033831in}}{\pgfqpoint{2.625522in}{4.038478in}}%
\pgfpathcurveto{\pgfqpoint{2.630169in}{4.043125in}}{\pgfqpoint{2.632780in}{4.049428in}}{\pgfqpoint{2.632780in}{4.056000in}}%
\pgfpathcurveto{\pgfqpoint{2.632780in}{4.062572in}}{\pgfqpoint{2.630169in}{4.068875in}}{\pgfqpoint{2.625522in}{4.073522in}}%
\pgfpathcurveto{\pgfqpoint{2.620875in}{4.078169in}}{\pgfqpoint{2.614572in}{4.080780in}}{\pgfqpoint{2.608000in}{4.080780in}}%
\pgfpathcurveto{\pgfqpoint{2.601428in}{4.080780in}}{\pgfqpoint{2.595125in}{4.078169in}}{\pgfqpoint{2.590478in}{4.073522in}}%
\pgfpathcurveto{\pgfqpoint{2.585831in}{4.068875in}}{\pgfqpoint{2.583220in}{4.062572in}}{\pgfqpoint{2.583220in}{4.056000in}}%
\pgfpathcurveto{\pgfqpoint{2.583220in}{4.049428in}}{\pgfqpoint{2.585831in}{4.043125in}}{\pgfqpoint{2.590478in}{4.038478in}}%
\pgfpathcurveto{\pgfqpoint{2.595125in}{4.033831in}}{\pgfqpoint{2.601428in}{4.031220in}}{\pgfqpoint{2.608000in}{4.031220in}}%
\pgfpathclose%
\pgfusepath{stroke,fill}%
\end{pgfscope}%
\begin{pgfscope}%
\pgfpathrectangle{\pgfqpoint{1.432000in}{0.528000in}}{\pgfqpoint{3.696000in}{3.696000in}}%
\pgfusepath{clip}%
\pgfsetbuttcap%
\pgfsetroundjoin%
\definecolor{currentfill}{rgb}{0.000000,0.000000,0.000000}%
\pgfsetfillcolor{currentfill}%
\pgfsetlinewidth{1.003750pt}%
\definecolor{currentstroke}{rgb}{0.000000,0.000000,0.000000}%
\pgfsetstrokecolor{currentstroke}%
\pgfsetdash{}{0pt}%
\pgfpathmoveto{\pgfqpoint{2.944000in}{0.671220in}}%
\pgfpathcurveto{\pgfqpoint{2.950572in}{0.671220in}}{\pgfqpoint{2.956875in}{0.673831in}}{\pgfqpoint{2.961522in}{0.678478in}}%
\pgfpathcurveto{\pgfqpoint{2.966169in}{0.683125in}}{\pgfqpoint{2.968780in}{0.689428in}}{\pgfqpoint{2.968780in}{0.696000in}}%
\pgfpathcurveto{\pgfqpoint{2.968780in}{0.702572in}}{\pgfqpoint{2.966169in}{0.708875in}}{\pgfqpoint{2.961522in}{0.713522in}}%
\pgfpathcurveto{\pgfqpoint{2.956875in}{0.718169in}}{\pgfqpoint{2.950572in}{0.720780in}}{\pgfqpoint{2.944000in}{0.720780in}}%
\pgfpathcurveto{\pgfqpoint{2.937428in}{0.720780in}}{\pgfqpoint{2.931125in}{0.718169in}}{\pgfqpoint{2.926478in}{0.713522in}}%
\pgfpathcurveto{\pgfqpoint{2.921831in}{0.708875in}}{\pgfqpoint{2.919220in}{0.702572in}}{\pgfqpoint{2.919220in}{0.696000in}}%
\pgfpathcurveto{\pgfqpoint{2.919220in}{0.689428in}}{\pgfqpoint{2.921831in}{0.683125in}}{\pgfqpoint{2.926478in}{0.678478in}}%
\pgfpathcurveto{\pgfqpoint{2.931125in}{0.673831in}}{\pgfqpoint{2.937428in}{0.671220in}}{\pgfqpoint{2.944000in}{0.671220in}}%
\pgfpathclose%
\pgfusepath{stroke,fill}%
\end{pgfscope}%
\begin{pgfscope}%
\pgfpathrectangle{\pgfqpoint{1.432000in}{0.528000in}}{\pgfqpoint{3.696000in}{3.696000in}}%
\pgfusepath{clip}%
\pgfsetbuttcap%
\pgfsetroundjoin%
\definecolor{currentfill}{rgb}{0.000000,0.000000,0.000000}%
\pgfsetfillcolor{currentfill}%
\pgfsetlinewidth{1.003750pt}%
\definecolor{currentstroke}{rgb}{0.000000,0.000000,0.000000}%
\pgfsetstrokecolor{currentstroke}%
\pgfsetdash{}{0pt}%
\pgfpathmoveto{\pgfqpoint{2.944000in}{1.007220in}}%
\pgfpathcurveto{\pgfqpoint{2.950572in}{1.007220in}}{\pgfqpoint{2.956875in}{1.009831in}}{\pgfqpoint{2.961522in}{1.014478in}}%
\pgfpathcurveto{\pgfqpoint{2.966169in}{1.019125in}}{\pgfqpoint{2.968780in}{1.025428in}}{\pgfqpoint{2.968780in}{1.032000in}}%
\pgfpathcurveto{\pgfqpoint{2.968780in}{1.038572in}}{\pgfqpoint{2.966169in}{1.044875in}}{\pgfqpoint{2.961522in}{1.049522in}}%
\pgfpathcurveto{\pgfqpoint{2.956875in}{1.054169in}}{\pgfqpoint{2.950572in}{1.056780in}}{\pgfqpoint{2.944000in}{1.056780in}}%
\pgfpathcurveto{\pgfqpoint{2.937428in}{1.056780in}}{\pgfqpoint{2.931125in}{1.054169in}}{\pgfqpoint{2.926478in}{1.049522in}}%
\pgfpathcurveto{\pgfqpoint{2.921831in}{1.044875in}}{\pgfqpoint{2.919220in}{1.038572in}}{\pgfqpoint{2.919220in}{1.032000in}}%
\pgfpathcurveto{\pgfqpoint{2.919220in}{1.025428in}}{\pgfqpoint{2.921831in}{1.019125in}}{\pgfqpoint{2.926478in}{1.014478in}}%
\pgfpathcurveto{\pgfqpoint{2.931125in}{1.009831in}}{\pgfqpoint{2.937428in}{1.007220in}}{\pgfqpoint{2.944000in}{1.007220in}}%
\pgfpathclose%
\pgfusepath{stroke,fill}%
\end{pgfscope}%
\begin{pgfscope}%
\pgfpathrectangle{\pgfqpoint{1.432000in}{0.528000in}}{\pgfqpoint{3.696000in}{3.696000in}}%
\pgfusepath{clip}%
\pgfsetbuttcap%
\pgfsetroundjoin%
\definecolor{currentfill}{rgb}{0.000000,0.000000,0.000000}%
\pgfsetfillcolor{currentfill}%
\pgfsetlinewidth{1.003750pt}%
\definecolor{currentstroke}{rgb}{0.000000,0.000000,0.000000}%
\pgfsetstrokecolor{currentstroke}%
\pgfsetdash{}{0pt}%
\pgfpathmoveto{\pgfqpoint{2.944000in}{1.343220in}}%
\pgfpathcurveto{\pgfqpoint{2.950572in}{1.343220in}}{\pgfqpoint{2.956875in}{1.345831in}}{\pgfqpoint{2.961522in}{1.350478in}}%
\pgfpathcurveto{\pgfqpoint{2.966169in}{1.355125in}}{\pgfqpoint{2.968780in}{1.361428in}}{\pgfqpoint{2.968780in}{1.368000in}}%
\pgfpathcurveto{\pgfqpoint{2.968780in}{1.374572in}}{\pgfqpoint{2.966169in}{1.380875in}}{\pgfqpoint{2.961522in}{1.385522in}}%
\pgfpathcurveto{\pgfqpoint{2.956875in}{1.390169in}}{\pgfqpoint{2.950572in}{1.392780in}}{\pgfqpoint{2.944000in}{1.392780in}}%
\pgfpathcurveto{\pgfqpoint{2.937428in}{1.392780in}}{\pgfqpoint{2.931125in}{1.390169in}}{\pgfqpoint{2.926478in}{1.385522in}}%
\pgfpathcurveto{\pgfqpoint{2.921831in}{1.380875in}}{\pgfqpoint{2.919220in}{1.374572in}}{\pgfqpoint{2.919220in}{1.368000in}}%
\pgfpathcurveto{\pgfqpoint{2.919220in}{1.361428in}}{\pgfqpoint{2.921831in}{1.355125in}}{\pgfqpoint{2.926478in}{1.350478in}}%
\pgfpathcurveto{\pgfqpoint{2.931125in}{1.345831in}}{\pgfqpoint{2.937428in}{1.343220in}}{\pgfqpoint{2.944000in}{1.343220in}}%
\pgfpathclose%
\pgfusepath{stroke,fill}%
\end{pgfscope}%
\begin{pgfscope}%
\pgfpathrectangle{\pgfqpoint{1.432000in}{0.528000in}}{\pgfqpoint{3.696000in}{3.696000in}}%
\pgfusepath{clip}%
\pgfsetbuttcap%
\pgfsetroundjoin%
\definecolor{currentfill}{rgb}{0.000000,0.000000,0.000000}%
\pgfsetfillcolor{currentfill}%
\pgfsetlinewidth{1.003750pt}%
\definecolor{currentstroke}{rgb}{0.000000,0.000000,0.000000}%
\pgfsetstrokecolor{currentstroke}%
\pgfsetdash{}{0pt}%
\pgfpathmoveto{\pgfqpoint{2.944000in}{1.679220in}}%
\pgfpathcurveto{\pgfqpoint{2.950572in}{1.679220in}}{\pgfqpoint{2.956875in}{1.681831in}}{\pgfqpoint{2.961522in}{1.686478in}}%
\pgfpathcurveto{\pgfqpoint{2.966169in}{1.691125in}}{\pgfqpoint{2.968780in}{1.697428in}}{\pgfqpoint{2.968780in}{1.704000in}}%
\pgfpathcurveto{\pgfqpoint{2.968780in}{1.710572in}}{\pgfqpoint{2.966169in}{1.716875in}}{\pgfqpoint{2.961522in}{1.721522in}}%
\pgfpathcurveto{\pgfqpoint{2.956875in}{1.726169in}}{\pgfqpoint{2.950572in}{1.728780in}}{\pgfqpoint{2.944000in}{1.728780in}}%
\pgfpathcurveto{\pgfqpoint{2.937428in}{1.728780in}}{\pgfqpoint{2.931125in}{1.726169in}}{\pgfqpoint{2.926478in}{1.721522in}}%
\pgfpathcurveto{\pgfqpoint{2.921831in}{1.716875in}}{\pgfqpoint{2.919220in}{1.710572in}}{\pgfqpoint{2.919220in}{1.704000in}}%
\pgfpathcurveto{\pgfqpoint{2.919220in}{1.697428in}}{\pgfqpoint{2.921831in}{1.691125in}}{\pgfqpoint{2.926478in}{1.686478in}}%
\pgfpathcurveto{\pgfqpoint{2.931125in}{1.681831in}}{\pgfqpoint{2.937428in}{1.679220in}}{\pgfqpoint{2.944000in}{1.679220in}}%
\pgfpathclose%
\pgfusepath{stroke,fill}%
\end{pgfscope}%
\begin{pgfscope}%
\pgfpathrectangle{\pgfqpoint{1.432000in}{0.528000in}}{\pgfqpoint{3.696000in}{3.696000in}}%
\pgfusepath{clip}%
\pgfsetbuttcap%
\pgfsetroundjoin%
\definecolor{currentfill}{rgb}{0.000000,0.000000,0.000000}%
\pgfsetfillcolor{currentfill}%
\pgfsetlinewidth{1.003750pt}%
\definecolor{currentstroke}{rgb}{0.000000,0.000000,0.000000}%
\pgfsetstrokecolor{currentstroke}%
\pgfsetdash{}{0pt}%
\pgfpathmoveto{\pgfqpoint{2.944000in}{2.015220in}}%
\pgfpathcurveto{\pgfqpoint{2.950572in}{2.015220in}}{\pgfqpoint{2.956875in}{2.017831in}}{\pgfqpoint{2.961522in}{2.022478in}}%
\pgfpathcurveto{\pgfqpoint{2.966169in}{2.027125in}}{\pgfqpoint{2.968780in}{2.033428in}}{\pgfqpoint{2.968780in}{2.040000in}}%
\pgfpathcurveto{\pgfqpoint{2.968780in}{2.046572in}}{\pgfqpoint{2.966169in}{2.052875in}}{\pgfqpoint{2.961522in}{2.057522in}}%
\pgfpathcurveto{\pgfqpoint{2.956875in}{2.062169in}}{\pgfqpoint{2.950572in}{2.064780in}}{\pgfqpoint{2.944000in}{2.064780in}}%
\pgfpathcurveto{\pgfqpoint{2.937428in}{2.064780in}}{\pgfqpoint{2.931125in}{2.062169in}}{\pgfqpoint{2.926478in}{2.057522in}}%
\pgfpathcurveto{\pgfqpoint{2.921831in}{2.052875in}}{\pgfqpoint{2.919220in}{2.046572in}}{\pgfqpoint{2.919220in}{2.040000in}}%
\pgfpathcurveto{\pgfqpoint{2.919220in}{2.033428in}}{\pgfqpoint{2.921831in}{2.027125in}}{\pgfqpoint{2.926478in}{2.022478in}}%
\pgfpathcurveto{\pgfqpoint{2.931125in}{2.017831in}}{\pgfqpoint{2.937428in}{2.015220in}}{\pgfqpoint{2.944000in}{2.015220in}}%
\pgfpathclose%
\pgfusepath{stroke,fill}%
\end{pgfscope}%
\begin{pgfscope}%
\pgfpathrectangle{\pgfqpoint{1.432000in}{0.528000in}}{\pgfqpoint{3.696000in}{3.696000in}}%
\pgfusepath{clip}%
\pgfsetbuttcap%
\pgfsetroundjoin%
\definecolor{currentfill}{rgb}{0.000000,0.000000,0.000000}%
\pgfsetfillcolor{currentfill}%
\pgfsetlinewidth{1.003750pt}%
\definecolor{currentstroke}{rgb}{0.000000,0.000000,0.000000}%
\pgfsetstrokecolor{currentstroke}%
\pgfsetdash{}{0pt}%
\pgfpathmoveto{\pgfqpoint{2.944000in}{2.351220in}}%
\pgfpathcurveto{\pgfqpoint{2.950572in}{2.351220in}}{\pgfqpoint{2.956875in}{2.353831in}}{\pgfqpoint{2.961522in}{2.358478in}}%
\pgfpathcurveto{\pgfqpoint{2.966169in}{2.363125in}}{\pgfqpoint{2.968780in}{2.369428in}}{\pgfqpoint{2.968780in}{2.376000in}}%
\pgfpathcurveto{\pgfqpoint{2.968780in}{2.382572in}}{\pgfqpoint{2.966169in}{2.388875in}}{\pgfqpoint{2.961522in}{2.393522in}}%
\pgfpathcurveto{\pgfqpoint{2.956875in}{2.398169in}}{\pgfqpoint{2.950572in}{2.400780in}}{\pgfqpoint{2.944000in}{2.400780in}}%
\pgfpathcurveto{\pgfqpoint{2.937428in}{2.400780in}}{\pgfqpoint{2.931125in}{2.398169in}}{\pgfqpoint{2.926478in}{2.393522in}}%
\pgfpathcurveto{\pgfqpoint{2.921831in}{2.388875in}}{\pgfqpoint{2.919220in}{2.382572in}}{\pgfqpoint{2.919220in}{2.376000in}}%
\pgfpathcurveto{\pgfqpoint{2.919220in}{2.369428in}}{\pgfqpoint{2.921831in}{2.363125in}}{\pgfqpoint{2.926478in}{2.358478in}}%
\pgfpathcurveto{\pgfqpoint{2.931125in}{2.353831in}}{\pgfqpoint{2.937428in}{2.351220in}}{\pgfqpoint{2.944000in}{2.351220in}}%
\pgfpathclose%
\pgfusepath{stroke,fill}%
\end{pgfscope}%
\begin{pgfscope}%
\pgfpathrectangle{\pgfqpoint{1.432000in}{0.528000in}}{\pgfqpoint{3.696000in}{3.696000in}}%
\pgfusepath{clip}%
\pgfsetbuttcap%
\pgfsetroundjoin%
\definecolor{currentfill}{rgb}{0.000000,0.000000,0.000000}%
\pgfsetfillcolor{currentfill}%
\pgfsetlinewidth{1.003750pt}%
\definecolor{currentstroke}{rgb}{0.000000,0.000000,0.000000}%
\pgfsetstrokecolor{currentstroke}%
\pgfsetdash{}{0pt}%
\pgfpathmoveto{\pgfqpoint{2.944000in}{2.687220in}}%
\pgfpathcurveto{\pgfqpoint{2.950572in}{2.687220in}}{\pgfqpoint{2.956875in}{2.689831in}}{\pgfqpoint{2.961522in}{2.694478in}}%
\pgfpathcurveto{\pgfqpoint{2.966169in}{2.699125in}}{\pgfqpoint{2.968780in}{2.705428in}}{\pgfqpoint{2.968780in}{2.712000in}}%
\pgfpathcurveto{\pgfqpoint{2.968780in}{2.718572in}}{\pgfqpoint{2.966169in}{2.724875in}}{\pgfqpoint{2.961522in}{2.729522in}}%
\pgfpathcurveto{\pgfqpoint{2.956875in}{2.734169in}}{\pgfqpoint{2.950572in}{2.736780in}}{\pgfqpoint{2.944000in}{2.736780in}}%
\pgfpathcurveto{\pgfqpoint{2.937428in}{2.736780in}}{\pgfqpoint{2.931125in}{2.734169in}}{\pgfqpoint{2.926478in}{2.729522in}}%
\pgfpathcurveto{\pgfqpoint{2.921831in}{2.724875in}}{\pgfqpoint{2.919220in}{2.718572in}}{\pgfqpoint{2.919220in}{2.712000in}}%
\pgfpathcurveto{\pgfqpoint{2.919220in}{2.705428in}}{\pgfqpoint{2.921831in}{2.699125in}}{\pgfqpoint{2.926478in}{2.694478in}}%
\pgfpathcurveto{\pgfqpoint{2.931125in}{2.689831in}}{\pgfqpoint{2.937428in}{2.687220in}}{\pgfqpoint{2.944000in}{2.687220in}}%
\pgfpathclose%
\pgfusepath{stroke,fill}%
\end{pgfscope}%
\begin{pgfscope}%
\pgfpathrectangle{\pgfqpoint{1.432000in}{0.528000in}}{\pgfqpoint{3.696000in}{3.696000in}}%
\pgfusepath{clip}%
\pgfsetbuttcap%
\pgfsetroundjoin%
\definecolor{currentfill}{rgb}{0.000000,0.000000,0.000000}%
\pgfsetfillcolor{currentfill}%
\pgfsetlinewidth{1.003750pt}%
\definecolor{currentstroke}{rgb}{0.000000,0.000000,0.000000}%
\pgfsetstrokecolor{currentstroke}%
\pgfsetdash{}{0pt}%
\pgfpathmoveto{\pgfqpoint{2.944000in}{3.023220in}}%
\pgfpathcurveto{\pgfqpoint{2.950572in}{3.023220in}}{\pgfqpoint{2.956875in}{3.025831in}}{\pgfqpoint{2.961522in}{3.030478in}}%
\pgfpathcurveto{\pgfqpoint{2.966169in}{3.035125in}}{\pgfqpoint{2.968780in}{3.041428in}}{\pgfqpoint{2.968780in}{3.048000in}}%
\pgfpathcurveto{\pgfqpoint{2.968780in}{3.054572in}}{\pgfqpoint{2.966169in}{3.060875in}}{\pgfqpoint{2.961522in}{3.065522in}}%
\pgfpathcurveto{\pgfqpoint{2.956875in}{3.070169in}}{\pgfqpoint{2.950572in}{3.072780in}}{\pgfqpoint{2.944000in}{3.072780in}}%
\pgfpathcurveto{\pgfqpoint{2.937428in}{3.072780in}}{\pgfqpoint{2.931125in}{3.070169in}}{\pgfqpoint{2.926478in}{3.065522in}}%
\pgfpathcurveto{\pgfqpoint{2.921831in}{3.060875in}}{\pgfqpoint{2.919220in}{3.054572in}}{\pgfqpoint{2.919220in}{3.048000in}}%
\pgfpathcurveto{\pgfqpoint{2.919220in}{3.041428in}}{\pgfqpoint{2.921831in}{3.035125in}}{\pgfqpoint{2.926478in}{3.030478in}}%
\pgfpathcurveto{\pgfqpoint{2.931125in}{3.025831in}}{\pgfqpoint{2.937428in}{3.023220in}}{\pgfqpoint{2.944000in}{3.023220in}}%
\pgfpathclose%
\pgfusepath{stroke,fill}%
\end{pgfscope}%
\begin{pgfscope}%
\pgfpathrectangle{\pgfqpoint{1.432000in}{0.528000in}}{\pgfqpoint{3.696000in}{3.696000in}}%
\pgfusepath{clip}%
\pgfsetbuttcap%
\pgfsetroundjoin%
\definecolor{currentfill}{rgb}{0.000000,0.000000,0.000000}%
\pgfsetfillcolor{currentfill}%
\pgfsetlinewidth{1.003750pt}%
\definecolor{currentstroke}{rgb}{0.000000,0.000000,0.000000}%
\pgfsetstrokecolor{currentstroke}%
\pgfsetdash{}{0pt}%
\pgfpathmoveto{\pgfqpoint{2.944000in}{3.359220in}}%
\pgfpathcurveto{\pgfqpoint{2.950572in}{3.359220in}}{\pgfqpoint{2.956875in}{3.361831in}}{\pgfqpoint{2.961522in}{3.366478in}}%
\pgfpathcurveto{\pgfqpoint{2.966169in}{3.371125in}}{\pgfqpoint{2.968780in}{3.377428in}}{\pgfqpoint{2.968780in}{3.384000in}}%
\pgfpathcurveto{\pgfqpoint{2.968780in}{3.390572in}}{\pgfqpoint{2.966169in}{3.396875in}}{\pgfqpoint{2.961522in}{3.401522in}}%
\pgfpathcurveto{\pgfqpoint{2.956875in}{3.406169in}}{\pgfqpoint{2.950572in}{3.408780in}}{\pgfqpoint{2.944000in}{3.408780in}}%
\pgfpathcurveto{\pgfqpoint{2.937428in}{3.408780in}}{\pgfqpoint{2.931125in}{3.406169in}}{\pgfqpoint{2.926478in}{3.401522in}}%
\pgfpathcurveto{\pgfqpoint{2.921831in}{3.396875in}}{\pgfqpoint{2.919220in}{3.390572in}}{\pgfqpoint{2.919220in}{3.384000in}}%
\pgfpathcurveto{\pgfqpoint{2.919220in}{3.377428in}}{\pgfqpoint{2.921831in}{3.371125in}}{\pgfqpoint{2.926478in}{3.366478in}}%
\pgfpathcurveto{\pgfqpoint{2.931125in}{3.361831in}}{\pgfqpoint{2.937428in}{3.359220in}}{\pgfqpoint{2.944000in}{3.359220in}}%
\pgfpathclose%
\pgfusepath{stroke,fill}%
\end{pgfscope}%
\begin{pgfscope}%
\pgfpathrectangle{\pgfqpoint{1.432000in}{0.528000in}}{\pgfqpoint{3.696000in}{3.696000in}}%
\pgfusepath{clip}%
\pgfsetbuttcap%
\pgfsetroundjoin%
\definecolor{currentfill}{rgb}{0.000000,0.000000,0.000000}%
\pgfsetfillcolor{currentfill}%
\pgfsetlinewidth{1.003750pt}%
\definecolor{currentstroke}{rgb}{0.000000,0.000000,0.000000}%
\pgfsetstrokecolor{currentstroke}%
\pgfsetdash{}{0pt}%
\pgfpathmoveto{\pgfqpoint{2.944000in}{3.695220in}}%
\pgfpathcurveto{\pgfqpoint{2.950572in}{3.695220in}}{\pgfqpoint{2.956875in}{3.697831in}}{\pgfqpoint{2.961522in}{3.702478in}}%
\pgfpathcurveto{\pgfqpoint{2.966169in}{3.707125in}}{\pgfqpoint{2.968780in}{3.713428in}}{\pgfqpoint{2.968780in}{3.720000in}}%
\pgfpathcurveto{\pgfqpoint{2.968780in}{3.726572in}}{\pgfqpoint{2.966169in}{3.732875in}}{\pgfqpoint{2.961522in}{3.737522in}}%
\pgfpathcurveto{\pgfqpoint{2.956875in}{3.742169in}}{\pgfqpoint{2.950572in}{3.744780in}}{\pgfqpoint{2.944000in}{3.744780in}}%
\pgfpathcurveto{\pgfqpoint{2.937428in}{3.744780in}}{\pgfqpoint{2.931125in}{3.742169in}}{\pgfqpoint{2.926478in}{3.737522in}}%
\pgfpathcurveto{\pgfqpoint{2.921831in}{3.732875in}}{\pgfqpoint{2.919220in}{3.726572in}}{\pgfqpoint{2.919220in}{3.720000in}}%
\pgfpathcurveto{\pgfqpoint{2.919220in}{3.713428in}}{\pgfqpoint{2.921831in}{3.707125in}}{\pgfqpoint{2.926478in}{3.702478in}}%
\pgfpathcurveto{\pgfqpoint{2.931125in}{3.697831in}}{\pgfqpoint{2.937428in}{3.695220in}}{\pgfqpoint{2.944000in}{3.695220in}}%
\pgfpathclose%
\pgfusepath{stroke,fill}%
\end{pgfscope}%
\begin{pgfscope}%
\pgfpathrectangle{\pgfqpoint{1.432000in}{0.528000in}}{\pgfqpoint{3.696000in}{3.696000in}}%
\pgfusepath{clip}%
\pgfsetbuttcap%
\pgfsetroundjoin%
\definecolor{currentfill}{rgb}{0.000000,0.000000,0.000000}%
\pgfsetfillcolor{currentfill}%
\pgfsetlinewidth{1.003750pt}%
\definecolor{currentstroke}{rgb}{0.000000,0.000000,0.000000}%
\pgfsetstrokecolor{currentstroke}%
\pgfsetdash{}{0pt}%
\pgfpathmoveto{\pgfqpoint{2.944000in}{4.031220in}}%
\pgfpathcurveto{\pgfqpoint{2.950572in}{4.031220in}}{\pgfqpoint{2.956875in}{4.033831in}}{\pgfqpoint{2.961522in}{4.038478in}}%
\pgfpathcurveto{\pgfqpoint{2.966169in}{4.043125in}}{\pgfqpoint{2.968780in}{4.049428in}}{\pgfqpoint{2.968780in}{4.056000in}}%
\pgfpathcurveto{\pgfqpoint{2.968780in}{4.062572in}}{\pgfqpoint{2.966169in}{4.068875in}}{\pgfqpoint{2.961522in}{4.073522in}}%
\pgfpathcurveto{\pgfqpoint{2.956875in}{4.078169in}}{\pgfqpoint{2.950572in}{4.080780in}}{\pgfqpoint{2.944000in}{4.080780in}}%
\pgfpathcurveto{\pgfqpoint{2.937428in}{4.080780in}}{\pgfqpoint{2.931125in}{4.078169in}}{\pgfqpoint{2.926478in}{4.073522in}}%
\pgfpathcurveto{\pgfqpoint{2.921831in}{4.068875in}}{\pgfqpoint{2.919220in}{4.062572in}}{\pgfqpoint{2.919220in}{4.056000in}}%
\pgfpathcurveto{\pgfqpoint{2.919220in}{4.049428in}}{\pgfqpoint{2.921831in}{4.043125in}}{\pgfqpoint{2.926478in}{4.038478in}}%
\pgfpathcurveto{\pgfqpoint{2.931125in}{4.033831in}}{\pgfqpoint{2.937428in}{4.031220in}}{\pgfqpoint{2.944000in}{4.031220in}}%
\pgfpathclose%
\pgfusepath{stroke,fill}%
\end{pgfscope}%
\begin{pgfscope}%
\pgfpathrectangle{\pgfqpoint{1.432000in}{0.528000in}}{\pgfqpoint{3.696000in}{3.696000in}}%
\pgfusepath{clip}%
\pgfsetbuttcap%
\pgfsetroundjoin%
\definecolor{currentfill}{rgb}{0.000000,0.000000,0.000000}%
\pgfsetfillcolor{currentfill}%
\pgfsetlinewidth{1.003750pt}%
\definecolor{currentstroke}{rgb}{0.000000,0.000000,0.000000}%
\pgfsetstrokecolor{currentstroke}%
\pgfsetdash{}{0pt}%
\pgfpathmoveto{\pgfqpoint{3.280000in}{0.671220in}}%
\pgfpathcurveto{\pgfqpoint{3.286572in}{0.671220in}}{\pgfqpoint{3.292875in}{0.673831in}}{\pgfqpoint{3.297522in}{0.678478in}}%
\pgfpathcurveto{\pgfqpoint{3.302169in}{0.683125in}}{\pgfqpoint{3.304780in}{0.689428in}}{\pgfqpoint{3.304780in}{0.696000in}}%
\pgfpathcurveto{\pgfqpoint{3.304780in}{0.702572in}}{\pgfqpoint{3.302169in}{0.708875in}}{\pgfqpoint{3.297522in}{0.713522in}}%
\pgfpathcurveto{\pgfqpoint{3.292875in}{0.718169in}}{\pgfqpoint{3.286572in}{0.720780in}}{\pgfqpoint{3.280000in}{0.720780in}}%
\pgfpathcurveto{\pgfqpoint{3.273428in}{0.720780in}}{\pgfqpoint{3.267125in}{0.718169in}}{\pgfqpoint{3.262478in}{0.713522in}}%
\pgfpathcurveto{\pgfqpoint{3.257831in}{0.708875in}}{\pgfqpoint{3.255220in}{0.702572in}}{\pgfqpoint{3.255220in}{0.696000in}}%
\pgfpathcurveto{\pgfqpoint{3.255220in}{0.689428in}}{\pgfqpoint{3.257831in}{0.683125in}}{\pgfqpoint{3.262478in}{0.678478in}}%
\pgfpathcurveto{\pgfqpoint{3.267125in}{0.673831in}}{\pgfqpoint{3.273428in}{0.671220in}}{\pgfqpoint{3.280000in}{0.671220in}}%
\pgfpathclose%
\pgfusepath{stroke,fill}%
\end{pgfscope}%
\begin{pgfscope}%
\pgfpathrectangle{\pgfqpoint{1.432000in}{0.528000in}}{\pgfqpoint{3.696000in}{3.696000in}}%
\pgfusepath{clip}%
\pgfsetbuttcap%
\pgfsetroundjoin%
\definecolor{currentfill}{rgb}{0.000000,0.000000,0.000000}%
\pgfsetfillcolor{currentfill}%
\pgfsetlinewidth{1.003750pt}%
\definecolor{currentstroke}{rgb}{0.000000,0.000000,0.000000}%
\pgfsetstrokecolor{currentstroke}%
\pgfsetdash{}{0pt}%
\pgfpathmoveto{\pgfqpoint{3.280000in}{1.007220in}}%
\pgfpathcurveto{\pgfqpoint{3.286572in}{1.007220in}}{\pgfqpoint{3.292875in}{1.009831in}}{\pgfqpoint{3.297522in}{1.014478in}}%
\pgfpathcurveto{\pgfqpoint{3.302169in}{1.019125in}}{\pgfqpoint{3.304780in}{1.025428in}}{\pgfqpoint{3.304780in}{1.032000in}}%
\pgfpathcurveto{\pgfqpoint{3.304780in}{1.038572in}}{\pgfqpoint{3.302169in}{1.044875in}}{\pgfqpoint{3.297522in}{1.049522in}}%
\pgfpathcurveto{\pgfqpoint{3.292875in}{1.054169in}}{\pgfqpoint{3.286572in}{1.056780in}}{\pgfqpoint{3.280000in}{1.056780in}}%
\pgfpathcurveto{\pgfqpoint{3.273428in}{1.056780in}}{\pgfqpoint{3.267125in}{1.054169in}}{\pgfqpoint{3.262478in}{1.049522in}}%
\pgfpathcurveto{\pgfqpoint{3.257831in}{1.044875in}}{\pgfqpoint{3.255220in}{1.038572in}}{\pgfqpoint{3.255220in}{1.032000in}}%
\pgfpathcurveto{\pgfqpoint{3.255220in}{1.025428in}}{\pgfqpoint{3.257831in}{1.019125in}}{\pgfqpoint{3.262478in}{1.014478in}}%
\pgfpathcurveto{\pgfqpoint{3.267125in}{1.009831in}}{\pgfqpoint{3.273428in}{1.007220in}}{\pgfqpoint{3.280000in}{1.007220in}}%
\pgfpathclose%
\pgfusepath{stroke,fill}%
\end{pgfscope}%
\begin{pgfscope}%
\pgfpathrectangle{\pgfqpoint{1.432000in}{0.528000in}}{\pgfqpoint{3.696000in}{3.696000in}}%
\pgfusepath{clip}%
\pgfsetbuttcap%
\pgfsetroundjoin%
\definecolor{currentfill}{rgb}{0.000000,0.000000,0.000000}%
\pgfsetfillcolor{currentfill}%
\pgfsetlinewidth{1.003750pt}%
\definecolor{currentstroke}{rgb}{0.000000,0.000000,0.000000}%
\pgfsetstrokecolor{currentstroke}%
\pgfsetdash{}{0pt}%
\pgfpathmoveto{\pgfqpoint{3.280000in}{1.343220in}}%
\pgfpathcurveto{\pgfqpoint{3.286572in}{1.343220in}}{\pgfqpoint{3.292875in}{1.345831in}}{\pgfqpoint{3.297522in}{1.350478in}}%
\pgfpathcurveto{\pgfqpoint{3.302169in}{1.355125in}}{\pgfqpoint{3.304780in}{1.361428in}}{\pgfqpoint{3.304780in}{1.368000in}}%
\pgfpathcurveto{\pgfqpoint{3.304780in}{1.374572in}}{\pgfqpoint{3.302169in}{1.380875in}}{\pgfqpoint{3.297522in}{1.385522in}}%
\pgfpathcurveto{\pgfqpoint{3.292875in}{1.390169in}}{\pgfqpoint{3.286572in}{1.392780in}}{\pgfqpoint{3.280000in}{1.392780in}}%
\pgfpathcurveto{\pgfqpoint{3.273428in}{1.392780in}}{\pgfqpoint{3.267125in}{1.390169in}}{\pgfqpoint{3.262478in}{1.385522in}}%
\pgfpathcurveto{\pgfqpoint{3.257831in}{1.380875in}}{\pgfqpoint{3.255220in}{1.374572in}}{\pgfqpoint{3.255220in}{1.368000in}}%
\pgfpathcurveto{\pgfqpoint{3.255220in}{1.361428in}}{\pgfqpoint{3.257831in}{1.355125in}}{\pgfqpoint{3.262478in}{1.350478in}}%
\pgfpathcurveto{\pgfqpoint{3.267125in}{1.345831in}}{\pgfqpoint{3.273428in}{1.343220in}}{\pgfqpoint{3.280000in}{1.343220in}}%
\pgfpathclose%
\pgfusepath{stroke,fill}%
\end{pgfscope}%
\begin{pgfscope}%
\pgfpathrectangle{\pgfqpoint{1.432000in}{0.528000in}}{\pgfqpoint{3.696000in}{3.696000in}}%
\pgfusepath{clip}%
\pgfsetbuttcap%
\pgfsetroundjoin%
\definecolor{currentfill}{rgb}{0.000000,0.000000,0.000000}%
\pgfsetfillcolor{currentfill}%
\pgfsetlinewidth{1.003750pt}%
\definecolor{currentstroke}{rgb}{0.000000,0.000000,0.000000}%
\pgfsetstrokecolor{currentstroke}%
\pgfsetdash{}{0pt}%
\pgfpathmoveto{\pgfqpoint{3.280000in}{1.679220in}}%
\pgfpathcurveto{\pgfqpoint{3.286572in}{1.679220in}}{\pgfqpoint{3.292875in}{1.681831in}}{\pgfqpoint{3.297522in}{1.686478in}}%
\pgfpathcurveto{\pgfqpoint{3.302169in}{1.691125in}}{\pgfqpoint{3.304780in}{1.697428in}}{\pgfqpoint{3.304780in}{1.704000in}}%
\pgfpathcurveto{\pgfqpoint{3.304780in}{1.710572in}}{\pgfqpoint{3.302169in}{1.716875in}}{\pgfqpoint{3.297522in}{1.721522in}}%
\pgfpathcurveto{\pgfqpoint{3.292875in}{1.726169in}}{\pgfqpoint{3.286572in}{1.728780in}}{\pgfqpoint{3.280000in}{1.728780in}}%
\pgfpathcurveto{\pgfqpoint{3.273428in}{1.728780in}}{\pgfqpoint{3.267125in}{1.726169in}}{\pgfqpoint{3.262478in}{1.721522in}}%
\pgfpathcurveto{\pgfqpoint{3.257831in}{1.716875in}}{\pgfqpoint{3.255220in}{1.710572in}}{\pgfqpoint{3.255220in}{1.704000in}}%
\pgfpathcurveto{\pgfqpoint{3.255220in}{1.697428in}}{\pgfqpoint{3.257831in}{1.691125in}}{\pgfqpoint{3.262478in}{1.686478in}}%
\pgfpathcurveto{\pgfqpoint{3.267125in}{1.681831in}}{\pgfqpoint{3.273428in}{1.679220in}}{\pgfqpoint{3.280000in}{1.679220in}}%
\pgfpathclose%
\pgfusepath{stroke,fill}%
\end{pgfscope}%
\begin{pgfscope}%
\pgfpathrectangle{\pgfqpoint{1.432000in}{0.528000in}}{\pgfqpoint{3.696000in}{3.696000in}}%
\pgfusepath{clip}%
\pgfsetbuttcap%
\pgfsetroundjoin%
\definecolor{currentfill}{rgb}{0.000000,0.000000,0.000000}%
\pgfsetfillcolor{currentfill}%
\pgfsetlinewidth{1.003750pt}%
\definecolor{currentstroke}{rgb}{0.000000,0.000000,0.000000}%
\pgfsetstrokecolor{currentstroke}%
\pgfsetdash{}{0pt}%
\pgfpathmoveto{\pgfqpoint{3.280000in}{2.015220in}}%
\pgfpathcurveto{\pgfqpoint{3.286572in}{2.015220in}}{\pgfqpoint{3.292875in}{2.017831in}}{\pgfqpoint{3.297522in}{2.022478in}}%
\pgfpathcurveto{\pgfqpoint{3.302169in}{2.027125in}}{\pgfqpoint{3.304780in}{2.033428in}}{\pgfqpoint{3.304780in}{2.040000in}}%
\pgfpathcurveto{\pgfqpoint{3.304780in}{2.046572in}}{\pgfqpoint{3.302169in}{2.052875in}}{\pgfqpoint{3.297522in}{2.057522in}}%
\pgfpathcurveto{\pgfqpoint{3.292875in}{2.062169in}}{\pgfqpoint{3.286572in}{2.064780in}}{\pgfqpoint{3.280000in}{2.064780in}}%
\pgfpathcurveto{\pgfqpoint{3.273428in}{2.064780in}}{\pgfqpoint{3.267125in}{2.062169in}}{\pgfqpoint{3.262478in}{2.057522in}}%
\pgfpathcurveto{\pgfqpoint{3.257831in}{2.052875in}}{\pgfqpoint{3.255220in}{2.046572in}}{\pgfqpoint{3.255220in}{2.040000in}}%
\pgfpathcurveto{\pgfqpoint{3.255220in}{2.033428in}}{\pgfqpoint{3.257831in}{2.027125in}}{\pgfqpoint{3.262478in}{2.022478in}}%
\pgfpathcurveto{\pgfqpoint{3.267125in}{2.017831in}}{\pgfqpoint{3.273428in}{2.015220in}}{\pgfqpoint{3.280000in}{2.015220in}}%
\pgfpathclose%
\pgfusepath{stroke,fill}%
\end{pgfscope}%
\begin{pgfscope}%
\pgfpathrectangle{\pgfqpoint{1.432000in}{0.528000in}}{\pgfqpoint{3.696000in}{3.696000in}}%
\pgfusepath{clip}%
\pgfsetbuttcap%
\pgfsetroundjoin%
\definecolor{currentfill}{rgb}{0.000000,0.000000,0.000000}%
\pgfsetfillcolor{currentfill}%
\pgfsetlinewidth{1.003750pt}%
\definecolor{currentstroke}{rgb}{0.000000,0.000000,0.000000}%
\pgfsetstrokecolor{currentstroke}%
\pgfsetdash{}{0pt}%
\pgfpathmoveto{\pgfqpoint{3.280000in}{2.351220in}}%
\pgfpathcurveto{\pgfqpoint{3.286572in}{2.351220in}}{\pgfqpoint{3.292875in}{2.353831in}}{\pgfqpoint{3.297522in}{2.358478in}}%
\pgfpathcurveto{\pgfqpoint{3.302169in}{2.363125in}}{\pgfqpoint{3.304780in}{2.369428in}}{\pgfqpoint{3.304780in}{2.376000in}}%
\pgfpathcurveto{\pgfqpoint{3.304780in}{2.382572in}}{\pgfqpoint{3.302169in}{2.388875in}}{\pgfqpoint{3.297522in}{2.393522in}}%
\pgfpathcurveto{\pgfqpoint{3.292875in}{2.398169in}}{\pgfqpoint{3.286572in}{2.400780in}}{\pgfqpoint{3.280000in}{2.400780in}}%
\pgfpathcurveto{\pgfqpoint{3.273428in}{2.400780in}}{\pgfqpoint{3.267125in}{2.398169in}}{\pgfqpoint{3.262478in}{2.393522in}}%
\pgfpathcurveto{\pgfqpoint{3.257831in}{2.388875in}}{\pgfqpoint{3.255220in}{2.382572in}}{\pgfqpoint{3.255220in}{2.376000in}}%
\pgfpathcurveto{\pgfqpoint{3.255220in}{2.369428in}}{\pgfqpoint{3.257831in}{2.363125in}}{\pgfqpoint{3.262478in}{2.358478in}}%
\pgfpathcurveto{\pgfqpoint{3.267125in}{2.353831in}}{\pgfqpoint{3.273428in}{2.351220in}}{\pgfqpoint{3.280000in}{2.351220in}}%
\pgfpathclose%
\pgfusepath{stroke,fill}%
\end{pgfscope}%
\begin{pgfscope}%
\pgfpathrectangle{\pgfqpoint{1.432000in}{0.528000in}}{\pgfqpoint{3.696000in}{3.696000in}}%
\pgfusepath{clip}%
\pgfsetbuttcap%
\pgfsetroundjoin%
\definecolor{currentfill}{rgb}{0.000000,0.000000,0.000000}%
\pgfsetfillcolor{currentfill}%
\pgfsetlinewidth{1.003750pt}%
\definecolor{currentstroke}{rgb}{0.000000,0.000000,0.000000}%
\pgfsetstrokecolor{currentstroke}%
\pgfsetdash{}{0pt}%
\pgfpathmoveto{\pgfqpoint{3.280000in}{2.687220in}}%
\pgfpathcurveto{\pgfqpoint{3.286572in}{2.687220in}}{\pgfqpoint{3.292875in}{2.689831in}}{\pgfqpoint{3.297522in}{2.694478in}}%
\pgfpathcurveto{\pgfqpoint{3.302169in}{2.699125in}}{\pgfqpoint{3.304780in}{2.705428in}}{\pgfqpoint{3.304780in}{2.712000in}}%
\pgfpathcurveto{\pgfqpoint{3.304780in}{2.718572in}}{\pgfqpoint{3.302169in}{2.724875in}}{\pgfqpoint{3.297522in}{2.729522in}}%
\pgfpathcurveto{\pgfqpoint{3.292875in}{2.734169in}}{\pgfqpoint{3.286572in}{2.736780in}}{\pgfqpoint{3.280000in}{2.736780in}}%
\pgfpathcurveto{\pgfqpoint{3.273428in}{2.736780in}}{\pgfqpoint{3.267125in}{2.734169in}}{\pgfqpoint{3.262478in}{2.729522in}}%
\pgfpathcurveto{\pgfqpoint{3.257831in}{2.724875in}}{\pgfqpoint{3.255220in}{2.718572in}}{\pgfqpoint{3.255220in}{2.712000in}}%
\pgfpathcurveto{\pgfqpoint{3.255220in}{2.705428in}}{\pgfqpoint{3.257831in}{2.699125in}}{\pgfqpoint{3.262478in}{2.694478in}}%
\pgfpathcurveto{\pgfqpoint{3.267125in}{2.689831in}}{\pgfqpoint{3.273428in}{2.687220in}}{\pgfqpoint{3.280000in}{2.687220in}}%
\pgfpathclose%
\pgfusepath{stroke,fill}%
\end{pgfscope}%
\begin{pgfscope}%
\pgfpathrectangle{\pgfqpoint{1.432000in}{0.528000in}}{\pgfqpoint{3.696000in}{3.696000in}}%
\pgfusepath{clip}%
\pgfsetbuttcap%
\pgfsetroundjoin%
\definecolor{currentfill}{rgb}{0.000000,0.000000,0.000000}%
\pgfsetfillcolor{currentfill}%
\pgfsetlinewidth{1.003750pt}%
\definecolor{currentstroke}{rgb}{0.000000,0.000000,0.000000}%
\pgfsetstrokecolor{currentstroke}%
\pgfsetdash{}{0pt}%
\pgfpathmoveto{\pgfqpoint{3.280000in}{3.023220in}}%
\pgfpathcurveto{\pgfqpoint{3.286572in}{3.023220in}}{\pgfqpoint{3.292875in}{3.025831in}}{\pgfqpoint{3.297522in}{3.030478in}}%
\pgfpathcurveto{\pgfqpoint{3.302169in}{3.035125in}}{\pgfqpoint{3.304780in}{3.041428in}}{\pgfqpoint{3.304780in}{3.048000in}}%
\pgfpathcurveto{\pgfqpoint{3.304780in}{3.054572in}}{\pgfqpoint{3.302169in}{3.060875in}}{\pgfqpoint{3.297522in}{3.065522in}}%
\pgfpathcurveto{\pgfqpoint{3.292875in}{3.070169in}}{\pgfqpoint{3.286572in}{3.072780in}}{\pgfqpoint{3.280000in}{3.072780in}}%
\pgfpathcurveto{\pgfqpoint{3.273428in}{3.072780in}}{\pgfqpoint{3.267125in}{3.070169in}}{\pgfqpoint{3.262478in}{3.065522in}}%
\pgfpathcurveto{\pgfqpoint{3.257831in}{3.060875in}}{\pgfqpoint{3.255220in}{3.054572in}}{\pgfqpoint{3.255220in}{3.048000in}}%
\pgfpathcurveto{\pgfqpoint{3.255220in}{3.041428in}}{\pgfqpoint{3.257831in}{3.035125in}}{\pgfqpoint{3.262478in}{3.030478in}}%
\pgfpathcurveto{\pgfqpoint{3.267125in}{3.025831in}}{\pgfqpoint{3.273428in}{3.023220in}}{\pgfqpoint{3.280000in}{3.023220in}}%
\pgfpathclose%
\pgfusepath{stroke,fill}%
\end{pgfscope}%
\begin{pgfscope}%
\pgfpathrectangle{\pgfqpoint{1.432000in}{0.528000in}}{\pgfqpoint{3.696000in}{3.696000in}}%
\pgfusepath{clip}%
\pgfsetbuttcap%
\pgfsetroundjoin%
\definecolor{currentfill}{rgb}{0.000000,0.000000,0.000000}%
\pgfsetfillcolor{currentfill}%
\pgfsetlinewidth{1.003750pt}%
\definecolor{currentstroke}{rgb}{0.000000,0.000000,0.000000}%
\pgfsetstrokecolor{currentstroke}%
\pgfsetdash{}{0pt}%
\pgfpathmoveto{\pgfqpoint{3.280000in}{3.359220in}}%
\pgfpathcurveto{\pgfqpoint{3.286572in}{3.359220in}}{\pgfqpoint{3.292875in}{3.361831in}}{\pgfqpoint{3.297522in}{3.366478in}}%
\pgfpathcurveto{\pgfqpoint{3.302169in}{3.371125in}}{\pgfqpoint{3.304780in}{3.377428in}}{\pgfqpoint{3.304780in}{3.384000in}}%
\pgfpathcurveto{\pgfqpoint{3.304780in}{3.390572in}}{\pgfqpoint{3.302169in}{3.396875in}}{\pgfqpoint{3.297522in}{3.401522in}}%
\pgfpathcurveto{\pgfqpoint{3.292875in}{3.406169in}}{\pgfqpoint{3.286572in}{3.408780in}}{\pgfqpoint{3.280000in}{3.408780in}}%
\pgfpathcurveto{\pgfqpoint{3.273428in}{3.408780in}}{\pgfqpoint{3.267125in}{3.406169in}}{\pgfqpoint{3.262478in}{3.401522in}}%
\pgfpathcurveto{\pgfqpoint{3.257831in}{3.396875in}}{\pgfqpoint{3.255220in}{3.390572in}}{\pgfqpoint{3.255220in}{3.384000in}}%
\pgfpathcurveto{\pgfqpoint{3.255220in}{3.377428in}}{\pgfqpoint{3.257831in}{3.371125in}}{\pgfqpoint{3.262478in}{3.366478in}}%
\pgfpathcurveto{\pgfqpoint{3.267125in}{3.361831in}}{\pgfqpoint{3.273428in}{3.359220in}}{\pgfqpoint{3.280000in}{3.359220in}}%
\pgfpathclose%
\pgfusepath{stroke,fill}%
\end{pgfscope}%
\begin{pgfscope}%
\pgfpathrectangle{\pgfqpoint{1.432000in}{0.528000in}}{\pgfqpoint{3.696000in}{3.696000in}}%
\pgfusepath{clip}%
\pgfsetbuttcap%
\pgfsetroundjoin%
\definecolor{currentfill}{rgb}{0.000000,0.000000,0.000000}%
\pgfsetfillcolor{currentfill}%
\pgfsetlinewidth{1.003750pt}%
\definecolor{currentstroke}{rgb}{0.000000,0.000000,0.000000}%
\pgfsetstrokecolor{currentstroke}%
\pgfsetdash{}{0pt}%
\pgfpathmoveto{\pgfqpoint{3.280000in}{3.695220in}}%
\pgfpathcurveto{\pgfqpoint{3.286572in}{3.695220in}}{\pgfqpoint{3.292875in}{3.697831in}}{\pgfqpoint{3.297522in}{3.702478in}}%
\pgfpathcurveto{\pgfqpoint{3.302169in}{3.707125in}}{\pgfqpoint{3.304780in}{3.713428in}}{\pgfqpoint{3.304780in}{3.720000in}}%
\pgfpathcurveto{\pgfqpoint{3.304780in}{3.726572in}}{\pgfqpoint{3.302169in}{3.732875in}}{\pgfqpoint{3.297522in}{3.737522in}}%
\pgfpathcurveto{\pgfqpoint{3.292875in}{3.742169in}}{\pgfqpoint{3.286572in}{3.744780in}}{\pgfqpoint{3.280000in}{3.744780in}}%
\pgfpathcurveto{\pgfqpoint{3.273428in}{3.744780in}}{\pgfqpoint{3.267125in}{3.742169in}}{\pgfqpoint{3.262478in}{3.737522in}}%
\pgfpathcurveto{\pgfqpoint{3.257831in}{3.732875in}}{\pgfqpoint{3.255220in}{3.726572in}}{\pgfqpoint{3.255220in}{3.720000in}}%
\pgfpathcurveto{\pgfqpoint{3.255220in}{3.713428in}}{\pgfqpoint{3.257831in}{3.707125in}}{\pgfqpoint{3.262478in}{3.702478in}}%
\pgfpathcurveto{\pgfqpoint{3.267125in}{3.697831in}}{\pgfqpoint{3.273428in}{3.695220in}}{\pgfqpoint{3.280000in}{3.695220in}}%
\pgfpathclose%
\pgfusepath{stroke,fill}%
\end{pgfscope}%
\begin{pgfscope}%
\pgfpathrectangle{\pgfqpoint{1.432000in}{0.528000in}}{\pgfqpoint{3.696000in}{3.696000in}}%
\pgfusepath{clip}%
\pgfsetbuttcap%
\pgfsetroundjoin%
\definecolor{currentfill}{rgb}{0.000000,0.000000,0.000000}%
\pgfsetfillcolor{currentfill}%
\pgfsetlinewidth{1.003750pt}%
\definecolor{currentstroke}{rgb}{0.000000,0.000000,0.000000}%
\pgfsetstrokecolor{currentstroke}%
\pgfsetdash{}{0pt}%
\pgfpathmoveto{\pgfqpoint{3.280000in}{4.031220in}}%
\pgfpathcurveto{\pgfqpoint{3.286572in}{4.031220in}}{\pgfqpoint{3.292875in}{4.033831in}}{\pgfqpoint{3.297522in}{4.038478in}}%
\pgfpathcurveto{\pgfqpoint{3.302169in}{4.043125in}}{\pgfqpoint{3.304780in}{4.049428in}}{\pgfqpoint{3.304780in}{4.056000in}}%
\pgfpathcurveto{\pgfqpoint{3.304780in}{4.062572in}}{\pgfqpoint{3.302169in}{4.068875in}}{\pgfqpoint{3.297522in}{4.073522in}}%
\pgfpathcurveto{\pgfqpoint{3.292875in}{4.078169in}}{\pgfqpoint{3.286572in}{4.080780in}}{\pgfqpoint{3.280000in}{4.080780in}}%
\pgfpathcurveto{\pgfqpoint{3.273428in}{4.080780in}}{\pgfqpoint{3.267125in}{4.078169in}}{\pgfqpoint{3.262478in}{4.073522in}}%
\pgfpathcurveto{\pgfqpoint{3.257831in}{4.068875in}}{\pgfqpoint{3.255220in}{4.062572in}}{\pgfqpoint{3.255220in}{4.056000in}}%
\pgfpathcurveto{\pgfqpoint{3.255220in}{4.049428in}}{\pgfqpoint{3.257831in}{4.043125in}}{\pgfqpoint{3.262478in}{4.038478in}}%
\pgfpathcurveto{\pgfqpoint{3.267125in}{4.033831in}}{\pgfqpoint{3.273428in}{4.031220in}}{\pgfqpoint{3.280000in}{4.031220in}}%
\pgfpathclose%
\pgfusepath{stroke,fill}%
\end{pgfscope}%
\begin{pgfscope}%
\pgfpathrectangle{\pgfqpoint{1.432000in}{0.528000in}}{\pgfqpoint{3.696000in}{3.696000in}}%
\pgfusepath{clip}%
\pgfsetbuttcap%
\pgfsetroundjoin%
\definecolor{currentfill}{rgb}{0.000000,0.000000,0.000000}%
\pgfsetfillcolor{currentfill}%
\pgfsetlinewidth{1.003750pt}%
\definecolor{currentstroke}{rgb}{0.000000,0.000000,0.000000}%
\pgfsetstrokecolor{currentstroke}%
\pgfsetdash{}{0pt}%
\pgfpathmoveto{\pgfqpoint{3.616000in}{0.671220in}}%
\pgfpathcurveto{\pgfqpoint{3.622572in}{0.671220in}}{\pgfqpoint{3.628875in}{0.673831in}}{\pgfqpoint{3.633522in}{0.678478in}}%
\pgfpathcurveto{\pgfqpoint{3.638169in}{0.683125in}}{\pgfqpoint{3.640780in}{0.689428in}}{\pgfqpoint{3.640780in}{0.696000in}}%
\pgfpathcurveto{\pgfqpoint{3.640780in}{0.702572in}}{\pgfqpoint{3.638169in}{0.708875in}}{\pgfqpoint{3.633522in}{0.713522in}}%
\pgfpathcurveto{\pgfqpoint{3.628875in}{0.718169in}}{\pgfqpoint{3.622572in}{0.720780in}}{\pgfqpoint{3.616000in}{0.720780in}}%
\pgfpathcurveto{\pgfqpoint{3.609428in}{0.720780in}}{\pgfqpoint{3.603125in}{0.718169in}}{\pgfqpoint{3.598478in}{0.713522in}}%
\pgfpathcurveto{\pgfqpoint{3.593831in}{0.708875in}}{\pgfqpoint{3.591220in}{0.702572in}}{\pgfqpoint{3.591220in}{0.696000in}}%
\pgfpathcurveto{\pgfqpoint{3.591220in}{0.689428in}}{\pgfqpoint{3.593831in}{0.683125in}}{\pgfqpoint{3.598478in}{0.678478in}}%
\pgfpathcurveto{\pgfqpoint{3.603125in}{0.673831in}}{\pgfqpoint{3.609428in}{0.671220in}}{\pgfqpoint{3.616000in}{0.671220in}}%
\pgfpathclose%
\pgfusepath{stroke,fill}%
\end{pgfscope}%
\begin{pgfscope}%
\pgfpathrectangle{\pgfqpoint{1.432000in}{0.528000in}}{\pgfqpoint{3.696000in}{3.696000in}}%
\pgfusepath{clip}%
\pgfsetbuttcap%
\pgfsetroundjoin%
\definecolor{currentfill}{rgb}{0.000000,0.000000,0.000000}%
\pgfsetfillcolor{currentfill}%
\pgfsetlinewidth{1.003750pt}%
\definecolor{currentstroke}{rgb}{0.000000,0.000000,0.000000}%
\pgfsetstrokecolor{currentstroke}%
\pgfsetdash{}{0pt}%
\pgfpathmoveto{\pgfqpoint{3.616000in}{1.007220in}}%
\pgfpathcurveto{\pgfqpoint{3.622572in}{1.007220in}}{\pgfqpoint{3.628875in}{1.009831in}}{\pgfqpoint{3.633522in}{1.014478in}}%
\pgfpathcurveto{\pgfqpoint{3.638169in}{1.019125in}}{\pgfqpoint{3.640780in}{1.025428in}}{\pgfqpoint{3.640780in}{1.032000in}}%
\pgfpathcurveto{\pgfqpoint{3.640780in}{1.038572in}}{\pgfqpoint{3.638169in}{1.044875in}}{\pgfqpoint{3.633522in}{1.049522in}}%
\pgfpathcurveto{\pgfqpoint{3.628875in}{1.054169in}}{\pgfqpoint{3.622572in}{1.056780in}}{\pgfqpoint{3.616000in}{1.056780in}}%
\pgfpathcurveto{\pgfqpoint{3.609428in}{1.056780in}}{\pgfqpoint{3.603125in}{1.054169in}}{\pgfqpoint{3.598478in}{1.049522in}}%
\pgfpathcurveto{\pgfqpoint{3.593831in}{1.044875in}}{\pgfqpoint{3.591220in}{1.038572in}}{\pgfqpoint{3.591220in}{1.032000in}}%
\pgfpathcurveto{\pgfqpoint{3.591220in}{1.025428in}}{\pgfqpoint{3.593831in}{1.019125in}}{\pgfqpoint{3.598478in}{1.014478in}}%
\pgfpathcurveto{\pgfqpoint{3.603125in}{1.009831in}}{\pgfqpoint{3.609428in}{1.007220in}}{\pgfqpoint{3.616000in}{1.007220in}}%
\pgfpathclose%
\pgfusepath{stroke,fill}%
\end{pgfscope}%
\begin{pgfscope}%
\pgfpathrectangle{\pgfqpoint{1.432000in}{0.528000in}}{\pgfqpoint{3.696000in}{3.696000in}}%
\pgfusepath{clip}%
\pgfsetbuttcap%
\pgfsetroundjoin%
\definecolor{currentfill}{rgb}{0.000000,0.000000,0.000000}%
\pgfsetfillcolor{currentfill}%
\pgfsetlinewidth{1.003750pt}%
\definecolor{currentstroke}{rgb}{0.000000,0.000000,0.000000}%
\pgfsetstrokecolor{currentstroke}%
\pgfsetdash{}{0pt}%
\pgfpathmoveto{\pgfqpoint{3.616000in}{1.343220in}}%
\pgfpathcurveto{\pgfqpoint{3.622572in}{1.343220in}}{\pgfqpoint{3.628875in}{1.345831in}}{\pgfqpoint{3.633522in}{1.350478in}}%
\pgfpathcurveto{\pgfqpoint{3.638169in}{1.355125in}}{\pgfqpoint{3.640780in}{1.361428in}}{\pgfqpoint{3.640780in}{1.368000in}}%
\pgfpathcurveto{\pgfqpoint{3.640780in}{1.374572in}}{\pgfqpoint{3.638169in}{1.380875in}}{\pgfqpoint{3.633522in}{1.385522in}}%
\pgfpathcurveto{\pgfqpoint{3.628875in}{1.390169in}}{\pgfqpoint{3.622572in}{1.392780in}}{\pgfqpoint{3.616000in}{1.392780in}}%
\pgfpathcurveto{\pgfqpoint{3.609428in}{1.392780in}}{\pgfqpoint{3.603125in}{1.390169in}}{\pgfqpoint{3.598478in}{1.385522in}}%
\pgfpathcurveto{\pgfqpoint{3.593831in}{1.380875in}}{\pgfqpoint{3.591220in}{1.374572in}}{\pgfqpoint{3.591220in}{1.368000in}}%
\pgfpathcurveto{\pgfqpoint{3.591220in}{1.361428in}}{\pgfqpoint{3.593831in}{1.355125in}}{\pgfqpoint{3.598478in}{1.350478in}}%
\pgfpathcurveto{\pgfqpoint{3.603125in}{1.345831in}}{\pgfqpoint{3.609428in}{1.343220in}}{\pgfqpoint{3.616000in}{1.343220in}}%
\pgfpathclose%
\pgfusepath{stroke,fill}%
\end{pgfscope}%
\begin{pgfscope}%
\pgfpathrectangle{\pgfqpoint{1.432000in}{0.528000in}}{\pgfqpoint{3.696000in}{3.696000in}}%
\pgfusepath{clip}%
\pgfsetbuttcap%
\pgfsetroundjoin%
\definecolor{currentfill}{rgb}{0.000000,0.000000,0.000000}%
\pgfsetfillcolor{currentfill}%
\pgfsetlinewidth{1.003750pt}%
\definecolor{currentstroke}{rgb}{0.000000,0.000000,0.000000}%
\pgfsetstrokecolor{currentstroke}%
\pgfsetdash{}{0pt}%
\pgfpathmoveto{\pgfqpoint{3.616000in}{1.679220in}}%
\pgfpathcurveto{\pgfqpoint{3.622572in}{1.679220in}}{\pgfqpoint{3.628875in}{1.681831in}}{\pgfqpoint{3.633522in}{1.686478in}}%
\pgfpathcurveto{\pgfqpoint{3.638169in}{1.691125in}}{\pgfqpoint{3.640780in}{1.697428in}}{\pgfqpoint{3.640780in}{1.704000in}}%
\pgfpathcurveto{\pgfqpoint{3.640780in}{1.710572in}}{\pgfqpoint{3.638169in}{1.716875in}}{\pgfqpoint{3.633522in}{1.721522in}}%
\pgfpathcurveto{\pgfqpoint{3.628875in}{1.726169in}}{\pgfqpoint{3.622572in}{1.728780in}}{\pgfqpoint{3.616000in}{1.728780in}}%
\pgfpathcurveto{\pgfqpoint{3.609428in}{1.728780in}}{\pgfqpoint{3.603125in}{1.726169in}}{\pgfqpoint{3.598478in}{1.721522in}}%
\pgfpathcurveto{\pgfqpoint{3.593831in}{1.716875in}}{\pgfqpoint{3.591220in}{1.710572in}}{\pgfqpoint{3.591220in}{1.704000in}}%
\pgfpathcurveto{\pgfqpoint{3.591220in}{1.697428in}}{\pgfqpoint{3.593831in}{1.691125in}}{\pgfqpoint{3.598478in}{1.686478in}}%
\pgfpathcurveto{\pgfqpoint{3.603125in}{1.681831in}}{\pgfqpoint{3.609428in}{1.679220in}}{\pgfqpoint{3.616000in}{1.679220in}}%
\pgfpathclose%
\pgfusepath{stroke,fill}%
\end{pgfscope}%
\begin{pgfscope}%
\pgfpathrectangle{\pgfqpoint{1.432000in}{0.528000in}}{\pgfqpoint{3.696000in}{3.696000in}}%
\pgfusepath{clip}%
\pgfsetbuttcap%
\pgfsetroundjoin%
\definecolor{currentfill}{rgb}{0.000000,0.000000,0.000000}%
\pgfsetfillcolor{currentfill}%
\pgfsetlinewidth{1.003750pt}%
\definecolor{currentstroke}{rgb}{0.000000,0.000000,0.000000}%
\pgfsetstrokecolor{currentstroke}%
\pgfsetdash{}{0pt}%
\pgfpathmoveto{\pgfqpoint{3.616000in}{2.015220in}}%
\pgfpathcurveto{\pgfqpoint{3.622572in}{2.015220in}}{\pgfqpoint{3.628875in}{2.017831in}}{\pgfqpoint{3.633522in}{2.022478in}}%
\pgfpathcurveto{\pgfqpoint{3.638169in}{2.027125in}}{\pgfqpoint{3.640780in}{2.033428in}}{\pgfqpoint{3.640780in}{2.040000in}}%
\pgfpathcurveto{\pgfqpoint{3.640780in}{2.046572in}}{\pgfqpoint{3.638169in}{2.052875in}}{\pgfqpoint{3.633522in}{2.057522in}}%
\pgfpathcurveto{\pgfqpoint{3.628875in}{2.062169in}}{\pgfqpoint{3.622572in}{2.064780in}}{\pgfqpoint{3.616000in}{2.064780in}}%
\pgfpathcurveto{\pgfqpoint{3.609428in}{2.064780in}}{\pgfqpoint{3.603125in}{2.062169in}}{\pgfqpoint{3.598478in}{2.057522in}}%
\pgfpathcurveto{\pgfqpoint{3.593831in}{2.052875in}}{\pgfqpoint{3.591220in}{2.046572in}}{\pgfqpoint{3.591220in}{2.040000in}}%
\pgfpathcurveto{\pgfqpoint{3.591220in}{2.033428in}}{\pgfqpoint{3.593831in}{2.027125in}}{\pgfqpoint{3.598478in}{2.022478in}}%
\pgfpathcurveto{\pgfqpoint{3.603125in}{2.017831in}}{\pgfqpoint{3.609428in}{2.015220in}}{\pgfqpoint{3.616000in}{2.015220in}}%
\pgfpathclose%
\pgfusepath{stroke,fill}%
\end{pgfscope}%
\begin{pgfscope}%
\pgfpathrectangle{\pgfqpoint{1.432000in}{0.528000in}}{\pgfqpoint{3.696000in}{3.696000in}}%
\pgfusepath{clip}%
\pgfsetbuttcap%
\pgfsetroundjoin%
\definecolor{currentfill}{rgb}{0.000000,0.000000,0.000000}%
\pgfsetfillcolor{currentfill}%
\pgfsetlinewidth{1.003750pt}%
\definecolor{currentstroke}{rgb}{0.000000,0.000000,0.000000}%
\pgfsetstrokecolor{currentstroke}%
\pgfsetdash{}{0pt}%
\pgfpathmoveto{\pgfqpoint{3.616000in}{2.351220in}}%
\pgfpathcurveto{\pgfqpoint{3.622572in}{2.351220in}}{\pgfqpoint{3.628875in}{2.353831in}}{\pgfqpoint{3.633522in}{2.358478in}}%
\pgfpathcurveto{\pgfqpoint{3.638169in}{2.363125in}}{\pgfqpoint{3.640780in}{2.369428in}}{\pgfqpoint{3.640780in}{2.376000in}}%
\pgfpathcurveto{\pgfqpoint{3.640780in}{2.382572in}}{\pgfqpoint{3.638169in}{2.388875in}}{\pgfqpoint{3.633522in}{2.393522in}}%
\pgfpathcurveto{\pgfqpoint{3.628875in}{2.398169in}}{\pgfqpoint{3.622572in}{2.400780in}}{\pgfqpoint{3.616000in}{2.400780in}}%
\pgfpathcurveto{\pgfqpoint{3.609428in}{2.400780in}}{\pgfqpoint{3.603125in}{2.398169in}}{\pgfqpoint{3.598478in}{2.393522in}}%
\pgfpathcurveto{\pgfqpoint{3.593831in}{2.388875in}}{\pgfqpoint{3.591220in}{2.382572in}}{\pgfqpoint{3.591220in}{2.376000in}}%
\pgfpathcurveto{\pgfqpoint{3.591220in}{2.369428in}}{\pgfqpoint{3.593831in}{2.363125in}}{\pgfqpoint{3.598478in}{2.358478in}}%
\pgfpathcurveto{\pgfqpoint{3.603125in}{2.353831in}}{\pgfqpoint{3.609428in}{2.351220in}}{\pgfqpoint{3.616000in}{2.351220in}}%
\pgfpathclose%
\pgfusepath{stroke,fill}%
\end{pgfscope}%
\begin{pgfscope}%
\pgfpathrectangle{\pgfqpoint{1.432000in}{0.528000in}}{\pgfqpoint{3.696000in}{3.696000in}}%
\pgfusepath{clip}%
\pgfsetbuttcap%
\pgfsetroundjoin%
\definecolor{currentfill}{rgb}{0.000000,0.000000,0.000000}%
\pgfsetfillcolor{currentfill}%
\pgfsetlinewidth{1.003750pt}%
\definecolor{currentstroke}{rgb}{0.000000,0.000000,0.000000}%
\pgfsetstrokecolor{currentstroke}%
\pgfsetdash{}{0pt}%
\pgfpathmoveto{\pgfqpoint{3.616000in}{2.687220in}}%
\pgfpathcurveto{\pgfqpoint{3.622572in}{2.687220in}}{\pgfqpoint{3.628875in}{2.689831in}}{\pgfqpoint{3.633522in}{2.694478in}}%
\pgfpathcurveto{\pgfqpoint{3.638169in}{2.699125in}}{\pgfqpoint{3.640780in}{2.705428in}}{\pgfqpoint{3.640780in}{2.712000in}}%
\pgfpathcurveto{\pgfqpoint{3.640780in}{2.718572in}}{\pgfqpoint{3.638169in}{2.724875in}}{\pgfqpoint{3.633522in}{2.729522in}}%
\pgfpathcurveto{\pgfqpoint{3.628875in}{2.734169in}}{\pgfqpoint{3.622572in}{2.736780in}}{\pgfqpoint{3.616000in}{2.736780in}}%
\pgfpathcurveto{\pgfqpoint{3.609428in}{2.736780in}}{\pgfqpoint{3.603125in}{2.734169in}}{\pgfqpoint{3.598478in}{2.729522in}}%
\pgfpathcurveto{\pgfqpoint{3.593831in}{2.724875in}}{\pgfqpoint{3.591220in}{2.718572in}}{\pgfqpoint{3.591220in}{2.712000in}}%
\pgfpathcurveto{\pgfqpoint{3.591220in}{2.705428in}}{\pgfqpoint{3.593831in}{2.699125in}}{\pgfqpoint{3.598478in}{2.694478in}}%
\pgfpathcurveto{\pgfqpoint{3.603125in}{2.689831in}}{\pgfqpoint{3.609428in}{2.687220in}}{\pgfqpoint{3.616000in}{2.687220in}}%
\pgfpathclose%
\pgfusepath{stroke,fill}%
\end{pgfscope}%
\begin{pgfscope}%
\pgfpathrectangle{\pgfqpoint{1.432000in}{0.528000in}}{\pgfqpoint{3.696000in}{3.696000in}}%
\pgfusepath{clip}%
\pgfsetbuttcap%
\pgfsetroundjoin%
\definecolor{currentfill}{rgb}{0.000000,0.000000,0.000000}%
\pgfsetfillcolor{currentfill}%
\pgfsetlinewidth{1.003750pt}%
\definecolor{currentstroke}{rgb}{0.000000,0.000000,0.000000}%
\pgfsetstrokecolor{currentstroke}%
\pgfsetdash{}{0pt}%
\pgfpathmoveto{\pgfqpoint{3.616000in}{3.023220in}}%
\pgfpathcurveto{\pgfqpoint{3.622572in}{3.023220in}}{\pgfqpoint{3.628875in}{3.025831in}}{\pgfqpoint{3.633522in}{3.030478in}}%
\pgfpathcurveto{\pgfqpoint{3.638169in}{3.035125in}}{\pgfqpoint{3.640780in}{3.041428in}}{\pgfqpoint{3.640780in}{3.048000in}}%
\pgfpathcurveto{\pgfqpoint{3.640780in}{3.054572in}}{\pgfqpoint{3.638169in}{3.060875in}}{\pgfqpoint{3.633522in}{3.065522in}}%
\pgfpathcurveto{\pgfqpoint{3.628875in}{3.070169in}}{\pgfqpoint{3.622572in}{3.072780in}}{\pgfqpoint{3.616000in}{3.072780in}}%
\pgfpathcurveto{\pgfqpoint{3.609428in}{3.072780in}}{\pgfqpoint{3.603125in}{3.070169in}}{\pgfqpoint{3.598478in}{3.065522in}}%
\pgfpathcurveto{\pgfqpoint{3.593831in}{3.060875in}}{\pgfqpoint{3.591220in}{3.054572in}}{\pgfqpoint{3.591220in}{3.048000in}}%
\pgfpathcurveto{\pgfqpoint{3.591220in}{3.041428in}}{\pgfqpoint{3.593831in}{3.035125in}}{\pgfqpoint{3.598478in}{3.030478in}}%
\pgfpathcurveto{\pgfqpoint{3.603125in}{3.025831in}}{\pgfqpoint{3.609428in}{3.023220in}}{\pgfqpoint{3.616000in}{3.023220in}}%
\pgfpathclose%
\pgfusepath{stroke,fill}%
\end{pgfscope}%
\begin{pgfscope}%
\pgfpathrectangle{\pgfqpoint{1.432000in}{0.528000in}}{\pgfqpoint{3.696000in}{3.696000in}}%
\pgfusepath{clip}%
\pgfsetbuttcap%
\pgfsetroundjoin%
\definecolor{currentfill}{rgb}{0.000000,0.000000,0.000000}%
\pgfsetfillcolor{currentfill}%
\pgfsetlinewidth{1.003750pt}%
\definecolor{currentstroke}{rgb}{0.000000,0.000000,0.000000}%
\pgfsetstrokecolor{currentstroke}%
\pgfsetdash{}{0pt}%
\pgfpathmoveto{\pgfqpoint{3.616000in}{3.359220in}}%
\pgfpathcurveto{\pgfqpoint{3.622572in}{3.359220in}}{\pgfqpoint{3.628875in}{3.361831in}}{\pgfqpoint{3.633522in}{3.366478in}}%
\pgfpathcurveto{\pgfqpoint{3.638169in}{3.371125in}}{\pgfqpoint{3.640780in}{3.377428in}}{\pgfqpoint{3.640780in}{3.384000in}}%
\pgfpathcurveto{\pgfqpoint{3.640780in}{3.390572in}}{\pgfqpoint{3.638169in}{3.396875in}}{\pgfqpoint{3.633522in}{3.401522in}}%
\pgfpathcurveto{\pgfqpoint{3.628875in}{3.406169in}}{\pgfqpoint{3.622572in}{3.408780in}}{\pgfqpoint{3.616000in}{3.408780in}}%
\pgfpathcurveto{\pgfqpoint{3.609428in}{3.408780in}}{\pgfqpoint{3.603125in}{3.406169in}}{\pgfqpoint{3.598478in}{3.401522in}}%
\pgfpathcurveto{\pgfqpoint{3.593831in}{3.396875in}}{\pgfqpoint{3.591220in}{3.390572in}}{\pgfqpoint{3.591220in}{3.384000in}}%
\pgfpathcurveto{\pgfqpoint{3.591220in}{3.377428in}}{\pgfqpoint{3.593831in}{3.371125in}}{\pgfqpoint{3.598478in}{3.366478in}}%
\pgfpathcurveto{\pgfqpoint{3.603125in}{3.361831in}}{\pgfqpoint{3.609428in}{3.359220in}}{\pgfqpoint{3.616000in}{3.359220in}}%
\pgfpathclose%
\pgfusepath{stroke,fill}%
\end{pgfscope}%
\begin{pgfscope}%
\pgfpathrectangle{\pgfqpoint{1.432000in}{0.528000in}}{\pgfqpoint{3.696000in}{3.696000in}}%
\pgfusepath{clip}%
\pgfsetbuttcap%
\pgfsetroundjoin%
\definecolor{currentfill}{rgb}{0.000000,0.000000,0.000000}%
\pgfsetfillcolor{currentfill}%
\pgfsetlinewidth{1.003750pt}%
\definecolor{currentstroke}{rgb}{0.000000,0.000000,0.000000}%
\pgfsetstrokecolor{currentstroke}%
\pgfsetdash{}{0pt}%
\pgfpathmoveto{\pgfqpoint{3.616000in}{3.695220in}}%
\pgfpathcurveto{\pgfqpoint{3.622572in}{3.695220in}}{\pgfqpoint{3.628875in}{3.697831in}}{\pgfqpoint{3.633522in}{3.702478in}}%
\pgfpathcurveto{\pgfqpoint{3.638169in}{3.707125in}}{\pgfqpoint{3.640780in}{3.713428in}}{\pgfqpoint{3.640780in}{3.720000in}}%
\pgfpathcurveto{\pgfqpoint{3.640780in}{3.726572in}}{\pgfqpoint{3.638169in}{3.732875in}}{\pgfqpoint{3.633522in}{3.737522in}}%
\pgfpathcurveto{\pgfqpoint{3.628875in}{3.742169in}}{\pgfqpoint{3.622572in}{3.744780in}}{\pgfqpoint{3.616000in}{3.744780in}}%
\pgfpathcurveto{\pgfqpoint{3.609428in}{3.744780in}}{\pgfqpoint{3.603125in}{3.742169in}}{\pgfqpoint{3.598478in}{3.737522in}}%
\pgfpathcurveto{\pgfqpoint{3.593831in}{3.732875in}}{\pgfqpoint{3.591220in}{3.726572in}}{\pgfqpoint{3.591220in}{3.720000in}}%
\pgfpathcurveto{\pgfqpoint{3.591220in}{3.713428in}}{\pgfqpoint{3.593831in}{3.707125in}}{\pgfqpoint{3.598478in}{3.702478in}}%
\pgfpathcurveto{\pgfqpoint{3.603125in}{3.697831in}}{\pgfqpoint{3.609428in}{3.695220in}}{\pgfqpoint{3.616000in}{3.695220in}}%
\pgfpathclose%
\pgfusepath{stroke,fill}%
\end{pgfscope}%
\begin{pgfscope}%
\pgfpathrectangle{\pgfqpoint{1.432000in}{0.528000in}}{\pgfqpoint{3.696000in}{3.696000in}}%
\pgfusepath{clip}%
\pgfsetbuttcap%
\pgfsetroundjoin%
\definecolor{currentfill}{rgb}{0.000000,0.000000,0.000000}%
\pgfsetfillcolor{currentfill}%
\pgfsetlinewidth{1.003750pt}%
\definecolor{currentstroke}{rgb}{0.000000,0.000000,0.000000}%
\pgfsetstrokecolor{currentstroke}%
\pgfsetdash{}{0pt}%
\pgfpathmoveto{\pgfqpoint{3.616000in}{4.031220in}}%
\pgfpathcurveto{\pgfqpoint{3.622572in}{4.031220in}}{\pgfqpoint{3.628875in}{4.033831in}}{\pgfqpoint{3.633522in}{4.038478in}}%
\pgfpathcurveto{\pgfqpoint{3.638169in}{4.043125in}}{\pgfqpoint{3.640780in}{4.049428in}}{\pgfqpoint{3.640780in}{4.056000in}}%
\pgfpathcurveto{\pgfqpoint{3.640780in}{4.062572in}}{\pgfqpoint{3.638169in}{4.068875in}}{\pgfqpoint{3.633522in}{4.073522in}}%
\pgfpathcurveto{\pgfqpoint{3.628875in}{4.078169in}}{\pgfqpoint{3.622572in}{4.080780in}}{\pgfqpoint{3.616000in}{4.080780in}}%
\pgfpathcurveto{\pgfqpoint{3.609428in}{4.080780in}}{\pgfqpoint{3.603125in}{4.078169in}}{\pgfqpoint{3.598478in}{4.073522in}}%
\pgfpathcurveto{\pgfqpoint{3.593831in}{4.068875in}}{\pgfqpoint{3.591220in}{4.062572in}}{\pgfqpoint{3.591220in}{4.056000in}}%
\pgfpathcurveto{\pgfqpoint{3.591220in}{4.049428in}}{\pgfqpoint{3.593831in}{4.043125in}}{\pgfqpoint{3.598478in}{4.038478in}}%
\pgfpathcurveto{\pgfqpoint{3.603125in}{4.033831in}}{\pgfqpoint{3.609428in}{4.031220in}}{\pgfqpoint{3.616000in}{4.031220in}}%
\pgfpathclose%
\pgfusepath{stroke,fill}%
\end{pgfscope}%
\begin{pgfscope}%
\pgfpathrectangle{\pgfqpoint{1.432000in}{0.528000in}}{\pgfqpoint{3.696000in}{3.696000in}}%
\pgfusepath{clip}%
\pgfsetbuttcap%
\pgfsetroundjoin%
\definecolor{currentfill}{rgb}{0.000000,0.000000,0.000000}%
\pgfsetfillcolor{currentfill}%
\pgfsetlinewidth{1.003750pt}%
\definecolor{currentstroke}{rgb}{0.000000,0.000000,0.000000}%
\pgfsetstrokecolor{currentstroke}%
\pgfsetdash{}{0pt}%
\pgfpathmoveto{\pgfqpoint{3.952000in}{0.671220in}}%
\pgfpathcurveto{\pgfqpoint{3.958572in}{0.671220in}}{\pgfqpoint{3.964875in}{0.673831in}}{\pgfqpoint{3.969522in}{0.678478in}}%
\pgfpathcurveto{\pgfqpoint{3.974169in}{0.683125in}}{\pgfqpoint{3.976780in}{0.689428in}}{\pgfqpoint{3.976780in}{0.696000in}}%
\pgfpathcurveto{\pgfqpoint{3.976780in}{0.702572in}}{\pgfqpoint{3.974169in}{0.708875in}}{\pgfqpoint{3.969522in}{0.713522in}}%
\pgfpathcurveto{\pgfqpoint{3.964875in}{0.718169in}}{\pgfqpoint{3.958572in}{0.720780in}}{\pgfqpoint{3.952000in}{0.720780in}}%
\pgfpathcurveto{\pgfqpoint{3.945428in}{0.720780in}}{\pgfqpoint{3.939125in}{0.718169in}}{\pgfqpoint{3.934478in}{0.713522in}}%
\pgfpathcurveto{\pgfqpoint{3.929831in}{0.708875in}}{\pgfqpoint{3.927220in}{0.702572in}}{\pgfqpoint{3.927220in}{0.696000in}}%
\pgfpathcurveto{\pgfqpoint{3.927220in}{0.689428in}}{\pgfqpoint{3.929831in}{0.683125in}}{\pgfqpoint{3.934478in}{0.678478in}}%
\pgfpathcurveto{\pgfqpoint{3.939125in}{0.673831in}}{\pgfqpoint{3.945428in}{0.671220in}}{\pgfqpoint{3.952000in}{0.671220in}}%
\pgfpathclose%
\pgfusepath{stroke,fill}%
\end{pgfscope}%
\begin{pgfscope}%
\pgfpathrectangle{\pgfqpoint{1.432000in}{0.528000in}}{\pgfqpoint{3.696000in}{3.696000in}}%
\pgfusepath{clip}%
\pgfsetbuttcap%
\pgfsetroundjoin%
\definecolor{currentfill}{rgb}{0.000000,0.000000,0.000000}%
\pgfsetfillcolor{currentfill}%
\pgfsetlinewidth{1.003750pt}%
\definecolor{currentstroke}{rgb}{0.000000,0.000000,0.000000}%
\pgfsetstrokecolor{currentstroke}%
\pgfsetdash{}{0pt}%
\pgfpathmoveto{\pgfqpoint{3.952000in}{1.007220in}}%
\pgfpathcurveto{\pgfqpoint{3.958572in}{1.007220in}}{\pgfqpoint{3.964875in}{1.009831in}}{\pgfqpoint{3.969522in}{1.014478in}}%
\pgfpathcurveto{\pgfqpoint{3.974169in}{1.019125in}}{\pgfqpoint{3.976780in}{1.025428in}}{\pgfqpoint{3.976780in}{1.032000in}}%
\pgfpathcurveto{\pgfqpoint{3.976780in}{1.038572in}}{\pgfqpoint{3.974169in}{1.044875in}}{\pgfqpoint{3.969522in}{1.049522in}}%
\pgfpathcurveto{\pgfqpoint{3.964875in}{1.054169in}}{\pgfqpoint{3.958572in}{1.056780in}}{\pgfqpoint{3.952000in}{1.056780in}}%
\pgfpathcurveto{\pgfqpoint{3.945428in}{1.056780in}}{\pgfqpoint{3.939125in}{1.054169in}}{\pgfqpoint{3.934478in}{1.049522in}}%
\pgfpathcurveto{\pgfqpoint{3.929831in}{1.044875in}}{\pgfqpoint{3.927220in}{1.038572in}}{\pgfqpoint{3.927220in}{1.032000in}}%
\pgfpathcurveto{\pgfqpoint{3.927220in}{1.025428in}}{\pgfqpoint{3.929831in}{1.019125in}}{\pgfqpoint{3.934478in}{1.014478in}}%
\pgfpathcurveto{\pgfqpoint{3.939125in}{1.009831in}}{\pgfqpoint{3.945428in}{1.007220in}}{\pgfqpoint{3.952000in}{1.007220in}}%
\pgfpathclose%
\pgfusepath{stroke,fill}%
\end{pgfscope}%
\begin{pgfscope}%
\pgfpathrectangle{\pgfqpoint{1.432000in}{0.528000in}}{\pgfqpoint{3.696000in}{3.696000in}}%
\pgfusepath{clip}%
\pgfsetbuttcap%
\pgfsetroundjoin%
\definecolor{currentfill}{rgb}{0.000000,0.000000,0.000000}%
\pgfsetfillcolor{currentfill}%
\pgfsetlinewidth{1.003750pt}%
\definecolor{currentstroke}{rgb}{0.000000,0.000000,0.000000}%
\pgfsetstrokecolor{currentstroke}%
\pgfsetdash{}{0pt}%
\pgfpathmoveto{\pgfqpoint{3.952000in}{1.343220in}}%
\pgfpathcurveto{\pgfqpoint{3.958572in}{1.343220in}}{\pgfqpoint{3.964875in}{1.345831in}}{\pgfqpoint{3.969522in}{1.350478in}}%
\pgfpathcurveto{\pgfqpoint{3.974169in}{1.355125in}}{\pgfqpoint{3.976780in}{1.361428in}}{\pgfqpoint{3.976780in}{1.368000in}}%
\pgfpathcurveto{\pgfqpoint{3.976780in}{1.374572in}}{\pgfqpoint{3.974169in}{1.380875in}}{\pgfqpoint{3.969522in}{1.385522in}}%
\pgfpathcurveto{\pgfqpoint{3.964875in}{1.390169in}}{\pgfqpoint{3.958572in}{1.392780in}}{\pgfqpoint{3.952000in}{1.392780in}}%
\pgfpathcurveto{\pgfqpoint{3.945428in}{1.392780in}}{\pgfqpoint{3.939125in}{1.390169in}}{\pgfqpoint{3.934478in}{1.385522in}}%
\pgfpathcurveto{\pgfqpoint{3.929831in}{1.380875in}}{\pgfqpoint{3.927220in}{1.374572in}}{\pgfqpoint{3.927220in}{1.368000in}}%
\pgfpathcurveto{\pgfqpoint{3.927220in}{1.361428in}}{\pgfqpoint{3.929831in}{1.355125in}}{\pgfqpoint{3.934478in}{1.350478in}}%
\pgfpathcurveto{\pgfqpoint{3.939125in}{1.345831in}}{\pgfqpoint{3.945428in}{1.343220in}}{\pgfqpoint{3.952000in}{1.343220in}}%
\pgfpathclose%
\pgfusepath{stroke,fill}%
\end{pgfscope}%
\begin{pgfscope}%
\pgfpathrectangle{\pgfqpoint{1.432000in}{0.528000in}}{\pgfqpoint{3.696000in}{3.696000in}}%
\pgfusepath{clip}%
\pgfsetbuttcap%
\pgfsetroundjoin%
\definecolor{currentfill}{rgb}{0.000000,0.000000,0.000000}%
\pgfsetfillcolor{currentfill}%
\pgfsetlinewidth{1.003750pt}%
\definecolor{currentstroke}{rgb}{0.000000,0.000000,0.000000}%
\pgfsetstrokecolor{currentstroke}%
\pgfsetdash{}{0pt}%
\pgfpathmoveto{\pgfqpoint{3.952000in}{1.679220in}}%
\pgfpathcurveto{\pgfqpoint{3.958572in}{1.679220in}}{\pgfqpoint{3.964875in}{1.681831in}}{\pgfqpoint{3.969522in}{1.686478in}}%
\pgfpathcurveto{\pgfqpoint{3.974169in}{1.691125in}}{\pgfqpoint{3.976780in}{1.697428in}}{\pgfqpoint{3.976780in}{1.704000in}}%
\pgfpathcurveto{\pgfqpoint{3.976780in}{1.710572in}}{\pgfqpoint{3.974169in}{1.716875in}}{\pgfqpoint{3.969522in}{1.721522in}}%
\pgfpathcurveto{\pgfqpoint{3.964875in}{1.726169in}}{\pgfqpoint{3.958572in}{1.728780in}}{\pgfqpoint{3.952000in}{1.728780in}}%
\pgfpathcurveto{\pgfqpoint{3.945428in}{1.728780in}}{\pgfqpoint{3.939125in}{1.726169in}}{\pgfqpoint{3.934478in}{1.721522in}}%
\pgfpathcurveto{\pgfqpoint{3.929831in}{1.716875in}}{\pgfqpoint{3.927220in}{1.710572in}}{\pgfqpoint{3.927220in}{1.704000in}}%
\pgfpathcurveto{\pgfqpoint{3.927220in}{1.697428in}}{\pgfqpoint{3.929831in}{1.691125in}}{\pgfqpoint{3.934478in}{1.686478in}}%
\pgfpathcurveto{\pgfqpoint{3.939125in}{1.681831in}}{\pgfqpoint{3.945428in}{1.679220in}}{\pgfqpoint{3.952000in}{1.679220in}}%
\pgfpathclose%
\pgfusepath{stroke,fill}%
\end{pgfscope}%
\begin{pgfscope}%
\pgfpathrectangle{\pgfqpoint{1.432000in}{0.528000in}}{\pgfqpoint{3.696000in}{3.696000in}}%
\pgfusepath{clip}%
\pgfsetbuttcap%
\pgfsetroundjoin%
\definecolor{currentfill}{rgb}{0.000000,0.000000,0.000000}%
\pgfsetfillcolor{currentfill}%
\pgfsetlinewidth{1.003750pt}%
\definecolor{currentstroke}{rgb}{0.000000,0.000000,0.000000}%
\pgfsetstrokecolor{currentstroke}%
\pgfsetdash{}{0pt}%
\pgfpathmoveto{\pgfqpoint{3.952000in}{2.015220in}}%
\pgfpathcurveto{\pgfqpoint{3.958572in}{2.015220in}}{\pgfqpoint{3.964875in}{2.017831in}}{\pgfqpoint{3.969522in}{2.022478in}}%
\pgfpathcurveto{\pgfqpoint{3.974169in}{2.027125in}}{\pgfqpoint{3.976780in}{2.033428in}}{\pgfqpoint{3.976780in}{2.040000in}}%
\pgfpathcurveto{\pgfqpoint{3.976780in}{2.046572in}}{\pgfqpoint{3.974169in}{2.052875in}}{\pgfqpoint{3.969522in}{2.057522in}}%
\pgfpathcurveto{\pgfqpoint{3.964875in}{2.062169in}}{\pgfqpoint{3.958572in}{2.064780in}}{\pgfqpoint{3.952000in}{2.064780in}}%
\pgfpathcurveto{\pgfqpoint{3.945428in}{2.064780in}}{\pgfqpoint{3.939125in}{2.062169in}}{\pgfqpoint{3.934478in}{2.057522in}}%
\pgfpathcurveto{\pgfqpoint{3.929831in}{2.052875in}}{\pgfqpoint{3.927220in}{2.046572in}}{\pgfqpoint{3.927220in}{2.040000in}}%
\pgfpathcurveto{\pgfqpoint{3.927220in}{2.033428in}}{\pgfqpoint{3.929831in}{2.027125in}}{\pgfqpoint{3.934478in}{2.022478in}}%
\pgfpathcurveto{\pgfqpoint{3.939125in}{2.017831in}}{\pgfqpoint{3.945428in}{2.015220in}}{\pgfqpoint{3.952000in}{2.015220in}}%
\pgfpathclose%
\pgfusepath{stroke,fill}%
\end{pgfscope}%
\begin{pgfscope}%
\pgfpathrectangle{\pgfqpoint{1.432000in}{0.528000in}}{\pgfqpoint{3.696000in}{3.696000in}}%
\pgfusepath{clip}%
\pgfsetbuttcap%
\pgfsetroundjoin%
\definecolor{currentfill}{rgb}{0.000000,0.000000,0.000000}%
\pgfsetfillcolor{currentfill}%
\pgfsetlinewidth{1.003750pt}%
\definecolor{currentstroke}{rgb}{0.000000,0.000000,0.000000}%
\pgfsetstrokecolor{currentstroke}%
\pgfsetdash{}{0pt}%
\pgfpathmoveto{\pgfqpoint{3.952000in}{2.351220in}}%
\pgfpathcurveto{\pgfqpoint{3.958572in}{2.351220in}}{\pgfqpoint{3.964875in}{2.353831in}}{\pgfqpoint{3.969522in}{2.358478in}}%
\pgfpathcurveto{\pgfqpoint{3.974169in}{2.363125in}}{\pgfqpoint{3.976780in}{2.369428in}}{\pgfqpoint{3.976780in}{2.376000in}}%
\pgfpathcurveto{\pgfqpoint{3.976780in}{2.382572in}}{\pgfqpoint{3.974169in}{2.388875in}}{\pgfqpoint{3.969522in}{2.393522in}}%
\pgfpathcurveto{\pgfqpoint{3.964875in}{2.398169in}}{\pgfqpoint{3.958572in}{2.400780in}}{\pgfqpoint{3.952000in}{2.400780in}}%
\pgfpathcurveto{\pgfqpoint{3.945428in}{2.400780in}}{\pgfqpoint{3.939125in}{2.398169in}}{\pgfqpoint{3.934478in}{2.393522in}}%
\pgfpathcurveto{\pgfqpoint{3.929831in}{2.388875in}}{\pgfqpoint{3.927220in}{2.382572in}}{\pgfqpoint{3.927220in}{2.376000in}}%
\pgfpathcurveto{\pgfqpoint{3.927220in}{2.369428in}}{\pgfqpoint{3.929831in}{2.363125in}}{\pgfqpoint{3.934478in}{2.358478in}}%
\pgfpathcurveto{\pgfqpoint{3.939125in}{2.353831in}}{\pgfqpoint{3.945428in}{2.351220in}}{\pgfqpoint{3.952000in}{2.351220in}}%
\pgfpathclose%
\pgfusepath{stroke,fill}%
\end{pgfscope}%
\begin{pgfscope}%
\pgfpathrectangle{\pgfqpoint{1.432000in}{0.528000in}}{\pgfqpoint{3.696000in}{3.696000in}}%
\pgfusepath{clip}%
\pgfsetbuttcap%
\pgfsetroundjoin%
\definecolor{currentfill}{rgb}{0.000000,0.000000,0.000000}%
\pgfsetfillcolor{currentfill}%
\pgfsetlinewidth{1.003750pt}%
\definecolor{currentstroke}{rgb}{0.000000,0.000000,0.000000}%
\pgfsetstrokecolor{currentstroke}%
\pgfsetdash{}{0pt}%
\pgfpathmoveto{\pgfqpoint{3.952000in}{2.687220in}}%
\pgfpathcurveto{\pgfqpoint{3.958572in}{2.687220in}}{\pgfqpoint{3.964875in}{2.689831in}}{\pgfqpoint{3.969522in}{2.694478in}}%
\pgfpathcurveto{\pgfqpoint{3.974169in}{2.699125in}}{\pgfqpoint{3.976780in}{2.705428in}}{\pgfqpoint{3.976780in}{2.712000in}}%
\pgfpathcurveto{\pgfqpoint{3.976780in}{2.718572in}}{\pgfqpoint{3.974169in}{2.724875in}}{\pgfqpoint{3.969522in}{2.729522in}}%
\pgfpathcurveto{\pgfqpoint{3.964875in}{2.734169in}}{\pgfqpoint{3.958572in}{2.736780in}}{\pgfqpoint{3.952000in}{2.736780in}}%
\pgfpathcurveto{\pgfqpoint{3.945428in}{2.736780in}}{\pgfqpoint{3.939125in}{2.734169in}}{\pgfqpoint{3.934478in}{2.729522in}}%
\pgfpathcurveto{\pgfqpoint{3.929831in}{2.724875in}}{\pgfqpoint{3.927220in}{2.718572in}}{\pgfqpoint{3.927220in}{2.712000in}}%
\pgfpathcurveto{\pgfqpoint{3.927220in}{2.705428in}}{\pgfqpoint{3.929831in}{2.699125in}}{\pgfqpoint{3.934478in}{2.694478in}}%
\pgfpathcurveto{\pgfqpoint{3.939125in}{2.689831in}}{\pgfqpoint{3.945428in}{2.687220in}}{\pgfqpoint{3.952000in}{2.687220in}}%
\pgfpathclose%
\pgfusepath{stroke,fill}%
\end{pgfscope}%
\begin{pgfscope}%
\pgfpathrectangle{\pgfqpoint{1.432000in}{0.528000in}}{\pgfqpoint{3.696000in}{3.696000in}}%
\pgfusepath{clip}%
\pgfsetbuttcap%
\pgfsetroundjoin%
\definecolor{currentfill}{rgb}{0.000000,0.000000,0.000000}%
\pgfsetfillcolor{currentfill}%
\pgfsetlinewidth{1.003750pt}%
\definecolor{currentstroke}{rgb}{0.000000,0.000000,0.000000}%
\pgfsetstrokecolor{currentstroke}%
\pgfsetdash{}{0pt}%
\pgfpathmoveto{\pgfqpoint{3.952000in}{3.023220in}}%
\pgfpathcurveto{\pgfqpoint{3.958572in}{3.023220in}}{\pgfqpoint{3.964875in}{3.025831in}}{\pgfqpoint{3.969522in}{3.030478in}}%
\pgfpathcurveto{\pgfqpoint{3.974169in}{3.035125in}}{\pgfqpoint{3.976780in}{3.041428in}}{\pgfqpoint{3.976780in}{3.048000in}}%
\pgfpathcurveto{\pgfqpoint{3.976780in}{3.054572in}}{\pgfqpoint{3.974169in}{3.060875in}}{\pgfqpoint{3.969522in}{3.065522in}}%
\pgfpathcurveto{\pgfqpoint{3.964875in}{3.070169in}}{\pgfqpoint{3.958572in}{3.072780in}}{\pgfqpoint{3.952000in}{3.072780in}}%
\pgfpathcurveto{\pgfqpoint{3.945428in}{3.072780in}}{\pgfqpoint{3.939125in}{3.070169in}}{\pgfqpoint{3.934478in}{3.065522in}}%
\pgfpathcurveto{\pgfqpoint{3.929831in}{3.060875in}}{\pgfqpoint{3.927220in}{3.054572in}}{\pgfqpoint{3.927220in}{3.048000in}}%
\pgfpathcurveto{\pgfqpoint{3.927220in}{3.041428in}}{\pgfqpoint{3.929831in}{3.035125in}}{\pgfqpoint{3.934478in}{3.030478in}}%
\pgfpathcurveto{\pgfqpoint{3.939125in}{3.025831in}}{\pgfqpoint{3.945428in}{3.023220in}}{\pgfqpoint{3.952000in}{3.023220in}}%
\pgfpathclose%
\pgfusepath{stroke,fill}%
\end{pgfscope}%
\begin{pgfscope}%
\pgfpathrectangle{\pgfqpoint{1.432000in}{0.528000in}}{\pgfqpoint{3.696000in}{3.696000in}}%
\pgfusepath{clip}%
\pgfsetbuttcap%
\pgfsetroundjoin%
\definecolor{currentfill}{rgb}{0.000000,0.000000,0.000000}%
\pgfsetfillcolor{currentfill}%
\pgfsetlinewidth{1.003750pt}%
\definecolor{currentstroke}{rgb}{0.000000,0.000000,0.000000}%
\pgfsetstrokecolor{currentstroke}%
\pgfsetdash{}{0pt}%
\pgfpathmoveto{\pgfqpoint{3.952000in}{3.359220in}}%
\pgfpathcurveto{\pgfqpoint{3.958572in}{3.359220in}}{\pgfqpoint{3.964875in}{3.361831in}}{\pgfqpoint{3.969522in}{3.366478in}}%
\pgfpathcurveto{\pgfqpoint{3.974169in}{3.371125in}}{\pgfqpoint{3.976780in}{3.377428in}}{\pgfqpoint{3.976780in}{3.384000in}}%
\pgfpathcurveto{\pgfqpoint{3.976780in}{3.390572in}}{\pgfqpoint{3.974169in}{3.396875in}}{\pgfqpoint{3.969522in}{3.401522in}}%
\pgfpathcurveto{\pgfqpoint{3.964875in}{3.406169in}}{\pgfqpoint{3.958572in}{3.408780in}}{\pgfqpoint{3.952000in}{3.408780in}}%
\pgfpathcurveto{\pgfqpoint{3.945428in}{3.408780in}}{\pgfqpoint{3.939125in}{3.406169in}}{\pgfqpoint{3.934478in}{3.401522in}}%
\pgfpathcurveto{\pgfqpoint{3.929831in}{3.396875in}}{\pgfqpoint{3.927220in}{3.390572in}}{\pgfqpoint{3.927220in}{3.384000in}}%
\pgfpathcurveto{\pgfqpoint{3.927220in}{3.377428in}}{\pgfqpoint{3.929831in}{3.371125in}}{\pgfqpoint{3.934478in}{3.366478in}}%
\pgfpathcurveto{\pgfqpoint{3.939125in}{3.361831in}}{\pgfqpoint{3.945428in}{3.359220in}}{\pgfqpoint{3.952000in}{3.359220in}}%
\pgfpathclose%
\pgfusepath{stroke,fill}%
\end{pgfscope}%
\begin{pgfscope}%
\pgfpathrectangle{\pgfqpoint{1.432000in}{0.528000in}}{\pgfqpoint{3.696000in}{3.696000in}}%
\pgfusepath{clip}%
\pgfsetbuttcap%
\pgfsetroundjoin%
\definecolor{currentfill}{rgb}{0.000000,0.000000,0.000000}%
\pgfsetfillcolor{currentfill}%
\pgfsetlinewidth{1.003750pt}%
\definecolor{currentstroke}{rgb}{0.000000,0.000000,0.000000}%
\pgfsetstrokecolor{currentstroke}%
\pgfsetdash{}{0pt}%
\pgfpathmoveto{\pgfqpoint{3.952000in}{3.695220in}}%
\pgfpathcurveto{\pgfqpoint{3.958572in}{3.695220in}}{\pgfqpoint{3.964875in}{3.697831in}}{\pgfqpoint{3.969522in}{3.702478in}}%
\pgfpathcurveto{\pgfqpoint{3.974169in}{3.707125in}}{\pgfqpoint{3.976780in}{3.713428in}}{\pgfqpoint{3.976780in}{3.720000in}}%
\pgfpathcurveto{\pgfqpoint{3.976780in}{3.726572in}}{\pgfqpoint{3.974169in}{3.732875in}}{\pgfqpoint{3.969522in}{3.737522in}}%
\pgfpathcurveto{\pgfqpoint{3.964875in}{3.742169in}}{\pgfqpoint{3.958572in}{3.744780in}}{\pgfqpoint{3.952000in}{3.744780in}}%
\pgfpathcurveto{\pgfqpoint{3.945428in}{3.744780in}}{\pgfqpoint{3.939125in}{3.742169in}}{\pgfqpoint{3.934478in}{3.737522in}}%
\pgfpathcurveto{\pgfqpoint{3.929831in}{3.732875in}}{\pgfqpoint{3.927220in}{3.726572in}}{\pgfqpoint{3.927220in}{3.720000in}}%
\pgfpathcurveto{\pgfqpoint{3.927220in}{3.713428in}}{\pgfqpoint{3.929831in}{3.707125in}}{\pgfqpoint{3.934478in}{3.702478in}}%
\pgfpathcurveto{\pgfqpoint{3.939125in}{3.697831in}}{\pgfqpoint{3.945428in}{3.695220in}}{\pgfqpoint{3.952000in}{3.695220in}}%
\pgfpathclose%
\pgfusepath{stroke,fill}%
\end{pgfscope}%
\begin{pgfscope}%
\pgfpathrectangle{\pgfqpoint{1.432000in}{0.528000in}}{\pgfqpoint{3.696000in}{3.696000in}}%
\pgfusepath{clip}%
\pgfsetbuttcap%
\pgfsetroundjoin%
\definecolor{currentfill}{rgb}{0.000000,0.000000,0.000000}%
\pgfsetfillcolor{currentfill}%
\pgfsetlinewidth{1.003750pt}%
\definecolor{currentstroke}{rgb}{0.000000,0.000000,0.000000}%
\pgfsetstrokecolor{currentstroke}%
\pgfsetdash{}{0pt}%
\pgfpathmoveto{\pgfqpoint{3.952000in}{4.031220in}}%
\pgfpathcurveto{\pgfqpoint{3.958572in}{4.031220in}}{\pgfqpoint{3.964875in}{4.033831in}}{\pgfqpoint{3.969522in}{4.038478in}}%
\pgfpathcurveto{\pgfqpoint{3.974169in}{4.043125in}}{\pgfqpoint{3.976780in}{4.049428in}}{\pgfqpoint{3.976780in}{4.056000in}}%
\pgfpathcurveto{\pgfqpoint{3.976780in}{4.062572in}}{\pgfqpoint{3.974169in}{4.068875in}}{\pgfqpoint{3.969522in}{4.073522in}}%
\pgfpathcurveto{\pgfqpoint{3.964875in}{4.078169in}}{\pgfqpoint{3.958572in}{4.080780in}}{\pgfqpoint{3.952000in}{4.080780in}}%
\pgfpathcurveto{\pgfqpoint{3.945428in}{4.080780in}}{\pgfqpoint{3.939125in}{4.078169in}}{\pgfqpoint{3.934478in}{4.073522in}}%
\pgfpathcurveto{\pgfqpoint{3.929831in}{4.068875in}}{\pgfqpoint{3.927220in}{4.062572in}}{\pgfqpoint{3.927220in}{4.056000in}}%
\pgfpathcurveto{\pgfqpoint{3.927220in}{4.049428in}}{\pgfqpoint{3.929831in}{4.043125in}}{\pgfqpoint{3.934478in}{4.038478in}}%
\pgfpathcurveto{\pgfqpoint{3.939125in}{4.033831in}}{\pgfqpoint{3.945428in}{4.031220in}}{\pgfqpoint{3.952000in}{4.031220in}}%
\pgfpathclose%
\pgfusepath{stroke,fill}%
\end{pgfscope}%
\begin{pgfscope}%
\pgfpathrectangle{\pgfqpoint{1.432000in}{0.528000in}}{\pgfqpoint{3.696000in}{3.696000in}}%
\pgfusepath{clip}%
\pgfsetbuttcap%
\pgfsetroundjoin%
\definecolor{currentfill}{rgb}{0.000000,0.000000,0.000000}%
\pgfsetfillcolor{currentfill}%
\pgfsetlinewidth{1.003750pt}%
\definecolor{currentstroke}{rgb}{0.000000,0.000000,0.000000}%
\pgfsetstrokecolor{currentstroke}%
\pgfsetdash{}{0pt}%
\pgfpathmoveto{\pgfqpoint{4.288000in}{0.671220in}}%
\pgfpathcurveto{\pgfqpoint{4.294572in}{0.671220in}}{\pgfqpoint{4.300875in}{0.673831in}}{\pgfqpoint{4.305522in}{0.678478in}}%
\pgfpathcurveto{\pgfqpoint{4.310169in}{0.683125in}}{\pgfqpoint{4.312780in}{0.689428in}}{\pgfqpoint{4.312780in}{0.696000in}}%
\pgfpathcurveto{\pgfqpoint{4.312780in}{0.702572in}}{\pgfqpoint{4.310169in}{0.708875in}}{\pgfqpoint{4.305522in}{0.713522in}}%
\pgfpathcurveto{\pgfqpoint{4.300875in}{0.718169in}}{\pgfqpoint{4.294572in}{0.720780in}}{\pgfqpoint{4.288000in}{0.720780in}}%
\pgfpathcurveto{\pgfqpoint{4.281428in}{0.720780in}}{\pgfqpoint{4.275125in}{0.718169in}}{\pgfqpoint{4.270478in}{0.713522in}}%
\pgfpathcurveto{\pgfqpoint{4.265831in}{0.708875in}}{\pgfqpoint{4.263220in}{0.702572in}}{\pgfqpoint{4.263220in}{0.696000in}}%
\pgfpathcurveto{\pgfqpoint{4.263220in}{0.689428in}}{\pgfqpoint{4.265831in}{0.683125in}}{\pgfqpoint{4.270478in}{0.678478in}}%
\pgfpathcurveto{\pgfqpoint{4.275125in}{0.673831in}}{\pgfqpoint{4.281428in}{0.671220in}}{\pgfqpoint{4.288000in}{0.671220in}}%
\pgfpathclose%
\pgfusepath{stroke,fill}%
\end{pgfscope}%
\begin{pgfscope}%
\pgfpathrectangle{\pgfqpoint{1.432000in}{0.528000in}}{\pgfqpoint{3.696000in}{3.696000in}}%
\pgfusepath{clip}%
\pgfsetbuttcap%
\pgfsetroundjoin%
\definecolor{currentfill}{rgb}{0.000000,0.000000,0.000000}%
\pgfsetfillcolor{currentfill}%
\pgfsetlinewidth{1.003750pt}%
\definecolor{currentstroke}{rgb}{0.000000,0.000000,0.000000}%
\pgfsetstrokecolor{currentstroke}%
\pgfsetdash{}{0pt}%
\pgfpathmoveto{\pgfqpoint{4.288000in}{1.007220in}}%
\pgfpathcurveto{\pgfqpoint{4.294572in}{1.007220in}}{\pgfqpoint{4.300875in}{1.009831in}}{\pgfqpoint{4.305522in}{1.014478in}}%
\pgfpathcurveto{\pgfqpoint{4.310169in}{1.019125in}}{\pgfqpoint{4.312780in}{1.025428in}}{\pgfqpoint{4.312780in}{1.032000in}}%
\pgfpathcurveto{\pgfqpoint{4.312780in}{1.038572in}}{\pgfqpoint{4.310169in}{1.044875in}}{\pgfqpoint{4.305522in}{1.049522in}}%
\pgfpathcurveto{\pgfqpoint{4.300875in}{1.054169in}}{\pgfqpoint{4.294572in}{1.056780in}}{\pgfqpoint{4.288000in}{1.056780in}}%
\pgfpathcurveto{\pgfqpoint{4.281428in}{1.056780in}}{\pgfqpoint{4.275125in}{1.054169in}}{\pgfqpoint{4.270478in}{1.049522in}}%
\pgfpathcurveto{\pgfqpoint{4.265831in}{1.044875in}}{\pgfqpoint{4.263220in}{1.038572in}}{\pgfqpoint{4.263220in}{1.032000in}}%
\pgfpathcurveto{\pgfqpoint{4.263220in}{1.025428in}}{\pgfqpoint{4.265831in}{1.019125in}}{\pgfqpoint{4.270478in}{1.014478in}}%
\pgfpathcurveto{\pgfqpoint{4.275125in}{1.009831in}}{\pgfqpoint{4.281428in}{1.007220in}}{\pgfqpoint{4.288000in}{1.007220in}}%
\pgfpathclose%
\pgfusepath{stroke,fill}%
\end{pgfscope}%
\begin{pgfscope}%
\pgfpathrectangle{\pgfqpoint{1.432000in}{0.528000in}}{\pgfqpoint{3.696000in}{3.696000in}}%
\pgfusepath{clip}%
\pgfsetbuttcap%
\pgfsetroundjoin%
\definecolor{currentfill}{rgb}{0.000000,0.000000,0.000000}%
\pgfsetfillcolor{currentfill}%
\pgfsetlinewidth{1.003750pt}%
\definecolor{currentstroke}{rgb}{0.000000,0.000000,0.000000}%
\pgfsetstrokecolor{currentstroke}%
\pgfsetdash{}{0pt}%
\pgfpathmoveto{\pgfqpoint{4.288000in}{1.343220in}}%
\pgfpathcurveto{\pgfqpoint{4.294572in}{1.343220in}}{\pgfqpoint{4.300875in}{1.345831in}}{\pgfqpoint{4.305522in}{1.350478in}}%
\pgfpathcurveto{\pgfqpoint{4.310169in}{1.355125in}}{\pgfqpoint{4.312780in}{1.361428in}}{\pgfqpoint{4.312780in}{1.368000in}}%
\pgfpathcurveto{\pgfqpoint{4.312780in}{1.374572in}}{\pgfqpoint{4.310169in}{1.380875in}}{\pgfqpoint{4.305522in}{1.385522in}}%
\pgfpathcurveto{\pgfqpoint{4.300875in}{1.390169in}}{\pgfqpoint{4.294572in}{1.392780in}}{\pgfqpoint{4.288000in}{1.392780in}}%
\pgfpathcurveto{\pgfqpoint{4.281428in}{1.392780in}}{\pgfqpoint{4.275125in}{1.390169in}}{\pgfqpoint{4.270478in}{1.385522in}}%
\pgfpathcurveto{\pgfqpoint{4.265831in}{1.380875in}}{\pgfqpoint{4.263220in}{1.374572in}}{\pgfqpoint{4.263220in}{1.368000in}}%
\pgfpathcurveto{\pgfqpoint{4.263220in}{1.361428in}}{\pgfqpoint{4.265831in}{1.355125in}}{\pgfqpoint{4.270478in}{1.350478in}}%
\pgfpathcurveto{\pgfqpoint{4.275125in}{1.345831in}}{\pgfqpoint{4.281428in}{1.343220in}}{\pgfqpoint{4.288000in}{1.343220in}}%
\pgfpathclose%
\pgfusepath{stroke,fill}%
\end{pgfscope}%
\begin{pgfscope}%
\pgfpathrectangle{\pgfqpoint{1.432000in}{0.528000in}}{\pgfqpoint{3.696000in}{3.696000in}}%
\pgfusepath{clip}%
\pgfsetbuttcap%
\pgfsetroundjoin%
\definecolor{currentfill}{rgb}{0.000000,0.000000,0.000000}%
\pgfsetfillcolor{currentfill}%
\pgfsetlinewidth{1.003750pt}%
\definecolor{currentstroke}{rgb}{0.000000,0.000000,0.000000}%
\pgfsetstrokecolor{currentstroke}%
\pgfsetdash{}{0pt}%
\pgfpathmoveto{\pgfqpoint{4.288000in}{1.679220in}}%
\pgfpathcurveto{\pgfqpoint{4.294572in}{1.679220in}}{\pgfqpoint{4.300875in}{1.681831in}}{\pgfqpoint{4.305522in}{1.686478in}}%
\pgfpathcurveto{\pgfqpoint{4.310169in}{1.691125in}}{\pgfqpoint{4.312780in}{1.697428in}}{\pgfqpoint{4.312780in}{1.704000in}}%
\pgfpathcurveto{\pgfqpoint{4.312780in}{1.710572in}}{\pgfqpoint{4.310169in}{1.716875in}}{\pgfqpoint{4.305522in}{1.721522in}}%
\pgfpathcurveto{\pgfqpoint{4.300875in}{1.726169in}}{\pgfqpoint{4.294572in}{1.728780in}}{\pgfqpoint{4.288000in}{1.728780in}}%
\pgfpathcurveto{\pgfqpoint{4.281428in}{1.728780in}}{\pgfqpoint{4.275125in}{1.726169in}}{\pgfqpoint{4.270478in}{1.721522in}}%
\pgfpathcurveto{\pgfqpoint{4.265831in}{1.716875in}}{\pgfqpoint{4.263220in}{1.710572in}}{\pgfqpoint{4.263220in}{1.704000in}}%
\pgfpathcurveto{\pgfqpoint{4.263220in}{1.697428in}}{\pgfqpoint{4.265831in}{1.691125in}}{\pgfqpoint{4.270478in}{1.686478in}}%
\pgfpathcurveto{\pgfqpoint{4.275125in}{1.681831in}}{\pgfqpoint{4.281428in}{1.679220in}}{\pgfqpoint{4.288000in}{1.679220in}}%
\pgfpathclose%
\pgfusepath{stroke,fill}%
\end{pgfscope}%
\begin{pgfscope}%
\pgfpathrectangle{\pgfqpoint{1.432000in}{0.528000in}}{\pgfqpoint{3.696000in}{3.696000in}}%
\pgfusepath{clip}%
\pgfsetbuttcap%
\pgfsetroundjoin%
\definecolor{currentfill}{rgb}{0.000000,0.000000,0.000000}%
\pgfsetfillcolor{currentfill}%
\pgfsetlinewidth{1.003750pt}%
\definecolor{currentstroke}{rgb}{0.000000,0.000000,0.000000}%
\pgfsetstrokecolor{currentstroke}%
\pgfsetdash{}{0pt}%
\pgfpathmoveto{\pgfqpoint{4.288000in}{2.015220in}}%
\pgfpathcurveto{\pgfqpoint{4.294572in}{2.015220in}}{\pgfqpoint{4.300875in}{2.017831in}}{\pgfqpoint{4.305522in}{2.022478in}}%
\pgfpathcurveto{\pgfqpoint{4.310169in}{2.027125in}}{\pgfqpoint{4.312780in}{2.033428in}}{\pgfqpoint{4.312780in}{2.040000in}}%
\pgfpathcurveto{\pgfqpoint{4.312780in}{2.046572in}}{\pgfqpoint{4.310169in}{2.052875in}}{\pgfqpoint{4.305522in}{2.057522in}}%
\pgfpathcurveto{\pgfqpoint{4.300875in}{2.062169in}}{\pgfqpoint{4.294572in}{2.064780in}}{\pgfqpoint{4.288000in}{2.064780in}}%
\pgfpathcurveto{\pgfqpoint{4.281428in}{2.064780in}}{\pgfqpoint{4.275125in}{2.062169in}}{\pgfqpoint{4.270478in}{2.057522in}}%
\pgfpathcurveto{\pgfqpoint{4.265831in}{2.052875in}}{\pgfqpoint{4.263220in}{2.046572in}}{\pgfqpoint{4.263220in}{2.040000in}}%
\pgfpathcurveto{\pgfqpoint{4.263220in}{2.033428in}}{\pgfqpoint{4.265831in}{2.027125in}}{\pgfqpoint{4.270478in}{2.022478in}}%
\pgfpathcurveto{\pgfqpoint{4.275125in}{2.017831in}}{\pgfqpoint{4.281428in}{2.015220in}}{\pgfqpoint{4.288000in}{2.015220in}}%
\pgfpathclose%
\pgfusepath{stroke,fill}%
\end{pgfscope}%
\begin{pgfscope}%
\pgfpathrectangle{\pgfqpoint{1.432000in}{0.528000in}}{\pgfqpoint{3.696000in}{3.696000in}}%
\pgfusepath{clip}%
\pgfsetbuttcap%
\pgfsetroundjoin%
\definecolor{currentfill}{rgb}{0.000000,0.000000,0.000000}%
\pgfsetfillcolor{currentfill}%
\pgfsetlinewidth{1.003750pt}%
\definecolor{currentstroke}{rgb}{0.000000,0.000000,0.000000}%
\pgfsetstrokecolor{currentstroke}%
\pgfsetdash{}{0pt}%
\pgfpathmoveto{\pgfqpoint{4.288000in}{2.351220in}}%
\pgfpathcurveto{\pgfqpoint{4.294572in}{2.351220in}}{\pgfqpoint{4.300875in}{2.353831in}}{\pgfqpoint{4.305522in}{2.358478in}}%
\pgfpathcurveto{\pgfqpoint{4.310169in}{2.363125in}}{\pgfqpoint{4.312780in}{2.369428in}}{\pgfqpoint{4.312780in}{2.376000in}}%
\pgfpathcurveto{\pgfqpoint{4.312780in}{2.382572in}}{\pgfqpoint{4.310169in}{2.388875in}}{\pgfqpoint{4.305522in}{2.393522in}}%
\pgfpathcurveto{\pgfqpoint{4.300875in}{2.398169in}}{\pgfqpoint{4.294572in}{2.400780in}}{\pgfqpoint{4.288000in}{2.400780in}}%
\pgfpathcurveto{\pgfqpoint{4.281428in}{2.400780in}}{\pgfqpoint{4.275125in}{2.398169in}}{\pgfqpoint{4.270478in}{2.393522in}}%
\pgfpathcurveto{\pgfqpoint{4.265831in}{2.388875in}}{\pgfqpoint{4.263220in}{2.382572in}}{\pgfqpoint{4.263220in}{2.376000in}}%
\pgfpathcurveto{\pgfqpoint{4.263220in}{2.369428in}}{\pgfqpoint{4.265831in}{2.363125in}}{\pgfqpoint{4.270478in}{2.358478in}}%
\pgfpathcurveto{\pgfqpoint{4.275125in}{2.353831in}}{\pgfqpoint{4.281428in}{2.351220in}}{\pgfqpoint{4.288000in}{2.351220in}}%
\pgfpathclose%
\pgfusepath{stroke,fill}%
\end{pgfscope}%
\begin{pgfscope}%
\pgfpathrectangle{\pgfqpoint{1.432000in}{0.528000in}}{\pgfqpoint{3.696000in}{3.696000in}}%
\pgfusepath{clip}%
\pgfsetbuttcap%
\pgfsetroundjoin%
\definecolor{currentfill}{rgb}{0.000000,0.000000,0.000000}%
\pgfsetfillcolor{currentfill}%
\pgfsetlinewidth{1.003750pt}%
\definecolor{currentstroke}{rgb}{0.000000,0.000000,0.000000}%
\pgfsetstrokecolor{currentstroke}%
\pgfsetdash{}{0pt}%
\pgfpathmoveto{\pgfqpoint{4.288000in}{2.687220in}}%
\pgfpathcurveto{\pgfqpoint{4.294572in}{2.687220in}}{\pgfqpoint{4.300875in}{2.689831in}}{\pgfqpoint{4.305522in}{2.694478in}}%
\pgfpathcurveto{\pgfqpoint{4.310169in}{2.699125in}}{\pgfqpoint{4.312780in}{2.705428in}}{\pgfqpoint{4.312780in}{2.712000in}}%
\pgfpathcurveto{\pgfqpoint{4.312780in}{2.718572in}}{\pgfqpoint{4.310169in}{2.724875in}}{\pgfqpoint{4.305522in}{2.729522in}}%
\pgfpathcurveto{\pgfqpoint{4.300875in}{2.734169in}}{\pgfqpoint{4.294572in}{2.736780in}}{\pgfqpoint{4.288000in}{2.736780in}}%
\pgfpathcurveto{\pgfqpoint{4.281428in}{2.736780in}}{\pgfqpoint{4.275125in}{2.734169in}}{\pgfqpoint{4.270478in}{2.729522in}}%
\pgfpathcurveto{\pgfqpoint{4.265831in}{2.724875in}}{\pgfqpoint{4.263220in}{2.718572in}}{\pgfqpoint{4.263220in}{2.712000in}}%
\pgfpathcurveto{\pgfqpoint{4.263220in}{2.705428in}}{\pgfqpoint{4.265831in}{2.699125in}}{\pgfqpoint{4.270478in}{2.694478in}}%
\pgfpathcurveto{\pgfqpoint{4.275125in}{2.689831in}}{\pgfqpoint{4.281428in}{2.687220in}}{\pgfqpoint{4.288000in}{2.687220in}}%
\pgfpathclose%
\pgfusepath{stroke,fill}%
\end{pgfscope}%
\begin{pgfscope}%
\pgfpathrectangle{\pgfqpoint{1.432000in}{0.528000in}}{\pgfqpoint{3.696000in}{3.696000in}}%
\pgfusepath{clip}%
\pgfsetbuttcap%
\pgfsetroundjoin%
\definecolor{currentfill}{rgb}{0.000000,0.000000,0.000000}%
\pgfsetfillcolor{currentfill}%
\pgfsetlinewidth{1.003750pt}%
\definecolor{currentstroke}{rgb}{0.000000,0.000000,0.000000}%
\pgfsetstrokecolor{currentstroke}%
\pgfsetdash{}{0pt}%
\pgfpathmoveto{\pgfqpoint{4.288000in}{3.023220in}}%
\pgfpathcurveto{\pgfqpoint{4.294572in}{3.023220in}}{\pgfqpoint{4.300875in}{3.025831in}}{\pgfqpoint{4.305522in}{3.030478in}}%
\pgfpathcurveto{\pgfqpoint{4.310169in}{3.035125in}}{\pgfqpoint{4.312780in}{3.041428in}}{\pgfqpoint{4.312780in}{3.048000in}}%
\pgfpathcurveto{\pgfqpoint{4.312780in}{3.054572in}}{\pgfqpoint{4.310169in}{3.060875in}}{\pgfqpoint{4.305522in}{3.065522in}}%
\pgfpathcurveto{\pgfqpoint{4.300875in}{3.070169in}}{\pgfqpoint{4.294572in}{3.072780in}}{\pgfqpoint{4.288000in}{3.072780in}}%
\pgfpathcurveto{\pgfqpoint{4.281428in}{3.072780in}}{\pgfqpoint{4.275125in}{3.070169in}}{\pgfqpoint{4.270478in}{3.065522in}}%
\pgfpathcurveto{\pgfqpoint{4.265831in}{3.060875in}}{\pgfqpoint{4.263220in}{3.054572in}}{\pgfqpoint{4.263220in}{3.048000in}}%
\pgfpathcurveto{\pgfqpoint{4.263220in}{3.041428in}}{\pgfqpoint{4.265831in}{3.035125in}}{\pgfqpoint{4.270478in}{3.030478in}}%
\pgfpathcurveto{\pgfqpoint{4.275125in}{3.025831in}}{\pgfqpoint{4.281428in}{3.023220in}}{\pgfqpoint{4.288000in}{3.023220in}}%
\pgfpathclose%
\pgfusepath{stroke,fill}%
\end{pgfscope}%
\begin{pgfscope}%
\pgfpathrectangle{\pgfqpoint{1.432000in}{0.528000in}}{\pgfqpoint{3.696000in}{3.696000in}}%
\pgfusepath{clip}%
\pgfsetbuttcap%
\pgfsetroundjoin%
\definecolor{currentfill}{rgb}{0.000000,0.000000,0.000000}%
\pgfsetfillcolor{currentfill}%
\pgfsetlinewidth{1.003750pt}%
\definecolor{currentstroke}{rgb}{0.000000,0.000000,0.000000}%
\pgfsetstrokecolor{currentstroke}%
\pgfsetdash{}{0pt}%
\pgfpathmoveto{\pgfqpoint{4.288000in}{3.359220in}}%
\pgfpathcurveto{\pgfqpoint{4.294572in}{3.359220in}}{\pgfqpoint{4.300875in}{3.361831in}}{\pgfqpoint{4.305522in}{3.366478in}}%
\pgfpathcurveto{\pgfqpoint{4.310169in}{3.371125in}}{\pgfqpoint{4.312780in}{3.377428in}}{\pgfqpoint{4.312780in}{3.384000in}}%
\pgfpathcurveto{\pgfqpoint{4.312780in}{3.390572in}}{\pgfqpoint{4.310169in}{3.396875in}}{\pgfqpoint{4.305522in}{3.401522in}}%
\pgfpathcurveto{\pgfqpoint{4.300875in}{3.406169in}}{\pgfqpoint{4.294572in}{3.408780in}}{\pgfqpoint{4.288000in}{3.408780in}}%
\pgfpathcurveto{\pgfqpoint{4.281428in}{3.408780in}}{\pgfqpoint{4.275125in}{3.406169in}}{\pgfqpoint{4.270478in}{3.401522in}}%
\pgfpathcurveto{\pgfqpoint{4.265831in}{3.396875in}}{\pgfqpoint{4.263220in}{3.390572in}}{\pgfqpoint{4.263220in}{3.384000in}}%
\pgfpathcurveto{\pgfqpoint{4.263220in}{3.377428in}}{\pgfqpoint{4.265831in}{3.371125in}}{\pgfqpoint{4.270478in}{3.366478in}}%
\pgfpathcurveto{\pgfqpoint{4.275125in}{3.361831in}}{\pgfqpoint{4.281428in}{3.359220in}}{\pgfqpoint{4.288000in}{3.359220in}}%
\pgfpathclose%
\pgfusepath{stroke,fill}%
\end{pgfscope}%
\begin{pgfscope}%
\pgfpathrectangle{\pgfqpoint{1.432000in}{0.528000in}}{\pgfqpoint{3.696000in}{3.696000in}}%
\pgfusepath{clip}%
\pgfsetbuttcap%
\pgfsetroundjoin%
\definecolor{currentfill}{rgb}{0.000000,0.000000,0.000000}%
\pgfsetfillcolor{currentfill}%
\pgfsetlinewidth{1.003750pt}%
\definecolor{currentstroke}{rgb}{0.000000,0.000000,0.000000}%
\pgfsetstrokecolor{currentstroke}%
\pgfsetdash{}{0pt}%
\pgfpathmoveto{\pgfqpoint{4.288000in}{3.695220in}}%
\pgfpathcurveto{\pgfqpoint{4.294572in}{3.695220in}}{\pgfqpoint{4.300875in}{3.697831in}}{\pgfqpoint{4.305522in}{3.702478in}}%
\pgfpathcurveto{\pgfqpoint{4.310169in}{3.707125in}}{\pgfqpoint{4.312780in}{3.713428in}}{\pgfqpoint{4.312780in}{3.720000in}}%
\pgfpathcurveto{\pgfqpoint{4.312780in}{3.726572in}}{\pgfqpoint{4.310169in}{3.732875in}}{\pgfqpoint{4.305522in}{3.737522in}}%
\pgfpathcurveto{\pgfqpoint{4.300875in}{3.742169in}}{\pgfqpoint{4.294572in}{3.744780in}}{\pgfqpoint{4.288000in}{3.744780in}}%
\pgfpathcurveto{\pgfqpoint{4.281428in}{3.744780in}}{\pgfqpoint{4.275125in}{3.742169in}}{\pgfqpoint{4.270478in}{3.737522in}}%
\pgfpathcurveto{\pgfqpoint{4.265831in}{3.732875in}}{\pgfqpoint{4.263220in}{3.726572in}}{\pgfqpoint{4.263220in}{3.720000in}}%
\pgfpathcurveto{\pgfqpoint{4.263220in}{3.713428in}}{\pgfqpoint{4.265831in}{3.707125in}}{\pgfqpoint{4.270478in}{3.702478in}}%
\pgfpathcurveto{\pgfqpoint{4.275125in}{3.697831in}}{\pgfqpoint{4.281428in}{3.695220in}}{\pgfqpoint{4.288000in}{3.695220in}}%
\pgfpathclose%
\pgfusepath{stroke,fill}%
\end{pgfscope}%
\begin{pgfscope}%
\pgfpathrectangle{\pgfqpoint{1.432000in}{0.528000in}}{\pgfqpoint{3.696000in}{3.696000in}}%
\pgfusepath{clip}%
\pgfsetbuttcap%
\pgfsetroundjoin%
\definecolor{currentfill}{rgb}{0.000000,0.000000,0.000000}%
\pgfsetfillcolor{currentfill}%
\pgfsetlinewidth{1.003750pt}%
\definecolor{currentstroke}{rgb}{0.000000,0.000000,0.000000}%
\pgfsetstrokecolor{currentstroke}%
\pgfsetdash{}{0pt}%
\pgfpathmoveto{\pgfqpoint{4.288000in}{4.031220in}}%
\pgfpathcurveto{\pgfqpoint{4.294572in}{4.031220in}}{\pgfqpoint{4.300875in}{4.033831in}}{\pgfqpoint{4.305522in}{4.038478in}}%
\pgfpathcurveto{\pgfqpoint{4.310169in}{4.043125in}}{\pgfqpoint{4.312780in}{4.049428in}}{\pgfqpoint{4.312780in}{4.056000in}}%
\pgfpathcurveto{\pgfqpoint{4.312780in}{4.062572in}}{\pgfqpoint{4.310169in}{4.068875in}}{\pgfqpoint{4.305522in}{4.073522in}}%
\pgfpathcurveto{\pgfqpoint{4.300875in}{4.078169in}}{\pgfqpoint{4.294572in}{4.080780in}}{\pgfqpoint{4.288000in}{4.080780in}}%
\pgfpathcurveto{\pgfqpoint{4.281428in}{4.080780in}}{\pgfqpoint{4.275125in}{4.078169in}}{\pgfqpoint{4.270478in}{4.073522in}}%
\pgfpathcurveto{\pgfqpoint{4.265831in}{4.068875in}}{\pgfqpoint{4.263220in}{4.062572in}}{\pgfqpoint{4.263220in}{4.056000in}}%
\pgfpathcurveto{\pgfqpoint{4.263220in}{4.049428in}}{\pgfqpoint{4.265831in}{4.043125in}}{\pgfqpoint{4.270478in}{4.038478in}}%
\pgfpathcurveto{\pgfqpoint{4.275125in}{4.033831in}}{\pgfqpoint{4.281428in}{4.031220in}}{\pgfqpoint{4.288000in}{4.031220in}}%
\pgfpathclose%
\pgfusepath{stroke,fill}%
\end{pgfscope}%
\begin{pgfscope}%
\pgfpathrectangle{\pgfqpoint{1.432000in}{0.528000in}}{\pgfqpoint{3.696000in}{3.696000in}}%
\pgfusepath{clip}%
\pgfsetbuttcap%
\pgfsetroundjoin%
\definecolor{currentfill}{rgb}{0.000000,0.000000,0.000000}%
\pgfsetfillcolor{currentfill}%
\pgfsetlinewidth{1.003750pt}%
\definecolor{currentstroke}{rgb}{0.000000,0.000000,0.000000}%
\pgfsetstrokecolor{currentstroke}%
\pgfsetdash{}{0pt}%
\pgfpathmoveto{\pgfqpoint{4.624000in}{0.671220in}}%
\pgfpathcurveto{\pgfqpoint{4.630572in}{0.671220in}}{\pgfqpoint{4.636875in}{0.673831in}}{\pgfqpoint{4.641522in}{0.678478in}}%
\pgfpathcurveto{\pgfqpoint{4.646169in}{0.683125in}}{\pgfqpoint{4.648780in}{0.689428in}}{\pgfqpoint{4.648780in}{0.696000in}}%
\pgfpathcurveto{\pgfqpoint{4.648780in}{0.702572in}}{\pgfqpoint{4.646169in}{0.708875in}}{\pgfqpoint{4.641522in}{0.713522in}}%
\pgfpathcurveto{\pgfqpoint{4.636875in}{0.718169in}}{\pgfqpoint{4.630572in}{0.720780in}}{\pgfqpoint{4.624000in}{0.720780in}}%
\pgfpathcurveto{\pgfqpoint{4.617428in}{0.720780in}}{\pgfqpoint{4.611125in}{0.718169in}}{\pgfqpoint{4.606478in}{0.713522in}}%
\pgfpathcurveto{\pgfqpoint{4.601831in}{0.708875in}}{\pgfqpoint{4.599220in}{0.702572in}}{\pgfqpoint{4.599220in}{0.696000in}}%
\pgfpathcurveto{\pgfqpoint{4.599220in}{0.689428in}}{\pgfqpoint{4.601831in}{0.683125in}}{\pgfqpoint{4.606478in}{0.678478in}}%
\pgfpathcurveto{\pgfqpoint{4.611125in}{0.673831in}}{\pgfqpoint{4.617428in}{0.671220in}}{\pgfqpoint{4.624000in}{0.671220in}}%
\pgfpathclose%
\pgfusepath{stroke,fill}%
\end{pgfscope}%
\begin{pgfscope}%
\pgfpathrectangle{\pgfqpoint{1.432000in}{0.528000in}}{\pgfqpoint{3.696000in}{3.696000in}}%
\pgfusepath{clip}%
\pgfsetbuttcap%
\pgfsetroundjoin%
\definecolor{currentfill}{rgb}{0.000000,0.000000,0.000000}%
\pgfsetfillcolor{currentfill}%
\pgfsetlinewidth{1.003750pt}%
\definecolor{currentstroke}{rgb}{0.000000,0.000000,0.000000}%
\pgfsetstrokecolor{currentstroke}%
\pgfsetdash{}{0pt}%
\pgfpathmoveto{\pgfqpoint{4.624000in}{1.007220in}}%
\pgfpathcurveto{\pgfqpoint{4.630572in}{1.007220in}}{\pgfqpoint{4.636875in}{1.009831in}}{\pgfqpoint{4.641522in}{1.014478in}}%
\pgfpathcurveto{\pgfqpoint{4.646169in}{1.019125in}}{\pgfqpoint{4.648780in}{1.025428in}}{\pgfqpoint{4.648780in}{1.032000in}}%
\pgfpathcurveto{\pgfqpoint{4.648780in}{1.038572in}}{\pgfqpoint{4.646169in}{1.044875in}}{\pgfqpoint{4.641522in}{1.049522in}}%
\pgfpathcurveto{\pgfqpoint{4.636875in}{1.054169in}}{\pgfqpoint{4.630572in}{1.056780in}}{\pgfqpoint{4.624000in}{1.056780in}}%
\pgfpathcurveto{\pgfqpoint{4.617428in}{1.056780in}}{\pgfqpoint{4.611125in}{1.054169in}}{\pgfqpoint{4.606478in}{1.049522in}}%
\pgfpathcurveto{\pgfqpoint{4.601831in}{1.044875in}}{\pgfqpoint{4.599220in}{1.038572in}}{\pgfqpoint{4.599220in}{1.032000in}}%
\pgfpathcurveto{\pgfqpoint{4.599220in}{1.025428in}}{\pgfqpoint{4.601831in}{1.019125in}}{\pgfqpoint{4.606478in}{1.014478in}}%
\pgfpathcurveto{\pgfqpoint{4.611125in}{1.009831in}}{\pgfqpoint{4.617428in}{1.007220in}}{\pgfqpoint{4.624000in}{1.007220in}}%
\pgfpathclose%
\pgfusepath{stroke,fill}%
\end{pgfscope}%
\begin{pgfscope}%
\pgfpathrectangle{\pgfqpoint{1.432000in}{0.528000in}}{\pgfqpoint{3.696000in}{3.696000in}}%
\pgfusepath{clip}%
\pgfsetbuttcap%
\pgfsetroundjoin%
\definecolor{currentfill}{rgb}{0.000000,0.000000,0.000000}%
\pgfsetfillcolor{currentfill}%
\pgfsetlinewidth{1.003750pt}%
\definecolor{currentstroke}{rgb}{0.000000,0.000000,0.000000}%
\pgfsetstrokecolor{currentstroke}%
\pgfsetdash{}{0pt}%
\pgfpathmoveto{\pgfqpoint{4.624000in}{1.343220in}}%
\pgfpathcurveto{\pgfqpoint{4.630572in}{1.343220in}}{\pgfqpoint{4.636875in}{1.345831in}}{\pgfqpoint{4.641522in}{1.350478in}}%
\pgfpathcurveto{\pgfqpoint{4.646169in}{1.355125in}}{\pgfqpoint{4.648780in}{1.361428in}}{\pgfqpoint{4.648780in}{1.368000in}}%
\pgfpathcurveto{\pgfqpoint{4.648780in}{1.374572in}}{\pgfqpoint{4.646169in}{1.380875in}}{\pgfqpoint{4.641522in}{1.385522in}}%
\pgfpathcurveto{\pgfqpoint{4.636875in}{1.390169in}}{\pgfqpoint{4.630572in}{1.392780in}}{\pgfqpoint{4.624000in}{1.392780in}}%
\pgfpathcurveto{\pgfqpoint{4.617428in}{1.392780in}}{\pgfqpoint{4.611125in}{1.390169in}}{\pgfqpoint{4.606478in}{1.385522in}}%
\pgfpathcurveto{\pgfqpoint{4.601831in}{1.380875in}}{\pgfqpoint{4.599220in}{1.374572in}}{\pgfqpoint{4.599220in}{1.368000in}}%
\pgfpathcurveto{\pgfqpoint{4.599220in}{1.361428in}}{\pgfqpoint{4.601831in}{1.355125in}}{\pgfqpoint{4.606478in}{1.350478in}}%
\pgfpathcurveto{\pgfqpoint{4.611125in}{1.345831in}}{\pgfqpoint{4.617428in}{1.343220in}}{\pgfqpoint{4.624000in}{1.343220in}}%
\pgfpathclose%
\pgfusepath{stroke,fill}%
\end{pgfscope}%
\begin{pgfscope}%
\pgfpathrectangle{\pgfqpoint{1.432000in}{0.528000in}}{\pgfqpoint{3.696000in}{3.696000in}}%
\pgfusepath{clip}%
\pgfsetbuttcap%
\pgfsetroundjoin%
\definecolor{currentfill}{rgb}{0.000000,0.000000,0.000000}%
\pgfsetfillcolor{currentfill}%
\pgfsetlinewidth{1.003750pt}%
\definecolor{currentstroke}{rgb}{0.000000,0.000000,0.000000}%
\pgfsetstrokecolor{currentstroke}%
\pgfsetdash{}{0pt}%
\pgfpathmoveto{\pgfqpoint{4.624000in}{1.679220in}}%
\pgfpathcurveto{\pgfqpoint{4.630572in}{1.679220in}}{\pgfqpoint{4.636875in}{1.681831in}}{\pgfqpoint{4.641522in}{1.686478in}}%
\pgfpathcurveto{\pgfqpoint{4.646169in}{1.691125in}}{\pgfqpoint{4.648780in}{1.697428in}}{\pgfqpoint{4.648780in}{1.704000in}}%
\pgfpathcurveto{\pgfqpoint{4.648780in}{1.710572in}}{\pgfqpoint{4.646169in}{1.716875in}}{\pgfqpoint{4.641522in}{1.721522in}}%
\pgfpathcurveto{\pgfqpoint{4.636875in}{1.726169in}}{\pgfqpoint{4.630572in}{1.728780in}}{\pgfqpoint{4.624000in}{1.728780in}}%
\pgfpathcurveto{\pgfqpoint{4.617428in}{1.728780in}}{\pgfqpoint{4.611125in}{1.726169in}}{\pgfqpoint{4.606478in}{1.721522in}}%
\pgfpathcurveto{\pgfqpoint{4.601831in}{1.716875in}}{\pgfqpoint{4.599220in}{1.710572in}}{\pgfqpoint{4.599220in}{1.704000in}}%
\pgfpathcurveto{\pgfqpoint{4.599220in}{1.697428in}}{\pgfqpoint{4.601831in}{1.691125in}}{\pgfqpoint{4.606478in}{1.686478in}}%
\pgfpathcurveto{\pgfqpoint{4.611125in}{1.681831in}}{\pgfqpoint{4.617428in}{1.679220in}}{\pgfqpoint{4.624000in}{1.679220in}}%
\pgfpathclose%
\pgfusepath{stroke,fill}%
\end{pgfscope}%
\begin{pgfscope}%
\pgfpathrectangle{\pgfqpoint{1.432000in}{0.528000in}}{\pgfqpoint{3.696000in}{3.696000in}}%
\pgfusepath{clip}%
\pgfsetbuttcap%
\pgfsetroundjoin%
\definecolor{currentfill}{rgb}{0.000000,0.000000,0.000000}%
\pgfsetfillcolor{currentfill}%
\pgfsetlinewidth{1.003750pt}%
\definecolor{currentstroke}{rgb}{0.000000,0.000000,0.000000}%
\pgfsetstrokecolor{currentstroke}%
\pgfsetdash{}{0pt}%
\pgfpathmoveto{\pgfqpoint{4.624000in}{2.015220in}}%
\pgfpathcurveto{\pgfqpoint{4.630572in}{2.015220in}}{\pgfqpoint{4.636875in}{2.017831in}}{\pgfqpoint{4.641522in}{2.022478in}}%
\pgfpathcurveto{\pgfqpoint{4.646169in}{2.027125in}}{\pgfqpoint{4.648780in}{2.033428in}}{\pgfqpoint{4.648780in}{2.040000in}}%
\pgfpathcurveto{\pgfqpoint{4.648780in}{2.046572in}}{\pgfqpoint{4.646169in}{2.052875in}}{\pgfqpoint{4.641522in}{2.057522in}}%
\pgfpathcurveto{\pgfqpoint{4.636875in}{2.062169in}}{\pgfqpoint{4.630572in}{2.064780in}}{\pgfqpoint{4.624000in}{2.064780in}}%
\pgfpathcurveto{\pgfqpoint{4.617428in}{2.064780in}}{\pgfqpoint{4.611125in}{2.062169in}}{\pgfqpoint{4.606478in}{2.057522in}}%
\pgfpathcurveto{\pgfqpoint{4.601831in}{2.052875in}}{\pgfqpoint{4.599220in}{2.046572in}}{\pgfqpoint{4.599220in}{2.040000in}}%
\pgfpathcurveto{\pgfqpoint{4.599220in}{2.033428in}}{\pgfqpoint{4.601831in}{2.027125in}}{\pgfqpoint{4.606478in}{2.022478in}}%
\pgfpathcurveto{\pgfqpoint{4.611125in}{2.017831in}}{\pgfqpoint{4.617428in}{2.015220in}}{\pgfqpoint{4.624000in}{2.015220in}}%
\pgfpathclose%
\pgfusepath{stroke,fill}%
\end{pgfscope}%
\begin{pgfscope}%
\pgfpathrectangle{\pgfqpoint{1.432000in}{0.528000in}}{\pgfqpoint{3.696000in}{3.696000in}}%
\pgfusepath{clip}%
\pgfsetbuttcap%
\pgfsetroundjoin%
\definecolor{currentfill}{rgb}{0.000000,0.000000,0.000000}%
\pgfsetfillcolor{currentfill}%
\pgfsetlinewidth{1.003750pt}%
\definecolor{currentstroke}{rgb}{0.000000,0.000000,0.000000}%
\pgfsetstrokecolor{currentstroke}%
\pgfsetdash{}{0pt}%
\pgfpathmoveto{\pgfqpoint{4.624000in}{2.351220in}}%
\pgfpathcurveto{\pgfqpoint{4.630572in}{2.351220in}}{\pgfqpoint{4.636875in}{2.353831in}}{\pgfqpoint{4.641522in}{2.358478in}}%
\pgfpathcurveto{\pgfqpoint{4.646169in}{2.363125in}}{\pgfqpoint{4.648780in}{2.369428in}}{\pgfqpoint{4.648780in}{2.376000in}}%
\pgfpathcurveto{\pgfqpoint{4.648780in}{2.382572in}}{\pgfqpoint{4.646169in}{2.388875in}}{\pgfqpoint{4.641522in}{2.393522in}}%
\pgfpathcurveto{\pgfqpoint{4.636875in}{2.398169in}}{\pgfqpoint{4.630572in}{2.400780in}}{\pgfqpoint{4.624000in}{2.400780in}}%
\pgfpathcurveto{\pgfqpoint{4.617428in}{2.400780in}}{\pgfqpoint{4.611125in}{2.398169in}}{\pgfqpoint{4.606478in}{2.393522in}}%
\pgfpathcurveto{\pgfqpoint{4.601831in}{2.388875in}}{\pgfqpoint{4.599220in}{2.382572in}}{\pgfqpoint{4.599220in}{2.376000in}}%
\pgfpathcurveto{\pgfqpoint{4.599220in}{2.369428in}}{\pgfqpoint{4.601831in}{2.363125in}}{\pgfqpoint{4.606478in}{2.358478in}}%
\pgfpathcurveto{\pgfqpoint{4.611125in}{2.353831in}}{\pgfqpoint{4.617428in}{2.351220in}}{\pgfqpoint{4.624000in}{2.351220in}}%
\pgfpathclose%
\pgfusepath{stroke,fill}%
\end{pgfscope}%
\begin{pgfscope}%
\pgfpathrectangle{\pgfqpoint{1.432000in}{0.528000in}}{\pgfqpoint{3.696000in}{3.696000in}}%
\pgfusepath{clip}%
\pgfsetbuttcap%
\pgfsetroundjoin%
\definecolor{currentfill}{rgb}{0.000000,0.000000,0.000000}%
\pgfsetfillcolor{currentfill}%
\pgfsetlinewidth{1.003750pt}%
\definecolor{currentstroke}{rgb}{0.000000,0.000000,0.000000}%
\pgfsetstrokecolor{currentstroke}%
\pgfsetdash{}{0pt}%
\pgfpathmoveto{\pgfqpoint{4.624000in}{2.687220in}}%
\pgfpathcurveto{\pgfqpoint{4.630572in}{2.687220in}}{\pgfqpoint{4.636875in}{2.689831in}}{\pgfqpoint{4.641522in}{2.694478in}}%
\pgfpathcurveto{\pgfqpoint{4.646169in}{2.699125in}}{\pgfqpoint{4.648780in}{2.705428in}}{\pgfqpoint{4.648780in}{2.712000in}}%
\pgfpathcurveto{\pgfqpoint{4.648780in}{2.718572in}}{\pgfqpoint{4.646169in}{2.724875in}}{\pgfqpoint{4.641522in}{2.729522in}}%
\pgfpathcurveto{\pgfqpoint{4.636875in}{2.734169in}}{\pgfqpoint{4.630572in}{2.736780in}}{\pgfqpoint{4.624000in}{2.736780in}}%
\pgfpathcurveto{\pgfqpoint{4.617428in}{2.736780in}}{\pgfqpoint{4.611125in}{2.734169in}}{\pgfqpoint{4.606478in}{2.729522in}}%
\pgfpathcurveto{\pgfqpoint{4.601831in}{2.724875in}}{\pgfqpoint{4.599220in}{2.718572in}}{\pgfqpoint{4.599220in}{2.712000in}}%
\pgfpathcurveto{\pgfqpoint{4.599220in}{2.705428in}}{\pgfqpoint{4.601831in}{2.699125in}}{\pgfqpoint{4.606478in}{2.694478in}}%
\pgfpathcurveto{\pgfqpoint{4.611125in}{2.689831in}}{\pgfqpoint{4.617428in}{2.687220in}}{\pgfqpoint{4.624000in}{2.687220in}}%
\pgfpathclose%
\pgfusepath{stroke,fill}%
\end{pgfscope}%
\begin{pgfscope}%
\pgfpathrectangle{\pgfqpoint{1.432000in}{0.528000in}}{\pgfqpoint{3.696000in}{3.696000in}}%
\pgfusepath{clip}%
\pgfsetbuttcap%
\pgfsetroundjoin%
\definecolor{currentfill}{rgb}{0.000000,0.000000,0.000000}%
\pgfsetfillcolor{currentfill}%
\pgfsetlinewidth{1.003750pt}%
\definecolor{currentstroke}{rgb}{0.000000,0.000000,0.000000}%
\pgfsetstrokecolor{currentstroke}%
\pgfsetdash{}{0pt}%
\pgfpathmoveto{\pgfqpoint{4.624000in}{3.023220in}}%
\pgfpathcurveto{\pgfqpoint{4.630572in}{3.023220in}}{\pgfqpoint{4.636875in}{3.025831in}}{\pgfqpoint{4.641522in}{3.030478in}}%
\pgfpathcurveto{\pgfqpoint{4.646169in}{3.035125in}}{\pgfqpoint{4.648780in}{3.041428in}}{\pgfqpoint{4.648780in}{3.048000in}}%
\pgfpathcurveto{\pgfqpoint{4.648780in}{3.054572in}}{\pgfqpoint{4.646169in}{3.060875in}}{\pgfqpoint{4.641522in}{3.065522in}}%
\pgfpathcurveto{\pgfqpoint{4.636875in}{3.070169in}}{\pgfqpoint{4.630572in}{3.072780in}}{\pgfqpoint{4.624000in}{3.072780in}}%
\pgfpathcurveto{\pgfqpoint{4.617428in}{3.072780in}}{\pgfqpoint{4.611125in}{3.070169in}}{\pgfqpoint{4.606478in}{3.065522in}}%
\pgfpathcurveto{\pgfqpoint{4.601831in}{3.060875in}}{\pgfqpoint{4.599220in}{3.054572in}}{\pgfqpoint{4.599220in}{3.048000in}}%
\pgfpathcurveto{\pgfqpoint{4.599220in}{3.041428in}}{\pgfqpoint{4.601831in}{3.035125in}}{\pgfqpoint{4.606478in}{3.030478in}}%
\pgfpathcurveto{\pgfqpoint{4.611125in}{3.025831in}}{\pgfqpoint{4.617428in}{3.023220in}}{\pgfqpoint{4.624000in}{3.023220in}}%
\pgfpathclose%
\pgfusepath{stroke,fill}%
\end{pgfscope}%
\begin{pgfscope}%
\pgfpathrectangle{\pgfqpoint{1.432000in}{0.528000in}}{\pgfqpoint{3.696000in}{3.696000in}}%
\pgfusepath{clip}%
\pgfsetbuttcap%
\pgfsetroundjoin%
\definecolor{currentfill}{rgb}{0.000000,0.000000,0.000000}%
\pgfsetfillcolor{currentfill}%
\pgfsetlinewidth{1.003750pt}%
\definecolor{currentstroke}{rgb}{0.000000,0.000000,0.000000}%
\pgfsetstrokecolor{currentstroke}%
\pgfsetdash{}{0pt}%
\pgfpathmoveto{\pgfqpoint{4.624000in}{3.359220in}}%
\pgfpathcurveto{\pgfqpoint{4.630572in}{3.359220in}}{\pgfqpoint{4.636875in}{3.361831in}}{\pgfqpoint{4.641522in}{3.366478in}}%
\pgfpathcurveto{\pgfqpoint{4.646169in}{3.371125in}}{\pgfqpoint{4.648780in}{3.377428in}}{\pgfqpoint{4.648780in}{3.384000in}}%
\pgfpathcurveto{\pgfqpoint{4.648780in}{3.390572in}}{\pgfqpoint{4.646169in}{3.396875in}}{\pgfqpoint{4.641522in}{3.401522in}}%
\pgfpathcurveto{\pgfqpoint{4.636875in}{3.406169in}}{\pgfqpoint{4.630572in}{3.408780in}}{\pgfqpoint{4.624000in}{3.408780in}}%
\pgfpathcurveto{\pgfqpoint{4.617428in}{3.408780in}}{\pgfqpoint{4.611125in}{3.406169in}}{\pgfqpoint{4.606478in}{3.401522in}}%
\pgfpathcurveto{\pgfqpoint{4.601831in}{3.396875in}}{\pgfqpoint{4.599220in}{3.390572in}}{\pgfqpoint{4.599220in}{3.384000in}}%
\pgfpathcurveto{\pgfqpoint{4.599220in}{3.377428in}}{\pgfqpoint{4.601831in}{3.371125in}}{\pgfqpoint{4.606478in}{3.366478in}}%
\pgfpathcurveto{\pgfqpoint{4.611125in}{3.361831in}}{\pgfqpoint{4.617428in}{3.359220in}}{\pgfqpoint{4.624000in}{3.359220in}}%
\pgfpathclose%
\pgfusepath{stroke,fill}%
\end{pgfscope}%
\begin{pgfscope}%
\pgfpathrectangle{\pgfqpoint{1.432000in}{0.528000in}}{\pgfqpoint{3.696000in}{3.696000in}}%
\pgfusepath{clip}%
\pgfsetbuttcap%
\pgfsetroundjoin%
\definecolor{currentfill}{rgb}{0.000000,0.000000,0.000000}%
\pgfsetfillcolor{currentfill}%
\pgfsetlinewidth{1.003750pt}%
\definecolor{currentstroke}{rgb}{0.000000,0.000000,0.000000}%
\pgfsetstrokecolor{currentstroke}%
\pgfsetdash{}{0pt}%
\pgfpathmoveto{\pgfqpoint{4.624000in}{3.695220in}}%
\pgfpathcurveto{\pgfqpoint{4.630572in}{3.695220in}}{\pgfqpoint{4.636875in}{3.697831in}}{\pgfqpoint{4.641522in}{3.702478in}}%
\pgfpathcurveto{\pgfqpoint{4.646169in}{3.707125in}}{\pgfqpoint{4.648780in}{3.713428in}}{\pgfqpoint{4.648780in}{3.720000in}}%
\pgfpathcurveto{\pgfqpoint{4.648780in}{3.726572in}}{\pgfqpoint{4.646169in}{3.732875in}}{\pgfqpoint{4.641522in}{3.737522in}}%
\pgfpathcurveto{\pgfqpoint{4.636875in}{3.742169in}}{\pgfqpoint{4.630572in}{3.744780in}}{\pgfqpoint{4.624000in}{3.744780in}}%
\pgfpathcurveto{\pgfqpoint{4.617428in}{3.744780in}}{\pgfqpoint{4.611125in}{3.742169in}}{\pgfqpoint{4.606478in}{3.737522in}}%
\pgfpathcurveto{\pgfqpoint{4.601831in}{3.732875in}}{\pgfqpoint{4.599220in}{3.726572in}}{\pgfqpoint{4.599220in}{3.720000in}}%
\pgfpathcurveto{\pgfqpoint{4.599220in}{3.713428in}}{\pgfqpoint{4.601831in}{3.707125in}}{\pgfqpoint{4.606478in}{3.702478in}}%
\pgfpathcurveto{\pgfqpoint{4.611125in}{3.697831in}}{\pgfqpoint{4.617428in}{3.695220in}}{\pgfqpoint{4.624000in}{3.695220in}}%
\pgfpathclose%
\pgfusepath{stroke,fill}%
\end{pgfscope}%
\begin{pgfscope}%
\pgfpathrectangle{\pgfqpoint{1.432000in}{0.528000in}}{\pgfqpoint{3.696000in}{3.696000in}}%
\pgfusepath{clip}%
\pgfsetbuttcap%
\pgfsetroundjoin%
\definecolor{currentfill}{rgb}{0.000000,0.000000,0.000000}%
\pgfsetfillcolor{currentfill}%
\pgfsetlinewidth{1.003750pt}%
\definecolor{currentstroke}{rgb}{0.000000,0.000000,0.000000}%
\pgfsetstrokecolor{currentstroke}%
\pgfsetdash{}{0pt}%
\pgfpathmoveto{\pgfqpoint{4.624000in}{4.031220in}}%
\pgfpathcurveto{\pgfqpoint{4.630572in}{4.031220in}}{\pgfqpoint{4.636875in}{4.033831in}}{\pgfqpoint{4.641522in}{4.038478in}}%
\pgfpathcurveto{\pgfqpoint{4.646169in}{4.043125in}}{\pgfqpoint{4.648780in}{4.049428in}}{\pgfqpoint{4.648780in}{4.056000in}}%
\pgfpathcurveto{\pgfqpoint{4.648780in}{4.062572in}}{\pgfqpoint{4.646169in}{4.068875in}}{\pgfqpoint{4.641522in}{4.073522in}}%
\pgfpathcurveto{\pgfqpoint{4.636875in}{4.078169in}}{\pgfqpoint{4.630572in}{4.080780in}}{\pgfqpoint{4.624000in}{4.080780in}}%
\pgfpathcurveto{\pgfqpoint{4.617428in}{4.080780in}}{\pgfqpoint{4.611125in}{4.078169in}}{\pgfqpoint{4.606478in}{4.073522in}}%
\pgfpathcurveto{\pgfqpoint{4.601831in}{4.068875in}}{\pgfqpoint{4.599220in}{4.062572in}}{\pgfqpoint{4.599220in}{4.056000in}}%
\pgfpathcurveto{\pgfqpoint{4.599220in}{4.049428in}}{\pgfqpoint{4.601831in}{4.043125in}}{\pgfqpoint{4.606478in}{4.038478in}}%
\pgfpathcurveto{\pgfqpoint{4.611125in}{4.033831in}}{\pgfqpoint{4.617428in}{4.031220in}}{\pgfqpoint{4.624000in}{4.031220in}}%
\pgfpathclose%
\pgfusepath{stroke,fill}%
\end{pgfscope}%
\begin{pgfscope}%
\pgfpathrectangle{\pgfqpoint{1.432000in}{0.528000in}}{\pgfqpoint{3.696000in}{3.696000in}}%
\pgfusepath{clip}%
\pgfsetbuttcap%
\pgfsetroundjoin%
\definecolor{currentfill}{rgb}{0.000000,0.000000,0.000000}%
\pgfsetfillcolor{currentfill}%
\pgfsetlinewidth{1.003750pt}%
\definecolor{currentstroke}{rgb}{0.000000,0.000000,0.000000}%
\pgfsetstrokecolor{currentstroke}%
\pgfsetdash{}{0pt}%
\pgfpathmoveto{\pgfqpoint{4.960000in}{0.671220in}}%
\pgfpathcurveto{\pgfqpoint{4.966572in}{0.671220in}}{\pgfqpoint{4.972875in}{0.673831in}}{\pgfqpoint{4.977522in}{0.678478in}}%
\pgfpathcurveto{\pgfqpoint{4.982169in}{0.683125in}}{\pgfqpoint{4.984780in}{0.689428in}}{\pgfqpoint{4.984780in}{0.696000in}}%
\pgfpathcurveto{\pgfqpoint{4.984780in}{0.702572in}}{\pgfqpoint{4.982169in}{0.708875in}}{\pgfqpoint{4.977522in}{0.713522in}}%
\pgfpathcurveto{\pgfqpoint{4.972875in}{0.718169in}}{\pgfqpoint{4.966572in}{0.720780in}}{\pgfqpoint{4.960000in}{0.720780in}}%
\pgfpathcurveto{\pgfqpoint{4.953428in}{0.720780in}}{\pgfqpoint{4.947125in}{0.718169in}}{\pgfqpoint{4.942478in}{0.713522in}}%
\pgfpathcurveto{\pgfqpoint{4.937831in}{0.708875in}}{\pgfqpoint{4.935220in}{0.702572in}}{\pgfqpoint{4.935220in}{0.696000in}}%
\pgfpathcurveto{\pgfqpoint{4.935220in}{0.689428in}}{\pgfqpoint{4.937831in}{0.683125in}}{\pgfqpoint{4.942478in}{0.678478in}}%
\pgfpathcurveto{\pgfqpoint{4.947125in}{0.673831in}}{\pgfqpoint{4.953428in}{0.671220in}}{\pgfqpoint{4.960000in}{0.671220in}}%
\pgfpathclose%
\pgfusepath{stroke,fill}%
\end{pgfscope}%
\begin{pgfscope}%
\pgfpathrectangle{\pgfqpoint{1.432000in}{0.528000in}}{\pgfqpoint{3.696000in}{3.696000in}}%
\pgfusepath{clip}%
\pgfsetbuttcap%
\pgfsetroundjoin%
\definecolor{currentfill}{rgb}{0.000000,0.000000,0.000000}%
\pgfsetfillcolor{currentfill}%
\pgfsetlinewidth{1.003750pt}%
\definecolor{currentstroke}{rgb}{0.000000,0.000000,0.000000}%
\pgfsetstrokecolor{currentstroke}%
\pgfsetdash{}{0pt}%
\pgfpathmoveto{\pgfqpoint{4.960000in}{1.007220in}}%
\pgfpathcurveto{\pgfqpoint{4.966572in}{1.007220in}}{\pgfqpoint{4.972875in}{1.009831in}}{\pgfqpoint{4.977522in}{1.014478in}}%
\pgfpathcurveto{\pgfqpoint{4.982169in}{1.019125in}}{\pgfqpoint{4.984780in}{1.025428in}}{\pgfqpoint{4.984780in}{1.032000in}}%
\pgfpathcurveto{\pgfqpoint{4.984780in}{1.038572in}}{\pgfqpoint{4.982169in}{1.044875in}}{\pgfqpoint{4.977522in}{1.049522in}}%
\pgfpathcurveto{\pgfqpoint{4.972875in}{1.054169in}}{\pgfqpoint{4.966572in}{1.056780in}}{\pgfqpoint{4.960000in}{1.056780in}}%
\pgfpathcurveto{\pgfqpoint{4.953428in}{1.056780in}}{\pgfqpoint{4.947125in}{1.054169in}}{\pgfqpoint{4.942478in}{1.049522in}}%
\pgfpathcurveto{\pgfqpoint{4.937831in}{1.044875in}}{\pgfqpoint{4.935220in}{1.038572in}}{\pgfqpoint{4.935220in}{1.032000in}}%
\pgfpathcurveto{\pgfqpoint{4.935220in}{1.025428in}}{\pgfqpoint{4.937831in}{1.019125in}}{\pgfqpoint{4.942478in}{1.014478in}}%
\pgfpathcurveto{\pgfqpoint{4.947125in}{1.009831in}}{\pgfqpoint{4.953428in}{1.007220in}}{\pgfqpoint{4.960000in}{1.007220in}}%
\pgfpathclose%
\pgfusepath{stroke,fill}%
\end{pgfscope}%
\begin{pgfscope}%
\pgfpathrectangle{\pgfqpoint{1.432000in}{0.528000in}}{\pgfqpoint{3.696000in}{3.696000in}}%
\pgfusepath{clip}%
\pgfsetbuttcap%
\pgfsetroundjoin%
\definecolor{currentfill}{rgb}{0.000000,0.000000,0.000000}%
\pgfsetfillcolor{currentfill}%
\pgfsetlinewidth{1.003750pt}%
\definecolor{currentstroke}{rgb}{0.000000,0.000000,0.000000}%
\pgfsetstrokecolor{currentstroke}%
\pgfsetdash{}{0pt}%
\pgfpathmoveto{\pgfqpoint{4.960000in}{1.343220in}}%
\pgfpathcurveto{\pgfqpoint{4.966572in}{1.343220in}}{\pgfqpoint{4.972875in}{1.345831in}}{\pgfqpoint{4.977522in}{1.350478in}}%
\pgfpathcurveto{\pgfqpoint{4.982169in}{1.355125in}}{\pgfqpoint{4.984780in}{1.361428in}}{\pgfqpoint{4.984780in}{1.368000in}}%
\pgfpathcurveto{\pgfqpoint{4.984780in}{1.374572in}}{\pgfqpoint{4.982169in}{1.380875in}}{\pgfqpoint{4.977522in}{1.385522in}}%
\pgfpathcurveto{\pgfqpoint{4.972875in}{1.390169in}}{\pgfqpoint{4.966572in}{1.392780in}}{\pgfqpoint{4.960000in}{1.392780in}}%
\pgfpathcurveto{\pgfqpoint{4.953428in}{1.392780in}}{\pgfqpoint{4.947125in}{1.390169in}}{\pgfqpoint{4.942478in}{1.385522in}}%
\pgfpathcurveto{\pgfqpoint{4.937831in}{1.380875in}}{\pgfqpoint{4.935220in}{1.374572in}}{\pgfqpoint{4.935220in}{1.368000in}}%
\pgfpathcurveto{\pgfqpoint{4.935220in}{1.361428in}}{\pgfqpoint{4.937831in}{1.355125in}}{\pgfqpoint{4.942478in}{1.350478in}}%
\pgfpathcurveto{\pgfqpoint{4.947125in}{1.345831in}}{\pgfqpoint{4.953428in}{1.343220in}}{\pgfqpoint{4.960000in}{1.343220in}}%
\pgfpathclose%
\pgfusepath{stroke,fill}%
\end{pgfscope}%
\begin{pgfscope}%
\pgfpathrectangle{\pgfqpoint{1.432000in}{0.528000in}}{\pgfqpoint{3.696000in}{3.696000in}}%
\pgfusepath{clip}%
\pgfsetbuttcap%
\pgfsetroundjoin%
\definecolor{currentfill}{rgb}{0.000000,0.000000,0.000000}%
\pgfsetfillcolor{currentfill}%
\pgfsetlinewidth{1.003750pt}%
\definecolor{currentstroke}{rgb}{0.000000,0.000000,0.000000}%
\pgfsetstrokecolor{currentstroke}%
\pgfsetdash{}{0pt}%
\pgfpathmoveto{\pgfqpoint{4.960000in}{1.679220in}}%
\pgfpathcurveto{\pgfqpoint{4.966572in}{1.679220in}}{\pgfqpoint{4.972875in}{1.681831in}}{\pgfqpoint{4.977522in}{1.686478in}}%
\pgfpathcurveto{\pgfqpoint{4.982169in}{1.691125in}}{\pgfqpoint{4.984780in}{1.697428in}}{\pgfqpoint{4.984780in}{1.704000in}}%
\pgfpathcurveto{\pgfqpoint{4.984780in}{1.710572in}}{\pgfqpoint{4.982169in}{1.716875in}}{\pgfqpoint{4.977522in}{1.721522in}}%
\pgfpathcurveto{\pgfqpoint{4.972875in}{1.726169in}}{\pgfqpoint{4.966572in}{1.728780in}}{\pgfqpoint{4.960000in}{1.728780in}}%
\pgfpathcurveto{\pgfqpoint{4.953428in}{1.728780in}}{\pgfqpoint{4.947125in}{1.726169in}}{\pgfqpoint{4.942478in}{1.721522in}}%
\pgfpathcurveto{\pgfqpoint{4.937831in}{1.716875in}}{\pgfqpoint{4.935220in}{1.710572in}}{\pgfqpoint{4.935220in}{1.704000in}}%
\pgfpathcurveto{\pgfqpoint{4.935220in}{1.697428in}}{\pgfqpoint{4.937831in}{1.691125in}}{\pgfqpoint{4.942478in}{1.686478in}}%
\pgfpathcurveto{\pgfqpoint{4.947125in}{1.681831in}}{\pgfqpoint{4.953428in}{1.679220in}}{\pgfqpoint{4.960000in}{1.679220in}}%
\pgfpathclose%
\pgfusepath{stroke,fill}%
\end{pgfscope}%
\begin{pgfscope}%
\pgfpathrectangle{\pgfqpoint{1.432000in}{0.528000in}}{\pgfqpoint{3.696000in}{3.696000in}}%
\pgfusepath{clip}%
\pgfsetbuttcap%
\pgfsetroundjoin%
\definecolor{currentfill}{rgb}{0.000000,0.000000,0.000000}%
\pgfsetfillcolor{currentfill}%
\pgfsetlinewidth{1.003750pt}%
\definecolor{currentstroke}{rgb}{0.000000,0.000000,0.000000}%
\pgfsetstrokecolor{currentstroke}%
\pgfsetdash{}{0pt}%
\pgfpathmoveto{\pgfqpoint{4.960000in}{2.015220in}}%
\pgfpathcurveto{\pgfqpoint{4.966572in}{2.015220in}}{\pgfqpoint{4.972875in}{2.017831in}}{\pgfqpoint{4.977522in}{2.022478in}}%
\pgfpathcurveto{\pgfqpoint{4.982169in}{2.027125in}}{\pgfqpoint{4.984780in}{2.033428in}}{\pgfqpoint{4.984780in}{2.040000in}}%
\pgfpathcurveto{\pgfqpoint{4.984780in}{2.046572in}}{\pgfqpoint{4.982169in}{2.052875in}}{\pgfqpoint{4.977522in}{2.057522in}}%
\pgfpathcurveto{\pgfqpoint{4.972875in}{2.062169in}}{\pgfqpoint{4.966572in}{2.064780in}}{\pgfqpoint{4.960000in}{2.064780in}}%
\pgfpathcurveto{\pgfqpoint{4.953428in}{2.064780in}}{\pgfqpoint{4.947125in}{2.062169in}}{\pgfqpoint{4.942478in}{2.057522in}}%
\pgfpathcurveto{\pgfqpoint{4.937831in}{2.052875in}}{\pgfqpoint{4.935220in}{2.046572in}}{\pgfqpoint{4.935220in}{2.040000in}}%
\pgfpathcurveto{\pgfqpoint{4.935220in}{2.033428in}}{\pgfqpoint{4.937831in}{2.027125in}}{\pgfqpoint{4.942478in}{2.022478in}}%
\pgfpathcurveto{\pgfqpoint{4.947125in}{2.017831in}}{\pgfqpoint{4.953428in}{2.015220in}}{\pgfqpoint{4.960000in}{2.015220in}}%
\pgfpathclose%
\pgfusepath{stroke,fill}%
\end{pgfscope}%
\begin{pgfscope}%
\pgfpathrectangle{\pgfqpoint{1.432000in}{0.528000in}}{\pgfqpoint{3.696000in}{3.696000in}}%
\pgfusepath{clip}%
\pgfsetbuttcap%
\pgfsetroundjoin%
\definecolor{currentfill}{rgb}{0.000000,0.000000,0.000000}%
\pgfsetfillcolor{currentfill}%
\pgfsetlinewidth{1.003750pt}%
\definecolor{currentstroke}{rgb}{0.000000,0.000000,0.000000}%
\pgfsetstrokecolor{currentstroke}%
\pgfsetdash{}{0pt}%
\pgfpathmoveto{\pgfqpoint{4.960000in}{2.351220in}}%
\pgfpathcurveto{\pgfqpoint{4.966572in}{2.351220in}}{\pgfqpoint{4.972875in}{2.353831in}}{\pgfqpoint{4.977522in}{2.358478in}}%
\pgfpathcurveto{\pgfqpoint{4.982169in}{2.363125in}}{\pgfqpoint{4.984780in}{2.369428in}}{\pgfqpoint{4.984780in}{2.376000in}}%
\pgfpathcurveto{\pgfqpoint{4.984780in}{2.382572in}}{\pgfqpoint{4.982169in}{2.388875in}}{\pgfqpoint{4.977522in}{2.393522in}}%
\pgfpathcurveto{\pgfqpoint{4.972875in}{2.398169in}}{\pgfqpoint{4.966572in}{2.400780in}}{\pgfqpoint{4.960000in}{2.400780in}}%
\pgfpathcurveto{\pgfqpoint{4.953428in}{2.400780in}}{\pgfqpoint{4.947125in}{2.398169in}}{\pgfqpoint{4.942478in}{2.393522in}}%
\pgfpathcurveto{\pgfqpoint{4.937831in}{2.388875in}}{\pgfqpoint{4.935220in}{2.382572in}}{\pgfqpoint{4.935220in}{2.376000in}}%
\pgfpathcurveto{\pgfqpoint{4.935220in}{2.369428in}}{\pgfqpoint{4.937831in}{2.363125in}}{\pgfqpoint{4.942478in}{2.358478in}}%
\pgfpathcurveto{\pgfqpoint{4.947125in}{2.353831in}}{\pgfqpoint{4.953428in}{2.351220in}}{\pgfqpoint{4.960000in}{2.351220in}}%
\pgfpathclose%
\pgfusepath{stroke,fill}%
\end{pgfscope}%
\begin{pgfscope}%
\pgfpathrectangle{\pgfqpoint{1.432000in}{0.528000in}}{\pgfqpoint{3.696000in}{3.696000in}}%
\pgfusepath{clip}%
\pgfsetbuttcap%
\pgfsetroundjoin%
\definecolor{currentfill}{rgb}{0.000000,0.000000,0.000000}%
\pgfsetfillcolor{currentfill}%
\pgfsetlinewidth{1.003750pt}%
\definecolor{currentstroke}{rgb}{0.000000,0.000000,0.000000}%
\pgfsetstrokecolor{currentstroke}%
\pgfsetdash{}{0pt}%
\pgfpathmoveto{\pgfqpoint{4.960000in}{2.687220in}}%
\pgfpathcurveto{\pgfqpoint{4.966572in}{2.687220in}}{\pgfqpoint{4.972875in}{2.689831in}}{\pgfqpoint{4.977522in}{2.694478in}}%
\pgfpathcurveto{\pgfqpoint{4.982169in}{2.699125in}}{\pgfqpoint{4.984780in}{2.705428in}}{\pgfqpoint{4.984780in}{2.712000in}}%
\pgfpathcurveto{\pgfqpoint{4.984780in}{2.718572in}}{\pgfqpoint{4.982169in}{2.724875in}}{\pgfqpoint{4.977522in}{2.729522in}}%
\pgfpathcurveto{\pgfqpoint{4.972875in}{2.734169in}}{\pgfqpoint{4.966572in}{2.736780in}}{\pgfqpoint{4.960000in}{2.736780in}}%
\pgfpathcurveto{\pgfqpoint{4.953428in}{2.736780in}}{\pgfqpoint{4.947125in}{2.734169in}}{\pgfqpoint{4.942478in}{2.729522in}}%
\pgfpathcurveto{\pgfqpoint{4.937831in}{2.724875in}}{\pgfqpoint{4.935220in}{2.718572in}}{\pgfqpoint{4.935220in}{2.712000in}}%
\pgfpathcurveto{\pgfqpoint{4.935220in}{2.705428in}}{\pgfqpoint{4.937831in}{2.699125in}}{\pgfqpoint{4.942478in}{2.694478in}}%
\pgfpathcurveto{\pgfqpoint{4.947125in}{2.689831in}}{\pgfqpoint{4.953428in}{2.687220in}}{\pgfqpoint{4.960000in}{2.687220in}}%
\pgfpathclose%
\pgfusepath{stroke,fill}%
\end{pgfscope}%
\begin{pgfscope}%
\pgfpathrectangle{\pgfqpoint{1.432000in}{0.528000in}}{\pgfqpoint{3.696000in}{3.696000in}}%
\pgfusepath{clip}%
\pgfsetbuttcap%
\pgfsetroundjoin%
\definecolor{currentfill}{rgb}{0.000000,0.000000,0.000000}%
\pgfsetfillcolor{currentfill}%
\pgfsetlinewidth{1.003750pt}%
\definecolor{currentstroke}{rgb}{0.000000,0.000000,0.000000}%
\pgfsetstrokecolor{currentstroke}%
\pgfsetdash{}{0pt}%
\pgfpathmoveto{\pgfqpoint{4.960000in}{3.023220in}}%
\pgfpathcurveto{\pgfqpoint{4.966572in}{3.023220in}}{\pgfqpoint{4.972875in}{3.025831in}}{\pgfqpoint{4.977522in}{3.030478in}}%
\pgfpathcurveto{\pgfqpoint{4.982169in}{3.035125in}}{\pgfqpoint{4.984780in}{3.041428in}}{\pgfqpoint{4.984780in}{3.048000in}}%
\pgfpathcurveto{\pgfqpoint{4.984780in}{3.054572in}}{\pgfqpoint{4.982169in}{3.060875in}}{\pgfqpoint{4.977522in}{3.065522in}}%
\pgfpathcurveto{\pgfqpoint{4.972875in}{3.070169in}}{\pgfqpoint{4.966572in}{3.072780in}}{\pgfqpoint{4.960000in}{3.072780in}}%
\pgfpathcurveto{\pgfqpoint{4.953428in}{3.072780in}}{\pgfqpoint{4.947125in}{3.070169in}}{\pgfqpoint{4.942478in}{3.065522in}}%
\pgfpathcurveto{\pgfqpoint{4.937831in}{3.060875in}}{\pgfqpoint{4.935220in}{3.054572in}}{\pgfqpoint{4.935220in}{3.048000in}}%
\pgfpathcurveto{\pgfqpoint{4.935220in}{3.041428in}}{\pgfqpoint{4.937831in}{3.035125in}}{\pgfqpoint{4.942478in}{3.030478in}}%
\pgfpathcurveto{\pgfqpoint{4.947125in}{3.025831in}}{\pgfqpoint{4.953428in}{3.023220in}}{\pgfqpoint{4.960000in}{3.023220in}}%
\pgfpathclose%
\pgfusepath{stroke,fill}%
\end{pgfscope}%
\begin{pgfscope}%
\pgfpathrectangle{\pgfqpoint{1.432000in}{0.528000in}}{\pgfqpoint{3.696000in}{3.696000in}}%
\pgfusepath{clip}%
\pgfsetbuttcap%
\pgfsetroundjoin%
\definecolor{currentfill}{rgb}{0.000000,0.000000,0.000000}%
\pgfsetfillcolor{currentfill}%
\pgfsetlinewidth{1.003750pt}%
\definecolor{currentstroke}{rgb}{0.000000,0.000000,0.000000}%
\pgfsetstrokecolor{currentstroke}%
\pgfsetdash{}{0pt}%
\pgfpathmoveto{\pgfqpoint{4.960000in}{3.359220in}}%
\pgfpathcurveto{\pgfqpoint{4.966572in}{3.359220in}}{\pgfqpoint{4.972875in}{3.361831in}}{\pgfqpoint{4.977522in}{3.366478in}}%
\pgfpathcurveto{\pgfqpoint{4.982169in}{3.371125in}}{\pgfqpoint{4.984780in}{3.377428in}}{\pgfqpoint{4.984780in}{3.384000in}}%
\pgfpathcurveto{\pgfqpoint{4.984780in}{3.390572in}}{\pgfqpoint{4.982169in}{3.396875in}}{\pgfqpoint{4.977522in}{3.401522in}}%
\pgfpathcurveto{\pgfqpoint{4.972875in}{3.406169in}}{\pgfqpoint{4.966572in}{3.408780in}}{\pgfqpoint{4.960000in}{3.408780in}}%
\pgfpathcurveto{\pgfqpoint{4.953428in}{3.408780in}}{\pgfqpoint{4.947125in}{3.406169in}}{\pgfqpoint{4.942478in}{3.401522in}}%
\pgfpathcurveto{\pgfqpoint{4.937831in}{3.396875in}}{\pgfqpoint{4.935220in}{3.390572in}}{\pgfqpoint{4.935220in}{3.384000in}}%
\pgfpathcurveto{\pgfqpoint{4.935220in}{3.377428in}}{\pgfqpoint{4.937831in}{3.371125in}}{\pgfqpoint{4.942478in}{3.366478in}}%
\pgfpathcurveto{\pgfqpoint{4.947125in}{3.361831in}}{\pgfqpoint{4.953428in}{3.359220in}}{\pgfqpoint{4.960000in}{3.359220in}}%
\pgfpathclose%
\pgfusepath{stroke,fill}%
\end{pgfscope}%
\begin{pgfscope}%
\pgfpathrectangle{\pgfqpoint{1.432000in}{0.528000in}}{\pgfqpoint{3.696000in}{3.696000in}}%
\pgfusepath{clip}%
\pgfsetbuttcap%
\pgfsetroundjoin%
\definecolor{currentfill}{rgb}{0.000000,0.000000,0.000000}%
\pgfsetfillcolor{currentfill}%
\pgfsetlinewidth{1.003750pt}%
\definecolor{currentstroke}{rgb}{0.000000,0.000000,0.000000}%
\pgfsetstrokecolor{currentstroke}%
\pgfsetdash{}{0pt}%
\pgfpathmoveto{\pgfqpoint{4.960000in}{3.695220in}}%
\pgfpathcurveto{\pgfqpoint{4.966572in}{3.695220in}}{\pgfqpoint{4.972875in}{3.697831in}}{\pgfqpoint{4.977522in}{3.702478in}}%
\pgfpathcurveto{\pgfqpoint{4.982169in}{3.707125in}}{\pgfqpoint{4.984780in}{3.713428in}}{\pgfqpoint{4.984780in}{3.720000in}}%
\pgfpathcurveto{\pgfqpoint{4.984780in}{3.726572in}}{\pgfqpoint{4.982169in}{3.732875in}}{\pgfqpoint{4.977522in}{3.737522in}}%
\pgfpathcurveto{\pgfqpoint{4.972875in}{3.742169in}}{\pgfqpoint{4.966572in}{3.744780in}}{\pgfqpoint{4.960000in}{3.744780in}}%
\pgfpathcurveto{\pgfqpoint{4.953428in}{3.744780in}}{\pgfqpoint{4.947125in}{3.742169in}}{\pgfqpoint{4.942478in}{3.737522in}}%
\pgfpathcurveto{\pgfqpoint{4.937831in}{3.732875in}}{\pgfqpoint{4.935220in}{3.726572in}}{\pgfqpoint{4.935220in}{3.720000in}}%
\pgfpathcurveto{\pgfqpoint{4.935220in}{3.713428in}}{\pgfqpoint{4.937831in}{3.707125in}}{\pgfqpoint{4.942478in}{3.702478in}}%
\pgfpathcurveto{\pgfqpoint{4.947125in}{3.697831in}}{\pgfqpoint{4.953428in}{3.695220in}}{\pgfqpoint{4.960000in}{3.695220in}}%
\pgfpathclose%
\pgfusepath{stroke,fill}%
\end{pgfscope}%
\begin{pgfscope}%
\pgfpathrectangle{\pgfqpoint{1.432000in}{0.528000in}}{\pgfqpoint{3.696000in}{3.696000in}}%
\pgfusepath{clip}%
\pgfsetbuttcap%
\pgfsetroundjoin%
\definecolor{currentfill}{rgb}{0.000000,0.000000,0.000000}%
\pgfsetfillcolor{currentfill}%
\pgfsetlinewidth{1.003750pt}%
\definecolor{currentstroke}{rgb}{0.000000,0.000000,0.000000}%
\pgfsetstrokecolor{currentstroke}%
\pgfsetdash{}{0pt}%
\pgfpathmoveto{\pgfqpoint{4.960000in}{4.031220in}}%
\pgfpathcurveto{\pgfqpoint{4.966572in}{4.031220in}}{\pgfqpoint{4.972875in}{4.033831in}}{\pgfqpoint{4.977522in}{4.038478in}}%
\pgfpathcurveto{\pgfqpoint{4.982169in}{4.043125in}}{\pgfqpoint{4.984780in}{4.049428in}}{\pgfqpoint{4.984780in}{4.056000in}}%
\pgfpathcurveto{\pgfqpoint{4.984780in}{4.062572in}}{\pgfqpoint{4.982169in}{4.068875in}}{\pgfqpoint{4.977522in}{4.073522in}}%
\pgfpathcurveto{\pgfqpoint{4.972875in}{4.078169in}}{\pgfqpoint{4.966572in}{4.080780in}}{\pgfqpoint{4.960000in}{4.080780in}}%
\pgfpathcurveto{\pgfqpoint{4.953428in}{4.080780in}}{\pgfqpoint{4.947125in}{4.078169in}}{\pgfqpoint{4.942478in}{4.073522in}}%
\pgfpathcurveto{\pgfqpoint{4.937831in}{4.068875in}}{\pgfqpoint{4.935220in}{4.062572in}}{\pgfqpoint{4.935220in}{4.056000in}}%
\pgfpathcurveto{\pgfqpoint{4.935220in}{4.049428in}}{\pgfqpoint{4.937831in}{4.043125in}}{\pgfqpoint{4.942478in}{4.038478in}}%
\pgfpathcurveto{\pgfqpoint{4.947125in}{4.033831in}}{\pgfqpoint{4.953428in}{4.031220in}}{\pgfqpoint{4.960000in}{4.031220in}}%
\pgfpathclose%
\pgfusepath{stroke,fill}%
\end{pgfscope}%
\begin{pgfscope}%
\pgfpathrectangle{\pgfqpoint{1.432000in}{0.528000in}}{\pgfqpoint{3.696000in}{3.696000in}}%
\pgfusepath{clip}%
\pgfsetbuttcap%
\pgfsetroundjoin%
\pgfsetlinewidth{1.505625pt}%
\definecolor{currentstroke}{rgb}{0.000000,0.000000,0.000000}%
\pgfsetstrokecolor{currentstroke}%
\pgfsetdash{}{0pt}%
\pgfpathmoveto{\pgfqpoint{1.600000in}{0.696000in}}%
\pgfpathlineto{\pgfqpoint{1.600000in}{1.032000in}}%
\pgfpathlineto{\pgfqpoint{1.936000in}{0.696000in}}%
\pgfpathlineto{\pgfqpoint{1.600000in}{0.696000in}}%
\pgfusepath{stroke}%
\end{pgfscope}%
\begin{pgfscope}%
\pgfpathrectangle{\pgfqpoint{1.432000in}{0.528000in}}{\pgfqpoint{3.696000in}{3.696000in}}%
\pgfusepath{clip}%
\pgfsetbuttcap%
\pgfsetroundjoin%
\pgfsetlinewidth{1.505625pt}%
\definecolor{currentstroke}{rgb}{0.000000,0.000000,0.000000}%
\pgfsetstrokecolor{currentstroke}%
\pgfsetdash{}{0pt}%
\pgfpathmoveto{\pgfqpoint{1.600000in}{1.032000in}}%
\pgfpathlineto{\pgfqpoint{1.936000in}{0.696000in}}%
\pgfpathlineto{\pgfqpoint{1.936000in}{1.032000in}}%
\pgfpathlineto{\pgfqpoint{1.600000in}{1.032000in}}%
\pgfusepath{stroke}%
\end{pgfscope}%
\begin{pgfscope}%
\pgfpathrectangle{\pgfqpoint{1.432000in}{0.528000in}}{\pgfqpoint{3.696000in}{3.696000in}}%
\pgfusepath{clip}%
\pgfsetbuttcap%
\pgfsetroundjoin%
\pgfsetlinewidth{1.505625pt}%
\definecolor{currentstroke}{rgb}{0.000000,0.000000,0.000000}%
\pgfsetstrokecolor{currentstroke}%
\pgfsetdash{}{0pt}%
\pgfpathmoveto{\pgfqpoint{1.936000in}{0.696000in}}%
\pgfpathlineto{\pgfqpoint{1.936000in}{1.032000in}}%
\pgfpathlineto{\pgfqpoint{2.272000in}{0.696000in}}%
\pgfpathlineto{\pgfqpoint{1.936000in}{0.696000in}}%
\pgfusepath{stroke}%
\end{pgfscope}%
\begin{pgfscope}%
\pgfpathrectangle{\pgfqpoint{1.432000in}{0.528000in}}{\pgfqpoint{3.696000in}{3.696000in}}%
\pgfusepath{clip}%
\pgfsetbuttcap%
\pgfsetroundjoin%
\pgfsetlinewidth{1.505625pt}%
\definecolor{currentstroke}{rgb}{0.000000,0.000000,0.000000}%
\pgfsetstrokecolor{currentstroke}%
\pgfsetdash{}{0pt}%
\pgfpathmoveto{\pgfqpoint{1.600000in}{1.032000in}}%
\pgfpathlineto{\pgfqpoint{1.936000in}{1.032000in}}%
\pgfpathlineto{\pgfqpoint{1.600000in}{1.368000in}}%
\pgfpathlineto{\pgfqpoint{1.600000in}{1.032000in}}%
\pgfusepath{stroke}%
\end{pgfscope}%
\begin{pgfscope}%
\pgfpathrectangle{\pgfqpoint{1.432000in}{0.528000in}}{\pgfqpoint{3.696000in}{3.696000in}}%
\pgfusepath{clip}%
\pgfsetbuttcap%
\pgfsetroundjoin%
\pgfsetlinewidth{1.505625pt}%
\definecolor{currentstroke}{rgb}{0.000000,0.000000,0.000000}%
\pgfsetstrokecolor{currentstroke}%
\pgfsetdash{}{0pt}%
\pgfpathmoveto{\pgfqpoint{1.936000in}{1.032000in}}%
\pgfpathlineto{\pgfqpoint{2.272000in}{0.696000in}}%
\pgfpathlineto{\pgfqpoint{2.272000in}{1.032000in}}%
\pgfpathlineto{\pgfqpoint{1.936000in}{1.032000in}}%
\pgfusepath{stroke}%
\end{pgfscope}%
\begin{pgfscope}%
\pgfpathrectangle{\pgfqpoint{1.432000in}{0.528000in}}{\pgfqpoint{3.696000in}{3.696000in}}%
\pgfusepath{clip}%
\pgfsetbuttcap%
\pgfsetroundjoin%
\pgfsetlinewidth{1.505625pt}%
\definecolor{currentstroke}{rgb}{0.000000,0.000000,0.000000}%
\pgfsetstrokecolor{currentstroke}%
\pgfsetdash{}{0pt}%
\pgfpathmoveto{\pgfqpoint{1.936000in}{1.032000in}}%
\pgfpathlineto{\pgfqpoint{1.600000in}{1.368000in}}%
\pgfpathlineto{\pgfqpoint{1.936000in}{1.368000in}}%
\pgfpathlineto{\pgfqpoint{1.936000in}{1.032000in}}%
\pgfusepath{stroke}%
\end{pgfscope}%
\begin{pgfscope}%
\pgfpathrectangle{\pgfqpoint{1.432000in}{0.528000in}}{\pgfqpoint{3.696000in}{3.696000in}}%
\pgfusepath{clip}%
\pgfsetbuttcap%
\pgfsetroundjoin%
\pgfsetlinewidth{1.505625pt}%
\definecolor{currentstroke}{rgb}{0.000000,0.000000,0.000000}%
\pgfsetstrokecolor{currentstroke}%
\pgfsetdash{}{0pt}%
\pgfpathmoveto{\pgfqpoint{2.272000in}{0.696000in}}%
\pgfpathlineto{\pgfqpoint{2.272000in}{1.032000in}}%
\pgfpathlineto{\pgfqpoint{2.608000in}{0.696000in}}%
\pgfpathlineto{\pgfqpoint{2.272000in}{0.696000in}}%
\pgfusepath{stroke}%
\end{pgfscope}%
\begin{pgfscope}%
\pgfpathrectangle{\pgfqpoint{1.432000in}{0.528000in}}{\pgfqpoint{3.696000in}{3.696000in}}%
\pgfusepath{clip}%
\pgfsetbuttcap%
\pgfsetroundjoin%
\pgfsetlinewidth{1.505625pt}%
\definecolor{currentstroke}{rgb}{0.000000,0.000000,0.000000}%
\pgfsetstrokecolor{currentstroke}%
\pgfsetdash{}{0pt}%
\pgfpathmoveto{\pgfqpoint{1.936000in}{1.032000in}}%
\pgfpathlineto{\pgfqpoint{2.272000in}{1.032000in}}%
\pgfpathlineto{\pgfqpoint{1.936000in}{1.368000in}}%
\pgfpathlineto{\pgfqpoint{1.936000in}{1.032000in}}%
\pgfusepath{stroke}%
\end{pgfscope}%
\begin{pgfscope}%
\pgfpathrectangle{\pgfqpoint{1.432000in}{0.528000in}}{\pgfqpoint{3.696000in}{3.696000in}}%
\pgfusepath{clip}%
\pgfsetbuttcap%
\pgfsetroundjoin%
\pgfsetlinewidth{1.505625pt}%
\definecolor{currentstroke}{rgb}{0.000000,0.000000,0.000000}%
\pgfsetstrokecolor{currentstroke}%
\pgfsetdash{}{0pt}%
\pgfpathmoveto{\pgfqpoint{1.600000in}{1.368000in}}%
\pgfpathlineto{\pgfqpoint{1.936000in}{1.368000in}}%
\pgfpathlineto{\pgfqpoint{1.600000in}{1.704000in}}%
\pgfpathlineto{\pgfqpoint{1.600000in}{1.368000in}}%
\pgfusepath{stroke}%
\end{pgfscope}%
\begin{pgfscope}%
\pgfpathrectangle{\pgfqpoint{1.432000in}{0.528000in}}{\pgfqpoint{3.696000in}{3.696000in}}%
\pgfusepath{clip}%
\pgfsetbuttcap%
\pgfsetroundjoin%
\pgfsetlinewidth{1.505625pt}%
\definecolor{currentstroke}{rgb}{0.000000,0.000000,0.000000}%
\pgfsetstrokecolor{currentstroke}%
\pgfsetdash{}{0pt}%
\pgfpathmoveto{\pgfqpoint{2.272000in}{1.032000in}}%
\pgfpathlineto{\pgfqpoint{2.608000in}{0.696000in}}%
\pgfpathlineto{\pgfqpoint{2.608000in}{1.032000in}}%
\pgfpathlineto{\pgfqpoint{2.272000in}{1.032000in}}%
\pgfusepath{stroke}%
\end{pgfscope}%
\begin{pgfscope}%
\pgfpathrectangle{\pgfqpoint{1.432000in}{0.528000in}}{\pgfqpoint{3.696000in}{3.696000in}}%
\pgfusepath{clip}%
\pgfsetbuttcap%
\pgfsetroundjoin%
\pgfsetlinewidth{1.505625pt}%
\definecolor{currentstroke}{rgb}{0.000000,0.000000,0.000000}%
\pgfsetstrokecolor{currentstroke}%
\pgfsetdash{}{0pt}%
\pgfpathmoveto{\pgfqpoint{2.272000in}{1.032000in}}%
\pgfpathlineto{\pgfqpoint{1.936000in}{1.368000in}}%
\pgfpathlineto{\pgfqpoint{2.272000in}{1.368000in}}%
\pgfpathlineto{\pgfqpoint{2.272000in}{1.032000in}}%
\pgfusepath{stroke}%
\end{pgfscope}%
\begin{pgfscope}%
\pgfpathrectangle{\pgfqpoint{1.432000in}{0.528000in}}{\pgfqpoint{3.696000in}{3.696000in}}%
\pgfusepath{clip}%
\pgfsetbuttcap%
\pgfsetroundjoin%
\pgfsetlinewidth{1.505625pt}%
\definecolor{currentstroke}{rgb}{0.000000,0.000000,0.000000}%
\pgfsetstrokecolor{currentstroke}%
\pgfsetdash{}{0pt}%
\pgfpathmoveto{\pgfqpoint{1.936000in}{1.368000in}}%
\pgfpathlineto{\pgfqpoint{1.600000in}{1.704000in}}%
\pgfpathlineto{\pgfqpoint{1.936000in}{1.704000in}}%
\pgfpathlineto{\pgfqpoint{1.936000in}{1.368000in}}%
\pgfusepath{stroke}%
\end{pgfscope}%
\begin{pgfscope}%
\pgfpathrectangle{\pgfqpoint{1.432000in}{0.528000in}}{\pgfqpoint{3.696000in}{3.696000in}}%
\pgfusepath{clip}%
\pgfsetbuttcap%
\pgfsetroundjoin%
\pgfsetlinewidth{1.505625pt}%
\definecolor{currentstroke}{rgb}{0.000000,0.000000,0.000000}%
\pgfsetstrokecolor{currentstroke}%
\pgfsetdash{}{0pt}%
\pgfpathmoveto{\pgfqpoint{2.608000in}{0.696000in}}%
\pgfpathlineto{\pgfqpoint{2.608000in}{1.032000in}}%
\pgfpathlineto{\pgfqpoint{2.944000in}{0.696000in}}%
\pgfpathlineto{\pgfqpoint{2.608000in}{0.696000in}}%
\pgfusepath{stroke}%
\end{pgfscope}%
\begin{pgfscope}%
\pgfpathrectangle{\pgfqpoint{1.432000in}{0.528000in}}{\pgfqpoint{3.696000in}{3.696000in}}%
\pgfusepath{clip}%
\pgfsetbuttcap%
\pgfsetroundjoin%
\pgfsetlinewidth{1.505625pt}%
\definecolor{currentstroke}{rgb}{0.000000,0.000000,0.000000}%
\pgfsetstrokecolor{currentstroke}%
\pgfsetdash{}{0pt}%
\pgfpathmoveto{\pgfqpoint{2.272000in}{1.032000in}}%
\pgfpathlineto{\pgfqpoint{2.608000in}{1.032000in}}%
\pgfpathlineto{\pgfqpoint{2.272000in}{1.368000in}}%
\pgfpathlineto{\pgfqpoint{2.272000in}{1.032000in}}%
\pgfusepath{stroke}%
\end{pgfscope}%
\begin{pgfscope}%
\pgfpathrectangle{\pgfqpoint{1.432000in}{0.528000in}}{\pgfqpoint{3.696000in}{3.696000in}}%
\pgfusepath{clip}%
\pgfsetbuttcap%
\pgfsetroundjoin%
\pgfsetlinewidth{1.505625pt}%
\definecolor{currentstroke}{rgb}{0.000000,0.000000,0.000000}%
\pgfsetstrokecolor{currentstroke}%
\pgfsetdash{}{0pt}%
\pgfpathmoveto{\pgfqpoint{1.936000in}{1.368000in}}%
\pgfpathlineto{\pgfqpoint{2.272000in}{1.368000in}}%
\pgfpathlineto{\pgfqpoint{1.936000in}{1.704000in}}%
\pgfpathlineto{\pgfqpoint{1.936000in}{1.368000in}}%
\pgfusepath{stroke}%
\end{pgfscope}%
\begin{pgfscope}%
\pgfpathrectangle{\pgfqpoint{1.432000in}{0.528000in}}{\pgfqpoint{3.696000in}{3.696000in}}%
\pgfusepath{clip}%
\pgfsetbuttcap%
\pgfsetroundjoin%
\pgfsetlinewidth{1.505625pt}%
\definecolor{currentstroke}{rgb}{0.000000,0.000000,0.000000}%
\pgfsetstrokecolor{currentstroke}%
\pgfsetdash{}{0pt}%
\pgfpathmoveto{\pgfqpoint{1.600000in}{1.704000in}}%
\pgfpathlineto{\pgfqpoint{1.936000in}{1.704000in}}%
\pgfpathlineto{\pgfqpoint{1.600000in}{2.040000in}}%
\pgfpathlineto{\pgfqpoint{1.600000in}{1.704000in}}%
\pgfusepath{stroke}%
\end{pgfscope}%
\begin{pgfscope}%
\pgfpathrectangle{\pgfqpoint{1.432000in}{0.528000in}}{\pgfqpoint{3.696000in}{3.696000in}}%
\pgfusepath{clip}%
\pgfsetbuttcap%
\pgfsetroundjoin%
\pgfsetlinewidth{1.505625pt}%
\definecolor{currentstroke}{rgb}{0.000000,0.000000,0.000000}%
\pgfsetstrokecolor{currentstroke}%
\pgfsetdash{}{0pt}%
\pgfpathmoveto{\pgfqpoint{2.608000in}{1.032000in}}%
\pgfpathlineto{\pgfqpoint{2.944000in}{0.696000in}}%
\pgfpathlineto{\pgfqpoint{2.944000in}{1.032000in}}%
\pgfpathlineto{\pgfqpoint{2.608000in}{1.032000in}}%
\pgfusepath{stroke}%
\end{pgfscope}%
\begin{pgfscope}%
\pgfpathrectangle{\pgfqpoint{1.432000in}{0.528000in}}{\pgfqpoint{3.696000in}{3.696000in}}%
\pgfusepath{clip}%
\pgfsetbuttcap%
\pgfsetroundjoin%
\pgfsetlinewidth{1.505625pt}%
\definecolor{currentstroke}{rgb}{0.000000,0.000000,0.000000}%
\pgfsetstrokecolor{currentstroke}%
\pgfsetdash{}{0pt}%
\pgfpathmoveto{\pgfqpoint{2.608000in}{1.032000in}}%
\pgfpathlineto{\pgfqpoint{2.272000in}{1.368000in}}%
\pgfpathlineto{\pgfqpoint{2.608000in}{1.368000in}}%
\pgfpathlineto{\pgfqpoint{2.608000in}{1.032000in}}%
\pgfusepath{stroke}%
\end{pgfscope}%
\begin{pgfscope}%
\pgfpathrectangle{\pgfqpoint{1.432000in}{0.528000in}}{\pgfqpoint{3.696000in}{3.696000in}}%
\pgfusepath{clip}%
\pgfsetbuttcap%
\pgfsetroundjoin%
\pgfsetlinewidth{1.505625pt}%
\definecolor{currentstroke}{rgb}{0.000000,0.000000,0.000000}%
\pgfsetstrokecolor{currentstroke}%
\pgfsetdash{}{0pt}%
\pgfpathmoveto{\pgfqpoint{2.272000in}{1.368000in}}%
\pgfpathlineto{\pgfqpoint{1.936000in}{1.704000in}}%
\pgfpathlineto{\pgfqpoint{2.272000in}{1.704000in}}%
\pgfpathlineto{\pgfqpoint{2.272000in}{1.368000in}}%
\pgfusepath{stroke}%
\end{pgfscope}%
\begin{pgfscope}%
\pgfpathrectangle{\pgfqpoint{1.432000in}{0.528000in}}{\pgfqpoint{3.696000in}{3.696000in}}%
\pgfusepath{clip}%
\pgfsetbuttcap%
\pgfsetroundjoin%
\pgfsetlinewidth{1.505625pt}%
\definecolor{currentstroke}{rgb}{0.000000,0.000000,0.000000}%
\pgfsetstrokecolor{currentstroke}%
\pgfsetdash{}{0pt}%
\pgfpathmoveto{\pgfqpoint{1.936000in}{1.704000in}}%
\pgfpathlineto{\pgfqpoint{1.600000in}{2.040000in}}%
\pgfpathlineto{\pgfqpoint{1.936000in}{2.040000in}}%
\pgfpathlineto{\pgfqpoint{1.936000in}{1.704000in}}%
\pgfusepath{stroke}%
\end{pgfscope}%
\begin{pgfscope}%
\pgfpathrectangle{\pgfqpoint{1.432000in}{0.528000in}}{\pgfqpoint{3.696000in}{3.696000in}}%
\pgfusepath{clip}%
\pgfsetbuttcap%
\pgfsetroundjoin%
\pgfsetlinewidth{1.505625pt}%
\definecolor{currentstroke}{rgb}{0.000000,0.000000,0.000000}%
\pgfsetstrokecolor{currentstroke}%
\pgfsetdash{}{0pt}%
\pgfpathmoveto{\pgfqpoint{2.944000in}{0.696000in}}%
\pgfpathlineto{\pgfqpoint{2.944000in}{1.032000in}}%
\pgfpathlineto{\pgfqpoint{3.280000in}{0.696000in}}%
\pgfpathlineto{\pgfqpoint{2.944000in}{0.696000in}}%
\pgfusepath{stroke}%
\end{pgfscope}%
\begin{pgfscope}%
\pgfpathrectangle{\pgfqpoint{1.432000in}{0.528000in}}{\pgfqpoint{3.696000in}{3.696000in}}%
\pgfusepath{clip}%
\pgfsetbuttcap%
\pgfsetroundjoin%
\pgfsetlinewidth{1.505625pt}%
\definecolor{currentstroke}{rgb}{0.000000,0.000000,0.000000}%
\pgfsetstrokecolor{currentstroke}%
\pgfsetdash{}{0pt}%
\pgfpathmoveto{\pgfqpoint{2.608000in}{1.032000in}}%
\pgfpathlineto{\pgfqpoint{2.944000in}{1.032000in}}%
\pgfpathlineto{\pgfqpoint{2.608000in}{1.368000in}}%
\pgfpathlineto{\pgfqpoint{2.608000in}{1.032000in}}%
\pgfusepath{stroke}%
\end{pgfscope}%
\begin{pgfscope}%
\pgfpathrectangle{\pgfqpoint{1.432000in}{0.528000in}}{\pgfqpoint{3.696000in}{3.696000in}}%
\pgfusepath{clip}%
\pgfsetbuttcap%
\pgfsetroundjoin%
\pgfsetlinewidth{1.505625pt}%
\definecolor{currentstroke}{rgb}{0.000000,0.000000,0.000000}%
\pgfsetstrokecolor{currentstroke}%
\pgfsetdash{}{0pt}%
\pgfpathmoveto{\pgfqpoint{2.272000in}{1.368000in}}%
\pgfpathlineto{\pgfqpoint{2.608000in}{1.368000in}}%
\pgfpathlineto{\pgfqpoint{2.272000in}{1.704000in}}%
\pgfpathlineto{\pgfqpoint{2.272000in}{1.368000in}}%
\pgfusepath{stroke}%
\end{pgfscope}%
\begin{pgfscope}%
\pgfpathrectangle{\pgfqpoint{1.432000in}{0.528000in}}{\pgfqpoint{3.696000in}{3.696000in}}%
\pgfusepath{clip}%
\pgfsetbuttcap%
\pgfsetroundjoin%
\pgfsetlinewidth{1.505625pt}%
\definecolor{currentstroke}{rgb}{0.000000,0.000000,0.000000}%
\pgfsetstrokecolor{currentstroke}%
\pgfsetdash{}{0pt}%
\pgfpathmoveto{\pgfqpoint{1.936000in}{1.704000in}}%
\pgfpathlineto{\pgfqpoint{2.272000in}{1.704000in}}%
\pgfpathlineto{\pgfqpoint{1.936000in}{2.040000in}}%
\pgfpathlineto{\pgfqpoint{1.936000in}{1.704000in}}%
\pgfusepath{stroke}%
\end{pgfscope}%
\begin{pgfscope}%
\pgfpathrectangle{\pgfqpoint{1.432000in}{0.528000in}}{\pgfqpoint{3.696000in}{3.696000in}}%
\pgfusepath{clip}%
\pgfsetbuttcap%
\pgfsetroundjoin%
\pgfsetlinewidth{1.505625pt}%
\definecolor{currentstroke}{rgb}{0.000000,0.000000,0.000000}%
\pgfsetstrokecolor{currentstroke}%
\pgfsetdash{}{0pt}%
\pgfpathmoveto{\pgfqpoint{1.600000in}{2.040000in}}%
\pgfpathlineto{\pgfqpoint{1.936000in}{2.040000in}}%
\pgfpathlineto{\pgfqpoint{1.600000in}{2.376000in}}%
\pgfpathlineto{\pgfqpoint{1.600000in}{2.040000in}}%
\pgfusepath{stroke}%
\end{pgfscope}%
\begin{pgfscope}%
\pgfpathrectangle{\pgfqpoint{1.432000in}{0.528000in}}{\pgfqpoint{3.696000in}{3.696000in}}%
\pgfusepath{clip}%
\pgfsetbuttcap%
\pgfsetroundjoin%
\pgfsetlinewidth{1.505625pt}%
\definecolor{currentstroke}{rgb}{0.000000,0.000000,0.000000}%
\pgfsetstrokecolor{currentstroke}%
\pgfsetdash{}{0pt}%
\pgfpathmoveto{\pgfqpoint{2.944000in}{1.032000in}}%
\pgfpathlineto{\pgfqpoint{3.280000in}{0.696000in}}%
\pgfpathlineto{\pgfqpoint{3.280000in}{1.032000in}}%
\pgfpathlineto{\pgfqpoint{2.944000in}{1.032000in}}%
\pgfusepath{stroke}%
\end{pgfscope}%
\begin{pgfscope}%
\pgfpathrectangle{\pgfqpoint{1.432000in}{0.528000in}}{\pgfqpoint{3.696000in}{3.696000in}}%
\pgfusepath{clip}%
\pgfsetbuttcap%
\pgfsetroundjoin%
\pgfsetlinewidth{1.505625pt}%
\definecolor{currentstroke}{rgb}{0.000000,0.000000,0.000000}%
\pgfsetstrokecolor{currentstroke}%
\pgfsetdash{}{0pt}%
\pgfpathmoveto{\pgfqpoint{2.944000in}{1.032000in}}%
\pgfpathlineto{\pgfqpoint{2.608000in}{1.368000in}}%
\pgfpathlineto{\pgfqpoint{2.944000in}{1.368000in}}%
\pgfpathlineto{\pgfqpoint{2.944000in}{1.032000in}}%
\pgfusepath{stroke}%
\end{pgfscope}%
\begin{pgfscope}%
\pgfpathrectangle{\pgfqpoint{1.432000in}{0.528000in}}{\pgfqpoint{3.696000in}{3.696000in}}%
\pgfusepath{clip}%
\pgfsetbuttcap%
\pgfsetroundjoin%
\pgfsetlinewidth{1.505625pt}%
\definecolor{currentstroke}{rgb}{0.000000,0.000000,0.000000}%
\pgfsetstrokecolor{currentstroke}%
\pgfsetdash{}{0pt}%
\pgfpathmoveto{\pgfqpoint{2.608000in}{1.368000in}}%
\pgfpathlineto{\pgfqpoint{2.272000in}{1.704000in}}%
\pgfpathlineto{\pgfqpoint{2.608000in}{1.704000in}}%
\pgfpathlineto{\pgfqpoint{2.608000in}{1.368000in}}%
\pgfusepath{stroke}%
\end{pgfscope}%
\begin{pgfscope}%
\pgfpathrectangle{\pgfqpoint{1.432000in}{0.528000in}}{\pgfqpoint{3.696000in}{3.696000in}}%
\pgfusepath{clip}%
\pgfsetbuttcap%
\pgfsetroundjoin%
\pgfsetlinewidth{1.505625pt}%
\definecolor{currentstroke}{rgb}{0.000000,0.000000,0.000000}%
\pgfsetstrokecolor{currentstroke}%
\pgfsetdash{}{0pt}%
\pgfpathmoveto{\pgfqpoint{2.272000in}{1.704000in}}%
\pgfpathlineto{\pgfqpoint{1.936000in}{2.040000in}}%
\pgfpathlineto{\pgfqpoint{2.272000in}{2.040000in}}%
\pgfpathlineto{\pgfqpoint{2.272000in}{1.704000in}}%
\pgfusepath{stroke}%
\end{pgfscope}%
\begin{pgfscope}%
\pgfpathrectangle{\pgfqpoint{1.432000in}{0.528000in}}{\pgfqpoint{3.696000in}{3.696000in}}%
\pgfusepath{clip}%
\pgfsetbuttcap%
\pgfsetroundjoin%
\pgfsetlinewidth{1.505625pt}%
\definecolor{currentstroke}{rgb}{0.000000,0.000000,0.000000}%
\pgfsetstrokecolor{currentstroke}%
\pgfsetdash{}{0pt}%
\pgfpathmoveto{\pgfqpoint{1.936000in}{2.040000in}}%
\pgfpathlineto{\pgfqpoint{1.600000in}{2.376000in}}%
\pgfpathlineto{\pgfqpoint{1.936000in}{2.376000in}}%
\pgfpathlineto{\pgfqpoint{1.936000in}{2.040000in}}%
\pgfusepath{stroke}%
\end{pgfscope}%
\begin{pgfscope}%
\pgfpathrectangle{\pgfqpoint{1.432000in}{0.528000in}}{\pgfqpoint{3.696000in}{3.696000in}}%
\pgfusepath{clip}%
\pgfsetbuttcap%
\pgfsetroundjoin%
\pgfsetlinewidth{1.505625pt}%
\definecolor{currentstroke}{rgb}{0.000000,0.000000,0.000000}%
\pgfsetstrokecolor{currentstroke}%
\pgfsetdash{}{0pt}%
\pgfpathmoveto{\pgfqpoint{3.280000in}{0.696000in}}%
\pgfpathlineto{\pgfqpoint{3.280000in}{1.032000in}}%
\pgfpathlineto{\pgfqpoint{3.616000in}{0.696000in}}%
\pgfpathlineto{\pgfqpoint{3.280000in}{0.696000in}}%
\pgfusepath{stroke}%
\end{pgfscope}%
\begin{pgfscope}%
\pgfpathrectangle{\pgfqpoint{1.432000in}{0.528000in}}{\pgfqpoint{3.696000in}{3.696000in}}%
\pgfusepath{clip}%
\pgfsetbuttcap%
\pgfsetroundjoin%
\pgfsetlinewidth{1.505625pt}%
\definecolor{currentstroke}{rgb}{0.000000,0.000000,0.000000}%
\pgfsetstrokecolor{currentstroke}%
\pgfsetdash{}{0pt}%
\pgfpathmoveto{\pgfqpoint{2.944000in}{1.032000in}}%
\pgfpathlineto{\pgfqpoint{3.280000in}{1.032000in}}%
\pgfpathlineto{\pgfqpoint{2.944000in}{1.368000in}}%
\pgfpathlineto{\pgfqpoint{2.944000in}{1.032000in}}%
\pgfusepath{stroke}%
\end{pgfscope}%
\begin{pgfscope}%
\pgfpathrectangle{\pgfqpoint{1.432000in}{0.528000in}}{\pgfqpoint{3.696000in}{3.696000in}}%
\pgfusepath{clip}%
\pgfsetbuttcap%
\pgfsetroundjoin%
\pgfsetlinewidth{1.505625pt}%
\definecolor{currentstroke}{rgb}{0.000000,0.000000,0.000000}%
\pgfsetstrokecolor{currentstroke}%
\pgfsetdash{}{0pt}%
\pgfpathmoveto{\pgfqpoint{2.608000in}{1.368000in}}%
\pgfpathlineto{\pgfqpoint{2.944000in}{1.368000in}}%
\pgfpathlineto{\pgfqpoint{2.608000in}{1.704000in}}%
\pgfpathlineto{\pgfqpoint{2.608000in}{1.368000in}}%
\pgfusepath{stroke}%
\end{pgfscope}%
\begin{pgfscope}%
\pgfpathrectangle{\pgfqpoint{1.432000in}{0.528000in}}{\pgfqpoint{3.696000in}{3.696000in}}%
\pgfusepath{clip}%
\pgfsetbuttcap%
\pgfsetroundjoin%
\pgfsetlinewidth{1.505625pt}%
\definecolor{currentstroke}{rgb}{0.000000,0.000000,0.000000}%
\pgfsetstrokecolor{currentstroke}%
\pgfsetdash{}{0pt}%
\pgfpathmoveto{\pgfqpoint{2.272000in}{1.704000in}}%
\pgfpathlineto{\pgfqpoint{2.608000in}{1.704000in}}%
\pgfpathlineto{\pgfqpoint{2.272000in}{2.040000in}}%
\pgfpathlineto{\pgfqpoint{2.272000in}{1.704000in}}%
\pgfusepath{stroke}%
\end{pgfscope}%
\begin{pgfscope}%
\pgfpathrectangle{\pgfqpoint{1.432000in}{0.528000in}}{\pgfqpoint{3.696000in}{3.696000in}}%
\pgfusepath{clip}%
\pgfsetbuttcap%
\pgfsetroundjoin%
\pgfsetlinewidth{1.505625pt}%
\definecolor{currentstroke}{rgb}{0.000000,0.000000,0.000000}%
\pgfsetstrokecolor{currentstroke}%
\pgfsetdash{}{0pt}%
\pgfpathmoveto{\pgfqpoint{1.936000in}{2.040000in}}%
\pgfpathlineto{\pgfqpoint{2.272000in}{2.040000in}}%
\pgfpathlineto{\pgfqpoint{1.936000in}{2.376000in}}%
\pgfpathlineto{\pgfqpoint{1.936000in}{2.040000in}}%
\pgfusepath{stroke}%
\end{pgfscope}%
\begin{pgfscope}%
\pgfpathrectangle{\pgfqpoint{1.432000in}{0.528000in}}{\pgfqpoint{3.696000in}{3.696000in}}%
\pgfusepath{clip}%
\pgfsetbuttcap%
\pgfsetroundjoin%
\pgfsetlinewidth{1.505625pt}%
\definecolor{currentstroke}{rgb}{0.000000,0.000000,0.000000}%
\pgfsetstrokecolor{currentstroke}%
\pgfsetdash{}{0pt}%
\pgfpathmoveto{\pgfqpoint{1.600000in}{2.376000in}}%
\pgfpathlineto{\pgfqpoint{1.936000in}{2.376000in}}%
\pgfpathlineto{\pgfqpoint{1.600000in}{2.712000in}}%
\pgfpathlineto{\pgfqpoint{1.600000in}{2.376000in}}%
\pgfusepath{stroke}%
\end{pgfscope}%
\begin{pgfscope}%
\pgfpathrectangle{\pgfqpoint{1.432000in}{0.528000in}}{\pgfqpoint{3.696000in}{3.696000in}}%
\pgfusepath{clip}%
\pgfsetbuttcap%
\pgfsetroundjoin%
\pgfsetlinewidth{1.505625pt}%
\definecolor{currentstroke}{rgb}{0.000000,0.000000,0.000000}%
\pgfsetstrokecolor{currentstroke}%
\pgfsetdash{}{0pt}%
\pgfpathmoveto{\pgfqpoint{3.280000in}{1.032000in}}%
\pgfpathlineto{\pgfqpoint{3.616000in}{0.696000in}}%
\pgfpathlineto{\pgfqpoint{3.616000in}{1.032000in}}%
\pgfpathlineto{\pgfqpoint{3.280000in}{1.032000in}}%
\pgfusepath{stroke}%
\end{pgfscope}%
\begin{pgfscope}%
\pgfpathrectangle{\pgfqpoint{1.432000in}{0.528000in}}{\pgfqpoint{3.696000in}{3.696000in}}%
\pgfusepath{clip}%
\pgfsetbuttcap%
\pgfsetroundjoin%
\pgfsetlinewidth{1.505625pt}%
\definecolor{currentstroke}{rgb}{0.000000,0.000000,0.000000}%
\pgfsetstrokecolor{currentstroke}%
\pgfsetdash{}{0pt}%
\pgfpathmoveto{\pgfqpoint{3.280000in}{1.032000in}}%
\pgfpathlineto{\pgfqpoint{2.944000in}{1.368000in}}%
\pgfpathlineto{\pgfqpoint{3.280000in}{1.368000in}}%
\pgfpathlineto{\pgfqpoint{3.280000in}{1.032000in}}%
\pgfusepath{stroke}%
\end{pgfscope}%
\begin{pgfscope}%
\pgfpathrectangle{\pgfqpoint{1.432000in}{0.528000in}}{\pgfqpoint{3.696000in}{3.696000in}}%
\pgfusepath{clip}%
\pgfsetbuttcap%
\pgfsetroundjoin%
\pgfsetlinewidth{1.505625pt}%
\definecolor{currentstroke}{rgb}{0.000000,0.000000,0.000000}%
\pgfsetstrokecolor{currentstroke}%
\pgfsetdash{}{0pt}%
\pgfpathmoveto{\pgfqpoint{2.944000in}{1.368000in}}%
\pgfpathlineto{\pgfqpoint{2.608000in}{1.704000in}}%
\pgfpathlineto{\pgfqpoint{2.944000in}{1.704000in}}%
\pgfpathlineto{\pgfqpoint{2.944000in}{1.368000in}}%
\pgfusepath{stroke}%
\end{pgfscope}%
\begin{pgfscope}%
\pgfpathrectangle{\pgfqpoint{1.432000in}{0.528000in}}{\pgfqpoint{3.696000in}{3.696000in}}%
\pgfusepath{clip}%
\pgfsetbuttcap%
\pgfsetroundjoin%
\pgfsetlinewidth{1.505625pt}%
\definecolor{currentstroke}{rgb}{0.000000,0.000000,0.000000}%
\pgfsetstrokecolor{currentstroke}%
\pgfsetdash{}{0pt}%
\pgfpathmoveto{\pgfqpoint{2.608000in}{1.704000in}}%
\pgfpathlineto{\pgfqpoint{2.272000in}{2.040000in}}%
\pgfpathlineto{\pgfqpoint{2.608000in}{2.040000in}}%
\pgfpathlineto{\pgfqpoint{2.608000in}{1.704000in}}%
\pgfusepath{stroke}%
\end{pgfscope}%
\begin{pgfscope}%
\pgfpathrectangle{\pgfqpoint{1.432000in}{0.528000in}}{\pgfqpoint{3.696000in}{3.696000in}}%
\pgfusepath{clip}%
\pgfsetbuttcap%
\pgfsetroundjoin%
\pgfsetlinewidth{1.505625pt}%
\definecolor{currentstroke}{rgb}{0.000000,0.000000,0.000000}%
\pgfsetstrokecolor{currentstroke}%
\pgfsetdash{}{0pt}%
\pgfpathmoveto{\pgfqpoint{2.272000in}{2.040000in}}%
\pgfpathlineto{\pgfqpoint{1.936000in}{2.376000in}}%
\pgfpathlineto{\pgfqpoint{2.272000in}{2.376000in}}%
\pgfpathlineto{\pgfqpoint{2.272000in}{2.040000in}}%
\pgfusepath{stroke}%
\end{pgfscope}%
\begin{pgfscope}%
\pgfpathrectangle{\pgfqpoint{1.432000in}{0.528000in}}{\pgfqpoint{3.696000in}{3.696000in}}%
\pgfusepath{clip}%
\pgfsetbuttcap%
\pgfsetroundjoin%
\pgfsetlinewidth{1.505625pt}%
\definecolor{currentstroke}{rgb}{0.000000,0.000000,0.000000}%
\pgfsetstrokecolor{currentstroke}%
\pgfsetdash{}{0pt}%
\pgfpathmoveto{\pgfqpoint{1.936000in}{2.376000in}}%
\pgfpathlineto{\pgfqpoint{1.600000in}{2.712000in}}%
\pgfpathlineto{\pgfqpoint{1.936000in}{2.712000in}}%
\pgfpathlineto{\pgfqpoint{1.936000in}{2.376000in}}%
\pgfusepath{stroke}%
\end{pgfscope}%
\begin{pgfscope}%
\pgfpathrectangle{\pgfqpoint{1.432000in}{0.528000in}}{\pgfqpoint{3.696000in}{3.696000in}}%
\pgfusepath{clip}%
\pgfsetbuttcap%
\pgfsetroundjoin%
\pgfsetlinewidth{1.505625pt}%
\definecolor{currentstroke}{rgb}{0.000000,0.000000,0.000000}%
\pgfsetstrokecolor{currentstroke}%
\pgfsetdash{}{0pt}%
\pgfpathmoveto{\pgfqpoint{3.616000in}{0.696000in}}%
\pgfpathlineto{\pgfqpoint{3.616000in}{1.032000in}}%
\pgfpathlineto{\pgfqpoint{3.952000in}{0.696000in}}%
\pgfpathlineto{\pgfqpoint{3.616000in}{0.696000in}}%
\pgfusepath{stroke}%
\end{pgfscope}%
\begin{pgfscope}%
\pgfpathrectangle{\pgfqpoint{1.432000in}{0.528000in}}{\pgfqpoint{3.696000in}{3.696000in}}%
\pgfusepath{clip}%
\pgfsetbuttcap%
\pgfsetroundjoin%
\pgfsetlinewidth{1.505625pt}%
\definecolor{currentstroke}{rgb}{0.000000,0.000000,0.000000}%
\pgfsetstrokecolor{currentstroke}%
\pgfsetdash{}{0pt}%
\pgfpathmoveto{\pgfqpoint{3.280000in}{1.032000in}}%
\pgfpathlineto{\pgfqpoint{3.616000in}{1.032000in}}%
\pgfpathlineto{\pgfqpoint{3.280000in}{1.368000in}}%
\pgfpathlineto{\pgfqpoint{3.280000in}{1.032000in}}%
\pgfusepath{stroke}%
\end{pgfscope}%
\begin{pgfscope}%
\pgfpathrectangle{\pgfqpoint{1.432000in}{0.528000in}}{\pgfqpoint{3.696000in}{3.696000in}}%
\pgfusepath{clip}%
\pgfsetbuttcap%
\pgfsetroundjoin%
\pgfsetlinewidth{1.505625pt}%
\definecolor{currentstroke}{rgb}{0.000000,0.000000,0.000000}%
\pgfsetstrokecolor{currentstroke}%
\pgfsetdash{}{0pt}%
\pgfpathmoveto{\pgfqpoint{2.944000in}{1.368000in}}%
\pgfpathlineto{\pgfqpoint{3.280000in}{1.368000in}}%
\pgfpathlineto{\pgfqpoint{2.944000in}{1.704000in}}%
\pgfpathlineto{\pgfqpoint{2.944000in}{1.368000in}}%
\pgfusepath{stroke}%
\end{pgfscope}%
\begin{pgfscope}%
\pgfpathrectangle{\pgfqpoint{1.432000in}{0.528000in}}{\pgfqpoint{3.696000in}{3.696000in}}%
\pgfusepath{clip}%
\pgfsetbuttcap%
\pgfsetroundjoin%
\pgfsetlinewidth{1.505625pt}%
\definecolor{currentstroke}{rgb}{0.000000,0.000000,0.000000}%
\pgfsetstrokecolor{currentstroke}%
\pgfsetdash{}{0pt}%
\pgfpathmoveto{\pgfqpoint{2.608000in}{1.704000in}}%
\pgfpathlineto{\pgfqpoint{2.944000in}{1.704000in}}%
\pgfpathlineto{\pgfqpoint{2.608000in}{2.040000in}}%
\pgfpathlineto{\pgfqpoint{2.608000in}{1.704000in}}%
\pgfusepath{stroke}%
\end{pgfscope}%
\begin{pgfscope}%
\pgfpathrectangle{\pgfqpoint{1.432000in}{0.528000in}}{\pgfqpoint{3.696000in}{3.696000in}}%
\pgfusepath{clip}%
\pgfsetbuttcap%
\pgfsetroundjoin%
\pgfsetlinewidth{1.505625pt}%
\definecolor{currentstroke}{rgb}{0.000000,0.000000,0.000000}%
\pgfsetstrokecolor{currentstroke}%
\pgfsetdash{}{0pt}%
\pgfpathmoveto{\pgfqpoint{2.272000in}{2.040000in}}%
\pgfpathlineto{\pgfqpoint{2.608000in}{2.040000in}}%
\pgfpathlineto{\pgfqpoint{2.272000in}{2.376000in}}%
\pgfpathlineto{\pgfqpoint{2.272000in}{2.040000in}}%
\pgfusepath{stroke}%
\end{pgfscope}%
\begin{pgfscope}%
\pgfpathrectangle{\pgfqpoint{1.432000in}{0.528000in}}{\pgfqpoint{3.696000in}{3.696000in}}%
\pgfusepath{clip}%
\pgfsetbuttcap%
\pgfsetroundjoin%
\pgfsetlinewidth{1.505625pt}%
\definecolor{currentstroke}{rgb}{0.000000,0.000000,0.000000}%
\pgfsetstrokecolor{currentstroke}%
\pgfsetdash{}{0pt}%
\pgfpathmoveto{\pgfqpoint{1.936000in}{2.376000in}}%
\pgfpathlineto{\pgfqpoint{2.272000in}{2.376000in}}%
\pgfpathlineto{\pgfqpoint{1.936000in}{2.712000in}}%
\pgfpathlineto{\pgfqpoint{1.936000in}{2.376000in}}%
\pgfusepath{stroke}%
\end{pgfscope}%
\begin{pgfscope}%
\pgfpathrectangle{\pgfqpoint{1.432000in}{0.528000in}}{\pgfqpoint{3.696000in}{3.696000in}}%
\pgfusepath{clip}%
\pgfsetbuttcap%
\pgfsetroundjoin%
\pgfsetlinewidth{1.505625pt}%
\definecolor{currentstroke}{rgb}{0.000000,0.000000,0.000000}%
\pgfsetstrokecolor{currentstroke}%
\pgfsetdash{}{0pt}%
\pgfpathmoveto{\pgfqpoint{1.600000in}{2.712000in}}%
\pgfpathlineto{\pgfqpoint{1.936000in}{2.712000in}}%
\pgfpathlineto{\pgfqpoint{1.600000in}{3.048000in}}%
\pgfpathlineto{\pgfqpoint{1.600000in}{2.712000in}}%
\pgfusepath{stroke}%
\end{pgfscope}%
\begin{pgfscope}%
\pgfpathrectangle{\pgfqpoint{1.432000in}{0.528000in}}{\pgfqpoint{3.696000in}{3.696000in}}%
\pgfusepath{clip}%
\pgfsetbuttcap%
\pgfsetroundjoin%
\pgfsetlinewidth{1.505625pt}%
\definecolor{currentstroke}{rgb}{0.000000,0.000000,0.000000}%
\pgfsetstrokecolor{currentstroke}%
\pgfsetdash{}{0pt}%
\pgfpathmoveto{\pgfqpoint{3.616000in}{1.032000in}}%
\pgfpathlineto{\pgfqpoint{3.952000in}{0.696000in}}%
\pgfpathlineto{\pgfqpoint{3.952000in}{1.032000in}}%
\pgfpathlineto{\pgfqpoint{3.616000in}{1.032000in}}%
\pgfusepath{stroke}%
\end{pgfscope}%
\begin{pgfscope}%
\pgfpathrectangle{\pgfqpoint{1.432000in}{0.528000in}}{\pgfqpoint{3.696000in}{3.696000in}}%
\pgfusepath{clip}%
\pgfsetbuttcap%
\pgfsetroundjoin%
\pgfsetlinewidth{1.505625pt}%
\definecolor{currentstroke}{rgb}{0.000000,0.000000,0.000000}%
\pgfsetstrokecolor{currentstroke}%
\pgfsetdash{}{0pt}%
\pgfpathmoveto{\pgfqpoint{3.616000in}{1.032000in}}%
\pgfpathlineto{\pgfqpoint{3.280000in}{1.368000in}}%
\pgfpathlineto{\pgfqpoint{3.616000in}{1.368000in}}%
\pgfpathlineto{\pgfqpoint{3.616000in}{1.032000in}}%
\pgfusepath{stroke}%
\end{pgfscope}%
\begin{pgfscope}%
\pgfpathrectangle{\pgfqpoint{1.432000in}{0.528000in}}{\pgfqpoint{3.696000in}{3.696000in}}%
\pgfusepath{clip}%
\pgfsetbuttcap%
\pgfsetroundjoin%
\pgfsetlinewidth{1.505625pt}%
\definecolor{currentstroke}{rgb}{0.000000,0.000000,0.000000}%
\pgfsetstrokecolor{currentstroke}%
\pgfsetdash{}{0pt}%
\pgfpathmoveto{\pgfqpoint{3.280000in}{1.368000in}}%
\pgfpathlineto{\pgfqpoint{2.944000in}{1.704000in}}%
\pgfpathlineto{\pgfqpoint{3.280000in}{1.704000in}}%
\pgfpathlineto{\pgfqpoint{3.280000in}{1.368000in}}%
\pgfusepath{stroke}%
\end{pgfscope}%
\begin{pgfscope}%
\pgfpathrectangle{\pgfqpoint{1.432000in}{0.528000in}}{\pgfqpoint{3.696000in}{3.696000in}}%
\pgfusepath{clip}%
\pgfsetbuttcap%
\pgfsetroundjoin%
\pgfsetlinewidth{1.505625pt}%
\definecolor{currentstroke}{rgb}{0.000000,0.000000,0.000000}%
\pgfsetstrokecolor{currentstroke}%
\pgfsetdash{}{0pt}%
\pgfpathmoveto{\pgfqpoint{2.944000in}{1.704000in}}%
\pgfpathlineto{\pgfqpoint{2.608000in}{2.040000in}}%
\pgfpathlineto{\pgfqpoint{2.944000in}{2.040000in}}%
\pgfpathlineto{\pgfqpoint{2.944000in}{1.704000in}}%
\pgfusepath{stroke}%
\end{pgfscope}%
\begin{pgfscope}%
\pgfpathrectangle{\pgfqpoint{1.432000in}{0.528000in}}{\pgfqpoint{3.696000in}{3.696000in}}%
\pgfusepath{clip}%
\pgfsetbuttcap%
\pgfsetroundjoin%
\pgfsetlinewidth{1.505625pt}%
\definecolor{currentstroke}{rgb}{0.000000,0.000000,0.000000}%
\pgfsetstrokecolor{currentstroke}%
\pgfsetdash{}{0pt}%
\pgfpathmoveto{\pgfqpoint{2.608000in}{2.040000in}}%
\pgfpathlineto{\pgfqpoint{2.272000in}{2.376000in}}%
\pgfpathlineto{\pgfqpoint{2.608000in}{2.376000in}}%
\pgfpathlineto{\pgfqpoint{2.608000in}{2.040000in}}%
\pgfusepath{stroke}%
\end{pgfscope}%
\begin{pgfscope}%
\pgfpathrectangle{\pgfqpoint{1.432000in}{0.528000in}}{\pgfqpoint{3.696000in}{3.696000in}}%
\pgfusepath{clip}%
\pgfsetbuttcap%
\pgfsetroundjoin%
\pgfsetlinewidth{1.505625pt}%
\definecolor{currentstroke}{rgb}{0.000000,0.000000,0.000000}%
\pgfsetstrokecolor{currentstroke}%
\pgfsetdash{}{0pt}%
\pgfpathmoveto{\pgfqpoint{2.272000in}{2.376000in}}%
\pgfpathlineto{\pgfqpoint{1.936000in}{2.712000in}}%
\pgfpathlineto{\pgfqpoint{2.272000in}{2.712000in}}%
\pgfpathlineto{\pgfqpoint{2.272000in}{2.376000in}}%
\pgfusepath{stroke}%
\end{pgfscope}%
\begin{pgfscope}%
\pgfpathrectangle{\pgfqpoint{1.432000in}{0.528000in}}{\pgfqpoint{3.696000in}{3.696000in}}%
\pgfusepath{clip}%
\pgfsetbuttcap%
\pgfsetroundjoin%
\pgfsetlinewidth{1.505625pt}%
\definecolor{currentstroke}{rgb}{0.000000,0.000000,0.000000}%
\pgfsetstrokecolor{currentstroke}%
\pgfsetdash{}{0pt}%
\pgfpathmoveto{\pgfqpoint{1.936000in}{2.712000in}}%
\pgfpathlineto{\pgfqpoint{1.600000in}{3.048000in}}%
\pgfpathlineto{\pgfqpoint{1.936000in}{3.048000in}}%
\pgfpathlineto{\pgfqpoint{1.936000in}{2.712000in}}%
\pgfusepath{stroke}%
\end{pgfscope}%
\begin{pgfscope}%
\pgfpathrectangle{\pgfqpoint{1.432000in}{0.528000in}}{\pgfqpoint{3.696000in}{3.696000in}}%
\pgfusepath{clip}%
\pgfsetbuttcap%
\pgfsetroundjoin%
\pgfsetlinewidth{1.505625pt}%
\definecolor{currentstroke}{rgb}{0.000000,0.000000,0.000000}%
\pgfsetstrokecolor{currentstroke}%
\pgfsetdash{}{0pt}%
\pgfpathmoveto{\pgfqpoint{3.952000in}{0.696000in}}%
\pgfpathlineto{\pgfqpoint{3.952000in}{1.032000in}}%
\pgfpathlineto{\pgfqpoint{4.288000in}{0.696000in}}%
\pgfpathlineto{\pgfqpoint{3.952000in}{0.696000in}}%
\pgfusepath{stroke}%
\end{pgfscope}%
\begin{pgfscope}%
\pgfpathrectangle{\pgfqpoint{1.432000in}{0.528000in}}{\pgfqpoint{3.696000in}{3.696000in}}%
\pgfusepath{clip}%
\pgfsetbuttcap%
\pgfsetroundjoin%
\pgfsetlinewidth{1.505625pt}%
\definecolor{currentstroke}{rgb}{0.000000,0.000000,0.000000}%
\pgfsetstrokecolor{currentstroke}%
\pgfsetdash{}{0pt}%
\pgfpathmoveto{\pgfqpoint{3.616000in}{1.032000in}}%
\pgfpathlineto{\pgfqpoint{3.952000in}{1.032000in}}%
\pgfpathlineto{\pgfqpoint{3.616000in}{1.368000in}}%
\pgfpathlineto{\pgfqpoint{3.616000in}{1.032000in}}%
\pgfusepath{stroke}%
\end{pgfscope}%
\begin{pgfscope}%
\pgfpathrectangle{\pgfqpoint{1.432000in}{0.528000in}}{\pgfqpoint{3.696000in}{3.696000in}}%
\pgfusepath{clip}%
\pgfsetbuttcap%
\pgfsetroundjoin%
\pgfsetlinewidth{1.505625pt}%
\definecolor{currentstroke}{rgb}{0.000000,0.000000,0.000000}%
\pgfsetstrokecolor{currentstroke}%
\pgfsetdash{}{0pt}%
\pgfpathmoveto{\pgfqpoint{3.280000in}{1.368000in}}%
\pgfpathlineto{\pgfqpoint{3.616000in}{1.368000in}}%
\pgfpathlineto{\pgfqpoint{3.280000in}{1.704000in}}%
\pgfpathlineto{\pgfqpoint{3.280000in}{1.368000in}}%
\pgfusepath{stroke}%
\end{pgfscope}%
\begin{pgfscope}%
\pgfpathrectangle{\pgfqpoint{1.432000in}{0.528000in}}{\pgfqpoint{3.696000in}{3.696000in}}%
\pgfusepath{clip}%
\pgfsetbuttcap%
\pgfsetroundjoin%
\pgfsetlinewidth{1.505625pt}%
\definecolor{currentstroke}{rgb}{0.000000,0.000000,0.000000}%
\pgfsetstrokecolor{currentstroke}%
\pgfsetdash{}{0pt}%
\pgfpathmoveto{\pgfqpoint{2.944000in}{1.704000in}}%
\pgfpathlineto{\pgfqpoint{3.280000in}{1.704000in}}%
\pgfpathlineto{\pgfqpoint{2.944000in}{2.040000in}}%
\pgfpathlineto{\pgfqpoint{2.944000in}{1.704000in}}%
\pgfusepath{stroke}%
\end{pgfscope}%
\begin{pgfscope}%
\pgfpathrectangle{\pgfqpoint{1.432000in}{0.528000in}}{\pgfqpoint{3.696000in}{3.696000in}}%
\pgfusepath{clip}%
\pgfsetbuttcap%
\pgfsetroundjoin%
\pgfsetlinewidth{1.505625pt}%
\definecolor{currentstroke}{rgb}{0.000000,0.000000,0.000000}%
\pgfsetstrokecolor{currentstroke}%
\pgfsetdash{}{0pt}%
\pgfpathmoveto{\pgfqpoint{2.608000in}{2.040000in}}%
\pgfpathlineto{\pgfqpoint{2.944000in}{2.040000in}}%
\pgfpathlineto{\pgfqpoint{2.608000in}{2.376000in}}%
\pgfpathlineto{\pgfqpoint{2.608000in}{2.040000in}}%
\pgfusepath{stroke}%
\end{pgfscope}%
\begin{pgfscope}%
\pgfpathrectangle{\pgfqpoint{1.432000in}{0.528000in}}{\pgfqpoint{3.696000in}{3.696000in}}%
\pgfusepath{clip}%
\pgfsetbuttcap%
\pgfsetroundjoin%
\pgfsetlinewidth{1.505625pt}%
\definecolor{currentstroke}{rgb}{0.000000,0.000000,0.000000}%
\pgfsetstrokecolor{currentstroke}%
\pgfsetdash{}{0pt}%
\pgfpathmoveto{\pgfqpoint{2.272000in}{2.376000in}}%
\pgfpathlineto{\pgfqpoint{2.608000in}{2.376000in}}%
\pgfpathlineto{\pgfqpoint{2.272000in}{2.712000in}}%
\pgfpathlineto{\pgfqpoint{2.272000in}{2.376000in}}%
\pgfusepath{stroke}%
\end{pgfscope}%
\begin{pgfscope}%
\pgfpathrectangle{\pgfqpoint{1.432000in}{0.528000in}}{\pgfqpoint{3.696000in}{3.696000in}}%
\pgfusepath{clip}%
\pgfsetbuttcap%
\pgfsetroundjoin%
\pgfsetlinewidth{1.505625pt}%
\definecolor{currentstroke}{rgb}{0.000000,0.000000,0.000000}%
\pgfsetstrokecolor{currentstroke}%
\pgfsetdash{}{0pt}%
\pgfpathmoveto{\pgfqpoint{1.936000in}{2.712000in}}%
\pgfpathlineto{\pgfqpoint{2.272000in}{2.712000in}}%
\pgfpathlineto{\pgfqpoint{1.936000in}{3.048000in}}%
\pgfpathlineto{\pgfqpoint{1.936000in}{2.712000in}}%
\pgfusepath{stroke}%
\end{pgfscope}%
\begin{pgfscope}%
\pgfpathrectangle{\pgfqpoint{1.432000in}{0.528000in}}{\pgfqpoint{3.696000in}{3.696000in}}%
\pgfusepath{clip}%
\pgfsetbuttcap%
\pgfsetroundjoin%
\pgfsetlinewidth{1.505625pt}%
\definecolor{currentstroke}{rgb}{0.000000,0.000000,0.000000}%
\pgfsetstrokecolor{currentstroke}%
\pgfsetdash{}{0pt}%
\pgfpathmoveto{\pgfqpoint{1.600000in}{3.048000in}}%
\pgfpathlineto{\pgfqpoint{1.936000in}{3.048000in}}%
\pgfpathlineto{\pgfqpoint{1.600000in}{3.384000in}}%
\pgfpathlineto{\pgfqpoint{1.600000in}{3.048000in}}%
\pgfusepath{stroke}%
\end{pgfscope}%
\begin{pgfscope}%
\pgfpathrectangle{\pgfqpoint{1.432000in}{0.528000in}}{\pgfqpoint{3.696000in}{3.696000in}}%
\pgfusepath{clip}%
\pgfsetbuttcap%
\pgfsetroundjoin%
\pgfsetlinewidth{1.505625pt}%
\definecolor{currentstroke}{rgb}{0.000000,0.000000,0.000000}%
\pgfsetstrokecolor{currentstroke}%
\pgfsetdash{}{0pt}%
\pgfpathmoveto{\pgfqpoint{3.952000in}{1.032000in}}%
\pgfpathlineto{\pgfqpoint{4.288000in}{0.696000in}}%
\pgfpathlineto{\pgfqpoint{4.288000in}{1.032000in}}%
\pgfpathlineto{\pgfqpoint{3.952000in}{1.032000in}}%
\pgfusepath{stroke}%
\end{pgfscope}%
\begin{pgfscope}%
\pgfpathrectangle{\pgfqpoint{1.432000in}{0.528000in}}{\pgfqpoint{3.696000in}{3.696000in}}%
\pgfusepath{clip}%
\pgfsetbuttcap%
\pgfsetroundjoin%
\pgfsetlinewidth{1.505625pt}%
\definecolor{currentstroke}{rgb}{0.000000,0.000000,0.000000}%
\pgfsetstrokecolor{currentstroke}%
\pgfsetdash{}{0pt}%
\pgfpathmoveto{\pgfqpoint{3.952000in}{1.032000in}}%
\pgfpathlineto{\pgfqpoint{3.616000in}{1.368000in}}%
\pgfpathlineto{\pgfqpoint{3.952000in}{1.368000in}}%
\pgfpathlineto{\pgfqpoint{3.952000in}{1.032000in}}%
\pgfusepath{stroke}%
\end{pgfscope}%
\begin{pgfscope}%
\pgfpathrectangle{\pgfqpoint{1.432000in}{0.528000in}}{\pgfqpoint{3.696000in}{3.696000in}}%
\pgfusepath{clip}%
\pgfsetbuttcap%
\pgfsetroundjoin%
\pgfsetlinewidth{1.505625pt}%
\definecolor{currentstroke}{rgb}{0.000000,0.000000,0.000000}%
\pgfsetstrokecolor{currentstroke}%
\pgfsetdash{}{0pt}%
\pgfpathmoveto{\pgfqpoint{3.616000in}{1.368000in}}%
\pgfpathlineto{\pgfqpoint{3.280000in}{1.704000in}}%
\pgfpathlineto{\pgfqpoint{3.616000in}{1.704000in}}%
\pgfpathlineto{\pgfqpoint{3.616000in}{1.368000in}}%
\pgfusepath{stroke}%
\end{pgfscope}%
\begin{pgfscope}%
\pgfpathrectangle{\pgfqpoint{1.432000in}{0.528000in}}{\pgfqpoint{3.696000in}{3.696000in}}%
\pgfusepath{clip}%
\pgfsetbuttcap%
\pgfsetroundjoin%
\pgfsetlinewidth{1.505625pt}%
\definecolor{currentstroke}{rgb}{0.000000,0.000000,0.000000}%
\pgfsetstrokecolor{currentstroke}%
\pgfsetdash{}{0pt}%
\pgfpathmoveto{\pgfqpoint{3.280000in}{1.704000in}}%
\pgfpathlineto{\pgfqpoint{2.944000in}{2.040000in}}%
\pgfpathlineto{\pgfqpoint{3.280000in}{2.040000in}}%
\pgfpathlineto{\pgfqpoint{3.280000in}{1.704000in}}%
\pgfusepath{stroke}%
\end{pgfscope}%
\begin{pgfscope}%
\pgfpathrectangle{\pgfqpoint{1.432000in}{0.528000in}}{\pgfqpoint{3.696000in}{3.696000in}}%
\pgfusepath{clip}%
\pgfsetbuttcap%
\pgfsetroundjoin%
\pgfsetlinewidth{1.505625pt}%
\definecolor{currentstroke}{rgb}{0.000000,0.000000,0.000000}%
\pgfsetstrokecolor{currentstroke}%
\pgfsetdash{}{0pt}%
\pgfpathmoveto{\pgfqpoint{2.944000in}{2.040000in}}%
\pgfpathlineto{\pgfqpoint{2.608000in}{2.376000in}}%
\pgfpathlineto{\pgfqpoint{2.944000in}{2.376000in}}%
\pgfpathlineto{\pgfqpoint{2.944000in}{2.040000in}}%
\pgfusepath{stroke}%
\end{pgfscope}%
\begin{pgfscope}%
\pgfpathrectangle{\pgfqpoint{1.432000in}{0.528000in}}{\pgfqpoint{3.696000in}{3.696000in}}%
\pgfusepath{clip}%
\pgfsetbuttcap%
\pgfsetroundjoin%
\pgfsetlinewidth{1.505625pt}%
\definecolor{currentstroke}{rgb}{0.000000,0.000000,0.000000}%
\pgfsetstrokecolor{currentstroke}%
\pgfsetdash{}{0pt}%
\pgfpathmoveto{\pgfqpoint{2.608000in}{2.376000in}}%
\pgfpathlineto{\pgfqpoint{2.272000in}{2.712000in}}%
\pgfpathlineto{\pgfqpoint{2.608000in}{2.712000in}}%
\pgfpathlineto{\pgfqpoint{2.608000in}{2.376000in}}%
\pgfusepath{stroke}%
\end{pgfscope}%
\begin{pgfscope}%
\pgfpathrectangle{\pgfqpoint{1.432000in}{0.528000in}}{\pgfqpoint{3.696000in}{3.696000in}}%
\pgfusepath{clip}%
\pgfsetbuttcap%
\pgfsetroundjoin%
\pgfsetlinewidth{1.505625pt}%
\definecolor{currentstroke}{rgb}{0.000000,0.000000,0.000000}%
\pgfsetstrokecolor{currentstroke}%
\pgfsetdash{}{0pt}%
\pgfpathmoveto{\pgfqpoint{2.272000in}{2.712000in}}%
\pgfpathlineto{\pgfqpoint{1.936000in}{3.048000in}}%
\pgfpathlineto{\pgfqpoint{2.272000in}{3.048000in}}%
\pgfpathlineto{\pgfqpoint{2.272000in}{2.712000in}}%
\pgfusepath{stroke}%
\end{pgfscope}%
\begin{pgfscope}%
\pgfpathrectangle{\pgfqpoint{1.432000in}{0.528000in}}{\pgfqpoint{3.696000in}{3.696000in}}%
\pgfusepath{clip}%
\pgfsetbuttcap%
\pgfsetroundjoin%
\pgfsetlinewidth{1.505625pt}%
\definecolor{currentstroke}{rgb}{0.000000,0.000000,0.000000}%
\pgfsetstrokecolor{currentstroke}%
\pgfsetdash{}{0pt}%
\pgfpathmoveto{\pgfqpoint{1.936000in}{3.048000in}}%
\pgfpathlineto{\pgfqpoint{1.600000in}{3.384000in}}%
\pgfpathlineto{\pgfqpoint{1.936000in}{3.384000in}}%
\pgfpathlineto{\pgfqpoint{1.936000in}{3.048000in}}%
\pgfusepath{stroke}%
\end{pgfscope}%
\begin{pgfscope}%
\pgfpathrectangle{\pgfqpoint{1.432000in}{0.528000in}}{\pgfqpoint{3.696000in}{3.696000in}}%
\pgfusepath{clip}%
\pgfsetbuttcap%
\pgfsetroundjoin%
\pgfsetlinewidth{1.505625pt}%
\definecolor{currentstroke}{rgb}{0.000000,0.000000,0.000000}%
\pgfsetstrokecolor{currentstroke}%
\pgfsetdash{}{0pt}%
\pgfpathmoveto{\pgfqpoint{4.288000in}{0.696000in}}%
\pgfpathlineto{\pgfqpoint{4.288000in}{1.032000in}}%
\pgfpathlineto{\pgfqpoint{4.624000in}{0.696000in}}%
\pgfpathlineto{\pgfqpoint{4.288000in}{0.696000in}}%
\pgfusepath{stroke}%
\end{pgfscope}%
\begin{pgfscope}%
\pgfpathrectangle{\pgfqpoint{1.432000in}{0.528000in}}{\pgfqpoint{3.696000in}{3.696000in}}%
\pgfusepath{clip}%
\pgfsetbuttcap%
\pgfsetroundjoin%
\pgfsetlinewidth{1.505625pt}%
\definecolor{currentstroke}{rgb}{0.000000,0.000000,0.000000}%
\pgfsetstrokecolor{currentstroke}%
\pgfsetdash{}{0pt}%
\pgfpathmoveto{\pgfqpoint{3.952000in}{1.032000in}}%
\pgfpathlineto{\pgfqpoint{4.288000in}{1.032000in}}%
\pgfpathlineto{\pgfqpoint{3.952000in}{1.368000in}}%
\pgfpathlineto{\pgfqpoint{3.952000in}{1.032000in}}%
\pgfusepath{stroke}%
\end{pgfscope}%
\begin{pgfscope}%
\pgfpathrectangle{\pgfqpoint{1.432000in}{0.528000in}}{\pgfqpoint{3.696000in}{3.696000in}}%
\pgfusepath{clip}%
\pgfsetbuttcap%
\pgfsetroundjoin%
\pgfsetlinewidth{1.505625pt}%
\definecolor{currentstroke}{rgb}{0.000000,0.000000,0.000000}%
\pgfsetstrokecolor{currentstroke}%
\pgfsetdash{}{0pt}%
\pgfpathmoveto{\pgfqpoint{3.616000in}{1.368000in}}%
\pgfpathlineto{\pgfqpoint{3.952000in}{1.368000in}}%
\pgfpathlineto{\pgfqpoint{3.616000in}{1.704000in}}%
\pgfpathlineto{\pgfqpoint{3.616000in}{1.368000in}}%
\pgfusepath{stroke}%
\end{pgfscope}%
\begin{pgfscope}%
\pgfpathrectangle{\pgfqpoint{1.432000in}{0.528000in}}{\pgfqpoint{3.696000in}{3.696000in}}%
\pgfusepath{clip}%
\pgfsetbuttcap%
\pgfsetroundjoin%
\pgfsetlinewidth{1.505625pt}%
\definecolor{currentstroke}{rgb}{0.000000,0.000000,0.000000}%
\pgfsetstrokecolor{currentstroke}%
\pgfsetdash{}{0pt}%
\pgfpathmoveto{\pgfqpoint{3.280000in}{1.704000in}}%
\pgfpathlineto{\pgfqpoint{3.616000in}{1.704000in}}%
\pgfpathlineto{\pgfqpoint{3.280000in}{2.040000in}}%
\pgfpathlineto{\pgfqpoint{3.280000in}{1.704000in}}%
\pgfusepath{stroke}%
\end{pgfscope}%
\begin{pgfscope}%
\pgfpathrectangle{\pgfqpoint{1.432000in}{0.528000in}}{\pgfqpoint{3.696000in}{3.696000in}}%
\pgfusepath{clip}%
\pgfsetbuttcap%
\pgfsetroundjoin%
\pgfsetlinewidth{1.505625pt}%
\definecolor{currentstroke}{rgb}{0.000000,0.000000,0.000000}%
\pgfsetstrokecolor{currentstroke}%
\pgfsetdash{}{0pt}%
\pgfpathmoveto{\pgfqpoint{2.944000in}{2.040000in}}%
\pgfpathlineto{\pgfqpoint{3.280000in}{2.040000in}}%
\pgfpathlineto{\pgfqpoint{2.944000in}{2.376000in}}%
\pgfpathlineto{\pgfqpoint{2.944000in}{2.040000in}}%
\pgfusepath{stroke}%
\end{pgfscope}%
\begin{pgfscope}%
\pgfpathrectangle{\pgfqpoint{1.432000in}{0.528000in}}{\pgfqpoint{3.696000in}{3.696000in}}%
\pgfusepath{clip}%
\pgfsetbuttcap%
\pgfsetroundjoin%
\pgfsetlinewidth{1.505625pt}%
\definecolor{currentstroke}{rgb}{0.000000,0.000000,0.000000}%
\pgfsetstrokecolor{currentstroke}%
\pgfsetdash{}{0pt}%
\pgfpathmoveto{\pgfqpoint{2.608000in}{2.376000in}}%
\pgfpathlineto{\pgfqpoint{2.944000in}{2.376000in}}%
\pgfpathlineto{\pgfqpoint{2.608000in}{2.712000in}}%
\pgfpathlineto{\pgfqpoint{2.608000in}{2.376000in}}%
\pgfusepath{stroke}%
\end{pgfscope}%
\begin{pgfscope}%
\pgfpathrectangle{\pgfqpoint{1.432000in}{0.528000in}}{\pgfqpoint{3.696000in}{3.696000in}}%
\pgfusepath{clip}%
\pgfsetbuttcap%
\pgfsetroundjoin%
\pgfsetlinewidth{1.505625pt}%
\definecolor{currentstroke}{rgb}{0.000000,0.000000,0.000000}%
\pgfsetstrokecolor{currentstroke}%
\pgfsetdash{}{0pt}%
\pgfpathmoveto{\pgfqpoint{2.272000in}{2.712000in}}%
\pgfpathlineto{\pgfqpoint{2.608000in}{2.712000in}}%
\pgfpathlineto{\pgfqpoint{2.272000in}{3.048000in}}%
\pgfpathlineto{\pgfqpoint{2.272000in}{2.712000in}}%
\pgfusepath{stroke}%
\end{pgfscope}%
\begin{pgfscope}%
\pgfpathrectangle{\pgfqpoint{1.432000in}{0.528000in}}{\pgfqpoint{3.696000in}{3.696000in}}%
\pgfusepath{clip}%
\pgfsetbuttcap%
\pgfsetroundjoin%
\pgfsetlinewidth{1.505625pt}%
\definecolor{currentstroke}{rgb}{0.000000,0.000000,0.000000}%
\pgfsetstrokecolor{currentstroke}%
\pgfsetdash{}{0pt}%
\pgfpathmoveto{\pgfqpoint{1.936000in}{3.048000in}}%
\pgfpathlineto{\pgfqpoint{2.272000in}{3.048000in}}%
\pgfpathlineto{\pgfqpoint{1.936000in}{3.384000in}}%
\pgfpathlineto{\pgfqpoint{1.936000in}{3.048000in}}%
\pgfusepath{stroke}%
\end{pgfscope}%
\begin{pgfscope}%
\pgfpathrectangle{\pgfqpoint{1.432000in}{0.528000in}}{\pgfqpoint{3.696000in}{3.696000in}}%
\pgfusepath{clip}%
\pgfsetbuttcap%
\pgfsetroundjoin%
\pgfsetlinewidth{1.505625pt}%
\definecolor{currentstroke}{rgb}{0.000000,0.000000,0.000000}%
\pgfsetstrokecolor{currentstroke}%
\pgfsetdash{}{0pt}%
\pgfpathmoveto{\pgfqpoint{1.600000in}{3.384000in}}%
\pgfpathlineto{\pgfqpoint{1.936000in}{3.384000in}}%
\pgfpathlineto{\pgfqpoint{1.600000in}{3.720000in}}%
\pgfpathlineto{\pgfqpoint{1.600000in}{3.384000in}}%
\pgfusepath{stroke}%
\end{pgfscope}%
\begin{pgfscope}%
\pgfpathrectangle{\pgfqpoint{1.432000in}{0.528000in}}{\pgfqpoint{3.696000in}{3.696000in}}%
\pgfusepath{clip}%
\pgfsetbuttcap%
\pgfsetroundjoin%
\pgfsetlinewidth{1.505625pt}%
\definecolor{currentstroke}{rgb}{0.000000,0.000000,0.000000}%
\pgfsetstrokecolor{currentstroke}%
\pgfsetdash{}{0pt}%
\pgfpathmoveto{\pgfqpoint{4.288000in}{1.032000in}}%
\pgfpathlineto{\pgfqpoint{4.624000in}{0.696000in}}%
\pgfpathlineto{\pgfqpoint{4.624000in}{1.032000in}}%
\pgfpathlineto{\pgfqpoint{4.288000in}{1.032000in}}%
\pgfusepath{stroke}%
\end{pgfscope}%
\begin{pgfscope}%
\pgfpathrectangle{\pgfqpoint{1.432000in}{0.528000in}}{\pgfqpoint{3.696000in}{3.696000in}}%
\pgfusepath{clip}%
\pgfsetbuttcap%
\pgfsetroundjoin%
\pgfsetlinewidth{1.505625pt}%
\definecolor{currentstroke}{rgb}{0.000000,0.000000,0.000000}%
\pgfsetstrokecolor{currentstroke}%
\pgfsetdash{}{0pt}%
\pgfpathmoveto{\pgfqpoint{4.288000in}{1.032000in}}%
\pgfpathlineto{\pgfqpoint{3.952000in}{1.368000in}}%
\pgfpathlineto{\pgfqpoint{4.288000in}{1.368000in}}%
\pgfpathlineto{\pgfqpoint{4.288000in}{1.032000in}}%
\pgfusepath{stroke}%
\end{pgfscope}%
\begin{pgfscope}%
\pgfpathrectangle{\pgfqpoint{1.432000in}{0.528000in}}{\pgfqpoint{3.696000in}{3.696000in}}%
\pgfusepath{clip}%
\pgfsetbuttcap%
\pgfsetroundjoin%
\pgfsetlinewidth{1.505625pt}%
\definecolor{currentstroke}{rgb}{0.000000,0.000000,0.000000}%
\pgfsetstrokecolor{currentstroke}%
\pgfsetdash{}{0pt}%
\pgfpathmoveto{\pgfqpoint{3.952000in}{1.368000in}}%
\pgfpathlineto{\pgfqpoint{3.616000in}{1.704000in}}%
\pgfpathlineto{\pgfqpoint{3.952000in}{1.704000in}}%
\pgfpathlineto{\pgfqpoint{3.952000in}{1.368000in}}%
\pgfusepath{stroke}%
\end{pgfscope}%
\begin{pgfscope}%
\pgfpathrectangle{\pgfqpoint{1.432000in}{0.528000in}}{\pgfqpoint{3.696000in}{3.696000in}}%
\pgfusepath{clip}%
\pgfsetbuttcap%
\pgfsetroundjoin%
\pgfsetlinewidth{1.505625pt}%
\definecolor{currentstroke}{rgb}{0.000000,0.000000,0.000000}%
\pgfsetstrokecolor{currentstroke}%
\pgfsetdash{}{0pt}%
\pgfpathmoveto{\pgfqpoint{3.616000in}{1.704000in}}%
\pgfpathlineto{\pgfqpoint{3.280000in}{2.040000in}}%
\pgfpathlineto{\pgfqpoint{3.616000in}{2.040000in}}%
\pgfpathlineto{\pgfqpoint{3.616000in}{1.704000in}}%
\pgfusepath{stroke}%
\end{pgfscope}%
\begin{pgfscope}%
\pgfpathrectangle{\pgfqpoint{1.432000in}{0.528000in}}{\pgfqpoint{3.696000in}{3.696000in}}%
\pgfusepath{clip}%
\pgfsetbuttcap%
\pgfsetroundjoin%
\pgfsetlinewidth{1.505625pt}%
\definecolor{currentstroke}{rgb}{0.000000,0.000000,0.000000}%
\pgfsetstrokecolor{currentstroke}%
\pgfsetdash{}{0pt}%
\pgfpathmoveto{\pgfqpoint{3.280000in}{2.040000in}}%
\pgfpathlineto{\pgfqpoint{2.944000in}{2.376000in}}%
\pgfpathlineto{\pgfqpoint{3.280000in}{2.376000in}}%
\pgfpathlineto{\pgfqpoint{3.280000in}{2.040000in}}%
\pgfusepath{stroke}%
\end{pgfscope}%
\begin{pgfscope}%
\pgfpathrectangle{\pgfqpoint{1.432000in}{0.528000in}}{\pgfqpoint{3.696000in}{3.696000in}}%
\pgfusepath{clip}%
\pgfsetbuttcap%
\pgfsetroundjoin%
\pgfsetlinewidth{1.505625pt}%
\definecolor{currentstroke}{rgb}{0.000000,0.000000,0.000000}%
\pgfsetstrokecolor{currentstroke}%
\pgfsetdash{}{0pt}%
\pgfpathmoveto{\pgfqpoint{2.944000in}{2.376000in}}%
\pgfpathlineto{\pgfqpoint{2.608000in}{2.712000in}}%
\pgfpathlineto{\pgfqpoint{2.944000in}{2.712000in}}%
\pgfpathlineto{\pgfqpoint{2.944000in}{2.376000in}}%
\pgfusepath{stroke}%
\end{pgfscope}%
\begin{pgfscope}%
\pgfpathrectangle{\pgfqpoint{1.432000in}{0.528000in}}{\pgfqpoint{3.696000in}{3.696000in}}%
\pgfusepath{clip}%
\pgfsetbuttcap%
\pgfsetroundjoin%
\pgfsetlinewidth{1.505625pt}%
\definecolor{currentstroke}{rgb}{0.000000,0.000000,0.000000}%
\pgfsetstrokecolor{currentstroke}%
\pgfsetdash{}{0pt}%
\pgfpathmoveto{\pgfqpoint{2.608000in}{2.712000in}}%
\pgfpathlineto{\pgfqpoint{2.272000in}{3.048000in}}%
\pgfpathlineto{\pgfqpoint{2.608000in}{3.048000in}}%
\pgfpathlineto{\pgfqpoint{2.608000in}{2.712000in}}%
\pgfusepath{stroke}%
\end{pgfscope}%
\begin{pgfscope}%
\pgfpathrectangle{\pgfqpoint{1.432000in}{0.528000in}}{\pgfqpoint{3.696000in}{3.696000in}}%
\pgfusepath{clip}%
\pgfsetbuttcap%
\pgfsetroundjoin%
\pgfsetlinewidth{1.505625pt}%
\definecolor{currentstroke}{rgb}{0.000000,0.000000,0.000000}%
\pgfsetstrokecolor{currentstroke}%
\pgfsetdash{}{0pt}%
\pgfpathmoveto{\pgfqpoint{2.272000in}{3.048000in}}%
\pgfpathlineto{\pgfqpoint{1.936000in}{3.384000in}}%
\pgfpathlineto{\pgfqpoint{2.272000in}{3.384000in}}%
\pgfpathlineto{\pgfqpoint{2.272000in}{3.048000in}}%
\pgfusepath{stroke}%
\end{pgfscope}%
\begin{pgfscope}%
\pgfpathrectangle{\pgfqpoint{1.432000in}{0.528000in}}{\pgfqpoint{3.696000in}{3.696000in}}%
\pgfusepath{clip}%
\pgfsetbuttcap%
\pgfsetroundjoin%
\pgfsetlinewidth{1.505625pt}%
\definecolor{currentstroke}{rgb}{0.000000,0.000000,0.000000}%
\pgfsetstrokecolor{currentstroke}%
\pgfsetdash{}{0pt}%
\pgfpathmoveto{\pgfqpoint{1.936000in}{3.384000in}}%
\pgfpathlineto{\pgfqpoint{1.600000in}{3.720000in}}%
\pgfpathlineto{\pgfqpoint{1.936000in}{3.720000in}}%
\pgfpathlineto{\pgfqpoint{1.936000in}{3.384000in}}%
\pgfusepath{stroke}%
\end{pgfscope}%
\begin{pgfscope}%
\pgfpathrectangle{\pgfqpoint{1.432000in}{0.528000in}}{\pgfqpoint{3.696000in}{3.696000in}}%
\pgfusepath{clip}%
\pgfsetbuttcap%
\pgfsetroundjoin%
\pgfsetlinewidth{1.505625pt}%
\definecolor{currentstroke}{rgb}{0.000000,0.000000,0.000000}%
\pgfsetstrokecolor{currentstroke}%
\pgfsetdash{}{0pt}%
\pgfpathmoveto{\pgfqpoint{4.624000in}{0.696000in}}%
\pgfpathlineto{\pgfqpoint{4.624000in}{1.032000in}}%
\pgfpathlineto{\pgfqpoint{4.960000in}{0.696000in}}%
\pgfpathlineto{\pgfqpoint{4.624000in}{0.696000in}}%
\pgfusepath{stroke}%
\end{pgfscope}%
\begin{pgfscope}%
\pgfpathrectangle{\pgfqpoint{1.432000in}{0.528000in}}{\pgfqpoint{3.696000in}{3.696000in}}%
\pgfusepath{clip}%
\pgfsetbuttcap%
\pgfsetroundjoin%
\pgfsetlinewidth{1.505625pt}%
\definecolor{currentstroke}{rgb}{0.000000,0.000000,0.000000}%
\pgfsetstrokecolor{currentstroke}%
\pgfsetdash{}{0pt}%
\pgfpathmoveto{\pgfqpoint{4.288000in}{1.032000in}}%
\pgfpathlineto{\pgfqpoint{4.624000in}{1.032000in}}%
\pgfpathlineto{\pgfqpoint{4.288000in}{1.368000in}}%
\pgfpathlineto{\pgfqpoint{4.288000in}{1.032000in}}%
\pgfusepath{stroke}%
\end{pgfscope}%
\begin{pgfscope}%
\pgfpathrectangle{\pgfqpoint{1.432000in}{0.528000in}}{\pgfqpoint{3.696000in}{3.696000in}}%
\pgfusepath{clip}%
\pgfsetbuttcap%
\pgfsetroundjoin%
\pgfsetlinewidth{1.505625pt}%
\definecolor{currentstroke}{rgb}{0.000000,0.000000,0.000000}%
\pgfsetstrokecolor{currentstroke}%
\pgfsetdash{}{0pt}%
\pgfpathmoveto{\pgfqpoint{3.952000in}{1.368000in}}%
\pgfpathlineto{\pgfqpoint{4.288000in}{1.368000in}}%
\pgfpathlineto{\pgfqpoint{3.952000in}{1.704000in}}%
\pgfpathlineto{\pgfqpoint{3.952000in}{1.368000in}}%
\pgfusepath{stroke}%
\end{pgfscope}%
\begin{pgfscope}%
\pgfpathrectangle{\pgfqpoint{1.432000in}{0.528000in}}{\pgfqpoint{3.696000in}{3.696000in}}%
\pgfusepath{clip}%
\pgfsetbuttcap%
\pgfsetroundjoin%
\pgfsetlinewidth{1.505625pt}%
\definecolor{currentstroke}{rgb}{0.000000,0.000000,0.000000}%
\pgfsetstrokecolor{currentstroke}%
\pgfsetdash{}{0pt}%
\pgfpathmoveto{\pgfqpoint{3.616000in}{1.704000in}}%
\pgfpathlineto{\pgfqpoint{3.952000in}{1.704000in}}%
\pgfpathlineto{\pgfqpoint{3.616000in}{2.040000in}}%
\pgfpathlineto{\pgfqpoint{3.616000in}{1.704000in}}%
\pgfusepath{stroke}%
\end{pgfscope}%
\begin{pgfscope}%
\pgfpathrectangle{\pgfqpoint{1.432000in}{0.528000in}}{\pgfqpoint{3.696000in}{3.696000in}}%
\pgfusepath{clip}%
\pgfsetbuttcap%
\pgfsetroundjoin%
\pgfsetlinewidth{1.505625pt}%
\definecolor{currentstroke}{rgb}{0.000000,0.000000,0.000000}%
\pgfsetstrokecolor{currentstroke}%
\pgfsetdash{}{0pt}%
\pgfpathmoveto{\pgfqpoint{3.280000in}{2.040000in}}%
\pgfpathlineto{\pgfqpoint{3.616000in}{2.040000in}}%
\pgfpathlineto{\pgfqpoint{3.280000in}{2.376000in}}%
\pgfpathlineto{\pgfqpoint{3.280000in}{2.040000in}}%
\pgfusepath{stroke}%
\end{pgfscope}%
\begin{pgfscope}%
\pgfpathrectangle{\pgfqpoint{1.432000in}{0.528000in}}{\pgfqpoint{3.696000in}{3.696000in}}%
\pgfusepath{clip}%
\pgfsetbuttcap%
\pgfsetroundjoin%
\pgfsetlinewidth{1.505625pt}%
\definecolor{currentstroke}{rgb}{0.000000,0.000000,0.000000}%
\pgfsetstrokecolor{currentstroke}%
\pgfsetdash{}{0pt}%
\pgfpathmoveto{\pgfqpoint{2.944000in}{2.376000in}}%
\pgfpathlineto{\pgfqpoint{3.280000in}{2.376000in}}%
\pgfpathlineto{\pgfqpoint{2.944000in}{2.712000in}}%
\pgfpathlineto{\pgfqpoint{2.944000in}{2.376000in}}%
\pgfusepath{stroke}%
\end{pgfscope}%
\begin{pgfscope}%
\pgfpathrectangle{\pgfqpoint{1.432000in}{0.528000in}}{\pgfqpoint{3.696000in}{3.696000in}}%
\pgfusepath{clip}%
\pgfsetbuttcap%
\pgfsetroundjoin%
\pgfsetlinewidth{1.505625pt}%
\definecolor{currentstroke}{rgb}{0.000000,0.000000,0.000000}%
\pgfsetstrokecolor{currentstroke}%
\pgfsetdash{}{0pt}%
\pgfpathmoveto{\pgfqpoint{2.608000in}{2.712000in}}%
\pgfpathlineto{\pgfqpoint{2.944000in}{2.712000in}}%
\pgfpathlineto{\pgfqpoint{2.608000in}{3.048000in}}%
\pgfpathlineto{\pgfqpoint{2.608000in}{2.712000in}}%
\pgfusepath{stroke}%
\end{pgfscope}%
\begin{pgfscope}%
\pgfpathrectangle{\pgfqpoint{1.432000in}{0.528000in}}{\pgfqpoint{3.696000in}{3.696000in}}%
\pgfusepath{clip}%
\pgfsetbuttcap%
\pgfsetroundjoin%
\pgfsetlinewidth{1.505625pt}%
\definecolor{currentstroke}{rgb}{0.000000,0.000000,0.000000}%
\pgfsetstrokecolor{currentstroke}%
\pgfsetdash{}{0pt}%
\pgfpathmoveto{\pgfqpoint{2.272000in}{3.048000in}}%
\pgfpathlineto{\pgfqpoint{2.608000in}{3.048000in}}%
\pgfpathlineto{\pgfqpoint{2.272000in}{3.384000in}}%
\pgfpathlineto{\pgfqpoint{2.272000in}{3.048000in}}%
\pgfusepath{stroke}%
\end{pgfscope}%
\begin{pgfscope}%
\pgfpathrectangle{\pgfqpoint{1.432000in}{0.528000in}}{\pgfqpoint{3.696000in}{3.696000in}}%
\pgfusepath{clip}%
\pgfsetbuttcap%
\pgfsetroundjoin%
\pgfsetlinewidth{1.505625pt}%
\definecolor{currentstroke}{rgb}{0.000000,0.000000,0.000000}%
\pgfsetstrokecolor{currentstroke}%
\pgfsetdash{}{0pt}%
\pgfpathmoveto{\pgfqpoint{1.936000in}{3.384000in}}%
\pgfpathlineto{\pgfqpoint{2.272000in}{3.384000in}}%
\pgfpathlineto{\pgfqpoint{1.936000in}{3.720000in}}%
\pgfpathlineto{\pgfqpoint{1.936000in}{3.384000in}}%
\pgfusepath{stroke}%
\end{pgfscope}%
\begin{pgfscope}%
\pgfpathrectangle{\pgfqpoint{1.432000in}{0.528000in}}{\pgfqpoint{3.696000in}{3.696000in}}%
\pgfusepath{clip}%
\pgfsetbuttcap%
\pgfsetroundjoin%
\pgfsetlinewidth{1.505625pt}%
\definecolor{currentstroke}{rgb}{0.000000,0.000000,0.000000}%
\pgfsetstrokecolor{currentstroke}%
\pgfsetdash{}{0pt}%
\pgfpathmoveto{\pgfqpoint{1.600000in}{3.720000in}}%
\pgfpathlineto{\pgfqpoint{1.936000in}{3.720000in}}%
\pgfpathlineto{\pgfqpoint{1.600000in}{4.056000in}}%
\pgfpathlineto{\pgfqpoint{1.600000in}{3.720000in}}%
\pgfusepath{stroke}%
\end{pgfscope}%
\begin{pgfscope}%
\pgfpathrectangle{\pgfqpoint{1.432000in}{0.528000in}}{\pgfqpoint{3.696000in}{3.696000in}}%
\pgfusepath{clip}%
\pgfsetbuttcap%
\pgfsetroundjoin%
\pgfsetlinewidth{1.505625pt}%
\definecolor{currentstroke}{rgb}{0.000000,0.000000,0.000000}%
\pgfsetstrokecolor{currentstroke}%
\pgfsetdash{}{0pt}%
\pgfpathmoveto{\pgfqpoint{4.624000in}{1.032000in}}%
\pgfpathlineto{\pgfqpoint{4.960000in}{0.696000in}}%
\pgfpathlineto{\pgfqpoint{4.960000in}{1.032000in}}%
\pgfpathlineto{\pgfqpoint{4.624000in}{1.032000in}}%
\pgfusepath{stroke}%
\end{pgfscope}%
\begin{pgfscope}%
\pgfpathrectangle{\pgfqpoint{1.432000in}{0.528000in}}{\pgfqpoint{3.696000in}{3.696000in}}%
\pgfusepath{clip}%
\pgfsetbuttcap%
\pgfsetroundjoin%
\pgfsetlinewidth{1.505625pt}%
\definecolor{currentstroke}{rgb}{0.000000,0.000000,0.000000}%
\pgfsetstrokecolor{currentstroke}%
\pgfsetdash{}{0pt}%
\pgfpathmoveto{\pgfqpoint{4.624000in}{1.032000in}}%
\pgfpathlineto{\pgfqpoint{4.288000in}{1.368000in}}%
\pgfpathlineto{\pgfqpoint{4.624000in}{1.368000in}}%
\pgfpathlineto{\pgfqpoint{4.624000in}{1.032000in}}%
\pgfusepath{stroke}%
\end{pgfscope}%
\begin{pgfscope}%
\pgfpathrectangle{\pgfqpoint{1.432000in}{0.528000in}}{\pgfqpoint{3.696000in}{3.696000in}}%
\pgfusepath{clip}%
\pgfsetbuttcap%
\pgfsetroundjoin%
\pgfsetlinewidth{1.505625pt}%
\definecolor{currentstroke}{rgb}{0.000000,0.000000,0.000000}%
\pgfsetstrokecolor{currentstroke}%
\pgfsetdash{}{0pt}%
\pgfpathmoveto{\pgfqpoint{4.288000in}{1.368000in}}%
\pgfpathlineto{\pgfqpoint{3.952000in}{1.704000in}}%
\pgfpathlineto{\pgfqpoint{4.288000in}{1.704000in}}%
\pgfpathlineto{\pgfqpoint{4.288000in}{1.368000in}}%
\pgfusepath{stroke}%
\end{pgfscope}%
\begin{pgfscope}%
\pgfpathrectangle{\pgfqpoint{1.432000in}{0.528000in}}{\pgfqpoint{3.696000in}{3.696000in}}%
\pgfusepath{clip}%
\pgfsetbuttcap%
\pgfsetroundjoin%
\pgfsetlinewidth{1.505625pt}%
\definecolor{currentstroke}{rgb}{0.000000,0.000000,0.000000}%
\pgfsetstrokecolor{currentstroke}%
\pgfsetdash{}{0pt}%
\pgfpathmoveto{\pgfqpoint{3.952000in}{1.704000in}}%
\pgfpathlineto{\pgfqpoint{3.616000in}{2.040000in}}%
\pgfpathlineto{\pgfqpoint{3.952000in}{2.040000in}}%
\pgfpathlineto{\pgfqpoint{3.952000in}{1.704000in}}%
\pgfusepath{stroke}%
\end{pgfscope}%
\begin{pgfscope}%
\pgfpathrectangle{\pgfqpoint{1.432000in}{0.528000in}}{\pgfqpoint{3.696000in}{3.696000in}}%
\pgfusepath{clip}%
\pgfsetbuttcap%
\pgfsetroundjoin%
\pgfsetlinewidth{1.505625pt}%
\definecolor{currentstroke}{rgb}{0.000000,0.000000,0.000000}%
\pgfsetstrokecolor{currentstroke}%
\pgfsetdash{}{0pt}%
\pgfpathmoveto{\pgfqpoint{3.616000in}{2.040000in}}%
\pgfpathlineto{\pgfqpoint{3.280000in}{2.376000in}}%
\pgfpathlineto{\pgfqpoint{3.616000in}{2.376000in}}%
\pgfpathlineto{\pgfqpoint{3.616000in}{2.040000in}}%
\pgfusepath{stroke}%
\end{pgfscope}%
\begin{pgfscope}%
\pgfpathrectangle{\pgfqpoint{1.432000in}{0.528000in}}{\pgfqpoint{3.696000in}{3.696000in}}%
\pgfusepath{clip}%
\pgfsetbuttcap%
\pgfsetroundjoin%
\pgfsetlinewidth{1.505625pt}%
\definecolor{currentstroke}{rgb}{0.000000,0.000000,0.000000}%
\pgfsetstrokecolor{currentstroke}%
\pgfsetdash{}{0pt}%
\pgfpathmoveto{\pgfqpoint{3.280000in}{2.376000in}}%
\pgfpathlineto{\pgfqpoint{2.944000in}{2.712000in}}%
\pgfpathlineto{\pgfqpoint{3.280000in}{2.712000in}}%
\pgfpathlineto{\pgfqpoint{3.280000in}{2.376000in}}%
\pgfusepath{stroke}%
\end{pgfscope}%
\begin{pgfscope}%
\pgfpathrectangle{\pgfqpoint{1.432000in}{0.528000in}}{\pgfqpoint{3.696000in}{3.696000in}}%
\pgfusepath{clip}%
\pgfsetbuttcap%
\pgfsetroundjoin%
\pgfsetlinewidth{1.505625pt}%
\definecolor{currentstroke}{rgb}{0.000000,0.000000,0.000000}%
\pgfsetstrokecolor{currentstroke}%
\pgfsetdash{}{0pt}%
\pgfpathmoveto{\pgfqpoint{2.944000in}{2.712000in}}%
\pgfpathlineto{\pgfqpoint{2.608000in}{3.048000in}}%
\pgfpathlineto{\pgfqpoint{2.944000in}{3.048000in}}%
\pgfpathlineto{\pgfqpoint{2.944000in}{2.712000in}}%
\pgfusepath{stroke}%
\end{pgfscope}%
\begin{pgfscope}%
\pgfpathrectangle{\pgfqpoint{1.432000in}{0.528000in}}{\pgfqpoint{3.696000in}{3.696000in}}%
\pgfusepath{clip}%
\pgfsetbuttcap%
\pgfsetroundjoin%
\pgfsetlinewidth{1.505625pt}%
\definecolor{currentstroke}{rgb}{0.000000,0.000000,0.000000}%
\pgfsetstrokecolor{currentstroke}%
\pgfsetdash{}{0pt}%
\pgfpathmoveto{\pgfqpoint{2.608000in}{3.048000in}}%
\pgfpathlineto{\pgfqpoint{2.272000in}{3.384000in}}%
\pgfpathlineto{\pgfqpoint{2.608000in}{3.384000in}}%
\pgfpathlineto{\pgfqpoint{2.608000in}{3.048000in}}%
\pgfusepath{stroke}%
\end{pgfscope}%
\begin{pgfscope}%
\pgfpathrectangle{\pgfqpoint{1.432000in}{0.528000in}}{\pgfqpoint{3.696000in}{3.696000in}}%
\pgfusepath{clip}%
\pgfsetbuttcap%
\pgfsetroundjoin%
\pgfsetlinewidth{1.505625pt}%
\definecolor{currentstroke}{rgb}{0.000000,0.000000,0.000000}%
\pgfsetstrokecolor{currentstroke}%
\pgfsetdash{}{0pt}%
\pgfpathmoveto{\pgfqpoint{2.272000in}{3.384000in}}%
\pgfpathlineto{\pgfqpoint{1.936000in}{3.720000in}}%
\pgfpathlineto{\pgfqpoint{2.272000in}{3.720000in}}%
\pgfpathlineto{\pgfqpoint{2.272000in}{3.384000in}}%
\pgfusepath{stroke}%
\end{pgfscope}%
\begin{pgfscope}%
\pgfpathrectangle{\pgfqpoint{1.432000in}{0.528000in}}{\pgfqpoint{3.696000in}{3.696000in}}%
\pgfusepath{clip}%
\pgfsetbuttcap%
\pgfsetroundjoin%
\pgfsetlinewidth{1.505625pt}%
\definecolor{currentstroke}{rgb}{0.000000,0.000000,0.000000}%
\pgfsetstrokecolor{currentstroke}%
\pgfsetdash{}{0pt}%
\pgfpathmoveto{\pgfqpoint{1.936000in}{3.720000in}}%
\pgfpathlineto{\pgfqpoint{1.600000in}{4.056000in}}%
\pgfpathlineto{\pgfqpoint{1.936000in}{4.056000in}}%
\pgfpathlineto{\pgfqpoint{1.936000in}{3.720000in}}%
\pgfusepath{stroke}%
\end{pgfscope}%
\begin{pgfscope}%
\pgfpathrectangle{\pgfqpoint{1.432000in}{0.528000in}}{\pgfqpoint{3.696000in}{3.696000in}}%
\pgfusepath{clip}%
\pgfsetbuttcap%
\pgfsetroundjoin%
\pgfsetlinewidth{1.505625pt}%
\definecolor{currentstroke}{rgb}{0.000000,0.000000,0.000000}%
\pgfsetstrokecolor{currentstroke}%
\pgfsetdash{}{0pt}%
\pgfpathmoveto{\pgfqpoint{4.624000in}{1.032000in}}%
\pgfpathlineto{\pgfqpoint{4.960000in}{1.032000in}}%
\pgfpathlineto{\pgfqpoint{4.624000in}{1.368000in}}%
\pgfpathlineto{\pgfqpoint{4.624000in}{1.032000in}}%
\pgfusepath{stroke}%
\end{pgfscope}%
\begin{pgfscope}%
\pgfpathrectangle{\pgfqpoint{1.432000in}{0.528000in}}{\pgfqpoint{3.696000in}{3.696000in}}%
\pgfusepath{clip}%
\pgfsetbuttcap%
\pgfsetroundjoin%
\pgfsetlinewidth{1.505625pt}%
\definecolor{currentstroke}{rgb}{0.000000,0.000000,0.000000}%
\pgfsetstrokecolor{currentstroke}%
\pgfsetdash{}{0pt}%
\pgfpathmoveto{\pgfqpoint{4.288000in}{1.368000in}}%
\pgfpathlineto{\pgfqpoint{4.624000in}{1.368000in}}%
\pgfpathlineto{\pgfqpoint{4.288000in}{1.704000in}}%
\pgfpathlineto{\pgfqpoint{4.288000in}{1.368000in}}%
\pgfusepath{stroke}%
\end{pgfscope}%
\begin{pgfscope}%
\pgfpathrectangle{\pgfqpoint{1.432000in}{0.528000in}}{\pgfqpoint{3.696000in}{3.696000in}}%
\pgfusepath{clip}%
\pgfsetbuttcap%
\pgfsetroundjoin%
\pgfsetlinewidth{1.505625pt}%
\definecolor{currentstroke}{rgb}{0.000000,0.000000,0.000000}%
\pgfsetstrokecolor{currentstroke}%
\pgfsetdash{}{0pt}%
\pgfpathmoveto{\pgfqpoint{3.952000in}{1.704000in}}%
\pgfpathlineto{\pgfqpoint{4.288000in}{1.704000in}}%
\pgfpathlineto{\pgfqpoint{3.952000in}{2.040000in}}%
\pgfpathlineto{\pgfqpoint{3.952000in}{1.704000in}}%
\pgfusepath{stroke}%
\end{pgfscope}%
\begin{pgfscope}%
\pgfpathrectangle{\pgfqpoint{1.432000in}{0.528000in}}{\pgfqpoint{3.696000in}{3.696000in}}%
\pgfusepath{clip}%
\pgfsetbuttcap%
\pgfsetroundjoin%
\pgfsetlinewidth{1.505625pt}%
\definecolor{currentstroke}{rgb}{0.000000,0.000000,0.000000}%
\pgfsetstrokecolor{currentstroke}%
\pgfsetdash{}{0pt}%
\pgfpathmoveto{\pgfqpoint{3.616000in}{2.040000in}}%
\pgfpathlineto{\pgfqpoint{3.952000in}{2.040000in}}%
\pgfpathlineto{\pgfqpoint{3.616000in}{2.376000in}}%
\pgfpathlineto{\pgfqpoint{3.616000in}{2.040000in}}%
\pgfusepath{stroke}%
\end{pgfscope}%
\begin{pgfscope}%
\pgfpathrectangle{\pgfqpoint{1.432000in}{0.528000in}}{\pgfqpoint{3.696000in}{3.696000in}}%
\pgfusepath{clip}%
\pgfsetbuttcap%
\pgfsetroundjoin%
\pgfsetlinewidth{1.505625pt}%
\definecolor{currentstroke}{rgb}{0.000000,0.000000,0.000000}%
\pgfsetstrokecolor{currentstroke}%
\pgfsetdash{}{0pt}%
\pgfpathmoveto{\pgfqpoint{3.280000in}{2.376000in}}%
\pgfpathlineto{\pgfqpoint{3.616000in}{2.376000in}}%
\pgfpathlineto{\pgfqpoint{3.280000in}{2.712000in}}%
\pgfpathlineto{\pgfqpoint{3.280000in}{2.376000in}}%
\pgfusepath{stroke}%
\end{pgfscope}%
\begin{pgfscope}%
\pgfpathrectangle{\pgfqpoint{1.432000in}{0.528000in}}{\pgfqpoint{3.696000in}{3.696000in}}%
\pgfusepath{clip}%
\pgfsetbuttcap%
\pgfsetroundjoin%
\pgfsetlinewidth{1.505625pt}%
\definecolor{currentstroke}{rgb}{0.000000,0.000000,0.000000}%
\pgfsetstrokecolor{currentstroke}%
\pgfsetdash{}{0pt}%
\pgfpathmoveto{\pgfqpoint{2.944000in}{2.712000in}}%
\pgfpathlineto{\pgfqpoint{3.280000in}{2.712000in}}%
\pgfpathlineto{\pgfqpoint{2.944000in}{3.048000in}}%
\pgfpathlineto{\pgfqpoint{2.944000in}{2.712000in}}%
\pgfusepath{stroke}%
\end{pgfscope}%
\begin{pgfscope}%
\pgfpathrectangle{\pgfqpoint{1.432000in}{0.528000in}}{\pgfqpoint{3.696000in}{3.696000in}}%
\pgfusepath{clip}%
\pgfsetbuttcap%
\pgfsetroundjoin%
\pgfsetlinewidth{1.505625pt}%
\definecolor{currentstroke}{rgb}{0.000000,0.000000,0.000000}%
\pgfsetstrokecolor{currentstroke}%
\pgfsetdash{}{0pt}%
\pgfpathmoveto{\pgfqpoint{2.608000in}{3.048000in}}%
\pgfpathlineto{\pgfqpoint{2.944000in}{3.048000in}}%
\pgfpathlineto{\pgfqpoint{2.608000in}{3.384000in}}%
\pgfpathlineto{\pgfqpoint{2.608000in}{3.048000in}}%
\pgfusepath{stroke}%
\end{pgfscope}%
\begin{pgfscope}%
\pgfpathrectangle{\pgfqpoint{1.432000in}{0.528000in}}{\pgfqpoint{3.696000in}{3.696000in}}%
\pgfusepath{clip}%
\pgfsetbuttcap%
\pgfsetroundjoin%
\pgfsetlinewidth{1.505625pt}%
\definecolor{currentstroke}{rgb}{0.000000,0.000000,0.000000}%
\pgfsetstrokecolor{currentstroke}%
\pgfsetdash{}{0pt}%
\pgfpathmoveto{\pgfqpoint{2.272000in}{3.384000in}}%
\pgfpathlineto{\pgfqpoint{2.608000in}{3.384000in}}%
\pgfpathlineto{\pgfqpoint{2.272000in}{3.720000in}}%
\pgfpathlineto{\pgfqpoint{2.272000in}{3.384000in}}%
\pgfusepath{stroke}%
\end{pgfscope}%
\begin{pgfscope}%
\pgfpathrectangle{\pgfqpoint{1.432000in}{0.528000in}}{\pgfqpoint{3.696000in}{3.696000in}}%
\pgfusepath{clip}%
\pgfsetbuttcap%
\pgfsetroundjoin%
\pgfsetlinewidth{1.505625pt}%
\definecolor{currentstroke}{rgb}{0.000000,0.000000,0.000000}%
\pgfsetstrokecolor{currentstroke}%
\pgfsetdash{}{0pt}%
\pgfpathmoveto{\pgfqpoint{1.936000in}{3.720000in}}%
\pgfpathlineto{\pgfqpoint{2.272000in}{3.720000in}}%
\pgfpathlineto{\pgfqpoint{1.936000in}{4.056000in}}%
\pgfpathlineto{\pgfqpoint{1.936000in}{3.720000in}}%
\pgfusepath{stroke}%
\end{pgfscope}%
\begin{pgfscope}%
\pgfpathrectangle{\pgfqpoint{1.432000in}{0.528000in}}{\pgfqpoint{3.696000in}{3.696000in}}%
\pgfusepath{clip}%
\pgfsetbuttcap%
\pgfsetroundjoin%
\pgfsetlinewidth{1.505625pt}%
\definecolor{currentstroke}{rgb}{0.000000,0.000000,0.000000}%
\pgfsetstrokecolor{currentstroke}%
\pgfsetdash{}{0pt}%
\pgfpathmoveto{\pgfqpoint{4.960000in}{1.032000in}}%
\pgfpathlineto{\pgfqpoint{4.624000in}{1.368000in}}%
\pgfpathlineto{\pgfqpoint{4.960000in}{1.368000in}}%
\pgfpathlineto{\pgfqpoint{4.960000in}{1.032000in}}%
\pgfusepath{stroke}%
\end{pgfscope}%
\begin{pgfscope}%
\pgfpathrectangle{\pgfqpoint{1.432000in}{0.528000in}}{\pgfqpoint{3.696000in}{3.696000in}}%
\pgfusepath{clip}%
\pgfsetbuttcap%
\pgfsetroundjoin%
\pgfsetlinewidth{1.505625pt}%
\definecolor{currentstroke}{rgb}{0.000000,0.000000,0.000000}%
\pgfsetstrokecolor{currentstroke}%
\pgfsetdash{}{0pt}%
\pgfpathmoveto{\pgfqpoint{4.624000in}{1.368000in}}%
\pgfpathlineto{\pgfqpoint{4.288000in}{1.704000in}}%
\pgfpathlineto{\pgfqpoint{4.624000in}{1.704000in}}%
\pgfpathlineto{\pgfqpoint{4.624000in}{1.368000in}}%
\pgfusepath{stroke}%
\end{pgfscope}%
\begin{pgfscope}%
\pgfpathrectangle{\pgfqpoint{1.432000in}{0.528000in}}{\pgfqpoint{3.696000in}{3.696000in}}%
\pgfusepath{clip}%
\pgfsetbuttcap%
\pgfsetroundjoin%
\pgfsetlinewidth{1.505625pt}%
\definecolor{currentstroke}{rgb}{0.000000,0.000000,0.000000}%
\pgfsetstrokecolor{currentstroke}%
\pgfsetdash{}{0pt}%
\pgfpathmoveto{\pgfqpoint{4.288000in}{1.704000in}}%
\pgfpathlineto{\pgfqpoint{3.952000in}{2.040000in}}%
\pgfpathlineto{\pgfqpoint{4.288000in}{2.040000in}}%
\pgfpathlineto{\pgfqpoint{4.288000in}{1.704000in}}%
\pgfusepath{stroke}%
\end{pgfscope}%
\begin{pgfscope}%
\pgfpathrectangle{\pgfqpoint{1.432000in}{0.528000in}}{\pgfqpoint{3.696000in}{3.696000in}}%
\pgfusepath{clip}%
\pgfsetbuttcap%
\pgfsetroundjoin%
\pgfsetlinewidth{1.505625pt}%
\definecolor{currentstroke}{rgb}{0.000000,0.000000,0.000000}%
\pgfsetstrokecolor{currentstroke}%
\pgfsetdash{}{0pt}%
\pgfpathmoveto{\pgfqpoint{3.952000in}{2.040000in}}%
\pgfpathlineto{\pgfqpoint{3.616000in}{2.376000in}}%
\pgfpathlineto{\pgfqpoint{3.952000in}{2.376000in}}%
\pgfpathlineto{\pgfqpoint{3.952000in}{2.040000in}}%
\pgfusepath{stroke}%
\end{pgfscope}%
\begin{pgfscope}%
\pgfpathrectangle{\pgfqpoint{1.432000in}{0.528000in}}{\pgfqpoint{3.696000in}{3.696000in}}%
\pgfusepath{clip}%
\pgfsetbuttcap%
\pgfsetroundjoin%
\pgfsetlinewidth{1.505625pt}%
\definecolor{currentstroke}{rgb}{0.000000,0.000000,0.000000}%
\pgfsetstrokecolor{currentstroke}%
\pgfsetdash{}{0pt}%
\pgfpathmoveto{\pgfqpoint{3.616000in}{2.376000in}}%
\pgfpathlineto{\pgfqpoint{3.280000in}{2.712000in}}%
\pgfpathlineto{\pgfqpoint{3.616000in}{2.712000in}}%
\pgfpathlineto{\pgfqpoint{3.616000in}{2.376000in}}%
\pgfusepath{stroke}%
\end{pgfscope}%
\begin{pgfscope}%
\pgfpathrectangle{\pgfqpoint{1.432000in}{0.528000in}}{\pgfqpoint{3.696000in}{3.696000in}}%
\pgfusepath{clip}%
\pgfsetbuttcap%
\pgfsetroundjoin%
\pgfsetlinewidth{1.505625pt}%
\definecolor{currentstroke}{rgb}{0.000000,0.000000,0.000000}%
\pgfsetstrokecolor{currentstroke}%
\pgfsetdash{}{0pt}%
\pgfpathmoveto{\pgfqpoint{3.280000in}{2.712000in}}%
\pgfpathlineto{\pgfqpoint{2.944000in}{3.048000in}}%
\pgfpathlineto{\pgfqpoint{3.280000in}{3.048000in}}%
\pgfpathlineto{\pgfqpoint{3.280000in}{2.712000in}}%
\pgfusepath{stroke}%
\end{pgfscope}%
\begin{pgfscope}%
\pgfpathrectangle{\pgfqpoint{1.432000in}{0.528000in}}{\pgfqpoint{3.696000in}{3.696000in}}%
\pgfusepath{clip}%
\pgfsetbuttcap%
\pgfsetroundjoin%
\pgfsetlinewidth{1.505625pt}%
\definecolor{currentstroke}{rgb}{0.000000,0.000000,0.000000}%
\pgfsetstrokecolor{currentstroke}%
\pgfsetdash{}{0pt}%
\pgfpathmoveto{\pgfqpoint{2.944000in}{3.048000in}}%
\pgfpathlineto{\pgfqpoint{2.608000in}{3.384000in}}%
\pgfpathlineto{\pgfqpoint{2.944000in}{3.384000in}}%
\pgfpathlineto{\pgfqpoint{2.944000in}{3.048000in}}%
\pgfusepath{stroke}%
\end{pgfscope}%
\begin{pgfscope}%
\pgfpathrectangle{\pgfqpoint{1.432000in}{0.528000in}}{\pgfqpoint{3.696000in}{3.696000in}}%
\pgfusepath{clip}%
\pgfsetbuttcap%
\pgfsetroundjoin%
\pgfsetlinewidth{1.505625pt}%
\definecolor{currentstroke}{rgb}{0.000000,0.000000,0.000000}%
\pgfsetstrokecolor{currentstroke}%
\pgfsetdash{}{0pt}%
\pgfpathmoveto{\pgfqpoint{2.608000in}{3.384000in}}%
\pgfpathlineto{\pgfqpoint{2.272000in}{3.720000in}}%
\pgfpathlineto{\pgfqpoint{2.608000in}{3.720000in}}%
\pgfpathlineto{\pgfqpoint{2.608000in}{3.384000in}}%
\pgfusepath{stroke}%
\end{pgfscope}%
\begin{pgfscope}%
\pgfpathrectangle{\pgfqpoint{1.432000in}{0.528000in}}{\pgfqpoint{3.696000in}{3.696000in}}%
\pgfusepath{clip}%
\pgfsetbuttcap%
\pgfsetroundjoin%
\pgfsetlinewidth{1.505625pt}%
\definecolor{currentstroke}{rgb}{0.000000,0.000000,0.000000}%
\pgfsetstrokecolor{currentstroke}%
\pgfsetdash{}{0pt}%
\pgfpathmoveto{\pgfqpoint{2.272000in}{3.720000in}}%
\pgfpathlineto{\pgfqpoint{1.936000in}{4.056000in}}%
\pgfpathlineto{\pgfqpoint{2.272000in}{4.056000in}}%
\pgfpathlineto{\pgfqpoint{2.272000in}{3.720000in}}%
\pgfusepath{stroke}%
\end{pgfscope}%
\begin{pgfscope}%
\pgfpathrectangle{\pgfqpoint{1.432000in}{0.528000in}}{\pgfqpoint{3.696000in}{3.696000in}}%
\pgfusepath{clip}%
\pgfsetbuttcap%
\pgfsetroundjoin%
\pgfsetlinewidth{1.505625pt}%
\definecolor{currentstroke}{rgb}{0.000000,0.000000,0.000000}%
\pgfsetstrokecolor{currentstroke}%
\pgfsetdash{}{0pt}%
\pgfpathmoveto{\pgfqpoint{4.624000in}{1.368000in}}%
\pgfpathlineto{\pgfqpoint{4.960000in}{1.368000in}}%
\pgfpathlineto{\pgfqpoint{4.624000in}{1.704000in}}%
\pgfpathlineto{\pgfqpoint{4.624000in}{1.368000in}}%
\pgfusepath{stroke}%
\end{pgfscope}%
\begin{pgfscope}%
\pgfpathrectangle{\pgfqpoint{1.432000in}{0.528000in}}{\pgfqpoint{3.696000in}{3.696000in}}%
\pgfusepath{clip}%
\pgfsetbuttcap%
\pgfsetroundjoin%
\pgfsetlinewidth{1.505625pt}%
\definecolor{currentstroke}{rgb}{0.000000,0.000000,0.000000}%
\pgfsetstrokecolor{currentstroke}%
\pgfsetdash{}{0pt}%
\pgfpathmoveto{\pgfqpoint{4.288000in}{1.704000in}}%
\pgfpathlineto{\pgfqpoint{4.624000in}{1.704000in}}%
\pgfpathlineto{\pgfqpoint{4.288000in}{2.040000in}}%
\pgfpathlineto{\pgfqpoint{4.288000in}{1.704000in}}%
\pgfusepath{stroke}%
\end{pgfscope}%
\begin{pgfscope}%
\pgfpathrectangle{\pgfqpoint{1.432000in}{0.528000in}}{\pgfqpoint{3.696000in}{3.696000in}}%
\pgfusepath{clip}%
\pgfsetbuttcap%
\pgfsetroundjoin%
\pgfsetlinewidth{1.505625pt}%
\definecolor{currentstroke}{rgb}{0.000000,0.000000,0.000000}%
\pgfsetstrokecolor{currentstroke}%
\pgfsetdash{}{0pt}%
\pgfpathmoveto{\pgfqpoint{3.952000in}{2.040000in}}%
\pgfpathlineto{\pgfqpoint{4.288000in}{2.040000in}}%
\pgfpathlineto{\pgfqpoint{3.952000in}{2.376000in}}%
\pgfpathlineto{\pgfqpoint{3.952000in}{2.040000in}}%
\pgfusepath{stroke}%
\end{pgfscope}%
\begin{pgfscope}%
\pgfpathrectangle{\pgfqpoint{1.432000in}{0.528000in}}{\pgfqpoint{3.696000in}{3.696000in}}%
\pgfusepath{clip}%
\pgfsetbuttcap%
\pgfsetroundjoin%
\pgfsetlinewidth{1.505625pt}%
\definecolor{currentstroke}{rgb}{0.000000,0.000000,0.000000}%
\pgfsetstrokecolor{currentstroke}%
\pgfsetdash{}{0pt}%
\pgfpathmoveto{\pgfqpoint{3.616000in}{2.376000in}}%
\pgfpathlineto{\pgfqpoint{3.952000in}{2.376000in}}%
\pgfpathlineto{\pgfqpoint{3.616000in}{2.712000in}}%
\pgfpathlineto{\pgfqpoint{3.616000in}{2.376000in}}%
\pgfusepath{stroke}%
\end{pgfscope}%
\begin{pgfscope}%
\pgfpathrectangle{\pgfqpoint{1.432000in}{0.528000in}}{\pgfqpoint{3.696000in}{3.696000in}}%
\pgfusepath{clip}%
\pgfsetbuttcap%
\pgfsetroundjoin%
\pgfsetlinewidth{1.505625pt}%
\definecolor{currentstroke}{rgb}{0.000000,0.000000,0.000000}%
\pgfsetstrokecolor{currentstroke}%
\pgfsetdash{}{0pt}%
\pgfpathmoveto{\pgfqpoint{3.280000in}{2.712000in}}%
\pgfpathlineto{\pgfqpoint{3.616000in}{2.712000in}}%
\pgfpathlineto{\pgfqpoint{3.280000in}{3.048000in}}%
\pgfpathlineto{\pgfqpoint{3.280000in}{2.712000in}}%
\pgfusepath{stroke}%
\end{pgfscope}%
\begin{pgfscope}%
\pgfpathrectangle{\pgfqpoint{1.432000in}{0.528000in}}{\pgfqpoint{3.696000in}{3.696000in}}%
\pgfusepath{clip}%
\pgfsetbuttcap%
\pgfsetroundjoin%
\pgfsetlinewidth{1.505625pt}%
\definecolor{currentstroke}{rgb}{0.000000,0.000000,0.000000}%
\pgfsetstrokecolor{currentstroke}%
\pgfsetdash{}{0pt}%
\pgfpathmoveto{\pgfqpoint{2.944000in}{3.048000in}}%
\pgfpathlineto{\pgfqpoint{3.280000in}{3.048000in}}%
\pgfpathlineto{\pgfqpoint{2.944000in}{3.384000in}}%
\pgfpathlineto{\pgfqpoint{2.944000in}{3.048000in}}%
\pgfusepath{stroke}%
\end{pgfscope}%
\begin{pgfscope}%
\pgfpathrectangle{\pgfqpoint{1.432000in}{0.528000in}}{\pgfqpoint{3.696000in}{3.696000in}}%
\pgfusepath{clip}%
\pgfsetbuttcap%
\pgfsetroundjoin%
\pgfsetlinewidth{1.505625pt}%
\definecolor{currentstroke}{rgb}{0.000000,0.000000,0.000000}%
\pgfsetstrokecolor{currentstroke}%
\pgfsetdash{}{0pt}%
\pgfpathmoveto{\pgfqpoint{2.608000in}{3.384000in}}%
\pgfpathlineto{\pgfqpoint{2.944000in}{3.384000in}}%
\pgfpathlineto{\pgfqpoint{2.608000in}{3.720000in}}%
\pgfpathlineto{\pgfqpoint{2.608000in}{3.384000in}}%
\pgfusepath{stroke}%
\end{pgfscope}%
\begin{pgfscope}%
\pgfpathrectangle{\pgfqpoint{1.432000in}{0.528000in}}{\pgfqpoint{3.696000in}{3.696000in}}%
\pgfusepath{clip}%
\pgfsetbuttcap%
\pgfsetroundjoin%
\pgfsetlinewidth{1.505625pt}%
\definecolor{currentstroke}{rgb}{0.000000,0.000000,0.000000}%
\pgfsetstrokecolor{currentstroke}%
\pgfsetdash{}{0pt}%
\pgfpathmoveto{\pgfqpoint{2.272000in}{3.720000in}}%
\pgfpathlineto{\pgfqpoint{2.608000in}{3.720000in}}%
\pgfpathlineto{\pgfqpoint{2.272000in}{4.056000in}}%
\pgfpathlineto{\pgfqpoint{2.272000in}{3.720000in}}%
\pgfusepath{stroke}%
\end{pgfscope}%
\begin{pgfscope}%
\pgfpathrectangle{\pgfqpoint{1.432000in}{0.528000in}}{\pgfqpoint{3.696000in}{3.696000in}}%
\pgfusepath{clip}%
\pgfsetbuttcap%
\pgfsetroundjoin%
\pgfsetlinewidth{1.505625pt}%
\definecolor{currentstroke}{rgb}{0.000000,0.000000,0.000000}%
\pgfsetstrokecolor{currentstroke}%
\pgfsetdash{}{0pt}%
\pgfpathmoveto{\pgfqpoint{4.960000in}{1.368000in}}%
\pgfpathlineto{\pgfqpoint{4.624000in}{1.704000in}}%
\pgfpathlineto{\pgfqpoint{4.960000in}{1.704000in}}%
\pgfpathlineto{\pgfqpoint{4.960000in}{1.368000in}}%
\pgfusepath{stroke}%
\end{pgfscope}%
\begin{pgfscope}%
\pgfpathrectangle{\pgfqpoint{1.432000in}{0.528000in}}{\pgfqpoint{3.696000in}{3.696000in}}%
\pgfusepath{clip}%
\pgfsetbuttcap%
\pgfsetroundjoin%
\pgfsetlinewidth{1.505625pt}%
\definecolor{currentstroke}{rgb}{0.000000,0.000000,0.000000}%
\pgfsetstrokecolor{currentstroke}%
\pgfsetdash{}{0pt}%
\pgfpathmoveto{\pgfqpoint{4.624000in}{1.704000in}}%
\pgfpathlineto{\pgfqpoint{4.288000in}{2.040000in}}%
\pgfpathlineto{\pgfqpoint{4.624000in}{2.040000in}}%
\pgfpathlineto{\pgfqpoint{4.624000in}{1.704000in}}%
\pgfusepath{stroke}%
\end{pgfscope}%
\begin{pgfscope}%
\pgfpathrectangle{\pgfqpoint{1.432000in}{0.528000in}}{\pgfqpoint{3.696000in}{3.696000in}}%
\pgfusepath{clip}%
\pgfsetbuttcap%
\pgfsetroundjoin%
\pgfsetlinewidth{1.505625pt}%
\definecolor{currentstroke}{rgb}{0.000000,0.000000,0.000000}%
\pgfsetstrokecolor{currentstroke}%
\pgfsetdash{}{0pt}%
\pgfpathmoveto{\pgfqpoint{4.288000in}{2.040000in}}%
\pgfpathlineto{\pgfqpoint{3.952000in}{2.376000in}}%
\pgfpathlineto{\pgfqpoint{4.288000in}{2.376000in}}%
\pgfpathlineto{\pgfqpoint{4.288000in}{2.040000in}}%
\pgfusepath{stroke}%
\end{pgfscope}%
\begin{pgfscope}%
\pgfpathrectangle{\pgfqpoint{1.432000in}{0.528000in}}{\pgfqpoint{3.696000in}{3.696000in}}%
\pgfusepath{clip}%
\pgfsetbuttcap%
\pgfsetroundjoin%
\pgfsetlinewidth{1.505625pt}%
\definecolor{currentstroke}{rgb}{0.000000,0.000000,0.000000}%
\pgfsetstrokecolor{currentstroke}%
\pgfsetdash{}{0pt}%
\pgfpathmoveto{\pgfqpoint{3.952000in}{2.376000in}}%
\pgfpathlineto{\pgfqpoint{3.616000in}{2.712000in}}%
\pgfpathlineto{\pgfqpoint{3.952000in}{2.712000in}}%
\pgfpathlineto{\pgfqpoint{3.952000in}{2.376000in}}%
\pgfusepath{stroke}%
\end{pgfscope}%
\begin{pgfscope}%
\pgfpathrectangle{\pgfqpoint{1.432000in}{0.528000in}}{\pgfqpoint{3.696000in}{3.696000in}}%
\pgfusepath{clip}%
\pgfsetbuttcap%
\pgfsetroundjoin%
\pgfsetlinewidth{1.505625pt}%
\definecolor{currentstroke}{rgb}{0.000000,0.000000,0.000000}%
\pgfsetstrokecolor{currentstroke}%
\pgfsetdash{}{0pt}%
\pgfpathmoveto{\pgfqpoint{3.616000in}{2.712000in}}%
\pgfpathlineto{\pgfqpoint{3.280000in}{3.048000in}}%
\pgfpathlineto{\pgfqpoint{3.616000in}{3.048000in}}%
\pgfpathlineto{\pgfqpoint{3.616000in}{2.712000in}}%
\pgfusepath{stroke}%
\end{pgfscope}%
\begin{pgfscope}%
\pgfpathrectangle{\pgfqpoint{1.432000in}{0.528000in}}{\pgfqpoint{3.696000in}{3.696000in}}%
\pgfusepath{clip}%
\pgfsetbuttcap%
\pgfsetroundjoin%
\pgfsetlinewidth{1.505625pt}%
\definecolor{currentstroke}{rgb}{0.000000,0.000000,0.000000}%
\pgfsetstrokecolor{currentstroke}%
\pgfsetdash{}{0pt}%
\pgfpathmoveto{\pgfqpoint{3.280000in}{3.048000in}}%
\pgfpathlineto{\pgfqpoint{2.944000in}{3.384000in}}%
\pgfpathlineto{\pgfqpoint{3.280000in}{3.384000in}}%
\pgfpathlineto{\pgfqpoint{3.280000in}{3.048000in}}%
\pgfusepath{stroke}%
\end{pgfscope}%
\begin{pgfscope}%
\pgfpathrectangle{\pgfqpoint{1.432000in}{0.528000in}}{\pgfqpoint{3.696000in}{3.696000in}}%
\pgfusepath{clip}%
\pgfsetbuttcap%
\pgfsetroundjoin%
\pgfsetlinewidth{1.505625pt}%
\definecolor{currentstroke}{rgb}{0.000000,0.000000,0.000000}%
\pgfsetstrokecolor{currentstroke}%
\pgfsetdash{}{0pt}%
\pgfpathmoveto{\pgfqpoint{2.944000in}{3.384000in}}%
\pgfpathlineto{\pgfqpoint{2.608000in}{3.720000in}}%
\pgfpathlineto{\pgfqpoint{2.944000in}{3.720000in}}%
\pgfpathlineto{\pgfqpoint{2.944000in}{3.384000in}}%
\pgfusepath{stroke}%
\end{pgfscope}%
\begin{pgfscope}%
\pgfpathrectangle{\pgfqpoint{1.432000in}{0.528000in}}{\pgfqpoint{3.696000in}{3.696000in}}%
\pgfusepath{clip}%
\pgfsetbuttcap%
\pgfsetroundjoin%
\pgfsetlinewidth{1.505625pt}%
\definecolor{currentstroke}{rgb}{0.000000,0.000000,0.000000}%
\pgfsetstrokecolor{currentstroke}%
\pgfsetdash{}{0pt}%
\pgfpathmoveto{\pgfqpoint{2.608000in}{3.720000in}}%
\pgfpathlineto{\pgfqpoint{2.272000in}{4.056000in}}%
\pgfpathlineto{\pgfqpoint{2.608000in}{4.056000in}}%
\pgfpathlineto{\pgfqpoint{2.608000in}{3.720000in}}%
\pgfusepath{stroke}%
\end{pgfscope}%
\begin{pgfscope}%
\pgfpathrectangle{\pgfqpoint{1.432000in}{0.528000in}}{\pgfqpoint{3.696000in}{3.696000in}}%
\pgfusepath{clip}%
\pgfsetbuttcap%
\pgfsetroundjoin%
\pgfsetlinewidth{1.505625pt}%
\definecolor{currentstroke}{rgb}{0.000000,0.000000,0.000000}%
\pgfsetstrokecolor{currentstroke}%
\pgfsetdash{}{0pt}%
\pgfpathmoveto{\pgfqpoint{4.624000in}{1.704000in}}%
\pgfpathlineto{\pgfqpoint{4.960000in}{1.704000in}}%
\pgfpathlineto{\pgfqpoint{4.624000in}{2.040000in}}%
\pgfpathlineto{\pgfqpoint{4.624000in}{1.704000in}}%
\pgfusepath{stroke}%
\end{pgfscope}%
\begin{pgfscope}%
\pgfpathrectangle{\pgfqpoint{1.432000in}{0.528000in}}{\pgfqpoint{3.696000in}{3.696000in}}%
\pgfusepath{clip}%
\pgfsetbuttcap%
\pgfsetroundjoin%
\pgfsetlinewidth{1.505625pt}%
\definecolor{currentstroke}{rgb}{0.000000,0.000000,0.000000}%
\pgfsetstrokecolor{currentstroke}%
\pgfsetdash{}{0pt}%
\pgfpathmoveto{\pgfqpoint{4.288000in}{2.040000in}}%
\pgfpathlineto{\pgfqpoint{4.624000in}{2.040000in}}%
\pgfpathlineto{\pgfqpoint{4.288000in}{2.376000in}}%
\pgfpathlineto{\pgfqpoint{4.288000in}{2.040000in}}%
\pgfusepath{stroke}%
\end{pgfscope}%
\begin{pgfscope}%
\pgfpathrectangle{\pgfqpoint{1.432000in}{0.528000in}}{\pgfqpoint{3.696000in}{3.696000in}}%
\pgfusepath{clip}%
\pgfsetbuttcap%
\pgfsetroundjoin%
\pgfsetlinewidth{1.505625pt}%
\definecolor{currentstroke}{rgb}{0.000000,0.000000,0.000000}%
\pgfsetstrokecolor{currentstroke}%
\pgfsetdash{}{0pt}%
\pgfpathmoveto{\pgfqpoint{3.952000in}{2.376000in}}%
\pgfpathlineto{\pgfqpoint{4.288000in}{2.376000in}}%
\pgfpathlineto{\pgfqpoint{3.952000in}{2.712000in}}%
\pgfpathlineto{\pgfqpoint{3.952000in}{2.376000in}}%
\pgfusepath{stroke}%
\end{pgfscope}%
\begin{pgfscope}%
\pgfpathrectangle{\pgfqpoint{1.432000in}{0.528000in}}{\pgfqpoint{3.696000in}{3.696000in}}%
\pgfusepath{clip}%
\pgfsetbuttcap%
\pgfsetroundjoin%
\pgfsetlinewidth{1.505625pt}%
\definecolor{currentstroke}{rgb}{0.000000,0.000000,0.000000}%
\pgfsetstrokecolor{currentstroke}%
\pgfsetdash{}{0pt}%
\pgfpathmoveto{\pgfqpoint{3.616000in}{2.712000in}}%
\pgfpathlineto{\pgfqpoint{3.952000in}{2.712000in}}%
\pgfpathlineto{\pgfqpoint{3.616000in}{3.048000in}}%
\pgfpathlineto{\pgfqpoint{3.616000in}{2.712000in}}%
\pgfusepath{stroke}%
\end{pgfscope}%
\begin{pgfscope}%
\pgfpathrectangle{\pgfqpoint{1.432000in}{0.528000in}}{\pgfqpoint{3.696000in}{3.696000in}}%
\pgfusepath{clip}%
\pgfsetbuttcap%
\pgfsetroundjoin%
\pgfsetlinewidth{1.505625pt}%
\definecolor{currentstroke}{rgb}{0.000000,0.000000,0.000000}%
\pgfsetstrokecolor{currentstroke}%
\pgfsetdash{}{0pt}%
\pgfpathmoveto{\pgfqpoint{3.280000in}{3.048000in}}%
\pgfpathlineto{\pgfqpoint{3.616000in}{3.048000in}}%
\pgfpathlineto{\pgfqpoint{3.280000in}{3.384000in}}%
\pgfpathlineto{\pgfqpoint{3.280000in}{3.048000in}}%
\pgfusepath{stroke}%
\end{pgfscope}%
\begin{pgfscope}%
\pgfpathrectangle{\pgfqpoint{1.432000in}{0.528000in}}{\pgfqpoint{3.696000in}{3.696000in}}%
\pgfusepath{clip}%
\pgfsetbuttcap%
\pgfsetroundjoin%
\pgfsetlinewidth{1.505625pt}%
\definecolor{currentstroke}{rgb}{0.000000,0.000000,0.000000}%
\pgfsetstrokecolor{currentstroke}%
\pgfsetdash{}{0pt}%
\pgfpathmoveto{\pgfqpoint{2.944000in}{3.384000in}}%
\pgfpathlineto{\pgfqpoint{3.280000in}{3.384000in}}%
\pgfpathlineto{\pgfqpoint{2.944000in}{3.720000in}}%
\pgfpathlineto{\pgfqpoint{2.944000in}{3.384000in}}%
\pgfusepath{stroke}%
\end{pgfscope}%
\begin{pgfscope}%
\pgfpathrectangle{\pgfqpoint{1.432000in}{0.528000in}}{\pgfqpoint{3.696000in}{3.696000in}}%
\pgfusepath{clip}%
\pgfsetbuttcap%
\pgfsetroundjoin%
\pgfsetlinewidth{1.505625pt}%
\definecolor{currentstroke}{rgb}{0.000000,0.000000,0.000000}%
\pgfsetstrokecolor{currentstroke}%
\pgfsetdash{}{0pt}%
\pgfpathmoveto{\pgfqpoint{2.608000in}{3.720000in}}%
\pgfpathlineto{\pgfqpoint{2.944000in}{3.720000in}}%
\pgfpathlineto{\pgfqpoint{2.608000in}{4.056000in}}%
\pgfpathlineto{\pgfqpoint{2.608000in}{3.720000in}}%
\pgfusepath{stroke}%
\end{pgfscope}%
\begin{pgfscope}%
\pgfpathrectangle{\pgfqpoint{1.432000in}{0.528000in}}{\pgfqpoint{3.696000in}{3.696000in}}%
\pgfusepath{clip}%
\pgfsetbuttcap%
\pgfsetroundjoin%
\pgfsetlinewidth{1.505625pt}%
\definecolor{currentstroke}{rgb}{0.000000,0.000000,0.000000}%
\pgfsetstrokecolor{currentstroke}%
\pgfsetdash{}{0pt}%
\pgfpathmoveto{\pgfqpoint{4.960000in}{1.704000in}}%
\pgfpathlineto{\pgfqpoint{4.624000in}{2.040000in}}%
\pgfpathlineto{\pgfqpoint{4.960000in}{2.040000in}}%
\pgfpathlineto{\pgfqpoint{4.960000in}{1.704000in}}%
\pgfusepath{stroke}%
\end{pgfscope}%
\begin{pgfscope}%
\pgfpathrectangle{\pgfqpoint{1.432000in}{0.528000in}}{\pgfqpoint{3.696000in}{3.696000in}}%
\pgfusepath{clip}%
\pgfsetbuttcap%
\pgfsetroundjoin%
\pgfsetlinewidth{1.505625pt}%
\definecolor{currentstroke}{rgb}{0.000000,0.000000,0.000000}%
\pgfsetstrokecolor{currentstroke}%
\pgfsetdash{}{0pt}%
\pgfpathmoveto{\pgfqpoint{4.624000in}{2.040000in}}%
\pgfpathlineto{\pgfqpoint{4.288000in}{2.376000in}}%
\pgfpathlineto{\pgfqpoint{4.624000in}{2.376000in}}%
\pgfpathlineto{\pgfqpoint{4.624000in}{2.040000in}}%
\pgfusepath{stroke}%
\end{pgfscope}%
\begin{pgfscope}%
\pgfpathrectangle{\pgfqpoint{1.432000in}{0.528000in}}{\pgfqpoint{3.696000in}{3.696000in}}%
\pgfusepath{clip}%
\pgfsetbuttcap%
\pgfsetroundjoin%
\pgfsetlinewidth{1.505625pt}%
\definecolor{currentstroke}{rgb}{0.000000,0.000000,0.000000}%
\pgfsetstrokecolor{currentstroke}%
\pgfsetdash{}{0pt}%
\pgfpathmoveto{\pgfqpoint{4.288000in}{2.376000in}}%
\pgfpathlineto{\pgfqpoint{3.952000in}{2.712000in}}%
\pgfpathlineto{\pgfqpoint{4.288000in}{2.712000in}}%
\pgfpathlineto{\pgfqpoint{4.288000in}{2.376000in}}%
\pgfusepath{stroke}%
\end{pgfscope}%
\begin{pgfscope}%
\pgfpathrectangle{\pgfqpoint{1.432000in}{0.528000in}}{\pgfqpoint{3.696000in}{3.696000in}}%
\pgfusepath{clip}%
\pgfsetbuttcap%
\pgfsetroundjoin%
\pgfsetlinewidth{1.505625pt}%
\definecolor{currentstroke}{rgb}{0.000000,0.000000,0.000000}%
\pgfsetstrokecolor{currentstroke}%
\pgfsetdash{}{0pt}%
\pgfpathmoveto{\pgfqpoint{3.952000in}{2.712000in}}%
\pgfpathlineto{\pgfqpoint{3.616000in}{3.048000in}}%
\pgfpathlineto{\pgfqpoint{3.952000in}{3.048000in}}%
\pgfpathlineto{\pgfqpoint{3.952000in}{2.712000in}}%
\pgfusepath{stroke}%
\end{pgfscope}%
\begin{pgfscope}%
\pgfpathrectangle{\pgfqpoint{1.432000in}{0.528000in}}{\pgfqpoint{3.696000in}{3.696000in}}%
\pgfusepath{clip}%
\pgfsetbuttcap%
\pgfsetroundjoin%
\pgfsetlinewidth{1.505625pt}%
\definecolor{currentstroke}{rgb}{0.000000,0.000000,0.000000}%
\pgfsetstrokecolor{currentstroke}%
\pgfsetdash{}{0pt}%
\pgfpathmoveto{\pgfqpoint{3.616000in}{3.048000in}}%
\pgfpathlineto{\pgfqpoint{3.280000in}{3.384000in}}%
\pgfpathlineto{\pgfqpoint{3.616000in}{3.384000in}}%
\pgfpathlineto{\pgfqpoint{3.616000in}{3.048000in}}%
\pgfusepath{stroke}%
\end{pgfscope}%
\begin{pgfscope}%
\pgfpathrectangle{\pgfqpoint{1.432000in}{0.528000in}}{\pgfqpoint{3.696000in}{3.696000in}}%
\pgfusepath{clip}%
\pgfsetbuttcap%
\pgfsetroundjoin%
\pgfsetlinewidth{1.505625pt}%
\definecolor{currentstroke}{rgb}{0.000000,0.000000,0.000000}%
\pgfsetstrokecolor{currentstroke}%
\pgfsetdash{}{0pt}%
\pgfpathmoveto{\pgfqpoint{3.280000in}{3.384000in}}%
\pgfpathlineto{\pgfqpoint{2.944000in}{3.720000in}}%
\pgfpathlineto{\pgfqpoint{3.280000in}{3.720000in}}%
\pgfpathlineto{\pgfqpoint{3.280000in}{3.384000in}}%
\pgfusepath{stroke}%
\end{pgfscope}%
\begin{pgfscope}%
\pgfpathrectangle{\pgfqpoint{1.432000in}{0.528000in}}{\pgfqpoint{3.696000in}{3.696000in}}%
\pgfusepath{clip}%
\pgfsetbuttcap%
\pgfsetroundjoin%
\pgfsetlinewidth{1.505625pt}%
\definecolor{currentstroke}{rgb}{0.000000,0.000000,0.000000}%
\pgfsetstrokecolor{currentstroke}%
\pgfsetdash{}{0pt}%
\pgfpathmoveto{\pgfqpoint{2.944000in}{3.720000in}}%
\pgfpathlineto{\pgfqpoint{2.608000in}{4.056000in}}%
\pgfpathlineto{\pgfqpoint{2.944000in}{4.056000in}}%
\pgfpathlineto{\pgfqpoint{2.944000in}{3.720000in}}%
\pgfusepath{stroke}%
\end{pgfscope}%
\begin{pgfscope}%
\pgfpathrectangle{\pgfqpoint{1.432000in}{0.528000in}}{\pgfqpoint{3.696000in}{3.696000in}}%
\pgfusepath{clip}%
\pgfsetbuttcap%
\pgfsetroundjoin%
\pgfsetlinewidth{1.505625pt}%
\definecolor{currentstroke}{rgb}{0.000000,0.000000,0.000000}%
\pgfsetstrokecolor{currentstroke}%
\pgfsetdash{}{0pt}%
\pgfpathmoveto{\pgfqpoint{4.624000in}{2.040000in}}%
\pgfpathlineto{\pgfqpoint{4.960000in}{2.040000in}}%
\pgfpathlineto{\pgfqpoint{4.624000in}{2.376000in}}%
\pgfpathlineto{\pgfqpoint{4.624000in}{2.040000in}}%
\pgfusepath{stroke}%
\end{pgfscope}%
\begin{pgfscope}%
\pgfpathrectangle{\pgfqpoint{1.432000in}{0.528000in}}{\pgfqpoint{3.696000in}{3.696000in}}%
\pgfusepath{clip}%
\pgfsetbuttcap%
\pgfsetroundjoin%
\pgfsetlinewidth{1.505625pt}%
\definecolor{currentstroke}{rgb}{0.000000,0.000000,0.000000}%
\pgfsetstrokecolor{currentstroke}%
\pgfsetdash{}{0pt}%
\pgfpathmoveto{\pgfqpoint{4.288000in}{2.376000in}}%
\pgfpathlineto{\pgfqpoint{4.624000in}{2.376000in}}%
\pgfpathlineto{\pgfqpoint{4.288000in}{2.712000in}}%
\pgfpathlineto{\pgfqpoint{4.288000in}{2.376000in}}%
\pgfusepath{stroke}%
\end{pgfscope}%
\begin{pgfscope}%
\pgfpathrectangle{\pgfqpoint{1.432000in}{0.528000in}}{\pgfqpoint{3.696000in}{3.696000in}}%
\pgfusepath{clip}%
\pgfsetbuttcap%
\pgfsetroundjoin%
\pgfsetlinewidth{1.505625pt}%
\definecolor{currentstroke}{rgb}{0.000000,0.000000,0.000000}%
\pgfsetstrokecolor{currentstroke}%
\pgfsetdash{}{0pt}%
\pgfpathmoveto{\pgfqpoint{3.952000in}{2.712000in}}%
\pgfpathlineto{\pgfqpoint{4.288000in}{2.712000in}}%
\pgfpathlineto{\pgfqpoint{3.952000in}{3.048000in}}%
\pgfpathlineto{\pgfqpoint{3.952000in}{2.712000in}}%
\pgfusepath{stroke}%
\end{pgfscope}%
\begin{pgfscope}%
\pgfpathrectangle{\pgfqpoint{1.432000in}{0.528000in}}{\pgfqpoint{3.696000in}{3.696000in}}%
\pgfusepath{clip}%
\pgfsetbuttcap%
\pgfsetroundjoin%
\pgfsetlinewidth{1.505625pt}%
\definecolor{currentstroke}{rgb}{0.000000,0.000000,0.000000}%
\pgfsetstrokecolor{currentstroke}%
\pgfsetdash{}{0pt}%
\pgfpathmoveto{\pgfqpoint{3.616000in}{3.048000in}}%
\pgfpathlineto{\pgfqpoint{3.952000in}{3.048000in}}%
\pgfpathlineto{\pgfqpoint{3.616000in}{3.384000in}}%
\pgfpathlineto{\pgfqpoint{3.616000in}{3.048000in}}%
\pgfusepath{stroke}%
\end{pgfscope}%
\begin{pgfscope}%
\pgfpathrectangle{\pgfqpoint{1.432000in}{0.528000in}}{\pgfqpoint{3.696000in}{3.696000in}}%
\pgfusepath{clip}%
\pgfsetbuttcap%
\pgfsetroundjoin%
\pgfsetlinewidth{1.505625pt}%
\definecolor{currentstroke}{rgb}{0.000000,0.000000,0.000000}%
\pgfsetstrokecolor{currentstroke}%
\pgfsetdash{}{0pt}%
\pgfpathmoveto{\pgfqpoint{3.280000in}{3.384000in}}%
\pgfpathlineto{\pgfqpoint{3.616000in}{3.384000in}}%
\pgfpathlineto{\pgfqpoint{3.280000in}{3.720000in}}%
\pgfpathlineto{\pgfqpoint{3.280000in}{3.384000in}}%
\pgfusepath{stroke}%
\end{pgfscope}%
\begin{pgfscope}%
\pgfpathrectangle{\pgfqpoint{1.432000in}{0.528000in}}{\pgfqpoint{3.696000in}{3.696000in}}%
\pgfusepath{clip}%
\pgfsetbuttcap%
\pgfsetroundjoin%
\pgfsetlinewidth{1.505625pt}%
\definecolor{currentstroke}{rgb}{0.000000,0.000000,0.000000}%
\pgfsetstrokecolor{currentstroke}%
\pgfsetdash{}{0pt}%
\pgfpathmoveto{\pgfqpoint{2.944000in}{3.720000in}}%
\pgfpathlineto{\pgfqpoint{3.280000in}{3.720000in}}%
\pgfpathlineto{\pgfqpoint{2.944000in}{4.056000in}}%
\pgfpathlineto{\pgfqpoint{2.944000in}{3.720000in}}%
\pgfusepath{stroke}%
\end{pgfscope}%
\begin{pgfscope}%
\pgfpathrectangle{\pgfqpoint{1.432000in}{0.528000in}}{\pgfqpoint{3.696000in}{3.696000in}}%
\pgfusepath{clip}%
\pgfsetbuttcap%
\pgfsetroundjoin%
\pgfsetlinewidth{1.505625pt}%
\definecolor{currentstroke}{rgb}{0.000000,0.000000,0.000000}%
\pgfsetstrokecolor{currentstroke}%
\pgfsetdash{}{0pt}%
\pgfpathmoveto{\pgfqpoint{4.960000in}{2.040000in}}%
\pgfpathlineto{\pgfqpoint{4.624000in}{2.376000in}}%
\pgfpathlineto{\pgfqpoint{4.960000in}{2.376000in}}%
\pgfpathlineto{\pgfqpoint{4.960000in}{2.040000in}}%
\pgfusepath{stroke}%
\end{pgfscope}%
\begin{pgfscope}%
\pgfpathrectangle{\pgfqpoint{1.432000in}{0.528000in}}{\pgfqpoint{3.696000in}{3.696000in}}%
\pgfusepath{clip}%
\pgfsetbuttcap%
\pgfsetroundjoin%
\pgfsetlinewidth{1.505625pt}%
\definecolor{currentstroke}{rgb}{0.000000,0.000000,0.000000}%
\pgfsetstrokecolor{currentstroke}%
\pgfsetdash{}{0pt}%
\pgfpathmoveto{\pgfqpoint{4.624000in}{2.376000in}}%
\pgfpathlineto{\pgfqpoint{4.288000in}{2.712000in}}%
\pgfpathlineto{\pgfqpoint{4.624000in}{2.712000in}}%
\pgfpathlineto{\pgfqpoint{4.624000in}{2.376000in}}%
\pgfusepath{stroke}%
\end{pgfscope}%
\begin{pgfscope}%
\pgfpathrectangle{\pgfqpoint{1.432000in}{0.528000in}}{\pgfqpoint{3.696000in}{3.696000in}}%
\pgfusepath{clip}%
\pgfsetbuttcap%
\pgfsetroundjoin%
\pgfsetlinewidth{1.505625pt}%
\definecolor{currentstroke}{rgb}{0.000000,0.000000,0.000000}%
\pgfsetstrokecolor{currentstroke}%
\pgfsetdash{}{0pt}%
\pgfpathmoveto{\pgfqpoint{4.288000in}{2.712000in}}%
\pgfpathlineto{\pgfqpoint{3.952000in}{3.048000in}}%
\pgfpathlineto{\pgfqpoint{4.288000in}{3.048000in}}%
\pgfpathlineto{\pgfqpoint{4.288000in}{2.712000in}}%
\pgfusepath{stroke}%
\end{pgfscope}%
\begin{pgfscope}%
\pgfpathrectangle{\pgfqpoint{1.432000in}{0.528000in}}{\pgfqpoint{3.696000in}{3.696000in}}%
\pgfusepath{clip}%
\pgfsetbuttcap%
\pgfsetroundjoin%
\pgfsetlinewidth{1.505625pt}%
\definecolor{currentstroke}{rgb}{0.000000,0.000000,0.000000}%
\pgfsetstrokecolor{currentstroke}%
\pgfsetdash{}{0pt}%
\pgfpathmoveto{\pgfqpoint{3.952000in}{3.048000in}}%
\pgfpathlineto{\pgfqpoint{3.616000in}{3.384000in}}%
\pgfpathlineto{\pgfqpoint{3.952000in}{3.384000in}}%
\pgfpathlineto{\pgfqpoint{3.952000in}{3.048000in}}%
\pgfusepath{stroke}%
\end{pgfscope}%
\begin{pgfscope}%
\pgfpathrectangle{\pgfqpoint{1.432000in}{0.528000in}}{\pgfqpoint{3.696000in}{3.696000in}}%
\pgfusepath{clip}%
\pgfsetbuttcap%
\pgfsetroundjoin%
\pgfsetlinewidth{1.505625pt}%
\definecolor{currentstroke}{rgb}{0.000000,0.000000,0.000000}%
\pgfsetstrokecolor{currentstroke}%
\pgfsetdash{}{0pt}%
\pgfpathmoveto{\pgfqpoint{3.616000in}{3.384000in}}%
\pgfpathlineto{\pgfqpoint{3.280000in}{3.720000in}}%
\pgfpathlineto{\pgfqpoint{3.616000in}{3.720000in}}%
\pgfpathlineto{\pgfqpoint{3.616000in}{3.384000in}}%
\pgfusepath{stroke}%
\end{pgfscope}%
\begin{pgfscope}%
\pgfpathrectangle{\pgfqpoint{1.432000in}{0.528000in}}{\pgfqpoint{3.696000in}{3.696000in}}%
\pgfusepath{clip}%
\pgfsetbuttcap%
\pgfsetroundjoin%
\pgfsetlinewidth{1.505625pt}%
\definecolor{currentstroke}{rgb}{0.000000,0.000000,0.000000}%
\pgfsetstrokecolor{currentstroke}%
\pgfsetdash{}{0pt}%
\pgfpathmoveto{\pgfqpoint{3.280000in}{3.720000in}}%
\pgfpathlineto{\pgfqpoint{2.944000in}{4.056000in}}%
\pgfpathlineto{\pgfqpoint{3.280000in}{4.056000in}}%
\pgfpathlineto{\pgfqpoint{3.280000in}{3.720000in}}%
\pgfusepath{stroke}%
\end{pgfscope}%
\begin{pgfscope}%
\pgfpathrectangle{\pgfqpoint{1.432000in}{0.528000in}}{\pgfqpoint{3.696000in}{3.696000in}}%
\pgfusepath{clip}%
\pgfsetbuttcap%
\pgfsetroundjoin%
\pgfsetlinewidth{1.505625pt}%
\definecolor{currentstroke}{rgb}{0.000000,0.000000,0.000000}%
\pgfsetstrokecolor{currentstroke}%
\pgfsetdash{}{0pt}%
\pgfpathmoveto{\pgfqpoint{4.624000in}{2.376000in}}%
\pgfpathlineto{\pgfqpoint{4.960000in}{2.376000in}}%
\pgfpathlineto{\pgfqpoint{4.624000in}{2.712000in}}%
\pgfpathlineto{\pgfqpoint{4.624000in}{2.376000in}}%
\pgfusepath{stroke}%
\end{pgfscope}%
\begin{pgfscope}%
\pgfpathrectangle{\pgfqpoint{1.432000in}{0.528000in}}{\pgfqpoint{3.696000in}{3.696000in}}%
\pgfusepath{clip}%
\pgfsetbuttcap%
\pgfsetroundjoin%
\pgfsetlinewidth{1.505625pt}%
\definecolor{currentstroke}{rgb}{0.000000,0.000000,0.000000}%
\pgfsetstrokecolor{currentstroke}%
\pgfsetdash{}{0pt}%
\pgfpathmoveto{\pgfqpoint{4.288000in}{2.712000in}}%
\pgfpathlineto{\pgfqpoint{4.624000in}{2.712000in}}%
\pgfpathlineto{\pgfqpoint{4.288000in}{3.048000in}}%
\pgfpathlineto{\pgfqpoint{4.288000in}{2.712000in}}%
\pgfusepath{stroke}%
\end{pgfscope}%
\begin{pgfscope}%
\pgfpathrectangle{\pgfqpoint{1.432000in}{0.528000in}}{\pgfqpoint{3.696000in}{3.696000in}}%
\pgfusepath{clip}%
\pgfsetbuttcap%
\pgfsetroundjoin%
\pgfsetlinewidth{1.505625pt}%
\definecolor{currentstroke}{rgb}{0.000000,0.000000,0.000000}%
\pgfsetstrokecolor{currentstroke}%
\pgfsetdash{}{0pt}%
\pgfpathmoveto{\pgfqpoint{3.952000in}{3.048000in}}%
\pgfpathlineto{\pgfqpoint{4.288000in}{3.048000in}}%
\pgfpathlineto{\pgfqpoint{3.952000in}{3.384000in}}%
\pgfpathlineto{\pgfqpoint{3.952000in}{3.048000in}}%
\pgfusepath{stroke}%
\end{pgfscope}%
\begin{pgfscope}%
\pgfpathrectangle{\pgfqpoint{1.432000in}{0.528000in}}{\pgfqpoint{3.696000in}{3.696000in}}%
\pgfusepath{clip}%
\pgfsetbuttcap%
\pgfsetroundjoin%
\pgfsetlinewidth{1.505625pt}%
\definecolor{currentstroke}{rgb}{0.000000,0.000000,0.000000}%
\pgfsetstrokecolor{currentstroke}%
\pgfsetdash{}{0pt}%
\pgfpathmoveto{\pgfqpoint{3.616000in}{3.384000in}}%
\pgfpathlineto{\pgfqpoint{3.952000in}{3.384000in}}%
\pgfpathlineto{\pgfqpoint{3.616000in}{3.720000in}}%
\pgfpathlineto{\pgfqpoint{3.616000in}{3.384000in}}%
\pgfusepath{stroke}%
\end{pgfscope}%
\begin{pgfscope}%
\pgfpathrectangle{\pgfqpoint{1.432000in}{0.528000in}}{\pgfqpoint{3.696000in}{3.696000in}}%
\pgfusepath{clip}%
\pgfsetbuttcap%
\pgfsetroundjoin%
\pgfsetlinewidth{1.505625pt}%
\definecolor{currentstroke}{rgb}{0.000000,0.000000,0.000000}%
\pgfsetstrokecolor{currentstroke}%
\pgfsetdash{}{0pt}%
\pgfpathmoveto{\pgfqpoint{3.280000in}{3.720000in}}%
\pgfpathlineto{\pgfqpoint{3.616000in}{3.720000in}}%
\pgfpathlineto{\pgfqpoint{3.280000in}{4.056000in}}%
\pgfpathlineto{\pgfqpoint{3.280000in}{3.720000in}}%
\pgfusepath{stroke}%
\end{pgfscope}%
\begin{pgfscope}%
\pgfpathrectangle{\pgfqpoint{1.432000in}{0.528000in}}{\pgfqpoint{3.696000in}{3.696000in}}%
\pgfusepath{clip}%
\pgfsetbuttcap%
\pgfsetroundjoin%
\pgfsetlinewidth{1.505625pt}%
\definecolor{currentstroke}{rgb}{0.000000,0.000000,0.000000}%
\pgfsetstrokecolor{currentstroke}%
\pgfsetdash{}{0pt}%
\pgfpathmoveto{\pgfqpoint{4.960000in}{2.376000in}}%
\pgfpathlineto{\pgfqpoint{4.624000in}{2.712000in}}%
\pgfpathlineto{\pgfqpoint{4.960000in}{2.712000in}}%
\pgfpathlineto{\pgfqpoint{4.960000in}{2.376000in}}%
\pgfusepath{stroke}%
\end{pgfscope}%
\begin{pgfscope}%
\pgfpathrectangle{\pgfqpoint{1.432000in}{0.528000in}}{\pgfqpoint{3.696000in}{3.696000in}}%
\pgfusepath{clip}%
\pgfsetbuttcap%
\pgfsetroundjoin%
\pgfsetlinewidth{1.505625pt}%
\definecolor{currentstroke}{rgb}{0.000000,0.000000,0.000000}%
\pgfsetstrokecolor{currentstroke}%
\pgfsetdash{}{0pt}%
\pgfpathmoveto{\pgfqpoint{4.624000in}{2.712000in}}%
\pgfpathlineto{\pgfqpoint{4.288000in}{3.048000in}}%
\pgfpathlineto{\pgfqpoint{4.624000in}{3.048000in}}%
\pgfpathlineto{\pgfqpoint{4.624000in}{2.712000in}}%
\pgfusepath{stroke}%
\end{pgfscope}%
\begin{pgfscope}%
\pgfpathrectangle{\pgfqpoint{1.432000in}{0.528000in}}{\pgfqpoint{3.696000in}{3.696000in}}%
\pgfusepath{clip}%
\pgfsetbuttcap%
\pgfsetroundjoin%
\pgfsetlinewidth{1.505625pt}%
\definecolor{currentstroke}{rgb}{0.000000,0.000000,0.000000}%
\pgfsetstrokecolor{currentstroke}%
\pgfsetdash{}{0pt}%
\pgfpathmoveto{\pgfqpoint{4.288000in}{3.048000in}}%
\pgfpathlineto{\pgfqpoint{3.952000in}{3.384000in}}%
\pgfpathlineto{\pgfqpoint{4.288000in}{3.384000in}}%
\pgfpathlineto{\pgfqpoint{4.288000in}{3.048000in}}%
\pgfusepath{stroke}%
\end{pgfscope}%
\begin{pgfscope}%
\pgfpathrectangle{\pgfqpoint{1.432000in}{0.528000in}}{\pgfqpoint{3.696000in}{3.696000in}}%
\pgfusepath{clip}%
\pgfsetbuttcap%
\pgfsetroundjoin%
\pgfsetlinewidth{1.505625pt}%
\definecolor{currentstroke}{rgb}{0.000000,0.000000,0.000000}%
\pgfsetstrokecolor{currentstroke}%
\pgfsetdash{}{0pt}%
\pgfpathmoveto{\pgfqpoint{3.952000in}{3.384000in}}%
\pgfpathlineto{\pgfqpoint{3.616000in}{3.720000in}}%
\pgfpathlineto{\pgfqpoint{3.952000in}{3.720000in}}%
\pgfpathlineto{\pgfqpoint{3.952000in}{3.384000in}}%
\pgfusepath{stroke}%
\end{pgfscope}%
\begin{pgfscope}%
\pgfpathrectangle{\pgfqpoint{1.432000in}{0.528000in}}{\pgfqpoint{3.696000in}{3.696000in}}%
\pgfusepath{clip}%
\pgfsetbuttcap%
\pgfsetroundjoin%
\pgfsetlinewidth{1.505625pt}%
\definecolor{currentstroke}{rgb}{0.000000,0.000000,0.000000}%
\pgfsetstrokecolor{currentstroke}%
\pgfsetdash{}{0pt}%
\pgfpathmoveto{\pgfqpoint{3.616000in}{3.720000in}}%
\pgfpathlineto{\pgfqpoint{3.280000in}{4.056000in}}%
\pgfpathlineto{\pgfqpoint{3.616000in}{4.056000in}}%
\pgfpathlineto{\pgfqpoint{3.616000in}{3.720000in}}%
\pgfusepath{stroke}%
\end{pgfscope}%
\begin{pgfscope}%
\pgfpathrectangle{\pgfqpoint{1.432000in}{0.528000in}}{\pgfqpoint{3.696000in}{3.696000in}}%
\pgfusepath{clip}%
\pgfsetbuttcap%
\pgfsetroundjoin%
\pgfsetlinewidth{1.505625pt}%
\definecolor{currentstroke}{rgb}{0.000000,0.000000,0.000000}%
\pgfsetstrokecolor{currentstroke}%
\pgfsetdash{}{0pt}%
\pgfpathmoveto{\pgfqpoint{4.624000in}{2.712000in}}%
\pgfpathlineto{\pgfqpoint{4.960000in}{2.712000in}}%
\pgfpathlineto{\pgfqpoint{4.624000in}{3.048000in}}%
\pgfpathlineto{\pgfqpoint{4.624000in}{2.712000in}}%
\pgfusepath{stroke}%
\end{pgfscope}%
\begin{pgfscope}%
\pgfpathrectangle{\pgfqpoint{1.432000in}{0.528000in}}{\pgfqpoint{3.696000in}{3.696000in}}%
\pgfusepath{clip}%
\pgfsetbuttcap%
\pgfsetroundjoin%
\pgfsetlinewidth{1.505625pt}%
\definecolor{currentstroke}{rgb}{0.000000,0.000000,0.000000}%
\pgfsetstrokecolor{currentstroke}%
\pgfsetdash{}{0pt}%
\pgfpathmoveto{\pgfqpoint{4.288000in}{3.048000in}}%
\pgfpathlineto{\pgfqpoint{4.624000in}{3.048000in}}%
\pgfpathlineto{\pgfqpoint{4.288000in}{3.384000in}}%
\pgfpathlineto{\pgfqpoint{4.288000in}{3.048000in}}%
\pgfusepath{stroke}%
\end{pgfscope}%
\begin{pgfscope}%
\pgfpathrectangle{\pgfqpoint{1.432000in}{0.528000in}}{\pgfqpoint{3.696000in}{3.696000in}}%
\pgfusepath{clip}%
\pgfsetbuttcap%
\pgfsetroundjoin%
\pgfsetlinewidth{1.505625pt}%
\definecolor{currentstroke}{rgb}{0.000000,0.000000,0.000000}%
\pgfsetstrokecolor{currentstroke}%
\pgfsetdash{}{0pt}%
\pgfpathmoveto{\pgfqpoint{3.952000in}{3.384000in}}%
\pgfpathlineto{\pgfqpoint{4.288000in}{3.384000in}}%
\pgfpathlineto{\pgfqpoint{3.952000in}{3.720000in}}%
\pgfpathlineto{\pgfqpoint{3.952000in}{3.384000in}}%
\pgfusepath{stroke}%
\end{pgfscope}%
\begin{pgfscope}%
\pgfpathrectangle{\pgfqpoint{1.432000in}{0.528000in}}{\pgfqpoint{3.696000in}{3.696000in}}%
\pgfusepath{clip}%
\pgfsetbuttcap%
\pgfsetroundjoin%
\pgfsetlinewidth{1.505625pt}%
\definecolor{currentstroke}{rgb}{0.000000,0.000000,0.000000}%
\pgfsetstrokecolor{currentstroke}%
\pgfsetdash{}{0pt}%
\pgfpathmoveto{\pgfqpoint{3.616000in}{3.720000in}}%
\pgfpathlineto{\pgfqpoint{3.952000in}{3.720000in}}%
\pgfpathlineto{\pgfqpoint{3.616000in}{4.056000in}}%
\pgfpathlineto{\pgfqpoint{3.616000in}{3.720000in}}%
\pgfusepath{stroke}%
\end{pgfscope}%
\begin{pgfscope}%
\pgfpathrectangle{\pgfqpoint{1.432000in}{0.528000in}}{\pgfqpoint{3.696000in}{3.696000in}}%
\pgfusepath{clip}%
\pgfsetbuttcap%
\pgfsetroundjoin%
\pgfsetlinewidth{1.505625pt}%
\definecolor{currentstroke}{rgb}{0.000000,0.000000,0.000000}%
\pgfsetstrokecolor{currentstroke}%
\pgfsetdash{}{0pt}%
\pgfpathmoveto{\pgfqpoint{4.960000in}{2.712000in}}%
\pgfpathlineto{\pgfqpoint{4.624000in}{3.048000in}}%
\pgfpathlineto{\pgfqpoint{4.960000in}{3.048000in}}%
\pgfpathlineto{\pgfqpoint{4.960000in}{2.712000in}}%
\pgfusepath{stroke}%
\end{pgfscope}%
\begin{pgfscope}%
\pgfpathrectangle{\pgfqpoint{1.432000in}{0.528000in}}{\pgfqpoint{3.696000in}{3.696000in}}%
\pgfusepath{clip}%
\pgfsetbuttcap%
\pgfsetroundjoin%
\pgfsetlinewidth{1.505625pt}%
\definecolor{currentstroke}{rgb}{0.000000,0.000000,0.000000}%
\pgfsetstrokecolor{currentstroke}%
\pgfsetdash{}{0pt}%
\pgfpathmoveto{\pgfqpoint{4.624000in}{3.048000in}}%
\pgfpathlineto{\pgfqpoint{4.288000in}{3.384000in}}%
\pgfpathlineto{\pgfqpoint{4.624000in}{3.384000in}}%
\pgfpathlineto{\pgfqpoint{4.624000in}{3.048000in}}%
\pgfusepath{stroke}%
\end{pgfscope}%
\begin{pgfscope}%
\pgfpathrectangle{\pgfqpoint{1.432000in}{0.528000in}}{\pgfqpoint{3.696000in}{3.696000in}}%
\pgfusepath{clip}%
\pgfsetbuttcap%
\pgfsetroundjoin%
\pgfsetlinewidth{1.505625pt}%
\definecolor{currentstroke}{rgb}{0.000000,0.000000,0.000000}%
\pgfsetstrokecolor{currentstroke}%
\pgfsetdash{}{0pt}%
\pgfpathmoveto{\pgfqpoint{4.288000in}{3.384000in}}%
\pgfpathlineto{\pgfqpoint{3.952000in}{3.720000in}}%
\pgfpathlineto{\pgfqpoint{4.288000in}{3.720000in}}%
\pgfpathlineto{\pgfqpoint{4.288000in}{3.384000in}}%
\pgfusepath{stroke}%
\end{pgfscope}%
\begin{pgfscope}%
\pgfpathrectangle{\pgfqpoint{1.432000in}{0.528000in}}{\pgfqpoint{3.696000in}{3.696000in}}%
\pgfusepath{clip}%
\pgfsetbuttcap%
\pgfsetroundjoin%
\pgfsetlinewidth{1.505625pt}%
\definecolor{currentstroke}{rgb}{0.000000,0.000000,0.000000}%
\pgfsetstrokecolor{currentstroke}%
\pgfsetdash{}{0pt}%
\pgfpathmoveto{\pgfqpoint{3.952000in}{3.720000in}}%
\pgfpathlineto{\pgfqpoint{3.616000in}{4.056000in}}%
\pgfpathlineto{\pgfqpoint{3.952000in}{4.056000in}}%
\pgfpathlineto{\pgfqpoint{3.952000in}{3.720000in}}%
\pgfusepath{stroke}%
\end{pgfscope}%
\begin{pgfscope}%
\pgfpathrectangle{\pgfqpoint{1.432000in}{0.528000in}}{\pgfqpoint{3.696000in}{3.696000in}}%
\pgfusepath{clip}%
\pgfsetbuttcap%
\pgfsetroundjoin%
\pgfsetlinewidth{1.505625pt}%
\definecolor{currentstroke}{rgb}{0.000000,0.000000,0.000000}%
\pgfsetstrokecolor{currentstroke}%
\pgfsetdash{}{0pt}%
\pgfpathmoveto{\pgfqpoint{4.624000in}{3.048000in}}%
\pgfpathlineto{\pgfqpoint{4.960000in}{3.048000in}}%
\pgfpathlineto{\pgfqpoint{4.624000in}{3.384000in}}%
\pgfpathlineto{\pgfqpoint{4.624000in}{3.048000in}}%
\pgfusepath{stroke}%
\end{pgfscope}%
\begin{pgfscope}%
\pgfpathrectangle{\pgfqpoint{1.432000in}{0.528000in}}{\pgfqpoint{3.696000in}{3.696000in}}%
\pgfusepath{clip}%
\pgfsetbuttcap%
\pgfsetroundjoin%
\pgfsetlinewidth{1.505625pt}%
\definecolor{currentstroke}{rgb}{0.000000,0.000000,0.000000}%
\pgfsetstrokecolor{currentstroke}%
\pgfsetdash{}{0pt}%
\pgfpathmoveto{\pgfqpoint{4.288000in}{3.384000in}}%
\pgfpathlineto{\pgfqpoint{4.624000in}{3.384000in}}%
\pgfpathlineto{\pgfqpoint{4.288000in}{3.720000in}}%
\pgfpathlineto{\pgfqpoint{4.288000in}{3.384000in}}%
\pgfusepath{stroke}%
\end{pgfscope}%
\begin{pgfscope}%
\pgfpathrectangle{\pgfqpoint{1.432000in}{0.528000in}}{\pgfqpoint{3.696000in}{3.696000in}}%
\pgfusepath{clip}%
\pgfsetbuttcap%
\pgfsetroundjoin%
\pgfsetlinewidth{1.505625pt}%
\definecolor{currentstroke}{rgb}{0.000000,0.000000,0.000000}%
\pgfsetstrokecolor{currentstroke}%
\pgfsetdash{}{0pt}%
\pgfpathmoveto{\pgfqpoint{3.952000in}{3.720000in}}%
\pgfpathlineto{\pgfqpoint{4.288000in}{3.720000in}}%
\pgfpathlineto{\pgfqpoint{3.952000in}{4.056000in}}%
\pgfpathlineto{\pgfqpoint{3.952000in}{3.720000in}}%
\pgfusepath{stroke}%
\end{pgfscope}%
\begin{pgfscope}%
\pgfpathrectangle{\pgfqpoint{1.432000in}{0.528000in}}{\pgfqpoint{3.696000in}{3.696000in}}%
\pgfusepath{clip}%
\pgfsetbuttcap%
\pgfsetroundjoin%
\pgfsetlinewidth{1.505625pt}%
\definecolor{currentstroke}{rgb}{0.000000,0.000000,0.000000}%
\pgfsetstrokecolor{currentstroke}%
\pgfsetdash{}{0pt}%
\pgfpathmoveto{\pgfqpoint{4.960000in}{3.048000in}}%
\pgfpathlineto{\pgfqpoint{4.624000in}{3.384000in}}%
\pgfpathlineto{\pgfqpoint{4.960000in}{3.384000in}}%
\pgfpathlineto{\pgfqpoint{4.960000in}{3.048000in}}%
\pgfusepath{stroke}%
\end{pgfscope}%
\begin{pgfscope}%
\pgfpathrectangle{\pgfqpoint{1.432000in}{0.528000in}}{\pgfqpoint{3.696000in}{3.696000in}}%
\pgfusepath{clip}%
\pgfsetbuttcap%
\pgfsetroundjoin%
\pgfsetlinewidth{1.505625pt}%
\definecolor{currentstroke}{rgb}{0.000000,0.000000,0.000000}%
\pgfsetstrokecolor{currentstroke}%
\pgfsetdash{}{0pt}%
\pgfpathmoveto{\pgfqpoint{4.624000in}{3.384000in}}%
\pgfpathlineto{\pgfqpoint{4.288000in}{3.720000in}}%
\pgfpathlineto{\pgfqpoint{4.624000in}{3.720000in}}%
\pgfpathlineto{\pgfqpoint{4.624000in}{3.384000in}}%
\pgfusepath{stroke}%
\end{pgfscope}%
\begin{pgfscope}%
\pgfpathrectangle{\pgfqpoint{1.432000in}{0.528000in}}{\pgfqpoint{3.696000in}{3.696000in}}%
\pgfusepath{clip}%
\pgfsetbuttcap%
\pgfsetroundjoin%
\pgfsetlinewidth{1.505625pt}%
\definecolor{currentstroke}{rgb}{0.000000,0.000000,0.000000}%
\pgfsetstrokecolor{currentstroke}%
\pgfsetdash{}{0pt}%
\pgfpathmoveto{\pgfqpoint{4.288000in}{3.720000in}}%
\pgfpathlineto{\pgfqpoint{3.952000in}{4.056000in}}%
\pgfpathlineto{\pgfqpoint{4.288000in}{4.056000in}}%
\pgfpathlineto{\pgfqpoint{4.288000in}{3.720000in}}%
\pgfusepath{stroke}%
\end{pgfscope}%
\begin{pgfscope}%
\pgfpathrectangle{\pgfqpoint{1.432000in}{0.528000in}}{\pgfqpoint{3.696000in}{3.696000in}}%
\pgfusepath{clip}%
\pgfsetbuttcap%
\pgfsetroundjoin%
\pgfsetlinewidth{1.505625pt}%
\definecolor{currentstroke}{rgb}{0.000000,0.000000,0.000000}%
\pgfsetstrokecolor{currentstroke}%
\pgfsetdash{}{0pt}%
\pgfpathmoveto{\pgfqpoint{4.624000in}{3.384000in}}%
\pgfpathlineto{\pgfqpoint{4.960000in}{3.384000in}}%
\pgfpathlineto{\pgfqpoint{4.624000in}{3.720000in}}%
\pgfpathlineto{\pgfqpoint{4.624000in}{3.384000in}}%
\pgfusepath{stroke}%
\end{pgfscope}%
\begin{pgfscope}%
\pgfpathrectangle{\pgfqpoint{1.432000in}{0.528000in}}{\pgfqpoint{3.696000in}{3.696000in}}%
\pgfusepath{clip}%
\pgfsetbuttcap%
\pgfsetroundjoin%
\pgfsetlinewidth{1.505625pt}%
\definecolor{currentstroke}{rgb}{0.000000,0.000000,0.000000}%
\pgfsetstrokecolor{currentstroke}%
\pgfsetdash{}{0pt}%
\pgfpathmoveto{\pgfqpoint{4.288000in}{3.720000in}}%
\pgfpathlineto{\pgfqpoint{4.624000in}{3.720000in}}%
\pgfpathlineto{\pgfqpoint{4.288000in}{4.056000in}}%
\pgfpathlineto{\pgfqpoint{4.288000in}{3.720000in}}%
\pgfusepath{stroke}%
\end{pgfscope}%
\begin{pgfscope}%
\pgfpathrectangle{\pgfqpoint{1.432000in}{0.528000in}}{\pgfqpoint{3.696000in}{3.696000in}}%
\pgfusepath{clip}%
\pgfsetbuttcap%
\pgfsetroundjoin%
\pgfsetlinewidth{1.505625pt}%
\definecolor{currentstroke}{rgb}{0.000000,0.000000,0.000000}%
\pgfsetstrokecolor{currentstroke}%
\pgfsetdash{}{0pt}%
\pgfpathmoveto{\pgfqpoint{4.960000in}{3.384000in}}%
\pgfpathlineto{\pgfqpoint{4.624000in}{3.720000in}}%
\pgfpathlineto{\pgfqpoint{4.960000in}{3.720000in}}%
\pgfpathlineto{\pgfqpoint{4.960000in}{3.384000in}}%
\pgfusepath{stroke}%
\end{pgfscope}%
\begin{pgfscope}%
\pgfpathrectangle{\pgfqpoint{1.432000in}{0.528000in}}{\pgfqpoint{3.696000in}{3.696000in}}%
\pgfusepath{clip}%
\pgfsetbuttcap%
\pgfsetroundjoin%
\pgfsetlinewidth{1.505625pt}%
\definecolor{currentstroke}{rgb}{0.000000,0.000000,0.000000}%
\pgfsetstrokecolor{currentstroke}%
\pgfsetdash{}{0pt}%
\pgfpathmoveto{\pgfqpoint{4.624000in}{3.720000in}}%
\pgfpathlineto{\pgfqpoint{4.288000in}{4.056000in}}%
\pgfpathlineto{\pgfqpoint{4.624000in}{4.056000in}}%
\pgfpathlineto{\pgfqpoint{4.624000in}{3.720000in}}%
\pgfusepath{stroke}%
\end{pgfscope}%
\begin{pgfscope}%
\pgfpathrectangle{\pgfqpoint{1.432000in}{0.528000in}}{\pgfqpoint{3.696000in}{3.696000in}}%
\pgfusepath{clip}%
\pgfsetbuttcap%
\pgfsetroundjoin%
\pgfsetlinewidth{1.505625pt}%
\definecolor{currentstroke}{rgb}{0.000000,0.000000,0.000000}%
\pgfsetstrokecolor{currentstroke}%
\pgfsetdash{}{0pt}%
\pgfpathmoveto{\pgfqpoint{4.624000in}{3.720000in}}%
\pgfpathlineto{\pgfqpoint{4.960000in}{3.720000in}}%
\pgfpathlineto{\pgfqpoint{4.624000in}{4.056000in}}%
\pgfpathlineto{\pgfqpoint{4.624000in}{3.720000in}}%
\pgfusepath{stroke}%
\end{pgfscope}%
\begin{pgfscope}%
\pgfpathrectangle{\pgfqpoint{1.432000in}{0.528000in}}{\pgfqpoint{3.696000in}{3.696000in}}%
\pgfusepath{clip}%
\pgfsetbuttcap%
\pgfsetroundjoin%
\pgfsetlinewidth{1.505625pt}%
\definecolor{currentstroke}{rgb}{0.000000,0.000000,0.000000}%
\pgfsetstrokecolor{currentstroke}%
\pgfsetdash{}{0pt}%
\pgfpathmoveto{\pgfqpoint{4.960000in}{3.720000in}}%
\pgfpathlineto{\pgfqpoint{4.624000in}{4.056000in}}%
\pgfpathlineto{\pgfqpoint{4.960000in}{4.056000in}}%
\pgfpathlineto{\pgfqpoint{4.960000in}{3.720000in}}%
\pgfusepath{stroke}%
\end{pgfscope}%
\begin{pgfscope}%
\pgfsetrectcap%
\pgfsetmiterjoin%
\pgfsetlinewidth{0.000000pt}%
\definecolor{currentstroke}{rgb}{0.000000,0.000000,0.000000}%
\pgfsetstrokecolor{currentstroke}%
\pgfsetstrokeopacity{0.000000}%
\pgfsetdash{}{0pt}%
\pgfpathmoveto{\pgfqpoint{1.432000in}{0.528000in}}%
\pgfpathlineto{\pgfqpoint{1.432000in}{4.224000in}}%
\pgfusepath{}%
\end{pgfscope}%
\begin{pgfscope}%
\pgfsetrectcap%
\pgfsetmiterjoin%
\pgfsetlinewidth{0.000000pt}%
\definecolor{currentstroke}{rgb}{0.000000,0.000000,0.000000}%
\pgfsetstrokecolor{currentstroke}%
\pgfsetstrokeopacity{0.000000}%
\pgfsetdash{}{0pt}%
\pgfpathmoveto{\pgfqpoint{5.128000in}{0.528000in}}%
\pgfpathlineto{\pgfqpoint{5.128000in}{4.224000in}}%
\pgfusepath{}%
\end{pgfscope}%
\begin{pgfscope}%
\pgfsetrectcap%
\pgfsetmiterjoin%
\pgfsetlinewidth{0.000000pt}%
\definecolor{currentstroke}{rgb}{0.000000,0.000000,0.000000}%
\pgfsetstrokecolor{currentstroke}%
\pgfsetstrokeopacity{0.000000}%
\pgfsetdash{}{0pt}%
\pgfpathmoveto{\pgfqpoint{1.432000in}{0.528000in}}%
\pgfpathlineto{\pgfqpoint{5.128000in}{0.528000in}}%
\pgfusepath{}%
\end{pgfscope}%
\begin{pgfscope}%
\pgfsetrectcap%
\pgfsetmiterjoin%
\pgfsetlinewidth{0.000000pt}%
\definecolor{currentstroke}{rgb}{0.000000,0.000000,0.000000}%
\pgfsetstrokecolor{currentstroke}%
\pgfsetstrokeopacity{0.000000}%
\pgfsetdash{}{0pt}%
\pgfpathmoveto{\pgfqpoint{1.432000in}{4.224000in}}%
\pgfpathlineto{\pgfqpoint{5.128000in}{4.224000in}}%
\pgfusepath{}%
\end{pgfscope}%
\end{pgfpicture}%
\makeatother%
\endgroup%

  \caption{A sample mesh, similar to those used in all the computations in this thesis.\label{fig:grid}}
  \end{figure}

\paragraph{Numerical setup}
All finite-element calculations were carried out using the Firedrake software library \cite{RaHaMiLaLuMcBeMaKe:16,LuVaRaBeRaHaKe:15}, which uses PETSc to perform its linear solves \cite{BaAbAsBrBrBuDaEiGrKaKnMaMcMiMuRuSaSmZaZhZh:18,BaGrMcSm:97,DaPaKlCo:11,BaAbAdBrBrBuDaDeEiGrKaKaKnMaMcMiMuRuSaSmZaZh:19} and Chaco \cite{HeLe:95} to perform graph partitions. PETSc uses the MUMPS \cite{AmDuLEKo:01,AmGuLEPr:06} solver to perform $LU$ factorisations and direct solves. When GMRES was used, the stopping criterion was a relative error (relative to the 2-norm of the right-hand side) of $10^{-5}$ or an absolute error of $10^{-50}.$ To generated QMC points, we made use of Dirk Nuyen's `Magic Point Shop' code \cite{Nu,KuNu:16}, which uses a base-2 lattice sequence with generating vector from \cite{CoKuNu:06}. Many of the computations were carried out on the Balena High Performance Computing (HPC) Service at the University of Bath.
