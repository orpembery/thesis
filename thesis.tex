\documentclass[draft]{book}

\ifdefined\revisionversion
\def\baselinestretch{2} % Modified from Higham, Second Edition, p. 109
\fi


\usepackage{amsthm}
\usepackage[show]{ed}
\usepackage{mathtools}
%\usepackage{amsthm} % Needed?
\usepackage{mleftright}
\ifdefined\supervisorversion

\else
        \usepackage{showkeys}
    \fi
    % Based on https://tex.stackexchange.com/a/1495

\usepackage{amssymb}
\usepackage{enumitem}
\usepackage[margin=2cm]{geometry}

\usepackage{thmtools} 
\usepackage{booktabs}
\usepackage{color} 
\usepackage{tikz} 
\usepgflibrary{shapes.misc}
\usetikzlibrary{positioning}
\usetikzlibrary{matrix} 
\usetikzlibrary{patterns}
\usetikzlibrary{calc}
\usepackage{bbm} % Allows blackboard bold 1
\usepackage{cases} % Gives numcases environment
\usepackage{longtable}
\usepackage{tabu}
\usepackage{makecell}
\usepackage{cellspace}
%\usepackage{adjustbox}
\usepackage{threeparttable}
\usepackage{pdflscape}
% The following two packages should come last
\usepackage[hidelinks]{hyperref}
% Found out about hidelinks in https://tex.stackexchange.com/a/50754
\usepackage[capitalise]{cleveref}



%%%%%%%%%%%%%% All the below macros have been taken from Anton Geraschenko's files at stacky.net %%%%%%%%%%%%%%%

%	Bold Letters
\renewcommand{\AA}{\mathbb A}
\newcommand{\BB}{\mathbb B}
\newcommand{\CC}{\mathbb C}
\newcommand{\DD}{\mathbb D}
\newcommand{\EE}{\mathbb E}
\newcommand{\FF}{\mathbb F}
\newcommand{\GG}{\mathbb G}
\newcommand{\HH}{\mathbb H}
\newcommand{\II}{\mathbb I}
\newcommand{\JJ}{\mathbb J}
\newcommand{\KK}{\mathbb K}
\newcommand{\LL}{\mathbb L}
\newcommand{\MM}{\mathbb M}
\newcommand{\NN}{\mathbb N}
\newcommand{\OO}{\mathbb O}
\newcommand{\PP}{\mathbb P}
\newcommand{\QQ}{\mathbb Q}
\newcommand{\RR}{\mathbb R}
\renewcommand{\SS}{\mathbb S}
\newcommand{\TT}{\mathbb T}
\newcommand{\UU}{\mathbb U}
\newcommand{\VV}{\mathbb V}
\newcommand{\WW}{\mathbb W}
\newcommand{\XX}{\mathbb X}
\newcommand{\YY}{\mathbb Y}
\newcommand{\ZZ}{\mathbb Z}

%Cal Letters
\newcommand{\cA}{\mathcal A}
\newcommand{\cB}{\mathcal B}
\newcommand{\cC}{\mathcal C}
\newcommand{\cD}{\mathcal D}
\newcommand{\cE}{\mathcal E}
\newcommand{\cF}{\mathcal F}
\newcommand{\cG}{\mathcal G}
\newcommand{\cH}{\mathcal H}
\newcommand{\cI}{\mathcal I}
\newcommand{\cJ}{\mathcal J}
\newcommand{\cK}{\mathcal K}
\newcommand{\cL}{\mathcal L}
\newcommand{\cM}{\mathcal M}
\newcommand{\cN}{\mathcal N}
\newcommand{\cO}{\mathcal O}
\newcommand{\cP}{\mathcal P}
\newcommand{\cQ}{\mathcal Q}
\newcommand{\cR}{\mathcal R}
\newcommand{\cS}{\mathcal S}
\newcommand{\cT}{\mathcal T}
\newcommand{\cU}{\mathcal U}
\newcommand{\cV}{\mathcal V}
\newcommand{\cW}{\mathcal W}
\newcommand{\cX}{\mathcal X}
\newcommand{\cY}{\mathcal Y}
\newcommand{\cZ}{\mathcal Z}

% Bold
%\newcommand{\bdT}{\mathbf T}
%\newcommand{\bda}{\mathbf a}
%\newcommand{\bdb}{\mathbf b}
%\newcommand{\bdc}{\mathbf c}
%\newcommand{\bdd}{\mathbf d}
%\newcommand{\bde}{\mathbf e}
%\newcommand{\bdf}{\mathbf f}
%\newcommand{\bdg}{\mathbf g}
%\newcommand{\bdh}{\mathbf h}
%\newcommand{\bdi}{\mathbf i}
%\newcommand{\bdj}{\mathbf j}
%\newcommand{\bdk}{\mathbf k}
%\newcommand{\bdl}{\mathbf l}
%\newcommand{\bdm}{\mathbf m}
%\newcommand{\bdn}{\mathbf n}
%\newcommand{\bdp}{\mathbf p}
%\newcommand{\bdq}{\mathbf q}
%\newcommand{\bdr}{\mathbf r}
%\newcommand{\bds}{\mathbf s}
%\newcommand{\bdt}{\mathbf t}
%\newcommand{\bdu}{\mathbf u}
%\newcommand{\bdv}{\mathbf v}
%\newcommand{\bdw}{\mathbf w}
%\newcommand{\bdx}{\mathbf x}
%\newcommand{\bdy}{\mathbf y}
%\newcommand{\bdz}{\mathbf z}
%\newcommand{\bdo}{\mathbf 0}

% Nicked from Euan Spence
\newcommand{\beq}{\begin{equation}}
\newcommand{\eeq}{\end{equation}}
\newcommand{\beqs}{\begin{equation*}}
\newcommand{\eeqs}{\end{equation*}}
\newcommand{\bit}{\begin{itemize}}
\newcommand{\eit}{\end{itemize}}
\newcommand{\ben}{\begin{enumerate}}
\newcommand{\een}{\end{enumerate}}
\newcommand{\bal}{\begin{align}}
\newcommand{\eal}{\end{align}}
\newcommand{\bals}{\begin{align*}}
\newcommand{\eals}{\end{align*}}
\newcommand{\bse}{\begin{subequations}}
\newcommand{\ese}{\end{subequations}}
\newcommand{\bpr}{\begin{proposition}}
\newcommand{\epr}{\end{proposition}}
\newcommand{\bre}{\begin{remark}}
\newcommand{\ere}{\end{remark}}
\newcommand{\bpf}{\begin{proof}}
\newcommand{\epf}{\end{proof}}
\newcommand{\ble}{\begin{lemma}}
\newcommand{\ele}{\end{lemma}}
\newcommand{\bco}{\begin{corollary}}
\newcommand{\eco}{\end{corollary}}
\newcommand{\bex}{\begin{example}}
\newcommand{\eex}{\end{example}}
\newcommand{\bth}{\begin{theorem}}
\newcommand{\enth}{\end{theorem}}
\newcommand{\bcon}{\begin{condition}}
\newcommand{\econ}{\end{condition}}
\newcommand{\bas}{\begin{assumption}}
\newcommand{\eas}{\end{assumption}}
\newcommand{\bde}{\begin{definition}}
\newcommand{\ede}{\end{definition}}



\newcommand{\ton}{\text{ on }}
\newcommand{\tin}{\text{ in }}
\newcommand{\tfa}{\text{ for all }}
\newcommand{\tfor}{\text{ for }}
\newcommand{\tas}{\text{ as }}
\newcommand{\tand}{\text{ and }}
\newcommand{\tst}{\text{ such that }}
\newcommand{\tif}{\text{ if }}
\newcommand{\tals}{\text{ almost surely}}



%%%% Owen's %%%%
\newcommand{\tfae}{\text{ for almost every }}

\newcommand{\bprob}{\begin{problem}}
\newcommand{\eprob}{\end{problem}}

\ifdefined\supervisorversion
    \newcommand{\optodo}[1]{}
\else
    \newcommand{\optodo}[1]{\ednote{OPTODO: #1}}
    \fi
    % Based on https://tex.stackexchange.com/a/1495
\newcommand{\minispace}{\;\!}

\newcommand{\AmatoI}{\matI{\Amato}}
\newcommand{\Amato}{\Amat_{1}}
\newcommand{\Amat}{\mat{A}}
\newcommand{\mat}[1]{\mathbf{#1}}
\newcommand{\Amatt}{\Amat_{2}}
\newcommand{\CMC}{C_{\mathrm{MC}}}
\newcommand{\CMLMC}{C_{\mathrm{MLMC}}}
\newcommand{\eps}{\varepsilon}
\newcommand{\grad}{\nabla}
\newcommand{\Ad}{A_{\mathrm{det}}}
\newcommand{\nd}{n_{\mathrm{det}}}
\newcommand{\fd}{f_{\mathrm{det}}}
\newcommand{\ud}{u_{\mathrm{det}}}
\newcommand{\Pmat}{\mat{P}}
\newcommand{\matI}[1]{#1^{-1}}
\newcommand{\PmatI}{\matI{\Pmat}}
\newcommand{\butilde}{\widetilde{\bu}}
\newcommand{\AmatI}{\matI{\Amat}}
\newcommand{\Imat}{\mat{I}}
\newcommand{\Amin}{A_{\min}}
\newcommand{\Amax}{A_{\max}}
\newcommand{\EXP}[1]{\EE\mleft[#1\mright]}
\newcommand{\ddPPomega}{\dd\PP(\omega)}
\newcommand{\dd}{\mathrm{d}}
\newcommand{\bzero}{\mathbf{0}}
\newcommand{\Dm}{D_{-}}
\newcommand{\Dp}{D_{+}}
\newcommand{\de}{\coloneqq}
\newcommand{\RRd}{\RR^{d}}
\newcommand{\Dmclos}{\clos{\Dm}}
\newcommand{\clos}[1]{\overline{#1}}
\newcommand{\GD}{\Gamma_{D}}
\newcommand{\LtDp}{\Lt{\Dp}}
\newcommand{\Lt}[1]{L^{2}\mleft(#1\mright)}
\newcommand{\gD}{g_{D}}
\newcommand{\Hh}[1]{H^{1/2}\mleft(#1\mright)}
\newcommand{\HhGD}{\Hh{\GD}}
\newcommand{\LiDpRRdtd}{\Li{\Dp;\RRdtd}}
\newcommand{\LiDpRR}{\Li{\Dp;\RR}}
\newcommand{\Li}[1]{L^{\infty}\mleft(#1\mright)}
\newcommand{\nmin}{n_{\min}}
\newcommand{\nmax}{n_{\max}}
\newcommand{\RRdtd}{\RR^{d\times d}}
\newcommand{\abs}[1]{\mleft|#1\mright|}
\newcommand{\bxi}{\boldsymbol{\xi}}
\newcommand{\bxibar}{\overline{\bxi}}
\newcommand{\CCd}{\CC^{d}}
\newcommand{\HolocDp}{\Holoc{\Dp}}
\newcommand{\Holoc}[1]{H^{1}_{\mathrm{loc}}\mleft(#1\mright)}
\newcommand{\defn}[1]{\emph{#1}}
\newcommand{\trGD}{\trace_{D}}
\newcommand{\trace}{\gamma}
\newcommand{\dudr}{\frac{\partial u}{\partial r}}
\newcommand{\bxhat}{\hat{\bx}}
\newcommand{\dn}{\partial_{\nu}}
\newcommand{\gI}{g_{I}}
\newcommand{\Dtilde}{\widetilde{D}}
\newcommand{\compcont}{\subset\subset}
\newcommand{\GI}{\Gamma_{I}}
\newcommand{\LtD}{\Lt{D}}
\newcommand{\LtGI}{\Lt{\GI}}
\newcommand{\LiDRR}{\Li{D;\RR}}
\DeclareMathOperator{\supp}{supp}
\newcommand{\LiDRRdtd}{\Li{D;\RRdtd}}
\DeclareMathOperator{\dist}{dist}
\newcommand{\HoD}{\Ho{D}}
\newcommand{\Ho}[1]{H^{1}\mleft(#1\mright)}
\newcommand{\trGI}{\trace_{I}}
\newcommand{\HozD}[1]{H^{1}_{0,\mathrm{D}}\mleft(#1\mright)}
\newcommand{\aE}{a} % For EDP
\newcommand{\ui}{u_{i}}
\newcommand{\us}{u_{s}}
\newcommand{\BR}{B_{R}}
\newcommand{\DR}{D_{R}}
\newcommand{\HozDDR}{\HozD{\DR}}
\newcommand{\FE}{F}
\newcommand{\IPRRd}[2]{\IP{#1}{#2}_{\RRd}}
\newcommand{\IP}[2]{\mleft<#1,#2\mright>}
\newcommand{\vbar}{\overline{v}}
\newcommand{\DPGR}[2]{\IP{#1}{#2}_{\GR}}
\newcommand{\TR}{T_{R}}
\newcommand{\trGR}{\trace_{R}}
\newcommand{\GR}{\Gamma_{R}}
\newcommand{\aT}{a_{T}} % For truncated
\newcommand{\FT}{F_{T}}
\newcommand{\restrict}{\big|} % Advice given in https://tex.stackexchange.com/questions/22252/how-to-typeset-function-restrictions
\newcommand{\NTADp}[1]{\mathrm{NT}_{\mathrm{Mat},\Dp}\mleft(#1\mright)}
\newcommand{\NTnDp}[1]{\mathrm{NT}_{\mathrm{scal},\Dp}\mleft(#1\mright)}
\newcommand{\NTAD}[1]{\mathrm{NT}_{\mathrm{Mat},D}\mleft(#1\mright)}
\newcommand{\NTnD}[1]{\mathrm{NT}_{\mathrm{scal},D}\mleft(#1\mright)}
\newcommand{\CzoDpclosRRdtd}{\Czo{\Dpclos;\RRdtd}}
\newcommand{\Czo}[1]{C^{0,1}\mleft(#1\mright)}
\newcommand{\Dpclos}{\clos{\Dp}}
\newcommand{\CzoDpclosRR}{\Czo{\Dpclos;\RR}}
\newcommand{\muo}{\mu_{1}}
\newcommand{\mut}{\mu_{2}}
\newcommand{\NLtDR}[1]{\N{#1}_{\LtDR}}
\newcommand{\LtDR}{\Lt{\DR}}
\newcommand{\N}[1]{\mleft\|#1\mright\|}
\newcommand{\Co}{C_{1}}
\newcommand{\LI}{L_{I}}
\newcommand{\NLtD}[1]{\N{#1}_{\LtD}}
\newcommand{\NLtGI}[1]{\N{#1}_{\LtGI}}
\newcommand{\gradGI}{\grad_{\GI}}
\newcommand{\Ct}{C_{2}}
\newcommand{\Cttilde}{\widetilde{C}_{2}}
\newcommand{\NW}[1]{\N{#1}_{1.k}}
\newcommand{\NLiD}[1]{\N{#1}_{\LiD}}
\newcommand{\LiD}{\Li{D}}
\newcommand{\Ki}{K_{i}}
\newcommand{\Dclos}{\clos{D}}
\newcommand{\interior}[1]{\mathring{#1}}
\newcommand{\Kj}{K_{j}}
\newcommand{\set}[1]{\mleft\{#1\mright\}}
\newcommand{\vh}{v_{h}}
\newcommand{\st}{:}
\newcommand{\polyp}[1]{\cP_{p}\mleft(#1\mright)}
\newcommand{\Kiclos}{\clos{\Ki}}
\newcommand{\Cz}[1]{C^{0}\mleft(#1\mright)}
\newcommand{\CzD}{\Cz{D}}
\DeclareMathOperator{\diam}{diam}
\newcommand{\Vhp}{V_{h,p}}
\newcommand{\vhptilde}{\widetilde{v}_{h,p}}
\newcommand{\HmD}{\Hm{D}}
\newcommand{\Hm}[1]{H^{m}\mleft(#1\mright)}
\newcommand{\NHsD}[1]{\N{#1}_{\HsD}}
\newcommand{\HsD}{\Hs{D}}
\newcommand{\Hs}[1]{H^{s}\mleft(#1\mright)}
\newcommand{\NHoD}[1]{\N{#1}_{\HoD}}
\newcommand{\uh}{u_{h}}
\newcommand{\uz}{u_{0}}
\newcommand{\Th}{\cT_{h}}
\newcommand{\tr}{\gamma}
%\newcommand{\dD}{\partial D}
\newcommand{\kz}{k_{0}}
%\newcommand{\HthtGD}{\Htht{\GD}}
\newcommand{\Htht}[1]{H^{3/2}\mleft(#1\mright)}
%\newcommand{\Woi
\newcommand{\aadj}{\widetilde{a}}
\newcommand{\vb}{\overline{v}}
\newcommand{\vo}{v_{1}}
\newcommand{\vt}{v_{2}}
\newcommand{\epsymb}{\star}
\newcommand{\api}{a_{\epsymb}}
\newcommand{\vtbar}{\overline{v}_{2}}
\newcommand{\CzoDRRRdtd}{\Czo{\DR;\RRdtd}}
\newcommand{\IPLtDR}[2]{\IP{#1}{#2}_{\LtDR}}
\newcommand{\HtDR}{\Ht{\DR}}
\newcommand{\Ht}[1]{H^{2}\mleft(#1\mright)}
\newcommand{\NHtDR}[1]{\N{#1}_{\HtDR}}
\newcommand{\CHtell}{C_{H^{2},\epsymb}}
\newcommand{\CHthh}{C_{H^2}}
\newcommand{\Nunsure}[1]{\N{#1}_{\HhGI}}
\newcommand{\HhGI}{\Hh{\GI}}
\newcommand{\NLtGD}[1]{\N{#1}_{\GD}}
\newcommand{\gradGD}{\grad_{\GD}}
\newcommand{\NLiDRRR}[1]{\N{#1}_{\LiDRRR}}
\newcommand{\LiDRRR}{\Li{\DR;\RR}}
\newcommand{\half}{\frac{1}{2}}
\newcommand{\Cfemo}{C_{\mathrm{FEM},1}}
\newcommand{\Cfemt}{C_{\mathrm{FEM},2}}
\newcommand{\Cfemth}{C_{\mathrm{FEM},3}}
\newcommand{\Npi}[1]{\N{#1}_{\epsymb}}
\newcommand{\quarter}{\frac{1}{4}}
\newcommand{\Ph}{\cP_{h}}
\newcommand{\Ih}{\cI_{h}}
\newcommand{\wh}{w_{h}}
\newcommand{\whtilde}{\widetilde{w}_{h}}
\newcommand{\NHoDR}[1]{\N{#1}_{\HoDR}}
\newcommand{\HoDR}{\Ho{\DR}}
\newcommand{\rhs}{M\mleft(f,\gI,\gD\mright)}
\newcommand{\LiDp}{\Li{\Dp}}
\newcommand{\LiDpCC}{\Li{\Dp;\CC}}
\newcommand{\bxz}{\mathbf{x}_{0}}
\newcommand{\Ball}[2]{B_{#1}\mleft(#2\mright)}
\newcommand{\bfzero}{\mathbf{0}}
\newcommand{\HhGR}{\Hh{\GR}}
\newcommand{\HmhGR}{\Hmh{\GR}}
\newcommand{\Hmh}[1]{H^{-1/2}\mleft(#1\mright)}
\newcommand{\Lap}{\Delta}
\newcommand{\Wmp}[1]{W^{m,p}\mleft(#1\mright)}
\newcommand{\WmpD}{\Wmp{D}}
\newcommand{\LtDpCC}{\Lt{\Dp;\CC}}
\newcommand{\trD}{\trGD}
\newcommand{\trI}{\trGI}
\newcommand{\CzoD}{\Czo{D}}
\newcommand{\WoiD}{\Woi{D}}
\newcommand{\Woi}[1]{W^{1,\infty}\mleft(#1\mright)}
\newcommand{\enlargement}[1]{#1_{+1}}
\newcommand{\beps}{\boldsymbol{\eps}}
\newcommand{\WoiDRRdtd}{\Woi{D;\RRdtd}}
\newcommand{\WoiDRR}{\Woi{D;\RR}}
\newcommand{\WoiDpclosRRdtd}{\Woi{\Dpclos;\RRdtd}}
\newcommand{\WoiDpclosRR}{\Woi{\Dpclos;\RR}}
\newcommand{\km}{k_{m}}
\newcommand{\um}{u_{m}}
\newcommand{\fm}{f_{m}}
\newcommand{\NLtDp}[1]{\N{#1}_{\LtDp}}
\newcommand{\NHtD}[1]{\N{#1}_{\HtD}}
\newcommand{\HtD}{\Ht{D}}
\newcommand{\uhat}{\hat{u}}

\newcommand{\tat}{\text{ at }}
\newcommand{\tforall}{\text{ for all }}
\newcommand{\twith}{\text{ with }}


\title{The Helmholtz Equation in Heterogeneous and Random Media: Analysis and Numerics}
\author{Owen Rhys Pembery}
\degree{Doctor of Philosophy}
\department{Department of Mathematical Sciences}
\degreemonthyear{September 2019}

\begin{document}

\maketitle

\setcounter{tocdepth}{2}
%\renewcommand{\familydefault}{\sfdefault}


\tableofcontents

\listoffigures
\listoftables
\listofalgorithms

\chapter{Introduction}\label{chap:intro}
\section{The subject of the thesis}

The subject of this thesis is rigorous theory and fast methods for the stochastic Helmholtz equation
\beq\label{eq:introhh}
\grad \cdot \mleft(A\grad u\mright) + k^2 \, n \, u = -f,
\eeq
where $A,$ $n,$ and $f$ are random fields (i.e., they are spatially heterogeneous and random). We are particularly interested in theory and methods that are applicable for large values of the wavenumber $k$, as the case of large $k$ is both of interest in applications and theoretically and computationally demanding.

The Helmholtz equation is the simplest possible model of wave propagation. It is the Fourier transform (in time) of the scalar wave equation
\beq\label{eq:introwave}
n \, \frac{\partial^2 U}{\partial t^2} - \grad \cdot\mleft(A\grad U\mright) = F;
\eeq
equivalently, one can seek solutions of \eqref{eq:introwave} that are time-harmonic, that is, have a single frequency of oscillation in time, and so can be written $U = e^{ikt}u$. In certain scenarios, 
\eqref{eq:introhh} can be also derived from the time-harmonic Maxwell's equations (the Fourier transform in time of Maxwell's Equations governing electromagnetics), see, e.g., \cite[Remark 2.1]{MoSp:19} for this derivation.

The physcial motivation for studying \eqref{eq:introhh} is, therefore, any physical scenario in which wave propagation can be modelled by either \eqref{eq:introwave} or Maxwell's equations. One prominent example of the usage of \eqref{eq:introwave} is in subsurface imaging, where the rock structures of the earth's crust are imaged by generating waves within the rock structures, and recording the reflections of the waves. The waves within the earth's rock structures satisfy the elastic wave equation, but under the so-called acoustic approximation, this can be approximated by \eqref{eq:introwave}; see \cite[Section 1.2]{Ch:15} for a physical derivation of the elastic wave equation, and \cite[Section 1.2.6]{Ch:15} for a derivation and discussion of the acoustic approximation. Other physical scenarios involving waves modelled by either \eqref{eq:introwave} or Maxwell's equations are the propagation of sound in an inviscid fluid\cite[Section 2.1]{CoKr:13}, and Microwave imaging (see, e.g., \cite[Section 6.4]{BoDoGrSpTo:19}). A mathematical and computational motivation for studying \eqref{eq:introhh} is that many of the difficulties one encounters when studying and numerically solving more complex wave-propagation models, such as the elastic wave equation, are also encoutered with \eqref{eq:introhh}. Therefore \eqref{eq:introhh} is an appropriate starting point for mathematical study of, and numerical algorithms for, wave propagation problems.

One commonality of all the examples given above is that they have high effective frequency, that is, the wavenumber $k$ is large. The wavenumber may be large because the physical frequency is large (as in, for example non-destructive testing, where waves of frequency $1\times10^5$--$2\times10^7$ Hz (see, e.g., \cite{Bi}) are passed through materials to image their interior), or because the waves are low-frequency, but propagate over a large domain (as in seismic imaging, where the wave frequencies are in the range 1-100 Hz (see, e.g., \cite{Sc}) but the domain of interest is on the kilometre scale\footnote{E.g., the SEG Overthrust model, a common benchmark for seismic imaging applications has domain size 20km $\times$ 20km $\times$ 4.65 km.}. These low-frequency, large-domain problems have many wavelengths in the domain, and hence, when they are scaled to a domain of size $\approx 1,$ they give rise to problems with large (effective) frequency.

We mentioned above that we are interested in the Helmholtz equation with \emph{heterogeneous} coefficients; again, this interest is driven by applications where the material parameters are not constant in space. For example, in subsurface imaging of the Earth's crust, there will be different rock types; these rocks may contain other materials, such as water, or oil, and above these rocks will be the sea; each of these materials will have different material properties, such as the density and Lam\'e parameters (see, e.g. \cite[Section 1.2.4]{Ch:15} for an explanation of how these parameters manifest themselves in \eqref{eq:introwave}).

We also stated above that we will consider the Helmholtz equation with \emph{random} coefficients, that is, we want to model physical situations where there is uncertainty in the material parameters. This uncertainty may arise in the inverse problem, where one has sent an incident wave into an unknown medium, recorded the scattered wave coming out of the medium, and wishes to reconstruct the medium itself. There will be uncertainties inherent in this process, for example, the scattered wave will only be recorded at discrete points in space, rather than everywhere, and these recordings will be subject to measurement error. These uncertanties in the measurement processes will result in uncertanties in the inferred properties of the medium. Alternatively, this uncertainty can arise in the forward problem, where we can assume we are already aware of uncertainty in our knowledge of the material parameters, and we wish to know properties of the wave passing through the uncertain medium. This occurs, for example, in radar imaging of ice sheets, where one wishes to know properties of the wave scattered by the ice, as in \cite{JiPi:18}.

This thesis will only focus on UQ for the forward problem, for two main reasons:
\ben
\item The Bayesian (statistical) inverse problem, whilst very relevant in applications, introduces other computational difficulties, unrelated to solving \eqref{eq:introhh}. Such difficulties are related to sampling the random coefficients $A$ and $n$; in wave propagation Bayesian inverse problems the distributions of $A$ and $n$ may well be multi-modal\ednote{I've been able to construct a simple 1-D example of this---it'll go in an appendix, but it's not written up yet.}\optodo{Finish example in appendix} (the stochastic analogue of solving an optimisation problem with multiple local minima), and constructing good samplers for such problems is an open research question\optodo{Find refs for this - ask Tom Pennington?}.
\item The forward problem and inverse problem share the common computational difficulty of needing to solve many (deterministic) realisations of \eqref{eq:introhh}. Whether the uncertainty in $A$ and $n$ has arisen as a result of the inverse or forward problem, most UQ algorithms will require many samples of the (random) solution of \eqref{eq:introhh}. As will be discussed below, obtaining one sample of the solution of \eqref{eq:introhh} is a considerable computational task, and so obtaining many (and `many' could easily mean thousands) of such samples is an even harder task. Reducing the computational cost of obtaining lots of samples of the solution of \eqref{eq:introhh} will be the main focus of the algorithms developed and studied in this thesis.
\een

We have just stated that it is hard to solve the (deterministic) Helmholtz equation
\beq\label{eq:introdet}
\grad \cdot (\Ad \grad \ud) + k^2 \,\nd\,\ud = -\fd,
\eeq
i.e., a single realisation of \eqref{eq:introhh}, numerically; we now provide some background on why this is the case. When solving \eqref{eq:introdet} numerically we discretise it to obtain a linear system
\beq\label{eq:intromat}
\Amat \bu = \bff.
\eeq
In this thesis we will be exclusively concerned with discretisation via finite elements, see \cref{chap:background} for the details of such a discretisation. The linear systems \eqref{eq:intromat} arising from standard finite-element discretisations of \eqref{eq:introdet} are hard to solve, as the matrices $\Amat$ are large, indefinite, and non-Hermitian. We now will briefly outline the reasons why the matrices $\Amat$ are large, before discussing how the size of the matrices $\Amat$ affects the solution strategies we use. For details on why the matrices $\Amat$ are indefinite and non-Hermitian, and how this affects solution strategies for \eqref{eq:intromat}, see \cref{chap:background}\optodo{Update this reference once this section has been written}

When the wavenumber $k$ is large, the matrices $\Amat$ are also large, because the number of degrees of freedom for standard numerical methods for the deterministic Helmholtz equation must increase as $k$ increases. One can see this by considering the problem of representing, or interpolating, the solution of \eqref{eq:introdet}; solutions $u$ of \eqref{eq:introdet} oscillate on a scale $1/k$, and therefore the number of degrees of freedom (interpolation points) must increase like $k^d$ (where $d$ is the spatial dimension) in order to keep the interpolation error for $u$ bounded. This need for increasing degrees of freedom with $k$ is illustrated in \cref{fig:introinterp}, where we see the interpolation error grows if the number of degrees of freedom is not increased with $k$. This dependence of the number of degrees of freedom on $k$ to achieve bounded interpolation error can be made more rigorous, see \cref{chap:background}\optodo{Update link once chapter written}. This maxim is referred to as a `fixed number of points per wavelength', as in practice one typically chooses to use 6--10 discretisation points in each dimension for each wavelength in the domain---this choice ensures the number of degrees of freedom increases like $k^d,$ and empirically keeps the interpolation error at a reasonable size. Hence, if one chooses enough number of degrees of freedom to preserve the interpolation error, the linear systems \eqref{eq:intromat} will have $k^d$ unknowns. 

\begin{figure}
\caption{\label{fig:introinterp} Figure showing interpolation error growing if mesh is not refined}
\end{figure}\optodo{Update Figure caption once figure is created}

However, discretising \eqref{eq:introdet} with a fixed number of points per wavelength is \emph{not} enough to keep the error in the finite-element solution of \eqref{eq:introdet} bounded as $k\rightarrow \infty.$ This is because standard-finite-element methods applied to the Helmholtz equation suffer from pollution, where the numerically calculated wave has a different wavelength to the true solution $u$, and so `drifts' away from $u$; moreover, this error increases as $k$ increases. See \cref{fig:intropoll} for an illustration of this phenomenon, and \cref{chap:background}\optodo{Better ref once chapter written} for an extended discussion of this phenomena.

\begin{figure}
\caption{\label{fig:intropoll} Figure showing pollution effect}
\end{figure}\optodo{Update Figure caption once figure is created}

In order to keep the finite-element error bounded when solving \eqref{eq:introdet}, once must over-refine the numerical grid. That is, rather than using a fixed number of points per wavelength, one must increase the number of points per wavelength as $k$ increases. To achieve bounded finite-element error, one must refine the finite-element mesh size $h$ like $k^{-3/2}$. Whilst this result has been known numerically for some time, it was only proven for \eqref{eq:introdet} with constant coefficients (on various domains and for various finite-element spaces) in \cite{IhBa:95a,Wu:14,DuWu:15,ChNi:18}, and the first proof (to our knowledge) for \eqref{eq:introdet} with heterogeneous coefficients  is contained in \cref{chap:background}. Choosing $h \sim k^{-3/2}$ means \eqref{eq:intromat} is a linear system of size $k^{3d/2}$, larger than if one merely wants the interpolation error to be bounded. Hence, requiring a bounded finite-element error gives rise to very large linear systems.

More briefly, if one wants the finite-element solution to be quasi-optimal (that is, up to a constant, the finite-element solution is the best approximation in the finite-element space), then one must over-refine even more, and take $h \sim k^{-2}$. This mesh condition will give rise to linear systems with $k^{2d}$ degres of freedom. See \cref{chap:background} for further details on the necessity of this mesh condition, and further discussion of all the mesh conditions discussed above.

In summary, numerically solving the Helmholtz equation gives rise to large linear systems, and the size of these linear systems increases as $k$ increases. The table \cref{tab:introlinsys} summarises the size of the linear systems one obtains for different values of $k$, depending on the spatial dimension, and the properties of the finite-element solution that one requires.
\begin{table}
\begin{tabular}{c|cccccc}
  &\multicolumn{2}{c}{Interpolation error bounded}&\multicolumn{2}{c}{Finite-element error bounded}&\multicolumn{2}{c}{Quasi-optimality}\\
    &\multicolumn{2}{c}{($h = 1/10 \times k^{-1}$)}&\multicolumn{2}{c}{($h = k^{-3/2}$)}&\multicolumn{2}{c}{($h = k^{-2}$)}\\
&2-D&3-D&2-D&3-D&2-D&3-D\\
\hline
$k$&&&&&&\\
$10$&$10^4$&$10^6$&$10^3$&$\approx 3 \times 10^4$&$10^4$&$10^6$\\
$100$&$10^6$&$10^9$&$10^6$&$10^9$&$10^8$&$10^{12}$\\
$1000$&$10^8$&$10^{12}$&$10^9$&$\approx 3 \times 10^{13}$&$10^{12}$&$10^{18}$
\end{tabular}
\caption{\label{tab:introlinsys}Table showing the number of degrees of freedom that would be required to obtain various properties of finite-dimensional approximations of the solution $u$ of \eqref{eq:introdet}, for various values of $k$, in 2 and 3 spatial dimensions.}
\end{table}

We now turn our attention to how one might solve the large, indefinite, non-Hermitian linear systems \eqref{eq:intromat}. One option is to solve the linear systems \eqref{eq:intromat} using a direct solver (solvers that, up to machine precision, invert the linear system \eqref{eq:intromat} exactly). Such solvers are incredibly competitive for solving \eqref{eq:introdet} in 2-D, if \eqref{eq:intromat} has up to $10^6$ unknowns; however, for larger linear systems \eqref{eq:intromat}, or those obtained from 3-D discretisations, direct solvers are not as competitive as so-called iterative solvers\ednote{Ivan---A good reference/introduction to this kind of stuff?}. An iterative solver is one that does not solve \eqref{eq:intromat} exactly, but rather produces a sequence of approximations to the solution of \eqref{eq:intromat}. A standard iterative solver to use for non-Hermitian linear systems is GMRES; this is the solver we will use throughout this thesis. However, it is hard to prove convergence results for GMRES applied to \eqref{eq:intromat} as the matrices $\Amat$ are typically indefinite\optodo{Understand why this is the case}, and moreover, numerical evidence shows that GMRES applied to \eqref{eq:intromat} can perform very badly (the number of interations to acheive convergence can grow dramatically with $k$). An explanation of how the wave-nature of the solution of the Helmholtz equation causes slow convergence of iterative methods for \eqref{eq:intromat} is explained in \cite[Section 2.1]{ErGa:12}, using a finite-difference approximation of the Helmholtz equation as an example.\optodo{Find an example of behaviour of non-preconditioned GMRES deteriorating as $k\rightarrow \infty.$}\optodo{Find ref for error going south as $k$increases}.

As GMRES applied to \eqref{eq:intromat} performs badly, we consider preconditioning \eqref{eq:intromat}, that is, solving the equivalent linear system
\beq\label{eq:intropre}
\PmatI\Amat \bu = \PmatI \bff
\eeq
for some matrix $\Pmat$. The idea of preconditioning is that the preconditioned matrix $\PmatI \Amat$ will have better properties than $\Amat,$ and therefore iterative solvers applied to \eqref{eq:intropre} will exhibit better convergence properties, but the solution of \eqref{eq:intropre} is the same as the solution of \eqref{eq:intromat}\footnote{In this exposition we have only considered left-preconditioning, that is, multiplying $\Amat$ from the left by $\PmatI$. However, one can also-consider right-preconditioning, that is, solving the linear system $\Amat \PmatI \butilde = \bff,$ the solution $\bu$ is then given by $\bu = \PmatI \butilde.$}.

The goal of preconditioning is to choose the preconditioner $\Pmat$ such that:
\ben
\item\label[listrequirement]{it:intropreone} The matrix $\PmatI \approx \AmatI$ (so that $\PmatI\Amat \approx \Imat$ and \eqref{eq:intropre} is close to the equation $\bu = \PmatI \bff$) and
\item\label[listrequirement]{it:intropretwo} The action of $\PmatI$ is cheap to compute.
\een
The ideal preconditioner from the point of view of \cref{it:intropreone} is $\AmatI$, however, if we could cheaply compute the action of $\AmatI,$ we could cheaply solve \eqref{eq:intromat}, and there would be no need for preconditioning. Hence, one needs to balance the \cref{it:intropreone,it:intropretwo} so that one obtains a good approximation of $\AmatI$ that is cheap to apply. There are several groups around the world working on the construction of good preconditioners for the Helmholtz equation, and this is an open research area. However, the construction of such preconditioners is not the focus of this thesis\optodo{Either summarise preconditioners here, somewhere else, or put a link to some overviews - Gander SIREV?}.

Aside from all the above issues in solving the deterministic Helmholtz equation \eqref{eq:introdet}, when seeking to perform UQ calculations for the stochastic Helmholtz equation \eqref{eq:introhh} one often needs to solve many realisations of \eqref{eq:introhh}, i.e., one needs to solve many (which we emphasise again, could easily be thousands) different deterministic Helmholtz problems which, as has just been shown, are each individually difficult to solve. This situation arises when using sampling-based methods such as Monte-Carlo or Stochastic-Collocation methods to compute properties of the solution $u$ of \eqref{eq:introhh}. Rigorously studying \eqref{eq:introhh}, devising computational techniques to reduce the cost of such UQ calculations, and rigorously justifying this reduction, is the subject of this thesis.

\section{The aims of the thesis}

The aims of this thesis are to conduct a detailed study of the Helmholtz equation in random media, with special regard to high-frequency behaviour, and to provide fast, rigorously justifiable UQ methods for the high-frequency Helmholtz equation. We will first prove well-posedness results and a priori bounds on the solution of the Helmholtz equation in classes of random media that are almost-surely nontrapping. This nontrapping assumption will allow us to obtain frequency-independent a priori bounds on the solution. These results on well-posedness and a priori bounds are crucial for the numerics that follow, as they show us the problems we are solving are well-posed, and the bounds we obtain will allow us to rigorously prove results about our numerical method. The only similar existing results in the literature were for media that are frequency-dependent perturbations of a constant background (in \cite{FeLiLo:15}), and our results are a major improvement on these, not least because they are frequency-independent.

We then seek to design numerical methods for the high-frequency Helmholtz equation that provide speedup over n\"aive numerical methods, in order to make UQ calculations for the Helmholtz equation more feasible. We also wish to analyse these numerical methods and show how their behaviour (both speedup and computational cost) depend on the wavenumber $k$. We propose two complementary numerical methods to speed up UQ calculations for the Helmholtz equation.

The first strategy, so-called nearby preconditioning, seeks to reduce the computational cost of assembling preconditioners for many deterministic Helmholtz problems. This reduction is achieved by re-using a preconditioner from one deterministic Helmholtz problem for other, nearby Helmholtz problems. We will investigate the effectiveness of this strategy, and see that, whilst its effectiveness degrades with increasing $k$, it still provides a speedup for a range of physically relevant $k$.

The second strategy, somewhat orthogonal to the first, is to use a Multi-Level Monte Carlo (MLMC) method to reduce the number of samples needed when performing UQ calculations. By reducing the number of samples needed, we will decrease the computational cost for UQ calculations. The analysis of this method requires an extension of standard MLMC theory to the case where the numerical error is dependent on a parameter besides the mesh size (in our case, the error in finite-element calculations depends on the wavenumber $k$). We will see that a MLMC approach does reduce cost, both in theory and in practice, and that the relative cost reduction with respect to a standard Monte Carlo method does not degrade as $k$ grows.


\section{The Main Acheivements of the Thesis}

The main acheivements of the thesis are as follows:

\ben
\item A general framework for proving well-posedness results and a priori bounds for stochastic elliptic PDEs. These general tools are used to prove \cref{it:achievements-bounds} below, but allow one to, in principle, conclude similar results to those in \cref{it:achievements-bounds} for a range of stochastic elliptic PDEs. They can be used in cases where the bilinear form given by the PDE is indefinite, such as for the time-harmonic Maxwell's equations\ednote{Including this as a separate `acheivement' is conditional on me obtaining some other good examples to which I can apply the general framework}.

\item\label[itemachievement]{it:achievements-bounds} Well-posedness results and a priori bounds on the solution of the Helmholtz equation in random media, results and bounds obtained are frequency-independent. The previous work in the literature proved such results and bounds under restrictions that became more stringent as the frequency increased.

\item A computational method (nearby preconditioning) that reduces the computational cost of solving many realisations of the Helmholtz equation in random media. The reduction in computational cost is gained by reusing the preconditioner from one realisation of the Helmholtz equation for subsequent `nearby' realisations. This computational method is rigorously analysed, and its effectiveness is precisely characterised, although this effectiveness does degrade as the frequency of the problems is increased.

\item Numerical experiments that show the rigorous analysis of nearby preconditioning may not be\optodo{change?} not sharp, and that the method is, in practice, more effective than can be rigorously proved. However, the effectiveness does still degrade as frequency is increased.

\item Analysis of the Monte-Carlo (MC) and Multi-Level Monte Carlo (MLMC) methods applied to the Helmholtz equation in random media. MLMC is a variance reduction technique that uses computations on a sequence of meshes to reduce the variance in UQ calculations, and therefore to reduce the number of realisations of the Helmholtz equation that need to be solved. We extend the existing abstract MLMC analysis in the literature to the case where the finite-element error is dependent on an additional parameter (here this parameter is the wavenumber $k$), and then apply this abstract analysis to the Helmholtz equation. We show that the error in MC is independent of the wavenumber $k$, and we show that MLMC gives a cost reduction over MC, with the relative cost reduction being independent of $k.$

\item Computational experiments for MLMC that show that, in many cases, the speedup one obtains using MLMC is greater than that predicting theoretically\ednote{I've no idea what the result of these computations will be at this point!}.
\een


\section{The Structure of the thesis}

In \cref{chap:background} we give background material on the (deterministic) Helmholtz equation and its discretisation via finite elements; this material will be necessary to understand the rest of this thesis. We provide an extended discussion of recent developments concerning the well-posedness of the deterministic heterogeneous Helmholtz equation and a priori bounds on its solution. We also give an overview of the theory of finite-element discretiations of the Helmholtz equation, and prove new results on the behaviour of the error when discretising the heterogeneous Helmholtz equation.

In \cref{chap:stochastic} we define three formulations of the stochastic exterior Dirichlet problem (SEDP) for the Helmholtz equation in random media. We prove well-posedness results for these formulations, and also prove a priori bounds on their solution that are explicit in all parameters of interest, especially the wavenumber $k.$ Crucially, using recent well-posedness results and a priori bounds obtained in \cite{GrPeSp:19} for the heterogeneous (but non-random) Helmholtz equation, we are able to prove such results and bounds for the stochastic Helmholtz equation under assumptions that are $k$-independent. We also give a general framework for proving such results for stochastic elliptic PDEs.

In \cref{chap:nbpc} we propose a computational technique, \emph{nearby preconditioning}, that speeds up the process of solving many realisations of the Helmholtz equation by reusing preconditioners from one realisation for (potentially) many subsequent `nearby' realisations. In this theoretical study, we assume that we have access to the action of an exact preconditioner for one Helmholtz problem, and study the convergence of GMRES for subsequent problems. That is, we investigate the convergence of GMRES applied to $\AmatoI\Amatt$, where $\Amato$ and $\Amatt$ are matrices arising from finite-element discretisations of the Helmholtz equation. We show that if the coefficients of the underlying PDEs are sufficiently close (or `nearby'), then GMRES applied to $\AmatoI\Amatt$ will converge in a number of iterations that is independent of $k.$ However, the conditions for 'sufficient closeness' that we prove depend on either $k$ or the mesh size $h$. We then provide numerical experiments showing the sharpness (in some cases) or the lack of sharpness (in other cases) of our proven results.

In \cref{chap:mlmc} we study the multi-level Monte Carlo (MLMC) method for reducing the variance in UQ calculations for the Helmholtz equation in random media. We first extend the abstract theory for MLMC to the sitation when there is an additional parameter (in our case, the wavenumber $k$), alongside the mesh size $h$, governing the size of the error in numerical approximations. Having extended the abstract theory, we then apply it to the Helmholtz equation for a variety of different quantities of interest, and we prove that MLMC gives a cost reduction over the Monte-Carlo method, and that this reduction is independent of $k$. We then investigate MLMC numerically, and find that in many cases the speedup we observe in numerics is better than the speedup we can prove rigorously.

%for MLMC later?:
%(that is the ratio $\CMC(\eps)/\CMLMC(\eps)$, where $\CMC(\eps)$ is the cost for the root-mean-sqaured


\chapter{PDE Theory of the Deterministic Helmholtz Equation and Theory of its Finite-Element Discretisation}\label{chap:background}
\chaptermark{PDE and FE Theory}
\section{Introduction}
This chapter has two main foci:
%    \setlist[enumerate]{restart} ????
\optodo{Sort numbering}
\ben
\item Recapping theory for the deterministic Helmholtz equation in heterogeneous media, focussing especially on well-posedness and a priori bounds on the solution, and
\item Recapping and extending theory for the finite-element method for the deterministic Helmholtz equation, especially error bounds.
  \een

  In this chapter we will first provide an overview of well-posedness results for the deterministic Helmholtz equation, and a priori bounds on its solution.\optodo{Put somewhere below that Fredholm gives a bound without dependence on $k$} Such results are more complicated than analagous results for the simpler stationary diffusion equation
  \beqs
\grad \cdot \mleft( A \grad u \mright) = -f,
\eeqs
as results for the Helmholtz equation depend on whether the medium described by the coefficients $A$ and $n$ is `trapping' or `nontrapping' (these terms will be discussed in more precision below). We will then move on to discussing the finite-element approximation of \eqref{eq:introdet}, focussing especially on error bounds and quasi-optimality, and the $k$-dependent mesh conditions under which such properties can be proven.

In \cref{sec:varform} we will define the deterministic Helmholtz problems we will study in this chapter (these are the deterministic analogues of the stochastic Helmholtz problems we will study), prove a straightforward lemma on the regularity of the solution of these problems, and show that these problems satisfy a so-called G\r{a}rding inequality. In \cref{sec:wpbounds} we will then state in some detail the well-posedness results and a priori bounds from \cite{GrPeSp:19}, these results will be crucial for our analysis of stochastic Helmholtz problems in \cref{chap:stochastic}. In \cref{sec:wpdisc} we will review recent and historical research efforts in this area. We then move on to the finite-element method for \eqref{eq:introdet}; in \cref{sec:fetheory} we recap basic concepts of the finite-element method, before proving new results on error bounds for the finite-element method for \eqref{eq:introdet} in \cref{sec:errbound}. We then give an overview of the literatureon error bounds and quasi-optimality for \eqref{eq:introdet} in \cref{sec:helmfedisc}.


\section{PDE Theory of the Deterministic Helmholtz Equation}
  

  We will begin by defining the two deterministic Helmholtz problems that we consider in this thesis (and the stochastic analogues of which we also consider). We first state these problems in `strong form' (that is, where derivatives are understood as distributional derivatives), in \cref{sec:varform}, when considering finite-element approximations of the problems, we will consider the variational formulations of these problems.\optodo{Somewhere explain convention on function spaces, and define matrix space} In this section we largely follow the presentation in \cite{GrPeSp:19}

  \bprob[Exterior Dirichlet Problem]\label{prob:edp}
  Let $\Dm$ be a bounded Lipschitz\optodo{Needed for statement?} open set such that the open complement $\Dp \de \RRd \setminus \Dmclos$ is connected. Let $\GD \de \partial \Dm.$ Given
  \bit
  \item $k > 0,$
\item $f \in \LtDp$ with compact support,
\item $\gD \in \HhGD,$
\item $n \in \LiDpRR$ such that $1-n$ has compact support and there exist $0 < \nmin < \nmax < \infty$ such that
  \beqs
\nmin \leq n(\bx) < \nmax \tfae \bx \in \Dp,
  \eeqs
\item $A \in \LiDpRRdtd$ such that $I-A$ has compact support, $A$ is symmetric, and there exist $0 < \Amin < \Amax < \infty$ such that
  \beqs
\Amin \abs{\bxi}^2 \leq \mleft(A(\bx) \bxi \mright) \cdot \bxibar < \Amax \abs{\bxi}^2 \tfa \bxi \in \CCd \tfae \bx \in \Dp,
  \eeqs
  \eit
  we say $u \in \HolocDp$ satisfies the \defn{exterior Dirichlet problem} if
  \beqs
\grad \cdot \mleft(A \grad u\mright) + k^2 n u = -f \tin \Dp,
\eeqs
\beq\label{eq:dbc}
\trGD u = \gD,
\eeq
and $u$ satisfies the Sommerfeld radiation condition\optodo{Make sure you understand this}
\beq\label{eq:sommerfeld}
\dudr(\bx) - iku(\bx) = o\mleft(\frac1{r^{(d-1)/2}}\mright)
\eeq
as $r \de \abs{\bx} \rightarrow \infty,$ uniformly in $\bxhat \de \bx/\abs{\bx}.$
\eprob

To interpret \cref{prob:edp} physically, we can think of $u$ as being the acoustic pressure field caused by the scattering of an incoming wave $\ui$ by the scatterer $\Dm.$\footnote{In the literature the scattered field is sometimes denotes $us,$ and $u$ denotes the total field $\ui+ \us.$} The Dirichlet boundary condition \eqref{eq:dbc} means $\Dm$ corresponds to a sound-soft scatterer, that is, one on which the total field $u + \ui$ vanishes, with $\gD = - \trGD \ui.$ The function $f$ represents other pressume sources in the domain, although if one is solely interested in scattering, $f=0.$

The Sommerfeld radiation condition \eqref{eq:sommerfeld} ensures that the solutions of \cref{prob:edp} correspond to physically `outgoing' waves\optodo{Find reference}, and also guarantees the uniqueness of solution to \cref{prob:edp}\optodo{Ref for this}. However, \cref{prob:edp} is posed on an infinite domain, which cannot be fully represented when discretising with the FEM. Therefore, one must truncated the infinite domain $\Dp$ and impose an artificial boundary condition on the external boundary of the truncated domain. Options for the truncated boundary condition include a perfectly matched layer which\optodo{check} mimics the whole of the external domain, or so-called FEM-BEM coupling\optodo{Check}, where a boundary element method is used to approximate the solution in the exterior of the truncated domain. However, in this thesis, we will use a simpler approach, imposing an \defn{impedance boundary condition}
\beq\label{eq:imptext}
\dn u - iku = \gI
\eeq
on the truncated boundary. If $\gI = 0,$ then \eqref{eq:imptext} can be seen as a first-order approximation to \eqref{eq:sommerfeld}. Truncating with an impedance boundary condition gives rise to the following alternative deterministic Helmholtz problem\optodo{Make a picture explaining EDP and TEDP on one pic}
\bprob[Truncated Exterior Dirichlet Problem]\label{prob:tedp}
 Let $\Dm$ be a bounded Lipschitz\optodo{Needed for statement?} open set such that the open complement $\Dp \de \RRd \setminus \Dmclos$ is connected. Let $\Dtilde$ be a bounded connected Lipschitz open set such that $\Dmclos \compcont \Dtilde.$ Let $D \de \Dtilde \setminus \Dmclos$, $\GD \de \partial \Dm,$ and $\GI \de \partial \Dtilde.$ Given
  \bit
  \item $k > 0,$
\item $f \in \LtD$
\item $\gD \in \HhGD,$
  \item $\gI \in \LtGI$\optodo{Understand why these are the function space requirements}
\item $n \in \LiDRR$ such that $\supp\mleft(1-n\mright) \compcont D$ and there exist $0 < \nmin < \nmax < \infty$ such that
  \beqs
\nmin \leq n(\bx) < \nmax \tfae \bx \in D,
  \eeqs
\item $A \in \LiDRRdtd$ such that $\supp\mleft(I-A\mright)\compcont D$,\ednote{Euan---in hetero, this requirement is instead phrased `$\dist(\supp(I-A),\GI) > 0'$. Is there any reason for phrasing it that way?} $A$ is symmetric, and there exist $0 < \Amin < \Amax < \infty$ such that
  \beqs
\Amin \abs{\bxi}^2 \leq \mleft(A(\bx) \bxi \mright) \cdot \bxibar < \Amax \abs{\bxi}^2 \tfa \bxi \in \CCd \tfae \bx \in D,
  \eeqs
  \eit
  we say $u \in \HoD$ satisfies the \defn{truncated exterior Dirichlet problem} if
  \beqs
\grad \cdot \mleft(A \grad u\mright) + k^2 n u = -f \tin D,
\eeqs
\beqs
\trGD u = \gD, \tand
\eeqs
\beq\label{eq:ibc}
\trGI \dn u - ik\trGI u = \gI.
\eeq
\eprob
Observe that, by construction, $\partial D = \GI \cup \GI$ and $\GD \cap \GI = \emptyset.$

Whilst the impedance boundary condition \eqref{eq:ibc} does not exactly mimic the Sommerfeld radiation condition \eqref{eq:sommerfeld}, the solutions of \cref{prob:tedp} are still `wave-like', and we will see below that solutions of \cref{prob:tedp} posses many of the same properties as solutions of \cref{prob:edp}. We also note that a common Helmholtz model problem in the numerical-analysis community is the \defn{interior impedance problem}, which is simply \cref{prob:tedp} in the case $\Dm = \emptyset.$

In order to approximation \cref{prob:edp,prob:tedp} by the finite-element method, we must instead work with the following 'variational forms' or 'weak forms' of \cref{prob:edp,prob:tedp}. For simplicity of exposition, we state both variational forms in the case $\gD = 0,$ although these can be generalised to the case $\gD \neq 0.$
\optodo{Insert ball notation somewhere}
\optodo{Define DtN}
\optodo{Define duality pairing}

\optodo{Fix cleverref referring to Problems as Theorems}


  \subsection{Well-posedness and bounds}\label{sec:wpbounds}

  We will now recap the well-posedness results and a priori bounds for \cref{prob:edp,prob:tedp} from \cite{GrPeSp:19}; these results will, in particular, be crucial for proving well-posedness results and a priori bounds for the stochastic analogues of \cref{prob:edp,prob:tedp} in \cref{chap:stochastic}. The novelty of these results is that they hold independently of $k,$ and that the a priori bounds we prove are explicit in $A$ and $n$. The explicitness in $A$ and $n$ is necessary in order to prove a priori bounds for stochastic $A$ and $n$, and we prove the results under conditions on $A$ and $n$ that are, in some sense `nontrapping'. Informally, a medium is `nontrapping' if all rays travelling through the medium escape in a uniform time; this definition, and the sense in which our conditions are `nontrapping', is discussed in \cref{sec:wpdisc} below.

  We first define the classes of $A$ and $n$ for which we will prove well-posedness results and a priori bounds.% Since we will need to consider the classes on both finite and infinite domains (for \cref{prob:tedp,prob:edp} respectively), we first define the classes for finite domains, before using this definition to define the classes for infinite domains.

  \optodo{Define star-shaped w.r.t. a point, ball}

%%   \bde[Class of nontrapping media on a finite domain]\label{def:NTfinite}
%% Let $\Dm$ be star-shaped with respect to the origin, and let $D$ be as in \cref{prob:tedp}. Let $A \in \CzoDclosRRdtd, n \in \CzoDclosRR$ and $\muo, \mut > 0.$ We say that 
%%   \ede
  
\bde[Class of nontrapping media]\label{def:NT}
Let $A \in \CzoDpclosRRdtd, n \in \CzoDpclosRR,$ and $\muo, \mut > 0.$ We say that $A \in \NTADp{\muo}$ if
\beqs
A(\bx) - \mleft(\bx \cdot \grad\mright)A(\bx) \geq \muo
\eeqs
in the sense of quadratic forms for almost every $\bx \in \Dp$. We say that $n \in \NTnDp{\mut}$ if
\beqs
n(\bx) + \bx \cdot \grad n(\bx) \geq \mut
\eeqs
for almost every $\bx \in \Dp.$

If $D$ is as in \cref{prob:tedp}, then we define $\NTAD{\muo}$ and $\NTnD{\mut}$ analagously.
\ede
\ednote{Both---Can you think of better notation for these conditions? I'm not overly keen to write $\mathrm{NT}_{\mathrm{A}}$ etc., as there is potentially confusion between then function $A$ and the $\mathrm{A}$ is the subscript. Or am I being overcautious?}

We can now prove well-posedness results and a priori bounds for the Helmholtz equation in the class of heterogeneous media we have just defined.

\bth[Well-posedness and bound for the EDP]\label{thm:edp}
If $\Dm, A, n,$ and $f$ satisfy the requirements in \cref{prob:edp}, $\Dm$ is star-shaped with respect to the origin, there exists $\muo, \mut > 0$ such that $A \in \NTADp{\muo}$ and $n \in \NTnDp{\mut}$, and $\gD = 0,$ then the solution of \cref{prob:edp} exists and is unique. Furthermore, given $R>0$ such that $\supp\mleft(I-A\mright),$ $\supp(1-n),$ and $\supp f$ are compactly contained in $\DR,$ then
\beqs
\muo \NLtDR{u}^2 + \mut k^2 \NLtDR{\grad u}^2 \leq \Co \NLtDR{f}^2,
\eeqs
for all $k>0,$ where
\beqs
\Co \de 4\mleft(\frac{R^2}{\muo} +\frac1{\mut}\mleft(R + \frac{d-1}{2k}\mright)^2\mright).
\eeqs
\enth

For the proof of \cref{thm:edp}, see \cite[Theorem 2.5]{GrPeSp:19}.

One can prove an analagous result to \cref{thm:edp} for \cref{prob:tedp} using the same techniques as in the proof of \cref{thm:edp}. However, the statement of the theorem is slightly more complicated due to the presence of the impedance boundary $\GI$, and its effect on the solution.

\bth[Well-posedness and bound for the TEDP]\label{thm:tedp}
If $\Dm, A, n, f,$ and $\gI$ satisfy the requirements in \cref{prob:tedp}, $\Dm$ is star-shaped with respect to the origin, $\Dtilde$, is star-shaped with respect to a ball, there exists $\muo, \mut > 0$ such that $A \in \NTAD{\muo}$ and $n \in \NTnD{\mut}$, and $\gD = 0,$ then the solution of \cref{prob:tedp} exists and is unique. Let:
\bit
\item $\LI \de \max_{\bx \in \GI} \abs{\bx}$ and
\item $a\LI$ be the radius of the ball with respect to which $\Dtilde$ is star-shaped.
    \eit
Then
\begin{multline*}
  \muo \NLtD{u}^2 + \mut k^2 \NLtD{\grad u}^2 + a\LI\NLtGI{\gradGI \trGI u}^2 + 2\LI k^2 \NLtGI{\trGI u}^2\\
  \leq \Ct \NLtDR{f}^2 + \Cttilde \NLtGI{\gI}^2
\end{multline*}
\optodo{Check that hetero grad boundary notation isn't hiding anything}
for all $k>0,$ where $\gradGI$ is the surface gradient on $\GI,$
\beqs
\Ct \de 4\mleft(\frac{\LI^2}{\muo} + \frac1{\mut}\mleft(\beta + \frac{d-1}{2k}\mright)^2\mright),
\eeqs
\beqs
\Cttilde \de 2\mleft(2\mleft(1+\frac2a\mright) + \frac\beta{\LI} + \frac{\mleft(d-1\mright)^2}4\mright)\LI,
\eeqs
and
\beqs
\beta \de \LI \mleft(2+\frac1{\mleft(k\LI\mright)^2} + 2\mleft(1+\frac2a\mright)\mright).
\eeqs
\enth

Observe that the above results are stated only in the case that $\gD = 0$. Whilst there is no mathematical difficulty in proving analagous results in the case $\gD \neq 0,$ the calculations in this case are more involved, as one must consider the surface gradient on the Dirichlet boundary, and this surface gradient depends on $A.$ In the case $A=I,$ these calculations are significantly simplified, and so in the case $A=I$ and $\gD \neq 0$ analagous results to \cref{thm:edp,thm:tedp} are proved in \cite[Theorem 2.19(ii)]{GrPeSp:19} (for \cref{prob:edp}) and \cite[Theorem A.6(iv)]{GrPeSp:19} (for \cref{prob:tedp}).

We highlight that these results are significant for the following two reasons.
\bit
\item These are the first $k$-explicit bounds on the solution of the Helmholtz equation in the case where both $A$ and $n$ are heterogeneous. As will discussed in more detail in \cref{sec:wpdisc} below, previous results were either not $k$-explicit, or did not have $A$ \emph{and} $n$ varying. The $k$-explicitness of these results is crucial for understanding how the solution of the Helmholtz equation (and numerical methods for its approximation) behave for large $k.$
  \item These are the first bounds explicit in $A$ and $n$ where the bound and the restrictions on $A$ and $n$ are independent of $k.$ Such bounds are crucial for the rigorous analysis of the analagous stochastic problems, and previous results in the literature only proved such bounds by imposing conditions on $A$ and $n$ that became more stringent as $k \rightarrow \infty;$ again, this will be more fully discussed in \cref{sec:wpdisc} below.
\eit

We remark that \cref{thm:edp,thm:tedp} are extended to wider classes of heterogeneous $A$ and $n$ and to the case $\gD \neq 0$ in \cite{GrPeSp:19}. As stated above, the case $\gD \neq 0$ (with $A=I$) is treated in \cite[Theorem 2.19(ii)]{GrPeSp:19} (for \cref{prob:edp}) and \cite[Theorem A.6(iv)]{GrPeSp:19} (for \cref{prob:tedp}), and the case $n=1$ is covered in \cite[Theorem 2.19(i)]{GrPeSp:19} (for \cref{prob:edp}) and \cite[Theorem A.6(ii)]{GrPeSp:19}. We highlight that when $A=I$ or $n=1$ the conditions on the other coefficient can be slightly weakened from those in \cref{def:NT}. When $A$ and $n$ are discontinuous, \cite[Condition 2.6]{GrPeSp:19} gives analogues of the conditions in \cref{def:NT}, and then the result corresponding to \cref{thm:edp} is proved in \cite[Theorem 2.7]{GrPeSp:19}. Letting $A$ and $n$ be $L^\infty$-perturbations of nontrapping media is discussed in \cite[Remark 2.15]{GrPeSp:19}, and relaxing the Lipschitz assumption on $\GD$ is outlined in \cite[Remark 2.13]{GrPeSp:19}, with the caveat that when $\GD$ is non-Lipschitz, we instead formulate \cref{prob:edp} as a variational problem, which is discussed in \cref{sec:varform} below. The above extensions and generalisations all can also be applied to \cref{prob:tedp}, as mentioned in \cite[p. 2916]{GrPeSp:19}.

\subsection{Discussion of results on well-posedness and a priori bounds for the Helmholtz equation}\label{sec:wpdisc}
\bit
\item Will review history of well-posedness and a priori bounds for Helmholtz
\item Will contrast with stationary diffusion (becuase of UQ)
  \eit

  \bit
\item Well-posedness $=$ existence, uniqueness, a priori bound (continuous dependence)
\item Want bound explicit in $k$ and parameters
  \item Also $k$- and parameter-explicit bounds needed for $k$- and $parameter$-explicit analysis of numerical methods
  \item And to understand how medium affects wave propagation
    \item A lot more subtle than stationary diffusion
  \eit

Recall for stationary diffusion\optodo{PUT SOMEWHERE ELSE?}:
\bit
\item Variational formulation has bounded and coercive form
\item Lax--Milgram gives existence, uniqueness, a priori bound explicit in coeffs
\eit
  
Helmholtz:
\bit
\item Above approach doesn't work---not coercive for large $k$
\item But do have G\r{a}rding - indentity $+$ compact
\item So use Fredholm alternative to show uniqueness $\implies$ existence and bound (bound not explicit in $k$)
\item Show uniqueness via
\bit
\item UCP - but don't get a priori bound explicit in $k,$ parameters\optodo{Why does SRc give uniqueness for homogeneous problems?}
\item Proving a priori bound directly - get uniqueness
\eit
\eit
\bit
\item Will now talk about bounds
\item Focus on $k$-dependence (and later) parameter-dependence
\item Long history of proving bounds on Helmholtz from 'analysis' community \cite{Bl:73,Va:75,BlKa:77,Bu:98,PeVe:99,PoVo:99a,PoVo:99b,CaPoVo:99,Bu:02,Be:03,Ca:12,CaLePa:12,NgVo:12,Sh:18}\optodo{Check each and every reference}\optodo{Check about Morawetz} and recent interest from NA \cite{Me:95,CuFe:06,He:07,EsMe:12,Sp:14,FeLiLo:15,Ch:15,BaSpWu:16,BrGaPe:17,BaChGo:17,SaTo:18,OhVe:18,GrSa:18,GaSpWu:18,GrPeSp:19,MoSp:19}
\item Worst-case scenario - bounds grows exponentially in $k$
\bit
\item If everything $C^\infty$, shown for $n=1$ in \cite{Bu:02} and if $A=1$ and $n$ Lipschitz in \cite{Sh:18}. For both jumping across same $C^\infty$ interface in \cite{Bu:98}.
\item Graham-Sauter \cite{GrSa:18} show exponential blow up in (properties of) coefficients, not $k$ in 1-D\ednote{Ivan, can we chat about your work with Stefan to check I've understood it correctly?}
\item \cite{MoSp:19} - numerical evidence for super-algebraic growith with trapping jumps - proved in \cite{PoVo:99a,PoVo:99b,CaPoVo:99}.
\eit
\item But only get exponential blow up when problem is trapping
\item When everything $C^\infty,$ can define rays (for penetrable and impenetrable obstacle) and thus define nontrapping
\item Nontrapping gives $1/k$ cut-off resolvent bound
\item Will get cases later where things aren't so smooth, but still get cut-off resolvent bound
  \item Call these nontrapping
\item Trapping of rays (angle of incidence = angle of reflection) stay for arbitrarily long time
\item Can get trapping if there are jumps in one direction, but not in other direction - \cite{MoSp:19}
\item Actually, trapping only happens at very few frequencies-\cite{MoSp:19,LaSpWu:19}
\item Related to idea of `resonant frequency'
\item Can be caused by obstacles or variations in speed - c.f. total internal reflection
\item If problem non-trapping (i.e. rays escape) then get bound independent of $k$ - see hetero\optodo{Find a reference for this}\optodo{iff?}
\item Therefore results proving $k$-indep bounds under some conditions can be seen as providing sufficient conditions for problem to be nontrapping.
\item Such results use either:
\bit
\item Multiplier techniques going back to Morawetz (and Rellich) \optodo{List results, demarcating homo and hetero} which give $k$-explicit bound directly, or
\item more technical microlocal tools, which give nontrapping, from which you can conclude bound\optodo{Look at Vainberg/Melrose--Sj\"ostrand}
\item Recent work on bounds explicit in parameters---motivated in part by UQ - \cite{FeLiLo:15,GrPeSp:19,PeSp:18} prove bounds under conditions that ensure nontrapping, \cite{GaSpWu:18} assume problem is nontrapping and get `length of longest ray'.
\eit
\item In conclusion, subtle how things depend on $k,$ coefficients, and more difficult than stationary diffusion.
\eit

\bit
\item Now shift attention to NA of Helmholtz, specifically, FEM.
  \item Will prove new result on error bounds for Helmholtz in hetero media, generalises homo media result.
  \eit

\section{Theory of the Discretisation of the Helmholtz Equation}\label{sec:helmfe}

  \subsection{Variational Formulations for the Helmholtz equation}\label{sec:varform}

\optodo{Chat}
  
\bprob[Variational formulation of EDP when $\gD = 0$]\label{prob:vedp}
Let $\Dp, A, n,$ and $f$ be as in \cref{prob:edp}. Choose $R>0$ such that $\supp f, \supp(I-A), \supp(1-n) \compcont \BR,$ and define $\DR \de \Dp \cap \BR.$

We say $u \in \HozDDR$ satisfies the \defn{variational formulation of the exterior Dirichlet problem} with $\gD = 0$ if
\beqs
\aE(u,v) = \FE(v) \tfa v \in \HozDDR,
\eeqs
where
\beqs
\aE(w,v) \de \int_{\DR} \mleft(\IPRRd{A \grad w}{\grad v} - k^2 n w \vbar\mright) - \DPGR{\TR \trGR w}{\trGR v}
\eeqs
and
\beqs
\FE(v) \de \int_{\DR} f\vbar
\eeqs
\eprob

\ble[Equivalence of formulations for the EDP]\label{lem:edpform}
\Cref{prob:edp,prob:vedp} are equivalent, i.e., if $u \in \HolocDp$ solves \cref{prob:edp} then $u\restrict_{\DR} \in \HozDDR$ (for $R$ as in \cref{prob:vedp}) and $u\restrict_{\DR}$ solves \cref{prob:vedp}.\optodo{Sort restriction notation}
\ele

For a proof of \cref{lem:edpform}, see \cite[Lemma 3.3]{GrPeSp:19}.

\bprob[Variational formulation of TEDP when $\gD = 0$]\label{prob:vtedp}
Let $D, A, n, f,$ and $\gI$ be as in \cref{prob:tedp}. We say $u \in \HozDDR$ satisfies the \defn{variational formulation of the truncated exterior Dirichlet problem} with $\gD = 0$ if
\beqs
\aT(u,v) = \FT(v) \tfa v \in \HozDDR,
\eeqs
where
\beqs
\aT(w,v) \de \int_{\DR} \mleft(\IPRRd{A \grad w}{\grad v} - k^2 n w \vbar\mright) - ik\int_{\GI} \trGI w\trGI \vbar
\eeqs
and
\beqs
\FT(v) \de \int_{\DR} f\vbar + \int_{\GI} \gI \trGI \vbar
\eeqs
\eprob
\ednote{Both---can we discuss whether it's worth stating the variational formulation of the EDP/TEDP with non-zero Dirichlet data? On the one hand, it's more technical, but on the other hand, if we want (e.g. for MLMC) to consider the far-field as a quantity of interest, it makes most sense to consider a `real' scattering problem, i.e., one with non-zero Dirichlet data.}

\ble[Equivalence of formulations for the TEDP]\label{lem:tedpform}
\Cref{prob:tedp,prob:vtedp} are equivalent, i.e., $u \in \HozDDR$ solves \cref{prob:tedp} if, and only if, $u$ solves \cref{prob:vtedp}.
\ele

For a proof of \cref{lem:tedpform}, see \cite[Lemma A.7]{GrPeSp:19}.
  
\subsection{Finite-element theory}\label{sec:fetheory}

Now give requisite background finite-element theory

\bit
\item Definition - Finite element/Finite element space
\item Definition - Triangulation
\item Definition - Mesh size
\item Definition - Nodal Finite element
\item Definition - Fe space (of order $p$) associated with a triangulation
\item Proof of $H^m$ regularity when $f$ etc. is smooth enough\optodo{Why did you think you needed more than $H^2$?}
\item Definition - Scott--Zhang quasi-interpolant
\item Lemma - approximation properties of Scott--Zhang quasi-interpolant
\item Definition - Finite-element problem
\item Remark - won't be considering variational crimes, like approximating geometry\optodo{At least understand this}
\eit

\subsection{Error bound for the heterogeneous IIP}\label{sec:errbound}
\bit
\item This section contains new work
\item Prove heterogeneous analogue of results on bounded error if $h \lesssim k^{-(2p+1)/2p}$
\item Proof technique uses `elliptic projection' idea - see Feng \& Wu, Wu, Du \& Wu, Chaumont-Frelet \& Nicaise, WU \& Zou.
\item Prove for IIP only, as need to use results on a related PDE from Chaumont-Frelet and Nicaise; only proved with impedance boundary condition - expect can extend to, e.g., exact DtN.
\item These bounds crucial for MLMC analysis in \cref{chap:mlmc}
\item Will prove for non-zero Dirichlet boundary condition, as motivation for MLMC is scattering
  \eit
  Include:
  \bit
\item G\r{a}rding inequality
  \eit

\subsection{Discussion of FEM for Helmholtz}\label{sec:helmfedisc}
\bit
\item Recall from intro $h\sim k^{-1}$ (for first order; fixed ppw as num. points $\sim 1/h$) means interpolation error bounded, in general fixed ppw, but condition varies.
\bit
\item Can see by interpolation bounds from, e.g., Scott-Zhang (see below)
\item In 1-D\optodo{Find out if it works in higher D} motivate also by Shannon--Nyquist theorem\optodo{Find Shannon Proc I. R. E., 31:10-21, '49 Communication in the prescence of noise}
\item Both make precise intuition that we have bounded errors under fixed points per wavelength
\eit
\item Also recall Helmholtz suffers from pollution - fixed ppw not enough to keep FEM error bounded
\item This section will summarise conditions under which error bounded, and related property of quasi-optimality holds
\item Plan:
\bit
\item Introduce concepts of bounded error and quasi-optimality
%\item Show (for contrast) that they are straightforward for stationary diffusion
\item Give overview of results on bounded error and Quasi-optimality for Helmholtz
\eit
\eit
\bit
\item Definition: error bounded
\item Would really like relative error bounded, however, not possible with current technology
\item Definition: Quasi-optimal
\item Natural question: what mesh size keeps error bounded?
\item Short history:
\bit
\item Bayliss, Goldstein, Turkel - computations showing $h\sim k^{-3/2}$ sufficient to keep error bounded
\item Ihlenburg, Babu\v{s}ka - prove error $\sim h^2k^3$ in 1-D if $h \sim k^{-1}$
\item Wu - error $\sim h^2k^3$ if $h \sim k^{-3/2}$ for 1st-order FEM
\item Du and Wu - error $\sim h^{2p}k^{2p+1}$ for higher-order FEM
\item Chaumont-Frelet \& Nicaise - similar result with corner singularities
  \item WU \& Zou - special case in heterogeneous media
\item Here in \cref{sec:errbound} first proof that $h\sim k^{-(2p+1)/2p}$ sufficient to keep relative error bounded in heterogeneous problems
\eit
\item For stationary diffusion, things are straightforward
\item Prove quasi optimality using C\'ea
\item Get FEM error bounded by using properties of interpolant, e.g., via Scott-Zhang.
\item No condition on $h$ needed, unlike Helmholtz
\item For Helmholtz Cea unavailable as not coercive
\item Proving Quasi-optimality more tricky
\item Instead use Aubin--Nitsche duality argument
\bit
\item Introduced by Nitsche\optodo{FOr what?}
\item Used by Aubin\optodo{For what}
\item Applied by Schatz to problems satisfying G\r{a}rding inequality
\item First applied to Helmholtz by Melenk
\item Obtain quasi-optimality under nontrapping assumptions and restrictive mesh condition $h \sim k^{-2}$---see \cite[Proposition 8.2.7]{Me:95} for homogeneous, \cite{GrSa:18} and \cite[Theorem 3]{GaSpWu:18} for heterogeneous, related bound in \cite[Section 1.4]{ChSpGiSm:17} sketched for parabolic trapping cases
\item Mesh condition is prohibitive for large $k$
\eit
\eit
Conclusion:
\bit
\item Rigorous NA of Helmholtz more demanding than, e.g., stationary diffusion
\item Mesh conditions require decreasing mesh size with $k$; esp. for quasi-optimality, very restrictive.
\eit


\optodo{Put size of linear systems in}
\optodo{Put in poor performance of GMRES?}
\optodo{Put in need for preconditioners?}


%%5 Can put the below somewhere %%

%% To keep the finite-element error bounded when solving \eqref{eq:introdet}, one must over-refine the numerical grid. That is, rather than using a fixed number of points per wavelength, one must increase the number of points per wavelength as $k$ increases. To achieve bounded finite-element error, one must refine the finite-element mesh size $h$ like $k^{-3/2}$. Whilst this result has been known numerically for some time, it was proven for \eqref{eq:introdet} only with constant coefficients (on various domains and for various finite-element spaces) in \cite{IhBa:95a,Wu:14,DuWu:15,ChNi:18}, and the first proof (to our knowledge) for \eqref{eq:introdet} with heterogeneous coefficients  is contained in \cref{chap:background}. Choosing $h \sim k^{-3/2}$ means \eqref{eq:intromat} is a linear system of size $k^{3d/2}$, larger than if one merely wants the interpolation error to be bounded. Hence, requiring a bounded finite-element error gives rise to very large linear systems.

%% More briefly, if one wants the finite-element solution to be quasi-optimal (that is, up to a constant, the finite-element solution is the best approximation in the finite-element space), then one must over-refine even more, and take $h \sim k^{-2}$. This mesh condition will give rise to linear systems with $k^{2d}$ degres of freedom. See \cref{chap:background} for further details on the necessity of this mesh condition, and further discussion of all the mesh conditions discussed above.\optodo{EDIT THE ABOVE TO REMOVE `OVER-REFINE'}



\section{Theory of the Finite-Element Discretisation of the Helmholtz Equation}\label{sec:helmfe}

We now shift our attention to the numerical analysis of the Helmholtz equation in heterogeneous media; in particular, we  study the finite-element method for the Helmholtz equation. We first state the variational formulations of the Helmholtz equation, define the finite element method, and recall results on the approximation properties of finite-element spaces. we then prove our main result, error bounds for the finite-element method for the heterogeneous Helmholtz equation, where the bounds hold for arbitrary (fixed) degree finite elements, and are explicit in $n$ and (in principle) $A$.

a new error bound for the finite-element method for the Helmholtz equation in heterogeneous media.

  \subsection{Variational Formulations for the Helmholtz equation}\label{sec:varform}
  The finite element method is based on the variational formulation of the Helmholtz equation; for simplicity of exposition, we state the variational formulation of \cref{prob:edp,prob:tedp} in the case $\gD = 0,$ although these can be generalised to the case $\gD \neq 0.$
  
\bprob[Variational formulation of EDP when $\gD = 0$]\label{prob:vedp}
Let $\Dp, A, n,$ and $f$ be as in \cref{prob:edp}. Choose $R>0$ such that $\supp f,$ $\supp(I-A),$ $\supp(1-n) \compcont \BR,$ and define $\DR \de \Dp \cap \BR.$

We say $u \in \HozDDR$ satisfies the \defn{variational formulation of the exterior Dirichlet problem} with $\gD = 0$ if
\beqs
\aE(u,v) = \FE(v)\quad \tfa v \in \HozDDR,
\eeqs
where
\beq\label{eq:aedp}
\aE(w,v) \de \int_{\DR} \mleft(\mleft(A \grad w\mright)\cdot\grad \vbar - k^2 n\minispace w \vbar\mright) - \DPGR{\DtN \trGR w}{\trGR v}
\eeq
and
\beq\label{eq:Ledp}
\FE(v) \de \int_{\DR} f\minispace\vbar,
\eeq
where $\DtN:\HhGR\rightarrow \HmhGR$ is the Dirichlet-to-Neumann map for the homogeneous Helmholtz equation $\Lap u + k^2 u = 0$ combined with the Sommerfeld radiation condition in the exterior of $\BR$; and $\DPGR{\cdot}{\cdot}$ is the duality pairing on $\GR.$
\eprob

\ble[Equivalence of formulations for the EDP]\label{lem:edpform}
\Cref{prob:edp,prob:vedp} are equivalent. If $u \in \HolocDp$ solves \cref{prob:edp} then $u\restrict_{\DR} \in \HozDDR$ and $u\restrict_{\DR}$ solves \cref{prob:vedp}  (for $R$ as in \cref{prob:vedp}). Conversely, if $u \in \HozDDR$ solves \cref{prob:vedp}, then $u$ solves \cref{prob:edp}, if $u$ is extended to $\HolocDp$ by the solution of the exterior Dirichlet problem (in the exterior of $\DR$) for the homogeneous Helmholtz equation with the Sommerfeld radiation condition (with Dirichlet data $\tr u$ on $\partial \BR$).
\ele

For a proof of \cref{lem:edpform}, see \cite[Lemma 3.3]{GrPeSp:19}.

\bprob[Variational formulation of TEDP when $\gD = 0$]\label{prob:vtedp}
Let $D, A, n, f,$ and $\gI$ be as in \cref{prob:tedp}. We say $u \in \HozDDR$ satisfies the \defn{variational formulation of the truncated exterior Dirichlet problem} with $\gD = 0$ if
\beq\label{eq:vtedp}
\aT(u,v) = \FT(v) \tfa v \in \HozDDR,
\eeq
where
\beq\label{eq:aT}
\aT(w,v) \de \int_{D} \big(\mleft(A \grad w\mright)\cdot\grad \vbar - k^2 n\minispace w \vbar\big) - ik\int_{\GI} \trGI w\minispace\trGI \vbar
\eeq
and
\beqs
\FT(v) \de \int_{D} f\minispace\vbar + \int_{\GI} \gI \minispace\trGI \vbar
\eeqs
\eprob

\ble[Equivalence of formulations for the TEDP]\label{lem:tedpform}
\Cref{prob:tedp,prob:vtedp} are equivalent, i.e., $u \in \HozDDR$ solves \cref{prob:tedp} if, and only if, $u$ solves \cref{prob:vtedp}.
\ele

For a proof of \cref{lem:tedpform}, see \cite[Lemma A.7]{GrPeSp:19}.
  
\subsection{Background Concepts in Finite-Element Theory}\label{sec:fetheory}

We now give a brief summary of elementary concepts in finite-element theory. In the interest of brevity, we focus only on those concepts that we be needed to
prove the new error bound for finite-element discretisations of the Helmholtz equation in \cref{sec:fem} below.

Throughout this thesis, we use the terms `mesh` and `triangulation' to refer to a triangulation of simplices in the sense of Ciarlet \cite[Paragraphs (FEM1) p. 61 and ($\cT_{h}$5) p. 71]{Ci:91}. Similarly, we use $h$ to denote the mesh size of a mesh\footnote{Frequently in this thesis we will consider families of meshes indexed by their mesh size $h$.}, and we use $\Vhp$ to denote the set of continuous piecewise-polynomials of degree $p$ on a mesh with mesh size $h$.
  %%   triangulation of the domain. (Note that in this thesis we use the words `mesh' and `triangulation' interchangably; strictly speaking, the term `triangulation' only makes sense in 2-d, but we ignore this technicality.

  %%   \bde[Triangulation]
  %%   A triangulation of a polygonal domain $D$ is a finite collection of sets $\Ki \subseteq \Dclos$ such that
  %%   \ben
  %% \item $\interior{\Ki} \cap \interior{\Kj} = \emptyset$ for $ i \neq j,$
  %% \item $\bigcup_i \Ki = \Dclos,$ and
  %%   \item Something about triangles that works in 3-D too\opntodo{Find ref - Brezzi/Johnson? Make sure to use simplical meshes.}.
  %%   \een
  %%   \ede\opntodo{Change to definition in Ciarlet, Handbook of numerical analysis}

%% \bde[Mesh Size]
%% Given a mesh $\cT = \set{\Ki}$ on a domain $D$, the \defn{mesh size} of $\cT$ is defined to be
%% \beqs
%% h = \max_{\Ki} \diam{\Ki}.
%% \eeqs
%% \ede

%%     We can now define finite-element spaces associated with a triangulation; we first define the space of polynomials on a set.

%% \bde[Set of polynomials]
%% For $K \subseteq \RRd,$ $\polyp{K}$ is the set of polynomials defined on $K$ with total degree at most $p.$
%% \ede
%%     \bde[Finite element space of degree $p$]\label{def:fespace}
%%     Given a triangulation $\set{\Ki}$ with mesh size $h$ of a domain $D$, the \defn{(continuous) piecewise-polynomial finite-element space of degree $p$} associated with $\set{\Ki}$ is
%%     \beqs
%% \Vhp \de \set{\vh : D \rightarrow \CC \st \vh \in \CzD \tand \vh\restrict_{\Ki} \in \polyp{\Kiclos} \tforall \Ki }.
%%     \eeqs
%%     \ede

Throughout this thesis, we only consider the $h$-finite-element method, i.e., the degree $p$ of the polynomials associated with the space is assumed fixed, and we consider refining $h.$ This is in contrast to the $p$-finite-element method, where $h$ is fixed and $p$ is increased, and the $hp$-finite-element method, where both $h$ and $p$ are increased according to some rule. $hp$-finite-element methods for the homogeneous Helmholtz equation were analysed by Melenk and Sauter in \cite{MeSa:10,MeSa:11}; however, their analysis relies on a fully $p$-explicit analysis of the best approximation error of the Helmholtz equation in $hp$-finite-element spaces; such analysis is currently not possible for the heterogeneous problem. Throughout this \lcnamecref{sec:fetheory} we use finite-element method to refer to the $h$-finite-element method.
%\ednote{I couldn't immediately see in these results that one obtains exponential convergence with respect to the number of DOFs---have I missed something, or is this result contained somewhere else?}, where they showed one can obtain an exponential rate of convergence with respect to the number of degrees of freedom.
%\opntodo{Get Schwab's hp book and look this up}

%    We adopt the terminology that $\vhptilde$ is the \defn{quasi-interpolant} of $v.$ For a proof of \cref{lem:scottzhang} see, e.g., \cite[Corollary 4.4.24, Remark 4.4.27]{BrSc:08}.

%%     \bre[The function $\vhptilde$]
%% The function $\vhptilde$ in \cref{lem:scottzhang} can be constructed using `averaged Taylor polynomials', see, e.g., \cite{ScZh:90},\cite[Section 4.4]{BrSc:08} for details\footnote{Observe that in \cite{BrSc:08}, the authors use different notation to us; they use $m$ to denote the regularity of $v$ and $p$ to denote the integrability of $v,$ i.e., in \cite{BrSc:08}, $v \in \WmpD.$}.
%%     \ere

    With the concept of a finite-element space in place, we can now define the finite-element approximation to the variational problems \cref{prob:vedp,prob:vtedp}.

    \bprob[Finite-element approximation of \cref{prob:vtedp}]\label{prob:fevtedp}
    We say that $\uh \in \Vhp$ is the \defn{finite-element approximation of $u$} (the solution to \cref{prob:vtedp}) if
    \beqs
    \aT(\uh,\vh) = \FT(\vh) \tforall \vh \in \Vhp.
    \eeqs
    The finite-element approximation of \cref{prob:vedp} is defined analagously.
    \eprob

    The following \lcnamecref{lem:scottzhang} shows how the approximation properties of the space $\Vhp$ depend on $h$ and $p$:
    \ble[Existence of best approximation]\label{lem:scottzhang}
    Let $v \in \HmD$, $m \geq 1,$ and $p \in \set{1,\ldots,m}$. Then there exist a constant $\CSZm>0$ independent of $v$ and $\vhptilde \in \Vhp$ (but dependent on $p$) such that

    \beqs
\NLtD{v - \vhptilde} \leq \CSZm h^m \NHmD{v}
    \eeqs
     and
    \beqs
\NHoD{v - \vhptilde} \leq \CSZm h^{m-1} \NHmD{v} \tfor m \geq 1.
    \eeqs
    \ele

    
    \bre[Not considering variational crimes]
    Throughout this thesis we assume that \cref{prob:fevtedp} is a conforming discretisation if \cref{prob:vtedp}, i.e. $\Vhp \subset \HozDDR.$ This inclusion is true if $\DR$ can be triangulated; otherwise, we must modify the definition of \cref{prob:fevtedp} and commit a variational crime by approximating the boundary of $\DR$ by a polygon, or introducing mesh elements with curved boundaries using interpolated boundary conditions or isoparametric finite elements. See, e.g., \cite[Chapter 10]{BrSc:08} for an overview of the additional errors introduced by variational crimes (although not in the context of the Helmholtz equation). In this thesis we  ignore such variational crimes, and the additional errors they induce; such analysis is standard. Therefore for the remainder of this thesis we assume that \cref{lem:scottzhang} holds, even if we have not triangulated the domain $D$ exactly.
    \ere
    

%%     \subsection{Error bound for the heterogeneous IIP}\label{sec:errbound}

%%     We now move on to present new work---bounds on the error $\uh-u$ between the finite-element approximation and the true solution of \cref{prob:vtedp}. These bounds, proven here for the Helmholtz equation in \emph{heterogeneous} media are generalisations of results already in the literature that the finite-element error in the $L^2$- and $H^1$-norms is bounded (as $k\rightarrow \infty$) provided $h \lesssim k^{-(2p+1)/2p}$. These results are crucial for our analysis of the multi-level Monte Carlo method for the Helmholtz equation in \cref{chap:mlmc}.

%% The proof of these results uses a so-called `elliptic projection' technique, introduced by Feng and Wu in \cite{FeWu:09} for Discontinuous Galerkin methods for the Helmholtz equation and modified by Du and Wu \cite{DuWu:15}, where the variational formulation of a PDE related to \cref{prob:vtedp} is used as part of the proof. We  only prove results for \cref{prob:fevtedp}, and not for the finite-element approximation of \cref{prob:vedp}, as our proof uses properties of the related PDE and these properties have only been proven with impedance boundary conditions (in the recent preprint \cite{ChNiTo:18}) and not with an exact Dirichlet-to-Neumann map on the truncation boundary. However, we imagine the results proven for the related PDE with an impedance boundary condition also hold for an exact Dirichlet-to-Neumann boundary condition, and so we anticipate that the results we prove here also hold for finite-element approximations of \cref{prob:vedp}.\opntodo{This paragraph needs rewriting---acknowledge Du \& Wu}

\subsection{Discussion of the finite-element method for the Helmholtz equation}\label{sec:helmfedisc}

We now discuss three properties of the $h$-finite-element method for the Helmholtz equation:
\bit
\item The relative error $\displaystyle \frac{\NHokD{u-\uh}}{\NHokD{u}}$ is bounded,
\item The error $\NHokD{u-\uh}$ is bounded in terms of norms of the data $f$ and $\gI$, and
  \item The finite-element method is quasi-optimal, $\displaystyle \NHokD{u-\uh} \lesssim \inf_{\vh \in \Vhp} \NHokD{u-\vh}.$
    \eit

    The key question we ask is: What mesh conditions lead to each of the three properties above? (I.e., how must one refine $h$ as $k$ increases in order for the finite-element method to satisfy each of the three properties above?)

    We will now summarise the state of the literature regarding each of the three properties above. We will:
    \ben
  \item Define each of the three properties formally,
  \item\label{it:sum2} State the sharpest results known in the literature,
  \item\label{it:sum3} Give a complete overview of all results in the literature concerning the three properties,and
    \item\label{it:sum4} Give a detailed discussion of proof techniques for each of the three properties.
      \een

We do \cref{it:sum2,it:sum3} because to our knowledge there is not a complete overview of all of these three properties anywhere in the literature. Moreover, research interest in these three properties has recently increased (e.g., the recent papers \cite{ChNi:18,LiWu:18,ChGaNiTo:18,ChNi:19} are all concerned with one or more of these properties, and have all been completed within the last two years), and so we believe it will be helpful and timely to provide a complete overview of the state of the literature. We do \cref{it:sum4} because collection of the proof techniques used for each of these properties are different to those used for standard finite-element proofs for, e.g., the stationary diffusion equation. Moreover, the different proof techniques are different but related, and we do not know of anywhere in the literature where these techniques are expounded, compared, and constrasted. In view of the resurgence in interest in the properties above, we hope that such a review will be valuable for the research community.

      

\paragraph{Intuition for fixed number of points per wavelength} Before discussing the three properties listed above, we first give a brief discussion of the commonly-used heuristic `take a fixed number of points (or elements) per wavelength'.  Recall from \cref{sec:numsolve} that if one takes the mesh size in the finite element method $h \sim 1/k$, then one expects the interpolation (or best approximation) error is bounded uniformly in $k$. More rigorously, as solutions of the Helmholtz equation typically\footnote{If $f$ and $\gD$ are independent of $k$,  combining \cref{thm:edp} and standard elliptic regularity results, e.g., \cite[Theorems 4.16 and 4.18]{Mc:00}, gives the fact that $\NHtD{u} \lesssim k$.} have $\NHtD{u} \sim k,$ one can bound the $H^1$-norm of the interpolation error if $h \sim 1/k$ using \cref{lem:scottzhang}.  As explained in \cref{sec:numsolve}, this restriction ensures there are a fixed number of discretisation points per wavelength of the solution, since the wavelength $\lambda = 2\pi/k.$

An alternative motivation for taking $h \sim 1/k$ is the Nyquist--Shannon sampling theorem (proved by Shannon in his seminal paper in information theory \cite[Theorem 1]{Sh:49}). The theorem states that any function $v$ (in 1-d) whose Fourier transform lies inside $[-k,k]$, for some $k>0$, is completely determined (via its Fourier series) by the point values $v(0)$, $v(\pm \mu),$ $v(\pm2\mu),  \ldots$, for any $\mu < 1/\mleft(2k\mright).$ Observe that $k$ is then the highest frequency present in $v$.  That is, if we interpolate $v$ at points spaced less than $ \lambda/(4\pi)=1/(2k)$ apart, where $\lambda = 2\pi/k$ is the wavelength associated with oscillations of frequency $k$, we can reconstruct $v$ from its Fourier transform. See, e.g., \cite[\S 5.21]{BaNaBe:00} for an explicit formula for reconstructioning $v$. Therefore, in the 1-d case we may reasonably expect that interpolating $v$ with points spaced less than $\lambda/(4\pi)$ apart will be a good approximation of $v$, as the Nyquist--Shannon theorem suggests such a spacing of points `captures' all of the oscillations in $v.$
%% The solution of the one-dimensional Helmholtz equation with constant coefficients is
%% \beq\label{eq:hh-1d}
%% u(x) = A \sin(kx) + B \cos(kx),
%% \eeq
%% (for some constants $A$ and $B$), and $\lambda = k/2\pi$ as $u$ has Fourier Transform
%% \beqs
%% \uhat(\xi) = \frac{A}{2i} \mleft(\delta\mleft(\xi-\frac{k}{2\pi}\mright) - \delta\mleft(\xi+\frac{k}{2\pi}\mright)\mright) + \frac{B}{2} \mleft(\delta\mleft(\xi-\frac{k}{2\pi}\mright) + \delta\mleft(\xi+\frac{k}{2\pi}\mright)\mright).
%% \eeqs
%% \opntodo{Find reference}
%% Therefore, if $u$ is sampled at regularly-spaced points that are strictly closer together than $\pi/k = 1/(2\lambda)$, then $u$ can be reconstructed perfectly from these samples.\opntodo{Consider writing something about mutliple dimensions} In conclusion, one expects the be able to interpolate the solution of the Helmholtz equation with uniform error in $k$ if $h \sim 1/k.$

%% However, as was stated in \cref{sec:numsolve}, the finite-element method for the Helmholtz equation suffers from pollution; and $h \sim 1/k$ is not sufficient to keep the finite-element error bounded as $k\rightarrow \infty.$ Therefore we  now provide an overview of previous work giving mesh conditions under which the finite-element error $\NW{u-\uh}$ is bounded as $k\rightarrow \infty$, as well as mesh conditions under which the finite-element method is quasi-optimal. Concerning errors, it would be more natural to consider the relative error $\NW{u-\uh}/\NW{u};$ and to investigate the conditions on $h$ which enable the relative error to be bounded as $k\rightarrow \infty.$ However, current technology typically only allows us to prove results for the error, not the relative error, and so we investigate the error instead.

%We will now briefly review the literature that seeks to answer the question `What mesh condition will keep the relative error bounded as $k \rightarrow \infty$?'


%% \bde[Bounded finite-element error]
%% Let $(\Th)_{h \in (0,1)}$ be a shape-regular family of triangulations of $\DR$, where we assume $h$ is a function of $k$ (e.g., $h(k) = 1/k$). We say that \defn{the error in the finite-element method is bounded uniformly in $k$} if for all $A,n$ in some class there exists $C(A,n)$ (but independent of $k$ and $h$) such that
%% \beqs
%% \NW{u-\uh} \leq C\mleft(\NLtD{f} + \NLtGD{\gD} + \NLtGI{\gI}\mright).
%% \eeqs
%% \ede



 %% In $d=2$ or $3$ with $D$ a convex polygon or polyhedron,  \cite[Theorem 6.1]{Wu:14} showed that the error (but not the relative error) in both the $H^1$ seminorm and the $L^2$ norm is bounded uniformly in $k$ if $h \lesssim k^{-3/2}$. The proof of \cite[Theorem 6.1]{Wu:14} showed an analagous result for a continuous interior-penalty method, and then took the limit as the penalty parameter tends to 0 to obtain the result for the continuous piecewise-linear finite-element method. In summary, the mesh condition $h \lesssim k^{-3/2}$ seems to be sufficient for the (relative) error to be bounded uniformly in $k.$

\subsubsection{Formal definition of properties of the finite-element method}\label{sec:femprops}

We now formally define the three properties mentioned above. These definitions may seem overly technical, however we believe their technical definition is necessary to capture the, at times, complex behaviour of finite-element methods for the Helmholtz equation. After stating the definitions, we give some examples of what they mean for finite-element discretisations of the Helmholtz equation. These definitions are based on \cite[Definition 2.3]{DiMoSp:19}.

In the following definitions, we consider the Helmholtz problems \cref{prob:vtedp}, and their discretisations \cref{prob:fevtedp} as parameterised by the wavenumber $k,$ that is, we write $u(k)$ for the solution of \cref{prob:vtedp} with wavenumber $k$, and $\uh(k)$ for its finite-element approximation.

\bde[$\hk{a}{b}$-accurate]\label{def:hkacc}
Given $a,b>0$, we that an $h$-finite-element method for the Helmholtz equation is \defn{$\hk{a}{b}$-accurate} if, given $0 < \eps < 1,$ and $\kz > 0,$ there exist $\Co(\kz), \Ct(\eps,\kz) > 0$ such that if
\beqs
hk^a \leq \Co
\eeqs
and
\beqs
hk^b \leq \Ct,
\eeqs
then the Galerkin solutions $\uh(k)$ satisfy
\beq\label{eq:hkacc}
\frac{\NHokD{u(k)-\uh(k)}}{\NHokD{u(k)}} \leq \eps
\eeq
for all $k \geq \kz.$

If $a=b$ we say the $h$-finite-element method is $\hka{a}$-accurate.
\ede

\bde[$\hk{a}{b}$-data-accurate]\label{def:hkdataacc}
Given $a,b>0$, we that an $h$-finite-element method for \cref{prob:vtedp} is \defn{$\hk{a}{b}$-data-accurate} if, given $0 < \eps < 1,$ and $\kz > 0,$ there exist $\Co(\kz), \Ct(\eps,\kz) > 0$ such that if
\beqs
hk^a \leq \Co
\eeqs
and
\beqs
hk^b \leq \Ct,
\eeqs
then the Galerkin solutions $\uh(k)$ satisfy
\beq\label{eq:hkdataacc}
\frac{\NHokD{u(k)-\uh(k)}}{\NLtD{f} + \NLtGI{\gI}} \leq \eps
\eeq
for all $k \geq \kz.$

If $a=b$ we say the $h$-finite-element method is $\hka{a}$-data-accurate.
\ede

Observe that the only difference between the definitions of $\hk{a}{b}$-accurate and $\hk{a}{b}$-data-accurate is that in \cref{eq:hkacc} the error is measured relative to the solution $u$, whereas in \cref{eq:hkdataacc} the error is measure relative to the `data' $f$ and $\gI$.

\bde[$\hka{a}$-quasi-optimal]\label{def:hkqo}
Given $a > 0,$ we say that an $h$-finite-element method for the Helmholtz equation is \defn{$\hka{a}$-quasi-optimal} if, given $\kz > 0$, there exist $\Co(\kz), \Cqo(\kz) > 0$ such that if
\beqs
hk^{a} \leq \Co,
\eeqs
then the Galerkin solutions $\uh(k)$ satisfy
\beqs
\NHokD{u(k) - \uh(k)} \leq \Cqo \inf_{\vh \in \Vhp} \NHokD{u(k) - \vh},
\eeqs
for all $k \geq \kz.$
\ede

\bre[Generalisations of $\hk{a}{b}$-data-accurate]
For discretisations of other Helmholtz problems (e.g., full-space problems with no truncation boundary $\GI$, or problems truncated with a perfectlly matched layer (PML)), then the denominator in \cref{eq:hkdataacc} should be adapted appropriately, e.g., the denominator should equal $\NLtD{f}$ for full-space problems and PML problems. In our discussion of the literature below (which includes such problems), we will make such an adaption without comment.
\ere

\bre[Relationship between $\hk{a}{b}$-accuracy and $\hk{a}{b}$-data-accuracy]
Observe that if an a priori bound such as \cref{eq:tedpbound} holds, then $\hk{a}{b}$-accuracy implies $\hk{a}b$-data-accuracy, as $\NHokD{u} \lesssim \NLtD{f} + \NLtGI{\gI}$.

However, in order to conclude $\hk{a}{b}$-accuracy from $\hk{a}b$-data-accuracy, one would need to show
\beq\label{eq:reversebound}
\NLtD{f} \lesssim \NHokD{u}.
\eeq
However, in general \cref{eq:reversebound} does not hold\ednote{Euan---can one write down an example of this \emph{easily}?}, and so one cannot, in general, conclude $\hk{a}{b}$-accuracy from $\hk{a}{b}$-data-accuracy.
\ere

\bre[The above conditions for heterogeneous problems]
Observe that for heterogeneous problems, the constants $\Co, \Ct$, and $\Cqo$ in \cref{def:hkacc,def:hkdataacc,def:hkqo} will all depend on the coefficients $A$ and $n.$ In general, this dependence is unknown. However, in our original work in \cref{sec:fem} below we prove that the $h$-finite-element method with polynomial degree $p$ is $\hka{(2p+1)/(2p)}$-data-accurate, where the dependence of $\Co$ and $\Ct$ on $n$ (and, in principle, $A$) is completely known.
\ere

To aid understanding of the above definitions, and to see how they connect with the usual presentation of finite-element results for the Helmholtz equation, we give the following illustrative example.

\bre[Example achieving $\hk{a}{b}$-data-accuracy]\label{rem:dataacc}
In \cite[Theorem 5.4]{ChGaNiTo:18}, Chaumont-Frelet, Gallistl, Nicaise, and Tomezyk proved\footnote{The result proved in \cite[Theorem 5.4]{ChGaNiTo:18} is slightly more complicated than that stated here; the right-hand side of \cref{eq:hkacceg} should be $\Cth \mleft(hk + h^{2p}k^{2p+1}\mright)\NLtD{f}.$ However, an analagous proof to the one presented here holds in the more complicated case.} that if $k \geq \kz$ and $h^{p+1}k^{p+2} \leq \Co$, then
\beq\label{eq:hkacceg}
\NHokD{u(k) - \uh(k)} \leq \Cth h^{2p}k^{2p+1}\NLtD{f}
\eeq
for some $\kz, \Co, \Cth > 0.$

For $0 < \eps < 1,$ if we take $\Ct = \eps/\Cth$ then it follows that the $h$-finite-element method is $\hk{(p+2)/(p+1)}{(2p+1)/(2p)}$-accurate.

We choose this example to illustrate:
\bit
\item That it is not always the case that $a=b$; here $a > b.$
\item That sometimes (as in this case) $a > b.$ I.e., whilst the mesh constraint `$hk^b$ sufficiently small' makes the finite-element error small, one can only apply this mesh constraint under the more restrictive mesh constraint `$hk^a$ sufficiently small'. Therefore, this result is somewhat limited in its practical applicability.
\eit
\ere

\subsubsection{Optimal mesh conditions for the finite-element method}

We now provide a brief overview of the optimal values of $a$ and $b$ for $\hk{a}{b}$-accuracy, $\hk{a}{b}$-data-accuracy, and $\hka{a}$-quasi-optimality. By `optimal', we mean the values of $a$ and $b$ that are smallest; these will correspond to the least restrictive conditions on the mesh size $h$. We will also, where possible, comment on whether these optimal conditions have been shown to be sharp. The literature reviews in this, and the next, section draw heavily on the literature reviews in \cite[pp. 182--183]{GrLoMeSp:14} and \cite[p. 112]{DiMoSp:19}. Unless otherwise stated, all the problems treated in this literature review are nontrapping (in the wider sense discussed under `\techtitle' in \cref{sec:wpdisc}), have constant coefficients, and and have a impedance boundary condition on at least part of the boundary.

\paragraph{$\hk{a}b$-accuracy} In 1-d the $h$-finite-element method has been proved to be $\hka{3/2}$-accurate for first-order finite elements by Ihlenburg and Babu\v{s}ka in \cite[Theorem 5 and Equation (3.25)]{IhBa:95a} and \cite[Equation (4.5.15)]{Ih:98}. For higher-order finite-elements (still in 1-d) they proved the $h$-finite-element method is $\hka{(2p+1)/(2p)}$-accurate in \cite[Corollary 3.2]{IhBa:97} and \cite[Theorem 4.27 and Equation 4.7.41]{Ih:98}. However, all the above results were only measured in the $H^1$ seminorm (and so we are slightly abusing the notation of $\hka{a}$-accuracy here), and the higher-order results were proved under the assumptions that $f \in \HpmoD,$ $u \in \HppoD,$ and $\SNHppoD{u} \sim k^p \SNHoD{u}.$ In addition, these results were proved using the properties of the `discrete Green's function' for the Helmholtz equation. This function was used to first prove an a priori bound on the finite-element approximation of the solution, and the error bounds were then concluded from this a priori bound. As the discrete Green's function is only known explicitly in 1-d, these results have not been generalised to higher dimensions.

These $\hk{a}b$-accuracy results were confirmed numerically (for $p=1$) by \cite[Figure 11]{IhBa:95a} and \cite[Figure 4.13]{Ih:98}, which showed that the relative error is bounded if $h \sim k^{-3/2}$ and by \cite[Figure 4.10]{Ih:98}, which showed that the relative error is not bounded if $h \sim k^{-1}.$ We note that the proofs in \cite{IhBa:95a,Ih:98} showed the finite-element error was bounded by $\SNHppoD{u}$, and then used the fact that
\beq\label{eq:ratio1d}
\frac{\SNHppoD{u}}{\SNHoD{u}} \sim k^p
\eeq
to bound the finite-element error by $\SNHoD{u},$ and hence conclude a bound on the relative error.

The relation \cref{eq:ratio1d} follows because the solution of the Helmholtz equation in 1-d is given by
\beq\label{eq:soln1d}
A \cos(kx) + B \sin(kx)
\eeq
(for some constants $A$ and $B$). Because \cref{eq:soln1d} only holds in 1-d, and therefore one can, in general, only prove \cref{eq:ratio1d} in 1-d, the proofs of $\hka{(2p+1)/(2p)}$-accuracy have not been extended to higher dimensions, although we conjecture they are true. The only computational results for higher dimensions are those by Bayliss, Goldstein, and Turkel, who showed in \cite[Section 3, Tables 1--3]{BaGoTu:85} that, for low wavenumbers $k \in (4.16,8.32)$ the relative error for first-order finite-elements is bounded if $h \sim k^{-3/2}$, but is not bounded if $h \sim k^{-1}.$

\paragraph{$\hk{a}b$-data-accuracy} In contrast to $\hk{a}b$-accuracy, there are many proofs (for different polynomial degrees and for different Helmholtz problems) of $\hk{a}b$-data-accuracy. The reason for far more proofs in this case is that one can usually show that the interpolation error $u -\Ih u$ is bounded by the data, and one can use this bound when bounding the finite-element error to conclude the finite-element error is bounded by the data.

The best results known to date are the same (in terms of $a$ and $b$) as those for $\hk{a}b$-accuracy, except results for $\hk{a}b$-data-accuracy hold in higher dimensions. In \cite[Theorem 5.1, Corollary 5.2]{DuWu:15} Du and Wu proved that the $h$-finite-element method for the IIP is $\hka{(2p+1)/(2p)}$-data-accurate for arbitrary (fixed) polynomial degree $p$ and $d = 2$ or $3$, provided $u \in \HppoD$ (although this result can be shown for lower-regularity solutions by combining \cite[Theorem 5.1]{DuWu:15} with \cite[Lemma 3.5]{DuWu:15}). This result was proved for the IIP in $d = 2$ or $3$ for $p=1$ by Wu in \cite{Wu:14}. The only computational evidence we know that in some way  confirms the sharpness of this result is that of Chaumont-Frelet and Nicaise \cite[Section 6.3, Figure 9]{ChNi:18}. They show that for scattering by a re-entrant corner, the relative $L^2$-error is bounded independently of $k$ provided $h^{2p}k^{2p+1}$ is bounded, for $p = 1,2,$ and $6.$ However, in general the result for $\hk{a}b$-data-accuracy coincides with the (sharp) best case for $\hk{a}b$-accuracy (in terms of the valuyes of $a$ and $b$), and so we expect the result for $\hk{a}b$-data-accuracy is also sharp.
\optodo{Update This ref and check all other instances}

\paragraph{$\hka{a}$-quasi-optimality} Finally, the best known result for $\hka{a}$-quasi-optimality for the Helmholtz equation is that the $h$-finite-element method is $\hka{(p+1)/p}$-quasi-optimal. This was first proved for $p=1$ in 1-d by Aziz, Kellog, and Stevens in \cite[Theorem 3.1]{AzKeSt:88} and Ihlenburg and Babu\v{s}ka in \cite[Theorem 3]{IhBa:95a} and \cite[Theorems 4.9 and 4.13]{Ih:98}, with \cite[Figures 7-9]{IhBa:95a} and \cite[Section 4.5.4 and Figures 4.11-4.12]{Ih:98} showing this result is sharp in 1-d (although Ihlenburg and Babu\v{s}ka only work in the $H^1$-semi norm). The result was proved for $p=1$ and $d=2$ for the IIP by Melenk \cite[Proposition 8.2.7]{Me:95}, and for higher-order finite elements (for the full-space problem, IIP, and EDP) by Melenk and Sauter in \cite[Corollary 5.6]{MeSa:10} and \cite[Theorem 5.8]{MeSa:11} respectively. This result was shown for a PML problem by Chaumont-Frelet, Gallistl, Nicaise, and Tomezyk in \cite[Theorem 5.1]{ChGaNiTo:18}, and extended to arbitrary time-harmonic wave propagation problems by Chaumont-Frelet and Nicaise in \cite[Theorem 2.15]{ChNi:19}, who showed that if the constant in the a priori bound \cref{eq:bgbound} grows like $k^\alpha,$ then the $h$-finite-element method is $\hka{(p+\alpha+1)/p}$-accurate. They give numerical experiments in \cite[Figure 8]{ChNi:19} showing the sharpness of $\hka{(p+1)/p}$-quasi-optimality for $p=1, 2$ for a heterogeneous IIP.

\paragraph{Link with dispersion error} When computing on regular grids, one can mathemtically analyse the `dispersion error' of the finite-element method for the Helmholtz equation; i.e., the difference between the wavenumber of the true solution $u$ (i.e., $k$) and the wavenumber of the approximation $\uh.$ We mention briefly that Ainsworth  analysed the dispersion error for the $hp$-finite-element method for the Helmholtz equation and proved the dispersion error is of the order $h^{2p}k^{2p+1}$ \cite[Equation (3.5)]{Ai:04}. Observe that this is the same order as the best-available results for $\hka{a}$-accuracy and $\hka{a}$-data-accuracy discussed above. Therefore, results from finite-element error analysis and dispersion analysis both suggest that the error (measured in a suitable sense in each case) is bounded if $h^{2p}k^{2p+1}$ is sufficiently small. See, e.g., \cite[Remark 5.3(a)]{DuWu:15} for more references on dispersion error analysis for the Helmholtz equation.

\paragraph{Observations} Observe that, in general, the mesh conditions required for quasi-optimality are more restrictive than those required for the error to be bounded. I.e., in general, if the finite-element method is $\hka{\ao}$-quasi-optimal and $\hka{\at}$-data-accurate, then $\ao > \at.$ Therefore, whilst one can show the finite-element method is $\hka{a}$-data-accurate (for some $a$) by first showing quasi-optimality, this approach will result in a more restrictive mesh condition than proving $\hka{a}$-data-accuracy directly. Also observe that the results for higher-order finite elements become less stringent as $p$ increases, i.e., the finite-element method is $\hka{(2p+1)/(2p)}$-data-accurate, and $(2p+1)/(2p) \downarrow 1$ as $p \rightarrow \infty.$

\paragraph{Complete summary of results in the literature} In \cref{tab:acc,tab:dataacc,tab:qo} we list all of the mathematical (as opposed to computational) results in the literature for $\hk{a}b$-accuracy, $\hk{a}b$-data-accuracy, and $\hka{a}$-quasi-optimality. We list these in chronological order, with any relevant comments in the `Notes' column. The `Proof technique' column details the method used in the proof; see \cref{sec:prooftechniques} below for an extended discussion of these techniques. However, we now make a few general comments on the history of these results.

Recall that proving quasi-optimality (or an error bound) for the finite-element method for the Helmholtz equation is more tricky than for the stationary diffusion equation \cref{eq:stdiff}. For the stationary diffusion equation, one immediately obtains quasi-optimality for \emph{any} mesh by C\'ea's Lemma (and one then obtains that the relative error is bounded by \cref{lem:scottzhang}). We emphasise again that this result holds for any mesh, with no restriction on $h.$

However, C\'ea's Lemma relies on the coercivity of the sesquilinear form, and the sesquilinear forms arising from standard discretisations of the Helmholtz equation are not coercive for large $k$. Therefore, to prove quasi-optimality, one instead uses the so-called Schatz argument, a modification of the standard Aubin--Nitsche duality argument\footnote{First introduced by Aubin \cite{Au:67} and Nitsche \cite{Ni:68} for coercive problems.} However, using the Schatz argument, one only obtains quasi-optimality under some $k$-dependent restriction on the mresh size $h.$

The Schatz argument was first used for problems satisfying a G\r{a}rding inequality by Schatz \cite{Sc:74} and first used for Helmholtz problems by Aziz, Kellogg, and Stephens \cite{AzKeSt:88}. In \cite{AzKeSt:88} they proved that in 1-d the finite-element method for the Helmholtz equation is $\hka{2}$-quasi-optimal, and this result was extended to $d=2$ by Melenk \cite{Me:95} in his PhD thesis. However, the Schatz argument was first presented in the framework we use below by Sauter \cite[Section 2]{Sa:06}. Observe that for large values of $k,$ $\hka{2}$-quasi-optimality and the related mesh restriction `$hk^2$ is sufficiently small' is computationally prohibitive---it would result in linear systems of size, e.g., $~10^{12}$, for the Helmholtz equation with $k=100$ in 3-d.

Because of the severe mesh restrictions required for quasi-optimality for Helmholtz problems, recent research efforts have been focused on directly proving error bounds for the finite-element method. The key proof technique has been so-called elliptic projection ideas; these ideas are at the heart of our results in \cref{sec:fem} below, and are discussed in more detail in \cref{sec:prooftechniques} below. To our knowledge, the first use of elliptic projections to prove error estimates for Galerkin approximations was by Wheeler\footnote{Mentioned in \cite{MaNo:03}.} \cite[Theorem 3.1 ff.]{Wh:73}, who proved bounds on the error for a nonlinear parabolic problem by splitting the error into the error from an elliptic projection and the remaining error between the elliptic projection and the Galerkin approximation. The idea of using elliptic projections for Helmholtz problems was first introduced to prove error bounds for discontinuous Galerkin methods for \cref{prob:tedp} by Feng and Wu \cite{FeWu:09,FeWu:11}, and then used for standard finite-element methods (or closely-related continuous-interior-penalty methods) beginning with the work of Wu and Zhu \cite{ZhWu:13,Wu:14}.

In this review we do not address $hp$-finite-element methods for the Helmholtz equation, that is, finite-element methods where one refines both $h$ and $p$. Such methods can be very effective; the error for such methods can converge exponentially with respect to the number of degrees of freedom (see, e.g., \cite[Theorem 4.51]{Sc:98}). However, rigorous error analysis for these methods requires finite-element-error bounds that are completely explicit in both $h$ and $p$.

The landmark results on $hp$-finite-element methods for the Helmholtz equation were achieved by Melenk and Sauter \cite{MeSa:10,MeSa:11} who used a novel splitting of the solution of the Helmholtz equation (see the comments underneath \cref{rem:osc} below) to show that the $hp$-finite-element method is quasi-optimal if $hk/p \leq \co$ and $p \geq \ct \ln k$ (for some constants $\co,\ct>0$). Melenk and Sauter proved this result for the full-space problem in \cite{MeSa:10} and for: (i) the exterior Dirichlet problem, and (ii) the interior impedance problem in an analytic domain or a 2-d convex polygon in \cite{MeSa:11}. These results were generalised to an arbitrary 2-d Lipschitz domain by Esterhazy and Melenk in \cite[Theorem 4.2]{EsMe:12}. Other results in the literature for $hp$-finite-element methods are those of Zhu and Wu \cite[Equation (1.7)]{ZhWu:13} who showed the $hp$-finite-element method for the Helmholtz equation is data-accurate\footnote{We do not use the $\hk{a}b$-accurate, etc., terminology for $hp$-methods, as it does not represent the interplay between the polynomial degree $p$ and the wavenumber $k$.} provided
\beqs
\frac{kh}p \leq C_0\mleft(\frac{p}k\mright)^{\frac1{p+1}},
\eeqs
where $C_0>0$ is some constant. Finally, in \cite[Corollary 3.2]{IhBa:97} Ihlenburg and Babu\v{s}ka proved a bound on the error for the finite-element method in 1-d that is explicit in $k,$ $h$, and $p$, although the bound is probably not sharp in $p,$ and therefore could not be used to prove results on $hp$-finite-element methods.

\afterpage{% Heard about from https://tex.stackexchange.com/questions/11471/how-to-wrap-text-around-landscape-page
\begin{landscape} % Heard about from https://tex.stackexchange.com/questions/19017/how-to-place-a-table-on-a-new-page-with-landscape-orientation-without-clearing-t/19021#19021 and https://tex.stackexchange.com/questions/25369/how-to-rotate-a-table 
\begin{table}[h]
  \centering
  \begin{threeparttable}[c]
    \begin{tabular}{Sc Sc Sc Sc}
  \toprule
  & $\hk{a}b$-accuracy & Notes & Proof technique\\
   \midrule
    \cite[Equation (3.25)]{IhBa:95a} & $\hk{1}{3/2}$ & \makecell{$d=1$, unit interval,\\$u(0)=0,$\\impedance boundary condition at $1$\\$H^1$ seminorm}&\makecell{Discrete Green's function\\(specific to $d=1$)}\\
    \cite[Theorem 4]{IhBa:95b} & $\hk{1}{3/2}$ & \makecell{$d=1$, unit interval,\\$u(0)=0,$\\impedance boundary condition at $1$\\$L^2$ norm\tnote{1}}&\makecell{Discrete Green's function\\error splitting using interpolant}\\
      \cite[Corollary 3.2]{IhBa:97}&$\hk{1}{(2p+1)/(2p)}$ & \makecell{$d=1$, unit interval,\\$u(0)=0,$\\impedance boundary condition at $1$\\$H^1$ seminorm}&\makecell{Discrete Green's function,\\error splitting using interpolant}\\
  \cite[Theorem 4.13 and equation (4.5.15)]{Ih:98}& $\hk{1}{3/2}$ & \makecell{$d=1$, unit interval,\\$u(0)=0,$\\impedance boundary condition at $1$\\$H^1$ seminorm}& Discrete Green's function\\
  \cite[Theorem 4.27 and equation (4.7.41)]{Ih:98}& $\hk{1}{(2p+1)/(2p)}$ & \makecell{$d=1$, unit interval,\\$u(0)=0,$\\impedance boundary condition at $1$\\$H^1$ seminorm}&\makecell{Discrete Green's function,\\error splitting using interpolant}\\ 
  \bottomrule
  \end{tabular}
  \begin{tablenotes}
\item [1] Actually, \cite[Theorem 4]{IhBa:95b} only proves a bound on the $L^2$-norm of the error in terms of the $H^2$-seminorm of the solution. However, when $d=1$, $\SNHtD{u} \sim k^2\NLtD{u}$ and so one can conclude $\hk{a}b$-accuracy.
  \end{tablenotes}
    \caption{$\hk{a}b$-accuracy for $h$-finite-element discretisations of the Helmholtz equation}\label{tab:acc}
\end{threeparttable}
\end{table}

\begin{table}[h]
  \centering
\begin{tabular}{Sc Sc Sc Sc}
  \toprule
 & $\hk{a}b$-data-accuracy  & Notes & Proof technique\\
  \midrule
  \cite[Theorem 5]{IhBa:95a} & $\hk{1}{3/2}$ & \makecell{$d=1$, unit interval,\\$u(0)=0,$\\impedance boundary condition at $1$}&\makecell{Discrete Green's function\\error splitting using interpolant}\\
      \cite[Corollary 4.2]{ZhWu:13}& $\hk{(p+2)/(p+1)}{(2p+1)/(2p)}$   &\makecell{$d=2,3$, IIP,\\$D$ smooth,\\bounds obtained for $hp$-finite-element method,\\so fully $p$-explicit}& Modified Schatz\\
      \cite[Theorem 5.1]{Wu:14} & $\hka{3/2}$  &\makecell{$d=2,3$, IIP,\\$D$ a star-shaped polygon/polyhedron}& Error splitting\\
      \cite[Corollary 5.2]{DuWu:15} & $\hka{(2p+1)/(2p)}$ &  $d=2,3$, IIP, $D$ smooth, star-shaped& Error splitting\\
        \cite[Theorem 5.5]{ChNi:18}& $\hk{(3+\sigma)/(1+\alpha)}{3/2}$   &\makecell{$d=2$, TEDP with re-entrant corners,\\a priori bound grows like $k^\sigma$\\$\alpha \in (1/2,1)$ related to\\strength of corner singularities}& Modified Schatz\\
        \cite[Theorem 4.3 and Remark 4.2(iv)]{LiWu:18}& $\hka{3/2}$   &\makecell{$d = 1,2,3$,\\full-space problem truncated with PML,\\posed in a ball}& Modified Schatz\\
        \cite[Lemma 3.3]{WuZo:18}&$\hka{3/2}$&\makecell{IIP, $D$ convex,\\$n$ heterogeneous with $\NLiD{n-1} \lesssim 1/k$,\\part of an argument for\\a nonlinear heterogeneous Helmholtz problem}&Error splitting\\
  \cite[Theorem 5.4]{ChGaNiTo:18}&$\hk{(p+2)/(p+1)}{(2p+1)/(2p)}$  & $d=2$, EDP truncated with PML, posed in a ball& Modified Schatz\\
\bottomrule
\end{tabular}
\caption{$\hk{a}b$-data-accuracy for $h$-finite-element discretisations of the Helmholtz equation}\label{tab:dataacc}
\end{table}


  \begin{table}[h]
    \centering
\begin{tabular}{Sc Sc Sc Sc}
  \toprule
& $\hka{a}$-quasi-optimality & Notes & Proof technique\\
  \midrule
  \cite[Theorem 3.1]{AzKeSt:88} & $\hka{2}$ & $d=1$ & Schatz\\
  %?  \cite{DoSaShSc:93}\\
  \cite[Theorem 3]{IhBa:95a} & $\hka{2}$ & $d=1,$ $H^1$ seminorm & Schatz\\
  \cite[Corollary 2]{IhBa:95a} & $\hka{0}$ & $d=1,$ $H^1$ seminorm &\makecell{Discrete Green's function\\error splitting using interpolant}\\
  %? \cite{IhBa:95b}
  \cite[Theorems 4.9 and 4.13]{Ih:98} & $\hka{2}$ & \makecell{$d=1,$ $H^1$ seminorm,\\\cite[Theorem 4.9]{Ih:98} is \cite[Theorem 3]{IhBa:95a}} & \makecell{Schatz, and error splitting\\using interpolant,\\respectively}\\
  \cite[Proposition 8.2.7]{Me:95} & $\hka{2}$ & \makecell{$d=2,$\\IIP,\\$D$ smooth and star-shaped or convex} & Schatz\\
  \cite[Corollary 5.6]{MeSa:10} & $\hka{(p+1)/p}$ & Full-space problem & Schatz\\
%  \cite[Theorem 5.8]{MeSa:11} & $\hka{(p+1)/p}$ & \makecell{$d = 2,3$,\\polynomial growth of constant in a priori bound,\\IIP ($\Dm$ analytic or Lipschitz convex polygon)\\and EDP ($\Dm$ analytic)} & Schatz\\
  \cite[Theorem 5.3]{ChNi:18} & $\hka{2+\sigma}$ & \makecell{$d=2$, TEDP with re-entrant corners\\a priori bound grows like $k^\sigma$} & Schatz\\
  \cite[Theorem 5.1]{ChGaNiTo:18} & $\hka{(p+1)/p}$&$d=2$, EDP truncated with PML, posed in a ball  &Schatz\\
    \cite{ChNi:19} & $\hka{(p+\alpha+1)/p}$ & \makecell{Arbitrary time-harmonic wave problem,\\a priori bound grows at rate $k^\alpha$} & Schatz\\
    \cite[Theorems 4.2 and 4.5, Remark 4.6(ii)]{GrSa:18} &$\hka{2}$ & \makecell{IIP, $D$ Lipschitz, star-shaped w.r.t. a ball,\\$n$ heterogenenous\\constants fully explicit in $n$}& Schatz\\
    \cite[Theorem 3]{GaSpWu:18} & $\hka{2}$ & \makecell{EDP,\\$A$ and $n$ heterogeneous,\\$\Dm,$ $A$, and $n$ $C^\infty$,\\constants fully explicit in $A$ and $n$} & Schatz\\
\bottomrule
\end{tabular}
\caption{$\hka{a}$-quasi-optimality for $h$-finite-element discretisations of the Helmholtz equation}\label{tab:qo}
\end{table}
\end{landscape}
}
\subsection{Extended discussion of proof techniques for finite-element errors for the Helmholtz equation}\label{sec:prooftechniques}
We now discuss in some detail the proof techniques used to obtain $\hk{a}b$-accuracy, $\hk{a}b$-data-accuracy, or $\hka{a}$-quasi-optimality.  We discuss these techniques because they are related, but different, and we hope that discussing the details, and comparing and contrasting these techniques will prove helpful to the research community, and will increase understanding of proofs that are, at times, a little technical. However, we also discuss these results to set the coefficient-explicit bounds in \cref{sec:fem} below in context.

We note that frequently quasi-optimality results are refereed to as \defn{asymptotic} error estimates, and accuracy or data-accuract results (when they are proved under weaker mesh constraints than asymptotic result) are referred to as \defn{pre-asymptotic} error estimates. This terminology is used because the mesh conditions to ensure quasi-optimality are more restrictive than those for bounded error, and therefore they hold for smaller values of $h.$

For simplicity's sake, our exposition below will assume that we are treating the homogeneous Helmholtz TEDP (i.e., $A=I$ and $n=1$) with $\gI = 0,$ and that the problem is nontrapping\footnote{Recall that we say that the problem is nontrapping if we have a $k$-independent a priori bound, i.e., $C$ in \cref{eq:bgbound} is independent of $k.$}. Also, we suppress all of the constants involved, instead opting to use $\lesssim$ notation, where $a \lesssim b$ if $a \leq C b,$ with $C$ independent of $k$ and $h$. The new results we present in \cref{sec:fem} consider heterogeneous problems that may be trapping, and state all of the constants involved explicitly, at the price of being more technical to state.

\subsubsection{Merits and drawbacks of each argument}
We first briefly discuss the positive and negative points for each of the two classes of argument we will outline below: (modified) duality arguments and splitting arguments.

The merits of duality arguments are their simplicity---we will state the arguments almost in their entirety below, and they are closely related with familiar duality arguments used in finite-element-error analysis for more familiar problems, such as the stationary diffusion equation. In addition, using these arguments for higher-order finite elements (or $hp$-finite-elements) one can easily conclude error bounds under certain mesh conditions, where both the bounds and the conditions are completely $h$-, $p$-, and $k$-explicit.% Such an analysis was performed for quasi-optimality of $hp$-finite-element methods for the Helmholtz equation in homogeneous media by Melenk and Sauter \cite{MeSa:10,MeSa:11}.

However, the main drawback of duality arguments is their lack of sharpness in the conditions imposed on $h$. For example, as detailed in \cref{rem:dataacc}, Chaumont-Frelet, Gallistl, Nicaise and Tomezyk used a duality argument in \cite{ChGaNiTo:18} to prove that for a given Helmholtz problem the pollution term is of the order $h^2k + h^{2p}k^{2p+1}$ provided $h^{p+1}k^{p+2}$ is sufficiently small. I.e., the mesh restriction under which they prove a finite-element error bound (`$h^{p+1}k^{p+2}$ is sufficiently small') is \emph{more restrictive} than the mesh restriction needed to bound the pollution error uniformly in $k$ (namely `$h^{2p}k^{2p+1}$ is sufficiently small').

In contrast, the merit of splitting arguments is that they can give mesh conditions that are sharp in their $h$-dependence; in \cite{DuWu:15}, Du and Wu proved that the pollution term in the finite-element error is of the order $h^{2p}k^{2p+1},$ under the restriction that $h^{2p}k^{2p+1}$ is sufficiently small, i.e., the mesh restriction under which they prove a finite-element error bound is \emph{of the same order} as the mesh restriction needed to bound the pollution error uniformly in $k.$

However, the drawback of splitting arguments is their complexity. They involve proving bounds on the solution of discrete Helmholtz problems; such bounds are complicated to prove, especially in the higher-order cases, and the constants involved depend on the polynomial degree $p$ in highly complicated ways. These drawbacks limit the likelihood that such bounds can be used for $hp$-finite-element methods, where the dependence on the polynomial degree must be known explicitly.


\subsubsection{(Modified) Duality arguments}
The first class of arguments we consider are (modified) duality arguments. These arguments are used to prove quasi-optimality, accuracy, and data-accuracy results; they include the more commonly known Schatz argument for Helmholtz problems. In these arguments, one uses a G\r{a}rding inequality satisfied by the Helmholtz equation to show that the error in the weighted $H^1$ norm is bounded by the best approximation error, plus error in the $L^2$ norm\footnote{Recall that for the stationary diffusion equation, one can show the finite-element error is bounded by the approximation error simply using coercivity and boundedness of the bilinear form---this is Cea's Lemma.}. One then uses a modification of the standard Aubin--Nitsche duality argument to bound the $L^2$ norm.

To show the error in the weighted $H^1$ norm is bounded by the error in the $L^2$ norm, observe
\begin{align}
\NHokD{u-\uh}^2 &\leq \Re{\aT(u-\uh,u-\uh)} + k^2 \NLtD{u-\uh}^2\nonumber\\
&= \Re{\aT(u-\uh,u-\vh)} + k^2 \NLtD{u-\uh}^2\text{ by Galerkin orthogonality}\nonumber\\
&\lesssim \NHokD{u-\uh}\NHokD{u-\vh} + k^2 \NLtD{u-\uh}^2,\label{eq:gardingerror}
\end{align}
for any $\vh \in \Vhp.$

The Schatz argument to prove quasi-optimality, outlined below, proceeds to show (for $p=1$)
\beq\label{eq:schatzbound}
k\NLtD{u-\uh} \lesssim hk^2 \NHokD{u-\uh},
\eeq
and combining \cref{eq:gardingerror,eq:schatzbound}, one has
\beqs
\NHokD{u-\uh}^2 \lesssim \NHokD{u-\uh}\NHokD{u-\vh} + \mleft(hk^2\mright)^2\NHokD{u-\uh}^2
\eeqs
cancelling a factor $\NHokD{u-\uh}$ from each side, and moving the final term onto the left-hand side. one obtains quasi-optimality if $hk^2$ is sufficiently small.

In contrast, so-called \defn{elliptic-projection-modified Schatz} (`modified Schatz' for short) arguments used to prove a finite-element error bound, also outlined below, show (for $p=1$)
\beq\label{eq:epbound}
k\NLtD{u-\uh} \lesssim h^2k^3 \NLtD{f},
\eeq
and then combine \cref{eq:gardingerror,eq:epbound} to show the finite-element error is bounded if $h^2k^3$ is sufficiently small---observe this is less restrictive than the requirement $hk^2$ is sufficiently small.
%% \begin{align}
%% \NW{u-\uh}^2 &\lesssim \NW{u-\uh}\NW{u-\vh} + h^2k^3 \NLtD{f}\nonumber\\
%% &\leq \eps \NW{u-\uh}^2 + \frac1\eps \NW{u-\vh}^2 +\mleft(h^2 k^3 \NLtD{f}\mright)^2\label{eq:epboundpart1}
%% \end{align}
%% Taking $\eps$ sufficiently small, and moving the first term in \cref{eq:epboundpart1} to the left-hand side, taking $\vh$ to be the best approximation of $u$ in $\Vhp,$ and using \cref{lem:scottzhang} and the fact that\footnote{For sufficiently smooth $\GD$ and $\GI$---see \cite[Remark 2.14]{GrPeSp:19}} we can show $\NHtD{u} \lesssim k\NLtD{f}$
%% \beqs
%% \NW{u-\uh} \lesssim \mleft(hk + h^2k^3\mright)\NLtD{f}.
%% \eeqs

We will now look over the Schatz argument in more detail, before considering modified Schatz arguments. We first establish some notation that will enable us to discuss best approximation errors for solutions of Helmholtz problems. We let $\solfem:\LtD\rightarrow \HoD$ denote the solution operator for \cref{prob:vtedp} with zero impedance boundary condition, and we let $\solfems$ denote the solution operator for the corresponding adjoint problem. That is, for any $\ftilde \in \LtD$ and for all $v \in \HozDD,$
\beqs
\aT(\solfem(\ftilde),v) = \IPLtD{\ftilde}{v}
\eeqs
and
\beqs
\aT(v,\solfems(\ftilde)) = \IPLtD{v}{\ftilde}.
\eeqs
One can, of course, define the corresponding solution operators in the case of non-zero impedance boundary conditions. We next define the approximability constants:
\beqs
\wba \de \sup_{\ftilde \in \LtD}\inf_{\vh \in \Vhp} \frac{\NHokD{\solfem(\ftilde) - \vh}}{\NLtD{\ftilde}}
\eeqs
and
\beqs
\wbaadj \de \sup_{\ftilde \in \LtD}\inf_{\vh \in \Vhp} \frac{\NHokD{\solfems(\ftilde) - \vh}}{\NLtD{\ftilde}}.
\eeqs
We are changing notation slightly from that normally prevelant in the literature; notation for $\wbaadj$ was first introduced by Sauter in \cite[Section 2.2]{Sa:06}, but $\wbaadj$ was instead denoted $\tilde{\eta}$ in \cite{Sa:06}. Later, $\wbaadj$ was denoted $\eta$ in \cite[Equation (4.5)]{MeSa:10}.  When dealing purely with the Schatz argument for quasi-optimality (as in \cite[Section 2.2]{Sa:06}) one only needs consider the approximation of adjoint problems in the duality-argument step, and hence one only needs notation for $\wbaadj$. However, in our exposition of elliptic-projection-based arguments below, we will also need to consider the approximation of standard Helmholtz problems, and hence we adopt the notation above.

\paragraph{The Schatz argument} We now recap the Schatz argument for proving quasi-optimality for an indefinite sesquilinear form. This recap will allow us to see where $\wbaadj$ enters the argument, and will also allow us to compare and contrast this argument with the arguments used to obtain a bounded finite-element error using an elliptic projection.

We let $\xi \in \HozDD$ solve the adjoint Helmholtz problem
\beqs
\aT(v,\xi) = \IPLtD{v}{u-\uh} \tforall v \in \HozDD.
\eeqs
Then, taking $v = u-\uh,$ we have
\begin{align}
  \NLtD{u-\uh}^2 &= \aT(u-\uh,\xi)\label{eq:schatz0}\\
  &=\aT(u-\uh,\xi - \vh) \text{ by Galerkin orthogonality for } u-\uh, \text{ for any } \vh \in \Vhp\nonumber\\
  &\lesssim \NHokD{u-\uh}\NHokD{\xi-\vh}\nonumber\\
  &\lesssim \NHokD{u-\uh} \wbaadj \NLtD{u-\uh}\nonumber.
\end{align}
Cancelling a factor of $\NLtD{u-\uh},$ we obtain
\beq\label{eq:schatz1}
\NLtD{u-\uh} \lesssim \wbaadj \NHokD{u-\uh}.
\eeq
Combining \cref{eq:gardingerror,eq:schatz1}, we have
\beqs
\NHokD{u-\uh}^2 \lesssim \NHokD{u-\uh}\NHokD{u-\vh} + \mleft(k\wbaadj\mright)^2 \NHokD{u-\uh}^2,
\eeqs
and hence by cancelling a factor $\NHokD{u-\uh}$ and taking the final term on to the left-hand side, we obtain
\beqs
\NHokD{u-\uh} \lesssim \inf_{\vh \in \Vhp} \NHokD{u-\vh} \tif k\wbaadj \text{ is sufficiently small}.
\eeqs
All results showing quasi-optimality for the Helmholtz equation (for different finite-element spaces and different domains) can then be seen as simply obtaining estimates on $\eta$ in these different scenarios; this literature is summarised in \cref{tab:qo,tab:acc,tab:dataacc} below.

\paragraph{Modified Schatz arguments} We now move on to compare and contrast the Schatz argument given above with elliptic-projection-modified-Schatz arguments. The main difference in the result of these arguments, compared to the Schatz argument given above, is that modified Schatz arguments only prove a \emph{bound} on the finite-element error, rather than proving quasi-optimality. However, these bounds are obtained under mesh restrictions that are \emph{less restrictive} than those required for quasi-optimality.

The elliptic projection of a function $ w \in \HozDD$ is the finite-element function $\Ph w$ that has the same action as $w$ on the finite-element space $\Vhp$ under some elliptic operator\footnote{The definition of the elliptic projection is dependent on the exact sesquilinear form used in the discretisation of the Helmholtz equation, and on the norm one is using to measure the error. For example, elliptic projection arguments originated in the study of discontinuous Galerkin methods for the Helmholtz equation (in \cite{FeWu:09,FeWu:11}); therefore the sesquilinear forms associated with the discretisation included penalty terms. These penalty terms were incorporated into the sesquilinear form $\aep$ and the norms used to measure the error also included these penalty terms. In the following exposition, we will work with standard finite-element discretisations of the Helmholtz equation and standard Sobolev norms, and so the elliptic projections we used will be based on this setting.}; i.e., $\Ph w$ is defined by
\beqs
\aep(\vh,\Ph w) = \aep(\vh,w) \tforall \vh \in \Vhp,
\eeqs
for some sesquilinear form $\aep$, such that $\Ph$ is well-defined. For Helmholtz problems\footnote{Recall we are considering discretisations of \cref{prob:vtedp}, and therefore the sesquilinear form we use is given by \cref{eq:aT}.}, choices for $\aep(\vo,\vt)$ used in the literature are either
\beq\label{eq:aepho}
\aep(\vo,\vt) = \IPLtD{\grad \vo}{\grad \vt},
\eeq
\beq\label{eq:aeplower}
\aep(\vo,\vt) = \IPLtD{\grad \vo}{\grad \vt} + \IPLtD{\vo}{\vt},
\eeq
or
\beq\label{eq:aepused}
\aep(\vo,\vt) = \IPLtD{\grad \vo}{\grad \vt} - ik\IPLtGI{\vo}{\vt}.
\eeq
These elliptic projections correspond to finding finite-element approximations of the solution of the PDEs
\begin{align}
  \Delta w &= F \tin D,\label{eq:sd1}\\
  w &= 0 \ton \GD,\tand\label{eq:sd2}\\
  \dn w &= 0 \ton \GI\label{eq:sd3};
\end{align}
\begin{align}
  \Delta w + w&= F \tin D,\label{eq:sd4}\\
  w &= 0 \ton \GD,\tand\label{eq:sd5}\\
  \dn w &= 0 \ton \GI;\label{eq:sd6}
\end{align}
or
\begin{align}
  \Delta w &= F \tin D,\label{eq:sd7}\\
  w &= 0 \ton \GD,\tand\label{eq:sd8}\\
  \dn w -ikw &= 0 \ton \GI\label{eq:sd9}
\end{align}
respectively, where $F$ is an appropriately chosen function. In the following exposition, we will assume $\aep$ is given by \cref{eq:aepused}.
Since $\Ph$ is a Galerkin projection, one can show that in its energy norm $\Nep{\cdot}$ (equivalent to the $k$-weighted $H^1$ norm $\NHokD{\cdot}$, with equivalence constants independent of $k$ via the multiplicative trace inequality and the Poincar\'e--Friedrich's inequality (\cref{lem:poincare} below) $\aep$ is coercive and continuous\footnote{For comment on showing coercivity and continuity for the other definitions of $\aep$, see \cref{rem:epdef} below.}, and hence is quasi-optimal:
\beq\label{eq:epho}
\Nep{w-\Ph w} \lesssim \inf_{\vh \in \Vhp} \Nep{w-\vh}.
\eeq
Also, by the Aubin--Nitsche duality argument (as $\aep$ is coercive),
\beq\label{eq:eplt}
\NLtD{w-\Ph w} \lesssim h \Nep{w-\Ph w}.
\eeq
\ednote{Euan---you said you wanted to have a discussion about what properties exactly are required for the elliptic projection---hopefully the exposition below is a starting point. I've gone with quasi-optimality for the time being, as it means one can see the analogues with the Schatz argument more clearly. But lets chat about this!}

Combining \cref{eq:epho,eq:eplt} is a modification of the Schatz argument above: We start from \cref{eq:schatz0}, but instead of introducing an arbitrary $\vh \in \Vhp$ into the second argument, we instead introduce the elliptic projection $\Ph \xi$ by Galerkin orthogonality for $u-\uh$:
\begin{align}
  \NLtD{u-\uh}^2 &= \aT(u-\uh,\xi-\Ph\xi)\nonumber\\
  &= \aep(u-\uh,\xi-\Ph\xi) - k^2\IPLtD{u-\uh}{\xi-\Ph\xi}\nonumber\\
  &= \aep(u-\Ih u,\xi-\Ph\xi) - k^2\IPLtD{u-\uh}{\xi-\Ph\xi} \text{ by Galerkin orthogonality for }\xi-\Ph\xi\nonumber\\
  &\lesssim \wbaadj \NLtD{f} \wba \NLtD{u-\uh} + k^2 \NLtD{u-\uh} \NLtD{\xi-\Ph\xi}\text{ by \cref{eq:epho}}\nonumber\\
  &\lesssim \wba\wbaadj \NLtD{f}\NLtD{u-\uh} + k^2 \NLtD{u-\uh} h\wba \NLtD{u-\uh}\text{ by \cref{eq:epho,eq:eplt}.}\label{eq:schatz2}
\end{align}
Therefore if $hk^2\wba$ is sufficiently small, the second term in \cref{eq:schatz2} can be absorbed into the left-hand side, and cancelling a factor of $\NLtD{u-\uh},$ we obtain
\beq\label{eq:ep1}
k \NLtD{u-\uh} \lesssim k \wba\wbaadj \NLtD{f} \tif hk^2\wba \text{ is sufficiently small,}
\eeq
this is \cref{eq:epbound}. (We consider $k\NLtD{u-\uh}$ so as to compare with the quasi-optimality results above, as the $L^2$ term in $\NHokD{\cdot}$ is also multiplied by a factor $k$.)

To obtain a bound on the error in the weighted $H^1$ norm\ednote{Both, Li and Wu \cite[Text after equation 4.15]{LiWu:18} use an error-spliiting-type technique to bound the $H^1$ error in terms of the $L^2$ error, but this only works because they're doing a PML problem, and so don't have a term defined on $\GI$. Would you mention this?}, we put \cref{eq:ep1} into \cref{eq:gardingerror} to get
\beqs
\NHokD{u-\uh}^2 \lesssim \NHokD{u-\uh}\NHokD{u-\vh} + \mleft(k\wba\wbaadj\mright)^2 \NLtD{f}^2,
\eeqs
and taking $\vh = \Ih v,$ by \cref{lem:scottzhang} we have
\beqs
\NHokD{u-\uh}^2 \lesssim \NHokD{u-\uh}\wba\NLtD{f} + \mleft(k\wba\wbaadj\mright)^2 \NLtD{f}^2,
\eeqs
We then obtain for any $\eps > 0$ by Cauchy's inequality \cref{eq:cauchy}
\beq\label{eq:ep2}
\NHokD{u-\uh}^2 \lesssim \eps \NHokD{u-\uh}^2 + \frac1\eps\wba^2 \NLtD{f}^2 + \mleft(k\wba\wbaadj\mright)^2 \NLtD{f}^2.
\eeq
Taking $\eps$ sufficiently small, moving the first term on the right-hand side of \cref{eq:ep2} to the left-hand side, and taking a square root, we obtain
\beq\label{eq:ep3}
\NHokD{u-\uh} \lesssim\mleft(\wba  + k\wba\wbaadj\mright) \NLtD{f} \tif hk^2\wba\text{ is sufficiently small}.
\eeq

The term $\wba\NLtD{f}$ on the right-hand side of \cref{eq:ep3} is, up to a constant, the best-approximation error for $u$ (i.e., the error when one interpolates/quasi-interpolates $u$---see \cref{sec:helmfedisc,lem:scottzhang} above), and the term $k\wba^2$ is the \defn{pollution} term arising from the numerical method. As for the quasi-optimality results above, we remark that different bounds on the finite-element error (for different finite-element space $\Vhp$ and different domains) can be thought of as proving bounds on $\eta$ in these different situations.
\ednote{Both---I don't yet have a clear explanation of \emph{why} this works. Currently, it seems just like a trick to me. I don't know if that's an issue!}
\subsubsection{Error-splitting arguments}\label{sec:errorsplit}
The second class of arguments used in the literature are error-splitting arguments, used to prove accuracy and data-accuracy results. In these arguments the finite-element error is split using the elliptic projection of the solution $u$. To begin, we make the trivial observation that
\beqs\label{eq:split1}
u-\uh = \mleft(u-\Ph u\mright) + \mleft(\Ph u - \uh\mright).
\eeqs
Proving an error bound for $u-\uh$ therefore reduces to proving an error bound for the elliptic projection error $u- \Ph u$ and proving a bound on the difference $\Ph u - \uh.$ The former can either be accomplished by showing quasi-optimality of the elliptic projection (as above) or by proving such an error bound directly. The former approach is taken in \cite{DuWu:15,LiWu:18,ChGaNiTo:18}, and the latter approach is taken in \cite[Lemma 5.2]{FeWu:09} (althought the proof is only contained in \cite[Lemma 5.2]{FeWu:08}), \cite[Lemma 4.3]{FeWu:11}, and \cite[Lemma 4.2]{Wu:14}. In \cite{FeWu:09,FeWu:11,Wu:14} the bound on the elliptic projection error is proved by observing that the sesquilinear form $\IPLtD{\grad \vo}{\grad \vt}$ is coercive on $\HozDD$, and then also controlling the additional term arising from the impedance boundary condition.

To bound the difference $\Ph u - \uh,$ one first shows that it solves a deterministic Helmholtz problem:
For $u-\uh,$ for any $\vh \in \Vhp$ we have $\aT(u-\uh,\vh) = 0$ and $\aep(u-\Ph u,\vh) = 0$ by Galerkin orthogonality for $u-\uh$ and $u-\Ph u$ respectively\footnote{For the arguments in this \namecref{sec:errorsplit}, we define the elliptic projection in the first argument, see \cref{eq:ellprojfirst} below.}. Therefore
\begin{align*}
  \aT(\Ph u - \uh,\vh) &= \aT(\Ph u - u,\vh) + \aT(u-\uh,\vh)\\
  &= \aT(\Ph u - u,\vh)\\
  &= \aep(\Ph u - u,\vh) - k^2\IPLtD{\Ph u - u}{\vh}\\
  &= - k^2\IPLtD{\Ph u - u}{\vh},
\end{align*}
that is, $\Ph u - \uh$ solves the \emph{discrete} Helmholtz problem
\beqs
\aT(\Ph u - \uh,\vh) = \IPLtD{\ftilde}{\vh} \tforall \vh \in \Vhp,
\eeqs
where $\ftilde = k^2\mleft(u-\Ph u\mright).$ One then uses the fact that $\Ph u - \uh$ satisfies a discrete Helmholtz problem to prove a bound on the difference directly.

In \cite{FeWu:09,FeWu:11,Wu:14} a discrete multiplier argument is used to prove a bound on $\Ph u - \uh$, reminiscent of the multiplier arguments used to prove a priori bounds on Helmholtz problems in \cref{sec:pdetheory}. In \cite{DuWu:15} an argument using higher-order (discrete) norms is used in an argument conceptually similar to the modified duality arguments above; this argument is the heart of the proof in \cref{sec:fem} below. In essence the argument in \cite{DuWu:15} reduces to showing the bounds
\beq\label{eq:duwu1}
\NLtD{\Ph u - \uh} \lesssim \mleft(h + \mleft(hk\mright)^p\mright)\NHokD{u-\Ph u} + \mleft(h^{p+1}k^2 + h^{2p}k^{p+2}\mright)\Nfn{p-1,h}{\Ph u - \uh}
\eeq
and
\beq\label{eq:duwu2}
\Nfn{p-1,h}{\Ph u - \uh} \lesssim h^{2-p} \NHokD{u- \Ph u} + k^{p-1} \NLtD{\Ph u - \uh},
\eeq
where $\Nfn{p-1,h}{\cdot}$ is a discrete norm analagous to the Sobolev norm of order $p-1$. \Cref{eq:duwu1,eq:duwu2} are then combined to show
\beqs
\NLtD{\Ph u - \uh} \lesssim \mleft(h + \mleft(hk\mright)^p\mright)\NHokD{u-\Ph u} + \mleft(\mleft(hk\mright)^{p+1} + h^{2p}k^{2p+1}\mright)\NLtD{\Ph u -\uh},
\eeqs
and the final term can be absorbed into the left-hand side if $h^{2p}k^{2p+1}$ is sufficiently small.

%\opntodo{For this section; for the first set you only need a bound on the elliptic projection error (with the right powers of $h$ and $k$). For the second one, We use quasi-optimality in $H^1$, the lower-order bounds in terms of the higher-order one, and a bound on the best approximation error. Maybe you could combine the first and third and get away with a bound on the elliptic projection error with the correct powers?}

Observe that when using the elliptic projection in such a splitting argument, there are two differences compared to the use of an elliptic projection in modified duality arguments:
\ben
\item One does not need the elliptic projection to be quasi-optimal \cref{eq:epho}, rather, one only need to bound the error $\NHokD{u-\Ph u}$, where $u$ solves a Helmholtz problem, in terms of $\NLtD{f}$.
\item The elliptic projection should be defined in the first argument, not the second, i.e.
  \beq\label{eq:ellprojfirst}
\aT(\Ph u,\vh) = \aT(u,\vh) \tforall \vh \in \Vhp.
  \eeq
\een


%% \subsubsection{Complete technical overview of the literature}\label{sec:litsum}
%% We now give a brief overview of the literature on rigorous quasi-optimality/error bounds for finite-element discretisations of the Helmholtz equation, where the proofs use one of the techniques outlined above. We record this information tersely, having given more insight into the various techniques above. In the columns marked `mesh condition' the quantities listed must be sufficiently small, e.g., if `mesh condition' reads `$hk^2$' then the condition is `$hk^2$ sufficiently small'. If a quantity depends on $p$, but `$p$ fixed' is not specified, then the result holds for varying $p,$ i.e., for $hp$-finite-element methods. IF $p$ does not appear, then the results are for first-order finite-element methods only.
%% Work
%% Mesh condition
%% Error bound
%% Notes
%\paragraph{Quasi-optimality}

%\opntodo{Would it be better to put bounds on $\eta$ here?}


%%5 Can put the below somewhere %%

%% To keep the finite-element error bounded when solving \eqref{eq:introdet}, one must over-refine the numerical grid. That is, rather than using a fixed number of points per wavelength, one must increase the number of points per wavelength as $k$ increases. To achieve bounded finite-element error, one must refine the finite-element mesh size $h$ like $k^{-3/2}$. Whilst this result has been known numerically for some time, it was proven for \eqref{eq:introdet} only with constant coefficients (on various domains and for various finite-element spaces) in \cite{IhBa:95a,Wu:14,DuWu:15,ChNi:18}, and the first proof (to our knowledge) for \eqref{eq:introdet} with heterogeneous coefficients  is contained in \cref{chap:background}. Choosing $h \sim k^{-3/2}$ means \eqref{eq:intromat} is a linear system of size $k^{3d/2}$, larger than if one merely wants the interpolation error to be bounded. Hence, requiring a bounded finite-element error gives rise to very large linear systems.

%% More briefly, if one wants the finite-element solution to be quasi-optimal (that is, up to a constant, the finite-element solution is the best approximation in the finite-element space), then one must over-refine even more, and take $h \sim k^{-2}$. This mesh condition will give rise to linear systems with $k^{2d}$ degres of freedom. See \cref{chap:background} for further details on the necessity of this mesh condition, and further discussion of all the mesh conditions discussed above.\opntodo{EDIT THE ABOVE TO REMOVE `OVER-REFINE'}


%% \subsection{New error bounds for the Helmholtz equation in heterogeneous media}\label{sec:heterr}
%% In this section, we prove that the finite-element approximation of the solution to the Helmholtz TEDP exists if $ h \lesssim k^{-3/2}.$ Moreover, we give an expression for the hidden constant that is completely explicit in $A$ and $n$, and we also prove a bound on the finite-element error, again completely explicit in $A$ and $n$. The argument in this section closely follows those in \cite{FeWu:11,ChNi:18} in its use of an elliptic projection argument to prove the required finite-element existence result and error bound. The paper \cite{FeWu:11} proved a similar result for the Helmholtz equation in homogeneous media, and \cite{ChNi:18} does so for the homogeneous Helmholtz equation with corner singularities.

%% Whilst we prove the results in this section for the TEDP, we expect that they can be extended to the Helmholtz Exterior Dirichlet Problem (EDP) where the infinite domain is truncated, and the Dirichlet-to-Neumann map is realised exactly on the truncated boundary. However, our proof below uses recently-proved bounds on the solution of a related problem to the TEDP from \cite{ChNiTo:18}; in order to extend our results to the EDP we would need analogues to the results in \cite{ChNiTo:18} for the EDP.

%% %% \paragraph{Problem Set-up} Let $\Dm$ be a bounded Lipschitz\ednote{We actually need this to be a $C^{k,\lambda}$ set, for $k+\lambda > 1.5,$ so that we can do the whole non-zero Dirichlet data thing. This is getting a bit complicated. I guess our options are (i) persevere, (ii) give up and just do the theory for zero Dirichlet data, or (iii) assume that we know $\ud,$ not just $\gD.$ Thoughts?} open set such that the open complement $\Dp\de \RRd\setminus \Dmclos$ is connected. Let $\Dtilde$ be a bounded connected Lipschitz open set such that $\Dmclos \subset\subset\Dtilde$. 
%% %% Let $D\de\Dtilde\setminus\Dm$, $\GD\de \partial \Dm$, and $\GI \de\partial \Dtilde$, so that $\partial D= \GD \cup \GI$ and $\GD\cap \GI = \emptyset$. Throughout $\tr$ will denote the trace onto the whole boundary $\dD,$ whereas $\trGI$ and $\trGD$ will denote the traces on $\GI$ and $\GD$ respectively. Throughout we assume there exists some $\kz > 0$ such that $k \geq \kz$. Let $\NW{v}$ denote the weighted $H^1$ norm on $\HoD$:
%% %% \beqs
%% %% \NW{v}^2 \de \NLtD{\grad v}^2 + k^2 \NLtD{v}^2.
%% %% \eeqs


%% %% Let
%% %% \bit
%% %% \item $f\in \LtD$ 
%% %% \item $\gD\in \HthtGD$,
%% %% \item $\gI\in \LtGI$
%% %% \item $n\in \LiDRR$ such that $\dist\mleft(\supp\mleft(1-n\mright),\GI\mright)>0$, satisfying
%% %% \beq
%% %% 0<\nmin \leq n\mleft(\bx\mright)\leq\nmax<\infty\,\, \text{ for almost every } \bx \in D,
%% %% \eeq
%% %% \item $A \in \WoiDRRdtd$ such that $\dist\mleft(\supp\mleft(I -A\mright),\GI\mright)>0$, $A$ is symmetric, and there exist $0<\Amin\leq \Amax<\infty$ such that
%% %% \beq\label{eq:AellEDP}
%% %%  \Amin |\bxi|^2\leq\mleft(A\mleft(\bx\mright) \bxi\mright) \cdot \overline{ \bxi}  \leq \Amax|\bxi|^2 \quad\text{ for almost every }\bx \in D \text{ and for all } \bxi\in \CCd.
%% %% \eeq
%% %% \eit
%% %we say $u\in \HoD$ satisfies the Helmholtz Truncated Exterior Dirichlet Problem (TEDP) if 
%% %\beqs
%% %\grad\cdot\mleft(A \grad u \mright) + k^2 n u = -f \quad \tin D,
%% %\eeqs
%% %\beqs
%% %\trGD u =\gD \quad\ton \GD,
%% %\eeqs
%% %and 
%% %\beq\label{eq:TEDP3}
%% %\dn u - \ii k  \trGI u = \gI \ton \GI.
%% %\eeq
%% In order to study the TEDP with $\gD\neq0,$ we must, in essence reformulate to the TEDP with $\gD=0$ but a different right-hand side for the domain term.

%% %% Define the space
%% %% \beqs
%% %% \HozDD \de \set{v \in \HoD \st \trGD u = 0}.
%% %% \eeqs
%% %and the sesquilinear form and antilinear functional
%% %The variational formulation of the TEDP with $\gD = 0$ is%\opntodo{Check exactly what's needed in hetero}
%% %
%% %\beq\label{eq:tedpz}
%% %\text{Find } u \in \HozDD\quad \tst\quad a(u,v) = F(v)\quad \tfa v \in \HozDD,
%% %\eeq
%% %
%% %where
%% %
%% %\beqs
%% %a(u,v) \de \int_D \mleft(A \grad u\mright)\cdot \grad \vb - k^2 n u\vb - ik \int_{\GI} \trGI u \trGI \vb\quad \tand\quad F(v) \de \int_D f\vb + \int_{\GI} \gI\trGI \vb.
%% %\eeqs
%% %
%% %% In order to deal with non-zero Dirichlet data $\gD,$ we let  $\ud \in \HtD$ be such that $\trGD \ud = \gD$, and $\esssup \ud \compcont D$. The proof that such a $\ud$ exists is in \cref{lem:ud}.
%% %% The variational formulation of the TEDP is then
%% %% \beq\label{eq:tedp}
%% %% \text{Find } u \in \HozDD\quad \tst\quad a(u,v) = F(v)\quad \tfa v \in \HozDD,
%% %% \eeq
%% %% where
%% %% \beqs
%% %% a(u,v) \de \int_D \mleft(A \grad u\mright)\cdot \grad \vb - k^2 n u\vb - ik \int_{\GI} \trGI u \trGI \vb
%% %% \eeqs
%% %% and
%% %% \beqs
%% %% F(v) \de  \int_D \mleft(f - \grad \cdot \mleft(A\grad \ud\mright) - k^2 n\ud\mright)\vb + \int_{\GI} \mleft(\gI-\dn\ud\mright)\trGI \vb.
%% %% \eeqs
%% %% The function $\us = u+ \ud$ is then the solution of the Helmholtz equation
%% %% \beqs
%% %% \grad\cdot\mleft(A \grad \us \mright) + k^2 n \us = -f \quad \tin D,
%% %% \eeqs
%% %% \beqs
%% %% \trGD \us =\gD \quad\ton \GD,
%% %% \eeqs
%% %% and 
%% %% \beq\label{eq:TEDP3}
%% %% \dn \us - \ii k  \trGI \us = \gI \ton \GI.
%% %% \eeq
%% %% \bre[Reducing the smoothness of $\gD$]
%% %% The assumption that $\gD \in \HthtGD$ is made so that the lifting $\ud$ of $\gD$ is in $\HtD$ (see \cref{app:ud}). As $\ud \in \HtD,$ the antilinear functional $F$ defined above is well-defined. We could reduce the smoothness of $\gD$ to $\HoGD$ (meaning $\ud \in \HthtD$) but this reduction in smoothness would then require us to reformulate the functional $F$ as
%% %% \beqs
%% %% F(v) = \int_D \mleft(A \grad \ud\mright)\cdot \grad \vb - k^2 n \ud \vb + f \vb + \int_{\GI} \mleft(\gI - \dn \ud\mright)\vb.
%% %% \eeqs\opntodo{Put a proof of this somewhere, in 28/2/19 notes}
%% %% With this reformulation, $F \in \HozDDprime,$ but does not have a representative function in $\LtD$. Our proofs below will use results from \cite{ChNiTo:18}, which are stated for the TEDP with zero Dirichlet boundary condition and $L^2$ right-hand side. To avoid the complications stated above, and to allow us to use the results in \cite{ChNiTo:18}, we therefore impose the additional smoothness on $\gD.$ Also, in the case with $F$ only in $\HozDDprime$, proving a priori bounds on the solution of the TEDP is more complicated (c.f., e.g., \cite[Theorem 2.5]{GrPeSp:19} and \cite[Corollary 2.16]{GrPeSp:19} which consider the analogous EDP). For the same reason, we assume $A \in \WoiDRRdtd;$ if we only had $A \in \LiDRRdtd,$ we could reformulate $F$ as outlined above, but we would have the same complications as just described.

%% We  restrict our meshes to the following class:
%% \bde[Shape-regular]
%% A family $(\Th)_{h \in (0,1)}$ of meshes of $\DR$ is said to be \defn{shape-regular} if there exists $\rho > 0$ such that for all $T \in \Th$ and for all $h \in (0,1]$
%%   \beqs
%% \diam B(T) \geq \rho \diam T,
%% \eeqs
%% where $B(T)$ is the largest ball contained in $T$ such that $T$ is star-shaped with respect to $B(T)$.
%% \ede

%% The fact that we cannot reduce the smoothness of $\gD$ further to $\HhGD$ is due to the Morawetz multiplier techniques used to obtain the a priori bounds in \cite{GrPeSp:19}, see \cite[(iii), p. 2874]{GrPeSp:19}.
%% %% \ere
%% Also, for later use we state the \defn{adjoint} problem.
%% \beq\label{eq:tedpadj}
%% \text{Find } u \in \HozDDR\quad \tst\quad \aadj(u,v) = F(v)\quad \tfa v \in \HozDDR,
%% \eeq
%% where
%% \beqs
%% \aadj(u,v) \de \int_D \mleft(A \grad u\mright)\cdot \grad \vb - k^2 n u\vb + ik \int_{\GI} \trGI u \trGI \vb.
%% \eeqs
%% %and
%% %\beqs
%% %\Fadj(v) \de \aadj(\uz,v) + \int_D f\vb + \int_{\GI} \gI\trGI \vb.
%% %\eeqs

%% The statement of the main result requires the following related sesquilinear form and \lcnamecref{lem:relatedwp}.

%% \bde[Related sesquilinear form]
%% For $\vo, \vt \in \HozDDR$ we define
%% \beqs
%% \api(\vo,\vt) = \int_D \IP{A \grad \vo}{\vt} - ik\int_{\GI} \vo\vtbar.
%% \eeqs
%% \ede

%% \ble[Related PDE is well-posed and solution is in $H^2$]\label{lem:relatedwp}
%% If $A \in \CzoDRRRdtd,$ then the solution $\psi \in \HozDDR$ of the related PDE
%% \beq\label{eq:relpde}
%% \api(u,v) = \IPLtDR{f}{v}\quad \tfa\quad v \in \HozDDR
%% \eeq
%% exists, is unique, is in $\HtDR,$ and satisfies the a priori bound
%% % \beqs\label{eq:relpdehobound}
%% % \NW{\psi} \lesssim \frac{\max\set{\Amin^{-1},1}}k,
%% % \eeqs
%% % and
%% % \beqs
%% % \NHoD{\psi} \lesssim \CHoell \NLtD{f}
%% % \eeqs
%% % and
%% \beqs
%% \NHtDR{\psi} \lesssim \CHtell \NLtDR{f}.
%% \eeqs
%% for some constant $\CHtell > 0$ depending on $A,$ but independent of $k.$
%% \ele

%% \bre[Proof of \cref{lem:relatedwp}]
%% \Cref{lem:relatedwp} is proved in \cite{ChNiTo:18}, although the dependence on $A$ is not made explicit.
%% \ere

%% %\paragraph{Finite-Element Set-up} Let $\Vh$ be the first-order linear finite-element space on some mesh on $D$ with mesh size $h.$

%% \bas[Existence, uniqueness, and an a priori bound]\label{ass:bound}
%% We assume that the coefficients $A$ and $n$ are such that for all $k \geq \kz$ the solutions of the \cref{prob:vtedp} and its adjoint \eqref{eq:tedpadj} exist, are unique, are in $\HtDR,$ and satisfy the bound
%% \beq\label{eq:hhbound}
%% \NHtDR{u} \lesssim \CHthh \,k \mleft(\NLtDR{f} + \Nunsure{g} + \NLtGD{\gradGD \gD} + k \NLtGD{\gD}\mright),
%% \eeq
%% where $\gradGD$ is the surface gradient on $\GD,$ $u$ is the solution of the TEDP or its adjoint, and $\CHthh >0$ is a constant dependent on $A$, $n,$ and possibly $k.$
%% % \footnote{Determining the dependence of $\CHthh$ on $A$ and $n$ could be tricky. It was done in \cite{ChScTe:13} for a $C^2$ domain with scalar $A$ and homogeneous Dirichlet boundary conditions.}
%%  \eas

%% %% \bde[Finite-element approximation] 
%% %%  The finite-element approximation to \eqref{eq:tedp} is the following:
%% %% \beq\label{eq:tedpfe}
%% %% \text{Find } \uh \in \Vh\quad \tst\quad \aT(\uh,\vh) = F(\vh)\quad \tfa \vh \in \Vh,
%% %% \eeq
%% %% \ede

%% The main theorem we prove is the following:

%% \bth[Finite-element-error bound]\label{thm:febound}
%% If $A \in \CzoDRRRdtd,$ $h \lesssim 1/k,$ \cref{ass:bound} holds, and
%% \beq\label{eq:hcond}
%% h \lesssim \mleft(\NLiDRRR{n} \mleft(\Amax + \half\mright)\CHtell \CHthh\mright)^{-1/2}k^{-3/2}, % There should be a factor of a half in front of the right-hand side of this, as it makes things clearer what's going on in the proof. However, since we're doing everything with \lesssim, a factor of a half doesn't matter. We could replace the half with any \eps in (0,1), but then the constant hidden in the \lesssim in \eqref{eq:hherrltbound} has a factor 1/\eps.
%% \eeq
%% the finite-element solution $\uh$ to the \cref{prob:fevtedp} exists, is unique, and satisfies the bounds
%% \beq\label{eq:hherrltbound}
%% \NLtD{u-\uh} \lesssim \Cfemo \mleft(hk\mright)^2 \mleft(\NLtD{f} + \Nunsure{\gI} + \NLtGD{\gradGD \gD} + k \NLtGD{\gD}\mright)
%% \eeq
%% and
%% \beq\label{eq:hherrwbound}
%% \NW{u-\uh} \lesssim \mleft(\Cfemt hk +  \Cfemth h^2k^3\mright)\mleft(\NLtD{f} + \Nunsure{\gI} + \NLtGD{\gradGD \gD} + k \NLtGD{\gD}\mright),
%% \eeq
%% where
%% \beqs
%% \Cfemo \de \mleft(\Amax + \half\mright)\CHthh^2,
%% \eeqs
%% \beqs
%% \Cfemt \de \frac{\Amax+\half}{\Amin} \CHthh,
%% \eeqs
%% \beqs
%% \Cfemth \de \frac{\mleft(\Amin+ \NLiDRRR{n}\mright)^{1/2}}{\Amin^{1/2}}\Cfemo,
%% \eeqs
%% and $u$ is the solution of \cref{prob:vtedp}.
%% \enth

%% \subsubsection{Properties of the Elliptic Projection, and a related PDE}

%% The proof technique we use below (adapted from \cite{FeWu:11,ChNi:18}) uses an `elliptic projection' of the solution of the TEDP using the related sesquilinear form $\api.$ We define the energy norm induced by the sesquilinear form $\api$:
%% \beqs
%% \Npi{\vo} = \sqrt{\abs{\api(\vo,\vo)}}.
%% \eeqs

%% \ble[Energy Norm is a norm]\label{lem:inducednorm}
%% The induced norm $\Npi{\cdot}$ is a norm on $\HoD.$
%% \ele

%% \bpf[Proof of \cref{lem:inducednorm}]
%% The main thing to check is that, for $v \in \HoD,$ $\Npi{v}=0 \implies v=0.$ By construction, if $\Npi{v}=0,$ then $\int_{D} \IP{A \grad v}{\grad v} =0$ and $\NLtGI{v}^2 = 0,$ as these are the real and imaginary parts of $\api(v,v).$ By \eqref{eq:AellEDP}, it follows that $\Amin \abs{\grad v}^2 \leq 0,$ and thus $v$ is constant. As $\NLtGI{v} = 0,$ it follows that $\trGI v =0,$ and hence by the trace theorem, as $v$ is constant, it follows that $v=0.$

%% Other properties of norms follow analagously as with any definition of an energy norm.
%% \epf
%% \ble[Energy norm is equivalent to weighted norm]\label{lem:normbound}
%% If $v \in \HoD,$ then
%% \beq\label{eq:boundew}
%% \Npi{v} \lesssim \sqrt{\Amax+\half}\NW{v}
%% \eeq
%% and
%% \beq\label{eq:boundwe}
%% \NW{v} \lesssim \max\set{\Amin^{-\half},1} \Npi{v}
%% \eeq
%% \ele

%% \bpf[Proof of Lemma \ref{lem:normbound}]
%% To show \eqref{eq:boundew}, for $ v \in \HoD$ we have
%% \begin{align*}
%%   \Npi{v}^2 &= \abs{\api(v,v)}\\
%%             &\lesssim \abs{\int_{D} \IP{A \grad v}{\grad v}} + k\NLtGI{v}^2 \\
%%             &\lesssim \abs{\int_{D} \IP{A \grad v}{\grad v}} + k\NLtD{v}\NHoD{v}, \text{ by the multiplicative trace inequality}\\
%%             &\lesssim \Amax \NLtD{\grad v}^2 + \half k^2 \NLtD{v}^2 + \half \NHoD{v}^2\\
%%   &\lesssim \mleft(\Amax+\half\mright)\NW{v}^2
%% \end{align*}
%% as required.

%% To show \eqref{eq:boundwe} we first show that, for $v \in \HoD,$ $\Npi{v} \gtrsim \min\set{\Amin^{\half},1} \mleft(\NLtD{\grad v} + k^{\half} \NLtGI{\trGI v}\mright)$:
%% \begin{align}
%%   \Npi{v} &= \mleft(\abs{\api(v,v)}\mright)^{\half}\nonumber\\
%%           &= \mleft(\mleft(\int_D \IP{A \grad v}{\grad v}\mright)^2 + k^2 \mleft(\int_{\GI}\abs{\trGI v}^2\mright)^2\mright)^{\quarter}\nonumber\\
%%   &\geq \mleft(\mleft(\int_D \Amin \abs{\grad v}\mright)^2 + k^2 \NLtGI{\trGI v}^4\mright)^{\quarter}\nonumber\\
%%           &= \mleft(\Amin^2 \NLtD{\grad v}^4 + k^2 \NLtGI{\trGI v}^4\mright)^{\quarter}\nonumber\\
%%           &\geq \min\set{\Amin^{\half},1}\mleft(\NLtD{\grad v}^4 + k^2 \NLtGI{\trGI v}^4\mright)^{\quarter}\nonumber\\
%%   &\gtrsim \min\set{\Amin^{\half},1} \mleft(\NLtD{\grad v} + k^{\half} \NLtGI{\trGI v}\mright), \text{ as } \mleft(x+y\mright)^4 \lesssim x^4 + y^4.\label{eq:Npifour}
%% \end{align}

%% We recall the fact that for $v \in \HoD,$
%% \beq\label{eq:poincarelike}
%% \NLtD{v} \lesssim \NLtD{\grad v} + \NLtGI{\trGI v},
%% \eeq
%% see, e.g., \cite[Equation (6.16)]{Sp:15}. We can then prove \eqref{eq:boundwe}:
%% \begin{align*}
%%    \NW{v} &\lesssim \NLtD{v}+ \NLtD{\grad v}\\
%%           &\lesssim \NLtGI{\trGI v}+ \NLtD{\grad v} + \NLtD{\grad v}\text{ by \eqref{eq:poincarelike}}\\
%%           &\lesssim k^{\half}\NLtGI{v} + \NLtD{\grad v}\\
%%   &\lesssim \max\set{\Amin^{-\half},1}\Npi{v}, \text{ by \eqref{eq:Npifour}.}
%% \end{align*}
%% \epf

%% % \ble[Bound on $L^2$ norm]\label{lem:ltbound}
%% % If $v \in \HoD$ then the bound
%% % \beqs
%% % \NLtD{v} \lesssim  \NLtGI{\trGI v} + \NLtD{\grad v}
%% % \eeqs
%% % holds.
%% % \ele

%% % \bpf[Proof of \cref{lem:ltbound}]
%% % \opntodo{Look at proof in IbyPs article}
%% % If $v \in \HozDD,$ then by the Poincar\'e inequality, we have that $\NLtD{v} \lesssim \NLtD{\grad v}.$ Alternatively, if $\GD = \emptyset$ and $\trGI v$ is constant, then $v - \trGI v \in \HozDD,$ and thus (abusing notation, and letting $\trGI v$ denote the value of the constant, and also a constant function defined on $D$ taking that value everywhere)
%% % \begin{align*}
%% %   \NLtD{v} &\leq \NLtD{\trGI v} + \NLtD{v-\trGI v}\\
%% %   &= \NLtGI{\trGI v} + \NLtD{v-\trGI v}\\
%% %            &\lesssim \NLtGI{\trGI v} +  \NLtD{\grad \mleft(v-\trGI v\mright)}\\
%% %              &= \NLtGI{\trGI v} + \NLtD{\grad v},
%% % \end{align*}
%% % as required.
%% % \epf

%% % \ble[Bound on weighted norm by energy norm]\label{lem:othernormbound}
%% % If $v \in \HozDD,$ or if $\GD = \emptyset$ and $\trGI v$ is constant, the bound
%% % \beq\label{eq:boundwe}
%% % \NW{v} \lesssim \max\set{\Amin^{-\half},1} \Npi{v}
%% % \eeq
%% % holds.
%% % \ele

%% % \bpf[Proof of \cref{lem:othernormbound}]
%% % \epf

%% We now define the elliptic projection of a function in $\HoD.$% and also define a related PDE that will be used in proving the approximation properties of the elliptic projection.

%% \bde[Elliptic Projection]
%% For $w \in \HoD$ we define the \defn{elliptic projection} $\Ph w \in \Vhp$ of $w$ by
%% \beq\label{eq:ellproj}
%% \api(\vh,\Ph w) = \api(\vh,w) \tfa \vh \in \Vhp.
%% \eeq
%% \ede

%% % \bde[Related PDE]\label{lem:relpde}
%% % Given $f \in \LtD$ we define the related (adjoint) PDE; find $\psi \in \HoD$ such that for all $v \in \HoD$
%% % \beq\label{eq:relpde}
%% % \api(\psi,v) = \IPLtD{f}{v}.
%% % \eeq\ede

%% % \bpf[Proof of \cref{lem:relatedwp}]
%% % By \eqref{eq:boundwe} we have, for $v \in \HozDD$
%% % \beqs
%% % \min\set{\Amin,1}\NW{v}^2 \lesssim \abs{\api(v,v)},
%% % \eeqs
%% % and we also have that
%% % \beqs
%% % \NWs{\IP{f}{\cdot}} \leq \frac1k \NLtD{f},
%% % \eeqs
%% % where $\NWs{\cdot}$ denotes the norm on $\HozDDs$ induced by $\NW{\cdot}.$ By the Lax--Milgram Theorem, we can therefore conclude that $\psi$ exists, is unique, and satisfies the bound
%% % \beqs
%% % \NW{\psi} \lesssim \frac{\max\set{\Amin^{-1},1}}{k}\NLtD{f}.
%% % \eeqs
%% % Use Grisvard Magic to get $H^2.$\opntodo{this}
%% % \epf

%% \ble[Properties of elliptic projection]\label{lem:ellprojbounds}
%% Let $A \in \CzoDRRRdtd.$ If $w \in \HtDR,$ then the elliptic projection $\Ph w$ exists, is unique, and the error satisfies the bounds
%% \beq\label{eq:ellprojenbound}
%% \Npi{w-\Ph w} \lesssim \sqrt{\Amax+\half}\,h\NHtDR{w},
%% \eeq
%% and
%% \beq\label{eq:ellprojltbound}
%% \NLtDR{w-\Ph w} \lesssim  \mleft(\Amax+\half\mright)\CHtell\,h^2\NHtDR{w}.
%% \eeq
%% \ele

%% \bpf[Proof of \cref{lem:ellprojbounds}]
%% We first assume $\Ph w$ exists. To show \eqref{eq:ellprojenbound} we apply C\'{e}a's Lemma in $\Vhp$ using the energy norm $\Npi{\cdot}$ to conclude
%% \beqs
%% \Npi{w-\Ph w} \leq \Npi{w-\Ih w}.
%% \eeqs
%% We then apply \cref{lem:normbound,lem:scottzhangbound} to conclude \eqref{eq:ellprojenbound}.

%% To prove \eqref{eq:ellprojltbound} we let $\psi$ solve the related PDE \eqref{eq:relpde} with $f = w-\Ph w.$ By \cref{lem:relatedwp} $\psi \in \HtDR$ and thus by  \cref{lem:normbound} and \cref{lem:scottzhangbound}
%% \beqs
%% \Npi{\psi - \Ih \psi} \lesssim \sqrt{\Amax + \half}\CHtell \,h\NLtDR{w-\Ph w}.
%% \eeqs

%% If we now set $v = w-\Ph w$ in \eqref{eq:relpde}, then we obtain
%% \begin{align}
%%   \NLtDR{w - \Ph w}^2 &= \api\mleft(\psi,w-\Ph w\mright)\nonumber\\
%%                      &= \api\mleft(\psi-\Ih \psi,w-\Ph w\mright) \text{ by Galerkin orthogonality for } w-\Ph w\nonumber\\
%%                      &\leq \Npi{\psi-\Ih \psi}\Npi{w-\Ph w}\nonumber\\
%%                        &\lesssim \sqrt{\Amax + \half}\CHtell \,h\NLtDR{w-\Ph w}\Npi{w-\Ph w}\label{eq:epltfinal}.
%% \end{align}
%% By cancelling $\NLtDR{w- \Ph w}$ from both sides of \eqref{eq:epltfinal} and using \eqref{eq:ellprojenbound} we obtain \eqref{eq:ellprojltbound}.

%% We have proved the bounds \eqref{eq:ellprojenbound} and \eqref{eq:ellprojltbound} under the assumption of existence. To show uniqueness, suppose $\wh, \whtilde$ both satisfy \eqref{eq:ellproj} (with $\Ph = \wh$ or $\whtilde$ respectively). Then by linearity, for all $\vh \in \Vhp,$
%% \beqs
%% \api\mleft(\vh,\Ph\mleft(\wh-\whtilde\mright)\mright) = \IP{\vh}{w-w} = 0.
%% \eeqs
%% That is, the function $\wh - \whtilde$ is an elliptic projection of the zero function.

%% Therefore, by \eqref{eq:ellprojltbound} $\NLtDR{0 - \mleft(\wh - \whtilde\mright)} \lesssim 0,$ i.e., $\wh = \whtilde.$ Therefore, if the elliptic projection $\Ph w$ exists, it is unique. As the space $\Vhp$ is finite-dimensional, by the Rank--Nullity Theorem, the uniqueness of $\Ph w$ implies its existence; hence $\Ph w$ exists, and is unique, as required.
%% \epf


%% \subsubsection{Proof of Main Result}

%% We let $\Ih$ denote the Scott--Zhang quasi-interpolant in $\Vhp$ (see \cite{ScZh:90}), and  use its following property.
%% \ble[Properties of Scott-Zhang interpolant]\label{lem:scottzhangbound}
%% let $h \lesssim 1/k.$ If $w \in \HtDR,$ then
%% \beq\label{eq:scottzhangbound}
%% \NW{w - \Ih w} \lesssim h \NHtDR{w}.
%% \eeq
%% \ele

%% \bpf[Proof of \cref{lem:scottzhangbound}]
%% The Scott-Zhang interpolant $\Ih w$ satisfies
%% \beq\label{eq:szlt}
%% \NLtDR{w-\Ih w} \lesssim h^2 \NHtDR{w}
%% \eeq
%% \and
%% \beq\label{eq:szho}
%% \NHoDR{w-\Ih w} \lesssim h \NHtDR{w}.
%% \eeq
%% Hence by the definition of $\NW{\cdot},$ by combining \eqref{eq:szlt} and \eqref{eq:szho} we have
%% \beqs
%% \NW{w-\Ih w} \lesssim h\mleft(1+hk\mright)\NHtDR{w}.
%% \eeqs
%% As $h\lesssim 1/k,$ \eqref{eq:scottzhangbound} follows.
%% \epf

%% The following \lcnamecref{cor:hhszbound} follows from \cref{ass:bound}, and is used to prove \cref{thm:febound}.

%% \bco\label{cor:hhszbound}
%% If $u$ is the solution of the Helmholtz Interior Impedance Problem (or its adjoint) then the error in the Scott--Zhang quasi-interpolant satisfies
%% \beq\label{eq:hhszbound}
%% \NW{u-\Ih u} \lesssim \CHthh hk \mleft(\NLtDR{f} + \Nunsure{g} + \NLtGD{\gradGD \gD} + k \NLtGD{\gD}\mright).
%% \eeq
%% \eco

%% The proof of the main theorem (\cref{thm:febound} below) also uses the fact that $a$ satisfies a G\r{a}rding inequality.
%% \ble[G\r{a}rding inequality]
%% If $v \in \HozDDR,$ then
%% \beq\label{eq:garding}
%% \Re\mleft(\aT(v,v)\mright) \geq \Amin \NW{v}^2 - k^2\mleft(\Amin + \NLiDRRR{n}\mright)\NLtDR{v}^2,
%% \eeq
%% where $\Re$ denotes the real part.
%% \ele

%% Finally, we recall \defn{Cauchy's inquality}: For all $a,b \in \RR$, and for all $\eps > 0,$
%% \beq\label{eq:cauchy}
%% ab \leq \frac{a^2}{2\eps} + \frac{\eps b^2}{2}.
%% \eeq

%% \bpf[Proof of \cref{cor:hhszbound}]
%% The proof follows from \cref{lem:scottzhangbound,ass:bound}.
%% \epf

%% We are now in a position to prove our main theorem.




%% \bpf[Proof of \cref{thm:febound}]
%% In this proof, for brevity we let
%% \beqs
%% \Mfg = \NLtDR{f} + \Nunsure{\gI} + \NLtGD{\gradGD \gD} + k \NLtGD{\gD}.
%% \eeqs
%% By \cref{ass:bound} the solution $u$ of the TEDP exists and is unique. Assume the finite-element solution $\uh$ exists. Let $\xi \in \HoD$ satisfy the adjoint TEDP \eqref{eq:tedpadj} with $f=u-\uh,$ $\gD=0,$ and $\gI=0.$ Taking complex conjugates, it follows that
%% \beq\label{eq:errordual}
%% \aT(v,\xi) = \IPLtDR{v}{u-\uh} \tfa v \in \HoDR.
%% \eeq
%% By \cref{ass:bound} $\xi$ exists, is unique and is in $\HtDR.$. Setting $v = u-\uh$ in \eqref{eq:errordual} we obtain
%% \begin{align*}
%%   \NLtDR{u-\uh}^2 &= a\mleft(u-\uh,\xi\mright)\\
%%                  &= a\mleft(u-\uh,\xi-\Ph \xi\mright) \quad\text{by Galerkin orthogonality for } u-\uh\\
%%                  &= \api\mleft(u-\uh,\xi-\Ph \xi\mright) - k^2 \IPLtDR{n\mleft(u-\uh\mright)}{\xi-\Ph \xi}\\
%%                  &= \api\mleft(u-\Ih u,\xi-\Ph \xi\mright) - k^2 \IPLtDR{n\mleft(u-\uh\mright)}{\xi-\Ph \xi}\\
%%   &\quad\quad\quad\text{by Galerkin orthogonality for }\xi  - \Ph \xi\\
%%                  &\leq \Npi{u-\Ih u}\Npi{\xi - \Ph \xi} + \NLiDRRR{n} k^2 \NLtDR{u-\uh}\NLtDR{\xi-\Ph \xi}\\
%%                  &\lesssim \sqrt{\Amax + \half}\, \CHthh\, hk \Mfg\Npi{\xi-\Ph \xi}\\
%%   &\quad\quad+  \NLiDRRR{n} k^2 \NLtDR{u-\uh}\NLtDR{\xi-\Ph \xi}\quad \text{by \eqref{eq:boundew} and \eqref{eq:hhszbound}}\\
%%                  &\lesssim \sqrt{\Amax + \half}\, \CHthh\, hk\Mfg\,\sqrt{\Amax + \half}\,h\NHtDR{\xi}\\
%%                  &\quad\quad  + \NLiDRRR{n} k^2 \NLtDR{u-\uh}\NLtDR{\xi-\Ph \xi}\quad \text{by \eqref{eq:ellprojenbound}}\\
%%                  &\lesssim \mleft(\Amax + \half\mright)\CHthh\, hk\Mfg \, \CHthh \,hk \NLtDR{u-\uh}\\
%% &\quad\quad  + \NLiDRRR{n} k^2 \NLtDR{u-\uh}\NLtDR{\xi-\Ph \xi}\quad \text{by \eqref{eq:hhbound}}\\
%% \end{align*}
%% Cancelling a factor of $\NLtDR{u-\uh}$ and rearranging terms we obtain
%% \beqs
%% \NLtDR{u-\uh} \lesssim \mleft(\Amax + \half\mright)\CHthh^2 \mleft(hk\mright)^2\Mfg + k^2 \NLiDRRR{n} \NLtDR{\xi - \Ph \xi}
%% \eeqs
%% and therefore
%% \begin{align*}
%% &  \NLtDR{u-\uh} \lesssim \mleft(\Amax + \half\mright)\CHthh^2 \mleft(hk\mright)^2 \Mfg\\
%% &\quad\quad  + h^2k^3 \NLiDRRR{n} \mleft(\Amax + \half\mright) \CHtell \CHthh \NLtDR{u-\uh}
%% \end{align*}
%%   using the definition of $\xi$, \eqref{eq:ellprojltbound}, and \eqref{eq:hhbound}Therefore if $h$ satisfies \eqref{eq:hcond} we obtain \eqref{eq:hherrltbound}.
%% % \beqs
%% % \half \NLtD{u-\uh} \lesssim \mleft(\Amax + \half\mright)\CHthh \mleft(hk\mright)^2 \mleft(\NLtD{f} + \Nunsure{g}\mright).
%% % \eeqs
%% % that is, if $\Cmess \de (\mleft(\NLiDRR{n} \mleft(\Amax + \half\mright) \CHthh \mright)^{-1/2},$ then
%% % \beqs
%% % \NLtD{u-\uh} \lesssim \mleft(\Amax + \half\mright)\CHthh \Cmess^2k^{-1} \mleft(\NLtD{f} + \Nunsure{g}\mright),
%% % \eeqs
%% To obtain the bound \eqref{eq:hherrwbound}, we use the G\r{a}rding inequality \eqref{eq:garding}:
%% \begin{align*}
%%   \Amin \NW{u-\uh}^2 &\leq \Re\mleft(a\mleft(u-\uh,u-\uh\mright)\mright) + k^2\mleft(\Amin+ \NLiDRRR{n}\mright) \NLtDR{u-\uh}^2\\
%%                      &= \Re\mleft(a\mleft(u-\uh,u-\Ih u\mright)\mright) + k^2\mleft(\Amin+ \NLiDRRR{n}\mright) \NLtDR{u-\uh}^2\\
%%   &\quad\quad\quad\text{by Galerkin orthogonality}\\
%%                      &\leq \mleft(\Amax+\half\mright) \NW{u-\uh}\NW{u-\Ih u} + k^2\mleft(\Amin+ \NLiDRRR{n}\mright) \NLtDR{u-\uh}^2\\
%%   &\quad\quad\quad\text{by \cref{lem:normbound}}\\
%%   &\leq \frac{\mleft(\Amax+\half\mright)^2}{2\Amin} \NW{u-\Ih u}^2 + \frac{\Amin}2 \NW{u-\uh}^2 + k^2\mleft(\Amin+ \NLiDRRR{n}\mright) \NLtDR{u-\uh}^2,\\
%% \end{align*}
%% by Cauchy's inequality \eqref{eq:cauchy} with $\eps = \Amin.$ Therefore,
%% \beqs
%% \NW{u-\uh}^2 \leq \frac2{\Amin} \mleft(\frac{\mleft(\Amax+\half\mright)^2}{2\Amin} \NW{u-\Ih u}^2+ k^2\mleft(\Amin+ \NLiDRRR{n}\mright) \NLtDR{u-\uh}^2\mright)
%% \eeqs
%% and hence
%% \beq\label{eq:hherrboundnearly}
%% \NW{u-\uh} \lesssim \frac1{\Amin^{1/2}} \mleft(\frac{\Amax+\half}{\Amin^{1/2}} \NW{u-\Ih u}+ k\mleft(\Amin+ \NLiDRRR{n}\mright)^{1/2} \NLtDR{u-\uh}\mright).
%% \eeq
%% By substituting \eqref{eq:hhszbound} and \eqref{eq:hherrltbound} into \eqref{eq:hherrboundnearly} we obtain \eqref{eq:hherrwbound}.

%% To show that $\uh$ exists, as in the proof of \cref{lem:ellprojbounds} we can use the error bound \eqref{eq:hherrltbound} to show that $\uh$ is unique, and we can then use the fact that $\Vhp$ is finite-dimensional to show that $\uh$ exists.
%% \epf

\section{New Finite-Element-Error bounds for the Heterogeneous Helmholtz Equation}\label{sec:fem}

We now prove our new error bounds for higher-order finite-element approximations of the solution of the Helmholtz equation in heterogeneous media--- we show that (for nontrapping media) the finite-element method applied to \cref{prob:tedp} is $\hka{(2p+1)/(2p)}$-data-accurate. Our results are a generalisation of results proved by Du and Wu \cite{DuWu:15} for higher-order finite-element approximations of the Helmholtz equation in homogeneous media; our results and proofs broadly follow those in \cite{DuWu:15}, with the main differences that:
\ben
\item We modify the proofs to cater for the heterogeneity of the coefficients, and
\item The dependence of our results on all of the constants involved is explicit.
\een
  In particular our results are explicit in $n$ and $k$ and are (in principle) explicit in $A$.

The proofs of our results have many parts, and appear technical, largely due to the burden of explicitly keeping track of all of the constants involved. However, in essence, the proof consists of two ideas:
\ben
\item Decompose the error $u-\uh = (u - \Ph u) + (\Ph u - \uh)$, where $\Ph u$ is an elliptic projection of $u$, and
\item Bound the error $u - \Ph u$ using the fact that $\Ph u$ is a Galerkin projection.
\item Bound the error $\Ph u - \uh$ in higher-order `discrete' norms, using the fact that $\Ph u - \uh$ solves a discrete Helmholtz problem.
\een

The structure of the remainder of this \lcnamecref{sec:fem} is as follows. In \cref{sec:errbounds} we state our new finite-element-error bounds. In \cref{sec:decomp} we prove a decomposition of the solution that allows us to prove a higher-order best approximation bound. In \cref{sec:anbackground} we collect together some routine analysis results needed for our proofs. In \cref{sec:errgalerkin} we prove error bounds for a number of different Galerkin projections (including the elliptic projection) that we use in subsequent proofs. In \cref{sec:discsob} we develop a notion of discrete derivatives and discrete norms, and prove properties of these norms. These norms will allow us to define higher-order discrete norms of functions in our finite-element space. Finally in \cref{sec:fembound} we prove our main finite-element error bounds.



\subsection{Main result: new finite-element-error bounds}\label{sec:errbounds}
As we are using higher-order finite elements (which we assume are of degree $p$), we will require extra smoothness assumptions on $A$, $n,$ and the boundaries $\GD$ and $\GI$. We also make simple assumptions on $k,$ $\nmin$, and $\NLiD{n}$ in order to simplify our calculations.
\bas[Assumptions for higher-$p$ finite-element method bounds]\label{ass:highp}
Assume
\bit
\item $\GD \neq \emptyset,$
\item $\GD$ and $\GI$ are $\Cpo$,
\item $\Aij \in \CpmooDclos$ for all $i,j$, and
\item $n \in \LiD \cap \HpmoD \cap \CfloordtD.$
%\item $f \in \HpmoD$
%  \item $\gI \in \HpmhGI$
  \eit
  \eas
  The assumption $\GD \neq \emptyset$ allows us to show the elliptic projection operator defined in \cref{eq:epdef} below is well-defined. The regularity assumptions on $\GD,$ $\GI$, and $A$ ensure that we can apply a shift theorem for a related stationary diffusion equation up to order $p-1$. The assumption on $n$ ensures that for all $m \in [0,p-1]$ and for all $v \in \HmD,$ $nv \in \HmD$ (see \cref{thm:banachalg} below). However, observe that we make no additional assumptions on the data $f$ and $\gI,$ and so the solution $u$ may not be smoother than $H^2.$

We make the following \lcnamecref{ass:htwo} on the solution of \cref{prob:vtedp}. Recall that the adjoint of \cref{prob:vtedp} is the same as \cref{prob:vtedp}, except \cref{eq:vtedp} is replaced by
\beq\label{eq:adjoint}
\aT(v,u) = \overline{\FT(v)} \tforall v \in \HozDDR.
\eeq

\bas\label{ass:htwo}
\Cref{prob:vtedp} (and its adjoint) has a unique solution $u$ in $\HtD$, and there exists $\CAnk>0$ (possibly dependent on $A$, $n,$ and $k$) such that
\beq\label{eq:generalhtwo}
k \NLtD{u} + \SNHoD{u} + \frac1k \SNHtD{u} \leq \CAnk \Cfg,
\eeq
where $\Cfg \de \NLtD{f} + \NLtGI{\gI}$.
\eas

Finally, we make the following \lcnamecref{ass:convenient} to simplify the proofs in this section. \Cref{ass:convenient} is by no means necessary, but greatly simplifies the proofs.

\bas[Assumptions for convenience of proofs]\label{ass:convenient}
Assume: $k \geq 1$, $\NLiD{n} \geq 1,$ $\nmin \leq 1$, $hk \leq 1$, and there exists $\Ctildemin > 0$ independent of $k$ such that $\CAnk k \geq \Ctildemin.$
\eas
%Note that one could require $\NHhGI{\gI}$ on the right-hand side of \cref{eq:generalhtwo} (as $\gI \in \HhGI$); however, since the bound in \cref{thm:tedp} only has $\NLtGI{\gI}$, we use the form \cref{eq:generalhtwo} to include \cref{thm:tedp}.

Throughout this \lcnamecref{sec:fem}, we adopt the following piece of notation.
\beqs
\nvar = \frac{\NLiD{n}}{\nmin},
\eeqs
%% and
%% \beqs
%% \En = \max\set{\frac{\NHsD{n}}{\nmin^2} \st s \in (d/2,p-1];\frac{\NWsiD{n}}{\nmin^2}\st s \in [2,d/2];1}\nvar.
%% \eeqs
%The quantity $\En$ arises from bounds on the `elliptic projection' mentioned above in weighted norms, see \cref{lem:ellprojerrw} below.

Our main \lcnamecref{thm:fembound} is the following \lcnamecref{thm:fembound}.
\bth[Higher-order error bound for the heterogeneous Helmholtz equation]\label{thm:fembound}
Let $u$ be the solution of \cref{prob:vtedp}. Under \cref{ass:highp,ass:htwo,ass:convenient}, there exist constants $\CFEMLt, \CFEMHo, \Chcond > 0$, independent of $h$, $k$ and $n$, such that if
\beq\label{eq:hfemcond}
h \leq \Chcond \Condn(n) \CAnk^{-\frac1{2p}}k^{-1-\frac1{2p}},
\eeq
then the finite-element solution $\uh$ exists, is unique, and satisfies the error bounds
\beq
\NLtD{u-\uh} \leq \CcorLt \CLtn(n)\mleft(h^2 + \CAnk^2 (hk)^{2p}\mright)\Cfg\tand\label{eq:femltbound}
\eeq
\beq
\NHokD{u-\uh} \leq \CcorHo \CHon(n)\mleft(h + \CAnk (hk)^p + \CAnk^2k (hk)^{2p}\mright)\Cfg,\label{eq:femhobound}
\eeq
where
\beqs
\Condn(n) = \Pfn{p-2}\mleft(\NLiD{n}\mright)^{-\frac1{p+1}}\mleft(\mleft(\En\nvar\mright)^{\half(\floor{\frac{p-1}2}+1)}\NLiD{n}\nmin^{-\frac{p}2}\mright)^{-\frac{p-1}{2p}},
\eeqs
\beqs
\CLtn(n) = \nvar^6\nmin^{-(p+1)} \mleft(\En \nvar\mright)^{\floor{\frac{p-1}2}+1}\Pfn{p-2}\mleft(\NLiD{n}\mright)\tand
\eeqs
\beqs
\CHon(n) = \mleft(1+\mleft(\En\nvar\mright)^{\half(\floor{\frac{p-1}2}+1)}\NLiD{n}\nmin^{-\frac{p}2}\mright)\CLtn,
\eeqs
where the polynomial $\Pfn(x)$ is defined in \cref{eq:p} below, and the $n$-dependent constant $\En$ is defined in \cref{eq:en} below.
\enth
%\bco[Higher-order-finite-element bound]\label{cor:fembound}
%Under the assumptions of \cref{thm:fembound},
%\eco

The proof of \cref{thm:fembound} is on \cpageref{page:thmfembound} below.

\bco[$\hka{(2p+1)/(2p)}$-data-accuracy]
Under the assumptions of \cref{thm:fembound}, if $\CAnk \lesssim 1$ (i.e., \cref{prob:tedp} is nontrapping), then the finite-element method is $\hka{(2p+1)/(2p)}$-data-accurate (where the constants $\Co$ and $\Ct$ in \cref{def:hkdataacc} depend on $A$ and $n$).
\eco

Whilst the calculations in this \lcnamecref{sec:fem} are explicit in all the constants involved, these dependencies are complex and, to a large extent, unnecessary to understand the flow of the arguments. Therefore, for ease of reference, the definition of all the constants in this \lcnamecref{sec:fem} (which are many-layered and interdependent) are given in \cref{app:constants}; i.e., any constant introduced or defined in this \lcnamecref{sec:fem} will be listed in \cref{app:constants}.



\subsubsection{Discussion of new finite-element-error bounds}

\bre[Relationship of new bounds to the work of Du and Wu]
In \cite{DuWu:15} Du and Wu proved that the error for high-order finite-element methods for the \emph{homogeneous} Helmholtz interior impedance problem is bounded provided $h^{2p}k^{2p+1}$ is sufficiently small. Our proof follows theirs, and achieves analagous results---observe that the condition \cref{eq:hfemcond} requires $h^{2p}k^{2p+1}$ to be sufficiently small, where the `sufficient smallness' condition depends on $A$ and $n$. We note as above that our results and proof have the following modifications to those of Du and Wu (the modifications listed in order of their impact upon the proof).
\ben
\item We prove bounds for \emph{heterogeneous} $A$ and $n$; in particular, in several places in the proof we must work with $n$-weighted inner products and norms; see \cref{rem:why} for more details on why $n$-weighted inner products and norms are required.
\item We explicitly track all of the constants involved in the proof---our results are completely explicit in $n$, and are in theory explicit in $A$ (see \cref{rem:explicita} below for information on the explicitness in $A$).
\item We allow for the possibility that the Helmholtz problem is trapping---the constant $\CAnk$ appearing in \cref{eq:hfemcond} may depend on $k$ (as well as $A$ and $n$). If the Helmholtz problem was nontrapping, $\CAnk$ would be independent of $k$.
  \item We assume the existence of a Dirichlet scatterer $\Dm,$ as opposed to only considering the interior impedance problem, where $\Dm=\emptyset.$
\een
\ere

\bre[Why are the results not fully explicit in $A$?]\label{rem:explicita}
The condition \cref{eq:hfemcond} and the bounds \cref{eq:femltbound,eq:femhobound} are not fully explicit in $A$, i.e. the constants $\Ccond, \CcorLt,$ and $\CcorHo$ may depend on $A$. This dependence is because the constants in the shift theorem for the stationary diffusion equation (\cref{thm:shift} below) are not explicit in their $A$-dependence. In principle one can determine this dependence (it is determined for a right-hand side in $\LtD$ and a solution in $\HtD$ in \cite[Appendix A]{ChScTe:13}).
\ere
\bre[Why the appearance of $\nvar$?]\label{rem:nvar}
The quantity $\nvar = \NLiD{n}/\nmin$ appears in multiple places in the definition of the $n$-dependent constants $\Condn(n), \CLtn(n),$ and $\CHon(n).$ This appearance is mainly due to the fact that in multiple places in the proof of \cref{thm:fembound}, we must convert from working in an $n$-weighted norm to a standard norm, and then later convert back to an $n$-weighted norm again.

This conversion is usually necessary because certain results (such as \cref{thm:shift,lem:poincare,thm:trace,thm:multiplicativetrace}) are only available in the literature in standard (non-$n$-weighted) norms, and so to apply these results we must first transfer to non-$n$-weighted norms, apply the results, and then tranfer back. If one could prove these results for $n$-weighted norms (under sufficient smoothness conditions on $n$) with constants that were completely explicit in $n$, then many of the instances of $\nvar$ could be removed.
\ere

\bre[Bounds not sharp in $n$]\label{rem:nsharp}
The $n$-dependence of the constants $\Condn(n), \CLtn(n),$ and $\CHon(n)$ is almost certainly not sharp, because
\ben
\item The proof of \cref{thm:fembound} is complex, and involves recursively applying bounds on finite-element functions (as in, e.g., the proof of \cref{lem:negdiscsum}, see \cref{eq:evenrecursivesum,eq:oddrecursive}). These arguments may well result in non-sharp $n$-dependence.
  \item As described in \cref{rem:nvar} above, many of the appearances of $\nvar$ in the constants $\Condn(n), \CLtn(n),$ and $\CHon(n)$ is purely due to changing norms, then changing back, and so it is possible that the dependence of these constants on $\nvar$ (and therefore $n$) is not sharp.
\een
\ere

\bre[$p$-dependence]
To prove error bounds for $hp$-finite-element methods, one must know explicitly how the error bounds \cref{eq:femltbound,eq:femhobound} depend on $p$ as well as $h$ and $k$. Such an analysis was carried out for the Helmholtz equation in homogeneous media by Melenk and Sauter in \cite{MeSa:10,MeSa:11} and for the related time-harmonic Maxwell's equations in homogeneous media in \cite{MeSa:18}. In principle (as our results are explicit in all of the constants involved) one could calculate the $p$-dependence, however, such dependence is likely to be highly complicated, and may very well not be sharp, for similar reasons that the $n$-dependence of our results may not be sharp, as outlined in \cref{rem:nsharp} above.
\ere

\bre[Special cases of $n$]
In order to understand the constants $\Condn(n), \CLtn(n),$ and $\CHon(n)$ it may be instructive to note their behaviour in the following three cases:
\bit
\item If $n=1,$ then $\Condn(n)=1, \CLtn(n)=1,$ and $\CHon(n)=2$.
\item If $\nmin$ is fixed, and $\NLiD{n} \rightarrow \infty,$ then $\Condn(n)\rightarrow 0,$ (i.e., the condition \cref{eq:hfemcond} becomes more restrictive), $\CLtn(n)\rightarrow \infty,$ and $\CHon(n)\rightarrow \infty$ (i.e., the right-hand side of the error bounds \cref{eq:femltbound,eq:femhobound} become larger).
\item If $\NLiD{n}$ is fixed, and $\nmin \rightarrow 0,$ then $\Condn(n)\rightarrow 0,$ $\CLtn(n)\rightarrow \infty,$ and $\CHon(n)\rightarrow \infty$ (i.e., the right-hand side of the error bounds \cref{eq:femltbound,eq:femhobound} become larger).
  \eit
  One can easily check each of these behaviours using the definition of $\nvar$, the definition of the constants $\Condn(n),$ $\CLtn(n)$, and $\CHon(n)$, and the definition of the polynomial $\Pfn{j}(x)$ in \cref{eq:p} below, and the definition of $\En$ in \cref{eq:en} below.
  \ere

  \bre[Extension to different boundary conditions on $\GI$]
  As in \cite[Remark 5.4(e)]{DuWu:15}, we remark that it is not obvious how to extend the proof of \cref{thm:fembound} to a Helmholtz problem with an exact Dirichlet-to-Neumann boundary condition on $\GI.$ In \cref{lem:higherbound,lem:ltthetahbound} below one must bound terms involving $\thetah$ on the truncation boundary $\GI$; these bounds are achieved using \cref{lem:boundarybound}, where we bound $\NLtGI{\thetah}$ by higher-order discrete norms of $\thetah$ and by the $k$-weighted $H^1$-norm of $\rho.$

  However, a crucial part of the proof of \cref{lem:boundarybound} is the fact that one has the bound $k \NLtGI{\thetah}^2 \lesssim \IPLtGI{\DtNapprox \thetah}{\thetah}$, where $\DtNapprox$ is the impedance approximation to the DtN map $\DtN,$ i.e., $\DtNapprox = ik.$ To replicate the proof of \cref{lem:boundarybound} for an exact DtN boundary condition, one would need to show $k \NLtGI{\thetah}^2 \lesssim \IPLtGI{\DtN \thetah}{\theta}$ for the \emph{exact} DtN map $\DtN.$ As mentioned in \cite[Remark 5.4(e)]{DuWu:15}, whether such a bound holds in general is an open question; such as bound has only been shown in \cite[Equation (3.4b)]{MeSa:10} for the case $\GI$ is a circle (in 2-d) or a sphere (in 3-d).
  \ere

\subsection{Decomposition of solution and best approximation bound}\label{sec:decomp}

For the first part of the proof of \cref{thm:fembound}, we prove a best approximation bound in $\Vhp$ for the solution of the Helmholtz equation following the presentation in \cite{ChNi:19} (although we explicitly keep track of the constants involved at each point). In order to obtain bounds for high $p$, we require the following shift \lcnamecref{thm:shift}:

\bth[Shift theorem]\label{thm:shift}
Under \cref{ass:highp}, for all integers $l \in \mleft[0,p-1\mright]$ there exists a constant $\CAl>0$ (depending on $A$) such that if $\ftilde \in \HlD$ and $\gItilde \in \HlphGI$, then there exists a unique $\utilde \in \HlptD$ such that $\utilde$ solves
\beq\label{eq:shifteq}
\grad \cdot \mleft(A \grad \utilde\mright) = -\ftilde,
\eeq
\beq\label{eq:shiftdbc}
\dn \utilde = \gItilde, \tand
\eeq
\beq\label{eq:shiftnbc}
\trD \utilde = 0
\eeq
and $\utilde$ satisfies the bound
\beq\label{eq:shift}
\NHlptD{\utilde} \leq \CAl \mleft(\NHlD{\ftilde} + \NHlphGI{\gItilde}\mright).
\eeq
\enth

\bpf[Proof of \cref{thm:shift}]
The uniqueness and existence of $\utilde$ (in $\HozDD$) follows from the Lax--Milgram theorem, as the variational formulation of \cref{eq:shifteq,eq:shiftdbc,eq:shiftnbc} is bounded and coercive. The proof for the higher regularity bounds uses standard elliptic regularity estimates in the interior and near the boundaries $\GD$ and $\GI$, and the work of the proof is combining these estimates. As a reference for these estimates we use \cite[pp. 137-138]{Mc:00}; \cref{ass:highp} means we can apply these estimates, as we have the necessary higher regularity of the coefficients and the boundaries $\GD$ and $\GI$.

To deal with the interior regularity and regularity near the boundary separately, we define the following subsets of $D$: $\Dint,\Dinttilde,\Dscat,$ and $\Dtrunc$ (see \cref{fig:shift-int,fig:shift-scat,fig:shift-trunc} for a schematic) with the following properties:
\bit
\item $\Dint \compcont \Dinttilde \compcont D$,
\item $\GD \subset \Dscatclos$
\item $\dist(\Dscat,\GI) > 0 $
  \item $\GI \subset \Dtruncclos$
\item $\dist(\Dtrunc,\GD) > 0 $
  \eit
\def\shiftscale{1.4}
  \begin{figure}
\begin{centering}
\begin{tikzpicture}[even odd rule]
  \input{diagram-defs-smooth}
\draw[scale=\shiftscale] % Truncation
\truncationsmooth
\obstaclesmooth
;
    

\draw[scale=\shiftscale] % The obstacle
\obstaclesmooth
;




% The labels
\draw ($\shiftscale/\domainscale*(-0.1,0)$) node[fill=white] {$\Dm$};
%\draw (-3.5,-4.75) node[fill=white] {$D:=\Dtilde\setminus \clos{\Dm}$};

\draw ($\shiftscale/\domainscale*(2.0,-0.5)$) node[fill=white] {$\GD$};
%\draw[scale=\shiftscale] (2.418221983093764, 0.6211657082460498) node[fill=white] {$\BR$};
\def\GIdraw{\draw ($\shiftscale/\domainscale*(5.1,3.1)$) node[fill=white] {$\GI$};}
\GIdraw

% D_int
\draw[scale=\shiftscale] 
(0,0) circle [radius=1.44]
(0,0) circle [radius=2.16];

%D_int_tilde
\filldraw[pattern=owennorthwest,scale=\shiftscale,dashed]
(0,0) circle [radius=1.2]
(0,0) circle [radius=2.4];

\draw ($\shiftscale*(1.27,1.27)$) node[fill=white] {$\Dint$};
\draw ($\shiftscale*(-2.06,1.51)$) node[fill=white] {$\Dinttilde$};

\end{tikzpicture}

\caption{A schematic of the sets $\Dint$ (between solid circles only) and $\Dinttilde$ (between dotted circles) from the proof of \cref{thm:shift}.}\label{fig:shift-int}
\end{centering}
  \end{figure}
    \begin{figure}
\begin{centering}
\begin{tikzpicture}[even odd rule]
  \input{diagram-defs-smooth}
\draw[scale=\shiftscale] % Truncation
\truncationsmooth
\obstaclesmooth
;
    

\draw[scale=\shiftscale] % The obstacle
\obstaclesmooth
;




% The labels
\draw ($\shiftscale/\domainscale*(-0.1,0)$) node[fill=white] {$\Dm$};
%\draw (-3.5,-4.75) node[fill=white] {$D:=\Dtilde\setminus \clos{\Dm}$};

\draw ($\shiftscale/\domainscale*(2.0,-0.5)$) node[fill=white] {$\GD$};
%\draw[scale=\shiftscale] (2.418221983093764, 0.6211657082460498) node[fill=white] {$\BR$};
\def\GIdraw{\draw ($\shiftscale/\domainscale*(5.1,3.1)$) node[fill=white] {$\GI$};}
\GIdraw

% D_scat
  \filldraw[pattern=owennorthwest,scale=\shiftscale] (0,0) circle [radius=2.1]
  \obstaclesmooth;

  \draw ($\shiftscale/\domainscale*(2.0,-0.5)$) node[fill=white] {$\GD$};

  \draw ($\shiftscale*(1.1,1.1)$) node[fill=white] {$\Dscat$};
\end{tikzpicture}

\caption{A schematic of the set $\Dscat$ from the proof of \cref{thm:shift}.}\label{fig:shift-scat}
\end{centering}
    \end{figure}
      \begin{figure}
\begin{centering}
\begin{tikzpicture}[even odd rule]
  \input{diagram-defs-smooth}
\draw[scale=\shiftscale] % Truncation
\truncationsmooth
\obstaclesmooth
;
    

\draw[scale=\shiftscale] % The obstacle
\obstaclesmooth
;




% The labels
\draw ($\shiftscale/\domainscale*(-0.1,0)$) node[fill=white] {$\Dm$};
%\draw (-3.5,-4.75) node[fill=white] {$D:=\Dtilde\setminus \clos{\Dm}$};

\draw ($\shiftscale/\domainscale*(2.0,-0.5)$) node[fill=white] {$\GD$};
%\draw[scale=\shiftscale] (2.418221983093764, 0.6211657082460498) node[fill=white] {$\BR$};
\def\GIdraw{\draw ($\shiftscale/\domainscale*(5.1,3.1)$) node[fill=white] {$\GI$};}
\GIdraw


% D_trunc
\draw[scale=\shiftscale] (0,0) circle [radius=1.4];
\end{tikzpicture}

\caption{A schematic of the set $\Dtrunc$ from the proof of \cref{thm:shift}.}\label{fig:shift-trunc}
\end{centering}
\end{figure}

    First applying interior regularity \cite[Theorem 4.16]{Mc:00} in $\Dinttilde,$ we obtain the bound
    \beq\label{eq:shiftint}
\NHlptDint{\utilde} \leq \CintAl \mleft(\NHoDinttilde{\utilde} + \NHlDinttilde{\ftilde}\mright).
\eeq
Applying regularity up to the boundary for Dirichlet data \cite[Theorem 4.18 (i)]{Mc:00} in $\Dscat,$ we obtain (as $\trD \utilde = 0$)
\beq\label{eq:shiftscat}
\NHlptDscat{\utilde} \leq \CscatAl \mleft(\NHoD{\utilde} + \NHlD{\ftilde}\mright)
\eeq
and similarly for Neumann data \cite[Theorem 4.18 (ii)]{Mc:00} in $\Dtrunc,$ we obtain
\beq\label{eq:shifttrunc}
\NHlptDtrunc{\utilde} \leq \CtruncAl \mleft(\NHoD{\utilde} + \NHlphGI{\dn \utilde} + \NHlD{\ftilde}\mright).
\eeq
Combining \cref{eq:shiftint,eq:shiftscat,eq:shifttrunc}, we obtain \cref{eq:shift}.
\epf




%% The following trace theorem is standard, see, e.g., \cite[Theorem 3.37]{Mc:00}.


The following \lcnamecref{lem:domainshift} follows from \cref{thm:shift}.

\bco\label{lem:domainshift}
Under \cref{ass:highp}, let $\ftilde \in \HlD$ and $\gtilde \in \HlpoD,$ for $0 \leq l \leq p-1$. If $\utilde \in \HlptD$ solves
\beqs
\grad \cdot \mleft(A\grad \utilde\mright) = -\ftilde,
\eeqs
\beqs \trGD u = 0,
\eeqs
and
\beqs
\dn u = \trGI \gtilde,
\eeqs
then
\beqs
\NHlptD{\utilde} \leq \CAl\mleft(1+\CTrlpo\mright)\mleft(\NHlD{\ftilde} + \NHlpoD{\trGI \gtilde}\mright).
\eeqs
\eco

The proof of \cref{lem:domainshift} requires the Trace theorem.

\bth[Trace Theorem]\label{thm:trace}
If $v \in \HmD,$ for $1/2 < m \leq p+1$, then there exists $\CTrm > 0$ independent of $v$ such that
\beqs
\NHmmhGI{\trI v} \leq \CTrm \NHmD{v}.
\eeqs
\enth

For a proof of \cref{thm:trace}, see \cite[Theorem 3.37]{Mc:00}

\bpf[Proof of \cref{lem:domainshift}]
By \cref{thm:shift,thm:trace}
\beqs
\NHlptD{\utilde} \leq \CAl \mleft(\NHlD{\ftilde} + \NHlphGI{\gtilde}\mright) \leq \CAl \mleft(\NHlD{\ftilde} + \CTrfn{l}\NHlpoD{\gtilde}\mright),
\eeqs
and the result follows.
\epf

We are now able to prove the following \lcnamecref{thm:expansion} that gives a decomposition of the solution $u$ of \cref{prob:vtedp} into lower-order, less oscillatory parts $\uj$ and a smoother, more-oscillatory part $\uosc$. The main advantage of this \lcnamecref{thm:expansion} is that it enables us to prove a higher-order best-approximation bound (\cref{lem:bestapprox} below) for solutions of the Helmholtz equation, \emph{even though} the solutions do not have high regularity.

\bth[Expansion of the solution of the Helmholtz equation]\label{thm:expansion}
Under \cref{ass:highp,ass:htwo} if $u$ is the solution of \cref{prob:tedp} or its adjoint then there exists $\uosc \in \HppoD$ and a sequence $\usj \in \HjptD,$ $j = 0,\ldots,p-2$ such that
\beq\label{eq:expansionid}
u = \uosc + \sum_{j=0}^{p-2} \usj.
\eeq
Furthermore,
\beq\label{eq:expansionuj}
\NHjptD{\usj} \leq \Cej \Pj(\NLiD{n}) k^j \Cfg,
\eeq
and
\beq\label{eq:expansionuosc}
\NHppoD{\uosc} \leq \Cosc \CAnk k^p \Cfg,
\eeq
where
\beq\label{eq:p}
\Pj(x) =
\begin{dcases}
1 & j = 0,1\\
x^{\floor{j/2}}& j \geq 2
\end{dcases}
\eeq.
%=\sum_{m=0}^{j-2}\pfn{j,m}x^m $
%% are polynomials of degree
%% \beq\label{eq:polydegree}
%% \begin{dcases}
%% \frac{j}2 & \tif  j \text{ is even}\\
%% \frac{j-1}2 \tif j  \text{ is odd}.
%% \end{dcases}
%% \eeq
%% (except for $j=0,1,$ where $\Pj$ is a polynomial of degree 0) given by the recurrence relation
%% \beq\label{eq:pjdef}
%% \Pj(x) = \CAj \mleft(1+\CTrjpo\mright)\mleft(x\Pfn{j-2}(x) + \Pfn{j-1}(x)\mright).
%% \eeq
% No idea if this coefficient stuf is right
%% \begin{align}
%% \label{eq:p1}\pfn{0,0}&= \CAz\\
%% \label{eq:p2}\pfn{1,0}&  = \CAz\CAo\mleft(1+\CTrt\mright)\\
%% \label{eq:p3}\pfn{j,m} &=
%% \begin{dcases}
%%  0, &\tif m \leq \floor{\frac{j-2}2}\\
%% \CAj\mleft(1+\CTrjpo\mright) \mleft(\pfn{j-2,m-1} + \pfn{j-1,m-1}\mright),&\tif \floor{\frac{j-2}2} < m \leq j-3
%% \end{dcases}\\
%% \label{eq:p4}\pfn{j,j-2}& = \CAj\mleft(1+\CTrjpo\mright) \pfn{j-1,j-3},
%% \end{align}
%% and $\Cosc$ is given by the recurrence relations
%% \begin{align}
%% \label{eq:osc1}\Coscfn{1} &= \CAo\mleft(1+\CTrt\mright),\\
%% \label{eq:osc2}\Coscfn{2} &= \CAt\mleft(1+\CTrth\mright)\mleft(1 + \CAo\mleft(1+\CTrt\mright)\mright)\tand\\
%% \label{eq:osc3}\Coscfn{j} &= \CAfn{j}\mleft(1+\CTrfn{j+1}\mright)\mleft(\Coscfn{j-1} + \Coscfn{j-2}\mright),
%% \end{align}
%% and
%% \beq\label{eq:cosc}
%% \Cosc = \frac{\Cefn{p-1}}{\max\set{1,\CAz}}.
%% \eeq
\enth

\bre[How oscillatory are the functions in \cref{thm:expansion}?]\label{rem:osc}
Recall that a higher power of $k$ appearing in an a priori bound indicates that a function is more oscillatory. In \cref{eq:expansionuj} below, the $(j+2)$th-order norm of $\usj$ is of order $k^j$ (i.e., the power of $k$ is \emph{two} orders of magnitude less than the order of the norm) whereas in \cref{eq:expansionuosc} the $(p+1)$st-order norm of $\uosc$ is of order $k^p$ (the power of $k$ is \emph{one} order of mangitude less that the order of the norm). Therefore, in this sense, $\uosc$ is `more oscillatory' than $\usj.$
\ere

\Cref{thm:expansion} is essentially just \cite[Theorem 1]{ChNi:19} in the particular case of a Helmholtz problem, but with the dependence on all the constants kept track of. The results in \cite{ChNi:19} are stated for a wider class of time-harmonic wave propagation problems, but the dependence on all of the constants is not made explicit.

We also point out (as in \cite[Section 7]{ChNi:19}) that \cref{thm:expansion} is reminiscent of results by Melenk and Sauter in \cite{MeSa:10,MeSa:11}who show, for the homogeneous Helmholtz equation that the solution $u$ can be decomposed as $u = \uHt + \uA,$ where $\uHt \in \HtD$ but is not oscillatory ($\NHtD{\uHt} \lesssim 1$), and $\uA$ is analytic but $\uA$ is oscillatory ($\NHmD{\uA} \lesssim k^{m-2}$ for all $m$)\footnote{These results are proved with no obstacle and $f$ given by a Dirac delta function in \cite[Lemma 3.5]{MeSa:10} and for: (i) the IIP with a bounded Lipschitz boundary that is either a 2-d polygon or analytic, or (ii) the EDP with an analytic scatterer, in \cite[Theorems 4.10, 4.20]{MeSa:11} respectively. In the former case the proof is under an assumption of a polynomial growth of the a priori bound \cite[Assumption 4.8]{MeSa:11}, i.e., $\CAnk$ is a polynomial in $k$.}. This decomposition is used in \cite{MeSa:10,MeSa:11} to prove convergence results for $hp$-finite element methods for the Helmholtz equation. %We note that the results in this \lcnamecref{sec:fem} could, in principle, be used in the analysis of $hp$-methods, as it is, in principle, possible to track the dependence of the constants on the polynomial degree $p.$

\bpf[Proof of \cref{thm:expansion}]
The idea of the proof is as follows. We write $u$ as a formal series expansion
\beq\label{eq:formalseries}
u = \sum_{j=0}^\infty \usj,
\eeq
and then substitute this series into the PDE \cref{eq:tedp} and the boundary condition \eqref{eq:ibc}. Equating powers of $k$, we derive a recursive sequence of stationary diffusion equations for the functions $\usj,$ with right-hand sides dependent on $\ujmo$ and $\ujmt$. We use this recursive sequence and \cref{lem:domainshift} to prove the a priori bounds \cref{eq:expansionuj}.

We then define the $l$th remainder $\rl = u - \sum_{j=0}^{l-1} \usj,$ and by applying the operator $\grad\cdot\mleft(A\grad \cdot\mright)$ with Neumann boundary conditions to $\rl$, we obtain a recursive sequence for the remainders $\rl,$ and can similarly prove a priori bounds for the functions $\rl$. The oscillatory function $\uosc$ is then just $\rpmo.$ The format of this proof is identical to that in \cite[Theorem 1]{ChNi:19}, except we keep track of all of the constants involved.

For the purposes of the proof, it is more convenient to define $\vj = \usj/k^j,$ so that the series expansion \cref{eq:formalseries} becomes\footnote{In \cite{ChNi:19} the notation is changed slighty, and the series expansion is defined as $u = \sum_{j=0}^\infty k^j \uj,$ i.e., the functions $\vj$ in our proof are denoted $\uj$ in \cite{ChNi:19}.}
\beq\label{eq:formalseriesv}
u = \sum_{j=0}^\infty k^j\vj
\eeq
as in \cite{ChNi:19}. Also, in this proof, all the boundary-value problems involve included a zero Dirichlet condition on the scatterer $\GD;$ we omit this boundary condition throughout the proof for brevity.

By applying the Helmholtz operator to the formal series \eqref{eq:formalseriesv} we obtain the following equations for $\vj \in \HjptD, j \geq 1$:
\beqs
\grad \cdot \mleft(A\grad \vz\mright) = -f \quad\tand\quad \dn \vz = \gI,
\eeqs
\beqs
\grad \cdot \mleft(A\grad \vo\mright) = 0\quad\tand\quad\dn \vo = i\vz,
\eeqs
and
\beq\label{eq:vj}
\grad \cdot \mleft(A\grad \vj\mright) = - n\vjmt\quad\tand\quad\dn \vz = i\vjmo \quad\tfor j \in \mleft[2,p-2\mright].
\eeq

By \cref{thm:shift} we immediately conclude the bound
\beq\label{eq:expuz}
\NHtD{\vz} \leq \CAz\Cfg \leq \Cefn{0}\Cfg,
\eeq
i.e., \cref{eq:expansionuj} for $j=0.$ By \cref{lem:domainshift,eq:expuz} we can conclude the bound
\beqs
\NHthD{\vo} \leq \CAo \mleft(1+\CTrt\mright)\NHtD{i\vz} \leq \max\set{1,\CAo}\mleft(1+\CTrfn{2}\mright)\Cefn{0} \Cfg = \Cefn{1}\Cfg,
\eeqs
i.e., \cref{eq:expansionuj} for $j=1$.

We prove the bound \cref{eq:expansionuj} for higher $j$ by induction. Let $j \in \mleft[2,p-2\mright]$ and suppose \cref{eq:expansionuj} holds for all $s \in [0,j-1].$ Using \cref{lem:domainshift}, we conclude that (using the observation that $\Cefn{j-1} = 2\max\set{1,\CAjmo}\mleft(1+\CTrj\mright)\Cefn{j-2}\geq \Cefn{j-2}$)
\begin{align*}
\NHfn{j+2}{D}{\vj} &\leq \CAj \mleft(1+\CTrjpo\mright)\mleft(\NLiD{n} \NHfn{j}{D}{\vfn{j-2}} + \NHfn{j+1}{D}{\vfn{j-1}}\mright)\\
&\leq \max\set{1,\CAj} \mleft(1+\CTrjpo\mright)\mleft(\NLiD{n} \Cefn{j-2} \Pfn{j-2}\mleft(\NLiD{n}\mright) + \Cefn{j-1} \Pfn{j-1}\mleft(\NLiD{n}\mright)\mright)\Cfg\\
&\leq 2\max\set{1,\CAj} \mleft(1+\CTrjpo\mright)\Cefn{j-1} \Pfn{j}(\NLiD{n})\Cfg\\
&= \Cefn{j} \Pfn{j}(\NLiD{n})\Cfg.
\end{align*}
%% Therefore the definition of the polynomials $\Pj$ \cref{eq:pjdef} holds (and it is straightforward to see that $\deg\mleft(\Pj\mright) = \deg\mleft(\Pfn{j-2}\mright)+1$, where $\deg$ denotes polynomial degree, and therefore since $\deg\mleft(\Pfn{0}\mright) = \deg\mleft(\Pfn{1}\mright)=0,$ \cref{eq:polydegree} holds. Hence, by the relationship between $\usj$ and $\vj$, we have the bound \cref{eq:expansionuj}.

We will now define the remainders $\rl$, and proceed similarly.
Let $\ro \in \HthD$ solve
\beqs
\grad \cdot \mleft(A\grad \ro\mright) = -k^2 u \tand \dn \ro = ik\trGI u.
\eeqs
Then by \cref{lem:domainshift}
\beqs%gin{align*}
\NHthD{\ro} \leq \CAo\mleft(1+\CTrt\mright)\mleft(k^2\NHoD{u} + k\NHtD{u}\mright)\leq \CAo\mleft(1+\CTrt\mright)k^2\CAnk\Cfg\leq \frac{\Cefn{1}}{\max\set{1,\CAz}}k^2\CAnk \Cfg
\eeqs%nd{align*}
by definition of $\Cefn{1}$.
Let $\rt \in \HfD$ solve
\beqs
\grad \cdot \mleft(A\grad \rt\mright) = -k^2 u \tand \dn \rt = ik\trGI \ro.
\eeqs
Then by \cref{lem:domainshift}
\begin{align*}
\NHfD{\rt} &\leq \CAt\mleft(1+\CTrth\mright)\mleft(k^2\NHtD{u} + k\NHthD{\ro}\mright)\\
&\leq \CAt\mleft(1+\CTrth\mright)\mleft(1 + \frac{\Cefn{1}}{\max\set{1,\CAz}}\mright)\CAnk k^3\Cfg\\
&\leq 2\CAt\mleft(1+\CTrth\mright)\frac{\Cefn{1}}{\max\set{1,\CAz}}\CAnk k^3 \Cfg\text{ as } \Cefn{1} \geq 1,\\
&\leq \frac{\Cefn{2}}{\max\set{1,\CAz}}\CAnk k^3\Cfg.
\end{align*}
Then for $j \geq 3,$ let $\rj \in \HjptD$ solve
\beqs
\grad \cdot \mleft(A\grad \rt\mright) = -k^2 \rjmt \tand \dn \rj = ik\trGI\rjmo.
\eeqs
And by induction and \cref{lem:domainshift} again, letting $\uosc = \rfn{p-1},$ we have \cref{eq:expansionuosc}. It is straightforward to see that $\rfn{p-1} + \sum_{j=1}^{p-2} \uj$ solves \cref{prob:tedp}, and therefore \cref{eq:expansionid} holds, as $u$ is unique.
\epf
%\{Just a note while I think about it - could you do DtN boundary conditions in this proof by testing with a different function when wanting to bound the $L^2$ norm on the boundary? I wonder if testing with the NtD of $u$ would mean you end up with a $\LtGI{u}^2$ term, and then you could use properties of the NtD map to bound other bits. Might foil $k$-dependency though.}
Using the expansion in \cref{thm:expansion}, we can prove the following error bound for the best approximation of $u$ in $\Vhp$:

%% We also have the following best-approximation error in higer-order Sobolev spaces:
%% \ble[Best approximation in $\HmD$]\label{lem:bestapproxhigh}
%% For integer $s$ in $[1,p+1]$, there exists $\Cinterps>0$ such that for every $v \in \HmD$ there exists $\vhhat \in \Vhp$ such that
%% \beqs
%% \NLtD{v - \vhhat} + h\NHoD{v-\vhhat} \leq \Cinterps h^s \SNHmD{v}.
%% \eeqs
%% \ele

\ble[Best approximation error bound]\label{lem:bestapprox}
Under \cref{ass:highp,ass:htwo}, there exist constants $\CFEMo, \CFEMt > 0$ independent of $k$ and $n$ (although dependent on $A$ and $p$) such that if $u$ solves \cref{prob:vtedp} or its adjoint, then there exists $\uhhat \in \Vhp$ such that
\beq\label{eq:bestapproxL2}
\NLtD{u-\uhhat} \leq \Pfn{p-2}\mleft(\NLiD{n}\mright)\mleft(\CFEMo  h^2 + \CFEMt \CAnk h\mleft(hk\mright)^p \mright)\Cfg,
\eeq
%% \beq\label{eq:bestapproxH1}
%% \NHoD{u-\uhhat} \leq \mleft(\CFEMo h + \CFEMt \CAnk \mleft(hk\mright)^p \mright)\Cfg \tand
%% \eeq
\beq\label{eq:bestapproxW}
\NHokD{u-\uhhat} \leq 2\Pfn{p-2}\mleft(\NLiD{n}\mright)\mleft(\CFEMo  h + \CFEMt \CAnk \mleft(hk\mright)^p \mright)\Cfg.
\eeq
\ele

%We write bounds for the standard and weighted $H^1$ norms separately, as we will use each individual bound in differents of proofs in \cref{sec:fembound}.

\bpf[Proof of \cref{lem:bestapprox}]
We apply \cref{lem:scottzhang} to all the $\usj$ and $\uosc$ in \cref{thm:expansion}, and obtain that there exist $\ujh \in \Vhp$ $j=0,\ldots,p-2$ and $\uosch \in\Vhp$ such that 
\beqs
\NLtD{\usj - \ujh} + h\NHoD{\usj - \ujh} \leq \Cinterpfn{j+2} \Cefn{j} \Pj\mleft(\NLiD{n}\mright) h^{j+2}k^j \Cfg
\eeqs
and
\beqs
\NLtD{\uosc - \uosch} + h\NHoD{\uosc - \uosch} \leq \Cinterpfn{p+1} \Cosc\CAnk h^{p+1}k^p \Cfg.
\eeqs
Therefore, by letting $\uhhat = \uosch + \sum_{j=0}^{p-2} \ujh,$ we have \cref{eq:bestapproxL2,eq:bestapproxW} (using the facts that $hk \leq 1$ and $\Pfn{p-2}(\NLiD{n}) \geq \Pfn{j}(\NLiD{n}) \geq 1$ for all $j \leq p-2$ since $\NLiD{n} \geq 1$).
\epf

\subsection{Routine analysis results}\label{sec:anbackground}
In this \lcnamecref{sec:anbackground} we collect together routine results that we use throughout the following proofs.

The following inequalities are straightforward to prove
\bit
\item For $N \in \NN$ and $a_1,\ldots,a_N > 0$
\beq\label{eq:simple}
\sqrt{\sum_{j=1}^N a_j^2} \leq \sum_{j=1}^N a_j.
\eeq
\item (Young's inequality) If $p,q \in (1,\infty)$ and $1/p + 1/q = 1$ then for all $a,b > 0$
  \beq\label{eq:young}
ab \leq \frac{a^p}p + \frac{b^q}q.
\eeq
\item (Cauchy's inequality) For all $\eps,a,b > 0$
  \beq\label{eq:cauchy}
  ab \leq \frac{a^2}{2\eps} + \frac{\eps b^2}2.
\eeq
  \eit
  




\bth[Multiplicative Trace Inequality]\label{thm:multiplicativetrace}%BS THm 1.6.6
There exists $\CMT > 0$ such that for all $v \in \HoD$
\beqs
\NLtdD{v} \leq \CMT \NLtD{v}^\half \NHoD{v}^\half.
\eeqs
\enth

\ble[Poincar\'{e}--Friedrichs Inequality]\label{lem:poincare}
Let $\Gamma \subseteq D$ have nonvanishing $d-1$-dimensional measure. There exist constants $\CP,\Ctilde > 0$ depending only on $D$ and $\Gamma$ such that for all $v \in \HoD$

\beqs
\NLtD{v}^2 \leq \CP \SNHoD{v}^2 + \Ctilde \NLtGI{v}^2.
\eeqs

In particular, taking $\Gamma = \GD,$ for all $v \in \HozDD$
\beq\label{eq:poincare}
\NLtD{v} \leq \CP \SNHoD{v}.
\eeq
\ele
For a proof of \cref{lem:poincare} see \cite[Lemma A.14]{ToWi:05}.

\bth[Multiplication in $\HmD$]\label{thm:banachalg}
\ben
\item\label[itempart]{it:ban1}If $m > d/2,$ then for all $\vo, \vt \in \HmD$, $\vo\vt \in \HmD$ and there exists a constant $\CBanfn{m} > 0$ independent of $\vo$ and $\vt$ such that
\beqs
\NHmD{\vo\vt} \leq \CBanfn{m} \NHmD{\vo}\NHmD{\vt}.
\eeqs
\item\label[itempart]{it:ban2}For any $m \in \NN,$ if $\vo \in \CmD$, $\supp\mleft(1-\vo\mright) \compcont D,$ and  $\vt \in \HmD,$ then
  \bit
\item$\vo \in \WmiD,$
\item$\vo\vt \in \HmD$, and
  \item there exists a constant $\Cprod{m} > 0$ independent of $\vo$ and $\vt$ such that%If $\vo \in \CrcompD$ for some $r \in \NN, r \geq m,$ and
\beq\label{eq:ban2}
\NHmD{\vo\vt} \leq \Cprod{m} \mleft(1+\NWmiD{\vo}\mright)\NHmD{\vt}.
\eeq
\eit
\een
\enth

\bpf[Proof of \cref{thm:banachalg}]
\Cref{it:ban1} is given in \cite[Section 1.8.1, Theorem]{Ma:11}. For \cref{it:ban2}, observe that as $1-\vo$ has compact support, $\vo \in \WmiD,$ and $1-\vo \in \CmcompD$. Therefore, by \cite[Theorem 3.20]{Mc:00}, there exists $\Cmclean{m} > 0$ such that\footnote{In \cite[Theorem 3.20]{Mc:00} $\Cmclean{m}$ is denoted $C_{m}$.}
\beqs
\NHmD{(1-\vo)\vt} \leq \Cmclean{m} \NWmiD{1-\vo}\NHmD{\vt}.
\eeqs
Therefore as $\NHmD{\vo\vt} \leq \NHmD{\mleft(\vo-1\mright)\vt} + \NHmD{\vt} = \NHmD{(1-\vo)\vt} + \NHmD{\vt}$, we have \cref{eq:ban2}.
\epf



\subsection{Error bounds for Galerkin projections}\label{sec:errgalerkin}
In this \lcnamecref{sec:errgalerkin} we state a sequence of error bounds in negative Sobolev norms for two different projection operators. The proofs of these error bounds are all simple modifications of the standard duality-argument proofs of finite-element errors in negative Sobolev norms, as in, e.g., \cite[Theorem 5.8.3]{BrSc:08}.
%\ednote{Both-should I write out at least one of these modified proofs?}. % Note to self, I have these proofs sketched out, but just use the argument from Brenner and Scott and use an L^2 inner product on the function, not their gradients, for the elliptic projection.
We first define the projections we use.

Define the elliptic projection $\Ph:\HozDD\rightarrow\Vhp$ by, for $w \in \HozDD$
\beq\label{eq:epdef}
\aep(\Ph w,\vh) = \aep(w,\vh) \tforall \vh \in \Vhp,
  \eeq
  where
  \beq\label{eq:aepdef}
  \aep(\vo,\vt) = \IPLtD{\grad \vo}{\grad \vt}.
  \eeq
%% and define the weighted elliptic projection $\Phn:\HozDD\rightarrow\Vhp$ by, for $w \in \HozDD$
%% \beqs
%% \IPLtD{A\grad\vh}{\grad\Phn w} = \IPLtDn{A\grad\vh}{\grad w} \tforall \vh \in \Vhp.
%% \eeqs
  Observe that $\Ph w$ is the finite-element approximation of the solution of the stationary diffusion problem with `diffusion coefficient' $A$ and right-hand side in $\HozDD'$ given by $\aep(w,\cdot).$
  
%% We define the $\LtD$-projection $\Qh:\HozDD\rightarrow \Vhp$ by, for $w \in \HozDD$
%% \beqs
%% \IPLtD{\Qh w}{\vh} = \IPLtD{w}{\vh} \tforall \vh \in \Vhp.
%% \eeqs
Define the $\LtD$ projection in the $n$-weighted norm $\Qhn:\HozDD\rightarrow \Vhp$ by, for $w \in \HozDD$
\beqs
\IPLtDn{\Qhn w}{\vh} = \IPLtDn{w}{\vh} \tforall \vh \in \Vhp,
\eeqs

where, for $v,w \in \LtD,$ $\IPLtDn{v}{w}$ is the $n$-weighted inner product
\beqs
\IPLtDn{v}{w} \de \int_{D} n v \wbar.
\eeqs

We also define the corresponding $n$-weighted $\LtD$ norm $\NLtDn{v} = \sqrt{\int_{D} n \abs{v}^2}$ and for $m \in \NN$, the $n$-weighted $\HmD$ norms
\beqs
\NHmDn{v}^2 \de \sum_{\alpha \st \abs{\alpha} \leq s}\NLtDn{D^\alpha v},
\eeqs
and the negative $n$-weighted Sobolev norms
\beq\label{eq:negweightnorm}
\NHnfn{-m}{D}{v} \de \sup_{w \in \HmD} \frac{\IPLtDn{v}{w}}{\NHmDn{w}}.
\eeq

Observe that, for $v \in \HmD,$
\beq\label{eq:nconv}
\nmin\NHmD{v} \leq \NHmDn{v} \leq \NLiD{n} \NHmD{v}.
\eeq
%We put the \emph{non-weighted} $H^s$-norm in the denominator of \cref{eq:negweightnorm}, but the \emph{weighted} inner product in the numerator for ease of manipulation in some of the following proofs---we could place the weighted $H^s$ norm in the denominator, and this would be equivalent to the current definition, up to a factor involving $n$.
%% so that we have, for all $v \in \LtD$
%% \beq\label{eq:neqweightid}
%% \NHmsD{n v} = \NHmsDn{v}.
%% \eeq
%% \beq\label{eq:negativeweightbounds}
%% \frac{\NHmsD{v}}{\NLiD{n}} \leq \NHmsDn{v} \leq \frac{NHmsD{v}}{\nmin}.
%% \eeq

The proofs of the error bounds for the Galerkin projections will use the following notation for the solution operator of a particular stationary diffusion problem. We let $\DeltaAI:\LtD\rightarrow\HtD$ denote the solution operator for the following stationary diffusion equation: given $\ftilde \in \LtD$ find $\utilde \in \HtD$ such that
\beq\label{eq:sdeq}
\grad \cdot \mleft(A\grad \utilde\mright) = -n\ftilde \text{ in } D
\eeq
\beq\label{eq:sddbc}
\trD \utilde = 0
\eeq
\beq\label{eq:sdnbc}
\dn \utilde = 0.
\eeq
%% $A$-weighted Laplacian; that is $\DeltaA w = \grad\cdot\mleft(\grad w\mright),$ and give it domain $\DomainDeltaA = \set{w \in \HtD \st \trD w = 0, \dn w = 0};$
%% hence $\DeltaA:\DomainDeltaA \rightarrow \LtD.$ Observe that, for any $f \in \LtD$ there exists $\wf \in \DomainDeltaA$ such that $\DeltaA \wf = -f.$
For the reason there is a factor $n$ on the right-hand side of \cref{eq:sdeq}, see \cref{rem:why} below. Observe that $\sdsol$ is well-defined by \cref{thm:shift} as $n \ftilde \in \LtD$. Also, observe that $\sdsol^{m}$ is well-defined for any $m \in \NN,$ as $\HtD \subseteq \LtD,$ and so one can place $\sdsol \ftilde$ on the right-hand side. 
For any $\ftilde \in \LtD$ and for any $v \in \HozDD,$ we have, by Green's identity,
\beqs
\int_D \mleft(A \grad \mleft(\DeltaAI\ftilde\mright)\mright)\cdot \grad \vb = \int_D n\ftilde \vb,
\eeqs
i.e.,
\beq\label{eq:deltaagreen}
\IPLtD{A\grad \mleft(\sdsol \ftilde\mright)}{\grad v} = \IPLtDn{\ftilde}{v}.
\eeq




We now state and prove error bounds for the two projections given above. As stated above, the proofs below are all modifications of the standard proof (in, e.g., \cite[Theorem 5.8.3]{BrSc:08}). We can, in essence, use the standard proof because all the projections defined above are Galerkin projections given by coercive and bounded sesquilinear forms on $\HoD$ (for $\Ph$) or $\LtD$ (for $\Qhn$).

\ble[Existence and uniqueness of Galerkin projections]\label{lem:eugp}
For any $w \in \HozDD$ the elliptic projection $\Ph w$ and the $n$-weighted $L^2$ projection $\Qhn w$ exist and are unique.
\ele

\bpf[Proof of \cref{lem:eugp}]
The existence and uniqueness of $\Ph w$ and $\Qhn w$ follows from the Lax--Milgram Theorem (see, e.g., \cite[Theorem 2.7.7]{BrSc:08}) applied in $\Vhp$ (as in, e.g., \cite[Corollary 2.7.13]{BrSc:08}), because $\Ph w$ and $\Qhn w$ are defined by sesquilinear forms that are continuous and coercive on $\HozDD$ (equipped with the $k$-weighted $H^1$ norm $\NHokD{\cdot}$) and $\LtD$ respectively.

Continuity and coercivity are immediate in the case of the $n$-weighted $L^2$ projection $\Qhn.$ For the elliptic projection $\Ph,$ continuity is immediate, and coercivity follows from the Poincar\'e inequality \cref{eq:poincare}, because we work in $\HozDD,$ and so all the functions we consider vanish on $\GD.$
\epf

\ble[Error bounds for elliptic projection]\label{lem:ellprojerr}
Under \cref{ass:highp}, for any integer $m \in [-1,p-1],$ there exists a constant $\Cmmo >0$ such that for all $w \in \HozDD$
\beq\label{eq:ellprojerr}
\NHmmD{w-\Ph w} \leq \Cmmo h^{s+1} \BAHoD{w}{\wh}.
\eeq
\ele

\ble[Error bounds for elliptic projection in $n$-weighted norms]\label{lem:ellprojerrw}
Under \cref{ass:highp}, for any integer $m \in [-1,p-1],$ there exists a constant $\Cwmm >0$ such that for all $w \in \HozDD$
\beqs
\NHnfn{-m}{D}{w-\Ph w} \leq \Cwmm \errn{m} h^{m+1} \BAHoDn{w}{\wh},
\eeqs
where
\beq\label{eq:errn}
\errn{m} =
\begin{dcases}
\frac{\NHmD{n}}{\nmin^2}\nvar &\tif m \in \mleft(\frac{d}2,p-1\mright]\\
\frac{\NWmiD{n}}{\nmin^2}\nvar &\tif m \in \mleft[1,\frac{d}2\mright]\\
\nvar &\tif m = -1,0.
\end{dcases}
\eeq
%% \ben
%% \item\label[itempart]{it:wep1} If $s \in (d/2,p-1],$ then
%% \beq\label{eq:wep1}%\label{eq:ellprojerr}
%% \NHnfn{-s}{D}{w-\Ph w} \leq \Cwmso \frac{\NHmD{n}}{\nmin^2}\nvar h^{s+1} \BAHoDn{w}{\wh}.
%% \eeq
%% \item\label[itempart]{it:wep2} If $s \in [2,d/2]$ then
%% \beq\label{eq:wep2}
%% \NHnfn{-s}{D}{w-\Ph w} \leq \Cwmso \frac{\NWsiD{n}}{\nmin^2}\nvar h^{s+1} \BAHoDn{w}{\wh}.
%% \eeq
%% \item\label[itempart]{it:wep4} For $s = 0$
%% \beq\label{eq:wep4}
%% \NLtDn{w-\Ph w} \leq \Cwz \nvar h \BAHoDn{w}{\wh}.
%% \eeq
%% \item\label[itempart]{it:wep3}Also, we have
%% \beq\label{eq:wep3}
%% \NHnfn{1}{D}{w-\Ph w} \leq \Cwo\nvar \BAHoDn{w}{\wh}.
%% \eeq
%% \een
\ele
We define
\beq\label{eq:en}
\En\de\max_{m = -1,\ldots,p-1}\set{\errn{m}}.
\eeq
\bpf[Proof of \cref{lem:ellprojerrw}]
For the case $m \geq 1$, let $\phi \in \HmD,$ and observe that by \cref{ass:highp,thm:banachalg} $n\phi\in \HmD$. Let $ \vtilde = \sdsol(\phi)$ and observe $\vtilde \in \Hfn{}{m+2}{D}$ by \cref{thm:shift}. Observe that for all $v \in \HozDD,$ by definition of $\vtilde$ and the $n$-weighted inner product $\IPLtDn{\cdot}{\cdot}$, we have
\beq\label{eq:vtilde}
\IPLtD{A \grad v}{\grad \vtilde} = \IPLtD{nv}{\phi} = \IPLtDn{v}{\phi}
\eeq
(where we multiply the complex conjugate of \cref{eq:sdeq} by $v$, and integrate by parts).
Taking $v = w-\Ph w$ in \cref{eq:vtilde}, we have
\begin{align}
\IPLtDn{v}{\phi} &= \IPLtD{A\grad\mleft(w-\Ph w\mright)}{\grad\mleft(v - \Ih v\mright)} \text{by Galerkin orthogonality for } \Ph\nonumber\\
&\leq \CSZfn{m+2}\NLiDop{A} \NHfn{m+2}{D}{\vtilde} h^{m+1} \SNHoD{w-\Ph w} \text{ by \cref{lem:scottzhang}}\nonumber\\
&\leq \CSZfn{m+2} \CAfn{m} \NLiDop{A} \NHfn{m}{D}{n\phi} h^{m+1}\SNHoD{w-\Ph w}\text{ by \cref{thm:shift}.}\label{eq:ellprojwpart}
\end{align}

For $m > d/2$ by \cref{thm:banachalg,eq:nconv} \cref{eq:ellprojwpart} is bounded above by
\beq\label{eq:wep1pt0}
\CSZfn{m+2} \CAfn{m} \CBanfn{m} \NLiDop{A} \frac{\NHfn{m}{D}{n}}{\nmin}\NHnfn{m}{D}{\phi} h^{m+1}\SNHoD{w-\Ph w}.
\eeq


Therefore we have, by definition of $\NHnfn{-m}{D}{\cdot}$ and $\NHoDn{\cdot}$,
\beq\label{eq:wep1pt1}
\NHnfn{-m}{D}{w-\Ph w} \leq \CSZfn{m+2}\CAfn{m} \CBanfn{m} \NLiDop{A} \frac{\NHmD{n}}{\nmin^2} h^{m+1} \NHoDn{w - \Ph w}.
\eeq
Applying C\'ea's Lemma to $\Ph$ in the $n$-weighted $H^1$ norm, we find
\beq\label{eq:wepcea}
\NHoDn{w-\Ph w} \leq \frac{2\NLiDop{A}}{\min\set{1,1/\CP^2}\Amin}\nvar\BAHoDn{w}{\wh},
\eeq
as $\Ph$ corresponds to a sesquilinear form with continuity constant $\NLiDop{A}/\nmin$ and coercivity constant
\beqs
\mleft(\Amin\min\set{1,1/\CP}\mright)/\mleft(2\NLiD{n}\mright).
\eeqs
Combining \cref{eq:wep1pt1,eq:wepcea}, we obtain \cref{eq:errn} in the case $m > d/2$.

For $m \in [1,d/2],$ \cref{thm:banachalg} yields, instead of \cref{eq:wep1pt0},
\beq\label{eq:wep2pt1}
2\CSZfn{m+2} \CAfn{m} \Cprod{m} \NLiDop{A} \frac{\NWmiD{n}}{\nmin}\NHnfn{m}{D}{\phi} h^{m+1}\SNHoD{w-\Ph w},
\eeq
(since $1 + \NWmiD{n} \leq 2\NWmiD{n},$ as $\NLiD{n} \geq 1$) and by a similar reasoning to that before, we obtain \cref{eq:errn}, in the case $m \in [1,d/2]$.

For the case $m=0$, using \cref{lem:ellprojerr} and then converting to the $n$-weighted $L^2$ norm yields \cref{eq:errn}. For $m=-1$, \cref{eq:wepcea} immediately gives \cref{eq:errn}. 
\epf

\bre[Subtleties regarding $\Ph$ for the IIP]\label{rem:epdef}
If we consider the IIP (i.e., we remove the assumption $\GD \neq \emptyset$ in \cref{ass:highp}), then we can no longer prove \nameCrefs{lem:eugp} \ref{lem:eugp}, \ref{lem:ellprojerr}, or \ref{lem:ellprojerrw}, because the sesquilinear form $\aep$ is no longer coercive on $\HozDD$. (Note $\HozDD = \HoD$ in this case.)

This lack of coercivity stems from the fact that we cannot apply the Poincar\'e inequality \cref{eq:poincare} to functions in $\HoD,$ because such functions are no longer zero on a portion of $\d D$ with non-zero $d-1$-dimensional measure. (Recall \cref{lem:poincare} requires such a property.) One can alternatively view this problem as arising from the fact that the stationary diffusion equation with Neumann boundary conditions does not have a unique solution (as one can simply add a constant to any solution). In the case $\GD=\emptyset,$ the PDE corresponding to the elliptic projection is precisely such a stationary diffusion equation: \cref{eq:sd1,eq:sd3} (\cref{eq:sd2} does not apply, as $\GD = \emptyset$).

The remedy for this lack of coercivity/lack of uniqueness is to change the sesquilinear form $\aep$ from \cref{eq:aepho} to either \cref{eq:aeplower} or \cref{eq:aepused}. This change corresponds to changing the PDE underlying the elliptic projection from \cref{eq:sd1,eq:sd3} to either \cref{eq:sd4,eq:sd6} or \cref{eq:sd7,eq:sd9} respectively. Either of these alternatives for $\aep$ is coercive in the $k$-weighted $H^1$ norm, and so the proof of \cref{lem:eugp} goes through as before.

However, if one defines $\aep$ using \cref{eq:aepused}, then one encounters the additional problem that one does not have a shift theorem (analagous to \cref{thm:shift}) for the underlying PDE \cref{eq:sd7,eq:sd9}. Such a shift theorem is used in the proof of \cref{lem:ellprojerr,lem:ellprojerrw} to prove bounds in negative-order norms. Whilst one could rewrite \cref{eq:sd7,eq:sd9} as
\beqs
\Delta w = F  \tin D \tand
\eeqs
\beqs
\dn w = ikw \ton \GI
\eeqs
and then use \cref{thm:shift} to obtain results analagous to \cref{thm:shift}, the constants in the resulting bounds \cref{eq:shift} would be \emph{$k$-dependent}. This $k$-dependence would mean the resulting proof of $\hka{(2p+1)/(wp)}$-data-accuracy would not work. Whilst a shift theorem for \cref{eq:sd7,eq:sd9} that \emph{is} $k$-independent has been proved by Chaumont-Frelet, Nicaise, and Tomezyk \cite[Theorems 3.1, 4.3, and 5.1]{ChNiTo:18}, this is a lowest-order shift theorem (i.e., the analogue of \cref{thm:shift} for $l=0$) and no mention is made in \cite{ChNiTo:18} of an extension to higher order.

Therefore, if one wishes to prove results for higher-order finite elements, the only reasonable course of action when considering the IIP is to use \cref{eq:aeplower} to define $\aep$. In this case one can then repeat the proof of \cref{thm:shift} almost verbatim (as the results from \cite{Mc:00} used in the proof of \cref{thm:shift} also hold for the PDE \cref{eq:sd4,eq:sd6}), and the proofs of \cref{lem:ellprojerr,lem:ellprojerrw} proceed as before. However, observe that this course of action works fr our proof because our proof technique (error splitting) allows us to bound terms arising on the truncation boundary $\GI$ (see \cref{lem:boundarybound} below). However, if one uses a modified Schatz technique, it is not clear to us how one would bound terms arising on the truncation boundary. Therefore, these terms on the tunrcation boundary must be incorporated into the elliptic projection, and so one must use \cref{eq:aepused} to define the elliptic projection. One must then accept that, with current technology, one will be limited to proving bounds for first-order methods, due to the lack of a shift theorem for the underlying PDE, as outlined above. (We note in passing that it was precisely this usage of an elliptic projection given by \cref{eq:aepused} by Chaumont-Frelet and Nicaise in \cite{ChNi:18} that led to the work \cite{ChNiTo:18}.)
\ere

%% We also give the following error bounds for the weighted elliptic projection, and give the proof, even though it is conceptually similar to the proof of \cref{lem:ellprojw}.\{Check the original one is still needed.}

%% \ble[Error bounds for weighted elliptic projection in $n$-weighted norms]\label{lem:wellprojerrw}
%% Let $s \in \ZZ.$ There exists a constant $\Cwmso >0$ such that for all $w \in \HozDD$
%% \ben
%% \item\label[itempart]{it:wwep1} If $s \in (d/2,p-1],$ and $n \in \HmD$ then
%% \beq\label{eq:wwep1}%\label{eq:ellprojerr}
%% \NHnfn{-s}{D}{w-\Phn w} \leq \Cwmso \frac{\NHmD{n}\NLiD{n}^2}{\nmin^3}h^{s+1} \BAHoDn{w}{\wh}.
%% \eeq
%% \item\label[itempart]{it:wwep2} If $s \in [1,d/2]$ and $n \in \CsD$, then
%% \beq\label{eq:wwep2}
%% \NHnfn{-s}{D}{w-\Phn w} \leq \Cwmso \frac{\NWsiD{n}\NLiD{n}^2}{\nmin^3} h^{s+1} \BAHoDn{w}{\wh}.
%% \eeq
%% \item\label[itempart]{it:wwep4} For $s = 0$
%% \beq\label{eq:wwep4}
%% \NLtDn{w-\Phn w} \leq \Cwfn{0,1} \frac{\NLiD{n}^3}{\nmin^3}h \BAHoDn{w}{\wh}
%% \eeq
%% \item\label[itempart]{it:wwep3} For $s = 1$
%% \beq\label{eq:it:wwep3}
%% \NHnfn{1}{D}{w-\Phn w} \leq \Cwfn{1,1}\frac{\NLiD{n}^2}{\nmin^2} \BAHoDn{w}{\wh}.
%% \eeq
%% \een
%% \ele

%% \bpf[Proof of \cref{lem:wellprojerrw}]
%% For \cref{it:wep1,it:wep2}, the proof follows the proof in \cite[Theorem 5.8.3]{BrSc:08}. Let $\phi \in \HmD,$ and observe that by \cref{thm:banachalg} $n\phi\in \HmD$ also. Let $ \vtilde = \sdsol(\phi),$ observe that by \cref{thm:shift} $\vtilde \in \Hfn{}{s+2}{D}$. For all $v \in \HozDD,$ we have
%% \beqs
%% \IPLtD{A \grad \vtilde}{\grad v} = \IPLtD{n\phi}{v} = \IPLtDn{\phi}{v}.
%% \eeqs
%% If we take $v = w-\Phn w,$ then we can compute
%% \begin{align}
%% \IPLtDn{\phi}{v} &= \IPLtD{A\grad\mleft(\vtilde - \vh\mright)}{\grad\mleft(w-\Phn w\mright)} \text{ for } \vh \text{ as in \cref{lem:scottzhang}, by Galerkin orthogonality for } \Phn\nonumber\\
%% &\leq \CSZfn{s+2}\NLiDop{A} \NHfn{s+2}{D}{\vtilde} h^{s+1} \SNHoD{w-\Phn w} \text{ by \cref{lem:scottzhang}}\nonumber\\
%% &\leq \CSZfn{s+2} \CAfn{s} \NLiDop{A} \NHfn{s}{D}{n\phi} h^{s+1}\SNHoD{w-\Phn w}\text{ by \cref{thm:shift}}\label{eq:ellprojwpart}
%% \end{align}

%% For \cref{it:wep1}, $s > d/2,$ and so by \cref{thm:banachalg}, \cref{eq:ellprojwpart} is bounded above by
%% \beq\label{eq:wep1pt0}
%% \CSZfn{s+2} \CAfn{s} \CBanfn{s} \NLiDop{A} \frac{\NHfn{s}{D}{n}}{\nmin}\NHnfn{s}{D}{\phi} h^{s+1}\SNHoD{w-\Phn w}
%% \eeq
%% by \cref{thm:banachalg,eq:nconv}.

%% Therefore we have, by definition of $\NHnfn{-s}{D}{\cdot}$ and $\NHoDn{\cdot}$,
%% \beq\label{eq:wep1pt1}
%% \NHnfn{-s}{D}{w-\Phn w} \leq \CSZfn{s+2}\CAfn{s} \CBanfn{s} \NLiDop{A} \frac{\NHmD{n}}{\nmin^2} h^{s+1} \NHoDn{w - \Phn w}.
%% \eeq
%% Applying C\'ea's Lemma to $\Phn$ in the weighted $H^1$ norm, we find
%% \beq\label{eq:wepcea}
%% \NHoDn{w-\Phn w} \leq \frac{2\NLiDop{A}\NLiD{n}^2}{\min\set{1,1/\CP^2}\Amin\nmin^2}\BAHoDn{w}{\wh},
%% \eeq
%% and then combining \cref{eq:wep1pt1,eq:wepcea}, we obtain \cref{eq:wep1}.

%% For \cref{it:wep2}, the application of \cref{thm:banachalg} yields, instead of \cref{eq:wep1pt0},
%% \beq\label{eq:wep2pt1}
%% 2\CSZfn{s+2} \CAfn{s} \Cprod{s} \NLiDop{A} \frac{\NWsiD{n}}{\nmin}\NHnfn{s}{D}{\phi} h^{s+1}\SNHoD{w-\Phn w},
%% \eeq
%% since $1 + \NWsiD{n} \leq 2\NWsiD{n},$ as $\NLiD{n} \geq 1,$ and by a similar reasoning to that before, we obtain \cref{eq:wep2}.

%% For \cref{it:wep3}, \cref{eq:wepcea} immediately gives the result. For \cref{it:wep4}, performing a standard duality argument in \emph{non-weighted} norms (and then using \cref{eq:it:wep3}, yields \cref{eq:wep4}.
%% \epf


%% The $\LtD$ projection satisfies the following error bound.
%% \ble[Error bounds for $\LtD$ projection]\label{lem:ltdprojerr}
%% Under \cref{ass:highp}, for any integer $m \in [0,p-1],$ for all $w \in \LtD$
%% \beqs
%% \NHmmD{w-\Qh w} \leq \Cmmz h^{m} \BALtD{w}{\wh}.
%% \eeqs
%% \ele

\ble[Error bounds for $n$-weighted $\LtD$ projection]\label{lem:wltdprojerr}
Under \cref{ass:highp}, for any integer $m \in [0,p-1],$ for all $w \in \HozDD$
\beq\label{eq:wltdprojerr}
\NHmmDn{w-\Qhn w} \leq \CSZfn{m} \frac{\NLiD{n}}{\nmin} h^{m} \BALtDn{w}{\wh}.
\eeq
\ele

\bpf[Proof of \cref{lem:wltdprojerr}]
Fix $\vtilde \in \HmD$. Trivially $\vtilde$ solves
\beq\label{eq:trivial}
\IPLtDn{v}{\vtilde} = \IPLtDn{v}{\vtilde} \tforall v \in \LtD.
\eeq
Then letting $v=w-\Qhn w$ in \cref{eq:trivial} and using Galerkin orthogonality we have
\beqs%gin{align*}
\IPLtDn{w-\Qhn w}{\vtilde} \leq \NLtDn{w-\Qhn w}\NLtDn{\vtilde-\Ih\vtilde}\leq \CSZfn{m} \frac{\NLiD{n}}{\nmin}\NLtDn{w-\Qhn w} h^m \NHmDn{\vtilde}.
\eeqs%nd{align*}
Taking the supremum over $\vtilde,$ we have
\beqs
\NHmmDn{w-\Qhn w} \leq \CSZfn{m} \frac{\NLiD{n}}{\nmin} h^m \NLtDn{w-\Qhn w},
\eeqs
and hence by C\'ea's Lemma (as the inner product $\IPLtDn{\cdot}{\cdot}$ is clearly bounded and coercive (with continuity and coercivity constants equal to 1) in the $n$-weighted $L^2$-norm $\NLtDn{\cdot}$) the result follows.
\epf

\subsection{Discrete Sobolev spaces}\label{sec:discsob}
When we analyse the high-order finite-element method, we need to measure higher-order norms of functions in the finite-element space $\Vhp.$ However, as these functions do not have higher-order weak derivatives, we must first define a notion of higher-order discrete derivatives, and then develop some theory of so-called discrete Sobolev spaces. We follow the presentation in \cite{DuWu:15}, albeit working in the heterogeneous case, and with some changes of notation. The main result of this \lcnamecref{sec:discsob} is \cref{lem:negdiscsum} below giving the relationship between negative-order discrete Sobolev norms and negative-order continuous Sobolev norms.

\bde[Discrete derivative operator]
Define the \defn{$A$-weighted discrete second derivative operator} $\Deltah:\Vhp\rightarrow\Vhp$ by, for $\wh \in \Vhp$,
\beq\label{eq:discderdef}
\IPLtDn{\Deltah \wh}{\vh} = \IPLtD{A \grad \wh}{\grad \vh} \tforall \vh \in \Vhp.
\eeq
\ede

\ble[Discrete derivative operator is well-defined]\label{lem:ddwd}
For any $\wh \in \Vhp,$ $\Deltah \wh$ exists and is unique.
\ele

\bpf[Proof of \cref{lem:ddwd}]
Choose an orthonormal (in the $n$-weighted inner product) basis  $(\phij)_j$ for $\Vhp.$ Write $\Delta\wh = \sum_j \wj \phij$, and take in turn $\vh = \phij$ for each $j;$ then \cref{eq:discderdef} is equivalent to the linear system $\Imat \bw = \bb,$ where $\bb_{j} = \IPLtDn{A \grad \wh}{\grad \phij}.$ The solution of this linear system clearly exists and is unique.
%% We equip $\Vhp$ with the $H^1$-norm. Observe that $\Deltah \wh$ satisfies the variational problem: Find $\vhtilde  \in \Vhp$ such that $\add(\uh,\vh) = \Ldd(\vh)$ for all $\vh \in \Vhp,$ where $\add(\uh,\vh) = \IPLtD{\uh}{\vh}$ and $\Ldd(\vh) = \IPLtD{A \grad \wh}{\grad \vh}.$ Observe that $\Ldd$ is bounded in $\Vhp$, as $\Ldd(\vh) \leq \NLiDop{A} \SNHoD{\wh}\SNHoD{\vh} \leq \NLiDop{A} \SNHoD{\wh}\NHoD{\vh},$ and $\add$ is coercive on $\Vhp,$ as, for $\vh \in \Vhp$, $\add(\vh,\vh) = \NLtD{\vh}^2 \geq \CinvVhp^2 \NHoD{\vh}^2$ by the standard inverse estimate\opntodo{Add in?}. Therefore, by the Lax--Milgram Theorem applied in $\Vhp$ (as $\Vhp$ is a finite-dimensional inner product space over a complete field, it is a Hilbert space), $\Deltah \wh$ exists and is unique.
\epf

%% By\opntodo{McLean Thm 4.12 - double check and maybe write out in more detail---exactly what operator are we talking about, especially if we want weak derivatives?}, there exists a sequence of eigenfunctions $\phio,\phit,\ldots \in \HoD$ of $\DeltaA$\opntodo{What exactly does McLean mean here, if they don't have second-order derivatives?} and corresponding eigenvalues $0 < \lambdao<\lambdat < \cdots \rightarrow\infty$ such that the eigenfunctions form a complete orthonormal system in $\LtD.$\opntodo{Maybe define this}

Since $A$ is real and symmetric, it is self-adjoint. Hence it follows that $\Deltah$ is self-adjoint in the $n$-weighted inner product, as
\beqs
\IPLtDn{\Deltah \wh}{\vh} = \IPLtD{A \grad \wh}{\grad \vh} = \IPLtD{\grad \wh}{A\grad \vh} = \overline{\IPLtDn{\Deltah \vh}{\wh}} = \IPLtDn{\wh}{\Deltah \vh}.
\eeqs
Therefore $\Deltah$ is diagonalisable, i.e., there exists a set of eigenfunctions $\phioh,\ldots,\phidimVhph$  with corresponding real eigenvalues\ednote{Both - is it common enough knowledge that a symmetric matrix has real eigenvalues, that I don't need to reference this?} $\lambdaoh, \ldots, \lambdadimVhph$ such that the $\phimh$ form an orthonormal (in the $n$-weighted inner product) basis of $\Vhp$.

\bde[Higher-order discrete derivative operators]\label{def:hodd}
For $\vh \in \Vhp$, if $\vh = \sum_{m=1}^{\dimVhp} \am \phimh,$ then for $j \in \RR$ define
\beqs
\Deltah^j \vh = \sum_{m=1}^{\dimVhp} \lambdamh^j \am \phimh.
\eeqs


\ede
Observe that for $\wh \in \Vhp,$ $\DeltahI\mleft(\Deltah\wh\mright) = \wh,$ i.e., one can think of $\DeltahI$ as being a `discrete solution operator' for the stationary diffusion equation \cref{eq:sdeq,eq:sddbc,eq:sdnbc}. I.e., $\DeltahI$ is a discrete counterpart to $\sdsol.$
%% Similarly, for $v \in \LtD,$ if $v = \sum_{m=1}^\infty \am \phim,$ then for $j \in \RR$ define
%% \beq\label{eq:deltaaseries}
%% \DeltaA^j v = \sum_{m=1}^\infty \lambdam^j \am \phim,
%% \eeq
%% if this series exists in $\LtD.$
%We let $\Domain{\DeltaA^j}$ denote the subset of $\LtD$ on which $\DeltaA^j$ is defined.
%% \bre[Negative powers of $\DeltaA$]
%% Observe that for \emph{every} $j \leq 0$, $\DeltaA^j v$ is defined for \emph{any} $v \in \LtD$ (i.e., $\Domain{\DeltaA^{j}} = \LtD$ for $j \leq 0$). For $\lambdam \geq 1,$ $\lambdam^j < \lambdam$, and only finitely many $\lambdam$ are in the interval $(0,1)$; therefore the series \cref{eq:deltaaseries} can be decomposed as a finite sum (for $\lambdam < 1$) and a convergent series (for $\lambdam \geq 1$).
%% \ere
%% \bre[Consistent Notation]
%% Observe that the notation $\DeltaA^j$ is consistent, i.e., $\DeltaA^0 v = v$, for $j \in \NN,$ $\DeltaA^j$ is equal to the $j$-fold application of $\DeltaA,$ and $\DeltaA^{-1}$ is the inverse of $\DeltaA.$\opntodo{Maybe just double-check this is watertight.}\opntodo{This needs work - need $A$ to be smooth enough to define proper higher-order derivatives.}
%% \ere
We can use the higher-order derivative operators to define discrete higher-order norms:

\bde[Discrete higher-order norm]
For $\vh \in \Vhp$ and $m \in \RR$, define
\beqs
\Nmhn{\vh} = \NLtDn{\Deltah^{m/2} v}.
\eeqs
\ede

%% \bde[$A$-weighted higher-order continuous norm]
%% For $v \in \LtD$ and $m \in \RR$, if $\DeltaA^{m/2} v$ exists, define
%% \beqs
%% \NmA{v} = \NLtD{\DeltaA^{m/2} v}.
%% \eeqs
%% \ede

%% In order to prove a relationship between the $A$-weighted higher order norm and the standard $H^m$ norms, we first must prove the following \lcnamecref{lem:normrelationshiptech}\opntodo{THIS NEEDS PROVING AND I DON'T KNOW HOW.}
%% \ble[Relationship between $\DeltaA^m$ and standard higher-order derivatives]\label{lem:normrelationshiptech}
%% For $m \in \NN,$ $\Domain{\DeltaA^{m/2}} \subseteq \HmD,$ and there exists a constant $\Cma > 0$ such that for all $v \in \Domain{\DeltaA^{m}}$
%% \beqs
%% \NHmD{v} \leq \Cma \NLtD{\DeltaA^{m/2}v}.
%% \eeqs\opntodo{The latter bit will probably use something shift-theorem-like, but you need to be careful because powers of the differential operator aren't defined standardly.}
%% \ele


%% \ble[Relationship between $A$-weighted and standard higher-order continuous norms]\label{lem:normrelationship}
%% For all $m \in \NNz,$ for all $v \in \HmmD,$
%% \beqs
%% \NHmmD{v} \geq \frac1{\Cma} \NmmA{v}.
%% \eeqs
%% \ele

%% \bpf[Proof of \cref{lem:normrelationship}]
%% For $v \in \HmmD,$ we have
%% \begin{align*}
%% \NHmmD{v} &= \sup_{w \in \HmD} \frac{\IPLtD{v}{w}}{\NHmD{w}}\\
%% &\geq \frac1{\Cma} \sup_{w \in \Domain{\DeltaA^{m/2}}} \frac{\IPLtD{v}{w}}{\NmA{w}} \text{ by \cref{lem:normrelationshiptech}}\\
%% & = \frac1{\Cma} \sup_{w \in \Domain{\DeltaA^{m/2}}} \frac{\IPLtD{\DeltaA^{-m/2}v}{\DeltaA^{m/2}w}}{\NmA{w}} \text{ by \cref{lem:intoip}}\\
%% &= \frac1{\Cma} \NmmA{v} \text{ the supremum is achieved when } \IPLtD{\DeltaA^{-m/2}v,\DeltaA^{m/2}w} = \NLtD{\DeltaA^{-m/2}v}\NLtD{\DeltaA^{m/2}w}
%% \end{align*}
%% \epf

\bre[Related literature for higher-order discrete norms]
To our knowledge, \cite{DuWu:15} is the only place in the literature where the above construction of higher-order discrete norms appears, although Thom\'{e}e defines these norms for negative integers $m$ in \cite[Equation above Lemma 1]{Th:80}. However, the idea of using a self-adjoint, coercive operator to define a norm can be found in, e.g., \cite[Section 2.1]{LiMa:72}, and the authors also observe \cite[Text at the bottom of page 9]{LiMa:72} that one can use a spectral decomposition to define arbitrary powers of such operators (analagous to \cref{def:hodd}). See \cite[Section 2.1]{BaKuMa:19} for a simpler exposition of defining a (fractional-order) Sobolev norm via an operator.

We observe in passing that \cref{def:hodd} is analagous to the spectral definition of the fractional Laplacian $\mleft(-\Delta\mright)^m$; see the recent review article \cite[Section 2.5.1]{LiPaGuSoGlZhMaCaMeAiKa:18} for an overview of this idea.
\ere

We will use the following \lcnamecref{lem:intoip} to bound the inner product of two discrete functions by their negative- and positive-higher-order discrete norms, or to transfer discrete derivatives from one argument of the inner product to the other.

\ble[Introduction of derivatives into inner product]\label{lem:intoip}
%% For $m \in \RR,$ $v \in \LtD$, and $w \in \LtD\cap\Domain{\DeltaA^{m/2}}$ we have
%% \beqs
%% \IPLtD{\DeltaAmmt v}{\DeltaAmt w} = \IPLtD{v}{w}.
%% \eeqs
%% Similarly, f
For $\vh, \wh \in \Vhp,$ and $m \in \RR$ we have
\beq\label{eq:feipsplit}
\IPLtDn{\vh}{\wh} = \IPLtDn{\Deltahmmt \vh}{\Deltahmt \wh} 
\eeq
and
\beq\label{eq:feiptrans}
\IPLtDn{\Deltah^m \vh}{\vh} = \IPLtDn{\Deltahmt \vh}{\Deltahmt \vh}.
\eeq
\ele
\bpf[Proof of \cref{lem:intoip}]
We only prove \cref{eq:feipsplit}, as the proof of \cref{eq:feiptrans} is analagous. Since $\vh,\wh \in \Vhp,$ there exist sequences $(\aj)_{j =1,\ldots,\dimVhp}$ and $(\bsl)_{l =1,\ldots,\dimVhp}$ such that $\vh = \sum_{j=1}^{\dimVhp} \aj\phij$ and $\wh = \sum_{l=1}^{\dimVhp} \bsl \phil.$ Then we have
\begin{align*}
\IPLtDn{\Deltahmmt \vh}{\Deltahmt \wh} &= \int_D n\mleft(\sum_{j=1}^{\dimVhp}\lambdaj^{-m/2} \aj\phij\mright)\overline{\mleft(\sum_{l=1}^{\dimVhp} \lambdal^{m/2}\bsl \phil\mright)}\\
&= \sum_{j,l=1}^{\dimVhp} \lambdaj^{-m/2} \lambdal^{m/2} \aj \bsl \int_D n \phij \philbar\quad \text{ as the } \lambdaj \text{ are real}\\
& =\sum_{j}^{\dimVhp} \aj \bsj \int_D n\abs{\phij}^2\quad \text{ as the } \phij \text{ are orthonormal}\\
&= \IPLtDn{\vh}{\wh},
\end{align*}
where the final line follows by repeating the above process in reverse without the factors $\lambdaj^{-m/2}$ and $\lambdal^{m/2}$.
\epf
The next \lcnamecref{cor:ipdiscbound} follows from \cref{lem:intoip} and the Cauchy--Schwarz inequality.

\bco[Inner product bounded by discrete norms]\label{cor:ipdiscbound}
If $\vh, \wh \in \Vhp,$ then for all $j \in \RR$
\beqs
\IPLtDn{\vh}{\wh} \leq \Njh{\vh}\Nmjh{\wh}.
\eeqs
\eco

We recall the standard inverse inequality for finite-element functions, so that we can prove an analagous inverse inequality for discrete norms.

\ble[Standard inverse inequality]\label{lem:inverseinequality}
There exists $\CinvVhp > 0$ such that for all $\vh \in \Vhp$
\beqs
\NHoD{\vh} \leq \CinvVhp h^{-1} \NLtD{\vh}.
\eeqs
\ele



\ble[Inverse inequality for discrete norms]\label{lem:inversediscrete}
For all $m \in \RR$, for all $\vh \in \Vhp$
\beqs
\Nmhn{\vh} \leq \Chinv \frac{1}{\nmin} h^{-1} \Nmmohn{\vh}.
\eeqs
\ele

\bpf[Proof of \cref{lem:inversediscrete}]
We only prove the case $m=1,$ as the other cases will follow immediately. We have
\begin{align*}
\Nfn{1,h,n}{\vh}^2 &= \IPLtDn{\Deltahh \vh}{\Deltahh \vh}\\
&= \IPLtDn{\Deltah \vh}{\vh} \text{ by \cref{eq:feiptrans}}\\
&= \IPLtD{A \grad \vh}{\grad \vh} \text{ by definition of } \Deltah\\
&\leq \NLiDop{A} \CinvVhp h^{-2}\frac{1}{\nmin^2} \NLtDn{\vh}^2
\end{align*}
by the standard inverse estimate, and the result follows as $\Nzhn{\cdot} = \NLtDn{\cdot}$.

For $m \neq 1$ we have
\begin{align*}
  \Nmhn{\vh} &= \NLtDn{\Deltah^{\frac{m}2}\vh}\\
  &= \NLtDn{\Deltah^{\half}\mleft(\Deltah^{\frac{m-1}2}\vh\mright)}\\
  &= \Nohn{\Deltah^{\frac{m-1}2}\vh}\\
  &\leq \Chinv \frac{1}{\nmin} h^{-1} \Nzhn{\Deltah^{\frac{m-1}2}\vh}\\
\end{align*}
by the result for $m=1,$ and the result for $m\neq 1$ follows.
\epf


\ble[Relationship between standard and discrete $H^1$ norms]\label{lem:h1contdisc}
Let $\vh \in \Vhp$. Then
\beqs
\SNHoD{\vh} \leq \Amin^{-\half} \Nohn{\vh}.
\eeqs
\ele

\bpf[Proof of \cref{lem:h1contdisc}]
We have, using \cref{eq:feiptrans},
\beqs
\Nohn{\vh}^2 = \IPLtDn{\Deltahh \vh}{\Deltahh \vh} = \IPLtDn{\Deltah \vh}{\vh}= \IPLtD{A \grad \vh}{\grad \vh} \geq \Amin \NLtD{\grad \vh}^2,
\eeqs
and the result follows.
\epf

To prove \cref{lem:negdiscsum} below we require the following \lcnamecref{lem:shiftnegativew} giving the shift theorem in negative $n$-weighted norms.

\ble[Shift theorem in negative $n$-weighted norms]\label{lem:shiftnegativew}
Let $\ftilde \in \LtD$ a d $m \in [0,p-1]$ be an integer. Under \cref{ass:highp} we have
\beq\label{eq:shiftnegativew}
\NHnfn{-m}{D}{\sdsol\ftilde} \leq \Cshiftfn{m} \shiftn{m} \NHnfn{-(m+2)}{D}{\ftilde},
\eeq
where
\beqs
\shiftn{m} \de 
\begin{dcases}
 \NHmD{n}\nvar &\tif m > \frac{d}2\\
\NWmiD{n}\nvar  &\tif m \in \mleft[1,\frac{d}2\mright]\\
\NLiD{n}\nvar &\tif m = 0\\
\NLiD{n}^2 &\tif m = -1.
\end{dcases}
\eeqs
\ele
We define
\beqs
\Rn = \max_{m = -1,\ldots,p-1}\set{\shiftn{m}}.
\eeqs

\bpf[Proof of \cref{lem:shiftnegativew}]
We first observe that the operator $\sdsolo$ is self-adjoint on $\LtD$. Let $\DeltaA:\HtD\rightarrow\LtD$ denote the stationary diffusion operator $\grad\cdot\mleft(A\grad\cdot\mright)$ with no boundary conditions applied. Then for any $\vo \in \LtD,$
\beq\label{eq:scomp}
\DeltaA \circ \sdsolo \vo = \vo.
\eeq
Moreover, by Green's Theorem, $\DeltaA$ is self-adjoint on the set $\set{v \in \HtD \st v \text{ satisfies \cref{eq:sddbc,eq:sdnbc}}},$ this set contains the image of $\sdsolo.$ Therefore, for any $\vo,\vt \in \LtD,$ we have
\beq\label{eq:solosa}
\IPLtD{\sdsolo \vo}{\vt} =\IPLtD{\sdsolo \vo}{\DeltaA \circ \sdsolo \vt} = \IPLtD{\DeltaA \circ \sdsolo \vo}{\sdsolo \vt} = \IPLtD{\vo}{\sdsolo \vt},
\eeq
using \cref{eq:scomp} and the fact that $\DeltaA$ is self adjoint. I.e., \cref{eq:solosa} shows $\sdsolo$ is self-adjoint on $\LtD.$

Observe that by \cref{thm:banachalg}, if $v \in \HmD$ then $nv \in \HmD,$ and therefore by \cref{thm:shift} $\sdsolo(nv) \in \HmptD.$ With these facts in place we can compute
\begin{align*}
\NHnfn{-m}{D}{\sdsol \ftilde} &= \sup_{v \in \HmD} \frac{\IPLtD{\sdsol \ftilde}{nv}}{\NHmDn{v}}\\
&= \sup_{v \in \HmD} \frac{\IPLtD{\sdsolo\mleft(n \ftilde\mright)}{nv}}{\NHmDn{v}}\\
&= \sup_{v \in \HmD} \frac{\IPLtDn{\ftilde}{\sdsolo\mleft(n v\mright)}}{\NHmDn{v}}\quad\tas \sdsolo \text{ is self-adjoint}\\
&\leq \sup_{v \in \HmD} \frac{\NHnfn{-(m+2)}{D}{\ftilde}\NHnfn{m+2}{D}{\sdsolo\mleft(nv\mright)} }{\NHmDn{v}}\\
&\leq \sup_{v \in \HmD}  \frac{\CAfn{m} \NLiD{n} \NHmD{nv} \NHnfn{-(m+2)}{D}{\ftilde}}{\NHmDn{v}}
%% & = \sup_{v \in \HmD} \frac{\IPLtD{ n\ftilde}{\sdsol(v)}}{\NHmD{v}}\text{ by definition of } \sdsol, \text{ and as } \sdsolo \text{ is self-adjoint}\\
%% &= \sup_{v \in \HmD} \frac{\IPLtDn{\ftilde}{\frac{\sdsol(nv)}n}}{\NHmD{v}}\\
%% &\leq \CBanfn{m+2}\sup_{v \in \HmD} \frac{\NHfn{-(m+2)}{D}{\ftilde}\NHfn{m+2}{D}{\sdsol(nv)}\NHfn{m+2}{D}{\frac1n}}{\NHmD{v}} \text{ by \cref{thm:banachalg}}\\
%% &\leq \CBanfn{m+2}\CAfn{m}\sup_{v \in \HmD} \frac{\NHfn{-(m+2)}{D}{\ftilde}\NHfn{m}{D}{nv}\NHfn{m+2}{D}{\frac1n}}{\NHmD{v}} \text{ by \cref{thm:shift}}\\
\end{align*}
and by applying \cref{thm:banachalg} to the term $\NHfn{m}{D}{nv}$ (or observing that $\NLtD{nv} \leq \NLiD{n}\NLtD{v}$ in the case $m=0$), the result follows, except for $m=-1.$

For $m=-1,$ we have, by the Lax--Milgram Theorem in non-weighted norms,
\beqs
\NHoD{\Sn \ftilde} \leq \NHmoD{n\ftilde}/\Amin.
\eeqs
One can straightforwardly show that
\beqs
\NHnfn{1}{D}{\Sn \ftilde} \leq \NLiD{n} \NHoD{\Sn \ftilde}
\eeqs
and
\beqs
\NHmoD{n\ftilde} \leq \NLiD{n} \NHnfn{-1}{D}{\ftilde},
\eeqs
and so the result follows.
\epf

%% \ble[Shift theorem in negative norms]\label{lem:shiftnegative}
%% For $m \in \NN,$ and $\ftilde \in \LtD,$ we have
%% \beqs
%% \NHmmD{\DeltaAI\ftilde} \leq \CAm \NHmmmtD{\ftilde}.
%% \eeqs
%% \ele

%% \bpf[Proof of \cref{lem:shiftnegative}]
%% Throughout the proof we let $\utilde$ denote $\DeltaAI\ftilde.$ As $\ftilde \in \LtD,$ it follows that $\utilde \in \HtD$ by \cref{thm:shift}, and so in particular $\utilde \in \HmmD.$ We will use the fact that $\sdsol$ is self-adjoint; this follows from the fact that the boundary-value problem \cref{eq:sdeq,eq:sddbc,eq:sdnbc} is self-adjoint\ednote{Both - do I need to show this? It's straightforward.}.% Note to self, have (Rv,w), where R is resolvent, for v in L^2, w in H^m. Have PR = Id (but not necessarily the other way round) so w = PRw. Then (Rv,w) = (Rv,PRw) = (PRv,Rw) (P self-adjoint) = (v,Rw). Denote differential operator (on the correct space) by P.
%% We can then compute
%% \begin{align*}
%% \NHmmD{\utilde} &=  \sup_{0 \neq \vtilde \in \HmD} \frac{\IPLtD{\ftilde}{\DeltaAI\vtilde}}{\NHmD{\vtilde}}\text{ as } \DeltaAI \text{ is self-adjoint}\\
%% &\leq \sup_{0 \neq \vtilde \in \HmD} \frac{\NHmmmtD{\ftilde}\CAm\NHmD{\vtilde}}{\NHmD{\vtilde}} \text{ by \cref{thm:shift}}\\
%% &= \CAm \NHmmmtD{\ftilde}
%% \end{align*}
%% as required. \epf

We can now prove the main result of this \lcnamecref{sec:discsob}.

\ble[Relationship between discrete and continuous negative-order norms]\label{lem:negdiscsum}
Under \cref{ass:highp}, for any integer $j \in [0,p+1],$ there exists a constant $\Csumj > 0$ such that for all $\vh \in \Vhp,$
\beq\label{eq:negdiscsum}
\Nfn{-j,h,n}{\vh} \leq \Csumj\mleft(\En\nvar\mright)^{\floor{\frac{j}2}}\NLiD{n} \sum_{m=0}^j h^{m} \NHnfn{-(j-m)}{D}{\vh}.
\eeq
\ele

\bpf[Proof of \cref{lem:negdiscsum}]
Let $\wh \in \Vhp,$ and define $\zh = \DeltahI \wh$ and $z = \DeltaAI \wh$ (observe $z$ is well-defined as $\Vhp \subseteq \LtD$). Then, for all $\vh \in \Vhp$, we have

\begin{align*}
\IPLtD{A \grad z}{\grad \vh} = \IPLtD{A \grad \mleft(\sdsol \wh\mright)}{\grad \vh} = \IPLtDn{\wh}{\vh},\quad \text{and}\\
\IPLtD{A \grad \zh}{\grad \vh} = \IPLtDn{\Deltah \zh}{\vh} = \IPLtDn{\wh}{\vh},
\end{align*}
where the equalities in the first line follow from the definition of $z$ and \cref{eq:deltaagreen}, and the equalities in the second line follows from \cref{eq:discderdef} and the definition of $\zh.$  Therefore (using the fact that $A$ is symmetric), for all $\vh \in \Vhp,$ $\IPLtD{A\grad z}{\grad \vh} = \IPLtD{A\grad \zh}{\grad \vh},$ i.e., $\zh = \Ph z.$

We now have, for $m \in [-1,p-1]$
\begin{align}
\NHnfn{-m}{D}{\DeltahI \wh} &\leq \NHnfn{-m}{D}{z} + \NHnfn{-m}{D}{z-\zh}\nonumber\\
&\leq \NHnfn{-m}{D}{z} + \Cwfn{-m}  \CAfn{0} \CSZfn{2} \errn{m} \NLiD{n} h^{m+2} \NLtD{\wh}\nonumber\\
&\quad\quad\text{ by \cref{lem:ellprojerrw,lem:scottzhang,thm:shift}, as }\zh = \Ph z\nonumber\\
&= \Cshiftfn{m} \shiftn{m} \NHnfn{-(m+2)}{D}{\wh} + \Cm \errn{m} \frac{\NLiD{n}}{\nmin} h^{m+2} \NLtDn{\wh}\label{eq:sumforrecursion}
\end{align}
by \cref{lem:shiftnegativew}.
From \cref{eq:sumforrecursion}, we can conclude that, for $l \in \NN$ and $\vh \in \Vhp$, writing $\wh = \Deltahmlpo \vh$,
\beq\label{eq:lrecursion}
\NHnfn{-m}{D}{\Deltahml \vh} \leq \Cshiftfn{m}\shiftn{m} \NHnfn{-(m+2)}{D}{\Deltahmlpo \vh} + \Cm \errn{m} \frac{\NLiD{n}}{\nmin} h^{m+2} \NLtDn{\Deltahmlpo \vh}
\eeq
as $\Deltahml = \DeltahI \Deltahmlpo$. We now use \cref{eq:lrecursion} recursively to bound $\Nfn{j,h,n}{\vh}$.

If $j = 2l,$ then one can show inductively using \cref{eq:lrecursion} that for any integer $t \in [0,l]$
\beq\label{eq:evenrecursivesum}
\Nfn{-2l,h,n}{\vh} \leq \mleft(\En\nvar\mright)^t\sum_{m=0}^t \Efn{m,t} h^{2m} \NHDfn{-2(t-m)}{\vh} ,
\eeq
where we define the constants $\Efn{m,t}$ inductively by
\begin{align}
\label{eq:Emt1}\Efn{0,0} &=1,\\
\label{eq:Emt2}\Efn{m,t} &= \Cshiftfn{2(t-1-m)} \Efn{m,t-1} \tfor 0 \leq m \leq t-1, \quad\tand\\
\label{eq:Emt3}\Efn{t,t} &= \sum_{m=0}^{t-1} \Csumrecfn{2(t-1-m)} \Efn{m,t-1}.
\end{align}
To see this recurrence, we prove the inductive step: suppose
\beqs
\Nfn{-2l,h,n}{\vh} \leq \mleft(\En\nvar\mright)^{t-1}\sum_{m=0}^{t-1} \Efn{m,t-1} h^{2m} \NHDfn{-2(t-1-m)}{\vh}.
\eeqs
Then using \cref{eq:lrecursion}, we have
\begin{align*}
\Nfn{-2l,h,n}{\vh} &\leq \mleft(\En\nvar\mright)^{t-1}\sum_{m=0}^{t-1} \Efn{m,t-1} h^{2m} \mleft(\Cshiftfn{2(t-1-m)}\shiftn{2(t-1-m)} \NHnfn{-(2(t-1-m)+2}{D}{\Deltah^{-l+t} \vh}\mright.\\
&\quad\quad\mleft.+ \Csumrecfn{2(t-1-m)} \errn{2(t-1-m)} \nvar h^{2(t-1-m)+2} \NLtDn{\Deltah^{-l+t} \vh}\mright),
\end{align*}
which upon rearranging, and using the fact that $\shiftn{2(t-1-m)} \leq \errn{2(t-1-m)}$ and $\nvar \geq 1,$ yields \cref{eq:evenrecursivesum}, with the recurrence \cref{eq:Emt1,eq:Emt2,eq:Emt3}.

If $j=2l+1,$ then we first reduce $\Nfn{-j,h}{\vh}$ to a point analagous to the even case, and then proceed as before. Let $\wh$ and $\zh$ be as at the beginning of the proof, and let $z$ solve the variational formulation\footnote{We use the variational formulation here, as we will need to bound the $H^1$-norm of $z$ by the $H^{-1}$-norm of $\wh$, which is immediate from the Lax--Milgram theorem.}  of \cref{eq:sdeq,eq:sddbc,eq:sdnbc} with $\ftilde = \wh$. Observe that we still have $\zh = \Ph z,$ and

\beq\label{eq:LMHmo}
\NHfn{1}{D}{z} \leq \frac1{\Amin} \NHfn{-1}{D}{n\wh}
\eeq
by the Lax--Milgram Theorem. Then
\begin{align}
\NLtDn{\Deltah^{-1/2}\wh} &= \NLtDn{\Deltah^{1/2} \zh}\nonumber\\
&= \IPLtDn{\Deltah \zh}{\zh} \text{ by \cref{eq:feiptrans}}\\
&= \IPLtD{A \grad \zh}{\grad \zh}\nonumber\\
&= \NLiDop{A} \NHoD{\Ph z}\nonumber\text{ by the Cauchy--Schwartz inequality and the definition of } \zh\\
&\leq \NLiDop{A}\mleft(\NHoD{z} + \Cprojfn{1}\NHoD{0 - z}\mright) \text{ by \cref{lem:ellprojerr}}\nonumber\\
&\leq \frac{\mleft(1+\Cprojfn{1}\mright)\NLiDop{A}}{\Amin}  \NHmoD{n\wh}\text{ by \cref{eq:LMHmo}}\label{eq:deltahhalf}\\
&\leq \frac{\mleft(1+\Cprojfn{1}\mright)\NLiDop{A}}{\Amin}\NLiD{n}  \NHnfn{-1}{D}{\wh}\text{ as in the proof of \cref{lem:shiftnegativew}}\nonumber
\end{align}
We now return to $\Njhn{\vh}$:
\begin{align*}
  \Nfn{j,h,n}{\vh} &= \NLtDn{\Deltah^{-l-1/2} \vh}\\
  &= \NLtDn{\Deltah^{-1/2} \Deltah^{-l} \vh}\\
&\leq \frac{\mleft(1+\Cprojfn{1}\mright)\NLiDop{A}}{\Amin}\NLiD{n}\NHnfn{-1}{D}{\Deltah^{-l}\vh}
\end{align*}
by \cref{eq:deltahhalf}.

Similarly to \cref{eq:evenrecursivesum}, one can use \cref{eq:lrecursion} recursively to show that, for any integer $t \in [0,l]$
\beq\label{eq:oddrecursive}
\NHnfn{-1}{D}{\Deltah^{-l}\vh} \leq \mleft(\En \nvar\mright)^t\mleft(\Etildefn{0,t} \NHnfn{-(2t+1)}{D}{\Deltah^{-l+t}\vh} + \sum_{m=0}^t \Etildefn{m,t} h^{2m+1}  \NHnfn{-2(t-m)}{D}{\Deltah^{-l+t}\vh}\mright),
\eeq
where  we define the $\Etildefn{m,t}$ inductively for $t \in [0,l]$ by
\begin{align}
\label{eq:Etilde1}\Etildefn{0,0} &= 1,\\
%\label{eq:Etilde2}\Etildefn{1,1} &= \Csumrecfn{1}\\
\label{eq:Etilde3}\Etildefn{0,t} &= \Etildefn{0,t-1}\Cshiftfn{2(t-1)+1)},\\
\label{eq:Etilde4}\Etildefn{m,t} &= \Etildefn{m,t-1}\Cshiftfn{2(t-1-m)}\tfor m = 1,\ldots,t-1,\quad\text{and}\\
\label{eq:Etilde5}\Etildefn{t,t} &= \Csumrecfn{2(t-1)+1} + \sum_{m=0}^{t-1}\Etildefn{m,t-1}\Csumrecfn{2(t-1-m)}.
\end{align}
To show \cref{eq:oddrecursive,eq:Etilde1,eq:Etilde3,eq:Etilde4,eq:Etilde5}, observe that \cref{eq:Etilde1} is immediate. We show \cref{eq:oddrecursive,eq:Etilde3,eq:Etilde4,eq:Etilde5} by induction. Suppose %\cref{eq:oddrecursive} holds for $t-1,$ then
\begin{align*}
  \NHnfn{-1}{D}{\Deltah^{-l}\vh} \leq \mleft(\En \nvar\mright)^{t-1}&\Bigg(\Etildefn{0,t-1} \NHnfn{-(2(t-1)+1)}{D}{\Deltah^{-l+t-1}\vh}\\
& \quad + \sum_{m=0}^{t-1} \Etildefn{m,t-1} h^{2m+1}  \NHnfn{-2(t-1-m)}{D}{\Deltah^{-l+t-1}\vh}\Bigg).
\end{align*}
Then
\begin{align}
  &\NHnfn{m}{D}{\Deltah^{-l}\vh} \leq\nonumber\\
  &\quad\mleft(\En \nvar\mright)^{t-1} \Etildefn{0,t-1}\bigg(\Cshiftfn{2(t-1)+1} \shiftn{2(t-1)+1}\NHnfn{-(2(t-1)+1)}{D}{\Deltah^{-l + t}\vh}\nonumber\\
    &\quad\quad\quad\quad\quad\quad\quad\quad\quad\quad\quad\quad\quad\quad+\Csumrecfn{2(t-1)t+1} \errn{2(t-1)+1}\nvar h^{2t+1} \NLtDn{\Deltah^{-l + t}\vh}\bigg)\nonumber\\
  &\quad+ \mleft(\En \nvar\mright)^{t-1}\sum_{m=1}^{t-1} \Etildefn{m,t-1} h^{2m+1}\Big(\Cshiftfn{2(t-1-m)} \shiftn{2(t-1-m)}\NHnfn{-2(t-1-m)}{D}{\Deltah^{-l + t}\vh}\nonumber\\
  &\quad\quad\quad\quad\quad\quad\quad\quad\quad\quad\quad\quad\quad\quad\quad\quad\quad\quad\quad\quad+ h^{2(t-m)} \Csumrecfn{2(t-1-m)}\errn{2(t-1-m)}\nvar \NLtDn{\Deltah^{-l + t}\vh}\Big)\label{eq:verylongeq}
%% &= \Etildefn{0,t-1} \CAfn{2t-1} \NHfn{-(2(t-1)+1)}{D}{\Deltah^{-l + t}\vh}\\
%% &\quad\quad+ \mleft(\Csumrecfn{2t-1}\sum_{m=1}^{t-1} \Etildefn{m,t-1}\CAfn{2(t-1-m)}\mright) h^{2t+1}\NLtD{\Deltah^{-l + t}\vh}\\
%% &\quad\quad+ \sum_{m=1}^{t-1} \Etildefn{m,t-1} h^{2m+1} \CAfn{2(t-1-m)}\NHfn{-2(t-m)}{D}{\Deltah^{-l + t}\vh},
\end{align}
and rearranging \cref{eq:verylongeq}, and using the facts that $\shiftn{m} \leq \errn{m}$ for all $m$ and $\nvar \geq 1,$ we obtain \cref{eq:oddrecursive} with the constants $\Etildefn{m,t}$ given by \cref{eq:Etilde1,eq:Etilde3,eq:Etilde4,eq:Etilde5}. Therefore, we conclude that if $j = 2l+1$
\begin{align}
\Nfn{-j,h,n}{\vh} \leq& \frac{\mleft(1+\Cprojfn{1}\mright)\NLiDop{A}}{\Amin}\NLiD{n}\mleft(\En \nvar\mright)^{l}\mleft(\Etildefn{0,l} \NHnfn{-(2l+1)}{D}{\vh}\mright.\nonumber\\
&\mleft.\hphantom{\frac{\mleft(1+\Cprojfn{1}\mright)\NLiDop{A}}{\Amin}\NLiD{n}\mleft(\En \nvar\mright)^{l}}
\quad\quad+ \sum_{m=0}^{l} \Etildefn{m,l} h^{2m+1}  \NHnfn{-2(l-m)}{D}{\vh}\mright)\label{eq:oddfinal}
\end{align}
and combining \cref{eq:evenrecursivesum} (with $t=l$) and \cref{eq:oddfinal}, the result follows.%\opntodo{Maybe put recursion in, but it's all in notes.}
\epf

\bre[Why is the factor $n$ only on the right-hand side of \cref{eq:sdeq}?]\label{rem:why}
The reason for defining $\Sn$ with a factor $n$ on the right-hand side of \cref{eq:sdeq} but not on the left-hand side is somewhat buried in the proof of \cref{thm:fembound} and its associated lemmas. However, we give an overview of the reason here.

All the bounds in the proofs of \cref{lem:boundarybound,lem:higherbound,lem:ltthetahbound} are in $n$-weighted discrete norms, because to prove bounds on the $n$-weighted $L^2$ projection $\Qhn$ we work in $n$-weighted higher-order discrete norms (as in \cref{lem:wltdprojerr}), rather than in non-weighted norms. Because we only work in $n$-weighted norms, we need \cref{lem:intoip} below to hold in the \emph{$n$-weighted} inner product. To prove \cref{lem:intoip} in the $n$-weighted inner product, we therefore use the $n$-weighted inner product on the left-hand side of \cref{eq:discderdef}, where the $\Deltah$ operator appears. Because we use the $n$-weighted inner product on the left-hand side of \cref{eq:discderdef}, we must then use the $n$-weighted inner product on the right-hand side of \cref{eq:deltaagreen}. Using this inner product in \cref{eq:deltaagreen} ensures the beginning of the proof of \cref{lem:negdiscsum} works---see the proof for more details.

We do not, however, put a factor $n$ on the left-hand side of \cref{eq:sdeq,eq:deltaagreen}. If we did, when we apply \cref{thm:shift} to $\Sn \ftilde$ (as we do in \cref{lem:shiftnegativew}), the resulting bounds would not be fully explicit in $n$ (because in \cref{thm:shift} we do not know explicitly how the constants depend on the `diffusion coefficient'). Not putting a factor $n$ on the left-hand side of \cref{eq:sdeq} is the reason why there is a \emph{non-weighted} inner product on the right-hand side of \cref{eq:discderdef} (so that the beginning of the proof of \cref{lem:negdiscsum}, as mentioned above, works).
\ere

\subsection{Proof of \cref{thm:fembound}}\label{sec:fembound}

Having established the necessary preliminary results about discrete Sobolev spaces, we are now in a position to prove our main theorem, \cref{thm:fembound}, which we do via a series of lemmas. The proof proceeds via a splitting argument, as discussed in \cref{sec:errorsplit}.

For ease of notation in the following lemmas, we follow the notation of \cite{DuWu:15} and let 
\beqs
\rho \de u - \Ph u, \tand
\eeqs
\beqs
\thetah \de \Ph u - \uh.
\eeqs


The following \lcnamecref{lem:simpleform} shows that $\thetah$ solves a discrete Helmholtz problem with data $k^2 \rho$ (in $D$) and $ik \rho$ (on $\GI$).
\ble[$\thetah$ solves a discrete Helmholtz problem]\label{lem:simpleform}
For any $\vh \in \Vhp,$
\beq\label{eq:thetaform}
\aT(\thetah,\vh) = k^2\IPLtDn{\Qhn\rho}{\vh} + ik \IPLtGI{\rho}{\vh}.
\eeq
\ele

\bpf[Proof of \cref{lem:simpleform}]
Let $\vh \in \Vhp.$ Then $\aT(\thetah,\vh) = \aT(u-\uh,\vh) - \aT(\rho,\vh) = -\aT(\rho,\vh)$ by Galerkin orthogonality. By definition of $a,$ we have $-\aT(\rho,\vh) = -\IPLtD{A\grad\rho}{\vh} + k^2 \IPLtD{n\rho}{\vh} + ik\IPLtGI{\rho}{\vh}.$ By Galerkin orthogonality for $\rho = u - \Ph u$, $\IPLtD{A\grad\rho}{\vh} = 0$, and so by the definition of the $n$-weighted $L^2$ inner product, and the $n$-weighted $L^2$-projection $\Qhn,$ the result follows.
\epf

As mentioned in \cref{rem:epdef}, the definition of the elliptic projection $\Ph$ (\cref{eq:epdef,eq:aepdef} above) uses a Neumann boundary condition on $\GI$, rather than an impedance boundary condition, and therefore the right-hand side of \cref{eq:thetaform} includes a term defined on the truncation boundary $\GI.$ If the definition of the elliptic projection used an imepdance boundary condition, this term would disappear. The presence of this term means that when we bound $\NLtDn{\thetah}$ and $\Npmohn{\thetah}$ in the proofs of \cref{lem:higherbound,lem:ltthetahbound} below, we will encounter terms involving $\NLtGI{\thetah}$. Therefore, we first prove a bound on $\NLtGI{\thetah}.$

\ble[Bound on $\NLtGI{\thetah}$ by $\Npmoh{\thetah}$]\label{lem:boundarybound}
Under the assumptions of \cref{thm:fembound}, we have
\beq\label{eq:boundarybound}
\NLtGI{\thetah}^2 \leq \Cboundaryo \mleft(\En\nvar\mright)^{2\mleft(\floor{\frac{p-1}2}+1\mright)}\NLiD{n}^4 k^2 h^{2p-1} \Nfn{p-1,h,n}{\thetah}^2 + \Cboundaryt h \NHokD{\rho}^2,
\eeq
\ele

\bpf[Proof of \cref{lem:boundarybound}]
In \cref{eq:thetaform}, let $\vh = \thetah,$ and take the imaginary part to obtain
\beq\label{eq:boundboundarybelow}
-k \NLtGI{\thetah}^2 \leq \Im k^2 \IPLtDn{\Qhn \rho}{\thetah} + \Re k \IPLtGI{\rho}{\thetah}.
\eeq
Therefore by \cref{cor:ipdiscbound}
\beq
\NLtGI{\thetah}^2 \leq  k \Nfn{1-p,h,n}{\Qhn \rho}\Nfn{p-1,h,n}{\thetah} + \NLtGI{\rho}\NLtGI{\thetah}.\label{eq:thetaboundarypart}
%% \nonumber\\
%% &\leq  k \NLiD{n}^2 \Nfn{1-p,h}{\Qhn \rho}\Nfn{p-1,h}{\thetah} + \NLtGI{\rho}\NLtGI{\thetah}
\eeq
We first bound the negative norm $\Nfn{1-p,h}{\Qhn\rho}$, to do this we use \cref{lem:negdiscsum}. However, as we will apply \cref{lem:negdiscsum} we need to estimate negative (standard) Sobolev norms of $\Qhn\rho$; for integers $m \in [0,p-1]$ we have (observing that $\Qhn \Ph u = \Ph u$ as $\Ph u \in \Vhp$).
\begin{align}
\NHnfn{-(p-1-m)}{D}{\Qhn\rho} &\leq \NHnfn{-(p-1-m)}{D}{\Qhn u - u} + \NHnfn{-(p-1-m)}{D}{u - \Ph u}\nonumber\\
&\leq \CSZfn{(p-1-m)} \nvar h^{(p-1-m)} \NLtDn{u-\Ph u}\nonumber\\
&\quad+ \Cwfn{-(p-1-m)}\errn{(p-1-m)} h^{p-m} \NHoDn{u - \Ph u}\nonumber\\
&\quad\quad\text{ by \cref{lem:ellprojerrw,lem:wltdprojerr}, taking } \wh = \Ph u \text{ in \cref{lem:ellprojerrw,eq:wltdprojerr},}\nonumber\\
&\leq \mleft(\CSZfn{(p-1-m)} \Cwz + \Cwfn{-(p-1-m)}\mright)\En\nvar\NLiD{n} h^{p-m} \NHokD{\rho}\label{eq:Qhnrhoneg}
\end{align}
by \cref{lem:ellprojerrw}. By \cref{lem:negdiscsum,eq:Qhnrhoneg} we have
\begin{align}
\Nfn{1-p,h,n}{\Qhn \rho}&\leq \mleft(\En\nvar\mright)^{\floor{\frac{p-1}2}}\NLiD{n} \Csumfn{p-1} \sum_{m=0}^{p-1} h^{m} \NHfn{-(p-1-m)}{D}{\Qhn \rho} \nonumber\\
&\leq \Cmess \mleft(\En\nvar\mright)^{\floor{\frac{p-1}2}+1}\NLiD{n}^2 h^p \NHokD{\rho}.\label{eq:Qhnrhosum}
\end{align}
To bound $\NLtGI{\rho}$appearing in  the second term on the right-hand side of \cref{eq:thetaboundarypart} we use \cref{thm:multiplicativetrace,lem:ellprojerr}. We take $\wh = \Ph u$ in \cref{eq:ellprojerr} and use the fact that $\NHoD{\cdot} \leq \NHokD{\cdot}$ to obtain
\beq\label{eq:rhomtbound}
\NLtGI{\rho} \leq \CMT\NHoD{\rho}^{1/2}\NLtD{\rho}^{1/2} \leq \CMT \Cprojfn{0}^{\half} h^\half \NHoD{\rho}\leq \CMT \Cprojfn{0}^{\half} h^{\half} \NHokD{\rho}.
\eeq
Therefore by \cref{eq:rhomtbound} and Young's inequality \cref{eq:young}, we obtain
\beq\label{eq:rhothetamt}
\NLtGI{\rho}\NLtGI{\thetah} \leq \half \CMT^2 \Cprojfn{0} h \NHokD{\rho}^2 + \half \NLtGI{\thetah}^2.
\eeq
By combining \cref{eq:thetaboundarypart,eq:Qhnrhosum,eq:rhothetamt} we have
\begin{align}
\NLtGI{\thetah}^2 &\leq k \Cmess \mleft(\En\nvar\mright)^{\floor{\frac{p-1}2}+1}\NLiD{n}^2 h^p \NHokD{\rho}\Nfn{p-1,h,n}{\thetah}\nonumber\\
&\quad+ \half \CMT^2 \Cprojfn{0} h\NHokD{\rho}^2 + \half \NLtGI{\thetah}^2.\label{eq:thetahboundnear}
\end{align}
By using Young's inequality on the first term in \cref{eq:thetahboundnear}, and moving the $\NLtGI{\thetah}^2/2$ term onto the left-hand side, we obtain \cref{eq:boundarybound}.
\epf

We can now prove the main two \lcnamecrefs{lem:ltthetahbound} in the proof of \cref{thm:fembound}.

\ble[Bound on higher-order discrete norms of $\thetah$ by $\NLtD{\thetah}$]\label{lem:higherbound}
Under the assumptions of \cref{thm:fembound}, for integer $m \in [1,p-1]$ there exist constants $\Chighmo,$ $\Chighmt > 0$ such that
\begin{align}
\Nfn{m,h,n}{\thetah} &\leq \Chighmo \mleft(\mleft(\En\nvar\mright)^{\half(\floor{\frac{p-1}2}+1)}\NLiD{n}\nmin^{-\frac{p}2}\mright)^mk^m \NLtD{\thetah}\nonumber\\
&\quad\quad+ \Chighmt \nvar^3\NLiD{n}\nmin^{1-m} h^{1-m} \NHokD{\rho}.\label{eq:chigh}
\end{align}
\ele
\bpf[Proof of \cref{lem:higherbound}]
By inserting the definitions of $a$ and $\Deltah$ into \cref{eq:thetaform} and rearranging, we have for any $\vh \in \Vhp$
\beqs
\IPLtDn{\Deltah \thetah}{\vh} = k^2 \IPLtDn{\thetah}{\vh} + k^2\IPLtDn{\Qhn \rho}{\vh} + ik \IPLtGI{\thetah}{\vh} + ik \IPLtGI{\rho}{\vh}.
\eeqs
Therefore, if we take $\vh = \Deltah^{m-1}\thetah$, by \cref{lem:intoip} we have
\beq\label{eq:deltahm}
\Nfn{m,h,n}{\thetah}^2 = k^2 \Nfn{m-1,h,n}{\thetah}^2 + k^2 \IPLtDn{\Deltah^{\frac{m-1}2} \Qhn \rho}{\Deltah^{\frac{m-1}2}\thetah} + ik\IPLtGI{\thetah}{\Deltah^{m-1} \thetah} + ik \IPLtGI{\rho}{\Deltah^{m-1} \thetah}.
\eeq
We now proceed to bound the two terms in \cref{eq:deltahm} defined on the truncation boundary $\GI.$ For the first term, we have
\begin{align}
\IPLtGI{\thetah}{\Deltah^{m-1}\thetah} &\leq \NLtGI{\thetah}\NLtGI{\Deltah^{m-1}\thetah}\nonumber\\
&\leq \CMT \CinvVhp^{1/2} \NLtGI{\thetah} h^{-\half} \NLtD{\Deltah^{m-1} \thetah}\nonumber\\
&\quad\quad\text{ by \cref{thm:multiplicativetrace,lem:inverseinequality}}\nonumber\\
&= \CMT \CinvVhp^{1/2} h^{-\half}\NLtGI{\thetah}\nmin^{-1}\Nfn{2m-2,h,n}{\thetah}\text{ by the definition of }\Nfn{2m-2,h,n}{\cdot},\nonumber\\
&\leq \CMT \CinvVhp^{1/2} \Chinv^{m-1} h^{-m+\half} \nmin^{-m} \NLtGI{\thetah}\Nfn{m-1,h,n}{\thetah}\nonumber\\
&\quad\text{ by \cref{lem:inversediscrete} applied } m-1 \text{ times,}\label{eq:useitagain}\\
&\leq \CMT \CinvVhp^{1/2} \Chinv^{m-1} h^{-m+\half} \nmin^{-m}\nonumber\\
&\quad\quad\mleft(\Cboundaryo^{\half} \mleft(\En\nvar\mright)^{\floor{\frac{p-1}2}+1}\NLiD{n}^2 k h^{p-\half} \Nfn{p-1,h,n}{\thetah} + \Cboundaryt^{\half} h^{\half} \NHokD{\rho}\mright)\Nfn{m-1,h,n}{\thetah}\nonumber\\
&\quad\quad\text{ by \cref{lem:boundarybound} and \cref{eq:simple}}\nonumber\\
&\leq \bigg(\CBo \mleft(\En\nvar\mright)^{\floor{\frac{p-1}2}+1}\NLiD{n}^2\nmin^{-p}k \Nfn{m-1,h,n}{\thetah}\nonumber\\
&\quad\quad\quad+ \CBt \nmin^{-m}h^{1-m} \NHokD{\rho}\bigg)\Nfn{m-1,h,n}{\thetah}\label{eq:firstboundary}
\end{align}
by \cref{lem:inversediscrete} applied $p-m$ times.

To bound the second boundary term in \cref{eq:deltahm}, we have
\beq\label{eq:secondboundarytemp}
\IPLtGI{\rho}{\Deltah^{m-1} \thetah} \leq \CMT \CinvVhp^{1/2} \Chinv^{m-1}\nmin^{-m}h^{\half-m}\NLtGI{\rho}\Nfn{m-1,h}{\thetah}
\eeq
using the same reasoning as we used to obtain \cref{eq:useitagain} above. By \cref{thm:multiplicativetrace,lem:ellprojerr} (with $\wh = \Ph u$) we have
\beq\label{eq:secondboundarytemp2}
\NLtGI{\rho} \leq \CMT \Cprojfn{0}^{\half} h^{\half} \NHokD{\rho}.
\eeq

Inserting \cref{eq:secondboundarytemp2} into \cref{eq:secondboundarytemp} we obtain
\beq\label{eq:secondboundary}
\IPLtD{\rho}{\Deltah^{m-1}\thetah} \leq \CBth \nmin^{-m}h^{1-m} \NHokD{\rho} \Nfn{m-1,h}{\thetah}.
\eeq

Therefore, from\renewcommand{\creflastconjunction}{,~} \cref{eq:deltahm,eq:firstboundary,eq:secondboundary},\renewcommand{\creflastconjunction}{\lastconj} and the Cauchy--Schwartz inequality, we have
\begin{align*}
\Nfn{m,h}{\thetah}^2 &\leq k^2 \Nfn{m-1,h}{\thetah}^2 + k^2 \Nfn{m-1,h,n}{\Qhn \rho}\Nfn{m-1,h,n}{\thetah}\nonumber\\
&\quad\quad+ k\mleft(\CBo \mleft(\En\nvar\mright)^{\floor{\frac{p-1}2}+1}\NLiD{n}^2\nmin^{-p}k \Nfn{m-1,h,n}{\thetah} + \CBt \nmin^{-m}h^{1-m} \NHokD{\rho}\mright)\Nfn{m-1,h,n}{\thetah}\nonumber\\
&\quad\quad+ k\CBth \nmin^{-m}h^{1-m} \NHokD{\rho} \Nfn{m-1,h}{\thetah}%\label{eq:thetahighnearlynearly}.
\end{align*}
Therefore using Young's inequality \cref{eq:young} we have
\begin{align*}
\Nfn{m,h}{\thetah}^2 &\leq k^2\mleft(\frac32 + \CBo \mleft(\En\nvar\mright)^{\floor{\frac{p-1}2}+1}\NLiD{n}^2\nmin^{-p} + \half \CBt \nmin^{-m} + \half\CBth \nmin^{-m}\mright)\Nfn{m-1,h,n}{\thetah}^2\\
&\quad\quad+ \frac{k^2}2 \Nfn{m-1,h,n}{\Qhn \rho}^2 +  h^{2(1-m)} \NHokD{\rho}^2,
\end{align*}
and by \cref{eq:simple}
\begin{align}
\Nfn{m,h}{\thetah} &\leq k\mleft(\sqrt{\frac32} + \CBo^{\half} \mleft(\En\nvar\mright)^{\half(\floor{\frac{p-1}2}+1)}\NLiD{n}\nmin^{-\frac{p}2} + \frac1{\sqrt{2}} \CBt^{\half} \nmin^{-\frac{m}2} + \frac1{\sqrt{2}}\CBth^{\half} \nmin^{-\frac{m}2}\mright)\Nfn{m-1,h,n}{\thetah}\nonumber\\
&\quad\quad+ \frac{k}{\sqrt{2}} \Nfn{m-1,h,n}{\Qhn \rho} +  h^{1-m} \NHokD{\rho},\label{eq:thetahighnearly}
\end{align}

We now proceed to bound $\Nfn{m-1,h,n}{\Qhn \rho}$: By \cref{lem:inversediscrete} we have
\begin{align}
  \Nfn{m-1,h,n}{\Qhn \rho} &\leq\Chinv^{m-1} \nmin^{1-m}h^{1-m} \NLtDn{\Qhn\rho}\nonumber\\
  &\leq \Chinv^{m-1} \nmin^{1-m}h^{1-m} \mleft(\NLtDn{\Qhn u - u} + \NLtDn{u-\Ph u}\mright)\label{eq:qhnmmo}
\end{align}
as in the proof of \cref{lem:boundarybound} using the fact that $\Qhn \Ph u = \Ph u.$

Using \cref{lem:wltdprojerr} we can bound the first of these terms by the second:
\beq\label{eq:firsttermbound}
\NLtDn{\Qhn u - u} \leq \CSZfn{0} \nvar \NLtDn{u-\Ph u}.
\eeq
We can also bound
\beq\label{eq:secondtermbound}
\NLtDn{u - \Ph u} \leq \Cwfn{0} \nvar^3h \NHnfn{1}{D}{\rho} \leq \Cwfn{0} \nvar^3\NLiD{n} h \NHokD{\rho}
\eeq
by \cref{lem:ellprojerrw}. Therefore by \cref{eq:qhnmmo,eq:firsttermbound,eq:secondtermbound} we have (as $kh \leq 1$ and $\nvar \geq 1$)
\beq\label{eq:qhnnearly}
k \Nfn{m-1,h}{\Qhn \rho} \leq \Chinv^{m-1}\Cwz \mleft(1+\CSZfn{0}\mright)\nvar^3\NLiD{n}\nmin^{1-m} h^{1-m}\NHokD{\rho}.
\eeq
Therefore using \cref{eq:thetahighnearly,eq:qhnnearly} (and the fact that $\NLiD{n}, \nmin^{-1} \geq 1,$ and so $\En, \nvar \geq1$)  we obtain

\begin{align}
  \Nfn{m,h}{\thetah} &\leq \CRecofn{m} \mleft(1+\mleft(\En\nvar\mright)^{\half(\floor{\frac{p-1}2}+1)}\NLiD{n}\nmin^{-\frac{p}2}\mright)k \Nfn{m-1,h}{\thetah}\nonumber\\
  &\quad+ \CRect \nvar^3\NLiD{n}\nmin^{1-m} h^{1-m} \NHokD{\rho}.\label{eq:readytorecurse}
\end{align}
Using \cref{eq:readytorecurse} recursively, and the facts that $\nmin^{-1} \geq 1$ and $hk \leq 1,$ we obtain \cref{eq:chigh}.
\epf

The following \lcnamecref{lem:continuity} is straightforward to prove, and used in the proof of \cref{lem:ltthetahbound} below
\ble[Continuity of $a$]\label{lem:continuity}
For any $\vo, \vt \in \HozDD,$
\beqs
\abs{\aT(\vo,\vt)} \leq \Cc \NLiD{n} \NHokD{\vo}\NHokD{\vt}.
\eeqs
\ele

\ble[Bound on $\NLtDn{\thetah}$ by $\Nfn{p-1,h,n}{\thetah}$]\label{lem:ltthetahbound}
Under the assumptions of \cref{thm:fembound},%there exist constants $\Cfirst, \Csecond > 0$ such that
\begin{align}
\NLtD{\thetah}&\leq \mleft(\Cfirst\NHokD{\rho} +\Csecond  k^2h^p\Nfn{p-1,h,n}{\thetah}\mright)\mleft(\En \nvar\mright)^{\floor{\frac{p-1}2}+1}\NLiD{n}^2\nonumber\\
&\quad\quad\Pfn{p-2}\mleft(\NLiD{n}\mright)\mleft(\CFEMotilde h + \CFEMttilde \CAnk (hk)^p\mright).\label{eq:ltthetahbound}
\end{align}
\ele

\bpf[Proof of \cref{lem:ltthetahbound}]
The proof initially uses the standard duality technique, but then becomes more complex than standard proofs. This complexity is due to the fact that (i) we are bounding $\thetah$, not the finite-element error $u-\uh$, and (ii) we are bounding $\thetah$ by its higher-order-discrete norms, rather than by its $H^1$-norm as in the Aubin--Nitsche argument.

Consider the adjoint variational problem: Find $w \in \HozDD$ such that for all $v \in \HozDD$
\beq\label{eq:adjointtheta}
\aT(v,w) = \IPLtD{v}{\thetah}.
\eeq
(I.e., $w$ solves the adjoint problem \cref{eq:adjoint} with right-hand side given by $n\thetah$.) Let $\eh \de u -\uh$ be the finite-element error, put $v = \eh$ in \cref{eq:adjointtheta} and take the complex conjugate\footnote{The reason we take the complex conjugate is that to apply Galerkin orthogonality for $w-\Ph w$, the term $w-\Ph w$ must be the first argument of $\aT$. Alternatively, one could define $\widetilde{\Ph}$ to be the analogue of the elliptic projection but defined in the second argument, show analogues of the error bounds \cref{lem:ellprojerr} and proceed with the proof of this \lcnamecref{lem:ltthetahbound}. However, for simplicity, we instead take the complex conjugate of \cref{eq:adjointtheta}.} to obtain
\beqs
\IPLtD{\thetah}{\eh} = \overline{\aT(\eh,w-\Ph w)} = \aep(w-\Ph w,\eh) - k^2 \IPLtDn{w-\Ph w}{\eh} + ik \NLtGI{w - \Ph w}{\eh}.
\eeqs
By Galerkin orthogonality for $w-\Ph w,$ we have (recalling $\eh = \rho + \thetah$)
\begin{align}
  \IPLtD{\thetah}{\eh} &= \aep(w-\Ph w,\rho) - k^2 \IPLtDn{w-\Ph w}{\eh} + ik \NLtGI{w - \Ph w}{\eh}\nonumber\\
&= \overline{\aT(\rho,w-\Ph w)}  -k^2 \IPLtDn{w-\Ph w}{\thetah} -ik \IPLtGI{w-\Ph w}{\thetah}.\label{eq:doublego}
%&=\aT(\rho,w-\Ph w) -k^2 \IPLtDn{\thetah}{w-\Ph w} -ik \IPLtGI{\thetah}{w-\Ph w}
\end{align}

Therefore as $\eh = \rho + \thetah$ we can rearrange \cref{eq:doublego} and use the Cauchy-Schwartz inequality to obtain
\begin{align}
\NLtD{\thetah}^2 &\leq \Cc\NLiD{n} \NHokD{\rho} \NHokD{w- \Ph w} + k^2 \abs{\IPLtDn{\thetah}{w- \Ph w}}\nonumber\\
&+ k \abs{\IPLtGI{\thetah}{w-\Ph w}} + \NLtDn{\rho}\NLtDn{\thetah}\label{eq:boundingLtwo}
\end{align}
By combining \cref{lem:bestapprox,lem:ellprojerr}, we can show (as $w$ satisfies an adjoint Helmholtz problem with right-hand side $\thetah$)
\beq\label{eq:wlt}
\NLtD{w - \Ph w} \leq \Cprojfn{0} \Pfn{p-2}\mleft(\NLiD{n}\mright)\mleft(\CFEMotilde h^2 + \CFEMttilde \CAnk h(hk)^p\mright)\NLtD{\thetah}
\eeq
and
\beq\label{eq:who}
\NHoD{w - \Ph w} \leq 2\Cprojfn{-1} \Pfn{p-2}\mleft(\NLiD{n}\mright)\mleft(\CFEMotilde h + \CFEMttilde \CAnk (hk)^p\mright)\NLtD{\thetah}.
\eeq
We will be able to use \cref{eq:wlt,eq:who} to bound terms involving $w - \Ph w$ in \cref{eq:boundingLtwo}. We first estimate the inner product terms in \cref{eq:boundingLtwo}:
\begin{align}
\abs{\IPLtDn{\thetah}{w- \Ph w}} &= \abs{\IPLtDn{\thetah}{\Qhn w - \Ph w}}\nonumber\\
&\leq \Nfn{p-1,h,n}{\thetah}\Nfn{1-p,h,n}{\Qhn w - \Ph w}\text{ by \cref{lem:intoip}}\nonumber\\
&\leq \Nfn{p-1,h,n}{\thetah} \Csumfn{p-1} \mleft(\En \nvar\mright)^{\floor{\frac{p-1}2}}\NLiD{n}\nonumber\\
&\quad\quad\sum_{m=0}^{p-1} h^{m}\mleft(\NHnfn{-(p-1-m)}{D}{\Qhn w - w}+ \NHnfn{-(p-1-m)}{D}{w - \Ph w}\mright) \text{ by \cref{lem:negdiscsum}}\nonumber\\
&\leq \Nfn{p-1,h,n}{\thetah} \Csumfn{p-1}\mleft(\En \nvar\mright)^{\floor{\frac{p-1}2}}\NLiD{n} h^{p-1}\nonumber\\
&\quad\quad\sum_{m=0}^{p-1} \mleft(\CSZfn{p-1-m}\nvar \NLtDn{w - \Ph w} + \Cwfn{p-1-m} \errn{p-1-m} h \NHoDn{w - \Ph w}\mright)\nonumber\\
&\quad\quad\text{ by \cref{lem:wltdprojerr,lem:ellprojerrw}, taking $\wh = \Ph u$ in \cref{lem:ellprojerrw} and \cref{eq:wltdprojerr},}\nonumber\\
&\leq \Nfn{p-1,h,n}{\thetah} \Csumfn{p-1}\sum_{m=0}^{p-1} \mleft(\CSZfn{p-1-m}\Cprojfn{0} + \Cwfn{p-1-m} \Cprojfn{-1}\mright)                         \mleft(\En \nvar\mright)^{\floor{\frac{p-1}2}+1}\NLiD{n}^2 \nonumber\\
&\quad\quad h^p \Pfn{p-2}\mleft(\NLiD{n}\mright)\mleft(\CFEMotilde h + \CFEMttilde \CAnk (hk)^p\mright) \NLtD{\thetah} \text{ by \cref{eq:wlt,eq:who},}\nonumber\\
&=\Nfn{p-1,h,n}{\thetah}  \Cfourteen \mleft(\En \nvar\mright)^{\floor{\frac{p-1}2}+1}\NLiD{n}^2 h^p \Pfn{p-2}\mleft(\NLiD{n}\mright)\nonumber\\
&\quad\quad\mleft(\CFEMotilde h + \CFEMttilde \CAnk (hk)^p\mright) \NLtD{\thetah} \label{eq:innerprod1}.
\end{align}

We now estimate the other inner product term
\begin{align}
\abs{\IPLtGI{\thetah}{w - \Ph w}} &\leq \CMT\CinvVhp\mleft(\Cboundaryo^{\half} \mleft(\En\nvar\mright)^{\floor{\frac{p-1}2}+1}\NLiD{n} k h^{p-\half} \Nfn{p-1,h,n}{\thetah}\mright.\nonumber\\
&\mleft.\quad\quad+ \Cboundaryt^{\half} h^{\half} \NHokD{\rho}\mright) h^{-\half}\NLtD{w - \Ph w}\nonumber\\
&\quad\quad\text{ by \cref{eq:simple}, \cref{lem:inverseinequality,lem:boundarybound,thm:multiplicativetrace},}
\nonumber\\
&\leq \CMT \CinvVhp \Cprojfn{0}\mleft(\Cboundaryo^{\half} \mleft(\En\nvar\mright)^{\floor{\frac{p-1}2}+1}\NLiD{n} k h^{p} \Nfn{p-1,h,n}{\thetah} + \Cboundaryt^{\half} h \NHokD{\rho}\mright)\nonumber\\
&\quad\quad \Pfn{p-2}\mleft(\NLiD{n}\mright)\mleft(\CFEMotilde h + \CFEMttilde \CAnk (hk)^p\mright)\NLtD{\thetah}\label{eq:innerprod2}
\end{align}
 by \cref{eq:wlt}.

We now insert \cref{eq:wlt,eq:who,eq:innerprod1,eq:innerprod2} into \cref{eq:boundingLtwo}:
\begin{align}
\NLtD{\thetah}^2 &\leq \Bigg[\Cc\NLiD{n} \NHokD{\rho} \mleft(\Cprojfn{0} + 2\Cprojfn{1}\mright)\\
&+k^2\Nfn{p-1,h,n}{\thetah}  \Cfourteen \mleft(\En \nvar\mright)^{\floor{\frac{p-1}2}+1}\NLiD{n}^2 h^p \\
&+k\CMT \CinvVhp \Cprojfn{0} \mleft(\Cboundaryo^{\half} \mleft(\En\nvar\mright)^{\floor{\frac{p-1}2}+1}\NLiD{n} k h^{p} \Nfn{p-1,h,n}{\thetah} + \Cboundaryt^{\half} h \NHokD{\rho}\mright)\Bigg]\nonumber\\
&\quad\quad \Pfn{p-2}\mleft(\NLiD{n}\mright)\mleft(\CFEMotilde h + \CFEMttilde \CAnk (hk)^p\mright)\NLtD{\thetah}\\
&+\NLtDn{\rho}\NLtDn{\thetah}\nonumber\\
%newline
&\leq \Bigg[\mleft(\Cc\mleft(\Cprojfn{0} + 2\Cprojfn{1}\mright)\NLiD{n} + \CMT \CinvVhp \Cprojfn{0} \Cboundaryt^{\half} + \Cprojfn{0}\nvar  \mright)\NHokD{\rho} \\
&+\mleft(  \Cfourteen \mleft(\En \nvar\mright)^{\floor{\frac{p-1}2}+1}\NLiD{n}^2 + \CMT \CinvVhp \Cprojfn{0}\Cboundaryo^{\half} \mleft(\En\nvar\mright)^{\floor{\frac{p-1}2}+1}\NLiD{n}\mright)k^2h^p\Nfn{p-1,h,n}{\thetah}\Bigg]\nonumber\\
&\quad\quad \Pfn{p-2}\mleft(\NLiD{n}\mright)\mleft(\CFEMotilde h + \CFEMttilde \CAnk (hk)^p\mright)\NLtD{\thetah}.\label{eq:nearlythere}
\end{align}
Rearranging \cref{eq:nearlythere} and using \cref{lem:ellprojerr} and the fact that $hk \leq 1$ we have
  \begin{align*}
\NLtD{\thetah}^2&\leq \Bigg[\mleft(\Cc\mleft(\Cprojfn{0} + 2\Cprojfn{1}\mright) + \CMT \CinvVhp \Cprojfn{0} \Cboundaryt^{\half} + \Cprojfn{0}  \mright)\NHokD{\rho} \nonumber\\
&+\mleft(  \Cfourteen  + \CMT \CinvVhp \Cprojfn{0}\Cboundaryo^{\half} \mright)k^2h^p\Nfn{p-1,h,n}{\thetah}\Bigg]\nonumber\\
&\quad\quad \mleft(\En \nvar\mright)^{\floor{\frac{p-1}2}+1}\NLiD{n}^2\Pfn{p-2}\mleft(\NLiD{n}\mright)\mleft(\CFEMotilde h + \CFEMttilde \CAnk (hk)^p\mright)\NLtD{\thetah}.
\end{align*}
  Using Young's inequality \cref{eq:young} to separate out the $\NLtD{\thetah}$ term on the right-hand side and then moving the resulting $\NLtD{\thetah}^2$ term to the left hand side, followed by \cref{eq:simple}, we obtain
\begin{align}
\NLtD{\thetah}&\leq \Bigg[\mleft(\Cc\mleft(\Cprojfn{0} + 2\Cprojfn{1}\mright) + \CMT \CinvVhp \Cprojfn{0} \Cboundaryt^{\half} + \Cprojfn{0}  \mright)\NHokD{\rho}\nonumber \\
&+\mleft(  \Cfourteen  + \CMT \CinvVhp \Cprojfn{0}\Cboundaryo^{\half} \mright)k^2h^p\Nfn{p-1,h,n}{\thetah}\Bigg]\nonumber\\
&\quad\quad \mleft(\En \nvar\mright)^{\floor{\frac{p-1}2}+1}\NLiD{n}^2\Pfn{p-2}\mleft(\NLiD{n}\mright)\mleft(\CFEMotilde h + \CFEMttilde \CAnk (hk)^p\mright)\label{eq:part3penultimate}.
\end{align}
Upon rearranging \cref{eq:part3penultimate} we obtain \cref{eq:ltthetahbound}.

\epf
With all our lemmas proved, we can now prove our main \lcnamecref{thm:fembound}.

\bpf[Proof of \cref{thm:fembound}]
\label{page:thmfembound}
By \cref{lem:higherbound} (with $m=p-1$) in \cref{eq:ltthetahbound} and using the fact that $\NLiD{n}, \En, \nvar \geq 1$, we have
\begin{align}
\NLtD{\thetah}&\leq \mleft(\Cfirst + \Csecond\Chighfn{p-1,2} \nvar^3\NLiD{n}\nmin^{2-p} \mright)\mleft(\En \nvar\mright)^{\floor{\frac{p-1}2}+1}\NLiD{n}^2\nonumber\\
&\quad\quad\Pfn{p-2}\mleft(\NLiD{n}\mright)\mleft(\CFEMotilde h + \CFEMttilde \CAnk (hk)^p\mright)\NHokD{\rho}\nonumber\\
&+\Csecond  \Chighfn{p-1,1} \mleft(\mleft(\En\nvar\mright)^{\half(\floor{\frac{p-1}2}+1)}\NLiD{n}\nmin^{-\frac{p}2}\mright)^{p-1}\mleft(\En \nvar\mright)^{\floor{\frac{p-1}2}+1}\NLiD{n}^2\nonumber\\
&\quad\quad\Pfn{p-2}\mleft(\NLiD{n}\mright)\mleft(\CFEMotilde (hk )^{p+1}+ \CFEMttilde \CAnk h^{2p}k^{2p+1}\mright)\NLtD{\thetah}\label{eq:doublehkdep}
\end{align}
Choosing $h$ according to \cref{eq:hfemcond}, \cref{eq:doublehkdep} simplifies to
\begin{align*}
\NLtD{\thetah} &\leq \mleft(\Cfirst + \Csecond\Chighfn{p-1,2} \nvar^4\nmin^{1-p} \mright)\mleft(\En \nvar\mright)^{\floor{\frac{p-1}2}+1}\NLiD{n}^2\\
&\quad\quad\Pfn{p-2}\mleft(\NLiD{n}\mright)\mleft(\CFEMotilde h + \CFEMttilde \CAnk (hk)^p\mright)\NHokD{\rho} + \half \NLtD{\thetah},
\end{align*}

and therefore it follows that
\beq\label{eq:ltboundwithrho}
\NLtD{\thetah} \leq \nvar^6\nmin^{-(p+1)} \mleft(\En \nvar\mright)^{\floor{\frac{p-1}2}+1} \Pfn{p-2}\mleft(\NLiD{n}\mright)\mleft(\CLtboundo h + \CLtboundt  \CAnk (hk)^p\mright)\NHokD{\rho}.
\eeq

It only now remains to bound the $k$-weighted $H^1$ norm of $\thetah.$ By \cref{lem:h1contdisc,lem:higherbound} (with $m=1$) we have
\begin{align}
\SNHoD{\thetah} &\leq \Amin^{\half} \mleft[\Chighfn{1,1} \mleft(\mleft(\En\nvar\mright)^{\half(\floor{\frac{p-1}2}+1)}\NLiD{n}\nmin^{-\frac{p}2}\mright)k \NLtD{\thetah}\mright.\nonumber\\
&\mleft.\quad\quad+ \Chighmt \nvar^3\NLiD{n} \NHokD{\rho}\mright],\label{eq:sntheta}
\end{align}

and by combining \cref{eq:sntheta,eq:ltboundwithrho} we obtain (as $hk \leq 1$ and $\nvar,\NLiD{n} \geq 1$)
\begin{align}
\SNHoD{\thetah} &\leq
\mleft(\mleft(\En\nvar\mright)^{\half(\floor{\frac{p-1}2}+1)}\NLiD{n}\nmin^{-\frac{p}2}\mright)\nvar^6\nmin^{-(p+1)} \mleft(\En \nvar\mright)^{\floor{\frac{p-1}2}+1}\Pfn{p-2}\mleft(\NLiD{n}\mright)\nonumber\\
&\quad\quad\mleft(\CHoboundo  +\CHoboundt \CAnk k(hk)^p\mright)\NHokD{\rho}.\label{eq:hoboundwithrho}
\end{align}
Therefore by combining \cref{eq:ltboundwithrho,eq:hoboundwithrho}, using \cref{lem:ellprojerr} with $m=1$ and \cref{lem:scottzhang} to bound $\NHokD{\rho}$, and using the fact that $hk \leq 1,$ we obtain \cref{eq:femltbound,eq:femhobound}.
\epf



\subsection{Constants from \cref{sec:fem}}\label{app:constants}
To summarise the constants used in \cref{sec:fem}, we use the following table, where the constants are given in logical order (i.e., constants lower in the table will only depend on constants higher in the table\ednote{This isn't the case yet.}). As well as giving the definitions of the constants, we also state the place (Theorem, Lemma, etc.) where they are defined. If a constant is not defined in terms of other constants, but rather is given in the statement of a Theorem or Lemma, then the `Definition' column is left blank.

Where a constant is only used inside a proof (and not in the statement of a theorem, or similar) then it will typically be numbered using the equation number of its first appearance.

\begin{longtabu}{ccc}
  \toprule
  Constant & Definition & Defined/Introduced\\
  \midrule
  \endhead
  $\CintAl$ & ---& \cite[Theorem 4.16]{Mc:00}\\
  $\CscatAl$ & ---& \cite[Theorem 4.18(i)]{Mc:00}\\
    $\CtruncAl$ & ---& \cite[Theorem 4.18(ii)]{Mc:00}\\
  $\CAl$ & $\CintAl + \CscatAl + \CtruncAl.$ & \cref{thm:shift}\\
  $\Cmclean{m}$&---&\cite[Theorem 3.20]{Mc:00}\\ 
  $\Cprod{m}$ &$1+\Cmclean{m}$&\cref{thm:banachalg} \\
  $\CAnk$ &---& \cref{ass:htwo}\\
  $\Cfg$ &---& \cref{ass:htwo}\\
  $\CTrs$ &---& \cref{thm:trace}\\
  $\CBanfn{s}$ & --- & \cref{thm:banachalg}\\
  $\Cej$ & $\displaystyle \begin{dcases}
  \max\set{1,\CAz} & j=0\\
  \mleft(\max\set{1,\CAz} \prod_{l=1}^j \max\set{1,\CAfn{l}} \mleft(1+\CTrfn{l+1}\mright)\mright)2^{j-1} &j \geq 1
  \end{dcases}$& \cref{thm:expansion}\\
%  $\Pj$ & See \cref{eq:p}& \cref{thm:expansion}\\
%  $\Coscfn{j}$ & See \cref{eq:osc1,eq:osc2,eq:osc3} & \cref{thm:expansion}\\
  $\Cosc$ & $\Cefn{p-1}/\max\set{1,\CAz}$ & \cref{thm:expansion}\\
  $\CSZs$ &---& \cref{lem:scottzhang}\\
  $\Cshiftfn{m}$ &$\displaystyle
  \begin{dcases}
  1/\Amin & m=-1\\
  \CAz & m = 0\\
  2\CAfn{m}\Cprod{m} & m \in [1,d/2]\\
\CAfn{m}\CBanfn{m} & m \in (d/2,p-1]
\end{dcases}
$
& \cref{lem:shiftnegativew}\\
  $\CFEMo$ & $\sum_{j=0}^{p-2}\Cinterpfn{j+2}\Cefn{j} $ & \cref{lem:bestapprox}\\
  $\CFEMt$ & $\CFEMt = \Cinterpfn{p+1}\Cosc$ & \cref{lem:bestapprox}\\
  $\Cmso$ &$\displaystyle
  \begin{dcases}
    \frac{2\NLiDop{A}}{\min\set{1,1/\CP^2}\Amin} & s = -1\\
      \frac{2\NLiDop{A}^2\CSZfn{2}\CAfn{0}}{\min\set{1,1/\CP^2}\Amin} & s = 0\\
  \frac{4\CSZfn{s+2}\CAfn{s}\Cprod{s}\NLiDop{A}^2}{\min\set{1,\frac1{\CP^2}}\Amin} & s \in (1,\frac{d}2]\\
\frac{2\CSZfn{s+2}\CAfn{s}\CBanfn{s}\NLiDop{A}^2}{\min\set{1,\frac1{\CP^2}}\Amin} & s \in (\frac{d}2,p-1]
  \end{dcases}
  $
& \cref{lem:ellprojerr} \\
  $\CinvVhp$ &---& \cref{lem:inverseinequality}\\
  $\Chinv$ & $\CinvVhp \NLiDop{A}^{1/2}$ & \cref{lem:inversediscrete}\\
  $\Cm$ & $\Cwfn{-m,1}  \CAfn{0} \CSZfn{2}$ & Proof of \cref{lem:negdiscsum}\\
  $\Efn{m,t}$ & See \cref{eq:Emt1,eq:Emt2,eq:Emt3} & Proof of \cref{lem:negdiscsum}\\
  $\Etildefn{m,t}$ & See \cref{eq:Etilde1,eq:Etilde3,eq:Etilde4} & Proof of \cref{lem:negdiscsum}\\
  $\Csumj$ &$\displaystyle\max_{m \in \set{0,\ldots,l}}\set{\Efn{m,l},\frac{\mleft(1+\Cfn{1,1}\mright)\NLiDop{A}}{\Amin}\Etildefn{m,l}}$ & Proof of \cref{lem:negdiscsum}\\
  $\Ctildemin$&---&\cref{thm:fembound}\\
  $\Chcond$ & $\displaystyle\frac1{4^{2p}} \mleft(\Csecond\Chighfn{p-1,1}\mright)^{-\frac1{2p}} \min\set{\CFEMo^{-\frac1{2p}},\CFEMt^{-\frac1{p+1}}}$ & \cref{thm:fembound}\\
  $\CFEMLt$ &$\Cfn{-1,1} \max\set{\CLtboundo,\CLtboundt}$&\cref{thm:fembound}\\
  $\CFEMHo$ &$\Cfn{-1,1}\max\set{\CHoboundo,\CHoboundt}$&\cref{thm:fembound}\\
  $\Cmess$ &$\Csumpmo \mleft(\sum_{m=0}^{p-1} \CSZfn{(p-1-m)} \Cwfn{0} + \Cwfn{-(p-1-m)}\mright)$&Proof of \cref{lem:boundarybound}\\
  $\CMT$ &---&\cref{thm:multiplicativetrace}\\
  $\Cboundaryo$ & $\Cmess^2$ & \cref{lem:boundarybound} \\
  $\Cboundaryt$ & $1 + \CMT^2 \Cfn{0,1}$ & \cref{lem:boundarybound}\\
  $\CBo$&$\CMT \CinvVhp^{1/2} \Chinv^{m-1} \Cboundaryo^{\half} \Chinv^{p-m}$&Proof of \cref{lem:higherbound}\\
  $\CBt$&$\CMT \CinvVhp^{1/2} \Chinv^{m-1} \Cboundaryt^{\half}$&Proof of \cref{lem:higherbound}\\
  $\CBth$&$\CMT^2 \CinvVhp^{\half} \Chinv^{m-1}\Cfn{0,1}^{\half}$&Proof of \cref{lem:higherbound}\\
  $\CReco$&$\displaystyle\sqrt{\frac32} + \CBo^{\half} + \frac{\CBt^{\half}}{\sqrt{2}}+ \frac{\CBth^{\half}}{\sqrt{2}}$&Proof of \cref{lem:higherbound}\\
  $\CRectfn{m}$&$\displaystyle\frac1{\sqrt{2}}\Chinv^{m-1}\Cwfn{0}\mleft(1+\CSZfn{0}\mright)+1$&Proof of \cref{lem:higherbound}\\
  $\Chighmo$ & 2$\CReco^m$&Proof of \cref{lem:higherbound}\\
  $\Chighmt$ &$\sum_{j=1}^{m}\CReco^{m-j} \CRectfn{j}$&Proof of \cref{lem:higherbound}\\
  $\Cfourteen$ &$\Csumfn{p-1}\sum_{m=0}^{p-1} \mleft(\CSZfn{p-1-m}\Cfn{0,1} + \Cwfn{p-1-m} \Cfn{-1,1}\mright)$&Proof of \cref{lem:ltthetahbound}\\
  $\Cfirst$ &$\Cc\mleft(\Cfn{0,1} + 2\Cfn{1,1}\mright) + \CMT \CinvVhp \Cfn{0,1} \Cboundaryt^{\half} + \Cfn{0,1}$&Proof of \cref{lem:ltthetahbound}\\
    $\Csecond$ &$\Cfourteen  + \CMT \CinvVhp \Cfn{0,1}\Cboundaryo^{\half}$&Proof of \cref{lem:ltthetahbound}\\
  $\CLtboundo$&$ 2\mleft(\Cfirst + \Csecond\Chighfn{p-1,1}\mright)\CFEMo$&Proof of \cref{thm:fembound}\\
  $\CLtboundt$&$2\mleft(\Cfirst + \Csecond\Chighfn{p-1,1}\mright)\CFEMt$&Proof of \cref{thm:fembound}\\
  $\CHoboundo$&$\Amin^{\half}\Chighfn{1,1}\CLtboundo+\Chighmt$&Proof of \cref{thm:fembound}\\
  $\CHoboundt$&$\Amin^{\half}\Chighfn{1,1}\CLtboundt$&Proof of \cref{thm:fembound}\\
  $\CFEMLt$&$\mleft(\Cfn{0,1} + \Cfn{-1,1}\mright) \max\set{\CLtboundo,\CLtboundt}$&\Cref{thm:fembound}\\
  $\CFEMHo$&$\mleft(\Cfn{0,1} + \Cfn{-1,1}\mright)\max\set{\CHoboundo+\CLtboundo,\CHoboundt+\CLtboundt}$&\Cref{thm:fembound}\\
  $\CcorLt$&$4\CFEMLt\max\set{\CFEMo,\CFEMt}$&\Cref{cor:fembound}\\
  $\CcorHo$&$8\CFEMHo\max\set{\CFEMo,\CFEMt}$&\Cref{cor:fembound}\\
  $\Cc$&$\displaystyle2\max\set{\NLiDop{A},1,\frac{\CMT^2}2}$&\cref{lem:continuity}\\
\bottomrule
\end{longtabu}


%\input{discrete-sobolev-spaces}

%\input{new-fem-bounds}

\chapter{Well-posedness of Formulations of the Stochastic Helmholtz Equation}\label{chap:stochastic}
\chaptermark{The Stochastic Helmholtz Equation}
\section{Introduction}\label{sec:intro}
The goals of this paper are to prove results on the well-posedness of variational formulations of the stochastic Helmholtz equation
\beq\label{eq:hh-intro}
\grad\cdot\mleft(A(\omega)\grad u(\omega)\mright) + k^2n(\omega)u(\omega) = -f(\omega),
\eeq
as well as a priori bounds on its solution that are explicit in the wavenumber $k$ and the material coefficients $A$ and $n.$


We consider \eqref{eq:hh-intro} with physical domain either $\RRd, \, d=2,3,$ or $\RR^d\setminus\Dmclos,$ where $\Dm$ (referred to as the \defn{obstacle}) is a bounded, Lipschitz, open set such that $\RRd\setminus\Dmclos$ is connected, and

\bit
\item $\omega$ is an element of the underlying probability space,
\item $A$ is a symmetric-positive-definite matrix-valued random field such that $\supp(I-A)$ is compact,
\item $n$ is a positive real-valued random field such that $\supp(1-n)$ is compact,
\item $f$ is a real-valued random field such that $\supp f$ is compact, and
  \item $k>0$ is the wavenumber,
  \eit
and we are particularly interested in the case where the wavenumber $k$ is large.

\paragraph{Motivation} The motivation for establishing well-posedness and proving a priori bounds on the solution of \eqref{eq:hh-intro} is the growing interest in Uncertainty Quantification (UQ) for the Helmholtz equation; see e.g.~\cite{XiSh:07,TsXiYi:11,BuGh:14,GaHa:15,FeLiLo:15,FeLiNi:18,LiWaZh:18,HiScScSc:15,BaCaHaZh:18}. (In this PDE context, by `UQ' we mean theory and algorithms for computing statistics of quantities of interest involving PDEs \emph{either} posed on a random domain \emph{or} having random coefficients.) There is a large literature on UQ for the stationary diffusion equation
\beq\label{eq:diffusion}
-\grad\cdot (\kappa(\omega) \grad u(\omega))=f(\omega),
\eeq
due in part to its large number of applications (e.g.~in modelling groundwater flow), and a priori bounds on the solution are vital for the rigorous analysis of UQ algorithms; see e.g.~\cite{BaTeZo:04,BaNoTe:07,Gi:10,MuSt:11,ChScTe:13}. In contrast, whilst \eqref{eq:hh-intro} has many applications (e.g.~in geophysics and electromagnetics), there is much less rigorous theory of UQ for the Helmholtz equation. The main reason for this is that the (deterministic) PDE theory of \eqref{eq:hh-intro} when $k$ is large is much more complicated that the analogous theory for \eqref{eq:diffusion}.
 
 \paragraph{Related previous work} To our knowledge, the only work that considers \eqref{eq:hh-intro} with large $k$ and attempts to establish either (i) well-posedness of variational formulations or (ii) a priori bounds is \cite{FeLiLo:15}, which considers both (i) and (ii) for \eqref{eq:hh-intro} posed in a bounded domain with an impedance boundary condition. We discuss the results of \cite{FeLiLo:15} further in \cref{sec:otherwork}, but we highlight here that (a) \cite{FeLiLo:15} considers $A=I$ and $n=1+\eta,$ with $\eta$ random and the magnitude of $\eta$ decreasing with $k,$  whereas we consider classes of $A$ and $n$ that allow $k$-independent random perturbations, and (b) in its well-posedness result, \cite{FeLiLo:15}  invokes Fredholm theory to conclude existence of a solution, but this relies on an incorrect assumption about compact inclusion of Bochner spaces---see \cref{sec:federico} below. In \cref{sec:otherwork} we also discuss the papers \cite{BuGh:14,JeSc:16,JeScZe:17,HiScScSc:15} on the theory of UQ for either \eqref{eq:hh-intro} or the related time-harmonic Maxwell's equations; in these papers either the $k$-explicit well-posedness is not a primary concern or $k$ is assumed to be small. Our hope is that the results in the present paper can be used in the rigorous theory of UQ for Helmholtz problems with large $k.$
 
\paragraph{The contributions of this paper} The main results in this paper, \cref{thm:hh-gen,thm:hh-hetero} below, concern well-posedness and a priori bounds for the solutions of various formulations of the stochastic Helmholtz equation; these formulations include those used in sampling-based UQ algorithms (\cref{prob:msedp,prob:somsedp} below) and in the stochastic Galerkin method (\cref{prob:svsedp} below). These are the first such results for arbitrarily large $k$ and for $A$ and $n$ varying independently of $k$. These results are proved by combining:
\ben
\item bounds for the Helmholtz equation in \cite{GrPeSp:19} with $A$ and $n$ deterministic but spatially-varying, with
\item general arguments (i.e.~not specific to Helmholtz) presented here for proving a priori bounds and well-posedness of variational formulations of linear elliptic SPDEs.
\een
Regarding 1: the $k$-dependence of the bounds on $u$ in terms of $f$ depends crucially on whether or not $A$, $n$, and $\Dm$ are such that there exist trapped rays. In the trapping case, the solution operator can grow exponentially in $k$ (see \cite{Ra:71,Bu:98,PoVo:99a,CaPo:02,Be:03} and \cite[Section 2.5]{BeChGrLaLi:11}, and the reviews in \cite[Section 6]{MoSp:19}, \cite[Section 1.1]{ChSpGiSm:17}, and \cite[Section 1]{GrPeSp:19}); in contrast, in the nontrapping case, the solution operator is bounded uniformly in $k$ (see \cite{Va:75,MeSj:78,Bu:02}). The bounds in \cite{GrPeSp:19} are under conditions on $A,n,$ and $\Dm$ that ensure nontrapping of rays; the significance of these bounds is that they are the first (deterministic) bounds for the Helmholtz scattering problem in which both $A$ and $n$ vary and the bounds are explicit in $A$ and $n$ (as well as in $k$). This feature of being explicit in $A$ and $n$ is crucial in allowing us to prove the results in this paper when $A$ and $n$ are random fields.

Regarding 2: the main reason these general arguments are needed is the fact that the variational formulations of both the deterministic and the stochastic Helmholtz equation are not coercive, and so one cannot use the Lax--Milgram theorem to conclude well-posedness and an a priori bound.  In the deterministic case, the remedy for the lack of coercivity of the Helmholtz equation is to use Fredholm theory, but this is \emph{not} applicable to the stochastic variational formulation of the Helmholtz equation because the necessary compactness results do not hold in Bochner spaces (see \cref{sec:federico} below). Our solution to this lack of coercivity and failure of Fredholm theory is to use well-posedness results and bounds from the deterministic case to prove results for the stochastic case. We work `pathwise' by integrating the deterministic results over probability space, identifying conditions under which the necessary quantities are indeed integrable. Our approach is given in a general framework that, given (i) deterministic well-posedness results and a priori bounds that are explicit in all the coefficients, and (ii) measurability and integrability conditions on the stochastic quantities, returns corresponding well-posedness results, a priori bounds, and equivalence results for different formulations of the stochastic problem. One reason we state our well-posedness results in general (i.e.~not only in the specific case of the Helmholtz equation) is that we expect that they can be used in the future to prove well-posedness results for the time-harmonic Maxwell's equations in random media. A nontechnical summary of the ideas behind our general well-posedness results is given in \cref{rem:nontechnical} below. Some of these results are similar in spirit to the results about the PDE \eqref{eq:diffusion} in \cite{Gi:10,MuSt:11} (which deal with the failure of Lax--Milgram for the stochastic variational problem for \eqref{eq:diffusion} in the case when the coefficient $\kappa$ is not uniformly bounded above and below), and our general arguments use some of the ideas and technical tools from these two papers. 
 

\subsection{Statement of main results}\label{sec:hh-results}

\paragraph{Notation and basic definitions}Let either (i) $\Dm \subset \RRd,$ $d=2,3,$ be a bounded Lipschitz open set such that $\bzero \in \Dm$ and the open complement $\Dp\de \RR^d\setminus \overline{\Dm}$ is connected, or (ii) $\Dm = \emptyset.$ Let $\GD = \partial \Dm.$ 
Fix $R>0$ and let $\BR$ be the ball of radius $R$ centred at the origin. Define $\GR := \partial \BR$ and $\DR \de \Dp \cap \BR$ (see \cref{fig:domain}). Let $\gamma$ denote the trace operator from $\DR$ to $\partial \DR = \GD \cup \GR$ and define $\HozDDR \de \set{v \in \HoDR \st \gamma v = 0 \ton \GD}.$ 
 
Let $\TrR: H^{1/2}(\Gamma_R) \rightarrow H^{-1/2}(\Gamma_R)$ be the Dirichlet-to-Neumann map for the deterministic equation $\Delta u+k^2 u=0$ posed in the exterior of $\BR$ with the Sommerfeld radiation condition 
\beq\label{eq:src}
\frac{\partial u}{\partial r}(\bx) - \ri ku(\bx) = o\mleft(\frac1{r^{(d-1)/2}}\mright) \text{ as } r\de\abs{\bx}\rightarrow \infty, \text{ uniformly in } \frac{\bx}{\abs{\bx}};
\eeq
see \cite[Section 2.6.3]{Ne:01} and \cite[Equations 3.5 and 3.6]{ChMo:08} for an explicit expression for $\TrR$  in terms of Hankel functions and Fourier series ($d=2$)/spherical harmonics ($d=3$). Let $\IPGR{\cdot}{\cdot}$ be the duality pairing on $\GR$ between $\HmhGR$ and $\HhGR$ and write $\dd\Leb$ for Lebesgue measure.

Let $\LiDpRRdtd$ be the set of all matrix-valued functions $A:\Dp\rightarrow\RRdtd$ such that $A_{i,j} \in \LiDpRR$ for all $i,j = 1,\ldots,d.$ Where the range of functions is $\CC$ we suppress the second argument in a function space, e.g.~we write $\LiDp$ for $\LiDpCC.$ We write $\Do \compcont \Dt$ if $\Do$ is a compact subset of the open set $\Dt.$ Let $\OFP$ be a complete probability space. Throughout this paper, unless stated otherwise we equip a topological space with its Borel $\sigma$-algebra. See \cref{app:mtBs} for a summary of the measure-theoretic concepts used in this paper. Let
\bit
\item $f:\Omega\rightarrow\LtDp$ be such that $\supp f \compcont \BR$ almost surely
\item $n:\Omega\rightarrow \LiDpRR$ be such that $\supp(1-n) \compcont \BR$ almost surely and there exist $\nmin, \nmax:\Omega\rightarrow \RR$ such that
%\beqs
$0 < \nmin(\omega) \leq n(\omega)(\bx) \leq \nmax(\omega)$
%\eeqs
for almost every $\bx \in \Dp$ almost surely, and
\item $A:\Omega\rightarrow\LiDpRRdtd$ be such that $\supp(I-A) \compcont \BR,\, A_{ij} = A_{ji}$ almost surely, and there exist $\Amin,\Amax:\Omega \rightarrow \RR$ such that $0 < \Amin(\omega) < \Amax(\omega)$ almost surely and
%\beqs
$\Amin(\omega)\abs{\bxi}^2 \leq \big(A(\omega)(\bx)\bxi\big)\cdot\bxi \leq \Amax(\omega)\abs{\bxi}^2$
%\eeqs
for almost every $\bx \in \Dp$ and for all $\bxi \in \CCd$ almost surely.
\eit
If $v:\Omega \rightarrow Z$ for some function space $Z$ of functions on $\RRd,$ we abuse notation slightly and write $v(\omega,\bx)$ instead of $v(\omega)(\bx).$


\def\domainscale{2}

\def\obsptone{(1.0, 0.0)}

\def\obspttwo{(0.7071067811865476, 0.7071067811865475)}

\def\obsptthree{(0.0, 1.0)}

\def\obsptfour{(-0.7071067811865475, 0.7071067811865476)}

\def\obsptfivey{0.0}

\def\obsptfive{(-1.0,\obsptfivey)}

\def\obsptsixx{-0.7071067811865477}

\def\obsptsix{(\obsptsixx, -0.7071067811865475)}

\def\obsptsevenx{0.0}

\def\obsptseveny{-1.0}

\def\obsptseven{(\obsptsevenx,\obsptseveny)}

\def\obseightptx{0.7071067811865474}

\def\obspteighty{-0.7071067811865477}

\def\obspteight{(\obseightptx,\obspteighty )}

\def\obslineone{\obsptone .. controls (1.0,0.3) and (1.0071067811865476, 0.7071067811865475) ..}

\def\obslinetwo{\obspttwo .. controls (0.4071067811865476, 0.7071067811865475) and (0.2, 0.8) ..}

\def\obslinethree{\obsptthree .. controls (-0.2,1.2) and (-0.5071067811865475, 0.9071067811865476) ..}

\def\obslinefour{\obsptfour .. controls (-0.9071067811865475, 0.5071067811865476) and (-1.2,0.2) ..}

\def\obslinefive{\obsptfive .. controls (-0.8,-0.2) and (-0.7071067811865477, -0.5071067811865475) ..}

\def\obslinesix{\obsptsix.. controls (-0.7071067811865477, -0.9071067811865475) and (-0.2,-0.8) ..}

\def\obslineseven{\obsptseven .. controls (0.2,-0.6) and (0.4071067811865474, -0.6071067811865477) .. }% The corner at the start of this line is the one where there's a 'proper' corner

\def\obslineeight{\obspteight .. controls (1.0071067811865474, -0.8071067811865477) and (1.0,-0.2) ..}

\def\obstacle{    \obslineone
     \obslinetwo
    \obslinethree
    \obslinefour
    \obslinefive
     \obslinesix
    \obslineseven
    \obslineeight
    cycle
}

\def\truncation{
    (3.5, 0.0) --
    (1.081559480312316, 3.3286978070330373) .. controls (-0.581559480312316, 2.3) ..
    (-2.8315594803123156, 2.0572483830236563) .. controls (-3.8315594803123156, 0.0572483830236563) and (-3.8315594803123165, -0.0572483830236554) ..
    (-2.8315594803123165, -2.0572483830236554) .. controls (-2.3315594803123165, -3.0572483830236554) and (-0.5815594803123152, -3.4286978070330377) ..
    (1.0815594803123152, -3.3286978070330377) --
    cycle
    }

\def\suppf{
(1.2,\obsptseveny) -- (1.2,1.5) -- (-1.5,1.5) -- (-1.5,\obsptfivey) --
\obslinefive
\obslinesix
\obsptseven --
cycle
}

\def\suppA{
%\obsptthree .. controls (3.0,3.0) and (-3.0,-6.0) ..  \obsptsix --
\obsptthree arc [start angle=90, end angle=-149, radius=1.5]
--\obsptsix
\obslinesix
\obslineseven
\obslineeight
\obslineone
\obslinetwo
\obsptthree
}

\def\ncontone{(-1.2,0.5)}

\def\suppn{
(-1.45,-0.6) ellipse [x radius = 1.0,y radius=0.6,rotate=-65]
%% (-1.8,0.5) .. controls \ncontone and (-0.9,0) ..
%% (-1.0,-1.5) .. controls (-1.5,-1.0) and (-1.5,-1.0) ..
%% (-1.4,-0.5) .. controls (-1.3,0.0) and () ..
%% cycle
}


    % Patterns for filling
    \pgfdeclarepatternformonly{owennortheast}
    {\pgfpointorigin}{\pgfpoint{1cm}{1cm}}
    {\pgfpoint{1cm}{1cm}}
    {
    \pgfpathmoveto{\pgfpointorigin}
    \pgfpathlineto{\pgfpoint{1cm}{1cm}}
    \pgfsetlinewidth{0.5\pgflinewidth}
    \pgfusepath{stroke}
    }
    

     \pgfdeclarepatternformonly{owennorthwest}
    {\pgfpointorigin}{\pgfpoint{1cm}{1cm}}
    {\pgfpoint{1cm}{1cm}}
    {
    \pgfpathmoveto{\pgfpoint{1cm}{0cm}}
    \pgfpathlineto{\pgfpoint{0cm}{1cm}}
   \pgfsetlinewidth{0.5\pgflinewidth}
    \pgfusepath{stroke}
    }

\pgfdeclarepatternformonly{owennorthwestsmall}
    {\pgfpoint{-1cm}{-1cm}}{\pgfpoint{1cm}{1cm}}
    {\pgfpoint{0.5cm}{0.5cm}}
    {
    \pgfpathmoveto{\pgfpoint{1cm}{0cm}}
    \pgfpathlineto{\pgfpoint{0cm}{1cm}}
   \pgfsetlinewidth{0.5\pgflinewidth}
    \pgfusepath{stroke}
    }
    
       \pgfdeclarepatternformonly{owennortheastsmall}
    {\pgfpoint{-1cm}{-1cm}}{\pgfpoint{1cm}{1cm}}
    {\pgfpoint{0.5cm}{0.5cm}}
    {
    \pgfpathmoveto{\pgfpointorigin}
    \pgfpathlineto{\pgfpoint{1cm}{1cm}}
    \pgfsetlinewidth{0.5\pgflinewidth}
    \pgfusepath{stroke}
    }

       \pgfdeclarepatternformonly{owengridsmall}
    {\pgfpoint{-1cm}{-1cm}}{\pgfpoint{1cm}{1cm}}
    {\pgfpoint{0.5cm}{0.5cm}}
    {
    \pgfpathmoveto{\pgfpoint{0.5cm}{0cm}}
    \pgfpathlineto{\pgfpoint{0.5cm}{1cm}}
    \pgfsetlinewidth{0.5\pgflinewidth}
    \pgfusepath{stroke}
    \pgfpathmoveto{\pgfpoint{0cm}{0.5cm}}
    \pgfpathlineto{\pgfpoint{1cm}{0.5cm}}
    \pgfsetlinewidth{0.5\pgflinewidth}
    \pgfusepath{stroke}
    }


\begin{tikzpicture}[even odd rule]   
\draw[scale=\domainscale] % Truncation
\truncation
\obstacle
;
    

\draw[scale=\domainscale] % The obstacle
\obstacle
;

\draw[scale=\domainscale,dashed] (0,0) circle [radius=2.25]; % The ball


\filldraw[pattern=owengridsmall,scale=\domainscale] % Support of f
\suppf
\obstacle
;
% support of 1-n
\filldraw[pattern=owennortheastsmall,scale=\domainscale]
\suppn
;

% support of I-A
\filldraw[pattern=owennorthwestsmall,scale=\domainscale]
\suppA
;

% The labels
\draw (-0.1,0) node[fill=white] {$\Dm$};
\draw (-3.5,-4.75) node[fill=white] {$D:=\Dtilde\setminus \clos{\Dm}$};

\draw (2.0,-0.5) node[fill=white] {$\GD$};
\draw[scale=\domainscale] (2.418221983093764, 0.6211657082460498) node[fill=white] {$\BR$};
\draw (5.1,3.1) node[fill=white] {$\GI$};

\draw (-1.8,2.3) node[fill=white] {$\supp f$};

\draw (0,-2.8) node[fill=white] {$\supp(I-A)$};

\draw (-2.9,-1.2) node[fill=white] {$\supp(1-n)$};
\end{tikzpicture}


\paragraph{Variational Formulations} We consider three different formulations of the  \emph{Helmholtz stochastic exterior Dirichlet problem} (stochastic EDP); \cref{prob:msedp,prob:somsedp,prob:svsedp} below.

Define the sesquilinear form $a(\omega)$ on $\HozDDR \times \HozDDR$ by
\beq\label{eq:SEDPa}
\mleft[a(\omega)\mright]\mleft(\vo,\vt\mright)\de\int_{D_R}\Big( \mleft( A(\omega) \grad \vo\mright)\cdot \grad \vtb 
 - k^2 n(\omega)\, \vo\,\vtb \Big)\dd\Leb- \big\langle T_R \gamma \vo,\gamma \vt\big\rangle_{\Gamma_R},
 \eeq
 and the antilinear functional $L(\omega)$ on $\HozDDR$ by
\beq\label{eq:SEDPL}
\mleft[L(\omega)\mright](\vt)\de \int_{D_R} f(\omega)\, \vtb\,\dd\Leb.
\eeq
Define the sesquilinear form $\as$ on $L^2\big(\Omega;H_{0,D}^1(D_R)\big)\times L^2\big(\Omega;H_{0,D}^1(D_R)\big)$ and the antilinear functional $\Ls$ on $L^2\big(\Omega;H_{0,D}^1(D_R)\big)$ by 
\beq\label{eq:SEDPas}
\as\mleft(\vo,\vt\mright)\de \int_\Omega \mleft[a(\omega)\mright]\mleft(\vo(\omega),\vt(\omega)\mright)\dd\PP(\omega)
\quad\text{ and } \quad
%\eeq
%
%\beq
%\label{eq:SEDPLs}
\Ls(\vt)\de \int_\Omega \mleft[L(\omega)\mright]\mleft(\vt(\omega)\mright)\dd\PP(\omega).
\eeq
We consider the following three problems:

\bprobvar{1}[Measurable EDP almost surely]\label{prob:msedp}
Find a measurable $u:\Omega\rightarrow\HozDDR$ such that
\vspace{-2ex}
\beqs
\mleft[a(\omega)\mright]\mleft(u(\omega),v\mright) = \mleft[L(\omega)\mright](v) \tforall v \in \HozDDR \text{ almost surely.}
\eeqs
\eprobvar

\bprobvar{2}[Second-order EDP almost surely]\label{prob:somsedp}
Find $u\in L^2\big(\Omega;H_{0,D}^1(D_R)\big)$ such that
\beqs
\mleft[a(\omega)\mright]\mleft(u(\omega),v\mright) = \mleft[L(\omega)\mright](v) \tforall v \in \HozDDR \text{ almost surely.}
\eeqs
\eprobvar

\bprobvar{3}[Stochastic variational EDP]\label{prob:svsedp}
Find $u\in L^2\big(\Omega;H_{0,D}^1(D_R)\big)$ such that
\beqs
\as(u,v) = \Ls(v) \tforall v \in \LtOHozDDR.
\eeqs
\eprobvar

%\bre[The relationship between \cref{prob:msedp,prob:somsedp,prob:svsedp}]\label{rem:lit}

\Cref{prob:somsedp} is the foundation of sampling-based UQ methods, such as Monte-Carlo and Stochastic-Collocation methods; its analogue for the stationary diffusion equation is well-studied in, e.g., \cite{XiHe:05,BaNoTe:07,NoTeWe:08a,Ch:12,ChScTe:13,TeJaWeGu:15,KuNu:16,HeLaSc:18}. Similarly \cref{prob:svsedp} is the foundation of the Stochastic Galerkin method (a finite element method in $\Omega \times D,$ where $D$ is the spatial domain), and is studied for the Helmholtz Interior Impedance Problem in \cite{FeLiLo:15}, and its analogue for the stationary diffusion equation is considered in, e.g., \cite{BaTeZo:04,KhSc:11,BaScZo:11,GuWeZh:14}.

%The prevalance of \cref{prob:somsedp,prob:svsedp} in the literature therefore raises two questions:
%\ben
%\item\label{q:whyone} Why consider \cref{prob:msedp}?
%\item\label{q:wpe} When are the well-posedness and equivalence of \cref{prob:msedp,prob:somsedp,prob:svsedp} non-trivial to establish?
%\een
%
%We address the first question in the next remark. Regarding the second question, in the stationary diffusion case when the coefficient $\kappa$ in \eqref{eq:diffusion} is uniformly bounded above and below on $\Omega \times D,$ the bilinear form for the analogue of \cref{prob:svsedp} is coercive on $\Omega \times D,$ and so we can obtain well-posedness and an a priori bound on the solution using the Lax--Milgram theorem. As we show in the general framework below, one can then obtain similar results for \cref{prob:somsedp,prob:msedp} under mild regularity and integrability assumptions. However, when the coefficient $\kappa$ is not bounded above and below on $\Omega \times D$ (as in the well-studied case of a lognormal random field), one can only show well-posedness and an a priori bound for the analogue of \cref{prob:somsedp} using a pathwise approach (as in \cite{ChScTe:13}) and for the analogue of \cref{prob:svsedp} by modifiying the functions spaces used, as in \cite{Gi:10,MuSt:11}.
%
%In the Helmholtz case, even when the coefficients $A$ and $n$ are bounded uniformly above and below, the sesquilinear form $\as$ in \cref{prob:svsedp} is not coercive on $\Omega\times \DR$, so one cannot use the Lax--Milgram theorem, and Fredholm theory (which can be used in the deterministic case to conclude existence and uniqueness) is not applicable, because the Bochner space $\LtOHoDR$ is not compactly contained in $\LtOLtDR$---see \cref{sec:federico}. Therefore, currently the only way to obtain well-posedness results and a priori bounds on the solution of \cref{prob:svsedp} is to use the results in this paper. We use a pathwise approach, imposing conditions on $A$ and $n$ such that we can construct a solution to \cref{prob:msedp}, and then showing this solution is also a solution of \cref{prob:somsedp,prob:svsedp} with the necessary properties. 
%\ere

\bre[Why consider \cref{prob:msedp}?]\label{rem:whyone}

The difference between \cref{prob:msedp,prob:somsedp} is that \cref{prob:msedp} requires no integrability of $u$ over $\Omega$, whereas \cref{prob:somsedp} requires $u \in \LtOHozDDRnormal$. Since all the theory for sampling-based UQ methods assume some integrability of the solution, the natural question is: why consider \cref{prob:msedp} at all?

The main reason we consider \cref{prob:msedp} is that, given the existing PDE theory for the Helmholtz equation, we can prove existence of a solution to \cref{prob:msedp} under general conditions on $A$ and $n$, but there is no current prospect of proving existence of a solution to \cref{prob:somsedp} under general conditions on $A$ and $n$. The explanation for this consists of the following three points:
\ben
\item The only two known ways to obtain a solution to \cref{prob:somsedp} are: (i) obtain a deterministic a priori bound, explicit in all parameters, and integrate (followed, e.g., in \cite{ChScTe:13} for \eqref{eq:diffusion} with lognormal coefficients) and (ii) obtain a solution to \cref{prob:svsedp} and show this is a solution to \cref{prob:somsedp}. In the Helmholtz case, doing (ii) is difficult as neither the Lax--Milgram theorem nor Fredholm theory is applicable (as explained in the introduction), and so we follow the approach in (i).
\item The only known bounds on the solution of the Helmholtz equation explicit in all parameters are those recently obtained for nontrapping scenarios in \cite{GrPeSp:19,GaSpWu:18}.
\item Obtaining a bound explicit in all parameters for a general class of $A$ and $n$, e.g., $A \in \WoiDRRRdtd$ and $n \in \LiDRRR$ is well beyond current techniques. Indeed, a general class of $A$ and $n$ will include both trapping and nontrapping scenarios, and such a bound would need to capture the exponential blow-up in $k$ for trapping $A$ and $n$, the uniform boundedness in $k$ for nontrapping $A$ and $n$, and be explicit in $A$ and $n$.
\een
Given this fact that there is no current prospect of proving existence of a solution to \cref{prob:somsedp} under general conditions on $A$ and $n$ we keep \cref{prob:msedp} so that we prove an (albeit weaker) existence result for the Helmholtz equation with general coefficients.


%or Fredholm theory are not
%
% We can show that $u \in \LtOHozDDRnormal$ by obtaining an a priori bound in $\HozDDR$ on the solution of the corresponding deterministic problem, where the bound is explicit in all of the stochastic parameters, checking conditions under which this bound can be integrated over $\Omega,$ and then integrating this deterministic bound over $\Omega$ to obtain a bound in $\LtOHozDDRnormal.$ This is precisely the approach followed in \cite{ChScTe:13} to obtain well-posedness results and an a priori bound for the stationary diffusion equation with a lognormal coefficient.
%
%However, in the Helmholtz case, as mentioned in \cref{sec:intro}, establishing an a priori bound on the corresponding deterministic problem that is explicit in all of the parameters of interest (such as $A$ and $n$) is difficult---the only such bounds where $A$ and $n$ both vary are those recently obtained in \cite{GrPeSp:19}, where these bounds are obtained under certain conditions on $A$ and $n$. Therefore, one can only prove that a solution of \cref{prob:somsedp,prob:svsedp} exists for small classes of $A$ and $n$---see \cref{thm:hh-hetero} below for such a result. Hence, given all the existing PDE theory for the Helmholtz equation, one \emph{cannot} prove \emph{existence} results for \cref{prob:somsedp,prob:svsedp} for general coefficients, including so-called trapping coefficients; \cref{prob:msedp} is the \emph{only} problem for which one can prove existence and uniqueness---see \cref{thm:hh-gen} below for this result.
%
%Moreover, \cref{prob:msedp} (or rather, its more general analogue \cref{prob:meas} in \cref{sec:notdef} below) can be used as a stepping stone to proving well-posedness results and a priori bounds for the more general analogues of \cref{prob:somsedp,prob:svsedp} (\cref{prob:lt,prob:svar} below). If one can show that a solution of \cref{prob:meas} exists (e.g., by constructing a solution pathwise), and that \cref{con:coeffstoform,con:A,con:coeffstofunc,con:L,con:cborel,con:C,con:B,con:K} in \cref{sec:cons} below hold, then one can show well-posedness and equivalence of all of \cref{prob:meas,prob:lt,prob:svar} and prove an a priori bound on the solution. See \cref{fig:ladder} for a summary of these results.
\ere

\bre[Measurability of $u$ in \cref{prob:msedp}]
It is natural to construct  the solution of \cref{prob:msedp} pathwise; that is, one defines $u(\omega)$ to be the solution of the deterministic problem with coefficients $A(\omega)$ and $n(\omega).$ However, is it then not obvious that $u$ is measurable.
In the proof of \cref{thm:hh-gen} below, we show that the measurability of $u$ follows from
\ben
\item a natural condition on the measurability of the coefficients and data (\cref{con:cborel} below), and 
\item the continuity of the map taking the coefficients of the deterministic PDE to the solution of the deterministic PDE (see \cref{lem:solcont} below).
  \een
\ere


In \cref{thm:hh-gen,thm:hh-hetero} we prove results on the well-posedness of \cref{prob:msedp,prob:somsedp,prob:svsedp} under conditions on $A,$ $n,$ $f,$ and $\Dm.$ Although $A,n,$ and $f$ are defined on $\Dp,$ since $\supp(I-A),$ $\supp(1-n),$ and $\supp f$ are compactly contained in  $\DR$ we can consider $A,n,$ and $f$ as functions on $\DR.$

\bcon[Regularity and stochastic regularity of $f,$ $A,$ and $n$]\label{con:hh-fAn}
The random fields $f, A,$ and $n$ satisfy $f \in \LtOLtDR,$  $A:\Omega \rightarrow \WoiDRRRdtd$ with $A \in \LiOLiDRRRdtd,$ and $n \in \LiOLiDRRR.$
\econ

\bth[Equivalence of variational problems]\label{thm:hh-gen}
Under \cref{con:hh-fAn}:
\bit
\item The maps $\as$ and $\Ls$ (defined by \eqref{eq:SEDPas}) are well-defined.
\item $u \in %\LtOHozDDR$ 
L^2\big(\Omega;H_{0,D}^1(D_R)\big)$
solves \cref{prob:somsedp} if and only if $u$ solves \cref{prob:svsedp}.
\item If $u \in
L^2\big(\Omega;H_{0,D}^1(D_R)\big)$
%\in \LtOHozDDR$ 
solves \cref{prob:somsedp}, then any member of the equivalence class of $u$ solves \cref{prob:msedp}.
\item The solution of \cref{prob:msedp} exists and is unique up to modification on a set of measure zero in $\Omega.$
\item The solution of \cref{prob:somsedp,prob:svsedp} is unique in $L^2\big(\Omega;H_{0,D}^1(D_R)\big)$. %$\LtOHozDDR.$
\eit
\enth

Observe that the only relationship between formulations not proved in \cref{thm:hh-gen} is: if $u:\Omega\rightarrow \HozDDR$  solves \cref{prob:msedp} then $u \in 
L^2\big(\Omega;H_{0,D}^1(D_R)\big)$
%\LtOHozDDR$ 
and $u$ solves \cref{prob:somsedp}. \cref{thm:hh-hetero} below includes this relationship, but we need additional assumptions on $A,n,$ and $\Dm.$

\bde[A particular class of (deterministic) nontrapping coefficients]\label{def:hh-nontrapping}
Let $\muo,\mut >0,$ $\Az \in \WoiDRRRdtd$ with $\supp(I-\Az) \compcont \BR$, and $\nz \in \WoiDRRR$ with $\supp(1-\nz) \compcont \BR.$ We write $\Az \in \NTA{\muo}$ and $\nz \in \NTn{\mut}$ if
\beq\label{eq:hh-Acond}
\Az(\bx) - \mleft(\bx\cdot\grad\mright)\Az(\bx) \geq \muo \quad \tand \quad \nz(\bx) + \bx\cdot\grad \nz(\bx) \geq \mut
\eeq
for almost every $\bx \in \DR,$ where the first inequality holds in the sense of quadratic forms.
\ede

The significance of the class of coefficients in \cref{def:hh-nontrapping} is that \cite[Theorem 2.5]{GrPeSp:19} proves bounds on the solution of \eqref{eq:hh-intro} for such $A$ and $n,$ where the constant in the bound only depends on $\muo, \mut, k, R, $ and $d.$

\bcon[{$k$-independent nontrapping conditions on (random) $A$ and $n$}]\label{con:hh-hetero}
The random fields $A$ and $n$ satisfy $A:\Omega\rightarrow\WoiDRRRdtd$ and $n:\Omega\rightarrow \WoiDRRR.$ Furthermore, there exist $\muo, \mut:\Omega\rightarrow \RR,$ independent of $f,$ with $\muo(\omega),\mut(\omega) > 0$ almost surely and $1/\muo,1/\mut \in \LtORR$  such that $A(\omega) \in \NTA{\muo(\omega)}$ almost surely and $n(\omega) \in \NTn{\mut(\omega)}$ almost surely.
\econ


\bde[Star-shaped]
The set $D \subseteq \RRd$ is \defn{star-shaped with respect to} the point $\bxz$ if for any $\bx \in D$ the line segment $\mleft[\bxz,\bx\mright] \subseteq D.$
\ede

\bth[{Equivalence of variational problems in a nontrapping case}]\label{thm:hh-hetero}
Let $\Dm$ be star-shaped with respect to the origin. Under \cref{con:hh-fAn,con:hh-hetero}:
\bit
\item The maps $\as$ and $\Ls$ (defined by \eqref{eq:SEDPas}) are well-defined.
\item \cref{prob:msedp,prob:somsedp,prob:svsedp} are all equivalent.
\item The solution $u \in %\LtOHozDDR$ 
L^2\big(\Omega;H_{0,D}^1(D_R)\big)$
of these problems exists, is unique, and, given $\kz > 0,$ satisfies the bound
\beq\label{eq:Sbound1}
\NLtOLtDR{\grad u}^2 + k^2\NLtOLtDR{u}^2\leq \NLoO{\Co} \NLtOLtDR{f}^2
\eeq
for all $k\geq\kz$, where $\Co:\Omega\rightarrow\RR$ is given by
\beq\label{eq:C1}
\Co = \max\mleft\{\frac1{\mu_1},\frac1{\mu_2}\mright\}\mleft(\frac{R^2}{\mu_1} + \frac{2}{\mu_2}\mleft(R+ \frac{d-1}{2\kz}\mright)^2\mright).
\eeq
\eit
\enth

As highlighted above, \cref{thm:hh-hetero} is obtained from combining
deterministic a priori bounds from \cite{GrPeSp:19} with the general
arguments in \cref{sec:general} about well-posedness of variational
formulations of stochastic PDEs. \cref{thm:hh-hetero} uses the most basic a priori bound proved
in \cite{GrPeSp:19} (from \cite[Theorem 2.5]{GrPeSp:19}), but \cite{GrPeSp:19}
contains several extensions of this bound. \cref{rem:planewave,rem:ones,rem:tedp,rem:jumps,rem:kdep}
outline the implications that these (deterministic) extensions have for
the stochastic Helmholtz equation.

\bre[Dirichlet boundary conditions on $\GD$ and plane-wave incidence]\label{rem:planewave}
The formulations of the stochastic EDP above assume that $u=0$ on the boundary $\GD.$ An important scattering problem for which $u \neq 0$ on $\GD$ is when $u$ is the field scattered by an incident plane wave; in this case $\gamma u = -\gamma \uI,$ where $\uI$ is the incident plane wave \cite[p. 107]{ChGrLaSp:12}.

The results in this paper can be easily extended to the case when $u\neq0$ on $\pDm$ using \cite[Theorem 2.19(ii)]{GrPeSp:19}  which proves a priori (deterministic) bounds in this case. One subtlety, however, is that $f$ is then not necessarily independent of $\muo$ and $\mut.$ Indeed in this case
%\beqs%\label{eq:fplanewave}
$f = -\grad\cdot\mleft(A\grad \uI\mright) - k^2n\uI$.
%\eeqs
If $\muo$ depends on $A$ and $\mut$ depends on $n$ then %\eqref{eq:fplanewave} shows that 
$f$ may be not be independent of $\muo$ and $\mut.$ One can produce an analogue of \cref{thm:hh-hetero} in the case where $f,\muo,$ and $\mut$ are dependent, but one requires $1/\muo, 1/\mut \in \LfO$ and $f \in \LfOLtD;$ see \cref{rem:notindep} below.
\ere

\bre[The case when either $n=1$ or $A=I$]\label{rem:ones}
When either $n=1$ or $A=I,$ \cite[Theorem 2.19]{GrPeSp:19} gives deterministic bounds under weaker conditions on $A$ and $n$ respectively; the corresponding results for the stochastic case are that:

\bit
\item 

When $n=1$  almost surely, the condition $A(\omega) \in \NTA{\muo(\omega)}$ in \cref{con:hh-hetero} can be improved to
%\beqs
$2A(\omega) - \mleft(\bx \cdot \nabla \mright)A(\omega) \geq \muo(\omega)$
%\eeqs
for almost every $\bx\in \Dp,$ almost surely.

\item When $A=I$ almost surely, the condition $n(\omega) \in \NTn{\mut(\omega)}$ in \cref{con:hh-hetero} can be improved to:
\beq\label{eq:nimproved}
2n(\omega) + \bx \cdot \nabla n(\omega) \geq \mut(\omega) \,\text{ for almost every $\bx \in \Dp$, almost surely}.
\eeq
%for almost every $\bx\in \Dp,$ almost surely.
\eit
\ere

\bre[Geometric interpretation of the conditions on $A$ and $n$ in \cref{def:hh-nontrapping}]
Recall that the $k\tendi$ asymptotics of solutions of the Helmholtz
equation are governed by the behaviour of rays (see, e.g.,
\cite{BaBu:91}). Given (deterministic) $\Az$ and $\nz,$ the Helmholtz EDP is \defn{nontrapping} if all rays starting
in $\DR$ and evolving according to the
Hamiltonian flow defined by the symbol of $\grad\cdot\mleft(\Az\grad u\mright) + k^2\nz u = -\fz$  escape from $\DR$ after some uniform time (see, e.g., \cite[Definition 1.1]{Bu:02}); the EDP is \defn{trapping} otherwise.
The $k$-dependence of the solution operator depends strongly on
whether the problem is trapping, and the type of trapping present;
see, e.g., the overview discussions in \cite[Section 1]{GrPeSp:19},
\cite[Section 1.1]{ChSpGiSm:17}.

The conditions on $A$ and $n$ in \cref{con:hh-hetero} and the star-shapedness restriction
on $\Dm$ are sufficient for the Helmholtz stochastic EDP to be nontrapping almost surely. As noted in \cref{rem:ones}, when $A=I$ almost surely the condition on $n$  can be improved  from that in \eqref{eq:hh-Acond} to \eqref{eq:nimproved} using \cite[Theorem 2.19(ii)]{GrPeSp:19}.
The condition \eqref{eq:nimproved} is equivalent  to nontrapping when $n$ is radial, i.e. $n(\omega,\bx)=
n(\omega,\abs{\bx})$. Indeed, if $n$ is radial and $2n(\omega,\bx)+ \bx \cdot\nabla n(\omega,\bx)<0$ at a point $\bx \in \RRd$,
then the deterministic Helmholtz EDP given by $n(\omega,\bx)$ is trapping; see \cite{Ra:71} and \cite[Theorem 7.7]{GrPeSp:19}.
\ere

\bre[The Helmholtz stochastic truncated exterior Dirichlet problem]\label{rem:tedp}
When applying the Galerkin method to \cref{prob:msedp,prob:somsedp,prob:svsedp}, the Dirichlet-to-Neumann map $\TrR$ is expensive to compute. Therefore, it is
common to approximate the DtN map on $\GR$ by an `absorbing
boundary condition' (see, e.g., \cite[Section 3.3]{Ih:98} and the references
therein), the  simplest of which is the impedance boundary condition
$\partial u/\partial \nu - \ri k u=0$. We call the Helmholtz stochastic EDP posed in $\DR$ with
an impedance boundary condition on $\GR$ the stochastic \emph{truncated
exterior Dirichlet problem} (stochastic TEDP). In fact, since we no longer need to know
the DtN map explicitly on the truncation boundary, the truncation
boundary can be arbitrary (i.e. it does not have to be just a circle/sphere). Note that in the case when the obstacle is the empty set, the TEDP is just the Interior Impedance Problem.

The results in this paper also hold for the stochastic TEDP (with arbitrary Lipschitz truncation boundary) under an analogue of \cref{con:hh-hetero} based on the deterministic bounds in \cite[Theorem A.6(i)]{GrPeSp:19} instead of \cite[Theorem 2.5]{GrPeSp:19}.
\ere

\bre[Discontinuous $A$ and $n$]\label{rem:jumps}
The requirements on $A$ and $n$ in \cref{def:hh-nontrapping} require them to be continuous (since  $\WoiDR = \CzoDR$ as $\DR$ is Lipschitz; see, e.g., \cite[Section 4.2.3, Theorem 5]{EvGa:92}). In addition to proving deterministic a priori bounds for the class of $A$ and $n$ in \cref{def:hh-nontrapping}, the paper \cite{GrPeSp:19} proves deterministic bounds for discontinuous $A$ and $n$ satisfying \eqref{eq:hh-Acond} in a distributional sense; see \cite[Theorem 2.7]{GrPeSp:19}. In this case, when moving outward from the obstacle to infinity,  $A$ can jump downwards and $n$ can
jump upwards on interfaces that are star-shaped. (When the jumps are in the opposite direction, the problem is trapping; see \cite{PoVo:99a} and \cite[Section 6]{MoSp:19}). The
well-posedness results and a priori bounds in this paper can therefore be adapted to prove results about the stochastic Helmholtz equation
for a class of random $A$ and $n$ that allows nontrapping jumps on randomly-placed star-shaped interfaces.
\ere

\bre[$k$-dependent $A$ and $n$]\label{rem:kdep}
In this paper we focus on random fields $A$ and $n$ varying independently of $k;$ this corresponds to a fixed physical
medium, characterised by $A$ and $n$, with waves of frequency $k$ passing through.
In \cref{sec:generating} below we construct $A$ and $n$ as ($k$-independent) $W^{1,\infty}$ perturbations of random fields $\Az$ and $\nz$ satisfying \cref{con:hh-hetero}.
We note, however, that results for $A$ and $n$ being
\emph{$k$-dependent} $L^\infty$ perturbations (i.e. rougher, but $k$-dependent perturbations) of $\Az$
and $\nz$ satisfying \cref{con:hh-hetero} can easily be obtained.

The basis for these bounds is observing that \emph{deterministic} a priori bounds hold when
(a) $A\in \NTA{\mu_1}$, $n = \nz + \eta,$ where $\nz \in \NTn{\mu_2}$
and $k\NLiDRRR{\eta}$ is sufficiently small, and
(b) $A=\Az+B$, $n=\nz+\eta$, where $\Az\in \NTA{\mu_1}$, $\nz \in
\NTn{\mu_2}$, $k\NLiDRRR{\eta}$ and $k \NWoiDRRRdtd{B}$ are
both sufficiently small, and $A, n,$ and $D_-$ are such that $u\in
H^2(D_R)$ (see, e.g., \cite[Theorem 4.18(i)]{Mc:00} or \cite[Theorems
2.3.3.2 and 2.4.2.5]{Gr:85} for these latter requirements). Given these deterministic bounds, the general arguments in this paper
can then be used to prove well-posedness of the analogous stochastic
problems.

To understand why bounds hold in the case (a), observe that one can
write the PDE as
\beq\label{eq:pert}
\nabla \cdot(A\grad u) +k^2 n_0 u = -f - k^2 \eta u;
\eeq
if $k\NLiDRRR{\eta}$ is sufficiently small then the contribution from
the $k^2 \eta u$ term on the right-hand side of \eqref{eq:pert} can be
absorbed into the $k^2\|u\|^2_{L^2(D_R)}$ term appearing on the left-hand
side of the bound (the deterministic analogue of \eqref{eq:Sbound1}).
In the case $\nz=1,$ this is essentially the argument used to prove the
a priori bound in \cite[Theorem 2.4]{FeLiLo:15} (see \cite[Remark
2.15]{GrPeSp:19}).
The reason bounds hold in the case (b) is similar, except now we need
the $H^2$ norm of $u$ on the left-hand side of the bound (as well as
the $H^1$ norm) to absorb the contribution from the $\nabla \cdot
(B\grad u)$ term on the right-hand side.
\ere


\subsection{Random fields satisfying \cref{con:hh-hetero}}\label{sec:generating}
The main focus of this \lcnamecref{chap:stochastic} is proving well-posedness of the variational formulations of the stochastic Helmholtz equation, and a priori bounds on the solution, for the most-general class of $A$ and $n$ allowed by the deterministic bounds in \cite{GrPeSp:19}. However, in this section, motivated by the Karhunen-Lo\`eve expansion (see e.g.~\cite[p.~201ff.]{LoPoSh:14}) and similar expansions of material coefficients for the stationary diffusion equation \cite[Section 2.1]{KuNu:16}, we consider $A$ and $n$ as series expansions around known non-random fields $\Az$ and $\nz$ satisfying \cref{con:hh-hetero} (i.e., \cref{con:hh-hetero} is satisfied for $\nz, \Az$ independent of $\omega \in \Omega$, and therefore $\muo,\mut$ independent of $\omega$).
Define
\beq\label{eq:nseries}
A(\omega,\bx) = \Az(\bx) + \sum_{j=1}^\infty \Yj(\omega) \Psij(\bx)\quad\text{and}\quad n(\omega,\bx) = \nz(\bx) + \sum_{j=1}^\infty\Zj(\omega) \psij(\bx) ,
\eeq
where:
\bit
\item $\supp\mleft(1-\Az\mright),\,\supp \mleft(I-\nz\mright) \compcont \BR,$
\item $\Az$ and $\nz$ satisfy \cref{con:hh-hetero} with $\muo$ and $\mut$ independent of $\omega \in \Omega$
\item $\Yj,\Zj \sim \Unif(-1/2,1/2)$ i.i.d.,
\item $\Psij \in \WoiDRRRdtd$ with $\supp \Psij \compcont \BR$ for all $j =1,\ldots,m$,
\beq\label{eq:Apsimeas}
\sum_{j=1}^\infty \NWoiDRRRdtd{\Psij} < \infty, \tand
\eeq
\beq\label{eq:Apsipos}
\sum_{j=1}^\infty \esssup_{\bx \in \DR} \NopCCd{\Psij} < 2\Azmin,
\eeq
where $\Azmin > 0$ is such that $\Azmin\abs{\bxi}^2 \leq \big(A(\bx)\bxi\big)\cdot \bxi$ for almost every $\bx \in \Dp$ and for all $\bxi \in \CCd$.
\item $\psij \in \WoiDRRR$ with $\supp \psij \compcont \BR$ for all $j = 1,\ldots,m$,
\beq\label{eq:npsimeas}
\sum_{j=1}^\infty \NWoiDRRR{\psij} < \infty, \tand
\eeq
\beq\label{eq:npsipos}
\sum_{j=1}^\infty \NLiDRRR{\psij} < 2 \nzmin,
\eeq
where $\nzmin \de \essinf_{\bx \in \DR} \nz(\bx).$
\eit
%Regarding the measurability of $n$ and $A$ defined by \eqref{eq:nseries}: the proof for this particular setting of mappings into a separable subspace of a general normed vector space is analagous to the proof in the case of a sum of real-valued measurable mappings, and so we do not give it here.

The assumptions \eqref{eq:Apsipos} and \eqref{eq:npsipos} ensure that $A > 0$ (in the sense of quadratic forms) and $n > 0$  almost surely, and the assumptions \eqref{eq:Apsimeas} and \eqref{eq:npsimeas} are used to prove $A$ and $n$ are measurable.

Regarding the measurability of $A$ and $n$ defined by \eqref{eq:nseries}: the proof that $A$ and $n$ given by \eqref{eq:nseries} are measurable is given in \cref{lem:seriesmeas}, and relies on the proof that the sum of measurable functions
is measurable. This latter result is standard, but we have not been able to find this result for this particular setting of mappings
into a separable subspace of a general normed vector space, and so we briefly give it in \cref{lem:summeas}. 

The following lemmas give sufficient conditions for the series in \eqref{eq:nseries} to satisfy \cref{con:hh-hetero}.

\ble[Series expansion of $A$ satisfies \cref{con:hh-hetero}]\label{lem:Agen}
Let $\mu > 0$, $\delta \in \mleft(0,1\mright).$ If $\Az \in \NTA{\mu},$ and
\beq\label{eq:Aseriescond}
\sum_{j=1}^\infty \esssup_{\bx \in \DR} \NopCCd{\Psij(\bx) - \mleft(\bx\cdot\grad\mright)\Psij(\bx)} \leq 2\delta\mu,
\eeq
then $A \in \NTA{(1-\delta)\mu}$ almost surely.
\ele

\bpf[Proof of \cref{lem:Agen}]
Since $\Az \in \NTA{\mu},$ we have 
\beq\label{eq:Aseries1}
\Big(\mleft(A(\omega,\bx) - \mleft(\bx\cdot\grad\mright)A(\omega,\bx)\mright)\bxi\Big)\cdot\bxib \geq\mu \abs{\bxi}^2+\sum_{j=1}^\infty  \Big(\Yj(\omega)\mleft(\Psij(\bx)- \mleft(\bx\cdot\grad\mright)\Psij(\bx)\mright)\bxi\Big)\cdot\bxib
\eeq
for all $\bxi \in \CCd,$ for almost every $\bx \in \DR,$ almost surely. As $\Yj \sim \Unif(-1/2,1/2)$ for all $j$ and the bound \eqref{eq:Aseriescond} holds, the right-hand side of \eqref{eq:Aseries1} is bounded below by
\beqs
\mu \abs{\bxi}^2 -\half 2\delta \mu \abs{\bxi}^2 = (1-\delta)\mu\abs{\bxi}^2\text{ almost surely.}
\eeqs
Since $\bxi \in \CCd$ was arbitrary, it follows that $A(\omega) \in \NTA{(1-\delta)\mu)}$ almost surely, as required.
\epf

\ble[Series expansion of $n$ satisfies \cref{con:hh-hetero}]\label{lem:ngen}
Let $\mu > 0$ and $\delta \in \mleft(0,1\mright).$ If $\nz \in \NTn{\mu}$ and
\beq\label{eq:nseriescond}
\sum_{j=1}^m\NLiDRRR{\psij(\bx) + \bx\cdot\grad\psij(\bx)} \leq 2\delta\mu,
\eeq
then $n \in \NTn{(1-\delta)\mu}.$
\ele

The proof of \cref{lem:ngen} is omitted, since it is similar to the proof of \cref{lem:Agen}; in fact it is simpler, because it involves scalars rather than matrices.


\subsection{Outline of the chapter} In \cref{sec:otherwork} we discuss our results in the context of related literature. In \cref{sec:general} we  state general results on a priori bounds and well-posedness for stochastic variational formulations. In \cref{sec:genproof} we prove the results in \cref{sec:general}. In \cref{sec:hhproof} we prove \cref{thm:hh-gen,thm:hh-hetero}.% In \cref{sec:federico} we discuss the failure of Fredholm theory for the stochastic variational formulation of Helmholtz problems. In \cref{app:mtBs} we recap results from measure theory and the theory of Bochner spaces.


\subsection{Discussion of the main results in the context of other work on UQ for time-harmonic wave equations}\label{sec:otherwork}

In this section we discuss existing results on well-posedness of \eqref{eq:hh-intro}, as well as analogous results for the elastic wave equation and the time-harmonic Maxwell's equations. The most closely-related work to this \lcnamecref{chap:stochastic} is \cite{FeLiLo:15} (and its analogue for elastic waves \cite{FeLo:17}), in that a large component of \cite{FeLiLo:15} consists of attempting to prove well-posedness and a priori bounds for the stochastic variational formulation (i.e.~\cref{prob:svsedp}) of the Helmholtz Interior Impedance Problem; i.e., \eqref{eq:hh-intro} with $A=I$ and stochastic $n$ posed in a bounded domain with an impedance boundary condition $\partial u/\partial \nu - ik u = g$. (Recall that this boundary condition is a simple approximation to the Dirichlet-to-Neumann map $\TrR$ defined above \eqref{eq:src}.) Under the assumption of existence, \cite{FeLiLo:15} shows that for any $k>0$ the solution is unique and satisfies an a priori bound of the form \eqref{eq:Sbound1} (with different constant $\Co$), provided $n=1+\eta$ where the random field $\eta$ satisfies (almost surely) $\N{\eta}_{L^\infty} \leq C/k$ for some $C>0$ independent of $k$. \cite{FeLiLo:15} then invokes Fredholm theory to conclude existence, but this relies on an incorrect assumption about compact inclusion of Bochner spaces---see \cref{sec:federico} below. However, combining \cref{thm:hh-gen,rem:tedp,rem:kdep} with $A=I$ and $\nz=1+\eta$ (with $\eta$ as above) produces an analogous result to \cref{thm:hh-hetero}, and gives a correct proof of \cite[Theorem 2.5]{FeLiLo:15}. Therefore the analysis of the Monte Carlo interior penalty discontinuous Galerkin method in \cite{FeLiLo:15} can proceed under the assumptions of \cref{thm:hh-gen,rem:tedp,rem:kdep}.

The papers \cite{HiScScSc:15} and \cite{Sc:17} consider the Helmholtz transmission problem with a stochastic interface, i.e.~\eqref{eq:hh-intro} posed in $\RRd$ with both $A$ and $n$ piecewise constant and jumping on a common, randomly-located interface. A component of this work is establishing well-posedness of \cref{prob:msedp} for this setup. To do this, the authors make the assumption that $k$ is small (to avoid problems with trapping mentioned above---see the comments after \cite[Theorem 4.3]{HiScScSc:15}); the sesquilinear form $a$ is then coercive and a priori bounds (in principle explicit in $A$ and $n$) follow in Sobolev norms \cite[Lemma 4.5]{HiScScSc:15} and H\"older norms \cite[Theorem 5.1 and Corollary 5.2]{Sc:17}. By \cref{rem:jumps}, the results of this \lcnamecref{chap:stochastic} can be used to obtain the analogous well-posedness result for large $k$ in the case of nontrapping jumps.

The paper \cite{BuGh:14} studies the \defn{Bayesian inverse problem} associated to \eqref{eq:hh-intro} with $A=I$ and $n=1$  posed in the  exterior of a Dirichlet obstacle. That is, \cite{BuGh:14} analyses computing the posterior distribution of the shape of the obstacle given noisy observations of the acoustic field in the exterior of the obstacle. A component of the analysis in \cite{BuGh:14} is the well-posedness of the forward problem for an obstacle with a variable boundary \cite[Proposition 3.5]{BuGh:14}. Instead of mapping the problem to one with  a fixed domain and variable $A$ and $n,$ \cite{BuGh:14} instead works with the variability of the obstacle directly, using boundary-integral equations. The $k$-dependence of the solution operator is not considered, but would enter in \cite[Lemma 3.1]{BuGh:14}.

%The paper \cite{KaPo:18} considers the time-harmonic Maxwell's equations posed on a fixed domain, with the material coefficient $\sigma$ defined by a random field on a subset of the domain. Part of the analysis of this paper is proving the well-posedness of the analogue of \cref{prob:somsedp} for the time-harmonic Maxwell's equations. The authors achieve this well-posedness result by working in a low-frequency regime, so that the analogue of the sesquilinear form $a$ is bounded and coercive (analagous to the approach in \cite{HiScScSc:15}), and therefore the Lax--Milgram Theorem can be applied to the deterministic problem to obtain existence, uniqueness, and a deterministic a priori bound. This bound is then converted into a bound for the stochastic problem using similar techniques to those in this paper. 

The papers \cite{JeScZe:17} and \cite{JeSc:16} consider the time-harmonic Maxwell's equations with (i) the material coefficients $\eps,\mu$ constant in the exterior of a perfectly-conducting random obstacle and (ii) $\eps,\mu$ piecewise constant and jumping on a common randomly located interface; in both cases these problems are mapped to problems where the domain/interface is fixed and $\eps$ and $\mu$ are random and heterogeneous. The papers \cite{JeScZe:17} and \cite{JeSc:16} essentially consider the analogue of \cref{prob:msedp} for the time-harmonic Maxwell's equations, obtaining well-posedness from the corresponding results for the related deterministic problems. Similarly the paper \cite{EsJe:19} considers analogues of \cref{prob:msedp} for the Helmholtz equation with constant coefficients and a random (penetrable or inpenetrable) obstacle. The paper \cite{EsJe:19} obtains \emph{deterministic} tensor-product boundary-integral equations for the statistical moments of $u.$


\section[General stochastic a priori bounds and well-posedness]{General results on proving a priori bounds and well-posedness of stochastic variational formulations}\label{sec:gen-framework}\label{sec:general}
In this section we state general results for proving a priori bounds and well-posedness results for variational formulations of linear time-independent SPDEs.

\subsection{Notation and definitions of the variational formulations}\label{sec:notdef}
Let $\OFP$ be a complete probability space. Let $X$ and $Y$ be separable Banach spaces over a field $\FF,$ (where $\FF = \RR$ or $\CC$).
Let $\homspace$ denote the space of bounded linear maps $X\rightarrow\Ys.$ Let $\cC$ be a topological space with topology $\TC.$ Given maps
\beqs
\coeff:\Omega\rightarrow\cC,\quad\toform:\cC \rightarrow \homspace,\quad\text{and } \rhs:\cC \rightarrow \Ys,
\eeqs let $\SAc:\LtOX \rightarrow \LtOYas$ and  $\SLc \in \LtOYas$ be defined by
\beq\label{eq:SA}
\big[\SAc(u)\big](v) \de \int_\Omega \big[\Acomega u(\omega)\big]\big(v(\omega)\big) \dd\PP(\omega) \quad\text{and} \quad 
\SLc(v) \de \int_\Omega \Lcomega\big(v(\omega)\big) \dd \PP(\omega)
\eeq
%for $v \in \LtOY,$ and let be defined by
%\beq\label{eq:SL}
%%\SLc(v) \de \int_\Omega \Lcomega\big(v(\omega)\big) \dd \PP(\omega)
%\eeq
for $v \in \LtOY.$ Recall that a bounded linear map $X\rightarrow\Ys$ is equivalent to a sesquilinear (or bilinear) form on $X \times Y;$ see e.g. \cite[Lemma 2.1.38]{SaSc:11}. To keep notation compact, we write $\Acomega=(\toform \circ c)(\omega)$ and $\Lcomega=(\rhs\circ c)(\omega).$



\bre[Interpretation of the space $\cC$]
The space $\cC$ is the `space of inputs'. For the stochastic Helmholtz EDP in \cref{sec:hh-results} the space $\cC$ is defined in \cref{def:cCHh} below, but the upshot of this definition is that for any $\omega \in \Omega$ the triple $(A(\omega),n(\omega),f(\omega))$ is an element of $\cC.$
The maps $\coeff,$ $\toform,$ and $\rhs$ are given by $\coeff = \mleft(A,n,f\mright),$ $\toform = a,$ and $\rhs = L,$ where $a$ and $L$ are given by \eqref{eq:SEDPa} and \eqref{eq:SEDPL} respectively and the equality $\toform=a$ is meant in the sense of the one-to-one correspondence between $\homspace$ and sesquilinear forms on $X\times Y.$
\ere

The following three problems are the analogues in this general setting of \cref{prob:msedp,prob:somsedp,prob:svsedp} in \cref{sec:intro}.

\bprobM[Measurable variational formulation almost surely]\label{prob:meas}
Find a measurable function $u:\Omega \rightarrow X$ such that 
\beq\label{eq:aeeq}
\Acomega u(\omega) = \Lcomega \tin \Ys
\eeq
almost surely.
\eprobvar

\bprobLT[Second-order moment variational formulation almost surely]\label{prob:lt}
Find $u \in \LtOX$ such that \eqref{eq:aeeq} holds almost surely.
\eprobvar

\bprobSVAR[Stochastic variational formulation]\label{prob:svar}
Find $u \in \LtOX$ such that
\beq\label{eq:stoeq}
\SAc u = \SLc \tin \LtOYas.
\eeq
\eprobvar

\bre[Immediate relationships between formulations]\label{rem:imm}
Since $\LtOX \subseteq \BorelOX$ (the space of all measurable functions $\Omega\rightarrow X$)
it is immediate that if $u$ solves \cref{prob:lt} then every member of the equivalence class of $u$ solves \cref{prob:meas}.
\ere

\subsection{Conditions on $\toform,$ $\rhs,$ and $\coeff$}\label{sec:cons}
We now state the conditions under which we prove results about the equivalence of \cref{prob:meas,prob:lt,prob:svar}.


\bconAo[$\toform$ is continuous] \label{con:coeffstoform}
The function $\toform:\cC \rightarrow \homspace$ is continuous, where we place the norm topology on $X,$ the dual norm topology on $\Ys$, and the operator norm topology on $\homspace.$
\econvar

\bconA[Regularity of {$\toform\circ\coeff$}]\label{con:A}
The map $\toform\circ\coeff \in \LiOhomspace.$
\econvar


We note that \cref{con:A} is violated in the well-studied case of a log-normal coefficient $\kappa$ for the stationary diffusion equation \eqref{eq:diffusion}; in order to ensure the stochastic variational formulation is well-defined in this case, one must change the space of test functions as in \cite{Gi:10,MuSt:11}

\bconLo[$\rhs$ is continuous] \label{con:coeffstofunc}
The function $\rhs:\cC \rightarrow \Ys$  is continuous, where we place the dual norm topology on $\Ys.$
\econvar

\bconL[Regularity of {$\rhs\circ\coeff$}]\label{con:L}
The map $\rhs\circ \coeff \in \LtOYs.$
\econvar

\bconCo[$\coeff$ is measurable]\label{con:cborel}
 The function $c:\Omega \rightarrow \cC$ is measurable.
\econvar

To state the next condition, we need to recall the following definition.

\bde[{$\PP$}-essentially separably valued {\cite[p26]{Ry:02}}]\label{def:sepval}
Let $\mleft(S,\Top{S}\mright)$ be a topological space. A function $h:\Omega\rightarrow S$ is \defn{$\PP$-essentially separably valued} if there exists $E \in \cF$ such that $\PP(E) = 1$ and $h(E)$ is contained in a separable subset of $S$.
\ede


\bconC[{$\coeff$} is {$\PP$}-essentially separably valued]\label{con:C}
The map $\coeff:\Omega\rightarrow\cC$  is $\PP$-essentially separably valued.
\econvar

\bre[Why do we need \cref{con:C}?]
The theory of Bochner spaces requires strong measurability of functions (see \cref{def:strongmeas,def:bochnerspace} below). However, the proof techniques used in this \lcnamecref{chap:stochastic} rely heavily on the measurability of functions (see \cref{def:meas} below). In separable spaces these two notions are equivalent (see \cref{cor:pettis}). However, some of the spaces we encounter (such as $\LiDRRR$) are not separable. Therefore, in our arguments we use \cref{con:C} along with the Pettis Measurability Theorem (\cref{thm:pettis} below) to conclude that measurable functions are strongly measurable.
\ere

\bconB[A priori bound almost surely]\label{con:B}
There exist $\Cj,\fj:\Omega\rightarrow \RR, \,j=1,\ldots,m$ such that $\Cj\fj \in \LoO$ for all $j=1,\ldots,m$ and the bound
\beq \label{eq:sbe1}
\NX{u(\omega)}^2 \leq \sum_{j=1}^m\Cj(\omega)\fj(\omega)
\eeq
holds almost surely.
\econvar

\bre[Notation in the a priori bound]
We use the notation $f_j$ in the right-hand side of \eqref{eq:sbe1} to emphasise the fact that typically these terms relate to the right-hand sides of the PDE in question. For the stochastic Helmholtz EDP, $m=1,$ $f_1 = \NLtD{f}^2,$ and $C_1$ is given by \eqref{eq:C1}.
\ere


\bconK[Uniqueness almost surely]\label{con:K}
$\ker\mleft(\Acomega\mright) = \set{0}$ $\PP$-almost surely.
\econvar

The condition  $\ker\mleft(\Acomega\mright) = \set{0}$ $\PP$-almost surely can be stated as: given $\L \in \Ys,$ for $\PP$-almost every $\omega \in \Omega$ the deterministic problem $\Acomega \uz = L$ has a unique solution,

\subsection{Results on the equivalence of \cref{prob:meas,,prob:lt,,prob:svar}}


\bth[Measurable solution implies second-order solution]\label{thm:3}
Under \cref{con:B}, if $u$ solves \cref{prob:meas} then $u$ solves \cref{prob:lt}  and satisfies the stochastic a priori bound
\beq\label{eq:sbresult}
\NLtOX{u}^2 \leq\sum_{j=1}^m \NLoO{C_jf_j}.
\eeq
\enth

The proof of \cref{thm:3} is on \cpageref{page:thm3proof} below.

Note that the right-hand side of the stochastic a priori bound \eqref{eq:sbresult} is the expectation of the right-hand side of the bound \eqref{eq:sbe1}.

\ble[Stochastic variational formulation well-defined]\label{lem:svarwelldefined}
Under \cref{con:coeffstoform,,con:A,,con:coeffstofunc,,con:L,,con:cborel,,con:C},  the maps $\SA$ and $\SL$ defined by \eqref{eq:SA} are well-defined in the sense that
\beq\label{eq:finite}
\mleft[\SA(\vo)\mright](\vt),\, \SL(\vt) < \infty \quad\text{for all } \vo \in \LtOX, \text{ for all }\vt \in \LtOY.
\eeq
\ele

The proof of \cref{lem:svarwelldefined} is on \cpageref{page:lemsvarwelldefinedproof} below.

\bth[Second-order solution implies stochastic variational solution]\label{thm:11}
Under \cref{con:coeffstofunc,,con:L,,con:cborel,,con:C}, if $u$ solves \cref{prob:lt} then $u$ solves \cref{prob:svar}.
\enth

The proof of \cref{thm:11} is on \cpageref{page:thm11proof} below.

\bth[Stochastic variational solution implies second-order solution]\label{thm:12}
If \cref{prob:svar} is well-defined and $u$ solves \cref{prob:svar}, then $u$ solves \cref{prob:lt}.
\enth

The proof of \cref{thm:12} is on \cpageref{page:thm12proof} below.

\Cref{thm:3,thm:11,thm:12,lem:svarwelldefined} are summarised in \cref{fig:ladder}.

\begin{figure}[h]
  \centering
  \scalebox{0.90}{
\begin{tikzpicture}

% Inspired by https://tex.stackexchange.com/questions/109102/anchor-arrow-start-position

\draw (0,1) node [rounded rectangle, fill=gray!45!white] (meas) {\Cref{prob:meas}};



\draw (0,-1) node [rounded rectangle, fill=gray!45!white] (lt) {\Cref{prob:lt}};


\draw (0,-3) node [rounded rectangle, fill=gray!45!white] (svar) {\Cref{prob:svar}};


\path node [left = \owenshift of meas.north] (meas top left) {};
\path node [right = \owenshift of meas.north] (meas top right) {};
\path node [left = \owenshift of meas.south] (meas bottom left) {};
\path node [right = \owenshift of meas.south] (meas bottom right) {};

\path node [left = \owenshift of lt.north] (lt top left) {};
\path node [right = \owenshift of lt.north] (lt top right) {};
\path node [left = \owenshift of lt.south] (lt bottom left) {};
\path node [right = \owenshift of lt.south] (lt bottom right) {};

\path node [left = \owenshift of svar.north] (svar top left) {};
\path node [right = \owenshift of svar.north] (svar top right) {};




% Arrows
\begin{scope}[->]


\draw
(meas bottom right)
--
node[right,align=center,text width=5cm] {Under \cref{con:B}, get stochastic a priori bound \eqref{eq:sbresult} (\Cref{thm:3})}
 (lt top right);

\draw
(lt top left)
--
node[left] {Immediate}
 (meas bottom left);

\draw
(lt bottom right)
--
node[right,align=center,text width=5cm] {Under \cref{con:coeffstofunc,,con:L,,con:cborel,,con:C},  (\Cref{thm:11})}
 (svar top right);

\draw
(svar top left)
--
node[left,align=center,text width=5cm] {If \cref{prob:svar} is well-defined (\cref{thm:12})}
(lt bottom left);
\end{scope}

\path node [align=center,below = \owentextshift of svar,text width=10cm] (svar wd) {Well-defined under  \cref{con:coeffstoform,,con:A,,con:coeffstofunc,,con:L,,con:cborel,,con:C} (\Cref{lem:svarwelldefined})};


\end{tikzpicture}
}
\caption[The relationship between the different variational formulations of stochastic PDEs]{The relationship between the variational formulations. An arrow from Problem P to Problem Q with Conditions R indicates `under Conditions R, the solution of Problem P is a solution of Problem Q'}\label{fig:ladder}
\end{figure}


\bre[\Cref{con:L} in \cref{thm:11}]
In \Cref{thm:11} we could replace \cref{con:L} with \cref{con:A}, and the result would still hold---see the proof for further details. However, \cref{con:L} is less restrictive than \cref{con:A}, as it only requires $L^2$ integrability of $\rhs\circ\coeff$ as opposed to essential boundedness of $\toform\circ\coeff.$ 
\ere

\ble[Showing uniqueness of the solution to \cref{prob:meas,prob:lt,prob:svar}]\label{lem:uniq}
If \cref{con:K} holds, then
\ben
\item\label[part]{part:uniq1} the solution to \cref{prob:meas} (if it exists) is unique up to modification on a set of $\PP$-measure 0 in $\Omega$, 
\item\label[part]{part:uniq2} the solution to \cref{prob:lt} (if it exists) is unique in $\LtOX,$ and 
\item\label[part]{part:uniq3} if \cref{prob:svar} is well-defined, the solution to \cref{prob:svar} (if it exists) is unique in $\LtOX.$
\een
\ele

The proof of \cref{lem:uniq} is on \cpageref{page:lemuniqproof} below.

\bre[Informal discussion on the ideas behind the equivalence results]\label{rem:nontechnical}
The diagram in \cref{fig:ladder} summarises the relationships between the variational formulations, and the conditions under which they hold. Moving `up' the left-hand side of the diagram, we prove a solution of \cref{prob:svar} is a solution of \cref{prob:lt} in \cref{thm:12}; the key idea in this theorem is to use a particular set of test functions and the general measure-theory result of \cref{lem:gotoae} below; this approach was used for the stationary diffusion equation \eqref{eq:diffusion} with log-normal coefficients in \cite{Gi:10}, and for a wider class of coefficients in \cite{MuSt:11}.

Moving `down' the right-hand side, we prove a solution of \cref{prob:meas} is a solution of \cref{prob:lt} in \cref{thm:3}; the key part of this proof is that the bound in \cref{con:B} gives information on the integrability of the solution $u.$ (In the case of \eqref{eq:diffusion} with uniformly coercive and bounded coefficient $\kappa,$ the analogous integrability result follows from the Lax--Milgram theorem; \cite[Proposition 2.4]{Ch:12} proves an equivalent result for \eqref{eq:diffusion} with lognormal coefficient $\kappa$ with an isotropic Lipschitz covariance function.) Proving a solution of \cref{prob:lt} is a solution of \cref{prob:svar} in \cref{thm:11} essentially amounts to posing conditions such that the quantities $\mleft[\Acomega\mleft(u(\omega)\mright)\mright] \mleft(v(\omega)\mright)$ and $\Lcomega \mleft(v(\omega)\mright)$ are Bochner integrable for any $v \in \LtOY,$ so that \eqref{eq:stoeq} makes sense. \Cref{lem:svarwelldefined} shows that the stronger property \eqref{eq:finite} holds, and requires stronger assumptions than \cref{thm:11}, since the proof of \cref{thm:11} uses the additional information that u solves \cref{prob:lt}.
\ere

\bre[Changing the condition $u \in \LtOX$]
Here we seek the solution $u \in \LtOX$ but we could instead require $u \in \LpOX,$ for some $p>0$ and require $\SA u = \SL$ in $\LqOYas,$ for some $q>0$ (i.e. use test functions in $\LqOY$). In this case, the proof of \cref{thm:12} would be nearly identical, as the space $\D$ of test functions used there %(see \eqref{eq:Dspace} below) 
is a subset of $\LqOY$ for all $q>0.$ One could also develop analogues of \cref{thm:3,thm:11,lem:svarwelldefined} in this setting---see e.g.~\cite[Theorem 3.20]{Gi:10} for an example of this approach for the stationary diffusion equation with lognormal diffusion coefficient.
\ere

\bre[Non-reliance on the Lax-Milgram theorem]
The above results hold for an arbitrary sesquilinear form and hence are applicable to a wide variety of PDEs; their main advantage is that they apply to PDEs whose stochastic variational formulations are not coercive. For example, as noted in \cref{sec:intro}, for the stationary diffusion equation \eqref{eq:diffusion} with coefficient $\kappa$ bounded uniformly below in $\omega,$ the bilinear form of \cref{prob:svar} is coercive; existence and uniqueness follow from the Lax-Milgram theorem, and hence the chain of results above leading to the well-posedness of \cref{prob:svar} is not necessary.
\ere

\bre[Overview of how these results are applied to the Helmholtz equation in \cref{sec:hhproof}]

We obtain the results for the Helmholtz equation via the following steps (which could also be applied to other SPDEs fitting into this framework):
\ben
\item\label[step]{it:step1pw} Define the map $\coeff$ (via $A,n,$ and $f$) such that for almost every $\omega \in \Omega$ there exists a solution of the deterministic Helmholtz EDP corresponding to $\coeff(\omega).$
\item\label[step]{it:step2pw} Define $u:\Omega\rightarrow X$ to map $\omega$ to the solution of the deterministic problem corresponding to $\coeff(\omega).$
\item Prove that \cref{con:coeffstoform,,con:A,,con:coeffstofunc,,con:L,,con:cborel,,con:C,,con:B,,con:K} hold, so that one can apply \cref{thm:3,,thm:11,,thm:12} along with \cref{lem:svarwelldefined,lem:uniq} to show \cref{prob:svsedp} is well-defined and $u$ is unique and satisfies \cref{prob:msedp,prob:somsedp,prob:svsedp}.
\een
\Cref{it:step1pw,it:step2pw} can be thought of as constructing a solution pathwise.
\ere


\section[Proof of the results in Section \MakeLowercase{\ref{sec:general}}]{Proof of the results in \cref{sec:general}}\label{sec:genproof}

\subsection{Preliminary lemmas}\label{sec:prelemmanary}
To simplify notation, we introduce the following definition.
\bde[Pairing map]
For fixed $\coeff:\Omega\rightarrow\cC,\,\toform:\Omega\rightarrow\homspace,$ given $v:\Omega\rightarrow X$ we define the map $\comp:\Omega\rightarrow \Ys$ by
\beq\label{eq:comp}
\comp(\omega) \de \mleft[\mleft(\toform\circ\coeff\mright)(\omega)\mright]\mleft(v(\omega)\mright).
\eeq
\ede

A key ingredient in proving that the stochastic variational formulation is well-defined (see \cref{lem:svarwelldefined}) is showing that the maps $\compu$ and $\fullrhs$ are measurable. Showing that $\fullrhs $ is measurable is straightforward (see \cref{lem:thetaborel} below), but showing that $\compu$ is measurable is not. This is because $\fullrhs$ depends on $\omega$ only through its dependence on $\coeff,$ but $\compu$ depends on $\omega$ through both the dependence of $\toform\circ\coeff$ on $\omega$ and the dependence of $u$ on $\omega;$ it is this dual dependence that causes the extra complication.

\ble[$\fullrhs$ is measurable] \label{lem:thetaborel}
Under \cref{con:coeffstofunc,con:cborel} the function $\fullrhs$ is measurable.
\ele

\begin{proof}[Proof of \cref{lem:thetaborel}]
The map $\coeff$ is measurable (by \cref{con:cborel})
  and $\rhs$ is continuous  (by \cref{con:coeffstofunc}), therefore \cref{lem:contplusmeas} implies that $\fullrhs $ is measurable.
\end{proof}

We now move on to the more-involved process of showing $\comp$ is measurable.

\bde[Product map]
For $v:\Omega\rightarrow X,$ let $\toprod:\Omega\rightarrow \homspace \times X$ be defined by $\toprod(\omega) = \big(\mleft(\toform\circ\coeff\mright)(\omega),v(\omega)\big).$
\ede

\ble[Product map is measurable]\label{lem:Pmeas}
When $\homspace \times X$ is equipped with the product top\-ol\-ogy,  if \cref{con:coeffstoform,con:cborel} hold, and if $v:\Omega\rightarrow X$ is measurable, then $\toprod:\Omega\rightarrow \homspace\times X$ is measurable.
\ele

\bpf[Proof of \cref{lem:Pmeas}]
By the result on the measurability of the Cartesian product of measureable functions (\cref{lem:measprod}), $\toprod$ is measurable with respect to $\big(\cF,\bigBorel{\homspace}\otimes \Borel{X}\big)$ (where $\cB$ denotes the Borel $\sigma$-algebra---see \cref{def:borelsigma}), as both of the coordinate functions $\toform\circ c$ and $v$ are measurable. Since $\homspace$ and $X$ are both metric spaces, they are both Hausdorff. As $X$ is separable, \cref{lem:bogachev} on the product of Borel $\sigma$-algebras implies $\bigBorel{\homspace} \otimes \Borel{X} = \bigBorel{\homspace \times X}.$ Hence $\toprod$ is measurable with respect to $\big(\cF,\bigBorel{\homspace\times X}\big).$
\epf

\bde[Evaluation map]
Let $Z$ be a separable Banach space. Define $\eval:\homspaceZ \times X \rightarrow \Zs$ by
\beq\label{eq:evaldef}
\eval\big(\mleft(\cH,v\mright)\big) \de \cH(v) \quad\tfor \cH \in \homspaceZ \tand v \in X.
\eeq
\ede

Observe that the pairing, product, and evaluation maps ($\comp, \toprod,$ and, $\evalY$ respectively) are related by $\comp = \evalY \circ \toprod.$

\ble[Evaluation map is continuous]\label{lem:vstarcont}
Let $Z$ be a separable Banach space. The map $\eval$ is continuous with respect to the product topology on $\homspaceZ \times X$ and the dual norm topology on $\Zs.$
\ele

The proof of \cref{lem:vstarcont} is straightforward and omitted.

\ble[$\comp$ is measurable] \label{lem:gammaborel}
If  \cref{con:coeffstoform,con:cborel} hold and $v$ is measurable, then the function $\comp$ as defined by \eqref{eq:comp} is measurable.
\ele

\begin{proof}[Proof of \cref{lem:gammaborel}]
By \cref{lem:Pmeas} $\toprod$ is measurable
 and by \cref{lem:vstarcont} $\evalY$ is continuous. Therefore \cref{lem:contplusmeas} implies that $\comp = \evalY \circ \toprod$ is measurable.
 \end{proof}



\subsection{Proofs of \cref{thm:3,thm:11,thm:12,lem:svarwelldefined,lem:uniq}}
\bpf[Proof of \cref{thm:3}]
\label{page:thm3proof}
We need to show $u:\Omega \rightarrow X$ is strongly measurable, satisfies the bound \eqref{eq:sbresult}, and therefore is Bochner integrable and is in the space $\LtOX.$ Our plan is to use \cref{cor:bochnersimple} to show $u$ is Bochner integrable, and establish \eqref{eq:sbresult} as a by-product. Since $u$ solves \cref{prob:meas}, $u$ is measurable. As $X$ is separable, it follows from \cref{cor:pettis} that $u$ is strongly measurable.
Define $N:X \rightarrow \RR$ by
$ N(v) \de \N{v}_X^2.$
Since $N$ is continuous, \cref{lem:contplusmeas} implies $N \circ u:\Omega \rightarrow \RR$ is measurable. 
Therefore, since both the left- and right-hand sides of \eqref{eq:sbe1} are measurable and \eqref{eq:sbe1} holds for almost every $\omega \in \Omega$ we can integrate \eqref{eq:sbe1} over $\Omega$ with respect to $\PP$ and obtain
\beq\label{eq:sbmid}
\int_\Omega \NX{u(\omega)}^2 \dd\PP(\omega) \leq \sum_{j=1}^m \NLoO{\Cj\fj},
\eeq
the right-hand side of which is finite since \cref{con:B} includes that $\Cj\fj \in \LoO$ for all $j = 1,\ldots,m.$ Since $u$ is strongly measurable, the bound \eqref{eq:sbmid} and \cref{cor:bochnersimple} with $p=2$ imply that $u$ is Bochner integrable. The norm $\NLtOX{u}$ is thus well-defined by \cref{def:bochnernorm} and \eqref{eq:sbmid} shows that \eqref{eq:sbresult} holds, and so in particular $\NLtOX{u} < \infty.$
\epf

\bpf[Proof of \cref{lem:svarwelldefined}]
\label{page:lemsvarwelldefinedproof}
We must show that for any $\vo \in \LtOX$ and any $\vt \in \LtOY$:
\bit
\item The quantities $\big[\Acomega \vo(\omega)\big]\big(\vt(\omega)\big)$ and $\Lcomega\big(\vt(\omega)\big)$ are Bochner integrable, so that the definitions of $\SA$ and $\SL$ as integrals over $\Omega$ make sense.
\item The maps $\SA(\vo)$ and $\SL$ are linear and bounded on $\LtOY,$ that is, $\SA:\LtOX\rightarrow\LtOYas$ and $\SL \in \LtOYas.$
\eit
It follows from these two points that $\SA$ and $\SL$ are well-defined.
Thanks to the groundwork laid in \cref{sec:prelemmanary}, the measurability of $\big[\Acomega \vo(\omega)\big]\big(\vt(\omega)\big)$ and $\Lcomega\big(\vt(\omega)\big)$  follows from \cref{lem:gammaborel,lem:thetaborel} (which need \cref{con:coeffstoform,con:coeffstofunc,con:C}).
Their $\PP$-essential separability follows from \cref{con:coeffstoform,con:coeffstofunc,con:C,lem:esssep} and thus their strong measurability follows from \cref{cor:pettis} on the equivalence of measurability and strong measurability when the image is separable. Their Bochner integrability then follows from the Bochner integrability condition in \cref{thm:bochnercond} (with $V=\FF$) and the Cauchy--Schwartz inequality since
\begin{align}
\int_\Omega \abs{\Lcomega\big(\vt(\omega)\big)}\dd\PP(\omega)&\leq \int_\Omega \NYs{\mleft(\rhs\circ\coeff\mright)(\omega)}\NY{\vt(\omega)}\dd\PP(\omega)\nonumber\\
&\leq \NLtOYs{\rhs\circ\coeff}\NLtOY{\vt},\label{eq:Lfirst}
\end{align}
which is finite by \cref{con:L}, and 
\begin{align}
\int_\Omega \Big|\big[\Acomega \vo(\omega)\big]\big(\vt(\omega)\big)\Big|\dd\PP(\omega) %&\leq \int_\Omega \NYs{\Acomega \vo(\omega)}\NY{\vt(\omega)}\dd\PP(\omega)\nonumber\\
&\leq \esssup_{\omega \in \Omega} \Nhomspace{\Acomega} \int_\Omega \NX{\vo(\omega)}\NY{\vt(\omega)}\dd\PP(\omega)\nonumber\\
&\leq \NLiOhomspace{\toform\circ\coeff}\NLtOX{\vo}\NLtOY{\vt},\label{eq:Afirst}
\end{align}
which is finite by \cref{con:A}.

We now show $\SL\in  \LtOYas$ and $\SA:\LtOX \rightarrow \LtOYas.$ Observe that 

\noindent $\abs{\SL(\vt)} \leq \int_\Omega \abs{\Lcomega\mleft(\vt(\omega)\mright)}\dd\PP(\omega)$ and $\abs{\mleft[\SA \mleft(\vo\mright)\mright](\vt)} \leq \int_\Omega \abs{\mleft[\Acomega \vo(\omega)\mright]\mleft(\vt(\omega)\mright)}\dd\PP(\omega)$ and thus by \eqref{eq:Lfirst} and \eqref{eq:Afirst} $\SL$ and $\SA(\vo)$ are bounded. They are clearly linear, and so it follows that $\SL \in \LtOYas$ and $\SA(\vo)\in \LtOYas,$ i.e., $\SA:\LtOX \rightarrow \LtOYas.$
\epf

\bpf[Proof of \cref{thm:11}]
\label{page:thm11proof}
In order to show that $u$ solves \cref{prob:svar}, we must show:
\ben
\item\label[point]{it:111} either the functional $\SLc \in \LtOYas$ or the functional $\SA(u) \in \LtOYas$, and
\item\label[point]{it:112} the equality \eqref{eq:stoeq} holds.
\een

For \cref{it:111} we show that $\SL \in \LtOYas,$ (since this is easier than showing $\SA(u) \in \LtOYas$); in fact the proof of this is contained in the proof of \cref{lem:svarwelldefined}.

For \cref{it:112}, since $u$ solves \cref{prob:lt}, for $\PP$-almost every $\omega \in \Omega$ we have
%\beqs
$\Acomega u(\omega) = \Lcomega$
%\eeqs
in $\Ys.$ Hence, for any $v \in \LtOY$ we have
\beq\label{eq:midwaytoeu2}
\big[\Acomega u(\omega)\big]\big(v(\omega)\big) = \Lcomega\big(v(\omega)\big)
\eeq
for $\PP$-almost every $\omega \in \Omega.$ Since $\SL \in \LtOYas$, the right-hand side of \eqref{eq:midwaytoeu2} is a strongly measurable function with finite integral. Hence the left-hand side of \eqref{eq:midwaytoeu2} is as well, and we can integrate over $\Omega$ to conclude
%\beqs
$\big[\SAc u\big](v) = \SLc(v) \tforall v \in \LtOY,$
%\eeqs
that is, $\SAc u = \SLc$ in $\LtOYas.$
\epf
The following lemma is needed for the proof of \cref{thm:12}.
\ble\label{lem:settheory}
Let $\diff:\Omega\times Y \rightarrow \FF.$  For $y \in Y,$ define $\Omegay \de \set{\omega \in \Omega \st \diff(\omega,y)=0}$ and define $\Omegat \de \set{\omega \in \Omega \st \diff(\omega,y)=0 \tforall y \in Y}.$ If
\bit
\item for all $\omega \in \Omega,$ $\diff(\omega,\cdot)$ is a continuous functional on $Y$ and
\item for all $y \in Y,$ the map $\diff(\cdot,y):\Omega\rightarrow \FF$ is measurable and $\PP(\Omegay)=1,$
  \eit
  then $\PP(\Omegat)=1.$
\ele

\bpf[Proof of \cref{lem:settheory}]
We must show that the set $\Omegat \in \cF,$ and $\PP(\Omegat)=1.$ Observe that, for any $y \in Y$, the set $\Omegay \in \cF,$ since  $\Omegay = \delta(\cdot,y)^{-1}\mleft(\set{0}\mright),$ which  is the preimage under a measurable map of a measurable set. 

Since $Y$ is a Hilbert space, it is separable, and therefore it has a countable dense subset $\mleft(\yn\mright)_{n \in \NN}.$ We will show that $\PP\mleft(\cap_{n \in \NN} \Omegayn\mright)=1$ and $\Omegat = \cap_{n \in \NN} \Omegayn.$ The set $\cap_{n \in \NN} \Omegayn \in \cF,$ as $\cF$ is a $\sigma$-algebra and $\PP\mleft(\cup_{n \in \NN} \Omegaync\mright) \leq \sum_{n \in \NN} \PP\mleft(\Omegaync\mright) = 0,$ and hence $\PP\mleft(\cap_{n\in\NN} \Omegayn\mright)=1.$ To next show $\Omegat =  \cap_{n \in \NN} \Omegayn$ we observe that $\Omegat = \cap_{y \in Y} \Omegay$ and $\cap_{y \in Y} \Omegay \subseteq \cap_{n \in \NN} \Omegayn.$ It therefore suffices to show $\cap_{n \in \NN} \Omegayn \subseteq \cap_{y \in Y} \Omegay$ to conclude $\Omegat =  \cap_{n \in \NN} \Omegayn.$


Fix $y \in Y.$ By density of $\mleft(\yn\mright)_{n \in \NN}$, there exists a subsequence $\ynmseq$ such that $\ynm \rightarrow y$ as $m \rightarrow \infty.$ Fix $\omega \in \cap_{n \in \NN} \Omegayn.$ Note that $\omega \in \cap_{m \in \NN} \Omegaynm;$ that is, for all $m \in \NN,$ $\diff(\omega,\ynm) =0.$ As $\diff(\omega,\cdot)$ is a continuous function on $Y$, $\diff(\omega,\ynm) \rightarrow \diff(\omega,y)$ as $m \rightarrow \infty.$ But as previously noted, $\diff(\omega,\ynm)=0$ for all $m \in \NN.$ Hence we must have $\diff(\omega,y)=0,$ and thus $\omega \in \Omegay.$ Since $\omega \in \cap_{n \in \NN} \Omegayn$ was arbitrary, it follows that $\cap_{n \in \NN} \Omegayn \subseteq \Omegay,$ and since $y \in Y$ was arbitrary, it follows that $\cap_{n \in \NN} \Omegayn \subseteq \cap_{y \in Y} \Omegay$ as required.
\epf

\bpf[Proof of \cref{thm:12}]
\label{page:thm12proof}
Let $u \in \LtOX$ solve \cref{prob:svar}. We need to show that $u$ solves \cref{prob:lt}. Observe that $u$ solving \cref{prob:lt} means $\Acomega(u(\omega)) = \mleft(\Lcomega\mright)(\omega)$ in $\Ys$ for almost every $\omega \in \Omega.$ We now use an idea from \cite[Theorem 3.3]{Gi:10}. Our plan is to use test functions of the form $y\Ind{E},$ where $y \in Y$ and $E \in \cF$ to reduce \cref{prob:svar} to the statement
\beqs
\int_E \mleft[\Acomega\big(u(\omega)\big)\mright]\big(y(\omega)\big) \dd\PP(\omega) = \int_E \mleft[\mleft(\Lcomega\mright)(\omega)\mright]\big(y(\omega)\big) \dd\PP(\omega)\quad \tforall E \in \cF
\eeqs
and then show this implies $u$ satisfies \cref{prob:lt} via \cref{lem:gotoae}.

First define the space
%\beqs%\label{eq:Dspace}
$\D := \set{y\Ind{E} \st y \in Y, E \in \cF}.$
%\eeqs
It is straightforward to see that the elements of $\D$ are maps from $\Omega$ to $Y.$ The fact that $\D \subseteq \LtOY$ follows via the following three steps:

\ben
\item The elements of $\D$ are measurable, indeed the indicator function of a measurable set is a measurable function $\Omega\rightarrow\RR,$ and multiplication by $y \in Y$ is a continuous function $\RR\rightarrow Y.$ Hence elements of $\D$ are measurable by \cref{lem:contplusmeas}.
\item As $Y$ is a separable Hilbert space, it follows from \cref{cor:pettis} that the elements of $\D$ are strongly measurable.
\item $\NLtOY{y\Ind{E}} = \sqrt{\PP\mleft(E\mright)}\NY{y} < \infty$ for all $y \in Y, E \in \cF.$
  \een

  

Since \cref{prob:svar} is well-defined, and $u$ solves \cref{prob:svar}, and $\D \subseteq \LtOY,$ we have that $\mleft[\SA u\mright](v) = \SL(v) \tforall v \in \D.$ Therefore, we have

\beq\label{eq:initialint}
\int_\Omega \mleft[\Acomega\mleft(u(\omega)\mright)\mright]\mleft(y\Ind{E}(\omega)\mright) \dd\PP(\omega) = \int_\Omega \mleft[\Lcomega\mright]\mleft(y\Ind{E}(\omega)\mright) \dd\PP(\omega)
\eeq
for all $y \in Y$ and $E \in \cF.$ If we define $\diff:\Omega\times Y \rightarrow \FF$ by $\diff(\omega,y) \de \mleft[\Acomega\mleft(u(\omega)\mright) - \Lcomega\mright]\mleft(y\mright)$ then, by the definition of $\Ind{E},$ \eqref{eq:initialint} becomes
\beq\label{eq:intoverE}
\int_E \diff(\omega,y) \dd\PP(\omega)=0\quad \tforall E \in \cF.
\eeq
 To conclude $u$ solves \cref{prob:lt} we must show $\diff(\omega,y)=0$ for all $y \in Y,$ almost surely. We will use \cref{lem:gotoae}, so the first step is to show that for all $y \in Y$ $\diff(\cdot,y)$ is Bochner integrable. This follows from the fact that \cref{prob:svar} is well-defined, and thus the quantities $\big[\Acomega \vo(\omega)\big]\big(\vt(\omega)\big)$ and $\Lcomega\big(\vt(\omega)\big)$ are Bochner integrable for any $\vo\in \LtOX,\vt \in \LtOY.$ In particular, they are Bochner integrable when $\vo=u,$ and $\vt=y\Ind{E}$ and thus their difference $\diff$ is Bochner integrable. Secondly, $\diff(\omega,\cdot)$ is a continuous function on $Y$ since $\Acomega\mleft(u(\omega)\mright)$ and $\mleft(\Lcomega\mright)(\omega) \in \Ys,$ for all $\omega \in \Omega.$

We now show $\diff(\omega,y)=0$ for all $y \in Y,$ almost surely. For $y \in Y$ define the set $\Omegay \de \set{\omega \in \Omega \st \diff(\omega,y)=0};$ by \eqref{eq:intoverE} and \cref{lem:gotoae} we have that $\PP(\Omegay)=1$ for all $y \in Y.$ By \cref{lem:settheory}, $  \diff(\omega,y)=0$ for all $y \in Y$, almost surely, that is, $\Acomega u(\omega) = \Lcomega$ almost surely; it follows that $u$ solves \cref{prob:lt}.
\epf

\bre[Connection with the argument in {\cite[Remark 2.2]{MuSt:11}}]
The argument in 

\noindent \cref{lem:settheory} and the final part of \cref{thm:12} closely mirrors the result in \cite[Remark 2.2]{MuSt:11}. Indeed, we prove in general that
\beqs
\PP\big(\diff(\omega,y)=0\big)=1\text{ for all } y \in Y \quad\text{implies} \quad\PP\big(\diff(\omega,y)=0\text{ for all } y \in Y\big)=1,
\eeqs
and \cite[Remark 2.2]{MuSt:11} shows an analogous result for the stationary diffusion equation \eqref{eq:diffusion} with non-uniformly coercive and unbounded coefficient $\kappa.$
\ere

\bpf[Proof of \cref{lem:uniq}]
\label{page:lemuniqproof}
\emph{Proof of \cref{part:uniq1}.} Suppose $\uo,\ut:\Omega\rightarrow X$ solve \cref{prob:meas}. Let $E = \set{\omega \in \Omega \st \uo(\omega) \neq \ut(\omega)}.$ Denote by $\Eo$ and $\Et$ the sets (of measure zero) where the variational problems for $\uo$ and $\ut$ fail to hold, i.e. $\Eo,\Et \in \cF$ with $\PP(\Eo)=\PP(\Et)=0$ and %such that  and
\beqs
\Acomega\mleft(\uo(\omega)\mright) \neq \Lcomega \tiff \omega \in \Eo,\quad\text{and}\quad \Acomega\mleft(\ut(\omega)\mright) \neq \Lcomega \tiff \omega \in \Et.
\eeqs As  $\ker\mleft(\Acomega\mright) = \set{0}$ $\PP$-almost surely, there exists $\Eth \in \cF$ such that $\PP(\Eth) = 0$ and
%\beqs

\noindent $\ker\mleft(\Acomega\mright) \neq \set{0} \tiff \omega \in \Eth.$
%\eeqs
We claim $E \subseteq \Eo\cup\Et\cup\Eth.$ Indeed, if $\uo(\omega) \neq \ut(\omega)$ then either: (i) at least one of $\uo$ and $\ut$ does not solve \cref{prob:meas} at $\omega$ or (ii) $\uo$ and $\ut$ both solve \cref{prob:meas} at $\omega,$ but $\ker\mleft(\Acomega\mright) \neq \set{0}.$
Since $\PP(E_j)=0, j = 1,2,3,$ we have $\PP(\Eo\cup\Et\cup\Eth) = 0.$  Therefore $E \in \cF$ and $\PP(E)=0$ since $\OFP$ is a complete probability space; hence $\uo = \ut$ almost surely, as required.

\emph{Proof of \cref{part:uniq2}.} By \cref{rem:imm}, if $\uo,\ut \in \LtOX$ solve \cref{prob:lt}, then all the representatives of the equivalence classes of $\uo$ and $\ut$ solve \cref{prob:meas}. Hence, by \cref{part:uniq1}, any representative of $\uo$ and any representative of $\ut$ differ only on some set (depending on the representatives) of $\PP$-measure zero in $\Omega.$ Therefore $\uo=\ut$ in $\LtOX,$ by definition of $\LtOX.$

\emph{Proof of \cref{part:uniq3}.} As \cref{prob:svar} is well-defined, by \cref{rem:imm,thm:12}, if $\uo$ and $\ut$ solve \cref{prob:svar}, then $\uo$ and $\ut$ also solve \cref{prob:meas}. We then repeat the reasoning in the proof of \cref{part:uniq2} to show $\uo=\ut$ in $\LtOX.$
\epf


\section{Proofs of \cref{thm:hh-gen,thm:hh-hetero}}\label{sec:hhproof}
In \cref{sec:placing} we place the Helmholtz stochastic EDP into the framework developed in \cref{sec:gen-framework}. In \cref{sec:hh-cond} we give sufficient conditions for the Helmholtz stochastic EDP to satisfy \cref{con:coeffstoform,,con:coeffstofunc,,con:cborel}, etc.. In \cref{sec:applying} we apply the general theory developed in \cref{sec:gen-framework} to prove \cref{thm:hh-gen,,thm:hh-hetero}.

\subsection{Placing the Helmholtz stochastic EDP into the framework of \cref{sec:gen-framework}}\label{sec:placing}
Recall $R>0$ is fixed. We let $X=Y=\HozDDR$ and define the norm $\NHokDR{v}^2 \de \NLtDR{\grad v}^2 + k^2 \NLtDR{v}^2$ on $\HozDDR.$ Throughout this section, $\Az,\nz,$ and $\fz$ will be deterministic functions. Recall that since the supports of $1-n,$ $I-A,$ and $f$ are compactly contained in $\BR$, we can consider $A, n,$ and $f$ as functions on $\DR$ rather than on $\Dp$. In order to define the space $\cC$ and the maps $\coeff,\toform,$ and $\rhs$ we define the following function spaces on $\DR$.

\bde[Compact-support spaces]\label{def:compsuppspace}
Let
\beqs
\LtRDR \de \set{\fz \in \LtDR \st \supp\mleft(\fz\mright) \compcont \BR}.
\eeqs
\begin{align*}
\LiRminDRRR \de \big\{&\nz \in \LiDRRR \st\supp\mleft(1-\nz\mright) \compcont \BR,\\
&\text{there exists } \alphanz > 0 \text{ such that } \nz(\bx) \geq \alphanz \text{ almost everywhere }\big\},\\
\hspace{-3cm}\LiRminDRRRdtd \de \Big\{&\Az \in \LiDRRRdtd \st\Az(\bx) \text{ is symmetric almost everywhere,}\\
&\supp\mleft(I-\Az\mright) \compcont \BR, \text{there exists } \alphaAz>0 \text{ s.~t.~}\alphaAz \leq \Az(\bx)\\
&\text{almost everywhere, in the sense of quadratic forms}\Big\},\text{ and}\\
%\end{align*}
%\begin{align*}
&\hspace{-3cm}\WoiRminDRRRdtd \de \set{\Az \in \LiRminDRRRdtd \st\Az \in \WoiDRRRdtd}.
\end{align*}
\ede

Observe that the norm on $\LiDRRR$ induces a metric on $\LiRminDRRR,$ and similarly for $\LiRDRRRdtd,$ $\WoiRminDRRRdtd,$ and $\LtRDR.$ These spaces are not vector spaces, and are not complete, but completeness and being a vector space is not required in what follows---we only need them to be metric spaces.


\bde[Deterministic form and functional]

\noindent For $\mleft(\Az,\nz,\fz\mright) \in \LiRDRRRdtd \times \LiRminDRRR \times \LtRDR$ let the sesquilinear form $\aGm$ on $\HozDDR \times \HozDDR$  and the antilinear functional $\Lh$ on $\HozDDR$ be given by
\begin{align*}
\aGm\mleft(\vo,\vt\mright) &\de \int_{D_R} \Big(\mleft(\Az \grad \vo\mright)\cdot \grad \vtb \rangle 
 - k^2 \nz\, \vo\,\vtb \Big)\dd\Leb- \big\langle T_R \gamma \vo,\gamma \vt\big\rangle_{\Gamma_R}, \quad\text{and}\\
 \Lh(\vt) &\de \int_{D_R} \fz\, \vtb\,\dd\Leb, \quad\text{ for } \vo, \vt \in \HozDDR.
\end{align*}
%for $\vo,\vt \in \HozDDR.$
\ede

\bprob[Helmholtz EDP]\label{prob:edpstoch}
For $\mleft(\Az,\nz,\fz\mright) \in \LiRDRRRdtd \times \LiRDRRR \times \LtRDR$ find 
$\uz \in \HozDDR$ such that $\aGm(\uz,v) = \Lh(v) \tforall v \in \HozDDR.$
\eprob

\bde[$\dinfty$ metric]\label{def:dinfty}
Let $\mleft(\Xo,\done\mright),\ldots,\mleft(\Xm,\dm\mright)$ be metric spaces. The \defn{$\dinfty$ metric} on the Cartesian product $\Xo\times\cdots\times\Xm$ is defined by
\beqs
\dinfty\mleft(\mleft(\xo,\ldots,\xm\mright),\mleft(\yo,\ldots,\ym\mright)\mright) \de \max_{j = 1,\ldots,m} \dmetj\mleft(\xj,\yj\mright).
\eeqs
\ede


\bde[The input space $\cC$]\label{def:cCHh}
We let
%\beqs
$\cC \de \WoiRminDRRRdtd \times \LiRminDRRR \times \LtRDR$
%\eeqs
with topology given by the $\dinfty$ metric.
\ede


\bde[The input map $\coeff$]\label{def:inputmap}
Define $\coeff:\Omega\rightarrow\cC$ by
%\beq\label{eq:hh-c}
$\coeff(\omega) = \mleft(A(\omega),n(\omega),f(\omega)\mright).$
%\eeq
\ede

\bde[The maps $\toform$ and $\rhs$ for the Helmholtz stochastic EDP]
Let
\beq\label{eq:hhform}
\toform\mleft(\mleft(\Az,\nz,\fz\mright)\mright) \de \aGm \quad \text{and} \quad \rhs\mleft(\mleft(\Az,\nz,\fz\mright)\mright) \de \Lh,
\eeq
where the definition of $\toform$ is understood in terms of the equivalence between $\homspace$ and sesquilinear forms on $X \times Y.$
\ede

\subsection{Verifying the Helmholtz stochastic EDP satisfies the general conditions in \cref{sec:gen-framework}}\label{sec:hh-cond}

\ble[{\Cref{con:cborel,con:C}} for Helmholtz stochastic EDP]\label{lem:hh-borelC}
If $A,n,$ and $f$ are strongly measurable, then $\coeff$ defined by  \cref{def:inputmap} satisfies \cref{con:cborel,con:C}.
\ele

\bpf
Since $A,n,$ and $f$ are strongly measurable, by \cref{thm:pettis} they are measurable and $\PP$-essentially separably valued. By \cref{lem:measprod}, it follows that $\coeff$ is measurable, so $\coeff$ satisfies \cref{con:cborel}. By \cref{lem:prodsep}, it follows that $\coeff$ is $\PP$-essentially separably valued, so $\coeff$ satisfies \cref{con:C}.
\epf

\ble[\Cref{con:coeffstoform,con:coeffstofunc} for Helmholtz stochastic EDP]\label{lem:hh-AL}
The maps $\toform$ and $\rhs$ given by \eqref{eq:hhform} satisfy \cref{con:coeffstoform,con:coeffstofunc}.
\ele

\bpf[Proof of \cref{lem:hh-AL}]
We need to show that if $(\Am,\nm,\fm) \rightarrow (\Az,\nz,\fz)$ in $\cC$ then $\toform((\Am,\nm,\fm)) \rightarrow \toform((\Az,\nz,\fz))$ in $\homspace,$ and similarly for $\rhs.$ We have, for $\vo \in X,\vt \in Y,$
\begin{align*}
&\biggabs{\Big[\big[\toform\mleft(\Am,\nm,\fm\mright) - \toform\mleft(\Az,\nz,\fz\mright)\big]\mleft(\vo\mright)\Big]\mleft(\vt\mright)}\\
&\quad\quad\quad\,\,= \abs{\int_{\DR} \Big(\big(\mleft(\Am-\Az\mright)\grad \vo\big)\cdot\grad\vtb- k^2 \mleft(\nm-\nz\mright) \vo\vtb\Big)\dd\Leb}\\
&\quad\quad\quad\,\,\leq\NLiDRRRdtd{\Am-\Az}\NLtDR{\grad \vo}\NLtDR{\grad \vt} 
\\
&\quad\quad\quad\,\,\quad\quad
+k^2\NLiDRRR{\nm-\nz}\NLtDR{\vo}\NLtDR{\vt}\\
&\quad\quad\quad\,\,\leq 2\dinfty((\Am,\nm,\fm),(\Az,\nz,\fz))\NHokDR{\vo}\NHokDR{\vt},
\end{align*}
Hence if $(\Am,\nm,\fm) \rightarrow (\Az,\nz,\fz)$ in $\cC,$ then $\toform((\Am,\nm,\fm)) \rightarrow \toform((\Az,\nz,\fz))$ in 

\noindent $\homspace.$ We also have
\beqs
\Bigabs{\big[\rhs\mleft(\mleft(\Am,\nm,\fm\mright),\mright) - \rhs\mleft(\mleft(\Az,\nz,\fz\mright)\mright)\big]\mleft(\vt\mright)} 
 = \abs{\int_{\DR} \mleft(\fm- \fz\mright)\vtb\,\dd\Leb}
 %&\hspace{-2cm}\leq\NLtDR{\fm-\fz}\NLtDR{\vt}
 \leq \NLtDR{\fm-\fz} \frac{\NHokDR{\vt}}{k}.
\eeqs
Hence if $(\Am,\nm,\fm) \rightarrow (\Az,\nz,\fz)$ in $\cC,$ then $\rhs((\Am,\nm,\fm)) \rightarrow \rhs((\Az,\nz,\fz))$ in $\Ys.$
\epf

\bde[The solution operator $\sol$]
Define $\sol:\cC\rightarrow \HozDDR$ by letting 

\noindent $\sol\mleft(\Az,\nz,\fz\mright) \in \HozDDR$ be the solution of the Helmholtz EDP (\cref{prob:edpstoch}).
\ede

\bth[$\sol$ is well defined]\label{thm:deteu}
For $\mleft(\Az,\nz,\fz\mright) \in \cC$ the solution $\sol\mleft(\mleft(\Az,\nz,\fz\mright)\mright)$ of the Helmholtz EDP (\cref{prob:edpstoch}) exists, is unique, and depends continuously on $\fz.$
\enth

\bpf[Proof of \cref{thm:deteu}] Since $\Real{-\langle \TrR \gamma v,\gamma v\rangle_{\GR}} \geq 0$ for all $v \in \HozDDR$ (see, e.g.~\cite[Theorem 2.6.4]{Ne:01}), $\aGm$ satisfies a G\r{a}rding inequality. Since the inclusion $\HozDDR \hookrightarrow \LtDR$ is compact, Fredholm theory shows that uniqueness implies well-posedness (see, e.g.~\cite[Theorem 2.34]{Mc:00}). Since $A$ is Lipschitz and $n$ is $L^\infty$, uniqueness follows from the unique continuation results in  \cite{JeKe:85,GaLi:87};
%\cite{GaLi:87};
see \cite[Section 2]{GrSa:18} for these results specifically applied to Helmholtz problems.
\epf


\ble[Continuity of solution operator for Helmholtz stochastic EDP]\label{lem:solcont}
For the 

\noindent  Helmholtz stochastic EDP, the solution operator $\sol:\cC\rightarrow\HozDDR$ is continuous.
\ele

\bpf[Sketch Proof of \cref{lem:solcont}]
Let $(\Az,\nz,\fz), (\Ao,\no,\fo) \in \cC,$ with $\sol((\Az,\nz,\fz)) = \uz$ and $\sol((\Ao,\no,\fo)) = \uo.$
Then for any $v \in \HozDDR$ we have, for $j=0,1,$

\beqs\label{eq:S1}
%\noindent$\label{eq:S1}
\mleft[\mleft[\toform((A_j, n_j, f_j))\mright](u_j)\mright](v) = \mleft[\rhs((A_j,n_j,f_j))\mright](v).% \tand
%\mleft[\mleft[\toform((\Ao,\no,\fo))\mright](\uo)\mright](v) = \mleft[\rhs((\Ao,\no,\fo))\mright](v).
\eeqs%$
Continuity of $\sol$ then follows from:

\ben
\item Deriving the Helmholtz equation with coefficients $\Az$ and $\nz$ satisfied by $\ud\de\uz-\uo.$
\item\label[point]{it:contrecall} Recalling that the well-posedness result of \cref{thm:deteu} holds when $\fz \in \LtRDR$ is replaced by a right-hand side in $(\HozDDR)^*$; see, e.g., \cite[Theorem 2.34]{Mc:00}.
\item Applying the result in \cref{it:contrecall} to obtain a bound
%  \beqs
  $\NHokDR{\ud} \leq C(\Az,\nz) \NHozDDRas{F}.$
%  \eeqs

\item Showing $\NHozDDRas{F}$ depends on $\NLtDR{\grad \uo},$ $\NLtDR{\uo},$ $\NLiDRRRdtd{\Ao-\Az},$ $\NLiDRRR{\no-\nz},$ and $\NLtD{\fz-\fo}.$
\item Eliminating the dependence on $\uo$ by writing $\uo = \uz-\ud$ and moving terms in $\ud$ to the left-hand side, to obtain a bound on $\ud$ of the form
  \begin{align*}
&\NLtDR{\grad \ud} + k\NLtDR{\ud} \\
&\qquad\leq \Ctilde\Big(\uz,\Az,\nz,\NLiDRRRdtd{\Ao-\Az},\NLiDRRR{\no-\nz},\NLtDR{\fz-\fo}\Big).
\end{align*}
\item Concluding that $\ud\rightarrow0$ in $\HozDDR$ as $(\Ao,\no,\fo) \rightarrow (\Az,\nz,\fz)$ in $\cC.$
\een\epf

\ble[\Cref{con:K} for the Helmholtz stochastic EDP]\label{lem:hh-K}
The Helmholtz stochastic EDP satisfies \cref{con:K}.
\ele

\bpf[Proof of \cref{lem:hh-K}]
This condition holds immediately from \cref{thm:deteu}.
%The fact that this %uniqueness 
%condition holds is immediate from \cref{thm:deteu}.
\epf

To prove that \cref{con:B} holds for the Helmholtz stochastic EDP, we first state the deterministic analogues of \cref{con:hh-hetero,thm:hh-hetero}.

\bcon[Nontrapping condition for Helmholtz EDP {\cite[Condition 2.4]{GrPeSp:19}}]\label{cond:1}
$d=2,3$, $\Dm$ is star-shaped with respect to the origin, $\Az\in \WoiDRRRdtd$, $\nz\in \WoiDRRR$, and there exist $\consto, \constt>0$ such that,
for almost every $\bx\in \Dp$,
\begin{align}\label{eq:A1}
&\Az(\bx) - (\bx\cdot\nabla)\Az(\bx) \geq \consto, \text{ in the sense of quadratic forms, and }\\
\label{eq:n1}
& \hspace{2cm}\nz(\bx)+ \bx\cdot\nabla \nz(\bx) \geq \constt. %\text{ for almost every }\bx\in \Dp.
\end{align}
\econ

\bth[Well-posedness of the Helmholtz EDP under \cref{cond:1} {\cite[Theorem 2.5]{GrPeSp:19}}]\label{thm:eubedp}
Let $(\Az,\nz,\fz) \in \cC$ and suppose $\Az$ and $\nz$ satisfy \cref{cond:1}. Then the solution of the Helmholtz EDP (\cref{prob:edpstoch}) exists and is unique. Furthermore, given $\kz > 0$ for all $k \geq \kz,$ the solution $\uz$ of the Helmholtz EDP satisfies the bound 
\beq\label{eq:heterobound1}
\consto \NLtDR{\grad \uz}^2 + \constt k^2 \NLtDR{\uz}^2\leq \Co \NLtDR{\fz}^2,\,\, \text{where} \,\, \Co := 4\mleft[\frac{R^2}{\consto} + \frac{1}{\constt}\mleft(R+ \frac{d-1}{2\kz}\mright)^2\mright].
\eeq
\enth

We can now prove \cref{con:B} holds for the Helmholtz stochastic EDP.

\ble[\Cref{con:B} for Helmholtz stochastic EDP]\label{lem:hh-B}
If \cref{con:hh-fAn,con:hh-hetero} hold, then \cref{con:B} holds for the Helmholtz stochastic EDP.
\ele

\bpf[Proof of \cref{lem:hh-B}]
As \cref{con:hh-hetero} holds,  \cref{cond:1} holds for $\PP$-almost every $\omega \in \Omega$ (with $\Az = A(\omega),$ $\nz = n(\omega),$ $\consto = \muo(\omega),$ and $\constt = \mut(\omega)$). Hence, by \cref{thm:eubedp} the bound \eqref{eq:sbe1} holds for all $k \geq \kz$, with $X = \HozDDR, m=1,$
\beqs
\Co(\omega) = \frac4{\min\set{\muo(\omega),\mut(\omega)}}\mleft[\frac{R^2}{\muo(\omega)} + \frac{1}{\mut(\omega)}\mleft(R+ \frac{d-1}{2\kz}\mright)^2\mright],
\eeqs
and $\fo = \NLtDR{f(\omega)}^2.$ It now remains to show that $\Co\,\NLtDR{f}^2 \in \LoO.$ We first show $\Co\,\NLtDR{f}^2$ is measurable and then show that it lies in $\LoO.$ To show measurability, we rewrite $\Co(\omega)$ as 
\beqs
\Co(\omega) =  \max\left\{\frac{2 R^2}{\mu_1^2(\omega)} + \frac{2}{\mu_1(\omega)\mu_2(\omega)}\left(R+ \frac{d-1}{2\kz}\right)^2,\frac{2 R^2}{\mu_1(\omega)\mu_2(\omega)} + \frac{2}{\mu_2^2(\omega)}\left(R+ \frac{d-1}{2\kz}\right)^2\right\}.
\eeqs
The functions $\muo^{-1}$ and $\mut^{-1}$ are measurable by assumption; to conclude $\Co$ is measurable we use the facts (see e.g. \cite[Theorems 19.C, 20.A]{Ha:74}): (i) the square of a measurable function is measurable, and (ii) the product, sum, and maximum of two measurable functions are measurable. Under 
\cref{con:hh-fAn}, the function $f$ lies in the Bochner space $\LtOLtDR.$ Therefore, $f$ is strongly measurable and hence $f$ is measurable  by \cref{thm:pettis}. The map $f\mapsto\NLtDR{f}^2$ is clearly continuous, and therefore $\fo$ is measurable by \cref{lem:contplusmeas}. As the product of two measurable functions is measurable, it follows that $\Co\,\NLtDR{f}^2$ is measurable.

We now show that $\Co\NLtDR{f}^2 \in \LoO.$ The assumptions $1/\muo,1/\mut \in \LtO$ and the Cauchy--Schwarz inequality imply $1/(\muo\mut) \in \LoO.$ Therefore the maps,
\beqs\omega \mapsto \frac{2 R^2}{\mu_1^2(\omega)} + \frac{2}{\mu_1(\omega)\mu_2(\omega)}\left(R+ \frac{d-1}{2\kz}\right)^2 \text{ and } \omega \mapsto \frac{2 R^2}{\mu_1(\omega)\mu_2(\omega)} + \frac{2}{\mu_2^2(\omega)}\left(R+ \frac{d-1}{2\kz}\right)^2
\eeqs
are in $\LoO.$ Since the maximum of two functions in $\LoO$ is also in $\LoO,$ it follows that $\Co \in \LoO.$ \cref{con:hh-fAn} implies that $\NLtDR{f}^2 \in \LoO.$

To conclude $\Co\NLtDR{f}^2 \in \LoO,$ observe that the only dependence of $\Co$ on $\omega$ is through $\muo$ and $\mut.$ As $\muo$ and $\mut$ are assumed independent of $f,$ and measurable functions of independent random variables are independent \cite[p.236]{Lo:77} it follows that $\Co$ and $\NLtDR{f}^2$ are independent, and therefore
\begin{align}\nonumber
&\hspace{-2ex}\NLoO{\Co\NLtDR{f}^2} = \int_\Omega \Co(\omega) \NLtDR{f(\omega)}^2 \dd\PP(\omega) 
\\ &= \mleft(\int_\Omega \Co(\omega) \dd\PP(\omega)\mright)\mleft(\int_\Omega \NLtDR{f(\omega)}^2 \dd\PP(\omega)\mright)= \NLoO{\Co}\NLoO{\NLtDR{f}^2} < \infty.\label{eq:normsplit}
\end{align}
Therefore $\Co\NLtD{f}^2 \in \LoO$ as required. We take the expectation (equivalently, the $L^1$ norm) of \eqref{eq:heterobound1} (with $\Az = A(\omega)$ etc.) and use \eqref{eq:normsplit} to obtain \eqref{eq:Sbound1}.
\epf

\bre[The case when $f,$ $\muo,$ and $\mut$ are not independent]\label{rem:notindep}
\Cref{rem:planewave} shows that for the physically relevant example of scattering by a plane wave, $f,$ $\muo,$ and $\mut$ may not be independent. In this case, if we replace the requirements in \cref{con:hh-hetero} that $f \in \LtOLtD$ and $1/\muo,\,1/\mut \in \LtO$ with the stronger requirements $f \in \LfOLtD$ and $1/\muo,\,1/\mut \in \LfO$, then one can obtain the bound
\beqs
\NLtOHozDDR{\grad u}^2 + k^2\NLtOHozDDR{ u}^2\leq \NLtO{\Co} \NLfOLtDR{f}^2.
\eeqs
Indeed, instead of independence, we use the Cauchy--Schwartz inequality in \eqref{eq:normsplit} to conclude
\beqs
\NLoO{\Co \NLtDR{f}^2}  \leq \NLtO{\Co}\NLtO{\NLtDR{f}^2} = \NLtO{\Co} \NLfOLtDR{f}^2.
\eeqs
\ere

\ble[\Cref{con:L} for Helmholtz stochastic EDP]\label{lem:hh-L}
If $f \in \LtOLtDR$ and $A$ and $n$ are strongly measurable, then \cref{con:L} holds for the Helmholtz stochastic EDP.
\ele

\bpf[Proof of \cref{lem:hh-L}]
Since $A,n,$ and $f$ are strongly measurable, \cref{con:cborel,con:C} hold by \cref{lem:hh-borelC}; i.e., $\coeff$ is both measurable and $\PP$-essentially separably valued. Furthermore, by \cref{thm:pettis} $\coeff$ is strongly measurable. By \cref{lem:hh-AL}, \cref{con:coeffstofunc} holds, so the map $\rhs$ is continuous. Hence, by \cref{lem:contplusstrong}, $\rhs\circ\coeff$ is strongly measurable. We also have that
%\beqs
$\NYs{\mleft(\rhs\circ\coeff\mright)(\omega)} = \NLtDR{f(\omega)}/k,$
%\eeqs
and thus $\rhs\circ\coeff \in \LtOYs$ since $f \in \LtOLtDR,$ i.e.~\cref{con:L} holds.
\epf

\ble[\Cref{con:A} for the Helmholtz stochastic EDP]\label{lem:hh-A}

\noindent If $A \in \LiOLiDRRRdtd,$ $n \in \LiOLiDRRR,$ and $f$ is strongly measurable, then \cref{con:A} holds for the Helmholtz stochastic EDP.
\ele

\bpf[Proof of \cref{lem:hh-A}]
A near-identical argument to that at the beginning of the proof of \cref{lem:hh-L} shows $\toform\circ\coeff$ is strongly measurable. Recall that the Dirichlet-to-Neumann operator $\TrR$ is continuous from $\HhGR$ to $\HmhGR,$ see e.g.~\cite[Theorem 2.6.4]{Ne:01}. Let $\vo \in X, \vt \in Y,$ and observe that the Cauchy--Schwartz inequality and these properties of $\TrR$
%, and the fact that the norm on $\LiDRRRdtd$ is defined as 
%\beqs
%\NLiDRRRdtd{\Az} = \esssup_{\bx \in \DR}\NopCCd{\Az(\bx)}
%\eeqs
imply that there exists $C(k) > 0$ such that 
\begin{align*}
\bigg|\Big[\mleft[\Acomega\mright](\vo)\Big](\vt)\bigg| &= \abs{\int_{\DR} \Big(\mleft(A(\omega) \grad \vo\mright)\cdot\grad \vtb - k^2n(\omega) \vo\vtb \Big)\dd\Leb- \IPGRbig{\DtN \vo}{\vt}}\\
&\hspace{-1cm}\leq \NLiDRRRdtd{A(\omega)}  \NLtDR{\grad \vo}\NLtDR{\grad \vt}\\
&\hspace{-1cm}\quad+ k^2 \NLiDRRR{n(\omega)}\NLtDR{\vo}\NLtDR{\vt}+ C(k) \NHhGR{\gamma\vo}\NHhGR{\gamma\vt},
\end{align*}
where we have used the fact that the two norms
\beq\label{eq:normsdef}
 \esssup_{\bx \in \DR} \NopCCd{A(\omega,\bx)} \quad\tand\quad\NLiDRRRdtd{A(\omega)} \de \max_{i,j \in \set{1,\ldots,d}} \NLiDRRR{A_{i,j}(\omega)}
\eeq
are equivalent.
Since the trace operator $\gamma$ is continuous from $\HoDR$ to $\HhGR$ (see, e.g.~\cite[Theorem 3.38]{Mc:00}), there exists $\Ctilde > 0$ such that
\beqs
\Nhomspace{\mleft(\toform\circ\coeff\mright)(\omega)}\leq\Ctilde\max\set{\NLiDRRRdtd{A(\omega)},\NLiDRRR{n(\omega)},C(k)}\NHokDR{\vo}\NHokDR{\vt}.
\eeqs
and hence $\toform\circ\coeff \in \LiOhomspace.$
\epf

\subsection{Proofs of \cref{thm:hh-gen,thm:hh-hetero}}\label{sec:applying}

\bpf[Proof of \cref{thm:hh-gen}]
We construct a solution of \cref{prob:msedp} by letting $u = \sol\circ\coeff$ (which is well-defined by \cref{thm:deteu}), and observe that, by construction, $\mleft[a(\omega)\mright]\mleft(u(\omega),v\mright) = \mleft[L(\omega)\mright](v)$ for all $v \in \HozDDR$ almost surely. It follows that $u$ is measurable by \cref{con:hh-fAn,lem:solcont,lem:solcont,lem:contplusmeas}, and so $u$ solves \cref{prob:msedp}. We therefore proceed to apply the general theory.

\Cref{con:coeffstoform,con:coeffstofunc} hold by \cref{lem:hh-AL};
\cref{con:A} holds by \cref{lem:hh-A};
\cref{con:L} holds by \cref{lem:hh-L};
 \cref{con:cborel,con:C} hold by \cref{lem:hh-borelC,con:hh-fAn};
and \cref{con:K} holds by \cref{lem:hh-K}. Therefore we can apply \cref{thm:11,thm:12,lem:svarwelldefined,lem:uniq} to conclude the results.
\epf

\bpf[Proof of \cref{thm:hh-hetero}]
All the conclusions of \cref{thm:hh-gen} hold, and we only need to show that if $u$ solves \cref{prob:msedp} then it also solves \cref{prob:somsedp}. \Cref{con:B} holds by \cref{con:hh-fAn,con:hh-hetero,lem:hh-B}. The result then follows from \cref{thm:3}.
\epf





\chapter{Nearby Preconditioning for the Helmholtz Equation}\label{chap:nbpc}
\chaptermark{Nearby Preconditioning}
\section{Introduction and Motivation from UQ}\label{sec:intronbpc}

\subsection{Motivation from uncertainty quantification for the Helmholtz equation} 
Consider the stochastic Helmholtz equation 
\beq\label{eq:nbpchh}
\nabla\cdot\big(A(\omega,\bx) \nabla u(\omega,\bx) \big) + k^2 n(\omega,\bx) u(\omega,\bx) =-f(\bx), \quad \bx\in\Dp,
\eeq
as defined in \cref{chap:stochastic}. If $Q(u)$ is some quantity of interest of the solution, then the simplest way to approximate $\EXP{Q(u)}$ is via a sampling-based method, i.e. using the approximation
\beq\label{eq:samplingexp}
\EXP{Q(u)} \approx \frac1N \sum_{l=1}^N Q(u(\omegasl)),
\eeq
where the $\omegasl$ are elements of the sample space $\Omega.$ To calculate the right-hand side of \cref{eq:samplingexp},  one must solve many deterministic Helmholtz problems, corresponding to different samples $\omega^l$, i.e. corresponding to different realisations of the coefficients $A(\omega,\cdot)$ and $n(\omega,\cdot)$.
Solving all these deterministic problems is a very computationally-intensive task because linear systems arising from discretisations of the Helmholtz equation are notoriously difficult to solve; see the discussion in \cref{sec:numsolve} above. In particular, direct solvers involving a sparse LU decomposition of the linear system have a computational cost of the order $\cO\mleft(N^{3/2}\mright)$ in 2-d  and $\cO\mleft(N^2\mright)$ in 3-d (see \cite[Section 1]{DuErRe:76} and \cite[Equation 3]{DuErRe:76}, respectively, for a particular regular grid).

However, if one already has access to the LU decomposition, then the cost of applying a direct solver using the LU decomposition is much cheaper; $\cO\mleft(N\log N\mright)$ in 2-d \cite[Section 1]{DuErRe:76} and $\cO\mleft(N^{4/3}\mright)$ in 3-d \cite[Equation 4]{DuErRe:76}. In the context of Uncertainty Quantification for the Helmholtz equation this reduction in cost when one has access to an LU decomposition suggests the following question: When can the LU decomposition corresponding to a particular realisation of \cref{eq:nbpchh} be used as a preconditioner for other realisations of \cref{eq:nbpchh}?

This question of reusing preconditioners is more widely applicable than just for LU decompositions. For \emph{any} preconditioner for the Helmholtz equation, one could ask when the preconditioner corresponding to one realisation of \cref{eq:nbpchh} can be re-used for other realisations. In this \lcnamecref{chap:nbpc}, for simplicity, we restrict our attention to the case where the preconditioner is an exact LU decomposition.

One expects this reuse of the preconditioner to work well if the two realisations are `nearby' in some sense. This idea of reusing preconditioners is the `nearby preconditioning' strategy proposed in this \lcnamecref{chap:nbpc}. To analyse this `nearby preconditioning' strategy rigorously, we first consider the following problem and question.

%% Therefore, in this \lcnamecref{chap:nbpc} we consider whether one can reuse the preconditioner from one solve for subsequent solves, and thereby reduce the overall computational effort.  

%%  We note that the above suggestion of re-using preconditioners should, at first glance, reduce the computational effort needed because Helmholtz preconditioners typically require more effort to \emph{construct} than they do to \emph{apply}.

%%  For example, if we calculate the action of $\AmatI$ exactly (say via an $LU$ factorisation) then (in 3-d) the cost of calculating the preconditioner is $\cO\mleft(N^2\mright)$ and the cost of applying the preconditioner is $\cO\mleft(N^{4/3}\mright)$ (where $N$ is the number of unknowns), see, e.g., \cite[Section 4]{GaZh:19}. This relationship between the cost of construction and application is also seen in recent Helmholtz preconditioners such as a sweeping preconditioners and a domain-decompostion preconditioners (see the recent review article \cite{GaZh:19} for an overview of many types of Helmholtz preconditioners). For these preconditioners one still performs $LU$-factorisations (or direct solves) on subdomains of $\Dp$ (see, e.g., \cite[Section 2]{GaZh:19}). Therefore resuing preconditioners, and thereby calculating fewer $LU$ factorisations may still lead to considerable savings. .

Let $\Aj, \nj$, $j=1,2$ satisfy the properties of $A$ and $n$ in \cref{prob:vedp} or \cref{prob:vtedp} (we will prove results for both problems), with $\uj$ the corresponding solution and $\Dm,$ $f$, etc. as in \cref{prob:vedp} or \cref{prob:vtedp}. Let $\Amatj$, $j=1,2,$ be the Galerkin matrices for the corresponding $h$-finite-element discretisations (see \cref{eq:matrixAjdef} below for a precise definition of $\Amatj$). We seek to answer:

% Inspired by the usage of enumitem here: https://tex.stackexchange.com/a/58714
\ben[label=Q1., ref=Q1]
\item\label[itemblank]{it:nbpcq1} How small must $\N{\Aso - \Ast}$ and 
$\N{\nso - \nst}$ be (in some norm to be defined, in terms of $k$-dependence) for GMRES 
applied to $(\Amat^{(1)})^{-1}\Amat^{(2)}$ to converge in a $k$-independent number of iterations
%$ to be a good preconditioner for $\Amat^{(2)}$
 for arbitrarily large $k$? 
 \een

 The rigorous answer of \cref{it:nbpcq1} is contained in \cref{cor:1,cor:1a} below. However, an informal statement of the answer to \cref{it:nbpcq1} is that if
 \beq\label{eq:informalconditionnbpc}
k\,
\NLi{\Aso-\Ast} \quad\text{ and } \quad k\,\NLi{\nso-\nst}
%\Big) 
\quad\text{ are both sufficiently small,}
\eeq
then GMRES applied to $(\Amat^{(1)})^{-1}\Amat^{(2)}$ in some weighted norm converges in a $k$-independent number of iterations (and a similar result for standard GMRES with \cref{eq:informalconditionnbpc} replaced by a slightly stronger condition).

 \subsection{Outline of the chapter}% Don't know why cleveref didn't work here
In \cref{sec:main} we state and discuss the main results of this \lcnamecref{chap:nbpc} on the effectiveness of nearby preconditioning, and give their analogues on the PDE level. In \cref{sec:num} we describe numerical experiments investigating the sharpness of the nearby-preconditioning results in \cref{sec:main}. In \cref{sec:3} we prove the results in \cref{sec:main}. In \cref{sec:weaknorm} we extend the results in \cref{sec:main} to hold in weaker spatial norms, and we describe numerical experiments investigating the sharpness of these new results. In \cref{sec:nbpcqmc} we then apply the idea of nearby preconditioning to a Quasi-Monte-Carlo (QMC) method for the stochastic Helmholtz equation; in \cref{sec:nbpcqmcnum} we describe two algorithms for applying nearby preconditioning to QMC methods and in \cref{sec:nbpcqmcnumerics} we describe numerical experiments on the effectiveness of nearby preconditioning applied to QMC methods. In \cref{sec:nbpclitreview} we briefly review the related literature. Finally, in \cref{sec:nbpcstochastic}, we show how one can prove probabilistic results about the behaviour of nearby preconditioning, and we describe numerical experiments that investigate these probabilistic results.

\section{Statement of the main results}\label{sec:main}

\subsection{Definition of variational problems and conditions used to prove main results}\label{sec:vpGm}
As this \lcnamecref{chap:nbpc} concerns finite-element discretisations of the Helmholtz equation, we will work with the variational formulation of the Helmholtz equation. However, because the arguments we use do not directly rely on the boundary condition used to truncate the computational domain, we will state our Helmholtz problems in sufficient generality to include both the EDP (\cref{prob:vedp}) and TEDP (\cref{prob:vtedp}) above.

\bprob[General variational Helmholtz problem]\label{prob:vgen}
Let $D, A,$ and $n$ be as in \cref{prob:tedp}. We say $u \in \HozDD$ satisfies the \defn{variational formulation of a general exterior Dirichlet problem} with $\gD = 0$ if
\beq\label{eq:vgen}
\aG(u,v) = \LG(v) \tfa v \in \HozDD,
\eeq
where
\beq\label{eq:agen}
\aG(w,v) \de \int_{D} \mleft(\mleft(A \grad w\mright)\cdot\grad \vbar - k^2 n\minispace w \vbar\mright) - \DPGI{\T \trGI w}{\trGI v},
\eeq
$\T:\HhGI\rightarrow \HmhGI$ is a bounded linear map,  $\DPGI{\cdot}{\cdot}$ is the duality pairing on $\GI,$ and $\LG  \in \HozDDs.$
\eprob

\bre[\Cref{prob:vgen} is a generalisation of \cref{prob:vedp,prob:vtedp}]
With the exception of some overlap in notation, it is straightforward to see that appropriate choices of $D,$ $\GI$, $\T$, and $\LG$ allow \cref{prob:vgen} to be either \cref{prob:vedp} or \cref{prob:vtedp}. Taking $D = D$, $\GI = \GR$, $\T = \DtN$ and $\LG(v) = \int_{D} f\minispace\vbar$ (for $f$ as in \cref{prob:vedp}) in \cref{prob:vgen}, we see \cref{prob:vgen} becomes \cref{prob:vedp}. Additionally, taking $D$ and $\GI$ in \cref{prob:vgen} to be the same as the $D$ and $\GI$ in \cref{prob:vtedp}, taking $\T=ik,$ and $\LG(v) = \int_{D} f\minispace\vbar + \int_{\GI} \gI \minispace\trGI \vbar$ (for $f$ and $\gI$ as in \cref{prob:vtedp}), \cref{prob:vgen} becomes \cref{prob:vtedp}.
\ere

\bre[\Cref{prob:vgen} allows for other boundary conditions]
The strength of the general formulation in \cref{prob:vgen} is that it allows us to treat a wide variety of Helmholtz problems at once. Indeed, \emph{any} Helmholtz problem that can be written in the form \cref{eq:vgen,eq:agen} and satisfies \cref{cond:1nbpc,cond:2} below can be treated using the analysis in this \lcnamecref{chap:nbpc}.
\ere 

For the remainder of this \lcnamecref{chap:nbpc}, we let $(\Vhp)_{h>0}$ be the family of finite-element spaces.

\bas[Properties of finite-element spaces]
We assume $(\Vhp)_{h>0}$ is a family of finite-dimensional subspaces of $\HozDD$, whose union is dense in $\HozDD$. Moreover, we assume $\Vhp$ consists of nodal finite-element functions given by piecewise-polynomials on a quasi-uniform simplicial mesh $\cTh$ with mesh-size $h$
%\ednote{Euan says: have problem that want to allow $C^{1,1}$ $\Dm$, so that statements later about $H^2$ regularity are covered, but easiest to define triangulation and hence subspaces on Lipschitz domains -- Euan to discuss with Ivan}
and fixed polynomial degree $p$.
\eas
Note that the dimension $N$ of $\Vhp$ satisfies $N\sim h^{-d}$, with hidden constant dependent on $p$. (The assumption of quasi-uniformity can, in principle, be relaxed, see \cref{rem:ggsqu} below.) As in \cref{rem:crimes} above, we ignore any variational crimes resulting from this discretisation. We now define the finite-element approximation of \cref{prob:vgen}.
\bprob[Finite-element approximation of \cref{prob:vgen}]\label{prob:fevgen}
    Find $\uh \in \Vhp$ such that
\beq\label{eq:galerkin}
\aG(\uh,\vh) = \LG(\vh) \tforall \vh \in \Vhp.
\eeq
We say that $\uh \in \Vhp$ is the \defn{finite-element approximation of $u$} (the solution to \cref{prob:vgen}).
\eprob
%Observe that implicit in our use of $\aE$ in \cref{eq:galerkin} is the fact that we are realising the Dirichlet-to-Neumann map $\TR$ exactly on $\GR.$
%Definition of Galerkin method

\subsection{Definition of finite-element matrices, weighted norms, and weighted GMRES} 

\subsubsection{Finite-element matrices and weighted norms}

We now define the matrices associated with our finite-element discretisation. Let $\{\phi_i, i= 1, \ldots, N\}$ be a basis for $\Vhp$ with each $\phi_i$ \emph{real-valued}.
Let 
\beq\label{eq:matrixSjdef}
\big(\Smat_{A}\big)_{ij}\de \int_D \big(A \nabla \phi_j)\cdot\nabla \phi_i, \quad
\big(\Mmat_{n}\big)_{ij}\de \int_D n\,\phi_i\, \phi_j,
\quad\tand\quad
\big(\Nmat\big)_{ij}\de \int_{\GR} \T (\gamma\phi_j) \,\gamma \phi_i
\eeq
be the stiffness, domain-mass, and boundary-mass matrices, respectively. Note that both $\Smat_A$ and $\Mmat_n$ are \emph{real-valued}, but in general $\Nmat$ is \emph{complex-valued} (because both the  DtN operator $\DtN$ and the impedance operator $ik$ are complex-valued).
Let
\beq\label{eq:matrixAdef}
\Amat \de \Smat_{A} - k^2 \Mmat_{n} - \Nmat,
\eeq
and let $u_h\de \sum_j \uvec_j \phi_j$. Then \cref{eq:galerkin} implies that the coefficient vector $\uvec = \mleft(\uveci\mright)_{i=1}^N \in \CCN$ satisfies
\beqs
\Amat \uvec = \fvec,
\eeqs
where $(\fvec)_i \de \FE(\phi_i)$.
Similarly to above we let 
\beq\label{eq:matrixAjdef}
\Amatj \de \Smat_{A^{(j)}} - k^2 \Mmat_{n^{(j)}} - \Nmat.
\eeq

Our main results about \cref{it:nbpcq1} are \cref{cor:1,cor:1a} below. \Cref{cor:1a} gives results in the \emph{Euclidean} norm on matrices, denoted by $\|\cdot\|_2$ (induced by the Euclidean norm on vectors), whereas \Cref{cor:1} gives results in the \emph{weighted} norms $\NDmatk{\cdot}$ and $\NDmatkI{\cdot}$. These weighted norms are induced by the corresponding vector norms
\beq\label{eq:Dk}
\NDmatk{\vvec}^2\de \big( \Dmatk \vvec, \vvec\big)_2 = %\big( (\Smat_I + k^2 \Mmat_1)\vvec,\vvec\big)_2 
\N{v_h}^2_{\HokD}
\quad \tand
\quad \NDmatkI{\vvec}^2\de \big( \Dmatk^{-1} \vvec, \vvec\big)_2 %= %\big( (\Smat_I + k^2 \Mmat_1)\vvec,\vvec\big)_2 
%\N{v_h}^2_{\HokD}
\eeq
where $\Dmatk$ is given in terms of familiar finite-element stiffness- and mass-matrices by
\beqs
\Dmat_k\de \Smat_I + k^2 \Mmat_1,
\eeqs
and $\vh =\sum_i \sfvi \phii$, where the $\phii$ are the finite-element basis functions.


As described in \cref{sec:wpdisc}, the PDE analysis of the Helmholtz equation naturally takes place in the norm $\NHokD{\cdot}$, and \cref{eq:Dk} shows that the norm $\NDmatk{\cdot}$ is simply the norm on the finite-element space induced by  $\NHokD{\cdot}$. %See \cref{eq:Dk2} and \cref{eq:Dk3} below for this norm expressed in terms of the Euc
The norms $\NDmatk{\cdot}$ and $\NDmatkI{\cdot}$ recently appeared in results about the convergence of domain-decomposition methods %in this norm are proved 
for the Helmholtz equation \cite{GrSpVa:17,GrSpZo:18}, and a related norm appeared in similar results for the time-harmonic Maxwell equations \cite{BoDoGrSpTo:19}. 

The statement and proof our main results, \cref{cor:1,cor:1a} will require the following \lcnamecref{lem:normequiv}.

\ble[Norm equivalences of FE functions]\label{lem:normequiv}
There exist $m_\pm >0$ and $s_+>0$, independent of $h$ (but dependent on $p$), such that
\beq\label{eq:normequiv1}
m_- h^{d/2} \N{\vvec}_2 \leq \N{v_h}_{\LtD} \leq m_+ h^{d/2} \N{\vvec}_2,
\eeq
and
\beq\label{eq:normequiv2}
\N{\nabla v_h}_{\LtD} \leq s_+ h^{d/2-1} \N{\vvec}_2,
\eeq
for all finite-element functions $v_h =\sum_i \vvec_i \phi_i \in \Vhp$, with $\vvec = \mleft(\vveci\mright)_{i=1}^N \in \CCN.$
\ele

The proof of \cref{lem:normequiv} is on \cpageref{page:normequivpf} below.

Written in terms of the matrices $\Mmat_1$ and $\Smat_I$ defined in \cref{eq:matrixSjdef}, the bounds \cref{eq:normequiv1} and \cref{eq:normequiv2} are, respectively, the bounds
\beqs
(\Mmat_1 \vvec,\vvec)_2 \sim h^d \N{\vvec}^2_2 \quad\tand\quad (\Smat_I \vvec,\vvec)_2 \lesssim h^{d-2} \N{\vvec}^2_2.
\eeqs


\bre[Relaxing the assumption of quasi-uniformity]\label{rem:ggsqu}
We assume that $\set{\Th}_{h>0}$ is a quasi-uniform family of meshes so that the proof of \cref{lem:normequiv} is straightforward. However, this assumption can almost certainly be relaxed. In \cite{GaGrSp:15} (on which the bulk of the arguments in this \lcnamecref{chap:nbpc} are based) Gander, Graham, and Spence prove results both for quasi-uniform meshes and also for shape-regular meshes (see \cite[Sections 3.4 and 4.1.2]{GaGrSp:15}). Given the results in \cite{GaGrSp:15} for shape-regular meshes are analagous to those they obtain for quasi-uniform meshes, we expect the results in this \lcnamecref{chap:nbpc} can also be extended to shape-regular meshes. However, we note that \cite{GaGrSp:15} only contains bounds on preconditioned mass matrices (analogous to \cref{lem:keylemma1} below) but not preconditioned stiffness matrices (analogous to \cref{lem:keylemma2} below. Therefore it remains open to prove that our results in this \lcnamecref{chap:nbpc} can be extended in their entirety to shape-regular meshes.
\ere



\subsubsection{Weighted GMRES}

We now give the set-up for weighted GMRES, first introduced in by Essai in \cite{Es:98}; we largely follow \cite[Section 5]{GrSpVa:17}. Consider the abstract  linear system 
% \begin{equation*}
$\matrixC \xvec = \dvec$
%\end{equation*}
in $\mathbb{C}^N$, where $\matrixC \in \CC^{N\times N}$ is invertible. Let $\xvecz$ be the initial guess, and define the initial residual $\rvec^0 \de \dvec- \Cmat \xvec^0$ and the standard Krylov spaces:  
\beqs  
\cK^m(\Cmat, \rvec^0) \de \mathrm{span}\big\{\matrixC^j \rvec^0 : j = 0, \ldots, m-1\big\}.
\eeqs
Analagously to the definition of $\NDk{\cdot}$ above, let $(\cdot , \cdot )_{\Dmat}$ denote the inner product on $\CC^n$ 
induced by some Hermitian positive-definite matrix $\Dmat$, i.e.~
%\begin{equation*}
$(\vvec,\wvec)_{\Dmat} \de (\Dmat \vvec, \wvec)_2,$
%\end{equation*} 
and let $\Vert \cdot \Vert_\Dmat$ be the induced norm. For $m \geq 1$, define the $m$th GMRES iterate $\xvec^m$  to be  the unique element of $\cK^m$ satisfying  the  
 minimal residual  property: 
$$ \ \Vert \rvecm \Vert_\Dmat \de \Vert \dvec - \matrixC \xvec^m \Vert_\Dmat \ = \ \min_{\xvec \in \cK^m(C, \rvec^0)} \Vert {\dvec} - {\matrixC} {\xvec} \Vert_\Dmat. $$
Observe that when $\Dmat = \Imat$ this is the standard GMRES algorithm. We also note that in general, weighted GMRES requires the use of weighted Arnoldi process, also introduced by Essai in \cite{Es:98}, see also the alternative implementations of the weighted Arnoldi process in \cite{GuPe:14}.




\subsection{Main results}

\Cref{cor:1,cor:1a} are proved under the following two \lcnamecrefs{cond:1nbpc}, which are the minimal conditions needed for the proof of \cref{cor:1,cor:1a}. Therefore, in particular, \cref{cond:2} is a very weak condition on the finite-element space $\Vhp,$ since it does not even require convergence (for fixed $k$) as the mesh is refined.

\begin{condition}[Nontrapping bound on $u^{(1)}$]\label{cond:1nbpc}
$\Aso, \nso,$ and $D$ are such that, given $f\in L^2(D)$
  the solution of \cref{prob:vgen} with
  \beq\label{eq:LGf}
  \LG(v) = \int_D f\vbar,
  \eeq
$u^{(1)}$, exists, is unique, and, given $k_0>0$, $u^{(1)}$ satisfies the bound 
\beq\label{eq:bound1}
\big\|u^{(1)}\big\|_{\HokD} \leq C^{(1)}_{\rm bound} \N{f}_{L^2(D)} \quad \tfa k\geq k_0,
\eeq
where $C^{(1)}_{\rm bound}$ is independent of $k$, but dependent on $\Aso, \nso, D$, and $k_0$.
\end{condition}

\begin{condition}[$k$-independent accuracy of the FE solution for $a^{(1)}(\cdot,\cdot)$]
\label{cond:2}

\

\noindent\ben

\item\label[itempart]{it:femasspt1} Given $\kz>0$, $h$ and $p$ are chosen to depend on $k$ such that for all $k \geq \kz$, if $f= n\sum_j \alpha_j\phi_j$ for some $\alpha_j \in \CC$ and  $n\in \LiDR$  (i.e.~$f$ is an arbitrary element of $\Vhp$ multiplied by $n$), then 
  \bit
\item The solution $u_h$ of \cref{prob:fevgen} (with $\aG = \aGo$, and $\LG(v)$ given by \cref{eq:LGf}) exists and is unique, and
\item The error bound
  \beq\label{eq:bound3}
\N{u-u_h}_{\HokD} \leq C^{(1)}_{\rm FEM1} \N{f}_{\LtD} \quad, 
\eeq
holds, where $C^{(1)}_{\rm FEM1}$  is independent of $k$ and $h$, but dependent on $\Aso, \nso, D, k_0$, and $p$.
  \eit

\item Given $\kz>0$, $h$ and $p$ are chosen to depend on $k$ such that for all $k \geq \kz$, if $\LG(v)= (A\nabla \widetilde{f},\nabla v)_{\LtD}$, where $A\in \LiDRRdtd$ and $\widetilde{f} \de \sum_j \alpha_j \phi_j$ with $\alpha_j\in \CC$  (i.e.~$\widetilde{f}$ is an arbitrary element of $\Vhp$), then,
  \bit
\item The solution $u_h$ of \cref{prob:fevgen} with $\aG = \aGo$ exists and is unique, and
\item The error bound
\beq\label{eq:bound4}
\N{u-u_h}_{\HokD} \leq C^{(1)}_{\rm FEM2}\,k\, \N{\LG}_{\HokDs} \quad, 
\eeq
holds, where $C^{(1)}_{\rm FEM2}$  is independent of $k$ and $h$, but dependent on $\Aso, \nso, D, k_0$, and $p$.  
  \eit
\een
\end{condition}

For details of when \cref{cond:1nbpc,cond:2} are satisfied (for \cref{prob:vedp,prob:vtedp}), see \cref{sec:pdetheory} (for \cref{cond:1nbpc}) and \cref{sec:helmfedisc} (for \cref{cond:2} \cref{it:femasspt1}). \Cref{cond:1nbpc,cond:2} can be informally stated as 
\bit
\item the obstacle $\Dm$ and the coefficients $\Aso$ and $\nso$ are such that $u^{(1)}$ exists, is unique, and the problem is \emph{nontrapping} (in the sense described in  \cref{sec:wpdisc} above), and
\item the meshsize $h$ and polynomial degree $p$ in the finite-element method are chosen to depend on $k$ to ensure that the 
finite-element approximation to the solution of the problem with coefficients $\Aso$ and $\nso$ exists, is unique, and has bounded error in the $H^1_k$-norm as
%Galerkin method (with coefficients $\Aso$ and $\no$) 
$k\tendi$. 
\eit 


\bre[\Cref{eq:bound4} has the same $k$-dependence as \cref{eq:bound3}]\label{rem:yesitis}
Observe that the bound \cref{eq:bound4} has the same $k$-dependence as \cref{eq:bound3} despite the fact that a factor $k$ appears on the right-hand side. If $\LG(v) = \int_{D} f\vbar,$ then
\begin{align*}
  \NHokDs{\LG} = \sup_{v \in \HozDD} \frac{\abs{\LG(v)}}{\NHokD{v}} &\leq\sup_{v \in \HozDD} \frac{\NLtD{f}\NLtD{v}}{\NHokD{v}}\\
  &\lesssim \frac1k \sup_{v \in \HozDD} \frac{\NLtD{f}\NLtD{v}}{\NLtD{v}}\\
  &= \frac{\NLtD{f}}k.
\end{align*}
The factor $k$ appears in \cref{eq:bound4} since we use the weighted norm $\NHokD{\cdot}$ in the definition of $\NHokDs{\cdot},$ rather than the unweighted norm $\NHoD{\cdot}.$
\ere

\bth[Answer to \cref{it:nbpcq1}: $k$-independent weighted GMRES iterations]\label{cor:1}

\

\noindent Let $\kz \geq 0,$ $k \geq \kz,$ and assume that $\Dm$, $\Aso$, and $\nso$ satisfy \cref{cond:1nbpc}, $h$ and $p$ satisfy \cref{cond:2}, and $\Ast$ and $\nst$ are as in \cref{prob:vgen}. Then there exist constants $\Co$ and $\Ct$  independent of $h$ and $k$ (but dependent on $\Dm, \Aso, \nso, p$, and $\kz$) such that if 
% there exists $C_2>0$, independent of $h$ and $k$ (but dependent on $\Dm, \Aso, \nso$, $p$, and $k_0$) and given explicitly in \cref{eq:C2} below,
% such that if 
\beq\label{eq:cond}
C_1 \,k \,\NLiDop{\Aso-\Ast} +C_2 \, k\, \NLiDRR{\nso-\nst}
\leq \frac{1}{2},
\eeq
then \emph{both} weighted GMRES working in $\|\cdot\|_{\Dmat_k}$ (and the associated inner product) applied to 
\beq\label{eq:pcsystem1}
(\Amat^{(1)})^{-1}\Amat^{(2)}\uvec = \fvec
\eeq
\emph{and} weighted GMRES working in $\|\cdot\|_{(\Dmat_k)^{-1}}$ (and the associated inner product) applied to 
\beq\label{eq:pcsystem2}
\Amat^{(2)}(\Amat^{(1)})^{-1}\vvec = \fvec
\eeq
 converge in a $k$-independent number of iterations.
 \enth

The constants $\Co$ and $\Ct$ are given explicitly  in \cref{eq:C1nbpc,eq:C2} below. The proof of \cref{cor:1} is on \cpageref{page:cor1proof} below.

\bth[Answer to \cref{it:nbpcq1}: $k$-independent (unweighted) GMRES iterations]\label{cor:1a}

\

\noindent Let $\kz \geq 0,$ $k \geq \kz,$ and assume that $\Dm$, $\Aso$, and $\nso$ satisfy \cref{cond:1nbpc}, $h$ and $p$ satisfy \cref{cond:2}, and $\Ast$ and $\nst$ are as in \cref{prob:vgen}. Let $C_1$ and $C_2$ be as in \cref{cor:1}, and let $s_{+}>0$ and $m_{\pm}>0$ be as in \cref{lem:normequiv} (note that all these constants are independent of $k$, $h$, and $p$). Then if 
% there exists $C_2>0$, independent of $h$ and $k$ (but dependent on $\Dm, \Aso, \nso$, $p$, and $k_0$) and given explicitly in \cref{eq:C2} below,
% such that if 
\beq\label{eq:conda}
 C_1 \,\left(\frac{s_+}{m_-}\right) \,\frac{1}{h} \,
\NLiDop{\Aso-\Ast} + C_2 \, \left(\frac{m_+}{m_-} \right)k \, \NLiDRR{\nso-\nst},
\leq \frac{1}{2}
\eeq
then standard GMRES (working in the Euclidean norm and inner product) applied to either of the equations \cref{eq:pcsystem1} or \cref{eq:pcsystem2}
%\beqs
%(\Amat^{(1)})^{-1}\Amat^{(2)}\uvec = \fvec\quad\text{ or } \quad\Amat^{(2)}(\Amat^{(1)})^{-1}\vvec = \fvec
%\eeqs
 converges in a $k$-independent number of iterations.
 \enth

 The proof of \cref{cor:1a} is on \cpageref{page:cor1aproof} below.

Three notes regarding \cref{cor:1,cor:1a}: (i) The $L^\infty$ norms of $\Ao-\At$ and $\no-\nt$ in \cref{cor:1,cor:1a} can be replaced by $L^p$ norms with $p < \infty$, at the price of making the conditions \cref{eq:cond,eq:conda} more restrictive; see \cref{sec:weaknorm} for more details. (ii) The $\NLiDop{\cdot}$ norm on matrix-valued functions appearing on the right-hand sides of \cref{eq:main1} and \cref{eq:main1a} is defined by \cref{eq:ineditsnorm1}.
(iii) The factor $1/2$ on the right-hand sides of \cref{eq:cond} and \cref{eq:conda} can be replaced by any number $<1$ and the result still holds, although the number of GMRES iterations may then be different---but is still independent of $k$.
%In \cref{sec:proofFEM}, the constant $C_2$ is expressed explicitly in terms of $C_1$.

\bre
When $h\sim  k^{-1}$, the bounds \cref{eq:cond,eq:conda} are identical in their $k$-dependence; however, when $h\ll k^{-1}$ (as one needs to take to overcome the pollution effect, as discussed in \cref{sec:helmfedisc}) the bound \cref{eq:conda} for standard GMRES is more pessimistic than the bound \cref{eq:cond} for weighted GMRES.
\ere

\bre[Link to the results of \cite{GaGrSp:15}]
A result analogous to the Euclidean-norm bounds in \cref{thm:1} was proved in \cite{GaGrSp:15} for the case that $\Aso= \Ast= I$, $\nst= 1$, and $\nso = 1 + i\eps/k^2$, with the `absorption parameter' or `shift' $\eps$ satisfying $0<\eps\lesssim k^2$. (Recall from \cref{rem:ggsqu} that the proof strategy used in this \lcnamecref{chap:nbpc} is based on the strategy in \cite{GaGrSp:15}.) The motivation for proving the results in \cite{GaGrSp:15} was that the so-called `shifted Laplacian preconditioning' of the Helmholtz equation is based on preconditioning (with these choices of parameters) $\Amatt$ with an approximation of $\Amato$. Similar to \cref{cor:1}, bounds on $\|\Imat -  (\Amato)^{-1}\Amatt \|_2$ and 
$\|\Imat - \Amatt  (\Amato)^{-1}\|_2$
 then give upper bounds on how large the `shift' $\eps$ can be for GMRES for $\AmatoI\Amatt$ to converge in a $k$-independent number of iterations in the case when the action of $(\Amato)^{-1}$ is computed exactly.

%\cite[Lemma 4.1]{GaGrSp:15}
The main differences between \cite{GaGrSp:15} and this work are that: (i)  \cite{GaGrSp:15} focused on the TEDP, not both the TEDP and the EDP,
(ii) \cite{GaGrSp:15} focused on the particular case that $\Dm$ is star-shaped with respect to a ball, finding a $k$- and $\eps$-explicit expression for $C^{(1)}_{\rm bound}$ in this case using Morawetz identities, whereas we assume the existence of $\Cboundo,$
(iii) \cite{GaGrSp:15} required a bound on 
$(\Amato)^{-1}\Mmat_{n}$, analogous to the bounds in \cref{lem:keylemma1} along with one on $(\Amato)^{-1}\Nmat$ (in the case that $T_R$ is approximated by $i k$), but \emph{not} on 
$(\Amato)^{-1}\Smat_{A}$, and (iv) \cite{GaGrSp:15} only proved bounds in the $\|\cdot\|_2$ norm.
%The result of \cref{thm:1}
\ere


\subsection{PDE analogues to \cref{cor:1,cor:1a} }
Numerical experiments in \cref{sec:num} below indicate that the condition \cref{eq:cond} is sharp, i.e., that the $k$ in \cref{eq:cond} cannot be replaced by $k^\alpha$ for $\alpha<1$. This indicated sharpness of \cref{eq:cond} is also supported by the PDE-result \cref{thm:2} below. Indeed, \cref{thm:2} % and \cref{lem:1} 
 shows that the condition
\beq\label{eq:sufficientlysmall}
k\,
\NLiDop{\Aso-\Ast} \quad\text{ and } \quad k\,\NLiDRR{\nso-\nst}
%\Big) 
\quad\text{ both sufficiently small}
\eeq
is not only an answer to \cref{it:nbpcq1} (about finite-element discretisations), but is also the natural answer to the analogue of \cref{it:nbpcq1} at the level of PDEs, namely 
\ben[label=Q2., ref=Q2]
\item\label[itemblank]{it:nbpcq2}
How small must $\NLiDop{\Aso - \Ast}$ and 
$\NLiDRR{\nso - \nst}$ be (in terms of their $k$-dependence) for the relative error in approximating 
%$u^{(1)}$ to be a good approximation to 
$u^{(2)}$ by $u^{(1)}$ to be bounded independently of $k$ for arbitrarily-large $k$? 
\een
\Cref{lem:sharp} shows that the condition ``$k\NLiDRR{\nso - \nst}$ sufficiently small" is the \emph{provably-sharp} answer to \cref{it:nbpcq2} when $\Aso= \Ast= I$.

%Before stating these PDE results, we define the weighted $H^1$ norm
%\beq\label{eq:1knorm}
%\N{v}^2_{\HokD} \de \N{\grad v}^2_{L^2(\D)} + k^2 \N{v}^2_{L^2(D)} \quad \tfor v \in H^1_{0,D}(D),
%\eeq
%where the space $H^1_{0,D}(D)$, defined by \cref{eq:spaceEDP} below, is the natural space containing the solution of the exterior Dirichlet problem. 
To state these PDE results, we use the notation for $a,b>0$ that $a\lesssim b$ when $a\leq C b$ for some $C>0$, independent of $k$, and $a\sim b$ if $a\lesssim b$ and $b\lesssim a$.


%The sharpness of \cref{eq:cond} and \cref{eq:main1} is also supported by the answer to the analogue of \cref{it:nbpcq1} at the level of PDEs. Indeed, the following \cref{thm:2} is the analogue of \cref{thm:1} 

\begin{theorem}[Answer to \cref{it:nbpcq2} (the PDE analogue of \cref{it:nbpcq1})]\label{thm:2}
%Given $f\in L^2(D)$ such that $\supp \, f \subset \BR$, 
Let $\kz > 0$ and $k \geq \kz.$ Let $\Dm$, $\Aso$, and $\nso$ satisfy \cref{cond:1nbpc} applied to \cref{prob:vedp} (so that the solution of \cref{prob:vedp} $\uso$ exists, is unique, and satisfies a $k$-independent a priori bound). Let $\Dm$, $\Ast$, and $\nst$ be such that $u^{(2)}$ exists
for any $f\in L^2(D)$ such that $\supp \, f \subset \BR$. 
Then, there exists $C_3>0$, independent of $k$ and given explicitly in terms of $\Dm$, $\Aso$, and $\nso$ in \cref{eq:C3} below, such that
\beq\label{eq:PDEbound}
\frac{\big\|u^{(1)}-u^{(2)}\big\|_{\HokD}
}{
\N{u^{(2)}}_{\HokD}
}\leq C_3 \,k\, \max\set{\NLiDop{\Aso-\Ast}\,,\, \NLiDRR{\nso-\nst}}
\eeq
for all $k\geq k_0$. 
\end{theorem}

The proof of \cref{thm:2} is on \cpageref{page:thm2proof} below.

\ble[Sharpness of the bound \cref{eq:PDEbound} when $\Aso = \Ast= I$]\label{lem:sharp}
There exist particular choices of  $f, \,\nso$, and $\nst$ (with $\nso\neq \nst$ both continuous) such that 
the corresponding solutions $u^{(1)}$ and $u^{(2)}$ of \cref{prob:edp} with $\Aso = \Ast= I$ exist, are unique, and satisfy
\beq\label{eq:sharp1}
\frac{\N{u^{(1)}-u^{(2)}}_{\HokD}
}{
\N{u^{(2)}}_{\HokD}
}
\sim 
\frac{\N{u^{(1)}-u^{(2)}}_{L^2(D)}
}{
\N{u^{(2)}}_{L^2(D)}
}\sim k \NLiDRR{\nso-\nst}.
\eeq
%\noi (ii) There exist $f, \Aso, \Ast$, (with $\Aso\not\equiv \Ast$), such that 
%the corresponding solutions $u^{(1)}$ and $u^{(2)}$ of the exterior Dirichlet problem with $\nso \equiv \nst\equiv 1$ exist, are unique, and satisfy
%%There exist $f\in L^2(D), \Aj \in C^{0,1}(D)$, $j=1,2$ (with $\Aso\not\equiv \Ast$), such that the corresponding solutions $u^{(1)}$ and $u^{(2)}$ of the exterior Dirichlet problem with $\nso \equiv \nst\equiv 1$ satisfy
%\beq\label{eq:sharp2}
%\frac{\N{u^{(1)}-u^{(2)}}_{\HokD}
%}{
%\N{u^{(2)}}_{\HokD}
%}
%\sim 
%\frac{\N{u^{(1)}-u^{(2)}}_{L^2(D)}
%}{
%\N{u^{(2)}}_{L^2(D)}
%}\sim k \big\|\Aso-\Ast\big\|_{L^\infty(D)}.
%\eeq
\ele

The proof of \cref{lem:sharp} is on \cpageref{page:lemsharpproof} below.

\bre[Physical interpretation for $k$-dependence]\label{rem:physical1k}
It is perhaps unsurprising that the condition \cref{eq:sufficientlysmall} is a sufficient condition to answer both \cref{it:nbpcq1} and \cref{it:nbpcq2}. Recall that $1/k$ is proportional to the wavelength $2\pi/k$ of the wave $u$ (at least when $A=I$ and $n=1$). As the wavelength is the natural length scale associated with the wave $u$, one expects perturbations of magnitude up to $\cO(1/k)$ to be `unseen' by the PDE or numerical method. This is exactly what we see; perturbations of size (up to) $1/k$ give bounded relative difference (in \cref{it:nbpcq2}) and bounded GMRES iterations for the nearby-preconditioned linear system (in \cref{it:nbpcq1}). Also, on a PDE level, perturbations of order $1/k$ being `unseen' by the PDE can also be seen in bounds proved for $u$ where $n = \no + \eta,$ with $\no$ nontrapping and $\NLiDRR{\eta} \lesssim 1/k,$ see \cref{rem:kdep} above.
\ere


\section[Numerical experiments investigating sharpness]{Numerical experiments investigating the\newline sharpness of Theorems {\ref{cor:1}} and {\ref{cor:1a}}}\label{sec:num}

The numerical experiments in this section seek to verify \cref{cor:1a,cor:1} for \cref{prob:vtedp}, and investigate their sharpness. More specifically, the experiments seek to verify whether the condition \cref{eq:cond} is:
\ben
\item sufficient, and
\item necessary
  \een
  for \emph{standard} GMRES applied to \cref{eq:pcsystem1} to converge in a number of iterations that is independent of $k.$

Based on the PDE results \cref{thm:2,lem:sharp} above, we expect that the condition \cref{eq:sufficientlysmall} is a necessary and sufficient condition for standard GMRES applied to \cref{eq:pcsystem1} to converge in a $k$-independent number of iterations, even though we can only prove this is a sufficient condition for \emph{weighted} GMRES. We expect this because \cref{eq:sufficientlysmall} is a sufficient condition for \cref{it:nbpcq2}, the PDE analogue of \cref{it:nbpcq1}. Indeed, this is exactly the behaviour we observe in numerical experiments; we see that if \cref{eq:sufficientlysmall} holds, then standard GMRES applied to \cref{eq:pcsystem1} converges in a $k$-independent number of iterations, and moreover, \cref{eq:sufficientlysmall} may be sharp. We now describe our numerical experiments in more detail.

To verify this expected behaviour, we perform numerical experiments with the setup described in \cref{app:compsetup} with $\Aso = I$ and $\nso = 1$. We define $f$ and $\gI$ to correspond to a plane wave incident from the bottom left passing through a homogeneous medium given by coefficients $\Aso$ and $\nso$. We perform experiments for $A$ and $n$ separately, i.e., first we perform experiments with $\Ast=I$ and $\nst$ varying, and then we perform experiments with $\Ast$ varying and $\nst=1.$ When we vary $\Ast$ we measure $\Aso-\Ast$ in the $\NLiDRRdtd{\cdot}$ norm, as this norm is easier to control than the $\NLiDop{\cdot}$ norm. However, these two norms are equivalent on $\LiDRRdtd$ (see the comment above \cref{eq:ineditsnorm1}).

We define $\Ast$ and $\nst$ to be piecewise constant (matrix-valued and real-valued respectively) on a $10\times10$ square grid, with their values on each square chosen independently at random from a $\Unif\mleft(1-\alpha,1+\alpha\mright)$ distribution, with $\alpha \in (0,1)$ chosen as described below. For $\Ast,$ we impose the restriction that on each square $\Ast$ is positive-definite almost surely. We solve the linear systems \cref{eq:pcsystem1} for $k = 20,40,60,80,100$ using standard GMRES and record the number of GMRES iterations taken to achieve a (relative) tolerance of $10^{-5}$ (relative to $\Nt{\bvec}$).

We perform experiments taking $\alpha = 0.5 \times k^{-\beta}$ for $\beta \in 0,0.1,\ldots,0.9,1.$ We expect that when $\beta \neq 1$  the number of GMRES iterations required for convergence will increase as $k$ increases, whereas we expect that when $\beta = 1$ the number of GMRES iterations required for convergence will remain bounded as $k$ increases, even though this behaviour for $\beta=1$ has only been proved for $\NLiDop{\Aso-\Ast}$ for weighted GMRES (compare the restrictions on $\NLiDop{\Aso-\Ast}$ in \cref{cor:1,cor:1a}).


In \cref{fig:linfinityA0,fig:linfinityA1,fig:linfinityA2,fig:linfinityn0,fig:linfinityn1,fig:linfinityn2}, when $\beta \in \set{0,\ldots,0.3}$ (for $\NLiDRRdtd{\Aso-\Ast}$) and $\beta \in \set{0,\ldots,0.5}$ (for $\NLiDRR{\nso-\nst}$) we see growth in the maximum number of GMRES iterations needed (over all realisations) to achieve convergence, otherwise we see that the number of GMRES iterations is bounded as $k$ increases. This behaviour is better than expected; as the number of GMRES iterations is apparently bounded for a range of $\beta < 1.$ However, we note that this behaviour could be (i) because we are in a pre-asymptotic regime, and the number of GMRES iterations would grow if we increased $k$ further, or (ii) the particular structure of $\nst$ (being piecewise constant, with the pieces independently, randomly chosen) could result in some kind of `averaging' behaviour, meaning the preconditioner is better than would otherwise be expected. However, we do not investigate these issues further in this thesis.

    \begin{figure}
      \centering
%% Creator: Matplotlib, PGF backend
%%
%% To include the figure in your LaTeX document, write
%%   \input{<filename>.pgf}
%%
%% Make sure the required packages are loaded in your preamble
%%   \usepackage{pgf}
%%
%% Figures using additional raster images can only be included by \input if
%% they are in the same directory as the main LaTeX file. For loading figures
%% from other directories you can use the `import` package
%%   \usepackage{import}
%% and then include the figures with
%%   \import{<path to file>}{<filename>.pgf}
%%
%% Matplotlib used the following preamble
%%   \usepackage{fontspec}
%%   \setmainfont{DejaVuSerif.ttf}[Path=/home/owen/progs/firedrake-complex/firedrake/lib/python3.5/site-packages/matplotlib/mpl-data/fonts/ttf/]
%%   \setsansfont{DejaVuSans.ttf}[Path=/home/owen/progs/firedrake-complex/firedrake/lib/python3.5/site-packages/matplotlib/mpl-data/fonts/ttf/]
%%   \setmonofont{DejaVuSansMono.ttf}[Path=/home/owen/progs/firedrake-complex/firedrake/lib/python3.5/site-packages/matplotlib/mpl-data/fonts/ttf/]
%%
\begingroup%
\makeatletter%
\begin{pgfpicture}%
\pgfpathrectangle{\pgfpointorigin}{\pgfqpoint{6.400000in}{4.800000in}}%
\pgfusepath{use as bounding box, clip}%
\begin{pgfscope}%
\pgfsetbuttcap%
\pgfsetmiterjoin%
\definecolor{currentfill}{rgb}{1.000000,1.000000,1.000000}%
\pgfsetfillcolor{currentfill}%
\pgfsetlinewidth{0.000000pt}%
\definecolor{currentstroke}{rgb}{1.000000,1.000000,1.000000}%
\pgfsetstrokecolor{currentstroke}%
\pgfsetdash{}{0pt}%
\pgfpathmoveto{\pgfqpoint{0.000000in}{0.000000in}}%
\pgfpathlineto{\pgfqpoint{6.400000in}{0.000000in}}%
\pgfpathlineto{\pgfqpoint{6.400000in}{4.800000in}}%
\pgfpathlineto{\pgfqpoint{0.000000in}{4.800000in}}%
\pgfpathclose%
\pgfusepath{fill}%
\end{pgfscope}%
\begin{pgfscope}%
\pgfsetbuttcap%
\pgfsetmiterjoin%
\definecolor{currentfill}{rgb}{1.000000,1.000000,1.000000}%
\pgfsetfillcolor{currentfill}%
\pgfsetlinewidth{0.000000pt}%
\definecolor{currentstroke}{rgb}{0.000000,0.000000,0.000000}%
\pgfsetstrokecolor{currentstroke}%
\pgfsetstrokeopacity{0.000000}%
\pgfsetdash{}{0pt}%
\pgfpathmoveto{\pgfqpoint{0.800000in}{0.528000in}}%
\pgfpathlineto{\pgfqpoint{5.760000in}{0.528000in}}%
\pgfpathlineto{\pgfqpoint{5.760000in}{4.224000in}}%
\pgfpathlineto{\pgfqpoint{0.800000in}{4.224000in}}%
\pgfpathclose%
\pgfusepath{fill}%
\end{pgfscope}%
\begin{pgfscope}%
\pgfpathrectangle{\pgfqpoint{0.800000in}{0.528000in}}{\pgfqpoint{4.960000in}{3.696000in}}%
\pgfusepath{clip}%
\pgfsetbuttcap%
\pgfsetroundjoin%
\definecolor{currentfill}{rgb}{0.000000,0.000000,0.000000}%
\pgfsetfillcolor{currentfill}%
\pgfsetlinewidth{1.003750pt}%
\definecolor{currentstroke}{rgb}{0.000000,0.000000,0.000000}%
\pgfsetstrokecolor{currentstroke}%
\pgfsetdash{}{0pt}%
\pgfpathmoveto{\pgfqpoint{1.025793in}{2.334333in}}%
\pgfpathcurveto{\pgfqpoint{1.036843in}{2.334333in}}{\pgfqpoint{1.047442in}{2.338724in}}{\pgfqpoint{1.055256in}{2.346537in}}%
\pgfpathcurveto{\pgfqpoint{1.063070in}{2.354351in}}{\pgfqpoint{1.067460in}{2.364950in}}{\pgfqpoint{1.067460in}{2.376000in}}%
\pgfpathcurveto{\pgfqpoint{1.067460in}{2.387050in}}{\pgfqpoint{1.063070in}{2.397649in}}{\pgfqpoint{1.055256in}{2.405463in}}%
\pgfpathcurveto{\pgfqpoint{1.047442in}{2.413276in}}{\pgfqpoint{1.036843in}{2.417667in}}{\pgfqpoint{1.025793in}{2.417667in}}%
\pgfpathcurveto{\pgfqpoint{1.014743in}{2.417667in}}{\pgfqpoint{1.004144in}{2.413276in}}{\pgfqpoint{0.996330in}{2.405463in}}%
\pgfpathcurveto{\pgfqpoint{0.988517in}{2.397649in}}{\pgfqpoint{0.984126in}{2.387050in}}{\pgfqpoint{0.984126in}{2.376000in}}%
\pgfpathcurveto{\pgfqpoint{0.984126in}{2.364950in}}{\pgfqpoint{0.988517in}{2.354351in}}{\pgfqpoint{0.996330in}{2.346537in}}%
\pgfpathcurveto{\pgfqpoint{1.004144in}{2.338724in}}{\pgfqpoint{1.014743in}{2.334333in}}{\pgfqpoint{1.025793in}{2.334333in}}%
\pgfpathclose%
\pgfusepath{stroke,fill}%
\end{pgfscope}%
\begin{pgfscope}%
\pgfpathrectangle{\pgfqpoint{0.800000in}{0.528000in}}{\pgfqpoint{4.960000in}{3.696000in}}%
\pgfusepath{clip}%
\pgfsetbuttcap%
\pgfsetroundjoin%
\definecolor{currentfill}{rgb}{0.000000,0.000000,0.000000}%
\pgfsetfillcolor{currentfill}%
\pgfsetlinewidth{1.003750pt}%
\definecolor{currentstroke}{rgb}{0.000000,0.000000,0.000000}%
\pgfsetstrokecolor{currentstroke}%
\pgfsetdash{}{0pt}%
\pgfpathmoveto{\pgfqpoint{1.025793in}{2.334333in}}%
\pgfpathcurveto{\pgfqpoint{1.036843in}{2.334333in}}{\pgfqpoint{1.047442in}{2.338724in}}{\pgfqpoint{1.055256in}{2.346537in}}%
\pgfpathcurveto{\pgfqpoint{1.063070in}{2.354351in}}{\pgfqpoint{1.067460in}{2.364950in}}{\pgfqpoint{1.067460in}{2.376000in}}%
\pgfpathcurveto{\pgfqpoint{1.067460in}{2.387050in}}{\pgfqpoint{1.063070in}{2.397649in}}{\pgfqpoint{1.055256in}{2.405463in}}%
\pgfpathcurveto{\pgfqpoint{1.047442in}{2.413276in}}{\pgfqpoint{1.036843in}{2.417667in}}{\pgfqpoint{1.025793in}{2.417667in}}%
\pgfpathcurveto{\pgfqpoint{1.014743in}{2.417667in}}{\pgfqpoint{1.004144in}{2.413276in}}{\pgfqpoint{0.996330in}{2.405463in}}%
\pgfpathcurveto{\pgfqpoint{0.988517in}{2.397649in}}{\pgfqpoint{0.984126in}{2.387050in}}{\pgfqpoint{0.984126in}{2.376000in}}%
\pgfpathcurveto{\pgfqpoint{0.984126in}{2.364950in}}{\pgfqpoint{0.988517in}{2.354351in}}{\pgfqpoint{0.996330in}{2.346537in}}%
\pgfpathcurveto{\pgfqpoint{1.004144in}{2.338724in}}{\pgfqpoint{1.014743in}{2.334333in}}{\pgfqpoint{1.025793in}{2.334333in}}%
\pgfpathclose%
\pgfusepath{stroke,fill}%
\end{pgfscope}%
\begin{pgfscope}%
\pgfpathrectangle{\pgfqpoint{0.800000in}{0.528000in}}{\pgfqpoint{4.960000in}{3.696000in}}%
\pgfusepath{clip}%
\pgfsetbuttcap%
\pgfsetroundjoin%
\definecolor{currentfill}{rgb}{0.000000,0.000000,0.000000}%
\pgfsetfillcolor{currentfill}%
\pgfsetlinewidth{1.003750pt}%
\definecolor{currentstroke}{rgb}{0.000000,0.000000,0.000000}%
\pgfsetstrokecolor{currentstroke}%
\pgfsetdash{}{0pt}%
\pgfpathmoveto{\pgfqpoint{1.025793in}{2.334333in}}%
\pgfpathcurveto{\pgfqpoint{1.036843in}{2.334333in}}{\pgfqpoint{1.047442in}{2.338724in}}{\pgfqpoint{1.055256in}{2.346537in}}%
\pgfpathcurveto{\pgfqpoint{1.063070in}{2.354351in}}{\pgfqpoint{1.067460in}{2.364950in}}{\pgfqpoint{1.067460in}{2.376000in}}%
\pgfpathcurveto{\pgfqpoint{1.067460in}{2.387050in}}{\pgfqpoint{1.063070in}{2.397649in}}{\pgfqpoint{1.055256in}{2.405463in}}%
\pgfpathcurveto{\pgfqpoint{1.047442in}{2.413276in}}{\pgfqpoint{1.036843in}{2.417667in}}{\pgfqpoint{1.025793in}{2.417667in}}%
\pgfpathcurveto{\pgfqpoint{1.014743in}{2.417667in}}{\pgfqpoint{1.004144in}{2.413276in}}{\pgfqpoint{0.996330in}{2.405463in}}%
\pgfpathcurveto{\pgfqpoint{0.988517in}{2.397649in}}{\pgfqpoint{0.984126in}{2.387050in}}{\pgfqpoint{0.984126in}{2.376000in}}%
\pgfpathcurveto{\pgfqpoint{0.984126in}{2.364950in}}{\pgfqpoint{0.988517in}{2.354351in}}{\pgfqpoint{0.996330in}{2.346537in}}%
\pgfpathcurveto{\pgfqpoint{1.004144in}{2.338724in}}{\pgfqpoint{1.014743in}{2.334333in}}{\pgfqpoint{1.025793in}{2.334333in}}%
\pgfpathclose%
\pgfusepath{stroke,fill}%
\end{pgfscope}%
\begin{pgfscope}%
\pgfpathrectangle{\pgfqpoint{0.800000in}{0.528000in}}{\pgfqpoint{4.960000in}{3.696000in}}%
\pgfusepath{clip}%
\pgfsetbuttcap%
\pgfsetroundjoin%
\definecolor{currentfill}{rgb}{0.000000,0.000000,0.000000}%
\pgfsetfillcolor{currentfill}%
\pgfsetlinewidth{1.003750pt}%
\definecolor{currentstroke}{rgb}{0.000000,0.000000,0.000000}%
\pgfsetstrokecolor{currentstroke}%
\pgfsetdash{}{0pt}%
\pgfpathmoveto{\pgfqpoint{1.025793in}{2.334333in}}%
\pgfpathcurveto{\pgfqpoint{1.036843in}{2.334333in}}{\pgfqpoint{1.047442in}{2.338724in}}{\pgfqpoint{1.055256in}{2.346537in}}%
\pgfpathcurveto{\pgfqpoint{1.063070in}{2.354351in}}{\pgfqpoint{1.067460in}{2.364950in}}{\pgfqpoint{1.067460in}{2.376000in}}%
\pgfpathcurveto{\pgfqpoint{1.067460in}{2.387050in}}{\pgfqpoint{1.063070in}{2.397649in}}{\pgfqpoint{1.055256in}{2.405463in}}%
\pgfpathcurveto{\pgfqpoint{1.047442in}{2.413276in}}{\pgfqpoint{1.036843in}{2.417667in}}{\pgfqpoint{1.025793in}{2.417667in}}%
\pgfpathcurveto{\pgfqpoint{1.014743in}{2.417667in}}{\pgfqpoint{1.004144in}{2.413276in}}{\pgfqpoint{0.996330in}{2.405463in}}%
\pgfpathcurveto{\pgfqpoint{0.988517in}{2.397649in}}{\pgfqpoint{0.984126in}{2.387050in}}{\pgfqpoint{0.984126in}{2.376000in}}%
\pgfpathcurveto{\pgfqpoint{0.984126in}{2.364950in}}{\pgfqpoint{0.988517in}{2.354351in}}{\pgfqpoint{0.996330in}{2.346537in}}%
\pgfpathcurveto{\pgfqpoint{1.004144in}{2.338724in}}{\pgfqpoint{1.014743in}{2.334333in}}{\pgfqpoint{1.025793in}{2.334333in}}%
\pgfpathclose%
\pgfusepath{stroke,fill}%
\end{pgfscope}%
\begin{pgfscope}%
\pgfpathrectangle{\pgfqpoint{0.800000in}{0.528000in}}{\pgfqpoint{4.960000in}{3.696000in}}%
\pgfusepath{clip}%
\pgfsetbuttcap%
\pgfsetroundjoin%
\definecolor{currentfill}{rgb}{0.000000,0.000000,0.000000}%
\pgfsetfillcolor{currentfill}%
\pgfsetlinewidth{1.003750pt}%
\definecolor{currentstroke}{rgb}{0.000000,0.000000,0.000000}%
\pgfsetstrokecolor{currentstroke}%
\pgfsetdash{}{0pt}%
\pgfpathmoveto{\pgfqpoint{1.025793in}{2.334333in}}%
\pgfpathcurveto{\pgfqpoint{1.036843in}{2.334333in}}{\pgfqpoint{1.047442in}{2.338724in}}{\pgfqpoint{1.055256in}{2.346537in}}%
\pgfpathcurveto{\pgfqpoint{1.063070in}{2.354351in}}{\pgfqpoint{1.067460in}{2.364950in}}{\pgfqpoint{1.067460in}{2.376000in}}%
\pgfpathcurveto{\pgfqpoint{1.067460in}{2.387050in}}{\pgfqpoint{1.063070in}{2.397649in}}{\pgfqpoint{1.055256in}{2.405463in}}%
\pgfpathcurveto{\pgfqpoint{1.047442in}{2.413276in}}{\pgfqpoint{1.036843in}{2.417667in}}{\pgfqpoint{1.025793in}{2.417667in}}%
\pgfpathcurveto{\pgfqpoint{1.014743in}{2.417667in}}{\pgfqpoint{1.004144in}{2.413276in}}{\pgfqpoint{0.996330in}{2.405463in}}%
\pgfpathcurveto{\pgfqpoint{0.988517in}{2.397649in}}{\pgfqpoint{0.984126in}{2.387050in}}{\pgfqpoint{0.984126in}{2.376000in}}%
\pgfpathcurveto{\pgfqpoint{0.984126in}{2.364950in}}{\pgfqpoint{0.988517in}{2.354351in}}{\pgfqpoint{0.996330in}{2.346537in}}%
\pgfpathcurveto{\pgfqpoint{1.004144in}{2.338724in}}{\pgfqpoint{1.014743in}{2.334333in}}{\pgfqpoint{1.025793in}{2.334333in}}%
\pgfpathclose%
\pgfusepath{stroke,fill}%
\end{pgfscope}%
\begin{pgfscope}%
\pgfpathrectangle{\pgfqpoint{0.800000in}{0.528000in}}{\pgfqpoint{4.960000in}{3.696000in}}%
\pgfusepath{clip}%
\pgfsetbuttcap%
\pgfsetroundjoin%
\definecolor{currentfill}{rgb}{0.000000,0.000000,0.000000}%
\pgfsetfillcolor{currentfill}%
\pgfsetlinewidth{1.003750pt}%
\definecolor{currentstroke}{rgb}{0.000000,0.000000,0.000000}%
\pgfsetstrokecolor{currentstroke}%
\pgfsetdash{}{0pt}%
\pgfpathmoveto{\pgfqpoint{1.025793in}{2.334333in}}%
\pgfpathcurveto{\pgfqpoint{1.036843in}{2.334333in}}{\pgfqpoint{1.047442in}{2.338724in}}{\pgfqpoint{1.055256in}{2.346537in}}%
\pgfpathcurveto{\pgfqpoint{1.063070in}{2.354351in}}{\pgfqpoint{1.067460in}{2.364950in}}{\pgfqpoint{1.067460in}{2.376000in}}%
\pgfpathcurveto{\pgfqpoint{1.067460in}{2.387050in}}{\pgfqpoint{1.063070in}{2.397649in}}{\pgfqpoint{1.055256in}{2.405463in}}%
\pgfpathcurveto{\pgfqpoint{1.047442in}{2.413276in}}{\pgfqpoint{1.036843in}{2.417667in}}{\pgfqpoint{1.025793in}{2.417667in}}%
\pgfpathcurveto{\pgfqpoint{1.014743in}{2.417667in}}{\pgfqpoint{1.004144in}{2.413276in}}{\pgfqpoint{0.996330in}{2.405463in}}%
\pgfpathcurveto{\pgfqpoint{0.988517in}{2.397649in}}{\pgfqpoint{0.984126in}{2.387050in}}{\pgfqpoint{0.984126in}{2.376000in}}%
\pgfpathcurveto{\pgfqpoint{0.984126in}{2.364950in}}{\pgfqpoint{0.988517in}{2.354351in}}{\pgfqpoint{0.996330in}{2.346537in}}%
\pgfpathcurveto{\pgfqpoint{1.004144in}{2.338724in}}{\pgfqpoint{1.014743in}{2.334333in}}{\pgfqpoint{1.025793in}{2.334333in}}%
\pgfpathclose%
\pgfusepath{stroke,fill}%
\end{pgfscope}%
\begin{pgfscope}%
\pgfpathrectangle{\pgfqpoint{0.800000in}{0.528000in}}{\pgfqpoint{4.960000in}{3.696000in}}%
\pgfusepath{clip}%
\pgfsetbuttcap%
\pgfsetroundjoin%
\definecolor{currentfill}{rgb}{0.000000,0.000000,0.000000}%
\pgfsetfillcolor{currentfill}%
\pgfsetlinewidth{1.003750pt}%
\definecolor{currentstroke}{rgb}{0.000000,0.000000,0.000000}%
\pgfsetstrokecolor{currentstroke}%
\pgfsetdash{}{0pt}%
\pgfpathmoveto{\pgfqpoint{1.025793in}{2.334333in}}%
\pgfpathcurveto{\pgfqpoint{1.036843in}{2.334333in}}{\pgfqpoint{1.047442in}{2.338724in}}{\pgfqpoint{1.055256in}{2.346537in}}%
\pgfpathcurveto{\pgfqpoint{1.063070in}{2.354351in}}{\pgfqpoint{1.067460in}{2.364950in}}{\pgfqpoint{1.067460in}{2.376000in}}%
\pgfpathcurveto{\pgfqpoint{1.067460in}{2.387050in}}{\pgfqpoint{1.063070in}{2.397649in}}{\pgfqpoint{1.055256in}{2.405463in}}%
\pgfpathcurveto{\pgfqpoint{1.047442in}{2.413276in}}{\pgfqpoint{1.036843in}{2.417667in}}{\pgfqpoint{1.025793in}{2.417667in}}%
\pgfpathcurveto{\pgfqpoint{1.014743in}{2.417667in}}{\pgfqpoint{1.004144in}{2.413276in}}{\pgfqpoint{0.996330in}{2.405463in}}%
\pgfpathcurveto{\pgfqpoint{0.988517in}{2.397649in}}{\pgfqpoint{0.984126in}{2.387050in}}{\pgfqpoint{0.984126in}{2.376000in}}%
\pgfpathcurveto{\pgfqpoint{0.984126in}{2.364950in}}{\pgfqpoint{0.988517in}{2.354351in}}{\pgfqpoint{0.996330in}{2.346537in}}%
\pgfpathcurveto{\pgfqpoint{1.004144in}{2.338724in}}{\pgfqpoint{1.014743in}{2.334333in}}{\pgfqpoint{1.025793in}{2.334333in}}%
\pgfpathclose%
\pgfusepath{stroke,fill}%
\end{pgfscope}%
\begin{pgfscope}%
\pgfpathrectangle{\pgfqpoint{0.800000in}{0.528000in}}{\pgfqpoint{4.960000in}{3.696000in}}%
\pgfusepath{clip}%
\pgfsetbuttcap%
\pgfsetroundjoin%
\definecolor{currentfill}{rgb}{0.000000,0.000000,0.000000}%
\pgfsetfillcolor{currentfill}%
\pgfsetlinewidth{1.003750pt}%
\definecolor{currentstroke}{rgb}{0.000000,0.000000,0.000000}%
\pgfsetstrokecolor{currentstroke}%
\pgfsetdash{}{0pt}%
\pgfpathmoveto{\pgfqpoint{1.025793in}{2.334333in}}%
\pgfpathcurveto{\pgfqpoint{1.036843in}{2.334333in}}{\pgfqpoint{1.047442in}{2.338724in}}{\pgfqpoint{1.055256in}{2.346537in}}%
\pgfpathcurveto{\pgfqpoint{1.063070in}{2.354351in}}{\pgfqpoint{1.067460in}{2.364950in}}{\pgfqpoint{1.067460in}{2.376000in}}%
\pgfpathcurveto{\pgfqpoint{1.067460in}{2.387050in}}{\pgfqpoint{1.063070in}{2.397649in}}{\pgfqpoint{1.055256in}{2.405463in}}%
\pgfpathcurveto{\pgfqpoint{1.047442in}{2.413276in}}{\pgfqpoint{1.036843in}{2.417667in}}{\pgfqpoint{1.025793in}{2.417667in}}%
\pgfpathcurveto{\pgfqpoint{1.014743in}{2.417667in}}{\pgfqpoint{1.004144in}{2.413276in}}{\pgfqpoint{0.996330in}{2.405463in}}%
\pgfpathcurveto{\pgfqpoint{0.988517in}{2.397649in}}{\pgfqpoint{0.984126in}{2.387050in}}{\pgfqpoint{0.984126in}{2.376000in}}%
\pgfpathcurveto{\pgfqpoint{0.984126in}{2.364950in}}{\pgfqpoint{0.988517in}{2.354351in}}{\pgfqpoint{0.996330in}{2.346537in}}%
\pgfpathcurveto{\pgfqpoint{1.004144in}{2.338724in}}{\pgfqpoint{1.014743in}{2.334333in}}{\pgfqpoint{1.025793in}{2.334333in}}%
\pgfpathclose%
\pgfusepath{stroke,fill}%
\end{pgfscope}%
\begin{pgfscope}%
\pgfpathrectangle{\pgfqpoint{0.800000in}{0.528000in}}{\pgfqpoint{4.960000in}{3.696000in}}%
\pgfusepath{clip}%
\pgfsetbuttcap%
\pgfsetroundjoin%
\definecolor{currentfill}{rgb}{0.000000,0.000000,0.000000}%
\pgfsetfillcolor{currentfill}%
\pgfsetlinewidth{1.003750pt}%
\definecolor{currentstroke}{rgb}{0.000000,0.000000,0.000000}%
\pgfsetstrokecolor{currentstroke}%
\pgfsetdash{}{0pt}%
\pgfpathmoveto{\pgfqpoint{1.025793in}{2.334333in}}%
\pgfpathcurveto{\pgfqpoint{1.036843in}{2.334333in}}{\pgfqpoint{1.047442in}{2.338724in}}{\pgfqpoint{1.055256in}{2.346537in}}%
\pgfpathcurveto{\pgfqpoint{1.063070in}{2.354351in}}{\pgfqpoint{1.067460in}{2.364950in}}{\pgfqpoint{1.067460in}{2.376000in}}%
\pgfpathcurveto{\pgfqpoint{1.067460in}{2.387050in}}{\pgfqpoint{1.063070in}{2.397649in}}{\pgfqpoint{1.055256in}{2.405463in}}%
\pgfpathcurveto{\pgfqpoint{1.047442in}{2.413276in}}{\pgfqpoint{1.036843in}{2.417667in}}{\pgfqpoint{1.025793in}{2.417667in}}%
\pgfpathcurveto{\pgfqpoint{1.014743in}{2.417667in}}{\pgfqpoint{1.004144in}{2.413276in}}{\pgfqpoint{0.996330in}{2.405463in}}%
\pgfpathcurveto{\pgfqpoint{0.988517in}{2.397649in}}{\pgfqpoint{0.984126in}{2.387050in}}{\pgfqpoint{0.984126in}{2.376000in}}%
\pgfpathcurveto{\pgfqpoint{0.984126in}{2.364950in}}{\pgfqpoint{0.988517in}{2.354351in}}{\pgfqpoint{0.996330in}{2.346537in}}%
\pgfpathcurveto{\pgfqpoint{1.004144in}{2.338724in}}{\pgfqpoint{1.014743in}{2.334333in}}{\pgfqpoint{1.025793in}{2.334333in}}%
\pgfpathclose%
\pgfusepath{stroke,fill}%
\end{pgfscope}%
\begin{pgfscope}%
\pgfpathrectangle{\pgfqpoint{0.800000in}{0.528000in}}{\pgfqpoint{4.960000in}{3.696000in}}%
\pgfusepath{clip}%
\pgfsetbuttcap%
\pgfsetroundjoin%
\definecolor{currentfill}{rgb}{0.000000,0.000000,0.000000}%
\pgfsetfillcolor{currentfill}%
\pgfsetlinewidth{1.003750pt}%
\definecolor{currentstroke}{rgb}{0.000000,0.000000,0.000000}%
\pgfsetstrokecolor{currentstroke}%
\pgfsetdash{}{0pt}%
\pgfpathmoveto{\pgfqpoint{1.025793in}{2.334333in}}%
\pgfpathcurveto{\pgfqpoint{1.036843in}{2.334333in}}{\pgfqpoint{1.047442in}{2.338724in}}{\pgfqpoint{1.055256in}{2.346537in}}%
\pgfpathcurveto{\pgfqpoint{1.063070in}{2.354351in}}{\pgfqpoint{1.067460in}{2.364950in}}{\pgfqpoint{1.067460in}{2.376000in}}%
\pgfpathcurveto{\pgfqpoint{1.067460in}{2.387050in}}{\pgfqpoint{1.063070in}{2.397649in}}{\pgfqpoint{1.055256in}{2.405463in}}%
\pgfpathcurveto{\pgfqpoint{1.047442in}{2.413276in}}{\pgfqpoint{1.036843in}{2.417667in}}{\pgfqpoint{1.025793in}{2.417667in}}%
\pgfpathcurveto{\pgfqpoint{1.014743in}{2.417667in}}{\pgfqpoint{1.004144in}{2.413276in}}{\pgfqpoint{0.996330in}{2.405463in}}%
\pgfpathcurveto{\pgfqpoint{0.988517in}{2.397649in}}{\pgfqpoint{0.984126in}{2.387050in}}{\pgfqpoint{0.984126in}{2.376000in}}%
\pgfpathcurveto{\pgfqpoint{0.984126in}{2.364950in}}{\pgfqpoint{0.988517in}{2.354351in}}{\pgfqpoint{0.996330in}{2.346537in}}%
\pgfpathcurveto{\pgfqpoint{1.004144in}{2.338724in}}{\pgfqpoint{1.014743in}{2.334333in}}{\pgfqpoint{1.025793in}{2.334333in}}%
\pgfpathclose%
\pgfusepath{stroke,fill}%
\end{pgfscope}%
\begin{pgfscope}%
\pgfpathrectangle{\pgfqpoint{0.800000in}{0.528000in}}{\pgfqpoint{4.960000in}{3.696000in}}%
\pgfusepath{clip}%
\pgfsetbuttcap%
\pgfsetroundjoin%
\definecolor{currentfill}{rgb}{0.000000,0.000000,0.000000}%
\pgfsetfillcolor{currentfill}%
\pgfsetlinewidth{1.003750pt}%
\definecolor{currentstroke}{rgb}{0.000000,0.000000,0.000000}%
\pgfsetstrokecolor{currentstroke}%
\pgfsetdash{}{0pt}%
\pgfpathmoveto{\pgfqpoint{1.025793in}{2.334333in}}%
\pgfpathcurveto{\pgfqpoint{1.036843in}{2.334333in}}{\pgfqpoint{1.047442in}{2.338724in}}{\pgfqpoint{1.055256in}{2.346537in}}%
\pgfpathcurveto{\pgfqpoint{1.063070in}{2.354351in}}{\pgfqpoint{1.067460in}{2.364950in}}{\pgfqpoint{1.067460in}{2.376000in}}%
\pgfpathcurveto{\pgfqpoint{1.067460in}{2.387050in}}{\pgfqpoint{1.063070in}{2.397649in}}{\pgfqpoint{1.055256in}{2.405463in}}%
\pgfpathcurveto{\pgfqpoint{1.047442in}{2.413276in}}{\pgfqpoint{1.036843in}{2.417667in}}{\pgfqpoint{1.025793in}{2.417667in}}%
\pgfpathcurveto{\pgfqpoint{1.014743in}{2.417667in}}{\pgfqpoint{1.004144in}{2.413276in}}{\pgfqpoint{0.996330in}{2.405463in}}%
\pgfpathcurveto{\pgfqpoint{0.988517in}{2.397649in}}{\pgfqpoint{0.984126in}{2.387050in}}{\pgfqpoint{0.984126in}{2.376000in}}%
\pgfpathcurveto{\pgfqpoint{0.984126in}{2.364950in}}{\pgfqpoint{0.988517in}{2.354351in}}{\pgfqpoint{0.996330in}{2.346537in}}%
\pgfpathcurveto{\pgfqpoint{1.004144in}{2.338724in}}{\pgfqpoint{1.014743in}{2.334333in}}{\pgfqpoint{1.025793in}{2.334333in}}%
\pgfpathclose%
\pgfusepath{stroke,fill}%
\end{pgfscope}%
\begin{pgfscope}%
\pgfpathrectangle{\pgfqpoint{0.800000in}{0.528000in}}{\pgfqpoint{4.960000in}{3.696000in}}%
\pgfusepath{clip}%
\pgfsetbuttcap%
\pgfsetroundjoin%
\definecolor{currentfill}{rgb}{0.000000,0.000000,0.000000}%
\pgfsetfillcolor{currentfill}%
\pgfsetlinewidth{1.003750pt}%
\definecolor{currentstroke}{rgb}{0.000000,0.000000,0.000000}%
\pgfsetstrokecolor{currentstroke}%
\pgfsetdash{}{0pt}%
\pgfpathmoveto{\pgfqpoint{1.025793in}{2.334333in}}%
\pgfpathcurveto{\pgfqpoint{1.036843in}{2.334333in}}{\pgfqpoint{1.047442in}{2.338724in}}{\pgfqpoint{1.055256in}{2.346537in}}%
\pgfpathcurveto{\pgfqpoint{1.063070in}{2.354351in}}{\pgfqpoint{1.067460in}{2.364950in}}{\pgfqpoint{1.067460in}{2.376000in}}%
\pgfpathcurveto{\pgfqpoint{1.067460in}{2.387050in}}{\pgfqpoint{1.063070in}{2.397649in}}{\pgfqpoint{1.055256in}{2.405463in}}%
\pgfpathcurveto{\pgfqpoint{1.047442in}{2.413276in}}{\pgfqpoint{1.036843in}{2.417667in}}{\pgfqpoint{1.025793in}{2.417667in}}%
\pgfpathcurveto{\pgfqpoint{1.014743in}{2.417667in}}{\pgfqpoint{1.004144in}{2.413276in}}{\pgfqpoint{0.996330in}{2.405463in}}%
\pgfpathcurveto{\pgfqpoint{0.988517in}{2.397649in}}{\pgfqpoint{0.984126in}{2.387050in}}{\pgfqpoint{0.984126in}{2.376000in}}%
\pgfpathcurveto{\pgfqpoint{0.984126in}{2.364950in}}{\pgfqpoint{0.988517in}{2.354351in}}{\pgfqpoint{0.996330in}{2.346537in}}%
\pgfpathcurveto{\pgfqpoint{1.004144in}{2.338724in}}{\pgfqpoint{1.014743in}{2.334333in}}{\pgfqpoint{1.025793in}{2.334333in}}%
\pgfpathclose%
\pgfusepath{stroke,fill}%
\end{pgfscope}%
\begin{pgfscope}%
\pgfpathrectangle{\pgfqpoint{0.800000in}{0.528000in}}{\pgfqpoint{4.960000in}{3.696000in}}%
\pgfusepath{clip}%
\pgfsetbuttcap%
\pgfsetroundjoin%
\definecolor{currentfill}{rgb}{0.000000,0.000000,0.000000}%
\pgfsetfillcolor{currentfill}%
\pgfsetlinewidth{1.003750pt}%
\definecolor{currentstroke}{rgb}{0.000000,0.000000,0.000000}%
\pgfsetstrokecolor{currentstroke}%
\pgfsetdash{}{0pt}%
\pgfpathmoveto{\pgfqpoint{1.025793in}{2.334333in}}%
\pgfpathcurveto{\pgfqpoint{1.036843in}{2.334333in}}{\pgfqpoint{1.047442in}{2.338724in}}{\pgfqpoint{1.055256in}{2.346537in}}%
\pgfpathcurveto{\pgfqpoint{1.063070in}{2.354351in}}{\pgfqpoint{1.067460in}{2.364950in}}{\pgfqpoint{1.067460in}{2.376000in}}%
\pgfpathcurveto{\pgfqpoint{1.067460in}{2.387050in}}{\pgfqpoint{1.063070in}{2.397649in}}{\pgfqpoint{1.055256in}{2.405463in}}%
\pgfpathcurveto{\pgfqpoint{1.047442in}{2.413276in}}{\pgfqpoint{1.036843in}{2.417667in}}{\pgfqpoint{1.025793in}{2.417667in}}%
\pgfpathcurveto{\pgfqpoint{1.014743in}{2.417667in}}{\pgfqpoint{1.004144in}{2.413276in}}{\pgfqpoint{0.996330in}{2.405463in}}%
\pgfpathcurveto{\pgfqpoint{0.988517in}{2.397649in}}{\pgfqpoint{0.984126in}{2.387050in}}{\pgfqpoint{0.984126in}{2.376000in}}%
\pgfpathcurveto{\pgfqpoint{0.984126in}{2.364950in}}{\pgfqpoint{0.988517in}{2.354351in}}{\pgfqpoint{0.996330in}{2.346537in}}%
\pgfpathcurveto{\pgfqpoint{1.004144in}{2.338724in}}{\pgfqpoint{1.014743in}{2.334333in}}{\pgfqpoint{1.025793in}{2.334333in}}%
\pgfpathclose%
\pgfusepath{stroke,fill}%
\end{pgfscope}%
\begin{pgfscope}%
\pgfpathrectangle{\pgfqpoint{0.800000in}{0.528000in}}{\pgfqpoint{4.960000in}{3.696000in}}%
\pgfusepath{clip}%
\pgfsetbuttcap%
\pgfsetroundjoin%
\definecolor{currentfill}{rgb}{0.000000,0.000000,0.000000}%
\pgfsetfillcolor{currentfill}%
\pgfsetlinewidth{1.003750pt}%
\definecolor{currentstroke}{rgb}{0.000000,0.000000,0.000000}%
\pgfsetstrokecolor{currentstroke}%
\pgfsetdash{}{0pt}%
\pgfpathmoveto{\pgfqpoint{1.025793in}{2.334333in}}%
\pgfpathcurveto{\pgfqpoint{1.036843in}{2.334333in}}{\pgfqpoint{1.047442in}{2.338724in}}{\pgfqpoint{1.055256in}{2.346537in}}%
\pgfpathcurveto{\pgfqpoint{1.063070in}{2.354351in}}{\pgfqpoint{1.067460in}{2.364950in}}{\pgfqpoint{1.067460in}{2.376000in}}%
\pgfpathcurveto{\pgfqpoint{1.067460in}{2.387050in}}{\pgfqpoint{1.063070in}{2.397649in}}{\pgfqpoint{1.055256in}{2.405463in}}%
\pgfpathcurveto{\pgfqpoint{1.047442in}{2.413276in}}{\pgfqpoint{1.036843in}{2.417667in}}{\pgfqpoint{1.025793in}{2.417667in}}%
\pgfpathcurveto{\pgfqpoint{1.014743in}{2.417667in}}{\pgfqpoint{1.004144in}{2.413276in}}{\pgfqpoint{0.996330in}{2.405463in}}%
\pgfpathcurveto{\pgfqpoint{0.988517in}{2.397649in}}{\pgfqpoint{0.984126in}{2.387050in}}{\pgfqpoint{0.984126in}{2.376000in}}%
\pgfpathcurveto{\pgfqpoint{0.984126in}{2.364950in}}{\pgfqpoint{0.988517in}{2.354351in}}{\pgfqpoint{0.996330in}{2.346537in}}%
\pgfpathcurveto{\pgfqpoint{1.004144in}{2.338724in}}{\pgfqpoint{1.014743in}{2.334333in}}{\pgfqpoint{1.025793in}{2.334333in}}%
\pgfpathclose%
\pgfusepath{stroke,fill}%
\end{pgfscope}%
\begin{pgfscope}%
\pgfpathrectangle{\pgfqpoint{0.800000in}{0.528000in}}{\pgfqpoint{4.960000in}{3.696000in}}%
\pgfusepath{clip}%
\pgfsetbuttcap%
\pgfsetroundjoin%
\definecolor{currentfill}{rgb}{0.000000,0.000000,0.000000}%
\pgfsetfillcolor{currentfill}%
\pgfsetlinewidth{1.003750pt}%
\definecolor{currentstroke}{rgb}{0.000000,0.000000,0.000000}%
\pgfsetstrokecolor{currentstroke}%
\pgfsetdash{}{0pt}%
\pgfpathmoveto{\pgfqpoint{1.025793in}{2.334333in}}%
\pgfpathcurveto{\pgfqpoint{1.036843in}{2.334333in}}{\pgfqpoint{1.047442in}{2.338724in}}{\pgfqpoint{1.055256in}{2.346537in}}%
\pgfpathcurveto{\pgfqpoint{1.063070in}{2.354351in}}{\pgfqpoint{1.067460in}{2.364950in}}{\pgfqpoint{1.067460in}{2.376000in}}%
\pgfpathcurveto{\pgfqpoint{1.067460in}{2.387050in}}{\pgfqpoint{1.063070in}{2.397649in}}{\pgfqpoint{1.055256in}{2.405463in}}%
\pgfpathcurveto{\pgfqpoint{1.047442in}{2.413276in}}{\pgfqpoint{1.036843in}{2.417667in}}{\pgfqpoint{1.025793in}{2.417667in}}%
\pgfpathcurveto{\pgfqpoint{1.014743in}{2.417667in}}{\pgfqpoint{1.004144in}{2.413276in}}{\pgfqpoint{0.996330in}{2.405463in}}%
\pgfpathcurveto{\pgfqpoint{0.988517in}{2.397649in}}{\pgfqpoint{0.984126in}{2.387050in}}{\pgfqpoint{0.984126in}{2.376000in}}%
\pgfpathcurveto{\pgfqpoint{0.984126in}{2.364950in}}{\pgfqpoint{0.988517in}{2.354351in}}{\pgfqpoint{0.996330in}{2.346537in}}%
\pgfpathcurveto{\pgfqpoint{1.004144in}{2.338724in}}{\pgfqpoint{1.014743in}{2.334333in}}{\pgfqpoint{1.025793in}{2.334333in}}%
\pgfpathclose%
\pgfusepath{stroke,fill}%
\end{pgfscope}%
\begin{pgfscope}%
\pgfpathrectangle{\pgfqpoint{0.800000in}{0.528000in}}{\pgfqpoint{4.960000in}{3.696000in}}%
\pgfusepath{clip}%
\pgfsetbuttcap%
\pgfsetroundjoin%
\definecolor{currentfill}{rgb}{0.000000,0.000000,0.000000}%
\pgfsetfillcolor{currentfill}%
\pgfsetlinewidth{1.003750pt}%
\definecolor{currentstroke}{rgb}{0.000000,0.000000,0.000000}%
\pgfsetstrokecolor{currentstroke}%
\pgfsetdash{}{0pt}%
\pgfpathmoveto{\pgfqpoint{1.025793in}{2.334333in}}%
\pgfpathcurveto{\pgfqpoint{1.036843in}{2.334333in}}{\pgfqpoint{1.047442in}{2.338724in}}{\pgfqpoint{1.055256in}{2.346537in}}%
\pgfpathcurveto{\pgfqpoint{1.063070in}{2.354351in}}{\pgfqpoint{1.067460in}{2.364950in}}{\pgfqpoint{1.067460in}{2.376000in}}%
\pgfpathcurveto{\pgfqpoint{1.067460in}{2.387050in}}{\pgfqpoint{1.063070in}{2.397649in}}{\pgfqpoint{1.055256in}{2.405463in}}%
\pgfpathcurveto{\pgfqpoint{1.047442in}{2.413276in}}{\pgfqpoint{1.036843in}{2.417667in}}{\pgfqpoint{1.025793in}{2.417667in}}%
\pgfpathcurveto{\pgfqpoint{1.014743in}{2.417667in}}{\pgfqpoint{1.004144in}{2.413276in}}{\pgfqpoint{0.996330in}{2.405463in}}%
\pgfpathcurveto{\pgfqpoint{0.988517in}{2.397649in}}{\pgfqpoint{0.984126in}{2.387050in}}{\pgfqpoint{0.984126in}{2.376000in}}%
\pgfpathcurveto{\pgfqpoint{0.984126in}{2.364950in}}{\pgfqpoint{0.988517in}{2.354351in}}{\pgfqpoint{0.996330in}{2.346537in}}%
\pgfpathcurveto{\pgfqpoint{1.004144in}{2.338724in}}{\pgfqpoint{1.014743in}{2.334333in}}{\pgfqpoint{1.025793in}{2.334333in}}%
\pgfpathclose%
\pgfusepath{stroke,fill}%
\end{pgfscope}%
\begin{pgfscope}%
\pgfpathrectangle{\pgfqpoint{0.800000in}{0.528000in}}{\pgfqpoint{4.960000in}{3.696000in}}%
\pgfusepath{clip}%
\pgfsetbuttcap%
\pgfsetroundjoin%
\definecolor{currentfill}{rgb}{0.000000,0.000000,0.000000}%
\pgfsetfillcolor{currentfill}%
\pgfsetlinewidth{1.003750pt}%
\definecolor{currentstroke}{rgb}{0.000000,0.000000,0.000000}%
\pgfsetstrokecolor{currentstroke}%
\pgfsetdash{}{0pt}%
\pgfpathmoveto{\pgfqpoint{1.025793in}{2.334333in}}%
\pgfpathcurveto{\pgfqpoint{1.036843in}{2.334333in}}{\pgfqpoint{1.047442in}{2.338724in}}{\pgfqpoint{1.055256in}{2.346537in}}%
\pgfpathcurveto{\pgfqpoint{1.063070in}{2.354351in}}{\pgfqpoint{1.067460in}{2.364950in}}{\pgfqpoint{1.067460in}{2.376000in}}%
\pgfpathcurveto{\pgfqpoint{1.067460in}{2.387050in}}{\pgfqpoint{1.063070in}{2.397649in}}{\pgfqpoint{1.055256in}{2.405463in}}%
\pgfpathcurveto{\pgfqpoint{1.047442in}{2.413276in}}{\pgfqpoint{1.036843in}{2.417667in}}{\pgfqpoint{1.025793in}{2.417667in}}%
\pgfpathcurveto{\pgfqpoint{1.014743in}{2.417667in}}{\pgfqpoint{1.004144in}{2.413276in}}{\pgfqpoint{0.996330in}{2.405463in}}%
\pgfpathcurveto{\pgfqpoint{0.988517in}{2.397649in}}{\pgfqpoint{0.984126in}{2.387050in}}{\pgfqpoint{0.984126in}{2.376000in}}%
\pgfpathcurveto{\pgfqpoint{0.984126in}{2.364950in}}{\pgfqpoint{0.988517in}{2.354351in}}{\pgfqpoint{0.996330in}{2.346537in}}%
\pgfpathcurveto{\pgfqpoint{1.004144in}{2.338724in}}{\pgfqpoint{1.014743in}{2.334333in}}{\pgfqpoint{1.025793in}{2.334333in}}%
\pgfpathclose%
\pgfusepath{stroke,fill}%
\end{pgfscope}%
\begin{pgfscope}%
\pgfpathrectangle{\pgfqpoint{0.800000in}{0.528000in}}{\pgfqpoint{4.960000in}{3.696000in}}%
\pgfusepath{clip}%
\pgfsetbuttcap%
\pgfsetroundjoin%
\definecolor{currentfill}{rgb}{0.000000,0.000000,0.000000}%
\pgfsetfillcolor{currentfill}%
\pgfsetlinewidth{1.003750pt}%
\definecolor{currentstroke}{rgb}{0.000000,0.000000,0.000000}%
\pgfsetstrokecolor{currentstroke}%
\pgfsetdash{}{0pt}%
\pgfpathmoveto{\pgfqpoint{1.025793in}{2.334333in}}%
\pgfpathcurveto{\pgfqpoint{1.036843in}{2.334333in}}{\pgfqpoint{1.047442in}{2.338724in}}{\pgfqpoint{1.055256in}{2.346537in}}%
\pgfpathcurveto{\pgfqpoint{1.063070in}{2.354351in}}{\pgfqpoint{1.067460in}{2.364950in}}{\pgfqpoint{1.067460in}{2.376000in}}%
\pgfpathcurveto{\pgfqpoint{1.067460in}{2.387050in}}{\pgfqpoint{1.063070in}{2.397649in}}{\pgfqpoint{1.055256in}{2.405463in}}%
\pgfpathcurveto{\pgfqpoint{1.047442in}{2.413276in}}{\pgfqpoint{1.036843in}{2.417667in}}{\pgfqpoint{1.025793in}{2.417667in}}%
\pgfpathcurveto{\pgfqpoint{1.014743in}{2.417667in}}{\pgfqpoint{1.004144in}{2.413276in}}{\pgfqpoint{0.996330in}{2.405463in}}%
\pgfpathcurveto{\pgfqpoint{0.988517in}{2.397649in}}{\pgfqpoint{0.984126in}{2.387050in}}{\pgfqpoint{0.984126in}{2.376000in}}%
\pgfpathcurveto{\pgfqpoint{0.984126in}{2.364950in}}{\pgfqpoint{0.988517in}{2.354351in}}{\pgfqpoint{0.996330in}{2.346537in}}%
\pgfpathcurveto{\pgfqpoint{1.004144in}{2.338724in}}{\pgfqpoint{1.014743in}{2.334333in}}{\pgfqpoint{1.025793in}{2.334333in}}%
\pgfpathclose%
\pgfusepath{stroke,fill}%
\end{pgfscope}%
\begin{pgfscope}%
\pgfpathrectangle{\pgfqpoint{0.800000in}{0.528000in}}{\pgfqpoint{4.960000in}{3.696000in}}%
\pgfusepath{clip}%
\pgfsetbuttcap%
\pgfsetroundjoin%
\definecolor{currentfill}{rgb}{0.000000,0.000000,0.000000}%
\pgfsetfillcolor{currentfill}%
\pgfsetlinewidth{1.003750pt}%
\definecolor{currentstroke}{rgb}{0.000000,0.000000,0.000000}%
\pgfsetstrokecolor{currentstroke}%
\pgfsetdash{}{0pt}%
\pgfpathmoveto{\pgfqpoint{1.025793in}{2.334333in}}%
\pgfpathcurveto{\pgfqpoint{1.036843in}{2.334333in}}{\pgfqpoint{1.047442in}{2.338724in}}{\pgfqpoint{1.055256in}{2.346537in}}%
\pgfpathcurveto{\pgfqpoint{1.063070in}{2.354351in}}{\pgfqpoint{1.067460in}{2.364950in}}{\pgfqpoint{1.067460in}{2.376000in}}%
\pgfpathcurveto{\pgfqpoint{1.067460in}{2.387050in}}{\pgfqpoint{1.063070in}{2.397649in}}{\pgfqpoint{1.055256in}{2.405463in}}%
\pgfpathcurveto{\pgfqpoint{1.047442in}{2.413276in}}{\pgfqpoint{1.036843in}{2.417667in}}{\pgfqpoint{1.025793in}{2.417667in}}%
\pgfpathcurveto{\pgfqpoint{1.014743in}{2.417667in}}{\pgfqpoint{1.004144in}{2.413276in}}{\pgfqpoint{0.996330in}{2.405463in}}%
\pgfpathcurveto{\pgfqpoint{0.988517in}{2.397649in}}{\pgfqpoint{0.984126in}{2.387050in}}{\pgfqpoint{0.984126in}{2.376000in}}%
\pgfpathcurveto{\pgfqpoint{0.984126in}{2.364950in}}{\pgfqpoint{0.988517in}{2.354351in}}{\pgfqpoint{0.996330in}{2.346537in}}%
\pgfpathcurveto{\pgfqpoint{1.004144in}{2.338724in}}{\pgfqpoint{1.014743in}{2.334333in}}{\pgfqpoint{1.025793in}{2.334333in}}%
\pgfpathclose%
\pgfusepath{stroke,fill}%
\end{pgfscope}%
\begin{pgfscope}%
\pgfpathrectangle{\pgfqpoint{0.800000in}{0.528000in}}{\pgfqpoint{4.960000in}{3.696000in}}%
\pgfusepath{clip}%
\pgfsetbuttcap%
\pgfsetroundjoin%
\definecolor{currentfill}{rgb}{0.000000,0.000000,0.000000}%
\pgfsetfillcolor{currentfill}%
\pgfsetlinewidth{1.003750pt}%
\definecolor{currentstroke}{rgb}{0.000000,0.000000,0.000000}%
\pgfsetstrokecolor{currentstroke}%
\pgfsetdash{}{0pt}%
\pgfpathmoveto{\pgfqpoint{1.025793in}{2.334333in}}%
\pgfpathcurveto{\pgfqpoint{1.036843in}{2.334333in}}{\pgfqpoint{1.047442in}{2.338724in}}{\pgfqpoint{1.055256in}{2.346537in}}%
\pgfpathcurveto{\pgfqpoint{1.063070in}{2.354351in}}{\pgfqpoint{1.067460in}{2.364950in}}{\pgfqpoint{1.067460in}{2.376000in}}%
\pgfpathcurveto{\pgfqpoint{1.067460in}{2.387050in}}{\pgfqpoint{1.063070in}{2.397649in}}{\pgfqpoint{1.055256in}{2.405463in}}%
\pgfpathcurveto{\pgfqpoint{1.047442in}{2.413276in}}{\pgfqpoint{1.036843in}{2.417667in}}{\pgfqpoint{1.025793in}{2.417667in}}%
\pgfpathcurveto{\pgfqpoint{1.014743in}{2.417667in}}{\pgfqpoint{1.004144in}{2.413276in}}{\pgfqpoint{0.996330in}{2.405463in}}%
\pgfpathcurveto{\pgfqpoint{0.988517in}{2.397649in}}{\pgfqpoint{0.984126in}{2.387050in}}{\pgfqpoint{0.984126in}{2.376000in}}%
\pgfpathcurveto{\pgfqpoint{0.984126in}{2.364950in}}{\pgfqpoint{0.988517in}{2.354351in}}{\pgfqpoint{0.996330in}{2.346537in}}%
\pgfpathcurveto{\pgfqpoint{1.004144in}{2.338724in}}{\pgfqpoint{1.014743in}{2.334333in}}{\pgfqpoint{1.025793in}{2.334333in}}%
\pgfpathclose%
\pgfusepath{stroke,fill}%
\end{pgfscope}%
\begin{pgfscope}%
\pgfpathrectangle{\pgfqpoint{0.800000in}{0.528000in}}{\pgfqpoint{4.960000in}{3.696000in}}%
\pgfusepath{clip}%
\pgfsetbuttcap%
\pgfsetroundjoin%
\definecolor{currentfill}{rgb}{0.000000,0.000000,0.000000}%
\pgfsetfillcolor{currentfill}%
\pgfsetlinewidth{1.003750pt}%
\definecolor{currentstroke}{rgb}{0.000000,0.000000,0.000000}%
\pgfsetstrokecolor{currentstroke}%
\pgfsetdash{}{0pt}%
\pgfpathmoveto{\pgfqpoint{1.025793in}{2.334333in}}%
\pgfpathcurveto{\pgfqpoint{1.036843in}{2.334333in}}{\pgfqpoint{1.047442in}{2.338724in}}{\pgfqpoint{1.055256in}{2.346537in}}%
\pgfpathcurveto{\pgfqpoint{1.063070in}{2.354351in}}{\pgfqpoint{1.067460in}{2.364950in}}{\pgfqpoint{1.067460in}{2.376000in}}%
\pgfpathcurveto{\pgfqpoint{1.067460in}{2.387050in}}{\pgfqpoint{1.063070in}{2.397649in}}{\pgfqpoint{1.055256in}{2.405463in}}%
\pgfpathcurveto{\pgfqpoint{1.047442in}{2.413276in}}{\pgfqpoint{1.036843in}{2.417667in}}{\pgfqpoint{1.025793in}{2.417667in}}%
\pgfpathcurveto{\pgfqpoint{1.014743in}{2.417667in}}{\pgfqpoint{1.004144in}{2.413276in}}{\pgfqpoint{0.996330in}{2.405463in}}%
\pgfpathcurveto{\pgfqpoint{0.988517in}{2.397649in}}{\pgfqpoint{0.984126in}{2.387050in}}{\pgfqpoint{0.984126in}{2.376000in}}%
\pgfpathcurveto{\pgfqpoint{0.984126in}{2.364950in}}{\pgfqpoint{0.988517in}{2.354351in}}{\pgfqpoint{0.996330in}{2.346537in}}%
\pgfpathcurveto{\pgfqpoint{1.004144in}{2.338724in}}{\pgfqpoint{1.014743in}{2.334333in}}{\pgfqpoint{1.025793in}{2.334333in}}%
\pgfpathclose%
\pgfusepath{stroke,fill}%
\end{pgfscope}%
\begin{pgfscope}%
\pgfpathrectangle{\pgfqpoint{0.800000in}{0.528000in}}{\pgfqpoint{4.960000in}{3.696000in}}%
\pgfusepath{clip}%
\pgfsetbuttcap%
\pgfsetroundjoin%
\definecolor{currentfill}{rgb}{0.000000,0.000000,0.000000}%
\pgfsetfillcolor{currentfill}%
\pgfsetlinewidth{1.003750pt}%
\definecolor{currentstroke}{rgb}{0.000000,0.000000,0.000000}%
\pgfsetstrokecolor{currentstroke}%
\pgfsetdash{}{0pt}%
\pgfpathmoveto{\pgfqpoint{1.025793in}{2.334333in}}%
\pgfpathcurveto{\pgfqpoint{1.036843in}{2.334333in}}{\pgfqpoint{1.047442in}{2.338724in}}{\pgfqpoint{1.055256in}{2.346537in}}%
\pgfpathcurveto{\pgfqpoint{1.063070in}{2.354351in}}{\pgfqpoint{1.067460in}{2.364950in}}{\pgfqpoint{1.067460in}{2.376000in}}%
\pgfpathcurveto{\pgfqpoint{1.067460in}{2.387050in}}{\pgfqpoint{1.063070in}{2.397649in}}{\pgfqpoint{1.055256in}{2.405463in}}%
\pgfpathcurveto{\pgfqpoint{1.047442in}{2.413276in}}{\pgfqpoint{1.036843in}{2.417667in}}{\pgfqpoint{1.025793in}{2.417667in}}%
\pgfpathcurveto{\pgfqpoint{1.014743in}{2.417667in}}{\pgfqpoint{1.004144in}{2.413276in}}{\pgfqpoint{0.996330in}{2.405463in}}%
\pgfpathcurveto{\pgfqpoint{0.988517in}{2.397649in}}{\pgfqpoint{0.984126in}{2.387050in}}{\pgfqpoint{0.984126in}{2.376000in}}%
\pgfpathcurveto{\pgfqpoint{0.984126in}{2.364950in}}{\pgfqpoint{0.988517in}{2.354351in}}{\pgfqpoint{0.996330in}{2.346537in}}%
\pgfpathcurveto{\pgfqpoint{1.004144in}{2.338724in}}{\pgfqpoint{1.014743in}{2.334333in}}{\pgfqpoint{1.025793in}{2.334333in}}%
\pgfpathclose%
\pgfusepath{stroke,fill}%
\end{pgfscope}%
\begin{pgfscope}%
\pgfpathrectangle{\pgfqpoint{0.800000in}{0.528000in}}{\pgfqpoint{4.960000in}{3.696000in}}%
\pgfusepath{clip}%
\pgfsetbuttcap%
\pgfsetroundjoin%
\definecolor{currentfill}{rgb}{0.000000,0.000000,0.000000}%
\pgfsetfillcolor{currentfill}%
\pgfsetlinewidth{1.003750pt}%
\definecolor{currentstroke}{rgb}{0.000000,0.000000,0.000000}%
\pgfsetstrokecolor{currentstroke}%
\pgfsetdash{}{0pt}%
\pgfpathmoveto{\pgfqpoint{1.025793in}{2.334333in}}%
\pgfpathcurveto{\pgfqpoint{1.036843in}{2.334333in}}{\pgfqpoint{1.047442in}{2.338724in}}{\pgfqpoint{1.055256in}{2.346537in}}%
\pgfpathcurveto{\pgfqpoint{1.063070in}{2.354351in}}{\pgfqpoint{1.067460in}{2.364950in}}{\pgfqpoint{1.067460in}{2.376000in}}%
\pgfpathcurveto{\pgfqpoint{1.067460in}{2.387050in}}{\pgfqpoint{1.063070in}{2.397649in}}{\pgfqpoint{1.055256in}{2.405463in}}%
\pgfpathcurveto{\pgfqpoint{1.047442in}{2.413276in}}{\pgfqpoint{1.036843in}{2.417667in}}{\pgfqpoint{1.025793in}{2.417667in}}%
\pgfpathcurveto{\pgfqpoint{1.014743in}{2.417667in}}{\pgfqpoint{1.004144in}{2.413276in}}{\pgfqpoint{0.996330in}{2.405463in}}%
\pgfpathcurveto{\pgfqpoint{0.988517in}{2.397649in}}{\pgfqpoint{0.984126in}{2.387050in}}{\pgfqpoint{0.984126in}{2.376000in}}%
\pgfpathcurveto{\pgfqpoint{0.984126in}{2.364950in}}{\pgfqpoint{0.988517in}{2.354351in}}{\pgfqpoint{0.996330in}{2.346537in}}%
\pgfpathcurveto{\pgfqpoint{1.004144in}{2.338724in}}{\pgfqpoint{1.014743in}{2.334333in}}{\pgfqpoint{1.025793in}{2.334333in}}%
\pgfpathclose%
\pgfusepath{stroke,fill}%
\end{pgfscope}%
\begin{pgfscope}%
\pgfpathrectangle{\pgfqpoint{0.800000in}{0.528000in}}{\pgfqpoint{4.960000in}{3.696000in}}%
\pgfusepath{clip}%
\pgfsetbuttcap%
\pgfsetroundjoin%
\definecolor{currentfill}{rgb}{0.000000,0.000000,0.000000}%
\pgfsetfillcolor{currentfill}%
\pgfsetlinewidth{1.003750pt}%
\definecolor{currentstroke}{rgb}{0.000000,0.000000,0.000000}%
\pgfsetstrokecolor{currentstroke}%
\pgfsetdash{}{0pt}%
\pgfpathmoveto{\pgfqpoint{1.025793in}{2.334333in}}%
\pgfpathcurveto{\pgfqpoint{1.036843in}{2.334333in}}{\pgfqpoint{1.047442in}{2.338724in}}{\pgfqpoint{1.055256in}{2.346537in}}%
\pgfpathcurveto{\pgfqpoint{1.063070in}{2.354351in}}{\pgfqpoint{1.067460in}{2.364950in}}{\pgfqpoint{1.067460in}{2.376000in}}%
\pgfpathcurveto{\pgfqpoint{1.067460in}{2.387050in}}{\pgfqpoint{1.063070in}{2.397649in}}{\pgfqpoint{1.055256in}{2.405463in}}%
\pgfpathcurveto{\pgfqpoint{1.047442in}{2.413276in}}{\pgfqpoint{1.036843in}{2.417667in}}{\pgfqpoint{1.025793in}{2.417667in}}%
\pgfpathcurveto{\pgfqpoint{1.014743in}{2.417667in}}{\pgfqpoint{1.004144in}{2.413276in}}{\pgfqpoint{0.996330in}{2.405463in}}%
\pgfpathcurveto{\pgfqpoint{0.988517in}{2.397649in}}{\pgfqpoint{0.984126in}{2.387050in}}{\pgfqpoint{0.984126in}{2.376000in}}%
\pgfpathcurveto{\pgfqpoint{0.984126in}{2.364950in}}{\pgfqpoint{0.988517in}{2.354351in}}{\pgfqpoint{0.996330in}{2.346537in}}%
\pgfpathcurveto{\pgfqpoint{1.004144in}{2.338724in}}{\pgfqpoint{1.014743in}{2.334333in}}{\pgfqpoint{1.025793in}{2.334333in}}%
\pgfpathclose%
\pgfusepath{stroke,fill}%
\end{pgfscope}%
\begin{pgfscope}%
\pgfpathrectangle{\pgfqpoint{0.800000in}{0.528000in}}{\pgfqpoint{4.960000in}{3.696000in}}%
\pgfusepath{clip}%
\pgfsetbuttcap%
\pgfsetroundjoin%
\definecolor{currentfill}{rgb}{0.000000,0.000000,0.000000}%
\pgfsetfillcolor{currentfill}%
\pgfsetlinewidth{1.003750pt}%
\definecolor{currentstroke}{rgb}{0.000000,0.000000,0.000000}%
\pgfsetstrokecolor{currentstroke}%
\pgfsetdash{}{0pt}%
\pgfpathmoveto{\pgfqpoint{1.025793in}{2.334333in}}%
\pgfpathcurveto{\pgfqpoint{1.036843in}{2.334333in}}{\pgfqpoint{1.047442in}{2.338724in}}{\pgfqpoint{1.055256in}{2.346537in}}%
\pgfpathcurveto{\pgfqpoint{1.063070in}{2.354351in}}{\pgfqpoint{1.067460in}{2.364950in}}{\pgfqpoint{1.067460in}{2.376000in}}%
\pgfpathcurveto{\pgfqpoint{1.067460in}{2.387050in}}{\pgfqpoint{1.063070in}{2.397649in}}{\pgfqpoint{1.055256in}{2.405463in}}%
\pgfpathcurveto{\pgfqpoint{1.047442in}{2.413276in}}{\pgfqpoint{1.036843in}{2.417667in}}{\pgfqpoint{1.025793in}{2.417667in}}%
\pgfpathcurveto{\pgfqpoint{1.014743in}{2.417667in}}{\pgfqpoint{1.004144in}{2.413276in}}{\pgfqpoint{0.996330in}{2.405463in}}%
\pgfpathcurveto{\pgfqpoint{0.988517in}{2.397649in}}{\pgfqpoint{0.984126in}{2.387050in}}{\pgfqpoint{0.984126in}{2.376000in}}%
\pgfpathcurveto{\pgfqpoint{0.984126in}{2.364950in}}{\pgfqpoint{0.988517in}{2.354351in}}{\pgfqpoint{0.996330in}{2.346537in}}%
\pgfpathcurveto{\pgfqpoint{1.004144in}{2.338724in}}{\pgfqpoint{1.014743in}{2.334333in}}{\pgfqpoint{1.025793in}{2.334333in}}%
\pgfpathclose%
\pgfusepath{stroke,fill}%
\end{pgfscope}%
\begin{pgfscope}%
\pgfpathrectangle{\pgfqpoint{0.800000in}{0.528000in}}{\pgfqpoint{4.960000in}{3.696000in}}%
\pgfusepath{clip}%
\pgfsetbuttcap%
\pgfsetroundjoin%
\definecolor{currentfill}{rgb}{0.000000,0.000000,0.000000}%
\pgfsetfillcolor{currentfill}%
\pgfsetlinewidth{1.003750pt}%
\definecolor{currentstroke}{rgb}{0.000000,0.000000,0.000000}%
\pgfsetstrokecolor{currentstroke}%
\pgfsetdash{}{0pt}%
\pgfpathmoveto{\pgfqpoint{1.025793in}{2.334333in}}%
\pgfpathcurveto{\pgfqpoint{1.036843in}{2.334333in}}{\pgfqpoint{1.047442in}{2.338724in}}{\pgfqpoint{1.055256in}{2.346537in}}%
\pgfpathcurveto{\pgfqpoint{1.063070in}{2.354351in}}{\pgfqpoint{1.067460in}{2.364950in}}{\pgfqpoint{1.067460in}{2.376000in}}%
\pgfpathcurveto{\pgfqpoint{1.067460in}{2.387050in}}{\pgfqpoint{1.063070in}{2.397649in}}{\pgfqpoint{1.055256in}{2.405463in}}%
\pgfpathcurveto{\pgfqpoint{1.047442in}{2.413276in}}{\pgfqpoint{1.036843in}{2.417667in}}{\pgfqpoint{1.025793in}{2.417667in}}%
\pgfpathcurveto{\pgfqpoint{1.014743in}{2.417667in}}{\pgfqpoint{1.004144in}{2.413276in}}{\pgfqpoint{0.996330in}{2.405463in}}%
\pgfpathcurveto{\pgfqpoint{0.988517in}{2.397649in}}{\pgfqpoint{0.984126in}{2.387050in}}{\pgfqpoint{0.984126in}{2.376000in}}%
\pgfpathcurveto{\pgfqpoint{0.984126in}{2.364950in}}{\pgfqpoint{0.988517in}{2.354351in}}{\pgfqpoint{0.996330in}{2.346537in}}%
\pgfpathcurveto{\pgfqpoint{1.004144in}{2.338724in}}{\pgfqpoint{1.014743in}{2.334333in}}{\pgfqpoint{1.025793in}{2.334333in}}%
\pgfpathclose%
\pgfusepath{stroke,fill}%
\end{pgfscope}%
\begin{pgfscope}%
\pgfpathrectangle{\pgfqpoint{0.800000in}{0.528000in}}{\pgfqpoint{4.960000in}{3.696000in}}%
\pgfusepath{clip}%
\pgfsetbuttcap%
\pgfsetroundjoin%
\definecolor{currentfill}{rgb}{0.000000,0.000000,0.000000}%
\pgfsetfillcolor{currentfill}%
\pgfsetlinewidth{1.003750pt}%
\definecolor{currentstroke}{rgb}{0.000000,0.000000,0.000000}%
\pgfsetstrokecolor{currentstroke}%
\pgfsetdash{}{0pt}%
\pgfpathmoveto{\pgfqpoint{1.025793in}{2.334333in}}%
\pgfpathcurveto{\pgfqpoint{1.036843in}{2.334333in}}{\pgfqpoint{1.047442in}{2.338724in}}{\pgfqpoint{1.055256in}{2.346537in}}%
\pgfpathcurveto{\pgfqpoint{1.063070in}{2.354351in}}{\pgfqpoint{1.067460in}{2.364950in}}{\pgfqpoint{1.067460in}{2.376000in}}%
\pgfpathcurveto{\pgfqpoint{1.067460in}{2.387050in}}{\pgfqpoint{1.063070in}{2.397649in}}{\pgfqpoint{1.055256in}{2.405463in}}%
\pgfpathcurveto{\pgfqpoint{1.047442in}{2.413276in}}{\pgfqpoint{1.036843in}{2.417667in}}{\pgfqpoint{1.025793in}{2.417667in}}%
\pgfpathcurveto{\pgfqpoint{1.014743in}{2.417667in}}{\pgfqpoint{1.004144in}{2.413276in}}{\pgfqpoint{0.996330in}{2.405463in}}%
\pgfpathcurveto{\pgfqpoint{0.988517in}{2.397649in}}{\pgfqpoint{0.984126in}{2.387050in}}{\pgfqpoint{0.984126in}{2.376000in}}%
\pgfpathcurveto{\pgfqpoint{0.984126in}{2.364950in}}{\pgfqpoint{0.988517in}{2.354351in}}{\pgfqpoint{0.996330in}{2.346537in}}%
\pgfpathcurveto{\pgfqpoint{1.004144in}{2.338724in}}{\pgfqpoint{1.014743in}{2.334333in}}{\pgfqpoint{1.025793in}{2.334333in}}%
\pgfpathclose%
\pgfusepath{stroke,fill}%
\end{pgfscope}%
\begin{pgfscope}%
\pgfpathrectangle{\pgfqpoint{0.800000in}{0.528000in}}{\pgfqpoint{4.960000in}{3.696000in}}%
\pgfusepath{clip}%
\pgfsetbuttcap%
\pgfsetroundjoin%
\definecolor{currentfill}{rgb}{0.000000,0.000000,0.000000}%
\pgfsetfillcolor{currentfill}%
\pgfsetlinewidth{1.003750pt}%
\definecolor{currentstroke}{rgb}{0.000000,0.000000,0.000000}%
\pgfsetstrokecolor{currentstroke}%
\pgfsetdash{}{0pt}%
\pgfpathmoveto{\pgfqpoint{1.025793in}{2.334333in}}%
\pgfpathcurveto{\pgfqpoint{1.036843in}{2.334333in}}{\pgfqpoint{1.047442in}{2.338724in}}{\pgfqpoint{1.055256in}{2.346537in}}%
\pgfpathcurveto{\pgfqpoint{1.063070in}{2.354351in}}{\pgfqpoint{1.067460in}{2.364950in}}{\pgfqpoint{1.067460in}{2.376000in}}%
\pgfpathcurveto{\pgfqpoint{1.067460in}{2.387050in}}{\pgfqpoint{1.063070in}{2.397649in}}{\pgfqpoint{1.055256in}{2.405463in}}%
\pgfpathcurveto{\pgfqpoint{1.047442in}{2.413276in}}{\pgfqpoint{1.036843in}{2.417667in}}{\pgfqpoint{1.025793in}{2.417667in}}%
\pgfpathcurveto{\pgfqpoint{1.014743in}{2.417667in}}{\pgfqpoint{1.004144in}{2.413276in}}{\pgfqpoint{0.996330in}{2.405463in}}%
\pgfpathcurveto{\pgfqpoint{0.988517in}{2.397649in}}{\pgfqpoint{0.984126in}{2.387050in}}{\pgfqpoint{0.984126in}{2.376000in}}%
\pgfpathcurveto{\pgfqpoint{0.984126in}{2.364950in}}{\pgfqpoint{0.988517in}{2.354351in}}{\pgfqpoint{0.996330in}{2.346537in}}%
\pgfpathcurveto{\pgfqpoint{1.004144in}{2.338724in}}{\pgfqpoint{1.014743in}{2.334333in}}{\pgfqpoint{1.025793in}{2.334333in}}%
\pgfpathclose%
\pgfusepath{stroke,fill}%
\end{pgfscope}%
\begin{pgfscope}%
\pgfpathrectangle{\pgfqpoint{0.800000in}{0.528000in}}{\pgfqpoint{4.960000in}{3.696000in}}%
\pgfusepath{clip}%
\pgfsetbuttcap%
\pgfsetroundjoin%
\definecolor{currentfill}{rgb}{0.000000,0.000000,0.000000}%
\pgfsetfillcolor{currentfill}%
\pgfsetlinewidth{1.003750pt}%
\definecolor{currentstroke}{rgb}{0.000000,0.000000,0.000000}%
\pgfsetstrokecolor{currentstroke}%
\pgfsetdash{}{0pt}%
\pgfpathmoveto{\pgfqpoint{1.025793in}{2.334333in}}%
\pgfpathcurveto{\pgfqpoint{1.036843in}{2.334333in}}{\pgfqpoint{1.047442in}{2.338724in}}{\pgfqpoint{1.055256in}{2.346537in}}%
\pgfpathcurveto{\pgfqpoint{1.063070in}{2.354351in}}{\pgfqpoint{1.067460in}{2.364950in}}{\pgfqpoint{1.067460in}{2.376000in}}%
\pgfpathcurveto{\pgfqpoint{1.067460in}{2.387050in}}{\pgfqpoint{1.063070in}{2.397649in}}{\pgfqpoint{1.055256in}{2.405463in}}%
\pgfpathcurveto{\pgfqpoint{1.047442in}{2.413276in}}{\pgfqpoint{1.036843in}{2.417667in}}{\pgfqpoint{1.025793in}{2.417667in}}%
\pgfpathcurveto{\pgfqpoint{1.014743in}{2.417667in}}{\pgfqpoint{1.004144in}{2.413276in}}{\pgfqpoint{0.996330in}{2.405463in}}%
\pgfpathcurveto{\pgfqpoint{0.988517in}{2.397649in}}{\pgfqpoint{0.984126in}{2.387050in}}{\pgfqpoint{0.984126in}{2.376000in}}%
\pgfpathcurveto{\pgfqpoint{0.984126in}{2.364950in}}{\pgfqpoint{0.988517in}{2.354351in}}{\pgfqpoint{0.996330in}{2.346537in}}%
\pgfpathcurveto{\pgfqpoint{1.004144in}{2.338724in}}{\pgfqpoint{1.014743in}{2.334333in}}{\pgfqpoint{1.025793in}{2.334333in}}%
\pgfpathclose%
\pgfusepath{stroke,fill}%
\end{pgfscope}%
\begin{pgfscope}%
\pgfpathrectangle{\pgfqpoint{0.800000in}{0.528000in}}{\pgfqpoint{4.960000in}{3.696000in}}%
\pgfusepath{clip}%
\pgfsetbuttcap%
\pgfsetroundjoin%
\definecolor{currentfill}{rgb}{0.000000,0.000000,0.000000}%
\pgfsetfillcolor{currentfill}%
\pgfsetlinewidth{1.003750pt}%
\definecolor{currentstroke}{rgb}{0.000000,0.000000,0.000000}%
\pgfsetstrokecolor{currentstroke}%
\pgfsetdash{}{0pt}%
\pgfpathmoveto{\pgfqpoint{1.025793in}{2.334333in}}%
\pgfpathcurveto{\pgfqpoint{1.036843in}{2.334333in}}{\pgfqpoint{1.047442in}{2.338724in}}{\pgfqpoint{1.055256in}{2.346537in}}%
\pgfpathcurveto{\pgfqpoint{1.063070in}{2.354351in}}{\pgfqpoint{1.067460in}{2.364950in}}{\pgfqpoint{1.067460in}{2.376000in}}%
\pgfpathcurveto{\pgfqpoint{1.067460in}{2.387050in}}{\pgfqpoint{1.063070in}{2.397649in}}{\pgfqpoint{1.055256in}{2.405463in}}%
\pgfpathcurveto{\pgfqpoint{1.047442in}{2.413276in}}{\pgfqpoint{1.036843in}{2.417667in}}{\pgfqpoint{1.025793in}{2.417667in}}%
\pgfpathcurveto{\pgfqpoint{1.014743in}{2.417667in}}{\pgfqpoint{1.004144in}{2.413276in}}{\pgfqpoint{0.996330in}{2.405463in}}%
\pgfpathcurveto{\pgfqpoint{0.988517in}{2.397649in}}{\pgfqpoint{0.984126in}{2.387050in}}{\pgfqpoint{0.984126in}{2.376000in}}%
\pgfpathcurveto{\pgfqpoint{0.984126in}{2.364950in}}{\pgfqpoint{0.988517in}{2.354351in}}{\pgfqpoint{0.996330in}{2.346537in}}%
\pgfpathcurveto{\pgfqpoint{1.004144in}{2.338724in}}{\pgfqpoint{1.014743in}{2.334333in}}{\pgfqpoint{1.025793in}{2.334333in}}%
\pgfpathclose%
\pgfusepath{stroke,fill}%
\end{pgfscope}%
\begin{pgfscope}%
\pgfpathrectangle{\pgfqpoint{0.800000in}{0.528000in}}{\pgfqpoint{4.960000in}{3.696000in}}%
\pgfusepath{clip}%
\pgfsetbuttcap%
\pgfsetroundjoin%
\definecolor{currentfill}{rgb}{0.000000,0.000000,0.000000}%
\pgfsetfillcolor{currentfill}%
\pgfsetlinewidth{1.003750pt}%
\definecolor{currentstroke}{rgb}{0.000000,0.000000,0.000000}%
\pgfsetstrokecolor{currentstroke}%
\pgfsetdash{}{0pt}%
\pgfpathmoveto{\pgfqpoint{1.025793in}{2.334333in}}%
\pgfpathcurveto{\pgfqpoint{1.036843in}{2.334333in}}{\pgfqpoint{1.047442in}{2.338724in}}{\pgfqpoint{1.055256in}{2.346537in}}%
\pgfpathcurveto{\pgfqpoint{1.063070in}{2.354351in}}{\pgfqpoint{1.067460in}{2.364950in}}{\pgfqpoint{1.067460in}{2.376000in}}%
\pgfpathcurveto{\pgfqpoint{1.067460in}{2.387050in}}{\pgfqpoint{1.063070in}{2.397649in}}{\pgfqpoint{1.055256in}{2.405463in}}%
\pgfpathcurveto{\pgfqpoint{1.047442in}{2.413276in}}{\pgfqpoint{1.036843in}{2.417667in}}{\pgfqpoint{1.025793in}{2.417667in}}%
\pgfpathcurveto{\pgfqpoint{1.014743in}{2.417667in}}{\pgfqpoint{1.004144in}{2.413276in}}{\pgfqpoint{0.996330in}{2.405463in}}%
\pgfpathcurveto{\pgfqpoint{0.988517in}{2.397649in}}{\pgfqpoint{0.984126in}{2.387050in}}{\pgfqpoint{0.984126in}{2.376000in}}%
\pgfpathcurveto{\pgfqpoint{0.984126in}{2.364950in}}{\pgfqpoint{0.988517in}{2.354351in}}{\pgfqpoint{0.996330in}{2.346537in}}%
\pgfpathcurveto{\pgfqpoint{1.004144in}{2.338724in}}{\pgfqpoint{1.014743in}{2.334333in}}{\pgfqpoint{1.025793in}{2.334333in}}%
\pgfpathclose%
\pgfusepath{stroke,fill}%
\end{pgfscope}%
\begin{pgfscope}%
\pgfpathrectangle{\pgfqpoint{0.800000in}{0.528000in}}{\pgfqpoint{4.960000in}{3.696000in}}%
\pgfusepath{clip}%
\pgfsetbuttcap%
\pgfsetroundjoin%
\definecolor{currentfill}{rgb}{0.000000,0.000000,0.000000}%
\pgfsetfillcolor{currentfill}%
\pgfsetlinewidth{1.003750pt}%
\definecolor{currentstroke}{rgb}{0.000000,0.000000,0.000000}%
\pgfsetstrokecolor{currentstroke}%
\pgfsetdash{}{0pt}%
\pgfpathmoveto{\pgfqpoint{1.025793in}{2.334333in}}%
\pgfpathcurveto{\pgfqpoint{1.036843in}{2.334333in}}{\pgfqpoint{1.047442in}{2.338724in}}{\pgfqpoint{1.055256in}{2.346537in}}%
\pgfpathcurveto{\pgfqpoint{1.063070in}{2.354351in}}{\pgfqpoint{1.067460in}{2.364950in}}{\pgfqpoint{1.067460in}{2.376000in}}%
\pgfpathcurveto{\pgfqpoint{1.067460in}{2.387050in}}{\pgfqpoint{1.063070in}{2.397649in}}{\pgfqpoint{1.055256in}{2.405463in}}%
\pgfpathcurveto{\pgfqpoint{1.047442in}{2.413276in}}{\pgfqpoint{1.036843in}{2.417667in}}{\pgfqpoint{1.025793in}{2.417667in}}%
\pgfpathcurveto{\pgfqpoint{1.014743in}{2.417667in}}{\pgfqpoint{1.004144in}{2.413276in}}{\pgfqpoint{0.996330in}{2.405463in}}%
\pgfpathcurveto{\pgfqpoint{0.988517in}{2.397649in}}{\pgfqpoint{0.984126in}{2.387050in}}{\pgfqpoint{0.984126in}{2.376000in}}%
\pgfpathcurveto{\pgfqpoint{0.984126in}{2.364950in}}{\pgfqpoint{0.988517in}{2.354351in}}{\pgfqpoint{0.996330in}{2.346537in}}%
\pgfpathcurveto{\pgfqpoint{1.004144in}{2.338724in}}{\pgfqpoint{1.014743in}{2.334333in}}{\pgfqpoint{1.025793in}{2.334333in}}%
\pgfpathclose%
\pgfusepath{stroke,fill}%
\end{pgfscope}%
\begin{pgfscope}%
\pgfpathrectangle{\pgfqpoint{0.800000in}{0.528000in}}{\pgfqpoint{4.960000in}{3.696000in}}%
\pgfusepath{clip}%
\pgfsetbuttcap%
\pgfsetroundjoin%
\definecolor{currentfill}{rgb}{0.000000,0.000000,0.000000}%
\pgfsetfillcolor{currentfill}%
\pgfsetlinewidth{1.003750pt}%
\definecolor{currentstroke}{rgb}{0.000000,0.000000,0.000000}%
\pgfsetstrokecolor{currentstroke}%
\pgfsetdash{}{0pt}%
\pgfpathmoveto{\pgfqpoint{1.025793in}{2.334333in}}%
\pgfpathcurveto{\pgfqpoint{1.036843in}{2.334333in}}{\pgfqpoint{1.047442in}{2.338724in}}{\pgfqpoint{1.055256in}{2.346537in}}%
\pgfpathcurveto{\pgfqpoint{1.063070in}{2.354351in}}{\pgfqpoint{1.067460in}{2.364950in}}{\pgfqpoint{1.067460in}{2.376000in}}%
\pgfpathcurveto{\pgfqpoint{1.067460in}{2.387050in}}{\pgfqpoint{1.063070in}{2.397649in}}{\pgfqpoint{1.055256in}{2.405463in}}%
\pgfpathcurveto{\pgfqpoint{1.047442in}{2.413276in}}{\pgfqpoint{1.036843in}{2.417667in}}{\pgfqpoint{1.025793in}{2.417667in}}%
\pgfpathcurveto{\pgfqpoint{1.014743in}{2.417667in}}{\pgfqpoint{1.004144in}{2.413276in}}{\pgfqpoint{0.996330in}{2.405463in}}%
\pgfpathcurveto{\pgfqpoint{0.988517in}{2.397649in}}{\pgfqpoint{0.984126in}{2.387050in}}{\pgfqpoint{0.984126in}{2.376000in}}%
\pgfpathcurveto{\pgfqpoint{0.984126in}{2.364950in}}{\pgfqpoint{0.988517in}{2.354351in}}{\pgfqpoint{0.996330in}{2.346537in}}%
\pgfpathcurveto{\pgfqpoint{1.004144in}{2.338724in}}{\pgfqpoint{1.014743in}{2.334333in}}{\pgfqpoint{1.025793in}{2.334333in}}%
\pgfpathclose%
\pgfusepath{stroke,fill}%
\end{pgfscope}%
\begin{pgfscope}%
\pgfpathrectangle{\pgfqpoint{0.800000in}{0.528000in}}{\pgfqpoint{4.960000in}{3.696000in}}%
\pgfusepath{clip}%
\pgfsetbuttcap%
\pgfsetroundjoin%
\definecolor{currentfill}{rgb}{0.000000,0.000000,0.000000}%
\pgfsetfillcolor{currentfill}%
\pgfsetlinewidth{1.003750pt}%
\definecolor{currentstroke}{rgb}{0.000000,0.000000,0.000000}%
\pgfsetstrokecolor{currentstroke}%
\pgfsetdash{}{0pt}%
\pgfpathmoveto{\pgfqpoint{1.025793in}{2.334333in}}%
\pgfpathcurveto{\pgfqpoint{1.036843in}{2.334333in}}{\pgfqpoint{1.047442in}{2.338724in}}{\pgfqpoint{1.055256in}{2.346537in}}%
\pgfpathcurveto{\pgfqpoint{1.063070in}{2.354351in}}{\pgfqpoint{1.067460in}{2.364950in}}{\pgfqpoint{1.067460in}{2.376000in}}%
\pgfpathcurveto{\pgfqpoint{1.067460in}{2.387050in}}{\pgfqpoint{1.063070in}{2.397649in}}{\pgfqpoint{1.055256in}{2.405463in}}%
\pgfpathcurveto{\pgfqpoint{1.047442in}{2.413276in}}{\pgfqpoint{1.036843in}{2.417667in}}{\pgfqpoint{1.025793in}{2.417667in}}%
\pgfpathcurveto{\pgfqpoint{1.014743in}{2.417667in}}{\pgfqpoint{1.004144in}{2.413276in}}{\pgfqpoint{0.996330in}{2.405463in}}%
\pgfpathcurveto{\pgfqpoint{0.988517in}{2.397649in}}{\pgfqpoint{0.984126in}{2.387050in}}{\pgfqpoint{0.984126in}{2.376000in}}%
\pgfpathcurveto{\pgfqpoint{0.984126in}{2.364950in}}{\pgfqpoint{0.988517in}{2.354351in}}{\pgfqpoint{0.996330in}{2.346537in}}%
\pgfpathcurveto{\pgfqpoint{1.004144in}{2.338724in}}{\pgfqpoint{1.014743in}{2.334333in}}{\pgfqpoint{1.025793in}{2.334333in}}%
\pgfpathclose%
\pgfusepath{stroke,fill}%
\end{pgfscope}%
\begin{pgfscope}%
\pgfpathrectangle{\pgfqpoint{0.800000in}{0.528000in}}{\pgfqpoint{4.960000in}{3.696000in}}%
\pgfusepath{clip}%
\pgfsetbuttcap%
\pgfsetroundjoin%
\definecolor{currentfill}{rgb}{0.000000,0.000000,0.000000}%
\pgfsetfillcolor{currentfill}%
\pgfsetlinewidth{1.003750pt}%
\definecolor{currentstroke}{rgb}{0.000000,0.000000,0.000000}%
\pgfsetstrokecolor{currentstroke}%
\pgfsetdash{}{0pt}%
\pgfpathmoveto{\pgfqpoint{1.025793in}{2.334333in}}%
\pgfpathcurveto{\pgfqpoint{1.036843in}{2.334333in}}{\pgfqpoint{1.047442in}{2.338724in}}{\pgfqpoint{1.055256in}{2.346537in}}%
\pgfpathcurveto{\pgfqpoint{1.063070in}{2.354351in}}{\pgfqpoint{1.067460in}{2.364950in}}{\pgfqpoint{1.067460in}{2.376000in}}%
\pgfpathcurveto{\pgfqpoint{1.067460in}{2.387050in}}{\pgfqpoint{1.063070in}{2.397649in}}{\pgfqpoint{1.055256in}{2.405463in}}%
\pgfpathcurveto{\pgfqpoint{1.047442in}{2.413276in}}{\pgfqpoint{1.036843in}{2.417667in}}{\pgfqpoint{1.025793in}{2.417667in}}%
\pgfpathcurveto{\pgfqpoint{1.014743in}{2.417667in}}{\pgfqpoint{1.004144in}{2.413276in}}{\pgfqpoint{0.996330in}{2.405463in}}%
\pgfpathcurveto{\pgfqpoint{0.988517in}{2.397649in}}{\pgfqpoint{0.984126in}{2.387050in}}{\pgfqpoint{0.984126in}{2.376000in}}%
\pgfpathcurveto{\pgfqpoint{0.984126in}{2.364950in}}{\pgfqpoint{0.988517in}{2.354351in}}{\pgfqpoint{0.996330in}{2.346537in}}%
\pgfpathcurveto{\pgfqpoint{1.004144in}{2.338724in}}{\pgfqpoint{1.014743in}{2.334333in}}{\pgfqpoint{1.025793in}{2.334333in}}%
\pgfpathclose%
\pgfusepath{stroke,fill}%
\end{pgfscope}%
\begin{pgfscope}%
\pgfpathrectangle{\pgfqpoint{0.800000in}{0.528000in}}{\pgfqpoint{4.960000in}{3.696000in}}%
\pgfusepath{clip}%
\pgfsetbuttcap%
\pgfsetroundjoin%
\definecolor{currentfill}{rgb}{0.000000,0.000000,0.000000}%
\pgfsetfillcolor{currentfill}%
\pgfsetlinewidth{1.003750pt}%
\definecolor{currentstroke}{rgb}{0.000000,0.000000,0.000000}%
\pgfsetstrokecolor{currentstroke}%
\pgfsetdash{}{0pt}%
\pgfpathmoveto{\pgfqpoint{1.025793in}{2.334333in}}%
\pgfpathcurveto{\pgfqpoint{1.036843in}{2.334333in}}{\pgfqpoint{1.047442in}{2.338724in}}{\pgfqpoint{1.055256in}{2.346537in}}%
\pgfpathcurveto{\pgfqpoint{1.063070in}{2.354351in}}{\pgfqpoint{1.067460in}{2.364950in}}{\pgfqpoint{1.067460in}{2.376000in}}%
\pgfpathcurveto{\pgfqpoint{1.067460in}{2.387050in}}{\pgfqpoint{1.063070in}{2.397649in}}{\pgfqpoint{1.055256in}{2.405463in}}%
\pgfpathcurveto{\pgfqpoint{1.047442in}{2.413276in}}{\pgfqpoint{1.036843in}{2.417667in}}{\pgfqpoint{1.025793in}{2.417667in}}%
\pgfpathcurveto{\pgfqpoint{1.014743in}{2.417667in}}{\pgfqpoint{1.004144in}{2.413276in}}{\pgfqpoint{0.996330in}{2.405463in}}%
\pgfpathcurveto{\pgfqpoint{0.988517in}{2.397649in}}{\pgfqpoint{0.984126in}{2.387050in}}{\pgfqpoint{0.984126in}{2.376000in}}%
\pgfpathcurveto{\pgfqpoint{0.984126in}{2.364950in}}{\pgfqpoint{0.988517in}{2.354351in}}{\pgfqpoint{0.996330in}{2.346537in}}%
\pgfpathcurveto{\pgfqpoint{1.004144in}{2.338724in}}{\pgfqpoint{1.014743in}{2.334333in}}{\pgfqpoint{1.025793in}{2.334333in}}%
\pgfpathclose%
\pgfusepath{stroke,fill}%
\end{pgfscope}%
\begin{pgfscope}%
\pgfpathrectangle{\pgfqpoint{0.800000in}{0.528000in}}{\pgfqpoint{4.960000in}{3.696000in}}%
\pgfusepath{clip}%
\pgfsetbuttcap%
\pgfsetroundjoin%
\definecolor{currentfill}{rgb}{0.000000,0.000000,0.000000}%
\pgfsetfillcolor{currentfill}%
\pgfsetlinewidth{1.003750pt}%
\definecolor{currentstroke}{rgb}{0.000000,0.000000,0.000000}%
\pgfsetstrokecolor{currentstroke}%
\pgfsetdash{}{0pt}%
\pgfpathmoveto{\pgfqpoint{1.025793in}{2.334333in}}%
\pgfpathcurveto{\pgfqpoint{1.036843in}{2.334333in}}{\pgfqpoint{1.047442in}{2.338724in}}{\pgfqpoint{1.055256in}{2.346537in}}%
\pgfpathcurveto{\pgfqpoint{1.063070in}{2.354351in}}{\pgfqpoint{1.067460in}{2.364950in}}{\pgfqpoint{1.067460in}{2.376000in}}%
\pgfpathcurveto{\pgfqpoint{1.067460in}{2.387050in}}{\pgfqpoint{1.063070in}{2.397649in}}{\pgfqpoint{1.055256in}{2.405463in}}%
\pgfpathcurveto{\pgfqpoint{1.047442in}{2.413276in}}{\pgfqpoint{1.036843in}{2.417667in}}{\pgfqpoint{1.025793in}{2.417667in}}%
\pgfpathcurveto{\pgfqpoint{1.014743in}{2.417667in}}{\pgfqpoint{1.004144in}{2.413276in}}{\pgfqpoint{0.996330in}{2.405463in}}%
\pgfpathcurveto{\pgfqpoint{0.988517in}{2.397649in}}{\pgfqpoint{0.984126in}{2.387050in}}{\pgfqpoint{0.984126in}{2.376000in}}%
\pgfpathcurveto{\pgfqpoint{0.984126in}{2.364950in}}{\pgfqpoint{0.988517in}{2.354351in}}{\pgfqpoint{0.996330in}{2.346537in}}%
\pgfpathcurveto{\pgfqpoint{1.004144in}{2.338724in}}{\pgfqpoint{1.014743in}{2.334333in}}{\pgfqpoint{1.025793in}{2.334333in}}%
\pgfpathclose%
\pgfusepath{stroke,fill}%
\end{pgfscope}%
\begin{pgfscope}%
\pgfpathrectangle{\pgfqpoint{0.800000in}{0.528000in}}{\pgfqpoint{4.960000in}{3.696000in}}%
\pgfusepath{clip}%
\pgfsetbuttcap%
\pgfsetroundjoin%
\definecolor{currentfill}{rgb}{0.000000,0.000000,0.000000}%
\pgfsetfillcolor{currentfill}%
\pgfsetlinewidth{1.003750pt}%
\definecolor{currentstroke}{rgb}{0.000000,0.000000,0.000000}%
\pgfsetstrokecolor{currentstroke}%
\pgfsetdash{}{0pt}%
\pgfpathmoveto{\pgfqpoint{1.025793in}{2.334333in}}%
\pgfpathcurveto{\pgfqpoint{1.036843in}{2.334333in}}{\pgfqpoint{1.047442in}{2.338724in}}{\pgfqpoint{1.055256in}{2.346537in}}%
\pgfpathcurveto{\pgfqpoint{1.063070in}{2.354351in}}{\pgfqpoint{1.067460in}{2.364950in}}{\pgfqpoint{1.067460in}{2.376000in}}%
\pgfpathcurveto{\pgfqpoint{1.067460in}{2.387050in}}{\pgfqpoint{1.063070in}{2.397649in}}{\pgfqpoint{1.055256in}{2.405463in}}%
\pgfpathcurveto{\pgfqpoint{1.047442in}{2.413276in}}{\pgfqpoint{1.036843in}{2.417667in}}{\pgfqpoint{1.025793in}{2.417667in}}%
\pgfpathcurveto{\pgfqpoint{1.014743in}{2.417667in}}{\pgfqpoint{1.004144in}{2.413276in}}{\pgfqpoint{0.996330in}{2.405463in}}%
\pgfpathcurveto{\pgfqpoint{0.988517in}{2.397649in}}{\pgfqpoint{0.984126in}{2.387050in}}{\pgfqpoint{0.984126in}{2.376000in}}%
\pgfpathcurveto{\pgfqpoint{0.984126in}{2.364950in}}{\pgfqpoint{0.988517in}{2.354351in}}{\pgfqpoint{0.996330in}{2.346537in}}%
\pgfpathcurveto{\pgfqpoint{1.004144in}{2.338724in}}{\pgfqpoint{1.014743in}{2.334333in}}{\pgfqpoint{1.025793in}{2.334333in}}%
\pgfpathclose%
\pgfusepath{stroke,fill}%
\end{pgfscope}%
\begin{pgfscope}%
\pgfpathrectangle{\pgfqpoint{0.800000in}{0.528000in}}{\pgfqpoint{4.960000in}{3.696000in}}%
\pgfusepath{clip}%
\pgfsetbuttcap%
\pgfsetroundjoin%
\definecolor{currentfill}{rgb}{0.000000,0.000000,0.000000}%
\pgfsetfillcolor{currentfill}%
\pgfsetlinewidth{1.003750pt}%
\definecolor{currentstroke}{rgb}{0.000000,0.000000,0.000000}%
\pgfsetstrokecolor{currentstroke}%
\pgfsetdash{}{0pt}%
\pgfpathmoveto{\pgfqpoint{1.025793in}{2.334333in}}%
\pgfpathcurveto{\pgfqpoint{1.036843in}{2.334333in}}{\pgfqpoint{1.047442in}{2.338724in}}{\pgfqpoint{1.055256in}{2.346537in}}%
\pgfpathcurveto{\pgfqpoint{1.063070in}{2.354351in}}{\pgfqpoint{1.067460in}{2.364950in}}{\pgfqpoint{1.067460in}{2.376000in}}%
\pgfpathcurveto{\pgfqpoint{1.067460in}{2.387050in}}{\pgfqpoint{1.063070in}{2.397649in}}{\pgfqpoint{1.055256in}{2.405463in}}%
\pgfpathcurveto{\pgfqpoint{1.047442in}{2.413276in}}{\pgfqpoint{1.036843in}{2.417667in}}{\pgfqpoint{1.025793in}{2.417667in}}%
\pgfpathcurveto{\pgfqpoint{1.014743in}{2.417667in}}{\pgfqpoint{1.004144in}{2.413276in}}{\pgfqpoint{0.996330in}{2.405463in}}%
\pgfpathcurveto{\pgfqpoint{0.988517in}{2.397649in}}{\pgfqpoint{0.984126in}{2.387050in}}{\pgfqpoint{0.984126in}{2.376000in}}%
\pgfpathcurveto{\pgfqpoint{0.984126in}{2.364950in}}{\pgfqpoint{0.988517in}{2.354351in}}{\pgfqpoint{0.996330in}{2.346537in}}%
\pgfpathcurveto{\pgfqpoint{1.004144in}{2.338724in}}{\pgfqpoint{1.014743in}{2.334333in}}{\pgfqpoint{1.025793in}{2.334333in}}%
\pgfpathclose%
\pgfusepath{stroke,fill}%
\end{pgfscope}%
\begin{pgfscope}%
\pgfpathrectangle{\pgfqpoint{0.800000in}{0.528000in}}{\pgfqpoint{4.960000in}{3.696000in}}%
\pgfusepath{clip}%
\pgfsetbuttcap%
\pgfsetroundjoin%
\definecolor{currentfill}{rgb}{0.000000,0.000000,0.000000}%
\pgfsetfillcolor{currentfill}%
\pgfsetlinewidth{1.003750pt}%
\definecolor{currentstroke}{rgb}{0.000000,0.000000,0.000000}%
\pgfsetstrokecolor{currentstroke}%
\pgfsetdash{}{0pt}%
\pgfpathmoveto{\pgfqpoint{1.025793in}{2.334333in}}%
\pgfpathcurveto{\pgfqpoint{1.036843in}{2.334333in}}{\pgfqpoint{1.047442in}{2.338724in}}{\pgfqpoint{1.055256in}{2.346537in}}%
\pgfpathcurveto{\pgfqpoint{1.063070in}{2.354351in}}{\pgfqpoint{1.067460in}{2.364950in}}{\pgfqpoint{1.067460in}{2.376000in}}%
\pgfpathcurveto{\pgfqpoint{1.067460in}{2.387050in}}{\pgfqpoint{1.063070in}{2.397649in}}{\pgfqpoint{1.055256in}{2.405463in}}%
\pgfpathcurveto{\pgfqpoint{1.047442in}{2.413276in}}{\pgfqpoint{1.036843in}{2.417667in}}{\pgfqpoint{1.025793in}{2.417667in}}%
\pgfpathcurveto{\pgfqpoint{1.014743in}{2.417667in}}{\pgfqpoint{1.004144in}{2.413276in}}{\pgfqpoint{0.996330in}{2.405463in}}%
\pgfpathcurveto{\pgfqpoint{0.988517in}{2.397649in}}{\pgfqpoint{0.984126in}{2.387050in}}{\pgfqpoint{0.984126in}{2.376000in}}%
\pgfpathcurveto{\pgfqpoint{0.984126in}{2.364950in}}{\pgfqpoint{0.988517in}{2.354351in}}{\pgfqpoint{0.996330in}{2.346537in}}%
\pgfpathcurveto{\pgfqpoint{1.004144in}{2.338724in}}{\pgfqpoint{1.014743in}{2.334333in}}{\pgfqpoint{1.025793in}{2.334333in}}%
\pgfpathclose%
\pgfusepath{stroke,fill}%
\end{pgfscope}%
\begin{pgfscope}%
\pgfpathrectangle{\pgfqpoint{0.800000in}{0.528000in}}{\pgfqpoint{4.960000in}{3.696000in}}%
\pgfusepath{clip}%
\pgfsetbuttcap%
\pgfsetroundjoin%
\definecolor{currentfill}{rgb}{0.000000,0.000000,0.000000}%
\pgfsetfillcolor{currentfill}%
\pgfsetlinewidth{1.003750pt}%
\definecolor{currentstroke}{rgb}{0.000000,0.000000,0.000000}%
\pgfsetstrokecolor{currentstroke}%
\pgfsetdash{}{0pt}%
\pgfpathmoveto{\pgfqpoint{1.025793in}{2.334333in}}%
\pgfpathcurveto{\pgfqpoint{1.036843in}{2.334333in}}{\pgfqpoint{1.047442in}{2.338724in}}{\pgfqpoint{1.055256in}{2.346537in}}%
\pgfpathcurveto{\pgfqpoint{1.063070in}{2.354351in}}{\pgfqpoint{1.067460in}{2.364950in}}{\pgfqpoint{1.067460in}{2.376000in}}%
\pgfpathcurveto{\pgfqpoint{1.067460in}{2.387050in}}{\pgfqpoint{1.063070in}{2.397649in}}{\pgfqpoint{1.055256in}{2.405463in}}%
\pgfpathcurveto{\pgfqpoint{1.047442in}{2.413276in}}{\pgfqpoint{1.036843in}{2.417667in}}{\pgfqpoint{1.025793in}{2.417667in}}%
\pgfpathcurveto{\pgfqpoint{1.014743in}{2.417667in}}{\pgfqpoint{1.004144in}{2.413276in}}{\pgfqpoint{0.996330in}{2.405463in}}%
\pgfpathcurveto{\pgfqpoint{0.988517in}{2.397649in}}{\pgfqpoint{0.984126in}{2.387050in}}{\pgfqpoint{0.984126in}{2.376000in}}%
\pgfpathcurveto{\pgfqpoint{0.984126in}{2.364950in}}{\pgfqpoint{0.988517in}{2.354351in}}{\pgfqpoint{0.996330in}{2.346537in}}%
\pgfpathcurveto{\pgfqpoint{1.004144in}{2.338724in}}{\pgfqpoint{1.014743in}{2.334333in}}{\pgfqpoint{1.025793in}{2.334333in}}%
\pgfpathclose%
\pgfusepath{stroke,fill}%
\end{pgfscope}%
\begin{pgfscope}%
\pgfpathrectangle{\pgfqpoint{0.800000in}{0.528000in}}{\pgfqpoint{4.960000in}{3.696000in}}%
\pgfusepath{clip}%
\pgfsetbuttcap%
\pgfsetroundjoin%
\definecolor{currentfill}{rgb}{0.000000,0.000000,0.000000}%
\pgfsetfillcolor{currentfill}%
\pgfsetlinewidth{1.003750pt}%
\definecolor{currentstroke}{rgb}{0.000000,0.000000,0.000000}%
\pgfsetstrokecolor{currentstroke}%
\pgfsetdash{}{0pt}%
\pgfpathmoveto{\pgfqpoint{1.025793in}{2.334333in}}%
\pgfpathcurveto{\pgfqpoint{1.036843in}{2.334333in}}{\pgfqpoint{1.047442in}{2.338724in}}{\pgfqpoint{1.055256in}{2.346537in}}%
\pgfpathcurveto{\pgfqpoint{1.063070in}{2.354351in}}{\pgfqpoint{1.067460in}{2.364950in}}{\pgfqpoint{1.067460in}{2.376000in}}%
\pgfpathcurveto{\pgfqpoint{1.067460in}{2.387050in}}{\pgfqpoint{1.063070in}{2.397649in}}{\pgfqpoint{1.055256in}{2.405463in}}%
\pgfpathcurveto{\pgfqpoint{1.047442in}{2.413276in}}{\pgfqpoint{1.036843in}{2.417667in}}{\pgfqpoint{1.025793in}{2.417667in}}%
\pgfpathcurveto{\pgfqpoint{1.014743in}{2.417667in}}{\pgfqpoint{1.004144in}{2.413276in}}{\pgfqpoint{0.996330in}{2.405463in}}%
\pgfpathcurveto{\pgfqpoint{0.988517in}{2.397649in}}{\pgfqpoint{0.984126in}{2.387050in}}{\pgfqpoint{0.984126in}{2.376000in}}%
\pgfpathcurveto{\pgfqpoint{0.984126in}{2.364950in}}{\pgfqpoint{0.988517in}{2.354351in}}{\pgfqpoint{0.996330in}{2.346537in}}%
\pgfpathcurveto{\pgfqpoint{1.004144in}{2.338724in}}{\pgfqpoint{1.014743in}{2.334333in}}{\pgfqpoint{1.025793in}{2.334333in}}%
\pgfpathclose%
\pgfusepath{stroke,fill}%
\end{pgfscope}%
\begin{pgfscope}%
\pgfpathrectangle{\pgfqpoint{0.800000in}{0.528000in}}{\pgfqpoint{4.960000in}{3.696000in}}%
\pgfusepath{clip}%
\pgfsetbuttcap%
\pgfsetroundjoin%
\definecolor{currentfill}{rgb}{0.000000,0.000000,0.000000}%
\pgfsetfillcolor{currentfill}%
\pgfsetlinewidth{1.003750pt}%
\definecolor{currentstroke}{rgb}{0.000000,0.000000,0.000000}%
\pgfsetstrokecolor{currentstroke}%
\pgfsetdash{}{0pt}%
\pgfpathmoveto{\pgfqpoint{1.025793in}{2.334333in}}%
\pgfpathcurveto{\pgfqpoint{1.036843in}{2.334333in}}{\pgfqpoint{1.047442in}{2.338724in}}{\pgfqpoint{1.055256in}{2.346537in}}%
\pgfpathcurveto{\pgfqpoint{1.063070in}{2.354351in}}{\pgfqpoint{1.067460in}{2.364950in}}{\pgfqpoint{1.067460in}{2.376000in}}%
\pgfpathcurveto{\pgfqpoint{1.067460in}{2.387050in}}{\pgfqpoint{1.063070in}{2.397649in}}{\pgfqpoint{1.055256in}{2.405463in}}%
\pgfpathcurveto{\pgfqpoint{1.047442in}{2.413276in}}{\pgfqpoint{1.036843in}{2.417667in}}{\pgfqpoint{1.025793in}{2.417667in}}%
\pgfpathcurveto{\pgfqpoint{1.014743in}{2.417667in}}{\pgfqpoint{1.004144in}{2.413276in}}{\pgfqpoint{0.996330in}{2.405463in}}%
\pgfpathcurveto{\pgfqpoint{0.988517in}{2.397649in}}{\pgfqpoint{0.984126in}{2.387050in}}{\pgfqpoint{0.984126in}{2.376000in}}%
\pgfpathcurveto{\pgfqpoint{0.984126in}{2.364950in}}{\pgfqpoint{0.988517in}{2.354351in}}{\pgfqpoint{0.996330in}{2.346537in}}%
\pgfpathcurveto{\pgfqpoint{1.004144in}{2.338724in}}{\pgfqpoint{1.014743in}{2.334333in}}{\pgfqpoint{1.025793in}{2.334333in}}%
\pgfpathclose%
\pgfusepath{stroke,fill}%
\end{pgfscope}%
\begin{pgfscope}%
\pgfpathrectangle{\pgfqpoint{0.800000in}{0.528000in}}{\pgfqpoint{4.960000in}{3.696000in}}%
\pgfusepath{clip}%
\pgfsetbuttcap%
\pgfsetroundjoin%
\definecolor{currentfill}{rgb}{0.000000,0.000000,0.000000}%
\pgfsetfillcolor{currentfill}%
\pgfsetlinewidth{1.003750pt}%
\definecolor{currentstroke}{rgb}{0.000000,0.000000,0.000000}%
\pgfsetstrokecolor{currentstroke}%
\pgfsetdash{}{0pt}%
\pgfpathmoveto{\pgfqpoint{1.025793in}{2.334333in}}%
\pgfpathcurveto{\pgfqpoint{1.036843in}{2.334333in}}{\pgfqpoint{1.047442in}{2.338724in}}{\pgfqpoint{1.055256in}{2.346537in}}%
\pgfpathcurveto{\pgfqpoint{1.063070in}{2.354351in}}{\pgfqpoint{1.067460in}{2.364950in}}{\pgfqpoint{1.067460in}{2.376000in}}%
\pgfpathcurveto{\pgfqpoint{1.067460in}{2.387050in}}{\pgfqpoint{1.063070in}{2.397649in}}{\pgfqpoint{1.055256in}{2.405463in}}%
\pgfpathcurveto{\pgfqpoint{1.047442in}{2.413276in}}{\pgfqpoint{1.036843in}{2.417667in}}{\pgfqpoint{1.025793in}{2.417667in}}%
\pgfpathcurveto{\pgfqpoint{1.014743in}{2.417667in}}{\pgfqpoint{1.004144in}{2.413276in}}{\pgfqpoint{0.996330in}{2.405463in}}%
\pgfpathcurveto{\pgfqpoint{0.988517in}{2.397649in}}{\pgfqpoint{0.984126in}{2.387050in}}{\pgfqpoint{0.984126in}{2.376000in}}%
\pgfpathcurveto{\pgfqpoint{0.984126in}{2.364950in}}{\pgfqpoint{0.988517in}{2.354351in}}{\pgfqpoint{0.996330in}{2.346537in}}%
\pgfpathcurveto{\pgfqpoint{1.004144in}{2.338724in}}{\pgfqpoint{1.014743in}{2.334333in}}{\pgfqpoint{1.025793in}{2.334333in}}%
\pgfpathclose%
\pgfusepath{stroke,fill}%
\end{pgfscope}%
\begin{pgfscope}%
\pgfpathrectangle{\pgfqpoint{0.800000in}{0.528000in}}{\pgfqpoint{4.960000in}{3.696000in}}%
\pgfusepath{clip}%
\pgfsetbuttcap%
\pgfsetroundjoin%
\definecolor{currentfill}{rgb}{0.000000,0.000000,0.000000}%
\pgfsetfillcolor{currentfill}%
\pgfsetlinewidth{1.003750pt}%
\definecolor{currentstroke}{rgb}{0.000000,0.000000,0.000000}%
\pgfsetstrokecolor{currentstroke}%
\pgfsetdash{}{0pt}%
\pgfpathmoveto{\pgfqpoint{1.025793in}{2.334333in}}%
\pgfpathcurveto{\pgfqpoint{1.036843in}{2.334333in}}{\pgfqpoint{1.047442in}{2.338724in}}{\pgfqpoint{1.055256in}{2.346537in}}%
\pgfpathcurveto{\pgfqpoint{1.063070in}{2.354351in}}{\pgfqpoint{1.067460in}{2.364950in}}{\pgfqpoint{1.067460in}{2.376000in}}%
\pgfpathcurveto{\pgfqpoint{1.067460in}{2.387050in}}{\pgfqpoint{1.063070in}{2.397649in}}{\pgfqpoint{1.055256in}{2.405463in}}%
\pgfpathcurveto{\pgfqpoint{1.047442in}{2.413276in}}{\pgfqpoint{1.036843in}{2.417667in}}{\pgfqpoint{1.025793in}{2.417667in}}%
\pgfpathcurveto{\pgfqpoint{1.014743in}{2.417667in}}{\pgfqpoint{1.004144in}{2.413276in}}{\pgfqpoint{0.996330in}{2.405463in}}%
\pgfpathcurveto{\pgfqpoint{0.988517in}{2.397649in}}{\pgfqpoint{0.984126in}{2.387050in}}{\pgfqpoint{0.984126in}{2.376000in}}%
\pgfpathcurveto{\pgfqpoint{0.984126in}{2.364950in}}{\pgfqpoint{0.988517in}{2.354351in}}{\pgfqpoint{0.996330in}{2.346537in}}%
\pgfpathcurveto{\pgfqpoint{1.004144in}{2.338724in}}{\pgfqpoint{1.014743in}{2.334333in}}{\pgfqpoint{1.025793in}{2.334333in}}%
\pgfpathclose%
\pgfusepath{stroke,fill}%
\end{pgfscope}%
\begin{pgfscope}%
\pgfpathrectangle{\pgfqpoint{0.800000in}{0.528000in}}{\pgfqpoint{4.960000in}{3.696000in}}%
\pgfusepath{clip}%
\pgfsetbuttcap%
\pgfsetroundjoin%
\definecolor{currentfill}{rgb}{0.000000,0.000000,0.000000}%
\pgfsetfillcolor{currentfill}%
\pgfsetlinewidth{1.003750pt}%
\definecolor{currentstroke}{rgb}{0.000000,0.000000,0.000000}%
\pgfsetstrokecolor{currentstroke}%
\pgfsetdash{}{0pt}%
\pgfpathmoveto{\pgfqpoint{1.025793in}{2.334333in}}%
\pgfpathcurveto{\pgfqpoint{1.036843in}{2.334333in}}{\pgfqpoint{1.047442in}{2.338724in}}{\pgfqpoint{1.055256in}{2.346537in}}%
\pgfpathcurveto{\pgfqpoint{1.063070in}{2.354351in}}{\pgfqpoint{1.067460in}{2.364950in}}{\pgfqpoint{1.067460in}{2.376000in}}%
\pgfpathcurveto{\pgfqpoint{1.067460in}{2.387050in}}{\pgfqpoint{1.063070in}{2.397649in}}{\pgfqpoint{1.055256in}{2.405463in}}%
\pgfpathcurveto{\pgfqpoint{1.047442in}{2.413276in}}{\pgfqpoint{1.036843in}{2.417667in}}{\pgfqpoint{1.025793in}{2.417667in}}%
\pgfpathcurveto{\pgfqpoint{1.014743in}{2.417667in}}{\pgfqpoint{1.004144in}{2.413276in}}{\pgfqpoint{0.996330in}{2.405463in}}%
\pgfpathcurveto{\pgfqpoint{0.988517in}{2.397649in}}{\pgfqpoint{0.984126in}{2.387050in}}{\pgfqpoint{0.984126in}{2.376000in}}%
\pgfpathcurveto{\pgfqpoint{0.984126in}{2.364950in}}{\pgfqpoint{0.988517in}{2.354351in}}{\pgfqpoint{0.996330in}{2.346537in}}%
\pgfpathcurveto{\pgfqpoint{1.004144in}{2.338724in}}{\pgfqpoint{1.014743in}{2.334333in}}{\pgfqpoint{1.025793in}{2.334333in}}%
\pgfpathclose%
\pgfusepath{stroke,fill}%
\end{pgfscope}%
\begin{pgfscope}%
\pgfpathrectangle{\pgfqpoint{0.800000in}{0.528000in}}{\pgfqpoint{4.960000in}{3.696000in}}%
\pgfusepath{clip}%
\pgfsetbuttcap%
\pgfsetroundjoin%
\definecolor{currentfill}{rgb}{0.000000,0.000000,0.000000}%
\pgfsetfillcolor{currentfill}%
\pgfsetlinewidth{1.003750pt}%
\definecolor{currentstroke}{rgb}{0.000000,0.000000,0.000000}%
\pgfsetstrokecolor{currentstroke}%
\pgfsetdash{}{0pt}%
\pgfpathmoveto{\pgfqpoint{1.025793in}{2.334333in}}%
\pgfpathcurveto{\pgfqpoint{1.036843in}{2.334333in}}{\pgfqpoint{1.047442in}{2.338724in}}{\pgfqpoint{1.055256in}{2.346537in}}%
\pgfpathcurveto{\pgfqpoint{1.063070in}{2.354351in}}{\pgfqpoint{1.067460in}{2.364950in}}{\pgfqpoint{1.067460in}{2.376000in}}%
\pgfpathcurveto{\pgfqpoint{1.067460in}{2.387050in}}{\pgfqpoint{1.063070in}{2.397649in}}{\pgfqpoint{1.055256in}{2.405463in}}%
\pgfpathcurveto{\pgfqpoint{1.047442in}{2.413276in}}{\pgfqpoint{1.036843in}{2.417667in}}{\pgfqpoint{1.025793in}{2.417667in}}%
\pgfpathcurveto{\pgfqpoint{1.014743in}{2.417667in}}{\pgfqpoint{1.004144in}{2.413276in}}{\pgfqpoint{0.996330in}{2.405463in}}%
\pgfpathcurveto{\pgfqpoint{0.988517in}{2.397649in}}{\pgfqpoint{0.984126in}{2.387050in}}{\pgfqpoint{0.984126in}{2.376000in}}%
\pgfpathcurveto{\pgfqpoint{0.984126in}{2.364950in}}{\pgfqpoint{0.988517in}{2.354351in}}{\pgfqpoint{0.996330in}{2.346537in}}%
\pgfpathcurveto{\pgfqpoint{1.004144in}{2.338724in}}{\pgfqpoint{1.014743in}{2.334333in}}{\pgfqpoint{1.025793in}{2.334333in}}%
\pgfpathclose%
\pgfusepath{stroke,fill}%
\end{pgfscope}%
\begin{pgfscope}%
\pgfpathrectangle{\pgfqpoint{0.800000in}{0.528000in}}{\pgfqpoint{4.960000in}{3.696000in}}%
\pgfusepath{clip}%
\pgfsetbuttcap%
\pgfsetroundjoin%
\definecolor{currentfill}{rgb}{0.000000,0.000000,0.000000}%
\pgfsetfillcolor{currentfill}%
\pgfsetlinewidth{1.003750pt}%
\definecolor{currentstroke}{rgb}{0.000000,0.000000,0.000000}%
\pgfsetstrokecolor{currentstroke}%
\pgfsetdash{}{0pt}%
\pgfpathmoveto{\pgfqpoint{1.025793in}{2.334333in}}%
\pgfpathcurveto{\pgfqpoint{1.036843in}{2.334333in}}{\pgfqpoint{1.047442in}{2.338724in}}{\pgfqpoint{1.055256in}{2.346537in}}%
\pgfpathcurveto{\pgfqpoint{1.063070in}{2.354351in}}{\pgfqpoint{1.067460in}{2.364950in}}{\pgfqpoint{1.067460in}{2.376000in}}%
\pgfpathcurveto{\pgfqpoint{1.067460in}{2.387050in}}{\pgfqpoint{1.063070in}{2.397649in}}{\pgfqpoint{1.055256in}{2.405463in}}%
\pgfpathcurveto{\pgfqpoint{1.047442in}{2.413276in}}{\pgfqpoint{1.036843in}{2.417667in}}{\pgfqpoint{1.025793in}{2.417667in}}%
\pgfpathcurveto{\pgfqpoint{1.014743in}{2.417667in}}{\pgfqpoint{1.004144in}{2.413276in}}{\pgfqpoint{0.996330in}{2.405463in}}%
\pgfpathcurveto{\pgfqpoint{0.988517in}{2.397649in}}{\pgfqpoint{0.984126in}{2.387050in}}{\pgfqpoint{0.984126in}{2.376000in}}%
\pgfpathcurveto{\pgfqpoint{0.984126in}{2.364950in}}{\pgfqpoint{0.988517in}{2.354351in}}{\pgfqpoint{0.996330in}{2.346537in}}%
\pgfpathcurveto{\pgfqpoint{1.004144in}{2.338724in}}{\pgfqpoint{1.014743in}{2.334333in}}{\pgfqpoint{1.025793in}{2.334333in}}%
\pgfpathclose%
\pgfusepath{stroke,fill}%
\end{pgfscope}%
\begin{pgfscope}%
\pgfpathrectangle{\pgfqpoint{0.800000in}{0.528000in}}{\pgfqpoint{4.960000in}{3.696000in}}%
\pgfusepath{clip}%
\pgfsetbuttcap%
\pgfsetroundjoin%
\definecolor{currentfill}{rgb}{0.000000,0.000000,0.000000}%
\pgfsetfillcolor{currentfill}%
\pgfsetlinewidth{1.003750pt}%
\definecolor{currentstroke}{rgb}{0.000000,0.000000,0.000000}%
\pgfsetstrokecolor{currentstroke}%
\pgfsetdash{}{0pt}%
\pgfpathmoveto{\pgfqpoint{1.025793in}{2.334333in}}%
\pgfpathcurveto{\pgfqpoint{1.036843in}{2.334333in}}{\pgfqpoint{1.047442in}{2.338724in}}{\pgfqpoint{1.055256in}{2.346537in}}%
\pgfpathcurveto{\pgfqpoint{1.063070in}{2.354351in}}{\pgfqpoint{1.067460in}{2.364950in}}{\pgfqpoint{1.067460in}{2.376000in}}%
\pgfpathcurveto{\pgfqpoint{1.067460in}{2.387050in}}{\pgfqpoint{1.063070in}{2.397649in}}{\pgfqpoint{1.055256in}{2.405463in}}%
\pgfpathcurveto{\pgfqpoint{1.047442in}{2.413276in}}{\pgfqpoint{1.036843in}{2.417667in}}{\pgfqpoint{1.025793in}{2.417667in}}%
\pgfpathcurveto{\pgfqpoint{1.014743in}{2.417667in}}{\pgfqpoint{1.004144in}{2.413276in}}{\pgfqpoint{0.996330in}{2.405463in}}%
\pgfpathcurveto{\pgfqpoint{0.988517in}{2.397649in}}{\pgfqpoint{0.984126in}{2.387050in}}{\pgfqpoint{0.984126in}{2.376000in}}%
\pgfpathcurveto{\pgfqpoint{0.984126in}{2.364950in}}{\pgfqpoint{0.988517in}{2.354351in}}{\pgfqpoint{0.996330in}{2.346537in}}%
\pgfpathcurveto{\pgfqpoint{1.004144in}{2.338724in}}{\pgfqpoint{1.014743in}{2.334333in}}{\pgfqpoint{1.025793in}{2.334333in}}%
\pgfpathclose%
\pgfusepath{stroke,fill}%
\end{pgfscope}%
\begin{pgfscope}%
\pgfpathrectangle{\pgfqpoint{0.800000in}{0.528000in}}{\pgfqpoint{4.960000in}{3.696000in}}%
\pgfusepath{clip}%
\pgfsetbuttcap%
\pgfsetroundjoin%
\definecolor{currentfill}{rgb}{0.000000,0.000000,0.000000}%
\pgfsetfillcolor{currentfill}%
\pgfsetlinewidth{1.003750pt}%
\definecolor{currentstroke}{rgb}{0.000000,0.000000,0.000000}%
\pgfsetstrokecolor{currentstroke}%
\pgfsetdash{}{0pt}%
\pgfpathmoveto{\pgfqpoint{1.025793in}{2.334333in}}%
\pgfpathcurveto{\pgfqpoint{1.036843in}{2.334333in}}{\pgfqpoint{1.047442in}{2.338724in}}{\pgfqpoint{1.055256in}{2.346537in}}%
\pgfpathcurveto{\pgfqpoint{1.063070in}{2.354351in}}{\pgfqpoint{1.067460in}{2.364950in}}{\pgfqpoint{1.067460in}{2.376000in}}%
\pgfpathcurveto{\pgfqpoint{1.067460in}{2.387050in}}{\pgfqpoint{1.063070in}{2.397649in}}{\pgfqpoint{1.055256in}{2.405463in}}%
\pgfpathcurveto{\pgfqpoint{1.047442in}{2.413276in}}{\pgfqpoint{1.036843in}{2.417667in}}{\pgfqpoint{1.025793in}{2.417667in}}%
\pgfpathcurveto{\pgfqpoint{1.014743in}{2.417667in}}{\pgfqpoint{1.004144in}{2.413276in}}{\pgfqpoint{0.996330in}{2.405463in}}%
\pgfpathcurveto{\pgfqpoint{0.988517in}{2.397649in}}{\pgfqpoint{0.984126in}{2.387050in}}{\pgfqpoint{0.984126in}{2.376000in}}%
\pgfpathcurveto{\pgfqpoint{0.984126in}{2.364950in}}{\pgfqpoint{0.988517in}{2.354351in}}{\pgfqpoint{0.996330in}{2.346537in}}%
\pgfpathcurveto{\pgfqpoint{1.004144in}{2.338724in}}{\pgfqpoint{1.014743in}{2.334333in}}{\pgfqpoint{1.025793in}{2.334333in}}%
\pgfpathclose%
\pgfusepath{stroke,fill}%
\end{pgfscope}%
\begin{pgfscope}%
\pgfpathrectangle{\pgfqpoint{0.800000in}{0.528000in}}{\pgfqpoint{4.960000in}{3.696000in}}%
\pgfusepath{clip}%
\pgfsetbuttcap%
\pgfsetroundjoin%
\definecolor{currentfill}{rgb}{0.000000,0.000000,0.000000}%
\pgfsetfillcolor{currentfill}%
\pgfsetlinewidth{1.003750pt}%
\definecolor{currentstroke}{rgb}{0.000000,0.000000,0.000000}%
\pgfsetstrokecolor{currentstroke}%
\pgfsetdash{}{0pt}%
\pgfpathmoveto{\pgfqpoint{1.025793in}{2.334333in}}%
\pgfpathcurveto{\pgfqpoint{1.036843in}{2.334333in}}{\pgfqpoint{1.047442in}{2.338724in}}{\pgfqpoint{1.055256in}{2.346537in}}%
\pgfpathcurveto{\pgfqpoint{1.063070in}{2.354351in}}{\pgfqpoint{1.067460in}{2.364950in}}{\pgfqpoint{1.067460in}{2.376000in}}%
\pgfpathcurveto{\pgfqpoint{1.067460in}{2.387050in}}{\pgfqpoint{1.063070in}{2.397649in}}{\pgfqpoint{1.055256in}{2.405463in}}%
\pgfpathcurveto{\pgfqpoint{1.047442in}{2.413276in}}{\pgfqpoint{1.036843in}{2.417667in}}{\pgfqpoint{1.025793in}{2.417667in}}%
\pgfpathcurveto{\pgfqpoint{1.014743in}{2.417667in}}{\pgfqpoint{1.004144in}{2.413276in}}{\pgfqpoint{0.996330in}{2.405463in}}%
\pgfpathcurveto{\pgfqpoint{0.988517in}{2.397649in}}{\pgfqpoint{0.984126in}{2.387050in}}{\pgfqpoint{0.984126in}{2.376000in}}%
\pgfpathcurveto{\pgfqpoint{0.984126in}{2.364950in}}{\pgfqpoint{0.988517in}{2.354351in}}{\pgfqpoint{0.996330in}{2.346537in}}%
\pgfpathcurveto{\pgfqpoint{1.004144in}{2.338724in}}{\pgfqpoint{1.014743in}{2.334333in}}{\pgfqpoint{1.025793in}{2.334333in}}%
\pgfpathclose%
\pgfusepath{stroke,fill}%
\end{pgfscope}%
\begin{pgfscope}%
\pgfpathrectangle{\pgfqpoint{0.800000in}{0.528000in}}{\pgfqpoint{4.960000in}{3.696000in}}%
\pgfusepath{clip}%
\pgfsetbuttcap%
\pgfsetroundjoin%
\definecolor{currentfill}{rgb}{0.000000,0.000000,0.000000}%
\pgfsetfillcolor{currentfill}%
\pgfsetlinewidth{1.003750pt}%
\definecolor{currentstroke}{rgb}{0.000000,0.000000,0.000000}%
\pgfsetstrokecolor{currentstroke}%
\pgfsetdash{}{0pt}%
\pgfpathmoveto{\pgfqpoint{1.025793in}{2.334333in}}%
\pgfpathcurveto{\pgfqpoint{1.036843in}{2.334333in}}{\pgfqpoint{1.047442in}{2.338724in}}{\pgfqpoint{1.055256in}{2.346537in}}%
\pgfpathcurveto{\pgfqpoint{1.063070in}{2.354351in}}{\pgfqpoint{1.067460in}{2.364950in}}{\pgfqpoint{1.067460in}{2.376000in}}%
\pgfpathcurveto{\pgfqpoint{1.067460in}{2.387050in}}{\pgfqpoint{1.063070in}{2.397649in}}{\pgfqpoint{1.055256in}{2.405463in}}%
\pgfpathcurveto{\pgfqpoint{1.047442in}{2.413276in}}{\pgfqpoint{1.036843in}{2.417667in}}{\pgfqpoint{1.025793in}{2.417667in}}%
\pgfpathcurveto{\pgfqpoint{1.014743in}{2.417667in}}{\pgfqpoint{1.004144in}{2.413276in}}{\pgfqpoint{0.996330in}{2.405463in}}%
\pgfpathcurveto{\pgfqpoint{0.988517in}{2.397649in}}{\pgfqpoint{0.984126in}{2.387050in}}{\pgfqpoint{0.984126in}{2.376000in}}%
\pgfpathcurveto{\pgfqpoint{0.984126in}{2.364950in}}{\pgfqpoint{0.988517in}{2.354351in}}{\pgfqpoint{0.996330in}{2.346537in}}%
\pgfpathcurveto{\pgfqpoint{1.004144in}{2.338724in}}{\pgfqpoint{1.014743in}{2.334333in}}{\pgfqpoint{1.025793in}{2.334333in}}%
\pgfpathclose%
\pgfusepath{stroke,fill}%
\end{pgfscope}%
\begin{pgfscope}%
\pgfpathrectangle{\pgfqpoint{0.800000in}{0.528000in}}{\pgfqpoint{4.960000in}{3.696000in}}%
\pgfusepath{clip}%
\pgfsetbuttcap%
\pgfsetroundjoin%
\definecolor{currentfill}{rgb}{0.000000,0.000000,0.000000}%
\pgfsetfillcolor{currentfill}%
\pgfsetlinewidth{1.003750pt}%
\definecolor{currentstroke}{rgb}{0.000000,0.000000,0.000000}%
\pgfsetstrokecolor{currentstroke}%
\pgfsetdash{}{0pt}%
\pgfpathmoveto{\pgfqpoint{1.025793in}{2.334333in}}%
\pgfpathcurveto{\pgfqpoint{1.036843in}{2.334333in}}{\pgfqpoint{1.047442in}{2.338724in}}{\pgfqpoint{1.055256in}{2.346537in}}%
\pgfpathcurveto{\pgfqpoint{1.063070in}{2.354351in}}{\pgfqpoint{1.067460in}{2.364950in}}{\pgfqpoint{1.067460in}{2.376000in}}%
\pgfpathcurveto{\pgfqpoint{1.067460in}{2.387050in}}{\pgfqpoint{1.063070in}{2.397649in}}{\pgfqpoint{1.055256in}{2.405463in}}%
\pgfpathcurveto{\pgfqpoint{1.047442in}{2.413276in}}{\pgfqpoint{1.036843in}{2.417667in}}{\pgfqpoint{1.025793in}{2.417667in}}%
\pgfpathcurveto{\pgfqpoint{1.014743in}{2.417667in}}{\pgfqpoint{1.004144in}{2.413276in}}{\pgfqpoint{0.996330in}{2.405463in}}%
\pgfpathcurveto{\pgfqpoint{0.988517in}{2.397649in}}{\pgfqpoint{0.984126in}{2.387050in}}{\pgfqpoint{0.984126in}{2.376000in}}%
\pgfpathcurveto{\pgfqpoint{0.984126in}{2.364950in}}{\pgfqpoint{0.988517in}{2.354351in}}{\pgfqpoint{0.996330in}{2.346537in}}%
\pgfpathcurveto{\pgfqpoint{1.004144in}{2.338724in}}{\pgfqpoint{1.014743in}{2.334333in}}{\pgfqpoint{1.025793in}{2.334333in}}%
\pgfpathclose%
\pgfusepath{stroke,fill}%
\end{pgfscope}%
\begin{pgfscope}%
\pgfpathrectangle{\pgfqpoint{0.800000in}{0.528000in}}{\pgfqpoint{4.960000in}{3.696000in}}%
\pgfusepath{clip}%
\pgfsetbuttcap%
\pgfsetroundjoin%
\definecolor{currentfill}{rgb}{0.000000,0.000000,0.000000}%
\pgfsetfillcolor{currentfill}%
\pgfsetlinewidth{1.003750pt}%
\definecolor{currentstroke}{rgb}{0.000000,0.000000,0.000000}%
\pgfsetstrokecolor{currentstroke}%
\pgfsetdash{}{0pt}%
\pgfpathmoveto{\pgfqpoint{1.025793in}{2.334333in}}%
\pgfpathcurveto{\pgfqpoint{1.036843in}{2.334333in}}{\pgfqpoint{1.047442in}{2.338724in}}{\pgfqpoint{1.055256in}{2.346537in}}%
\pgfpathcurveto{\pgfqpoint{1.063070in}{2.354351in}}{\pgfqpoint{1.067460in}{2.364950in}}{\pgfqpoint{1.067460in}{2.376000in}}%
\pgfpathcurveto{\pgfqpoint{1.067460in}{2.387050in}}{\pgfqpoint{1.063070in}{2.397649in}}{\pgfqpoint{1.055256in}{2.405463in}}%
\pgfpathcurveto{\pgfqpoint{1.047442in}{2.413276in}}{\pgfqpoint{1.036843in}{2.417667in}}{\pgfqpoint{1.025793in}{2.417667in}}%
\pgfpathcurveto{\pgfqpoint{1.014743in}{2.417667in}}{\pgfqpoint{1.004144in}{2.413276in}}{\pgfqpoint{0.996330in}{2.405463in}}%
\pgfpathcurveto{\pgfqpoint{0.988517in}{2.397649in}}{\pgfqpoint{0.984126in}{2.387050in}}{\pgfqpoint{0.984126in}{2.376000in}}%
\pgfpathcurveto{\pgfqpoint{0.984126in}{2.364950in}}{\pgfqpoint{0.988517in}{2.354351in}}{\pgfqpoint{0.996330in}{2.346537in}}%
\pgfpathcurveto{\pgfqpoint{1.004144in}{2.338724in}}{\pgfqpoint{1.014743in}{2.334333in}}{\pgfqpoint{1.025793in}{2.334333in}}%
\pgfpathclose%
\pgfusepath{stroke,fill}%
\end{pgfscope}%
\begin{pgfscope}%
\pgfpathrectangle{\pgfqpoint{0.800000in}{0.528000in}}{\pgfqpoint{4.960000in}{3.696000in}}%
\pgfusepath{clip}%
\pgfsetbuttcap%
\pgfsetroundjoin%
\definecolor{currentfill}{rgb}{0.000000,0.000000,0.000000}%
\pgfsetfillcolor{currentfill}%
\pgfsetlinewidth{1.003750pt}%
\definecolor{currentstroke}{rgb}{0.000000,0.000000,0.000000}%
\pgfsetstrokecolor{currentstroke}%
\pgfsetdash{}{0pt}%
\pgfpathmoveto{\pgfqpoint{1.025793in}{2.334333in}}%
\pgfpathcurveto{\pgfqpoint{1.036843in}{2.334333in}}{\pgfqpoint{1.047442in}{2.338724in}}{\pgfqpoint{1.055256in}{2.346537in}}%
\pgfpathcurveto{\pgfqpoint{1.063070in}{2.354351in}}{\pgfqpoint{1.067460in}{2.364950in}}{\pgfqpoint{1.067460in}{2.376000in}}%
\pgfpathcurveto{\pgfqpoint{1.067460in}{2.387050in}}{\pgfqpoint{1.063070in}{2.397649in}}{\pgfqpoint{1.055256in}{2.405463in}}%
\pgfpathcurveto{\pgfqpoint{1.047442in}{2.413276in}}{\pgfqpoint{1.036843in}{2.417667in}}{\pgfqpoint{1.025793in}{2.417667in}}%
\pgfpathcurveto{\pgfqpoint{1.014743in}{2.417667in}}{\pgfqpoint{1.004144in}{2.413276in}}{\pgfqpoint{0.996330in}{2.405463in}}%
\pgfpathcurveto{\pgfqpoint{0.988517in}{2.397649in}}{\pgfqpoint{0.984126in}{2.387050in}}{\pgfqpoint{0.984126in}{2.376000in}}%
\pgfpathcurveto{\pgfqpoint{0.984126in}{2.364950in}}{\pgfqpoint{0.988517in}{2.354351in}}{\pgfqpoint{0.996330in}{2.346537in}}%
\pgfpathcurveto{\pgfqpoint{1.004144in}{2.338724in}}{\pgfqpoint{1.014743in}{2.334333in}}{\pgfqpoint{1.025793in}{2.334333in}}%
\pgfpathclose%
\pgfusepath{stroke,fill}%
\end{pgfscope}%
\begin{pgfscope}%
\pgfpathrectangle{\pgfqpoint{0.800000in}{0.528000in}}{\pgfqpoint{4.960000in}{3.696000in}}%
\pgfusepath{clip}%
\pgfsetbuttcap%
\pgfsetroundjoin%
\definecolor{currentfill}{rgb}{0.000000,0.000000,0.000000}%
\pgfsetfillcolor{currentfill}%
\pgfsetlinewidth{1.003750pt}%
\definecolor{currentstroke}{rgb}{0.000000,0.000000,0.000000}%
\pgfsetstrokecolor{currentstroke}%
\pgfsetdash{}{0pt}%
\pgfpathmoveto{\pgfqpoint{1.025793in}{2.334333in}}%
\pgfpathcurveto{\pgfqpoint{1.036843in}{2.334333in}}{\pgfqpoint{1.047442in}{2.338724in}}{\pgfqpoint{1.055256in}{2.346537in}}%
\pgfpathcurveto{\pgfqpoint{1.063070in}{2.354351in}}{\pgfqpoint{1.067460in}{2.364950in}}{\pgfqpoint{1.067460in}{2.376000in}}%
\pgfpathcurveto{\pgfqpoint{1.067460in}{2.387050in}}{\pgfqpoint{1.063070in}{2.397649in}}{\pgfqpoint{1.055256in}{2.405463in}}%
\pgfpathcurveto{\pgfqpoint{1.047442in}{2.413276in}}{\pgfqpoint{1.036843in}{2.417667in}}{\pgfqpoint{1.025793in}{2.417667in}}%
\pgfpathcurveto{\pgfqpoint{1.014743in}{2.417667in}}{\pgfqpoint{1.004144in}{2.413276in}}{\pgfqpoint{0.996330in}{2.405463in}}%
\pgfpathcurveto{\pgfqpoint{0.988517in}{2.397649in}}{\pgfqpoint{0.984126in}{2.387050in}}{\pgfqpoint{0.984126in}{2.376000in}}%
\pgfpathcurveto{\pgfqpoint{0.984126in}{2.364950in}}{\pgfqpoint{0.988517in}{2.354351in}}{\pgfqpoint{0.996330in}{2.346537in}}%
\pgfpathcurveto{\pgfqpoint{1.004144in}{2.338724in}}{\pgfqpoint{1.014743in}{2.334333in}}{\pgfqpoint{1.025793in}{2.334333in}}%
\pgfpathclose%
\pgfusepath{stroke,fill}%
\end{pgfscope}%
\begin{pgfscope}%
\pgfpathrectangle{\pgfqpoint{0.800000in}{0.528000in}}{\pgfqpoint{4.960000in}{3.696000in}}%
\pgfusepath{clip}%
\pgfsetbuttcap%
\pgfsetroundjoin%
\definecolor{currentfill}{rgb}{0.000000,0.000000,0.000000}%
\pgfsetfillcolor{currentfill}%
\pgfsetlinewidth{1.003750pt}%
\definecolor{currentstroke}{rgb}{0.000000,0.000000,0.000000}%
\pgfsetstrokecolor{currentstroke}%
\pgfsetdash{}{0pt}%
\pgfpathmoveto{\pgfqpoint{1.025793in}{2.334333in}}%
\pgfpathcurveto{\pgfqpoint{1.036843in}{2.334333in}}{\pgfqpoint{1.047442in}{2.338724in}}{\pgfqpoint{1.055256in}{2.346537in}}%
\pgfpathcurveto{\pgfqpoint{1.063070in}{2.354351in}}{\pgfqpoint{1.067460in}{2.364950in}}{\pgfqpoint{1.067460in}{2.376000in}}%
\pgfpathcurveto{\pgfqpoint{1.067460in}{2.387050in}}{\pgfqpoint{1.063070in}{2.397649in}}{\pgfqpoint{1.055256in}{2.405463in}}%
\pgfpathcurveto{\pgfqpoint{1.047442in}{2.413276in}}{\pgfqpoint{1.036843in}{2.417667in}}{\pgfqpoint{1.025793in}{2.417667in}}%
\pgfpathcurveto{\pgfqpoint{1.014743in}{2.417667in}}{\pgfqpoint{1.004144in}{2.413276in}}{\pgfqpoint{0.996330in}{2.405463in}}%
\pgfpathcurveto{\pgfqpoint{0.988517in}{2.397649in}}{\pgfqpoint{0.984126in}{2.387050in}}{\pgfqpoint{0.984126in}{2.376000in}}%
\pgfpathcurveto{\pgfqpoint{0.984126in}{2.364950in}}{\pgfqpoint{0.988517in}{2.354351in}}{\pgfqpoint{0.996330in}{2.346537in}}%
\pgfpathcurveto{\pgfqpoint{1.004144in}{2.338724in}}{\pgfqpoint{1.014743in}{2.334333in}}{\pgfqpoint{1.025793in}{2.334333in}}%
\pgfpathclose%
\pgfusepath{stroke,fill}%
\end{pgfscope}%
\begin{pgfscope}%
\pgfpathrectangle{\pgfqpoint{0.800000in}{0.528000in}}{\pgfqpoint{4.960000in}{3.696000in}}%
\pgfusepath{clip}%
\pgfsetbuttcap%
\pgfsetroundjoin%
\definecolor{currentfill}{rgb}{0.000000,0.000000,0.000000}%
\pgfsetfillcolor{currentfill}%
\pgfsetlinewidth{1.003750pt}%
\definecolor{currentstroke}{rgb}{0.000000,0.000000,0.000000}%
\pgfsetstrokecolor{currentstroke}%
\pgfsetdash{}{0pt}%
\pgfpathmoveto{\pgfqpoint{1.025793in}{2.334333in}}%
\pgfpathcurveto{\pgfqpoint{1.036843in}{2.334333in}}{\pgfqpoint{1.047442in}{2.338724in}}{\pgfqpoint{1.055256in}{2.346537in}}%
\pgfpathcurveto{\pgfqpoint{1.063070in}{2.354351in}}{\pgfqpoint{1.067460in}{2.364950in}}{\pgfqpoint{1.067460in}{2.376000in}}%
\pgfpathcurveto{\pgfqpoint{1.067460in}{2.387050in}}{\pgfqpoint{1.063070in}{2.397649in}}{\pgfqpoint{1.055256in}{2.405463in}}%
\pgfpathcurveto{\pgfqpoint{1.047442in}{2.413276in}}{\pgfqpoint{1.036843in}{2.417667in}}{\pgfqpoint{1.025793in}{2.417667in}}%
\pgfpathcurveto{\pgfqpoint{1.014743in}{2.417667in}}{\pgfqpoint{1.004144in}{2.413276in}}{\pgfqpoint{0.996330in}{2.405463in}}%
\pgfpathcurveto{\pgfqpoint{0.988517in}{2.397649in}}{\pgfqpoint{0.984126in}{2.387050in}}{\pgfqpoint{0.984126in}{2.376000in}}%
\pgfpathcurveto{\pgfqpoint{0.984126in}{2.364950in}}{\pgfqpoint{0.988517in}{2.354351in}}{\pgfqpoint{0.996330in}{2.346537in}}%
\pgfpathcurveto{\pgfqpoint{1.004144in}{2.338724in}}{\pgfqpoint{1.014743in}{2.334333in}}{\pgfqpoint{1.025793in}{2.334333in}}%
\pgfpathclose%
\pgfusepath{stroke,fill}%
\end{pgfscope}%
\begin{pgfscope}%
\pgfpathrectangle{\pgfqpoint{0.800000in}{0.528000in}}{\pgfqpoint{4.960000in}{3.696000in}}%
\pgfusepath{clip}%
\pgfsetbuttcap%
\pgfsetroundjoin%
\definecolor{currentfill}{rgb}{0.000000,0.000000,0.000000}%
\pgfsetfillcolor{currentfill}%
\pgfsetlinewidth{1.003750pt}%
\definecolor{currentstroke}{rgb}{0.000000,0.000000,0.000000}%
\pgfsetstrokecolor{currentstroke}%
\pgfsetdash{}{0pt}%
\pgfpathmoveto{\pgfqpoint{1.025793in}{2.334333in}}%
\pgfpathcurveto{\pgfqpoint{1.036843in}{2.334333in}}{\pgfqpoint{1.047442in}{2.338724in}}{\pgfqpoint{1.055256in}{2.346537in}}%
\pgfpathcurveto{\pgfqpoint{1.063070in}{2.354351in}}{\pgfqpoint{1.067460in}{2.364950in}}{\pgfqpoint{1.067460in}{2.376000in}}%
\pgfpathcurveto{\pgfqpoint{1.067460in}{2.387050in}}{\pgfqpoint{1.063070in}{2.397649in}}{\pgfqpoint{1.055256in}{2.405463in}}%
\pgfpathcurveto{\pgfqpoint{1.047442in}{2.413276in}}{\pgfqpoint{1.036843in}{2.417667in}}{\pgfqpoint{1.025793in}{2.417667in}}%
\pgfpathcurveto{\pgfqpoint{1.014743in}{2.417667in}}{\pgfqpoint{1.004144in}{2.413276in}}{\pgfqpoint{0.996330in}{2.405463in}}%
\pgfpathcurveto{\pgfqpoint{0.988517in}{2.397649in}}{\pgfqpoint{0.984126in}{2.387050in}}{\pgfqpoint{0.984126in}{2.376000in}}%
\pgfpathcurveto{\pgfqpoint{0.984126in}{2.364950in}}{\pgfqpoint{0.988517in}{2.354351in}}{\pgfqpoint{0.996330in}{2.346537in}}%
\pgfpathcurveto{\pgfqpoint{1.004144in}{2.338724in}}{\pgfqpoint{1.014743in}{2.334333in}}{\pgfqpoint{1.025793in}{2.334333in}}%
\pgfpathclose%
\pgfusepath{stroke,fill}%
\end{pgfscope}%
\begin{pgfscope}%
\pgfpathrectangle{\pgfqpoint{0.800000in}{0.528000in}}{\pgfqpoint{4.960000in}{3.696000in}}%
\pgfusepath{clip}%
\pgfsetbuttcap%
\pgfsetroundjoin%
\definecolor{currentfill}{rgb}{0.000000,0.000000,0.000000}%
\pgfsetfillcolor{currentfill}%
\pgfsetlinewidth{1.003750pt}%
\definecolor{currentstroke}{rgb}{0.000000,0.000000,0.000000}%
\pgfsetstrokecolor{currentstroke}%
\pgfsetdash{}{0pt}%
\pgfpathmoveto{\pgfqpoint{1.025793in}{2.334333in}}%
\pgfpathcurveto{\pgfqpoint{1.036843in}{2.334333in}}{\pgfqpoint{1.047442in}{2.338724in}}{\pgfqpoint{1.055256in}{2.346537in}}%
\pgfpathcurveto{\pgfqpoint{1.063070in}{2.354351in}}{\pgfqpoint{1.067460in}{2.364950in}}{\pgfqpoint{1.067460in}{2.376000in}}%
\pgfpathcurveto{\pgfqpoint{1.067460in}{2.387050in}}{\pgfqpoint{1.063070in}{2.397649in}}{\pgfqpoint{1.055256in}{2.405463in}}%
\pgfpathcurveto{\pgfqpoint{1.047442in}{2.413276in}}{\pgfqpoint{1.036843in}{2.417667in}}{\pgfqpoint{1.025793in}{2.417667in}}%
\pgfpathcurveto{\pgfqpoint{1.014743in}{2.417667in}}{\pgfqpoint{1.004144in}{2.413276in}}{\pgfqpoint{0.996330in}{2.405463in}}%
\pgfpathcurveto{\pgfqpoint{0.988517in}{2.397649in}}{\pgfqpoint{0.984126in}{2.387050in}}{\pgfqpoint{0.984126in}{2.376000in}}%
\pgfpathcurveto{\pgfqpoint{0.984126in}{2.364950in}}{\pgfqpoint{0.988517in}{2.354351in}}{\pgfqpoint{0.996330in}{2.346537in}}%
\pgfpathcurveto{\pgfqpoint{1.004144in}{2.338724in}}{\pgfqpoint{1.014743in}{2.334333in}}{\pgfqpoint{1.025793in}{2.334333in}}%
\pgfpathclose%
\pgfusepath{stroke,fill}%
\end{pgfscope}%
\begin{pgfscope}%
\pgfpathrectangle{\pgfqpoint{0.800000in}{0.528000in}}{\pgfqpoint{4.960000in}{3.696000in}}%
\pgfusepath{clip}%
\pgfsetbuttcap%
\pgfsetroundjoin%
\definecolor{currentfill}{rgb}{0.000000,0.000000,0.000000}%
\pgfsetfillcolor{currentfill}%
\pgfsetlinewidth{1.003750pt}%
\definecolor{currentstroke}{rgb}{0.000000,0.000000,0.000000}%
\pgfsetstrokecolor{currentstroke}%
\pgfsetdash{}{0pt}%
\pgfpathmoveto{\pgfqpoint{1.025793in}{2.334333in}}%
\pgfpathcurveto{\pgfqpoint{1.036843in}{2.334333in}}{\pgfqpoint{1.047442in}{2.338724in}}{\pgfqpoint{1.055256in}{2.346537in}}%
\pgfpathcurveto{\pgfqpoint{1.063070in}{2.354351in}}{\pgfqpoint{1.067460in}{2.364950in}}{\pgfqpoint{1.067460in}{2.376000in}}%
\pgfpathcurveto{\pgfqpoint{1.067460in}{2.387050in}}{\pgfqpoint{1.063070in}{2.397649in}}{\pgfqpoint{1.055256in}{2.405463in}}%
\pgfpathcurveto{\pgfqpoint{1.047442in}{2.413276in}}{\pgfqpoint{1.036843in}{2.417667in}}{\pgfqpoint{1.025793in}{2.417667in}}%
\pgfpathcurveto{\pgfqpoint{1.014743in}{2.417667in}}{\pgfqpoint{1.004144in}{2.413276in}}{\pgfqpoint{0.996330in}{2.405463in}}%
\pgfpathcurveto{\pgfqpoint{0.988517in}{2.397649in}}{\pgfqpoint{0.984126in}{2.387050in}}{\pgfqpoint{0.984126in}{2.376000in}}%
\pgfpathcurveto{\pgfqpoint{0.984126in}{2.364950in}}{\pgfqpoint{0.988517in}{2.354351in}}{\pgfqpoint{0.996330in}{2.346537in}}%
\pgfpathcurveto{\pgfqpoint{1.004144in}{2.338724in}}{\pgfqpoint{1.014743in}{2.334333in}}{\pgfqpoint{1.025793in}{2.334333in}}%
\pgfpathclose%
\pgfusepath{stroke,fill}%
\end{pgfscope}%
\begin{pgfscope}%
\pgfpathrectangle{\pgfqpoint{0.800000in}{0.528000in}}{\pgfqpoint{4.960000in}{3.696000in}}%
\pgfusepath{clip}%
\pgfsetbuttcap%
\pgfsetroundjoin%
\definecolor{currentfill}{rgb}{0.000000,0.000000,0.000000}%
\pgfsetfillcolor{currentfill}%
\pgfsetlinewidth{1.003750pt}%
\definecolor{currentstroke}{rgb}{0.000000,0.000000,0.000000}%
\pgfsetstrokecolor{currentstroke}%
\pgfsetdash{}{0pt}%
\pgfpathmoveto{\pgfqpoint{1.025793in}{2.334333in}}%
\pgfpathcurveto{\pgfqpoint{1.036843in}{2.334333in}}{\pgfqpoint{1.047442in}{2.338724in}}{\pgfqpoint{1.055256in}{2.346537in}}%
\pgfpathcurveto{\pgfqpoint{1.063070in}{2.354351in}}{\pgfqpoint{1.067460in}{2.364950in}}{\pgfqpoint{1.067460in}{2.376000in}}%
\pgfpathcurveto{\pgfqpoint{1.067460in}{2.387050in}}{\pgfqpoint{1.063070in}{2.397649in}}{\pgfqpoint{1.055256in}{2.405463in}}%
\pgfpathcurveto{\pgfqpoint{1.047442in}{2.413276in}}{\pgfqpoint{1.036843in}{2.417667in}}{\pgfqpoint{1.025793in}{2.417667in}}%
\pgfpathcurveto{\pgfqpoint{1.014743in}{2.417667in}}{\pgfqpoint{1.004144in}{2.413276in}}{\pgfqpoint{0.996330in}{2.405463in}}%
\pgfpathcurveto{\pgfqpoint{0.988517in}{2.397649in}}{\pgfqpoint{0.984126in}{2.387050in}}{\pgfqpoint{0.984126in}{2.376000in}}%
\pgfpathcurveto{\pgfqpoint{0.984126in}{2.364950in}}{\pgfqpoint{0.988517in}{2.354351in}}{\pgfqpoint{0.996330in}{2.346537in}}%
\pgfpathcurveto{\pgfqpoint{1.004144in}{2.338724in}}{\pgfqpoint{1.014743in}{2.334333in}}{\pgfqpoint{1.025793in}{2.334333in}}%
\pgfpathclose%
\pgfusepath{stroke,fill}%
\end{pgfscope}%
\begin{pgfscope}%
\pgfpathrectangle{\pgfqpoint{0.800000in}{0.528000in}}{\pgfqpoint{4.960000in}{3.696000in}}%
\pgfusepath{clip}%
\pgfsetbuttcap%
\pgfsetroundjoin%
\definecolor{currentfill}{rgb}{0.000000,0.000000,0.000000}%
\pgfsetfillcolor{currentfill}%
\pgfsetlinewidth{1.003750pt}%
\definecolor{currentstroke}{rgb}{0.000000,0.000000,0.000000}%
\pgfsetstrokecolor{currentstroke}%
\pgfsetdash{}{0pt}%
\pgfpathmoveto{\pgfqpoint{1.025793in}{2.334333in}}%
\pgfpathcurveto{\pgfqpoint{1.036843in}{2.334333in}}{\pgfqpoint{1.047442in}{2.338724in}}{\pgfqpoint{1.055256in}{2.346537in}}%
\pgfpathcurveto{\pgfqpoint{1.063070in}{2.354351in}}{\pgfqpoint{1.067460in}{2.364950in}}{\pgfqpoint{1.067460in}{2.376000in}}%
\pgfpathcurveto{\pgfqpoint{1.067460in}{2.387050in}}{\pgfqpoint{1.063070in}{2.397649in}}{\pgfqpoint{1.055256in}{2.405463in}}%
\pgfpathcurveto{\pgfqpoint{1.047442in}{2.413276in}}{\pgfqpoint{1.036843in}{2.417667in}}{\pgfqpoint{1.025793in}{2.417667in}}%
\pgfpathcurveto{\pgfqpoint{1.014743in}{2.417667in}}{\pgfqpoint{1.004144in}{2.413276in}}{\pgfqpoint{0.996330in}{2.405463in}}%
\pgfpathcurveto{\pgfqpoint{0.988517in}{2.397649in}}{\pgfqpoint{0.984126in}{2.387050in}}{\pgfqpoint{0.984126in}{2.376000in}}%
\pgfpathcurveto{\pgfqpoint{0.984126in}{2.364950in}}{\pgfqpoint{0.988517in}{2.354351in}}{\pgfqpoint{0.996330in}{2.346537in}}%
\pgfpathcurveto{\pgfqpoint{1.004144in}{2.338724in}}{\pgfqpoint{1.014743in}{2.334333in}}{\pgfqpoint{1.025793in}{2.334333in}}%
\pgfpathclose%
\pgfusepath{stroke,fill}%
\end{pgfscope}%
\begin{pgfscope}%
\pgfpathrectangle{\pgfqpoint{0.800000in}{0.528000in}}{\pgfqpoint{4.960000in}{3.696000in}}%
\pgfusepath{clip}%
\pgfsetbuttcap%
\pgfsetroundjoin%
\definecolor{currentfill}{rgb}{0.000000,0.000000,0.000000}%
\pgfsetfillcolor{currentfill}%
\pgfsetlinewidth{1.003750pt}%
\definecolor{currentstroke}{rgb}{0.000000,0.000000,0.000000}%
\pgfsetstrokecolor{currentstroke}%
\pgfsetdash{}{0pt}%
\pgfpathmoveto{\pgfqpoint{1.025793in}{2.334333in}}%
\pgfpathcurveto{\pgfqpoint{1.036843in}{2.334333in}}{\pgfqpoint{1.047442in}{2.338724in}}{\pgfqpoint{1.055256in}{2.346537in}}%
\pgfpathcurveto{\pgfqpoint{1.063070in}{2.354351in}}{\pgfqpoint{1.067460in}{2.364950in}}{\pgfqpoint{1.067460in}{2.376000in}}%
\pgfpathcurveto{\pgfqpoint{1.067460in}{2.387050in}}{\pgfqpoint{1.063070in}{2.397649in}}{\pgfqpoint{1.055256in}{2.405463in}}%
\pgfpathcurveto{\pgfqpoint{1.047442in}{2.413276in}}{\pgfqpoint{1.036843in}{2.417667in}}{\pgfqpoint{1.025793in}{2.417667in}}%
\pgfpathcurveto{\pgfqpoint{1.014743in}{2.417667in}}{\pgfqpoint{1.004144in}{2.413276in}}{\pgfqpoint{0.996330in}{2.405463in}}%
\pgfpathcurveto{\pgfqpoint{0.988517in}{2.397649in}}{\pgfqpoint{0.984126in}{2.387050in}}{\pgfqpoint{0.984126in}{2.376000in}}%
\pgfpathcurveto{\pgfqpoint{0.984126in}{2.364950in}}{\pgfqpoint{0.988517in}{2.354351in}}{\pgfqpoint{0.996330in}{2.346537in}}%
\pgfpathcurveto{\pgfqpoint{1.004144in}{2.338724in}}{\pgfqpoint{1.014743in}{2.334333in}}{\pgfqpoint{1.025793in}{2.334333in}}%
\pgfpathclose%
\pgfusepath{stroke,fill}%
\end{pgfscope}%
\begin{pgfscope}%
\pgfpathrectangle{\pgfqpoint{0.800000in}{0.528000in}}{\pgfqpoint{4.960000in}{3.696000in}}%
\pgfusepath{clip}%
\pgfsetbuttcap%
\pgfsetroundjoin%
\definecolor{currentfill}{rgb}{0.000000,0.000000,0.000000}%
\pgfsetfillcolor{currentfill}%
\pgfsetlinewidth{1.003750pt}%
\definecolor{currentstroke}{rgb}{0.000000,0.000000,0.000000}%
\pgfsetstrokecolor{currentstroke}%
\pgfsetdash{}{0pt}%
\pgfpathmoveto{\pgfqpoint{1.025793in}{2.334333in}}%
\pgfpathcurveto{\pgfqpoint{1.036843in}{2.334333in}}{\pgfqpoint{1.047442in}{2.338724in}}{\pgfqpoint{1.055256in}{2.346537in}}%
\pgfpathcurveto{\pgfqpoint{1.063070in}{2.354351in}}{\pgfqpoint{1.067460in}{2.364950in}}{\pgfqpoint{1.067460in}{2.376000in}}%
\pgfpathcurveto{\pgfqpoint{1.067460in}{2.387050in}}{\pgfqpoint{1.063070in}{2.397649in}}{\pgfqpoint{1.055256in}{2.405463in}}%
\pgfpathcurveto{\pgfqpoint{1.047442in}{2.413276in}}{\pgfqpoint{1.036843in}{2.417667in}}{\pgfqpoint{1.025793in}{2.417667in}}%
\pgfpathcurveto{\pgfqpoint{1.014743in}{2.417667in}}{\pgfqpoint{1.004144in}{2.413276in}}{\pgfqpoint{0.996330in}{2.405463in}}%
\pgfpathcurveto{\pgfqpoint{0.988517in}{2.397649in}}{\pgfqpoint{0.984126in}{2.387050in}}{\pgfqpoint{0.984126in}{2.376000in}}%
\pgfpathcurveto{\pgfqpoint{0.984126in}{2.364950in}}{\pgfqpoint{0.988517in}{2.354351in}}{\pgfqpoint{0.996330in}{2.346537in}}%
\pgfpathcurveto{\pgfqpoint{1.004144in}{2.338724in}}{\pgfqpoint{1.014743in}{2.334333in}}{\pgfqpoint{1.025793in}{2.334333in}}%
\pgfpathclose%
\pgfusepath{stroke,fill}%
\end{pgfscope}%
\begin{pgfscope}%
\pgfpathrectangle{\pgfqpoint{0.800000in}{0.528000in}}{\pgfqpoint{4.960000in}{3.696000in}}%
\pgfusepath{clip}%
\pgfsetbuttcap%
\pgfsetroundjoin%
\definecolor{currentfill}{rgb}{0.000000,0.000000,0.000000}%
\pgfsetfillcolor{currentfill}%
\pgfsetlinewidth{1.003750pt}%
\definecolor{currentstroke}{rgb}{0.000000,0.000000,0.000000}%
\pgfsetstrokecolor{currentstroke}%
\pgfsetdash{}{0pt}%
\pgfpathmoveto{\pgfqpoint{1.025793in}{2.334333in}}%
\pgfpathcurveto{\pgfqpoint{1.036843in}{2.334333in}}{\pgfqpoint{1.047442in}{2.338724in}}{\pgfqpoint{1.055256in}{2.346537in}}%
\pgfpathcurveto{\pgfqpoint{1.063070in}{2.354351in}}{\pgfqpoint{1.067460in}{2.364950in}}{\pgfqpoint{1.067460in}{2.376000in}}%
\pgfpathcurveto{\pgfqpoint{1.067460in}{2.387050in}}{\pgfqpoint{1.063070in}{2.397649in}}{\pgfqpoint{1.055256in}{2.405463in}}%
\pgfpathcurveto{\pgfqpoint{1.047442in}{2.413276in}}{\pgfqpoint{1.036843in}{2.417667in}}{\pgfqpoint{1.025793in}{2.417667in}}%
\pgfpathcurveto{\pgfqpoint{1.014743in}{2.417667in}}{\pgfqpoint{1.004144in}{2.413276in}}{\pgfqpoint{0.996330in}{2.405463in}}%
\pgfpathcurveto{\pgfqpoint{0.988517in}{2.397649in}}{\pgfqpoint{0.984126in}{2.387050in}}{\pgfqpoint{0.984126in}{2.376000in}}%
\pgfpathcurveto{\pgfqpoint{0.984126in}{2.364950in}}{\pgfqpoint{0.988517in}{2.354351in}}{\pgfqpoint{0.996330in}{2.346537in}}%
\pgfpathcurveto{\pgfqpoint{1.004144in}{2.338724in}}{\pgfqpoint{1.014743in}{2.334333in}}{\pgfqpoint{1.025793in}{2.334333in}}%
\pgfpathclose%
\pgfusepath{stroke,fill}%
\end{pgfscope}%
\begin{pgfscope}%
\pgfpathrectangle{\pgfqpoint{0.800000in}{0.528000in}}{\pgfqpoint{4.960000in}{3.696000in}}%
\pgfusepath{clip}%
\pgfsetbuttcap%
\pgfsetroundjoin%
\definecolor{currentfill}{rgb}{0.000000,0.000000,0.000000}%
\pgfsetfillcolor{currentfill}%
\pgfsetlinewidth{1.003750pt}%
\definecolor{currentstroke}{rgb}{0.000000,0.000000,0.000000}%
\pgfsetstrokecolor{currentstroke}%
\pgfsetdash{}{0pt}%
\pgfpathmoveto{\pgfqpoint{1.025793in}{2.334333in}}%
\pgfpathcurveto{\pgfqpoint{1.036843in}{2.334333in}}{\pgfqpoint{1.047442in}{2.338724in}}{\pgfqpoint{1.055256in}{2.346537in}}%
\pgfpathcurveto{\pgfqpoint{1.063070in}{2.354351in}}{\pgfqpoint{1.067460in}{2.364950in}}{\pgfqpoint{1.067460in}{2.376000in}}%
\pgfpathcurveto{\pgfqpoint{1.067460in}{2.387050in}}{\pgfqpoint{1.063070in}{2.397649in}}{\pgfqpoint{1.055256in}{2.405463in}}%
\pgfpathcurveto{\pgfqpoint{1.047442in}{2.413276in}}{\pgfqpoint{1.036843in}{2.417667in}}{\pgfqpoint{1.025793in}{2.417667in}}%
\pgfpathcurveto{\pgfqpoint{1.014743in}{2.417667in}}{\pgfqpoint{1.004144in}{2.413276in}}{\pgfqpoint{0.996330in}{2.405463in}}%
\pgfpathcurveto{\pgfqpoint{0.988517in}{2.397649in}}{\pgfqpoint{0.984126in}{2.387050in}}{\pgfqpoint{0.984126in}{2.376000in}}%
\pgfpathcurveto{\pgfqpoint{0.984126in}{2.364950in}}{\pgfqpoint{0.988517in}{2.354351in}}{\pgfqpoint{0.996330in}{2.346537in}}%
\pgfpathcurveto{\pgfqpoint{1.004144in}{2.338724in}}{\pgfqpoint{1.014743in}{2.334333in}}{\pgfqpoint{1.025793in}{2.334333in}}%
\pgfpathclose%
\pgfusepath{stroke,fill}%
\end{pgfscope}%
\begin{pgfscope}%
\pgfpathrectangle{\pgfqpoint{0.800000in}{0.528000in}}{\pgfqpoint{4.960000in}{3.696000in}}%
\pgfusepath{clip}%
\pgfsetbuttcap%
\pgfsetroundjoin%
\definecolor{currentfill}{rgb}{0.000000,0.000000,0.000000}%
\pgfsetfillcolor{currentfill}%
\pgfsetlinewidth{1.003750pt}%
\definecolor{currentstroke}{rgb}{0.000000,0.000000,0.000000}%
\pgfsetstrokecolor{currentstroke}%
\pgfsetdash{}{0pt}%
\pgfpathmoveto{\pgfqpoint{1.025793in}{2.334333in}}%
\pgfpathcurveto{\pgfqpoint{1.036843in}{2.334333in}}{\pgfqpoint{1.047442in}{2.338724in}}{\pgfqpoint{1.055256in}{2.346537in}}%
\pgfpathcurveto{\pgfqpoint{1.063070in}{2.354351in}}{\pgfqpoint{1.067460in}{2.364950in}}{\pgfqpoint{1.067460in}{2.376000in}}%
\pgfpathcurveto{\pgfqpoint{1.067460in}{2.387050in}}{\pgfqpoint{1.063070in}{2.397649in}}{\pgfqpoint{1.055256in}{2.405463in}}%
\pgfpathcurveto{\pgfqpoint{1.047442in}{2.413276in}}{\pgfqpoint{1.036843in}{2.417667in}}{\pgfqpoint{1.025793in}{2.417667in}}%
\pgfpathcurveto{\pgfqpoint{1.014743in}{2.417667in}}{\pgfqpoint{1.004144in}{2.413276in}}{\pgfqpoint{0.996330in}{2.405463in}}%
\pgfpathcurveto{\pgfqpoint{0.988517in}{2.397649in}}{\pgfqpoint{0.984126in}{2.387050in}}{\pgfqpoint{0.984126in}{2.376000in}}%
\pgfpathcurveto{\pgfqpoint{0.984126in}{2.364950in}}{\pgfqpoint{0.988517in}{2.354351in}}{\pgfqpoint{0.996330in}{2.346537in}}%
\pgfpathcurveto{\pgfqpoint{1.004144in}{2.338724in}}{\pgfqpoint{1.014743in}{2.334333in}}{\pgfqpoint{1.025793in}{2.334333in}}%
\pgfpathclose%
\pgfusepath{stroke,fill}%
\end{pgfscope}%
\begin{pgfscope}%
\pgfpathrectangle{\pgfqpoint{0.800000in}{0.528000in}}{\pgfqpoint{4.960000in}{3.696000in}}%
\pgfusepath{clip}%
\pgfsetbuttcap%
\pgfsetroundjoin%
\definecolor{currentfill}{rgb}{0.000000,0.000000,0.000000}%
\pgfsetfillcolor{currentfill}%
\pgfsetlinewidth{1.003750pt}%
\definecolor{currentstroke}{rgb}{0.000000,0.000000,0.000000}%
\pgfsetstrokecolor{currentstroke}%
\pgfsetdash{}{0pt}%
\pgfpathmoveto{\pgfqpoint{1.025793in}{2.334333in}}%
\pgfpathcurveto{\pgfqpoint{1.036843in}{2.334333in}}{\pgfqpoint{1.047442in}{2.338724in}}{\pgfqpoint{1.055256in}{2.346537in}}%
\pgfpathcurveto{\pgfqpoint{1.063070in}{2.354351in}}{\pgfqpoint{1.067460in}{2.364950in}}{\pgfqpoint{1.067460in}{2.376000in}}%
\pgfpathcurveto{\pgfqpoint{1.067460in}{2.387050in}}{\pgfqpoint{1.063070in}{2.397649in}}{\pgfqpoint{1.055256in}{2.405463in}}%
\pgfpathcurveto{\pgfqpoint{1.047442in}{2.413276in}}{\pgfqpoint{1.036843in}{2.417667in}}{\pgfqpoint{1.025793in}{2.417667in}}%
\pgfpathcurveto{\pgfqpoint{1.014743in}{2.417667in}}{\pgfqpoint{1.004144in}{2.413276in}}{\pgfqpoint{0.996330in}{2.405463in}}%
\pgfpathcurveto{\pgfqpoint{0.988517in}{2.397649in}}{\pgfqpoint{0.984126in}{2.387050in}}{\pgfqpoint{0.984126in}{2.376000in}}%
\pgfpathcurveto{\pgfqpoint{0.984126in}{2.364950in}}{\pgfqpoint{0.988517in}{2.354351in}}{\pgfqpoint{0.996330in}{2.346537in}}%
\pgfpathcurveto{\pgfqpoint{1.004144in}{2.338724in}}{\pgfqpoint{1.014743in}{2.334333in}}{\pgfqpoint{1.025793in}{2.334333in}}%
\pgfpathclose%
\pgfusepath{stroke,fill}%
\end{pgfscope}%
\begin{pgfscope}%
\pgfpathrectangle{\pgfqpoint{0.800000in}{0.528000in}}{\pgfqpoint{4.960000in}{3.696000in}}%
\pgfusepath{clip}%
\pgfsetbuttcap%
\pgfsetroundjoin%
\definecolor{currentfill}{rgb}{0.000000,0.000000,0.000000}%
\pgfsetfillcolor{currentfill}%
\pgfsetlinewidth{1.003750pt}%
\definecolor{currentstroke}{rgb}{0.000000,0.000000,0.000000}%
\pgfsetstrokecolor{currentstroke}%
\pgfsetdash{}{0pt}%
\pgfpathmoveto{\pgfqpoint{1.025793in}{2.334333in}}%
\pgfpathcurveto{\pgfqpoint{1.036843in}{2.334333in}}{\pgfqpoint{1.047442in}{2.338724in}}{\pgfqpoint{1.055256in}{2.346537in}}%
\pgfpathcurveto{\pgfqpoint{1.063070in}{2.354351in}}{\pgfqpoint{1.067460in}{2.364950in}}{\pgfqpoint{1.067460in}{2.376000in}}%
\pgfpathcurveto{\pgfqpoint{1.067460in}{2.387050in}}{\pgfqpoint{1.063070in}{2.397649in}}{\pgfqpoint{1.055256in}{2.405463in}}%
\pgfpathcurveto{\pgfqpoint{1.047442in}{2.413276in}}{\pgfqpoint{1.036843in}{2.417667in}}{\pgfqpoint{1.025793in}{2.417667in}}%
\pgfpathcurveto{\pgfqpoint{1.014743in}{2.417667in}}{\pgfqpoint{1.004144in}{2.413276in}}{\pgfqpoint{0.996330in}{2.405463in}}%
\pgfpathcurveto{\pgfqpoint{0.988517in}{2.397649in}}{\pgfqpoint{0.984126in}{2.387050in}}{\pgfqpoint{0.984126in}{2.376000in}}%
\pgfpathcurveto{\pgfqpoint{0.984126in}{2.364950in}}{\pgfqpoint{0.988517in}{2.354351in}}{\pgfqpoint{0.996330in}{2.346537in}}%
\pgfpathcurveto{\pgfqpoint{1.004144in}{2.338724in}}{\pgfqpoint{1.014743in}{2.334333in}}{\pgfqpoint{1.025793in}{2.334333in}}%
\pgfpathclose%
\pgfusepath{stroke,fill}%
\end{pgfscope}%
\begin{pgfscope}%
\pgfpathrectangle{\pgfqpoint{0.800000in}{0.528000in}}{\pgfqpoint{4.960000in}{3.696000in}}%
\pgfusepath{clip}%
\pgfsetbuttcap%
\pgfsetroundjoin%
\definecolor{currentfill}{rgb}{0.000000,0.000000,0.000000}%
\pgfsetfillcolor{currentfill}%
\pgfsetlinewidth{1.003750pt}%
\definecolor{currentstroke}{rgb}{0.000000,0.000000,0.000000}%
\pgfsetstrokecolor{currentstroke}%
\pgfsetdash{}{0pt}%
\pgfpathmoveto{\pgfqpoint{1.025793in}{2.334333in}}%
\pgfpathcurveto{\pgfqpoint{1.036843in}{2.334333in}}{\pgfqpoint{1.047442in}{2.338724in}}{\pgfqpoint{1.055256in}{2.346537in}}%
\pgfpathcurveto{\pgfqpoint{1.063070in}{2.354351in}}{\pgfqpoint{1.067460in}{2.364950in}}{\pgfqpoint{1.067460in}{2.376000in}}%
\pgfpathcurveto{\pgfqpoint{1.067460in}{2.387050in}}{\pgfqpoint{1.063070in}{2.397649in}}{\pgfqpoint{1.055256in}{2.405463in}}%
\pgfpathcurveto{\pgfqpoint{1.047442in}{2.413276in}}{\pgfqpoint{1.036843in}{2.417667in}}{\pgfqpoint{1.025793in}{2.417667in}}%
\pgfpathcurveto{\pgfqpoint{1.014743in}{2.417667in}}{\pgfqpoint{1.004144in}{2.413276in}}{\pgfqpoint{0.996330in}{2.405463in}}%
\pgfpathcurveto{\pgfqpoint{0.988517in}{2.397649in}}{\pgfqpoint{0.984126in}{2.387050in}}{\pgfqpoint{0.984126in}{2.376000in}}%
\pgfpathcurveto{\pgfqpoint{0.984126in}{2.364950in}}{\pgfqpoint{0.988517in}{2.354351in}}{\pgfqpoint{0.996330in}{2.346537in}}%
\pgfpathcurveto{\pgfqpoint{1.004144in}{2.338724in}}{\pgfqpoint{1.014743in}{2.334333in}}{\pgfqpoint{1.025793in}{2.334333in}}%
\pgfpathclose%
\pgfusepath{stroke,fill}%
\end{pgfscope}%
\begin{pgfscope}%
\pgfpathrectangle{\pgfqpoint{0.800000in}{0.528000in}}{\pgfqpoint{4.960000in}{3.696000in}}%
\pgfusepath{clip}%
\pgfsetbuttcap%
\pgfsetroundjoin%
\definecolor{currentfill}{rgb}{0.000000,0.000000,0.000000}%
\pgfsetfillcolor{currentfill}%
\pgfsetlinewidth{1.003750pt}%
\definecolor{currentstroke}{rgb}{0.000000,0.000000,0.000000}%
\pgfsetstrokecolor{currentstroke}%
\pgfsetdash{}{0pt}%
\pgfpathmoveto{\pgfqpoint{1.025793in}{2.334333in}}%
\pgfpathcurveto{\pgfqpoint{1.036843in}{2.334333in}}{\pgfqpoint{1.047442in}{2.338724in}}{\pgfqpoint{1.055256in}{2.346537in}}%
\pgfpathcurveto{\pgfqpoint{1.063070in}{2.354351in}}{\pgfqpoint{1.067460in}{2.364950in}}{\pgfqpoint{1.067460in}{2.376000in}}%
\pgfpathcurveto{\pgfqpoint{1.067460in}{2.387050in}}{\pgfqpoint{1.063070in}{2.397649in}}{\pgfqpoint{1.055256in}{2.405463in}}%
\pgfpathcurveto{\pgfqpoint{1.047442in}{2.413276in}}{\pgfqpoint{1.036843in}{2.417667in}}{\pgfqpoint{1.025793in}{2.417667in}}%
\pgfpathcurveto{\pgfqpoint{1.014743in}{2.417667in}}{\pgfqpoint{1.004144in}{2.413276in}}{\pgfqpoint{0.996330in}{2.405463in}}%
\pgfpathcurveto{\pgfqpoint{0.988517in}{2.397649in}}{\pgfqpoint{0.984126in}{2.387050in}}{\pgfqpoint{0.984126in}{2.376000in}}%
\pgfpathcurveto{\pgfqpoint{0.984126in}{2.364950in}}{\pgfqpoint{0.988517in}{2.354351in}}{\pgfqpoint{0.996330in}{2.346537in}}%
\pgfpathcurveto{\pgfqpoint{1.004144in}{2.338724in}}{\pgfqpoint{1.014743in}{2.334333in}}{\pgfqpoint{1.025793in}{2.334333in}}%
\pgfpathclose%
\pgfusepath{stroke,fill}%
\end{pgfscope}%
\begin{pgfscope}%
\pgfpathrectangle{\pgfqpoint{0.800000in}{0.528000in}}{\pgfqpoint{4.960000in}{3.696000in}}%
\pgfusepath{clip}%
\pgfsetbuttcap%
\pgfsetroundjoin%
\definecolor{currentfill}{rgb}{0.000000,0.000000,0.000000}%
\pgfsetfillcolor{currentfill}%
\pgfsetlinewidth{1.003750pt}%
\definecolor{currentstroke}{rgb}{0.000000,0.000000,0.000000}%
\pgfsetstrokecolor{currentstroke}%
\pgfsetdash{}{0pt}%
\pgfpathmoveto{\pgfqpoint{1.025793in}{2.334333in}}%
\pgfpathcurveto{\pgfqpoint{1.036843in}{2.334333in}}{\pgfqpoint{1.047442in}{2.338724in}}{\pgfqpoint{1.055256in}{2.346537in}}%
\pgfpathcurveto{\pgfqpoint{1.063070in}{2.354351in}}{\pgfqpoint{1.067460in}{2.364950in}}{\pgfqpoint{1.067460in}{2.376000in}}%
\pgfpathcurveto{\pgfqpoint{1.067460in}{2.387050in}}{\pgfqpoint{1.063070in}{2.397649in}}{\pgfqpoint{1.055256in}{2.405463in}}%
\pgfpathcurveto{\pgfqpoint{1.047442in}{2.413276in}}{\pgfqpoint{1.036843in}{2.417667in}}{\pgfqpoint{1.025793in}{2.417667in}}%
\pgfpathcurveto{\pgfqpoint{1.014743in}{2.417667in}}{\pgfqpoint{1.004144in}{2.413276in}}{\pgfqpoint{0.996330in}{2.405463in}}%
\pgfpathcurveto{\pgfqpoint{0.988517in}{2.397649in}}{\pgfqpoint{0.984126in}{2.387050in}}{\pgfqpoint{0.984126in}{2.376000in}}%
\pgfpathcurveto{\pgfqpoint{0.984126in}{2.364950in}}{\pgfqpoint{0.988517in}{2.354351in}}{\pgfqpoint{0.996330in}{2.346537in}}%
\pgfpathcurveto{\pgfqpoint{1.004144in}{2.338724in}}{\pgfqpoint{1.014743in}{2.334333in}}{\pgfqpoint{1.025793in}{2.334333in}}%
\pgfpathclose%
\pgfusepath{stroke,fill}%
\end{pgfscope}%
\begin{pgfscope}%
\pgfpathrectangle{\pgfqpoint{0.800000in}{0.528000in}}{\pgfqpoint{4.960000in}{3.696000in}}%
\pgfusepath{clip}%
\pgfsetbuttcap%
\pgfsetroundjoin%
\definecolor{currentfill}{rgb}{0.000000,0.000000,0.000000}%
\pgfsetfillcolor{currentfill}%
\pgfsetlinewidth{1.003750pt}%
\definecolor{currentstroke}{rgb}{0.000000,0.000000,0.000000}%
\pgfsetstrokecolor{currentstroke}%
\pgfsetdash{}{0pt}%
\pgfpathmoveto{\pgfqpoint{1.025793in}{2.334333in}}%
\pgfpathcurveto{\pgfqpoint{1.036843in}{2.334333in}}{\pgfqpoint{1.047442in}{2.338724in}}{\pgfqpoint{1.055256in}{2.346537in}}%
\pgfpathcurveto{\pgfqpoint{1.063070in}{2.354351in}}{\pgfqpoint{1.067460in}{2.364950in}}{\pgfqpoint{1.067460in}{2.376000in}}%
\pgfpathcurveto{\pgfqpoint{1.067460in}{2.387050in}}{\pgfqpoint{1.063070in}{2.397649in}}{\pgfqpoint{1.055256in}{2.405463in}}%
\pgfpathcurveto{\pgfqpoint{1.047442in}{2.413276in}}{\pgfqpoint{1.036843in}{2.417667in}}{\pgfqpoint{1.025793in}{2.417667in}}%
\pgfpathcurveto{\pgfqpoint{1.014743in}{2.417667in}}{\pgfqpoint{1.004144in}{2.413276in}}{\pgfqpoint{0.996330in}{2.405463in}}%
\pgfpathcurveto{\pgfqpoint{0.988517in}{2.397649in}}{\pgfqpoint{0.984126in}{2.387050in}}{\pgfqpoint{0.984126in}{2.376000in}}%
\pgfpathcurveto{\pgfqpoint{0.984126in}{2.364950in}}{\pgfqpoint{0.988517in}{2.354351in}}{\pgfqpoint{0.996330in}{2.346537in}}%
\pgfpathcurveto{\pgfqpoint{1.004144in}{2.338724in}}{\pgfqpoint{1.014743in}{2.334333in}}{\pgfqpoint{1.025793in}{2.334333in}}%
\pgfpathclose%
\pgfusepath{stroke,fill}%
\end{pgfscope}%
\begin{pgfscope}%
\pgfpathrectangle{\pgfqpoint{0.800000in}{0.528000in}}{\pgfqpoint{4.960000in}{3.696000in}}%
\pgfusepath{clip}%
\pgfsetbuttcap%
\pgfsetroundjoin%
\definecolor{currentfill}{rgb}{0.000000,0.000000,0.000000}%
\pgfsetfillcolor{currentfill}%
\pgfsetlinewidth{1.003750pt}%
\definecolor{currentstroke}{rgb}{0.000000,0.000000,0.000000}%
\pgfsetstrokecolor{currentstroke}%
\pgfsetdash{}{0pt}%
\pgfpathmoveto{\pgfqpoint{1.025793in}{2.334333in}}%
\pgfpathcurveto{\pgfqpoint{1.036843in}{2.334333in}}{\pgfqpoint{1.047442in}{2.338724in}}{\pgfqpoint{1.055256in}{2.346537in}}%
\pgfpathcurveto{\pgfqpoint{1.063070in}{2.354351in}}{\pgfqpoint{1.067460in}{2.364950in}}{\pgfqpoint{1.067460in}{2.376000in}}%
\pgfpathcurveto{\pgfqpoint{1.067460in}{2.387050in}}{\pgfqpoint{1.063070in}{2.397649in}}{\pgfqpoint{1.055256in}{2.405463in}}%
\pgfpathcurveto{\pgfqpoint{1.047442in}{2.413276in}}{\pgfqpoint{1.036843in}{2.417667in}}{\pgfqpoint{1.025793in}{2.417667in}}%
\pgfpathcurveto{\pgfqpoint{1.014743in}{2.417667in}}{\pgfqpoint{1.004144in}{2.413276in}}{\pgfqpoint{0.996330in}{2.405463in}}%
\pgfpathcurveto{\pgfqpoint{0.988517in}{2.397649in}}{\pgfqpoint{0.984126in}{2.387050in}}{\pgfqpoint{0.984126in}{2.376000in}}%
\pgfpathcurveto{\pgfqpoint{0.984126in}{2.364950in}}{\pgfqpoint{0.988517in}{2.354351in}}{\pgfqpoint{0.996330in}{2.346537in}}%
\pgfpathcurveto{\pgfqpoint{1.004144in}{2.338724in}}{\pgfqpoint{1.014743in}{2.334333in}}{\pgfqpoint{1.025793in}{2.334333in}}%
\pgfpathclose%
\pgfusepath{stroke,fill}%
\end{pgfscope}%
\begin{pgfscope}%
\pgfpathrectangle{\pgfqpoint{0.800000in}{0.528000in}}{\pgfqpoint{4.960000in}{3.696000in}}%
\pgfusepath{clip}%
\pgfsetbuttcap%
\pgfsetroundjoin%
\definecolor{currentfill}{rgb}{0.000000,0.000000,0.000000}%
\pgfsetfillcolor{currentfill}%
\pgfsetlinewidth{1.003750pt}%
\definecolor{currentstroke}{rgb}{0.000000,0.000000,0.000000}%
\pgfsetstrokecolor{currentstroke}%
\pgfsetdash{}{0pt}%
\pgfpathmoveto{\pgfqpoint{1.025793in}{2.334333in}}%
\pgfpathcurveto{\pgfqpoint{1.036843in}{2.334333in}}{\pgfqpoint{1.047442in}{2.338724in}}{\pgfqpoint{1.055256in}{2.346537in}}%
\pgfpathcurveto{\pgfqpoint{1.063070in}{2.354351in}}{\pgfqpoint{1.067460in}{2.364950in}}{\pgfqpoint{1.067460in}{2.376000in}}%
\pgfpathcurveto{\pgfqpoint{1.067460in}{2.387050in}}{\pgfqpoint{1.063070in}{2.397649in}}{\pgfqpoint{1.055256in}{2.405463in}}%
\pgfpathcurveto{\pgfqpoint{1.047442in}{2.413276in}}{\pgfqpoint{1.036843in}{2.417667in}}{\pgfqpoint{1.025793in}{2.417667in}}%
\pgfpathcurveto{\pgfqpoint{1.014743in}{2.417667in}}{\pgfqpoint{1.004144in}{2.413276in}}{\pgfqpoint{0.996330in}{2.405463in}}%
\pgfpathcurveto{\pgfqpoint{0.988517in}{2.397649in}}{\pgfqpoint{0.984126in}{2.387050in}}{\pgfqpoint{0.984126in}{2.376000in}}%
\pgfpathcurveto{\pgfqpoint{0.984126in}{2.364950in}}{\pgfqpoint{0.988517in}{2.354351in}}{\pgfqpoint{0.996330in}{2.346537in}}%
\pgfpathcurveto{\pgfqpoint{1.004144in}{2.338724in}}{\pgfqpoint{1.014743in}{2.334333in}}{\pgfqpoint{1.025793in}{2.334333in}}%
\pgfpathclose%
\pgfusepath{stroke,fill}%
\end{pgfscope}%
\begin{pgfscope}%
\pgfpathrectangle{\pgfqpoint{0.800000in}{0.528000in}}{\pgfqpoint{4.960000in}{3.696000in}}%
\pgfusepath{clip}%
\pgfsetbuttcap%
\pgfsetroundjoin%
\definecolor{currentfill}{rgb}{0.000000,0.000000,0.000000}%
\pgfsetfillcolor{currentfill}%
\pgfsetlinewidth{1.003750pt}%
\definecolor{currentstroke}{rgb}{0.000000,0.000000,0.000000}%
\pgfsetstrokecolor{currentstroke}%
\pgfsetdash{}{0pt}%
\pgfpathmoveto{\pgfqpoint{1.025793in}{2.334333in}}%
\pgfpathcurveto{\pgfqpoint{1.036843in}{2.334333in}}{\pgfqpoint{1.047442in}{2.338724in}}{\pgfqpoint{1.055256in}{2.346537in}}%
\pgfpathcurveto{\pgfqpoint{1.063070in}{2.354351in}}{\pgfqpoint{1.067460in}{2.364950in}}{\pgfqpoint{1.067460in}{2.376000in}}%
\pgfpathcurveto{\pgfqpoint{1.067460in}{2.387050in}}{\pgfqpoint{1.063070in}{2.397649in}}{\pgfqpoint{1.055256in}{2.405463in}}%
\pgfpathcurveto{\pgfqpoint{1.047442in}{2.413276in}}{\pgfqpoint{1.036843in}{2.417667in}}{\pgfqpoint{1.025793in}{2.417667in}}%
\pgfpathcurveto{\pgfqpoint{1.014743in}{2.417667in}}{\pgfqpoint{1.004144in}{2.413276in}}{\pgfqpoint{0.996330in}{2.405463in}}%
\pgfpathcurveto{\pgfqpoint{0.988517in}{2.397649in}}{\pgfqpoint{0.984126in}{2.387050in}}{\pgfqpoint{0.984126in}{2.376000in}}%
\pgfpathcurveto{\pgfqpoint{0.984126in}{2.364950in}}{\pgfqpoint{0.988517in}{2.354351in}}{\pgfqpoint{0.996330in}{2.346537in}}%
\pgfpathcurveto{\pgfqpoint{1.004144in}{2.338724in}}{\pgfqpoint{1.014743in}{2.334333in}}{\pgfqpoint{1.025793in}{2.334333in}}%
\pgfpathclose%
\pgfusepath{stroke,fill}%
\end{pgfscope}%
\begin{pgfscope}%
\pgfpathrectangle{\pgfqpoint{0.800000in}{0.528000in}}{\pgfqpoint{4.960000in}{3.696000in}}%
\pgfusepath{clip}%
\pgfsetbuttcap%
\pgfsetroundjoin%
\definecolor{currentfill}{rgb}{0.000000,0.000000,0.000000}%
\pgfsetfillcolor{currentfill}%
\pgfsetlinewidth{1.003750pt}%
\definecolor{currentstroke}{rgb}{0.000000,0.000000,0.000000}%
\pgfsetstrokecolor{currentstroke}%
\pgfsetdash{}{0pt}%
\pgfpathmoveto{\pgfqpoint{1.025793in}{2.334333in}}%
\pgfpathcurveto{\pgfqpoint{1.036843in}{2.334333in}}{\pgfqpoint{1.047442in}{2.338724in}}{\pgfqpoint{1.055256in}{2.346537in}}%
\pgfpathcurveto{\pgfqpoint{1.063070in}{2.354351in}}{\pgfqpoint{1.067460in}{2.364950in}}{\pgfqpoint{1.067460in}{2.376000in}}%
\pgfpathcurveto{\pgfqpoint{1.067460in}{2.387050in}}{\pgfqpoint{1.063070in}{2.397649in}}{\pgfqpoint{1.055256in}{2.405463in}}%
\pgfpathcurveto{\pgfqpoint{1.047442in}{2.413276in}}{\pgfqpoint{1.036843in}{2.417667in}}{\pgfqpoint{1.025793in}{2.417667in}}%
\pgfpathcurveto{\pgfqpoint{1.014743in}{2.417667in}}{\pgfqpoint{1.004144in}{2.413276in}}{\pgfqpoint{0.996330in}{2.405463in}}%
\pgfpathcurveto{\pgfqpoint{0.988517in}{2.397649in}}{\pgfqpoint{0.984126in}{2.387050in}}{\pgfqpoint{0.984126in}{2.376000in}}%
\pgfpathcurveto{\pgfqpoint{0.984126in}{2.364950in}}{\pgfqpoint{0.988517in}{2.354351in}}{\pgfqpoint{0.996330in}{2.346537in}}%
\pgfpathcurveto{\pgfqpoint{1.004144in}{2.338724in}}{\pgfqpoint{1.014743in}{2.334333in}}{\pgfqpoint{1.025793in}{2.334333in}}%
\pgfpathclose%
\pgfusepath{stroke,fill}%
\end{pgfscope}%
\begin{pgfscope}%
\pgfpathrectangle{\pgfqpoint{0.800000in}{0.528000in}}{\pgfqpoint{4.960000in}{3.696000in}}%
\pgfusepath{clip}%
\pgfsetbuttcap%
\pgfsetroundjoin%
\definecolor{currentfill}{rgb}{0.000000,0.000000,0.000000}%
\pgfsetfillcolor{currentfill}%
\pgfsetlinewidth{1.003750pt}%
\definecolor{currentstroke}{rgb}{0.000000,0.000000,0.000000}%
\pgfsetstrokecolor{currentstroke}%
\pgfsetdash{}{0pt}%
\pgfpathmoveto{\pgfqpoint{1.025793in}{2.334333in}}%
\pgfpathcurveto{\pgfqpoint{1.036843in}{2.334333in}}{\pgfqpoint{1.047442in}{2.338724in}}{\pgfqpoint{1.055256in}{2.346537in}}%
\pgfpathcurveto{\pgfqpoint{1.063070in}{2.354351in}}{\pgfqpoint{1.067460in}{2.364950in}}{\pgfqpoint{1.067460in}{2.376000in}}%
\pgfpathcurveto{\pgfqpoint{1.067460in}{2.387050in}}{\pgfqpoint{1.063070in}{2.397649in}}{\pgfqpoint{1.055256in}{2.405463in}}%
\pgfpathcurveto{\pgfqpoint{1.047442in}{2.413276in}}{\pgfqpoint{1.036843in}{2.417667in}}{\pgfqpoint{1.025793in}{2.417667in}}%
\pgfpathcurveto{\pgfqpoint{1.014743in}{2.417667in}}{\pgfqpoint{1.004144in}{2.413276in}}{\pgfqpoint{0.996330in}{2.405463in}}%
\pgfpathcurveto{\pgfqpoint{0.988517in}{2.397649in}}{\pgfqpoint{0.984126in}{2.387050in}}{\pgfqpoint{0.984126in}{2.376000in}}%
\pgfpathcurveto{\pgfqpoint{0.984126in}{2.364950in}}{\pgfqpoint{0.988517in}{2.354351in}}{\pgfqpoint{0.996330in}{2.346537in}}%
\pgfpathcurveto{\pgfqpoint{1.004144in}{2.338724in}}{\pgfqpoint{1.014743in}{2.334333in}}{\pgfqpoint{1.025793in}{2.334333in}}%
\pgfpathclose%
\pgfusepath{stroke,fill}%
\end{pgfscope}%
\begin{pgfscope}%
\pgfpathrectangle{\pgfqpoint{0.800000in}{0.528000in}}{\pgfqpoint{4.960000in}{3.696000in}}%
\pgfusepath{clip}%
\pgfsetbuttcap%
\pgfsetroundjoin%
\definecolor{currentfill}{rgb}{0.000000,0.000000,0.000000}%
\pgfsetfillcolor{currentfill}%
\pgfsetlinewidth{1.003750pt}%
\definecolor{currentstroke}{rgb}{0.000000,0.000000,0.000000}%
\pgfsetstrokecolor{currentstroke}%
\pgfsetdash{}{0pt}%
\pgfpathmoveto{\pgfqpoint{1.025793in}{2.334333in}}%
\pgfpathcurveto{\pgfqpoint{1.036843in}{2.334333in}}{\pgfqpoint{1.047442in}{2.338724in}}{\pgfqpoint{1.055256in}{2.346537in}}%
\pgfpathcurveto{\pgfqpoint{1.063070in}{2.354351in}}{\pgfqpoint{1.067460in}{2.364950in}}{\pgfqpoint{1.067460in}{2.376000in}}%
\pgfpathcurveto{\pgfqpoint{1.067460in}{2.387050in}}{\pgfqpoint{1.063070in}{2.397649in}}{\pgfqpoint{1.055256in}{2.405463in}}%
\pgfpathcurveto{\pgfqpoint{1.047442in}{2.413276in}}{\pgfqpoint{1.036843in}{2.417667in}}{\pgfqpoint{1.025793in}{2.417667in}}%
\pgfpathcurveto{\pgfqpoint{1.014743in}{2.417667in}}{\pgfqpoint{1.004144in}{2.413276in}}{\pgfqpoint{0.996330in}{2.405463in}}%
\pgfpathcurveto{\pgfqpoint{0.988517in}{2.397649in}}{\pgfqpoint{0.984126in}{2.387050in}}{\pgfqpoint{0.984126in}{2.376000in}}%
\pgfpathcurveto{\pgfqpoint{0.984126in}{2.364950in}}{\pgfqpoint{0.988517in}{2.354351in}}{\pgfqpoint{0.996330in}{2.346537in}}%
\pgfpathcurveto{\pgfqpoint{1.004144in}{2.338724in}}{\pgfqpoint{1.014743in}{2.334333in}}{\pgfqpoint{1.025793in}{2.334333in}}%
\pgfpathclose%
\pgfusepath{stroke,fill}%
\end{pgfscope}%
\begin{pgfscope}%
\pgfpathrectangle{\pgfqpoint{0.800000in}{0.528000in}}{\pgfqpoint{4.960000in}{3.696000in}}%
\pgfusepath{clip}%
\pgfsetbuttcap%
\pgfsetroundjoin%
\definecolor{currentfill}{rgb}{0.000000,0.000000,0.000000}%
\pgfsetfillcolor{currentfill}%
\pgfsetlinewidth{1.003750pt}%
\definecolor{currentstroke}{rgb}{0.000000,0.000000,0.000000}%
\pgfsetstrokecolor{currentstroke}%
\pgfsetdash{}{0pt}%
\pgfpathmoveto{\pgfqpoint{1.025793in}{2.334333in}}%
\pgfpathcurveto{\pgfqpoint{1.036843in}{2.334333in}}{\pgfqpoint{1.047442in}{2.338724in}}{\pgfqpoint{1.055256in}{2.346537in}}%
\pgfpathcurveto{\pgfqpoint{1.063070in}{2.354351in}}{\pgfqpoint{1.067460in}{2.364950in}}{\pgfqpoint{1.067460in}{2.376000in}}%
\pgfpathcurveto{\pgfqpoint{1.067460in}{2.387050in}}{\pgfqpoint{1.063070in}{2.397649in}}{\pgfqpoint{1.055256in}{2.405463in}}%
\pgfpathcurveto{\pgfqpoint{1.047442in}{2.413276in}}{\pgfqpoint{1.036843in}{2.417667in}}{\pgfqpoint{1.025793in}{2.417667in}}%
\pgfpathcurveto{\pgfqpoint{1.014743in}{2.417667in}}{\pgfqpoint{1.004144in}{2.413276in}}{\pgfqpoint{0.996330in}{2.405463in}}%
\pgfpathcurveto{\pgfqpoint{0.988517in}{2.397649in}}{\pgfqpoint{0.984126in}{2.387050in}}{\pgfqpoint{0.984126in}{2.376000in}}%
\pgfpathcurveto{\pgfqpoint{0.984126in}{2.364950in}}{\pgfqpoint{0.988517in}{2.354351in}}{\pgfqpoint{0.996330in}{2.346537in}}%
\pgfpathcurveto{\pgfqpoint{1.004144in}{2.338724in}}{\pgfqpoint{1.014743in}{2.334333in}}{\pgfqpoint{1.025793in}{2.334333in}}%
\pgfpathclose%
\pgfusepath{stroke,fill}%
\end{pgfscope}%
\begin{pgfscope}%
\pgfpathrectangle{\pgfqpoint{0.800000in}{0.528000in}}{\pgfqpoint{4.960000in}{3.696000in}}%
\pgfusepath{clip}%
\pgfsetbuttcap%
\pgfsetroundjoin%
\definecolor{currentfill}{rgb}{0.000000,0.000000,0.000000}%
\pgfsetfillcolor{currentfill}%
\pgfsetlinewidth{1.003750pt}%
\definecolor{currentstroke}{rgb}{0.000000,0.000000,0.000000}%
\pgfsetstrokecolor{currentstroke}%
\pgfsetdash{}{0pt}%
\pgfpathmoveto{\pgfqpoint{1.025793in}{2.334333in}}%
\pgfpathcurveto{\pgfqpoint{1.036843in}{2.334333in}}{\pgfqpoint{1.047442in}{2.338724in}}{\pgfqpoint{1.055256in}{2.346537in}}%
\pgfpathcurveto{\pgfqpoint{1.063070in}{2.354351in}}{\pgfqpoint{1.067460in}{2.364950in}}{\pgfqpoint{1.067460in}{2.376000in}}%
\pgfpathcurveto{\pgfqpoint{1.067460in}{2.387050in}}{\pgfqpoint{1.063070in}{2.397649in}}{\pgfqpoint{1.055256in}{2.405463in}}%
\pgfpathcurveto{\pgfqpoint{1.047442in}{2.413276in}}{\pgfqpoint{1.036843in}{2.417667in}}{\pgfqpoint{1.025793in}{2.417667in}}%
\pgfpathcurveto{\pgfqpoint{1.014743in}{2.417667in}}{\pgfqpoint{1.004144in}{2.413276in}}{\pgfqpoint{0.996330in}{2.405463in}}%
\pgfpathcurveto{\pgfqpoint{0.988517in}{2.397649in}}{\pgfqpoint{0.984126in}{2.387050in}}{\pgfqpoint{0.984126in}{2.376000in}}%
\pgfpathcurveto{\pgfqpoint{0.984126in}{2.364950in}}{\pgfqpoint{0.988517in}{2.354351in}}{\pgfqpoint{0.996330in}{2.346537in}}%
\pgfpathcurveto{\pgfqpoint{1.004144in}{2.338724in}}{\pgfqpoint{1.014743in}{2.334333in}}{\pgfqpoint{1.025793in}{2.334333in}}%
\pgfpathclose%
\pgfusepath{stroke,fill}%
\end{pgfscope}%
\begin{pgfscope}%
\pgfpathrectangle{\pgfqpoint{0.800000in}{0.528000in}}{\pgfqpoint{4.960000in}{3.696000in}}%
\pgfusepath{clip}%
\pgfsetbuttcap%
\pgfsetroundjoin%
\definecolor{currentfill}{rgb}{0.000000,0.000000,0.000000}%
\pgfsetfillcolor{currentfill}%
\pgfsetlinewidth{1.003750pt}%
\definecolor{currentstroke}{rgb}{0.000000,0.000000,0.000000}%
\pgfsetstrokecolor{currentstroke}%
\pgfsetdash{}{0pt}%
\pgfpathmoveto{\pgfqpoint{1.025793in}{2.334333in}}%
\pgfpathcurveto{\pgfqpoint{1.036843in}{2.334333in}}{\pgfqpoint{1.047442in}{2.338724in}}{\pgfqpoint{1.055256in}{2.346537in}}%
\pgfpathcurveto{\pgfqpoint{1.063070in}{2.354351in}}{\pgfqpoint{1.067460in}{2.364950in}}{\pgfqpoint{1.067460in}{2.376000in}}%
\pgfpathcurveto{\pgfqpoint{1.067460in}{2.387050in}}{\pgfqpoint{1.063070in}{2.397649in}}{\pgfqpoint{1.055256in}{2.405463in}}%
\pgfpathcurveto{\pgfqpoint{1.047442in}{2.413276in}}{\pgfqpoint{1.036843in}{2.417667in}}{\pgfqpoint{1.025793in}{2.417667in}}%
\pgfpathcurveto{\pgfqpoint{1.014743in}{2.417667in}}{\pgfqpoint{1.004144in}{2.413276in}}{\pgfqpoint{0.996330in}{2.405463in}}%
\pgfpathcurveto{\pgfqpoint{0.988517in}{2.397649in}}{\pgfqpoint{0.984126in}{2.387050in}}{\pgfqpoint{0.984126in}{2.376000in}}%
\pgfpathcurveto{\pgfqpoint{0.984126in}{2.364950in}}{\pgfqpoint{0.988517in}{2.354351in}}{\pgfqpoint{0.996330in}{2.346537in}}%
\pgfpathcurveto{\pgfqpoint{1.004144in}{2.338724in}}{\pgfqpoint{1.014743in}{2.334333in}}{\pgfqpoint{1.025793in}{2.334333in}}%
\pgfpathclose%
\pgfusepath{stroke,fill}%
\end{pgfscope}%
\begin{pgfscope}%
\pgfpathrectangle{\pgfqpoint{0.800000in}{0.528000in}}{\pgfqpoint{4.960000in}{3.696000in}}%
\pgfusepath{clip}%
\pgfsetbuttcap%
\pgfsetroundjoin%
\definecolor{currentfill}{rgb}{0.000000,0.000000,0.000000}%
\pgfsetfillcolor{currentfill}%
\pgfsetlinewidth{1.003750pt}%
\definecolor{currentstroke}{rgb}{0.000000,0.000000,0.000000}%
\pgfsetstrokecolor{currentstroke}%
\pgfsetdash{}{0pt}%
\pgfpathmoveto{\pgfqpoint{1.025793in}{2.334333in}}%
\pgfpathcurveto{\pgfqpoint{1.036843in}{2.334333in}}{\pgfqpoint{1.047442in}{2.338724in}}{\pgfqpoint{1.055256in}{2.346537in}}%
\pgfpathcurveto{\pgfqpoint{1.063070in}{2.354351in}}{\pgfqpoint{1.067460in}{2.364950in}}{\pgfqpoint{1.067460in}{2.376000in}}%
\pgfpathcurveto{\pgfqpoint{1.067460in}{2.387050in}}{\pgfqpoint{1.063070in}{2.397649in}}{\pgfqpoint{1.055256in}{2.405463in}}%
\pgfpathcurveto{\pgfqpoint{1.047442in}{2.413276in}}{\pgfqpoint{1.036843in}{2.417667in}}{\pgfqpoint{1.025793in}{2.417667in}}%
\pgfpathcurveto{\pgfqpoint{1.014743in}{2.417667in}}{\pgfqpoint{1.004144in}{2.413276in}}{\pgfqpoint{0.996330in}{2.405463in}}%
\pgfpathcurveto{\pgfqpoint{0.988517in}{2.397649in}}{\pgfqpoint{0.984126in}{2.387050in}}{\pgfqpoint{0.984126in}{2.376000in}}%
\pgfpathcurveto{\pgfqpoint{0.984126in}{2.364950in}}{\pgfqpoint{0.988517in}{2.354351in}}{\pgfqpoint{0.996330in}{2.346537in}}%
\pgfpathcurveto{\pgfqpoint{1.004144in}{2.338724in}}{\pgfqpoint{1.014743in}{2.334333in}}{\pgfqpoint{1.025793in}{2.334333in}}%
\pgfpathclose%
\pgfusepath{stroke,fill}%
\end{pgfscope}%
\begin{pgfscope}%
\pgfpathrectangle{\pgfqpoint{0.800000in}{0.528000in}}{\pgfqpoint{4.960000in}{3.696000in}}%
\pgfusepath{clip}%
\pgfsetbuttcap%
\pgfsetroundjoin%
\definecolor{currentfill}{rgb}{0.000000,0.000000,0.000000}%
\pgfsetfillcolor{currentfill}%
\pgfsetlinewidth{1.003750pt}%
\definecolor{currentstroke}{rgb}{0.000000,0.000000,0.000000}%
\pgfsetstrokecolor{currentstroke}%
\pgfsetdash{}{0pt}%
\pgfpathmoveto{\pgfqpoint{1.025793in}{2.334333in}}%
\pgfpathcurveto{\pgfqpoint{1.036843in}{2.334333in}}{\pgfqpoint{1.047442in}{2.338724in}}{\pgfqpoint{1.055256in}{2.346537in}}%
\pgfpathcurveto{\pgfqpoint{1.063070in}{2.354351in}}{\pgfqpoint{1.067460in}{2.364950in}}{\pgfqpoint{1.067460in}{2.376000in}}%
\pgfpathcurveto{\pgfqpoint{1.067460in}{2.387050in}}{\pgfqpoint{1.063070in}{2.397649in}}{\pgfqpoint{1.055256in}{2.405463in}}%
\pgfpathcurveto{\pgfqpoint{1.047442in}{2.413276in}}{\pgfqpoint{1.036843in}{2.417667in}}{\pgfqpoint{1.025793in}{2.417667in}}%
\pgfpathcurveto{\pgfqpoint{1.014743in}{2.417667in}}{\pgfqpoint{1.004144in}{2.413276in}}{\pgfqpoint{0.996330in}{2.405463in}}%
\pgfpathcurveto{\pgfqpoint{0.988517in}{2.397649in}}{\pgfqpoint{0.984126in}{2.387050in}}{\pgfqpoint{0.984126in}{2.376000in}}%
\pgfpathcurveto{\pgfqpoint{0.984126in}{2.364950in}}{\pgfqpoint{0.988517in}{2.354351in}}{\pgfqpoint{0.996330in}{2.346537in}}%
\pgfpathcurveto{\pgfqpoint{1.004144in}{2.338724in}}{\pgfqpoint{1.014743in}{2.334333in}}{\pgfqpoint{1.025793in}{2.334333in}}%
\pgfpathclose%
\pgfusepath{stroke,fill}%
\end{pgfscope}%
\begin{pgfscope}%
\pgfpathrectangle{\pgfqpoint{0.800000in}{0.528000in}}{\pgfqpoint{4.960000in}{3.696000in}}%
\pgfusepath{clip}%
\pgfsetbuttcap%
\pgfsetroundjoin%
\definecolor{currentfill}{rgb}{0.000000,0.000000,0.000000}%
\pgfsetfillcolor{currentfill}%
\pgfsetlinewidth{1.003750pt}%
\definecolor{currentstroke}{rgb}{0.000000,0.000000,0.000000}%
\pgfsetstrokecolor{currentstroke}%
\pgfsetdash{}{0pt}%
\pgfpathmoveto{\pgfqpoint{1.025793in}{2.334333in}}%
\pgfpathcurveto{\pgfqpoint{1.036843in}{2.334333in}}{\pgfqpoint{1.047442in}{2.338724in}}{\pgfqpoint{1.055256in}{2.346537in}}%
\pgfpathcurveto{\pgfqpoint{1.063070in}{2.354351in}}{\pgfqpoint{1.067460in}{2.364950in}}{\pgfqpoint{1.067460in}{2.376000in}}%
\pgfpathcurveto{\pgfqpoint{1.067460in}{2.387050in}}{\pgfqpoint{1.063070in}{2.397649in}}{\pgfqpoint{1.055256in}{2.405463in}}%
\pgfpathcurveto{\pgfqpoint{1.047442in}{2.413276in}}{\pgfqpoint{1.036843in}{2.417667in}}{\pgfqpoint{1.025793in}{2.417667in}}%
\pgfpathcurveto{\pgfqpoint{1.014743in}{2.417667in}}{\pgfqpoint{1.004144in}{2.413276in}}{\pgfqpoint{0.996330in}{2.405463in}}%
\pgfpathcurveto{\pgfqpoint{0.988517in}{2.397649in}}{\pgfqpoint{0.984126in}{2.387050in}}{\pgfqpoint{0.984126in}{2.376000in}}%
\pgfpathcurveto{\pgfqpoint{0.984126in}{2.364950in}}{\pgfqpoint{0.988517in}{2.354351in}}{\pgfqpoint{0.996330in}{2.346537in}}%
\pgfpathcurveto{\pgfqpoint{1.004144in}{2.338724in}}{\pgfqpoint{1.014743in}{2.334333in}}{\pgfqpoint{1.025793in}{2.334333in}}%
\pgfpathclose%
\pgfusepath{stroke,fill}%
\end{pgfscope}%
\begin{pgfscope}%
\pgfpathrectangle{\pgfqpoint{0.800000in}{0.528000in}}{\pgfqpoint{4.960000in}{3.696000in}}%
\pgfusepath{clip}%
\pgfsetbuttcap%
\pgfsetroundjoin%
\definecolor{currentfill}{rgb}{0.000000,0.000000,0.000000}%
\pgfsetfillcolor{currentfill}%
\pgfsetlinewidth{1.003750pt}%
\definecolor{currentstroke}{rgb}{0.000000,0.000000,0.000000}%
\pgfsetstrokecolor{currentstroke}%
\pgfsetdash{}{0pt}%
\pgfpathmoveto{\pgfqpoint{1.025793in}{2.334333in}}%
\pgfpathcurveto{\pgfqpoint{1.036843in}{2.334333in}}{\pgfqpoint{1.047442in}{2.338724in}}{\pgfqpoint{1.055256in}{2.346537in}}%
\pgfpathcurveto{\pgfqpoint{1.063070in}{2.354351in}}{\pgfqpoint{1.067460in}{2.364950in}}{\pgfqpoint{1.067460in}{2.376000in}}%
\pgfpathcurveto{\pgfqpoint{1.067460in}{2.387050in}}{\pgfqpoint{1.063070in}{2.397649in}}{\pgfqpoint{1.055256in}{2.405463in}}%
\pgfpathcurveto{\pgfqpoint{1.047442in}{2.413276in}}{\pgfqpoint{1.036843in}{2.417667in}}{\pgfqpoint{1.025793in}{2.417667in}}%
\pgfpathcurveto{\pgfqpoint{1.014743in}{2.417667in}}{\pgfqpoint{1.004144in}{2.413276in}}{\pgfqpoint{0.996330in}{2.405463in}}%
\pgfpathcurveto{\pgfqpoint{0.988517in}{2.397649in}}{\pgfqpoint{0.984126in}{2.387050in}}{\pgfqpoint{0.984126in}{2.376000in}}%
\pgfpathcurveto{\pgfqpoint{0.984126in}{2.364950in}}{\pgfqpoint{0.988517in}{2.354351in}}{\pgfqpoint{0.996330in}{2.346537in}}%
\pgfpathcurveto{\pgfqpoint{1.004144in}{2.338724in}}{\pgfqpoint{1.014743in}{2.334333in}}{\pgfqpoint{1.025793in}{2.334333in}}%
\pgfpathclose%
\pgfusepath{stroke,fill}%
\end{pgfscope}%
\begin{pgfscope}%
\pgfpathrectangle{\pgfqpoint{0.800000in}{0.528000in}}{\pgfqpoint{4.960000in}{3.696000in}}%
\pgfusepath{clip}%
\pgfsetbuttcap%
\pgfsetroundjoin%
\definecolor{currentfill}{rgb}{0.000000,0.000000,0.000000}%
\pgfsetfillcolor{currentfill}%
\pgfsetlinewidth{1.003750pt}%
\definecolor{currentstroke}{rgb}{0.000000,0.000000,0.000000}%
\pgfsetstrokecolor{currentstroke}%
\pgfsetdash{}{0pt}%
\pgfpathmoveto{\pgfqpoint{1.025793in}{2.334333in}}%
\pgfpathcurveto{\pgfqpoint{1.036843in}{2.334333in}}{\pgfqpoint{1.047442in}{2.338724in}}{\pgfqpoint{1.055256in}{2.346537in}}%
\pgfpathcurveto{\pgfqpoint{1.063070in}{2.354351in}}{\pgfqpoint{1.067460in}{2.364950in}}{\pgfqpoint{1.067460in}{2.376000in}}%
\pgfpathcurveto{\pgfqpoint{1.067460in}{2.387050in}}{\pgfqpoint{1.063070in}{2.397649in}}{\pgfqpoint{1.055256in}{2.405463in}}%
\pgfpathcurveto{\pgfqpoint{1.047442in}{2.413276in}}{\pgfqpoint{1.036843in}{2.417667in}}{\pgfqpoint{1.025793in}{2.417667in}}%
\pgfpathcurveto{\pgfqpoint{1.014743in}{2.417667in}}{\pgfqpoint{1.004144in}{2.413276in}}{\pgfqpoint{0.996330in}{2.405463in}}%
\pgfpathcurveto{\pgfqpoint{0.988517in}{2.397649in}}{\pgfqpoint{0.984126in}{2.387050in}}{\pgfqpoint{0.984126in}{2.376000in}}%
\pgfpathcurveto{\pgfqpoint{0.984126in}{2.364950in}}{\pgfqpoint{0.988517in}{2.354351in}}{\pgfqpoint{0.996330in}{2.346537in}}%
\pgfpathcurveto{\pgfqpoint{1.004144in}{2.338724in}}{\pgfqpoint{1.014743in}{2.334333in}}{\pgfqpoint{1.025793in}{2.334333in}}%
\pgfpathclose%
\pgfusepath{stroke,fill}%
\end{pgfscope}%
\begin{pgfscope}%
\pgfpathrectangle{\pgfqpoint{0.800000in}{0.528000in}}{\pgfqpoint{4.960000in}{3.696000in}}%
\pgfusepath{clip}%
\pgfsetbuttcap%
\pgfsetroundjoin%
\definecolor{currentfill}{rgb}{0.000000,0.000000,0.000000}%
\pgfsetfillcolor{currentfill}%
\pgfsetlinewidth{1.003750pt}%
\definecolor{currentstroke}{rgb}{0.000000,0.000000,0.000000}%
\pgfsetstrokecolor{currentstroke}%
\pgfsetdash{}{0pt}%
\pgfpathmoveto{\pgfqpoint{1.025793in}{2.334333in}}%
\pgfpathcurveto{\pgfqpoint{1.036843in}{2.334333in}}{\pgfqpoint{1.047442in}{2.338724in}}{\pgfqpoint{1.055256in}{2.346537in}}%
\pgfpathcurveto{\pgfqpoint{1.063070in}{2.354351in}}{\pgfqpoint{1.067460in}{2.364950in}}{\pgfqpoint{1.067460in}{2.376000in}}%
\pgfpathcurveto{\pgfqpoint{1.067460in}{2.387050in}}{\pgfqpoint{1.063070in}{2.397649in}}{\pgfqpoint{1.055256in}{2.405463in}}%
\pgfpathcurveto{\pgfqpoint{1.047442in}{2.413276in}}{\pgfqpoint{1.036843in}{2.417667in}}{\pgfqpoint{1.025793in}{2.417667in}}%
\pgfpathcurveto{\pgfqpoint{1.014743in}{2.417667in}}{\pgfqpoint{1.004144in}{2.413276in}}{\pgfqpoint{0.996330in}{2.405463in}}%
\pgfpathcurveto{\pgfqpoint{0.988517in}{2.397649in}}{\pgfqpoint{0.984126in}{2.387050in}}{\pgfqpoint{0.984126in}{2.376000in}}%
\pgfpathcurveto{\pgfqpoint{0.984126in}{2.364950in}}{\pgfqpoint{0.988517in}{2.354351in}}{\pgfqpoint{0.996330in}{2.346537in}}%
\pgfpathcurveto{\pgfqpoint{1.004144in}{2.338724in}}{\pgfqpoint{1.014743in}{2.334333in}}{\pgfqpoint{1.025793in}{2.334333in}}%
\pgfpathclose%
\pgfusepath{stroke,fill}%
\end{pgfscope}%
\begin{pgfscope}%
\pgfpathrectangle{\pgfqpoint{0.800000in}{0.528000in}}{\pgfqpoint{4.960000in}{3.696000in}}%
\pgfusepath{clip}%
\pgfsetbuttcap%
\pgfsetroundjoin%
\definecolor{currentfill}{rgb}{0.000000,0.000000,0.000000}%
\pgfsetfillcolor{currentfill}%
\pgfsetlinewidth{1.003750pt}%
\definecolor{currentstroke}{rgb}{0.000000,0.000000,0.000000}%
\pgfsetstrokecolor{currentstroke}%
\pgfsetdash{}{0pt}%
\pgfpathmoveto{\pgfqpoint{1.025793in}{2.334333in}}%
\pgfpathcurveto{\pgfqpoint{1.036843in}{2.334333in}}{\pgfqpoint{1.047442in}{2.338724in}}{\pgfqpoint{1.055256in}{2.346537in}}%
\pgfpathcurveto{\pgfqpoint{1.063070in}{2.354351in}}{\pgfqpoint{1.067460in}{2.364950in}}{\pgfqpoint{1.067460in}{2.376000in}}%
\pgfpathcurveto{\pgfqpoint{1.067460in}{2.387050in}}{\pgfqpoint{1.063070in}{2.397649in}}{\pgfqpoint{1.055256in}{2.405463in}}%
\pgfpathcurveto{\pgfqpoint{1.047442in}{2.413276in}}{\pgfqpoint{1.036843in}{2.417667in}}{\pgfqpoint{1.025793in}{2.417667in}}%
\pgfpathcurveto{\pgfqpoint{1.014743in}{2.417667in}}{\pgfqpoint{1.004144in}{2.413276in}}{\pgfqpoint{0.996330in}{2.405463in}}%
\pgfpathcurveto{\pgfqpoint{0.988517in}{2.397649in}}{\pgfqpoint{0.984126in}{2.387050in}}{\pgfqpoint{0.984126in}{2.376000in}}%
\pgfpathcurveto{\pgfqpoint{0.984126in}{2.364950in}}{\pgfqpoint{0.988517in}{2.354351in}}{\pgfqpoint{0.996330in}{2.346537in}}%
\pgfpathcurveto{\pgfqpoint{1.004144in}{2.338724in}}{\pgfqpoint{1.014743in}{2.334333in}}{\pgfqpoint{1.025793in}{2.334333in}}%
\pgfpathclose%
\pgfusepath{stroke,fill}%
\end{pgfscope}%
\begin{pgfscope}%
\pgfpathrectangle{\pgfqpoint{0.800000in}{0.528000in}}{\pgfqpoint{4.960000in}{3.696000in}}%
\pgfusepath{clip}%
\pgfsetbuttcap%
\pgfsetroundjoin%
\definecolor{currentfill}{rgb}{0.000000,0.000000,0.000000}%
\pgfsetfillcolor{currentfill}%
\pgfsetlinewidth{1.003750pt}%
\definecolor{currentstroke}{rgb}{0.000000,0.000000,0.000000}%
\pgfsetstrokecolor{currentstroke}%
\pgfsetdash{}{0pt}%
\pgfpathmoveto{\pgfqpoint{1.025793in}{2.334333in}}%
\pgfpathcurveto{\pgfqpoint{1.036843in}{2.334333in}}{\pgfqpoint{1.047442in}{2.338724in}}{\pgfqpoint{1.055256in}{2.346537in}}%
\pgfpathcurveto{\pgfqpoint{1.063070in}{2.354351in}}{\pgfqpoint{1.067460in}{2.364950in}}{\pgfqpoint{1.067460in}{2.376000in}}%
\pgfpathcurveto{\pgfqpoint{1.067460in}{2.387050in}}{\pgfqpoint{1.063070in}{2.397649in}}{\pgfqpoint{1.055256in}{2.405463in}}%
\pgfpathcurveto{\pgfqpoint{1.047442in}{2.413276in}}{\pgfqpoint{1.036843in}{2.417667in}}{\pgfqpoint{1.025793in}{2.417667in}}%
\pgfpathcurveto{\pgfqpoint{1.014743in}{2.417667in}}{\pgfqpoint{1.004144in}{2.413276in}}{\pgfqpoint{0.996330in}{2.405463in}}%
\pgfpathcurveto{\pgfqpoint{0.988517in}{2.397649in}}{\pgfqpoint{0.984126in}{2.387050in}}{\pgfqpoint{0.984126in}{2.376000in}}%
\pgfpathcurveto{\pgfqpoint{0.984126in}{2.364950in}}{\pgfqpoint{0.988517in}{2.354351in}}{\pgfqpoint{0.996330in}{2.346537in}}%
\pgfpathcurveto{\pgfqpoint{1.004144in}{2.338724in}}{\pgfqpoint{1.014743in}{2.334333in}}{\pgfqpoint{1.025793in}{2.334333in}}%
\pgfpathclose%
\pgfusepath{stroke,fill}%
\end{pgfscope}%
\begin{pgfscope}%
\pgfpathrectangle{\pgfqpoint{0.800000in}{0.528000in}}{\pgfqpoint{4.960000in}{3.696000in}}%
\pgfusepath{clip}%
\pgfsetbuttcap%
\pgfsetroundjoin%
\definecolor{currentfill}{rgb}{0.000000,0.000000,0.000000}%
\pgfsetfillcolor{currentfill}%
\pgfsetlinewidth{1.003750pt}%
\definecolor{currentstroke}{rgb}{0.000000,0.000000,0.000000}%
\pgfsetstrokecolor{currentstroke}%
\pgfsetdash{}{0pt}%
\pgfpathmoveto{\pgfqpoint{1.025793in}{2.334333in}}%
\pgfpathcurveto{\pgfqpoint{1.036843in}{2.334333in}}{\pgfqpoint{1.047442in}{2.338724in}}{\pgfqpoint{1.055256in}{2.346537in}}%
\pgfpathcurveto{\pgfqpoint{1.063070in}{2.354351in}}{\pgfqpoint{1.067460in}{2.364950in}}{\pgfqpoint{1.067460in}{2.376000in}}%
\pgfpathcurveto{\pgfqpoint{1.067460in}{2.387050in}}{\pgfqpoint{1.063070in}{2.397649in}}{\pgfqpoint{1.055256in}{2.405463in}}%
\pgfpathcurveto{\pgfqpoint{1.047442in}{2.413276in}}{\pgfqpoint{1.036843in}{2.417667in}}{\pgfqpoint{1.025793in}{2.417667in}}%
\pgfpathcurveto{\pgfqpoint{1.014743in}{2.417667in}}{\pgfqpoint{1.004144in}{2.413276in}}{\pgfqpoint{0.996330in}{2.405463in}}%
\pgfpathcurveto{\pgfqpoint{0.988517in}{2.397649in}}{\pgfqpoint{0.984126in}{2.387050in}}{\pgfqpoint{0.984126in}{2.376000in}}%
\pgfpathcurveto{\pgfqpoint{0.984126in}{2.364950in}}{\pgfqpoint{0.988517in}{2.354351in}}{\pgfqpoint{0.996330in}{2.346537in}}%
\pgfpathcurveto{\pgfqpoint{1.004144in}{2.338724in}}{\pgfqpoint{1.014743in}{2.334333in}}{\pgfqpoint{1.025793in}{2.334333in}}%
\pgfpathclose%
\pgfusepath{stroke,fill}%
\end{pgfscope}%
\begin{pgfscope}%
\pgfpathrectangle{\pgfqpoint{0.800000in}{0.528000in}}{\pgfqpoint{4.960000in}{3.696000in}}%
\pgfusepath{clip}%
\pgfsetbuttcap%
\pgfsetroundjoin%
\definecolor{currentfill}{rgb}{0.000000,0.000000,0.000000}%
\pgfsetfillcolor{currentfill}%
\pgfsetlinewidth{1.003750pt}%
\definecolor{currentstroke}{rgb}{0.000000,0.000000,0.000000}%
\pgfsetstrokecolor{currentstroke}%
\pgfsetdash{}{0pt}%
\pgfpathmoveto{\pgfqpoint{1.025793in}{2.334333in}}%
\pgfpathcurveto{\pgfqpoint{1.036843in}{2.334333in}}{\pgfqpoint{1.047442in}{2.338724in}}{\pgfqpoint{1.055256in}{2.346537in}}%
\pgfpathcurveto{\pgfqpoint{1.063070in}{2.354351in}}{\pgfqpoint{1.067460in}{2.364950in}}{\pgfqpoint{1.067460in}{2.376000in}}%
\pgfpathcurveto{\pgfqpoint{1.067460in}{2.387050in}}{\pgfqpoint{1.063070in}{2.397649in}}{\pgfqpoint{1.055256in}{2.405463in}}%
\pgfpathcurveto{\pgfqpoint{1.047442in}{2.413276in}}{\pgfqpoint{1.036843in}{2.417667in}}{\pgfqpoint{1.025793in}{2.417667in}}%
\pgfpathcurveto{\pgfqpoint{1.014743in}{2.417667in}}{\pgfqpoint{1.004144in}{2.413276in}}{\pgfqpoint{0.996330in}{2.405463in}}%
\pgfpathcurveto{\pgfqpoint{0.988517in}{2.397649in}}{\pgfqpoint{0.984126in}{2.387050in}}{\pgfqpoint{0.984126in}{2.376000in}}%
\pgfpathcurveto{\pgfqpoint{0.984126in}{2.364950in}}{\pgfqpoint{0.988517in}{2.354351in}}{\pgfqpoint{0.996330in}{2.346537in}}%
\pgfpathcurveto{\pgfqpoint{1.004144in}{2.338724in}}{\pgfqpoint{1.014743in}{2.334333in}}{\pgfqpoint{1.025793in}{2.334333in}}%
\pgfpathclose%
\pgfusepath{stroke,fill}%
\end{pgfscope}%
\begin{pgfscope}%
\pgfpathrectangle{\pgfqpoint{0.800000in}{0.528000in}}{\pgfqpoint{4.960000in}{3.696000in}}%
\pgfusepath{clip}%
\pgfsetbuttcap%
\pgfsetroundjoin%
\definecolor{currentfill}{rgb}{0.000000,0.000000,0.000000}%
\pgfsetfillcolor{currentfill}%
\pgfsetlinewidth{1.003750pt}%
\definecolor{currentstroke}{rgb}{0.000000,0.000000,0.000000}%
\pgfsetstrokecolor{currentstroke}%
\pgfsetdash{}{0pt}%
\pgfpathmoveto{\pgfqpoint{1.025793in}{2.334333in}}%
\pgfpathcurveto{\pgfqpoint{1.036843in}{2.334333in}}{\pgfqpoint{1.047442in}{2.338724in}}{\pgfqpoint{1.055256in}{2.346537in}}%
\pgfpathcurveto{\pgfqpoint{1.063070in}{2.354351in}}{\pgfqpoint{1.067460in}{2.364950in}}{\pgfqpoint{1.067460in}{2.376000in}}%
\pgfpathcurveto{\pgfqpoint{1.067460in}{2.387050in}}{\pgfqpoint{1.063070in}{2.397649in}}{\pgfqpoint{1.055256in}{2.405463in}}%
\pgfpathcurveto{\pgfqpoint{1.047442in}{2.413276in}}{\pgfqpoint{1.036843in}{2.417667in}}{\pgfqpoint{1.025793in}{2.417667in}}%
\pgfpathcurveto{\pgfqpoint{1.014743in}{2.417667in}}{\pgfqpoint{1.004144in}{2.413276in}}{\pgfqpoint{0.996330in}{2.405463in}}%
\pgfpathcurveto{\pgfqpoint{0.988517in}{2.397649in}}{\pgfqpoint{0.984126in}{2.387050in}}{\pgfqpoint{0.984126in}{2.376000in}}%
\pgfpathcurveto{\pgfqpoint{0.984126in}{2.364950in}}{\pgfqpoint{0.988517in}{2.354351in}}{\pgfqpoint{0.996330in}{2.346537in}}%
\pgfpathcurveto{\pgfqpoint{1.004144in}{2.338724in}}{\pgfqpoint{1.014743in}{2.334333in}}{\pgfqpoint{1.025793in}{2.334333in}}%
\pgfpathclose%
\pgfusepath{stroke,fill}%
\end{pgfscope}%
\begin{pgfscope}%
\pgfpathrectangle{\pgfqpoint{0.800000in}{0.528000in}}{\pgfqpoint{4.960000in}{3.696000in}}%
\pgfusepath{clip}%
\pgfsetbuttcap%
\pgfsetroundjoin%
\definecolor{currentfill}{rgb}{0.000000,0.000000,0.000000}%
\pgfsetfillcolor{currentfill}%
\pgfsetlinewidth{1.003750pt}%
\definecolor{currentstroke}{rgb}{0.000000,0.000000,0.000000}%
\pgfsetstrokecolor{currentstroke}%
\pgfsetdash{}{0pt}%
\pgfpathmoveto{\pgfqpoint{1.025793in}{2.334333in}}%
\pgfpathcurveto{\pgfqpoint{1.036843in}{2.334333in}}{\pgfqpoint{1.047442in}{2.338724in}}{\pgfqpoint{1.055256in}{2.346537in}}%
\pgfpathcurveto{\pgfqpoint{1.063070in}{2.354351in}}{\pgfqpoint{1.067460in}{2.364950in}}{\pgfqpoint{1.067460in}{2.376000in}}%
\pgfpathcurveto{\pgfqpoint{1.067460in}{2.387050in}}{\pgfqpoint{1.063070in}{2.397649in}}{\pgfqpoint{1.055256in}{2.405463in}}%
\pgfpathcurveto{\pgfqpoint{1.047442in}{2.413276in}}{\pgfqpoint{1.036843in}{2.417667in}}{\pgfqpoint{1.025793in}{2.417667in}}%
\pgfpathcurveto{\pgfqpoint{1.014743in}{2.417667in}}{\pgfqpoint{1.004144in}{2.413276in}}{\pgfqpoint{0.996330in}{2.405463in}}%
\pgfpathcurveto{\pgfqpoint{0.988517in}{2.397649in}}{\pgfqpoint{0.984126in}{2.387050in}}{\pgfqpoint{0.984126in}{2.376000in}}%
\pgfpathcurveto{\pgfqpoint{0.984126in}{2.364950in}}{\pgfqpoint{0.988517in}{2.354351in}}{\pgfqpoint{0.996330in}{2.346537in}}%
\pgfpathcurveto{\pgfqpoint{1.004144in}{2.338724in}}{\pgfqpoint{1.014743in}{2.334333in}}{\pgfqpoint{1.025793in}{2.334333in}}%
\pgfpathclose%
\pgfusepath{stroke,fill}%
\end{pgfscope}%
\begin{pgfscope}%
\pgfpathrectangle{\pgfqpoint{0.800000in}{0.528000in}}{\pgfqpoint{4.960000in}{3.696000in}}%
\pgfusepath{clip}%
\pgfsetbuttcap%
\pgfsetroundjoin%
\definecolor{currentfill}{rgb}{0.000000,0.000000,0.000000}%
\pgfsetfillcolor{currentfill}%
\pgfsetlinewidth{1.003750pt}%
\definecolor{currentstroke}{rgb}{0.000000,0.000000,0.000000}%
\pgfsetstrokecolor{currentstroke}%
\pgfsetdash{}{0pt}%
\pgfpathmoveto{\pgfqpoint{1.025793in}{2.334333in}}%
\pgfpathcurveto{\pgfqpoint{1.036843in}{2.334333in}}{\pgfqpoint{1.047442in}{2.338724in}}{\pgfqpoint{1.055256in}{2.346537in}}%
\pgfpathcurveto{\pgfqpoint{1.063070in}{2.354351in}}{\pgfqpoint{1.067460in}{2.364950in}}{\pgfqpoint{1.067460in}{2.376000in}}%
\pgfpathcurveto{\pgfqpoint{1.067460in}{2.387050in}}{\pgfqpoint{1.063070in}{2.397649in}}{\pgfqpoint{1.055256in}{2.405463in}}%
\pgfpathcurveto{\pgfqpoint{1.047442in}{2.413276in}}{\pgfqpoint{1.036843in}{2.417667in}}{\pgfqpoint{1.025793in}{2.417667in}}%
\pgfpathcurveto{\pgfqpoint{1.014743in}{2.417667in}}{\pgfqpoint{1.004144in}{2.413276in}}{\pgfqpoint{0.996330in}{2.405463in}}%
\pgfpathcurveto{\pgfqpoint{0.988517in}{2.397649in}}{\pgfqpoint{0.984126in}{2.387050in}}{\pgfqpoint{0.984126in}{2.376000in}}%
\pgfpathcurveto{\pgfqpoint{0.984126in}{2.364950in}}{\pgfqpoint{0.988517in}{2.354351in}}{\pgfqpoint{0.996330in}{2.346537in}}%
\pgfpathcurveto{\pgfqpoint{1.004144in}{2.338724in}}{\pgfqpoint{1.014743in}{2.334333in}}{\pgfqpoint{1.025793in}{2.334333in}}%
\pgfpathclose%
\pgfusepath{stroke,fill}%
\end{pgfscope}%
\begin{pgfscope}%
\pgfpathrectangle{\pgfqpoint{0.800000in}{0.528000in}}{\pgfqpoint{4.960000in}{3.696000in}}%
\pgfusepath{clip}%
\pgfsetbuttcap%
\pgfsetroundjoin%
\definecolor{currentfill}{rgb}{0.000000,0.000000,0.000000}%
\pgfsetfillcolor{currentfill}%
\pgfsetlinewidth{1.003750pt}%
\definecolor{currentstroke}{rgb}{0.000000,0.000000,0.000000}%
\pgfsetstrokecolor{currentstroke}%
\pgfsetdash{}{0pt}%
\pgfpathmoveto{\pgfqpoint{1.025793in}{2.334333in}}%
\pgfpathcurveto{\pgfqpoint{1.036843in}{2.334333in}}{\pgfqpoint{1.047442in}{2.338724in}}{\pgfqpoint{1.055256in}{2.346537in}}%
\pgfpathcurveto{\pgfqpoint{1.063070in}{2.354351in}}{\pgfqpoint{1.067460in}{2.364950in}}{\pgfqpoint{1.067460in}{2.376000in}}%
\pgfpathcurveto{\pgfqpoint{1.067460in}{2.387050in}}{\pgfqpoint{1.063070in}{2.397649in}}{\pgfqpoint{1.055256in}{2.405463in}}%
\pgfpathcurveto{\pgfqpoint{1.047442in}{2.413276in}}{\pgfqpoint{1.036843in}{2.417667in}}{\pgfqpoint{1.025793in}{2.417667in}}%
\pgfpathcurveto{\pgfqpoint{1.014743in}{2.417667in}}{\pgfqpoint{1.004144in}{2.413276in}}{\pgfqpoint{0.996330in}{2.405463in}}%
\pgfpathcurveto{\pgfqpoint{0.988517in}{2.397649in}}{\pgfqpoint{0.984126in}{2.387050in}}{\pgfqpoint{0.984126in}{2.376000in}}%
\pgfpathcurveto{\pgfqpoint{0.984126in}{2.364950in}}{\pgfqpoint{0.988517in}{2.354351in}}{\pgfqpoint{0.996330in}{2.346537in}}%
\pgfpathcurveto{\pgfqpoint{1.004144in}{2.338724in}}{\pgfqpoint{1.014743in}{2.334333in}}{\pgfqpoint{1.025793in}{2.334333in}}%
\pgfpathclose%
\pgfusepath{stroke,fill}%
\end{pgfscope}%
\begin{pgfscope}%
\pgfpathrectangle{\pgfqpoint{0.800000in}{0.528000in}}{\pgfqpoint{4.960000in}{3.696000in}}%
\pgfusepath{clip}%
\pgfsetbuttcap%
\pgfsetroundjoin%
\definecolor{currentfill}{rgb}{0.000000,0.000000,0.000000}%
\pgfsetfillcolor{currentfill}%
\pgfsetlinewidth{1.003750pt}%
\definecolor{currentstroke}{rgb}{0.000000,0.000000,0.000000}%
\pgfsetstrokecolor{currentstroke}%
\pgfsetdash{}{0pt}%
\pgfpathmoveto{\pgfqpoint{1.025793in}{2.334333in}}%
\pgfpathcurveto{\pgfqpoint{1.036843in}{2.334333in}}{\pgfqpoint{1.047442in}{2.338724in}}{\pgfqpoint{1.055256in}{2.346537in}}%
\pgfpathcurveto{\pgfqpoint{1.063070in}{2.354351in}}{\pgfqpoint{1.067460in}{2.364950in}}{\pgfqpoint{1.067460in}{2.376000in}}%
\pgfpathcurveto{\pgfqpoint{1.067460in}{2.387050in}}{\pgfqpoint{1.063070in}{2.397649in}}{\pgfqpoint{1.055256in}{2.405463in}}%
\pgfpathcurveto{\pgfqpoint{1.047442in}{2.413276in}}{\pgfqpoint{1.036843in}{2.417667in}}{\pgfqpoint{1.025793in}{2.417667in}}%
\pgfpathcurveto{\pgfqpoint{1.014743in}{2.417667in}}{\pgfqpoint{1.004144in}{2.413276in}}{\pgfqpoint{0.996330in}{2.405463in}}%
\pgfpathcurveto{\pgfqpoint{0.988517in}{2.397649in}}{\pgfqpoint{0.984126in}{2.387050in}}{\pgfqpoint{0.984126in}{2.376000in}}%
\pgfpathcurveto{\pgfqpoint{0.984126in}{2.364950in}}{\pgfqpoint{0.988517in}{2.354351in}}{\pgfqpoint{0.996330in}{2.346537in}}%
\pgfpathcurveto{\pgfqpoint{1.004144in}{2.338724in}}{\pgfqpoint{1.014743in}{2.334333in}}{\pgfqpoint{1.025793in}{2.334333in}}%
\pgfpathclose%
\pgfusepath{stroke,fill}%
\end{pgfscope}%
\begin{pgfscope}%
\pgfpathrectangle{\pgfqpoint{0.800000in}{0.528000in}}{\pgfqpoint{4.960000in}{3.696000in}}%
\pgfusepath{clip}%
\pgfsetbuttcap%
\pgfsetroundjoin%
\definecolor{currentfill}{rgb}{0.000000,0.000000,0.000000}%
\pgfsetfillcolor{currentfill}%
\pgfsetlinewidth{1.003750pt}%
\definecolor{currentstroke}{rgb}{0.000000,0.000000,0.000000}%
\pgfsetstrokecolor{currentstroke}%
\pgfsetdash{}{0pt}%
\pgfpathmoveto{\pgfqpoint{1.025793in}{2.334333in}}%
\pgfpathcurveto{\pgfqpoint{1.036843in}{2.334333in}}{\pgfqpoint{1.047442in}{2.338724in}}{\pgfqpoint{1.055256in}{2.346537in}}%
\pgfpathcurveto{\pgfqpoint{1.063070in}{2.354351in}}{\pgfqpoint{1.067460in}{2.364950in}}{\pgfqpoint{1.067460in}{2.376000in}}%
\pgfpathcurveto{\pgfqpoint{1.067460in}{2.387050in}}{\pgfqpoint{1.063070in}{2.397649in}}{\pgfqpoint{1.055256in}{2.405463in}}%
\pgfpathcurveto{\pgfqpoint{1.047442in}{2.413276in}}{\pgfqpoint{1.036843in}{2.417667in}}{\pgfqpoint{1.025793in}{2.417667in}}%
\pgfpathcurveto{\pgfqpoint{1.014743in}{2.417667in}}{\pgfqpoint{1.004144in}{2.413276in}}{\pgfqpoint{0.996330in}{2.405463in}}%
\pgfpathcurveto{\pgfqpoint{0.988517in}{2.397649in}}{\pgfqpoint{0.984126in}{2.387050in}}{\pgfqpoint{0.984126in}{2.376000in}}%
\pgfpathcurveto{\pgfqpoint{0.984126in}{2.364950in}}{\pgfqpoint{0.988517in}{2.354351in}}{\pgfqpoint{0.996330in}{2.346537in}}%
\pgfpathcurveto{\pgfqpoint{1.004144in}{2.338724in}}{\pgfqpoint{1.014743in}{2.334333in}}{\pgfqpoint{1.025793in}{2.334333in}}%
\pgfpathclose%
\pgfusepath{stroke,fill}%
\end{pgfscope}%
\begin{pgfscope}%
\pgfpathrectangle{\pgfqpoint{0.800000in}{0.528000in}}{\pgfqpoint{4.960000in}{3.696000in}}%
\pgfusepath{clip}%
\pgfsetbuttcap%
\pgfsetroundjoin%
\definecolor{currentfill}{rgb}{0.000000,0.000000,0.000000}%
\pgfsetfillcolor{currentfill}%
\pgfsetlinewidth{1.003750pt}%
\definecolor{currentstroke}{rgb}{0.000000,0.000000,0.000000}%
\pgfsetstrokecolor{currentstroke}%
\pgfsetdash{}{0pt}%
\pgfpathmoveto{\pgfqpoint{1.025793in}{2.334333in}}%
\pgfpathcurveto{\pgfqpoint{1.036843in}{2.334333in}}{\pgfqpoint{1.047442in}{2.338724in}}{\pgfqpoint{1.055256in}{2.346537in}}%
\pgfpathcurveto{\pgfqpoint{1.063070in}{2.354351in}}{\pgfqpoint{1.067460in}{2.364950in}}{\pgfqpoint{1.067460in}{2.376000in}}%
\pgfpathcurveto{\pgfqpoint{1.067460in}{2.387050in}}{\pgfqpoint{1.063070in}{2.397649in}}{\pgfqpoint{1.055256in}{2.405463in}}%
\pgfpathcurveto{\pgfqpoint{1.047442in}{2.413276in}}{\pgfqpoint{1.036843in}{2.417667in}}{\pgfqpoint{1.025793in}{2.417667in}}%
\pgfpathcurveto{\pgfqpoint{1.014743in}{2.417667in}}{\pgfqpoint{1.004144in}{2.413276in}}{\pgfqpoint{0.996330in}{2.405463in}}%
\pgfpathcurveto{\pgfqpoint{0.988517in}{2.397649in}}{\pgfqpoint{0.984126in}{2.387050in}}{\pgfqpoint{0.984126in}{2.376000in}}%
\pgfpathcurveto{\pgfqpoint{0.984126in}{2.364950in}}{\pgfqpoint{0.988517in}{2.354351in}}{\pgfqpoint{0.996330in}{2.346537in}}%
\pgfpathcurveto{\pgfqpoint{1.004144in}{2.338724in}}{\pgfqpoint{1.014743in}{2.334333in}}{\pgfqpoint{1.025793in}{2.334333in}}%
\pgfpathclose%
\pgfusepath{stroke,fill}%
\end{pgfscope}%
\begin{pgfscope}%
\pgfpathrectangle{\pgfqpoint{0.800000in}{0.528000in}}{\pgfqpoint{4.960000in}{3.696000in}}%
\pgfusepath{clip}%
\pgfsetbuttcap%
\pgfsetroundjoin%
\definecolor{currentfill}{rgb}{0.000000,0.000000,0.000000}%
\pgfsetfillcolor{currentfill}%
\pgfsetlinewidth{1.003750pt}%
\definecolor{currentstroke}{rgb}{0.000000,0.000000,0.000000}%
\pgfsetstrokecolor{currentstroke}%
\pgfsetdash{}{0pt}%
\pgfpathmoveto{\pgfqpoint{2.145481in}{2.334333in}}%
\pgfpathcurveto{\pgfqpoint{2.156531in}{2.334333in}}{\pgfqpoint{2.167130in}{2.338724in}}{\pgfqpoint{2.174944in}{2.346537in}}%
\pgfpathcurveto{\pgfqpoint{2.182758in}{2.354351in}}{\pgfqpoint{2.187148in}{2.364950in}}{\pgfqpoint{2.187148in}{2.376000in}}%
\pgfpathcurveto{\pgfqpoint{2.187148in}{2.387050in}}{\pgfqpoint{2.182758in}{2.397649in}}{\pgfqpoint{2.174944in}{2.405463in}}%
\pgfpathcurveto{\pgfqpoint{2.167130in}{2.413276in}}{\pgfqpoint{2.156531in}{2.417667in}}{\pgfqpoint{2.145481in}{2.417667in}}%
\pgfpathcurveto{\pgfqpoint{2.134431in}{2.417667in}}{\pgfqpoint{2.123832in}{2.413276in}}{\pgfqpoint{2.116018in}{2.405463in}}%
\pgfpathcurveto{\pgfqpoint{2.108205in}{2.397649in}}{\pgfqpoint{2.103815in}{2.387050in}}{\pgfqpoint{2.103815in}{2.376000in}}%
\pgfpathcurveto{\pgfqpoint{2.103815in}{2.364950in}}{\pgfqpoint{2.108205in}{2.354351in}}{\pgfqpoint{2.116018in}{2.346537in}}%
\pgfpathcurveto{\pgfqpoint{2.123832in}{2.338724in}}{\pgfqpoint{2.134431in}{2.334333in}}{\pgfqpoint{2.145481in}{2.334333in}}%
\pgfpathclose%
\pgfusepath{stroke,fill}%
\end{pgfscope}%
\begin{pgfscope}%
\pgfpathrectangle{\pgfqpoint{0.800000in}{0.528000in}}{\pgfqpoint{4.960000in}{3.696000in}}%
\pgfusepath{clip}%
\pgfsetbuttcap%
\pgfsetroundjoin%
\definecolor{currentfill}{rgb}{0.000000,0.000000,0.000000}%
\pgfsetfillcolor{currentfill}%
\pgfsetlinewidth{1.003750pt}%
\definecolor{currentstroke}{rgb}{0.000000,0.000000,0.000000}%
\pgfsetstrokecolor{currentstroke}%
\pgfsetdash{}{0pt}%
\pgfpathmoveto{\pgfqpoint{2.145481in}{2.334333in}}%
\pgfpathcurveto{\pgfqpoint{2.156531in}{2.334333in}}{\pgfqpoint{2.167130in}{2.338724in}}{\pgfqpoint{2.174944in}{2.346537in}}%
\pgfpathcurveto{\pgfqpoint{2.182758in}{2.354351in}}{\pgfqpoint{2.187148in}{2.364950in}}{\pgfqpoint{2.187148in}{2.376000in}}%
\pgfpathcurveto{\pgfqpoint{2.187148in}{2.387050in}}{\pgfqpoint{2.182758in}{2.397649in}}{\pgfqpoint{2.174944in}{2.405463in}}%
\pgfpathcurveto{\pgfqpoint{2.167130in}{2.413276in}}{\pgfqpoint{2.156531in}{2.417667in}}{\pgfqpoint{2.145481in}{2.417667in}}%
\pgfpathcurveto{\pgfqpoint{2.134431in}{2.417667in}}{\pgfqpoint{2.123832in}{2.413276in}}{\pgfqpoint{2.116018in}{2.405463in}}%
\pgfpathcurveto{\pgfqpoint{2.108205in}{2.397649in}}{\pgfqpoint{2.103815in}{2.387050in}}{\pgfqpoint{2.103815in}{2.376000in}}%
\pgfpathcurveto{\pgfqpoint{2.103815in}{2.364950in}}{\pgfqpoint{2.108205in}{2.354351in}}{\pgfqpoint{2.116018in}{2.346537in}}%
\pgfpathcurveto{\pgfqpoint{2.123832in}{2.338724in}}{\pgfqpoint{2.134431in}{2.334333in}}{\pgfqpoint{2.145481in}{2.334333in}}%
\pgfpathclose%
\pgfusepath{stroke,fill}%
\end{pgfscope}%
\begin{pgfscope}%
\pgfpathrectangle{\pgfqpoint{0.800000in}{0.528000in}}{\pgfqpoint{4.960000in}{3.696000in}}%
\pgfusepath{clip}%
\pgfsetbuttcap%
\pgfsetroundjoin%
\definecolor{currentfill}{rgb}{0.000000,0.000000,0.000000}%
\pgfsetfillcolor{currentfill}%
\pgfsetlinewidth{1.003750pt}%
\definecolor{currentstroke}{rgb}{0.000000,0.000000,0.000000}%
\pgfsetstrokecolor{currentstroke}%
\pgfsetdash{}{0pt}%
\pgfpathmoveto{\pgfqpoint{2.145481in}{2.334333in}}%
\pgfpathcurveto{\pgfqpoint{2.156531in}{2.334333in}}{\pgfqpoint{2.167130in}{2.338724in}}{\pgfqpoint{2.174944in}{2.346537in}}%
\pgfpathcurveto{\pgfqpoint{2.182758in}{2.354351in}}{\pgfqpoint{2.187148in}{2.364950in}}{\pgfqpoint{2.187148in}{2.376000in}}%
\pgfpathcurveto{\pgfqpoint{2.187148in}{2.387050in}}{\pgfqpoint{2.182758in}{2.397649in}}{\pgfqpoint{2.174944in}{2.405463in}}%
\pgfpathcurveto{\pgfqpoint{2.167130in}{2.413276in}}{\pgfqpoint{2.156531in}{2.417667in}}{\pgfqpoint{2.145481in}{2.417667in}}%
\pgfpathcurveto{\pgfqpoint{2.134431in}{2.417667in}}{\pgfqpoint{2.123832in}{2.413276in}}{\pgfqpoint{2.116018in}{2.405463in}}%
\pgfpathcurveto{\pgfqpoint{2.108205in}{2.397649in}}{\pgfqpoint{2.103815in}{2.387050in}}{\pgfqpoint{2.103815in}{2.376000in}}%
\pgfpathcurveto{\pgfqpoint{2.103815in}{2.364950in}}{\pgfqpoint{2.108205in}{2.354351in}}{\pgfqpoint{2.116018in}{2.346537in}}%
\pgfpathcurveto{\pgfqpoint{2.123832in}{2.338724in}}{\pgfqpoint{2.134431in}{2.334333in}}{\pgfqpoint{2.145481in}{2.334333in}}%
\pgfpathclose%
\pgfusepath{stroke,fill}%
\end{pgfscope}%
\begin{pgfscope}%
\pgfpathrectangle{\pgfqpoint{0.800000in}{0.528000in}}{\pgfqpoint{4.960000in}{3.696000in}}%
\pgfusepath{clip}%
\pgfsetbuttcap%
\pgfsetroundjoin%
\definecolor{currentfill}{rgb}{0.000000,0.000000,0.000000}%
\pgfsetfillcolor{currentfill}%
\pgfsetlinewidth{1.003750pt}%
\definecolor{currentstroke}{rgb}{0.000000,0.000000,0.000000}%
\pgfsetstrokecolor{currentstroke}%
\pgfsetdash{}{0pt}%
\pgfpathmoveto{\pgfqpoint{2.145481in}{2.334333in}}%
\pgfpathcurveto{\pgfqpoint{2.156531in}{2.334333in}}{\pgfqpoint{2.167130in}{2.338724in}}{\pgfqpoint{2.174944in}{2.346537in}}%
\pgfpathcurveto{\pgfqpoint{2.182758in}{2.354351in}}{\pgfqpoint{2.187148in}{2.364950in}}{\pgfqpoint{2.187148in}{2.376000in}}%
\pgfpathcurveto{\pgfqpoint{2.187148in}{2.387050in}}{\pgfqpoint{2.182758in}{2.397649in}}{\pgfqpoint{2.174944in}{2.405463in}}%
\pgfpathcurveto{\pgfqpoint{2.167130in}{2.413276in}}{\pgfqpoint{2.156531in}{2.417667in}}{\pgfqpoint{2.145481in}{2.417667in}}%
\pgfpathcurveto{\pgfqpoint{2.134431in}{2.417667in}}{\pgfqpoint{2.123832in}{2.413276in}}{\pgfqpoint{2.116018in}{2.405463in}}%
\pgfpathcurveto{\pgfqpoint{2.108205in}{2.397649in}}{\pgfqpoint{2.103815in}{2.387050in}}{\pgfqpoint{2.103815in}{2.376000in}}%
\pgfpathcurveto{\pgfqpoint{2.103815in}{2.364950in}}{\pgfqpoint{2.108205in}{2.354351in}}{\pgfqpoint{2.116018in}{2.346537in}}%
\pgfpathcurveto{\pgfqpoint{2.123832in}{2.338724in}}{\pgfqpoint{2.134431in}{2.334333in}}{\pgfqpoint{2.145481in}{2.334333in}}%
\pgfpathclose%
\pgfusepath{stroke,fill}%
\end{pgfscope}%
\begin{pgfscope}%
\pgfpathrectangle{\pgfqpoint{0.800000in}{0.528000in}}{\pgfqpoint{4.960000in}{3.696000in}}%
\pgfusepath{clip}%
\pgfsetbuttcap%
\pgfsetroundjoin%
\definecolor{currentfill}{rgb}{0.000000,0.000000,0.000000}%
\pgfsetfillcolor{currentfill}%
\pgfsetlinewidth{1.003750pt}%
\definecolor{currentstroke}{rgb}{0.000000,0.000000,0.000000}%
\pgfsetstrokecolor{currentstroke}%
\pgfsetdash{}{0pt}%
\pgfpathmoveto{\pgfqpoint{2.145481in}{2.334333in}}%
\pgfpathcurveto{\pgfqpoint{2.156531in}{2.334333in}}{\pgfqpoint{2.167130in}{2.338724in}}{\pgfqpoint{2.174944in}{2.346537in}}%
\pgfpathcurveto{\pgfqpoint{2.182758in}{2.354351in}}{\pgfqpoint{2.187148in}{2.364950in}}{\pgfqpoint{2.187148in}{2.376000in}}%
\pgfpathcurveto{\pgfqpoint{2.187148in}{2.387050in}}{\pgfqpoint{2.182758in}{2.397649in}}{\pgfqpoint{2.174944in}{2.405463in}}%
\pgfpathcurveto{\pgfqpoint{2.167130in}{2.413276in}}{\pgfqpoint{2.156531in}{2.417667in}}{\pgfqpoint{2.145481in}{2.417667in}}%
\pgfpathcurveto{\pgfqpoint{2.134431in}{2.417667in}}{\pgfqpoint{2.123832in}{2.413276in}}{\pgfqpoint{2.116018in}{2.405463in}}%
\pgfpathcurveto{\pgfqpoint{2.108205in}{2.397649in}}{\pgfqpoint{2.103815in}{2.387050in}}{\pgfqpoint{2.103815in}{2.376000in}}%
\pgfpathcurveto{\pgfqpoint{2.103815in}{2.364950in}}{\pgfqpoint{2.108205in}{2.354351in}}{\pgfqpoint{2.116018in}{2.346537in}}%
\pgfpathcurveto{\pgfqpoint{2.123832in}{2.338724in}}{\pgfqpoint{2.134431in}{2.334333in}}{\pgfqpoint{2.145481in}{2.334333in}}%
\pgfpathclose%
\pgfusepath{stroke,fill}%
\end{pgfscope}%
\begin{pgfscope}%
\pgfpathrectangle{\pgfqpoint{0.800000in}{0.528000in}}{\pgfqpoint{4.960000in}{3.696000in}}%
\pgfusepath{clip}%
\pgfsetbuttcap%
\pgfsetroundjoin%
\definecolor{currentfill}{rgb}{0.000000,0.000000,0.000000}%
\pgfsetfillcolor{currentfill}%
\pgfsetlinewidth{1.003750pt}%
\definecolor{currentstroke}{rgb}{0.000000,0.000000,0.000000}%
\pgfsetstrokecolor{currentstroke}%
\pgfsetdash{}{0pt}%
\pgfpathmoveto{\pgfqpoint{2.145481in}{2.334333in}}%
\pgfpathcurveto{\pgfqpoint{2.156531in}{2.334333in}}{\pgfqpoint{2.167130in}{2.338724in}}{\pgfqpoint{2.174944in}{2.346537in}}%
\pgfpathcurveto{\pgfqpoint{2.182758in}{2.354351in}}{\pgfqpoint{2.187148in}{2.364950in}}{\pgfqpoint{2.187148in}{2.376000in}}%
\pgfpathcurveto{\pgfqpoint{2.187148in}{2.387050in}}{\pgfqpoint{2.182758in}{2.397649in}}{\pgfqpoint{2.174944in}{2.405463in}}%
\pgfpathcurveto{\pgfqpoint{2.167130in}{2.413276in}}{\pgfqpoint{2.156531in}{2.417667in}}{\pgfqpoint{2.145481in}{2.417667in}}%
\pgfpathcurveto{\pgfqpoint{2.134431in}{2.417667in}}{\pgfqpoint{2.123832in}{2.413276in}}{\pgfqpoint{2.116018in}{2.405463in}}%
\pgfpathcurveto{\pgfqpoint{2.108205in}{2.397649in}}{\pgfqpoint{2.103815in}{2.387050in}}{\pgfqpoint{2.103815in}{2.376000in}}%
\pgfpathcurveto{\pgfqpoint{2.103815in}{2.364950in}}{\pgfqpoint{2.108205in}{2.354351in}}{\pgfqpoint{2.116018in}{2.346537in}}%
\pgfpathcurveto{\pgfqpoint{2.123832in}{2.338724in}}{\pgfqpoint{2.134431in}{2.334333in}}{\pgfqpoint{2.145481in}{2.334333in}}%
\pgfpathclose%
\pgfusepath{stroke,fill}%
\end{pgfscope}%
\begin{pgfscope}%
\pgfpathrectangle{\pgfqpoint{0.800000in}{0.528000in}}{\pgfqpoint{4.960000in}{3.696000in}}%
\pgfusepath{clip}%
\pgfsetbuttcap%
\pgfsetroundjoin%
\definecolor{currentfill}{rgb}{0.000000,0.000000,0.000000}%
\pgfsetfillcolor{currentfill}%
\pgfsetlinewidth{1.003750pt}%
\definecolor{currentstroke}{rgb}{0.000000,0.000000,0.000000}%
\pgfsetstrokecolor{currentstroke}%
\pgfsetdash{}{0pt}%
\pgfpathmoveto{\pgfqpoint{2.145481in}{2.334333in}}%
\pgfpathcurveto{\pgfqpoint{2.156531in}{2.334333in}}{\pgfqpoint{2.167130in}{2.338724in}}{\pgfqpoint{2.174944in}{2.346537in}}%
\pgfpathcurveto{\pgfqpoint{2.182758in}{2.354351in}}{\pgfqpoint{2.187148in}{2.364950in}}{\pgfqpoint{2.187148in}{2.376000in}}%
\pgfpathcurveto{\pgfqpoint{2.187148in}{2.387050in}}{\pgfqpoint{2.182758in}{2.397649in}}{\pgfqpoint{2.174944in}{2.405463in}}%
\pgfpathcurveto{\pgfqpoint{2.167130in}{2.413276in}}{\pgfqpoint{2.156531in}{2.417667in}}{\pgfqpoint{2.145481in}{2.417667in}}%
\pgfpathcurveto{\pgfqpoint{2.134431in}{2.417667in}}{\pgfqpoint{2.123832in}{2.413276in}}{\pgfqpoint{2.116018in}{2.405463in}}%
\pgfpathcurveto{\pgfqpoint{2.108205in}{2.397649in}}{\pgfqpoint{2.103815in}{2.387050in}}{\pgfqpoint{2.103815in}{2.376000in}}%
\pgfpathcurveto{\pgfqpoint{2.103815in}{2.364950in}}{\pgfqpoint{2.108205in}{2.354351in}}{\pgfqpoint{2.116018in}{2.346537in}}%
\pgfpathcurveto{\pgfqpoint{2.123832in}{2.338724in}}{\pgfqpoint{2.134431in}{2.334333in}}{\pgfqpoint{2.145481in}{2.334333in}}%
\pgfpathclose%
\pgfusepath{stroke,fill}%
\end{pgfscope}%
\begin{pgfscope}%
\pgfpathrectangle{\pgfqpoint{0.800000in}{0.528000in}}{\pgfqpoint{4.960000in}{3.696000in}}%
\pgfusepath{clip}%
\pgfsetbuttcap%
\pgfsetroundjoin%
\definecolor{currentfill}{rgb}{0.000000,0.000000,0.000000}%
\pgfsetfillcolor{currentfill}%
\pgfsetlinewidth{1.003750pt}%
\definecolor{currentstroke}{rgb}{0.000000,0.000000,0.000000}%
\pgfsetstrokecolor{currentstroke}%
\pgfsetdash{}{0pt}%
\pgfpathmoveto{\pgfqpoint{2.145481in}{2.334333in}}%
\pgfpathcurveto{\pgfqpoint{2.156531in}{2.334333in}}{\pgfqpoint{2.167130in}{2.338724in}}{\pgfqpoint{2.174944in}{2.346537in}}%
\pgfpathcurveto{\pgfqpoint{2.182758in}{2.354351in}}{\pgfqpoint{2.187148in}{2.364950in}}{\pgfqpoint{2.187148in}{2.376000in}}%
\pgfpathcurveto{\pgfqpoint{2.187148in}{2.387050in}}{\pgfqpoint{2.182758in}{2.397649in}}{\pgfqpoint{2.174944in}{2.405463in}}%
\pgfpathcurveto{\pgfqpoint{2.167130in}{2.413276in}}{\pgfqpoint{2.156531in}{2.417667in}}{\pgfqpoint{2.145481in}{2.417667in}}%
\pgfpathcurveto{\pgfqpoint{2.134431in}{2.417667in}}{\pgfqpoint{2.123832in}{2.413276in}}{\pgfqpoint{2.116018in}{2.405463in}}%
\pgfpathcurveto{\pgfqpoint{2.108205in}{2.397649in}}{\pgfqpoint{2.103815in}{2.387050in}}{\pgfqpoint{2.103815in}{2.376000in}}%
\pgfpathcurveto{\pgfqpoint{2.103815in}{2.364950in}}{\pgfqpoint{2.108205in}{2.354351in}}{\pgfqpoint{2.116018in}{2.346537in}}%
\pgfpathcurveto{\pgfqpoint{2.123832in}{2.338724in}}{\pgfqpoint{2.134431in}{2.334333in}}{\pgfqpoint{2.145481in}{2.334333in}}%
\pgfpathclose%
\pgfusepath{stroke,fill}%
\end{pgfscope}%
\begin{pgfscope}%
\pgfpathrectangle{\pgfqpoint{0.800000in}{0.528000in}}{\pgfqpoint{4.960000in}{3.696000in}}%
\pgfusepath{clip}%
\pgfsetbuttcap%
\pgfsetroundjoin%
\definecolor{currentfill}{rgb}{0.000000,0.000000,0.000000}%
\pgfsetfillcolor{currentfill}%
\pgfsetlinewidth{1.003750pt}%
\definecolor{currentstroke}{rgb}{0.000000,0.000000,0.000000}%
\pgfsetstrokecolor{currentstroke}%
\pgfsetdash{}{0pt}%
\pgfpathmoveto{\pgfqpoint{2.145481in}{2.334333in}}%
\pgfpathcurveto{\pgfqpoint{2.156531in}{2.334333in}}{\pgfqpoint{2.167130in}{2.338724in}}{\pgfqpoint{2.174944in}{2.346537in}}%
\pgfpathcurveto{\pgfqpoint{2.182758in}{2.354351in}}{\pgfqpoint{2.187148in}{2.364950in}}{\pgfqpoint{2.187148in}{2.376000in}}%
\pgfpathcurveto{\pgfqpoint{2.187148in}{2.387050in}}{\pgfqpoint{2.182758in}{2.397649in}}{\pgfqpoint{2.174944in}{2.405463in}}%
\pgfpathcurveto{\pgfqpoint{2.167130in}{2.413276in}}{\pgfqpoint{2.156531in}{2.417667in}}{\pgfqpoint{2.145481in}{2.417667in}}%
\pgfpathcurveto{\pgfqpoint{2.134431in}{2.417667in}}{\pgfqpoint{2.123832in}{2.413276in}}{\pgfqpoint{2.116018in}{2.405463in}}%
\pgfpathcurveto{\pgfqpoint{2.108205in}{2.397649in}}{\pgfqpoint{2.103815in}{2.387050in}}{\pgfqpoint{2.103815in}{2.376000in}}%
\pgfpathcurveto{\pgfqpoint{2.103815in}{2.364950in}}{\pgfqpoint{2.108205in}{2.354351in}}{\pgfqpoint{2.116018in}{2.346537in}}%
\pgfpathcurveto{\pgfqpoint{2.123832in}{2.338724in}}{\pgfqpoint{2.134431in}{2.334333in}}{\pgfqpoint{2.145481in}{2.334333in}}%
\pgfpathclose%
\pgfusepath{stroke,fill}%
\end{pgfscope}%
\begin{pgfscope}%
\pgfpathrectangle{\pgfqpoint{0.800000in}{0.528000in}}{\pgfqpoint{4.960000in}{3.696000in}}%
\pgfusepath{clip}%
\pgfsetbuttcap%
\pgfsetroundjoin%
\definecolor{currentfill}{rgb}{0.000000,0.000000,0.000000}%
\pgfsetfillcolor{currentfill}%
\pgfsetlinewidth{1.003750pt}%
\definecolor{currentstroke}{rgb}{0.000000,0.000000,0.000000}%
\pgfsetstrokecolor{currentstroke}%
\pgfsetdash{}{0pt}%
\pgfpathmoveto{\pgfqpoint{2.145481in}{2.334333in}}%
\pgfpathcurveto{\pgfqpoint{2.156531in}{2.334333in}}{\pgfqpoint{2.167130in}{2.338724in}}{\pgfqpoint{2.174944in}{2.346537in}}%
\pgfpathcurveto{\pgfqpoint{2.182758in}{2.354351in}}{\pgfqpoint{2.187148in}{2.364950in}}{\pgfqpoint{2.187148in}{2.376000in}}%
\pgfpathcurveto{\pgfqpoint{2.187148in}{2.387050in}}{\pgfqpoint{2.182758in}{2.397649in}}{\pgfqpoint{2.174944in}{2.405463in}}%
\pgfpathcurveto{\pgfqpoint{2.167130in}{2.413276in}}{\pgfqpoint{2.156531in}{2.417667in}}{\pgfqpoint{2.145481in}{2.417667in}}%
\pgfpathcurveto{\pgfqpoint{2.134431in}{2.417667in}}{\pgfqpoint{2.123832in}{2.413276in}}{\pgfqpoint{2.116018in}{2.405463in}}%
\pgfpathcurveto{\pgfqpoint{2.108205in}{2.397649in}}{\pgfqpoint{2.103815in}{2.387050in}}{\pgfqpoint{2.103815in}{2.376000in}}%
\pgfpathcurveto{\pgfqpoint{2.103815in}{2.364950in}}{\pgfqpoint{2.108205in}{2.354351in}}{\pgfqpoint{2.116018in}{2.346537in}}%
\pgfpathcurveto{\pgfqpoint{2.123832in}{2.338724in}}{\pgfqpoint{2.134431in}{2.334333in}}{\pgfqpoint{2.145481in}{2.334333in}}%
\pgfpathclose%
\pgfusepath{stroke,fill}%
\end{pgfscope}%
\begin{pgfscope}%
\pgfpathrectangle{\pgfqpoint{0.800000in}{0.528000in}}{\pgfqpoint{4.960000in}{3.696000in}}%
\pgfusepath{clip}%
\pgfsetbuttcap%
\pgfsetroundjoin%
\definecolor{currentfill}{rgb}{0.000000,0.000000,0.000000}%
\pgfsetfillcolor{currentfill}%
\pgfsetlinewidth{1.003750pt}%
\definecolor{currentstroke}{rgb}{0.000000,0.000000,0.000000}%
\pgfsetstrokecolor{currentstroke}%
\pgfsetdash{}{0pt}%
\pgfpathmoveto{\pgfqpoint{2.145481in}{2.334333in}}%
\pgfpathcurveto{\pgfqpoint{2.156531in}{2.334333in}}{\pgfqpoint{2.167130in}{2.338724in}}{\pgfqpoint{2.174944in}{2.346537in}}%
\pgfpathcurveto{\pgfqpoint{2.182758in}{2.354351in}}{\pgfqpoint{2.187148in}{2.364950in}}{\pgfqpoint{2.187148in}{2.376000in}}%
\pgfpathcurveto{\pgfqpoint{2.187148in}{2.387050in}}{\pgfqpoint{2.182758in}{2.397649in}}{\pgfqpoint{2.174944in}{2.405463in}}%
\pgfpathcurveto{\pgfqpoint{2.167130in}{2.413276in}}{\pgfqpoint{2.156531in}{2.417667in}}{\pgfqpoint{2.145481in}{2.417667in}}%
\pgfpathcurveto{\pgfqpoint{2.134431in}{2.417667in}}{\pgfqpoint{2.123832in}{2.413276in}}{\pgfqpoint{2.116018in}{2.405463in}}%
\pgfpathcurveto{\pgfqpoint{2.108205in}{2.397649in}}{\pgfqpoint{2.103815in}{2.387050in}}{\pgfqpoint{2.103815in}{2.376000in}}%
\pgfpathcurveto{\pgfqpoint{2.103815in}{2.364950in}}{\pgfqpoint{2.108205in}{2.354351in}}{\pgfqpoint{2.116018in}{2.346537in}}%
\pgfpathcurveto{\pgfqpoint{2.123832in}{2.338724in}}{\pgfqpoint{2.134431in}{2.334333in}}{\pgfqpoint{2.145481in}{2.334333in}}%
\pgfpathclose%
\pgfusepath{stroke,fill}%
\end{pgfscope}%
\begin{pgfscope}%
\pgfpathrectangle{\pgfqpoint{0.800000in}{0.528000in}}{\pgfqpoint{4.960000in}{3.696000in}}%
\pgfusepath{clip}%
\pgfsetbuttcap%
\pgfsetroundjoin%
\definecolor{currentfill}{rgb}{0.000000,0.000000,0.000000}%
\pgfsetfillcolor{currentfill}%
\pgfsetlinewidth{1.003750pt}%
\definecolor{currentstroke}{rgb}{0.000000,0.000000,0.000000}%
\pgfsetstrokecolor{currentstroke}%
\pgfsetdash{}{0pt}%
\pgfpathmoveto{\pgfqpoint{2.145481in}{2.334333in}}%
\pgfpathcurveto{\pgfqpoint{2.156531in}{2.334333in}}{\pgfqpoint{2.167130in}{2.338724in}}{\pgfqpoint{2.174944in}{2.346537in}}%
\pgfpathcurveto{\pgfqpoint{2.182758in}{2.354351in}}{\pgfqpoint{2.187148in}{2.364950in}}{\pgfqpoint{2.187148in}{2.376000in}}%
\pgfpathcurveto{\pgfqpoint{2.187148in}{2.387050in}}{\pgfqpoint{2.182758in}{2.397649in}}{\pgfqpoint{2.174944in}{2.405463in}}%
\pgfpathcurveto{\pgfqpoint{2.167130in}{2.413276in}}{\pgfqpoint{2.156531in}{2.417667in}}{\pgfqpoint{2.145481in}{2.417667in}}%
\pgfpathcurveto{\pgfqpoint{2.134431in}{2.417667in}}{\pgfqpoint{2.123832in}{2.413276in}}{\pgfqpoint{2.116018in}{2.405463in}}%
\pgfpathcurveto{\pgfqpoint{2.108205in}{2.397649in}}{\pgfqpoint{2.103815in}{2.387050in}}{\pgfqpoint{2.103815in}{2.376000in}}%
\pgfpathcurveto{\pgfqpoint{2.103815in}{2.364950in}}{\pgfqpoint{2.108205in}{2.354351in}}{\pgfqpoint{2.116018in}{2.346537in}}%
\pgfpathcurveto{\pgfqpoint{2.123832in}{2.338724in}}{\pgfqpoint{2.134431in}{2.334333in}}{\pgfqpoint{2.145481in}{2.334333in}}%
\pgfpathclose%
\pgfusepath{stroke,fill}%
\end{pgfscope}%
\begin{pgfscope}%
\pgfpathrectangle{\pgfqpoint{0.800000in}{0.528000in}}{\pgfqpoint{4.960000in}{3.696000in}}%
\pgfusepath{clip}%
\pgfsetbuttcap%
\pgfsetroundjoin%
\definecolor{currentfill}{rgb}{0.000000,0.000000,0.000000}%
\pgfsetfillcolor{currentfill}%
\pgfsetlinewidth{1.003750pt}%
\definecolor{currentstroke}{rgb}{0.000000,0.000000,0.000000}%
\pgfsetstrokecolor{currentstroke}%
\pgfsetdash{}{0pt}%
\pgfpathmoveto{\pgfqpoint{2.145481in}{2.334333in}}%
\pgfpathcurveto{\pgfqpoint{2.156531in}{2.334333in}}{\pgfqpoint{2.167130in}{2.338724in}}{\pgfqpoint{2.174944in}{2.346537in}}%
\pgfpathcurveto{\pgfqpoint{2.182758in}{2.354351in}}{\pgfqpoint{2.187148in}{2.364950in}}{\pgfqpoint{2.187148in}{2.376000in}}%
\pgfpathcurveto{\pgfqpoint{2.187148in}{2.387050in}}{\pgfqpoint{2.182758in}{2.397649in}}{\pgfqpoint{2.174944in}{2.405463in}}%
\pgfpathcurveto{\pgfqpoint{2.167130in}{2.413276in}}{\pgfqpoint{2.156531in}{2.417667in}}{\pgfqpoint{2.145481in}{2.417667in}}%
\pgfpathcurveto{\pgfqpoint{2.134431in}{2.417667in}}{\pgfqpoint{2.123832in}{2.413276in}}{\pgfqpoint{2.116018in}{2.405463in}}%
\pgfpathcurveto{\pgfqpoint{2.108205in}{2.397649in}}{\pgfqpoint{2.103815in}{2.387050in}}{\pgfqpoint{2.103815in}{2.376000in}}%
\pgfpathcurveto{\pgfqpoint{2.103815in}{2.364950in}}{\pgfqpoint{2.108205in}{2.354351in}}{\pgfqpoint{2.116018in}{2.346537in}}%
\pgfpathcurveto{\pgfqpoint{2.123832in}{2.338724in}}{\pgfqpoint{2.134431in}{2.334333in}}{\pgfqpoint{2.145481in}{2.334333in}}%
\pgfpathclose%
\pgfusepath{stroke,fill}%
\end{pgfscope}%
\begin{pgfscope}%
\pgfpathrectangle{\pgfqpoint{0.800000in}{0.528000in}}{\pgfqpoint{4.960000in}{3.696000in}}%
\pgfusepath{clip}%
\pgfsetbuttcap%
\pgfsetroundjoin%
\definecolor{currentfill}{rgb}{0.000000,0.000000,0.000000}%
\pgfsetfillcolor{currentfill}%
\pgfsetlinewidth{1.003750pt}%
\definecolor{currentstroke}{rgb}{0.000000,0.000000,0.000000}%
\pgfsetstrokecolor{currentstroke}%
\pgfsetdash{}{0pt}%
\pgfpathmoveto{\pgfqpoint{2.145481in}{2.334333in}}%
\pgfpathcurveto{\pgfqpoint{2.156531in}{2.334333in}}{\pgfqpoint{2.167130in}{2.338724in}}{\pgfqpoint{2.174944in}{2.346537in}}%
\pgfpathcurveto{\pgfqpoint{2.182758in}{2.354351in}}{\pgfqpoint{2.187148in}{2.364950in}}{\pgfqpoint{2.187148in}{2.376000in}}%
\pgfpathcurveto{\pgfqpoint{2.187148in}{2.387050in}}{\pgfqpoint{2.182758in}{2.397649in}}{\pgfqpoint{2.174944in}{2.405463in}}%
\pgfpathcurveto{\pgfqpoint{2.167130in}{2.413276in}}{\pgfqpoint{2.156531in}{2.417667in}}{\pgfqpoint{2.145481in}{2.417667in}}%
\pgfpathcurveto{\pgfqpoint{2.134431in}{2.417667in}}{\pgfqpoint{2.123832in}{2.413276in}}{\pgfqpoint{2.116018in}{2.405463in}}%
\pgfpathcurveto{\pgfqpoint{2.108205in}{2.397649in}}{\pgfqpoint{2.103815in}{2.387050in}}{\pgfqpoint{2.103815in}{2.376000in}}%
\pgfpathcurveto{\pgfqpoint{2.103815in}{2.364950in}}{\pgfqpoint{2.108205in}{2.354351in}}{\pgfqpoint{2.116018in}{2.346537in}}%
\pgfpathcurveto{\pgfqpoint{2.123832in}{2.338724in}}{\pgfqpoint{2.134431in}{2.334333in}}{\pgfqpoint{2.145481in}{2.334333in}}%
\pgfpathclose%
\pgfusepath{stroke,fill}%
\end{pgfscope}%
\begin{pgfscope}%
\pgfpathrectangle{\pgfqpoint{0.800000in}{0.528000in}}{\pgfqpoint{4.960000in}{3.696000in}}%
\pgfusepath{clip}%
\pgfsetbuttcap%
\pgfsetroundjoin%
\definecolor{currentfill}{rgb}{0.000000,0.000000,0.000000}%
\pgfsetfillcolor{currentfill}%
\pgfsetlinewidth{1.003750pt}%
\definecolor{currentstroke}{rgb}{0.000000,0.000000,0.000000}%
\pgfsetstrokecolor{currentstroke}%
\pgfsetdash{}{0pt}%
\pgfpathmoveto{\pgfqpoint{2.145481in}{2.334333in}}%
\pgfpathcurveto{\pgfqpoint{2.156531in}{2.334333in}}{\pgfqpoint{2.167130in}{2.338724in}}{\pgfqpoint{2.174944in}{2.346537in}}%
\pgfpathcurveto{\pgfqpoint{2.182758in}{2.354351in}}{\pgfqpoint{2.187148in}{2.364950in}}{\pgfqpoint{2.187148in}{2.376000in}}%
\pgfpathcurveto{\pgfqpoint{2.187148in}{2.387050in}}{\pgfqpoint{2.182758in}{2.397649in}}{\pgfqpoint{2.174944in}{2.405463in}}%
\pgfpathcurveto{\pgfqpoint{2.167130in}{2.413276in}}{\pgfqpoint{2.156531in}{2.417667in}}{\pgfqpoint{2.145481in}{2.417667in}}%
\pgfpathcurveto{\pgfqpoint{2.134431in}{2.417667in}}{\pgfqpoint{2.123832in}{2.413276in}}{\pgfqpoint{2.116018in}{2.405463in}}%
\pgfpathcurveto{\pgfqpoint{2.108205in}{2.397649in}}{\pgfqpoint{2.103815in}{2.387050in}}{\pgfqpoint{2.103815in}{2.376000in}}%
\pgfpathcurveto{\pgfqpoint{2.103815in}{2.364950in}}{\pgfqpoint{2.108205in}{2.354351in}}{\pgfqpoint{2.116018in}{2.346537in}}%
\pgfpathcurveto{\pgfqpoint{2.123832in}{2.338724in}}{\pgfqpoint{2.134431in}{2.334333in}}{\pgfqpoint{2.145481in}{2.334333in}}%
\pgfpathclose%
\pgfusepath{stroke,fill}%
\end{pgfscope}%
\begin{pgfscope}%
\pgfpathrectangle{\pgfqpoint{0.800000in}{0.528000in}}{\pgfqpoint{4.960000in}{3.696000in}}%
\pgfusepath{clip}%
\pgfsetbuttcap%
\pgfsetroundjoin%
\definecolor{currentfill}{rgb}{0.000000,0.000000,0.000000}%
\pgfsetfillcolor{currentfill}%
\pgfsetlinewidth{1.003750pt}%
\definecolor{currentstroke}{rgb}{0.000000,0.000000,0.000000}%
\pgfsetstrokecolor{currentstroke}%
\pgfsetdash{}{0pt}%
\pgfpathmoveto{\pgfqpoint{2.145481in}{2.334333in}}%
\pgfpathcurveto{\pgfqpoint{2.156531in}{2.334333in}}{\pgfqpoint{2.167130in}{2.338724in}}{\pgfqpoint{2.174944in}{2.346537in}}%
\pgfpathcurveto{\pgfqpoint{2.182758in}{2.354351in}}{\pgfqpoint{2.187148in}{2.364950in}}{\pgfqpoint{2.187148in}{2.376000in}}%
\pgfpathcurveto{\pgfqpoint{2.187148in}{2.387050in}}{\pgfqpoint{2.182758in}{2.397649in}}{\pgfqpoint{2.174944in}{2.405463in}}%
\pgfpathcurveto{\pgfqpoint{2.167130in}{2.413276in}}{\pgfqpoint{2.156531in}{2.417667in}}{\pgfqpoint{2.145481in}{2.417667in}}%
\pgfpathcurveto{\pgfqpoint{2.134431in}{2.417667in}}{\pgfqpoint{2.123832in}{2.413276in}}{\pgfqpoint{2.116018in}{2.405463in}}%
\pgfpathcurveto{\pgfqpoint{2.108205in}{2.397649in}}{\pgfqpoint{2.103815in}{2.387050in}}{\pgfqpoint{2.103815in}{2.376000in}}%
\pgfpathcurveto{\pgfqpoint{2.103815in}{2.364950in}}{\pgfqpoint{2.108205in}{2.354351in}}{\pgfqpoint{2.116018in}{2.346537in}}%
\pgfpathcurveto{\pgfqpoint{2.123832in}{2.338724in}}{\pgfqpoint{2.134431in}{2.334333in}}{\pgfqpoint{2.145481in}{2.334333in}}%
\pgfpathclose%
\pgfusepath{stroke,fill}%
\end{pgfscope}%
\begin{pgfscope}%
\pgfpathrectangle{\pgfqpoint{0.800000in}{0.528000in}}{\pgfqpoint{4.960000in}{3.696000in}}%
\pgfusepath{clip}%
\pgfsetbuttcap%
\pgfsetroundjoin%
\definecolor{currentfill}{rgb}{0.000000,0.000000,0.000000}%
\pgfsetfillcolor{currentfill}%
\pgfsetlinewidth{1.003750pt}%
\definecolor{currentstroke}{rgb}{0.000000,0.000000,0.000000}%
\pgfsetstrokecolor{currentstroke}%
\pgfsetdash{}{0pt}%
\pgfpathmoveto{\pgfqpoint{2.145481in}{2.334333in}}%
\pgfpathcurveto{\pgfqpoint{2.156531in}{2.334333in}}{\pgfqpoint{2.167130in}{2.338724in}}{\pgfqpoint{2.174944in}{2.346537in}}%
\pgfpathcurveto{\pgfqpoint{2.182758in}{2.354351in}}{\pgfqpoint{2.187148in}{2.364950in}}{\pgfqpoint{2.187148in}{2.376000in}}%
\pgfpathcurveto{\pgfqpoint{2.187148in}{2.387050in}}{\pgfqpoint{2.182758in}{2.397649in}}{\pgfqpoint{2.174944in}{2.405463in}}%
\pgfpathcurveto{\pgfqpoint{2.167130in}{2.413276in}}{\pgfqpoint{2.156531in}{2.417667in}}{\pgfqpoint{2.145481in}{2.417667in}}%
\pgfpathcurveto{\pgfqpoint{2.134431in}{2.417667in}}{\pgfqpoint{2.123832in}{2.413276in}}{\pgfqpoint{2.116018in}{2.405463in}}%
\pgfpathcurveto{\pgfqpoint{2.108205in}{2.397649in}}{\pgfqpoint{2.103815in}{2.387050in}}{\pgfqpoint{2.103815in}{2.376000in}}%
\pgfpathcurveto{\pgfqpoint{2.103815in}{2.364950in}}{\pgfqpoint{2.108205in}{2.354351in}}{\pgfqpoint{2.116018in}{2.346537in}}%
\pgfpathcurveto{\pgfqpoint{2.123832in}{2.338724in}}{\pgfqpoint{2.134431in}{2.334333in}}{\pgfqpoint{2.145481in}{2.334333in}}%
\pgfpathclose%
\pgfusepath{stroke,fill}%
\end{pgfscope}%
\begin{pgfscope}%
\pgfpathrectangle{\pgfqpoint{0.800000in}{0.528000in}}{\pgfqpoint{4.960000in}{3.696000in}}%
\pgfusepath{clip}%
\pgfsetbuttcap%
\pgfsetroundjoin%
\definecolor{currentfill}{rgb}{0.000000,0.000000,0.000000}%
\pgfsetfillcolor{currentfill}%
\pgfsetlinewidth{1.003750pt}%
\definecolor{currentstroke}{rgb}{0.000000,0.000000,0.000000}%
\pgfsetstrokecolor{currentstroke}%
\pgfsetdash{}{0pt}%
\pgfpathmoveto{\pgfqpoint{2.145481in}{2.334333in}}%
\pgfpathcurveto{\pgfqpoint{2.156531in}{2.334333in}}{\pgfqpoint{2.167130in}{2.338724in}}{\pgfqpoint{2.174944in}{2.346537in}}%
\pgfpathcurveto{\pgfqpoint{2.182758in}{2.354351in}}{\pgfqpoint{2.187148in}{2.364950in}}{\pgfqpoint{2.187148in}{2.376000in}}%
\pgfpathcurveto{\pgfqpoint{2.187148in}{2.387050in}}{\pgfqpoint{2.182758in}{2.397649in}}{\pgfqpoint{2.174944in}{2.405463in}}%
\pgfpathcurveto{\pgfqpoint{2.167130in}{2.413276in}}{\pgfqpoint{2.156531in}{2.417667in}}{\pgfqpoint{2.145481in}{2.417667in}}%
\pgfpathcurveto{\pgfqpoint{2.134431in}{2.417667in}}{\pgfqpoint{2.123832in}{2.413276in}}{\pgfqpoint{2.116018in}{2.405463in}}%
\pgfpathcurveto{\pgfqpoint{2.108205in}{2.397649in}}{\pgfqpoint{2.103815in}{2.387050in}}{\pgfqpoint{2.103815in}{2.376000in}}%
\pgfpathcurveto{\pgfqpoint{2.103815in}{2.364950in}}{\pgfqpoint{2.108205in}{2.354351in}}{\pgfqpoint{2.116018in}{2.346537in}}%
\pgfpathcurveto{\pgfqpoint{2.123832in}{2.338724in}}{\pgfqpoint{2.134431in}{2.334333in}}{\pgfqpoint{2.145481in}{2.334333in}}%
\pgfpathclose%
\pgfusepath{stroke,fill}%
\end{pgfscope}%
\begin{pgfscope}%
\pgfpathrectangle{\pgfqpoint{0.800000in}{0.528000in}}{\pgfqpoint{4.960000in}{3.696000in}}%
\pgfusepath{clip}%
\pgfsetbuttcap%
\pgfsetroundjoin%
\definecolor{currentfill}{rgb}{0.000000,0.000000,0.000000}%
\pgfsetfillcolor{currentfill}%
\pgfsetlinewidth{1.003750pt}%
\definecolor{currentstroke}{rgb}{0.000000,0.000000,0.000000}%
\pgfsetstrokecolor{currentstroke}%
\pgfsetdash{}{0pt}%
\pgfpathmoveto{\pgfqpoint{2.145481in}{2.334333in}}%
\pgfpathcurveto{\pgfqpoint{2.156531in}{2.334333in}}{\pgfqpoint{2.167130in}{2.338724in}}{\pgfqpoint{2.174944in}{2.346537in}}%
\pgfpathcurveto{\pgfqpoint{2.182758in}{2.354351in}}{\pgfqpoint{2.187148in}{2.364950in}}{\pgfqpoint{2.187148in}{2.376000in}}%
\pgfpathcurveto{\pgfqpoint{2.187148in}{2.387050in}}{\pgfqpoint{2.182758in}{2.397649in}}{\pgfqpoint{2.174944in}{2.405463in}}%
\pgfpathcurveto{\pgfqpoint{2.167130in}{2.413276in}}{\pgfqpoint{2.156531in}{2.417667in}}{\pgfqpoint{2.145481in}{2.417667in}}%
\pgfpathcurveto{\pgfqpoint{2.134431in}{2.417667in}}{\pgfqpoint{2.123832in}{2.413276in}}{\pgfqpoint{2.116018in}{2.405463in}}%
\pgfpathcurveto{\pgfqpoint{2.108205in}{2.397649in}}{\pgfqpoint{2.103815in}{2.387050in}}{\pgfqpoint{2.103815in}{2.376000in}}%
\pgfpathcurveto{\pgfqpoint{2.103815in}{2.364950in}}{\pgfqpoint{2.108205in}{2.354351in}}{\pgfqpoint{2.116018in}{2.346537in}}%
\pgfpathcurveto{\pgfqpoint{2.123832in}{2.338724in}}{\pgfqpoint{2.134431in}{2.334333in}}{\pgfqpoint{2.145481in}{2.334333in}}%
\pgfpathclose%
\pgfusepath{stroke,fill}%
\end{pgfscope}%
\begin{pgfscope}%
\pgfpathrectangle{\pgfqpoint{0.800000in}{0.528000in}}{\pgfqpoint{4.960000in}{3.696000in}}%
\pgfusepath{clip}%
\pgfsetbuttcap%
\pgfsetroundjoin%
\definecolor{currentfill}{rgb}{0.000000,0.000000,0.000000}%
\pgfsetfillcolor{currentfill}%
\pgfsetlinewidth{1.003750pt}%
\definecolor{currentstroke}{rgb}{0.000000,0.000000,0.000000}%
\pgfsetstrokecolor{currentstroke}%
\pgfsetdash{}{0pt}%
\pgfpathmoveto{\pgfqpoint{2.145481in}{2.334333in}}%
\pgfpathcurveto{\pgfqpoint{2.156531in}{2.334333in}}{\pgfqpoint{2.167130in}{2.338724in}}{\pgfqpoint{2.174944in}{2.346537in}}%
\pgfpathcurveto{\pgfqpoint{2.182758in}{2.354351in}}{\pgfqpoint{2.187148in}{2.364950in}}{\pgfqpoint{2.187148in}{2.376000in}}%
\pgfpathcurveto{\pgfqpoint{2.187148in}{2.387050in}}{\pgfqpoint{2.182758in}{2.397649in}}{\pgfqpoint{2.174944in}{2.405463in}}%
\pgfpathcurveto{\pgfqpoint{2.167130in}{2.413276in}}{\pgfqpoint{2.156531in}{2.417667in}}{\pgfqpoint{2.145481in}{2.417667in}}%
\pgfpathcurveto{\pgfqpoint{2.134431in}{2.417667in}}{\pgfqpoint{2.123832in}{2.413276in}}{\pgfqpoint{2.116018in}{2.405463in}}%
\pgfpathcurveto{\pgfqpoint{2.108205in}{2.397649in}}{\pgfqpoint{2.103815in}{2.387050in}}{\pgfqpoint{2.103815in}{2.376000in}}%
\pgfpathcurveto{\pgfqpoint{2.103815in}{2.364950in}}{\pgfqpoint{2.108205in}{2.354351in}}{\pgfqpoint{2.116018in}{2.346537in}}%
\pgfpathcurveto{\pgfqpoint{2.123832in}{2.338724in}}{\pgfqpoint{2.134431in}{2.334333in}}{\pgfqpoint{2.145481in}{2.334333in}}%
\pgfpathclose%
\pgfusepath{stroke,fill}%
\end{pgfscope}%
\begin{pgfscope}%
\pgfpathrectangle{\pgfqpoint{0.800000in}{0.528000in}}{\pgfqpoint{4.960000in}{3.696000in}}%
\pgfusepath{clip}%
\pgfsetbuttcap%
\pgfsetroundjoin%
\definecolor{currentfill}{rgb}{0.000000,0.000000,0.000000}%
\pgfsetfillcolor{currentfill}%
\pgfsetlinewidth{1.003750pt}%
\definecolor{currentstroke}{rgb}{0.000000,0.000000,0.000000}%
\pgfsetstrokecolor{currentstroke}%
\pgfsetdash{}{0pt}%
\pgfpathmoveto{\pgfqpoint{2.145481in}{2.334333in}}%
\pgfpathcurveto{\pgfqpoint{2.156531in}{2.334333in}}{\pgfqpoint{2.167130in}{2.338724in}}{\pgfqpoint{2.174944in}{2.346537in}}%
\pgfpathcurveto{\pgfqpoint{2.182758in}{2.354351in}}{\pgfqpoint{2.187148in}{2.364950in}}{\pgfqpoint{2.187148in}{2.376000in}}%
\pgfpathcurveto{\pgfqpoint{2.187148in}{2.387050in}}{\pgfqpoint{2.182758in}{2.397649in}}{\pgfqpoint{2.174944in}{2.405463in}}%
\pgfpathcurveto{\pgfqpoint{2.167130in}{2.413276in}}{\pgfqpoint{2.156531in}{2.417667in}}{\pgfqpoint{2.145481in}{2.417667in}}%
\pgfpathcurveto{\pgfqpoint{2.134431in}{2.417667in}}{\pgfqpoint{2.123832in}{2.413276in}}{\pgfqpoint{2.116018in}{2.405463in}}%
\pgfpathcurveto{\pgfqpoint{2.108205in}{2.397649in}}{\pgfqpoint{2.103815in}{2.387050in}}{\pgfqpoint{2.103815in}{2.376000in}}%
\pgfpathcurveto{\pgfqpoint{2.103815in}{2.364950in}}{\pgfqpoint{2.108205in}{2.354351in}}{\pgfqpoint{2.116018in}{2.346537in}}%
\pgfpathcurveto{\pgfqpoint{2.123832in}{2.338724in}}{\pgfqpoint{2.134431in}{2.334333in}}{\pgfqpoint{2.145481in}{2.334333in}}%
\pgfpathclose%
\pgfusepath{stroke,fill}%
\end{pgfscope}%
\begin{pgfscope}%
\pgfpathrectangle{\pgfqpoint{0.800000in}{0.528000in}}{\pgfqpoint{4.960000in}{3.696000in}}%
\pgfusepath{clip}%
\pgfsetbuttcap%
\pgfsetroundjoin%
\definecolor{currentfill}{rgb}{0.000000,0.000000,0.000000}%
\pgfsetfillcolor{currentfill}%
\pgfsetlinewidth{1.003750pt}%
\definecolor{currentstroke}{rgb}{0.000000,0.000000,0.000000}%
\pgfsetstrokecolor{currentstroke}%
\pgfsetdash{}{0pt}%
\pgfpathmoveto{\pgfqpoint{2.145481in}{2.334333in}}%
\pgfpathcurveto{\pgfqpoint{2.156531in}{2.334333in}}{\pgfqpoint{2.167130in}{2.338724in}}{\pgfqpoint{2.174944in}{2.346537in}}%
\pgfpathcurveto{\pgfqpoint{2.182758in}{2.354351in}}{\pgfqpoint{2.187148in}{2.364950in}}{\pgfqpoint{2.187148in}{2.376000in}}%
\pgfpathcurveto{\pgfqpoint{2.187148in}{2.387050in}}{\pgfqpoint{2.182758in}{2.397649in}}{\pgfqpoint{2.174944in}{2.405463in}}%
\pgfpathcurveto{\pgfqpoint{2.167130in}{2.413276in}}{\pgfqpoint{2.156531in}{2.417667in}}{\pgfqpoint{2.145481in}{2.417667in}}%
\pgfpathcurveto{\pgfqpoint{2.134431in}{2.417667in}}{\pgfqpoint{2.123832in}{2.413276in}}{\pgfqpoint{2.116018in}{2.405463in}}%
\pgfpathcurveto{\pgfqpoint{2.108205in}{2.397649in}}{\pgfqpoint{2.103815in}{2.387050in}}{\pgfqpoint{2.103815in}{2.376000in}}%
\pgfpathcurveto{\pgfqpoint{2.103815in}{2.364950in}}{\pgfqpoint{2.108205in}{2.354351in}}{\pgfqpoint{2.116018in}{2.346537in}}%
\pgfpathcurveto{\pgfqpoint{2.123832in}{2.338724in}}{\pgfqpoint{2.134431in}{2.334333in}}{\pgfqpoint{2.145481in}{2.334333in}}%
\pgfpathclose%
\pgfusepath{stroke,fill}%
\end{pgfscope}%
\begin{pgfscope}%
\pgfpathrectangle{\pgfqpoint{0.800000in}{0.528000in}}{\pgfqpoint{4.960000in}{3.696000in}}%
\pgfusepath{clip}%
\pgfsetbuttcap%
\pgfsetroundjoin%
\definecolor{currentfill}{rgb}{0.000000,0.000000,0.000000}%
\pgfsetfillcolor{currentfill}%
\pgfsetlinewidth{1.003750pt}%
\definecolor{currentstroke}{rgb}{0.000000,0.000000,0.000000}%
\pgfsetstrokecolor{currentstroke}%
\pgfsetdash{}{0pt}%
\pgfpathmoveto{\pgfqpoint{2.145481in}{2.334333in}}%
\pgfpathcurveto{\pgfqpoint{2.156531in}{2.334333in}}{\pgfqpoint{2.167130in}{2.338724in}}{\pgfqpoint{2.174944in}{2.346537in}}%
\pgfpathcurveto{\pgfqpoint{2.182758in}{2.354351in}}{\pgfqpoint{2.187148in}{2.364950in}}{\pgfqpoint{2.187148in}{2.376000in}}%
\pgfpathcurveto{\pgfqpoint{2.187148in}{2.387050in}}{\pgfqpoint{2.182758in}{2.397649in}}{\pgfqpoint{2.174944in}{2.405463in}}%
\pgfpathcurveto{\pgfqpoint{2.167130in}{2.413276in}}{\pgfqpoint{2.156531in}{2.417667in}}{\pgfqpoint{2.145481in}{2.417667in}}%
\pgfpathcurveto{\pgfqpoint{2.134431in}{2.417667in}}{\pgfqpoint{2.123832in}{2.413276in}}{\pgfqpoint{2.116018in}{2.405463in}}%
\pgfpathcurveto{\pgfqpoint{2.108205in}{2.397649in}}{\pgfqpoint{2.103815in}{2.387050in}}{\pgfqpoint{2.103815in}{2.376000in}}%
\pgfpathcurveto{\pgfqpoint{2.103815in}{2.364950in}}{\pgfqpoint{2.108205in}{2.354351in}}{\pgfqpoint{2.116018in}{2.346537in}}%
\pgfpathcurveto{\pgfqpoint{2.123832in}{2.338724in}}{\pgfqpoint{2.134431in}{2.334333in}}{\pgfqpoint{2.145481in}{2.334333in}}%
\pgfpathclose%
\pgfusepath{stroke,fill}%
\end{pgfscope}%
\begin{pgfscope}%
\pgfpathrectangle{\pgfqpoint{0.800000in}{0.528000in}}{\pgfqpoint{4.960000in}{3.696000in}}%
\pgfusepath{clip}%
\pgfsetbuttcap%
\pgfsetroundjoin%
\definecolor{currentfill}{rgb}{0.000000,0.000000,0.000000}%
\pgfsetfillcolor{currentfill}%
\pgfsetlinewidth{1.003750pt}%
\definecolor{currentstroke}{rgb}{0.000000,0.000000,0.000000}%
\pgfsetstrokecolor{currentstroke}%
\pgfsetdash{}{0pt}%
\pgfpathmoveto{\pgfqpoint{2.145481in}{2.334333in}}%
\pgfpathcurveto{\pgfqpoint{2.156531in}{2.334333in}}{\pgfqpoint{2.167130in}{2.338724in}}{\pgfqpoint{2.174944in}{2.346537in}}%
\pgfpathcurveto{\pgfqpoint{2.182758in}{2.354351in}}{\pgfqpoint{2.187148in}{2.364950in}}{\pgfqpoint{2.187148in}{2.376000in}}%
\pgfpathcurveto{\pgfqpoint{2.187148in}{2.387050in}}{\pgfqpoint{2.182758in}{2.397649in}}{\pgfqpoint{2.174944in}{2.405463in}}%
\pgfpathcurveto{\pgfqpoint{2.167130in}{2.413276in}}{\pgfqpoint{2.156531in}{2.417667in}}{\pgfqpoint{2.145481in}{2.417667in}}%
\pgfpathcurveto{\pgfqpoint{2.134431in}{2.417667in}}{\pgfqpoint{2.123832in}{2.413276in}}{\pgfqpoint{2.116018in}{2.405463in}}%
\pgfpathcurveto{\pgfqpoint{2.108205in}{2.397649in}}{\pgfqpoint{2.103815in}{2.387050in}}{\pgfqpoint{2.103815in}{2.376000in}}%
\pgfpathcurveto{\pgfqpoint{2.103815in}{2.364950in}}{\pgfqpoint{2.108205in}{2.354351in}}{\pgfqpoint{2.116018in}{2.346537in}}%
\pgfpathcurveto{\pgfqpoint{2.123832in}{2.338724in}}{\pgfqpoint{2.134431in}{2.334333in}}{\pgfqpoint{2.145481in}{2.334333in}}%
\pgfpathclose%
\pgfusepath{stroke,fill}%
\end{pgfscope}%
\begin{pgfscope}%
\pgfpathrectangle{\pgfqpoint{0.800000in}{0.528000in}}{\pgfqpoint{4.960000in}{3.696000in}}%
\pgfusepath{clip}%
\pgfsetbuttcap%
\pgfsetroundjoin%
\definecolor{currentfill}{rgb}{0.000000,0.000000,0.000000}%
\pgfsetfillcolor{currentfill}%
\pgfsetlinewidth{1.003750pt}%
\definecolor{currentstroke}{rgb}{0.000000,0.000000,0.000000}%
\pgfsetstrokecolor{currentstroke}%
\pgfsetdash{}{0pt}%
\pgfpathmoveto{\pgfqpoint{2.145481in}{2.334333in}}%
\pgfpathcurveto{\pgfqpoint{2.156531in}{2.334333in}}{\pgfqpoint{2.167130in}{2.338724in}}{\pgfqpoint{2.174944in}{2.346537in}}%
\pgfpathcurveto{\pgfqpoint{2.182758in}{2.354351in}}{\pgfqpoint{2.187148in}{2.364950in}}{\pgfqpoint{2.187148in}{2.376000in}}%
\pgfpathcurveto{\pgfqpoint{2.187148in}{2.387050in}}{\pgfqpoint{2.182758in}{2.397649in}}{\pgfqpoint{2.174944in}{2.405463in}}%
\pgfpathcurveto{\pgfqpoint{2.167130in}{2.413276in}}{\pgfqpoint{2.156531in}{2.417667in}}{\pgfqpoint{2.145481in}{2.417667in}}%
\pgfpathcurveto{\pgfqpoint{2.134431in}{2.417667in}}{\pgfqpoint{2.123832in}{2.413276in}}{\pgfqpoint{2.116018in}{2.405463in}}%
\pgfpathcurveto{\pgfqpoint{2.108205in}{2.397649in}}{\pgfqpoint{2.103815in}{2.387050in}}{\pgfqpoint{2.103815in}{2.376000in}}%
\pgfpathcurveto{\pgfqpoint{2.103815in}{2.364950in}}{\pgfqpoint{2.108205in}{2.354351in}}{\pgfqpoint{2.116018in}{2.346537in}}%
\pgfpathcurveto{\pgfqpoint{2.123832in}{2.338724in}}{\pgfqpoint{2.134431in}{2.334333in}}{\pgfqpoint{2.145481in}{2.334333in}}%
\pgfpathclose%
\pgfusepath{stroke,fill}%
\end{pgfscope}%
\begin{pgfscope}%
\pgfpathrectangle{\pgfqpoint{0.800000in}{0.528000in}}{\pgfqpoint{4.960000in}{3.696000in}}%
\pgfusepath{clip}%
\pgfsetbuttcap%
\pgfsetroundjoin%
\definecolor{currentfill}{rgb}{0.000000,0.000000,0.000000}%
\pgfsetfillcolor{currentfill}%
\pgfsetlinewidth{1.003750pt}%
\definecolor{currentstroke}{rgb}{0.000000,0.000000,0.000000}%
\pgfsetstrokecolor{currentstroke}%
\pgfsetdash{}{0pt}%
\pgfpathmoveto{\pgfqpoint{2.145481in}{2.334333in}}%
\pgfpathcurveto{\pgfqpoint{2.156531in}{2.334333in}}{\pgfqpoint{2.167130in}{2.338724in}}{\pgfqpoint{2.174944in}{2.346537in}}%
\pgfpathcurveto{\pgfqpoint{2.182758in}{2.354351in}}{\pgfqpoint{2.187148in}{2.364950in}}{\pgfqpoint{2.187148in}{2.376000in}}%
\pgfpathcurveto{\pgfqpoint{2.187148in}{2.387050in}}{\pgfqpoint{2.182758in}{2.397649in}}{\pgfqpoint{2.174944in}{2.405463in}}%
\pgfpathcurveto{\pgfqpoint{2.167130in}{2.413276in}}{\pgfqpoint{2.156531in}{2.417667in}}{\pgfqpoint{2.145481in}{2.417667in}}%
\pgfpathcurveto{\pgfqpoint{2.134431in}{2.417667in}}{\pgfqpoint{2.123832in}{2.413276in}}{\pgfqpoint{2.116018in}{2.405463in}}%
\pgfpathcurveto{\pgfqpoint{2.108205in}{2.397649in}}{\pgfqpoint{2.103815in}{2.387050in}}{\pgfqpoint{2.103815in}{2.376000in}}%
\pgfpathcurveto{\pgfqpoint{2.103815in}{2.364950in}}{\pgfqpoint{2.108205in}{2.354351in}}{\pgfqpoint{2.116018in}{2.346537in}}%
\pgfpathcurveto{\pgfqpoint{2.123832in}{2.338724in}}{\pgfqpoint{2.134431in}{2.334333in}}{\pgfqpoint{2.145481in}{2.334333in}}%
\pgfpathclose%
\pgfusepath{stroke,fill}%
\end{pgfscope}%
\begin{pgfscope}%
\pgfpathrectangle{\pgfqpoint{0.800000in}{0.528000in}}{\pgfqpoint{4.960000in}{3.696000in}}%
\pgfusepath{clip}%
\pgfsetbuttcap%
\pgfsetroundjoin%
\definecolor{currentfill}{rgb}{0.000000,0.000000,0.000000}%
\pgfsetfillcolor{currentfill}%
\pgfsetlinewidth{1.003750pt}%
\definecolor{currentstroke}{rgb}{0.000000,0.000000,0.000000}%
\pgfsetstrokecolor{currentstroke}%
\pgfsetdash{}{0pt}%
\pgfpathmoveto{\pgfqpoint{2.145481in}{2.334333in}}%
\pgfpathcurveto{\pgfqpoint{2.156531in}{2.334333in}}{\pgfqpoint{2.167130in}{2.338724in}}{\pgfqpoint{2.174944in}{2.346537in}}%
\pgfpathcurveto{\pgfqpoint{2.182758in}{2.354351in}}{\pgfqpoint{2.187148in}{2.364950in}}{\pgfqpoint{2.187148in}{2.376000in}}%
\pgfpathcurveto{\pgfqpoint{2.187148in}{2.387050in}}{\pgfqpoint{2.182758in}{2.397649in}}{\pgfqpoint{2.174944in}{2.405463in}}%
\pgfpathcurveto{\pgfqpoint{2.167130in}{2.413276in}}{\pgfqpoint{2.156531in}{2.417667in}}{\pgfqpoint{2.145481in}{2.417667in}}%
\pgfpathcurveto{\pgfqpoint{2.134431in}{2.417667in}}{\pgfqpoint{2.123832in}{2.413276in}}{\pgfqpoint{2.116018in}{2.405463in}}%
\pgfpathcurveto{\pgfqpoint{2.108205in}{2.397649in}}{\pgfqpoint{2.103815in}{2.387050in}}{\pgfqpoint{2.103815in}{2.376000in}}%
\pgfpathcurveto{\pgfqpoint{2.103815in}{2.364950in}}{\pgfqpoint{2.108205in}{2.354351in}}{\pgfqpoint{2.116018in}{2.346537in}}%
\pgfpathcurveto{\pgfqpoint{2.123832in}{2.338724in}}{\pgfqpoint{2.134431in}{2.334333in}}{\pgfqpoint{2.145481in}{2.334333in}}%
\pgfpathclose%
\pgfusepath{stroke,fill}%
\end{pgfscope}%
\begin{pgfscope}%
\pgfpathrectangle{\pgfqpoint{0.800000in}{0.528000in}}{\pgfqpoint{4.960000in}{3.696000in}}%
\pgfusepath{clip}%
\pgfsetbuttcap%
\pgfsetroundjoin%
\definecolor{currentfill}{rgb}{0.000000,0.000000,0.000000}%
\pgfsetfillcolor{currentfill}%
\pgfsetlinewidth{1.003750pt}%
\definecolor{currentstroke}{rgb}{0.000000,0.000000,0.000000}%
\pgfsetstrokecolor{currentstroke}%
\pgfsetdash{}{0pt}%
\pgfpathmoveto{\pgfqpoint{2.145481in}{2.334333in}}%
\pgfpathcurveto{\pgfqpoint{2.156531in}{2.334333in}}{\pgfqpoint{2.167130in}{2.338724in}}{\pgfqpoint{2.174944in}{2.346537in}}%
\pgfpathcurveto{\pgfqpoint{2.182758in}{2.354351in}}{\pgfqpoint{2.187148in}{2.364950in}}{\pgfqpoint{2.187148in}{2.376000in}}%
\pgfpathcurveto{\pgfqpoint{2.187148in}{2.387050in}}{\pgfqpoint{2.182758in}{2.397649in}}{\pgfqpoint{2.174944in}{2.405463in}}%
\pgfpathcurveto{\pgfqpoint{2.167130in}{2.413276in}}{\pgfqpoint{2.156531in}{2.417667in}}{\pgfqpoint{2.145481in}{2.417667in}}%
\pgfpathcurveto{\pgfqpoint{2.134431in}{2.417667in}}{\pgfqpoint{2.123832in}{2.413276in}}{\pgfqpoint{2.116018in}{2.405463in}}%
\pgfpathcurveto{\pgfqpoint{2.108205in}{2.397649in}}{\pgfqpoint{2.103815in}{2.387050in}}{\pgfqpoint{2.103815in}{2.376000in}}%
\pgfpathcurveto{\pgfqpoint{2.103815in}{2.364950in}}{\pgfqpoint{2.108205in}{2.354351in}}{\pgfqpoint{2.116018in}{2.346537in}}%
\pgfpathcurveto{\pgfqpoint{2.123832in}{2.338724in}}{\pgfqpoint{2.134431in}{2.334333in}}{\pgfqpoint{2.145481in}{2.334333in}}%
\pgfpathclose%
\pgfusepath{stroke,fill}%
\end{pgfscope}%
\begin{pgfscope}%
\pgfpathrectangle{\pgfqpoint{0.800000in}{0.528000in}}{\pgfqpoint{4.960000in}{3.696000in}}%
\pgfusepath{clip}%
\pgfsetbuttcap%
\pgfsetroundjoin%
\definecolor{currentfill}{rgb}{0.000000,0.000000,0.000000}%
\pgfsetfillcolor{currentfill}%
\pgfsetlinewidth{1.003750pt}%
\definecolor{currentstroke}{rgb}{0.000000,0.000000,0.000000}%
\pgfsetstrokecolor{currentstroke}%
\pgfsetdash{}{0pt}%
\pgfpathmoveto{\pgfqpoint{2.145481in}{2.334333in}}%
\pgfpathcurveto{\pgfqpoint{2.156531in}{2.334333in}}{\pgfqpoint{2.167130in}{2.338724in}}{\pgfqpoint{2.174944in}{2.346537in}}%
\pgfpathcurveto{\pgfqpoint{2.182758in}{2.354351in}}{\pgfqpoint{2.187148in}{2.364950in}}{\pgfqpoint{2.187148in}{2.376000in}}%
\pgfpathcurveto{\pgfqpoint{2.187148in}{2.387050in}}{\pgfqpoint{2.182758in}{2.397649in}}{\pgfqpoint{2.174944in}{2.405463in}}%
\pgfpathcurveto{\pgfqpoint{2.167130in}{2.413276in}}{\pgfqpoint{2.156531in}{2.417667in}}{\pgfqpoint{2.145481in}{2.417667in}}%
\pgfpathcurveto{\pgfqpoint{2.134431in}{2.417667in}}{\pgfqpoint{2.123832in}{2.413276in}}{\pgfqpoint{2.116018in}{2.405463in}}%
\pgfpathcurveto{\pgfqpoint{2.108205in}{2.397649in}}{\pgfqpoint{2.103815in}{2.387050in}}{\pgfqpoint{2.103815in}{2.376000in}}%
\pgfpathcurveto{\pgfqpoint{2.103815in}{2.364950in}}{\pgfqpoint{2.108205in}{2.354351in}}{\pgfqpoint{2.116018in}{2.346537in}}%
\pgfpathcurveto{\pgfqpoint{2.123832in}{2.338724in}}{\pgfqpoint{2.134431in}{2.334333in}}{\pgfqpoint{2.145481in}{2.334333in}}%
\pgfpathclose%
\pgfusepath{stroke,fill}%
\end{pgfscope}%
\begin{pgfscope}%
\pgfpathrectangle{\pgfqpoint{0.800000in}{0.528000in}}{\pgfqpoint{4.960000in}{3.696000in}}%
\pgfusepath{clip}%
\pgfsetbuttcap%
\pgfsetroundjoin%
\definecolor{currentfill}{rgb}{0.000000,0.000000,0.000000}%
\pgfsetfillcolor{currentfill}%
\pgfsetlinewidth{1.003750pt}%
\definecolor{currentstroke}{rgb}{0.000000,0.000000,0.000000}%
\pgfsetstrokecolor{currentstroke}%
\pgfsetdash{}{0pt}%
\pgfpathmoveto{\pgfqpoint{2.145481in}{2.334333in}}%
\pgfpathcurveto{\pgfqpoint{2.156531in}{2.334333in}}{\pgfqpoint{2.167130in}{2.338724in}}{\pgfqpoint{2.174944in}{2.346537in}}%
\pgfpathcurveto{\pgfqpoint{2.182758in}{2.354351in}}{\pgfqpoint{2.187148in}{2.364950in}}{\pgfqpoint{2.187148in}{2.376000in}}%
\pgfpathcurveto{\pgfqpoint{2.187148in}{2.387050in}}{\pgfqpoint{2.182758in}{2.397649in}}{\pgfqpoint{2.174944in}{2.405463in}}%
\pgfpathcurveto{\pgfqpoint{2.167130in}{2.413276in}}{\pgfqpoint{2.156531in}{2.417667in}}{\pgfqpoint{2.145481in}{2.417667in}}%
\pgfpathcurveto{\pgfqpoint{2.134431in}{2.417667in}}{\pgfqpoint{2.123832in}{2.413276in}}{\pgfqpoint{2.116018in}{2.405463in}}%
\pgfpathcurveto{\pgfqpoint{2.108205in}{2.397649in}}{\pgfqpoint{2.103815in}{2.387050in}}{\pgfqpoint{2.103815in}{2.376000in}}%
\pgfpathcurveto{\pgfqpoint{2.103815in}{2.364950in}}{\pgfqpoint{2.108205in}{2.354351in}}{\pgfqpoint{2.116018in}{2.346537in}}%
\pgfpathcurveto{\pgfqpoint{2.123832in}{2.338724in}}{\pgfqpoint{2.134431in}{2.334333in}}{\pgfqpoint{2.145481in}{2.334333in}}%
\pgfpathclose%
\pgfusepath{stroke,fill}%
\end{pgfscope}%
\begin{pgfscope}%
\pgfpathrectangle{\pgfqpoint{0.800000in}{0.528000in}}{\pgfqpoint{4.960000in}{3.696000in}}%
\pgfusepath{clip}%
\pgfsetbuttcap%
\pgfsetroundjoin%
\definecolor{currentfill}{rgb}{0.000000,0.000000,0.000000}%
\pgfsetfillcolor{currentfill}%
\pgfsetlinewidth{1.003750pt}%
\definecolor{currentstroke}{rgb}{0.000000,0.000000,0.000000}%
\pgfsetstrokecolor{currentstroke}%
\pgfsetdash{}{0pt}%
\pgfpathmoveto{\pgfqpoint{2.145481in}{2.334333in}}%
\pgfpathcurveto{\pgfqpoint{2.156531in}{2.334333in}}{\pgfqpoint{2.167130in}{2.338724in}}{\pgfqpoint{2.174944in}{2.346537in}}%
\pgfpathcurveto{\pgfqpoint{2.182758in}{2.354351in}}{\pgfqpoint{2.187148in}{2.364950in}}{\pgfqpoint{2.187148in}{2.376000in}}%
\pgfpathcurveto{\pgfqpoint{2.187148in}{2.387050in}}{\pgfqpoint{2.182758in}{2.397649in}}{\pgfqpoint{2.174944in}{2.405463in}}%
\pgfpathcurveto{\pgfqpoint{2.167130in}{2.413276in}}{\pgfqpoint{2.156531in}{2.417667in}}{\pgfqpoint{2.145481in}{2.417667in}}%
\pgfpathcurveto{\pgfqpoint{2.134431in}{2.417667in}}{\pgfqpoint{2.123832in}{2.413276in}}{\pgfqpoint{2.116018in}{2.405463in}}%
\pgfpathcurveto{\pgfqpoint{2.108205in}{2.397649in}}{\pgfqpoint{2.103815in}{2.387050in}}{\pgfqpoint{2.103815in}{2.376000in}}%
\pgfpathcurveto{\pgfqpoint{2.103815in}{2.364950in}}{\pgfqpoint{2.108205in}{2.354351in}}{\pgfqpoint{2.116018in}{2.346537in}}%
\pgfpathcurveto{\pgfqpoint{2.123832in}{2.338724in}}{\pgfqpoint{2.134431in}{2.334333in}}{\pgfqpoint{2.145481in}{2.334333in}}%
\pgfpathclose%
\pgfusepath{stroke,fill}%
\end{pgfscope}%
\begin{pgfscope}%
\pgfpathrectangle{\pgfqpoint{0.800000in}{0.528000in}}{\pgfqpoint{4.960000in}{3.696000in}}%
\pgfusepath{clip}%
\pgfsetbuttcap%
\pgfsetroundjoin%
\definecolor{currentfill}{rgb}{0.000000,0.000000,0.000000}%
\pgfsetfillcolor{currentfill}%
\pgfsetlinewidth{1.003750pt}%
\definecolor{currentstroke}{rgb}{0.000000,0.000000,0.000000}%
\pgfsetstrokecolor{currentstroke}%
\pgfsetdash{}{0pt}%
\pgfpathmoveto{\pgfqpoint{2.145481in}{2.334333in}}%
\pgfpathcurveto{\pgfqpoint{2.156531in}{2.334333in}}{\pgfqpoint{2.167130in}{2.338724in}}{\pgfqpoint{2.174944in}{2.346537in}}%
\pgfpathcurveto{\pgfqpoint{2.182758in}{2.354351in}}{\pgfqpoint{2.187148in}{2.364950in}}{\pgfqpoint{2.187148in}{2.376000in}}%
\pgfpathcurveto{\pgfqpoint{2.187148in}{2.387050in}}{\pgfqpoint{2.182758in}{2.397649in}}{\pgfqpoint{2.174944in}{2.405463in}}%
\pgfpathcurveto{\pgfqpoint{2.167130in}{2.413276in}}{\pgfqpoint{2.156531in}{2.417667in}}{\pgfqpoint{2.145481in}{2.417667in}}%
\pgfpathcurveto{\pgfqpoint{2.134431in}{2.417667in}}{\pgfqpoint{2.123832in}{2.413276in}}{\pgfqpoint{2.116018in}{2.405463in}}%
\pgfpathcurveto{\pgfqpoint{2.108205in}{2.397649in}}{\pgfqpoint{2.103815in}{2.387050in}}{\pgfqpoint{2.103815in}{2.376000in}}%
\pgfpathcurveto{\pgfqpoint{2.103815in}{2.364950in}}{\pgfqpoint{2.108205in}{2.354351in}}{\pgfqpoint{2.116018in}{2.346537in}}%
\pgfpathcurveto{\pgfqpoint{2.123832in}{2.338724in}}{\pgfqpoint{2.134431in}{2.334333in}}{\pgfqpoint{2.145481in}{2.334333in}}%
\pgfpathclose%
\pgfusepath{stroke,fill}%
\end{pgfscope}%
\begin{pgfscope}%
\pgfpathrectangle{\pgfqpoint{0.800000in}{0.528000in}}{\pgfqpoint{4.960000in}{3.696000in}}%
\pgfusepath{clip}%
\pgfsetbuttcap%
\pgfsetroundjoin%
\definecolor{currentfill}{rgb}{0.000000,0.000000,0.000000}%
\pgfsetfillcolor{currentfill}%
\pgfsetlinewidth{1.003750pt}%
\definecolor{currentstroke}{rgb}{0.000000,0.000000,0.000000}%
\pgfsetstrokecolor{currentstroke}%
\pgfsetdash{}{0pt}%
\pgfpathmoveto{\pgfqpoint{2.145481in}{2.334333in}}%
\pgfpathcurveto{\pgfqpoint{2.156531in}{2.334333in}}{\pgfqpoint{2.167130in}{2.338724in}}{\pgfqpoint{2.174944in}{2.346537in}}%
\pgfpathcurveto{\pgfqpoint{2.182758in}{2.354351in}}{\pgfqpoint{2.187148in}{2.364950in}}{\pgfqpoint{2.187148in}{2.376000in}}%
\pgfpathcurveto{\pgfqpoint{2.187148in}{2.387050in}}{\pgfqpoint{2.182758in}{2.397649in}}{\pgfqpoint{2.174944in}{2.405463in}}%
\pgfpathcurveto{\pgfqpoint{2.167130in}{2.413276in}}{\pgfqpoint{2.156531in}{2.417667in}}{\pgfqpoint{2.145481in}{2.417667in}}%
\pgfpathcurveto{\pgfqpoint{2.134431in}{2.417667in}}{\pgfqpoint{2.123832in}{2.413276in}}{\pgfqpoint{2.116018in}{2.405463in}}%
\pgfpathcurveto{\pgfqpoint{2.108205in}{2.397649in}}{\pgfqpoint{2.103815in}{2.387050in}}{\pgfqpoint{2.103815in}{2.376000in}}%
\pgfpathcurveto{\pgfqpoint{2.103815in}{2.364950in}}{\pgfqpoint{2.108205in}{2.354351in}}{\pgfqpoint{2.116018in}{2.346537in}}%
\pgfpathcurveto{\pgfqpoint{2.123832in}{2.338724in}}{\pgfqpoint{2.134431in}{2.334333in}}{\pgfqpoint{2.145481in}{2.334333in}}%
\pgfpathclose%
\pgfusepath{stroke,fill}%
\end{pgfscope}%
\begin{pgfscope}%
\pgfpathrectangle{\pgfqpoint{0.800000in}{0.528000in}}{\pgfqpoint{4.960000in}{3.696000in}}%
\pgfusepath{clip}%
\pgfsetbuttcap%
\pgfsetroundjoin%
\definecolor{currentfill}{rgb}{0.000000,0.000000,0.000000}%
\pgfsetfillcolor{currentfill}%
\pgfsetlinewidth{1.003750pt}%
\definecolor{currentstroke}{rgb}{0.000000,0.000000,0.000000}%
\pgfsetstrokecolor{currentstroke}%
\pgfsetdash{}{0pt}%
\pgfpathmoveto{\pgfqpoint{2.145481in}{2.334333in}}%
\pgfpathcurveto{\pgfqpoint{2.156531in}{2.334333in}}{\pgfqpoint{2.167130in}{2.338724in}}{\pgfqpoint{2.174944in}{2.346537in}}%
\pgfpathcurveto{\pgfqpoint{2.182758in}{2.354351in}}{\pgfqpoint{2.187148in}{2.364950in}}{\pgfqpoint{2.187148in}{2.376000in}}%
\pgfpathcurveto{\pgfqpoint{2.187148in}{2.387050in}}{\pgfqpoint{2.182758in}{2.397649in}}{\pgfqpoint{2.174944in}{2.405463in}}%
\pgfpathcurveto{\pgfqpoint{2.167130in}{2.413276in}}{\pgfqpoint{2.156531in}{2.417667in}}{\pgfqpoint{2.145481in}{2.417667in}}%
\pgfpathcurveto{\pgfqpoint{2.134431in}{2.417667in}}{\pgfqpoint{2.123832in}{2.413276in}}{\pgfqpoint{2.116018in}{2.405463in}}%
\pgfpathcurveto{\pgfqpoint{2.108205in}{2.397649in}}{\pgfqpoint{2.103815in}{2.387050in}}{\pgfqpoint{2.103815in}{2.376000in}}%
\pgfpathcurveto{\pgfqpoint{2.103815in}{2.364950in}}{\pgfqpoint{2.108205in}{2.354351in}}{\pgfqpoint{2.116018in}{2.346537in}}%
\pgfpathcurveto{\pgfqpoint{2.123832in}{2.338724in}}{\pgfqpoint{2.134431in}{2.334333in}}{\pgfqpoint{2.145481in}{2.334333in}}%
\pgfpathclose%
\pgfusepath{stroke,fill}%
\end{pgfscope}%
\begin{pgfscope}%
\pgfpathrectangle{\pgfqpoint{0.800000in}{0.528000in}}{\pgfqpoint{4.960000in}{3.696000in}}%
\pgfusepath{clip}%
\pgfsetbuttcap%
\pgfsetroundjoin%
\definecolor{currentfill}{rgb}{0.000000,0.000000,0.000000}%
\pgfsetfillcolor{currentfill}%
\pgfsetlinewidth{1.003750pt}%
\definecolor{currentstroke}{rgb}{0.000000,0.000000,0.000000}%
\pgfsetstrokecolor{currentstroke}%
\pgfsetdash{}{0pt}%
\pgfpathmoveto{\pgfqpoint{2.145481in}{2.334333in}}%
\pgfpathcurveto{\pgfqpoint{2.156531in}{2.334333in}}{\pgfqpoint{2.167130in}{2.338724in}}{\pgfqpoint{2.174944in}{2.346537in}}%
\pgfpathcurveto{\pgfqpoint{2.182758in}{2.354351in}}{\pgfqpoint{2.187148in}{2.364950in}}{\pgfqpoint{2.187148in}{2.376000in}}%
\pgfpathcurveto{\pgfqpoint{2.187148in}{2.387050in}}{\pgfqpoint{2.182758in}{2.397649in}}{\pgfqpoint{2.174944in}{2.405463in}}%
\pgfpathcurveto{\pgfqpoint{2.167130in}{2.413276in}}{\pgfqpoint{2.156531in}{2.417667in}}{\pgfqpoint{2.145481in}{2.417667in}}%
\pgfpathcurveto{\pgfqpoint{2.134431in}{2.417667in}}{\pgfqpoint{2.123832in}{2.413276in}}{\pgfqpoint{2.116018in}{2.405463in}}%
\pgfpathcurveto{\pgfqpoint{2.108205in}{2.397649in}}{\pgfqpoint{2.103815in}{2.387050in}}{\pgfqpoint{2.103815in}{2.376000in}}%
\pgfpathcurveto{\pgfqpoint{2.103815in}{2.364950in}}{\pgfqpoint{2.108205in}{2.354351in}}{\pgfqpoint{2.116018in}{2.346537in}}%
\pgfpathcurveto{\pgfqpoint{2.123832in}{2.338724in}}{\pgfqpoint{2.134431in}{2.334333in}}{\pgfqpoint{2.145481in}{2.334333in}}%
\pgfpathclose%
\pgfusepath{stroke,fill}%
\end{pgfscope}%
\begin{pgfscope}%
\pgfpathrectangle{\pgfqpoint{0.800000in}{0.528000in}}{\pgfqpoint{4.960000in}{3.696000in}}%
\pgfusepath{clip}%
\pgfsetbuttcap%
\pgfsetroundjoin%
\definecolor{currentfill}{rgb}{0.000000,0.000000,0.000000}%
\pgfsetfillcolor{currentfill}%
\pgfsetlinewidth{1.003750pt}%
\definecolor{currentstroke}{rgb}{0.000000,0.000000,0.000000}%
\pgfsetstrokecolor{currentstroke}%
\pgfsetdash{}{0pt}%
\pgfpathmoveto{\pgfqpoint{2.145481in}{2.334333in}}%
\pgfpathcurveto{\pgfqpoint{2.156531in}{2.334333in}}{\pgfqpoint{2.167130in}{2.338724in}}{\pgfqpoint{2.174944in}{2.346537in}}%
\pgfpathcurveto{\pgfqpoint{2.182758in}{2.354351in}}{\pgfqpoint{2.187148in}{2.364950in}}{\pgfqpoint{2.187148in}{2.376000in}}%
\pgfpathcurveto{\pgfqpoint{2.187148in}{2.387050in}}{\pgfqpoint{2.182758in}{2.397649in}}{\pgfqpoint{2.174944in}{2.405463in}}%
\pgfpathcurveto{\pgfqpoint{2.167130in}{2.413276in}}{\pgfqpoint{2.156531in}{2.417667in}}{\pgfqpoint{2.145481in}{2.417667in}}%
\pgfpathcurveto{\pgfqpoint{2.134431in}{2.417667in}}{\pgfqpoint{2.123832in}{2.413276in}}{\pgfqpoint{2.116018in}{2.405463in}}%
\pgfpathcurveto{\pgfqpoint{2.108205in}{2.397649in}}{\pgfqpoint{2.103815in}{2.387050in}}{\pgfqpoint{2.103815in}{2.376000in}}%
\pgfpathcurveto{\pgfqpoint{2.103815in}{2.364950in}}{\pgfqpoint{2.108205in}{2.354351in}}{\pgfqpoint{2.116018in}{2.346537in}}%
\pgfpathcurveto{\pgfqpoint{2.123832in}{2.338724in}}{\pgfqpoint{2.134431in}{2.334333in}}{\pgfqpoint{2.145481in}{2.334333in}}%
\pgfpathclose%
\pgfusepath{stroke,fill}%
\end{pgfscope}%
\begin{pgfscope}%
\pgfpathrectangle{\pgfqpoint{0.800000in}{0.528000in}}{\pgfqpoint{4.960000in}{3.696000in}}%
\pgfusepath{clip}%
\pgfsetbuttcap%
\pgfsetroundjoin%
\definecolor{currentfill}{rgb}{0.000000,0.000000,0.000000}%
\pgfsetfillcolor{currentfill}%
\pgfsetlinewidth{1.003750pt}%
\definecolor{currentstroke}{rgb}{0.000000,0.000000,0.000000}%
\pgfsetstrokecolor{currentstroke}%
\pgfsetdash{}{0pt}%
\pgfpathmoveto{\pgfqpoint{2.145481in}{2.334333in}}%
\pgfpathcurveto{\pgfqpoint{2.156531in}{2.334333in}}{\pgfqpoint{2.167130in}{2.338724in}}{\pgfqpoint{2.174944in}{2.346537in}}%
\pgfpathcurveto{\pgfqpoint{2.182758in}{2.354351in}}{\pgfqpoint{2.187148in}{2.364950in}}{\pgfqpoint{2.187148in}{2.376000in}}%
\pgfpathcurveto{\pgfqpoint{2.187148in}{2.387050in}}{\pgfqpoint{2.182758in}{2.397649in}}{\pgfqpoint{2.174944in}{2.405463in}}%
\pgfpathcurveto{\pgfqpoint{2.167130in}{2.413276in}}{\pgfqpoint{2.156531in}{2.417667in}}{\pgfqpoint{2.145481in}{2.417667in}}%
\pgfpathcurveto{\pgfqpoint{2.134431in}{2.417667in}}{\pgfqpoint{2.123832in}{2.413276in}}{\pgfqpoint{2.116018in}{2.405463in}}%
\pgfpathcurveto{\pgfqpoint{2.108205in}{2.397649in}}{\pgfqpoint{2.103815in}{2.387050in}}{\pgfqpoint{2.103815in}{2.376000in}}%
\pgfpathcurveto{\pgfqpoint{2.103815in}{2.364950in}}{\pgfqpoint{2.108205in}{2.354351in}}{\pgfqpoint{2.116018in}{2.346537in}}%
\pgfpathcurveto{\pgfqpoint{2.123832in}{2.338724in}}{\pgfqpoint{2.134431in}{2.334333in}}{\pgfqpoint{2.145481in}{2.334333in}}%
\pgfpathclose%
\pgfusepath{stroke,fill}%
\end{pgfscope}%
\begin{pgfscope}%
\pgfpathrectangle{\pgfqpoint{0.800000in}{0.528000in}}{\pgfqpoint{4.960000in}{3.696000in}}%
\pgfusepath{clip}%
\pgfsetbuttcap%
\pgfsetroundjoin%
\definecolor{currentfill}{rgb}{0.000000,0.000000,0.000000}%
\pgfsetfillcolor{currentfill}%
\pgfsetlinewidth{1.003750pt}%
\definecolor{currentstroke}{rgb}{0.000000,0.000000,0.000000}%
\pgfsetstrokecolor{currentstroke}%
\pgfsetdash{}{0pt}%
\pgfpathmoveto{\pgfqpoint{2.145481in}{2.334333in}}%
\pgfpathcurveto{\pgfqpoint{2.156531in}{2.334333in}}{\pgfqpoint{2.167130in}{2.338724in}}{\pgfqpoint{2.174944in}{2.346537in}}%
\pgfpathcurveto{\pgfqpoint{2.182758in}{2.354351in}}{\pgfqpoint{2.187148in}{2.364950in}}{\pgfqpoint{2.187148in}{2.376000in}}%
\pgfpathcurveto{\pgfqpoint{2.187148in}{2.387050in}}{\pgfqpoint{2.182758in}{2.397649in}}{\pgfqpoint{2.174944in}{2.405463in}}%
\pgfpathcurveto{\pgfqpoint{2.167130in}{2.413276in}}{\pgfqpoint{2.156531in}{2.417667in}}{\pgfqpoint{2.145481in}{2.417667in}}%
\pgfpathcurveto{\pgfqpoint{2.134431in}{2.417667in}}{\pgfqpoint{2.123832in}{2.413276in}}{\pgfqpoint{2.116018in}{2.405463in}}%
\pgfpathcurveto{\pgfqpoint{2.108205in}{2.397649in}}{\pgfqpoint{2.103815in}{2.387050in}}{\pgfqpoint{2.103815in}{2.376000in}}%
\pgfpathcurveto{\pgfqpoint{2.103815in}{2.364950in}}{\pgfqpoint{2.108205in}{2.354351in}}{\pgfqpoint{2.116018in}{2.346537in}}%
\pgfpathcurveto{\pgfqpoint{2.123832in}{2.338724in}}{\pgfqpoint{2.134431in}{2.334333in}}{\pgfqpoint{2.145481in}{2.334333in}}%
\pgfpathclose%
\pgfusepath{stroke,fill}%
\end{pgfscope}%
\begin{pgfscope}%
\pgfpathrectangle{\pgfqpoint{0.800000in}{0.528000in}}{\pgfqpoint{4.960000in}{3.696000in}}%
\pgfusepath{clip}%
\pgfsetbuttcap%
\pgfsetroundjoin%
\definecolor{currentfill}{rgb}{0.000000,0.000000,0.000000}%
\pgfsetfillcolor{currentfill}%
\pgfsetlinewidth{1.003750pt}%
\definecolor{currentstroke}{rgb}{0.000000,0.000000,0.000000}%
\pgfsetstrokecolor{currentstroke}%
\pgfsetdash{}{0pt}%
\pgfpathmoveto{\pgfqpoint{2.145481in}{2.334333in}}%
\pgfpathcurveto{\pgfqpoint{2.156531in}{2.334333in}}{\pgfqpoint{2.167130in}{2.338724in}}{\pgfqpoint{2.174944in}{2.346537in}}%
\pgfpathcurveto{\pgfqpoint{2.182758in}{2.354351in}}{\pgfqpoint{2.187148in}{2.364950in}}{\pgfqpoint{2.187148in}{2.376000in}}%
\pgfpathcurveto{\pgfqpoint{2.187148in}{2.387050in}}{\pgfqpoint{2.182758in}{2.397649in}}{\pgfqpoint{2.174944in}{2.405463in}}%
\pgfpathcurveto{\pgfqpoint{2.167130in}{2.413276in}}{\pgfqpoint{2.156531in}{2.417667in}}{\pgfqpoint{2.145481in}{2.417667in}}%
\pgfpathcurveto{\pgfqpoint{2.134431in}{2.417667in}}{\pgfqpoint{2.123832in}{2.413276in}}{\pgfqpoint{2.116018in}{2.405463in}}%
\pgfpathcurveto{\pgfqpoint{2.108205in}{2.397649in}}{\pgfqpoint{2.103815in}{2.387050in}}{\pgfqpoint{2.103815in}{2.376000in}}%
\pgfpathcurveto{\pgfqpoint{2.103815in}{2.364950in}}{\pgfqpoint{2.108205in}{2.354351in}}{\pgfqpoint{2.116018in}{2.346537in}}%
\pgfpathcurveto{\pgfqpoint{2.123832in}{2.338724in}}{\pgfqpoint{2.134431in}{2.334333in}}{\pgfqpoint{2.145481in}{2.334333in}}%
\pgfpathclose%
\pgfusepath{stroke,fill}%
\end{pgfscope}%
\begin{pgfscope}%
\pgfpathrectangle{\pgfqpoint{0.800000in}{0.528000in}}{\pgfqpoint{4.960000in}{3.696000in}}%
\pgfusepath{clip}%
\pgfsetbuttcap%
\pgfsetroundjoin%
\definecolor{currentfill}{rgb}{0.000000,0.000000,0.000000}%
\pgfsetfillcolor{currentfill}%
\pgfsetlinewidth{1.003750pt}%
\definecolor{currentstroke}{rgb}{0.000000,0.000000,0.000000}%
\pgfsetstrokecolor{currentstroke}%
\pgfsetdash{}{0pt}%
\pgfpathmoveto{\pgfqpoint{2.145481in}{2.334333in}}%
\pgfpathcurveto{\pgfqpoint{2.156531in}{2.334333in}}{\pgfqpoint{2.167130in}{2.338724in}}{\pgfqpoint{2.174944in}{2.346537in}}%
\pgfpathcurveto{\pgfqpoint{2.182758in}{2.354351in}}{\pgfqpoint{2.187148in}{2.364950in}}{\pgfqpoint{2.187148in}{2.376000in}}%
\pgfpathcurveto{\pgfqpoint{2.187148in}{2.387050in}}{\pgfqpoint{2.182758in}{2.397649in}}{\pgfqpoint{2.174944in}{2.405463in}}%
\pgfpathcurveto{\pgfqpoint{2.167130in}{2.413276in}}{\pgfqpoint{2.156531in}{2.417667in}}{\pgfqpoint{2.145481in}{2.417667in}}%
\pgfpathcurveto{\pgfqpoint{2.134431in}{2.417667in}}{\pgfqpoint{2.123832in}{2.413276in}}{\pgfqpoint{2.116018in}{2.405463in}}%
\pgfpathcurveto{\pgfqpoint{2.108205in}{2.397649in}}{\pgfqpoint{2.103815in}{2.387050in}}{\pgfqpoint{2.103815in}{2.376000in}}%
\pgfpathcurveto{\pgfqpoint{2.103815in}{2.364950in}}{\pgfqpoint{2.108205in}{2.354351in}}{\pgfqpoint{2.116018in}{2.346537in}}%
\pgfpathcurveto{\pgfqpoint{2.123832in}{2.338724in}}{\pgfqpoint{2.134431in}{2.334333in}}{\pgfqpoint{2.145481in}{2.334333in}}%
\pgfpathclose%
\pgfusepath{stroke,fill}%
\end{pgfscope}%
\begin{pgfscope}%
\pgfpathrectangle{\pgfqpoint{0.800000in}{0.528000in}}{\pgfqpoint{4.960000in}{3.696000in}}%
\pgfusepath{clip}%
\pgfsetbuttcap%
\pgfsetroundjoin%
\definecolor{currentfill}{rgb}{0.000000,0.000000,0.000000}%
\pgfsetfillcolor{currentfill}%
\pgfsetlinewidth{1.003750pt}%
\definecolor{currentstroke}{rgb}{0.000000,0.000000,0.000000}%
\pgfsetstrokecolor{currentstroke}%
\pgfsetdash{}{0pt}%
\pgfpathmoveto{\pgfqpoint{2.145481in}{2.334333in}}%
\pgfpathcurveto{\pgfqpoint{2.156531in}{2.334333in}}{\pgfqpoint{2.167130in}{2.338724in}}{\pgfqpoint{2.174944in}{2.346537in}}%
\pgfpathcurveto{\pgfqpoint{2.182758in}{2.354351in}}{\pgfqpoint{2.187148in}{2.364950in}}{\pgfqpoint{2.187148in}{2.376000in}}%
\pgfpathcurveto{\pgfqpoint{2.187148in}{2.387050in}}{\pgfqpoint{2.182758in}{2.397649in}}{\pgfqpoint{2.174944in}{2.405463in}}%
\pgfpathcurveto{\pgfqpoint{2.167130in}{2.413276in}}{\pgfqpoint{2.156531in}{2.417667in}}{\pgfqpoint{2.145481in}{2.417667in}}%
\pgfpathcurveto{\pgfqpoint{2.134431in}{2.417667in}}{\pgfqpoint{2.123832in}{2.413276in}}{\pgfqpoint{2.116018in}{2.405463in}}%
\pgfpathcurveto{\pgfqpoint{2.108205in}{2.397649in}}{\pgfqpoint{2.103815in}{2.387050in}}{\pgfqpoint{2.103815in}{2.376000in}}%
\pgfpathcurveto{\pgfqpoint{2.103815in}{2.364950in}}{\pgfqpoint{2.108205in}{2.354351in}}{\pgfqpoint{2.116018in}{2.346537in}}%
\pgfpathcurveto{\pgfqpoint{2.123832in}{2.338724in}}{\pgfqpoint{2.134431in}{2.334333in}}{\pgfqpoint{2.145481in}{2.334333in}}%
\pgfpathclose%
\pgfusepath{stroke,fill}%
\end{pgfscope}%
\begin{pgfscope}%
\pgfpathrectangle{\pgfqpoint{0.800000in}{0.528000in}}{\pgfqpoint{4.960000in}{3.696000in}}%
\pgfusepath{clip}%
\pgfsetbuttcap%
\pgfsetroundjoin%
\definecolor{currentfill}{rgb}{0.000000,0.000000,0.000000}%
\pgfsetfillcolor{currentfill}%
\pgfsetlinewidth{1.003750pt}%
\definecolor{currentstroke}{rgb}{0.000000,0.000000,0.000000}%
\pgfsetstrokecolor{currentstroke}%
\pgfsetdash{}{0pt}%
\pgfpathmoveto{\pgfqpoint{2.145481in}{2.334333in}}%
\pgfpathcurveto{\pgfqpoint{2.156531in}{2.334333in}}{\pgfqpoint{2.167130in}{2.338724in}}{\pgfqpoint{2.174944in}{2.346537in}}%
\pgfpathcurveto{\pgfqpoint{2.182758in}{2.354351in}}{\pgfqpoint{2.187148in}{2.364950in}}{\pgfqpoint{2.187148in}{2.376000in}}%
\pgfpathcurveto{\pgfqpoint{2.187148in}{2.387050in}}{\pgfqpoint{2.182758in}{2.397649in}}{\pgfqpoint{2.174944in}{2.405463in}}%
\pgfpathcurveto{\pgfqpoint{2.167130in}{2.413276in}}{\pgfqpoint{2.156531in}{2.417667in}}{\pgfqpoint{2.145481in}{2.417667in}}%
\pgfpathcurveto{\pgfqpoint{2.134431in}{2.417667in}}{\pgfqpoint{2.123832in}{2.413276in}}{\pgfqpoint{2.116018in}{2.405463in}}%
\pgfpathcurveto{\pgfqpoint{2.108205in}{2.397649in}}{\pgfqpoint{2.103815in}{2.387050in}}{\pgfqpoint{2.103815in}{2.376000in}}%
\pgfpathcurveto{\pgfqpoint{2.103815in}{2.364950in}}{\pgfqpoint{2.108205in}{2.354351in}}{\pgfqpoint{2.116018in}{2.346537in}}%
\pgfpathcurveto{\pgfqpoint{2.123832in}{2.338724in}}{\pgfqpoint{2.134431in}{2.334333in}}{\pgfqpoint{2.145481in}{2.334333in}}%
\pgfpathclose%
\pgfusepath{stroke,fill}%
\end{pgfscope}%
\begin{pgfscope}%
\pgfpathrectangle{\pgfqpoint{0.800000in}{0.528000in}}{\pgfqpoint{4.960000in}{3.696000in}}%
\pgfusepath{clip}%
\pgfsetbuttcap%
\pgfsetroundjoin%
\definecolor{currentfill}{rgb}{0.000000,0.000000,0.000000}%
\pgfsetfillcolor{currentfill}%
\pgfsetlinewidth{1.003750pt}%
\definecolor{currentstroke}{rgb}{0.000000,0.000000,0.000000}%
\pgfsetstrokecolor{currentstroke}%
\pgfsetdash{}{0pt}%
\pgfpathmoveto{\pgfqpoint{2.145481in}{2.334333in}}%
\pgfpathcurveto{\pgfqpoint{2.156531in}{2.334333in}}{\pgfqpoint{2.167130in}{2.338724in}}{\pgfqpoint{2.174944in}{2.346537in}}%
\pgfpathcurveto{\pgfqpoint{2.182758in}{2.354351in}}{\pgfqpoint{2.187148in}{2.364950in}}{\pgfqpoint{2.187148in}{2.376000in}}%
\pgfpathcurveto{\pgfqpoint{2.187148in}{2.387050in}}{\pgfqpoint{2.182758in}{2.397649in}}{\pgfqpoint{2.174944in}{2.405463in}}%
\pgfpathcurveto{\pgfqpoint{2.167130in}{2.413276in}}{\pgfqpoint{2.156531in}{2.417667in}}{\pgfqpoint{2.145481in}{2.417667in}}%
\pgfpathcurveto{\pgfqpoint{2.134431in}{2.417667in}}{\pgfqpoint{2.123832in}{2.413276in}}{\pgfqpoint{2.116018in}{2.405463in}}%
\pgfpathcurveto{\pgfqpoint{2.108205in}{2.397649in}}{\pgfqpoint{2.103815in}{2.387050in}}{\pgfqpoint{2.103815in}{2.376000in}}%
\pgfpathcurveto{\pgfqpoint{2.103815in}{2.364950in}}{\pgfqpoint{2.108205in}{2.354351in}}{\pgfqpoint{2.116018in}{2.346537in}}%
\pgfpathcurveto{\pgfqpoint{2.123832in}{2.338724in}}{\pgfqpoint{2.134431in}{2.334333in}}{\pgfqpoint{2.145481in}{2.334333in}}%
\pgfpathclose%
\pgfusepath{stroke,fill}%
\end{pgfscope}%
\begin{pgfscope}%
\pgfpathrectangle{\pgfqpoint{0.800000in}{0.528000in}}{\pgfqpoint{4.960000in}{3.696000in}}%
\pgfusepath{clip}%
\pgfsetbuttcap%
\pgfsetroundjoin%
\definecolor{currentfill}{rgb}{0.000000,0.000000,0.000000}%
\pgfsetfillcolor{currentfill}%
\pgfsetlinewidth{1.003750pt}%
\definecolor{currentstroke}{rgb}{0.000000,0.000000,0.000000}%
\pgfsetstrokecolor{currentstroke}%
\pgfsetdash{}{0pt}%
\pgfpathmoveto{\pgfqpoint{2.145481in}{2.334333in}}%
\pgfpathcurveto{\pgfqpoint{2.156531in}{2.334333in}}{\pgfqpoint{2.167130in}{2.338724in}}{\pgfqpoint{2.174944in}{2.346537in}}%
\pgfpathcurveto{\pgfqpoint{2.182758in}{2.354351in}}{\pgfqpoint{2.187148in}{2.364950in}}{\pgfqpoint{2.187148in}{2.376000in}}%
\pgfpathcurveto{\pgfqpoint{2.187148in}{2.387050in}}{\pgfqpoint{2.182758in}{2.397649in}}{\pgfqpoint{2.174944in}{2.405463in}}%
\pgfpathcurveto{\pgfqpoint{2.167130in}{2.413276in}}{\pgfqpoint{2.156531in}{2.417667in}}{\pgfqpoint{2.145481in}{2.417667in}}%
\pgfpathcurveto{\pgfqpoint{2.134431in}{2.417667in}}{\pgfqpoint{2.123832in}{2.413276in}}{\pgfqpoint{2.116018in}{2.405463in}}%
\pgfpathcurveto{\pgfqpoint{2.108205in}{2.397649in}}{\pgfqpoint{2.103815in}{2.387050in}}{\pgfqpoint{2.103815in}{2.376000in}}%
\pgfpathcurveto{\pgfqpoint{2.103815in}{2.364950in}}{\pgfqpoint{2.108205in}{2.354351in}}{\pgfqpoint{2.116018in}{2.346537in}}%
\pgfpathcurveto{\pgfqpoint{2.123832in}{2.338724in}}{\pgfqpoint{2.134431in}{2.334333in}}{\pgfqpoint{2.145481in}{2.334333in}}%
\pgfpathclose%
\pgfusepath{stroke,fill}%
\end{pgfscope}%
\begin{pgfscope}%
\pgfpathrectangle{\pgfqpoint{0.800000in}{0.528000in}}{\pgfqpoint{4.960000in}{3.696000in}}%
\pgfusepath{clip}%
\pgfsetbuttcap%
\pgfsetroundjoin%
\definecolor{currentfill}{rgb}{0.000000,0.000000,0.000000}%
\pgfsetfillcolor{currentfill}%
\pgfsetlinewidth{1.003750pt}%
\definecolor{currentstroke}{rgb}{0.000000,0.000000,0.000000}%
\pgfsetstrokecolor{currentstroke}%
\pgfsetdash{}{0pt}%
\pgfpathmoveto{\pgfqpoint{2.145481in}{2.334333in}}%
\pgfpathcurveto{\pgfqpoint{2.156531in}{2.334333in}}{\pgfqpoint{2.167130in}{2.338724in}}{\pgfqpoint{2.174944in}{2.346537in}}%
\pgfpathcurveto{\pgfqpoint{2.182758in}{2.354351in}}{\pgfqpoint{2.187148in}{2.364950in}}{\pgfqpoint{2.187148in}{2.376000in}}%
\pgfpathcurveto{\pgfqpoint{2.187148in}{2.387050in}}{\pgfqpoint{2.182758in}{2.397649in}}{\pgfqpoint{2.174944in}{2.405463in}}%
\pgfpathcurveto{\pgfqpoint{2.167130in}{2.413276in}}{\pgfqpoint{2.156531in}{2.417667in}}{\pgfqpoint{2.145481in}{2.417667in}}%
\pgfpathcurveto{\pgfqpoint{2.134431in}{2.417667in}}{\pgfqpoint{2.123832in}{2.413276in}}{\pgfqpoint{2.116018in}{2.405463in}}%
\pgfpathcurveto{\pgfqpoint{2.108205in}{2.397649in}}{\pgfqpoint{2.103815in}{2.387050in}}{\pgfqpoint{2.103815in}{2.376000in}}%
\pgfpathcurveto{\pgfqpoint{2.103815in}{2.364950in}}{\pgfqpoint{2.108205in}{2.354351in}}{\pgfqpoint{2.116018in}{2.346537in}}%
\pgfpathcurveto{\pgfqpoint{2.123832in}{2.338724in}}{\pgfqpoint{2.134431in}{2.334333in}}{\pgfqpoint{2.145481in}{2.334333in}}%
\pgfpathclose%
\pgfusepath{stroke,fill}%
\end{pgfscope}%
\begin{pgfscope}%
\pgfpathrectangle{\pgfqpoint{0.800000in}{0.528000in}}{\pgfqpoint{4.960000in}{3.696000in}}%
\pgfusepath{clip}%
\pgfsetbuttcap%
\pgfsetroundjoin%
\definecolor{currentfill}{rgb}{0.000000,0.000000,0.000000}%
\pgfsetfillcolor{currentfill}%
\pgfsetlinewidth{1.003750pt}%
\definecolor{currentstroke}{rgb}{0.000000,0.000000,0.000000}%
\pgfsetstrokecolor{currentstroke}%
\pgfsetdash{}{0pt}%
\pgfpathmoveto{\pgfqpoint{2.145481in}{2.334333in}}%
\pgfpathcurveto{\pgfqpoint{2.156531in}{2.334333in}}{\pgfqpoint{2.167130in}{2.338724in}}{\pgfqpoint{2.174944in}{2.346537in}}%
\pgfpathcurveto{\pgfqpoint{2.182758in}{2.354351in}}{\pgfqpoint{2.187148in}{2.364950in}}{\pgfqpoint{2.187148in}{2.376000in}}%
\pgfpathcurveto{\pgfqpoint{2.187148in}{2.387050in}}{\pgfqpoint{2.182758in}{2.397649in}}{\pgfqpoint{2.174944in}{2.405463in}}%
\pgfpathcurveto{\pgfqpoint{2.167130in}{2.413276in}}{\pgfqpoint{2.156531in}{2.417667in}}{\pgfqpoint{2.145481in}{2.417667in}}%
\pgfpathcurveto{\pgfqpoint{2.134431in}{2.417667in}}{\pgfqpoint{2.123832in}{2.413276in}}{\pgfqpoint{2.116018in}{2.405463in}}%
\pgfpathcurveto{\pgfqpoint{2.108205in}{2.397649in}}{\pgfqpoint{2.103815in}{2.387050in}}{\pgfqpoint{2.103815in}{2.376000in}}%
\pgfpathcurveto{\pgfqpoint{2.103815in}{2.364950in}}{\pgfqpoint{2.108205in}{2.354351in}}{\pgfqpoint{2.116018in}{2.346537in}}%
\pgfpathcurveto{\pgfqpoint{2.123832in}{2.338724in}}{\pgfqpoint{2.134431in}{2.334333in}}{\pgfqpoint{2.145481in}{2.334333in}}%
\pgfpathclose%
\pgfusepath{stroke,fill}%
\end{pgfscope}%
\begin{pgfscope}%
\pgfpathrectangle{\pgfqpoint{0.800000in}{0.528000in}}{\pgfqpoint{4.960000in}{3.696000in}}%
\pgfusepath{clip}%
\pgfsetbuttcap%
\pgfsetroundjoin%
\definecolor{currentfill}{rgb}{0.000000,0.000000,0.000000}%
\pgfsetfillcolor{currentfill}%
\pgfsetlinewidth{1.003750pt}%
\definecolor{currentstroke}{rgb}{0.000000,0.000000,0.000000}%
\pgfsetstrokecolor{currentstroke}%
\pgfsetdash{}{0pt}%
\pgfpathmoveto{\pgfqpoint{2.145481in}{2.334333in}}%
\pgfpathcurveto{\pgfqpoint{2.156531in}{2.334333in}}{\pgfqpoint{2.167130in}{2.338724in}}{\pgfqpoint{2.174944in}{2.346537in}}%
\pgfpathcurveto{\pgfqpoint{2.182758in}{2.354351in}}{\pgfqpoint{2.187148in}{2.364950in}}{\pgfqpoint{2.187148in}{2.376000in}}%
\pgfpathcurveto{\pgfqpoint{2.187148in}{2.387050in}}{\pgfqpoint{2.182758in}{2.397649in}}{\pgfqpoint{2.174944in}{2.405463in}}%
\pgfpathcurveto{\pgfqpoint{2.167130in}{2.413276in}}{\pgfqpoint{2.156531in}{2.417667in}}{\pgfqpoint{2.145481in}{2.417667in}}%
\pgfpathcurveto{\pgfqpoint{2.134431in}{2.417667in}}{\pgfqpoint{2.123832in}{2.413276in}}{\pgfqpoint{2.116018in}{2.405463in}}%
\pgfpathcurveto{\pgfqpoint{2.108205in}{2.397649in}}{\pgfqpoint{2.103815in}{2.387050in}}{\pgfqpoint{2.103815in}{2.376000in}}%
\pgfpathcurveto{\pgfqpoint{2.103815in}{2.364950in}}{\pgfqpoint{2.108205in}{2.354351in}}{\pgfqpoint{2.116018in}{2.346537in}}%
\pgfpathcurveto{\pgfqpoint{2.123832in}{2.338724in}}{\pgfqpoint{2.134431in}{2.334333in}}{\pgfqpoint{2.145481in}{2.334333in}}%
\pgfpathclose%
\pgfusepath{stroke,fill}%
\end{pgfscope}%
\begin{pgfscope}%
\pgfpathrectangle{\pgfqpoint{0.800000in}{0.528000in}}{\pgfqpoint{4.960000in}{3.696000in}}%
\pgfusepath{clip}%
\pgfsetbuttcap%
\pgfsetroundjoin%
\definecolor{currentfill}{rgb}{0.000000,0.000000,0.000000}%
\pgfsetfillcolor{currentfill}%
\pgfsetlinewidth{1.003750pt}%
\definecolor{currentstroke}{rgb}{0.000000,0.000000,0.000000}%
\pgfsetstrokecolor{currentstroke}%
\pgfsetdash{}{0pt}%
\pgfpathmoveto{\pgfqpoint{2.145481in}{2.334333in}}%
\pgfpathcurveto{\pgfqpoint{2.156531in}{2.334333in}}{\pgfqpoint{2.167130in}{2.338724in}}{\pgfqpoint{2.174944in}{2.346537in}}%
\pgfpathcurveto{\pgfqpoint{2.182758in}{2.354351in}}{\pgfqpoint{2.187148in}{2.364950in}}{\pgfqpoint{2.187148in}{2.376000in}}%
\pgfpathcurveto{\pgfqpoint{2.187148in}{2.387050in}}{\pgfqpoint{2.182758in}{2.397649in}}{\pgfqpoint{2.174944in}{2.405463in}}%
\pgfpathcurveto{\pgfqpoint{2.167130in}{2.413276in}}{\pgfqpoint{2.156531in}{2.417667in}}{\pgfqpoint{2.145481in}{2.417667in}}%
\pgfpathcurveto{\pgfqpoint{2.134431in}{2.417667in}}{\pgfqpoint{2.123832in}{2.413276in}}{\pgfqpoint{2.116018in}{2.405463in}}%
\pgfpathcurveto{\pgfqpoint{2.108205in}{2.397649in}}{\pgfqpoint{2.103815in}{2.387050in}}{\pgfqpoint{2.103815in}{2.376000in}}%
\pgfpathcurveto{\pgfqpoint{2.103815in}{2.364950in}}{\pgfqpoint{2.108205in}{2.354351in}}{\pgfqpoint{2.116018in}{2.346537in}}%
\pgfpathcurveto{\pgfqpoint{2.123832in}{2.338724in}}{\pgfqpoint{2.134431in}{2.334333in}}{\pgfqpoint{2.145481in}{2.334333in}}%
\pgfpathclose%
\pgfusepath{stroke,fill}%
\end{pgfscope}%
\begin{pgfscope}%
\pgfpathrectangle{\pgfqpoint{0.800000in}{0.528000in}}{\pgfqpoint{4.960000in}{3.696000in}}%
\pgfusepath{clip}%
\pgfsetbuttcap%
\pgfsetroundjoin%
\definecolor{currentfill}{rgb}{0.000000,0.000000,0.000000}%
\pgfsetfillcolor{currentfill}%
\pgfsetlinewidth{1.003750pt}%
\definecolor{currentstroke}{rgb}{0.000000,0.000000,0.000000}%
\pgfsetstrokecolor{currentstroke}%
\pgfsetdash{}{0pt}%
\pgfpathmoveto{\pgfqpoint{2.145481in}{2.334333in}}%
\pgfpathcurveto{\pgfqpoint{2.156531in}{2.334333in}}{\pgfqpoint{2.167130in}{2.338724in}}{\pgfqpoint{2.174944in}{2.346537in}}%
\pgfpathcurveto{\pgfqpoint{2.182758in}{2.354351in}}{\pgfqpoint{2.187148in}{2.364950in}}{\pgfqpoint{2.187148in}{2.376000in}}%
\pgfpathcurveto{\pgfqpoint{2.187148in}{2.387050in}}{\pgfqpoint{2.182758in}{2.397649in}}{\pgfqpoint{2.174944in}{2.405463in}}%
\pgfpathcurveto{\pgfqpoint{2.167130in}{2.413276in}}{\pgfqpoint{2.156531in}{2.417667in}}{\pgfqpoint{2.145481in}{2.417667in}}%
\pgfpathcurveto{\pgfqpoint{2.134431in}{2.417667in}}{\pgfqpoint{2.123832in}{2.413276in}}{\pgfqpoint{2.116018in}{2.405463in}}%
\pgfpathcurveto{\pgfqpoint{2.108205in}{2.397649in}}{\pgfqpoint{2.103815in}{2.387050in}}{\pgfqpoint{2.103815in}{2.376000in}}%
\pgfpathcurveto{\pgfqpoint{2.103815in}{2.364950in}}{\pgfqpoint{2.108205in}{2.354351in}}{\pgfqpoint{2.116018in}{2.346537in}}%
\pgfpathcurveto{\pgfqpoint{2.123832in}{2.338724in}}{\pgfqpoint{2.134431in}{2.334333in}}{\pgfqpoint{2.145481in}{2.334333in}}%
\pgfpathclose%
\pgfusepath{stroke,fill}%
\end{pgfscope}%
\begin{pgfscope}%
\pgfpathrectangle{\pgfqpoint{0.800000in}{0.528000in}}{\pgfqpoint{4.960000in}{3.696000in}}%
\pgfusepath{clip}%
\pgfsetbuttcap%
\pgfsetroundjoin%
\definecolor{currentfill}{rgb}{0.000000,0.000000,0.000000}%
\pgfsetfillcolor{currentfill}%
\pgfsetlinewidth{1.003750pt}%
\definecolor{currentstroke}{rgb}{0.000000,0.000000,0.000000}%
\pgfsetstrokecolor{currentstroke}%
\pgfsetdash{}{0pt}%
\pgfpathmoveto{\pgfqpoint{2.145481in}{2.334333in}}%
\pgfpathcurveto{\pgfqpoint{2.156531in}{2.334333in}}{\pgfqpoint{2.167130in}{2.338724in}}{\pgfqpoint{2.174944in}{2.346537in}}%
\pgfpathcurveto{\pgfqpoint{2.182758in}{2.354351in}}{\pgfqpoint{2.187148in}{2.364950in}}{\pgfqpoint{2.187148in}{2.376000in}}%
\pgfpathcurveto{\pgfqpoint{2.187148in}{2.387050in}}{\pgfqpoint{2.182758in}{2.397649in}}{\pgfqpoint{2.174944in}{2.405463in}}%
\pgfpathcurveto{\pgfqpoint{2.167130in}{2.413276in}}{\pgfqpoint{2.156531in}{2.417667in}}{\pgfqpoint{2.145481in}{2.417667in}}%
\pgfpathcurveto{\pgfqpoint{2.134431in}{2.417667in}}{\pgfqpoint{2.123832in}{2.413276in}}{\pgfqpoint{2.116018in}{2.405463in}}%
\pgfpathcurveto{\pgfqpoint{2.108205in}{2.397649in}}{\pgfqpoint{2.103815in}{2.387050in}}{\pgfqpoint{2.103815in}{2.376000in}}%
\pgfpathcurveto{\pgfqpoint{2.103815in}{2.364950in}}{\pgfqpoint{2.108205in}{2.354351in}}{\pgfqpoint{2.116018in}{2.346537in}}%
\pgfpathcurveto{\pgfqpoint{2.123832in}{2.338724in}}{\pgfqpoint{2.134431in}{2.334333in}}{\pgfqpoint{2.145481in}{2.334333in}}%
\pgfpathclose%
\pgfusepath{stroke,fill}%
\end{pgfscope}%
\begin{pgfscope}%
\pgfpathrectangle{\pgfqpoint{0.800000in}{0.528000in}}{\pgfqpoint{4.960000in}{3.696000in}}%
\pgfusepath{clip}%
\pgfsetbuttcap%
\pgfsetroundjoin%
\definecolor{currentfill}{rgb}{0.000000,0.000000,0.000000}%
\pgfsetfillcolor{currentfill}%
\pgfsetlinewidth{1.003750pt}%
\definecolor{currentstroke}{rgb}{0.000000,0.000000,0.000000}%
\pgfsetstrokecolor{currentstroke}%
\pgfsetdash{}{0pt}%
\pgfpathmoveto{\pgfqpoint{2.145481in}{2.334333in}}%
\pgfpathcurveto{\pgfqpoint{2.156531in}{2.334333in}}{\pgfqpoint{2.167130in}{2.338724in}}{\pgfqpoint{2.174944in}{2.346537in}}%
\pgfpathcurveto{\pgfqpoint{2.182758in}{2.354351in}}{\pgfqpoint{2.187148in}{2.364950in}}{\pgfqpoint{2.187148in}{2.376000in}}%
\pgfpathcurveto{\pgfqpoint{2.187148in}{2.387050in}}{\pgfqpoint{2.182758in}{2.397649in}}{\pgfqpoint{2.174944in}{2.405463in}}%
\pgfpathcurveto{\pgfqpoint{2.167130in}{2.413276in}}{\pgfqpoint{2.156531in}{2.417667in}}{\pgfqpoint{2.145481in}{2.417667in}}%
\pgfpathcurveto{\pgfqpoint{2.134431in}{2.417667in}}{\pgfqpoint{2.123832in}{2.413276in}}{\pgfqpoint{2.116018in}{2.405463in}}%
\pgfpathcurveto{\pgfqpoint{2.108205in}{2.397649in}}{\pgfqpoint{2.103815in}{2.387050in}}{\pgfqpoint{2.103815in}{2.376000in}}%
\pgfpathcurveto{\pgfqpoint{2.103815in}{2.364950in}}{\pgfqpoint{2.108205in}{2.354351in}}{\pgfqpoint{2.116018in}{2.346537in}}%
\pgfpathcurveto{\pgfqpoint{2.123832in}{2.338724in}}{\pgfqpoint{2.134431in}{2.334333in}}{\pgfqpoint{2.145481in}{2.334333in}}%
\pgfpathclose%
\pgfusepath{stroke,fill}%
\end{pgfscope}%
\begin{pgfscope}%
\pgfpathrectangle{\pgfqpoint{0.800000in}{0.528000in}}{\pgfqpoint{4.960000in}{3.696000in}}%
\pgfusepath{clip}%
\pgfsetbuttcap%
\pgfsetroundjoin%
\definecolor{currentfill}{rgb}{0.000000,0.000000,0.000000}%
\pgfsetfillcolor{currentfill}%
\pgfsetlinewidth{1.003750pt}%
\definecolor{currentstroke}{rgb}{0.000000,0.000000,0.000000}%
\pgfsetstrokecolor{currentstroke}%
\pgfsetdash{}{0pt}%
\pgfpathmoveto{\pgfqpoint{2.145481in}{2.334333in}}%
\pgfpathcurveto{\pgfqpoint{2.156531in}{2.334333in}}{\pgfqpoint{2.167130in}{2.338724in}}{\pgfqpoint{2.174944in}{2.346537in}}%
\pgfpathcurveto{\pgfqpoint{2.182758in}{2.354351in}}{\pgfqpoint{2.187148in}{2.364950in}}{\pgfqpoint{2.187148in}{2.376000in}}%
\pgfpathcurveto{\pgfqpoint{2.187148in}{2.387050in}}{\pgfqpoint{2.182758in}{2.397649in}}{\pgfqpoint{2.174944in}{2.405463in}}%
\pgfpathcurveto{\pgfqpoint{2.167130in}{2.413276in}}{\pgfqpoint{2.156531in}{2.417667in}}{\pgfqpoint{2.145481in}{2.417667in}}%
\pgfpathcurveto{\pgfqpoint{2.134431in}{2.417667in}}{\pgfqpoint{2.123832in}{2.413276in}}{\pgfqpoint{2.116018in}{2.405463in}}%
\pgfpathcurveto{\pgfqpoint{2.108205in}{2.397649in}}{\pgfqpoint{2.103815in}{2.387050in}}{\pgfqpoint{2.103815in}{2.376000in}}%
\pgfpathcurveto{\pgfqpoint{2.103815in}{2.364950in}}{\pgfqpoint{2.108205in}{2.354351in}}{\pgfqpoint{2.116018in}{2.346537in}}%
\pgfpathcurveto{\pgfqpoint{2.123832in}{2.338724in}}{\pgfqpoint{2.134431in}{2.334333in}}{\pgfqpoint{2.145481in}{2.334333in}}%
\pgfpathclose%
\pgfusepath{stroke,fill}%
\end{pgfscope}%
\begin{pgfscope}%
\pgfpathrectangle{\pgfqpoint{0.800000in}{0.528000in}}{\pgfqpoint{4.960000in}{3.696000in}}%
\pgfusepath{clip}%
\pgfsetbuttcap%
\pgfsetroundjoin%
\definecolor{currentfill}{rgb}{0.000000,0.000000,0.000000}%
\pgfsetfillcolor{currentfill}%
\pgfsetlinewidth{1.003750pt}%
\definecolor{currentstroke}{rgb}{0.000000,0.000000,0.000000}%
\pgfsetstrokecolor{currentstroke}%
\pgfsetdash{}{0pt}%
\pgfpathmoveto{\pgfqpoint{2.145481in}{2.334333in}}%
\pgfpathcurveto{\pgfqpoint{2.156531in}{2.334333in}}{\pgfqpoint{2.167130in}{2.338724in}}{\pgfqpoint{2.174944in}{2.346537in}}%
\pgfpathcurveto{\pgfqpoint{2.182758in}{2.354351in}}{\pgfqpoint{2.187148in}{2.364950in}}{\pgfqpoint{2.187148in}{2.376000in}}%
\pgfpathcurveto{\pgfqpoint{2.187148in}{2.387050in}}{\pgfqpoint{2.182758in}{2.397649in}}{\pgfqpoint{2.174944in}{2.405463in}}%
\pgfpathcurveto{\pgfqpoint{2.167130in}{2.413276in}}{\pgfqpoint{2.156531in}{2.417667in}}{\pgfqpoint{2.145481in}{2.417667in}}%
\pgfpathcurveto{\pgfqpoint{2.134431in}{2.417667in}}{\pgfqpoint{2.123832in}{2.413276in}}{\pgfqpoint{2.116018in}{2.405463in}}%
\pgfpathcurveto{\pgfqpoint{2.108205in}{2.397649in}}{\pgfqpoint{2.103815in}{2.387050in}}{\pgfqpoint{2.103815in}{2.376000in}}%
\pgfpathcurveto{\pgfqpoint{2.103815in}{2.364950in}}{\pgfqpoint{2.108205in}{2.354351in}}{\pgfqpoint{2.116018in}{2.346537in}}%
\pgfpathcurveto{\pgfqpoint{2.123832in}{2.338724in}}{\pgfqpoint{2.134431in}{2.334333in}}{\pgfqpoint{2.145481in}{2.334333in}}%
\pgfpathclose%
\pgfusepath{stroke,fill}%
\end{pgfscope}%
\begin{pgfscope}%
\pgfpathrectangle{\pgfqpoint{0.800000in}{0.528000in}}{\pgfqpoint{4.960000in}{3.696000in}}%
\pgfusepath{clip}%
\pgfsetbuttcap%
\pgfsetroundjoin%
\definecolor{currentfill}{rgb}{0.000000,0.000000,0.000000}%
\pgfsetfillcolor{currentfill}%
\pgfsetlinewidth{1.003750pt}%
\definecolor{currentstroke}{rgb}{0.000000,0.000000,0.000000}%
\pgfsetstrokecolor{currentstroke}%
\pgfsetdash{}{0pt}%
\pgfpathmoveto{\pgfqpoint{2.145481in}{2.334333in}}%
\pgfpathcurveto{\pgfqpoint{2.156531in}{2.334333in}}{\pgfqpoint{2.167130in}{2.338724in}}{\pgfqpoint{2.174944in}{2.346537in}}%
\pgfpathcurveto{\pgfqpoint{2.182758in}{2.354351in}}{\pgfqpoint{2.187148in}{2.364950in}}{\pgfqpoint{2.187148in}{2.376000in}}%
\pgfpathcurveto{\pgfqpoint{2.187148in}{2.387050in}}{\pgfqpoint{2.182758in}{2.397649in}}{\pgfqpoint{2.174944in}{2.405463in}}%
\pgfpathcurveto{\pgfqpoint{2.167130in}{2.413276in}}{\pgfqpoint{2.156531in}{2.417667in}}{\pgfqpoint{2.145481in}{2.417667in}}%
\pgfpathcurveto{\pgfqpoint{2.134431in}{2.417667in}}{\pgfqpoint{2.123832in}{2.413276in}}{\pgfqpoint{2.116018in}{2.405463in}}%
\pgfpathcurveto{\pgfqpoint{2.108205in}{2.397649in}}{\pgfqpoint{2.103815in}{2.387050in}}{\pgfqpoint{2.103815in}{2.376000in}}%
\pgfpathcurveto{\pgfqpoint{2.103815in}{2.364950in}}{\pgfqpoint{2.108205in}{2.354351in}}{\pgfqpoint{2.116018in}{2.346537in}}%
\pgfpathcurveto{\pgfqpoint{2.123832in}{2.338724in}}{\pgfqpoint{2.134431in}{2.334333in}}{\pgfqpoint{2.145481in}{2.334333in}}%
\pgfpathclose%
\pgfusepath{stroke,fill}%
\end{pgfscope}%
\begin{pgfscope}%
\pgfpathrectangle{\pgfqpoint{0.800000in}{0.528000in}}{\pgfqpoint{4.960000in}{3.696000in}}%
\pgfusepath{clip}%
\pgfsetbuttcap%
\pgfsetroundjoin%
\definecolor{currentfill}{rgb}{0.000000,0.000000,0.000000}%
\pgfsetfillcolor{currentfill}%
\pgfsetlinewidth{1.003750pt}%
\definecolor{currentstroke}{rgb}{0.000000,0.000000,0.000000}%
\pgfsetstrokecolor{currentstroke}%
\pgfsetdash{}{0pt}%
\pgfpathmoveto{\pgfqpoint{2.145481in}{2.334333in}}%
\pgfpathcurveto{\pgfqpoint{2.156531in}{2.334333in}}{\pgfqpoint{2.167130in}{2.338724in}}{\pgfqpoint{2.174944in}{2.346537in}}%
\pgfpathcurveto{\pgfqpoint{2.182758in}{2.354351in}}{\pgfqpoint{2.187148in}{2.364950in}}{\pgfqpoint{2.187148in}{2.376000in}}%
\pgfpathcurveto{\pgfqpoint{2.187148in}{2.387050in}}{\pgfqpoint{2.182758in}{2.397649in}}{\pgfqpoint{2.174944in}{2.405463in}}%
\pgfpathcurveto{\pgfqpoint{2.167130in}{2.413276in}}{\pgfqpoint{2.156531in}{2.417667in}}{\pgfqpoint{2.145481in}{2.417667in}}%
\pgfpathcurveto{\pgfqpoint{2.134431in}{2.417667in}}{\pgfqpoint{2.123832in}{2.413276in}}{\pgfqpoint{2.116018in}{2.405463in}}%
\pgfpathcurveto{\pgfqpoint{2.108205in}{2.397649in}}{\pgfqpoint{2.103815in}{2.387050in}}{\pgfqpoint{2.103815in}{2.376000in}}%
\pgfpathcurveto{\pgfqpoint{2.103815in}{2.364950in}}{\pgfqpoint{2.108205in}{2.354351in}}{\pgfqpoint{2.116018in}{2.346537in}}%
\pgfpathcurveto{\pgfqpoint{2.123832in}{2.338724in}}{\pgfqpoint{2.134431in}{2.334333in}}{\pgfqpoint{2.145481in}{2.334333in}}%
\pgfpathclose%
\pgfusepath{stroke,fill}%
\end{pgfscope}%
\begin{pgfscope}%
\pgfpathrectangle{\pgfqpoint{0.800000in}{0.528000in}}{\pgfqpoint{4.960000in}{3.696000in}}%
\pgfusepath{clip}%
\pgfsetbuttcap%
\pgfsetroundjoin%
\definecolor{currentfill}{rgb}{0.000000,0.000000,0.000000}%
\pgfsetfillcolor{currentfill}%
\pgfsetlinewidth{1.003750pt}%
\definecolor{currentstroke}{rgb}{0.000000,0.000000,0.000000}%
\pgfsetstrokecolor{currentstroke}%
\pgfsetdash{}{0pt}%
\pgfpathmoveto{\pgfqpoint{2.145481in}{2.334333in}}%
\pgfpathcurveto{\pgfqpoint{2.156531in}{2.334333in}}{\pgfqpoint{2.167130in}{2.338724in}}{\pgfqpoint{2.174944in}{2.346537in}}%
\pgfpathcurveto{\pgfqpoint{2.182758in}{2.354351in}}{\pgfqpoint{2.187148in}{2.364950in}}{\pgfqpoint{2.187148in}{2.376000in}}%
\pgfpathcurveto{\pgfqpoint{2.187148in}{2.387050in}}{\pgfqpoint{2.182758in}{2.397649in}}{\pgfqpoint{2.174944in}{2.405463in}}%
\pgfpathcurveto{\pgfqpoint{2.167130in}{2.413276in}}{\pgfqpoint{2.156531in}{2.417667in}}{\pgfqpoint{2.145481in}{2.417667in}}%
\pgfpathcurveto{\pgfqpoint{2.134431in}{2.417667in}}{\pgfqpoint{2.123832in}{2.413276in}}{\pgfqpoint{2.116018in}{2.405463in}}%
\pgfpathcurveto{\pgfqpoint{2.108205in}{2.397649in}}{\pgfqpoint{2.103815in}{2.387050in}}{\pgfqpoint{2.103815in}{2.376000in}}%
\pgfpathcurveto{\pgfqpoint{2.103815in}{2.364950in}}{\pgfqpoint{2.108205in}{2.354351in}}{\pgfqpoint{2.116018in}{2.346537in}}%
\pgfpathcurveto{\pgfqpoint{2.123832in}{2.338724in}}{\pgfqpoint{2.134431in}{2.334333in}}{\pgfqpoint{2.145481in}{2.334333in}}%
\pgfpathclose%
\pgfusepath{stroke,fill}%
\end{pgfscope}%
\begin{pgfscope}%
\pgfpathrectangle{\pgfqpoint{0.800000in}{0.528000in}}{\pgfqpoint{4.960000in}{3.696000in}}%
\pgfusepath{clip}%
\pgfsetbuttcap%
\pgfsetroundjoin%
\definecolor{currentfill}{rgb}{0.000000,0.000000,0.000000}%
\pgfsetfillcolor{currentfill}%
\pgfsetlinewidth{1.003750pt}%
\definecolor{currentstroke}{rgb}{0.000000,0.000000,0.000000}%
\pgfsetstrokecolor{currentstroke}%
\pgfsetdash{}{0pt}%
\pgfpathmoveto{\pgfqpoint{2.145481in}{2.334333in}}%
\pgfpathcurveto{\pgfqpoint{2.156531in}{2.334333in}}{\pgfqpoint{2.167130in}{2.338724in}}{\pgfqpoint{2.174944in}{2.346537in}}%
\pgfpathcurveto{\pgfqpoint{2.182758in}{2.354351in}}{\pgfqpoint{2.187148in}{2.364950in}}{\pgfqpoint{2.187148in}{2.376000in}}%
\pgfpathcurveto{\pgfqpoint{2.187148in}{2.387050in}}{\pgfqpoint{2.182758in}{2.397649in}}{\pgfqpoint{2.174944in}{2.405463in}}%
\pgfpathcurveto{\pgfqpoint{2.167130in}{2.413276in}}{\pgfqpoint{2.156531in}{2.417667in}}{\pgfqpoint{2.145481in}{2.417667in}}%
\pgfpathcurveto{\pgfqpoint{2.134431in}{2.417667in}}{\pgfqpoint{2.123832in}{2.413276in}}{\pgfqpoint{2.116018in}{2.405463in}}%
\pgfpathcurveto{\pgfqpoint{2.108205in}{2.397649in}}{\pgfqpoint{2.103815in}{2.387050in}}{\pgfqpoint{2.103815in}{2.376000in}}%
\pgfpathcurveto{\pgfqpoint{2.103815in}{2.364950in}}{\pgfqpoint{2.108205in}{2.354351in}}{\pgfqpoint{2.116018in}{2.346537in}}%
\pgfpathcurveto{\pgfqpoint{2.123832in}{2.338724in}}{\pgfqpoint{2.134431in}{2.334333in}}{\pgfqpoint{2.145481in}{2.334333in}}%
\pgfpathclose%
\pgfusepath{stroke,fill}%
\end{pgfscope}%
\begin{pgfscope}%
\pgfpathrectangle{\pgfqpoint{0.800000in}{0.528000in}}{\pgfqpoint{4.960000in}{3.696000in}}%
\pgfusepath{clip}%
\pgfsetbuttcap%
\pgfsetroundjoin%
\definecolor{currentfill}{rgb}{0.000000,0.000000,0.000000}%
\pgfsetfillcolor{currentfill}%
\pgfsetlinewidth{1.003750pt}%
\definecolor{currentstroke}{rgb}{0.000000,0.000000,0.000000}%
\pgfsetstrokecolor{currentstroke}%
\pgfsetdash{}{0pt}%
\pgfpathmoveto{\pgfqpoint{2.145481in}{2.334333in}}%
\pgfpathcurveto{\pgfqpoint{2.156531in}{2.334333in}}{\pgfqpoint{2.167130in}{2.338724in}}{\pgfqpoint{2.174944in}{2.346537in}}%
\pgfpathcurveto{\pgfqpoint{2.182758in}{2.354351in}}{\pgfqpoint{2.187148in}{2.364950in}}{\pgfqpoint{2.187148in}{2.376000in}}%
\pgfpathcurveto{\pgfqpoint{2.187148in}{2.387050in}}{\pgfqpoint{2.182758in}{2.397649in}}{\pgfqpoint{2.174944in}{2.405463in}}%
\pgfpathcurveto{\pgfqpoint{2.167130in}{2.413276in}}{\pgfqpoint{2.156531in}{2.417667in}}{\pgfqpoint{2.145481in}{2.417667in}}%
\pgfpathcurveto{\pgfqpoint{2.134431in}{2.417667in}}{\pgfqpoint{2.123832in}{2.413276in}}{\pgfqpoint{2.116018in}{2.405463in}}%
\pgfpathcurveto{\pgfqpoint{2.108205in}{2.397649in}}{\pgfqpoint{2.103815in}{2.387050in}}{\pgfqpoint{2.103815in}{2.376000in}}%
\pgfpathcurveto{\pgfqpoint{2.103815in}{2.364950in}}{\pgfqpoint{2.108205in}{2.354351in}}{\pgfqpoint{2.116018in}{2.346537in}}%
\pgfpathcurveto{\pgfqpoint{2.123832in}{2.338724in}}{\pgfqpoint{2.134431in}{2.334333in}}{\pgfqpoint{2.145481in}{2.334333in}}%
\pgfpathclose%
\pgfusepath{stroke,fill}%
\end{pgfscope}%
\begin{pgfscope}%
\pgfpathrectangle{\pgfqpoint{0.800000in}{0.528000in}}{\pgfqpoint{4.960000in}{3.696000in}}%
\pgfusepath{clip}%
\pgfsetbuttcap%
\pgfsetroundjoin%
\definecolor{currentfill}{rgb}{0.000000,0.000000,0.000000}%
\pgfsetfillcolor{currentfill}%
\pgfsetlinewidth{1.003750pt}%
\definecolor{currentstroke}{rgb}{0.000000,0.000000,0.000000}%
\pgfsetstrokecolor{currentstroke}%
\pgfsetdash{}{0pt}%
\pgfpathmoveto{\pgfqpoint{2.145481in}{2.334333in}}%
\pgfpathcurveto{\pgfqpoint{2.156531in}{2.334333in}}{\pgfqpoint{2.167130in}{2.338724in}}{\pgfqpoint{2.174944in}{2.346537in}}%
\pgfpathcurveto{\pgfqpoint{2.182758in}{2.354351in}}{\pgfqpoint{2.187148in}{2.364950in}}{\pgfqpoint{2.187148in}{2.376000in}}%
\pgfpathcurveto{\pgfqpoint{2.187148in}{2.387050in}}{\pgfqpoint{2.182758in}{2.397649in}}{\pgfqpoint{2.174944in}{2.405463in}}%
\pgfpathcurveto{\pgfqpoint{2.167130in}{2.413276in}}{\pgfqpoint{2.156531in}{2.417667in}}{\pgfqpoint{2.145481in}{2.417667in}}%
\pgfpathcurveto{\pgfqpoint{2.134431in}{2.417667in}}{\pgfqpoint{2.123832in}{2.413276in}}{\pgfqpoint{2.116018in}{2.405463in}}%
\pgfpathcurveto{\pgfqpoint{2.108205in}{2.397649in}}{\pgfqpoint{2.103815in}{2.387050in}}{\pgfqpoint{2.103815in}{2.376000in}}%
\pgfpathcurveto{\pgfqpoint{2.103815in}{2.364950in}}{\pgfqpoint{2.108205in}{2.354351in}}{\pgfqpoint{2.116018in}{2.346537in}}%
\pgfpathcurveto{\pgfqpoint{2.123832in}{2.338724in}}{\pgfqpoint{2.134431in}{2.334333in}}{\pgfqpoint{2.145481in}{2.334333in}}%
\pgfpathclose%
\pgfusepath{stroke,fill}%
\end{pgfscope}%
\begin{pgfscope}%
\pgfpathrectangle{\pgfqpoint{0.800000in}{0.528000in}}{\pgfqpoint{4.960000in}{3.696000in}}%
\pgfusepath{clip}%
\pgfsetbuttcap%
\pgfsetroundjoin%
\definecolor{currentfill}{rgb}{0.000000,0.000000,0.000000}%
\pgfsetfillcolor{currentfill}%
\pgfsetlinewidth{1.003750pt}%
\definecolor{currentstroke}{rgb}{0.000000,0.000000,0.000000}%
\pgfsetstrokecolor{currentstroke}%
\pgfsetdash{}{0pt}%
\pgfpathmoveto{\pgfqpoint{2.145481in}{2.334333in}}%
\pgfpathcurveto{\pgfqpoint{2.156531in}{2.334333in}}{\pgfqpoint{2.167130in}{2.338724in}}{\pgfqpoint{2.174944in}{2.346537in}}%
\pgfpathcurveto{\pgfqpoint{2.182758in}{2.354351in}}{\pgfqpoint{2.187148in}{2.364950in}}{\pgfqpoint{2.187148in}{2.376000in}}%
\pgfpathcurveto{\pgfqpoint{2.187148in}{2.387050in}}{\pgfqpoint{2.182758in}{2.397649in}}{\pgfqpoint{2.174944in}{2.405463in}}%
\pgfpathcurveto{\pgfqpoint{2.167130in}{2.413276in}}{\pgfqpoint{2.156531in}{2.417667in}}{\pgfqpoint{2.145481in}{2.417667in}}%
\pgfpathcurveto{\pgfqpoint{2.134431in}{2.417667in}}{\pgfqpoint{2.123832in}{2.413276in}}{\pgfqpoint{2.116018in}{2.405463in}}%
\pgfpathcurveto{\pgfqpoint{2.108205in}{2.397649in}}{\pgfqpoint{2.103815in}{2.387050in}}{\pgfqpoint{2.103815in}{2.376000in}}%
\pgfpathcurveto{\pgfqpoint{2.103815in}{2.364950in}}{\pgfqpoint{2.108205in}{2.354351in}}{\pgfqpoint{2.116018in}{2.346537in}}%
\pgfpathcurveto{\pgfqpoint{2.123832in}{2.338724in}}{\pgfqpoint{2.134431in}{2.334333in}}{\pgfqpoint{2.145481in}{2.334333in}}%
\pgfpathclose%
\pgfusepath{stroke,fill}%
\end{pgfscope}%
\begin{pgfscope}%
\pgfpathrectangle{\pgfqpoint{0.800000in}{0.528000in}}{\pgfqpoint{4.960000in}{3.696000in}}%
\pgfusepath{clip}%
\pgfsetbuttcap%
\pgfsetroundjoin%
\definecolor{currentfill}{rgb}{0.000000,0.000000,0.000000}%
\pgfsetfillcolor{currentfill}%
\pgfsetlinewidth{1.003750pt}%
\definecolor{currentstroke}{rgb}{0.000000,0.000000,0.000000}%
\pgfsetstrokecolor{currentstroke}%
\pgfsetdash{}{0pt}%
\pgfpathmoveto{\pgfqpoint{2.145481in}{2.334333in}}%
\pgfpathcurveto{\pgfqpoint{2.156531in}{2.334333in}}{\pgfqpoint{2.167130in}{2.338724in}}{\pgfqpoint{2.174944in}{2.346537in}}%
\pgfpathcurveto{\pgfqpoint{2.182758in}{2.354351in}}{\pgfqpoint{2.187148in}{2.364950in}}{\pgfqpoint{2.187148in}{2.376000in}}%
\pgfpathcurveto{\pgfqpoint{2.187148in}{2.387050in}}{\pgfqpoint{2.182758in}{2.397649in}}{\pgfqpoint{2.174944in}{2.405463in}}%
\pgfpathcurveto{\pgfqpoint{2.167130in}{2.413276in}}{\pgfqpoint{2.156531in}{2.417667in}}{\pgfqpoint{2.145481in}{2.417667in}}%
\pgfpathcurveto{\pgfqpoint{2.134431in}{2.417667in}}{\pgfqpoint{2.123832in}{2.413276in}}{\pgfqpoint{2.116018in}{2.405463in}}%
\pgfpathcurveto{\pgfqpoint{2.108205in}{2.397649in}}{\pgfqpoint{2.103815in}{2.387050in}}{\pgfqpoint{2.103815in}{2.376000in}}%
\pgfpathcurveto{\pgfqpoint{2.103815in}{2.364950in}}{\pgfqpoint{2.108205in}{2.354351in}}{\pgfqpoint{2.116018in}{2.346537in}}%
\pgfpathcurveto{\pgfqpoint{2.123832in}{2.338724in}}{\pgfqpoint{2.134431in}{2.334333in}}{\pgfqpoint{2.145481in}{2.334333in}}%
\pgfpathclose%
\pgfusepath{stroke,fill}%
\end{pgfscope}%
\begin{pgfscope}%
\pgfpathrectangle{\pgfqpoint{0.800000in}{0.528000in}}{\pgfqpoint{4.960000in}{3.696000in}}%
\pgfusepath{clip}%
\pgfsetbuttcap%
\pgfsetroundjoin%
\definecolor{currentfill}{rgb}{0.000000,0.000000,0.000000}%
\pgfsetfillcolor{currentfill}%
\pgfsetlinewidth{1.003750pt}%
\definecolor{currentstroke}{rgb}{0.000000,0.000000,0.000000}%
\pgfsetstrokecolor{currentstroke}%
\pgfsetdash{}{0pt}%
\pgfpathmoveto{\pgfqpoint{2.145481in}{2.334333in}}%
\pgfpathcurveto{\pgfqpoint{2.156531in}{2.334333in}}{\pgfqpoint{2.167130in}{2.338724in}}{\pgfqpoint{2.174944in}{2.346537in}}%
\pgfpathcurveto{\pgfqpoint{2.182758in}{2.354351in}}{\pgfqpoint{2.187148in}{2.364950in}}{\pgfqpoint{2.187148in}{2.376000in}}%
\pgfpathcurveto{\pgfqpoint{2.187148in}{2.387050in}}{\pgfqpoint{2.182758in}{2.397649in}}{\pgfqpoint{2.174944in}{2.405463in}}%
\pgfpathcurveto{\pgfqpoint{2.167130in}{2.413276in}}{\pgfqpoint{2.156531in}{2.417667in}}{\pgfqpoint{2.145481in}{2.417667in}}%
\pgfpathcurveto{\pgfqpoint{2.134431in}{2.417667in}}{\pgfqpoint{2.123832in}{2.413276in}}{\pgfqpoint{2.116018in}{2.405463in}}%
\pgfpathcurveto{\pgfqpoint{2.108205in}{2.397649in}}{\pgfqpoint{2.103815in}{2.387050in}}{\pgfqpoint{2.103815in}{2.376000in}}%
\pgfpathcurveto{\pgfqpoint{2.103815in}{2.364950in}}{\pgfqpoint{2.108205in}{2.354351in}}{\pgfqpoint{2.116018in}{2.346537in}}%
\pgfpathcurveto{\pgfqpoint{2.123832in}{2.338724in}}{\pgfqpoint{2.134431in}{2.334333in}}{\pgfqpoint{2.145481in}{2.334333in}}%
\pgfpathclose%
\pgfusepath{stroke,fill}%
\end{pgfscope}%
\begin{pgfscope}%
\pgfpathrectangle{\pgfqpoint{0.800000in}{0.528000in}}{\pgfqpoint{4.960000in}{3.696000in}}%
\pgfusepath{clip}%
\pgfsetbuttcap%
\pgfsetroundjoin%
\definecolor{currentfill}{rgb}{0.000000,0.000000,0.000000}%
\pgfsetfillcolor{currentfill}%
\pgfsetlinewidth{1.003750pt}%
\definecolor{currentstroke}{rgb}{0.000000,0.000000,0.000000}%
\pgfsetstrokecolor{currentstroke}%
\pgfsetdash{}{0pt}%
\pgfpathmoveto{\pgfqpoint{2.145481in}{2.334333in}}%
\pgfpathcurveto{\pgfqpoint{2.156531in}{2.334333in}}{\pgfqpoint{2.167130in}{2.338724in}}{\pgfqpoint{2.174944in}{2.346537in}}%
\pgfpathcurveto{\pgfqpoint{2.182758in}{2.354351in}}{\pgfqpoint{2.187148in}{2.364950in}}{\pgfqpoint{2.187148in}{2.376000in}}%
\pgfpathcurveto{\pgfqpoint{2.187148in}{2.387050in}}{\pgfqpoint{2.182758in}{2.397649in}}{\pgfqpoint{2.174944in}{2.405463in}}%
\pgfpathcurveto{\pgfqpoint{2.167130in}{2.413276in}}{\pgfqpoint{2.156531in}{2.417667in}}{\pgfqpoint{2.145481in}{2.417667in}}%
\pgfpathcurveto{\pgfqpoint{2.134431in}{2.417667in}}{\pgfqpoint{2.123832in}{2.413276in}}{\pgfqpoint{2.116018in}{2.405463in}}%
\pgfpathcurveto{\pgfqpoint{2.108205in}{2.397649in}}{\pgfqpoint{2.103815in}{2.387050in}}{\pgfqpoint{2.103815in}{2.376000in}}%
\pgfpathcurveto{\pgfqpoint{2.103815in}{2.364950in}}{\pgfqpoint{2.108205in}{2.354351in}}{\pgfqpoint{2.116018in}{2.346537in}}%
\pgfpathcurveto{\pgfqpoint{2.123832in}{2.338724in}}{\pgfqpoint{2.134431in}{2.334333in}}{\pgfqpoint{2.145481in}{2.334333in}}%
\pgfpathclose%
\pgfusepath{stroke,fill}%
\end{pgfscope}%
\begin{pgfscope}%
\pgfpathrectangle{\pgfqpoint{0.800000in}{0.528000in}}{\pgfqpoint{4.960000in}{3.696000in}}%
\pgfusepath{clip}%
\pgfsetbuttcap%
\pgfsetroundjoin%
\definecolor{currentfill}{rgb}{0.000000,0.000000,0.000000}%
\pgfsetfillcolor{currentfill}%
\pgfsetlinewidth{1.003750pt}%
\definecolor{currentstroke}{rgb}{0.000000,0.000000,0.000000}%
\pgfsetstrokecolor{currentstroke}%
\pgfsetdash{}{0pt}%
\pgfpathmoveto{\pgfqpoint{2.145481in}{2.334333in}}%
\pgfpathcurveto{\pgfqpoint{2.156531in}{2.334333in}}{\pgfqpoint{2.167130in}{2.338724in}}{\pgfqpoint{2.174944in}{2.346537in}}%
\pgfpathcurveto{\pgfqpoint{2.182758in}{2.354351in}}{\pgfqpoint{2.187148in}{2.364950in}}{\pgfqpoint{2.187148in}{2.376000in}}%
\pgfpathcurveto{\pgfqpoint{2.187148in}{2.387050in}}{\pgfqpoint{2.182758in}{2.397649in}}{\pgfqpoint{2.174944in}{2.405463in}}%
\pgfpathcurveto{\pgfqpoint{2.167130in}{2.413276in}}{\pgfqpoint{2.156531in}{2.417667in}}{\pgfqpoint{2.145481in}{2.417667in}}%
\pgfpathcurveto{\pgfqpoint{2.134431in}{2.417667in}}{\pgfqpoint{2.123832in}{2.413276in}}{\pgfqpoint{2.116018in}{2.405463in}}%
\pgfpathcurveto{\pgfqpoint{2.108205in}{2.397649in}}{\pgfqpoint{2.103815in}{2.387050in}}{\pgfqpoint{2.103815in}{2.376000in}}%
\pgfpathcurveto{\pgfqpoint{2.103815in}{2.364950in}}{\pgfqpoint{2.108205in}{2.354351in}}{\pgfqpoint{2.116018in}{2.346537in}}%
\pgfpathcurveto{\pgfqpoint{2.123832in}{2.338724in}}{\pgfqpoint{2.134431in}{2.334333in}}{\pgfqpoint{2.145481in}{2.334333in}}%
\pgfpathclose%
\pgfusepath{stroke,fill}%
\end{pgfscope}%
\begin{pgfscope}%
\pgfpathrectangle{\pgfqpoint{0.800000in}{0.528000in}}{\pgfqpoint{4.960000in}{3.696000in}}%
\pgfusepath{clip}%
\pgfsetbuttcap%
\pgfsetroundjoin%
\definecolor{currentfill}{rgb}{0.000000,0.000000,0.000000}%
\pgfsetfillcolor{currentfill}%
\pgfsetlinewidth{1.003750pt}%
\definecolor{currentstroke}{rgb}{0.000000,0.000000,0.000000}%
\pgfsetstrokecolor{currentstroke}%
\pgfsetdash{}{0pt}%
\pgfpathmoveto{\pgfqpoint{2.145481in}{2.334333in}}%
\pgfpathcurveto{\pgfqpoint{2.156531in}{2.334333in}}{\pgfqpoint{2.167130in}{2.338724in}}{\pgfqpoint{2.174944in}{2.346537in}}%
\pgfpathcurveto{\pgfqpoint{2.182758in}{2.354351in}}{\pgfqpoint{2.187148in}{2.364950in}}{\pgfqpoint{2.187148in}{2.376000in}}%
\pgfpathcurveto{\pgfqpoint{2.187148in}{2.387050in}}{\pgfqpoint{2.182758in}{2.397649in}}{\pgfqpoint{2.174944in}{2.405463in}}%
\pgfpathcurveto{\pgfqpoint{2.167130in}{2.413276in}}{\pgfqpoint{2.156531in}{2.417667in}}{\pgfqpoint{2.145481in}{2.417667in}}%
\pgfpathcurveto{\pgfqpoint{2.134431in}{2.417667in}}{\pgfqpoint{2.123832in}{2.413276in}}{\pgfqpoint{2.116018in}{2.405463in}}%
\pgfpathcurveto{\pgfqpoint{2.108205in}{2.397649in}}{\pgfqpoint{2.103815in}{2.387050in}}{\pgfqpoint{2.103815in}{2.376000in}}%
\pgfpathcurveto{\pgfqpoint{2.103815in}{2.364950in}}{\pgfqpoint{2.108205in}{2.354351in}}{\pgfqpoint{2.116018in}{2.346537in}}%
\pgfpathcurveto{\pgfqpoint{2.123832in}{2.338724in}}{\pgfqpoint{2.134431in}{2.334333in}}{\pgfqpoint{2.145481in}{2.334333in}}%
\pgfpathclose%
\pgfusepath{stroke,fill}%
\end{pgfscope}%
\begin{pgfscope}%
\pgfpathrectangle{\pgfqpoint{0.800000in}{0.528000in}}{\pgfqpoint{4.960000in}{3.696000in}}%
\pgfusepath{clip}%
\pgfsetbuttcap%
\pgfsetroundjoin%
\definecolor{currentfill}{rgb}{0.000000,0.000000,0.000000}%
\pgfsetfillcolor{currentfill}%
\pgfsetlinewidth{1.003750pt}%
\definecolor{currentstroke}{rgb}{0.000000,0.000000,0.000000}%
\pgfsetstrokecolor{currentstroke}%
\pgfsetdash{}{0pt}%
\pgfpathmoveto{\pgfqpoint{2.145481in}{2.334333in}}%
\pgfpathcurveto{\pgfqpoint{2.156531in}{2.334333in}}{\pgfqpoint{2.167130in}{2.338724in}}{\pgfqpoint{2.174944in}{2.346537in}}%
\pgfpathcurveto{\pgfqpoint{2.182758in}{2.354351in}}{\pgfqpoint{2.187148in}{2.364950in}}{\pgfqpoint{2.187148in}{2.376000in}}%
\pgfpathcurveto{\pgfqpoint{2.187148in}{2.387050in}}{\pgfqpoint{2.182758in}{2.397649in}}{\pgfqpoint{2.174944in}{2.405463in}}%
\pgfpathcurveto{\pgfqpoint{2.167130in}{2.413276in}}{\pgfqpoint{2.156531in}{2.417667in}}{\pgfqpoint{2.145481in}{2.417667in}}%
\pgfpathcurveto{\pgfqpoint{2.134431in}{2.417667in}}{\pgfqpoint{2.123832in}{2.413276in}}{\pgfqpoint{2.116018in}{2.405463in}}%
\pgfpathcurveto{\pgfqpoint{2.108205in}{2.397649in}}{\pgfqpoint{2.103815in}{2.387050in}}{\pgfqpoint{2.103815in}{2.376000in}}%
\pgfpathcurveto{\pgfqpoint{2.103815in}{2.364950in}}{\pgfqpoint{2.108205in}{2.354351in}}{\pgfqpoint{2.116018in}{2.346537in}}%
\pgfpathcurveto{\pgfqpoint{2.123832in}{2.338724in}}{\pgfqpoint{2.134431in}{2.334333in}}{\pgfqpoint{2.145481in}{2.334333in}}%
\pgfpathclose%
\pgfusepath{stroke,fill}%
\end{pgfscope}%
\begin{pgfscope}%
\pgfpathrectangle{\pgfqpoint{0.800000in}{0.528000in}}{\pgfqpoint{4.960000in}{3.696000in}}%
\pgfusepath{clip}%
\pgfsetbuttcap%
\pgfsetroundjoin%
\definecolor{currentfill}{rgb}{0.000000,0.000000,0.000000}%
\pgfsetfillcolor{currentfill}%
\pgfsetlinewidth{1.003750pt}%
\definecolor{currentstroke}{rgb}{0.000000,0.000000,0.000000}%
\pgfsetstrokecolor{currentstroke}%
\pgfsetdash{}{0pt}%
\pgfpathmoveto{\pgfqpoint{2.145481in}{2.334333in}}%
\pgfpathcurveto{\pgfqpoint{2.156531in}{2.334333in}}{\pgfqpoint{2.167130in}{2.338724in}}{\pgfqpoint{2.174944in}{2.346537in}}%
\pgfpathcurveto{\pgfqpoint{2.182758in}{2.354351in}}{\pgfqpoint{2.187148in}{2.364950in}}{\pgfqpoint{2.187148in}{2.376000in}}%
\pgfpathcurveto{\pgfqpoint{2.187148in}{2.387050in}}{\pgfqpoint{2.182758in}{2.397649in}}{\pgfqpoint{2.174944in}{2.405463in}}%
\pgfpathcurveto{\pgfqpoint{2.167130in}{2.413276in}}{\pgfqpoint{2.156531in}{2.417667in}}{\pgfqpoint{2.145481in}{2.417667in}}%
\pgfpathcurveto{\pgfqpoint{2.134431in}{2.417667in}}{\pgfqpoint{2.123832in}{2.413276in}}{\pgfqpoint{2.116018in}{2.405463in}}%
\pgfpathcurveto{\pgfqpoint{2.108205in}{2.397649in}}{\pgfqpoint{2.103815in}{2.387050in}}{\pgfqpoint{2.103815in}{2.376000in}}%
\pgfpathcurveto{\pgfqpoint{2.103815in}{2.364950in}}{\pgfqpoint{2.108205in}{2.354351in}}{\pgfqpoint{2.116018in}{2.346537in}}%
\pgfpathcurveto{\pgfqpoint{2.123832in}{2.338724in}}{\pgfqpoint{2.134431in}{2.334333in}}{\pgfqpoint{2.145481in}{2.334333in}}%
\pgfpathclose%
\pgfusepath{stroke,fill}%
\end{pgfscope}%
\begin{pgfscope}%
\pgfpathrectangle{\pgfqpoint{0.800000in}{0.528000in}}{\pgfqpoint{4.960000in}{3.696000in}}%
\pgfusepath{clip}%
\pgfsetbuttcap%
\pgfsetroundjoin%
\definecolor{currentfill}{rgb}{0.000000,0.000000,0.000000}%
\pgfsetfillcolor{currentfill}%
\pgfsetlinewidth{1.003750pt}%
\definecolor{currentstroke}{rgb}{0.000000,0.000000,0.000000}%
\pgfsetstrokecolor{currentstroke}%
\pgfsetdash{}{0pt}%
\pgfpathmoveto{\pgfqpoint{2.145481in}{2.334333in}}%
\pgfpathcurveto{\pgfqpoint{2.156531in}{2.334333in}}{\pgfqpoint{2.167130in}{2.338724in}}{\pgfqpoint{2.174944in}{2.346537in}}%
\pgfpathcurveto{\pgfqpoint{2.182758in}{2.354351in}}{\pgfqpoint{2.187148in}{2.364950in}}{\pgfqpoint{2.187148in}{2.376000in}}%
\pgfpathcurveto{\pgfqpoint{2.187148in}{2.387050in}}{\pgfqpoint{2.182758in}{2.397649in}}{\pgfqpoint{2.174944in}{2.405463in}}%
\pgfpathcurveto{\pgfqpoint{2.167130in}{2.413276in}}{\pgfqpoint{2.156531in}{2.417667in}}{\pgfqpoint{2.145481in}{2.417667in}}%
\pgfpathcurveto{\pgfqpoint{2.134431in}{2.417667in}}{\pgfqpoint{2.123832in}{2.413276in}}{\pgfqpoint{2.116018in}{2.405463in}}%
\pgfpathcurveto{\pgfqpoint{2.108205in}{2.397649in}}{\pgfqpoint{2.103815in}{2.387050in}}{\pgfqpoint{2.103815in}{2.376000in}}%
\pgfpathcurveto{\pgfqpoint{2.103815in}{2.364950in}}{\pgfqpoint{2.108205in}{2.354351in}}{\pgfqpoint{2.116018in}{2.346537in}}%
\pgfpathcurveto{\pgfqpoint{2.123832in}{2.338724in}}{\pgfqpoint{2.134431in}{2.334333in}}{\pgfqpoint{2.145481in}{2.334333in}}%
\pgfpathclose%
\pgfusepath{stroke,fill}%
\end{pgfscope}%
\begin{pgfscope}%
\pgfpathrectangle{\pgfqpoint{0.800000in}{0.528000in}}{\pgfqpoint{4.960000in}{3.696000in}}%
\pgfusepath{clip}%
\pgfsetbuttcap%
\pgfsetroundjoin%
\definecolor{currentfill}{rgb}{0.000000,0.000000,0.000000}%
\pgfsetfillcolor{currentfill}%
\pgfsetlinewidth{1.003750pt}%
\definecolor{currentstroke}{rgb}{0.000000,0.000000,0.000000}%
\pgfsetstrokecolor{currentstroke}%
\pgfsetdash{}{0pt}%
\pgfpathmoveto{\pgfqpoint{2.145481in}{2.334333in}}%
\pgfpathcurveto{\pgfqpoint{2.156531in}{2.334333in}}{\pgfqpoint{2.167130in}{2.338724in}}{\pgfqpoint{2.174944in}{2.346537in}}%
\pgfpathcurveto{\pgfqpoint{2.182758in}{2.354351in}}{\pgfqpoint{2.187148in}{2.364950in}}{\pgfqpoint{2.187148in}{2.376000in}}%
\pgfpathcurveto{\pgfqpoint{2.187148in}{2.387050in}}{\pgfqpoint{2.182758in}{2.397649in}}{\pgfqpoint{2.174944in}{2.405463in}}%
\pgfpathcurveto{\pgfqpoint{2.167130in}{2.413276in}}{\pgfqpoint{2.156531in}{2.417667in}}{\pgfqpoint{2.145481in}{2.417667in}}%
\pgfpathcurveto{\pgfqpoint{2.134431in}{2.417667in}}{\pgfqpoint{2.123832in}{2.413276in}}{\pgfqpoint{2.116018in}{2.405463in}}%
\pgfpathcurveto{\pgfqpoint{2.108205in}{2.397649in}}{\pgfqpoint{2.103815in}{2.387050in}}{\pgfqpoint{2.103815in}{2.376000in}}%
\pgfpathcurveto{\pgfqpoint{2.103815in}{2.364950in}}{\pgfqpoint{2.108205in}{2.354351in}}{\pgfqpoint{2.116018in}{2.346537in}}%
\pgfpathcurveto{\pgfqpoint{2.123832in}{2.338724in}}{\pgfqpoint{2.134431in}{2.334333in}}{\pgfqpoint{2.145481in}{2.334333in}}%
\pgfpathclose%
\pgfusepath{stroke,fill}%
\end{pgfscope}%
\begin{pgfscope}%
\pgfpathrectangle{\pgfqpoint{0.800000in}{0.528000in}}{\pgfqpoint{4.960000in}{3.696000in}}%
\pgfusepath{clip}%
\pgfsetbuttcap%
\pgfsetroundjoin%
\definecolor{currentfill}{rgb}{0.000000,0.000000,0.000000}%
\pgfsetfillcolor{currentfill}%
\pgfsetlinewidth{1.003750pt}%
\definecolor{currentstroke}{rgb}{0.000000,0.000000,0.000000}%
\pgfsetstrokecolor{currentstroke}%
\pgfsetdash{}{0pt}%
\pgfpathmoveto{\pgfqpoint{2.145481in}{2.334333in}}%
\pgfpathcurveto{\pgfqpoint{2.156531in}{2.334333in}}{\pgfqpoint{2.167130in}{2.338724in}}{\pgfqpoint{2.174944in}{2.346537in}}%
\pgfpathcurveto{\pgfqpoint{2.182758in}{2.354351in}}{\pgfqpoint{2.187148in}{2.364950in}}{\pgfqpoint{2.187148in}{2.376000in}}%
\pgfpathcurveto{\pgfqpoint{2.187148in}{2.387050in}}{\pgfqpoint{2.182758in}{2.397649in}}{\pgfqpoint{2.174944in}{2.405463in}}%
\pgfpathcurveto{\pgfqpoint{2.167130in}{2.413276in}}{\pgfqpoint{2.156531in}{2.417667in}}{\pgfqpoint{2.145481in}{2.417667in}}%
\pgfpathcurveto{\pgfqpoint{2.134431in}{2.417667in}}{\pgfqpoint{2.123832in}{2.413276in}}{\pgfqpoint{2.116018in}{2.405463in}}%
\pgfpathcurveto{\pgfqpoint{2.108205in}{2.397649in}}{\pgfqpoint{2.103815in}{2.387050in}}{\pgfqpoint{2.103815in}{2.376000in}}%
\pgfpathcurveto{\pgfqpoint{2.103815in}{2.364950in}}{\pgfqpoint{2.108205in}{2.354351in}}{\pgfqpoint{2.116018in}{2.346537in}}%
\pgfpathcurveto{\pgfqpoint{2.123832in}{2.338724in}}{\pgfqpoint{2.134431in}{2.334333in}}{\pgfqpoint{2.145481in}{2.334333in}}%
\pgfpathclose%
\pgfusepath{stroke,fill}%
\end{pgfscope}%
\begin{pgfscope}%
\pgfpathrectangle{\pgfqpoint{0.800000in}{0.528000in}}{\pgfqpoint{4.960000in}{3.696000in}}%
\pgfusepath{clip}%
\pgfsetbuttcap%
\pgfsetroundjoin%
\definecolor{currentfill}{rgb}{0.000000,0.000000,0.000000}%
\pgfsetfillcolor{currentfill}%
\pgfsetlinewidth{1.003750pt}%
\definecolor{currentstroke}{rgb}{0.000000,0.000000,0.000000}%
\pgfsetstrokecolor{currentstroke}%
\pgfsetdash{}{0pt}%
\pgfpathmoveto{\pgfqpoint{2.145481in}{2.334333in}}%
\pgfpathcurveto{\pgfqpoint{2.156531in}{2.334333in}}{\pgfqpoint{2.167130in}{2.338724in}}{\pgfqpoint{2.174944in}{2.346537in}}%
\pgfpathcurveto{\pgfqpoint{2.182758in}{2.354351in}}{\pgfqpoint{2.187148in}{2.364950in}}{\pgfqpoint{2.187148in}{2.376000in}}%
\pgfpathcurveto{\pgfqpoint{2.187148in}{2.387050in}}{\pgfqpoint{2.182758in}{2.397649in}}{\pgfqpoint{2.174944in}{2.405463in}}%
\pgfpathcurveto{\pgfqpoint{2.167130in}{2.413276in}}{\pgfqpoint{2.156531in}{2.417667in}}{\pgfqpoint{2.145481in}{2.417667in}}%
\pgfpathcurveto{\pgfqpoint{2.134431in}{2.417667in}}{\pgfqpoint{2.123832in}{2.413276in}}{\pgfqpoint{2.116018in}{2.405463in}}%
\pgfpathcurveto{\pgfqpoint{2.108205in}{2.397649in}}{\pgfqpoint{2.103815in}{2.387050in}}{\pgfqpoint{2.103815in}{2.376000in}}%
\pgfpathcurveto{\pgfqpoint{2.103815in}{2.364950in}}{\pgfqpoint{2.108205in}{2.354351in}}{\pgfqpoint{2.116018in}{2.346537in}}%
\pgfpathcurveto{\pgfqpoint{2.123832in}{2.338724in}}{\pgfqpoint{2.134431in}{2.334333in}}{\pgfqpoint{2.145481in}{2.334333in}}%
\pgfpathclose%
\pgfusepath{stroke,fill}%
\end{pgfscope}%
\begin{pgfscope}%
\pgfpathrectangle{\pgfqpoint{0.800000in}{0.528000in}}{\pgfqpoint{4.960000in}{3.696000in}}%
\pgfusepath{clip}%
\pgfsetbuttcap%
\pgfsetroundjoin%
\definecolor{currentfill}{rgb}{0.000000,0.000000,0.000000}%
\pgfsetfillcolor{currentfill}%
\pgfsetlinewidth{1.003750pt}%
\definecolor{currentstroke}{rgb}{0.000000,0.000000,0.000000}%
\pgfsetstrokecolor{currentstroke}%
\pgfsetdash{}{0pt}%
\pgfpathmoveto{\pgfqpoint{2.145481in}{2.334333in}}%
\pgfpathcurveto{\pgfqpoint{2.156531in}{2.334333in}}{\pgfqpoint{2.167130in}{2.338724in}}{\pgfqpoint{2.174944in}{2.346537in}}%
\pgfpathcurveto{\pgfqpoint{2.182758in}{2.354351in}}{\pgfqpoint{2.187148in}{2.364950in}}{\pgfqpoint{2.187148in}{2.376000in}}%
\pgfpathcurveto{\pgfqpoint{2.187148in}{2.387050in}}{\pgfqpoint{2.182758in}{2.397649in}}{\pgfqpoint{2.174944in}{2.405463in}}%
\pgfpathcurveto{\pgfqpoint{2.167130in}{2.413276in}}{\pgfqpoint{2.156531in}{2.417667in}}{\pgfqpoint{2.145481in}{2.417667in}}%
\pgfpathcurveto{\pgfqpoint{2.134431in}{2.417667in}}{\pgfqpoint{2.123832in}{2.413276in}}{\pgfqpoint{2.116018in}{2.405463in}}%
\pgfpathcurveto{\pgfqpoint{2.108205in}{2.397649in}}{\pgfqpoint{2.103815in}{2.387050in}}{\pgfqpoint{2.103815in}{2.376000in}}%
\pgfpathcurveto{\pgfqpoint{2.103815in}{2.364950in}}{\pgfqpoint{2.108205in}{2.354351in}}{\pgfqpoint{2.116018in}{2.346537in}}%
\pgfpathcurveto{\pgfqpoint{2.123832in}{2.338724in}}{\pgfqpoint{2.134431in}{2.334333in}}{\pgfqpoint{2.145481in}{2.334333in}}%
\pgfpathclose%
\pgfusepath{stroke,fill}%
\end{pgfscope}%
\begin{pgfscope}%
\pgfpathrectangle{\pgfqpoint{0.800000in}{0.528000in}}{\pgfqpoint{4.960000in}{3.696000in}}%
\pgfusepath{clip}%
\pgfsetbuttcap%
\pgfsetroundjoin%
\definecolor{currentfill}{rgb}{0.000000,0.000000,0.000000}%
\pgfsetfillcolor{currentfill}%
\pgfsetlinewidth{1.003750pt}%
\definecolor{currentstroke}{rgb}{0.000000,0.000000,0.000000}%
\pgfsetstrokecolor{currentstroke}%
\pgfsetdash{}{0pt}%
\pgfpathmoveto{\pgfqpoint{2.145481in}{2.334333in}}%
\pgfpathcurveto{\pgfqpoint{2.156531in}{2.334333in}}{\pgfqpoint{2.167130in}{2.338724in}}{\pgfqpoint{2.174944in}{2.346537in}}%
\pgfpathcurveto{\pgfqpoint{2.182758in}{2.354351in}}{\pgfqpoint{2.187148in}{2.364950in}}{\pgfqpoint{2.187148in}{2.376000in}}%
\pgfpathcurveto{\pgfqpoint{2.187148in}{2.387050in}}{\pgfqpoint{2.182758in}{2.397649in}}{\pgfqpoint{2.174944in}{2.405463in}}%
\pgfpathcurveto{\pgfqpoint{2.167130in}{2.413276in}}{\pgfqpoint{2.156531in}{2.417667in}}{\pgfqpoint{2.145481in}{2.417667in}}%
\pgfpathcurveto{\pgfqpoint{2.134431in}{2.417667in}}{\pgfqpoint{2.123832in}{2.413276in}}{\pgfqpoint{2.116018in}{2.405463in}}%
\pgfpathcurveto{\pgfqpoint{2.108205in}{2.397649in}}{\pgfqpoint{2.103815in}{2.387050in}}{\pgfqpoint{2.103815in}{2.376000in}}%
\pgfpathcurveto{\pgfqpoint{2.103815in}{2.364950in}}{\pgfqpoint{2.108205in}{2.354351in}}{\pgfqpoint{2.116018in}{2.346537in}}%
\pgfpathcurveto{\pgfqpoint{2.123832in}{2.338724in}}{\pgfqpoint{2.134431in}{2.334333in}}{\pgfqpoint{2.145481in}{2.334333in}}%
\pgfpathclose%
\pgfusepath{stroke,fill}%
\end{pgfscope}%
\begin{pgfscope}%
\pgfpathrectangle{\pgfqpoint{0.800000in}{0.528000in}}{\pgfqpoint{4.960000in}{3.696000in}}%
\pgfusepath{clip}%
\pgfsetbuttcap%
\pgfsetroundjoin%
\definecolor{currentfill}{rgb}{0.000000,0.000000,0.000000}%
\pgfsetfillcolor{currentfill}%
\pgfsetlinewidth{1.003750pt}%
\definecolor{currentstroke}{rgb}{0.000000,0.000000,0.000000}%
\pgfsetstrokecolor{currentstroke}%
\pgfsetdash{}{0pt}%
\pgfpathmoveto{\pgfqpoint{2.145481in}{2.334333in}}%
\pgfpathcurveto{\pgfqpoint{2.156531in}{2.334333in}}{\pgfqpoint{2.167130in}{2.338724in}}{\pgfqpoint{2.174944in}{2.346537in}}%
\pgfpathcurveto{\pgfqpoint{2.182758in}{2.354351in}}{\pgfqpoint{2.187148in}{2.364950in}}{\pgfqpoint{2.187148in}{2.376000in}}%
\pgfpathcurveto{\pgfqpoint{2.187148in}{2.387050in}}{\pgfqpoint{2.182758in}{2.397649in}}{\pgfqpoint{2.174944in}{2.405463in}}%
\pgfpathcurveto{\pgfqpoint{2.167130in}{2.413276in}}{\pgfqpoint{2.156531in}{2.417667in}}{\pgfqpoint{2.145481in}{2.417667in}}%
\pgfpathcurveto{\pgfqpoint{2.134431in}{2.417667in}}{\pgfqpoint{2.123832in}{2.413276in}}{\pgfqpoint{2.116018in}{2.405463in}}%
\pgfpathcurveto{\pgfqpoint{2.108205in}{2.397649in}}{\pgfqpoint{2.103815in}{2.387050in}}{\pgfqpoint{2.103815in}{2.376000in}}%
\pgfpathcurveto{\pgfqpoint{2.103815in}{2.364950in}}{\pgfqpoint{2.108205in}{2.354351in}}{\pgfqpoint{2.116018in}{2.346537in}}%
\pgfpathcurveto{\pgfqpoint{2.123832in}{2.338724in}}{\pgfqpoint{2.134431in}{2.334333in}}{\pgfqpoint{2.145481in}{2.334333in}}%
\pgfpathclose%
\pgfusepath{stroke,fill}%
\end{pgfscope}%
\begin{pgfscope}%
\pgfpathrectangle{\pgfqpoint{0.800000in}{0.528000in}}{\pgfqpoint{4.960000in}{3.696000in}}%
\pgfusepath{clip}%
\pgfsetbuttcap%
\pgfsetroundjoin%
\definecolor{currentfill}{rgb}{0.000000,0.000000,0.000000}%
\pgfsetfillcolor{currentfill}%
\pgfsetlinewidth{1.003750pt}%
\definecolor{currentstroke}{rgb}{0.000000,0.000000,0.000000}%
\pgfsetstrokecolor{currentstroke}%
\pgfsetdash{}{0pt}%
\pgfpathmoveto{\pgfqpoint{2.145481in}{2.334333in}}%
\pgfpathcurveto{\pgfqpoint{2.156531in}{2.334333in}}{\pgfqpoint{2.167130in}{2.338724in}}{\pgfqpoint{2.174944in}{2.346537in}}%
\pgfpathcurveto{\pgfqpoint{2.182758in}{2.354351in}}{\pgfqpoint{2.187148in}{2.364950in}}{\pgfqpoint{2.187148in}{2.376000in}}%
\pgfpathcurveto{\pgfqpoint{2.187148in}{2.387050in}}{\pgfqpoint{2.182758in}{2.397649in}}{\pgfqpoint{2.174944in}{2.405463in}}%
\pgfpathcurveto{\pgfqpoint{2.167130in}{2.413276in}}{\pgfqpoint{2.156531in}{2.417667in}}{\pgfqpoint{2.145481in}{2.417667in}}%
\pgfpathcurveto{\pgfqpoint{2.134431in}{2.417667in}}{\pgfqpoint{2.123832in}{2.413276in}}{\pgfqpoint{2.116018in}{2.405463in}}%
\pgfpathcurveto{\pgfqpoint{2.108205in}{2.397649in}}{\pgfqpoint{2.103815in}{2.387050in}}{\pgfqpoint{2.103815in}{2.376000in}}%
\pgfpathcurveto{\pgfqpoint{2.103815in}{2.364950in}}{\pgfqpoint{2.108205in}{2.354351in}}{\pgfqpoint{2.116018in}{2.346537in}}%
\pgfpathcurveto{\pgfqpoint{2.123832in}{2.338724in}}{\pgfqpoint{2.134431in}{2.334333in}}{\pgfqpoint{2.145481in}{2.334333in}}%
\pgfpathclose%
\pgfusepath{stroke,fill}%
\end{pgfscope}%
\begin{pgfscope}%
\pgfpathrectangle{\pgfqpoint{0.800000in}{0.528000in}}{\pgfqpoint{4.960000in}{3.696000in}}%
\pgfusepath{clip}%
\pgfsetbuttcap%
\pgfsetroundjoin%
\definecolor{currentfill}{rgb}{0.000000,0.000000,0.000000}%
\pgfsetfillcolor{currentfill}%
\pgfsetlinewidth{1.003750pt}%
\definecolor{currentstroke}{rgb}{0.000000,0.000000,0.000000}%
\pgfsetstrokecolor{currentstroke}%
\pgfsetdash{}{0pt}%
\pgfpathmoveto{\pgfqpoint{2.145481in}{2.334333in}}%
\pgfpathcurveto{\pgfqpoint{2.156531in}{2.334333in}}{\pgfqpoint{2.167130in}{2.338724in}}{\pgfqpoint{2.174944in}{2.346537in}}%
\pgfpathcurveto{\pgfqpoint{2.182758in}{2.354351in}}{\pgfqpoint{2.187148in}{2.364950in}}{\pgfqpoint{2.187148in}{2.376000in}}%
\pgfpathcurveto{\pgfqpoint{2.187148in}{2.387050in}}{\pgfqpoint{2.182758in}{2.397649in}}{\pgfqpoint{2.174944in}{2.405463in}}%
\pgfpathcurveto{\pgfqpoint{2.167130in}{2.413276in}}{\pgfqpoint{2.156531in}{2.417667in}}{\pgfqpoint{2.145481in}{2.417667in}}%
\pgfpathcurveto{\pgfqpoint{2.134431in}{2.417667in}}{\pgfqpoint{2.123832in}{2.413276in}}{\pgfqpoint{2.116018in}{2.405463in}}%
\pgfpathcurveto{\pgfqpoint{2.108205in}{2.397649in}}{\pgfqpoint{2.103815in}{2.387050in}}{\pgfqpoint{2.103815in}{2.376000in}}%
\pgfpathcurveto{\pgfqpoint{2.103815in}{2.364950in}}{\pgfqpoint{2.108205in}{2.354351in}}{\pgfqpoint{2.116018in}{2.346537in}}%
\pgfpathcurveto{\pgfqpoint{2.123832in}{2.338724in}}{\pgfqpoint{2.134431in}{2.334333in}}{\pgfqpoint{2.145481in}{2.334333in}}%
\pgfpathclose%
\pgfusepath{stroke,fill}%
\end{pgfscope}%
\begin{pgfscope}%
\pgfpathrectangle{\pgfqpoint{0.800000in}{0.528000in}}{\pgfqpoint{4.960000in}{3.696000in}}%
\pgfusepath{clip}%
\pgfsetbuttcap%
\pgfsetroundjoin%
\definecolor{currentfill}{rgb}{0.000000,0.000000,0.000000}%
\pgfsetfillcolor{currentfill}%
\pgfsetlinewidth{1.003750pt}%
\definecolor{currentstroke}{rgb}{0.000000,0.000000,0.000000}%
\pgfsetstrokecolor{currentstroke}%
\pgfsetdash{}{0pt}%
\pgfpathmoveto{\pgfqpoint{2.145481in}{2.334333in}}%
\pgfpathcurveto{\pgfqpoint{2.156531in}{2.334333in}}{\pgfqpoint{2.167130in}{2.338724in}}{\pgfqpoint{2.174944in}{2.346537in}}%
\pgfpathcurveto{\pgfqpoint{2.182758in}{2.354351in}}{\pgfqpoint{2.187148in}{2.364950in}}{\pgfqpoint{2.187148in}{2.376000in}}%
\pgfpathcurveto{\pgfqpoint{2.187148in}{2.387050in}}{\pgfqpoint{2.182758in}{2.397649in}}{\pgfqpoint{2.174944in}{2.405463in}}%
\pgfpathcurveto{\pgfqpoint{2.167130in}{2.413276in}}{\pgfqpoint{2.156531in}{2.417667in}}{\pgfqpoint{2.145481in}{2.417667in}}%
\pgfpathcurveto{\pgfqpoint{2.134431in}{2.417667in}}{\pgfqpoint{2.123832in}{2.413276in}}{\pgfqpoint{2.116018in}{2.405463in}}%
\pgfpathcurveto{\pgfqpoint{2.108205in}{2.397649in}}{\pgfqpoint{2.103815in}{2.387050in}}{\pgfqpoint{2.103815in}{2.376000in}}%
\pgfpathcurveto{\pgfqpoint{2.103815in}{2.364950in}}{\pgfqpoint{2.108205in}{2.354351in}}{\pgfqpoint{2.116018in}{2.346537in}}%
\pgfpathcurveto{\pgfqpoint{2.123832in}{2.338724in}}{\pgfqpoint{2.134431in}{2.334333in}}{\pgfqpoint{2.145481in}{2.334333in}}%
\pgfpathclose%
\pgfusepath{stroke,fill}%
\end{pgfscope}%
\begin{pgfscope}%
\pgfpathrectangle{\pgfqpoint{0.800000in}{0.528000in}}{\pgfqpoint{4.960000in}{3.696000in}}%
\pgfusepath{clip}%
\pgfsetbuttcap%
\pgfsetroundjoin%
\definecolor{currentfill}{rgb}{0.000000,0.000000,0.000000}%
\pgfsetfillcolor{currentfill}%
\pgfsetlinewidth{1.003750pt}%
\definecolor{currentstroke}{rgb}{0.000000,0.000000,0.000000}%
\pgfsetstrokecolor{currentstroke}%
\pgfsetdash{}{0pt}%
\pgfpathmoveto{\pgfqpoint{2.145481in}{2.334333in}}%
\pgfpathcurveto{\pgfqpoint{2.156531in}{2.334333in}}{\pgfqpoint{2.167130in}{2.338724in}}{\pgfqpoint{2.174944in}{2.346537in}}%
\pgfpathcurveto{\pgfqpoint{2.182758in}{2.354351in}}{\pgfqpoint{2.187148in}{2.364950in}}{\pgfqpoint{2.187148in}{2.376000in}}%
\pgfpathcurveto{\pgfqpoint{2.187148in}{2.387050in}}{\pgfqpoint{2.182758in}{2.397649in}}{\pgfqpoint{2.174944in}{2.405463in}}%
\pgfpathcurveto{\pgfqpoint{2.167130in}{2.413276in}}{\pgfqpoint{2.156531in}{2.417667in}}{\pgfqpoint{2.145481in}{2.417667in}}%
\pgfpathcurveto{\pgfqpoint{2.134431in}{2.417667in}}{\pgfqpoint{2.123832in}{2.413276in}}{\pgfqpoint{2.116018in}{2.405463in}}%
\pgfpathcurveto{\pgfqpoint{2.108205in}{2.397649in}}{\pgfqpoint{2.103815in}{2.387050in}}{\pgfqpoint{2.103815in}{2.376000in}}%
\pgfpathcurveto{\pgfqpoint{2.103815in}{2.364950in}}{\pgfqpoint{2.108205in}{2.354351in}}{\pgfqpoint{2.116018in}{2.346537in}}%
\pgfpathcurveto{\pgfqpoint{2.123832in}{2.338724in}}{\pgfqpoint{2.134431in}{2.334333in}}{\pgfqpoint{2.145481in}{2.334333in}}%
\pgfpathclose%
\pgfusepath{stroke,fill}%
\end{pgfscope}%
\begin{pgfscope}%
\pgfpathrectangle{\pgfqpoint{0.800000in}{0.528000in}}{\pgfqpoint{4.960000in}{3.696000in}}%
\pgfusepath{clip}%
\pgfsetbuttcap%
\pgfsetroundjoin%
\definecolor{currentfill}{rgb}{0.000000,0.000000,0.000000}%
\pgfsetfillcolor{currentfill}%
\pgfsetlinewidth{1.003750pt}%
\definecolor{currentstroke}{rgb}{0.000000,0.000000,0.000000}%
\pgfsetstrokecolor{currentstroke}%
\pgfsetdash{}{0pt}%
\pgfpathmoveto{\pgfqpoint{2.145481in}{2.334333in}}%
\pgfpathcurveto{\pgfqpoint{2.156531in}{2.334333in}}{\pgfqpoint{2.167130in}{2.338724in}}{\pgfqpoint{2.174944in}{2.346537in}}%
\pgfpathcurveto{\pgfqpoint{2.182758in}{2.354351in}}{\pgfqpoint{2.187148in}{2.364950in}}{\pgfqpoint{2.187148in}{2.376000in}}%
\pgfpathcurveto{\pgfqpoint{2.187148in}{2.387050in}}{\pgfqpoint{2.182758in}{2.397649in}}{\pgfqpoint{2.174944in}{2.405463in}}%
\pgfpathcurveto{\pgfqpoint{2.167130in}{2.413276in}}{\pgfqpoint{2.156531in}{2.417667in}}{\pgfqpoint{2.145481in}{2.417667in}}%
\pgfpathcurveto{\pgfqpoint{2.134431in}{2.417667in}}{\pgfqpoint{2.123832in}{2.413276in}}{\pgfqpoint{2.116018in}{2.405463in}}%
\pgfpathcurveto{\pgfqpoint{2.108205in}{2.397649in}}{\pgfqpoint{2.103815in}{2.387050in}}{\pgfqpoint{2.103815in}{2.376000in}}%
\pgfpathcurveto{\pgfqpoint{2.103815in}{2.364950in}}{\pgfqpoint{2.108205in}{2.354351in}}{\pgfqpoint{2.116018in}{2.346537in}}%
\pgfpathcurveto{\pgfqpoint{2.123832in}{2.338724in}}{\pgfqpoint{2.134431in}{2.334333in}}{\pgfqpoint{2.145481in}{2.334333in}}%
\pgfpathclose%
\pgfusepath{stroke,fill}%
\end{pgfscope}%
\begin{pgfscope}%
\pgfpathrectangle{\pgfqpoint{0.800000in}{0.528000in}}{\pgfqpoint{4.960000in}{3.696000in}}%
\pgfusepath{clip}%
\pgfsetbuttcap%
\pgfsetroundjoin%
\definecolor{currentfill}{rgb}{0.000000,0.000000,0.000000}%
\pgfsetfillcolor{currentfill}%
\pgfsetlinewidth{1.003750pt}%
\definecolor{currentstroke}{rgb}{0.000000,0.000000,0.000000}%
\pgfsetstrokecolor{currentstroke}%
\pgfsetdash{}{0pt}%
\pgfpathmoveto{\pgfqpoint{2.145481in}{2.334333in}}%
\pgfpathcurveto{\pgfqpoint{2.156531in}{2.334333in}}{\pgfqpoint{2.167130in}{2.338724in}}{\pgfqpoint{2.174944in}{2.346537in}}%
\pgfpathcurveto{\pgfqpoint{2.182758in}{2.354351in}}{\pgfqpoint{2.187148in}{2.364950in}}{\pgfqpoint{2.187148in}{2.376000in}}%
\pgfpathcurveto{\pgfqpoint{2.187148in}{2.387050in}}{\pgfqpoint{2.182758in}{2.397649in}}{\pgfqpoint{2.174944in}{2.405463in}}%
\pgfpathcurveto{\pgfqpoint{2.167130in}{2.413276in}}{\pgfqpoint{2.156531in}{2.417667in}}{\pgfqpoint{2.145481in}{2.417667in}}%
\pgfpathcurveto{\pgfqpoint{2.134431in}{2.417667in}}{\pgfqpoint{2.123832in}{2.413276in}}{\pgfqpoint{2.116018in}{2.405463in}}%
\pgfpathcurveto{\pgfqpoint{2.108205in}{2.397649in}}{\pgfqpoint{2.103815in}{2.387050in}}{\pgfqpoint{2.103815in}{2.376000in}}%
\pgfpathcurveto{\pgfqpoint{2.103815in}{2.364950in}}{\pgfqpoint{2.108205in}{2.354351in}}{\pgfqpoint{2.116018in}{2.346537in}}%
\pgfpathcurveto{\pgfqpoint{2.123832in}{2.338724in}}{\pgfqpoint{2.134431in}{2.334333in}}{\pgfqpoint{2.145481in}{2.334333in}}%
\pgfpathclose%
\pgfusepath{stroke,fill}%
\end{pgfscope}%
\begin{pgfscope}%
\pgfpathrectangle{\pgfqpoint{0.800000in}{0.528000in}}{\pgfqpoint{4.960000in}{3.696000in}}%
\pgfusepath{clip}%
\pgfsetbuttcap%
\pgfsetroundjoin%
\definecolor{currentfill}{rgb}{0.000000,0.000000,0.000000}%
\pgfsetfillcolor{currentfill}%
\pgfsetlinewidth{1.003750pt}%
\definecolor{currentstroke}{rgb}{0.000000,0.000000,0.000000}%
\pgfsetstrokecolor{currentstroke}%
\pgfsetdash{}{0pt}%
\pgfpathmoveto{\pgfqpoint{2.145481in}{2.334333in}}%
\pgfpathcurveto{\pgfqpoint{2.156531in}{2.334333in}}{\pgfqpoint{2.167130in}{2.338724in}}{\pgfqpoint{2.174944in}{2.346537in}}%
\pgfpathcurveto{\pgfqpoint{2.182758in}{2.354351in}}{\pgfqpoint{2.187148in}{2.364950in}}{\pgfqpoint{2.187148in}{2.376000in}}%
\pgfpathcurveto{\pgfqpoint{2.187148in}{2.387050in}}{\pgfqpoint{2.182758in}{2.397649in}}{\pgfqpoint{2.174944in}{2.405463in}}%
\pgfpathcurveto{\pgfqpoint{2.167130in}{2.413276in}}{\pgfqpoint{2.156531in}{2.417667in}}{\pgfqpoint{2.145481in}{2.417667in}}%
\pgfpathcurveto{\pgfqpoint{2.134431in}{2.417667in}}{\pgfqpoint{2.123832in}{2.413276in}}{\pgfqpoint{2.116018in}{2.405463in}}%
\pgfpathcurveto{\pgfqpoint{2.108205in}{2.397649in}}{\pgfqpoint{2.103815in}{2.387050in}}{\pgfqpoint{2.103815in}{2.376000in}}%
\pgfpathcurveto{\pgfqpoint{2.103815in}{2.364950in}}{\pgfqpoint{2.108205in}{2.354351in}}{\pgfqpoint{2.116018in}{2.346537in}}%
\pgfpathcurveto{\pgfqpoint{2.123832in}{2.338724in}}{\pgfqpoint{2.134431in}{2.334333in}}{\pgfqpoint{2.145481in}{2.334333in}}%
\pgfpathclose%
\pgfusepath{stroke,fill}%
\end{pgfscope}%
\begin{pgfscope}%
\pgfpathrectangle{\pgfqpoint{0.800000in}{0.528000in}}{\pgfqpoint{4.960000in}{3.696000in}}%
\pgfusepath{clip}%
\pgfsetbuttcap%
\pgfsetroundjoin%
\definecolor{currentfill}{rgb}{0.000000,0.000000,0.000000}%
\pgfsetfillcolor{currentfill}%
\pgfsetlinewidth{1.003750pt}%
\definecolor{currentstroke}{rgb}{0.000000,0.000000,0.000000}%
\pgfsetstrokecolor{currentstroke}%
\pgfsetdash{}{0pt}%
\pgfpathmoveto{\pgfqpoint{2.145481in}{2.334333in}}%
\pgfpathcurveto{\pgfqpoint{2.156531in}{2.334333in}}{\pgfqpoint{2.167130in}{2.338724in}}{\pgfqpoint{2.174944in}{2.346537in}}%
\pgfpathcurveto{\pgfqpoint{2.182758in}{2.354351in}}{\pgfqpoint{2.187148in}{2.364950in}}{\pgfqpoint{2.187148in}{2.376000in}}%
\pgfpathcurveto{\pgfqpoint{2.187148in}{2.387050in}}{\pgfqpoint{2.182758in}{2.397649in}}{\pgfqpoint{2.174944in}{2.405463in}}%
\pgfpathcurveto{\pgfqpoint{2.167130in}{2.413276in}}{\pgfqpoint{2.156531in}{2.417667in}}{\pgfqpoint{2.145481in}{2.417667in}}%
\pgfpathcurveto{\pgfqpoint{2.134431in}{2.417667in}}{\pgfqpoint{2.123832in}{2.413276in}}{\pgfqpoint{2.116018in}{2.405463in}}%
\pgfpathcurveto{\pgfqpoint{2.108205in}{2.397649in}}{\pgfqpoint{2.103815in}{2.387050in}}{\pgfqpoint{2.103815in}{2.376000in}}%
\pgfpathcurveto{\pgfqpoint{2.103815in}{2.364950in}}{\pgfqpoint{2.108205in}{2.354351in}}{\pgfqpoint{2.116018in}{2.346537in}}%
\pgfpathcurveto{\pgfqpoint{2.123832in}{2.338724in}}{\pgfqpoint{2.134431in}{2.334333in}}{\pgfqpoint{2.145481in}{2.334333in}}%
\pgfpathclose%
\pgfusepath{stroke,fill}%
\end{pgfscope}%
\begin{pgfscope}%
\pgfpathrectangle{\pgfqpoint{0.800000in}{0.528000in}}{\pgfqpoint{4.960000in}{3.696000in}}%
\pgfusepath{clip}%
\pgfsetbuttcap%
\pgfsetroundjoin%
\definecolor{currentfill}{rgb}{0.000000,0.000000,0.000000}%
\pgfsetfillcolor{currentfill}%
\pgfsetlinewidth{1.003750pt}%
\definecolor{currentstroke}{rgb}{0.000000,0.000000,0.000000}%
\pgfsetstrokecolor{currentstroke}%
\pgfsetdash{}{0pt}%
\pgfpathmoveto{\pgfqpoint{2.145481in}{2.334333in}}%
\pgfpathcurveto{\pgfqpoint{2.156531in}{2.334333in}}{\pgfqpoint{2.167130in}{2.338724in}}{\pgfqpoint{2.174944in}{2.346537in}}%
\pgfpathcurveto{\pgfqpoint{2.182758in}{2.354351in}}{\pgfqpoint{2.187148in}{2.364950in}}{\pgfqpoint{2.187148in}{2.376000in}}%
\pgfpathcurveto{\pgfqpoint{2.187148in}{2.387050in}}{\pgfqpoint{2.182758in}{2.397649in}}{\pgfqpoint{2.174944in}{2.405463in}}%
\pgfpathcurveto{\pgfqpoint{2.167130in}{2.413276in}}{\pgfqpoint{2.156531in}{2.417667in}}{\pgfqpoint{2.145481in}{2.417667in}}%
\pgfpathcurveto{\pgfqpoint{2.134431in}{2.417667in}}{\pgfqpoint{2.123832in}{2.413276in}}{\pgfqpoint{2.116018in}{2.405463in}}%
\pgfpathcurveto{\pgfqpoint{2.108205in}{2.397649in}}{\pgfqpoint{2.103815in}{2.387050in}}{\pgfqpoint{2.103815in}{2.376000in}}%
\pgfpathcurveto{\pgfqpoint{2.103815in}{2.364950in}}{\pgfqpoint{2.108205in}{2.354351in}}{\pgfqpoint{2.116018in}{2.346537in}}%
\pgfpathcurveto{\pgfqpoint{2.123832in}{2.338724in}}{\pgfqpoint{2.134431in}{2.334333in}}{\pgfqpoint{2.145481in}{2.334333in}}%
\pgfpathclose%
\pgfusepath{stroke,fill}%
\end{pgfscope}%
\begin{pgfscope}%
\pgfpathrectangle{\pgfqpoint{0.800000in}{0.528000in}}{\pgfqpoint{4.960000in}{3.696000in}}%
\pgfusepath{clip}%
\pgfsetbuttcap%
\pgfsetroundjoin%
\definecolor{currentfill}{rgb}{0.000000,0.000000,0.000000}%
\pgfsetfillcolor{currentfill}%
\pgfsetlinewidth{1.003750pt}%
\definecolor{currentstroke}{rgb}{0.000000,0.000000,0.000000}%
\pgfsetstrokecolor{currentstroke}%
\pgfsetdash{}{0pt}%
\pgfpathmoveto{\pgfqpoint{2.145481in}{2.334333in}}%
\pgfpathcurveto{\pgfqpoint{2.156531in}{2.334333in}}{\pgfqpoint{2.167130in}{2.338724in}}{\pgfqpoint{2.174944in}{2.346537in}}%
\pgfpathcurveto{\pgfqpoint{2.182758in}{2.354351in}}{\pgfqpoint{2.187148in}{2.364950in}}{\pgfqpoint{2.187148in}{2.376000in}}%
\pgfpathcurveto{\pgfqpoint{2.187148in}{2.387050in}}{\pgfqpoint{2.182758in}{2.397649in}}{\pgfqpoint{2.174944in}{2.405463in}}%
\pgfpathcurveto{\pgfqpoint{2.167130in}{2.413276in}}{\pgfqpoint{2.156531in}{2.417667in}}{\pgfqpoint{2.145481in}{2.417667in}}%
\pgfpathcurveto{\pgfqpoint{2.134431in}{2.417667in}}{\pgfqpoint{2.123832in}{2.413276in}}{\pgfqpoint{2.116018in}{2.405463in}}%
\pgfpathcurveto{\pgfqpoint{2.108205in}{2.397649in}}{\pgfqpoint{2.103815in}{2.387050in}}{\pgfqpoint{2.103815in}{2.376000in}}%
\pgfpathcurveto{\pgfqpoint{2.103815in}{2.364950in}}{\pgfqpoint{2.108205in}{2.354351in}}{\pgfqpoint{2.116018in}{2.346537in}}%
\pgfpathcurveto{\pgfqpoint{2.123832in}{2.338724in}}{\pgfqpoint{2.134431in}{2.334333in}}{\pgfqpoint{2.145481in}{2.334333in}}%
\pgfpathclose%
\pgfusepath{stroke,fill}%
\end{pgfscope}%
\begin{pgfscope}%
\pgfpathrectangle{\pgfqpoint{0.800000in}{0.528000in}}{\pgfqpoint{4.960000in}{3.696000in}}%
\pgfusepath{clip}%
\pgfsetbuttcap%
\pgfsetroundjoin%
\definecolor{currentfill}{rgb}{0.000000,0.000000,0.000000}%
\pgfsetfillcolor{currentfill}%
\pgfsetlinewidth{1.003750pt}%
\definecolor{currentstroke}{rgb}{0.000000,0.000000,0.000000}%
\pgfsetstrokecolor{currentstroke}%
\pgfsetdash{}{0pt}%
\pgfpathmoveto{\pgfqpoint{2.145481in}{2.334333in}}%
\pgfpathcurveto{\pgfqpoint{2.156531in}{2.334333in}}{\pgfqpoint{2.167130in}{2.338724in}}{\pgfqpoint{2.174944in}{2.346537in}}%
\pgfpathcurveto{\pgfqpoint{2.182758in}{2.354351in}}{\pgfqpoint{2.187148in}{2.364950in}}{\pgfqpoint{2.187148in}{2.376000in}}%
\pgfpathcurveto{\pgfqpoint{2.187148in}{2.387050in}}{\pgfqpoint{2.182758in}{2.397649in}}{\pgfqpoint{2.174944in}{2.405463in}}%
\pgfpathcurveto{\pgfqpoint{2.167130in}{2.413276in}}{\pgfqpoint{2.156531in}{2.417667in}}{\pgfqpoint{2.145481in}{2.417667in}}%
\pgfpathcurveto{\pgfqpoint{2.134431in}{2.417667in}}{\pgfqpoint{2.123832in}{2.413276in}}{\pgfqpoint{2.116018in}{2.405463in}}%
\pgfpathcurveto{\pgfqpoint{2.108205in}{2.397649in}}{\pgfqpoint{2.103815in}{2.387050in}}{\pgfqpoint{2.103815in}{2.376000in}}%
\pgfpathcurveto{\pgfqpoint{2.103815in}{2.364950in}}{\pgfqpoint{2.108205in}{2.354351in}}{\pgfqpoint{2.116018in}{2.346537in}}%
\pgfpathcurveto{\pgfqpoint{2.123832in}{2.338724in}}{\pgfqpoint{2.134431in}{2.334333in}}{\pgfqpoint{2.145481in}{2.334333in}}%
\pgfpathclose%
\pgfusepath{stroke,fill}%
\end{pgfscope}%
\begin{pgfscope}%
\pgfpathrectangle{\pgfqpoint{0.800000in}{0.528000in}}{\pgfqpoint{4.960000in}{3.696000in}}%
\pgfusepath{clip}%
\pgfsetbuttcap%
\pgfsetroundjoin%
\definecolor{currentfill}{rgb}{0.000000,0.000000,0.000000}%
\pgfsetfillcolor{currentfill}%
\pgfsetlinewidth{1.003750pt}%
\definecolor{currentstroke}{rgb}{0.000000,0.000000,0.000000}%
\pgfsetstrokecolor{currentstroke}%
\pgfsetdash{}{0pt}%
\pgfpathmoveto{\pgfqpoint{2.145481in}{2.334333in}}%
\pgfpathcurveto{\pgfqpoint{2.156531in}{2.334333in}}{\pgfqpoint{2.167130in}{2.338724in}}{\pgfqpoint{2.174944in}{2.346537in}}%
\pgfpathcurveto{\pgfqpoint{2.182758in}{2.354351in}}{\pgfqpoint{2.187148in}{2.364950in}}{\pgfqpoint{2.187148in}{2.376000in}}%
\pgfpathcurveto{\pgfqpoint{2.187148in}{2.387050in}}{\pgfqpoint{2.182758in}{2.397649in}}{\pgfqpoint{2.174944in}{2.405463in}}%
\pgfpathcurveto{\pgfqpoint{2.167130in}{2.413276in}}{\pgfqpoint{2.156531in}{2.417667in}}{\pgfqpoint{2.145481in}{2.417667in}}%
\pgfpathcurveto{\pgfqpoint{2.134431in}{2.417667in}}{\pgfqpoint{2.123832in}{2.413276in}}{\pgfqpoint{2.116018in}{2.405463in}}%
\pgfpathcurveto{\pgfqpoint{2.108205in}{2.397649in}}{\pgfqpoint{2.103815in}{2.387050in}}{\pgfqpoint{2.103815in}{2.376000in}}%
\pgfpathcurveto{\pgfqpoint{2.103815in}{2.364950in}}{\pgfqpoint{2.108205in}{2.354351in}}{\pgfqpoint{2.116018in}{2.346537in}}%
\pgfpathcurveto{\pgfqpoint{2.123832in}{2.338724in}}{\pgfqpoint{2.134431in}{2.334333in}}{\pgfqpoint{2.145481in}{2.334333in}}%
\pgfpathclose%
\pgfusepath{stroke,fill}%
\end{pgfscope}%
\begin{pgfscope}%
\pgfpathrectangle{\pgfqpoint{0.800000in}{0.528000in}}{\pgfqpoint{4.960000in}{3.696000in}}%
\pgfusepath{clip}%
\pgfsetbuttcap%
\pgfsetroundjoin%
\definecolor{currentfill}{rgb}{0.000000,0.000000,0.000000}%
\pgfsetfillcolor{currentfill}%
\pgfsetlinewidth{1.003750pt}%
\definecolor{currentstroke}{rgb}{0.000000,0.000000,0.000000}%
\pgfsetstrokecolor{currentstroke}%
\pgfsetdash{}{0pt}%
\pgfpathmoveto{\pgfqpoint{2.145481in}{2.334333in}}%
\pgfpathcurveto{\pgfqpoint{2.156531in}{2.334333in}}{\pgfqpoint{2.167130in}{2.338724in}}{\pgfqpoint{2.174944in}{2.346537in}}%
\pgfpathcurveto{\pgfqpoint{2.182758in}{2.354351in}}{\pgfqpoint{2.187148in}{2.364950in}}{\pgfqpoint{2.187148in}{2.376000in}}%
\pgfpathcurveto{\pgfqpoint{2.187148in}{2.387050in}}{\pgfqpoint{2.182758in}{2.397649in}}{\pgfqpoint{2.174944in}{2.405463in}}%
\pgfpathcurveto{\pgfqpoint{2.167130in}{2.413276in}}{\pgfqpoint{2.156531in}{2.417667in}}{\pgfqpoint{2.145481in}{2.417667in}}%
\pgfpathcurveto{\pgfqpoint{2.134431in}{2.417667in}}{\pgfqpoint{2.123832in}{2.413276in}}{\pgfqpoint{2.116018in}{2.405463in}}%
\pgfpathcurveto{\pgfqpoint{2.108205in}{2.397649in}}{\pgfqpoint{2.103815in}{2.387050in}}{\pgfqpoint{2.103815in}{2.376000in}}%
\pgfpathcurveto{\pgfqpoint{2.103815in}{2.364950in}}{\pgfqpoint{2.108205in}{2.354351in}}{\pgfqpoint{2.116018in}{2.346537in}}%
\pgfpathcurveto{\pgfqpoint{2.123832in}{2.338724in}}{\pgfqpoint{2.134431in}{2.334333in}}{\pgfqpoint{2.145481in}{2.334333in}}%
\pgfpathclose%
\pgfusepath{stroke,fill}%
\end{pgfscope}%
\begin{pgfscope}%
\pgfpathrectangle{\pgfqpoint{0.800000in}{0.528000in}}{\pgfqpoint{4.960000in}{3.696000in}}%
\pgfusepath{clip}%
\pgfsetbuttcap%
\pgfsetroundjoin%
\definecolor{currentfill}{rgb}{0.000000,0.000000,0.000000}%
\pgfsetfillcolor{currentfill}%
\pgfsetlinewidth{1.003750pt}%
\definecolor{currentstroke}{rgb}{0.000000,0.000000,0.000000}%
\pgfsetstrokecolor{currentstroke}%
\pgfsetdash{}{0pt}%
\pgfpathmoveto{\pgfqpoint{2.145481in}{2.334333in}}%
\pgfpathcurveto{\pgfqpoint{2.156531in}{2.334333in}}{\pgfqpoint{2.167130in}{2.338724in}}{\pgfqpoint{2.174944in}{2.346537in}}%
\pgfpathcurveto{\pgfqpoint{2.182758in}{2.354351in}}{\pgfqpoint{2.187148in}{2.364950in}}{\pgfqpoint{2.187148in}{2.376000in}}%
\pgfpathcurveto{\pgfqpoint{2.187148in}{2.387050in}}{\pgfqpoint{2.182758in}{2.397649in}}{\pgfqpoint{2.174944in}{2.405463in}}%
\pgfpathcurveto{\pgfqpoint{2.167130in}{2.413276in}}{\pgfqpoint{2.156531in}{2.417667in}}{\pgfqpoint{2.145481in}{2.417667in}}%
\pgfpathcurveto{\pgfqpoint{2.134431in}{2.417667in}}{\pgfqpoint{2.123832in}{2.413276in}}{\pgfqpoint{2.116018in}{2.405463in}}%
\pgfpathcurveto{\pgfqpoint{2.108205in}{2.397649in}}{\pgfqpoint{2.103815in}{2.387050in}}{\pgfqpoint{2.103815in}{2.376000in}}%
\pgfpathcurveto{\pgfqpoint{2.103815in}{2.364950in}}{\pgfqpoint{2.108205in}{2.354351in}}{\pgfqpoint{2.116018in}{2.346537in}}%
\pgfpathcurveto{\pgfqpoint{2.123832in}{2.338724in}}{\pgfqpoint{2.134431in}{2.334333in}}{\pgfqpoint{2.145481in}{2.334333in}}%
\pgfpathclose%
\pgfusepath{stroke,fill}%
\end{pgfscope}%
\begin{pgfscope}%
\pgfpathrectangle{\pgfqpoint{0.800000in}{0.528000in}}{\pgfqpoint{4.960000in}{3.696000in}}%
\pgfusepath{clip}%
\pgfsetbuttcap%
\pgfsetroundjoin%
\definecolor{currentfill}{rgb}{0.000000,0.000000,0.000000}%
\pgfsetfillcolor{currentfill}%
\pgfsetlinewidth{1.003750pt}%
\definecolor{currentstroke}{rgb}{0.000000,0.000000,0.000000}%
\pgfsetstrokecolor{currentstroke}%
\pgfsetdash{}{0pt}%
\pgfpathmoveto{\pgfqpoint{2.145481in}{2.334333in}}%
\pgfpathcurveto{\pgfqpoint{2.156531in}{2.334333in}}{\pgfqpoint{2.167130in}{2.338724in}}{\pgfqpoint{2.174944in}{2.346537in}}%
\pgfpathcurveto{\pgfqpoint{2.182758in}{2.354351in}}{\pgfqpoint{2.187148in}{2.364950in}}{\pgfqpoint{2.187148in}{2.376000in}}%
\pgfpathcurveto{\pgfqpoint{2.187148in}{2.387050in}}{\pgfqpoint{2.182758in}{2.397649in}}{\pgfqpoint{2.174944in}{2.405463in}}%
\pgfpathcurveto{\pgfqpoint{2.167130in}{2.413276in}}{\pgfqpoint{2.156531in}{2.417667in}}{\pgfqpoint{2.145481in}{2.417667in}}%
\pgfpathcurveto{\pgfqpoint{2.134431in}{2.417667in}}{\pgfqpoint{2.123832in}{2.413276in}}{\pgfqpoint{2.116018in}{2.405463in}}%
\pgfpathcurveto{\pgfqpoint{2.108205in}{2.397649in}}{\pgfqpoint{2.103815in}{2.387050in}}{\pgfqpoint{2.103815in}{2.376000in}}%
\pgfpathcurveto{\pgfqpoint{2.103815in}{2.364950in}}{\pgfqpoint{2.108205in}{2.354351in}}{\pgfqpoint{2.116018in}{2.346537in}}%
\pgfpathcurveto{\pgfqpoint{2.123832in}{2.338724in}}{\pgfqpoint{2.134431in}{2.334333in}}{\pgfqpoint{2.145481in}{2.334333in}}%
\pgfpathclose%
\pgfusepath{stroke,fill}%
\end{pgfscope}%
\begin{pgfscope}%
\pgfpathrectangle{\pgfqpoint{0.800000in}{0.528000in}}{\pgfqpoint{4.960000in}{3.696000in}}%
\pgfusepath{clip}%
\pgfsetbuttcap%
\pgfsetroundjoin%
\definecolor{currentfill}{rgb}{0.000000,0.000000,0.000000}%
\pgfsetfillcolor{currentfill}%
\pgfsetlinewidth{1.003750pt}%
\definecolor{currentstroke}{rgb}{0.000000,0.000000,0.000000}%
\pgfsetstrokecolor{currentstroke}%
\pgfsetdash{}{0pt}%
\pgfpathmoveto{\pgfqpoint{2.145481in}{2.334333in}}%
\pgfpathcurveto{\pgfqpoint{2.156531in}{2.334333in}}{\pgfqpoint{2.167130in}{2.338724in}}{\pgfqpoint{2.174944in}{2.346537in}}%
\pgfpathcurveto{\pgfqpoint{2.182758in}{2.354351in}}{\pgfqpoint{2.187148in}{2.364950in}}{\pgfqpoint{2.187148in}{2.376000in}}%
\pgfpathcurveto{\pgfqpoint{2.187148in}{2.387050in}}{\pgfqpoint{2.182758in}{2.397649in}}{\pgfqpoint{2.174944in}{2.405463in}}%
\pgfpathcurveto{\pgfqpoint{2.167130in}{2.413276in}}{\pgfqpoint{2.156531in}{2.417667in}}{\pgfqpoint{2.145481in}{2.417667in}}%
\pgfpathcurveto{\pgfqpoint{2.134431in}{2.417667in}}{\pgfqpoint{2.123832in}{2.413276in}}{\pgfqpoint{2.116018in}{2.405463in}}%
\pgfpathcurveto{\pgfqpoint{2.108205in}{2.397649in}}{\pgfqpoint{2.103815in}{2.387050in}}{\pgfqpoint{2.103815in}{2.376000in}}%
\pgfpathcurveto{\pgfqpoint{2.103815in}{2.364950in}}{\pgfqpoint{2.108205in}{2.354351in}}{\pgfqpoint{2.116018in}{2.346537in}}%
\pgfpathcurveto{\pgfqpoint{2.123832in}{2.338724in}}{\pgfqpoint{2.134431in}{2.334333in}}{\pgfqpoint{2.145481in}{2.334333in}}%
\pgfpathclose%
\pgfusepath{stroke,fill}%
\end{pgfscope}%
\begin{pgfscope}%
\pgfpathrectangle{\pgfqpoint{0.800000in}{0.528000in}}{\pgfqpoint{4.960000in}{3.696000in}}%
\pgfusepath{clip}%
\pgfsetbuttcap%
\pgfsetroundjoin%
\definecolor{currentfill}{rgb}{0.000000,0.000000,0.000000}%
\pgfsetfillcolor{currentfill}%
\pgfsetlinewidth{1.003750pt}%
\definecolor{currentstroke}{rgb}{0.000000,0.000000,0.000000}%
\pgfsetstrokecolor{currentstroke}%
\pgfsetdash{}{0pt}%
\pgfpathmoveto{\pgfqpoint{2.145481in}{2.334333in}}%
\pgfpathcurveto{\pgfqpoint{2.156531in}{2.334333in}}{\pgfqpoint{2.167130in}{2.338724in}}{\pgfqpoint{2.174944in}{2.346537in}}%
\pgfpathcurveto{\pgfqpoint{2.182758in}{2.354351in}}{\pgfqpoint{2.187148in}{2.364950in}}{\pgfqpoint{2.187148in}{2.376000in}}%
\pgfpathcurveto{\pgfqpoint{2.187148in}{2.387050in}}{\pgfqpoint{2.182758in}{2.397649in}}{\pgfqpoint{2.174944in}{2.405463in}}%
\pgfpathcurveto{\pgfqpoint{2.167130in}{2.413276in}}{\pgfqpoint{2.156531in}{2.417667in}}{\pgfqpoint{2.145481in}{2.417667in}}%
\pgfpathcurveto{\pgfqpoint{2.134431in}{2.417667in}}{\pgfqpoint{2.123832in}{2.413276in}}{\pgfqpoint{2.116018in}{2.405463in}}%
\pgfpathcurveto{\pgfqpoint{2.108205in}{2.397649in}}{\pgfqpoint{2.103815in}{2.387050in}}{\pgfqpoint{2.103815in}{2.376000in}}%
\pgfpathcurveto{\pgfqpoint{2.103815in}{2.364950in}}{\pgfqpoint{2.108205in}{2.354351in}}{\pgfqpoint{2.116018in}{2.346537in}}%
\pgfpathcurveto{\pgfqpoint{2.123832in}{2.338724in}}{\pgfqpoint{2.134431in}{2.334333in}}{\pgfqpoint{2.145481in}{2.334333in}}%
\pgfpathclose%
\pgfusepath{stroke,fill}%
\end{pgfscope}%
\begin{pgfscope}%
\pgfpathrectangle{\pgfqpoint{0.800000in}{0.528000in}}{\pgfqpoint{4.960000in}{3.696000in}}%
\pgfusepath{clip}%
\pgfsetbuttcap%
\pgfsetroundjoin%
\definecolor{currentfill}{rgb}{0.000000,0.000000,0.000000}%
\pgfsetfillcolor{currentfill}%
\pgfsetlinewidth{1.003750pt}%
\definecolor{currentstroke}{rgb}{0.000000,0.000000,0.000000}%
\pgfsetstrokecolor{currentstroke}%
\pgfsetdash{}{0pt}%
\pgfpathmoveto{\pgfqpoint{2.145481in}{2.334333in}}%
\pgfpathcurveto{\pgfqpoint{2.156531in}{2.334333in}}{\pgfqpoint{2.167130in}{2.338724in}}{\pgfqpoint{2.174944in}{2.346537in}}%
\pgfpathcurveto{\pgfqpoint{2.182758in}{2.354351in}}{\pgfqpoint{2.187148in}{2.364950in}}{\pgfqpoint{2.187148in}{2.376000in}}%
\pgfpathcurveto{\pgfqpoint{2.187148in}{2.387050in}}{\pgfqpoint{2.182758in}{2.397649in}}{\pgfqpoint{2.174944in}{2.405463in}}%
\pgfpathcurveto{\pgfqpoint{2.167130in}{2.413276in}}{\pgfqpoint{2.156531in}{2.417667in}}{\pgfqpoint{2.145481in}{2.417667in}}%
\pgfpathcurveto{\pgfqpoint{2.134431in}{2.417667in}}{\pgfqpoint{2.123832in}{2.413276in}}{\pgfqpoint{2.116018in}{2.405463in}}%
\pgfpathcurveto{\pgfqpoint{2.108205in}{2.397649in}}{\pgfqpoint{2.103815in}{2.387050in}}{\pgfqpoint{2.103815in}{2.376000in}}%
\pgfpathcurveto{\pgfqpoint{2.103815in}{2.364950in}}{\pgfqpoint{2.108205in}{2.354351in}}{\pgfqpoint{2.116018in}{2.346537in}}%
\pgfpathcurveto{\pgfqpoint{2.123832in}{2.338724in}}{\pgfqpoint{2.134431in}{2.334333in}}{\pgfqpoint{2.145481in}{2.334333in}}%
\pgfpathclose%
\pgfusepath{stroke,fill}%
\end{pgfscope}%
\begin{pgfscope}%
\pgfpathrectangle{\pgfqpoint{0.800000in}{0.528000in}}{\pgfqpoint{4.960000in}{3.696000in}}%
\pgfusepath{clip}%
\pgfsetbuttcap%
\pgfsetroundjoin%
\definecolor{currentfill}{rgb}{0.000000,0.000000,0.000000}%
\pgfsetfillcolor{currentfill}%
\pgfsetlinewidth{1.003750pt}%
\definecolor{currentstroke}{rgb}{0.000000,0.000000,0.000000}%
\pgfsetstrokecolor{currentstroke}%
\pgfsetdash{}{0pt}%
\pgfpathmoveto{\pgfqpoint{2.145481in}{2.334333in}}%
\pgfpathcurveto{\pgfqpoint{2.156531in}{2.334333in}}{\pgfqpoint{2.167130in}{2.338724in}}{\pgfqpoint{2.174944in}{2.346537in}}%
\pgfpathcurveto{\pgfqpoint{2.182758in}{2.354351in}}{\pgfqpoint{2.187148in}{2.364950in}}{\pgfqpoint{2.187148in}{2.376000in}}%
\pgfpathcurveto{\pgfqpoint{2.187148in}{2.387050in}}{\pgfqpoint{2.182758in}{2.397649in}}{\pgfqpoint{2.174944in}{2.405463in}}%
\pgfpathcurveto{\pgfqpoint{2.167130in}{2.413276in}}{\pgfqpoint{2.156531in}{2.417667in}}{\pgfqpoint{2.145481in}{2.417667in}}%
\pgfpathcurveto{\pgfqpoint{2.134431in}{2.417667in}}{\pgfqpoint{2.123832in}{2.413276in}}{\pgfqpoint{2.116018in}{2.405463in}}%
\pgfpathcurveto{\pgfqpoint{2.108205in}{2.397649in}}{\pgfqpoint{2.103815in}{2.387050in}}{\pgfqpoint{2.103815in}{2.376000in}}%
\pgfpathcurveto{\pgfqpoint{2.103815in}{2.364950in}}{\pgfqpoint{2.108205in}{2.354351in}}{\pgfqpoint{2.116018in}{2.346537in}}%
\pgfpathcurveto{\pgfqpoint{2.123832in}{2.338724in}}{\pgfqpoint{2.134431in}{2.334333in}}{\pgfqpoint{2.145481in}{2.334333in}}%
\pgfpathclose%
\pgfusepath{stroke,fill}%
\end{pgfscope}%
\begin{pgfscope}%
\pgfpathrectangle{\pgfqpoint{0.800000in}{0.528000in}}{\pgfqpoint{4.960000in}{3.696000in}}%
\pgfusepath{clip}%
\pgfsetbuttcap%
\pgfsetroundjoin%
\definecolor{currentfill}{rgb}{0.000000,0.000000,0.000000}%
\pgfsetfillcolor{currentfill}%
\pgfsetlinewidth{1.003750pt}%
\definecolor{currentstroke}{rgb}{0.000000,0.000000,0.000000}%
\pgfsetstrokecolor{currentstroke}%
\pgfsetdash{}{0pt}%
\pgfpathmoveto{\pgfqpoint{2.145481in}{2.334333in}}%
\pgfpathcurveto{\pgfqpoint{2.156531in}{2.334333in}}{\pgfqpoint{2.167130in}{2.338724in}}{\pgfqpoint{2.174944in}{2.346537in}}%
\pgfpathcurveto{\pgfqpoint{2.182758in}{2.354351in}}{\pgfqpoint{2.187148in}{2.364950in}}{\pgfqpoint{2.187148in}{2.376000in}}%
\pgfpathcurveto{\pgfqpoint{2.187148in}{2.387050in}}{\pgfqpoint{2.182758in}{2.397649in}}{\pgfqpoint{2.174944in}{2.405463in}}%
\pgfpathcurveto{\pgfqpoint{2.167130in}{2.413276in}}{\pgfqpoint{2.156531in}{2.417667in}}{\pgfqpoint{2.145481in}{2.417667in}}%
\pgfpathcurveto{\pgfqpoint{2.134431in}{2.417667in}}{\pgfqpoint{2.123832in}{2.413276in}}{\pgfqpoint{2.116018in}{2.405463in}}%
\pgfpathcurveto{\pgfqpoint{2.108205in}{2.397649in}}{\pgfqpoint{2.103815in}{2.387050in}}{\pgfqpoint{2.103815in}{2.376000in}}%
\pgfpathcurveto{\pgfqpoint{2.103815in}{2.364950in}}{\pgfqpoint{2.108205in}{2.354351in}}{\pgfqpoint{2.116018in}{2.346537in}}%
\pgfpathcurveto{\pgfqpoint{2.123832in}{2.338724in}}{\pgfqpoint{2.134431in}{2.334333in}}{\pgfqpoint{2.145481in}{2.334333in}}%
\pgfpathclose%
\pgfusepath{stroke,fill}%
\end{pgfscope}%
\begin{pgfscope}%
\pgfpathrectangle{\pgfqpoint{0.800000in}{0.528000in}}{\pgfqpoint{4.960000in}{3.696000in}}%
\pgfusepath{clip}%
\pgfsetbuttcap%
\pgfsetroundjoin%
\definecolor{currentfill}{rgb}{0.000000,0.000000,0.000000}%
\pgfsetfillcolor{currentfill}%
\pgfsetlinewidth{1.003750pt}%
\definecolor{currentstroke}{rgb}{0.000000,0.000000,0.000000}%
\pgfsetstrokecolor{currentstroke}%
\pgfsetdash{}{0pt}%
\pgfpathmoveto{\pgfqpoint{2.145481in}{2.334333in}}%
\pgfpathcurveto{\pgfqpoint{2.156531in}{2.334333in}}{\pgfqpoint{2.167130in}{2.338724in}}{\pgfqpoint{2.174944in}{2.346537in}}%
\pgfpathcurveto{\pgfqpoint{2.182758in}{2.354351in}}{\pgfqpoint{2.187148in}{2.364950in}}{\pgfqpoint{2.187148in}{2.376000in}}%
\pgfpathcurveto{\pgfqpoint{2.187148in}{2.387050in}}{\pgfqpoint{2.182758in}{2.397649in}}{\pgfqpoint{2.174944in}{2.405463in}}%
\pgfpathcurveto{\pgfqpoint{2.167130in}{2.413276in}}{\pgfqpoint{2.156531in}{2.417667in}}{\pgfqpoint{2.145481in}{2.417667in}}%
\pgfpathcurveto{\pgfqpoint{2.134431in}{2.417667in}}{\pgfqpoint{2.123832in}{2.413276in}}{\pgfqpoint{2.116018in}{2.405463in}}%
\pgfpathcurveto{\pgfqpoint{2.108205in}{2.397649in}}{\pgfqpoint{2.103815in}{2.387050in}}{\pgfqpoint{2.103815in}{2.376000in}}%
\pgfpathcurveto{\pgfqpoint{2.103815in}{2.364950in}}{\pgfqpoint{2.108205in}{2.354351in}}{\pgfqpoint{2.116018in}{2.346537in}}%
\pgfpathcurveto{\pgfqpoint{2.123832in}{2.338724in}}{\pgfqpoint{2.134431in}{2.334333in}}{\pgfqpoint{2.145481in}{2.334333in}}%
\pgfpathclose%
\pgfusepath{stroke,fill}%
\end{pgfscope}%
\begin{pgfscope}%
\pgfpathrectangle{\pgfqpoint{0.800000in}{0.528000in}}{\pgfqpoint{4.960000in}{3.696000in}}%
\pgfusepath{clip}%
\pgfsetbuttcap%
\pgfsetroundjoin%
\definecolor{currentfill}{rgb}{0.000000,0.000000,0.000000}%
\pgfsetfillcolor{currentfill}%
\pgfsetlinewidth{1.003750pt}%
\definecolor{currentstroke}{rgb}{0.000000,0.000000,0.000000}%
\pgfsetstrokecolor{currentstroke}%
\pgfsetdash{}{0pt}%
\pgfpathmoveto{\pgfqpoint{2.145481in}{2.334333in}}%
\pgfpathcurveto{\pgfqpoint{2.156531in}{2.334333in}}{\pgfqpoint{2.167130in}{2.338724in}}{\pgfqpoint{2.174944in}{2.346537in}}%
\pgfpathcurveto{\pgfqpoint{2.182758in}{2.354351in}}{\pgfqpoint{2.187148in}{2.364950in}}{\pgfqpoint{2.187148in}{2.376000in}}%
\pgfpathcurveto{\pgfqpoint{2.187148in}{2.387050in}}{\pgfqpoint{2.182758in}{2.397649in}}{\pgfqpoint{2.174944in}{2.405463in}}%
\pgfpathcurveto{\pgfqpoint{2.167130in}{2.413276in}}{\pgfqpoint{2.156531in}{2.417667in}}{\pgfqpoint{2.145481in}{2.417667in}}%
\pgfpathcurveto{\pgfqpoint{2.134431in}{2.417667in}}{\pgfqpoint{2.123832in}{2.413276in}}{\pgfqpoint{2.116018in}{2.405463in}}%
\pgfpathcurveto{\pgfqpoint{2.108205in}{2.397649in}}{\pgfqpoint{2.103815in}{2.387050in}}{\pgfqpoint{2.103815in}{2.376000in}}%
\pgfpathcurveto{\pgfqpoint{2.103815in}{2.364950in}}{\pgfqpoint{2.108205in}{2.354351in}}{\pgfqpoint{2.116018in}{2.346537in}}%
\pgfpathcurveto{\pgfqpoint{2.123832in}{2.338724in}}{\pgfqpoint{2.134431in}{2.334333in}}{\pgfqpoint{2.145481in}{2.334333in}}%
\pgfpathclose%
\pgfusepath{stroke,fill}%
\end{pgfscope}%
\begin{pgfscope}%
\pgfpathrectangle{\pgfqpoint{0.800000in}{0.528000in}}{\pgfqpoint{4.960000in}{3.696000in}}%
\pgfusepath{clip}%
\pgfsetbuttcap%
\pgfsetroundjoin%
\definecolor{currentfill}{rgb}{0.000000,0.000000,0.000000}%
\pgfsetfillcolor{currentfill}%
\pgfsetlinewidth{1.003750pt}%
\definecolor{currentstroke}{rgb}{0.000000,0.000000,0.000000}%
\pgfsetstrokecolor{currentstroke}%
\pgfsetdash{}{0pt}%
\pgfpathmoveto{\pgfqpoint{2.145481in}{2.334333in}}%
\pgfpathcurveto{\pgfqpoint{2.156531in}{2.334333in}}{\pgfqpoint{2.167130in}{2.338724in}}{\pgfqpoint{2.174944in}{2.346537in}}%
\pgfpathcurveto{\pgfqpoint{2.182758in}{2.354351in}}{\pgfqpoint{2.187148in}{2.364950in}}{\pgfqpoint{2.187148in}{2.376000in}}%
\pgfpathcurveto{\pgfqpoint{2.187148in}{2.387050in}}{\pgfqpoint{2.182758in}{2.397649in}}{\pgfqpoint{2.174944in}{2.405463in}}%
\pgfpathcurveto{\pgfqpoint{2.167130in}{2.413276in}}{\pgfqpoint{2.156531in}{2.417667in}}{\pgfqpoint{2.145481in}{2.417667in}}%
\pgfpathcurveto{\pgfqpoint{2.134431in}{2.417667in}}{\pgfqpoint{2.123832in}{2.413276in}}{\pgfqpoint{2.116018in}{2.405463in}}%
\pgfpathcurveto{\pgfqpoint{2.108205in}{2.397649in}}{\pgfqpoint{2.103815in}{2.387050in}}{\pgfqpoint{2.103815in}{2.376000in}}%
\pgfpathcurveto{\pgfqpoint{2.103815in}{2.364950in}}{\pgfqpoint{2.108205in}{2.354351in}}{\pgfqpoint{2.116018in}{2.346537in}}%
\pgfpathcurveto{\pgfqpoint{2.123832in}{2.338724in}}{\pgfqpoint{2.134431in}{2.334333in}}{\pgfqpoint{2.145481in}{2.334333in}}%
\pgfpathclose%
\pgfusepath{stroke,fill}%
\end{pgfscope}%
\begin{pgfscope}%
\pgfpathrectangle{\pgfqpoint{0.800000in}{0.528000in}}{\pgfqpoint{4.960000in}{3.696000in}}%
\pgfusepath{clip}%
\pgfsetbuttcap%
\pgfsetroundjoin%
\definecolor{currentfill}{rgb}{0.000000,0.000000,0.000000}%
\pgfsetfillcolor{currentfill}%
\pgfsetlinewidth{1.003750pt}%
\definecolor{currentstroke}{rgb}{0.000000,0.000000,0.000000}%
\pgfsetstrokecolor{currentstroke}%
\pgfsetdash{}{0pt}%
\pgfpathmoveto{\pgfqpoint{2.145481in}{2.334333in}}%
\pgfpathcurveto{\pgfqpoint{2.156531in}{2.334333in}}{\pgfqpoint{2.167130in}{2.338724in}}{\pgfqpoint{2.174944in}{2.346537in}}%
\pgfpathcurveto{\pgfqpoint{2.182758in}{2.354351in}}{\pgfqpoint{2.187148in}{2.364950in}}{\pgfqpoint{2.187148in}{2.376000in}}%
\pgfpathcurveto{\pgfqpoint{2.187148in}{2.387050in}}{\pgfqpoint{2.182758in}{2.397649in}}{\pgfqpoint{2.174944in}{2.405463in}}%
\pgfpathcurveto{\pgfqpoint{2.167130in}{2.413276in}}{\pgfqpoint{2.156531in}{2.417667in}}{\pgfqpoint{2.145481in}{2.417667in}}%
\pgfpathcurveto{\pgfqpoint{2.134431in}{2.417667in}}{\pgfqpoint{2.123832in}{2.413276in}}{\pgfqpoint{2.116018in}{2.405463in}}%
\pgfpathcurveto{\pgfqpoint{2.108205in}{2.397649in}}{\pgfqpoint{2.103815in}{2.387050in}}{\pgfqpoint{2.103815in}{2.376000in}}%
\pgfpathcurveto{\pgfqpoint{2.103815in}{2.364950in}}{\pgfqpoint{2.108205in}{2.354351in}}{\pgfqpoint{2.116018in}{2.346537in}}%
\pgfpathcurveto{\pgfqpoint{2.123832in}{2.338724in}}{\pgfqpoint{2.134431in}{2.334333in}}{\pgfqpoint{2.145481in}{2.334333in}}%
\pgfpathclose%
\pgfusepath{stroke,fill}%
\end{pgfscope}%
\begin{pgfscope}%
\pgfpathrectangle{\pgfqpoint{0.800000in}{0.528000in}}{\pgfqpoint{4.960000in}{3.696000in}}%
\pgfusepath{clip}%
\pgfsetbuttcap%
\pgfsetroundjoin%
\definecolor{currentfill}{rgb}{0.000000,0.000000,0.000000}%
\pgfsetfillcolor{currentfill}%
\pgfsetlinewidth{1.003750pt}%
\definecolor{currentstroke}{rgb}{0.000000,0.000000,0.000000}%
\pgfsetstrokecolor{currentstroke}%
\pgfsetdash{}{0pt}%
\pgfpathmoveto{\pgfqpoint{2.145481in}{2.334333in}}%
\pgfpathcurveto{\pgfqpoint{2.156531in}{2.334333in}}{\pgfqpoint{2.167130in}{2.338724in}}{\pgfqpoint{2.174944in}{2.346537in}}%
\pgfpathcurveto{\pgfqpoint{2.182758in}{2.354351in}}{\pgfqpoint{2.187148in}{2.364950in}}{\pgfqpoint{2.187148in}{2.376000in}}%
\pgfpathcurveto{\pgfqpoint{2.187148in}{2.387050in}}{\pgfqpoint{2.182758in}{2.397649in}}{\pgfqpoint{2.174944in}{2.405463in}}%
\pgfpathcurveto{\pgfqpoint{2.167130in}{2.413276in}}{\pgfqpoint{2.156531in}{2.417667in}}{\pgfqpoint{2.145481in}{2.417667in}}%
\pgfpathcurveto{\pgfqpoint{2.134431in}{2.417667in}}{\pgfqpoint{2.123832in}{2.413276in}}{\pgfqpoint{2.116018in}{2.405463in}}%
\pgfpathcurveto{\pgfqpoint{2.108205in}{2.397649in}}{\pgfqpoint{2.103815in}{2.387050in}}{\pgfqpoint{2.103815in}{2.376000in}}%
\pgfpathcurveto{\pgfqpoint{2.103815in}{2.364950in}}{\pgfqpoint{2.108205in}{2.354351in}}{\pgfqpoint{2.116018in}{2.346537in}}%
\pgfpathcurveto{\pgfqpoint{2.123832in}{2.338724in}}{\pgfqpoint{2.134431in}{2.334333in}}{\pgfqpoint{2.145481in}{2.334333in}}%
\pgfpathclose%
\pgfusepath{stroke,fill}%
\end{pgfscope}%
\begin{pgfscope}%
\pgfpathrectangle{\pgfqpoint{0.800000in}{0.528000in}}{\pgfqpoint{4.960000in}{3.696000in}}%
\pgfusepath{clip}%
\pgfsetbuttcap%
\pgfsetroundjoin%
\definecolor{currentfill}{rgb}{0.000000,0.000000,0.000000}%
\pgfsetfillcolor{currentfill}%
\pgfsetlinewidth{1.003750pt}%
\definecolor{currentstroke}{rgb}{0.000000,0.000000,0.000000}%
\pgfsetstrokecolor{currentstroke}%
\pgfsetdash{}{0pt}%
\pgfpathmoveto{\pgfqpoint{2.145481in}{2.334333in}}%
\pgfpathcurveto{\pgfqpoint{2.156531in}{2.334333in}}{\pgfqpoint{2.167130in}{2.338724in}}{\pgfqpoint{2.174944in}{2.346537in}}%
\pgfpathcurveto{\pgfqpoint{2.182758in}{2.354351in}}{\pgfqpoint{2.187148in}{2.364950in}}{\pgfqpoint{2.187148in}{2.376000in}}%
\pgfpathcurveto{\pgfqpoint{2.187148in}{2.387050in}}{\pgfqpoint{2.182758in}{2.397649in}}{\pgfqpoint{2.174944in}{2.405463in}}%
\pgfpathcurveto{\pgfqpoint{2.167130in}{2.413276in}}{\pgfqpoint{2.156531in}{2.417667in}}{\pgfqpoint{2.145481in}{2.417667in}}%
\pgfpathcurveto{\pgfqpoint{2.134431in}{2.417667in}}{\pgfqpoint{2.123832in}{2.413276in}}{\pgfqpoint{2.116018in}{2.405463in}}%
\pgfpathcurveto{\pgfqpoint{2.108205in}{2.397649in}}{\pgfqpoint{2.103815in}{2.387050in}}{\pgfqpoint{2.103815in}{2.376000in}}%
\pgfpathcurveto{\pgfqpoint{2.103815in}{2.364950in}}{\pgfqpoint{2.108205in}{2.354351in}}{\pgfqpoint{2.116018in}{2.346537in}}%
\pgfpathcurveto{\pgfqpoint{2.123832in}{2.338724in}}{\pgfqpoint{2.134431in}{2.334333in}}{\pgfqpoint{2.145481in}{2.334333in}}%
\pgfpathclose%
\pgfusepath{stroke,fill}%
\end{pgfscope}%
\begin{pgfscope}%
\pgfpathrectangle{\pgfqpoint{0.800000in}{0.528000in}}{\pgfqpoint{4.960000in}{3.696000in}}%
\pgfusepath{clip}%
\pgfsetbuttcap%
\pgfsetroundjoin%
\definecolor{currentfill}{rgb}{0.000000,0.000000,0.000000}%
\pgfsetfillcolor{currentfill}%
\pgfsetlinewidth{1.003750pt}%
\definecolor{currentstroke}{rgb}{0.000000,0.000000,0.000000}%
\pgfsetstrokecolor{currentstroke}%
\pgfsetdash{}{0pt}%
\pgfpathmoveto{\pgfqpoint{3.265169in}{2.334333in}}%
\pgfpathcurveto{\pgfqpoint{3.276219in}{2.334333in}}{\pgfqpoint{3.286818in}{2.338724in}}{\pgfqpoint{3.294632in}{2.346537in}}%
\pgfpathcurveto{\pgfqpoint{3.302446in}{2.354351in}}{\pgfqpoint{3.306836in}{2.364950in}}{\pgfqpoint{3.306836in}{2.376000in}}%
\pgfpathcurveto{\pgfqpoint{3.306836in}{2.387050in}}{\pgfqpoint{3.302446in}{2.397649in}}{\pgfqpoint{3.294632in}{2.405463in}}%
\pgfpathcurveto{\pgfqpoint{3.286818in}{2.413276in}}{\pgfqpoint{3.276219in}{2.417667in}}{\pgfqpoint{3.265169in}{2.417667in}}%
\pgfpathcurveto{\pgfqpoint{3.254119in}{2.417667in}}{\pgfqpoint{3.243520in}{2.413276in}}{\pgfqpoint{3.235707in}{2.405463in}}%
\pgfpathcurveto{\pgfqpoint{3.227893in}{2.397649in}}{\pgfqpoint{3.223503in}{2.387050in}}{\pgfqpoint{3.223503in}{2.376000in}}%
\pgfpathcurveto{\pgfqpoint{3.223503in}{2.364950in}}{\pgfqpoint{3.227893in}{2.354351in}}{\pgfqpoint{3.235707in}{2.346537in}}%
\pgfpathcurveto{\pgfqpoint{3.243520in}{2.338724in}}{\pgfqpoint{3.254119in}{2.334333in}}{\pgfqpoint{3.265169in}{2.334333in}}%
\pgfpathclose%
\pgfusepath{stroke,fill}%
\end{pgfscope}%
\begin{pgfscope}%
\pgfpathrectangle{\pgfqpoint{0.800000in}{0.528000in}}{\pgfqpoint{4.960000in}{3.696000in}}%
\pgfusepath{clip}%
\pgfsetbuttcap%
\pgfsetroundjoin%
\definecolor{currentfill}{rgb}{0.000000,0.000000,0.000000}%
\pgfsetfillcolor{currentfill}%
\pgfsetlinewidth{1.003750pt}%
\definecolor{currentstroke}{rgb}{0.000000,0.000000,0.000000}%
\pgfsetstrokecolor{currentstroke}%
\pgfsetdash{}{0pt}%
\pgfpathmoveto{\pgfqpoint{3.265169in}{2.334333in}}%
\pgfpathcurveto{\pgfqpoint{3.276219in}{2.334333in}}{\pgfqpoint{3.286818in}{2.338724in}}{\pgfqpoint{3.294632in}{2.346537in}}%
\pgfpathcurveto{\pgfqpoint{3.302446in}{2.354351in}}{\pgfqpoint{3.306836in}{2.364950in}}{\pgfqpoint{3.306836in}{2.376000in}}%
\pgfpathcurveto{\pgfqpoint{3.306836in}{2.387050in}}{\pgfqpoint{3.302446in}{2.397649in}}{\pgfqpoint{3.294632in}{2.405463in}}%
\pgfpathcurveto{\pgfqpoint{3.286818in}{2.413276in}}{\pgfqpoint{3.276219in}{2.417667in}}{\pgfqpoint{3.265169in}{2.417667in}}%
\pgfpathcurveto{\pgfqpoint{3.254119in}{2.417667in}}{\pgfqpoint{3.243520in}{2.413276in}}{\pgfqpoint{3.235707in}{2.405463in}}%
\pgfpathcurveto{\pgfqpoint{3.227893in}{2.397649in}}{\pgfqpoint{3.223503in}{2.387050in}}{\pgfqpoint{3.223503in}{2.376000in}}%
\pgfpathcurveto{\pgfqpoint{3.223503in}{2.364950in}}{\pgfqpoint{3.227893in}{2.354351in}}{\pgfqpoint{3.235707in}{2.346537in}}%
\pgfpathcurveto{\pgfqpoint{3.243520in}{2.338724in}}{\pgfqpoint{3.254119in}{2.334333in}}{\pgfqpoint{3.265169in}{2.334333in}}%
\pgfpathclose%
\pgfusepath{stroke,fill}%
\end{pgfscope}%
\begin{pgfscope}%
\pgfpathrectangle{\pgfqpoint{0.800000in}{0.528000in}}{\pgfqpoint{4.960000in}{3.696000in}}%
\pgfusepath{clip}%
\pgfsetbuttcap%
\pgfsetroundjoin%
\definecolor{currentfill}{rgb}{0.000000,0.000000,0.000000}%
\pgfsetfillcolor{currentfill}%
\pgfsetlinewidth{1.003750pt}%
\definecolor{currentstroke}{rgb}{0.000000,0.000000,0.000000}%
\pgfsetstrokecolor{currentstroke}%
\pgfsetdash{}{0pt}%
\pgfpathmoveto{\pgfqpoint{3.265169in}{2.334333in}}%
\pgfpathcurveto{\pgfqpoint{3.276219in}{2.334333in}}{\pgfqpoint{3.286818in}{2.338724in}}{\pgfqpoint{3.294632in}{2.346537in}}%
\pgfpathcurveto{\pgfqpoint{3.302446in}{2.354351in}}{\pgfqpoint{3.306836in}{2.364950in}}{\pgfqpoint{3.306836in}{2.376000in}}%
\pgfpathcurveto{\pgfqpoint{3.306836in}{2.387050in}}{\pgfqpoint{3.302446in}{2.397649in}}{\pgfqpoint{3.294632in}{2.405463in}}%
\pgfpathcurveto{\pgfqpoint{3.286818in}{2.413276in}}{\pgfqpoint{3.276219in}{2.417667in}}{\pgfqpoint{3.265169in}{2.417667in}}%
\pgfpathcurveto{\pgfqpoint{3.254119in}{2.417667in}}{\pgfqpoint{3.243520in}{2.413276in}}{\pgfqpoint{3.235707in}{2.405463in}}%
\pgfpathcurveto{\pgfqpoint{3.227893in}{2.397649in}}{\pgfqpoint{3.223503in}{2.387050in}}{\pgfqpoint{3.223503in}{2.376000in}}%
\pgfpathcurveto{\pgfqpoint{3.223503in}{2.364950in}}{\pgfqpoint{3.227893in}{2.354351in}}{\pgfqpoint{3.235707in}{2.346537in}}%
\pgfpathcurveto{\pgfqpoint{3.243520in}{2.338724in}}{\pgfqpoint{3.254119in}{2.334333in}}{\pgfqpoint{3.265169in}{2.334333in}}%
\pgfpathclose%
\pgfusepath{stroke,fill}%
\end{pgfscope}%
\begin{pgfscope}%
\pgfpathrectangle{\pgfqpoint{0.800000in}{0.528000in}}{\pgfqpoint{4.960000in}{3.696000in}}%
\pgfusepath{clip}%
\pgfsetbuttcap%
\pgfsetroundjoin%
\definecolor{currentfill}{rgb}{0.000000,0.000000,0.000000}%
\pgfsetfillcolor{currentfill}%
\pgfsetlinewidth{1.003750pt}%
\definecolor{currentstroke}{rgb}{0.000000,0.000000,0.000000}%
\pgfsetstrokecolor{currentstroke}%
\pgfsetdash{}{0pt}%
\pgfpathmoveto{\pgfqpoint{3.265169in}{2.334333in}}%
\pgfpathcurveto{\pgfqpoint{3.276219in}{2.334333in}}{\pgfqpoint{3.286818in}{2.338724in}}{\pgfqpoint{3.294632in}{2.346537in}}%
\pgfpathcurveto{\pgfqpoint{3.302446in}{2.354351in}}{\pgfqpoint{3.306836in}{2.364950in}}{\pgfqpoint{3.306836in}{2.376000in}}%
\pgfpathcurveto{\pgfqpoint{3.306836in}{2.387050in}}{\pgfqpoint{3.302446in}{2.397649in}}{\pgfqpoint{3.294632in}{2.405463in}}%
\pgfpathcurveto{\pgfqpoint{3.286818in}{2.413276in}}{\pgfqpoint{3.276219in}{2.417667in}}{\pgfqpoint{3.265169in}{2.417667in}}%
\pgfpathcurveto{\pgfqpoint{3.254119in}{2.417667in}}{\pgfqpoint{3.243520in}{2.413276in}}{\pgfqpoint{3.235707in}{2.405463in}}%
\pgfpathcurveto{\pgfqpoint{3.227893in}{2.397649in}}{\pgfqpoint{3.223503in}{2.387050in}}{\pgfqpoint{3.223503in}{2.376000in}}%
\pgfpathcurveto{\pgfqpoint{3.223503in}{2.364950in}}{\pgfqpoint{3.227893in}{2.354351in}}{\pgfqpoint{3.235707in}{2.346537in}}%
\pgfpathcurveto{\pgfqpoint{3.243520in}{2.338724in}}{\pgfqpoint{3.254119in}{2.334333in}}{\pgfqpoint{3.265169in}{2.334333in}}%
\pgfpathclose%
\pgfusepath{stroke,fill}%
\end{pgfscope}%
\begin{pgfscope}%
\pgfpathrectangle{\pgfqpoint{0.800000in}{0.528000in}}{\pgfqpoint{4.960000in}{3.696000in}}%
\pgfusepath{clip}%
\pgfsetbuttcap%
\pgfsetroundjoin%
\definecolor{currentfill}{rgb}{0.000000,0.000000,0.000000}%
\pgfsetfillcolor{currentfill}%
\pgfsetlinewidth{1.003750pt}%
\definecolor{currentstroke}{rgb}{0.000000,0.000000,0.000000}%
\pgfsetstrokecolor{currentstroke}%
\pgfsetdash{}{0pt}%
\pgfpathmoveto{\pgfqpoint{3.265169in}{2.334333in}}%
\pgfpathcurveto{\pgfqpoint{3.276219in}{2.334333in}}{\pgfqpoint{3.286818in}{2.338724in}}{\pgfqpoint{3.294632in}{2.346537in}}%
\pgfpathcurveto{\pgfqpoint{3.302446in}{2.354351in}}{\pgfqpoint{3.306836in}{2.364950in}}{\pgfqpoint{3.306836in}{2.376000in}}%
\pgfpathcurveto{\pgfqpoint{3.306836in}{2.387050in}}{\pgfqpoint{3.302446in}{2.397649in}}{\pgfqpoint{3.294632in}{2.405463in}}%
\pgfpathcurveto{\pgfqpoint{3.286818in}{2.413276in}}{\pgfqpoint{3.276219in}{2.417667in}}{\pgfqpoint{3.265169in}{2.417667in}}%
\pgfpathcurveto{\pgfqpoint{3.254119in}{2.417667in}}{\pgfqpoint{3.243520in}{2.413276in}}{\pgfqpoint{3.235707in}{2.405463in}}%
\pgfpathcurveto{\pgfqpoint{3.227893in}{2.397649in}}{\pgfqpoint{3.223503in}{2.387050in}}{\pgfqpoint{3.223503in}{2.376000in}}%
\pgfpathcurveto{\pgfqpoint{3.223503in}{2.364950in}}{\pgfqpoint{3.227893in}{2.354351in}}{\pgfqpoint{3.235707in}{2.346537in}}%
\pgfpathcurveto{\pgfqpoint{3.243520in}{2.338724in}}{\pgfqpoint{3.254119in}{2.334333in}}{\pgfqpoint{3.265169in}{2.334333in}}%
\pgfpathclose%
\pgfusepath{stroke,fill}%
\end{pgfscope}%
\begin{pgfscope}%
\pgfpathrectangle{\pgfqpoint{0.800000in}{0.528000in}}{\pgfqpoint{4.960000in}{3.696000in}}%
\pgfusepath{clip}%
\pgfsetbuttcap%
\pgfsetroundjoin%
\definecolor{currentfill}{rgb}{0.000000,0.000000,0.000000}%
\pgfsetfillcolor{currentfill}%
\pgfsetlinewidth{1.003750pt}%
\definecolor{currentstroke}{rgb}{0.000000,0.000000,0.000000}%
\pgfsetstrokecolor{currentstroke}%
\pgfsetdash{}{0pt}%
\pgfpathmoveto{\pgfqpoint{3.265169in}{2.334333in}}%
\pgfpathcurveto{\pgfqpoint{3.276219in}{2.334333in}}{\pgfqpoint{3.286818in}{2.338724in}}{\pgfqpoint{3.294632in}{2.346537in}}%
\pgfpathcurveto{\pgfqpoint{3.302446in}{2.354351in}}{\pgfqpoint{3.306836in}{2.364950in}}{\pgfqpoint{3.306836in}{2.376000in}}%
\pgfpathcurveto{\pgfqpoint{3.306836in}{2.387050in}}{\pgfqpoint{3.302446in}{2.397649in}}{\pgfqpoint{3.294632in}{2.405463in}}%
\pgfpathcurveto{\pgfqpoint{3.286818in}{2.413276in}}{\pgfqpoint{3.276219in}{2.417667in}}{\pgfqpoint{3.265169in}{2.417667in}}%
\pgfpathcurveto{\pgfqpoint{3.254119in}{2.417667in}}{\pgfqpoint{3.243520in}{2.413276in}}{\pgfqpoint{3.235707in}{2.405463in}}%
\pgfpathcurveto{\pgfqpoint{3.227893in}{2.397649in}}{\pgfqpoint{3.223503in}{2.387050in}}{\pgfqpoint{3.223503in}{2.376000in}}%
\pgfpathcurveto{\pgfqpoint{3.223503in}{2.364950in}}{\pgfqpoint{3.227893in}{2.354351in}}{\pgfqpoint{3.235707in}{2.346537in}}%
\pgfpathcurveto{\pgfqpoint{3.243520in}{2.338724in}}{\pgfqpoint{3.254119in}{2.334333in}}{\pgfqpoint{3.265169in}{2.334333in}}%
\pgfpathclose%
\pgfusepath{stroke,fill}%
\end{pgfscope}%
\begin{pgfscope}%
\pgfpathrectangle{\pgfqpoint{0.800000in}{0.528000in}}{\pgfqpoint{4.960000in}{3.696000in}}%
\pgfusepath{clip}%
\pgfsetbuttcap%
\pgfsetroundjoin%
\definecolor{currentfill}{rgb}{0.000000,0.000000,0.000000}%
\pgfsetfillcolor{currentfill}%
\pgfsetlinewidth{1.003750pt}%
\definecolor{currentstroke}{rgb}{0.000000,0.000000,0.000000}%
\pgfsetstrokecolor{currentstroke}%
\pgfsetdash{}{0pt}%
\pgfpathmoveto{\pgfqpoint{3.265169in}{2.334333in}}%
\pgfpathcurveto{\pgfqpoint{3.276219in}{2.334333in}}{\pgfqpoint{3.286818in}{2.338724in}}{\pgfqpoint{3.294632in}{2.346537in}}%
\pgfpathcurveto{\pgfqpoint{3.302446in}{2.354351in}}{\pgfqpoint{3.306836in}{2.364950in}}{\pgfqpoint{3.306836in}{2.376000in}}%
\pgfpathcurveto{\pgfqpoint{3.306836in}{2.387050in}}{\pgfqpoint{3.302446in}{2.397649in}}{\pgfqpoint{3.294632in}{2.405463in}}%
\pgfpathcurveto{\pgfqpoint{3.286818in}{2.413276in}}{\pgfqpoint{3.276219in}{2.417667in}}{\pgfqpoint{3.265169in}{2.417667in}}%
\pgfpathcurveto{\pgfqpoint{3.254119in}{2.417667in}}{\pgfqpoint{3.243520in}{2.413276in}}{\pgfqpoint{3.235707in}{2.405463in}}%
\pgfpathcurveto{\pgfqpoint{3.227893in}{2.397649in}}{\pgfqpoint{3.223503in}{2.387050in}}{\pgfqpoint{3.223503in}{2.376000in}}%
\pgfpathcurveto{\pgfqpoint{3.223503in}{2.364950in}}{\pgfqpoint{3.227893in}{2.354351in}}{\pgfqpoint{3.235707in}{2.346537in}}%
\pgfpathcurveto{\pgfqpoint{3.243520in}{2.338724in}}{\pgfqpoint{3.254119in}{2.334333in}}{\pgfqpoint{3.265169in}{2.334333in}}%
\pgfpathclose%
\pgfusepath{stroke,fill}%
\end{pgfscope}%
\begin{pgfscope}%
\pgfpathrectangle{\pgfqpoint{0.800000in}{0.528000in}}{\pgfqpoint{4.960000in}{3.696000in}}%
\pgfusepath{clip}%
\pgfsetbuttcap%
\pgfsetroundjoin%
\definecolor{currentfill}{rgb}{0.000000,0.000000,0.000000}%
\pgfsetfillcolor{currentfill}%
\pgfsetlinewidth{1.003750pt}%
\definecolor{currentstroke}{rgb}{0.000000,0.000000,0.000000}%
\pgfsetstrokecolor{currentstroke}%
\pgfsetdash{}{0pt}%
\pgfpathmoveto{\pgfqpoint{3.265169in}{2.334333in}}%
\pgfpathcurveto{\pgfqpoint{3.276219in}{2.334333in}}{\pgfqpoint{3.286818in}{2.338724in}}{\pgfqpoint{3.294632in}{2.346537in}}%
\pgfpathcurveto{\pgfqpoint{3.302446in}{2.354351in}}{\pgfqpoint{3.306836in}{2.364950in}}{\pgfqpoint{3.306836in}{2.376000in}}%
\pgfpathcurveto{\pgfqpoint{3.306836in}{2.387050in}}{\pgfqpoint{3.302446in}{2.397649in}}{\pgfqpoint{3.294632in}{2.405463in}}%
\pgfpathcurveto{\pgfqpoint{3.286818in}{2.413276in}}{\pgfqpoint{3.276219in}{2.417667in}}{\pgfqpoint{3.265169in}{2.417667in}}%
\pgfpathcurveto{\pgfqpoint{3.254119in}{2.417667in}}{\pgfqpoint{3.243520in}{2.413276in}}{\pgfqpoint{3.235707in}{2.405463in}}%
\pgfpathcurveto{\pgfqpoint{3.227893in}{2.397649in}}{\pgfqpoint{3.223503in}{2.387050in}}{\pgfqpoint{3.223503in}{2.376000in}}%
\pgfpathcurveto{\pgfqpoint{3.223503in}{2.364950in}}{\pgfqpoint{3.227893in}{2.354351in}}{\pgfqpoint{3.235707in}{2.346537in}}%
\pgfpathcurveto{\pgfqpoint{3.243520in}{2.338724in}}{\pgfqpoint{3.254119in}{2.334333in}}{\pgfqpoint{3.265169in}{2.334333in}}%
\pgfpathclose%
\pgfusepath{stroke,fill}%
\end{pgfscope}%
\begin{pgfscope}%
\pgfpathrectangle{\pgfqpoint{0.800000in}{0.528000in}}{\pgfqpoint{4.960000in}{3.696000in}}%
\pgfusepath{clip}%
\pgfsetbuttcap%
\pgfsetroundjoin%
\definecolor{currentfill}{rgb}{0.000000,0.000000,0.000000}%
\pgfsetfillcolor{currentfill}%
\pgfsetlinewidth{1.003750pt}%
\definecolor{currentstroke}{rgb}{0.000000,0.000000,0.000000}%
\pgfsetstrokecolor{currentstroke}%
\pgfsetdash{}{0pt}%
\pgfpathmoveto{\pgfqpoint{3.265169in}{2.334333in}}%
\pgfpathcurveto{\pgfqpoint{3.276219in}{2.334333in}}{\pgfqpoint{3.286818in}{2.338724in}}{\pgfqpoint{3.294632in}{2.346537in}}%
\pgfpathcurveto{\pgfqpoint{3.302446in}{2.354351in}}{\pgfqpoint{3.306836in}{2.364950in}}{\pgfqpoint{3.306836in}{2.376000in}}%
\pgfpathcurveto{\pgfqpoint{3.306836in}{2.387050in}}{\pgfqpoint{3.302446in}{2.397649in}}{\pgfqpoint{3.294632in}{2.405463in}}%
\pgfpathcurveto{\pgfqpoint{3.286818in}{2.413276in}}{\pgfqpoint{3.276219in}{2.417667in}}{\pgfqpoint{3.265169in}{2.417667in}}%
\pgfpathcurveto{\pgfqpoint{3.254119in}{2.417667in}}{\pgfqpoint{3.243520in}{2.413276in}}{\pgfqpoint{3.235707in}{2.405463in}}%
\pgfpathcurveto{\pgfqpoint{3.227893in}{2.397649in}}{\pgfqpoint{3.223503in}{2.387050in}}{\pgfqpoint{3.223503in}{2.376000in}}%
\pgfpathcurveto{\pgfqpoint{3.223503in}{2.364950in}}{\pgfqpoint{3.227893in}{2.354351in}}{\pgfqpoint{3.235707in}{2.346537in}}%
\pgfpathcurveto{\pgfqpoint{3.243520in}{2.338724in}}{\pgfqpoint{3.254119in}{2.334333in}}{\pgfqpoint{3.265169in}{2.334333in}}%
\pgfpathclose%
\pgfusepath{stroke,fill}%
\end{pgfscope}%
\begin{pgfscope}%
\pgfpathrectangle{\pgfqpoint{0.800000in}{0.528000in}}{\pgfqpoint{4.960000in}{3.696000in}}%
\pgfusepath{clip}%
\pgfsetbuttcap%
\pgfsetroundjoin%
\definecolor{currentfill}{rgb}{0.000000,0.000000,0.000000}%
\pgfsetfillcolor{currentfill}%
\pgfsetlinewidth{1.003750pt}%
\definecolor{currentstroke}{rgb}{0.000000,0.000000,0.000000}%
\pgfsetstrokecolor{currentstroke}%
\pgfsetdash{}{0pt}%
\pgfpathmoveto{\pgfqpoint{3.265169in}{2.334333in}}%
\pgfpathcurveto{\pgfqpoint{3.276219in}{2.334333in}}{\pgfqpoint{3.286818in}{2.338724in}}{\pgfqpoint{3.294632in}{2.346537in}}%
\pgfpathcurveto{\pgfqpoint{3.302446in}{2.354351in}}{\pgfqpoint{3.306836in}{2.364950in}}{\pgfqpoint{3.306836in}{2.376000in}}%
\pgfpathcurveto{\pgfqpoint{3.306836in}{2.387050in}}{\pgfqpoint{3.302446in}{2.397649in}}{\pgfqpoint{3.294632in}{2.405463in}}%
\pgfpathcurveto{\pgfqpoint{3.286818in}{2.413276in}}{\pgfqpoint{3.276219in}{2.417667in}}{\pgfqpoint{3.265169in}{2.417667in}}%
\pgfpathcurveto{\pgfqpoint{3.254119in}{2.417667in}}{\pgfqpoint{3.243520in}{2.413276in}}{\pgfqpoint{3.235707in}{2.405463in}}%
\pgfpathcurveto{\pgfqpoint{3.227893in}{2.397649in}}{\pgfqpoint{3.223503in}{2.387050in}}{\pgfqpoint{3.223503in}{2.376000in}}%
\pgfpathcurveto{\pgfqpoint{3.223503in}{2.364950in}}{\pgfqpoint{3.227893in}{2.354351in}}{\pgfqpoint{3.235707in}{2.346537in}}%
\pgfpathcurveto{\pgfqpoint{3.243520in}{2.338724in}}{\pgfqpoint{3.254119in}{2.334333in}}{\pgfqpoint{3.265169in}{2.334333in}}%
\pgfpathclose%
\pgfusepath{stroke,fill}%
\end{pgfscope}%
\begin{pgfscope}%
\pgfpathrectangle{\pgfqpoint{0.800000in}{0.528000in}}{\pgfqpoint{4.960000in}{3.696000in}}%
\pgfusepath{clip}%
\pgfsetbuttcap%
\pgfsetroundjoin%
\definecolor{currentfill}{rgb}{0.000000,0.000000,0.000000}%
\pgfsetfillcolor{currentfill}%
\pgfsetlinewidth{1.003750pt}%
\definecolor{currentstroke}{rgb}{0.000000,0.000000,0.000000}%
\pgfsetstrokecolor{currentstroke}%
\pgfsetdash{}{0pt}%
\pgfpathmoveto{\pgfqpoint{3.265169in}{2.334333in}}%
\pgfpathcurveto{\pgfqpoint{3.276219in}{2.334333in}}{\pgfqpoint{3.286818in}{2.338724in}}{\pgfqpoint{3.294632in}{2.346537in}}%
\pgfpathcurveto{\pgfqpoint{3.302446in}{2.354351in}}{\pgfqpoint{3.306836in}{2.364950in}}{\pgfqpoint{3.306836in}{2.376000in}}%
\pgfpathcurveto{\pgfqpoint{3.306836in}{2.387050in}}{\pgfqpoint{3.302446in}{2.397649in}}{\pgfqpoint{3.294632in}{2.405463in}}%
\pgfpathcurveto{\pgfqpoint{3.286818in}{2.413276in}}{\pgfqpoint{3.276219in}{2.417667in}}{\pgfqpoint{3.265169in}{2.417667in}}%
\pgfpathcurveto{\pgfqpoint{3.254119in}{2.417667in}}{\pgfqpoint{3.243520in}{2.413276in}}{\pgfqpoint{3.235707in}{2.405463in}}%
\pgfpathcurveto{\pgfqpoint{3.227893in}{2.397649in}}{\pgfqpoint{3.223503in}{2.387050in}}{\pgfqpoint{3.223503in}{2.376000in}}%
\pgfpathcurveto{\pgfqpoint{3.223503in}{2.364950in}}{\pgfqpoint{3.227893in}{2.354351in}}{\pgfqpoint{3.235707in}{2.346537in}}%
\pgfpathcurveto{\pgfqpoint{3.243520in}{2.338724in}}{\pgfqpoint{3.254119in}{2.334333in}}{\pgfqpoint{3.265169in}{2.334333in}}%
\pgfpathclose%
\pgfusepath{stroke,fill}%
\end{pgfscope}%
\begin{pgfscope}%
\pgfpathrectangle{\pgfqpoint{0.800000in}{0.528000in}}{\pgfqpoint{4.960000in}{3.696000in}}%
\pgfusepath{clip}%
\pgfsetbuttcap%
\pgfsetroundjoin%
\definecolor{currentfill}{rgb}{0.000000,0.000000,0.000000}%
\pgfsetfillcolor{currentfill}%
\pgfsetlinewidth{1.003750pt}%
\definecolor{currentstroke}{rgb}{0.000000,0.000000,0.000000}%
\pgfsetstrokecolor{currentstroke}%
\pgfsetdash{}{0pt}%
\pgfpathmoveto{\pgfqpoint{3.265169in}{2.334333in}}%
\pgfpathcurveto{\pgfqpoint{3.276219in}{2.334333in}}{\pgfqpoint{3.286818in}{2.338724in}}{\pgfqpoint{3.294632in}{2.346537in}}%
\pgfpathcurveto{\pgfqpoint{3.302446in}{2.354351in}}{\pgfqpoint{3.306836in}{2.364950in}}{\pgfqpoint{3.306836in}{2.376000in}}%
\pgfpathcurveto{\pgfqpoint{3.306836in}{2.387050in}}{\pgfqpoint{3.302446in}{2.397649in}}{\pgfqpoint{3.294632in}{2.405463in}}%
\pgfpathcurveto{\pgfqpoint{3.286818in}{2.413276in}}{\pgfqpoint{3.276219in}{2.417667in}}{\pgfqpoint{3.265169in}{2.417667in}}%
\pgfpathcurveto{\pgfqpoint{3.254119in}{2.417667in}}{\pgfqpoint{3.243520in}{2.413276in}}{\pgfqpoint{3.235707in}{2.405463in}}%
\pgfpathcurveto{\pgfqpoint{3.227893in}{2.397649in}}{\pgfqpoint{3.223503in}{2.387050in}}{\pgfqpoint{3.223503in}{2.376000in}}%
\pgfpathcurveto{\pgfqpoint{3.223503in}{2.364950in}}{\pgfqpoint{3.227893in}{2.354351in}}{\pgfqpoint{3.235707in}{2.346537in}}%
\pgfpathcurveto{\pgfqpoint{3.243520in}{2.338724in}}{\pgfqpoint{3.254119in}{2.334333in}}{\pgfqpoint{3.265169in}{2.334333in}}%
\pgfpathclose%
\pgfusepath{stroke,fill}%
\end{pgfscope}%
\begin{pgfscope}%
\pgfpathrectangle{\pgfqpoint{0.800000in}{0.528000in}}{\pgfqpoint{4.960000in}{3.696000in}}%
\pgfusepath{clip}%
\pgfsetbuttcap%
\pgfsetroundjoin%
\definecolor{currentfill}{rgb}{0.000000,0.000000,0.000000}%
\pgfsetfillcolor{currentfill}%
\pgfsetlinewidth{1.003750pt}%
\definecolor{currentstroke}{rgb}{0.000000,0.000000,0.000000}%
\pgfsetstrokecolor{currentstroke}%
\pgfsetdash{}{0pt}%
\pgfpathmoveto{\pgfqpoint{3.265169in}{2.334333in}}%
\pgfpathcurveto{\pgfqpoint{3.276219in}{2.334333in}}{\pgfqpoint{3.286818in}{2.338724in}}{\pgfqpoint{3.294632in}{2.346537in}}%
\pgfpathcurveto{\pgfqpoint{3.302446in}{2.354351in}}{\pgfqpoint{3.306836in}{2.364950in}}{\pgfqpoint{3.306836in}{2.376000in}}%
\pgfpathcurveto{\pgfqpoint{3.306836in}{2.387050in}}{\pgfqpoint{3.302446in}{2.397649in}}{\pgfqpoint{3.294632in}{2.405463in}}%
\pgfpathcurveto{\pgfqpoint{3.286818in}{2.413276in}}{\pgfqpoint{3.276219in}{2.417667in}}{\pgfqpoint{3.265169in}{2.417667in}}%
\pgfpathcurveto{\pgfqpoint{3.254119in}{2.417667in}}{\pgfqpoint{3.243520in}{2.413276in}}{\pgfqpoint{3.235707in}{2.405463in}}%
\pgfpathcurveto{\pgfqpoint{3.227893in}{2.397649in}}{\pgfqpoint{3.223503in}{2.387050in}}{\pgfqpoint{3.223503in}{2.376000in}}%
\pgfpathcurveto{\pgfqpoint{3.223503in}{2.364950in}}{\pgfqpoint{3.227893in}{2.354351in}}{\pgfqpoint{3.235707in}{2.346537in}}%
\pgfpathcurveto{\pgfqpoint{3.243520in}{2.338724in}}{\pgfqpoint{3.254119in}{2.334333in}}{\pgfqpoint{3.265169in}{2.334333in}}%
\pgfpathclose%
\pgfusepath{stroke,fill}%
\end{pgfscope}%
\begin{pgfscope}%
\pgfpathrectangle{\pgfqpoint{0.800000in}{0.528000in}}{\pgfqpoint{4.960000in}{3.696000in}}%
\pgfusepath{clip}%
\pgfsetbuttcap%
\pgfsetroundjoin%
\definecolor{currentfill}{rgb}{0.000000,0.000000,0.000000}%
\pgfsetfillcolor{currentfill}%
\pgfsetlinewidth{1.003750pt}%
\definecolor{currentstroke}{rgb}{0.000000,0.000000,0.000000}%
\pgfsetstrokecolor{currentstroke}%
\pgfsetdash{}{0pt}%
\pgfpathmoveto{\pgfqpoint{3.265169in}{2.334333in}}%
\pgfpathcurveto{\pgfqpoint{3.276219in}{2.334333in}}{\pgfqpoint{3.286818in}{2.338724in}}{\pgfqpoint{3.294632in}{2.346537in}}%
\pgfpathcurveto{\pgfqpoint{3.302446in}{2.354351in}}{\pgfqpoint{3.306836in}{2.364950in}}{\pgfqpoint{3.306836in}{2.376000in}}%
\pgfpathcurveto{\pgfqpoint{3.306836in}{2.387050in}}{\pgfqpoint{3.302446in}{2.397649in}}{\pgfqpoint{3.294632in}{2.405463in}}%
\pgfpathcurveto{\pgfqpoint{3.286818in}{2.413276in}}{\pgfqpoint{3.276219in}{2.417667in}}{\pgfqpoint{3.265169in}{2.417667in}}%
\pgfpathcurveto{\pgfqpoint{3.254119in}{2.417667in}}{\pgfqpoint{3.243520in}{2.413276in}}{\pgfqpoint{3.235707in}{2.405463in}}%
\pgfpathcurveto{\pgfqpoint{3.227893in}{2.397649in}}{\pgfqpoint{3.223503in}{2.387050in}}{\pgfqpoint{3.223503in}{2.376000in}}%
\pgfpathcurveto{\pgfqpoint{3.223503in}{2.364950in}}{\pgfqpoint{3.227893in}{2.354351in}}{\pgfqpoint{3.235707in}{2.346537in}}%
\pgfpathcurveto{\pgfqpoint{3.243520in}{2.338724in}}{\pgfqpoint{3.254119in}{2.334333in}}{\pgfqpoint{3.265169in}{2.334333in}}%
\pgfpathclose%
\pgfusepath{stroke,fill}%
\end{pgfscope}%
\begin{pgfscope}%
\pgfpathrectangle{\pgfqpoint{0.800000in}{0.528000in}}{\pgfqpoint{4.960000in}{3.696000in}}%
\pgfusepath{clip}%
\pgfsetbuttcap%
\pgfsetroundjoin%
\definecolor{currentfill}{rgb}{0.000000,0.000000,0.000000}%
\pgfsetfillcolor{currentfill}%
\pgfsetlinewidth{1.003750pt}%
\definecolor{currentstroke}{rgb}{0.000000,0.000000,0.000000}%
\pgfsetstrokecolor{currentstroke}%
\pgfsetdash{}{0pt}%
\pgfpathmoveto{\pgfqpoint{3.265169in}{2.334333in}}%
\pgfpathcurveto{\pgfqpoint{3.276219in}{2.334333in}}{\pgfqpoint{3.286818in}{2.338724in}}{\pgfqpoint{3.294632in}{2.346537in}}%
\pgfpathcurveto{\pgfqpoint{3.302446in}{2.354351in}}{\pgfqpoint{3.306836in}{2.364950in}}{\pgfqpoint{3.306836in}{2.376000in}}%
\pgfpathcurveto{\pgfqpoint{3.306836in}{2.387050in}}{\pgfqpoint{3.302446in}{2.397649in}}{\pgfqpoint{3.294632in}{2.405463in}}%
\pgfpathcurveto{\pgfqpoint{3.286818in}{2.413276in}}{\pgfqpoint{3.276219in}{2.417667in}}{\pgfqpoint{3.265169in}{2.417667in}}%
\pgfpathcurveto{\pgfqpoint{3.254119in}{2.417667in}}{\pgfqpoint{3.243520in}{2.413276in}}{\pgfqpoint{3.235707in}{2.405463in}}%
\pgfpathcurveto{\pgfqpoint{3.227893in}{2.397649in}}{\pgfqpoint{3.223503in}{2.387050in}}{\pgfqpoint{3.223503in}{2.376000in}}%
\pgfpathcurveto{\pgfqpoint{3.223503in}{2.364950in}}{\pgfqpoint{3.227893in}{2.354351in}}{\pgfqpoint{3.235707in}{2.346537in}}%
\pgfpathcurveto{\pgfqpoint{3.243520in}{2.338724in}}{\pgfqpoint{3.254119in}{2.334333in}}{\pgfqpoint{3.265169in}{2.334333in}}%
\pgfpathclose%
\pgfusepath{stroke,fill}%
\end{pgfscope}%
\begin{pgfscope}%
\pgfpathrectangle{\pgfqpoint{0.800000in}{0.528000in}}{\pgfqpoint{4.960000in}{3.696000in}}%
\pgfusepath{clip}%
\pgfsetbuttcap%
\pgfsetroundjoin%
\definecolor{currentfill}{rgb}{0.000000,0.000000,0.000000}%
\pgfsetfillcolor{currentfill}%
\pgfsetlinewidth{1.003750pt}%
\definecolor{currentstroke}{rgb}{0.000000,0.000000,0.000000}%
\pgfsetstrokecolor{currentstroke}%
\pgfsetdash{}{0pt}%
\pgfpathmoveto{\pgfqpoint{3.265169in}{2.334333in}}%
\pgfpathcurveto{\pgfqpoint{3.276219in}{2.334333in}}{\pgfqpoint{3.286818in}{2.338724in}}{\pgfqpoint{3.294632in}{2.346537in}}%
\pgfpathcurveto{\pgfqpoint{3.302446in}{2.354351in}}{\pgfqpoint{3.306836in}{2.364950in}}{\pgfqpoint{3.306836in}{2.376000in}}%
\pgfpathcurveto{\pgfqpoint{3.306836in}{2.387050in}}{\pgfqpoint{3.302446in}{2.397649in}}{\pgfqpoint{3.294632in}{2.405463in}}%
\pgfpathcurveto{\pgfqpoint{3.286818in}{2.413276in}}{\pgfqpoint{3.276219in}{2.417667in}}{\pgfqpoint{3.265169in}{2.417667in}}%
\pgfpathcurveto{\pgfqpoint{3.254119in}{2.417667in}}{\pgfqpoint{3.243520in}{2.413276in}}{\pgfqpoint{3.235707in}{2.405463in}}%
\pgfpathcurveto{\pgfqpoint{3.227893in}{2.397649in}}{\pgfqpoint{3.223503in}{2.387050in}}{\pgfqpoint{3.223503in}{2.376000in}}%
\pgfpathcurveto{\pgfqpoint{3.223503in}{2.364950in}}{\pgfqpoint{3.227893in}{2.354351in}}{\pgfqpoint{3.235707in}{2.346537in}}%
\pgfpathcurveto{\pgfqpoint{3.243520in}{2.338724in}}{\pgfqpoint{3.254119in}{2.334333in}}{\pgfqpoint{3.265169in}{2.334333in}}%
\pgfpathclose%
\pgfusepath{stroke,fill}%
\end{pgfscope}%
\begin{pgfscope}%
\pgfpathrectangle{\pgfqpoint{0.800000in}{0.528000in}}{\pgfqpoint{4.960000in}{3.696000in}}%
\pgfusepath{clip}%
\pgfsetbuttcap%
\pgfsetroundjoin%
\definecolor{currentfill}{rgb}{0.000000,0.000000,0.000000}%
\pgfsetfillcolor{currentfill}%
\pgfsetlinewidth{1.003750pt}%
\definecolor{currentstroke}{rgb}{0.000000,0.000000,0.000000}%
\pgfsetstrokecolor{currentstroke}%
\pgfsetdash{}{0pt}%
\pgfpathmoveto{\pgfqpoint{3.265169in}{2.334333in}}%
\pgfpathcurveto{\pgfqpoint{3.276219in}{2.334333in}}{\pgfqpoint{3.286818in}{2.338724in}}{\pgfqpoint{3.294632in}{2.346537in}}%
\pgfpathcurveto{\pgfqpoint{3.302446in}{2.354351in}}{\pgfqpoint{3.306836in}{2.364950in}}{\pgfqpoint{3.306836in}{2.376000in}}%
\pgfpathcurveto{\pgfqpoint{3.306836in}{2.387050in}}{\pgfqpoint{3.302446in}{2.397649in}}{\pgfqpoint{3.294632in}{2.405463in}}%
\pgfpathcurveto{\pgfqpoint{3.286818in}{2.413276in}}{\pgfqpoint{3.276219in}{2.417667in}}{\pgfqpoint{3.265169in}{2.417667in}}%
\pgfpathcurveto{\pgfqpoint{3.254119in}{2.417667in}}{\pgfqpoint{3.243520in}{2.413276in}}{\pgfqpoint{3.235707in}{2.405463in}}%
\pgfpathcurveto{\pgfqpoint{3.227893in}{2.397649in}}{\pgfqpoint{3.223503in}{2.387050in}}{\pgfqpoint{3.223503in}{2.376000in}}%
\pgfpathcurveto{\pgfqpoint{3.223503in}{2.364950in}}{\pgfqpoint{3.227893in}{2.354351in}}{\pgfqpoint{3.235707in}{2.346537in}}%
\pgfpathcurveto{\pgfqpoint{3.243520in}{2.338724in}}{\pgfqpoint{3.254119in}{2.334333in}}{\pgfqpoint{3.265169in}{2.334333in}}%
\pgfpathclose%
\pgfusepath{stroke,fill}%
\end{pgfscope}%
\begin{pgfscope}%
\pgfpathrectangle{\pgfqpoint{0.800000in}{0.528000in}}{\pgfqpoint{4.960000in}{3.696000in}}%
\pgfusepath{clip}%
\pgfsetbuttcap%
\pgfsetroundjoin%
\definecolor{currentfill}{rgb}{0.000000,0.000000,0.000000}%
\pgfsetfillcolor{currentfill}%
\pgfsetlinewidth{1.003750pt}%
\definecolor{currentstroke}{rgb}{0.000000,0.000000,0.000000}%
\pgfsetstrokecolor{currentstroke}%
\pgfsetdash{}{0pt}%
\pgfpathmoveto{\pgfqpoint{3.265169in}{2.334333in}}%
\pgfpathcurveto{\pgfqpoint{3.276219in}{2.334333in}}{\pgfqpoint{3.286818in}{2.338724in}}{\pgfqpoint{3.294632in}{2.346537in}}%
\pgfpathcurveto{\pgfqpoint{3.302446in}{2.354351in}}{\pgfqpoint{3.306836in}{2.364950in}}{\pgfqpoint{3.306836in}{2.376000in}}%
\pgfpathcurveto{\pgfqpoint{3.306836in}{2.387050in}}{\pgfqpoint{3.302446in}{2.397649in}}{\pgfqpoint{3.294632in}{2.405463in}}%
\pgfpathcurveto{\pgfqpoint{3.286818in}{2.413276in}}{\pgfqpoint{3.276219in}{2.417667in}}{\pgfqpoint{3.265169in}{2.417667in}}%
\pgfpathcurveto{\pgfqpoint{3.254119in}{2.417667in}}{\pgfqpoint{3.243520in}{2.413276in}}{\pgfqpoint{3.235707in}{2.405463in}}%
\pgfpathcurveto{\pgfqpoint{3.227893in}{2.397649in}}{\pgfqpoint{3.223503in}{2.387050in}}{\pgfqpoint{3.223503in}{2.376000in}}%
\pgfpathcurveto{\pgfqpoint{3.223503in}{2.364950in}}{\pgfqpoint{3.227893in}{2.354351in}}{\pgfqpoint{3.235707in}{2.346537in}}%
\pgfpathcurveto{\pgfqpoint{3.243520in}{2.338724in}}{\pgfqpoint{3.254119in}{2.334333in}}{\pgfqpoint{3.265169in}{2.334333in}}%
\pgfpathclose%
\pgfusepath{stroke,fill}%
\end{pgfscope}%
\begin{pgfscope}%
\pgfpathrectangle{\pgfqpoint{0.800000in}{0.528000in}}{\pgfqpoint{4.960000in}{3.696000in}}%
\pgfusepath{clip}%
\pgfsetbuttcap%
\pgfsetroundjoin%
\definecolor{currentfill}{rgb}{0.000000,0.000000,0.000000}%
\pgfsetfillcolor{currentfill}%
\pgfsetlinewidth{1.003750pt}%
\definecolor{currentstroke}{rgb}{0.000000,0.000000,0.000000}%
\pgfsetstrokecolor{currentstroke}%
\pgfsetdash{}{0pt}%
\pgfpathmoveto{\pgfqpoint{3.265169in}{2.334333in}}%
\pgfpathcurveto{\pgfqpoint{3.276219in}{2.334333in}}{\pgfqpoint{3.286818in}{2.338724in}}{\pgfqpoint{3.294632in}{2.346537in}}%
\pgfpathcurveto{\pgfqpoint{3.302446in}{2.354351in}}{\pgfqpoint{3.306836in}{2.364950in}}{\pgfqpoint{3.306836in}{2.376000in}}%
\pgfpathcurveto{\pgfqpoint{3.306836in}{2.387050in}}{\pgfqpoint{3.302446in}{2.397649in}}{\pgfqpoint{3.294632in}{2.405463in}}%
\pgfpathcurveto{\pgfqpoint{3.286818in}{2.413276in}}{\pgfqpoint{3.276219in}{2.417667in}}{\pgfqpoint{3.265169in}{2.417667in}}%
\pgfpathcurveto{\pgfqpoint{3.254119in}{2.417667in}}{\pgfqpoint{3.243520in}{2.413276in}}{\pgfqpoint{3.235707in}{2.405463in}}%
\pgfpathcurveto{\pgfqpoint{3.227893in}{2.397649in}}{\pgfqpoint{3.223503in}{2.387050in}}{\pgfqpoint{3.223503in}{2.376000in}}%
\pgfpathcurveto{\pgfqpoint{3.223503in}{2.364950in}}{\pgfqpoint{3.227893in}{2.354351in}}{\pgfqpoint{3.235707in}{2.346537in}}%
\pgfpathcurveto{\pgfqpoint{3.243520in}{2.338724in}}{\pgfqpoint{3.254119in}{2.334333in}}{\pgfqpoint{3.265169in}{2.334333in}}%
\pgfpathclose%
\pgfusepath{stroke,fill}%
\end{pgfscope}%
\begin{pgfscope}%
\pgfpathrectangle{\pgfqpoint{0.800000in}{0.528000in}}{\pgfqpoint{4.960000in}{3.696000in}}%
\pgfusepath{clip}%
\pgfsetbuttcap%
\pgfsetroundjoin%
\definecolor{currentfill}{rgb}{0.000000,0.000000,0.000000}%
\pgfsetfillcolor{currentfill}%
\pgfsetlinewidth{1.003750pt}%
\definecolor{currentstroke}{rgb}{0.000000,0.000000,0.000000}%
\pgfsetstrokecolor{currentstroke}%
\pgfsetdash{}{0pt}%
\pgfpathmoveto{\pgfqpoint{3.265169in}{2.334333in}}%
\pgfpathcurveto{\pgfqpoint{3.276219in}{2.334333in}}{\pgfqpoint{3.286818in}{2.338724in}}{\pgfqpoint{3.294632in}{2.346537in}}%
\pgfpathcurveto{\pgfqpoint{3.302446in}{2.354351in}}{\pgfqpoint{3.306836in}{2.364950in}}{\pgfqpoint{3.306836in}{2.376000in}}%
\pgfpathcurveto{\pgfqpoint{3.306836in}{2.387050in}}{\pgfqpoint{3.302446in}{2.397649in}}{\pgfqpoint{3.294632in}{2.405463in}}%
\pgfpathcurveto{\pgfqpoint{3.286818in}{2.413276in}}{\pgfqpoint{3.276219in}{2.417667in}}{\pgfqpoint{3.265169in}{2.417667in}}%
\pgfpathcurveto{\pgfqpoint{3.254119in}{2.417667in}}{\pgfqpoint{3.243520in}{2.413276in}}{\pgfqpoint{3.235707in}{2.405463in}}%
\pgfpathcurveto{\pgfqpoint{3.227893in}{2.397649in}}{\pgfqpoint{3.223503in}{2.387050in}}{\pgfqpoint{3.223503in}{2.376000in}}%
\pgfpathcurveto{\pgfqpoint{3.223503in}{2.364950in}}{\pgfqpoint{3.227893in}{2.354351in}}{\pgfqpoint{3.235707in}{2.346537in}}%
\pgfpathcurveto{\pgfqpoint{3.243520in}{2.338724in}}{\pgfqpoint{3.254119in}{2.334333in}}{\pgfqpoint{3.265169in}{2.334333in}}%
\pgfpathclose%
\pgfusepath{stroke,fill}%
\end{pgfscope}%
\begin{pgfscope}%
\pgfpathrectangle{\pgfqpoint{0.800000in}{0.528000in}}{\pgfqpoint{4.960000in}{3.696000in}}%
\pgfusepath{clip}%
\pgfsetbuttcap%
\pgfsetroundjoin%
\definecolor{currentfill}{rgb}{0.000000,0.000000,0.000000}%
\pgfsetfillcolor{currentfill}%
\pgfsetlinewidth{1.003750pt}%
\definecolor{currentstroke}{rgb}{0.000000,0.000000,0.000000}%
\pgfsetstrokecolor{currentstroke}%
\pgfsetdash{}{0pt}%
\pgfpathmoveto{\pgfqpoint{3.265169in}{2.334333in}}%
\pgfpathcurveto{\pgfqpoint{3.276219in}{2.334333in}}{\pgfqpoint{3.286818in}{2.338724in}}{\pgfqpoint{3.294632in}{2.346537in}}%
\pgfpathcurveto{\pgfqpoint{3.302446in}{2.354351in}}{\pgfqpoint{3.306836in}{2.364950in}}{\pgfqpoint{3.306836in}{2.376000in}}%
\pgfpathcurveto{\pgfqpoint{3.306836in}{2.387050in}}{\pgfqpoint{3.302446in}{2.397649in}}{\pgfqpoint{3.294632in}{2.405463in}}%
\pgfpathcurveto{\pgfqpoint{3.286818in}{2.413276in}}{\pgfqpoint{3.276219in}{2.417667in}}{\pgfqpoint{3.265169in}{2.417667in}}%
\pgfpathcurveto{\pgfqpoint{3.254119in}{2.417667in}}{\pgfqpoint{3.243520in}{2.413276in}}{\pgfqpoint{3.235707in}{2.405463in}}%
\pgfpathcurveto{\pgfqpoint{3.227893in}{2.397649in}}{\pgfqpoint{3.223503in}{2.387050in}}{\pgfqpoint{3.223503in}{2.376000in}}%
\pgfpathcurveto{\pgfqpoint{3.223503in}{2.364950in}}{\pgfqpoint{3.227893in}{2.354351in}}{\pgfqpoint{3.235707in}{2.346537in}}%
\pgfpathcurveto{\pgfqpoint{3.243520in}{2.338724in}}{\pgfqpoint{3.254119in}{2.334333in}}{\pgfqpoint{3.265169in}{2.334333in}}%
\pgfpathclose%
\pgfusepath{stroke,fill}%
\end{pgfscope}%
\begin{pgfscope}%
\pgfpathrectangle{\pgfqpoint{0.800000in}{0.528000in}}{\pgfqpoint{4.960000in}{3.696000in}}%
\pgfusepath{clip}%
\pgfsetbuttcap%
\pgfsetroundjoin%
\definecolor{currentfill}{rgb}{0.000000,0.000000,0.000000}%
\pgfsetfillcolor{currentfill}%
\pgfsetlinewidth{1.003750pt}%
\definecolor{currentstroke}{rgb}{0.000000,0.000000,0.000000}%
\pgfsetstrokecolor{currentstroke}%
\pgfsetdash{}{0pt}%
\pgfpathmoveto{\pgfqpoint{3.265169in}{2.334333in}}%
\pgfpathcurveto{\pgfqpoint{3.276219in}{2.334333in}}{\pgfqpoint{3.286818in}{2.338724in}}{\pgfqpoint{3.294632in}{2.346537in}}%
\pgfpathcurveto{\pgfqpoint{3.302446in}{2.354351in}}{\pgfqpoint{3.306836in}{2.364950in}}{\pgfqpoint{3.306836in}{2.376000in}}%
\pgfpathcurveto{\pgfqpoint{3.306836in}{2.387050in}}{\pgfqpoint{3.302446in}{2.397649in}}{\pgfqpoint{3.294632in}{2.405463in}}%
\pgfpathcurveto{\pgfqpoint{3.286818in}{2.413276in}}{\pgfqpoint{3.276219in}{2.417667in}}{\pgfqpoint{3.265169in}{2.417667in}}%
\pgfpathcurveto{\pgfqpoint{3.254119in}{2.417667in}}{\pgfqpoint{3.243520in}{2.413276in}}{\pgfqpoint{3.235707in}{2.405463in}}%
\pgfpathcurveto{\pgfqpoint{3.227893in}{2.397649in}}{\pgfqpoint{3.223503in}{2.387050in}}{\pgfqpoint{3.223503in}{2.376000in}}%
\pgfpathcurveto{\pgfqpoint{3.223503in}{2.364950in}}{\pgfqpoint{3.227893in}{2.354351in}}{\pgfqpoint{3.235707in}{2.346537in}}%
\pgfpathcurveto{\pgfqpoint{3.243520in}{2.338724in}}{\pgfqpoint{3.254119in}{2.334333in}}{\pgfqpoint{3.265169in}{2.334333in}}%
\pgfpathclose%
\pgfusepath{stroke,fill}%
\end{pgfscope}%
\begin{pgfscope}%
\pgfpathrectangle{\pgfqpoint{0.800000in}{0.528000in}}{\pgfqpoint{4.960000in}{3.696000in}}%
\pgfusepath{clip}%
\pgfsetbuttcap%
\pgfsetroundjoin%
\definecolor{currentfill}{rgb}{0.000000,0.000000,0.000000}%
\pgfsetfillcolor{currentfill}%
\pgfsetlinewidth{1.003750pt}%
\definecolor{currentstroke}{rgb}{0.000000,0.000000,0.000000}%
\pgfsetstrokecolor{currentstroke}%
\pgfsetdash{}{0pt}%
\pgfpathmoveto{\pgfqpoint{3.265169in}{2.334333in}}%
\pgfpathcurveto{\pgfqpoint{3.276219in}{2.334333in}}{\pgfqpoint{3.286818in}{2.338724in}}{\pgfqpoint{3.294632in}{2.346537in}}%
\pgfpathcurveto{\pgfqpoint{3.302446in}{2.354351in}}{\pgfqpoint{3.306836in}{2.364950in}}{\pgfqpoint{3.306836in}{2.376000in}}%
\pgfpathcurveto{\pgfqpoint{3.306836in}{2.387050in}}{\pgfqpoint{3.302446in}{2.397649in}}{\pgfqpoint{3.294632in}{2.405463in}}%
\pgfpathcurveto{\pgfqpoint{3.286818in}{2.413276in}}{\pgfqpoint{3.276219in}{2.417667in}}{\pgfqpoint{3.265169in}{2.417667in}}%
\pgfpathcurveto{\pgfqpoint{3.254119in}{2.417667in}}{\pgfqpoint{3.243520in}{2.413276in}}{\pgfqpoint{3.235707in}{2.405463in}}%
\pgfpathcurveto{\pgfqpoint{3.227893in}{2.397649in}}{\pgfqpoint{3.223503in}{2.387050in}}{\pgfqpoint{3.223503in}{2.376000in}}%
\pgfpathcurveto{\pgfqpoint{3.223503in}{2.364950in}}{\pgfqpoint{3.227893in}{2.354351in}}{\pgfqpoint{3.235707in}{2.346537in}}%
\pgfpathcurveto{\pgfqpoint{3.243520in}{2.338724in}}{\pgfqpoint{3.254119in}{2.334333in}}{\pgfqpoint{3.265169in}{2.334333in}}%
\pgfpathclose%
\pgfusepath{stroke,fill}%
\end{pgfscope}%
\begin{pgfscope}%
\pgfpathrectangle{\pgfqpoint{0.800000in}{0.528000in}}{\pgfqpoint{4.960000in}{3.696000in}}%
\pgfusepath{clip}%
\pgfsetbuttcap%
\pgfsetroundjoin%
\definecolor{currentfill}{rgb}{0.000000,0.000000,0.000000}%
\pgfsetfillcolor{currentfill}%
\pgfsetlinewidth{1.003750pt}%
\definecolor{currentstroke}{rgb}{0.000000,0.000000,0.000000}%
\pgfsetstrokecolor{currentstroke}%
\pgfsetdash{}{0pt}%
\pgfpathmoveto{\pgfqpoint{3.265169in}{2.334333in}}%
\pgfpathcurveto{\pgfqpoint{3.276219in}{2.334333in}}{\pgfqpoint{3.286818in}{2.338724in}}{\pgfqpoint{3.294632in}{2.346537in}}%
\pgfpathcurveto{\pgfqpoint{3.302446in}{2.354351in}}{\pgfqpoint{3.306836in}{2.364950in}}{\pgfqpoint{3.306836in}{2.376000in}}%
\pgfpathcurveto{\pgfqpoint{3.306836in}{2.387050in}}{\pgfqpoint{3.302446in}{2.397649in}}{\pgfqpoint{3.294632in}{2.405463in}}%
\pgfpathcurveto{\pgfqpoint{3.286818in}{2.413276in}}{\pgfqpoint{3.276219in}{2.417667in}}{\pgfqpoint{3.265169in}{2.417667in}}%
\pgfpathcurveto{\pgfqpoint{3.254119in}{2.417667in}}{\pgfqpoint{3.243520in}{2.413276in}}{\pgfqpoint{3.235707in}{2.405463in}}%
\pgfpathcurveto{\pgfqpoint{3.227893in}{2.397649in}}{\pgfqpoint{3.223503in}{2.387050in}}{\pgfqpoint{3.223503in}{2.376000in}}%
\pgfpathcurveto{\pgfqpoint{3.223503in}{2.364950in}}{\pgfqpoint{3.227893in}{2.354351in}}{\pgfqpoint{3.235707in}{2.346537in}}%
\pgfpathcurveto{\pgfqpoint{3.243520in}{2.338724in}}{\pgfqpoint{3.254119in}{2.334333in}}{\pgfqpoint{3.265169in}{2.334333in}}%
\pgfpathclose%
\pgfusepath{stroke,fill}%
\end{pgfscope}%
\begin{pgfscope}%
\pgfpathrectangle{\pgfqpoint{0.800000in}{0.528000in}}{\pgfqpoint{4.960000in}{3.696000in}}%
\pgfusepath{clip}%
\pgfsetbuttcap%
\pgfsetroundjoin%
\definecolor{currentfill}{rgb}{0.000000,0.000000,0.000000}%
\pgfsetfillcolor{currentfill}%
\pgfsetlinewidth{1.003750pt}%
\definecolor{currentstroke}{rgb}{0.000000,0.000000,0.000000}%
\pgfsetstrokecolor{currentstroke}%
\pgfsetdash{}{0pt}%
\pgfpathmoveto{\pgfqpoint{3.265169in}{2.334333in}}%
\pgfpathcurveto{\pgfqpoint{3.276219in}{2.334333in}}{\pgfqpoint{3.286818in}{2.338724in}}{\pgfqpoint{3.294632in}{2.346537in}}%
\pgfpathcurveto{\pgfqpoint{3.302446in}{2.354351in}}{\pgfqpoint{3.306836in}{2.364950in}}{\pgfqpoint{3.306836in}{2.376000in}}%
\pgfpathcurveto{\pgfqpoint{3.306836in}{2.387050in}}{\pgfqpoint{3.302446in}{2.397649in}}{\pgfqpoint{3.294632in}{2.405463in}}%
\pgfpathcurveto{\pgfqpoint{3.286818in}{2.413276in}}{\pgfqpoint{3.276219in}{2.417667in}}{\pgfqpoint{3.265169in}{2.417667in}}%
\pgfpathcurveto{\pgfqpoint{3.254119in}{2.417667in}}{\pgfqpoint{3.243520in}{2.413276in}}{\pgfqpoint{3.235707in}{2.405463in}}%
\pgfpathcurveto{\pgfqpoint{3.227893in}{2.397649in}}{\pgfqpoint{3.223503in}{2.387050in}}{\pgfqpoint{3.223503in}{2.376000in}}%
\pgfpathcurveto{\pgfqpoint{3.223503in}{2.364950in}}{\pgfqpoint{3.227893in}{2.354351in}}{\pgfqpoint{3.235707in}{2.346537in}}%
\pgfpathcurveto{\pgfqpoint{3.243520in}{2.338724in}}{\pgfqpoint{3.254119in}{2.334333in}}{\pgfqpoint{3.265169in}{2.334333in}}%
\pgfpathclose%
\pgfusepath{stroke,fill}%
\end{pgfscope}%
\begin{pgfscope}%
\pgfpathrectangle{\pgfqpoint{0.800000in}{0.528000in}}{\pgfqpoint{4.960000in}{3.696000in}}%
\pgfusepath{clip}%
\pgfsetbuttcap%
\pgfsetroundjoin%
\definecolor{currentfill}{rgb}{0.000000,0.000000,0.000000}%
\pgfsetfillcolor{currentfill}%
\pgfsetlinewidth{1.003750pt}%
\definecolor{currentstroke}{rgb}{0.000000,0.000000,0.000000}%
\pgfsetstrokecolor{currentstroke}%
\pgfsetdash{}{0pt}%
\pgfpathmoveto{\pgfqpoint{3.265169in}{2.334333in}}%
\pgfpathcurveto{\pgfqpoint{3.276219in}{2.334333in}}{\pgfqpoint{3.286818in}{2.338724in}}{\pgfqpoint{3.294632in}{2.346537in}}%
\pgfpathcurveto{\pgfqpoint{3.302446in}{2.354351in}}{\pgfqpoint{3.306836in}{2.364950in}}{\pgfqpoint{3.306836in}{2.376000in}}%
\pgfpathcurveto{\pgfqpoint{3.306836in}{2.387050in}}{\pgfqpoint{3.302446in}{2.397649in}}{\pgfqpoint{3.294632in}{2.405463in}}%
\pgfpathcurveto{\pgfqpoint{3.286818in}{2.413276in}}{\pgfqpoint{3.276219in}{2.417667in}}{\pgfqpoint{3.265169in}{2.417667in}}%
\pgfpathcurveto{\pgfqpoint{3.254119in}{2.417667in}}{\pgfqpoint{3.243520in}{2.413276in}}{\pgfqpoint{3.235707in}{2.405463in}}%
\pgfpathcurveto{\pgfqpoint{3.227893in}{2.397649in}}{\pgfqpoint{3.223503in}{2.387050in}}{\pgfqpoint{3.223503in}{2.376000in}}%
\pgfpathcurveto{\pgfqpoint{3.223503in}{2.364950in}}{\pgfqpoint{3.227893in}{2.354351in}}{\pgfqpoint{3.235707in}{2.346537in}}%
\pgfpathcurveto{\pgfqpoint{3.243520in}{2.338724in}}{\pgfqpoint{3.254119in}{2.334333in}}{\pgfqpoint{3.265169in}{2.334333in}}%
\pgfpathclose%
\pgfusepath{stroke,fill}%
\end{pgfscope}%
\begin{pgfscope}%
\pgfpathrectangle{\pgfqpoint{0.800000in}{0.528000in}}{\pgfqpoint{4.960000in}{3.696000in}}%
\pgfusepath{clip}%
\pgfsetbuttcap%
\pgfsetroundjoin%
\definecolor{currentfill}{rgb}{0.000000,0.000000,0.000000}%
\pgfsetfillcolor{currentfill}%
\pgfsetlinewidth{1.003750pt}%
\definecolor{currentstroke}{rgb}{0.000000,0.000000,0.000000}%
\pgfsetstrokecolor{currentstroke}%
\pgfsetdash{}{0pt}%
\pgfpathmoveto{\pgfqpoint{3.265169in}{2.334333in}}%
\pgfpathcurveto{\pgfqpoint{3.276219in}{2.334333in}}{\pgfqpoint{3.286818in}{2.338724in}}{\pgfqpoint{3.294632in}{2.346537in}}%
\pgfpathcurveto{\pgfqpoint{3.302446in}{2.354351in}}{\pgfqpoint{3.306836in}{2.364950in}}{\pgfqpoint{3.306836in}{2.376000in}}%
\pgfpathcurveto{\pgfqpoint{3.306836in}{2.387050in}}{\pgfqpoint{3.302446in}{2.397649in}}{\pgfqpoint{3.294632in}{2.405463in}}%
\pgfpathcurveto{\pgfqpoint{3.286818in}{2.413276in}}{\pgfqpoint{3.276219in}{2.417667in}}{\pgfqpoint{3.265169in}{2.417667in}}%
\pgfpathcurveto{\pgfqpoint{3.254119in}{2.417667in}}{\pgfqpoint{3.243520in}{2.413276in}}{\pgfqpoint{3.235707in}{2.405463in}}%
\pgfpathcurveto{\pgfqpoint{3.227893in}{2.397649in}}{\pgfqpoint{3.223503in}{2.387050in}}{\pgfqpoint{3.223503in}{2.376000in}}%
\pgfpathcurveto{\pgfqpoint{3.223503in}{2.364950in}}{\pgfqpoint{3.227893in}{2.354351in}}{\pgfqpoint{3.235707in}{2.346537in}}%
\pgfpathcurveto{\pgfqpoint{3.243520in}{2.338724in}}{\pgfqpoint{3.254119in}{2.334333in}}{\pgfqpoint{3.265169in}{2.334333in}}%
\pgfpathclose%
\pgfusepath{stroke,fill}%
\end{pgfscope}%
\begin{pgfscope}%
\pgfpathrectangle{\pgfqpoint{0.800000in}{0.528000in}}{\pgfqpoint{4.960000in}{3.696000in}}%
\pgfusepath{clip}%
\pgfsetbuttcap%
\pgfsetroundjoin%
\definecolor{currentfill}{rgb}{0.000000,0.000000,0.000000}%
\pgfsetfillcolor{currentfill}%
\pgfsetlinewidth{1.003750pt}%
\definecolor{currentstroke}{rgb}{0.000000,0.000000,0.000000}%
\pgfsetstrokecolor{currentstroke}%
\pgfsetdash{}{0pt}%
\pgfpathmoveto{\pgfqpoint{3.265169in}{2.334333in}}%
\pgfpathcurveto{\pgfqpoint{3.276219in}{2.334333in}}{\pgfqpoint{3.286818in}{2.338724in}}{\pgfqpoint{3.294632in}{2.346537in}}%
\pgfpathcurveto{\pgfqpoint{3.302446in}{2.354351in}}{\pgfqpoint{3.306836in}{2.364950in}}{\pgfqpoint{3.306836in}{2.376000in}}%
\pgfpathcurveto{\pgfqpoint{3.306836in}{2.387050in}}{\pgfqpoint{3.302446in}{2.397649in}}{\pgfqpoint{3.294632in}{2.405463in}}%
\pgfpathcurveto{\pgfqpoint{3.286818in}{2.413276in}}{\pgfqpoint{3.276219in}{2.417667in}}{\pgfqpoint{3.265169in}{2.417667in}}%
\pgfpathcurveto{\pgfqpoint{3.254119in}{2.417667in}}{\pgfqpoint{3.243520in}{2.413276in}}{\pgfqpoint{3.235707in}{2.405463in}}%
\pgfpathcurveto{\pgfqpoint{3.227893in}{2.397649in}}{\pgfqpoint{3.223503in}{2.387050in}}{\pgfqpoint{3.223503in}{2.376000in}}%
\pgfpathcurveto{\pgfqpoint{3.223503in}{2.364950in}}{\pgfqpoint{3.227893in}{2.354351in}}{\pgfqpoint{3.235707in}{2.346537in}}%
\pgfpathcurveto{\pgfqpoint{3.243520in}{2.338724in}}{\pgfqpoint{3.254119in}{2.334333in}}{\pgfqpoint{3.265169in}{2.334333in}}%
\pgfpathclose%
\pgfusepath{stroke,fill}%
\end{pgfscope}%
\begin{pgfscope}%
\pgfpathrectangle{\pgfqpoint{0.800000in}{0.528000in}}{\pgfqpoint{4.960000in}{3.696000in}}%
\pgfusepath{clip}%
\pgfsetbuttcap%
\pgfsetroundjoin%
\definecolor{currentfill}{rgb}{0.000000,0.000000,0.000000}%
\pgfsetfillcolor{currentfill}%
\pgfsetlinewidth{1.003750pt}%
\definecolor{currentstroke}{rgb}{0.000000,0.000000,0.000000}%
\pgfsetstrokecolor{currentstroke}%
\pgfsetdash{}{0pt}%
\pgfpathmoveto{\pgfqpoint{3.265169in}{2.334333in}}%
\pgfpathcurveto{\pgfqpoint{3.276219in}{2.334333in}}{\pgfqpoint{3.286818in}{2.338724in}}{\pgfqpoint{3.294632in}{2.346537in}}%
\pgfpathcurveto{\pgfqpoint{3.302446in}{2.354351in}}{\pgfqpoint{3.306836in}{2.364950in}}{\pgfqpoint{3.306836in}{2.376000in}}%
\pgfpathcurveto{\pgfqpoint{3.306836in}{2.387050in}}{\pgfqpoint{3.302446in}{2.397649in}}{\pgfqpoint{3.294632in}{2.405463in}}%
\pgfpathcurveto{\pgfqpoint{3.286818in}{2.413276in}}{\pgfqpoint{3.276219in}{2.417667in}}{\pgfqpoint{3.265169in}{2.417667in}}%
\pgfpathcurveto{\pgfqpoint{3.254119in}{2.417667in}}{\pgfqpoint{3.243520in}{2.413276in}}{\pgfqpoint{3.235707in}{2.405463in}}%
\pgfpathcurveto{\pgfqpoint{3.227893in}{2.397649in}}{\pgfqpoint{3.223503in}{2.387050in}}{\pgfqpoint{3.223503in}{2.376000in}}%
\pgfpathcurveto{\pgfqpoint{3.223503in}{2.364950in}}{\pgfqpoint{3.227893in}{2.354351in}}{\pgfqpoint{3.235707in}{2.346537in}}%
\pgfpathcurveto{\pgfqpoint{3.243520in}{2.338724in}}{\pgfqpoint{3.254119in}{2.334333in}}{\pgfqpoint{3.265169in}{2.334333in}}%
\pgfpathclose%
\pgfusepath{stroke,fill}%
\end{pgfscope}%
\begin{pgfscope}%
\pgfpathrectangle{\pgfqpoint{0.800000in}{0.528000in}}{\pgfqpoint{4.960000in}{3.696000in}}%
\pgfusepath{clip}%
\pgfsetbuttcap%
\pgfsetroundjoin%
\definecolor{currentfill}{rgb}{0.000000,0.000000,0.000000}%
\pgfsetfillcolor{currentfill}%
\pgfsetlinewidth{1.003750pt}%
\definecolor{currentstroke}{rgb}{0.000000,0.000000,0.000000}%
\pgfsetstrokecolor{currentstroke}%
\pgfsetdash{}{0pt}%
\pgfpathmoveto{\pgfqpoint{3.265169in}{2.334333in}}%
\pgfpathcurveto{\pgfqpoint{3.276219in}{2.334333in}}{\pgfqpoint{3.286818in}{2.338724in}}{\pgfqpoint{3.294632in}{2.346537in}}%
\pgfpathcurveto{\pgfqpoint{3.302446in}{2.354351in}}{\pgfqpoint{3.306836in}{2.364950in}}{\pgfqpoint{3.306836in}{2.376000in}}%
\pgfpathcurveto{\pgfqpoint{3.306836in}{2.387050in}}{\pgfqpoint{3.302446in}{2.397649in}}{\pgfqpoint{3.294632in}{2.405463in}}%
\pgfpathcurveto{\pgfqpoint{3.286818in}{2.413276in}}{\pgfqpoint{3.276219in}{2.417667in}}{\pgfqpoint{3.265169in}{2.417667in}}%
\pgfpathcurveto{\pgfqpoint{3.254119in}{2.417667in}}{\pgfqpoint{3.243520in}{2.413276in}}{\pgfqpoint{3.235707in}{2.405463in}}%
\pgfpathcurveto{\pgfqpoint{3.227893in}{2.397649in}}{\pgfqpoint{3.223503in}{2.387050in}}{\pgfqpoint{3.223503in}{2.376000in}}%
\pgfpathcurveto{\pgfqpoint{3.223503in}{2.364950in}}{\pgfqpoint{3.227893in}{2.354351in}}{\pgfqpoint{3.235707in}{2.346537in}}%
\pgfpathcurveto{\pgfqpoint{3.243520in}{2.338724in}}{\pgfqpoint{3.254119in}{2.334333in}}{\pgfqpoint{3.265169in}{2.334333in}}%
\pgfpathclose%
\pgfusepath{stroke,fill}%
\end{pgfscope}%
\begin{pgfscope}%
\pgfpathrectangle{\pgfqpoint{0.800000in}{0.528000in}}{\pgfqpoint{4.960000in}{3.696000in}}%
\pgfusepath{clip}%
\pgfsetbuttcap%
\pgfsetroundjoin%
\definecolor{currentfill}{rgb}{0.000000,0.000000,0.000000}%
\pgfsetfillcolor{currentfill}%
\pgfsetlinewidth{1.003750pt}%
\definecolor{currentstroke}{rgb}{0.000000,0.000000,0.000000}%
\pgfsetstrokecolor{currentstroke}%
\pgfsetdash{}{0pt}%
\pgfpathmoveto{\pgfqpoint{3.265169in}{2.334333in}}%
\pgfpathcurveto{\pgfqpoint{3.276219in}{2.334333in}}{\pgfqpoint{3.286818in}{2.338724in}}{\pgfqpoint{3.294632in}{2.346537in}}%
\pgfpathcurveto{\pgfqpoint{3.302446in}{2.354351in}}{\pgfqpoint{3.306836in}{2.364950in}}{\pgfqpoint{3.306836in}{2.376000in}}%
\pgfpathcurveto{\pgfqpoint{3.306836in}{2.387050in}}{\pgfqpoint{3.302446in}{2.397649in}}{\pgfqpoint{3.294632in}{2.405463in}}%
\pgfpathcurveto{\pgfqpoint{3.286818in}{2.413276in}}{\pgfqpoint{3.276219in}{2.417667in}}{\pgfqpoint{3.265169in}{2.417667in}}%
\pgfpathcurveto{\pgfqpoint{3.254119in}{2.417667in}}{\pgfqpoint{3.243520in}{2.413276in}}{\pgfqpoint{3.235707in}{2.405463in}}%
\pgfpathcurveto{\pgfqpoint{3.227893in}{2.397649in}}{\pgfqpoint{3.223503in}{2.387050in}}{\pgfqpoint{3.223503in}{2.376000in}}%
\pgfpathcurveto{\pgfqpoint{3.223503in}{2.364950in}}{\pgfqpoint{3.227893in}{2.354351in}}{\pgfqpoint{3.235707in}{2.346537in}}%
\pgfpathcurveto{\pgfqpoint{3.243520in}{2.338724in}}{\pgfqpoint{3.254119in}{2.334333in}}{\pgfqpoint{3.265169in}{2.334333in}}%
\pgfpathclose%
\pgfusepath{stroke,fill}%
\end{pgfscope}%
\begin{pgfscope}%
\pgfpathrectangle{\pgfqpoint{0.800000in}{0.528000in}}{\pgfqpoint{4.960000in}{3.696000in}}%
\pgfusepath{clip}%
\pgfsetbuttcap%
\pgfsetroundjoin%
\definecolor{currentfill}{rgb}{0.000000,0.000000,0.000000}%
\pgfsetfillcolor{currentfill}%
\pgfsetlinewidth{1.003750pt}%
\definecolor{currentstroke}{rgb}{0.000000,0.000000,0.000000}%
\pgfsetstrokecolor{currentstroke}%
\pgfsetdash{}{0pt}%
\pgfpathmoveto{\pgfqpoint{3.265169in}{2.334333in}}%
\pgfpathcurveto{\pgfqpoint{3.276219in}{2.334333in}}{\pgfqpoint{3.286818in}{2.338724in}}{\pgfqpoint{3.294632in}{2.346537in}}%
\pgfpathcurveto{\pgfqpoint{3.302446in}{2.354351in}}{\pgfqpoint{3.306836in}{2.364950in}}{\pgfqpoint{3.306836in}{2.376000in}}%
\pgfpathcurveto{\pgfqpoint{3.306836in}{2.387050in}}{\pgfqpoint{3.302446in}{2.397649in}}{\pgfqpoint{3.294632in}{2.405463in}}%
\pgfpathcurveto{\pgfqpoint{3.286818in}{2.413276in}}{\pgfqpoint{3.276219in}{2.417667in}}{\pgfqpoint{3.265169in}{2.417667in}}%
\pgfpathcurveto{\pgfqpoint{3.254119in}{2.417667in}}{\pgfqpoint{3.243520in}{2.413276in}}{\pgfqpoint{3.235707in}{2.405463in}}%
\pgfpathcurveto{\pgfqpoint{3.227893in}{2.397649in}}{\pgfqpoint{3.223503in}{2.387050in}}{\pgfqpoint{3.223503in}{2.376000in}}%
\pgfpathcurveto{\pgfqpoint{3.223503in}{2.364950in}}{\pgfqpoint{3.227893in}{2.354351in}}{\pgfqpoint{3.235707in}{2.346537in}}%
\pgfpathcurveto{\pgfqpoint{3.243520in}{2.338724in}}{\pgfqpoint{3.254119in}{2.334333in}}{\pgfqpoint{3.265169in}{2.334333in}}%
\pgfpathclose%
\pgfusepath{stroke,fill}%
\end{pgfscope}%
\begin{pgfscope}%
\pgfpathrectangle{\pgfqpoint{0.800000in}{0.528000in}}{\pgfqpoint{4.960000in}{3.696000in}}%
\pgfusepath{clip}%
\pgfsetbuttcap%
\pgfsetroundjoin%
\definecolor{currentfill}{rgb}{0.000000,0.000000,0.000000}%
\pgfsetfillcolor{currentfill}%
\pgfsetlinewidth{1.003750pt}%
\definecolor{currentstroke}{rgb}{0.000000,0.000000,0.000000}%
\pgfsetstrokecolor{currentstroke}%
\pgfsetdash{}{0pt}%
\pgfpathmoveto{\pgfqpoint{3.265169in}{2.334333in}}%
\pgfpathcurveto{\pgfqpoint{3.276219in}{2.334333in}}{\pgfqpoint{3.286818in}{2.338724in}}{\pgfqpoint{3.294632in}{2.346537in}}%
\pgfpathcurveto{\pgfqpoint{3.302446in}{2.354351in}}{\pgfqpoint{3.306836in}{2.364950in}}{\pgfqpoint{3.306836in}{2.376000in}}%
\pgfpathcurveto{\pgfqpoint{3.306836in}{2.387050in}}{\pgfqpoint{3.302446in}{2.397649in}}{\pgfqpoint{3.294632in}{2.405463in}}%
\pgfpathcurveto{\pgfqpoint{3.286818in}{2.413276in}}{\pgfqpoint{3.276219in}{2.417667in}}{\pgfqpoint{3.265169in}{2.417667in}}%
\pgfpathcurveto{\pgfqpoint{3.254119in}{2.417667in}}{\pgfqpoint{3.243520in}{2.413276in}}{\pgfqpoint{3.235707in}{2.405463in}}%
\pgfpathcurveto{\pgfqpoint{3.227893in}{2.397649in}}{\pgfqpoint{3.223503in}{2.387050in}}{\pgfqpoint{3.223503in}{2.376000in}}%
\pgfpathcurveto{\pgfqpoint{3.223503in}{2.364950in}}{\pgfqpoint{3.227893in}{2.354351in}}{\pgfqpoint{3.235707in}{2.346537in}}%
\pgfpathcurveto{\pgfqpoint{3.243520in}{2.338724in}}{\pgfqpoint{3.254119in}{2.334333in}}{\pgfqpoint{3.265169in}{2.334333in}}%
\pgfpathclose%
\pgfusepath{stroke,fill}%
\end{pgfscope}%
\begin{pgfscope}%
\pgfpathrectangle{\pgfqpoint{0.800000in}{0.528000in}}{\pgfqpoint{4.960000in}{3.696000in}}%
\pgfusepath{clip}%
\pgfsetbuttcap%
\pgfsetroundjoin%
\definecolor{currentfill}{rgb}{0.000000,0.000000,0.000000}%
\pgfsetfillcolor{currentfill}%
\pgfsetlinewidth{1.003750pt}%
\definecolor{currentstroke}{rgb}{0.000000,0.000000,0.000000}%
\pgfsetstrokecolor{currentstroke}%
\pgfsetdash{}{0pt}%
\pgfpathmoveto{\pgfqpoint{3.265169in}{2.334333in}}%
\pgfpathcurveto{\pgfqpoint{3.276219in}{2.334333in}}{\pgfqpoint{3.286818in}{2.338724in}}{\pgfqpoint{3.294632in}{2.346537in}}%
\pgfpathcurveto{\pgfqpoint{3.302446in}{2.354351in}}{\pgfqpoint{3.306836in}{2.364950in}}{\pgfqpoint{3.306836in}{2.376000in}}%
\pgfpathcurveto{\pgfqpoint{3.306836in}{2.387050in}}{\pgfqpoint{3.302446in}{2.397649in}}{\pgfqpoint{3.294632in}{2.405463in}}%
\pgfpathcurveto{\pgfqpoint{3.286818in}{2.413276in}}{\pgfqpoint{3.276219in}{2.417667in}}{\pgfqpoint{3.265169in}{2.417667in}}%
\pgfpathcurveto{\pgfqpoint{3.254119in}{2.417667in}}{\pgfqpoint{3.243520in}{2.413276in}}{\pgfqpoint{3.235707in}{2.405463in}}%
\pgfpathcurveto{\pgfqpoint{3.227893in}{2.397649in}}{\pgfqpoint{3.223503in}{2.387050in}}{\pgfqpoint{3.223503in}{2.376000in}}%
\pgfpathcurveto{\pgfqpoint{3.223503in}{2.364950in}}{\pgfqpoint{3.227893in}{2.354351in}}{\pgfqpoint{3.235707in}{2.346537in}}%
\pgfpathcurveto{\pgfqpoint{3.243520in}{2.338724in}}{\pgfqpoint{3.254119in}{2.334333in}}{\pgfqpoint{3.265169in}{2.334333in}}%
\pgfpathclose%
\pgfusepath{stroke,fill}%
\end{pgfscope}%
\begin{pgfscope}%
\pgfpathrectangle{\pgfqpoint{0.800000in}{0.528000in}}{\pgfqpoint{4.960000in}{3.696000in}}%
\pgfusepath{clip}%
\pgfsetbuttcap%
\pgfsetroundjoin%
\definecolor{currentfill}{rgb}{0.000000,0.000000,0.000000}%
\pgfsetfillcolor{currentfill}%
\pgfsetlinewidth{1.003750pt}%
\definecolor{currentstroke}{rgb}{0.000000,0.000000,0.000000}%
\pgfsetstrokecolor{currentstroke}%
\pgfsetdash{}{0pt}%
\pgfpathmoveto{\pgfqpoint{3.265169in}{2.334333in}}%
\pgfpathcurveto{\pgfqpoint{3.276219in}{2.334333in}}{\pgfqpoint{3.286818in}{2.338724in}}{\pgfqpoint{3.294632in}{2.346537in}}%
\pgfpathcurveto{\pgfqpoint{3.302446in}{2.354351in}}{\pgfqpoint{3.306836in}{2.364950in}}{\pgfqpoint{3.306836in}{2.376000in}}%
\pgfpathcurveto{\pgfqpoint{3.306836in}{2.387050in}}{\pgfqpoint{3.302446in}{2.397649in}}{\pgfqpoint{3.294632in}{2.405463in}}%
\pgfpathcurveto{\pgfqpoint{3.286818in}{2.413276in}}{\pgfqpoint{3.276219in}{2.417667in}}{\pgfqpoint{3.265169in}{2.417667in}}%
\pgfpathcurveto{\pgfqpoint{3.254119in}{2.417667in}}{\pgfqpoint{3.243520in}{2.413276in}}{\pgfqpoint{3.235707in}{2.405463in}}%
\pgfpathcurveto{\pgfqpoint{3.227893in}{2.397649in}}{\pgfqpoint{3.223503in}{2.387050in}}{\pgfqpoint{3.223503in}{2.376000in}}%
\pgfpathcurveto{\pgfqpoint{3.223503in}{2.364950in}}{\pgfqpoint{3.227893in}{2.354351in}}{\pgfqpoint{3.235707in}{2.346537in}}%
\pgfpathcurveto{\pgfqpoint{3.243520in}{2.338724in}}{\pgfqpoint{3.254119in}{2.334333in}}{\pgfqpoint{3.265169in}{2.334333in}}%
\pgfpathclose%
\pgfusepath{stroke,fill}%
\end{pgfscope}%
\begin{pgfscope}%
\pgfpathrectangle{\pgfqpoint{0.800000in}{0.528000in}}{\pgfqpoint{4.960000in}{3.696000in}}%
\pgfusepath{clip}%
\pgfsetbuttcap%
\pgfsetroundjoin%
\definecolor{currentfill}{rgb}{0.000000,0.000000,0.000000}%
\pgfsetfillcolor{currentfill}%
\pgfsetlinewidth{1.003750pt}%
\definecolor{currentstroke}{rgb}{0.000000,0.000000,0.000000}%
\pgfsetstrokecolor{currentstroke}%
\pgfsetdash{}{0pt}%
\pgfpathmoveto{\pgfqpoint{3.265169in}{2.334333in}}%
\pgfpathcurveto{\pgfqpoint{3.276219in}{2.334333in}}{\pgfqpoint{3.286818in}{2.338724in}}{\pgfqpoint{3.294632in}{2.346537in}}%
\pgfpathcurveto{\pgfqpoint{3.302446in}{2.354351in}}{\pgfqpoint{3.306836in}{2.364950in}}{\pgfqpoint{3.306836in}{2.376000in}}%
\pgfpathcurveto{\pgfqpoint{3.306836in}{2.387050in}}{\pgfqpoint{3.302446in}{2.397649in}}{\pgfqpoint{3.294632in}{2.405463in}}%
\pgfpathcurveto{\pgfqpoint{3.286818in}{2.413276in}}{\pgfqpoint{3.276219in}{2.417667in}}{\pgfqpoint{3.265169in}{2.417667in}}%
\pgfpathcurveto{\pgfqpoint{3.254119in}{2.417667in}}{\pgfqpoint{3.243520in}{2.413276in}}{\pgfqpoint{3.235707in}{2.405463in}}%
\pgfpathcurveto{\pgfqpoint{3.227893in}{2.397649in}}{\pgfqpoint{3.223503in}{2.387050in}}{\pgfqpoint{3.223503in}{2.376000in}}%
\pgfpathcurveto{\pgfqpoint{3.223503in}{2.364950in}}{\pgfqpoint{3.227893in}{2.354351in}}{\pgfqpoint{3.235707in}{2.346537in}}%
\pgfpathcurveto{\pgfqpoint{3.243520in}{2.338724in}}{\pgfqpoint{3.254119in}{2.334333in}}{\pgfqpoint{3.265169in}{2.334333in}}%
\pgfpathclose%
\pgfusepath{stroke,fill}%
\end{pgfscope}%
\begin{pgfscope}%
\pgfpathrectangle{\pgfqpoint{0.800000in}{0.528000in}}{\pgfqpoint{4.960000in}{3.696000in}}%
\pgfusepath{clip}%
\pgfsetbuttcap%
\pgfsetroundjoin%
\definecolor{currentfill}{rgb}{0.000000,0.000000,0.000000}%
\pgfsetfillcolor{currentfill}%
\pgfsetlinewidth{1.003750pt}%
\definecolor{currentstroke}{rgb}{0.000000,0.000000,0.000000}%
\pgfsetstrokecolor{currentstroke}%
\pgfsetdash{}{0pt}%
\pgfpathmoveto{\pgfqpoint{3.265169in}{2.334333in}}%
\pgfpathcurveto{\pgfqpoint{3.276219in}{2.334333in}}{\pgfqpoint{3.286818in}{2.338724in}}{\pgfqpoint{3.294632in}{2.346537in}}%
\pgfpathcurveto{\pgfqpoint{3.302446in}{2.354351in}}{\pgfqpoint{3.306836in}{2.364950in}}{\pgfqpoint{3.306836in}{2.376000in}}%
\pgfpathcurveto{\pgfqpoint{3.306836in}{2.387050in}}{\pgfqpoint{3.302446in}{2.397649in}}{\pgfqpoint{3.294632in}{2.405463in}}%
\pgfpathcurveto{\pgfqpoint{3.286818in}{2.413276in}}{\pgfqpoint{3.276219in}{2.417667in}}{\pgfqpoint{3.265169in}{2.417667in}}%
\pgfpathcurveto{\pgfqpoint{3.254119in}{2.417667in}}{\pgfqpoint{3.243520in}{2.413276in}}{\pgfqpoint{3.235707in}{2.405463in}}%
\pgfpathcurveto{\pgfqpoint{3.227893in}{2.397649in}}{\pgfqpoint{3.223503in}{2.387050in}}{\pgfqpoint{3.223503in}{2.376000in}}%
\pgfpathcurveto{\pgfqpoint{3.223503in}{2.364950in}}{\pgfqpoint{3.227893in}{2.354351in}}{\pgfqpoint{3.235707in}{2.346537in}}%
\pgfpathcurveto{\pgfqpoint{3.243520in}{2.338724in}}{\pgfqpoint{3.254119in}{2.334333in}}{\pgfqpoint{3.265169in}{2.334333in}}%
\pgfpathclose%
\pgfusepath{stroke,fill}%
\end{pgfscope}%
\begin{pgfscope}%
\pgfpathrectangle{\pgfqpoint{0.800000in}{0.528000in}}{\pgfqpoint{4.960000in}{3.696000in}}%
\pgfusepath{clip}%
\pgfsetbuttcap%
\pgfsetroundjoin%
\definecolor{currentfill}{rgb}{0.000000,0.000000,0.000000}%
\pgfsetfillcolor{currentfill}%
\pgfsetlinewidth{1.003750pt}%
\definecolor{currentstroke}{rgb}{0.000000,0.000000,0.000000}%
\pgfsetstrokecolor{currentstroke}%
\pgfsetdash{}{0pt}%
\pgfpathmoveto{\pgfqpoint{3.265169in}{2.334333in}}%
\pgfpathcurveto{\pgfqpoint{3.276219in}{2.334333in}}{\pgfqpoint{3.286818in}{2.338724in}}{\pgfqpoint{3.294632in}{2.346537in}}%
\pgfpathcurveto{\pgfqpoint{3.302446in}{2.354351in}}{\pgfqpoint{3.306836in}{2.364950in}}{\pgfqpoint{3.306836in}{2.376000in}}%
\pgfpathcurveto{\pgfqpoint{3.306836in}{2.387050in}}{\pgfqpoint{3.302446in}{2.397649in}}{\pgfqpoint{3.294632in}{2.405463in}}%
\pgfpathcurveto{\pgfqpoint{3.286818in}{2.413276in}}{\pgfqpoint{3.276219in}{2.417667in}}{\pgfqpoint{3.265169in}{2.417667in}}%
\pgfpathcurveto{\pgfqpoint{3.254119in}{2.417667in}}{\pgfqpoint{3.243520in}{2.413276in}}{\pgfqpoint{3.235707in}{2.405463in}}%
\pgfpathcurveto{\pgfqpoint{3.227893in}{2.397649in}}{\pgfqpoint{3.223503in}{2.387050in}}{\pgfqpoint{3.223503in}{2.376000in}}%
\pgfpathcurveto{\pgfqpoint{3.223503in}{2.364950in}}{\pgfqpoint{3.227893in}{2.354351in}}{\pgfqpoint{3.235707in}{2.346537in}}%
\pgfpathcurveto{\pgfqpoint{3.243520in}{2.338724in}}{\pgfqpoint{3.254119in}{2.334333in}}{\pgfqpoint{3.265169in}{2.334333in}}%
\pgfpathclose%
\pgfusepath{stroke,fill}%
\end{pgfscope}%
\begin{pgfscope}%
\pgfpathrectangle{\pgfqpoint{0.800000in}{0.528000in}}{\pgfqpoint{4.960000in}{3.696000in}}%
\pgfusepath{clip}%
\pgfsetbuttcap%
\pgfsetroundjoin%
\definecolor{currentfill}{rgb}{0.000000,0.000000,0.000000}%
\pgfsetfillcolor{currentfill}%
\pgfsetlinewidth{1.003750pt}%
\definecolor{currentstroke}{rgb}{0.000000,0.000000,0.000000}%
\pgfsetstrokecolor{currentstroke}%
\pgfsetdash{}{0pt}%
\pgfpathmoveto{\pgfqpoint{3.265169in}{2.334333in}}%
\pgfpathcurveto{\pgfqpoint{3.276219in}{2.334333in}}{\pgfqpoint{3.286818in}{2.338724in}}{\pgfqpoint{3.294632in}{2.346537in}}%
\pgfpathcurveto{\pgfqpoint{3.302446in}{2.354351in}}{\pgfqpoint{3.306836in}{2.364950in}}{\pgfqpoint{3.306836in}{2.376000in}}%
\pgfpathcurveto{\pgfqpoint{3.306836in}{2.387050in}}{\pgfqpoint{3.302446in}{2.397649in}}{\pgfqpoint{3.294632in}{2.405463in}}%
\pgfpathcurveto{\pgfqpoint{3.286818in}{2.413276in}}{\pgfqpoint{3.276219in}{2.417667in}}{\pgfqpoint{3.265169in}{2.417667in}}%
\pgfpathcurveto{\pgfqpoint{3.254119in}{2.417667in}}{\pgfqpoint{3.243520in}{2.413276in}}{\pgfqpoint{3.235707in}{2.405463in}}%
\pgfpathcurveto{\pgfqpoint{3.227893in}{2.397649in}}{\pgfqpoint{3.223503in}{2.387050in}}{\pgfqpoint{3.223503in}{2.376000in}}%
\pgfpathcurveto{\pgfqpoint{3.223503in}{2.364950in}}{\pgfqpoint{3.227893in}{2.354351in}}{\pgfqpoint{3.235707in}{2.346537in}}%
\pgfpathcurveto{\pgfqpoint{3.243520in}{2.338724in}}{\pgfqpoint{3.254119in}{2.334333in}}{\pgfqpoint{3.265169in}{2.334333in}}%
\pgfpathclose%
\pgfusepath{stroke,fill}%
\end{pgfscope}%
\begin{pgfscope}%
\pgfpathrectangle{\pgfqpoint{0.800000in}{0.528000in}}{\pgfqpoint{4.960000in}{3.696000in}}%
\pgfusepath{clip}%
\pgfsetbuttcap%
\pgfsetroundjoin%
\definecolor{currentfill}{rgb}{0.000000,0.000000,0.000000}%
\pgfsetfillcolor{currentfill}%
\pgfsetlinewidth{1.003750pt}%
\definecolor{currentstroke}{rgb}{0.000000,0.000000,0.000000}%
\pgfsetstrokecolor{currentstroke}%
\pgfsetdash{}{0pt}%
\pgfpathmoveto{\pgfqpoint{3.265169in}{2.334333in}}%
\pgfpathcurveto{\pgfqpoint{3.276219in}{2.334333in}}{\pgfqpoint{3.286818in}{2.338724in}}{\pgfqpoint{3.294632in}{2.346537in}}%
\pgfpathcurveto{\pgfqpoint{3.302446in}{2.354351in}}{\pgfqpoint{3.306836in}{2.364950in}}{\pgfqpoint{3.306836in}{2.376000in}}%
\pgfpathcurveto{\pgfqpoint{3.306836in}{2.387050in}}{\pgfqpoint{3.302446in}{2.397649in}}{\pgfqpoint{3.294632in}{2.405463in}}%
\pgfpathcurveto{\pgfqpoint{3.286818in}{2.413276in}}{\pgfqpoint{3.276219in}{2.417667in}}{\pgfqpoint{3.265169in}{2.417667in}}%
\pgfpathcurveto{\pgfqpoint{3.254119in}{2.417667in}}{\pgfqpoint{3.243520in}{2.413276in}}{\pgfqpoint{3.235707in}{2.405463in}}%
\pgfpathcurveto{\pgfqpoint{3.227893in}{2.397649in}}{\pgfqpoint{3.223503in}{2.387050in}}{\pgfqpoint{3.223503in}{2.376000in}}%
\pgfpathcurveto{\pgfqpoint{3.223503in}{2.364950in}}{\pgfqpoint{3.227893in}{2.354351in}}{\pgfqpoint{3.235707in}{2.346537in}}%
\pgfpathcurveto{\pgfqpoint{3.243520in}{2.338724in}}{\pgfqpoint{3.254119in}{2.334333in}}{\pgfqpoint{3.265169in}{2.334333in}}%
\pgfpathclose%
\pgfusepath{stroke,fill}%
\end{pgfscope}%
\begin{pgfscope}%
\pgfpathrectangle{\pgfqpoint{0.800000in}{0.528000in}}{\pgfqpoint{4.960000in}{3.696000in}}%
\pgfusepath{clip}%
\pgfsetbuttcap%
\pgfsetroundjoin%
\definecolor{currentfill}{rgb}{0.000000,0.000000,0.000000}%
\pgfsetfillcolor{currentfill}%
\pgfsetlinewidth{1.003750pt}%
\definecolor{currentstroke}{rgb}{0.000000,0.000000,0.000000}%
\pgfsetstrokecolor{currentstroke}%
\pgfsetdash{}{0pt}%
\pgfpathmoveto{\pgfqpoint{3.265169in}{2.334333in}}%
\pgfpathcurveto{\pgfqpoint{3.276219in}{2.334333in}}{\pgfqpoint{3.286818in}{2.338724in}}{\pgfqpoint{3.294632in}{2.346537in}}%
\pgfpathcurveto{\pgfqpoint{3.302446in}{2.354351in}}{\pgfqpoint{3.306836in}{2.364950in}}{\pgfqpoint{3.306836in}{2.376000in}}%
\pgfpathcurveto{\pgfqpoint{3.306836in}{2.387050in}}{\pgfqpoint{3.302446in}{2.397649in}}{\pgfqpoint{3.294632in}{2.405463in}}%
\pgfpathcurveto{\pgfqpoint{3.286818in}{2.413276in}}{\pgfqpoint{3.276219in}{2.417667in}}{\pgfqpoint{3.265169in}{2.417667in}}%
\pgfpathcurveto{\pgfqpoint{3.254119in}{2.417667in}}{\pgfqpoint{3.243520in}{2.413276in}}{\pgfqpoint{3.235707in}{2.405463in}}%
\pgfpathcurveto{\pgfqpoint{3.227893in}{2.397649in}}{\pgfqpoint{3.223503in}{2.387050in}}{\pgfqpoint{3.223503in}{2.376000in}}%
\pgfpathcurveto{\pgfqpoint{3.223503in}{2.364950in}}{\pgfqpoint{3.227893in}{2.354351in}}{\pgfqpoint{3.235707in}{2.346537in}}%
\pgfpathcurveto{\pgfqpoint{3.243520in}{2.338724in}}{\pgfqpoint{3.254119in}{2.334333in}}{\pgfqpoint{3.265169in}{2.334333in}}%
\pgfpathclose%
\pgfusepath{stroke,fill}%
\end{pgfscope}%
\begin{pgfscope}%
\pgfpathrectangle{\pgfqpoint{0.800000in}{0.528000in}}{\pgfqpoint{4.960000in}{3.696000in}}%
\pgfusepath{clip}%
\pgfsetbuttcap%
\pgfsetroundjoin%
\definecolor{currentfill}{rgb}{0.000000,0.000000,0.000000}%
\pgfsetfillcolor{currentfill}%
\pgfsetlinewidth{1.003750pt}%
\definecolor{currentstroke}{rgb}{0.000000,0.000000,0.000000}%
\pgfsetstrokecolor{currentstroke}%
\pgfsetdash{}{0pt}%
\pgfpathmoveto{\pgfqpoint{3.265169in}{2.334333in}}%
\pgfpathcurveto{\pgfqpoint{3.276219in}{2.334333in}}{\pgfqpoint{3.286818in}{2.338724in}}{\pgfqpoint{3.294632in}{2.346537in}}%
\pgfpathcurveto{\pgfqpoint{3.302446in}{2.354351in}}{\pgfqpoint{3.306836in}{2.364950in}}{\pgfqpoint{3.306836in}{2.376000in}}%
\pgfpathcurveto{\pgfqpoint{3.306836in}{2.387050in}}{\pgfqpoint{3.302446in}{2.397649in}}{\pgfqpoint{3.294632in}{2.405463in}}%
\pgfpathcurveto{\pgfqpoint{3.286818in}{2.413276in}}{\pgfqpoint{3.276219in}{2.417667in}}{\pgfqpoint{3.265169in}{2.417667in}}%
\pgfpathcurveto{\pgfqpoint{3.254119in}{2.417667in}}{\pgfqpoint{3.243520in}{2.413276in}}{\pgfqpoint{3.235707in}{2.405463in}}%
\pgfpathcurveto{\pgfqpoint{3.227893in}{2.397649in}}{\pgfqpoint{3.223503in}{2.387050in}}{\pgfqpoint{3.223503in}{2.376000in}}%
\pgfpathcurveto{\pgfqpoint{3.223503in}{2.364950in}}{\pgfqpoint{3.227893in}{2.354351in}}{\pgfqpoint{3.235707in}{2.346537in}}%
\pgfpathcurveto{\pgfqpoint{3.243520in}{2.338724in}}{\pgfqpoint{3.254119in}{2.334333in}}{\pgfqpoint{3.265169in}{2.334333in}}%
\pgfpathclose%
\pgfusepath{stroke,fill}%
\end{pgfscope}%
\begin{pgfscope}%
\pgfpathrectangle{\pgfqpoint{0.800000in}{0.528000in}}{\pgfqpoint{4.960000in}{3.696000in}}%
\pgfusepath{clip}%
\pgfsetbuttcap%
\pgfsetroundjoin%
\definecolor{currentfill}{rgb}{0.000000,0.000000,0.000000}%
\pgfsetfillcolor{currentfill}%
\pgfsetlinewidth{1.003750pt}%
\definecolor{currentstroke}{rgb}{0.000000,0.000000,0.000000}%
\pgfsetstrokecolor{currentstroke}%
\pgfsetdash{}{0pt}%
\pgfpathmoveto{\pgfqpoint{3.265169in}{2.334333in}}%
\pgfpathcurveto{\pgfqpoint{3.276219in}{2.334333in}}{\pgfqpoint{3.286818in}{2.338724in}}{\pgfqpoint{3.294632in}{2.346537in}}%
\pgfpathcurveto{\pgfqpoint{3.302446in}{2.354351in}}{\pgfqpoint{3.306836in}{2.364950in}}{\pgfqpoint{3.306836in}{2.376000in}}%
\pgfpathcurveto{\pgfqpoint{3.306836in}{2.387050in}}{\pgfqpoint{3.302446in}{2.397649in}}{\pgfqpoint{3.294632in}{2.405463in}}%
\pgfpathcurveto{\pgfqpoint{3.286818in}{2.413276in}}{\pgfqpoint{3.276219in}{2.417667in}}{\pgfqpoint{3.265169in}{2.417667in}}%
\pgfpathcurveto{\pgfqpoint{3.254119in}{2.417667in}}{\pgfqpoint{3.243520in}{2.413276in}}{\pgfqpoint{3.235707in}{2.405463in}}%
\pgfpathcurveto{\pgfqpoint{3.227893in}{2.397649in}}{\pgfqpoint{3.223503in}{2.387050in}}{\pgfqpoint{3.223503in}{2.376000in}}%
\pgfpathcurveto{\pgfqpoint{3.223503in}{2.364950in}}{\pgfqpoint{3.227893in}{2.354351in}}{\pgfqpoint{3.235707in}{2.346537in}}%
\pgfpathcurveto{\pgfqpoint{3.243520in}{2.338724in}}{\pgfqpoint{3.254119in}{2.334333in}}{\pgfqpoint{3.265169in}{2.334333in}}%
\pgfpathclose%
\pgfusepath{stroke,fill}%
\end{pgfscope}%
\begin{pgfscope}%
\pgfpathrectangle{\pgfqpoint{0.800000in}{0.528000in}}{\pgfqpoint{4.960000in}{3.696000in}}%
\pgfusepath{clip}%
\pgfsetbuttcap%
\pgfsetroundjoin%
\definecolor{currentfill}{rgb}{0.000000,0.000000,0.000000}%
\pgfsetfillcolor{currentfill}%
\pgfsetlinewidth{1.003750pt}%
\definecolor{currentstroke}{rgb}{0.000000,0.000000,0.000000}%
\pgfsetstrokecolor{currentstroke}%
\pgfsetdash{}{0pt}%
\pgfpathmoveto{\pgfqpoint{3.265169in}{2.334333in}}%
\pgfpathcurveto{\pgfqpoint{3.276219in}{2.334333in}}{\pgfqpoint{3.286818in}{2.338724in}}{\pgfqpoint{3.294632in}{2.346537in}}%
\pgfpathcurveto{\pgfqpoint{3.302446in}{2.354351in}}{\pgfqpoint{3.306836in}{2.364950in}}{\pgfqpoint{3.306836in}{2.376000in}}%
\pgfpathcurveto{\pgfqpoint{3.306836in}{2.387050in}}{\pgfqpoint{3.302446in}{2.397649in}}{\pgfqpoint{3.294632in}{2.405463in}}%
\pgfpathcurveto{\pgfqpoint{3.286818in}{2.413276in}}{\pgfqpoint{3.276219in}{2.417667in}}{\pgfqpoint{3.265169in}{2.417667in}}%
\pgfpathcurveto{\pgfqpoint{3.254119in}{2.417667in}}{\pgfqpoint{3.243520in}{2.413276in}}{\pgfqpoint{3.235707in}{2.405463in}}%
\pgfpathcurveto{\pgfqpoint{3.227893in}{2.397649in}}{\pgfqpoint{3.223503in}{2.387050in}}{\pgfqpoint{3.223503in}{2.376000in}}%
\pgfpathcurveto{\pgfqpoint{3.223503in}{2.364950in}}{\pgfqpoint{3.227893in}{2.354351in}}{\pgfqpoint{3.235707in}{2.346537in}}%
\pgfpathcurveto{\pgfqpoint{3.243520in}{2.338724in}}{\pgfqpoint{3.254119in}{2.334333in}}{\pgfqpoint{3.265169in}{2.334333in}}%
\pgfpathclose%
\pgfusepath{stroke,fill}%
\end{pgfscope}%
\begin{pgfscope}%
\pgfpathrectangle{\pgfqpoint{0.800000in}{0.528000in}}{\pgfqpoint{4.960000in}{3.696000in}}%
\pgfusepath{clip}%
\pgfsetbuttcap%
\pgfsetroundjoin%
\definecolor{currentfill}{rgb}{0.000000,0.000000,0.000000}%
\pgfsetfillcolor{currentfill}%
\pgfsetlinewidth{1.003750pt}%
\definecolor{currentstroke}{rgb}{0.000000,0.000000,0.000000}%
\pgfsetstrokecolor{currentstroke}%
\pgfsetdash{}{0pt}%
\pgfpathmoveto{\pgfqpoint{3.265169in}{2.334333in}}%
\pgfpathcurveto{\pgfqpoint{3.276219in}{2.334333in}}{\pgfqpoint{3.286818in}{2.338724in}}{\pgfqpoint{3.294632in}{2.346537in}}%
\pgfpathcurveto{\pgfqpoint{3.302446in}{2.354351in}}{\pgfqpoint{3.306836in}{2.364950in}}{\pgfqpoint{3.306836in}{2.376000in}}%
\pgfpathcurveto{\pgfqpoint{3.306836in}{2.387050in}}{\pgfqpoint{3.302446in}{2.397649in}}{\pgfqpoint{3.294632in}{2.405463in}}%
\pgfpathcurveto{\pgfqpoint{3.286818in}{2.413276in}}{\pgfqpoint{3.276219in}{2.417667in}}{\pgfqpoint{3.265169in}{2.417667in}}%
\pgfpathcurveto{\pgfqpoint{3.254119in}{2.417667in}}{\pgfqpoint{3.243520in}{2.413276in}}{\pgfqpoint{3.235707in}{2.405463in}}%
\pgfpathcurveto{\pgfqpoint{3.227893in}{2.397649in}}{\pgfqpoint{3.223503in}{2.387050in}}{\pgfqpoint{3.223503in}{2.376000in}}%
\pgfpathcurveto{\pgfqpoint{3.223503in}{2.364950in}}{\pgfqpoint{3.227893in}{2.354351in}}{\pgfqpoint{3.235707in}{2.346537in}}%
\pgfpathcurveto{\pgfqpoint{3.243520in}{2.338724in}}{\pgfqpoint{3.254119in}{2.334333in}}{\pgfqpoint{3.265169in}{2.334333in}}%
\pgfpathclose%
\pgfusepath{stroke,fill}%
\end{pgfscope}%
\begin{pgfscope}%
\pgfpathrectangle{\pgfqpoint{0.800000in}{0.528000in}}{\pgfqpoint{4.960000in}{3.696000in}}%
\pgfusepath{clip}%
\pgfsetbuttcap%
\pgfsetroundjoin%
\definecolor{currentfill}{rgb}{0.000000,0.000000,0.000000}%
\pgfsetfillcolor{currentfill}%
\pgfsetlinewidth{1.003750pt}%
\definecolor{currentstroke}{rgb}{0.000000,0.000000,0.000000}%
\pgfsetstrokecolor{currentstroke}%
\pgfsetdash{}{0pt}%
\pgfpathmoveto{\pgfqpoint{3.265169in}{2.334333in}}%
\pgfpathcurveto{\pgfqpoint{3.276219in}{2.334333in}}{\pgfqpoint{3.286818in}{2.338724in}}{\pgfqpoint{3.294632in}{2.346537in}}%
\pgfpathcurveto{\pgfqpoint{3.302446in}{2.354351in}}{\pgfqpoint{3.306836in}{2.364950in}}{\pgfqpoint{3.306836in}{2.376000in}}%
\pgfpathcurveto{\pgfqpoint{3.306836in}{2.387050in}}{\pgfqpoint{3.302446in}{2.397649in}}{\pgfqpoint{3.294632in}{2.405463in}}%
\pgfpathcurveto{\pgfqpoint{3.286818in}{2.413276in}}{\pgfqpoint{3.276219in}{2.417667in}}{\pgfqpoint{3.265169in}{2.417667in}}%
\pgfpathcurveto{\pgfqpoint{3.254119in}{2.417667in}}{\pgfqpoint{3.243520in}{2.413276in}}{\pgfqpoint{3.235707in}{2.405463in}}%
\pgfpathcurveto{\pgfqpoint{3.227893in}{2.397649in}}{\pgfqpoint{3.223503in}{2.387050in}}{\pgfqpoint{3.223503in}{2.376000in}}%
\pgfpathcurveto{\pgfqpoint{3.223503in}{2.364950in}}{\pgfqpoint{3.227893in}{2.354351in}}{\pgfqpoint{3.235707in}{2.346537in}}%
\pgfpathcurveto{\pgfqpoint{3.243520in}{2.338724in}}{\pgfqpoint{3.254119in}{2.334333in}}{\pgfqpoint{3.265169in}{2.334333in}}%
\pgfpathclose%
\pgfusepath{stroke,fill}%
\end{pgfscope}%
\begin{pgfscope}%
\pgfpathrectangle{\pgfqpoint{0.800000in}{0.528000in}}{\pgfqpoint{4.960000in}{3.696000in}}%
\pgfusepath{clip}%
\pgfsetbuttcap%
\pgfsetroundjoin%
\definecolor{currentfill}{rgb}{0.000000,0.000000,0.000000}%
\pgfsetfillcolor{currentfill}%
\pgfsetlinewidth{1.003750pt}%
\definecolor{currentstroke}{rgb}{0.000000,0.000000,0.000000}%
\pgfsetstrokecolor{currentstroke}%
\pgfsetdash{}{0pt}%
\pgfpathmoveto{\pgfqpoint{3.265169in}{2.334333in}}%
\pgfpathcurveto{\pgfqpoint{3.276219in}{2.334333in}}{\pgfqpoint{3.286818in}{2.338724in}}{\pgfqpoint{3.294632in}{2.346537in}}%
\pgfpathcurveto{\pgfqpoint{3.302446in}{2.354351in}}{\pgfqpoint{3.306836in}{2.364950in}}{\pgfqpoint{3.306836in}{2.376000in}}%
\pgfpathcurveto{\pgfqpoint{3.306836in}{2.387050in}}{\pgfqpoint{3.302446in}{2.397649in}}{\pgfqpoint{3.294632in}{2.405463in}}%
\pgfpathcurveto{\pgfqpoint{3.286818in}{2.413276in}}{\pgfqpoint{3.276219in}{2.417667in}}{\pgfqpoint{3.265169in}{2.417667in}}%
\pgfpathcurveto{\pgfqpoint{3.254119in}{2.417667in}}{\pgfqpoint{3.243520in}{2.413276in}}{\pgfqpoint{3.235707in}{2.405463in}}%
\pgfpathcurveto{\pgfqpoint{3.227893in}{2.397649in}}{\pgfqpoint{3.223503in}{2.387050in}}{\pgfqpoint{3.223503in}{2.376000in}}%
\pgfpathcurveto{\pgfqpoint{3.223503in}{2.364950in}}{\pgfqpoint{3.227893in}{2.354351in}}{\pgfqpoint{3.235707in}{2.346537in}}%
\pgfpathcurveto{\pgfqpoint{3.243520in}{2.338724in}}{\pgfqpoint{3.254119in}{2.334333in}}{\pgfqpoint{3.265169in}{2.334333in}}%
\pgfpathclose%
\pgfusepath{stroke,fill}%
\end{pgfscope}%
\begin{pgfscope}%
\pgfpathrectangle{\pgfqpoint{0.800000in}{0.528000in}}{\pgfqpoint{4.960000in}{3.696000in}}%
\pgfusepath{clip}%
\pgfsetbuttcap%
\pgfsetroundjoin%
\definecolor{currentfill}{rgb}{0.000000,0.000000,0.000000}%
\pgfsetfillcolor{currentfill}%
\pgfsetlinewidth{1.003750pt}%
\definecolor{currentstroke}{rgb}{0.000000,0.000000,0.000000}%
\pgfsetstrokecolor{currentstroke}%
\pgfsetdash{}{0pt}%
\pgfpathmoveto{\pgfqpoint{3.265169in}{2.334333in}}%
\pgfpathcurveto{\pgfqpoint{3.276219in}{2.334333in}}{\pgfqpoint{3.286818in}{2.338724in}}{\pgfqpoint{3.294632in}{2.346537in}}%
\pgfpathcurveto{\pgfqpoint{3.302446in}{2.354351in}}{\pgfqpoint{3.306836in}{2.364950in}}{\pgfqpoint{3.306836in}{2.376000in}}%
\pgfpathcurveto{\pgfqpoint{3.306836in}{2.387050in}}{\pgfqpoint{3.302446in}{2.397649in}}{\pgfqpoint{3.294632in}{2.405463in}}%
\pgfpathcurveto{\pgfqpoint{3.286818in}{2.413276in}}{\pgfqpoint{3.276219in}{2.417667in}}{\pgfqpoint{3.265169in}{2.417667in}}%
\pgfpathcurveto{\pgfqpoint{3.254119in}{2.417667in}}{\pgfqpoint{3.243520in}{2.413276in}}{\pgfqpoint{3.235707in}{2.405463in}}%
\pgfpathcurveto{\pgfqpoint{3.227893in}{2.397649in}}{\pgfqpoint{3.223503in}{2.387050in}}{\pgfqpoint{3.223503in}{2.376000in}}%
\pgfpathcurveto{\pgfqpoint{3.223503in}{2.364950in}}{\pgfqpoint{3.227893in}{2.354351in}}{\pgfqpoint{3.235707in}{2.346537in}}%
\pgfpathcurveto{\pgfqpoint{3.243520in}{2.338724in}}{\pgfqpoint{3.254119in}{2.334333in}}{\pgfqpoint{3.265169in}{2.334333in}}%
\pgfpathclose%
\pgfusepath{stroke,fill}%
\end{pgfscope}%
\begin{pgfscope}%
\pgfpathrectangle{\pgfqpoint{0.800000in}{0.528000in}}{\pgfqpoint{4.960000in}{3.696000in}}%
\pgfusepath{clip}%
\pgfsetbuttcap%
\pgfsetroundjoin%
\definecolor{currentfill}{rgb}{0.000000,0.000000,0.000000}%
\pgfsetfillcolor{currentfill}%
\pgfsetlinewidth{1.003750pt}%
\definecolor{currentstroke}{rgb}{0.000000,0.000000,0.000000}%
\pgfsetstrokecolor{currentstroke}%
\pgfsetdash{}{0pt}%
\pgfpathmoveto{\pgfqpoint{3.265169in}{2.334333in}}%
\pgfpathcurveto{\pgfqpoint{3.276219in}{2.334333in}}{\pgfqpoint{3.286818in}{2.338724in}}{\pgfqpoint{3.294632in}{2.346537in}}%
\pgfpathcurveto{\pgfqpoint{3.302446in}{2.354351in}}{\pgfqpoint{3.306836in}{2.364950in}}{\pgfqpoint{3.306836in}{2.376000in}}%
\pgfpathcurveto{\pgfqpoint{3.306836in}{2.387050in}}{\pgfqpoint{3.302446in}{2.397649in}}{\pgfqpoint{3.294632in}{2.405463in}}%
\pgfpathcurveto{\pgfqpoint{3.286818in}{2.413276in}}{\pgfqpoint{3.276219in}{2.417667in}}{\pgfqpoint{3.265169in}{2.417667in}}%
\pgfpathcurveto{\pgfqpoint{3.254119in}{2.417667in}}{\pgfqpoint{3.243520in}{2.413276in}}{\pgfqpoint{3.235707in}{2.405463in}}%
\pgfpathcurveto{\pgfqpoint{3.227893in}{2.397649in}}{\pgfqpoint{3.223503in}{2.387050in}}{\pgfqpoint{3.223503in}{2.376000in}}%
\pgfpathcurveto{\pgfqpoint{3.223503in}{2.364950in}}{\pgfqpoint{3.227893in}{2.354351in}}{\pgfqpoint{3.235707in}{2.346537in}}%
\pgfpathcurveto{\pgfqpoint{3.243520in}{2.338724in}}{\pgfqpoint{3.254119in}{2.334333in}}{\pgfqpoint{3.265169in}{2.334333in}}%
\pgfpathclose%
\pgfusepath{stroke,fill}%
\end{pgfscope}%
\begin{pgfscope}%
\pgfpathrectangle{\pgfqpoint{0.800000in}{0.528000in}}{\pgfqpoint{4.960000in}{3.696000in}}%
\pgfusepath{clip}%
\pgfsetbuttcap%
\pgfsetroundjoin%
\definecolor{currentfill}{rgb}{0.000000,0.000000,0.000000}%
\pgfsetfillcolor{currentfill}%
\pgfsetlinewidth{1.003750pt}%
\definecolor{currentstroke}{rgb}{0.000000,0.000000,0.000000}%
\pgfsetstrokecolor{currentstroke}%
\pgfsetdash{}{0pt}%
\pgfpathmoveto{\pgfqpoint{3.265169in}{2.334333in}}%
\pgfpathcurveto{\pgfqpoint{3.276219in}{2.334333in}}{\pgfqpoint{3.286818in}{2.338724in}}{\pgfqpoint{3.294632in}{2.346537in}}%
\pgfpathcurveto{\pgfqpoint{3.302446in}{2.354351in}}{\pgfqpoint{3.306836in}{2.364950in}}{\pgfqpoint{3.306836in}{2.376000in}}%
\pgfpathcurveto{\pgfqpoint{3.306836in}{2.387050in}}{\pgfqpoint{3.302446in}{2.397649in}}{\pgfqpoint{3.294632in}{2.405463in}}%
\pgfpathcurveto{\pgfqpoint{3.286818in}{2.413276in}}{\pgfqpoint{3.276219in}{2.417667in}}{\pgfqpoint{3.265169in}{2.417667in}}%
\pgfpathcurveto{\pgfqpoint{3.254119in}{2.417667in}}{\pgfqpoint{3.243520in}{2.413276in}}{\pgfqpoint{3.235707in}{2.405463in}}%
\pgfpathcurveto{\pgfqpoint{3.227893in}{2.397649in}}{\pgfqpoint{3.223503in}{2.387050in}}{\pgfqpoint{3.223503in}{2.376000in}}%
\pgfpathcurveto{\pgfqpoint{3.223503in}{2.364950in}}{\pgfqpoint{3.227893in}{2.354351in}}{\pgfqpoint{3.235707in}{2.346537in}}%
\pgfpathcurveto{\pgfqpoint{3.243520in}{2.338724in}}{\pgfqpoint{3.254119in}{2.334333in}}{\pgfqpoint{3.265169in}{2.334333in}}%
\pgfpathclose%
\pgfusepath{stroke,fill}%
\end{pgfscope}%
\begin{pgfscope}%
\pgfpathrectangle{\pgfqpoint{0.800000in}{0.528000in}}{\pgfqpoint{4.960000in}{3.696000in}}%
\pgfusepath{clip}%
\pgfsetbuttcap%
\pgfsetroundjoin%
\definecolor{currentfill}{rgb}{0.000000,0.000000,0.000000}%
\pgfsetfillcolor{currentfill}%
\pgfsetlinewidth{1.003750pt}%
\definecolor{currentstroke}{rgb}{0.000000,0.000000,0.000000}%
\pgfsetstrokecolor{currentstroke}%
\pgfsetdash{}{0pt}%
\pgfpathmoveto{\pgfqpoint{3.265169in}{2.334333in}}%
\pgfpathcurveto{\pgfqpoint{3.276219in}{2.334333in}}{\pgfqpoint{3.286818in}{2.338724in}}{\pgfqpoint{3.294632in}{2.346537in}}%
\pgfpathcurveto{\pgfqpoint{3.302446in}{2.354351in}}{\pgfqpoint{3.306836in}{2.364950in}}{\pgfqpoint{3.306836in}{2.376000in}}%
\pgfpathcurveto{\pgfqpoint{3.306836in}{2.387050in}}{\pgfqpoint{3.302446in}{2.397649in}}{\pgfqpoint{3.294632in}{2.405463in}}%
\pgfpathcurveto{\pgfqpoint{3.286818in}{2.413276in}}{\pgfqpoint{3.276219in}{2.417667in}}{\pgfqpoint{3.265169in}{2.417667in}}%
\pgfpathcurveto{\pgfqpoint{3.254119in}{2.417667in}}{\pgfqpoint{3.243520in}{2.413276in}}{\pgfqpoint{3.235707in}{2.405463in}}%
\pgfpathcurveto{\pgfqpoint{3.227893in}{2.397649in}}{\pgfqpoint{3.223503in}{2.387050in}}{\pgfqpoint{3.223503in}{2.376000in}}%
\pgfpathcurveto{\pgfqpoint{3.223503in}{2.364950in}}{\pgfqpoint{3.227893in}{2.354351in}}{\pgfqpoint{3.235707in}{2.346537in}}%
\pgfpathcurveto{\pgfqpoint{3.243520in}{2.338724in}}{\pgfqpoint{3.254119in}{2.334333in}}{\pgfqpoint{3.265169in}{2.334333in}}%
\pgfpathclose%
\pgfusepath{stroke,fill}%
\end{pgfscope}%
\begin{pgfscope}%
\pgfpathrectangle{\pgfqpoint{0.800000in}{0.528000in}}{\pgfqpoint{4.960000in}{3.696000in}}%
\pgfusepath{clip}%
\pgfsetbuttcap%
\pgfsetroundjoin%
\definecolor{currentfill}{rgb}{0.000000,0.000000,0.000000}%
\pgfsetfillcolor{currentfill}%
\pgfsetlinewidth{1.003750pt}%
\definecolor{currentstroke}{rgb}{0.000000,0.000000,0.000000}%
\pgfsetstrokecolor{currentstroke}%
\pgfsetdash{}{0pt}%
\pgfpathmoveto{\pgfqpoint{3.265169in}{2.334333in}}%
\pgfpathcurveto{\pgfqpoint{3.276219in}{2.334333in}}{\pgfqpoint{3.286818in}{2.338724in}}{\pgfqpoint{3.294632in}{2.346537in}}%
\pgfpathcurveto{\pgfqpoint{3.302446in}{2.354351in}}{\pgfqpoint{3.306836in}{2.364950in}}{\pgfqpoint{3.306836in}{2.376000in}}%
\pgfpathcurveto{\pgfqpoint{3.306836in}{2.387050in}}{\pgfqpoint{3.302446in}{2.397649in}}{\pgfqpoint{3.294632in}{2.405463in}}%
\pgfpathcurveto{\pgfqpoint{3.286818in}{2.413276in}}{\pgfqpoint{3.276219in}{2.417667in}}{\pgfqpoint{3.265169in}{2.417667in}}%
\pgfpathcurveto{\pgfqpoint{3.254119in}{2.417667in}}{\pgfqpoint{3.243520in}{2.413276in}}{\pgfqpoint{3.235707in}{2.405463in}}%
\pgfpathcurveto{\pgfqpoint{3.227893in}{2.397649in}}{\pgfqpoint{3.223503in}{2.387050in}}{\pgfqpoint{3.223503in}{2.376000in}}%
\pgfpathcurveto{\pgfqpoint{3.223503in}{2.364950in}}{\pgfqpoint{3.227893in}{2.354351in}}{\pgfqpoint{3.235707in}{2.346537in}}%
\pgfpathcurveto{\pgfqpoint{3.243520in}{2.338724in}}{\pgfqpoint{3.254119in}{2.334333in}}{\pgfqpoint{3.265169in}{2.334333in}}%
\pgfpathclose%
\pgfusepath{stroke,fill}%
\end{pgfscope}%
\begin{pgfscope}%
\pgfpathrectangle{\pgfqpoint{0.800000in}{0.528000in}}{\pgfqpoint{4.960000in}{3.696000in}}%
\pgfusepath{clip}%
\pgfsetbuttcap%
\pgfsetroundjoin%
\definecolor{currentfill}{rgb}{0.000000,0.000000,0.000000}%
\pgfsetfillcolor{currentfill}%
\pgfsetlinewidth{1.003750pt}%
\definecolor{currentstroke}{rgb}{0.000000,0.000000,0.000000}%
\pgfsetstrokecolor{currentstroke}%
\pgfsetdash{}{0pt}%
\pgfpathmoveto{\pgfqpoint{3.265169in}{2.334333in}}%
\pgfpathcurveto{\pgfqpoint{3.276219in}{2.334333in}}{\pgfqpoint{3.286818in}{2.338724in}}{\pgfqpoint{3.294632in}{2.346537in}}%
\pgfpathcurveto{\pgfqpoint{3.302446in}{2.354351in}}{\pgfqpoint{3.306836in}{2.364950in}}{\pgfqpoint{3.306836in}{2.376000in}}%
\pgfpathcurveto{\pgfqpoint{3.306836in}{2.387050in}}{\pgfqpoint{3.302446in}{2.397649in}}{\pgfqpoint{3.294632in}{2.405463in}}%
\pgfpathcurveto{\pgfqpoint{3.286818in}{2.413276in}}{\pgfqpoint{3.276219in}{2.417667in}}{\pgfqpoint{3.265169in}{2.417667in}}%
\pgfpathcurveto{\pgfqpoint{3.254119in}{2.417667in}}{\pgfqpoint{3.243520in}{2.413276in}}{\pgfqpoint{3.235707in}{2.405463in}}%
\pgfpathcurveto{\pgfqpoint{3.227893in}{2.397649in}}{\pgfqpoint{3.223503in}{2.387050in}}{\pgfqpoint{3.223503in}{2.376000in}}%
\pgfpathcurveto{\pgfqpoint{3.223503in}{2.364950in}}{\pgfqpoint{3.227893in}{2.354351in}}{\pgfqpoint{3.235707in}{2.346537in}}%
\pgfpathcurveto{\pgfqpoint{3.243520in}{2.338724in}}{\pgfqpoint{3.254119in}{2.334333in}}{\pgfqpoint{3.265169in}{2.334333in}}%
\pgfpathclose%
\pgfusepath{stroke,fill}%
\end{pgfscope}%
\begin{pgfscope}%
\pgfpathrectangle{\pgfqpoint{0.800000in}{0.528000in}}{\pgfqpoint{4.960000in}{3.696000in}}%
\pgfusepath{clip}%
\pgfsetbuttcap%
\pgfsetroundjoin%
\definecolor{currentfill}{rgb}{0.000000,0.000000,0.000000}%
\pgfsetfillcolor{currentfill}%
\pgfsetlinewidth{1.003750pt}%
\definecolor{currentstroke}{rgb}{0.000000,0.000000,0.000000}%
\pgfsetstrokecolor{currentstroke}%
\pgfsetdash{}{0pt}%
\pgfpathmoveto{\pgfqpoint{3.265169in}{2.334333in}}%
\pgfpathcurveto{\pgfqpoint{3.276219in}{2.334333in}}{\pgfqpoint{3.286818in}{2.338724in}}{\pgfqpoint{3.294632in}{2.346537in}}%
\pgfpathcurveto{\pgfqpoint{3.302446in}{2.354351in}}{\pgfqpoint{3.306836in}{2.364950in}}{\pgfqpoint{3.306836in}{2.376000in}}%
\pgfpathcurveto{\pgfqpoint{3.306836in}{2.387050in}}{\pgfqpoint{3.302446in}{2.397649in}}{\pgfqpoint{3.294632in}{2.405463in}}%
\pgfpathcurveto{\pgfqpoint{3.286818in}{2.413276in}}{\pgfqpoint{3.276219in}{2.417667in}}{\pgfqpoint{3.265169in}{2.417667in}}%
\pgfpathcurveto{\pgfqpoint{3.254119in}{2.417667in}}{\pgfqpoint{3.243520in}{2.413276in}}{\pgfqpoint{3.235707in}{2.405463in}}%
\pgfpathcurveto{\pgfqpoint{3.227893in}{2.397649in}}{\pgfqpoint{3.223503in}{2.387050in}}{\pgfqpoint{3.223503in}{2.376000in}}%
\pgfpathcurveto{\pgfqpoint{3.223503in}{2.364950in}}{\pgfqpoint{3.227893in}{2.354351in}}{\pgfqpoint{3.235707in}{2.346537in}}%
\pgfpathcurveto{\pgfqpoint{3.243520in}{2.338724in}}{\pgfqpoint{3.254119in}{2.334333in}}{\pgfqpoint{3.265169in}{2.334333in}}%
\pgfpathclose%
\pgfusepath{stroke,fill}%
\end{pgfscope}%
\begin{pgfscope}%
\pgfpathrectangle{\pgfqpoint{0.800000in}{0.528000in}}{\pgfqpoint{4.960000in}{3.696000in}}%
\pgfusepath{clip}%
\pgfsetbuttcap%
\pgfsetroundjoin%
\definecolor{currentfill}{rgb}{0.000000,0.000000,0.000000}%
\pgfsetfillcolor{currentfill}%
\pgfsetlinewidth{1.003750pt}%
\definecolor{currentstroke}{rgb}{0.000000,0.000000,0.000000}%
\pgfsetstrokecolor{currentstroke}%
\pgfsetdash{}{0pt}%
\pgfpathmoveto{\pgfqpoint{3.265169in}{2.334333in}}%
\pgfpathcurveto{\pgfqpoint{3.276219in}{2.334333in}}{\pgfqpoint{3.286818in}{2.338724in}}{\pgfqpoint{3.294632in}{2.346537in}}%
\pgfpathcurveto{\pgfqpoint{3.302446in}{2.354351in}}{\pgfqpoint{3.306836in}{2.364950in}}{\pgfqpoint{3.306836in}{2.376000in}}%
\pgfpathcurveto{\pgfqpoint{3.306836in}{2.387050in}}{\pgfqpoint{3.302446in}{2.397649in}}{\pgfqpoint{3.294632in}{2.405463in}}%
\pgfpathcurveto{\pgfqpoint{3.286818in}{2.413276in}}{\pgfqpoint{3.276219in}{2.417667in}}{\pgfqpoint{3.265169in}{2.417667in}}%
\pgfpathcurveto{\pgfqpoint{3.254119in}{2.417667in}}{\pgfqpoint{3.243520in}{2.413276in}}{\pgfqpoint{3.235707in}{2.405463in}}%
\pgfpathcurveto{\pgfqpoint{3.227893in}{2.397649in}}{\pgfqpoint{3.223503in}{2.387050in}}{\pgfqpoint{3.223503in}{2.376000in}}%
\pgfpathcurveto{\pgfqpoint{3.223503in}{2.364950in}}{\pgfqpoint{3.227893in}{2.354351in}}{\pgfqpoint{3.235707in}{2.346537in}}%
\pgfpathcurveto{\pgfqpoint{3.243520in}{2.338724in}}{\pgfqpoint{3.254119in}{2.334333in}}{\pgfqpoint{3.265169in}{2.334333in}}%
\pgfpathclose%
\pgfusepath{stroke,fill}%
\end{pgfscope}%
\begin{pgfscope}%
\pgfpathrectangle{\pgfqpoint{0.800000in}{0.528000in}}{\pgfqpoint{4.960000in}{3.696000in}}%
\pgfusepath{clip}%
\pgfsetbuttcap%
\pgfsetroundjoin%
\definecolor{currentfill}{rgb}{0.000000,0.000000,0.000000}%
\pgfsetfillcolor{currentfill}%
\pgfsetlinewidth{1.003750pt}%
\definecolor{currentstroke}{rgb}{0.000000,0.000000,0.000000}%
\pgfsetstrokecolor{currentstroke}%
\pgfsetdash{}{0pt}%
\pgfpathmoveto{\pgfqpoint{3.265169in}{2.334333in}}%
\pgfpathcurveto{\pgfqpoint{3.276219in}{2.334333in}}{\pgfqpoint{3.286818in}{2.338724in}}{\pgfqpoint{3.294632in}{2.346537in}}%
\pgfpathcurveto{\pgfqpoint{3.302446in}{2.354351in}}{\pgfqpoint{3.306836in}{2.364950in}}{\pgfqpoint{3.306836in}{2.376000in}}%
\pgfpathcurveto{\pgfqpoint{3.306836in}{2.387050in}}{\pgfqpoint{3.302446in}{2.397649in}}{\pgfqpoint{3.294632in}{2.405463in}}%
\pgfpathcurveto{\pgfqpoint{3.286818in}{2.413276in}}{\pgfqpoint{3.276219in}{2.417667in}}{\pgfqpoint{3.265169in}{2.417667in}}%
\pgfpathcurveto{\pgfqpoint{3.254119in}{2.417667in}}{\pgfqpoint{3.243520in}{2.413276in}}{\pgfqpoint{3.235707in}{2.405463in}}%
\pgfpathcurveto{\pgfqpoint{3.227893in}{2.397649in}}{\pgfqpoint{3.223503in}{2.387050in}}{\pgfqpoint{3.223503in}{2.376000in}}%
\pgfpathcurveto{\pgfqpoint{3.223503in}{2.364950in}}{\pgfqpoint{3.227893in}{2.354351in}}{\pgfqpoint{3.235707in}{2.346537in}}%
\pgfpathcurveto{\pgfqpoint{3.243520in}{2.338724in}}{\pgfqpoint{3.254119in}{2.334333in}}{\pgfqpoint{3.265169in}{2.334333in}}%
\pgfpathclose%
\pgfusepath{stroke,fill}%
\end{pgfscope}%
\begin{pgfscope}%
\pgfpathrectangle{\pgfqpoint{0.800000in}{0.528000in}}{\pgfqpoint{4.960000in}{3.696000in}}%
\pgfusepath{clip}%
\pgfsetbuttcap%
\pgfsetroundjoin%
\definecolor{currentfill}{rgb}{0.000000,0.000000,0.000000}%
\pgfsetfillcolor{currentfill}%
\pgfsetlinewidth{1.003750pt}%
\definecolor{currentstroke}{rgb}{0.000000,0.000000,0.000000}%
\pgfsetstrokecolor{currentstroke}%
\pgfsetdash{}{0pt}%
\pgfpathmoveto{\pgfqpoint{3.265169in}{2.334333in}}%
\pgfpathcurveto{\pgfqpoint{3.276219in}{2.334333in}}{\pgfqpoint{3.286818in}{2.338724in}}{\pgfqpoint{3.294632in}{2.346537in}}%
\pgfpathcurveto{\pgfqpoint{3.302446in}{2.354351in}}{\pgfqpoint{3.306836in}{2.364950in}}{\pgfqpoint{3.306836in}{2.376000in}}%
\pgfpathcurveto{\pgfqpoint{3.306836in}{2.387050in}}{\pgfqpoint{3.302446in}{2.397649in}}{\pgfqpoint{3.294632in}{2.405463in}}%
\pgfpathcurveto{\pgfqpoint{3.286818in}{2.413276in}}{\pgfqpoint{3.276219in}{2.417667in}}{\pgfqpoint{3.265169in}{2.417667in}}%
\pgfpathcurveto{\pgfqpoint{3.254119in}{2.417667in}}{\pgfqpoint{3.243520in}{2.413276in}}{\pgfqpoint{3.235707in}{2.405463in}}%
\pgfpathcurveto{\pgfqpoint{3.227893in}{2.397649in}}{\pgfqpoint{3.223503in}{2.387050in}}{\pgfqpoint{3.223503in}{2.376000in}}%
\pgfpathcurveto{\pgfqpoint{3.223503in}{2.364950in}}{\pgfqpoint{3.227893in}{2.354351in}}{\pgfqpoint{3.235707in}{2.346537in}}%
\pgfpathcurveto{\pgfqpoint{3.243520in}{2.338724in}}{\pgfqpoint{3.254119in}{2.334333in}}{\pgfqpoint{3.265169in}{2.334333in}}%
\pgfpathclose%
\pgfusepath{stroke,fill}%
\end{pgfscope}%
\begin{pgfscope}%
\pgfpathrectangle{\pgfqpoint{0.800000in}{0.528000in}}{\pgfqpoint{4.960000in}{3.696000in}}%
\pgfusepath{clip}%
\pgfsetbuttcap%
\pgfsetroundjoin%
\definecolor{currentfill}{rgb}{0.000000,0.000000,0.000000}%
\pgfsetfillcolor{currentfill}%
\pgfsetlinewidth{1.003750pt}%
\definecolor{currentstroke}{rgb}{0.000000,0.000000,0.000000}%
\pgfsetstrokecolor{currentstroke}%
\pgfsetdash{}{0pt}%
\pgfpathmoveto{\pgfqpoint{3.265169in}{2.334333in}}%
\pgfpathcurveto{\pgfqpoint{3.276219in}{2.334333in}}{\pgfqpoint{3.286818in}{2.338724in}}{\pgfqpoint{3.294632in}{2.346537in}}%
\pgfpathcurveto{\pgfqpoint{3.302446in}{2.354351in}}{\pgfqpoint{3.306836in}{2.364950in}}{\pgfqpoint{3.306836in}{2.376000in}}%
\pgfpathcurveto{\pgfqpoint{3.306836in}{2.387050in}}{\pgfqpoint{3.302446in}{2.397649in}}{\pgfqpoint{3.294632in}{2.405463in}}%
\pgfpathcurveto{\pgfqpoint{3.286818in}{2.413276in}}{\pgfqpoint{3.276219in}{2.417667in}}{\pgfqpoint{3.265169in}{2.417667in}}%
\pgfpathcurveto{\pgfqpoint{3.254119in}{2.417667in}}{\pgfqpoint{3.243520in}{2.413276in}}{\pgfqpoint{3.235707in}{2.405463in}}%
\pgfpathcurveto{\pgfqpoint{3.227893in}{2.397649in}}{\pgfqpoint{3.223503in}{2.387050in}}{\pgfqpoint{3.223503in}{2.376000in}}%
\pgfpathcurveto{\pgfqpoint{3.223503in}{2.364950in}}{\pgfqpoint{3.227893in}{2.354351in}}{\pgfqpoint{3.235707in}{2.346537in}}%
\pgfpathcurveto{\pgfqpoint{3.243520in}{2.338724in}}{\pgfqpoint{3.254119in}{2.334333in}}{\pgfqpoint{3.265169in}{2.334333in}}%
\pgfpathclose%
\pgfusepath{stroke,fill}%
\end{pgfscope}%
\begin{pgfscope}%
\pgfpathrectangle{\pgfqpoint{0.800000in}{0.528000in}}{\pgfqpoint{4.960000in}{3.696000in}}%
\pgfusepath{clip}%
\pgfsetbuttcap%
\pgfsetroundjoin%
\definecolor{currentfill}{rgb}{0.000000,0.000000,0.000000}%
\pgfsetfillcolor{currentfill}%
\pgfsetlinewidth{1.003750pt}%
\definecolor{currentstroke}{rgb}{0.000000,0.000000,0.000000}%
\pgfsetstrokecolor{currentstroke}%
\pgfsetdash{}{0pt}%
\pgfpathmoveto{\pgfqpoint{3.265169in}{2.334333in}}%
\pgfpathcurveto{\pgfqpoint{3.276219in}{2.334333in}}{\pgfqpoint{3.286818in}{2.338724in}}{\pgfqpoint{3.294632in}{2.346537in}}%
\pgfpathcurveto{\pgfqpoint{3.302446in}{2.354351in}}{\pgfqpoint{3.306836in}{2.364950in}}{\pgfqpoint{3.306836in}{2.376000in}}%
\pgfpathcurveto{\pgfqpoint{3.306836in}{2.387050in}}{\pgfqpoint{3.302446in}{2.397649in}}{\pgfqpoint{3.294632in}{2.405463in}}%
\pgfpathcurveto{\pgfqpoint{3.286818in}{2.413276in}}{\pgfqpoint{3.276219in}{2.417667in}}{\pgfqpoint{3.265169in}{2.417667in}}%
\pgfpathcurveto{\pgfqpoint{3.254119in}{2.417667in}}{\pgfqpoint{3.243520in}{2.413276in}}{\pgfqpoint{3.235707in}{2.405463in}}%
\pgfpathcurveto{\pgfqpoint{3.227893in}{2.397649in}}{\pgfqpoint{3.223503in}{2.387050in}}{\pgfqpoint{3.223503in}{2.376000in}}%
\pgfpathcurveto{\pgfqpoint{3.223503in}{2.364950in}}{\pgfqpoint{3.227893in}{2.354351in}}{\pgfqpoint{3.235707in}{2.346537in}}%
\pgfpathcurveto{\pgfqpoint{3.243520in}{2.338724in}}{\pgfqpoint{3.254119in}{2.334333in}}{\pgfqpoint{3.265169in}{2.334333in}}%
\pgfpathclose%
\pgfusepath{stroke,fill}%
\end{pgfscope}%
\begin{pgfscope}%
\pgfpathrectangle{\pgfqpoint{0.800000in}{0.528000in}}{\pgfqpoint{4.960000in}{3.696000in}}%
\pgfusepath{clip}%
\pgfsetbuttcap%
\pgfsetroundjoin%
\definecolor{currentfill}{rgb}{0.000000,0.000000,0.000000}%
\pgfsetfillcolor{currentfill}%
\pgfsetlinewidth{1.003750pt}%
\definecolor{currentstroke}{rgb}{0.000000,0.000000,0.000000}%
\pgfsetstrokecolor{currentstroke}%
\pgfsetdash{}{0pt}%
\pgfpathmoveto{\pgfqpoint{3.265169in}{2.334333in}}%
\pgfpathcurveto{\pgfqpoint{3.276219in}{2.334333in}}{\pgfqpoint{3.286818in}{2.338724in}}{\pgfqpoint{3.294632in}{2.346537in}}%
\pgfpathcurveto{\pgfqpoint{3.302446in}{2.354351in}}{\pgfqpoint{3.306836in}{2.364950in}}{\pgfqpoint{3.306836in}{2.376000in}}%
\pgfpathcurveto{\pgfqpoint{3.306836in}{2.387050in}}{\pgfqpoint{3.302446in}{2.397649in}}{\pgfqpoint{3.294632in}{2.405463in}}%
\pgfpathcurveto{\pgfqpoint{3.286818in}{2.413276in}}{\pgfqpoint{3.276219in}{2.417667in}}{\pgfqpoint{3.265169in}{2.417667in}}%
\pgfpathcurveto{\pgfqpoint{3.254119in}{2.417667in}}{\pgfqpoint{3.243520in}{2.413276in}}{\pgfqpoint{3.235707in}{2.405463in}}%
\pgfpathcurveto{\pgfqpoint{3.227893in}{2.397649in}}{\pgfqpoint{3.223503in}{2.387050in}}{\pgfqpoint{3.223503in}{2.376000in}}%
\pgfpathcurveto{\pgfqpoint{3.223503in}{2.364950in}}{\pgfqpoint{3.227893in}{2.354351in}}{\pgfqpoint{3.235707in}{2.346537in}}%
\pgfpathcurveto{\pgfqpoint{3.243520in}{2.338724in}}{\pgfqpoint{3.254119in}{2.334333in}}{\pgfqpoint{3.265169in}{2.334333in}}%
\pgfpathclose%
\pgfusepath{stroke,fill}%
\end{pgfscope}%
\begin{pgfscope}%
\pgfpathrectangle{\pgfqpoint{0.800000in}{0.528000in}}{\pgfqpoint{4.960000in}{3.696000in}}%
\pgfusepath{clip}%
\pgfsetbuttcap%
\pgfsetroundjoin%
\definecolor{currentfill}{rgb}{0.000000,0.000000,0.000000}%
\pgfsetfillcolor{currentfill}%
\pgfsetlinewidth{1.003750pt}%
\definecolor{currentstroke}{rgb}{0.000000,0.000000,0.000000}%
\pgfsetstrokecolor{currentstroke}%
\pgfsetdash{}{0pt}%
\pgfpathmoveto{\pgfqpoint{3.265169in}{2.334333in}}%
\pgfpathcurveto{\pgfqpoint{3.276219in}{2.334333in}}{\pgfqpoint{3.286818in}{2.338724in}}{\pgfqpoint{3.294632in}{2.346537in}}%
\pgfpathcurveto{\pgfqpoint{3.302446in}{2.354351in}}{\pgfqpoint{3.306836in}{2.364950in}}{\pgfqpoint{3.306836in}{2.376000in}}%
\pgfpathcurveto{\pgfqpoint{3.306836in}{2.387050in}}{\pgfqpoint{3.302446in}{2.397649in}}{\pgfqpoint{3.294632in}{2.405463in}}%
\pgfpathcurveto{\pgfqpoint{3.286818in}{2.413276in}}{\pgfqpoint{3.276219in}{2.417667in}}{\pgfqpoint{3.265169in}{2.417667in}}%
\pgfpathcurveto{\pgfqpoint{3.254119in}{2.417667in}}{\pgfqpoint{3.243520in}{2.413276in}}{\pgfqpoint{3.235707in}{2.405463in}}%
\pgfpathcurveto{\pgfqpoint{3.227893in}{2.397649in}}{\pgfqpoint{3.223503in}{2.387050in}}{\pgfqpoint{3.223503in}{2.376000in}}%
\pgfpathcurveto{\pgfqpoint{3.223503in}{2.364950in}}{\pgfqpoint{3.227893in}{2.354351in}}{\pgfqpoint{3.235707in}{2.346537in}}%
\pgfpathcurveto{\pgfqpoint{3.243520in}{2.338724in}}{\pgfqpoint{3.254119in}{2.334333in}}{\pgfqpoint{3.265169in}{2.334333in}}%
\pgfpathclose%
\pgfusepath{stroke,fill}%
\end{pgfscope}%
\begin{pgfscope}%
\pgfpathrectangle{\pgfqpoint{0.800000in}{0.528000in}}{\pgfqpoint{4.960000in}{3.696000in}}%
\pgfusepath{clip}%
\pgfsetbuttcap%
\pgfsetroundjoin%
\definecolor{currentfill}{rgb}{0.000000,0.000000,0.000000}%
\pgfsetfillcolor{currentfill}%
\pgfsetlinewidth{1.003750pt}%
\definecolor{currentstroke}{rgb}{0.000000,0.000000,0.000000}%
\pgfsetstrokecolor{currentstroke}%
\pgfsetdash{}{0pt}%
\pgfpathmoveto{\pgfqpoint{3.265169in}{2.334333in}}%
\pgfpathcurveto{\pgfqpoint{3.276219in}{2.334333in}}{\pgfqpoint{3.286818in}{2.338724in}}{\pgfqpoint{3.294632in}{2.346537in}}%
\pgfpathcurveto{\pgfqpoint{3.302446in}{2.354351in}}{\pgfqpoint{3.306836in}{2.364950in}}{\pgfqpoint{3.306836in}{2.376000in}}%
\pgfpathcurveto{\pgfqpoint{3.306836in}{2.387050in}}{\pgfqpoint{3.302446in}{2.397649in}}{\pgfqpoint{3.294632in}{2.405463in}}%
\pgfpathcurveto{\pgfqpoint{3.286818in}{2.413276in}}{\pgfqpoint{3.276219in}{2.417667in}}{\pgfqpoint{3.265169in}{2.417667in}}%
\pgfpathcurveto{\pgfqpoint{3.254119in}{2.417667in}}{\pgfqpoint{3.243520in}{2.413276in}}{\pgfqpoint{3.235707in}{2.405463in}}%
\pgfpathcurveto{\pgfqpoint{3.227893in}{2.397649in}}{\pgfqpoint{3.223503in}{2.387050in}}{\pgfqpoint{3.223503in}{2.376000in}}%
\pgfpathcurveto{\pgfqpoint{3.223503in}{2.364950in}}{\pgfqpoint{3.227893in}{2.354351in}}{\pgfqpoint{3.235707in}{2.346537in}}%
\pgfpathcurveto{\pgfqpoint{3.243520in}{2.338724in}}{\pgfqpoint{3.254119in}{2.334333in}}{\pgfqpoint{3.265169in}{2.334333in}}%
\pgfpathclose%
\pgfusepath{stroke,fill}%
\end{pgfscope}%
\begin{pgfscope}%
\pgfpathrectangle{\pgfqpoint{0.800000in}{0.528000in}}{\pgfqpoint{4.960000in}{3.696000in}}%
\pgfusepath{clip}%
\pgfsetbuttcap%
\pgfsetroundjoin%
\definecolor{currentfill}{rgb}{0.000000,0.000000,0.000000}%
\pgfsetfillcolor{currentfill}%
\pgfsetlinewidth{1.003750pt}%
\definecolor{currentstroke}{rgb}{0.000000,0.000000,0.000000}%
\pgfsetstrokecolor{currentstroke}%
\pgfsetdash{}{0pt}%
\pgfpathmoveto{\pgfqpoint{3.265169in}{2.334333in}}%
\pgfpathcurveto{\pgfqpoint{3.276219in}{2.334333in}}{\pgfqpoint{3.286818in}{2.338724in}}{\pgfqpoint{3.294632in}{2.346537in}}%
\pgfpathcurveto{\pgfqpoint{3.302446in}{2.354351in}}{\pgfqpoint{3.306836in}{2.364950in}}{\pgfqpoint{3.306836in}{2.376000in}}%
\pgfpathcurveto{\pgfqpoint{3.306836in}{2.387050in}}{\pgfqpoint{3.302446in}{2.397649in}}{\pgfqpoint{3.294632in}{2.405463in}}%
\pgfpathcurveto{\pgfqpoint{3.286818in}{2.413276in}}{\pgfqpoint{3.276219in}{2.417667in}}{\pgfqpoint{3.265169in}{2.417667in}}%
\pgfpathcurveto{\pgfqpoint{3.254119in}{2.417667in}}{\pgfqpoint{3.243520in}{2.413276in}}{\pgfqpoint{3.235707in}{2.405463in}}%
\pgfpathcurveto{\pgfqpoint{3.227893in}{2.397649in}}{\pgfqpoint{3.223503in}{2.387050in}}{\pgfqpoint{3.223503in}{2.376000in}}%
\pgfpathcurveto{\pgfqpoint{3.223503in}{2.364950in}}{\pgfqpoint{3.227893in}{2.354351in}}{\pgfqpoint{3.235707in}{2.346537in}}%
\pgfpathcurveto{\pgfqpoint{3.243520in}{2.338724in}}{\pgfqpoint{3.254119in}{2.334333in}}{\pgfqpoint{3.265169in}{2.334333in}}%
\pgfpathclose%
\pgfusepath{stroke,fill}%
\end{pgfscope}%
\begin{pgfscope}%
\pgfpathrectangle{\pgfqpoint{0.800000in}{0.528000in}}{\pgfqpoint{4.960000in}{3.696000in}}%
\pgfusepath{clip}%
\pgfsetbuttcap%
\pgfsetroundjoin%
\definecolor{currentfill}{rgb}{0.000000,0.000000,0.000000}%
\pgfsetfillcolor{currentfill}%
\pgfsetlinewidth{1.003750pt}%
\definecolor{currentstroke}{rgb}{0.000000,0.000000,0.000000}%
\pgfsetstrokecolor{currentstroke}%
\pgfsetdash{}{0pt}%
\pgfpathmoveto{\pgfqpoint{3.265169in}{2.334333in}}%
\pgfpathcurveto{\pgfqpoint{3.276219in}{2.334333in}}{\pgfqpoint{3.286818in}{2.338724in}}{\pgfqpoint{3.294632in}{2.346537in}}%
\pgfpathcurveto{\pgfqpoint{3.302446in}{2.354351in}}{\pgfqpoint{3.306836in}{2.364950in}}{\pgfqpoint{3.306836in}{2.376000in}}%
\pgfpathcurveto{\pgfqpoint{3.306836in}{2.387050in}}{\pgfqpoint{3.302446in}{2.397649in}}{\pgfqpoint{3.294632in}{2.405463in}}%
\pgfpathcurveto{\pgfqpoint{3.286818in}{2.413276in}}{\pgfqpoint{3.276219in}{2.417667in}}{\pgfqpoint{3.265169in}{2.417667in}}%
\pgfpathcurveto{\pgfqpoint{3.254119in}{2.417667in}}{\pgfqpoint{3.243520in}{2.413276in}}{\pgfqpoint{3.235707in}{2.405463in}}%
\pgfpathcurveto{\pgfqpoint{3.227893in}{2.397649in}}{\pgfqpoint{3.223503in}{2.387050in}}{\pgfqpoint{3.223503in}{2.376000in}}%
\pgfpathcurveto{\pgfqpoint{3.223503in}{2.364950in}}{\pgfqpoint{3.227893in}{2.354351in}}{\pgfqpoint{3.235707in}{2.346537in}}%
\pgfpathcurveto{\pgfqpoint{3.243520in}{2.338724in}}{\pgfqpoint{3.254119in}{2.334333in}}{\pgfqpoint{3.265169in}{2.334333in}}%
\pgfpathclose%
\pgfusepath{stroke,fill}%
\end{pgfscope}%
\begin{pgfscope}%
\pgfpathrectangle{\pgfqpoint{0.800000in}{0.528000in}}{\pgfqpoint{4.960000in}{3.696000in}}%
\pgfusepath{clip}%
\pgfsetbuttcap%
\pgfsetroundjoin%
\definecolor{currentfill}{rgb}{0.000000,0.000000,0.000000}%
\pgfsetfillcolor{currentfill}%
\pgfsetlinewidth{1.003750pt}%
\definecolor{currentstroke}{rgb}{0.000000,0.000000,0.000000}%
\pgfsetstrokecolor{currentstroke}%
\pgfsetdash{}{0pt}%
\pgfpathmoveto{\pgfqpoint{3.265169in}{2.334333in}}%
\pgfpathcurveto{\pgfqpoint{3.276219in}{2.334333in}}{\pgfqpoint{3.286818in}{2.338724in}}{\pgfqpoint{3.294632in}{2.346537in}}%
\pgfpathcurveto{\pgfqpoint{3.302446in}{2.354351in}}{\pgfqpoint{3.306836in}{2.364950in}}{\pgfqpoint{3.306836in}{2.376000in}}%
\pgfpathcurveto{\pgfqpoint{3.306836in}{2.387050in}}{\pgfqpoint{3.302446in}{2.397649in}}{\pgfqpoint{3.294632in}{2.405463in}}%
\pgfpathcurveto{\pgfqpoint{3.286818in}{2.413276in}}{\pgfqpoint{3.276219in}{2.417667in}}{\pgfqpoint{3.265169in}{2.417667in}}%
\pgfpathcurveto{\pgfqpoint{3.254119in}{2.417667in}}{\pgfqpoint{3.243520in}{2.413276in}}{\pgfqpoint{3.235707in}{2.405463in}}%
\pgfpathcurveto{\pgfqpoint{3.227893in}{2.397649in}}{\pgfqpoint{3.223503in}{2.387050in}}{\pgfqpoint{3.223503in}{2.376000in}}%
\pgfpathcurveto{\pgfqpoint{3.223503in}{2.364950in}}{\pgfqpoint{3.227893in}{2.354351in}}{\pgfqpoint{3.235707in}{2.346537in}}%
\pgfpathcurveto{\pgfqpoint{3.243520in}{2.338724in}}{\pgfqpoint{3.254119in}{2.334333in}}{\pgfqpoint{3.265169in}{2.334333in}}%
\pgfpathclose%
\pgfusepath{stroke,fill}%
\end{pgfscope}%
\begin{pgfscope}%
\pgfpathrectangle{\pgfqpoint{0.800000in}{0.528000in}}{\pgfqpoint{4.960000in}{3.696000in}}%
\pgfusepath{clip}%
\pgfsetbuttcap%
\pgfsetroundjoin%
\definecolor{currentfill}{rgb}{0.000000,0.000000,0.000000}%
\pgfsetfillcolor{currentfill}%
\pgfsetlinewidth{1.003750pt}%
\definecolor{currentstroke}{rgb}{0.000000,0.000000,0.000000}%
\pgfsetstrokecolor{currentstroke}%
\pgfsetdash{}{0pt}%
\pgfpathmoveto{\pgfqpoint{3.265169in}{2.334333in}}%
\pgfpathcurveto{\pgfqpoint{3.276219in}{2.334333in}}{\pgfqpoint{3.286818in}{2.338724in}}{\pgfqpoint{3.294632in}{2.346537in}}%
\pgfpathcurveto{\pgfqpoint{3.302446in}{2.354351in}}{\pgfqpoint{3.306836in}{2.364950in}}{\pgfqpoint{3.306836in}{2.376000in}}%
\pgfpathcurveto{\pgfqpoint{3.306836in}{2.387050in}}{\pgfqpoint{3.302446in}{2.397649in}}{\pgfqpoint{3.294632in}{2.405463in}}%
\pgfpathcurveto{\pgfqpoint{3.286818in}{2.413276in}}{\pgfqpoint{3.276219in}{2.417667in}}{\pgfqpoint{3.265169in}{2.417667in}}%
\pgfpathcurveto{\pgfqpoint{3.254119in}{2.417667in}}{\pgfqpoint{3.243520in}{2.413276in}}{\pgfqpoint{3.235707in}{2.405463in}}%
\pgfpathcurveto{\pgfqpoint{3.227893in}{2.397649in}}{\pgfqpoint{3.223503in}{2.387050in}}{\pgfqpoint{3.223503in}{2.376000in}}%
\pgfpathcurveto{\pgfqpoint{3.223503in}{2.364950in}}{\pgfqpoint{3.227893in}{2.354351in}}{\pgfqpoint{3.235707in}{2.346537in}}%
\pgfpathcurveto{\pgfqpoint{3.243520in}{2.338724in}}{\pgfqpoint{3.254119in}{2.334333in}}{\pgfqpoint{3.265169in}{2.334333in}}%
\pgfpathclose%
\pgfusepath{stroke,fill}%
\end{pgfscope}%
\begin{pgfscope}%
\pgfpathrectangle{\pgfqpoint{0.800000in}{0.528000in}}{\pgfqpoint{4.960000in}{3.696000in}}%
\pgfusepath{clip}%
\pgfsetbuttcap%
\pgfsetroundjoin%
\definecolor{currentfill}{rgb}{0.000000,0.000000,0.000000}%
\pgfsetfillcolor{currentfill}%
\pgfsetlinewidth{1.003750pt}%
\definecolor{currentstroke}{rgb}{0.000000,0.000000,0.000000}%
\pgfsetstrokecolor{currentstroke}%
\pgfsetdash{}{0pt}%
\pgfpathmoveto{\pgfqpoint{3.265169in}{2.334333in}}%
\pgfpathcurveto{\pgfqpoint{3.276219in}{2.334333in}}{\pgfqpoint{3.286818in}{2.338724in}}{\pgfqpoint{3.294632in}{2.346537in}}%
\pgfpathcurveto{\pgfqpoint{3.302446in}{2.354351in}}{\pgfqpoint{3.306836in}{2.364950in}}{\pgfqpoint{3.306836in}{2.376000in}}%
\pgfpathcurveto{\pgfqpoint{3.306836in}{2.387050in}}{\pgfqpoint{3.302446in}{2.397649in}}{\pgfqpoint{3.294632in}{2.405463in}}%
\pgfpathcurveto{\pgfqpoint{3.286818in}{2.413276in}}{\pgfqpoint{3.276219in}{2.417667in}}{\pgfqpoint{3.265169in}{2.417667in}}%
\pgfpathcurveto{\pgfqpoint{3.254119in}{2.417667in}}{\pgfqpoint{3.243520in}{2.413276in}}{\pgfqpoint{3.235707in}{2.405463in}}%
\pgfpathcurveto{\pgfqpoint{3.227893in}{2.397649in}}{\pgfqpoint{3.223503in}{2.387050in}}{\pgfqpoint{3.223503in}{2.376000in}}%
\pgfpathcurveto{\pgfqpoint{3.223503in}{2.364950in}}{\pgfqpoint{3.227893in}{2.354351in}}{\pgfqpoint{3.235707in}{2.346537in}}%
\pgfpathcurveto{\pgfqpoint{3.243520in}{2.338724in}}{\pgfqpoint{3.254119in}{2.334333in}}{\pgfqpoint{3.265169in}{2.334333in}}%
\pgfpathclose%
\pgfusepath{stroke,fill}%
\end{pgfscope}%
\begin{pgfscope}%
\pgfpathrectangle{\pgfqpoint{0.800000in}{0.528000in}}{\pgfqpoint{4.960000in}{3.696000in}}%
\pgfusepath{clip}%
\pgfsetbuttcap%
\pgfsetroundjoin%
\definecolor{currentfill}{rgb}{0.000000,0.000000,0.000000}%
\pgfsetfillcolor{currentfill}%
\pgfsetlinewidth{1.003750pt}%
\definecolor{currentstroke}{rgb}{0.000000,0.000000,0.000000}%
\pgfsetstrokecolor{currentstroke}%
\pgfsetdash{}{0pt}%
\pgfpathmoveto{\pgfqpoint{3.265169in}{2.334333in}}%
\pgfpathcurveto{\pgfqpoint{3.276219in}{2.334333in}}{\pgfqpoint{3.286818in}{2.338724in}}{\pgfqpoint{3.294632in}{2.346537in}}%
\pgfpathcurveto{\pgfqpoint{3.302446in}{2.354351in}}{\pgfqpoint{3.306836in}{2.364950in}}{\pgfqpoint{3.306836in}{2.376000in}}%
\pgfpathcurveto{\pgfqpoint{3.306836in}{2.387050in}}{\pgfqpoint{3.302446in}{2.397649in}}{\pgfqpoint{3.294632in}{2.405463in}}%
\pgfpathcurveto{\pgfqpoint{3.286818in}{2.413276in}}{\pgfqpoint{3.276219in}{2.417667in}}{\pgfqpoint{3.265169in}{2.417667in}}%
\pgfpathcurveto{\pgfqpoint{3.254119in}{2.417667in}}{\pgfqpoint{3.243520in}{2.413276in}}{\pgfqpoint{3.235707in}{2.405463in}}%
\pgfpathcurveto{\pgfqpoint{3.227893in}{2.397649in}}{\pgfqpoint{3.223503in}{2.387050in}}{\pgfqpoint{3.223503in}{2.376000in}}%
\pgfpathcurveto{\pgfqpoint{3.223503in}{2.364950in}}{\pgfqpoint{3.227893in}{2.354351in}}{\pgfqpoint{3.235707in}{2.346537in}}%
\pgfpathcurveto{\pgfqpoint{3.243520in}{2.338724in}}{\pgfqpoint{3.254119in}{2.334333in}}{\pgfqpoint{3.265169in}{2.334333in}}%
\pgfpathclose%
\pgfusepath{stroke,fill}%
\end{pgfscope}%
\begin{pgfscope}%
\pgfpathrectangle{\pgfqpoint{0.800000in}{0.528000in}}{\pgfqpoint{4.960000in}{3.696000in}}%
\pgfusepath{clip}%
\pgfsetbuttcap%
\pgfsetroundjoin%
\definecolor{currentfill}{rgb}{0.000000,0.000000,0.000000}%
\pgfsetfillcolor{currentfill}%
\pgfsetlinewidth{1.003750pt}%
\definecolor{currentstroke}{rgb}{0.000000,0.000000,0.000000}%
\pgfsetstrokecolor{currentstroke}%
\pgfsetdash{}{0pt}%
\pgfpathmoveto{\pgfqpoint{3.265169in}{2.334333in}}%
\pgfpathcurveto{\pgfqpoint{3.276219in}{2.334333in}}{\pgfqpoint{3.286818in}{2.338724in}}{\pgfqpoint{3.294632in}{2.346537in}}%
\pgfpathcurveto{\pgfqpoint{3.302446in}{2.354351in}}{\pgfqpoint{3.306836in}{2.364950in}}{\pgfqpoint{3.306836in}{2.376000in}}%
\pgfpathcurveto{\pgfqpoint{3.306836in}{2.387050in}}{\pgfqpoint{3.302446in}{2.397649in}}{\pgfqpoint{3.294632in}{2.405463in}}%
\pgfpathcurveto{\pgfqpoint{3.286818in}{2.413276in}}{\pgfqpoint{3.276219in}{2.417667in}}{\pgfqpoint{3.265169in}{2.417667in}}%
\pgfpathcurveto{\pgfqpoint{3.254119in}{2.417667in}}{\pgfqpoint{3.243520in}{2.413276in}}{\pgfqpoint{3.235707in}{2.405463in}}%
\pgfpathcurveto{\pgfqpoint{3.227893in}{2.397649in}}{\pgfqpoint{3.223503in}{2.387050in}}{\pgfqpoint{3.223503in}{2.376000in}}%
\pgfpathcurveto{\pgfqpoint{3.223503in}{2.364950in}}{\pgfqpoint{3.227893in}{2.354351in}}{\pgfqpoint{3.235707in}{2.346537in}}%
\pgfpathcurveto{\pgfqpoint{3.243520in}{2.338724in}}{\pgfqpoint{3.254119in}{2.334333in}}{\pgfqpoint{3.265169in}{2.334333in}}%
\pgfpathclose%
\pgfusepath{stroke,fill}%
\end{pgfscope}%
\begin{pgfscope}%
\pgfpathrectangle{\pgfqpoint{0.800000in}{0.528000in}}{\pgfqpoint{4.960000in}{3.696000in}}%
\pgfusepath{clip}%
\pgfsetbuttcap%
\pgfsetroundjoin%
\definecolor{currentfill}{rgb}{0.000000,0.000000,0.000000}%
\pgfsetfillcolor{currentfill}%
\pgfsetlinewidth{1.003750pt}%
\definecolor{currentstroke}{rgb}{0.000000,0.000000,0.000000}%
\pgfsetstrokecolor{currentstroke}%
\pgfsetdash{}{0pt}%
\pgfpathmoveto{\pgfqpoint{3.265169in}{2.334333in}}%
\pgfpathcurveto{\pgfqpoint{3.276219in}{2.334333in}}{\pgfqpoint{3.286818in}{2.338724in}}{\pgfqpoint{3.294632in}{2.346537in}}%
\pgfpathcurveto{\pgfqpoint{3.302446in}{2.354351in}}{\pgfqpoint{3.306836in}{2.364950in}}{\pgfqpoint{3.306836in}{2.376000in}}%
\pgfpathcurveto{\pgfqpoint{3.306836in}{2.387050in}}{\pgfqpoint{3.302446in}{2.397649in}}{\pgfqpoint{3.294632in}{2.405463in}}%
\pgfpathcurveto{\pgfqpoint{3.286818in}{2.413276in}}{\pgfqpoint{3.276219in}{2.417667in}}{\pgfqpoint{3.265169in}{2.417667in}}%
\pgfpathcurveto{\pgfqpoint{3.254119in}{2.417667in}}{\pgfqpoint{3.243520in}{2.413276in}}{\pgfqpoint{3.235707in}{2.405463in}}%
\pgfpathcurveto{\pgfqpoint{3.227893in}{2.397649in}}{\pgfqpoint{3.223503in}{2.387050in}}{\pgfqpoint{3.223503in}{2.376000in}}%
\pgfpathcurveto{\pgfqpoint{3.223503in}{2.364950in}}{\pgfqpoint{3.227893in}{2.354351in}}{\pgfqpoint{3.235707in}{2.346537in}}%
\pgfpathcurveto{\pgfqpoint{3.243520in}{2.338724in}}{\pgfqpoint{3.254119in}{2.334333in}}{\pgfqpoint{3.265169in}{2.334333in}}%
\pgfpathclose%
\pgfusepath{stroke,fill}%
\end{pgfscope}%
\begin{pgfscope}%
\pgfpathrectangle{\pgfqpoint{0.800000in}{0.528000in}}{\pgfqpoint{4.960000in}{3.696000in}}%
\pgfusepath{clip}%
\pgfsetbuttcap%
\pgfsetroundjoin%
\definecolor{currentfill}{rgb}{0.000000,0.000000,0.000000}%
\pgfsetfillcolor{currentfill}%
\pgfsetlinewidth{1.003750pt}%
\definecolor{currentstroke}{rgb}{0.000000,0.000000,0.000000}%
\pgfsetstrokecolor{currentstroke}%
\pgfsetdash{}{0pt}%
\pgfpathmoveto{\pgfqpoint{3.265169in}{2.334333in}}%
\pgfpathcurveto{\pgfqpoint{3.276219in}{2.334333in}}{\pgfqpoint{3.286818in}{2.338724in}}{\pgfqpoint{3.294632in}{2.346537in}}%
\pgfpathcurveto{\pgfqpoint{3.302446in}{2.354351in}}{\pgfqpoint{3.306836in}{2.364950in}}{\pgfqpoint{3.306836in}{2.376000in}}%
\pgfpathcurveto{\pgfqpoint{3.306836in}{2.387050in}}{\pgfqpoint{3.302446in}{2.397649in}}{\pgfqpoint{3.294632in}{2.405463in}}%
\pgfpathcurveto{\pgfqpoint{3.286818in}{2.413276in}}{\pgfqpoint{3.276219in}{2.417667in}}{\pgfqpoint{3.265169in}{2.417667in}}%
\pgfpathcurveto{\pgfqpoint{3.254119in}{2.417667in}}{\pgfqpoint{3.243520in}{2.413276in}}{\pgfqpoint{3.235707in}{2.405463in}}%
\pgfpathcurveto{\pgfqpoint{3.227893in}{2.397649in}}{\pgfqpoint{3.223503in}{2.387050in}}{\pgfqpoint{3.223503in}{2.376000in}}%
\pgfpathcurveto{\pgfqpoint{3.223503in}{2.364950in}}{\pgfqpoint{3.227893in}{2.354351in}}{\pgfqpoint{3.235707in}{2.346537in}}%
\pgfpathcurveto{\pgfqpoint{3.243520in}{2.338724in}}{\pgfqpoint{3.254119in}{2.334333in}}{\pgfqpoint{3.265169in}{2.334333in}}%
\pgfpathclose%
\pgfusepath{stroke,fill}%
\end{pgfscope}%
\begin{pgfscope}%
\pgfpathrectangle{\pgfqpoint{0.800000in}{0.528000in}}{\pgfqpoint{4.960000in}{3.696000in}}%
\pgfusepath{clip}%
\pgfsetbuttcap%
\pgfsetroundjoin%
\definecolor{currentfill}{rgb}{0.000000,0.000000,0.000000}%
\pgfsetfillcolor{currentfill}%
\pgfsetlinewidth{1.003750pt}%
\definecolor{currentstroke}{rgb}{0.000000,0.000000,0.000000}%
\pgfsetstrokecolor{currentstroke}%
\pgfsetdash{}{0pt}%
\pgfpathmoveto{\pgfqpoint{3.265169in}{2.334333in}}%
\pgfpathcurveto{\pgfqpoint{3.276219in}{2.334333in}}{\pgfqpoint{3.286818in}{2.338724in}}{\pgfqpoint{3.294632in}{2.346537in}}%
\pgfpathcurveto{\pgfqpoint{3.302446in}{2.354351in}}{\pgfqpoint{3.306836in}{2.364950in}}{\pgfqpoint{3.306836in}{2.376000in}}%
\pgfpathcurveto{\pgfqpoint{3.306836in}{2.387050in}}{\pgfqpoint{3.302446in}{2.397649in}}{\pgfqpoint{3.294632in}{2.405463in}}%
\pgfpathcurveto{\pgfqpoint{3.286818in}{2.413276in}}{\pgfqpoint{3.276219in}{2.417667in}}{\pgfqpoint{3.265169in}{2.417667in}}%
\pgfpathcurveto{\pgfqpoint{3.254119in}{2.417667in}}{\pgfqpoint{3.243520in}{2.413276in}}{\pgfqpoint{3.235707in}{2.405463in}}%
\pgfpathcurveto{\pgfqpoint{3.227893in}{2.397649in}}{\pgfqpoint{3.223503in}{2.387050in}}{\pgfqpoint{3.223503in}{2.376000in}}%
\pgfpathcurveto{\pgfqpoint{3.223503in}{2.364950in}}{\pgfqpoint{3.227893in}{2.354351in}}{\pgfqpoint{3.235707in}{2.346537in}}%
\pgfpathcurveto{\pgfqpoint{3.243520in}{2.338724in}}{\pgfqpoint{3.254119in}{2.334333in}}{\pgfqpoint{3.265169in}{2.334333in}}%
\pgfpathclose%
\pgfusepath{stroke,fill}%
\end{pgfscope}%
\begin{pgfscope}%
\pgfpathrectangle{\pgfqpoint{0.800000in}{0.528000in}}{\pgfqpoint{4.960000in}{3.696000in}}%
\pgfusepath{clip}%
\pgfsetbuttcap%
\pgfsetroundjoin%
\definecolor{currentfill}{rgb}{0.000000,0.000000,0.000000}%
\pgfsetfillcolor{currentfill}%
\pgfsetlinewidth{1.003750pt}%
\definecolor{currentstroke}{rgb}{0.000000,0.000000,0.000000}%
\pgfsetstrokecolor{currentstroke}%
\pgfsetdash{}{0pt}%
\pgfpathmoveto{\pgfqpoint{3.265169in}{2.334333in}}%
\pgfpathcurveto{\pgfqpoint{3.276219in}{2.334333in}}{\pgfqpoint{3.286818in}{2.338724in}}{\pgfqpoint{3.294632in}{2.346537in}}%
\pgfpathcurveto{\pgfqpoint{3.302446in}{2.354351in}}{\pgfqpoint{3.306836in}{2.364950in}}{\pgfqpoint{3.306836in}{2.376000in}}%
\pgfpathcurveto{\pgfqpoint{3.306836in}{2.387050in}}{\pgfqpoint{3.302446in}{2.397649in}}{\pgfqpoint{3.294632in}{2.405463in}}%
\pgfpathcurveto{\pgfqpoint{3.286818in}{2.413276in}}{\pgfqpoint{3.276219in}{2.417667in}}{\pgfqpoint{3.265169in}{2.417667in}}%
\pgfpathcurveto{\pgfqpoint{3.254119in}{2.417667in}}{\pgfqpoint{3.243520in}{2.413276in}}{\pgfqpoint{3.235707in}{2.405463in}}%
\pgfpathcurveto{\pgfqpoint{3.227893in}{2.397649in}}{\pgfqpoint{3.223503in}{2.387050in}}{\pgfqpoint{3.223503in}{2.376000in}}%
\pgfpathcurveto{\pgfqpoint{3.223503in}{2.364950in}}{\pgfqpoint{3.227893in}{2.354351in}}{\pgfqpoint{3.235707in}{2.346537in}}%
\pgfpathcurveto{\pgfqpoint{3.243520in}{2.338724in}}{\pgfqpoint{3.254119in}{2.334333in}}{\pgfqpoint{3.265169in}{2.334333in}}%
\pgfpathclose%
\pgfusepath{stroke,fill}%
\end{pgfscope}%
\begin{pgfscope}%
\pgfpathrectangle{\pgfqpoint{0.800000in}{0.528000in}}{\pgfqpoint{4.960000in}{3.696000in}}%
\pgfusepath{clip}%
\pgfsetbuttcap%
\pgfsetroundjoin%
\definecolor{currentfill}{rgb}{0.000000,0.000000,0.000000}%
\pgfsetfillcolor{currentfill}%
\pgfsetlinewidth{1.003750pt}%
\definecolor{currentstroke}{rgb}{0.000000,0.000000,0.000000}%
\pgfsetstrokecolor{currentstroke}%
\pgfsetdash{}{0pt}%
\pgfpathmoveto{\pgfqpoint{3.265169in}{2.334333in}}%
\pgfpathcurveto{\pgfqpoint{3.276219in}{2.334333in}}{\pgfqpoint{3.286818in}{2.338724in}}{\pgfqpoint{3.294632in}{2.346537in}}%
\pgfpathcurveto{\pgfqpoint{3.302446in}{2.354351in}}{\pgfqpoint{3.306836in}{2.364950in}}{\pgfqpoint{3.306836in}{2.376000in}}%
\pgfpathcurveto{\pgfqpoint{3.306836in}{2.387050in}}{\pgfqpoint{3.302446in}{2.397649in}}{\pgfqpoint{3.294632in}{2.405463in}}%
\pgfpathcurveto{\pgfqpoint{3.286818in}{2.413276in}}{\pgfqpoint{3.276219in}{2.417667in}}{\pgfqpoint{3.265169in}{2.417667in}}%
\pgfpathcurveto{\pgfqpoint{3.254119in}{2.417667in}}{\pgfqpoint{3.243520in}{2.413276in}}{\pgfqpoint{3.235707in}{2.405463in}}%
\pgfpathcurveto{\pgfqpoint{3.227893in}{2.397649in}}{\pgfqpoint{3.223503in}{2.387050in}}{\pgfqpoint{3.223503in}{2.376000in}}%
\pgfpathcurveto{\pgfqpoint{3.223503in}{2.364950in}}{\pgfqpoint{3.227893in}{2.354351in}}{\pgfqpoint{3.235707in}{2.346537in}}%
\pgfpathcurveto{\pgfqpoint{3.243520in}{2.338724in}}{\pgfqpoint{3.254119in}{2.334333in}}{\pgfqpoint{3.265169in}{2.334333in}}%
\pgfpathclose%
\pgfusepath{stroke,fill}%
\end{pgfscope}%
\begin{pgfscope}%
\pgfpathrectangle{\pgfqpoint{0.800000in}{0.528000in}}{\pgfqpoint{4.960000in}{3.696000in}}%
\pgfusepath{clip}%
\pgfsetbuttcap%
\pgfsetroundjoin%
\definecolor{currentfill}{rgb}{0.000000,0.000000,0.000000}%
\pgfsetfillcolor{currentfill}%
\pgfsetlinewidth{1.003750pt}%
\definecolor{currentstroke}{rgb}{0.000000,0.000000,0.000000}%
\pgfsetstrokecolor{currentstroke}%
\pgfsetdash{}{0pt}%
\pgfpathmoveto{\pgfqpoint{3.265169in}{2.334333in}}%
\pgfpathcurveto{\pgfqpoint{3.276219in}{2.334333in}}{\pgfqpoint{3.286818in}{2.338724in}}{\pgfqpoint{3.294632in}{2.346537in}}%
\pgfpathcurveto{\pgfqpoint{3.302446in}{2.354351in}}{\pgfqpoint{3.306836in}{2.364950in}}{\pgfqpoint{3.306836in}{2.376000in}}%
\pgfpathcurveto{\pgfqpoint{3.306836in}{2.387050in}}{\pgfqpoint{3.302446in}{2.397649in}}{\pgfqpoint{3.294632in}{2.405463in}}%
\pgfpathcurveto{\pgfqpoint{3.286818in}{2.413276in}}{\pgfqpoint{3.276219in}{2.417667in}}{\pgfqpoint{3.265169in}{2.417667in}}%
\pgfpathcurveto{\pgfqpoint{3.254119in}{2.417667in}}{\pgfqpoint{3.243520in}{2.413276in}}{\pgfqpoint{3.235707in}{2.405463in}}%
\pgfpathcurveto{\pgfqpoint{3.227893in}{2.397649in}}{\pgfqpoint{3.223503in}{2.387050in}}{\pgfqpoint{3.223503in}{2.376000in}}%
\pgfpathcurveto{\pgfqpoint{3.223503in}{2.364950in}}{\pgfqpoint{3.227893in}{2.354351in}}{\pgfqpoint{3.235707in}{2.346537in}}%
\pgfpathcurveto{\pgfqpoint{3.243520in}{2.338724in}}{\pgfqpoint{3.254119in}{2.334333in}}{\pgfqpoint{3.265169in}{2.334333in}}%
\pgfpathclose%
\pgfusepath{stroke,fill}%
\end{pgfscope}%
\begin{pgfscope}%
\pgfpathrectangle{\pgfqpoint{0.800000in}{0.528000in}}{\pgfqpoint{4.960000in}{3.696000in}}%
\pgfusepath{clip}%
\pgfsetbuttcap%
\pgfsetroundjoin%
\definecolor{currentfill}{rgb}{0.000000,0.000000,0.000000}%
\pgfsetfillcolor{currentfill}%
\pgfsetlinewidth{1.003750pt}%
\definecolor{currentstroke}{rgb}{0.000000,0.000000,0.000000}%
\pgfsetstrokecolor{currentstroke}%
\pgfsetdash{}{0pt}%
\pgfpathmoveto{\pgfqpoint{3.265169in}{2.334333in}}%
\pgfpathcurveto{\pgfqpoint{3.276219in}{2.334333in}}{\pgfqpoint{3.286818in}{2.338724in}}{\pgfqpoint{3.294632in}{2.346537in}}%
\pgfpathcurveto{\pgfqpoint{3.302446in}{2.354351in}}{\pgfqpoint{3.306836in}{2.364950in}}{\pgfqpoint{3.306836in}{2.376000in}}%
\pgfpathcurveto{\pgfqpoint{3.306836in}{2.387050in}}{\pgfqpoint{3.302446in}{2.397649in}}{\pgfqpoint{3.294632in}{2.405463in}}%
\pgfpathcurveto{\pgfqpoint{3.286818in}{2.413276in}}{\pgfqpoint{3.276219in}{2.417667in}}{\pgfqpoint{3.265169in}{2.417667in}}%
\pgfpathcurveto{\pgfqpoint{3.254119in}{2.417667in}}{\pgfqpoint{3.243520in}{2.413276in}}{\pgfqpoint{3.235707in}{2.405463in}}%
\pgfpathcurveto{\pgfqpoint{3.227893in}{2.397649in}}{\pgfqpoint{3.223503in}{2.387050in}}{\pgfqpoint{3.223503in}{2.376000in}}%
\pgfpathcurveto{\pgfqpoint{3.223503in}{2.364950in}}{\pgfqpoint{3.227893in}{2.354351in}}{\pgfqpoint{3.235707in}{2.346537in}}%
\pgfpathcurveto{\pgfqpoint{3.243520in}{2.338724in}}{\pgfqpoint{3.254119in}{2.334333in}}{\pgfqpoint{3.265169in}{2.334333in}}%
\pgfpathclose%
\pgfusepath{stroke,fill}%
\end{pgfscope}%
\begin{pgfscope}%
\pgfpathrectangle{\pgfqpoint{0.800000in}{0.528000in}}{\pgfqpoint{4.960000in}{3.696000in}}%
\pgfusepath{clip}%
\pgfsetbuttcap%
\pgfsetroundjoin%
\definecolor{currentfill}{rgb}{0.000000,0.000000,0.000000}%
\pgfsetfillcolor{currentfill}%
\pgfsetlinewidth{1.003750pt}%
\definecolor{currentstroke}{rgb}{0.000000,0.000000,0.000000}%
\pgfsetstrokecolor{currentstroke}%
\pgfsetdash{}{0pt}%
\pgfpathmoveto{\pgfqpoint{3.265169in}{2.334333in}}%
\pgfpathcurveto{\pgfqpoint{3.276219in}{2.334333in}}{\pgfqpoint{3.286818in}{2.338724in}}{\pgfqpoint{3.294632in}{2.346537in}}%
\pgfpathcurveto{\pgfqpoint{3.302446in}{2.354351in}}{\pgfqpoint{3.306836in}{2.364950in}}{\pgfqpoint{3.306836in}{2.376000in}}%
\pgfpathcurveto{\pgfqpoint{3.306836in}{2.387050in}}{\pgfqpoint{3.302446in}{2.397649in}}{\pgfqpoint{3.294632in}{2.405463in}}%
\pgfpathcurveto{\pgfqpoint{3.286818in}{2.413276in}}{\pgfqpoint{3.276219in}{2.417667in}}{\pgfqpoint{3.265169in}{2.417667in}}%
\pgfpathcurveto{\pgfqpoint{3.254119in}{2.417667in}}{\pgfqpoint{3.243520in}{2.413276in}}{\pgfqpoint{3.235707in}{2.405463in}}%
\pgfpathcurveto{\pgfqpoint{3.227893in}{2.397649in}}{\pgfqpoint{3.223503in}{2.387050in}}{\pgfqpoint{3.223503in}{2.376000in}}%
\pgfpathcurveto{\pgfqpoint{3.223503in}{2.364950in}}{\pgfqpoint{3.227893in}{2.354351in}}{\pgfqpoint{3.235707in}{2.346537in}}%
\pgfpathcurveto{\pgfqpoint{3.243520in}{2.338724in}}{\pgfqpoint{3.254119in}{2.334333in}}{\pgfqpoint{3.265169in}{2.334333in}}%
\pgfpathclose%
\pgfusepath{stroke,fill}%
\end{pgfscope}%
\begin{pgfscope}%
\pgfpathrectangle{\pgfqpoint{0.800000in}{0.528000in}}{\pgfqpoint{4.960000in}{3.696000in}}%
\pgfusepath{clip}%
\pgfsetbuttcap%
\pgfsetroundjoin%
\definecolor{currentfill}{rgb}{0.000000,0.000000,0.000000}%
\pgfsetfillcolor{currentfill}%
\pgfsetlinewidth{1.003750pt}%
\definecolor{currentstroke}{rgb}{0.000000,0.000000,0.000000}%
\pgfsetstrokecolor{currentstroke}%
\pgfsetdash{}{0pt}%
\pgfpathmoveto{\pgfqpoint{3.265169in}{2.334333in}}%
\pgfpathcurveto{\pgfqpoint{3.276219in}{2.334333in}}{\pgfqpoint{3.286818in}{2.338724in}}{\pgfqpoint{3.294632in}{2.346537in}}%
\pgfpathcurveto{\pgfqpoint{3.302446in}{2.354351in}}{\pgfqpoint{3.306836in}{2.364950in}}{\pgfqpoint{3.306836in}{2.376000in}}%
\pgfpathcurveto{\pgfqpoint{3.306836in}{2.387050in}}{\pgfqpoint{3.302446in}{2.397649in}}{\pgfqpoint{3.294632in}{2.405463in}}%
\pgfpathcurveto{\pgfqpoint{3.286818in}{2.413276in}}{\pgfqpoint{3.276219in}{2.417667in}}{\pgfqpoint{3.265169in}{2.417667in}}%
\pgfpathcurveto{\pgfqpoint{3.254119in}{2.417667in}}{\pgfqpoint{3.243520in}{2.413276in}}{\pgfqpoint{3.235707in}{2.405463in}}%
\pgfpathcurveto{\pgfqpoint{3.227893in}{2.397649in}}{\pgfqpoint{3.223503in}{2.387050in}}{\pgfqpoint{3.223503in}{2.376000in}}%
\pgfpathcurveto{\pgfqpoint{3.223503in}{2.364950in}}{\pgfqpoint{3.227893in}{2.354351in}}{\pgfqpoint{3.235707in}{2.346537in}}%
\pgfpathcurveto{\pgfqpoint{3.243520in}{2.338724in}}{\pgfqpoint{3.254119in}{2.334333in}}{\pgfqpoint{3.265169in}{2.334333in}}%
\pgfpathclose%
\pgfusepath{stroke,fill}%
\end{pgfscope}%
\begin{pgfscope}%
\pgfpathrectangle{\pgfqpoint{0.800000in}{0.528000in}}{\pgfqpoint{4.960000in}{3.696000in}}%
\pgfusepath{clip}%
\pgfsetbuttcap%
\pgfsetroundjoin%
\definecolor{currentfill}{rgb}{0.000000,0.000000,0.000000}%
\pgfsetfillcolor{currentfill}%
\pgfsetlinewidth{1.003750pt}%
\definecolor{currentstroke}{rgb}{0.000000,0.000000,0.000000}%
\pgfsetstrokecolor{currentstroke}%
\pgfsetdash{}{0pt}%
\pgfpathmoveto{\pgfqpoint{3.265169in}{2.334333in}}%
\pgfpathcurveto{\pgfqpoint{3.276219in}{2.334333in}}{\pgfqpoint{3.286818in}{2.338724in}}{\pgfqpoint{3.294632in}{2.346537in}}%
\pgfpathcurveto{\pgfqpoint{3.302446in}{2.354351in}}{\pgfqpoint{3.306836in}{2.364950in}}{\pgfqpoint{3.306836in}{2.376000in}}%
\pgfpathcurveto{\pgfqpoint{3.306836in}{2.387050in}}{\pgfqpoint{3.302446in}{2.397649in}}{\pgfqpoint{3.294632in}{2.405463in}}%
\pgfpathcurveto{\pgfqpoint{3.286818in}{2.413276in}}{\pgfqpoint{3.276219in}{2.417667in}}{\pgfqpoint{3.265169in}{2.417667in}}%
\pgfpathcurveto{\pgfqpoint{3.254119in}{2.417667in}}{\pgfqpoint{3.243520in}{2.413276in}}{\pgfqpoint{3.235707in}{2.405463in}}%
\pgfpathcurveto{\pgfqpoint{3.227893in}{2.397649in}}{\pgfqpoint{3.223503in}{2.387050in}}{\pgfqpoint{3.223503in}{2.376000in}}%
\pgfpathcurveto{\pgfqpoint{3.223503in}{2.364950in}}{\pgfqpoint{3.227893in}{2.354351in}}{\pgfqpoint{3.235707in}{2.346537in}}%
\pgfpathcurveto{\pgfqpoint{3.243520in}{2.338724in}}{\pgfqpoint{3.254119in}{2.334333in}}{\pgfqpoint{3.265169in}{2.334333in}}%
\pgfpathclose%
\pgfusepath{stroke,fill}%
\end{pgfscope}%
\begin{pgfscope}%
\pgfpathrectangle{\pgfqpoint{0.800000in}{0.528000in}}{\pgfqpoint{4.960000in}{3.696000in}}%
\pgfusepath{clip}%
\pgfsetbuttcap%
\pgfsetroundjoin%
\definecolor{currentfill}{rgb}{0.000000,0.000000,0.000000}%
\pgfsetfillcolor{currentfill}%
\pgfsetlinewidth{1.003750pt}%
\definecolor{currentstroke}{rgb}{0.000000,0.000000,0.000000}%
\pgfsetstrokecolor{currentstroke}%
\pgfsetdash{}{0pt}%
\pgfpathmoveto{\pgfqpoint{3.265169in}{2.334333in}}%
\pgfpathcurveto{\pgfqpoint{3.276219in}{2.334333in}}{\pgfqpoint{3.286818in}{2.338724in}}{\pgfqpoint{3.294632in}{2.346537in}}%
\pgfpathcurveto{\pgfqpoint{3.302446in}{2.354351in}}{\pgfqpoint{3.306836in}{2.364950in}}{\pgfqpoint{3.306836in}{2.376000in}}%
\pgfpathcurveto{\pgfqpoint{3.306836in}{2.387050in}}{\pgfqpoint{3.302446in}{2.397649in}}{\pgfqpoint{3.294632in}{2.405463in}}%
\pgfpathcurveto{\pgfqpoint{3.286818in}{2.413276in}}{\pgfqpoint{3.276219in}{2.417667in}}{\pgfqpoint{3.265169in}{2.417667in}}%
\pgfpathcurveto{\pgfqpoint{3.254119in}{2.417667in}}{\pgfqpoint{3.243520in}{2.413276in}}{\pgfqpoint{3.235707in}{2.405463in}}%
\pgfpathcurveto{\pgfqpoint{3.227893in}{2.397649in}}{\pgfqpoint{3.223503in}{2.387050in}}{\pgfqpoint{3.223503in}{2.376000in}}%
\pgfpathcurveto{\pgfqpoint{3.223503in}{2.364950in}}{\pgfqpoint{3.227893in}{2.354351in}}{\pgfqpoint{3.235707in}{2.346537in}}%
\pgfpathcurveto{\pgfqpoint{3.243520in}{2.338724in}}{\pgfqpoint{3.254119in}{2.334333in}}{\pgfqpoint{3.265169in}{2.334333in}}%
\pgfpathclose%
\pgfusepath{stroke,fill}%
\end{pgfscope}%
\begin{pgfscope}%
\pgfpathrectangle{\pgfqpoint{0.800000in}{0.528000in}}{\pgfqpoint{4.960000in}{3.696000in}}%
\pgfusepath{clip}%
\pgfsetbuttcap%
\pgfsetroundjoin%
\definecolor{currentfill}{rgb}{0.000000,0.000000,0.000000}%
\pgfsetfillcolor{currentfill}%
\pgfsetlinewidth{1.003750pt}%
\definecolor{currentstroke}{rgb}{0.000000,0.000000,0.000000}%
\pgfsetstrokecolor{currentstroke}%
\pgfsetdash{}{0pt}%
\pgfpathmoveto{\pgfqpoint{3.265169in}{2.334333in}}%
\pgfpathcurveto{\pgfqpoint{3.276219in}{2.334333in}}{\pgfqpoint{3.286818in}{2.338724in}}{\pgfqpoint{3.294632in}{2.346537in}}%
\pgfpathcurveto{\pgfqpoint{3.302446in}{2.354351in}}{\pgfqpoint{3.306836in}{2.364950in}}{\pgfqpoint{3.306836in}{2.376000in}}%
\pgfpathcurveto{\pgfqpoint{3.306836in}{2.387050in}}{\pgfqpoint{3.302446in}{2.397649in}}{\pgfqpoint{3.294632in}{2.405463in}}%
\pgfpathcurveto{\pgfqpoint{3.286818in}{2.413276in}}{\pgfqpoint{3.276219in}{2.417667in}}{\pgfqpoint{3.265169in}{2.417667in}}%
\pgfpathcurveto{\pgfqpoint{3.254119in}{2.417667in}}{\pgfqpoint{3.243520in}{2.413276in}}{\pgfqpoint{3.235707in}{2.405463in}}%
\pgfpathcurveto{\pgfqpoint{3.227893in}{2.397649in}}{\pgfqpoint{3.223503in}{2.387050in}}{\pgfqpoint{3.223503in}{2.376000in}}%
\pgfpathcurveto{\pgfqpoint{3.223503in}{2.364950in}}{\pgfqpoint{3.227893in}{2.354351in}}{\pgfqpoint{3.235707in}{2.346537in}}%
\pgfpathcurveto{\pgfqpoint{3.243520in}{2.338724in}}{\pgfqpoint{3.254119in}{2.334333in}}{\pgfqpoint{3.265169in}{2.334333in}}%
\pgfpathclose%
\pgfusepath{stroke,fill}%
\end{pgfscope}%
\begin{pgfscope}%
\pgfpathrectangle{\pgfqpoint{0.800000in}{0.528000in}}{\pgfqpoint{4.960000in}{3.696000in}}%
\pgfusepath{clip}%
\pgfsetbuttcap%
\pgfsetroundjoin%
\definecolor{currentfill}{rgb}{0.000000,0.000000,0.000000}%
\pgfsetfillcolor{currentfill}%
\pgfsetlinewidth{1.003750pt}%
\definecolor{currentstroke}{rgb}{0.000000,0.000000,0.000000}%
\pgfsetstrokecolor{currentstroke}%
\pgfsetdash{}{0pt}%
\pgfpathmoveto{\pgfqpoint{3.265169in}{2.334333in}}%
\pgfpathcurveto{\pgfqpoint{3.276219in}{2.334333in}}{\pgfqpoint{3.286818in}{2.338724in}}{\pgfqpoint{3.294632in}{2.346537in}}%
\pgfpathcurveto{\pgfqpoint{3.302446in}{2.354351in}}{\pgfqpoint{3.306836in}{2.364950in}}{\pgfqpoint{3.306836in}{2.376000in}}%
\pgfpathcurveto{\pgfqpoint{3.306836in}{2.387050in}}{\pgfqpoint{3.302446in}{2.397649in}}{\pgfqpoint{3.294632in}{2.405463in}}%
\pgfpathcurveto{\pgfqpoint{3.286818in}{2.413276in}}{\pgfqpoint{3.276219in}{2.417667in}}{\pgfqpoint{3.265169in}{2.417667in}}%
\pgfpathcurveto{\pgfqpoint{3.254119in}{2.417667in}}{\pgfqpoint{3.243520in}{2.413276in}}{\pgfqpoint{3.235707in}{2.405463in}}%
\pgfpathcurveto{\pgfqpoint{3.227893in}{2.397649in}}{\pgfqpoint{3.223503in}{2.387050in}}{\pgfqpoint{3.223503in}{2.376000in}}%
\pgfpathcurveto{\pgfqpoint{3.223503in}{2.364950in}}{\pgfqpoint{3.227893in}{2.354351in}}{\pgfqpoint{3.235707in}{2.346537in}}%
\pgfpathcurveto{\pgfqpoint{3.243520in}{2.338724in}}{\pgfqpoint{3.254119in}{2.334333in}}{\pgfqpoint{3.265169in}{2.334333in}}%
\pgfpathclose%
\pgfusepath{stroke,fill}%
\end{pgfscope}%
\begin{pgfscope}%
\pgfpathrectangle{\pgfqpoint{0.800000in}{0.528000in}}{\pgfqpoint{4.960000in}{3.696000in}}%
\pgfusepath{clip}%
\pgfsetbuttcap%
\pgfsetroundjoin%
\definecolor{currentfill}{rgb}{0.000000,0.000000,0.000000}%
\pgfsetfillcolor{currentfill}%
\pgfsetlinewidth{1.003750pt}%
\definecolor{currentstroke}{rgb}{0.000000,0.000000,0.000000}%
\pgfsetstrokecolor{currentstroke}%
\pgfsetdash{}{0pt}%
\pgfpathmoveto{\pgfqpoint{3.265169in}{2.334333in}}%
\pgfpathcurveto{\pgfqpoint{3.276219in}{2.334333in}}{\pgfqpoint{3.286818in}{2.338724in}}{\pgfqpoint{3.294632in}{2.346537in}}%
\pgfpathcurveto{\pgfqpoint{3.302446in}{2.354351in}}{\pgfqpoint{3.306836in}{2.364950in}}{\pgfqpoint{3.306836in}{2.376000in}}%
\pgfpathcurveto{\pgfqpoint{3.306836in}{2.387050in}}{\pgfqpoint{3.302446in}{2.397649in}}{\pgfqpoint{3.294632in}{2.405463in}}%
\pgfpathcurveto{\pgfqpoint{3.286818in}{2.413276in}}{\pgfqpoint{3.276219in}{2.417667in}}{\pgfqpoint{3.265169in}{2.417667in}}%
\pgfpathcurveto{\pgfqpoint{3.254119in}{2.417667in}}{\pgfqpoint{3.243520in}{2.413276in}}{\pgfqpoint{3.235707in}{2.405463in}}%
\pgfpathcurveto{\pgfqpoint{3.227893in}{2.397649in}}{\pgfqpoint{3.223503in}{2.387050in}}{\pgfqpoint{3.223503in}{2.376000in}}%
\pgfpathcurveto{\pgfqpoint{3.223503in}{2.364950in}}{\pgfqpoint{3.227893in}{2.354351in}}{\pgfqpoint{3.235707in}{2.346537in}}%
\pgfpathcurveto{\pgfqpoint{3.243520in}{2.338724in}}{\pgfqpoint{3.254119in}{2.334333in}}{\pgfqpoint{3.265169in}{2.334333in}}%
\pgfpathclose%
\pgfusepath{stroke,fill}%
\end{pgfscope}%
\begin{pgfscope}%
\pgfpathrectangle{\pgfqpoint{0.800000in}{0.528000in}}{\pgfqpoint{4.960000in}{3.696000in}}%
\pgfusepath{clip}%
\pgfsetbuttcap%
\pgfsetroundjoin%
\definecolor{currentfill}{rgb}{0.000000,0.000000,0.000000}%
\pgfsetfillcolor{currentfill}%
\pgfsetlinewidth{1.003750pt}%
\definecolor{currentstroke}{rgb}{0.000000,0.000000,0.000000}%
\pgfsetstrokecolor{currentstroke}%
\pgfsetdash{}{0pt}%
\pgfpathmoveto{\pgfqpoint{3.265169in}{2.334333in}}%
\pgfpathcurveto{\pgfqpoint{3.276219in}{2.334333in}}{\pgfqpoint{3.286818in}{2.338724in}}{\pgfqpoint{3.294632in}{2.346537in}}%
\pgfpathcurveto{\pgfqpoint{3.302446in}{2.354351in}}{\pgfqpoint{3.306836in}{2.364950in}}{\pgfqpoint{3.306836in}{2.376000in}}%
\pgfpathcurveto{\pgfqpoint{3.306836in}{2.387050in}}{\pgfqpoint{3.302446in}{2.397649in}}{\pgfqpoint{3.294632in}{2.405463in}}%
\pgfpathcurveto{\pgfqpoint{3.286818in}{2.413276in}}{\pgfqpoint{3.276219in}{2.417667in}}{\pgfqpoint{3.265169in}{2.417667in}}%
\pgfpathcurveto{\pgfqpoint{3.254119in}{2.417667in}}{\pgfqpoint{3.243520in}{2.413276in}}{\pgfqpoint{3.235707in}{2.405463in}}%
\pgfpathcurveto{\pgfqpoint{3.227893in}{2.397649in}}{\pgfqpoint{3.223503in}{2.387050in}}{\pgfqpoint{3.223503in}{2.376000in}}%
\pgfpathcurveto{\pgfqpoint{3.223503in}{2.364950in}}{\pgfqpoint{3.227893in}{2.354351in}}{\pgfqpoint{3.235707in}{2.346537in}}%
\pgfpathcurveto{\pgfqpoint{3.243520in}{2.338724in}}{\pgfqpoint{3.254119in}{2.334333in}}{\pgfqpoint{3.265169in}{2.334333in}}%
\pgfpathclose%
\pgfusepath{stroke,fill}%
\end{pgfscope}%
\begin{pgfscope}%
\pgfpathrectangle{\pgfqpoint{0.800000in}{0.528000in}}{\pgfqpoint{4.960000in}{3.696000in}}%
\pgfusepath{clip}%
\pgfsetbuttcap%
\pgfsetroundjoin%
\definecolor{currentfill}{rgb}{0.000000,0.000000,0.000000}%
\pgfsetfillcolor{currentfill}%
\pgfsetlinewidth{1.003750pt}%
\definecolor{currentstroke}{rgb}{0.000000,0.000000,0.000000}%
\pgfsetstrokecolor{currentstroke}%
\pgfsetdash{}{0pt}%
\pgfpathmoveto{\pgfqpoint{3.265169in}{2.334333in}}%
\pgfpathcurveto{\pgfqpoint{3.276219in}{2.334333in}}{\pgfqpoint{3.286818in}{2.338724in}}{\pgfqpoint{3.294632in}{2.346537in}}%
\pgfpathcurveto{\pgfqpoint{3.302446in}{2.354351in}}{\pgfqpoint{3.306836in}{2.364950in}}{\pgfqpoint{3.306836in}{2.376000in}}%
\pgfpathcurveto{\pgfqpoint{3.306836in}{2.387050in}}{\pgfqpoint{3.302446in}{2.397649in}}{\pgfqpoint{3.294632in}{2.405463in}}%
\pgfpathcurveto{\pgfqpoint{3.286818in}{2.413276in}}{\pgfqpoint{3.276219in}{2.417667in}}{\pgfqpoint{3.265169in}{2.417667in}}%
\pgfpathcurveto{\pgfqpoint{3.254119in}{2.417667in}}{\pgfqpoint{3.243520in}{2.413276in}}{\pgfqpoint{3.235707in}{2.405463in}}%
\pgfpathcurveto{\pgfqpoint{3.227893in}{2.397649in}}{\pgfqpoint{3.223503in}{2.387050in}}{\pgfqpoint{3.223503in}{2.376000in}}%
\pgfpathcurveto{\pgfqpoint{3.223503in}{2.364950in}}{\pgfqpoint{3.227893in}{2.354351in}}{\pgfqpoint{3.235707in}{2.346537in}}%
\pgfpathcurveto{\pgfqpoint{3.243520in}{2.338724in}}{\pgfqpoint{3.254119in}{2.334333in}}{\pgfqpoint{3.265169in}{2.334333in}}%
\pgfpathclose%
\pgfusepath{stroke,fill}%
\end{pgfscope}%
\begin{pgfscope}%
\pgfpathrectangle{\pgfqpoint{0.800000in}{0.528000in}}{\pgfqpoint{4.960000in}{3.696000in}}%
\pgfusepath{clip}%
\pgfsetbuttcap%
\pgfsetroundjoin%
\definecolor{currentfill}{rgb}{0.000000,0.000000,0.000000}%
\pgfsetfillcolor{currentfill}%
\pgfsetlinewidth{1.003750pt}%
\definecolor{currentstroke}{rgb}{0.000000,0.000000,0.000000}%
\pgfsetstrokecolor{currentstroke}%
\pgfsetdash{}{0pt}%
\pgfpathmoveto{\pgfqpoint{3.265169in}{2.334333in}}%
\pgfpathcurveto{\pgfqpoint{3.276219in}{2.334333in}}{\pgfqpoint{3.286818in}{2.338724in}}{\pgfqpoint{3.294632in}{2.346537in}}%
\pgfpathcurveto{\pgfqpoint{3.302446in}{2.354351in}}{\pgfqpoint{3.306836in}{2.364950in}}{\pgfqpoint{3.306836in}{2.376000in}}%
\pgfpathcurveto{\pgfqpoint{3.306836in}{2.387050in}}{\pgfqpoint{3.302446in}{2.397649in}}{\pgfqpoint{3.294632in}{2.405463in}}%
\pgfpathcurveto{\pgfqpoint{3.286818in}{2.413276in}}{\pgfqpoint{3.276219in}{2.417667in}}{\pgfqpoint{3.265169in}{2.417667in}}%
\pgfpathcurveto{\pgfqpoint{3.254119in}{2.417667in}}{\pgfqpoint{3.243520in}{2.413276in}}{\pgfqpoint{3.235707in}{2.405463in}}%
\pgfpathcurveto{\pgfqpoint{3.227893in}{2.397649in}}{\pgfqpoint{3.223503in}{2.387050in}}{\pgfqpoint{3.223503in}{2.376000in}}%
\pgfpathcurveto{\pgfqpoint{3.223503in}{2.364950in}}{\pgfqpoint{3.227893in}{2.354351in}}{\pgfqpoint{3.235707in}{2.346537in}}%
\pgfpathcurveto{\pgfqpoint{3.243520in}{2.338724in}}{\pgfqpoint{3.254119in}{2.334333in}}{\pgfqpoint{3.265169in}{2.334333in}}%
\pgfpathclose%
\pgfusepath{stroke,fill}%
\end{pgfscope}%
\begin{pgfscope}%
\pgfpathrectangle{\pgfqpoint{0.800000in}{0.528000in}}{\pgfqpoint{4.960000in}{3.696000in}}%
\pgfusepath{clip}%
\pgfsetbuttcap%
\pgfsetroundjoin%
\definecolor{currentfill}{rgb}{0.000000,0.000000,0.000000}%
\pgfsetfillcolor{currentfill}%
\pgfsetlinewidth{1.003750pt}%
\definecolor{currentstroke}{rgb}{0.000000,0.000000,0.000000}%
\pgfsetstrokecolor{currentstroke}%
\pgfsetdash{}{0pt}%
\pgfpathmoveto{\pgfqpoint{3.265169in}{2.334333in}}%
\pgfpathcurveto{\pgfqpoint{3.276219in}{2.334333in}}{\pgfqpoint{3.286818in}{2.338724in}}{\pgfqpoint{3.294632in}{2.346537in}}%
\pgfpathcurveto{\pgfqpoint{3.302446in}{2.354351in}}{\pgfqpoint{3.306836in}{2.364950in}}{\pgfqpoint{3.306836in}{2.376000in}}%
\pgfpathcurveto{\pgfqpoint{3.306836in}{2.387050in}}{\pgfqpoint{3.302446in}{2.397649in}}{\pgfqpoint{3.294632in}{2.405463in}}%
\pgfpathcurveto{\pgfqpoint{3.286818in}{2.413276in}}{\pgfqpoint{3.276219in}{2.417667in}}{\pgfqpoint{3.265169in}{2.417667in}}%
\pgfpathcurveto{\pgfqpoint{3.254119in}{2.417667in}}{\pgfqpoint{3.243520in}{2.413276in}}{\pgfqpoint{3.235707in}{2.405463in}}%
\pgfpathcurveto{\pgfqpoint{3.227893in}{2.397649in}}{\pgfqpoint{3.223503in}{2.387050in}}{\pgfqpoint{3.223503in}{2.376000in}}%
\pgfpathcurveto{\pgfqpoint{3.223503in}{2.364950in}}{\pgfqpoint{3.227893in}{2.354351in}}{\pgfqpoint{3.235707in}{2.346537in}}%
\pgfpathcurveto{\pgfqpoint{3.243520in}{2.338724in}}{\pgfqpoint{3.254119in}{2.334333in}}{\pgfqpoint{3.265169in}{2.334333in}}%
\pgfpathclose%
\pgfusepath{stroke,fill}%
\end{pgfscope}%
\begin{pgfscope}%
\pgfpathrectangle{\pgfqpoint{0.800000in}{0.528000in}}{\pgfqpoint{4.960000in}{3.696000in}}%
\pgfusepath{clip}%
\pgfsetbuttcap%
\pgfsetroundjoin%
\definecolor{currentfill}{rgb}{0.000000,0.000000,0.000000}%
\pgfsetfillcolor{currentfill}%
\pgfsetlinewidth{1.003750pt}%
\definecolor{currentstroke}{rgb}{0.000000,0.000000,0.000000}%
\pgfsetstrokecolor{currentstroke}%
\pgfsetdash{}{0pt}%
\pgfpathmoveto{\pgfqpoint{3.265169in}{2.334333in}}%
\pgfpathcurveto{\pgfqpoint{3.276219in}{2.334333in}}{\pgfqpoint{3.286818in}{2.338724in}}{\pgfqpoint{3.294632in}{2.346537in}}%
\pgfpathcurveto{\pgfqpoint{3.302446in}{2.354351in}}{\pgfqpoint{3.306836in}{2.364950in}}{\pgfqpoint{3.306836in}{2.376000in}}%
\pgfpathcurveto{\pgfqpoint{3.306836in}{2.387050in}}{\pgfqpoint{3.302446in}{2.397649in}}{\pgfqpoint{3.294632in}{2.405463in}}%
\pgfpathcurveto{\pgfqpoint{3.286818in}{2.413276in}}{\pgfqpoint{3.276219in}{2.417667in}}{\pgfqpoint{3.265169in}{2.417667in}}%
\pgfpathcurveto{\pgfqpoint{3.254119in}{2.417667in}}{\pgfqpoint{3.243520in}{2.413276in}}{\pgfqpoint{3.235707in}{2.405463in}}%
\pgfpathcurveto{\pgfqpoint{3.227893in}{2.397649in}}{\pgfqpoint{3.223503in}{2.387050in}}{\pgfqpoint{3.223503in}{2.376000in}}%
\pgfpathcurveto{\pgfqpoint{3.223503in}{2.364950in}}{\pgfqpoint{3.227893in}{2.354351in}}{\pgfqpoint{3.235707in}{2.346537in}}%
\pgfpathcurveto{\pgfqpoint{3.243520in}{2.338724in}}{\pgfqpoint{3.254119in}{2.334333in}}{\pgfqpoint{3.265169in}{2.334333in}}%
\pgfpathclose%
\pgfusepath{stroke,fill}%
\end{pgfscope}%
\begin{pgfscope}%
\pgfpathrectangle{\pgfqpoint{0.800000in}{0.528000in}}{\pgfqpoint{4.960000in}{3.696000in}}%
\pgfusepath{clip}%
\pgfsetbuttcap%
\pgfsetroundjoin%
\definecolor{currentfill}{rgb}{0.000000,0.000000,0.000000}%
\pgfsetfillcolor{currentfill}%
\pgfsetlinewidth{1.003750pt}%
\definecolor{currentstroke}{rgb}{0.000000,0.000000,0.000000}%
\pgfsetstrokecolor{currentstroke}%
\pgfsetdash{}{0pt}%
\pgfpathmoveto{\pgfqpoint{3.265169in}{2.334333in}}%
\pgfpathcurveto{\pgfqpoint{3.276219in}{2.334333in}}{\pgfqpoint{3.286818in}{2.338724in}}{\pgfqpoint{3.294632in}{2.346537in}}%
\pgfpathcurveto{\pgfqpoint{3.302446in}{2.354351in}}{\pgfqpoint{3.306836in}{2.364950in}}{\pgfqpoint{3.306836in}{2.376000in}}%
\pgfpathcurveto{\pgfqpoint{3.306836in}{2.387050in}}{\pgfqpoint{3.302446in}{2.397649in}}{\pgfqpoint{3.294632in}{2.405463in}}%
\pgfpathcurveto{\pgfqpoint{3.286818in}{2.413276in}}{\pgfqpoint{3.276219in}{2.417667in}}{\pgfqpoint{3.265169in}{2.417667in}}%
\pgfpathcurveto{\pgfqpoint{3.254119in}{2.417667in}}{\pgfqpoint{3.243520in}{2.413276in}}{\pgfqpoint{3.235707in}{2.405463in}}%
\pgfpathcurveto{\pgfqpoint{3.227893in}{2.397649in}}{\pgfqpoint{3.223503in}{2.387050in}}{\pgfqpoint{3.223503in}{2.376000in}}%
\pgfpathcurveto{\pgfqpoint{3.223503in}{2.364950in}}{\pgfqpoint{3.227893in}{2.354351in}}{\pgfqpoint{3.235707in}{2.346537in}}%
\pgfpathcurveto{\pgfqpoint{3.243520in}{2.338724in}}{\pgfqpoint{3.254119in}{2.334333in}}{\pgfqpoint{3.265169in}{2.334333in}}%
\pgfpathclose%
\pgfusepath{stroke,fill}%
\end{pgfscope}%
\begin{pgfscope}%
\pgfpathrectangle{\pgfqpoint{0.800000in}{0.528000in}}{\pgfqpoint{4.960000in}{3.696000in}}%
\pgfusepath{clip}%
\pgfsetbuttcap%
\pgfsetroundjoin%
\definecolor{currentfill}{rgb}{0.000000,0.000000,0.000000}%
\pgfsetfillcolor{currentfill}%
\pgfsetlinewidth{1.003750pt}%
\definecolor{currentstroke}{rgb}{0.000000,0.000000,0.000000}%
\pgfsetstrokecolor{currentstroke}%
\pgfsetdash{}{0pt}%
\pgfpathmoveto{\pgfqpoint{3.265169in}{2.334333in}}%
\pgfpathcurveto{\pgfqpoint{3.276219in}{2.334333in}}{\pgfqpoint{3.286818in}{2.338724in}}{\pgfqpoint{3.294632in}{2.346537in}}%
\pgfpathcurveto{\pgfqpoint{3.302446in}{2.354351in}}{\pgfqpoint{3.306836in}{2.364950in}}{\pgfqpoint{3.306836in}{2.376000in}}%
\pgfpathcurveto{\pgfqpoint{3.306836in}{2.387050in}}{\pgfqpoint{3.302446in}{2.397649in}}{\pgfqpoint{3.294632in}{2.405463in}}%
\pgfpathcurveto{\pgfqpoint{3.286818in}{2.413276in}}{\pgfqpoint{3.276219in}{2.417667in}}{\pgfqpoint{3.265169in}{2.417667in}}%
\pgfpathcurveto{\pgfqpoint{3.254119in}{2.417667in}}{\pgfqpoint{3.243520in}{2.413276in}}{\pgfqpoint{3.235707in}{2.405463in}}%
\pgfpathcurveto{\pgfqpoint{3.227893in}{2.397649in}}{\pgfqpoint{3.223503in}{2.387050in}}{\pgfqpoint{3.223503in}{2.376000in}}%
\pgfpathcurveto{\pgfqpoint{3.223503in}{2.364950in}}{\pgfqpoint{3.227893in}{2.354351in}}{\pgfqpoint{3.235707in}{2.346537in}}%
\pgfpathcurveto{\pgfqpoint{3.243520in}{2.338724in}}{\pgfqpoint{3.254119in}{2.334333in}}{\pgfqpoint{3.265169in}{2.334333in}}%
\pgfpathclose%
\pgfusepath{stroke,fill}%
\end{pgfscope}%
\begin{pgfscope}%
\pgfpathrectangle{\pgfqpoint{0.800000in}{0.528000in}}{\pgfqpoint{4.960000in}{3.696000in}}%
\pgfusepath{clip}%
\pgfsetbuttcap%
\pgfsetroundjoin%
\definecolor{currentfill}{rgb}{0.000000,0.000000,0.000000}%
\pgfsetfillcolor{currentfill}%
\pgfsetlinewidth{1.003750pt}%
\definecolor{currentstroke}{rgb}{0.000000,0.000000,0.000000}%
\pgfsetstrokecolor{currentstroke}%
\pgfsetdash{}{0pt}%
\pgfpathmoveto{\pgfqpoint{3.265169in}{2.334333in}}%
\pgfpathcurveto{\pgfqpoint{3.276219in}{2.334333in}}{\pgfqpoint{3.286818in}{2.338724in}}{\pgfqpoint{3.294632in}{2.346537in}}%
\pgfpathcurveto{\pgfqpoint{3.302446in}{2.354351in}}{\pgfqpoint{3.306836in}{2.364950in}}{\pgfqpoint{3.306836in}{2.376000in}}%
\pgfpathcurveto{\pgfqpoint{3.306836in}{2.387050in}}{\pgfqpoint{3.302446in}{2.397649in}}{\pgfqpoint{3.294632in}{2.405463in}}%
\pgfpathcurveto{\pgfqpoint{3.286818in}{2.413276in}}{\pgfqpoint{3.276219in}{2.417667in}}{\pgfqpoint{3.265169in}{2.417667in}}%
\pgfpathcurveto{\pgfqpoint{3.254119in}{2.417667in}}{\pgfqpoint{3.243520in}{2.413276in}}{\pgfqpoint{3.235707in}{2.405463in}}%
\pgfpathcurveto{\pgfqpoint{3.227893in}{2.397649in}}{\pgfqpoint{3.223503in}{2.387050in}}{\pgfqpoint{3.223503in}{2.376000in}}%
\pgfpathcurveto{\pgfqpoint{3.223503in}{2.364950in}}{\pgfqpoint{3.227893in}{2.354351in}}{\pgfqpoint{3.235707in}{2.346537in}}%
\pgfpathcurveto{\pgfqpoint{3.243520in}{2.338724in}}{\pgfqpoint{3.254119in}{2.334333in}}{\pgfqpoint{3.265169in}{2.334333in}}%
\pgfpathclose%
\pgfusepath{stroke,fill}%
\end{pgfscope}%
\begin{pgfscope}%
\pgfpathrectangle{\pgfqpoint{0.800000in}{0.528000in}}{\pgfqpoint{4.960000in}{3.696000in}}%
\pgfusepath{clip}%
\pgfsetbuttcap%
\pgfsetroundjoin%
\definecolor{currentfill}{rgb}{0.000000,0.000000,0.000000}%
\pgfsetfillcolor{currentfill}%
\pgfsetlinewidth{1.003750pt}%
\definecolor{currentstroke}{rgb}{0.000000,0.000000,0.000000}%
\pgfsetstrokecolor{currentstroke}%
\pgfsetdash{}{0pt}%
\pgfpathmoveto{\pgfqpoint{3.265169in}{2.334333in}}%
\pgfpathcurveto{\pgfqpoint{3.276219in}{2.334333in}}{\pgfqpoint{3.286818in}{2.338724in}}{\pgfqpoint{3.294632in}{2.346537in}}%
\pgfpathcurveto{\pgfqpoint{3.302446in}{2.354351in}}{\pgfqpoint{3.306836in}{2.364950in}}{\pgfqpoint{3.306836in}{2.376000in}}%
\pgfpathcurveto{\pgfqpoint{3.306836in}{2.387050in}}{\pgfqpoint{3.302446in}{2.397649in}}{\pgfqpoint{3.294632in}{2.405463in}}%
\pgfpathcurveto{\pgfqpoint{3.286818in}{2.413276in}}{\pgfqpoint{3.276219in}{2.417667in}}{\pgfqpoint{3.265169in}{2.417667in}}%
\pgfpathcurveto{\pgfqpoint{3.254119in}{2.417667in}}{\pgfqpoint{3.243520in}{2.413276in}}{\pgfqpoint{3.235707in}{2.405463in}}%
\pgfpathcurveto{\pgfqpoint{3.227893in}{2.397649in}}{\pgfqpoint{3.223503in}{2.387050in}}{\pgfqpoint{3.223503in}{2.376000in}}%
\pgfpathcurveto{\pgfqpoint{3.223503in}{2.364950in}}{\pgfqpoint{3.227893in}{2.354351in}}{\pgfqpoint{3.235707in}{2.346537in}}%
\pgfpathcurveto{\pgfqpoint{3.243520in}{2.338724in}}{\pgfqpoint{3.254119in}{2.334333in}}{\pgfqpoint{3.265169in}{2.334333in}}%
\pgfpathclose%
\pgfusepath{stroke,fill}%
\end{pgfscope}%
\begin{pgfscope}%
\pgfpathrectangle{\pgfqpoint{0.800000in}{0.528000in}}{\pgfqpoint{4.960000in}{3.696000in}}%
\pgfusepath{clip}%
\pgfsetbuttcap%
\pgfsetroundjoin%
\definecolor{currentfill}{rgb}{0.000000,0.000000,0.000000}%
\pgfsetfillcolor{currentfill}%
\pgfsetlinewidth{1.003750pt}%
\definecolor{currentstroke}{rgb}{0.000000,0.000000,0.000000}%
\pgfsetstrokecolor{currentstroke}%
\pgfsetdash{}{0pt}%
\pgfpathmoveto{\pgfqpoint{3.265169in}{2.334333in}}%
\pgfpathcurveto{\pgfqpoint{3.276219in}{2.334333in}}{\pgfqpoint{3.286818in}{2.338724in}}{\pgfqpoint{3.294632in}{2.346537in}}%
\pgfpathcurveto{\pgfqpoint{3.302446in}{2.354351in}}{\pgfqpoint{3.306836in}{2.364950in}}{\pgfqpoint{3.306836in}{2.376000in}}%
\pgfpathcurveto{\pgfqpoint{3.306836in}{2.387050in}}{\pgfqpoint{3.302446in}{2.397649in}}{\pgfqpoint{3.294632in}{2.405463in}}%
\pgfpathcurveto{\pgfqpoint{3.286818in}{2.413276in}}{\pgfqpoint{3.276219in}{2.417667in}}{\pgfqpoint{3.265169in}{2.417667in}}%
\pgfpathcurveto{\pgfqpoint{3.254119in}{2.417667in}}{\pgfqpoint{3.243520in}{2.413276in}}{\pgfqpoint{3.235707in}{2.405463in}}%
\pgfpathcurveto{\pgfqpoint{3.227893in}{2.397649in}}{\pgfqpoint{3.223503in}{2.387050in}}{\pgfqpoint{3.223503in}{2.376000in}}%
\pgfpathcurveto{\pgfqpoint{3.223503in}{2.364950in}}{\pgfqpoint{3.227893in}{2.354351in}}{\pgfqpoint{3.235707in}{2.346537in}}%
\pgfpathcurveto{\pgfqpoint{3.243520in}{2.338724in}}{\pgfqpoint{3.254119in}{2.334333in}}{\pgfqpoint{3.265169in}{2.334333in}}%
\pgfpathclose%
\pgfusepath{stroke,fill}%
\end{pgfscope}%
\begin{pgfscope}%
\pgfpathrectangle{\pgfqpoint{0.800000in}{0.528000in}}{\pgfqpoint{4.960000in}{3.696000in}}%
\pgfusepath{clip}%
\pgfsetbuttcap%
\pgfsetroundjoin%
\definecolor{currentfill}{rgb}{0.000000,0.000000,0.000000}%
\pgfsetfillcolor{currentfill}%
\pgfsetlinewidth{1.003750pt}%
\definecolor{currentstroke}{rgb}{0.000000,0.000000,0.000000}%
\pgfsetstrokecolor{currentstroke}%
\pgfsetdash{}{0pt}%
\pgfpathmoveto{\pgfqpoint{3.265169in}{2.334333in}}%
\pgfpathcurveto{\pgfqpoint{3.276219in}{2.334333in}}{\pgfqpoint{3.286818in}{2.338724in}}{\pgfqpoint{3.294632in}{2.346537in}}%
\pgfpathcurveto{\pgfqpoint{3.302446in}{2.354351in}}{\pgfqpoint{3.306836in}{2.364950in}}{\pgfqpoint{3.306836in}{2.376000in}}%
\pgfpathcurveto{\pgfqpoint{3.306836in}{2.387050in}}{\pgfqpoint{3.302446in}{2.397649in}}{\pgfqpoint{3.294632in}{2.405463in}}%
\pgfpathcurveto{\pgfqpoint{3.286818in}{2.413276in}}{\pgfqpoint{3.276219in}{2.417667in}}{\pgfqpoint{3.265169in}{2.417667in}}%
\pgfpathcurveto{\pgfqpoint{3.254119in}{2.417667in}}{\pgfqpoint{3.243520in}{2.413276in}}{\pgfqpoint{3.235707in}{2.405463in}}%
\pgfpathcurveto{\pgfqpoint{3.227893in}{2.397649in}}{\pgfqpoint{3.223503in}{2.387050in}}{\pgfqpoint{3.223503in}{2.376000in}}%
\pgfpathcurveto{\pgfqpoint{3.223503in}{2.364950in}}{\pgfqpoint{3.227893in}{2.354351in}}{\pgfqpoint{3.235707in}{2.346537in}}%
\pgfpathcurveto{\pgfqpoint{3.243520in}{2.338724in}}{\pgfqpoint{3.254119in}{2.334333in}}{\pgfqpoint{3.265169in}{2.334333in}}%
\pgfpathclose%
\pgfusepath{stroke,fill}%
\end{pgfscope}%
\begin{pgfscope}%
\pgfpathrectangle{\pgfqpoint{0.800000in}{0.528000in}}{\pgfqpoint{4.960000in}{3.696000in}}%
\pgfusepath{clip}%
\pgfsetbuttcap%
\pgfsetroundjoin%
\definecolor{currentfill}{rgb}{0.000000,0.000000,0.000000}%
\pgfsetfillcolor{currentfill}%
\pgfsetlinewidth{1.003750pt}%
\definecolor{currentstroke}{rgb}{0.000000,0.000000,0.000000}%
\pgfsetstrokecolor{currentstroke}%
\pgfsetdash{}{0pt}%
\pgfpathmoveto{\pgfqpoint{3.265169in}{2.334333in}}%
\pgfpathcurveto{\pgfqpoint{3.276219in}{2.334333in}}{\pgfqpoint{3.286818in}{2.338724in}}{\pgfqpoint{3.294632in}{2.346537in}}%
\pgfpathcurveto{\pgfqpoint{3.302446in}{2.354351in}}{\pgfqpoint{3.306836in}{2.364950in}}{\pgfqpoint{3.306836in}{2.376000in}}%
\pgfpathcurveto{\pgfqpoint{3.306836in}{2.387050in}}{\pgfqpoint{3.302446in}{2.397649in}}{\pgfqpoint{3.294632in}{2.405463in}}%
\pgfpathcurveto{\pgfqpoint{3.286818in}{2.413276in}}{\pgfqpoint{3.276219in}{2.417667in}}{\pgfqpoint{3.265169in}{2.417667in}}%
\pgfpathcurveto{\pgfqpoint{3.254119in}{2.417667in}}{\pgfqpoint{3.243520in}{2.413276in}}{\pgfqpoint{3.235707in}{2.405463in}}%
\pgfpathcurveto{\pgfqpoint{3.227893in}{2.397649in}}{\pgfqpoint{3.223503in}{2.387050in}}{\pgfqpoint{3.223503in}{2.376000in}}%
\pgfpathcurveto{\pgfqpoint{3.223503in}{2.364950in}}{\pgfqpoint{3.227893in}{2.354351in}}{\pgfqpoint{3.235707in}{2.346537in}}%
\pgfpathcurveto{\pgfqpoint{3.243520in}{2.338724in}}{\pgfqpoint{3.254119in}{2.334333in}}{\pgfqpoint{3.265169in}{2.334333in}}%
\pgfpathclose%
\pgfusepath{stroke,fill}%
\end{pgfscope}%
\begin{pgfscope}%
\pgfpathrectangle{\pgfqpoint{0.800000in}{0.528000in}}{\pgfqpoint{4.960000in}{3.696000in}}%
\pgfusepath{clip}%
\pgfsetbuttcap%
\pgfsetroundjoin%
\definecolor{currentfill}{rgb}{0.000000,0.000000,0.000000}%
\pgfsetfillcolor{currentfill}%
\pgfsetlinewidth{1.003750pt}%
\definecolor{currentstroke}{rgb}{0.000000,0.000000,0.000000}%
\pgfsetstrokecolor{currentstroke}%
\pgfsetdash{}{0pt}%
\pgfpathmoveto{\pgfqpoint{3.265169in}{2.334333in}}%
\pgfpathcurveto{\pgfqpoint{3.276219in}{2.334333in}}{\pgfqpoint{3.286818in}{2.338724in}}{\pgfqpoint{3.294632in}{2.346537in}}%
\pgfpathcurveto{\pgfqpoint{3.302446in}{2.354351in}}{\pgfqpoint{3.306836in}{2.364950in}}{\pgfqpoint{3.306836in}{2.376000in}}%
\pgfpathcurveto{\pgfqpoint{3.306836in}{2.387050in}}{\pgfqpoint{3.302446in}{2.397649in}}{\pgfqpoint{3.294632in}{2.405463in}}%
\pgfpathcurveto{\pgfqpoint{3.286818in}{2.413276in}}{\pgfqpoint{3.276219in}{2.417667in}}{\pgfqpoint{3.265169in}{2.417667in}}%
\pgfpathcurveto{\pgfqpoint{3.254119in}{2.417667in}}{\pgfqpoint{3.243520in}{2.413276in}}{\pgfqpoint{3.235707in}{2.405463in}}%
\pgfpathcurveto{\pgfqpoint{3.227893in}{2.397649in}}{\pgfqpoint{3.223503in}{2.387050in}}{\pgfqpoint{3.223503in}{2.376000in}}%
\pgfpathcurveto{\pgfqpoint{3.223503in}{2.364950in}}{\pgfqpoint{3.227893in}{2.354351in}}{\pgfqpoint{3.235707in}{2.346537in}}%
\pgfpathcurveto{\pgfqpoint{3.243520in}{2.338724in}}{\pgfqpoint{3.254119in}{2.334333in}}{\pgfqpoint{3.265169in}{2.334333in}}%
\pgfpathclose%
\pgfusepath{stroke,fill}%
\end{pgfscope}%
\begin{pgfscope}%
\pgfpathrectangle{\pgfqpoint{0.800000in}{0.528000in}}{\pgfqpoint{4.960000in}{3.696000in}}%
\pgfusepath{clip}%
\pgfsetbuttcap%
\pgfsetroundjoin%
\definecolor{currentfill}{rgb}{0.000000,0.000000,0.000000}%
\pgfsetfillcolor{currentfill}%
\pgfsetlinewidth{1.003750pt}%
\definecolor{currentstroke}{rgb}{0.000000,0.000000,0.000000}%
\pgfsetstrokecolor{currentstroke}%
\pgfsetdash{}{0pt}%
\pgfpathmoveto{\pgfqpoint{3.265169in}{2.334333in}}%
\pgfpathcurveto{\pgfqpoint{3.276219in}{2.334333in}}{\pgfqpoint{3.286818in}{2.338724in}}{\pgfqpoint{3.294632in}{2.346537in}}%
\pgfpathcurveto{\pgfqpoint{3.302446in}{2.354351in}}{\pgfqpoint{3.306836in}{2.364950in}}{\pgfqpoint{3.306836in}{2.376000in}}%
\pgfpathcurveto{\pgfqpoint{3.306836in}{2.387050in}}{\pgfqpoint{3.302446in}{2.397649in}}{\pgfqpoint{3.294632in}{2.405463in}}%
\pgfpathcurveto{\pgfqpoint{3.286818in}{2.413276in}}{\pgfqpoint{3.276219in}{2.417667in}}{\pgfqpoint{3.265169in}{2.417667in}}%
\pgfpathcurveto{\pgfqpoint{3.254119in}{2.417667in}}{\pgfqpoint{3.243520in}{2.413276in}}{\pgfqpoint{3.235707in}{2.405463in}}%
\pgfpathcurveto{\pgfqpoint{3.227893in}{2.397649in}}{\pgfqpoint{3.223503in}{2.387050in}}{\pgfqpoint{3.223503in}{2.376000in}}%
\pgfpathcurveto{\pgfqpoint{3.223503in}{2.364950in}}{\pgfqpoint{3.227893in}{2.354351in}}{\pgfqpoint{3.235707in}{2.346537in}}%
\pgfpathcurveto{\pgfqpoint{3.243520in}{2.338724in}}{\pgfqpoint{3.254119in}{2.334333in}}{\pgfqpoint{3.265169in}{2.334333in}}%
\pgfpathclose%
\pgfusepath{stroke,fill}%
\end{pgfscope}%
\begin{pgfscope}%
\pgfpathrectangle{\pgfqpoint{0.800000in}{0.528000in}}{\pgfqpoint{4.960000in}{3.696000in}}%
\pgfusepath{clip}%
\pgfsetbuttcap%
\pgfsetroundjoin%
\definecolor{currentfill}{rgb}{0.000000,0.000000,0.000000}%
\pgfsetfillcolor{currentfill}%
\pgfsetlinewidth{1.003750pt}%
\definecolor{currentstroke}{rgb}{0.000000,0.000000,0.000000}%
\pgfsetstrokecolor{currentstroke}%
\pgfsetdash{}{0pt}%
\pgfpathmoveto{\pgfqpoint{3.265169in}{2.334333in}}%
\pgfpathcurveto{\pgfqpoint{3.276219in}{2.334333in}}{\pgfqpoint{3.286818in}{2.338724in}}{\pgfqpoint{3.294632in}{2.346537in}}%
\pgfpathcurveto{\pgfqpoint{3.302446in}{2.354351in}}{\pgfqpoint{3.306836in}{2.364950in}}{\pgfqpoint{3.306836in}{2.376000in}}%
\pgfpathcurveto{\pgfqpoint{3.306836in}{2.387050in}}{\pgfqpoint{3.302446in}{2.397649in}}{\pgfqpoint{3.294632in}{2.405463in}}%
\pgfpathcurveto{\pgfqpoint{3.286818in}{2.413276in}}{\pgfqpoint{3.276219in}{2.417667in}}{\pgfqpoint{3.265169in}{2.417667in}}%
\pgfpathcurveto{\pgfqpoint{3.254119in}{2.417667in}}{\pgfqpoint{3.243520in}{2.413276in}}{\pgfqpoint{3.235707in}{2.405463in}}%
\pgfpathcurveto{\pgfqpoint{3.227893in}{2.397649in}}{\pgfqpoint{3.223503in}{2.387050in}}{\pgfqpoint{3.223503in}{2.376000in}}%
\pgfpathcurveto{\pgfqpoint{3.223503in}{2.364950in}}{\pgfqpoint{3.227893in}{2.354351in}}{\pgfqpoint{3.235707in}{2.346537in}}%
\pgfpathcurveto{\pgfqpoint{3.243520in}{2.338724in}}{\pgfqpoint{3.254119in}{2.334333in}}{\pgfqpoint{3.265169in}{2.334333in}}%
\pgfpathclose%
\pgfusepath{stroke,fill}%
\end{pgfscope}%
\begin{pgfscope}%
\pgfpathrectangle{\pgfqpoint{0.800000in}{0.528000in}}{\pgfqpoint{4.960000in}{3.696000in}}%
\pgfusepath{clip}%
\pgfsetbuttcap%
\pgfsetroundjoin%
\definecolor{currentfill}{rgb}{0.000000,0.000000,0.000000}%
\pgfsetfillcolor{currentfill}%
\pgfsetlinewidth{1.003750pt}%
\definecolor{currentstroke}{rgb}{0.000000,0.000000,0.000000}%
\pgfsetstrokecolor{currentstroke}%
\pgfsetdash{}{0pt}%
\pgfpathmoveto{\pgfqpoint{3.265169in}{2.334333in}}%
\pgfpathcurveto{\pgfqpoint{3.276219in}{2.334333in}}{\pgfqpoint{3.286818in}{2.338724in}}{\pgfqpoint{3.294632in}{2.346537in}}%
\pgfpathcurveto{\pgfqpoint{3.302446in}{2.354351in}}{\pgfqpoint{3.306836in}{2.364950in}}{\pgfqpoint{3.306836in}{2.376000in}}%
\pgfpathcurveto{\pgfqpoint{3.306836in}{2.387050in}}{\pgfqpoint{3.302446in}{2.397649in}}{\pgfqpoint{3.294632in}{2.405463in}}%
\pgfpathcurveto{\pgfqpoint{3.286818in}{2.413276in}}{\pgfqpoint{3.276219in}{2.417667in}}{\pgfqpoint{3.265169in}{2.417667in}}%
\pgfpathcurveto{\pgfqpoint{3.254119in}{2.417667in}}{\pgfqpoint{3.243520in}{2.413276in}}{\pgfqpoint{3.235707in}{2.405463in}}%
\pgfpathcurveto{\pgfqpoint{3.227893in}{2.397649in}}{\pgfqpoint{3.223503in}{2.387050in}}{\pgfqpoint{3.223503in}{2.376000in}}%
\pgfpathcurveto{\pgfqpoint{3.223503in}{2.364950in}}{\pgfqpoint{3.227893in}{2.354351in}}{\pgfqpoint{3.235707in}{2.346537in}}%
\pgfpathcurveto{\pgfqpoint{3.243520in}{2.338724in}}{\pgfqpoint{3.254119in}{2.334333in}}{\pgfqpoint{3.265169in}{2.334333in}}%
\pgfpathclose%
\pgfusepath{stroke,fill}%
\end{pgfscope}%
\begin{pgfscope}%
\pgfpathrectangle{\pgfqpoint{0.800000in}{0.528000in}}{\pgfqpoint{4.960000in}{3.696000in}}%
\pgfusepath{clip}%
\pgfsetbuttcap%
\pgfsetroundjoin%
\definecolor{currentfill}{rgb}{0.000000,0.000000,0.000000}%
\pgfsetfillcolor{currentfill}%
\pgfsetlinewidth{1.003750pt}%
\definecolor{currentstroke}{rgb}{0.000000,0.000000,0.000000}%
\pgfsetstrokecolor{currentstroke}%
\pgfsetdash{}{0pt}%
\pgfpathmoveto{\pgfqpoint{3.265169in}{2.334333in}}%
\pgfpathcurveto{\pgfqpoint{3.276219in}{2.334333in}}{\pgfqpoint{3.286818in}{2.338724in}}{\pgfqpoint{3.294632in}{2.346537in}}%
\pgfpathcurveto{\pgfqpoint{3.302446in}{2.354351in}}{\pgfqpoint{3.306836in}{2.364950in}}{\pgfqpoint{3.306836in}{2.376000in}}%
\pgfpathcurveto{\pgfqpoint{3.306836in}{2.387050in}}{\pgfqpoint{3.302446in}{2.397649in}}{\pgfqpoint{3.294632in}{2.405463in}}%
\pgfpathcurveto{\pgfqpoint{3.286818in}{2.413276in}}{\pgfqpoint{3.276219in}{2.417667in}}{\pgfqpoint{3.265169in}{2.417667in}}%
\pgfpathcurveto{\pgfqpoint{3.254119in}{2.417667in}}{\pgfqpoint{3.243520in}{2.413276in}}{\pgfqpoint{3.235707in}{2.405463in}}%
\pgfpathcurveto{\pgfqpoint{3.227893in}{2.397649in}}{\pgfqpoint{3.223503in}{2.387050in}}{\pgfqpoint{3.223503in}{2.376000in}}%
\pgfpathcurveto{\pgfqpoint{3.223503in}{2.364950in}}{\pgfqpoint{3.227893in}{2.354351in}}{\pgfqpoint{3.235707in}{2.346537in}}%
\pgfpathcurveto{\pgfqpoint{3.243520in}{2.338724in}}{\pgfqpoint{3.254119in}{2.334333in}}{\pgfqpoint{3.265169in}{2.334333in}}%
\pgfpathclose%
\pgfusepath{stroke,fill}%
\end{pgfscope}%
\begin{pgfscope}%
\pgfpathrectangle{\pgfqpoint{0.800000in}{0.528000in}}{\pgfqpoint{4.960000in}{3.696000in}}%
\pgfusepath{clip}%
\pgfsetbuttcap%
\pgfsetroundjoin%
\definecolor{currentfill}{rgb}{0.000000,0.000000,0.000000}%
\pgfsetfillcolor{currentfill}%
\pgfsetlinewidth{1.003750pt}%
\definecolor{currentstroke}{rgb}{0.000000,0.000000,0.000000}%
\pgfsetstrokecolor{currentstroke}%
\pgfsetdash{}{0pt}%
\pgfpathmoveto{\pgfqpoint{4.384857in}{2.334333in}}%
\pgfpathcurveto{\pgfqpoint{4.395908in}{2.334333in}}{\pgfqpoint{4.406507in}{2.338724in}}{\pgfqpoint{4.414320in}{2.346537in}}%
\pgfpathcurveto{\pgfqpoint{4.422134in}{2.354351in}}{\pgfqpoint{4.426524in}{2.364950in}}{\pgfqpoint{4.426524in}{2.376000in}}%
\pgfpathcurveto{\pgfqpoint{4.426524in}{2.387050in}}{\pgfqpoint{4.422134in}{2.397649in}}{\pgfqpoint{4.414320in}{2.405463in}}%
\pgfpathcurveto{\pgfqpoint{4.406507in}{2.413276in}}{\pgfqpoint{4.395908in}{2.417667in}}{\pgfqpoint{4.384857in}{2.417667in}}%
\pgfpathcurveto{\pgfqpoint{4.373807in}{2.417667in}}{\pgfqpoint{4.363208in}{2.413276in}}{\pgfqpoint{4.355395in}{2.405463in}}%
\pgfpathcurveto{\pgfqpoint{4.347581in}{2.397649in}}{\pgfqpoint{4.343191in}{2.387050in}}{\pgfqpoint{4.343191in}{2.376000in}}%
\pgfpathcurveto{\pgfqpoint{4.343191in}{2.364950in}}{\pgfqpoint{4.347581in}{2.354351in}}{\pgfqpoint{4.355395in}{2.346537in}}%
\pgfpathcurveto{\pgfqpoint{4.363208in}{2.338724in}}{\pgfqpoint{4.373807in}{2.334333in}}{\pgfqpoint{4.384857in}{2.334333in}}%
\pgfpathclose%
\pgfusepath{stroke,fill}%
\end{pgfscope}%
\begin{pgfscope}%
\pgfpathrectangle{\pgfqpoint{0.800000in}{0.528000in}}{\pgfqpoint{4.960000in}{3.696000in}}%
\pgfusepath{clip}%
\pgfsetbuttcap%
\pgfsetroundjoin%
\definecolor{currentfill}{rgb}{0.000000,0.000000,0.000000}%
\pgfsetfillcolor{currentfill}%
\pgfsetlinewidth{1.003750pt}%
\definecolor{currentstroke}{rgb}{0.000000,0.000000,0.000000}%
\pgfsetstrokecolor{currentstroke}%
\pgfsetdash{}{0pt}%
\pgfpathmoveto{\pgfqpoint{4.384857in}{2.334333in}}%
\pgfpathcurveto{\pgfqpoint{4.395908in}{2.334333in}}{\pgfqpoint{4.406507in}{2.338724in}}{\pgfqpoint{4.414320in}{2.346537in}}%
\pgfpathcurveto{\pgfqpoint{4.422134in}{2.354351in}}{\pgfqpoint{4.426524in}{2.364950in}}{\pgfqpoint{4.426524in}{2.376000in}}%
\pgfpathcurveto{\pgfqpoint{4.426524in}{2.387050in}}{\pgfqpoint{4.422134in}{2.397649in}}{\pgfqpoint{4.414320in}{2.405463in}}%
\pgfpathcurveto{\pgfqpoint{4.406507in}{2.413276in}}{\pgfqpoint{4.395908in}{2.417667in}}{\pgfqpoint{4.384857in}{2.417667in}}%
\pgfpathcurveto{\pgfqpoint{4.373807in}{2.417667in}}{\pgfqpoint{4.363208in}{2.413276in}}{\pgfqpoint{4.355395in}{2.405463in}}%
\pgfpathcurveto{\pgfqpoint{4.347581in}{2.397649in}}{\pgfqpoint{4.343191in}{2.387050in}}{\pgfqpoint{4.343191in}{2.376000in}}%
\pgfpathcurveto{\pgfqpoint{4.343191in}{2.364950in}}{\pgfqpoint{4.347581in}{2.354351in}}{\pgfqpoint{4.355395in}{2.346537in}}%
\pgfpathcurveto{\pgfqpoint{4.363208in}{2.338724in}}{\pgfqpoint{4.373807in}{2.334333in}}{\pgfqpoint{4.384857in}{2.334333in}}%
\pgfpathclose%
\pgfusepath{stroke,fill}%
\end{pgfscope}%
\begin{pgfscope}%
\pgfpathrectangle{\pgfqpoint{0.800000in}{0.528000in}}{\pgfqpoint{4.960000in}{3.696000in}}%
\pgfusepath{clip}%
\pgfsetbuttcap%
\pgfsetroundjoin%
\definecolor{currentfill}{rgb}{0.000000,0.000000,0.000000}%
\pgfsetfillcolor{currentfill}%
\pgfsetlinewidth{1.003750pt}%
\definecolor{currentstroke}{rgb}{0.000000,0.000000,0.000000}%
\pgfsetstrokecolor{currentstroke}%
\pgfsetdash{}{0pt}%
\pgfpathmoveto{\pgfqpoint{4.384857in}{2.334333in}}%
\pgfpathcurveto{\pgfqpoint{4.395908in}{2.334333in}}{\pgfqpoint{4.406507in}{2.338724in}}{\pgfqpoint{4.414320in}{2.346537in}}%
\pgfpathcurveto{\pgfqpoint{4.422134in}{2.354351in}}{\pgfqpoint{4.426524in}{2.364950in}}{\pgfqpoint{4.426524in}{2.376000in}}%
\pgfpathcurveto{\pgfqpoint{4.426524in}{2.387050in}}{\pgfqpoint{4.422134in}{2.397649in}}{\pgfqpoint{4.414320in}{2.405463in}}%
\pgfpathcurveto{\pgfqpoint{4.406507in}{2.413276in}}{\pgfqpoint{4.395908in}{2.417667in}}{\pgfqpoint{4.384857in}{2.417667in}}%
\pgfpathcurveto{\pgfqpoint{4.373807in}{2.417667in}}{\pgfqpoint{4.363208in}{2.413276in}}{\pgfqpoint{4.355395in}{2.405463in}}%
\pgfpathcurveto{\pgfqpoint{4.347581in}{2.397649in}}{\pgfqpoint{4.343191in}{2.387050in}}{\pgfqpoint{4.343191in}{2.376000in}}%
\pgfpathcurveto{\pgfqpoint{4.343191in}{2.364950in}}{\pgfqpoint{4.347581in}{2.354351in}}{\pgfqpoint{4.355395in}{2.346537in}}%
\pgfpathcurveto{\pgfqpoint{4.363208in}{2.338724in}}{\pgfqpoint{4.373807in}{2.334333in}}{\pgfqpoint{4.384857in}{2.334333in}}%
\pgfpathclose%
\pgfusepath{stroke,fill}%
\end{pgfscope}%
\begin{pgfscope}%
\pgfpathrectangle{\pgfqpoint{0.800000in}{0.528000in}}{\pgfqpoint{4.960000in}{3.696000in}}%
\pgfusepath{clip}%
\pgfsetbuttcap%
\pgfsetroundjoin%
\definecolor{currentfill}{rgb}{0.000000,0.000000,0.000000}%
\pgfsetfillcolor{currentfill}%
\pgfsetlinewidth{1.003750pt}%
\definecolor{currentstroke}{rgb}{0.000000,0.000000,0.000000}%
\pgfsetstrokecolor{currentstroke}%
\pgfsetdash{}{0pt}%
\pgfpathmoveto{\pgfqpoint{4.384857in}{2.334333in}}%
\pgfpathcurveto{\pgfqpoint{4.395908in}{2.334333in}}{\pgfqpoint{4.406507in}{2.338724in}}{\pgfqpoint{4.414320in}{2.346537in}}%
\pgfpathcurveto{\pgfqpoint{4.422134in}{2.354351in}}{\pgfqpoint{4.426524in}{2.364950in}}{\pgfqpoint{4.426524in}{2.376000in}}%
\pgfpathcurveto{\pgfqpoint{4.426524in}{2.387050in}}{\pgfqpoint{4.422134in}{2.397649in}}{\pgfqpoint{4.414320in}{2.405463in}}%
\pgfpathcurveto{\pgfqpoint{4.406507in}{2.413276in}}{\pgfqpoint{4.395908in}{2.417667in}}{\pgfqpoint{4.384857in}{2.417667in}}%
\pgfpathcurveto{\pgfqpoint{4.373807in}{2.417667in}}{\pgfqpoint{4.363208in}{2.413276in}}{\pgfqpoint{4.355395in}{2.405463in}}%
\pgfpathcurveto{\pgfqpoint{4.347581in}{2.397649in}}{\pgfqpoint{4.343191in}{2.387050in}}{\pgfqpoint{4.343191in}{2.376000in}}%
\pgfpathcurveto{\pgfqpoint{4.343191in}{2.364950in}}{\pgfqpoint{4.347581in}{2.354351in}}{\pgfqpoint{4.355395in}{2.346537in}}%
\pgfpathcurveto{\pgfqpoint{4.363208in}{2.338724in}}{\pgfqpoint{4.373807in}{2.334333in}}{\pgfqpoint{4.384857in}{2.334333in}}%
\pgfpathclose%
\pgfusepath{stroke,fill}%
\end{pgfscope}%
\begin{pgfscope}%
\pgfpathrectangle{\pgfqpoint{0.800000in}{0.528000in}}{\pgfqpoint{4.960000in}{3.696000in}}%
\pgfusepath{clip}%
\pgfsetbuttcap%
\pgfsetroundjoin%
\definecolor{currentfill}{rgb}{0.000000,0.000000,0.000000}%
\pgfsetfillcolor{currentfill}%
\pgfsetlinewidth{1.003750pt}%
\definecolor{currentstroke}{rgb}{0.000000,0.000000,0.000000}%
\pgfsetstrokecolor{currentstroke}%
\pgfsetdash{}{0pt}%
\pgfpathmoveto{\pgfqpoint{4.384857in}{2.334333in}}%
\pgfpathcurveto{\pgfqpoint{4.395908in}{2.334333in}}{\pgfqpoint{4.406507in}{2.338724in}}{\pgfqpoint{4.414320in}{2.346537in}}%
\pgfpathcurveto{\pgfqpoint{4.422134in}{2.354351in}}{\pgfqpoint{4.426524in}{2.364950in}}{\pgfqpoint{4.426524in}{2.376000in}}%
\pgfpathcurveto{\pgfqpoint{4.426524in}{2.387050in}}{\pgfqpoint{4.422134in}{2.397649in}}{\pgfqpoint{4.414320in}{2.405463in}}%
\pgfpathcurveto{\pgfqpoint{4.406507in}{2.413276in}}{\pgfqpoint{4.395908in}{2.417667in}}{\pgfqpoint{4.384857in}{2.417667in}}%
\pgfpathcurveto{\pgfqpoint{4.373807in}{2.417667in}}{\pgfqpoint{4.363208in}{2.413276in}}{\pgfqpoint{4.355395in}{2.405463in}}%
\pgfpathcurveto{\pgfqpoint{4.347581in}{2.397649in}}{\pgfqpoint{4.343191in}{2.387050in}}{\pgfqpoint{4.343191in}{2.376000in}}%
\pgfpathcurveto{\pgfqpoint{4.343191in}{2.364950in}}{\pgfqpoint{4.347581in}{2.354351in}}{\pgfqpoint{4.355395in}{2.346537in}}%
\pgfpathcurveto{\pgfqpoint{4.363208in}{2.338724in}}{\pgfqpoint{4.373807in}{2.334333in}}{\pgfqpoint{4.384857in}{2.334333in}}%
\pgfpathclose%
\pgfusepath{stroke,fill}%
\end{pgfscope}%
\begin{pgfscope}%
\pgfpathrectangle{\pgfqpoint{0.800000in}{0.528000in}}{\pgfqpoint{4.960000in}{3.696000in}}%
\pgfusepath{clip}%
\pgfsetbuttcap%
\pgfsetroundjoin%
\definecolor{currentfill}{rgb}{0.000000,0.000000,0.000000}%
\pgfsetfillcolor{currentfill}%
\pgfsetlinewidth{1.003750pt}%
\definecolor{currentstroke}{rgb}{0.000000,0.000000,0.000000}%
\pgfsetstrokecolor{currentstroke}%
\pgfsetdash{}{0pt}%
\pgfpathmoveto{\pgfqpoint{4.384857in}{2.334333in}}%
\pgfpathcurveto{\pgfqpoint{4.395908in}{2.334333in}}{\pgfqpoint{4.406507in}{2.338724in}}{\pgfqpoint{4.414320in}{2.346537in}}%
\pgfpathcurveto{\pgfqpoint{4.422134in}{2.354351in}}{\pgfqpoint{4.426524in}{2.364950in}}{\pgfqpoint{4.426524in}{2.376000in}}%
\pgfpathcurveto{\pgfqpoint{4.426524in}{2.387050in}}{\pgfqpoint{4.422134in}{2.397649in}}{\pgfqpoint{4.414320in}{2.405463in}}%
\pgfpathcurveto{\pgfqpoint{4.406507in}{2.413276in}}{\pgfqpoint{4.395908in}{2.417667in}}{\pgfqpoint{4.384857in}{2.417667in}}%
\pgfpathcurveto{\pgfqpoint{4.373807in}{2.417667in}}{\pgfqpoint{4.363208in}{2.413276in}}{\pgfqpoint{4.355395in}{2.405463in}}%
\pgfpathcurveto{\pgfqpoint{4.347581in}{2.397649in}}{\pgfqpoint{4.343191in}{2.387050in}}{\pgfqpoint{4.343191in}{2.376000in}}%
\pgfpathcurveto{\pgfqpoint{4.343191in}{2.364950in}}{\pgfqpoint{4.347581in}{2.354351in}}{\pgfqpoint{4.355395in}{2.346537in}}%
\pgfpathcurveto{\pgfqpoint{4.363208in}{2.338724in}}{\pgfqpoint{4.373807in}{2.334333in}}{\pgfqpoint{4.384857in}{2.334333in}}%
\pgfpathclose%
\pgfusepath{stroke,fill}%
\end{pgfscope}%
\begin{pgfscope}%
\pgfpathrectangle{\pgfqpoint{0.800000in}{0.528000in}}{\pgfqpoint{4.960000in}{3.696000in}}%
\pgfusepath{clip}%
\pgfsetbuttcap%
\pgfsetroundjoin%
\definecolor{currentfill}{rgb}{0.000000,0.000000,0.000000}%
\pgfsetfillcolor{currentfill}%
\pgfsetlinewidth{1.003750pt}%
\definecolor{currentstroke}{rgb}{0.000000,0.000000,0.000000}%
\pgfsetstrokecolor{currentstroke}%
\pgfsetdash{}{0pt}%
\pgfpathmoveto{\pgfqpoint{4.384857in}{2.334333in}}%
\pgfpathcurveto{\pgfqpoint{4.395908in}{2.334333in}}{\pgfqpoint{4.406507in}{2.338724in}}{\pgfqpoint{4.414320in}{2.346537in}}%
\pgfpathcurveto{\pgfqpoint{4.422134in}{2.354351in}}{\pgfqpoint{4.426524in}{2.364950in}}{\pgfqpoint{4.426524in}{2.376000in}}%
\pgfpathcurveto{\pgfqpoint{4.426524in}{2.387050in}}{\pgfqpoint{4.422134in}{2.397649in}}{\pgfqpoint{4.414320in}{2.405463in}}%
\pgfpathcurveto{\pgfqpoint{4.406507in}{2.413276in}}{\pgfqpoint{4.395908in}{2.417667in}}{\pgfqpoint{4.384857in}{2.417667in}}%
\pgfpathcurveto{\pgfqpoint{4.373807in}{2.417667in}}{\pgfqpoint{4.363208in}{2.413276in}}{\pgfqpoint{4.355395in}{2.405463in}}%
\pgfpathcurveto{\pgfqpoint{4.347581in}{2.397649in}}{\pgfqpoint{4.343191in}{2.387050in}}{\pgfqpoint{4.343191in}{2.376000in}}%
\pgfpathcurveto{\pgfqpoint{4.343191in}{2.364950in}}{\pgfqpoint{4.347581in}{2.354351in}}{\pgfqpoint{4.355395in}{2.346537in}}%
\pgfpathcurveto{\pgfqpoint{4.363208in}{2.338724in}}{\pgfqpoint{4.373807in}{2.334333in}}{\pgfqpoint{4.384857in}{2.334333in}}%
\pgfpathclose%
\pgfusepath{stroke,fill}%
\end{pgfscope}%
\begin{pgfscope}%
\pgfpathrectangle{\pgfqpoint{0.800000in}{0.528000in}}{\pgfqpoint{4.960000in}{3.696000in}}%
\pgfusepath{clip}%
\pgfsetbuttcap%
\pgfsetroundjoin%
\definecolor{currentfill}{rgb}{0.000000,0.000000,0.000000}%
\pgfsetfillcolor{currentfill}%
\pgfsetlinewidth{1.003750pt}%
\definecolor{currentstroke}{rgb}{0.000000,0.000000,0.000000}%
\pgfsetstrokecolor{currentstroke}%
\pgfsetdash{}{0pt}%
\pgfpathmoveto{\pgfqpoint{4.384857in}{2.334333in}}%
\pgfpathcurveto{\pgfqpoint{4.395908in}{2.334333in}}{\pgfqpoint{4.406507in}{2.338724in}}{\pgfqpoint{4.414320in}{2.346537in}}%
\pgfpathcurveto{\pgfqpoint{4.422134in}{2.354351in}}{\pgfqpoint{4.426524in}{2.364950in}}{\pgfqpoint{4.426524in}{2.376000in}}%
\pgfpathcurveto{\pgfqpoint{4.426524in}{2.387050in}}{\pgfqpoint{4.422134in}{2.397649in}}{\pgfqpoint{4.414320in}{2.405463in}}%
\pgfpathcurveto{\pgfqpoint{4.406507in}{2.413276in}}{\pgfqpoint{4.395908in}{2.417667in}}{\pgfqpoint{4.384857in}{2.417667in}}%
\pgfpathcurveto{\pgfqpoint{4.373807in}{2.417667in}}{\pgfqpoint{4.363208in}{2.413276in}}{\pgfqpoint{4.355395in}{2.405463in}}%
\pgfpathcurveto{\pgfqpoint{4.347581in}{2.397649in}}{\pgfqpoint{4.343191in}{2.387050in}}{\pgfqpoint{4.343191in}{2.376000in}}%
\pgfpathcurveto{\pgfqpoint{4.343191in}{2.364950in}}{\pgfqpoint{4.347581in}{2.354351in}}{\pgfqpoint{4.355395in}{2.346537in}}%
\pgfpathcurveto{\pgfqpoint{4.363208in}{2.338724in}}{\pgfqpoint{4.373807in}{2.334333in}}{\pgfqpoint{4.384857in}{2.334333in}}%
\pgfpathclose%
\pgfusepath{stroke,fill}%
\end{pgfscope}%
\begin{pgfscope}%
\pgfpathrectangle{\pgfqpoint{0.800000in}{0.528000in}}{\pgfqpoint{4.960000in}{3.696000in}}%
\pgfusepath{clip}%
\pgfsetbuttcap%
\pgfsetroundjoin%
\definecolor{currentfill}{rgb}{0.000000,0.000000,0.000000}%
\pgfsetfillcolor{currentfill}%
\pgfsetlinewidth{1.003750pt}%
\definecolor{currentstroke}{rgb}{0.000000,0.000000,0.000000}%
\pgfsetstrokecolor{currentstroke}%
\pgfsetdash{}{0pt}%
\pgfpathmoveto{\pgfqpoint{4.384857in}{2.334333in}}%
\pgfpathcurveto{\pgfqpoint{4.395908in}{2.334333in}}{\pgfqpoint{4.406507in}{2.338724in}}{\pgfqpoint{4.414320in}{2.346537in}}%
\pgfpathcurveto{\pgfqpoint{4.422134in}{2.354351in}}{\pgfqpoint{4.426524in}{2.364950in}}{\pgfqpoint{4.426524in}{2.376000in}}%
\pgfpathcurveto{\pgfqpoint{4.426524in}{2.387050in}}{\pgfqpoint{4.422134in}{2.397649in}}{\pgfqpoint{4.414320in}{2.405463in}}%
\pgfpathcurveto{\pgfqpoint{4.406507in}{2.413276in}}{\pgfqpoint{4.395908in}{2.417667in}}{\pgfqpoint{4.384857in}{2.417667in}}%
\pgfpathcurveto{\pgfqpoint{4.373807in}{2.417667in}}{\pgfqpoint{4.363208in}{2.413276in}}{\pgfqpoint{4.355395in}{2.405463in}}%
\pgfpathcurveto{\pgfqpoint{4.347581in}{2.397649in}}{\pgfqpoint{4.343191in}{2.387050in}}{\pgfqpoint{4.343191in}{2.376000in}}%
\pgfpathcurveto{\pgfqpoint{4.343191in}{2.364950in}}{\pgfqpoint{4.347581in}{2.354351in}}{\pgfqpoint{4.355395in}{2.346537in}}%
\pgfpathcurveto{\pgfqpoint{4.363208in}{2.338724in}}{\pgfqpoint{4.373807in}{2.334333in}}{\pgfqpoint{4.384857in}{2.334333in}}%
\pgfpathclose%
\pgfusepath{stroke,fill}%
\end{pgfscope}%
\begin{pgfscope}%
\pgfpathrectangle{\pgfqpoint{0.800000in}{0.528000in}}{\pgfqpoint{4.960000in}{3.696000in}}%
\pgfusepath{clip}%
\pgfsetbuttcap%
\pgfsetroundjoin%
\definecolor{currentfill}{rgb}{0.000000,0.000000,0.000000}%
\pgfsetfillcolor{currentfill}%
\pgfsetlinewidth{1.003750pt}%
\definecolor{currentstroke}{rgb}{0.000000,0.000000,0.000000}%
\pgfsetstrokecolor{currentstroke}%
\pgfsetdash{}{0pt}%
\pgfpathmoveto{\pgfqpoint{4.384857in}{2.334333in}}%
\pgfpathcurveto{\pgfqpoint{4.395908in}{2.334333in}}{\pgfqpoint{4.406507in}{2.338724in}}{\pgfqpoint{4.414320in}{2.346537in}}%
\pgfpathcurveto{\pgfqpoint{4.422134in}{2.354351in}}{\pgfqpoint{4.426524in}{2.364950in}}{\pgfqpoint{4.426524in}{2.376000in}}%
\pgfpathcurveto{\pgfqpoint{4.426524in}{2.387050in}}{\pgfqpoint{4.422134in}{2.397649in}}{\pgfqpoint{4.414320in}{2.405463in}}%
\pgfpathcurveto{\pgfqpoint{4.406507in}{2.413276in}}{\pgfqpoint{4.395908in}{2.417667in}}{\pgfqpoint{4.384857in}{2.417667in}}%
\pgfpathcurveto{\pgfqpoint{4.373807in}{2.417667in}}{\pgfqpoint{4.363208in}{2.413276in}}{\pgfqpoint{4.355395in}{2.405463in}}%
\pgfpathcurveto{\pgfqpoint{4.347581in}{2.397649in}}{\pgfqpoint{4.343191in}{2.387050in}}{\pgfqpoint{4.343191in}{2.376000in}}%
\pgfpathcurveto{\pgfqpoint{4.343191in}{2.364950in}}{\pgfqpoint{4.347581in}{2.354351in}}{\pgfqpoint{4.355395in}{2.346537in}}%
\pgfpathcurveto{\pgfqpoint{4.363208in}{2.338724in}}{\pgfqpoint{4.373807in}{2.334333in}}{\pgfqpoint{4.384857in}{2.334333in}}%
\pgfpathclose%
\pgfusepath{stroke,fill}%
\end{pgfscope}%
\begin{pgfscope}%
\pgfpathrectangle{\pgfqpoint{0.800000in}{0.528000in}}{\pgfqpoint{4.960000in}{3.696000in}}%
\pgfusepath{clip}%
\pgfsetbuttcap%
\pgfsetroundjoin%
\definecolor{currentfill}{rgb}{0.000000,0.000000,0.000000}%
\pgfsetfillcolor{currentfill}%
\pgfsetlinewidth{1.003750pt}%
\definecolor{currentstroke}{rgb}{0.000000,0.000000,0.000000}%
\pgfsetstrokecolor{currentstroke}%
\pgfsetdash{}{0pt}%
\pgfpathmoveto{\pgfqpoint{4.384857in}{2.334333in}}%
\pgfpathcurveto{\pgfqpoint{4.395908in}{2.334333in}}{\pgfqpoint{4.406507in}{2.338724in}}{\pgfqpoint{4.414320in}{2.346537in}}%
\pgfpathcurveto{\pgfqpoint{4.422134in}{2.354351in}}{\pgfqpoint{4.426524in}{2.364950in}}{\pgfqpoint{4.426524in}{2.376000in}}%
\pgfpathcurveto{\pgfqpoint{4.426524in}{2.387050in}}{\pgfqpoint{4.422134in}{2.397649in}}{\pgfqpoint{4.414320in}{2.405463in}}%
\pgfpathcurveto{\pgfqpoint{4.406507in}{2.413276in}}{\pgfqpoint{4.395908in}{2.417667in}}{\pgfqpoint{4.384857in}{2.417667in}}%
\pgfpathcurveto{\pgfqpoint{4.373807in}{2.417667in}}{\pgfqpoint{4.363208in}{2.413276in}}{\pgfqpoint{4.355395in}{2.405463in}}%
\pgfpathcurveto{\pgfqpoint{4.347581in}{2.397649in}}{\pgfqpoint{4.343191in}{2.387050in}}{\pgfqpoint{4.343191in}{2.376000in}}%
\pgfpathcurveto{\pgfqpoint{4.343191in}{2.364950in}}{\pgfqpoint{4.347581in}{2.354351in}}{\pgfqpoint{4.355395in}{2.346537in}}%
\pgfpathcurveto{\pgfqpoint{4.363208in}{2.338724in}}{\pgfqpoint{4.373807in}{2.334333in}}{\pgfqpoint{4.384857in}{2.334333in}}%
\pgfpathclose%
\pgfusepath{stroke,fill}%
\end{pgfscope}%
\begin{pgfscope}%
\pgfpathrectangle{\pgfqpoint{0.800000in}{0.528000in}}{\pgfqpoint{4.960000in}{3.696000in}}%
\pgfusepath{clip}%
\pgfsetbuttcap%
\pgfsetroundjoin%
\definecolor{currentfill}{rgb}{0.000000,0.000000,0.000000}%
\pgfsetfillcolor{currentfill}%
\pgfsetlinewidth{1.003750pt}%
\definecolor{currentstroke}{rgb}{0.000000,0.000000,0.000000}%
\pgfsetstrokecolor{currentstroke}%
\pgfsetdash{}{0pt}%
\pgfpathmoveto{\pgfqpoint{4.384857in}{2.334333in}}%
\pgfpathcurveto{\pgfqpoint{4.395908in}{2.334333in}}{\pgfqpoint{4.406507in}{2.338724in}}{\pgfqpoint{4.414320in}{2.346537in}}%
\pgfpathcurveto{\pgfqpoint{4.422134in}{2.354351in}}{\pgfqpoint{4.426524in}{2.364950in}}{\pgfqpoint{4.426524in}{2.376000in}}%
\pgfpathcurveto{\pgfqpoint{4.426524in}{2.387050in}}{\pgfqpoint{4.422134in}{2.397649in}}{\pgfqpoint{4.414320in}{2.405463in}}%
\pgfpathcurveto{\pgfqpoint{4.406507in}{2.413276in}}{\pgfqpoint{4.395908in}{2.417667in}}{\pgfqpoint{4.384857in}{2.417667in}}%
\pgfpathcurveto{\pgfqpoint{4.373807in}{2.417667in}}{\pgfqpoint{4.363208in}{2.413276in}}{\pgfqpoint{4.355395in}{2.405463in}}%
\pgfpathcurveto{\pgfqpoint{4.347581in}{2.397649in}}{\pgfqpoint{4.343191in}{2.387050in}}{\pgfqpoint{4.343191in}{2.376000in}}%
\pgfpathcurveto{\pgfqpoint{4.343191in}{2.364950in}}{\pgfqpoint{4.347581in}{2.354351in}}{\pgfqpoint{4.355395in}{2.346537in}}%
\pgfpathcurveto{\pgfqpoint{4.363208in}{2.338724in}}{\pgfqpoint{4.373807in}{2.334333in}}{\pgfqpoint{4.384857in}{2.334333in}}%
\pgfpathclose%
\pgfusepath{stroke,fill}%
\end{pgfscope}%
\begin{pgfscope}%
\pgfpathrectangle{\pgfqpoint{0.800000in}{0.528000in}}{\pgfqpoint{4.960000in}{3.696000in}}%
\pgfusepath{clip}%
\pgfsetbuttcap%
\pgfsetroundjoin%
\definecolor{currentfill}{rgb}{0.000000,0.000000,0.000000}%
\pgfsetfillcolor{currentfill}%
\pgfsetlinewidth{1.003750pt}%
\definecolor{currentstroke}{rgb}{0.000000,0.000000,0.000000}%
\pgfsetstrokecolor{currentstroke}%
\pgfsetdash{}{0pt}%
\pgfpathmoveto{\pgfqpoint{4.384857in}{2.334333in}}%
\pgfpathcurveto{\pgfqpoint{4.395908in}{2.334333in}}{\pgfqpoint{4.406507in}{2.338724in}}{\pgfqpoint{4.414320in}{2.346537in}}%
\pgfpathcurveto{\pgfqpoint{4.422134in}{2.354351in}}{\pgfqpoint{4.426524in}{2.364950in}}{\pgfqpoint{4.426524in}{2.376000in}}%
\pgfpathcurveto{\pgfqpoint{4.426524in}{2.387050in}}{\pgfqpoint{4.422134in}{2.397649in}}{\pgfqpoint{4.414320in}{2.405463in}}%
\pgfpathcurveto{\pgfqpoint{4.406507in}{2.413276in}}{\pgfqpoint{4.395908in}{2.417667in}}{\pgfqpoint{4.384857in}{2.417667in}}%
\pgfpathcurveto{\pgfqpoint{4.373807in}{2.417667in}}{\pgfqpoint{4.363208in}{2.413276in}}{\pgfqpoint{4.355395in}{2.405463in}}%
\pgfpathcurveto{\pgfqpoint{4.347581in}{2.397649in}}{\pgfqpoint{4.343191in}{2.387050in}}{\pgfqpoint{4.343191in}{2.376000in}}%
\pgfpathcurveto{\pgfqpoint{4.343191in}{2.364950in}}{\pgfqpoint{4.347581in}{2.354351in}}{\pgfqpoint{4.355395in}{2.346537in}}%
\pgfpathcurveto{\pgfqpoint{4.363208in}{2.338724in}}{\pgfqpoint{4.373807in}{2.334333in}}{\pgfqpoint{4.384857in}{2.334333in}}%
\pgfpathclose%
\pgfusepath{stroke,fill}%
\end{pgfscope}%
\begin{pgfscope}%
\pgfpathrectangle{\pgfqpoint{0.800000in}{0.528000in}}{\pgfqpoint{4.960000in}{3.696000in}}%
\pgfusepath{clip}%
\pgfsetbuttcap%
\pgfsetroundjoin%
\definecolor{currentfill}{rgb}{0.000000,0.000000,0.000000}%
\pgfsetfillcolor{currentfill}%
\pgfsetlinewidth{1.003750pt}%
\definecolor{currentstroke}{rgb}{0.000000,0.000000,0.000000}%
\pgfsetstrokecolor{currentstroke}%
\pgfsetdash{}{0pt}%
\pgfpathmoveto{\pgfqpoint{4.384857in}{2.334333in}}%
\pgfpathcurveto{\pgfqpoint{4.395908in}{2.334333in}}{\pgfqpoint{4.406507in}{2.338724in}}{\pgfqpoint{4.414320in}{2.346537in}}%
\pgfpathcurveto{\pgfqpoint{4.422134in}{2.354351in}}{\pgfqpoint{4.426524in}{2.364950in}}{\pgfqpoint{4.426524in}{2.376000in}}%
\pgfpathcurveto{\pgfqpoint{4.426524in}{2.387050in}}{\pgfqpoint{4.422134in}{2.397649in}}{\pgfqpoint{4.414320in}{2.405463in}}%
\pgfpathcurveto{\pgfqpoint{4.406507in}{2.413276in}}{\pgfqpoint{4.395908in}{2.417667in}}{\pgfqpoint{4.384857in}{2.417667in}}%
\pgfpathcurveto{\pgfqpoint{4.373807in}{2.417667in}}{\pgfqpoint{4.363208in}{2.413276in}}{\pgfqpoint{4.355395in}{2.405463in}}%
\pgfpathcurveto{\pgfqpoint{4.347581in}{2.397649in}}{\pgfqpoint{4.343191in}{2.387050in}}{\pgfqpoint{4.343191in}{2.376000in}}%
\pgfpathcurveto{\pgfqpoint{4.343191in}{2.364950in}}{\pgfqpoint{4.347581in}{2.354351in}}{\pgfqpoint{4.355395in}{2.346537in}}%
\pgfpathcurveto{\pgfqpoint{4.363208in}{2.338724in}}{\pgfqpoint{4.373807in}{2.334333in}}{\pgfqpoint{4.384857in}{2.334333in}}%
\pgfpathclose%
\pgfusepath{stroke,fill}%
\end{pgfscope}%
\begin{pgfscope}%
\pgfpathrectangle{\pgfqpoint{0.800000in}{0.528000in}}{\pgfqpoint{4.960000in}{3.696000in}}%
\pgfusepath{clip}%
\pgfsetbuttcap%
\pgfsetroundjoin%
\definecolor{currentfill}{rgb}{0.000000,0.000000,0.000000}%
\pgfsetfillcolor{currentfill}%
\pgfsetlinewidth{1.003750pt}%
\definecolor{currentstroke}{rgb}{0.000000,0.000000,0.000000}%
\pgfsetstrokecolor{currentstroke}%
\pgfsetdash{}{0pt}%
\pgfpathmoveto{\pgfqpoint{4.384857in}{2.334333in}}%
\pgfpathcurveto{\pgfqpoint{4.395908in}{2.334333in}}{\pgfqpoint{4.406507in}{2.338724in}}{\pgfqpoint{4.414320in}{2.346537in}}%
\pgfpathcurveto{\pgfqpoint{4.422134in}{2.354351in}}{\pgfqpoint{4.426524in}{2.364950in}}{\pgfqpoint{4.426524in}{2.376000in}}%
\pgfpathcurveto{\pgfqpoint{4.426524in}{2.387050in}}{\pgfqpoint{4.422134in}{2.397649in}}{\pgfqpoint{4.414320in}{2.405463in}}%
\pgfpathcurveto{\pgfqpoint{4.406507in}{2.413276in}}{\pgfqpoint{4.395908in}{2.417667in}}{\pgfqpoint{4.384857in}{2.417667in}}%
\pgfpathcurveto{\pgfqpoint{4.373807in}{2.417667in}}{\pgfqpoint{4.363208in}{2.413276in}}{\pgfqpoint{4.355395in}{2.405463in}}%
\pgfpathcurveto{\pgfqpoint{4.347581in}{2.397649in}}{\pgfqpoint{4.343191in}{2.387050in}}{\pgfqpoint{4.343191in}{2.376000in}}%
\pgfpathcurveto{\pgfqpoint{4.343191in}{2.364950in}}{\pgfqpoint{4.347581in}{2.354351in}}{\pgfqpoint{4.355395in}{2.346537in}}%
\pgfpathcurveto{\pgfqpoint{4.363208in}{2.338724in}}{\pgfqpoint{4.373807in}{2.334333in}}{\pgfqpoint{4.384857in}{2.334333in}}%
\pgfpathclose%
\pgfusepath{stroke,fill}%
\end{pgfscope}%
\begin{pgfscope}%
\pgfpathrectangle{\pgfqpoint{0.800000in}{0.528000in}}{\pgfqpoint{4.960000in}{3.696000in}}%
\pgfusepath{clip}%
\pgfsetbuttcap%
\pgfsetroundjoin%
\definecolor{currentfill}{rgb}{0.000000,0.000000,0.000000}%
\pgfsetfillcolor{currentfill}%
\pgfsetlinewidth{1.003750pt}%
\definecolor{currentstroke}{rgb}{0.000000,0.000000,0.000000}%
\pgfsetstrokecolor{currentstroke}%
\pgfsetdash{}{0pt}%
\pgfpathmoveto{\pgfqpoint{4.384857in}{2.334333in}}%
\pgfpathcurveto{\pgfqpoint{4.395908in}{2.334333in}}{\pgfqpoint{4.406507in}{2.338724in}}{\pgfqpoint{4.414320in}{2.346537in}}%
\pgfpathcurveto{\pgfqpoint{4.422134in}{2.354351in}}{\pgfqpoint{4.426524in}{2.364950in}}{\pgfqpoint{4.426524in}{2.376000in}}%
\pgfpathcurveto{\pgfqpoint{4.426524in}{2.387050in}}{\pgfqpoint{4.422134in}{2.397649in}}{\pgfqpoint{4.414320in}{2.405463in}}%
\pgfpathcurveto{\pgfqpoint{4.406507in}{2.413276in}}{\pgfqpoint{4.395908in}{2.417667in}}{\pgfqpoint{4.384857in}{2.417667in}}%
\pgfpathcurveto{\pgfqpoint{4.373807in}{2.417667in}}{\pgfqpoint{4.363208in}{2.413276in}}{\pgfqpoint{4.355395in}{2.405463in}}%
\pgfpathcurveto{\pgfqpoint{4.347581in}{2.397649in}}{\pgfqpoint{4.343191in}{2.387050in}}{\pgfqpoint{4.343191in}{2.376000in}}%
\pgfpathcurveto{\pgfqpoint{4.343191in}{2.364950in}}{\pgfqpoint{4.347581in}{2.354351in}}{\pgfqpoint{4.355395in}{2.346537in}}%
\pgfpathcurveto{\pgfqpoint{4.363208in}{2.338724in}}{\pgfqpoint{4.373807in}{2.334333in}}{\pgfqpoint{4.384857in}{2.334333in}}%
\pgfpathclose%
\pgfusepath{stroke,fill}%
\end{pgfscope}%
\begin{pgfscope}%
\pgfpathrectangle{\pgfqpoint{0.800000in}{0.528000in}}{\pgfqpoint{4.960000in}{3.696000in}}%
\pgfusepath{clip}%
\pgfsetbuttcap%
\pgfsetroundjoin%
\definecolor{currentfill}{rgb}{0.000000,0.000000,0.000000}%
\pgfsetfillcolor{currentfill}%
\pgfsetlinewidth{1.003750pt}%
\definecolor{currentstroke}{rgb}{0.000000,0.000000,0.000000}%
\pgfsetstrokecolor{currentstroke}%
\pgfsetdash{}{0pt}%
\pgfpathmoveto{\pgfqpoint{4.384857in}{2.334333in}}%
\pgfpathcurveto{\pgfqpoint{4.395908in}{2.334333in}}{\pgfqpoint{4.406507in}{2.338724in}}{\pgfqpoint{4.414320in}{2.346537in}}%
\pgfpathcurveto{\pgfqpoint{4.422134in}{2.354351in}}{\pgfqpoint{4.426524in}{2.364950in}}{\pgfqpoint{4.426524in}{2.376000in}}%
\pgfpathcurveto{\pgfqpoint{4.426524in}{2.387050in}}{\pgfqpoint{4.422134in}{2.397649in}}{\pgfqpoint{4.414320in}{2.405463in}}%
\pgfpathcurveto{\pgfqpoint{4.406507in}{2.413276in}}{\pgfqpoint{4.395908in}{2.417667in}}{\pgfqpoint{4.384857in}{2.417667in}}%
\pgfpathcurveto{\pgfqpoint{4.373807in}{2.417667in}}{\pgfqpoint{4.363208in}{2.413276in}}{\pgfqpoint{4.355395in}{2.405463in}}%
\pgfpathcurveto{\pgfqpoint{4.347581in}{2.397649in}}{\pgfqpoint{4.343191in}{2.387050in}}{\pgfqpoint{4.343191in}{2.376000in}}%
\pgfpathcurveto{\pgfqpoint{4.343191in}{2.364950in}}{\pgfqpoint{4.347581in}{2.354351in}}{\pgfqpoint{4.355395in}{2.346537in}}%
\pgfpathcurveto{\pgfqpoint{4.363208in}{2.338724in}}{\pgfqpoint{4.373807in}{2.334333in}}{\pgfqpoint{4.384857in}{2.334333in}}%
\pgfpathclose%
\pgfusepath{stroke,fill}%
\end{pgfscope}%
\begin{pgfscope}%
\pgfpathrectangle{\pgfqpoint{0.800000in}{0.528000in}}{\pgfqpoint{4.960000in}{3.696000in}}%
\pgfusepath{clip}%
\pgfsetbuttcap%
\pgfsetroundjoin%
\definecolor{currentfill}{rgb}{0.000000,0.000000,0.000000}%
\pgfsetfillcolor{currentfill}%
\pgfsetlinewidth{1.003750pt}%
\definecolor{currentstroke}{rgb}{0.000000,0.000000,0.000000}%
\pgfsetstrokecolor{currentstroke}%
\pgfsetdash{}{0pt}%
\pgfpathmoveto{\pgfqpoint{4.384857in}{2.334333in}}%
\pgfpathcurveto{\pgfqpoint{4.395908in}{2.334333in}}{\pgfqpoint{4.406507in}{2.338724in}}{\pgfqpoint{4.414320in}{2.346537in}}%
\pgfpathcurveto{\pgfqpoint{4.422134in}{2.354351in}}{\pgfqpoint{4.426524in}{2.364950in}}{\pgfqpoint{4.426524in}{2.376000in}}%
\pgfpathcurveto{\pgfqpoint{4.426524in}{2.387050in}}{\pgfqpoint{4.422134in}{2.397649in}}{\pgfqpoint{4.414320in}{2.405463in}}%
\pgfpathcurveto{\pgfqpoint{4.406507in}{2.413276in}}{\pgfqpoint{4.395908in}{2.417667in}}{\pgfqpoint{4.384857in}{2.417667in}}%
\pgfpathcurveto{\pgfqpoint{4.373807in}{2.417667in}}{\pgfqpoint{4.363208in}{2.413276in}}{\pgfqpoint{4.355395in}{2.405463in}}%
\pgfpathcurveto{\pgfqpoint{4.347581in}{2.397649in}}{\pgfqpoint{4.343191in}{2.387050in}}{\pgfqpoint{4.343191in}{2.376000in}}%
\pgfpathcurveto{\pgfqpoint{4.343191in}{2.364950in}}{\pgfqpoint{4.347581in}{2.354351in}}{\pgfqpoint{4.355395in}{2.346537in}}%
\pgfpathcurveto{\pgfqpoint{4.363208in}{2.338724in}}{\pgfqpoint{4.373807in}{2.334333in}}{\pgfqpoint{4.384857in}{2.334333in}}%
\pgfpathclose%
\pgfusepath{stroke,fill}%
\end{pgfscope}%
\begin{pgfscope}%
\pgfpathrectangle{\pgfqpoint{0.800000in}{0.528000in}}{\pgfqpoint{4.960000in}{3.696000in}}%
\pgfusepath{clip}%
\pgfsetbuttcap%
\pgfsetroundjoin%
\definecolor{currentfill}{rgb}{0.000000,0.000000,0.000000}%
\pgfsetfillcolor{currentfill}%
\pgfsetlinewidth{1.003750pt}%
\definecolor{currentstroke}{rgb}{0.000000,0.000000,0.000000}%
\pgfsetstrokecolor{currentstroke}%
\pgfsetdash{}{0pt}%
\pgfpathmoveto{\pgfqpoint{4.384857in}{2.334333in}}%
\pgfpathcurveto{\pgfqpoint{4.395908in}{2.334333in}}{\pgfqpoint{4.406507in}{2.338724in}}{\pgfqpoint{4.414320in}{2.346537in}}%
\pgfpathcurveto{\pgfqpoint{4.422134in}{2.354351in}}{\pgfqpoint{4.426524in}{2.364950in}}{\pgfqpoint{4.426524in}{2.376000in}}%
\pgfpathcurveto{\pgfqpoint{4.426524in}{2.387050in}}{\pgfqpoint{4.422134in}{2.397649in}}{\pgfqpoint{4.414320in}{2.405463in}}%
\pgfpathcurveto{\pgfqpoint{4.406507in}{2.413276in}}{\pgfqpoint{4.395908in}{2.417667in}}{\pgfqpoint{4.384857in}{2.417667in}}%
\pgfpathcurveto{\pgfqpoint{4.373807in}{2.417667in}}{\pgfqpoint{4.363208in}{2.413276in}}{\pgfqpoint{4.355395in}{2.405463in}}%
\pgfpathcurveto{\pgfqpoint{4.347581in}{2.397649in}}{\pgfqpoint{4.343191in}{2.387050in}}{\pgfqpoint{4.343191in}{2.376000in}}%
\pgfpathcurveto{\pgfqpoint{4.343191in}{2.364950in}}{\pgfqpoint{4.347581in}{2.354351in}}{\pgfqpoint{4.355395in}{2.346537in}}%
\pgfpathcurveto{\pgfqpoint{4.363208in}{2.338724in}}{\pgfqpoint{4.373807in}{2.334333in}}{\pgfqpoint{4.384857in}{2.334333in}}%
\pgfpathclose%
\pgfusepath{stroke,fill}%
\end{pgfscope}%
\begin{pgfscope}%
\pgfpathrectangle{\pgfqpoint{0.800000in}{0.528000in}}{\pgfqpoint{4.960000in}{3.696000in}}%
\pgfusepath{clip}%
\pgfsetbuttcap%
\pgfsetroundjoin%
\definecolor{currentfill}{rgb}{0.000000,0.000000,0.000000}%
\pgfsetfillcolor{currentfill}%
\pgfsetlinewidth{1.003750pt}%
\definecolor{currentstroke}{rgb}{0.000000,0.000000,0.000000}%
\pgfsetstrokecolor{currentstroke}%
\pgfsetdash{}{0pt}%
\pgfpathmoveto{\pgfqpoint{4.384857in}{2.334333in}}%
\pgfpathcurveto{\pgfqpoint{4.395908in}{2.334333in}}{\pgfqpoint{4.406507in}{2.338724in}}{\pgfqpoint{4.414320in}{2.346537in}}%
\pgfpathcurveto{\pgfqpoint{4.422134in}{2.354351in}}{\pgfqpoint{4.426524in}{2.364950in}}{\pgfqpoint{4.426524in}{2.376000in}}%
\pgfpathcurveto{\pgfqpoint{4.426524in}{2.387050in}}{\pgfqpoint{4.422134in}{2.397649in}}{\pgfqpoint{4.414320in}{2.405463in}}%
\pgfpathcurveto{\pgfqpoint{4.406507in}{2.413276in}}{\pgfqpoint{4.395908in}{2.417667in}}{\pgfqpoint{4.384857in}{2.417667in}}%
\pgfpathcurveto{\pgfqpoint{4.373807in}{2.417667in}}{\pgfqpoint{4.363208in}{2.413276in}}{\pgfqpoint{4.355395in}{2.405463in}}%
\pgfpathcurveto{\pgfqpoint{4.347581in}{2.397649in}}{\pgfqpoint{4.343191in}{2.387050in}}{\pgfqpoint{4.343191in}{2.376000in}}%
\pgfpathcurveto{\pgfqpoint{4.343191in}{2.364950in}}{\pgfqpoint{4.347581in}{2.354351in}}{\pgfqpoint{4.355395in}{2.346537in}}%
\pgfpathcurveto{\pgfqpoint{4.363208in}{2.338724in}}{\pgfqpoint{4.373807in}{2.334333in}}{\pgfqpoint{4.384857in}{2.334333in}}%
\pgfpathclose%
\pgfusepath{stroke,fill}%
\end{pgfscope}%
\begin{pgfscope}%
\pgfpathrectangle{\pgfqpoint{0.800000in}{0.528000in}}{\pgfqpoint{4.960000in}{3.696000in}}%
\pgfusepath{clip}%
\pgfsetbuttcap%
\pgfsetroundjoin%
\definecolor{currentfill}{rgb}{0.000000,0.000000,0.000000}%
\pgfsetfillcolor{currentfill}%
\pgfsetlinewidth{1.003750pt}%
\definecolor{currentstroke}{rgb}{0.000000,0.000000,0.000000}%
\pgfsetstrokecolor{currentstroke}%
\pgfsetdash{}{0pt}%
\pgfpathmoveto{\pgfqpoint{4.384857in}{2.334333in}}%
\pgfpathcurveto{\pgfqpoint{4.395908in}{2.334333in}}{\pgfqpoint{4.406507in}{2.338724in}}{\pgfqpoint{4.414320in}{2.346537in}}%
\pgfpathcurveto{\pgfqpoint{4.422134in}{2.354351in}}{\pgfqpoint{4.426524in}{2.364950in}}{\pgfqpoint{4.426524in}{2.376000in}}%
\pgfpathcurveto{\pgfqpoint{4.426524in}{2.387050in}}{\pgfqpoint{4.422134in}{2.397649in}}{\pgfqpoint{4.414320in}{2.405463in}}%
\pgfpathcurveto{\pgfqpoint{4.406507in}{2.413276in}}{\pgfqpoint{4.395908in}{2.417667in}}{\pgfqpoint{4.384857in}{2.417667in}}%
\pgfpathcurveto{\pgfqpoint{4.373807in}{2.417667in}}{\pgfqpoint{4.363208in}{2.413276in}}{\pgfqpoint{4.355395in}{2.405463in}}%
\pgfpathcurveto{\pgfqpoint{4.347581in}{2.397649in}}{\pgfqpoint{4.343191in}{2.387050in}}{\pgfqpoint{4.343191in}{2.376000in}}%
\pgfpathcurveto{\pgfqpoint{4.343191in}{2.364950in}}{\pgfqpoint{4.347581in}{2.354351in}}{\pgfqpoint{4.355395in}{2.346537in}}%
\pgfpathcurveto{\pgfqpoint{4.363208in}{2.338724in}}{\pgfqpoint{4.373807in}{2.334333in}}{\pgfqpoint{4.384857in}{2.334333in}}%
\pgfpathclose%
\pgfusepath{stroke,fill}%
\end{pgfscope}%
\begin{pgfscope}%
\pgfpathrectangle{\pgfqpoint{0.800000in}{0.528000in}}{\pgfqpoint{4.960000in}{3.696000in}}%
\pgfusepath{clip}%
\pgfsetbuttcap%
\pgfsetroundjoin%
\definecolor{currentfill}{rgb}{0.000000,0.000000,0.000000}%
\pgfsetfillcolor{currentfill}%
\pgfsetlinewidth{1.003750pt}%
\definecolor{currentstroke}{rgb}{0.000000,0.000000,0.000000}%
\pgfsetstrokecolor{currentstroke}%
\pgfsetdash{}{0pt}%
\pgfpathmoveto{\pgfqpoint{4.384857in}{2.334333in}}%
\pgfpathcurveto{\pgfqpoint{4.395908in}{2.334333in}}{\pgfqpoint{4.406507in}{2.338724in}}{\pgfqpoint{4.414320in}{2.346537in}}%
\pgfpathcurveto{\pgfqpoint{4.422134in}{2.354351in}}{\pgfqpoint{4.426524in}{2.364950in}}{\pgfqpoint{4.426524in}{2.376000in}}%
\pgfpathcurveto{\pgfqpoint{4.426524in}{2.387050in}}{\pgfqpoint{4.422134in}{2.397649in}}{\pgfqpoint{4.414320in}{2.405463in}}%
\pgfpathcurveto{\pgfqpoint{4.406507in}{2.413276in}}{\pgfqpoint{4.395908in}{2.417667in}}{\pgfqpoint{4.384857in}{2.417667in}}%
\pgfpathcurveto{\pgfqpoint{4.373807in}{2.417667in}}{\pgfqpoint{4.363208in}{2.413276in}}{\pgfqpoint{4.355395in}{2.405463in}}%
\pgfpathcurveto{\pgfqpoint{4.347581in}{2.397649in}}{\pgfqpoint{4.343191in}{2.387050in}}{\pgfqpoint{4.343191in}{2.376000in}}%
\pgfpathcurveto{\pgfqpoint{4.343191in}{2.364950in}}{\pgfqpoint{4.347581in}{2.354351in}}{\pgfqpoint{4.355395in}{2.346537in}}%
\pgfpathcurveto{\pgfqpoint{4.363208in}{2.338724in}}{\pgfqpoint{4.373807in}{2.334333in}}{\pgfqpoint{4.384857in}{2.334333in}}%
\pgfpathclose%
\pgfusepath{stroke,fill}%
\end{pgfscope}%
\begin{pgfscope}%
\pgfpathrectangle{\pgfqpoint{0.800000in}{0.528000in}}{\pgfqpoint{4.960000in}{3.696000in}}%
\pgfusepath{clip}%
\pgfsetbuttcap%
\pgfsetroundjoin%
\definecolor{currentfill}{rgb}{0.000000,0.000000,0.000000}%
\pgfsetfillcolor{currentfill}%
\pgfsetlinewidth{1.003750pt}%
\definecolor{currentstroke}{rgb}{0.000000,0.000000,0.000000}%
\pgfsetstrokecolor{currentstroke}%
\pgfsetdash{}{0pt}%
\pgfpathmoveto{\pgfqpoint{4.384857in}{2.334333in}}%
\pgfpathcurveto{\pgfqpoint{4.395908in}{2.334333in}}{\pgfqpoint{4.406507in}{2.338724in}}{\pgfqpoint{4.414320in}{2.346537in}}%
\pgfpathcurveto{\pgfqpoint{4.422134in}{2.354351in}}{\pgfqpoint{4.426524in}{2.364950in}}{\pgfqpoint{4.426524in}{2.376000in}}%
\pgfpathcurveto{\pgfqpoint{4.426524in}{2.387050in}}{\pgfqpoint{4.422134in}{2.397649in}}{\pgfqpoint{4.414320in}{2.405463in}}%
\pgfpathcurveto{\pgfqpoint{4.406507in}{2.413276in}}{\pgfqpoint{4.395908in}{2.417667in}}{\pgfqpoint{4.384857in}{2.417667in}}%
\pgfpathcurveto{\pgfqpoint{4.373807in}{2.417667in}}{\pgfqpoint{4.363208in}{2.413276in}}{\pgfqpoint{4.355395in}{2.405463in}}%
\pgfpathcurveto{\pgfqpoint{4.347581in}{2.397649in}}{\pgfqpoint{4.343191in}{2.387050in}}{\pgfqpoint{4.343191in}{2.376000in}}%
\pgfpathcurveto{\pgfqpoint{4.343191in}{2.364950in}}{\pgfqpoint{4.347581in}{2.354351in}}{\pgfqpoint{4.355395in}{2.346537in}}%
\pgfpathcurveto{\pgfqpoint{4.363208in}{2.338724in}}{\pgfqpoint{4.373807in}{2.334333in}}{\pgfqpoint{4.384857in}{2.334333in}}%
\pgfpathclose%
\pgfusepath{stroke,fill}%
\end{pgfscope}%
\begin{pgfscope}%
\pgfpathrectangle{\pgfqpoint{0.800000in}{0.528000in}}{\pgfqpoint{4.960000in}{3.696000in}}%
\pgfusepath{clip}%
\pgfsetbuttcap%
\pgfsetroundjoin%
\definecolor{currentfill}{rgb}{0.000000,0.000000,0.000000}%
\pgfsetfillcolor{currentfill}%
\pgfsetlinewidth{1.003750pt}%
\definecolor{currentstroke}{rgb}{0.000000,0.000000,0.000000}%
\pgfsetstrokecolor{currentstroke}%
\pgfsetdash{}{0pt}%
\pgfpathmoveto{\pgfqpoint{4.384857in}{2.334333in}}%
\pgfpathcurveto{\pgfqpoint{4.395908in}{2.334333in}}{\pgfqpoint{4.406507in}{2.338724in}}{\pgfqpoint{4.414320in}{2.346537in}}%
\pgfpathcurveto{\pgfqpoint{4.422134in}{2.354351in}}{\pgfqpoint{4.426524in}{2.364950in}}{\pgfqpoint{4.426524in}{2.376000in}}%
\pgfpathcurveto{\pgfqpoint{4.426524in}{2.387050in}}{\pgfqpoint{4.422134in}{2.397649in}}{\pgfqpoint{4.414320in}{2.405463in}}%
\pgfpathcurveto{\pgfqpoint{4.406507in}{2.413276in}}{\pgfqpoint{4.395908in}{2.417667in}}{\pgfqpoint{4.384857in}{2.417667in}}%
\pgfpathcurveto{\pgfqpoint{4.373807in}{2.417667in}}{\pgfqpoint{4.363208in}{2.413276in}}{\pgfqpoint{4.355395in}{2.405463in}}%
\pgfpathcurveto{\pgfqpoint{4.347581in}{2.397649in}}{\pgfqpoint{4.343191in}{2.387050in}}{\pgfqpoint{4.343191in}{2.376000in}}%
\pgfpathcurveto{\pgfqpoint{4.343191in}{2.364950in}}{\pgfqpoint{4.347581in}{2.354351in}}{\pgfqpoint{4.355395in}{2.346537in}}%
\pgfpathcurveto{\pgfqpoint{4.363208in}{2.338724in}}{\pgfqpoint{4.373807in}{2.334333in}}{\pgfqpoint{4.384857in}{2.334333in}}%
\pgfpathclose%
\pgfusepath{stroke,fill}%
\end{pgfscope}%
\begin{pgfscope}%
\pgfpathrectangle{\pgfqpoint{0.800000in}{0.528000in}}{\pgfqpoint{4.960000in}{3.696000in}}%
\pgfusepath{clip}%
\pgfsetbuttcap%
\pgfsetroundjoin%
\definecolor{currentfill}{rgb}{0.000000,0.000000,0.000000}%
\pgfsetfillcolor{currentfill}%
\pgfsetlinewidth{1.003750pt}%
\definecolor{currentstroke}{rgb}{0.000000,0.000000,0.000000}%
\pgfsetstrokecolor{currentstroke}%
\pgfsetdash{}{0pt}%
\pgfpathmoveto{\pgfqpoint{4.384857in}{2.334333in}}%
\pgfpathcurveto{\pgfqpoint{4.395908in}{2.334333in}}{\pgfqpoint{4.406507in}{2.338724in}}{\pgfqpoint{4.414320in}{2.346537in}}%
\pgfpathcurveto{\pgfqpoint{4.422134in}{2.354351in}}{\pgfqpoint{4.426524in}{2.364950in}}{\pgfqpoint{4.426524in}{2.376000in}}%
\pgfpathcurveto{\pgfqpoint{4.426524in}{2.387050in}}{\pgfqpoint{4.422134in}{2.397649in}}{\pgfqpoint{4.414320in}{2.405463in}}%
\pgfpathcurveto{\pgfqpoint{4.406507in}{2.413276in}}{\pgfqpoint{4.395908in}{2.417667in}}{\pgfqpoint{4.384857in}{2.417667in}}%
\pgfpathcurveto{\pgfqpoint{4.373807in}{2.417667in}}{\pgfqpoint{4.363208in}{2.413276in}}{\pgfqpoint{4.355395in}{2.405463in}}%
\pgfpathcurveto{\pgfqpoint{4.347581in}{2.397649in}}{\pgfqpoint{4.343191in}{2.387050in}}{\pgfqpoint{4.343191in}{2.376000in}}%
\pgfpathcurveto{\pgfqpoint{4.343191in}{2.364950in}}{\pgfqpoint{4.347581in}{2.354351in}}{\pgfqpoint{4.355395in}{2.346537in}}%
\pgfpathcurveto{\pgfqpoint{4.363208in}{2.338724in}}{\pgfqpoint{4.373807in}{2.334333in}}{\pgfqpoint{4.384857in}{2.334333in}}%
\pgfpathclose%
\pgfusepath{stroke,fill}%
\end{pgfscope}%
\begin{pgfscope}%
\pgfpathrectangle{\pgfqpoint{0.800000in}{0.528000in}}{\pgfqpoint{4.960000in}{3.696000in}}%
\pgfusepath{clip}%
\pgfsetbuttcap%
\pgfsetroundjoin%
\definecolor{currentfill}{rgb}{0.000000,0.000000,0.000000}%
\pgfsetfillcolor{currentfill}%
\pgfsetlinewidth{1.003750pt}%
\definecolor{currentstroke}{rgb}{0.000000,0.000000,0.000000}%
\pgfsetstrokecolor{currentstroke}%
\pgfsetdash{}{0pt}%
\pgfpathmoveto{\pgfqpoint{4.384857in}{2.334333in}}%
\pgfpathcurveto{\pgfqpoint{4.395908in}{2.334333in}}{\pgfqpoint{4.406507in}{2.338724in}}{\pgfqpoint{4.414320in}{2.346537in}}%
\pgfpathcurveto{\pgfqpoint{4.422134in}{2.354351in}}{\pgfqpoint{4.426524in}{2.364950in}}{\pgfqpoint{4.426524in}{2.376000in}}%
\pgfpathcurveto{\pgfqpoint{4.426524in}{2.387050in}}{\pgfqpoint{4.422134in}{2.397649in}}{\pgfqpoint{4.414320in}{2.405463in}}%
\pgfpathcurveto{\pgfqpoint{4.406507in}{2.413276in}}{\pgfqpoint{4.395908in}{2.417667in}}{\pgfqpoint{4.384857in}{2.417667in}}%
\pgfpathcurveto{\pgfqpoint{4.373807in}{2.417667in}}{\pgfqpoint{4.363208in}{2.413276in}}{\pgfqpoint{4.355395in}{2.405463in}}%
\pgfpathcurveto{\pgfqpoint{4.347581in}{2.397649in}}{\pgfqpoint{4.343191in}{2.387050in}}{\pgfqpoint{4.343191in}{2.376000in}}%
\pgfpathcurveto{\pgfqpoint{4.343191in}{2.364950in}}{\pgfqpoint{4.347581in}{2.354351in}}{\pgfqpoint{4.355395in}{2.346537in}}%
\pgfpathcurveto{\pgfqpoint{4.363208in}{2.338724in}}{\pgfqpoint{4.373807in}{2.334333in}}{\pgfqpoint{4.384857in}{2.334333in}}%
\pgfpathclose%
\pgfusepath{stroke,fill}%
\end{pgfscope}%
\begin{pgfscope}%
\pgfpathrectangle{\pgfqpoint{0.800000in}{0.528000in}}{\pgfqpoint{4.960000in}{3.696000in}}%
\pgfusepath{clip}%
\pgfsetbuttcap%
\pgfsetroundjoin%
\definecolor{currentfill}{rgb}{0.000000,0.000000,0.000000}%
\pgfsetfillcolor{currentfill}%
\pgfsetlinewidth{1.003750pt}%
\definecolor{currentstroke}{rgb}{0.000000,0.000000,0.000000}%
\pgfsetstrokecolor{currentstroke}%
\pgfsetdash{}{0pt}%
\pgfpathmoveto{\pgfqpoint{4.384857in}{2.334333in}}%
\pgfpathcurveto{\pgfqpoint{4.395908in}{2.334333in}}{\pgfqpoint{4.406507in}{2.338724in}}{\pgfqpoint{4.414320in}{2.346537in}}%
\pgfpathcurveto{\pgfqpoint{4.422134in}{2.354351in}}{\pgfqpoint{4.426524in}{2.364950in}}{\pgfqpoint{4.426524in}{2.376000in}}%
\pgfpathcurveto{\pgfqpoint{4.426524in}{2.387050in}}{\pgfqpoint{4.422134in}{2.397649in}}{\pgfqpoint{4.414320in}{2.405463in}}%
\pgfpathcurveto{\pgfqpoint{4.406507in}{2.413276in}}{\pgfqpoint{4.395908in}{2.417667in}}{\pgfqpoint{4.384857in}{2.417667in}}%
\pgfpathcurveto{\pgfqpoint{4.373807in}{2.417667in}}{\pgfqpoint{4.363208in}{2.413276in}}{\pgfqpoint{4.355395in}{2.405463in}}%
\pgfpathcurveto{\pgfqpoint{4.347581in}{2.397649in}}{\pgfqpoint{4.343191in}{2.387050in}}{\pgfqpoint{4.343191in}{2.376000in}}%
\pgfpathcurveto{\pgfqpoint{4.343191in}{2.364950in}}{\pgfqpoint{4.347581in}{2.354351in}}{\pgfqpoint{4.355395in}{2.346537in}}%
\pgfpathcurveto{\pgfqpoint{4.363208in}{2.338724in}}{\pgfqpoint{4.373807in}{2.334333in}}{\pgfqpoint{4.384857in}{2.334333in}}%
\pgfpathclose%
\pgfusepath{stroke,fill}%
\end{pgfscope}%
\begin{pgfscope}%
\pgfpathrectangle{\pgfqpoint{0.800000in}{0.528000in}}{\pgfqpoint{4.960000in}{3.696000in}}%
\pgfusepath{clip}%
\pgfsetbuttcap%
\pgfsetroundjoin%
\definecolor{currentfill}{rgb}{0.000000,0.000000,0.000000}%
\pgfsetfillcolor{currentfill}%
\pgfsetlinewidth{1.003750pt}%
\definecolor{currentstroke}{rgb}{0.000000,0.000000,0.000000}%
\pgfsetstrokecolor{currentstroke}%
\pgfsetdash{}{0pt}%
\pgfpathmoveto{\pgfqpoint{4.384857in}{2.334333in}}%
\pgfpathcurveto{\pgfqpoint{4.395908in}{2.334333in}}{\pgfqpoint{4.406507in}{2.338724in}}{\pgfqpoint{4.414320in}{2.346537in}}%
\pgfpathcurveto{\pgfqpoint{4.422134in}{2.354351in}}{\pgfqpoint{4.426524in}{2.364950in}}{\pgfqpoint{4.426524in}{2.376000in}}%
\pgfpathcurveto{\pgfqpoint{4.426524in}{2.387050in}}{\pgfqpoint{4.422134in}{2.397649in}}{\pgfqpoint{4.414320in}{2.405463in}}%
\pgfpathcurveto{\pgfqpoint{4.406507in}{2.413276in}}{\pgfqpoint{4.395908in}{2.417667in}}{\pgfqpoint{4.384857in}{2.417667in}}%
\pgfpathcurveto{\pgfqpoint{4.373807in}{2.417667in}}{\pgfqpoint{4.363208in}{2.413276in}}{\pgfqpoint{4.355395in}{2.405463in}}%
\pgfpathcurveto{\pgfqpoint{4.347581in}{2.397649in}}{\pgfqpoint{4.343191in}{2.387050in}}{\pgfqpoint{4.343191in}{2.376000in}}%
\pgfpathcurveto{\pgfqpoint{4.343191in}{2.364950in}}{\pgfqpoint{4.347581in}{2.354351in}}{\pgfqpoint{4.355395in}{2.346537in}}%
\pgfpathcurveto{\pgfqpoint{4.363208in}{2.338724in}}{\pgfqpoint{4.373807in}{2.334333in}}{\pgfqpoint{4.384857in}{2.334333in}}%
\pgfpathclose%
\pgfusepath{stroke,fill}%
\end{pgfscope}%
\begin{pgfscope}%
\pgfpathrectangle{\pgfqpoint{0.800000in}{0.528000in}}{\pgfqpoint{4.960000in}{3.696000in}}%
\pgfusepath{clip}%
\pgfsetbuttcap%
\pgfsetroundjoin%
\definecolor{currentfill}{rgb}{0.000000,0.000000,0.000000}%
\pgfsetfillcolor{currentfill}%
\pgfsetlinewidth{1.003750pt}%
\definecolor{currentstroke}{rgb}{0.000000,0.000000,0.000000}%
\pgfsetstrokecolor{currentstroke}%
\pgfsetdash{}{0pt}%
\pgfpathmoveto{\pgfqpoint{4.384857in}{2.334333in}}%
\pgfpathcurveto{\pgfqpoint{4.395908in}{2.334333in}}{\pgfqpoint{4.406507in}{2.338724in}}{\pgfqpoint{4.414320in}{2.346537in}}%
\pgfpathcurveto{\pgfqpoint{4.422134in}{2.354351in}}{\pgfqpoint{4.426524in}{2.364950in}}{\pgfqpoint{4.426524in}{2.376000in}}%
\pgfpathcurveto{\pgfqpoint{4.426524in}{2.387050in}}{\pgfqpoint{4.422134in}{2.397649in}}{\pgfqpoint{4.414320in}{2.405463in}}%
\pgfpathcurveto{\pgfqpoint{4.406507in}{2.413276in}}{\pgfqpoint{4.395908in}{2.417667in}}{\pgfqpoint{4.384857in}{2.417667in}}%
\pgfpathcurveto{\pgfqpoint{4.373807in}{2.417667in}}{\pgfqpoint{4.363208in}{2.413276in}}{\pgfqpoint{4.355395in}{2.405463in}}%
\pgfpathcurveto{\pgfqpoint{4.347581in}{2.397649in}}{\pgfqpoint{4.343191in}{2.387050in}}{\pgfqpoint{4.343191in}{2.376000in}}%
\pgfpathcurveto{\pgfqpoint{4.343191in}{2.364950in}}{\pgfqpoint{4.347581in}{2.354351in}}{\pgfqpoint{4.355395in}{2.346537in}}%
\pgfpathcurveto{\pgfqpoint{4.363208in}{2.338724in}}{\pgfqpoint{4.373807in}{2.334333in}}{\pgfqpoint{4.384857in}{2.334333in}}%
\pgfpathclose%
\pgfusepath{stroke,fill}%
\end{pgfscope}%
\begin{pgfscope}%
\pgfpathrectangle{\pgfqpoint{0.800000in}{0.528000in}}{\pgfqpoint{4.960000in}{3.696000in}}%
\pgfusepath{clip}%
\pgfsetbuttcap%
\pgfsetroundjoin%
\definecolor{currentfill}{rgb}{0.000000,0.000000,0.000000}%
\pgfsetfillcolor{currentfill}%
\pgfsetlinewidth{1.003750pt}%
\definecolor{currentstroke}{rgb}{0.000000,0.000000,0.000000}%
\pgfsetstrokecolor{currentstroke}%
\pgfsetdash{}{0pt}%
\pgfpathmoveto{\pgfqpoint{4.384857in}{2.334333in}}%
\pgfpathcurveto{\pgfqpoint{4.395908in}{2.334333in}}{\pgfqpoint{4.406507in}{2.338724in}}{\pgfqpoint{4.414320in}{2.346537in}}%
\pgfpathcurveto{\pgfqpoint{4.422134in}{2.354351in}}{\pgfqpoint{4.426524in}{2.364950in}}{\pgfqpoint{4.426524in}{2.376000in}}%
\pgfpathcurveto{\pgfqpoint{4.426524in}{2.387050in}}{\pgfqpoint{4.422134in}{2.397649in}}{\pgfqpoint{4.414320in}{2.405463in}}%
\pgfpathcurveto{\pgfqpoint{4.406507in}{2.413276in}}{\pgfqpoint{4.395908in}{2.417667in}}{\pgfqpoint{4.384857in}{2.417667in}}%
\pgfpathcurveto{\pgfqpoint{4.373807in}{2.417667in}}{\pgfqpoint{4.363208in}{2.413276in}}{\pgfqpoint{4.355395in}{2.405463in}}%
\pgfpathcurveto{\pgfqpoint{4.347581in}{2.397649in}}{\pgfqpoint{4.343191in}{2.387050in}}{\pgfqpoint{4.343191in}{2.376000in}}%
\pgfpathcurveto{\pgfqpoint{4.343191in}{2.364950in}}{\pgfqpoint{4.347581in}{2.354351in}}{\pgfqpoint{4.355395in}{2.346537in}}%
\pgfpathcurveto{\pgfqpoint{4.363208in}{2.338724in}}{\pgfqpoint{4.373807in}{2.334333in}}{\pgfqpoint{4.384857in}{2.334333in}}%
\pgfpathclose%
\pgfusepath{stroke,fill}%
\end{pgfscope}%
\begin{pgfscope}%
\pgfpathrectangle{\pgfqpoint{0.800000in}{0.528000in}}{\pgfqpoint{4.960000in}{3.696000in}}%
\pgfusepath{clip}%
\pgfsetbuttcap%
\pgfsetroundjoin%
\definecolor{currentfill}{rgb}{0.000000,0.000000,0.000000}%
\pgfsetfillcolor{currentfill}%
\pgfsetlinewidth{1.003750pt}%
\definecolor{currentstroke}{rgb}{0.000000,0.000000,0.000000}%
\pgfsetstrokecolor{currentstroke}%
\pgfsetdash{}{0pt}%
\pgfpathmoveto{\pgfqpoint{4.384857in}{2.334333in}}%
\pgfpathcurveto{\pgfqpoint{4.395908in}{2.334333in}}{\pgfqpoint{4.406507in}{2.338724in}}{\pgfqpoint{4.414320in}{2.346537in}}%
\pgfpathcurveto{\pgfqpoint{4.422134in}{2.354351in}}{\pgfqpoint{4.426524in}{2.364950in}}{\pgfqpoint{4.426524in}{2.376000in}}%
\pgfpathcurveto{\pgfqpoint{4.426524in}{2.387050in}}{\pgfqpoint{4.422134in}{2.397649in}}{\pgfqpoint{4.414320in}{2.405463in}}%
\pgfpathcurveto{\pgfqpoint{4.406507in}{2.413276in}}{\pgfqpoint{4.395908in}{2.417667in}}{\pgfqpoint{4.384857in}{2.417667in}}%
\pgfpathcurveto{\pgfqpoint{4.373807in}{2.417667in}}{\pgfqpoint{4.363208in}{2.413276in}}{\pgfqpoint{4.355395in}{2.405463in}}%
\pgfpathcurveto{\pgfqpoint{4.347581in}{2.397649in}}{\pgfqpoint{4.343191in}{2.387050in}}{\pgfqpoint{4.343191in}{2.376000in}}%
\pgfpathcurveto{\pgfqpoint{4.343191in}{2.364950in}}{\pgfqpoint{4.347581in}{2.354351in}}{\pgfqpoint{4.355395in}{2.346537in}}%
\pgfpathcurveto{\pgfqpoint{4.363208in}{2.338724in}}{\pgfqpoint{4.373807in}{2.334333in}}{\pgfqpoint{4.384857in}{2.334333in}}%
\pgfpathclose%
\pgfusepath{stroke,fill}%
\end{pgfscope}%
\begin{pgfscope}%
\pgfpathrectangle{\pgfqpoint{0.800000in}{0.528000in}}{\pgfqpoint{4.960000in}{3.696000in}}%
\pgfusepath{clip}%
\pgfsetbuttcap%
\pgfsetroundjoin%
\definecolor{currentfill}{rgb}{0.000000,0.000000,0.000000}%
\pgfsetfillcolor{currentfill}%
\pgfsetlinewidth{1.003750pt}%
\definecolor{currentstroke}{rgb}{0.000000,0.000000,0.000000}%
\pgfsetstrokecolor{currentstroke}%
\pgfsetdash{}{0pt}%
\pgfpathmoveto{\pgfqpoint{4.384857in}{2.334333in}}%
\pgfpathcurveto{\pgfqpoint{4.395908in}{2.334333in}}{\pgfqpoint{4.406507in}{2.338724in}}{\pgfqpoint{4.414320in}{2.346537in}}%
\pgfpathcurveto{\pgfqpoint{4.422134in}{2.354351in}}{\pgfqpoint{4.426524in}{2.364950in}}{\pgfqpoint{4.426524in}{2.376000in}}%
\pgfpathcurveto{\pgfqpoint{4.426524in}{2.387050in}}{\pgfqpoint{4.422134in}{2.397649in}}{\pgfqpoint{4.414320in}{2.405463in}}%
\pgfpathcurveto{\pgfqpoint{4.406507in}{2.413276in}}{\pgfqpoint{4.395908in}{2.417667in}}{\pgfqpoint{4.384857in}{2.417667in}}%
\pgfpathcurveto{\pgfqpoint{4.373807in}{2.417667in}}{\pgfqpoint{4.363208in}{2.413276in}}{\pgfqpoint{4.355395in}{2.405463in}}%
\pgfpathcurveto{\pgfqpoint{4.347581in}{2.397649in}}{\pgfqpoint{4.343191in}{2.387050in}}{\pgfqpoint{4.343191in}{2.376000in}}%
\pgfpathcurveto{\pgfqpoint{4.343191in}{2.364950in}}{\pgfqpoint{4.347581in}{2.354351in}}{\pgfqpoint{4.355395in}{2.346537in}}%
\pgfpathcurveto{\pgfqpoint{4.363208in}{2.338724in}}{\pgfqpoint{4.373807in}{2.334333in}}{\pgfqpoint{4.384857in}{2.334333in}}%
\pgfpathclose%
\pgfusepath{stroke,fill}%
\end{pgfscope}%
\begin{pgfscope}%
\pgfpathrectangle{\pgfqpoint{0.800000in}{0.528000in}}{\pgfqpoint{4.960000in}{3.696000in}}%
\pgfusepath{clip}%
\pgfsetbuttcap%
\pgfsetroundjoin%
\definecolor{currentfill}{rgb}{0.000000,0.000000,0.000000}%
\pgfsetfillcolor{currentfill}%
\pgfsetlinewidth{1.003750pt}%
\definecolor{currentstroke}{rgb}{0.000000,0.000000,0.000000}%
\pgfsetstrokecolor{currentstroke}%
\pgfsetdash{}{0pt}%
\pgfpathmoveto{\pgfqpoint{4.384857in}{2.334333in}}%
\pgfpathcurveto{\pgfqpoint{4.395908in}{2.334333in}}{\pgfqpoint{4.406507in}{2.338724in}}{\pgfqpoint{4.414320in}{2.346537in}}%
\pgfpathcurveto{\pgfqpoint{4.422134in}{2.354351in}}{\pgfqpoint{4.426524in}{2.364950in}}{\pgfqpoint{4.426524in}{2.376000in}}%
\pgfpathcurveto{\pgfqpoint{4.426524in}{2.387050in}}{\pgfqpoint{4.422134in}{2.397649in}}{\pgfqpoint{4.414320in}{2.405463in}}%
\pgfpathcurveto{\pgfqpoint{4.406507in}{2.413276in}}{\pgfqpoint{4.395908in}{2.417667in}}{\pgfqpoint{4.384857in}{2.417667in}}%
\pgfpathcurveto{\pgfqpoint{4.373807in}{2.417667in}}{\pgfqpoint{4.363208in}{2.413276in}}{\pgfqpoint{4.355395in}{2.405463in}}%
\pgfpathcurveto{\pgfqpoint{4.347581in}{2.397649in}}{\pgfqpoint{4.343191in}{2.387050in}}{\pgfqpoint{4.343191in}{2.376000in}}%
\pgfpathcurveto{\pgfqpoint{4.343191in}{2.364950in}}{\pgfqpoint{4.347581in}{2.354351in}}{\pgfqpoint{4.355395in}{2.346537in}}%
\pgfpathcurveto{\pgfqpoint{4.363208in}{2.338724in}}{\pgfqpoint{4.373807in}{2.334333in}}{\pgfqpoint{4.384857in}{2.334333in}}%
\pgfpathclose%
\pgfusepath{stroke,fill}%
\end{pgfscope}%
\begin{pgfscope}%
\pgfpathrectangle{\pgfqpoint{0.800000in}{0.528000in}}{\pgfqpoint{4.960000in}{3.696000in}}%
\pgfusepath{clip}%
\pgfsetbuttcap%
\pgfsetroundjoin%
\definecolor{currentfill}{rgb}{0.000000,0.000000,0.000000}%
\pgfsetfillcolor{currentfill}%
\pgfsetlinewidth{1.003750pt}%
\definecolor{currentstroke}{rgb}{0.000000,0.000000,0.000000}%
\pgfsetstrokecolor{currentstroke}%
\pgfsetdash{}{0pt}%
\pgfpathmoveto{\pgfqpoint{4.384857in}{2.334333in}}%
\pgfpathcurveto{\pgfqpoint{4.395908in}{2.334333in}}{\pgfqpoint{4.406507in}{2.338724in}}{\pgfqpoint{4.414320in}{2.346537in}}%
\pgfpathcurveto{\pgfqpoint{4.422134in}{2.354351in}}{\pgfqpoint{4.426524in}{2.364950in}}{\pgfqpoint{4.426524in}{2.376000in}}%
\pgfpathcurveto{\pgfqpoint{4.426524in}{2.387050in}}{\pgfqpoint{4.422134in}{2.397649in}}{\pgfqpoint{4.414320in}{2.405463in}}%
\pgfpathcurveto{\pgfqpoint{4.406507in}{2.413276in}}{\pgfqpoint{4.395908in}{2.417667in}}{\pgfqpoint{4.384857in}{2.417667in}}%
\pgfpathcurveto{\pgfqpoint{4.373807in}{2.417667in}}{\pgfqpoint{4.363208in}{2.413276in}}{\pgfqpoint{4.355395in}{2.405463in}}%
\pgfpathcurveto{\pgfqpoint{4.347581in}{2.397649in}}{\pgfqpoint{4.343191in}{2.387050in}}{\pgfqpoint{4.343191in}{2.376000in}}%
\pgfpathcurveto{\pgfqpoint{4.343191in}{2.364950in}}{\pgfqpoint{4.347581in}{2.354351in}}{\pgfqpoint{4.355395in}{2.346537in}}%
\pgfpathcurveto{\pgfqpoint{4.363208in}{2.338724in}}{\pgfqpoint{4.373807in}{2.334333in}}{\pgfqpoint{4.384857in}{2.334333in}}%
\pgfpathclose%
\pgfusepath{stroke,fill}%
\end{pgfscope}%
\begin{pgfscope}%
\pgfpathrectangle{\pgfqpoint{0.800000in}{0.528000in}}{\pgfqpoint{4.960000in}{3.696000in}}%
\pgfusepath{clip}%
\pgfsetbuttcap%
\pgfsetroundjoin%
\definecolor{currentfill}{rgb}{0.000000,0.000000,0.000000}%
\pgfsetfillcolor{currentfill}%
\pgfsetlinewidth{1.003750pt}%
\definecolor{currentstroke}{rgb}{0.000000,0.000000,0.000000}%
\pgfsetstrokecolor{currentstroke}%
\pgfsetdash{}{0pt}%
\pgfpathmoveto{\pgfqpoint{4.384857in}{2.334333in}}%
\pgfpathcurveto{\pgfqpoint{4.395908in}{2.334333in}}{\pgfqpoint{4.406507in}{2.338724in}}{\pgfqpoint{4.414320in}{2.346537in}}%
\pgfpathcurveto{\pgfqpoint{4.422134in}{2.354351in}}{\pgfqpoint{4.426524in}{2.364950in}}{\pgfqpoint{4.426524in}{2.376000in}}%
\pgfpathcurveto{\pgfqpoint{4.426524in}{2.387050in}}{\pgfqpoint{4.422134in}{2.397649in}}{\pgfqpoint{4.414320in}{2.405463in}}%
\pgfpathcurveto{\pgfqpoint{4.406507in}{2.413276in}}{\pgfqpoint{4.395908in}{2.417667in}}{\pgfqpoint{4.384857in}{2.417667in}}%
\pgfpathcurveto{\pgfqpoint{4.373807in}{2.417667in}}{\pgfqpoint{4.363208in}{2.413276in}}{\pgfqpoint{4.355395in}{2.405463in}}%
\pgfpathcurveto{\pgfqpoint{4.347581in}{2.397649in}}{\pgfqpoint{4.343191in}{2.387050in}}{\pgfqpoint{4.343191in}{2.376000in}}%
\pgfpathcurveto{\pgfqpoint{4.343191in}{2.364950in}}{\pgfqpoint{4.347581in}{2.354351in}}{\pgfqpoint{4.355395in}{2.346537in}}%
\pgfpathcurveto{\pgfqpoint{4.363208in}{2.338724in}}{\pgfqpoint{4.373807in}{2.334333in}}{\pgfqpoint{4.384857in}{2.334333in}}%
\pgfpathclose%
\pgfusepath{stroke,fill}%
\end{pgfscope}%
\begin{pgfscope}%
\pgfpathrectangle{\pgfqpoint{0.800000in}{0.528000in}}{\pgfqpoint{4.960000in}{3.696000in}}%
\pgfusepath{clip}%
\pgfsetbuttcap%
\pgfsetroundjoin%
\definecolor{currentfill}{rgb}{0.000000,0.000000,0.000000}%
\pgfsetfillcolor{currentfill}%
\pgfsetlinewidth{1.003750pt}%
\definecolor{currentstroke}{rgb}{0.000000,0.000000,0.000000}%
\pgfsetstrokecolor{currentstroke}%
\pgfsetdash{}{0pt}%
\pgfpathmoveto{\pgfqpoint{4.384857in}{2.334333in}}%
\pgfpathcurveto{\pgfqpoint{4.395908in}{2.334333in}}{\pgfqpoint{4.406507in}{2.338724in}}{\pgfqpoint{4.414320in}{2.346537in}}%
\pgfpathcurveto{\pgfqpoint{4.422134in}{2.354351in}}{\pgfqpoint{4.426524in}{2.364950in}}{\pgfqpoint{4.426524in}{2.376000in}}%
\pgfpathcurveto{\pgfqpoint{4.426524in}{2.387050in}}{\pgfqpoint{4.422134in}{2.397649in}}{\pgfqpoint{4.414320in}{2.405463in}}%
\pgfpathcurveto{\pgfqpoint{4.406507in}{2.413276in}}{\pgfqpoint{4.395908in}{2.417667in}}{\pgfqpoint{4.384857in}{2.417667in}}%
\pgfpathcurveto{\pgfqpoint{4.373807in}{2.417667in}}{\pgfqpoint{4.363208in}{2.413276in}}{\pgfqpoint{4.355395in}{2.405463in}}%
\pgfpathcurveto{\pgfqpoint{4.347581in}{2.397649in}}{\pgfqpoint{4.343191in}{2.387050in}}{\pgfqpoint{4.343191in}{2.376000in}}%
\pgfpathcurveto{\pgfqpoint{4.343191in}{2.364950in}}{\pgfqpoint{4.347581in}{2.354351in}}{\pgfqpoint{4.355395in}{2.346537in}}%
\pgfpathcurveto{\pgfqpoint{4.363208in}{2.338724in}}{\pgfqpoint{4.373807in}{2.334333in}}{\pgfqpoint{4.384857in}{2.334333in}}%
\pgfpathclose%
\pgfusepath{stroke,fill}%
\end{pgfscope}%
\begin{pgfscope}%
\pgfpathrectangle{\pgfqpoint{0.800000in}{0.528000in}}{\pgfqpoint{4.960000in}{3.696000in}}%
\pgfusepath{clip}%
\pgfsetbuttcap%
\pgfsetroundjoin%
\definecolor{currentfill}{rgb}{0.000000,0.000000,0.000000}%
\pgfsetfillcolor{currentfill}%
\pgfsetlinewidth{1.003750pt}%
\definecolor{currentstroke}{rgb}{0.000000,0.000000,0.000000}%
\pgfsetstrokecolor{currentstroke}%
\pgfsetdash{}{0pt}%
\pgfpathmoveto{\pgfqpoint{4.384857in}{2.334333in}}%
\pgfpathcurveto{\pgfqpoint{4.395908in}{2.334333in}}{\pgfqpoint{4.406507in}{2.338724in}}{\pgfqpoint{4.414320in}{2.346537in}}%
\pgfpathcurveto{\pgfqpoint{4.422134in}{2.354351in}}{\pgfqpoint{4.426524in}{2.364950in}}{\pgfqpoint{4.426524in}{2.376000in}}%
\pgfpathcurveto{\pgfqpoint{4.426524in}{2.387050in}}{\pgfqpoint{4.422134in}{2.397649in}}{\pgfqpoint{4.414320in}{2.405463in}}%
\pgfpathcurveto{\pgfqpoint{4.406507in}{2.413276in}}{\pgfqpoint{4.395908in}{2.417667in}}{\pgfqpoint{4.384857in}{2.417667in}}%
\pgfpathcurveto{\pgfqpoint{4.373807in}{2.417667in}}{\pgfqpoint{4.363208in}{2.413276in}}{\pgfqpoint{4.355395in}{2.405463in}}%
\pgfpathcurveto{\pgfqpoint{4.347581in}{2.397649in}}{\pgfqpoint{4.343191in}{2.387050in}}{\pgfqpoint{4.343191in}{2.376000in}}%
\pgfpathcurveto{\pgfqpoint{4.343191in}{2.364950in}}{\pgfqpoint{4.347581in}{2.354351in}}{\pgfqpoint{4.355395in}{2.346537in}}%
\pgfpathcurveto{\pgfqpoint{4.363208in}{2.338724in}}{\pgfqpoint{4.373807in}{2.334333in}}{\pgfqpoint{4.384857in}{2.334333in}}%
\pgfpathclose%
\pgfusepath{stroke,fill}%
\end{pgfscope}%
\begin{pgfscope}%
\pgfpathrectangle{\pgfqpoint{0.800000in}{0.528000in}}{\pgfqpoint{4.960000in}{3.696000in}}%
\pgfusepath{clip}%
\pgfsetbuttcap%
\pgfsetroundjoin%
\definecolor{currentfill}{rgb}{0.000000,0.000000,0.000000}%
\pgfsetfillcolor{currentfill}%
\pgfsetlinewidth{1.003750pt}%
\definecolor{currentstroke}{rgb}{0.000000,0.000000,0.000000}%
\pgfsetstrokecolor{currentstroke}%
\pgfsetdash{}{0pt}%
\pgfpathmoveto{\pgfqpoint{4.384857in}{2.334333in}}%
\pgfpathcurveto{\pgfqpoint{4.395908in}{2.334333in}}{\pgfqpoint{4.406507in}{2.338724in}}{\pgfqpoint{4.414320in}{2.346537in}}%
\pgfpathcurveto{\pgfqpoint{4.422134in}{2.354351in}}{\pgfqpoint{4.426524in}{2.364950in}}{\pgfqpoint{4.426524in}{2.376000in}}%
\pgfpathcurveto{\pgfqpoint{4.426524in}{2.387050in}}{\pgfqpoint{4.422134in}{2.397649in}}{\pgfqpoint{4.414320in}{2.405463in}}%
\pgfpathcurveto{\pgfqpoint{4.406507in}{2.413276in}}{\pgfqpoint{4.395908in}{2.417667in}}{\pgfqpoint{4.384857in}{2.417667in}}%
\pgfpathcurveto{\pgfqpoint{4.373807in}{2.417667in}}{\pgfqpoint{4.363208in}{2.413276in}}{\pgfqpoint{4.355395in}{2.405463in}}%
\pgfpathcurveto{\pgfqpoint{4.347581in}{2.397649in}}{\pgfqpoint{4.343191in}{2.387050in}}{\pgfqpoint{4.343191in}{2.376000in}}%
\pgfpathcurveto{\pgfqpoint{4.343191in}{2.364950in}}{\pgfqpoint{4.347581in}{2.354351in}}{\pgfqpoint{4.355395in}{2.346537in}}%
\pgfpathcurveto{\pgfqpoint{4.363208in}{2.338724in}}{\pgfqpoint{4.373807in}{2.334333in}}{\pgfqpoint{4.384857in}{2.334333in}}%
\pgfpathclose%
\pgfusepath{stroke,fill}%
\end{pgfscope}%
\begin{pgfscope}%
\pgfpathrectangle{\pgfqpoint{0.800000in}{0.528000in}}{\pgfqpoint{4.960000in}{3.696000in}}%
\pgfusepath{clip}%
\pgfsetbuttcap%
\pgfsetroundjoin%
\definecolor{currentfill}{rgb}{0.000000,0.000000,0.000000}%
\pgfsetfillcolor{currentfill}%
\pgfsetlinewidth{1.003750pt}%
\definecolor{currentstroke}{rgb}{0.000000,0.000000,0.000000}%
\pgfsetstrokecolor{currentstroke}%
\pgfsetdash{}{0pt}%
\pgfpathmoveto{\pgfqpoint{4.384857in}{2.334333in}}%
\pgfpathcurveto{\pgfqpoint{4.395908in}{2.334333in}}{\pgfqpoint{4.406507in}{2.338724in}}{\pgfqpoint{4.414320in}{2.346537in}}%
\pgfpathcurveto{\pgfqpoint{4.422134in}{2.354351in}}{\pgfqpoint{4.426524in}{2.364950in}}{\pgfqpoint{4.426524in}{2.376000in}}%
\pgfpathcurveto{\pgfqpoint{4.426524in}{2.387050in}}{\pgfqpoint{4.422134in}{2.397649in}}{\pgfqpoint{4.414320in}{2.405463in}}%
\pgfpathcurveto{\pgfqpoint{4.406507in}{2.413276in}}{\pgfqpoint{4.395908in}{2.417667in}}{\pgfqpoint{4.384857in}{2.417667in}}%
\pgfpathcurveto{\pgfqpoint{4.373807in}{2.417667in}}{\pgfqpoint{4.363208in}{2.413276in}}{\pgfqpoint{4.355395in}{2.405463in}}%
\pgfpathcurveto{\pgfqpoint{4.347581in}{2.397649in}}{\pgfqpoint{4.343191in}{2.387050in}}{\pgfqpoint{4.343191in}{2.376000in}}%
\pgfpathcurveto{\pgfqpoint{4.343191in}{2.364950in}}{\pgfqpoint{4.347581in}{2.354351in}}{\pgfqpoint{4.355395in}{2.346537in}}%
\pgfpathcurveto{\pgfqpoint{4.363208in}{2.338724in}}{\pgfqpoint{4.373807in}{2.334333in}}{\pgfqpoint{4.384857in}{2.334333in}}%
\pgfpathclose%
\pgfusepath{stroke,fill}%
\end{pgfscope}%
\begin{pgfscope}%
\pgfpathrectangle{\pgfqpoint{0.800000in}{0.528000in}}{\pgfqpoint{4.960000in}{3.696000in}}%
\pgfusepath{clip}%
\pgfsetbuttcap%
\pgfsetroundjoin%
\definecolor{currentfill}{rgb}{0.000000,0.000000,0.000000}%
\pgfsetfillcolor{currentfill}%
\pgfsetlinewidth{1.003750pt}%
\definecolor{currentstroke}{rgb}{0.000000,0.000000,0.000000}%
\pgfsetstrokecolor{currentstroke}%
\pgfsetdash{}{0pt}%
\pgfpathmoveto{\pgfqpoint{4.384857in}{2.334333in}}%
\pgfpathcurveto{\pgfqpoint{4.395908in}{2.334333in}}{\pgfqpoint{4.406507in}{2.338724in}}{\pgfqpoint{4.414320in}{2.346537in}}%
\pgfpathcurveto{\pgfqpoint{4.422134in}{2.354351in}}{\pgfqpoint{4.426524in}{2.364950in}}{\pgfqpoint{4.426524in}{2.376000in}}%
\pgfpathcurveto{\pgfqpoint{4.426524in}{2.387050in}}{\pgfqpoint{4.422134in}{2.397649in}}{\pgfqpoint{4.414320in}{2.405463in}}%
\pgfpathcurveto{\pgfqpoint{4.406507in}{2.413276in}}{\pgfqpoint{4.395908in}{2.417667in}}{\pgfqpoint{4.384857in}{2.417667in}}%
\pgfpathcurveto{\pgfqpoint{4.373807in}{2.417667in}}{\pgfqpoint{4.363208in}{2.413276in}}{\pgfqpoint{4.355395in}{2.405463in}}%
\pgfpathcurveto{\pgfqpoint{4.347581in}{2.397649in}}{\pgfqpoint{4.343191in}{2.387050in}}{\pgfqpoint{4.343191in}{2.376000in}}%
\pgfpathcurveto{\pgfqpoint{4.343191in}{2.364950in}}{\pgfqpoint{4.347581in}{2.354351in}}{\pgfqpoint{4.355395in}{2.346537in}}%
\pgfpathcurveto{\pgfqpoint{4.363208in}{2.338724in}}{\pgfqpoint{4.373807in}{2.334333in}}{\pgfqpoint{4.384857in}{2.334333in}}%
\pgfpathclose%
\pgfusepath{stroke,fill}%
\end{pgfscope}%
\begin{pgfscope}%
\pgfpathrectangle{\pgfqpoint{0.800000in}{0.528000in}}{\pgfqpoint{4.960000in}{3.696000in}}%
\pgfusepath{clip}%
\pgfsetbuttcap%
\pgfsetroundjoin%
\definecolor{currentfill}{rgb}{0.000000,0.000000,0.000000}%
\pgfsetfillcolor{currentfill}%
\pgfsetlinewidth{1.003750pt}%
\definecolor{currentstroke}{rgb}{0.000000,0.000000,0.000000}%
\pgfsetstrokecolor{currentstroke}%
\pgfsetdash{}{0pt}%
\pgfpathmoveto{\pgfqpoint{4.384857in}{2.334333in}}%
\pgfpathcurveto{\pgfqpoint{4.395908in}{2.334333in}}{\pgfqpoint{4.406507in}{2.338724in}}{\pgfqpoint{4.414320in}{2.346537in}}%
\pgfpathcurveto{\pgfqpoint{4.422134in}{2.354351in}}{\pgfqpoint{4.426524in}{2.364950in}}{\pgfqpoint{4.426524in}{2.376000in}}%
\pgfpathcurveto{\pgfqpoint{4.426524in}{2.387050in}}{\pgfqpoint{4.422134in}{2.397649in}}{\pgfqpoint{4.414320in}{2.405463in}}%
\pgfpathcurveto{\pgfqpoint{4.406507in}{2.413276in}}{\pgfqpoint{4.395908in}{2.417667in}}{\pgfqpoint{4.384857in}{2.417667in}}%
\pgfpathcurveto{\pgfqpoint{4.373807in}{2.417667in}}{\pgfqpoint{4.363208in}{2.413276in}}{\pgfqpoint{4.355395in}{2.405463in}}%
\pgfpathcurveto{\pgfqpoint{4.347581in}{2.397649in}}{\pgfqpoint{4.343191in}{2.387050in}}{\pgfqpoint{4.343191in}{2.376000in}}%
\pgfpathcurveto{\pgfqpoint{4.343191in}{2.364950in}}{\pgfqpoint{4.347581in}{2.354351in}}{\pgfqpoint{4.355395in}{2.346537in}}%
\pgfpathcurveto{\pgfqpoint{4.363208in}{2.338724in}}{\pgfqpoint{4.373807in}{2.334333in}}{\pgfqpoint{4.384857in}{2.334333in}}%
\pgfpathclose%
\pgfusepath{stroke,fill}%
\end{pgfscope}%
\begin{pgfscope}%
\pgfpathrectangle{\pgfqpoint{0.800000in}{0.528000in}}{\pgfqpoint{4.960000in}{3.696000in}}%
\pgfusepath{clip}%
\pgfsetbuttcap%
\pgfsetroundjoin%
\definecolor{currentfill}{rgb}{0.000000,0.000000,0.000000}%
\pgfsetfillcolor{currentfill}%
\pgfsetlinewidth{1.003750pt}%
\definecolor{currentstroke}{rgb}{0.000000,0.000000,0.000000}%
\pgfsetstrokecolor{currentstroke}%
\pgfsetdash{}{0pt}%
\pgfpathmoveto{\pgfqpoint{4.384857in}{2.334333in}}%
\pgfpathcurveto{\pgfqpoint{4.395908in}{2.334333in}}{\pgfqpoint{4.406507in}{2.338724in}}{\pgfqpoint{4.414320in}{2.346537in}}%
\pgfpathcurveto{\pgfqpoint{4.422134in}{2.354351in}}{\pgfqpoint{4.426524in}{2.364950in}}{\pgfqpoint{4.426524in}{2.376000in}}%
\pgfpathcurveto{\pgfqpoint{4.426524in}{2.387050in}}{\pgfqpoint{4.422134in}{2.397649in}}{\pgfqpoint{4.414320in}{2.405463in}}%
\pgfpathcurveto{\pgfqpoint{4.406507in}{2.413276in}}{\pgfqpoint{4.395908in}{2.417667in}}{\pgfqpoint{4.384857in}{2.417667in}}%
\pgfpathcurveto{\pgfqpoint{4.373807in}{2.417667in}}{\pgfqpoint{4.363208in}{2.413276in}}{\pgfqpoint{4.355395in}{2.405463in}}%
\pgfpathcurveto{\pgfqpoint{4.347581in}{2.397649in}}{\pgfqpoint{4.343191in}{2.387050in}}{\pgfqpoint{4.343191in}{2.376000in}}%
\pgfpathcurveto{\pgfqpoint{4.343191in}{2.364950in}}{\pgfqpoint{4.347581in}{2.354351in}}{\pgfqpoint{4.355395in}{2.346537in}}%
\pgfpathcurveto{\pgfqpoint{4.363208in}{2.338724in}}{\pgfqpoint{4.373807in}{2.334333in}}{\pgfqpoint{4.384857in}{2.334333in}}%
\pgfpathclose%
\pgfusepath{stroke,fill}%
\end{pgfscope}%
\begin{pgfscope}%
\pgfpathrectangle{\pgfqpoint{0.800000in}{0.528000in}}{\pgfqpoint{4.960000in}{3.696000in}}%
\pgfusepath{clip}%
\pgfsetbuttcap%
\pgfsetroundjoin%
\definecolor{currentfill}{rgb}{0.000000,0.000000,0.000000}%
\pgfsetfillcolor{currentfill}%
\pgfsetlinewidth{1.003750pt}%
\definecolor{currentstroke}{rgb}{0.000000,0.000000,0.000000}%
\pgfsetstrokecolor{currentstroke}%
\pgfsetdash{}{0pt}%
\pgfpathmoveto{\pgfqpoint{4.384857in}{2.334333in}}%
\pgfpathcurveto{\pgfqpoint{4.395908in}{2.334333in}}{\pgfqpoint{4.406507in}{2.338724in}}{\pgfqpoint{4.414320in}{2.346537in}}%
\pgfpathcurveto{\pgfqpoint{4.422134in}{2.354351in}}{\pgfqpoint{4.426524in}{2.364950in}}{\pgfqpoint{4.426524in}{2.376000in}}%
\pgfpathcurveto{\pgfqpoint{4.426524in}{2.387050in}}{\pgfqpoint{4.422134in}{2.397649in}}{\pgfqpoint{4.414320in}{2.405463in}}%
\pgfpathcurveto{\pgfqpoint{4.406507in}{2.413276in}}{\pgfqpoint{4.395908in}{2.417667in}}{\pgfqpoint{4.384857in}{2.417667in}}%
\pgfpathcurveto{\pgfqpoint{4.373807in}{2.417667in}}{\pgfqpoint{4.363208in}{2.413276in}}{\pgfqpoint{4.355395in}{2.405463in}}%
\pgfpathcurveto{\pgfqpoint{4.347581in}{2.397649in}}{\pgfqpoint{4.343191in}{2.387050in}}{\pgfqpoint{4.343191in}{2.376000in}}%
\pgfpathcurveto{\pgfqpoint{4.343191in}{2.364950in}}{\pgfqpoint{4.347581in}{2.354351in}}{\pgfqpoint{4.355395in}{2.346537in}}%
\pgfpathcurveto{\pgfqpoint{4.363208in}{2.338724in}}{\pgfqpoint{4.373807in}{2.334333in}}{\pgfqpoint{4.384857in}{2.334333in}}%
\pgfpathclose%
\pgfusepath{stroke,fill}%
\end{pgfscope}%
\begin{pgfscope}%
\pgfpathrectangle{\pgfqpoint{0.800000in}{0.528000in}}{\pgfqpoint{4.960000in}{3.696000in}}%
\pgfusepath{clip}%
\pgfsetbuttcap%
\pgfsetroundjoin%
\definecolor{currentfill}{rgb}{0.000000,0.000000,0.000000}%
\pgfsetfillcolor{currentfill}%
\pgfsetlinewidth{1.003750pt}%
\definecolor{currentstroke}{rgb}{0.000000,0.000000,0.000000}%
\pgfsetstrokecolor{currentstroke}%
\pgfsetdash{}{0pt}%
\pgfpathmoveto{\pgfqpoint{4.384857in}{2.334333in}}%
\pgfpathcurveto{\pgfqpoint{4.395908in}{2.334333in}}{\pgfqpoint{4.406507in}{2.338724in}}{\pgfqpoint{4.414320in}{2.346537in}}%
\pgfpathcurveto{\pgfqpoint{4.422134in}{2.354351in}}{\pgfqpoint{4.426524in}{2.364950in}}{\pgfqpoint{4.426524in}{2.376000in}}%
\pgfpathcurveto{\pgfqpoint{4.426524in}{2.387050in}}{\pgfqpoint{4.422134in}{2.397649in}}{\pgfqpoint{4.414320in}{2.405463in}}%
\pgfpathcurveto{\pgfqpoint{4.406507in}{2.413276in}}{\pgfqpoint{4.395908in}{2.417667in}}{\pgfqpoint{4.384857in}{2.417667in}}%
\pgfpathcurveto{\pgfqpoint{4.373807in}{2.417667in}}{\pgfqpoint{4.363208in}{2.413276in}}{\pgfqpoint{4.355395in}{2.405463in}}%
\pgfpathcurveto{\pgfqpoint{4.347581in}{2.397649in}}{\pgfqpoint{4.343191in}{2.387050in}}{\pgfqpoint{4.343191in}{2.376000in}}%
\pgfpathcurveto{\pgfqpoint{4.343191in}{2.364950in}}{\pgfqpoint{4.347581in}{2.354351in}}{\pgfqpoint{4.355395in}{2.346537in}}%
\pgfpathcurveto{\pgfqpoint{4.363208in}{2.338724in}}{\pgfqpoint{4.373807in}{2.334333in}}{\pgfqpoint{4.384857in}{2.334333in}}%
\pgfpathclose%
\pgfusepath{stroke,fill}%
\end{pgfscope}%
\begin{pgfscope}%
\pgfpathrectangle{\pgfqpoint{0.800000in}{0.528000in}}{\pgfqpoint{4.960000in}{3.696000in}}%
\pgfusepath{clip}%
\pgfsetbuttcap%
\pgfsetroundjoin%
\definecolor{currentfill}{rgb}{0.000000,0.000000,0.000000}%
\pgfsetfillcolor{currentfill}%
\pgfsetlinewidth{1.003750pt}%
\definecolor{currentstroke}{rgb}{0.000000,0.000000,0.000000}%
\pgfsetstrokecolor{currentstroke}%
\pgfsetdash{}{0pt}%
\pgfpathmoveto{\pgfqpoint{4.384857in}{2.334333in}}%
\pgfpathcurveto{\pgfqpoint{4.395908in}{2.334333in}}{\pgfqpoint{4.406507in}{2.338724in}}{\pgfqpoint{4.414320in}{2.346537in}}%
\pgfpathcurveto{\pgfqpoint{4.422134in}{2.354351in}}{\pgfqpoint{4.426524in}{2.364950in}}{\pgfqpoint{4.426524in}{2.376000in}}%
\pgfpathcurveto{\pgfqpoint{4.426524in}{2.387050in}}{\pgfqpoint{4.422134in}{2.397649in}}{\pgfqpoint{4.414320in}{2.405463in}}%
\pgfpathcurveto{\pgfqpoint{4.406507in}{2.413276in}}{\pgfqpoint{4.395908in}{2.417667in}}{\pgfqpoint{4.384857in}{2.417667in}}%
\pgfpathcurveto{\pgfqpoint{4.373807in}{2.417667in}}{\pgfqpoint{4.363208in}{2.413276in}}{\pgfqpoint{4.355395in}{2.405463in}}%
\pgfpathcurveto{\pgfqpoint{4.347581in}{2.397649in}}{\pgfqpoint{4.343191in}{2.387050in}}{\pgfqpoint{4.343191in}{2.376000in}}%
\pgfpathcurveto{\pgfqpoint{4.343191in}{2.364950in}}{\pgfqpoint{4.347581in}{2.354351in}}{\pgfqpoint{4.355395in}{2.346537in}}%
\pgfpathcurveto{\pgfqpoint{4.363208in}{2.338724in}}{\pgfqpoint{4.373807in}{2.334333in}}{\pgfqpoint{4.384857in}{2.334333in}}%
\pgfpathclose%
\pgfusepath{stroke,fill}%
\end{pgfscope}%
\begin{pgfscope}%
\pgfpathrectangle{\pgfqpoint{0.800000in}{0.528000in}}{\pgfqpoint{4.960000in}{3.696000in}}%
\pgfusepath{clip}%
\pgfsetbuttcap%
\pgfsetroundjoin%
\definecolor{currentfill}{rgb}{0.000000,0.000000,0.000000}%
\pgfsetfillcolor{currentfill}%
\pgfsetlinewidth{1.003750pt}%
\definecolor{currentstroke}{rgb}{0.000000,0.000000,0.000000}%
\pgfsetstrokecolor{currentstroke}%
\pgfsetdash{}{0pt}%
\pgfpathmoveto{\pgfqpoint{4.384857in}{2.334333in}}%
\pgfpathcurveto{\pgfqpoint{4.395908in}{2.334333in}}{\pgfqpoint{4.406507in}{2.338724in}}{\pgfqpoint{4.414320in}{2.346537in}}%
\pgfpathcurveto{\pgfqpoint{4.422134in}{2.354351in}}{\pgfqpoint{4.426524in}{2.364950in}}{\pgfqpoint{4.426524in}{2.376000in}}%
\pgfpathcurveto{\pgfqpoint{4.426524in}{2.387050in}}{\pgfqpoint{4.422134in}{2.397649in}}{\pgfqpoint{4.414320in}{2.405463in}}%
\pgfpathcurveto{\pgfqpoint{4.406507in}{2.413276in}}{\pgfqpoint{4.395908in}{2.417667in}}{\pgfqpoint{4.384857in}{2.417667in}}%
\pgfpathcurveto{\pgfqpoint{4.373807in}{2.417667in}}{\pgfqpoint{4.363208in}{2.413276in}}{\pgfqpoint{4.355395in}{2.405463in}}%
\pgfpathcurveto{\pgfqpoint{4.347581in}{2.397649in}}{\pgfqpoint{4.343191in}{2.387050in}}{\pgfqpoint{4.343191in}{2.376000in}}%
\pgfpathcurveto{\pgfqpoint{4.343191in}{2.364950in}}{\pgfqpoint{4.347581in}{2.354351in}}{\pgfqpoint{4.355395in}{2.346537in}}%
\pgfpathcurveto{\pgfqpoint{4.363208in}{2.338724in}}{\pgfqpoint{4.373807in}{2.334333in}}{\pgfqpoint{4.384857in}{2.334333in}}%
\pgfpathclose%
\pgfusepath{stroke,fill}%
\end{pgfscope}%
\begin{pgfscope}%
\pgfpathrectangle{\pgfqpoint{0.800000in}{0.528000in}}{\pgfqpoint{4.960000in}{3.696000in}}%
\pgfusepath{clip}%
\pgfsetbuttcap%
\pgfsetroundjoin%
\definecolor{currentfill}{rgb}{0.000000,0.000000,0.000000}%
\pgfsetfillcolor{currentfill}%
\pgfsetlinewidth{1.003750pt}%
\definecolor{currentstroke}{rgb}{0.000000,0.000000,0.000000}%
\pgfsetstrokecolor{currentstroke}%
\pgfsetdash{}{0pt}%
\pgfpathmoveto{\pgfqpoint{4.384857in}{2.334333in}}%
\pgfpathcurveto{\pgfqpoint{4.395908in}{2.334333in}}{\pgfqpoint{4.406507in}{2.338724in}}{\pgfqpoint{4.414320in}{2.346537in}}%
\pgfpathcurveto{\pgfqpoint{4.422134in}{2.354351in}}{\pgfqpoint{4.426524in}{2.364950in}}{\pgfqpoint{4.426524in}{2.376000in}}%
\pgfpathcurveto{\pgfqpoint{4.426524in}{2.387050in}}{\pgfqpoint{4.422134in}{2.397649in}}{\pgfqpoint{4.414320in}{2.405463in}}%
\pgfpathcurveto{\pgfqpoint{4.406507in}{2.413276in}}{\pgfqpoint{4.395908in}{2.417667in}}{\pgfqpoint{4.384857in}{2.417667in}}%
\pgfpathcurveto{\pgfqpoint{4.373807in}{2.417667in}}{\pgfqpoint{4.363208in}{2.413276in}}{\pgfqpoint{4.355395in}{2.405463in}}%
\pgfpathcurveto{\pgfqpoint{4.347581in}{2.397649in}}{\pgfqpoint{4.343191in}{2.387050in}}{\pgfqpoint{4.343191in}{2.376000in}}%
\pgfpathcurveto{\pgfqpoint{4.343191in}{2.364950in}}{\pgfqpoint{4.347581in}{2.354351in}}{\pgfqpoint{4.355395in}{2.346537in}}%
\pgfpathcurveto{\pgfqpoint{4.363208in}{2.338724in}}{\pgfqpoint{4.373807in}{2.334333in}}{\pgfqpoint{4.384857in}{2.334333in}}%
\pgfpathclose%
\pgfusepath{stroke,fill}%
\end{pgfscope}%
\begin{pgfscope}%
\pgfpathrectangle{\pgfqpoint{0.800000in}{0.528000in}}{\pgfqpoint{4.960000in}{3.696000in}}%
\pgfusepath{clip}%
\pgfsetbuttcap%
\pgfsetroundjoin%
\definecolor{currentfill}{rgb}{0.000000,0.000000,0.000000}%
\pgfsetfillcolor{currentfill}%
\pgfsetlinewidth{1.003750pt}%
\definecolor{currentstroke}{rgb}{0.000000,0.000000,0.000000}%
\pgfsetstrokecolor{currentstroke}%
\pgfsetdash{}{0pt}%
\pgfpathmoveto{\pgfqpoint{4.384857in}{2.334333in}}%
\pgfpathcurveto{\pgfqpoint{4.395908in}{2.334333in}}{\pgfqpoint{4.406507in}{2.338724in}}{\pgfqpoint{4.414320in}{2.346537in}}%
\pgfpathcurveto{\pgfqpoint{4.422134in}{2.354351in}}{\pgfqpoint{4.426524in}{2.364950in}}{\pgfqpoint{4.426524in}{2.376000in}}%
\pgfpathcurveto{\pgfqpoint{4.426524in}{2.387050in}}{\pgfqpoint{4.422134in}{2.397649in}}{\pgfqpoint{4.414320in}{2.405463in}}%
\pgfpathcurveto{\pgfqpoint{4.406507in}{2.413276in}}{\pgfqpoint{4.395908in}{2.417667in}}{\pgfqpoint{4.384857in}{2.417667in}}%
\pgfpathcurveto{\pgfqpoint{4.373807in}{2.417667in}}{\pgfqpoint{4.363208in}{2.413276in}}{\pgfqpoint{4.355395in}{2.405463in}}%
\pgfpathcurveto{\pgfqpoint{4.347581in}{2.397649in}}{\pgfqpoint{4.343191in}{2.387050in}}{\pgfqpoint{4.343191in}{2.376000in}}%
\pgfpathcurveto{\pgfqpoint{4.343191in}{2.364950in}}{\pgfqpoint{4.347581in}{2.354351in}}{\pgfqpoint{4.355395in}{2.346537in}}%
\pgfpathcurveto{\pgfqpoint{4.363208in}{2.338724in}}{\pgfqpoint{4.373807in}{2.334333in}}{\pgfqpoint{4.384857in}{2.334333in}}%
\pgfpathclose%
\pgfusepath{stroke,fill}%
\end{pgfscope}%
\begin{pgfscope}%
\pgfpathrectangle{\pgfqpoint{0.800000in}{0.528000in}}{\pgfqpoint{4.960000in}{3.696000in}}%
\pgfusepath{clip}%
\pgfsetbuttcap%
\pgfsetroundjoin%
\definecolor{currentfill}{rgb}{0.000000,0.000000,0.000000}%
\pgfsetfillcolor{currentfill}%
\pgfsetlinewidth{1.003750pt}%
\definecolor{currentstroke}{rgb}{0.000000,0.000000,0.000000}%
\pgfsetstrokecolor{currentstroke}%
\pgfsetdash{}{0pt}%
\pgfpathmoveto{\pgfqpoint{4.384857in}{2.334333in}}%
\pgfpathcurveto{\pgfqpoint{4.395908in}{2.334333in}}{\pgfqpoint{4.406507in}{2.338724in}}{\pgfqpoint{4.414320in}{2.346537in}}%
\pgfpathcurveto{\pgfqpoint{4.422134in}{2.354351in}}{\pgfqpoint{4.426524in}{2.364950in}}{\pgfqpoint{4.426524in}{2.376000in}}%
\pgfpathcurveto{\pgfqpoint{4.426524in}{2.387050in}}{\pgfqpoint{4.422134in}{2.397649in}}{\pgfqpoint{4.414320in}{2.405463in}}%
\pgfpathcurveto{\pgfqpoint{4.406507in}{2.413276in}}{\pgfqpoint{4.395908in}{2.417667in}}{\pgfqpoint{4.384857in}{2.417667in}}%
\pgfpathcurveto{\pgfqpoint{4.373807in}{2.417667in}}{\pgfqpoint{4.363208in}{2.413276in}}{\pgfqpoint{4.355395in}{2.405463in}}%
\pgfpathcurveto{\pgfqpoint{4.347581in}{2.397649in}}{\pgfqpoint{4.343191in}{2.387050in}}{\pgfqpoint{4.343191in}{2.376000in}}%
\pgfpathcurveto{\pgfqpoint{4.343191in}{2.364950in}}{\pgfqpoint{4.347581in}{2.354351in}}{\pgfqpoint{4.355395in}{2.346537in}}%
\pgfpathcurveto{\pgfqpoint{4.363208in}{2.338724in}}{\pgfqpoint{4.373807in}{2.334333in}}{\pgfqpoint{4.384857in}{2.334333in}}%
\pgfpathclose%
\pgfusepath{stroke,fill}%
\end{pgfscope}%
\begin{pgfscope}%
\pgfpathrectangle{\pgfqpoint{0.800000in}{0.528000in}}{\pgfqpoint{4.960000in}{3.696000in}}%
\pgfusepath{clip}%
\pgfsetbuttcap%
\pgfsetroundjoin%
\definecolor{currentfill}{rgb}{0.000000,0.000000,0.000000}%
\pgfsetfillcolor{currentfill}%
\pgfsetlinewidth{1.003750pt}%
\definecolor{currentstroke}{rgb}{0.000000,0.000000,0.000000}%
\pgfsetstrokecolor{currentstroke}%
\pgfsetdash{}{0pt}%
\pgfpathmoveto{\pgfqpoint{4.384857in}{2.334333in}}%
\pgfpathcurveto{\pgfqpoint{4.395908in}{2.334333in}}{\pgfqpoint{4.406507in}{2.338724in}}{\pgfqpoint{4.414320in}{2.346537in}}%
\pgfpathcurveto{\pgfqpoint{4.422134in}{2.354351in}}{\pgfqpoint{4.426524in}{2.364950in}}{\pgfqpoint{4.426524in}{2.376000in}}%
\pgfpathcurveto{\pgfqpoint{4.426524in}{2.387050in}}{\pgfqpoint{4.422134in}{2.397649in}}{\pgfqpoint{4.414320in}{2.405463in}}%
\pgfpathcurveto{\pgfqpoint{4.406507in}{2.413276in}}{\pgfqpoint{4.395908in}{2.417667in}}{\pgfqpoint{4.384857in}{2.417667in}}%
\pgfpathcurveto{\pgfqpoint{4.373807in}{2.417667in}}{\pgfqpoint{4.363208in}{2.413276in}}{\pgfqpoint{4.355395in}{2.405463in}}%
\pgfpathcurveto{\pgfqpoint{4.347581in}{2.397649in}}{\pgfqpoint{4.343191in}{2.387050in}}{\pgfqpoint{4.343191in}{2.376000in}}%
\pgfpathcurveto{\pgfqpoint{4.343191in}{2.364950in}}{\pgfqpoint{4.347581in}{2.354351in}}{\pgfqpoint{4.355395in}{2.346537in}}%
\pgfpathcurveto{\pgfqpoint{4.363208in}{2.338724in}}{\pgfqpoint{4.373807in}{2.334333in}}{\pgfqpoint{4.384857in}{2.334333in}}%
\pgfpathclose%
\pgfusepath{stroke,fill}%
\end{pgfscope}%
\begin{pgfscope}%
\pgfpathrectangle{\pgfqpoint{0.800000in}{0.528000in}}{\pgfqpoint{4.960000in}{3.696000in}}%
\pgfusepath{clip}%
\pgfsetbuttcap%
\pgfsetroundjoin%
\definecolor{currentfill}{rgb}{0.000000,0.000000,0.000000}%
\pgfsetfillcolor{currentfill}%
\pgfsetlinewidth{1.003750pt}%
\definecolor{currentstroke}{rgb}{0.000000,0.000000,0.000000}%
\pgfsetstrokecolor{currentstroke}%
\pgfsetdash{}{0pt}%
\pgfpathmoveto{\pgfqpoint{4.384857in}{2.334333in}}%
\pgfpathcurveto{\pgfqpoint{4.395908in}{2.334333in}}{\pgfqpoint{4.406507in}{2.338724in}}{\pgfqpoint{4.414320in}{2.346537in}}%
\pgfpathcurveto{\pgfqpoint{4.422134in}{2.354351in}}{\pgfqpoint{4.426524in}{2.364950in}}{\pgfqpoint{4.426524in}{2.376000in}}%
\pgfpathcurveto{\pgfqpoint{4.426524in}{2.387050in}}{\pgfqpoint{4.422134in}{2.397649in}}{\pgfqpoint{4.414320in}{2.405463in}}%
\pgfpathcurveto{\pgfqpoint{4.406507in}{2.413276in}}{\pgfqpoint{4.395908in}{2.417667in}}{\pgfqpoint{4.384857in}{2.417667in}}%
\pgfpathcurveto{\pgfqpoint{4.373807in}{2.417667in}}{\pgfqpoint{4.363208in}{2.413276in}}{\pgfqpoint{4.355395in}{2.405463in}}%
\pgfpathcurveto{\pgfqpoint{4.347581in}{2.397649in}}{\pgfqpoint{4.343191in}{2.387050in}}{\pgfqpoint{4.343191in}{2.376000in}}%
\pgfpathcurveto{\pgfqpoint{4.343191in}{2.364950in}}{\pgfqpoint{4.347581in}{2.354351in}}{\pgfqpoint{4.355395in}{2.346537in}}%
\pgfpathcurveto{\pgfqpoint{4.363208in}{2.338724in}}{\pgfqpoint{4.373807in}{2.334333in}}{\pgfqpoint{4.384857in}{2.334333in}}%
\pgfpathclose%
\pgfusepath{stroke,fill}%
\end{pgfscope}%
\begin{pgfscope}%
\pgfpathrectangle{\pgfqpoint{0.800000in}{0.528000in}}{\pgfqpoint{4.960000in}{3.696000in}}%
\pgfusepath{clip}%
\pgfsetbuttcap%
\pgfsetroundjoin%
\definecolor{currentfill}{rgb}{0.000000,0.000000,0.000000}%
\pgfsetfillcolor{currentfill}%
\pgfsetlinewidth{1.003750pt}%
\definecolor{currentstroke}{rgb}{0.000000,0.000000,0.000000}%
\pgfsetstrokecolor{currentstroke}%
\pgfsetdash{}{0pt}%
\pgfpathmoveto{\pgfqpoint{4.384857in}{2.334333in}}%
\pgfpathcurveto{\pgfqpoint{4.395908in}{2.334333in}}{\pgfqpoint{4.406507in}{2.338724in}}{\pgfqpoint{4.414320in}{2.346537in}}%
\pgfpathcurveto{\pgfqpoint{4.422134in}{2.354351in}}{\pgfqpoint{4.426524in}{2.364950in}}{\pgfqpoint{4.426524in}{2.376000in}}%
\pgfpathcurveto{\pgfqpoint{4.426524in}{2.387050in}}{\pgfqpoint{4.422134in}{2.397649in}}{\pgfqpoint{4.414320in}{2.405463in}}%
\pgfpathcurveto{\pgfqpoint{4.406507in}{2.413276in}}{\pgfqpoint{4.395908in}{2.417667in}}{\pgfqpoint{4.384857in}{2.417667in}}%
\pgfpathcurveto{\pgfqpoint{4.373807in}{2.417667in}}{\pgfqpoint{4.363208in}{2.413276in}}{\pgfqpoint{4.355395in}{2.405463in}}%
\pgfpathcurveto{\pgfqpoint{4.347581in}{2.397649in}}{\pgfqpoint{4.343191in}{2.387050in}}{\pgfqpoint{4.343191in}{2.376000in}}%
\pgfpathcurveto{\pgfqpoint{4.343191in}{2.364950in}}{\pgfqpoint{4.347581in}{2.354351in}}{\pgfqpoint{4.355395in}{2.346537in}}%
\pgfpathcurveto{\pgfqpoint{4.363208in}{2.338724in}}{\pgfqpoint{4.373807in}{2.334333in}}{\pgfqpoint{4.384857in}{2.334333in}}%
\pgfpathclose%
\pgfusepath{stroke,fill}%
\end{pgfscope}%
\begin{pgfscope}%
\pgfpathrectangle{\pgfqpoint{0.800000in}{0.528000in}}{\pgfqpoint{4.960000in}{3.696000in}}%
\pgfusepath{clip}%
\pgfsetbuttcap%
\pgfsetroundjoin%
\definecolor{currentfill}{rgb}{0.000000,0.000000,0.000000}%
\pgfsetfillcolor{currentfill}%
\pgfsetlinewidth{1.003750pt}%
\definecolor{currentstroke}{rgb}{0.000000,0.000000,0.000000}%
\pgfsetstrokecolor{currentstroke}%
\pgfsetdash{}{0pt}%
\pgfpathmoveto{\pgfqpoint{4.384857in}{2.334333in}}%
\pgfpathcurveto{\pgfqpoint{4.395908in}{2.334333in}}{\pgfqpoint{4.406507in}{2.338724in}}{\pgfqpoint{4.414320in}{2.346537in}}%
\pgfpathcurveto{\pgfqpoint{4.422134in}{2.354351in}}{\pgfqpoint{4.426524in}{2.364950in}}{\pgfqpoint{4.426524in}{2.376000in}}%
\pgfpathcurveto{\pgfqpoint{4.426524in}{2.387050in}}{\pgfqpoint{4.422134in}{2.397649in}}{\pgfqpoint{4.414320in}{2.405463in}}%
\pgfpathcurveto{\pgfqpoint{4.406507in}{2.413276in}}{\pgfqpoint{4.395908in}{2.417667in}}{\pgfqpoint{4.384857in}{2.417667in}}%
\pgfpathcurveto{\pgfqpoint{4.373807in}{2.417667in}}{\pgfqpoint{4.363208in}{2.413276in}}{\pgfqpoint{4.355395in}{2.405463in}}%
\pgfpathcurveto{\pgfqpoint{4.347581in}{2.397649in}}{\pgfqpoint{4.343191in}{2.387050in}}{\pgfqpoint{4.343191in}{2.376000in}}%
\pgfpathcurveto{\pgfqpoint{4.343191in}{2.364950in}}{\pgfqpoint{4.347581in}{2.354351in}}{\pgfqpoint{4.355395in}{2.346537in}}%
\pgfpathcurveto{\pgfqpoint{4.363208in}{2.338724in}}{\pgfqpoint{4.373807in}{2.334333in}}{\pgfqpoint{4.384857in}{2.334333in}}%
\pgfpathclose%
\pgfusepath{stroke,fill}%
\end{pgfscope}%
\begin{pgfscope}%
\pgfpathrectangle{\pgfqpoint{0.800000in}{0.528000in}}{\pgfqpoint{4.960000in}{3.696000in}}%
\pgfusepath{clip}%
\pgfsetbuttcap%
\pgfsetroundjoin%
\definecolor{currentfill}{rgb}{0.000000,0.000000,0.000000}%
\pgfsetfillcolor{currentfill}%
\pgfsetlinewidth{1.003750pt}%
\definecolor{currentstroke}{rgb}{0.000000,0.000000,0.000000}%
\pgfsetstrokecolor{currentstroke}%
\pgfsetdash{}{0pt}%
\pgfpathmoveto{\pgfqpoint{4.384857in}{2.334333in}}%
\pgfpathcurveto{\pgfqpoint{4.395908in}{2.334333in}}{\pgfqpoint{4.406507in}{2.338724in}}{\pgfqpoint{4.414320in}{2.346537in}}%
\pgfpathcurveto{\pgfqpoint{4.422134in}{2.354351in}}{\pgfqpoint{4.426524in}{2.364950in}}{\pgfqpoint{4.426524in}{2.376000in}}%
\pgfpathcurveto{\pgfqpoint{4.426524in}{2.387050in}}{\pgfqpoint{4.422134in}{2.397649in}}{\pgfqpoint{4.414320in}{2.405463in}}%
\pgfpathcurveto{\pgfqpoint{4.406507in}{2.413276in}}{\pgfqpoint{4.395908in}{2.417667in}}{\pgfqpoint{4.384857in}{2.417667in}}%
\pgfpathcurveto{\pgfqpoint{4.373807in}{2.417667in}}{\pgfqpoint{4.363208in}{2.413276in}}{\pgfqpoint{4.355395in}{2.405463in}}%
\pgfpathcurveto{\pgfqpoint{4.347581in}{2.397649in}}{\pgfqpoint{4.343191in}{2.387050in}}{\pgfqpoint{4.343191in}{2.376000in}}%
\pgfpathcurveto{\pgfqpoint{4.343191in}{2.364950in}}{\pgfqpoint{4.347581in}{2.354351in}}{\pgfqpoint{4.355395in}{2.346537in}}%
\pgfpathcurveto{\pgfqpoint{4.363208in}{2.338724in}}{\pgfqpoint{4.373807in}{2.334333in}}{\pgfqpoint{4.384857in}{2.334333in}}%
\pgfpathclose%
\pgfusepath{stroke,fill}%
\end{pgfscope}%
\begin{pgfscope}%
\pgfpathrectangle{\pgfqpoint{0.800000in}{0.528000in}}{\pgfqpoint{4.960000in}{3.696000in}}%
\pgfusepath{clip}%
\pgfsetbuttcap%
\pgfsetroundjoin%
\definecolor{currentfill}{rgb}{0.000000,0.000000,0.000000}%
\pgfsetfillcolor{currentfill}%
\pgfsetlinewidth{1.003750pt}%
\definecolor{currentstroke}{rgb}{0.000000,0.000000,0.000000}%
\pgfsetstrokecolor{currentstroke}%
\pgfsetdash{}{0pt}%
\pgfpathmoveto{\pgfqpoint{4.384857in}{2.334333in}}%
\pgfpathcurveto{\pgfqpoint{4.395908in}{2.334333in}}{\pgfqpoint{4.406507in}{2.338724in}}{\pgfqpoint{4.414320in}{2.346537in}}%
\pgfpathcurveto{\pgfqpoint{4.422134in}{2.354351in}}{\pgfqpoint{4.426524in}{2.364950in}}{\pgfqpoint{4.426524in}{2.376000in}}%
\pgfpathcurveto{\pgfqpoint{4.426524in}{2.387050in}}{\pgfqpoint{4.422134in}{2.397649in}}{\pgfqpoint{4.414320in}{2.405463in}}%
\pgfpathcurveto{\pgfqpoint{4.406507in}{2.413276in}}{\pgfqpoint{4.395908in}{2.417667in}}{\pgfqpoint{4.384857in}{2.417667in}}%
\pgfpathcurveto{\pgfqpoint{4.373807in}{2.417667in}}{\pgfqpoint{4.363208in}{2.413276in}}{\pgfqpoint{4.355395in}{2.405463in}}%
\pgfpathcurveto{\pgfqpoint{4.347581in}{2.397649in}}{\pgfqpoint{4.343191in}{2.387050in}}{\pgfqpoint{4.343191in}{2.376000in}}%
\pgfpathcurveto{\pgfqpoint{4.343191in}{2.364950in}}{\pgfqpoint{4.347581in}{2.354351in}}{\pgfqpoint{4.355395in}{2.346537in}}%
\pgfpathcurveto{\pgfqpoint{4.363208in}{2.338724in}}{\pgfqpoint{4.373807in}{2.334333in}}{\pgfqpoint{4.384857in}{2.334333in}}%
\pgfpathclose%
\pgfusepath{stroke,fill}%
\end{pgfscope}%
\begin{pgfscope}%
\pgfpathrectangle{\pgfqpoint{0.800000in}{0.528000in}}{\pgfqpoint{4.960000in}{3.696000in}}%
\pgfusepath{clip}%
\pgfsetbuttcap%
\pgfsetroundjoin%
\definecolor{currentfill}{rgb}{0.000000,0.000000,0.000000}%
\pgfsetfillcolor{currentfill}%
\pgfsetlinewidth{1.003750pt}%
\definecolor{currentstroke}{rgb}{0.000000,0.000000,0.000000}%
\pgfsetstrokecolor{currentstroke}%
\pgfsetdash{}{0pt}%
\pgfpathmoveto{\pgfqpoint{4.384857in}{2.334333in}}%
\pgfpathcurveto{\pgfqpoint{4.395908in}{2.334333in}}{\pgfqpoint{4.406507in}{2.338724in}}{\pgfqpoint{4.414320in}{2.346537in}}%
\pgfpathcurveto{\pgfqpoint{4.422134in}{2.354351in}}{\pgfqpoint{4.426524in}{2.364950in}}{\pgfqpoint{4.426524in}{2.376000in}}%
\pgfpathcurveto{\pgfqpoint{4.426524in}{2.387050in}}{\pgfqpoint{4.422134in}{2.397649in}}{\pgfqpoint{4.414320in}{2.405463in}}%
\pgfpathcurveto{\pgfqpoint{4.406507in}{2.413276in}}{\pgfqpoint{4.395908in}{2.417667in}}{\pgfqpoint{4.384857in}{2.417667in}}%
\pgfpathcurveto{\pgfqpoint{4.373807in}{2.417667in}}{\pgfqpoint{4.363208in}{2.413276in}}{\pgfqpoint{4.355395in}{2.405463in}}%
\pgfpathcurveto{\pgfqpoint{4.347581in}{2.397649in}}{\pgfqpoint{4.343191in}{2.387050in}}{\pgfqpoint{4.343191in}{2.376000in}}%
\pgfpathcurveto{\pgfqpoint{4.343191in}{2.364950in}}{\pgfqpoint{4.347581in}{2.354351in}}{\pgfqpoint{4.355395in}{2.346537in}}%
\pgfpathcurveto{\pgfqpoint{4.363208in}{2.338724in}}{\pgfqpoint{4.373807in}{2.334333in}}{\pgfqpoint{4.384857in}{2.334333in}}%
\pgfpathclose%
\pgfusepath{stroke,fill}%
\end{pgfscope}%
\begin{pgfscope}%
\pgfpathrectangle{\pgfqpoint{0.800000in}{0.528000in}}{\pgfqpoint{4.960000in}{3.696000in}}%
\pgfusepath{clip}%
\pgfsetbuttcap%
\pgfsetroundjoin%
\definecolor{currentfill}{rgb}{0.000000,0.000000,0.000000}%
\pgfsetfillcolor{currentfill}%
\pgfsetlinewidth{1.003750pt}%
\definecolor{currentstroke}{rgb}{0.000000,0.000000,0.000000}%
\pgfsetstrokecolor{currentstroke}%
\pgfsetdash{}{0pt}%
\pgfpathmoveto{\pgfqpoint{4.384857in}{2.334333in}}%
\pgfpathcurveto{\pgfqpoint{4.395908in}{2.334333in}}{\pgfqpoint{4.406507in}{2.338724in}}{\pgfqpoint{4.414320in}{2.346537in}}%
\pgfpathcurveto{\pgfqpoint{4.422134in}{2.354351in}}{\pgfqpoint{4.426524in}{2.364950in}}{\pgfqpoint{4.426524in}{2.376000in}}%
\pgfpathcurveto{\pgfqpoint{4.426524in}{2.387050in}}{\pgfqpoint{4.422134in}{2.397649in}}{\pgfqpoint{4.414320in}{2.405463in}}%
\pgfpathcurveto{\pgfqpoint{4.406507in}{2.413276in}}{\pgfqpoint{4.395908in}{2.417667in}}{\pgfqpoint{4.384857in}{2.417667in}}%
\pgfpathcurveto{\pgfqpoint{4.373807in}{2.417667in}}{\pgfqpoint{4.363208in}{2.413276in}}{\pgfqpoint{4.355395in}{2.405463in}}%
\pgfpathcurveto{\pgfqpoint{4.347581in}{2.397649in}}{\pgfqpoint{4.343191in}{2.387050in}}{\pgfqpoint{4.343191in}{2.376000in}}%
\pgfpathcurveto{\pgfqpoint{4.343191in}{2.364950in}}{\pgfqpoint{4.347581in}{2.354351in}}{\pgfqpoint{4.355395in}{2.346537in}}%
\pgfpathcurveto{\pgfqpoint{4.363208in}{2.338724in}}{\pgfqpoint{4.373807in}{2.334333in}}{\pgfqpoint{4.384857in}{2.334333in}}%
\pgfpathclose%
\pgfusepath{stroke,fill}%
\end{pgfscope}%
\begin{pgfscope}%
\pgfpathrectangle{\pgfqpoint{0.800000in}{0.528000in}}{\pgfqpoint{4.960000in}{3.696000in}}%
\pgfusepath{clip}%
\pgfsetbuttcap%
\pgfsetroundjoin%
\definecolor{currentfill}{rgb}{0.000000,0.000000,0.000000}%
\pgfsetfillcolor{currentfill}%
\pgfsetlinewidth{1.003750pt}%
\definecolor{currentstroke}{rgb}{0.000000,0.000000,0.000000}%
\pgfsetstrokecolor{currentstroke}%
\pgfsetdash{}{0pt}%
\pgfpathmoveto{\pgfqpoint{4.384857in}{2.334333in}}%
\pgfpathcurveto{\pgfqpoint{4.395908in}{2.334333in}}{\pgfqpoint{4.406507in}{2.338724in}}{\pgfqpoint{4.414320in}{2.346537in}}%
\pgfpathcurveto{\pgfqpoint{4.422134in}{2.354351in}}{\pgfqpoint{4.426524in}{2.364950in}}{\pgfqpoint{4.426524in}{2.376000in}}%
\pgfpathcurveto{\pgfqpoint{4.426524in}{2.387050in}}{\pgfqpoint{4.422134in}{2.397649in}}{\pgfqpoint{4.414320in}{2.405463in}}%
\pgfpathcurveto{\pgfqpoint{4.406507in}{2.413276in}}{\pgfqpoint{4.395908in}{2.417667in}}{\pgfqpoint{4.384857in}{2.417667in}}%
\pgfpathcurveto{\pgfqpoint{4.373807in}{2.417667in}}{\pgfqpoint{4.363208in}{2.413276in}}{\pgfqpoint{4.355395in}{2.405463in}}%
\pgfpathcurveto{\pgfqpoint{4.347581in}{2.397649in}}{\pgfqpoint{4.343191in}{2.387050in}}{\pgfqpoint{4.343191in}{2.376000in}}%
\pgfpathcurveto{\pgfqpoint{4.343191in}{2.364950in}}{\pgfqpoint{4.347581in}{2.354351in}}{\pgfqpoint{4.355395in}{2.346537in}}%
\pgfpathcurveto{\pgfqpoint{4.363208in}{2.338724in}}{\pgfqpoint{4.373807in}{2.334333in}}{\pgfqpoint{4.384857in}{2.334333in}}%
\pgfpathclose%
\pgfusepath{stroke,fill}%
\end{pgfscope}%
\begin{pgfscope}%
\pgfpathrectangle{\pgfqpoint{0.800000in}{0.528000in}}{\pgfqpoint{4.960000in}{3.696000in}}%
\pgfusepath{clip}%
\pgfsetbuttcap%
\pgfsetroundjoin%
\definecolor{currentfill}{rgb}{0.000000,0.000000,0.000000}%
\pgfsetfillcolor{currentfill}%
\pgfsetlinewidth{1.003750pt}%
\definecolor{currentstroke}{rgb}{0.000000,0.000000,0.000000}%
\pgfsetstrokecolor{currentstroke}%
\pgfsetdash{}{0pt}%
\pgfpathmoveto{\pgfqpoint{4.384857in}{2.334333in}}%
\pgfpathcurveto{\pgfqpoint{4.395908in}{2.334333in}}{\pgfqpoint{4.406507in}{2.338724in}}{\pgfqpoint{4.414320in}{2.346537in}}%
\pgfpathcurveto{\pgfqpoint{4.422134in}{2.354351in}}{\pgfqpoint{4.426524in}{2.364950in}}{\pgfqpoint{4.426524in}{2.376000in}}%
\pgfpathcurveto{\pgfqpoint{4.426524in}{2.387050in}}{\pgfqpoint{4.422134in}{2.397649in}}{\pgfqpoint{4.414320in}{2.405463in}}%
\pgfpathcurveto{\pgfqpoint{4.406507in}{2.413276in}}{\pgfqpoint{4.395908in}{2.417667in}}{\pgfqpoint{4.384857in}{2.417667in}}%
\pgfpathcurveto{\pgfqpoint{4.373807in}{2.417667in}}{\pgfqpoint{4.363208in}{2.413276in}}{\pgfqpoint{4.355395in}{2.405463in}}%
\pgfpathcurveto{\pgfqpoint{4.347581in}{2.397649in}}{\pgfqpoint{4.343191in}{2.387050in}}{\pgfqpoint{4.343191in}{2.376000in}}%
\pgfpathcurveto{\pgfqpoint{4.343191in}{2.364950in}}{\pgfqpoint{4.347581in}{2.354351in}}{\pgfqpoint{4.355395in}{2.346537in}}%
\pgfpathcurveto{\pgfqpoint{4.363208in}{2.338724in}}{\pgfqpoint{4.373807in}{2.334333in}}{\pgfqpoint{4.384857in}{2.334333in}}%
\pgfpathclose%
\pgfusepath{stroke,fill}%
\end{pgfscope}%
\begin{pgfscope}%
\pgfpathrectangle{\pgfqpoint{0.800000in}{0.528000in}}{\pgfqpoint{4.960000in}{3.696000in}}%
\pgfusepath{clip}%
\pgfsetbuttcap%
\pgfsetroundjoin%
\definecolor{currentfill}{rgb}{0.000000,0.000000,0.000000}%
\pgfsetfillcolor{currentfill}%
\pgfsetlinewidth{1.003750pt}%
\definecolor{currentstroke}{rgb}{0.000000,0.000000,0.000000}%
\pgfsetstrokecolor{currentstroke}%
\pgfsetdash{}{0pt}%
\pgfpathmoveto{\pgfqpoint{4.384857in}{2.334333in}}%
\pgfpathcurveto{\pgfqpoint{4.395908in}{2.334333in}}{\pgfqpoint{4.406507in}{2.338724in}}{\pgfqpoint{4.414320in}{2.346537in}}%
\pgfpathcurveto{\pgfqpoint{4.422134in}{2.354351in}}{\pgfqpoint{4.426524in}{2.364950in}}{\pgfqpoint{4.426524in}{2.376000in}}%
\pgfpathcurveto{\pgfqpoint{4.426524in}{2.387050in}}{\pgfqpoint{4.422134in}{2.397649in}}{\pgfqpoint{4.414320in}{2.405463in}}%
\pgfpathcurveto{\pgfqpoint{4.406507in}{2.413276in}}{\pgfqpoint{4.395908in}{2.417667in}}{\pgfqpoint{4.384857in}{2.417667in}}%
\pgfpathcurveto{\pgfqpoint{4.373807in}{2.417667in}}{\pgfqpoint{4.363208in}{2.413276in}}{\pgfqpoint{4.355395in}{2.405463in}}%
\pgfpathcurveto{\pgfqpoint{4.347581in}{2.397649in}}{\pgfqpoint{4.343191in}{2.387050in}}{\pgfqpoint{4.343191in}{2.376000in}}%
\pgfpathcurveto{\pgfqpoint{4.343191in}{2.364950in}}{\pgfqpoint{4.347581in}{2.354351in}}{\pgfqpoint{4.355395in}{2.346537in}}%
\pgfpathcurveto{\pgfqpoint{4.363208in}{2.338724in}}{\pgfqpoint{4.373807in}{2.334333in}}{\pgfqpoint{4.384857in}{2.334333in}}%
\pgfpathclose%
\pgfusepath{stroke,fill}%
\end{pgfscope}%
\begin{pgfscope}%
\pgfpathrectangle{\pgfqpoint{0.800000in}{0.528000in}}{\pgfqpoint{4.960000in}{3.696000in}}%
\pgfusepath{clip}%
\pgfsetbuttcap%
\pgfsetroundjoin%
\definecolor{currentfill}{rgb}{0.000000,0.000000,0.000000}%
\pgfsetfillcolor{currentfill}%
\pgfsetlinewidth{1.003750pt}%
\definecolor{currentstroke}{rgb}{0.000000,0.000000,0.000000}%
\pgfsetstrokecolor{currentstroke}%
\pgfsetdash{}{0pt}%
\pgfpathmoveto{\pgfqpoint{4.384857in}{2.334333in}}%
\pgfpathcurveto{\pgfqpoint{4.395908in}{2.334333in}}{\pgfqpoint{4.406507in}{2.338724in}}{\pgfqpoint{4.414320in}{2.346537in}}%
\pgfpathcurveto{\pgfqpoint{4.422134in}{2.354351in}}{\pgfqpoint{4.426524in}{2.364950in}}{\pgfqpoint{4.426524in}{2.376000in}}%
\pgfpathcurveto{\pgfqpoint{4.426524in}{2.387050in}}{\pgfqpoint{4.422134in}{2.397649in}}{\pgfqpoint{4.414320in}{2.405463in}}%
\pgfpathcurveto{\pgfqpoint{4.406507in}{2.413276in}}{\pgfqpoint{4.395908in}{2.417667in}}{\pgfqpoint{4.384857in}{2.417667in}}%
\pgfpathcurveto{\pgfqpoint{4.373807in}{2.417667in}}{\pgfqpoint{4.363208in}{2.413276in}}{\pgfqpoint{4.355395in}{2.405463in}}%
\pgfpathcurveto{\pgfqpoint{4.347581in}{2.397649in}}{\pgfqpoint{4.343191in}{2.387050in}}{\pgfqpoint{4.343191in}{2.376000in}}%
\pgfpathcurveto{\pgfqpoint{4.343191in}{2.364950in}}{\pgfqpoint{4.347581in}{2.354351in}}{\pgfqpoint{4.355395in}{2.346537in}}%
\pgfpathcurveto{\pgfqpoint{4.363208in}{2.338724in}}{\pgfqpoint{4.373807in}{2.334333in}}{\pgfqpoint{4.384857in}{2.334333in}}%
\pgfpathclose%
\pgfusepath{stroke,fill}%
\end{pgfscope}%
\begin{pgfscope}%
\pgfpathrectangle{\pgfqpoint{0.800000in}{0.528000in}}{\pgfqpoint{4.960000in}{3.696000in}}%
\pgfusepath{clip}%
\pgfsetbuttcap%
\pgfsetroundjoin%
\definecolor{currentfill}{rgb}{0.000000,0.000000,0.000000}%
\pgfsetfillcolor{currentfill}%
\pgfsetlinewidth{1.003750pt}%
\definecolor{currentstroke}{rgb}{0.000000,0.000000,0.000000}%
\pgfsetstrokecolor{currentstroke}%
\pgfsetdash{}{0pt}%
\pgfpathmoveto{\pgfqpoint{4.384857in}{2.334333in}}%
\pgfpathcurveto{\pgfqpoint{4.395908in}{2.334333in}}{\pgfqpoint{4.406507in}{2.338724in}}{\pgfqpoint{4.414320in}{2.346537in}}%
\pgfpathcurveto{\pgfqpoint{4.422134in}{2.354351in}}{\pgfqpoint{4.426524in}{2.364950in}}{\pgfqpoint{4.426524in}{2.376000in}}%
\pgfpathcurveto{\pgfqpoint{4.426524in}{2.387050in}}{\pgfqpoint{4.422134in}{2.397649in}}{\pgfqpoint{4.414320in}{2.405463in}}%
\pgfpathcurveto{\pgfqpoint{4.406507in}{2.413276in}}{\pgfqpoint{4.395908in}{2.417667in}}{\pgfqpoint{4.384857in}{2.417667in}}%
\pgfpathcurveto{\pgfqpoint{4.373807in}{2.417667in}}{\pgfqpoint{4.363208in}{2.413276in}}{\pgfqpoint{4.355395in}{2.405463in}}%
\pgfpathcurveto{\pgfqpoint{4.347581in}{2.397649in}}{\pgfqpoint{4.343191in}{2.387050in}}{\pgfqpoint{4.343191in}{2.376000in}}%
\pgfpathcurveto{\pgfqpoint{4.343191in}{2.364950in}}{\pgfqpoint{4.347581in}{2.354351in}}{\pgfqpoint{4.355395in}{2.346537in}}%
\pgfpathcurveto{\pgfqpoint{4.363208in}{2.338724in}}{\pgfqpoint{4.373807in}{2.334333in}}{\pgfqpoint{4.384857in}{2.334333in}}%
\pgfpathclose%
\pgfusepath{stroke,fill}%
\end{pgfscope}%
\begin{pgfscope}%
\pgfpathrectangle{\pgfqpoint{0.800000in}{0.528000in}}{\pgfqpoint{4.960000in}{3.696000in}}%
\pgfusepath{clip}%
\pgfsetbuttcap%
\pgfsetroundjoin%
\definecolor{currentfill}{rgb}{0.000000,0.000000,0.000000}%
\pgfsetfillcolor{currentfill}%
\pgfsetlinewidth{1.003750pt}%
\definecolor{currentstroke}{rgb}{0.000000,0.000000,0.000000}%
\pgfsetstrokecolor{currentstroke}%
\pgfsetdash{}{0pt}%
\pgfpathmoveto{\pgfqpoint{4.384857in}{2.334333in}}%
\pgfpathcurveto{\pgfqpoint{4.395908in}{2.334333in}}{\pgfqpoint{4.406507in}{2.338724in}}{\pgfqpoint{4.414320in}{2.346537in}}%
\pgfpathcurveto{\pgfqpoint{4.422134in}{2.354351in}}{\pgfqpoint{4.426524in}{2.364950in}}{\pgfqpoint{4.426524in}{2.376000in}}%
\pgfpathcurveto{\pgfqpoint{4.426524in}{2.387050in}}{\pgfqpoint{4.422134in}{2.397649in}}{\pgfqpoint{4.414320in}{2.405463in}}%
\pgfpathcurveto{\pgfqpoint{4.406507in}{2.413276in}}{\pgfqpoint{4.395908in}{2.417667in}}{\pgfqpoint{4.384857in}{2.417667in}}%
\pgfpathcurveto{\pgfqpoint{4.373807in}{2.417667in}}{\pgfqpoint{4.363208in}{2.413276in}}{\pgfqpoint{4.355395in}{2.405463in}}%
\pgfpathcurveto{\pgfqpoint{4.347581in}{2.397649in}}{\pgfqpoint{4.343191in}{2.387050in}}{\pgfqpoint{4.343191in}{2.376000in}}%
\pgfpathcurveto{\pgfqpoint{4.343191in}{2.364950in}}{\pgfqpoint{4.347581in}{2.354351in}}{\pgfqpoint{4.355395in}{2.346537in}}%
\pgfpathcurveto{\pgfqpoint{4.363208in}{2.338724in}}{\pgfqpoint{4.373807in}{2.334333in}}{\pgfqpoint{4.384857in}{2.334333in}}%
\pgfpathclose%
\pgfusepath{stroke,fill}%
\end{pgfscope}%
\begin{pgfscope}%
\pgfpathrectangle{\pgfqpoint{0.800000in}{0.528000in}}{\pgfqpoint{4.960000in}{3.696000in}}%
\pgfusepath{clip}%
\pgfsetbuttcap%
\pgfsetroundjoin%
\definecolor{currentfill}{rgb}{0.000000,0.000000,0.000000}%
\pgfsetfillcolor{currentfill}%
\pgfsetlinewidth{1.003750pt}%
\definecolor{currentstroke}{rgb}{0.000000,0.000000,0.000000}%
\pgfsetstrokecolor{currentstroke}%
\pgfsetdash{}{0pt}%
\pgfpathmoveto{\pgfqpoint{4.384857in}{2.334333in}}%
\pgfpathcurveto{\pgfqpoint{4.395908in}{2.334333in}}{\pgfqpoint{4.406507in}{2.338724in}}{\pgfqpoint{4.414320in}{2.346537in}}%
\pgfpathcurveto{\pgfqpoint{4.422134in}{2.354351in}}{\pgfqpoint{4.426524in}{2.364950in}}{\pgfqpoint{4.426524in}{2.376000in}}%
\pgfpathcurveto{\pgfqpoint{4.426524in}{2.387050in}}{\pgfqpoint{4.422134in}{2.397649in}}{\pgfqpoint{4.414320in}{2.405463in}}%
\pgfpathcurveto{\pgfqpoint{4.406507in}{2.413276in}}{\pgfqpoint{4.395908in}{2.417667in}}{\pgfqpoint{4.384857in}{2.417667in}}%
\pgfpathcurveto{\pgfqpoint{4.373807in}{2.417667in}}{\pgfqpoint{4.363208in}{2.413276in}}{\pgfqpoint{4.355395in}{2.405463in}}%
\pgfpathcurveto{\pgfqpoint{4.347581in}{2.397649in}}{\pgfqpoint{4.343191in}{2.387050in}}{\pgfqpoint{4.343191in}{2.376000in}}%
\pgfpathcurveto{\pgfqpoint{4.343191in}{2.364950in}}{\pgfqpoint{4.347581in}{2.354351in}}{\pgfqpoint{4.355395in}{2.346537in}}%
\pgfpathcurveto{\pgfqpoint{4.363208in}{2.338724in}}{\pgfqpoint{4.373807in}{2.334333in}}{\pgfqpoint{4.384857in}{2.334333in}}%
\pgfpathclose%
\pgfusepath{stroke,fill}%
\end{pgfscope}%
\begin{pgfscope}%
\pgfpathrectangle{\pgfqpoint{0.800000in}{0.528000in}}{\pgfqpoint{4.960000in}{3.696000in}}%
\pgfusepath{clip}%
\pgfsetbuttcap%
\pgfsetroundjoin%
\definecolor{currentfill}{rgb}{0.000000,0.000000,0.000000}%
\pgfsetfillcolor{currentfill}%
\pgfsetlinewidth{1.003750pt}%
\definecolor{currentstroke}{rgb}{0.000000,0.000000,0.000000}%
\pgfsetstrokecolor{currentstroke}%
\pgfsetdash{}{0pt}%
\pgfpathmoveto{\pgfqpoint{4.384857in}{2.334333in}}%
\pgfpathcurveto{\pgfqpoint{4.395908in}{2.334333in}}{\pgfqpoint{4.406507in}{2.338724in}}{\pgfqpoint{4.414320in}{2.346537in}}%
\pgfpathcurveto{\pgfqpoint{4.422134in}{2.354351in}}{\pgfqpoint{4.426524in}{2.364950in}}{\pgfqpoint{4.426524in}{2.376000in}}%
\pgfpathcurveto{\pgfqpoint{4.426524in}{2.387050in}}{\pgfqpoint{4.422134in}{2.397649in}}{\pgfqpoint{4.414320in}{2.405463in}}%
\pgfpathcurveto{\pgfqpoint{4.406507in}{2.413276in}}{\pgfqpoint{4.395908in}{2.417667in}}{\pgfqpoint{4.384857in}{2.417667in}}%
\pgfpathcurveto{\pgfqpoint{4.373807in}{2.417667in}}{\pgfqpoint{4.363208in}{2.413276in}}{\pgfqpoint{4.355395in}{2.405463in}}%
\pgfpathcurveto{\pgfqpoint{4.347581in}{2.397649in}}{\pgfqpoint{4.343191in}{2.387050in}}{\pgfqpoint{4.343191in}{2.376000in}}%
\pgfpathcurveto{\pgfqpoint{4.343191in}{2.364950in}}{\pgfqpoint{4.347581in}{2.354351in}}{\pgfqpoint{4.355395in}{2.346537in}}%
\pgfpathcurveto{\pgfqpoint{4.363208in}{2.338724in}}{\pgfqpoint{4.373807in}{2.334333in}}{\pgfqpoint{4.384857in}{2.334333in}}%
\pgfpathclose%
\pgfusepath{stroke,fill}%
\end{pgfscope}%
\begin{pgfscope}%
\pgfpathrectangle{\pgfqpoint{0.800000in}{0.528000in}}{\pgfqpoint{4.960000in}{3.696000in}}%
\pgfusepath{clip}%
\pgfsetbuttcap%
\pgfsetroundjoin%
\definecolor{currentfill}{rgb}{0.000000,0.000000,0.000000}%
\pgfsetfillcolor{currentfill}%
\pgfsetlinewidth{1.003750pt}%
\definecolor{currentstroke}{rgb}{0.000000,0.000000,0.000000}%
\pgfsetstrokecolor{currentstroke}%
\pgfsetdash{}{0pt}%
\pgfpathmoveto{\pgfqpoint{4.384857in}{2.334333in}}%
\pgfpathcurveto{\pgfqpoint{4.395908in}{2.334333in}}{\pgfqpoint{4.406507in}{2.338724in}}{\pgfqpoint{4.414320in}{2.346537in}}%
\pgfpathcurveto{\pgfqpoint{4.422134in}{2.354351in}}{\pgfqpoint{4.426524in}{2.364950in}}{\pgfqpoint{4.426524in}{2.376000in}}%
\pgfpathcurveto{\pgfqpoint{4.426524in}{2.387050in}}{\pgfqpoint{4.422134in}{2.397649in}}{\pgfqpoint{4.414320in}{2.405463in}}%
\pgfpathcurveto{\pgfqpoint{4.406507in}{2.413276in}}{\pgfqpoint{4.395908in}{2.417667in}}{\pgfqpoint{4.384857in}{2.417667in}}%
\pgfpathcurveto{\pgfqpoint{4.373807in}{2.417667in}}{\pgfqpoint{4.363208in}{2.413276in}}{\pgfqpoint{4.355395in}{2.405463in}}%
\pgfpathcurveto{\pgfqpoint{4.347581in}{2.397649in}}{\pgfqpoint{4.343191in}{2.387050in}}{\pgfqpoint{4.343191in}{2.376000in}}%
\pgfpathcurveto{\pgfqpoint{4.343191in}{2.364950in}}{\pgfqpoint{4.347581in}{2.354351in}}{\pgfqpoint{4.355395in}{2.346537in}}%
\pgfpathcurveto{\pgfqpoint{4.363208in}{2.338724in}}{\pgfqpoint{4.373807in}{2.334333in}}{\pgfqpoint{4.384857in}{2.334333in}}%
\pgfpathclose%
\pgfusepath{stroke,fill}%
\end{pgfscope}%
\begin{pgfscope}%
\pgfpathrectangle{\pgfqpoint{0.800000in}{0.528000in}}{\pgfqpoint{4.960000in}{3.696000in}}%
\pgfusepath{clip}%
\pgfsetbuttcap%
\pgfsetroundjoin%
\definecolor{currentfill}{rgb}{0.000000,0.000000,0.000000}%
\pgfsetfillcolor{currentfill}%
\pgfsetlinewidth{1.003750pt}%
\definecolor{currentstroke}{rgb}{0.000000,0.000000,0.000000}%
\pgfsetstrokecolor{currentstroke}%
\pgfsetdash{}{0pt}%
\pgfpathmoveto{\pgfqpoint{4.384857in}{2.334333in}}%
\pgfpathcurveto{\pgfqpoint{4.395908in}{2.334333in}}{\pgfqpoint{4.406507in}{2.338724in}}{\pgfqpoint{4.414320in}{2.346537in}}%
\pgfpathcurveto{\pgfqpoint{4.422134in}{2.354351in}}{\pgfqpoint{4.426524in}{2.364950in}}{\pgfqpoint{4.426524in}{2.376000in}}%
\pgfpathcurveto{\pgfqpoint{4.426524in}{2.387050in}}{\pgfqpoint{4.422134in}{2.397649in}}{\pgfqpoint{4.414320in}{2.405463in}}%
\pgfpathcurveto{\pgfqpoint{4.406507in}{2.413276in}}{\pgfqpoint{4.395908in}{2.417667in}}{\pgfqpoint{4.384857in}{2.417667in}}%
\pgfpathcurveto{\pgfqpoint{4.373807in}{2.417667in}}{\pgfqpoint{4.363208in}{2.413276in}}{\pgfqpoint{4.355395in}{2.405463in}}%
\pgfpathcurveto{\pgfqpoint{4.347581in}{2.397649in}}{\pgfqpoint{4.343191in}{2.387050in}}{\pgfqpoint{4.343191in}{2.376000in}}%
\pgfpathcurveto{\pgfqpoint{4.343191in}{2.364950in}}{\pgfqpoint{4.347581in}{2.354351in}}{\pgfqpoint{4.355395in}{2.346537in}}%
\pgfpathcurveto{\pgfqpoint{4.363208in}{2.338724in}}{\pgfqpoint{4.373807in}{2.334333in}}{\pgfqpoint{4.384857in}{2.334333in}}%
\pgfpathclose%
\pgfusepath{stroke,fill}%
\end{pgfscope}%
\begin{pgfscope}%
\pgfpathrectangle{\pgfqpoint{0.800000in}{0.528000in}}{\pgfqpoint{4.960000in}{3.696000in}}%
\pgfusepath{clip}%
\pgfsetbuttcap%
\pgfsetroundjoin%
\definecolor{currentfill}{rgb}{0.000000,0.000000,0.000000}%
\pgfsetfillcolor{currentfill}%
\pgfsetlinewidth{1.003750pt}%
\definecolor{currentstroke}{rgb}{0.000000,0.000000,0.000000}%
\pgfsetstrokecolor{currentstroke}%
\pgfsetdash{}{0pt}%
\pgfpathmoveto{\pgfqpoint{4.384857in}{2.334333in}}%
\pgfpathcurveto{\pgfqpoint{4.395908in}{2.334333in}}{\pgfqpoint{4.406507in}{2.338724in}}{\pgfqpoint{4.414320in}{2.346537in}}%
\pgfpathcurveto{\pgfqpoint{4.422134in}{2.354351in}}{\pgfqpoint{4.426524in}{2.364950in}}{\pgfqpoint{4.426524in}{2.376000in}}%
\pgfpathcurveto{\pgfqpoint{4.426524in}{2.387050in}}{\pgfqpoint{4.422134in}{2.397649in}}{\pgfqpoint{4.414320in}{2.405463in}}%
\pgfpathcurveto{\pgfqpoint{4.406507in}{2.413276in}}{\pgfqpoint{4.395908in}{2.417667in}}{\pgfqpoint{4.384857in}{2.417667in}}%
\pgfpathcurveto{\pgfqpoint{4.373807in}{2.417667in}}{\pgfqpoint{4.363208in}{2.413276in}}{\pgfqpoint{4.355395in}{2.405463in}}%
\pgfpathcurveto{\pgfqpoint{4.347581in}{2.397649in}}{\pgfqpoint{4.343191in}{2.387050in}}{\pgfqpoint{4.343191in}{2.376000in}}%
\pgfpathcurveto{\pgfqpoint{4.343191in}{2.364950in}}{\pgfqpoint{4.347581in}{2.354351in}}{\pgfqpoint{4.355395in}{2.346537in}}%
\pgfpathcurveto{\pgfqpoint{4.363208in}{2.338724in}}{\pgfqpoint{4.373807in}{2.334333in}}{\pgfqpoint{4.384857in}{2.334333in}}%
\pgfpathclose%
\pgfusepath{stroke,fill}%
\end{pgfscope}%
\begin{pgfscope}%
\pgfpathrectangle{\pgfqpoint{0.800000in}{0.528000in}}{\pgfqpoint{4.960000in}{3.696000in}}%
\pgfusepath{clip}%
\pgfsetbuttcap%
\pgfsetroundjoin%
\definecolor{currentfill}{rgb}{0.000000,0.000000,0.000000}%
\pgfsetfillcolor{currentfill}%
\pgfsetlinewidth{1.003750pt}%
\definecolor{currentstroke}{rgb}{0.000000,0.000000,0.000000}%
\pgfsetstrokecolor{currentstroke}%
\pgfsetdash{}{0pt}%
\pgfpathmoveto{\pgfqpoint{4.384857in}{2.334333in}}%
\pgfpathcurveto{\pgfqpoint{4.395908in}{2.334333in}}{\pgfqpoint{4.406507in}{2.338724in}}{\pgfqpoint{4.414320in}{2.346537in}}%
\pgfpathcurveto{\pgfqpoint{4.422134in}{2.354351in}}{\pgfqpoint{4.426524in}{2.364950in}}{\pgfqpoint{4.426524in}{2.376000in}}%
\pgfpathcurveto{\pgfqpoint{4.426524in}{2.387050in}}{\pgfqpoint{4.422134in}{2.397649in}}{\pgfqpoint{4.414320in}{2.405463in}}%
\pgfpathcurveto{\pgfqpoint{4.406507in}{2.413276in}}{\pgfqpoint{4.395908in}{2.417667in}}{\pgfqpoint{4.384857in}{2.417667in}}%
\pgfpathcurveto{\pgfqpoint{4.373807in}{2.417667in}}{\pgfqpoint{4.363208in}{2.413276in}}{\pgfqpoint{4.355395in}{2.405463in}}%
\pgfpathcurveto{\pgfqpoint{4.347581in}{2.397649in}}{\pgfqpoint{4.343191in}{2.387050in}}{\pgfqpoint{4.343191in}{2.376000in}}%
\pgfpathcurveto{\pgfqpoint{4.343191in}{2.364950in}}{\pgfqpoint{4.347581in}{2.354351in}}{\pgfqpoint{4.355395in}{2.346537in}}%
\pgfpathcurveto{\pgfqpoint{4.363208in}{2.338724in}}{\pgfqpoint{4.373807in}{2.334333in}}{\pgfqpoint{4.384857in}{2.334333in}}%
\pgfpathclose%
\pgfusepath{stroke,fill}%
\end{pgfscope}%
\begin{pgfscope}%
\pgfpathrectangle{\pgfqpoint{0.800000in}{0.528000in}}{\pgfqpoint{4.960000in}{3.696000in}}%
\pgfusepath{clip}%
\pgfsetbuttcap%
\pgfsetroundjoin%
\definecolor{currentfill}{rgb}{0.000000,0.000000,0.000000}%
\pgfsetfillcolor{currentfill}%
\pgfsetlinewidth{1.003750pt}%
\definecolor{currentstroke}{rgb}{0.000000,0.000000,0.000000}%
\pgfsetstrokecolor{currentstroke}%
\pgfsetdash{}{0pt}%
\pgfpathmoveto{\pgfqpoint{4.384857in}{2.334333in}}%
\pgfpathcurveto{\pgfqpoint{4.395908in}{2.334333in}}{\pgfqpoint{4.406507in}{2.338724in}}{\pgfqpoint{4.414320in}{2.346537in}}%
\pgfpathcurveto{\pgfqpoint{4.422134in}{2.354351in}}{\pgfqpoint{4.426524in}{2.364950in}}{\pgfqpoint{4.426524in}{2.376000in}}%
\pgfpathcurveto{\pgfqpoint{4.426524in}{2.387050in}}{\pgfqpoint{4.422134in}{2.397649in}}{\pgfqpoint{4.414320in}{2.405463in}}%
\pgfpathcurveto{\pgfqpoint{4.406507in}{2.413276in}}{\pgfqpoint{4.395908in}{2.417667in}}{\pgfqpoint{4.384857in}{2.417667in}}%
\pgfpathcurveto{\pgfqpoint{4.373807in}{2.417667in}}{\pgfqpoint{4.363208in}{2.413276in}}{\pgfqpoint{4.355395in}{2.405463in}}%
\pgfpathcurveto{\pgfqpoint{4.347581in}{2.397649in}}{\pgfqpoint{4.343191in}{2.387050in}}{\pgfqpoint{4.343191in}{2.376000in}}%
\pgfpathcurveto{\pgfqpoint{4.343191in}{2.364950in}}{\pgfqpoint{4.347581in}{2.354351in}}{\pgfqpoint{4.355395in}{2.346537in}}%
\pgfpathcurveto{\pgfqpoint{4.363208in}{2.338724in}}{\pgfqpoint{4.373807in}{2.334333in}}{\pgfqpoint{4.384857in}{2.334333in}}%
\pgfpathclose%
\pgfusepath{stroke,fill}%
\end{pgfscope}%
\begin{pgfscope}%
\pgfpathrectangle{\pgfqpoint{0.800000in}{0.528000in}}{\pgfqpoint{4.960000in}{3.696000in}}%
\pgfusepath{clip}%
\pgfsetbuttcap%
\pgfsetroundjoin%
\definecolor{currentfill}{rgb}{0.000000,0.000000,0.000000}%
\pgfsetfillcolor{currentfill}%
\pgfsetlinewidth{1.003750pt}%
\definecolor{currentstroke}{rgb}{0.000000,0.000000,0.000000}%
\pgfsetstrokecolor{currentstroke}%
\pgfsetdash{}{0pt}%
\pgfpathmoveto{\pgfqpoint{4.384857in}{2.334333in}}%
\pgfpathcurveto{\pgfqpoint{4.395908in}{2.334333in}}{\pgfqpoint{4.406507in}{2.338724in}}{\pgfqpoint{4.414320in}{2.346537in}}%
\pgfpathcurveto{\pgfqpoint{4.422134in}{2.354351in}}{\pgfqpoint{4.426524in}{2.364950in}}{\pgfqpoint{4.426524in}{2.376000in}}%
\pgfpathcurveto{\pgfqpoint{4.426524in}{2.387050in}}{\pgfqpoint{4.422134in}{2.397649in}}{\pgfqpoint{4.414320in}{2.405463in}}%
\pgfpathcurveto{\pgfqpoint{4.406507in}{2.413276in}}{\pgfqpoint{4.395908in}{2.417667in}}{\pgfqpoint{4.384857in}{2.417667in}}%
\pgfpathcurveto{\pgfqpoint{4.373807in}{2.417667in}}{\pgfqpoint{4.363208in}{2.413276in}}{\pgfqpoint{4.355395in}{2.405463in}}%
\pgfpathcurveto{\pgfqpoint{4.347581in}{2.397649in}}{\pgfqpoint{4.343191in}{2.387050in}}{\pgfqpoint{4.343191in}{2.376000in}}%
\pgfpathcurveto{\pgfqpoint{4.343191in}{2.364950in}}{\pgfqpoint{4.347581in}{2.354351in}}{\pgfqpoint{4.355395in}{2.346537in}}%
\pgfpathcurveto{\pgfqpoint{4.363208in}{2.338724in}}{\pgfqpoint{4.373807in}{2.334333in}}{\pgfqpoint{4.384857in}{2.334333in}}%
\pgfpathclose%
\pgfusepath{stroke,fill}%
\end{pgfscope}%
\begin{pgfscope}%
\pgfpathrectangle{\pgfqpoint{0.800000in}{0.528000in}}{\pgfqpoint{4.960000in}{3.696000in}}%
\pgfusepath{clip}%
\pgfsetbuttcap%
\pgfsetroundjoin%
\definecolor{currentfill}{rgb}{0.000000,0.000000,0.000000}%
\pgfsetfillcolor{currentfill}%
\pgfsetlinewidth{1.003750pt}%
\definecolor{currentstroke}{rgb}{0.000000,0.000000,0.000000}%
\pgfsetstrokecolor{currentstroke}%
\pgfsetdash{}{0pt}%
\pgfpathmoveto{\pgfqpoint{4.384857in}{2.334333in}}%
\pgfpathcurveto{\pgfqpoint{4.395908in}{2.334333in}}{\pgfqpoint{4.406507in}{2.338724in}}{\pgfqpoint{4.414320in}{2.346537in}}%
\pgfpathcurveto{\pgfqpoint{4.422134in}{2.354351in}}{\pgfqpoint{4.426524in}{2.364950in}}{\pgfqpoint{4.426524in}{2.376000in}}%
\pgfpathcurveto{\pgfqpoint{4.426524in}{2.387050in}}{\pgfqpoint{4.422134in}{2.397649in}}{\pgfqpoint{4.414320in}{2.405463in}}%
\pgfpathcurveto{\pgfqpoint{4.406507in}{2.413276in}}{\pgfqpoint{4.395908in}{2.417667in}}{\pgfqpoint{4.384857in}{2.417667in}}%
\pgfpathcurveto{\pgfqpoint{4.373807in}{2.417667in}}{\pgfqpoint{4.363208in}{2.413276in}}{\pgfqpoint{4.355395in}{2.405463in}}%
\pgfpathcurveto{\pgfqpoint{4.347581in}{2.397649in}}{\pgfqpoint{4.343191in}{2.387050in}}{\pgfqpoint{4.343191in}{2.376000in}}%
\pgfpathcurveto{\pgfqpoint{4.343191in}{2.364950in}}{\pgfqpoint{4.347581in}{2.354351in}}{\pgfqpoint{4.355395in}{2.346537in}}%
\pgfpathcurveto{\pgfqpoint{4.363208in}{2.338724in}}{\pgfqpoint{4.373807in}{2.334333in}}{\pgfqpoint{4.384857in}{2.334333in}}%
\pgfpathclose%
\pgfusepath{stroke,fill}%
\end{pgfscope}%
\begin{pgfscope}%
\pgfpathrectangle{\pgfqpoint{0.800000in}{0.528000in}}{\pgfqpoint{4.960000in}{3.696000in}}%
\pgfusepath{clip}%
\pgfsetbuttcap%
\pgfsetroundjoin%
\definecolor{currentfill}{rgb}{0.000000,0.000000,0.000000}%
\pgfsetfillcolor{currentfill}%
\pgfsetlinewidth{1.003750pt}%
\definecolor{currentstroke}{rgb}{0.000000,0.000000,0.000000}%
\pgfsetstrokecolor{currentstroke}%
\pgfsetdash{}{0pt}%
\pgfpathmoveto{\pgfqpoint{4.384857in}{2.334333in}}%
\pgfpathcurveto{\pgfqpoint{4.395908in}{2.334333in}}{\pgfqpoint{4.406507in}{2.338724in}}{\pgfqpoint{4.414320in}{2.346537in}}%
\pgfpathcurveto{\pgfqpoint{4.422134in}{2.354351in}}{\pgfqpoint{4.426524in}{2.364950in}}{\pgfqpoint{4.426524in}{2.376000in}}%
\pgfpathcurveto{\pgfqpoint{4.426524in}{2.387050in}}{\pgfqpoint{4.422134in}{2.397649in}}{\pgfqpoint{4.414320in}{2.405463in}}%
\pgfpathcurveto{\pgfqpoint{4.406507in}{2.413276in}}{\pgfqpoint{4.395908in}{2.417667in}}{\pgfqpoint{4.384857in}{2.417667in}}%
\pgfpathcurveto{\pgfqpoint{4.373807in}{2.417667in}}{\pgfqpoint{4.363208in}{2.413276in}}{\pgfqpoint{4.355395in}{2.405463in}}%
\pgfpathcurveto{\pgfqpoint{4.347581in}{2.397649in}}{\pgfqpoint{4.343191in}{2.387050in}}{\pgfqpoint{4.343191in}{2.376000in}}%
\pgfpathcurveto{\pgfqpoint{4.343191in}{2.364950in}}{\pgfqpoint{4.347581in}{2.354351in}}{\pgfqpoint{4.355395in}{2.346537in}}%
\pgfpathcurveto{\pgfqpoint{4.363208in}{2.338724in}}{\pgfqpoint{4.373807in}{2.334333in}}{\pgfqpoint{4.384857in}{2.334333in}}%
\pgfpathclose%
\pgfusepath{stroke,fill}%
\end{pgfscope}%
\begin{pgfscope}%
\pgfpathrectangle{\pgfqpoint{0.800000in}{0.528000in}}{\pgfqpoint{4.960000in}{3.696000in}}%
\pgfusepath{clip}%
\pgfsetbuttcap%
\pgfsetroundjoin%
\definecolor{currentfill}{rgb}{0.000000,0.000000,0.000000}%
\pgfsetfillcolor{currentfill}%
\pgfsetlinewidth{1.003750pt}%
\definecolor{currentstroke}{rgb}{0.000000,0.000000,0.000000}%
\pgfsetstrokecolor{currentstroke}%
\pgfsetdash{}{0pt}%
\pgfpathmoveto{\pgfqpoint{4.384857in}{2.334333in}}%
\pgfpathcurveto{\pgfqpoint{4.395908in}{2.334333in}}{\pgfqpoint{4.406507in}{2.338724in}}{\pgfqpoint{4.414320in}{2.346537in}}%
\pgfpathcurveto{\pgfqpoint{4.422134in}{2.354351in}}{\pgfqpoint{4.426524in}{2.364950in}}{\pgfqpoint{4.426524in}{2.376000in}}%
\pgfpathcurveto{\pgfqpoint{4.426524in}{2.387050in}}{\pgfqpoint{4.422134in}{2.397649in}}{\pgfqpoint{4.414320in}{2.405463in}}%
\pgfpathcurveto{\pgfqpoint{4.406507in}{2.413276in}}{\pgfqpoint{4.395908in}{2.417667in}}{\pgfqpoint{4.384857in}{2.417667in}}%
\pgfpathcurveto{\pgfqpoint{4.373807in}{2.417667in}}{\pgfqpoint{4.363208in}{2.413276in}}{\pgfqpoint{4.355395in}{2.405463in}}%
\pgfpathcurveto{\pgfqpoint{4.347581in}{2.397649in}}{\pgfqpoint{4.343191in}{2.387050in}}{\pgfqpoint{4.343191in}{2.376000in}}%
\pgfpathcurveto{\pgfqpoint{4.343191in}{2.364950in}}{\pgfqpoint{4.347581in}{2.354351in}}{\pgfqpoint{4.355395in}{2.346537in}}%
\pgfpathcurveto{\pgfqpoint{4.363208in}{2.338724in}}{\pgfqpoint{4.373807in}{2.334333in}}{\pgfqpoint{4.384857in}{2.334333in}}%
\pgfpathclose%
\pgfusepath{stroke,fill}%
\end{pgfscope}%
\begin{pgfscope}%
\pgfpathrectangle{\pgfqpoint{0.800000in}{0.528000in}}{\pgfqpoint{4.960000in}{3.696000in}}%
\pgfusepath{clip}%
\pgfsetbuttcap%
\pgfsetroundjoin%
\definecolor{currentfill}{rgb}{0.000000,0.000000,0.000000}%
\pgfsetfillcolor{currentfill}%
\pgfsetlinewidth{1.003750pt}%
\definecolor{currentstroke}{rgb}{0.000000,0.000000,0.000000}%
\pgfsetstrokecolor{currentstroke}%
\pgfsetdash{}{0pt}%
\pgfpathmoveto{\pgfqpoint{4.384857in}{2.334333in}}%
\pgfpathcurveto{\pgfqpoint{4.395908in}{2.334333in}}{\pgfqpoint{4.406507in}{2.338724in}}{\pgfqpoint{4.414320in}{2.346537in}}%
\pgfpathcurveto{\pgfqpoint{4.422134in}{2.354351in}}{\pgfqpoint{4.426524in}{2.364950in}}{\pgfqpoint{4.426524in}{2.376000in}}%
\pgfpathcurveto{\pgfqpoint{4.426524in}{2.387050in}}{\pgfqpoint{4.422134in}{2.397649in}}{\pgfqpoint{4.414320in}{2.405463in}}%
\pgfpathcurveto{\pgfqpoint{4.406507in}{2.413276in}}{\pgfqpoint{4.395908in}{2.417667in}}{\pgfqpoint{4.384857in}{2.417667in}}%
\pgfpathcurveto{\pgfqpoint{4.373807in}{2.417667in}}{\pgfqpoint{4.363208in}{2.413276in}}{\pgfqpoint{4.355395in}{2.405463in}}%
\pgfpathcurveto{\pgfqpoint{4.347581in}{2.397649in}}{\pgfqpoint{4.343191in}{2.387050in}}{\pgfqpoint{4.343191in}{2.376000in}}%
\pgfpathcurveto{\pgfqpoint{4.343191in}{2.364950in}}{\pgfqpoint{4.347581in}{2.354351in}}{\pgfqpoint{4.355395in}{2.346537in}}%
\pgfpathcurveto{\pgfqpoint{4.363208in}{2.338724in}}{\pgfqpoint{4.373807in}{2.334333in}}{\pgfqpoint{4.384857in}{2.334333in}}%
\pgfpathclose%
\pgfusepath{stroke,fill}%
\end{pgfscope}%
\begin{pgfscope}%
\pgfpathrectangle{\pgfqpoint{0.800000in}{0.528000in}}{\pgfqpoint{4.960000in}{3.696000in}}%
\pgfusepath{clip}%
\pgfsetbuttcap%
\pgfsetroundjoin%
\definecolor{currentfill}{rgb}{0.000000,0.000000,0.000000}%
\pgfsetfillcolor{currentfill}%
\pgfsetlinewidth{1.003750pt}%
\definecolor{currentstroke}{rgb}{0.000000,0.000000,0.000000}%
\pgfsetstrokecolor{currentstroke}%
\pgfsetdash{}{0pt}%
\pgfpathmoveto{\pgfqpoint{4.384857in}{2.334333in}}%
\pgfpathcurveto{\pgfqpoint{4.395908in}{2.334333in}}{\pgfqpoint{4.406507in}{2.338724in}}{\pgfqpoint{4.414320in}{2.346537in}}%
\pgfpathcurveto{\pgfqpoint{4.422134in}{2.354351in}}{\pgfqpoint{4.426524in}{2.364950in}}{\pgfqpoint{4.426524in}{2.376000in}}%
\pgfpathcurveto{\pgfqpoint{4.426524in}{2.387050in}}{\pgfqpoint{4.422134in}{2.397649in}}{\pgfqpoint{4.414320in}{2.405463in}}%
\pgfpathcurveto{\pgfqpoint{4.406507in}{2.413276in}}{\pgfqpoint{4.395908in}{2.417667in}}{\pgfqpoint{4.384857in}{2.417667in}}%
\pgfpathcurveto{\pgfqpoint{4.373807in}{2.417667in}}{\pgfqpoint{4.363208in}{2.413276in}}{\pgfqpoint{4.355395in}{2.405463in}}%
\pgfpathcurveto{\pgfqpoint{4.347581in}{2.397649in}}{\pgfqpoint{4.343191in}{2.387050in}}{\pgfqpoint{4.343191in}{2.376000in}}%
\pgfpathcurveto{\pgfqpoint{4.343191in}{2.364950in}}{\pgfqpoint{4.347581in}{2.354351in}}{\pgfqpoint{4.355395in}{2.346537in}}%
\pgfpathcurveto{\pgfqpoint{4.363208in}{2.338724in}}{\pgfqpoint{4.373807in}{2.334333in}}{\pgfqpoint{4.384857in}{2.334333in}}%
\pgfpathclose%
\pgfusepath{stroke,fill}%
\end{pgfscope}%
\begin{pgfscope}%
\pgfpathrectangle{\pgfqpoint{0.800000in}{0.528000in}}{\pgfqpoint{4.960000in}{3.696000in}}%
\pgfusepath{clip}%
\pgfsetbuttcap%
\pgfsetroundjoin%
\definecolor{currentfill}{rgb}{0.000000,0.000000,0.000000}%
\pgfsetfillcolor{currentfill}%
\pgfsetlinewidth{1.003750pt}%
\definecolor{currentstroke}{rgb}{0.000000,0.000000,0.000000}%
\pgfsetstrokecolor{currentstroke}%
\pgfsetdash{}{0pt}%
\pgfpathmoveto{\pgfqpoint{4.384857in}{2.334333in}}%
\pgfpathcurveto{\pgfqpoint{4.395908in}{2.334333in}}{\pgfqpoint{4.406507in}{2.338724in}}{\pgfqpoint{4.414320in}{2.346537in}}%
\pgfpathcurveto{\pgfqpoint{4.422134in}{2.354351in}}{\pgfqpoint{4.426524in}{2.364950in}}{\pgfqpoint{4.426524in}{2.376000in}}%
\pgfpathcurveto{\pgfqpoint{4.426524in}{2.387050in}}{\pgfqpoint{4.422134in}{2.397649in}}{\pgfqpoint{4.414320in}{2.405463in}}%
\pgfpathcurveto{\pgfqpoint{4.406507in}{2.413276in}}{\pgfqpoint{4.395908in}{2.417667in}}{\pgfqpoint{4.384857in}{2.417667in}}%
\pgfpathcurveto{\pgfqpoint{4.373807in}{2.417667in}}{\pgfqpoint{4.363208in}{2.413276in}}{\pgfqpoint{4.355395in}{2.405463in}}%
\pgfpathcurveto{\pgfqpoint{4.347581in}{2.397649in}}{\pgfqpoint{4.343191in}{2.387050in}}{\pgfqpoint{4.343191in}{2.376000in}}%
\pgfpathcurveto{\pgfqpoint{4.343191in}{2.364950in}}{\pgfqpoint{4.347581in}{2.354351in}}{\pgfqpoint{4.355395in}{2.346537in}}%
\pgfpathcurveto{\pgfqpoint{4.363208in}{2.338724in}}{\pgfqpoint{4.373807in}{2.334333in}}{\pgfqpoint{4.384857in}{2.334333in}}%
\pgfpathclose%
\pgfusepath{stroke,fill}%
\end{pgfscope}%
\begin{pgfscope}%
\pgfpathrectangle{\pgfqpoint{0.800000in}{0.528000in}}{\pgfqpoint{4.960000in}{3.696000in}}%
\pgfusepath{clip}%
\pgfsetbuttcap%
\pgfsetroundjoin%
\definecolor{currentfill}{rgb}{0.000000,0.000000,0.000000}%
\pgfsetfillcolor{currentfill}%
\pgfsetlinewidth{1.003750pt}%
\definecolor{currentstroke}{rgb}{0.000000,0.000000,0.000000}%
\pgfsetstrokecolor{currentstroke}%
\pgfsetdash{}{0pt}%
\pgfpathmoveto{\pgfqpoint{4.384857in}{2.334333in}}%
\pgfpathcurveto{\pgfqpoint{4.395908in}{2.334333in}}{\pgfqpoint{4.406507in}{2.338724in}}{\pgfqpoint{4.414320in}{2.346537in}}%
\pgfpathcurveto{\pgfqpoint{4.422134in}{2.354351in}}{\pgfqpoint{4.426524in}{2.364950in}}{\pgfqpoint{4.426524in}{2.376000in}}%
\pgfpathcurveto{\pgfqpoint{4.426524in}{2.387050in}}{\pgfqpoint{4.422134in}{2.397649in}}{\pgfqpoint{4.414320in}{2.405463in}}%
\pgfpathcurveto{\pgfqpoint{4.406507in}{2.413276in}}{\pgfqpoint{4.395908in}{2.417667in}}{\pgfqpoint{4.384857in}{2.417667in}}%
\pgfpathcurveto{\pgfqpoint{4.373807in}{2.417667in}}{\pgfqpoint{4.363208in}{2.413276in}}{\pgfqpoint{4.355395in}{2.405463in}}%
\pgfpathcurveto{\pgfqpoint{4.347581in}{2.397649in}}{\pgfqpoint{4.343191in}{2.387050in}}{\pgfqpoint{4.343191in}{2.376000in}}%
\pgfpathcurveto{\pgfqpoint{4.343191in}{2.364950in}}{\pgfqpoint{4.347581in}{2.354351in}}{\pgfqpoint{4.355395in}{2.346537in}}%
\pgfpathcurveto{\pgfqpoint{4.363208in}{2.338724in}}{\pgfqpoint{4.373807in}{2.334333in}}{\pgfqpoint{4.384857in}{2.334333in}}%
\pgfpathclose%
\pgfusepath{stroke,fill}%
\end{pgfscope}%
\begin{pgfscope}%
\pgfpathrectangle{\pgfqpoint{0.800000in}{0.528000in}}{\pgfqpoint{4.960000in}{3.696000in}}%
\pgfusepath{clip}%
\pgfsetbuttcap%
\pgfsetroundjoin%
\definecolor{currentfill}{rgb}{0.000000,0.000000,0.000000}%
\pgfsetfillcolor{currentfill}%
\pgfsetlinewidth{1.003750pt}%
\definecolor{currentstroke}{rgb}{0.000000,0.000000,0.000000}%
\pgfsetstrokecolor{currentstroke}%
\pgfsetdash{}{0pt}%
\pgfpathmoveto{\pgfqpoint{4.384857in}{2.334333in}}%
\pgfpathcurveto{\pgfqpoint{4.395908in}{2.334333in}}{\pgfqpoint{4.406507in}{2.338724in}}{\pgfqpoint{4.414320in}{2.346537in}}%
\pgfpathcurveto{\pgfqpoint{4.422134in}{2.354351in}}{\pgfqpoint{4.426524in}{2.364950in}}{\pgfqpoint{4.426524in}{2.376000in}}%
\pgfpathcurveto{\pgfqpoint{4.426524in}{2.387050in}}{\pgfqpoint{4.422134in}{2.397649in}}{\pgfqpoint{4.414320in}{2.405463in}}%
\pgfpathcurveto{\pgfqpoint{4.406507in}{2.413276in}}{\pgfqpoint{4.395908in}{2.417667in}}{\pgfqpoint{4.384857in}{2.417667in}}%
\pgfpathcurveto{\pgfqpoint{4.373807in}{2.417667in}}{\pgfqpoint{4.363208in}{2.413276in}}{\pgfqpoint{4.355395in}{2.405463in}}%
\pgfpathcurveto{\pgfqpoint{4.347581in}{2.397649in}}{\pgfqpoint{4.343191in}{2.387050in}}{\pgfqpoint{4.343191in}{2.376000in}}%
\pgfpathcurveto{\pgfqpoint{4.343191in}{2.364950in}}{\pgfqpoint{4.347581in}{2.354351in}}{\pgfqpoint{4.355395in}{2.346537in}}%
\pgfpathcurveto{\pgfqpoint{4.363208in}{2.338724in}}{\pgfqpoint{4.373807in}{2.334333in}}{\pgfqpoint{4.384857in}{2.334333in}}%
\pgfpathclose%
\pgfusepath{stroke,fill}%
\end{pgfscope}%
\begin{pgfscope}%
\pgfpathrectangle{\pgfqpoint{0.800000in}{0.528000in}}{\pgfqpoint{4.960000in}{3.696000in}}%
\pgfusepath{clip}%
\pgfsetbuttcap%
\pgfsetroundjoin%
\definecolor{currentfill}{rgb}{0.000000,0.000000,0.000000}%
\pgfsetfillcolor{currentfill}%
\pgfsetlinewidth{1.003750pt}%
\definecolor{currentstroke}{rgb}{0.000000,0.000000,0.000000}%
\pgfsetstrokecolor{currentstroke}%
\pgfsetdash{}{0pt}%
\pgfpathmoveto{\pgfqpoint{4.384857in}{2.334333in}}%
\pgfpathcurveto{\pgfqpoint{4.395908in}{2.334333in}}{\pgfqpoint{4.406507in}{2.338724in}}{\pgfqpoint{4.414320in}{2.346537in}}%
\pgfpathcurveto{\pgfqpoint{4.422134in}{2.354351in}}{\pgfqpoint{4.426524in}{2.364950in}}{\pgfqpoint{4.426524in}{2.376000in}}%
\pgfpathcurveto{\pgfqpoint{4.426524in}{2.387050in}}{\pgfqpoint{4.422134in}{2.397649in}}{\pgfqpoint{4.414320in}{2.405463in}}%
\pgfpathcurveto{\pgfqpoint{4.406507in}{2.413276in}}{\pgfqpoint{4.395908in}{2.417667in}}{\pgfqpoint{4.384857in}{2.417667in}}%
\pgfpathcurveto{\pgfqpoint{4.373807in}{2.417667in}}{\pgfqpoint{4.363208in}{2.413276in}}{\pgfqpoint{4.355395in}{2.405463in}}%
\pgfpathcurveto{\pgfqpoint{4.347581in}{2.397649in}}{\pgfqpoint{4.343191in}{2.387050in}}{\pgfqpoint{4.343191in}{2.376000in}}%
\pgfpathcurveto{\pgfqpoint{4.343191in}{2.364950in}}{\pgfqpoint{4.347581in}{2.354351in}}{\pgfqpoint{4.355395in}{2.346537in}}%
\pgfpathcurveto{\pgfqpoint{4.363208in}{2.338724in}}{\pgfqpoint{4.373807in}{2.334333in}}{\pgfqpoint{4.384857in}{2.334333in}}%
\pgfpathclose%
\pgfusepath{stroke,fill}%
\end{pgfscope}%
\begin{pgfscope}%
\pgfpathrectangle{\pgfqpoint{0.800000in}{0.528000in}}{\pgfqpoint{4.960000in}{3.696000in}}%
\pgfusepath{clip}%
\pgfsetbuttcap%
\pgfsetroundjoin%
\definecolor{currentfill}{rgb}{0.000000,0.000000,0.000000}%
\pgfsetfillcolor{currentfill}%
\pgfsetlinewidth{1.003750pt}%
\definecolor{currentstroke}{rgb}{0.000000,0.000000,0.000000}%
\pgfsetstrokecolor{currentstroke}%
\pgfsetdash{}{0pt}%
\pgfpathmoveto{\pgfqpoint{4.384857in}{2.334333in}}%
\pgfpathcurveto{\pgfqpoint{4.395908in}{2.334333in}}{\pgfqpoint{4.406507in}{2.338724in}}{\pgfqpoint{4.414320in}{2.346537in}}%
\pgfpathcurveto{\pgfqpoint{4.422134in}{2.354351in}}{\pgfqpoint{4.426524in}{2.364950in}}{\pgfqpoint{4.426524in}{2.376000in}}%
\pgfpathcurveto{\pgfqpoint{4.426524in}{2.387050in}}{\pgfqpoint{4.422134in}{2.397649in}}{\pgfqpoint{4.414320in}{2.405463in}}%
\pgfpathcurveto{\pgfqpoint{4.406507in}{2.413276in}}{\pgfqpoint{4.395908in}{2.417667in}}{\pgfqpoint{4.384857in}{2.417667in}}%
\pgfpathcurveto{\pgfqpoint{4.373807in}{2.417667in}}{\pgfqpoint{4.363208in}{2.413276in}}{\pgfqpoint{4.355395in}{2.405463in}}%
\pgfpathcurveto{\pgfqpoint{4.347581in}{2.397649in}}{\pgfqpoint{4.343191in}{2.387050in}}{\pgfqpoint{4.343191in}{2.376000in}}%
\pgfpathcurveto{\pgfqpoint{4.343191in}{2.364950in}}{\pgfqpoint{4.347581in}{2.354351in}}{\pgfqpoint{4.355395in}{2.346537in}}%
\pgfpathcurveto{\pgfqpoint{4.363208in}{2.338724in}}{\pgfqpoint{4.373807in}{2.334333in}}{\pgfqpoint{4.384857in}{2.334333in}}%
\pgfpathclose%
\pgfusepath{stroke,fill}%
\end{pgfscope}%
\begin{pgfscope}%
\pgfpathrectangle{\pgfqpoint{0.800000in}{0.528000in}}{\pgfqpoint{4.960000in}{3.696000in}}%
\pgfusepath{clip}%
\pgfsetbuttcap%
\pgfsetroundjoin%
\definecolor{currentfill}{rgb}{0.000000,0.000000,0.000000}%
\pgfsetfillcolor{currentfill}%
\pgfsetlinewidth{1.003750pt}%
\definecolor{currentstroke}{rgb}{0.000000,0.000000,0.000000}%
\pgfsetstrokecolor{currentstroke}%
\pgfsetdash{}{0pt}%
\pgfpathmoveto{\pgfqpoint{4.384857in}{2.334333in}}%
\pgfpathcurveto{\pgfqpoint{4.395908in}{2.334333in}}{\pgfqpoint{4.406507in}{2.338724in}}{\pgfqpoint{4.414320in}{2.346537in}}%
\pgfpathcurveto{\pgfqpoint{4.422134in}{2.354351in}}{\pgfqpoint{4.426524in}{2.364950in}}{\pgfqpoint{4.426524in}{2.376000in}}%
\pgfpathcurveto{\pgfqpoint{4.426524in}{2.387050in}}{\pgfqpoint{4.422134in}{2.397649in}}{\pgfqpoint{4.414320in}{2.405463in}}%
\pgfpathcurveto{\pgfqpoint{4.406507in}{2.413276in}}{\pgfqpoint{4.395908in}{2.417667in}}{\pgfqpoint{4.384857in}{2.417667in}}%
\pgfpathcurveto{\pgfqpoint{4.373807in}{2.417667in}}{\pgfqpoint{4.363208in}{2.413276in}}{\pgfqpoint{4.355395in}{2.405463in}}%
\pgfpathcurveto{\pgfqpoint{4.347581in}{2.397649in}}{\pgfqpoint{4.343191in}{2.387050in}}{\pgfqpoint{4.343191in}{2.376000in}}%
\pgfpathcurveto{\pgfqpoint{4.343191in}{2.364950in}}{\pgfqpoint{4.347581in}{2.354351in}}{\pgfqpoint{4.355395in}{2.346537in}}%
\pgfpathcurveto{\pgfqpoint{4.363208in}{2.338724in}}{\pgfqpoint{4.373807in}{2.334333in}}{\pgfqpoint{4.384857in}{2.334333in}}%
\pgfpathclose%
\pgfusepath{stroke,fill}%
\end{pgfscope}%
\begin{pgfscope}%
\pgfpathrectangle{\pgfqpoint{0.800000in}{0.528000in}}{\pgfqpoint{4.960000in}{3.696000in}}%
\pgfusepath{clip}%
\pgfsetbuttcap%
\pgfsetroundjoin%
\definecolor{currentfill}{rgb}{0.000000,0.000000,0.000000}%
\pgfsetfillcolor{currentfill}%
\pgfsetlinewidth{1.003750pt}%
\definecolor{currentstroke}{rgb}{0.000000,0.000000,0.000000}%
\pgfsetstrokecolor{currentstroke}%
\pgfsetdash{}{0pt}%
\pgfpathmoveto{\pgfqpoint{4.384857in}{2.334333in}}%
\pgfpathcurveto{\pgfqpoint{4.395908in}{2.334333in}}{\pgfqpoint{4.406507in}{2.338724in}}{\pgfqpoint{4.414320in}{2.346537in}}%
\pgfpathcurveto{\pgfqpoint{4.422134in}{2.354351in}}{\pgfqpoint{4.426524in}{2.364950in}}{\pgfqpoint{4.426524in}{2.376000in}}%
\pgfpathcurveto{\pgfqpoint{4.426524in}{2.387050in}}{\pgfqpoint{4.422134in}{2.397649in}}{\pgfqpoint{4.414320in}{2.405463in}}%
\pgfpathcurveto{\pgfqpoint{4.406507in}{2.413276in}}{\pgfqpoint{4.395908in}{2.417667in}}{\pgfqpoint{4.384857in}{2.417667in}}%
\pgfpathcurveto{\pgfqpoint{4.373807in}{2.417667in}}{\pgfqpoint{4.363208in}{2.413276in}}{\pgfqpoint{4.355395in}{2.405463in}}%
\pgfpathcurveto{\pgfqpoint{4.347581in}{2.397649in}}{\pgfqpoint{4.343191in}{2.387050in}}{\pgfqpoint{4.343191in}{2.376000in}}%
\pgfpathcurveto{\pgfqpoint{4.343191in}{2.364950in}}{\pgfqpoint{4.347581in}{2.354351in}}{\pgfqpoint{4.355395in}{2.346537in}}%
\pgfpathcurveto{\pgfqpoint{4.363208in}{2.338724in}}{\pgfqpoint{4.373807in}{2.334333in}}{\pgfqpoint{4.384857in}{2.334333in}}%
\pgfpathclose%
\pgfusepath{stroke,fill}%
\end{pgfscope}%
\begin{pgfscope}%
\pgfpathrectangle{\pgfqpoint{0.800000in}{0.528000in}}{\pgfqpoint{4.960000in}{3.696000in}}%
\pgfusepath{clip}%
\pgfsetbuttcap%
\pgfsetroundjoin%
\definecolor{currentfill}{rgb}{0.000000,0.000000,0.000000}%
\pgfsetfillcolor{currentfill}%
\pgfsetlinewidth{1.003750pt}%
\definecolor{currentstroke}{rgb}{0.000000,0.000000,0.000000}%
\pgfsetstrokecolor{currentstroke}%
\pgfsetdash{}{0pt}%
\pgfpathmoveto{\pgfqpoint{4.384857in}{2.334333in}}%
\pgfpathcurveto{\pgfqpoint{4.395908in}{2.334333in}}{\pgfqpoint{4.406507in}{2.338724in}}{\pgfqpoint{4.414320in}{2.346537in}}%
\pgfpathcurveto{\pgfqpoint{4.422134in}{2.354351in}}{\pgfqpoint{4.426524in}{2.364950in}}{\pgfqpoint{4.426524in}{2.376000in}}%
\pgfpathcurveto{\pgfqpoint{4.426524in}{2.387050in}}{\pgfqpoint{4.422134in}{2.397649in}}{\pgfqpoint{4.414320in}{2.405463in}}%
\pgfpathcurveto{\pgfqpoint{4.406507in}{2.413276in}}{\pgfqpoint{4.395908in}{2.417667in}}{\pgfqpoint{4.384857in}{2.417667in}}%
\pgfpathcurveto{\pgfqpoint{4.373807in}{2.417667in}}{\pgfqpoint{4.363208in}{2.413276in}}{\pgfqpoint{4.355395in}{2.405463in}}%
\pgfpathcurveto{\pgfqpoint{4.347581in}{2.397649in}}{\pgfqpoint{4.343191in}{2.387050in}}{\pgfqpoint{4.343191in}{2.376000in}}%
\pgfpathcurveto{\pgfqpoint{4.343191in}{2.364950in}}{\pgfqpoint{4.347581in}{2.354351in}}{\pgfqpoint{4.355395in}{2.346537in}}%
\pgfpathcurveto{\pgfqpoint{4.363208in}{2.338724in}}{\pgfqpoint{4.373807in}{2.334333in}}{\pgfqpoint{4.384857in}{2.334333in}}%
\pgfpathclose%
\pgfusepath{stroke,fill}%
\end{pgfscope}%
\begin{pgfscope}%
\pgfpathrectangle{\pgfqpoint{0.800000in}{0.528000in}}{\pgfqpoint{4.960000in}{3.696000in}}%
\pgfusepath{clip}%
\pgfsetbuttcap%
\pgfsetroundjoin%
\definecolor{currentfill}{rgb}{0.000000,0.000000,0.000000}%
\pgfsetfillcolor{currentfill}%
\pgfsetlinewidth{1.003750pt}%
\definecolor{currentstroke}{rgb}{0.000000,0.000000,0.000000}%
\pgfsetstrokecolor{currentstroke}%
\pgfsetdash{}{0pt}%
\pgfpathmoveto{\pgfqpoint{4.384857in}{2.334333in}}%
\pgfpathcurveto{\pgfqpoint{4.395908in}{2.334333in}}{\pgfqpoint{4.406507in}{2.338724in}}{\pgfqpoint{4.414320in}{2.346537in}}%
\pgfpathcurveto{\pgfqpoint{4.422134in}{2.354351in}}{\pgfqpoint{4.426524in}{2.364950in}}{\pgfqpoint{4.426524in}{2.376000in}}%
\pgfpathcurveto{\pgfqpoint{4.426524in}{2.387050in}}{\pgfqpoint{4.422134in}{2.397649in}}{\pgfqpoint{4.414320in}{2.405463in}}%
\pgfpathcurveto{\pgfqpoint{4.406507in}{2.413276in}}{\pgfqpoint{4.395908in}{2.417667in}}{\pgfqpoint{4.384857in}{2.417667in}}%
\pgfpathcurveto{\pgfqpoint{4.373807in}{2.417667in}}{\pgfqpoint{4.363208in}{2.413276in}}{\pgfqpoint{4.355395in}{2.405463in}}%
\pgfpathcurveto{\pgfqpoint{4.347581in}{2.397649in}}{\pgfqpoint{4.343191in}{2.387050in}}{\pgfqpoint{4.343191in}{2.376000in}}%
\pgfpathcurveto{\pgfqpoint{4.343191in}{2.364950in}}{\pgfqpoint{4.347581in}{2.354351in}}{\pgfqpoint{4.355395in}{2.346537in}}%
\pgfpathcurveto{\pgfqpoint{4.363208in}{2.338724in}}{\pgfqpoint{4.373807in}{2.334333in}}{\pgfqpoint{4.384857in}{2.334333in}}%
\pgfpathclose%
\pgfusepath{stroke,fill}%
\end{pgfscope}%
\begin{pgfscope}%
\pgfpathrectangle{\pgfqpoint{0.800000in}{0.528000in}}{\pgfqpoint{4.960000in}{3.696000in}}%
\pgfusepath{clip}%
\pgfsetbuttcap%
\pgfsetroundjoin%
\definecolor{currentfill}{rgb}{0.000000,0.000000,0.000000}%
\pgfsetfillcolor{currentfill}%
\pgfsetlinewidth{1.003750pt}%
\definecolor{currentstroke}{rgb}{0.000000,0.000000,0.000000}%
\pgfsetstrokecolor{currentstroke}%
\pgfsetdash{}{0pt}%
\pgfpathmoveto{\pgfqpoint{4.384857in}{2.334333in}}%
\pgfpathcurveto{\pgfqpoint{4.395908in}{2.334333in}}{\pgfqpoint{4.406507in}{2.338724in}}{\pgfqpoint{4.414320in}{2.346537in}}%
\pgfpathcurveto{\pgfqpoint{4.422134in}{2.354351in}}{\pgfqpoint{4.426524in}{2.364950in}}{\pgfqpoint{4.426524in}{2.376000in}}%
\pgfpathcurveto{\pgfqpoint{4.426524in}{2.387050in}}{\pgfqpoint{4.422134in}{2.397649in}}{\pgfqpoint{4.414320in}{2.405463in}}%
\pgfpathcurveto{\pgfqpoint{4.406507in}{2.413276in}}{\pgfqpoint{4.395908in}{2.417667in}}{\pgfqpoint{4.384857in}{2.417667in}}%
\pgfpathcurveto{\pgfqpoint{4.373807in}{2.417667in}}{\pgfqpoint{4.363208in}{2.413276in}}{\pgfqpoint{4.355395in}{2.405463in}}%
\pgfpathcurveto{\pgfqpoint{4.347581in}{2.397649in}}{\pgfqpoint{4.343191in}{2.387050in}}{\pgfqpoint{4.343191in}{2.376000in}}%
\pgfpathcurveto{\pgfqpoint{4.343191in}{2.364950in}}{\pgfqpoint{4.347581in}{2.354351in}}{\pgfqpoint{4.355395in}{2.346537in}}%
\pgfpathcurveto{\pgfqpoint{4.363208in}{2.338724in}}{\pgfqpoint{4.373807in}{2.334333in}}{\pgfqpoint{4.384857in}{2.334333in}}%
\pgfpathclose%
\pgfusepath{stroke,fill}%
\end{pgfscope}%
\begin{pgfscope}%
\pgfpathrectangle{\pgfqpoint{0.800000in}{0.528000in}}{\pgfqpoint{4.960000in}{3.696000in}}%
\pgfusepath{clip}%
\pgfsetbuttcap%
\pgfsetroundjoin%
\definecolor{currentfill}{rgb}{0.000000,0.000000,0.000000}%
\pgfsetfillcolor{currentfill}%
\pgfsetlinewidth{1.003750pt}%
\definecolor{currentstroke}{rgb}{0.000000,0.000000,0.000000}%
\pgfsetstrokecolor{currentstroke}%
\pgfsetdash{}{0pt}%
\pgfpathmoveto{\pgfqpoint{4.384857in}{2.334333in}}%
\pgfpathcurveto{\pgfqpoint{4.395908in}{2.334333in}}{\pgfqpoint{4.406507in}{2.338724in}}{\pgfqpoint{4.414320in}{2.346537in}}%
\pgfpathcurveto{\pgfqpoint{4.422134in}{2.354351in}}{\pgfqpoint{4.426524in}{2.364950in}}{\pgfqpoint{4.426524in}{2.376000in}}%
\pgfpathcurveto{\pgfqpoint{4.426524in}{2.387050in}}{\pgfqpoint{4.422134in}{2.397649in}}{\pgfqpoint{4.414320in}{2.405463in}}%
\pgfpathcurveto{\pgfqpoint{4.406507in}{2.413276in}}{\pgfqpoint{4.395908in}{2.417667in}}{\pgfqpoint{4.384857in}{2.417667in}}%
\pgfpathcurveto{\pgfqpoint{4.373807in}{2.417667in}}{\pgfqpoint{4.363208in}{2.413276in}}{\pgfqpoint{4.355395in}{2.405463in}}%
\pgfpathcurveto{\pgfqpoint{4.347581in}{2.397649in}}{\pgfqpoint{4.343191in}{2.387050in}}{\pgfqpoint{4.343191in}{2.376000in}}%
\pgfpathcurveto{\pgfqpoint{4.343191in}{2.364950in}}{\pgfqpoint{4.347581in}{2.354351in}}{\pgfqpoint{4.355395in}{2.346537in}}%
\pgfpathcurveto{\pgfqpoint{4.363208in}{2.338724in}}{\pgfqpoint{4.373807in}{2.334333in}}{\pgfqpoint{4.384857in}{2.334333in}}%
\pgfpathclose%
\pgfusepath{stroke,fill}%
\end{pgfscope}%
\begin{pgfscope}%
\pgfpathrectangle{\pgfqpoint{0.800000in}{0.528000in}}{\pgfqpoint{4.960000in}{3.696000in}}%
\pgfusepath{clip}%
\pgfsetbuttcap%
\pgfsetroundjoin%
\definecolor{currentfill}{rgb}{0.000000,0.000000,0.000000}%
\pgfsetfillcolor{currentfill}%
\pgfsetlinewidth{1.003750pt}%
\definecolor{currentstroke}{rgb}{0.000000,0.000000,0.000000}%
\pgfsetstrokecolor{currentstroke}%
\pgfsetdash{}{0pt}%
\pgfpathmoveto{\pgfqpoint{4.384857in}{2.334333in}}%
\pgfpathcurveto{\pgfqpoint{4.395908in}{2.334333in}}{\pgfqpoint{4.406507in}{2.338724in}}{\pgfqpoint{4.414320in}{2.346537in}}%
\pgfpathcurveto{\pgfqpoint{4.422134in}{2.354351in}}{\pgfqpoint{4.426524in}{2.364950in}}{\pgfqpoint{4.426524in}{2.376000in}}%
\pgfpathcurveto{\pgfqpoint{4.426524in}{2.387050in}}{\pgfqpoint{4.422134in}{2.397649in}}{\pgfqpoint{4.414320in}{2.405463in}}%
\pgfpathcurveto{\pgfqpoint{4.406507in}{2.413276in}}{\pgfqpoint{4.395908in}{2.417667in}}{\pgfqpoint{4.384857in}{2.417667in}}%
\pgfpathcurveto{\pgfqpoint{4.373807in}{2.417667in}}{\pgfqpoint{4.363208in}{2.413276in}}{\pgfqpoint{4.355395in}{2.405463in}}%
\pgfpathcurveto{\pgfqpoint{4.347581in}{2.397649in}}{\pgfqpoint{4.343191in}{2.387050in}}{\pgfqpoint{4.343191in}{2.376000in}}%
\pgfpathcurveto{\pgfqpoint{4.343191in}{2.364950in}}{\pgfqpoint{4.347581in}{2.354351in}}{\pgfqpoint{4.355395in}{2.346537in}}%
\pgfpathcurveto{\pgfqpoint{4.363208in}{2.338724in}}{\pgfqpoint{4.373807in}{2.334333in}}{\pgfqpoint{4.384857in}{2.334333in}}%
\pgfpathclose%
\pgfusepath{stroke,fill}%
\end{pgfscope}%
\begin{pgfscope}%
\pgfpathrectangle{\pgfqpoint{0.800000in}{0.528000in}}{\pgfqpoint{4.960000in}{3.696000in}}%
\pgfusepath{clip}%
\pgfsetbuttcap%
\pgfsetroundjoin%
\definecolor{currentfill}{rgb}{0.000000,0.000000,0.000000}%
\pgfsetfillcolor{currentfill}%
\pgfsetlinewidth{1.003750pt}%
\definecolor{currentstroke}{rgb}{0.000000,0.000000,0.000000}%
\pgfsetstrokecolor{currentstroke}%
\pgfsetdash{}{0pt}%
\pgfpathmoveto{\pgfqpoint{4.384857in}{2.334333in}}%
\pgfpathcurveto{\pgfqpoint{4.395908in}{2.334333in}}{\pgfqpoint{4.406507in}{2.338724in}}{\pgfqpoint{4.414320in}{2.346537in}}%
\pgfpathcurveto{\pgfqpoint{4.422134in}{2.354351in}}{\pgfqpoint{4.426524in}{2.364950in}}{\pgfqpoint{4.426524in}{2.376000in}}%
\pgfpathcurveto{\pgfqpoint{4.426524in}{2.387050in}}{\pgfqpoint{4.422134in}{2.397649in}}{\pgfqpoint{4.414320in}{2.405463in}}%
\pgfpathcurveto{\pgfqpoint{4.406507in}{2.413276in}}{\pgfqpoint{4.395908in}{2.417667in}}{\pgfqpoint{4.384857in}{2.417667in}}%
\pgfpathcurveto{\pgfqpoint{4.373807in}{2.417667in}}{\pgfqpoint{4.363208in}{2.413276in}}{\pgfqpoint{4.355395in}{2.405463in}}%
\pgfpathcurveto{\pgfqpoint{4.347581in}{2.397649in}}{\pgfqpoint{4.343191in}{2.387050in}}{\pgfqpoint{4.343191in}{2.376000in}}%
\pgfpathcurveto{\pgfqpoint{4.343191in}{2.364950in}}{\pgfqpoint{4.347581in}{2.354351in}}{\pgfqpoint{4.355395in}{2.346537in}}%
\pgfpathcurveto{\pgfqpoint{4.363208in}{2.338724in}}{\pgfqpoint{4.373807in}{2.334333in}}{\pgfqpoint{4.384857in}{2.334333in}}%
\pgfpathclose%
\pgfusepath{stroke,fill}%
\end{pgfscope}%
\begin{pgfscope}%
\pgfpathrectangle{\pgfqpoint{0.800000in}{0.528000in}}{\pgfqpoint{4.960000in}{3.696000in}}%
\pgfusepath{clip}%
\pgfsetbuttcap%
\pgfsetroundjoin%
\definecolor{currentfill}{rgb}{0.000000,0.000000,0.000000}%
\pgfsetfillcolor{currentfill}%
\pgfsetlinewidth{1.003750pt}%
\definecolor{currentstroke}{rgb}{0.000000,0.000000,0.000000}%
\pgfsetstrokecolor{currentstroke}%
\pgfsetdash{}{0pt}%
\pgfpathmoveto{\pgfqpoint{4.384857in}{2.334333in}}%
\pgfpathcurveto{\pgfqpoint{4.395908in}{2.334333in}}{\pgfqpoint{4.406507in}{2.338724in}}{\pgfqpoint{4.414320in}{2.346537in}}%
\pgfpathcurveto{\pgfqpoint{4.422134in}{2.354351in}}{\pgfqpoint{4.426524in}{2.364950in}}{\pgfqpoint{4.426524in}{2.376000in}}%
\pgfpathcurveto{\pgfqpoint{4.426524in}{2.387050in}}{\pgfqpoint{4.422134in}{2.397649in}}{\pgfqpoint{4.414320in}{2.405463in}}%
\pgfpathcurveto{\pgfqpoint{4.406507in}{2.413276in}}{\pgfqpoint{4.395908in}{2.417667in}}{\pgfqpoint{4.384857in}{2.417667in}}%
\pgfpathcurveto{\pgfqpoint{4.373807in}{2.417667in}}{\pgfqpoint{4.363208in}{2.413276in}}{\pgfqpoint{4.355395in}{2.405463in}}%
\pgfpathcurveto{\pgfqpoint{4.347581in}{2.397649in}}{\pgfqpoint{4.343191in}{2.387050in}}{\pgfqpoint{4.343191in}{2.376000in}}%
\pgfpathcurveto{\pgfqpoint{4.343191in}{2.364950in}}{\pgfqpoint{4.347581in}{2.354351in}}{\pgfqpoint{4.355395in}{2.346537in}}%
\pgfpathcurveto{\pgfqpoint{4.363208in}{2.338724in}}{\pgfqpoint{4.373807in}{2.334333in}}{\pgfqpoint{4.384857in}{2.334333in}}%
\pgfpathclose%
\pgfusepath{stroke,fill}%
\end{pgfscope}%
\begin{pgfscope}%
\pgfpathrectangle{\pgfqpoint{0.800000in}{0.528000in}}{\pgfqpoint{4.960000in}{3.696000in}}%
\pgfusepath{clip}%
\pgfsetbuttcap%
\pgfsetroundjoin%
\definecolor{currentfill}{rgb}{0.000000,0.000000,0.000000}%
\pgfsetfillcolor{currentfill}%
\pgfsetlinewidth{1.003750pt}%
\definecolor{currentstroke}{rgb}{0.000000,0.000000,0.000000}%
\pgfsetstrokecolor{currentstroke}%
\pgfsetdash{}{0pt}%
\pgfpathmoveto{\pgfqpoint{4.384857in}{2.334333in}}%
\pgfpathcurveto{\pgfqpoint{4.395908in}{2.334333in}}{\pgfqpoint{4.406507in}{2.338724in}}{\pgfqpoint{4.414320in}{2.346537in}}%
\pgfpathcurveto{\pgfqpoint{4.422134in}{2.354351in}}{\pgfqpoint{4.426524in}{2.364950in}}{\pgfqpoint{4.426524in}{2.376000in}}%
\pgfpathcurveto{\pgfqpoint{4.426524in}{2.387050in}}{\pgfqpoint{4.422134in}{2.397649in}}{\pgfqpoint{4.414320in}{2.405463in}}%
\pgfpathcurveto{\pgfqpoint{4.406507in}{2.413276in}}{\pgfqpoint{4.395908in}{2.417667in}}{\pgfqpoint{4.384857in}{2.417667in}}%
\pgfpathcurveto{\pgfqpoint{4.373807in}{2.417667in}}{\pgfqpoint{4.363208in}{2.413276in}}{\pgfqpoint{4.355395in}{2.405463in}}%
\pgfpathcurveto{\pgfqpoint{4.347581in}{2.397649in}}{\pgfqpoint{4.343191in}{2.387050in}}{\pgfqpoint{4.343191in}{2.376000in}}%
\pgfpathcurveto{\pgfqpoint{4.343191in}{2.364950in}}{\pgfqpoint{4.347581in}{2.354351in}}{\pgfqpoint{4.355395in}{2.346537in}}%
\pgfpathcurveto{\pgfqpoint{4.363208in}{2.338724in}}{\pgfqpoint{4.373807in}{2.334333in}}{\pgfqpoint{4.384857in}{2.334333in}}%
\pgfpathclose%
\pgfusepath{stroke,fill}%
\end{pgfscope}%
\begin{pgfscope}%
\pgfpathrectangle{\pgfqpoint{0.800000in}{0.528000in}}{\pgfqpoint{4.960000in}{3.696000in}}%
\pgfusepath{clip}%
\pgfsetbuttcap%
\pgfsetroundjoin%
\definecolor{currentfill}{rgb}{0.000000,0.000000,0.000000}%
\pgfsetfillcolor{currentfill}%
\pgfsetlinewidth{1.003750pt}%
\definecolor{currentstroke}{rgb}{0.000000,0.000000,0.000000}%
\pgfsetstrokecolor{currentstroke}%
\pgfsetdash{}{0pt}%
\pgfpathmoveto{\pgfqpoint{4.384857in}{2.334333in}}%
\pgfpathcurveto{\pgfqpoint{4.395908in}{2.334333in}}{\pgfqpoint{4.406507in}{2.338724in}}{\pgfqpoint{4.414320in}{2.346537in}}%
\pgfpathcurveto{\pgfqpoint{4.422134in}{2.354351in}}{\pgfqpoint{4.426524in}{2.364950in}}{\pgfqpoint{4.426524in}{2.376000in}}%
\pgfpathcurveto{\pgfqpoint{4.426524in}{2.387050in}}{\pgfqpoint{4.422134in}{2.397649in}}{\pgfqpoint{4.414320in}{2.405463in}}%
\pgfpathcurveto{\pgfqpoint{4.406507in}{2.413276in}}{\pgfqpoint{4.395908in}{2.417667in}}{\pgfqpoint{4.384857in}{2.417667in}}%
\pgfpathcurveto{\pgfqpoint{4.373807in}{2.417667in}}{\pgfqpoint{4.363208in}{2.413276in}}{\pgfqpoint{4.355395in}{2.405463in}}%
\pgfpathcurveto{\pgfqpoint{4.347581in}{2.397649in}}{\pgfqpoint{4.343191in}{2.387050in}}{\pgfqpoint{4.343191in}{2.376000in}}%
\pgfpathcurveto{\pgfqpoint{4.343191in}{2.364950in}}{\pgfqpoint{4.347581in}{2.354351in}}{\pgfqpoint{4.355395in}{2.346537in}}%
\pgfpathcurveto{\pgfqpoint{4.363208in}{2.338724in}}{\pgfqpoint{4.373807in}{2.334333in}}{\pgfqpoint{4.384857in}{2.334333in}}%
\pgfpathclose%
\pgfusepath{stroke,fill}%
\end{pgfscope}%
\begin{pgfscope}%
\pgfpathrectangle{\pgfqpoint{0.800000in}{0.528000in}}{\pgfqpoint{4.960000in}{3.696000in}}%
\pgfusepath{clip}%
\pgfsetbuttcap%
\pgfsetroundjoin%
\definecolor{currentfill}{rgb}{0.000000,0.000000,0.000000}%
\pgfsetfillcolor{currentfill}%
\pgfsetlinewidth{1.003750pt}%
\definecolor{currentstroke}{rgb}{0.000000,0.000000,0.000000}%
\pgfsetstrokecolor{currentstroke}%
\pgfsetdash{}{0pt}%
\pgfpathmoveto{\pgfqpoint{4.384857in}{2.334333in}}%
\pgfpathcurveto{\pgfqpoint{4.395908in}{2.334333in}}{\pgfqpoint{4.406507in}{2.338724in}}{\pgfqpoint{4.414320in}{2.346537in}}%
\pgfpathcurveto{\pgfqpoint{4.422134in}{2.354351in}}{\pgfqpoint{4.426524in}{2.364950in}}{\pgfqpoint{4.426524in}{2.376000in}}%
\pgfpathcurveto{\pgfqpoint{4.426524in}{2.387050in}}{\pgfqpoint{4.422134in}{2.397649in}}{\pgfqpoint{4.414320in}{2.405463in}}%
\pgfpathcurveto{\pgfqpoint{4.406507in}{2.413276in}}{\pgfqpoint{4.395908in}{2.417667in}}{\pgfqpoint{4.384857in}{2.417667in}}%
\pgfpathcurveto{\pgfqpoint{4.373807in}{2.417667in}}{\pgfqpoint{4.363208in}{2.413276in}}{\pgfqpoint{4.355395in}{2.405463in}}%
\pgfpathcurveto{\pgfqpoint{4.347581in}{2.397649in}}{\pgfqpoint{4.343191in}{2.387050in}}{\pgfqpoint{4.343191in}{2.376000in}}%
\pgfpathcurveto{\pgfqpoint{4.343191in}{2.364950in}}{\pgfqpoint{4.347581in}{2.354351in}}{\pgfqpoint{4.355395in}{2.346537in}}%
\pgfpathcurveto{\pgfqpoint{4.363208in}{2.338724in}}{\pgfqpoint{4.373807in}{2.334333in}}{\pgfqpoint{4.384857in}{2.334333in}}%
\pgfpathclose%
\pgfusepath{stroke,fill}%
\end{pgfscope}%
\begin{pgfscope}%
\pgfpathrectangle{\pgfqpoint{0.800000in}{0.528000in}}{\pgfqpoint{4.960000in}{3.696000in}}%
\pgfusepath{clip}%
\pgfsetbuttcap%
\pgfsetroundjoin%
\definecolor{currentfill}{rgb}{0.000000,0.000000,0.000000}%
\pgfsetfillcolor{currentfill}%
\pgfsetlinewidth{1.003750pt}%
\definecolor{currentstroke}{rgb}{0.000000,0.000000,0.000000}%
\pgfsetstrokecolor{currentstroke}%
\pgfsetdash{}{0pt}%
\pgfpathmoveto{\pgfqpoint{4.384857in}{2.334333in}}%
\pgfpathcurveto{\pgfqpoint{4.395908in}{2.334333in}}{\pgfqpoint{4.406507in}{2.338724in}}{\pgfqpoint{4.414320in}{2.346537in}}%
\pgfpathcurveto{\pgfqpoint{4.422134in}{2.354351in}}{\pgfqpoint{4.426524in}{2.364950in}}{\pgfqpoint{4.426524in}{2.376000in}}%
\pgfpathcurveto{\pgfqpoint{4.426524in}{2.387050in}}{\pgfqpoint{4.422134in}{2.397649in}}{\pgfqpoint{4.414320in}{2.405463in}}%
\pgfpathcurveto{\pgfqpoint{4.406507in}{2.413276in}}{\pgfqpoint{4.395908in}{2.417667in}}{\pgfqpoint{4.384857in}{2.417667in}}%
\pgfpathcurveto{\pgfqpoint{4.373807in}{2.417667in}}{\pgfqpoint{4.363208in}{2.413276in}}{\pgfqpoint{4.355395in}{2.405463in}}%
\pgfpathcurveto{\pgfqpoint{4.347581in}{2.397649in}}{\pgfqpoint{4.343191in}{2.387050in}}{\pgfqpoint{4.343191in}{2.376000in}}%
\pgfpathcurveto{\pgfqpoint{4.343191in}{2.364950in}}{\pgfqpoint{4.347581in}{2.354351in}}{\pgfqpoint{4.355395in}{2.346537in}}%
\pgfpathcurveto{\pgfqpoint{4.363208in}{2.338724in}}{\pgfqpoint{4.373807in}{2.334333in}}{\pgfqpoint{4.384857in}{2.334333in}}%
\pgfpathclose%
\pgfusepath{stroke,fill}%
\end{pgfscope}%
\begin{pgfscope}%
\pgfpathrectangle{\pgfqpoint{0.800000in}{0.528000in}}{\pgfqpoint{4.960000in}{3.696000in}}%
\pgfusepath{clip}%
\pgfsetbuttcap%
\pgfsetroundjoin%
\definecolor{currentfill}{rgb}{0.000000,0.000000,0.000000}%
\pgfsetfillcolor{currentfill}%
\pgfsetlinewidth{1.003750pt}%
\definecolor{currentstroke}{rgb}{0.000000,0.000000,0.000000}%
\pgfsetstrokecolor{currentstroke}%
\pgfsetdash{}{0pt}%
\pgfpathmoveto{\pgfqpoint{4.384857in}{2.334333in}}%
\pgfpathcurveto{\pgfqpoint{4.395908in}{2.334333in}}{\pgfqpoint{4.406507in}{2.338724in}}{\pgfqpoint{4.414320in}{2.346537in}}%
\pgfpathcurveto{\pgfqpoint{4.422134in}{2.354351in}}{\pgfqpoint{4.426524in}{2.364950in}}{\pgfqpoint{4.426524in}{2.376000in}}%
\pgfpathcurveto{\pgfqpoint{4.426524in}{2.387050in}}{\pgfqpoint{4.422134in}{2.397649in}}{\pgfqpoint{4.414320in}{2.405463in}}%
\pgfpathcurveto{\pgfqpoint{4.406507in}{2.413276in}}{\pgfqpoint{4.395908in}{2.417667in}}{\pgfqpoint{4.384857in}{2.417667in}}%
\pgfpathcurveto{\pgfqpoint{4.373807in}{2.417667in}}{\pgfqpoint{4.363208in}{2.413276in}}{\pgfqpoint{4.355395in}{2.405463in}}%
\pgfpathcurveto{\pgfqpoint{4.347581in}{2.397649in}}{\pgfqpoint{4.343191in}{2.387050in}}{\pgfqpoint{4.343191in}{2.376000in}}%
\pgfpathcurveto{\pgfqpoint{4.343191in}{2.364950in}}{\pgfqpoint{4.347581in}{2.354351in}}{\pgfqpoint{4.355395in}{2.346537in}}%
\pgfpathcurveto{\pgfqpoint{4.363208in}{2.338724in}}{\pgfqpoint{4.373807in}{2.334333in}}{\pgfqpoint{4.384857in}{2.334333in}}%
\pgfpathclose%
\pgfusepath{stroke,fill}%
\end{pgfscope}%
\begin{pgfscope}%
\pgfpathrectangle{\pgfqpoint{0.800000in}{0.528000in}}{\pgfqpoint{4.960000in}{3.696000in}}%
\pgfusepath{clip}%
\pgfsetbuttcap%
\pgfsetroundjoin%
\definecolor{currentfill}{rgb}{0.000000,0.000000,0.000000}%
\pgfsetfillcolor{currentfill}%
\pgfsetlinewidth{1.003750pt}%
\definecolor{currentstroke}{rgb}{0.000000,0.000000,0.000000}%
\pgfsetstrokecolor{currentstroke}%
\pgfsetdash{}{0pt}%
\pgfpathmoveto{\pgfqpoint{4.384857in}{2.334333in}}%
\pgfpathcurveto{\pgfqpoint{4.395908in}{2.334333in}}{\pgfqpoint{4.406507in}{2.338724in}}{\pgfqpoint{4.414320in}{2.346537in}}%
\pgfpathcurveto{\pgfqpoint{4.422134in}{2.354351in}}{\pgfqpoint{4.426524in}{2.364950in}}{\pgfqpoint{4.426524in}{2.376000in}}%
\pgfpathcurveto{\pgfqpoint{4.426524in}{2.387050in}}{\pgfqpoint{4.422134in}{2.397649in}}{\pgfqpoint{4.414320in}{2.405463in}}%
\pgfpathcurveto{\pgfqpoint{4.406507in}{2.413276in}}{\pgfqpoint{4.395908in}{2.417667in}}{\pgfqpoint{4.384857in}{2.417667in}}%
\pgfpathcurveto{\pgfqpoint{4.373807in}{2.417667in}}{\pgfqpoint{4.363208in}{2.413276in}}{\pgfqpoint{4.355395in}{2.405463in}}%
\pgfpathcurveto{\pgfqpoint{4.347581in}{2.397649in}}{\pgfqpoint{4.343191in}{2.387050in}}{\pgfqpoint{4.343191in}{2.376000in}}%
\pgfpathcurveto{\pgfqpoint{4.343191in}{2.364950in}}{\pgfqpoint{4.347581in}{2.354351in}}{\pgfqpoint{4.355395in}{2.346537in}}%
\pgfpathcurveto{\pgfqpoint{4.363208in}{2.338724in}}{\pgfqpoint{4.373807in}{2.334333in}}{\pgfqpoint{4.384857in}{2.334333in}}%
\pgfpathclose%
\pgfusepath{stroke,fill}%
\end{pgfscope}%
\begin{pgfscope}%
\pgfpathrectangle{\pgfqpoint{0.800000in}{0.528000in}}{\pgfqpoint{4.960000in}{3.696000in}}%
\pgfusepath{clip}%
\pgfsetbuttcap%
\pgfsetroundjoin%
\definecolor{currentfill}{rgb}{0.000000,0.000000,0.000000}%
\pgfsetfillcolor{currentfill}%
\pgfsetlinewidth{1.003750pt}%
\definecolor{currentstroke}{rgb}{0.000000,0.000000,0.000000}%
\pgfsetstrokecolor{currentstroke}%
\pgfsetdash{}{0pt}%
\pgfpathmoveto{\pgfqpoint{4.384857in}{2.334333in}}%
\pgfpathcurveto{\pgfqpoint{4.395908in}{2.334333in}}{\pgfqpoint{4.406507in}{2.338724in}}{\pgfqpoint{4.414320in}{2.346537in}}%
\pgfpathcurveto{\pgfqpoint{4.422134in}{2.354351in}}{\pgfqpoint{4.426524in}{2.364950in}}{\pgfqpoint{4.426524in}{2.376000in}}%
\pgfpathcurveto{\pgfqpoint{4.426524in}{2.387050in}}{\pgfqpoint{4.422134in}{2.397649in}}{\pgfqpoint{4.414320in}{2.405463in}}%
\pgfpathcurveto{\pgfqpoint{4.406507in}{2.413276in}}{\pgfqpoint{4.395908in}{2.417667in}}{\pgfqpoint{4.384857in}{2.417667in}}%
\pgfpathcurveto{\pgfqpoint{4.373807in}{2.417667in}}{\pgfqpoint{4.363208in}{2.413276in}}{\pgfqpoint{4.355395in}{2.405463in}}%
\pgfpathcurveto{\pgfqpoint{4.347581in}{2.397649in}}{\pgfqpoint{4.343191in}{2.387050in}}{\pgfqpoint{4.343191in}{2.376000in}}%
\pgfpathcurveto{\pgfqpoint{4.343191in}{2.364950in}}{\pgfqpoint{4.347581in}{2.354351in}}{\pgfqpoint{4.355395in}{2.346537in}}%
\pgfpathcurveto{\pgfqpoint{4.363208in}{2.338724in}}{\pgfqpoint{4.373807in}{2.334333in}}{\pgfqpoint{4.384857in}{2.334333in}}%
\pgfpathclose%
\pgfusepath{stroke,fill}%
\end{pgfscope}%
\begin{pgfscope}%
\pgfpathrectangle{\pgfqpoint{0.800000in}{0.528000in}}{\pgfqpoint{4.960000in}{3.696000in}}%
\pgfusepath{clip}%
\pgfsetbuttcap%
\pgfsetroundjoin%
\definecolor{currentfill}{rgb}{0.000000,0.000000,0.000000}%
\pgfsetfillcolor{currentfill}%
\pgfsetlinewidth{1.003750pt}%
\definecolor{currentstroke}{rgb}{0.000000,0.000000,0.000000}%
\pgfsetstrokecolor{currentstroke}%
\pgfsetdash{}{0pt}%
\pgfpathmoveto{\pgfqpoint{4.384857in}{2.334333in}}%
\pgfpathcurveto{\pgfqpoint{4.395908in}{2.334333in}}{\pgfqpoint{4.406507in}{2.338724in}}{\pgfqpoint{4.414320in}{2.346537in}}%
\pgfpathcurveto{\pgfqpoint{4.422134in}{2.354351in}}{\pgfqpoint{4.426524in}{2.364950in}}{\pgfqpoint{4.426524in}{2.376000in}}%
\pgfpathcurveto{\pgfqpoint{4.426524in}{2.387050in}}{\pgfqpoint{4.422134in}{2.397649in}}{\pgfqpoint{4.414320in}{2.405463in}}%
\pgfpathcurveto{\pgfqpoint{4.406507in}{2.413276in}}{\pgfqpoint{4.395908in}{2.417667in}}{\pgfqpoint{4.384857in}{2.417667in}}%
\pgfpathcurveto{\pgfqpoint{4.373807in}{2.417667in}}{\pgfqpoint{4.363208in}{2.413276in}}{\pgfqpoint{4.355395in}{2.405463in}}%
\pgfpathcurveto{\pgfqpoint{4.347581in}{2.397649in}}{\pgfqpoint{4.343191in}{2.387050in}}{\pgfqpoint{4.343191in}{2.376000in}}%
\pgfpathcurveto{\pgfqpoint{4.343191in}{2.364950in}}{\pgfqpoint{4.347581in}{2.354351in}}{\pgfqpoint{4.355395in}{2.346537in}}%
\pgfpathcurveto{\pgfqpoint{4.363208in}{2.338724in}}{\pgfqpoint{4.373807in}{2.334333in}}{\pgfqpoint{4.384857in}{2.334333in}}%
\pgfpathclose%
\pgfusepath{stroke,fill}%
\end{pgfscope}%
\begin{pgfscope}%
\pgfpathrectangle{\pgfqpoint{0.800000in}{0.528000in}}{\pgfqpoint{4.960000in}{3.696000in}}%
\pgfusepath{clip}%
\pgfsetbuttcap%
\pgfsetroundjoin%
\definecolor{currentfill}{rgb}{0.000000,0.000000,0.000000}%
\pgfsetfillcolor{currentfill}%
\pgfsetlinewidth{1.003750pt}%
\definecolor{currentstroke}{rgb}{0.000000,0.000000,0.000000}%
\pgfsetstrokecolor{currentstroke}%
\pgfsetdash{}{0pt}%
\pgfpathmoveto{\pgfqpoint{4.384857in}{2.334333in}}%
\pgfpathcurveto{\pgfqpoint{4.395908in}{2.334333in}}{\pgfqpoint{4.406507in}{2.338724in}}{\pgfqpoint{4.414320in}{2.346537in}}%
\pgfpathcurveto{\pgfqpoint{4.422134in}{2.354351in}}{\pgfqpoint{4.426524in}{2.364950in}}{\pgfqpoint{4.426524in}{2.376000in}}%
\pgfpathcurveto{\pgfqpoint{4.426524in}{2.387050in}}{\pgfqpoint{4.422134in}{2.397649in}}{\pgfqpoint{4.414320in}{2.405463in}}%
\pgfpathcurveto{\pgfqpoint{4.406507in}{2.413276in}}{\pgfqpoint{4.395908in}{2.417667in}}{\pgfqpoint{4.384857in}{2.417667in}}%
\pgfpathcurveto{\pgfqpoint{4.373807in}{2.417667in}}{\pgfqpoint{4.363208in}{2.413276in}}{\pgfqpoint{4.355395in}{2.405463in}}%
\pgfpathcurveto{\pgfqpoint{4.347581in}{2.397649in}}{\pgfqpoint{4.343191in}{2.387050in}}{\pgfqpoint{4.343191in}{2.376000in}}%
\pgfpathcurveto{\pgfqpoint{4.343191in}{2.364950in}}{\pgfqpoint{4.347581in}{2.354351in}}{\pgfqpoint{4.355395in}{2.346537in}}%
\pgfpathcurveto{\pgfqpoint{4.363208in}{2.338724in}}{\pgfqpoint{4.373807in}{2.334333in}}{\pgfqpoint{4.384857in}{2.334333in}}%
\pgfpathclose%
\pgfusepath{stroke,fill}%
\end{pgfscope}%
\begin{pgfscope}%
\pgfpathrectangle{\pgfqpoint{0.800000in}{0.528000in}}{\pgfqpoint{4.960000in}{3.696000in}}%
\pgfusepath{clip}%
\pgfsetbuttcap%
\pgfsetroundjoin%
\definecolor{currentfill}{rgb}{0.000000,0.000000,0.000000}%
\pgfsetfillcolor{currentfill}%
\pgfsetlinewidth{1.003750pt}%
\definecolor{currentstroke}{rgb}{0.000000,0.000000,0.000000}%
\pgfsetstrokecolor{currentstroke}%
\pgfsetdash{}{0pt}%
\pgfpathmoveto{\pgfqpoint{4.384857in}{2.334333in}}%
\pgfpathcurveto{\pgfqpoint{4.395908in}{2.334333in}}{\pgfqpoint{4.406507in}{2.338724in}}{\pgfqpoint{4.414320in}{2.346537in}}%
\pgfpathcurveto{\pgfqpoint{4.422134in}{2.354351in}}{\pgfqpoint{4.426524in}{2.364950in}}{\pgfqpoint{4.426524in}{2.376000in}}%
\pgfpathcurveto{\pgfqpoint{4.426524in}{2.387050in}}{\pgfqpoint{4.422134in}{2.397649in}}{\pgfqpoint{4.414320in}{2.405463in}}%
\pgfpathcurveto{\pgfqpoint{4.406507in}{2.413276in}}{\pgfqpoint{4.395908in}{2.417667in}}{\pgfqpoint{4.384857in}{2.417667in}}%
\pgfpathcurveto{\pgfqpoint{4.373807in}{2.417667in}}{\pgfqpoint{4.363208in}{2.413276in}}{\pgfqpoint{4.355395in}{2.405463in}}%
\pgfpathcurveto{\pgfqpoint{4.347581in}{2.397649in}}{\pgfqpoint{4.343191in}{2.387050in}}{\pgfqpoint{4.343191in}{2.376000in}}%
\pgfpathcurveto{\pgfqpoint{4.343191in}{2.364950in}}{\pgfqpoint{4.347581in}{2.354351in}}{\pgfqpoint{4.355395in}{2.346537in}}%
\pgfpathcurveto{\pgfqpoint{4.363208in}{2.338724in}}{\pgfqpoint{4.373807in}{2.334333in}}{\pgfqpoint{4.384857in}{2.334333in}}%
\pgfpathclose%
\pgfusepath{stroke,fill}%
\end{pgfscope}%
\begin{pgfscope}%
\pgfpathrectangle{\pgfqpoint{0.800000in}{0.528000in}}{\pgfqpoint{4.960000in}{3.696000in}}%
\pgfusepath{clip}%
\pgfsetbuttcap%
\pgfsetroundjoin%
\definecolor{currentfill}{rgb}{0.000000,0.000000,0.000000}%
\pgfsetfillcolor{currentfill}%
\pgfsetlinewidth{1.003750pt}%
\definecolor{currentstroke}{rgb}{0.000000,0.000000,0.000000}%
\pgfsetstrokecolor{currentstroke}%
\pgfsetdash{}{0pt}%
\pgfpathmoveto{\pgfqpoint{5.504545in}{2.334333in}}%
\pgfpathcurveto{\pgfqpoint{5.515596in}{2.334333in}}{\pgfqpoint{5.526195in}{2.338724in}}{\pgfqpoint{5.534008in}{2.346537in}}%
\pgfpathcurveto{\pgfqpoint{5.541822in}{2.354351in}}{\pgfqpoint{5.546212in}{2.364950in}}{\pgfqpoint{5.546212in}{2.376000in}}%
\pgfpathcurveto{\pgfqpoint{5.546212in}{2.387050in}}{\pgfqpoint{5.541822in}{2.397649in}}{\pgfqpoint{5.534008in}{2.405463in}}%
\pgfpathcurveto{\pgfqpoint{5.526195in}{2.413276in}}{\pgfqpoint{5.515596in}{2.417667in}}{\pgfqpoint{5.504545in}{2.417667in}}%
\pgfpathcurveto{\pgfqpoint{5.493495in}{2.417667in}}{\pgfqpoint{5.482896in}{2.413276in}}{\pgfqpoint{5.475083in}{2.405463in}}%
\pgfpathcurveto{\pgfqpoint{5.467269in}{2.397649in}}{\pgfqpoint{5.462879in}{2.387050in}}{\pgfqpoint{5.462879in}{2.376000in}}%
\pgfpathcurveto{\pgfqpoint{5.462879in}{2.364950in}}{\pgfqpoint{5.467269in}{2.354351in}}{\pgfqpoint{5.475083in}{2.346537in}}%
\pgfpathcurveto{\pgfqpoint{5.482896in}{2.338724in}}{\pgfqpoint{5.493495in}{2.334333in}}{\pgfqpoint{5.504545in}{2.334333in}}%
\pgfpathclose%
\pgfusepath{stroke,fill}%
\end{pgfscope}%
\begin{pgfscope}%
\pgfpathrectangle{\pgfqpoint{0.800000in}{0.528000in}}{\pgfqpoint{4.960000in}{3.696000in}}%
\pgfusepath{clip}%
\pgfsetbuttcap%
\pgfsetroundjoin%
\definecolor{currentfill}{rgb}{0.000000,0.000000,0.000000}%
\pgfsetfillcolor{currentfill}%
\pgfsetlinewidth{1.003750pt}%
\definecolor{currentstroke}{rgb}{0.000000,0.000000,0.000000}%
\pgfsetstrokecolor{currentstroke}%
\pgfsetdash{}{0pt}%
\pgfpathmoveto{\pgfqpoint{5.504545in}{2.334333in}}%
\pgfpathcurveto{\pgfqpoint{5.515596in}{2.334333in}}{\pgfqpoint{5.526195in}{2.338724in}}{\pgfqpoint{5.534008in}{2.346537in}}%
\pgfpathcurveto{\pgfqpoint{5.541822in}{2.354351in}}{\pgfqpoint{5.546212in}{2.364950in}}{\pgfqpoint{5.546212in}{2.376000in}}%
\pgfpathcurveto{\pgfqpoint{5.546212in}{2.387050in}}{\pgfqpoint{5.541822in}{2.397649in}}{\pgfqpoint{5.534008in}{2.405463in}}%
\pgfpathcurveto{\pgfqpoint{5.526195in}{2.413276in}}{\pgfqpoint{5.515596in}{2.417667in}}{\pgfqpoint{5.504545in}{2.417667in}}%
\pgfpathcurveto{\pgfqpoint{5.493495in}{2.417667in}}{\pgfqpoint{5.482896in}{2.413276in}}{\pgfqpoint{5.475083in}{2.405463in}}%
\pgfpathcurveto{\pgfqpoint{5.467269in}{2.397649in}}{\pgfqpoint{5.462879in}{2.387050in}}{\pgfqpoint{5.462879in}{2.376000in}}%
\pgfpathcurveto{\pgfqpoint{5.462879in}{2.364950in}}{\pgfqpoint{5.467269in}{2.354351in}}{\pgfqpoint{5.475083in}{2.346537in}}%
\pgfpathcurveto{\pgfqpoint{5.482896in}{2.338724in}}{\pgfqpoint{5.493495in}{2.334333in}}{\pgfqpoint{5.504545in}{2.334333in}}%
\pgfpathclose%
\pgfusepath{stroke,fill}%
\end{pgfscope}%
\begin{pgfscope}%
\pgfpathrectangle{\pgfqpoint{0.800000in}{0.528000in}}{\pgfqpoint{4.960000in}{3.696000in}}%
\pgfusepath{clip}%
\pgfsetbuttcap%
\pgfsetroundjoin%
\definecolor{currentfill}{rgb}{0.000000,0.000000,0.000000}%
\pgfsetfillcolor{currentfill}%
\pgfsetlinewidth{1.003750pt}%
\definecolor{currentstroke}{rgb}{0.000000,0.000000,0.000000}%
\pgfsetstrokecolor{currentstroke}%
\pgfsetdash{}{0pt}%
\pgfpathmoveto{\pgfqpoint{5.504545in}{2.334333in}}%
\pgfpathcurveto{\pgfqpoint{5.515596in}{2.334333in}}{\pgfqpoint{5.526195in}{2.338724in}}{\pgfqpoint{5.534008in}{2.346537in}}%
\pgfpathcurveto{\pgfqpoint{5.541822in}{2.354351in}}{\pgfqpoint{5.546212in}{2.364950in}}{\pgfqpoint{5.546212in}{2.376000in}}%
\pgfpathcurveto{\pgfqpoint{5.546212in}{2.387050in}}{\pgfqpoint{5.541822in}{2.397649in}}{\pgfqpoint{5.534008in}{2.405463in}}%
\pgfpathcurveto{\pgfqpoint{5.526195in}{2.413276in}}{\pgfqpoint{5.515596in}{2.417667in}}{\pgfqpoint{5.504545in}{2.417667in}}%
\pgfpathcurveto{\pgfqpoint{5.493495in}{2.417667in}}{\pgfqpoint{5.482896in}{2.413276in}}{\pgfqpoint{5.475083in}{2.405463in}}%
\pgfpathcurveto{\pgfqpoint{5.467269in}{2.397649in}}{\pgfqpoint{5.462879in}{2.387050in}}{\pgfqpoint{5.462879in}{2.376000in}}%
\pgfpathcurveto{\pgfqpoint{5.462879in}{2.364950in}}{\pgfqpoint{5.467269in}{2.354351in}}{\pgfqpoint{5.475083in}{2.346537in}}%
\pgfpathcurveto{\pgfqpoint{5.482896in}{2.338724in}}{\pgfqpoint{5.493495in}{2.334333in}}{\pgfqpoint{5.504545in}{2.334333in}}%
\pgfpathclose%
\pgfusepath{stroke,fill}%
\end{pgfscope}%
\begin{pgfscope}%
\pgfpathrectangle{\pgfqpoint{0.800000in}{0.528000in}}{\pgfqpoint{4.960000in}{3.696000in}}%
\pgfusepath{clip}%
\pgfsetbuttcap%
\pgfsetroundjoin%
\definecolor{currentfill}{rgb}{0.000000,0.000000,0.000000}%
\pgfsetfillcolor{currentfill}%
\pgfsetlinewidth{1.003750pt}%
\definecolor{currentstroke}{rgb}{0.000000,0.000000,0.000000}%
\pgfsetstrokecolor{currentstroke}%
\pgfsetdash{}{0pt}%
\pgfpathmoveto{\pgfqpoint{5.504545in}{2.334333in}}%
\pgfpathcurveto{\pgfqpoint{5.515596in}{2.334333in}}{\pgfqpoint{5.526195in}{2.338724in}}{\pgfqpoint{5.534008in}{2.346537in}}%
\pgfpathcurveto{\pgfqpoint{5.541822in}{2.354351in}}{\pgfqpoint{5.546212in}{2.364950in}}{\pgfqpoint{5.546212in}{2.376000in}}%
\pgfpathcurveto{\pgfqpoint{5.546212in}{2.387050in}}{\pgfqpoint{5.541822in}{2.397649in}}{\pgfqpoint{5.534008in}{2.405463in}}%
\pgfpathcurveto{\pgfqpoint{5.526195in}{2.413276in}}{\pgfqpoint{5.515596in}{2.417667in}}{\pgfqpoint{5.504545in}{2.417667in}}%
\pgfpathcurveto{\pgfqpoint{5.493495in}{2.417667in}}{\pgfqpoint{5.482896in}{2.413276in}}{\pgfqpoint{5.475083in}{2.405463in}}%
\pgfpathcurveto{\pgfqpoint{5.467269in}{2.397649in}}{\pgfqpoint{5.462879in}{2.387050in}}{\pgfqpoint{5.462879in}{2.376000in}}%
\pgfpathcurveto{\pgfqpoint{5.462879in}{2.364950in}}{\pgfqpoint{5.467269in}{2.354351in}}{\pgfqpoint{5.475083in}{2.346537in}}%
\pgfpathcurveto{\pgfqpoint{5.482896in}{2.338724in}}{\pgfqpoint{5.493495in}{2.334333in}}{\pgfqpoint{5.504545in}{2.334333in}}%
\pgfpathclose%
\pgfusepath{stroke,fill}%
\end{pgfscope}%
\begin{pgfscope}%
\pgfpathrectangle{\pgfqpoint{0.800000in}{0.528000in}}{\pgfqpoint{4.960000in}{3.696000in}}%
\pgfusepath{clip}%
\pgfsetbuttcap%
\pgfsetroundjoin%
\definecolor{currentfill}{rgb}{0.000000,0.000000,0.000000}%
\pgfsetfillcolor{currentfill}%
\pgfsetlinewidth{1.003750pt}%
\definecolor{currentstroke}{rgb}{0.000000,0.000000,0.000000}%
\pgfsetstrokecolor{currentstroke}%
\pgfsetdash{}{0pt}%
\pgfpathmoveto{\pgfqpoint{5.504545in}{2.334333in}}%
\pgfpathcurveto{\pgfqpoint{5.515596in}{2.334333in}}{\pgfqpoint{5.526195in}{2.338724in}}{\pgfqpoint{5.534008in}{2.346537in}}%
\pgfpathcurveto{\pgfqpoint{5.541822in}{2.354351in}}{\pgfqpoint{5.546212in}{2.364950in}}{\pgfqpoint{5.546212in}{2.376000in}}%
\pgfpathcurveto{\pgfqpoint{5.546212in}{2.387050in}}{\pgfqpoint{5.541822in}{2.397649in}}{\pgfqpoint{5.534008in}{2.405463in}}%
\pgfpathcurveto{\pgfqpoint{5.526195in}{2.413276in}}{\pgfqpoint{5.515596in}{2.417667in}}{\pgfqpoint{5.504545in}{2.417667in}}%
\pgfpathcurveto{\pgfqpoint{5.493495in}{2.417667in}}{\pgfqpoint{5.482896in}{2.413276in}}{\pgfqpoint{5.475083in}{2.405463in}}%
\pgfpathcurveto{\pgfqpoint{5.467269in}{2.397649in}}{\pgfqpoint{5.462879in}{2.387050in}}{\pgfqpoint{5.462879in}{2.376000in}}%
\pgfpathcurveto{\pgfqpoint{5.462879in}{2.364950in}}{\pgfqpoint{5.467269in}{2.354351in}}{\pgfqpoint{5.475083in}{2.346537in}}%
\pgfpathcurveto{\pgfqpoint{5.482896in}{2.338724in}}{\pgfqpoint{5.493495in}{2.334333in}}{\pgfqpoint{5.504545in}{2.334333in}}%
\pgfpathclose%
\pgfusepath{stroke,fill}%
\end{pgfscope}%
\begin{pgfscope}%
\pgfpathrectangle{\pgfqpoint{0.800000in}{0.528000in}}{\pgfqpoint{4.960000in}{3.696000in}}%
\pgfusepath{clip}%
\pgfsetbuttcap%
\pgfsetroundjoin%
\definecolor{currentfill}{rgb}{0.000000,0.000000,0.000000}%
\pgfsetfillcolor{currentfill}%
\pgfsetlinewidth{1.003750pt}%
\definecolor{currentstroke}{rgb}{0.000000,0.000000,0.000000}%
\pgfsetstrokecolor{currentstroke}%
\pgfsetdash{}{0pt}%
\pgfpathmoveto{\pgfqpoint{5.504545in}{2.334333in}}%
\pgfpathcurveto{\pgfqpoint{5.515596in}{2.334333in}}{\pgfqpoint{5.526195in}{2.338724in}}{\pgfqpoint{5.534008in}{2.346537in}}%
\pgfpathcurveto{\pgfqpoint{5.541822in}{2.354351in}}{\pgfqpoint{5.546212in}{2.364950in}}{\pgfqpoint{5.546212in}{2.376000in}}%
\pgfpathcurveto{\pgfqpoint{5.546212in}{2.387050in}}{\pgfqpoint{5.541822in}{2.397649in}}{\pgfqpoint{5.534008in}{2.405463in}}%
\pgfpathcurveto{\pgfqpoint{5.526195in}{2.413276in}}{\pgfqpoint{5.515596in}{2.417667in}}{\pgfqpoint{5.504545in}{2.417667in}}%
\pgfpathcurveto{\pgfqpoint{5.493495in}{2.417667in}}{\pgfqpoint{5.482896in}{2.413276in}}{\pgfqpoint{5.475083in}{2.405463in}}%
\pgfpathcurveto{\pgfqpoint{5.467269in}{2.397649in}}{\pgfqpoint{5.462879in}{2.387050in}}{\pgfqpoint{5.462879in}{2.376000in}}%
\pgfpathcurveto{\pgfqpoint{5.462879in}{2.364950in}}{\pgfqpoint{5.467269in}{2.354351in}}{\pgfqpoint{5.475083in}{2.346537in}}%
\pgfpathcurveto{\pgfqpoint{5.482896in}{2.338724in}}{\pgfqpoint{5.493495in}{2.334333in}}{\pgfqpoint{5.504545in}{2.334333in}}%
\pgfpathclose%
\pgfusepath{stroke,fill}%
\end{pgfscope}%
\begin{pgfscope}%
\pgfpathrectangle{\pgfqpoint{0.800000in}{0.528000in}}{\pgfqpoint{4.960000in}{3.696000in}}%
\pgfusepath{clip}%
\pgfsetbuttcap%
\pgfsetroundjoin%
\definecolor{currentfill}{rgb}{0.000000,0.000000,0.000000}%
\pgfsetfillcolor{currentfill}%
\pgfsetlinewidth{1.003750pt}%
\definecolor{currentstroke}{rgb}{0.000000,0.000000,0.000000}%
\pgfsetstrokecolor{currentstroke}%
\pgfsetdash{}{0pt}%
\pgfpathmoveto{\pgfqpoint{5.504545in}{2.334333in}}%
\pgfpathcurveto{\pgfqpoint{5.515596in}{2.334333in}}{\pgfqpoint{5.526195in}{2.338724in}}{\pgfqpoint{5.534008in}{2.346537in}}%
\pgfpathcurveto{\pgfqpoint{5.541822in}{2.354351in}}{\pgfqpoint{5.546212in}{2.364950in}}{\pgfqpoint{5.546212in}{2.376000in}}%
\pgfpathcurveto{\pgfqpoint{5.546212in}{2.387050in}}{\pgfqpoint{5.541822in}{2.397649in}}{\pgfqpoint{5.534008in}{2.405463in}}%
\pgfpathcurveto{\pgfqpoint{5.526195in}{2.413276in}}{\pgfqpoint{5.515596in}{2.417667in}}{\pgfqpoint{5.504545in}{2.417667in}}%
\pgfpathcurveto{\pgfqpoint{5.493495in}{2.417667in}}{\pgfqpoint{5.482896in}{2.413276in}}{\pgfqpoint{5.475083in}{2.405463in}}%
\pgfpathcurveto{\pgfqpoint{5.467269in}{2.397649in}}{\pgfqpoint{5.462879in}{2.387050in}}{\pgfqpoint{5.462879in}{2.376000in}}%
\pgfpathcurveto{\pgfqpoint{5.462879in}{2.364950in}}{\pgfqpoint{5.467269in}{2.354351in}}{\pgfqpoint{5.475083in}{2.346537in}}%
\pgfpathcurveto{\pgfqpoint{5.482896in}{2.338724in}}{\pgfqpoint{5.493495in}{2.334333in}}{\pgfqpoint{5.504545in}{2.334333in}}%
\pgfpathclose%
\pgfusepath{stroke,fill}%
\end{pgfscope}%
\begin{pgfscope}%
\pgfpathrectangle{\pgfqpoint{0.800000in}{0.528000in}}{\pgfqpoint{4.960000in}{3.696000in}}%
\pgfusepath{clip}%
\pgfsetbuttcap%
\pgfsetroundjoin%
\definecolor{currentfill}{rgb}{0.000000,0.000000,0.000000}%
\pgfsetfillcolor{currentfill}%
\pgfsetlinewidth{1.003750pt}%
\definecolor{currentstroke}{rgb}{0.000000,0.000000,0.000000}%
\pgfsetstrokecolor{currentstroke}%
\pgfsetdash{}{0pt}%
\pgfpathmoveto{\pgfqpoint{5.504545in}{2.334333in}}%
\pgfpathcurveto{\pgfqpoint{5.515596in}{2.334333in}}{\pgfqpoint{5.526195in}{2.338724in}}{\pgfqpoint{5.534008in}{2.346537in}}%
\pgfpathcurveto{\pgfqpoint{5.541822in}{2.354351in}}{\pgfqpoint{5.546212in}{2.364950in}}{\pgfqpoint{5.546212in}{2.376000in}}%
\pgfpathcurveto{\pgfqpoint{5.546212in}{2.387050in}}{\pgfqpoint{5.541822in}{2.397649in}}{\pgfqpoint{5.534008in}{2.405463in}}%
\pgfpathcurveto{\pgfqpoint{5.526195in}{2.413276in}}{\pgfqpoint{5.515596in}{2.417667in}}{\pgfqpoint{5.504545in}{2.417667in}}%
\pgfpathcurveto{\pgfqpoint{5.493495in}{2.417667in}}{\pgfqpoint{5.482896in}{2.413276in}}{\pgfqpoint{5.475083in}{2.405463in}}%
\pgfpathcurveto{\pgfqpoint{5.467269in}{2.397649in}}{\pgfqpoint{5.462879in}{2.387050in}}{\pgfqpoint{5.462879in}{2.376000in}}%
\pgfpathcurveto{\pgfqpoint{5.462879in}{2.364950in}}{\pgfqpoint{5.467269in}{2.354351in}}{\pgfqpoint{5.475083in}{2.346537in}}%
\pgfpathcurveto{\pgfqpoint{5.482896in}{2.338724in}}{\pgfqpoint{5.493495in}{2.334333in}}{\pgfqpoint{5.504545in}{2.334333in}}%
\pgfpathclose%
\pgfusepath{stroke,fill}%
\end{pgfscope}%
\begin{pgfscope}%
\pgfpathrectangle{\pgfqpoint{0.800000in}{0.528000in}}{\pgfqpoint{4.960000in}{3.696000in}}%
\pgfusepath{clip}%
\pgfsetbuttcap%
\pgfsetroundjoin%
\definecolor{currentfill}{rgb}{0.000000,0.000000,0.000000}%
\pgfsetfillcolor{currentfill}%
\pgfsetlinewidth{1.003750pt}%
\definecolor{currentstroke}{rgb}{0.000000,0.000000,0.000000}%
\pgfsetstrokecolor{currentstroke}%
\pgfsetdash{}{0pt}%
\pgfpathmoveto{\pgfqpoint{5.504545in}{2.334333in}}%
\pgfpathcurveto{\pgfqpoint{5.515596in}{2.334333in}}{\pgfqpoint{5.526195in}{2.338724in}}{\pgfqpoint{5.534008in}{2.346537in}}%
\pgfpathcurveto{\pgfqpoint{5.541822in}{2.354351in}}{\pgfqpoint{5.546212in}{2.364950in}}{\pgfqpoint{5.546212in}{2.376000in}}%
\pgfpathcurveto{\pgfqpoint{5.546212in}{2.387050in}}{\pgfqpoint{5.541822in}{2.397649in}}{\pgfqpoint{5.534008in}{2.405463in}}%
\pgfpathcurveto{\pgfqpoint{5.526195in}{2.413276in}}{\pgfqpoint{5.515596in}{2.417667in}}{\pgfqpoint{5.504545in}{2.417667in}}%
\pgfpathcurveto{\pgfqpoint{5.493495in}{2.417667in}}{\pgfqpoint{5.482896in}{2.413276in}}{\pgfqpoint{5.475083in}{2.405463in}}%
\pgfpathcurveto{\pgfqpoint{5.467269in}{2.397649in}}{\pgfqpoint{5.462879in}{2.387050in}}{\pgfqpoint{5.462879in}{2.376000in}}%
\pgfpathcurveto{\pgfqpoint{5.462879in}{2.364950in}}{\pgfqpoint{5.467269in}{2.354351in}}{\pgfqpoint{5.475083in}{2.346537in}}%
\pgfpathcurveto{\pgfqpoint{5.482896in}{2.338724in}}{\pgfqpoint{5.493495in}{2.334333in}}{\pgfqpoint{5.504545in}{2.334333in}}%
\pgfpathclose%
\pgfusepath{stroke,fill}%
\end{pgfscope}%
\begin{pgfscope}%
\pgfpathrectangle{\pgfqpoint{0.800000in}{0.528000in}}{\pgfqpoint{4.960000in}{3.696000in}}%
\pgfusepath{clip}%
\pgfsetbuttcap%
\pgfsetroundjoin%
\definecolor{currentfill}{rgb}{0.000000,0.000000,0.000000}%
\pgfsetfillcolor{currentfill}%
\pgfsetlinewidth{1.003750pt}%
\definecolor{currentstroke}{rgb}{0.000000,0.000000,0.000000}%
\pgfsetstrokecolor{currentstroke}%
\pgfsetdash{}{0pt}%
\pgfpathmoveto{\pgfqpoint{5.504545in}{2.334333in}}%
\pgfpathcurveto{\pgfqpoint{5.515596in}{2.334333in}}{\pgfqpoint{5.526195in}{2.338724in}}{\pgfqpoint{5.534008in}{2.346537in}}%
\pgfpathcurveto{\pgfqpoint{5.541822in}{2.354351in}}{\pgfqpoint{5.546212in}{2.364950in}}{\pgfqpoint{5.546212in}{2.376000in}}%
\pgfpathcurveto{\pgfqpoint{5.546212in}{2.387050in}}{\pgfqpoint{5.541822in}{2.397649in}}{\pgfqpoint{5.534008in}{2.405463in}}%
\pgfpathcurveto{\pgfqpoint{5.526195in}{2.413276in}}{\pgfqpoint{5.515596in}{2.417667in}}{\pgfqpoint{5.504545in}{2.417667in}}%
\pgfpathcurveto{\pgfqpoint{5.493495in}{2.417667in}}{\pgfqpoint{5.482896in}{2.413276in}}{\pgfqpoint{5.475083in}{2.405463in}}%
\pgfpathcurveto{\pgfqpoint{5.467269in}{2.397649in}}{\pgfqpoint{5.462879in}{2.387050in}}{\pgfqpoint{5.462879in}{2.376000in}}%
\pgfpathcurveto{\pgfqpoint{5.462879in}{2.364950in}}{\pgfqpoint{5.467269in}{2.354351in}}{\pgfqpoint{5.475083in}{2.346537in}}%
\pgfpathcurveto{\pgfqpoint{5.482896in}{2.338724in}}{\pgfqpoint{5.493495in}{2.334333in}}{\pgfqpoint{5.504545in}{2.334333in}}%
\pgfpathclose%
\pgfusepath{stroke,fill}%
\end{pgfscope}%
\begin{pgfscope}%
\pgfpathrectangle{\pgfqpoint{0.800000in}{0.528000in}}{\pgfqpoint{4.960000in}{3.696000in}}%
\pgfusepath{clip}%
\pgfsetbuttcap%
\pgfsetroundjoin%
\definecolor{currentfill}{rgb}{0.000000,0.000000,0.000000}%
\pgfsetfillcolor{currentfill}%
\pgfsetlinewidth{1.003750pt}%
\definecolor{currentstroke}{rgb}{0.000000,0.000000,0.000000}%
\pgfsetstrokecolor{currentstroke}%
\pgfsetdash{}{0pt}%
\pgfpathmoveto{\pgfqpoint{5.504545in}{2.334333in}}%
\pgfpathcurveto{\pgfqpoint{5.515596in}{2.334333in}}{\pgfqpoint{5.526195in}{2.338724in}}{\pgfqpoint{5.534008in}{2.346537in}}%
\pgfpathcurveto{\pgfqpoint{5.541822in}{2.354351in}}{\pgfqpoint{5.546212in}{2.364950in}}{\pgfqpoint{5.546212in}{2.376000in}}%
\pgfpathcurveto{\pgfqpoint{5.546212in}{2.387050in}}{\pgfqpoint{5.541822in}{2.397649in}}{\pgfqpoint{5.534008in}{2.405463in}}%
\pgfpathcurveto{\pgfqpoint{5.526195in}{2.413276in}}{\pgfqpoint{5.515596in}{2.417667in}}{\pgfqpoint{5.504545in}{2.417667in}}%
\pgfpathcurveto{\pgfqpoint{5.493495in}{2.417667in}}{\pgfqpoint{5.482896in}{2.413276in}}{\pgfqpoint{5.475083in}{2.405463in}}%
\pgfpathcurveto{\pgfqpoint{5.467269in}{2.397649in}}{\pgfqpoint{5.462879in}{2.387050in}}{\pgfqpoint{5.462879in}{2.376000in}}%
\pgfpathcurveto{\pgfqpoint{5.462879in}{2.364950in}}{\pgfqpoint{5.467269in}{2.354351in}}{\pgfqpoint{5.475083in}{2.346537in}}%
\pgfpathcurveto{\pgfqpoint{5.482896in}{2.338724in}}{\pgfqpoint{5.493495in}{2.334333in}}{\pgfqpoint{5.504545in}{2.334333in}}%
\pgfpathclose%
\pgfusepath{stroke,fill}%
\end{pgfscope}%
\begin{pgfscope}%
\pgfpathrectangle{\pgfqpoint{0.800000in}{0.528000in}}{\pgfqpoint{4.960000in}{3.696000in}}%
\pgfusepath{clip}%
\pgfsetbuttcap%
\pgfsetroundjoin%
\definecolor{currentfill}{rgb}{0.000000,0.000000,0.000000}%
\pgfsetfillcolor{currentfill}%
\pgfsetlinewidth{1.003750pt}%
\definecolor{currentstroke}{rgb}{0.000000,0.000000,0.000000}%
\pgfsetstrokecolor{currentstroke}%
\pgfsetdash{}{0pt}%
\pgfpathmoveto{\pgfqpoint{5.504545in}{2.334333in}}%
\pgfpathcurveto{\pgfqpoint{5.515596in}{2.334333in}}{\pgfqpoint{5.526195in}{2.338724in}}{\pgfqpoint{5.534008in}{2.346537in}}%
\pgfpathcurveto{\pgfqpoint{5.541822in}{2.354351in}}{\pgfqpoint{5.546212in}{2.364950in}}{\pgfqpoint{5.546212in}{2.376000in}}%
\pgfpathcurveto{\pgfqpoint{5.546212in}{2.387050in}}{\pgfqpoint{5.541822in}{2.397649in}}{\pgfqpoint{5.534008in}{2.405463in}}%
\pgfpathcurveto{\pgfqpoint{5.526195in}{2.413276in}}{\pgfqpoint{5.515596in}{2.417667in}}{\pgfqpoint{5.504545in}{2.417667in}}%
\pgfpathcurveto{\pgfqpoint{5.493495in}{2.417667in}}{\pgfqpoint{5.482896in}{2.413276in}}{\pgfqpoint{5.475083in}{2.405463in}}%
\pgfpathcurveto{\pgfqpoint{5.467269in}{2.397649in}}{\pgfqpoint{5.462879in}{2.387050in}}{\pgfqpoint{5.462879in}{2.376000in}}%
\pgfpathcurveto{\pgfqpoint{5.462879in}{2.364950in}}{\pgfqpoint{5.467269in}{2.354351in}}{\pgfqpoint{5.475083in}{2.346537in}}%
\pgfpathcurveto{\pgfqpoint{5.482896in}{2.338724in}}{\pgfqpoint{5.493495in}{2.334333in}}{\pgfqpoint{5.504545in}{2.334333in}}%
\pgfpathclose%
\pgfusepath{stroke,fill}%
\end{pgfscope}%
\begin{pgfscope}%
\pgfpathrectangle{\pgfqpoint{0.800000in}{0.528000in}}{\pgfqpoint{4.960000in}{3.696000in}}%
\pgfusepath{clip}%
\pgfsetbuttcap%
\pgfsetroundjoin%
\definecolor{currentfill}{rgb}{0.000000,0.000000,0.000000}%
\pgfsetfillcolor{currentfill}%
\pgfsetlinewidth{1.003750pt}%
\definecolor{currentstroke}{rgb}{0.000000,0.000000,0.000000}%
\pgfsetstrokecolor{currentstroke}%
\pgfsetdash{}{0pt}%
\pgfpathmoveto{\pgfqpoint{5.504545in}{2.334333in}}%
\pgfpathcurveto{\pgfqpoint{5.515596in}{2.334333in}}{\pgfqpoint{5.526195in}{2.338724in}}{\pgfqpoint{5.534008in}{2.346537in}}%
\pgfpathcurveto{\pgfqpoint{5.541822in}{2.354351in}}{\pgfqpoint{5.546212in}{2.364950in}}{\pgfqpoint{5.546212in}{2.376000in}}%
\pgfpathcurveto{\pgfqpoint{5.546212in}{2.387050in}}{\pgfqpoint{5.541822in}{2.397649in}}{\pgfqpoint{5.534008in}{2.405463in}}%
\pgfpathcurveto{\pgfqpoint{5.526195in}{2.413276in}}{\pgfqpoint{5.515596in}{2.417667in}}{\pgfqpoint{5.504545in}{2.417667in}}%
\pgfpathcurveto{\pgfqpoint{5.493495in}{2.417667in}}{\pgfqpoint{5.482896in}{2.413276in}}{\pgfqpoint{5.475083in}{2.405463in}}%
\pgfpathcurveto{\pgfqpoint{5.467269in}{2.397649in}}{\pgfqpoint{5.462879in}{2.387050in}}{\pgfqpoint{5.462879in}{2.376000in}}%
\pgfpathcurveto{\pgfqpoint{5.462879in}{2.364950in}}{\pgfqpoint{5.467269in}{2.354351in}}{\pgfqpoint{5.475083in}{2.346537in}}%
\pgfpathcurveto{\pgfqpoint{5.482896in}{2.338724in}}{\pgfqpoint{5.493495in}{2.334333in}}{\pgfqpoint{5.504545in}{2.334333in}}%
\pgfpathclose%
\pgfusepath{stroke,fill}%
\end{pgfscope}%
\begin{pgfscope}%
\pgfpathrectangle{\pgfqpoint{0.800000in}{0.528000in}}{\pgfqpoint{4.960000in}{3.696000in}}%
\pgfusepath{clip}%
\pgfsetbuttcap%
\pgfsetroundjoin%
\definecolor{currentfill}{rgb}{0.000000,0.000000,0.000000}%
\pgfsetfillcolor{currentfill}%
\pgfsetlinewidth{1.003750pt}%
\definecolor{currentstroke}{rgb}{0.000000,0.000000,0.000000}%
\pgfsetstrokecolor{currentstroke}%
\pgfsetdash{}{0pt}%
\pgfpathmoveto{\pgfqpoint{5.504545in}{2.334333in}}%
\pgfpathcurveto{\pgfqpoint{5.515596in}{2.334333in}}{\pgfqpoint{5.526195in}{2.338724in}}{\pgfqpoint{5.534008in}{2.346537in}}%
\pgfpathcurveto{\pgfqpoint{5.541822in}{2.354351in}}{\pgfqpoint{5.546212in}{2.364950in}}{\pgfqpoint{5.546212in}{2.376000in}}%
\pgfpathcurveto{\pgfqpoint{5.546212in}{2.387050in}}{\pgfqpoint{5.541822in}{2.397649in}}{\pgfqpoint{5.534008in}{2.405463in}}%
\pgfpathcurveto{\pgfqpoint{5.526195in}{2.413276in}}{\pgfqpoint{5.515596in}{2.417667in}}{\pgfqpoint{5.504545in}{2.417667in}}%
\pgfpathcurveto{\pgfqpoint{5.493495in}{2.417667in}}{\pgfqpoint{5.482896in}{2.413276in}}{\pgfqpoint{5.475083in}{2.405463in}}%
\pgfpathcurveto{\pgfqpoint{5.467269in}{2.397649in}}{\pgfqpoint{5.462879in}{2.387050in}}{\pgfqpoint{5.462879in}{2.376000in}}%
\pgfpathcurveto{\pgfqpoint{5.462879in}{2.364950in}}{\pgfqpoint{5.467269in}{2.354351in}}{\pgfqpoint{5.475083in}{2.346537in}}%
\pgfpathcurveto{\pgfqpoint{5.482896in}{2.338724in}}{\pgfqpoint{5.493495in}{2.334333in}}{\pgfqpoint{5.504545in}{2.334333in}}%
\pgfpathclose%
\pgfusepath{stroke,fill}%
\end{pgfscope}%
\begin{pgfscope}%
\pgfpathrectangle{\pgfqpoint{0.800000in}{0.528000in}}{\pgfqpoint{4.960000in}{3.696000in}}%
\pgfusepath{clip}%
\pgfsetbuttcap%
\pgfsetroundjoin%
\definecolor{currentfill}{rgb}{0.000000,0.000000,0.000000}%
\pgfsetfillcolor{currentfill}%
\pgfsetlinewidth{1.003750pt}%
\definecolor{currentstroke}{rgb}{0.000000,0.000000,0.000000}%
\pgfsetstrokecolor{currentstroke}%
\pgfsetdash{}{0pt}%
\pgfpathmoveto{\pgfqpoint{5.504545in}{2.334333in}}%
\pgfpathcurveto{\pgfqpoint{5.515596in}{2.334333in}}{\pgfqpoint{5.526195in}{2.338724in}}{\pgfqpoint{5.534008in}{2.346537in}}%
\pgfpathcurveto{\pgfqpoint{5.541822in}{2.354351in}}{\pgfqpoint{5.546212in}{2.364950in}}{\pgfqpoint{5.546212in}{2.376000in}}%
\pgfpathcurveto{\pgfqpoint{5.546212in}{2.387050in}}{\pgfqpoint{5.541822in}{2.397649in}}{\pgfqpoint{5.534008in}{2.405463in}}%
\pgfpathcurveto{\pgfqpoint{5.526195in}{2.413276in}}{\pgfqpoint{5.515596in}{2.417667in}}{\pgfqpoint{5.504545in}{2.417667in}}%
\pgfpathcurveto{\pgfqpoint{5.493495in}{2.417667in}}{\pgfqpoint{5.482896in}{2.413276in}}{\pgfqpoint{5.475083in}{2.405463in}}%
\pgfpathcurveto{\pgfqpoint{5.467269in}{2.397649in}}{\pgfqpoint{5.462879in}{2.387050in}}{\pgfqpoint{5.462879in}{2.376000in}}%
\pgfpathcurveto{\pgfqpoint{5.462879in}{2.364950in}}{\pgfqpoint{5.467269in}{2.354351in}}{\pgfqpoint{5.475083in}{2.346537in}}%
\pgfpathcurveto{\pgfqpoint{5.482896in}{2.338724in}}{\pgfqpoint{5.493495in}{2.334333in}}{\pgfqpoint{5.504545in}{2.334333in}}%
\pgfpathclose%
\pgfusepath{stroke,fill}%
\end{pgfscope}%
\begin{pgfscope}%
\pgfpathrectangle{\pgfqpoint{0.800000in}{0.528000in}}{\pgfqpoint{4.960000in}{3.696000in}}%
\pgfusepath{clip}%
\pgfsetbuttcap%
\pgfsetroundjoin%
\definecolor{currentfill}{rgb}{0.000000,0.000000,0.000000}%
\pgfsetfillcolor{currentfill}%
\pgfsetlinewidth{1.003750pt}%
\definecolor{currentstroke}{rgb}{0.000000,0.000000,0.000000}%
\pgfsetstrokecolor{currentstroke}%
\pgfsetdash{}{0pt}%
\pgfpathmoveto{\pgfqpoint{5.504545in}{2.334333in}}%
\pgfpathcurveto{\pgfqpoint{5.515596in}{2.334333in}}{\pgfqpoint{5.526195in}{2.338724in}}{\pgfqpoint{5.534008in}{2.346537in}}%
\pgfpathcurveto{\pgfqpoint{5.541822in}{2.354351in}}{\pgfqpoint{5.546212in}{2.364950in}}{\pgfqpoint{5.546212in}{2.376000in}}%
\pgfpathcurveto{\pgfqpoint{5.546212in}{2.387050in}}{\pgfqpoint{5.541822in}{2.397649in}}{\pgfqpoint{5.534008in}{2.405463in}}%
\pgfpathcurveto{\pgfqpoint{5.526195in}{2.413276in}}{\pgfqpoint{5.515596in}{2.417667in}}{\pgfqpoint{5.504545in}{2.417667in}}%
\pgfpathcurveto{\pgfqpoint{5.493495in}{2.417667in}}{\pgfqpoint{5.482896in}{2.413276in}}{\pgfqpoint{5.475083in}{2.405463in}}%
\pgfpathcurveto{\pgfqpoint{5.467269in}{2.397649in}}{\pgfqpoint{5.462879in}{2.387050in}}{\pgfqpoint{5.462879in}{2.376000in}}%
\pgfpathcurveto{\pgfqpoint{5.462879in}{2.364950in}}{\pgfqpoint{5.467269in}{2.354351in}}{\pgfqpoint{5.475083in}{2.346537in}}%
\pgfpathcurveto{\pgfqpoint{5.482896in}{2.338724in}}{\pgfqpoint{5.493495in}{2.334333in}}{\pgfqpoint{5.504545in}{2.334333in}}%
\pgfpathclose%
\pgfusepath{stroke,fill}%
\end{pgfscope}%
\begin{pgfscope}%
\pgfpathrectangle{\pgfqpoint{0.800000in}{0.528000in}}{\pgfqpoint{4.960000in}{3.696000in}}%
\pgfusepath{clip}%
\pgfsetbuttcap%
\pgfsetroundjoin%
\definecolor{currentfill}{rgb}{0.000000,0.000000,0.000000}%
\pgfsetfillcolor{currentfill}%
\pgfsetlinewidth{1.003750pt}%
\definecolor{currentstroke}{rgb}{0.000000,0.000000,0.000000}%
\pgfsetstrokecolor{currentstroke}%
\pgfsetdash{}{0pt}%
\pgfpathmoveto{\pgfqpoint{5.504545in}{2.334333in}}%
\pgfpathcurveto{\pgfqpoint{5.515596in}{2.334333in}}{\pgfqpoint{5.526195in}{2.338724in}}{\pgfqpoint{5.534008in}{2.346537in}}%
\pgfpathcurveto{\pgfqpoint{5.541822in}{2.354351in}}{\pgfqpoint{5.546212in}{2.364950in}}{\pgfqpoint{5.546212in}{2.376000in}}%
\pgfpathcurveto{\pgfqpoint{5.546212in}{2.387050in}}{\pgfqpoint{5.541822in}{2.397649in}}{\pgfqpoint{5.534008in}{2.405463in}}%
\pgfpathcurveto{\pgfqpoint{5.526195in}{2.413276in}}{\pgfqpoint{5.515596in}{2.417667in}}{\pgfqpoint{5.504545in}{2.417667in}}%
\pgfpathcurveto{\pgfqpoint{5.493495in}{2.417667in}}{\pgfqpoint{5.482896in}{2.413276in}}{\pgfqpoint{5.475083in}{2.405463in}}%
\pgfpathcurveto{\pgfqpoint{5.467269in}{2.397649in}}{\pgfqpoint{5.462879in}{2.387050in}}{\pgfqpoint{5.462879in}{2.376000in}}%
\pgfpathcurveto{\pgfqpoint{5.462879in}{2.364950in}}{\pgfqpoint{5.467269in}{2.354351in}}{\pgfqpoint{5.475083in}{2.346537in}}%
\pgfpathcurveto{\pgfqpoint{5.482896in}{2.338724in}}{\pgfqpoint{5.493495in}{2.334333in}}{\pgfqpoint{5.504545in}{2.334333in}}%
\pgfpathclose%
\pgfusepath{stroke,fill}%
\end{pgfscope}%
\begin{pgfscope}%
\pgfpathrectangle{\pgfqpoint{0.800000in}{0.528000in}}{\pgfqpoint{4.960000in}{3.696000in}}%
\pgfusepath{clip}%
\pgfsetbuttcap%
\pgfsetroundjoin%
\definecolor{currentfill}{rgb}{0.000000,0.000000,0.000000}%
\pgfsetfillcolor{currentfill}%
\pgfsetlinewidth{1.003750pt}%
\definecolor{currentstroke}{rgb}{0.000000,0.000000,0.000000}%
\pgfsetstrokecolor{currentstroke}%
\pgfsetdash{}{0pt}%
\pgfpathmoveto{\pgfqpoint{5.504545in}{2.334333in}}%
\pgfpathcurveto{\pgfqpoint{5.515596in}{2.334333in}}{\pgfqpoint{5.526195in}{2.338724in}}{\pgfqpoint{5.534008in}{2.346537in}}%
\pgfpathcurveto{\pgfqpoint{5.541822in}{2.354351in}}{\pgfqpoint{5.546212in}{2.364950in}}{\pgfqpoint{5.546212in}{2.376000in}}%
\pgfpathcurveto{\pgfqpoint{5.546212in}{2.387050in}}{\pgfqpoint{5.541822in}{2.397649in}}{\pgfqpoint{5.534008in}{2.405463in}}%
\pgfpathcurveto{\pgfqpoint{5.526195in}{2.413276in}}{\pgfqpoint{5.515596in}{2.417667in}}{\pgfqpoint{5.504545in}{2.417667in}}%
\pgfpathcurveto{\pgfqpoint{5.493495in}{2.417667in}}{\pgfqpoint{5.482896in}{2.413276in}}{\pgfqpoint{5.475083in}{2.405463in}}%
\pgfpathcurveto{\pgfqpoint{5.467269in}{2.397649in}}{\pgfqpoint{5.462879in}{2.387050in}}{\pgfqpoint{5.462879in}{2.376000in}}%
\pgfpathcurveto{\pgfqpoint{5.462879in}{2.364950in}}{\pgfqpoint{5.467269in}{2.354351in}}{\pgfqpoint{5.475083in}{2.346537in}}%
\pgfpathcurveto{\pgfqpoint{5.482896in}{2.338724in}}{\pgfqpoint{5.493495in}{2.334333in}}{\pgfqpoint{5.504545in}{2.334333in}}%
\pgfpathclose%
\pgfusepath{stroke,fill}%
\end{pgfscope}%
\begin{pgfscope}%
\pgfpathrectangle{\pgfqpoint{0.800000in}{0.528000in}}{\pgfqpoint{4.960000in}{3.696000in}}%
\pgfusepath{clip}%
\pgfsetbuttcap%
\pgfsetroundjoin%
\definecolor{currentfill}{rgb}{0.000000,0.000000,0.000000}%
\pgfsetfillcolor{currentfill}%
\pgfsetlinewidth{1.003750pt}%
\definecolor{currentstroke}{rgb}{0.000000,0.000000,0.000000}%
\pgfsetstrokecolor{currentstroke}%
\pgfsetdash{}{0pt}%
\pgfpathmoveto{\pgfqpoint{5.504545in}{2.334333in}}%
\pgfpathcurveto{\pgfqpoint{5.515596in}{2.334333in}}{\pgfqpoint{5.526195in}{2.338724in}}{\pgfqpoint{5.534008in}{2.346537in}}%
\pgfpathcurveto{\pgfqpoint{5.541822in}{2.354351in}}{\pgfqpoint{5.546212in}{2.364950in}}{\pgfqpoint{5.546212in}{2.376000in}}%
\pgfpathcurveto{\pgfqpoint{5.546212in}{2.387050in}}{\pgfqpoint{5.541822in}{2.397649in}}{\pgfqpoint{5.534008in}{2.405463in}}%
\pgfpathcurveto{\pgfqpoint{5.526195in}{2.413276in}}{\pgfqpoint{5.515596in}{2.417667in}}{\pgfqpoint{5.504545in}{2.417667in}}%
\pgfpathcurveto{\pgfqpoint{5.493495in}{2.417667in}}{\pgfqpoint{5.482896in}{2.413276in}}{\pgfqpoint{5.475083in}{2.405463in}}%
\pgfpathcurveto{\pgfqpoint{5.467269in}{2.397649in}}{\pgfqpoint{5.462879in}{2.387050in}}{\pgfqpoint{5.462879in}{2.376000in}}%
\pgfpathcurveto{\pgfqpoint{5.462879in}{2.364950in}}{\pgfqpoint{5.467269in}{2.354351in}}{\pgfqpoint{5.475083in}{2.346537in}}%
\pgfpathcurveto{\pgfqpoint{5.482896in}{2.338724in}}{\pgfqpoint{5.493495in}{2.334333in}}{\pgfqpoint{5.504545in}{2.334333in}}%
\pgfpathclose%
\pgfusepath{stroke,fill}%
\end{pgfscope}%
\begin{pgfscope}%
\pgfpathrectangle{\pgfqpoint{0.800000in}{0.528000in}}{\pgfqpoint{4.960000in}{3.696000in}}%
\pgfusepath{clip}%
\pgfsetbuttcap%
\pgfsetroundjoin%
\definecolor{currentfill}{rgb}{0.000000,0.000000,0.000000}%
\pgfsetfillcolor{currentfill}%
\pgfsetlinewidth{1.003750pt}%
\definecolor{currentstroke}{rgb}{0.000000,0.000000,0.000000}%
\pgfsetstrokecolor{currentstroke}%
\pgfsetdash{}{0pt}%
\pgfpathmoveto{\pgfqpoint{5.504545in}{2.334333in}}%
\pgfpathcurveto{\pgfqpoint{5.515596in}{2.334333in}}{\pgfqpoint{5.526195in}{2.338724in}}{\pgfqpoint{5.534008in}{2.346537in}}%
\pgfpathcurveto{\pgfqpoint{5.541822in}{2.354351in}}{\pgfqpoint{5.546212in}{2.364950in}}{\pgfqpoint{5.546212in}{2.376000in}}%
\pgfpathcurveto{\pgfqpoint{5.546212in}{2.387050in}}{\pgfqpoint{5.541822in}{2.397649in}}{\pgfqpoint{5.534008in}{2.405463in}}%
\pgfpathcurveto{\pgfqpoint{5.526195in}{2.413276in}}{\pgfqpoint{5.515596in}{2.417667in}}{\pgfqpoint{5.504545in}{2.417667in}}%
\pgfpathcurveto{\pgfqpoint{5.493495in}{2.417667in}}{\pgfqpoint{5.482896in}{2.413276in}}{\pgfqpoint{5.475083in}{2.405463in}}%
\pgfpathcurveto{\pgfqpoint{5.467269in}{2.397649in}}{\pgfqpoint{5.462879in}{2.387050in}}{\pgfqpoint{5.462879in}{2.376000in}}%
\pgfpathcurveto{\pgfqpoint{5.462879in}{2.364950in}}{\pgfqpoint{5.467269in}{2.354351in}}{\pgfqpoint{5.475083in}{2.346537in}}%
\pgfpathcurveto{\pgfqpoint{5.482896in}{2.338724in}}{\pgfqpoint{5.493495in}{2.334333in}}{\pgfqpoint{5.504545in}{2.334333in}}%
\pgfpathclose%
\pgfusepath{stroke,fill}%
\end{pgfscope}%
\begin{pgfscope}%
\pgfpathrectangle{\pgfqpoint{0.800000in}{0.528000in}}{\pgfqpoint{4.960000in}{3.696000in}}%
\pgfusepath{clip}%
\pgfsetbuttcap%
\pgfsetroundjoin%
\definecolor{currentfill}{rgb}{0.000000,0.000000,0.000000}%
\pgfsetfillcolor{currentfill}%
\pgfsetlinewidth{1.003750pt}%
\definecolor{currentstroke}{rgb}{0.000000,0.000000,0.000000}%
\pgfsetstrokecolor{currentstroke}%
\pgfsetdash{}{0pt}%
\pgfpathmoveto{\pgfqpoint{5.504545in}{2.334333in}}%
\pgfpathcurveto{\pgfqpoint{5.515596in}{2.334333in}}{\pgfqpoint{5.526195in}{2.338724in}}{\pgfqpoint{5.534008in}{2.346537in}}%
\pgfpathcurveto{\pgfqpoint{5.541822in}{2.354351in}}{\pgfqpoint{5.546212in}{2.364950in}}{\pgfqpoint{5.546212in}{2.376000in}}%
\pgfpathcurveto{\pgfqpoint{5.546212in}{2.387050in}}{\pgfqpoint{5.541822in}{2.397649in}}{\pgfqpoint{5.534008in}{2.405463in}}%
\pgfpathcurveto{\pgfqpoint{5.526195in}{2.413276in}}{\pgfqpoint{5.515596in}{2.417667in}}{\pgfqpoint{5.504545in}{2.417667in}}%
\pgfpathcurveto{\pgfqpoint{5.493495in}{2.417667in}}{\pgfqpoint{5.482896in}{2.413276in}}{\pgfqpoint{5.475083in}{2.405463in}}%
\pgfpathcurveto{\pgfqpoint{5.467269in}{2.397649in}}{\pgfqpoint{5.462879in}{2.387050in}}{\pgfqpoint{5.462879in}{2.376000in}}%
\pgfpathcurveto{\pgfqpoint{5.462879in}{2.364950in}}{\pgfqpoint{5.467269in}{2.354351in}}{\pgfqpoint{5.475083in}{2.346537in}}%
\pgfpathcurveto{\pgfqpoint{5.482896in}{2.338724in}}{\pgfqpoint{5.493495in}{2.334333in}}{\pgfqpoint{5.504545in}{2.334333in}}%
\pgfpathclose%
\pgfusepath{stroke,fill}%
\end{pgfscope}%
\begin{pgfscope}%
\pgfpathrectangle{\pgfqpoint{0.800000in}{0.528000in}}{\pgfqpoint{4.960000in}{3.696000in}}%
\pgfusepath{clip}%
\pgfsetbuttcap%
\pgfsetroundjoin%
\definecolor{currentfill}{rgb}{0.000000,0.000000,0.000000}%
\pgfsetfillcolor{currentfill}%
\pgfsetlinewidth{1.003750pt}%
\definecolor{currentstroke}{rgb}{0.000000,0.000000,0.000000}%
\pgfsetstrokecolor{currentstroke}%
\pgfsetdash{}{0pt}%
\pgfpathmoveto{\pgfqpoint{5.504545in}{2.334333in}}%
\pgfpathcurveto{\pgfqpoint{5.515596in}{2.334333in}}{\pgfqpoint{5.526195in}{2.338724in}}{\pgfqpoint{5.534008in}{2.346537in}}%
\pgfpathcurveto{\pgfqpoint{5.541822in}{2.354351in}}{\pgfqpoint{5.546212in}{2.364950in}}{\pgfqpoint{5.546212in}{2.376000in}}%
\pgfpathcurveto{\pgfqpoint{5.546212in}{2.387050in}}{\pgfqpoint{5.541822in}{2.397649in}}{\pgfqpoint{5.534008in}{2.405463in}}%
\pgfpathcurveto{\pgfqpoint{5.526195in}{2.413276in}}{\pgfqpoint{5.515596in}{2.417667in}}{\pgfqpoint{5.504545in}{2.417667in}}%
\pgfpathcurveto{\pgfqpoint{5.493495in}{2.417667in}}{\pgfqpoint{5.482896in}{2.413276in}}{\pgfqpoint{5.475083in}{2.405463in}}%
\pgfpathcurveto{\pgfqpoint{5.467269in}{2.397649in}}{\pgfqpoint{5.462879in}{2.387050in}}{\pgfqpoint{5.462879in}{2.376000in}}%
\pgfpathcurveto{\pgfqpoint{5.462879in}{2.364950in}}{\pgfqpoint{5.467269in}{2.354351in}}{\pgfqpoint{5.475083in}{2.346537in}}%
\pgfpathcurveto{\pgfqpoint{5.482896in}{2.338724in}}{\pgfqpoint{5.493495in}{2.334333in}}{\pgfqpoint{5.504545in}{2.334333in}}%
\pgfpathclose%
\pgfusepath{stroke,fill}%
\end{pgfscope}%
\begin{pgfscope}%
\pgfpathrectangle{\pgfqpoint{0.800000in}{0.528000in}}{\pgfqpoint{4.960000in}{3.696000in}}%
\pgfusepath{clip}%
\pgfsetbuttcap%
\pgfsetroundjoin%
\definecolor{currentfill}{rgb}{0.000000,0.000000,0.000000}%
\pgfsetfillcolor{currentfill}%
\pgfsetlinewidth{1.003750pt}%
\definecolor{currentstroke}{rgb}{0.000000,0.000000,0.000000}%
\pgfsetstrokecolor{currentstroke}%
\pgfsetdash{}{0pt}%
\pgfpathmoveto{\pgfqpoint{5.504545in}{2.334333in}}%
\pgfpathcurveto{\pgfqpoint{5.515596in}{2.334333in}}{\pgfqpoint{5.526195in}{2.338724in}}{\pgfqpoint{5.534008in}{2.346537in}}%
\pgfpathcurveto{\pgfqpoint{5.541822in}{2.354351in}}{\pgfqpoint{5.546212in}{2.364950in}}{\pgfqpoint{5.546212in}{2.376000in}}%
\pgfpathcurveto{\pgfqpoint{5.546212in}{2.387050in}}{\pgfqpoint{5.541822in}{2.397649in}}{\pgfqpoint{5.534008in}{2.405463in}}%
\pgfpathcurveto{\pgfqpoint{5.526195in}{2.413276in}}{\pgfqpoint{5.515596in}{2.417667in}}{\pgfqpoint{5.504545in}{2.417667in}}%
\pgfpathcurveto{\pgfqpoint{5.493495in}{2.417667in}}{\pgfqpoint{5.482896in}{2.413276in}}{\pgfqpoint{5.475083in}{2.405463in}}%
\pgfpathcurveto{\pgfqpoint{5.467269in}{2.397649in}}{\pgfqpoint{5.462879in}{2.387050in}}{\pgfqpoint{5.462879in}{2.376000in}}%
\pgfpathcurveto{\pgfqpoint{5.462879in}{2.364950in}}{\pgfqpoint{5.467269in}{2.354351in}}{\pgfqpoint{5.475083in}{2.346537in}}%
\pgfpathcurveto{\pgfqpoint{5.482896in}{2.338724in}}{\pgfqpoint{5.493495in}{2.334333in}}{\pgfqpoint{5.504545in}{2.334333in}}%
\pgfpathclose%
\pgfusepath{stroke,fill}%
\end{pgfscope}%
\begin{pgfscope}%
\pgfpathrectangle{\pgfqpoint{0.800000in}{0.528000in}}{\pgfqpoint{4.960000in}{3.696000in}}%
\pgfusepath{clip}%
\pgfsetbuttcap%
\pgfsetroundjoin%
\definecolor{currentfill}{rgb}{0.000000,0.000000,0.000000}%
\pgfsetfillcolor{currentfill}%
\pgfsetlinewidth{1.003750pt}%
\definecolor{currentstroke}{rgb}{0.000000,0.000000,0.000000}%
\pgfsetstrokecolor{currentstroke}%
\pgfsetdash{}{0pt}%
\pgfpathmoveto{\pgfqpoint{5.504545in}{2.334333in}}%
\pgfpathcurveto{\pgfqpoint{5.515596in}{2.334333in}}{\pgfqpoint{5.526195in}{2.338724in}}{\pgfqpoint{5.534008in}{2.346537in}}%
\pgfpathcurveto{\pgfqpoint{5.541822in}{2.354351in}}{\pgfqpoint{5.546212in}{2.364950in}}{\pgfqpoint{5.546212in}{2.376000in}}%
\pgfpathcurveto{\pgfqpoint{5.546212in}{2.387050in}}{\pgfqpoint{5.541822in}{2.397649in}}{\pgfqpoint{5.534008in}{2.405463in}}%
\pgfpathcurveto{\pgfqpoint{5.526195in}{2.413276in}}{\pgfqpoint{5.515596in}{2.417667in}}{\pgfqpoint{5.504545in}{2.417667in}}%
\pgfpathcurveto{\pgfqpoint{5.493495in}{2.417667in}}{\pgfqpoint{5.482896in}{2.413276in}}{\pgfqpoint{5.475083in}{2.405463in}}%
\pgfpathcurveto{\pgfqpoint{5.467269in}{2.397649in}}{\pgfqpoint{5.462879in}{2.387050in}}{\pgfqpoint{5.462879in}{2.376000in}}%
\pgfpathcurveto{\pgfqpoint{5.462879in}{2.364950in}}{\pgfqpoint{5.467269in}{2.354351in}}{\pgfqpoint{5.475083in}{2.346537in}}%
\pgfpathcurveto{\pgfqpoint{5.482896in}{2.338724in}}{\pgfqpoint{5.493495in}{2.334333in}}{\pgfqpoint{5.504545in}{2.334333in}}%
\pgfpathclose%
\pgfusepath{stroke,fill}%
\end{pgfscope}%
\begin{pgfscope}%
\pgfpathrectangle{\pgfqpoint{0.800000in}{0.528000in}}{\pgfqpoint{4.960000in}{3.696000in}}%
\pgfusepath{clip}%
\pgfsetbuttcap%
\pgfsetroundjoin%
\definecolor{currentfill}{rgb}{0.000000,0.000000,0.000000}%
\pgfsetfillcolor{currentfill}%
\pgfsetlinewidth{1.003750pt}%
\definecolor{currentstroke}{rgb}{0.000000,0.000000,0.000000}%
\pgfsetstrokecolor{currentstroke}%
\pgfsetdash{}{0pt}%
\pgfpathmoveto{\pgfqpoint{5.504545in}{2.334333in}}%
\pgfpathcurveto{\pgfqpoint{5.515596in}{2.334333in}}{\pgfqpoint{5.526195in}{2.338724in}}{\pgfqpoint{5.534008in}{2.346537in}}%
\pgfpathcurveto{\pgfqpoint{5.541822in}{2.354351in}}{\pgfqpoint{5.546212in}{2.364950in}}{\pgfqpoint{5.546212in}{2.376000in}}%
\pgfpathcurveto{\pgfqpoint{5.546212in}{2.387050in}}{\pgfqpoint{5.541822in}{2.397649in}}{\pgfqpoint{5.534008in}{2.405463in}}%
\pgfpathcurveto{\pgfqpoint{5.526195in}{2.413276in}}{\pgfqpoint{5.515596in}{2.417667in}}{\pgfqpoint{5.504545in}{2.417667in}}%
\pgfpathcurveto{\pgfqpoint{5.493495in}{2.417667in}}{\pgfqpoint{5.482896in}{2.413276in}}{\pgfqpoint{5.475083in}{2.405463in}}%
\pgfpathcurveto{\pgfqpoint{5.467269in}{2.397649in}}{\pgfqpoint{5.462879in}{2.387050in}}{\pgfqpoint{5.462879in}{2.376000in}}%
\pgfpathcurveto{\pgfqpoint{5.462879in}{2.364950in}}{\pgfqpoint{5.467269in}{2.354351in}}{\pgfqpoint{5.475083in}{2.346537in}}%
\pgfpathcurveto{\pgfqpoint{5.482896in}{2.338724in}}{\pgfqpoint{5.493495in}{2.334333in}}{\pgfqpoint{5.504545in}{2.334333in}}%
\pgfpathclose%
\pgfusepath{stroke,fill}%
\end{pgfscope}%
\begin{pgfscope}%
\pgfpathrectangle{\pgfqpoint{0.800000in}{0.528000in}}{\pgfqpoint{4.960000in}{3.696000in}}%
\pgfusepath{clip}%
\pgfsetbuttcap%
\pgfsetroundjoin%
\definecolor{currentfill}{rgb}{0.000000,0.000000,0.000000}%
\pgfsetfillcolor{currentfill}%
\pgfsetlinewidth{1.003750pt}%
\definecolor{currentstroke}{rgb}{0.000000,0.000000,0.000000}%
\pgfsetstrokecolor{currentstroke}%
\pgfsetdash{}{0pt}%
\pgfpathmoveto{\pgfqpoint{5.504545in}{2.334333in}}%
\pgfpathcurveto{\pgfqpoint{5.515596in}{2.334333in}}{\pgfqpoint{5.526195in}{2.338724in}}{\pgfqpoint{5.534008in}{2.346537in}}%
\pgfpathcurveto{\pgfqpoint{5.541822in}{2.354351in}}{\pgfqpoint{5.546212in}{2.364950in}}{\pgfqpoint{5.546212in}{2.376000in}}%
\pgfpathcurveto{\pgfqpoint{5.546212in}{2.387050in}}{\pgfqpoint{5.541822in}{2.397649in}}{\pgfqpoint{5.534008in}{2.405463in}}%
\pgfpathcurveto{\pgfqpoint{5.526195in}{2.413276in}}{\pgfqpoint{5.515596in}{2.417667in}}{\pgfqpoint{5.504545in}{2.417667in}}%
\pgfpathcurveto{\pgfqpoint{5.493495in}{2.417667in}}{\pgfqpoint{5.482896in}{2.413276in}}{\pgfqpoint{5.475083in}{2.405463in}}%
\pgfpathcurveto{\pgfqpoint{5.467269in}{2.397649in}}{\pgfqpoint{5.462879in}{2.387050in}}{\pgfqpoint{5.462879in}{2.376000in}}%
\pgfpathcurveto{\pgfqpoint{5.462879in}{2.364950in}}{\pgfqpoint{5.467269in}{2.354351in}}{\pgfqpoint{5.475083in}{2.346537in}}%
\pgfpathcurveto{\pgfqpoint{5.482896in}{2.338724in}}{\pgfqpoint{5.493495in}{2.334333in}}{\pgfqpoint{5.504545in}{2.334333in}}%
\pgfpathclose%
\pgfusepath{stroke,fill}%
\end{pgfscope}%
\begin{pgfscope}%
\pgfpathrectangle{\pgfqpoint{0.800000in}{0.528000in}}{\pgfqpoint{4.960000in}{3.696000in}}%
\pgfusepath{clip}%
\pgfsetbuttcap%
\pgfsetroundjoin%
\definecolor{currentfill}{rgb}{0.000000,0.000000,0.000000}%
\pgfsetfillcolor{currentfill}%
\pgfsetlinewidth{1.003750pt}%
\definecolor{currentstroke}{rgb}{0.000000,0.000000,0.000000}%
\pgfsetstrokecolor{currentstroke}%
\pgfsetdash{}{0pt}%
\pgfpathmoveto{\pgfqpoint{5.504545in}{2.334333in}}%
\pgfpathcurveto{\pgfqpoint{5.515596in}{2.334333in}}{\pgfqpoint{5.526195in}{2.338724in}}{\pgfqpoint{5.534008in}{2.346537in}}%
\pgfpathcurveto{\pgfqpoint{5.541822in}{2.354351in}}{\pgfqpoint{5.546212in}{2.364950in}}{\pgfqpoint{5.546212in}{2.376000in}}%
\pgfpathcurveto{\pgfqpoint{5.546212in}{2.387050in}}{\pgfqpoint{5.541822in}{2.397649in}}{\pgfqpoint{5.534008in}{2.405463in}}%
\pgfpathcurveto{\pgfqpoint{5.526195in}{2.413276in}}{\pgfqpoint{5.515596in}{2.417667in}}{\pgfqpoint{5.504545in}{2.417667in}}%
\pgfpathcurveto{\pgfqpoint{5.493495in}{2.417667in}}{\pgfqpoint{5.482896in}{2.413276in}}{\pgfqpoint{5.475083in}{2.405463in}}%
\pgfpathcurveto{\pgfqpoint{5.467269in}{2.397649in}}{\pgfqpoint{5.462879in}{2.387050in}}{\pgfqpoint{5.462879in}{2.376000in}}%
\pgfpathcurveto{\pgfqpoint{5.462879in}{2.364950in}}{\pgfqpoint{5.467269in}{2.354351in}}{\pgfqpoint{5.475083in}{2.346537in}}%
\pgfpathcurveto{\pgfqpoint{5.482896in}{2.338724in}}{\pgfqpoint{5.493495in}{2.334333in}}{\pgfqpoint{5.504545in}{2.334333in}}%
\pgfpathclose%
\pgfusepath{stroke,fill}%
\end{pgfscope}%
\begin{pgfscope}%
\pgfpathrectangle{\pgfqpoint{0.800000in}{0.528000in}}{\pgfqpoint{4.960000in}{3.696000in}}%
\pgfusepath{clip}%
\pgfsetbuttcap%
\pgfsetroundjoin%
\definecolor{currentfill}{rgb}{0.000000,0.000000,0.000000}%
\pgfsetfillcolor{currentfill}%
\pgfsetlinewidth{1.003750pt}%
\definecolor{currentstroke}{rgb}{0.000000,0.000000,0.000000}%
\pgfsetstrokecolor{currentstroke}%
\pgfsetdash{}{0pt}%
\pgfpathmoveto{\pgfqpoint{5.504545in}{2.334333in}}%
\pgfpathcurveto{\pgfqpoint{5.515596in}{2.334333in}}{\pgfqpoint{5.526195in}{2.338724in}}{\pgfqpoint{5.534008in}{2.346537in}}%
\pgfpathcurveto{\pgfqpoint{5.541822in}{2.354351in}}{\pgfqpoint{5.546212in}{2.364950in}}{\pgfqpoint{5.546212in}{2.376000in}}%
\pgfpathcurveto{\pgfqpoint{5.546212in}{2.387050in}}{\pgfqpoint{5.541822in}{2.397649in}}{\pgfqpoint{5.534008in}{2.405463in}}%
\pgfpathcurveto{\pgfqpoint{5.526195in}{2.413276in}}{\pgfqpoint{5.515596in}{2.417667in}}{\pgfqpoint{5.504545in}{2.417667in}}%
\pgfpathcurveto{\pgfqpoint{5.493495in}{2.417667in}}{\pgfqpoint{5.482896in}{2.413276in}}{\pgfqpoint{5.475083in}{2.405463in}}%
\pgfpathcurveto{\pgfqpoint{5.467269in}{2.397649in}}{\pgfqpoint{5.462879in}{2.387050in}}{\pgfqpoint{5.462879in}{2.376000in}}%
\pgfpathcurveto{\pgfqpoint{5.462879in}{2.364950in}}{\pgfqpoint{5.467269in}{2.354351in}}{\pgfqpoint{5.475083in}{2.346537in}}%
\pgfpathcurveto{\pgfqpoint{5.482896in}{2.338724in}}{\pgfqpoint{5.493495in}{2.334333in}}{\pgfqpoint{5.504545in}{2.334333in}}%
\pgfpathclose%
\pgfusepath{stroke,fill}%
\end{pgfscope}%
\begin{pgfscope}%
\pgfpathrectangle{\pgfqpoint{0.800000in}{0.528000in}}{\pgfqpoint{4.960000in}{3.696000in}}%
\pgfusepath{clip}%
\pgfsetbuttcap%
\pgfsetroundjoin%
\definecolor{currentfill}{rgb}{0.000000,0.000000,0.000000}%
\pgfsetfillcolor{currentfill}%
\pgfsetlinewidth{1.003750pt}%
\definecolor{currentstroke}{rgb}{0.000000,0.000000,0.000000}%
\pgfsetstrokecolor{currentstroke}%
\pgfsetdash{}{0pt}%
\pgfpathmoveto{\pgfqpoint{5.504545in}{2.334333in}}%
\pgfpathcurveto{\pgfqpoint{5.515596in}{2.334333in}}{\pgfqpoint{5.526195in}{2.338724in}}{\pgfqpoint{5.534008in}{2.346537in}}%
\pgfpathcurveto{\pgfqpoint{5.541822in}{2.354351in}}{\pgfqpoint{5.546212in}{2.364950in}}{\pgfqpoint{5.546212in}{2.376000in}}%
\pgfpathcurveto{\pgfqpoint{5.546212in}{2.387050in}}{\pgfqpoint{5.541822in}{2.397649in}}{\pgfqpoint{5.534008in}{2.405463in}}%
\pgfpathcurveto{\pgfqpoint{5.526195in}{2.413276in}}{\pgfqpoint{5.515596in}{2.417667in}}{\pgfqpoint{5.504545in}{2.417667in}}%
\pgfpathcurveto{\pgfqpoint{5.493495in}{2.417667in}}{\pgfqpoint{5.482896in}{2.413276in}}{\pgfqpoint{5.475083in}{2.405463in}}%
\pgfpathcurveto{\pgfqpoint{5.467269in}{2.397649in}}{\pgfqpoint{5.462879in}{2.387050in}}{\pgfqpoint{5.462879in}{2.376000in}}%
\pgfpathcurveto{\pgfqpoint{5.462879in}{2.364950in}}{\pgfqpoint{5.467269in}{2.354351in}}{\pgfqpoint{5.475083in}{2.346537in}}%
\pgfpathcurveto{\pgfqpoint{5.482896in}{2.338724in}}{\pgfqpoint{5.493495in}{2.334333in}}{\pgfqpoint{5.504545in}{2.334333in}}%
\pgfpathclose%
\pgfusepath{stroke,fill}%
\end{pgfscope}%
\begin{pgfscope}%
\pgfpathrectangle{\pgfqpoint{0.800000in}{0.528000in}}{\pgfqpoint{4.960000in}{3.696000in}}%
\pgfusepath{clip}%
\pgfsetbuttcap%
\pgfsetroundjoin%
\definecolor{currentfill}{rgb}{0.000000,0.000000,0.000000}%
\pgfsetfillcolor{currentfill}%
\pgfsetlinewidth{1.003750pt}%
\definecolor{currentstroke}{rgb}{0.000000,0.000000,0.000000}%
\pgfsetstrokecolor{currentstroke}%
\pgfsetdash{}{0pt}%
\pgfpathmoveto{\pgfqpoint{5.504545in}{2.334333in}}%
\pgfpathcurveto{\pgfqpoint{5.515596in}{2.334333in}}{\pgfqpoint{5.526195in}{2.338724in}}{\pgfqpoint{5.534008in}{2.346537in}}%
\pgfpathcurveto{\pgfqpoint{5.541822in}{2.354351in}}{\pgfqpoint{5.546212in}{2.364950in}}{\pgfqpoint{5.546212in}{2.376000in}}%
\pgfpathcurveto{\pgfqpoint{5.546212in}{2.387050in}}{\pgfqpoint{5.541822in}{2.397649in}}{\pgfqpoint{5.534008in}{2.405463in}}%
\pgfpathcurveto{\pgfqpoint{5.526195in}{2.413276in}}{\pgfqpoint{5.515596in}{2.417667in}}{\pgfqpoint{5.504545in}{2.417667in}}%
\pgfpathcurveto{\pgfqpoint{5.493495in}{2.417667in}}{\pgfqpoint{5.482896in}{2.413276in}}{\pgfqpoint{5.475083in}{2.405463in}}%
\pgfpathcurveto{\pgfqpoint{5.467269in}{2.397649in}}{\pgfqpoint{5.462879in}{2.387050in}}{\pgfqpoint{5.462879in}{2.376000in}}%
\pgfpathcurveto{\pgfqpoint{5.462879in}{2.364950in}}{\pgfqpoint{5.467269in}{2.354351in}}{\pgfqpoint{5.475083in}{2.346537in}}%
\pgfpathcurveto{\pgfqpoint{5.482896in}{2.338724in}}{\pgfqpoint{5.493495in}{2.334333in}}{\pgfqpoint{5.504545in}{2.334333in}}%
\pgfpathclose%
\pgfusepath{stroke,fill}%
\end{pgfscope}%
\begin{pgfscope}%
\pgfpathrectangle{\pgfqpoint{0.800000in}{0.528000in}}{\pgfqpoint{4.960000in}{3.696000in}}%
\pgfusepath{clip}%
\pgfsetbuttcap%
\pgfsetroundjoin%
\definecolor{currentfill}{rgb}{0.000000,0.000000,0.000000}%
\pgfsetfillcolor{currentfill}%
\pgfsetlinewidth{1.003750pt}%
\definecolor{currentstroke}{rgb}{0.000000,0.000000,0.000000}%
\pgfsetstrokecolor{currentstroke}%
\pgfsetdash{}{0pt}%
\pgfpathmoveto{\pgfqpoint{5.504545in}{2.334333in}}%
\pgfpathcurveto{\pgfqpoint{5.515596in}{2.334333in}}{\pgfqpoint{5.526195in}{2.338724in}}{\pgfqpoint{5.534008in}{2.346537in}}%
\pgfpathcurveto{\pgfqpoint{5.541822in}{2.354351in}}{\pgfqpoint{5.546212in}{2.364950in}}{\pgfqpoint{5.546212in}{2.376000in}}%
\pgfpathcurveto{\pgfqpoint{5.546212in}{2.387050in}}{\pgfqpoint{5.541822in}{2.397649in}}{\pgfqpoint{5.534008in}{2.405463in}}%
\pgfpathcurveto{\pgfqpoint{5.526195in}{2.413276in}}{\pgfqpoint{5.515596in}{2.417667in}}{\pgfqpoint{5.504545in}{2.417667in}}%
\pgfpathcurveto{\pgfqpoint{5.493495in}{2.417667in}}{\pgfqpoint{5.482896in}{2.413276in}}{\pgfqpoint{5.475083in}{2.405463in}}%
\pgfpathcurveto{\pgfqpoint{5.467269in}{2.397649in}}{\pgfqpoint{5.462879in}{2.387050in}}{\pgfqpoint{5.462879in}{2.376000in}}%
\pgfpathcurveto{\pgfqpoint{5.462879in}{2.364950in}}{\pgfqpoint{5.467269in}{2.354351in}}{\pgfqpoint{5.475083in}{2.346537in}}%
\pgfpathcurveto{\pgfqpoint{5.482896in}{2.338724in}}{\pgfqpoint{5.493495in}{2.334333in}}{\pgfqpoint{5.504545in}{2.334333in}}%
\pgfpathclose%
\pgfusepath{stroke,fill}%
\end{pgfscope}%
\begin{pgfscope}%
\pgfpathrectangle{\pgfqpoint{0.800000in}{0.528000in}}{\pgfqpoint{4.960000in}{3.696000in}}%
\pgfusepath{clip}%
\pgfsetbuttcap%
\pgfsetroundjoin%
\definecolor{currentfill}{rgb}{0.000000,0.000000,0.000000}%
\pgfsetfillcolor{currentfill}%
\pgfsetlinewidth{1.003750pt}%
\definecolor{currentstroke}{rgb}{0.000000,0.000000,0.000000}%
\pgfsetstrokecolor{currentstroke}%
\pgfsetdash{}{0pt}%
\pgfpathmoveto{\pgfqpoint{5.504545in}{2.334333in}}%
\pgfpathcurveto{\pgfqpoint{5.515596in}{2.334333in}}{\pgfqpoint{5.526195in}{2.338724in}}{\pgfqpoint{5.534008in}{2.346537in}}%
\pgfpathcurveto{\pgfqpoint{5.541822in}{2.354351in}}{\pgfqpoint{5.546212in}{2.364950in}}{\pgfqpoint{5.546212in}{2.376000in}}%
\pgfpathcurveto{\pgfqpoint{5.546212in}{2.387050in}}{\pgfqpoint{5.541822in}{2.397649in}}{\pgfqpoint{5.534008in}{2.405463in}}%
\pgfpathcurveto{\pgfqpoint{5.526195in}{2.413276in}}{\pgfqpoint{5.515596in}{2.417667in}}{\pgfqpoint{5.504545in}{2.417667in}}%
\pgfpathcurveto{\pgfqpoint{5.493495in}{2.417667in}}{\pgfqpoint{5.482896in}{2.413276in}}{\pgfqpoint{5.475083in}{2.405463in}}%
\pgfpathcurveto{\pgfqpoint{5.467269in}{2.397649in}}{\pgfqpoint{5.462879in}{2.387050in}}{\pgfqpoint{5.462879in}{2.376000in}}%
\pgfpathcurveto{\pgfqpoint{5.462879in}{2.364950in}}{\pgfqpoint{5.467269in}{2.354351in}}{\pgfqpoint{5.475083in}{2.346537in}}%
\pgfpathcurveto{\pgfqpoint{5.482896in}{2.338724in}}{\pgfqpoint{5.493495in}{2.334333in}}{\pgfqpoint{5.504545in}{2.334333in}}%
\pgfpathclose%
\pgfusepath{stroke,fill}%
\end{pgfscope}%
\begin{pgfscope}%
\pgfpathrectangle{\pgfqpoint{0.800000in}{0.528000in}}{\pgfqpoint{4.960000in}{3.696000in}}%
\pgfusepath{clip}%
\pgfsetbuttcap%
\pgfsetroundjoin%
\definecolor{currentfill}{rgb}{0.000000,0.000000,0.000000}%
\pgfsetfillcolor{currentfill}%
\pgfsetlinewidth{1.003750pt}%
\definecolor{currentstroke}{rgb}{0.000000,0.000000,0.000000}%
\pgfsetstrokecolor{currentstroke}%
\pgfsetdash{}{0pt}%
\pgfpathmoveto{\pgfqpoint{5.504545in}{2.334333in}}%
\pgfpathcurveto{\pgfqpoint{5.515596in}{2.334333in}}{\pgfqpoint{5.526195in}{2.338724in}}{\pgfqpoint{5.534008in}{2.346537in}}%
\pgfpathcurveto{\pgfqpoint{5.541822in}{2.354351in}}{\pgfqpoint{5.546212in}{2.364950in}}{\pgfqpoint{5.546212in}{2.376000in}}%
\pgfpathcurveto{\pgfqpoint{5.546212in}{2.387050in}}{\pgfqpoint{5.541822in}{2.397649in}}{\pgfqpoint{5.534008in}{2.405463in}}%
\pgfpathcurveto{\pgfqpoint{5.526195in}{2.413276in}}{\pgfqpoint{5.515596in}{2.417667in}}{\pgfqpoint{5.504545in}{2.417667in}}%
\pgfpathcurveto{\pgfqpoint{5.493495in}{2.417667in}}{\pgfqpoint{5.482896in}{2.413276in}}{\pgfqpoint{5.475083in}{2.405463in}}%
\pgfpathcurveto{\pgfqpoint{5.467269in}{2.397649in}}{\pgfqpoint{5.462879in}{2.387050in}}{\pgfqpoint{5.462879in}{2.376000in}}%
\pgfpathcurveto{\pgfqpoint{5.462879in}{2.364950in}}{\pgfqpoint{5.467269in}{2.354351in}}{\pgfqpoint{5.475083in}{2.346537in}}%
\pgfpathcurveto{\pgfqpoint{5.482896in}{2.338724in}}{\pgfqpoint{5.493495in}{2.334333in}}{\pgfqpoint{5.504545in}{2.334333in}}%
\pgfpathclose%
\pgfusepath{stroke,fill}%
\end{pgfscope}%
\begin{pgfscope}%
\pgfpathrectangle{\pgfqpoint{0.800000in}{0.528000in}}{\pgfqpoint{4.960000in}{3.696000in}}%
\pgfusepath{clip}%
\pgfsetbuttcap%
\pgfsetroundjoin%
\definecolor{currentfill}{rgb}{0.000000,0.000000,0.000000}%
\pgfsetfillcolor{currentfill}%
\pgfsetlinewidth{1.003750pt}%
\definecolor{currentstroke}{rgb}{0.000000,0.000000,0.000000}%
\pgfsetstrokecolor{currentstroke}%
\pgfsetdash{}{0pt}%
\pgfpathmoveto{\pgfqpoint{5.504545in}{2.334333in}}%
\pgfpathcurveto{\pgfqpoint{5.515596in}{2.334333in}}{\pgfqpoint{5.526195in}{2.338724in}}{\pgfqpoint{5.534008in}{2.346537in}}%
\pgfpathcurveto{\pgfqpoint{5.541822in}{2.354351in}}{\pgfqpoint{5.546212in}{2.364950in}}{\pgfqpoint{5.546212in}{2.376000in}}%
\pgfpathcurveto{\pgfqpoint{5.546212in}{2.387050in}}{\pgfqpoint{5.541822in}{2.397649in}}{\pgfqpoint{5.534008in}{2.405463in}}%
\pgfpathcurveto{\pgfqpoint{5.526195in}{2.413276in}}{\pgfqpoint{5.515596in}{2.417667in}}{\pgfqpoint{5.504545in}{2.417667in}}%
\pgfpathcurveto{\pgfqpoint{5.493495in}{2.417667in}}{\pgfqpoint{5.482896in}{2.413276in}}{\pgfqpoint{5.475083in}{2.405463in}}%
\pgfpathcurveto{\pgfqpoint{5.467269in}{2.397649in}}{\pgfqpoint{5.462879in}{2.387050in}}{\pgfqpoint{5.462879in}{2.376000in}}%
\pgfpathcurveto{\pgfqpoint{5.462879in}{2.364950in}}{\pgfqpoint{5.467269in}{2.354351in}}{\pgfqpoint{5.475083in}{2.346537in}}%
\pgfpathcurveto{\pgfqpoint{5.482896in}{2.338724in}}{\pgfqpoint{5.493495in}{2.334333in}}{\pgfqpoint{5.504545in}{2.334333in}}%
\pgfpathclose%
\pgfusepath{stroke,fill}%
\end{pgfscope}%
\begin{pgfscope}%
\pgfpathrectangle{\pgfqpoint{0.800000in}{0.528000in}}{\pgfqpoint{4.960000in}{3.696000in}}%
\pgfusepath{clip}%
\pgfsetbuttcap%
\pgfsetroundjoin%
\definecolor{currentfill}{rgb}{0.000000,0.000000,0.000000}%
\pgfsetfillcolor{currentfill}%
\pgfsetlinewidth{1.003750pt}%
\definecolor{currentstroke}{rgb}{0.000000,0.000000,0.000000}%
\pgfsetstrokecolor{currentstroke}%
\pgfsetdash{}{0pt}%
\pgfpathmoveto{\pgfqpoint{5.504545in}{2.334333in}}%
\pgfpathcurveto{\pgfqpoint{5.515596in}{2.334333in}}{\pgfqpoint{5.526195in}{2.338724in}}{\pgfqpoint{5.534008in}{2.346537in}}%
\pgfpathcurveto{\pgfqpoint{5.541822in}{2.354351in}}{\pgfqpoint{5.546212in}{2.364950in}}{\pgfqpoint{5.546212in}{2.376000in}}%
\pgfpathcurveto{\pgfqpoint{5.546212in}{2.387050in}}{\pgfqpoint{5.541822in}{2.397649in}}{\pgfqpoint{5.534008in}{2.405463in}}%
\pgfpathcurveto{\pgfqpoint{5.526195in}{2.413276in}}{\pgfqpoint{5.515596in}{2.417667in}}{\pgfqpoint{5.504545in}{2.417667in}}%
\pgfpathcurveto{\pgfqpoint{5.493495in}{2.417667in}}{\pgfqpoint{5.482896in}{2.413276in}}{\pgfqpoint{5.475083in}{2.405463in}}%
\pgfpathcurveto{\pgfqpoint{5.467269in}{2.397649in}}{\pgfqpoint{5.462879in}{2.387050in}}{\pgfqpoint{5.462879in}{2.376000in}}%
\pgfpathcurveto{\pgfqpoint{5.462879in}{2.364950in}}{\pgfqpoint{5.467269in}{2.354351in}}{\pgfqpoint{5.475083in}{2.346537in}}%
\pgfpathcurveto{\pgfqpoint{5.482896in}{2.338724in}}{\pgfqpoint{5.493495in}{2.334333in}}{\pgfqpoint{5.504545in}{2.334333in}}%
\pgfpathclose%
\pgfusepath{stroke,fill}%
\end{pgfscope}%
\begin{pgfscope}%
\pgfpathrectangle{\pgfqpoint{0.800000in}{0.528000in}}{\pgfqpoint{4.960000in}{3.696000in}}%
\pgfusepath{clip}%
\pgfsetbuttcap%
\pgfsetroundjoin%
\definecolor{currentfill}{rgb}{0.000000,0.000000,0.000000}%
\pgfsetfillcolor{currentfill}%
\pgfsetlinewidth{1.003750pt}%
\definecolor{currentstroke}{rgb}{0.000000,0.000000,0.000000}%
\pgfsetstrokecolor{currentstroke}%
\pgfsetdash{}{0pt}%
\pgfpathmoveto{\pgfqpoint{5.504545in}{2.334333in}}%
\pgfpathcurveto{\pgfqpoint{5.515596in}{2.334333in}}{\pgfqpoint{5.526195in}{2.338724in}}{\pgfqpoint{5.534008in}{2.346537in}}%
\pgfpathcurveto{\pgfqpoint{5.541822in}{2.354351in}}{\pgfqpoint{5.546212in}{2.364950in}}{\pgfqpoint{5.546212in}{2.376000in}}%
\pgfpathcurveto{\pgfqpoint{5.546212in}{2.387050in}}{\pgfqpoint{5.541822in}{2.397649in}}{\pgfqpoint{5.534008in}{2.405463in}}%
\pgfpathcurveto{\pgfqpoint{5.526195in}{2.413276in}}{\pgfqpoint{5.515596in}{2.417667in}}{\pgfqpoint{5.504545in}{2.417667in}}%
\pgfpathcurveto{\pgfqpoint{5.493495in}{2.417667in}}{\pgfqpoint{5.482896in}{2.413276in}}{\pgfqpoint{5.475083in}{2.405463in}}%
\pgfpathcurveto{\pgfqpoint{5.467269in}{2.397649in}}{\pgfqpoint{5.462879in}{2.387050in}}{\pgfqpoint{5.462879in}{2.376000in}}%
\pgfpathcurveto{\pgfqpoint{5.462879in}{2.364950in}}{\pgfqpoint{5.467269in}{2.354351in}}{\pgfqpoint{5.475083in}{2.346537in}}%
\pgfpathcurveto{\pgfqpoint{5.482896in}{2.338724in}}{\pgfqpoint{5.493495in}{2.334333in}}{\pgfqpoint{5.504545in}{2.334333in}}%
\pgfpathclose%
\pgfusepath{stroke,fill}%
\end{pgfscope}%
\begin{pgfscope}%
\pgfpathrectangle{\pgfqpoint{0.800000in}{0.528000in}}{\pgfqpoint{4.960000in}{3.696000in}}%
\pgfusepath{clip}%
\pgfsetbuttcap%
\pgfsetroundjoin%
\definecolor{currentfill}{rgb}{0.000000,0.000000,0.000000}%
\pgfsetfillcolor{currentfill}%
\pgfsetlinewidth{1.003750pt}%
\definecolor{currentstroke}{rgb}{0.000000,0.000000,0.000000}%
\pgfsetstrokecolor{currentstroke}%
\pgfsetdash{}{0pt}%
\pgfpathmoveto{\pgfqpoint{5.504545in}{2.334333in}}%
\pgfpathcurveto{\pgfqpoint{5.515596in}{2.334333in}}{\pgfqpoint{5.526195in}{2.338724in}}{\pgfqpoint{5.534008in}{2.346537in}}%
\pgfpathcurveto{\pgfqpoint{5.541822in}{2.354351in}}{\pgfqpoint{5.546212in}{2.364950in}}{\pgfqpoint{5.546212in}{2.376000in}}%
\pgfpathcurveto{\pgfqpoint{5.546212in}{2.387050in}}{\pgfqpoint{5.541822in}{2.397649in}}{\pgfqpoint{5.534008in}{2.405463in}}%
\pgfpathcurveto{\pgfqpoint{5.526195in}{2.413276in}}{\pgfqpoint{5.515596in}{2.417667in}}{\pgfqpoint{5.504545in}{2.417667in}}%
\pgfpathcurveto{\pgfqpoint{5.493495in}{2.417667in}}{\pgfqpoint{5.482896in}{2.413276in}}{\pgfqpoint{5.475083in}{2.405463in}}%
\pgfpathcurveto{\pgfqpoint{5.467269in}{2.397649in}}{\pgfqpoint{5.462879in}{2.387050in}}{\pgfqpoint{5.462879in}{2.376000in}}%
\pgfpathcurveto{\pgfqpoint{5.462879in}{2.364950in}}{\pgfqpoint{5.467269in}{2.354351in}}{\pgfqpoint{5.475083in}{2.346537in}}%
\pgfpathcurveto{\pgfqpoint{5.482896in}{2.338724in}}{\pgfqpoint{5.493495in}{2.334333in}}{\pgfqpoint{5.504545in}{2.334333in}}%
\pgfpathclose%
\pgfusepath{stroke,fill}%
\end{pgfscope}%
\begin{pgfscope}%
\pgfpathrectangle{\pgfqpoint{0.800000in}{0.528000in}}{\pgfqpoint{4.960000in}{3.696000in}}%
\pgfusepath{clip}%
\pgfsetbuttcap%
\pgfsetroundjoin%
\definecolor{currentfill}{rgb}{0.000000,0.000000,0.000000}%
\pgfsetfillcolor{currentfill}%
\pgfsetlinewidth{1.003750pt}%
\definecolor{currentstroke}{rgb}{0.000000,0.000000,0.000000}%
\pgfsetstrokecolor{currentstroke}%
\pgfsetdash{}{0pt}%
\pgfpathmoveto{\pgfqpoint{5.504545in}{2.334333in}}%
\pgfpathcurveto{\pgfqpoint{5.515596in}{2.334333in}}{\pgfqpoint{5.526195in}{2.338724in}}{\pgfqpoint{5.534008in}{2.346537in}}%
\pgfpathcurveto{\pgfqpoint{5.541822in}{2.354351in}}{\pgfqpoint{5.546212in}{2.364950in}}{\pgfqpoint{5.546212in}{2.376000in}}%
\pgfpathcurveto{\pgfqpoint{5.546212in}{2.387050in}}{\pgfqpoint{5.541822in}{2.397649in}}{\pgfqpoint{5.534008in}{2.405463in}}%
\pgfpathcurveto{\pgfqpoint{5.526195in}{2.413276in}}{\pgfqpoint{5.515596in}{2.417667in}}{\pgfqpoint{5.504545in}{2.417667in}}%
\pgfpathcurveto{\pgfqpoint{5.493495in}{2.417667in}}{\pgfqpoint{5.482896in}{2.413276in}}{\pgfqpoint{5.475083in}{2.405463in}}%
\pgfpathcurveto{\pgfqpoint{5.467269in}{2.397649in}}{\pgfqpoint{5.462879in}{2.387050in}}{\pgfqpoint{5.462879in}{2.376000in}}%
\pgfpathcurveto{\pgfqpoint{5.462879in}{2.364950in}}{\pgfqpoint{5.467269in}{2.354351in}}{\pgfqpoint{5.475083in}{2.346537in}}%
\pgfpathcurveto{\pgfqpoint{5.482896in}{2.338724in}}{\pgfqpoint{5.493495in}{2.334333in}}{\pgfqpoint{5.504545in}{2.334333in}}%
\pgfpathclose%
\pgfusepath{stroke,fill}%
\end{pgfscope}%
\begin{pgfscope}%
\pgfpathrectangle{\pgfqpoint{0.800000in}{0.528000in}}{\pgfqpoint{4.960000in}{3.696000in}}%
\pgfusepath{clip}%
\pgfsetbuttcap%
\pgfsetroundjoin%
\definecolor{currentfill}{rgb}{0.000000,0.000000,0.000000}%
\pgfsetfillcolor{currentfill}%
\pgfsetlinewidth{1.003750pt}%
\definecolor{currentstroke}{rgb}{0.000000,0.000000,0.000000}%
\pgfsetstrokecolor{currentstroke}%
\pgfsetdash{}{0pt}%
\pgfpathmoveto{\pgfqpoint{5.504545in}{2.334333in}}%
\pgfpathcurveto{\pgfqpoint{5.515596in}{2.334333in}}{\pgfqpoint{5.526195in}{2.338724in}}{\pgfqpoint{5.534008in}{2.346537in}}%
\pgfpathcurveto{\pgfqpoint{5.541822in}{2.354351in}}{\pgfqpoint{5.546212in}{2.364950in}}{\pgfqpoint{5.546212in}{2.376000in}}%
\pgfpathcurveto{\pgfqpoint{5.546212in}{2.387050in}}{\pgfqpoint{5.541822in}{2.397649in}}{\pgfqpoint{5.534008in}{2.405463in}}%
\pgfpathcurveto{\pgfqpoint{5.526195in}{2.413276in}}{\pgfqpoint{5.515596in}{2.417667in}}{\pgfqpoint{5.504545in}{2.417667in}}%
\pgfpathcurveto{\pgfqpoint{5.493495in}{2.417667in}}{\pgfqpoint{5.482896in}{2.413276in}}{\pgfqpoint{5.475083in}{2.405463in}}%
\pgfpathcurveto{\pgfqpoint{5.467269in}{2.397649in}}{\pgfqpoint{5.462879in}{2.387050in}}{\pgfqpoint{5.462879in}{2.376000in}}%
\pgfpathcurveto{\pgfqpoint{5.462879in}{2.364950in}}{\pgfqpoint{5.467269in}{2.354351in}}{\pgfqpoint{5.475083in}{2.346537in}}%
\pgfpathcurveto{\pgfqpoint{5.482896in}{2.338724in}}{\pgfqpoint{5.493495in}{2.334333in}}{\pgfqpoint{5.504545in}{2.334333in}}%
\pgfpathclose%
\pgfusepath{stroke,fill}%
\end{pgfscope}%
\begin{pgfscope}%
\pgfpathrectangle{\pgfqpoint{0.800000in}{0.528000in}}{\pgfqpoint{4.960000in}{3.696000in}}%
\pgfusepath{clip}%
\pgfsetbuttcap%
\pgfsetroundjoin%
\definecolor{currentfill}{rgb}{0.000000,0.000000,0.000000}%
\pgfsetfillcolor{currentfill}%
\pgfsetlinewidth{1.003750pt}%
\definecolor{currentstroke}{rgb}{0.000000,0.000000,0.000000}%
\pgfsetstrokecolor{currentstroke}%
\pgfsetdash{}{0pt}%
\pgfpathmoveto{\pgfqpoint{5.504545in}{2.334333in}}%
\pgfpathcurveto{\pgfqpoint{5.515596in}{2.334333in}}{\pgfqpoint{5.526195in}{2.338724in}}{\pgfqpoint{5.534008in}{2.346537in}}%
\pgfpathcurveto{\pgfqpoint{5.541822in}{2.354351in}}{\pgfqpoint{5.546212in}{2.364950in}}{\pgfqpoint{5.546212in}{2.376000in}}%
\pgfpathcurveto{\pgfqpoint{5.546212in}{2.387050in}}{\pgfqpoint{5.541822in}{2.397649in}}{\pgfqpoint{5.534008in}{2.405463in}}%
\pgfpathcurveto{\pgfqpoint{5.526195in}{2.413276in}}{\pgfqpoint{5.515596in}{2.417667in}}{\pgfqpoint{5.504545in}{2.417667in}}%
\pgfpathcurveto{\pgfqpoint{5.493495in}{2.417667in}}{\pgfqpoint{5.482896in}{2.413276in}}{\pgfqpoint{5.475083in}{2.405463in}}%
\pgfpathcurveto{\pgfqpoint{5.467269in}{2.397649in}}{\pgfqpoint{5.462879in}{2.387050in}}{\pgfqpoint{5.462879in}{2.376000in}}%
\pgfpathcurveto{\pgfqpoint{5.462879in}{2.364950in}}{\pgfqpoint{5.467269in}{2.354351in}}{\pgfqpoint{5.475083in}{2.346537in}}%
\pgfpathcurveto{\pgfqpoint{5.482896in}{2.338724in}}{\pgfqpoint{5.493495in}{2.334333in}}{\pgfqpoint{5.504545in}{2.334333in}}%
\pgfpathclose%
\pgfusepath{stroke,fill}%
\end{pgfscope}%
\begin{pgfscope}%
\pgfpathrectangle{\pgfqpoint{0.800000in}{0.528000in}}{\pgfqpoint{4.960000in}{3.696000in}}%
\pgfusepath{clip}%
\pgfsetbuttcap%
\pgfsetroundjoin%
\definecolor{currentfill}{rgb}{0.000000,0.000000,0.000000}%
\pgfsetfillcolor{currentfill}%
\pgfsetlinewidth{1.003750pt}%
\definecolor{currentstroke}{rgb}{0.000000,0.000000,0.000000}%
\pgfsetstrokecolor{currentstroke}%
\pgfsetdash{}{0pt}%
\pgfpathmoveto{\pgfqpoint{5.504545in}{2.334333in}}%
\pgfpathcurveto{\pgfqpoint{5.515596in}{2.334333in}}{\pgfqpoint{5.526195in}{2.338724in}}{\pgfqpoint{5.534008in}{2.346537in}}%
\pgfpathcurveto{\pgfqpoint{5.541822in}{2.354351in}}{\pgfqpoint{5.546212in}{2.364950in}}{\pgfqpoint{5.546212in}{2.376000in}}%
\pgfpathcurveto{\pgfqpoint{5.546212in}{2.387050in}}{\pgfqpoint{5.541822in}{2.397649in}}{\pgfqpoint{5.534008in}{2.405463in}}%
\pgfpathcurveto{\pgfqpoint{5.526195in}{2.413276in}}{\pgfqpoint{5.515596in}{2.417667in}}{\pgfqpoint{5.504545in}{2.417667in}}%
\pgfpathcurveto{\pgfqpoint{5.493495in}{2.417667in}}{\pgfqpoint{5.482896in}{2.413276in}}{\pgfqpoint{5.475083in}{2.405463in}}%
\pgfpathcurveto{\pgfqpoint{5.467269in}{2.397649in}}{\pgfqpoint{5.462879in}{2.387050in}}{\pgfqpoint{5.462879in}{2.376000in}}%
\pgfpathcurveto{\pgfqpoint{5.462879in}{2.364950in}}{\pgfqpoint{5.467269in}{2.354351in}}{\pgfqpoint{5.475083in}{2.346537in}}%
\pgfpathcurveto{\pgfqpoint{5.482896in}{2.338724in}}{\pgfqpoint{5.493495in}{2.334333in}}{\pgfqpoint{5.504545in}{2.334333in}}%
\pgfpathclose%
\pgfusepath{stroke,fill}%
\end{pgfscope}%
\begin{pgfscope}%
\pgfpathrectangle{\pgfqpoint{0.800000in}{0.528000in}}{\pgfqpoint{4.960000in}{3.696000in}}%
\pgfusepath{clip}%
\pgfsetbuttcap%
\pgfsetroundjoin%
\definecolor{currentfill}{rgb}{0.000000,0.000000,0.000000}%
\pgfsetfillcolor{currentfill}%
\pgfsetlinewidth{1.003750pt}%
\definecolor{currentstroke}{rgb}{0.000000,0.000000,0.000000}%
\pgfsetstrokecolor{currentstroke}%
\pgfsetdash{}{0pt}%
\pgfpathmoveto{\pgfqpoint{5.504545in}{2.334333in}}%
\pgfpathcurveto{\pgfqpoint{5.515596in}{2.334333in}}{\pgfqpoint{5.526195in}{2.338724in}}{\pgfqpoint{5.534008in}{2.346537in}}%
\pgfpathcurveto{\pgfqpoint{5.541822in}{2.354351in}}{\pgfqpoint{5.546212in}{2.364950in}}{\pgfqpoint{5.546212in}{2.376000in}}%
\pgfpathcurveto{\pgfqpoint{5.546212in}{2.387050in}}{\pgfqpoint{5.541822in}{2.397649in}}{\pgfqpoint{5.534008in}{2.405463in}}%
\pgfpathcurveto{\pgfqpoint{5.526195in}{2.413276in}}{\pgfqpoint{5.515596in}{2.417667in}}{\pgfqpoint{5.504545in}{2.417667in}}%
\pgfpathcurveto{\pgfqpoint{5.493495in}{2.417667in}}{\pgfqpoint{5.482896in}{2.413276in}}{\pgfqpoint{5.475083in}{2.405463in}}%
\pgfpathcurveto{\pgfqpoint{5.467269in}{2.397649in}}{\pgfqpoint{5.462879in}{2.387050in}}{\pgfqpoint{5.462879in}{2.376000in}}%
\pgfpathcurveto{\pgfqpoint{5.462879in}{2.364950in}}{\pgfqpoint{5.467269in}{2.354351in}}{\pgfqpoint{5.475083in}{2.346537in}}%
\pgfpathcurveto{\pgfqpoint{5.482896in}{2.338724in}}{\pgfqpoint{5.493495in}{2.334333in}}{\pgfqpoint{5.504545in}{2.334333in}}%
\pgfpathclose%
\pgfusepath{stroke,fill}%
\end{pgfscope}%
\begin{pgfscope}%
\pgfpathrectangle{\pgfqpoint{0.800000in}{0.528000in}}{\pgfqpoint{4.960000in}{3.696000in}}%
\pgfusepath{clip}%
\pgfsetbuttcap%
\pgfsetroundjoin%
\definecolor{currentfill}{rgb}{0.000000,0.000000,0.000000}%
\pgfsetfillcolor{currentfill}%
\pgfsetlinewidth{1.003750pt}%
\definecolor{currentstroke}{rgb}{0.000000,0.000000,0.000000}%
\pgfsetstrokecolor{currentstroke}%
\pgfsetdash{}{0pt}%
\pgfpathmoveto{\pgfqpoint{5.504545in}{2.334333in}}%
\pgfpathcurveto{\pgfqpoint{5.515596in}{2.334333in}}{\pgfqpoint{5.526195in}{2.338724in}}{\pgfqpoint{5.534008in}{2.346537in}}%
\pgfpathcurveto{\pgfqpoint{5.541822in}{2.354351in}}{\pgfqpoint{5.546212in}{2.364950in}}{\pgfqpoint{5.546212in}{2.376000in}}%
\pgfpathcurveto{\pgfqpoint{5.546212in}{2.387050in}}{\pgfqpoint{5.541822in}{2.397649in}}{\pgfqpoint{5.534008in}{2.405463in}}%
\pgfpathcurveto{\pgfqpoint{5.526195in}{2.413276in}}{\pgfqpoint{5.515596in}{2.417667in}}{\pgfqpoint{5.504545in}{2.417667in}}%
\pgfpathcurveto{\pgfqpoint{5.493495in}{2.417667in}}{\pgfqpoint{5.482896in}{2.413276in}}{\pgfqpoint{5.475083in}{2.405463in}}%
\pgfpathcurveto{\pgfqpoint{5.467269in}{2.397649in}}{\pgfqpoint{5.462879in}{2.387050in}}{\pgfqpoint{5.462879in}{2.376000in}}%
\pgfpathcurveto{\pgfqpoint{5.462879in}{2.364950in}}{\pgfqpoint{5.467269in}{2.354351in}}{\pgfqpoint{5.475083in}{2.346537in}}%
\pgfpathcurveto{\pgfqpoint{5.482896in}{2.338724in}}{\pgfqpoint{5.493495in}{2.334333in}}{\pgfqpoint{5.504545in}{2.334333in}}%
\pgfpathclose%
\pgfusepath{stroke,fill}%
\end{pgfscope}%
\begin{pgfscope}%
\pgfpathrectangle{\pgfqpoint{0.800000in}{0.528000in}}{\pgfqpoint{4.960000in}{3.696000in}}%
\pgfusepath{clip}%
\pgfsetbuttcap%
\pgfsetroundjoin%
\definecolor{currentfill}{rgb}{0.000000,0.000000,0.000000}%
\pgfsetfillcolor{currentfill}%
\pgfsetlinewidth{1.003750pt}%
\definecolor{currentstroke}{rgb}{0.000000,0.000000,0.000000}%
\pgfsetstrokecolor{currentstroke}%
\pgfsetdash{}{0pt}%
\pgfpathmoveto{\pgfqpoint{5.504545in}{2.334333in}}%
\pgfpathcurveto{\pgfqpoint{5.515596in}{2.334333in}}{\pgfqpoint{5.526195in}{2.338724in}}{\pgfqpoint{5.534008in}{2.346537in}}%
\pgfpathcurveto{\pgfqpoint{5.541822in}{2.354351in}}{\pgfqpoint{5.546212in}{2.364950in}}{\pgfqpoint{5.546212in}{2.376000in}}%
\pgfpathcurveto{\pgfqpoint{5.546212in}{2.387050in}}{\pgfqpoint{5.541822in}{2.397649in}}{\pgfqpoint{5.534008in}{2.405463in}}%
\pgfpathcurveto{\pgfqpoint{5.526195in}{2.413276in}}{\pgfqpoint{5.515596in}{2.417667in}}{\pgfqpoint{5.504545in}{2.417667in}}%
\pgfpathcurveto{\pgfqpoint{5.493495in}{2.417667in}}{\pgfqpoint{5.482896in}{2.413276in}}{\pgfqpoint{5.475083in}{2.405463in}}%
\pgfpathcurveto{\pgfqpoint{5.467269in}{2.397649in}}{\pgfqpoint{5.462879in}{2.387050in}}{\pgfqpoint{5.462879in}{2.376000in}}%
\pgfpathcurveto{\pgfqpoint{5.462879in}{2.364950in}}{\pgfqpoint{5.467269in}{2.354351in}}{\pgfqpoint{5.475083in}{2.346537in}}%
\pgfpathcurveto{\pgfqpoint{5.482896in}{2.338724in}}{\pgfqpoint{5.493495in}{2.334333in}}{\pgfqpoint{5.504545in}{2.334333in}}%
\pgfpathclose%
\pgfusepath{stroke,fill}%
\end{pgfscope}%
\begin{pgfscope}%
\pgfpathrectangle{\pgfqpoint{0.800000in}{0.528000in}}{\pgfqpoint{4.960000in}{3.696000in}}%
\pgfusepath{clip}%
\pgfsetbuttcap%
\pgfsetroundjoin%
\definecolor{currentfill}{rgb}{0.000000,0.000000,0.000000}%
\pgfsetfillcolor{currentfill}%
\pgfsetlinewidth{1.003750pt}%
\definecolor{currentstroke}{rgb}{0.000000,0.000000,0.000000}%
\pgfsetstrokecolor{currentstroke}%
\pgfsetdash{}{0pt}%
\pgfpathmoveto{\pgfqpoint{5.504545in}{2.334333in}}%
\pgfpathcurveto{\pgfqpoint{5.515596in}{2.334333in}}{\pgfqpoint{5.526195in}{2.338724in}}{\pgfqpoint{5.534008in}{2.346537in}}%
\pgfpathcurveto{\pgfqpoint{5.541822in}{2.354351in}}{\pgfqpoint{5.546212in}{2.364950in}}{\pgfqpoint{5.546212in}{2.376000in}}%
\pgfpathcurveto{\pgfqpoint{5.546212in}{2.387050in}}{\pgfqpoint{5.541822in}{2.397649in}}{\pgfqpoint{5.534008in}{2.405463in}}%
\pgfpathcurveto{\pgfqpoint{5.526195in}{2.413276in}}{\pgfqpoint{5.515596in}{2.417667in}}{\pgfqpoint{5.504545in}{2.417667in}}%
\pgfpathcurveto{\pgfqpoint{5.493495in}{2.417667in}}{\pgfqpoint{5.482896in}{2.413276in}}{\pgfqpoint{5.475083in}{2.405463in}}%
\pgfpathcurveto{\pgfqpoint{5.467269in}{2.397649in}}{\pgfqpoint{5.462879in}{2.387050in}}{\pgfqpoint{5.462879in}{2.376000in}}%
\pgfpathcurveto{\pgfqpoint{5.462879in}{2.364950in}}{\pgfqpoint{5.467269in}{2.354351in}}{\pgfqpoint{5.475083in}{2.346537in}}%
\pgfpathcurveto{\pgfqpoint{5.482896in}{2.338724in}}{\pgfqpoint{5.493495in}{2.334333in}}{\pgfqpoint{5.504545in}{2.334333in}}%
\pgfpathclose%
\pgfusepath{stroke,fill}%
\end{pgfscope}%
\begin{pgfscope}%
\pgfpathrectangle{\pgfqpoint{0.800000in}{0.528000in}}{\pgfqpoint{4.960000in}{3.696000in}}%
\pgfusepath{clip}%
\pgfsetbuttcap%
\pgfsetroundjoin%
\definecolor{currentfill}{rgb}{0.000000,0.000000,0.000000}%
\pgfsetfillcolor{currentfill}%
\pgfsetlinewidth{1.003750pt}%
\definecolor{currentstroke}{rgb}{0.000000,0.000000,0.000000}%
\pgfsetstrokecolor{currentstroke}%
\pgfsetdash{}{0pt}%
\pgfpathmoveto{\pgfqpoint{5.504545in}{2.334333in}}%
\pgfpathcurveto{\pgfqpoint{5.515596in}{2.334333in}}{\pgfqpoint{5.526195in}{2.338724in}}{\pgfqpoint{5.534008in}{2.346537in}}%
\pgfpathcurveto{\pgfqpoint{5.541822in}{2.354351in}}{\pgfqpoint{5.546212in}{2.364950in}}{\pgfqpoint{5.546212in}{2.376000in}}%
\pgfpathcurveto{\pgfqpoint{5.546212in}{2.387050in}}{\pgfqpoint{5.541822in}{2.397649in}}{\pgfqpoint{5.534008in}{2.405463in}}%
\pgfpathcurveto{\pgfqpoint{5.526195in}{2.413276in}}{\pgfqpoint{5.515596in}{2.417667in}}{\pgfqpoint{5.504545in}{2.417667in}}%
\pgfpathcurveto{\pgfqpoint{5.493495in}{2.417667in}}{\pgfqpoint{5.482896in}{2.413276in}}{\pgfqpoint{5.475083in}{2.405463in}}%
\pgfpathcurveto{\pgfqpoint{5.467269in}{2.397649in}}{\pgfqpoint{5.462879in}{2.387050in}}{\pgfqpoint{5.462879in}{2.376000in}}%
\pgfpathcurveto{\pgfqpoint{5.462879in}{2.364950in}}{\pgfqpoint{5.467269in}{2.354351in}}{\pgfqpoint{5.475083in}{2.346537in}}%
\pgfpathcurveto{\pgfqpoint{5.482896in}{2.338724in}}{\pgfqpoint{5.493495in}{2.334333in}}{\pgfqpoint{5.504545in}{2.334333in}}%
\pgfpathclose%
\pgfusepath{stroke,fill}%
\end{pgfscope}%
\begin{pgfscope}%
\pgfpathrectangle{\pgfqpoint{0.800000in}{0.528000in}}{\pgfqpoint{4.960000in}{3.696000in}}%
\pgfusepath{clip}%
\pgfsetbuttcap%
\pgfsetroundjoin%
\definecolor{currentfill}{rgb}{0.000000,0.000000,0.000000}%
\pgfsetfillcolor{currentfill}%
\pgfsetlinewidth{1.003750pt}%
\definecolor{currentstroke}{rgb}{0.000000,0.000000,0.000000}%
\pgfsetstrokecolor{currentstroke}%
\pgfsetdash{}{0pt}%
\pgfpathmoveto{\pgfqpoint{5.504545in}{2.334333in}}%
\pgfpathcurveto{\pgfqpoint{5.515596in}{2.334333in}}{\pgfqpoint{5.526195in}{2.338724in}}{\pgfqpoint{5.534008in}{2.346537in}}%
\pgfpathcurveto{\pgfqpoint{5.541822in}{2.354351in}}{\pgfqpoint{5.546212in}{2.364950in}}{\pgfqpoint{5.546212in}{2.376000in}}%
\pgfpathcurveto{\pgfqpoint{5.546212in}{2.387050in}}{\pgfqpoint{5.541822in}{2.397649in}}{\pgfqpoint{5.534008in}{2.405463in}}%
\pgfpathcurveto{\pgfqpoint{5.526195in}{2.413276in}}{\pgfqpoint{5.515596in}{2.417667in}}{\pgfqpoint{5.504545in}{2.417667in}}%
\pgfpathcurveto{\pgfqpoint{5.493495in}{2.417667in}}{\pgfqpoint{5.482896in}{2.413276in}}{\pgfqpoint{5.475083in}{2.405463in}}%
\pgfpathcurveto{\pgfqpoint{5.467269in}{2.397649in}}{\pgfqpoint{5.462879in}{2.387050in}}{\pgfqpoint{5.462879in}{2.376000in}}%
\pgfpathcurveto{\pgfqpoint{5.462879in}{2.364950in}}{\pgfqpoint{5.467269in}{2.354351in}}{\pgfqpoint{5.475083in}{2.346537in}}%
\pgfpathcurveto{\pgfqpoint{5.482896in}{2.338724in}}{\pgfqpoint{5.493495in}{2.334333in}}{\pgfqpoint{5.504545in}{2.334333in}}%
\pgfpathclose%
\pgfusepath{stroke,fill}%
\end{pgfscope}%
\begin{pgfscope}%
\pgfpathrectangle{\pgfqpoint{0.800000in}{0.528000in}}{\pgfqpoint{4.960000in}{3.696000in}}%
\pgfusepath{clip}%
\pgfsetbuttcap%
\pgfsetroundjoin%
\definecolor{currentfill}{rgb}{0.000000,0.000000,0.000000}%
\pgfsetfillcolor{currentfill}%
\pgfsetlinewidth{1.003750pt}%
\definecolor{currentstroke}{rgb}{0.000000,0.000000,0.000000}%
\pgfsetstrokecolor{currentstroke}%
\pgfsetdash{}{0pt}%
\pgfpathmoveto{\pgfqpoint{5.504545in}{2.334333in}}%
\pgfpathcurveto{\pgfqpoint{5.515596in}{2.334333in}}{\pgfqpoint{5.526195in}{2.338724in}}{\pgfqpoint{5.534008in}{2.346537in}}%
\pgfpathcurveto{\pgfqpoint{5.541822in}{2.354351in}}{\pgfqpoint{5.546212in}{2.364950in}}{\pgfqpoint{5.546212in}{2.376000in}}%
\pgfpathcurveto{\pgfqpoint{5.546212in}{2.387050in}}{\pgfqpoint{5.541822in}{2.397649in}}{\pgfqpoint{5.534008in}{2.405463in}}%
\pgfpathcurveto{\pgfqpoint{5.526195in}{2.413276in}}{\pgfqpoint{5.515596in}{2.417667in}}{\pgfqpoint{5.504545in}{2.417667in}}%
\pgfpathcurveto{\pgfqpoint{5.493495in}{2.417667in}}{\pgfqpoint{5.482896in}{2.413276in}}{\pgfqpoint{5.475083in}{2.405463in}}%
\pgfpathcurveto{\pgfqpoint{5.467269in}{2.397649in}}{\pgfqpoint{5.462879in}{2.387050in}}{\pgfqpoint{5.462879in}{2.376000in}}%
\pgfpathcurveto{\pgfqpoint{5.462879in}{2.364950in}}{\pgfqpoint{5.467269in}{2.354351in}}{\pgfqpoint{5.475083in}{2.346537in}}%
\pgfpathcurveto{\pgfqpoint{5.482896in}{2.338724in}}{\pgfqpoint{5.493495in}{2.334333in}}{\pgfqpoint{5.504545in}{2.334333in}}%
\pgfpathclose%
\pgfusepath{stroke,fill}%
\end{pgfscope}%
\begin{pgfscope}%
\pgfpathrectangle{\pgfqpoint{0.800000in}{0.528000in}}{\pgfqpoint{4.960000in}{3.696000in}}%
\pgfusepath{clip}%
\pgfsetbuttcap%
\pgfsetroundjoin%
\definecolor{currentfill}{rgb}{0.000000,0.000000,0.000000}%
\pgfsetfillcolor{currentfill}%
\pgfsetlinewidth{1.003750pt}%
\definecolor{currentstroke}{rgb}{0.000000,0.000000,0.000000}%
\pgfsetstrokecolor{currentstroke}%
\pgfsetdash{}{0pt}%
\pgfpathmoveto{\pgfqpoint{5.504545in}{2.334333in}}%
\pgfpathcurveto{\pgfqpoint{5.515596in}{2.334333in}}{\pgfqpoint{5.526195in}{2.338724in}}{\pgfqpoint{5.534008in}{2.346537in}}%
\pgfpathcurveto{\pgfqpoint{5.541822in}{2.354351in}}{\pgfqpoint{5.546212in}{2.364950in}}{\pgfqpoint{5.546212in}{2.376000in}}%
\pgfpathcurveto{\pgfqpoint{5.546212in}{2.387050in}}{\pgfqpoint{5.541822in}{2.397649in}}{\pgfqpoint{5.534008in}{2.405463in}}%
\pgfpathcurveto{\pgfqpoint{5.526195in}{2.413276in}}{\pgfqpoint{5.515596in}{2.417667in}}{\pgfqpoint{5.504545in}{2.417667in}}%
\pgfpathcurveto{\pgfqpoint{5.493495in}{2.417667in}}{\pgfqpoint{5.482896in}{2.413276in}}{\pgfqpoint{5.475083in}{2.405463in}}%
\pgfpathcurveto{\pgfqpoint{5.467269in}{2.397649in}}{\pgfqpoint{5.462879in}{2.387050in}}{\pgfqpoint{5.462879in}{2.376000in}}%
\pgfpathcurveto{\pgfqpoint{5.462879in}{2.364950in}}{\pgfqpoint{5.467269in}{2.354351in}}{\pgfqpoint{5.475083in}{2.346537in}}%
\pgfpathcurveto{\pgfqpoint{5.482896in}{2.338724in}}{\pgfqpoint{5.493495in}{2.334333in}}{\pgfqpoint{5.504545in}{2.334333in}}%
\pgfpathclose%
\pgfusepath{stroke,fill}%
\end{pgfscope}%
\begin{pgfscope}%
\pgfpathrectangle{\pgfqpoint{0.800000in}{0.528000in}}{\pgfqpoint{4.960000in}{3.696000in}}%
\pgfusepath{clip}%
\pgfsetbuttcap%
\pgfsetroundjoin%
\definecolor{currentfill}{rgb}{0.000000,0.000000,0.000000}%
\pgfsetfillcolor{currentfill}%
\pgfsetlinewidth{1.003750pt}%
\definecolor{currentstroke}{rgb}{0.000000,0.000000,0.000000}%
\pgfsetstrokecolor{currentstroke}%
\pgfsetdash{}{0pt}%
\pgfpathmoveto{\pgfqpoint{5.504545in}{2.334333in}}%
\pgfpathcurveto{\pgfqpoint{5.515596in}{2.334333in}}{\pgfqpoint{5.526195in}{2.338724in}}{\pgfqpoint{5.534008in}{2.346537in}}%
\pgfpathcurveto{\pgfqpoint{5.541822in}{2.354351in}}{\pgfqpoint{5.546212in}{2.364950in}}{\pgfqpoint{5.546212in}{2.376000in}}%
\pgfpathcurveto{\pgfqpoint{5.546212in}{2.387050in}}{\pgfqpoint{5.541822in}{2.397649in}}{\pgfqpoint{5.534008in}{2.405463in}}%
\pgfpathcurveto{\pgfqpoint{5.526195in}{2.413276in}}{\pgfqpoint{5.515596in}{2.417667in}}{\pgfqpoint{5.504545in}{2.417667in}}%
\pgfpathcurveto{\pgfqpoint{5.493495in}{2.417667in}}{\pgfqpoint{5.482896in}{2.413276in}}{\pgfqpoint{5.475083in}{2.405463in}}%
\pgfpathcurveto{\pgfqpoint{5.467269in}{2.397649in}}{\pgfqpoint{5.462879in}{2.387050in}}{\pgfqpoint{5.462879in}{2.376000in}}%
\pgfpathcurveto{\pgfqpoint{5.462879in}{2.364950in}}{\pgfqpoint{5.467269in}{2.354351in}}{\pgfqpoint{5.475083in}{2.346537in}}%
\pgfpathcurveto{\pgfqpoint{5.482896in}{2.338724in}}{\pgfqpoint{5.493495in}{2.334333in}}{\pgfqpoint{5.504545in}{2.334333in}}%
\pgfpathclose%
\pgfusepath{stroke,fill}%
\end{pgfscope}%
\begin{pgfscope}%
\pgfpathrectangle{\pgfqpoint{0.800000in}{0.528000in}}{\pgfqpoint{4.960000in}{3.696000in}}%
\pgfusepath{clip}%
\pgfsetbuttcap%
\pgfsetroundjoin%
\definecolor{currentfill}{rgb}{0.000000,0.000000,0.000000}%
\pgfsetfillcolor{currentfill}%
\pgfsetlinewidth{1.003750pt}%
\definecolor{currentstroke}{rgb}{0.000000,0.000000,0.000000}%
\pgfsetstrokecolor{currentstroke}%
\pgfsetdash{}{0pt}%
\pgfpathmoveto{\pgfqpoint{5.504545in}{2.334333in}}%
\pgfpathcurveto{\pgfqpoint{5.515596in}{2.334333in}}{\pgfqpoint{5.526195in}{2.338724in}}{\pgfqpoint{5.534008in}{2.346537in}}%
\pgfpathcurveto{\pgfqpoint{5.541822in}{2.354351in}}{\pgfqpoint{5.546212in}{2.364950in}}{\pgfqpoint{5.546212in}{2.376000in}}%
\pgfpathcurveto{\pgfqpoint{5.546212in}{2.387050in}}{\pgfqpoint{5.541822in}{2.397649in}}{\pgfqpoint{5.534008in}{2.405463in}}%
\pgfpathcurveto{\pgfqpoint{5.526195in}{2.413276in}}{\pgfqpoint{5.515596in}{2.417667in}}{\pgfqpoint{5.504545in}{2.417667in}}%
\pgfpathcurveto{\pgfqpoint{5.493495in}{2.417667in}}{\pgfqpoint{5.482896in}{2.413276in}}{\pgfqpoint{5.475083in}{2.405463in}}%
\pgfpathcurveto{\pgfqpoint{5.467269in}{2.397649in}}{\pgfqpoint{5.462879in}{2.387050in}}{\pgfqpoint{5.462879in}{2.376000in}}%
\pgfpathcurveto{\pgfqpoint{5.462879in}{2.364950in}}{\pgfqpoint{5.467269in}{2.354351in}}{\pgfqpoint{5.475083in}{2.346537in}}%
\pgfpathcurveto{\pgfqpoint{5.482896in}{2.338724in}}{\pgfqpoint{5.493495in}{2.334333in}}{\pgfqpoint{5.504545in}{2.334333in}}%
\pgfpathclose%
\pgfusepath{stroke,fill}%
\end{pgfscope}%
\begin{pgfscope}%
\pgfpathrectangle{\pgfqpoint{0.800000in}{0.528000in}}{\pgfqpoint{4.960000in}{3.696000in}}%
\pgfusepath{clip}%
\pgfsetbuttcap%
\pgfsetroundjoin%
\definecolor{currentfill}{rgb}{0.000000,0.000000,0.000000}%
\pgfsetfillcolor{currentfill}%
\pgfsetlinewidth{1.003750pt}%
\definecolor{currentstroke}{rgb}{0.000000,0.000000,0.000000}%
\pgfsetstrokecolor{currentstroke}%
\pgfsetdash{}{0pt}%
\pgfpathmoveto{\pgfqpoint{5.504545in}{2.334333in}}%
\pgfpathcurveto{\pgfqpoint{5.515596in}{2.334333in}}{\pgfqpoint{5.526195in}{2.338724in}}{\pgfqpoint{5.534008in}{2.346537in}}%
\pgfpathcurveto{\pgfqpoint{5.541822in}{2.354351in}}{\pgfqpoint{5.546212in}{2.364950in}}{\pgfqpoint{5.546212in}{2.376000in}}%
\pgfpathcurveto{\pgfqpoint{5.546212in}{2.387050in}}{\pgfqpoint{5.541822in}{2.397649in}}{\pgfqpoint{5.534008in}{2.405463in}}%
\pgfpathcurveto{\pgfqpoint{5.526195in}{2.413276in}}{\pgfqpoint{5.515596in}{2.417667in}}{\pgfqpoint{5.504545in}{2.417667in}}%
\pgfpathcurveto{\pgfqpoint{5.493495in}{2.417667in}}{\pgfqpoint{5.482896in}{2.413276in}}{\pgfqpoint{5.475083in}{2.405463in}}%
\pgfpathcurveto{\pgfqpoint{5.467269in}{2.397649in}}{\pgfqpoint{5.462879in}{2.387050in}}{\pgfqpoint{5.462879in}{2.376000in}}%
\pgfpathcurveto{\pgfqpoint{5.462879in}{2.364950in}}{\pgfqpoint{5.467269in}{2.354351in}}{\pgfqpoint{5.475083in}{2.346537in}}%
\pgfpathcurveto{\pgfqpoint{5.482896in}{2.338724in}}{\pgfqpoint{5.493495in}{2.334333in}}{\pgfqpoint{5.504545in}{2.334333in}}%
\pgfpathclose%
\pgfusepath{stroke,fill}%
\end{pgfscope}%
\begin{pgfscope}%
\pgfpathrectangle{\pgfqpoint{0.800000in}{0.528000in}}{\pgfqpoint{4.960000in}{3.696000in}}%
\pgfusepath{clip}%
\pgfsetbuttcap%
\pgfsetroundjoin%
\definecolor{currentfill}{rgb}{0.000000,0.000000,0.000000}%
\pgfsetfillcolor{currentfill}%
\pgfsetlinewidth{1.003750pt}%
\definecolor{currentstroke}{rgb}{0.000000,0.000000,0.000000}%
\pgfsetstrokecolor{currentstroke}%
\pgfsetdash{}{0pt}%
\pgfpathmoveto{\pgfqpoint{5.504545in}{2.334333in}}%
\pgfpathcurveto{\pgfqpoint{5.515596in}{2.334333in}}{\pgfqpoint{5.526195in}{2.338724in}}{\pgfqpoint{5.534008in}{2.346537in}}%
\pgfpathcurveto{\pgfqpoint{5.541822in}{2.354351in}}{\pgfqpoint{5.546212in}{2.364950in}}{\pgfqpoint{5.546212in}{2.376000in}}%
\pgfpathcurveto{\pgfqpoint{5.546212in}{2.387050in}}{\pgfqpoint{5.541822in}{2.397649in}}{\pgfqpoint{5.534008in}{2.405463in}}%
\pgfpathcurveto{\pgfqpoint{5.526195in}{2.413276in}}{\pgfqpoint{5.515596in}{2.417667in}}{\pgfqpoint{5.504545in}{2.417667in}}%
\pgfpathcurveto{\pgfqpoint{5.493495in}{2.417667in}}{\pgfqpoint{5.482896in}{2.413276in}}{\pgfqpoint{5.475083in}{2.405463in}}%
\pgfpathcurveto{\pgfqpoint{5.467269in}{2.397649in}}{\pgfqpoint{5.462879in}{2.387050in}}{\pgfqpoint{5.462879in}{2.376000in}}%
\pgfpathcurveto{\pgfqpoint{5.462879in}{2.364950in}}{\pgfqpoint{5.467269in}{2.354351in}}{\pgfqpoint{5.475083in}{2.346537in}}%
\pgfpathcurveto{\pgfqpoint{5.482896in}{2.338724in}}{\pgfqpoint{5.493495in}{2.334333in}}{\pgfqpoint{5.504545in}{2.334333in}}%
\pgfpathclose%
\pgfusepath{stroke,fill}%
\end{pgfscope}%
\begin{pgfscope}%
\pgfpathrectangle{\pgfqpoint{0.800000in}{0.528000in}}{\pgfqpoint{4.960000in}{3.696000in}}%
\pgfusepath{clip}%
\pgfsetbuttcap%
\pgfsetroundjoin%
\definecolor{currentfill}{rgb}{0.000000,0.000000,0.000000}%
\pgfsetfillcolor{currentfill}%
\pgfsetlinewidth{1.003750pt}%
\definecolor{currentstroke}{rgb}{0.000000,0.000000,0.000000}%
\pgfsetstrokecolor{currentstroke}%
\pgfsetdash{}{0pt}%
\pgfpathmoveto{\pgfqpoint{5.504545in}{2.334333in}}%
\pgfpathcurveto{\pgfqpoint{5.515596in}{2.334333in}}{\pgfqpoint{5.526195in}{2.338724in}}{\pgfqpoint{5.534008in}{2.346537in}}%
\pgfpathcurveto{\pgfqpoint{5.541822in}{2.354351in}}{\pgfqpoint{5.546212in}{2.364950in}}{\pgfqpoint{5.546212in}{2.376000in}}%
\pgfpathcurveto{\pgfqpoint{5.546212in}{2.387050in}}{\pgfqpoint{5.541822in}{2.397649in}}{\pgfqpoint{5.534008in}{2.405463in}}%
\pgfpathcurveto{\pgfqpoint{5.526195in}{2.413276in}}{\pgfqpoint{5.515596in}{2.417667in}}{\pgfqpoint{5.504545in}{2.417667in}}%
\pgfpathcurveto{\pgfqpoint{5.493495in}{2.417667in}}{\pgfqpoint{5.482896in}{2.413276in}}{\pgfqpoint{5.475083in}{2.405463in}}%
\pgfpathcurveto{\pgfqpoint{5.467269in}{2.397649in}}{\pgfqpoint{5.462879in}{2.387050in}}{\pgfqpoint{5.462879in}{2.376000in}}%
\pgfpathcurveto{\pgfqpoint{5.462879in}{2.364950in}}{\pgfqpoint{5.467269in}{2.354351in}}{\pgfqpoint{5.475083in}{2.346537in}}%
\pgfpathcurveto{\pgfqpoint{5.482896in}{2.338724in}}{\pgfqpoint{5.493495in}{2.334333in}}{\pgfqpoint{5.504545in}{2.334333in}}%
\pgfpathclose%
\pgfusepath{stroke,fill}%
\end{pgfscope}%
\begin{pgfscope}%
\pgfpathrectangle{\pgfqpoint{0.800000in}{0.528000in}}{\pgfqpoint{4.960000in}{3.696000in}}%
\pgfusepath{clip}%
\pgfsetbuttcap%
\pgfsetroundjoin%
\definecolor{currentfill}{rgb}{0.000000,0.000000,0.000000}%
\pgfsetfillcolor{currentfill}%
\pgfsetlinewidth{1.003750pt}%
\definecolor{currentstroke}{rgb}{0.000000,0.000000,0.000000}%
\pgfsetstrokecolor{currentstroke}%
\pgfsetdash{}{0pt}%
\pgfpathmoveto{\pgfqpoint{5.504545in}{2.334333in}}%
\pgfpathcurveto{\pgfqpoint{5.515596in}{2.334333in}}{\pgfqpoint{5.526195in}{2.338724in}}{\pgfqpoint{5.534008in}{2.346537in}}%
\pgfpathcurveto{\pgfqpoint{5.541822in}{2.354351in}}{\pgfqpoint{5.546212in}{2.364950in}}{\pgfqpoint{5.546212in}{2.376000in}}%
\pgfpathcurveto{\pgfqpoint{5.546212in}{2.387050in}}{\pgfqpoint{5.541822in}{2.397649in}}{\pgfqpoint{5.534008in}{2.405463in}}%
\pgfpathcurveto{\pgfqpoint{5.526195in}{2.413276in}}{\pgfqpoint{5.515596in}{2.417667in}}{\pgfqpoint{5.504545in}{2.417667in}}%
\pgfpathcurveto{\pgfqpoint{5.493495in}{2.417667in}}{\pgfqpoint{5.482896in}{2.413276in}}{\pgfqpoint{5.475083in}{2.405463in}}%
\pgfpathcurveto{\pgfqpoint{5.467269in}{2.397649in}}{\pgfqpoint{5.462879in}{2.387050in}}{\pgfqpoint{5.462879in}{2.376000in}}%
\pgfpathcurveto{\pgfqpoint{5.462879in}{2.364950in}}{\pgfqpoint{5.467269in}{2.354351in}}{\pgfqpoint{5.475083in}{2.346537in}}%
\pgfpathcurveto{\pgfqpoint{5.482896in}{2.338724in}}{\pgfqpoint{5.493495in}{2.334333in}}{\pgfqpoint{5.504545in}{2.334333in}}%
\pgfpathclose%
\pgfusepath{stroke,fill}%
\end{pgfscope}%
\begin{pgfscope}%
\pgfpathrectangle{\pgfqpoint{0.800000in}{0.528000in}}{\pgfqpoint{4.960000in}{3.696000in}}%
\pgfusepath{clip}%
\pgfsetbuttcap%
\pgfsetroundjoin%
\definecolor{currentfill}{rgb}{0.000000,0.000000,0.000000}%
\pgfsetfillcolor{currentfill}%
\pgfsetlinewidth{1.003750pt}%
\definecolor{currentstroke}{rgb}{0.000000,0.000000,0.000000}%
\pgfsetstrokecolor{currentstroke}%
\pgfsetdash{}{0pt}%
\pgfpathmoveto{\pgfqpoint{5.504545in}{2.334333in}}%
\pgfpathcurveto{\pgfqpoint{5.515596in}{2.334333in}}{\pgfqpoint{5.526195in}{2.338724in}}{\pgfqpoint{5.534008in}{2.346537in}}%
\pgfpathcurveto{\pgfqpoint{5.541822in}{2.354351in}}{\pgfqpoint{5.546212in}{2.364950in}}{\pgfqpoint{5.546212in}{2.376000in}}%
\pgfpathcurveto{\pgfqpoint{5.546212in}{2.387050in}}{\pgfqpoint{5.541822in}{2.397649in}}{\pgfqpoint{5.534008in}{2.405463in}}%
\pgfpathcurveto{\pgfqpoint{5.526195in}{2.413276in}}{\pgfqpoint{5.515596in}{2.417667in}}{\pgfqpoint{5.504545in}{2.417667in}}%
\pgfpathcurveto{\pgfqpoint{5.493495in}{2.417667in}}{\pgfqpoint{5.482896in}{2.413276in}}{\pgfqpoint{5.475083in}{2.405463in}}%
\pgfpathcurveto{\pgfqpoint{5.467269in}{2.397649in}}{\pgfqpoint{5.462879in}{2.387050in}}{\pgfqpoint{5.462879in}{2.376000in}}%
\pgfpathcurveto{\pgfqpoint{5.462879in}{2.364950in}}{\pgfqpoint{5.467269in}{2.354351in}}{\pgfqpoint{5.475083in}{2.346537in}}%
\pgfpathcurveto{\pgfqpoint{5.482896in}{2.338724in}}{\pgfqpoint{5.493495in}{2.334333in}}{\pgfqpoint{5.504545in}{2.334333in}}%
\pgfpathclose%
\pgfusepath{stroke,fill}%
\end{pgfscope}%
\begin{pgfscope}%
\pgfpathrectangle{\pgfqpoint{0.800000in}{0.528000in}}{\pgfqpoint{4.960000in}{3.696000in}}%
\pgfusepath{clip}%
\pgfsetbuttcap%
\pgfsetroundjoin%
\definecolor{currentfill}{rgb}{0.000000,0.000000,0.000000}%
\pgfsetfillcolor{currentfill}%
\pgfsetlinewidth{1.003750pt}%
\definecolor{currentstroke}{rgb}{0.000000,0.000000,0.000000}%
\pgfsetstrokecolor{currentstroke}%
\pgfsetdash{}{0pt}%
\pgfpathmoveto{\pgfqpoint{5.504545in}{2.334333in}}%
\pgfpathcurveto{\pgfqpoint{5.515596in}{2.334333in}}{\pgfqpoint{5.526195in}{2.338724in}}{\pgfqpoint{5.534008in}{2.346537in}}%
\pgfpathcurveto{\pgfqpoint{5.541822in}{2.354351in}}{\pgfqpoint{5.546212in}{2.364950in}}{\pgfqpoint{5.546212in}{2.376000in}}%
\pgfpathcurveto{\pgfqpoint{5.546212in}{2.387050in}}{\pgfqpoint{5.541822in}{2.397649in}}{\pgfqpoint{5.534008in}{2.405463in}}%
\pgfpathcurveto{\pgfqpoint{5.526195in}{2.413276in}}{\pgfqpoint{5.515596in}{2.417667in}}{\pgfqpoint{5.504545in}{2.417667in}}%
\pgfpathcurveto{\pgfqpoint{5.493495in}{2.417667in}}{\pgfqpoint{5.482896in}{2.413276in}}{\pgfqpoint{5.475083in}{2.405463in}}%
\pgfpathcurveto{\pgfqpoint{5.467269in}{2.397649in}}{\pgfqpoint{5.462879in}{2.387050in}}{\pgfqpoint{5.462879in}{2.376000in}}%
\pgfpathcurveto{\pgfqpoint{5.462879in}{2.364950in}}{\pgfqpoint{5.467269in}{2.354351in}}{\pgfqpoint{5.475083in}{2.346537in}}%
\pgfpathcurveto{\pgfqpoint{5.482896in}{2.338724in}}{\pgfqpoint{5.493495in}{2.334333in}}{\pgfqpoint{5.504545in}{2.334333in}}%
\pgfpathclose%
\pgfusepath{stroke,fill}%
\end{pgfscope}%
\begin{pgfscope}%
\pgfpathrectangle{\pgfqpoint{0.800000in}{0.528000in}}{\pgfqpoint{4.960000in}{3.696000in}}%
\pgfusepath{clip}%
\pgfsetbuttcap%
\pgfsetroundjoin%
\definecolor{currentfill}{rgb}{0.000000,0.000000,0.000000}%
\pgfsetfillcolor{currentfill}%
\pgfsetlinewidth{1.003750pt}%
\definecolor{currentstroke}{rgb}{0.000000,0.000000,0.000000}%
\pgfsetstrokecolor{currentstroke}%
\pgfsetdash{}{0pt}%
\pgfpathmoveto{\pgfqpoint{5.504545in}{2.334333in}}%
\pgfpathcurveto{\pgfqpoint{5.515596in}{2.334333in}}{\pgfqpoint{5.526195in}{2.338724in}}{\pgfqpoint{5.534008in}{2.346537in}}%
\pgfpathcurveto{\pgfqpoint{5.541822in}{2.354351in}}{\pgfqpoint{5.546212in}{2.364950in}}{\pgfqpoint{5.546212in}{2.376000in}}%
\pgfpathcurveto{\pgfqpoint{5.546212in}{2.387050in}}{\pgfqpoint{5.541822in}{2.397649in}}{\pgfqpoint{5.534008in}{2.405463in}}%
\pgfpathcurveto{\pgfqpoint{5.526195in}{2.413276in}}{\pgfqpoint{5.515596in}{2.417667in}}{\pgfqpoint{5.504545in}{2.417667in}}%
\pgfpathcurveto{\pgfqpoint{5.493495in}{2.417667in}}{\pgfqpoint{5.482896in}{2.413276in}}{\pgfqpoint{5.475083in}{2.405463in}}%
\pgfpathcurveto{\pgfqpoint{5.467269in}{2.397649in}}{\pgfqpoint{5.462879in}{2.387050in}}{\pgfqpoint{5.462879in}{2.376000in}}%
\pgfpathcurveto{\pgfqpoint{5.462879in}{2.364950in}}{\pgfqpoint{5.467269in}{2.354351in}}{\pgfqpoint{5.475083in}{2.346537in}}%
\pgfpathcurveto{\pgfqpoint{5.482896in}{2.338724in}}{\pgfqpoint{5.493495in}{2.334333in}}{\pgfqpoint{5.504545in}{2.334333in}}%
\pgfpathclose%
\pgfusepath{stroke,fill}%
\end{pgfscope}%
\begin{pgfscope}%
\pgfpathrectangle{\pgfqpoint{0.800000in}{0.528000in}}{\pgfqpoint{4.960000in}{3.696000in}}%
\pgfusepath{clip}%
\pgfsetbuttcap%
\pgfsetroundjoin%
\definecolor{currentfill}{rgb}{0.000000,0.000000,0.000000}%
\pgfsetfillcolor{currentfill}%
\pgfsetlinewidth{1.003750pt}%
\definecolor{currentstroke}{rgb}{0.000000,0.000000,0.000000}%
\pgfsetstrokecolor{currentstroke}%
\pgfsetdash{}{0pt}%
\pgfpathmoveto{\pgfqpoint{5.504545in}{2.334333in}}%
\pgfpathcurveto{\pgfqpoint{5.515596in}{2.334333in}}{\pgfqpoint{5.526195in}{2.338724in}}{\pgfqpoint{5.534008in}{2.346537in}}%
\pgfpathcurveto{\pgfqpoint{5.541822in}{2.354351in}}{\pgfqpoint{5.546212in}{2.364950in}}{\pgfqpoint{5.546212in}{2.376000in}}%
\pgfpathcurveto{\pgfqpoint{5.546212in}{2.387050in}}{\pgfqpoint{5.541822in}{2.397649in}}{\pgfqpoint{5.534008in}{2.405463in}}%
\pgfpathcurveto{\pgfqpoint{5.526195in}{2.413276in}}{\pgfqpoint{5.515596in}{2.417667in}}{\pgfqpoint{5.504545in}{2.417667in}}%
\pgfpathcurveto{\pgfqpoint{5.493495in}{2.417667in}}{\pgfqpoint{5.482896in}{2.413276in}}{\pgfqpoint{5.475083in}{2.405463in}}%
\pgfpathcurveto{\pgfqpoint{5.467269in}{2.397649in}}{\pgfqpoint{5.462879in}{2.387050in}}{\pgfqpoint{5.462879in}{2.376000in}}%
\pgfpathcurveto{\pgfqpoint{5.462879in}{2.364950in}}{\pgfqpoint{5.467269in}{2.354351in}}{\pgfqpoint{5.475083in}{2.346537in}}%
\pgfpathcurveto{\pgfqpoint{5.482896in}{2.338724in}}{\pgfqpoint{5.493495in}{2.334333in}}{\pgfqpoint{5.504545in}{2.334333in}}%
\pgfpathclose%
\pgfusepath{stroke,fill}%
\end{pgfscope}%
\begin{pgfscope}%
\pgfpathrectangle{\pgfqpoint{0.800000in}{0.528000in}}{\pgfqpoint{4.960000in}{3.696000in}}%
\pgfusepath{clip}%
\pgfsetbuttcap%
\pgfsetroundjoin%
\definecolor{currentfill}{rgb}{0.000000,0.000000,0.000000}%
\pgfsetfillcolor{currentfill}%
\pgfsetlinewidth{1.003750pt}%
\definecolor{currentstroke}{rgb}{0.000000,0.000000,0.000000}%
\pgfsetstrokecolor{currentstroke}%
\pgfsetdash{}{0pt}%
\pgfpathmoveto{\pgfqpoint{5.504545in}{2.334333in}}%
\pgfpathcurveto{\pgfqpoint{5.515596in}{2.334333in}}{\pgfqpoint{5.526195in}{2.338724in}}{\pgfqpoint{5.534008in}{2.346537in}}%
\pgfpathcurveto{\pgfqpoint{5.541822in}{2.354351in}}{\pgfqpoint{5.546212in}{2.364950in}}{\pgfqpoint{5.546212in}{2.376000in}}%
\pgfpathcurveto{\pgfqpoint{5.546212in}{2.387050in}}{\pgfqpoint{5.541822in}{2.397649in}}{\pgfqpoint{5.534008in}{2.405463in}}%
\pgfpathcurveto{\pgfqpoint{5.526195in}{2.413276in}}{\pgfqpoint{5.515596in}{2.417667in}}{\pgfqpoint{5.504545in}{2.417667in}}%
\pgfpathcurveto{\pgfqpoint{5.493495in}{2.417667in}}{\pgfqpoint{5.482896in}{2.413276in}}{\pgfqpoint{5.475083in}{2.405463in}}%
\pgfpathcurveto{\pgfqpoint{5.467269in}{2.397649in}}{\pgfqpoint{5.462879in}{2.387050in}}{\pgfqpoint{5.462879in}{2.376000in}}%
\pgfpathcurveto{\pgfqpoint{5.462879in}{2.364950in}}{\pgfqpoint{5.467269in}{2.354351in}}{\pgfqpoint{5.475083in}{2.346537in}}%
\pgfpathcurveto{\pgfqpoint{5.482896in}{2.338724in}}{\pgfqpoint{5.493495in}{2.334333in}}{\pgfqpoint{5.504545in}{2.334333in}}%
\pgfpathclose%
\pgfusepath{stroke,fill}%
\end{pgfscope}%
\begin{pgfscope}%
\pgfpathrectangle{\pgfqpoint{0.800000in}{0.528000in}}{\pgfqpoint{4.960000in}{3.696000in}}%
\pgfusepath{clip}%
\pgfsetbuttcap%
\pgfsetroundjoin%
\definecolor{currentfill}{rgb}{0.000000,0.000000,0.000000}%
\pgfsetfillcolor{currentfill}%
\pgfsetlinewidth{1.003750pt}%
\definecolor{currentstroke}{rgb}{0.000000,0.000000,0.000000}%
\pgfsetstrokecolor{currentstroke}%
\pgfsetdash{}{0pt}%
\pgfpathmoveto{\pgfqpoint{5.504545in}{2.334333in}}%
\pgfpathcurveto{\pgfqpoint{5.515596in}{2.334333in}}{\pgfqpoint{5.526195in}{2.338724in}}{\pgfqpoint{5.534008in}{2.346537in}}%
\pgfpathcurveto{\pgfqpoint{5.541822in}{2.354351in}}{\pgfqpoint{5.546212in}{2.364950in}}{\pgfqpoint{5.546212in}{2.376000in}}%
\pgfpathcurveto{\pgfqpoint{5.546212in}{2.387050in}}{\pgfqpoint{5.541822in}{2.397649in}}{\pgfqpoint{5.534008in}{2.405463in}}%
\pgfpathcurveto{\pgfqpoint{5.526195in}{2.413276in}}{\pgfqpoint{5.515596in}{2.417667in}}{\pgfqpoint{5.504545in}{2.417667in}}%
\pgfpathcurveto{\pgfqpoint{5.493495in}{2.417667in}}{\pgfqpoint{5.482896in}{2.413276in}}{\pgfqpoint{5.475083in}{2.405463in}}%
\pgfpathcurveto{\pgfqpoint{5.467269in}{2.397649in}}{\pgfqpoint{5.462879in}{2.387050in}}{\pgfqpoint{5.462879in}{2.376000in}}%
\pgfpathcurveto{\pgfqpoint{5.462879in}{2.364950in}}{\pgfqpoint{5.467269in}{2.354351in}}{\pgfqpoint{5.475083in}{2.346537in}}%
\pgfpathcurveto{\pgfqpoint{5.482896in}{2.338724in}}{\pgfqpoint{5.493495in}{2.334333in}}{\pgfqpoint{5.504545in}{2.334333in}}%
\pgfpathclose%
\pgfusepath{stroke,fill}%
\end{pgfscope}%
\begin{pgfscope}%
\pgfpathrectangle{\pgfqpoint{0.800000in}{0.528000in}}{\pgfqpoint{4.960000in}{3.696000in}}%
\pgfusepath{clip}%
\pgfsetbuttcap%
\pgfsetroundjoin%
\definecolor{currentfill}{rgb}{0.000000,0.000000,0.000000}%
\pgfsetfillcolor{currentfill}%
\pgfsetlinewidth{1.003750pt}%
\definecolor{currentstroke}{rgb}{0.000000,0.000000,0.000000}%
\pgfsetstrokecolor{currentstroke}%
\pgfsetdash{}{0pt}%
\pgfpathmoveto{\pgfqpoint{5.504545in}{2.334333in}}%
\pgfpathcurveto{\pgfqpoint{5.515596in}{2.334333in}}{\pgfqpoint{5.526195in}{2.338724in}}{\pgfqpoint{5.534008in}{2.346537in}}%
\pgfpathcurveto{\pgfqpoint{5.541822in}{2.354351in}}{\pgfqpoint{5.546212in}{2.364950in}}{\pgfqpoint{5.546212in}{2.376000in}}%
\pgfpathcurveto{\pgfqpoint{5.546212in}{2.387050in}}{\pgfqpoint{5.541822in}{2.397649in}}{\pgfqpoint{5.534008in}{2.405463in}}%
\pgfpathcurveto{\pgfqpoint{5.526195in}{2.413276in}}{\pgfqpoint{5.515596in}{2.417667in}}{\pgfqpoint{5.504545in}{2.417667in}}%
\pgfpathcurveto{\pgfqpoint{5.493495in}{2.417667in}}{\pgfqpoint{5.482896in}{2.413276in}}{\pgfqpoint{5.475083in}{2.405463in}}%
\pgfpathcurveto{\pgfqpoint{5.467269in}{2.397649in}}{\pgfqpoint{5.462879in}{2.387050in}}{\pgfqpoint{5.462879in}{2.376000in}}%
\pgfpathcurveto{\pgfqpoint{5.462879in}{2.364950in}}{\pgfqpoint{5.467269in}{2.354351in}}{\pgfqpoint{5.475083in}{2.346537in}}%
\pgfpathcurveto{\pgfqpoint{5.482896in}{2.338724in}}{\pgfqpoint{5.493495in}{2.334333in}}{\pgfqpoint{5.504545in}{2.334333in}}%
\pgfpathclose%
\pgfusepath{stroke,fill}%
\end{pgfscope}%
\begin{pgfscope}%
\pgfpathrectangle{\pgfqpoint{0.800000in}{0.528000in}}{\pgfqpoint{4.960000in}{3.696000in}}%
\pgfusepath{clip}%
\pgfsetbuttcap%
\pgfsetroundjoin%
\definecolor{currentfill}{rgb}{0.000000,0.000000,0.000000}%
\pgfsetfillcolor{currentfill}%
\pgfsetlinewidth{1.003750pt}%
\definecolor{currentstroke}{rgb}{0.000000,0.000000,0.000000}%
\pgfsetstrokecolor{currentstroke}%
\pgfsetdash{}{0pt}%
\pgfpathmoveto{\pgfqpoint{5.504545in}{2.334333in}}%
\pgfpathcurveto{\pgfqpoint{5.515596in}{2.334333in}}{\pgfqpoint{5.526195in}{2.338724in}}{\pgfqpoint{5.534008in}{2.346537in}}%
\pgfpathcurveto{\pgfqpoint{5.541822in}{2.354351in}}{\pgfqpoint{5.546212in}{2.364950in}}{\pgfqpoint{5.546212in}{2.376000in}}%
\pgfpathcurveto{\pgfqpoint{5.546212in}{2.387050in}}{\pgfqpoint{5.541822in}{2.397649in}}{\pgfqpoint{5.534008in}{2.405463in}}%
\pgfpathcurveto{\pgfqpoint{5.526195in}{2.413276in}}{\pgfqpoint{5.515596in}{2.417667in}}{\pgfqpoint{5.504545in}{2.417667in}}%
\pgfpathcurveto{\pgfqpoint{5.493495in}{2.417667in}}{\pgfqpoint{5.482896in}{2.413276in}}{\pgfqpoint{5.475083in}{2.405463in}}%
\pgfpathcurveto{\pgfqpoint{5.467269in}{2.397649in}}{\pgfqpoint{5.462879in}{2.387050in}}{\pgfqpoint{5.462879in}{2.376000in}}%
\pgfpathcurveto{\pgfqpoint{5.462879in}{2.364950in}}{\pgfqpoint{5.467269in}{2.354351in}}{\pgfqpoint{5.475083in}{2.346537in}}%
\pgfpathcurveto{\pgfqpoint{5.482896in}{2.338724in}}{\pgfqpoint{5.493495in}{2.334333in}}{\pgfqpoint{5.504545in}{2.334333in}}%
\pgfpathclose%
\pgfusepath{stroke,fill}%
\end{pgfscope}%
\begin{pgfscope}%
\pgfpathrectangle{\pgfqpoint{0.800000in}{0.528000in}}{\pgfqpoint{4.960000in}{3.696000in}}%
\pgfusepath{clip}%
\pgfsetbuttcap%
\pgfsetroundjoin%
\definecolor{currentfill}{rgb}{0.000000,0.000000,0.000000}%
\pgfsetfillcolor{currentfill}%
\pgfsetlinewidth{1.003750pt}%
\definecolor{currentstroke}{rgb}{0.000000,0.000000,0.000000}%
\pgfsetstrokecolor{currentstroke}%
\pgfsetdash{}{0pt}%
\pgfpathmoveto{\pgfqpoint{5.504545in}{2.334333in}}%
\pgfpathcurveto{\pgfqpoint{5.515596in}{2.334333in}}{\pgfqpoint{5.526195in}{2.338724in}}{\pgfqpoint{5.534008in}{2.346537in}}%
\pgfpathcurveto{\pgfqpoint{5.541822in}{2.354351in}}{\pgfqpoint{5.546212in}{2.364950in}}{\pgfqpoint{5.546212in}{2.376000in}}%
\pgfpathcurveto{\pgfqpoint{5.546212in}{2.387050in}}{\pgfqpoint{5.541822in}{2.397649in}}{\pgfqpoint{5.534008in}{2.405463in}}%
\pgfpathcurveto{\pgfqpoint{5.526195in}{2.413276in}}{\pgfqpoint{5.515596in}{2.417667in}}{\pgfqpoint{5.504545in}{2.417667in}}%
\pgfpathcurveto{\pgfqpoint{5.493495in}{2.417667in}}{\pgfqpoint{5.482896in}{2.413276in}}{\pgfqpoint{5.475083in}{2.405463in}}%
\pgfpathcurveto{\pgfqpoint{5.467269in}{2.397649in}}{\pgfqpoint{5.462879in}{2.387050in}}{\pgfqpoint{5.462879in}{2.376000in}}%
\pgfpathcurveto{\pgfqpoint{5.462879in}{2.364950in}}{\pgfqpoint{5.467269in}{2.354351in}}{\pgfqpoint{5.475083in}{2.346537in}}%
\pgfpathcurveto{\pgfqpoint{5.482896in}{2.338724in}}{\pgfqpoint{5.493495in}{2.334333in}}{\pgfqpoint{5.504545in}{2.334333in}}%
\pgfpathclose%
\pgfusepath{stroke,fill}%
\end{pgfscope}%
\begin{pgfscope}%
\pgfpathrectangle{\pgfqpoint{0.800000in}{0.528000in}}{\pgfqpoint{4.960000in}{3.696000in}}%
\pgfusepath{clip}%
\pgfsetbuttcap%
\pgfsetroundjoin%
\definecolor{currentfill}{rgb}{0.000000,0.000000,0.000000}%
\pgfsetfillcolor{currentfill}%
\pgfsetlinewidth{1.003750pt}%
\definecolor{currentstroke}{rgb}{0.000000,0.000000,0.000000}%
\pgfsetstrokecolor{currentstroke}%
\pgfsetdash{}{0pt}%
\pgfpathmoveto{\pgfqpoint{5.504545in}{2.334333in}}%
\pgfpathcurveto{\pgfqpoint{5.515596in}{2.334333in}}{\pgfqpoint{5.526195in}{2.338724in}}{\pgfqpoint{5.534008in}{2.346537in}}%
\pgfpathcurveto{\pgfqpoint{5.541822in}{2.354351in}}{\pgfqpoint{5.546212in}{2.364950in}}{\pgfqpoint{5.546212in}{2.376000in}}%
\pgfpathcurveto{\pgfqpoint{5.546212in}{2.387050in}}{\pgfqpoint{5.541822in}{2.397649in}}{\pgfqpoint{5.534008in}{2.405463in}}%
\pgfpathcurveto{\pgfqpoint{5.526195in}{2.413276in}}{\pgfqpoint{5.515596in}{2.417667in}}{\pgfqpoint{5.504545in}{2.417667in}}%
\pgfpathcurveto{\pgfqpoint{5.493495in}{2.417667in}}{\pgfqpoint{5.482896in}{2.413276in}}{\pgfqpoint{5.475083in}{2.405463in}}%
\pgfpathcurveto{\pgfqpoint{5.467269in}{2.397649in}}{\pgfqpoint{5.462879in}{2.387050in}}{\pgfqpoint{5.462879in}{2.376000in}}%
\pgfpathcurveto{\pgfqpoint{5.462879in}{2.364950in}}{\pgfqpoint{5.467269in}{2.354351in}}{\pgfqpoint{5.475083in}{2.346537in}}%
\pgfpathcurveto{\pgfqpoint{5.482896in}{2.338724in}}{\pgfqpoint{5.493495in}{2.334333in}}{\pgfqpoint{5.504545in}{2.334333in}}%
\pgfpathclose%
\pgfusepath{stroke,fill}%
\end{pgfscope}%
\begin{pgfscope}%
\pgfpathrectangle{\pgfqpoint{0.800000in}{0.528000in}}{\pgfqpoint{4.960000in}{3.696000in}}%
\pgfusepath{clip}%
\pgfsetbuttcap%
\pgfsetroundjoin%
\definecolor{currentfill}{rgb}{0.000000,0.000000,0.000000}%
\pgfsetfillcolor{currentfill}%
\pgfsetlinewidth{1.003750pt}%
\definecolor{currentstroke}{rgb}{0.000000,0.000000,0.000000}%
\pgfsetstrokecolor{currentstroke}%
\pgfsetdash{}{0pt}%
\pgfpathmoveto{\pgfqpoint{5.504545in}{2.334333in}}%
\pgfpathcurveto{\pgfqpoint{5.515596in}{2.334333in}}{\pgfqpoint{5.526195in}{2.338724in}}{\pgfqpoint{5.534008in}{2.346537in}}%
\pgfpathcurveto{\pgfqpoint{5.541822in}{2.354351in}}{\pgfqpoint{5.546212in}{2.364950in}}{\pgfqpoint{5.546212in}{2.376000in}}%
\pgfpathcurveto{\pgfqpoint{5.546212in}{2.387050in}}{\pgfqpoint{5.541822in}{2.397649in}}{\pgfqpoint{5.534008in}{2.405463in}}%
\pgfpathcurveto{\pgfqpoint{5.526195in}{2.413276in}}{\pgfqpoint{5.515596in}{2.417667in}}{\pgfqpoint{5.504545in}{2.417667in}}%
\pgfpathcurveto{\pgfqpoint{5.493495in}{2.417667in}}{\pgfqpoint{5.482896in}{2.413276in}}{\pgfqpoint{5.475083in}{2.405463in}}%
\pgfpathcurveto{\pgfqpoint{5.467269in}{2.397649in}}{\pgfqpoint{5.462879in}{2.387050in}}{\pgfqpoint{5.462879in}{2.376000in}}%
\pgfpathcurveto{\pgfqpoint{5.462879in}{2.364950in}}{\pgfqpoint{5.467269in}{2.354351in}}{\pgfqpoint{5.475083in}{2.346537in}}%
\pgfpathcurveto{\pgfqpoint{5.482896in}{2.338724in}}{\pgfqpoint{5.493495in}{2.334333in}}{\pgfqpoint{5.504545in}{2.334333in}}%
\pgfpathclose%
\pgfusepath{stroke,fill}%
\end{pgfscope}%
\begin{pgfscope}%
\pgfpathrectangle{\pgfqpoint{0.800000in}{0.528000in}}{\pgfqpoint{4.960000in}{3.696000in}}%
\pgfusepath{clip}%
\pgfsetbuttcap%
\pgfsetroundjoin%
\definecolor{currentfill}{rgb}{0.000000,0.000000,0.000000}%
\pgfsetfillcolor{currentfill}%
\pgfsetlinewidth{1.003750pt}%
\definecolor{currentstroke}{rgb}{0.000000,0.000000,0.000000}%
\pgfsetstrokecolor{currentstroke}%
\pgfsetdash{}{0pt}%
\pgfpathmoveto{\pgfqpoint{5.504545in}{2.334333in}}%
\pgfpathcurveto{\pgfqpoint{5.515596in}{2.334333in}}{\pgfqpoint{5.526195in}{2.338724in}}{\pgfqpoint{5.534008in}{2.346537in}}%
\pgfpathcurveto{\pgfqpoint{5.541822in}{2.354351in}}{\pgfqpoint{5.546212in}{2.364950in}}{\pgfqpoint{5.546212in}{2.376000in}}%
\pgfpathcurveto{\pgfqpoint{5.546212in}{2.387050in}}{\pgfqpoint{5.541822in}{2.397649in}}{\pgfqpoint{5.534008in}{2.405463in}}%
\pgfpathcurveto{\pgfqpoint{5.526195in}{2.413276in}}{\pgfqpoint{5.515596in}{2.417667in}}{\pgfqpoint{5.504545in}{2.417667in}}%
\pgfpathcurveto{\pgfqpoint{5.493495in}{2.417667in}}{\pgfqpoint{5.482896in}{2.413276in}}{\pgfqpoint{5.475083in}{2.405463in}}%
\pgfpathcurveto{\pgfqpoint{5.467269in}{2.397649in}}{\pgfqpoint{5.462879in}{2.387050in}}{\pgfqpoint{5.462879in}{2.376000in}}%
\pgfpathcurveto{\pgfqpoint{5.462879in}{2.364950in}}{\pgfqpoint{5.467269in}{2.354351in}}{\pgfqpoint{5.475083in}{2.346537in}}%
\pgfpathcurveto{\pgfqpoint{5.482896in}{2.338724in}}{\pgfqpoint{5.493495in}{2.334333in}}{\pgfqpoint{5.504545in}{2.334333in}}%
\pgfpathclose%
\pgfusepath{stroke,fill}%
\end{pgfscope}%
\begin{pgfscope}%
\pgfpathrectangle{\pgfqpoint{0.800000in}{0.528000in}}{\pgfqpoint{4.960000in}{3.696000in}}%
\pgfusepath{clip}%
\pgfsetbuttcap%
\pgfsetroundjoin%
\definecolor{currentfill}{rgb}{0.000000,0.000000,0.000000}%
\pgfsetfillcolor{currentfill}%
\pgfsetlinewidth{1.003750pt}%
\definecolor{currentstroke}{rgb}{0.000000,0.000000,0.000000}%
\pgfsetstrokecolor{currentstroke}%
\pgfsetdash{}{0pt}%
\pgfpathmoveto{\pgfqpoint{5.504545in}{2.334333in}}%
\pgfpathcurveto{\pgfqpoint{5.515596in}{2.334333in}}{\pgfqpoint{5.526195in}{2.338724in}}{\pgfqpoint{5.534008in}{2.346537in}}%
\pgfpathcurveto{\pgfqpoint{5.541822in}{2.354351in}}{\pgfqpoint{5.546212in}{2.364950in}}{\pgfqpoint{5.546212in}{2.376000in}}%
\pgfpathcurveto{\pgfqpoint{5.546212in}{2.387050in}}{\pgfqpoint{5.541822in}{2.397649in}}{\pgfqpoint{5.534008in}{2.405463in}}%
\pgfpathcurveto{\pgfqpoint{5.526195in}{2.413276in}}{\pgfqpoint{5.515596in}{2.417667in}}{\pgfqpoint{5.504545in}{2.417667in}}%
\pgfpathcurveto{\pgfqpoint{5.493495in}{2.417667in}}{\pgfqpoint{5.482896in}{2.413276in}}{\pgfqpoint{5.475083in}{2.405463in}}%
\pgfpathcurveto{\pgfqpoint{5.467269in}{2.397649in}}{\pgfqpoint{5.462879in}{2.387050in}}{\pgfqpoint{5.462879in}{2.376000in}}%
\pgfpathcurveto{\pgfqpoint{5.462879in}{2.364950in}}{\pgfqpoint{5.467269in}{2.354351in}}{\pgfqpoint{5.475083in}{2.346537in}}%
\pgfpathcurveto{\pgfqpoint{5.482896in}{2.338724in}}{\pgfqpoint{5.493495in}{2.334333in}}{\pgfqpoint{5.504545in}{2.334333in}}%
\pgfpathclose%
\pgfusepath{stroke,fill}%
\end{pgfscope}%
\begin{pgfscope}%
\pgfpathrectangle{\pgfqpoint{0.800000in}{0.528000in}}{\pgfqpoint{4.960000in}{3.696000in}}%
\pgfusepath{clip}%
\pgfsetbuttcap%
\pgfsetroundjoin%
\definecolor{currentfill}{rgb}{0.000000,0.000000,0.000000}%
\pgfsetfillcolor{currentfill}%
\pgfsetlinewidth{1.003750pt}%
\definecolor{currentstroke}{rgb}{0.000000,0.000000,0.000000}%
\pgfsetstrokecolor{currentstroke}%
\pgfsetdash{}{0pt}%
\pgfpathmoveto{\pgfqpoint{5.504545in}{2.334333in}}%
\pgfpathcurveto{\pgfqpoint{5.515596in}{2.334333in}}{\pgfqpoint{5.526195in}{2.338724in}}{\pgfqpoint{5.534008in}{2.346537in}}%
\pgfpathcurveto{\pgfqpoint{5.541822in}{2.354351in}}{\pgfqpoint{5.546212in}{2.364950in}}{\pgfqpoint{5.546212in}{2.376000in}}%
\pgfpathcurveto{\pgfqpoint{5.546212in}{2.387050in}}{\pgfqpoint{5.541822in}{2.397649in}}{\pgfqpoint{5.534008in}{2.405463in}}%
\pgfpathcurveto{\pgfqpoint{5.526195in}{2.413276in}}{\pgfqpoint{5.515596in}{2.417667in}}{\pgfqpoint{5.504545in}{2.417667in}}%
\pgfpathcurveto{\pgfqpoint{5.493495in}{2.417667in}}{\pgfqpoint{5.482896in}{2.413276in}}{\pgfqpoint{5.475083in}{2.405463in}}%
\pgfpathcurveto{\pgfqpoint{5.467269in}{2.397649in}}{\pgfqpoint{5.462879in}{2.387050in}}{\pgfqpoint{5.462879in}{2.376000in}}%
\pgfpathcurveto{\pgfqpoint{5.462879in}{2.364950in}}{\pgfqpoint{5.467269in}{2.354351in}}{\pgfqpoint{5.475083in}{2.346537in}}%
\pgfpathcurveto{\pgfqpoint{5.482896in}{2.338724in}}{\pgfqpoint{5.493495in}{2.334333in}}{\pgfqpoint{5.504545in}{2.334333in}}%
\pgfpathclose%
\pgfusepath{stroke,fill}%
\end{pgfscope}%
\begin{pgfscope}%
\pgfpathrectangle{\pgfqpoint{0.800000in}{0.528000in}}{\pgfqpoint{4.960000in}{3.696000in}}%
\pgfusepath{clip}%
\pgfsetbuttcap%
\pgfsetroundjoin%
\definecolor{currentfill}{rgb}{0.000000,0.000000,0.000000}%
\pgfsetfillcolor{currentfill}%
\pgfsetlinewidth{1.003750pt}%
\definecolor{currentstroke}{rgb}{0.000000,0.000000,0.000000}%
\pgfsetstrokecolor{currentstroke}%
\pgfsetdash{}{0pt}%
\pgfpathmoveto{\pgfqpoint{5.504545in}{2.334333in}}%
\pgfpathcurveto{\pgfqpoint{5.515596in}{2.334333in}}{\pgfqpoint{5.526195in}{2.338724in}}{\pgfqpoint{5.534008in}{2.346537in}}%
\pgfpathcurveto{\pgfqpoint{5.541822in}{2.354351in}}{\pgfqpoint{5.546212in}{2.364950in}}{\pgfqpoint{5.546212in}{2.376000in}}%
\pgfpathcurveto{\pgfqpoint{5.546212in}{2.387050in}}{\pgfqpoint{5.541822in}{2.397649in}}{\pgfqpoint{5.534008in}{2.405463in}}%
\pgfpathcurveto{\pgfqpoint{5.526195in}{2.413276in}}{\pgfqpoint{5.515596in}{2.417667in}}{\pgfqpoint{5.504545in}{2.417667in}}%
\pgfpathcurveto{\pgfqpoint{5.493495in}{2.417667in}}{\pgfqpoint{5.482896in}{2.413276in}}{\pgfqpoint{5.475083in}{2.405463in}}%
\pgfpathcurveto{\pgfqpoint{5.467269in}{2.397649in}}{\pgfqpoint{5.462879in}{2.387050in}}{\pgfqpoint{5.462879in}{2.376000in}}%
\pgfpathcurveto{\pgfqpoint{5.462879in}{2.364950in}}{\pgfqpoint{5.467269in}{2.354351in}}{\pgfqpoint{5.475083in}{2.346537in}}%
\pgfpathcurveto{\pgfqpoint{5.482896in}{2.338724in}}{\pgfqpoint{5.493495in}{2.334333in}}{\pgfqpoint{5.504545in}{2.334333in}}%
\pgfpathclose%
\pgfusepath{stroke,fill}%
\end{pgfscope}%
\begin{pgfscope}%
\pgfpathrectangle{\pgfqpoint{0.800000in}{0.528000in}}{\pgfqpoint{4.960000in}{3.696000in}}%
\pgfusepath{clip}%
\pgfsetbuttcap%
\pgfsetroundjoin%
\definecolor{currentfill}{rgb}{0.000000,0.000000,0.000000}%
\pgfsetfillcolor{currentfill}%
\pgfsetlinewidth{1.003750pt}%
\definecolor{currentstroke}{rgb}{0.000000,0.000000,0.000000}%
\pgfsetstrokecolor{currentstroke}%
\pgfsetdash{}{0pt}%
\pgfpathmoveto{\pgfqpoint{5.504545in}{2.334333in}}%
\pgfpathcurveto{\pgfqpoint{5.515596in}{2.334333in}}{\pgfqpoint{5.526195in}{2.338724in}}{\pgfqpoint{5.534008in}{2.346537in}}%
\pgfpathcurveto{\pgfqpoint{5.541822in}{2.354351in}}{\pgfqpoint{5.546212in}{2.364950in}}{\pgfqpoint{5.546212in}{2.376000in}}%
\pgfpathcurveto{\pgfqpoint{5.546212in}{2.387050in}}{\pgfqpoint{5.541822in}{2.397649in}}{\pgfqpoint{5.534008in}{2.405463in}}%
\pgfpathcurveto{\pgfqpoint{5.526195in}{2.413276in}}{\pgfqpoint{5.515596in}{2.417667in}}{\pgfqpoint{5.504545in}{2.417667in}}%
\pgfpathcurveto{\pgfqpoint{5.493495in}{2.417667in}}{\pgfqpoint{5.482896in}{2.413276in}}{\pgfqpoint{5.475083in}{2.405463in}}%
\pgfpathcurveto{\pgfqpoint{5.467269in}{2.397649in}}{\pgfqpoint{5.462879in}{2.387050in}}{\pgfqpoint{5.462879in}{2.376000in}}%
\pgfpathcurveto{\pgfqpoint{5.462879in}{2.364950in}}{\pgfqpoint{5.467269in}{2.354351in}}{\pgfqpoint{5.475083in}{2.346537in}}%
\pgfpathcurveto{\pgfqpoint{5.482896in}{2.338724in}}{\pgfqpoint{5.493495in}{2.334333in}}{\pgfqpoint{5.504545in}{2.334333in}}%
\pgfpathclose%
\pgfusepath{stroke,fill}%
\end{pgfscope}%
\begin{pgfscope}%
\pgfpathrectangle{\pgfqpoint{0.800000in}{0.528000in}}{\pgfqpoint{4.960000in}{3.696000in}}%
\pgfusepath{clip}%
\pgfsetbuttcap%
\pgfsetroundjoin%
\definecolor{currentfill}{rgb}{0.000000,0.000000,0.000000}%
\pgfsetfillcolor{currentfill}%
\pgfsetlinewidth{1.003750pt}%
\definecolor{currentstroke}{rgb}{0.000000,0.000000,0.000000}%
\pgfsetstrokecolor{currentstroke}%
\pgfsetdash{}{0pt}%
\pgfpathmoveto{\pgfqpoint{5.504545in}{2.334333in}}%
\pgfpathcurveto{\pgfqpoint{5.515596in}{2.334333in}}{\pgfqpoint{5.526195in}{2.338724in}}{\pgfqpoint{5.534008in}{2.346537in}}%
\pgfpathcurveto{\pgfqpoint{5.541822in}{2.354351in}}{\pgfqpoint{5.546212in}{2.364950in}}{\pgfqpoint{5.546212in}{2.376000in}}%
\pgfpathcurveto{\pgfqpoint{5.546212in}{2.387050in}}{\pgfqpoint{5.541822in}{2.397649in}}{\pgfqpoint{5.534008in}{2.405463in}}%
\pgfpathcurveto{\pgfqpoint{5.526195in}{2.413276in}}{\pgfqpoint{5.515596in}{2.417667in}}{\pgfqpoint{5.504545in}{2.417667in}}%
\pgfpathcurveto{\pgfqpoint{5.493495in}{2.417667in}}{\pgfqpoint{5.482896in}{2.413276in}}{\pgfqpoint{5.475083in}{2.405463in}}%
\pgfpathcurveto{\pgfqpoint{5.467269in}{2.397649in}}{\pgfqpoint{5.462879in}{2.387050in}}{\pgfqpoint{5.462879in}{2.376000in}}%
\pgfpathcurveto{\pgfqpoint{5.462879in}{2.364950in}}{\pgfqpoint{5.467269in}{2.354351in}}{\pgfqpoint{5.475083in}{2.346537in}}%
\pgfpathcurveto{\pgfqpoint{5.482896in}{2.338724in}}{\pgfqpoint{5.493495in}{2.334333in}}{\pgfqpoint{5.504545in}{2.334333in}}%
\pgfpathclose%
\pgfusepath{stroke,fill}%
\end{pgfscope}%
\begin{pgfscope}%
\pgfpathrectangle{\pgfqpoint{0.800000in}{0.528000in}}{\pgfqpoint{4.960000in}{3.696000in}}%
\pgfusepath{clip}%
\pgfsetbuttcap%
\pgfsetroundjoin%
\definecolor{currentfill}{rgb}{0.000000,0.000000,0.000000}%
\pgfsetfillcolor{currentfill}%
\pgfsetlinewidth{1.003750pt}%
\definecolor{currentstroke}{rgb}{0.000000,0.000000,0.000000}%
\pgfsetstrokecolor{currentstroke}%
\pgfsetdash{}{0pt}%
\pgfpathmoveto{\pgfqpoint{5.504545in}{2.334333in}}%
\pgfpathcurveto{\pgfqpoint{5.515596in}{2.334333in}}{\pgfqpoint{5.526195in}{2.338724in}}{\pgfqpoint{5.534008in}{2.346537in}}%
\pgfpathcurveto{\pgfqpoint{5.541822in}{2.354351in}}{\pgfqpoint{5.546212in}{2.364950in}}{\pgfqpoint{5.546212in}{2.376000in}}%
\pgfpathcurveto{\pgfqpoint{5.546212in}{2.387050in}}{\pgfqpoint{5.541822in}{2.397649in}}{\pgfqpoint{5.534008in}{2.405463in}}%
\pgfpathcurveto{\pgfqpoint{5.526195in}{2.413276in}}{\pgfqpoint{5.515596in}{2.417667in}}{\pgfqpoint{5.504545in}{2.417667in}}%
\pgfpathcurveto{\pgfqpoint{5.493495in}{2.417667in}}{\pgfqpoint{5.482896in}{2.413276in}}{\pgfqpoint{5.475083in}{2.405463in}}%
\pgfpathcurveto{\pgfqpoint{5.467269in}{2.397649in}}{\pgfqpoint{5.462879in}{2.387050in}}{\pgfqpoint{5.462879in}{2.376000in}}%
\pgfpathcurveto{\pgfqpoint{5.462879in}{2.364950in}}{\pgfqpoint{5.467269in}{2.354351in}}{\pgfqpoint{5.475083in}{2.346537in}}%
\pgfpathcurveto{\pgfqpoint{5.482896in}{2.338724in}}{\pgfqpoint{5.493495in}{2.334333in}}{\pgfqpoint{5.504545in}{2.334333in}}%
\pgfpathclose%
\pgfusepath{stroke,fill}%
\end{pgfscope}%
\begin{pgfscope}%
\pgfpathrectangle{\pgfqpoint{0.800000in}{0.528000in}}{\pgfqpoint{4.960000in}{3.696000in}}%
\pgfusepath{clip}%
\pgfsetbuttcap%
\pgfsetroundjoin%
\definecolor{currentfill}{rgb}{0.000000,0.000000,0.000000}%
\pgfsetfillcolor{currentfill}%
\pgfsetlinewidth{1.003750pt}%
\definecolor{currentstroke}{rgb}{0.000000,0.000000,0.000000}%
\pgfsetstrokecolor{currentstroke}%
\pgfsetdash{}{0pt}%
\pgfpathmoveto{\pgfqpoint{5.504545in}{2.334333in}}%
\pgfpathcurveto{\pgfqpoint{5.515596in}{2.334333in}}{\pgfqpoint{5.526195in}{2.338724in}}{\pgfqpoint{5.534008in}{2.346537in}}%
\pgfpathcurveto{\pgfqpoint{5.541822in}{2.354351in}}{\pgfqpoint{5.546212in}{2.364950in}}{\pgfqpoint{5.546212in}{2.376000in}}%
\pgfpathcurveto{\pgfqpoint{5.546212in}{2.387050in}}{\pgfqpoint{5.541822in}{2.397649in}}{\pgfqpoint{5.534008in}{2.405463in}}%
\pgfpathcurveto{\pgfqpoint{5.526195in}{2.413276in}}{\pgfqpoint{5.515596in}{2.417667in}}{\pgfqpoint{5.504545in}{2.417667in}}%
\pgfpathcurveto{\pgfqpoint{5.493495in}{2.417667in}}{\pgfqpoint{5.482896in}{2.413276in}}{\pgfqpoint{5.475083in}{2.405463in}}%
\pgfpathcurveto{\pgfqpoint{5.467269in}{2.397649in}}{\pgfqpoint{5.462879in}{2.387050in}}{\pgfqpoint{5.462879in}{2.376000in}}%
\pgfpathcurveto{\pgfqpoint{5.462879in}{2.364950in}}{\pgfqpoint{5.467269in}{2.354351in}}{\pgfqpoint{5.475083in}{2.346537in}}%
\pgfpathcurveto{\pgfqpoint{5.482896in}{2.338724in}}{\pgfqpoint{5.493495in}{2.334333in}}{\pgfqpoint{5.504545in}{2.334333in}}%
\pgfpathclose%
\pgfusepath{stroke,fill}%
\end{pgfscope}%
\begin{pgfscope}%
\pgfpathrectangle{\pgfqpoint{0.800000in}{0.528000in}}{\pgfqpoint{4.960000in}{3.696000in}}%
\pgfusepath{clip}%
\pgfsetbuttcap%
\pgfsetroundjoin%
\definecolor{currentfill}{rgb}{0.000000,0.000000,0.000000}%
\pgfsetfillcolor{currentfill}%
\pgfsetlinewidth{1.003750pt}%
\definecolor{currentstroke}{rgb}{0.000000,0.000000,0.000000}%
\pgfsetstrokecolor{currentstroke}%
\pgfsetdash{}{0pt}%
\pgfpathmoveto{\pgfqpoint{5.504545in}{2.334333in}}%
\pgfpathcurveto{\pgfqpoint{5.515596in}{2.334333in}}{\pgfqpoint{5.526195in}{2.338724in}}{\pgfqpoint{5.534008in}{2.346537in}}%
\pgfpathcurveto{\pgfqpoint{5.541822in}{2.354351in}}{\pgfqpoint{5.546212in}{2.364950in}}{\pgfqpoint{5.546212in}{2.376000in}}%
\pgfpathcurveto{\pgfqpoint{5.546212in}{2.387050in}}{\pgfqpoint{5.541822in}{2.397649in}}{\pgfqpoint{5.534008in}{2.405463in}}%
\pgfpathcurveto{\pgfqpoint{5.526195in}{2.413276in}}{\pgfqpoint{5.515596in}{2.417667in}}{\pgfqpoint{5.504545in}{2.417667in}}%
\pgfpathcurveto{\pgfqpoint{5.493495in}{2.417667in}}{\pgfqpoint{5.482896in}{2.413276in}}{\pgfqpoint{5.475083in}{2.405463in}}%
\pgfpathcurveto{\pgfqpoint{5.467269in}{2.397649in}}{\pgfqpoint{5.462879in}{2.387050in}}{\pgfqpoint{5.462879in}{2.376000in}}%
\pgfpathcurveto{\pgfqpoint{5.462879in}{2.364950in}}{\pgfqpoint{5.467269in}{2.354351in}}{\pgfqpoint{5.475083in}{2.346537in}}%
\pgfpathcurveto{\pgfqpoint{5.482896in}{2.338724in}}{\pgfqpoint{5.493495in}{2.334333in}}{\pgfqpoint{5.504545in}{2.334333in}}%
\pgfpathclose%
\pgfusepath{stroke,fill}%
\end{pgfscope}%
\begin{pgfscope}%
\pgfpathrectangle{\pgfqpoint{0.800000in}{0.528000in}}{\pgfqpoint{4.960000in}{3.696000in}}%
\pgfusepath{clip}%
\pgfsetbuttcap%
\pgfsetroundjoin%
\definecolor{currentfill}{rgb}{0.000000,0.000000,0.000000}%
\pgfsetfillcolor{currentfill}%
\pgfsetlinewidth{1.003750pt}%
\definecolor{currentstroke}{rgb}{0.000000,0.000000,0.000000}%
\pgfsetstrokecolor{currentstroke}%
\pgfsetdash{}{0pt}%
\pgfpathmoveto{\pgfqpoint{5.504545in}{2.334333in}}%
\pgfpathcurveto{\pgfqpoint{5.515596in}{2.334333in}}{\pgfqpoint{5.526195in}{2.338724in}}{\pgfqpoint{5.534008in}{2.346537in}}%
\pgfpathcurveto{\pgfqpoint{5.541822in}{2.354351in}}{\pgfqpoint{5.546212in}{2.364950in}}{\pgfqpoint{5.546212in}{2.376000in}}%
\pgfpathcurveto{\pgfqpoint{5.546212in}{2.387050in}}{\pgfqpoint{5.541822in}{2.397649in}}{\pgfqpoint{5.534008in}{2.405463in}}%
\pgfpathcurveto{\pgfqpoint{5.526195in}{2.413276in}}{\pgfqpoint{5.515596in}{2.417667in}}{\pgfqpoint{5.504545in}{2.417667in}}%
\pgfpathcurveto{\pgfqpoint{5.493495in}{2.417667in}}{\pgfqpoint{5.482896in}{2.413276in}}{\pgfqpoint{5.475083in}{2.405463in}}%
\pgfpathcurveto{\pgfqpoint{5.467269in}{2.397649in}}{\pgfqpoint{5.462879in}{2.387050in}}{\pgfqpoint{5.462879in}{2.376000in}}%
\pgfpathcurveto{\pgfqpoint{5.462879in}{2.364950in}}{\pgfqpoint{5.467269in}{2.354351in}}{\pgfqpoint{5.475083in}{2.346537in}}%
\pgfpathcurveto{\pgfqpoint{5.482896in}{2.338724in}}{\pgfqpoint{5.493495in}{2.334333in}}{\pgfqpoint{5.504545in}{2.334333in}}%
\pgfpathclose%
\pgfusepath{stroke,fill}%
\end{pgfscope}%
\begin{pgfscope}%
\pgfpathrectangle{\pgfqpoint{0.800000in}{0.528000in}}{\pgfqpoint{4.960000in}{3.696000in}}%
\pgfusepath{clip}%
\pgfsetbuttcap%
\pgfsetroundjoin%
\definecolor{currentfill}{rgb}{0.000000,0.000000,0.000000}%
\pgfsetfillcolor{currentfill}%
\pgfsetlinewidth{1.003750pt}%
\definecolor{currentstroke}{rgb}{0.000000,0.000000,0.000000}%
\pgfsetstrokecolor{currentstroke}%
\pgfsetdash{}{0pt}%
\pgfpathmoveto{\pgfqpoint{5.504545in}{2.334333in}}%
\pgfpathcurveto{\pgfqpoint{5.515596in}{2.334333in}}{\pgfqpoint{5.526195in}{2.338724in}}{\pgfqpoint{5.534008in}{2.346537in}}%
\pgfpathcurveto{\pgfqpoint{5.541822in}{2.354351in}}{\pgfqpoint{5.546212in}{2.364950in}}{\pgfqpoint{5.546212in}{2.376000in}}%
\pgfpathcurveto{\pgfqpoint{5.546212in}{2.387050in}}{\pgfqpoint{5.541822in}{2.397649in}}{\pgfqpoint{5.534008in}{2.405463in}}%
\pgfpathcurveto{\pgfqpoint{5.526195in}{2.413276in}}{\pgfqpoint{5.515596in}{2.417667in}}{\pgfqpoint{5.504545in}{2.417667in}}%
\pgfpathcurveto{\pgfqpoint{5.493495in}{2.417667in}}{\pgfqpoint{5.482896in}{2.413276in}}{\pgfqpoint{5.475083in}{2.405463in}}%
\pgfpathcurveto{\pgfqpoint{5.467269in}{2.397649in}}{\pgfqpoint{5.462879in}{2.387050in}}{\pgfqpoint{5.462879in}{2.376000in}}%
\pgfpathcurveto{\pgfqpoint{5.462879in}{2.364950in}}{\pgfqpoint{5.467269in}{2.354351in}}{\pgfqpoint{5.475083in}{2.346537in}}%
\pgfpathcurveto{\pgfqpoint{5.482896in}{2.338724in}}{\pgfqpoint{5.493495in}{2.334333in}}{\pgfqpoint{5.504545in}{2.334333in}}%
\pgfpathclose%
\pgfusepath{stroke,fill}%
\end{pgfscope}%
\begin{pgfscope}%
\pgfpathrectangle{\pgfqpoint{0.800000in}{0.528000in}}{\pgfqpoint{4.960000in}{3.696000in}}%
\pgfusepath{clip}%
\pgfsetbuttcap%
\pgfsetroundjoin%
\definecolor{currentfill}{rgb}{0.000000,0.000000,0.000000}%
\pgfsetfillcolor{currentfill}%
\pgfsetlinewidth{1.003750pt}%
\definecolor{currentstroke}{rgb}{0.000000,0.000000,0.000000}%
\pgfsetstrokecolor{currentstroke}%
\pgfsetdash{}{0pt}%
\pgfpathmoveto{\pgfqpoint{5.504545in}{2.334333in}}%
\pgfpathcurveto{\pgfqpoint{5.515596in}{2.334333in}}{\pgfqpoint{5.526195in}{2.338724in}}{\pgfqpoint{5.534008in}{2.346537in}}%
\pgfpathcurveto{\pgfqpoint{5.541822in}{2.354351in}}{\pgfqpoint{5.546212in}{2.364950in}}{\pgfqpoint{5.546212in}{2.376000in}}%
\pgfpathcurveto{\pgfqpoint{5.546212in}{2.387050in}}{\pgfqpoint{5.541822in}{2.397649in}}{\pgfqpoint{5.534008in}{2.405463in}}%
\pgfpathcurveto{\pgfqpoint{5.526195in}{2.413276in}}{\pgfqpoint{5.515596in}{2.417667in}}{\pgfqpoint{5.504545in}{2.417667in}}%
\pgfpathcurveto{\pgfqpoint{5.493495in}{2.417667in}}{\pgfqpoint{5.482896in}{2.413276in}}{\pgfqpoint{5.475083in}{2.405463in}}%
\pgfpathcurveto{\pgfqpoint{5.467269in}{2.397649in}}{\pgfqpoint{5.462879in}{2.387050in}}{\pgfqpoint{5.462879in}{2.376000in}}%
\pgfpathcurveto{\pgfqpoint{5.462879in}{2.364950in}}{\pgfqpoint{5.467269in}{2.354351in}}{\pgfqpoint{5.475083in}{2.346537in}}%
\pgfpathcurveto{\pgfqpoint{5.482896in}{2.338724in}}{\pgfqpoint{5.493495in}{2.334333in}}{\pgfqpoint{5.504545in}{2.334333in}}%
\pgfpathclose%
\pgfusepath{stroke,fill}%
\end{pgfscope}%
\begin{pgfscope}%
\pgfpathrectangle{\pgfqpoint{0.800000in}{0.528000in}}{\pgfqpoint{4.960000in}{3.696000in}}%
\pgfusepath{clip}%
\pgfsetbuttcap%
\pgfsetroundjoin%
\definecolor{currentfill}{rgb}{0.000000,0.000000,0.000000}%
\pgfsetfillcolor{currentfill}%
\pgfsetlinewidth{1.003750pt}%
\definecolor{currentstroke}{rgb}{0.000000,0.000000,0.000000}%
\pgfsetstrokecolor{currentstroke}%
\pgfsetdash{}{0pt}%
\pgfpathmoveto{\pgfqpoint{5.504545in}{2.334333in}}%
\pgfpathcurveto{\pgfqpoint{5.515596in}{2.334333in}}{\pgfqpoint{5.526195in}{2.338724in}}{\pgfqpoint{5.534008in}{2.346537in}}%
\pgfpathcurveto{\pgfqpoint{5.541822in}{2.354351in}}{\pgfqpoint{5.546212in}{2.364950in}}{\pgfqpoint{5.546212in}{2.376000in}}%
\pgfpathcurveto{\pgfqpoint{5.546212in}{2.387050in}}{\pgfqpoint{5.541822in}{2.397649in}}{\pgfqpoint{5.534008in}{2.405463in}}%
\pgfpathcurveto{\pgfqpoint{5.526195in}{2.413276in}}{\pgfqpoint{5.515596in}{2.417667in}}{\pgfqpoint{5.504545in}{2.417667in}}%
\pgfpathcurveto{\pgfqpoint{5.493495in}{2.417667in}}{\pgfqpoint{5.482896in}{2.413276in}}{\pgfqpoint{5.475083in}{2.405463in}}%
\pgfpathcurveto{\pgfqpoint{5.467269in}{2.397649in}}{\pgfqpoint{5.462879in}{2.387050in}}{\pgfqpoint{5.462879in}{2.376000in}}%
\pgfpathcurveto{\pgfqpoint{5.462879in}{2.364950in}}{\pgfqpoint{5.467269in}{2.354351in}}{\pgfqpoint{5.475083in}{2.346537in}}%
\pgfpathcurveto{\pgfqpoint{5.482896in}{2.338724in}}{\pgfqpoint{5.493495in}{2.334333in}}{\pgfqpoint{5.504545in}{2.334333in}}%
\pgfpathclose%
\pgfusepath{stroke,fill}%
\end{pgfscope}%
\begin{pgfscope}%
\pgfpathrectangle{\pgfqpoint{0.800000in}{0.528000in}}{\pgfqpoint{4.960000in}{3.696000in}}%
\pgfusepath{clip}%
\pgfsetbuttcap%
\pgfsetroundjoin%
\definecolor{currentfill}{rgb}{0.000000,0.000000,0.000000}%
\pgfsetfillcolor{currentfill}%
\pgfsetlinewidth{1.003750pt}%
\definecolor{currentstroke}{rgb}{0.000000,0.000000,0.000000}%
\pgfsetstrokecolor{currentstroke}%
\pgfsetdash{}{0pt}%
\pgfpathmoveto{\pgfqpoint{5.504545in}{2.334333in}}%
\pgfpathcurveto{\pgfqpoint{5.515596in}{2.334333in}}{\pgfqpoint{5.526195in}{2.338724in}}{\pgfqpoint{5.534008in}{2.346537in}}%
\pgfpathcurveto{\pgfqpoint{5.541822in}{2.354351in}}{\pgfqpoint{5.546212in}{2.364950in}}{\pgfqpoint{5.546212in}{2.376000in}}%
\pgfpathcurveto{\pgfqpoint{5.546212in}{2.387050in}}{\pgfqpoint{5.541822in}{2.397649in}}{\pgfqpoint{5.534008in}{2.405463in}}%
\pgfpathcurveto{\pgfqpoint{5.526195in}{2.413276in}}{\pgfqpoint{5.515596in}{2.417667in}}{\pgfqpoint{5.504545in}{2.417667in}}%
\pgfpathcurveto{\pgfqpoint{5.493495in}{2.417667in}}{\pgfqpoint{5.482896in}{2.413276in}}{\pgfqpoint{5.475083in}{2.405463in}}%
\pgfpathcurveto{\pgfqpoint{5.467269in}{2.397649in}}{\pgfqpoint{5.462879in}{2.387050in}}{\pgfqpoint{5.462879in}{2.376000in}}%
\pgfpathcurveto{\pgfqpoint{5.462879in}{2.364950in}}{\pgfqpoint{5.467269in}{2.354351in}}{\pgfqpoint{5.475083in}{2.346537in}}%
\pgfpathcurveto{\pgfqpoint{5.482896in}{2.338724in}}{\pgfqpoint{5.493495in}{2.334333in}}{\pgfqpoint{5.504545in}{2.334333in}}%
\pgfpathclose%
\pgfusepath{stroke,fill}%
\end{pgfscope}%
\begin{pgfscope}%
\pgfpathrectangle{\pgfqpoint{0.800000in}{0.528000in}}{\pgfqpoint{4.960000in}{3.696000in}}%
\pgfusepath{clip}%
\pgfsetbuttcap%
\pgfsetroundjoin%
\definecolor{currentfill}{rgb}{0.000000,0.000000,0.000000}%
\pgfsetfillcolor{currentfill}%
\pgfsetlinewidth{1.003750pt}%
\definecolor{currentstroke}{rgb}{0.000000,0.000000,0.000000}%
\pgfsetstrokecolor{currentstroke}%
\pgfsetdash{}{0pt}%
\pgfpathmoveto{\pgfqpoint{5.504545in}{2.334333in}}%
\pgfpathcurveto{\pgfqpoint{5.515596in}{2.334333in}}{\pgfqpoint{5.526195in}{2.338724in}}{\pgfqpoint{5.534008in}{2.346537in}}%
\pgfpathcurveto{\pgfqpoint{5.541822in}{2.354351in}}{\pgfqpoint{5.546212in}{2.364950in}}{\pgfqpoint{5.546212in}{2.376000in}}%
\pgfpathcurveto{\pgfqpoint{5.546212in}{2.387050in}}{\pgfqpoint{5.541822in}{2.397649in}}{\pgfqpoint{5.534008in}{2.405463in}}%
\pgfpathcurveto{\pgfqpoint{5.526195in}{2.413276in}}{\pgfqpoint{5.515596in}{2.417667in}}{\pgfqpoint{5.504545in}{2.417667in}}%
\pgfpathcurveto{\pgfqpoint{5.493495in}{2.417667in}}{\pgfqpoint{5.482896in}{2.413276in}}{\pgfqpoint{5.475083in}{2.405463in}}%
\pgfpathcurveto{\pgfqpoint{5.467269in}{2.397649in}}{\pgfqpoint{5.462879in}{2.387050in}}{\pgfqpoint{5.462879in}{2.376000in}}%
\pgfpathcurveto{\pgfqpoint{5.462879in}{2.364950in}}{\pgfqpoint{5.467269in}{2.354351in}}{\pgfqpoint{5.475083in}{2.346537in}}%
\pgfpathcurveto{\pgfqpoint{5.482896in}{2.338724in}}{\pgfqpoint{5.493495in}{2.334333in}}{\pgfqpoint{5.504545in}{2.334333in}}%
\pgfpathclose%
\pgfusepath{stroke,fill}%
\end{pgfscope}%
\begin{pgfscope}%
\pgfpathrectangle{\pgfqpoint{0.800000in}{0.528000in}}{\pgfqpoint{4.960000in}{3.696000in}}%
\pgfusepath{clip}%
\pgfsetbuttcap%
\pgfsetroundjoin%
\definecolor{currentfill}{rgb}{0.000000,0.000000,0.000000}%
\pgfsetfillcolor{currentfill}%
\pgfsetlinewidth{1.003750pt}%
\definecolor{currentstroke}{rgb}{0.000000,0.000000,0.000000}%
\pgfsetstrokecolor{currentstroke}%
\pgfsetdash{}{0pt}%
\pgfpathmoveto{\pgfqpoint{5.504545in}{2.334333in}}%
\pgfpathcurveto{\pgfqpoint{5.515596in}{2.334333in}}{\pgfqpoint{5.526195in}{2.338724in}}{\pgfqpoint{5.534008in}{2.346537in}}%
\pgfpathcurveto{\pgfqpoint{5.541822in}{2.354351in}}{\pgfqpoint{5.546212in}{2.364950in}}{\pgfqpoint{5.546212in}{2.376000in}}%
\pgfpathcurveto{\pgfqpoint{5.546212in}{2.387050in}}{\pgfqpoint{5.541822in}{2.397649in}}{\pgfqpoint{5.534008in}{2.405463in}}%
\pgfpathcurveto{\pgfqpoint{5.526195in}{2.413276in}}{\pgfqpoint{5.515596in}{2.417667in}}{\pgfqpoint{5.504545in}{2.417667in}}%
\pgfpathcurveto{\pgfqpoint{5.493495in}{2.417667in}}{\pgfqpoint{5.482896in}{2.413276in}}{\pgfqpoint{5.475083in}{2.405463in}}%
\pgfpathcurveto{\pgfqpoint{5.467269in}{2.397649in}}{\pgfqpoint{5.462879in}{2.387050in}}{\pgfqpoint{5.462879in}{2.376000in}}%
\pgfpathcurveto{\pgfqpoint{5.462879in}{2.364950in}}{\pgfqpoint{5.467269in}{2.354351in}}{\pgfqpoint{5.475083in}{2.346537in}}%
\pgfpathcurveto{\pgfqpoint{5.482896in}{2.338724in}}{\pgfqpoint{5.493495in}{2.334333in}}{\pgfqpoint{5.504545in}{2.334333in}}%
\pgfpathclose%
\pgfusepath{stroke,fill}%
\end{pgfscope}%
\begin{pgfscope}%
\pgfpathrectangle{\pgfqpoint{0.800000in}{0.528000in}}{\pgfqpoint{4.960000in}{3.696000in}}%
\pgfusepath{clip}%
\pgfsetbuttcap%
\pgfsetroundjoin%
\definecolor{currentfill}{rgb}{0.000000,0.000000,0.000000}%
\pgfsetfillcolor{currentfill}%
\pgfsetlinewidth{1.003750pt}%
\definecolor{currentstroke}{rgb}{0.000000,0.000000,0.000000}%
\pgfsetstrokecolor{currentstroke}%
\pgfsetdash{}{0pt}%
\pgfpathmoveto{\pgfqpoint{5.504545in}{2.334333in}}%
\pgfpathcurveto{\pgfqpoint{5.515596in}{2.334333in}}{\pgfqpoint{5.526195in}{2.338724in}}{\pgfqpoint{5.534008in}{2.346537in}}%
\pgfpathcurveto{\pgfqpoint{5.541822in}{2.354351in}}{\pgfqpoint{5.546212in}{2.364950in}}{\pgfqpoint{5.546212in}{2.376000in}}%
\pgfpathcurveto{\pgfqpoint{5.546212in}{2.387050in}}{\pgfqpoint{5.541822in}{2.397649in}}{\pgfqpoint{5.534008in}{2.405463in}}%
\pgfpathcurveto{\pgfqpoint{5.526195in}{2.413276in}}{\pgfqpoint{5.515596in}{2.417667in}}{\pgfqpoint{5.504545in}{2.417667in}}%
\pgfpathcurveto{\pgfqpoint{5.493495in}{2.417667in}}{\pgfqpoint{5.482896in}{2.413276in}}{\pgfqpoint{5.475083in}{2.405463in}}%
\pgfpathcurveto{\pgfqpoint{5.467269in}{2.397649in}}{\pgfqpoint{5.462879in}{2.387050in}}{\pgfqpoint{5.462879in}{2.376000in}}%
\pgfpathcurveto{\pgfqpoint{5.462879in}{2.364950in}}{\pgfqpoint{5.467269in}{2.354351in}}{\pgfqpoint{5.475083in}{2.346537in}}%
\pgfpathcurveto{\pgfqpoint{5.482896in}{2.338724in}}{\pgfqpoint{5.493495in}{2.334333in}}{\pgfqpoint{5.504545in}{2.334333in}}%
\pgfpathclose%
\pgfusepath{stroke,fill}%
\end{pgfscope}%
\begin{pgfscope}%
\pgfpathrectangle{\pgfqpoint{0.800000in}{0.528000in}}{\pgfqpoint{4.960000in}{3.696000in}}%
\pgfusepath{clip}%
\pgfsetbuttcap%
\pgfsetroundjoin%
\definecolor{currentfill}{rgb}{0.000000,0.000000,0.000000}%
\pgfsetfillcolor{currentfill}%
\pgfsetlinewidth{1.003750pt}%
\definecolor{currentstroke}{rgb}{0.000000,0.000000,0.000000}%
\pgfsetstrokecolor{currentstroke}%
\pgfsetdash{}{0pt}%
\pgfpathmoveto{\pgfqpoint{5.504545in}{2.334333in}}%
\pgfpathcurveto{\pgfqpoint{5.515596in}{2.334333in}}{\pgfqpoint{5.526195in}{2.338724in}}{\pgfqpoint{5.534008in}{2.346537in}}%
\pgfpathcurveto{\pgfqpoint{5.541822in}{2.354351in}}{\pgfqpoint{5.546212in}{2.364950in}}{\pgfqpoint{5.546212in}{2.376000in}}%
\pgfpathcurveto{\pgfqpoint{5.546212in}{2.387050in}}{\pgfqpoint{5.541822in}{2.397649in}}{\pgfqpoint{5.534008in}{2.405463in}}%
\pgfpathcurveto{\pgfqpoint{5.526195in}{2.413276in}}{\pgfqpoint{5.515596in}{2.417667in}}{\pgfqpoint{5.504545in}{2.417667in}}%
\pgfpathcurveto{\pgfqpoint{5.493495in}{2.417667in}}{\pgfqpoint{5.482896in}{2.413276in}}{\pgfqpoint{5.475083in}{2.405463in}}%
\pgfpathcurveto{\pgfqpoint{5.467269in}{2.397649in}}{\pgfqpoint{5.462879in}{2.387050in}}{\pgfqpoint{5.462879in}{2.376000in}}%
\pgfpathcurveto{\pgfqpoint{5.462879in}{2.364950in}}{\pgfqpoint{5.467269in}{2.354351in}}{\pgfqpoint{5.475083in}{2.346537in}}%
\pgfpathcurveto{\pgfqpoint{5.482896in}{2.338724in}}{\pgfqpoint{5.493495in}{2.334333in}}{\pgfqpoint{5.504545in}{2.334333in}}%
\pgfpathclose%
\pgfusepath{stroke,fill}%
\end{pgfscope}%
\begin{pgfscope}%
\pgfpathrectangle{\pgfqpoint{0.800000in}{0.528000in}}{\pgfqpoint{4.960000in}{3.696000in}}%
\pgfusepath{clip}%
\pgfsetbuttcap%
\pgfsetroundjoin%
\definecolor{currentfill}{rgb}{0.000000,0.000000,0.000000}%
\pgfsetfillcolor{currentfill}%
\pgfsetlinewidth{1.003750pt}%
\definecolor{currentstroke}{rgb}{0.000000,0.000000,0.000000}%
\pgfsetstrokecolor{currentstroke}%
\pgfsetdash{}{0pt}%
\pgfpathmoveto{\pgfqpoint{5.504545in}{2.334333in}}%
\pgfpathcurveto{\pgfqpoint{5.515596in}{2.334333in}}{\pgfqpoint{5.526195in}{2.338724in}}{\pgfqpoint{5.534008in}{2.346537in}}%
\pgfpathcurveto{\pgfqpoint{5.541822in}{2.354351in}}{\pgfqpoint{5.546212in}{2.364950in}}{\pgfqpoint{5.546212in}{2.376000in}}%
\pgfpathcurveto{\pgfqpoint{5.546212in}{2.387050in}}{\pgfqpoint{5.541822in}{2.397649in}}{\pgfqpoint{5.534008in}{2.405463in}}%
\pgfpathcurveto{\pgfqpoint{5.526195in}{2.413276in}}{\pgfqpoint{5.515596in}{2.417667in}}{\pgfqpoint{5.504545in}{2.417667in}}%
\pgfpathcurveto{\pgfqpoint{5.493495in}{2.417667in}}{\pgfqpoint{5.482896in}{2.413276in}}{\pgfqpoint{5.475083in}{2.405463in}}%
\pgfpathcurveto{\pgfqpoint{5.467269in}{2.397649in}}{\pgfqpoint{5.462879in}{2.387050in}}{\pgfqpoint{5.462879in}{2.376000in}}%
\pgfpathcurveto{\pgfqpoint{5.462879in}{2.364950in}}{\pgfqpoint{5.467269in}{2.354351in}}{\pgfqpoint{5.475083in}{2.346537in}}%
\pgfpathcurveto{\pgfqpoint{5.482896in}{2.338724in}}{\pgfqpoint{5.493495in}{2.334333in}}{\pgfqpoint{5.504545in}{2.334333in}}%
\pgfpathclose%
\pgfusepath{stroke,fill}%
\end{pgfscope}%
\begin{pgfscope}%
\pgfpathrectangle{\pgfqpoint{0.800000in}{0.528000in}}{\pgfqpoint{4.960000in}{3.696000in}}%
\pgfusepath{clip}%
\pgfsetbuttcap%
\pgfsetroundjoin%
\definecolor{currentfill}{rgb}{0.000000,0.000000,0.000000}%
\pgfsetfillcolor{currentfill}%
\pgfsetlinewidth{1.003750pt}%
\definecolor{currentstroke}{rgb}{0.000000,0.000000,0.000000}%
\pgfsetstrokecolor{currentstroke}%
\pgfsetdash{}{0pt}%
\pgfpathmoveto{\pgfqpoint{5.504545in}{2.334333in}}%
\pgfpathcurveto{\pgfqpoint{5.515596in}{2.334333in}}{\pgfqpoint{5.526195in}{2.338724in}}{\pgfqpoint{5.534008in}{2.346537in}}%
\pgfpathcurveto{\pgfqpoint{5.541822in}{2.354351in}}{\pgfqpoint{5.546212in}{2.364950in}}{\pgfqpoint{5.546212in}{2.376000in}}%
\pgfpathcurveto{\pgfqpoint{5.546212in}{2.387050in}}{\pgfqpoint{5.541822in}{2.397649in}}{\pgfqpoint{5.534008in}{2.405463in}}%
\pgfpathcurveto{\pgfqpoint{5.526195in}{2.413276in}}{\pgfqpoint{5.515596in}{2.417667in}}{\pgfqpoint{5.504545in}{2.417667in}}%
\pgfpathcurveto{\pgfqpoint{5.493495in}{2.417667in}}{\pgfqpoint{5.482896in}{2.413276in}}{\pgfqpoint{5.475083in}{2.405463in}}%
\pgfpathcurveto{\pgfqpoint{5.467269in}{2.397649in}}{\pgfqpoint{5.462879in}{2.387050in}}{\pgfqpoint{5.462879in}{2.376000in}}%
\pgfpathcurveto{\pgfqpoint{5.462879in}{2.364950in}}{\pgfqpoint{5.467269in}{2.354351in}}{\pgfqpoint{5.475083in}{2.346537in}}%
\pgfpathcurveto{\pgfqpoint{5.482896in}{2.338724in}}{\pgfqpoint{5.493495in}{2.334333in}}{\pgfqpoint{5.504545in}{2.334333in}}%
\pgfpathclose%
\pgfusepath{stroke,fill}%
\end{pgfscope}%
\begin{pgfscope}%
\pgfpathrectangle{\pgfqpoint{0.800000in}{0.528000in}}{\pgfqpoint{4.960000in}{3.696000in}}%
\pgfusepath{clip}%
\pgfsetbuttcap%
\pgfsetroundjoin%
\definecolor{currentfill}{rgb}{0.000000,0.000000,0.000000}%
\pgfsetfillcolor{currentfill}%
\pgfsetlinewidth{1.003750pt}%
\definecolor{currentstroke}{rgb}{0.000000,0.000000,0.000000}%
\pgfsetstrokecolor{currentstroke}%
\pgfsetdash{}{0pt}%
\pgfpathmoveto{\pgfqpoint{5.504545in}{2.334333in}}%
\pgfpathcurveto{\pgfqpoint{5.515596in}{2.334333in}}{\pgfqpoint{5.526195in}{2.338724in}}{\pgfqpoint{5.534008in}{2.346537in}}%
\pgfpathcurveto{\pgfqpoint{5.541822in}{2.354351in}}{\pgfqpoint{5.546212in}{2.364950in}}{\pgfqpoint{5.546212in}{2.376000in}}%
\pgfpathcurveto{\pgfqpoint{5.546212in}{2.387050in}}{\pgfqpoint{5.541822in}{2.397649in}}{\pgfqpoint{5.534008in}{2.405463in}}%
\pgfpathcurveto{\pgfqpoint{5.526195in}{2.413276in}}{\pgfqpoint{5.515596in}{2.417667in}}{\pgfqpoint{5.504545in}{2.417667in}}%
\pgfpathcurveto{\pgfqpoint{5.493495in}{2.417667in}}{\pgfqpoint{5.482896in}{2.413276in}}{\pgfqpoint{5.475083in}{2.405463in}}%
\pgfpathcurveto{\pgfqpoint{5.467269in}{2.397649in}}{\pgfqpoint{5.462879in}{2.387050in}}{\pgfqpoint{5.462879in}{2.376000in}}%
\pgfpathcurveto{\pgfqpoint{5.462879in}{2.364950in}}{\pgfqpoint{5.467269in}{2.354351in}}{\pgfqpoint{5.475083in}{2.346537in}}%
\pgfpathcurveto{\pgfqpoint{5.482896in}{2.338724in}}{\pgfqpoint{5.493495in}{2.334333in}}{\pgfqpoint{5.504545in}{2.334333in}}%
\pgfpathclose%
\pgfusepath{stroke,fill}%
\end{pgfscope}%
\begin{pgfscope}%
\pgfpathrectangle{\pgfqpoint{0.800000in}{0.528000in}}{\pgfqpoint{4.960000in}{3.696000in}}%
\pgfusepath{clip}%
\pgfsetbuttcap%
\pgfsetroundjoin%
\definecolor{currentfill}{rgb}{0.000000,0.000000,0.000000}%
\pgfsetfillcolor{currentfill}%
\pgfsetlinewidth{1.003750pt}%
\definecolor{currentstroke}{rgb}{0.000000,0.000000,0.000000}%
\pgfsetstrokecolor{currentstroke}%
\pgfsetdash{}{0pt}%
\pgfpathmoveto{\pgfqpoint{5.504545in}{2.334333in}}%
\pgfpathcurveto{\pgfqpoint{5.515596in}{2.334333in}}{\pgfqpoint{5.526195in}{2.338724in}}{\pgfqpoint{5.534008in}{2.346537in}}%
\pgfpathcurveto{\pgfqpoint{5.541822in}{2.354351in}}{\pgfqpoint{5.546212in}{2.364950in}}{\pgfqpoint{5.546212in}{2.376000in}}%
\pgfpathcurveto{\pgfqpoint{5.546212in}{2.387050in}}{\pgfqpoint{5.541822in}{2.397649in}}{\pgfqpoint{5.534008in}{2.405463in}}%
\pgfpathcurveto{\pgfqpoint{5.526195in}{2.413276in}}{\pgfqpoint{5.515596in}{2.417667in}}{\pgfqpoint{5.504545in}{2.417667in}}%
\pgfpathcurveto{\pgfqpoint{5.493495in}{2.417667in}}{\pgfqpoint{5.482896in}{2.413276in}}{\pgfqpoint{5.475083in}{2.405463in}}%
\pgfpathcurveto{\pgfqpoint{5.467269in}{2.397649in}}{\pgfqpoint{5.462879in}{2.387050in}}{\pgfqpoint{5.462879in}{2.376000in}}%
\pgfpathcurveto{\pgfqpoint{5.462879in}{2.364950in}}{\pgfqpoint{5.467269in}{2.354351in}}{\pgfqpoint{5.475083in}{2.346537in}}%
\pgfpathcurveto{\pgfqpoint{5.482896in}{2.338724in}}{\pgfqpoint{5.493495in}{2.334333in}}{\pgfqpoint{5.504545in}{2.334333in}}%
\pgfpathclose%
\pgfusepath{stroke,fill}%
\end{pgfscope}%
\begin{pgfscope}%
\pgfpathrectangle{\pgfqpoint{0.800000in}{0.528000in}}{\pgfqpoint{4.960000in}{3.696000in}}%
\pgfusepath{clip}%
\pgfsetbuttcap%
\pgfsetroundjoin%
\definecolor{currentfill}{rgb}{0.000000,0.000000,0.000000}%
\pgfsetfillcolor{currentfill}%
\pgfsetlinewidth{1.003750pt}%
\definecolor{currentstroke}{rgb}{0.000000,0.000000,0.000000}%
\pgfsetstrokecolor{currentstroke}%
\pgfsetdash{}{0pt}%
\pgfpathmoveto{\pgfqpoint{5.504545in}{2.334333in}}%
\pgfpathcurveto{\pgfqpoint{5.515596in}{2.334333in}}{\pgfqpoint{5.526195in}{2.338724in}}{\pgfqpoint{5.534008in}{2.346537in}}%
\pgfpathcurveto{\pgfqpoint{5.541822in}{2.354351in}}{\pgfqpoint{5.546212in}{2.364950in}}{\pgfqpoint{5.546212in}{2.376000in}}%
\pgfpathcurveto{\pgfqpoint{5.546212in}{2.387050in}}{\pgfqpoint{5.541822in}{2.397649in}}{\pgfqpoint{5.534008in}{2.405463in}}%
\pgfpathcurveto{\pgfqpoint{5.526195in}{2.413276in}}{\pgfqpoint{5.515596in}{2.417667in}}{\pgfqpoint{5.504545in}{2.417667in}}%
\pgfpathcurveto{\pgfqpoint{5.493495in}{2.417667in}}{\pgfqpoint{5.482896in}{2.413276in}}{\pgfqpoint{5.475083in}{2.405463in}}%
\pgfpathcurveto{\pgfqpoint{5.467269in}{2.397649in}}{\pgfqpoint{5.462879in}{2.387050in}}{\pgfqpoint{5.462879in}{2.376000in}}%
\pgfpathcurveto{\pgfqpoint{5.462879in}{2.364950in}}{\pgfqpoint{5.467269in}{2.354351in}}{\pgfqpoint{5.475083in}{2.346537in}}%
\pgfpathcurveto{\pgfqpoint{5.482896in}{2.338724in}}{\pgfqpoint{5.493495in}{2.334333in}}{\pgfqpoint{5.504545in}{2.334333in}}%
\pgfpathclose%
\pgfusepath{stroke,fill}%
\end{pgfscope}%
\begin{pgfscope}%
\pgfpathrectangle{\pgfqpoint{0.800000in}{0.528000in}}{\pgfqpoint{4.960000in}{3.696000in}}%
\pgfusepath{clip}%
\pgfsetbuttcap%
\pgfsetroundjoin%
\definecolor{currentfill}{rgb}{0.000000,0.000000,0.000000}%
\pgfsetfillcolor{currentfill}%
\pgfsetlinewidth{1.003750pt}%
\definecolor{currentstroke}{rgb}{0.000000,0.000000,0.000000}%
\pgfsetstrokecolor{currentstroke}%
\pgfsetdash{}{0pt}%
\pgfpathmoveto{\pgfqpoint{5.504545in}{2.334333in}}%
\pgfpathcurveto{\pgfqpoint{5.515596in}{2.334333in}}{\pgfqpoint{5.526195in}{2.338724in}}{\pgfqpoint{5.534008in}{2.346537in}}%
\pgfpathcurveto{\pgfqpoint{5.541822in}{2.354351in}}{\pgfqpoint{5.546212in}{2.364950in}}{\pgfqpoint{5.546212in}{2.376000in}}%
\pgfpathcurveto{\pgfqpoint{5.546212in}{2.387050in}}{\pgfqpoint{5.541822in}{2.397649in}}{\pgfqpoint{5.534008in}{2.405463in}}%
\pgfpathcurveto{\pgfqpoint{5.526195in}{2.413276in}}{\pgfqpoint{5.515596in}{2.417667in}}{\pgfqpoint{5.504545in}{2.417667in}}%
\pgfpathcurveto{\pgfqpoint{5.493495in}{2.417667in}}{\pgfqpoint{5.482896in}{2.413276in}}{\pgfqpoint{5.475083in}{2.405463in}}%
\pgfpathcurveto{\pgfqpoint{5.467269in}{2.397649in}}{\pgfqpoint{5.462879in}{2.387050in}}{\pgfqpoint{5.462879in}{2.376000in}}%
\pgfpathcurveto{\pgfqpoint{5.462879in}{2.364950in}}{\pgfqpoint{5.467269in}{2.354351in}}{\pgfqpoint{5.475083in}{2.346537in}}%
\pgfpathcurveto{\pgfqpoint{5.482896in}{2.338724in}}{\pgfqpoint{5.493495in}{2.334333in}}{\pgfqpoint{5.504545in}{2.334333in}}%
\pgfpathclose%
\pgfusepath{stroke,fill}%
\end{pgfscope}%
\begin{pgfscope}%
\pgfpathrectangle{\pgfqpoint{0.800000in}{0.528000in}}{\pgfqpoint{4.960000in}{3.696000in}}%
\pgfusepath{clip}%
\pgfsetbuttcap%
\pgfsetroundjoin%
\definecolor{currentfill}{rgb}{0.000000,0.000000,0.000000}%
\pgfsetfillcolor{currentfill}%
\pgfsetlinewidth{1.003750pt}%
\definecolor{currentstroke}{rgb}{0.000000,0.000000,0.000000}%
\pgfsetstrokecolor{currentstroke}%
\pgfsetdash{}{0pt}%
\pgfpathmoveto{\pgfqpoint{5.504545in}{2.334333in}}%
\pgfpathcurveto{\pgfqpoint{5.515596in}{2.334333in}}{\pgfqpoint{5.526195in}{2.338724in}}{\pgfqpoint{5.534008in}{2.346537in}}%
\pgfpathcurveto{\pgfqpoint{5.541822in}{2.354351in}}{\pgfqpoint{5.546212in}{2.364950in}}{\pgfqpoint{5.546212in}{2.376000in}}%
\pgfpathcurveto{\pgfqpoint{5.546212in}{2.387050in}}{\pgfqpoint{5.541822in}{2.397649in}}{\pgfqpoint{5.534008in}{2.405463in}}%
\pgfpathcurveto{\pgfqpoint{5.526195in}{2.413276in}}{\pgfqpoint{5.515596in}{2.417667in}}{\pgfqpoint{5.504545in}{2.417667in}}%
\pgfpathcurveto{\pgfqpoint{5.493495in}{2.417667in}}{\pgfqpoint{5.482896in}{2.413276in}}{\pgfqpoint{5.475083in}{2.405463in}}%
\pgfpathcurveto{\pgfqpoint{5.467269in}{2.397649in}}{\pgfqpoint{5.462879in}{2.387050in}}{\pgfqpoint{5.462879in}{2.376000in}}%
\pgfpathcurveto{\pgfqpoint{5.462879in}{2.364950in}}{\pgfqpoint{5.467269in}{2.354351in}}{\pgfqpoint{5.475083in}{2.346537in}}%
\pgfpathcurveto{\pgfqpoint{5.482896in}{2.338724in}}{\pgfqpoint{5.493495in}{2.334333in}}{\pgfqpoint{5.504545in}{2.334333in}}%
\pgfpathclose%
\pgfusepath{stroke,fill}%
\end{pgfscope}%
\begin{pgfscope}%
\pgfpathrectangle{\pgfqpoint{0.800000in}{0.528000in}}{\pgfqpoint{4.960000in}{3.696000in}}%
\pgfusepath{clip}%
\pgfsetbuttcap%
\pgfsetroundjoin%
\definecolor{currentfill}{rgb}{0.000000,0.000000,0.000000}%
\pgfsetfillcolor{currentfill}%
\pgfsetlinewidth{1.003750pt}%
\definecolor{currentstroke}{rgb}{0.000000,0.000000,0.000000}%
\pgfsetstrokecolor{currentstroke}%
\pgfsetdash{}{0pt}%
\pgfpathmoveto{\pgfqpoint{5.504545in}{2.334333in}}%
\pgfpathcurveto{\pgfqpoint{5.515596in}{2.334333in}}{\pgfqpoint{5.526195in}{2.338724in}}{\pgfqpoint{5.534008in}{2.346537in}}%
\pgfpathcurveto{\pgfqpoint{5.541822in}{2.354351in}}{\pgfqpoint{5.546212in}{2.364950in}}{\pgfqpoint{5.546212in}{2.376000in}}%
\pgfpathcurveto{\pgfqpoint{5.546212in}{2.387050in}}{\pgfqpoint{5.541822in}{2.397649in}}{\pgfqpoint{5.534008in}{2.405463in}}%
\pgfpathcurveto{\pgfqpoint{5.526195in}{2.413276in}}{\pgfqpoint{5.515596in}{2.417667in}}{\pgfqpoint{5.504545in}{2.417667in}}%
\pgfpathcurveto{\pgfqpoint{5.493495in}{2.417667in}}{\pgfqpoint{5.482896in}{2.413276in}}{\pgfqpoint{5.475083in}{2.405463in}}%
\pgfpathcurveto{\pgfqpoint{5.467269in}{2.397649in}}{\pgfqpoint{5.462879in}{2.387050in}}{\pgfqpoint{5.462879in}{2.376000in}}%
\pgfpathcurveto{\pgfqpoint{5.462879in}{2.364950in}}{\pgfqpoint{5.467269in}{2.354351in}}{\pgfqpoint{5.475083in}{2.346537in}}%
\pgfpathcurveto{\pgfqpoint{5.482896in}{2.338724in}}{\pgfqpoint{5.493495in}{2.334333in}}{\pgfqpoint{5.504545in}{2.334333in}}%
\pgfpathclose%
\pgfusepath{stroke,fill}%
\end{pgfscope}%
\begin{pgfscope}%
\pgfpathrectangle{\pgfqpoint{0.800000in}{0.528000in}}{\pgfqpoint{4.960000in}{3.696000in}}%
\pgfusepath{clip}%
\pgfsetbuttcap%
\pgfsetroundjoin%
\definecolor{currentfill}{rgb}{0.000000,0.000000,0.000000}%
\pgfsetfillcolor{currentfill}%
\pgfsetlinewidth{1.003750pt}%
\definecolor{currentstroke}{rgb}{0.000000,0.000000,0.000000}%
\pgfsetstrokecolor{currentstroke}%
\pgfsetdash{}{0pt}%
\pgfpathmoveto{\pgfqpoint{5.504545in}{2.334333in}}%
\pgfpathcurveto{\pgfqpoint{5.515596in}{2.334333in}}{\pgfqpoint{5.526195in}{2.338724in}}{\pgfqpoint{5.534008in}{2.346537in}}%
\pgfpathcurveto{\pgfqpoint{5.541822in}{2.354351in}}{\pgfqpoint{5.546212in}{2.364950in}}{\pgfqpoint{5.546212in}{2.376000in}}%
\pgfpathcurveto{\pgfqpoint{5.546212in}{2.387050in}}{\pgfqpoint{5.541822in}{2.397649in}}{\pgfqpoint{5.534008in}{2.405463in}}%
\pgfpathcurveto{\pgfqpoint{5.526195in}{2.413276in}}{\pgfqpoint{5.515596in}{2.417667in}}{\pgfqpoint{5.504545in}{2.417667in}}%
\pgfpathcurveto{\pgfqpoint{5.493495in}{2.417667in}}{\pgfqpoint{5.482896in}{2.413276in}}{\pgfqpoint{5.475083in}{2.405463in}}%
\pgfpathcurveto{\pgfqpoint{5.467269in}{2.397649in}}{\pgfqpoint{5.462879in}{2.387050in}}{\pgfqpoint{5.462879in}{2.376000in}}%
\pgfpathcurveto{\pgfqpoint{5.462879in}{2.364950in}}{\pgfqpoint{5.467269in}{2.354351in}}{\pgfqpoint{5.475083in}{2.346537in}}%
\pgfpathcurveto{\pgfqpoint{5.482896in}{2.338724in}}{\pgfqpoint{5.493495in}{2.334333in}}{\pgfqpoint{5.504545in}{2.334333in}}%
\pgfpathclose%
\pgfusepath{stroke,fill}%
\end{pgfscope}%
\begin{pgfscope}%
\pgfpathrectangle{\pgfqpoint{0.800000in}{0.528000in}}{\pgfqpoint{4.960000in}{3.696000in}}%
\pgfusepath{clip}%
\pgfsetbuttcap%
\pgfsetroundjoin%
\definecolor{currentfill}{rgb}{0.000000,0.000000,0.000000}%
\pgfsetfillcolor{currentfill}%
\pgfsetlinewidth{1.003750pt}%
\definecolor{currentstroke}{rgb}{0.000000,0.000000,0.000000}%
\pgfsetstrokecolor{currentstroke}%
\pgfsetdash{}{0pt}%
\pgfpathmoveto{\pgfqpoint{5.504545in}{2.334333in}}%
\pgfpathcurveto{\pgfqpoint{5.515596in}{2.334333in}}{\pgfqpoint{5.526195in}{2.338724in}}{\pgfqpoint{5.534008in}{2.346537in}}%
\pgfpathcurveto{\pgfqpoint{5.541822in}{2.354351in}}{\pgfqpoint{5.546212in}{2.364950in}}{\pgfqpoint{5.546212in}{2.376000in}}%
\pgfpathcurveto{\pgfqpoint{5.546212in}{2.387050in}}{\pgfqpoint{5.541822in}{2.397649in}}{\pgfqpoint{5.534008in}{2.405463in}}%
\pgfpathcurveto{\pgfqpoint{5.526195in}{2.413276in}}{\pgfqpoint{5.515596in}{2.417667in}}{\pgfqpoint{5.504545in}{2.417667in}}%
\pgfpathcurveto{\pgfqpoint{5.493495in}{2.417667in}}{\pgfqpoint{5.482896in}{2.413276in}}{\pgfqpoint{5.475083in}{2.405463in}}%
\pgfpathcurveto{\pgfqpoint{5.467269in}{2.397649in}}{\pgfqpoint{5.462879in}{2.387050in}}{\pgfqpoint{5.462879in}{2.376000in}}%
\pgfpathcurveto{\pgfqpoint{5.462879in}{2.364950in}}{\pgfqpoint{5.467269in}{2.354351in}}{\pgfqpoint{5.475083in}{2.346537in}}%
\pgfpathcurveto{\pgfqpoint{5.482896in}{2.338724in}}{\pgfqpoint{5.493495in}{2.334333in}}{\pgfqpoint{5.504545in}{2.334333in}}%
\pgfpathclose%
\pgfusepath{stroke,fill}%
\end{pgfscope}%
\begin{pgfscope}%
\pgfpathrectangle{\pgfqpoint{0.800000in}{0.528000in}}{\pgfqpoint{4.960000in}{3.696000in}}%
\pgfusepath{clip}%
\pgfsetbuttcap%
\pgfsetroundjoin%
\definecolor{currentfill}{rgb}{0.000000,0.000000,0.000000}%
\pgfsetfillcolor{currentfill}%
\pgfsetlinewidth{1.003750pt}%
\definecolor{currentstroke}{rgb}{0.000000,0.000000,0.000000}%
\pgfsetstrokecolor{currentstroke}%
\pgfsetdash{}{0pt}%
\pgfpathmoveto{\pgfqpoint{5.504545in}{2.334333in}}%
\pgfpathcurveto{\pgfqpoint{5.515596in}{2.334333in}}{\pgfqpoint{5.526195in}{2.338724in}}{\pgfqpoint{5.534008in}{2.346537in}}%
\pgfpathcurveto{\pgfqpoint{5.541822in}{2.354351in}}{\pgfqpoint{5.546212in}{2.364950in}}{\pgfqpoint{5.546212in}{2.376000in}}%
\pgfpathcurveto{\pgfqpoint{5.546212in}{2.387050in}}{\pgfqpoint{5.541822in}{2.397649in}}{\pgfqpoint{5.534008in}{2.405463in}}%
\pgfpathcurveto{\pgfqpoint{5.526195in}{2.413276in}}{\pgfqpoint{5.515596in}{2.417667in}}{\pgfqpoint{5.504545in}{2.417667in}}%
\pgfpathcurveto{\pgfqpoint{5.493495in}{2.417667in}}{\pgfqpoint{5.482896in}{2.413276in}}{\pgfqpoint{5.475083in}{2.405463in}}%
\pgfpathcurveto{\pgfqpoint{5.467269in}{2.397649in}}{\pgfqpoint{5.462879in}{2.387050in}}{\pgfqpoint{5.462879in}{2.376000in}}%
\pgfpathcurveto{\pgfqpoint{5.462879in}{2.364950in}}{\pgfqpoint{5.467269in}{2.354351in}}{\pgfqpoint{5.475083in}{2.346537in}}%
\pgfpathcurveto{\pgfqpoint{5.482896in}{2.338724in}}{\pgfqpoint{5.493495in}{2.334333in}}{\pgfqpoint{5.504545in}{2.334333in}}%
\pgfpathclose%
\pgfusepath{stroke,fill}%
\end{pgfscope}%
\begin{pgfscope}%
\pgfpathrectangle{\pgfqpoint{0.800000in}{0.528000in}}{\pgfqpoint{4.960000in}{3.696000in}}%
\pgfusepath{clip}%
\pgfsetbuttcap%
\pgfsetroundjoin%
\definecolor{currentfill}{rgb}{0.000000,0.000000,0.000000}%
\pgfsetfillcolor{currentfill}%
\pgfsetlinewidth{1.003750pt}%
\definecolor{currentstroke}{rgb}{0.000000,0.000000,0.000000}%
\pgfsetstrokecolor{currentstroke}%
\pgfsetdash{}{0pt}%
\pgfpathmoveto{\pgfqpoint{5.504545in}{2.334333in}}%
\pgfpathcurveto{\pgfqpoint{5.515596in}{2.334333in}}{\pgfqpoint{5.526195in}{2.338724in}}{\pgfqpoint{5.534008in}{2.346537in}}%
\pgfpathcurveto{\pgfqpoint{5.541822in}{2.354351in}}{\pgfqpoint{5.546212in}{2.364950in}}{\pgfqpoint{5.546212in}{2.376000in}}%
\pgfpathcurveto{\pgfqpoint{5.546212in}{2.387050in}}{\pgfqpoint{5.541822in}{2.397649in}}{\pgfqpoint{5.534008in}{2.405463in}}%
\pgfpathcurveto{\pgfqpoint{5.526195in}{2.413276in}}{\pgfqpoint{5.515596in}{2.417667in}}{\pgfqpoint{5.504545in}{2.417667in}}%
\pgfpathcurveto{\pgfqpoint{5.493495in}{2.417667in}}{\pgfqpoint{5.482896in}{2.413276in}}{\pgfqpoint{5.475083in}{2.405463in}}%
\pgfpathcurveto{\pgfqpoint{5.467269in}{2.397649in}}{\pgfqpoint{5.462879in}{2.387050in}}{\pgfqpoint{5.462879in}{2.376000in}}%
\pgfpathcurveto{\pgfqpoint{5.462879in}{2.364950in}}{\pgfqpoint{5.467269in}{2.354351in}}{\pgfqpoint{5.475083in}{2.346537in}}%
\pgfpathcurveto{\pgfqpoint{5.482896in}{2.338724in}}{\pgfqpoint{5.493495in}{2.334333in}}{\pgfqpoint{5.504545in}{2.334333in}}%
\pgfpathclose%
\pgfusepath{stroke,fill}%
\end{pgfscope}%
\begin{pgfscope}%
\pgfpathrectangle{\pgfqpoint{0.800000in}{0.528000in}}{\pgfqpoint{4.960000in}{3.696000in}}%
\pgfusepath{clip}%
\pgfsetbuttcap%
\pgfsetroundjoin%
\definecolor{currentfill}{rgb}{0.000000,0.000000,0.000000}%
\pgfsetfillcolor{currentfill}%
\pgfsetlinewidth{1.003750pt}%
\definecolor{currentstroke}{rgb}{0.000000,0.000000,0.000000}%
\pgfsetstrokecolor{currentstroke}%
\pgfsetdash{}{0pt}%
\pgfpathmoveto{\pgfqpoint{5.504545in}{2.334333in}}%
\pgfpathcurveto{\pgfqpoint{5.515596in}{2.334333in}}{\pgfqpoint{5.526195in}{2.338724in}}{\pgfqpoint{5.534008in}{2.346537in}}%
\pgfpathcurveto{\pgfqpoint{5.541822in}{2.354351in}}{\pgfqpoint{5.546212in}{2.364950in}}{\pgfqpoint{5.546212in}{2.376000in}}%
\pgfpathcurveto{\pgfqpoint{5.546212in}{2.387050in}}{\pgfqpoint{5.541822in}{2.397649in}}{\pgfqpoint{5.534008in}{2.405463in}}%
\pgfpathcurveto{\pgfqpoint{5.526195in}{2.413276in}}{\pgfqpoint{5.515596in}{2.417667in}}{\pgfqpoint{5.504545in}{2.417667in}}%
\pgfpathcurveto{\pgfqpoint{5.493495in}{2.417667in}}{\pgfqpoint{5.482896in}{2.413276in}}{\pgfqpoint{5.475083in}{2.405463in}}%
\pgfpathcurveto{\pgfqpoint{5.467269in}{2.397649in}}{\pgfqpoint{5.462879in}{2.387050in}}{\pgfqpoint{5.462879in}{2.376000in}}%
\pgfpathcurveto{\pgfqpoint{5.462879in}{2.364950in}}{\pgfqpoint{5.467269in}{2.354351in}}{\pgfqpoint{5.475083in}{2.346537in}}%
\pgfpathcurveto{\pgfqpoint{5.482896in}{2.338724in}}{\pgfqpoint{5.493495in}{2.334333in}}{\pgfqpoint{5.504545in}{2.334333in}}%
\pgfpathclose%
\pgfusepath{stroke,fill}%
\end{pgfscope}%
\begin{pgfscope}%
\pgfpathrectangle{\pgfqpoint{0.800000in}{0.528000in}}{\pgfqpoint{4.960000in}{3.696000in}}%
\pgfusepath{clip}%
\pgfsetbuttcap%
\pgfsetroundjoin%
\definecolor{currentfill}{rgb}{0.000000,0.000000,0.000000}%
\pgfsetfillcolor{currentfill}%
\pgfsetlinewidth{1.003750pt}%
\definecolor{currentstroke}{rgb}{0.000000,0.000000,0.000000}%
\pgfsetstrokecolor{currentstroke}%
\pgfsetdash{}{0pt}%
\pgfpathmoveto{\pgfqpoint{5.504545in}{2.334333in}}%
\pgfpathcurveto{\pgfqpoint{5.515596in}{2.334333in}}{\pgfqpoint{5.526195in}{2.338724in}}{\pgfqpoint{5.534008in}{2.346537in}}%
\pgfpathcurveto{\pgfqpoint{5.541822in}{2.354351in}}{\pgfqpoint{5.546212in}{2.364950in}}{\pgfqpoint{5.546212in}{2.376000in}}%
\pgfpathcurveto{\pgfqpoint{5.546212in}{2.387050in}}{\pgfqpoint{5.541822in}{2.397649in}}{\pgfqpoint{5.534008in}{2.405463in}}%
\pgfpathcurveto{\pgfqpoint{5.526195in}{2.413276in}}{\pgfqpoint{5.515596in}{2.417667in}}{\pgfqpoint{5.504545in}{2.417667in}}%
\pgfpathcurveto{\pgfqpoint{5.493495in}{2.417667in}}{\pgfqpoint{5.482896in}{2.413276in}}{\pgfqpoint{5.475083in}{2.405463in}}%
\pgfpathcurveto{\pgfqpoint{5.467269in}{2.397649in}}{\pgfqpoint{5.462879in}{2.387050in}}{\pgfqpoint{5.462879in}{2.376000in}}%
\pgfpathcurveto{\pgfqpoint{5.462879in}{2.364950in}}{\pgfqpoint{5.467269in}{2.354351in}}{\pgfqpoint{5.475083in}{2.346537in}}%
\pgfpathcurveto{\pgfqpoint{5.482896in}{2.338724in}}{\pgfqpoint{5.493495in}{2.334333in}}{\pgfqpoint{5.504545in}{2.334333in}}%
\pgfpathclose%
\pgfusepath{stroke,fill}%
\end{pgfscope}%
\begin{pgfscope}%
\pgfpathrectangle{\pgfqpoint{0.800000in}{0.528000in}}{\pgfqpoint{4.960000in}{3.696000in}}%
\pgfusepath{clip}%
\pgfsetbuttcap%
\pgfsetroundjoin%
\definecolor{currentfill}{rgb}{0.000000,0.000000,0.000000}%
\pgfsetfillcolor{currentfill}%
\pgfsetlinewidth{1.003750pt}%
\definecolor{currentstroke}{rgb}{0.000000,0.000000,0.000000}%
\pgfsetstrokecolor{currentstroke}%
\pgfsetdash{}{0pt}%
\pgfpathmoveto{\pgfqpoint{5.504545in}{2.334333in}}%
\pgfpathcurveto{\pgfqpoint{5.515596in}{2.334333in}}{\pgfqpoint{5.526195in}{2.338724in}}{\pgfqpoint{5.534008in}{2.346537in}}%
\pgfpathcurveto{\pgfqpoint{5.541822in}{2.354351in}}{\pgfqpoint{5.546212in}{2.364950in}}{\pgfqpoint{5.546212in}{2.376000in}}%
\pgfpathcurveto{\pgfqpoint{5.546212in}{2.387050in}}{\pgfqpoint{5.541822in}{2.397649in}}{\pgfqpoint{5.534008in}{2.405463in}}%
\pgfpathcurveto{\pgfqpoint{5.526195in}{2.413276in}}{\pgfqpoint{5.515596in}{2.417667in}}{\pgfqpoint{5.504545in}{2.417667in}}%
\pgfpathcurveto{\pgfqpoint{5.493495in}{2.417667in}}{\pgfqpoint{5.482896in}{2.413276in}}{\pgfqpoint{5.475083in}{2.405463in}}%
\pgfpathcurveto{\pgfqpoint{5.467269in}{2.397649in}}{\pgfqpoint{5.462879in}{2.387050in}}{\pgfqpoint{5.462879in}{2.376000in}}%
\pgfpathcurveto{\pgfqpoint{5.462879in}{2.364950in}}{\pgfqpoint{5.467269in}{2.354351in}}{\pgfqpoint{5.475083in}{2.346537in}}%
\pgfpathcurveto{\pgfqpoint{5.482896in}{2.338724in}}{\pgfqpoint{5.493495in}{2.334333in}}{\pgfqpoint{5.504545in}{2.334333in}}%
\pgfpathclose%
\pgfusepath{stroke,fill}%
\end{pgfscope}%
\begin{pgfscope}%
\pgfpathrectangle{\pgfqpoint{0.800000in}{0.528000in}}{\pgfqpoint{4.960000in}{3.696000in}}%
\pgfusepath{clip}%
\pgfsetbuttcap%
\pgfsetroundjoin%
\definecolor{currentfill}{rgb}{0.000000,0.000000,0.000000}%
\pgfsetfillcolor{currentfill}%
\pgfsetlinewidth{1.003750pt}%
\definecolor{currentstroke}{rgb}{0.000000,0.000000,0.000000}%
\pgfsetstrokecolor{currentstroke}%
\pgfsetdash{}{0pt}%
\pgfpathmoveto{\pgfqpoint{5.504545in}{2.334333in}}%
\pgfpathcurveto{\pgfqpoint{5.515596in}{2.334333in}}{\pgfqpoint{5.526195in}{2.338724in}}{\pgfqpoint{5.534008in}{2.346537in}}%
\pgfpathcurveto{\pgfqpoint{5.541822in}{2.354351in}}{\pgfqpoint{5.546212in}{2.364950in}}{\pgfqpoint{5.546212in}{2.376000in}}%
\pgfpathcurveto{\pgfqpoint{5.546212in}{2.387050in}}{\pgfqpoint{5.541822in}{2.397649in}}{\pgfqpoint{5.534008in}{2.405463in}}%
\pgfpathcurveto{\pgfqpoint{5.526195in}{2.413276in}}{\pgfqpoint{5.515596in}{2.417667in}}{\pgfqpoint{5.504545in}{2.417667in}}%
\pgfpathcurveto{\pgfqpoint{5.493495in}{2.417667in}}{\pgfqpoint{5.482896in}{2.413276in}}{\pgfqpoint{5.475083in}{2.405463in}}%
\pgfpathcurveto{\pgfqpoint{5.467269in}{2.397649in}}{\pgfqpoint{5.462879in}{2.387050in}}{\pgfqpoint{5.462879in}{2.376000in}}%
\pgfpathcurveto{\pgfqpoint{5.462879in}{2.364950in}}{\pgfqpoint{5.467269in}{2.354351in}}{\pgfqpoint{5.475083in}{2.346537in}}%
\pgfpathcurveto{\pgfqpoint{5.482896in}{2.338724in}}{\pgfqpoint{5.493495in}{2.334333in}}{\pgfqpoint{5.504545in}{2.334333in}}%
\pgfpathclose%
\pgfusepath{stroke,fill}%
\end{pgfscope}%
\begin{pgfscope}%
\pgfsetbuttcap%
\pgfsetroundjoin%
\definecolor{currentfill}{rgb}{0.000000,0.000000,0.000000}%
\pgfsetfillcolor{currentfill}%
\pgfsetlinewidth{0.803000pt}%
\definecolor{currentstroke}{rgb}{0.000000,0.000000,0.000000}%
\pgfsetstrokecolor{currentstroke}%
\pgfsetdash{}{0pt}%
\pgfsys@defobject{currentmarker}{\pgfqpoint{0.000000in}{-0.048611in}}{\pgfqpoint{0.000000in}{0.000000in}}{%
\pgfpathmoveto{\pgfqpoint{0.000000in}{0.000000in}}%
\pgfpathlineto{\pgfqpoint{0.000000in}{-0.048611in}}%
\pgfusepath{stroke,fill}%
}%
\begin{pgfscope}%
\pgfsys@transformshift{1.025793in}{0.528000in}%
\pgfsys@useobject{currentmarker}{}%
\end{pgfscope}%
\end{pgfscope}%
\begin{pgfscope}%
\definecolor{textcolor}{rgb}{0.000000,0.000000,0.000000}%
\pgfsetstrokecolor{textcolor}%
\pgfsetfillcolor{textcolor}%
\pgftext[x=1.025793in,y=0.430778in,,top]{\color{textcolor}\sffamily\fontsize{10.000000}{12.000000}\selectfont 20}%
\end{pgfscope}%
\begin{pgfscope}%
\pgfsetbuttcap%
\pgfsetroundjoin%
\definecolor{currentfill}{rgb}{0.000000,0.000000,0.000000}%
\pgfsetfillcolor{currentfill}%
\pgfsetlinewidth{0.803000pt}%
\definecolor{currentstroke}{rgb}{0.000000,0.000000,0.000000}%
\pgfsetstrokecolor{currentstroke}%
\pgfsetdash{}{0pt}%
\pgfsys@defobject{currentmarker}{\pgfqpoint{0.000000in}{-0.048611in}}{\pgfqpoint{0.000000in}{0.000000in}}{%
\pgfpathmoveto{\pgfqpoint{0.000000in}{0.000000in}}%
\pgfpathlineto{\pgfqpoint{0.000000in}{-0.048611in}}%
\pgfusepath{stroke,fill}%
}%
\begin{pgfscope}%
\pgfsys@transformshift{2.145481in}{0.528000in}%
\pgfsys@useobject{currentmarker}{}%
\end{pgfscope}%
\end{pgfscope}%
\begin{pgfscope}%
\definecolor{textcolor}{rgb}{0.000000,0.000000,0.000000}%
\pgfsetstrokecolor{textcolor}%
\pgfsetfillcolor{textcolor}%
\pgftext[x=2.145481in,y=0.430778in,,top]{\color{textcolor}\sffamily\fontsize{10.000000}{12.000000}\selectfont 40}%
\end{pgfscope}%
\begin{pgfscope}%
\pgfsetbuttcap%
\pgfsetroundjoin%
\definecolor{currentfill}{rgb}{0.000000,0.000000,0.000000}%
\pgfsetfillcolor{currentfill}%
\pgfsetlinewidth{0.803000pt}%
\definecolor{currentstroke}{rgb}{0.000000,0.000000,0.000000}%
\pgfsetstrokecolor{currentstroke}%
\pgfsetdash{}{0pt}%
\pgfsys@defobject{currentmarker}{\pgfqpoint{0.000000in}{-0.048611in}}{\pgfqpoint{0.000000in}{0.000000in}}{%
\pgfpathmoveto{\pgfqpoint{0.000000in}{0.000000in}}%
\pgfpathlineto{\pgfqpoint{0.000000in}{-0.048611in}}%
\pgfusepath{stroke,fill}%
}%
\begin{pgfscope}%
\pgfsys@transformshift{3.265169in}{0.528000in}%
\pgfsys@useobject{currentmarker}{}%
\end{pgfscope}%
\end{pgfscope}%
\begin{pgfscope}%
\definecolor{textcolor}{rgb}{0.000000,0.000000,0.000000}%
\pgfsetstrokecolor{textcolor}%
\pgfsetfillcolor{textcolor}%
\pgftext[x=3.265169in,y=0.430778in,,top]{\color{textcolor}\sffamily\fontsize{10.000000}{12.000000}\selectfont 60}%
\end{pgfscope}%
\begin{pgfscope}%
\pgfsetbuttcap%
\pgfsetroundjoin%
\definecolor{currentfill}{rgb}{0.000000,0.000000,0.000000}%
\pgfsetfillcolor{currentfill}%
\pgfsetlinewidth{0.803000pt}%
\definecolor{currentstroke}{rgb}{0.000000,0.000000,0.000000}%
\pgfsetstrokecolor{currentstroke}%
\pgfsetdash{}{0pt}%
\pgfsys@defobject{currentmarker}{\pgfqpoint{0.000000in}{-0.048611in}}{\pgfqpoint{0.000000in}{0.000000in}}{%
\pgfpathmoveto{\pgfqpoint{0.000000in}{0.000000in}}%
\pgfpathlineto{\pgfqpoint{0.000000in}{-0.048611in}}%
\pgfusepath{stroke,fill}%
}%
\begin{pgfscope}%
\pgfsys@transformshift{4.384857in}{0.528000in}%
\pgfsys@useobject{currentmarker}{}%
\end{pgfscope}%
\end{pgfscope}%
\begin{pgfscope}%
\definecolor{textcolor}{rgb}{0.000000,0.000000,0.000000}%
\pgfsetstrokecolor{textcolor}%
\pgfsetfillcolor{textcolor}%
\pgftext[x=4.384857in,y=0.430778in,,top]{\color{textcolor}\sffamily\fontsize{10.000000}{12.000000}\selectfont 80}%
\end{pgfscope}%
\begin{pgfscope}%
\pgfsetbuttcap%
\pgfsetroundjoin%
\definecolor{currentfill}{rgb}{0.000000,0.000000,0.000000}%
\pgfsetfillcolor{currentfill}%
\pgfsetlinewidth{0.803000pt}%
\definecolor{currentstroke}{rgb}{0.000000,0.000000,0.000000}%
\pgfsetstrokecolor{currentstroke}%
\pgfsetdash{}{0pt}%
\pgfsys@defobject{currentmarker}{\pgfqpoint{0.000000in}{-0.048611in}}{\pgfqpoint{0.000000in}{0.000000in}}{%
\pgfpathmoveto{\pgfqpoint{0.000000in}{0.000000in}}%
\pgfpathlineto{\pgfqpoint{0.000000in}{-0.048611in}}%
\pgfusepath{stroke,fill}%
}%
\begin{pgfscope}%
\pgfsys@transformshift{5.504545in}{0.528000in}%
\pgfsys@useobject{currentmarker}{}%
\end{pgfscope}%
\end{pgfscope}%
\begin{pgfscope}%
\definecolor{textcolor}{rgb}{0.000000,0.000000,0.000000}%
\pgfsetstrokecolor{textcolor}%
\pgfsetfillcolor{textcolor}%
\pgftext[x=5.504545in,y=0.430778in,,top]{\color{textcolor}\sffamily\fontsize{10.000000}{12.000000}\selectfont 100}%
\end{pgfscope}%
\begin{pgfscope}%
\definecolor{textcolor}{rgb}{0.000000,0.000000,0.000000}%
\pgfsetstrokecolor{textcolor}%
\pgfsetfillcolor{textcolor}%
\pgftext[x=3.280000in,y=0.240809in,,top]{\color{textcolor}\sffamily\fontsize{10.000000}{12.000000}\selectfont \(\displaystyle k\)}%
\end{pgfscope}%
\begin{pgfscope}%
\pgfsetbuttcap%
\pgfsetroundjoin%
\definecolor{currentfill}{rgb}{0.000000,0.000000,0.000000}%
\pgfsetfillcolor{currentfill}%
\pgfsetlinewidth{0.803000pt}%
\definecolor{currentstroke}{rgb}{0.000000,0.000000,0.000000}%
\pgfsetstrokecolor{currentstroke}%
\pgfsetdash{}{0pt}%
\pgfsys@defobject{currentmarker}{\pgfqpoint{-0.048611in}{0.000000in}}{\pgfqpoint{0.000000in}{0.000000in}}{%
\pgfpathmoveto{\pgfqpoint{0.000000in}{0.000000in}}%
\pgfpathlineto{\pgfqpoint{-0.048611in}{0.000000in}}%
\pgfusepath{stroke,fill}%
}%
\begin{pgfscope}%
\pgfsys@transformshift{0.800000in}{0.720192in}%
\pgfsys@useobject{currentmarker}{}%
\end{pgfscope}%
\end{pgfscope}%
\begin{pgfscope}%
\definecolor{textcolor}{rgb}{0.000000,0.000000,0.000000}%
\pgfsetstrokecolor{textcolor}%
\pgfsetfillcolor{textcolor}%
\pgftext[x=0.305168in,y=0.667430in,left,base]{\color{textcolor}\sffamily\fontsize{10.000000}{12.000000}\selectfont 3.992}%
\end{pgfscope}%
\begin{pgfscope}%
\pgfsetbuttcap%
\pgfsetroundjoin%
\definecolor{currentfill}{rgb}{0.000000,0.000000,0.000000}%
\pgfsetfillcolor{currentfill}%
\pgfsetlinewidth{0.803000pt}%
\definecolor{currentstroke}{rgb}{0.000000,0.000000,0.000000}%
\pgfsetstrokecolor{currentstroke}%
\pgfsetdash{}{0pt}%
\pgfsys@defobject{currentmarker}{\pgfqpoint{-0.048611in}{0.000000in}}{\pgfqpoint{0.000000in}{0.000000in}}{%
\pgfpathmoveto{\pgfqpoint{0.000000in}{0.000000in}}%
\pgfpathlineto{\pgfqpoint{-0.048611in}{0.000000in}}%
\pgfusepath{stroke,fill}%
}%
\begin{pgfscope}%
\pgfsys@transformshift{0.800000in}{1.134144in}%
\pgfsys@useobject{currentmarker}{}%
\end{pgfscope}%
\end{pgfscope}%
\begin{pgfscope}%
\definecolor{textcolor}{rgb}{0.000000,0.000000,0.000000}%
\pgfsetstrokecolor{textcolor}%
\pgfsetfillcolor{textcolor}%
\pgftext[x=0.305168in,y=1.081382in,left,base]{\color{textcolor}\sffamily\fontsize{10.000000}{12.000000}\selectfont 3.994}%
\end{pgfscope}%
\begin{pgfscope}%
\pgfsetbuttcap%
\pgfsetroundjoin%
\definecolor{currentfill}{rgb}{0.000000,0.000000,0.000000}%
\pgfsetfillcolor{currentfill}%
\pgfsetlinewidth{0.803000pt}%
\definecolor{currentstroke}{rgb}{0.000000,0.000000,0.000000}%
\pgfsetstrokecolor{currentstroke}%
\pgfsetdash{}{0pt}%
\pgfsys@defobject{currentmarker}{\pgfqpoint{-0.048611in}{0.000000in}}{\pgfqpoint{0.000000in}{0.000000in}}{%
\pgfpathmoveto{\pgfqpoint{0.000000in}{0.000000in}}%
\pgfpathlineto{\pgfqpoint{-0.048611in}{0.000000in}}%
\pgfusepath{stroke,fill}%
}%
\begin{pgfscope}%
\pgfsys@transformshift{0.800000in}{1.548096in}%
\pgfsys@useobject{currentmarker}{}%
\end{pgfscope}%
\end{pgfscope}%
\begin{pgfscope}%
\definecolor{textcolor}{rgb}{0.000000,0.000000,0.000000}%
\pgfsetstrokecolor{textcolor}%
\pgfsetfillcolor{textcolor}%
\pgftext[x=0.305168in,y=1.495334in,left,base]{\color{textcolor}\sffamily\fontsize{10.000000}{12.000000}\selectfont 3.996}%
\end{pgfscope}%
\begin{pgfscope}%
\pgfsetbuttcap%
\pgfsetroundjoin%
\definecolor{currentfill}{rgb}{0.000000,0.000000,0.000000}%
\pgfsetfillcolor{currentfill}%
\pgfsetlinewidth{0.803000pt}%
\definecolor{currentstroke}{rgb}{0.000000,0.000000,0.000000}%
\pgfsetstrokecolor{currentstroke}%
\pgfsetdash{}{0pt}%
\pgfsys@defobject{currentmarker}{\pgfqpoint{-0.048611in}{0.000000in}}{\pgfqpoint{0.000000in}{0.000000in}}{%
\pgfpathmoveto{\pgfqpoint{0.000000in}{0.000000in}}%
\pgfpathlineto{\pgfqpoint{-0.048611in}{0.000000in}}%
\pgfusepath{stroke,fill}%
}%
\begin{pgfscope}%
\pgfsys@transformshift{0.800000in}{1.962048in}%
\pgfsys@useobject{currentmarker}{}%
\end{pgfscope}%
\end{pgfscope}%
\begin{pgfscope}%
\definecolor{textcolor}{rgb}{0.000000,0.000000,0.000000}%
\pgfsetstrokecolor{textcolor}%
\pgfsetfillcolor{textcolor}%
\pgftext[x=0.305168in,y=1.909286in,left,base]{\color{textcolor}\sffamily\fontsize{10.000000}{12.000000}\selectfont 3.998}%
\end{pgfscope}%
\begin{pgfscope}%
\pgfsetbuttcap%
\pgfsetroundjoin%
\definecolor{currentfill}{rgb}{0.000000,0.000000,0.000000}%
\pgfsetfillcolor{currentfill}%
\pgfsetlinewidth{0.803000pt}%
\definecolor{currentstroke}{rgb}{0.000000,0.000000,0.000000}%
\pgfsetstrokecolor{currentstroke}%
\pgfsetdash{}{0pt}%
\pgfsys@defobject{currentmarker}{\pgfqpoint{-0.048611in}{0.000000in}}{\pgfqpoint{0.000000in}{0.000000in}}{%
\pgfpathmoveto{\pgfqpoint{0.000000in}{0.000000in}}%
\pgfpathlineto{\pgfqpoint{-0.048611in}{0.000000in}}%
\pgfusepath{stroke,fill}%
}%
\begin{pgfscope}%
\pgfsys@transformshift{0.800000in}{2.376000in}%
\pgfsys@useobject{currentmarker}{}%
\end{pgfscope}%
\end{pgfscope}%
\begin{pgfscope}%
\definecolor{textcolor}{rgb}{0.000000,0.000000,0.000000}%
\pgfsetstrokecolor{textcolor}%
\pgfsetfillcolor{textcolor}%
\pgftext[x=0.305168in,y=2.323238in,left,base]{\color{textcolor}\sffamily\fontsize{10.000000}{12.000000}\selectfont 4.000}%
\end{pgfscope}%
\begin{pgfscope}%
\pgfsetbuttcap%
\pgfsetroundjoin%
\definecolor{currentfill}{rgb}{0.000000,0.000000,0.000000}%
\pgfsetfillcolor{currentfill}%
\pgfsetlinewidth{0.803000pt}%
\definecolor{currentstroke}{rgb}{0.000000,0.000000,0.000000}%
\pgfsetstrokecolor{currentstroke}%
\pgfsetdash{}{0pt}%
\pgfsys@defobject{currentmarker}{\pgfqpoint{-0.048611in}{0.000000in}}{\pgfqpoint{0.000000in}{0.000000in}}{%
\pgfpathmoveto{\pgfqpoint{0.000000in}{0.000000in}}%
\pgfpathlineto{\pgfqpoint{-0.048611in}{0.000000in}}%
\pgfusepath{stroke,fill}%
}%
\begin{pgfscope}%
\pgfsys@transformshift{0.800000in}{2.789952in}%
\pgfsys@useobject{currentmarker}{}%
\end{pgfscope}%
\end{pgfscope}%
\begin{pgfscope}%
\definecolor{textcolor}{rgb}{0.000000,0.000000,0.000000}%
\pgfsetstrokecolor{textcolor}%
\pgfsetfillcolor{textcolor}%
\pgftext[x=0.305168in,y=2.737190in,left,base]{\color{textcolor}\sffamily\fontsize{10.000000}{12.000000}\selectfont 4.002}%
\end{pgfscope}%
\begin{pgfscope}%
\pgfsetbuttcap%
\pgfsetroundjoin%
\definecolor{currentfill}{rgb}{0.000000,0.000000,0.000000}%
\pgfsetfillcolor{currentfill}%
\pgfsetlinewidth{0.803000pt}%
\definecolor{currentstroke}{rgb}{0.000000,0.000000,0.000000}%
\pgfsetstrokecolor{currentstroke}%
\pgfsetdash{}{0pt}%
\pgfsys@defobject{currentmarker}{\pgfqpoint{-0.048611in}{0.000000in}}{\pgfqpoint{0.000000in}{0.000000in}}{%
\pgfpathmoveto{\pgfqpoint{0.000000in}{0.000000in}}%
\pgfpathlineto{\pgfqpoint{-0.048611in}{0.000000in}}%
\pgfusepath{stroke,fill}%
}%
\begin{pgfscope}%
\pgfsys@transformshift{0.800000in}{3.203904in}%
\pgfsys@useobject{currentmarker}{}%
\end{pgfscope}%
\end{pgfscope}%
\begin{pgfscope}%
\definecolor{textcolor}{rgb}{0.000000,0.000000,0.000000}%
\pgfsetstrokecolor{textcolor}%
\pgfsetfillcolor{textcolor}%
\pgftext[x=0.305168in,y=3.151142in,left,base]{\color{textcolor}\sffamily\fontsize{10.000000}{12.000000}\selectfont 4.004}%
\end{pgfscope}%
\begin{pgfscope}%
\pgfsetbuttcap%
\pgfsetroundjoin%
\definecolor{currentfill}{rgb}{0.000000,0.000000,0.000000}%
\pgfsetfillcolor{currentfill}%
\pgfsetlinewidth{0.803000pt}%
\definecolor{currentstroke}{rgb}{0.000000,0.000000,0.000000}%
\pgfsetstrokecolor{currentstroke}%
\pgfsetdash{}{0pt}%
\pgfsys@defobject{currentmarker}{\pgfqpoint{-0.048611in}{0.000000in}}{\pgfqpoint{0.000000in}{0.000000in}}{%
\pgfpathmoveto{\pgfqpoint{0.000000in}{0.000000in}}%
\pgfpathlineto{\pgfqpoint{-0.048611in}{0.000000in}}%
\pgfusepath{stroke,fill}%
}%
\begin{pgfscope}%
\pgfsys@transformshift{0.800000in}{3.617856in}%
\pgfsys@useobject{currentmarker}{}%
\end{pgfscope}%
\end{pgfscope}%
\begin{pgfscope}%
\definecolor{textcolor}{rgb}{0.000000,0.000000,0.000000}%
\pgfsetstrokecolor{textcolor}%
\pgfsetfillcolor{textcolor}%
\pgftext[x=0.305168in,y=3.565094in,left,base]{\color{textcolor}\sffamily\fontsize{10.000000}{12.000000}\selectfont 4.006}%
\end{pgfscope}%
\begin{pgfscope}%
\pgfsetbuttcap%
\pgfsetroundjoin%
\definecolor{currentfill}{rgb}{0.000000,0.000000,0.000000}%
\pgfsetfillcolor{currentfill}%
\pgfsetlinewidth{0.803000pt}%
\definecolor{currentstroke}{rgb}{0.000000,0.000000,0.000000}%
\pgfsetstrokecolor{currentstroke}%
\pgfsetdash{}{0pt}%
\pgfsys@defobject{currentmarker}{\pgfqpoint{-0.048611in}{0.000000in}}{\pgfqpoint{0.000000in}{0.000000in}}{%
\pgfpathmoveto{\pgfqpoint{0.000000in}{0.000000in}}%
\pgfpathlineto{\pgfqpoint{-0.048611in}{0.000000in}}%
\pgfusepath{stroke,fill}%
}%
\begin{pgfscope}%
\pgfsys@transformshift{0.800000in}{4.031808in}%
\pgfsys@useobject{currentmarker}{}%
\end{pgfscope}%
\end{pgfscope}%
\begin{pgfscope}%
\definecolor{textcolor}{rgb}{0.000000,0.000000,0.000000}%
\pgfsetstrokecolor{textcolor}%
\pgfsetfillcolor{textcolor}%
\pgftext[x=0.305168in,y=3.979046in,left,base]{\color{textcolor}\sffamily\fontsize{10.000000}{12.000000}\selectfont 4.008}%
\end{pgfscope}%
\begin{pgfscope}%
\definecolor{textcolor}{rgb}{0.000000,0.000000,0.000000}%
\pgfsetstrokecolor{textcolor}%
\pgfsetfillcolor{textcolor}%
\pgftext[x=0.249612in,y=2.376000in,,bottom,rotate=90.000000]{\color{textcolor}\sffamily\fontsize{10.000000}{12.000000}\selectfont Number of GMRES Iterations}%
\end{pgfscope}%
\begin{pgfscope}%
\pgfsetrectcap%
\pgfsetmiterjoin%
\pgfsetlinewidth{0.803000pt}%
\definecolor{currentstroke}{rgb}{0.000000,0.000000,0.000000}%
\pgfsetstrokecolor{currentstroke}%
\pgfsetdash{}{0pt}%
\pgfpathmoveto{\pgfqpoint{0.800000in}{0.528000in}}%
\pgfpathlineto{\pgfqpoint{0.800000in}{4.224000in}}%
\pgfusepath{stroke}%
\end{pgfscope}%
\begin{pgfscope}%
\pgfsetrectcap%
\pgfsetmiterjoin%
\pgfsetlinewidth{0.803000pt}%
\definecolor{currentstroke}{rgb}{0.000000,0.000000,0.000000}%
\pgfsetstrokecolor{currentstroke}%
\pgfsetdash{}{0pt}%
\pgfpathmoveto{\pgfqpoint{5.760000in}{0.528000in}}%
\pgfpathlineto{\pgfqpoint{5.760000in}{4.224000in}}%
\pgfusepath{stroke}%
\end{pgfscope}%
\begin{pgfscope}%
\pgfsetrectcap%
\pgfsetmiterjoin%
\pgfsetlinewidth{0.803000pt}%
\definecolor{currentstroke}{rgb}{0.000000,0.000000,0.000000}%
\pgfsetstrokecolor{currentstroke}%
\pgfsetdash{}{0pt}%
\pgfpathmoveto{\pgfqpoint{0.800000in}{0.528000in}}%
\pgfpathlineto{\pgfqpoint{5.760000in}{0.528000in}}%
\pgfusepath{stroke}%
\end{pgfscope}%
\begin{pgfscope}%
\pgfsetrectcap%
\pgfsetmiterjoin%
\pgfsetlinewidth{0.803000pt}%
\definecolor{currentstroke}{rgb}{0.000000,0.000000,0.000000}%
\pgfsetstrokecolor{currentstroke}%
\pgfsetdash{}{0pt}%
\pgfpathmoveto{\pgfqpoint{0.800000in}{4.224000in}}%
\pgfpathlineto{\pgfqpoint{5.760000in}{4.224000in}}%
\pgfusepath{stroke}%
\end{pgfscope}%
\end{pgfpicture}%
\makeatother%
\endgroup%

  \caption[Maximum GMRES iteration counts when $\NLiDRRdtd{\Aso-\Ast} = 0.5\times  k^{-\beta}$ for $\beta = 0,0.1,0.2,0.3.$]{Maximum GMRES iteration counts for solving systems with matrix $\AmatoI\Amatt$, where $\nso=\nst=1$ and $\NLiDRRdtd{\Aso-\Ast} = 0.5\times  k^{-\beta}$ for $\beta = 0,0.1,0.2,0.3.$}\label{fig:linfinityA0}
    \end{figure}
    
    \begin{figure}
      \centering
%% Creator: Matplotlib, PGF backend
%%
%% To include the figure in your LaTeX document, write
%%   \input{<filename>.pgf}
%%
%% Make sure the required packages are loaded in your preamble
%%   \usepackage{pgf}
%%
%% Figures using additional raster images can only be included by \input if
%% they are in the same directory as the main LaTeX file. For loading figures
%% from other directories you can use the `import` package
%%   \usepackage{import}
%% and then include the figures with
%%   \import{<path to file>}{<filename>.pgf}
%%
%% Matplotlib used the following preamble
%%   \usepackage{fontspec}
%%   \setmainfont{DejaVuSerif.ttf}[Path=/home/owen/progs/firedrake-complex/firedrake/lib/python3.5/site-packages/matplotlib/mpl-data/fonts/ttf/]
%%   \setsansfont{DejaVuSans.ttf}[Path=/home/owen/progs/firedrake-complex/firedrake/lib/python3.5/site-packages/matplotlib/mpl-data/fonts/ttf/]
%%   \setmonofont{DejaVuSansMono.ttf}[Path=/home/owen/progs/firedrake-complex/firedrake/lib/python3.5/site-packages/matplotlib/mpl-data/fonts/ttf/]
%%
\begingroup%
\makeatletter%
\begin{pgfpicture}%
\pgfpathrectangle{\pgfpointorigin}{\pgfqpoint{3.000000in}{3.000000in}}%
\pgfusepath{use as bounding box, clip}%
\begin{pgfscope}%
\pgfsetbuttcap%
\pgfsetmiterjoin%
\definecolor{currentfill}{rgb}{1.000000,1.000000,1.000000}%
\pgfsetfillcolor{currentfill}%
\pgfsetlinewidth{0.000000pt}%
\definecolor{currentstroke}{rgb}{1.000000,1.000000,1.000000}%
\pgfsetstrokecolor{currentstroke}%
\pgfsetdash{}{0pt}%
\pgfpathmoveto{\pgfqpoint{0.000000in}{0.000000in}}%
\pgfpathlineto{\pgfqpoint{3.000000in}{0.000000in}}%
\pgfpathlineto{\pgfqpoint{3.000000in}{3.000000in}}%
\pgfpathlineto{\pgfqpoint{0.000000in}{3.000000in}}%
\pgfpathclose%
\pgfusepath{fill}%
\end{pgfscope}%
\begin{pgfscope}%
\pgfsetbuttcap%
\pgfsetmiterjoin%
\definecolor{currentfill}{rgb}{1.000000,1.000000,1.000000}%
\pgfsetfillcolor{currentfill}%
\pgfsetlinewidth{0.000000pt}%
\definecolor{currentstroke}{rgb}{0.000000,0.000000,0.000000}%
\pgfsetstrokecolor{currentstroke}%
\pgfsetstrokeopacity{0.000000}%
\pgfsetdash{}{0pt}%
\pgfpathmoveto{\pgfqpoint{0.375000in}{0.330000in}}%
\pgfpathlineto{\pgfqpoint{2.700000in}{0.330000in}}%
\pgfpathlineto{\pgfqpoint{2.700000in}{2.640000in}}%
\pgfpathlineto{\pgfqpoint{0.375000in}{2.640000in}}%
\pgfpathclose%
\pgfusepath{fill}%
\end{pgfscope}%
\begin{pgfscope}%
\pgfpathrectangle{\pgfqpoint{0.375000in}{0.330000in}}{\pgfqpoint{2.325000in}{2.310000in}}%
\pgfusepath{clip}%
\pgfsetbuttcap%
\pgfsetroundjoin%
\definecolor{currentfill}{rgb}{0.000000,0.000000,0.000000}%
\pgfsetfillcolor{currentfill}%
\pgfsetlinewidth{1.003750pt}%
\definecolor{currentstroke}{rgb}{0.000000,0.000000,0.000000}%
\pgfsetstrokecolor{currentstroke}%
\pgfsetdash{}{0pt}%
\pgfpathmoveto{\pgfqpoint{0.459750in}{0.412000in}}%
\pgfpathcurveto{\pgfqpoint{0.470800in}{0.412000in}}{\pgfqpoint{0.481399in}{0.416390in}}{\pgfqpoint{0.489213in}{0.424204in}}%
\pgfpathcurveto{\pgfqpoint{0.497026in}{0.432017in}}{\pgfqpoint{0.501417in}{0.442616in}}{\pgfqpoint{0.501417in}{0.453666in}}%
\pgfpathcurveto{\pgfqpoint{0.501417in}{0.464716in}}{\pgfqpoint{0.497026in}{0.475315in}}{\pgfqpoint{0.489213in}{0.483129in}}%
\pgfpathcurveto{\pgfqpoint{0.481399in}{0.490943in}}{\pgfqpoint{0.470800in}{0.495333in}}{\pgfqpoint{0.459750in}{0.495333in}}%
\pgfpathcurveto{\pgfqpoint{0.448700in}{0.495333in}}{\pgfqpoint{0.438101in}{0.490943in}}{\pgfqpoint{0.430287in}{0.483129in}}%
\pgfpathcurveto{\pgfqpoint{0.422474in}{0.475315in}}{\pgfqpoint{0.418083in}{0.464716in}}{\pgfqpoint{0.418083in}{0.453666in}}%
\pgfpathcurveto{\pgfqpoint{0.418083in}{0.442616in}}{\pgfqpoint{0.422474in}{0.432017in}}{\pgfqpoint{0.430287in}{0.424204in}}%
\pgfpathcurveto{\pgfqpoint{0.438101in}{0.416390in}}{\pgfqpoint{0.448700in}{0.412000in}}{\pgfqpoint{0.459750in}{0.412000in}}%
\pgfpathclose%
\pgfusepath{stroke,fill}%
\end{pgfscope}%
\begin{pgfscope}%
\pgfpathrectangle{\pgfqpoint{0.375000in}{0.330000in}}{\pgfqpoint{2.325000in}{2.310000in}}%
\pgfusepath{clip}%
\pgfsetbuttcap%
\pgfsetroundjoin%
\definecolor{currentfill}{rgb}{0.000000,0.000000,0.000000}%
\pgfsetfillcolor{currentfill}%
\pgfsetlinewidth{1.003750pt}%
\definecolor{currentstroke}{rgb}{0.000000,0.000000,0.000000}%
\pgfsetstrokecolor{currentstroke}%
\pgfsetdash{}{0pt}%
\pgfpathmoveto{\pgfqpoint{0.459750in}{0.412000in}}%
\pgfpathcurveto{\pgfqpoint{0.470800in}{0.412000in}}{\pgfqpoint{0.481399in}{0.416390in}}{\pgfqpoint{0.489213in}{0.424204in}}%
\pgfpathcurveto{\pgfqpoint{0.497026in}{0.432017in}}{\pgfqpoint{0.501417in}{0.442616in}}{\pgfqpoint{0.501417in}{0.453666in}}%
\pgfpathcurveto{\pgfqpoint{0.501417in}{0.464716in}}{\pgfqpoint{0.497026in}{0.475315in}}{\pgfqpoint{0.489213in}{0.483129in}}%
\pgfpathcurveto{\pgfqpoint{0.481399in}{0.490943in}}{\pgfqpoint{0.470800in}{0.495333in}}{\pgfqpoint{0.459750in}{0.495333in}}%
\pgfpathcurveto{\pgfqpoint{0.448700in}{0.495333in}}{\pgfqpoint{0.438101in}{0.490943in}}{\pgfqpoint{0.430287in}{0.483129in}}%
\pgfpathcurveto{\pgfqpoint{0.422474in}{0.475315in}}{\pgfqpoint{0.418083in}{0.464716in}}{\pgfqpoint{0.418083in}{0.453666in}}%
\pgfpathcurveto{\pgfqpoint{0.418083in}{0.442616in}}{\pgfqpoint{0.422474in}{0.432017in}}{\pgfqpoint{0.430287in}{0.424204in}}%
\pgfpathcurveto{\pgfqpoint{0.438101in}{0.416390in}}{\pgfqpoint{0.448700in}{0.412000in}}{\pgfqpoint{0.459750in}{0.412000in}}%
\pgfpathclose%
\pgfusepath{stroke,fill}%
\end{pgfscope}%
\begin{pgfscope}%
\pgfpathrectangle{\pgfqpoint{0.375000in}{0.330000in}}{\pgfqpoint{2.325000in}{2.310000in}}%
\pgfusepath{clip}%
\pgfsetbuttcap%
\pgfsetroundjoin%
\definecolor{currentfill}{rgb}{0.000000,0.000000,0.000000}%
\pgfsetfillcolor{currentfill}%
\pgfsetlinewidth{1.003750pt}%
\definecolor{currentstroke}{rgb}{0.000000,0.000000,0.000000}%
\pgfsetstrokecolor{currentstroke}%
\pgfsetdash{}{0pt}%
\pgfpathmoveto{\pgfqpoint{0.459750in}{0.412000in}}%
\pgfpathcurveto{\pgfqpoint{0.470800in}{0.412000in}}{\pgfqpoint{0.481399in}{0.416390in}}{\pgfqpoint{0.489213in}{0.424204in}}%
\pgfpathcurveto{\pgfqpoint{0.497026in}{0.432017in}}{\pgfqpoint{0.501417in}{0.442616in}}{\pgfqpoint{0.501417in}{0.453666in}}%
\pgfpathcurveto{\pgfqpoint{0.501417in}{0.464716in}}{\pgfqpoint{0.497026in}{0.475315in}}{\pgfqpoint{0.489213in}{0.483129in}}%
\pgfpathcurveto{\pgfqpoint{0.481399in}{0.490943in}}{\pgfqpoint{0.470800in}{0.495333in}}{\pgfqpoint{0.459750in}{0.495333in}}%
\pgfpathcurveto{\pgfqpoint{0.448700in}{0.495333in}}{\pgfqpoint{0.438101in}{0.490943in}}{\pgfqpoint{0.430287in}{0.483129in}}%
\pgfpathcurveto{\pgfqpoint{0.422474in}{0.475315in}}{\pgfqpoint{0.418083in}{0.464716in}}{\pgfqpoint{0.418083in}{0.453666in}}%
\pgfpathcurveto{\pgfqpoint{0.418083in}{0.442616in}}{\pgfqpoint{0.422474in}{0.432017in}}{\pgfqpoint{0.430287in}{0.424204in}}%
\pgfpathcurveto{\pgfqpoint{0.438101in}{0.416390in}}{\pgfqpoint{0.448700in}{0.412000in}}{\pgfqpoint{0.459750in}{0.412000in}}%
\pgfpathclose%
\pgfusepath{stroke,fill}%
\end{pgfscope}%
\begin{pgfscope}%
\pgfpathrectangle{\pgfqpoint{0.375000in}{0.330000in}}{\pgfqpoint{2.325000in}{2.310000in}}%
\pgfusepath{clip}%
\pgfsetbuttcap%
\pgfsetroundjoin%
\definecolor{currentfill}{rgb}{0.000000,0.000000,0.000000}%
\pgfsetfillcolor{currentfill}%
\pgfsetlinewidth{1.003750pt}%
\definecolor{currentstroke}{rgb}{0.000000,0.000000,0.000000}%
\pgfsetstrokecolor{currentstroke}%
\pgfsetdash{}{0pt}%
\pgfpathmoveto{\pgfqpoint{0.459750in}{0.412000in}}%
\pgfpathcurveto{\pgfqpoint{0.470800in}{0.412000in}}{\pgfqpoint{0.481399in}{0.416390in}}{\pgfqpoint{0.489213in}{0.424204in}}%
\pgfpathcurveto{\pgfqpoint{0.497026in}{0.432017in}}{\pgfqpoint{0.501417in}{0.442616in}}{\pgfqpoint{0.501417in}{0.453666in}}%
\pgfpathcurveto{\pgfqpoint{0.501417in}{0.464716in}}{\pgfqpoint{0.497026in}{0.475315in}}{\pgfqpoint{0.489213in}{0.483129in}}%
\pgfpathcurveto{\pgfqpoint{0.481399in}{0.490943in}}{\pgfqpoint{0.470800in}{0.495333in}}{\pgfqpoint{0.459750in}{0.495333in}}%
\pgfpathcurveto{\pgfqpoint{0.448700in}{0.495333in}}{\pgfqpoint{0.438101in}{0.490943in}}{\pgfqpoint{0.430287in}{0.483129in}}%
\pgfpathcurveto{\pgfqpoint{0.422474in}{0.475315in}}{\pgfqpoint{0.418083in}{0.464716in}}{\pgfqpoint{0.418083in}{0.453666in}}%
\pgfpathcurveto{\pgfqpoint{0.418083in}{0.442616in}}{\pgfqpoint{0.422474in}{0.432017in}}{\pgfqpoint{0.430287in}{0.424204in}}%
\pgfpathcurveto{\pgfqpoint{0.438101in}{0.416390in}}{\pgfqpoint{0.448700in}{0.412000in}}{\pgfqpoint{0.459750in}{0.412000in}}%
\pgfpathclose%
\pgfusepath{stroke,fill}%
\end{pgfscope}%
\begin{pgfscope}%
\pgfpathrectangle{\pgfqpoint{0.375000in}{0.330000in}}{\pgfqpoint{2.325000in}{2.310000in}}%
\pgfusepath{clip}%
\pgfsetbuttcap%
\pgfsetroundjoin%
\definecolor{currentfill}{rgb}{0.000000,0.000000,0.000000}%
\pgfsetfillcolor{currentfill}%
\pgfsetlinewidth{1.003750pt}%
\definecolor{currentstroke}{rgb}{0.000000,0.000000,0.000000}%
\pgfsetstrokecolor{currentstroke}%
\pgfsetdash{}{0pt}%
\pgfpathmoveto{\pgfqpoint{0.459750in}{0.412000in}}%
\pgfpathcurveto{\pgfqpoint{0.470800in}{0.412000in}}{\pgfqpoint{0.481399in}{0.416390in}}{\pgfqpoint{0.489213in}{0.424204in}}%
\pgfpathcurveto{\pgfqpoint{0.497026in}{0.432017in}}{\pgfqpoint{0.501417in}{0.442616in}}{\pgfqpoint{0.501417in}{0.453666in}}%
\pgfpathcurveto{\pgfqpoint{0.501417in}{0.464716in}}{\pgfqpoint{0.497026in}{0.475315in}}{\pgfqpoint{0.489213in}{0.483129in}}%
\pgfpathcurveto{\pgfqpoint{0.481399in}{0.490943in}}{\pgfqpoint{0.470800in}{0.495333in}}{\pgfqpoint{0.459750in}{0.495333in}}%
\pgfpathcurveto{\pgfqpoint{0.448700in}{0.495333in}}{\pgfqpoint{0.438101in}{0.490943in}}{\pgfqpoint{0.430287in}{0.483129in}}%
\pgfpathcurveto{\pgfqpoint{0.422474in}{0.475315in}}{\pgfqpoint{0.418083in}{0.464716in}}{\pgfqpoint{0.418083in}{0.453666in}}%
\pgfpathcurveto{\pgfqpoint{0.418083in}{0.442616in}}{\pgfqpoint{0.422474in}{0.432017in}}{\pgfqpoint{0.430287in}{0.424204in}}%
\pgfpathcurveto{\pgfqpoint{0.438101in}{0.416390in}}{\pgfqpoint{0.448700in}{0.412000in}}{\pgfqpoint{0.459750in}{0.412000in}}%
\pgfpathclose%
\pgfusepath{stroke,fill}%
\end{pgfscope}%
\begin{pgfscope}%
\pgfpathrectangle{\pgfqpoint{0.375000in}{0.330000in}}{\pgfqpoint{2.325000in}{2.310000in}}%
\pgfusepath{clip}%
\pgfsetbuttcap%
\pgfsetroundjoin%
\definecolor{currentfill}{rgb}{0.000000,0.000000,0.000000}%
\pgfsetfillcolor{currentfill}%
\pgfsetlinewidth{1.003750pt}%
\definecolor{currentstroke}{rgb}{0.000000,0.000000,0.000000}%
\pgfsetstrokecolor{currentstroke}%
\pgfsetdash{}{0pt}%
\pgfpathmoveto{\pgfqpoint{0.459750in}{0.412000in}}%
\pgfpathcurveto{\pgfqpoint{0.470800in}{0.412000in}}{\pgfqpoint{0.481399in}{0.416390in}}{\pgfqpoint{0.489213in}{0.424204in}}%
\pgfpathcurveto{\pgfqpoint{0.497026in}{0.432017in}}{\pgfqpoint{0.501417in}{0.442616in}}{\pgfqpoint{0.501417in}{0.453666in}}%
\pgfpathcurveto{\pgfqpoint{0.501417in}{0.464716in}}{\pgfqpoint{0.497026in}{0.475315in}}{\pgfqpoint{0.489213in}{0.483129in}}%
\pgfpathcurveto{\pgfqpoint{0.481399in}{0.490943in}}{\pgfqpoint{0.470800in}{0.495333in}}{\pgfqpoint{0.459750in}{0.495333in}}%
\pgfpathcurveto{\pgfqpoint{0.448700in}{0.495333in}}{\pgfqpoint{0.438101in}{0.490943in}}{\pgfqpoint{0.430287in}{0.483129in}}%
\pgfpathcurveto{\pgfqpoint{0.422474in}{0.475315in}}{\pgfqpoint{0.418083in}{0.464716in}}{\pgfqpoint{0.418083in}{0.453666in}}%
\pgfpathcurveto{\pgfqpoint{0.418083in}{0.442616in}}{\pgfqpoint{0.422474in}{0.432017in}}{\pgfqpoint{0.430287in}{0.424204in}}%
\pgfpathcurveto{\pgfqpoint{0.438101in}{0.416390in}}{\pgfqpoint{0.448700in}{0.412000in}}{\pgfqpoint{0.459750in}{0.412000in}}%
\pgfpathclose%
\pgfusepath{stroke,fill}%
\end{pgfscope}%
\begin{pgfscope}%
\pgfpathrectangle{\pgfqpoint{0.375000in}{0.330000in}}{\pgfqpoint{2.325000in}{2.310000in}}%
\pgfusepath{clip}%
\pgfsetbuttcap%
\pgfsetroundjoin%
\definecolor{currentfill}{rgb}{0.000000,0.000000,0.000000}%
\pgfsetfillcolor{currentfill}%
\pgfsetlinewidth{1.003750pt}%
\definecolor{currentstroke}{rgb}{0.000000,0.000000,0.000000}%
\pgfsetstrokecolor{currentstroke}%
\pgfsetdash{}{0pt}%
\pgfpathmoveto{\pgfqpoint{0.459750in}{0.412000in}}%
\pgfpathcurveto{\pgfqpoint{0.470800in}{0.412000in}}{\pgfqpoint{0.481399in}{0.416390in}}{\pgfqpoint{0.489213in}{0.424204in}}%
\pgfpathcurveto{\pgfqpoint{0.497026in}{0.432017in}}{\pgfqpoint{0.501417in}{0.442616in}}{\pgfqpoint{0.501417in}{0.453666in}}%
\pgfpathcurveto{\pgfqpoint{0.501417in}{0.464716in}}{\pgfqpoint{0.497026in}{0.475315in}}{\pgfqpoint{0.489213in}{0.483129in}}%
\pgfpathcurveto{\pgfqpoint{0.481399in}{0.490943in}}{\pgfqpoint{0.470800in}{0.495333in}}{\pgfqpoint{0.459750in}{0.495333in}}%
\pgfpathcurveto{\pgfqpoint{0.448700in}{0.495333in}}{\pgfqpoint{0.438101in}{0.490943in}}{\pgfqpoint{0.430287in}{0.483129in}}%
\pgfpathcurveto{\pgfqpoint{0.422474in}{0.475315in}}{\pgfqpoint{0.418083in}{0.464716in}}{\pgfqpoint{0.418083in}{0.453666in}}%
\pgfpathcurveto{\pgfqpoint{0.418083in}{0.442616in}}{\pgfqpoint{0.422474in}{0.432017in}}{\pgfqpoint{0.430287in}{0.424204in}}%
\pgfpathcurveto{\pgfqpoint{0.438101in}{0.416390in}}{\pgfqpoint{0.448700in}{0.412000in}}{\pgfqpoint{0.459750in}{0.412000in}}%
\pgfpathclose%
\pgfusepath{stroke,fill}%
\end{pgfscope}%
\begin{pgfscope}%
\pgfpathrectangle{\pgfqpoint{0.375000in}{0.330000in}}{\pgfqpoint{2.325000in}{2.310000in}}%
\pgfusepath{clip}%
\pgfsetbuttcap%
\pgfsetroundjoin%
\definecolor{currentfill}{rgb}{0.000000,0.000000,0.000000}%
\pgfsetfillcolor{currentfill}%
\pgfsetlinewidth{1.003750pt}%
\definecolor{currentstroke}{rgb}{0.000000,0.000000,0.000000}%
\pgfsetstrokecolor{currentstroke}%
\pgfsetdash{}{0pt}%
\pgfpathmoveto{\pgfqpoint{0.459750in}{0.412000in}}%
\pgfpathcurveto{\pgfqpoint{0.470800in}{0.412000in}}{\pgfqpoint{0.481399in}{0.416390in}}{\pgfqpoint{0.489213in}{0.424204in}}%
\pgfpathcurveto{\pgfqpoint{0.497026in}{0.432017in}}{\pgfqpoint{0.501417in}{0.442616in}}{\pgfqpoint{0.501417in}{0.453666in}}%
\pgfpathcurveto{\pgfqpoint{0.501417in}{0.464716in}}{\pgfqpoint{0.497026in}{0.475315in}}{\pgfqpoint{0.489213in}{0.483129in}}%
\pgfpathcurveto{\pgfqpoint{0.481399in}{0.490943in}}{\pgfqpoint{0.470800in}{0.495333in}}{\pgfqpoint{0.459750in}{0.495333in}}%
\pgfpathcurveto{\pgfqpoint{0.448700in}{0.495333in}}{\pgfqpoint{0.438101in}{0.490943in}}{\pgfqpoint{0.430287in}{0.483129in}}%
\pgfpathcurveto{\pgfqpoint{0.422474in}{0.475315in}}{\pgfqpoint{0.418083in}{0.464716in}}{\pgfqpoint{0.418083in}{0.453666in}}%
\pgfpathcurveto{\pgfqpoint{0.418083in}{0.442616in}}{\pgfqpoint{0.422474in}{0.432017in}}{\pgfqpoint{0.430287in}{0.424204in}}%
\pgfpathcurveto{\pgfqpoint{0.438101in}{0.416390in}}{\pgfqpoint{0.448700in}{0.412000in}}{\pgfqpoint{0.459750in}{0.412000in}}%
\pgfpathclose%
\pgfusepath{stroke,fill}%
\end{pgfscope}%
\begin{pgfscope}%
\pgfpathrectangle{\pgfqpoint{0.375000in}{0.330000in}}{\pgfqpoint{2.325000in}{2.310000in}}%
\pgfusepath{clip}%
\pgfsetbuttcap%
\pgfsetroundjoin%
\definecolor{currentfill}{rgb}{0.000000,0.000000,0.000000}%
\pgfsetfillcolor{currentfill}%
\pgfsetlinewidth{1.003750pt}%
\definecolor{currentstroke}{rgb}{0.000000,0.000000,0.000000}%
\pgfsetstrokecolor{currentstroke}%
\pgfsetdash{}{0pt}%
\pgfpathmoveto{\pgfqpoint{0.459750in}{0.412000in}}%
\pgfpathcurveto{\pgfqpoint{0.470800in}{0.412000in}}{\pgfqpoint{0.481399in}{0.416390in}}{\pgfqpoint{0.489213in}{0.424204in}}%
\pgfpathcurveto{\pgfqpoint{0.497026in}{0.432017in}}{\pgfqpoint{0.501417in}{0.442616in}}{\pgfqpoint{0.501417in}{0.453666in}}%
\pgfpathcurveto{\pgfqpoint{0.501417in}{0.464716in}}{\pgfqpoint{0.497026in}{0.475315in}}{\pgfqpoint{0.489213in}{0.483129in}}%
\pgfpathcurveto{\pgfqpoint{0.481399in}{0.490943in}}{\pgfqpoint{0.470800in}{0.495333in}}{\pgfqpoint{0.459750in}{0.495333in}}%
\pgfpathcurveto{\pgfqpoint{0.448700in}{0.495333in}}{\pgfqpoint{0.438101in}{0.490943in}}{\pgfqpoint{0.430287in}{0.483129in}}%
\pgfpathcurveto{\pgfqpoint{0.422474in}{0.475315in}}{\pgfqpoint{0.418083in}{0.464716in}}{\pgfqpoint{0.418083in}{0.453666in}}%
\pgfpathcurveto{\pgfqpoint{0.418083in}{0.442616in}}{\pgfqpoint{0.422474in}{0.432017in}}{\pgfqpoint{0.430287in}{0.424204in}}%
\pgfpathcurveto{\pgfqpoint{0.438101in}{0.416390in}}{\pgfqpoint{0.448700in}{0.412000in}}{\pgfqpoint{0.459750in}{0.412000in}}%
\pgfpathclose%
\pgfusepath{stroke,fill}%
\end{pgfscope}%
\begin{pgfscope}%
\pgfpathrectangle{\pgfqpoint{0.375000in}{0.330000in}}{\pgfqpoint{2.325000in}{2.310000in}}%
\pgfusepath{clip}%
\pgfsetbuttcap%
\pgfsetroundjoin%
\definecolor{currentfill}{rgb}{0.000000,0.000000,0.000000}%
\pgfsetfillcolor{currentfill}%
\pgfsetlinewidth{1.003750pt}%
\definecolor{currentstroke}{rgb}{0.000000,0.000000,0.000000}%
\pgfsetstrokecolor{currentstroke}%
\pgfsetdash{}{0pt}%
\pgfpathmoveto{\pgfqpoint{0.459750in}{0.412000in}}%
\pgfpathcurveto{\pgfqpoint{0.470800in}{0.412000in}}{\pgfqpoint{0.481399in}{0.416390in}}{\pgfqpoint{0.489213in}{0.424204in}}%
\pgfpathcurveto{\pgfqpoint{0.497026in}{0.432017in}}{\pgfqpoint{0.501417in}{0.442616in}}{\pgfqpoint{0.501417in}{0.453666in}}%
\pgfpathcurveto{\pgfqpoint{0.501417in}{0.464716in}}{\pgfqpoint{0.497026in}{0.475315in}}{\pgfqpoint{0.489213in}{0.483129in}}%
\pgfpathcurveto{\pgfqpoint{0.481399in}{0.490943in}}{\pgfqpoint{0.470800in}{0.495333in}}{\pgfqpoint{0.459750in}{0.495333in}}%
\pgfpathcurveto{\pgfqpoint{0.448700in}{0.495333in}}{\pgfqpoint{0.438101in}{0.490943in}}{\pgfqpoint{0.430287in}{0.483129in}}%
\pgfpathcurveto{\pgfqpoint{0.422474in}{0.475315in}}{\pgfqpoint{0.418083in}{0.464716in}}{\pgfqpoint{0.418083in}{0.453666in}}%
\pgfpathcurveto{\pgfqpoint{0.418083in}{0.442616in}}{\pgfqpoint{0.422474in}{0.432017in}}{\pgfqpoint{0.430287in}{0.424204in}}%
\pgfpathcurveto{\pgfqpoint{0.438101in}{0.416390in}}{\pgfqpoint{0.448700in}{0.412000in}}{\pgfqpoint{0.459750in}{0.412000in}}%
\pgfpathclose%
\pgfusepath{stroke,fill}%
\end{pgfscope}%
\begin{pgfscope}%
\pgfpathrectangle{\pgfqpoint{0.375000in}{0.330000in}}{\pgfqpoint{2.325000in}{2.310000in}}%
\pgfusepath{clip}%
\pgfsetbuttcap%
\pgfsetroundjoin%
\definecolor{currentfill}{rgb}{0.000000,0.000000,0.000000}%
\pgfsetfillcolor{currentfill}%
\pgfsetlinewidth{1.003750pt}%
\definecolor{currentstroke}{rgb}{0.000000,0.000000,0.000000}%
\pgfsetstrokecolor{currentstroke}%
\pgfsetdash{}{0pt}%
\pgfpathmoveto{\pgfqpoint{0.459750in}{1.443291in}}%
\pgfpathcurveto{\pgfqpoint{0.470800in}{1.443291in}}{\pgfqpoint{0.481399in}{1.447682in}}{\pgfqpoint{0.489213in}{1.455495in}}%
\pgfpathcurveto{\pgfqpoint{0.497026in}{1.463309in}}{\pgfqpoint{0.501417in}{1.473908in}}{\pgfqpoint{0.501417in}{1.484958in}}%
\pgfpathcurveto{\pgfqpoint{0.501417in}{1.496008in}}{\pgfqpoint{0.497026in}{1.506607in}}{\pgfqpoint{0.489213in}{1.514421in}}%
\pgfpathcurveto{\pgfqpoint{0.481399in}{1.522235in}}{\pgfqpoint{0.470800in}{1.526625in}}{\pgfqpoint{0.459750in}{1.526625in}}%
\pgfpathcurveto{\pgfqpoint{0.448700in}{1.526625in}}{\pgfqpoint{0.438101in}{1.522235in}}{\pgfqpoint{0.430287in}{1.514421in}}%
\pgfpathcurveto{\pgfqpoint{0.422474in}{1.506607in}}{\pgfqpoint{0.418083in}{1.496008in}}{\pgfqpoint{0.418083in}{1.484958in}}%
\pgfpathcurveto{\pgfqpoint{0.418083in}{1.473908in}}{\pgfqpoint{0.422474in}{1.463309in}}{\pgfqpoint{0.430287in}{1.455495in}}%
\pgfpathcurveto{\pgfqpoint{0.438101in}{1.447682in}}{\pgfqpoint{0.448700in}{1.443291in}}{\pgfqpoint{0.459750in}{1.443291in}}%
\pgfpathclose%
\pgfusepath{stroke,fill}%
\end{pgfscope}%
\begin{pgfscope}%
\pgfpathrectangle{\pgfqpoint{0.375000in}{0.330000in}}{\pgfqpoint{2.325000in}{2.310000in}}%
\pgfusepath{clip}%
\pgfsetbuttcap%
\pgfsetroundjoin%
\definecolor{currentfill}{rgb}{0.000000,0.000000,0.000000}%
\pgfsetfillcolor{currentfill}%
\pgfsetlinewidth{1.003750pt}%
\definecolor{currentstroke}{rgb}{0.000000,0.000000,0.000000}%
\pgfsetstrokecolor{currentstroke}%
\pgfsetdash{}{0pt}%
\pgfpathmoveto{\pgfqpoint{0.459750in}{1.443291in}}%
\pgfpathcurveto{\pgfqpoint{0.470800in}{1.443291in}}{\pgfqpoint{0.481399in}{1.447682in}}{\pgfqpoint{0.489213in}{1.455495in}}%
\pgfpathcurveto{\pgfqpoint{0.497026in}{1.463309in}}{\pgfqpoint{0.501417in}{1.473908in}}{\pgfqpoint{0.501417in}{1.484958in}}%
\pgfpathcurveto{\pgfqpoint{0.501417in}{1.496008in}}{\pgfqpoint{0.497026in}{1.506607in}}{\pgfqpoint{0.489213in}{1.514421in}}%
\pgfpathcurveto{\pgfqpoint{0.481399in}{1.522235in}}{\pgfqpoint{0.470800in}{1.526625in}}{\pgfqpoint{0.459750in}{1.526625in}}%
\pgfpathcurveto{\pgfqpoint{0.448700in}{1.526625in}}{\pgfqpoint{0.438101in}{1.522235in}}{\pgfqpoint{0.430287in}{1.514421in}}%
\pgfpathcurveto{\pgfqpoint{0.422474in}{1.506607in}}{\pgfqpoint{0.418083in}{1.496008in}}{\pgfqpoint{0.418083in}{1.484958in}}%
\pgfpathcurveto{\pgfqpoint{0.418083in}{1.473908in}}{\pgfqpoint{0.422474in}{1.463309in}}{\pgfqpoint{0.430287in}{1.455495in}}%
\pgfpathcurveto{\pgfqpoint{0.438101in}{1.447682in}}{\pgfqpoint{0.448700in}{1.443291in}}{\pgfqpoint{0.459750in}{1.443291in}}%
\pgfpathclose%
\pgfusepath{stroke,fill}%
\end{pgfscope}%
\begin{pgfscope}%
\pgfpathrectangle{\pgfqpoint{0.375000in}{0.330000in}}{\pgfqpoint{2.325000in}{2.310000in}}%
\pgfusepath{clip}%
\pgfsetbuttcap%
\pgfsetroundjoin%
\definecolor{currentfill}{rgb}{0.000000,0.000000,0.000000}%
\pgfsetfillcolor{currentfill}%
\pgfsetlinewidth{1.003750pt}%
\definecolor{currentstroke}{rgb}{0.000000,0.000000,0.000000}%
\pgfsetstrokecolor{currentstroke}%
\pgfsetdash{}{0pt}%
\pgfpathmoveto{\pgfqpoint{0.459750in}{0.412000in}}%
\pgfpathcurveto{\pgfqpoint{0.470800in}{0.412000in}}{\pgfqpoint{0.481399in}{0.416390in}}{\pgfqpoint{0.489213in}{0.424204in}}%
\pgfpathcurveto{\pgfqpoint{0.497026in}{0.432017in}}{\pgfqpoint{0.501417in}{0.442616in}}{\pgfqpoint{0.501417in}{0.453666in}}%
\pgfpathcurveto{\pgfqpoint{0.501417in}{0.464716in}}{\pgfqpoint{0.497026in}{0.475315in}}{\pgfqpoint{0.489213in}{0.483129in}}%
\pgfpathcurveto{\pgfqpoint{0.481399in}{0.490943in}}{\pgfqpoint{0.470800in}{0.495333in}}{\pgfqpoint{0.459750in}{0.495333in}}%
\pgfpathcurveto{\pgfqpoint{0.448700in}{0.495333in}}{\pgfqpoint{0.438101in}{0.490943in}}{\pgfqpoint{0.430287in}{0.483129in}}%
\pgfpathcurveto{\pgfqpoint{0.422474in}{0.475315in}}{\pgfqpoint{0.418083in}{0.464716in}}{\pgfqpoint{0.418083in}{0.453666in}}%
\pgfpathcurveto{\pgfqpoint{0.418083in}{0.442616in}}{\pgfqpoint{0.422474in}{0.432017in}}{\pgfqpoint{0.430287in}{0.424204in}}%
\pgfpathcurveto{\pgfqpoint{0.438101in}{0.416390in}}{\pgfqpoint{0.448700in}{0.412000in}}{\pgfqpoint{0.459750in}{0.412000in}}%
\pgfpathclose%
\pgfusepath{stroke,fill}%
\end{pgfscope}%
\begin{pgfscope}%
\pgfpathrectangle{\pgfqpoint{0.375000in}{0.330000in}}{\pgfqpoint{2.325000in}{2.310000in}}%
\pgfusepath{clip}%
\pgfsetbuttcap%
\pgfsetroundjoin%
\definecolor{currentfill}{rgb}{0.000000,0.000000,0.000000}%
\pgfsetfillcolor{currentfill}%
\pgfsetlinewidth{1.003750pt}%
\definecolor{currentstroke}{rgb}{0.000000,0.000000,0.000000}%
\pgfsetstrokecolor{currentstroke}%
\pgfsetdash{}{0pt}%
\pgfpathmoveto{\pgfqpoint{0.459750in}{0.412000in}}%
\pgfpathcurveto{\pgfqpoint{0.470800in}{0.412000in}}{\pgfqpoint{0.481399in}{0.416390in}}{\pgfqpoint{0.489213in}{0.424204in}}%
\pgfpathcurveto{\pgfqpoint{0.497026in}{0.432017in}}{\pgfqpoint{0.501417in}{0.442616in}}{\pgfqpoint{0.501417in}{0.453666in}}%
\pgfpathcurveto{\pgfqpoint{0.501417in}{0.464716in}}{\pgfqpoint{0.497026in}{0.475315in}}{\pgfqpoint{0.489213in}{0.483129in}}%
\pgfpathcurveto{\pgfqpoint{0.481399in}{0.490943in}}{\pgfqpoint{0.470800in}{0.495333in}}{\pgfqpoint{0.459750in}{0.495333in}}%
\pgfpathcurveto{\pgfqpoint{0.448700in}{0.495333in}}{\pgfqpoint{0.438101in}{0.490943in}}{\pgfqpoint{0.430287in}{0.483129in}}%
\pgfpathcurveto{\pgfqpoint{0.422474in}{0.475315in}}{\pgfqpoint{0.418083in}{0.464716in}}{\pgfqpoint{0.418083in}{0.453666in}}%
\pgfpathcurveto{\pgfqpoint{0.418083in}{0.442616in}}{\pgfqpoint{0.422474in}{0.432017in}}{\pgfqpoint{0.430287in}{0.424204in}}%
\pgfpathcurveto{\pgfqpoint{0.438101in}{0.416390in}}{\pgfqpoint{0.448700in}{0.412000in}}{\pgfqpoint{0.459750in}{0.412000in}}%
\pgfpathclose%
\pgfusepath{stroke,fill}%
\end{pgfscope}%
\begin{pgfscope}%
\pgfpathrectangle{\pgfqpoint{0.375000in}{0.330000in}}{\pgfqpoint{2.325000in}{2.310000in}}%
\pgfusepath{clip}%
\pgfsetbuttcap%
\pgfsetroundjoin%
\definecolor{currentfill}{rgb}{0.000000,0.000000,0.000000}%
\pgfsetfillcolor{currentfill}%
\pgfsetlinewidth{1.003750pt}%
\definecolor{currentstroke}{rgb}{0.000000,0.000000,0.000000}%
\pgfsetstrokecolor{currentstroke}%
\pgfsetdash{}{0pt}%
\pgfpathmoveto{\pgfqpoint{0.459750in}{0.412000in}}%
\pgfpathcurveto{\pgfqpoint{0.470800in}{0.412000in}}{\pgfqpoint{0.481399in}{0.416390in}}{\pgfqpoint{0.489213in}{0.424204in}}%
\pgfpathcurveto{\pgfqpoint{0.497026in}{0.432017in}}{\pgfqpoint{0.501417in}{0.442616in}}{\pgfqpoint{0.501417in}{0.453666in}}%
\pgfpathcurveto{\pgfqpoint{0.501417in}{0.464716in}}{\pgfqpoint{0.497026in}{0.475315in}}{\pgfqpoint{0.489213in}{0.483129in}}%
\pgfpathcurveto{\pgfqpoint{0.481399in}{0.490943in}}{\pgfqpoint{0.470800in}{0.495333in}}{\pgfqpoint{0.459750in}{0.495333in}}%
\pgfpathcurveto{\pgfqpoint{0.448700in}{0.495333in}}{\pgfqpoint{0.438101in}{0.490943in}}{\pgfqpoint{0.430287in}{0.483129in}}%
\pgfpathcurveto{\pgfqpoint{0.422474in}{0.475315in}}{\pgfqpoint{0.418083in}{0.464716in}}{\pgfqpoint{0.418083in}{0.453666in}}%
\pgfpathcurveto{\pgfqpoint{0.418083in}{0.442616in}}{\pgfqpoint{0.422474in}{0.432017in}}{\pgfqpoint{0.430287in}{0.424204in}}%
\pgfpathcurveto{\pgfqpoint{0.438101in}{0.416390in}}{\pgfqpoint{0.448700in}{0.412000in}}{\pgfqpoint{0.459750in}{0.412000in}}%
\pgfpathclose%
\pgfusepath{stroke,fill}%
\end{pgfscope}%
\begin{pgfscope}%
\pgfpathrectangle{\pgfqpoint{0.375000in}{0.330000in}}{\pgfqpoint{2.325000in}{2.310000in}}%
\pgfusepath{clip}%
\pgfsetbuttcap%
\pgfsetroundjoin%
\definecolor{currentfill}{rgb}{0.000000,0.000000,0.000000}%
\pgfsetfillcolor{currentfill}%
\pgfsetlinewidth{1.003750pt}%
\definecolor{currentstroke}{rgb}{0.000000,0.000000,0.000000}%
\pgfsetstrokecolor{currentstroke}%
\pgfsetdash{}{0pt}%
\pgfpathmoveto{\pgfqpoint{0.459750in}{0.412000in}}%
\pgfpathcurveto{\pgfqpoint{0.470800in}{0.412000in}}{\pgfqpoint{0.481399in}{0.416390in}}{\pgfqpoint{0.489213in}{0.424204in}}%
\pgfpathcurveto{\pgfqpoint{0.497026in}{0.432017in}}{\pgfqpoint{0.501417in}{0.442616in}}{\pgfqpoint{0.501417in}{0.453666in}}%
\pgfpathcurveto{\pgfqpoint{0.501417in}{0.464716in}}{\pgfqpoint{0.497026in}{0.475315in}}{\pgfqpoint{0.489213in}{0.483129in}}%
\pgfpathcurveto{\pgfqpoint{0.481399in}{0.490943in}}{\pgfqpoint{0.470800in}{0.495333in}}{\pgfqpoint{0.459750in}{0.495333in}}%
\pgfpathcurveto{\pgfqpoint{0.448700in}{0.495333in}}{\pgfqpoint{0.438101in}{0.490943in}}{\pgfqpoint{0.430287in}{0.483129in}}%
\pgfpathcurveto{\pgfqpoint{0.422474in}{0.475315in}}{\pgfqpoint{0.418083in}{0.464716in}}{\pgfqpoint{0.418083in}{0.453666in}}%
\pgfpathcurveto{\pgfqpoint{0.418083in}{0.442616in}}{\pgfqpoint{0.422474in}{0.432017in}}{\pgfqpoint{0.430287in}{0.424204in}}%
\pgfpathcurveto{\pgfqpoint{0.438101in}{0.416390in}}{\pgfqpoint{0.448700in}{0.412000in}}{\pgfqpoint{0.459750in}{0.412000in}}%
\pgfpathclose%
\pgfusepath{stroke,fill}%
\end{pgfscope}%
\begin{pgfscope}%
\pgfpathrectangle{\pgfqpoint{0.375000in}{0.330000in}}{\pgfqpoint{2.325000in}{2.310000in}}%
\pgfusepath{clip}%
\pgfsetbuttcap%
\pgfsetroundjoin%
\definecolor{currentfill}{rgb}{0.000000,0.000000,0.000000}%
\pgfsetfillcolor{currentfill}%
\pgfsetlinewidth{1.003750pt}%
\definecolor{currentstroke}{rgb}{0.000000,0.000000,0.000000}%
\pgfsetstrokecolor{currentstroke}%
\pgfsetdash{}{0pt}%
\pgfpathmoveto{\pgfqpoint{0.459750in}{0.412000in}}%
\pgfpathcurveto{\pgfqpoint{0.470800in}{0.412000in}}{\pgfqpoint{0.481399in}{0.416390in}}{\pgfqpoint{0.489213in}{0.424204in}}%
\pgfpathcurveto{\pgfqpoint{0.497026in}{0.432017in}}{\pgfqpoint{0.501417in}{0.442616in}}{\pgfqpoint{0.501417in}{0.453666in}}%
\pgfpathcurveto{\pgfqpoint{0.501417in}{0.464716in}}{\pgfqpoint{0.497026in}{0.475315in}}{\pgfqpoint{0.489213in}{0.483129in}}%
\pgfpathcurveto{\pgfqpoint{0.481399in}{0.490943in}}{\pgfqpoint{0.470800in}{0.495333in}}{\pgfqpoint{0.459750in}{0.495333in}}%
\pgfpathcurveto{\pgfqpoint{0.448700in}{0.495333in}}{\pgfqpoint{0.438101in}{0.490943in}}{\pgfqpoint{0.430287in}{0.483129in}}%
\pgfpathcurveto{\pgfqpoint{0.422474in}{0.475315in}}{\pgfqpoint{0.418083in}{0.464716in}}{\pgfqpoint{0.418083in}{0.453666in}}%
\pgfpathcurveto{\pgfqpoint{0.418083in}{0.442616in}}{\pgfqpoint{0.422474in}{0.432017in}}{\pgfqpoint{0.430287in}{0.424204in}}%
\pgfpathcurveto{\pgfqpoint{0.438101in}{0.416390in}}{\pgfqpoint{0.448700in}{0.412000in}}{\pgfqpoint{0.459750in}{0.412000in}}%
\pgfpathclose%
\pgfusepath{stroke,fill}%
\end{pgfscope}%
\begin{pgfscope}%
\pgfpathrectangle{\pgfqpoint{0.375000in}{0.330000in}}{\pgfqpoint{2.325000in}{2.310000in}}%
\pgfusepath{clip}%
\pgfsetbuttcap%
\pgfsetroundjoin%
\definecolor{currentfill}{rgb}{0.000000,0.000000,0.000000}%
\pgfsetfillcolor{currentfill}%
\pgfsetlinewidth{1.003750pt}%
\definecolor{currentstroke}{rgb}{0.000000,0.000000,0.000000}%
\pgfsetstrokecolor{currentstroke}%
\pgfsetdash{}{0pt}%
\pgfpathmoveto{\pgfqpoint{0.459750in}{0.412000in}}%
\pgfpathcurveto{\pgfqpoint{0.470800in}{0.412000in}}{\pgfqpoint{0.481399in}{0.416390in}}{\pgfqpoint{0.489213in}{0.424204in}}%
\pgfpathcurveto{\pgfqpoint{0.497026in}{0.432017in}}{\pgfqpoint{0.501417in}{0.442616in}}{\pgfqpoint{0.501417in}{0.453666in}}%
\pgfpathcurveto{\pgfqpoint{0.501417in}{0.464716in}}{\pgfqpoint{0.497026in}{0.475315in}}{\pgfqpoint{0.489213in}{0.483129in}}%
\pgfpathcurveto{\pgfqpoint{0.481399in}{0.490943in}}{\pgfqpoint{0.470800in}{0.495333in}}{\pgfqpoint{0.459750in}{0.495333in}}%
\pgfpathcurveto{\pgfqpoint{0.448700in}{0.495333in}}{\pgfqpoint{0.438101in}{0.490943in}}{\pgfqpoint{0.430287in}{0.483129in}}%
\pgfpathcurveto{\pgfqpoint{0.422474in}{0.475315in}}{\pgfqpoint{0.418083in}{0.464716in}}{\pgfqpoint{0.418083in}{0.453666in}}%
\pgfpathcurveto{\pgfqpoint{0.418083in}{0.442616in}}{\pgfqpoint{0.422474in}{0.432017in}}{\pgfqpoint{0.430287in}{0.424204in}}%
\pgfpathcurveto{\pgfqpoint{0.438101in}{0.416390in}}{\pgfqpoint{0.448700in}{0.412000in}}{\pgfqpoint{0.459750in}{0.412000in}}%
\pgfpathclose%
\pgfusepath{stroke,fill}%
\end{pgfscope}%
\begin{pgfscope}%
\pgfpathrectangle{\pgfqpoint{0.375000in}{0.330000in}}{\pgfqpoint{2.325000in}{2.310000in}}%
\pgfusepath{clip}%
\pgfsetbuttcap%
\pgfsetroundjoin%
\definecolor{currentfill}{rgb}{0.000000,0.000000,0.000000}%
\pgfsetfillcolor{currentfill}%
\pgfsetlinewidth{1.003750pt}%
\definecolor{currentstroke}{rgb}{0.000000,0.000000,0.000000}%
\pgfsetstrokecolor{currentstroke}%
\pgfsetdash{}{0pt}%
\pgfpathmoveto{\pgfqpoint{0.459750in}{0.412000in}}%
\pgfpathcurveto{\pgfqpoint{0.470800in}{0.412000in}}{\pgfqpoint{0.481399in}{0.416390in}}{\pgfqpoint{0.489213in}{0.424204in}}%
\pgfpathcurveto{\pgfqpoint{0.497026in}{0.432017in}}{\pgfqpoint{0.501417in}{0.442616in}}{\pgfqpoint{0.501417in}{0.453666in}}%
\pgfpathcurveto{\pgfqpoint{0.501417in}{0.464716in}}{\pgfqpoint{0.497026in}{0.475315in}}{\pgfqpoint{0.489213in}{0.483129in}}%
\pgfpathcurveto{\pgfqpoint{0.481399in}{0.490943in}}{\pgfqpoint{0.470800in}{0.495333in}}{\pgfqpoint{0.459750in}{0.495333in}}%
\pgfpathcurveto{\pgfqpoint{0.448700in}{0.495333in}}{\pgfqpoint{0.438101in}{0.490943in}}{\pgfqpoint{0.430287in}{0.483129in}}%
\pgfpathcurveto{\pgfqpoint{0.422474in}{0.475315in}}{\pgfqpoint{0.418083in}{0.464716in}}{\pgfqpoint{0.418083in}{0.453666in}}%
\pgfpathcurveto{\pgfqpoint{0.418083in}{0.442616in}}{\pgfqpoint{0.422474in}{0.432017in}}{\pgfqpoint{0.430287in}{0.424204in}}%
\pgfpathcurveto{\pgfqpoint{0.438101in}{0.416390in}}{\pgfqpoint{0.448700in}{0.412000in}}{\pgfqpoint{0.459750in}{0.412000in}}%
\pgfpathclose%
\pgfusepath{stroke,fill}%
\end{pgfscope}%
\begin{pgfscope}%
\pgfpathrectangle{\pgfqpoint{0.375000in}{0.330000in}}{\pgfqpoint{2.325000in}{2.310000in}}%
\pgfusepath{clip}%
\pgfsetbuttcap%
\pgfsetroundjoin%
\definecolor{currentfill}{rgb}{0.000000,0.000000,0.000000}%
\pgfsetfillcolor{currentfill}%
\pgfsetlinewidth{1.003750pt}%
\definecolor{currentstroke}{rgb}{0.000000,0.000000,0.000000}%
\pgfsetstrokecolor{currentstroke}%
\pgfsetdash{}{0pt}%
\pgfpathmoveto{\pgfqpoint{0.459750in}{0.412000in}}%
\pgfpathcurveto{\pgfqpoint{0.470800in}{0.412000in}}{\pgfqpoint{0.481399in}{0.416390in}}{\pgfqpoint{0.489213in}{0.424204in}}%
\pgfpathcurveto{\pgfqpoint{0.497026in}{0.432017in}}{\pgfqpoint{0.501417in}{0.442616in}}{\pgfqpoint{0.501417in}{0.453666in}}%
\pgfpathcurveto{\pgfqpoint{0.501417in}{0.464716in}}{\pgfqpoint{0.497026in}{0.475315in}}{\pgfqpoint{0.489213in}{0.483129in}}%
\pgfpathcurveto{\pgfqpoint{0.481399in}{0.490943in}}{\pgfqpoint{0.470800in}{0.495333in}}{\pgfqpoint{0.459750in}{0.495333in}}%
\pgfpathcurveto{\pgfqpoint{0.448700in}{0.495333in}}{\pgfqpoint{0.438101in}{0.490943in}}{\pgfqpoint{0.430287in}{0.483129in}}%
\pgfpathcurveto{\pgfqpoint{0.422474in}{0.475315in}}{\pgfqpoint{0.418083in}{0.464716in}}{\pgfqpoint{0.418083in}{0.453666in}}%
\pgfpathcurveto{\pgfqpoint{0.418083in}{0.442616in}}{\pgfqpoint{0.422474in}{0.432017in}}{\pgfqpoint{0.430287in}{0.424204in}}%
\pgfpathcurveto{\pgfqpoint{0.438101in}{0.416390in}}{\pgfqpoint{0.448700in}{0.412000in}}{\pgfqpoint{0.459750in}{0.412000in}}%
\pgfpathclose%
\pgfusepath{stroke,fill}%
\end{pgfscope}%
\begin{pgfscope}%
\pgfpathrectangle{\pgfqpoint{0.375000in}{0.330000in}}{\pgfqpoint{2.325000in}{2.310000in}}%
\pgfusepath{clip}%
\pgfsetbuttcap%
\pgfsetroundjoin%
\definecolor{currentfill}{rgb}{0.000000,0.000000,0.000000}%
\pgfsetfillcolor{currentfill}%
\pgfsetlinewidth{1.003750pt}%
\definecolor{currentstroke}{rgb}{0.000000,0.000000,0.000000}%
\pgfsetstrokecolor{currentstroke}%
\pgfsetdash{}{0pt}%
\pgfpathmoveto{\pgfqpoint{0.459750in}{0.412000in}}%
\pgfpathcurveto{\pgfqpoint{0.470800in}{0.412000in}}{\pgfqpoint{0.481399in}{0.416390in}}{\pgfqpoint{0.489213in}{0.424204in}}%
\pgfpathcurveto{\pgfqpoint{0.497026in}{0.432017in}}{\pgfqpoint{0.501417in}{0.442616in}}{\pgfqpoint{0.501417in}{0.453666in}}%
\pgfpathcurveto{\pgfqpoint{0.501417in}{0.464716in}}{\pgfqpoint{0.497026in}{0.475315in}}{\pgfqpoint{0.489213in}{0.483129in}}%
\pgfpathcurveto{\pgfqpoint{0.481399in}{0.490943in}}{\pgfqpoint{0.470800in}{0.495333in}}{\pgfqpoint{0.459750in}{0.495333in}}%
\pgfpathcurveto{\pgfqpoint{0.448700in}{0.495333in}}{\pgfqpoint{0.438101in}{0.490943in}}{\pgfqpoint{0.430287in}{0.483129in}}%
\pgfpathcurveto{\pgfqpoint{0.422474in}{0.475315in}}{\pgfqpoint{0.418083in}{0.464716in}}{\pgfqpoint{0.418083in}{0.453666in}}%
\pgfpathcurveto{\pgfqpoint{0.418083in}{0.442616in}}{\pgfqpoint{0.422474in}{0.432017in}}{\pgfqpoint{0.430287in}{0.424204in}}%
\pgfpathcurveto{\pgfqpoint{0.438101in}{0.416390in}}{\pgfqpoint{0.448700in}{0.412000in}}{\pgfqpoint{0.459750in}{0.412000in}}%
\pgfpathclose%
\pgfusepath{stroke,fill}%
\end{pgfscope}%
\begin{pgfscope}%
\pgfpathrectangle{\pgfqpoint{0.375000in}{0.330000in}}{\pgfqpoint{2.325000in}{2.310000in}}%
\pgfusepath{clip}%
\pgfsetbuttcap%
\pgfsetroundjoin%
\definecolor{currentfill}{rgb}{0.000000,0.000000,0.000000}%
\pgfsetfillcolor{currentfill}%
\pgfsetlinewidth{1.003750pt}%
\definecolor{currentstroke}{rgb}{0.000000,0.000000,0.000000}%
\pgfsetstrokecolor{currentstroke}%
\pgfsetdash{}{0pt}%
\pgfpathmoveto{\pgfqpoint{0.459750in}{0.412000in}}%
\pgfpathcurveto{\pgfqpoint{0.470800in}{0.412000in}}{\pgfqpoint{0.481399in}{0.416390in}}{\pgfqpoint{0.489213in}{0.424204in}}%
\pgfpathcurveto{\pgfqpoint{0.497026in}{0.432017in}}{\pgfqpoint{0.501417in}{0.442616in}}{\pgfqpoint{0.501417in}{0.453666in}}%
\pgfpathcurveto{\pgfqpoint{0.501417in}{0.464716in}}{\pgfqpoint{0.497026in}{0.475315in}}{\pgfqpoint{0.489213in}{0.483129in}}%
\pgfpathcurveto{\pgfqpoint{0.481399in}{0.490943in}}{\pgfqpoint{0.470800in}{0.495333in}}{\pgfqpoint{0.459750in}{0.495333in}}%
\pgfpathcurveto{\pgfqpoint{0.448700in}{0.495333in}}{\pgfqpoint{0.438101in}{0.490943in}}{\pgfqpoint{0.430287in}{0.483129in}}%
\pgfpathcurveto{\pgfqpoint{0.422474in}{0.475315in}}{\pgfqpoint{0.418083in}{0.464716in}}{\pgfqpoint{0.418083in}{0.453666in}}%
\pgfpathcurveto{\pgfqpoint{0.418083in}{0.442616in}}{\pgfqpoint{0.422474in}{0.432017in}}{\pgfqpoint{0.430287in}{0.424204in}}%
\pgfpathcurveto{\pgfqpoint{0.438101in}{0.416390in}}{\pgfqpoint{0.448700in}{0.412000in}}{\pgfqpoint{0.459750in}{0.412000in}}%
\pgfpathclose%
\pgfusepath{stroke,fill}%
\end{pgfscope}%
\begin{pgfscope}%
\pgfpathrectangle{\pgfqpoint{0.375000in}{0.330000in}}{\pgfqpoint{2.325000in}{2.310000in}}%
\pgfusepath{clip}%
\pgfsetbuttcap%
\pgfsetroundjoin%
\definecolor{currentfill}{rgb}{0.000000,0.000000,0.000000}%
\pgfsetfillcolor{currentfill}%
\pgfsetlinewidth{1.003750pt}%
\definecolor{currentstroke}{rgb}{0.000000,0.000000,0.000000}%
\pgfsetstrokecolor{currentstroke}%
\pgfsetdash{}{0pt}%
\pgfpathmoveto{\pgfqpoint{0.459750in}{0.412000in}}%
\pgfpathcurveto{\pgfqpoint{0.470800in}{0.412000in}}{\pgfqpoint{0.481399in}{0.416390in}}{\pgfqpoint{0.489213in}{0.424204in}}%
\pgfpathcurveto{\pgfqpoint{0.497026in}{0.432017in}}{\pgfqpoint{0.501417in}{0.442616in}}{\pgfqpoint{0.501417in}{0.453666in}}%
\pgfpathcurveto{\pgfqpoint{0.501417in}{0.464716in}}{\pgfqpoint{0.497026in}{0.475315in}}{\pgfqpoint{0.489213in}{0.483129in}}%
\pgfpathcurveto{\pgfqpoint{0.481399in}{0.490943in}}{\pgfqpoint{0.470800in}{0.495333in}}{\pgfqpoint{0.459750in}{0.495333in}}%
\pgfpathcurveto{\pgfqpoint{0.448700in}{0.495333in}}{\pgfqpoint{0.438101in}{0.490943in}}{\pgfqpoint{0.430287in}{0.483129in}}%
\pgfpathcurveto{\pgfqpoint{0.422474in}{0.475315in}}{\pgfqpoint{0.418083in}{0.464716in}}{\pgfqpoint{0.418083in}{0.453666in}}%
\pgfpathcurveto{\pgfqpoint{0.418083in}{0.442616in}}{\pgfqpoint{0.422474in}{0.432017in}}{\pgfqpoint{0.430287in}{0.424204in}}%
\pgfpathcurveto{\pgfqpoint{0.438101in}{0.416390in}}{\pgfqpoint{0.448700in}{0.412000in}}{\pgfqpoint{0.459750in}{0.412000in}}%
\pgfpathclose%
\pgfusepath{stroke,fill}%
\end{pgfscope}%
\begin{pgfscope}%
\pgfpathrectangle{\pgfqpoint{0.375000in}{0.330000in}}{\pgfqpoint{2.325000in}{2.310000in}}%
\pgfusepath{clip}%
\pgfsetbuttcap%
\pgfsetroundjoin%
\definecolor{currentfill}{rgb}{0.000000,0.000000,0.000000}%
\pgfsetfillcolor{currentfill}%
\pgfsetlinewidth{1.003750pt}%
\definecolor{currentstroke}{rgb}{0.000000,0.000000,0.000000}%
\pgfsetstrokecolor{currentstroke}%
\pgfsetdash{}{0pt}%
\pgfpathmoveto{\pgfqpoint{0.459750in}{1.443291in}}%
\pgfpathcurveto{\pgfqpoint{0.470800in}{1.443291in}}{\pgfqpoint{0.481399in}{1.447682in}}{\pgfqpoint{0.489213in}{1.455495in}}%
\pgfpathcurveto{\pgfqpoint{0.497026in}{1.463309in}}{\pgfqpoint{0.501417in}{1.473908in}}{\pgfqpoint{0.501417in}{1.484958in}}%
\pgfpathcurveto{\pgfqpoint{0.501417in}{1.496008in}}{\pgfqpoint{0.497026in}{1.506607in}}{\pgfqpoint{0.489213in}{1.514421in}}%
\pgfpathcurveto{\pgfqpoint{0.481399in}{1.522235in}}{\pgfqpoint{0.470800in}{1.526625in}}{\pgfqpoint{0.459750in}{1.526625in}}%
\pgfpathcurveto{\pgfqpoint{0.448700in}{1.526625in}}{\pgfqpoint{0.438101in}{1.522235in}}{\pgfqpoint{0.430287in}{1.514421in}}%
\pgfpathcurveto{\pgfqpoint{0.422474in}{1.506607in}}{\pgfqpoint{0.418083in}{1.496008in}}{\pgfqpoint{0.418083in}{1.484958in}}%
\pgfpathcurveto{\pgfqpoint{0.418083in}{1.473908in}}{\pgfqpoint{0.422474in}{1.463309in}}{\pgfqpoint{0.430287in}{1.455495in}}%
\pgfpathcurveto{\pgfqpoint{0.438101in}{1.447682in}}{\pgfqpoint{0.448700in}{1.443291in}}{\pgfqpoint{0.459750in}{1.443291in}}%
\pgfpathclose%
\pgfusepath{stroke,fill}%
\end{pgfscope}%
\begin{pgfscope}%
\pgfpathrectangle{\pgfqpoint{0.375000in}{0.330000in}}{\pgfqpoint{2.325000in}{2.310000in}}%
\pgfusepath{clip}%
\pgfsetbuttcap%
\pgfsetroundjoin%
\definecolor{currentfill}{rgb}{0.000000,0.000000,0.000000}%
\pgfsetfillcolor{currentfill}%
\pgfsetlinewidth{1.003750pt}%
\definecolor{currentstroke}{rgb}{0.000000,0.000000,0.000000}%
\pgfsetstrokecolor{currentstroke}%
\pgfsetdash{}{0pt}%
\pgfpathmoveto{\pgfqpoint{0.459750in}{0.412000in}}%
\pgfpathcurveto{\pgfqpoint{0.470800in}{0.412000in}}{\pgfqpoint{0.481399in}{0.416390in}}{\pgfqpoint{0.489213in}{0.424204in}}%
\pgfpathcurveto{\pgfqpoint{0.497026in}{0.432017in}}{\pgfqpoint{0.501417in}{0.442616in}}{\pgfqpoint{0.501417in}{0.453666in}}%
\pgfpathcurveto{\pgfqpoint{0.501417in}{0.464716in}}{\pgfqpoint{0.497026in}{0.475315in}}{\pgfqpoint{0.489213in}{0.483129in}}%
\pgfpathcurveto{\pgfqpoint{0.481399in}{0.490943in}}{\pgfqpoint{0.470800in}{0.495333in}}{\pgfqpoint{0.459750in}{0.495333in}}%
\pgfpathcurveto{\pgfqpoint{0.448700in}{0.495333in}}{\pgfqpoint{0.438101in}{0.490943in}}{\pgfqpoint{0.430287in}{0.483129in}}%
\pgfpathcurveto{\pgfqpoint{0.422474in}{0.475315in}}{\pgfqpoint{0.418083in}{0.464716in}}{\pgfqpoint{0.418083in}{0.453666in}}%
\pgfpathcurveto{\pgfqpoint{0.418083in}{0.442616in}}{\pgfqpoint{0.422474in}{0.432017in}}{\pgfqpoint{0.430287in}{0.424204in}}%
\pgfpathcurveto{\pgfqpoint{0.438101in}{0.416390in}}{\pgfqpoint{0.448700in}{0.412000in}}{\pgfqpoint{0.459750in}{0.412000in}}%
\pgfpathclose%
\pgfusepath{stroke,fill}%
\end{pgfscope}%
\begin{pgfscope}%
\pgfpathrectangle{\pgfqpoint{0.375000in}{0.330000in}}{\pgfqpoint{2.325000in}{2.310000in}}%
\pgfusepath{clip}%
\pgfsetbuttcap%
\pgfsetroundjoin%
\definecolor{currentfill}{rgb}{0.000000,0.000000,0.000000}%
\pgfsetfillcolor{currentfill}%
\pgfsetlinewidth{1.003750pt}%
\definecolor{currentstroke}{rgb}{0.000000,0.000000,0.000000}%
\pgfsetstrokecolor{currentstroke}%
\pgfsetdash{}{0pt}%
\pgfpathmoveto{\pgfqpoint{0.459750in}{0.412000in}}%
\pgfpathcurveto{\pgfqpoint{0.470800in}{0.412000in}}{\pgfqpoint{0.481399in}{0.416390in}}{\pgfqpoint{0.489213in}{0.424204in}}%
\pgfpathcurveto{\pgfqpoint{0.497026in}{0.432017in}}{\pgfqpoint{0.501417in}{0.442616in}}{\pgfqpoint{0.501417in}{0.453666in}}%
\pgfpathcurveto{\pgfqpoint{0.501417in}{0.464716in}}{\pgfqpoint{0.497026in}{0.475315in}}{\pgfqpoint{0.489213in}{0.483129in}}%
\pgfpathcurveto{\pgfqpoint{0.481399in}{0.490943in}}{\pgfqpoint{0.470800in}{0.495333in}}{\pgfqpoint{0.459750in}{0.495333in}}%
\pgfpathcurveto{\pgfqpoint{0.448700in}{0.495333in}}{\pgfqpoint{0.438101in}{0.490943in}}{\pgfqpoint{0.430287in}{0.483129in}}%
\pgfpathcurveto{\pgfqpoint{0.422474in}{0.475315in}}{\pgfqpoint{0.418083in}{0.464716in}}{\pgfqpoint{0.418083in}{0.453666in}}%
\pgfpathcurveto{\pgfqpoint{0.418083in}{0.442616in}}{\pgfqpoint{0.422474in}{0.432017in}}{\pgfqpoint{0.430287in}{0.424204in}}%
\pgfpathcurveto{\pgfqpoint{0.438101in}{0.416390in}}{\pgfqpoint{0.448700in}{0.412000in}}{\pgfqpoint{0.459750in}{0.412000in}}%
\pgfpathclose%
\pgfusepath{stroke,fill}%
\end{pgfscope}%
\begin{pgfscope}%
\pgfpathrectangle{\pgfqpoint{0.375000in}{0.330000in}}{\pgfqpoint{2.325000in}{2.310000in}}%
\pgfusepath{clip}%
\pgfsetbuttcap%
\pgfsetroundjoin%
\definecolor{currentfill}{rgb}{0.000000,0.000000,0.000000}%
\pgfsetfillcolor{currentfill}%
\pgfsetlinewidth{1.003750pt}%
\definecolor{currentstroke}{rgb}{0.000000,0.000000,0.000000}%
\pgfsetstrokecolor{currentstroke}%
\pgfsetdash{}{0pt}%
\pgfpathmoveto{\pgfqpoint{0.459750in}{0.412000in}}%
\pgfpathcurveto{\pgfqpoint{0.470800in}{0.412000in}}{\pgfqpoint{0.481399in}{0.416390in}}{\pgfqpoint{0.489213in}{0.424204in}}%
\pgfpathcurveto{\pgfqpoint{0.497026in}{0.432017in}}{\pgfqpoint{0.501417in}{0.442616in}}{\pgfqpoint{0.501417in}{0.453666in}}%
\pgfpathcurveto{\pgfqpoint{0.501417in}{0.464716in}}{\pgfqpoint{0.497026in}{0.475315in}}{\pgfqpoint{0.489213in}{0.483129in}}%
\pgfpathcurveto{\pgfqpoint{0.481399in}{0.490943in}}{\pgfqpoint{0.470800in}{0.495333in}}{\pgfqpoint{0.459750in}{0.495333in}}%
\pgfpathcurveto{\pgfqpoint{0.448700in}{0.495333in}}{\pgfqpoint{0.438101in}{0.490943in}}{\pgfqpoint{0.430287in}{0.483129in}}%
\pgfpathcurveto{\pgfqpoint{0.422474in}{0.475315in}}{\pgfqpoint{0.418083in}{0.464716in}}{\pgfqpoint{0.418083in}{0.453666in}}%
\pgfpathcurveto{\pgfqpoint{0.418083in}{0.442616in}}{\pgfqpoint{0.422474in}{0.432017in}}{\pgfqpoint{0.430287in}{0.424204in}}%
\pgfpathcurveto{\pgfqpoint{0.438101in}{0.416390in}}{\pgfqpoint{0.448700in}{0.412000in}}{\pgfqpoint{0.459750in}{0.412000in}}%
\pgfpathclose%
\pgfusepath{stroke,fill}%
\end{pgfscope}%
\begin{pgfscope}%
\pgfpathrectangle{\pgfqpoint{0.375000in}{0.330000in}}{\pgfqpoint{2.325000in}{2.310000in}}%
\pgfusepath{clip}%
\pgfsetbuttcap%
\pgfsetroundjoin%
\definecolor{currentfill}{rgb}{0.000000,0.000000,0.000000}%
\pgfsetfillcolor{currentfill}%
\pgfsetlinewidth{1.003750pt}%
\definecolor{currentstroke}{rgb}{0.000000,0.000000,0.000000}%
\pgfsetstrokecolor{currentstroke}%
\pgfsetdash{}{0pt}%
\pgfpathmoveto{\pgfqpoint{0.459750in}{0.412000in}}%
\pgfpathcurveto{\pgfqpoint{0.470800in}{0.412000in}}{\pgfqpoint{0.481399in}{0.416390in}}{\pgfqpoint{0.489213in}{0.424204in}}%
\pgfpathcurveto{\pgfqpoint{0.497026in}{0.432017in}}{\pgfqpoint{0.501417in}{0.442616in}}{\pgfqpoint{0.501417in}{0.453666in}}%
\pgfpathcurveto{\pgfqpoint{0.501417in}{0.464716in}}{\pgfqpoint{0.497026in}{0.475315in}}{\pgfqpoint{0.489213in}{0.483129in}}%
\pgfpathcurveto{\pgfqpoint{0.481399in}{0.490943in}}{\pgfqpoint{0.470800in}{0.495333in}}{\pgfqpoint{0.459750in}{0.495333in}}%
\pgfpathcurveto{\pgfqpoint{0.448700in}{0.495333in}}{\pgfqpoint{0.438101in}{0.490943in}}{\pgfqpoint{0.430287in}{0.483129in}}%
\pgfpathcurveto{\pgfqpoint{0.422474in}{0.475315in}}{\pgfqpoint{0.418083in}{0.464716in}}{\pgfqpoint{0.418083in}{0.453666in}}%
\pgfpathcurveto{\pgfqpoint{0.418083in}{0.442616in}}{\pgfqpoint{0.422474in}{0.432017in}}{\pgfqpoint{0.430287in}{0.424204in}}%
\pgfpathcurveto{\pgfqpoint{0.438101in}{0.416390in}}{\pgfqpoint{0.448700in}{0.412000in}}{\pgfqpoint{0.459750in}{0.412000in}}%
\pgfpathclose%
\pgfusepath{stroke,fill}%
\end{pgfscope}%
\begin{pgfscope}%
\pgfpathrectangle{\pgfqpoint{0.375000in}{0.330000in}}{\pgfqpoint{2.325000in}{2.310000in}}%
\pgfusepath{clip}%
\pgfsetbuttcap%
\pgfsetroundjoin%
\definecolor{currentfill}{rgb}{0.000000,0.000000,0.000000}%
\pgfsetfillcolor{currentfill}%
\pgfsetlinewidth{1.003750pt}%
\definecolor{currentstroke}{rgb}{0.000000,0.000000,0.000000}%
\pgfsetstrokecolor{currentstroke}%
\pgfsetdash{}{0pt}%
\pgfpathmoveto{\pgfqpoint{0.459750in}{1.443291in}}%
\pgfpathcurveto{\pgfqpoint{0.470800in}{1.443291in}}{\pgfqpoint{0.481399in}{1.447682in}}{\pgfqpoint{0.489213in}{1.455495in}}%
\pgfpathcurveto{\pgfqpoint{0.497026in}{1.463309in}}{\pgfqpoint{0.501417in}{1.473908in}}{\pgfqpoint{0.501417in}{1.484958in}}%
\pgfpathcurveto{\pgfqpoint{0.501417in}{1.496008in}}{\pgfqpoint{0.497026in}{1.506607in}}{\pgfqpoint{0.489213in}{1.514421in}}%
\pgfpathcurveto{\pgfqpoint{0.481399in}{1.522235in}}{\pgfqpoint{0.470800in}{1.526625in}}{\pgfqpoint{0.459750in}{1.526625in}}%
\pgfpathcurveto{\pgfqpoint{0.448700in}{1.526625in}}{\pgfqpoint{0.438101in}{1.522235in}}{\pgfqpoint{0.430287in}{1.514421in}}%
\pgfpathcurveto{\pgfqpoint{0.422474in}{1.506607in}}{\pgfqpoint{0.418083in}{1.496008in}}{\pgfqpoint{0.418083in}{1.484958in}}%
\pgfpathcurveto{\pgfqpoint{0.418083in}{1.473908in}}{\pgfqpoint{0.422474in}{1.463309in}}{\pgfqpoint{0.430287in}{1.455495in}}%
\pgfpathcurveto{\pgfqpoint{0.438101in}{1.447682in}}{\pgfqpoint{0.448700in}{1.443291in}}{\pgfqpoint{0.459750in}{1.443291in}}%
\pgfpathclose%
\pgfusepath{stroke,fill}%
\end{pgfscope}%
\begin{pgfscope}%
\pgfpathrectangle{\pgfqpoint{0.375000in}{0.330000in}}{\pgfqpoint{2.325000in}{2.310000in}}%
\pgfusepath{clip}%
\pgfsetbuttcap%
\pgfsetroundjoin%
\definecolor{currentfill}{rgb}{0.000000,0.000000,0.000000}%
\pgfsetfillcolor{currentfill}%
\pgfsetlinewidth{1.003750pt}%
\definecolor{currentstroke}{rgb}{0.000000,0.000000,0.000000}%
\pgfsetstrokecolor{currentstroke}%
\pgfsetdash{}{0pt}%
\pgfpathmoveto{\pgfqpoint{0.459750in}{1.443291in}}%
\pgfpathcurveto{\pgfqpoint{0.470800in}{1.443291in}}{\pgfqpoint{0.481399in}{1.447682in}}{\pgfqpoint{0.489213in}{1.455495in}}%
\pgfpathcurveto{\pgfqpoint{0.497026in}{1.463309in}}{\pgfqpoint{0.501417in}{1.473908in}}{\pgfqpoint{0.501417in}{1.484958in}}%
\pgfpathcurveto{\pgfqpoint{0.501417in}{1.496008in}}{\pgfqpoint{0.497026in}{1.506607in}}{\pgfqpoint{0.489213in}{1.514421in}}%
\pgfpathcurveto{\pgfqpoint{0.481399in}{1.522235in}}{\pgfqpoint{0.470800in}{1.526625in}}{\pgfqpoint{0.459750in}{1.526625in}}%
\pgfpathcurveto{\pgfqpoint{0.448700in}{1.526625in}}{\pgfqpoint{0.438101in}{1.522235in}}{\pgfqpoint{0.430287in}{1.514421in}}%
\pgfpathcurveto{\pgfqpoint{0.422474in}{1.506607in}}{\pgfqpoint{0.418083in}{1.496008in}}{\pgfqpoint{0.418083in}{1.484958in}}%
\pgfpathcurveto{\pgfqpoint{0.418083in}{1.473908in}}{\pgfqpoint{0.422474in}{1.463309in}}{\pgfqpoint{0.430287in}{1.455495in}}%
\pgfpathcurveto{\pgfqpoint{0.438101in}{1.447682in}}{\pgfqpoint{0.448700in}{1.443291in}}{\pgfqpoint{0.459750in}{1.443291in}}%
\pgfpathclose%
\pgfusepath{stroke,fill}%
\end{pgfscope}%
\begin{pgfscope}%
\pgfpathrectangle{\pgfqpoint{0.375000in}{0.330000in}}{\pgfqpoint{2.325000in}{2.310000in}}%
\pgfusepath{clip}%
\pgfsetbuttcap%
\pgfsetroundjoin%
\definecolor{currentfill}{rgb}{0.000000,0.000000,0.000000}%
\pgfsetfillcolor{currentfill}%
\pgfsetlinewidth{1.003750pt}%
\definecolor{currentstroke}{rgb}{0.000000,0.000000,0.000000}%
\pgfsetstrokecolor{currentstroke}%
\pgfsetdash{}{0pt}%
\pgfpathmoveto{\pgfqpoint{0.459750in}{1.443291in}}%
\pgfpathcurveto{\pgfqpoint{0.470800in}{1.443291in}}{\pgfqpoint{0.481399in}{1.447682in}}{\pgfqpoint{0.489213in}{1.455495in}}%
\pgfpathcurveto{\pgfqpoint{0.497026in}{1.463309in}}{\pgfqpoint{0.501417in}{1.473908in}}{\pgfqpoint{0.501417in}{1.484958in}}%
\pgfpathcurveto{\pgfqpoint{0.501417in}{1.496008in}}{\pgfqpoint{0.497026in}{1.506607in}}{\pgfqpoint{0.489213in}{1.514421in}}%
\pgfpathcurveto{\pgfqpoint{0.481399in}{1.522235in}}{\pgfqpoint{0.470800in}{1.526625in}}{\pgfqpoint{0.459750in}{1.526625in}}%
\pgfpathcurveto{\pgfqpoint{0.448700in}{1.526625in}}{\pgfqpoint{0.438101in}{1.522235in}}{\pgfqpoint{0.430287in}{1.514421in}}%
\pgfpathcurveto{\pgfqpoint{0.422474in}{1.506607in}}{\pgfqpoint{0.418083in}{1.496008in}}{\pgfqpoint{0.418083in}{1.484958in}}%
\pgfpathcurveto{\pgfqpoint{0.418083in}{1.473908in}}{\pgfqpoint{0.422474in}{1.463309in}}{\pgfqpoint{0.430287in}{1.455495in}}%
\pgfpathcurveto{\pgfqpoint{0.438101in}{1.447682in}}{\pgfqpoint{0.448700in}{1.443291in}}{\pgfqpoint{0.459750in}{1.443291in}}%
\pgfpathclose%
\pgfusepath{stroke,fill}%
\end{pgfscope}%
\begin{pgfscope}%
\pgfpathrectangle{\pgfqpoint{0.375000in}{0.330000in}}{\pgfqpoint{2.325000in}{2.310000in}}%
\pgfusepath{clip}%
\pgfsetbuttcap%
\pgfsetroundjoin%
\definecolor{currentfill}{rgb}{0.000000,0.000000,0.000000}%
\pgfsetfillcolor{currentfill}%
\pgfsetlinewidth{1.003750pt}%
\definecolor{currentstroke}{rgb}{0.000000,0.000000,0.000000}%
\pgfsetstrokecolor{currentstroke}%
\pgfsetdash{}{0pt}%
\pgfpathmoveto{\pgfqpoint{0.459750in}{1.443291in}}%
\pgfpathcurveto{\pgfqpoint{0.470800in}{1.443291in}}{\pgfqpoint{0.481399in}{1.447682in}}{\pgfqpoint{0.489213in}{1.455495in}}%
\pgfpathcurveto{\pgfqpoint{0.497026in}{1.463309in}}{\pgfqpoint{0.501417in}{1.473908in}}{\pgfqpoint{0.501417in}{1.484958in}}%
\pgfpathcurveto{\pgfqpoint{0.501417in}{1.496008in}}{\pgfqpoint{0.497026in}{1.506607in}}{\pgfqpoint{0.489213in}{1.514421in}}%
\pgfpathcurveto{\pgfqpoint{0.481399in}{1.522235in}}{\pgfqpoint{0.470800in}{1.526625in}}{\pgfqpoint{0.459750in}{1.526625in}}%
\pgfpathcurveto{\pgfqpoint{0.448700in}{1.526625in}}{\pgfqpoint{0.438101in}{1.522235in}}{\pgfqpoint{0.430287in}{1.514421in}}%
\pgfpathcurveto{\pgfqpoint{0.422474in}{1.506607in}}{\pgfqpoint{0.418083in}{1.496008in}}{\pgfqpoint{0.418083in}{1.484958in}}%
\pgfpathcurveto{\pgfqpoint{0.418083in}{1.473908in}}{\pgfqpoint{0.422474in}{1.463309in}}{\pgfqpoint{0.430287in}{1.455495in}}%
\pgfpathcurveto{\pgfqpoint{0.438101in}{1.447682in}}{\pgfqpoint{0.448700in}{1.443291in}}{\pgfqpoint{0.459750in}{1.443291in}}%
\pgfpathclose%
\pgfusepath{stroke,fill}%
\end{pgfscope}%
\begin{pgfscope}%
\pgfpathrectangle{\pgfqpoint{0.375000in}{0.330000in}}{\pgfqpoint{2.325000in}{2.310000in}}%
\pgfusepath{clip}%
\pgfsetbuttcap%
\pgfsetroundjoin%
\definecolor{currentfill}{rgb}{0.000000,0.000000,0.000000}%
\pgfsetfillcolor{currentfill}%
\pgfsetlinewidth{1.003750pt}%
\definecolor{currentstroke}{rgb}{0.000000,0.000000,0.000000}%
\pgfsetstrokecolor{currentstroke}%
\pgfsetdash{}{0pt}%
\pgfpathmoveto{\pgfqpoint{0.459750in}{0.412000in}}%
\pgfpathcurveto{\pgfqpoint{0.470800in}{0.412000in}}{\pgfqpoint{0.481399in}{0.416390in}}{\pgfqpoint{0.489213in}{0.424204in}}%
\pgfpathcurveto{\pgfqpoint{0.497026in}{0.432017in}}{\pgfqpoint{0.501417in}{0.442616in}}{\pgfqpoint{0.501417in}{0.453666in}}%
\pgfpathcurveto{\pgfqpoint{0.501417in}{0.464716in}}{\pgfqpoint{0.497026in}{0.475315in}}{\pgfqpoint{0.489213in}{0.483129in}}%
\pgfpathcurveto{\pgfqpoint{0.481399in}{0.490943in}}{\pgfqpoint{0.470800in}{0.495333in}}{\pgfqpoint{0.459750in}{0.495333in}}%
\pgfpathcurveto{\pgfqpoint{0.448700in}{0.495333in}}{\pgfqpoint{0.438101in}{0.490943in}}{\pgfqpoint{0.430287in}{0.483129in}}%
\pgfpathcurveto{\pgfqpoint{0.422474in}{0.475315in}}{\pgfqpoint{0.418083in}{0.464716in}}{\pgfqpoint{0.418083in}{0.453666in}}%
\pgfpathcurveto{\pgfqpoint{0.418083in}{0.442616in}}{\pgfqpoint{0.422474in}{0.432017in}}{\pgfqpoint{0.430287in}{0.424204in}}%
\pgfpathcurveto{\pgfqpoint{0.438101in}{0.416390in}}{\pgfqpoint{0.448700in}{0.412000in}}{\pgfqpoint{0.459750in}{0.412000in}}%
\pgfpathclose%
\pgfusepath{stroke,fill}%
\end{pgfscope}%
\begin{pgfscope}%
\pgfpathrectangle{\pgfqpoint{0.375000in}{0.330000in}}{\pgfqpoint{2.325000in}{2.310000in}}%
\pgfusepath{clip}%
\pgfsetbuttcap%
\pgfsetroundjoin%
\definecolor{currentfill}{rgb}{0.000000,0.000000,0.000000}%
\pgfsetfillcolor{currentfill}%
\pgfsetlinewidth{1.003750pt}%
\definecolor{currentstroke}{rgb}{0.000000,0.000000,0.000000}%
\pgfsetstrokecolor{currentstroke}%
\pgfsetdash{}{0pt}%
\pgfpathmoveto{\pgfqpoint{0.459750in}{0.412000in}}%
\pgfpathcurveto{\pgfqpoint{0.470800in}{0.412000in}}{\pgfqpoint{0.481399in}{0.416390in}}{\pgfqpoint{0.489213in}{0.424204in}}%
\pgfpathcurveto{\pgfqpoint{0.497026in}{0.432017in}}{\pgfqpoint{0.501417in}{0.442616in}}{\pgfqpoint{0.501417in}{0.453666in}}%
\pgfpathcurveto{\pgfqpoint{0.501417in}{0.464716in}}{\pgfqpoint{0.497026in}{0.475315in}}{\pgfqpoint{0.489213in}{0.483129in}}%
\pgfpathcurveto{\pgfqpoint{0.481399in}{0.490943in}}{\pgfqpoint{0.470800in}{0.495333in}}{\pgfqpoint{0.459750in}{0.495333in}}%
\pgfpathcurveto{\pgfqpoint{0.448700in}{0.495333in}}{\pgfqpoint{0.438101in}{0.490943in}}{\pgfqpoint{0.430287in}{0.483129in}}%
\pgfpathcurveto{\pgfqpoint{0.422474in}{0.475315in}}{\pgfqpoint{0.418083in}{0.464716in}}{\pgfqpoint{0.418083in}{0.453666in}}%
\pgfpathcurveto{\pgfqpoint{0.418083in}{0.442616in}}{\pgfqpoint{0.422474in}{0.432017in}}{\pgfqpoint{0.430287in}{0.424204in}}%
\pgfpathcurveto{\pgfqpoint{0.438101in}{0.416390in}}{\pgfqpoint{0.448700in}{0.412000in}}{\pgfqpoint{0.459750in}{0.412000in}}%
\pgfpathclose%
\pgfusepath{stroke,fill}%
\end{pgfscope}%
\begin{pgfscope}%
\pgfpathrectangle{\pgfqpoint{0.375000in}{0.330000in}}{\pgfqpoint{2.325000in}{2.310000in}}%
\pgfusepath{clip}%
\pgfsetbuttcap%
\pgfsetroundjoin%
\definecolor{currentfill}{rgb}{0.000000,0.000000,0.000000}%
\pgfsetfillcolor{currentfill}%
\pgfsetlinewidth{1.003750pt}%
\definecolor{currentstroke}{rgb}{0.000000,0.000000,0.000000}%
\pgfsetstrokecolor{currentstroke}%
\pgfsetdash{}{0pt}%
\pgfpathmoveto{\pgfqpoint{0.459750in}{0.412000in}}%
\pgfpathcurveto{\pgfqpoint{0.470800in}{0.412000in}}{\pgfqpoint{0.481399in}{0.416390in}}{\pgfqpoint{0.489213in}{0.424204in}}%
\pgfpathcurveto{\pgfqpoint{0.497026in}{0.432017in}}{\pgfqpoint{0.501417in}{0.442616in}}{\pgfqpoint{0.501417in}{0.453666in}}%
\pgfpathcurveto{\pgfqpoint{0.501417in}{0.464716in}}{\pgfqpoint{0.497026in}{0.475315in}}{\pgfqpoint{0.489213in}{0.483129in}}%
\pgfpathcurveto{\pgfqpoint{0.481399in}{0.490943in}}{\pgfqpoint{0.470800in}{0.495333in}}{\pgfqpoint{0.459750in}{0.495333in}}%
\pgfpathcurveto{\pgfqpoint{0.448700in}{0.495333in}}{\pgfqpoint{0.438101in}{0.490943in}}{\pgfqpoint{0.430287in}{0.483129in}}%
\pgfpathcurveto{\pgfqpoint{0.422474in}{0.475315in}}{\pgfqpoint{0.418083in}{0.464716in}}{\pgfqpoint{0.418083in}{0.453666in}}%
\pgfpathcurveto{\pgfqpoint{0.418083in}{0.442616in}}{\pgfqpoint{0.422474in}{0.432017in}}{\pgfqpoint{0.430287in}{0.424204in}}%
\pgfpathcurveto{\pgfqpoint{0.438101in}{0.416390in}}{\pgfqpoint{0.448700in}{0.412000in}}{\pgfqpoint{0.459750in}{0.412000in}}%
\pgfpathclose%
\pgfusepath{stroke,fill}%
\end{pgfscope}%
\begin{pgfscope}%
\pgfpathrectangle{\pgfqpoint{0.375000in}{0.330000in}}{\pgfqpoint{2.325000in}{2.310000in}}%
\pgfusepath{clip}%
\pgfsetbuttcap%
\pgfsetroundjoin%
\definecolor{currentfill}{rgb}{0.000000,0.000000,0.000000}%
\pgfsetfillcolor{currentfill}%
\pgfsetlinewidth{1.003750pt}%
\definecolor{currentstroke}{rgb}{0.000000,0.000000,0.000000}%
\pgfsetstrokecolor{currentstroke}%
\pgfsetdash{}{0pt}%
\pgfpathmoveto{\pgfqpoint{0.459750in}{0.412000in}}%
\pgfpathcurveto{\pgfqpoint{0.470800in}{0.412000in}}{\pgfqpoint{0.481399in}{0.416390in}}{\pgfqpoint{0.489213in}{0.424204in}}%
\pgfpathcurveto{\pgfqpoint{0.497026in}{0.432017in}}{\pgfqpoint{0.501417in}{0.442616in}}{\pgfqpoint{0.501417in}{0.453666in}}%
\pgfpathcurveto{\pgfqpoint{0.501417in}{0.464716in}}{\pgfqpoint{0.497026in}{0.475315in}}{\pgfqpoint{0.489213in}{0.483129in}}%
\pgfpathcurveto{\pgfqpoint{0.481399in}{0.490943in}}{\pgfqpoint{0.470800in}{0.495333in}}{\pgfqpoint{0.459750in}{0.495333in}}%
\pgfpathcurveto{\pgfqpoint{0.448700in}{0.495333in}}{\pgfqpoint{0.438101in}{0.490943in}}{\pgfqpoint{0.430287in}{0.483129in}}%
\pgfpathcurveto{\pgfqpoint{0.422474in}{0.475315in}}{\pgfqpoint{0.418083in}{0.464716in}}{\pgfqpoint{0.418083in}{0.453666in}}%
\pgfpathcurveto{\pgfqpoint{0.418083in}{0.442616in}}{\pgfqpoint{0.422474in}{0.432017in}}{\pgfqpoint{0.430287in}{0.424204in}}%
\pgfpathcurveto{\pgfqpoint{0.438101in}{0.416390in}}{\pgfqpoint{0.448700in}{0.412000in}}{\pgfqpoint{0.459750in}{0.412000in}}%
\pgfpathclose%
\pgfusepath{stroke,fill}%
\end{pgfscope}%
\begin{pgfscope}%
\pgfpathrectangle{\pgfqpoint{0.375000in}{0.330000in}}{\pgfqpoint{2.325000in}{2.310000in}}%
\pgfusepath{clip}%
\pgfsetbuttcap%
\pgfsetroundjoin%
\definecolor{currentfill}{rgb}{0.000000,0.000000,0.000000}%
\pgfsetfillcolor{currentfill}%
\pgfsetlinewidth{1.003750pt}%
\definecolor{currentstroke}{rgb}{0.000000,0.000000,0.000000}%
\pgfsetstrokecolor{currentstroke}%
\pgfsetdash{}{0pt}%
\pgfpathmoveto{\pgfqpoint{0.459750in}{0.412000in}}%
\pgfpathcurveto{\pgfqpoint{0.470800in}{0.412000in}}{\pgfqpoint{0.481399in}{0.416390in}}{\pgfqpoint{0.489213in}{0.424204in}}%
\pgfpathcurveto{\pgfqpoint{0.497026in}{0.432017in}}{\pgfqpoint{0.501417in}{0.442616in}}{\pgfqpoint{0.501417in}{0.453666in}}%
\pgfpathcurveto{\pgfqpoint{0.501417in}{0.464716in}}{\pgfqpoint{0.497026in}{0.475315in}}{\pgfqpoint{0.489213in}{0.483129in}}%
\pgfpathcurveto{\pgfqpoint{0.481399in}{0.490943in}}{\pgfqpoint{0.470800in}{0.495333in}}{\pgfqpoint{0.459750in}{0.495333in}}%
\pgfpathcurveto{\pgfqpoint{0.448700in}{0.495333in}}{\pgfqpoint{0.438101in}{0.490943in}}{\pgfqpoint{0.430287in}{0.483129in}}%
\pgfpathcurveto{\pgfqpoint{0.422474in}{0.475315in}}{\pgfqpoint{0.418083in}{0.464716in}}{\pgfqpoint{0.418083in}{0.453666in}}%
\pgfpathcurveto{\pgfqpoint{0.418083in}{0.442616in}}{\pgfqpoint{0.422474in}{0.432017in}}{\pgfqpoint{0.430287in}{0.424204in}}%
\pgfpathcurveto{\pgfqpoint{0.438101in}{0.416390in}}{\pgfqpoint{0.448700in}{0.412000in}}{\pgfqpoint{0.459750in}{0.412000in}}%
\pgfpathclose%
\pgfusepath{stroke,fill}%
\end{pgfscope}%
\begin{pgfscope}%
\pgfpathrectangle{\pgfqpoint{0.375000in}{0.330000in}}{\pgfqpoint{2.325000in}{2.310000in}}%
\pgfusepath{clip}%
\pgfsetbuttcap%
\pgfsetroundjoin%
\definecolor{currentfill}{rgb}{0.000000,0.000000,0.000000}%
\pgfsetfillcolor{currentfill}%
\pgfsetlinewidth{1.003750pt}%
\definecolor{currentstroke}{rgb}{0.000000,0.000000,0.000000}%
\pgfsetstrokecolor{currentstroke}%
\pgfsetdash{}{0pt}%
\pgfpathmoveto{\pgfqpoint{0.459750in}{0.412000in}}%
\pgfpathcurveto{\pgfqpoint{0.470800in}{0.412000in}}{\pgfqpoint{0.481399in}{0.416390in}}{\pgfqpoint{0.489213in}{0.424204in}}%
\pgfpathcurveto{\pgfqpoint{0.497026in}{0.432017in}}{\pgfqpoint{0.501417in}{0.442616in}}{\pgfqpoint{0.501417in}{0.453666in}}%
\pgfpathcurveto{\pgfqpoint{0.501417in}{0.464716in}}{\pgfqpoint{0.497026in}{0.475315in}}{\pgfqpoint{0.489213in}{0.483129in}}%
\pgfpathcurveto{\pgfqpoint{0.481399in}{0.490943in}}{\pgfqpoint{0.470800in}{0.495333in}}{\pgfqpoint{0.459750in}{0.495333in}}%
\pgfpathcurveto{\pgfqpoint{0.448700in}{0.495333in}}{\pgfqpoint{0.438101in}{0.490943in}}{\pgfqpoint{0.430287in}{0.483129in}}%
\pgfpathcurveto{\pgfqpoint{0.422474in}{0.475315in}}{\pgfqpoint{0.418083in}{0.464716in}}{\pgfqpoint{0.418083in}{0.453666in}}%
\pgfpathcurveto{\pgfqpoint{0.418083in}{0.442616in}}{\pgfqpoint{0.422474in}{0.432017in}}{\pgfqpoint{0.430287in}{0.424204in}}%
\pgfpathcurveto{\pgfqpoint{0.438101in}{0.416390in}}{\pgfqpoint{0.448700in}{0.412000in}}{\pgfqpoint{0.459750in}{0.412000in}}%
\pgfpathclose%
\pgfusepath{stroke,fill}%
\end{pgfscope}%
\begin{pgfscope}%
\pgfpathrectangle{\pgfqpoint{0.375000in}{0.330000in}}{\pgfqpoint{2.325000in}{2.310000in}}%
\pgfusepath{clip}%
\pgfsetbuttcap%
\pgfsetroundjoin%
\definecolor{currentfill}{rgb}{0.000000,0.000000,0.000000}%
\pgfsetfillcolor{currentfill}%
\pgfsetlinewidth{1.003750pt}%
\definecolor{currentstroke}{rgb}{0.000000,0.000000,0.000000}%
\pgfsetstrokecolor{currentstroke}%
\pgfsetdash{}{0pt}%
\pgfpathmoveto{\pgfqpoint{0.459750in}{1.443291in}}%
\pgfpathcurveto{\pgfqpoint{0.470800in}{1.443291in}}{\pgfqpoint{0.481399in}{1.447682in}}{\pgfqpoint{0.489213in}{1.455495in}}%
\pgfpathcurveto{\pgfqpoint{0.497026in}{1.463309in}}{\pgfqpoint{0.501417in}{1.473908in}}{\pgfqpoint{0.501417in}{1.484958in}}%
\pgfpathcurveto{\pgfqpoint{0.501417in}{1.496008in}}{\pgfqpoint{0.497026in}{1.506607in}}{\pgfqpoint{0.489213in}{1.514421in}}%
\pgfpathcurveto{\pgfqpoint{0.481399in}{1.522235in}}{\pgfqpoint{0.470800in}{1.526625in}}{\pgfqpoint{0.459750in}{1.526625in}}%
\pgfpathcurveto{\pgfqpoint{0.448700in}{1.526625in}}{\pgfqpoint{0.438101in}{1.522235in}}{\pgfqpoint{0.430287in}{1.514421in}}%
\pgfpathcurveto{\pgfqpoint{0.422474in}{1.506607in}}{\pgfqpoint{0.418083in}{1.496008in}}{\pgfqpoint{0.418083in}{1.484958in}}%
\pgfpathcurveto{\pgfqpoint{0.418083in}{1.473908in}}{\pgfqpoint{0.422474in}{1.463309in}}{\pgfqpoint{0.430287in}{1.455495in}}%
\pgfpathcurveto{\pgfqpoint{0.438101in}{1.447682in}}{\pgfqpoint{0.448700in}{1.443291in}}{\pgfqpoint{0.459750in}{1.443291in}}%
\pgfpathclose%
\pgfusepath{stroke,fill}%
\end{pgfscope}%
\begin{pgfscope}%
\pgfpathrectangle{\pgfqpoint{0.375000in}{0.330000in}}{\pgfqpoint{2.325000in}{2.310000in}}%
\pgfusepath{clip}%
\pgfsetbuttcap%
\pgfsetroundjoin%
\definecolor{currentfill}{rgb}{0.000000,0.000000,0.000000}%
\pgfsetfillcolor{currentfill}%
\pgfsetlinewidth{1.003750pt}%
\definecolor{currentstroke}{rgb}{0.000000,0.000000,0.000000}%
\pgfsetstrokecolor{currentstroke}%
\pgfsetdash{}{0pt}%
\pgfpathmoveto{\pgfqpoint{0.459750in}{1.443291in}}%
\pgfpathcurveto{\pgfqpoint{0.470800in}{1.443291in}}{\pgfqpoint{0.481399in}{1.447682in}}{\pgfqpoint{0.489213in}{1.455495in}}%
\pgfpathcurveto{\pgfqpoint{0.497026in}{1.463309in}}{\pgfqpoint{0.501417in}{1.473908in}}{\pgfqpoint{0.501417in}{1.484958in}}%
\pgfpathcurveto{\pgfqpoint{0.501417in}{1.496008in}}{\pgfqpoint{0.497026in}{1.506607in}}{\pgfqpoint{0.489213in}{1.514421in}}%
\pgfpathcurveto{\pgfqpoint{0.481399in}{1.522235in}}{\pgfqpoint{0.470800in}{1.526625in}}{\pgfqpoint{0.459750in}{1.526625in}}%
\pgfpathcurveto{\pgfqpoint{0.448700in}{1.526625in}}{\pgfqpoint{0.438101in}{1.522235in}}{\pgfqpoint{0.430287in}{1.514421in}}%
\pgfpathcurveto{\pgfqpoint{0.422474in}{1.506607in}}{\pgfqpoint{0.418083in}{1.496008in}}{\pgfqpoint{0.418083in}{1.484958in}}%
\pgfpathcurveto{\pgfqpoint{0.418083in}{1.473908in}}{\pgfqpoint{0.422474in}{1.463309in}}{\pgfqpoint{0.430287in}{1.455495in}}%
\pgfpathcurveto{\pgfqpoint{0.438101in}{1.447682in}}{\pgfqpoint{0.448700in}{1.443291in}}{\pgfqpoint{0.459750in}{1.443291in}}%
\pgfpathclose%
\pgfusepath{stroke,fill}%
\end{pgfscope}%
\begin{pgfscope}%
\pgfpathrectangle{\pgfqpoint{0.375000in}{0.330000in}}{\pgfqpoint{2.325000in}{2.310000in}}%
\pgfusepath{clip}%
\pgfsetbuttcap%
\pgfsetroundjoin%
\definecolor{currentfill}{rgb}{0.000000,0.000000,0.000000}%
\pgfsetfillcolor{currentfill}%
\pgfsetlinewidth{1.003750pt}%
\definecolor{currentstroke}{rgb}{0.000000,0.000000,0.000000}%
\pgfsetstrokecolor{currentstroke}%
\pgfsetdash{}{0pt}%
\pgfpathmoveto{\pgfqpoint{0.459750in}{0.412000in}}%
\pgfpathcurveto{\pgfqpoint{0.470800in}{0.412000in}}{\pgfqpoint{0.481399in}{0.416390in}}{\pgfqpoint{0.489213in}{0.424204in}}%
\pgfpathcurveto{\pgfqpoint{0.497026in}{0.432017in}}{\pgfqpoint{0.501417in}{0.442616in}}{\pgfqpoint{0.501417in}{0.453666in}}%
\pgfpathcurveto{\pgfqpoint{0.501417in}{0.464716in}}{\pgfqpoint{0.497026in}{0.475315in}}{\pgfqpoint{0.489213in}{0.483129in}}%
\pgfpathcurveto{\pgfqpoint{0.481399in}{0.490943in}}{\pgfqpoint{0.470800in}{0.495333in}}{\pgfqpoint{0.459750in}{0.495333in}}%
\pgfpathcurveto{\pgfqpoint{0.448700in}{0.495333in}}{\pgfqpoint{0.438101in}{0.490943in}}{\pgfqpoint{0.430287in}{0.483129in}}%
\pgfpathcurveto{\pgfqpoint{0.422474in}{0.475315in}}{\pgfqpoint{0.418083in}{0.464716in}}{\pgfqpoint{0.418083in}{0.453666in}}%
\pgfpathcurveto{\pgfqpoint{0.418083in}{0.442616in}}{\pgfqpoint{0.422474in}{0.432017in}}{\pgfqpoint{0.430287in}{0.424204in}}%
\pgfpathcurveto{\pgfqpoint{0.438101in}{0.416390in}}{\pgfqpoint{0.448700in}{0.412000in}}{\pgfqpoint{0.459750in}{0.412000in}}%
\pgfpathclose%
\pgfusepath{stroke,fill}%
\end{pgfscope}%
\begin{pgfscope}%
\pgfpathrectangle{\pgfqpoint{0.375000in}{0.330000in}}{\pgfqpoint{2.325000in}{2.310000in}}%
\pgfusepath{clip}%
\pgfsetbuttcap%
\pgfsetroundjoin%
\definecolor{currentfill}{rgb}{0.000000,0.000000,0.000000}%
\pgfsetfillcolor{currentfill}%
\pgfsetlinewidth{1.003750pt}%
\definecolor{currentstroke}{rgb}{0.000000,0.000000,0.000000}%
\pgfsetstrokecolor{currentstroke}%
\pgfsetdash{}{0pt}%
\pgfpathmoveto{\pgfqpoint{0.459750in}{0.412000in}}%
\pgfpathcurveto{\pgfqpoint{0.470800in}{0.412000in}}{\pgfqpoint{0.481399in}{0.416390in}}{\pgfqpoint{0.489213in}{0.424204in}}%
\pgfpathcurveto{\pgfqpoint{0.497026in}{0.432017in}}{\pgfqpoint{0.501417in}{0.442616in}}{\pgfqpoint{0.501417in}{0.453666in}}%
\pgfpathcurveto{\pgfqpoint{0.501417in}{0.464716in}}{\pgfqpoint{0.497026in}{0.475315in}}{\pgfqpoint{0.489213in}{0.483129in}}%
\pgfpathcurveto{\pgfqpoint{0.481399in}{0.490943in}}{\pgfqpoint{0.470800in}{0.495333in}}{\pgfqpoint{0.459750in}{0.495333in}}%
\pgfpathcurveto{\pgfqpoint{0.448700in}{0.495333in}}{\pgfqpoint{0.438101in}{0.490943in}}{\pgfqpoint{0.430287in}{0.483129in}}%
\pgfpathcurveto{\pgfqpoint{0.422474in}{0.475315in}}{\pgfqpoint{0.418083in}{0.464716in}}{\pgfqpoint{0.418083in}{0.453666in}}%
\pgfpathcurveto{\pgfqpoint{0.418083in}{0.442616in}}{\pgfqpoint{0.422474in}{0.432017in}}{\pgfqpoint{0.430287in}{0.424204in}}%
\pgfpathcurveto{\pgfqpoint{0.438101in}{0.416390in}}{\pgfqpoint{0.448700in}{0.412000in}}{\pgfqpoint{0.459750in}{0.412000in}}%
\pgfpathclose%
\pgfusepath{stroke,fill}%
\end{pgfscope}%
\begin{pgfscope}%
\pgfpathrectangle{\pgfqpoint{0.375000in}{0.330000in}}{\pgfqpoint{2.325000in}{2.310000in}}%
\pgfusepath{clip}%
\pgfsetbuttcap%
\pgfsetroundjoin%
\definecolor{currentfill}{rgb}{0.000000,0.000000,0.000000}%
\pgfsetfillcolor{currentfill}%
\pgfsetlinewidth{1.003750pt}%
\definecolor{currentstroke}{rgb}{0.000000,0.000000,0.000000}%
\pgfsetstrokecolor{currentstroke}%
\pgfsetdash{}{0pt}%
\pgfpathmoveto{\pgfqpoint{0.459750in}{0.412000in}}%
\pgfpathcurveto{\pgfqpoint{0.470800in}{0.412000in}}{\pgfqpoint{0.481399in}{0.416390in}}{\pgfqpoint{0.489213in}{0.424204in}}%
\pgfpathcurveto{\pgfqpoint{0.497026in}{0.432017in}}{\pgfqpoint{0.501417in}{0.442616in}}{\pgfqpoint{0.501417in}{0.453666in}}%
\pgfpathcurveto{\pgfqpoint{0.501417in}{0.464716in}}{\pgfqpoint{0.497026in}{0.475315in}}{\pgfqpoint{0.489213in}{0.483129in}}%
\pgfpathcurveto{\pgfqpoint{0.481399in}{0.490943in}}{\pgfqpoint{0.470800in}{0.495333in}}{\pgfqpoint{0.459750in}{0.495333in}}%
\pgfpathcurveto{\pgfqpoint{0.448700in}{0.495333in}}{\pgfqpoint{0.438101in}{0.490943in}}{\pgfqpoint{0.430287in}{0.483129in}}%
\pgfpathcurveto{\pgfqpoint{0.422474in}{0.475315in}}{\pgfqpoint{0.418083in}{0.464716in}}{\pgfqpoint{0.418083in}{0.453666in}}%
\pgfpathcurveto{\pgfqpoint{0.418083in}{0.442616in}}{\pgfqpoint{0.422474in}{0.432017in}}{\pgfqpoint{0.430287in}{0.424204in}}%
\pgfpathcurveto{\pgfqpoint{0.438101in}{0.416390in}}{\pgfqpoint{0.448700in}{0.412000in}}{\pgfqpoint{0.459750in}{0.412000in}}%
\pgfpathclose%
\pgfusepath{stroke,fill}%
\end{pgfscope}%
\begin{pgfscope}%
\pgfpathrectangle{\pgfqpoint{0.375000in}{0.330000in}}{\pgfqpoint{2.325000in}{2.310000in}}%
\pgfusepath{clip}%
\pgfsetbuttcap%
\pgfsetroundjoin%
\definecolor{currentfill}{rgb}{0.000000,0.000000,0.000000}%
\pgfsetfillcolor{currentfill}%
\pgfsetlinewidth{1.003750pt}%
\definecolor{currentstroke}{rgb}{0.000000,0.000000,0.000000}%
\pgfsetstrokecolor{currentstroke}%
\pgfsetdash{}{0pt}%
\pgfpathmoveto{\pgfqpoint{0.459750in}{0.412000in}}%
\pgfpathcurveto{\pgfqpoint{0.470800in}{0.412000in}}{\pgfqpoint{0.481399in}{0.416390in}}{\pgfqpoint{0.489213in}{0.424204in}}%
\pgfpathcurveto{\pgfqpoint{0.497026in}{0.432017in}}{\pgfqpoint{0.501417in}{0.442616in}}{\pgfqpoint{0.501417in}{0.453666in}}%
\pgfpathcurveto{\pgfqpoint{0.501417in}{0.464716in}}{\pgfqpoint{0.497026in}{0.475315in}}{\pgfqpoint{0.489213in}{0.483129in}}%
\pgfpathcurveto{\pgfqpoint{0.481399in}{0.490943in}}{\pgfqpoint{0.470800in}{0.495333in}}{\pgfqpoint{0.459750in}{0.495333in}}%
\pgfpathcurveto{\pgfqpoint{0.448700in}{0.495333in}}{\pgfqpoint{0.438101in}{0.490943in}}{\pgfqpoint{0.430287in}{0.483129in}}%
\pgfpathcurveto{\pgfqpoint{0.422474in}{0.475315in}}{\pgfqpoint{0.418083in}{0.464716in}}{\pgfqpoint{0.418083in}{0.453666in}}%
\pgfpathcurveto{\pgfqpoint{0.418083in}{0.442616in}}{\pgfqpoint{0.422474in}{0.432017in}}{\pgfqpoint{0.430287in}{0.424204in}}%
\pgfpathcurveto{\pgfqpoint{0.438101in}{0.416390in}}{\pgfqpoint{0.448700in}{0.412000in}}{\pgfqpoint{0.459750in}{0.412000in}}%
\pgfpathclose%
\pgfusepath{stroke,fill}%
\end{pgfscope}%
\begin{pgfscope}%
\pgfpathrectangle{\pgfqpoint{0.375000in}{0.330000in}}{\pgfqpoint{2.325000in}{2.310000in}}%
\pgfusepath{clip}%
\pgfsetbuttcap%
\pgfsetroundjoin%
\definecolor{currentfill}{rgb}{0.000000,0.000000,0.000000}%
\pgfsetfillcolor{currentfill}%
\pgfsetlinewidth{1.003750pt}%
\definecolor{currentstroke}{rgb}{0.000000,0.000000,0.000000}%
\pgfsetstrokecolor{currentstroke}%
\pgfsetdash{}{0pt}%
\pgfpathmoveto{\pgfqpoint{0.459750in}{0.412000in}}%
\pgfpathcurveto{\pgfqpoint{0.470800in}{0.412000in}}{\pgfqpoint{0.481399in}{0.416390in}}{\pgfqpoint{0.489213in}{0.424204in}}%
\pgfpathcurveto{\pgfqpoint{0.497026in}{0.432017in}}{\pgfqpoint{0.501417in}{0.442616in}}{\pgfqpoint{0.501417in}{0.453666in}}%
\pgfpathcurveto{\pgfqpoint{0.501417in}{0.464716in}}{\pgfqpoint{0.497026in}{0.475315in}}{\pgfqpoint{0.489213in}{0.483129in}}%
\pgfpathcurveto{\pgfqpoint{0.481399in}{0.490943in}}{\pgfqpoint{0.470800in}{0.495333in}}{\pgfqpoint{0.459750in}{0.495333in}}%
\pgfpathcurveto{\pgfqpoint{0.448700in}{0.495333in}}{\pgfqpoint{0.438101in}{0.490943in}}{\pgfqpoint{0.430287in}{0.483129in}}%
\pgfpathcurveto{\pgfqpoint{0.422474in}{0.475315in}}{\pgfqpoint{0.418083in}{0.464716in}}{\pgfqpoint{0.418083in}{0.453666in}}%
\pgfpathcurveto{\pgfqpoint{0.418083in}{0.442616in}}{\pgfqpoint{0.422474in}{0.432017in}}{\pgfqpoint{0.430287in}{0.424204in}}%
\pgfpathcurveto{\pgfqpoint{0.438101in}{0.416390in}}{\pgfqpoint{0.448700in}{0.412000in}}{\pgfqpoint{0.459750in}{0.412000in}}%
\pgfpathclose%
\pgfusepath{stroke,fill}%
\end{pgfscope}%
\begin{pgfscope}%
\pgfpathrectangle{\pgfqpoint{0.375000in}{0.330000in}}{\pgfqpoint{2.325000in}{2.310000in}}%
\pgfusepath{clip}%
\pgfsetbuttcap%
\pgfsetroundjoin%
\definecolor{currentfill}{rgb}{0.000000,0.000000,0.000000}%
\pgfsetfillcolor{currentfill}%
\pgfsetlinewidth{1.003750pt}%
\definecolor{currentstroke}{rgb}{0.000000,0.000000,0.000000}%
\pgfsetstrokecolor{currentstroke}%
\pgfsetdash{}{0pt}%
\pgfpathmoveto{\pgfqpoint{0.459750in}{0.412000in}}%
\pgfpathcurveto{\pgfqpoint{0.470800in}{0.412000in}}{\pgfqpoint{0.481399in}{0.416390in}}{\pgfqpoint{0.489213in}{0.424204in}}%
\pgfpathcurveto{\pgfqpoint{0.497026in}{0.432017in}}{\pgfqpoint{0.501417in}{0.442616in}}{\pgfqpoint{0.501417in}{0.453666in}}%
\pgfpathcurveto{\pgfqpoint{0.501417in}{0.464716in}}{\pgfqpoint{0.497026in}{0.475315in}}{\pgfqpoint{0.489213in}{0.483129in}}%
\pgfpathcurveto{\pgfqpoint{0.481399in}{0.490943in}}{\pgfqpoint{0.470800in}{0.495333in}}{\pgfqpoint{0.459750in}{0.495333in}}%
\pgfpathcurveto{\pgfqpoint{0.448700in}{0.495333in}}{\pgfqpoint{0.438101in}{0.490943in}}{\pgfqpoint{0.430287in}{0.483129in}}%
\pgfpathcurveto{\pgfqpoint{0.422474in}{0.475315in}}{\pgfqpoint{0.418083in}{0.464716in}}{\pgfqpoint{0.418083in}{0.453666in}}%
\pgfpathcurveto{\pgfqpoint{0.418083in}{0.442616in}}{\pgfqpoint{0.422474in}{0.432017in}}{\pgfqpoint{0.430287in}{0.424204in}}%
\pgfpathcurveto{\pgfqpoint{0.438101in}{0.416390in}}{\pgfqpoint{0.448700in}{0.412000in}}{\pgfqpoint{0.459750in}{0.412000in}}%
\pgfpathclose%
\pgfusepath{stroke,fill}%
\end{pgfscope}%
\begin{pgfscope}%
\pgfpathrectangle{\pgfqpoint{0.375000in}{0.330000in}}{\pgfqpoint{2.325000in}{2.310000in}}%
\pgfusepath{clip}%
\pgfsetbuttcap%
\pgfsetroundjoin%
\definecolor{currentfill}{rgb}{0.000000,0.000000,0.000000}%
\pgfsetfillcolor{currentfill}%
\pgfsetlinewidth{1.003750pt}%
\definecolor{currentstroke}{rgb}{0.000000,0.000000,0.000000}%
\pgfsetstrokecolor{currentstroke}%
\pgfsetdash{}{0pt}%
\pgfpathmoveto{\pgfqpoint{0.459750in}{1.443291in}}%
\pgfpathcurveto{\pgfqpoint{0.470800in}{1.443291in}}{\pgfqpoint{0.481399in}{1.447682in}}{\pgfqpoint{0.489213in}{1.455495in}}%
\pgfpathcurveto{\pgfqpoint{0.497026in}{1.463309in}}{\pgfqpoint{0.501417in}{1.473908in}}{\pgfqpoint{0.501417in}{1.484958in}}%
\pgfpathcurveto{\pgfqpoint{0.501417in}{1.496008in}}{\pgfqpoint{0.497026in}{1.506607in}}{\pgfqpoint{0.489213in}{1.514421in}}%
\pgfpathcurveto{\pgfqpoint{0.481399in}{1.522235in}}{\pgfqpoint{0.470800in}{1.526625in}}{\pgfqpoint{0.459750in}{1.526625in}}%
\pgfpathcurveto{\pgfqpoint{0.448700in}{1.526625in}}{\pgfqpoint{0.438101in}{1.522235in}}{\pgfqpoint{0.430287in}{1.514421in}}%
\pgfpathcurveto{\pgfqpoint{0.422474in}{1.506607in}}{\pgfqpoint{0.418083in}{1.496008in}}{\pgfqpoint{0.418083in}{1.484958in}}%
\pgfpathcurveto{\pgfqpoint{0.418083in}{1.473908in}}{\pgfqpoint{0.422474in}{1.463309in}}{\pgfqpoint{0.430287in}{1.455495in}}%
\pgfpathcurveto{\pgfqpoint{0.438101in}{1.447682in}}{\pgfqpoint{0.448700in}{1.443291in}}{\pgfqpoint{0.459750in}{1.443291in}}%
\pgfpathclose%
\pgfusepath{stroke,fill}%
\end{pgfscope}%
\begin{pgfscope}%
\pgfpathrectangle{\pgfqpoint{0.375000in}{0.330000in}}{\pgfqpoint{2.325000in}{2.310000in}}%
\pgfusepath{clip}%
\pgfsetbuttcap%
\pgfsetroundjoin%
\definecolor{currentfill}{rgb}{0.000000,0.000000,0.000000}%
\pgfsetfillcolor{currentfill}%
\pgfsetlinewidth{1.003750pt}%
\definecolor{currentstroke}{rgb}{0.000000,0.000000,0.000000}%
\pgfsetstrokecolor{currentstroke}%
\pgfsetdash{}{0pt}%
\pgfpathmoveto{\pgfqpoint{0.459750in}{1.443291in}}%
\pgfpathcurveto{\pgfqpoint{0.470800in}{1.443291in}}{\pgfqpoint{0.481399in}{1.447682in}}{\pgfqpoint{0.489213in}{1.455495in}}%
\pgfpathcurveto{\pgfqpoint{0.497026in}{1.463309in}}{\pgfqpoint{0.501417in}{1.473908in}}{\pgfqpoint{0.501417in}{1.484958in}}%
\pgfpathcurveto{\pgfqpoint{0.501417in}{1.496008in}}{\pgfqpoint{0.497026in}{1.506607in}}{\pgfqpoint{0.489213in}{1.514421in}}%
\pgfpathcurveto{\pgfqpoint{0.481399in}{1.522235in}}{\pgfqpoint{0.470800in}{1.526625in}}{\pgfqpoint{0.459750in}{1.526625in}}%
\pgfpathcurveto{\pgfqpoint{0.448700in}{1.526625in}}{\pgfqpoint{0.438101in}{1.522235in}}{\pgfqpoint{0.430287in}{1.514421in}}%
\pgfpathcurveto{\pgfqpoint{0.422474in}{1.506607in}}{\pgfqpoint{0.418083in}{1.496008in}}{\pgfqpoint{0.418083in}{1.484958in}}%
\pgfpathcurveto{\pgfqpoint{0.418083in}{1.473908in}}{\pgfqpoint{0.422474in}{1.463309in}}{\pgfqpoint{0.430287in}{1.455495in}}%
\pgfpathcurveto{\pgfqpoint{0.438101in}{1.447682in}}{\pgfqpoint{0.448700in}{1.443291in}}{\pgfqpoint{0.459750in}{1.443291in}}%
\pgfpathclose%
\pgfusepath{stroke,fill}%
\end{pgfscope}%
\begin{pgfscope}%
\pgfpathrectangle{\pgfqpoint{0.375000in}{0.330000in}}{\pgfqpoint{2.325000in}{2.310000in}}%
\pgfusepath{clip}%
\pgfsetbuttcap%
\pgfsetroundjoin%
\definecolor{currentfill}{rgb}{0.000000,0.000000,0.000000}%
\pgfsetfillcolor{currentfill}%
\pgfsetlinewidth{1.003750pt}%
\definecolor{currentstroke}{rgb}{0.000000,0.000000,0.000000}%
\pgfsetstrokecolor{currentstroke}%
\pgfsetdash{}{0pt}%
\pgfpathmoveto{\pgfqpoint{0.459750in}{0.412000in}}%
\pgfpathcurveto{\pgfqpoint{0.470800in}{0.412000in}}{\pgfqpoint{0.481399in}{0.416390in}}{\pgfqpoint{0.489213in}{0.424204in}}%
\pgfpathcurveto{\pgfqpoint{0.497026in}{0.432017in}}{\pgfqpoint{0.501417in}{0.442616in}}{\pgfqpoint{0.501417in}{0.453666in}}%
\pgfpathcurveto{\pgfqpoint{0.501417in}{0.464716in}}{\pgfqpoint{0.497026in}{0.475315in}}{\pgfqpoint{0.489213in}{0.483129in}}%
\pgfpathcurveto{\pgfqpoint{0.481399in}{0.490943in}}{\pgfqpoint{0.470800in}{0.495333in}}{\pgfqpoint{0.459750in}{0.495333in}}%
\pgfpathcurveto{\pgfqpoint{0.448700in}{0.495333in}}{\pgfqpoint{0.438101in}{0.490943in}}{\pgfqpoint{0.430287in}{0.483129in}}%
\pgfpathcurveto{\pgfqpoint{0.422474in}{0.475315in}}{\pgfqpoint{0.418083in}{0.464716in}}{\pgfqpoint{0.418083in}{0.453666in}}%
\pgfpathcurveto{\pgfqpoint{0.418083in}{0.442616in}}{\pgfqpoint{0.422474in}{0.432017in}}{\pgfqpoint{0.430287in}{0.424204in}}%
\pgfpathcurveto{\pgfqpoint{0.438101in}{0.416390in}}{\pgfqpoint{0.448700in}{0.412000in}}{\pgfqpoint{0.459750in}{0.412000in}}%
\pgfpathclose%
\pgfusepath{stroke,fill}%
\end{pgfscope}%
\begin{pgfscope}%
\pgfpathrectangle{\pgfqpoint{0.375000in}{0.330000in}}{\pgfqpoint{2.325000in}{2.310000in}}%
\pgfusepath{clip}%
\pgfsetbuttcap%
\pgfsetroundjoin%
\definecolor{currentfill}{rgb}{0.000000,0.000000,0.000000}%
\pgfsetfillcolor{currentfill}%
\pgfsetlinewidth{1.003750pt}%
\definecolor{currentstroke}{rgb}{0.000000,0.000000,0.000000}%
\pgfsetstrokecolor{currentstroke}%
\pgfsetdash{}{0pt}%
\pgfpathmoveto{\pgfqpoint{0.459750in}{0.412000in}}%
\pgfpathcurveto{\pgfqpoint{0.470800in}{0.412000in}}{\pgfqpoint{0.481399in}{0.416390in}}{\pgfqpoint{0.489213in}{0.424204in}}%
\pgfpathcurveto{\pgfqpoint{0.497026in}{0.432017in}}{\pgfqpoint{0.501417in}{0.442616in}}{\pgfqpoint{0.501417in}{0.453666in}}%
\pgfpathcurveto{\pgfqpoint{0.501417in}{0.464716in}}{\pgfqpoint{0.497026in}{0.475315in}}{\pgfqpoint{0.489213in}{0.483129in}}%
\pgfpathcurveto{\pgfqpoint{0.481399in}{0.490943in}}{\pgfqpoint{0.470800in}{0.495333in}}{\pgfqpoint{0.459750in}{0.495333in}}%
\pgfpathcurveto{\pgfqpoint{0.448700in}{0.495333in}}{\pgfqpoint{0.438101in}{0.490943in}}{\pgfqpoint{0.430287in}{0.483129in}}%
\pgfpathcurveto{\pgfqpoint{0.422474in}{0.475315in}}{\pgfqpoint{0.418083in}{0.464716in}}{\pgfqpoint{0.418083in}{0.453666in}}%
\pgfpathcurveto{\pgfqpoint{0.418083in}{0.442616in}}{\pgfqpoint{0.422474in}{0.432017in}}{\pgfqpoint{0.430287in}{0.424204in}}%
\pgfpathcurveto{\pgfqpoint{0.438101in}{0.416390in}}{\pgfqpoint{0.448700in}{0.412000in}}{\pgfqpoint{0.459750in}{0.412000in}}%
\pgfpathclose%
\pgfusepath{stroke,fill}%
\end{pgfscope}%
\begin{pgfscope}%
\pgfpathrectangle{\pgfqpoint{0.375000in}{0.330000in}}{\pgfqpoint{2.325000in}{2.310000in}}%
\pgfusepath{clip}%
\pgfsetbuttcap%
\pgfsetroundjoin%
\definecolor{currentfill}{rgb}{0.000000,0.000000,0.000000}%
\pgfsetfillcolor{currentfill}%
\pgfsetlinewidth{1.003750pt}%
\definecolor{currentstroke}{rgb}{0.000000,0.000000,0.000000}%
\pgfsetstrokecolor{currentstroke}%
\pgfsetdash{}{0pt}%
\pgfpathmoveto{\pgfqpoint{0.459750in}{0.412000in}}%
\pgfpathcurveto{\pgfqpoint{0.470800in}{0.412000in}}{\pgfqpoint{0.481399in}{0.416390in}}{\pgfqpoint{0.489213in}{0.424204in}}%
\pgfpathcurveto{\pgfqpoint{0.497026in}{0.432017in}}{\pgfqpoint{0.501417in}{0.442616in}}{\pgfqpoint{0.501417in}{0.453666in}}%
\pgfpathcurveto{\pgfqpoint{0.501417in}{0.464716in}}{\pgfqpoint{0.497026in}{0.475315in}}{\pgfqpoint{0.489213in}{0.483129in}}%
\pgfpathcurveto{\pgfqpoint{0.481399in}{0.490943in}}{\pgfqpoint{0.470800in}{0.495333in}}{\pgfqpoint{0.459750in}{0.495333in}}%
\pgfpathcurveto{\pgfqpoint{0.448700in}{0.495333in}}{\pgfqpoint{0.438101in}{0.490943in}}{\pgfqpoint{0.430287in}{0.483129in}}%
\pgfpathcurveto{\pgfqpoint{0.422474in}{0.475315in}}{\pgfqpoint{0.418083in}{0.464716in}}{\pgfqpoint{0.418083in}{0.453666in}}%
\pgfpathcurveto{\pgfqpoint{0.418083in}{0.442616in}}{\pgfqpoint{0.422474in}{0.432017in}}{\pgfqpoint{0.430287in}{0.424204in}}%
\pgfpathcurveto{\pgfqpoint{0.438101in}{0.416390in}}{\pgfqpoint{0.448700in}{0.412000in}}{\pgfqpoint{0.459750in}{0.412000in}}%
\pgfpathclose%
\pgfusepath{stroke,fill}%
\end{pgfscope}%
\begin{pgfscope}%
\pgfpathrectangle{\pgfqpoint{0.375000in}{0.330000in}}{\pgfqpoint{2.325000in}{2.310000in}}%
\pgfusepath{clip}%
\pgfsetbuttcap%
\pgfsetroundjoin%
\definecolor{currentfill}{rgb}{0.000000,0.000000,0.000000}%
\pgfsetfillcolor{currentfill}%
\pgfsetlinewidth{1.003750pt}%
\definecolor{currentstroke}{rgb}{0.000000,0.000000,0.000000}%
\pgfsetstrokecolor{currentstroke}%
\pgfsetdash{}{0pt}%
\pgfpathmoveto{\pgfqpoint{0.459750in}{0.412000in}}%
\pgfpathcurveto{\pgfqpoint{0.470800in}{0.412000in}}{\pgfqpoint{0.481399in}{0.416390in}}{\pgfqpoint{0.489213in}{0.424204in}}%
\pgfpathcurveto{\pgfqpoint{0.497026in}{0.432017in}}{\pgfqpoint{0.501417in}{0.442616in}}{\pgfqpoint{0.501417in}{0.453666in}}%
\pgfpathcurveto{\pgfqpoint{0.501417in}{0.464716in}}{\pgfqpoint{0.497026in}{0.475315in}}{\pgfqpoint{0.489213in}{0.483129in}}%
\pgfpathcurveto{\pgfqpoint{0.481399in}{0.490943in}}{\pgfqpoint{0.470800in}{0.495333in}}{\pgfqpoint{0.459750in}{0.495333in}}%
\pgfpathcurveto{\pgfqpoint{0.448700in}{0.495333in}}{\pgfqpoint{0.438101in}{0.490943in}}{\pgfqpoint{0.430287in}{0.483129in}}%
\pgfpathcurveto{\pgfqpoint{0.422474in}{0.475315in}}{\pgfqpoint{0.418083in}{0.464716in}}{\pgfqpoint{0.418083in}{0.453666in}}%
\pgfpathcurveto{\pgfqpoint{0.418083in}{0.442616in}}{\pgfqpoint{0.422474in}{0.432017in}}{\pgfqpoint{0.430287in}{0.424204in}}%
\pgfpathcurveto{\pgfqpoint{0.438101in}{0.416390in}}{\pgfqpoint{0.448700in}{0.412000in}}{\pgfqpoint{0.459750in}{0.412000in}}%
\pgfpathclose%
\pgfusepath{stroke,fill}%
\end{pgfscope}%
\begin{pgfscope}%
\pgfpathrectangle{\pgfqpoint{0.375000in}{0.330000in}}{\pgfqpoint{2.325000in}{2.310000in}}%
\pgfusepath{clip}%
\pgfsetbuttcap%
\pgfsetroundjoin%
\definecolor{currentfill}{rgb}{0.000000,0.000000,0.000000}%
\pgfsetfillcolor{currentfill}%
\pgfsetlinewidth{1.003750pt}%
\definecolor{currentstroke}{rgb}{0.000000,0.000000,0.000000}%
\pgfsetstrokecolor{currentstroke}%
\pgfsetdash{}{0pt}%
\pgfpathmoveto{\pgfqpoint{0.459750in}{1.443291in}}%
\pgfpathcurveto{\pgfqpoint{0.470800in}{1.443291in}}{\pgfqpoint{0.481399in}{1.447682in}}{\pgfqpoint{0.489213in}{1.455495in}}%
\pgfpathcurveto{\pgfqpoint{0.497026in}{1.463309in}}{\pgfqpoint{0.501417in}{1.473908in}}{\pgfqpoint{0.501417in}{1.484958in}}%
\pgfpathcurveto{\pgfqpoint{0.501417in}{1.496008in}}{\pgfqpoint{0.497026in}{1.506607in}}{\pgfqpoint{0.489213in}{1.514421in}}%
\pgfpathcurveto{\pgfqpoint{0.481399in}{1.522235in}}{\pgfqpoint{0.470800in}{1.526625in}}{\pgfqpoint{0.459750in}{1.526625in}}%
\pgfpathcurveto{\pgfqpoint{0.448700in}{1.526625in}}{\pgfqpoint{0.438101in}{1.522235in}}{\pgfqpoint{0.430287in}{1.514421in}}%
\pgfpathcurveto{\pgfqpoint{0.422474in}{1.506607in}}{\pgfqpoint{0.418083in}{1.496008in}}{\pgfqpoint{0.418083in}{1.484958in}}%
\pgfpathcurveto{\pgfqpoint{0.418083in}{1.473908in}}{\pgfqpoint{0.422474in}{1.463309in}}{\pgfqpoint{0.430287in}{1.455495in}}%
\pgfpathcurveto{\pgfqpoint{0.438101in}{1.447682in}}{\pgfqpoint{0.448700in}{1.443291in}}{\pgfqpoint{0.459750in}{1.443291in}}%
\pgfpathclose%
\pgfusepath{stroke,fill}%
\end{pgfscope}%
\begin{pgfscope}%
\pgfpathrectangle{\pgfqpoint{0.375000in}{0.330000in}}{\pgfqpoint{2.325000in}{2.310000in}}%
\pgfusepath{clip}%
\pgfsetbuttcap%
\pgfsetroundjoin%
\definecolor{currentfill}{rgb}{0.000000,0.000000,0.000000}%
\pgfsetfillcolor{currentfill}%
\pgfsetlinewidth{1.003750pt}%
\definecolor{currentstroke}{rgb}{0.000000,0.000000,0.000000}%
\pgfsetstrokecolor{currentstroke}%
\pgfsetdash{}{0pt}%
\pgfpathmoveto{\pgfqpoint{0.459750in}{0.412000in}}%
\pgfpathcurveto{\pgfqpoint{0.470800in}{0.412000in}}{\pgfqpoint{0.481399in}{0.416390in}}{\pgfqpoint{0.489213in}{0.424204in}}%
\pgfpathcurveto{\pgfqpoint{0.497026in}{0.432017in}}{\pgfqpoint{0.501417in}{0.442616in}}{\pgfqpoint{0.501417in}{0.453666in}}%
\pgfpathcurveto{\pgfqpoint{0.501417in}{0.464716in}}{\pgfqpoint{0.497026in}{0.475315in}}{\pgfqpoint{0.489213in}{0.483129in}}%
\pgfpathcurveto{\pgfqpoint{0.481399in}{0.490943in}}{\pgfqpoint{0.470800in}{0.495333in}}{\pgfqpoint{0.459750in}{0.495333in}}%
\pgfpathcurveto{\pgfqpoint{0.448700in}{0.495333in}}{\pgfqpoint{0.438101in}{0.490943in}}{\pgfqpoint{0.430287in}{0.483129in}}%
\pgfpathcurveto{\pgfqpoint{0.422474in}{0.475315in}}{\pgfqpoint{0.418083in}{0.464716in}}{\pgfqpoint{0.418083in}{0.453666in}}%
\pgfpathcurveto{\pgfqpoint{0.418083in}{0.442616in}}{\pgfqpoint{0.422474in}{0.432017in}}{\pgfqpoint{0.430287in}{0.424204in}}%
\pgfpathcurveto{\pgfqpoint{0.438101in}{0.416390in}}{\pgfqpoint{0.448700in}{0.412000in}}{\pgfqpoint{0.459750in}{0.412000in}}%
\pgfpathclose%
\pgfusepath{stroke,fill}%
\end{pgfscope}%
\begin{pgfscope}%
\pgfpathrectangle{\pgfqpoint{0.375000in}{0.330000in}}{\pgfqpoint{2.325000in}{2.310000in}}%
\pgfusepath{clip}%
\pgfsetbuttcap%
\pgfsetroundjoin%
\definecolor{currentfill}{rgb}{0.000000,0.000000,0.000000}%
\pgfsetfillcolor{currentfill}%
\pgfsetlinewidth{1.003750pt}%
\definecolor{currentstroke}{rgb}{0.000000,0.000000,0.000000}%
\pgfsetstrokecolor{currentstroke}%
\pgfsetdash{}{0pt}%
\pgfpathmoveto{\pgfqpoint{0.459750in}{0.412000in}}%
\pgfpathcurveto{\pgfqpoint{0.470800in}{0.412000in}}{\pgfqpoint{0.481399in}{0.416390in}}{\pgfqpoint{0.489213in}{0.424204in}}%
\pgfpathcurveto{\pgfqpoint{0.497026in}{0.432017in}}{\pgfqpoint{0.501417in}{0.442616in}}{\pgfqpoint{0.501417in}{0.453666in}}%
\pgfpathcurveto{\pgfqpoint{0.501417in}{0.464716in}}{\pgfqpoint{0.497026in}{0.475315in}}{\pgfqpoint{0.489213in}{0.483129in}}%
\pgfpathcurveto{\pgfqpoint{0.481399in}{0.490943in}}{\pgfqpoint{0.470800in}{0.495333in}}{\pgfqpoint{0.459750in}{0.495333in}}%
\pgfpathcurveto{\pgfqpoint{0.448700in}{0.495333in}}{\pgfqpoint{0.438101in}{0.490943in}}{\pgfqpoint{0.430287in}{0.483129in}}%
\pgfpathcurveto{\pgfqpoint{0.422474in}{0.475315in}}{\pgfqpoint{0.418083in}{0.464716in}}{\pgfqpoint{0.418083in}{0.453666in}}%
\pgfpathcurveto{\pgfqpoint{0.418083in}{0.442616in}}{\pgfqpoint{0.422474in}{0.432017in}}{\pgfqpoint{0.430287in}{0.424204in}}%
\pgfpathcurveto{\pgfqpoint{0.438101in}{0.416390in}}{\pgfqpoint{0.448700in}{0.412000in}}{\pgfqpoint{0.459750in}{0.412000in}}%
\pgfpathclose%
\pgfusepath{stroke,fill}%
\end{pgfscope}%
\begin{pgfscope}%
\pgfpathrectangle{\pgfqpoint{0.375000in}{0.330000in}}{\pgfqpoint{2.325000in}{2.310000in}}%
\pgfusepath{clip}%
\pgfsetbuttcap%
\pgfsetroundjoin%
\definecolor{currentfill}{rgb}{0.000000,0.000000,0.000000}%
\pgfsetfillcolor{currentfill}%
\pgfsetlinewidth{1.003750pt}%
\definecolor{currentstroke}{rgb}{0.000000,0.000000,0.000000}%
\pgfsetstrokecolor{currentstroke}%
\pgfsetdash{}{0pt}%
\pgfpathmoveto{\pgfqpoint{0.459750in}{0.412000in}}%
\pgfpathcurveto{\pgfqpoint{0.470800in}{0.412000in}}{\pgfqpoint{0.481399in}{0.416390in}}{\pgfqpoint{0.489213in}{0.424204in}}%
\pgfpathcurveto{\pgfqpoint{0.497026in}{0.432017in}}{\pgfqpoint{0.501417in}{0.442616in}}{\pgfqpoint{0.501417in}{0.453666in}}%
\pgfpathcurveto{\pgfqpoint{0.501417in}{0.464716in}}{\pgfqpoint{0.497026in}{0.475315in}}{\pgfqpoint{0.489213in}{0.483129in}}%
\pgfpathcurveto{\pgfqpoint{0.481399in}{0.490943in}}{\pgfqpoint{0.470800in}{0.495333in}}{\pgfqpoint{0.459750in}{0.495333in}}%
\pgfpathcurveto{\pgfqpoint{0.448700in}{0.495333in}}{\pgfqpoint{0.438101in}{0.490943in}}{\pgfqpoint{0.430287in}{0.483129in}}%
\pgfpathcurveto{\pgfqpoint{0.422474in}{0.475315in}}{\pgfqpoint{0.418083in}{0.464716in}}{\pgfqpoint{0.418083in}{0.453666in}}%
\pgfpathcurveto{\pgfqpoint{0.418083in}{0.442616in}}{\pgfqpoint{0.422474in}{0.432017in}}{\pgfqpoint{0.430287in}{0.424204in}}%
\pgfpathcurveto{\pgfqpoint{0.438101in}{0.416390in}}{\pgfqpoint{0.448700in}{0.412000in}}{\pgfqpoint{0.459750in}{0.412000in}}%
\pgfpathclose%
\pgfusepath{stroke,fill}%
\end{pgfscope}%
\begin{pgfscope}%
\pgfpathrectangle{\pgfqpoint{0.375000in}{0.330000in}}{\pgfqpoint{2.325000in}{2.310000in}}%
\pgfusepath{clip}%
\pgfsetbuttcap%
\pgfsetroundjoin%
\definecolor{currentfill}{rgb}{0.000000,0.000000,0.000000}%
\pgfsetfillcolor{currentfill}%
\pgfsetlinewidth{1.003750pt}%
\definecolor{currentstroke}{rgb}{0.000000,0.000000,0.000000}%
\pgfsetstrokecolor{currentstroke}%
\pgfsetdash{}{0pt}%
\pgfpathmoveto{\pgfqpoint{0.459750in}{0.412000in}}%
\pgfpathcurveto{\pgfqpoint{0.470800in}{0.412000in}}{\pgfqpoint{0.481399in}{0.416390in}}{\pgfqpoint{0.489213in}{0.424204in}}%
\pgfpathcurveto{\pgfqpoint{0.497026in}{0.432017in}}{\pgfqpoint{0.501417in}{0.442616in}}{\pgfqpoint{0.501417in}{0.453666in}}%
\pgfpathcurveto{\pgfqpoint{0.501417in}{0.464716in}}{\pgfqpoint{0.497026in}{0.475315in}}{\pgfqpoint{0.489213in}{0.483129in}}%
\pgfpathcurveto{\pgfqpoint{0.481399in}{0.490943in}}{\pgfqpoint{0.470800in}{0.495333in}}{\pgfqpoint{0.459750in}{0.495333in}}%
\pgfpathcurveto{\pgfqpoint{0.448700in}{0.495333in}}{\pgfqpoint{0.438101in}{0.490943in}}{\pgfqpoint{0.430287in}{0.483129in}}%
\pgfpathcurveto{\pgfqpoint{0.422474in}{0.475315in}}{\pgfqpoint{0.418083in}{0.464716in}}{\pgfqpoint{0.418083in}{0.453666in}}%
\pgfpathcurveto{\pgfqpoint{0.418083in}{0.442616in}}{\pgfqpoint{0.422474in}{0.432017in}}{\pgfqpoint{0.430287in}{0.424204in}}%
\pgfpathcurveto{\pgfqpoint{0.438101in}{0.416390in}}{\pgfqpoint{0.448700in}{0.412000in}}{\pgfqpoint{0.459750in}{0.412000in}}%
\pgfpathclose%
\pgfusepath{stroke,fill}%
\end{pgfscope}%
\begin{pgfscope}%
\pgfpathrectangle{\pgfqpoint{0.375000in}{0.330000in}}{\pgfqpoint{2.325000in}{2.310000in}}%
\pgfusepath{clip}%
\pgfsetbuttcap%
\pgfsetroundjoin%
\definecolor{currentfill}{rgb}{0.000000,0.000000,0.000000}%
\pgfsetfillcolor{currentfill}%
\pgfsetlinewidth{1.003750pt}%
\definecolor{currentstroke}{rgb}{0.000000,0.000000,0.000000}%
\pgfsetstrokecolor{currentstroke}%
\pgfsetdash{}{0pt}%
\pgfpathmoveto{\pgfqpoint{0.459750in}{0.412000in}}%
\pgfpathcurveto{\pgfqpoint{0.470800in}{0.412000in}}{\pgfqpoint{0.481399in}{0.416390in}}{\pgfqpoint{0.489213in}{0.424204in}}%
\pgfpathcurveto{\pgfqpoint{0.497026in}{0.432017in}}{\pgfqpoint{0.501417in}{0.442616in}}{\pgfqpoint{0.501417in}{0.453666in}}%
\pgfpathcurveto{\pgfqpoint{0.501417in}{0.464716in}}{\pgfqpoint{0.497026in}{0.475315in}}{\pgfqpoint{0.489213in}{0.483129in}}%
\pgfpathcurveto{\pgfqpoint{0.481399in}{0.490943in}}{\pgfqpoint{0.470800in}{0.495333in}}{\pgfqpoint{0.459750in}{0.495333in}}%
\pgfpathcurveto{\pgfqpoint{0.448700in}{0.495333in}}{\pgfqpoint{0.438101in}{0.490943in}}{\pgfqpoint{0.430287in}{0.483129in}}%
\pgfpathcurveto{\pgfqpoint{0.422474in}{0.475315in}}{\pgfqpoint{0.418083in}{0.464716in}}{\pgfqpoint{0.418083in}{0.453666in}}%
\pgfpathcurveto{\pgfqpoint{0.418083in}{0.442616in}}{\pgfqpoint{0.422474in}{0.432017in}}{\pgfqpoint{0.430287in}{0.424204in}}%
\pgfpathcurveto{\pgfqpoint{0.438101in}{0.416390in}}{\pgfqpoint{0.448700in}{0.412000in}}{\pgfqpoint{0.459750in}{0.412000in}}%
\pgfpathclose%
\pgfusepath{stroke,fill}%
\end{pgfscope}%
\begin{pgfscope}%
\pgfpathrectangle{\pgfqpoint{0.375000in}{0.330000in}}{\pgfqpoint{2.325000in}{2.310000in}}%
\pgfusepath{clip}%
\pgfsetbuttcap%
\pgfsetroundjoin%
\definecolor{currentfill}{rgb}{0.000000,0.000000,0.000000}%
\pgfsetfillcolor{currentfill}%
\pgfsetlinewidth{1.003750pt}%
\definecolor{currentstroke}{rgb}{0.000000,0.000000,0.000000}%
\pgfsetstrokecolor{currentstroke}%
\pgfsetdash{}{0pt}%
\pgfpathmoveto{\pgfqpoint{0.459750in}{0.412000in}}%
\pgfpathcurveto{\pgfqpoint{0.470800in}{0.412000in}}{\pgfqpoint{0.481399in}{0.416390in}}{\pgfqpoint{0.489213in}{0.424204in}}%
\pgfpathcurveto{\pgfqpoint{0.497026in}{0.432017in}}{\pgfqpoint{0.501417in}{0.442616in}}{\pgfqpoint{0.501417in}{0.453666in}}%
\pgfpathcurveto{\pgfqpoint{0.501417in}{0.464716in}}{\pgfqpoint{0.497026in}{0.475315in}}{\pgfqpoint{0.489213in}{0.483129in}}%
\pgfpathcurveto{\pgfqpoint{0.481399in}{0.490943in}}{\pgfqpoint{0.470800in}{0.495333in}}{\pgfqpoint{0.459750in}{0.495333in}}%
\pgfpathcurveto{\pgfqpoint{0.448700in}{0.495333in}}{\pgfqpoint{0.438101in}{0.490943in}}{\pgfqpoint{0.430287in}{0.483129in}}%
\pgfpathcurveto{\pgfqpoint{0.422474in}{0.475315in}}{\pgfqpoint{0.418083in}{0.464716in}}{\pgfqpoint{0.418083in}{0.453666in}}%
\pgfpathcurveto{\pgfqpoint{0.418083in}{0.442616in}}{\pgfqpoint{0.422474in}{0.432017in}}{\pgfqpoint{0.430287in}{0.424204in}}%
\pgfpathcurveto{\pgfqpoint{0.438101in}{0.416390in}}{\pgfqpoint{0.448700in}{0.412000in}}{\pgfqpoint{0.459750in}{0.412000in}}%
\pgfpathclose%
\pgfusepath{stroke,fill}%
\end{pgfscope}%
\begin{pgfscope}%
\pgfpathrectangle{\pgfqpoint{0.375000in}{0.330000in}}{\pgfqpoint{2.325000in}{2.310000in}}%
\pgfusepath{clip}%
\pgfsetbuttcap%
\pgfsetroundjoin%
\definecolor{currentfill}{rgb}{0.000000,0.000000,0.000000}%
\pgfsetfillcolor{currentfill}%
\pgfsetlinewidth{1.003750pt}%
\definecolor{currentstroke}{rgb}{0.000000,0.000000,0.000000}%
\pgfsetstrokecolor{currentstroke}%
\pgfsetdash{}{0pt}%
\pgfpathmoveto{\pgfqpoint{0.459750in}{0.412000in}}%
\pgfpathcurveto{\pgfqpoint{0.470800in}{0.412000in}}{\pgfqpoint{0.481399in}{0.416390in}}{\pgfqpoint{0.489213in}{0.424204in}}%
\pgfpathcurveto{\pgfqpoint{0.497026in}{0.432017in}}{\pgfqpoint{0.501417in}{0.442616in}}{\pgfqpoint{0.501417in}{0.453666in}}%
\pgfpathcurveto{\pgfqpoint{0.501417in}{0.464716in}}{\pgfqpoint{0.497026in}{0.475315in}}{\pgfqpoint{0.489213in}{0.483129in}}%
\pgfpathcurveto{\pgfqpoint{0.481399in}{0.490943in}}{\pgfqpoint{0.470800in}{0.495333in}}{\pgfqpoint{0.459750in}{0.495333in}}%
\pgfpathcurveto{\pgfqpoint{0.448700in}{0.495333in}}{\pgfqpoint{0.438101in}{0.490943in}}{\pgfqpoint{0.430287in}{0.483129in}}%
\pgfpathcurveto{\pgfqpoint{0.422474in}{0.475315in}}{\pgfqpoint{0.418083in}{0.464716in}}{\pgfqpoint{0.418083in}{0.453666in}}%
\pgfpathcurveto{\pgfqpoint{0.418083in}{0.442616in}}{\pgfqpoint{0.422474in}{0.432017in}}{\pgfqpoint{0.430287in}{0.424204in}}%
\pgfpathcurveto{\pgfqpoint{0.438101in}{0.416390in}}{\pgfqpoint{0.448700in}{0.412000in}}{\pgfqpoint{0.459750in}{0.412000in}}%
\pgfpathclose%
\pgfusepath{stroke,fill}%
\end{pgfscope}%
\begin{pgfscope}%
\pgfpathrectangle{\pgfqpoint{0.375000in}{0.330000in}}{\pgfqpoint{2.325000in}{2.310000in}}%
\pgfusepath{clip}%
\pgfsetbuttcap%
\pgfsetroundjoin%
\definecolor{currentfill}{rgb}{0.000000,0.000000,0.000000}%
\pgfsetfillcolor{currentfill}%
\pgfsetlinewidth{1.003750pt}%
\definecolor{currentstroke}{rgb}{0.000000,0.000000,0.000000}%
\pgfsetstrokecolor{currentstroke}%
\pgfsetdash{}{0pt}%
\pgfpathmoveto{\pgfqpoint{0.459750in}{0.412000in}}%
\pgfpathcurveto{\pgfqpoint{0.470800in}{0.412000in}}{\pgfqpoint{0.481399in}{0.416390in}}{\pgfqpoint{0.489213in}{0.424204in}}%
\pgfpathcurveto{\pgfqpoint{0.497026in}{0.432017in}}{\pgfqpoint{0.501417in}{0.442616in}}{\pgfqpoint{0.501417in}{0.453666in}}%
\pgfpathcurveto{\pgfqpoint{0.501417in}{0.464716in}}{\pgfqpoint{0.497026in}{0.475315in}}{\pgfqpoint{0.489213in}{0.483129in}}%
\pgfpathcurveto{\pgfqpoint{0.481399in}{0.490943in}}{\pgfqpoint{0.470800in}{0.495333in}}{\pgfqpoint{0.459750in}{0.495333in}}%
\pgfpathcurveto{\pgfqpoint{0.448700in}{0.495333in}}{\pgfqpoint{0.438101in}{0.490943in}}{\pgfqpoint{0.430287in}{0.483129in}}%
\pgfpathcurveto{\pgfqpoint{0.422474in}{0.475315in}}{\pgfqpoint{0.418083in}{0.464716in}}{\pgfqpoint{0.418083in}{0.453666in}}%
\pgfpathcurveto{\pgfqpoint{0.418083in}{0.442616in}}{\pgfqpoint{0.422474in}{0.432017in}}{\pgfqpoint{0.430287in}{0.424204in}}%
\pgfpathcurveto{\pgfqpoint{0.438101in}{0.416390in}}{\pgfqpoint{0.448700in}{0.412000in}}{\pgfqpoint{0.459750in}{0.412000in}}%
\pgfpathclose%
\pgfusepath{stroke,fill}%
\end{pgfscope}%
\begin{pgfscope}%
\pgfpathrectangle{\pgfqpoint{0.375000in}{0.330000in}}{\pgfqpoint{2.325000in}{2.310000in}}%
\pgfusepath{clip}%
\pgfsetbuttcap%
\pgfsetroundjoin%
\definecolor{currentfill}{rgb}{0.000000,0.000000,0.000000}%
\pgfsetfillcolor{currentfill}%
\pgfsetlinewidth{1.003750pt}%
\definecolor{currentstroke}{rgb}{0.000000,0.000000,0.000000}%
\pgfsetstrokecolor{currentstroke}%
\pgfsetdash{}{0pt}%
\pgfpathmoveto{\pgfqpoint{0.459750in}{0.412000in}}%
\pgfpathcurveto{\pgfqpoint{0.470800in}{0.412000in}}{\pgfqpoint{0.481399in}{0.416390in}}{\pgfqpoint{0.489213in}{0.424204in}}%
\pgfpathcurveto{\pgfqpoint{0.497026in}{0.432017in}}{\pgfqpoint{0.501417in}{0.442616in}}{\pgfqpoint{0.501417in}{0.453666in}}%
\pgfpathcurveto{\pgfqpoint{0.501417in}{0.464716in}}{\pgfqpoint{0.497026in}{0.475315in}}{\pgfqpoint{0.489213in}{0.483129in}}%
\pgfpathcurveto{\pgfqpoint{0.481399in}{0.490943in}}{\pgfqpoint{0.470800in}{0.495333in}}{\pgfqpoint{0.459750in}{0.495333in}}%
\pgfpathcurveto{\pgfqpoint{0.448700in}{0.495333in}}{\pgfqpoint{0.438101in}{0.490943in}}{\pgfqpoint{0.430287in}{0.483129in}}%
\pgfpathcurveto{\pgfqpoint{0.422474in}{0.475315in}}{\pgfqpoint{0.418083in}{0.464716in}}{\pgfqpoint{0.418083in}{0.453666in}}%
\pgfpathcurveto{\pgfqpoint{0.418083in}{0.442616in}}{\pgfqpoint{0.422474in}{0.432017in}}{\pgfqpoint{0.430287in}{0.424204in}}%
\pgfpathcurveto{\pgfqpoint{0.438101in}{0.416390in}}{\pgfqpoint{0.448700in}{0.412000in}}{\pgfqpoint{0.459750in}{0.412000in}}%
\pgfpathclose%
\pgfusepath{stroke,fill}%
\end{pgfscope}%
\begin{pgfscope}%
\pgfpathrectangle{\pgfqpoint{0.375000in}{0.330000in}}{\pgfqpoint{2.325000in}{2.310000in}}%
\pgfusepath{clip}%
\pgfsetbuttcap%
\pgfsetroundjoin%
\definecolor{currentfill}{rgb}{0.000000,0.000000,0.000000}%
\pgfsetfillcolor{currentfill}%
\pgfsetlinewidth{1.003750pt}%
\definecolor{currentstroke}{rgb}{0.000000,0.000000,0.000000}%
\pgfsetstrokecolor{currentstroke}%
\pgfsetdash{}{0pt}%
\pgfpathmoveto{\pgfqpoint{0.459750in}{0.412000in}}%
\pgfpathcurveto{\pgfqpoint{0.470800in}{0.412000in}}{\pgfqpoint{0.481399in}{0.416390in}}{\pgfqpoint{0.489213in}{0.424204in}}%
\pgfpathcurveto{\pgfqpoint{0.497026in}{0.432017in}}{\pgfqpoint{0.501417in}{0.442616in}}{\pgfqpoint{0.501417in}{0.453666in}}%
\pgfpathcurveto{\pgfqpoint{0.501417in}{0.464716in}}{\pgfqpoint{0.497026in}{0.475315in}}{\pgfqpoint{0.489213in}{0.483129in}}%
\pgfpathcurveto{\pgfqpoint{0.481399in}{0.490943in}}{\pgfqpoint{0.470800in}{0.495333in}}{\pgfqpoint{0.459750in}{0.495333in}}%
\pgfpathcurveto{\pgfqpoint{0.448700in}{0.495333in}}{\pgfqpoint{0.438101in}{0.490943in}}{\pgfqpoint{0.430287in}{0.483129in}}%
\pgfpathcurveto{\pgfqpoint{0.422474in}{0.475315in}}{\pgfqpoint{0.418083in}{0.464716in}}{\pgfqpoint{0.418083in}{0.453666in}}%
\pgfpathcurveto{\pgfqpoint{0.418083in}{0.442616in}}{\pgfqpoint{0.422474in}{0.432017in}}{\pgfqpoint{0.430287in}{0.424204in}}%
\pgfpathcurveto{\pgfqpoint{0.438101in}{0.416390in}}{\pgfqpoint{0.448700in}{0.412000in}}{\pgfqpoint{0.459750in}{0.412000in}}%
\pgfpathclose%
\pgfusepath{stroke,fill}%
\end{pgfscope}%
\begin{pgfscope}%
\pgfpathrectangle{\pgfqpoint{0.375000in}{0.330000in}}{\pgfqpoint{2.325000in}{2.310000in}}%
\pgfusepath{clip}%
\pgfsetbuttcap%
\pgfsetroundjoin%
\definecolor{currentfill}{rgb}{0.000000,0.000000,0.000000}%
\pgfsetfillcolor{currentfill}%
\pgfsetlinewidth{1.003750pt}%
\definecolor{currentstroke}{rgb}{0.000000,0.000000,0.000000}%
\pgfsetstrokecolor{currentstroke}%
\pgfsetdash{}{0pt}%
\pgfpathmoveto{\pgfqpoint{0.459750in}{0.412000in}}%
\pgfpathcurveto{\pgfqpoint{0.470800in}{0.412000in}}{\pgfqpoint{0.481399in}{0.416390in}}{\pgfqpoint{0.489213in}{0.424204in}}%
\pgfpathcurveto{\pgfqpoint{0.497026in}{0.432017in}}{\pgfqpoint{0.501417in}{0.442616in}}{\pgfqpoint{0.501417in}{0.453666in}}%
\pgfpathcurveto{\pgfqpoint{0.501417in}{0.464716in}}{\pgfqpoint{0.497026in}{0.475315in}}{\pgfqpoint{0.489213in}{0.483129in}}%
\pgfpathcurveto{\pgfqpoint{0.481399in}{0.490943in}}{\pgfqpoint{0.470800in}{0.495333in}}{\pgfqpoint{0.459750in}{0.495333in}}%
\pgfpathcurveto{\pgfqpoint{0.448700in}{0.495333in}}{\pgfqpoint{0.438101in}{0.490943in}}{\pgfqpoint{0.430287in}{0.483129in}}%
\pgfpathcurveto{\pgfqpoint{0.422474in}{0.475315in}}{\pgfqpoint{0.418083in}{0.464716in}}{\pgfqpoint{0.418083in}{0.453666in}}%
\pgfpathcurveto{\pgfqpoint{0.418083in}{0.442616in}}{\pgfqpoint{0.422474in}{0.432017in}}{\pgfqpoint{0.430287in}{0.424204in}}%
\pgfpathcurveto{\pgfqpoint{0.438101in}{0.416390in}}{\pgfqpoint{0.448700in}{0.412000in}}{\pgfqpoint{0.459750in}{0.412000in}}%
\pgfpathclose%
\pgfusepath{stroke,fill}%
\end{pgfscope}%
\begin{pgfscope}%
\pgfpathrectangle{\pgfqpoint{0.375000in}{0.330000in}}{\pgfqpoint{2.325000in}{2.310000in}}%
\pgfusepath{clip}%
\pgfsetbuttcap%
\pgfsetroundjoin%
\definecolor{currentfill}{rgb}{0.000000,0.000000,0.000000}%
\pgfsetfillcolor{currentfill}%
\pgfsetlinewidth{1.003750pt}%
\definecolor{currentstroke}{rgb}{0.000000,0.000000,0.000000}%
\pgfsetstrokecolor{currentstroke}%
\pgfsetdash{}{0pt}%
\pgfpathmoveto{\pgfqpoint{0.459750in}{0.412000in}}%
\pgfpathcurveto{\pgfqpoint{0.470800in}{0.412000in}}{\pgfqpoint{0.481399in}{0.416390in}}{\pgfqpoint{0.489213in}{0.424204in}}%
\pgfpathcurveto{\pgfqpoint{0.497026in}{0.432017in}}{\pgfqpoint{0.501417in}{0.442616in}}{\pgfqpoint{0.501417in}{0.453666in}}%
\pgfpathcurveto{\pgfqpoint{0.501417in}{0.464716in}}{\pgfqpoint{0.497026in}{0.475315in}}{\pgfqpoint{0.489213in}{0.483129in}}%
\pgfpathcurveto{\pgfqpoint{0.481399in}{0.490943in}}{\pgfqpoint{0.470800in}{0.495333in}}{\pgfqpoint{0.459750in}{0.495333in}}%
\pgfpathcurveto{\pgfqpoint{0.448700in}{0.495333in}}{\pgfqpoint{0.438101in}{0.490943in}}{\pgfqpoint{0.430287in}{0.483129in}}%
\pgfpathcurveto{\pgfqpoint{0.422474in}{0.475315in}}{\pgfqpoint{0.418083in}{0.464716in}}{\pgfqpoint{0.418083in}{0.453666in}}%
\pgfpathcurveto{\pgfqpoint{0.418083in}{0.442616in}}{\pgfqpoint{0.422474in}{0.432017in}}{\pgfqpoint{0.430287in}{0.424204in}}%
\pgfpathcurveto{\pgfqpoint{0.438101in}{0.416390in}}{\pgfqpoint{0.448700in}{0.412000in}}{\pgfqpoint{0.459750in}{0.412000in}}%
\pgfpathclose%
\pgfusepath{stroke,fill}%
\end{pgfscope}%
\begin{pgfscope}%
\pgfpathrectangle{\pgfqpoint{0.375000in}{0.330000in}}{\pgfqpoint{2.325000in}{2.310000in}}%
\pgfusepath{clip}%
\pgfsetbuttcap%
\pgfsetroundjoin%
\definecolor{currentfill}{rgb}{0.000000,0.000000,0.000000}%
\pgfsetfillcolor{currentfill}%
\pgfsetlinewidth{1.003750pt}%
\definecolor{currentstroke}{rgb}{0.000000,0.000000,0.000000}%
\pgfsetstrokecolor{currentstroke}%
\pgfsetdash{}{0pt}%
\pgfpathmoveto{\pgfqpoint{0.459750in}{0.412000in}}%
\pgfpathcurveto{\pgfqpoint{0.470800in}{0.412000in}}{\pgfqpoint{0.481399in}{0.416390in}}{\pgfqpoint{0.489213in}{0.424204in}}%
\pgfpathcurveto{\pgfqpoint{0.497026in}{0.432017in}}{\pgfqpoint{0.501417in}{0.442616in}}{\pgfqpoint{0.501417in}{0.453666in}}%
\pgfpathcurveto{\pgfqpoint{0.501417in}{0.464716in}}{\pgfqpoint{0.497026in}{0.475315in}}{\pgfqpoint{0.489213in}{0.483129in}}%
\pgfpathcurveto{\pgfqpoint{0.481399in}{0.490943in}}{\pgfqpoint{0.470800in}{0.495333in}}{\pgfqpoint{0.459750in}{0.495333in}}%
\pgfpathcurveto{\pgfqpoint{0.448700in}{0.495333in}}{\pgfqpoint{0.438101in}{0.490943in}}{\pgfqpoint{0.430287in}{0.483129in}}%
\pgfpathcurveto{\pgfqpoint{0.422474in}{0.475315in}}{\pgfqpoint{0.418083in}{0.464716in}}{\pgfqpoint{0.418083in}{0.453666in}}%
\pgfpathcurveto{\pgfqpoint{0.418083in}{0.442616in}}{\pgfqpoint{0.422474in}{0.432017in}}{\pgfqpoint{0.430287in}{0.424204in}}%
\pgfpathcurveto{\pgfqpoint{0.438101in}{0.416390in}}{\pgfqpoint{0.448700in}{0.412000in}}{\pgfqpoint{0.459750in}{0.412000in}}%
\pgfpathclose%
\pgfusepath{stroke,fill}%
\end{pgfscope}%
\begin{pgfscope}%
\pgfpathrectangle{\pgfqpoint{0.375000in}{0.330000in}}{\pgfqpoint{2.325000in}{2.310000in}}%
\pgfusepath{clip}%
\pgfsetbuttcap%
\pgfsetroundjoin%
\definecolor{currentfill}{rgb}{0.000000,0.000000,0.000000}%
\pgfsetfillcolor{currentfill}%
\pgfsetlinewidth{1.003750pt}%
\definecolor{currentstroke}{rgb}{0.000000,0.000000,0.000000}%
\pgfsetstrokecolor{currentstroke}%
\pgfsetdash{}{0pt}%
\pgfpathmoveto{\pgfqpoint{0.459750in}{1.443291in}}%
\pgfpathcurveto{\pgfqpoint{0.470800in}{1.443291in}}{\pgfqpoint{0.481399in}{1.447682in}}{\pgfqpoint{0.489213in}{1.455495in}}%
\pgfpathcurveto{\pgfqpoint{0.497026in}{1.463309in}}{\pgfqpoint{0.501417in}{1.473908in}}{\pgfqpoint{0.501417in}{1.484958in}}%
\pgfpathcurveto{\pgfqpoint{0.501417in}{1.496008in}}{\pgfqpoint{0.497026in}{1.506607in}}{\pgfqpoint{0.489213in}{1.514421in}}%
\pgfpathcurveto{\pgfqpoint{0.481399in}{1.522235in}}{\pgfqpoint{0.470800in}{1.526625in}}{\pgfqpoint{0.459750in}{1.526625in}}%
\pgfpathcurveto{\pgfqpoint{0.448700in}{1.526625in}}{\pgfqpoint{0.438101in}{1.522235in}}{\pgfqpoint{0.430287in}{1.514421in}}%
\pgfpathcurveto{\pgfqpoint{0.422474in}{1.506607in}}{\pgfqpoint{0.418083in}{1.496008in}}{\pgfqpoint{0.418083in}{1.484958in}}%
\pgfpathcurveto{\pgfqpoint{0.418083in}{1.473908in}}{\pgfqpoint{0.422474in}{1.463309in}}{\pgfqpoint{0.430287in}{1.455495in}}%
\pgfpathcurveto{\pgfqpoint{0.438101in}{1.447682in}}{\pgfqpoint{0.448700in}{1.443291in}}{\pgfqpoint{0.459750in}{1.443291in}}%
\pgfpathclose%
\pgfusepath{stroke,fill}%
\end{pgfscope}%
\begin{pgfscope}%
\pgfpathrectangle{\pgfqpoint{0.375000in}{0.330000in}}{\pgfqpoint{2.325000in}{2.310000in}}%
\pgfusepath{clip}%
\pgfsetbuttcap%
\pgfsetroundjoin%
\definecolor{currentfill}{rgb}{0.000000,0.000000,0.000000}%
\pgfsetfillcolor{currentfill}%
\pgfsetlinewidth{1.003750pt}%
\definecolor{currentstroke}{rgb}{0.000000,0.000000,0.000000}%
\pgfsetstrokecolor{currentstroke}%
\pgfsetdash{}{0pt}%
\pgfpathmoveto{\pgfqpoint{0.459750in}{0.412000in}}%
\pgfpathcurveto{\pgfqpoint{0.470800in}{0.412000in}}{\pgfqpoint{0.481399in}{0.416390in}}{\pgfqpoint{0.489213in}{0.424204in}}%
\pgfpathcurveto{\pgfqpoint{0.497026in}{0.432017in}}{\pgfqpoint{0.501417in}{0.442616in}}{\pgfqpoint{0.501417in}{0.453666in}}%
\pgfpathcurveto{\pgfqpoint{0.501417in}{0.464716in}}{\pgfqpoint{0.497026in}{0.475315in}}{\pgfqpoint{0.489213in}{0.483129in}}%
\pgfpathcurveto{\pgfqpoint{0.481399in}{0.490943in}}{\pgfqpoint{0.470800in}{0.495333in}}{\pgfqpoint{0.459750in}{0.495333in}}%
\pgfpathcurveto{\pgfqpoint{0.448700in}{0.495333in}}{\pgfqpoint{0.438101in}{0.490943in}}{\pgfqpoint{0.430287in}{0.483129in}}%
\pgfpathcurveto{\pgfqpoint{0.422474in}{0.475315in}}{\pgfqpoint{0.418083in}{0.464716in}}{\pgfqpoint{0.418083in}{0.453666in}}%
\pgfpathcurveto{\pgfqpoint{0.418083in}{0.442616in}}{\pgfqpoint{0.422474in}{0.432017in}}{\pgfqpoint{0.430287in}{0.424204in}}%
\pgfpathcurveto{\pgfqpoint{0.438101in}{0.416390in}}{\pgfqpoint{0.448700in}{0.412000in}}{\pgfqpoint{0.459750in}{0.412000in}}%
\pgfpathclose%
\pgfusepath{stroke,fill}%
\end{pgfscope}%
\begin{pgfscope}%
\pgfpathrectangle{\pgfqpoint{0.375000in}{0.330000in}}{\pgfqpoint{2.325000in}{2.310000in}}%
\pgfusepath{clip}%
\pgfsetbuttcap%
\pgfsetroundjoin%
\definecolor{currentfill}{rgb}{0.000000,0.000000,0.000000}%
\pgfsetfillcolor{currentfill}%
\pgfsetlinewidth{1.003750pt}%
\definecolor{currentstroke}{rgb}{0.000000,0.000000,0.000000}%
\pgfsetstrokecolor{currentstroke}%
\pgfsetdash{}{0pt}%
\pgfpathmoveto{\pgfqpoint{0.459750in}{0.412000in}}%
\pgfpathcurveto{\pgfqpoint{0.470800in}{0.412000in}}{\pgfqpoint{0.481399in}{0.416390in}}{\pgfqpoint{0.489213in}{0.424204in}}%
\pgfpathcurveto{\pgfqpoint{0.497026in}{0.432017in}}{\pgfqpoint{0.501417in}{0.442616in}}{\pgfqpoint{0.501417in}{0.453666in}}%
\pgfpathcurveto{\pgfqpoint{0.501417in}{0.464716in}}{\pgfqpoint{0.497026in}{0.475315in}}{\pgfqpoint{0.489213in}{0.483129in}}%
\pgfpathcurveto{\pgfqpoint{0.481399in}{0.490943in}}{\pgfqpoint{0.470800in}{0.495333in}}{\pgfqpoint{0.459750in}{0.495333in}}%
\pgfpathcurveto{\pgfqpoint{0.448700in}{0.495333in}}{\pgfqpoint{0.438101in}{0.490943in}}{\pgfqpoint{0.430287in}{0.483129in}}%
\pgfpathcurveto{\pgfqpoint{0.422474in}{0.475315in}}{\pgfqpoint{0.418083in}{0.464716in}}{\pgfqpoint{0.418083in}{0.453666in}}%
\pgfpathcurveto{\pgfqpoint{0.418083in}{0.442616in}}{\pgfqpoint{0.422474in}{0.432017in}}{\pgfqpoint{0.430287in}{0.424204in}}%
\pgfpathcurveto{\pgfqpoint{0.438101in}{0.416390in}}{\pgfqpoint{0.448700in}{0.412000in}}{\pgfqpoint{0.459750in}{0.412000in}}%
\pgfpathclose%
\pgfusepath{stroke,fill}%
\end{pgfscope}%
\begin{pgfscope}%
\pgfpathrectangle{\pgfqpoint{0.375000in}{0.330000in}}{\pgfqpoint{2.325000in}{2.310000in}}%
\pgfusepath{clip}%
\pgfsetbuttcap%
\pgfsetroundjoin%
\definecolor{currentfill}{rgb}{0.000000,0.000000,0.000000}%
\pgfsetfillcolor{currentfill}%
\pgfsetlinewidth{1.003750pt}%
\definecolor{currentstroke}{rgb}{0.000000,0.000000,0.000000}%
\pgfsetstrokecolor{currentstroke}%
\pgfsetdash{}{0pt}%
\pgfpathmoveto{\pgfqpoint{0.459750in}{0.412000in}}%
\pgfpathcurveto{\pgfqpoint{0.470800in}{0.412000in}}{\pgfqpoint{0.481399in}{0.416390in}}{\pgfqpoint{0.489213in}{0.424204in}}%
\pgfpathcurveto{\pgfqpoint{0.497026in}{0.432017in}}{\pgfqpoint{0.501417in}{0.442616in}}{\pgfqpoint{0.501417in}{0.453666in}}%
\pgfpathcurveto{\pgfqpoint{0.501417in}{0.464716in}}{\pgfqpoint{0.497026in}{0.475315in}}{\pgfqpoint{0.489213in}{0.483129in}}%
\pgfpathcurveto{\pgfqpoint{0.481399in}{0.490943in}}{\pgfqpoint{0.470800in}{0.495333in}}{\pgfqpoint{0.459750in}{0.495333in}}%
\pgfpathcurveto{\pgfqpoint{0.448700in}{0.495333in}}{\pgfqpoint{0.438101in}{0.490943in}}{\pgfqpoint{0.430287in}{0.483129in}}%
\pgfpathcurveto{\pgfqpoint{0.422474in}{0.475315in}}{\pgfqpoint{0.418083in}{0.464716in}}{\pgfqpoint{0.418083in}{0.453666in}}%
\pgfpathcurveto{\pgfqpoint{0.418083in}{0.442616in}}{\pgfqpoint{0.422474in}{0.432017in}}{\pgfqpoint{0.430287in}{0.424204in}}%
\pgfpathcurveto{\pgfqpoint{0.438101in}{0.416390in}}{\pgfqpoint{0.448700in}{0.412000in}}{\pgfqpoint{0.459750in}{0.412000in}}%
\pgfpathclose%
\pgfusepath{stroke,fill}%
\end{pgfscope}%
\begin{pgfscope}%
\pgfpathrectangle{\pgfqpoint{0.375000in}{0.330000in}}{\pgfqpoint{2.325000in}{2.310000in}}%
\pgfusepath{clip}%
\pgfsetbuttcap%
\pgfsetroundjoin%
\definecolor{currentfill}{rgb}{0.000000,0.000000,0.000000}%
\pgfsetfillcolor{currentfill}%
\pgfsetlinewidth{1.003750pt}%
\definecolor{currentstroke}{rgb}{0.000000,0.000000,0.000000}%
\pgfsetstrokecolor{currentstroke}%
\pgfsetdash{}{0pt}%
\pgfpathmoveto{\pgfqpoint{0.459750in}{1.443291in}}%
\pgfpathcurveto{\pgfqpoint{0.470800in}{1.443291in}}{\pgfqpoint{0.481399in}{1.447682in}}{\pgfqpoint{0.489213in}{1.455495in}}%
\pgfpathcurveto{\pgfqpoint{0.497026in}{1.463309in}}{\pgfqpoint{0.501417in}{1.473908in}}{\pgfqpoint{0.501417in}{1.484958in}}%
\pgfpathcurveto{\pgfqpoint{0.501417in}{1.496008in}}{\pgfqpoint{0.497026in}{1.506607in}}{\pgfqpoint{0.489213in}{1.514421in}}%
\pgfpathcurveto{\pgfqpoint{0.481399in}{1.522235in}}{\pgfqpoint{0.470800in}{1.526625in}}{\pgfqpoint{0.459750in}{1.526625in}}%
\pgfpathcurveto{\pgfqpoint{0.448700in}{1.526625in}}{\pgfqpoint{0.438101in}{1.522235in}}{\pgfqpoint{0.430287in}{1.514421in}}%
\pgfpathcurveto{\pgfqpoint{0.422474in}{1.506607in}}{\pgfqpoint{0.418083in}{1.496008in}}{\pgfqpoint{0.418083in}{1.484958in}}%
\pgfpathcurveto{\pgfqpoint{0.418083in}{1.473908in}}{\pgfqpoint{0.422474in}{1.463309in}}{\pgfqpoint{0.430287in}{1.455495in}}%
\pgfpathcurveto{\pgfqpoint{0.438101in}{1.447682in}}{\pgfqpoint{0.448700in}{1.443291in}}{\pgfqpoint{0.459750in}{1.443291in}}%
\pgfpathclose%
\pgfusepath{stroke,fill}%
\end{pgfscope}%
\begin{pgfscope}%
\pgfpathrectangle{\pgfqpoint{0.375000in}{0.330000in}}{\pgfqpoint{2.325000in}{2.310000in}}%
\pgfusepath{clip}%
\pgfsetbuttcap%
\pgfsetroundjoin%
\definecolor{currentfill}{rgb}{0.000000,0.000000,0.000000}%
\pgfsetfillcolor{currentfill}%
\pgfsetlinewidth{1.003750pt}%
\definecolor{currentstroke}{rgb}{0.000000,0.000000,0.000000}%
\pgfsetstrokecolor{currentstroke}%
\pgfsetdash{}{0pt}%
\pgfpathmoveto{\pgfqpoint{0.459750in}{0.412000in}}%
\pgfpathcurveto{\pgfqpoint{0.470800in}{0.412000in}}{\pgfqpoint{0.481399in}{0.416390in}}{\pgfqpoint{0.489213in}{0.424204in}}%
\pgfpathcurveto{\pgfqpoint{0.497026in}{0.432017in}}{\pgfqpoint{0.501417in}{0.442616in}}{\pgfqpoint{0.501417in}{0.453666in}}%
\pgfpathcurveto{\pgfqpoint{0.501417in}{0.464716in}}{\pgfqpoint{0.497026in}{0.475315in}}{\pgfqpoint{0.489213in}{0.483129in}}%
\pgfpathcurveto{\pgfqpoint{0.481399in}{0.490943in}}{\pgfqpoint{0.470800in}{0.495333in}}{\pgfqpoint{0.459750in}{0.495333in}}%
\pgfpathcurveto{\pgfqpoint{0.448700in}{0.495333in}}{\pgfqpoint{0.438101in}{0.490943in}}{\pgfqpoint{0.430287in}{0.483129in}}%
\pgfpathcurveto{\pgfqpoint{0.422474in}{0.475315in}}{\pgfqpoint{0.418083in}{0.464716in}}{\pgfqpoint{0.418083in}{0.453666in}}%
\pgfpathcurveto{\pgfqpoint{0.418083in}{0.442616in}}{\pgfqpoint{0.422474in}{0.432017in}}{\pgfqpoint{0.430287in}{0.424204in}}%
\pgfpathcurveto{\pgfqpoint{0.438101in}{0.416390in}}{\pgfqpoint{0.448700in}{0.412000in}}{\pgfqpoint{0.459750in}{0.412000in}}%
\pgfpathclose%
\pgfusepath{stroke,fill}%
\end{pgfscope}%
\begin{pgfscope}%
\pgfpathrectangle{\pgfqpoint{0.375000in}{0.330000in}}{\pgfqpoint{2.325000in}{2.310000in}}%
\pgfusepath{clip}%
\pgfsetbuttcap%
\pgfsetroundjoin%
\definecolor{currentfill}{rgb}{0.000000,0.000000,0.000000}%
\pgfsetfillcolor{currentfill}%
\pgfsetlinewidth{1.003750pt}%
\definecolor{currentstroke}{rgb}{0.000000,0.000000,0.000000}%
\pgfsetstrokecolor{currentstroke}%
\pgfsetdash{}{0pt}%
\pgfpathmoveto{\pgfqpoint{0.459750in}{0.412000in}}%
\pgfpathcurveto{\pgfqpoint{0.470800in}{0.412000in}}{\pgfqpoint{0.481399in}{0.416390in}}{\pgfqpoint{0.489213in}{0.424204in}}%
\pgfpathcurveto{\pgfqpoint{0.497026in}{0.432017in}}{\pgfqpoint{0.501417in}{0.442616in}}{\pgfqpoint{0.501417in}{0.453666in}}%
\pgfpathcurveto{\pgfqpoint{0.501417in}{0.464716in}}{\pgfqpoint{0.497026in}{0.475315in}}{\pgfqpoint{0.489213in}{0.483129in}}%
\pgfpathcurveto{\pgfqpoint{0.481399in}{0.490943in}}{\pgfqpoint{0.470800in}{0.495333in}}{\pgfqpoint{0.459750in}{0.495333in}}%
\pgfpathcurveto{\pgfqpoint{0.448700in}{0.495333in}}{\pgfqpoint{0.438101in}{0.490943in}}{\pgfqpoint{0.430287in}{0.483129in}}%
\pgfpathcurveto{\pgfqpoint{0.422474in}{0.475315in}}{\pgfqpoint{0.418083in}{0.464716in}}{\pgfqpoint{0.418083in}{0.453666in}}%
\pgfpathcurveto{\pgfqpoint{0.418083in}{0.442616in}}{\pgfqpoint{0.422474in}{0.432017in}}{\pgfqpoint{0.430287in}{0.424204in}}%
\pgfpathcurveto{\pgfqpoint{0.438101in}{0.416390in}}{\pgfqpoint{0.448700in}{0.412000in}}{\pgfqpoint{0.459750in}{0.412000in}}%
\pgfpathclose%
\pgfusepath{stroke,fill}%
\end{pgfscope}%
\begin{pgfscope}%
\pgfpathrectangle{\pgfqpoint{0.375000in}{0.330000in}}{\pgfqpoint{2.325000in}{2.310000in}}%
\pgfusepath{clip}%
\pgfsetbuttcap%
\pgfsetroundjoin%
\definecolor{currentfill}{rgb}{0.000000,0.000000,0.000000}%
\pgfsetfillcolor{currentfill}%
\pgfsetlinewidth{1.003750pt}%
\definecolor{currentstroke}{rgb}{0.000000,0.000000,0.000000}%
\pgfsetstrokecolor{currentstroke}%
\pgfsetdash{}{0pt}%
\pgfpathmoveto{\pgfqpoint{0.459750in}{0.412000in}}%
\pgfpathcurveto{\pgfqpoint{0.470800in}{0.412000in}}{\pgfqpoint{0.481399in}{0.416390in}}{\pgfqpoint{0.489213in}{0.424204in}}%
\pgfpathcurveto{\pgfqpoint{0.497026in}{0.432017in}}{\pgfqpoint{0.501417in}{0.442616in}}{\pgfqpoint{0.501417in}{0.453666in}}%
\pgfpathcurveto{\pgfqpoint{0.501417in}{0.464716in}}{\pgfqpoint{0.497026in}{0.475315in}}{\pgfqpoint{0.489213in}{0.483129in}}%
\pgfpathcurveto{\pgfqpoint{0.481399in}{0.490943in}}{\pgfqpoint{0.470800in}{0.495333in}}{\pgfqpoint{0.459750in}{0.495333in}}%
\pgfpathcurveto{\pgfqpoint{0.448700in}{0.495333in}}{\pgfqpoint{0.438101in}{0.490943in}}{\pgfqpoint{0.430287in}{0.483129in}}%
\pgfpathcurveto{\pgfqpoint{0.422474in}{0.475315in}}{\pgfqpoint{0.418083in}{0.464716in}}{\pgfqpoint{0.418083in}{0.453666in}}%
\pgfpathcurveto{\pgfqpoint{0.418083in}{0.442616in}}{\pgfqpoint{0.422474in}{0.432017in}}{\pgfqpoint{0.430287in}{0.424204in}}%
\pgfpathcurveto{\pgfqpoint{0.438101in}{0.416390in}}{\pgfqpoint{0.448700in}{0.412000in}}{\pgfqpoint{0.459750in}{0.412000in}}%
\pgfpathclose%
\pgfusepath{stroke,fill}%
\end{pgfscope}%
\begin{pgfscope}%
\pgfpathrectangle{\pgfqpoint{0.375000in}{0.330000in}}{\pgfqpoint{2.325000in}{2.310000in}}%
\pgfusepath{clip}%
\pgfsetbuttcap%
\pgfsetroundjoin%
\definecolor{currentfill}{rgb}{0.000000,0.000000,0.000000}%
\pgfsetfillcolor{currentfill}%
\pgfsetlinewidth{1.003750pt}%
\definecolor{currentstroke}{rgb}{0.000000,0.000000,0.000000}%
\pgfsetstrokecolor{currentstroke}%
\pgfsetdash{}{0pt}%
\pgfpathmoveto{\pgfqpoint{0.459750in}{0.412000in}}%
\pgfpathcurveto{\pgfqpoint{0.470800in}{0.412000in}}{\pgfqpoint{0.481399in}{0.416390in}}{\pgfqpoint{0.489213in}{0.424204in}}%
\pgfpathcurveto{\pgfqpoint{0.497026in}{0.432017in}}{\pgfqpoint{0.501417in}{0.442616in}}{\pgfqpoint{0.501417in}{0.453666in}}%
\pgfpathcurveto{\pgfqpoint{0.501417in}{0.464716in}}{\pgfqpoint{0.497026in}{0.475315in}}{\pgfqpoint{0.489213in}{0.483129in}}%
\pgfpathcurveto{\pgfqpoint{0.481399in}{0.490943in}}{\pgfqpoint{0.470800in}{0.495333in}}{\pgfqpoint{0.459750in}{0.495333in}}%
\pgfpathcurveto{\pgfqpoint{0.448700in}{0.495333in}}{\pgfqpoint{0.438101in}{0.490943in}}{\pgfqpoint{0.430287in}{0.483129in}}%
\pgfpathcurveto{\pgfqpoint{0.422474in}{0.475315in}}{\pgfqpoint{0.418083in}{0.464716in}}{\pgfqpoint{0.418083in}{0.453666in}}%
\pgfpathcurveto{\pgfqpoint{0.418083in}{0.442616in}}{\pgfqpoint{0.422474in}{0.432017in}}{\pgfqpoint{0.430287in}{0.424204in}}%
\pgfpathcurveto{\pgfqpoint{0.438101in}{0.416390in}}{\pgfqpoint{0.448700in}{0.412000in}}{\pgfqpoint{0.459750in}{0.412000in}}%
\pgfpathclose%
\pgfusepath{stroke,fill}%
\end{pgfscope}%
\begin{pgfscope}%
\pgfpathrectangle{\pgfqpoint{0.375000in}{0.330000in}}{\pgfqpoint{2.325000in}{2.310000in}}%
\pgfusepath{clip}%
\pgfsetbuttcap%
\pgfsetroundjoin%
\definecolor{currentfill}{rgb}{0.000000,0.000000,0.000000}%
\pgfsetfillcolor{currentfill}%
\pgfsetlinewidth{1.003750pt}%
\definecolor{currentstroke}{rgb}{0.000000,0.000000,0.000000}%
\pgfsetstrokecolor{currentstroke}%
\pgfsetdash{}{0pt}%
\pgfpathmoveto{\pgfqpoint{0.459750in}{0.412000in}}%
\pgfpathcurveto{\pgfqpoint{0.470800in}{0.412000in}}{\pgfqpoint{0.481399in}{0.416390in}}{\pgfqpoint{0.489213in}{0.424204in}}%
\pgfpathcurveto{\pgfqpoint{0.497026in}{0.432017in}}{\pgfqpoint{0.501417in}{0.442616in}}{\pgfqpoint{0.501417in}{0.453666in}}%
\pgfpathcurveto{\pgfqpoint{0.501417in}{0.464716in}}{\pgfqpoint{0.497026in}{0.475315in}}{\pgfqpoint{0.489213in}{0.483129in}}%
\pgfpathcurveto{\pgfqpoint{0.481399in}{0.490943in}}{\pgfqpoint{0.470800in}{0.495333in}}{\pgfqpoint{0.459750in}{0.495333in}}%
\pgfpathcurveto{\pgfqpoint{0.448700in}{0.495333in}}{\pgfqpoint{0.438101in}{0.490943in}}{\pgfqpoint{0.430287in}{0.483129in}}%
\pgfpathcurveto{\pgfqpoint{0.422474in}{0.475315in}}{\pgfqpoint{0.418083in}{0.464716in}}{\pgfqpoint{0.418083in}{0.453666in}}%
\pgfpathcurveto{\pgfqpoint{0.418083in}{0.442616in}}{\pgfqpoint{0.422474in}{0.432017in}}{\pgfqpoint{0.430287in}{0.424204in}}%
\pgfpathcurveto{\pgfqpoint{0.438101in}{0.416390in}}{\pgfqpoint{0.448700in}{0.412000in}}{\pgfqpoint{0.459750in}{0.412000in}}%
\pgfpathclose%
\pgfusepath{stroke,fill}%
\end{pgfscope}%
\begin{pgfscope}%
\pgfpathrectangle{\pgfqpoint{0.375000in}{0.330000in}}{\pgfqpoint{2.325000in}{2.310000in}}%
\pgfusepath{clip}%
\pgfsetbuttcap%
\pgfsetroundjoin%
\definecolor{currentfill}{rgb}{0.000000,0.000000,0.000000}%
\pgfsetfillcolor{currentfill}%
\pgfsetlinewidth{1.003750pt}%
\definecolor{currentstroke}{rgb}{0.000000,0.000000,0.000000}%
\pgfsetstrokecolor{currentstroke}%
\pgfsetdash{}{0pt}%
\pgfpathmoveto{\pgfqpoint{0.459750in}{0.412000in}}%
\pgfpathcurveto{\pgfqpoint{0.470800in}{0.412000in}}{\pgfqpoint{0.481399in}{0.416390in}}{\pgfqpoint{0.489213in}{0.424204in}}%
\pgfpathcurveto{\pgfqpoint{0.497026in}{0.432017in}}{\pgfqpoint{0.501417in}{0.442616in}}{\pgfqpoint{0.501417in}{0.453666in}}%
\pgfpathcurveto{\pgfqpoint{0.501417in}{0.464716in}}{\pgfqpoint{0.497026in}{0.475315in}}{\pgfqpoint{0.489213in}{0.483129in}}%
\pgfpathcurveto{\pgfqpoint{0.481399in}{0.490943in}}{\pgfqpoint{0.470800in}{0.495333in}}{\pgfqpoint{0.459750in}{0.495333in}}%
\pgfpathcurveto{\pgfqpoint{0.448700in}{0.495333in}}{\pgfqpoint{0.438101in}{0.490943in}}{\pgfqpoint{0.430287in}{0.483129in}}%
\pgfpathcurveto{\pgfqpoint{0.422474in}{0.475315in}}{\pgfqpoint{0.418083in}{0.464716in}}{\pgfqpoint{0.418083in}{0.453666in}}%
\pgfpathcurveto{\pgfqpoint{0.418083in}{0.442616in}}{\pgfqpoint{0.422474in}{0.432017in}}{\pgfqpoint{0.430287in}{0.424204in}}%
\pgfpathcurveto{\pgfqpoint{0.438101in}{0.416390in}}{\pgfqpoint{0.448700in}{0.412000in}}{\pgfqpoint{0.459750in}{0.412000in}}%
\pgfpathclose%
\pgfusepath{stroke,fill}%
\end{pgfscope}%
\begin{pgfscope}%
\pgfpathrectangle{\pgfqpoint{0.375000in}{0.330000in}}{\pgfqpoint{2.325000in}{2.310000in}}%
\pgfusepath{clip}%
\pgfsetbuttcap%
\pgfsetroundjoin%
\definecolor{currentfill}{rgb}{0.000000,0.000000,0.000000}%
\pgfsetfillcolor{currentfill}%
\pgfsetlinewidth{1.003750pt}%
\definecolor{currentstroke}{rgb}{0.000000,0.000000,0.000000}%
\pgfsetstrokecolor{currentstroke}%
\pgfsetdash{}{0pt}%
\pgfpathmoveto{\pgfqpoint{0.459750in}{0.412000in}}%
\pgfpathcurveto{\pgfqpoint{0.470800in}{0.412000in}}{\pgfqpoint{0.481399in}{0.416390in}}{\pgfqpoint{0.489213in}{0.424204in}}%
\pgfpathcurveto{\pgfqpoint{0.497026in}{0.432017in}}{\pgfqpoint{0.501417in}{0.442616in}}{\pgfqpoint{0.501417in}{0.453666in}}%
\pgfpathcurveto{\pgfqpoint{0.501417in}{0.464716in}}{\pgfqpoint{0.497026in}{0.475315in}}{\pgfqpoint{0.489213in}{0.483129in}}%
\pgfpathcurveto{\pgfqpoint{0.481399in}{0.490943in}}{\pgfqpoint{0.470800in}{0.495333in}}{\pgfqpoint{0.459750in}{0.495333in}}%
\pgfpathcurveto{\pgfqpoint{0.448700in}{0.495333in}}{\pgfqpoint{0.438101in}{0.490943in}}{\pgfqpoint{0.430287in}{0.483129in}}%
\pgfpathcurveto{\pgfqpoint{0.422474in}{0.475315in}}{\pgfqpoint{0.418083in}{0.464716in}}{\pgfqpoint{0.418083in}{0.453666in}}%
\pgfpathcurveto{\pgfqpoint{0.418083in}{0.442616in}}{\pgfqpoint{0.422474in}{0.432017in}}{\pgfqpoint{0.430287in}{0.424204in}}%
\pgfpathcurveto{\pgfqpoint{0.438101in}{0.416390in}}{\pgfqpoint{0.448700in}{0.412000in}}{\pgfqpoint{0.459750in}{0.412000in}}%
\pgfpathclose%
\pgfusepath{stroke,fill}%
\end{pgfscope}%
\begin{pgfscope}%
\pgfpathrectangle{\pgfqpoint{0.375000in}{0.330000in}}{\pgfqpoint{2.325000in}{2.310000in}}%
\pgfusepath{clip}%
\pgfsetbuttcap%
\pgfsetroundjoin%
\definecolor{currentfill}{rgb}{0.000000,0.000000,0.000000}%
\pgfsetfillcolor{currentfill}%
\pgfsetlinewidth{1.003750pt}%
\definecolor{currentstroke}{rgb}{0.000000,0.000000,0.000000}%
\pgfsetstrokecolor{currentstroke}%
\pgfsetdash{}{0pt}%
\pgfpathmoveto{\pgfqpoint{0.459750in}{0.412000in}}%
\pgfpathcurveto{\pgfqpoint{0.470800in}{0.412000in}}{\pgfqpoint{0.481399in}{0.416390in}}{\pgfqpoint{0.489213in}{0.424204in}}%
\pgfpathcurveto{\pgfqpoint{0.497026in}{0.432017in}}{\pgfqpoint{0.501417in}{0.442616in}}{\pgfqpoint{0.501417in}{0.453666in}}%
\pgfpathcurveto{\pgfqpoint{0.501417in}{0.464716in}}{\pgfqpoint{0.497026in}{0.475315in}}{\pgfqpoint{0.489213in}{0.483129in}}%
\pgfpathcurveto{\pgfqpoint{0.481399in}{0.490943in}}{\pgfqpoint{0.470800in}{0.495333in}}{\pgfqpoint{0.459750in}{0.495333in}}%
\pgfpathcurveto{\pgfqpoint{0.448700in}{0.495333in}}{\pgfqpoint{0.438101in}{0.490943in}}{\pgfqpoint{0.430287in}{0.483129in}}%
\pgfpathcurveto{\pgfqpoint{0.422474in}{0.475315in}}{\pgfqpoint{0.418083in}{0.464716in}}{\pgfqpoint{0.418083in}{0.453666in}}%
\pgfpathcurveto{\pgfqpoint{0.418083in}{0.442616in}}{\pgfqpoint{0.422474in}{0.432017in}}{\pgfqpoint{0.430287in}{0.424204in}}%
\pgfpathcurveto{\pgfqpoint{0.438101in}{0.416390in}}{\pgfqpoint{0.448700in}{0.412000in}}{\pgfqpoint{0.459750in}{0.412000in}}%
\pgfpathclose%
\pgfusepath{stroke,fill}%
\end{pgfscope}%
\begin{pgfscope}%
\pgfpathrectangle{\pgfqpoint{0.375000in}{0.330000in}}{\pgfqpoint{2.325000in}{2.310000in}}%
\pgfusepath{clip}%
\pgfsetbuttcap%
\pgfsetroundjoin%
\definecolor{currentfill}{rgb}{0.000000,0.000000,0.000000}%
\pgfsetfillcolor{currentfill}%
\pgfsetlinewidth{1.003750pt}%
\definecolor{currentstroke}{rgb}{0.000000,0.000000,0.000000}%
\pgfsetstrokecolor{currentstroke}%
\pgfsetdash{}{0pt}%
\pgfpathmoveto{\pgfqpoint{0.459750in}{0.412000in}}%
\pgfpathcurveto{\pgfqpoint{0.470800in}{0.412000in}}{\pgfqpoint{0.481399in}{0.416390in}}{\pgfqpoint{0.489213in}{0.424204in}}%
\pgfpathcurveto{\pgfqpoint{0.497026in}{0.432017in}}{\pgfqpoint{0.501417in}{0.442616in}}{\pgfqpoint{0.501417in}{0.453666in}}%
\pgfpathcurveto{\pgfqpoint{0.501417in}{0.464716in}}{\pgfqpoint{0.497026in}{0.475315in}}{\pgfqpoint{0.489213in}{0.483129in}}%
\pgfpathcurveto{\pgfqpoint{0.481399in}{0.490943in}}{\pgfqpoint{0.470800in}{0.495333in}}{\pgfqpoint{0.459750in}{0.495333in}}%
\pgfpathcurveto{\pgfqpoint{0.448700in}{0.495333in}}{\pgfqpoint{0.438101in}{0.490943in}}{\pgfqpoint{0.430287in}{0.483129in}}%
\pgfpathcurveto{\pgfqpoint{0.422474in}{0.475315in}}{\pgfqpoint{0.418083in}{0.464716in}}{\pgfqpoint{0.418083in}{0.453666in}}%
\pgfpathcurveto{\pgfqpoint{0.418083in}{0.442616in}}{\pgfqpoint{0.422474in}{0.432017in}}{\pgfqpoint{0.430287in}{0.424204in}}%
\pgfpathcurveto{\pgfqpoint{0.438101in}{0.416390in}}{\pgfqpoint{0.448700in}{0.412000in}}{\pgfqpoint{0.459750in}{0.412000in}}%
\pgfpathclose%
\pgfusepath{stroke,fill}%
\end{pgfscope}%
\begin{pgfscope}%
\pgfpathrectangle{\pgfqpoint{0.375000in}{0.330000in}}{\pgfqpoint{2.325000in}{2.310000in}}%
\pgfusepath{clip}%
\pgfsetbuttcap%
\pgfsetroundjoin%
\definecolor{currentfill}{rgb}{0.000000,0.000000,0.000000}%
\pgfsetfillcolor{currentfill}%
\pgfsetlinewidth{1.003750pt}%
\definecolor{currentstroke}{rgb}{0.000000,0.000000,0.000000}%
\pgfsetstrokecolor{currentstroke}%
\pgfsetdash{}{0pt}%
\pgfpathmoveto{\pgfqpoint{0.459750in}{0.412000in}}%
\pgfpathcurveto{\pgfqpoint{0.470800in}{0.412000in}}{\pgfqpoint{0.481399in}{0.416390in}}{\pgfqpoint{0.489213in}{0.424204in}}%
\pgfpathcurveto{\pgfqpoint{0.497026in}{0.432017in}}{\pgfqpoint{0.501417in}{0.442616in}}{\pgfqpoint{0.501417in}{0.453666in}}%
\pgfpathcurveto{\pgfqpoint{0.501417in}{0.464716in}}{\pgfqpoint{0.497026in}{0.475315in}}{\pgfqpoint{0.489213in}{0.483129in}}%
\pgfpathcurveto{\pgfqpoint{0.481399in}{0.490943in}}{\pgfqpoint{0.470800in}{0.495333in}}{\pgfqpoint{0.459750in}{0.495333in}}%
\pgfpathcurveto{\pgfqpoint{0.448700in}{0.495333in}}{\pgfqpoint{0.438101in}{0.490943in}}{\pgfqpoint{0.430287in}{0.483129in}}%
\pgfpathcurveto{\pgfqpoint{0.422474in}{0.475315in}}{\pgfqpoint{0.418083in}{0.464716in}}{\pgfqpoint{0.418083in}{0.453666in}}%
\pgfpathcurveto{\pgfqpoint{0.418083in}{0.442616in}}{\pgfqpoint{0.422474in}{0.432017in}}{\pgfqpoint{0.430287in}{0.424204in}}%
\pgfpathcurveto{\pgfqpoint{0.438101in}{0.416390in}}{\pgfqpoint{0.448700in}{0.412000in}}{\pgfqpoint{0.459750in}{0.412000in}}%
\pgfpathclose%
\pgfusepath{stroke,fill}%
\end{pgfscope}%
\begin{pgfscope}%
\pgfpathrectangle{\pgfqpoint{0.375000in}{0.330000in}}{\pgfqpoint{2.325000in}{2.310000in}}%
\pgfusepath{clip}%
\pgfsetbuttcap%
\pgfsetroundjoin%
\definecolor{currentfill}{rgb}{0.000000,0.000000,0.000000}%
\pgfsetfillcolor{currentfill}%
\pgfsetlinewidth{1.003750pt}%
\definecolor{currentstroke}{rgb}{0.000000,0.000000,0.000000}%
\pgfsetstrokecolor{currentstroke}%
\pgfsetdash{}{0pt}%
\pgfpathmoveto{\pgfqpoint{0.459750in}{0.412000in}}%
\pgfpathcurveto{\pgfqpoint{0.470800in}{0.412000in}}{\pgfqpoint{0.481399in}{0.416390in}}{\pgfqpoint{0.489213in}{0.424204in}}%
\pgfpathcurveto{\pgfqpoint{0.497026in}{0.432017in}}{\pgfqpoint{0.501417in}{0.442616in}}{\pgfqpoint{0.501417in}{0.453666in}}%
\pgfpathcurveto{\pgfqpoint{0.501417in}{0.464716in}}{\pgfqpoint{0.497026in}{0.475315in}}{\pgfqpoint{0.489213in}{0.483129in}}%
\pgfpathcurveto{\pgfqpoint{0.481399in}{0.490943in}}{\pgfqpoint{0.470800in}{0.495333in}}{\pgfqpoint{0.459750in}{0.495333in}}%
\pgfpathcurveto{\pgfqpoint{0.448700in}{0.495333in}}{\pgfqpoint{0.438101in}{0.490943in}}{\pgfqpoint{0.430287in}{0.483129in}}%
\pgfpathcurveto{\pgfqpoint{0.422474in}{0.475315in}}{\pgfqpoint{0.418083in}{0.464716in}}{\pgfqpoint{0.418083in}{0.453666in}}%
\pgfpathcurveto{\pgfqpoint{0.418083in}{0.442616in}}{\pgfqpoint{0.422474in}{0.432017in}}{\pgfqpoint{0.430287in}{0.424204in}}%
\pgfpathcurveto{\pgfqpoint{0.438101in}{0.416390in}}{\pgfqpoint{0.448700in}{0.412000in}}{\pgfqpoint{0.459750in}{0.412000in}}%
\pgfpathclose%
\pgfusepath{stroke,fill}%
\end{pgfscope}%
\begin{pgfscope}%
\pgfpathrectangle{\pgfqpoint{0.375000in}{0.330000in}}{\pgfqpoint{2.325000in}{2.310000in}}%
\pgfusepath{clip}%
\pgfsetbuttcap%
\pgfsetroundjoin%
\definecolor{currentfill}{rgb}{0.000000,0.000000,0.000000}%
\pgfsetfillcolor{currentfill}%
\pgfsetlinewidth{1.003750pt}%
\definecolor{currentstroke}{rgb}{0.000000,0.000000,0.000000}%
\pgfsetstrokecolor{currentstroke}%
\pgfsetdash{}{0pt}%
\pgfpathmoveto{\pgfqpoint{0.459750in}{1.443291in}}%
\pgfpathcurveto{\pgfqpoint{0.470800in}{1.443291in}}{\pgfqpoint{0.481399in}{1.447682in}}{\pgfqpoint{0.489213in}{1.455495in}}%
\pgfpathcurveto{\pgfqpoint{0.497026in}{1.463309in}}{\pgfqpoint{0.501417in}{1.473908in}}{\pgfqpoint{0.501417in}{1.484958in}}%
\pgfpathcurveto{\pgfqpoint{0.501417in}{1.496008in}}{\pgfqpoint{0.497026in}{1.506607in}}{\pgfqpoint{0.489213in}{1.514421in}}%
\pgfpathcurveto{\pgfqpoint{0.481399in}{1.522235in}}{\pgfqpoint{0.470800in}{1.526625in}}{\pgfqpoint{0.459750in}{1.526625in}}%
\pgfpathcurveto{\pgfqpoint{0.448700in}{1.526625in}}{\pgfqpoint{0.438101in}{1.522235in}}{\pgfqpoint{0.430287in}{1.514421in}}%
\pgfpathcurveto{\pgfqpoint{0.422474in}{1.506607in}}{\pgfqpoint{0.418083in}{1.496008in}}{\pgfqpoint{0.418083in}{1.484958in}}%
\pgfpathcurveto{\pgfqpoint{0.418083in}{1.473908in}}{\pgfqpoint{0.422474in}{1.463309in}}{\pgfqpoint{0.430287in}{1.455495in}}%
\pgfpathcurveto{\pgfqpoint{0.438101in}{1.447682in}}{\pgfqpoint{0.448700in}{1.443291in}}{\pgfqpoint{0.459750in}{1.443291in}}%
\pgfpathclose%
\pgfusepath{stroke,fill}%
\end{pgfscope}%
\begin{pgfscope}%
\pgfpathrectangle{\pgfqpoint{0.375000in}{0.330000in}}{\pgfqpoint{2.325000in}{2.310000in}}%
\pgfusepath{clip}%
\pgfsetbuttcap%
\pgfsetroundjoin%
\definecolor{currentfill}{rgb}{0.000000,0.000000,0.000000}%
\pgfsetfillcolor{currentfill}%
\pgfsetlinewidth{1.003750pt}%
\definecolor{currentstroke}{rgb}{0.000000,0.000000,0.000000}%
\pgfsetstrokecolor{currentstroke}%
\pgfsetdash{}{0pt}%
\pgfpathmoveto{\pgfqpoint{0.459750in}{0.412000in}}%
\pgfpathcurveto{\pgfqpoint{0.470800in}{0.412000in}}{\pgfqpoint{0.481399in}{0.416390in}}{\pgfqpoint{0.489213in}{0.424204in}}%
\pgfpathcurveto{\pgfqpoint{0.497026in}{0.432017in}}{\pgfqpoint{0.501417in}{0.442616in}}{\pgfqpoint{0.501417in}{0.453666in}}%
\pgfpathcurveto{\pgfqpoint{0.501417in}{0.464716in}}{\pgfqpoint{0.497026in}{0.475315in}}{\pgfqpoint{0.489213in}{0.483129in}}%
\pgfpathcurveto{\pgfqpoint{0.481399in}{0.490943in}}{\pgfqpoint{0.470800in}{0.495333in}}{\pgfqpoint{0.459750in}{0.495333in}}%
\pgfpathcurveto{\pgfqpoint{0.448700in}{0.495333in}}{\pgfqpoint{0.438101in}{0.490943in}}{\pgfqpoint{0.430287in}{0.483129in}}%
\pgfpathcurveto{\pgfqpoint{0.422474in}{0.475315in}}{\pgfqpoint{0.418083in}{0.464716in}}{\pgfqpoint{0.418083in}{0.453666in}}%
\pgfpathcurveto{\pgfqpoint{0.418083in}{0.442616in}}{\pgfqpoint{0.422474in}{0.432017in}}{\pgfqpoint{0.430287in}{0.424204in}}%
\pgfpathcurveto{\pgfqpoint{0.438101in}{0.416390in}}{\pgfqpoint{0.448700in}{0.412000in}}{\pgfqpoint{0.459750in}{0.412000in}}%
\pgfpathclose%
\pgfusepath{stroke,fill}%
\end{pgfscope}%
\begin{pgfscope}%
\pgfpathrectangle{\pgfqpoint{0.375000in}{0.330000in}}{\pgfqpoint{2.325000in}{2.310000in}}%
\pgfusepath{clip}%
\pgfsetbuttcap%
\pgfsetroundjoin%
\definecolor{currentfill}{rgb}{0.000000,0.000000,0.000000}%
\pgfsetfillcolor{currentfill}%
\pgfsetlinewidth{1.003750pt}%
\definecolor{currentstroke}{rgb}{0.000000,0.000000,0.000000}%
\pgfsetstrokecolor{currentstroke}%
\pgfsetdash{}{0pt}%
\pgfpathmoveto{\pgfqpoint{0.459750in}{0.412000in}}%
\pgfpathcurveto{\pgfqpoint{0.470800in}{0.412000in}}{\pgfqpoint{0.481399in}{0.416390in}}{\pgfqpoint{0.489213in}{0.424204in}}%
\pgfpathcurveto{\pgfqpoint{0.497026in}{0.432017in}}{\pgfqpoint{0.501417in}{0.442616in}}{\pgfqpoint{0.501417in}{0.453666in}}%
\pgfpathcurveto{\pgfqpoint{0.501417in}{0.464716in}}{\pgfqpoint{0.497026in}{0.475315in}}{\pgfqpoint{0.489213in}{0.483129in}}%
\pgfpathcurveto{\pgfqpoint{0.481399in}{0.490943in}}{\pgfqpoint{0.470800in}{0.495333in}}{\pgfqpoint{0.459750in}{0.495333in}}%
\pgfpathcurveto{\pgfqpoint{0.448700in}{0.495333in}}{\pgfqpoint{0.438101in}{0.490943in}}{\pgfqpoint{0.430287in}{0.483129in}}%
\pgfpathcurveto{\pgfqpoint{0.422474in}{0.475315in}}{\pgfqpoint{0.418083in}{0.464716in}}{\pgfqpoint{0.418083in}{0.453666in}}%
\pgfpathcurveto{\pgfqpoint{0.418083in}{0.442616in}}{\pgfqpoint{0.422474in}{0.432017in}}{\pgfqpoint{0.430287in}{0.424204in}}%
\pgfpathcurveto{\pgfqpoint{0.438101in}{0.416390in}}{\pgfqpoint{0.448700in}{0.412000in}}{\pgfqpoint{0.459750in}{0.412000in}}%
\pgfpathclose%
\pgfusepath{stroke,fill}%
\end{pgfscope}%
\begin{pgfscope}%
\pgfpathrectangle{\pgfqpoint{0.375000in}{0.330000in}}{\pgfqpoint{2.325000in}{2.310000in}}%
\pgfusepath{clip}%
\pgfsetbuttcap%
\pgfsetroundjoin%
\definecolor{currentfill}{rgb}{0.000000,0.000000,0.000000}%
\pgfsetfillcolor{currentfill}%
\pgfsetlinewidth{1.003750pt}%
\definecolor{currentstroke}{rgb}{0.000000,0.000000,0.000000}%
\pgfsetstrokecolor{currentstroke}%
\pgfsetdash{}{0pt}%
\pgfpathmoveto{\pgfqpoint{0.459750in}{0.412000in}}%
\pgfpathcurveto{\pgfqpoint{0.470800in}{0.412000in}}{\pgfqpoint{0.481399in}{0.416390in}}{\pgfqpoint{0.489213in}{0.424204in}}%
\pgfpathcurveto{\pgfqpoint{0.497026in}{0.432017in}}{\pgfqpoint{0.501417in}{0.442616in}}{\pgfqpoint{0.501417in}{0.453666in}}%
\pgfpathcurveto{\pgfqpoint{0.501417in}{0.464716in}}{\pgfqpoint{0.497026in}{0.475315in}}{\pgfqpoint{0.489213in}{0.483129in}}%
\pgfpathcurveto{\pgfqpoint{0.481399in}{0.490943in}}{\pgfqpoint{0.470800in}{0.495333in}}{\pgfqpoint{0.459750in}{0.495333in}}%
\pgfpathcurveto{\pgfqpoint{0.448700in}{0.495333in}}{\pgfqpoint{0.438101in}{0.490943in}}{\pgfqpoint{0.430287in}{0.483129in}}%
\pgfpathcurveto{\pgfqpoint{0.422474in}{0.475315in}}{\pgfqpoint{0.418083in}{0.464716in}}{\pgfqpoint{0.418083in}{0.453666in}}%
\pgfpathcurveto{\pgfqpoint{0.418083in}{0.442616in}}{\pgfqpoint{0.422474in}{0.432017in}}{\pgfqpoint{0.430287in}{0.424204in}}%
\pgfpathcurveto{\pgfqpoint{0.438101in}{0.416390in}}{\pgfqpoint{0.448700in}{0.412000in}}{\pgfqpoint{0.459750in}{0.412000in}}%
\pgfpathclose%
\pgfusepath{stroke,fill}%
\end{pgfscope}%
\begin{pgfscope}%
\pgfpathrectangle{\pgfqpoint{0.375000in}{0.330000in}}{\pgfqpoint{2.325000in}{2.310000in}}%
\pgfusepath{clip}%
\pgfsetbuttcap%
\pgfsetroundjoin%
\definecolor{currentfill}{rgb}{0.000000,0.000000,0.000000}%
\pgfsetfillcolor{currentfill}%
\pgfsetlinewidth{1.003750pt}%
\definecolor{currentstroke}{rgb}{0.000000,0.000000,0.000000}%
\pgfsetstrokecolor{currentstroke}%
\pgfsetdash{}{0pt}%
\pgfpathmoveto{\pgfqpoint{0.459750in}{1.443291in}}%
\pgfpathcurveto{\pgfqpoint{0.470800in}{1.443291in}}{\pgfqpoint{0.481399in}{1.447682in}}{\pgfqpoint{0.489213in}{1.455495in}}%
\pgfpathcurveto{\pgfqpoint{0.497026in}{1.463309in}}{\pgfqpoint{0.501417in}{1.473908in}}{\pgfqpoint{0.501417in}{1.484958in}}%
\pgfpathcurveto{\pgfqpoint{0.501417in}{1.496008in}}{\pgfqpoint{0.497026in}{1.506607in}}{\pgfqpoint{0.489213in}{1.514421in}}%
\pgfpathcurveto{\pgfqpoint{0.481399in}{1.522235in}}{\pgfqpoint{0.470800in}{1.526625in}}{\pgfqpoint{0.459750in}{1.526625in}}%
\pgfpathcurveto{\pgfqpoint{0.448700in}{1.526625in}}{\pgfqpoint{0.438101in}{1.522235in}}{\pgfqpoint{0.430287in}{1.514421in}}%
\pgfpathcurveto{\pgfqpoint{0.422474in}{1.506607in}}{\pgfqpoint{0.418083in}{1.496008in}}{\pgfqpoint{0.418083in}{1.484958in}}%
\pgfpathcurveto{\pgfqpoint{0.418083in}{1.473908in}}{\pgfqpoint{0.422474in}{1.463309in}}{\pgfqpoint{0.430287in}{1.455495in}}%
\pgfpathcurveto{\pgfqpoint{0.438101in}{1.447682in}}{\pgfqpoint{0.448700in}{1.443291in}}{\pgfqpoint{0.459750in}{1.443291in}}%
\pgfpathclose%
\pgfusepath{stroke,fill}%
\end{pgfscope}%
\begin{pgfscope}%
\pgfpathrectangle{\pgfqpoint{0.375000in}{0.330000in}}{\pgfqpoint{2.325000in}{2.310000in}}%
\pgfusepath{clip}%
\pgfsetbuttcap%
\pgfsetroundjoin%
\definecolor{currentfill}{rgb}{0.000000,0.000000,0.000000}%
\pgfsetfillcolor{currentfill}%
\pgfsetlinewidth{1.003750pt}%
\definecolor{currentstroke}{rgb}{0.000000,0.000000,0.000000}%
\pgfsetstrokecolor{currentstroke}%
\pgfsetdash{}{0pt}%
\pgfpathmoveto{\pgfqpoint{0.459750in}{0.412000in}}%
\pgfpathcurveto{\pgfqpoint{0.470800in}{0.412000in}}{\pgfqpoint{0.481399in}{0.416390in}}{\pgfqpoint{0.489213in}{0.424204in}}%
\pgfpathcurveto{\pgfqpoint{0.497026in}{0.432017in}}{\pgfqpoint{0.501417in}{0.442616in}}{\pgfqpoint{0.501417in}{0.453666in}}%
\pgfpathcurveto{\pgfqpoint{0.501417in}{0.464716in}}{\pgfqpoint{0.497026in}{0.475315in}}{\pgfqpoint{0.489213in}{0.483129in}}%
\pgfpathcurveto{\pgfqpoint{0.481399in}{0.490943in}}{\pgfqpoint{0.470800in}{0.495333in}}{\pgfqpoint{0.459750in}{0.495333in}}%
\pgfpathcurveto{\pgfqpoint{0.448700in}{0.495333in}}{\pgfqpoint{0.438101in}{0.490943in}}{\pgfqpoint{0.430287in}{0.483129in}}%
\pgfpathcurveto{\pgfqpoint{0.422474in}{0.475315in}}{\pgfqpoint{0.418083in}{0.464716in}}{\pgfqpoint{0.418083in}{0.453666in}}%
\pgfpathcurveto{\pgfqpoint{0.418083in}{0.442616in}}{\pgfqpoint{0.422474in}{0.432017in}}{\pgfqpoint{0.430287in}{0.424204in}}%
\pgfpathcurveto{\pgfqpoint{0.438101in}{0.416390in}}{\pgfqpoint{0.448700in}{0.412000in}}{\pgfqpoint{0.459750in}{0.412000in}}%
\pgfpathclose%
\pgfusepath{stroke,fill}%
\end{pgfscope}%
\begin{pgfscope}%
\pgfpathrectangle{\pgfqpoint{0.375000in}{0.330000in}}{\pgfqpoint{2.325000in}{2.310000in}}%
\pgfusepath{clip}%
\pgfsetbuttcap%
\pgfsetroundjoin%
\definecolor{currentfill}{rgb}{0.000000,0.000000,0.000000}%
\pgfsetfillcolor{currentfill}%
\pgfsetlinewidth{1.003750pt}%
\definecolor{currentstroke}{rgb}{0.000000,0.000000,0.000000}%
\pgfsetstrokecolor{currentstroke}%
\pgfsetdash{}{0pt}%
\pgfpathmoveto{\pgfqpoint{0.459750in}{1.443291in}}%
\pgfpathcurveto{\pgfqpoint{0.470800in}{1.443291in}}{\pgfqpoint{0.481399in}{1.447682in}}{\pgfqpoint{0.489213in}{1.455495in}}%
\pgfpathcurveto{\pgfqpoint{0.497026in}{1.463309in}}{\pgfqpoint{0.501417in}{1.473908in}}{\pgfqpoint{0.501417in}{1.484958in}}%
\pgfpathcurveto{\pgfqpoint{0.501417in}{1.496008in}}{\pgfqpoint{0.497026in}{1.506607in}}{\pgfqpoint{0.489213in}{1.514421in}}%
\pgfpathcurveto{\pgfqpoint{0.481399in}{1.522235in}}{\pgfqpoint{0.470800in}{1.526625in}}{\pgfqpoint{0.459750in}{1.526625in}}%
\pgfpathcurveto{\pgfqpoint{0.448700in}{1.526625in}}{\pgfqpoint{0.438101in}{1.522235in}}{\pgfqpoint{0.430287in}{1.514421in}}%
\pgfpathcurveto{\pgfqpoint{0.422474in}{1.506607in}}{\pgfqpoint{0.418083in}{1.496008in}}{\pgfqpoint{0.418083in}{1.484958in}}%
\pgfpathcurveto{\pgfqpoint{0.418083in}{1.473908in}}{\pgfqpoint{0.422474in}{1.463309in}}{\pgfqpoint{0.430287in}{1.455495in}}%
\pgfpathcurveto{\pgfqpoint{0.438101in}{1.447682in}}{\pgfqpoint{0.448700in}{1.443291in}}{\pgfqpoint{0.459750in}{1.443291in}}%
\pgfpathclose%
\pgfusepath{stroke,fill}%
\end{pgfscope}%
\begin{pgfscope}%
\pgfpathrectangle{\pgfqpoint{0.375000in}{0.330000in}}{\pgfqpoint{2.325000in}{2.310000in}}%
\pgfusepath{clip}%
\pgfsetbuttcap%
\pgfsetroundjoin%
\definecolor{currentfill}{rgb}{0.000000,0.000000,0.000000}%
\pgfsetfillcolor{currentfill}%
\pgfsetlinewidth{1.003750pt}%
\definecolor{currentstroke}{rgb}{0.000000,0.000000,0.000000}%
\pgfsetstrokecolor{currentstroke}%
\pgfsetdash{}{0pt}%
\pgfpathmoveto{\pgfqpoint{0.459750in}{0.412000in}}%
\pgfpathcurveto{\pgfqpoint{0.470800in}{0.412000in}}{\pgfqpoint{0.481399in}{0.416390in}}{\pgfqpoint{0.489213in}{0.424204in}}%
\pgfpathcurveto{\pgfqpoint{0.497026in}{0.432017in}}{\pgfqpoint{0.501417in}{0.442616in}}{\pgfqpoint{0.501417in}{0.453666in}}%
\pgfpathcurveto{\pgfqpoint{0.501417in}{0.464716in}}{\pgfqpoint{0.497026in}{0.475315in}}{\pgfqpoint{0.489213in}{0.483129in}}%
\pgfpathcurveto{\pgfqpoint{0.481399in}{0.490943in}}{\pgfqpoint{0.470800in}{0.495333in}}{\pgfqpoint{0.459750in}{0.495333in}}%
\pgfpathcurveto{\pgfqpoint{0.448700in}{0.495333in}}{\pgfqpoint{0.438101in}{0.490943in}}{\pgfqpoint{0.430287in}{0.483129in}}%
\pgfpathcurveto{\pgfqpoint{0.422474in}{0.475315in}}{\pgfqpoint{0.418083in}{0.464716in}}{\pgfqpoint{0.418083in}{0.453666in}}%
\pgfpathcurveto{\pgfqpoint{0.418083in}{0.442616in}}{\pgfqpoint{0.422474in}{0.432017in}}{\pgfqpoint{0.430287in}{0.424204in}}%
\pgfpathcurveto{\pgfqpoint{0.438101in}{0.416390in}}{\pgfqpoint{0.448700in}{0.412000in}}{\pgfqpoint{0.459750in}{0.412000in}}%
\pgfpathclose%
\pgfusepath{stroke,fill}%
\end{pgfscope}%
\begin{pgfscope}%
\pgfpathrectangle{\pgfqpoint{0.375000in}{0.330000in}}{\pgfqpoint{2.325000in}{2.310000in}}%
\pgfusepath{clip}%
\pgfsetbuttcap%
\pgfsetroundjoin%
\definecolor{currentfill}{rgb}{0.000000,0.000000,0.000000}%
\pgfsetfillcolor{currentfill}%
\pgfsetlinewidth{1.003750pt}%
\definecolor{currentstroke}{rgb}{0.000000,0.000000,0.000000}%
\pgfsetstrokecolor{currentstroke}%
\pgfsetdash{}{0pt}%
\pgfpathmoveto{\pgfqpoint{0.459750in}{0.412000in}}%
\pgfpathcurveto{\pgfqpoint{0.470800in}{0.412000in}}{\pgfqpoint{0.481399in}{0.416390in}}{\pgfqpoint{0.489213in}{0.424204in}}%
\pgfpathcurveto{\pgfqpoint{0.497026in}{0.432017in}}{\pgfqpoint{0.501417in}{0.442616in}}{\pgfqpoint{0.501417in}{0.453666in}}%
\pgfpathcurveto{\pgfqpoint{0.501417in}{0.464716in}}{\pgfqpoint{0.497026in}{0.475315in}}{\pgfqpoint{0.489213in}{0.483129in}}%
\pgfpathcurveto{\pgfqpoint{0.481399in}{0.490943in}}{\pgfqpoint{0.470800in}{0.495333in}}{\pgfqpoint{0.459750in}{0.495333in}}%
\pgfpathcurveto{\pgfqpoint{0.448700in}{0.495333in}}{\pgfqpoint{0.438101in}{0.490943in}}{\pgfqpoint{0.430287in}{0.483129in}}%
\pgfpathcurveto{\pgfqpoint{0.422474in}{0.475315in}}{\pgfqpoint{0.418083in}{0.464716in}}{\pgfqpoint{0.418083in}{0.453666in}}%
\pgfpathcurveto{\pgfqpoint{0.418083in}{0.442616in}}{\pgfqpoint{0.422474in}{0.432017in}}{\pgfqpoint{0.430287in}{0.424204in}}%
\pgfpathcurveto{\pgfqpoint{0.438101in}{0.416390in}}{\pgfqpoint{0.448700in}{0.412000in}}{\pgfqpoint{0.459750in}{0.412000in}}%
\pgfpathclose%
\pgfusepath{stroke,fill}%
\end{pgfscope}%
\begin{pgfscope}%
\pgfpathrectangle{\pgfqpoint{0.375000in}{0.330000in}}{\pgfqpoint{2.325000in}{2.310000in}}%
\pgfusepath{clip}%
\pgfsetbuttcap%
\pgfsetroundjoin%
\definecolor{currentfill}{rgb}{0.000000,0.000000,0.000000}%
\pgfsetfillcolor{currentfill}%
\pgfsetlinewidth{1.003750pt}%
\definecolor{currentstroke}{rgb}{0.000000,0.000000,0.000000}%
\pgfsetstrokecolor{currentstroke}%
\pgfsetdash{}{0pt}%
\pgfpathmoveto{\pgfqpoint{0.459750in}{0.412000in}}%
\pgfpathcurveto{\pgfqpoint{0.470800in}{0.412000in}}{\pgfqpoint{0.481399in}{0.416390in}}{\pgfqpoint{0.489213in}{0.424204in}}%
\pgfpathcurveto{\pgfqpoint{0.497026in}{0.432017in}}{\pgfqpoint{0.501417in}{0.442616in}}{\pgfqpoint{0.501417in}{0.453666in}}%
\pgfpathcurveto{\pgfqpoint{0.501417in}{0.464716in}}{\pgfqpoint{0.497026in}{0.475315in}}{\pgfqpoint{0.489213in}{0.483129in}}%
\pgfpathcurveto{\pgfqpoint{0.481399in}{0.490943in}}{\pgfqpoint{0.470800in}{0.495333in}}{\pgfqpoint{0.459750in}{0.495333in}}%
\pgfpathcurveto{\pgfqpoint{0.448700in}{0.495333in}}{\pgfqpoint{0.438101in}{0.490943in}}{\pgfqpoint{0.430287in}{0.483129in}}%
\pgfpathcurveto{\pgfqpoint{0.422474in}{0.475315in}}{\pgfqpoint{0.418083in}{0.464716in}}{\pgfqpoint{0.418083in}{0.453666in}}%
\pgfpathcurveto{\pgfqpoint{0.418083in}{0.442616in}}{\pgfqpoint{0.422474in}{0.432017in}}{\pgfqpoint{0.430287in}{0.424204in}}%
\pgfpathcurveto{\pgfqpoint{0.438101in}{0.416390in}}{\pgfqpoint{0.448700in}{0.412000in}}{\pgfqpoint{0.459750in}{0.412000in}}%
\pgfpathclose%
\pgfusepath{stroke,fill}%
\end{pgfscope}%
\begin{pgfscope}%
\pgfpathrectangle{\pgfqpoint{0.375000in}{0.330000in}}{\pgfqpoint{2.325000in}{2.310000in}}%
\pgfusepath{clip}%
\pgfsetbuttcap%
\pgfsetroundjoin%
\definecolor{currentfill}{rgb}{0.000000,0.000000,0.000000}%
\pgfsetfillcolor{currentfill}%
\pgfsetlinewidth{1.003750pt}%
\definecolor{currentstroke}{rgb}{0.000000,0.000000,0.000000}%
\pgfsetstrokecolor{currentstroke}%
\pgfsetdash{}{0pt}%
\pgfpathmoveto{\pgfqpoint{0.459750in}{0.412000in}}%
\pgfpathcurveto{\pgfqpoint{0.470800in}{0.412000in}}{\pgfqpoint{0.481399in}{0.416390in}}{\pgfqpoint{0.489213in}{0.424204in}}%
\pgfpathcurveto{\pgfqpoint{0.497026in}{0.432017in}}{\pgfqpoint{0.501417in}{0.442616in}}{\pgfqpoint{0.501417in}{0.453666in}}%
\pgfpathcurveto{\pgfqpoint{0.501417in}{0.464716in}}{\pgfqpoint{0.497026in}{0.475315in}}{\pgfqpoint{0.489213in}{0.483129in}}%
\pgfpathcurveto{\pgfqpoint{0.481399in}{0.490943in}}{\pgfqpoint{0.470800in}{0.495333in}}{\pgfqpoint{0.459750in}{0.495333in}}%
\pgfpathcurveto{\pgfqpoint{0.448700in}{0.495333in}}{\pgfqpoint{0.438101in}{0.490943in}}{\pgfqpoint{0.430287in}{0.483129in}}%
\pgfpathcurveto{\pgfqpoint{0.422474in}{0.475315in}}{\pgfqpoint{0.418083in}{0.464716in}}{\pgfqpoint{0.418083in}{0.453666in}}%
\pgfpathcurveto{\pgfqpoint{0.418083in}{0.442616in}}{\pgfqpoint{0.422474in}{0.432017in}}{\pgfqpoint{0.430287in}{0.424204in}}%
\pgfpathcurveto{\pgfqpoint{0.438101in}{0.416390in}}{\pgfqpoint{0.448700in}{0.412000in}}{\pgfqpoint{0.459750in}{0.412000in}}%
\pgfpathclose%
\pgfusepath{stroke,fill}%
\end{pgfscope}%
\begin{pgfscope}%
\pgfpathrectangle{\pgfqpoint{0.375000in}{0.330000in}}{\pgfqpoint{2.325000in}{2.310000in}}%
\pgfusepath{clip}%
\pgfsetbuttcap%
\pgfsetroundjoin%
\definecolor{currentfill}{rgb}{0.000000,0.000000,0.000000}%
\pgfsetfillcolor{currentfill}%
\pgfsetlinewidth{1.003750pt}%
\definecolor{currentstroke}{rgb}{0.000000,0.000000,0.000000}%
\pgfsetstrokecolor{currentstroke}%
\pgfsetdash{}{0pt}%
\pgfpathmoveto{\pgfqpoint{0.459750in}{0.412000in}}%
\pgfpathcurveto{\pgfqpoint{0.470800in}{0.412000in}}{\pgfqpoint{0.481399in}{0.416390in}}{\pgfqpoint{0.489213in}{0.424204in}}%
\pgfpathcurveto{\pgfqpoint{0.497026in}{0.432017in}}{\pgfqpoint{0.501417in}{0.442616in}}{\pgfqpoint{0.501417in}{0.453666in}}%
\pgfpathcurveto{\pgfqpoint{0.501417in}{0.464716in}}{\pgfqpoint{0.497026in}{0.475315in}}{\pgfqpoint{0.489213in}{0.483129in}}%
\pgfpathcurveto{\pgfqpoint{0.481399in}{0.490943in}}{\pgfqpoint{0.470800in}{0.495333in}}{\pgfqpoint{0.459750in}{0.495333in}}%
\pgfpathcurveto{\pgfqpoint{0.448700in}{0.495333in}}{\pgfqpoint{0.438101in}{0.490943in}}{\pgfqpoint{0.430287in}{0.483129in}}%
\pgfpathcurveto{\pgfqpoint{0.422474in}{0.475315in}}{\pgfqpoint{0.418083in}{0.464716in}}{\pgfqpoint{0.418083in}{0.453666in}}%
\pgfpathcurveto{\pgfqpoint{0.418083in}{0.442616in}}{\pgfqpoint{0.422474in}{0.432017in}}{\pgfqpoint{0.430287in}{0.424204in}}%
\pgfpathcurveto{\pgfqpoint{0.438101in}{0.416390in}}{\pgfqpoint{0.448700in}{0.412000in}}{\pgfqpoint{0.459750in}{0.412000in}}%
\pgfpathclose%
\pgfusepath{stroke,fill}%
\end{pgfscope}%
\begin{pgfscope}%
\pgfpathrectangle{\pgfqpoint{0.375000in}{0.330000in}}{\pgfqpoint{2.325000in}{2.310000in}}%
\pgfusepath{clip}%
\pgfsetbuttcap%
\pgfsetroundjoin%
\definecolor{currentfill}{rgb}{0.000000,0.000000,0.000000}%
\pgfsetfillcolor{currentfill}%
\pgfsetlinewidth{1.003750pt}%
\definecolor{currentstroke}{rgb}{0.000000,0.000000,0.000000}%
\pgfsetstrokecolor{currentstroke}%
\pgfsetdash{}{0pt}%
\pgfpathmoveto{\pgfqpoint{0.459750in}{0.412000in}}%
\pgfpathcurveto{\pgfqpoint{0.470800in}{0.412000in}}{\pgfqpoint{0.481399in}{0.416390in}}{\pgfqpoint{0.489213in}{0.424204in}}%
\pgfpathcurveto{\pgfqpoint{0.497026in}{0.432017in}}{\pgfqpoint{0.501417in}{0.442616in}}{\pgfqpoint{0.501417in}{0.453666in}}%
\pgfpathcurveto{\pgfqpoint{0.501417in}{0.464716in}}{\pgfqpoint{0.497026in}{0.475315in}}{\pgfqpoint{0.489213in}{0.483129in}}%
\pgfpathcurveto{\pgfqpoint{0.481399in}{0.490943in}}{\pgfqpoint{0.470800in}{0.495333in}}{\pgfqpoint{0.459750in}{0.495333in}}%
\pgfpathcurveto{\pgfqpoint{0.448700in}{0.495333in}}{\pgfqpoint{0.438101in}{0.490943in}}{\pgfqpoint{0.430287in}{0.483129in}}%
\pgfpathcurveto{\pgfqpoint{0.422474in}{0.475315in}}{\pgfqpoint{0.418083in}{0.464716in}}{\pgfqpoint{0.418083in}{0.453666in}}%
\pgfpathcurveto{\pgfqpoint{0.418083in}{0.442616in}}{\pgfqpoint{0.422474in}{0.432017in}}{\pgfqpoint{0.430287in}{0.424204in}}%
\pgfpathcurveto{\pgfqpoint{0.438101in}{0.416390in}}{\pgfqpoint{0.448700in}{0.412000in}}{\pgfqpoint{0.459750in}{0.412000in}}%
\pgfpathclose%
\pgfusepath{stroke,fill}%
\end{pgfscope}%
\begin{pgfscope}%
\pgfpathrectangle{\pgfqpoint{0.375000in}{0.330000in}}{\pgfqpoint{2.325000in}{2.310000in}}%
\pgfusepath{clip}%
\pgfsetbuttcap%
\pgfsetroundjoin%
\definecolor{currentfill}{rgb}{0.000000,0.000000,0.000000}%
\pgfsetfillcolor{currentfill}%
\pgfsetlinewidth{1.003750pt}%
\definecolor{currentstroke}{rgb}{0.000000,0.000000,0.000000}%
\pgfsetstrokecolor{currentstroke}%
\pgfsetdash{}{0pt}%
\pgfpathmoveto{\pgfqpoint{0.459750in}{0.412000in}}%
\pgfpathcurveto{\pgfqpoint{0.470800in}{0.412000in}}{\pgfqpoint{0.481399in}{0.416390in}}{\pgfqpoint{0.489213in}{0.424204in}}%
\pgfpathcurveto{\pgfqpoint{0.497026in}{0.432017in}}{\pgfqpoint{0.501417in}{0.442616in}}{\pgfqpoint{0.501417in}{0.453666in}}%
\pgfpathcurveto{\pgfqpoint{0.501417in}{0.464716in}}{\pgfqpoint{0.497026in}{0.475315in}}{\pgfqpoint{0.489213in}{0.483129in}}%
\pgfpathcurveto{\pgfqpoint{0.481399in}{0.490943in}}{\pgfqpoint{0.470800in}{0.495333in}}{\pgfqpoint{0.459750in}{0.495333in}}%
\pgfpathcurveto{\pgfqpoint{0.448700in}{0.495333in}}{\pgfqpoint{0.438101in}{0.490943in}}{\pgfqpoint{0.430287in}{0.483129in}}%
\pgfpathcurveto{\pgfqpoint{0.422474in}{0.475315in}}{\pgfqpoint{0.418083in}{0.464716in}}{\pgfqpoint{0.418083in}{0.453666in}}%
\pgfpathcurveto{\pgfqpoint{0.418083in}{0.442616in}}{\pgfqpoint{0.422474in}{0.432017in}}{\pgfqpoint{0.430287in}{0.424204in}}%
\pgfpathcurveto{\pgfqpoint{0.438101in}{0.416390in}}{\pgfqpoint{0.448700in}{0.412000in}}{\pgfqpoint{0.459750in}{0.412000in}}%
\pgfpathclose%
\pgfusepath{stroke,fill}%
\end{pgfscope}%
\begin{pgfscope}%
\pgfpathrectangle{\pgfqpoint{0.375000in}{0.330000in}}{\pgfqpoint{2.325000in}{2.310000in}}%
\pgfusepath{clip}%
\pgfsetbuttcap%
\pgfsetroundjoin%
\definecolor{currentfill}{rgb}{0.000000,0.000000,0.000000}%
\pgfsetfillcolor{currentfill}%
\pgfsetlinewidth{1.003750pt}%
\definecolor{currentstroke}{rgb}{0.000000,0.000000,0.000000}%
\pgfsetstrokecolor{currentstroke}%
\pgfsetdash{}{0pt}%
\pgfpathmoveto{\pgfqpoint{0.459750in}{0.412000in}}%
\pgfpathcurveto{\pgfqpoint{0.470800in}{0.412000in}}{\pgfqpoint{0.481399in}{0.416390in}}{\pgfqpoint{0.489213in}{0.424204in}}%
\pgfpathcurveto{\pgfqpoint{0.497026in}{0.432017in}}{\pgfqpoint{0.501417in}{0.442616in}}{\pgfqpoint{0.501417in}{0.453666in}}%
\pgfpathcurveto{\pgfqpoint{0.501417in}{0.464716in}}{\pgfqpoint{0.497026in}{0.475315in}}{\pgfqpoint{0.489213in}{0.483129in}}%
\pgfpathcurveto{\pgfqpoint{0.481399in}{0.490943in}}{\pgfqpoint{0.470800in}{0.495333in}}{\pgfqpoint{0.459750in}{0.495333in}}%
\pgfpathcurveto{\pgfqpoint{0.448700in}{0.495333in}}{\pgfqpoint{0.438101in}{0.490943in}}{\pgfqpoint{0.430287in}{0.483129in}}%
\pgfpathcurveto{\pgfqpoint{0.422474in}{0.475315in}}{\pgfqpoint{0.418083in}{0.464716in}}{\pgfqpoint{0.418083in}{0.453666in}}%
\pgfpathcurveto{\pgfqpoint{0.418083in}{0.442616in}}{\pgfqpoint{0.422474in}{0.432017in}}{\pgfqpoint{0.430287in}{0.424204in}}%
\pgfpathcurveto{\pgfqpoint{0.438101in}{0.416390in}}{\pgfqpoint{0.448700in}{0.412000in}}{\pgfqpoint{0.459750in}{0.412000in}}%
\pgfpathclose%
\pgfusepath{stroke,fill}%
\end{pgfscope}%
\begin{pgfscope}%
\pgfpathrectangle{\pgfqpoint{0.375000in}{0.330000in}}{\pgfqpoint{2.325000in}{2.310000in}}%
\pgfusepath{clip}%
\pgfsetbuttcap%
\pgfsetroundjoin%
\definecolor{currentfill}{rgb}{0.000000,0.000000,0.000000}%
\pgfsetfillcolor{currentfill}%
\pgfsetlinewidth{1.003750pt}%
\definecolor{currentstroke}{rgb}{0.000000,0.000000,0.000000}%
\pgfsetstrokecolor{currentstroke}%
\pgfsetdash{}{0pt}%
\pgfpathmoveto{\pgfqpoint{0.459750in}{0.412000in}}%
\pgfpathcurveto{\pgfqpoint{0.470800in}{0.412000in}}{\pgfqpoint{0.481399in}{0.416390in}}{\pgfqpoint{0.489213in}{0.424204in}}%
\pgfpathcurveto{\pgfqpoint{0.497026in}{0.432017in}}{\pgfqpoint{0.501417in}{0.442616in}}{\pgfqpoint{0.501417in}{0.453666in}}%
\pgfpathcurveto{\pgfqpoint{0.501417in}{0.464716in}}{\pgfqpoint{0.497026in}{0.475315in}}{\pgfqpoint{0.489213in}{0.483129in}}%
\pgfpathcurveto{\pgfqpoint{0.481399in}{0.490943in}}{\pgfqpoint{0.470800in}{0.495333in}}{\pgfqpoint{0.459750in}{0.495333in}}%
\pgfpathcurveto{\pgfqpoint{0.448700in}{0.495333in}}{\pgfqpoint{0.438101in}{0.490943in}}{\pgfqpoint{0.430287in}{0.483129in}}%
\pgfpathcurveto{\pgfqpoint{0.422474in}{0.475315in}}{\pgfqpoint{0.418083in}{0.464716in}}{\pgfqpoint{0.418083in}{0.453666in}}%
\pgfpathcurveto{\pgfqpoint{0.418083in}{0.442616in}}{\pgfqpoint{0.422474in}{0.432017in}}{\pgfqpoint{0.430287in}{0.424204in}}%
\pgfpathcurveto{\pgfqpoint{0.438101in}{0.416390in}}{\pgfqpoint{0.448700in}{0.412000in}}{\pgfqpoint{0.459750in}{0.412000in}}%
\pgfpathclose%
\pgfusepath{stroke,fill}%
\end{pgfscope}%
\begin{pgfscope}%
\pgfpathrectangle{\pgfqpoint{0.375000in}{0.330000in}}{\pgfqpoint{2.325000in}{2.310000in}}%
\pgfusepath{clip}%
\pgfsetbuttcap%
\pgfsetroundjoin%
\definecolor{currentfill}{rgb}{0.000000,0.000000,0.000000}%
\pgfsetfillcolor{currentfill}%
\pgfsetlinewidth{1.003750pt}%
\definecolor{currentstroke}{rgb}{0.000000,0.000000,0.000000}%
\pgfsetstrokecolor{currentstroke}%
\pgfsetdash{}{0pt}%
\pgfpathmoveto{\pgfqpoint{0.459750in}{0.412000in}}%
\pgfpathcurveto{\pgfqpoint{0.470800in}{0.412000in}}{\pgfqpoint{0.481399in}{0.416390in}}{\pgfqpoint{0.489213in}{0.424204in}}%
\pgfpathcurveto{\pgfqpoint{0.497026in}{0.432017in}}{\pgfqpoint{0.501417in}{0.442616in}}{\pgfqpoint{0.501417in}{0.453666in}}%
\pgfpathcurveto{\pgfqpoint{0.501417in}{0.464716in}}{\pgfqpoint{0.497026in}{0.475315in}}{\pgfqpoint{0.489213in}{0.483129in}}%
\pgfpathcurveto{\pgfqpoint{0.481399in}{0.490943in}}{\pgfqpoint{0.470800in}{0.495333in}}{\pgfqpoint{0.459750in}{0.495333in}}%
\pgfpathcurveto{\pgfqpoint{0.448700in}{0.495333in}}{\pgfqpoint{0.438101in}{0.490943in}}{\pgfqpoint{0.430287in}{0.483129in}}%
\pgfpathcurveto{\pgfqpoint{0.422474in}{0.475315in}}{\pgfqpoint{0.418083in}{0.464716in}}{\pgfqpoint{0.418083in}{0.453666in}}%
\pgfpathcurveto{\pgfqpoint{0.418083in}{0.442616in}}{\pgfqpoint{0.422474in}{0.432017in}}{\pgfqpoint{0.430287in}{0.424204in}}%
\pgfpathcurveto{\pgfqpoint{0.438101in}{0.416390in}}{\pgfqpoint{0.448700in}{0.412000in}}{\pgfqpoint{0.459750in}{0.412000in}}%
\pgfpathclose%
\pgfusepath{stroke,fill}%
\end{pgfscope}%
\begin{pgfscope}%
\pgfpathrectangle{\pgfqpoint{0.375000in}{0.330000in}}{\pgfqpoint{2.325000in}{2.310000in}}%
\pgfusepath{clip}%
\pgfsetbuttcap%
\pgfsetroundjoin%
\definecolor{currentfill}{rgb}{0.000000,0.000000,0.000000}%
\pgfsetfillcolor{currentfill}%
\pgfsetlinewidth{1.003750pt}%
\definecolor{currentstroke}{rgb}{0.000000,0.000000,0.000000}%
\pgfsetstrokecolor{currentstroke}%
\pgfsetdash{}{0pt}%
\pgfpathmoveto{\pgfqpoint{0.459750in}{0.412000in}}%
\pgfpathcurveto{\pgfqpoint{0.470800in}{0.412000in}}{\pgfqpoint{0.481399in}{0.416390in}}{\pgfqpoint{0.489213in}{0.424204in}}%
\pgfpathcurveto{\pgfqpoint{0.497026in}{0.432017in}}{\pgfqpoint{0.501417in}{0.442616in}}{\pgfqpoint{0.501417in}{0.453666in}}%
\pgfpathcurveto{\pgfqpoint{0.501417in}{0.464716in}}{\pgfqpoint{0.497026in}{0.475315in}}{\pgfqpoint{0.489213in}{0.483129in}}%
\pgfpathcurveto{\pgfqpoint{0.481399in}{0.490943in}}{\pgfqpoint{0.470800in}{0.495333in}}{\pgfqpoint{0.459750in}{0.495333in}}%
\pgfpathcurveto{\pgfqpoint{0.448700in}{0.495333in}}{\pgfqpoint{0.438101in}{0.490943in}}{\pgfqpoint{0.430287in}{0.483129in}}%
\pgfpathcurveto{\pgfqpoint{0.422474in}{0.475315in}}{\pgfqpoint{0.418083in}{0.464716in}}{\pgfqpoint{0.418083in}{0.453666in}}%
\pgfpathcurveto{\pgfqpoint{0.418083in}{0.442616in}}{\pgfqpoint{0.422474in}{0.432017in}}{\pgfqpoint{0.430287in}{0.424204in}}%
\pgfpathcurveto{\pgfqpoint{0.438101in}{0.416390in}}{\pgfqpoint{0.448700in}{0.412000in}}{\pgfqpoint{0.459750in}{0.412000in}}%
\pgfpathclose%
\pgfusepath{stroke,fill}%
\end{pgfscope}%
\begin{pgfscope}%
\pgfpathrectangle{\pgfqpoint{0.375000in}{0.330000in}}{\pgfqpoint{2.325000in}{2.310000in}}%
\pgfusepath{clip}%
\pgfsetbuttcap%
\pgfsetroundjoin%
\definecolor{currentfill}{rgb}{0.000000,0.000000,0.000000}%
\pgfsetfillcolor{currentfill}%
\pgfsetlinewidth{1.003750pt}%
\definecolor{currentstroke}{rgb}{0.000000,0.000000,0.000000}%
\pgfsetstrokecolor{currentstroke}%
\pgfsetdash{}{0pt}%
\pgfpathmoveto{\pgfqpoint{1.019812in}{1.443291in}}%
\pgfpathcurveto{\pgfqpoint{1.030863in}{1.443291in}}{\pgfqpoint{1.041462in}{1.447682in}}{\pgfqpoint{1.049275in}{1.455495in}}%
\pgfpathcurveto{\pgfqpoint{1.057089in}{1.463309in}}{\pgfqpoint{1.061479in}{1.473908in}}{\pgfqpoint{1.061479in}{1.484958in}}%
\pgfpathcurveto{\pgfqpoint{1.061479in}{1.496008in}}{\pgfqpoint{1.057089in}{1.506607in}}{\pgfqpoint{1.049275in}{1.514421in}}%
\pgfpathcurveto{\pgfqpoint{1.041462in}{1.522235in}}{\pgfqpoint{1.030863in}{1.526625in}}{\pgfqpoint{1.019812in}{1.526625in}}%
\pgfpathcurveto{\pgfqpoint{1.008762in}{1.526625in}}{\pgfqpoint{0.998163in}{1.522235in}}{\pgfqpoint{0.990350in}{1.514421in}}%
\pgfpathcurveto{\pgfqpoint{0.982536in}{1.506607in}}{\pgfqpoint{0.978146in}{1.496008in}}{\pgfqpoint{0.978146in}{1.484958in}}%
\pgfpathcurveto{\pgfqpoint{0.978146in}{1.473908in}}{\pgfqpoint{0.982536in}{1.463309in}}{\pgfqpoint{0.990350in}{1.455495in}}%
\pgfpathcurveto{\pgfqpoint{0.998163in}{1.447682in}}{\pgfqpoint{1.008762in}{1.443291in}}{\pgfqpoint{1.019812in}{1.443291in}}%
\pgfpathclose%
\pgfusepath{stroke,fill}%
\end{pgfscope}%
\begin{pgfscope}%
\pgfpathrectangle{\pgfqpoint{0.375000in}{0.330000in}}{\pgfqpoint{2.325000in}{2.310000in}}%
\pgfusepath{clip}%
\pgfsetbuttcap%
\pgfsetroundjoin%
\definecolor{currentfill}{rgb}{0.000000,0.000000,0.000000}%
\pgfsetfillcolor{currentfill}%
\pgfsetlinewidth{1.003750pt}%
\definecolor{currentstroke}{rgb}{0.000000,0.000000,0.000000}%
\pgfsetstrokecolor{currentstroke}%
\pgfsetdash{}{0pt}%
\pgfpathmoveto{\pgfqpoint{1.019812in}{1.443291in}}%
\pgfpathcurveto{\pgfqpoint{1.030863in}{1.443291in}}{\pgfqpoint{1.041462in}{1.447682in}}{\pgfqpoint{1.049275in}{1.455495in}}%
\pgfpathcurveto{\pgfqpoint{1.057089in}{1.463309in}}{\pgfqpoint{1.061479in}{1.473908in}}{\pgfqpoint{1.061479in}{1.484958in}}%
\pgfpathcurveto{\pgfqpoint{1.061479in}{1.496008in}}{\pgfqpoint{1.057089in}{1.506607in}}{\pgfqpoint{1.049275in}{1.514421in}}%
\pgfpathcurveto{\pgfqpoint{1.041462in}{1.522235in}}{\pgfqpoint{1.030863in}{1.526625in}}{\pgfqpoint{1.019812in}{1.526625in}}%
\pgfpathcurveto{\pgfqpoint{1.008762in}{1.526625in}}{\pgfqpoint{0.998163in}{1.522235in}}{\pgfqpoint{0.990350in}{1.514421in}}%
\pgfpathcurveto{\pgfqpoint{0.982536in}{1.506607in}}{\pgfqpoint{0.978146in}{1.496008in}}{\pgfqpoint{0.978146in}{1.484958in}}%
\pgfpathcurveto{\pgfqpoint{0.978146in}{1.473908in}}{\pgfqpoint{0.982536in}{1.463309in}}{\pgfqpoint{0.990350in}{1.455495in}}%
\pgfpathcurveto{\pgfqpoint{0.998163in}{1.447682in}}{\pgfqpoint{1.008762in}{1.443291in}}{\pgfqpoint{1.019812in}{1.443291in}}%
\pgfpathclose%
\pgfusepath{stroke,fill}%
\end{pgfscope}%
\begin{pgfscope}%
\pgfpathrectangle{\pgfqpoint{0.375000in}{0.330000in}}{\pgfqpoint{2.325000in}{2.310000in}}%
\pgfusepath{clip}%
\pgfsetbuttcap%
\pgfsetroundjoin%
\definecolor{currentfill}{rgb}{0.000000,0.000000,0.000000}%
\pgfsetfillcolor{currentfill}%
\pgfsetlinewidth{1.003750pt}%
\definecolor{currentstroke}{rgb}{0.000000,0.000000,0.000000}%
\pgfsetstrokecolor{currentstroke}%
\pgfsetdash{}{0pt}%
\pgfpathmoveto{\pgfqpoint{1.019812in}{1.443291in}}%
\pgfpathcurveto{\pgfqpoint{1.030863in}{1.443291in}}{\pgfqpoint{1.041462in}{1.447682in}}{\pgfqpoint{1.049275in}{1.455495in}}%
\pgfpathcurveto{\pgfqpoint{1.057089in}{1.463309in}}{\pgfqpoint{1.061479in}{1.473908in}}{\pgfqpoint{1.061479in}{1.484958in}}%
\pgfpathcurveto{\pgfqpoint{1.061479in}{1.496008in}}{\pgfqpoint{1.057089in}{1.506607in}}{\pgfqpoint{1.049275in}{1.514421in}}%
\pgfpathcurveto{\pgfqpoint{1.041462in}{1.522235in}}{\pgfqpoint{1.030863in}{1.526625in}}{\pgfqpoint{1.019812in}{1.526625in}}%
\pgfpathcurveto{\pgfqpoint{1.008762in}{1.526625in}}{\pgfqpoint{0.998163in}{1.522235in}}{\pgfqpoint{0.990350in}{1.514421in}}%
\pgfpathcurveto{\pgfqpoint{0.982536in}{1.506607in}}{\pgfqpoint{0.978146in}{1.496008in}}{\pgfqpoint{0.978146in}{1.484958in}}%
\pgfpathcurveto{\pgfqpoint{0.978146in}{1.473908in}}{\pgfqpoint{0.982536in}{1.463309in}}{\pgfqpoint{0.990350in}{1.455495in}}%
\pgfpathcurveto{\pgfqpoint{0.998163in}{1.447682in}}{\pgfqpoint{1.008762in}{1.443291in}}{\pgfqpoint{1.019812in}{1.443291in}}%
\pgfpathclose%
\pgfusepath{stroke,fill}%
\end{pgfscope}%
\begin{pgfscope}%
\pgfpathrectangle{\pgfqpoint{0.375000in}{0.330000in}}{\pgfqpoint{2.325000in}{2.310000in}}%
\pgfusepath{clip}%
\pgfsetbuttcap%
\pgfsetroundjoin%
\definecolor{currentfill}{rgb}{0.000000,0.000000,0.000000}%
\pgfsetfillcolor{currentfill}%
\pgfsetlinewidth{1.003750pt}%
\definecolor{currentstroke}{rgb}{0.000000,0.000000,0.000000}%
\pgfsetstrokecolor{currentstroke}%
\pgfsetdash{}{0pt}%
\pgfpathmoveto{\pgfqpoint{1.019812in}{1.443291in}}%
\pgfpathcurveto{\pgfqpoint{1.030863in}{1.443291in}}{\pgfqpoint{1.041462in}{1.447682in}}{\pgfqpoint{1.049275in}{1.455495in}}%
\pgfpathcurveto{\pgfqpoint{1.057089in}{1.463309in}}{\pgfqpoint{1.061479in}{1.473908in}}{\pgfqpoint{1.061479in}{1.484958in}}%
\pgfpathcurveto{\pgfqpoint{1.061479in}{1.496008in}}{\pgfqpoint{1.057089in}{1.506607in}}{\pgfqpoint{1.049275in}{1.514421in}}%
\pgfpathcurveto{\pgfqpoint{1.041462in}{1.522235in}}{\pgfqpoint{1.030863in}{1.526625in}}{\pgfqpoint{1.019812in}{1.526625in}}%
\pgfpathcurveto{\pgfqpoint{1.008762in}{1.526625in}}{\pgfqpoint{0.998163in}{1.522235in}}{\pgfqpoint{0.990350in}{1.514421in}}%
\pgfpathcurveto{\pgfqpoint{0.982536in}{1.506607in}}{\pgfqpoint{0.978146in}{1.496008in}}{\pgfqpoint{0.978146in}{1.484958in}}%
\pgfpathcurveto{\pgfqpoint{0.978146in}{1.473908in}}{\pgfqpoint{0.982536in}{1.463309in}}{\pgfqpoint{0.990350in}{1.455495in}}%
\pgfpathcurveto{\pgfqpoint{0.998163in}{1.447682in}}{\pgfqpoint{1.008762in}{1.443291in}}{\pgfqpoint{1.019812in}{1.443291in}}%
\pgfpathclose%
\pgfusepath{stroke,fill}%
\end{pgfscope}%
\begin{pgfscope}%
\pgfpathrectangle{\pgfqpoint{0.375000in}{0.330000in}}{\pgfqpoint{2.325000in}{2.310000in}}%
\pgfusepath{clip}%
\pgfsetbuttcap%
\pgfsetroundjoin%
\definecolor{currentfill}{rgb}{0.000000,0.000000,0.000000}%
\pgfsetfillcolor{currentfill}%
\pgfsetlinewidth{1.003750pt}%
\definecolor{currentstroke}{rgb}{0.000000,0.000000,0.000000}%
\pgfsetstrokecolor{currentstroke}%
\pgfsetdash{}{0pt}%
\pgfpathmoveto{\pgfqpoint{1.019812in}{1.443291in}}%
\pgfpathcurveto{\pgfqpoint{1.030863in}{1.443291in}}{\pgfqpoint{1.041462in}{1.447682in}}{\pgfqpoint{1.049275in}{1.455495in}}%
\pgfpathcurveto{\pgfqpoint{1.057089in}{1.463309in}}{\pgfqpoint{1.061479in}{1.473908in}}{\pgfqpoint{1.061479in}{1.484958in}}%
\pgfpathcurveto{\pgfqpoint{1.061479in}{1.496008in}}{\pgfqpoint{1.057089in}{1.506607in}}{\pgfqpoint{1.049275in}{1.514421in}}%
\pgfpathcurveto{\pgfqpoint{1.041462in}{1.522235in}}{\pgfqpoint{1.030863in}{1.526625in}}{\pgfqpoint{1.019812in}{1.526625in}}%
\pgfpathcurveto{\pgfqpoint{1.008762in}{1.526625in}}{\pgfqpoint{0.998163in}{1.522235in}}{\pgfqpoint{0.990350in}{1.514421in}}%
\pgfpathcurveto{\pgfqpoint{0.982536in}{1.506607in}}{\pgfqpoint{0.978146in}{1.496008in}}{\pgfqpoint{0.978146in}{1.484958in}}%
\pgfpathcurveto{\pgfqpoint{0.978146in}{1.473908in}}{\pgfqpoint{0.982536in}{1.463309in}}{\pgfqpoint{0.990350in}{1.455495in}}%
\pgfpathcurveto{\pgfqpoint{0.998163in}{1.447682in}}{\pgfqpoint{1.008762in}{1.443291in}}{\pgfqpoint{1.019812in}{1.443291in}}%
\pgfpathclose%
\pgfusepath{stroke,fill}%
\end{pgfscope}%
\begin{pgfscope}%
\pgfpathrectangle{\pgfqpoint{0.375000in}{0.330000in}}{\pgfqpoint{2.325000in}{2.310000in}}%
\pgfusepath{clip}%
\pgfsetbuttcap%
\pgfsetroundjoin%
\definecolor{currentfill}{rgb}{0.000000,0.000000,0.000000}%
\pgfsetfillcolor{currentfill}%
\pgfsetlinewidth{1.003750pt}%
\definecolor{currentstroke}{rgb}{0.000000,0.000000,0.000000}%
\pgfsetstrokecolor{currentstroke}%
\pgfsetdash{}{0pt}%
\pgfpathmoveto{\pgfqpoint{1.019812in}{1.443291in}}%
\pgfpathcurveto{\pgfqpoint{1.030863in}{1.443291in}}{\pgfqpoint{1.041462in}{1.447682in}}{\pgfqpoint{1.049275in}{1.455495in}}%
\pgfpathcurveto{\pgfqpoint{1.057089in}{1.463309in}}{\pgfqpoint{1.061479in}{1.473908in}}{\pgfqpoint{1.061479in}{1.484958in}}%
\pgfpathcurveto{\pgfqpoint{1.061479in}{1.496008in}}{\pgfqpoint{1.057089in}{1.506607in}}{\pgfqpoint{1.049275in}{1.514421in}}%
\pgfpathcurveto{\pgfqpoint{1.041462in}{1.522235in}}{\pgfqpoint{1.030863in}{1.526625in}}{\pgfqpoint{1.019812in}{1.526625in}}%
\pgfpathcurveto{\pgfqpoint{1.008762in}{1.526625in}}{\pgfqpoint{0.998163in}{1.522235in}}{\pgfqpoint{0.990350in}{1.514421in}}%
\pgfpathcurveto{\pgfqpoint{0.982536in}{1.506607in}}{\pgfqpoint{0.978146in}{1.496008in}}{\pgfqpoint{0.978146in}{1.484958in}}%
\pgfpathcurveto{\pgfqpoint{0.978146in}{1.473908in}}{\pgfqpoint{0.982536in}{1.463309in}}{\pgfqpoint{0.990350in}{1.455495in}}%
\pgfpathcurveto{\pgfqpoint{0.998163in}{1.447682in}}{\pgfqpoint{1.008762in}{1.443291in}}{\pgfqpoint{1.019812in}{1.443291in}}%
\pgfpathclose%
\pgfusepath{stroke,fill}%
\end{pgfscope}%
\begin{pgfscope}%
\pgfpathrectangle{\pgfqpoint{0.375000in}{0.330000in}}{\pgfqpoint{2.325000in}{2.310000in}}%
\pgfusepath{clip}%
\pgfsetbuttcap%
\pgfsetroundjoin%
\definecolor{currentfill}{rgb}{0.000000,0.000000,0.000000}%
\pgfsetfillcolor{currentfill}%
\pgfsetlinewidth{1.003750pt}%
\definecolor{currentstroke}{rgb}{0.000000,0.000000,0.000000}%
\pgfsetstrokecolor{currentstroke}%
\pgfsetdash{}{0pt}%
\pgfpathmoveto{\pgfqpoint{1.019812in}{1.443291in}}%
\pgfpathcurveto{\pgfqpoint{1.030863in}{1.443291in}}{\pgfqpoint{1.041462in}{1.447682in}}{\pgfqpoint{1.049275in}{1.455495in}}%
\pgfpathcurveto{\pgfqpoint{1.057089in}{1.463309in}}{\pgfqpoint{1.061479in}{1.473908in}}{\pgfqpoint{1.061479in}{1.484958in}}%
\pgfpathcurveto{\pgfqpoint{1.061479in}{1.496008in}}{\pgfqpoint{1.057089in}{1.506607in}}{\pgfqpoint{1.049275in}{1.514421in}}%
\pgfpathcurveto{\pgfqpoint{1.041462in}{1.522235in}}{\pgfqpoint{1.030863in}{1.526625in}}{\pgfqpoint{1.019812in}{1.526625in}}%
\pgfpathcurveto{\pgfqpoint{1.008762in}{1.526625in}}{\pgfqpoint{0.998163in}{1.522235in}}{\pgfqpoint{0.990350in}{1.514421in}}%
\pgfpathcurveto{\pgfqpoint{0.982536in}{1.506607in}}{\pgfqpoint{0.978146in}{1.496008in}}{\pgfqpoint{0.978146in}{1.484958in}}%
\pgfpathcurveto{\pgfqpoint{0.978146in}{1.473908in}}{\pgfqpoint{0.982536in}{1.463309in}}{\pgfqpoint{0.990350in}{1.455495in}}%
\pgfpathcurveto{\pgfqpoint{0.998163in}{1.447682in}}{\pgfqpoint{1.008762in}{1.443291in}}{\pgfqpoint{1.019812in}{1.443291in}}%
\pgfpathclose%
\pgfusepath{stroke,fill}%
\end{pgfscope}%
\begin{pgfscope}%
\pgfpathrectangle{\pgfqpoint{0.375000in}{0.330000in}}{\pgfqpoint{2.325000in}{2.310000in}}%
\pgfusepath{clip}%
\pgfsetbuttcap%
\pgfsetroundjoin%
\definecolor{currentfill}{rgb}{0.000000,0.000000,0.000000}%
\pgfsetfillcolor{currentfill}%
\pgfsetlinewidth{1.003750pt}%
\definecolor{currentstroke}{rgb}{0.000000,0.000000,0.000000}%
\pgfsetstrokecolor{currentstroke}%
\pgfsetdash{}{0pt}%
\pgfpathmoveto{\pgfqpoint{1.019812in}{1.443291in}}%
\pgfpathcurveto{\pgfqpoint{1.030863in}{1.443291in}}{\pgfqpoint{1.041462in}{1.447682in}}{\pgfqpoint{1.049275in}{1.455495in}}%
\pgfpathcurveto{\pgfqpoint{1.057089in}{1.463309in}}{\pgfqpoint{1.061479in}{1.473908in}}{\pgfqpoint{1.061479in}{1.484958in}}%
\pgfpathcurveto{\pgfqpoint{1.061479in}{1.496008in}}{\pgfqpoint{1.057089in}{1.506607in}}{\pgfqpoint{1.049275in}{1.514421in}}%
\pgfpathcurveto{\pgfqpoint{1.041462in}{1.522235in}}{\pgfqpoint{1.030863in}{1.526625in}}{\pgfqpoint{1.019812in}{1.526625in}}%
\pgfpathcurveto{\pgfqpoint{1.008762in}{1.526625in}}{\pgfqpoint{0.998163in}{1.522235in}}{\pgfqpoint{0.990350in}{1.514421in}}%
\pgfpathcurveto{\pgfqpoint{0.982536in}{1.506607in}}{\pgfqpoint{0.978146in}{1.496008in}}{\pgfqpoint{0.978146in}{1.484958in}}%
\pgfpathcurveto{\pgfqpoint{0.978146in}{1.473908in}}{\pgfqpoint{0.982536in}{1.463309in}}{\pgfqpoint{0.990350in}{1.455495in}}%
\pgfpathcurveto{\pgfqpoint{0.998163in}{1.447682in}}{\pgfqpoint{1.008762in}{1.443291in}}{\pgfqpoint{1.019812in}{1.443291in}}%
\pgfpathclose%
\pgfusepath{stroke,fill}%
\end{pgfscope}%
\begin{pgfscope}%
\pgfpathrectangle{\pgfqpoint{0.375000in}{0.330000in}}{\pgfqpoint{2.325000in}{2.310000in}}%
\pgfusepath{clip}%
\pgfsetbuttcap%
\pgfsetroundjoin%
\definecolor{currentfill}{rgb}{0.000000,0.000000,0.000000}%
\pgfsetfillcolor{currentfill}%
\pgfsetlinewidth{1.003750pt}%
\definecolor{currentstroke}{rgb}{0.000000,0.000000,0.000000}%
\pgfsetstrokecolor{currentstroke}%
\pgfsetdash{}{0pt}%
\pgfpathmoveto{\pgfqpoint{1.019812in}{1.443291in}}%
\pgfpathcurveto{\pgfqpoint{1.030863in}{1.443291in}}{\pgfqpoint{1.041462in}{1.447682in}}{\pgfqpoint{1.049275in}{1.455495in}}%
\pgfpathcurveto{\pgfqpoint{1.057089in}{1.463309in}}{\pgfqpoint{1.061479in}{1.473908in}}{\pgfqpoint{1.061479in}{1.484958in}}%
\pgfpathcurveto{\pgfqpoint{1.061479in}{1.496008in}}{\pgfqpoint{1.057089in}{1.506607in}}{\pgfqpoint{1.049275in}{1.514421in}}%
\pgfpathcurveto{\pgfqpoint{1.041462in}{1.522235in}}{\pgfqpoint{1.030863in}{1.526625in}}{\pgfqpoint{1.019812in}{1.526625in}}%
\pgfpathcurveto{\pgfqpoint{1.008762in}{1.526625in}}{\pgfqpoint{0.998163in}{1.522235in}}{\pgfqpoint{0.990350in}{1.514421in}}%
\pgfpathcurveto{\pgfqpoint{0.982536in}{1.506607in}}{\pgfqpoint{0.978146in}{1.496008in}}{\pgfqpoint{0.978146in}{1.484958in}}%
\pgfpathcurveto{\pgfqpoint{0.978146in}{1.473908in}}{\pgfqpoint{0.982536in}{1.463309in}}{\pgfqpoint{0.990350in}{1.455495in}}%
\pgfpathcurveto{\pgfqpoint{0.998163in}{1.447682in}}{\pgfqpoint{1.008762in}{1.443291in}}{\pgfqpoint{1.019812in}{1.443291in}}%
\pgfpathclose%
\pgfusepath{stroke,fill}%
\end{pgfscope}%
\begin{pgfscope}%
\pgfpathrectangle{\pgfqpoint{0.375000in}{0.330000in}}{\pgfqpoint{2.325000in}{2.310000in}}%
\pgfusepath{clip}%
\pgfsetbuttcap%
\pgfsetroundjoin%
\definecolor{currentfill}{rgb}{0.000000,0.000000,0.000000}%
\pgfsetfillcolor{currentfill}%
\pgfsetlinewidth{1.003750pt}%
\definecolor{currentstroke}{rgb}{0.000000,0.000000,0.000000}%
\pgfsetstrokecolor{currentstroke}%
\pgfsetdash{}{0pt}%
\pgfpathmoveto{\pgfqpoint{1.019812in}{1.443291in}}%
\pgfpathcurveto{\pgfqpoint{1.030863in}{1.443291in}}{\pgfqpoint{1.041462in}{1.447682in}}{\pgfqpoint{1.049275in}{1.455495in}}%
\pgfpathcurveto{\pgfqpoint{1.057089in}{1.463309in}}{\pgfqpoint{1.061479in}{1.473908in}}{\pgfqpoint{1.061479in}{1.484958in}}%
\pgfpathcurveto{\pgfqpoint{1.061479in}{1.496008in}}{\pgfqpoint{1.057089in}{1.506607in}}{\pgfqpoint{1.049275in}{1.514421in}}%
\pgfpathcurveto{\pgfqpoint{1.041462in}{1.522235in}}{\pgfqpoint{1.030863in}{1.526625in}}{\pgfqpoint{1.019812in}{1.526625in}}%
\pgfpathcurveto{\pgfqpoint{1.008762in}{1.526625in}}{\pgfqpoint{0.998163in}{1.522235in}}{\pgfqpoint{0.990350in}{1.514421in}}%
\pgfpathcurveto{\pgfqpoint{0.982536in}{1.506607in}}{\pgfqpoint{0.978146in}{1.496008in}}{\pgfqpoint{0.978146in}{1.484958in}}%
\pgfpathcurveto{\pgfqpoint{0.978146in}{1.473908in}}{\pgfqpoint{0.982536in}{1.463309in}}{\pgfqpoint{0.990350in}{1.455495in}}%
\pgfpathcurveto{\pgfqpoint{0.998163in}{1.447682in}}{\pgfqpoint{1.008762in}{1.443291in}}{\pgfqpoint{1.019812in}{1.443291in}}%
\pgfpathclose%
\pgfusepath{stroke,fill}%
\end{pgfscope}%
\begin{pgfscope}%
\pgfpathrectangle{\pgfqpoint{0.375000in}{0.330000in}}{\pgfqpoint{2.325000in}{2.310000in}}%
\pgfusepath{clip}%
\pgfsetbuttcap%
\pgfsetroundjoin%
\definecolor{currentfill}{rgb}{0.000000,0.000000,0.000000}%
\pgfsetfillcolor{currentfill}%
\pgfsetlinewidth{1.003750pt}%
\definecolor{currentstroke}{rgb}{0.000000,0.000000,0.000000}%
\pgfsetstrokecolor{currentstroke}%
\pgfsetdash{}{0pt}%
\pgfpathmoveto{\pgfqpoint{1.019812in}{2.474583in}}%
\pgfpathcurveto{\pgfqpoint{1.030863in}{2.474583in}}{\pgfqpoint{1.041462in}{2.478974in}}{\pgfqpoint{1.049275in}{2.486787in}}%
\pgfpathcurveto{\pgfqpoint{1.057089in}{2.494601in}}{\pgfqpoint{1.061479in}{2.505200in}}{\pgfqpoint{1.061479in}{2.516250in}}%
\pgfpathcurveto{\pgfqpoint{1.061479in}{2.527300in}}{\pgfqpoint{1.057089in}{2.537899in}}{\pgfqpoint{1.049275in}{2.545713in}}%
\pgfpathcurveto{\pgfqpoint{1.041462in}{2.553526in}}{\pgfqpoint{1.030863in}{2.557917in}}{\pgfqpoint{1.019812in}{2.557917in}}%
\pgfpathcurveto{\pgfqpoint{1.008762in}{2.557917in}}{\pgfqpoint{0.998163in}{2.553526in}}{\pgfqpoint{0.990350in}{2.545713in}}%
\pgfpathcurveto{\pgfqpoint{0.982536in}{2.537899in}}{\pgfqpoint{0.978146in}{2.527300in}}{\pgfqpoint{0.978146in}{2.516250in}}%
\pgfpathcurveto{\pgfqpoint{0.978146in}{2.505200in}}{\pgfqpoint{0.982536in}{2.494601in}}{\pgfqpoint{0.990350in}{2.486787in}}%
\pgfpathcurveto{\pgfqpoint{0.998163in}{2.478974in}}{\pgfqpoint{1.008762in}{2.474583in}}{\pgfqpoint{1.019812in}{2.474583in}}%
\pgfpathclose%
\pgfusepath{stroke,fill}%
\end{pgfscope}%
\begin{pgfscope}%
\pgfpathrectangle{\pgfqpoint{0.375000in}{0.330000in}}{\pgfqpoint{2.325000in}{2.310000in}}%
\pgfusepath{clip}%
\pgfsetbuttcap%
\pgfsetroundjoin%
\definecolor{currentfill}{rgb}{0.000000,0.000000,0.000000}%
\pgfsetfillcolor{currentfill}%
\pgfsetlinewidth{1.003750pt}%
\definecolor{currentstroke}{rgb}{0.000000,0.000000,0.000000}%
\pgfsetstrokecolor{currentstroke}%
\pgfsetdash{}{0pt}%
\pgfpathmoveto{\pgfqpoint{1.019812in}{1.443291in}}%
\pgfpathcurveto{\pgfqpoint{1.030863in}{1.443291in}}{\pgfqpoint{1.041462in}{1.447682in}}{\pgfqpoint{1.049275in}{1.455495in}}%
\pgfpathcurveto{\pgfqpoint{1.057089in}{1.463309in}}{\pgfqpoint{1.061479in}{1.473908in}}{\pgfqpoint{1.061479in}{1.484958in}}%
\pgfpathcurveto{\pgfqpoint{1.061479in}{1.496008in}}{\pgfqpoint{1.057089in}{1.506607in}}{\pgfqpoint{1.049275in}{1.514421in}}%
\pgfpathcurveto{\pgfqpoint{1.041462in}{1.522235in}}{\pgfqpoint{1.030863in}{1.526625in}}{\pgfqpoint{1.019812in}{1.526625in}}%
\pgfpathcurveto{\pgfqpoint{1.008762in}{1.526625in}}{\pgfqpoint{0.998163in}{1.522235in}}{\pgfqpoint{0.990350in}{1.514421in}}%
\pgfpathcurveto{\pgfqpoint{0.982536in}{1.506607in}}{\pgfqpoint{0.978146in}{1.496008in}}{\pgfqpoint{0.978146in}{1.484958in}}%
\pgfpathcurveto{\pgfqpoint{0.978146in}{1.473908in}}{\pgfqpoint{0.982536in}{1.463309in}}{\pgfqpoint{0.990350in}{1.455495in}}%
\pgfpathcurveto{\pgfqpoint{0.998163in}{1.447682in}}{\pgfqpoint{1.008762in}{1.443291in}}{\pgfqpoint{1.019812in}{1.443291in}}%
\pgfpathclose%
\pgfusepath{stroke,fill}%
\end{pgfscope}%
\begin{pgfscope}%
\pgfpathrectangle{\pgfqpoint{0.375000in}{0.330000in}}{\pgfqpoint{2.325000in}{2.310000in}}%
\pgfusepath{clip}%
\pgfsetbuttcap%
\pgfsetroundjoin%
\definecolor{currentfill}{rgb}{0.000000,0.000000,0.000000}%
\pgfsetfillcolor{currentfill}%
\pgfsetlinewidth{1.003750pt}%
\definecolor{currentstroke}{rgb}{0.000000,0.000000,0.000000}%
\pgfsetstrokecolor{currentstroke}%
\pgfsetdash{}{0pt}%
\pgfpathmoveto{\pgfqpoint{1.019812in}{1.443291in}}%
\pgfpathcurveto{\pgfqpoint{1.030863in}{1.443291in}}{\pgfqpoint{1.041462in}{1.447682in}}{\pgfqpoint{1.049275in}{1.455495in}}%
\pgfpathcurveto{\pgfqpoint{1.057089in}{1.463309in}}{\pgfqpoint{1.061479in}{1.473908in}}{\pgfqpoint{1.061479in}{1.484958in}}%
\pgfpathcurveto{\pgfqpoint{1.061479in}{1.496008in}}{\pgfqpoint{1.057089in}{1.506607in}}{\pgfqpoint{1.049275in}{1.514421in}}%
\pgfpathcurveto{\pgfqpoint{1.041462in}{1.522235in}}{\pgfqpoint{1.030863in}{1.526625in}}{\pgfqpoint{1.019812in}{1.526625in}}%
\pgfpathcurveto{\pgfqpoint{1.008762in}{1.526625in}}{\pgfqpoint{0.998163in}{1.522235in}}{\pgfqpoint{0.990350in}{1.514421in}}%
\pgfpathcurveto{\pgfqpoint{0.982536in}{1.506607in}}{\pgfqpoint{0.978146in}{1.496008in}}{\pgfqpoint{0.978146in}{1.484958in}}%
\pgfpathcurveto{\pgfqpoint{0.978146in}{1.473908in}}{\pgfqpoint{0.982536in}{1.463309in}}{\pgfqpoint{0.990350in}{1.455495in}}%
\pgfpathcurveto{\pgfqpoint{0.998163in}{1.447682in}}{\pgfqpoint{1.008762in}{1.443291in}}{\pgfqpoint{1.019812in}{1.443291in}}%
\pgfpathclose%
\pgfusepath{stroke,fill}%
\end{pgfscope}%
\begin{pgfscope}%
\pgfpathrectangle{\pgfqpoint{0.375000in}{0.330000in}}{\pgfqpoint{2.325000in}{2.310000in}}%
\pgfusepath{clip}%
\pgfsetbuttcap%
\pgfsetroundjoin%
\definecolor{currentfill}{rgb}{0.000000,0.000000,0.000000}%
\pgfsetfillcolor{currentfill}%
\pgfsetlinewidth{1.003750pt}%
\definecolor{currentstroke}{rgb}{0.000000,0.000000,0.000000}%
\pgfsetstrokecolor{currentstroke}%
\pgfsetdash{}{0pt}%
\pgfpathmoveto{\pgfqpoint{1.019812in}{1.443291in}}%
\pgfpathcurveto{\pgfqpoint{1.030863in}{1.443291in}}{\pgfqpoint{1.041462in}{1.447682in}}{\pgfqpoint{1.049275in}{1.455495in}}%
\pgfpathcurveto{\pgfqpoint{1.057089in}{1.463309in}}{\pgfqpoint{1.061479in}{1.473908in}}{\pgfqpoint{1.061479in}{1.484958in}}%
\pgfpathcurveto{\pgfqpoint{1.061479in}{1.496008in}}{\pgfqpoint{1.057089in}{1.506607in}}{\pgfqpoint{1.049275in}{1.514421in}}%
\pgfpathcurveto{\pgfqpoint{1.041462in}{1.522235in}}{\pgfqpoint{1.030863in}{1.526625in}}{\pgfqpoint{1.019812in}{1.526625in}}%
\pgfpathcurveto{\pgfqpoint{1.008762in}{1.526625in}}{\pgfqpoint{0.998163in}{1.522235in}}{\pgfqpoint{0.990350in}{1.514421in}}%
\pgfpathcurveto{\pgfqpoint{0.982536in}{1.506607in}}{\pgfqpoint{0.978146in}{1.496008in}}{\pgfqpoint{0.978146in}{1.484958in}}%
\pgfpathcurveto{\pgfqpoint{0.978146in}{1.473908in}}{\pgfqpoint{0.982536in}{1.463309in}}{\pgfqpoint{0.990350in}{1.455495in}}%
\pgfpathcurveto{\pgfqpoint{0.998163in}{1.447682in}}{\pgfqpoint{1.008762in}{1.443291in}}{\pgfqpoint{1.019812in}{1.443291in}}%
\pgfpathclose%
\pgfusepath{stroke,fill}%
\end{pgfscope}%
\begin{pgfscope}%
\pgfpathrectangle{\pgfqpoint{0.375000in}{0.330000in}}{\pgfqpoint{2.325000in}{2.310000in}}%
\pgfusepath{clip}%
\pgfsetbuttcap%
\pgfsetroundjoin%
\definecolor{currentfill}{rgb}{0.000000,0.000000,0.000000}%
\pgfsetfillcolor{currentfill}%
\pgfsetlinewidth{1.003750pt}%
\definecolor{currentstroke}{rgb}{0.000000,0.000000,0.000000}%
\pgfsetstrokecolor{currentstroke}%
\pgfsetdash{}{0pt}%
\pgfpathmoveto{\pgfqpoint{1.019812in}{1.443291in}}%
\pgfpathcurveto{\pgfqpoint{1.030863in}{1.443291in}}{\pgfqpoint{1.041462in}{1.447682in}}{\pgfqpoint{1.049275in}{1.455495in}}%
\pgfpathcurveto{\pgfqpoint{1.057089in}{1.463309in}}{\pgfqpoint{1.061479in}{1.473908in}}{\pgfqpoint{1.061479in}{1.484958in}}%
\pgfpathcurveto{\pgfqpoint{1.061479in}{1.496008in}}{\pgfqpoint{1.057089in}{1.506607in}}{\pgfqpoint{1.049275in}{1.514421in}}%
\pgfpathcurveto{\pgfqpoint{1.041462in}{1.522235in}}{\pgfqpoint{1.030863in}{1.526625in}}{\pgfqpoint{1.019812in}{1.526625in}}%
\pgfpathcurveto{\pgfqpoint{1.008762in}{1.526625in}}{\pgfqpoint{0.998163in}{1.522235in}}{\pgfqpoint{0.990350in}{1.514421in}}%
\pgfpathcurveto{\pgfqpoint{0.982536in}{1.506607in}}{\pgfqpoint{0.978146in}{1.496008in}}{\pgfqpoint{0.978146in}{1.484958in}}%
\pgfpathcurveto{\pgfqpoint{0.978146in}{1.473908in}}{\pgfqpoint{0.982536in}{1.463309in}}{\pgfqpoint{0.990350in}{1.455495in}}%
\pgfpathcurveto{\pgfqpoint{0.998163in}{1.447682in}}{\pgfqpoint{1.008762in}{1.443291in}}{\pgfqpoint{1.019812in}{1.443291in}}%
\pgfpathclose%
\pgfusepath{stroke,fill}%
\end{pgfscope}%
\begin{pgfscope}%
\pgfpathrectangle{\pgfqpoint{0.375000in}{0.330000in}}{\pgfqpoint{2.325000in}{2.310000in}}%
\pgfusepath{clip}%
\pgfsetbuttcap%
\pgfsetroundjoin%
\definecolor{currentfill}{rgb}{0.000000,0.000000,0.000000}%
\pgfsetfillcolor{currentfill}%
\pgfsetlinewidth{1.003750pt}%
\definecolor{currentstroke}{rgb}{0.000000,0.000000,0.000000}%
\pgfsetstrokecolor{currentstroke}%
\pgfsetdash{}{0pt}%
\pgfpathmoveto{\pgfqpoint{1.019812in}{1.443291in}}%
\pgfpathcurveto{\pgfqpoint{1.030863in}{1.443291in}}{\pgfqpoint{1.041462in}{1.447682in}}{\pgfqpoint{1.049275in}{1.455495in}}%
\pgfpathcurveto{\pgfqpoint{1.057089in}{1.463309in}}{\pgfqpoint{1.061479in}{1.473908in}}{\pgfqpoint{1.061479in}{1.484958in}}%
\pgfpathcurveto{\pgfqpoint{1.061479in}{1.496008in}}{\pgfqpoint{1.057089in}{1.506607in}}{\pgfqpoint{1.049275in}{1.514421in}}%
\pgfpathcurveto{\pgfqpoint{1.041462in}{1.522235in}}{\pgfqpoint{1.030863in}{1.526625in}}{\pgfqpoint{1.019812in}{1.526625in}}%
\pgfpathcurveto{\pgfqpoint{1.008762in}{1.526625in}}{\pgfqpoint{0.998163in}{1.522235in}}{\pgfqpoint{0.990350in}{1.514421in}}%
\pgfpathcurveto{\pgfqpoint{0.982536in}{1.506607in}}{\pgfqpoint{0.978146in}{1.496008in}}{\pgfqpoint{0.978146in}{1.484958in}}%
\pgfpathcurveto{\pgfqpoint{0.978146in}{1.473908in}}{\pgfqpoint{0.982536in}{1.463309in}}{\pgfqpoint{0.990350in}{1.455495in}}%
\pgfpathcurveto{\pgfqpoint{0.998163in}{1.447682in}}{\pgfqpoint{1.008762in}{1.443291in}}{\pgfqpoint{1.019812in}{1.443291in}}%
\pgfpathclose%
\pgfusepath{stroke,fill}%
\end{pgfscope}%
\begin{pgfscope}%
\pgfpathrectangle{\pgfqpoint{0.375000in}{0.330000in}}{\pgfqpoint{2.325000in}{2.310000in}}%
\pgfusepath{clip}%
\pgfsetbuttcap%
\pgfsetroundjoin%
\definecolor{currentfill}{rgb}{0.000000,0.000000,0.000000}%
\pgfsetfillcolor{currentfill}%
\pgfsetlinewidth{1.003750pt}%
\definecolor{currentstroke}{rgb}{0.000000,0.000000,0.000000}%
\pgfsetstrokecolor{currentstroke}%
\pgfsetdash{}{0pt}%
\pgfpathmoveto{\pgfqpoint{1.019812in}{1.443291in}}%
\pgfpathcurveto{\pgfqpoint{1.030863in}{1.443291in}}{\pgfqpoint{1.041462in}{1.447682in}}{\pgfqpoint{1.049275in}{1.455495in}}%
\pgfpathcurveto{\pgfqpoint{1.057089in}{1.463309in}}{\pgfqpoint{1.061479in}{1.473908in}}{\pgfqpoint{1.061479in}{1.484958in}}%
\pgfpathcurveto{\pgfqpoint{1.061479in}{1.496008in}}{\pgfqpoint{1.057089in}{1.506607in}}{\pgfqpoint{1.049275in}{1.514421in}}%
\pgfpathcurveto{\pgfqpoint{1.041462in}{1.522235in}}{\pgfqpoint{1.030863in}{1.526625in}}{\pgfqpoint{1.019812in}{1.526625in}}%
\pgfpathcurveto{\pgfqpoint{1.008762in}{1.526625in}}{\pgfqpoint{0.998163in}{1.522235in}}{\pgfqpoint{0.990350in}{1.514421in}}%
\pgfpathcurveto{\pgfqpoint{0.982536in}{1.506607in}}{\pgfqpoint{0.978146in}{1.496008in}}{\pgfqpoint{0.978146in}{1.484958in}}%
\pgfpathcurveto{\pgfqpoint{0.978146in}{1.473908in}}{\pgfqpoint{0.982536in}{1.463309in}}{\pgfqpoint{0.990350in}{1.455495in}}%
\pgfpathcurveto{\pgfqpoint{0.998163in}{1.447682in}}{\pgfqpoint{1.008762in}{1.443291in}}{\pgfqpoint{1.019812in}{1.443291in}}%
\pgfpathclose%
\pgfusepath{stroke,fill}%
\end{pgfscope}%
\begin{pgfscope}%
\pgfpathrectangle{\pgfqpoint{0.375000in}{0.330000in}}{\pgfqpoint{2.325000in}{2.310000in}}%
\pgfusepath{clip}%
\pgfsetbuttcap%
\pgfsetroundjoin%
\definecolor{currentfill}{rgb}{0.000000,0.000000,0.000000}%
\pgfsetfillcolor{currentfill}%
\pgfsetlinewidth{1.003750pt}%
\definecolor{currentstroke}{rgb}{0.000000,0.000000,0.000000}%
\pgfsetstrokecolor{currentstroke}%
\pgfsetdash{}{0pt}%
\pgfpathmoveto{\pgfqpoint{1.019812in}{1.443291in}}%
\pgfpathcurveto{\pgfqpoint{1.030863in}{1.443291in}}{\pgfqpoint{1.041462in}{1.447682in}}{\pgfqpoint{1.049275in}{1.455495in}}%
\pgfpathcurveto{\pgfqpoint{1.057089in}{1.463309in}}{\pgfqpoint{1.061479in}{1.473908in}}{\pgfqpoint{1.061479in}{1.484958in}}%
\pgfpathcurveto{\pgfqpoint{1.061479in}{1.496008in}}{\pgfqpoint{1.057089in}{1.506607in}}{\pgfqpoint{1.049275in}{1.514421in}}%
\pgfpathcurveto{\pgfqpoint{1.041462in}{1.522235in}}{\pgfqpoint{1.030863in}{1.526625in}}{\pgfqpoint{1.019812in}{1.526625in}}%
\pgfpathcurveto{\pgfqpoint{1.008762in}{1.526625in}}{\pgfqpoint{0.998163in}{1.522235in}}{\pgfqpoint{0.990350in}{1.514421in}}%
\pgfpathcurveto{\pgfqpoint{0.982536in}{1.506607in}}{\pgfqpoint{0.978146in}{1.496008in}}{\pgfqpoint{0.978146in}{1.484958in}}%
\pgfpathcurveto{\pgfqpoint{0.978146in}{1.473908in}}{\pgfqpoint{0.982536in}{1.463309in}}{\pgfqpoint{0.990350in}{1.455495in}}%
\pgfpathcurveto{\pgfqpoint{0.998163in}{1.447682in}}{\pgfqpoint{1.008762in}{1.443291in}}{\pgfqpoint{1.019812in}{1.443291in}}%
\pgfpathclose%
\pgfusepath{stroke,fill}%
\end{pgfscope}%
\begin{pgfscope}%
\pgfpathrectangle{\pgfqpoint{0.375000in}{0.330000in}}{\pgfqpoint{2.325000in}{2.310000in}}%
\pgfusepath{clip}%
\pgfsetbuttcap%
\pgfsetroundjoin%
\definecolor{currentfill}{rgb}{0.000000,0.000000,0.000000}%
\pgfsetfillcolor{currentfill}%
\pgfsetlinewidth{1.003750pt}%
\definecolor{currentstroke}{rgb}{0.000000,0.000000,0.000000}%
\pgfsetstrokecolor{currentstroke}%
\pgfsetdash{}{0pt}%
\pgfpathmoveto{\pgfqpoint{1.019812in}{1.443291in}}%
\pgfpathcurveto{\pgfqpoint{1.030863in}{1.443291in}}{\pgfqpoint{1.041462in}{1.447682in}}{\pgfqpoint{1.049275in}{1.455495in}}%
\pgfpathcurveto{\pgfqpoint{1.057089in}{1.463309in}}{\pgfqpoint{1.061479in}{1.473908in}}{\pgfqpoint{1.061479in}{1.484958in}}%
\pgfpathcurveto{\pgfqpoint{1.061479in}{1.496008in}}{\pgfqpoint{1.057089in}{1.506607in}}{\pgfqpoint{1.049275in}{1.514421in}}%
\pgfpathcurveto{\pgfqpoint{1.041462in}{1.522235in}}{\pgfqpoint{1.030863in}{1.526625in}}{\pgfqpoint{1.019812in}{1.526625in}}%
\pgfpathcurveto{\pgfqpoint{1.008762in}{1.526625in}}{\pgfqpoint{0.998163in}{1.522235in}}{\pgfqpoint{0.990350in}{1.514421in}}%
\pgfpathcurveto{\pgfqpoint{0.982536in}{1.506607in}}{\pgfqpoint{0.978146in}{1.496008in}}{\pgfqpoint{0.978146in}{1.484958in}}%
\pgfpathcurveto{\pgfqpoint{0.978146in}{1.473908in}}{\pgfqpoint{0.982536in}{1.463309in}}{\pgfqpoint{0.990350in}{1.455495in}}%
\pgfpathcurveto{\pgfqpoint{0.998163in}{1.447682in}}{\pgfqpoint{1.008762in}{1.443291in}}{\pgfqpoint{1.019812in}{1.443291in}}%
\pgfpathclose%
\pgfusepath{stroke,fill}%
\end{pgfscope}%
\begin{pgfscope}%
\pgfpathrectangle{\pgfqpoint{0.375000in}{0.330000in}}{\pgfqpoint{2.325000in}{2.310000in}}%
\pgfusepath{clip}%
\pgfsetbuttcap%
\pgfsetroundjoin%
\definecolor{currentfill}{rgb}{0.000000,0.000000,0.000000}%
\pgfsetfillcolor{currentfill}%
\pgfsetlinewidth{1.003750pt}%
\definecolor{currentstroke}{rgb}{0.000000,0.000000,0.000000}%
\pgfsetstrokecolor{currentstroke}%
\pgfsetdash{}{0pt}%
\pgfpathmoveto{\pgfqpoint{1.019812in}{1.443291in}}%
\pgfpathcurveto{\pgfqpoint{1.030863in}{1.443291in}}{\pgfqpoint{1.041462in}{1.447682in}}{\pgfqpoint{1.049275in}{1.455495in}}%
\pgfpathcurveto{\pgfqpoint{1.057089in}{1.463309in}}{\pgfqpoint{1.061479in}{1.473908in}}{\pgfqpoint{1.061479in}{1.484958in}}%
\pgfpathcurveto{\pgfqpoint{1.061479in}{1.496008in}}{\pgfqpoint{1.057089in}{1.506607in}}{\pgfqpoint{1.049275in}{1.514421in}}%
\pgfpathcurveto{\pgfqpoint{1.041462in}{1.522235in}}{\pgfqpoint{1.030863in}{1.526625in}}{\pgfqpoint{1.019812in}{1.526625in}}%
\pgfpathcurveto{\pgfqpoint{1.008762in}{1.526625in}}{\pgfqpoint{0.998163in}{1.522235in}}{\pgfqpoint{0.990350in}{1.514421in}}%
\pgfpathcurveto{\pgfqpoint{0.982536in}{1.506607in}}{\pgfqpoint{0.978146in}{1.496008in}}{\pgfqpoint{0.978146in}{1.484958in}}%
\pgfpathcurveto{\pgfqpoint{0.978146in}{1.473908in}}{\pgfqpoint{0.982536in}{1.463309in}}{\pgfqpoint{0.990350in}{1.455495in}}%
\pgfpathcurveto{\pgfqpoint{0.998163in}{1.447682in}}{\pgfqpoint{1.008762in}{1.443291in}}{\pgfqpoint{1.019812in}{1.443291in}}%
\pgfpathclose%
\pgfusepath{stroke,fill}%
\end{pgfscope}%
\begin{pgfscope}%
\pgfpathrectangle{\pgfqpoint{0.375000in}{0.330000in}}{\pgfqpoint{2.325000in}{2.310000in}}%
\pgfusepath{clip}%
\pgfsetbuttcap%
\pgfsetroundjoin%
\definecolor{currentfill}{rgb}{0.000000,0.000000,0.000000}%
\pgfsetfillcolor{currentfill}%
\pgfsetlinewidth{1.003750pt}%
\definecolor{currentstroke}{rgb}{0.000000,0.000000,0.000000}%
\pgfsetstrokecolor{currentstroke}%
\pgfsetdash{}{0pt}%
\pgfpathmoveto{\pgfqpoint{1.019812in}{1.443291in}}%
\pgfpathcurveto{\pgfqpoint{1.030863in}{1.443291in}}{\pgfqpoint{1.041462in}{1.447682in}}{\pgfqpoint{1.049275in}{1.455495in}}%
\pgfpathcurveto{\pgfqpoint{1.057089in}{1.463309in}}{\pgfqpoint{1.061479in}{1.473908in}}{\pgfqpoint{1.061479in}{1.484958in}}%
\pgfpathcurveto{\pgfqpoint{1.061479in}{1.496008in}}{\pgfqpoint{1.057089in}{1.506607in}}{\pgfqpoint{1.049275in}{1.514421in}}%
\pgfpathcurveto{\pgfqpoint{1.041462in}{1.522235in}}{\pgfqpoint{1.030863in}{1.526625in}}{\pgfqpoint{1.019812in}{1.526625in}}%
\pgfpathcurveto{\pgfqpoint{1.008762in}{1.526625in}}{\pgfqpoint{0.998163in}{1.522235in}}{\pgfqpoint{0.990350in}{1.514421in}}%
\pgfpathcurveto{\pgfqpoint{0.982536in}{1.506607in}}{\pgfqpoint{0.978146in}{1.496008in}}{\pgfqpoint{0.978146in}{1.484958in}}%
\pgfpathcurveto{\pgfqpoint{0.978146in}{1.473908in}}{\pgfqpoint{0.982536in}{1.463309in}}{\pgfqpoint{0.990350in}{1.455495in}}%
\pgfpathcurveto{\pgfqpoint{0.998163in}{1.447682in}}{\pgfqpoint{1.008762in}{1.443291in}}{\pgfqpoint{1.019812in}{1.443291in}}%
\pgfpathclose%
\pgfusepath{stroke,fill}%
\end{pgfscope}%
\begin{pgfscope}%
\pgfpathrectangle{\pgfqpoint{0.375000in}{0.330000in}}{\pgfqpoint{2.325000in}{2.310000in}}%
\pgfusepath{clip}%
\pgfsetbuttcap%
\pgfsetroundjoin%
\definecolor{currentfill}{rgb}{0.000000,0.000000,0.000000}%
\pgfsetfillcolor{currentfill}%
\pgfsetlinewidth{1.003750pt}%
\definecolor{currentstroke}{rgb}{0.000000,0.000000,0.000000}%
\pgfsetstrokecolor{currentstroke}%
\pgfsetdash{}{0pt}%
\pgfpathmoveto{\pgfqpoint{1.019812in}{1.443291in}}%
\pgfpathcurveto{\pgfqpoint{1.030863in}{1.443291in}}{\pgfqpoint{1.041462in}{1.447682in}}{\pgfqpoint{1.049275in}{1.455495in}}%
\pgfpathcurveto{\pgfqpoint{1.057089in}{1.463309in}}{\pgfqpoint{1.061479in}{1.473908in}}{\pgfqpoint{1.061479in}{1.484958in}}%
\pgfpathcurveto{\pgfqpoint{1.061479in}{1.496008in}}{\pgfqpoint{1.057089in}{1.506607in}}{\pgfqpoint{1.049275in}{1.514421in}}%
\pgfpathcurveto{\pgfqpoint{1.041462in}{1.522235in}}{\pgfqpoint{1.030863in}{1.526625in}}{\pgfqpoint{1.019812in}{1.526625in}}%
\pgfpathcurveto{\pgfqpoint{1.008762in}{1.526625in}}{\pgfqpoint{0.998163in}{1.522235in}}{\pgfqpoint{0.990350in}{1.514421in}}%
\pgfpathcurveto{\pgfqpoint{0.982536in}{1.506607in}}{\pgfqpoint{0.978146in}{1.496008in}}{\pgfqpoint{0.978146in}{1.484958in}}%
\pgfpathcurveto{\pgfqpoint{0.978146in}{1.473908in}}{\pgfqpoint{0.982536in}{1.463309in}}{\pgfqpoint{0.990350in}{1.455495in}}%
\pgfpathcurveto{\pgfqpoint{0.998163in}{1.447682in}}{\pgfqpoint{1.008762in}{1.443291in}}{\pgfqpoint{1.019812in}{1.443291in}}%
\pgfpathclose%
\pgfusepath{stroke,fill}%
\end{pgfscope}%
\begin{pgfscope}%
\pgfpathrectangle{\pgfqpoint{0.375000in}{0.330000in}}{\pgfqpoint{2.325000in}{2.310000in}}%
\pgfusepath{clip}%
\pgfsetbuttcap%
\pgfsetroundjoin%
\definecolor{currentfill}{rgb}{0.000000,0.000000,0.000000}%
\pgfsetfillcolor{currentfill}%
\pgfsetlinewidth{1.003750pt}%
\definecolor{currentstroke}{rgb}{0.000000,0.000000,0.000000}%
\pgfsetstrokecolor{currentstroke}%
\pgfsetdash{}{0pt}%
\pgfpathmoveto{\pgfqpoint{1.019812in}{1.443291in}}%
\pgfpathcurveto{\pgfqpoint{1.030863in}{1.443291in}}{\pgfqpoint{1.041462in}{1.447682in}}{\pgfqpoint{1.049275in}{1.455495in}}%
\pgfpathcurveto{\pgfqpoint{1.057089in}{1.463309in}}{\pgfqpoint{1.061479in}{1.473908in}}{\pgfqpoint{1.061479in}{1.484958in}}%
\pgfpathcurveto{\pgfqpoint{1.061479in}{1.496008in}}{\pgfqpoint{1.057089in}{1.506607in}}{\pgfqpoint{1.049275in}{1.514421in}}%
\pgfpathcurveto{\pgfqpoint{1.041462in}{1.522235in}}{\pgfqpoint{1.030863in}{1.526625in}}{\pgfqpoint{1.019812in}{1.526625in}}%
\pgfpathcurveto{\pgfqpoint{1.008762in}{1.526625in}}{\pgfqpoint{0.998163in}{1.522235in}}{\pgfqpoint{0.990350in}{1.514421in}}%
\pgfpathcurveto{\pgfqpoint{0.982536in}{1.506607in}}{\pgfqpoint{0.978146in}{1.496008in}}{\pgfqpoint{0.978146in}{1.484958in}}%
\pgfpathcurveto{\pgfqpoint{0.978146in}{1.473908in}}{\pgfqpoint{0.982536in}{1.463309in}}{\pgfqpoint{0.990350in}{1.455495in}}%
\pgfpathcurveto{\pgfqpoint{0.998163in}{1.447682in}}{\pgfqpoint{1.008762in}{1.443291in}}{\pgfqpoint{1.019812in}{1.443291in}}%
\pgfpathclose%
\pgfusepath{stroke,fill}%
\end{pgfscope}%
\begin{pgfscope}%
\pgfpathrectangle{\pgfqpoint{0.375000in}{0.330000in}}{\pgfqpoint{2.325000in}{2.310000in}}%
\pgfusepath{clip}%
\pgfsetbuttcap%
\pgfsetroundjoin%
\definecolor{currentfill}{rgb}{0.000000,0.000000,0.000000}%
\pgfsetfillcolor{currentfill}%
\pgfsetlinewidth{1.003750pt}%
\definecolor{currentstroke}{rgb}{0.000000,0.000000,0.000000}%
\pgfsetstrokecolor{currentstroke}%
\pgfsetdash{}{0pt}%
\pgfpathmoveto{\pgfqpoint{1.019812in}{1.443291in}}%
\pgfpathcurveto{\pgfqpoint{1.030863in}{1.443291in}}{\pgfqpoint{1.041462in}{1.447682in}}{\pgfqpoint{1.049275in}{1.455495in}}%
\pgfpathcurveto{\pgfqpoint{1.057089in}{1.463309in}}{\pgfqpoint{1.061479in}{1.473908in}}{\pgfqpoint{1.061479in}{1.484958in}}%
\pgfpathcurveto{\pgfqpoint{1.061479in}{1.496008in}}{\pgfqpoint{1.057089in}{1.506607in}}{\pgfqpoint{1.049275in}{1.514421in}}%
\pgfpathcurveto{\pgfqpoint{1.041462in}{1.522235in}}{\pgfqpoint{1.030863in}{1.526625in}}{\pgfqpoint{1.019812in}{1.526625in}}%
\pgfpathcurveto{\pgfqpoint{1.008762in}{1.526625in}}{\pgfqpoint{0.998163in}{1.522235in}}{\pgfqpoint{0.990350in}{1.514421in}}%
\pgfpathcurveto{\pgfqpoint{0.982536in}{1.506607in}}{\pgfqpoint{0.978146in}{1.496008in}}{\pgfqpoint{0.978146in}{1.484958in}}%
\pgfpathcurveto{\pgfqpoint{0.978146in}{1.473908in}}{\pgfqpoint{0.982536in}{1.463309in}}{\pgfqpoint{0.990350in}{1.455495in}}%
\pgfpathcurveto{\pgfqpoint{0.998163in}{1.447682in}}{\pgfqpoint{1.008762in}{1.443291in}}{\pgfqpoint{1.019812in}{1.443291in}}%
\pgfpathclose%
\pgfusepath{stroke,fill}%
\end{pgfscope}%
\begin{pgfscope}%
\pgfpathrectangle{\pgfqpoint{0.375000in}{0.330000in}}{\pgfqpoint{2.325000in}{2.310000in}}%
\pgfusepath{clip}%
\pgfsetbuttcap%
\pgfsetroundjoin%
\definecolor{currentfill}{rgb}{0.000000,0.000000,0.000000}%
\pgfsetfillcolor{currentfill}%
\pgfsetlinewidth{1.003750pt}%
\definecolor{currentstroke}{rgb}{0.000000,0.000000,0.000000}%
\pgfsetstrokecolor{currentstroke}%
\pgfsetdash{}{0pt}%
\pgfpathmoveto{\pgfqpoint{1.019812in}{1.443291in}}%
\pgfpathcurveto{\pgfqpoint{1.030863in}{1.443291in}}{\pgfqpoint{1.041462in}{1.447682in}}{\pgfqpoint{1.049275in}{1.455495in}}%
\pgfpathcurveto{\pgfqpoint{1.057089in}{1.463309in}}{\pgfqpoint{1.061479in}{1.473908in}}{\pgfqpoint{1.061479in}{1.484958in}}%
\pgfpathcurveto{\pgfqpoint{1.061479in}{1.496008in}}{\pgfqpoint{1.057089in}{1.506607in}}{\pgfqpoint{1.049275in}{1.514421in}}%
\pgfpathcurveto{\pgfqpoint{1.041462in}{1.522235in}}{\pgfqpoint{1.030863in}{1.526625in}}{\pgfqpoint{1.019812in}{1.526625in}}%
\pgfpathcurveto{\pgfqpoint{1.008762in}{1.526625in}}{\pgfqpoint{0.998163in}{1.522235in}}{\pgfqpoint{0.990350in}{1.514421in}}%
\pgfpathcurveto{\pgfqpoint{0.982536in}{1.506607in}}{\pgfqpoint{0.978146in}{1.496008in}}{\pgfqpoint{0.978146in}{1.484958in}}%
\pgfpathcurveto{\pgfqpoint{0.978146in}{1.473908in}}{\pgfqpoint{0.982536in}{1.463309in}}{\pgfqpoint{0.990350in}{1.455495in}}%
\pgfpathcurveto{\pgfqpoint{0.998163in}{1.447682in}}{\pgfqpoint{1.008762in}{1.443291in}}{\pgfqpoint{1.019812in}{1.443291in}}%
\pgfpathclose%
\pgfusepath{stroke,fill}%
\end{pgfscope}%
\begin{pgfscope}%
\pgfpathrectangle{\pgfqpoint{0.375000in}{0.330000in}}{\pgfqpoint{2.325000in}{2.310000in}}%
\pgfusepath{clip}%
\pgfsetbuttcap%
\pgfsetroundjoin%
\definecolor{currentfill}{rgb}{0.000000,0.000000,0.000000}%
\pgfsetfillcolor{currentfill}%
\pgfsetlinewidth{1.003750pt}%
\definecolor{currentstroke}{rgb}{0.000000,0.000000,0.000000}%
\pgfsetstrokecolor{currentstroke}%
\pgfsetdash{}{0pt}%
\pgfpathmoveto{\pgfqpoint{1.019812in}{1.443291in}}%
\pgfpathcurveto{\pgfqpoint{1.030863in}{1.443291in}}{\pgfqpoint{1.041462in}{1.447682in}}{\pgfqpoint{1.049275in}{1.455495in}}%
\pgfpathcurveto{\pgfqpoint{1.057089in}{1.463309in}}{\pgfqpoint{1.061479in}{1.473908in}}{\pgfqpoint{1.061479in}{1.484958in}}%
\pgfpathcurveto{\pgfqpoint{1.061479in}{1.496008in}}{\pgfqpoint{1.057089in}{1.506607in}}{\pgfqpoint{1.049275in}{1.514421in}}%
\pgfpathcurveto{\pgfqpoint{1.041462in}{1.522235in}}{\pgfqpoint{1.030863in}{1.526625in}}{\pgfqpoint{1.019812in}{1.526625in}}%
\pgfpathcurveto{\pgfqpoint{1.008762in}{1.526625in}}{\pgfqpoint{0.998163in}{1.522235in}}{\pgfqpoint{0.990350in}{1.514421in}}%
\pgfpathcurveto{\pgfqpoint{0.982536in}{1.506607in}}{\pgfqpoint{0.978146in}{1.496008in}}{\pgfqpoint{0.978146in}{1.484958in}}%
\pgfpathcurveto{\pgfqpoint{0.978146in}{1.473908in}}{\pgfqpoint{0.982536in}{1.463309in}}{\pgfqpoint{0.990350in}{1.455495in}}%
\pgfpathcurveto{\pgfqpoint{0.998163in}{1.447682in}}{\pgfqpoint{1.008762in}{1.443291in}}{\pgfqpoint{1.019812in}{1.443291in}}%
\pgfpathclose%
\pgfusepath{stroke,fill}%
\end{pgfscope}%
\begin{pgfscope}%
\pgfpathrectangle{\pgfqpoint{0.375000in}{0.330000in}}{\pgfqpoint{2.325000in}{2.310000in}}%
\pgfusepath{clip}%
\pgfsetbuttcap%
\pgfsetroundjoin%
\definecolor{currentfill}{rgb}{0.000000,0.000000,0.000000}%
\pgfsetfillcolor{currentfill}%
\pgfsetlinewidth{1.003750pt}%
\definecolor{currentstroke}{rgb}{0.000000,0.000000,0.000000}%
\pgfsetstrokecolor{currentstroke}%
\pgfsetdash{}{0pt}%
\pgfpathmoveto{\pgfqpoint{1.019812in}{1.443291in}}%
\pgfpathcurveto{\pgfqpoint{1.030863in}{1.443291in}}{\pgfqpoint{1.041462in}{1.447682in}}{\pgfqpoint{1.049275in}{1.455495in}}%
\pgfpathcurveto{\pgfqpoint{1.057089in}{1.463309in}}{\pgfqpoint{1.061479in}{1.473908in}}{\pgfqpoint{1.061479in}{1.484958in}}%
\pgfpathcurveto{\pgfqpoint{1.061479in}{1.496008in}}{\pgfqpoint{1.057089in}{1.506607in}}{\pgfqpoint{1.049275in}{1.514421in}}%
\pgfpathcurveto{\pgfqpoint{1.041462in}{1.522235in}}{\pgfqpoint{1.030863in}{1.526625in}}{\pgfqpoint{1.019812in}{1.526625in}}%
\pgfpathcurveto{\pgfqpoint{1.008762in}{1.526625in}}{\pgfqpoint{0.998163in}{1.522235in}}{\pgfqpoint{0.990350in}{1.514421in}}%
\pgfpathcurveto{\pgfqpoint{0.982536in}{1.506607in}}{\pgfqpoint{0.978146in}{1.496008in}}{\pgfqpoint{0.978146in}{1.484958in}}%
\pgfpathcurveto{\pgfqpoint{0.978146in}{1.473908in}}{\pgfqpoint{0.982536in}{1.463309in}}{\pgfqpoint{0.990350in}{1.455495in}}%
\pgfpathcurveto{\pgfqpoint{0.998163in}{1.447682in}}{\pgfqpoint{1.008762in}{1.443291in}}{\pgfqpoint{1.019812in}{1.443291in}}%
\pgfpathclose%
\pgfusepath{stroke,fill}%
\end{pgfscope}%
\begin{pgfscope}%
\pgfpathrectangle{\pgfqpoint{0.375000in}{0.330000in}}{\pgfqpoint{2.325000in}{2.310000in}}%
\pgfusepath{clip}%
\pgfsetbuttcap%
\pgfsetroundjoin%
\definecolor{currentfill}{rgb}{0.000000,0.000000,0.000000}%
\pgfsetfillcolor{currentfill}%
\pgfsetlinewidth{1.003750pt}%
\definecolor{currentstroke}{rgb}{0.000000,0.000000,0.000000}%
\pgfsetstrokecolor{currentstroke}%
\pgfsetdash{}{0pt}%
\pgfpathmoveto{\pgfqpoint{1.019812in}{1.443291in}}%
\pgfpathcurveto{\pgfqpoint{1.030863in}{1.443291in}}{\pgfqpoint{1.041462in}{1.447682in}}{\pgfqpoint{1.049275in}{1.455495in}}%
\pgfpathcurveto{\pgfqpoint{1.057089in}{1.463309in}}{\pgfqpoint{1.061479in}{1.473908in}}{\pgfqpoint{1.061479in}{1.484958in}}%
\pgfpathcurveto{\pgfqpoint{1.061479in}{1.496008in}}{\pgfqpoint{1.057089in}{1.506607in}}{\pgfqpoint{1.049275in}{1.514421in}}%
\pgfpathcurveto{\pgfqpoint{1.041462in}{1.522235in}}{\pgfqpoint{1.030863in}{1.526625in}}{\pgfqpoint{1.019812in}{1.526625in}}%
\pgfpathcurveto{\pgfqpoint{1.008762in}{1.526625in}}{\pgfqpoint{0.998163in}{1.522235in}}{\pgfqpoint{0.990350in}{1.514421in}}%
\pgfpathcurveto{\pgfqpoint{0.982536in}{1.506607in}}{\pgfqpoint{0.978146in}{1.496008in}}{\pgfqpoint{0.978146in}{1.484958in}}%
\pgfpathcurveto{\pgfqpoint{0.978146in}{1.473908in}}{\pgfqpoint{0.982536in}{1.463309in}}{\pgfqpoint{0.990350in}{1.455495in}}%
\pgfpathcurveto{\pgfqpoint{0.998163in}{1.447682in}}{\pgfqpoint{1.008762in}{1.443291in}}{\pgfqpoint{1.019812in}{1.443291in}}%
\pgfpathclose%
\pgfusepath{stroke,fill}%
\end{pgfscope}%
\begin{pgfscope}%
\pgfpathrectangle{\pgfqpoint{0.375000in}{0.330000in}}{\pgfqpoint{2.325000in}{2.310000in}}%
\pgfusepath{clip}%
\pgfsetbuttcap%
\pgfsetroundjoin%
\definecolor{currentfill}{rgb}{0.000000,0.000000,0.000000}%
\pgfsetfillcolor{currentfill}%
\pgfsetlinewidth{1.003750pt}%
\definecolor{currentstroke}{rgb}{0.000000,0.000000,0.000000}%
\pgfsetstrokecolor{currentstroke}%
\pgfsetdash{}{0pt}%
\pgfpathmoveto{\pgfqpoint{1.019812in}{1.443291in}}%
\pgfpathcurveto{\pgfqpoint{1.030863in}{1.443291in}}{\pgfqpoint{1.041462in}{1.447682in}}{\pgfqpoint{1.049275in}{1.455495in}}%
\pgfpathcurveto{\pgfqpoint{1.057089in}{1.463309in}}{\pgfqpoint{1.061479in}{1.473908in}}{\pgfqpoint{1.061479in}{1.484958in}}%
\pgfpathcurveto{\pgfqpoint{1.061479in}{1.496008in}}{\pgfqpoint{1.057089in}{1.506607in}}{\pgfqpoint{1.049275in}{1.514421in}}%
\pgfpathcurveto{\pgfqpoint{1.041462in}{1.522235in}}{\pgfqpoint{1.030863in}{1.526625in}}{\pgfqpoint{1.019812in}{1.526625in}}%
\pgfpathcurveto{\pgfqpoint{1.008762in}{1.526625in}}{\pgfqpoint{0.998163in}{1.522235in}}{\pgfqpoint{0.990350in}{1.514421in}}%
\pgfpathcurveto{\pgfqpoint{0.982536in}{1.506607in}}{\pgfqpoint{0.978146in}{1.496008in}}{\pgfqpoint{0.978146in}{1.484958in}}%
\pgfpathcurveto{\pgfqpoint{0.978146in}{1.473908in}}{\pgfqpoint{0.982536in}{1.463309in}}{\pgfqpoint{0.990350in}{1.455495in}}%
\pgfpathcurveto{\pgfqpoint{0.998163in}{1.447682in}}{\pgfqpoint{1.008762in}{1.443291in}}{\pgfqpoint{1.019812in}{1.443291in}}%
\pgfpathclose%
\pgfusepath{stroke,fill}%
\end{pgfscope}%
\begin{pgfscope}%
\pgfpathrectangle{\pgfqpoint{0.375000in}{0.330000in}}{\pgfqpoint{2.325000in}{2.310000in}}%
\pgfusepath{clip}%
\pgfsetbuttcap%
\pgfsetroundjoin%
\definecolor{currentfill}{rgb}{0.000000,0.000000,0.000000}%
\pgfsetfillcolor{currentfill}%
\pgfsetlinewidth{1.003750pt}%
\definecolor{currentstroke}{rgb}{0.000000,0.000000,0.000000}%
\pgfsetstrokecolor{currentstroke}%
\pgfsetdash{}{0pt}%
\pgfpathmoveto{\pgfqpoint{1.019812in}{1.443291in}}%
\pgfpathcurveto{\pgfqpoint{1.030863in}{1.443291in}}{\pgfqpoint{1.041462in}{1.447682in}}{\pgfqpoint{1.049275in}{1.455495in}}%
\pgfpathcurveto{\pgfqpoint{1.057089in}{1.463309in}}{\pgfqpoint{1.061479in}{1.473908in}}{\pgfqpoint{1.061479in}{1.484958in}}%
\pgfpathcurveto{\pgfqpoint{1.061479in}{1.496008in}}{\pgfqpoint{1.057089in}{1.506607in}}{\pgfqpoint{1.049275in}{1.514421in}}%
\pgfpathcurveto{\pgfqpoint{1.041462in}{1.522235in}}{\pgfqpoint{1.030863in}{1.526625in}}{\pgfqpoint{1.019812in}{1.526625in}}%
\pgfpathcurveto{\pgfqpoint{1.008762in}{1.526625in}}{\pgfqpoint{0.998163in}{1.522235in}}{\pgfqpoint{0.990350in}{1.514421in}}%
\pgfpathcurveto{\pgfqpoint{0.982536in}{1.506607in}}{\pgfqpoint{0.978146in}{1.496008in}}{\pgfqpoint{0.978146in}{1.484958in}}%
\pgfpathcurveto{\pgfqpoint{0.978146in}{1.473908in}}{\pgfqpoint{0.982536in}{1.463309in}}{\pgfqpoint{0.990350in}{1.455495in}}%
\pgfpathcurveto{\pgfqpoint{0.998163in}{1.447682in}}{\pgfqpoint{1.008762in}{1.443291in}}{\pgfqpoint{1.019812in}{1.443291in}}%
\pgfpathclose%
\pgfusepath{stroke,fill}%
\end{pgfscope}%
\begin{pgfscope}%
\pgfpathrectangle{\pgfqpoint{0.375000in}{0.330000in}}{\pgfqpoint{2.325000in}{2.310000in}}%
\pgfusepath{clip}%
\pgfsetbuttcap%
\pgfsetroundjoin%
\definecolor{currentfill}{rgb}{0.000000,0.000000,0.000000}%
\pgfsetfillcolor{currentfill}%
\pgfsetlinewidth{1.003750pt}%
\definecolor{currentstroke}{rgb}{0.000000,0.000000,0.000000}%
\pgfsetstrokecolor{currentstroke}%
\pgfsetdash{}{0pt}%
\pgfpathmoveto{\pgfqpoint{1.019812in}{1.443291in}}%
\pgfpathcurveto{\pgfqpoint{1.030863in}{1.443291in}}{\pgfqpoint{1.041462in}{1.447682in}}{\pgfqpoint{1.049275in}{1.455495in}}%
\pgfpathcurveto{\pgfqpoint{1.057089in}{1.463309in}}{\pgfqpoint{1.061479in}{1.473908in}}{\pgfqpoint{1.061479in}{1.484958in}}%
\pgfpathcurveto{\pgfqpoint{1.061479in}{1.496008in}}{\pgfqpoint{1.057089in}{1.506607in}}{\pgfqpoint{1.049275in}{1.514421in}}%
\pgfpathcurveto{\pgfqpoint{1.041462in}{1.522235in}}{\pgfqpoint{1.030863in}{1.526625in}}{\pgfqpoint{1.019812in}{1.526625in}}%
\pgfpathcurveto{\pgfqpoint{1.008762in}{1.526625in}}{\pgfqpoint{0.998163in}{1.522235in}}{\pgfqpoint{0.990350in}{1.514421in}}%
\pgfpathcurveto{\pgfqpoint{0.982536in}{1.506607in}}{\pgfqpoint{0.978146in}{1.496008in}}{\pgfqpoint{0.978146in}{1.484958in}}%
\pgfpathcurveto{\pgfqpoint{0.978146in}{1.473908in}}{\pgfqpoint{0.982536in}{1.463309in}}{\pgfqpoint{0.990350in}{1.455495in}}%
\pgfpathcurveto{\pgfqpoint{0.998163in}{1.447682in}}{\pgfqpoint{1.008762in}{1.443291in}}{\pgfqpoint{1.019812in}{1.443291in}}%
\pgfpathclose%
\pgfusepath{stroke,fill}%
\end{pgfscope}%
\begin{pgfscope}%
\pgfpathrectangle{\pgfqpoint{0.375000in}{0.330000in}}{\pgfqpoint{2.325000in}{2.310000in}}%
\pgfusepath{clip}%
\pgfsetbuttcap%
\pgfsetroundjoin%
\definecolor{currentfill}{rgb}{0.000000,0.000000,0.000000}%
\pgfsetfillcolor{currentfill}%
\pgfsetlinewidth{1.003750pt}%
\definecolor{currentstroke}{rgb}{0.000000,0.000000,0.000000}%
\pgfsetstrokecolor{currentstroke}%
\pgfsetdash{}{0pt}%
\pgfpathmoveto{\pgfqpoint{1.019812in}{1.443291in}}%
\pgfpathcurveto{\pgfqpoint{1.030863in}{1.443291in}}{\pgfqpoint{1.041462in}{1.447682in}}{\pgfqpoint{1.049275in}{1.455495in}}%
\pgfpathcurveto{\pgfqpoint{1.057089in}{1.463309in}}{\pgfqpoint{1.061479in}{1.473908in}}{\pgfqpoint{1.061479in}{1.484958in}}%
\pgfpathcurveto{\pgfqpoint{1.061479in}{1.496008in}}{\pgfqpoint{1.057089in}{1.506607in}}{\pgfqpoint{1.049275in}{1.514421in}}%
\pgfpathcurveto{\pgfqpoint{1.041462in}{1.522235in}}{\pgfqpoint{1.030863in}{1.526625in}}{\pgfqpoint{1.019812in}{1.526625in}}%
\pgfpathcurveto{\pgfqpoint{1.008762in}{1.526625in}}{\pgfqpoint{0.998163in}{1.522235in}}{\pgfqpoint{0.990350in}{1.514421in}}%
\pgfpathcurveto{\pgfqpoint{0.982536in}{1.506607in}}{\pgfqpoint{0.978146in}{1.496008in}}{\pgfqpoint{0.978146in}{1.484958in}}%
\pgfpathcurveto{\pgfqpoint{0.978146in}{1.473908in}}{\pgfqpoint{0.982536in}{1.463309in}}{\pgfqpoint{0.990350in}{1.455495in}}%
\pgfpathcurveto{\pgfqpoint{0.998163in}{1.447682in}}{\pgfqpoint{1.008762in}{1.443291in}}{\pgfqpoint{1.019812in}{1.443291in}}%
\pgfpathclose%
\pgfusepath{stroke,fill}%
\end{pgfscope}%
\begin{pgfscope}%
\pgfpathrectangle{\pgfqpoint{0.375000in}{0.330000in}}{\pgfqpoint{2.325000in}{2.310000in}}%
\pgfusepath{clip}%
\pgfsetbuttcap%
\pgfsetroundjoin%
\definecolor{currentfill}{rgb}{0.000000,0.000000,0.000000}%
\pgfsetfillcolor{currentfill}%
\pgfsetlinewidth{1.003750pt}%
\definecolor{currentstroke}{rgb}{0.000000,0.000000,0.000000}%
\pgfsetstrokecolor{currentstroke}%
\pgfsetdash{}{0pt}%
\pgfpathmoveto{\pgfqpoint{1.019812in}{1.443291in}}%
\pgfpathcurveto{\pgfqpoint{1.030863in}{1.443291in}}{\pgfqpoint{1.041462in}{1.447682in}}{\pgfqpoint{1.049275in}{1.455495in}}%
\pgfpathcurveto{\pgfqpoint{1.057089in}{1.463309in}}{\pgfqpoint{1.061479in}{1.473908in}}{\pgfqpoint{1.061479in}{1.484958in}}%
\pgfpathcurveto{\pgfqpoint{1.061479in}{1.496008in}}{\pgfqpoint{1.057089in}{1.506607in}}{\pgfqpoint{1.049275in}{1.514421in}}%
\pgfpathcurveto{\pgfqpoint{1.041462in}{1.522235in}}{\pgfqpoint{1.030863in}{1.526625in}}{\pgfqpoint{1.019812in}{1.526625in}}%
\pgfpathcurveto{\pgfqpoint{1.008762in}{1.526625in}}{\pgfqpoint{0.998163in}{1.522235in}}{\pgfqpoint{0.990350in}{1.514421in}}%
\pgfpathcurveto{\pgfqpoint{0.982536in}{1.506607in}}{\pgfqpoint{0.978146in}{1.496008in}}{\pgfqpoint{0.978146in}{1.484958in}}%
\pgfpathcurveto{\pgfqpoint{0.978146in}{1.473908in}}{\pgfqpoint{0.982536in}{1.463309in}}{\pgfqpoint{0.990350in}{1.455495in}}%
\pgfpathcurveto{\pgfqpoint{0.998163in}{1.447682in}}{\pgfqpoint{1.008762in}{1.443291in}}{\pgfqpoint{1.019812in}{1.443291in}}%
\pgfpathclose%
\pgfusepath{stroke,fill}%
\end{pgfscope}%
\begin{pgfscope}%
\pgfpathrectangle{\pgfqpoint{0.375000in}{0.330000in}}{\pgfqpoint{2.325000in}{2.310000in}}%
\pgfusepath{clip}%
\pgfsetbuttcap%
\pgfsetroundjoin%
\definecolor{currentfill}{rgb}{0.000000,0.000000,0.000000}%
\pgfsetfillcolor{currentfill}%
\pgfsetlinewidth{1.003750pt}%
\definecolor{currentstroke}{rgb}{0.000000,0.000000,0.000000}%
\pgfsetstrokecolor{currentstroke}%
\pgfsetdash{}{0pt}%
\pgfpathmoveto{\pgfqpoint{1.019812in}{1.443291in}}%
\pgfpathcurveto{\pgfqpoint{1.030863in}{1.443291in}}{\pgfqpoint{1.041462in}{1.447682in}}{\pgfqpoint{1.049275in}{1.455495in}}%
\pgfpathcurveto{\pgfqpoint{1.057089in}{1.463309in}}{\pgfqpoint{1.061479in}{1.473908in}}{\pgfqpoint{1.061479in}{1.484958in}}%
\pgfpathcurveto{\pgfqpoint{1.061479in}{1.496008in}}{\pgfqpoint{1.057089in}{1.506607in}}{\pgfqpoint{1.049275in}{1.514421in}}%
\pgfpathcurveto{\pgfqpoint{1.041462in}{1.522235in}}{\pgfqpoint{1.030863in}{1.526625in}}{\pgfqpoint{1.019812in}{1.526625in}}%
\pgfpathcurveto{\pgfqpoint{1.008762in}{1.526625in}}{\pgfqpoint{0.998163in}{1.522235in}}{\pgfqpoint{0.990350in}{1.514421in}}%
\pgfpathcurveto{\pgfqpoint{0.982536in}{1.506607in}}{\pgfqpoint{0.978146in}{1.496008in}}{\pgfqpoint{0.978146in}{1.484958in}}%
\pgfpathcurveto{\pgfqpoint{0.978146in}{1.473908in}}{\pgfqpoint{0.982536in}{1.463309in}}{\pgfqpoint{0.990350in}{1.455495in}}%
\pgfpathcurveto{\pgfqpoint{0.998163in}{1.447682in}}{\pgfqpoint{1.008762in}{1.443291in}}{\pgfqpoint{1.019812in}{1.443291in}}%
\pgfpathclose%
\pgfusepath{stroke,fill}%
\end{pgfscope}%
\begin{pgfscope}%
\pgfpathrectangle{\pgfqpoint{0.375000in}{0.330000in}}{\pgfqpoint{2.325000in}{2.310000in}}%
\pgfusepath{clip}%
\pgfsetbuttcap%
\pgfsetroundjoin%
\definecolor{currentfill}{rgb}{0.000000,0.000000,0.000000}%
\pgfsetfillcolor{currentfill}%
\pgfsetlinewidth{1.003750pt}%
\definecolor{currentstroke}{rgb}{0.000000,0.000000,0.000000}%
\pgfsetstrokecolor{currentstroke}%
\pgfsetdash{}{0pt}%
\pgfpathmoveto{\pgfqpoint{1.019812in}{1.443291in}}%
\pgfpathcurveto{\pgfqpoint{1.030863in}{1.443291in}}{\pgfqpoint{1.041462in}{1.447682in}}{\pgfqpoint{1.049275in}{1.455495in}}%
\pgfpathcurveto{\pgfqpoint{1.057089in}{1.463309in}}{\pgfqpoint{1.061479in}{1.473908in}}{\pgfqpoint{1.061479in}{1.484958in}}%
\pgfpathcurveto{\pgfqpoint{1.061479in}{1.496008in}}{\pgfqpoint{1.057089in}{1.506607in}}{\pgfqpoint{1.049275in}{1.514421in}}%
\pgfpathcurveto{\pgfqpoint{1.041462in}{1.522235in}}{\pgfqpoint{1.030863in}{1.526625in}}{\pgfqpoint{1.019812in}{1.526625in}}%
\pgfpathcurveto{\pgfqpoint{1.008762in}{1.526625in}}{\pgfqpoint{0.998163in}{1.522235in}}{\pgfqpoint{0.990350in}{1.514421in}}%
\pgfpathcurveto{\pgfqpoint{0.982536in}{1.506607in}}{\pgfqpoint{0.978146in}{1.496008in}}{\pgfqpoint{0.978146in}{1.484958in}}%
\pgfpathcurveto{\pgfqpoint{0.978146in}{1.473908in}}{\pgfqpoint{0.982536in}{1.463309in}}{\pgfqpoint{0.990350in}{1.455495in}}%
\pgfpathcurveto{\pgfqpoint{0.998163in}{1.447682in}}{\pgfqpoint{1.008762in}{1.443291in}}{\pgfqpoint{1.019812in}{1.443291in}}%
\pgfpathclose%
\pgfusepath{stroke,fill}%
\end{pgfscope}%
\begin{pgfscope}%
\pgfpathrectangle{\pgfqpoint{0.375000in}{0.330000in}}{\pgfqpoint{2.325000in}{2.310000in}}%
\pgfusepath{clip}%
\pgfsetbuttcap%
\pgfsetroundjoin%
\definecolor{currentfill}{rgb}{0.000000,0.000000,0.000000}%
\pgfsetfillcolor{currentfill}%
\pgfsetlinewidth{1.003750pt}%
\definecolor{currentstroke}{rgb}{0.000000,0.000000,0.000000}%
\pgfsetstrokecolor{currentstroke}%
\pgfsetdash{}{0pt}%
\pgfpathmoveto{\pgfqpoint{1.019812in}{1.443291in}}%
\pgfpathcurveto{\pgfqpoint{1.030863in}{1.443291in}}{\pgfqpoint{1.041462in}{1.447682in}}{\pgfqpoint{1.049275in}{1.455495in}}%
\pgfpathcurveto{\pgfqpoint{1.057089in}{1.463309in}}{\pgfqpoint{1.061479in}{1.473908in}}{\pgfqpoint{1.061479in}{1.484958in}}%
\pgfpathcurveto{\pgfqpoint{1.061479in}{1.496008in}}{\pgfqpoint{1.057089in}{1.506607in}}{\pgfqpoint{1.049275in}{1.514421in}}%
\pgfpathcurveto{\pgfqpoint{1.041462in}{1.522235in}}{\pgfqpoint{1.030863in}{1.526625in}}{\pgfqpoint{1.019812in}{1.526625in}}%
\pgfpathcurveto{\pgfqpoint{1.008762in}{1.526625in}}{\pgfqpoint{0.998163in}{1.522235in}}{\pgfqpoint{0.990350in}{1.514421in}}%
\pgfpathcurveto{\pgfqpoint{0.982536in}{1.506607in}}{\pgfqpoint{0.978146in}{1.496008in}}{\pgfqpoint{0.978146in}{1.484958in}}%
\pgfpathcurveto{\pgfqpoint{0.978146in}{1.473908in}}{\pgfqpoint{0.982536in}{1.463309in}}{\pgfqpoint{0.990350in}{1.455495in}}%
\pgfpathcurveto{\pgfqpoint{0.998163in}{1.447682in}}{\pgfqpoint{1.008762in}{1.443291in}}{\pgfqpoint{1.019812in}{1.443291in}}%
\pgfpathclose%
\pgfusepath{stroke,fill}%
\end{pgfscope}%
\begin{pgfscope}%
\pgfpathrectangle{\pgfqpoint{0.375000in}{0.330000in}}{\pgfqpoint{2.325000in}{2.310000in}}%
\pgfusepath{clip}%
\pgfsetbuttcap%
\pgfsetroundjoin%
\definecolor{currentfill}{rgb}{0.000000,0.000000,0.000000}%
\pgfsetfillcolor{currentfill}%
\pgfsetlinewidth{1.003750pt}%
\definecolor{currentstroke}{rgb}{0.000000,0.000000,0.000000}%
\pgfsetstrokecolor{currentstroke}%
\pgfsetdash{}{0pt}%
\pgfpathmoveto{\pgfqpoint{1.019812in}{1.443291in}}%
\pgfpathcurveto{\pgfqpoint{1.030863in}{1.443291in}}{\pgfqpoint{1.041462in}{1.447682in}}{\pgfqpoint{1.049275in}{1.455495in}}%
\pgfpathcurveto{\pgfqpoint{1.057089in}{1.463309in}}{\pgfqpoint{1.061479in}{1.473908in}}{\pgfqpoint{1.061479in}{1.484958in}}%
\pgfpathcurveto{\pgfqpoint{1.061479in}{1.496008in}}{\pgfqpoint{1.057089in}{1.506607in}}{\pgfqpoint{1.049275in}{1.514421in}}%
\pgfpathcurveto{\pgfqpoint{1.041462in}{1.522235in}}{\pgfqpoint{1.030863in}{1.526625in}}{\pgfqpoint{1.019812in}{1.526625in}}%
\pgfpathcurveto{\pgfqpoint{1.008762in}{1.526625in}}{\pgfqpoint{0.998163in}{1.522235in}}{\pgfqpoint{0.990350in}{1.514421in}}%
\pgfpathcurveto{\pgfqpoint{0.982536in}{1.506607in}}{\pgfqpoint{0.978146in}{1.496008in}}{\pgfqpoint{0.978146in}{1.484958in}}%
\pgfpathcurveto{\pgfqpoint{0.978146in}{1.473908in}}{\pgfqpoint{0.982536in}{1.463309in}}{\pgfqpoint{0.990350in}{1.455495in}}%
\pgfpathcurveto{\pgfqpoint{0.998163in}{1.447682in}}{\pgfqpoint{1.008762in}{1.443291in}}{\pgfqpoint{1.019812in}{1.443291in}}%
\pgfpathclose%
\pgfusepath{stroke,fill}%
\end{pgfscope}%
\begin{pgfscope}%
\pgfpathrectangle{\pgfqpoint{0.375000in}{0.330000in}}{\pgfqpoint{2.325000in}{2.310000in}}%
\pgfusepath{clip}%
\pgfsetbuttcap%
\pgfsetroundjoin%
\definecolor{currentfill}{rgb}{0.000000,0.000000,0.000000}%
\pgfsetfillcolor{currentfill}%
\pgfsetlinewidth{1.003750pt}%
\definecolor{currentstroke}{rgb}{0.000000,0.000000,0.000000}%
\pgfsetstrokecolor{currentstroke}%
\pgfsetdash{}{0pt}%
\pgfpathmoveto{\pgfqpoint{1.019812in}{1.443291in}}%
\pgfpathcurveto{\pgfqpoint{1.030863in}{1.443291in}}{\pgfqpoint{1.041462in}{1.447682in}}{\pgfqpoint{1.049275in}{1.455495in}}%
\pgfpathcurveto{\pgfqpoint{1.057089in}{1.463309in}}{\pgfqpoint{1.061479in}{1.473908in}}{\pgfqpoint{1.061479in}{1.484958in}}%
\pgfpathcurveto{\pgfqpoint{1.061479in}{1.496008in}}{\pgfqpoint{1.057089in}{1.506607in}}{\pgfqpoint{1.049275in}{1.514421in}}%
\pgfpathcurveto{\pgfqpoint{1.041462in}{1.522235in}}{\pgfqpoint{1.030863in}{1.526625in}}{\pgfqpoint{1.019812in}{1.526625in}}%
\pgfpathcurveto{\pgfqpoint{1.008762in}{1.526625in}}{\pgfqpoint{0.998163in}{1.522235in}}{\pgfqpoint{0.990350in}{1.514421in}}%
\pgfpathcurveto{\pgfqpoint{0.982536in}{1.506607in}}{\pgfqpoint{0.978146in}{1.496008in}}{\pgfqpoint{0.978146in}{1.484958in}}%
\pgfpathcurveto{\pgfqpoint{0.978146in}{1.473908in}}{\pgfqpoint{0.982536in}{1.463309in}}{\pgfqpoint{0.990350in}{1.455495in}}%
\pgfpathcurveto{\pgfqpoint{0.998163in}{1.447682in}}{\pgfqpoint{1.008762in}{1.443291in}}{\pgfqpoint{1.019812in}{1.443291in}}%
\pgfpathclose%
\pgfusepath{stroke,fill}%
\end{pgfscope}%
\begin{pgfscope}%
\pgfpathrectangle{\pgfqpoint{0.375000in}{0.330000in}}{\pgfqpoint{2.325000in}{2.310000in}}%
\pgfusepath{clip}%
\pgfsetbuttcap%
\pgfsetroundjoin%
\definecolor{currentfill}{rgb}{0.000000,0.000000,0.000000}%
\pgfsetfillcolor{currentfill}%
\pgfsetlinewidth{1.003750pt}%
\definecolor{currentstroke}{rgb}{0.000000,0.000000,0.000000}%
\pgfsetstrokecolor{currentstroke}%
\pgfsetdash{}{0pt}%
\pgfpathmoveto{\pgfqpoint{1.019812in}{1.443291in}}%
\pgfpathcurveto{\pgfqpoint{1.030863in}{1.443291in}}{\pgfqpoint{1.041462in}{1.447682in}}{\pgfqpoint{1.049275in}{1.455495in}}%
\pgfpathcurveto{\pgfqpoint{1.057089in}{1.463309in}}{\pgfqpoint{1.061479in}{1.473908in}}{\pgfqpoint{1.061479in}{1.484958in}}%
\pgfpathcurveto{\pgfqpoint{1.061479in}{1.496008in}}{\pgfqpoint{1.057089in}{1.506607in}}{\pgfqpoint{1.049275in}{1.514421in}}%
\pgfpathcurveto{\pgfqpoint{1.041462in}{1.522235in}}{\pgfqpoint{1.030863in}{1.526625in}}{\pgfqpoint{1.019812in}{1.526625in}}%
\pgfpathcurveto{\pgfqpoint{1.008762in}{1.526625in}}{\pgfqpoint{0.998163in}{1.522235in}}{\pgfqpoint{0.990350in}{1.514421in}}%
\pgfpathcurveto{\pgfqpoint{0.982536in}{1.506607in}}{\pgfqpoint{0.978146in}{1.496008in}}{\pgfqpoint{0.978146in}{1.484958in}}%
\pgfpathcurveto{\pgfqpoint{0.978146in}{1.473908in}}{\pgfqpoint{0.982536in}{1.463309in}}{\pgfqpoint{0.990350in}{1.455495in}}%
\pgfpathcurveto{\pgfqpoint{0.998163in}{1.447682in}}{\pgfqpoint{1.008762in}{1.443291in}}{\pgfqpoint{1.019812in}{1.443291in}}%
\pgfpathclose%
\pgfusepath{stroke,fill}%
\end{pgfscope}%
\begin{pgfscope}%
\pgfpathrectangle{\pgfqpoint{0.375000in}{0.330000in}}{\pgfqpoint{2.325000in}{2.310000in}}%
\pgfusepath{clip}%
\pgfsetbuttcap%
\pgfsetroundjoin%
\definecolor{currentfill}{rgb}{0.000000,0.000000,0.000000}%
\pgfsetfillcolor{currentfill}%
\pgfsetlinewidth{1.003750pt}%
\definecolor{currentstroke}{rgb}{0.000000,0.000000,0.000000}%
\pgfsetstrokecolor{currentstroke}%
\pgfsetdash{}{0pt}%
\pgfpathmoveto{\pgfqpoint{1.019812in}{1.443291in}}%
\pgfpathcurveto{\pgfqpoint{1.030863in}{1.443291in}}{\pgfqpoint{1.041462in}{1.447682in}}{\pgfqpoint{1.049275in}{1.455495in}}%
\pgfpathcurveto{\pgfqpoint{1.057089in}{1.463309in}}{\pgfqpoint{1.061479in}{1.473908in}}{\pgfqpoint{1.061479in}{1.484958in}}%
\pgfpathcurveto{\pgfqpoint{1.061479in}{1.496008in}}{\pgfqpoint{1.057089in}{1.506607in}}{\pgfqpoint{1.049275in}{1.514421in}}%
\pgfpathcurveto{\pgfqpoint{1.041462in}{1.522235in}}{\pgfqpoint{1.030863in}{1.526625in}}{\pgfqpoint{1.019812in}{1.526625in}}%
\pgfpathcurveto{\pgfqpoint{1.008762in}{1.526625in}}{\pgfqpoint{0.998163in}{1.522235in}}{\pgfqpoint{0.990350in}{1.514421in}}%
\pgfpathcurveto{\pgfqpoint{0.982536in}{1.506607in}}{\pgfqpoint{0.978146in}{1.496008in}}{\pgfqpoint{0.978146in}{1.484958in}}%
\pgfpathcurveto{\pgfqpoint{0.978146in}{1.473908in}}{\pgfqpoint{0.982536in}{1.463309in}}{\pgfqpoint{0.990350in}{1.455495in}}%
\pgfpathcurveto{\pgfqpoint{0.998163in}{1.447682in}}{\pgfqpoint{1.008762in}{1.443291in}}{\pgfqpoint{1.019812in}{1.443291in}}%
\pgfpathclose%
\pgfusepath{stroke,fill}%
\end{pgfscope}%
\begin{pgfscope}%
\pgfpathrectangle{\pgfqpoint{0.375000in}{0.330000in}}{\pgfqpoint{2.325000in}{2.310000in}}%
\pgfusepath{clip}%
\pgfsetbuttcap%
\pgfsetroundjoin%
\definecolor{currentfill}{rgb}{0.000000,0.000000,0.000000}%
\pgfsetfillcolor{currentfill}%
\pgfsetlinewidth{1.003750pt}%
\definecolor{currentstroke}{rgb}{0.000000,0.000000,0.000000}%
\pgfsetstrokecolor{currentstroke}%
\pgfsetdash{}{0pt}%
\pgfpathmoveto{\pgfqpoint{1.019812in}{1.443291in}}%
\pgfpathcurveto{\pgfqpoint{1.030863in}{1.443291in}}{\pgfqpoint{1.041462in}{1.447682in}}{\pgfqpoint{1.049275in}{1.455495in}}%
\pgfpathcurveto{\pgfqpoint{1.057089in}{1.463309in}}{\pgfqpoint{1.061479in}{1.473908in}}{\pgfqpoint{1.061479in}{1.484958in}}%
\pgfpathcurveto{\pgfqpoint{1.061479in}{1.496008in}}{\pgfqpoint{1.057089in}{1.506607in}}{\pgfqpoint{1.049275in}{1.514421in}}%
\pgfpathcurveto{\pgfqpoint{1.041462in}{1.522235in}}{\pgfqpoint{1.030863in}{1.526625in}}{\pgfqpoint{1.019812in}{1.526625in}}%
\pgfpathcurveto{\pgfqpoint{1.008762in}{1.526625in}}{\pgfqpoint{0.998163in}{1.522235in}}{\pgfqpoint{0.990350in}{1.514421in}}%
\pgfpathcurveto{\pgfqpoint{0.982536in}{1.506607in}}{\pgfqpoint{0.978146in}{1.496008in}}{\pgfqpoint{0.978146in}{1.484958in}}%
\pgfpathcurveto{\pgfqpoint{0.978146in}{1.473908in}}{\pgfqpoint{0.982536in}{1.463309in}}{\pgfqpoint{0.990350in}{1.455495in}}%
\pgfpathcurveto{\pgfqpoint{0.998163in}{1.447682in}}{\pgfqpoint{1.008762in}{1.443291in}}{\pgfqpoint{1.019812in}{1.443291in}}%
\pgfpathclose%
\pgfusepath{stroke,fill}%
\end{pgfscope}%
\begin{pgfscope}%
\pgfpathrectangle{\pgfqpoint{0.375000in}{0.330000in}}{\pgfqpoint{2.325000in}{2.310000in}}%
\pgfusepath{clip}%
\pgfsetbuttcap%
\pgfsetroundjoin%
\definecolor{currentfill}{rgb}{0.000000,0.000000,0.000000}%
\pgfsetfillcolor{currentfill}%
\pgfsetlinewidth{1.003750pt}%
\definecolor{currentstroke}{rgb}{0.000000,0.000000,0.000000}%
\pgfsetstrokecolor{currentstroke}%
\pgfsetdash{}{0pt}%
\pgfpathmoveto{\pgfqpoint{1.019812in}{1.443291in}}%
\pgfpathcurveto{\pgfqpoint{1.030863in}{1.443291in}}{\pgfqpoint{1.041462in}{1.447682in}}{\pgfqpoint{1.049275in}{1.455495in}}%
\pgfpathcurveto{\pgfqpoint{1.057089in}{1.463309in}}{\pgfqpoint{1.061479in}{1.473908in}}{\pgfqpoint{1.061479in}{1.484958in}}%
\pgfpathcurveto{\pgfqpoint{1.061479in}{1.496008in}}{\pgfqpoint{1.057089in}{1.506607in}}{\pgfqpoint{1.049275in}{1.514421in}}%
\pgfpathcurveto{\pgfqpoint{1.041462in}{1.522235in}}{\pgfqpoint{1.030863in}{1.526625in}}{\pgfqpoint{1.019812in}{1.526625in}}%
\pgfpathcurveto{\pgfqpoint{1.008762in}{1.526625in}}{\pgfqpoint{0.998163in}{1.522235in}}{\pgfqpoint{0.990350in}{1.514421in}}%
\pgfpathcurveto{\pgfqpoint{0.982536in}{1.506607in}}{\pgfqpoint{0.978146in}{1.496008in}}{\pgfqpoint{0.978146in}{1.484958in}}%
\pgfpathcurveto{\pgfqpoint{0.978146in}{1.473908in}}{\pgfqpoint{0.982536in}{1.463309in}}{\pgfqpoint{0.990350in}{1.455495in}}%
\pgfpathcurveto{\pgfqpoint{0.998163in}{1.447682in}}{\pgfqpoint{1.008762in}{1.443291in}}{\pgfqpoint{1.019812in}{1.443291in}}%
\pgfpathclose%
\pgfusepath{stroke,fill}%
\end{pgfscope}%
\begin{pgfscope}%
\pgfpathrectangle{\pgfqpoint{0.375000in}{0.330000in}}{\pgfqpoint{2.325000in}{2.310000in}}%
\pgfusepath{clip}%
\pgfsetbuttcap%
\pgfsetroundjoin%
\definecolor{currentfill}{rgb}{0.000000,0.000000,0.000000}%
\pgfsetfillcolor{currentfill}%
\pgfsetlinewidth{1.003750pt}%
\definecolor{currentstroke}{rgb}{0.000000,0.000000,0.000000}%
\pgfsetstrokecolor{currentstroke}%
\pgfsetdash{}{0pt}%
\pgfpathmoveto{\pgfqpoint{1.019812in}{1.443291in}}%
\pgfpathcurveto{\pgfqpoint{1.030863in}{1.443291in}}{\pgfqpoint{1.041462in}{1.447682in}}{\pgfqpoint{1.049275in}{1.455495in}}%
\pgfpathcurveto{\pgfqpoint{1.057089in}{1.463309in}}{\pgfqpoint{1.061479in}{1.473908in}}{\pgfqpoint{1.061479in}{1.484958in}}%
\pgfpathcurveto{\pgfqpoint{1.061479in}{1.496008in}}{\pgfqpoint{1.057089in}{1.506607in}}{\pgfqpoint{1.049275in}{1.514421in}}%
\pgfpathcurveto{\pgfqpoint{1.041462in}{1.522235in}}{\pgfqpoint{1.030863in}{1.526625in}}{\pgfqpoint{1.019812in}{1.526625in}}%
\pgfpathcurveto{\pgfqpoint{1.008762in}{1.526625in}}{\pgfqpoint{0.998163in}{1.522235in}}{\pgfqpoint{0.990350in}{1.514421in}}%
\pgfpathcurveto{\pgfqpoint{0.982536in}{1.506607in}}{\pgfqpoint{0.978146in}{1.496008in}}{\pgfqpoint{0.978146in}{1.484958in}}%
\pgfpathcurveto{\pgfqpoint{0.978146in}{1.473908in}}{\pgfqpoint{0.982536in}{1.463309in}}{\pgfqpoint{0.990350in}{1.455495in}}%
\pgfpathcurveto{\pgfqpoint{0.998163in}{1.447682in}}{\pgfqpoint{1.008762in}{1.443291in}}{\pgfqpoint{1.019812in}{1.443291in}}%
\pgfpathclose%
\pgfusepath{stroke,fill}%
\end{pgfscope}%
\begin{pgfscope}%
\pgfpathrectangle{\pgfqpoint{0.375000in}{0.330000in}}{\pgfqpoint{2.325000in}{2.310000in}}%
\pgfusepath{clip}%
\pgfsetbuttcap%
\pgfsetroundjoin%
\definecolor{currentfill}{rgb}{0.000000,0.000000,0.000000}%
\pgfsetfillcolor{currentfill}%
\pgfsetlinewidth{1.003750pt}%
\definecolor{currentstroke}{rgb}{0.000000,0.000000,0.000000}%
\pgfsetstrokecolor{currentstroke}%
\pgfsetdash{}{0pt}%
\pgfpathmoveto{\pgfqpoint{1.019812in}{1.443291in}}%
\pgfpathcurveto{\pgfqpoint{1.030863in}{1.443291in}}{\pgfqpoint{1.041462in}{1.447682in}}{\pgfqpoint{1.049275in}{1.455495in}}%
\pgfpathcurveto{\pgfqpoint{1.057089in}{1.463309in}}{\pgfqpoint{1.061479in}{1.473908in}}{\pgfqpoint{1.061479in}{1.484958in}}%
\pgfpathcurveto{\pgfqpoint{1.061479in}{1.496008in}}{\pgfqpoint{1.057089in}{1.506607in}}{\pgfqpoint{1.049275in}{1.514421in}}%
\pgfpathcurveto{\pgfqpoint{1.041462in}{1.522235in}}{\pgfqpoint{1.030863in}{1.526625in}}{\pgfqpoint{1.019812in}{1.526625in}}%
\pgfpathcurveto{\pgfqpoint{1.008762in}{1.526625in}}{\pgfqpoint{0.998163in}{1.522235in}}{\pgfqpoint{0.990350in}{1.514421in}}%
\pgfpathcurveto{\pgfqpoint{0.982536in}{1.506607in}}{\pgfqpoint{0.978146in}{1.496008in}}{\pgfqpoint{0.978146in}{1.484958in}}%
\pgfpathcurveto{\pgfqpoint{0.978146in}{1.473908in}}{\pgfqpoint{0.982536in}{1.463309in}}{\pgfqpoint{0.990350in}{1.455495in}}%
\pgfpathcurveto{\pgfqpoint{0.998163in}{1.447682in}}{\pgfqpoint{1.008762in}{1.443291in}}{\pgfqpoint{1.019812in}{1.443291in}}%
\pgfpathclose%
\pgfusepath{stroke,fill}%
\end{pgfscope}%
\begin{pgfscope}%
\pgfpathrectangle{\pgfqpoint{0.375000in}{0.330000in}}{\pgfqpoint{2.325000in}{2.310000in}}%
\pgfusepath{clip}%
\pgfsetbuttcap%
\pgfsetroundjoin%
\definecolor{currentfill}{rgb}{0.000000,0.000000,0.000000}%
\pgfsetfillcolor{currentfill}%
\pgfsetlinewidth{1.003750pt}%
\definecolor{currentstroke}{rgb}{0.000000,0.000000,0.000000}%
\pgfsetstrokecolor{currentstroke}%
\pgfsetdash{}{0pt}%
\pgfpathmoveto{\pgfqpoint{1.019812in}{1.443291in}}%
\pgfpathcurveto{\pgfqpoint{1.030863in}{1.443291in}}{\pgfqpoint{1.041462in}{1.447682in}}{\pgfqpoint{1.049275in}{1.455495in}}%
\pgfpathcurveto{\pgfqpoint{1.057089in}{1.463309in}}{\pgfqpoint{1.061479in}{1.473908in}}{\pgfqpoint{1.061479in}{1.484958in}}%
\pgfpathcurveto{\pgfqpoint{1.061479in}{1.496008in}}{\pgfqpoint{1.057089in}{1.506607in}}{\pgfqpoint{1.049275in}{1.514421in}}%
\pgfpathcurveto{\pgfqpoint{1.041462in}{1.522235in}}{\pgfqpoint{1.030863in}{1.526625in}}{\pgfqpoint{1.019812in}{1.526625in}}%
\pgfpathcurveto{\pgfqpoint{1.008762in}{1.526625in}}{\pgfqpoint{0.998163in}{1.522235in}}{\pgfqpoint{0.990350in}{1.514421in}}%
\pgfpathcurveto{\pgfqpoint{0.982536in}{1.506607in}}{\pgfqpoint{0.978146in}{1.496008in}}{\pgfqpoint{0.978146in}{1.484958in}}%
\pgfpathcurveto{\pgfqpoint{0.978146in}{1.473908in}}{\pgfqpoint{0.982536in}{1.463309in}}{\pgfqpoint{0.990350in}{1.455495in}}%
\pgfpathcurveto{\pgfqpoint{0.998163in}{1.447682in}}{\pgfqpoint{1.008762in}{1.443291in}}{\pgfqpoint{1.019812in}{1.443291in}}%
\pgfpathclose%
\pgfusepath{stroke,fill}%
\end{pgfscope}%
\begin{pgfscope}%
\pgfpathrectangle{\pgfqpoint{0.375000in}{0.330000in}}{\pgfqpoint{2.325000in}{2.310000in}}%
\pgfusepath{clip}%
\pgfsetbuttcap%
\pgfsetroundjoin%
\definecolor{currentfill}{rgb}{0.000000,0.000000,0.000000}%
\pgfsetfillcolor{currentfill}%
\pgfsetlinewidth{1.003750pt}%
\definecolor{currentstroke}{rgb}{0.000000,0.000000,0.000000}%
\pgfsetstrokecolor{currentstroke}%
\pgfsetdash{}{0pt}%
\pgfpathmoveto{\pgfqpoint{1.019812in}{1.443291in}}%
\pgfpathcurveto{\pgfqpoint{1.030863in}{1.443291in}}{\pgfqpoint{1.041462in}{1.447682in}}{\pgfqpoint{1.049275in}{1.455495in}}%
\pgfpathcurveto{\pgfqpoint{1.057089in}{1.463309in}}{\pgfqpoint{1.061479in}{1.473908in}}{\pgfqpoint{1.061479in}{1.484958in}}%
\pgfpathcurveto{\pgfqpoint{1.061479in}{1.496008in}}{\pgfqpoint{1.057089in}{1.506607in}}{\pgfqpoint{1.049275in}{1.514421in}}%
\pgfpathcurveto{\pgfqpoint{1.041462in}{1.522235in}}{\pgfqpoint{1.030863in}{1.526625in}}{\pgfqpoint{1.019812in}{1.526625in}}%
\pgfpathcurveto{\pgfqpoint{1.008762in}{1.526625in}}{\pgfqpoint{0.998163in}{1.522235in}}{\pgfqpoint{0.990350in}{1.514421in}}%
\pgfpathcurveto{\pgfqpoint{0.982536in}{1.506607in}}{\pgfqpoint{0.978146in}{1.496008in}}{\pgfqpoint{0.978146in}{1.484958in}}%
\pgfpathcurveto{\pgfqpoint{0.978146in}{1.473908in}}{\pgfqpoint{0.982536in}{1.463309in}}{\pgfqpoint{0.990350in}{1.455495in}}%
\pgfpathcurveto{\pgfqpoint{0.998163in}{1.447682in}}{\pgfqpoint{1.008762in}{1.443291in}}{\pgfqpoint{1.019812in}{1.443291in}}%
\pgfpathclose%
\pgfusepath{stroke,fill}%
\end{pgfscope}%
\begin{pgfscope}%
\pgfpathrectangle{\pgfqpoint{0.375000in}{0.330000in}}{\pgfqpoint{2.325000in}{2.310000in}}%
\pgfusepath{clip}%
\pgfsetbuttcap%
\pgfsetroundjoin%
\definecolor{currentfill}{rgb}{0.000000,0.000000,0.000000}%
\pgfsetfillcolor{currentfill}%
\pgfsetlinewidth{1.003750pt}%
\definecolor{currentstroke}{rgb}{0.000000,0.000000,0.000000}%
\pgfsetstrokecolor{currentstroke}%
\pgfsetdash{}{0pt}%
\pgfpathmoveto{\pgfqpoint{1.019812in}{1.443291in}}%
\pgfpathcurveto{\pgfqpoint{1.030863in}{1.443291in}}{\pgfqpoint{1.041462in}{1.447682in}}{\pgfqpoint{1.049275in}{1.455495in}}%
\pgfpathcurveto{\pgfqpoint{1.057089in}{1.463309in}}{\pgfqpoint{1.061479in}{1.473908in}}{\pgfqpoint{1.061479in}{1.484958in}}%
\pgfpathcurveto{\pgfqpoint{1.061479in}{1.496008in}}{\pgfqpoint{1.057089in}{1.506607in}}{\pgfqpoint{1.049275in}{1.514421in}}%
\pgfpathcurveto{\pgfqpoint{1.041462in}{1.522235in}}{\pgfqpoint{1.030863in}{1.526625in}}{\pgfqpoint{1.019812in}{1.526625in}}%
\pgfpathcurveto{\pgfqpoint{1.008762in}{1.526625in}}{\pgfqpoint{0.998163in}{1.522235in}}{\pgfqpoint{0.990350in}{1.514421in}}%
\pgfpathcurveto{\pgfqpoint{0.982536in}{1.506607in}}{\pgfqpoint{0.978146in}{1.496008in}}{\pgfqpoint{0.978146in}{1.484958in}}%
\pgfpathcurveto{\pgfqpoint{0.978146in}{1.473908in}}{\pgfqpoint{0.982536in}{1.463309in}}{\pgfqpoint{0.990350in}{1.455495in}}%
\pgfpathcurveto{\pgfqpoint{0.998163in}{1.447682in}}{\pgfqpoint{1.008762in}{1.443291in}}{\pgfqpoint{1.019812in}{1.443291in}}%
\pgfpathclose%
\pgfusepath{stroke,fill}%
\end{pgfscope}%
\begin{pgfscope}%
\pgfpathrectangle{\pgfqpoint{0.375000in}{0.330000in}}{\pgfqpoint{2.325000in}{2.310000in}}%
\pgfusepath{clip}%
\pgfsetbuttcap%
\pgfsetroundjoin%
\definecolor{currentfill}{rgb}{0.000000,0.000000,0.000000}%
\pgfsetfillcolor{currentfill}%
\pgfsetlinewidth{1.003750pt}%
\definecolor{currentstroke}{rgb}{0.000000,0.000000,0.000000}%
\pgfsetstrokecolor{currentstroke}%
\pgfsetdash{}{0pt}%
\pgfpathmoveto{\pgfqpoint{1.019812in}{2.474583in}}%
\pgfpathcurveto{\pgfqpoint{1.030863in}{2.474583in}}{\pgfqpoint{1.041462in}{2.478974in}}{\pgfqpoint{1.049275in}{2.486787in}}%
\pgfpathcurveto{\pgfqpoint{1.057089in}{2.494601in}}{\pgfqpoint{1.061479in}{2.505200in}}{\pgfqpoint{1.061479in}{2.516250in}}%
\pgfpathcurveto{\pgfqpoint{1.061479in}{2.527300in}}{\pgfqpoint{1.057089in}{2.537899in}}{\pgfqpoint{1.049275in}{2.545713in}}%
\pgfpathcurveto{\pgfqpoint{1.041462in}{2.553526in}}{\pgfqpoint{1.030863in}{2.557917in}}{\pgfqpoint{1.019812in}{2.557917in}}%
\pgfpathcurveto{\pgfqpoint{1.008762in}{2.557917in}}{\pgfqpoint{0.998163in}{2.553526in}}{\pgfqpoint{0.990350in}{2.545713in}}%
\pgfpathcurveto{\pgfqpoint{0.982536in}{2.537899in}}{\pgfqpoint{0.978146in}{2.527300in}}{\pgfqpoint{0.978146in}{2.516250in}}%
\pgfpathcurveto{\pgfqpoint{0.978146in}{2.505200in}}{\pgfqpoint{0.982536in}{2.494601in}}{\pgfqpoint{0.990350in}{2.486787in}}%
\pgfpathcurveto{\pgfqpoint{0.998163in}{2.478974in}}{\pgfqpoint{1.008762in}{2.474583in}}{\pgfqpoint{1.019812in}{2.474583in}}%
\pgfpathclose%
\pgfusepath{stroke,fill}%
\end{pgfscope}%
\begin{pgfscope}%
\pgfpathrectangle{\pgfqpoint{0.375000in}{0.330000in}}{\pgfqpoint{2.325000in}{2.310000in}}%
\pgfusepath{clip}%
\pgfsetbuttcap%
\pgfsetroundjoin%
\definecolor{currentfill}{rgb}{0.000000,0.000000,0.000000}%
\pgfsetfillcolor{currentfill}%
\pgfsetlinewidth{1.003750pt}%
\definecolor{currentstroke}{rgb}{0.000000,0.000000,0.000000}%
\pgfsetstrokecolor{currentstroke}%
\pgfsetdash{}{0pt}%
\pgfpathmoveto{\pgfqpoint{1.019812in}{1.443291in}}%
\pgfpathcurveto{\pgfqpoint{1.030863in}{1.443291in}}{\pgfqpoint{1.041462in}{1.447682in}}{\pgfqpoint{1.049275in}{1.455495in}}%
\pgfpathcurveto{\pgfqpoint{1.057089in}{1.463309in}}{\pgfqpoint{1.061479in}{1.473908in}}{\pgfqpoint{1.061479in}{1.484958in}}%
\pgfpathcurveto{\pgfqpoint{1.061479in}{1.496008in}}{\pgfqpoint{1.057089in}{1.506607in}}{\pgfqpoint{1.049275in}{1.514421in}}%
\pgfpathcurveto{\pgfqpoint{1.041462in}{1.522235in}}{\pgfqpoint{1.030863in}{1.526625in}}{\pgfqpoint{1.019812in}{1.526625in}}%
\pgfpathcurveto{\pgfqpoint{1.008762in}{1.526625in}}{\pgfqpoint{0.998163in}{1.522235in}}{\pgfqpoint{0.990350in}{1.514421in}}%
\pgfpathcurveto{\pgfqpoint{0.982536in}{1.506607in}}{\pgfqpoint{0.978146in}{1.496008in}}{\pgfqpoint{0.978146in}{1.484958in}}%
\pgfpathcurveto{\pgfqpoint{0.978146in}{1.473908in}}{\pgfqpoint{0.982536in}{1.463309in}}{\pgfqpoint{0.990350in}{1.455495in}}%
\pgfpathcurveto{\pgfqpoint{0.998163in}{1.447682in}}{\pgfqpoint{1.008762in}{1.443291in}}{\pgfqpoint{1.019812in}{1.443291in}}%
\pgfpathclose%
\pgfusepath{stroke,fill}%
\end{pgfscope}%
\begin{pgfscope}%
\pgfpathrectangle{\pgfqpoint{0.375000in}{0.330000in}}{\pgfqpoint{2.325000in}{2.310000in}}%
\pgfusepath{clip}%
\pgfsetbuttcap%
\pgfsetroundjoin%
\definecolor{currentfill}{rgb}{0.000000,0.000000,0.000000}%
\pgfsetfillcolor{currentfill}%
\pgfsetlinewidth{1.003750pt}%
\definecolor{currentstroke}{rgb}{0.000000,0.000000,0.000000}%
\pgfsetstrokecolor{currentstroke}%
\pgfsetdash{}{0pt}%
\pgfpathmoveto{\pgfqpoint{1.019812in}{1.443291in}}%
\pgfpathcurveto{\pgfqpoint{1.030863in}{1.443291in}}{\pgfqpoint{1.041462in}{1.447682in}}{\pgfqpoint{1.049275in}{1.455495in}}%
\pgfpathcurveto{\pgfqpoint{1.057089in}{1.463309in}}{\pgfqpoint{1.061479in}{1.473908in}}{\pgfqpoint{1.061479in}{1.484958in}}%
\pgfpathcurveto{\pgfqpoint{1.061479in}{1.496008in}}{\pgfqpoint{1.057089in}{1.506607in}}{\pgfqpoint{1.049275in}{1.514421in}}%
\pgfpathcurveto{\pgfqpoint{1.041462in}{1.522235in}}{\pgfqpoint{1.030863in}{1.526625in}}{\pgfqpoint{1.019812in}{1.526625in}}%
\pgfpathcurveto{\pgfqpoint{1.008762in}{1.526625in}}{\pgfqpoint{0.998163in}{1.522235in}}{\pgfqpoint{0.990350in}{1.514421in}}%
\pgfpathcurveto{\pgfqpoint{0.982536in}{1.506607in}}{\pgfqpoint{0.978146in}{1.496008in}}{\pgfqpoint{0.978146in}{1.484958in}}%
\pgfpathcurveto{\pgfqpoint{0.978146in}{1.473908in}}{\pgfqpoint{0.982536in}{1.463309in}}{\pgfqpoint{0.990350in}{1.455495in}}%
\pgfpathcurveto{\pgfqpoint{0.998163in}{1.447682in}}{\pgfqpoint{1.008762in}{1.443291in}}{\pgfqpoint{1.019812in}{1.443291in}}%
\pgfpathclose%
\pgfusepath{stroke,fill}%
\end{pgfscope}%
\begin{pgfscope}%
\pgfpathrectangle{\pgfqpoint{0.375000in}{0.330000in}}{\pgfqpoint{2.325000in}{2.310000in}}%
\pgfusepath{clip}%
\pgfsetbuttcap%
\pgfsetroundjoin%
\definecolor{currentfill}{rgb}{0.000000,0.000000,0.000000}%
\pgfsetfillcolor{currentfill}%
\pgfsetlinewidth{1.003750pt}%
\definecolor{currentstroke}{rgb}{0.000000,0.000000,0.000000}%
\pgfsetstrokecolor{currentstroke}%
\pgfsetdash{}{0pt}%
\pgfpathmoveto{\pgfqpoint{1.019812in}{1.443291in}}%
\pgfpathcurveto{\pgfqpoint{1.030863in}{1.443291in}}{\pgfqpoint{1.041462in}{1.447682in}}{\pgfqpoint{1.049275in}{1.455495in}}%
\pgfpathcurveto{\pgfqpoint{1.057089in}{1.463309in}}{\pgfqpoint{1.061479in}{1.473908in}}{\pgfqpoint{1.061479in}{1.484958in}}%
\pgfpathcurveto{\pgfqpoint{1.061479in}{1.496008in}}{\pgfqpoint{1.057089in}{1.506607in}}{\pgfqpoint{1.049275in}{1.514421in}}%
\pgfpathcurveto{\pgfqpoint{1.041462in}{1.522235in}}{\pgfqpoint{1.030863in}{1.526625in}}{\pgfqpoint{1.019812in}{1.526625in}}%
\pgfpathcurveto{\pgfqpoint{1.008762in}{1.526625in}}{\pgfqpoint{0.998163in}{1.522235in}}{\pgfqpoint{0.990350in}{1.514421in}}%
\pgfpathcurveto{\pgfqpoint{0.982536in}{1.506607in}}{\pgfqpoint{0.978146in}{1.496008in}}{\pgfqpoint{0.978146in}{1.484958in}}%
\pgfpathcurveto{\pgfqpoint{0.978146in}{1.473908in}}{\pgfqpoint{0.982536in}{1.463309in}}{\pgfqpoint{0.990350in}{1.455495in}}%
\pgfpathcurveto{\pgfqpoint{0.998163in}{1.447682in}}{\pgfqpoint{1.008762in}{1.443291in}}{\pgfqpoint{1.019812in}{1.443291in}}%
\pgfpathclose%
\pgfusepath{stroke,fill}%
\end{pgfscope}%
\begin{pgfscope}%
\pgfpathrectangle{\pgfqpoint{0.375000in}{0.330000in}}{\pgfqpoint{2.325000in}{2.310000in}}%
\pgfusepath{clip}%
\pgfsetbuttcap%
\pgfsetroundjoin%
\definecolor{currentfill}{rgb}{0.000000,0.000000,0.000000}%
\pgfsetfillcolor{currentfill}%
\pgfsetlinewidth{1.003750pt}%
\definecolor{currentstroke}{rgb}{0.000000,0.000000,0.000000}%
\pgfsetstrokecolor{currentstroke}%
\pgfsetdash{}{0pt}%
\pgfpathmoveto{\pgfqpoint{1.019812in}{1.443291in}}%
\pgfpathcurveto{\pgfqpoint{1.030863in}{1.443291in}}{\pgfqpoint{1.041462in}{1.447682in}}{\pgfqpoint{1.049275in}{1.455495in}}%
\pgfpathcurveto{\pgfqpoint{1.057089in}{1.463309in}}{\pgfqpoint{1.061479in}{1.473908in}}{\pgfqpoint{1.061479in}{1.484958in}}%
\pgfpathcurveto{\pgfqpoint{1.061479in}{1.496008in}}{\pgfqpoint{1.057089in}{1.506607in}}{\pgfqpoint{1.049275in}{1.514421in}}%
\pgfpathcurveto{\pgfqpoint{1.041462in}{1.522235in}}{\pgfqpoint{1.030863in}{1.526625in}}{\pgfqpoint{1.019812in}{1.526625in}}%
\pgfpathcurveto{\pgfqpoint{1.008762in}{1.526625in}}{\pgfqpoint{0.998163in}{1.522235in}}{\pgfqpoint{0.990350in}{1.514421in}}%
\pgfpathcurveto{\pgfqpoint{0.982536in}{1.506607in}}{\pgfqpoint{0.978146in}{1.496008in}}{\pgfqpoint{0.978146in}{1.484958in}}%
\pgfpathcurveto{\pgfqpoint{0.978146in}{1.473908in}}{\pgfqpoint{0.982536in}{1.463309in}}{\pgfqpoint{0.990350in}{1.455495in}}%
\pgfpathcurveto{\pgfqpoint{0.998163in}{1.447682in}}{\pgfqpoint{1.008762in}{1.443291in}}{\pgfqpoint{1.019812in}{1.443291in}}%
\pgfpathclose%
\pgfusepath{stroke,fill}%
\end{pgfscope}%
\begin{pgfscope}%
\pgfpathrectangle{\pgfqpoint{0.375000in}{0.330000in}}{\pgfqpoint{2.325000in}{2.310000in}}%
\pgfusepath{clip}%
\pgfsetbuttcap%
\pgfsetroundjoin%
\definecolor{currentfill}{rgb}{0.000000,0.000000,0.000000}%
\pgfsetfillcolor{currentfill}%
\pgfsetlinewidth{1.003750pt}%
\definecolor{currentstroke}{rgb}{0.000000,0.000000,0.000000}%
\pgfsetstrokecolor{currentstroke}%
\pgfsetdash{}{0pt}%
\pgfpathmoveto{\pgfqpoint{1.019812in}{2.474583in}}%
\pgfpathcurveto{\pgfqpoint{1.030863in}{2.474583in}}{\pgfqpoint{1.041462in}{2.478974in}}{\pgfqpoint{1.049275in}{2.486787in}}%
\pgfpathcurveto{\pgfqpoint{1.057089in}{2.494601in}}{\pgfqpoint{1.061479in}{2.505200in}}{\pgfqpoint{1.061479in}{2.516250in}}%
\pgfpathcurveto{\pgfqpoint{1.061479in}{2.527300in}}{\pgfqpoint{1.057089in}{2.537899in}}{\pgfqpoint{1.049275in}{2.545713in}}%
\pgfpathcurveto{\pgfqpoint{1.041462in}{2.553526in}}{\pgfqpoint{1.030863in}{2.557917in}}{\pgfqpoint{1.019812in}{2.557917in}}%
\pgfpathcurveto{\pgfqpoint{1.008762in}{2.557917in}}{\pgfqpoint{0.998163in}{2.553526in}}{\pgfqpoint{0.990350in}{2.545713in}}%
\pgfpathcurveto{\pgfqpoint{0.982536in}{2.537899in}}{\pgfqpoint{0.978146in}{2.527300in}}{\pgfqpoint{0.978146in}{2.516250in}}%
\pgfpathcurveto{\pgfqpoint{0.978146in}{2.505200in}}{\pgfqpoint{0.982536in}{2.494601in}}{\pgfqpoint{0.990350in}{2.486787in}}%
\pgfpathcurveto{\pgfqpoint{0.998163in}{2.478974in}}{\pgfqpoint{1.008762in}{2.474583in}}{\pgfqpoint{1.019812in}{2.474583in}}%
\pgfpathclose%
\pgfusepath{stroke,fill}%
\end{pgfscope}%
\begin{pgfscope}%
\pgfpathrectangle{\pgfqpoint{0.375000in}{0.330000in}}{\pgfqpoint{2.325000in}{2.310000in}}%
\pgfusepath{clip}%
\pgfsetbuttcap%
\pgfsetroundjoin%
\definecolor{currentfill}{rgb}{0.000000,0.000000,0.000000}%
\pgfsetfillcolor{currentfill}%
\pgfsetlinewidth{1.003750pt}%
\definecolor{currentstroke}{rgb}{0.000000,0.000000,0.000000}%
\pgfsetstrokecolor{currentstroke}%
\pgfsetdash{}{0pt}%
\pgfpathmoveto{\pgfqpoint{1.019812in}{1.443291in}}%
\pgfpathcurveto{\pgfqpoint{1.030863in}{1.443291in}}{\pgfqpoint{1.041462in}{1.447682in}}{\pgfqpoint{1.049275in}{1.455495in}}%
\pgfpathcurveto{\pgfqpoint{1.057089in}{1.463309in}}{\pgfqpoint{1.061479in}{1.473908in}}{\pgfqpoint{1.061479in}{1.484958in}}%
\pgfpathcurveto{\pgfqpoint{1.061479in}{1.496008in}}{\pgfqpoint{1.057089in}{1.506607in}}{\pgfqpoint{1.049275in}{1.514421in}}%
\pgfpathcurveto{\pgfqpoint{1.041462in}{1.522235in}}{\pgfqpoint{1.030863in}{1.526625in}}{\pgfqpoint{1.019812in}{1.526625in}}%
\pgfpathcurveto{\pgfqpoint{1.008762in}{1.526625in}}{\pgfqpoint{0.998163in}{1.522235in}}{\pgfqpoint{0.990350in}{1.514421in}}%
\pgfpathcurveto{\pgfqpoint{0.982536in}{1.506607in}}{\pgfqpoint{0.978146in}{1.496008in}}{\pgfqpoint{0.978146in}{1.484958in}}%
\pgfpathcurveto{\pgfqpoint{0.978146in}{1.473908in}}{\pgfqpoint{0.982536in}{1.463309in}}{\pgfqpoint{0.990350in}{1.455495in}}%
\pgfpathcurveto{\pgfqpoint{0.998163in}{1.447682in}}{\pgfqpoint{1.008762in}{1.443291in}}{\pgfqpoint{1.019812in}{1.443291in}}%
\pgfpathclose%
\pgfusepath{stroke,fill}%
\end{pgfscope}%
\begin{pgfscope}%
\pgfpathrectangle{\pgfqpoint{0.375000in}{0.330000in}}{\pgfqpoint{2.325000in}{2.310000in}}%
\pgfusepath{clip}%
\pgfsetbuttcap%
\pgfsetroundjoin%
\definecolor{currentfill}{rgb}{0.000000,0.000000,0.000000}%
\pgfsetfillcolor{currentfill}%
\pgfsetlinewidth{1.003750pt}%
\definecolor{currentstroke}{rgb}{0.000000,0.000000,0.000000}%
\pgfsetstrokecolor{currentstroke}%
\pgfsetdash{}{0pt}%
\pgfpathmoveto{\pgfqpoint{1.019812in}{1.443291in}}%
\pgfpathcurveto{\pgfqpoint{1.030863in}{1.443291in}}{\pgfqpoint{1.041462in}{1.447682in}}{\pgfqpoint{1.049275in}{1.455495in}}%
\pgfpathcurveto{\pgfqpoint{1.057089in}{1.463309in}}{\pgfqpoint{1.061479in}{1.473908in}}{\pgfqpoint{1.061479in}{1.484958in}}%
\pgfpathcurveto{\pgfqpoint{1.061479in}{1.496008in}}{\pgfqpoint{1.057089in}{1.506607in}}{\pgfqpoint{1.049275in}{1.514421in}}%
\pgfpathcurveto{\pgfqpoint{1.041462in}{1.522235in}}{\pgfqpoint{1.030863in}{1.526625in}}{\pgfqpoint{1.019812in}{1.526625in}}%
\pgfpathcurveto{\pgfqpoint{1.008762in}{1.526625in}}{\pgfqpoint{0.998163in}{1.522235in}}{\pgfqpoint{0.990350in}{1.514421in}}%
\pgfpathcurveto{\pgfqpoint{0.982536in}{1.506607in}}{\pgfqpoint{0.978146in}{1.496008in}}{\pgfqpoint{0.978146in}{1.484958in}}%
\pgfpathcurveto{\pgfqpoint{0.978146in}{1.473908in}}{\pgfqpoint{0.982536in}{1.463309in}}{\pgfqpoint{0.990350in}{1.455495in}}%
\pgfpathcurveto{\pgfqpoint{0.998163in}{1.447682in}}{\pgfqpoint{1.008762in}{1.443291in}}{\pgfqpoint{1.019812in}{1.443291in}}%
\pgfpathclose%
\pgfusepath{stroke,fill}%
\end{pgfscope}%
\begin{pgfscope}%
\pgfpathrectangle{\pgfqpoint{0.375000in}{0.330000in}}{\pgfqpoint{2.325000in}{2.310000in}}%
\pgfusepath{clip}%
\pgfsetbuttcap%
\pgfsetroundjoin%
\definecolor{currentfill}{rgb}{0.000000,0.000000,0.000000}%
\pgfsetfillcolor{currentfill}%
\pgfsetlinewidth{1.003750pt}%
\definecolor{currentstroke}{rgb}{0.000000,0.000000,0.000000}%
\pgfsetstrokecolor{currentstroke}%
\pgfsetdash{}{0pt}%
\pgfpathmoveto{\pgfqpoint{1.019812in}{1.443291in}}%
\pgfpathcurveto{\pgfqpoint{1.030863in}{1.443291in}}{\pgfqpoint{1.041462in}{1.447682in}}{\pgfqpoint{1.049275in}{1.455495in}}%
\pgfpathcurveto{\pgfqpoint{1.057089in}{1.463309in}}{\pgfqpoint{1.061479in}{1.473908in}}{\pgfqpoint{1.061479in}{1.484958in}}%
\pgfpathcurveto{\pgfqpoint{1.061479in}{1.496008in}}{\pgfqpoint{1.057089in}{1.506607in}}{\pgfqpoint{1.049275in}{1.514421in}}%
\pgfpathcurveto{\pgfqpoint{1.041462in}{1.522235in}}{\pgfqpoint{1.030863in}{1.526625in}}{\pgfqpoint{1.019812in}{1.526625in}}%
\pgfpathcurveto{\pgfqpoint{1.008762in}{1.526625in}}{\pgfqpoint{0.998163in}{1.522235in}}{\pgfqpoint{0.990350in}{1.514421in}}%
\pgfpathcurveto{\pgfqpoint{0.982536in}{1.506607in}}{\pgfqpoint{0.978146in}{1.496008in}}{\pgfqpoint{0.978146in}{1.484958in}}%
\pgfpathcurveto{\pgfqpoint{0.978146in}{1.473908in}}{\pgfqpoint{0.982536in}{1.463309in}}{\pgfqpoint{0.990350in}{1.455495in}}%
\pgfpathcurveto{\pgfqpoint{0.998163in}{1.447682in}}{\pgfqpoint{1.008762in}{1.443291in}}{\pgfqpoint{1.019812in}{1.443291in}}%
\pgfpathclose%
\pgfusepath{stroke,fill}%
\end{pgfscope}%
\begin{pgfscope}%
\pgfpathrectangle{\pgfqpoint{0.375000in}{0.330000in}}{\pgfqpoint{2.325000in}{2.310000in}}%
\pgfusepath{clip}%
\pgfsetbuttcap%
\pgfsetroundjoin%
\definecolor{currentfill}{rgb}{0.000000,0.000000,0.000000}%
\pgfsetfillcolor{currentfill}%
\pgfsetlinewidth{1.003750pt}%
\definecolor{currentstroke}{rgb}{0.000000,0.000000,0.000000}%
\pgfsetstrokecolor{currentstroke}%
\pgfsetdash{}{0pt}%
\pgfpathmoveto{\pgfqpoint{1.019812in}{1.443291in}}%
\pgfpathcurveto{\pgfqpoint{1.030863in}{1.443291in}}{\pgfqpoint{1.041462in}{1.447682in}}{\pgfqpoint{1.049275in}{1.455495in}}%
\pgfpathcurveto{\pgfqpoint{1.057089in}{1.463309in}}{\pgfqpoint{1.061479in}{1.473908in}}{\pgfqpoint{1.061479in}{1.484958in}}%
\pgfpathcurveto{\pgfqpoint{1.061479in}{1.496008in}}{\pgfqpoint{1.057089in}{1.506607in}}{\pgfqpoint{1.049275in}{1.514421in}}%
\pgfpathcurveto{\pgfqpoint{1.041462in}{1.522235in}}{\pgfqpoint{1.030863in}{1.526625in}}{\pgfqpoint{1.019812in}{1.526625in}}%
\pgfpathcurveto{\pgfqpoint{1.008762in}{1.526625in}}{\pgfqpoint{0.998163in}{1.522235in}}{\pgfqpoint{0.990350in}{1.514421in}}%
\pgfpathcurveto{\pgfqpoint{0.982536in}{1.506607in}}{\pgfqpoint{0.978146in}{1.496008in}}{\pgfqpoint{0.978146in}{1.484958in}}%
\pgfpathcurveto{\pgfqpoint{0.978146in}{1.473908in}}{\pgfqpoint{0.982536in}{1.463309in}}{\pgfqpoint{0.990350in}{1.455495in}}%
\pgfpathcurveto{\pgfqpoint{0.998163in}{1.447682in}}{\pgfqpoint{1.008762in}{1.443291in}}{\pgfqpoint{1.019812in}{1.443291in}}%
\pgfpathclose%
\pgfusepath{stroke,fill}%
\end{pgfscope}%
\begin{pgfscope}%
\pgfpathrectangle{\pgfqpoint{0.375000in}{0.330000in}}{\pgfqpoint{2.325000in}{2.310000in}}%
\pgfusepath{clip}%
\pgfsetbuttcap%
\pgfsetroundjoin%
\definecolor{currentfill}{rgb}{0.000000,0.000000,0.000000}%
\pgfsetfillcolor{currentfill}%
\pgfsetlinewidth{1.003750pt}%
\definecolor{currentstroke}{rgb}{0.000000,0.000000,0.000000}%
\pgfsetstrokecolor{currentstroke}%
\pgfsetdash{}{0pt}%
\pgfpathmoveto{\pgfqpoint{1.019812in}{0.412000in}}%
\pgfpathcurveto{\pgfqpoint{1.030863in}{0.412000in}}{\pgfqpoint{1.041462in}{0.416390in}}{\pgfqpoint{1.049275in}{0.424204in}}%
\pgfpathcurveto{\pgfqpoint{1.057089in}{0.432017in}}{\pgfqpoint{1.061479in}{0.442616in}}{\pgfqpoint{1.061479in}{0.453666in}}%
\pgfpathcurveto{\pgfqpoint{1.061479in}{0.464716in}}{\pgfqpoint{1.057089in}{0.475315in}}{\pgfqpoint{1.049275in}{0.483129in}}%
\pgfpathcurveto{\pgfqpoint{1.041462in}{0.490943in}}{\pgfqpoint{1.030863in}{0.495333in}}{\pgfqpoint{1.019812in}{0.495333in}}%
\pgfpathcurveto{\pgfqpoint{1.008762in}{0.495333in}}{\pgfqpoint{0.998163in}{0.490943in}}{\pgfqpoint{0.990350in}{0.483129in}}%
\pgfpathcurveto{\pgfqpoint{0.982536in}{0.475315in}}{\pgfqpoint{0.978146in}{0.464716in}}{\pgfqpoint{0.978146in}{0.453666in}}%
\pgfpathcurveto{\pgfqpoint{0.978146in}{0.442616in}}{\pgfqpoint{0.982536in}{0.432017in}}{\pgfqpoint{0.990350in}{0.424204in}}%
\pgfpathcurveto{\pgfqpoint{0.998163in}{0.416390in}}{\pgfqpoint{1.008762in}{0.412000in}}{\pgfqpoint{1.019812in}{0.412000in}}%
\pgfpathclose%
\pgfusepath{stroke,fill}%
\end{pgfscope}%
\begin{pgfscope}%
\pgfpathrectangle{\pgfqpoint{0.375000in}{0.330000in}}{\pgfqpoint{2.325000in}{2.310000in}}%
\pgfusepath{clip}%
\pgfsetbuttcap%
\pgfsetroundjoin%
\definecolor{currentfill}{rgb}{0.000000,0.000000,0.000000}%
\pgfsetfillcolor{currentfill}%
\pgfsetlinewidth{1.003750pt}%
\definecolor{currentstroke}{rgb}{0.000000,0.000000,0.000000}%
\pgfsetstrokecolor{currentstroke}%
\pgfsetdash{}{0pt}%
\pgfpathmoveto{\pgfqpoint{1.019812in}{2.474583in}}%
\pgfpathcurveto{\pgfqpoint{1.030863in}{2.474583in}}{\pgfqpoint{1.041462in}{2.478974in}}{\pgfqpoint{1.049275in}{2.486787in}}%
\pgfpathcurveto{\pgfqpoint{1.057089in}{2.494601in}}{\pgfqpoint{1.061479in}{2.505200in}}{\pgfqpoint{1.061479in}{2.516250in}}%
\pgfpathcurveto{\pgfqpoint{1.061479in}{2.527300in}}{\pgfqpoint{1.057089in}{2.537899in}}{\pgfqpoint{1.049275in}{2.545713in}}%
\pgfpathcurveto{\pgfqpoint{1.041462in}{2.553526in}}{\pgfqpoint{1.030863in}{2.557917in}}{\pgfqpoint{1.019812in}{2.557917in}}%
\pgfpathcurveto{\pgfqpoint{1.008762in}{2.557917in}}{\pgfqpoint{0.998163in}{2.553526in}}{\pgfqpoint{0.990350in}{2.545713in}}%
\pgfpathcurveto{\pgfqpoint{0.982536in}{2.537899in}}{\pgfqpoint{0.978146in}{2.527300in}}{\pgfqpoint{0.978146in}{2.516250in}}%
\pgfpathcurveto{\pgfqpoint{0.978146in}{2.505200in}}{\pgfqpoint{0.982536in}{2.494601in}}{\pgfqpoint{0.990350in}{2.486787in}}%
\pgfpathcurveto{\pgfqpoint{0.998163in}{2.478974in}}{\pgfqpoint{1.008762in}{2.474583in}}{\pgfqpoint{1.019812in}{2.474583in}}%
\pgfpathclose%
\pgfusepath{stroke,fill}%
\end{pgfscope}%
\begin{pgfscope}%
\pgfpathrectangle{\pgfqpoint{0.375000in}{0.330000in}}{\pgfqpoint{2.325000in}{2.310000in}}%
\pgfusepath{clip}%
\pgfsetbuttcap%
\pgfsetroundjoin%
\definecolor{currentfill}{rgb}{0.000000,0.000000,0.000000}%
\pgfsetfillcolor{currentfill}%
\pgfsetlinewidth{1.003750pt}%
\definecolor{currentstroke}{rgb}{0.000000,0.000000,0.000000}%
\pgfsetstrokecolor{currentstroke}%
\pgfsetdash{}{0pt}%
\pgfpathmoveto{\pgfqpoint{1.019812in}{1.443291in}}%
\pgfpathcurveto{\pgfqpoint{1.030863in}{1.443291in}}{\pgfqpoint{1.041462in}{1.447682in}}{\pgfqpoint{1.049275in}{1.455495in}}%
\pgfpathcurveto{\pgfqpoint{1.057089in}{1.463309in}}{\pgfqpoint{1.061479in}{1.473908in}}{\pgfqpoint{1.061479in}{1.484958in}}%
\pgfpathcurveto{\pgfqpoint{1.061479in}{1.496008in}}{\pgfqpoint{1.057089in}{1.506607in}}{\pgfqpoint{1.049275in}{1.514421in}}%
\pgfpathcurveto{\pgfqpoint{1.041462in}{1.522235in}}{\pgfqpoint{1.030863in}{1.526625in}}{\pgfqpoint{1.019812in}{1.526625in}}%
\pgfpathcurveto{\pgfqpoint{1.008762in}{1.526625in}}{\pgfqpoint{0.998163in}{1.522235in}}{\pgfqpoint{0.990350in}{1.514421in}}%
\pgfpathcurveto{\pgfqpoint{0.982536in}{1.506607in}}{\pgfqpoint{0.978146in}{1.496008in}}{\pgfqpoint{0.978146in}{1.484958in}}%
\pgfpathcurveto{\pgfqpoint{0.978146in}{1.473908in}}{\pgfqpoint{0.982536in}{1.463309in}}{\pgfqpoint{0.990350in}{1.455495in}}%
\pgfpathcurveto{\pgfqpoint{0.998163in}{1.447682in}}{\pgfqpoint{1.008762in}{1.443291in}}{\pgfqpoint{1.019812in}{1.443291in}}%
\pgfpathclose%
\pgfusepath{stroke,fill}%
\end{pgfscope}%
\begin{pgfscope}%
\pgfpathrectangle{\pgfqpoint{0.375000in}{0.330000in}}{\pgfqpoint{2.325000in}{2.310000in}}%
\pgfusepath{clip}%
\pgfsetbuttcap%
\pgfsetroundjoin%
\definecolor{currentfill}{rgb}{0.000000,0.000000,0.000000}%
\pgfsetfillcolor{currentfill}%
\pgfsetlinewidth{1.003750pt}%
\definecolor{currentstroke}{rgb}{0.000000,0.000000,0.000000}%
\pgfsetstrokecolor{currentstroke}%
\pgfsetdash{}{0pt}%
\pgfpathmoveto{\pgfqpoint{1.019812in}{1.443291in}}%
\pgfpathcurveto{\pgfqpoint{1.030863in}{1.443291in}}{\pgfqpoint{1.041462in}{1.447682in}}{\pgfqpoint{1.049275in}{1.455495in}}%
\pgfpathcurveto{\pgfqpoint{1.057089in}{1.463309in}}{\pgfqpoint{1.061479in}{1.473908in}}{\pgfqpoint{1.061479in}{1.484958in}}%
\pgfpathcurveto{\pgfqpoint{1.061479in}{1.496008in}}{\pgfqpoint{1.057089in}{1.506607in}}{\pgfqpoint{1.049275in}{1.514421in}}%
\pgfpathcurveto{\pgfqpoint{1.041462in}{1.522235in}}{\pgfqpoint{1.030863in}{1.526625in}}{\pgfqpoint{1.019812in}{1.526625in}}%
\pgfpathcurveto{\pgfqpoint{1.008762in}{1.526625in}}{\pgfqpoint{0.998163in}{1.522235in}}{\pgfqpoint{0.990350in}{1.514421in}}%
\pgfpathcurveto{\pgfqpoint{0.982536in}{1.506607in}}{\pgfqpoint{0.978146in}{1.496008in}}{\pgfqpoint{0.978146in}{1.484958in}}%
\pgfpathcurveto{\pgfqpoint{0.978146in}{1.473908in}}{\pgfqpoint{0.982536in}{1.463309in}}{\pgfqpoint{0.990350in}{1.455495in}}%
\pgfpathcurveto{\pgfqpoint{0.998163in}{1.447682in}}{\pgfqpoint{1.008762in}{1.443291in}}{\pgfqpoint{1.019812in}{1.443291in}}%
\pgfpathclose%
\pgfusepath{stroke,fill}%
\end{pgfscope}%
\begin{pgfscope}%
\pgfpathrectangle{\pgfqpoint{0.375000in}{0.330000in}}{\pgfqpoint{2.325000in}{2.310000in}}%
\pgfusepath{clip}%
\pgfsetbuttcap%
\pgfsetroundjoin%
\definecolor{currentfill}{rgb}{0.000000,0.000000,0.000000}%
\pgfsetfillcolor{currentfill}%
\pgfsetlinewidth{1.003750pt}%
\definecolor{currentstroke}{rgb}{0.000000,0.000000,0.000000}%
\pgfsetstrokecolor{currentstroke}%
\pgfsetdash{}{0pt}%
\pgfpathmoveto{\pgfqpoint{1.019812in}{1.443291in}}%
\pgfpathcurveto{\pgfqpoint{1.030863in}{1.443291in}}{\pgfqpoint{1.041462in}{1.447682in}}{\pgfqpoint{1.049275in}{1.455495in}}%
\pgfpathcurveto{\pgfqpoint{1.057089in}{1.463309in}}{\pgfqpoint{1.061479in}{1.473908in}}{\pgfqpoint{1.061479in}{1.484958in}}%
\pgfpathcurveto{\pgfqpoint{1.061479in}{1.496008in}}{\pgfqpoint{1.057089in}{1.506607in}}{\pgfqpoint{1.049275in}{1.514421in}}%
\pgfpathcurveto{\pgfqpoint{1.041462in}{1.522235in}}{\pgfqpoint{1.030863in}{1.526625in}}{\pgfqpoint{1.019812in}{1.526625in}}%
\pgfpathcurveto{\pgfqpoint{1.008762in}{1.526625in}}{\pgfqpoint{0.998163in}{1.522235in}}{\pgfqpoint{0.990350in}{1.514421in}}%
\pgfpathcurveto{\pgfqpoint{0.982536in}{1.506607in}}{\pgfqpoint{0.978146in}{1.496008in}}{\pgfqpoint{0.978146in}{1.484958in}}%
\pgfpathcurveto{\pgfqpoint{0.978146in}{1.473908in}}{\pgfqpoint{0.982536in}{1.463309in}}{\pgfqpoint{0.990350in}{1.455495in}}%
\pgfpathcurveto{\pgfqpoint{0.998163in}{1.447682in}}{\pgfqpoint{1.008762in}{1.443291in}}{\pgfqpoint{1.019812in}{1.443291in}}%
\pgfpathclose%
\pgfusepath{stroke,fill}%
\end{pgfscope}%
\begin{pgfscope}%
\pgfpathrectangle{\pgfqpoint{0.375000in}{0.330000in}}{\pgfqpoint{2.325000in}{2.310000in}}%
\pgfusepath{clip}%
\pgfsetbuttcap%
\pgfsetroundjoin%
\definecolor{currentfill}{rgb}{0.000000,0.000000,0.000000}%
\pgfsetfillcolor{currentfill}%
\pgfsetlinewidth{1.003750pt}%
\definecolor{currentstroke}{rgb}{0.000000,0.000000,0.000000}%
\pgfsetstrokecolor{currentstroke}%
\pgfsetdash{}{0pt}%
\pgfpathmoveto{\pgfqpoint{1.019812in}{1.443291in}}%
\pgfpathcurveto{\pgfqpoint{1.030863in}{1.443291in}}{\pgfqpoint{1.041462in}{1.447682in}}{\pgfqpoint{1.049275in}{1.455495in}}%
\pgfpathcurveto{\pgfqpoint{1.057089in}{1.463309in}}{\pgfqpoint{1.061479in}{1.473908in}}{\pgfqpoint{1.061479in}{1.484958in}}%
\pgfpathcurveto{\pgfqpoint{1.061479in}{1.496008in}}{\pgfqpoint{1.057089in}{1.506607in}}{\pgfqpoint{1.049275in}{1.514421in}}%
\pgfpathcurveto{\pgfqpoint{1.041462in}{1.522235in}}{\pgfqpoint{1.030863in}{1.526625in}}{\pgfqpoint{1.019812in}{1.526625in}}%
\pgfpathcurveto{\pgfqpoint{1.008762in}{1.526625in}}{\pgfqpoint{0.998163in}{1.522235in}}{\pgfqpoint{0.990350in}{1.514421in}}%
\pgfpathcurveto{\pgfqpoint{0.982536in}{1.506607in}}{\pgfqpoint{0.978146in}{1.496008in}}{\pgfqpoint{0.978146in}{1.484958in}}%
\pgfpathcurveto{\pgfqpoint{0.978146in}{1.473908in}}{\pgfqpoint{0.982536in}{1.463309in}}{\pgfqpoint{0.990350in}{1.455495in}}%
\pgfpathcurveto{\pgfqpoint{0.998163in}{1.447682in}}{\pgfqpoint{1.008762in}{1.443291in}}{\pgfqpoint{1.019812in}{1.443291in}}%
\pgfpathclose%
\pgfusepath{stroke,fill}%
\end{pgfscope}%
\begin{pgfscope}%
\pgfpathrectangle{\pgfqpoint{0.375000in}{0.330000in}}{\pgfqpoint{2.325000in}{2.310000in}}%
\pgfusepath{clip}%
\pgfsetbuttcap%
\pgfsetroundjoin%
\definecolor{currentfill}{rgb}{0.000000,0.000000,0.000000}%
\pgfsetfillcolor{currentfill}%
\pgfsetlinewidth{1.003750pt}%
\definecolor{currentstroke}{rgb}{0.000000,0.000000,0.000000}%
\pgfsetstrokecolor{currentstroke}%
\pgfsetdash{}{0pt}%
\pgfpathmoveto{\pgfqpoint{1.019812in}{1.443291in}}%
\pgfpathcurveto{\pgfqpoint{1.030863in}{1.443291in}}{\pgfqpoint{1.041462in}{1.447682in}}{\pgfqpoint{1.049275in}{1.455495in}}%
\pgfpathcurveto{\pgfqpoint{1.057089in}{1.463309in}}{\pgfqpoint{1.061479in}{1.473908in}}{\pgfqpoint{1.061479in}{1.484958in}}%
\pgfpathcurveto{\pgfqpoint{1.061479in}{1.496008in}}{\pgfqpoint{1.057089in}{1.506607in}}{\pgfqpoint{1.049275in}{1.514421in}}%
\pgfpathcurveto{\pgfqpoint{1.041462in}{1.522235in}}{\pgfqpoint{1.030863in}{1.526625in}}{\pgfqpoint{1.019812in}{1.526625in}}%
\pgfpathcurveto{\pgfqpoint{1.008762in}{1.526625in}}{\pgfqpoint{0.998163in}{1.522235in}}{\pgfqpoint{0.990350in}{1.514421in}}%
\pgfpathcurveto{\pgfqpoint{0.982536in}{1.506607in}}{\pgfqpoint{0.978146in}{1.496008in}}{\pgfqpoint{0.978146in}{1.484958in}}%
\pgfpathcurveto{\pgfqpoint{0.978146in}{1.473908in}}{\pgfqpoint{0.982536in}{1.463309in}}{\pgfqpoint{0.990350in}{1.455495in}}%
\pgfpathcurveto{\pgfqpoint{0.998163in}{1.447682in}}{\pgfqpoint{1.008762in}{1.443291in}}{\pgfqpoint{1.019812in}{1.443291in}}%
\pgfpathclose%
\pgfusepath{stroke,fill}%
\end{pgfscope}%
\begin{pgfscope}%
\pgfpathrectangle{\pgfqpoint{0.375000in}{0.330000in}}{\pgfqpoint{2.325000in}{2.310000in}}%
\pgfusepath{clip}%
\pgfsetbuttcap%
\pgfsetroundjoin%
\definecolor{currentfill}{rgb}{0.000000,0.000000,0.000000}%
\pgfsetfillcolor{currentfill}%
\pgfsetlinewidth{1.003750pt}%
\definecolor{currentstroke}{rgb}{0.000000,0.000000,0.000000}%
\pgfsetstrokecolor{currentstroke}%
\pgfsetdash{}{0pt}%
\pgfpathmoveto{\pgfqpoint{1.019812in}{1.443291in}}%
\pgfpathcurveto{\pgfqpoint{1.030863in}{1.443291in}}{\pgfqpoint{1.041462in}{1.447682in}}{\pgfqpoint{1.049275in}{1.455495in}}%
\pgfpathcurveto{\pgfqpoint{1.057089in}{1.463309in}}{\pgfqpoint{1.061479in}{1.473908in}}{\pgfqpoint{1.061479in}{1.484958in}}%
\pgfpathcurveto{\pgfqpoint{1.061479in}{1.496008in}}{\pgfqpoint{1.057089in}{1.506607in}}{\pgfqpoint{1.049275in}{1.514421in}}%
\pgfpathcurveto{\pgfqpoint{1.041462in}{1.522235in}}{\pgfqpoint{1.030863in}{1.526625in}}{\pgfqpoint{1.019812in}{1.526625in}}%
\pgfpathcurveto{\pgfqpoint{1.008762in}{1.526625in}}{\pgfqpoint{0.998163in}{1.522235in}}{\pgfqpoint{0.990350in}{1.514421in}}%
\pgfpathcurveto{\pgfqpoint{0.982536in}{1.506607in}}{\pgfqpoint{0.978146in}{1.496008in}}{\pgfqpoint{0.978146in}{1.484958in}}%
\pgfpathcurveto{\pgfqpoint{0.978146in}{1.473908in}}{\pgfqpoint{0.982536in}{1.463309in}}{\pgfqpoint{0.990350in}{1.455495in}}%
\pgfpathcurveto{\pgfqpoint{0.998163in}{1.447682in}}{\pgfqpoint{1.008762in}{1.443291in}}{\pgfqpoint{1.019812in}{1.443291in}}%
\pgfpathclose%
\pgfusepath{stroke,fill}%
\end{pgfscope}%
\begin{pgfscope}%
\pgfpathrectangle{\pgfqpoint{0.375000in}{0.330000in}}{\pgfqpoint{2.325000in}{2.310000in}}%
\pgfusepath{clip}%
\pgfsetbuttcap%
\pgfsetroundjoin%
\definecolor{currentfill}{rgb}{0.000000,0.000000,0.000000}%
\pgfsetfillcolor{currentfill}%
\pgfsetlinewidth{1.003750pt}%
\definecolor{currentstroke}{rgb}{0.000000,0.000000,0.000000}%
\pgfsetstrokecolor{currentstroke}%
\pgfsetdash{}{0pt}%
\pgfpathmoveto{\pgfqpoint{1.019812in}{1.443291in}}%
\pgfpathcurveto{\pgfqpoint{1.030863in}{1.443291in}}{\pgfqpoint{1.041462in}{1.447682in}}{\pgfqpoint{1.049275in}{1.455495in}}%
\pgfpathcurveto{\pgfqpoint{1.057089in}{1.463309in}}{\pgfqpoint{1.061479in}{1.473908in}}{\pgfqpoint{1.061479in}{1.484958in}}%
\pgfpathcurveto{\pgfqpoint{1.061479in}{1.496008in}}{\pgfqpoint{1.057089in}{1.506607in}}{\pgfqpoint{1.049275in}{1.514421in}}%
\pgfpathcurveto{\pgfqpoint{1.041462in}{1.522235in}}{\pgfqpoint{1.030863in}{1.526625in}}{\pgfqpoint{1.019812in}{1.526625in}}%
\pgfpathcurveto{\pgfqpoint{1.008762in}{1.526625in}}{\pgfqpoint{0.998163in}{1.522235in}}{\pgfqpoint{0.990350in}{1.514421in}}%
\pgfpathcurveto{\pgfqpoint{0.982536in}{1.506607in}}{\pgfqpoint{0.978146in}{1.496008in}}{\pgfqpoint{0.978146in}{1.484958in}}%
\pgfpathcurveto{\pgfqpoint{0.978146in}{1.473908in}}{\pgfqpoint{0.982536in}{1.463309in}}{\pgfqpoint{0.990350in}{1.455495in}}%
\pgfpathcurveto{\pgfqpoint{0.998163in}{1.447682in}}{\pgfqpoint{1.008762in}{1.443291in}}{\pgfqpoint{1.019812in}{1.443291in}}%
\pgfpathclose%
\pgfusepath{stroke,fill}%
\end{pgfscope}%
\begin{pgfscope}%
\pgfpathrectangle{\pgfqpoint{0.375000in}{0.330000in}}{\pgfqpoint{2.325000in}{2.310000in}}%
\pgfusepath{clip}%
\pgfsetbuttcap%
\pgfsetroundjoin%
\definecolor{currentfill}{rgb}{0.000000,0.000000,0.000000}%
\pgfsetfillcolor{currentfill}%
\pgfsetlinewidth{1.003750pt}%
\definecolor{currentstroke}{rgb}{0.000000,0.000000,0.000000}%
\pgfsetstrokecolor{currentstroke}%
\pgfsetdash{}{0pt}%
\pgfpathmoveto{\pgfqpoint{1.019812in}{1.443291in}}%
\pgfpathcurveto{\pgfqpoint{1.030863in}{1.443291in}}{\pgfqpoint{1.041462in}{1.447682in}}{\pgfqpoint{1.049275in}{1.455495in}}%
\pgfpathcurveto{\pgfqpoint{1.057089in}{1.463309in}}{\pgfqpoint{1.061479in}{1.473908in}}{\pgfqpoint{1.061479in}{1.484958in}}%
\pgfpathcurveto{\pgfqpoint{1.061479in}{1.496008in}}{\pgfqpoint{1.057089in}{1.506607in}}{\pgfqpoint{1.049275in}{1.514421in}}%
\pgfpathcurveto{\pgfqpoint{1.041462in}{1.522235in}}{\pgfqpoint{1.030863in}{1.526625in}}{\pgfqpoint{1.019812in}{1.526625in}}%
\pgfpathcurveto{\pgfqpoint{1.008762in}{1.526625in}}{\pgfqpoint{0.998163in}{1.522235in}}{\pgfqpoint{0.990350in}{1.514421in}}%
\pgfpathcurveto{\pgfqpoint{0.982536in}{1.506607in}}{\pgfqpoint{0.978146in}{1.496008in}}{\pgfqpoint{0.978146in}{1.484958in}}%
\pgfpathcurveto{\pgfqpoint{0.978146in}{1.473908in}}{\pgfqpoint{0.982536in}{1.463309in}}{\pgfqpoint{0.990350in}{1.455495in}}%
\pgfpathcurveto{\pgfqpoint{0.998163in}{1.447682in}}{\pgfqpoint{1.008762in}{1.443291in}}{\pgfqpoint{1.019812in}{1.443291in}}%
\pgfpathclose%
\pgfusepath{stroke,fill}%
\end{pgfscope}%
\begin{pgfscope}%
\pgfpathrectangle{\pgfqpoint{0.375000in}{0.330000in}}{\pgfqpoint{2.325000in}{2.310000in}}%
\pgfusepath{clip}%
\pgfsetbuttcap%
\pgfsetroundjoin%
\definecolor{currentfill}{rgb}{0.000000,0.000000,0.000000}%
\pgfsetfillcolor{currentfill}%
\pgfsetlinewidth{1.003750pt}%
\definecolor{currentstroke}{rgb}{0.000000,0.000000,0.000000}%
\pgfsetstrokecolor{currentstroke}%
\pgfsetdash{}{0pt}%
\pgfpathmoveto{\pgfqpoint{1.019812in}{1.443291in}}%
\pgfpathcurveto{\pgfqpoint{1.030863in}{1.443291in}}{\pgfqpoint{1.041462in}{1.447682in}}{\pgfqpoint{1.049275in}{1.455495in}}%
\pgfpathcurveto{\pgfqpoint{1.057089in}{1.463309in}}{\pgfqpoint{1.061479in}{1.473908in}}{\pgfqpoint{1.061479in}{1.484958in}}%
\pgfpathcurveto{\pgfqpoint{1.061479in}{1.496008in}}{\pgfqpoint{1.057089in}{1.506607in}}{\pgfqpoint{1.049275in}{1.514421in}}%
\pgfpathcurveto{\pgfqpoint{1.041462in}{1.522235in}}{\pgfqpoint{1.030863in}{1.526625in}}{\pgfqpoint{1.019812in}{1.526625in}}%
\pgfpathcurveto{\pgfqpoint{1.008762in}{1.526625in}}{\pgfqpoint{0.998163in}{1.522235in}}{\pgfqpoint{0.990350in}{1.514421in}}%
\pgfpathcurveto{\pgfqpoint{0.982536in}{1.506607in}}{\pgfqpoint{0.978146in}{1.496008in}}{\pgfqpoint{0.978146in}{1.484958in}}%
\pgfpathcurveto{\pgfqpoint{0.978146in}{1.473908in}}{\pgfqpoint{0.982536in}{1.463309in}}{\pgfqpoint{0.990350in}{1.455495in}}%
\pgfpathcurveto{\pgfqpoint{0.998163in}{1.447682in}}{\pgfqpoint{1.008762in}{1.443291in}}{\pgfqpoint{1.019812in}{1.443291in}}%
\pgfpathclose%
\pgfusepath{stroke,fill}%
\end{pgfscope}%
\begin{pgfscope}%
\pgfpathrectangle{\pgfqpoint{0.375000in}{0.330000in}}{\pgfqpoint{2.325000in}{2.310000in}}%
\pgfusepath{clip}%
\pgfsetbuttcap%
\pgfsetroundjoin%
\definecolor{currentfill}{rgb}{0.000000,0.000000,0.000000}%
\pgfsetfillcolor{currentfill}%
\pgfsetlinewidth{1.003750pt}%
\definecolor{currentstroke}{rgb}{0.000000,0.000000,0.000000}%
\pgfsetstrokecolor{currentstroke}%
\pgfsetdash{}{0pt}%
\pgfpathmoveto{\pgfqpoint{1.019812in}{1.443291in}}%
\pgfpathcurveto{\pgfqpoint{1.030863in}{1.443291in}}{\pgfqpoint{1.041462in}{1.447682in}}{\pgfqpoint{1.049275in}{1.455495in}}%
\pgfpathcurveto{\pgfqpoint{1.057089in}{1.463309in}}{\pgfqpoint{1.061479in}{1.473908in}}{\pgfqpoint{1.061479in}{1.484958in}}%
\pgfpathcurveto{\pgfqpoint{1.061479in}{1.496008in}}{\pgfqpoint{1.057089in}{1.506607in}}{\pgfqpoint{1.049275in}{1.514421in}}%
\pgfpathcurveto{\pgfqpoint{1.041462in}{1.522235in}}{\pgfqpoint{1.030863in}{1.526625in}}{\pgfqpoint{1.019812in}{1.526625in}}%
\pgfpathcurveto{\pgfqpoint{1.008762in}{1.526625in}}{\pgfqpoint{0.998163in}{1.522235in}}{\pgfqpoint{0.990350in}{1.514421in}}%
\pgfpathcurveto{\pgfqpoint{0.982536in}{1.506607in}}{\pgfqpoint{0.978146in}{1.496008in}}{\pgfqpoint{0.978146in}{1.484958in}}%
\pgfpathcurveto{\pgfqpoint{0.978146in}{1.473908in}}{\pgfqpoint{0.982536in}{1.463309in}}{\pgfqpoint{0.990350in}{1.455495in}}%
\pgfpathcurveto{\pgfqpoint{0.998163in}{1.447682in}}{\pgfqpoint{1.008762in}{1.443291in}}{\pgfqpoint{1.019812in}{1.443291in}}%
\pgfpathclose%
\pgfusepath{stroke,fill}%
\end{pgfscope}%
\begin{pgfscope}%
\pgfpathrectangle{\pgfqpoint{0.375000in}{0.330000in}}{\pgfqpoint{2.325000in}{2.310000in}}%
\pgfusepath{clip}%
\pgfsetbuttcap%
\pgfsetroundjoin%
\definecolor{currentfill}{rgb}{0.000000,0.000000,0.000000}%
\pgfsetfillcolor{currentfill}%
\pgfsetlinewidth{1.003750pt}%
\definecolor{currentstroke}{rgb}{0.000000,0.000000,0.000000}%
\pgfsetstrokecolor{currentstroke}%
\pgfsetdash{}{0pt}%
\pgfpathmoveto{\pgfqpoint{1.019812in}{1.443291in}}%
\pgfpathcurveto{\pgfqpoint{1.030863in}{1.443291in}}{\pgfqpoint{1.041462in}{1.447682in}}{\pgfqpoint{1.049275in}{1.455495in}}%
\pgfpathcurveto{\pgfqpoint{1.057089in}{1.463309in}}{\pgfqpoint{1.061479in}{1.473908in}}{\pgfqpoint{1.061479in}{1.484958in}}%
\pgfpathcurveto{\pgfqpoint{1.061479in}{1.496008in}}{\pgfqpoint{1.057089in}{1.506607in}}{\pgfqpoint{1.049275in}{1.514421in}}%
\pgfpathcurveto{\pgfqpoint{1.041462in}{1.522235in}}{\pgfqpoint{1.030863in}{1.526625in}}{\pgfqpoint{1.019812in}{1.526625in}}%
\pgfpathcurveto{\pgfqpoint{1.008762in}{1.526625in}}{\pgfqpoint{0.998163in}{1.522235in}}{\pgfqpoint{0.990350in}{1.514421in}}%
\pgfpathcurveto{\pgfqpoint{0.982536in}{1.506607in}}{\pgfqpoint{0.978146in}{1.496008in}}{\pgfqpoint{0.978146in}{1.484958in}}%
\pgfpathcurveto{\pgfqpoint{0.978146in}{1.473908in}}{\pgfqpoint{0.982536in}{1.463309in}}{\pgfqpoint{0.990350in}{1.455495in}}%
\pgfpathcurveto{\pgfqpoint{0.998163in}{1.447682in}}{\pgfqpoint{1.008762in}{1.443291in}}{\pgfqpoint{1.019812in}{1.443291in}}%
\pgfpathclose%
\pgfusepath{stroke,fill}%
\end{pgfscope}%
\begin{pgfscope}%
\pgfpathrectangle{\pgfqpoint{0.375000in}{0.330000in}}{\pgfqpoint{2.325000in}{2.310000in}}%
\pgfusepath{clip}%
\pgfsetbuttcap%
\pgfsetroundjoin%
\definecolor{currentfill}{rgb}{0.000000,0.000000,0.000000}%
\pgfsetfillcolor{currentfill}%
\pgfsetlinewidth{1.003750pt}%
\definecolor{currentstroke}{rgb}{0.000000,0.000000,0.000000}%
\pgfsetstrokecolor{currentstroke}%
\pgfsetdash{}{0pt}%
\pgfpathmoveto{\pgfqpoint{1.019812in}{2.474583in}}%
\pgfpathcurveto{\pgfqpoint{1.030863in}{2.474583in}}{\pgfqpoint{1.041462in}{2.478974in}}{\pgfqpoint{1.049275in}{2.486787in}}%
\pgfpathcurveto{\pgfqpoint{1.057089in}{2.494601in}}{\pgfqpoint{1.061479in}{2.505200in}}{\pgfqpoint{1.061479in}{2.516250in}}%
\pgfpathcurveto{\pgfqpoint{1.061479in}{2.527300in}}{\pgfqpoint{1.057089in}{2.537899in}}{\pgfqpoint{1.049275in}{2.545713in}}%
\pgfpathcurveto{\pgfqpoint{1.041462in}{2.553526in}}{\pgfqpoint{1.030863in}{2.557917in}}{\pgfqpoint{1.019812in}{2.557917in}}%
\pgfpathcurveto{\pgfqpoint{1.008762in}{2.557917in}}{\pgfqpoint{0.998163in}{2.553526in}}{\pgfqpoint{0.990350in}{2.545713in}}%
\pgfpathcurveto{\pgfqpoint{0.982536in}{2.537899in}}{\pgfqpoint{0.978146in}{2.527300in}}{\pgfqpoint{0.978146in}{2.516250in}}%
\pgfpathcurveto{\pgfqpoint{0.978146in}{2.505200in}}{\pgfqpoint{0.982536in}{2.494601in}}{\pgfqpoint{0.990350in}{2.486787in}}%
\pgfpathcurveto{\pgfqpoint{0.998163in}{2.478974in}}{\pgfqpoint{1.008762in}{2.474583in}}{\pgfqpoint{1.019812in}{2.474583in}}%
\pgfpathclose%
\pgfusepath{stroke,fill}%
\end{pgfscope}%
\begin{pgfscope}%
\pgfpathrectangle{\pgfqpoint{0.375000in}{0.330000in}}{\pgfqpoint{2.325000in}{2.310000in}}%
\pgfusepath{clip}%
\pgfsetbuttcap%
\pgfsetroundjoin%
\definecolor{currentfill}{rgb}{0.000000,0.000000,0.000000}%
\pgfsetfillcolor{currentfill}%
\pgfsetlinewidth{1.003750pt}%
\definecolor{currentstroke}{rgb}{0.000000,0.000000,0.000000}%
\pgfsetstrokecolor{currentstroke}%
\pgfsetdash{}{0pt}%
\pgfpathmoveto{\pgfqpoint{1.019812in}{1.443291in}}%
\pgfpathcurveto{\pgfqpoint{1.030863in}{1.443291in}}{\pgfqpoint{1.041462in}{1.447682in}}{\pgfqpoint{1.049275in}{1.455495in}}%
\pgfpathcurveto{\pgfqpoint{1.057089in}{1.463309in}}{\pgfqpoint{1.061479in}{1.473908in}}{\pgfqpoint{1.061479in}{1.484958in}}%
\pgfpathcurveto{\pgfqpoint{1.061479in}{1.496008in}}{\pgfqpoint{1.057089in}{1.506607in}}{\pgfqpoint{1.049275in}{1.514421in}}%
\pgfpathcurveto{\pgfqpoint{1.041462in}{1.522235in}}{\pgfqpoint{1.030863in}{1.526625in}}{\pgfqpoint{1.019812in}{1.526625in}}%
\pgfpathcurveto{\pgfqpoint{1.008762in}{1.526625in}}{\pgfqpoint{0.998163in}{1.522235in}}{\pgfqpoint{0.990350in}{1.514421in}}%
\pgfpathcurveto{\pgfqpoint{0.982536in}{1.506607in}}{\pgfqpoint{0.978146in}{1.496008in}}{\pgfqpoint{0.978146in}{1.484958in}}%
\pgfpathcurveto{\pgfqpoint{0.978146in}{1.473908in}}{\pgfqpoint{0.982536in}{1.463309in}}{\pgfqpoint{0.990350in}{1.455495in}}%
\pgfpathcurveto{\pgfqpoint{0.998163in}{1.447682in}}{\pgfqpoint{1.008762in}{1.443291in}}{\pgfqpoint{1.019812in}{1.443291in}}%
\pgfpathclose%
\pgfusepath{stroke,fill}%
\end{pgfscope}%
\begin{pgfscope}%
\pgfpathrectangle{\pgfqpoint{0.375000in}{0.330000in}}{\pgfqpoint{2.325000in}{2.310000in}}%
\pgfusepath{clip}%
\pgfsetbuttcap%
\pgfsetroundjoin%
\definecolor{currentfill}{rgb}{0.000000,0.000000,0.000000}%
\pgfsetfillcolor{currentfill}%
\pgfsetlinewidth{1.003750pt}%
\definecolor{currentstroke}{rgb}{0.000000,0.000000,0.000000}%
\pgfsetstrokecolor{currentstroke}%
\pgfsetdash{}{0pt}%
\pgfpathmoveto{\pgfqpoint{1.019812in}{1.443291in}}%
\pgfpathcurveto{\pgfqpoint{1.030863in}{1.443291in}}{\pgfqpoint{1.041462in}{1.447682in}}{\pgfqpoint{1.049275in}{1.455495in}}%
\pgfpathcurveto{\pgfqpoint{1.057089in}{1.463309in}}{\pgfqpoint{1.061479in}{1.473908in}}{\pgfqpoint{1.061479in}{1.484958in}}%
\pgfpathcurveto{\pgfqpoint{1.061479in}{1.496008in}}{\pgfqpoint{1.057089in}{1.506607in}}{\pgfqpoint{1.049275in}{1.514421in}}%
\pgfpathcurveto{\pgfqpoint{1.041462in}{1.522235in}}{\pgfqpoint{1.030863in}{1.526625in}}{\pgfqpoint{1.019812in}{1.526625in}}%
\pgfpathcurveto{\pgfqpoint{1.008762in}{1.526625in}}{\pgfqpoint{0.998163in}{1.522235in}}{\pgfqpoint{0.990350in}{1.514421in}}%
\pgfpathcurveto{\pgfqpoint{0.982536in}{1.506607in}}{\pgfqpoint{0.978146in}{1.496008in}}{\pgfqpoint{0.978146in}{1.484958in}}%
\pgfpathcurveto{\pgfqpoint{0.978146in}{1.473908in}}{\pgfqpoint{0.982536in}{1.463309in}}{\pgfqpoint{0.990350in}{1.455495in}}%
\pgfpathcurveto{\pgfqpoint{0.998163in}{1.447682in}}{\pgfqpoint{1.008762in}{1.443291in}}{\pgfqpoint{1.019812in}{1.443291in}}%
\pgfpathclose%
\pgfusepath{stroke,fill}%
\end{pgfscope}%
\begin{pgfscope}%
\pgfpathrectangle{\pgfqpoint{0.375000in}{0.330000in}}{\pgfqpoint{2.325000in}{2.310000in}}%
\pgfusepath{clip}%
\pgfsetbuttcap%
\pgfsetroundjoin%
\definecolor{currentfill}{rgb}{0.000000,0.000000,0.000000}%
\pgfsetfillcolor{currentfill}%
\pgfsetlinewidth{1.003750pt}%
\definecolor{currentstroke}{rgb}{0.000000,0.000000,0.000000}%
\pgfsetstrokecolor{currentstroke}%
\pgfsetdash{}{0pt}%
\pgfpathmoveto{\pgfqpoint{1.019812in}{1.443291in}}%
\pgfpathcurveto{\pgfqpoint{1.030863in}{1.443291in}}{\pgfqpoint{1.041462in}{1.447682in}}{\pgfqpoint{1.049275in}{1.455495in}}%
\pgfpathcurveto{\pgfqpoint{1.057089in}{1.463309in}}{\pgfqpoint{1.061479in}{1.473908in}}{\pgfqpoint{1.061479in}{1.484958in}}%
\pgfpathcurveto{\pgfqpoint{1.061479in}{1.496008in}}{\pgfqpoint{1.057089in}{1.506607in}}{\pgfqpoint{1.049275in}{1.514421in}}%
\pgfpathcurveto{\pgfqpoint{1.041462in}{1.522235in}}{\pgfqpoint{1.030863in}{1.526625in}}{\pgfqpoint{1.019812in}{1.526625in}}%
\pgfpathcurveto{\pgfqpoint{1.008762in}{1.526625in}}{\pgfqpoint{0.998163in}{1.522235in}}{\pgfqpoint{0.990350in}{1.514421in}}%
\pgfpathcurveto{\pgfqpoint{0.982536in}{1.506607in}}{\pgfqpoint{0.978146in}{1.496008in}}{\pgfqpoint{0.978146in}{1.484958in}}%
\pgfpathcurveto{\pgfqpoint{0.978146in}{1.473908in}}{\pgfqpoint{0.982536in}{1.463309in}}{\pgfqpoint{0.990350in}{1.455495in}}%
\pgfpathcurveto{\pgfqpoint{0.998163in}{1.447682in}}{\pgfqpoint{1.008762in}{1.443291in}}{\pgfqpoint{1.019812in}{1.443291in}}%
\pgfpathclose%
\pgfusepath{stroke,fill}%
\end{pgfscope}%
\begin{pgfscope}%
\pgfpathrectangle{\pgfqpoint{0.375000in}{0.330000in}}{\pgfqpoint{2.325000in}{2.310000in}}%
\pgfusepath{clip}%
\pgfsetbuttcap%
\pgfsetroundjoin%
\definecolor{currentfill}{rgb}{0.000000,0.000000,0.000000}%
\pgfsetfillcolor{currentfill}%
\pgfsetlinewidth{1.003750pt}%
\definecolor{currentstroke}{rgb}{0.000000,0.000000,0.000000}%
\pgfsetstrokecolor{currentstroke}%
\pgfsetdash{}{0pt}%
\pgfpathmoveto{\pgfqpoint{1.019812in}{1.443291in}}%
\pgfpathcurveto{\pgfqpoint{1.030863in}{1.443291in}}{\pgfqpoint{1.041462in}{1.447682in}}{\pgfqpoint{1.049275in}{1.455495in}}%
\pgfpathcurveto{\pgfqpoint{1.057089in}{1.463309in}}{\pgfqpoint{1.061479in}{1.473908in}}{\pgfqpoint{1.061479in}{1.484958in}}%
\pgfpathcurveto{\pgfqpoint{1.061479in}{1.496008in}}{\pgfqpoint{1.057089in}{1.506607in}}{\pgfqpoint{1.049275in}{1.514421in}}%
\pgfpathcurveto{\pgfqpoint{1.041462in}{1.522235in}}{\pgfqpoint{1.030863in}{1.526625in}}{\pgfqpoint{1.019812in}{1.526625in}}%
\pgfpathcurveto{\pgfqpoint{1.008762in}{1.526625in}}{\pgfqpoint{0.998163in}{1.522235in}}{\pgfqpoint{0.990350in}{1.514421in}}%
\pgfpathcurveto{\pgfqpoint{0.982536in}{1.506607in}}{\pgfqpoint{0.978146in}{1.496008in}}{\pgfqpoint{0.978146in}{1.484958in}}%
\pgfpathcurveto{\pgfqpoint{0.978146in}{1.473908in}}{\pgfqpoint{0.982536in}{1.463309in}}{\pgfqpoint{0.990350in}{1.455495in}}%
\pgfpathcurveto{\pgfqpoint{0.998163in}{1.447682in}}{\pgfqpoint{1.008762in}{1.443291in}}{\pgfqpoint{1.019812in}{1.443291in}}%
\pgfpathclose%
\pgfusepath{stroke,fill}%
\end{pgfscope}%
\begin{pgfscope}%
\pgfpathrectangle{\pgfqpoint{0.375000in}{0.330000in}}{\pgfqpoint{2.325000in}{2.310000in}}%
\pgfusepath{clip}%
\pgfsetbuttcap%
\pgfsetroundjoin%
\definecolor{currentfill}{rgb}{0.000000,0.000000,0.000000}%
\pgfsetfillcolor{currentfill}%
\pgfsetlinewidth{1.003750pt}%
\definecolor{currentstroke}{rgb}{0.000000,0.000000,0.000000}%
\pgfsetstrokecolor{currentstroke}%
\pgfsetdash{}{0pt}%
\pgfpathmoveto{\pgfqpoint{1.019812in}{1.443291in}}%
\pgfpathcurveto{\pgfqpoint{1.030863in}{1.443291in}}{\pgfqpoint{1.041462in}{1.447682in}}{\pgfqpoint{1.049275in}{1.455495in}}%
\pgfpathcurveto{\pgfqpoint{1.057089in}{1.463309in}}{\pgfqpoint{1.061479in}{1.473908in}}{\pgfqpoint{1.061479in}{1.484958in}}%
\pgfpathcurveto{\pgfqpoint{1.061479in}{1.496008in}}{\pgfqpoint{1.057089in}{1.506607in}}{\pgfqpoint{1.049275in}{1.514421in}}%
\pgfpathcurveto{\pgfqpoint{1.041462in}{1.522235in}}{\pgfqpoint{1.030863in}{1.526625in}}{\pgfqpoint{1.019812in}{1.526625in}}%
\pgfpathcurveto{\pgfqpoint{1.008762in}{1.526625in}}{\pgfqpoint{0.998163in}{1.522235in}}{\pgfqpoint{0.990350in}{1.514421in}}%
\pgfpathcurveto{\pgfqpoint{0.982536in}{1.506607in}}{\pgfqpoint{0.978146in}{1.496008in}}{\pgfqpoint{0.978146in}{1.484958in}}%
\pgfpathcurveto{\pgfqpoint{0.978146in}{1.473908in}}{\pgfqpoint{0.982536in}{1.463309in}}{\pgfqpoint{0.990350in}{1.455495in}}%
\pgfpathcurveto{\pgfqpoint{0.998163in}{1.447682in}}{\pgfqpoint{1.008762in}{1.443291in}}{\pgfqpoint{1.019812in}{1.443291in}}%
\pgfpathclose%
\pgfusepath{stroke,fill}%
\end{pgfscope}%
\begin{pgfscope}%
\pgfpathrectangle{\pgfqpoint{0.375000in}{0.330000in}}{\pgfqpoint{2.325000in}{2.310000in}}%
\pgfusepath{clip}%
\pgfsetbuttcap%
\pgfsetroundjoin%
\definecolor{currentfill}{rgb}{0.000000,0.000000,0.000000}%
\pgfsetfillcolor{currentfill}%
\pgfsetlinewidth{1.003750pt}%
\definecolor{currentstroke}{rgb}{0.000000,0.000000,0.000000}%
\pgfsetstrokecolor{currentstroke}%
\pgfsetdash{}{0pt}%
\pgfpathmoveto{\pgfqpoint{1.019812in}{2.474583in}}%
\pgfpathcurveto{\pgfqpoint{1.030863in}{2.474583in}}{\pgfqpoint{1.041462in}{2.478974in}}{\pgfqpoint{1.049275in}{2.486787in}}%
\pgfpathcurveto{\pgfqpoint{1.057089in}{2.494601in}}{\pgfqpoint{1.061479in}{2.505200in}}{\pgfqpoint{1.061479in}{2.516250in}}%
\pgfpathcurveto{\pgfqpoint{1.061479in}{2.527300in}}{\pgfqpoint{1.057089in}{2.537899in}}{\pgfqpoint{1.049275in}{2.545713in}}%
\pgfpathcurveto{\pgfqpoint{1.041462in}{2.553526in}}{\pgfqpoint{1.030863in}{2.557917in}}{\pgfqpoint{1.019812in}{2.557917in}}%
\pgfpathcurveto{\pgfqpoint{1.008762in}{2.557917in}}{\pgfqpoint{0.998163in}{2.553526in}}{\pgfqpoint{0.990350in}{2.545713in}}%
\pgfpathcurveto{\pgfqpoint{0.982536in}{2.537899in}}{\pgfqpoint{0.978146in}{2.527300in}}{\pgfqpoint{0.978146in}{2.516250in}}%
\pgfpathcurveto{\pgfqpoint{0.978146in}{2.505200in}}{\pgfqpoint{0.982536in}{2.494601in}}{\pgfqpoint{0.990350in}{2.486787in}}%
\pgfpathcurveto{\pgfqpoint{0.998163in}{2.478974in}}{\pgfqpoint{1.008762in}{2.474583in}}{\pgfqpoint{1.019812in}{2.474583in}}%
\pgfpathclose%
\pgfusepath{stroke,fill}%
\end{pgfscope}%
\begin{pgfscope}%
\pgfpathrectangle{\pgfqpoint{0.375000in}{0.330000in}}{\pgfqpoint{2.325000in}{2.310000in}}%
\pgfusepath{clip}%
\pgfsetbuttcap%
\pgfsetroundjoin%
\definecolor{currentfill}{rgb}{0.000000,0.000000,0.000000}%
\pgfsetfillcolor{currentfill}%
\pgfsetlinewidth{1.003750pt}%
\definecolor{currentstroke}{rgb}{0.000000,0.000000,0.000000}%
\pgfsetstrokecolor{currentstroke}%
\pgfsetdash{}{0pt}%
\pgfpathmoveto{\pgfqpoint{1.019812in}{1.443291in}}%
\pgfpathcurveto{\pgfqpoint{1.030863in}{1.443291in}}{\pgfqpoint{1.041462in}{1.447682in}}{\pgfqpoint{1.049275in}{1.455495in}}%
\pgfpathcurveto{\pgfqpoint{1.057089in}{1.463309in}}{\pgfqpoint{1.061479in}{1.473908in}}{\pgfqpoint{1.061479in}{1.484958in}}%
\pgfpathcurveto{\pgfqpoint{1.061479in}{1.496008in}}{\pgfqpoint{1.057089in}{1.506607in}}{\pgfqpoint{1.049275in}{1.514421in}}%
\pgfpathcurveto{\pgfqpoint{1.041462in}{1.522235in}}{\pgfqpoint{1.030863in}{1.526625in}}{\pgfqpoint{1.019812in}{1.526625in}}%
\pgfpathcurveto{\pgfqpoint{1.008762in}{1.526625in}}{\pgfqpoint{0.998163in}{1.522235in}}{\pgfqpoint{0.990350in}{1.514421in}}%
\pgfpathcurveto{\pgfqpoint{0.982536in}{1.506607in}}{\pgfqpoint{0.978146in}{1.496008in}}{\pgfqpoint{0.978146in}{1.484958in}}%
\pgfpathcurveto{\pgfqpoint{0.978146in}{1.473908in}}{\pgfqpoint{0.982536in}{1.463309in}}{\pgfqpoint{0.990350in}{1.455495in}}%
\pgfpathcurveto{\pgfqpoint{0.998163in}{1.447682in}}{\pgfqpoint{1.008762in}{1.443291in}}{\pgfqpoint{1.019812in}{1.443291in}}%
\pgfpathclose%
\pgfusepath{stroke,fill}%
\end{pgfscope}%
\begin{pgfscope}%
\pgfpathrectangle{\pgfqpoint{0.375000in}{0.330000in}}{\pgfqpoint{2.325000in}{2.310000in}}%
\pgfusepath{clip}%
\pgfsetbuttcap%
\pgfsetroundjoin%
\definecolor{currentfill}{rgb}{0.000000,0.000000,0.000000}%
\pgfsetfillcolor{currentfill}%
\pgfsetlinewidth{1.003750pt}%
\definecolor{currentstroke}{rgb}{0.000000,0.000000,0.000000}%
\pgfsetstrokecolor{currentstroke}%
\pgfsetdash{}{0pt}%
\pgfpathmoveto{\pgfqpoint{1.019812in}{1.443291in}}%
\pgfpathcurveto{\pgfqpoint{1.030863in}{1.443291in}}{\pgfqpoint{1.041462in}{1.447682in}}{\pgfqpoint{1.049275in}{1.455495in}}%
\pgfpathcurveto{\pgfqpoint{1.057089in}{1.463309in}}{\pgfqpoint{1.061479in}{1.473908in}}{\pgfqpoint{1.061479in}{1.484958in}}%
\pgfpathcurveto{\pgfqpoint{1.061479in}{1.496008in}}{\pgfqpoint{1.057089in}{1.506607in}}{\pgfqpoint{1.049275in}{1.514421in}}%
\pgfpathcurveto{\pgfqpoint{1.041462in}{1.522235in}}{\pgfqpoint{1.030863in}{1.526625in}}{\pgfqpoint{1.019812in}{1.526625in}}%
\pgfpathcurveto{\pgfqpoint{1.008762in}{1.526625in}}{\pgfqpoint{0.998163in}{1.522235in}}{\pgfqpoint{0.990350in}{1.514421in}}%
\pgfpathcurveto{\pgfqpoint{0.982536in}{1.506607in}}{\pgfqpoint{0.978146in}{1.496008in}}{\pgfqpoint{0.978146in}{1.484958in}}%
\pgfpathcurveto{\pgfqpoint{0.978146in}{1.473908in}}{\pgfqpoint{0.982536in}{1.463309in}}{\pgfqpoint{0.990350in}{1.455495in}}%
\pgfpathcurveto{\pgfqpoint{0.998163in}{1.447682in}}{\pgfqpoint{1.008762in}{1.443291in}}{\pgfqpoint{1.019812in}{1.443291in}}%
\pgfpathclose%
\pgfusepath{stroke,fill}%
\end{pgfscope}%
\begin{pgfscope}%
\pgfpathrectangle{\pgfqpoint{0.375000in}{0.330000in}}{\pgfqpoint{2.325000in}{2.310000in}}%
\pgfusepath{clip}%
\pgfsetbuttcap%
\pgfsetroundjoin%
\definecolor{currentfill}{rgb}{0.000000,0.000000,0.000000}%
\pgfsetfillcolor{currentfill}%
\pgfsetlinewidth{1.003750pt}%
\definecolor{currentstroke}{rgb}{0.000000,0.000000,0.000000}%
\pgfsetstrokecolor{currentstroke}%
\pgfsetdash{}{0pt}%
\pgfpathmoveto{\pgfqpoint{1.019812in}{1.443291in}}%
\pgfpathcurveto{\pgfqpoint{1.030863in}{1.443291in}}{\pgfqpoint{1.041462in}{1.447682in}}{\pgfqpoint{1.049275in}{1.455495in}}%
\pgfpathcurveto{\pgfqpoint{1.057089in}{1.463309in}}{\pgfqpoint{1.061479in}{1.473908in}}{\pgfqpoint{1.061479in}{1.484958in}}%
\pgfpathcurveto{\pgfqpoint{1.061479in}{1.496008in}}{\pgfqpoint{1.057089in}{1.506607in}}{\pgfqpoint{1.049275in}{1.514421in}}%
\pgfpathcurveto{\pgfqpoint{1.041462in}{1.522235in}}{\pgfqpoint{1.030863in}{1.526625in}}{\pgfqpoint{1.019812in}{1.526625in}}%
\pgfpathcurveto{\pgfqpoint{1.008762in}{1.526625in}}{\pgfqpoint{0.998163in}{1.522235in}}{\pgfqpoint{0.990350in}{1.514421in}}%
\pgfpathcurveto{\pgfqpoint{0.982536in}{1.506607in}}{\pgfqpoint{0.978146in}{1.496008in}}{\pgfqpoint{0.978146in}{1.484958in}}%
\pgfpathcurveto{\pgfqpoint{0.978146in}{1.473908in}}{\pgfqpoint{0.982536in}{1.463309in}}{\pgfqpoint{0.990350in}{1.455495in}}%
\pgfpathcurveto{\pgfqpoint{0.998163in}{1.447682in}}{\pgfqpoint{1.008762in}{1.443291in}}{\pgfqpoint{1.019812in}{1.443291in}}%
\pgfpathclose%
\pgfusepath{stroke,fill}%
\end{pgfscope}%
\begin{pgfscope}%
\pgfpathrectangle{\pgfqpoint{0.375000in}{0.330000in}}{\pgfqpoint{2.325000in}{2.310000in}}%
\pgfusepath{clip}%
\pgfsetbuttcap%
\pgfsetroundjoin%
\definecolor{currentfill}{rgb}{0.000000,0.000000,0.000000}%
\pgfsetfillcolor{currentfill}%
\pgfsetlinewidth{1.003750pt}%
\definecolor{currentstroke}{rgb}{0.000000,0.000000,0.000000}%
\pgfsetstrokecolor{currentstroke}%
\pgfsetdash{}{0pt}%
\pgfpathmoveto{\pgfqpoint{1.019812in}{1.443291in}}%
\pgfpathcurveto{\pgfqpoint{1.030863in}{1.443291in}}{\pgfqpoint{1.041462in}{1.447682in}}{\pgfqpoint{1.049275in}{1.455495in}}%
\pgfpathcurveto{\pgfqpoint{1.057089in}{1.463309in}}{\pgfqpoint{1.061479in}{1.473908in}}{\pgfqpoint{1.061479in}{1.484958in}}%
\pgfpathcurveto{\pgfqpoint{1.061479in}{1.496008in}}{\pgfqpoint{1.057089in}{1.506607in}}{\pgfqpoint{1.049275in}{1.514421in}}%
\pgfpathcurveto{\pgfqpoint{1.041462in}{1.522235in}}{\pgfqpoint{1.030863in}{1.526625in}}{\pgfqpoint{1.019812in}{1.526625in}}%
\pgfpathcurveto{\pgfqpoint{1.008762in}{1.526625in}}{\pgfqpoint{0.998163in}{1.522235in}}{\pgfqpoint{0.990350in}{1.514421in}}%
\pgfpathcurveto{\pgfqpoint{0.982536in}{1.506607in}}{\pgfqpoint{0.978146in}{1.496008in}}{\pgfqpoint{0.978146in}{1.484958in}}%
\pgfpathcurveto{\pgfqpoint{0.978146in}{1.473908in}}{\pgfqpoint{0.982536in}{1.463309in}}{\pgfqpoint{0.990350in}{1.455495in}}%
\pgfpathcurveto{\pgfqpoint{0.998163in}{1.447682in}}{\pgfqpoint{1.008762in}{1.443291in}}{\pgfqpoint{1.019812in}{1.443291in}}%
\pgfpathclose%
\pgfusepath{stroke,fill}%
\end{pgfscope}%
\begin{pgfscope}%
\pgfpathrectangle{\pgfqpoint{0.375000in}{0.330000in}}{\pgfqpoint{2.325000in}{2.310000in}}%
\pgfusepath{clip}%
\pgfsetbuttcap%
\pgfsetroundjoin%
\definecolor{currentfill}{rgb}{0.000000,0.000000,0.000000}%
\pgfsetfillcolor{currentfill}%
\pgfsetlinewidth{1.003750pt}%
\definecolor{currentstroke}{rgb}{0.000000,0.000000,0.000000}%
\pgfsetstrokecolor{currentstroke}%
\pgfsetdash{}{0pt}%
\pgfpathmoveto{\pgfqpoint{1.019812in}{1.443291in}}%
\pgfpathcurveto{\pgfqpoint{1.030863in}{1.443291in}}{\pgfqpoint{1.041462in}{1.447682in}}{\pgfqpoint{1.049275in}{1.455495in}}%
\pgfpathcurveto{\pgfqpoint{1.057089in}{1.463309in}}{\pgfqpoint{1.061479in}{1.473908in}}{\pgfqpoint{1.061479in}{1.484958in}}%
\pgfpathcurveto{\pgfqpoint{1.061479in}{1.496008in}}{\pgfqpoint{1.057089in}{1.506607in}}{\pgfqpoint{1.049275in}{1.514421in}}%
\pgfpathcurveto{\pgfqpoint{1.041462in}{1.522235in}}{\pgfqpoint{1.030863in}{1.526625in}}{\pgfqpoint{1.019812in}{1.526625in}}%
\pgfpathcurveto{\pgfqpoint{1.008762in}{1.526625in}}{\pgfqpoint{0.998163in}{1.522235in}}{\pgfqpoint{0.990350in}{1.514421in}}%
\pgfpathcurveto{\pgfqpoint{0.982536in}{1.506607in}}{\pgfqpoint{0.978146in}{1.496008in}}{\pgfqpoint{0.978146in}{1.484958in}}%
\pgfpathcurveto{\pgfqpoint{0.978146in}{1.473908in}}{\pgfqpoint{0.982536in}{1.463309in}}{\pgfqpoint{0.990350in}{1.455495in}}%
\pgfpathcurveto{\pgfqpoint{0.998163in}{1.447682in}}{\pgfqpoint{1.008762in}{1.443291in}}{\pgfqpoint{1.019812in}{1.443291in}}%
\pgfpathclose%
\pgfusepath{stroke,fill}%
\end{pgfscope}%
\begin{pgfscope}%
\pgfpathrectangle{\pgfqpoint{0.375000in}{0.330000in}}{\pgfqpoint{2.325000in}{2.310000in}}%
\pgfusepath{clip}%
\pgfsetbuttcap%
\pgfsetroundjoin%
\definecolor{currentfill}{rgb}{0.000000,0.000000,0.000000}%
\pgfsetfillcolor{currentfill}%
\pgfsetlinewidth{1.003750pt}%
\definecolor{currentstroke}{rgb}{0.000000,0.000000,0.000000}%
\pgfsetstrokecolor{currentstroke}%
\pgfsetdash{}{0pt}%
\pgfpathmoveto{\pgfqpoint{1.019812in}{2.474583in}}%
\pgfpathcurveto{\pgfqpoint{1.030863in}{2.474583in}}{\pgfqpoint{1.041462in}{2.478974in}}{\pgfqpoint{1.049275in}{2.486787in}}%
\pgfpathcurveto{\pgfqpoint{1.057089in}{2.494601in}}{\pgfqpoint{1.061479in}{2.505200in}}{\pgfqpoint{1.061479in}{2.516250in}}%
\pgfpathcurveto{\pgfqpoint{1.061479in}{2.527300in}}{\pgfqpoint{1.057089in}{2.537899in}}{\pgfqpoint{1.049275in}{2.545713in}}%
\pgfpathcurveto{\pgfqpoint{1.041462in}{2.553526in}}{\pgfqpoint{1.030863in}{2.557917in}}{\pgfqpoint{1.019812in}{2.557917in}}%
\pgfpathcurveto{\pgfqpoint{1.008762in}{2.557917in}}{\pgfqpoint{0.998163in}{2.553526in}}{\pgfqpoint{0.990350in}{2.545713in}}%
\pgfpathcurveto{\pgfqpoint{0.982536in}{2.537899in}}{\pgfqpoint{0.978146in}{2.527300in}}{\pgfqpoint{0.978146in}{2.516250in}}%
\pgfpathcurveto{\pgfqpoint{0.978146in}{2.505200in}}{\pgfqpoint{0.982536in}{2.494601in}}{\pgfqpoint{0.990350in}{2.486787in}}%
\pgfpathcurveto{\pgfqpoint{0.998163in}{2.478974in}}{\pgfqpoint{1.008762in}{2.474583in}}{\pgfqpoint{1.019812in}{2.474583in}}%
\pgfpathclose%
\pgfusepath{stroke,fill}%
\end{pgfscope}%
\begin{pgfscope}%
\pgfpathrectangle{\pgfqpoint{0.375000in}{0.330000in}}{\pgfqpoint{2.325000in}{2.310000in}}%
\pgfusepath{clip}%
\pgfsetbuttcap%
\pgfsetroundjoin%
\definecolor{currentfill}{rgb}{0.000000,0.000000,0.000000}%
\pgfsetfillcolor{currentfill}%
\pgfsetlinewidth{1.003750pt}%
\definecolor{currentstroke}{rgb}{0.000000,0.000000,0.000000}%
\pgfsetstrokecolor{currentstroke}%
\pgfsetdash{}{0pt}%
\pgfpathmoveto{\pgfqpoint{1.019812in}{1.443291in}}%
\pgfpathcurveto{\pgfqpoint{1.030863in}{1.443291in}}{\pgfqpoint{1.041462in}{1.447682in}}{\pgfqpoint{1.049275in}{1.455495in}}%
\pgfpathcurveto{\pgfqpoint{1.057089in}{1.463309in}}{\pgfqpoint{1.061479in}{1.473908in}}{\pgfqpoint{1.061479in}{1.484958in}}%
\pgfpathcurveto{\pgfqpoint{1.061479in}{1.496008in}}{\pgfqpoint{1.057089in}{1.506607in}}{\pgfqpoint{1.049275in}{1.514421in}}%
\pgfpathcurveto{\pgfqpoint{1.041462in}{1.522235in}}{\pgfqpoint{1.030863in}{1.526625in}}{\pgfqpoint{1.019812in}{1.526625in}}%
\pgfpathcurveto{\pgfqpoint{1.008762in}{1.526625in}}{\pgfqpoint{0.998163in}{1.522235in}}{\pgfqpoint{0.990350in}{1.514421in}}%
\pgfpathcurveto{\pgfqpoint{0.982536in}{1.506607in}}{\pgfqpoint{0.978146in}{1.496008in}}{\pgfqpoint{0.978146in}{1.484958in}}%
\pgfpathcurveto{\pgfqpoint{0.978146in}{1.473908in}}{\pgfqpoint{0.982536in}{1.463309in}}{\pgfqpoint{0.990350in}{1.455495in}}%
\pgfpathcurveto{\pgfqpoint{0.998163in}{1.447682in}}{\pgfqpoint{1.008762in}{1.443291in}}{\pgfqpoint{1.019812in}{1.443291in}}%
\pgfpathclose%
\pgfusepath{stroke,fill}%
\end{pgfscope}%
\begin{pgfscope}%
\pgfpathrectangle{\pgfqpoint{0.375000in}{0.330000in}}{\pgfqpoint{2.325000in}{2.310000in}}%
\pgfusepath{clip}%
\pgfsetbuttcap%
\pgfsetroundjoin%
\definecolor{currentfill}{rgb}{0.000000,0.000000,0.000000}%
\pgfsetfillcolor{currentfill}%
\pgfsetlinewidth{1.003750pt}%
\definecolor{currentstroke}{rgb}{0.000000,0.000000,0.000000}%
\pgfsetstrokecolor{currentstroke}%
\pgfsetdash{}{0pt}%
\pgfpathmoveto{\pgfqpoint{1.019812in}{1.443291in}}%
\pgfpathcurveto{\pgfqpoint{1.030863in}{1.443291in}}{\pgfqpoint{1.041462in}{1.447682in}}{\pgfqpoint{1.049275in}{1.455495in}}%
\pgfpathcurveto{\pgfqpoint{1.057089in}{1.463309in}}{\pgfqpoint{1.061479in}{1.473908in}}{\pgfqpoint{1.061479in}{1.484958in}}%
\pgfpathcurveto{\pgfqpoint{1.061479in}{1.496008in}}{\pgfqpoint{1.057089in}{1.506607in}}{\pgfqpoint{1.049275in}{1.514421in}}%
\pgfpathcurveto{\pgfqpoint{1.041462in}{1.522235in}}{\pgfqpoint{1.030863in}{1.526625in}}{\pgfqpoint{1.019812in}{1.526625in}}%
\pgfpathcurveto{\pgfqpoint{1.008762in}{1.526625in}}{\pgfqpoint{0.998163in}{1.522235in}}{\pgfqpoint{0.990350in}{1.514421in}}%
\pgfpathcurveto{\pgfqpoint{0.982536in}{1.506607in}}{\pgfqpoint{0.978146in}{1.496008in}}{\pgfqpoint{0.978146in}{1.484958in}}%
\pgfpathcurveto{\pgfqpoint{0.978146in}{1.473908in}}{\pgfqpoint{0.982536in}{1.463309in}}{\pgfqpoint{0.990350in}{1.455495in}}%
\pgfpathcurveto{\pgfqpoint{0.998163in}{1.447682in}}{\pgfqpoint{1.008762in}{1.443291in}}{\pgfqpoint{1.019812in}{1.443291in}}%
\pgfpathclose%
\pgfusepath{stroke,fill}%
\end{pgfscope}%
\begin{pgfscope}%
\pgfpathrectangle{\pgfqpoint{0.375000in}{0.330000in}}{\pgfqpoint{2.325000in}{2.310000in}}%
\pgfusepath{clip}%
\pgfsetbuttcap%
\pgfsetroundjoin%
\definecolor{currentfill}{rgb}{0.000000,0.000000,0.000000}%
\pgfsetfillcolor{currentfill}%
\pgfsetlinewidth{1.003750pt}%
\definecolor{currentstroke}{rgb}{0.000000,0.000000,0.000000}%
\pgfsetstrokecolor{currentstroke}%
\pgfsetdash{}{0pt}%
\pgfpathmoveto{\pgfqpoint{1.019812in}{1.443291in}}%
\pgfpathcurveto{\pgfqpoint{1.030863in}{1.443291in}}{\pgfqpoint{1.041462in}{1.447682in}}{\pgfqpoint{1.049275in}{1.455495in}}%
\pgfpathcurveto{\pgfqpoint{1.057089in}{1.463309in}}{\pgfqpoint{1.061479in}{1.473908in}}{\pgfqpoint{1.061479in}{1.484958in}}%
\pgfpathcurveto{\pgfqpoint{1.061479in}{1.496008in}}{\pgfqpoint{1.057089in}{1.506607in}}{\pgfqpoint{1.049275in}{1.514421in}}%
\pgfpathcurveto{\pgfqpoint{1.041462in}{1.522235in}}{\pgfqpoint{1.030863in}{1.526625in}}{\pgfqpoint{1.019812in}{1.526625in}}%
\pgfpathcurveto{\pgfqpoint{1.008762in}{1.526625in}}{\pgfqpoint{0.998163in}{1.522235in}}{\pgfqpoint{0.990350in}{1.514421in}}%
\pgfpathcurveto{\pgfqpoint{0.982536in}{1.506607in}}{\pgfqpoint{0.978146in}{1.496008in}}{\pgfqpoint{0.978146in}{1.484958in}}%
\pgfpathcurveto{\pgfqpoint{0.978146in}{1.473908in}}{\pgfqpoint{0.982536in}{1.463309in}}{\pgfqpoint{0.990350in}{1.455495in}}%
\pgfpathcurveto{\pgfqpoint{0.998163in}{1.447682in}}{\pgfqpoint{1.008762in}{1.443291in}}{\pgfqpoint{1.019812in}{1.443291in}}%
\pgfpathclose%
\pgfusepath{stroke,fill}%
\end{pgfscope}%
\begin{pgfscope}%
\pgfpathrectangle{\pgfqpoint{0.375000in}{0.330000in}}{\pgfqpoint{2.325000in}{2.310000in}}%
\pgfusepath{clip}%
\pgfsetbuttcap%
\pgfsetroundjoin%
\definecolor{currentfill}{rgb}{0.000000,0.000000,0.000000}%
\pgfsetfillcolor{currentfill}%
\pgfsetlinewidth{1.003750pt}%
\definecolor{currentstroke}{rgb}{0.000000,0.000000,0.000000}%
\pgfsetstrokecolor{currentstroke}%
\pgfsetdash{}{0pt}%
\pgfpathmoveto{\pgfqpoint{1.019812in}{1.443291in}}%
\pgfpathcurveto{\pgfqpoint{1.030863in}{1.443291in}}{\pgfqpoint{1.041462in}{1.447682in}}{\pgfqpoint{1.049275in}{1.455495in}}%
\pgfpathcurveto{\pgfqpoint{1.057089in}{1.463309in}}{\pgfqpoint{1.061479in}{1.473908in}}{\pgfqpoint{1.061479in}{1.484958in}}%
\pgfpathcurveto{\pgfqpoint{1.061479in}{1.496008in}}{\pgfqpoint{1.057089in}{1.506607in}}{\pgfqpoint{1.049275in}{1.514421in}}%
\pgfpathcurveto{\pgfqpoint{1.041462in}{1.522235in}}{\pgfqpoint{1.030863in}{1.526625in}}{\pgfqpoint{1.019812in}{1.526625in}}%
\pgfpathcurveto{\pgfqpoint{1.008762in}{1.526625in}}{\pgfqpoint{0.998163in}{1.522235in}}{\pgfqpoint{0.990350in}{1.514421in}}%
\pgfpathcurveto{\pgfqpoint{0.982536in}{1.506607in}}{\pgfqpoint{0.978146in}{1.496008in}}{\pgfqpoint{0.978146in}{1.484958in}}%
\pgfpathcurveto{\pgfqpoint{0.978146in}{1.473908in}}{\pgfqpoint{0.982536in}{1.463309in}}{\pgfqpoint{0.990350in}{1.455495in}}%
\pgfpathcurveto{\pgfqpoint{0.998163in}{1.447682in}}{\pgfqpoint{1.008762in}{1.443291in}}{\pgfqpoint{1.019812in}{1.443291in}}%
\pgfpathclose%
\pgfusepath{stroke,fill}%
\end{pgfscope}%
\begin{pgfscope}%
\pgfpathrectangle{\pgfqpoint{0.375000in}{0.330000in}}{\pgfqpoint{2.325000in}{2.310000in}}%
\pgfusepath{clip}%
\pgfsetbuttcap%
\pgfsetroundjoin%
\definecolor{currentfill}{rgb}{0.000000,0.000000,0.000000}%
\pgfsetfillcolor{currentfill}%
\pgfsetlinewidth{1.003750pt}%
\definecolor{currentstroke}{rgb}{0.000000,0.000000,0.000000}%
\pgfsetstrokecolor{currentstroke}%
\pgfsetdash{}{0pt}%
\pgfpathmoveto{\pgfqpoint{1.019812in}{1.443291in}}%
\pgfpathcurveto{\pgfqpoint{1.030863in}{1.443291in}}{\pgfqpoint{1.041462in}{1.447682in}}{\pgfqpoint{1.049275in}{1.455495in}}%
\pgfpathcurveto{\pgfqpoint{1.057089in}{1.463309in}}{\pgfqpoint{1.061479in}{1.473908in}}{\pgfqpoint{1.061479in}{1.484958in}}%
\pgfpathcurveto{\pgfqpoint{1.061479in}{1.496008in}}{\pgfqpoint{1.057089in}{1.506607in}}{\pgfqpoint{1.049275in}{1.514421in}}%
\pgfpathcurveto{\pgfqpoint{1.041462in}{1.522235in}}{\pgfqpoint{1.030863in}{1.526625in}}{\pgfqpoint{1.019812in}{1.526625in}}%
\pgfpathcurveto{\pgfqpoint{1.008762in}{1.526625in}}{\pgfqpoint{0.998163in}{1.522235in}}{\pgfqpoint{0.990350in}{1.514421in}}%
\pgfpathcurveto{\pgfqpoint{0.982536in}{1.506607in}}{\pgfqpoint{0.978146in}{1.496008in}}{\pgfqpoint{0.978146in}{1.484958in}}%
\pgfpathcurveto{\pgfqpoint{0.978146in}{1.473908in}}{\pgfqpoint{0.982536in}{1.463309in}}{\pgfqpoint{0.990350in}{1.455495in}}%
\pgfpathcurveto{\pgfqpoint{0.998163in}{1.447682in}}{\pgfqpoint{1.008762in}{1.443291in}}{\pgfqpoint{1.019812in}{1.443291in}}%
\pgfpathclose%
\pgfusepath{stroke,fill}%
\end{pgfscope}%
\begin{pgfscope}%
\pgfpathrectangle{\pgfqpoint{0.375000in}{0.330000in}}{\pgfqpoint{2.325000in}{2.310000in}}%
\pgfusepath{clip}%
\pgfsetbuttcap%
\pgfsetroundjoin%
\definecolor{currentfill}{rgb}{0.000000,0.000000,0.000000}%
\pgfsetfillcolor{currentfill}%
\pgfsetlinewidth{1.003750pt}%
\definecolor{currentstroke}{rgb}{0.000000,0.000000,0.000000}%
\pgfsetstrokecolor{currentstroke}%
\pgfsetdash{}{0pt}%
\pgfpathmoveto{\pgfqpoint{1.019812in}{1.443291in}}%
\pgfpathcurveto{\pgfqpoint{1.030863in}{1.443291in}}{\pgfqpoint{1.041462in}{1.447682in}}{\pgfqpoint{1.049275in}{1.455495in}}%
\pgfpathcurveto{\pgfqpoint{1.057089in}{1.463309in}}{\pgfqpoint{1.061479in}{1.473908in}}{\pgfqpoint{1.061479in}{1.484958in}}%
\pgfpathcurveto{\pgfqpoint{1.061479in}{1.496008in}}{\pgfqpoint{1.057089in}{1.506607in}}{\pgfqpoint{1.049275in}{1.514421in}}%
\pgfpathcurveto{\pgfqpoint{1.041462in}{1.522235in}}{\pgfqpoint{1.030863in}{1.526625in}}{\pgfqpoint{1.019812in}{1.526625in}}%
\pgfpathcurveto{\pgfqpoint{1.008762in}{1.526625in}}{\pgfqpoint{0.998163in}{1.522235in}}{\pgfqpoint{0.990350in}{1.514421in}}%
\pgfpathcurveto{\pgfqpoint{0.982536in}{1.506607in}}{\pgfqpoint{0.978146in}{1.496008in}}{\pgfqpoint{0.978146in}{1.484958in}}%
\pgfpathcurveto{\pgfqpoint{0.978146in}{1.473908in}}{\pgfqpoint{0.982536in}{1.463309in}}{\pgfqpoint{0.990350in}{1.455495in}}%
\pgfpathcurveto{\pgfqpoint{0.998163in}{1.447682in}}{\pgfqpoint{1.008762in}{1.443291in}}{\pgfqpoint{1.019812in}{1.443291in}}%
\pgfpathclose%
\pgfusepath{stroke,fill}%
\end{pgfscope}%
\begin{pgfscope}%
\pgfpathrectangle{\pgfqpoint{0.375000in}{0.330000in}}{\pgfqpoint{2.325000in}{2.310000in}}%
\pgfusepath{clip}%
\pgfsetbuttcap%
\pgfsetroundjoin%
\definecolor{currentfill}{rgb}{0.000000,0.000000,0.000000}%
\pgfsetfillcolor{currentfill}%
\pgfsetlinewidth{1.003750pt}%
\definecolor{currentstroke}{rgb}{0.000000,0.000000,0.000000}%
\pgfsetstrokecolor{currentstroke}%
\pgfsetdash{}{0pt}%
\pgfpathmoveto{\pgfqpoint{1.019812in}{1.443291in}}%
\pgfpathcurveto{\pgfqpoint{1.030863in}{1.443291in}}{\pgfqpoint{1.041462in}{1.447682in}}{\pgfqpoint{1.049275in}{1.455495in}}%
\pgfpathcurveto{\pgfqpoint{1.057089in}{1.463309in}}{\pgfqpoint{1.061479in}{1.473908in}}{\pgfqpoint{1.061479in}{1.484958in}}%
\pgfpathcurveto{\pgfqpoint{1.061479in}{1.496008in}}{\pgfqpoint{1.057089in}{1.506607in}}{\pgfqpoint{1.049275in}{1.514421in}}%
\pgfpathcurveto{\pgfqpoint{1.041462in}{1.522235in}}{\pgfqpoint{1.030863in}{1.526625in}}{\pgfqpoint{1.019812in}{1.526625in}}%
\pgfpathcurveto{\pgfqpoint{1.008762in}{1.526625in}}{\pgfqpoint{0.998163in}{1.522235in}}{\pgfqpoint{0.990350in}{1.514421in}}%
\pgfpathcurveto{\pgfqpoint{0.982536in}{1.506607in}}{\pgfqpoint{0.978146in}{1.496008in}}{\pgfqpoint{0.978146in}{1.484958in}}%
\pgfpathcurveto{\pgfqpoint{0.978146in}{1.473908in}}{\pgfqpoint{0.982536in}{1.463309in}}{\pgfqpoint{0.990350in}{1.455495in}}%
\pgfpathcurveto{\pgfqpoint{0.998163in}{1.447682in}}{\pgfqpoint{1.008762in}{1.443291in}}{\pgfqpoint{1.019812in}{1.443291in}}%
\pgfpathclose%
\pgfusepath{stroke,fill}%
\end{pgfscope}%
\begin{pgfscope}%
\pgfpathrectangle{\pgfqpoint{0.375000in}{0.330000in}}{\pgfqpoint{2.325000in}{2.310000in}}%
\pgfusepath{clip}%
\pgfsetbuttcap%
\pgfsetroundjoin%
\definecolor{currentfill}{rgb}{0.000000,0.000000,0.000000}%
\pgfsetfillcolor{currentfill}%
\pgfsetlinewidth{1.003750pt}%
\definecolor{currentstroke}{rgb}{0.000000,0.000000,0.000000}%
\pgfsetstrokecolor{currentstroke}%
\pgfsetdash{}{0pt}%
\pgfpathmoveto{\pgfqpoint{1.019812in}{1.443291in}}%
\pgfpathcurveto{\pgfqpoint{1.030863in}{1.443291in}}{\pgfqpoint{1.041462in}{1.447682in}}{\pgfqpoint{1.049275in}{1.455495in}}%
\pgfpathcurveto{\pgfqpoint{1.057089in}{1.463309in}}{\pgfqpoint{1.061479in}{1.473908in}}{\pgfqpoint{1.061479in}{1.484958in}}%
\pgfpathcurveto{\pgfqpoint{1.061479in}{1.496008in}}{\pgfqpoint{1.057089in}{1.506607in}}{\pgfqpoint{1.049275in}{1.514421in}}%
\pgfpathcurveto{\pgfqpoint{1.041462in}{1.522235in}}{\pgfqpoint{1.030863in}{1.526625in}}{\pgfqpoint{1.019812in}{1.526625in}}%
\pgfpathcurveto{\pgfqpoint{1.008762in}{1.526625in}}{\pgfqpoint{0.998163in}{1.522235in}}{\pgfqpoint{0.990350in}{1.514421in}}%
\pgfpathcurveto{\pgfqpoint{0.982536in}{1.506607in}}{\pgfqpoint{0.978146in}{1.496008in}}{\pgfqpoint{0.978146in}{1.484958in}}%
\pgfpathcurveto{\pgfqpoint{0.978146in}{1.473908in}}{\pgfqpoint{0.982536in}{1.463309in}}{\pgfqpoint{0.990350in}{1.455495in}}%
\pgfpathcurveto{\pgfqpoint{0.998163in}{1.447682in}}{\pgfqpoint{1.008762in}{1.443291in}}{\pgfqpoint{1.019812in}{1.443291in}}%
\pgfpathclose%
\pgfusepath{stroke,fill}%
\end{pgfscope}%
\begin{pgfscope}%
\pgfpathrectangle{\pgfqpoint{0.375000in}{0.330000in}}{\pgfqpoint{2.325000in}{2.310000in}}%
\pgfusepath{clip}%
\pgfsetbuttcap%
\pgfsetroundjoin%
\definecolor{currentfill}{rgb}{0.000000,0.000000,0.000000}%
\pgfsetfillcolor{currentfill}%
\pgfsetlinewidth{1.003750pt}%
\definecolor{currentstroke}{rgb}{0.000000,0.000000,0.000000}%
\pgfsetstrokecolor{currentstroke}%
\pgfsetdash{}{0pt}%
\pgfpathmoveto{\pgfqpoint{1.019812in}{1.443291in}}%
\pgfpathcurveto{\pgfqpoint{1.030863in}{1.443291in}}{\pgfqpoint{1.041462in}{1.447682in}}{\pgfqpoint{1.049275in}{1.455495in}}%
\pgfpathcurveto{\pgfqpoint{1.057089in}{1.463309in}}{\pgfqpoint{1.061479in}{1.473908in}}{\pgfqpoint{1.061479in}{1.484958in}}%
\pgfpathcurveto{\pgfqpoint{1.061479in}{1.496008in}}{\pgfqpoint{1.057089in}{1.506607in}}{\pgfqpoint{1.049275in}{1.514421in}}%
\pgfpathcurveto{\pgfqpoint{1.041462in}{1.522235in}}{\pgfqpoint{1.030863in}{1.526625in}}{\pgfqpoint{1.019812in}{1.526625in}}%
\pgfpathcurveto{\pgfqpoint{1.008762in}{1.526625in}}{\pgfqpoint{0.998163in}{1.522235in}}{\pgfqpoint{0.990350in}{1.514421in}}%
\pgfpathcurveto{\pgfqpoint{0.982536in}{1.506607in}}{\pgfqpoint{0.978146in}{1.496008in}}{\pgfqpoint{0.978146in}{1.484958in}}%
\pgfpathcurveto{\pgfqpoint{0.978146in}{1.473908in}}{\pgfqpoint{0.982536in}{1.463309in}}{\pgfqpoint{0.990350in}{1.455495in}}%
\pgfpathcurveto{\pgfqpoint{0.998163in}{1.447682in}}{\pgfqpoint{1.008762in}{1.443291in}}{\pgfqpoint{1.019812in}{1.443291in}}%
\pgfpathclose%
\pgfusepath{stroke,fill}%
\end{pgfscope}%
\begin{pgfscope}%
\pgfpathrectangle{\pgfqpoint{0.375000in}{0.330000in}}{\pgfqpoint{2.325000in}{2.310000in}}%
\pgfusepath{clip}%
\pgfsetbuttcap%
\pgfsetroundjoin%
\definecolor{currentfill}{rgb}{0.000000,0.000000,0.000000}%
\pgfsetfillcolor{currentfill}%
\pgfsetlinewidth{1.003750pt}%
\definecolor{currentstroke}{rgb}{0.000000,0.000000,0.000000}%
\pgfsetstrokecolor{currentstroke}%
\pgfsetdash{}{0pt}%
\pgfpathmoveto{\pgfqpoint{1.019812in}{1.443291in}}%
\pgfpathcurveto{\pgfqpoint{1.030863in}{1.443291in}}{\pgfqpoint{1.041462in}{1.447682in}}{\pgfqpoint{1.049275in}{1.455495in}}%
\pgfpathcurveto{\pgfqpoint{1.057089in}{1.463309in}}{\pgfqpoint{1.061479in}{1.473908in}}{\pgfqpoint{1.061479in}{1.484958in}}%
\pgfpathcurveto{\pgfqpoint{1.061479in}{1.496008in}}{\pgfqpoint{1.057089in}{1.506607in}}{\pgfqpoint{1.049275in}{1.514421in}}%
\pgfpathcurveto{\pgfqpoint{1.041462in}{1.522235in}}{\pgfqpoint{1.030863in}{1.526625in}}{\pgfqpoint{1.019812in}{1.526625in}}%
\pgfpathcurveto{\pgfqpoint{1.008762in}{1.526625in}}{\pgfqpoint{0.998163in}{1.522235in}}{\pgfqpoint{0.990350in}{1.514421in}}%
\pgfpathcurveto{\pgfqpoint{0.982536in}{1.506607in}}{\pgfqpoint{0.978146in}{1.496008in}}{\pgfqpoint{0.978146in}{1.484958in}}%
\pgfpathcurveto{\pgfqpoint{0.978146in}{1.473908in}}{\pgfqpoint{0.982536in}{1.463309in}}{\pgfqpoint{0.990350in}{1.455495in}}%
\pgfpathcurveto{\pgfqpoint{0.998163in}{1.447682in}}{\pgfqpoint{1.008762in}{1.443291in}}{\pgfqpoint{1.019812in}{1.443291in}}%
\pgfpathclose%
\pgfusepath{stroke,fill}%
\end{pgfscope}%
\begin{pgfscope}%
\pgfpathrectangle{\pgfqpoint{0.375000in}{0.330000in}}{\pgfqpoint{2.325000in}{2.310000in}}%
\pgfusepath{clip}%
\pgfsetbuttcap%
\pgfsetroundjoin%
\definecolor{currentfill}{rgb}{0.000000,0.000000,0.000000}%
\pgfsetfillcolor{currentfill}%
\pgfsetlinewidth{1.003750pt}%
\definecolor{currentstroke}{rgb}{0.000000,0.000000,0.000000}%
\pgfsetstrokecolor{currentstroke}%
\pgfsetdash{}{0pt}%
\pgfpathmoveto{\pgfqpoint{1.019812in}{1.443291in}}%
\pgfpathcurveto{\pgfqpoint{1.030863in}{1.443291in}}{\pgfqpoint{1.041462in}{1.447682in}}{\pgfqpoint{1.049275in}{1.455495in}}%
\pgfpathcurveto{\pgfqpoint{1.057089in}{1.463309in}}{\pgfqpoint{1.061479in}{1.473908in}}{\pgfqpoint{1.061479in}{1.484958in}}%
\pgfpathcurveto{\pgfqpoint{1.061479in}{1.496008in}}{\pgfqpoint{1.057089in}{1.506607in}}{\pgfqpoint{1.049275in}{1.514421in}}%
\pgfpathcurveto{\pgfqpoint{1.041462in}{1.522235in}}{\pgfqpoint{1.030863in}{1.526625in}}{\pgfqpoint{1.019812in}{1.526625in}}%
\pgfpathcurveto{\pgfqpoint{1.008762in}{1.526625in}}{\pgfqpoint{0.998163in}{1.522235in}}{\pgfqpoint{0.990350in}{1.514421in}}%
\pgfpathcurveto{\pgfqpoint{0.982536in}{1.506607in}}{\pgfqpoint{0.978146in}{1.496008in}}{\pgfqpoint{0.978146in}{1.484958in}}%
\pgfpathcurveto{\pgfqpoint{0.978146in}{1.473908in}}{\pgfqpoint{0.982536in}{1.463309in}}{\pgfqpoint{0.990350in}{1.455495in}}%
\pgfpathcurveto{\pgfqpoint{0.998163in}{1.447682in}}{\pgfqpoint{1.008762in}{1.443291in}}{\pgfqpoint{1.019812in}{1.443291in}}%
\pgfpathclose%
\pgfusepath{stroke,fill}%
\end{pgfscope}%
\begin{pgfscope}%
\pgfpathrectangle{\pgfqpoint{0.375000in}{0.330000in}}{\pgfqpoint{2.325000in}{2.310000in}}%
\pgfusepath{clip}%
\pgfsetbuttcap%
\pgfsetroundjoin%
\definecolor{currentfill}{rgb}{0.000000,0.000000,0.000000}%
\pgfsetfillcolor{currentfill}%
\pgfsetlinewidth{1.003750pt}%
\definecolor{currentstroke}{rgb}{0.000000,0.000000,0.000000}%
\pgfsetstrokecolor{currentstroke}%
\pgfsetdash{}{0pt}%
\pgfpathmoveto{\pgfqpoint{1.019812in}{1.443291in}}%
\pgfpathcurveto{\pgfqpoint{1.030863in}{1.443291in}}{\pgfqpoint{1.041462in}{1.447682in}}{\pgfqpoint{1.049275in}{1.455495in}}%
\pgfpathcurveto{\pgfqpoint{1.057089in}{1.463309in}}{\pgfqpoint{1.061479in}{1.473908in}}{\pgfqpoint{1.061479in}{1.484958in}}%
\pgfpathcurveto{\pgfqpoint{1.061479in}{1.496008in}}{\pgfqpoint{1.057089in}{1.506607in}}{\pgfqpoint{1.049275in}{1.514421in}}%
\pgfpathcurveto{\pgfqpoint{1.041462in}{1.522235in}}{\pgfqpoint{1.030863in}{1.526625in}}{\pgfqpoint{1.019812in}{1.526625in}}%
\pgfpathcurveto{\pgfqpoint{1.008762in}{1.526625in}}{\pgfqpoint{0.998163in}{1.522235in}}{\pgfqpoint{0.990350in}{1.514421in}}%
\pgfpathcurveto{\pgfqpoint{0.982536in}{1.506607in}}{\pgfqpoint{0.978146in}{1.496008in}}{\pgfqpoint{0.978146in}{1.484958in}}%
\pgfpathcurveto{\pgfqpoint{0.978146in}{1.473908in}}{\pgfqpoint{0.982536in}{1.463309in}}{\pgfqpoint{0.990350in}{1.455495in}}%
\pgfpathcurveto{\pgfqpoint{0.998163in}{1.447682in}}{\pgfqpoint{1.008762in}{1.443291in}}{\pgfqpoint{1.019812in}{1.443291in}}%
\pgfpathclose%
\pgfusepath{stroke,fill}%
\end{pgfscope}%
\begin{pgfscope}%
\pgfpathrectangle{\pgfqpoint{0.375000in}{0.330000in}}{\pgfqpoint{2.325000in}{2.310000in}}%
\pgfusepath{clip}%
\pgfsetbuttcap%
\pgfsetroundjoin%
\definecolor{currentfill}{rgb}{0.000000,0.000000,0.000000}%
\pgfsetfillcolor{currentfill}%
\pgfsetlinewidth{1.003750pt}%
\definecolor{currentstroke}{rgb}{0.000000,0.000000,0.000000}%
\pgfsetstrokecolor{currentstroke}%
\pgfsetdash{}{0pt}%
\pgfpathmoveto{\pgfqpoint{1.019812in}{1.443291in}}%
\pgfpathcurveto{\pgfqpoint{1.030863in}{1.443291in}}{\pgfqpoint{1.041462in}{1.447682in}}{\pgfqpoint{1.049275in}{1.455495in}}%
\pgfpathcurveto{\pgfqpoint{1.057089in}{1.463309in}}{\pgfqpoint{1.061479in}{1.473908in}}{\pgfqpoint{1.061479in}{1.484958in}}%
\pgfpathcurveto{\pgfqpoint{1.061479in}{1.496008in}}{\pgfqpoint{1.057089in}{1.506607in}}{\pgfqpoint{1.049275in}{1.514421in}}%
\pgfpathcurveto{\pgfqpoint{1.041462in}{1.522235in}}{\pgfqpoint{1.030863in}{1.526625in}}{\pgfqpoint{1.019812in}{1.526625in}}%
\pgfpathcurveto{\pgfqpoint{1.008762in}{1.526625in}}{\pgfqpoint{0.998163in}{1.522235in}}{\pgfqpoint{0.990350in}{1.514421in}}%
\pgfpathcurveto{\pgfqpoint{0.982536in}{1.506607in}}{\pgfqpoint{0.978146in}{1.496008in}}{\pgfqpoint{0.978146in}{1.484958in}}%
\pgfpathcurveto{\pgfqpoint{0.978146in}{1.473908in}}{\pgfqpoint{0.982536in}{1.463309in}}{\pgfqpoint{0.990350in}{1.455495in}}%
\pgfpathcurveto{\pgfqpoint{0.998163in}{1.447682in}}{\pgfqpoint{1.008762in}{1.443291in}}{\pgfqpoint{1.019812in}{1.443291in}}%
\pgfpathclose%
\pgfusepath{stroke,fill}%
\end{pgfscope}%
\begin{pgfscope}%
\pgfpathrectangle{\pgfqpoint{0.375000in}{0.330000in}}{\pgfqpoint{2.325000in}{2.310000in}}%
\pgfusepath{clip}%
\pgfsetbuttcap%
\pgfsetroundjoin%
\definecolor{currentfill}{rgb}{0.000000,0.000000,0.000000}%
\pgfsetfillcolor{currentfill}%
\pgfsetlinewidth{1.003750pt}%
\definecolor{currentstroke}{rgb}{0.000000,0.000000,0.000000}%
\pgfsetstrokecolor{currentstroke}%
\pgfsetdash{}{0pt}%
\pgfpathmoveto{\pgfqpoint{1.019812in}{1.443291in}}%
\pgfpathcurveto{\pgfqpoint{1.030863in}{1.443291in}}{\pgfqpoint{1.041462in}{1.447682in}}{\pgfqpoint{1.049275in}{1.455495in}}%
\pgfpathcurveto{\pgfqpoint{1.057089in}{1.463309in}}{\pgfqpoint{1.061479in}{1.473908in}}{\pgfqpoint{1.061479in}{1.484958in}}%
\pgfpathcurveto{\pgfqpoint{1.061479in}{1.496008in}}{\pgfqpoint{1.057089in}{1.506607in}}{\pgfqpoint{1.049275in}{1.514421in}}%
\pgfpathcurveto{\pgfqpoint{1.041462in}{1.522235in}}{\pgfqpoint{1.030863in}{1.526625in}}{\pgfqpoint{1.019812in}{1.526625in}}%
\pgfpathcurveto{\pgfqpoint{1.008762in}{1.526625in}}{\pgfqpoint{0.998163in}{1.522235in}}{\pgfqpoint{0.990350in}{1.514421in}}%
\pgfpathcurveto{\pgfqpoint{0.982536in}{1.506607in}}{\pgfqpoint{0.978146in}{1.496008in}}{\pgfqpoint{0.978146in}{1.484958in}}%
\pgfpathcurveto{\pgfqpoint{0.978146in}{1.473908in}}{\pgfqpoint{0.982536in}{1.463309in}}{\pgfqpoint{0.990350in}{1.455495in}}%
\pgfpathcurveto{\pgfqpoint{0.998163in}{1.447682in}}{\pgfqpoint{1.008762in}{1.443291in}}{\pgfqpoint{1.019812in}{1.443291in}}%
\pgfpathclose%
\pgfusepath{stroke,fill}%
\end{pgfscope}%
\begin{pgfscope}%
\pgfpathrectangle{\pgfqpoint{0.375000in}{0.330000in}}{\pgfqpoint{2.325000in}{2.310000in}}%
\pgfusepath{clip}%
\pgfsetbuttcap%
\pgfsetroundjoin%
\definecolor{currentfill}{rgb}{0.000000,0.000000,0.000000}%
\pgfsetfillcolor{currentfill}%
\pgfsetlinewidth{1.003750pt}%
\definecolor{currentstroke}{rgb}{0.000000,0.000000,0.000000}%
\pgfsetstrokecolor{currentstroke}%
\pgfsetdash{}{0pt}%
\pgfpathmoveto{\pgfqpoint{1.019812in}{1.443291in}}%
\pgfpathcurveto{\pgfqpoint{1.030863in}{1.443291in}}{\pgfqpoint{1.041462in}{1.447682in}}{\pgfqpoint{1.049275in}{1.455495in}}%
\pgfpathcurveto{\pgfqpoint{1.057089in}{1.463309in}}{\pgfqpoint{1.061479in}{1.473908in}}{\pgfqpoint{1.061479in}{1.484958in}}%
\pgfpathcurveto{\pgfqpoint{1.061479in}{1.496008in}}{\pgfqpoint{1.057089in}{1.506607in}}{\pgfqpoint{1.049275in}{1.514421in}}%
\pgfpathcurveto{\pgfqpoint{1.041462in}{1.522235in}}{\pgfqpoint{1.030863in}{1.526625in}}{\pgfqpoint{1.019812in}{1.526625in}}%
\pgfpathcurveto{\pgfqpoint{1.008762in}{1.526625in}}{\pgfqpoint{0.998163in}{1.522235in}}{\pgfqpoint{0.990350in}{1.514421in}}%
\pgfpathcurveto{\pgfqpoint{0.982536in}{1.506607in}}{\pgfqpoint{0.978146in}{1.496008in}}{\pgfqpoint{0.978146in}{1.484958in}}%
\pgfpathcurveto{\pgfqpoint{0.978146in}{1.473908in}}{\pgfqpoint{0.982536in}{1.463309in}}{\pgfqpoint{0.990350in}{1.455495in}}%
\pgfpathcurveto{\pgfqpoint{0.998163in}{1.447682in}}{\pgfqpoint{1.008762in}{1.443291in}}{\pgfqpoint{1.019812in}{1.443291in}}%
\pgfpathclose%
\pgfusepath{stroke,fill}%
\end{pgfscope}%
\begin{pgfscope}%
\pgfpathrectangle{\pgfqpoint{0.375000in}{0.330000in}}{\pgfqpoint{2.325000in}{2.310000in}}%
\pgfusepath{clip}%
\pgfsetbuttcap%
\pgfsetroundjoin%
\definecolor{currentfill}{rgb}{0.000000,0.000000,0.000000}%
\pgfsetfillcolor{currentfill}%
\pgfsetlinewidth{1.003750pt}%
\definecolor{currentstroke}{rgb}{0.000000,0.000000,0.000000}%
\pgfsetstrokecolor{currentstroke}%
\pgfsetdash{}{0pt}%
\pgfpathmoveto{\pgfqpoint{1.019812in}{1.443291in}}%
\pgfpathcurveto{\pgfqpoint{1.030863in}{1.443291in}}{\pgfqpoint{1.041462in}{1.447682in}}{\pgfqpoint{1.049275in}{1.455495in}}%
\pgfpathcurveto{\pgfqpoint{1.057089in}{1.463309in}}{\pgfqpoint{1.061479in}{1.473908in}}{\pgfqpoint{1.061479in}{1.484958in}}%
\pgfpathcurveto{\pgfqpoint{1.061479in}{1.496008in}}{\pgfqpoint{1.057089in}{1.506607in}}{\pgfqpoint{1.049275in}{1.514421in}}%
\pgfpathcurveto{\pgfqpoint{1.041462in}{1.522235in}}{\pgfqpoint{1.030863in}{1.526625in}}{\pgfqpoint{1.019812in}{1.526625in}}%
\pgfpathcurveto{\pgfqpoint{1.008762in}{1.526625in}}{\pgfqpoint{0.998163in}{1.522235in}}{\pgfqpoint{0.990350in}{1.514421in}}%
\pgfpathcurveto{\pgfqpoint{0.982536in}{1.506607in}}{\pgfqpoint{0.978146in}{1.496008in}}{\pgfqpoint{0.978146in}{1.484958in}}%
\pgfpathcurveto{\pgfqpoint{0.978146in}{1.473908in}}{\pgfqpoint{0.982536in}{1.463309in}}{\pgfqpoint{0.990350in}{1.455495in}}%
\pgfpathcurveto{\pgfqpoint{0.998163in}{1.447682in}}{\pgfqpoint{1.008762in}{1.443291in}}{\pgfqpoint{1.019812in}{1.443291in}}%
\pgfpathclose%
\pgfusepath{stroke,fill}%
\end{pgfscope}%
\begin{pgfscope}%
\pgfpathrectangle{\pgfqpoint{0.375000in}{0.330000in}}{\pgfqpoint{2.325000in}{2.310000in}}%
\pgfusepath{clip}%
\pgfsetbuttcap%
\pgfsetroundjoin%
\definecolor{currentfill}{rgb}{0.000000,0.000000,0.000000}%
\pgfsetfillcolor{currentfill}%
\pgfsetlinewidth{1.003750pt}%
\definecolor{currentstroke}{rgb}{0.000000,0.000000,0.000000}%
\pgfsetstrokecolor{currentstroke}%
\pgfsetdash{}{0pt}%
\pgfpathmoveto{\pgfqpoint{1.019812in}{1.443291in}}%
\pgfpathcurveto{\pgfqpoint{1.030863in}{1.443291in}}{\pgfqpoint{1.041462in}{1.447682in}}{\pgfqpoint{1.049275in}{1.455495in}}%
\pgfpathcurveto{\pgfqpoint{1.057089in}{1.463309in}}{\pgfqpoint{1.061479in}{1.473908in}}{\pgfqpoint{1.061479in}{1.484958in}}%
\pgfpathcurveto{\pgfqpoint{1.061479in}{1.496008in}}{\pgfqpoint{1.057089in}{1.506607in}}{\pgfqpoint{1.049275in}{1.514421in}}%
\pgfpathcurveto{\pgfqpoint{1.041462in}{1.522235in}}{\pgfqpoint{1.030863in}{1.526625in}}{\pgfqpoint{1.019812in}{1.526625in}}%
\pgfpathcurveto{\pgfqpoint{1.008762in}{1.526625in}}{\pgfqpoint{0.998163in}{1.522235in}}{\pgfqpoint{0.990350in}{1.514421in}}%
\pgfpathcurveto{\pgfqpoint{0.982536in}{1.506607in}}{\pgfqpoint{0.978146in}{1.496008in}}{\pgfqpoint{0.978146in}{1.484958in}}%
\pgfpathcurveto{\pgfqpoint{0.978146in}{1.473908in}}{\pgfqpoint{0.982536in}{1.463309in}}{\pgfqpoint{0.990350in}{1.455495in}}%
\pgfpathcurveto{\pgfqpoint{0.998163in}{1.447682in}}{\pgfqpoint{1.008762in}{1.443291in}}{\pgfqpoint{1.019812in}{1.443291in}}%
\pgfpathclose%
\pgfusepath{stroke,fill}%
\end{pgfscope}%
\begin{pgfscope}%
\pgfpathrectangle{\pgfqpoint{0.375000in}{0.330000in}}{\pgfqpoint{2.325000in}{2.310000in}}%
\pgfusepath{clip}%
\pgfsetbuttcap%
\pgfsetroundjoin%
\definecolor{currentfill}{rgb}{0.000000,0.000000,0.000000}%
\pgfsetfillcolor{currentfill}%
\pgfsetlinewidth{1.003750pt}%
\definecolor{currentstroke}{rgb}{0.000000,0.000000,0.000000}%
\pgfsetstrokecolor{currentstroke}%
\pgfsetdash{}{0pt}%
\pgfpathmoveto{\pgfqpoint{1.579875in}{1.443291in}}%
\pgfpathcurveto{\pgfqpoint{1.590925in}{1.443291in}}{\pgfqpoint{1.601524in}{1.447682in}}{\pgfqpoint{1.609338in}{1.455495in}}%
\pgfpathcurveto{\pgfqpoint{1.617151in}{1.463309in}}{\pgfqpoint{1.621542in}{1.473908in}}{\pgfqpoint{1.621542in}{1.484958in}}%
\pgfpathcurveto{\pgfqpoint{1.621542in}{1.496008in}}{\pgfqpoint{1.617151in}{1.506607in}}{\pgfqpoint{1.609338in}{1.514421in}}%
\pgfpathcurveto{\pgfqpoint{1.601524in}{1.522235in}}{\pgfqpoint{1.590925in}{1.526625in}}{\pgfqpoint{1.579875in}{1.526625in}}%
\pgfpathcurveto{\pgfqpoint{1.568825in}{1.526625in}}{\pgfqpoint{1.558226in}{1.522235in}}{\pgfqpoint{1.550412in}{1.514421in}}%
\pgfpathcurveto{\pgfqpoint{1.542599in}{1.506607in}}{\pgfqpoint{1.538208in}{1.496008in}}{\pgfqpoint{1.538208in}{1.484958in}}%
\pgfpathcurveto{\pgfqpoint{1.538208in}{1.473908in}}{\pgfqpoint{1.542599in}{1.463309in}}{\pgfqpoint{1.550412in}{1.455495in}}%
\pgfpathcurveto{\pgfqpoint{1.558226in}{1.447682in}}{\pgfqpoint{1.568825in}{1.443291in}}{\pgfqpoint{1.579875in}{1.443291in}}%
\pgfpathclose%
\pgfusepath{stroke,fill}%
\end{pgfscope}%
\begin{pgfscope}%
\pgfpathrectangle{\pgfqpoint{0.375000in}{0.330000in}}{\pgfqpoint{2.325000in}{2.310000in}}%
\pgfusepath{clip}%
\pgfsetbuttcap%
\pgfsetroundjoin%
\definecolor{currentfill}{rgb}{0.000000,0.000000,0.000000}%
\pgfsetfillcolor{currentfill}%
\pgfsetlinewidth{1.003750pt}%
\definecolor{currentstroke}{rgb}{0.000000,0.000000,0.000000}%
\pgfsetstrokecolor{currentstroke}%
\pgfsetdash{}{0pt}%
\pgfpathmoveto{\pgfqpoint{1.579875in}{1.443291in}}%
\pgfpathcurveto{\pgfqpoint{1.590925in}{1.443291in}}{\pgfqpoint{1.601524in}{1.447682in}}{\pgfqpoint{1.609338in}{1.455495in}}%
\pgfpathcurveto{\pgfqpoint{1.617151in}{1.463309in}}{\pgfqpoint{1.621542in}{1.473908in}}{\pgfqpoint{1.621542in}{1.484958in}}%
\pgfpathcurveto{\pgfqpoint{1.621542in}{1.496008in}}{\pgfqpoint{1.617151in}{1.506607in}}{\pgfqpoint{1.609338in}{1.514421in}}%
\pgfpathcurveto{\pgfqpoint{1.601524in}{1.522235in}}{\pgfqpoint{1.590925in}{1.526625in}}{\pgfqpoint{1.579875in}{1.526625in}}%
\pgfpathcurveto{\pgfqpoint{1.568825in}{1.526625in}}{\pgfqpoint{1.558226in}{1.522235in}}{\pgfqpoint{1.550412in}{1.514421in}}%
\pgfpathcurveto{\pgfqpoint{1.542599in}{1.506607in}}{\pgfqpoint{1.538208in}{1.496008in}}{\pgfqpoint{1.538208in}{1.484958in}}%
\pgfpathcurveto{\pgfqpoint{1.538208in}{1.473908in}}{\pgfqpoint{1.542599in}{1.463309in}}{\pgfqpoint{1.550412in}{1.455495in}}%
\pgfpathcurveto{\pgfqpoint{1.558226in}{1.447682in}}{\pgfqpoint{1.568825in}{1.443291in}}{\pgfqpoint{1.579875in}{1.443291in}}%
\pgfpathclose%
\pgfusepath{stroke,fill}%
\end{pgfscope}%
\begin{pgfscope}%
\pgfpathrectangle{\pgfqpoint{0.375000in}{0.330000in}}{\pgfqpoint{2.325000in}{2.310000in}}%
\pgfusepath{clip}%
\pgfsetbuttcap%
\pgfsetroundjoin%
\definecolor{currentfill}{rgb}{0.000000,0.000000,0.000000}%
\pgfsetfillcolor{currentfill}%
\pgfsetlinewidth{1.003750pt}%
\definecolor{currentstroke}{rgb}{0.000000,0.000000,0.000000}%
\pgfsetstrokecolor{currentstroke}%
\pgfsetdash{}{0pt}%
\pgfpathmoveto{\pgfqpoint{1.579875in}{1.443291in}}%
\pgfpathcurveto{\pgfqpoint{1.590925in}{1.443291in}}{\pgfqpoint{1.601524in}{1.447682in}}{\pgfqpoint{1.609338in}{1.455495in}}%
\pgfpathcurveto{\pgfqpoint{1.617151in}{1.463309in}}{\pgfqpoint{1.621542in}{1.473908in}}{\pgfqpoint{1.621542in}{1.484958in}}%
\pgfpathcurveto{\pgfqpoint{1.621542in}{1.496008in}}{\pgfqpoint{1.617151in}{1.506607in}}{\pgfqpoint{1.609338in}{1.514421in}}%
\pgfpathcurveto{\pgfqpoint{1.601524in}{1.522235in}}{\pgfqpoint{1.590925in}{1.526625in}}{\pgfqpoint{1.579875in}{1.526625in}}%
\pgfpathcurveto{\pgfqpoint{1.568825in}{1.526625in}}{\pgfqpoint{1.558226in}{1.522235in}}{\pgfqpoint{1.550412in}{1.514421in}}%
\pgfpathcurveto{\pgfqpoint{1.542599in}{1.506607in}}{\pgfqpoint{1.538208in}{1.496008in}}{\pgfqpoint{1.538208in}{1.484958in}}%
\pgfpathcurveto{\pgfqpoint{1.538208in}{1.473908in}}{\pgfqpoint{1.542599in}{1.463309in}}{\pgfqpoint{1.550412in}{1.455495in}}%
\pgfpathcurveto{\pgfqpoint{1.558226in}{1.447682in}}{\pgfqpoint{1.568825in}{1.443291in}}{\pgfqpoint{1.579875in}{1.443291in}}%
\pgfpathclose%
\pgfusepath{stroke,fill}%
\end{pgfscope}%
\begin{pgfscope}%
\pgfpathrectangle{\pgfqpoint{0.375000in}{0.330000in}}{\pgfqpoint{2.325000in}{2.310000in}}%
\pgfusepath{clip}%
\pgfsetbuttcap%
\pgfsetroundjoin%
\definecolor{currentfill}{rgb}{0.000000,0.000000,0.000000}%
\pgfsetfillcolor{currentfill}%
\pgfsetlinewidth{1.003750pt}%
\definecolor{currentstroke}{rgb}{0.000000,0.000000,0.000000}%
\pgfsetstrokecolor{currentstroke}%
\pgfsetdash{}{0pt}%
\pgfpathmoveto{\pgfqpoint{1.579875in}{1.443291in}}%
\pgfpathcurveto{\pgfqpoint{1.590925in}{1.443291in}}{\pgfqpoint{1.601524in}{1.447682in}}{\pgfqpoint{1.609338in}{1.455495in}}%
\pgfpathcurveto{\pgfqpoint{1.617151in}{1.463309in}}{\pgfqpoint{1.621542in}{1.473908in}}{\pgfqpoint{1.621542in}{1.484958in}}%
\pgfpathcurveto{\pgfqpoint{1.621542in}{1.496008in}}{\pgfqpoint{1.617151in}{1.506607in}}{\pgfqpoint{1.609338in}{1.514421in}}%
\pgfpathcurveto{\pgfqpoint{1.601524in}{1.522235in}}{\pgfqpoint{1.590925in}{1.526625in}}{\pgfqpoint{1.579875in}{1.526625in}}%
\pgfpathcurveto{\pgfqpoint{1.568825in}{1.526625in}}{\pgfqpoint{1.558226in}{1.522235in}}{\pgfqpoint{1.550412in}{1.514421in}}%
\pgfpathcurveto{\pgfqpoint{1.542599in}{1.506607in}}{\pgfqpoint{1.538208in}{1.496008in}}{\pgfqpoint{1.538208in}{1.484958in}}%
\pgfpathcurveto{\pgfqpoint{1.538208in}{1.473908in}}{\pgfqpoint{1.542599in}{1.463309in}}{\pgfqpoint{1.550412in}{1.455495in}}%
\pgfpathcurveto{\pgfqpoint{1.558226in}{1.447682in}}{\pgfqpoint{1.568825in}{1.443291in}}{\pgfqpoint{1.579875in}{1.443291in}}%
\pgfpathclose%
\pgfusepath{stroke,fill}%
\end{pgfscope}%
\begin{pgfscope}%
\pgfpathrectangle{\pgfqpoint{0.375000in}{0.330000in}}{\pgfqpoint{2.325000in}{2.310000in}}%
\pgfusepath{clip}%
\pgfsetbuttcap%
\pgfsetroundjoin%
\definecolor{currentfill}{rgb}{0.000000,0.000000,0.000000}%
\pgfsetfillcolor{currentfill}%
\pgfsetlinewidth{1.003750pt}%
\definecolor{currentstroke}{rgb}{0.000000,0.000000,0.000000}%
\pgfsetstrokecolor{currentstroke}%
\pgfsetdash{}{0pt}%
\pgfpathmoveto{\pgfqpoint{1.579875in}{1.443291in}}%
\pgfpathcurveto{\pgfqpoint{1.590925in}{1.443291in}}{\pgfqpoint{1.601524in}{1.447682in}}{\pgfqpoint{1.609338in}{1.455495in}}%
\pgfpathcurveto{\pgfqpoint{1.617151in}{1.463309in}}{\pgfqpoint{1.621542in}{1.473908in}}{\pgfqpoint{1.621542in}{1.484958in}}%
\pgfpathcurveto{\pgfqpoint{1.621542in}{1.496008in}}{\pgfqpoint{1.617151in}{1.506607in}}{\pgfqpoint{1.609338in}{1.514421in}}%
\pgfpathcurveto{\pgfqpoint{1.601524in}{1.522235in}}{\pgfqpoint{1.590925in}{1.526625in}}{\pgfqpoint{1.579875in}{1.526625in}}%
\pgfpathcurveto{\pgfqpoint{1.568825in}{1.526625in}}{\pgfqpoint{1.558226in}{1.522235in}}{\pgfqpoint{1.550412in}{1.514421in}}%
\pgfpathcurveto{\pgfqpoint{1.542599in}{1.506607in}}{\pgfqpoint{1.538208in}{1.496008in}}{\pgfqpoint{1.538208in}{1.484958in}}%
\pgfpathcurveto{\pgfqpoint{1.538208in}{1.473908in}}{\pgfqpoint{1.542599in}{1.463309in}}{\pgfqpoint{1.550412in}{1.455495in}}%
\pgfpathcurveto{\pgfqpoint{1.558226in}{1.447682in}}{\pgfqpoint{1.568825in}{1.443291in}}{\pgfqpoint{1.579875in}{1.443291in}}%
\pgfpathclose%
\pgfusepath{stroke,fill}%
\end{pgfscope}%
\begin{pgfscope}%
\pgfpathrectangle{\pgfqpoint{0.375000in}{0.330000in}}{\pgfqpoint{2.325000in}{2.310000in}}%
\pgfusepath{clip}%
\pgfsetbuttcap%
\pgfsetroundjoin%
\definecolor{currentfill}{rgb}{0.000000,0.000000,0.000000}%
\pgfsetfillcolor{currentfill}%
\pgfsetlinewidth{1.003750pt}%
\definecolor{currentstroke}{rgb}{0.000000,0.000000,0.000000}%
\pgfsetstrokecolor{currentstroke}%
\pgfsetdash{}{0pt}%
\pgfpathmoveto{\pgfqpoint{1.579875in}{1.443291in}}%
\pgfpathcurveto{\pgfqpoint{1.590925in}{1.443291in}}{\pgfqpoint{1.601524in}{1.447682in}}{\pgfqpoint{1.609338in}{1.455495in}}%
\pgfpathcurveto{\pgfqpoint{1.617151in}{1.463309in}}{\pgfqpoint{1.621542in}{1.473908in}}{\pgfqpoint{1.621542in}{1.484958in}}%
\pgfpathcurveto{\pgfqpoint{1.621542in}{1.496008in}}{\pgfqpoint{1.617151in}{1.506607in}}{\pgfqpoint{1.609338in}{1.514421in}}%
\pgfpathcurveto{\pgfqpoint{1.601524in}{1.522235in}}{\pgfqpoint{1.590925in}{1.526625in}}{\pgfqpoint{1.579875in}{1.526625in}}%
\pgfpathcurveto{\pgfqpoint{1.568825in}{1.526625in}}{\pgfqpoint{1.558226in}{1.522235in}}{\pgfqpoint{1.550412in}{1.514421in}}%
\pgfpathcurveto{\pgfqpoint{1.542599in}{1.506607in}}{\pgfqpoint{1.538208in}{1.496008in}}{\pgfqpoint{1.538208in}{1.484958in}}%
\pgfpathcurveto{\pgfqpoint{1.538208in}{1.473908in}}{\pgfqpoint{1.542599in}{1.463309in}}{\pgfqpoint{1.550412in}{1.455495in}}%
\pgfpathcurveto{\pgfqpoint{1.558226in}{1.447682in}}{\pgfqpoint{1.568825in}{1.443291in}}{\pgfqpoint{1.579875in}{1.443291in}}%
\pgfpathclose%
\pgfusepath{stroke,fill}%
\end{pgfscope}%
\begin{pgfscope}%
\pgfpathrectangle{\pgfqpoint{0.375000in}{0.330000in}}{\pgfqpoint{2.325000in}{2.310000in}}%
\pgfusepath{clip}%
\pgfsetbuttcap%
\pgfsetroundjoin%
\definecolor{currentfill}{rgb}{0.000000,0.000000,0.000000}%
\pgfsetfillcolor{currentfill}%
\pgfsetlinewidth{1.003750pt}%
\definecolor{currentstroke}{rgb}{0.000000,0.000000,0.000000}%
\pgfsetstrokecolor{currentstroke}%
\pgfsetdash{}{0pt}%
\pgfpathmoveto{\pgfqpoint{1.579875in}{1.443291in}}%
\pgfpathcurveto{\pgfqpoint{1.590925in}{1.443291in}}{\pgfqpoint{1.601524in}{1.447682in}}{\pgfqpoint{1.609338in}{1.455495in}}%
\pgfpathcurveto{\pgfqpoint{1.617151in}{1.463309in}}{\pgfqpoint{1.621542in}{1.473908in}}{\pgfqpoint{1.621542in}{1.484958in}}%
\pgfpathcurveto{\pgfqpoint{1.621542in}{1.496008in}}{\pgfqpoint{1.617151in}{1.506607in}}{\pgfqpoint{1.609338in}{1.514421in}}%
\pgfpathcurveto{\pgfqpoint{1.601524in}{1.522235in}}{\pgfqpoint{1.590925in}{1.526625in}}{\pgfqpoint{1.579875in}{1.526625in}}%
\pgfpathcurveto{\pgfqpoint{1.568825in}{1.526625in}}{\pgfqpoint{1.558226in}{1.522235in}}{\pgfqpoint{1.550412in}{1.514421in}}%
\pgfpathcurveto{\pgfqpoint{1.542599in}{1.506607in}}{\pgfqpoint{1.538208in}{1.496008in}}{\pgfqpoint{1.538208in}{1.484958in}}%
\pgfpathcurveto{\pgfqpoint{1.538208in}{1.473908in}}{\pgfqpoint{1.542599in}{1.463309in}}{\pgfqpoint{1.550412in}{1.455495in}}%
\pgfpathcurveto{\pgfqpoint{1.558226in}{1.447682in}}{\pgfqpoint{1.568825in}{1.443291in}}{\pgfqpoint{1.579875in}{1.443291in}}%
\pgfpathclose%
\pgfusepath{stroke,fill}%
\end{pgfscope}%
\begin{pgfscope}%
\pgfpathrectangle{\pgfqpoint{0.375000in}{0.330000in}}{\pgfqpoint{2.325000in}{2.310000in}}%
\pgfusepath{clip}%
\pgfsetbuttcap%
\pgfsetroundjoin%
\definecolor{currentfill}{rgb}{0.000000,0.000000,0.000000}%
\pgfsetfillcolor{currentfill}%
\pgfsetlinewidth{1.003750pt}%
\definecolor{currentstroke}{rgb}{0.000000,0.000000,0.000000}%
\pgfsetstrokecolor{currentstroke}%
\pgfsetdash{}{0pt}%
\pgfpathmoveto{\pgfqpoint{1.579875in}{1.443291in}}%
\pgfpathcurveto{\pgfqpoint{1.590925in}{1.443291in}}{\pgfqpoint{1.601524in}{1.447682in}}{\pgfqpoint{1.609338in}{1.455495in}}%
\pgfpathcurveto{\pgfqpoint{1.617151in}{1.463309in}}{\pgfqpoint{1.621542in}{1.473908in}}{\pgfqpoint{1.621542in}{1.484958in}}%
\pgfpathcurveto{\pgfqpoint{1.621542in}{1.496008in}}{\pgfqpoint{1.617151in}{1.506607in}}{\pgfqpoint{1.609338in}{1.514421in}}%
\pgfpathcurveto{\pgfqpoint{1.601524in}{1.522235in}}{\pgfqpoint{1.590925in}{1.526625in}}{\pgfqpoint{1.579875in}{1.526625in}}%
\pgfpathcurveto{\pgfqpoint{1.568825in}{1.526625in}}{\pgfqpoint{1.558226in}{1.522235in}}{\pgfqpoint{1.550412in}{1.514421in}}%
\pgfpathcurveto{\pgfqpoint{1.542599in}{1.506607in}}{\pgfqpoint{1.538208in}{1.496008in}}{\pgfqpoint{1.538208in}{1.484958in}}%
\pgfpathcurveto{\pgfqpoint{1.538208in}{1.473908in}}{\pgfqpoint{1.542599in}{1.463309in}}{\pgfqpoint{1.550412in}{1.455495in}}%
\pgfpathcurveto{\pgfqpoint{1.558226in}{1.447682in}}{\pgfqpoint{1.568825in}{1.443291in}}{\pgfqpoint{1.579875in}{1.443291in}}%
\pgfpathclose%
\pgfusepath{stroke,fill}%
\end{pgfscope}%
\begin{pgfscope}%
\pgfpathrectangle{\pgfqpoint{0.375000in}{0.330000in}}{\pgfqpoint{2.325000in}{2.310000in}}%
\pgfusepath{clip}%
\pgfsetbuttcap%
\pgfsetroundjoin%
\definecolor{currentfill}{rgb}{0.000000,0.000000,0.000000}%
\pgfsetfillcolor{currentfill}%
\pgfsetlinewidth{1.003750pt}%
\definecolor{currentstroke}{rgb}{0.000000,0.000000,0.000000}%
\pgfsetstrokecolor{currentstroke}%
\pgfsetdash{}{0pt}%
\pgfpathmoveto{\pgfqpoint{1.579875in}{1.443291in}}%
\pgfpathcurveto{\pgfqpoint{1.590925in}{1.443291in}}{\pgfqpoint{1.601524in}{1.447682in}}{\pgfqpoint{1.609338in}{1.455495in}}%
\pgfpathcurveto{\pgfqpoint{1.617151in}{1.463309in}}{\pgfqpoint{1.621542in}{1.473908in}}{\pgfqpoint{1.621542in}{1.484958in}}%
\pgfpathcurveto{\pgfqpoint{1.621542in}{1.496008in}}{\pgfqpoint{1.617151in}{1.506607in}}{\pgfqpoint{1.609338in}{1.514421in}}%
\pgfpathcurveto{\pgfqpoint{1.601524in}{1.522235in}}{\pgfqpoint{1.590925in}{1.526625in}}{\pgfqpoint{1.579875in}{1.526625in}}%
\pgfpathcurveto{\pgfqpoint{1.568825in}{1.526625in}}{\pgfqpoint{1.558226in}{1.522235in}}{\pgfqpoint{1.550412in}{1.514421in}}%
\pgfpathcurveto{\pgfqpoint{1.542599in}{1.506607in}}{\pgfqpoint{1.538208in}{1.496008in}}{\pgfqpoint{1.538208in}{1.484958in}}%
\pgfpathcurveto{\pgfqpoint{1.538208in}{1.473908in}}{\pgfqpoint{1.542599in}{1.463309in}}{\pgfqpoint{1.550412in}{1.455495in}}%
\pgfpathcurveto{\pgfqpoint{1.558226in}{1.447682in}}{\pgfqpoint{1.568825in}{1.443291in}}{\pgfqpoint{1.579875in}{1.443291in}}%
\pgfpathclose%
\pgfusepath{stroke,fill}%
\end{pgfscope}%
\begin{pgfscope}%
\pgfpathrectangle{\pgfqpoint{0.375000in}{0.330000in}}{\pgfqpoint{2.325000in}{2.310000in}}%
\pgfusepath{clip}%
\pgfsetbuttcap%
\pgfsetroundjoin%
\definecolor{currentfill}{rgb}{0.000000,0.000000,0.000000}%
\pgfsetfillcolor{currentfill}%
\pgfsetlinewidth{1.003750pt}%
\definecolor{currentstroke}{rgb}{0.000000,0.000000,0.000000}%
\pgfsetstrokecolor{currentstroke}%
\pgfsetdash{}{0pt}%
\pgfpathmoveto{\pgfqpoint{1.579875in}{1.443291in}}%
\pgfpathcurveto{\pgfqpoint{1.590925in}{1.443291in}}{\pgfqpoint{1.601524in}{1.447682in}}{\pgfqpoint{1.609338in}{1.455495in}}%
\pgfpathcurveto{\pgfqpoint{1.617151in}{1.463309in}}{\pgfqpoint{1.621542in}{1.473908in}}{\pgfqpoint{1.621542in}{1.484958in}}%
\pgfpathcurveto{\pgfqpoint{1.621542in}{1.496008in}}{\pgfqpoint{1.617151in}{1.506607in}}{\pgfqpoint{1.609338in}{1.514421in}}%
\pgfpathcurveto{\pgfqpoint{1.601524in}{1.522235in}}{\pgfqpoint{1.590925in}{1.526625in}}{\pgfqpoint{1.579875in}{1.526625in}}%
\pgfpathcurveto{\pgfqpoint{1.568825in}{1.526625in}}{\pgfqpoint{1.558226in}{1.522235in}}{\pgfqpoint{1.550412in}{1.514421in}}%
\pgfpathcurveto{\pgfqpoint{1.542599in}{1.506607in}}{\pgfqpoint{1.538208in}{1.496008in}}{\pgfqpoint{1.538208in}{1.484958in}}%
\pgfpathcurveto{\pgfqpoint{1.538208in}{1.473908in}}{\pgfqpoint{1.542599in}{1.463309in}}{\pgfqpoint{1.550412in}{1.455495in}}%
\pgfpathcurveto{\pgfqpoint{1.558226in}{1.447682in}}{\pgfqpoint{1.568825in}{1.443291in}}{\pgfqpoint{1.579875in}{1.443291in}}%
\pgfpathclose%
\pgfusepath{stroke,fill}%
\end{pgfscope}%
\begin{pgfscope}%
\pgfpathrectangle{\pgfqpoint{0.375000in}{0.330000in}}{\pgfqpoint{2.325000in}{2.310000in}}%
\pgfusepath{clip}%
\pgfsetbuttcap%
\pgfsetroundjoin%
\definecolor{currentfill}{rgb}{0.000000,0.000000,0.000000}%
\pgfsetfillcolor{currentfill}%
\pgfsetlinewidth{1.003750pt}%
\definecolor{currentstroke}{rgb}{0.000000,0.000000,0.000000}%
\pgfsetstrokecolor{currentstroke}%
\pgfsetdash{}{0pt}%
\pgfpathmoveto{\pgfqpoint{1.579875in}{1.443291in}}%
\pgfpathcurveto{\pgfqpoint{1.590925in}{1.443291in}}{\pgfqpoint{1.601524in}{1.447682in}}{\pgfqpoint{1.609338in}{1.455495in}}%
\pgfpathcurveto{\pgfqpoint{1.617151in}{1.463309in}}{\pgfqpoint{1.621542in}{1.473908in}}{\pgfqpoint{1.621542in}{1.484958in}}%
\pgfpathcurveto{\pgfqpoint{1.621542in}{1.496008in}}{\pgfqpoint{1.617151in}{1.506607in}}{\pgfqpoint{1.609338in}{1.514421in}}%
\pgfpathcurveto{\pgfqpoint{1.601524in}{1.522235in}}{\pgfqpoint{1.590925in}{1.526625in}}{\pgfqpoint{1.579875in}{1.526625in}}%
\pgfpathcurveto{\pgfqpoint{1.568825in}{1.526625in}}{\pgfqpoint{1.558226in}{1.522235in}}{\pgfqpoint{1.550412in}{1.514421in}}%
\pgfpathcurveto{\pgfqpoint{1.542599in}{1.506607in}}{\pgfqpoint{1.538208in}{1.496008in}}{\pgfqpoint{1.538208in}{1.484958in}}%
\pgfpathcurveto{\pgfqpoint{1.538208in}{1.473908in}}{\pgfqpoint{1.542599in}{1.463309in}}{\pgfqpoint{1.550412in}{1.455495in}}%
\pgfpathcurveto{\pgfqpoint{1.558226in}{1.447682in}}{\pgfqpoint{1.568825in}{1.443291in}}{\pgfqpoint{1.579875in}{1.443291in}}%
\pgfpathclose%
\pgfusepath{stroke,fill}%
\end{pgfscope}%
\begin{pgfscope}%
\pgfpathrectangle{\pgfqpoint{0.375000in}{0.330000in}}{\pgfqpoint{2.325000in}{2.310000in}}%
\pgfusepath{clip}%
\pgfsetbuttcap%
\pgfsetroundjoin%
\definecolor{currentfill}{rgb}{0.000000,0.000000,0.000000}%
\pgfsetfillcolor{currentfill}%
\pgfsetlinewidth{1.003750pt}%
\definecolor{currentstroke}{rgb}{0.000000,0.000000,0.000000}%
\pgfsetstrokecolor{currentstroke}%
\pgfsetdash{}{0pt}%
\pgfpathmoveto{\pgfqpoint{1.579875in}{2.474583in}}%
\pgfpathcurveto{\pgfqpoint{1.590925in}{2.474583in}}{\pgfqpoint{1.601524in}{2.478974in}}{\pgfqpoint{1.609338in}{2.486787in}}%
\pgfpathcurveto{\pgfqpoint{1.617151in}{2.494601in}}{\pgfqpoint{1.621542in}{2.505200in}}{\pgfqpoint{1.621542in}{2.516250in}}%
\pgfpathcurveto{\pgfqpoint{1.621542in}{2.527300in}}{\pgfqpoint{1.617151in}{2.537899in}}{\pgfqpoint{1.609338in}{2.545713in}}%
\pgfpathcurveto{\pgfqpoint{1.601524in}{2.553526in}}{\pgfqpoint{1.590925in}{2.557917in}}{\pgfqpoint{1.579875in}{2.557917in}}%
\pgfpathcurveto{\pgfqpoint{1.568825in}{2.557917in}}{\pgfqpoint{1.558226in}{2.553526in}}{\pgfqpoint{1.550412in}{2.545713in}}%
\pgfpathcurveto{\pgfqpoint{1.542599in}{2.537899in}}{\pgfqpoint{1.538208in}{2.527300in}}{\pgfqpoint{1.538208in}{2.516250in}}%
\pgfpathcurveto{\pgfqpoint{1.538208in}{2.505200in}}{\pgfqpoint{1.542599in}{2.494601in}}{\pgfqpoint{1.550412in}{2.486787in}}%
\pgfpathcurveto{\pgfqpoint{1.558226in}{2.478974in}}{\pgfqpoint{1.568825in}{2.474583in}}{\pgfqpoint{1.579875in}{2.474583in}}%
\pgfpathclose%
\pgfusepath{stroke,fill}%
\end{pgfscope}%
\begin{pgfscope}%
\pgfpathrectangle{\pgfqpoint{0.375000in}{0.330000in}}{\pgfqpoint{2.325000in}{2.310000in}}%
\pgfusepath{clip}%
\pgfsetbuttcap%
\pgfsetroundjoin%
\definecolor{currentfill}{rgb}{0.000000,0.000000,0.000000}%
\pgfsetfillcolor{currentfill}%
\pgfsetlinewidth{1.003750pt}%
\definecolor{currentstroke}{rgb}{0.000000,0.000000,0.000000}%
\pgfsetstrokecolor{currentstroke}%
\pgfsetdash{}{0pt}%
\pgfpathmoveto{\pgfqpoint{1.579875in}{2.474583in}}%
\pgfpathcurveto{\pgfqpoint{1.590925in}{2.474583in}}{\pgfqpoint{1.601524in}{2.478974in}}{\pgfqpoint{1.609338in}{2.486787in}}%
\pgfpathcurveto{\pgfqpoint{1.617151in}{2.494601in}}{\pgfqpoint{1.621542in}{2.505200in}}{\pgfqpoint{1.621542in}{2.516250in}}%
\pgfpathcurveto{\pgfqpoint{1.621542in}{2.527300in}}{\pgfqpoint{1.617151in}{2.537899in}}{\pgfqpoint{1.609338in}{2.545713in}}%
\pgfpathcurveto{\pgfqpoint{1.601524in}{2.553526in}}{\pgfqpoint{1.590925in}{2.557917in}}{\pgfqpoint{1.579875in}{2.557917in}}%
\pgfpathcurveto{\pgfqpoint{1.568825in}{2.557917in}}{\pgfqpoint{1.558226in}{2.553526in}}{\pgfqpoint{1.550412in}{2.545713in}}%
\pgfpathcurveto{\pgfqpoint{1.542599in}{2.537899in}}{\pgfqpoint{1.538208in}{2.527300in}}{\pgfqpoint{1.538208in}{2.516250in}}%
\pgfpathcurveto{\pgfqpoint{1.538208in}{2.505200in}}{\pgfqpoint{1.542599in}{2.494601in}}{\pgfqpoint{1.550412in}{2.486787in}}%
\pgfpathcurveto{\pgfqpoint{1.558226in}{2.478974in}}{\pgfqpoint{1.568825in}{2.474583in}}{\pgfqpoint{1.579875in}{2.474583in}}%
\pgfpathclose%
\pgfusepath{stroke,fill}%
\end{pgfscope}%
\begin{pgfscope}%
\pgfpathrectangle{\pgfqpoint{0.375000in}{0.330000in}}{\pgfqpoint{2.325000in}{2.310000in}}%
\pgfusepath{clip}%
\pgfsetbuttcap%
\pgfsetroundjoin%
\definecolor{currentfill}{rgb}{0.000000,0.000000,0.000000}%
\pgfsetfillcolor{currentfill}%
\pgfsetlinewidth{1.003750pt}%
\definecolor{currentstroke}{rgb}{0.000000,0.000000,0.000000}%
\pgfsetstrokecolor{currentstroke}%
\pgfsetdash{}{0pt}%
\pgfpathmoveto{\pgfqpoint{1.579875in}{1.443291in}}%
\pgfpathcurveto{\pgfqpoint{1.590925in}{1.443291in}}{\pgfqpoint{1.601524in}{1.447682in}}{\pgfqpoint{1.609338in}{1.455495in}}%
\pgfpathcurveto{\pgfqpoint{1.617151in}{1.463309in}}{\pgfqpoint{1.621542in}{1.473908in}}{\pgfqpoint{1.621542in}{1.484958in}}%
\pgfpathcurveto{\pgfqpoint{1.621542in}{1.496008in}}{\pgfqpoint{1.617151in}{1.506607in}}{\pgfqpoint{1.609338in}{1.514421in}}%
\pgfpathcurveto{\pgfqpoint{1.601524in}{1.522235in}}{\pgfqpoint{1.590925in}{1.526625in}}{\pgfqpoint{1.579875in}{1.526625in}}%
\pgfpathcurveto{\pgfqpoint{1.568825in}{1.526625in}}{\pgfqpoint{1.558226in}{1.522235in}}{\pgfqpoint{1.550412in}{1.514421in}}%
\pgfpathcurveto{\pgfqpoint{1.542599in}{1.506607in}}{\pgfqpoint{1.538208in}{1.496008in}}{\pgfqpoint{1.538208in}{1.484958in}}%
\pgfpathcurveto{\pgfqpoint{1.538208in}{1.473908in}}{\pgfqpoint{1.542599in}{1.463309in}}{\pgfqpoint{1.550412in}{1.455495in}}%
\pgfpathcurveto{\pgfqpoint{1.558226in}{1.447682in}}{\pgfqpoint{1.568825in}{1.443291in}}{\pgfqpoint{1.579875in}{1.443291in}}%
\pgfpathclose%
\pgfusepath{stroke,fill}%
\end{pgfscope}%
\begin{pgfscope}%
\pgfpathrectangle{\pgfqpoint{0.375000in}{0.330000in}}{\pgfqpoint{2.325000in}{2.310000in}}%
\pgfusepath{clip}%
\pgfsetbuttcap%
\pgfsetroundjoin%
\definecolor{currentfill}{rgb}{0.000000,0.000000,0.000000}%
\pgfsetfillcolor{currentfill}%
\pgfsetlinewidth{1.003750pt}%
\definecolor{currentstroke}{rgb}{0.000000,0.000000,0.000000}%
\pgfsetstrokecolor{currentstroke}%
\pgfsetdash{}{0pt}%
\pgfpathmoveto{\pgfqpoint{1.579875in}{2.474583in}}%
\pgfpathcurveto{\pgfqpoint{1.590925in}{2.474583in}}{\pgfqpoint{1.601524in}{2.478974in}}{\pgfqpoint{1.609338in}{2.486787in}}%
\pgfpathcurveto{\pgfqpoint{1.617151in}{2.494601in}}{\pgfqpoint{1.621542in}{2.505200in}}{\pgfqpoint{1.621542in}{2.516250in}}%
\pgfpathcurveto{\pgfqpoint{1.621542in}{2.527300in}}{\pgfqpoint{1.617151in}{2.537899in}}{\pgfqpoint{1.609338in}{2.545713in}}%
\pgfpathcurveto{\pgfqpoint{1.601524in}{2.553526in}}{\pgfqpoint{1.590925in}{2.557917in}}{\pgfqpoint{1.579875in}{2.557917in}}%
\pgfpathcurveto{\pgfqpoint{1.568825in}{2.557917in}}{\pgfqpoint{1.558226in}{2.553526in}}{\pgfqpoint{1.550412in}{2.545713in}}%
\pgfpathcurveto{\pgfqpoint{1.542599in}{2.537899in}}{\pgfqpoint{1.538208in}{2.527300in}}{\pgfqpoint{1.538208in}{2.516250in}}%
\pgfpathcurveto{\pgfqpoint{1.538208in}{2.505200in}}{\pgfqpoint{1.542599in}{2.494601in}}{\pgfqpoint{1.550412in}{2.486787in}}%
\pgfpathcurveto{\pgfqpoint{1.558226in}{2.478974in}}{\pgfqpoint{1.568825in}{2.474583in}}{\pgfqpoint{1.579875in}{2.474583in}}%
\pgfpathclose%
\pgfusepath{stroke,fill}%
\end{pgfscope}%
\begin{pgfscope}%
\pgfpathrectangle{\pgfqpoint{0.375000in}{0.330000in}}{\pgfqpoint{2.325000in}{2.310000in}}%
\pgfusepath{clip}%
\pgfsetbuttcap%
\pgfsetroundjoin%
\definecolor{currentfill}{rgb}{0.000000,0.000000,0.000000}%
\pgfsetfillcolor{currentfill}%
\pgfsetlinewidth{1.003750pt}%
\definecolor{currentstroke}{rgb}{0.000000,0.000000,0.000000}%
\pgfsetstrokecolor{currentstroke}%
\pgfsetdash{}{0pt}%
\pgfpathmoveto{\pgfqpoint{1.579875in}{1.443291in}}%
\pgfpathcurveto{\pgfqpoint{1.590925in}{1.443291in}}{\pgfqpoint{1.601524in}{1.447682in}}{\pgfqpoint{1.609338in}{1.455495in}}%
\pgfpathcurveto{\pgfqpoint{1.617151in}{1.463309in}}{\pgfqpoint{1.621542in}{1.473908in}}{\pgfqpoint{1.621542in}{1.484958in}}%
\pgfpathcurveto{\pgfqpoint{1.621542in}{1.496008in}}{\pgfqpoint{1.617151in}{1.506607in}}{\pgfqpoint{1.609338in}{1.514421in}}%
\pgfpathcurveto{\pgfqpoint{1.601524in}{1.522235in}}{\pgfqpoint{1.590925in}{1.526625in}}{\pgfqpoint{1.579875in}{1.526625in}}%
\pgfpathcurveto{\pgfqpoint{1.568825in}{1.526625in}}{\pgfqpoint{1.558226in}{1.522235in}}{\pgfqpoint{1.550412in}{1.514421in}}%
\pgfpathcurveto{\pgfqpoint{1.542599in}{1.506607in}}{\pgfqpoint{1.538208in}{1.496008in}}{\pgfqpoint{1.538208in}{1.484958in}}%
\pgfpathcurveto{\pgfqpoint{1.538208in}{1.473908in}}{\pgfqpoint{1.542599in}{1.463309in}}{\pgfqpoint{1.550412in}{1.455495in}}%
\pgfpathcurveto{\pgfqpoint{1.558226in}{1.447682in}}{\pgfqpoint{1.568825in}{1.443291in}}{\pgfqpoint{1.579875in}{1.443291in}}%
\pgfpathclose%
\pgfusepath{stroke,fill}%
\end{pgfscope}%
\begin{pgfscope}%
\pgfpathrectangle{\pgfqpoint{0.375000in}{0.330000in}}{\pgfqpoint{2.325000in}{2.310000in}}%
\pgfusepath{clip}%
\pgfsetbuttcap%
\pgfsetroundjoin%
\definecolor{currentfill}{rgb}{0.000000,0.000000,0.000000}%
\pgfsetfillcolor{currentfill}%
\pgfsetlinewidth{1.003750pt}%
\definecolor{currentstroke}{rgb}{0.000000,0.000000,0.000000}%
\pgfsetstrokecolor{currentstroke}%
\pgfsetdash{}{0pt}%
\pgfpathmoveto{\pgfqpoint{1.579875in}{1.443291in}}%
\pgfpathcurveto{\pgfqpoint{1.590925in}{1.443291in}}{\pgfqpoint{1.601524in}{1.447682in}}{\pgfqpoint{1.609338in}{1.455495in}}%
\pgfpathcurveto{\pgfqpoint{1.617151in}{1.463309in}}{\pgfqpoint{1.621542in}{1.473908in}}{\pgfqpoint{1.621542in}{1.484958in}}%
\pgfpathcurveto{\pgfqpoint{1.621542in}{1.496008in}}{\pgfqpoint{1.617151in}{1.506607in}}{\pgfqpoint{1.609338in}{1.514421in}}%
\pgfpathcurveto{\pgfqpoint{1.601524in}{1.522235in}}{\pgfqpoint{1.590925in}{1.526625in}}{\pgfqpoint{1.579875in}{1.526625in}}%
\pgfpathcurveto{\pgfqpoint{1.568825in}{1.526625in}}{\pgfqpoint{1.558226in}{1.522235in}}{\pgfqpoint{1.550412in}{1.514421in}}%
\pgfpathcurveto{\pgfqpoint{1.542599in}{1.506607in}}{\pgfqpoint{1.538208in}{1.496008in}}{\pgfqpoint{1.538208in}{1.484958in}}%
\pgfpathcurveto{\pgfqpoint{1.538208in}{1.473908in}}{\pgfqpoint{1.542599in}{1.463309in}}{\pgfqpoint{1.550412in}{1.455495in}}%
\pgfpathcurveto{\pgfqpoint{1.558226in}{1.447682in}}{\pgfqpoint{1.568825in}{1.443291in}}{\pgfqpoint{1.579875in}{1.443291in}}%
\pgfpathclose%
\pgfusepath{stroke,fill}%
\end{pgfscope}%
\begin{pgfscope}%
\pgfpathrectangle{\pgfqpoint{0.375000in}{0.330000in}}{\pgfqpoint{2.325000in}{2.310000in}}%
\pgfusepath{clip}%
\pgfsetbuttcap%
\pgfsetroundjoin%
\definecolor{currentfill}{rgb}{0.000000,0.000000,0.000000}%
\pgfsetfillcolor{currentfill}%
\pgfsetlinewidth{1.003750pt}%
\definecolor{currentstroke}{rgb}{0.000000,0.000000,0.000000}%
\pgfsetstrokecolor{currentstroke}%
\pgfsetdash{}{0pt}%
\pgfpathmoveto{\pgfqpoint{1.579875in}{1.443291in}}%
\pgfpathcurveto{\pgfqpoint{1.590925in}{1.443291in}}{\pgfqpoint{1.601524in}{1.447682in}}{\pgfqpoint{1.609338in}{1.455495in}}%
\pgfpathcurveto{\pgfqpoint{1.617151in}{1.463309in}}{\pgfqpoint{1.621542in}{1.473908in}}{\pgfqpoint{1.621542in}{1.484958in}}%
\pgfpathcurveto{\pgfqpoint{1.621542in}{1.496008in}}{\pgfqpoint{1.617151in}{1.506607in}}{\pgfqpoint{1.609338in}{1.514421in}}%
\pgfpathcurveto{\pgfqpoint{1.601524in}{1.522235in}}{\pgfqpoint{1.590925in}{1.526625in}}{\pgfqpoint{1.579875in}{1.526625in}}%
\pgfpathcurveto{\pgfqpoint{1.568825in}{1.526625in}}{\pgfqpoint{1.558226in}{1.522235in}}{\pgfqpoint{1.550412in}{1.514421in}}%
\pgfpathcurveto{\pgfqpoint{1.542599in}{1.506607in}}{\pgfqpoint{1.538208in}{1.496008in}}{\pgfqpoint{1.538208in}{1.484958in}}%
\pgfpathcurveto{\pgfqpoint{1.538208in}{1.473908in}}{\pgfqpoint{1.542599in}{1.463309in}}{\pgfqpoint{1.550412in}{1.455495in}}%
\pgfpathcurveto{\pgfqpoint{1.558226in}{1.447682in}}{\pgfqpoint{1.568825in}{1.443291in}}{\pgfqpoint{1.579875in}{1.443291in}}%
\pgfpathclose%
\pgfusepath{stroke,fill}%
\end{pgfscope}%
\begin{pgfscope}%
\pgfpathrectangle{\pgfqpoint{0.375000in}{0.330000in}}{\pgfqpoint{2.325000in}{2.310000in}}%
\pgfusepath{clip}%
\pgfsetbuttcap%
\pgfsetroundjoin%
\definecolor{currentfill}{rgb}{0.000000,0.000000,0.000000}%
\pgfsetfillcolor{currentfill}%
\pgfsetlinewidth{1.003750pt}%
\definecolor{currentstroke}{rgb}{0.000000,0.000000,0.000000}%
\pgfsetstrokecolor{currentstroke}%
\pgfsetdash{}{0pt}%
\pgfpathmoveto{\pgfqpoint{1.579875in}{1.443291in}}%
\pgfpathcurveto{\pgfqpoint{1.590925in}{1.443291in}}{\pgfqpoint{1.601524in}{1.447682in}}{\pgfqpoint{1.609338in}{1.455495in}}%
\pgfpathcurveto{\pgfqpoint{1.617151in}{1.463309in}}{\pgfqpoint{1.621542in}{1.473908in}}{\pgfqpoint{1.621542in}{1.484958in}}%
\pgfpathcurveto{\pgfqpoint{1.621542in}{1.496008in}}{\pgfqpoint{1.617151in}{1.506607in}}{\pgfqpoint{1.609338in}{1.514421in}}%
\pgfpathcurveto{\pgfqpoint{1.601524in}{1.522235in}}{\pgfqpoint{1.590925in}{1.526625in}}{\pgfqpoint{1.579875in}{1.526625in}}%
\pgfpathcurveto{\pgfqpoint{1.568825in}{1.526625in}}{\pgfqpoint{1.558226in}{1.522235in}}{\pgfqpoint{1.550412in}{1.514421in}}%
\pgfpathcurveto{\pgfqpoint{1.542599in}{1.506607in}}{\pgfqpoint{1.538208in}{1.496008in}}{\pgfqpoint{1.538208in}{1.484958in}}%
\pgfpathcurveto{\pgfqpoint{1.538208in}{1.473908in}}{\pgfqpoint{1.542599in}{1.463309in}}{\pgfqpoint{1.550412in}{1.455495in}}%
\pgfpathcurveto{\pgfqpoint{1.558226in}{1.447682in}}{\pgfqpoint{1.568825in}{1.443291in}}{\pgfqpoint{1.579875in}{1.443291in}}%
\pgfpathclose%
\pgfusepath{stroke,fill}%
\end{pgfscope}%
\begin{pgfscope}%
\pgfpathrectangle{\pgfqpoint{0.375000in}{0.330000in}}{\pgfqpoint{2.325000in}{2.310000in}}%
\pgfusepath{clip}%
\pgfsetbuttcap%
\pgfsetroundjoin%
\definecolor{currentfill}{rgb}{0.000000,0.000000,0.000000}%
\pgfsetfillcolor{currentfill}%
\pgfsetlinewidth{1.003750pt}%
\definecolor{currentstroke}{rgb}{0.000000,0.000000,0.000000}%
\pgfsetstrokecolor{currentstroke}%
\pgfsetdash{}{0pt}%
\pgfpathmoveto{\pgfqpoint{1.579875in}{1.443291in}}%
\pgfpathcurveto{\pgfqpoint{1.590925in}{1.443291in}}{\pgfqpoint{1.601524in}{1.447682in}}{\pgfqpoint{1.609338in}{1.455495in}}%
\pgfpathcurveto{\pgfqpoint{1.617151in}{1.463309in}}{\pgfqpoint{1.621542in}{1.473908in}}{\pgfqpoint{1.621542in}{1.484958in}}%
\pgfpathcurveto{\pgfqpoint{1.621542in}{1.496008in}}{\pgfqpoint{1.617151in}{1.506607in}}{\pgfqpoint{1.609338in}{1.514421in}}%
\pgfpathcurveto{\pgfqpoint{1.601524in}{1.522235in}}{\pgfqpoint{1.590925in}{1.526625in}}{\pgfqpoint{1.579875in}{1.526625in}}%
\pgfpathcurveto{\pgfqpoint{1.568825in}{1.526625in}}{\pgfqpoint{1.558226in}{1.522235in}}{\pgfqpoint{1.550412in}{1.514421in}}%
\pgfpathcurveto{\pgfqpoint{1.542599in}{1.506607in}}{\pgfqpoint{1.538208in}{1.496008in}}{\pgfqpoint{1.538208in}{1.484958in}}%
\pgfpathcurveto{\pgfqpoint{1.538208in}{1.473908in}}{\pgfqpoint{1.542599in}{1.463309in}}{\pgfqpoint{1.550412in}{1.455495in}}%
\pgfpathcurveto{\pgfqpoint{1.558226in}{1.447682in}}{\pgfqpoint{1.568825in}{1.443291in}}{\pgfqpoint{1.579875in}{1.443291in}}%
\pgfpathclose%
\pgfusepath{stroke,fill}%
\end{pgfscope}%
\begin{pgfscope}%
\pgfpathrectangle{\pgfqpoint{0.375000in}{0.330000in}}{\pgfqpoint{2.325000in}{2.310000in}}%
\pgfusepath{clip}%
\pgfsetbuttcap%
\pgfsetroundjoin%
\definecolor{currentfill}{rgb}{0.000000,0.000000,0.000000}%
\pgfsetfillcolor{currentfill}%
\pgfsetlinewidth{1.003750pt}%
\definecolor{currentstroke}{rgb}{0.000000,0.000000,0.000000}%
\pgfsetstrokecolor{currentstroke}%
\pgfsetdash{}{0pt}%
\pgfpathmoveto{\pgfqpoint{1.579875in}{1.443291in}}%
\pgfpathcurveto{\pgfqpoint{1.590925in}{1.443291in}}{\pgfqpoint{1.601524in}{1.447682in}}{\pgfqpoint{1.609338in}{1.455495in}}%
\pgfpathcurveto{\pgfqpoint{1.617151in}{1.463309in}}{\pgfqpoint{1.621542in}{1.473908in}}{\pgfqpoint{1.621542in}{1.484958in}}%
\pgfpathcurveto{\pgfqpoint{1.621542in}{1.496008in}}{\pgfqpoint{1.617151in}{1.506607in}}{\pgfqpoint{1.609338in}{1.514421in}}%
\pgfpathcurveto{\pgfqpoint{1.601524in}{1.522235in}}{\pgfqpoint{1.590925in}{1.526625in}}{\pgfqpoint{1.579875in}{1.526625in}}%
\pgfpathcurveto{\pgfqpoint{1.568825in}{1.526625in}}{\pgfqpoint{1.558226in}{1.522235in}}{\pgfqpoint{1.550412in}{1.514421in}}%
\pgfpathcurveto{\pgfqpoint{1.542599in}{1.506607in}}{\pgfqpoint{1.538208in}{1.496008in}}{\pgfqpoint{1.538208in}{1.484958in}}%
\pgfpathcurveto{\pgfqpoint{1.538208in}{1.473908in}}{\pgfqpoint{1.542599in}{1.463309in}}{\pgfqpoint{1.550412in}{1.455495in}}%
\pgfpathcurveto{\pgfqpoint{1.558226in}{1.447682in}}{\pgfqpoint{1.568825in}{1.443291in}}{\pgfqpoint{1.579875in}{1.443291in}}%
\pgfpathclose%
\pgfusepath{stroke,fill}%
\end{pgfscope}%
\begin{pgfscope}%
\pgfpathrectangle{\pgfqpoint{0.375000in}{0.330000in}}{\pgfqpoint{2.325000in}{2.310000in}}%
\pgfusepath{clip}%
\pgfsetbuttcap%
\pgfsetroundjoin%
\definecolor{currentfill}{rgb}{0.000000,0.000000,0.000000}%
\pgfsetfillcolor{currentfill}%
\pgfsetlinewidth{1.003750pt}%
\definecolor{currentstroke}{rgb}{0.000000,0.000000,0.000000}%
\pgfsetstrokecolor{currentstroke}%
\pgfsetdash{}{0pt}%
\pgfpathmoveto{\pgfqpoint{1.579875in}{1.443291in}}%
\pgfpathcurveto{\pgfqpoint{1.590925in}{1.443291in}}{\pgfqpoint{1.601524in}{1.447682in}}{\pgfqpoint{1.609338in}{1.455495in}}%
\pgfpathcurveto{\pgfqpoint{1.617151in}{1.463309in}}{\pgfqpoint{1.621542in}{1.473908in}}{\pgfqpoint{1.621542in}{1.484958in}}%
\pgfpathcurveto{\pgfqpoint{1.621542in}{1.496008in}}{\pgfqpoint{1.617151in}{1.506607in}}{\pgfqpoint{1.609338in}{1.514421in}}%
\pgfpathcurveto{\pgfqpoint{1.601524in}{1.522235in}}{\pgfqpoint{1.590925in}{1.526625in}}{\pgfqpoint{1.579875in}{1.526625in}}%
\pgfpathcurveto{\pgfqpoint{1.568825in}{1.526625in}}{\pgfqpoint{1.558226in}{1.522235in}}{\pgfqpoint{1.550412in}{1.514421in}}%
\pgfpathcurveto{\pgfqpoint{1.542599in}{1.506607in}}{\pgfqpoint{1.538208in}{1.496008in}}{\pgfqpoint{1.538208in}{1.484958in}}%
\pgfpathcurveto{\pgfqpoint{1.538208in}{1.473908in}}{\pgfqpoint{1.542599in}{1.463309in}}{\pgfqpoint{1.550412in}{1.455495in}}%
\pgfpathcurveto{\pgfqpoint{1.558226in}{1.447682in}}{\pgfqpoint{1.568825in}{1.443291in}}{\pgfqpoint{1.579875in}{1.443291in}}%
\pgfpathclose%
\pgfusepath{stroke,fill}%
\end{pgfscope}%
\begin{pgfscope}%
\pgfpathrectangle{\pgfqpoint{0.375000in}{0.330000in}}{\pgfqpoint{2.325000in}{2.310000in}}%
\pgfusepath{clip}%
\pgfsetbuttcap%
\pgfsetroundjoin%
\definecolor{currentfill}{rgb}{0.000000,0.000000,0.000000}%
\pgfsetfillcolor{currentfill}%
\pgfsetlinewidth{1.003750pt}%
\definecolor{currentstroke}{rgb}{0.000000,0.000000,0.000000}%
\pgfsetstrokecolor{currentstroke}%
\pgfsetdash{}{0pt}%
\pgfpathmoveto{\pgfqpoint{1.579875in}{1.443291in}}%
\pgfpathcurveto{\pgfqpoint{1.590925in}{1.443291in}}{\pgfqpoint{1.601524in}{1.447682in}}{\pgfqpoint{1.609338in}{1.455495in}}%
\pgfpathcurveto{\pgfqpoint{1.617151in}{1.463309in}}{\pgfqpoint{1.621542in}{1.473908in}}{\pgfqpoint{1.621542in}{1.484958in}}%
\pgfpathcurveto{\pgfqpoint{1.621542in}{1.496008in}}{\pgfqpoint{1.617151in}{1.506607in}}{\pgfqpoint{1.609338in}{1.514421in}}%
\pgfpathcurveto{\pgfqpoint{1.601524in}{1.522235in}}{\pgfqpoint{1.590925in}{1.526625in}}{\pgfqpoint{1.579875in}{1.526625in}}%
\pgfpathcurveto{\pgfqpoint{1.568825in}{1.526625in}}{\pgfqpoint{1.558226in}{1.522235in}}{\pgfqpoint{1.550412in}{1.514421in}}%
\pgfpathcurveto{\pgfqpoint{1.542599in}{1.506607in}}{\pgfqpoint{1.538208in}{1.496008in}}{\pgfqpoint{1.538208in}{1.484958in}}%
\pgfpathcurveto{\pgfqpoint{1.538208in}{1.473908in}}{\pgfqpoint{1.542599in}{1.463309in}}{\pgfqpoint{1.550412in}{1.455495in}}%
\pgfpathcurveto{\pgfqpoint{1.558226in}{1.447682in}}{\pgfqpoint{1.568825in}{1.443291in}}{\pgfqpoint{1.579875in}{1.443291in}}%
\pgfpathclose%
\pgfusepath{stroke,fill}%
\end{pgfscope}%
\begin{pgfscope}%
\pgfpathrectangle{\pgfqpoint{0.375000in}{0.330000in}}{\pgfqpoint{2.325000in}{2.310000in}}%
\pgfusepath{clip}%
\pgfsetbuttcap%
\pgfsetroundjoin%
\definecolor{currentfill}{rgb}{0.000000,0.000000,0.000000}%
\pgfsetfillcolor{currentfill}%
\pgfsetlinewidth{1.003750pt}%
\definecolor{currentstroke}{rgb}{0.000000,0.000000,0.000000}%
\pgfsetstrokecolor{currentstroke}%
\pgfsetdash{}{0pt}%
\pgfpathmoveto{\pgfqpoint{1.579875in}{1.443291in}}%
\pgfpathcurveto{\pgfqpoint{1.590925in}{1.443291in}}{\pgfqpoint{1.601524in}{1.447682in}}{\pgfqpoint{1.609338in}{1.455495in}}%
\pgfpathcurveto{\pgfqpoint{1.617151in}{1.463309in}}{\pgfqpoint{1.621542in}{1.473908in}}{\pgfqpoint{1.621542in}{1.484958in}}%
\pgfpathcurveto{\pgfqpoint{1.621542in}{1.496008in}}{\pgfqpoint{1.617151in}{1.506607in}}{\pgfqpoint{1.609338in}{1.514421in}}%
\pgfpathcurveto{\pgfqpoint{1.601524in}{1.522235in}}{\pgfqpoint{1.590925in}{1.526625in}}{\pgfqpoint{1.579875in}{1.526625in}}%
\pgfpathcurveto{\pgfqpoint{1.568825in}{1.526625in}}{\pgfqpoint{1.558226in}{1.522235in}}{\pgfqpoint{1.550412in}{1.514421in}}%
\pgfpathcurveto{\pgfqpoint{1.542599in}{1.506607in}}{\pgfqpoint{1.538208in}{1.496008in}}{\pgfqpoint{1.538208in}{1.484958in}}%
\pgfpathcurveto{\pgfqpoint{1.538208in}{1.473908in}}{\pgfqpoint{1.542599in}{1.463309in}}{\pgfqpoint{1.550412in}{1.455495in}}%
\pgfpathcurveto{\pgfqpoint{1.558226in}{1.447682in}}{\pgfqpoint{1.568825in}{1.443291in}}{\pgfqpoint{1.579875in}{1.443291in}}%
\pgfpathclose%
\pgfusepath{stroke,fill}%
\end{pgfscope}%
\begin{pgfscope}%
\pgfpathrectangle{\pgfqpoint{0.375000in}{0.330000in}}{\pgfqpoint{2.325000in}{2.310000in}}%
\pgfusepath{clip}%
\pgfsetbuttcap%
\pgfsetroundjoin%
\definecolor{currentfill}{rgb}{0.000000,0.000000,0.000000}%
\pgfsetfillcolor{currentfill}%
\pgfsetlinewidth{1.003750pt}%
\definecolor{currentstroke}{rgb}{0.000000,0.000000,0.000000}%
\pgfsetstrokecolor{currentstroke}%
\pgfsetdash{}{0pt}%
\pgfpathmoveto{\pgfqpoint{1.579875in}{1.443291in}}%
\pgfpathcurveto{\pgfqpoint{1.590925in}{1.443291in}}{\pgfqpoint{1.601524in}{1.447682in}}{\pgfqpoint{1.609338in}{1.455495in}}%
\pgfpathcurveto{\pgfqpoint{1.617151in}{1.463309in}}{\pgfqpoint{1.621542in}{1.473908in}}{\pgfqpoint{1.621542in}{1.484958in}}%
\pgfpathcurveto{\pgfqpoint{1.621542in}{1.496008in}}{\pgfqpoint{1.617151in}{1.506607in}}{\pgfqpoint{1.609338in}{1.514421in}}%
\pgfpathcurveto{\pgfqpoint{1.601524in}{1.522235in}}{\pgfqpoint{1.590925in}{1.526625in}}{\pgfqpoint{1.579875in}{1.526625in}}%
\pgfpathcurveto{\pgfqpoint{1.568825in}{1.526625in}}{\pgfqpoint{1.558226in}{1.522235in}}{\pgfqpoint{1.550412in}{1.514421in}}%
\pgfpathcurveto{\pgfqpoint{1.542599in}{1.506607in}}{\pgfqpoint{1.538208in}{1.496008in}}{\pgfqpoint{1.538208in}{1.484958in}}%
\pgfpathcurveto{\pgfqpoint{1.538208in}{1.473908in}}{\pgfqpoint{1.542599in}{1.463309in}}{\pgfqpoint{1.550412in}{1.455495in}}%
\pgfpathcurveto{\pgfqpoint{1.558226in}{1.447682in}}{\pgfqpoint{1.568825in}{1.443291in}}{\pgfqpoint{1.579875in}{1.443291in}}%
\pgfpathclose%
\pgfusepath{stroke,fill}%
\end{pgfscope}%
\begin{pgfscope}%
\pgfpathrectangle{\pgfqpoint{0.375000in}{0.330000in}}{\pgfqpoint{2.325000in}{2.310000in}}%
\pgfusepath{clip}%
\pgfsetbuttcap%
\pgfsetroundjoin%
\definecolor{currentfill}{rgb}{0.000000,0.000000,0.000000}%
\pgfsetfillcolor{currentfill}%
\pgfsetlinewidth{1.003750pt}%
\definecolor{currentstroke}{rgb}{0.000000,0.000000,0.000000}%
\pgfsetstrokecolor{currentstroke}%
\pgfsetdash{}{0pt}%
\pgfpathmoveto{\pgfqpoint{1.579875in}{1.443291in}}%
\pgfpathcurveto{\pgfqpoint{1.590925in}{1.443291in}}{\pgfqpoint{1.601524in}{1.447682in}}{\pgfqpoint{1.609338in}{1.455495in}}%
\pgfpathcurveto{\pgfqpoint{1.617151in}{1.463309in}}{\pgfqpoint{1.621542in}{1.473908in}}{\pgfqpoint{1.621542in}{1.484958in}}%
\pgfpathcurveto{\pgfqpoint{1.621542in}{1.496008in}}{\pgfqpoint{1.617151in}{1.506607in}}{\pgfqpoint{1.609338in}{1.514421in}}%
\pgfpathcurveto{\pgfqpoint{1.601524in}{1.522235in}}{\pgfqpoint{1.590925in}{1.526625in}}{\pgfqpoint{1.579875in}{1.526625in}}%
\pgfpathcurveto{\pgfqpoint{1.568825in}{1.526625in}}{\pgfqpoint{1.558226in}{1.522235in}}{\pgfqpoint{1.550412in}{1.514421in}}%
\pgfpathcurveto{\pgfqpoint{1.542599in}{1.506607in}}{\pgfqpoint{1.538208in}{1.496008in}}{\pgfqpoint{1.538208in}{1.484958in}}%
\pgfpathcurveto{\pgfqpoint{1.538208in}{1.473908in}}{\pgfqpoint{1.542599in}{1.463309in}}{\pgfqpoint{1.550412in}{1.455495in}}%
\pgfpathcurveto{\pgfqpoint{1.558226in}{1.447682in}}{\pgfqpoint{1.568825in}{1.443291in}}{\pgfqpoint{1.579875in}{1.443291in}}%
\pgfpathclose%
\pgfusepath{stroke,fill}%
\end{pgfscope}%
\begin{pgfscope}%
\pgfpathrectangle{\pgfqpoint{0.375000in}{0.330000in}}{\pgfqpoint{2.325000in}{2.310000in}}%
\pgfusepath{clip}%
\pgfsetbuttcap%
\pgfsetroundjoin%
\definecolor{currentfill}{rgb}{0.000000,0.000000,0.000000}%
\pgfsetfillcolor{currentfill}%
\pgfsetlinewidth{1.003750pt}%
\definecolor{currentstroke}{rgb}{0.000000,0.000000,0.000000}%
\pgfsetstrokecolor{currentstroke}%
\pgfsetdash{}{0pt}%
\pgfpathmoveto{\pgfqpoint{1.579875in}{1.443291in}}%
\pgfpathcurveto{\pgfqpoint{1.590925in}{1.443291in}}{\pgfqpoint{1.601524in}{1.447682in}}{\pgfqpoint{1.609338in}{1.455495in}}%
\pgfpathcurveto{\pgfqpoint{1.617151in}{1.463309in}}{\pgfqpoint{1.621542in}{1.473908in}}{\pgfqpoint{1.621542in}{1.484958in}}%
\pgfpathcurveto{\pgfqpoint{1.621542in}{1.496008in}}{\pgfqpoint{1.617151in}{1.506607in}}{\pgfqpoint{1.609338in}{1.514421in}}%
\pgfpathcurveto{\pgfqpoint{1.601524in}{1.522235in}}{\pgfqpoint{1.590925in}{1.526625in}}{\pgfqpoint{1.579875in}{1.526625in}}%
\pgfpathcurveto{\pgfqpoint{1.568825in}{1.526625in}}{\pgfqpoint{1.558226in}{1.522235in}}{\pgfqpoint{1.550412in}{1.514421in}}%
\pgfpathcurveto{\pgfqpoint{1.542599in}{1.506607in}}{\pgfqpoint{1.538208in}{1.496008in}}{\pgfqpoint{1.538208in}{1.484958in}}%
\pgfpathcurveto{\pgfqpoint{1.538208in}{1.473908in}}{\pgfqpoint{1.542599in}{1.463309in}}{\pgfqpoint{1.550412in}{1.455495in}}%
\pgfpathcurveto{\pgfqpoint{1.558226in}{1.447682in}}{\pgfqpoint{1.568825in}{1.443291in}}{\pgfqpoint{1.579875in}{1.443291in}}%
\pgfpathclose%
\pgfusepath{stroke,fill}%
\end{pgfscope}%
\begin{pgfscope}%
\pgfpathrectangle{\pgfqpoint{0.375000in}{0.330000in}}{\pgfqpoint{2.325000in}{2.310000in}}%
\pgfusepath{clip}%
\pgfsetbuttcap%
\pgfsetroundjoin%
\definecolor{currentfill}{rgb}{0.000000,0.000000,0.000000}%
\pgfsetfillcolor{currentfill}%
\pgfsetlinewidth{1.003750pt}%
\definecolor{currentstroke}{rgb}{0.000000,0.000000,0.000000}%
\pgfsetstrokecolor{currentstroke}%
\pgfsetdash{}{0pt}%
\pgfpathmoveto{\pgfqpoint{1.579875in}{1.443291in}}%
\pgfpathcurveto{\pgfqpoint{1.590925in}{1.443291in}}{\pgfqpoint{1.601524in}{1.447682in}}{\pgfqpoint{1.609338in}{1.455495in}}%
\pgfpathcurveto{\pgfqpoint{1.617151in}{1.463309in}}{\pgfqpoint{1.621542in}{1.473908in}}{\pgfqpoint{1.621542in}{1.484958in}}%
\pgfpathcurveto{\pgfqpoint{1.621542in}{1.496008in}}{\pgfqpoint{1.617151in}{1.506607in}}{\pgfqpoint{1.609338in}{1.514421in}}%
\pgfpathcurveto{\pgfqpoint{1.601524in}{1.522235in}}{\pgfqpoint{1.590925in}{1.526625in}}{\pgfqpoint{1.579875in}{1.526625in}}%
\pgfpathcurveto{\pgfqpoint{1.568825in}{1.526625in}}{\pgfqpoint{1.558226in}{1.522235in}}{\pgfqpoint{1.550412in}{1.514421in}}%
\pgfpathcurveto{\pgfqpoint{1.542599in}{1.506607in}}{\pgfqpoint{1.538208in}{1.496008in}}{\pgfqpoint{1.538208in}{1.484958in}}%
\pgfpathcurveto{\pgfqpoint{1.538208in}{1.473908in}}{\pgfqpoint{1.542599in}{1.463309in}}{\pgfqpoint{1.550412in}{1.455495in}}%
\pgfpathcurveto{\pgfqpoint{1.558226in}{1.447682in}}{\pgfqpoint{1.568825in}{1.443291in}}{\pgfqpoint{1.579875in}{1.443291in}}%
\pgfpathclose%
\pgfusepath{stroke,fill}%
\end{pgfscope}%
\begin{pgfscope}%
\pgfpathrectangle{\pgfqpoint{0.375000in}{0.330000in}}{\pgfqpoint{2.325000in}{2.310000in}}%
\pgfusepath{clip}%
\pgfsetbuttcap%
\pgfsetroundjoin%
\definecolor{currentfill}{rgb}{0.000000,0.000000,0.000000}%
\pgfsetfillcolor{currentfill}%
\pgfsetlinewidth{1.003750pt}%
\definecolor{currentstroke}{rgb}{0.000000,0.000000,0.000000}%
\pgfsetstrokecolor{currentstroke}%
\pgfsetdash{}{0pt}%
\pgfpathmoveto{\pgfqpoint{1.579875in}{1.443291in}}%
\pgfpathcurveto{\pgfqpoint{1.590925in}{1.443291in}}{\pgfqpoint{1.601524in}{1.447682in}}{\pgfqpoint{1.609338in}{1.455495in}}%
\pgfpathcurveto{\pgfqpoint{1.617151in}{1.463309in}}{\pgfqpoint{1.621542in}{1.473908in}}{\pgfqpoint{1.621542in}{1.484958in}}%
\pgfpathcurveto{\pgfqpoint{1.621542in}{1.496008in}}{\pgfqpoint{1.617151in}{1.506607in}}{\pgfqpoint{1.609338in}{1.514421in}}%
\pgfpathcurveto{\pgfqpoint{1.601524in}{1.522235in}}{\pgfqpoint{1.590925in}{1.526625in}}{\pgfqpoint{1.579875in}{1.526625in}}%
\pgfpathcurveto{\pgfqpoint{1.568825in}{1.526625in}}{\pgfqpoint{1.558226in}{1.522235in}}{\pgfqpoint{1.550412in}{1.514421in}}%
\pgfpathcurveto{\pgfqpoint{1.542599in}{1.506607in}}{\pgfqpoint{1.538208in}{1.496008in}}{\pgfqpoint{1.538208in}{1.484958in}}%
\pgfpathcurveto{\pgfqpoint{1.538208in}{1.473908in}}{\pgfqpoint{1.542599in}{1.463309in}}{\pgfqpoint{1.550412in}{1.455495in}}%
\pgfpathcurveto{\pgfqpoint{1.558226in}{1.447682in}}{\pgfqpoint{1.568825in}{1.443291in}}{\pgfqpoint{1.579875in}{1.443291in}}%
\pgfpathclose%
\pgfusepath{stroke,fill}%
\end{pgfscope}%
\begin{pgfscope}%
\pgfpathrectangle{\pgfqpoint{0.375000in}{0.330000in}}{\pgfqpoint{2.325000in}{2.310000in}}%
\pgfusepath{clip}%
\pgfsetbuttcap%
\pgfsetroundjoin%
\definecolor{currentfill}{rgb}{0.000000,0.000000,0.000000}%
\pgfsetfillcolor{currentfill}%
\pgfsetlinewidth{1.003750pt}%
\definecolor{currentstroke}{rgb}{0.000000,0.000000,0.000000}%
\pgfsetstrokecolor{currentstroke}%
\pgfsetdash{}{0pt}%
\pgfpathmoveto{\pgfqpoint{1.579875in}{1.443291in}}%
\pgfpathcurveto{\pgfqpoint{1.590925in}{1.443291in}}{\pgfqpoint{1.601524in}{1.447682in}}{\pgfqpoint{1.609338in}{1.455495in}}%
\pgfpathcurveto{\pgfqpoint{1.617151in}{1.463309in}}{\pgfqpoint{1.621542in}{1.473908in}}{\pgfqpoint{1.621542in}{1.484958in}}%
\pgfpathcurveto{\pgfqpoint{1.621542in}{1.496008in}}{\pgfqpoint{1.617151in}{1.506607in}}{\pgfqpoint{1.609338in}{1.514421in}}%
\pgfpathcurveto{\pgfqpoint{1.601524in}{1.522235in}}{\pgfqpoint{1.590925in}{1.526625in}}{\pgfqpoint{1.579875in}{1.526625in}}%
\pgfpathcurveto{\pgfqpoint{1.568825in}{1.526625in}}{\pgfqpoint{1.558226in}{1.522235in}}{\pgfqpoint{1.550412in}{1.514421in}}%
\pgfpathcurveto{\pgfqpoint{1.542599in}{1.506607in}}{\pgfqpoint{1.538208in}{1.496008in}}{\pgfqpoint{1.538208in}{1.484958in}}%
\pgfpathcurveto{\pgfqpoint{1.538208in}{1.473908in}}{\pgfqpoint{1.542599in}{1.463309in}}{\pgfqpoint{1.550412in}{1.455495in}}%
\pgfpathcurveto{\pgfqpoint{1.558226in}{1.447682in}}{\pgfqpoint{1.568825in}{1.443291in}}{\pgfqpoint{1.579875in}{1.443291in}}%
\pgfpathclose%
\pgfusepath{stroke,fill}%
\end{pgfscope}%
\begin{pgfscope}%
\pgfpathrectangle{\pgfqpoint{0.375000in}{0.330000in}}{\pgfqpoint{2.325000in}{2.310000in}}%
\pgfusepath{clip}%
\pgfsetbuttcap%
\pgfsetroundjoin%
\definecolor{currentfill}{rgb}{0.000000,0.000000,0.000000}%
\pgfsetfillcolor{currentfill}%
\pgfsetlinewidth{1.003750pt}%
\definecolor{currentstroke}{rgb}{0.000000,0.000000,0.000000}%
\pgfsetstrokecolor{currentstroke}%
\pgfsetdash{}{0pt}%
\pgfpathmoveto{\pgfqpoint{1.579875in}{1.443291in}}%
\pgfpathcurveto{\pgfqpoint{1.590925in}{1.443291in}}{\pgfqpoint{1.601524in}{1.447682in}}{\pgfqpoint{1.609338in}{1.455495in}}%
\pgfpathcurveto{\pgfqpoint{1.617151in}{1.463309in}}{\pgfqpoint{1.621542in}{1.473908in}}{\pgfqpoint{1.621542in}{1.484958in}}%
\pgfpathcurveto{\pgfqpoint{1.621542in}{1.496008in}}{\pgfqpoint{1.617151in}{1.506607in}}{\pgfqpoint{1.609338in}{1.514421in}}%
\pgfpathcurveto{\pgfqpoint{1.601524in}{1.522235in}}{\pgfqpoint{1.590925in}{1.526625in}}{\pgfqpoint{1.579875in}{1.526625in}}%
\pgfpathcurveto{\pgfqpoint{1.568825in}{1.526625in}}{\pgfqpoint{1.558226in}{1.522235in}}{\pgfqpoint{1.550412in}{1.514421in}}%
\pgfpathcurveto{\pgfqpoint{1.542599in}{1.506607in}}{\pgfqpoint{1.538208in}{1.496008in}}{\pgfqpoint{1.538208in}{1.484958in}}%
\pgfpathcurveto{\pgfqpoint{1.538208in}{1.473908in}}{\pgfqpoint{1.542599in}{1.463309in}}{\pgfqpoint{1.550412in}{1.455495in}}%
\pgfpathcurveto{\pgfqpoint{1.558226in}{1.447682in}}{\pgfqpoint{1.568825in}{1.443291in}}{\pgfqpoint{1.579875in}{1.443291in}}%
\pgfpathclose%
\pgfusepath{stroke,fill}%
\end{pgfscope}%
\begin{pgfscope}%
\pgfpathrectangle{\pgfqpoint{0.375000in}{0.330000in}}{\pgfqpoint{2.325000in}{2.310000in}}%
\pgfusepath{clip}%
\pgfsetbuttcap%
\pgfsetroundjoin%
\definecolor{currentfill}{rgb}{0.000000,0.000000,0.000000}%
\pgfsetfillcolor{currentfill}%
\pgfsetlinewidth{1.003750pt}%
\definecolor{currentstroke}{rgb}{0.000000,0.000000,0.000000}%
\pgfsetstrokecolor{currentstroke}%
\pgfsetdash{}{0pt}%
\pgfpathmoveto{\pgfqpoint{1.579875in}{1.443291in}}%
\pgfpathcurveto{\pgfqpoint{1.590925in}{1.443291in}}{\pgfqpoint{1.601524in}{1.447682in}}{\pgfqpoint{1.609338in}{1.455495in}}%
\pgfpathcurveto{\pgfqpoint{1.617151in}{1.463309in}}{\pgfqpoint{1.621542in}{1.473908in}}{\pgfqpoint{1.621542in}{1.484958in}}%
\pgfpathcurveto{\pgfqpoint{1.621542in}{1.496008in}}{\pgfqpoint{1.617151in}{1.506607in}}{\pgfqpoint{1.609338in}{1.514421in}}%
\pgfpathcurveto{\pgfqpoint{1.601524in}{1.522235in}}{\pgfqpoint{1.590925in}{1.526625in}}{\pgfqpoint{1.579875in}{1.526625in}}%
\pgfpathcurveto{\pgfqpoint{1.568825in}{1.526625in}}{\pgfqpoint{1.558226in}{1.522235in}}{\pgfqpoint{1.550412in}{1.514421in}}%
\pgfpathcurveto{\pgfqpoint{1.542599in}{1.506607in}}{\pgfqpoint{1.538208in}{1.496008in}}{\pgfqpoint{1.538208in}{1.484958in}}%
\pgfpathcurveto{\pgfqpoint{1.538208in}{1.473908in}}{\pgfqpoint{1.542599in}{1.463309in}}{\pgfqpoint{1.550412in}{1.455495in}}%
\pgfpathcurveto{\pgfqpoint{1.558226in}{1.447682in}}{\pgfqpoint{1.568825in}{1.443291in}}{\pgfqpoint{1.579875in}{1.443291in}}%
\pgfpathclose%
\pgfusepath{stroke,fill}%
\end{pgfscope}%
\begin{pgfscope}%
\pgfpathrectangle{\pgfqpoint{0.375000in}{0.330000in}}{\pgfqpoint{2.325000in}{2.310000in}}%
\pgfusepath{clip}%
\pgfsetbuttcap%
\pgfsetroundjoin%
\definecolor{currentfill}{rgb}{0.000000,0.000000,0.000000}%
\pgfsetfillcolor{currentfill}%
\pgfsetlinewidth{1.003750pt}%
\definecolor{currentstroke}{rgb}{0.000000,0.000000,0.000000}%
\pgfsetstrokecolor{currentstroke}%
\pgfsetdash{}{0pt}%
\pgfpathmoveto{\pgfqpoint{1.579875in}{1.443291in}}%
\pgfpathcurveto{\pgfqpoint{1.590925in}{1.443291in}}{\pgfqpoint{1.601524in}{1.447682in}}{\pgfqpoint{1.609338in}{1.455495in}}%
\pgfpathcurveto{\pgfqpoint{1.617151in}{1.463309in}}{\pgfqpoint{1.621542in}{1.473908in}}{\pgfqpoint{1.621542in}{1.484958in}}%
\pgfpathcurveto{\pgfqpoint{1.621542in}{1.496008in}}{\pgfqpoint{1.617151in}{1.506607in}}{\pgfqpoint{1.609338in}{1.514421in}}%
\pgfpathcurveto{\pgfqpoint{1.601524in}{1.522235in}}{\pgfqpoint{1.590925in}{1.526625in}}{\pgfqpoint{1.579875in}{1.526625in}}%
\pgfpathcurveto{\pgfqpoint{1.568825in}{1.526625in}}{\pgfqpoint{1.558226in}{1.522235in}}{\pgfqpoint{1.550412in}{1.514421in}}%
\pgfpathcurveto{\pgfqpoint{1.542599in}{1.506607in}}{\pgfqpoint{1.538208in}{1.496008in}}{\pgfqpoint{1.538208in}{1.484958in}}%
\pgfpathcurveto{\pgfqpoint{1.538208in}{1.473908in}}{\pgfqpoint{1.542599in}{1.463309in}}{\pgfqpoint{1.550412in}{1.455495in}}%
\pgfpathcurveto{\pgfqpoint{1.558226in}{1.447682in}}{\pgfqpoint{1.568825in}{1.443291in}}{\pgfqpoint{1.579875in}{1.443291in}}%
\pgfpathclose%
\pgfusepath{stroke,fill}%
\end{pgfscope}%
\begin{pgfscope}%
\pgfpathrectangle{\pgfqpoint{0.375000in}{0.330000in}}{\pgfqpoint{2.325000in}{2.310000in}}%
\pgfusepath{clip}%
\pgfsetbuttcap%
\pgfsetroundjoin%
\definecolor{currentfill}{rgb}{0.000000,0.000000,0.000000}%
\pgfsetfillcolor{currentfill}%
\pgfsetlinewidth{1.003750pt}%
\definecolor{currentstroke}{rgb}{0.000000,0.000000,0.000000}%
\pgfsetstrokecolor{currentstroke}%
\pgfsetdash{}{0pt}%
\pgfpathmoveto{\pgfqpoint{1.579875in}{2.474583in}}%
\pgfpathcurveto{\pgfqpoint{1.590925in}{2.474583in}}{\pgfqpoint{1.601524in}{2.478974in}}{\pgfqpoint{1.609338in}{2.486787in}}%
\pgfpathcurveto{\pgfqpoint{1.617151in}{2.494601in}}{\pgfqpoint{1.621542in}{2.505200in}}{\pgfqpoint{1.621542in}{2.516250in}}%
\pgfpathcurveto{\pgfqpoint{1.621542in}{2.527300in}}{\pgfqpoint{1.617151in}{2.537899in}}{\pgfqpoint{1.609338in}{2.545713in}}%
\pgfpathcurveto{\pgfqpoint{1.601524in}{2.553526in}}{\pgfqpoint{1.590925in}{2.557917in}}{\pgfqpoint{1.579875in}{2.557917in}}%
\pgfpathcurveto{\pgfqpoint{1.568825in}{2.557917in}}{\pgfqpoint{1.558226in}{2.553526in}}{\pgfqpoint{1.550412in}{2.545713in}}%
\pgfpathcurveto{\pgfqpoint{1.542599in}{2.537899in}}{\pgfqpoint{1.538208in}{2.527300in}}{\pgfqpoint{1.538208in}{2.516250in}}%
\pgfpathcurveto{\pgfqpoint{1.538208in}{2.505200in}}{\pgfqpoint{1.542599in}{2.494601in}}{\pgfqpoint{1.550412in}{2.486787in}}%
\pgfpathcurveto{\pgfqpoint{1.558226in}{2.478974in}}{\pgfqpoint{1.568825in}{2.474583in}}{\pgfqpoint{1.579875in}{2.474583in}}%
\pgfpathclose%
\pgfusepath{stroke,fill}%
\end{pgfscope}%
\begin{pgfscope}%
\pgfpathrectangle{\pgfqpoint{0.375000in}{0.330000in}}{\pgfqpoint{2.325000in}{2.310000in}}%
\pgfusepath{clip}%
\pgfsetbuttcap%
\pgfsetroundjoin%
\definecolor{currentfill}{rgb}{0.000000,0.000000,0.000000}%
\pgfsetfillcolor{currentfill}%
\pgfsetlinewidth{1.003750pt}%
\definecolor{currentstroke}{rgb}{0.000000,0.000000,0.000000}%
\pgfsetstrokecolor{currentstroke}%
\pgfsetdash{}{0pt}%
\pgfpathmoveto{\pgfqpoint{1.579875in}{1.443291in}}%
\pgfpathcurveto{\pgfqpoint{1.590925in}{1.443291in}}{\pgfqpoint{1.601524in}{1.447682in}}{\pgfqpoint{1.609338in}{1.455495in}}%
\pgfpathcurveto{\pgfqpoint{1.617151in}{1.463309in}}{\pgfqpoint{1.621542in}{1.473908in}}{\pgfqpoint{1.621542in}{1.484958in}}%
\pgfpathcurveto{\pgfqpoint{1.621542in}{1.496008in}}{\pgfqpoint{1.617151in}{1.506607in}}{\pgfqpoint{1.609338in}{1.514421in}}%
\pgfpathcurveto{\pgfqpoint{1.601524in}{1.522235in}}{\pgfqpoint{1.590925in}{1.526625in}}{\pgfqpoint{1.579875in}{1.526625in}}%
\pgfpathcurveto{\pgfqpoint{1.568825in}{1.526625in}}{\pgfqpoint{1.558226in}{1.522235in}}{\pgfqpoint{1.550412in}{1.514421in}}%
\pgfpathcurveto{\pgfqpoint{1.542599in}{1.506607in}}{\pgfqpoint{1.538208in}{1.496008in}}{\pgfqpoint{1.538208in}{1.484958in}}%
\pgfpathcurveto{\pgfqpoint{1.538208in}{1.473908in}}{\pgfqpoint{1.542599in}{1.463309in}}{\pgfqpoint{1.550412in}{1.455495in}}%
\pgfpathcurveto{\pgfqpoint{1.558226in}{1.447682in}}{\pgfqpoint{1.568825in}{1.443291in}}{\pgfqpoint{1.579875in}{1.443291in}}%
\pgfpathclose%
\pgfusepath{stroke,fill}%
\end{pgfscope}%
\begin{pgfscope}%
\pgfpathrectangle{\pgfqpoint{0.375000in}{0.330000in}}{\pgfqpoint{2.325000in}{2.310000in}}%
\pgfusepath{clip}%
\pgfsetbuttcap%
\pgfsetroundjoin%
\definecolor{currentfill}{rgb}{0.000000,0.000000,0.000000}%
\pgfsetfillcolor{currentfill}%
\pgfsetlinewidth{1.003750pt}%
\definecolor{currentstroke}{rgb}{0.000000,0.000000,0.000000}%
\pgfsetstrokecolor{currentstroke}%
\pgfsetdash{}{0pt}%
\pgfpathmoveto{\pgfqpoint{1.579875in}{1.443291in}}%
\pgfpathcurveto{\pgfqpoint{1.590925in}{1.443291in}}{\pgfqpoint{1.601524in}{1.447682in}}{\pgfqpoint{1.609338in}{1.455495in}}%
\pgfpathcurveto{\pgfqpoint{1.617151in}{1.463309in}}{\pgfqpoint{1.621542in}{1.473908in}}{\pgfqpoint{1.621542in}{1.484958in}}%
\pgfpathcurveto{\pgfqpoint{1.621542in}{1.496008in}}{\pgfqpoint{1.617151in}{1.506607in}}{\pgfqpoint{1.609338in}{1.514421in}}%
\pgfpathcurveto{\pgfqpoint{1.601524in}{1.522235in}}{\pgfqpoint{1.590925in}{1.526625in}}{\pgfqpoint{1.579875in}{1.526625in}}%
\pgfpathcurveto{\pgfqpoint{1.568825in}{1.526625in}}{\pgfqpoint{1.558226in}{1.522235in}}{\pgfqpoint{1.550412in}{1.514421in}}%
\pgfpathcurveto{\pgfqpoint{1.542599in}{1.506607in}}{\pgfqpoint{1.538208in}{1.496008in}}{\pgfqpoint{1.538208in}{1.484958in}}%
\pgfpathcurveto{\pgfqpoint{1.538208in}{1.473908in}}{\pgfqpoint{1.542599in}{1.463309in}}{\pgfqpoint{1.550412in}{1.455495in}}%
\pgfpathcurveto{\pgfqpoint{1.558226in}{1.447682in}}{\pgfqpoint{1.568825in}{1.443291in}}{\pgfqpoint{1.579875in}{1.443291in}}%
\pgfpathclose%
\pgfusepath{stroke,fill}%
\end{pgfscope}%
\begin{pgfscope}%
\pgfpathrectangle{\pgfqpoint{0.375000in}{0.330000in}}{\pgfqpoint{2.325000in}{2.310000in}}%
\pgfusepath{clip}%
\pgfsetbuttcap%
\pgfsetroundjoin%
\definecolor{currentfill}{rgb}{0.000000,0.000000,0.000000}%
\pgfsetfillcolor{currentfill}%
\pgfsetlinewidth{1.003750pt}%
\definecolor{currentstroke}{rgb}{0.000000,0.000000,0.000000}%
\pgfsetstrokecolor{currentstroke}%
\pgfsetdash{}{0pt}%
\pgfpathmoveto{\pgfqpoint{1.579875in}{2.474583in}}%
\pgfpathcurveto{\pgfqpoint{1.590925in}{2.474583in}}{\pgfqpoint{1.601524in}{2.478974in}}{\pgfqpoint{1.609338in}{2.486787in}}%
\pgfpathcurveto{\pgfqpoint{1.617151in}{2.494601in}}{\pgfqpoint{1.621542in}{2.505200in}}{\pgfqpoint{1.621542in}{2.516250in}}%
\pgfpathcurveto{\pgfqpoint{1.621542in}{2.527300in}}{\pgfqpoint{1.617151in}{2.537899in}}{\pgfqpoint{1.609338in}{2.545713in}}%
\pgfpathcurveto{\pgfqpoint{1.601524in}{2.553526in}}{\pgfqpoint{1.590925in}{2.557917in}}{\pgfqpoint{1.579875in}{2.557917in}}%
\pgfpathcurveto{\pgfqpoint{1.568825in}{2.557917in}}{\pgfqpoint{1.558226in}{2.553526in}}{\pgfqpoint{1.550412in}{2.545713in}}%
\pgfpathcurveto{\pgfqpoint{1.542599in}{2.537899in}}{\pgfqpoint{1.538208in}{2.527300in}}{\pgfqpoint{1.538208in}{2.516250in}}%
\pgfpathcurveto{\pgfqpoint{1.538208in}{2.505200in}}{\pgfqpoint{1.542599in}{2.494601in}}{\pgfqpoint{1.550412in}{2.486787in}}%
\pgfpathcurveto{\pgfqpoint{1.558226in}{2.478974in}}{\pgfqpoint{1.568825in}{2.474583in}}{\pgfqpoint{1.579875in}{2.474583in}}%
\pgfpathclose%
\pgfusepath{stroke,fill}%
\end{pgfscope}%
\begin{pgfscope}%
\pgfpathrectangle{\pgfqpoint{0.375000in}{0.330000in}}{\pgfqpoint{2.325000in}{2.310000in}}%
\pgfusepath{clip}%
\pgfsetbuttcap%
\pgfsetroundjoin%
\definecolor{currentfill}{rgb}{0.000000,0.000000,0.000000}%
\pgfsetfillcolor{currentfill}%
\pgfsetlinewidth{1.003750pt}%
\definecolor{currentstroke}{rgb}{0.000000,0.000000,0.000000}%
\pgfsetstrokecolor{currentstroke}%
\pgfsetdash{}{0pt}%
\pgfpathmoveto{\pgfqpoint{1.579875in}{1.443291in}}%
\pgfpathcurveto{\pgfqpoint{1.590925in}{1.443291in}}{\pgfqpoint{1.601524in}{1.447682in}}{\pgfqpoint{1.609338in}{1.455495in}}%
\pgfpathcurveto{\pgfqpoint{1.617151in}{1.463309in}}{\pgfqpoint{1.621542in}{1.473908in}}{\pgfqpoint{1.621542in}{1.484958in}}%
\pgfpathcurveto{\pgfqpoint{1.621542in}{1.496008in}}{\pgfqpoint{1.617151in}{1.506607in}}{\pgfqpoint{1.609338in}{1.514421in}}%
\pgfpathcurveto{\pgfqpoint{1.601524in}{1.522235in}}{\pgfqpoint{1.590925in}{1.526625in}}{\pgfqpoint{1.579875in}{1.526625in}}%
\pgfpathcurveto{\pgfqpoint{1.568825in}{1.526625in}}{\pgfqpoint{1.558226in}{1.522235in}}{\pgfqpoint{1.550412in}{1.514421in}}%
\pgfpathcurveto{\pgfqpoint{1.542599in}{1.506607in}}{\pgfqpoint{1.538208in}{1.496008in}}{\pgfqpoint{1.538208in}{1.484958in}}%
\pgfpathcurveto{\pgfqpoint{1.538208in}{1.473908in}}{\pgfqpoint{1.542599in}{1.463309in}}{\pgfqpoint{1.550412in}{1.455495in}}%
\pgfpathcurveto{\pgfqpoint{1.558226in}{1.447682in}}{\pgfqpoint{1.568825in}{1.443291in}}{\pgfqpoint{1.579875in}{1.443291in}}%
\pgfpathclose%
\pgfusepath{stroke,fill}%
\end{pgfscope}%
\begin{pgfscope}%
\pgfpathrectangle{\pgfqpoint{0.375000in}{0.330000in}}{\pgfqpoint{2.325000in}{2.310000in}}%
\pgfusepath{clip}%
\pgfsetbuttcap%
\pgfsetroundjoin%
\definecolor{currentfill}{rgb}{0.000000,0.000000,0.000000}%
\pgfsetfillcolor{currentfill}%
\pgfsetlinewidth{1.003750pt}%
\definecolor{currentstroke}{rgb}{0.000000,0.000000,0.000000}%
\pgfsetstrokecolor{currentstroke}%
\pgfsetdash{}{0pt}%
\pgfpathmoveto{\pgfqpoint{1.579875in}{1.443291in}}%
\pgfpathcurveto{\pgfqpoint{1.590925in}{1.443291in}}{\pgfqpoint{1.601524in}{1.447682in}}{\pgfqpoint{1.609338in}{1.455495in}}%
\pgfpathcurveto{\pgfqpoint{1.617151in}{1.463309in}}{\pgfqpoint{1.621542in}{1.473908in}}{\pgfqpoint{1.621542in}{1.484958in}}%
\pgfpathcurveto{\pgfqpoint{1.621542in}{1.496008in}}{\pgfqpoint{1.617151in}{1.506607in}}{\pgfqpoint{1.609338in}{1.514421in}}%
\pgfpathcurveto{\pgfqpoint{1.601524in}{1.522235in}}{\pgfqpoint{1.590925in}{1.526625in}}{\pgfqpoint{1.579875in}{1.526625in}}%
\pgfpathcurveto{\pgfqpoint{1.568825in}{1.526625in}}{\pgfqpoint{1.558226in}{1.522235in}}{\pgfqpoint{1.550412in}{1.514421in}}%
\pgfpathcurveto{\pgfqpoint{1.542599in}{1.506607in}}{\pgfqpoint{1.538208in}{1.496008in}}{\pgfqpoint{1.538208in}{1.484958in}}%
\pgfpathcurveto{\pgfqpoint{1.538208in}{1.473908in}}{\pgfqpoint{1.542599in}{1.463309in}}{\pgfqpoint{1.550412in}{1.455495in}}%
\pgfpathcurveto{\pgfqpoint{1.558226in}{1.447682in}}{\pgfqpoint{1.568825in}{1.443291in}}{\pgfqpoint{1.579875in}{1.443291in}}%
\pgfpathclose%
\pgfusepath{stroke,fill}%
\end{pgfscope}%
\begin{pgfscope}%
\pgfpathrectangle{\pgfqpoint{0.375000in}{0.330000in}}{\pgfqpoint{2.325000in}{2.310000in}}%
\pgfusepath{clip}%
\pgfsetbuttcap%
\pgfsetroundjoin%
\definecolor{currentfill}{rgb}{0.000000,0.000000,0.000000}%
\pgfsetfillcolor{currentfill}%
\pgfsetlinewidth{1.003750pt}%
\definecolor{currentstroke}{rgb}{0.000000,0.000000,0.000000}%
\pgfsetstrokecolor{currentstroke}%
\pgfsetdash{}{0pt}%
\pgfpathmoveto{\pgfqpoint{1.579875in}{1.443291in}}%
\pgfpathcurveto{\pgfqpoint{1.590925in}{1.443291in}}{\pgfqpoint{1.601524in}{1.447682in}}{\pgfqpoint{1.609338in}{1.455495in}}%
\pgfpathcurveto{\pgfqpoint{1.617151in}{1.463309in}}{\pgfqpoint{1.621542in}{1.473908in}}{\pgfqpoint{1.621542in}{1.484958in}}%
\pgfpathcurveto{\pgfqpoint{1.621542in}{1.496008in}}{\pgfqpoint{1.617151in}{1.506607in}}{\pgfqpoint{1.609338in}{1.514421in}}%
\pgfpathcurveto{\pgfqpoint{1.601524in}{1.522235in}}{\pgfqpoint{1.590925in}{1.526625in}}{\pgfqpoint{1.579875in}{1.526625in}}%
\pgfpathcurveto{\pgfqpoint{1.568825in}{1.526625in}}{\pgfqpoint{1.558226in}{1.522235in}}{\pgfqpoint{1.550412in}{1.514421in}}%
\pgfpathcurveto{\pgfqpoint{1.542599in}{1.506607in}}{\pgfqpoint{1.538208in}{1.496008in}}{\pgfqpoint{1.538208in}{1.484958in}}%
\pgfpathcurveto{\pgfqpoint{1.538208in}{1.473908in}}{\pgfqpoint{1.542599in}{1.463309in}}{\pgfqpoint{1.550412in}{1.455495in}}%
\pgfpathcurveto{\pgfqpoint{1.558226in}{1.447682in}}{\pgfqpoint{1.568825in}{1.443291in}}{\pgfqpoint{1.579875in}{1.443291in}}%
\pgfpathclose%
\pgfusepath{stroke,fill}%
\end{pgfscope}%
\begin{pgfscope}%
\pgfpathrectangle{\pgfqpoint{0.375000in}{0.330000in}}{\pgfqpoint{2.325000in}{2.310000in}}%
\pgfusepath{clip}%
\pgfsetbuttcap%
\pgfsetroundjoin%
\definecolor{currentfill}{rgb}{0.000000,0.000000,0.000000}%
\pgfsetfillcolor{currentfill}%
\pgfsetlinewidth{1.003750pt}%
\definecolor{currentstroke}{rgb}{0.000000,0.000000,0.000000}%
\pgfsetstrokecolor{currentstroke}%
\pgfsetdash{}{0pt}%
\pgfpathmoveto{\pgfqpoint{1.579875in}{1.443291in}}%
\pgfpathcurveto{\pgfqpoint{1.590925in}{1.443291in}}{\pgfqpoint{1.601524in}{1.447682in}}{\pgfqpoint{1.609338in}{1.455495in}}%
\pgfpathcurveto{\pgfqpoint{1.617151in}{1.463309in}}{\pgfqpoint{1.621542in}{1.473908in}}{\pgfqpoint{1.621542in}{1.484958in}}%
\pgfpathcurveto{\pgfqpoint{1.621542in}{1.496008in}}{\pgfqpoint{1.617151in}{1.506607in}}{\pgfqpoint{1.609338in}{1.514421in}}%
\pgfpathcurveto{\pgfqpoint{1.601524in}{1.522235in}}{\pgfqpoint{1.590925in}{1.526625in}}{\pgfqpoint{1.579875in}{1.526625in}}%
\pgfpathcurveto{\pgfqpoint{1.568825in}{1.526625in}}{\pgfqpoint{1.558226in}{1.522235in}}{\pgfqpoint{1.550412in}{1.514421in}}%
\pgfpathcurveto{\pgfqpoint{1.542599in}{1.506607in}}{\pgfqpoint{1.538208in}{1.496008in}}{\pgfqpoint{1.538208in}{1.484958in}}%
\pgfpathcurveto{\pgfqpoint{1.538208in}{1.473908in}}{\pgfqpoint{1.542599in}{1.463309in}}{\pgfqpoint{1.550412in}{1.455495in}}%
\pgfpathcurveto{\pgfqpoint{1.558226in}{1.447682in}}{\pgfqpoint{1.568825in}{1.443291in}}{\pgfqpoint{1.579875in}{1.443291in}}%
\pgfpathclose%
\pgfusepath{stroke,fill}%
\end{pgfscope}%
\begin{pgfscope}%
\pgfpathrectangle{\pgfqpoint{0.375000in}{0.330000in}}{\pgfqpoint{2.325000in}{2.310000in}}%
\pgfusepath{clip}%
\pgfsetbuttcap%
\pgfsetroundjoin%
\definecolor{currentfill}{rgb}{0.000000,0.000000,0.000000}%
\pgfsetfillcolor{currentfill}%
\pgfsetlinewidth{1.003750pt}%
\definecolor{currentstroke}{rgb}{0.000000,0.000000,0.000000}%
\pgfsetstrokecolor{currentstroke}%
\pgfsetdash{}{0pt}%
\pgfpathmoveto{\pgfqpoint{1.579875in}{1.443291in}}%
\pgfpathcurveto{\pgfqpoint{1.590925in}{1.443291in}}{\pgfqpoint{1.601524in}{1.447682in}}{\pgfqpoint{1.609338in}{1.455495in}}%
\pgfpathcurveto{\pgfqpoint{1.617151in}{1.463309in}}{\pgfqpoint{1.621542in}{1.473908in}}{\pgfqpoint{1.621542in}{1.484958in}}%
\pgfpathcurveto{\pgfqpoint{1.621542in}{1.496008in}}{\pgfqpoint{1.617151in}{1.506607in}}{\pgfqpoint{1.609338in}{1.514421in}}%
\pgfpathcurveto{\pgfqpoint{1.601524in}{1.522235in}}{\pgfqpoint{1.590925in}{1.526625in}}{\pgfqpoint{1.579875in}{1.526625in}}%
\pgfpathcurveto{\pgfqpoint{1.568825in}{1.526625in}}{\pgfqpoint{1.558226in}{1.522235in}}{\pgfqpoint{1.550412in}{1.514421in}}%
\pgfpathcurveto{\pgfqpoint{1.542599in}{1.506607in}}{\pgfqpoint{1.538208in}{1.496008in}}{\pgfqpoint{1.538208in}{1.484958in}}%
\pgfpathcurveto{\pgfqpoint{1.538208in}{1.473908in}}{\pgfqpoint{1.542599in}{1.463309in}}{\pgfqpoint{1.550412in}{1.455495in}}%
\pgfpathcurveto{\pgfqpoint{1.558226in}{1.447682in}}{\pgfqpoint{1.568825in}{1.443291in}}{\pgfqpoint{1.579875in}{1.443291in}}%
\pgfpathclose%
\pgfusepath{stroke,fill}%
\end{pgfscope}%
\begin{pgfscope}%
\pgfpathrectangle{\pgfqpoint{0.375000in}{0.330000in}}{\pgfqpoint{2.325000in}{2.310000in}}%
\pgfusepath{clip}%
\pgfsetbuttcap%
\pgfsetroundjoin%
\definecolor{currentfill}{rgb}{0.000000,0.000000,0.000000}%
\pgfsetfillcolor{currentfill}%
\pgfsetlinewidth{1.003750pt}%
\definecolor{currentstroke}{rgb}{0.000000,0.000000,0.000000}%
\pgfsetstrokecolor{currentstroke}%
\pgfsetdash{}{0pt}%
\pgfpathmoveto{\pgfqpoint{1.579875in}{1.443291in}}%
\pgfpathcurveto{\pgfqpoint{1.590925in}{1.443291in}}{\pgfqpoint{1.601524in}{1.447682in}}{\pgfqpoint{1.609338in}{1.455495in}}%
\pgfpathcurveto{\pgfqpoint{1.617151in}{1.463309in}}{\pgfqpoint{1.621542in}{1.473908in}}{\pgfqpoint{1.621542in}{1.484958in}}%
\pgfpathcurveto{\pgfqpoint{1.621542in}{1.496008in}}{\pgfqpoint{1.617151in}{1.506607in}}{\pgfqpoint{1.609338in}{1.514421in}}%
\pgfpathcurveto{\pgfqpoint{1.601524in}{1.522235in}}{\pgfqpoint{1.590925in}{1.526625in}}{\pgfqpoint{1.579875in}{1.526625in}}%
\pgfpathcurveto{\pgfqpoint{1.568825in}{1.526625in}}{\pgfqpoint{1.558226in}{1.522235in}}{\pgfqpoint{1.550412in}{1.514421in}}%
\pgfpathcurveto{\pgfqpoint{1.542599in}{1.506607in}}{\pgfqpoint{1.538208in}{1.496008in}}{\pgfqpoint{1.538208in}{1.484958in}}%
\pgfpathcurveto{\pgfqpoint{1.538208in}{1.473908in}}{\pgfqpoint{1.542599in}{1.463309in}}{\pgfqpoint{1.550412in}{1.455495in}}%
\pgfpathcurveto{\pgfqpoint{1.558226in}{1.447682in}}{\pgfqpoint{1.568825in}{1.443291in}}{\pgfqpoint{1.579875in}{1.443291in}}%
\pgfpathclose%
\pgfusepath{stroke,fill}%
\end{pgfscope}%
\begin{pgfscope}%
\pgfpathrectangle{\pgfqpoint{0.375000in}{0.330000in}}{\pgfqpoint{2.325000in}{2.310000in}}%
\pgfusepath{clip}%
\pgfsetbuttcap%
\pgfsetroundjoin%
\definecolor{currentfill}{rgb}{0.000000,0.000000,0.000000}%
\pgfsetfillcolor{currentfill}%
\pgfsetlinewidth{1.003750pt}%
\definecolor{currentstroke}{rgb}{0.000000,0.000000,0.000000}%
\pgfsetstrokecolor{currentstroke}%
\pgfsetdash{}{0pt}%
\pgfpathmoveto{\pgfqpoint{1.579875in}{1.443291in}}%
\pgfpathcurveto{\pgfqpoint{1.590925in}{1.443291in}}{\pgfqpoint{1.601524in}{1.447682in}}{\pgfqpoint{1.609338in}{1.455495in}}%
\pgfpathcurveto{\pgfqpoint{1.617151in}{1.463309in}}{\pgfqpoint{1.621542in}{1.473908in}}{\pgfqpoint{1.621542in}{1.484958in}}%
\pgfpathcurveto{\pgfqpoint{1.621542in}{1.496008in}}{\pgfqpoint{1.617151in}{1.506607in}}{\pgfqpoint{1.609338in}{1.514421in}}%
\pgfpathcurveto{\pgfqpoint{1.601524in}{1.522235in}}{\pgfqpoint{1.590925in}{1.526625in}}{\pgfqpoint{1.579875in}{1.526625in}}%
\pgfpathcurveto{\pgfqpoint{1.568825in}{1.526625in}}{\pgfqpoint{1.558226in}{1.522235in}}{\pgfqpoint{1.550412in}{1.514421in}}%
\pgfpathcurveto{\pgfqpoint{1.542599in}{1.506607in}}{\pgfqpoint{1.538208in}{1.496008in}}{\pgfqpoint{1.538208in}{1.484958in}}%
\pgfpathcurveto{\pgfqpoint{1.538208in}{1.473908in}}{\pgfqpoint{1.542599in}{1.463309in}}{\pgfqpoint{1.550412in}{1.455495in}}%
\pgfpathcurveto{\pgfqpoint{1.558226in}{1.447682in}}{\pgfqpoint{1.568825in}{1.443291in}}{\pgfqpoint{1.579875in}{1.443291in}}%
\pgfpathclose%
\pgfusepath{stroke,fill}%
\end{pgfscope}%
\begin{pgfscope}%
\pgfpathrectangle{\pgfqpoint{0.375000in}{0.330000in}}{\pgfqpoint{2.325000in}{2.310000in}}%
\pgfusepath{clip}%
\pgfsetbuttcap%
\pgfsetroundjoin%
\definecolor{currentfill}{rgb}{0.000000,0.000000,0.000000}%
\pgfsetfillcolor{currentfill}%
\pgfsetlinewidth{1.003750pt}%
\definecolor{currentstroke}{rgb}{0.000000,0.000000,0.000000}%
\pgfsetstrokecolor{currentstroke}%
\pgfsetdash{}{0pt}%
\pgfpathmoveto{\pgfqpoint{1.579875in}{1.443291in}}%
\pgfpathcurveto{\pgfqpoint{1.590925in}{1.443291in}}{\pgfqpoint{1.601524in}{1.447682in}}{\pgfqpoint{1.609338in}{1.455495in}}%
\pgfpathcurveto{\pgfqpoint{1.617151in}{1.463309in}}{\pgfqpoint{1.621542in}{1.473908in}}{\pgfqpoint{1.621542in}{1.484958in}}%
\pgfpathcurveto{\pgfqpoint{1.621542in}{1.496008in}}{\pgfqpoint{1.617151in}{1.506607in}}{\pgfqpoint{1.609338in}{1.514421in}}%
\pgfpathcurveto{\pgfqpoint{1.601524in}{1.522235in}}{\pgfqpoint{1.590925in}{1.526625in}}{\pgfqpoint{1.579875in}{1.526625in}}%
\pgfpathcurveto{\pgfqpoint{1.568825in}{1.526625in}}{\pgfqpoint{1.558226in}{1.522235in}}{\pgfqpoint{1.550412in}{1.514421in}}%
\pgfpathcurveto{\pgfqpoint{1.542599in}{1.506607in}}{\pgfqpoint{1.538208in}{1.496008in}}{\pgfqpoint{1.538208in}{1.484958in}}%
\pgfpathcurveto{\pgfqpoint{1.538208in}{1.473908in}}{\pgfqpoint{1.542599in}{1.463309in}}{\pgfqpoint{1.550412in}{1.455495in}}%
\pgfpathcurveto{\pgfqpoint{1.558226in}{1.447682in}}{\pgfqpoint{1.568825in}{1.443291in}}{\pgfqpoint{1.579875in}{1.443291in}}%
\pgfpathclose%
\pgfusepath{stroke,fill}%
\end{pgfscope}%
\begin{pgfscope}%
\pgfpathrectangle{\pgfqpoint{0.375000in}{0.330000in}}{\pgfqpoint{2.325000in}{2.310000in}}%
\pgfusepath{clip}%
\pgfsetbuttcap%
\pgfsetroundjoin%
\definecolor{currentfill}{rgb}{0.000000,0.000000,0.000000}%
\pgfsetfillcolor{currentfill}%
\pgfsetlinewidth{1.003750pt}%
\definecolor{currentstroke}{rgb}{0.000000,0.000000,0.000000}%
\pgfsetstrokecolor{currentstroke}%
\pgfsetdash{}{0pt}%
\pgfpathmoveto{\pgfqpoint{1.579875in}{1.443291in}}%
\pgfpathcurveto{\pgfqpoint{1.590925in}{1.443291in}}{\pgfqpoint{1.601524in}{1.447682in}}{\pgfqpoint{1.609338in}{1.455495in}}%
\pgfpathcurveto{\pgfqpoint{1.617151in}{1.463309in}}{\pgfqpoint{1.621542in}{1.473908in}}{\pgfqpoint{1.621542in}{1.484958in}}%
\pgfpathcurveto{\pgfqpoint{1.621542in}{1.496008in}}{\pgfqpoint{1.617151in}{1.506607in}}{\pgfqpoint{1.609338in}{1.514421in}}%
\pgfpathcurveto{\pgfqpoint{1.601524in}{1.522235in}}{\pgfqpoint{1.590925in}{1.526625in}}{\pgfqpoint{1.579875in}{1.526625in}}%
\pgfpathcurveto{\pgfqpoint{1.568825in}{1.526625in}}{\pgfqpoint{1.558226in}{1.522235in}}{\pgfqpoint{1.550412in}{1.514421in}}%
\pgfpathcurveto{\pgfqpoint{1.542599in}{1.506607in}}{\pgfqpoint{1.538208in}{1.496008in}}{\pgfqpoint{1.538208in}{1.484958in}}%
\pgfpathcurveto{\pgfqpoint{1.538208in}{1.473908in}}{\pgfqpoint{1.542599in}{1.463309in}}{\pgfqpoint{1.550412in}{1.455495in}}%
\pgfpathcurveto{\pgfqpoint{1.558226in}{1.447682in}}{\pgfqpoint{1.568825in}{1.443291in}}{\pgfqpoint{1.579875in}{1.443291in}}%
\pgfpathclose%
\pgfusepath{stroke,fill}%
\end{pgfscope}%
\begin{pgfscope}%
\pgfpathrectangle{\pgfqpoint{0.375000in}{0.330000in}}{\pgfqpoint{2.325000in}{2.310000in}}%
\pgfusepath{clip}%
\pgfsetbuttcap%
\pgfsetroundjoin%
\definecolor{currentfill}{rgb}{0.000000,0.000000,0.000000}%
\pgfsetfillcolor{currentfill}%
\pgfsetlinewidth{1.003750pt}%
\definecolor{currentstroke}{rgb}{0.000000,0.000000,0.000000}%
\pgfsetstrokecolor{currentstroke}%
\pgfsetdash{}{0pt}%
\pgfpathmoveto{\pgfqpoint{1.579875in}{1.443291in}}%
\pgfpathcurveto{\pgfqpoint{1.590925in}{1.443291in}}{\pgfqpoint{1.601524in}{1.447682in}}{\pgfqpoint{1.609338in}{1.455495in}}%
\pgfpathcurveto{\pgfqpoint{1.617151in}{1.463309in}}{\pgfqpoint{1.621542in}{1.473908in}}{\pgfqpoint{1.621542in}{1.484958in}}%
\pgfpathcurveto{\pgfqpoint{1.621542in}{1.496008in}}{\pgfqpoint{1.617151in}{1.506607in}}{\pgfqpoint{1.609338in}{1.514421in}}%
\pgfpathcurveto{\pgfqpoint{1.601524in}{1.522235in}}{\pgfqpoint{1.590925in}{1.526625in}}{\pgfqpoint{1.579875in}{1.526625in}}%
\pgfpathcurveto{\pgfqpoint{1.568825in}{1.526625in}}{\pgfqpoint{1.558226in}{1.522235in}}{\pgfqpoint{1.550412in}{1.514421in}}%
\pgfpathcurveto{\pgfqpoint{1.542599in}{1.506607in}}{\pgfqpoint{1.538208in}{1.496008in}}{\pgfqpoint{1.538208in}{1.484958in}}%
\pgfpathcurveto{\pgfqpoint{1.538208in}{1.473908in}}{\pgfqpoint{1.542599in}{1.463309in}}{\pgfqpoint{1.550412in}{1.455495in}}%
\pgfpathcurveto{\pgfqpoint{1.558226in}{1.447682in}}{\pgfqpoint{1.568825in}{1.443291in}}{\pgfqpoint{1.579875in}{1.443291in}}%
\pgfpathclose%
\pgfusepath{stroke,fill}%
\end{pgfscope}%
\begin{pgfscope}%
\pgfpathrectangle{\pgfqpoint{0.375000in}{0.330000in}}{\pgfqpoint{2.325000in}{2.310000in}}%
\pgfusepath{clip}%
\pgfsetbuttcap%
\pgfsetroundjoin%
\definecolor{currentfill}{rgb}{0.000000,0.000000,0.000000}%
\pgfsetfillcolor{currentfill}%
\pgfsetlinewidth{1.003750pt}%
\definecolor{currentstroke}{rgb}{0.000000,0.000000,0.000000}%
\pgfsetstrokecolor{currentstroke}%
\pgfsetdash{}{0pt}%
\pgfpathmoveto{\pgfqpoint{1.579875in}{1.443291in}}%
\pgfpathcurveto{\pgfqpoint{1.590925in}{1.443291in}}{\pgfqpoint{1.601524in}{1.447682in}}{\pgfqpoint{1.609338in}{1.455495in}}%
\pgfpathcurveto{\pgfqpoint{1.617151in}{1.463309in}}{\pgfqpoint{1.621542in}{1.473908in}}{\pgfqpoint{1.621542in}{1.484958in}}%
\pgfpathcurveto{\pgfqpoint{1.621542in}{1.496008in}}{\pgfqpoint{1.617151in}{1.506607in}}{\pgfqpoint{1.609338in}{1.514421in}}%
\pgfpathcurveto{\pgfqpoint{1.601524in}{1.522235in}}{\pgfqpoint{1.590925in}{1.526625in}}{\pgfqpoint{1.579875in}{1.526625in}}%
\pgfpathcurveto{\pgfqpoint{1.568825in}{1.526625in}}{\pgfqpoint{1.558226in}{1.522235in}}{\pgfqpoint{1.550412in}{1.514421in}}%
\pgfpathcurveto{\pgfqpoint{1.542599in}{1.506607in}}{\pgfqpoint{1.538208in}{1.496008in}}{\pgfqpoint{1.538208in}{1.484958in}}%
\pgfpathcurveto{\pgfqpoint{1.538208in}{1.473908in}}{\pgfqpoint{1.542599in}{1.463309in}}{\pgfqpoint{1.550412in}{1.455495in}}%
\pgfpathcurveto{\pgfqpoint{1.558226in}{1.447682in}}{\pgfqpoint{1.568825in}{1.443291in}}{\pgfqpoint{1.579875in}{1.443291in}}%
\pgfpathclose%
\pgfusepath{stroke,fill}%
\end{pgfscope}%
\begin{pgfscope}%
\pgfpathrectangle{\pgfqpoint{0.375000in}{0.330000in}}{\pgfqpoint{2.325000in}{2.310000in}}%
\pgfusepath{clip}%
\pgfsetbuttcap%
\pgfsetroundjoin%
\definecolor{currentfill}{rgb}{0.000000,0.000000,0.000000}%
\pgfsetfillcolor{currentfill}%
\pgfsetlinewidth{1.003750pt}%
\definecolor{currentstroke}{rgb}{0.000000,0.000000,0.000000}%
\pgfsetstrokecolor{currentstroke}%
\pgfsetdash{}{0pt}%
\pgfpathmoveto{\pgfqpoint{1.579875in}{1.443291in}}%
\pgfpathcurveto{\pgfqpoint{1.590925in}{1.443291in}}{\pgfqpoint{1.601524in}{1.447682in}}{\pgfqpoint{1.609338in}{1.455495in}}%
\pgfpathcurveto{\pgfqpoint{1.617151in}{1.463309in}}{\pgfqpoint{1.621542in}{1.473908in}}{\pgfqpoint{1.621542in}{1.484958in}}%
\pgfpathcurveto{\pgfqpoint{1.621542in}{1.496008in}}{\pgfqpoint{1.617151in}{1.506607in}}{\pgfqpoint{1.609338in}{1.514421in}}%
\pgfpathcurveto{\pgfqpoint{1.601524in}{1.522235in}}{\pgfqpoint{1.590925in}{1.526625in}}{\pgfqpoint{1.579875in}{1.526625in}}%
\pgfpathcurveto{\pgfqpoint{1.568825in}{1.526625in}}{\pgfqpoint{1.558226in}{1.522235in}}{\pgfqpoint{1.550412in}{1.514421in}}%
\pgfpathcurveto{\pgfqpoint{1.542599in}{1.506607in}}{\pgfqpoint{1.538208in}{1.496008in}}{\pgfqpoint{1.538208in}{1.484958in}}%
\pgfpathcurveto{\pgfqpoint{1.538208in}{1.473908in}}{\pgfqpoint{1.542599in}{1.463309in}}{\pgfqpoint{1.550412in}{1.455495in}}%
\pgfpathcurveto{\pgfqpoint{1.558226in}{1.447682in}}{\pgfqpoint{1.568825in}{1.443291in}}{\pgfqpoint{1.579875in}{1.443291in}}%
\pgfpathclose%
\pgfusepath{stroke,fill}%
\end{pgfscope}%
\begin{pgfscope}%
\pgfpathrectangle{\pgfqpoint{0.375000in}{0.330000in}}{\pgfqpoint{2.325000in}{2.310000in}}%
\pgfusepath{clip}%
\pgfsetbuttcap%
\pgfsetroundjoin%
\definecolor{currentfill}{rgb}{0.000000,0.000000,0.000000}%
\pgfsetfillcolor{currentfill}%
\pgfsetlinewidth{1.003750pt}%
\definecolor{currentstroke}{rgb}{0.000000,0.000000,0.000000}%
\pgfsetstrokecolor{currentstroke}%
\pgfsetdash{}{0pt}%
\pgfpathmoveto{\pgfqpoint{1.579875in}{2.474583in}}%
\pgfpathcurveto{\pgfqpoint{1.590925in}{2.474583in}}{\pgfqpoint{1.601524in}{2.478974in}}{\pgfqpoint{1.609338in}{2.486787in}}%
\pgfpathcurveto{\pgfqpoint{1.617151in}{2.494601in}}{\pgfqpoint{1.621542in}{2.505200in}}{\pgfqpoint{1.621542in}{2.516250in}}%
\pgfpathcurveto{\pgfqpoint{1.621542in}{2.527300in}}{\pgfqpoint{1.617151in}{2.537899in}}{\pgfqpoint{1.609338in}{2.545713in}}%
\pgfpathcurveto{\pgfqpoint{1.601524in}{2.553526in}}{\pgfqpoint{1.590925in}{2.557917in}}{\pgfqpoint{1.579875in}{2.557917in}}%
\pgfpathcurveto{\pgfqpoint{1.568825in}{2.557917in}}{\pgfqpoint{1.558226in}{2.553526in}}{\pgfqpoint{1.550412in}{2.545713in}}%
\pgfpathcurveto{\pgfqpoint{1.542599in}{2.537899in}}{\pgfqpoint{1.538208in}{2.527300in}}{\pgfqpoint{1.538208in}{2.516250in}}%
\pgfpathcurveto{\pgfqpoint{1.538208in}{2.505200in}}{\pgfqpoint{1.542599in}{2.494601in}}{\pgfqpoint{1.550412in}{2.486787in}}%
\pgfpathcurveto{\pgfqpoint{1.558226in}{2.478974in}}{\pgfqpoint{1.568825in}{2.474583in}}{\pgfqpoint{1.579875in}{2.474583in}}%
\pgfpathclose%
\pgfusepath{stroke,fill}%
\end{pgfscope}%
\begin{pgfscope}%
\pgfpathrectangle{\pgfqpoint{0.375000in}{0.330000in}}{\pgfqpoint{2.325000in}{2.310000in}}%
\pgfusepath{clip}%
\pgfsetbuttcap%
\pgfsetroundjoin%
\definecolor{currentfill}{rgb}{0.000000,0.000000,0.000000}%
\pgfsetfillcolor{currentfill}%
\pgfsetlinewidth{1.003750pt}%
\definecolor{currentstroke}{rgb}{0.000000,0.000000,0.000000}%
\pgfsetstrokecolor{currentstroke}%
\pgfsetdash{}{0pt}%
\pgfpathmoveto{\pgfqpoint{1.579875in}{1.443291in}}%
\pgfpathcurveto{\pgfqpoint{1.590925in}{1.443291in}}{\pgfqpoint{1.601524in}{1.447682in}}{\pgfqpoint{1.609338in}{1.455495in}}%
\pgfpathcurveto{\pgfqpoint{1.617151in}{1.463309in}}{\pgfqpoint{1.621542in}{1.473908in}}{\pgfqpoint{1.621542in}{1.484958in}}%
\pgfpathcurveto{\pgfqpoint{1.621542in}{1.496008in}}{\pgfqpoint{1.617151in}{1.506607in}}{\pgfqpoint{1.609338in}{1.514421in}}%
\pgfpathcurveto{\pgfqpoint{1.601524in}{1.522235in}}{\pgfqpoint{1.590925in}{1.526625in}}{\pgfqpoint{1.579875in}{1.526625in}}%
\pgfpathcurveto{\pgfqpoint{1.568825in}{1.526625in}}{\pgfqpoint{1.558226in}{1.522235in}}{\pgfqpoint{1.550412in}{1.514421in}}%
\pgfpathcurveto{\pgfqpoint{1.542599in}{1.506607in}}{\pgfqpoint{1.538208in}{1.496008in}}{\pgfqpoint{1.538208in}{1.484958in}}%
\pgfpathcurveto{\pgfqpoint{1.538208in}{1.473908in}}{\pgfqpoint{1.542599in}{1.463309in}}{\pgfqpoint{1.550412in}{1.455495in}}%
\pgfpathcurveto{\pgfqpoint{1.558226in}{1.447682in}}{\pgfqpoint{1.568825in}{1.443291in}}{\pgfqpoint{1.579875in}{1.443291in}}%
\pgfpathclose%
\pgfusepath{stroke,fill}%
\end{pgfscope}%
\begin{pgfscope}%
\pgfpathrectangle{\pgfqpoint{0.375000in}{0.330000in}}{\pgfqpoint{2.325000in}{2.310000in}}%
\pgfusepath{clip}%
\pgfsetbuttcap%
\pgfsetroundjoin%
\definecolor{currentfill}{rgb}{0.000000,0.000000,0.000000}%
\pgfsetfillcolor{currentfill}%
\pgfsetlinewidth{1.003750pt}%
\definecolor{currentstroke}{rgb}{0.000000,0.000000,0.000000}%
\pgfsetstrokecolor{currentstroke}%
\pgfsetdash{}{0pt}%
\pgfpathmoveto{\pgfqpoint{1.579875in}{1.443291in}}%
\pgfpathcurveto{\pgfqpoint{1.590925in}{1.443291in}}{\pgfqpoint{1.601524in}{1.447682in}}{\pgfqpoint{1.609338in}{1.455495in}}%
\pgfpathcurveto{\pgfqpoint{1.617151in}{1.463309in}}{\pgfqpoint{1.621542in}{1.473908in}}{\pgfqpoint{1.621542in}{1.484958in}}%
\pgfpathcurveto{\pgfqpoint{1.621542in}{1.496008in}}{\pgfqpoint{1.617151in}{1.506607in}}{\pgfqpoint{1.609338in}{1.514421in}}%
\pgfpathcurveto{\pgfqpoint{1.601524in}{1.522235in}}{\pgfqpoint{1.590925in}{1.526625in}}{\pgfqpoint{1.579875in}{1.526625in}}%
\pgfpathcurveto{\pgfqpoint{1.568825in}{1.526625in}}{\pgfqpoint{1.558226in}{1.522235in}}{\pgfqpoint{1.550412in}{1.514421in}}%
\pgfpathcurveto{\pgfqpoint{1.542599in}{1.506607in}}{\pgfqpoint{1.538208in}{1.496008in}}{\pgfqpoint{1.538208in}{1.484958in}}%
\pgfpathcurveto{\pgfqpoint{1.538208in}{1.473908in}}{\pgfqpoint{1.542599in}{1.463309in}}{\pgfqpoint{1.550412in}{1.455495in}}%
\pgfpathcurveto{\pgfqpoint{1.558226in}{1.447682in}}{\pgfqpoint{1.568825in}{1.443291in}}{\pgfqpoint{1.579875in}{1.443291in}}%
\pgfpathclose%
\pgfusepath{stroke,fill}%
\end{pgfscope}%
\begin{pgfscope}%
\pgfpathrectangle{\pgfqpoint{0.375000in}{0.330000in}}{\pgfqpoint{2.325000in}{2.310000in}}%
\pgfusepath{clip}%
\pgfsetbuttcap%
\pgfsetroundjoin%
\definecolor{currentfill}{rgb}{0.000000,0.000000,0.000000}%
\pgfsetfillcolor{currentfill}%
\pgfsetlinewidth{1.003750pt}%
\definecolor{currentstroke}{rgb}{0.000000,0.000000,0.000000}%
\pgfsetstrokecolor{currentstroke}%
\pgfsetdash{}{0pt}%
\pgfpathmoveto{\pgfqpoint{1.579875in}{2.474583in}}%
\pgfpathcurveto{\pgfqpoint{1.590925in}{2.474583in}}{\pgfqpoint{1.601524in}{2.478974in}}{\pgfqpoint{1.609338in}{2.486787in}}%
\pgfpathcurveto{\pgfqpoint{1.617151in}{2.494601in}}{\pgfqpoint{1.621542in}{2.505200in}}{\pgfqpoint{1.621542in}{2.516250in}}%
\pgfpathcurveto{\pgfqpoint{1.621542in}{2.527300in}}{\pgfqpoint{1.617151in}{2.537899in}}{\pgfqpoint{1.609338in}{2.545713in}}%
\pgfpathcurveto{\pgfqpoint{1.601524in}{2.553526in}}{\pgfqpoint{1.590925in}{2.557917in}}{\pgfqpoint{1.579875in}{2.557917in}}%
\pgfpathcurveto{\pgfqpoint{1.568825in}{2.557917in}}{\pgfqpoint{1.558226in}{2.553526in}}{\pgfqpoint{1.550412in}{2.545713in}}%
\pgfpathcurveto{\pgfqpoint{1.542599in}{2.537899in}}{\pgfqpoint{1.538208in}{2.527300in}}{\pgfqpoint{1.538208in}{2.516250in}}%
\pgfpathcurveto{\pgfqpoint{1.538208in}{2.505200in}}{\pgfqpoint{1.542599in}{2.494601in}}{\pgfqpoint{1.550412in}{2.486787in}}%
\pgfpathcurveto{\pgfqpoint{1.558226in}{2.478974in}}{\pgfqpoint{1.568825in}{2.474583in}}{\pgfqpoint{1.579875in}{2.474583in}}%
\pgfpathclose%
\pgfusepath{stroke,fill}%
\end{pgfscope}%
\begin{pgfscope}%
\pgfpathrectangle{\pgfqpoint{0.375000in}{0.330000in}}{\pgfqpoint{2.325000in}{2.310000in}}%
\pgfusepath{clip}%
\pgfsetbuttcap%
\pgfsetroundjoin%
\definecolor{currentfill}{rgb}{0.000000,0.000000,0.000000}%
\pgfsetfillcolor{currentfill}%
\pgfsetlinewidth{1.003750pt}%
\definecolor{currentstroke}{rgb}{0.000000,0.000000,0.000000}%
\pgfsetstrokecolor{currentstroke}%
\pgfsetdash{}{0pt}%
\pgfpathmoveto{\pgfqpoint{1.579875in}{1.443291in}}%
\pgfpathcurveto{\pgfqpoint{1.590925in}{1.443291in}}{\pgfqpoint{1.601524in}{1.447682in}}{\pgfqpoint{1.609338in}{1.455495in}}%
\pgfpathcurveto{\pgfqpoint{1.617151in}{1.463309in}}{\pgfqpoint{1.621542in}{1.473908in}}{\pgfqpoint{1.621542in}{1.484958in}}%
\pgfpathcurveto{\pgfqpoint{1.621542in}{1.496008in}}{\pgfqpoint{1.617151in}{1.506607in}}{\pgfqpoint{1.609338in}{1.514421in}}%
\pgfpathcurveto{\pgfqpoint{1.601524in}{1.522235in}}{\pgfqpoint{1.590925in}{1.526625in}}{\pgfqpoint{1.579875in}{1.526625in}}%
\pgfpathcurveto{\pgfqpoint{1.568825in}{1.526625in}}{\pgfqpoint{1.558226in}{1.522235in}}{\pgfqpoint{1.550412in}{1.514421in}}%
\pgfpathcurveto{\pgfqpoint{1.542599in}{1.506607in}}{\pgfqpoint{1.538208in}{1.496008in}}{\pgfqpoint{1.538208in}{1.484958in}}%
\pgfpathcurveto{\pgfqpoint{1.538208in}{1.473908in}}{\pgfqpoint{1.542599in}{1.463309in}}{\pgfqpoint{1.550412in}{1.455495in}}%
\pgfpathcurveto{\pgfqpoint{1.558226in}{1.447682in}}{\pgfqpoint{1.568825in}{1.443291in}}{\pgfqpoint{1.579875in}{1.443291in}}%
\pgfpathclose%
\pgfusepath{stroke,fill}%
\end{pgfscope}%
\begin{pgfscope}%
\pgfpathrectangle{\pgfqpoint{0.375000in}{0.330000in}}{\pgfqpoint{2.325000in}{2.310000in}}%
\pgfusepath{clip}%
\pgfsetbuttcap%
\pgfsetroundjoin%
\definecolor{currentfill}{rgb}{0.000000,0.000000,0.000000}%
\pgfsetfillcolor{currentfill}%
\pgfsetlinewidth{1.003750pt}%
\definecolor{currentstroke}{rgb}{0.000000,0.000000,0.000000}%
\pgfsetstrokecolor{currentstroke}%
\pgfsetdash{}{0pt}%
\pgfpathmoveto{\pgfqpoint{1.579875in}{1.443291in}}%
\pgfpathcurveto{\pgfqpoint{1.590925in}{1.443291in}}{\pgfqpoint{1.601524in}{1.447682in}}{\pgfqpoint{1.609338in}{1.455495in}}%
\pgfpathcurveto{\pgfqpoint{1.617151in}{1.463309in}}{\pgfqpoint{1.621542in}{1.473908in}}{\pgfqpoint{1.621542in}{1.484958in}}%
\pgfpathcurveto{\pgfqpoint{1.621542in}{1.496008in}}{\pgfqpoint{1.617151in}{1.506607in}}{\pgfqpoint{1.609338in}{1.514421in}}%
\pgfpathcurveto{\pgfqpoint{1.601524in}{1.522235in}}{\pgfqpoint{1.590925in}{1.526625in}}{\pgfqpoint{1.579875in}{1.526625in}}%
\pgfpathcurveto{\pgfqpoint{1.568825in}{1.526625in}}{\pgfqpoint{1.558226in}{1.522235in}}{\pgfqpoint{1.550412in}{1.514421in}}%
\pgfpathcurveto{\pgfqpoint{1.542599in}{1.506607in}}{\pgfqpoint{1.538208in}{1.496008in}}{\pgfqpoint{1.538208in}{1.484958in}}%
\pgfpathcurveto{\pgfqpoint{1.538208in}{1.473908in}}{\pgfqpoint{1.542599in}{1.463309in}}{\pgfqpoint{1.550412in}{1.455495in}}%
\pgfpathcurveto{\pgfqpoint{1.558226in}{1.447682in}}{\pgfqpoint{1.568825in}{1.443291in}}{\pgfqpoint{1.579875in}{1.443291in}}%
\pgfpathclose%
\pgfusepath{stroke,fill}%
\end{pgfscope}%
\begin{pgfscope}%
\pgfpathrectangle{\pgfqpoint{0.375000in}{0.330000in}}{\pgfqpoint{2.325000in}{2.310000in}}%
\pgfusepath{clip}%
\pgfsetbuttcap%
\pgfsetroundjoin%
\definecolor{currentfill}{rgb}{0.000000,0.000000,0.000000}%
\pgfsetfillcolor{currentfill}%
\pgfsetlinewidth{1.003750pt}%
\definecolor{currentstroke}{rgb}{0.000000,0.000000,0.000000}%
\pgfsetstrokecolor{currentstroke}%
\pgfsetdash{}{0pt}%
\pgfpathmoveto{\pgfqpoint{1.579875in}{2.474583in}}%
\pgfpathcurveto{\pgfqpoint{1.590925in}{2.474583in}}{\pgfqpoint{1.601524in}{2.478974in}}{\pgfqpoint{1.609338in}{2.486787in}}%
\pgfpathcurveto{\pgfqpoint{1.617151in}{2.494601in}}{\pgfqpoint{1.621542in}{2.505200in}}{\pgfqpoint{1.621542in}{2.516250in}}%
\pgfpathcurveto{\pgfqpoint{1.621542in}{2.527300in}}{\pgfqpoint{1.617151in}{2.537899in}}{\pgfqpoint{1.609338in}{2.545713in}}%
\pgfpathcurveto{\pgfqpoint{1.601524in}{2.553526in}}{\pgfqpoint{1.590925in}{2.557917in}}{\pgfqpoint{1.579875in}{2.557917in}}%
\pgfpathcurveto{\pgfqpoint{1.568825in}{2.557917in}}{\pgfqpoint{1.558226in}{2.553526in}}{\pgfqpoint{1.550412in}{2.545713in}}%
\pgfpathcurveto{\pgfqpoint{1.542599in}{2.537899in}}{\pgfqpoint{1.538208in}{2.527300in}}{\pgfqpoint{1.538208in}{2.516250in}}%
\pgfpathcurveto{\pgfqpoint{1.538208in}{2.505200in}}{\pgfqpoint{1.542599in}{2.494601in}}{\pgfqpoint{1.550412in}{2.486787in}}%
\pgfpathcurveto{\pgfqpoint{1.558226in}{2.478974in}}{\pgfqpoint{1.568825in}{2.474583in}}{\pgfqpoint{1.579875in}{2.474583in}}%
\pgfpathclose%
\pgfusepath{stroke,fill}%
\end{pgfscope}%
\begin{pgfscope}%
\pgfpathrectangle{\pgfqpoint{0.375000in}{0.330000in}}{\pgfqpoint{2.325000in}{2.310000in}}%
\pgfusepath{clip}%
\pgfsetbuttcap%
\pgfsetroundjoin%
\definecolor{currentfill}{rgb}{0.000000,0.000000,0.000000}%
\pgfsetfillcolor{currentfill}%
\pgfsetlinewidth{1.003750pt}%
\definecolor{currentstroke}{rgb}{0.000000,0.000000,0.000000}%
\pgfsetstrokecolor{currentstroke}%
\pgfsetdash{}{0pt}%
\pgfpathmoveto{\pgfqpoint{1.579875in}{1.443291in}}%
\pgfpathcurveto{\pgfqpoint{1.590925in}{1.443291in}}{\pgfqpoint{1.601524in}{1.447682in}}{\pgfqpoint{1.609338in}{1.455495in}}%
\pgfpathcurveto{\pgfqpoint{1.617151in}{1.463309in}}{\pgfqpoint{1.621542in}{1.473908in}}{\pgfqpoint{1.621542in}{1.484958in}}%
\pgfpathcurveto{\pgfqpoint{1.621542in}{1.496008in}}{\pgfqpoint{1.617151in}{1.506607in}}{\pgfqpoint{1.609338in}{1.514421in}}%
\pgfpathcurveto{\pgfqpoint{1.601524in}{1.522235in}}{\pgfqpoint{1.590925in}{1.526625in}}{\pgfqpoint{1.579875in}{1.526625in}}%
\pgfpathcurveto{\pgfqpoint{1.568825in}{1.526625in}}{\pgfqpoint{1.558226in}{1.522235in}}{\pgfqpoint{1.550412in}{1.514421in}}%
\pgfpathcurveto{\pgfqpoint{1.542599in}{1.506607in}}{\pgfqpoint{1.538208in}{1.496008in}}{\pgfqpoint{1.538208in}{1.484958in}}%
\pgfpathcurveto{\pgfqpoint{1.538208in}{1.473908in}}{\pgfqpoint{1.542599in}{1.463309in}}{\pgfqpoint{1.550412in}{1.455495in}}%
\pgfpathcurveto{\pgfqpoint{1.558226in}{1.447682in}}{\pgfqpoint{1.568825in}{1.443291in}}{\pgfqpoint{1.579875in}{1.443291in}}%
\pgfpathclose%
\pgfusepath{stroke,fill}%
\end{pgfscope}%
\begin{pgfscope}%
\pgfpathrectangle{\pgfqpoint{0.375000in}{0.330000in}}{\pgfqpoint{2.325000in}{2.310000in}}%
\pgfusepath{clip}%
\pgfsetbuttcap%
\pgfsetroundjoin%
\definecolor{currentfill}{rgb}{0.000000,0.000000,0.000000}%
\pgfsetfillcolor{currentfill}%
\pgfsetlinewidth{1.003750pt}%
\definecolor{currentstroke}{rgb}{0.000000,0.000000,0.000000}%
\pgfsetstrokecolor{currentstroke}%
\pgfsetdash{}{0pt}%
\pgfpathmoveto{\pgfqpoint{1.579875in}{1.443291in}}%
\pgfpathcurveto{\pgfqpoint{1.590925in}{1.443291in}}{\pgfqpoint{1.601524in}{1.447682in}}{\pgfqpoint{1.609338in}{1.455495in}}%
\pgfpathcurveto{\pgfqpoint{1.617151in}{1.463309in}}{\pgfqpoint{1.621542in}{1.473908in}}{\pgfqpoint{1.621542in}{1.484958in}}%
\pgfpathcurveto{\pgfqpoint{1.621542in}{1.496008in}}{\pgfqpoint{1.617151in}{1.506607in}}{\pgfqpoint{1.609338in}{1.514421in}}%
\pgfpathcurveto{\pgfqpoint{1.601524in}{1.522235in}}{\pgfqpoint{1.590925in}{1.526625in}}{\pgfqpoint{1.579875in}{1.526625in}}%
\pgfpathcurveto{\pgfqpoint{1.568825in}{1.526625in}}{\pgfqpoint{1.558226in}{1.522235in}}{\pgfqpoint{1.550412in}{1.514421in}}%
\pgfpathcurveto{\pgfqpoint{1.542599in}{1.506607in}}{\pgfqpoint{1.538208in}{1.496008in}}{\pgfqpoint{1.538208in}{1.484958in}}%
\pgfpathcurveto{\pgfqpoint{1.538208in}{1.473908in}}{\pgfqpoint{1.542599in}{1.463309in}}{\pgfqpoint{1.550412in}{1.455495in}}%
\pgfpathcurveto{\pgfqpoint{1.558226in}{1.447682in}}{\pgfqpoint{1.568825in}{1.443291in}}{\pgfqpoint{1.579875in}{1.443291in}}%
\pgfpathclose%
\pgfusepath{stroke,fill}%
\end{pgfscope}%
\begin{pgfscope}%
\pgfpathrectangle{\pgfqpoint{0.375000in}{0.330000in}}{\pgfqpoint{2.325000in}{2.310000in}}%
\pgfusepath{clip}%
\pgfsetbuttcap%
\pgfsetroundjoin%
\definecolor{currentfill}{rgb}{0.000000,0.000000,0.000000}%
\pgfsetfillcolor{currentfill}%
\pgfsetlinewidth{1.003750pt}%
\definecolor{currentstroke}{rgb}{0.000000,0.000000,0.000000}%
\pgfsetstrokecolor{currentstroke}%
\pgfsetdash{}{0pt}%
\pgfpathmoveto{\pgfqpoint{1.579875in}{2.474583in}}%
\pgfpathcurveto{\pgfqpoint{1.590925in}{2.474583in}}{\pgfqpoint{1.601524in}{2.478974in}}{\pgfqpoint{1.609338in}{2.486787in}}%
\pgfpathcurveto{\pgfqpoint{1.617151in}{2.494601in}}{\pgfqpoint{1.621542in}{2.505200in}}{\pgfqpoint{1.621542in}{2.516250in}}%
\pgfpathcurveto{\pgfqpoint{1.621542in}{2.527300in}}{\pgfqpoint{1.617151in}{2.537899in}}{\pgfqpoint{1.609338in}{2.545713in}}%
\pgfpathcurveto{\pgfqpoint{1.601524in}{2.553526in}}{\pgfqpoint{1.590925in}{2.557917in}}{\pgfqpoint{1.579875in}{2.557917in}}%
\pgfpathcurveto{\pgfqpoint{1.568825in}{2.557917in}}{\pgfqpoint{1.558226in}{2.553526in}}{\pgfqpoint{1.550412in}{2.545713in}}%
\pgfpathcurveto{\pgfqpoint{1.542599in}{2.537899in}}{\pgfqpoint{1.538208in}{2.527300in}}{\pgfqpoint{1.538208in}{2.516250in}}%
\pgfpathcurveto{\pgfqpoint{1.538208in}{2.505200in}}{\pgfqpoint{1.542599in}{2.494601in}}{\pgfqpoint{1.550412in}{2.486787in}}%
\pgfpathcurveto{\pgfqpoint{1.558226in}{2.478974in}}{\pgfqpoint{1.568825in}{2.474583in}}{\pgfqpoint{1.579875in}{2.474583in}}%
\pgfpathclose%
\pgfusepath{stroke,fill}%
\end{pgfscope}%
\begin{pgfscope}%
\pgfpathrectangle{\pgfqpoint{0.375000in}{0.330000in}}{\pgfqpoint{2.325000in}{2.310000in}}%
\pgfusepath{clip}%
\pgfsetbuttcap%
\pgfsetroundjoin%
\definecolor{currentfill}{rgb}{0.000000,0.000000,0.000000}%
\pgfsetfillcolor{currentfill}%
\pgfsetlinewidth{1.003750pt}%
\definecolor{currentstroke}{rgb}{0.000000,0.000000,0.000000}%
\pgfsetstrokecolor{currentstroke}%
\pgfsetdash{}{0pt}%
\pgfpathmoveto{\pgfqpoint{1.579875in}{1.443291in}}%
\pgfpathcurveto{\pgfqpoint{1.590925in}{1.443291in}}{\pgfqpoint{1.601524in}{1.447682in}}{\pgfqpoint{1.609338in}{1.455495in}}%
\pgfpathcurveto{\pgfqpoint{1.617151in}{1.463309in}}{\pgfqpoint{1.621542in}{1.473908in}}{\pgfqpoint{1.621542in}{1.484958in}}%
\pgfpathcurveto{\pgfqpoint{1.621542in}{1.496008in}}{\pgfqpoint{1.617151in}{1.506607in}}{\pgfqpoint{1.609338in}{1.514421in}}%
\pgfpathcurveto{\pgfqpoint{1.601524in}{1.522235in}}{\pgfqpoint{1.590925in}{1.526625in}}{\pgfqpoint{1.579875in}{1.526625in}}%
\pgfpathcurveto{\pgfqpoint{1.568825in}{1.526625in}}{\pgfqpoint{1.558226in}{1.522235in}}{\pgfqpoint{1.550412in}{1.514421in}}%
\pgfpathcurveto{\pgfqpoint{1.542599in}{1.506607in}}{\pgfqpoint{1.538208in}{1.496008in}}{\pgfqpoint{1.538208in}{1.484958in}}%
\pgfpathcurveto{\pgfqpoint{1.538208in}{1.473908in}}{\pgfqpoint{1.542599in}{1.463309in}}{\pgfqpoint{1.550412in}{1.455495in}}%
\pgfpathcurveto{\pgfqpoint{1.558226in}{1.447682in}}{\pgfqpoint{1.568825in}{1.443291in}}{\pgfqpoint{1.579875in}{1.443291in}}%
\pgfpathclose%
\pgfusepath{stroke,fill}%
\end{pgfscope}%
\begin{pgfscope}%
\pgfpathrectangle{\pgfqpoint{0.375000in}{0.330000in}}{\pgfqpoint{2.325000in}{2.310000in}}%
\pgfusepath{clip}%
\pgfsetbuttcap%
\pgfsetroundjoin%
\definecolor{currentfill}{rgb}{0.000000,0.000000,0.000000}%
\pgfsetfillcolor{currentfill}%
\pgfsetlinewidth{1.003750pt}%
\definecolor{currentstroke}{rgb}{0.000000,0.000000,0.000000}%
\pgfsetstrokecolor{currentstroke}%
\pgfsetdash{}{0pt}%
\pgfpathmoveto{\pgfqpoint{1.579875in}{1.443291in}}%
\pgfpathcurveto{\pgfqpoint{1.590925in}{1.443291in}}{\pgfqpoint{1.601524in}{1.447682in}}{\pgfqpoint{1.609338in}{1.455495in}}%
\pgfpathcurveto{\pgfqpoint{1.617151in}{1.463309in}}{\pgfqpoint{1.621542in}{1.473908in}}{\pgfqpoint{1.621542in}{1.484958in}}%
\pgfpathcurveto{\pgfqpoint{1.621542in}{1.496008in}}{\pgfqpoint{1.617151in}{1.506607in}}{\pgfqpoint{1.609338in}{1.514421in}}%
\pgfpathcurveto{\pgfqpoint{1.601524in}{1.522235in}}{\pgfqpoint{1.590925in}{1.526625in}}{\pgfqpoint{1.579875in}{1.526625in}}%
\pgfpathcurveto{\pgfqpoint{1.568825in}{1.526625in}}{\pgfqpoint{1.558226in}{1.522235in}}{\pgfqpoint{1.550412in}{1.514421in}}%
\pgfpathcurveto{\pgfqpoint{1.542599in}{1.506607in}}{\pgfqpoint{1.538208in}{1.496008in}}{\pgfqpoint{1.538208in}{1.484958in}}%
\pgfpathcurveto{\pgfqpoint{1.538208in}{1.473908in}}{\pgfqpoint{1.542599in}{1.463309in}}{\pgfqpoint{1.550412in}{1.455495in}}%
\pgfpathcurveto{\pgfqpoint{1.558226in}{1.447682in}}{\pgfqpoint{1.568825in}{1.443291in}}{\pgfqpoint{1.579875in}{1.443291in}}%
\pgfpathclose%
\pgfusepath{stroke,fill}%
\end{pgfscope}%
\begin{pgfscope}%
\pgfpathrectangle{\pgfqpoint{0.375000in}{0.330000in}}{\pgfqpoint{2.325000in}{2.310000in}}%
\pgfusepath{clip}%
\pgfsetbuttcap%
\pgfsetroundjoin%
\definecolor{currentfill}{rgb}{0.000000,0.000000,0.000000}%
\pgfsetfillcolor{currentfill}%
\pgfsetlinewidth{1.003750pt}%
\definecolor{currentstroke}{rgb}{0.000000,0.000000,0.000000}%
\pgfsetstrokecolor{currentstroke}%
\pgfsetdash{}{0pt}%
\pgfpathmoveto{\pgfqpoint{1.579875in}{1.443291in}}%
\pgfpathcurveto{\pgfqpoint{1.590925in}{1.443291in}}{\pgfqpoint{1.601524in}{1.447682in}}{\pgfqpoint{1.609338in}{1.455495in}}%
\pgfpathcurveto{\pgfqpoint{1.617151in}{1.463309in}}{\pgfqpoint{1.621542in}{1.473908in}}{\pgfqpoint{1.621542in}{1.484958in}}%
\pgfpathcurveto{\pgfqpoint{1.621542in}{1.496008in}}{\pgfqpoint{1.617151in}{1.506607in}}{\pgfqpoint{1.609338in}{1.514421in}}%
\pgfpathcurveto{\pgfqpoint{1.601524in}{1.522235in}}{\pgfqpoint{1.590925in}{1.526625in}}{\pgfqpoint{1.579875in}{1.526625in}}%
\pgfpathcurveto{\pgfqpoint{1.568825in}{1.526625in}}{\pgfqpoint{1.558226in}{1.522235in}}{\pgfqpoint{1.550412in}{1.514421in}}%
\pgfpathcurveto{\pgfqpoint{1.542599in}{1.506607in}}{\pgfqpoint{1.538208in}{1.496008in}}{\pgfqpoint{1.538208in}{1.484958in}}%
\pgfpathcurveto{\pgfqpoint{1.538208in}{1.473908in}}{\pgfqpoint{1.542599in}{1.463309in}}{\pgfqpoint{1.550412in}{1.455495in}}%
\pgfpathcurveto{\pgfqpoint{1.558226in}{1.447682in}}{\pgfqpoint{1.568825in}{1.443291in}}{\pgfqpoint{1.579875in}{1.443291in}}%
\pgfpathclose%
\pgfusepath{stroke,fill}%
\end{pgfscope}%
\begin{pgfscope}%
\pgfpathrectangle{\pgfqpoint{0.375000in}{0.330000in}}{\pgfqpoint{2.325000in}{2.310000in}}%
\pgfusepath{clip}%
\pgfsetbuttcap%
\pgfsetroundjoin%
\definecolor{currentfill}{rgb}{0.000000,0.000000,0.000000}%
\pgfsetfillcolor{currentfill}%
\pgfsetlinewidth{1.003750pt}%
\definecolor{currentstroke}{rgb}{0.000000,0.000000,0.000000}%
\pgfsetstrokecolor{currentstroke}%
\pgfsetdash{}{0pt}%
\pgfpathmoveto{\pgfqpoint{1.579875in}{2.474583in}}%
\pgfpathcurveto{\pgfqpoint{1.590925in}{2.474583in}}{\pgfqpoint{1.601524in}{2.478974in}}{\pgfqpoint{1.609338in}{2.486787in}}%
\pgfpathcurveto{\pgfqpoint{1.617151in}{2.494601in}}{\pgfqpoint{1.621542in}{2.505200in}}{\pgfqpoint{1.621542in}{2.516250in}}%
\pgfpathcurveto{\pgfqpoint{1.621542in}{2.527300in}}{\pgfqpoint{1.617151in}{2.537899in}}{\pgfqpoint{1.609338in}{2.545713in}}%
\pgfpathcurveto{\pgfqpoint{1.601524in}{2.553526in}}{\pgfqpoint{1.590925in}{2.557917in}}{\pgfqpoint{1.579875in}{2.557917in}}%
\pgfpathcurveto{\pgfqpoint{1.568825in}{2.557917in}}{\pgfqpoint{1.558226in}{2.553526in}}{\pgfqpoint{1.550412in}{2.545713in}}%
\pgfpathcurveto{\pgfqpoint{1.542599in}{2.537899in}}{\pgfqpoint{1.538208in}{2.527300in}}{\pgfqpoint{1.538208in}{2.516250in}}%
\pgfpathcurveto{\pgfqpoint{1.538208in}{2.505200in}}{\pgfqpoint{1.542599in}{2.494601in}}{\pgfqpoint{1.550412in}{2.486787in}}%
\pgfpathcurveto{\pgfqpoint{1.558226in}{2.478974in}}{\pgfqpoint{1.568825in}{2.474583in}}{\pgfqpoint{1.579875in}{2.474583in}}%
\pgfpathclose%
\pgfusepath{stroke,fill}%
\end{pgfscope}%
\begin{pgfscope}%
\pgfpathrectangle{\pgfqpoint{0.375000in}{0.330000in}}{\pgfqpoint{2.325000in}{2.310000in}}%
\pgfusepath{clip}%
\pgfsetbuttcap%
\pgfsetroundjoin%
\definecolor{currentfill}{rgb}{0.000000,0.000000,0.000000}%
\pgfsetfillcolor{currentfill}%
\pgfsetlinewidth{1.003750pt}%
\definecolor{currentstroke}{rgb}{0.000000,0.000000,0.000000}%
\pgfsetstrokecolor{currentstroke}%
\pgfsetdash{}{0pt}%
\pgfpathmoveto{\pgfqpoint{1.579875in}{1.443291in}}%
\pgfpathcurveto{\pgfqpoint{1.590925in}{1.443291in}}{\pgfqpoint{1.601524in}{1.447682in}}{\pgfqpoint{1.609338in}{1.455495in}}%
\pgfpathcurveto{\pgfqpoint{1.617151in}{1.463309in}}{\pgfqpoint{1.621542in}{1.473908in}}{\pgfqpoint{1.621542in}{1.484958in}}%
\pgfpathcurveto{\pgfqpoint{1.621542in}{1.496008in}}{\pgfqpoint{1.617151in}{1.506607in}}{\pgfqpoint{1.609338in}{1.514421in}}%
\pgfpathcurveto{\pgfqpoint{1.601524in}{1.522235in}}{\pgfqpoint{1.590925in}{1.526625in}}{\pgfqpoint{1.579875in}{1.526625in}}%
\pgfpathcurveto{\pgfqpoint{1.568825in}{1.526625in}}{\pgfqpoint{1.558226in}{1.522235in}}{\pgfqpoint{1.550412in}{1.514421in}}%
\pgfpathcurveto{\pgfqpoint{1.542599in}{1.506607in}}{\pgfqpoint{1.538208in}{1.496008in}}{\pgfqpoint{1.538208in}{1.484958in}}%
\pgfpathcurveto{\pgfqpoint{1.538208in}{1.473908in}}{\pgfqpoint{1.542599in}{1.463309in}}{\pgfqpoint{1.550412in}{1.455495in}}%
\pgfpathcurveto{\pgfqpoint{1.558226in}{1.447682in}}{\pgfqpoint{1.568825in}{1.443291in}}{\pgfqpoint{1.579875in}{1.443291in}}%
\pgfpathclose%
\pgfusepath{stroke,fill}%
\end{pgfscope}%
\begin{pgfscope}%
\pgfpathrectangle{\pgfqpoint{0.375000in}{0.330000in}}{\pgfqpoint{2.325000in}{2.310000in}}%
\pgfusepath{clip}%
\pgfsetbuttcap%
\pgfsetroundjoin%
\definecolor{currentfill}{rgb}{0.000000,0.000000,0.000000}%
\pgfsetfillcolor{currentfill}%
\pgfsetlinewidth{1.003750pt}%
\definecolor{currentstroke}{rgb}{0.000000,0.000000,0.000000}%
\pgfsetstrokecolor{currentstroke}%
\pgfsetdash{}{0pt}%
\pgfpathmoveto{\pgfqpoint{1.579875in}{1.443291in}}%
\pgfpathcurveto{\pgfqpoint{1.590925in}{1.443291in}}{\pgfqpoint{1.601524in}{1.447682in}}{\pgfqpoint{1.609338in}{1.455495in}}%
\pgfpathcurveto{\pgfqpoint{1.617151in}{1.463309in}}{\pgfqpoint{1.621542in}{1.473908in}}{\pgfqpoint{1.621542in}{1.484958in}}%
\pgfpathcurveto{\pgfqpoint{1.621542in}{1.496008in}}{\pgfqpoint{1.617151in}{1.506607in}}{\pgfqpoint{1.609338in}{1.514421in}}%
\pgfpathcurveto{\pgfqpoint{1.601524in}{1.522235in}}{\pgfqpoint{1.590925in}{1.526625in}}{\pgfqpoint{1.579875in}{1.526625in}}%
\pgfpathcurveto{\pgfqpoint{1.568825in}{1.526625in}}{\pgfqpoint{1.558226in}{1.522235in}}{\pgfqpoint{1.550412in}{1.514421in}}%
\pgfpathcurveto{\pgfqpoint{1.542599in}{1.506607in}}{\pgfqpoint{1.538208in}{1.496008in}}{\pgfqpoint{1.538208in}{1.484958in}}%
\pgfpathcurveto{\pgfqpoint{1.538208in}{1.473908in}}{\pgfqpoint{1.542599in}{1.463309in}}{\pgfqpoint{1.550412in}{1.455495in}}%
\pgfpathcurveto{\pgfqpoint{1.558226in}{1.447682in}}{\pgfqpoint{1.568825in}{1.443291in}}{\pgfqpoint{1.579875in}{1.443291in}}%
\pgfpathclose%
\pgfusepath{stroke,fill}%
\end{pgfscope}%
\begin{pgfscope}%
\pgfpathrectangle{\pgfqpoint{0.375000in}{0.330000in}}{\pgfqpoint{2.325000in}{2.310000in}}%
\pgfusepath{clip}%
\pgfsetbuttcap%
\pgfsetroundjoin%
\definecolor{currentfill}{rgb}{0.000000,0.000000,0.000000}%
\pgfsetfillcolor{currentfill}%
\pgfsetlinewidth{1.003750pt}%
\definecolor{currentstroke}{rgb}{0.000000,0.000000,0.000000}%
\pgfsetstrokecolor{currentstroke}%
\pgfsetdash{}{0pt}%
\pgfpathmoveto{\pgfqpoint{1.579875in}{1.443291in}}%
\pgfpathcurveto{\pgfqpoint{1.590925in}{1.443291in}}{\pgfqpoint{1.601524in}{1.447682in}}{\pgfqpoint{1.609338in}{1.455495in}}%
\pgfpathcurveto{\pgfqpoint{1.617151in}{1.463309in}}{\pgfqpoint{1.621542in}{1.473908in}}{\pgfqpoint{1.621542in}{1.484958in}}%
\pgfpathcurveto{\pgfqpoint{1.621542in}{1.496008in}}{\pgfqpoint{1.617151in}{1.506607in}}{\pgfqpoint{1.609338in}{1.514421in}}%
\pgfpathcurveto{\pgfqpoint{1.601524in}{1.522235in}}{\pgfqpoint{1.590925in}{1.526625in}}{\pgfqpoint{1.579875in}{1.526625in}}%
\pgfpathcurveto{\pgfqpoint{1.568825in}{1.526625in}}{\pgfqpoint{1.558226in}{1.522235in}}{\pgfqpoint{1.550412in}{1.514421in}}%
\pgfpathcurveto{\pgfqpoint{1.542599in}{1.506607in}}{\pgfqpoint{1.538208in}{1.496008in}}{\pgfqpoint{1.538208in}{1.484958in}}%
\pgfpathcurveto{\pgfqpoint{1.538208in}{1.473908in}}{\pgfqpoint{1.542599in}{1.463309in}}{\pgfqpoint{1.550412in}{1.455495in}}%
\pgfpathcurveto{\pgfqpoint{1.558226in}{1.447682in}}{\pgfqpoint{1.568825in}{1.443291in}}{\pgfqpoint{1.579875in}{1.443291in}}%
\pgfpathclose%
\pgfusepath{stroke,fill}%
\end{pgfscope}%
\begin{pgfscope}%
\pgfpathrectangle{\pgfqpoint{0.375000in}{0.330000in}}{\pgfqpoint{2.325000in}{2.310000in}}%
\pgfusepath{clip}%
\pgfsetbuttcap%
\pgfsetroundjoin%
\definecolor{currentfill}{rgb}{0.000000,0.000000,0.000000}%
\pgfsetfillcolor{currentfill}%
\pgfsetlinewidth{1.003750pt}%
\definecolor{currentstroke}{rgb}{0.000000,0.000000,0.000000}%
\pgfsetstrokecolor{currentstroke}%
\pgfsetdash{}{0pt}%
\pgfpathmoveto{\pgfqpoint{1.579875in}{1.443291in}}%
\pgfpathcurveto{\pgfqpoint{1.590925in}{1.443291in}}{\pgfqpoint{1.601524in}{1.447682in}}{\pgfqpoint{1.609338in}{1.455495in}}%
\pgfpathcurveto{\pgfqpoint{1.617151in}{1.463309in}}{\pgfqpoint{1.621542in}{1.473908in}}{\pgfqpoint{1.621542in}{1.484958in}}%
\pgfpathcurveto{\pgfqpoint{1.621542in}{1.496008in}}{\pgfqpoint{1.617151in}{1.506607in}}{\pgfqpoint{1.609338in}{1.514421in}}%
\pgfpathcurveto{\pgfqpoint{1.601524in}{1.522235in}}{\pgfqpoint{1.590925in}{1.526625in}}{\pgfqpoint{1.579875in}{1.526625in}}%
\pgfpathcurveto{\pgfqpoint{1.568825in}{1.526625in}}{\pgfqpoint{1.558226in}{1.522235in}}{\pgfqpoint{1.550412in}{1.514421in}}%
\pgfpathcurveto{\pgfqpoint{1.542599in}{1.506607in}}{\pgfqpoint{1.538208in}{1.496008in}}{\pgfqpoint{1.538208in}{1.484958in}}%
\pgfpathcurveto{\pgfqpoint{1.538208in}{1.473908in}}{\pgfqpoint{1.542599in}{1.463309in}}{\pgfqpoint{1.550412in}{1.455495in}}%
\pgfpathcurveto{\pgfqpoint{1.558226in}{1.447682in}}{\pgfqpoint{1.568825in}{1.443291in}}{\pgfqpoint{1.579875in}{1.443291in}}%
\pgfpathclose%
\pgfusepath{stroke,fill}%
\end{pgfscope}%
\begin{pgfscope}%
\pgfpathrectangle{\pgfqpoint{0.375000in}{0.330000in}}{\pgfqpoint{2.325000in}{2.310000in}}%
\pgfusepath{clip}%
\pgfsetbuttcap%
\pgfsetroundjoin%
\definecolor{currentfill}{rgb}{0.000000,0.000000,0.000000}%
\pgfsetfillcolor{currentfill}%
\pgfsetlinewidth{1.003750pt}%
\definecolor{currentstroke}{rgb}{0.000000,0.000000,0.000000}%
\pgfsetstrokecolor{currentstroke}%
\pgfsetdash{}{0pt}%
\pgfpathmoveto{\pgfqpoint{1.579875in}{1.443291in}}%
\pgfpathcurveto{\pgfqpoint{1.590925in}{1.443291in}}{\pgfqpoint{1.601524in}{1.447682in}}{\pgfqpoint{1.609338in}{1.455495in}}%
\pgfpathcurveto{\pgfqpoint{1.617151in}{1.463309in}}{\pgfqpoint{1.621542in}{1.473908in}}{\pgfqpoint{1.621542in}{1.484958in}}%
\pgfpathcurveto{\pgfqpoint{1.621542in}{1.496008in}}{\pgfqpoint{1.617151in}{1.506607in}}{\pgfqpoint{1.609338in}{1.514421in}}%
\pgfpathcurveto{\pgfqpoint{1.601524in}{1.522235in}}{\pgfqpoint{1.590925in}{1.526625in}}{\pgfqpoint{1.579875in}{1.526625in}}%
\pgfpathcurveto{\pgfqpoint{1.568825in}{1.526625in}}{\pgfqpoint{1.558226in}{1.522235in}}{\pgfqpoint{1.550412in}{1.514421in}}%
\pgfpathcurveto{\pgfqpoint{1.542599in}{1.506607in}}{\pgfqpoint{1.538208in}{1.496008in}}{\pgfqpoint{1.538208in}{1.484958in}}%
\pgfpathcurveto{\pgfqpoint{1.538208in}{1.473908in}}{\pgfqpoint{1.542599in}{1.463309in}}{\pgfqpoint{1.550412in}{1.455495in}}%
\pgfpathcurveto{\pgfqpoint{1.558226in}{1.447682in}}{\pgfqpoint{1.568825in}{1.443291in}}{\pgfqpoint{1.579875in}{1.443291in}}%
\pgfpathclose%
\pgfusepath{stroke,fill}%
\end{pgfscope}%
\begin{pgfscope}%
\pgfpathrectangle{\pgfqpoint{0.375000in}{0.330000in}}{\pgfqpoint{2.325000in}{2.310000in}}%
\pgfusepath{clip}%
\pgfsetbuttcap%
\pgfsetroundjoin%
\definecolor{currentfill}{rgb}{0.000000,0.000000,0.000000}%
\pgfsetfillcolor{currentfill}%
\pgfsetlinewidth{1.003750pt}%
\definecolor{currentstroke}{rgb}{0.000000,0.000000,0.000000}%
\pgfsetstrokecolor{currentstroke}%
\pgfsetdash{}{0pt}%
\pgfpathmoveto{\pgfqpoint{1.579875in}{1.443291in}}%
\pgfpathcurveto{\pgfqpoint{1.590925in}{1.443291in}}{\pgfqpoint{1.601524in}{1.447682in}}{\pgfqpoint{1.609338in}{1.455495in}}%
\pgfpathcurveto{\pgfqpoint{1.617151in}{1.463309in}}{\pgfqpoint{1.621542in}{1.473908in}}{\pgfqpoint{1.621542in}{1.484958in}}%
\pgfpathcurveto{\pgfqpoint{1.621542in}{1.496008in}}{\pgfqpoint{1.617151in}{1.506607in}}{\pgfqpoint{1.609338in}{1.514421in}}%
\pgfpathcurveto{\pgfqpoint{1.601524in}{1.522235in}}{\pgfqpoint{1.590925in}{1.526625in}}{\pgfqpoint{1.579875in}{1.526625in}}%
\pgfpathcurveto{\pgfqpoint{1.568825in}{1.526625in}}{\pgfqpoint{1.558226in}{1.522235in}}{\pgfqpoint{1.550412in}{1.514421in}}%
\pgfpathcurveto{\pgfqpoint{1.542599in}{1.506607in}}{\pgfqpoint{1.538208in}{1.496008in}}{\pgfqpoint{1.538208in}{1.484958in}}%
\pgfpathcurveto{\pgfqpoint{1.538208in}{1.473908in}}{\pgfqpoint{1.542599in}{1.463309in}}{\pgfqpoint{1.550412in}{1.455495in}}%
\pgfpathcurveto{\pgfqpoint{1.558226in}{1.447682in}}{\pgfqpoint{1.568825in}{1.443291in}}{\pgfqpoint{1.579875in}{1.443291in}}%
\pgfpathclose%
\pgfusepath{stroke,fill}%
\end{pgfscope}%
\begin{pgfscope}%
\pgfpathrectangle{\pgfqpoint{0.375000in}{0.330000in}}{\pgfqpoint{2.325000in}{2.310000in}}%
\pgfusepath{clip}%
\pgfsetbuttcap%
\pgfsetroundjoin%
\definecolor{currentfill}{rgb}{0.000000,0.000000,0.000000}%
\pgfsetfillcolor{currentfill}%
\pgfsetlinewidth{1.003750pt}%
\definecolor{currentstroke}{rgb}{0.000000,0.000000,0.000000}%
\pgfsetstrokecolor{currentstroke}%
\pgfsetdash{}{0pt}%
\pgfpathmoveto{\pgfqpoint{1.579875in}{1.443291in}}%
\pgfpathcurveto{\pgfqpoint{1.590925in}{1.443291in}}{\pgfqpoint{1.601524in}{1.447682in}}{\pgfqpoint{1.609338in}{1.455495in}}%
\pgfpathcurveto{\pgfqpoint{1.617151in}{1.463309in}}{\pgfqpoint{1.621542in}{1.473908in}}{\pgfqpoint{1.621542in}{1.484958in}}%
\pgfpathcurveto{\pgfqpoint{1.621542in}{1.496008in}}{\pgfqpoint{1.617151in}{1.506607in}}{\pgfqpoint{1.609338in}{1.514421in}}%
\pgfpathcurveto{\pgfqpoint{1.601524in}{1.522235in}}{\pgfqpoint{1.590925in}{1.526625in}}{\pgfqpoint{1.579875in}{1.526625in}}%
\pgfpathcurveto{\pgfqpoint{1.568825in}{1.526625in}}{\pgfqpoint{1.558226in}{1.522235in}}{\pgfqpoint{1.550412in}{1.514421in}}%
\pgfpathcurveto{\pgfqpoint{1.542599in}{1.506607in}}{\pgfqpoint{1.538208in}{1.496008in}}{\pgfqpoint{1.538208in}{1.484958in}}%
\pgfpathcurveto{\pgfqpoint{1.538208in}{1.473908in}}{\pgfqpoint{1.542599in}{1.463309in}}{\pgfqpoint{1.550412in}{1.455495in}}%
\pgfpathcurveto{\pgfqpoint{1.558226in}{1.447682in}}{\pgfqpoint{1.568825in}{1.443291in}}{\pgfqpoint{1.579875in}{1.443291in}}%
\pgfpathclose%
\pgfusepath{stroke,fill}%
\end{pgfscope}%
\begin{pgfscope}%
\pgfpathrectangle{\pgfqpoint{0.375000in}{0.330000in}}{\pgfqpoint{2.325000in}{2.310000in}}%
\pgfusepath{clip}%
\pgfsetbuttcap%
\pgfsetroundjoin%
\definecolor{currentfill}{rgb}{0.000000,0.000000,0.000000}%
\pgfsetfillcolor{currentfill}%
\pgfsetlinewidth{1.003750pt}%
\definecolor{currentstroke}{rgb}{0.000000,0.000000,0.000000}%
\pgfsetstrokecolor{currentstroke}%
\pgfsetdash{}{0pt}%
\pgfpathmoveto{\pgfqpoint{1.579875in}{2.474583in}}%
\pgfpathcurveto{\pgfqpoint{1.590925in}{2.474583in}}{\pgfqpoint{1.601524in}{2.478974in}}{\pgfqpoint{1.609338in}{2.486787in}}%
\pgfpathcurveto{\pgfqpoint{1.617151in}{2.494601in}}{\pgfqpoint{1.621542in}{2.505200in}}{\pgfqpoint{1.621542in}{2.516250in}}%
\pgfpathcurveto{\pgfqpoint{1.621542in}{2.527300in}}{\pgfqpoint{1.617151in}{2.537899in}}{\pgfqpoint{1.609338in}{2.545713in}}%
\pgfpathcurveto{\pgfqpoint{1.601524in}{2.553526in}}{\pgfqpoint{1.590925in}{2.557917in}}{\pgfqpoint{1.579875in}{2.557917in}}%
\pgfpathcurveto{\pgfqpoint{1.568825in}{2.557917in}}{\pgfqpoint{1.558226in}{2.553526in}}{\pgfqpoint{1.550412in}{2.545713in}}%
\pgfpathcurveto{\pgfqpoint{1.542599in}{2.537899in}}{\pgfqpoint{1.538208in}{2.527300in}}{\pgfqpoint{1.538208in}{2.516250in}}%
\pgfpathcurveto{\pgfqpoint{1.538208in}{2.505200in}}{\pgfqpoint{1.542599in}{2.494601in}}{\pgfqpoint{1.550412in}{2.486787in}}%
\pgfpathcurveto{\pgfqpoint{1.558226in}{2.478974in}}{\pgfqpoint{1.568825in}{2.474583in}}{\pgfqpoint{1.579875in}{2.474583in}}%
\pgfpathclose%
\pgfusepath{stroke,fill}%
\end{pgfscope}%
\begin{pgfscope}%
\pgfpathrectangle{\pgfqpoint{0.375000in}{0.330000in}}{\pgfqpoint{2.325000in}{2.310000in}}%
\pgfusepath{clip}%
\pgfsetbuttcap%
\pgfsetroundjoin%
\definecolor{currentfill}{rgb}{0.000000,0.000000,0.000000}%
\pgfsetfillcolor{currentfill}%
\pgfsetlinewidth{1.003750pt}%
\definecolor{currentstroke}{rgb}{0.000000,0.000000,0.000000}%
\pgfsetstrokecolor{currentstroke}%
\pgfsetdash{}{0pt}%
\pgfpathmoveto{\pgfqpoint{1.579875in}{1.443291in}}%
\pgfpathcurveto{\pgfqpoint{1.590925in}{1.443291in}}{\pgfqpoint{1.601524in}{1.447682in}}{\pgfqpoint{1.609338in}{1.455495in}}%
\pgfpathcurveto{\pgfqpoint{1.617151in}{1.463309in}}{\pgfqpoint{1.621542in}{1.473908in}}{\pgfqpoint{1.621542in}{1.484958in}}%
\pgfpathcurveto{\pgfqpoint{1.621542in}{1.496008in}}{\pgfqpoint{1.617151in}{1.506607in}}{\pgfqpoint{1.609338in}{1.514421in}}%
\pgfpathcurveto{\pgfqpoint{1.601524in}{1.522235in}}{\pgfqpoint{1.590925in}{1.526625in}}{\pgfqpoint{1.579875in}{1.526625in}}%
\pgfpathcurveto{\pgfqpoint{1.568825in}{1.526625in}}{\pgfqpoint{1.558226in}{1.522235in}}{\pgfqpoint{1.550412in}{1.514421in}}%
\pgfpathcurveto{\pgfqpoint{1.542599in}{1.506607in}}{\pgfqpoint{1.538208in}{1.496008in}}{\pgfqpoint{1.538208in}{1.484958in}}%
\pgfpathcurveto{\pgfqpoint{1.538208in}{1.473908in}}{\pgfqpoint{1.542599in}{1.463309in}}{\pgfqpoint{1.550412in}{1.455495in}}%
\pgfpathcurveto{\pgfqpoint{1.558226in}{1.447682in}}{\pgfqpoint{1.568825in}{1.443291in}}{\pgfqpoint{1.579875in}{1.443291in}}%
\pgfpathclose%
\pgfusepath{stroke,fill}%
\end{pgfscope}%
\begin{pgfscope}%
\pgfpathrectangle{\pgfqpoint{0.375000in}{0.330000in}}{\pgfqpoint{2.325000in}{2.310000in}}%
\pgfusepath{clip}%
\pgfsetbuttcap%
\pgfsetroundjoin%
\definecolor{currentfill}{rgb}{0.000000,0.000000,0.000000}%
\pgfsetfillcolor{currentfill}%
\pgfsetlinewidth{1.003750pt}%
\definecolor{currentstroke}{rgb}{0.000000,0.000000,0.000000}%
\pgfsetstrokecolor{currentstroke}%
\pgfsetdash{}{0pt}%
\pgfpathmoveto{\pgfqpoint{1.579875in}{1.443291in}}%
\pgfpathcurveto{\pgfqpoint{1.590925in}{1.443291in}}{\pgfqpoint{1.601524in}{1.447682in}}{\pgfqpoint{1.609338in}{1.455495in}}%
\pgfpathcurveto{\pgfqpoint{1.617151in}{1.463309in}}{\pgfqpoint{1.621542in}{1.473908in}}{\pgfqpoint{1.621542in}{1.484958in}}%
\pgfpathcurveto{\pgfqpoint{1.621542in}{1.496008in}}{\pgfqpoint{1.617151in}{1.506607in}}{\pgfqpoint{1.609338in}{1.514421in}}%
\pgfpathcurveto{\pgfqpoint{1.601524in}{1.522235in}}{\pgfqpoint{1.590925in}{1.526625in}}{\pgfqpoint{1.579875in}{1.526625in}}%
\pgfpathcurveto{\pgfqpoint{1.568825in}{1.526625in}}{\pgfqpoint{1.558226in}{1.522235in}}{\pgfqpoint{1.550412in}{1.514421in}}%
\pgfpathcurveto{\pgfqpoint{1.542599in}{1.506607in}}{\pgfqpoint{1.538208in}{1.496008in}}{\pgfqpoint{1.538208in}{1.484958in}}%
\pgfpathcurveto{\pgfqpoint{1.538208in}{1.473908in}}{\pgfqpoint{1.542599in}{1.463309in}}{\pgfqpoint{1.550412in}{1.455495in}}%
\pgfpathcurveto{\pgfqpoint{1.558226in}{1.447682in}}{\pgfqpoint{1.568825in}{1.443291in}}{\pgfqpoint{1.579875in}{1.443291in}}%
\pgfpathclose%
\pgfusepath{stroke,fill}%
\end{pgfscope}%
\begin{pgfscope}%
\pgfpathrectangle{\pgfqpoint{0.375000in}{0.330000in}}{\pgfqpoint{2.325000in}{2.310000in}}%
\pgfusepath{clip}%
\pgfsetbuttcap%
\pgfsetroundjoin%
\definecolor{currentfill}{rgb}{0.000000,0.000000,0.000000}%
\pgfsetfillcolor{currentfill}%
\pgfsetlinewidth{1.003750pt}%
\definecolor{currentstroke}{rgb}{0.000000,0.000000,0.000000}%
\pgfsetstrokecolor{currentstroke}%
\pgfsetdash{}{0pt}%
\pgfpathmoveto{\pgfqpoint{1.579875in}{1.443291in}}%
\pgfpathcurveto{\pgfqpoint{1.590925in}{1.443291in}}{\pgfqpoint{1.601524in}{1.447682in}}{\pgfqpoint{1.609338in}{1.455495in}}%
\pgfpathcurveto{\pgfqpoint{1.617151in}{1.463309in}}{\pgfqpoint{1.621542in}{1.473908in}}{\pgfqpoint{1.621542in}{1.484958in}}%
\pgfpathcurveto{\pgfqpoint{1.621542in}{1.496008in}}{\pgfqpoint{1.617151in}{1.506607in}}{\pgfqpoint{1.609338in}{1.514421in}}%
\pgfpathcurveto{\pgfqpoint{1.601524in}{1.522235in}}{\pgfqpoint{1.590925in}{1.526625in}}{\pgfqpoint{1.579875in}{1.526625in}}%
\pgfpathcurveto{\pgfqpoint{1.568825in}{1.526625in}}{\pgfqpoint{1.558226in}{1.522235in}}{\pgfqpoint{1.550412in}{1.514421in}}%
\pgfpathcurveto{\pgfqpoint{1.542599in}{1.506607in}}{\pgfqpoint{1.538208in}{1.496008in}}{\pgfqpoint{1.538208in}{1.484958in}}%
\pgfpathcurveto{\pgfqpoint{1.538208in}{1.473908in}}{\pgfqpoint{1.542599in}{1.463309in}}{\pgfqpoint{1.550412in}{1.455495in}}%
\pgfpathcurveto{\pgfqpoint{1.558226in}{1.447682in}}{\pgfqpoint{1.568825in}{1.443291in}}{\pgfqpoint{1.579875in}{1.443291in}}%
\pgfpathclose%
\pgfusepath{stroke,fill}%
\end{pgfscope}%
\begin{pgfscope}%
\pgfpathrectangle{\pgfqpoint{0.375000in}{0.330000in}}{\pgfqpoint{2.325000in}{2.310000in}}%
\pgfusepath{clip}%
\pgfsetbuttcap%
\pgfsetroundjoin%
\definecolor{currentfill}{rgb}{0.000000,0.000000,0.000000}%
\pgfsetfillcolor{currentfill}%
\pgfsetlinewidth{1.003750pt}%
\definecolor{currentstroke}{rgb}{0.000000,0.000000,0.000000}%
\pgfsetstrokecolor{currentstroke}%
\pgfsetdash{}{0pt}%
\pgfpathmoveto{\pgfqpoint{1.579875in}{1.443291in}}%
\pgfpathcurveto{\pgfqpoint{1.590925in}{1.443291in}}{\pgfqpoint{1.601524in}{1.447682in}}{\pgfqpoint{1.609338in}{1.455495in}}%
\pgfpathcurveto{\pgfqpoint{1.617151in}{1.463309in}}{\pgfqpoint{1.621542in}{1.473908in}}{\pgfqpoint{1.621542in}{1.484958in}}%
\pgfpathcurveto{\pgfqpoint{1.621542in}{1.496008in}}{\pgfqpoint{1.617151in}{1.506607in}}{\pgfqpoint{1.609338in}{1.514421in}}%
\pgfpathcurveto{\pgfqpoint{1.601524in}{1.522235in}}{\pgfqpoint{1.590925in}{1.526625in}}{\pgfqpoint{1.579875in}{1.526625in}}%
\pgfpathcurveto{\pgfqpoint{1.568825in}{1.526625in}}{\pgfqpoint{1.558226in}{1.522235in}}{\pgfqpoint{1.550412in}{1.514421in}}%
\pgfpathcurveto{\pgfqpoint{1.542599in}{1.506607in}}{\pgfqpoint{1.538208in}{1.496008in}}{\pgfqpoint{1.538208in}{1.484958in}}%
\pgfpathcurveto{\pgfqpoint{1.538208in}{1.473908in}}{\pgfqpoint{1.542599in}{1.463309in}}{\pgfqpoint{1.550412in}{1.455495in}}%
\pgfpathcurveto{\pgfqpoint{1.558226in}{1.447682in}}{\pgfqpoint{1.568825in}{1.443291in}}{\pgfqpoint{1.579875in}{1.443291in}}%
\pgfpathclose%
\pgfusepath{stroke,fill}%
\end{pgfscope}%
\begin{pgfscope}%
\pgfpathrectangle{\pgfqpoint{0.375000in}{0.330000in}}{\pgfqpoint{2.325000in}{2.310000in}}%
\pgfusepath{clip}%
\pgfsetbuttcap%
\pgfsetroundjoin%
\definecolor{currentfill}{rgb}{0.000000,0.000000,0.000000}%
\pgfsetfillcolor{currentfill}%
\pgfsetlinewidth{1.003750pt}%
\definecolor{currentstroke}{rgb}{0.000000,0.000000,0.000000}%
\pgfsetstrokecolor{currentstroke}%
\pgfsetdash{}{0pt}%
\pgfpathmoveto{\pgfqpoint{1.579875in}{1.443291in}}%
\pgfpathcurveto{\pgfqpoint{1.590925in}{1.443291in}}{\pgfqpoint{1.601524in}{1.447682in}}{\pgfqpoint{1.609338in}{1.455495in}}%
\pgfpathcurveto{\pgfqpoint{1.617151in}{1.463309in}}{\pgfqpoint{1.621542in}{1.473908in}}{\pgfqpoint{1.621542in}{1.484958in}}%
\pgfpathcurveto{\pgfqpoint{1.621542in}{1.496008in}}{\pgfqpoint{1.617151in}{1.506607in}}{\pgfqpoint{1.609338in}{1.514421in}}%
\pgfpathcurveto{\pgfqpoint{1.601524in}{1.522235in}}{\pgfqpoint{1.590925in}{1.526625in}}{\pgfqpoint{1.579875in}{1.526625in}}%
\pgfpathcurveto{\pgfqpoint{1.568825in}{1.526625in}}{\pgfqpoint{1.558226in}{1.522235in}}{\pgfqpoint{1.550412in}{1.514421in}}%
\pgfpathcurveto{\pgfqpoint{1.542599in}{1.506607in}}{\pgfqpoint{1.538208in}{1.496008in}}{\pgfqpoint{1.538208in}{1.484958in}}%
\pgfpathcurveto{\pgfqpoint{1.538208in}{1.473908in}}{\pgfqpoint{1.542599in}{1.463309in}}{\pgfqpoint{1.550412in}{1.455495in}}%
\pgfpathcurveto{\pgfqpoint{1.558226in}{1.447682in}}{\pgfqpoint{1.568825in}{1.443291in}}{\pgfqpoint{1.579875in}{1.443291in}}%
\pgfpathclose%
\pgfusepath{stroke,fill}%
\end{pgfscope}%
\begin{pgfscope}%
\pgfpathrectangle{\pgfqpoint{0.375000in}{0.330000in}}{\pgfqpoint{2.325000in}{2.310000in}}%
\pgfusepath{clip}%
\pgfsetbuttcap%
\pgfsetroundjoin%
\definecolor{currentfill}{rgb}{0.000000,0.000000,0.000000}%
\pgfsetfillcolor{currentfill}%
\pgfsetlinewidth{1.003750pt}%
\definecolor{currentstroke}{rgb}{0.000000,0.000000,0.000000}%
\pgfsetstrokecolor{currentstroke}%
\pgfsetdash{}{0pt}%
\pgfpathmoveto{\pgfqpoint{1.579875in}{1.443291in}}%
\pgfpathcurveto{\pgfqpoint{1.590925in}{1.443291in}}{\pgfqpoint{1.601524in}{1.447682in}}{\pgfqpoint{1.609338in}{1.455495in}}%
\pgfpathcurveto{\pgfqpoint{1.617151in}{1.463309in}}{\pgfqpoint{1.621542in}{1.473908in}}{\pgfqpoint{1.621542in}{1.484958in}}%
\pgfpathcurveto{\pgfqpoint{1.621542in}{1.496008in}}{\pgfqpoint{1.617151in}{1.506607in}}{\pgfqpoint{1.609338in}{1.514421in}}%
\pgfpathcurveto{\pgfqpoint{1.601524in}{1.522235in}}{\pgfqpoint{1.590925in}{1.526625in}}{\pgfqpoint{1.579875in}{1.526625in}}%
\pgfpathcurveto{\pgfqpoint{1.568825in}{1.526625in}}{\pgfqpoint{1.558226in}{1.522235in}}{\pgfqpoint{1.550412in}{1.514421in}}%
\pgfpathcurveto{\pgfqpoint{1.542599in}{1.506607in}}{\pgfqpoint{1.538208in}{1.496008in}}{\pgfqpoint{1.538208in}{1.484958in}}%
\pgfpathcurveto{\pgfqpoint{1.538208in}{1.473908in}}{\pgfqpoint{1.542599in}{1.463309in}}{\pgfqpoint{1.550412in}{1.455495in}}%
\pgfpathcurveto{\pgfqpoint{1.558226in}{1.447682in}}{\pgfqpoint{1.568825in}{1.443291in}}{\pgfqpoint{1.579875in}{1.443291in}}%
\pgfpathclose%
\pgfusepath{stroke,fill}%
\end{pgfscope}%
\begin{pgfscope}%
\pgfpathrectangle{\pgfqpoint{0.375000in}{0.330000in}}{\pgfqpoint{2.325000in}{2.310000in}}%
\pgfusepath{clip}%
\pgfsetbuttcap%
\pgfsetroundjoin%
\definecolor{currentfill}{rgb}{0.000000,0.000000,0.000000}%
\pgfsetfillcolor{currentfill}%
\pgfsetlinewidth{1.003750pt}%
\definecolor{currentstroke}{rgb}{0.000000,0.000000,0.000000}%
\pgfsetstrokecolor{currentstroke}%
\pgfsetdash{}{0pt}%
\pgfpathmoveto{\pgfqpoint{1.579875in}{1.443291in}}%
\pgfpathcurveto{\pgfqpoint{1.590925in}{1.443291in}}{\pgfqpoint{1.601524in}{1.447682in}}{\pgfqpoint{1.609338in}{1.455495in}}%
\pgfpathcurveto{\pgfqpoint{1.617151in}{1.463309in}}{\pgfqpoint{1.621542in}{1.473908in}}{\pgfqpoint{1.621542in}{1.484958in}}%
\pgfpathcurveto{\pgfqpoint{1.621542in}{1.496008in}}{\pgfqpoint{1.617151in}{1.506607in}}{\pgfqpoint{1.609338in}{1.514421in}}%
\pgfpathcurveto{\pgfqpoint{1.601524in}{1.522235in}}{\pgfqpoint{1.590925in}{1.526625in}}{\pgfqpoint{1.579875in}{1.526625in}}%
\pgfpathcurveto{\pgfqpoint{1.568825in}{1.526625in}}{\pgfqpoint{1.558226in}{1.522235in}}{\pgfqpoint{1.550412in}{1.514421in}}%
\pgfpathcurveto{\pgfqpoint{1.542599in}{1.506607in}}{\pgfqpoint{1.538208in}{1.496008in}}{\pgfqpoint{1.538208in}{1.484958in}}%
\pgfpathcurveto{\pgfqpoint{1.538208in}{1.473908in}}{\pgfqpoint{1.542599in}{1.463309in}}{\pgfqpoint{1.550412in}{1.455495in}}%
\pgfpathcurveto{\pgfqpoint{1.558226in}{1.447682in}}{\pgfqpoint{1.568825in}{1.443291in}}{\pgfqpoint{1.579875in}{1.443291in}}%
\pgfpathclose%
\pgfusepath{stroke,fill}%
\end{pgfscope}%
\begin{pgfscope}%
\pgfpathrectangle{\pgfqpoint{0.375000in}{0.330000in}}{\pgfqpoint{2.325000in}{2.310000in}}%
\pgfusepath{clip}%
\pgfsetbuttcap%
\pgfsetroundjoin%
\definecolor{currentfill}{rgb}{0.000000,0.000000,0.000000}%
\pgfsetfillcolor{currentfill}%
\pgfsetlinewidth{1.003750pt}%
\definecolor{currentstroke}{rgb}{0.000000,0.000000,0.000000}%
\pgfsetstrokecolor{currentstroke}%
\pgfsetdash{}{0pt}%
\pgfpathmoveto{\pgfqpoint{1.579875in}{2.474583in}}%
\pgfpathcurveto{\pgfqpoint{1.590925in}{2.474583in}}{\pgfqpoint{1.601524in}{2.478974in}}{\pgfqpoint{1.609338in}{2.486787in}}%
\pgfpathcurveto{\pgfqpoint{1.617151in}{2.494601in}}{\pgfqpoint{1.621542in}{2.505200in}}{\pgfqpoint{1.621542in}{2.516250in}}%
\pgfpathcurveto{\pgfqpoint{1.621542in}{2.527300in}}{\pgfqpoint{1.617151in}{2.537899in}}{\pgfqpoint{1.609338in}{2.545713in}}%
\pgfpathcurveto{\pgfqpoint{1.601524in}{2.553526in}}{\pgfqpoint{1.590925in}{2.557917in}}{\pgfqpoint{1.579875in}{2.557917in}}%
\pgfpathcurveto{\pgfqpoint{1.568825in}{2.557917in}}{\pgfqpoint{1.558226in}{2.553526in}}{\pgfqpoint{1.550412in}{2.545713in}}%
\pgfpathcurveto{\pgfqpoint{1.542599in}{2.537899in}}{\pgfqpoint{1.538208in}{2.527300in}}{\pgfqpoint{1.538208in}{2.516250in}}%
\pgfpathcurveto{\pgfqpoint{1.538208in}{2.505200in}}{\pgfqpoint{1.542599in}{2.494601in}}{\pgfqpoint{1.550412in}{2.486787in}}%
\pgfpathcurveto{\pgfqpoint{1.558226in}{2.478974in}}{\pgfqpoint{1.568825in}{2.474583in}}{\pgfqpoint{1.579875in}{2.474583in}}%
\pgfpathclose%
\pgfusepath{stroke,fill}%
\end{pgfscope}%
\begin{pgfscope}%
\pgfpathrectangle{\pgfqpoint{0.375000in}{0.330000in}}{\pgfqpoint{2.325000in}{2.310000in}}%
\pgfusepath{clip}%
\pgfsetbuttcap%
\pgfsetroundjoin%
\definecolor{currentfill}{rgb}{0.000000,0.000000,0.000000}%
\pgfsetfillcolor{currentfill}%
\pgfsetlinewidth{1.003750pt}%
\definecolor{currentstroke}{rgb}{0.000000,0.000000,0.000000}%
\pgfsetstrokecolor{currentstroke}%
\pgfsetdash{}{0pt}%
\pgfpathmoveto{\pgfqpoint{1.579875in}{1.443291in}}%
\pgfpathcurveto{\pgfqpoint{1.590925in}{1.443291in}}{\pgfqpoint{1.601524in}{1.447682in}}{\pgfqpoint{1.609338in}{1.455495in}}%
\pgfpathcurveto{\pgfqpoint{1.617151in}{1.463309in}}{\pgfqpoint{1.621542in}{1.473908in}}{\pgfqpoint{1.621542in}{1.484958in}}%
\pgfpathcurveto{\pgfqpoint{1.621542in}{1.496008in}}{\pgfqpoint{1.617151in}{1.506607in}}{\pgfqpoint{1.609338in}{1.514421in}}%
\pgfpathcurveto{\pgfqpoint{1.601524in}{1.522235in}}{\pgfqpoint{1.590925in}{1.526625in}}{\pgfqpoint{1.579875in}{1.526625in}}%
\pgfpathcurveto{\pgfqpoint{1.568825in}{1.526625in}}{\pgfqpoint{1.558226in}{1.522235in}}{\pgfqpoint{1.550412in}{1.514421in}}%
\pgfpathcurveto{\pgfqpoint{1.542599in}{1.506607in}}{\pgfqpoint{1.538208in}{1.496008in}}{\pgfqpoint{1.538208in}{1.484958in}}%
\pgfpathcurveto{\pgfqpoint{1.538208in}{1.473908in}}{\pgfqpoint{1.542599in}{1.463309in}}{\pgfqpoint{1.550412in}{1.455495in}}%
\pgfpathcurveto{\pgfqpoint{1.558226in}{1.447682in}}{\pgfqpoint{1.568825in}{1.443291in}}{\pgfqpoint{1.579875in}{1.443291in}}%
\pgfpathclose%
\pgfusepath{stroke,fill}%
\end{pgfscope}%
\begin{pgfscope}%
\pgfpathrectangle{\pgfqpoint{0.375000in}{0.330000in}}{\pgfqpoint{2.325000in}{2.310000in}}%
\pgfusepath{clip}%
\pgfsetbuttcap%
\pgfsetroundjoin%
\definecolor{currentfill}{rgb}{0.000000,0.000000,0.000000}%
\pgfsetfillcolor{currentfill}%
\pgfsetlinewidth{1.003750pt}%
\definecolor{currentstroke}{rgb}{0.000000,0.000000,0.000000}%
\pgfsetstrokecolor{currentstroke}%
\pgfsetdash{}{0pt}%
\pgfpathmoveto{\pgfqpoint{1.579875in}{1.443291in}}%
\pgfpathcurveto{\pgfqpoint{1.590925in}{1.443291in}}{\pgfqpoint{1.601524in}{1.447682in}}{\pgfqpoint{1.609338in}{1.455495in}}%
\pgfpathcurveto{\pgfqpoint{1.617151in}{1.463309in}}{\pgfqpoint{1.621542in}{1.473908in}}{\pgfqpoint{1.621542in}{1.484958in}}%
\pgfpathcurveto{\pgfqpoint{1.621542in}{1.496008in}}{\pgfqpoint{1.617151in}{1.506607in}}{\pgfqpoint{1.609338in}{1.514421in}}%
\pgfpathcurveto{\pgfqpoint{1.601524in}{1.522235in}}{\pgfqpoint{1.590925in}{1.526625in}}{\pgfqpoint{1.579875in}{1.526625in}}%
\pgfpathcurveto{\pgfqpoint{1.568825in}{1.526625in}}{\pgfqpoint{1.558226in}{1.522235in}}{\pgfqpoint{1.550412in}{1.514421in}}%
\pgfpathcurveto{\pgfqpoint{1.542599in}{1.506607in}}{\pgfqpoint{1.538208in}{1.496008in}}{\pgfqpoint{1.538208in}{1.484958in}}%
\pgfpathcurveto{\pgfqpoint{1.538208in}{1.473908in}}{\pgfqpoint{1.542599in}{1.463309in}}{\pgfqpoint{1.550412in}{1.455495in}}%
\pgfpathcurveto{\pgfqpoint{1.558226in}{1.447682in}}{\pgfqpoint{1.568825in}{1.443291in}}{\pgfqpoint{1.579875in}{1.443291in}}%
\pgfpathclose%
\pgfusepath{stroke,fill}%
\end{pgfscope}%
\begin{pgfscope}%
\pgfpathrectangle{\pgfqpoint{0.375000in}{0.330000in}}{\pgfqpoint{2.325000in}{2.310000in}}%
\pgfusepath{clip}%
\pgfsetbuttcap%
\pgfsetroundjoin%
\definecolor{currentfill}{rgb}{0.000000,0.000000,0.000000}%
\pgfsetfillcolor{currentfill}%
\pgfsetlinewidth{1.003750pt}%
\definecolor{currentstroke}{rgb}{0.000000,0.000000,0.000000}%
\pgfsetstrokecolor{currentstroke}%
\pgfsetdash{}{0pt}%
\pgfpathmoveto{\pgfqpoint{1.579875in}{1.443291in}}%
\pgfpathcurveto{\pgfqpoint{1.590925in}{1.443291in}}{\pgfqpoint{1.601524in}{1.447682in}}{\pgfqpoint{1.609338in}{1.455495in}}%
\pgfpathcurveto{\pgfqpoint{1.617151in}{1.463309in}}{\pgfqpoint{1.621542in}{1.473908in}}{\pgfqpoint{1.621542in}{1.484958in}}%
\pgfpathcurveto{\pgfqpoint{1.621542in}{1.496008in}}{\pgfqpoint{1.617151in}{1.506607in}}{\pgfqpoint{1.609338in}{1.514421in}}%
\pgfpathcurveto{\pgfqpoint{1.601524in}{1.522235in}}{\pgfqpoint{1.590925in}{1.526625in}}{\pgfqpoint{1.579875in}{1.526625in}}%
\pgfpathcurveto{\pgfqpoint{1.568825in}{1.526625in}}{\pgfqpoint{1.558226in}{1.522235in}}{\pgfqpoint{1.550412in}{1.514421in}}%
\pgfpathcurveto{\pgfqpoint{1.542599in}{1.506607in}}{\pgfqpoint{1.538208in}{1.496008in}}{\pgfqpoint{1.538208in}{1.484958in}}%
\pgfpathcurveto{\pgfqpoint{1.538208in}{1.473908in}}{\pgfqpoint{1.542599in}{1.463309in}}{\pgfqpoint{1.550412in}{1.455495in}}%
\pgfpathcurveto{\pgfqpoint{1.558226in}{1.447682in}}{\pgfqpoint{1.568825in}{1.443291in}}{\pgfqpoint{1.579875in}{1.443291in}}%
\pgfpathclose%
\pgfusepath{stroke,fill}%
\end{pgfscope}%
\begin{pgfscope}%
\pgfpathrectangle{\pgfqpoint{0.375000in}{0.330000in}}{\pgfqpoint{2.325000in}{2.310000in}}%
\pgfusepath{clip}%
\pgfsetbuttcap%
\pgfsetroundjoin%
\definecolor{currentfill}{rgb}{0.000000,0.000000,0.000000}%
\pgfsetfillcolor{currentfill}%
\pgfsetlinewidth{1.003750pt}%
\definecolor{currentstroke}{rgb}{0.000000,0.000000,0.000000}%
\pgfsetstrokecolor{currentstroke}%
\pgfsetdash{}{0pt}%
\pgfpathmoveto{\pgfqpoint{1.579875in}{2.474583in}}%
\pgfpathcurveto{\pgfqpoint{1.590925in}{2.474583in}}{\pgfqpoint{1.601524in}{2.478974in}}{\pgfqpoint{1.609338in}{2.486787in}}%
\pgfpathcurveto{\pgfqpoint{1.617151in}{2.494601in}}{\pgfqpoint{1.621542in}{2.505200in}}{\pgfqpoint{1.621542in}{2.516250in}}%
\pgfpathcurveto{\pgfqpoint{1.621542in}{2.527300in}}{\pgfqpoint{1.617151in}{2.537899in}}{\pgfqpoint{1.609338in}{2.545713in}}%
\pgfpathcurveto{\pgfqpoint{1.601524in}{2.553526in}}{\pgfqpoint{1.590925in}{2.557917in}}{\pgfqpoint{1.579875in}{2.557917in}}%
\pgfpathcurveto{\pgfqpoint{1.568825in}{2.557917in}}{\pgfqpoint{1.558226in}{2.553526in}}{\pgfqpoint{1.550412in}{2.545713in}}%
\pgfpathcurveto{\pgfqpoint{1.542599in}{2.537899in}}{\pgfqpoint{1.538208in}{2.527300in}}{\pgfqpoint{1.538208in}{2.516250in}}%
\pgfpathcurveto{\pgfqpoint{1.538208in}{2.505200in}}{\pgfqpoint{1.542599in}{2.494601in}}{\pgfqpoint{1.550412in}{2.486787in}}%
\pgfpathcurveto{\pgfqpoint{1.558226in}{2.478974in}}{\pgfqpoint{1.568825in}{2.474583in}}{\pgfqpoint{1.579875in}{2.474583in}}%
\pgfpathclose%
\pgfusepath{stroke,fill}%
\end{pgfscope}%
\begin{pgfscope}%
\pgfpathrectangle{\pgfqpoint{0.375000in}{0.330000in}}{\pgfqpoint{2.325000in}{2.310000in}}%
\pgfusepath{clip}%
\pgfsetbuttcap%
\pgfsetroundjoin%
\definecolor{currentfill}{rgb}{0.000000,0.000000,0.000000}%
\pgfsetfillcolor{currentfill}%
\pgfsetlinewidth{1.003750pt}%
\definecolor{currentstroke}{rgb}{0.000000,0.000000,0.000000}%
\pgfsetstrokecolor{currentstroke}%
\pgfsetdash{}{0pt}%
\pgfpathmoveto{\pgfqpoint{1.579875in}{1.443291in}}%
\pgfpathcurveto{\pgfqpoint{1.590925in}{1.443291in}}{\pgfqpoint{1.601524in}{1.447682in}}{\pgfqpoint{1.609338in}{1.455495in}}%
\pgfpathcurveto{\pgfqpoint{1.617151in}{1.463309in}}{\pgfqpoint{1.621542in}{1.473908in}}{\pgfqpoint{1.621542in}{1.484958in}}%
\pgfpathcurveto{\pgfqpoint{1.621542in}{1.496008in}}{\pgfqpoint{1.617151in}{1.506607in}}{\pgfqpoint{1.609338in}{1.514421in}}%
\pgfpathcurveto{\pgfqpoint{1.601524in}{1.522235in}}{\pgfqpoint{1.590925in}{1.526625in}}{\pgfqpoint{1.579875in}{1.526625in}}%
\pgfpathcurveto{\pgfqpoint{1.568825in}{1.526625in}}{\pgfqpoint{1.558226in}{1.522235in}}{\pgfqpoint{1.550412in}{1.514421in}}%
\pgfpathcurveto{\pgfqpoint{1.542599in}{1.506607in}}{\pgfqpoint{1.538208in}{1.496008in}}{\pgfqpoint{1.538208in}{1.484958in}}%
\pgfpathcurveto{\pgfqpoint{1.538208in}{1.473908in}}{\pgfqpoint{1.542599in}{1.463309in}}{\pgfqpoint{1.550412in}{1.455495in}}%
\pgfpathcurveto{\pgfqpoint{1.558226in}{1.447682in}}{\pgfqpoint{1.568825in}{1.443291in}}{\pgfqpoint{1.579875in}{1.443291in}}%
\pgfpathclose%
\pgfusepath{stroke,fill}%
\end{pgfscope}%
\begin{pgfscope}%
\pgfpathrectangle{\pgfqpoint{0.375000in}{0.330000in}}{\pgfqpoint{2.325000in}{2.310000in}}%
\pgfusepath{clip}%
\pgfsetbuttcap%
\pgfsetroundjoin%
\definecolor{currentfill}{rgb}{0.000000,0.000000,0.000000}%
\pgfsetfillcolor{currentfill}%
\pgfsetlinewidth{1.003750pt}%
\definecolor{currentstroke}{rgb}{0.000000,0.000000,0.000000}%
\pgfsetstrokecolor{currentstroke}%
\pgfsetdash{}{0pt}%
\pgfpathmoveto{\pgfqpoint{1.579875in}{1.443291in}}%
\pgfpathcurveto{\pgfqpoint{1.590925in}{1.443291in}}{\pgfqpoint{1.601524in}{1.447682in}}{\pgfqpoint{1.609338in}{1.455495in}}%
\pgfpathcurveto{\pgfqpoint{1.617151in}{1.463309in}}{\pgfqpoint{1.621542in}{1.473908in}}{\pgfqpoint{1.621542in}{1.484958in}}%
\pgfpathcurveto{\pgfqpoint{1.621542in}{1.496008in}}{\pgfqpoint{1.617151in}{1.506607in}}{\pgfqpoint{1.609338in}{1.514421in}}%
\pgfpathcurveto{\pgfqpoint{1.601524in}{1.522235in}}{\pgfqpoint{1.590925in}{1.526625in}}{\pgfqpoint{1.579875in}{1.526625in}}%
\pgfpathcurveto{\pgfqpoint{1.568825in}{1.526625in}}{\pgfqpoint{1.558226in}{1.522235in}}{\pgfqpoint{1.550412in}{1.514421in}}%
\pgfpathcurveto{\pgfqpoint{1.542599in}{1.506607in}}{\pgfqpoint{1.538208in}{1.496008in}}{\pgfqpoint{1.538208in}{1.484958in}}%
\pgfpathcurveto{\pgfqpoint{1.538208in}{1.473908in}}{\pgfqpoint{1.542599in}{1.463309in}}{\pgfqpoint{1.550412in}{1.455495in}}%
\pgfpathcurveto{\pgfqpoint{1.558226in}{1.447682in}}{\pgfqpoint{1.568825in}{1.443291in}}{\pgfqpoint{1.579875in}{1.443291in}}%
\pgfpathclose%
\pgfusepath{stroke,fill}%
\end{pgfscope}%
\begin{pgfscope}%
\pgfpathrectangle{\pgfqpoint{0.375000in}{0.330000in}}{\pgfqpoint{2.325000in}{2.310000in}}%
\pgfusepath{clip}%
\pgfsetbuttcap%
\pgfsetroundjoin%
\definecolor{currentfill}{rgb}{0.000000,0.000000,0.000000}%
\pgfsetfillcolor{currentfill}%
\pgfsetlinewidth{1.003750pt}%
\definecolor{currentstroke}{rgb}{0.000000,0.000000,0.000000}%
\pgfsetstrokecolor{currentstroke}%
\pgfsetdash{}{0pt}%
\pgfpathmoveto{\pgfqpoint{1.579875in}{1.443291in}}%
\pgfpathcurveto{\pgfqpoint{1.590925in}{1.443291in}}{\pgfqpoint{1.601524in}{1.447682in}}{\pgfqpoint{1.609338in}{1.455495in}}%
\pgfpathcurveto{\pgfqpoint{1.617151in}{1.463309in}}{\pgfqpoint{1.621542in}{1.473908in}}{\pgfqpoint{1.621542in}{1.484958in}}%
\pgfpathcurveto{\pgfqpoint{1.621542in}{1.496008in}}{\pgfqpoint{1.617151in}{1.506607in}}{\pgfqpoint{1.609338in}{1.514421in}}%
\pgfpathcurveto{\pgfqpoint{1.601524in}{1.522235in}}{\pgfqpoint{1.590925in}{1.526625in}}{\pgfqpoint{1.579875in}{1.526625in}}%
\pgfpathcurveto{\pgfqpoint{1.568825in}{1.526625in}}{\pgfqpoint{1.558226in}{1.522235in}}{\pgfqpoint{1.550412in}{1.514421in}}%
\pgfpathcurveto{\pgfqpoint{1.542599in}{1.506607in}}{\pgfqpoint{1.538208in}{1.496008in}}{\pgfqpoint{1.538208in}{1.484958in}}%
\pgfpathcurveto{\pgfqpoint{1.538208in}{1.473908in}}{\pgfqpoint{1.542599in}{1.463309in}}{\pgfqpoint{1.550412in}{1.455495in}}%
\pgfpathcurveto{\pgfqpoint{1.558226in}{1.447682in}}{\pgfqpoint{1.568825in}{1.443291in}}{\pgfqpoint{1.579875in}{1.443291in}}%
\pgfpathclose%
\pgfusepath{stroke,fill}%
\end{pgfscope}%
\begin{pgfscope}%
\pgfpathrectangle{\pgfqpoint{0.375000in}{0.330000in}}{\pgfqpoint{2.325000in}{2.310000in}}%
\pgfusepath{clip}%
\pgfsetbuttcap%
\pgfsetroundjoin%
\definecolor{currentfill}{rgb}{0.000000,0.000000,0.000000}%
\pgfsetfillcolor{currentfill}%
\pgfsetlinewidth{1.003750pt}%
\definecolor{currentstroke}{rgb}{0.000000,0.000000,0.000000}%
\pgfsetstrokecolor{currentstroke}%
\pgfsetdash{}{0pt}%
\pgfpathmoveto{\pgfqpoint{1.579875in}{1.443291in}}%
\pgfpathcurveto{\pgfqpoint{1.590925in}{1.443291in}}{\pgfqpoint{1.601524in}{1.447682in}}{\pgfqpoint{1.609338in}{1.455495in}}%
\pgfpathcurveto{\pgfqpoint{1.617151in}{1.463309in}}{\pgfqpoint{1.621542in}{1.473908in}}{\pgfqpoint{1.621542in}{1.484958in}}%
\pgfpathcurveto{\pgfqpoint{1.621542in}{1.496008in}}{\pgfqpoint{1.617151in}{1.506607in}}{\pgfqpoint{1.609338in}{1.514421in}}%
\pgfpathcurveto{\pgfqpoint{1.601524in}{1.522235in}}{\pgfqpoint{1.590925in}{1.526625in}}{\pgfqpoint{1.579875in}{1.526625in}}%
\pgfpathcurveto{\pgfqpoint{1.568825in}{1.526625in}}{\pgfqpoint{1.558226in}{1.522235in}}{\pgfqpoint{1.550412in}{1.514421in}}%
\pgfpathcurveto{\pgfqpoint{1.542599in}{1.506607in}}{\pgfqpoint{1.538208in}{1.496008in}}{\pgfqpoint{1.538208in}{1.484958in}}%
\pgfpathcurveto{\pgfqpoint{1.538208in}{1.473908in}}{\pgfqpoint{1.542599in}{1.463309in}}{\pgfqpoint{1.550412in}{1.455495in}}%
\pgfpathcurveto{\pgfqpoint{1.558226in}{1.447682in}}{\pgfqpoint{1.568825in}{1.443291in}}{\pgfqpoint{1.579875in}{1.443291in}}%
\pgfpathclose%
\pgfusepath{stroke,fill}%
\end{pgfscope}%
\begin{pgfscope}%
\pgfpathrectangle{\pgfqpoint{0.375000in}{0.330000in}}{\pgfqpoint{2.325000in}{2.310000in}}%
\pgfusepath{clip}%
\pgfsetbuttcap%
\pgfsetroundjoin%
\definecolor{currentfill}{rgb}{0.000000,0.000000,0.000000}%
\pgfsetfillcolor{currentfill}%
\pgfsetlinewidth{1.003750pt}%
\definecolor{currentstroke}{rgb}{0.000000,0.000000,0.000000}%
\pgfsetstrokecolor{currentstroke}%
\pgfsetdash{}{0pt}%
\pgfpathmoveto{\pgfqpoint{1.579875in}{1.443291in}}%
\pgfpathcurveto{\pgfqpoint{1.590925in}{1.443291in}}{\pgfqpoint{1.601524in}{1.447682in}}{\pgfqpoint{1.609338in}{1.455495in}}%
\pgfpathcurveto{\pgfqpoint{1.617151in}{1.463309in}}{\pgfqpoint{1.621542in}{1.473908in}}{\pgfqpoint{1.621542in}{1.484958in}}%
\pgfpathcurveto{\pgfqpoint{1.621542in}{1.496008in}}{\pgfqpoint{1.617151in}{1.506607in}}{\pgfqpoint{1.609338in}{1.514421in}}%
\pgfpathcurveto{\pgfqpoint{1.601524in}{1.522235in}}{\pgfqpoint{1.590925in}{1.526625in}}{\pgfqpoint{1.579875in}{1.526625in}}%
\pgfpathcurveto{\pgfqpoint{1.568825in}{1.526625in}}{\pgfqpoint{1.558226in}{1.522235in}}{\pgfqpoint{1.550412in}{1.514421in}}%
\pgfpathcurveto{\pgfqpoint{1.542599in}{1.506607in}}{\pgfqpoint{1.538208in}{1.496008in}}{\pgfqpoint{1.538208in}{1.484958in}}%
\pgfpathcurveto{\pgfqpoint{1.538208in}{1.473908in}}{\pgfqpoint{1.542599in}{1.463309in}}{\pgfqpoint{1.550412in}{1.455495in}}%
\pgfpathcurveto{\pgfqpoint{1.558226in}{1.447682in}}{\pgfqpoint{1.568825in}{1.443291in}}{\pgfqpoint{1.579875in}{1.443291in}}%
\pgfpathclose%
\pgfusepath{stroke,fill}%
\end{pgfscope}%
\begin{pgfscope}%
\pgfpathrectangle{\pgfqpoint{0.375000in}{0.330000in}}{\pgfqpoint{2.325000in}{2.310000in}}%
\pgfusepath{clip}%
\pgfsetbuttcap%
\pgfsetroundjoin%
\definecolor{currentfill}{rgb}{0.000000,0.000000,0.000000}%
\pgfsetfillcolor{currentfill}%
\pgfsetlinewidth{1.003750pt}%
\definecolor{currentstroke}{rgb}{0.000000,0.000000,0.000000}%
\pgfsetstrokecolor{currentstroke}%
\pgfsetdash{}{0pt}%
\pgfpathmoveto{\pgfqpoint{1.579875in}{2.474583in}}%
\pgfpathcurveto{\pgfqpoint{1.590925in}{2.474583in}}{\pgfqpoint{1.601524in}{2.478974in}}{\pgfqpoint{1.609338in}{2.486787in}}%
\pgfpathcurveto{\pgfqpoint{1.617151in}{2.494601in}}{\pgfqpoint{1.621542in}{2.505200in}}{\pgfqpoint{1.621542in}{2.516250in}}%
\pgfpathcurveto{\pgfqpoint{1.621542in}{2.527300in}}{\pgfqpoint{1.617151in}{2.537899in}}{\pgfqpoint{1.609338in}{2.545713in}}%
\pgfpathcurveto{\pgfqpoint{1.601524in}{2.553526in}}{\pgfqpoint{1.590925in}{2.557917in}}{\pgfqpoint{1.579875in}{2.557917in}}%
\pgfpathcurveto{\pgfqpoint{1.568825in}{2.557917in}}{\pgfqpoint{1.558226in}{2.553526in}}{\pgfqpoint{1.550412in}{2.545713in}}%
\pgfpathcurveto{\pgfqpoint{1.542599in}{2.537899in}}{\pgfqpoint{1.538208in}{2.527300in}}{\pgfqpoint{1.538208in}{2.516250in}}%
\pgfpathcurveto{\pgfqpoint{1.538208in}{2.505200in}}{\pgfqpoint{1.542599in}{2.494601in}}{\pgfqpoint{1.550412in}{2.486787in}}%
\pgfpathcurveto{\pgfqpoint{1.558226in}{2.478974in}}{\pgfqpoint{1.568825in}{2.474583in}}{\pgfqpoint{1.579875in}{2.474583in}}%
\pgfpathclose%
\pgfusepath{stroke,fill}%
\end{pgfscope}%
\begin{pgfscope}%
\pgfpathrectangle{\pgfqpoint{0.375000in}{0.330000in}}{\pgfqpoint{2.325000in}{2.310000in}}%
\pgfusepath{clip}%
\pgfsetbuttcap%
\pgfsetroundjoin%
\definecolor{currentfill}{rgb}{0.000000,0.000000,0.000000}%
\pgfsetfillcolor{currentfill}%
\pgfsetlinewidth{1.003750pt}%
\definecolor{currentstroke}{rgb}{0.000000,0.000000,0.000000}%
\pgfsetstrokecolor{currentstroke}%
\pgfsetdash{}{0pt}%
\pgfpathmoveto{\pgfqpoint{1.579875in}{1.443291in}}%
\pgfpathcurveto{\pgfqpoint{1.590925in}{1.443291in}}{\pgfqpoint{1.601524in}{1.447682in}}{\pgfqpoint{1.609338in}{1.455495in}}%
\pgfpathcurveto{\pgfqpoint{1.617151in}{1.463309in}}{\pgfqpoint{1.621542in}{1.473908in}}{\pgfqpoint{1.621542in}{1.484958in}}%
\pgfpathcurveto{\pgfqpoint{1.621542in}{1.496008in}}{\pgfqpoint{1.617151in}{1.506607in}}{\pgfqpoint{1.609338in}{1.514421in}}%
\pgfpathcurveto{\pgfqpoint{1.601524in}{1.522235in}}{\pgfqpoint{1.590925in}{1.526625in}}{\pgfqpoint{1.579875in}{1.526625in}}%
\pgfpathcurveto{\pgfqpoint{1.568825in}{1.526625in}}{\pgfqpoint{1.558226in}{1.522235in}}{\pgfqpoint{1.550412in}{1.514421in}}%
\pgfpathcurveto{\pgfqpoint{1.542599in}{1.506607in}}{\pgfqpoint{1.538208in}{1.496008in}}{\pgfqpoint{1.538208in}{1.484958in}}%
\pgfpathcurveto{\pgfqpoint{1.538208in}{1.473908in}}{\pgfqpoint{1.542599in}{1.463309in}}{\pgfqpoint{1.550412in}{1.455495in}}%
\pgfpathcurveto{\pgfqpoint{1.558226in}{1.447682in}}{\pgfqpoint{1.568825in}{1.443291in}}{\pgfqpoint{1.579875in}{1.443291in}}%
\pgfpathclose%
\pgfusepath{stroke,fill}%
\end{pgfscope}%
\begin{pgfscope}%
\pgfpathrectangle{\pgfqpoint{0.375000in}{0.330000in}}{\pgfqpoint{2.325000in}{2.310000in}}%
\pgfusepath{clip}%
\pgfsetbuttcap%
\pgfsetroundjoin%
\definecolor{currentfill}{rgb}{0.000000,0.000000,0.000000}%
\pgfsetfillcolor{currentfill}%
\pgfsetlinewidth{1.003750pt}%
\definecolor{currentstroke}{rgb}{0.000000,0.000000,0.000000}%
\pgfsetstrokecolor{currentstroke}%
\pgfsetdash{}{0pt}%
\pgfpathmoveto{\pgfqpoint{1.579875in}{1.443291in}}%
\pgfpathcurveto{\pgfqpoint{1.590925in}{1.443291in}}{\pgfqpoint{1.601524in}{1.447682in}}{\pgfqpoint{1.609338in}{1.455495in}}%
\pgfpathcurveto{\pgfqpoint{1.617151in}{1.463309in}}{\pgfqpoint{1.621542in}{1.473908in}}{\pgfqpoint{1.621542in}{1.484958in}}%
\pgfpathcurveto{\pgfqpoint{1.621542in}{1.496008in}}{\pgfqpoint{1.617151in}{1.506607in}}{\pgfqpoint{1.609338in}{1.514421in}}%
\pgfpathcurveto{\pgfqpoint{1.601524in}{1.522235in}}{\pgfqpoint{1.590925in}{1.526625in}}{\pgfqpoint{1.579875in}{1.526625in}}%
\pgfpathcurveto{\pgfqpoint{1.568825in}{1.526625in}}{\pgfqpoint{1.558226in}{1.522235in}}{\pgfqpoint{1.550412in}{1.514421in}}%
\pgfpathcurveto{\pgfqpoint{1.542599in}{1.506607in}}{\pgfqpoint{1.538208in}{1.496008in}}{\pgfqpoint{1.538208in}{1.484958in}}%
\pgfpathcurveto{\pgfqpoint{1.538208in}{1.473908in}}{\pgfqpoint{1.542599in}{1.463309in}}{\pgfqpoint{1.550412in}{1.455495in}}%
\pgfpathcurveto{\pgfqpoint{1.558226in}{1.447682in}}{\pgfqpoint{1.568825in}{1.443291in}}{\pgfqpoint{1.579875in}{1.443291in}}%
\pgfpathclose%
\pgfusepath{stroke,fill}%
\end{pgfscope}%
\begin{pgfscope}%
\pgfpathrectangle{\pgfqpoint{0.375000in}{0.330000in}}{\pgfqpoint{2.325000in}{2.310000in}}%
\pgfusepath{clip}%
\pgfsetbuttcap%
\pgfsetroundjoin%
\definecolor{currentfill}{rgb}{0.000000,0.000000,0.000000}%
\pgfsetfillcolor{currentfill}%
\pgfsetlinewidth{1.003750pt}%
\definecolor{currentstroke}{rgb}{0.000000,0.000000,0.000000}%
\pgfsetstrokecolor{currentstroke}%
\pgfsetdash{}{0pt}%
\pgfpathmoveto{\pgfqpoint{1.579875in}{1.443291in}}%
\pgfpathcurveto{\pgfqpoint{1.590925in}{1.443291in}}{\pgfqpoint{1.601524in}{1.447682in}}{\pgfqpoint{1.609338in}{1.455495in}}%
\pgfpathcurveto{\pgfqpoint{1.617151in}{1.463309in}}{\pgfqpoint{1.621542in}{1.473908in}}{\pgfqpoint{1.621542in}{1.484958in}}%
\pgfpathcurveto{\pgfqpoint{1.621542in}{1.496008in}}{\pgfqpoint{1.617151in}{1.506607in}}{\pgfqpoint{1.609338in}{1.514421in}}%
\pgfpathcurveto{\pgfqpoint{1.601524in}{1.522235in}}{\pgfqpoint{1.590925in}{1.526625in}}{\pgfqpoint{1.579875in}{1.526625in}}%
\pgfpathcurveto{\pgfqpoint{1.568825in}{1.526625in}}{\pgfqpoint{1.558226in}{1.522235in}}{\pgfqpoint{1.550412in}{1.514421in}}%
\pgfpathcurveto{\pgfqpoint{1.542599in}{1.506607in}}{\pgfqpoint{1.538208in}{1.496008in}}{\pgfqpoint{1.538208in}{1.484958in}}%
\pgfpathcurveto{\pgfqpoint{1.538208in}{1.473908in}}{\pgfqpoint{1.542599in}{1.463309in}}{\pgfqpoint{1.550412in}{1.455495in}}%
\pgfpathcurveto{\pgfqpoint{1.558226in}{1.447682in}}{\pgfqpoint{1.568825in}{1.443291in}}{\pgfqpoint{1.579875in}{1.443291in}}%
\pgfpathclose%
\pgfusepath{stroke,fill}%
\end{pgfscope}%
\begin{pgfscope}%
\pgfpathrectangle{\pgfqpoint{0.375000in}{0.330000in}}{\pgfqpoint{2.325000in}{2.310000in}}%
\pgfusepath{clip}%
\pgfsetbuttcap%
\pgfsetroundjoin%
\definecolor{currentfill}{rgb}{0.000000,0.000000,0.000000}%
\pgfsetfillcolor{currentfill}%
\pgfsetlinewidth{1.003750pt}%
\definecolor{currentstroke}{rgb}{0.000000,0.000000,0.000000}%
\pgfsetstrokecolor{currentstroke}%
\pgfsetdash{}{0pt}%
\pgfpathmoveto{\pgfqpoint{1.579875in}{1.443291in}}%
\pgfpathcurveto{\pgfqpoint{1.590925in}{1.443291in}}{\pgfqpoint{1.601524in}{1.447682in}}{\pgfqpoint{1.609338in}{1.455495in}}%
\pgfpathcurveto{\pgfqpoint{1.617151in}{1.463309in}}{\pgfqpoint{1.621542in}{1.473908in}}{\pgfqpoint{1.621542in}{1.484958in}}%
\pgfpathcurveto{\pgfqpoint{1.621542in}{1.496008in}}{\pgfqpoint{1.617151in}{1.506607in}}{\pgfqpoint{1.609338in}{1.514421in}}%
\pgfpathcurveto{\pgfqpoint{1.601524in}{1.522235in}}{\pgfqpoint{1.590925in}{1.526625in}}{\pgfqpoint{1.579875in}{1.526625in}}%
\pgfpathcurveto{\pgfqpoint{1.568825in}{1.526625in}}{\pgfqpoint{1.558226in}{1.522235in}}{\pgfqpoint{1.550412in}{1.514421in}}%
\pgfpathcurveto{\pgfqpoint{1.542599in}{1.506607in}}{\pgfqpoint{1.538208in}{1.496008in}}{\pgfqpoint{1.538208in}{1.484958in}}%
\pgfpathcurveto{\pgfqpoint{1.538208in}{1.473908in}}{\pgfqpoint{1.542599in}{1.463309in}}{\pgfqpoint{1.550412in}{1.455495in}}%
\pgfpathcurveto{\pgfqpoint{1.558226in}{1.447682in}}{\pgfqpoint{1.568825in}{1.443291in}}{\pgfqpoint{1.579875in}{1.443291in}}%
\pgfpathclose%
\pgfusepath{stroke,fill}%
\end{pgfscope}%
\begin{pgfscope}%
\pgfpathrectangle{\pgfqpoint{0.375000in}{0.330000in}}{\pgfqpoint{2.325000in}{2.310000in}}%
\pgfusepath{clip}%
\pgfsetbuttcap%
\pgfsetroundjoin%
\definecolor{currentfill}{rgb}{0.000000,0.000000,0.000000}%
\pgfsetfillcolor{currentfill}%
\pgfsetlinewidth{1.003750pt}%
\definecolor{currentstroke}{rgb}{0.000000,0.000000,0.000000}%
\pgfsetstrokecolor{currentstroke}%
\pgfsetdash{}{0pt}%
\pgfpathmoveto{\pgfqpoint{1.579875in}{2.474583in}}%
\pgfpathcurveto{\pgfqpoint{1.590925in}{2.474583in}}{\pgfqpoint{1.601524in}{2.478974in}}{\pgfqpoint{1.609338in}{2.486787in}}%
\pgfpathcurveto{\pgfqpoint{1.617151in}{2.494601in}}{\pgfqpoint{1.621542in}{2.505200in}}{\pgfqpoint{1.621542in}{2.516250in}}%
\pgfpathcurveto{\pgfqpoint{1.621542in}{2.527300in}}{\pgfqpoint{1.617151in}{2.537899in}}{\pgfqpoint{1.609338in}{2.545713in}}%
\pgfpathcurveto{\pgfqpoint{1.601524in}{2.553526in}}{\pgfqpoint{1.590925in}{2.557917in}}{\pgfqpoint{1.579875in}{2.557917in}}%
\pgfpathcurveto{\pgfqpoint{1.568825in}{2.557917in}}{\pgfqpoint{1.558226in}{2.553526in}}{\pgfqpoint{1.550412in}{2.545713in}}%
\pgfpathcurveto{\pgfqpoint{1.542599in}{2.537899in}}{\pgfqpoint{1.538208in}{2.527300in}}{\pgfqpoint{1.538208in}{2.516250in}}%
\pgfpathcurveto{\pgfqpoint{1.538208in}{2.505200in}}{\pgfqpoint{1.542599in}{2.494601in}}{\pgfqpoint{1.550412in}{2.486787in}}%
\pgfpathcurveto{\pgfqpoint{1.558226in}{2.478974in}}{\pgfqpoint{1.568825in}{2.474583in}}{\pgfqpoint{1.579875in}{2.474583in}}%
\pgfpathclose%
\pgfusepath{stroke,fill}%
\end{pgfscope}%
\begin{pgfscope}%
\pgfpathrectangle{\pgfqpoint{0.375000in}{0.330000in}}{\pgfqpoint{2.325000in}{2.310000in}}%
\pgfusepath{clip}%
\pgfsetbuttcap%
\pgfsetroundjoin%
\definecolor{currentfill}{rgb}{0.000000,0.000000,0.000000}%
\pgfsetfillcolor{currentfill}%
\pgfsetlinewidth{1.003750pt}%
\definecolor{currentstroke}{rgb}{0.000000,0.000000,0.000000}%
\pgfsetstrokecolor{currentstroke}%
\pgfsetdash{}{0pt}%
\pgfpathmoveto{\pgfqpoint{1.579875in}{1.443291in}}%
\pgfpathcurveto{\pgfqpoint{1.590925in}{1.443291in}}{\pgfqpoint{1.601524in}{1.447682in}}{\pgfqpoint{1.609338in}{1.455495in}}%
\pgfpathcurveto{\pgfqpoint{1.617151in}{1.463309in}}{\pgfqpoint{1.621542in}{1.473908in}}{\pgfqpoint{1.621542in}{1.484958in}}%
\pgfpathcurveto{\pgfqpoint{1.621542in}{1.496008in}}{\pgfqpoint{1.617151in}{1.506607in}}{\pgfqpoint{1.609338in}{1.514421in}}%
\pgfpathcurveto{\pgfqpoint{1.601524in}{1.522235in}}{\pgfqpoint{1.590925in}{1.526625in}}{\pgfqpoint{1.579875in}{1.526625in}}%
\pgfpathcurveto{\pgfqpoint{1.568825in}{1.526625in}}{\pgfqpoint{1.558226in}{1.522235in}}{\pgfqpoint{1.550412in}{1.514421in}}%
\pgfpathcurveto{\pgfqpoint{1.542599in}{1.506607in}}{\pgfqpoint{1.538208in}{1.496008in}}{\pgfqpoint{1.538208in}{1.484958in}}%
\pgfpathcurveto{\pgfqpoint{1.538208in}{1.473908in}}{\pgfqpoint{1.542599in}{1.463309in}}{\pgfqpoint{1.550412in}{1.455495in}}%
\pgfpathcurveto{\pgfqpoint{1.558226in}{1.447682in}}{\pgfqpoint{1.568825in}{1.443291in}}{\pgfqpoint{1.579875in}{1.443291in}}%
\pgfpathclose%
\pgfusepath{stroke,fill}%
\end{pgfscope}%
\begin{pgfscope}%
\pgfpathrectangle{\pgfqpoint{0.375000in}{0.330000in}}{\pgfqpoint{2.325000in}{2.310000in}}%
\pgfusepath{clip}%
\pgfsetbuttcap%
\pgfsetroundjoin%
\definecolor{currentfill}{rgb}{0.000000,0.000000,0.000000}%
\pgfsetfillcolor{currentfill}%
\pgfsetlinewidth{1.003750pt}%
\definecolor{currentstroke}{rgb}{0.000000,0.000000,0.000000}%
\pgfsetstrokecolor{currentstroke}%
\pgfsetdash{}{0pt}%
\pgfpathmoveto{\pgfqpoint{1.579875in}{1.443291in}}%
\pgfpathcurveto{\pgfqpoint{1.590925in}{1.443291in}}{\pgfqpoint{1.601524in}{1.447682in}}{\pgfqpoint{1.609338in}{1.455495in}}%
\pgfpathcurveto{\pgfqpoint{1.617151in}{1.463309in}}{\pgfqpoint{1.621542in}{1.473908in}}{\pgfqpoint{1.621542in}{1.484958in}}%
\pgfpathcurveto{\pgfqpoint{1.621542in}{1.496008in}}{\pgfqpoint{1.617151in}{1.506607in}}{\pgfqpoint{1.609338in}{1.514421in}}%
\pgfpathcurveto{\pgfqpoint{1.601524in}{1.522235in}}{\pgfqpoint{1.590925in}{1.526625in}}{\pgfqpoint{1.579875in}{1.526625in}}%
\pgfpathcurveto{\pgfqpoint{1.568825in}{1.526625in}}{\pgfqpoint{1.558226in}{1.522235in}}{\pgfqpoint{1.550412in}{1.514421in}}%
\pgfpathcurveto{\pgfqpoint{1.542599in}{1.506607in}}{\pgfqpoint{1.538208in}{1.496008in}}{\pgfqpoint{1.538208in}{1.484958in}}%
\pgfpathcurveto{\pgfqpoint{1.538208in}{1.473908in}}{\pgfqpoint{1.542599in}{1.463309in}}{\pgfqpoint{1.550412in}{1.455495in}}%
\pgfpathcurveto{\pgfqpoint{1.558226in}{1.447682in}}{\pgfqpoint{1.568825in}{1.443291in}}{\pgfqpoint{1.579875in}{1.443291in}}%
\pgfpathclose%
\pgfusepath{stroke,fill}%
\end{pgfscope}%
\begin{pgfscope}%
\pgfpathrectangle{\pgfqpoint{0.375000in}{0.330000in}}{\pgfqpoint{2.325000in}{2.310000in}}%
\pgfusepath{clip}%
\pgfsetbuttcap%
\pgfsetroundjoin%
\definecolor{currentfill}{rgb}{0.000000,0.000000,0.000000}%
\pgfsetfillcolor{currentfill}%
\pgfsetlinewidth{1.003750pt}%
\definecolor{currentstroke}{rgb}{0.000000,0.000000,0.000000}%
\pgfsetstrokecolor{currentstroke}%
\pgfsetdash{}{0pt}%
\pgfpathmoveto{\pgfqpoint{1.579875in}{1.443291in}}%
\pgfpathcurveto{\pgfqpoint{1.590925in}{1.443291in}}{\pgfqpoint{1.601524in}{1.447682in}}{\pgfqpoint{1.609338in}{1.455495in}}%
\pgfpathcurveto{\pgfqpoint{1.617151in}{1.463309in}}{\pgfqpoint{1.621542in}{1.473908in}}{\pgfqpoint{1.621542in}{1.484958in}}%
\pgfpathcurveto{\pgfqpoint{1.621542in}{1.496008in}}{\pgfqpoint{1.617151in}{1.506607in}}{\pgfqpoint{1.609338in}{1.514421in}}%
\pgfpathcurveto{\pgfqpoint{1.601524in}{1.522235in}}{\pgfqpoint{1.590925in}{1.526625in}}{\pgfqpoint{1.579875in}{1.526625in}}%
\pgfpathcurveto{\pgfqpoint{1.568825in}{1.526625in}}{\pgfqpoint{1.558226in}{1.522235in}}{\pgfqpoint{1.550412in}{1.514421in}}%
\pgfpathcurveto{\pgfqpoint{1.542599in}{1.506607in}}{\pgfqpoint{1.538208in}{1.496008in}}{\pgfqpoint{1.538208in}{1.484958in}}%
\pgfpathcurveto{\pgfqpoint{1.538208in}{1.473908in}}{\pgfqpoint{1.542599in}{1.463309in}}{\pgfqpoint{1.550412in}{1.455495in}}%
\pgfpathcurveto{\pgfqpoint{1.558226in}{1.447682in}}{\pgfqpoint{1.568825in}{1.443291in}}{\pgfqpoint{1.579875in}{1.443291in}}%
\pgfpathclose%
\pgfusepath{stroke,fill}%
\end{pgfscope}%
\begin{pgfscope}%
\pgfpathrectangle{\pgfqpoint{0.375000in}{0.330000in}}{\pgfqpoint{2.325000in}{2.310000in}}%
\pgfusepath{clip}%
\pgfsetbuttcap%
\pgfsetroundjoin%
\definecolor{currentfill}{rgb}{0.000000,0.000000,0.000000}%
\pgfsetfillcolor{currentfill}%
\pgfsetlinewidth{1.003750pt}%
\definecolor{currentstroke}{rgb}{0.000000,0.000000,0.000000}%
\pgfsetstrokecolor{currentstroke}%
\pgfsetdash{}{0pt}%
\pgfpathmoveto{\pgfqpoint{1.579875in}{1.443291in}}%
\pgfpathcurveto{\pgfqpoint{1.590925in}{1.443291in}}{\pgfqpoint{1.601524in}{1.447682in}}{\pgfqpoint{1.609338in}{1.455495in}}%
\pgfpathcurveto{\pgfqpoint{1.617151in}{1.463309in}}{\pgfqpoint{1.621542in}{1.473908in}}{\pgfqpoint{1.621542in}{1.484958in}}%
\pgfpathcurveto{\pgfqpoint{1.621542in}{1.496008in}}{\pgfqpoint{1.617151in}{1.506607in}}{\pgfqpoint{1.609338in}{1.514421in}}%
\pgfpathcurveto{\pgfqpoint{1.601524in}{1.522235in}}{\pgfqpoint{1.590925in}{1.526625in}}{\pgfqpoint{1.579875in}{1.526625in}}%
\pgfpathcurveto{\pgfqpoint{1.568825in}{1.526625in}}{\pgfqpoint{1.558226in}{1.522235in}}{\pgfqpoint{1.550412in}{1.514421in}}%
\pgfpathcurveto{\pgfqpoint{1.542599in}{1.506607in}}{\pgfqpoint{1.538208in}{1.496008in}}{\pgfqpoint{1.538208in}{1.484958in}}%
\pgfpathcurveto{\pgfqpoint{1.538208in}{1.473908in}}{\pgfqpoint{1.542599in}{1.463309in}}{\pgfqpoint{1.550412in}{1.455495in}}%
\pgfpathcurveto{\pgfqpoint{1.558226in}{1.447682in}}{\pgfqpoint{1.568825in}{1.443291in}}{\pgfqpoint{1.579875in}{1.443291in}}%
\pgfpathclose%
\pgfusepath{stroke,fill}%
\end{pgfscope}%
\begin{pgfscope}%
\pgfpathrectangle{\pgfqpoint{0.375000in}{0.330000in}}{\pgfqpoint{2.325000in}{2.310000in}}%
\pgfusepath{clip}%
\pgfsetbuttcap%
\pgfsetroundjoin%
\definecolor{currentfill}{rgb}{0.000000,0.000000,0.000000}%
\pgfsetfillcolor{currentfill}%
\pgfsetlinewidth{1.003750pt}%
\definecolor{currentstroke}{rgb}{0.000000,0.000000,0.000000}%
\pgfsetstrokecolor{currentstroke}%
\pgfsetdash{}{0pt}%
\pgfpathmoveto{\pgfqpoint{1.579875in}{1.443291in}}%
\pgfpathcurveto{\pgfqpoint{1.590925in}{1.443291in}}{\pgfqpoint{1.601524in}{1.447682in}}{\pgfqpoint{1.609338in}{1.455495in}}%
\pgfpathcurveto{\pgfqpoint{1.617151in}{1.463309in}}{\pgfqpoint{1.621542in}{1.473908in}}{\pgfqpoint{1.621542in}{1.484958in}}%
\pgfpathcurveto{\pgfqpoint{1.621542in}{1.496008in}}{\pgfqpoint{1.617151in}{1.506607in}}{\pgfqpoint{1.609338in}{1.514421in}}%
\pgfpathcurveto{\pgfqpoint{1.601524in}{1.522235in}}{\pgfqpoint{1.590925in}{1.526625in}}{\pgfqpoint{1.579875in}{1.526625in}}%
\pgfpathcurveto{\pgfqpoint{1.568825in}{1.526625in}}{\pgfqpoint{1.558226in}{1.522235in}}{\pgfqpoint{1.550412in}{1.514421in}}%
\pgfpathcurveto{\pgfqpoint{1.542599in}{1.506607in}}{\pgfqpoint{1.538208in}{1.496008in}}{\pgfqpoint{1.538208in}{1.484958in}}%
\pgfpathcurveto{\pgfqpoint{1.538208in}{1.473908in}}{\pgfqpoint{1.542599in}{1.463309in}}{\pgfqpoint{1.550412in}{1.455495in}}%
\pgfpathcurveto{\pgfqpoint{1.558226in}{1.447682in}}{\pgfqpoint{1.568825in}{1.443291in}}{\pgfqpoint{1.579875in}{1.443291in}}%
\pgfpathclose%
\pgfusepath{stroke,fill}%
\end{pgfscope}%
\begin{pgfscope}%
\pgfpathrectangle{\pgfqpoint{0.375000in}{0.330000in}}{\pgfqpoint{2.325000in}{2.310000in}}%
\pgfusepath{clip}%
\pgfsetbuttcap%
\pgfsetroundjoin%
\definecolor{currentfill}{rgb}{0.000000,0.000000,0.000000}%
\pgfsetfillcolor{currentfill}%
\pgfsetlinewidth{1.003750pt}%
\definecolor{currentstroke}{rgb}{0.000000,0.000000,0.000000}%
\pgfsetstrokecolor{currentstroke}%
\pgfsetdash{}{0pt}%
\pgfpathmoveto{\pgfqpoint{1.579875in}{1.443291in}}%
\pgfpathcurveto{\pgfqpoint{1.590925in}{1.443291in}}{\pgfqpoint{1.601524in}{1.447682in}}{\pgfqpoint{1.609338in}{1.455495in}}%
\pgfpathcurveto{\pgfqpoint{1.617151in}{1.463309in}}{\pgfqpoint{1.621542in}{1.473908in}}{\pgfqpoint{1.621542in}{1.484958in}}%
\pgfpathcurveto{\pgfqpoint{1.621542in}{1.496008in}}{\pgfqpoint{1.617151in}{1.506607in}}{\pgfqpoint{1.609338in}{1.514421in}}%
\pgfpathcurveto{\pgfqpoint{1.601524in}{1.522235in}}{\pgfqpoint{1.590925in}{1.526625in}}{\pgfqpoint{1.579875in}{1.526625in}}%
\pgfpathcurveto{\pgfqpoint{1.568825in}{1.526625in}}{\pgfqpoint{1.558226in}{1.522235in}}{\pgfqpoint{1.550412in}{1.514421in}}%
\pgfpathcurveto{\pgfqpoint{1.542599in}{1.506607in}}{\pgfqpoint{1.538208in}{1.496008in}}{\pgfqpoint{1.538208in}{1.484958in}}%
\pgfpathcurveto{\pgfqpoint{1.538208in}{1.473908in}}{\pgfqpoint{1.542599in}{1.463309in}}{\pgfqpoint{1.550412in}{1.455495in}}%
\pgfpathcurveto{\pgfqpoint{1.558226in}{1.447682in}}{\pgfqpoint{1.568825in}{1.443291in}}{\pgfqpoint{1.579875in}{1.443291in}}%
\pgfpathclose%
\pgfusepath{stroke,fill}%
\end{pgfscope}%
\begin{pgfscope}%
\pgfpathrectangle{\pgfqpoint{0.375000in}{0.330000in}}{\pgfqpoint{2.325000in}{2.310000in}}%
\pgfusepath{clip}%
\pgfsetbuttcap%
\pgfsetroundjoin%
\definecolor{currentfill}{rgb}{0.000000,0.000000,0.000000}%
\pgfsetfillcolor{currentfill}%
\pgfsetlinewidth{1.003750pt}%
\definecolor{currentstroke}{rgb}{0.000000,0.000000,0.000000}%
\pgfsetstrokecolor{currentstroke}%
\pgfsetdash{}{0pt}%
\pgfpathmoveto{\pgfqpoint{2.139937in}{1.443291in}}%
\pgfpathcurveto{\pgfqpoint{2.150988in}{1.443291in}}{\pgfqpoint{2.161587in}{1.447682in}}{\pgfqpoint{2.169400in}{1.455495in}}%
\pgfpathcurveto{\pgfqpoint{2.177214in}{1.463309in}}{\pgfqpoint{2.181604in}{1.473908in}}{\pgfqpoint{2.181604in}{1.484958in}}%
\pgfpathcurveto{\pgfqpoint{2.181604in}{1.496008in}}{\pgfqpoint{2.177214in}{1.506607in}}{\pgfqpoint{2.169400in}{1.514421in}}%
\pgfpathcurveto{\pgfqpoint{2.161587in}{1.522235in}}{\pgfqpoint{2.150988in}{1.526625in}}{\pgfqpoint{2.139937in}{1.526625in}}%
\pgfpathcurveto{\pgfqpoint{2.128887in}{1.526625in}}{\pgfqpoint{2.118288in}{1.522235in}}{\pgfqpoint{2.110475in}{1.514421in}}%
\pgfpathcurveto{\pgfqpoint{2.102661in}{1.506607in}}{\pgfqpoint{2.098271in}{1.496008in}}{\pgfqpoint{2.098271in}{1.484958in}}%
\pgfpathcurveto{\pgfqpoint{2.098271in}{1.473908in}}{\pgfqpoint{2.102661in}{1.463309in}}{\pgfqpoint{2.110475in}{1.455495in}}%
\pgfpathcurveto{\pgfqpoint{2.118288in}{1.447682in}}{\pgfqpoint{2.128887in}{1.443291in}}{\pgfqpoint{2.139937in}{1.443291in}}%
\pgfpathclose%
\pgfusepath{stroke,fill}%
\end{pgfscope}%
\begin{pgfscope}%
\pgfpathrectangle{\pgfqpoint{0.375000in}{0.330000in}}{\pgfqpoint{2.325000in}{2.310000in}}%
\pgfusepath{clip}%
\pgfsetbuttcap%
\pgfsetroundjoin%
\definecolor{currentfill}{rgb}{0.000000,0.000000,0.000000}%
\pgfsetfillcolor{currentfill}%
\pgfsetlinewidth{1.003750pt}%
\definecolor{currentstroke}{rgb}{0.000000,0.000000,0.000000}%
\pgfsetstrokecolor{currentstroke}%
\pgfsetdash{}{0pt}%
\pgfpathmoveto{\pgfqpoint{2.139937in}{2.474583in}}%
\pgfpathcurveto{\pgfqpoint{2.150988in}{2.474583in}}{\pgfqpoint{2.161587in}{2.478974in}}{\pgfqpoint{2.169400in}{2.486787in}}%
\pgfpathcurveto{\pgfqpoint{2.177214in}{2.494601in}}{\pgfqpoint{2.181604in}{2.505200in}}{\pgfqpoint{2.181604in}{2.516250in}}%
\pgfpathcurveto{\pgfqpoint{2.181604in}{2.527300in}}{\pgfqpoint{2.177214in}{2.537899in}}{\pgfqpoint{2.169400in}{2.545713in}}%
\pgfpathcurveto{\pgfqpoint{2.161587in}{2.553526in}}{\pgfqpoint{2.150988in}{2.557917in}}{\pgfqpoint{2.139937in}{2.557917in}}%
\pgfpathcurveto{\pgfqpoint{2.128887in}{2.557917in}}{\pgfqpoint{2.118288in}{2.553526in}}{\pgfqpoint{2.110475in}{2.545713in}}%
\pgfpathcurveto{\pgfqpoint{2.102661in}{2.537899in}}{\pgfqpoint{2.098271in}{2.527300in}}{\pgfqpoint{2.098271in}{2.516250in}}%
\pgfpathcurveto{\pgfqpoint{2.098271in}{2.505200in}}{\pgfqpoint{2.102661in}{2.494601in}}{\pgfqpoint{2.110475in}{2.486787in}}%
\pgfpathcurveto{\pgfqpoint{2.118288in}{2.478974in}}{\pgfqpoint{2.128887in}{2.474583in}}{\pgfqpoint{2.139937in}{2.474583in}}%
\pgfpathclose%
\pgfusepath{stroke,fill}%
\end{pgfscope}%
\begin{pgfscope}%
\pgfpathrectangle{\pgfqpoint{0.375000in}{0.330000in}}{\pgfqpoint{2.325000in}{2.310000in}}%
\pgfusepath{clip}%
\pgfsetbuttcap%
\pgfsetroundjoin%
\definecolor{currentfill}{rgb}{0.000000,0.000000,0.000000}%
\pgfsetfillcolor{currentfill}%
\pgfsetlinewidth{1.003750pt}%
\definecolor{currentstroke}{rgb}{0.000000,0.000000,0.000000}%
\pgfsetstrokecolor{currentstroke}%
\pgfsetdash{}{0pt}%
\pgfpathmoveto{\pgfqpoint{2.139937in}{1.443291in}}%
\pgfpathcurveto{\pgfqpoint{2.150988in}{1.443291in}}{\pgfqpoint{2.161587in}{1.447682in}}{\pgfqpoint{2.169400in}{1.455495in}}%
\pgfpathcurveto{\pgfqpoint{2.177214in}{1.463309in}}{\pgfqpoint{2.181604in}{1.473908in}}{\pgfqpoint{2.181604in}{1.484958in}}%
\pgfpathcurveto{\pgfqpoint{2.181604in}{1.496008in}}{\pgfqpoint{2.177214in}{1.506607in}}{\pgfqpoint{2.169400in}{1.514421in}}%
\pgfpathcurveto{\pgfqpoint{2.161587in}{1.522235in}}{\pgfqpoint{2.150988in}{1.526625in}}{\pgfqpoint{2.139937in}{1.526625in}}%
\pgfpathcurveto{\pgfqpoint{2.128887in}{1.526625in}}{\pgfqpoint{2.118288in}{1.522235in}}{\pgfqpoint{2.110475in}{1.514421in}}%
\pgfpathcurveto{\pgfqpoint{2.102661in}{1.506607in}}{\pgfqpoint{2.098271in}{1.496008in}}{\pgfqpoint{2.098271in}{1.484958in}}%
\pgfpathcurveto{\pgfqpoint{2.098271in}{1.473908in}}{\pgfqpoint{2.102661in}{1.463309in}}{\pgfqpoint{2.110475in}{1.455495in}}%
\pgfpathcurveto{\pgfqpoint{2.118288in}{1.447682in}}{\pgfqpoint{2.128887in}{1.443291in}}{\pgfqpoint{2.139937in}{1.443291in}}%
\pgfpathclose%
\pgfusepath{stroke,fill}%
\end{pgfscope}%
\begin{pgfscope}%
\pgfpathrectangle{\pgfqpoint{0.375000in}{0.330000in}}{\pgfqpoint{2.325000in}{2.310000in}}%
\pgfusepath{clip}%
\pgfsetbuttcap%
\pgfsetroundjoin%
\definecolor{currentfill}{rgb}{0.000000,0.000000,0.000000}%
\pgfsetfillcolor{currentfill}%
\pgfsetlinewidth{1.003750pt}%
\definecolor{currentstroke}{rgb}{0.000000,0.000000,0.000000}%
\pgfsetstrokecolor{currentstroke}%
\pgfsetdash{}{0pt}%
\pgfpathmoveto{\pgfqpoint{2.139937in}{2.474583in}}%
\pgfpathcurveto{\pgfqpoint{2.150988in}{2.474583in}}{\pgfqpoint{2.161587in}{2.478974in}}{\pgfqpoint{2.169400in}{2.486787in}}%
\pgfpathcurveto{\pgfqpoint{2.177214in}{2.494601in}}{\pgfqpoint{2.181604in}{2.505200in}}{\pgfqpoint{2.181604in}{2.516250in}}%
\pgfpathcurveto{\pgfqpoint{2.181604in}{2.527300in}}{\pgfqpoint{2.177214in}{2.537899in}}{\pgfqpoint{2.169400in}{2.545713in}}%
\pgfpathcurveto{\pgfqpoint{2.161587in}{2.553526in}}{\pgfqpoint{2.150988in}{2.557917in}}{\pgfqpoint{2.139937in}{2.557917in}}%
\pgfpathcurveto{\pgfqpoint{2.128887in}{2.557917in}}{\pgfqpoint{2.118288in}{2.553526in}}{\pgfqpoint{2.110475in}{2.545713in}}%
\pgfpathcurveto{\pgfqpoint{2.102661in}{2.537899in}}{\pgfqpoint{2.098271in}{2.527300in}}{\pgfqpoint{2.098271in}{2.516250in}}%
\pgfpathcurveto{\pgfqpoint{2.098271in}{2.505200in}}{\pgfqpoint{2.102661in}{2.494601in}}{\pgfqpoint{2.110475in}{2.486787in}}%
\pgfpathcurveto{\pgfqpoint{2.118288in}{2.478974in}}{\pgfqpoint{2.128887in}{2.474583in}}{\pgfqpoint{2.139937in}{2.474583in}}%
\pgfpathclose%
\pgfusepath{stroke,fill}%
\end{pgfscope}%
\begin{pgfscope}%
\pgfpathrectangle{\pgfqpoint{0.375000in}{0.330000in}}{\pgfqpoint{2.325000in}{2.310000in}}%
\pgfusepath{clip}%
\pgfsetbuttcap%
\pgfsetroundjoin%
\definecolor{currentfill}{rgb}{0.000000,0.000000,0.000000}%
\pgfsetfillcolor{currentfill}%
\pgfsetlinewidth{1.003750pt}%
\definecolor{currentstroke}{rgb}{0.000000,0.000000,0.000000}%
\pgfsetstrokecolor{currentstroke}%
\pgfsetdash{}{0pt}%
\pgfpathmoveto{\pgfqpoint{2.139937in}{1.443291in}}%
\pgfpathcurveto{\pgfqpoint{2.150988in}{1.443291in}}{\pgfqpoint{2.161587in}{1.447682in}}{\pgfqpoint{2.169400in}{1.455495in}}%
\pgfpathcurveto{\pgfqpoint{2.177214in}{1.463309in}}{\pgfqpoint{2.181604in}{1.473908in}}{\pgfqpoint{2.181604in}{1.484958in}}%
\pgfpathcurveto{\pgfqpoint{2.181604in}{1.496008in}}{\pgfqpoint{2.177214in}{1.506607in}}{\pgfqpoint{2.169400in}{1.514421in}}%
\pgfpathcurveto{\pgfqpoint{2.161587in}{1.522235in}}{\pgfqpoint{2.150988in}{1.526625in}}{\pgfqpoint{2.139937in}{1.526625in}}%
\pgfpathcurveto{\pgfqpoint{2.128887in}{1.526625in}}{\pgfqpoint{2.118288in}{1.522235in}}{\pgfqpoint{2.110475in}{1.514421in}}%
\pgfpathcurveto{\pgfqpoint{2.102661in}{1.506607in}}{\pgfqpoint{2.098271in}{1.496008in}}{\pgfqpoint{2.098271in}{1.484958in}}%
\pgfpathcurveto{\pgfqpoint{2.098271in}{1.473908in}}{\pgfqpoint{2.102661in}{1.463309in}}{\pgfqpoint{2.110475in}{1.455495in}}%
\pgfpathcurveto{\pgfqpoint{2.118288in}{1.447682in}}{\pgfqpoint{2.128887in}{1.443291in}}{\pgfqpoint{2.139937in}{1.443291in}}%
\pgfpathclose%
\pgfusepath{stroke,fill}%
\end{pgfscope}%
\begin{pgfscope}%
\pgfpathrectangle{\pgfqpoint{0.375000in}{0.330000in}}{\pgfqpoint{2.325000in}{2.310000in}}%
\pgfusepath{clip}%
\pgfsetbuttcap%
\pgfsetroundjoin%
\definecolor{currentfill}{rgb}{0.000000,0.000000,0.000000}%
\pgfsetfillcolor{currentfill}%
\pgfsetlinewidth{1.003750pt}%
\definecolor{currentstroke}{rgb}{0.000000,0.000000,0.000000}%
\pgfsetstrokecolor{currentstroke}%
\pgfsetdash{}{0pt}%
\pgfpathmoveto{\pgfqpoint{2.139937in}{1.443291in}}%
\pgfpathcurveto{\pgfqpoint{2.150988in}{1.443291in}}{\pgfqpoint{2.161587in}{1.447682in}}{\pgfqpoint{2.169400in}{1.455495in}}%
\pgfpathcurveto{\pgfqpoint{2.177214in}{1.463309in}}{\pgfqpoint{2.181604in}{1.473908in}}{\pgfqpoint{2.181604in}{1.484958in}}%
\pgfpathcurveto{\pgfqpoint{2.181604in}{1.496008in}}{\pgfqpoint{2.177214in}{1.506607in}}{\pgfqpoint{2.169400in}{1.514421in}}%
\pgfpathcurveto{\pgfqpoint{2.161587in}{1.522235in}}{\pgfqpoint{2.150988in}{1.526625in}}{\pgfqpoint{2.139937in}{1.526625in}}%
\pgfpathcurveto{\pgfqpoint{2.128887in}{1.526625in}}{\pgfqpoint{2.118288in}{1.522235in}}{\pgfqpoint{2.110475in}{1.514421in}}%
\pgfpathcurveto{\pgfqpoint{2.102661in}{1.506607in}}{\pgfqpoint{2.098271in}{1.496008in}}{\pgfqpoint{2.098271in}{1.484958in}}%
\pgfpathcurveto{\pgfqpoint{2.098271in}{1.473908in}}{\pgfqpoint{2.102661in}{1.463309in}}{\pgfqpoint{2.110475in}{1.455495in}}%
\pgfpathcurveto{\pgfqpoint{2.118288in}{1.447682in}}{\pgfqpoint{2.128887in}{1.443291in}}{\pgfqpoint{2.139937in}{1.443291in}}%
\pgfpathclose%
\pgfusepath{stroke,fill}%
\end{pgfscope}%
\begin{pgfscope}%
\pgfpathrectangle{\pgfqpoint{0.375000in}{0.330000in}}{\pgfqpoint{2.325000in}{2.310000in}}%
\pgfusepath{clip}%
\pgfsetbuttcap%
\pgfsetroundjoin%
\definecolor{currentfill}{rgb}{0.000000,0.000000,0.000000}%
\pgfsetfillcolor{currentfill}%
\pgfsetlinewidth{1.003750pt}%
\definecolor{currentstroke}{rgb}{0.000000,0.000000,0.000000}%
\pgfsetstrokecolor{currentstroke}%
\pgfsetdash{}{0pt}%
\pgfpathmoveto{\pgfqpoint{2.139937in}{2.474583in}}%
\pgfpathcurveto{\pgfqpoint{2.150988in}{2.474583in}}{\pgfqpoint{2.161587in}{2.478974in}}{\pgfqpoint{2.169400in}{2.486787in}}%
\pgfpathcurveto{\pgfqpoint{2.177214in}{2.494601in}}{\pgfqpoint{2.181604in}{2.505200in}}{\pgfqpoint{2.181604in}{2.516250in}}%
\pgfpathcurveto{\pgfqpoint{2.181604in}{2.527300in}}{\pgfqpoint{2.177214in}{2.537899in}}{\pgfqpoint{2.169400in}{2.545713in}}%
\pgfpathcurveto{\pgfqpoint{2.161587in}{2.553526in}}{\pgfqpoint{2.150988in}{2.557917in}}{\pgfqpoint{2.139937in}{2.557917in}}%
\pgfpathcurveto{\pgfqpoint{2.128887in}{2.557917in}}{\pgfqpoint{2.118288in}{2.553526in}}{\pgfqpoint{2.110475in}{2.545713in}}%
\pgfpathcurveto{\pgfqpoint{2.102661in}{2.537899in}}{\pgfqpoint{2.098271in}{2.527300in}}{\pgfqpoint{2.098271in}{2.516250in}}%
\pgfpathcurveto{\pgfqpoint{2.098271in}{2.505200in}}{\pgfqpoint{2.102661in}{2.494601in}}{\pgfqpoint{2.110475in}{2.486787in}}%
\pgfpathcurveto{\pgfqpoint{2.118288in}{2.478974in}}{\pgfqpoint{2.128887in}{2.474583in}}{\pgfqpoint{2.139937in}{2.474583in}}%
\pgfpathclose%
\pgfusepath{stroke,fill}%
\end{pgfscope}%
\begin{pgfscope}%
\pgfpathrectangle{\pgfqpoint{0.375000in}{0.330000in}}{\pgfqpoint{2.325000in}{2.310000in}}%
\pgfusepath{clip}%
\pgfsetbuttcap%
\pgfsetroundjoin%
\definecolor{currentfill}{rgb}{0.000000,0.000000,0.000000}%
\pgfsetfillcolor{currentfill}%
\pgfsetlinewidth{1.003750pt}%
\definecolor{currentstroke}{rgb}{0.000000,0.000000,0.000000}%
\pgfsetstrokecolor{currentstroke}%
\pgfsetdash{}{0pt}%
\pgfpathmoveto{\pgfqpoint{2.139937in}{2.474583in}}%
\pgfpathcurveto{\pgfqpoint{2.150988in}{2.474583in}}{\pgfqpoint{2.161587in}{2.478974in}}{\pgfqpoint{2.169400in}{2.486787in}}%
\pgfpathcurveto{\pgfqpoint{2.177214in}{2.494601in}}{\pgfqpoint{2.181604in}{2.505200in}}{\pgfqpoint{2.181604in}{2.516250in}}%
\pgfpathcurveto{\pgfqpoint{2.181604in}{2.527300in}}{\pgfqpoint{2.177214in}{2.537899in}}{\pgfqpoint{2.169400in}{2.545713in}}%
\pgfpathcurveto{\pgfqpoint{2.161587in}{2.553526in}}{\pgfqpoint{2.150988in}{2.557917in}}{\pgfqpoint{2.139937in}{2.557917in}}%
\pgfpathcurveto{\pgfqpoint{2.128887in}{2.557917in}}{\pgfqpoint{2.118288in}{2.553526in}}{\pgfqpoint{2.110475in}{2.545713in}}%
\pgfpathcurveto{\pgfqpoint{2.102661in}{2.537899in}}{\pgfqpoint{2.098271in}{2.527300in}}{\pgfqpoint{2.098271in}{2.516250in}}%
\pgfpathcurveto{\pgfqpoint{2.098271in}{2.505200in}}{\pgfqpoint{2.102661in}{2.494601in}}{\pgfqpoint{2.110475in}{2.486787in}}%
\pgfpathcurveto{\pgfqpoint{2.118288in}{2.478974in}}{\pgfqpoint{2.128887in}{2.474583in}}{\pgfqpoint{2.139937in}{2.474583in}}%
\pgfpathclose%
\pgfusepath{stroke,fill}%
\end{pgfscope}%
\begin{pgfscope}%
\pgfpathrectangle{\pgfqpoint{0.375000in}{0.330000in}}{\pgfqpoint{2.325000in}{2.310000in}}%
\pgfusepath{clip}%
\pgfsetbuttcap%
\pgfsetroundjoin%
\definecolor{currentfill}{rgb}{0.000000,0.000000,0.000000}%
\pgfsetfillcolor{currentfill}%
\pgfsetlinewidth{1.003750pt}%
\definecolor{currentstroke}{rgb}{0.000000,0.000000,0.000000}%
\pgfsetstrokecolor{currentstroke}%
\pgfsetdash{}{0pt}%
\pgfpathmoveto{\pgfqpoint{2.139937in}{1.443291in}}%
\pgfpathcurveto{\pgfqpoint{2.150988in}{1.443291in}}{\pgfqpoint{2.161587in}{1.447682in}}{\pgfqpoint{2.169400in}{1.455495in}}%
\pgfpathcurveto{\pgfqpoint{2.177214in}{1.463309in}}{\pgfqpoint{2.181604in}{1.473908in}}{\pgfqpoint{2.181604in}{1.484958in}}%
\pgfpathcurveto{\pgfqpoint{2.181604in}{1.496008in}}{\pgfqpoint{2.177214in}{1.506607in}}{\pgfqpoint{2.169400in}{1.514421in}}%
\pgfpathcurveto{\pgfqpoint{2.161587in}{1.522235in}}{\pgfqpoint{2.150988in}{1.526625in}}{\pgfqpoint{2.139937in}{1.526625in}}%
\pgfpathcurveto{\pgfqpoint{2.128887in}{1.526625in}}{\pgfqpoint{2.118288in}{1.522235in}}{\pgfqpoint{2.110475in}{1.514421in}}%
\pgfpathcurveto{\pgfqpoint{2.102661in}{1.506607in}}{\pgfqpoint{2.098271in}{1.496008in}}{\pgfqpoint{2.098271in}{1.484958in}}%
\pgfpathcurveto{\pgfqpoint{2.098271in}{1.473908in}}{\pgfqpoint{2.102661in}{1.463309in}}{\pgfqpoint{2.110475in}{1.455495in}}%
\pgfpathcurveto{\pgfqpoint{2.118288in}{1.447682in}}{\pgfqpoint{2.128887in}{1.443291in}}{\pgfqpoint{2.139937in}{1.443291in}}%
\pgfpathclose%
\pgfusepath{stroke,fill}%
\end{pgfscope}%
\begin{pgfscope}%
\pgfpathrectangle{\pgfqpoint{0.375000in}{0.330000in}}{\pgfqpoint{2.325000in}{2.310000in}}%
\pgfusepath{clip}%
\pgfsetbuttcap%
\pgfsetroundjoin%
\definecolor{currentfill}{rgb}{0.000000,0.000000,0.000000}%
\pgfsetfillcolor{currentfill}%
\pgfsetlinewidth{1.003750pt}%
\definecolor{currentstroke}{rgb}{0.000000,0.000000,0.000000}%
\pgfsetstrokecolor{currentstroke}%
\pgfsetdash{}{0pt}%
\pgfpathmoveto{\pgfqpoint{2.139937in}{1.443291in}}%
\pgfpathcurveto{\pgfqpoint{2.150988in}{1.443291in}}{\pgfqpoint{2.161587in}{1.447682in}}{\pgfqpoint{2.169400in}{1.455495in}}%
\pgfpathcurveto{\pgfqpoint{2.177214in}{1.463309in}}{\pgfqpoint{2.181604in}{1.473908in}}{\pgfqpoint{2.181604in}{1.484958in}}%
\pgfpathcurveto{\pgfqpoint{2.181604in}{1.496008in}}{\pgfqpoint{2.177214in}{1.506607in}}{\pgfqpoint{2.169400in}{1.514421in}}%
\pgfpathcurveto{\pgfqpoint{2.161587in}{1.522235in}}{\pgfqpoint{2.150988in}{1.526625in}}{\pgfqpoint{2.139937in}{1.526625in}}%
\pgfpathcurveto{\pgfqpoint{2.128887in}{1.526625in}}{\pgfqpoint{2.118288in}{1.522235in}}{\pgfqpoint{2.110475in}{1.514421in}}%
\pgfpathcurveto{\pgfqpoint{2.102661in}{1.506607in}}{\pgfqpoint{2.098271in}{1.496008in}}{\pgfqpoint{2.098271in}{1.484958in}}%
\pgfpathcurveto{\pgfqpoint{2.098271in}{1.473908in}}{\pgfqpoint{2.102661in}{1.463309in}}{\pgfqpoint{2.110475in}{1.455495in}}%
\pgfpathcurveto{\pgfqpoint{2.118288in}{1.447682in}}{\pgfqpoint{2.128887in}{1.443291in}}{\pgfqpoint{2.139937in}{1.443291in}}%
\pgfpathclose%
\pgfusepath{stroke,fill}%
\end{pgfscope}%
\begin{pgfscope}%
\pgfpathrectangle{\pgfqpoint{0.375000in}{0.330000in}}{\pgfqpoint{2.325000in}{2.310000in}}%
\pgfusepath{clip}%
\pgfsetbuttcap%
\pgfsetroundjoin%
\definecolor{currentfill}{rgb}{0.000000,0.000000,0.000000}%
\pgfsetfillcolor{currentfill}%
\pgfsetlinewidth{1.003750pt}%
\definecolor{currentstroke}{rgb}{0.000000,0.000000,0.000000}%
\pgfsetstrokecolor{currentstroke}%
\pgfsetdash{}{0pt}%
\pgfpathmoveto{\pgfqpoint{2.139937in}{2.474583in}}%
\pgfpathcurveto{\pgfqpoint{2.150988in}{2.474583in}}{\pgfqpoint{2.161587in}{2.478974in}}{\pgfqpoint{2.169400in}{2.486787in}}%
\pgfpathcurveto{\pgfqpoint{2.177214in}{2.494601in}}{\pgfqpoint{2.181604in}{2.505200in}}{\pgfqpoint{2.181604in}{2.516250in}}%
\pgfpathcurveto{\pgfqpoint{2.181604in}{2.527300in}}{\pgfqpoint{2.177214in}{2.537899in}}{\pgfqpoint{2.169400in}{2.545713in}}%
\pgfpathcurveto{\pgfqpoint{2.161587in}{2.553526in}}{\pgfqpoint{2.150988in}{2.557917in}}{\pgfqpoint{2.139937in}{2.557917in}}%
\pgfpathcurveto{\pgfqpoint{2.128887in}{2.557917in}}{\pgfqpoint{2.118288in}{2.553526in}}{\pgfqpoint{2.110475in}{2.545713in}}%
\pgfpathcurveto{\pgfqpoint{2.102661in}{2.537899in}}{\pgfqpoint{2.098271in}{2.527300in}}{\pgfqpoint{2.098271in}{2.516250in}}%
\pgfpathcurveto{\pgfqpoint{2.098271in}{2.505200in}}{\pgfqpoint{2.102661in}{2.494601in}}{\pgfqpoint{2.110475in}{2.486787in}}%
\pgfpathcurveto{\pgfqpoint{2.118288in}{2.478974in}}{\pgfqpoint{2.128887in}{2.474583in}}{\pgfqpoint{2.139937in}{2.474583in}}%
\pgfpathclose%
\pgfusepath{stroke,fill}%
\end{pgfscope}%
\begin{pgfscope}%
\pgfpathrectangle{\pgfqpoint{0.375000in}{0.330000in}}{\pgfqpoint{2.325000in}{2.310000in}}%
\pgfusepath{clip}%
\pgfsetbuttcap%
\pgfsetroundjoin%
\definecolor{currentfill}{rgb}{0.000000,0.000000,0.000000}%
\pgfsetfillcolor{currentfill}%
\pgfsetlinewidth{1.003750pt}%
\definecolor{currentstroke}{rgb}{0.000000,0.000000,0.000000}%
\pgfsetstrokecolor{currentstroke}%
\pgfsetdash{}{0pt}%
\pgfpathmoveto{\pgfqpoint{2.139937in}{1.443291in}}%
\pgfpathcurveto{\pgfqpoint{2.150988in}{1.443291in}}{\pgfqpoint{2.161587in}{1.447682in}}{\pgfqpoint{2.169400in}{1.455495in}}%
\pgfpathcurveto{\pgfqpoint{2.177214in}{1.463309in}}{\pgfqpoint{2.181604in}{1.473908in}}{\pgfqpoint{2.181604in}{1.484958in}}%
\pgfpathcurveto{\pgfqpoint{2.181604in}{1.496008in}}{\pgfqpoint{2.177214in}{1.506607in}}{\pgfqpoint{2.169400in}{1.514421in}}%
\pgfpathcurveto{\pgfqpoint{2.161587in}{1.522235in}}{\pgfqpoint{2.150988in}{1.526625in}}{\pgfqpoint{2.139937in}{1.526625in}}%
\pgfpathcurveto{\pgfqpoint{2.128887in}{1.526625in}}{\pgfqpoint{2.118288in}{1.522235in}}{\pgfqpoint{2.110475in}{1.514421in}}%
\pgfpathcurveto{\pgfqpoint{2.102661in}{1.506607in}}{\pgfqpoint{2.098271in}{1.496008in}}{\pgfqpoint{2.098271in}{1.484958in}}%
\pgfpathcurveto{\pgfqpoint{2.098271in}{1.473908in}}{\pgfqpoint{2.102661in}{1.463309in}}{\pgfqpoint{2.110475in}{1.455495in}}%
\pgfpathcurveto{\pgfqpoint{2.118288in}{1.447682in}}{\pgfqpoint{2.128887in}{1.443291in}}{\pgfqpoint{2.139937in}{1.443291in}}%
\pgfpathclose%
\pgfusepath{stroke,fill}%
\end{pgfscope}%
\begin{pgfscope}%
\pgfpathrectangle{\pgfqpoint{0.375000in}{0.330000in}}{\pgfqpoint{2.325000in}{2.310000in}}%
\pgfusepath{clip}%
\pgfsetbuttcap%
\pgfsetroundjoin%
\definecolor{currentfill}{rgb}{0.000000,0.000000,0.000000}%
\pgfsetfillcolor{currentfill}%
\pgfsetlinewidth{1.003750pt}%
\definecolor{currentstroke}{rgb}{0.000000,0.000000,0.000000}%
\pgfsetstrokecolor{currentstroke}%
\pgfsetdash{}{0pt}%
\pgfpathmoveto{\pgfqpoint{2.139937in}{1.443291in}}%
\pgfpathcurveto{\pgfqpoint{2.150988in}{1.443291in}}{\pgfqpoint{2.161587in}{1.447682in}}{\pgfqpoint{2.169400in}{1.455495in}}%
\pgfpathcurveto{\pgfqpoint{2.177214in}{1.463309in}}{\pgfqpoint{2.181604in}{1.473908in}}{\pgfqpoint{2.181604in}{1.484958in}}%
\pgfpathcurveto{\pgfqpoint{2.181604in}{1.496008in}}{\pgfqpoint{2.177214in}{1.506607in}}{\pgfqpoint{2.169400in}{1.514421in}}%
\pgfpathcurveto{\pgfqpoint{2.161587in}{1.522235in}}{\pgfqpoint{2.150988in}{1.526625in}}{\pgfqpoint{2.139937in}{1.526625in}}%
\pgfpathcurveto{\pgfqpoint{2.128887in}{1.526625in}}{\pgfqpoint{2.118288in}{1.522235in}}{\pgfqpoint{2.110475in}{1.514421in}}%
\pgfpathcurveto{\pgfqpoint{2.102661in}{1.506607in}}{\pgfqpoint{2.098271in}{1.496008in}}{\pgfqpoint{2.098271in}{1.484958in}}%
\pgfpathcurveto{\pgfqpoint{2.098271in}{1.473908in}}{\pgfqpoint{2.102661in}{1.463309in}}{\pgfqpoint{2.110475in}{1.455495in}}%
\pgfpathcurveto{\pgfqpoint{2.118288in}{1.447682in}}{\pgfqpoint{2.128887in}{1.443291in}}{\pgfqpoint{2.139937in}{1.443291in}}%
\pgfpathclose%
\pgfusepath{stroke,fill}%
\end{pgfscope}%
\begin{pgfscope}%
\pgfpathrectangle{\pgfqpoint{0.375000in}{0.330000in}}{\pgfqpoint{2.325000in}{2.310000in}}%
\pgfusepath{clip}%
\pgfsetbuttcap%
\pgfsetroundjoin%
\definecolor{currentfill}{rgb}{0.000000,0.000000,0.000000}%
\pgfsetfillcolor{currentfill}%
\pgfsetlinewidth{1.003750pt}%
\definecolor{currentstroke}{rgb}{0.000000,0.000000,0.000000}%
\pgfsetstrokecolor{currentstroke}%
\pgfsetdash{}{0pt}%
\pgfpathmoveto{\pgfqpoint{2.139937in}{1.443291in}}%
\pgfpathcurveto{\pgfqpoint{2.150988in}{1.443291in}}{\pgfqpoint{2.161587in}{1.447682in}}{\pgfqpoint{2.169400in}{1.455495in}}%
\pgfpathcurveto{\pgfqpoint{2.177214in}{1.463309in}}{\pgfqpoint{2.181604in}{1.473908in}}{\pgfqpoint{2.181604in}{1.484958in}}%
\pgfpathcurveto{\pgfqpoint{2.181604in}{1.496008in}}{\pgfqpoint{2.177214in}{1.506607in}}{\pgfqpoint{2.169400in}{1.514421in}}%
\pgfpathcurveto{\pgfqpoint{2.161587in}{1.522235in}}{\pgfqpoint{2.150988in}{1.526625in}}{\pgfqpoint{2.139937in}{1.526625in}}%
\pgfpathcurveto{\pgfqpoint{2.128887in}{1.526625in}}{\pgfqpoint{2.118288in}{1.522235in}}{\pgfqpoint{2.110475in}{1.514421in}}%
\pgfpathcurveto{\pgfqpoint{2.102661in}{1.506607in}}{\pgfqpoint{2.098271in}{1.496008in}}{\pgfqpoint{2.098271in}{1.484958in}}%
\pgfpathcurveto{\pgfqpoint{2.098271in}{1.473908in}}{\pgfqpoint{2.102661in}{1.463309in}}{\pgfqpoint{2.110475in}{1.455495in}}%
\pgfpathcurveto{\pgfqpoint{2.118288in}{1.447682in}}{\pgfqpoint{2.128887in}{1.443291in}}{\pgfqpoint{2.139937in}{1.443291in}}%
\pgfpathclose%
\pgfusepath{stroke,fill}%
\end{pgfscope}%
\begin{pgfscope}%
\pgfpathrectangle{\pgfqpoint{0.375000in}{0.330000in}}{\pgfqpoint{2.325000in}{2.310000in}}%
\pgfusepath{clip}%
\pgfsetbuttcap%
\pgfsetroundjoin%
\definecolor{currentfill}{rgb}{0.000000,0.000000,0.000000}%
\pgfsetfillcolor{currentfill}%
\pgfsetlinewidth{1.003750pt}%
\definecolor{currentstroke}{rgb}{0.000000,0.000000,0.000000}%
\pgfsetstrokecolor{currentstroke}%
\pgfsetdash{}{0pt}%
\pgfpathmoveto{\pgfqpoint{2.139937in}{2.474583in}}%
\pgfpathcurveto{\pgfqpoint{2.150988in}{2.474583in}}{\pgfqpoint{2.161587in}{2.478974in}}{\pgfqpoint{2.169400in}{2.486787in}}%
\pgfpathcurveto{\pgfqpoint{2.177214in}{2.494601in}}{\pgfqpoint{2.181604in}{2.505200in}}{\pgfqpoint{2.181604in}{2.516250in}}%
\pgfpathcurveto{\pgfqpoint{2.181604in}{2.527300in}}{\pgfqpoint{2.177214in}{2.537899in}}{\pgfqpoint{2.169400in}{2.545713in}}%
\pgfpathcurveto{\pgfqpoint{2.161587in}{2.553526in}}{\pgfqpoint{2.150988in}{2.557917in}}{\pgfqpoint{2.139937in}{2.557917in}}%
\pgfpathcurveto{\pgfqpoint{2.128887in}{2.557917in}}{\pgfqpoint{2.118288in}{2.553526in}}{\pgfqpoint{2.110475in}{2.545713in}}%
\pgfpathcurveto{\pgfqpoint{2.102661in}{2.537899in}}{\pgfqpoint{2.098271in}{2.527300in}}{\pgfqpoint{2.098271in}{2.516250in}}%
\pgfpathcurveto{\pgfqpoint{2.098271in}{2.505200in}}{\pgfqpoint{2.102661in}{2.494601in}}{\pgfqpoint{2.110475in}{2.486787in}}%
\pgfpathcurveto{\pgfqpoint{2.118288in}{2.478974in}}{\pgfqpoint{2.128887in}{2.474583in}}{\pgfqpoint{2.139937in}{2.474583in}}%
\pgfpathclose%
\pgfusepath{stroke,fill}%
\end{pgfscope}%
\begin{pgfscope}%
\pgfpathrectangle{\pgfqpoint{0.375000in}{0.330000in}}{\pgfqpoint{2.325000in}{2.310000in}}%
\pgfusepath{clip}%
\pgfsetbuttcap%
\pgfsetroundjoin%
\definecolor{currentfill}{rgb}{0.000000,0.000000,0.000000}%
\pgfsetfillcolor{currentfill}%
\pgfsetlinewidth{1.003750pt}%
\definecolor{currentstroke}{rgb}{0.000000,0.000000,0.000000}%
\pgfsetstrokecolor{currentstroke}%
\pgfsetdash{}{0pt}%
\pgfpathmoveto{\pgfqpoint{2.139937in}{2.474583in}}%
\pgfpathcurveto{\pgfqpoint{2.150988in}{2.474583in}}{\pgfqpoint{2.161587in}{2.478974in}}{\pgfqpoint{2.169400in}{2.486787in}}%
\pgfpathcurveto{\pgfqpoint{2.177214in}{2.494601in}}{\pgfqpoint{2.181604in}{2.505200in}}{\pgfqpoint{2.181604in}{2.516250in}}%
\pgfpathcurveto{\pgfqpoint{2.181604in}{2.527300in}}{\pgfqpoint{2.177214in}{2.537899in}}{\pgfqpoint{2.169400in}{2.545713in}}%
\pgfpathcurveto{\pgfqpoint{2.161587in}{2.553526in}}{\pgfqpoint{2.150988in}{2.557917in}}{\pgfqpoint{2.139937in}{2.557917in}}%
\pgfpathcurveto{\pgfqpoint{2.128887in}{2.557917in}}{\pgfqpoint{2.118288in}{2.553526in}}{\pgfqpoint{2.110475in}{2.545713in}}%
\pgfpathcurveto{\pgfqpoint{2.102661in}{2.537899in}}{\pgfqpoint{2.098271in}{2.527300in}}{\pgfqpoint{2.098271in}{2.516250in}}%
\pgfpathcurveto{\pgfqpoint{2.098271in}{2.505200in}}{\pgfqpoint{2.102661in}{2.494601in}}{\pgfqpoint{2.110475in}{2.486787in}}%
\pgfpathcurveto{\pgfqpoint{2.118288in}{2.478974in}}{\pgfqpoint{2.128887in}{2.474583in}}{\pgfqpoint{2.139937in}{2.474583in}}%
\pgfpathclose%
\pgfusepath{stroke,fill}%
\end{pgfscope}%
\begin{pgfscope}%
\pgfpathrectangle{\pgfqpoint{0.375000in}{0.330000in}}{\pgfqpoint{2.325000in}{2.310000in}}%
\pgfusepath{clip}%
\pgfsetbuttcap%
\pgfsetroundjoin%
\definecolor{currentfill}{rgb}{0.000000,0.000000,0.000000}%
\pgfsetfillcolor{currentfill}%
\pgfsetlinewidth{1.003750pt}%
\definecolor{currentstroke}{rgb}{0.000000,0.000000,0.000000}%
\pgfsetstrokecolor{currentstroke}%
\pgfsetdash{}{0pt}%
\pgfpathmoveto{\pgfqpoint{2.139937in}{1.443291in}}%
\pgfpathcurveto{\pgfqpoint{2.150988in}{1.443291in}}{\pgfqpoint{2.161587in}{1.447682in}}{\pgfqpoint{2.169400in}{1.455495in}}%
\pgfpathcurveto{\pgfqpoint{2.177214in}{1.463309in}}{\pgfqpoint{2.181604in}{1.473908in}}{\pgfqpoint{2.181604in}{1.484958in}}%
\pgfpathcurveto{\pgfqpoint{2.181604in}{1.496008in}}{\pgfqpoint{2.177214in}{1.506607in}}{\pgfqpoint{2.169400in}{1.514421in}}%
\pgfpathcurveto{\pgfqpoint{2.161587in}{1.522235in}}{\pgfqpoint{2.150988in}{1.526625in}}{\pgfqpoint{2.139937in}{1.526625in}}%
\pgfpathcurveto{\pgfqpoint{2.128887in}{1.526625in}}{\pgfqpoint{2.118288in}{1.522235in}}{\pgfqpoint{2.110475in}{1.514421in}}%
\pgfpathcurveto{\pgfqpoint{2.102661in}{1.506607in}}{\pgfqpoint{2.098271in}{1.496008in}}{\pgfqpoint{2.098271in}{1.484958in}}%
\pgfpathcurveto{\pgfqpoint{2.098271in}{1.473908in}}{\pgfqpoint{2.102661in}{1.463309in}}{\pgfqpoint{2.110475in}{1.455495in}}%
\pgfpathcurveto{\pgfqpoint{2.118288in}{1.447682in}}{\pgfqpoint{2.128887in}{1.443291in}}{\pgfqpoint{2.139937in}{1.443291in}}%
\pgfpathclose%
\pgfusepath{stroke,fill}%
\end{pgfscope}%
\begin{pgfscope}%
\pgfpathrectangle{\pgfqpoint{0.375000in}{0.330000in}}{\pgfqpoint{2.325000in}{2.310000in}}%
\pgfusepath{clip}%
\pgfsetbuttcap%
\pgfsetroundjoin%
\definecolor{currentfill}{rgb}{0.000000,0.000000,0.000000}%
\pgfsetfillcolor{currentfill}%
\pgfsetlinewidth{1.003750pt}%
\definecolor{currentstroke}{rgb}{0.000000,0.000000,0.000000}%
\pgfsetstrokecolor{currentstroke}%
\pgfsetdash{}{0pt}%
\pgfpathmoveto{\pgfqpoint{2.139937in}{2.474583in}}%
\pgfpathcurveto{\pgfqpoint{2.150988in}{2.474583in}}{\pgfqpoint{2.161587in}{2.478974in}}{\pgfqpoint{2.169400in}{2.486787in}}%
\pgfpathcurveto{\pgfqpoint{2.177214in}{2.494601in}}{\pgfqpoint{2.181604in}{2.505200in}}{\pgfqpoint{2.181604in}{2.516250in}}%
\pgfpathcurveto{\pgfqpoint{2.181604in}{2.527300in}}{\pgfqpoint{2.177214in}{2.537899in}}{\pgfqpoint{2.169400in}{2.545713in}}%
\pgfpathcurveto{\pgfqpoint{2.161587in}{2.553526in}}{\pgfqpoint{2.150988in}{2.557917in}}{\pgfqpoint{2.139937in}{2.557917in}}%
\pgfpathcurveto{\pgfqpoint{2.128887in}{2.557917in}}{\pgfqpoint{2.118288in}{2.553526in}}{\pgfqpoint{2.110475in}{2.545713in}}%
\pgfpathcurveto{\pgfqpoint{2.102661in}{2.537899in}}{\pgfqpoint{2.098271in}{2.527300in}}{\pgfqpoint{2.098271in}{2.516250in}}%
\pgfpathcurveto{\pgfqpoint{2.098271in}{2.505200in}}{\pgfqpoint{2.102661in}{2.494601in}}{\pgfqpoint{2.110475in}{2.486787in}}%
\pgfpathcurveto{\pgfqpoint{2.118288in}{2.478974in}}{\pgfqpoint{2.128887in}{2.474583in}}{\pgfqpoint{2.139937in}{2.474583in}}%
\pgfpathclose%
\pgfusepath{stroke,fill}%
\end{pgfscope}%
\begin{pgfscope}%
\pgfpathrectangle{\pgfqpoint{0.375000in}{0.330000in}}{\pgfqpoint{2.325000in}{2.310000in}}%
\pgfusepath{clip}%
\pgfsetbuttcap%
\pgfsetroundjoin%
\definecolor{currentfill}{rgb}{0.000000,0.000000,0.000000}%
\pgfsetfillcolor{currentfill}%
\pgfsetlinewidth{1.003750pt}%
\definecolor{currentstroke}{rgb}{0.000000,0.000000,0.000000}%
\pgfsetstrokecolor{currentstroke}%
\pgfsetdash{}{0pt}%
\pgfpathmoveto{\pgfqpoint{2.139937in}{2.474583in}}%
\pgfpathcurveto{\pgfqpoint{2.150988in}{2.474583in}}{\pgfqpoint{2.161587in}{2.478974in}}{\pgfqpoint{2.169400in}{2.486787in}}%
\pgfpathcurveto{\pgfqpoint{2.177214in}{2.494601in}}{\pgfqpoint{2.181604in}{2.505200in}}{\pgfqpoint{2.181604in}{2.516250in}}%
\pgfpathcurveto{\pgfqpoint{2.181604in}{2.527300in}}{\pgfqpoint{2.177214in}{2.537899in}}{\pgfqpoint{2.169400in}{2.545713in}}%
\pgfpathcurveto{\pgfqpoint{2.161587in}{2.553526in}}{\pgfqpoint{2.150988in}{2.557917in}}{\pgfqpoint{2.139937in}{2.557917in}}%
\pgfpathcurveto{\pgfqpoint{2.128887in}{2.557917in}}{\pgfqpoint{2.118288in}{2.553526in}}{\pgfqpoint{2.110475in}{2.545713in}}%
\pgfpathcurveto{\pgfqpoint{2.102661in}{2.537899in}}{\pgfqpoint{2.098271in}{2.527300in}}{\pgfqpoint{2.098271in}{2.516250in}}%
\pgfpathcurveto{\pgfqpoint{2.098271in}{2.505200in}}{\pgfqpoint{2.102661in}{2.494601in}}{\pgfqpoint{2.110475in}{2.486787in}}%
\pgfpathcurveto{\pgfqpoint{2.118288in}{2.478974in}}{\pgfqpoint{2.128887in}{2.474583in}}{\pgfqpoint{2.139937in}{2.474583in}}%
\pgfpathclose%
\pgfusepath{stroke,fill}%
\end{pgfscope}%
\begin{pgfscope}%
\pgfpathrectangle{\pgfqpoint{0.375000in}{0.330000in}}{\pgfqpoint{2.325000in}{2.310000in}}%
\pgfusepath{clip}%
\pgfsetbuttcap%
\pgfsetroundjoin%
\definecolor{currentfill}{rgb}{0.000000,0.000000,0.000000}%
\pgfsetfillcolor{currentfill}%
\pgfsetlinewidth{1.003750pt}%
\definecolor{currentstroke}{rgb}{0.000000,0.000000,0.000000}%
\pgfsetstrokecolor{currentstroke}%
\pgfsetdash{}{0pt}%
\pgfpathmoveto{\pgfqpoint{2.139937in}{2.474583in}}%
\pgfpathcurveto{\pgfqpoint{2.150988in}{2.474583in}}{\pgfqpoint{2.161587in}{2.478974in}}{\pgfqpoint{2.169400in}{2.486787in}}%
\pgfpathcurveto{\pgfqpoint{2.177214in}{2.494601in}}{\pgfqpoint{2.181604in}{2.505200in}}{\pgfqpoint{2.181604in}{2.516250in}}%
\pgfpathcurveto{\pgfqpoint{2.181604in}{2.527300in}}{\pgfqpoint{2.177214in}{2.537899in}}{\pgfqpoint{2.169400in}{2.545713in}}%
\pgfpathcurveto{\pgfqpoint{2.161587in}{2.553526in}}{\pgfqpoint{2.150988in}{2.557917in}}{\pgfqpoint{2.139937in}{2.557917in}}%
\pgfpathcurveto{\pgfqpoint{2.128887in}{2.557917in}}{\pgfqpoint{2.118288in}{2.553526in}}{\pgfqpoint{2.110475in}{2.545713in}}%
\pgfpathcurveto{\pgfqpoint{2.102661in}{2.537899in}}{\pgfqpoint{2.098271in}{2.527300in}}{\pgfqpoint{2.098271in}{2.516250in}}%
\pgfpathcurveto{\pgfqpoint{2.098271in}{2.505200in}}{\pgfqpoint{2.102661in}{2.494601in}}{\pgfqpoint{2.110475in}{2.486787in}}%
\pgfpathcurveto{\pgfqpoint{2.118288in}{2.478974in}}{\pgfqpoint{2.128887in}{2.474583in}}{\pgfqpoint{2.139937in}{2.474583in}}%
\pgfpathclose%
\pgfusepath{stroke,fill}%
\end{pgfscope}%
\begin{pgfscope}%
\pgfpathrectangle{\pgfqpoint{0.375000in}{0.330000in}}{\pgfqpoint{2.325000in}{2.310000in}}%
\pgfusepath{clip}%
\pgfsetbuttcap%
\pgfsetroundjoin%
\definecolor{currentfill}{rgb}{0.000000,0.000000,0.000000}%
\pgfsetfillcolor{currentfill}%
\pgfsetlinewidth{1.003750pt}%
\definecolor{currentstroke}{rgb}{0.000000,0.000000,0.000000}%
\pgfsetstrokecolor{currentstroke}%
\pgfsetdash{}{0pt}%
\pgfpathmoveto{\pgfqpoint{2.139937in}{1.443291in}}%
\pgfpathcurveto{\pgfqpoint{2.150988in}{1.443291in}}{\pgfqpoint{2.161587in}{1.447682in}}{\pgfqpoint{2.169400in}{1.455495in}}%
\pgfpathcurveto{\pgfqpoint{2.177214in}{1.463309in}}{\pgfqpoint{2.181604in}{1.473908in}}{\pgfqpoint{2.181604in}{1.484958in}}%
\pgfpathcurveto{\pgfqpoint{2.181604in}{1.496008in}}{\pgfqpoint{2.177214in}{1.506607in}}{\pgfqpoint{2.169400in}{1.514421in}}%
\pgfpathcurveto{\pgfqpoint{2.161587in}{1.522235in}}{\pgfqpoint{2.150988in}{1.526625in}}{\pgfqpoint{2.139937in}{1.526625in}}%
\pgfpathcurveto{\pgfqpoint{2.128887in}{1.526625in}}{\pgfqpoint{2.118288in}{1.522235in}}{\pgfqpoint{2.110475in}{1.514421in}}%
\pgfpathcurveto{\pgfqpoint{2.102661in}{1.506607in}}{\pgfqpoint{2.098271in}{1.496008in}}{\pgfqpoint{2.098271in}{1.484958in}}%
\pgfpathcurveto{\pgfqpoint{2.098271in}{1.473908in}}{\pgfqpoint{2.102661in}{1.463309in}}{\pgfqpoint{2.110475in}{1.455495in}}%
\pgfpathcurveto{\pgfqpoint{2.118288in}{1.447682in}}{\pgfqpoint{2.128887in}{1.443291in}}{\pgfqpoint{2.139937in}{1.443291in}}%
\pgfpathclose%
\pgfusepath{stroke,fill}%
\end{pgfscope}%
\begin{pgfscope}%
\pgfpathrectangle{\pgfqpoint{0.375000in}{0.330000in}}{\pgfqpoint{2.325000in}{2.310000in}}%
\pgfusepath{clip}%
\pgfsetbuttcap%
\pgfsetroundjoin%
\definecolor{currentfill}{rgb}{0.000000,0.000000,0.000000}%
\pgfsetfillcolor{currentfill}%
\pgfsetlinewidth{1.003750pt}%
\definecolor{currentstroke}{rgb}{0.000000,0.000000,0.000000}%
\pgfsetstrokecolor{currentstroke}%
\pgfsetdash{}{0pt}%
\pgfpathmoveto{\pgfqpoint{2.139937in}{1.443291in}}%
\pgfpathcurveto{\pgfqpoint{2.150988in}{1.443291in}}{\pgfqpoint{2.161587in}{1.447682in}}{\pgfqpoint{2.169400in}{1.455495in}}%
\pgfpathcurveto{\pgfqpoint{2.177214in}{1.463309in}}{\pgfqpoint{2.181604in}{1.473908in}}{\pgfqpoint{2.181604in}{1.484958in}}%
\pgfpathcurveto{\pgfqpoint{2.181604in}{1.496008in}}{\pgfqpoint{2.177214in}{1.506607in}}{\pgfqpoint{2.169400in}{1.514421in}}%
\pgfpathcurveto{\pgfqpoint{2.161587in}{1.522235in}}{\pgfqpoint{2.150988in}{1.526625in}}{\pgfqpoint{2.139937in}{1.526625in}}%
\pgfpathcurveto{\pgfqpoint{2.128887in}{1.526625in}}{\pgfqpoint{2.118288in}{1.522235in}}{\pgfqpoint{2.110475in}{1.514421in}}%
\pgfpathcurveto{\pgfqpoint{2.102661in}{1.506607in}}{\pgfqpoint{2.098271in}{1.496008in}}{\pgfqpoint{2.098271in}{1.484958in}}%
\pgfpathcurveto{\pgfqpoint{2.098271in}{1.473908in}}{\pgfqpoint{2.102661in}{1.463309in}}{\pgfqpoint{2.110475in}{1.455495in}}%
\pgfpathcurveto{\pgfqpoint{2.118288in}{1.447682in}}{\pgfqpoint{2.128887in}{1.443291in}}{\pgfqpoint{2.139937in}{1.443291in}}%
\pgfpathclose%
\pgfusepath{stroke,fill}%
\end{pgfscope}%
\begin{pgfscope}%
\pgfpathrectangle{\pgfqpoint{0.375000in}{0.330000in}}{\pgfqpoint{2.325000in}{2.310000in}}%
\pgfusepath{clip}%
\pgfsetbuttcap%
\pgfsetroundjoin%
\definecolor{currentfill}{rgb}{0.000000,0.000000,0.000000}%
\pgfsetfillcolor{currentfill}%
\pgfsetlinewidth{1.003750pt}%
\definecolor{currentstroke}{rgb}{0.000000,0.000000,0.000000}%
\pgfsetstrokecolor{currentstroke}%
\pgfsetdash{}{0pt}%
\pgfpathmoveto{\pgfqpoint{2.139937in}{2.474583in}}%
\pgfpathcurveto{\pgfqpoint{2.150988in}{2.474583in}}{\pgfqpoint{2.161587in}{2.478974in}}{\pgfqpoint{2.169400in}{2.486787in}}%
\pgfpathcurveto{\pgfqpoint{2.177214in}{2.494601in}}{\pgfqpoint{2.181604in}{2.505200in}}{\pgfqpoint{2.181604in}{2.516250in}}%
\pgfpathcurveto{\pgfqpoint{2.181604in}{2.527300in}}{\pgfqpoint{2.177214in}{2.537899in}}{\pgfqpoint{2.169400in}{2.545713in}}%
\pgfpathcurveto{\pgfqpoint{2.161587in}{2.553526in}}{\pgfqpoint{2.150988in}{2.557917in}}{\pgfqpoint{2.139937in}{2.557917in}}%
\pgfpathcurveto{\pgfqpoint{2.128887in}{2.557917in}}{\pgfqpoint{2.118288in}{2.553526in}}{\pgfqpoint{2.110475in}{2.545713in}}%
\pgfpathcurveto{\pgfqpoint{2.102661in}{2.537899in}}{\pgfqpoint{2.098271in}{2.527300in}}{\pgfqpoint{2.098271in}{2.516250in}}%
\pgfpathcurveto{\pgfqpoint{2.098271in}{2.505200in}}{\pgfqpoint{2.102661in}{2.494601in}}{\pgfqpoint{2.110475in}{2.486787in}}%
\pgfpathcurveto{\pgfqpoint{2.118288in}{2.478974in}}{\pgfqpoint{2.128887in}{2.474583in}}{\pgfqpoint{2.139937in}{2.474583in}}%
\pgfpathclose%
\pgfusepath{stroke,fill}%
\end{pgfscope}%
\begin{pgfscope}%
\pgfpathrectangle{\pgfqpoint{0.375000in}{0.330000in}}{\pgfqpoint{2.325000in}{2.310000in}}%
\pgfusepath{clip}%
\pgfsetbuttcap%
\pgfsetroundjoin%
\definecolor{currentfill}{rgb}{0.000000,0.000000,0.000000}%
\pgfsetfillcolor{currentfill}%
\pgfsetlinewidth{1.003750pt}%
\definecolor{currentstroke}{rgb}{0.000000,0.000000,0.000000}%
\pgfsetstrokecolor{currentstroke}%
\pgfsetdash{}{0pt}%
\pgfpathmoveto{\pgfqpoint{2.139937in}{2.474583in}}%
\pgfpathcurveto{\pgfqpoint{2.150988in}{2.474583in}}{\pgfqpoint{2.161587in}{2.478974in}}{\pgfqpoint{2.169400in}{2.486787in}}%
\pgfpathcurveto{\pgfqpoint{2.177214in}{2.494601in}}{\pgfqpoint{2.181604in}{2.505200in}}{\pgfqpoint{2.181604in}{2.516250in}}%
\pgfpathcurveto{\pgfqpoint{2.181604in}{2.527300in}}{\pgfqpoint{2.177214in}{2.537899in}}{\pgfqpoint{2.169400in}{2.545713in}}%
\pgfpathcurveto{\pgfqpoint{2.161587in}{2.553526in}}{\pgfqpoint{2.150988in}{2.557917in}}{\pgfqpoint{2.139937in}{2.557917in}}%
\pgfpathcurveto{\pgfqpoint{2.128887in}{2.557917in}}{\pgfqpoint{2.118288in}{2.553526in}}{\pgfqpoint{2.110475in}{2.545713in}}%
\pgfpathcurveto{\pgfqpoint{2.102661in}{2.537899in}}{\pgfqpoint{2.098271in}{2.527300in}}{\pgfqpoint{2.098271in}{2.516250in}}%
\pgfpathcurveto{\pgfqpoint{2.098271in}{2.505200in}}{\pgfqpoint{2.102661in}{2.494601in}}{\pgfqpoint{2.110475in}{2.486787in}}%
\pgfpathcurveto{\pgfqpoint{2.118288in}{2.478974in}}{\pgfqpoint{2.128887in}{2.474583in}}{\pgfqpoint{2.139937in}{2.474583in}}%
\pgfpathclose%
\pgfusepath{stroke,fill}%
\end{pgfscope}%
\begin{pgfscope}%
\pgfpathrectangle{\pgfqpoint{0.375000in}{0.330000in}}{\pgfqpoint{2.325000in}{2.310000in}}%
\pgfusepath{clip}%
\pgfsetbuttcap%
\pgfsetroundjoin%
\definecolor{currentfill}{rgb}{0.000000,0.000000,0.000000}%
\pgfsetfillcolor{currentfill}%
\pgfsetlinewidth{1.003750pt}%
\definecolor{currentstroke}{rgb}{0.000000,0.000000,0.000000}%
\pgfsetstrokecolor{currentstroke}%
\pgfsetdash{}{0pt}%
\pgfpathmoveto{\pgfqpoint{2.139937in}{1.443291in}}%
\pgfpathcurveto{\pgfqpoint{2.150988in}{1.443291in}}{\pgfqpoint{2.161587in}{1.447682in}}{\pgfqpoint{2.169400in}{1.455495in}}%
\pgfpathcurveto{\pgfqpoint{2.177214in}{1.463309in}}{\pgfqpoint{2.181604in}{1.473908in}}{\pgfqpoint{2.181604in}{1.484958in}}%
\pgfpathcurveto{\pgfqpoint{2.181604in}{1.496008in}}{\pgfqpoint{2.177214in}{1.506607in}}{\pgfqpoint{2.169400in}{1.514421in}}%
\pgfpathcurveto{\pgfqpoint{2.161587in}{1.522235in}}{\pgfqpoint{2.150988in}{1.526625in}}{\pgfqpoint{2.139937in}{1.526625in}}%
\pgfpathcurveto{\pgfqpoint{2.128887in}{1.526625in}}{\pgfqpoint{2.118288in}{1.522235in}}{\pgfqpoint{2.110475in}{1.514421in}}%
\pgfpathcurveto{\pgfqpoint{2.102661in}{1.506607in}}{\pgfqpoint{2.098271in}{1.496008in}}{\pgfqpoint{2.098271in}{1.484958in}}%
\pgfpathcurveto{\pgfqpoint{2.098271in}{1.473908in}}{\pgfqpoint{2.102661in}{1.463309in}}{\pgfqpoint{2.110475in}{1.455495in}}%
\pgfpathcurveto{\pgfqpoint{2.118288in}{1.447682in}}{\pgfqpoint{2.128887in}{1.443291in}}{\pgfqpoint{2.139937in}{1.443291in}}%
\pgfpathclose%
\pgfusepath{stroke,fill}%
\end{pgfscope}%
\begin{pgfscope}%
\pgfpathrectangle{\pgfqpoint{0.375000in}{0.330000in}}{\pgfqpoint{2.325000in}{2.310000in}}%
\pgfusepath{clip}%
\pgfsetbuttcap%
\pgfsetroundjoin%
\definecolor{currentfill}{rgb}{0.000000,0.000000,0.000000}%
\pgfsetfillcolor{currentfill}%
\pgfsetlinewidth{1.003750pt}%
\definecolor{currentstroke}{rgb}{0.000000,0.000000,0.000000}%
\pgfsetstrokecolor{currentstroke}%
\pgfsetdash{}{0pt}%
\pgfpathmoveto{\pgfqpoint{2.139937in}{1.443291in}}%
\pgfpathcurveto{\pgfqpoint{2.150988in}{1.443291in}}{\pgfqpoint{2.161587in}{1.447682in}}{\pgfqpoint{2.169400in}{1.455495in}}%
\pgfpathcurveto{\pgfqpoint{2.177214in}{1.463309in}}{\pgfqpoint{2.181604in}{1.473908in}}{\pgfqpoint{2.181604in}{1.484958in}}%
\pgfpathcurveto{\pgfqpoint{2.181604in}{1.496008in}}{\pgfqpoint{2.177214in}{1.506607in}}{\pgfqpoint{2.169400in}{1.514421in}}%
\pgfpathcurveto{\pgfqpoint{2.161587in}{1.522235in}}{\pgfqpoint{2.150988in}{1.526625in}}{\pgfqpoint{2.139937in}{1.526625in}}%
\pgfpathcurveto{\pgfqpoint{2.128887in}{1.526625in}}{\pgfqpoint{2.118288in}{1.522235in}}{\pgfqpoint{2.110475in}{1.514421in}}%
\pgfpathcurveto{\pgfqpoint{2.102661in}{1.506607in}}{\pgfqpoint{2.098271in}{1.496008in}}{\pgfqpoint{2.098271in}{1.484958in}}%
\pgfpathcurveto{\pgfqpoint{2.098271in}{1.473908in}}{\pgfqpoint{2.102661in}{1.463309in}}{\pgfqpoint{2.110475in}{1.455495in}}%
\pgfpathcurveto{\pgfqpoint{2.118288in}{1.447682in}}{\pgfqpoint{2.128887in}{1.443291in}}{\pgfqpoint{2.139937in}{1.443291in}}%
\pgfpathclose%
\pgfusepath{stroke,fill}%
\end{pgfscope}%
\begin{pgfscope}%
\pgfpathrectangle{\pgfqpoint{0.375000in}{0.330000in}}{\pgfqpoint{2.325000in}{2.310000in}}%
\pgfusepath{clip}%
\pgfsetbuttcap%
\pgfsetroundjoin%
\definecolor{currentfill}{rgb}{0.000000,0.000000,0.000000}%
\pgfsetfillcolor{currentfill}%
\pgfsetlinewidth{1.003750pt}%
\definecolor{currentstroke}{rgb}{0.000000,0.000000,0.000000}%
\pgfsetstrokecolor{currentstroke}%
\pgfsetdash{}{0pt}%
\pgfpathmoveto{\pgfqpoint{2.139937in}{2.474583in}}%
\pgfpathcurveto{\pgfqpoint{2.150988in}{2.474583in}}{\pgfqpoint{2.161587in}{2.478974in}}{\pgfqpoint{2.169400in}{2.486787in}}%
\pgfpathcurveto{\pgfqpoint{2.177214in}{2.494601in}}{\pgfqpoint{2.181604in}{2.505200in}}{\pgfqpoint{2.181604in}{2.516250in}}%
\pgfpathcurveto{\pgfqpoint{2.181604in}{2.527300in}}{\pgfqpoint{2.177214in}{2.537899in}}{\pgfqpoint{2.169400in}{2.545713in}}%
\pgfpathcurveto{\pgfqpoint{2.161587in}{2.553526in}}{\pgfqpoint{2.150988in}{2.557917in}}{\pgfqpoint{2.139937in}{2.557917in}}%
\pgfpathcurveto{\pgfqpoint{2.128887in}{2.557917in}}{\pgfqpoint{2.118288in}{2.553526in}}{\pgfqpoint{2.110475in}{2.545713in}}%
\pgfpathcurveto{\pgfqpoint{2.102661in}{2.537899in}}{\pgfqpoint{2.098271in}{2.527300in}}{\pgfqpoint{2.098271in}{2.516250in}}%
\pgfpathcurveto{\pgfqpoint{2.098271in}{2.505200in}}{\pgfqpoint{2.102661in}{2.494601in}}{\pgfqpoint{2.110475in}{2.486787in}}%
\pgfpathcurveto{\pgfqpoint{2.118288in}{2.478974in}}{\pgfqpoint{2.128887in}{2.474583in}}{\pgfqpoint{2.139937in}{2.474583in}}%
\pgfpathclose%
\pgfusepath{stroke,fill}%
\end{pgfscope}%
\begin{pgfscope}%
\pgfpathrectangle{\pgfqpoint{0.375000in}{0.330000in}}{\pgfqpoint{2.325000in}{2.310000in}}%
\pgfusepath{clip}%
\pgfsetbuttcap%
\pgfsetroundjoin%
\definecolor{currentfill}{rgb}{0.000000,0.000000,0.000000}%
\pgfsetfillcolor{currentfill}%
\pgfsetlinewidth{1.003750pt}%
\definecolor{currentstroke}{rgb}{0.000000,0.000000,0.000000}%
\pgfsetstrokecolor{currentstroke}%
\pgfsetdash{}{0pt}%
\pgfpathmoveto{\pgfqpoint{2.139937in}{1.443291in}}%
\pgfpathcurveto{\pgfqpoint{2.150988in}{1.443291in}}{\pgfqpoint{2.161587in}{1.447682in}}{\pgfqpoint{2.169400in}{1.455495in}}%
\pgfpathcurveto{\pgfqpoint{2.177214in}{1.463309in}}{\pgfqpoint{2.181604in}{1.473908in}}{\pgfqpoint{2.181604in}{1.484958in}}%
\pgfpathcurveto{\pgfqpoint{2.181604in}{1.496008in}}{\pgfqpoint{2.177214in}{1.506607in}}{\pgfqpoint{2.169400in}{1.514421in}}%
\pgfpathcurveto{\pgfqpoint{2.161587in}{1.522235in}}{\pgfqpoint{2.150988in}{1.526625in}}{\pgfqpoint{2.139937in}{1.526625in}}%
\pgfpathcurveto{\pgfqpoint{2.128887in}{1.526625in}}{\pgfqpoint{2.118288in}{1.522235in}}{\pgfqpoint{2.110475in}{1.514421in}}%
\pgfpathcurveto{\pgfqpoint{2.102661in}{1.506607in}}{\pgfqpoint{2.098271in}{1.496008in}}{\pgfqpoint{2.098271in}{1.484958in}}%
\pgfpathcurveto{\pgfqpoint{2.098271in}{1.473908in}}{\pgfqpoint{2.102661in}{1.463309in}}{\pgfqpoint{2.110475in}{1.455495in}}%
\pgfpathcurveto{\pgfqpoint{2.118288in}{1.447682in}}{\pgfqpoint{2.128887in}{1.443291in}}{\pgfqpoint{2.139937in}{1.443291in}}%
\pgfpathclose%
\pgfusepath{stroke,fill}%
\end{pgfscope}%
\begin{pgfscope}%
\pgfpathrectangle{\pgfqpoint{0.375000in}{0.330000in}}{\pgfqpoint{2.325000in}{2.310000in}}%
\pgfusepath{clip}%
\pgfsetbuttcap%
\pgfsetroundjoin%
\definecolor{currentfill}{rgb}{0.000000,0.000000,0.000000}%
\pgfsetfillcolor{currentfill}%
\pgfsetlinewidth{1.003750pt}%
\definecolor{currentstroke}{rgb}{0.000000,0.000000,0.000000}%
\pgfsetstrokecolor{currentstroke}%
\pgfsetdash{}{0pt}%
\pgfpathmoveto{\pgfqpoint{2.139937in}{2.474583in}}%
\pgfpathcurveto{\pgfqpoint{2.150988in}{2.474583in}}{\pgfqpoint{2.161587in}{2.478974in}}{\pgfqpoint{2.169400in}{2.486787in}}%
\pgfpathcurveto{\pgfqpoint{2.177214in}{2.494601in}}{\pgfqpoint{2.181604in}{2.505200in}}{\pgfqpoint{2.181604in}{2.516250in}}%
\pgfpathcurveto{\pgfqpoint{2.181604in}{2.527300in}}{\pgfqpoint{2.177214in}{2.537899in}}{\pgfqpoint{2.169400in}{2.545713in}}%
\pgfpathcurveto{\pgfqpoint{2.161587in}{2.553526in}}{\pgfqpoint{2.150988in}{2.557917in}}{\pgfqpoint{2.139937in}{2.557917in}}%
\pgfpathcurveto{\pgfqpoint{2.128887in}{2.557917in}}{\pgfqpoint{2.118288in}{2.553526in}}{\pgfqpoint{2.110475in}{2.545713in}}%
\pgfpathcurveto{\pgfqpoint{2.102661in}{2.537899in}}{\pgfqpoint{2.098271in}{2.527300in}}{\pgfqpoint{2.098271in}{2.516250in}}%
\pgfpathcurveto{\pgfqpoint{2.098271in}{2.505200in}}{\pgfqpoint{2.102661in}{2.494601in}}{\pgfqpoint{2.110475in}{2.486787in}}%
\pgfpathcurveto{\pgfqpoint{2.118288in}{2.478974in}}{\pgfqpoint{2.128887in}{2.474583in}}{\pgfqpoint{2.139937in}{2.474583in}}%
\pgfpathclose%
\pgfusepath{stroke,fill}%
\end{pgfscope}%
\begin{pgfscope}%
\pgfpathrectangle{\pgfqpoint{0.375000in}{0.330000in}}{\pgfqpoint{2.325000in}{2.310000in}}%
\pgfusepath{clip}%
\pgfsetbuttcap%
\pgfsetroundjoin%
\definecolor{currentfill}{rgb}{0.000000,0.000000,0.000000}%
\pgfsetfillcolor{currentfill}%
\pgfsetlinewidth{1.003750pt}%
\definecolor{currentstroke}{rgb}{0.000000,0.000000,0.000000}%
\pgfsetstrokecolor{currentstroke}%
\pgfsetdash{}{0pt}%
\pgfpathmoveto{\pgfqpoint{2.139937in}{1.443291in}}%
\pgfpathcurveto{\pgfqpoint{2.150988in}{1.443291in}}{\pgfqpoint{2.161587in}{1.447682in}}{\pgfqpoint{2.169400in}{1.455495in}}%
\pgfpathcurveto{\pgfqpoint{2.177214in}{1.463309in}}{\pgfqpoint{2.181604in}{1.473908in}}{\pgfqpoint{2.181604in}{1.484958in}}%
\pgfpathcurveto{\pgfqpoint{2.181604in}{1.496008in}}{\pgfqpoint{2.177214in}{1.506607in}}{\pgfqpoint{2.169400in}{1.514421in}}%
\pgfpathcurveto{\pgfqpoint{2.161587in}{1.522235in}}{\pgfqpoint{2.150988in}{1.526625in}}{\pgfqpoint{2.139937in}{1.526625in}}%
\pgfpathcurveto{\pgfqpoint{2.128887in}{1.526625in}}{\pgfqpoint{2.118288in}{1.522235in}}{\pgfqpoint{2.110475in}{1.514421in}}%
\pgfpathcurveto{\pgfqpoint{2.102661in}{1.506607in}}{\pgfqpoint{2.098271in}{1.496008in}}{\pgfqpoint{2.098271in}{1.484958in}}%
\pgfpathcurveto{\pgfqpoint{2.098271in}{1.473908in}}{\pgfqpoint{2.102661in}{1.463309in}}{\pgfqpoint{2.110475in}{1.455495in}}%
\pgfpathcurveto{\pgfqpoint{2.118288in}{1.447682in}}{\pgfqpoint{2.128887in}{1.443291in}}{\pgfqpoint{2.139937in}{1.443291in}}%
\pgfpathclose%
\pgfusepath{stroke,fill}%
\end{pgfscope}%
\begin{pgfscope}%
\pgfpathrectangle{\pgfqpoint{0.375000in}{0.330000in}}{\pgfqpoint{2.325000in}{2.310000in}}%
\pgfusepath{clip}%
\pgfsetbuttcap%
\pgfsetroundjoin%
\definecolor{currentfill}{rgb}{0.000000,0.000000,0.000000}%
\pgfsetfillcolor{currentfill}%
\pgfsetlinewidth{1.003750pt}%
\definecolor{currentstroke}{rgb}{0.000000,0.000000,0.000000}%
\pgfsetstrokecolor{currentstroke}%
\pgfsetdash{}{0pt}%
\pgfpathmoveto{\pgfqpoint{2.139937in}{1.443291in}}%
\pgfpathcurveto{\pgfqpoint{2.150988in}{1.443291in}}{\pgfqpoint{2.161587in}{1.447682in}}{\pgfqpoint{2.169400in}{1.455495in}}%
\pgfpathcurveto{\pgfqpoint{2.177214in}{1.463309in}}{\pgfqpoint{2.181604in}{1.473908in}}{\pgfqpoint{2.181604in}{1.484958in}}%
\pgfpathcurveto{\pgfqpoint{2.181604in}{1.496008in}}{\pgfqpoint{2.177214in}{1.506607in}}{\pgfqpoint{2.169400in}{1.514421in}}%
\pgfpathcurveto{\pgfqpoint{2.161587in}{1.522235in}}{\pgfqpoint{2.150988in}{1.526625in}}{\pgfqpoint{2.139937in}{1.526625in}}%
\pgfpathcurveto{\pgfqpoint{2.128887in}{1.526625in}}{\pgfqpoint{2.118288in}{1.522235in}}{\pgfqpoint{2.110475in}{1.514421in}}%
\pgfpathcurveto{\pgfqpoint{2.102661in}{1.506607in}}{\pgfqpoint{2.098271in}{1.496008in}}{\pgfqpoint{2.098271in}{1.484958in}}%
\pgfpathcurveto{\pgfqpoint{2.098271in}{1.473908in}}{\pgfqpoint{2.102661in}{1.463309in}}{\pgfqpoint{2.110475in}{1.455495in}}%
\pgfpathcurveto{\pgfqpoint{2.118288in}{1.447682in}}{\pgfqpoint{2.128887in}{1.443291in}}{\pgfqpoint{2.139937in}{1.443291in}}%
\pgfpathclose%
\pgfusepath{stroke,fill}%
\end{pgfscope}%
\begin{pgfscope}%
\pgfpathrectangle{\pgfqpoint{0.375000in}{0.330000in}}{\pgfqpoint{2.325000in}{2.310000in}}%
\pgfusepath{clip}%
\pgfsetbuttcap%
\pgfsetroundjoin%
\definecolor{currentfill}{rgb}{0.000000,0.000000,0.000000}%
\pgfsetfillcolor{currentfill}%
\pgfsetlinewidth{1.003750pt}%
\definecolor{currentstroke}{rgb}{0.000000,0.000000,0.000000}%
\pgfsetstrokecolor{currentstroke}%
\pgfsetdash{}{0pt}%
\pgfpathmoveto{\pgfqpoint{2.139937in}{2.474583in}}%
\pgfpathcurveto{\pgfqpoint{2.150988in}{2.474583in}}{\pgfqpoint{2.161587in}{2.478974in}}{\pgfqpoint{2.169400in}{2.486787in}}%
\pgfpathcurveto{\pgfqpoint{2.177214in}{2.494601in}}{\pgfqpoint{2.181604in}{2.505200in}}{\pgfqpoint{2.181604in}{2.516250in}}%
\pgfpathcurveto{\pgfqpoint{2.181604in}{2.527300in}}{\pgfqpoint{2.177214in}{2.537899in}}{\pgfqpoint{2.169400in}{2.545713in}}%
\pgfpathcurveto{\pgfqpoint{2.161587in}{2.553526in}}{\pgfqpoint{2.150988in}{2.557917in}}{\pgfqpoint{2.139937in}{2.557917in}}%
\pgfpathcurveto{\pgfqpoint{2.128887in}{2.557917in}}{\pgfqpoint{2.118288in}{2.553526in}}{\pgfqpoint{2.110475in}{2.545713in}}%
\pgfpathcurveto{\pgfqpoint{2.102661in}{2.537899in}}{\pgfqpoint{2.098271in}{2.527300in}}{\pgfqpoint{2.098271in}{2.516250in}}%
\pgfpathcurveto{\pgfqpoint{2.098271in}{2.505200in}}{\pgfqpoint{2.102661in}{2.494601in}}{\pgfqpoint{2.110475in}{2.486787in}}%
\pgfpathcurveto{\pgfqpoint{2.118288in}{2.478974in}}{\pgfqpoint{2.128887in}{2.474583in}}{\pgfqpoint{2.139937in}{2.474583in}}%
\pgfpathclose%
\pgfusepath{stroke,fill}%
\end{pgfscope}%
\begin{pgfscope}%
\pgfpathrectangle{\pgfqpoint{0.375000in}{0.330000in}}{\pgfqpoint{2.325000in}{2.310000in}}%
\pgfusepath{clip}%
\pgfsetbuttcap%
\pgfsetroundjoin%
\definecolor{currentfill}{rgb}{0.000000,0.000000,0.000000}%
\pgfsetfillcolor{currentfill}%
\pgfsetlinewidth{1.003750pt}%
\definecolor{currentstroke}{rgb}{0.000000,0.000000,0.000000}%
\pgfsetstrokecolor{currentstroke}%
\pgfsetdash{}{0pt}%
\pgfpathmoveto{\pgfqpoint{2.139937in}{1.443291in}}%
\pgfpathcurveto{\pgfqpoint{2.150988in}{1.443291in}}{\pgfqpoint{2.161587in}{1.447682in}}{\pgfqpoint{2.169400in}{1.455495in}}%
\pgfpathcurveto{\pgfqpoint{2.177214in}{1.463309in}}{\pgfqpoint{2.181604in}{1.473908in}}{\pgfqpoint{2.181604in}{1.484958in}}%
\pgfpathcurveto{\pgfqpoint{2.181604in}{1.496008in}}{\pgfqpoint{2.177214in}{1.506607in}}{\pgfqpoint{2.169400in}{1.514421in}}%
\pgfpathcurveto{\pgfqpoint{2.161587in}{1.522235in}}{\pgfqpoint{2.150988in}{1.526625in}}{\pgfqpoint{2.139937in}{1.526625in}}%
\pgfpathcurveto{\pgfqpoint{2.128887in}{1.526625in}}{\pgfqpoint{2.118288in}{1.522235in}}{\pgfqpoint{2.110475in}{1.514421in}}%
\pgfpathcurveto{\pgfqpoint{2.102661in}{1.506607in}}{\pgfqpoint{2.098271in}{1.496008in}}{\pgfqpoint{2.098271in}{1.484958in}}%
\pgfpathcurveto{\pgfqpoint{2.098271in}{1.473908in}}{\pgfqpoint{2.102661in}{1.463309in}}{\pgfqpoint{2.110475in}{1.455495in}}%
\pgfpathcurveto{\pgfqpoint{2.118288in}{1.447682in}}{\pgfqpoint{2.128887in}{1.443291in}}{\pgfqpoint{2.139937in}{1.443291in}}%
\pgfpathclose%
\pgfusepath{stroke,fill}%
\end{pgfscope}%
\begin{pgfscope}%
\pgfpathrectangle{\pgfqpoint{0.375000in}{0.330000in}}{\pgfqpoint{2.325000in}{2.310000in}}%
\pgfusepath{clip}%
\pgfsetbuttcap%
\pgfsetroundjoin%
\definecolor{currentfill}{rgb}{0.000000,0.000000,0.000000}%
\pgfsetfillcolor{currentfill}%
\pgfsetlinewidth{1.003750pt}%
\definecolor{currentstroke}{rgb}{0.000000,0.000000,0.000000}%
\pgfsetstrokecolor{currentstroke}%
\pgfsetdash{}{0pt}%
\pgfpathmoveto{\pgfqpoint{2.139937in}{2.474583in}}%
\pgfpathcurveto{\pgfqpoint{2.150988in}{2.474583in}}{\pgfqpoint{2.161587in}{2.478974in}}{\pgfqpoint{2.169400in}{2.486787in}}%
\pgfpathcurveto{\pgfqpoint{2.177214in}{2.494601in}}{\pgfqpoint{2.181604in}{2.505200in}}{\pgfqpoint{2.181604in}{2.516250in}}%
\pgfpathcurveto{\pgfqpoint{2.181604in}{2.527300in}}{\pgfqpoint{2.177214in}{2.537899in}}{\pgfqpoint{2.169400in}{2.545713in}}%
\pgfpathcurveto{\pgfqpoint{2.161587in}{2.553526in}}{\pgfqpoint{2.150988in}{2.557917in}}{\pgfqpoint{2.139937in}{2.557917in}}%
\pgfpathcurveto{\pgfqpoint{2.128887in}{2.557917in}}{\pgfqpoint{2.118288in}{2.553526in}}{\pgfqpoint{2.110475in}{2.545713in}}%
\pgfpathcurveto{\pgfqpoint{2.102661in}{2.537899in}}{\pgfqpoint{2.098271in}{2.527300in}}{\pgfqpoint{2.098271in}{2.516250in}}%
\pgfpathcurveto{\pgfqpoint{2.098271in}{2.505200in}}{\pgfqpoint{2.102661in}{2.494601in}}{\pgfqpoint{2.110475in}{2.486787in}}%
\pgfpathcurveto{\pgfqpoint{2.118288in}{2.478974in}}{\pgfqpoint{2.128887in}{2.474583in}}{\pgfqpoint{2.139937in}{2.474583in}}%
\pgfpathclose%
\pgfusepath{stroke,fill}%
\end{pgfscope}%
\begin{pgfscope}%
\pgfpathrectangle{\pgfqpoint{0.375000in}{0.330000in}}{\pgfqpoint{2.325000in}{2.310000in}}%
\pgfusepath{clip}%
\pgfsetbuttcap%
\pgfsetroundjoin%
\definecolor{currentfill}{rgb}{0.000000,0.000000,0.000000}%
\pgfsetfillcolor{currentfill}%
\pgfsetlinewidth{1.003750pt}%
\definecolor{currentstroke}{rgb}{0.000000,0.000000,0.000000}%
\pgfsetstrokecolor{currentstroke}%
\pgfsetdash{}{0pt}%
\pgfpathmoveto{\pgfqpoint{2.139937in}{2.474583in}}%
\pgfpathcurveto{\pgfqpoint{2.150988in}{2.474583in}}{\pgfqpoint{2.161587in}{2.478974in}}{\pgfqpoint{2.169400in}{2.486787in}}%
\pgfpathcurveto{\pgfqpoint{2.177214in}{2.494601in}}{\pgfqpoint{2.181604in}{2.505200in}}{\pgfqpoint{2.181604in}{2.516250in}}%
\pgfpathcurveto{\pgfqpoint{2.181604in}{2.527300in}}{\pgfqpoint{2.177214in}{2.537899in}}{\pgfqpoint{2.169400in}{2.545713in}}%
\pgfpathcurveto{\pgfqpoint{2.161587in}{2.553526in}}{\pgfqpoint{2.150988in}{2.557917in}}{\pgfqpoint{2.139937in}{2.557917in}}%
\pgfpathcurveto{\pgfqpoint{2.128887in}{2.557917in}}{\pgfqpoint{2.118288in}{2.553526in}}{\pgfqpoint{2.110475in}{2.545713in}}%
\pgfpathcurveto{\pgfqpoint{2.102661in}{2.537899in}}{\pgfqpoint{2.098271in}{2.527300in}}{\pgfqpoint{2.098271in}{2.516250in}}%
\pgfpathcurveto{\pgfqpoint{2.098271in}{2.505200in}}{\pgfqpoint{2.102661in}{2.494601in}}{\pgfqpoint{2.110475in}{2.486787in}}%
\pgfpathcurveto{\pgfqpoint{2.118288in}{2.478974in}}{\pgfqpoint{2.128887in}{2.474583in}}{\pgfqpoint{2.139937in}{2.474583in}}%
\pgfpathclose%
\pgfusepath{stroke,fill}%
\end{pgfscope}%
\begin{pgfscope}%
\pgfpathrectangle{\pgfqpoint{0.375000in}{0.330000in}}{\pgfqpoint{2.325000in}{2.310000in}}%
\pgfusepath{clip}%
\pgfsetbuttcap%
\pgfsetroundjoin%
\definecolor{currentfill}{rgb}{0.000000,0.000000,0.000000}%
\pgfsetfillcolor{currentfill}%
\pgfsetlinewidth{1.003750pt}%
\definecolor{currentstroke}{rgb}{0.000000,0.000000,0.000000}%
\pgfsetstrokecolor{currentstroke}%
\pgfsetdash{}{0pt}%
\pgfpathmoveto{\pgfqpoint{2.139937in}{1.443291in}}%
\pgfpathcurveto{\pgfqpoint{2.150988in}{1.443291in}}{\pgfqpoint{2.161587in}{1.447682in}}{\pgfqpoint{2.169400in}{1.455495in}}%
\pgfpathcurveto{\pgfqpoint{2.177214in}{1.463309in}}{\pgfqpoint{2.181604in}{1.473908in}}{\pgfqpoint{2.181604in}{1.484958in}}%
\pgfpathcurveto{\pgfqpoint{2.181604in}{1.496008in}}{\pgfqpoint{2.177214in}{1.506607in}}{\pgfqpoint{2.169400in}{1.514421in}}%
\pgfpathcurveto{\pgfqpoint{2.161587in}{1.522235in}}{\pgfqpoint{2.150988in}{1.526625in}}{\pgfqpoint{2.139937in}{1.526625in}}%
\pgfpathcurveto{\pgfqpoint{2.128887in}{1.526625in}}{\pgfqpoint{2.118288in}{1.522235in}}{\pgfqpoint{2.110475in}{1.514421in}}%
\pgfpathcurveto{\pgfqpoint{2.102661in}{1.506607in}}{\pgfqpoint{2.098271in}{1.496008in}}{\pgfqpoint{2.098271in}{1.484958in}}%
\pgfpathcurveto{\pgfqpoint{2.098271in}{1.473908in}}{\pgfqpoint{2.102661in}{1.463309in}}{\pgfqpoint{2.110475in}{1.455495in}}%
\pgfpathcurveto{\pgfqpoint{2.118288in}{1.447682in}}{\pgfqpoint{2.128887in}{1.443291in}}{\pgfqpoint{2.139937in}{1.443291in}}%
\pgfpathclose%
\pgfusepath{stroke,fill}%
\end{pgfscope}%
\begin{pgfscope}%
\pgfpathrectangle{\pgfqpoint{0.375000in}{0.330000in}}{\pgfqpoint{2.325000in}{2.310000in}}%
\pgfusepath{clip}%
\pgfsetbuttcap%
\pgfsetroundjoin%
\definecolor{currentfill}{rgb}{0.000000,0.000000,0.000000}%
\pgfsetfillcolor{currentfill}%
\pgfsetlinewidth{1.003750pt}%
\definecolor{currentstroke}{rgb}{0.000000,0.000000,0.000000}%
\pgfsetstrokecolor{currentstroke}%
\pgfsetdash{}{0pt}%
\pgfpathmoveto{\pgfqpoint{2.139937in}{1.443291in}}%
\pgfpathcurveto{\pgfqpoint{2.150988in}{1.443291in}}{\pgfqpoint{2.161587in}{1.447682in}}{\pgfqpoint{2.169400in}{1.455495in}}%
\pgfpathcurveto{\pgfqpoint{2.177214in}{1.463309in}}{\pgfqpoint{2.181604in}{1.473908in}}{\pgfqpoint{2.181604in}{1.484958in}}%
\pgfpathcurveto{\pgfqpoint{2.181604in}{1.496008in}}{\pgfqpoint{2.177214in}{1.506607in}}{\pgfqpoint{2.169400in}{1.514421in}}%
\pgfpathcurveto{\pgfqpoint{2.161587in}{1.522235in}}{\pgfqpoint{2.150988in}{1.526625in}}{\pgfqpoint{2.139937in}{1.526625in}}%
\pgfpathcurveto{\pgfqpoint{2.128887in}{1.526625in}}{\pgfqpoint{2.118288in}{1.522235in}}{\pgfqpoint{2.110475in}{1.514421in}}%
\pgfpathcurveto{\pgfqpoint{2.102661in}{1.506607in}}{\pgfqpoint{2.098271in}{1.496008in}}{\pgfqpoint{2.098271in}{1.484958in}}%
\pgfpathcurveto{\pgfqpoint{2.098271in}{1.473908in}}{\pgfqpoint{2.102661in}{1.463309in}}{\pgfqpoint{2.110475in}{1.455495in}}%
\pgfpathcurveto{\pgfqpoint{2.118288in}{1.447682in}}{\pgfqpoint{2.128887in}{1.443291in}}{\pgfqpoint{2.139937in}{1.443291in}}%
\pgfpathclose%
\pgfusepath{stroke,fill}%
\end{pgfscope}%
\begin{pgfscope}%
\pgfpathrectangle{\pgfqpoint{0.375000in}{0.330000in}}{\pgfqpoint{2.325000in}{2.310000in}}%
\pgfusepath{clip}%
\pgfsetbuttcap%
\pgfsetroundjoin%
\definecolor{currentfill}{rgb}{0.000000,0.000000,0.000000}%
\pgfsetfillcolor{currentfill}%
\pgfsetlinewidth{1.003750pt}%
\definecolor{currentstroke}{rgb}{0.000000,0.000000,0.000000}%
\pgfsetstrokecolor{currentstroke}%
\pgfsetdash{}{0pt}%
\pgfpathmoveto{\pgfqpoint{2.139937in}{1.443291in}}%
\pgfpathcurveto{\pgfqpoint{2.150988in}{1.443291in}}{\pgfqpoint{2.161587in}{1.447682in}}{\pgfqpoint{2.169400in}{1.455495in}}%
\pgfpathcurveto{\pgfqpoint{2.177214in}{1.463309in}}{\pgfqpoint{2.181604in}{1.473908in}}{\pgfqpoint{2.181604in}{1.484958in}}%
\pgfpathcurveto{\pgfqpoint{2.181604in}{1.496008in}}{\pgfqpoint{2.177214in}{1.506607in}}{\pgfqpoint{2.169400in}{1.514421in}}%
\pgfpathcurveto{\pgfqpoint{2.161587in}{1.522235in}}{\pgfqpoint{2.150988in}{1.526625in}}{\pgfqpoint{2.139937in}{1.526625in}}%
\pgfpathcurveto{\pgfqpoint{2.128887in}{1.526625in}}{\pgfqpoint{2.118288in}{1.522235in}}{\pgfqpoint{2.110475in}{1.514421in}}%
\pgfpathcurveto{\pgfqpoint{2.102661in}{1.506607in}}{\pgfqpoint{2.098271in}{1.496008in}}{\pgfqpoint{2.098271in}{1.484958in}}%
\pgfpathcurveto{\pgfqpoint{2.098271in}{1.473908in}}{\pgfqpoint{2.102661in}{1.463309in}}{\pgfqpoint{2.110475in}{1.455495in}}%
\pgfpathcurveto{\pgfqpoint{2.118288in}{1.447682in}}{\pgfqpoint{2.128887in}{1.443291in}}{\pgfqpoint{2.139937in}{1.443291in}}%
\pgfpathclose%
\pgfusepath{stroke,fill}%
\end{pgfscope}%
\begin{pgfscope}%
\pgfpathrectangle{\pgfqpoint{0.375000in}{0.330000in}}{\pgfqpoint{2.325000in}{2.310000in}}%
\pgfusepath{clip}%
\pgfsetbuttcap%
\pgfsetroundjoin%
\definecolor{currentfill}{rgb}{0.000000,0.000000,0.000000}%
\pgfsetfillcolor{currentfill}%
\pgfsetlinewidth{1.003750pt}%
\definecolor{currentstroke}{rgb}{0.000000,0.000000,0.000000}%
\pgfsetstrokecolor{currentstroke}%
\pgfsetdash{}{0pt}%
\pgfpathmoveto{\pgfqpoint{2.139937in}{1.443291in}}%
\pgfpathcurveto{\pgfqpoint{2.150988in}{1.443291in}}{\pgfqpoint{2.161587in}{1.447682in}}{\pgfqpoint{2.169400in}{1.455495in}}%
\pgfpathcurveto{\pgfqpoint{2.177214in}{1.463309in}}{\pgfqpoint{2.181604in}{1.473908in}}{\pgfqpoint{2.181604in}{1.484958in}}%
\pgfpathcurveto{\pgfqpoint{2.181604in}{1.496008in}}{\pgfqpoint{2.177214in}{1.506607in}}{\pgfqpoint{2.169400in}{1.514421in}}%
\pgfpathcurveto{\pgfqpoint{2.161587in}{1.522235in}}{\pgfqpoint{2.150988in}{1.526625in}}{\pgfqpoint{2.139937in}{1.526625in}}%
\pgfpathcurveto{\pgfqpoint{2.128887in}{1.526625in}}{\pgfqpoint{2.118288in}{1.522235in}}{\pgfqpoint{2.110475in}{1.514421in}}%
\pgfpathcurveto{\pgfqpoint{2.102661in}{1.506607in}}{\pgfqpoint{2.098271in}{1.496008in}}{\pgfqpoint{2.098271in}{1.484958in}}%
\pgfpathcurveto{\pgfqpoint{2.098271in}{1.473908in}}{\pgfqpoint{2.102661in}{1.463309in}}{\pgfqpoint{2.110475in}{1.455495in}}%
\pgfpathcurveto{\pgfqpoint{2.118288in}{1.447682in}}{\pgfqpoint{2.128887in}{1.443291in}}{\pgfqpoint{2.139937in}{1.443291in}}%
\pgfpathclose%
\pgfusepath{stroke,fill}%
\end{pgfscope}%
\begin{pgfscope}%
\pgfpathrectangle{\pgfqpoint{0.375000in}{0.330000in}}{\pgfqpoint{2.325000in}{2.310000in}}%
\pgfusepath{clip}%
\pgfsetbuttcap%
\pgfsetroundjoin%
\definecolor{currentfill}{rgb}{0.000000,0.000000,0.000000}%
\pgfsetfillcolor{currentfill}%
\pgfsetlinewidth{1.003750pt}%
\definecolor{currentstroke}{rgb}{0.000000,0.000000,0.000000}%
\pgfsetstrokecolor{currentstroke}%
\pgfsetdash{}{0pt}%
\pgfpathmoveto{\pgfqpoint{2.139937in}{2.474583in}}%
\pgfpathcurveto{\pgfqpoint{2.150988in}{2.474583in}}{\pgfqpoint{2.161587in}{2.478974in}}{\pgfqpoint{2.169400in}{2.486787in}}%
\pgfpathcurveto{\pgfqpoint{2.177214in}{2.494601in}}{\pgfqpoint{2.181604in}{2.505200in}}{\pgfqpoint{2.181604in}{2.516250in}}%
\pgfpathcurveto{\pgfqpoint{2.181604in}{2.527300in}}{\pgfqpoint{2.177214in}{2.537899in}}{\pgfqpoint{2.169400in}{2.545713in}}%
\pgfpathcurveto{\pgfqpoint{2.161587in}{2.553526in}}{\pgfqpoint{2.150988in}{2.557917in}}{\pgfqpoint{2.139937in}{2.557917in}}%
\pgfpathcurveto{\pgfqpoint{2.128887in}{2.557917in}}{\pgfqpoint{2.118288in}{2.553526in}}{\pgfqpoint{2.110475in}{2.545713in}}%
\pgfpathcurveto{\pgfqpoint{2.102661in}{2.537899in}}{\pgfqpoint{2.098271in}{2.527300in}}{\pgfqpoint{2.098271in}{2.516250in}}%
\pgfpathcurveto{\pgfqpoint{2.098271in}{2.505200in}}{\pgfqpoint{2.102661in}{2.494601in}}{\pgfqpoint{2.110475in}{2.486787in}}%
\pgfpathcurveto{\pgfqpoint{2.118288in}{2.478974in}}{\pgfqpoint{2.128887in}{2.474583in}}{\pgfqpoint{2.139937in}{2.474583in}}%
\pgfpathclose%
\pgfusepath{stroke,fill}%
\end{pgfscope}%
\begin{pgfscope}%
\pgfpathrectangle{\pgfqpoint{0.375000in}{0.330000in}}{\pgfqpoint{2.325000in}{2.310000in}}%
\pgfusepath{clip}%
\pgfsetbuttcap%
\pgfsetroundjoin%
\definecolor{currentfill}{rgb}{0.000000,0.000000,0.000000}%
\pgfsetfillcolor{currentfill}%
\pgfsetlinewidth{1.003750pt}%
\definecolor{currentstroke}{rgb}{0.000000,0.000000,0.000000}%
\pgfsetstrokecolor{currentstroke}%
\pgfsetdash{}{0pt}%
\pgfpathmoveto{\pgfqpoint{2.139937in}{2.474583in}}%
\pgfpathcurveto{\pgfqpoint{2.150988in}{2.474583in}}{\pgfqpoint{2.161587in}{2.478974in}}{\pgfqpoint{2.169400in}{2.486787in}}%
\pgfpathcurveto{\pgfqpoint{2.177214in}{2.494601in}}{\pgfqpoint{2.181604in}{2.505200in}}{\pgfqpoint{2.181604in}{2.516250in}}%
\pgfpathcurveto{\pgfqpoint{2.181604in}{2.527300in}}{\pgfqpoint{2.177214in}{2.537899in}}{\pgfqpoint{2.169400in}{2.545713in}}%
\pgfpathcurveto{\pgfqpoint{2.161587in}{2.553526in}}{\pgfqpoint{2.150988in}{2.557917in}}{\pgfqpoint{2.139937in}{2.557917in}}%
\pgfpathcurveto{\pgfqpoint{2.128887in}{2.557917in}}{\pgfqpoint{2.118288in}{2.553526in}}{\pgfqpoint{2.110475in}{2.545713in}}%
\pgfpathcurveto{\pgfqpoint{2.102661in}{2.537899in}}{\pgfqpoint{2.098271in}{2.527300in}}{\pgfqpoint{2.098271in}{2.516250in}}%
\pgfpathcurveto{\pgfqpoint{2.098271in}{2.505200in}}{\pgfqpoint{2.102661in}{2.494601in}}{\pgfqpoint{2.110475in}{2.486787in}}%
\pgfpathcurveto{\pgfqpoint{2.118288in}{2.478974in}}{\pgfqpoint{2.128887in}{2.474583in}}{\pgfqpoint{2.139937in}{2.474583in}}%
\pgfpathclose%
\pgfusepath{stroke,fill}%
\end{pgfscope}%
\begin{pgfscope}%
\pgfpathrectangle{\pgfqpoint{0.375000in}{0.330000in}}{\pgfqpoint{2.325000in}{2.310000in}}%
\pgfusepath{clip}%
\pgfsetbuttcap%
\pgfsetroundjoin%
\definecolor{currentfill}{rgb}{0.000000,0.000000,0.000000}%
\pgfsetfillcolor{currentfill}%
\pgfsetlinewidth{1.003750pt}%
\definecolor{currentstroke}{rgb}{0.000000,0.000000,0.000000}%
\pgfsetstrokecolor{currentstroke}%
\pgfsetdash{}{0pt}%
\pgfpathmoveto{\pgfqpoint{2.139937in}{2.474583in}}%
\pgfpathcurveto{\pgfqpoint{2.150988in}{2.474583in}}{\pgfqpoint{2.161587in}{2.478974in}}{\pgfqpoint{2.169400in}{2.486787in}}%
\pgfpathcurveto{\pgfqpoint{2.177214in}{2.494601in}}{\pgfqpoint{2.181604in}{2.505200in}}{\pgfqpoint{2.181604in}{2.516250in}}%
\pgfpathcurveto{\pgfqpoint{2.181604in}{2.527300in}}{\pgfqpoint{2.177214in}{2.537899in}}{\pgfqpoint{2.169400in}{2.545713in}}%
\pgfpathcurveto{\pgfqpoint{2.161587in}{2.553526in}}{\pgfqpoint{2.150988in}{2.557917in}}{\pgfqpoint{2.139937in}{2.557917in}}%
\pgfpathcurveto{\pgfqpoint{2.128887in}{2.557917in}}{\pgfqpoint{2.118288in}{2.553526in}}{\pgfqpoint{2.110475in}{2.545713in}}%
\pgfpathcurveto{\pgfqpoint{2.102661in}{2.537899in}}{\pgfqpoint{2.098271in}{2.527300in}}{\pgfqpoint{2.098271in}{2.516250in}}%
\pgfpathcurveto{\pgfqpoint{2.098271in}{2.505200in}}{\pgfqpoint{2.102661in}{2.494601in}}{\pgfqpoint{2.110475in}{2.486787in}}%
\pgfpathcurveto{\pgfqpoint{2.118288in}{2.478974in}}{\pgfqpoint{2.128887in}{2.474583in}}{\pgfqpoint{2.139937in}{2.474583in}}%
\pgfpathclose%
\pgfusepath{stroke,fill}%
\end{pgfscope}%
\begin{pgfscope}%
\pgfpathrectangle{\pgfqpoint{0.375000in}{0.330000in}}{\pgfqpoint{2.325000in}{2.310000in}}%
\pgfusepath{clip}%
\pgfsetbuttcap%
\pgfsetroundjoin%
\definecolor{currentfill}{rgb}{0.000000,0.000000,0.000000}%
\pgfsetfillcolor{currentfill}%
\pgfsetlinewidth{1.003750pt}%
\definecolor{currentstroke}{rgb}{0.000000,0.000000,0.000000}%
\pgfsetstrokecolor{currentstroke}%
\pgfsetdash{}{0pt}%
\pgfpathmoveto{\pgfqpoint{2.139937in}{2.474583in}}%
\pgfpathcurveto{\pgfqpoint{2.150988in}{2.474583in}}{\pgfqpoint{2.161587in}{2.478974in}}{\pgfqpoint{2.169400in}{2.486787in}}%
\pgfpathcurveto{\pgfqpoint{2.177214in}{2.494601in}}{\pgfqpoint{2.181604in}{2.505200in}}{\pgfqpoint{2.181604in}{2.516250in}}%
\pgfpathcurveto{\pgfqpoint{2.181604in}{2.527300in}}{\pgfqpoint{2.177214in}{2.537899in}}{\pgfqpoint{2.169400in}{2.545713in}}%
\pgfpathcurveto{\pgfqpoint{2.161587in}{2.553526in}}{\pgfqpoint{2.150988in}{2.557917in}}{\pgfqpoint{2.139937in}{2.557917in}}%
\pgfpathcurveto{\pgfqpoint{2.128887in}{2.557917in}}{\pgfqpoint{2.118288in}{2.553526in}}{\pgfqpoint{2.110475in}{2.545713in}}%
\pgfpathcurveto{\pgfqpoint{2.102661in}{2.537899in}}{\pgfqpoint{2.098271in}{2.527300in}}{\pgfqpoint{2.098271in}{2.516250in}}%
\pgfpathcurveto{\pgfqpoint{2.098271in}{2.505200in}}{\pgfqpoint{2.102661in}{2.494601in}}{\pgfqpoint{2.110475in}{2.486787in}}%
\pgfpathcurveto{\pgfqpoint{2.118288in}{2.478974in}}{\pgfqpoint{2.128887in}{2.474583in}}{\pgfqpoint{2.139937in}{2.474583in}}%
\pgfpathclose%
\pgfusepath{stroke,fill}%
\end{pgfscope}%
\begin{pgfscope}%
\pgfpathrectangle{\pgfqpoint{0.375000in}{0.330000in}}{\pgfqpoint{2.325000in}{2.310000in}}%
\pgfusepath{clip}%
\pgfsetbuttcap%
\pgfsetroundjoin%
\definecolor{currentfill}{rgb}{0.000000,0.000000,0.000000}%
\pgfsetfillcolor{currentfill}%
\pgfsetlinewidth{1.003750pt}%
\definecolor{currentstroke}{rgb}{0.000000,0.000000,0.000000}%
\pgfsetstrokecolor{currentstroke}%
\pgfsetdash{}{0pt}%
\pgfpathmoveto{\pgfqpoint{2.139937in}{2.474583in}}%
\pgfpathcurveto{\pgfqpoint{2.150988in}{2.474583in}}{\pgfqpoint{2.161587in}{2.478974in}}{\pgfqpoint{2.169400in}{2.486787in}}%
\pgfpathcurveto{\pgfqpoint{2.177214in}{2.494601in}}{\pgfqpoint{2.181604in}{2.505200in}}{\pgfqpoint{2.181604in}{2.516250in}}%
\pgfpathcurveto{\pgfqpoint{2.181604in}{2.527300in}}{\pgfqpoint{2.177214in}{2.537899in}}{\pgfqpoint{2.169400in}{2.545713in}}%
\pgfpathcurveto{\pgfqpoint{2.161587in}{2.553526in}}{\pgfqpoint{2.150988in}{2.557917in}}{\pgfqpoint{2.139937in}{2.557917in}}%
\pgfpathcurveto{\pgfqpoint{2.128887in}{2.557917in}}{\pgfqpoint{2.118288in}{2.553526in}}{\pgfqpoint{2.110475in}{2.545713in}}%
\pgfpathcurveto{\pgfqpoint{2.102661in}{2.537899in}}{\pgfqpoint{2.098271in}{2.527300in}}{\pgfqpoint{2.098271in}{2.516250in}}%
\pgfpathcurveto{\pgfqpoint{2.098271in}{2.505200in}}{\pgfqpoint{2.102661in}{2.494601in}}{\pgfqpoint{2.110475in}{2.486787in}}%
\pgfpathcurveto{\pgfqpoint{2.118288in}{2.478974in}}{\pgfqpoint{2.128887in}{2.474583in}}{\pgfqpoint{2.139937in}{2.474583in}}%
\pgfpathclose%
\pgfusepath{stroke,fill}%
\end{pgfscope}%
\begin{pgfscope}%
\pgfpathrectangle{\pgfqpoint{0.375000in}{0.330000in}}{\pgfqpoint{2.325000in}{2.310000in}}%
\pgfusepath{clip}%
\pgfsetbuttcap%
\pgfsetroundjoin%
\definecolor{currentfill}{rgb}{0.000000,0.000000,0.000000}%
\pgfsetfillcolor{currentfill}%
\pgfsetlinewidth{1.003750pt}%
\definecolor{currentstroke}{rgb}{0.000000,0.000000,0.000000}%
\pgfsetstrokecolor{currentstroke}%
\pgfsetdash{}{0pt}%
\pgfpathmoveto{\pgfqpoint{2.139937in}{2.474583in}}%
\pgfpathcurveto{\pgfqpoint{2.150988in}{2.474583in}}{\pgfqpoint{2.161587in}{2.478974in}}{\pgfqpoint{2.169400in}{2.486787in}}%
\pgfpathcurveto{\pgfqpoint{2.177214in}{2.494601in}}{\pgfqpoint{2.181604in}{2.505200in}}{\pgfqpoint{2.181604in}{2.516250in}}%
\pgfpathcurveto{\pgfqpoint{2.181604in}{2.527300in}}{\pgfqpoint{2.177214in}{2.537899in}}{\pgfqpoint{2.169400in}{2.545713in}}%
\pgfpathcurveto{\pgfqpoint{2.161587in}{2.553526in}}{\pgfqpoint{2.150988in}{2.557917in}}{\pgfqpoint{2.139937in}{2.557917in}}%
\pgfpathcurveto{\pgfqpoint{2.128887in}{2.557917in}}{\pgfqpoint{2.118288in}{2.553526in}}{\pgfqpoint{2.110475in}{2.545713in}}%
\pgfpathcurveto{\pgfqpoint{2.102661in}{2.537899in}}{\pgfqpoint{2.098271in}{2.527300in}}{\pgfqpoint{2.098271in}{2.516250in}}%
\pgfpathcurveto{\pgfqpoint{2.098271in}{2.505200in}}{\pgfqpoint{2.102661in}{2.494601in}}{\pgfqpoint{2.110475in}{2.486787in}}%
\pgfpathcurveto{\pgfqpoint{2.118288in}{2.478974in}}{\pgfqpoint{2.128887in}{2.474583in}}{\pgfqpoint{2.139937in}{2.474583in}}%
\pgfpathclose%
\pgfusepath{stroke,fill}%
\end{pgfscope}%
\begin{pgfscope}%
\pgfpathrectangle{\pgfqpoint{0.375000in}{0.330000in}}{\pgfqpoint{2.325000in}{2.310000in}}%
\pgfusepath{clip}%
\pgfsetbuttcap%
\pgfsetroundjoin%
\definecolor{currentfill}{rgb}{0.000000,0.000000,0.000000}%
\pgfsetfillcolor{currentfill}%
\pgfsetlinewidth{1.003750pt}%
\definecolor{currentstroke}{rgb}{0.000000,0.000000,0.000000}%
\pgfsetstrokecolor{currentstroke}%
\pgfsetdash{}{0pt}%
\pgfpathmoveto{\pgfqpoint{2.139937in}{1.443291in}}%
\pgfpathcurveto{\pgfqpoint{2.150988in}{1.443291in}}{\pgfqpoint{2.161587in}{1.447682in}}{\pgfqpoint{2.169400in}{1.455495in}}%
\pgfpathcurveto{\pgfqpoint{2.177214in}{1.463309in}}{\pgfqpoint{2.181604in}{1.473908in}}{\pgfqpoint{2.181604in}{1.484958in}}%
\pgfpathcurveto{\pgfqpoint{2.181604in}{1.496008in}}{\pgfqpoint{2.177214in}{1.506607in}}{\pgfqpoint{2.169400in}{1.514421in}}%
\pgfpathcurveto{\pgfqpoint{2.161587in}{1.522235in}}{\pgfqpoint{2.150988in}{1.526625in}}{\pgfqpoint{2.139937in}{1.526625in}}%
\pgfpathcurveto{\pgfqpoint{2.128887in}{1.526625in}}{\pgfqpoint{2.118288in}{1.522235in}}{\pgfqpoint{2.110475in}{1.514421in}}%
\pgfpathcurveto{\pgfqpoint{2.102661in}{1.506607in}}{\pgfqpoint{2.098271in}{1.496008in}}{\pgfqpoint{2.098271in}{1.484958in}}%
\pgfpathcurveto{\pgfqpoint{2.098271in}{1.473908in}}{\pgfqpoint{2.102661in}{1.463309in}}{\pgfqpoint{2.110475in}{1.455495in}}%
\pgfpathcurveto{\pgfqpoint{2.118288in}{1.447682in}}{\pgfqpoint{2.128887in}{1.443291in}}{\pgfqpoint{2.139937in}{1.443291in}}%
\pgfpathclose%
\pgfusepath{stroke,fill}%
\end{pgfscope}%
\begin{pgfscope}%
\pgfpathrectangle{\pgfqpoint{0.375000in}{0.330000in}}{\pgfqpoint{2.325000in}{2.310000in}}%
\pgfusepath{clip}%
\pgfsetbuttcap%
\pgfsetroundjoin%
\definecolor{currentfill}{rgb}{0.000000,0.000000,0.000000}%
\pgfsetfillcolor{currentfill}%
\pgfsetlinewidth{1.003750pt}%
\definecolor{currentstroke}{rgb}{0.000000,0.000000,0.000000}%
\pgfsetstrokecolor{currentstroke}%
\pgfsetdash{}{0pt}%
\pgfpathmoveto{\pgfqpoint{2.139937in}{2.474583in}}%
\pgfpathcurveto{\pgfqpoint{2.150988in}{2.474583in}}{\pgfqpoint{2.161587in}{2.478974in}}{\pgfqpoint{2.169400in}{2.486787in}}%
\pgfpathcurveto{\pgfqpoint{2.177214in}{2.494601in}}{\pgfqpoint{2.181604in}{2.505200in}}{\pgfqpoint{2.181604in}{2.516250in}}%
\pgfpathcurveto{\pgfqpoint{2.181604in}{2.527300in}}{\pgfqpoint{2.177214in}{2.537899in}}{\pgfqpoint{2.169400in}{2.545713in}}%
\pgfpathcurveto{\pgfqpoint{2.161587in}{2.553526in}}{\pgfqpoint{2.150988in}{2.557917in}}{\pgfqpoint{2.139937in}{2.557917in}}%
\pgfpathcurveto{\pgfqpoint{2.128887in}{2.557917in}}{\pgfqpoint{2.118288in}{2.553526in}}{\pgfqpoint{2.110475in}{2.545713in}}%
\pgfpathcurveto{\pgfqpoint{2.102661in}{2.537899in}}{\pgfqpoint{2.098271in}{2.527300in}}{\pgfqpoint{2.098271in}{2.516250in}}%
\pgfpathcurveto{\pgfqpoint{2.098271in}{2.505200in}}{\pgfqpoint{2.102661in}{2.494601in}}{\pgfqpoint{2.110475in}{2.486787in}}%
\pgfpathcurveto{\pgfqpoint{2.118288in}{2.478974in}}{\pgfqpoint{2.128887in}{2.474583in}}{\pgfqpoint{2.139937in}{2.474583in}}%
\pgfpathclose%
\pgfusepath{stroke,fill}%
\end{pgfscope}%
\begin{pgfscope}%
\pgfpathrectangle{\pgfqpoint{0.375000in}{0.330000in}}{\pgfqpoint{2.325000in}{2.310000in}}%
\pgfusepath{clip}%
\pgfsetbuttcap%
\pgfsetroundjoin%
\definecolor{currentfill}{rgb}{0.000000,0.000000,0.000000}%
\pgfsetfillcolor{currentfill}%
\pgfsetlinewidth{1.003750pt}%
\definecolor{currentstroke}{rgb}{0.000000,0.000000,0.000000}%
\pgfsetstrokecolor{currentstroke}%
\pgfsetdash{}{0pt}%
\pgfpathmoveto{\pgfqpoint{2.139937in}{1.443291in}}%
\pgfpathcurveto{\pgfqpoint{2.150988in}{1.443291in}}{\pgfqpoint{2.161587in}{1.447682in}}{\pgfqpoint{2.169400in}{1.455495in}}%
\pgfpathcurveto{\pgfqpoint{2.177214in}{1.463309in}}{\pgfqpoint{2.181604in}{1.473908in}}{\pgfqpoint{2.181604in}{1.484958in}}%
\pgfpathcurveto{\pgfqpoint{2.181604in}{1.496008in}}{\pgfqpoint{2.177214in}{1.506607in}}{\pgfqpoint{2.169400in}{1.514421in}}%
\pgfpathcurveto{\pgfqpoint{2.161587in}{1.522235in}}{\pgfqpoint{2.150988in}{1.526625in}}{\pgfqpoint{2.139937in}{1.526625in}}%
\pgfpathcurveto{\pgfqpoint{2.128887in}{1.526625in}}{\pgfqpoint{2.118288in}{1.522235in}}{\pgfqpoint{2.110475in}{1.514421in}}%
\pgfpathcurveto{\pgfqpoint{2.102661in}{1.506607in}}{\pgfqpoint{2.098271in}{1.496008in}}{\pgfqpoint{2.098271in}{1.484958in}}%
\pgfpathcurveto{\pgfqpoint{2.098271in}{1.473908in}}{\pgfqpoint{2.102661in}{1.463309in}}{\pgfqpoint{2.110475in}{1.455495in}}%
\pgfpathcurveto{\pgfqpoint{2.118288in}{1.447682in}}{\pgfqpoint{2.128887in}{1.443291in}}{\pgfqpoint{2.139937in}{1.443291in}}%
\pgfpathclose%
\pgfusepath{stroke,fill}%
\end{pgfscope}%
\begin{pgfscope}%
\pgfpathrectangle{\pgfqpoint{0.375000in}{0.330000in}}{\pgfqpoint{2.325000in}{2.310000in}}%
\pgfusepath{clip}%
\pgfsetbuttcap%
\pgfsetroundjoin%
\definecolor{currentfill}{rgb}{0.000000,0.000000,0.000000}%
\pgfsetfillcolor{currentfill}%
\pgfsetlinewidth{1.003750pt}%
\definecolor{currentstroke}{rgb}{0.000000,0.000000,0.000000}%
\pgfsetstrokecolor{currentstroke}%
\pgfsetdash{}{0pt}%
\pgfpathmoveto{\pgfqpoint{2.139937in}{1.443291in}}%
\pgfpathcurveto{\pgfqpoint{2.150988in}{1.443291in}}{\pgfqpoint{2.161587in}{1.447682in}}{\pgfqpoint{2.169400in}{1.455495in}}%
\pgfpathcurveto{\pgfqpoint{2.177214in}{1.463309in}}{\pgfqpoint{2.181604in}{1.473908in}}{\pgfqpoint{2.181604in}{1.484958in}}%
\pgfpathcurveto{\pgfqpoint{2.181604in}{1.496008in}}{\pgfqpoint{2.177214in}{1.506607in}}{\pgfqpoint{2.169400in}{1.514421in}}%
\pgfpathcurveto{\pgfqpoint{2.161587in}{1.522235in}}{\pgfqpoint{2.150988in}{1.526625in}}{\pgfqpoint{2.139937in}{1.526625in}}%
\pgfpathcurveto{\pgfqpoint{2.128887in}{1.526625in}}{\pgfqpoint{2.118288in}{1.522235in}}{\pgfqpoint{2.110475in}{1.514421in}}%
\pgfpathcurveto{\pgfqpoint{2.102661in}{1.506607in}}{\pgfqpoint{2.098271in}{1.496008in}}{\pgfqpoint{2.098271in}{1.484958in}}%
\pgfpathcurveto{\pgfqpoint{2.098271in}{1.473908in}}{\pgfqpoint{2.102661in}{1.463309in}}{\pgfqpoint{2.110475in}{1.455495in}}%
\pgfpathcurveto{\pgfqpoint{2.118288in}{1.447682in}}{\pgfqpoint{2.128887in}{1.443291in}}{\pgfqpoint{2.139937in}{1.443291in}}%
\pgfpathclose%
\pgfusepath{stroke,fill}%
\end{pgfscope}%
\begin{pgfscope}%
\pgfpathrectangle{\pgfqpoint{0.375000in}{0.330000in}}{\pgfqpoint{2.325000in}{2.310000in}}%
\pgfusepath{clip}%
\pgfsetbuttcap%
\pgfsetroundjoin%
\definecolor{currentfill}{rgb}{0.000000,0.000000,0.000000}%
\pgfsetfillcolor{currentfill}%
\pgfsetlinewidth{1.003750pt}%
\definecolor{currentstroke}{rgb}{0.000000,0.000000,0.000000}%
\pgfsetstrokecolor{currentstroke}%
\pgfsetdash{}{0pt}%
\pgfpathmoveto{\pgfqpoint{2.139937in}{2.474583in}}%
\pgfpathcurveto{\pgfqpoint{2.150988in}{2.474583in}}{\pgfqpoint{2.161587in}{2.478974in}}{\pgfqpoint{2.169400in}{2.486787in}}%
\pgfpathcurveto{\pgfqpoint{2.177214in}{2.494601in}}{\pgfqpoint{2.181604in}{2.505200in}}{\pgfqpoint{2.181604in}{2.516250in}}%
\pgfpathcurveto{\pgfqpoint{2.181604in}{2.527300in}}{\pgfqpoint{2.177214in}{2.537899in}}{\pgfqpoint{2.169400in}{2.545713in}}%
\pgfpathcurveto{\pgfqpoint{2.161587in}{2.553526in}}{\pgfqpoint{2.150988in}{2.557917in}}{\pgfqpoint{2.139937in}{2.557917in}}%
\pgfpathcurveto{\pgfqpoint{2.128887in}{2.557917in}}{\pgfqpoint{2.118288in}{2.553526in}}{\pgfqpoint{2.110475in}{2.545713in}}%
\pgfpathcurveto{\pgfqpoint{2.102661in}{2.537899in}}{\pgfqpoint{2.098271in}{2.527300in}}{\pgfqpoint{2.098271in}{2.516250in}}%
\pgfpathcurveto{\pgfqpoint{2.098271in}{2.505200in}}{\pgfqpoint{2.102661in}{2.494601in}}{\pgfqpoint{2.110475in}{2.486787in}}%
\pgfpathcurveto{\pgfqpoint{2.118288in}{2.478974in}}{\pgfqpoint{2.128887in}{2.474583in}}{\pgfqpoint{2.139937in}{2.474583in}}%
\pgfpathclose%
\pgfusepath{stroke,fill}%
\end{pgfscope}%
\begin{pgfscope}%
\pgfpathrectangle{\pgfqpoint{0.375000in}{0.330000in}}{\pgfqpoint{2.325000in}{2.310000in}}%
\pgfusepath{clip}%
\pgfsetbuttcap%
\pgfsetroundjoin%
\definecolor{currentfill}{rgb}{0.000000,0.000000,0.000000}%
\pgfsetfillcolor{currentfill}%
\pgfsetlinewidth{1.003750pt}%
\definecolor{currentstroke}{rgb}{0.000000,0.000000,0.000000}%
\pgfsetstrokecolor{currentstroke}%
\pgfsetdash{}{0pt}%
\pgfpathmoveto{\pgfqpoint{2.139937in}{2.474583in}}%
\pgfpathcurveto{\pgfqpoint{2.150988in}{2.474583in}}{\pgfqpoint{2.161587in}{2.478974in}}{\pgfqpoint{2.169400in}{2.486787in}}%
\pgfpathcurveto{\pgfqpoint{2.177214in}{2.494601in}}{\pgfqpoint{2.181604in}{2.505200in}}{\pgfqpoint{2.181604in}{2.516250in}}%
\pgfpathcurveto{\pgfqpoint{2.181604in}{2.527300in}}{\pgfqpoint{2.177214in}{2.537899in}}{\pgfqpoint{2.169400in}{2.545713in}}%
\pgfpathcurveto{\pgfqpoint{2.161587in}{2.553526in}}{\pgfqpoint{2.150988in}{2.557917in}}{\pgfqpoint{2.139937in}{2.557917in}}%
\pgfpathcurveto{\pgfqpoint{2.128887in}{2.557917in}}{\pgfqpoint{2.118288in}{2.553526in}}{\pgfqpoint{2.110475in}{2.545713in}}%
\pgfpathcurveto{\pgfqpoint{2.102661in}{2.537899in}}{\pgfqpoint{2.098271in}{2.527300in}}{\pgfqpoint{2.098271in}{2.516250in}}%
\pgfpathcurveto{\pgfqpoint{2.098271in}{2.505200in}}{\pgfqpoint{2.102661in}{2.494601in}}{\pgfqpoint{2.110475in}{2.486787in}}%
\pgfpathcurveto{\pgfqpoint{2.118288in}{2.478974in}}{\pgfqpoint{2.128887in}{2.474583in}}{\pgfqpoint{2.139937in}{2.474583in}}%
\pgfpathclose%
\pgfusepath{stroke,fill}%
\end{pgfscope}%
\begin{pgfscope}%
\pgfpathrectangle{\pgfqpoint{0.375000in}{0.330000in}}{\pgfqpoint{2.325000in}{2.310000in}}%
\pgfusepath{clip}%
\pgfsetbuttcap%
\pgfsetroundjoin%
\definecolor{currentfill}{rgb}{0.000000,0.000000,0.000000}%
\pgfsetfillcolor{currentfill}%
\pgfsetlinewidth{1.003750pt}%
\definecolor{currentstroke}{rgb}{0.000000,0.000000,0.000000}%
\pgfsetstrokecolor{currentstroke}%
\pgfsetdash{}{0pt}%
\pgfpathmoveto{\pgfqpoint{2.139937in}{1.443291in}}%
\pgfpathcurveto{\pgfqpoint{2.150988in}{1.443291in}}{\pgfqpoint{2.161587in}{1.447682in}}{\pgfqpoint{2.169400in}{1.455495in}}%
\pgfpathcurveto{\pgfqpoint{2.177214in}{1.463309in}}{\pgfqpoint{2.181604in}{1.473908in}}{\pgfqpoint{2.181604in}{1.484958in}}%
\pgfpathcurveto{\pgfqpoint{2.181604in}{1.496008in}}{\pgfqpoint{2.177214in}{1.506607in}}{\pgfqpoint{2.169400in}{1.514421in}}%
\pgfpathcurveto{\pgfqpoint{2.161587in}{1.522235in}}{\pgfqpoint{2.150988in}{1.526625in}}{\pgfqpoint{2.139937in}{1.526625in}}%
\pgfpathcurveto{\pgfqpoint{2.128887in}{1.526625in}}{\pgfqpoint{2.118288in}{1.522235in}}{\pgfqpoint{2.110475in}{1.514421in}}%
\pgfpathcurveto{\pgfqpoint{2.102661in}{1.506607in}}{\pgfqpoint{2.098271in}{1.496008in}}{\pgfqpoint{2.098271in}{1.484958in}}%
\pgfpathcurveto{\pgfqpoint{2.098271in}{1.473908in}}{\pgfqpoint{2.102661in}{1.463309in}}{\pgfqpoint{2.110475in}{1.455495in}}%
\pgfpathcurveto{\pgfqpoint{2.118288in}{1.447682in}}{\pgfqpoint{2.128887in}{1.443291in}}{\pgfqpoint{2.139937in}{1.443291in}}%
\pgfpathclose%
\pgfusepath{stroke,fill}%
\end{pgfscope}%
\begin{pgfscope}%
\pgfpathrectangle{\pgfqpoint{0.375000in}{0.330000in}}{\pgfqpoint{2.325000in}{2.310000in}}%
\pgfusepath{clip}%
\pgfsetbuttcap%
\pgfsetroundjoin%
\definecolor{currentfill}{rgb}{0.000000,0.000000,0.000000}%
\pgfsetfillcolor{currentfill}%
\pgfsetlinewidth{1.003750pt}%
\definecolor{currentstroke}{rgb}{0.000000,0.000000,0.000000}%
\pgfsetstrokecolor{currentstroke}%
\pgfsetdash{}{0pt}%
\pgfpathmoveto{\pgfqpoint{2.139937in}{2.474583in}}%
\pgfpathcurveto{\pgfqpoint{2.150988in}{2.474583in}}{\pgfqpoint{2.161587in}{2.478974in}}{\pgfqpoint{2.169400in}{2.486787in}}%
\pgfpathcurveto{\pgfqpoint{2.177214in}{2.494601in}}{\pgfqpoint{2.181604in}{2.505200in}}{\pgfqpoint{2.181604in}{2.516250in}}%
\pgfpathcurveto{\pgfqpoint{2.181604in}{2.527300in}}{\pgfqpoint{2.177214in}{2.537899in}}{\pgfqpoint{2.169400in}{2.545713in}}%
\pgfpathcurveto{\pgfqpoint{2.161587in}{2.553526in}}{\pgfqpoint{2.150988in}{2.557917in}}{\pgfqpoint{2.139937in}{2.557917in}}%
\pgfpathcurveto{\pgfqpoint{2.128887in}{2.557917in}}{\pgfqpoint{2.118288in}{2.553526in}}{\pgfqpoint{2.110475in}{2.545713in}}%
\pgfpathcurveto{\pgfqpoint{2.102661in}{2.537899in}}{\pgfqpoint{2.098271in}{2.527300in}}{\pgfqpoint{2.098271in}{2.516250in}}%
\pgfpathcurveto{\pgfqpoint{2.098271in}{2.505200in}}{\pgfqpoint{2.102661in}{2.494601in}}{\pgfqpoint{2.110475in}{2.486787in}}%
\pgfpathcurveto{\pgfqpoint{2.118288in}{2.478974in}}{\pgfqpoint{2.128887in}{2.474583in}}{\pgfqpoint{2.139937in}{2.474583in}}%
\pgfpathclose%
\pgfusepath{stroke,fill}%
\end{pgfscope}%
\begin{pgfscope}%
\pgfpathrectangle{\pgfqpoint{0.375000in}{0.330000in}}{\pgfqpoint{2.325000in}{2.310000in}}%
\pgfusepath{clip}%
\pgfsetbuttcap%
\pgfsetroundjoin%
\definecolor{currentfill}{rgb}{0.000000,0.000000,0.000000}%
\pgfsetfillcolor{currentfill}%
\pgfsetlinewidth{1.003750pt}%
\definecolor{currentstroke}{rgb}{0.000000,0.000000,0.000000}%
\pgfsetstrokecolor{currentstroke}%
\pgfsetdash{}{0pt}%
\pgfpathmoveto{\pgfqpoint{2.139937in}{2.474583in}}%
\pgfpathcurveto{\pgfqpoint{2.150988in}{2.474583in}}{\pgfqpoint{2.161587in}{2.478974in}}{\pgfqpoint{2.169400in}{2.486787in}}%
\pgfpathcurveto{\pgfqpoint{2.177214in}{2.494601in}}{\pgfqpoint{2.181604in}{2.505200in}}{\pgfqpoint{2.181604in}{2.516250in}}%
\pgfpathcurveto{\pgfqpoint{2.181604in}{2.527300in}}{\pgfqpoint{2.177214in}{2.537899in}}{\pgfqpoint{2.169400in}{2.545713in}}%
\pgfpathcurveto{\pgfqpoint{2.161587in}{2.553526in}}{\pgfqpoint{2.150988in}{2.557917in}}{\pgfqpoint{2.139937in}{2.557917in}}%
\pgfpathcurveto{\pgfqpoint{2.128887in}{2.557917in}}{\pgfqpoint{2.118288in}{2.553526in}}{\pgfqpoint{2.110475in}{2.545713in}}%
\pgfpathcurveto{\pgfqpoint{2.102661in}{2.537899in}}{\pgfqpoint{2.098271in}{2.527300in}}{\pgfqpoint{2.098271in}{2.516250in}}%
\pgfpathcurveto{\pgfqpoint{2.098271in}{2.505200in}}{\pgfqpoint{2.102661in}{2.494601in}}{\pgfqpoint{2.110475in}{2.486787in}}%
\pgfpathcurveto{\pgfqpoint{2.118288in}{2.478974in}}{\pgfqpoint{2.128887in}{2.474583in}}{\pgfqpoint{2.139937in}{2.474583in}}%
\pgfpathclose%
\pgfusepath{stroke,fill}%
\end{pgfscope}%
\begin{pgfscope}%
\pgfpathrectangle{\pgfqpoint{0.375000in}{0.330000in}}{\pgfqpoint{2.325000in}{2.310000in}}%
\pgfusepath{clip}%
\pgfsetbuttcap%
\pgfsetroundjoin%
\definecolor{currentfill}{rgb}{0.000000,0.000000,0.000000}%
\pgfsetfillcolor{currentfill}%
\pgfsetlinewidth{1.003750pt}%
\definecolor{currentstroke}{rgb}{0.000000,0.000000,0.000000}%
\pgfsetstrokecolor{currentstroke}%
\pgfsetdash{}{0pt}%
\pgfpathmoveto{\pgfqpoint{2.139937in}{2.474583in}}%
\pgfpathcurveto{\pgfqpoint{2.150988in}{2.474583in}}{\pgfqpoint{2.161587in}{2.478974in}}{\pgfqpoint{2.169400in}{2.486787in}}%
\pgfpathcurveto{\pgfqpoint{2.177214in}{2.494601in}}{\pgfqpoint{2.181604in}{2.505200in}}{\pgfqpoint{2.181604in}{2.516250in}}%
\pgfpathcurveto{\pgfqpoint{2.181604in}{2.527300in}}{\pgfqpoint{2.177214in}{2.537899in}}{\pgfqpoint{2.169400in}{2.545713in}}%
\pgfpathcurveto{\pgfqpoint{2.161587in}{2.553526in}}{\pgfqpoint{2.150988in}{2.557917in}}{\pgfqpoint{2.139937in}{2.557917in}}%
\pgfpathcurveto{\pgfqpoint{2.128887in}{2.557917in}}{\pgfqpoint{2.118288in}{2.553526in}}{\pgfqpoint{2.110475in}{2.545713in}}%
\pgfpathcurveto{\pgfqpoint{2.102661in}{2.537899in}}{\pgfqpoint{2.098271in}{2.527300in}}{\pgfqpoint{2.098271in}{2.516250in}}%
\pgfpathcurveto{\pgfqpoint{2.098271in}{2.505200in}}{\pgfqpoint{2.102661in}{2.494601in}}{\pgfqpoint{2.110475in}{2.486787in}}%
\pgfpathcurveto{\pgfqpoint{2.118288in}{2.478974in}}{\pgfqpoint{2.128887in}{2.474583in}}{\pgfqpoint{2.139937in}{2.474583in}}%
\pgfpathclose%
\pgfusepath{stroke,fill}%
\end{pgfscope}%
\begin{pgfscope}%
\pgfpathrectangle{\pgfqpoint{0.375000in}{0.330000in}}{\pgfqpoint{2.325000in}{2.310000in}}%
\pgfusepath{clip}%
\pgfsetbuttcap%
\pgfsetroundjoin%
\definecolor{currentfill}{rgb}{0.000000,0.000000,0.000000}%
\pgfsetfillcolor{currentfill}%
\pgfsetlinewidth{1.003750pt}%
\definecolor{currentstroke}{rgb}{0.000000,0.000000,0.000000}%
\pgfsetstrokecolor{currentstroke}%
\pgfsetdash{}{0pt}%
\pgfpathmoveto{\pgfqpoint{2.139937in}{2.474583in}}%
\pgfpathcurveto{\pgfqpoint{2.150988in}{2.474583in}}{\pgfqpoint{2.161587in}{2.478974in}}{\pgfqpoint{2.169400in}{2.486787in}}%
\pgfpathcurveto{\pgfqpoint{2.177214in}{2.494601in}}{\pgfqpoint{2.181604in}{2.505200in}}{\pgfqpoint{2.181604in}{2.516250in}}%
\pgfpathcurveto{\pgfqpoint{2.181604in}{2.527300in}}{\pgfqpoint{2.177214in}{2.537899in}}{\pgfqpoint{2.169400in}{2.545713in}}%
\pgfpathcurveto{\pgfqpoint{2.161587in}{2.553526in}}{\pgfqpoint{2.150988in}{2.557917in}}{\pgfqpoint{2.139937in}{2.557917in}}%
\pgfpathcurveto{\pgfqpoint{2.128887in}{2.557917in}}{\pgfqpoint{2.118288in}{2.553526in}}{\pgfqpoint{2.110475in}{2.545713in}}%
\pgfpathcurveto{\pgfqpoint{2.102661in}{2.537899in}}{\pgfqpoint{2.098271in}{2.527300in}}{\pgfqpoint{2.098271in}{2.516250in}}%
\pgfpathcurveto{\pgfqpoint{2.098271in}{2.505200in}}{\pgfqpoint{2.102661in}{2.494601in}}{\pgfqpoint{2.110475in}{2.486787in}}%
\pgfpathcurveto{\pgfqpoint{2.118288in}{2.478974in}}{\pgfqpoint{2.128887in}{2.474583in}}{\pgfqpoint{2.139937in}{2.474583in}}%
\pgfpathclose%
\pgfusepath{stroke,fill}%
\end{pgfscope}%
\begin{pgfscope}%
\pgfpathrectangle{\pgfqpoint{0.375000in}{0.330000in}}{\pgfqpoint{2.325000in}{2.310000in}}%
\pgfusepath{clip}%
\pgfsetbuttcap%
\pgfsetroundjoin%
\definecolor{currentfill}{rgb}{0.000000,0.000000,0.000000}%
\pgfsetfillcolor{currentfill}%
\pgfsetlinewidth{1.003750pt}%
\definecolor{currentstroke}{rgb}{0.000000,0.000000,0.000000}%
\pgfsetstrokecolor{currentstroke}%
\pgfsetdash{}{0pt}%
\pgfpathmoveto{\pgfqpoint{2.139937in}{2.474583in}}%
\pgfpathcurveto{\pgfqpoint{2.150988in}{2.474583in}}{\pgfqpoint{2.161587in}{2.478974in}}{\pgfqpoint{2.169400in}{2.486787in}}%
\pgfpathcurveto{\pgfqpoint{2.177214in}{2.494601in}}{\pgfqpoint{2.181604in}{2.505200in}}{\pgfqpoint{2.181604in}{2.516250in}}%
\pgfpathcurveto{\pgfqpoint{2.181604in}{2.527300in}}{\pgfqpoint{2.177214in}{2.537899in}}{\pgfqpoint{2.169400in}{2.545713in}}%
\pgfpathcurveto{\pgfqpoint{2.161587in}{2.553526in}}{\pgfqpoint{2.150988in}{2.557917in}}{\pgfqpoint{2.139937in}{2.557917in}}%
\pgfpathcurveto{\pgfqpoint{2.128887in}{2.557917in}}{\pgfqpoint{2.118288in}{2.553526in}}{\pgfqpoint{2.110475in}{2.545713in}}%
\pgfpathcurveto{\pgfqpoint{2.102661in}{2.537899in}}{\pgfqpoint{2.098271in}{2.527300in}}{\pgfqpoint{2.098271in}{2.516250in}}%
\pgfpathcurveto{\pgfqpoint{2.098271in}{2.505200in}}{\pgfqpoint{2.102661in}{2.494601in}}{\pgfqpoint{2.110475in}{2.486787in}}%
\pgfpathcurveto{\pgfqpoint{2.118288in}{2.478974in}}{\pgfqpoint{2.128887in}{2.474583in}}{\pgfqpoint{2.139937in}{2.474583in}}%
\pgfpathclose%
\pgfusepath{stroke,fill}%
\end{pgfscope}%
\begin{pgfscope}%
\pgfpathrectangle{\pgfqpoint{0.375000in}{0.330000in}}{\pgfqpoint{2.325000in}{2.310000in}}%
\pgfusepath{clip}%
\pgfsetbuttcap%
\pgfsetroundjoin%
\definecolor{currentfill}{rgb}{0.000000,0.000000,0.000000}%
\pgfsetfillcolor{currentfill}%
\pgfsetlinewidth{1.003750pt}%
\definecolor{currentstroke}{rgb}{0.000000,0.000000,0.000000}%
\pgfsetstrokecolor{currentstroke}%
\pgfsetdash{}{0pt}%
\pgfpathmoveto{\pgfqpoint{2.139937in}{1.443291in}}%
\pgfpathcurveto{\pgfqpoint{2.150988in}{1.443291in}}{\pgfqpoint{2.161587in}{1.447682in}}{\pgfqpoint{2.169400in}{1.455495in}}%
\pgfpathcurveto{\pgfqpoint{2.177214in}{1.463309in}}{\pgfqpoint{2.181604in}{1.473908in}}{\pgfqpoint{2.181604in}{1.484958in}}%
\pgfpathcurveto{\pgfqpoint{2.181604in}{1.496008in}}{\pgfqpoint{2.177214in}{1.506607in}}{\pgfqpoint{2.169400in}{1.514421in}}%
\pgfpathcurveto{\pgfqpoint{2.161587in}{1.522235in}}{\pgfqpoint{2.150988in}{1.526625in}}{\pgfqpoint{2.139937in}{1.526625in}}%
\pgfpathcurveto{\pgfqpoint{2.128887in}{1.526625in}}{\pgfqpoint{2.118288in}{1.522235in}}{\pgfqpoint{2.110475in}{1.514421in}}%
\pgfpathcurveto{\pgfqpoint{2.102661in}{1.506607in}}{\pgfqpoint{2.098271in}{1.496008in}}{\pgfqpoint{2.098271in}{1.484958in}}%
\pgfpathcurveto{\pgfqpoint{2.098271in}{1.473908in}}{\pgfqpoint{2.102661in}{1.463309in}}{\pgfqpoint{2.110475in}{1.455495in}}%
\pgfpathcurveto{\pgfqpoint{2.118288in}{1.447682in}}{\pgfqpoint{2.128887in}{1.443291in}}{\pgfqpoint{2.139937in}{1.443291in}}%
\pgfpathclose%
\pgfusepath{stroke,fill}%
\end{pgfscope}%
\begin{pgfscope}%
\pgfpathrectangle{\pgfqpoint{0.375000in}{0.330000in}}{\pgfqpoint{2.325000in}{2.310000in}}%
\pgfusepath{clip}%
\pgfsetbuttcap%
\pgfsetroundjoin%
\definecolor{currentfill}{rgb}{0.000000,0.000000,0.000000}%
\pgfsetfillcolor{currentfill}%
\pgfsetlinewidth{1.003750pt}%
\definecolor{currentstroke}{rgb}{0.000000,0.000000,0.000000}%
\pgfsetstrokecolor{currentstroke}%
\pgfsetdash{}{0pt}%
\pgfpathmoveto{\pgfqpoint{2.139937in}{2.474583in}}%
\pgfpathcurveto{\pgfqpoint{2.150988in}{2.474583in}}{\pgfqpoint{2.161587in}{2.478974in}}{\pgfqpoint{2.169400in}{2.486787in}}%
\pgfpathcurveto{\pgfqpoint{2.177214in}{2.494601in}}{\pgfqpoint{2.181604in}{2.505200in}}{\pgfqpoint{2.181604in}{2.516250in}}%
\pgfpathcurveto{\pgfqpoint{2.181604in}{2.527300in}}{\pgfqpoint{2.177214in}{2.537899in}}{\pgfqpoint{2.169400in}{2.545713in}}%
\pgfpathcurveto{\pgfqpoint{2.161587in}{2.553526in}}{\pgfqpoint{2.150988in}{2.557917in}}{\pgfqpoint{2.139937in}{2.557917in}}%
\pgfpathcurveto{\pgfqpoint{2.128887in}{2.557917in}}{\pgfqpoint{2.118288in}{2.553526in}}{\pgfqpoint{2.110475in}{2.545713in}}%
\pgfpathcurveto{\pgfqpoint{2.102661in}{2.537899in}}{\pgfqpoint{2.098271in}{2.527300in}}{\pgfqpoint{2.098271in}{2.516250in}}%
\pgfpathcurveto{\pgfqpoint{2.098271in}{2.505200in}}{\pgfqpoint{2.102661in}{2.494601in}}{\pgfqpoint{2.110475in}{2.486787in}}%
\pgfpathcurveto{\pgfqpoint{2.118288in}{2.478974in}}{\pgfqpoint{2.128887in}{2.474583in}}{\pgfqpoint{2.139937in}{2.474583in}}%
\pgfpathclose%
\pgfusepath{stroke,fill}%
\end{pgfscope}%
\begin{pgfscope}%
\pgfpathrectangle{\pgfqpoint{0.375000in}{0.330000in}}{\pgfqpoint{2.325000in}{2.310000in}}%
\pgfusepath{clip}%
\pgfsetbuttcap%
\pgfsetroundjoin%
\definecolor{currentfill}{rgb}{0.000000,0.000000,0.000000}%
\pgfsetfillcolor{currentfill}%
\pgfsetlinewidth{1.003750pt}%
\definecolor{currentstroke}{rgb}{0.000000,0.000000,0.000000}%
\pgfsetstrokecolor{currentstroke}%
\pgfsetdash{}{0pt}%
\pgfpathmoveto{\pgfqpoint{2.139937in}{2.474583in}}%
\pgfpathcurveto{\pgfqpoint{2.150988in}{2.474583in}}{\pgfqpoint{2.161587in}{2.478974in}}{\pgfqpoint{2.169400in}{2.486787in}}%
\pgfpathcurveto{\pgfqpoint{2.177214in}{2.494601in}}{\pgfqpoint{2.181604in}{2.505200in}}{\pgfqpoint{2.181604in}{2.516250in}}%
\pgfpathcurveto{\pgfqpoint{2.181604in}{2.527300in}}{\pgfqpoint{2.177214in}{2.537899in}}{\pgfqpoint{2.169400in}{2.545713in}}%
\pgfpathcurveto{\pgfqpoint{2.161587in}{2.553526in}}{\pgfqpoint{2.150988in}{2.557917in}}{\pgfqpoint{2.139937in}{2.557917in}}%
\pgfpathcurveto{\pgfqpoint{2.128887in}{2.557917in}}{\pgfqpoint{2.118288in}{2.553526in}}{\pgfqpoint{2.110475in}{2.545713in}}%
\pgfpathcurveto{\pgfqpoint{2.102661in}{2.537899in}}{\pgfqpoint{2.098271in}{2.527300in}}{\pgfqpoint{2.098271in}{2.516250in}}%
\pgfpathcurveto{\pgfqpoint{2.098271in}{2.505200in}}{\pgfqpoint{2.102661in}{2.494601in}}{\pgfqpoint{2.110475in}{2.486787in}}%
\pgfpathcurveto{\pgfqpoint{2.118288in}{2.478974in}}{\pgfqpoint{2.128887in}{2.474583in}}{\pgfqpoint{2.139937in}{2.474583in}}%
\pgfpathclose%
\pgfusepath{stroke,fill}%
\end{pgfscope}%
\begin{pgfscope}%
\pgfpathrectangle{\pgfqpoint{0.375000in}{0.330000in}}{\pgfqpoint{2.325000in}{2.310000in}}%
\pgfusepath{clip}%
\pgfsetbuttcap%
\pgfsetroundjoin%
\definecolor{currentfill}{rgb}{0.000000,0.000000,0.000000}%
\pgfsetfillcolor{currentfill}%
\pgfsetlinewidth{1.003750pt}%
\definecolor{currentstroke}{rgb}{0.000000,0.000000,0.000000}%
\pgfsetstrokecolor{currentstroke}%
\pgfsetdash{}{0pt}%
\pgfpathmoveto{\pgfqpoint{2.139937in}{1.443291in}}%
\pgfpathcurveto{\pgfqpoint{2.150988in}{1.443291in}}{\pgfqpoint{2.161587in}{1.447682in}}{\pgfqpoint{2.169400in}{1.455495in}}%
\pgfpathcurveto{\pgfqpoint{2.177214in}{1.463309in}}{\pgfqpoint{2.181604in}{1.473908in}}{\pgfqpoint{2.181604in}{1.484958in}}%
\pgfpathcurveto{\pgfqpoint{2.181604in}{1.496008in}}{\pgfqpoint{2.177214in}{1.506607in}}{\pgfqpoint{2.169400in}{1.514421in}}%
\pgfpathcurveto{\pgfqpoint{2.161587in}{1.522235in}}{\pgfqpoint{2.150988in}{1.526625in}}{\pgfqpoint{2.139937in}{1.526625in}}%
\pgfpathcurveto{\pgfqpoint{2.128887in}{1.526625in}}{\pgfqpoint{2.118288in}{1.522235in}}{\pgfqpoint{2.110475in}{1.514421in}}%
\pgfpathcurveto{\pgfqpoint{2.102661in}{1.506607in}}{\pgfqpoint{2.098271in}{1.496008in}}{\pgfqpoint{2.098271in}{1.484958in}}%
\pgfpathcurveto{\pgfqpoint{2.098271in}{1.473908in}}{\pgfqpoint{2.102661in}{1.463309in}}{\pgfqpoint{2.110475in}{1.455495in}}%
\pgfpathcurveto{\pgfqpoint{2.118288in}{1.447682in}}{\pgfqpoint{2.128887in}{1.443291in}}{\pgfqpoint{2.139937in}{1.443291in}}%
\pgfpathclose%
\pgfusepath{stroke,fill}%
\end{pgfscope}%
\begin{pgfscope}%
\pgfpathrectangle{\pgfqpoint{0.375000in}{0.330000in}}{\pgfqpoint{2.325000in}{2.310000in}}%
\pgfusepath{clip}%
\pgfsetbuttcap%
\pgfsetroundjoin%
\definecolor{currentfill}{rgb}{0.000000,0.000000,0.000000}%
\pgfsetfillcolor{currentfill}%
\pgfsetlinewidth{1.003750pt}%
\definecolor{currentstroke}{rgb}{0.000000,0.000000,0.000000}%
\pgfsetstrokecolor{currentstroke}%
\pgfsetdash{}{0pt}%
\pgfpathmoveto{\pgfqpoint{2.139937in}{1.443291in}}%
\pgfpathcurveto{\pgfqpoint{2.150988in}{1.443291in}}{\pgfqpoint{2.161587in}{1.447682in}}{\pgfqpoint{2.169400in}{1.455495in}}%
\pgfpathcurveto{\pgfqpoint{2.177214in}{1.463309in}}{\pgfqpoint{2.181604in}{1.473908in}}{\pgfqpoint{2.181604in}{1.484958in}}%
\pgfpathcurveto{\pgfqpoint{2.181604in}{1.496008in}}{\pgfqpoint{2.177214in}{1.506607in}}{\pgfqpoint{2.169400in}{1.514421in}}%
\pgfpathcurveto{\pgfqpoint{2.161587in}{1.522235in}}{\pgfqpoint{2.150988in}{1.526625in}}{\pgfqpoint{2.139937in}{1.526625in}}%
\pgfpathcurveto{\pgfqpoint{2.128887in}{1.526625in}}{\pgfqpoint{2.118288in}{1.522235in}}{\pgfqpoint{2.110475in}{1.514421in}}%
\pgfpathcurveto{\pgfqpoint{2.102661in}{1.506607in}}{\pgfqpoint{2.098271in}{1.496008in}}{\pgfqpoint{2.098271in}{1.484958in}}%
\pgfpathcurveto{\pgfqpoint{2.098271in}{1.473908in}}{\pgfqpoint{2.102661in}{1.463309in}}{\pgfqpoint{2.110475in}{1.455495in}}%
\pgfpathcurveto{\pgfqpoint{2.118288in}{1.447682in}}{\pgfqpoint{2.128887in}{1.443291in}}{\pgfqpoint{2.139937in}{1.443291in}}%
\pgfpathclose%
\pgfusepath{stroke,fill}%
\end{pgfscope}%
\begin{pgfscope}%
\pgfpathrectangle{\pgfqpoint{0.375000in}{0.330000in}}{\pgfqpoint{2.325000in}{2.310000in}}%
\pgfusepath{clip}%
\pgfsetbuttcap%
\pgfsetroundjoin%
\definecolor{currentfill}{rgb}{0.000000,0.000000,0.000000}%
\pgfsetfillcolor{currentfill}%
\pgfsetlinewidth{1.003750pt}%
\definecolor{currentstroke}{rgb}{0.000000,0.000000,0.000000}%
\pgfsetstrokecolor{currentstroke}%
\pgfsetdash{}{0pt}%
\pgfpathmoveto{\pgfqpoint{2.139937in}{1.443291in}}%
\pgfpathcurveto{\pgfqpoint{2.150988in}{1.443291in}}{\pgfqpoint{2.161587in}{1.447682in}}{\pgfqpoint{2.169400in}{1.455495in}}%
\pgfpathcurveto{\pgfqpoint{2.177214in}{1.463309in}}{\pgfqpoint{2.181604in}{1.473908in}}{\pgfqpoint{2.181604in}{1.484958in}}%
\pgfpathcurveto{\pgfqpoint{2.181604in}{1.496008in}}{\pgfqpoint{2.177214in}{1.506607in}}{\pgfqpoint{2.169400in}{1.514421in}}%
\pgfpathcurveto{\pgfqpoint{2.161587in}{1.522235in}}{\pgfqpoint{2.150988in}{1.526625in}}{\pgfqpoint{2.139937in}{1.526625in}}%
\pgfpathcurveto{\pgfqpoint{2.128887in}{1.526625in}}{\pgfqpoint{2.118288in}{1.522235in}}{\pgfqpoint{2.110475in}{1.514421in}}%
\pgfpathcurveto{\pgfqpoint{2.102661in}{1.506607in}}{\pgfqpoint{2.098271in}{1.496008in}}{\pgfqpoint{2.098271in}{1.484958in}}%
\pgfpathcurveto{\pgfqpoint{2.098271in}{1.473908in}}{\pgfqpoint{2.102661in}{1.463309in}}{\pgfqpoint{2.110475in}{1.455495in}}%
\pgfpathcurveto{\pgfqpoint{2.118288in}{1.447682in}}{\pgfqpoint{2.128887in}{1.443291in}}{\pgfqpoint{2.139937in}{1.443291in}}%
\pgfpathclose%
\pgfusepath{stroke,fill}%
\end{pgfscope}%
\begin{pgfscope}%
\pgfpathrectangle{\pgfqpoint{0.375000in}{0.330000in}}{\pgfqpoint{2.325000in}{2.310000in}}%
\pgfusepath{clip}%
\pgfsetbuttcap%
\pgfsetroundjoin%
\definecolor{currentfill}{rgb}{0.000000,0.000000,0.000000}%
\pgfsetfillcolor{currentfill}%
\pgfsetlinewidth{1.003750pt}%
\definecolor{currentstroke}{rgb}{0.000000,0.000000,0.000000}%
\pgfsetstrokecolor{currentstroke}%
\pgfsetdash{}{0pt}%
\pgfpathmoveto{\pgfqpoint{2.139937in}{1.443291in}}%
\pgfpathcurveto{\pgfqpoint{2.150988in}{1.443291in}}{\pgfqpoint{2.161587in}{1.447682in}}{\pgfqpoint{2.169400in}{1.455495in}}%
\pgfpathcurveto{\pgfqpoint{2.177214in}{1.463309in}}{\pgfqpoint{2.181604in}{1.473908in}}{\pgfqpoint{2.181604in}{1.484958in}}%
\pgfpathcurveto{\pgfqpoint{2.181604in}{1.496008in}}{\pgfqpoint{2.177214in}{1.506607in}}{\pgfqpoint{2.169400in}{1.514421in}}%
\pgfpathcurveto{\pgfqpoint{2.161587in}{1.522235in}}{\pgfqpoint{2.150988in}{1.526625in}}{\pgfqpoint{2.139937in}{1.526625in}}%
\pgfpathcurveto{\pgfqpoint{2.128887in}{1.526625in}}{\pgfqpoint{2.118288in}{1.522235in}}{\pgfqpoint{2.110475in}{1.514421in}}%
\pgfpathcurveto{\pgfqpoint{2.102661in}{1.506607in}}{\pgfqpoint{2.098271in}{1.496008in}}{\pgfqpoint{2.098271in}{1.484958in}}%
\pgfpathcurveto{\pgfqpoint{2.098271in}{1.473908in}}{\pgfqpoint{2.102661in}{1.463309in}}{\pgfqpoint{2.110475in}{1.455495in}}%
\pgfpathcurveto{\pgfqpoint{2.118288in}{1.447682in}}{\pgfqpoint{2.128887in}{1.443291in}}{\pgfqpoint{2.139937in}{1.443291in}}%
\pgfpathclose%
\pgfusepath{stroke,fill}%
\end{pgfscope}%
\begin{pgfscope}%
\pgfpathrectangle{\pgfqpoint{0.375000in}{0.330000in}}{\pgfqpoint{2.325000in}{2.310000in}}%
\pgfusepath{clip}%
\pgfsetbuttcap%
\pgfsetroundjoin%
\definecolor{currentfill}{rgb}{0.000000,0.000000,0.000000}%
\pgfsetfillcolor{currentfill}%
\pgfsetlinewidth{1.003750pt}%
\definecolor{currentstroke}{rgb}{0.000000,0.000000,0.000000}%
\pgfsetstrokecolor{currentstroke}%
\pgfsetdash{}{0pt}%
\pgfpathmoveto{\pgfqpoint{2.139937in}{2.474583in}}%
\pgfpathcurveto{\pgfqpoint{2.150988in}{2.474583in}}{\pgfqpoint{2.161587in}{2.478974in}}{\pgfqpoint{2.169400in}{2.486787in}}%
\pgfpathcurveto{\pgfqpoint{2.177214in}{2.494601in}}{\pgfqpoint{2.181604in}{2.505200in}}{\pgfqpoint{2.181604in}{2.516250in}}%
\pgfpathcurveto{\pgfqpoint{2.181604in}{2.527300in}}{\pgfqpoint{2.177214in}{2.537899in}}{\pgfqpoint{2.169400in}{2.545713in}}%
\pgfpathcurveto{\pgfqpoint{2.161587in}{2.553526in}}{\pgfqpoint{2.150988in}{2.557917in}}{\pgfqpoint{2.139937in}{2.557917in}}%
\pgfpathcurveto{\pgfqpoint{2.128887in}{2.557917in}}{\pgfqpoint{2.118288in}{2.553526in}}{\pgfqpoint{2.110475in}{2.545713in}}%
\pgfpathcurveto{\pgfqpoint{2.102661in}{2.537899in}}{\pgfqpoint{2.098271in}{2.527300in}}{\pgfqpoint{2.098271in}{2.516250in}}%
\pgfpathcurveto{\pgfqpoint{2.098271in}{2.505200in}}{\pgfqpoint{2.102661in}{2.494601in}}{\pgfqpoint{2.110475in}{2.486787in}}%
\pgfpathcurveto{\pgfqpoint{2.118288in}{2.478974in}}{\pgfqpoint{2.128887in}{2.474583in}}{\pgfqpoint{2.139937in}{2.474583in}}%
\pgfpathclose%
\pgfusepath{stroke,fill}%
\end{pgfscope}%
\begin{pgfscope}%
\pgfpathrectangle{\pgfqpoint{0.375000in}{0.330000in}}{\pgfqpoint{2.325000in}{2.310000in}}%
\pgfusepath{clip}%
\pgfsetbuttcap%
\pgfsetroundjoin%
\definecolor{currentfill}{rgb}{0.000000,0.000000,0.000000}%
\pgfsetfillcolor{currentfill}%
\pgfsetlinewidth{1.003750pt}%
\definecolor{currentstroke}{rgb}{0.000000,0.000000,0.000000}%
\pgfsetstrokecolor{currentstroke}%
\pgfsetdash{}{0pt}%
\pgfpathmoveto{\pgfqpoint{2.139937in}{2.474583in}}%
\pgfpathcurveto{\pgfqpoint{2.150988in}{2.474583in}}{\pgfqpoint{2.161587in}{2.478974in}}{\pgfqpoint{2.169400in}{2.486787in}}%
\pgfpathcurveto{\pgfqpoint{2.177214in}{2.494601in}}{\pgfqpoint{2.181604in}{2.505200in}}{\pgfqpoint{2.181604in}{2.516250in}}%
\pgfpathcurveto{\pgfqpoint{2.181604in}{2.527300in}}{\pgfqpoint{2.177214in}{2.537899in}}{\pgfqpoint{2.169400in}{2.545713in}}%
\pgfpathcurveto{\pgfqpoint{2.161587in}{2.553526in}}{\pgfqpoint{2.150988in}{2.557917in}}{\pgfqpoint{2.139937in}{2.557917in}}%
\pgfpathcurveto{\pgfqpoint{2.128887in}{2.557917in}}{\pgfqpoint{2.118288in}{2.553526in}}{\pgfqpoint{2.110475in}{2.545713in}}%
\pgfpathcurveto{\pgfqpoint{2.102661in}{2.537899in}}{\pgfqpoint{2.098271in}{2.527300in}}{\pgfqpoint{2.098271in}{2.516250in}}%
\pgfpathcurveto{\pgfqpoint{2.098271in}{2.505200in}}{\pgfqpoint{2.102661in}{2.494601in}}{\pgfqpoint{2.110475in}{2.486787in}}%
\pgfpathcurveto{\pgfqpoint{2.118288in}{2.478974in}}{\pgfqpoint{2.128887in}{2.474583in}}{\pgfqpoint{2.139937in}{2.474583in}}%
\pgfpathclose%
\pgfusepath{stroke,fill}%
\end{pgfscope}%
\begin{pgfscope}%
\pgfpathrectangle{\pgfqpoint{0.375000in}{0.330000in}}{\pgfqpoint{2.325000in}{2.310000in}}%
\pgfusepath{clip}%
\pgfsetbuttcap%
\pgfsetroundjoin%
\definecolor{currentfill}{rgb}{0.000000,0.000000,0.000000}%
\pgfsetfillcolor{currentfill}%
\pgfsetlinewidth{1.003750pt}%
\definecolor{currentstroke}{rgb}{0.000000,0.000000,0.000000}%
\pgfsetstrokecolor{currentstroke}%
\pgfsetdash{}{0pt}%
\pgfpathmoveto{\pgfqpoint{2.139937in}{1.443291in}}%
\pgfpathcurveto{\pgfqpoint{2.150988in}{1.443291in}}{\pgfqpoint{2.161587in}{1.447682in}}{\pgfqpoint{2.169400in}{1.455495in}}%
\pgfpathcurveto{\pgfqpoint{2.177214in}{1.463309in}}{\pgfqpoint{2.181604in}{1.473908in}}{\pgfqpoint{2.181604in}{1.484958in}}%
\pgfpathcurveto{\pgfqpoint{2.181604in}{1.496008in}}{\pgfqpoint{2.177214in}{1.506607in}}{\pgfqpoint{2.169400in}{1.514421in}}%
\pgfpathcurveto{\pgfqpoint{2.161587in}{1.522235in}}{\pgfqpoint{2.150988in}{1.526625in}}{\pgfqpoint{2.139937in}{1.526625in}}%
\pgfpathcurveto{\pgfqpoint{2.128887in}{1.526625in}}{\pgfqpoint{2.118288in}{1.522235in}}{\pgfqpoint{2.110475in}{1.514421in}}%
\pgfpathcurveto{\pgfqpoint{2.102661in}{1.506607in}}{\pgfqpoint{2.098271in}{1.496008in}}{\pgfqpoint{2.098271in}{1.484958in}}%
\pgfpathcurveto{\pgfqpoint{2.098271in}{1.473908in}}{\pgfqpoint{2.102661in}{1.463309in}}{\pgfqpoint{2.110475in}{1.455495in}}%
\pgfpathcurveto{\pgfqpoint{2.118288in}{1.447682in}}{\pgfqpoint{2.128887in}{1.443291in}}{\pgfqpoint{2.139937in}{1.443291in}}%
\pgfpathclose%
\pgfusepath{stroke,fill}%
\end{pgfscope}%
\begin{pgfscope}%
\pgfpathrectangle{\pgfqpoint{0.375000in}{0.330000in}}{\pgfqpoint{2.325000in}{2.310000in}}%
\pgfusepath{clip}%
\pgfsetbuttcap%
\pgfsetroundjoin%
\definecolor{currentfill}{rgb}{0.000000,0.000000,0.000000}%
\pgfsetfillcolor{currentfill}%
\pgfsetlinewidth{1.003750pt}%
\definecolor{currentstroke}{rgb}{0.000000,0.000000,0.000000}%
\pgfsetstrokecolor{currentstroke}%
\pgfsetdash{}{0pt}%
\pgfpathmoveto{\pgfqpoint{2.139937in}{2.474583in}}%
\pgfpathcurveto{\pgfqpoint{2.150988in}{2.474583in}}{\pgfqpoint{2.161587in}{2.478974in}}{\pgfqpoint{2.169400in}{2.486787in}}%
\pgfpathcurveto{\pgfqpoint{2.177214in}{2.494601in}}{\pgfqpoint{2.181604in}{2.505200in}}{\pgfqpoint{2.181604in}{2.516250in}}%
\pgfpathcurveto{\pgfqpoint{2.181604in}{2.527300in}}{\pgfqpoint{2.177214in}{2.537899in}}{\pgfqpoint{2.169400in}{2.545713in}}%
\pgfpathcurveto{\pgfqpoint{2.161587in}{2.553526in}}{\pgfqpoint{2.150988in}{2.557917in}}{\pgfqpoint{2.139937in}{2.557917in}}%
\pgfpathcurveto{\pgfqpoint{2.128887in}{2.557917in}}{\pgfqpoint{2.118288in}{2.553526in}}{\pgfqpoint{2.110475in}{2.545713in}}%
\pgfpathcurveto{\pgfqpoint{2.102661in}{2.537899in}}{\pgfqpoint{2.098271in}{2.527300in}}{\pgfqpoint{2.098271in}{2.516250in}}%
\pgfpathcurveto{\pgfqpoint{2.098271in}{2.505200in}}{\pgfqpoint{2.102661in}{2.494601in}}{\pgfqpoint{2.110475in}{2.486787in}}%
\pgfpathcurveto{\pgfqpoint{2.118288in}{2.478974in}}{\pgfqpoint{2.128887in}{2.474583in}}{\pgfqpoint{2.139937in}{2.474583in}}%
\pgfpathclose%
\pgfusepath{stroke,fill}%
\end{pgfscope}%
\begin{pgfscope}%
\pgfpathrectangle{\pgfqpoint{0.375000in}{0.330000in}}{\pgfqpoint{2.325000in}{2.310000in}}%
\pgfusepath{clip}%
\pgfsetbuttcap%
\pgfsetroundjoin%
\definecolor{currentfill}{rgb}{0.000000,0.000000,0.000000}%
\pgfsetfillcolor{currentfill}%
\pgfsetlinewidth{1.003750pt}%
\definecolor{currentstroke}{rgb}{0.000000,0.000000,0.000000}%
\pgfsetstrokecolor{currentstroke}%
\pgfsetdash{}{0pt}%
\pgfpathmoveto{\pgfqpoint{2.139937in}{2.474583in}}%
\pgfpathcurveto{\pgfqpoint{2.150988in}{2.474583in}}{\pgfqpoint{2.161587in}{2.478974in}}{\pgfqpoint{2.169400in}{2.486787in}}%
\pgfpathcurveto{\pgfqpoint{2.177214in}{2.494601in}}{\pgfqpoint{2.181604in}{2.505200in}}{\pgfqpoint{2.181604in}{2.516250in}}%
\pgfpathcurveto{\pgfqpoint{2.181604in}{2.527300in}}{\pgfqpoint{2.177214in}{2.537899in}}{\pgfqpoint{2.169400in}{2.545713in}}%
\pgfpathcurveto{\pgfqpoint{2.161587in}{2.553526in}}{\pgfqpoint{2.150988in}{2.557917in}}{\pgfqpoint{2.139937in}{2.557917in}}%
\pgfpathcurveto{\pgfqpoint{2.128887in}{2.557917in}}{\pgfqpoint{2.118288in}{2.553526in}}{\pgfqpoint{2.110475in}{2.545713in}}%
\pgfpathcurveto{\pgfqpoint{2.102661in}{2.537899in}}{\pgfqpoint{2.098271in}{2.527300in}}{\pgfqpoint{2.098271in}{2.516250in}}%
\pgfpathcurveto{\pgfqpoint{2.098271in}{2.505200in}}{\pgfqpoint{2.102661in}{2.494601in}}{\pgfqpoint{2.110475in}{2.486787in}}%
\pgfpathcurveto{\pgfqpoint{2.118288in}{2.478974in}}{\pgfqpoint{2.128887in}{2.474583in}}{\pgfqpoint{2.139937in}{2.474583in}}%
\pgfpathclose%
\pgfusepath{stroke,fill}%
\end{pgfscope}%
\begin{pgfscope}%
\pgfpathrectangle{\pgfqpoint{0.375000in}{0.330000in}}{\pgfqpoint{2.325000in}{2.310000in}}%
\pgfusepath{clip}%
\pgfsetbuttcap%
\pgfsetroundjoin%
\definecolor{currentfill}{rgb}{0.000000,0.000000,0.000000}%
\pgfsetfillcolor{currentfill}%
\pgfsetlinewidth{1.003750pt}%
\definecolor{currentstroke}{rgb}{0.000000,0.000000,0.000000}%
\pgfsetstrokecolor{currentstroke}%
\pgfsetdash{}{0pt}%
\pgfpathmoveto{\pgfqpoint{2.139937in}{2.474583in}}%
\pgfpathcurveto{\pgfqpoint{2.150988in}{2.474583in}}{\pgfqpoint{2.161587in}{2.478974in}}{\pgfqpoint{2.169400in}{2.486787in}}%
\pgfpathcurveto{\pgfqpoint{2.177214in}{2.494601in}}{\pgfqpoint{2.181604in}{2.505200in}}{\pgfqpoint{2.181604in}{2.516250in}}%
\pgfpathcurveto{\pgfqpoint{2.181604in}{2.527300in}}{\pgfqpoint{2.177214in}{2.537899in}}{\pgfqpoint{2.169400in}{2.545713in}}%
\pgfpathcurveto{\pgfqpoint{2.161587in}{2.553526in}}{\pgfqpoint{2.150988in}{2.557917in}}{\pgfqpoint{2.139937in}{2.557917in}}%
\pgfpathcurveto{\pgfqpoint{2.128887in}{2.557917in}}{\pgfqpoint{2.118288in}{2.553526in}}{\pgfqpoint{2.110475in}{2.545713in}}%
\pgfpathcurveto{\pgfqpoint{2.102661in}{2.537899in}}{\pgfqpoint{2.098271in}{2.527300in}}{\pgfqpoint{2.098271in}{2.516250in}}%
\pgfpathcurveto{\pgfqpoint{2.098271in}{2.505200in}}{\pgfqpoint{2.102661in}{2.494601in}}{\pgfqpoint{2.110475in}{2.486787in}}%
\pgfpathcurveto{\pgfqpoint{2.118288in}{2.478974in}}{\pgfqpoint{2.128887in}{2.474583in}}{\pgfqpoint{2.139937in}{2.474583in}}%
\pgfpathclose%
\pgfusepath{stroke,fill}%
\end{pgfscope}%
\begin{pgfscope}%
\pgfpathrectangle{\pgfqpoint{0.375000in}{0.330000in}}{\pgfqpoint{2.325000in}{2.310000in}}%
\pgfusepath{clip}%
\pgfsetbuttcap%
\pgfsetroundjoin%
\definecolor{currentfill}{rgb}{0.000000,0.000000,0.000000}%
\pgfsetfillcolor{currentfill}%
\pgfsetlinewidth{1.003750pt}%
\definecolor{currentstroke}{rgb}{0.000000,0.000000,0.000000}%
\pgfsetstrokecolor{currentstroke}%
\pgfsetdash{}{0pt}%
\pgfpathmoveto{\pgfqpoint{2.139937in}{1.443291in}}%
\pgfpathcurveto{\pgfqpoint{2.150988in}{1.443291in}}{\pgfqpoint{2.161587in}{1.447682in}}{\pgfqpoint{2.169400in}{1.455495in}}%
\pgfpathcurveto{\pgfqpoint{2.177214in}{1.463309in}}{\pgfqpoint{2.181604in}{1.473908in}}{\pgfqpoint{2.181604in}{1.484958in}}%
\pgfpathcurveto{\pgfqpoint{2.181604in}{1.496008in}}{\pgfqpoint{2.177214in}{1.506607in}}{\pgfqpoint{2.169400in}{1.514421in}}%
\pgfpathcurveto{\pgfqpoint{2.161587in}{1.522235in}}{\pgfqpoint{2.150988in}{1.526625in}}{\pgfqpoint{2.139937in}{1.526625in}}%
\pgfpathcurveto{\pgfqpoint{2.128887in}{1.526625in}}{\pgfqpoint{2.118288in}{1.522235in}}{\pgfqpoint{2.110475in}{1.514421in}}%
\pgfpathcurveto{\pgfqpoint{2.102661in}{1.506607in}}{\pgfqpoint{2.098271in}{1.496008in}}{\pgfqpoint{2.098271in}{1.484958in}}%
\pgfpathcurveto{\pgfqpoint{2.098271in}{1.473908in}}{\pgfqpoint{2.102661in}{1.463309in}}{\pgfqpoint{2.110475in}{1.455495in}}%
\pgfpathcurveto{\pgfqpoint{2.118288in}{1.447682in}}{\pgfqpoint{2.128887in}{1.443291in}}{\pgfqpoint{2.139937in}{1.443291in}}%
\pgfpathclose%
\pgfusepath{stroke,fill}%
\end{pgfscope}%
\begin{pgfscope}%
\pgfpathrectangle{\pgfqpoint{0.375000in}{0.330000in}}{\pgfqpoint{2.325000in}{2.310000in}}%
\pgfusepath{clip}%
\pgfsetbuttcap%
\pgfsetroundjoin%
\definecolor{currentfill}{rgb}{0.000000,0.000000,0.000000}%
\pgfsetfillcolor{currentfill}%
\pgfsetlinewidth{1.003750pt}%
\definecolor{currentstroke}{rgb}{0.000000,0.000000,0.000000}%
\pgfsetstrokecolor{currentstroke}%
\pgfsetdash{}{0pt}%
\pgfpathmoveto{\pgfqpoint{2.139937in}{1.443291in}}%
\pgfpathcurveto{\pgfqpoint{2.150988in}{1.443291in}}{\pgfqpoint{2.161587in}{1.447682in}}{\pgfqpoint{2.169400in}{1.455495in}}%
\pgfpathcurveto{\pgfqpoint{2.177214in}{1.463309in}}{\pgfqpoint{2.181604in}{1.473908in}}{\pgfqpoint{2.181604in}{1.484958in}}%
\pgfpathcurveto{\pgfqpoint{2.181604in}{1.496008in}}{\pgfqpoint{2.177214in}{1.506607in}}{\pgfqpoint{2.169400in}{1.514421in}}%
\pgfpathcurveto{\pgfqpoint{2.161587in}{1.522235in}}{\pgfqpoint{2.150988in}{1.526625in}}{\pgfqpoint{2.139937in}{1.526625in}}%
\pgfpathcurveto{\pgfqpoint{2.128887in}{1.526625in}}{\pgfqpoint{2.118288in}{1.522235in}}{\pgfqpoint{2.110475in}{1.514421in}}%
\pgfpathcurveto{\pgfqpoint{2.102661in}{1.506607in}}{\pgfqpoint{2.098271in}{1.496008in}}{\pgfqpoint{2.098271in}{1.484958in}}%
\pgfpathcurveto{\pgfqpoint{2.098271in}{1.473908in}}{\pgfqpoint{2.102661in}{1.463309in}}{\pgfqpoint{2.110475in}{1.455495in}}%
\pgfpathcurveto{\pgfqpoint{2.118288in}{1.447682in}}{\pgfqpoint{2.128887in}{1.443291in}}{\pgfqpoint{2.139937in}{1.443291in}}%
\pgfpathclose%
\pgfusepath{stroke,fill}%
\end{pgfscope}%
\begin{pgfscope}%
\pgfpathrectangle{\pgfqpoint{0.375000in}{0.330000in}}{\pgfqpoint{2.325000in}{2.310000in}}%
\pgfusepath{clip}%
\pgfsetbuttcap%
\pgfsetroundjoin%
\definecolor{currentfill}{rgb}{0.000000,0.000000,0.000000}%
\pgfsetfillcolor{currentfill}%
\pgfsetlinewidth{1.003750pt}%
\definecolor{currentstroke}{rgb}{0.000000,0.000000,0.000000}%
\pgfsetstrokecolor{currentstroke}%
\pgfsetdash{}{0pt}%
\pgfpathmoveto{\pgfqpoint{2.139937in}{2.474583in}}%
\pgfpathcurveto{\pgfqpoint{2.150988in}{2.474583in}}{\pgfqpoint{2.161587in}{2.478974in}}{\pgfqpoint{2.169400in}{2.486787in}}%
\pgfpathcurveto{\pgfqpoint{2.177214in}{2.494601in}}{\pgfqpoint{2.181604in}{2.505200in}}{\pgfqpoint{2.181604in}{2.516250in}}%
\pgfpathcurveto{\pgfqpoint{2.181604in}{2.527300in}}{\pgfqpoint{2.177214in}{2.537899in}}{\pgfqpoint{2.169400in}{2.545713in}}%
\pgfpathcurveto{\pgfqpoint{2.161587in}{2.553526in}}{\pgfqpoint{2.150988in}{2.557917in}}{\pgfqpoint{2.139937in}{2.557917in}}%
\pgfpathcurveto{\pgfqpoint{2.128887in}{2.557917in}}{\pgfqpoint{2.118288in}{2.553526in}}{\pgfqpoint{2.110475in}{2.545713in}}%
\pgfpathcurveto{\pgfqpoint{2.102661in}{2.537899in}}{\pgfqpoint{2.098271in}{2.527300in}}{\pgfqpoint{2.098271in}{2.516250in}}%
\pgfpathcurveto{\pgfqpoint{2.098271in}{2.505200in}}{\pgfqpoint{2.102661in}{2.494601in}}{\pgfqpoint{2.110475in}{2.486787in}}%
\pgfpathcurveto{\pgfqpoint{2.118288in}{2.478974in}}{\pgfqpoint{2.128887in}{2.474583in}}{\pgfqpoint{2.139937in}{2.474583in}}%
\pgfpathclose%
\pgfusepath{stroke,fill}%
\end{pgfscope}%
\begin{pgfscope}%
\pgfpathrectangle{\pgfqpoint{0.375000in}{0.330000in}}{\pgfqpoint{2.325000in}{2.310000in}}%
\pgfusepath{clip}%
\pgfsetbuttcap%
\pgfsetroundjoin%
\definecolor{currentfill}{rgb}{0.000000,0.000000,0.000000}%
\pgfsetfillcolor{currentfill}%
\pgfsetlinewidth{1.003750pt}%
\definecolor{currentstroke}{rgb}{0.000000,0.000000,0.000000}%
\pgfsetstrokecolor{currentstroke}%
\pgfsetdash{}{0pt}%
\pgfpathmoveto{\pgfqpoint{2.139937in}{1.443291in}}%
\pgfpathcurveto{\pgfqpoint{2.150988in}{1.443291in}}{\pgfqpoint{2.161587in}{1.447682in}}{\pgfqpoint{2.169400in}{1.455495in}}%
\pgfpathcurveto{\pgfqpoint{2.177214in}{1.463309in}}{\pgfqpoint{2.181604in}{1.473908in}}{\pgfqpoint{2.181604in}{1.484958in}}%
\pgfpathcurveto{\pgfqpoint{2.181604in}{1.496008in}}{\pgfqpoint{2.177214in}{1.506607in}}{\pgfqpoint{2.169400in}{1.514421in}}%
\pgfpathcurveto{\pgfqpoint{2.161587in}{1.522235in}}{\pgfqpoint{2.150988in}{1.526625in}}{\pgfqpoint{2.139937in}{1.526625in}}%
\pgfpathcurveto{\pgfqpoint{2.128887in}{1.526625in}}{\pgfqpoint{2.118288in}{1.522235in}}{\pgfqpoint{2.110475in}{1.514421in}}%
\pgfpathcurveto{\pgfqpoint{2.102661in}{1.506607in}}{\pgfqpoint{2.098271in}{1.496008in}}{\pgfqpoint{2.098271in}{1.484958in}}%
\pgfpathcurveto{\pgfqpoint{2.098271in}{1.473908in}}{\pgfqpoint{2.102661in}{1.463309in}}{\pgfqpoint{2.110475in}{1.455495in}}%
\pgfpathcurveto{\pgfqpoint{2.118288in}{1.447682in}}{\pgfqpoint{2.128887in}{1.443291in}}{\pgfqpoint{2.139937in}{1.443291in}}%
\pgfpathclose%
\pgfusepath{stroke,fill}%
\end{pgfscope}%
\begin{pgfscope}%
\pgfpathrectangle{\pgfqpoint{0.375000in}{0.330000in}}{\pgfqpoint{2.325000in}{2.310000in}}%
\pgfusepath{clip}%
\pgfsetbuttcap%
\pgfsetroundjoin%
\definecolor{currentfill}{rgb}{0.000000,0.000000,0.000000}%
\pgfsetfillcolor{currentfill}%
\pgfsetlinewidth{1.003750pt}%
\definecolor{currentstroke}{rgb}{0.000000,0.000000,0.000000}%
\pgfsetstrokecolor{currentstroke}%
\pgfsetdash{}{0pt}%
\pgfpathmoveto{\pgfqpoint{2.139937in}{2.474583in}}%
\pgfpathcurveto{\pgfqpoint{2.150988in}{2.474583in}}{\pgfqpoint{2.161587in}{2.478974in}}{\pgfqpoint{2.169400in}{2.486787in}}%
\pgfpathcurveto{\pgfqpoint{2.177214in}{2.494601in}}{\pgfqpoint{2.181604in}{2.505200in}}{\pgfqpoint{2.181604in}{2.516250in}}%
\pgfpathcurveto{\pgfqpoint{2.181604in}{2.527300in}}{\pgfqpoint{2.177214in}{2.537899in}}{\pgfqpoint{2.169400in}{2.545713in}}%
\pgfpathcurveto{\pgfqpoint{2.161587in}{2.553526in}}{\pgfqpoint{2.150988in}{2.557917in}}{\pgfqpoint{2.139937in}{2.557917in}}%
\pgfpathcurveto{\pgfqpoint{2.128887in}{2.557917in}}{\pgfqpoint{2.118288in}{2.553526in}}{\pgfqpoint{2.110475in}{2.545713in}}%
\pgfpathcurveto{\pgfqpoint{2.102661in}{2.537899in}}{\pgfqpoint{2.098271in}{2.527300in}}{\pgfqpoint{2.098271in}{2.516250in}}%
\pgfpathcurveto{\pgfqpoint{2.098271in}{2.505200in}}{\pgfqpoint{2.102661in}{2.494601in}}{\pgfqpoint{2.110475in}{2.486787in}}%
\pgfpathcurveto{\pgfqpoint{2.118288in}{2.478974in}}{\pgfqpoint{2.128887in}{2.474583in}}{\pgfqpoint{2.139937in}{2.474583in}}%
\pgfpathclose%
\pgfusepath{stroke,fill}%
\end{pgfscope}%
\begin{pgfscope}%
\pgfpathrectangle{\pgfqpoint{0.375000in}{0.330000in}}{\pgfqpoint{2.325000in}{2.310000in}}%
\pgfusepath{clip}%
\pgfsetbuttcap%
\pgfsetroundjoin%
\definecolor{currentfill}{rgb}{0.000000,0.000000,0.000000}%
\pgfsetfillcolor{currentfill}%
\pgfsetlinewidth{1.003750pt}%
\definecolor{currentstroke}{rgb}{0.000000,0.000000,0.000000}%
\pgfsetstrokecolor{currentstroke}%
\pgfsetdash{}{0pt}%
\pgfpathmoveto{\pgfqpoint{2.139937in}{1.443291in}}%
\pgfpathcurveto{\pgfqpoint{2.150988in}{1.443291in}}{\pgfqpoint{2.161587in}{1.447682in}}{\pgfqpoint{2.169400in}{1.455495in}}%
\pgfpathcurveto{\pgfqpoint{2.177214in}{1.463309in}}{\pgfqpoint{2.181604in}{1.473908in}}{\pgfqpoint{2.181604in}{1.484958in}}%
\pgfpathcurveto{\pgfqpoint{2.181604in}{1.496008in}}{\pgfqpoint{2.177214in}{1.506607in}}{\pgfqpoint{2.169400in}{1.514421in}}%
\pgfpathcurveto{\pgfqpoint{2.161587in}{1.522235in}}{\pgfqpoint{2.150988in}{1.526625in}}{\pgfqpoint{2.139937in}{1.526625in}}%
\pgfpathcurveto{\pgfqpoint{2.128887in}{1.526625in}}{\pgfqpoint{2.118288in}{1.522235in}}{\pgfqpoint{2.110475in}{1.514421in}}%
\pgfpathcurveto{\pgfqpoint{2.102661in}{1.506607in}}{\pgfqpoint{2.098271in}{1.496008in}}{\pgfqpoint{2.098271in}{1.484958in}}%
\pgfpathcurveto{\pgfqpoint{2.098271in}{1.473908in}}{\pgfqpoint{2.102661in}{1.463309in}}{\pgfqpoint{2.110475in}{1.455495in}}%
\pgfpathcurveto{\pgfqpoint{2.118288in}{1.447682in}}{\pgfqpoint{2.128887in}{1.443291in}}{\pgfqpoint{2.139937in}{1.443291in}}%
\pgfpathclose%
\pgfusepath{stroke,fill}%
\end{pgfscope}%
\begin{pgfscope}%
\pgfpathrectangle{\pgfqpoint{0.375000in}{0.330000in}}{\pgfqpoint{2.325000in}{2.310000in}}%
\pgfusepath{clip}%
\pgfsetbuttcap%
\pgfsetroundjoin%
\definecolor{currentfill}{rgb}{0.000000,0.000000,0.000000}%
\pgfsetfillcolor{currentfill}%
\pgfsetlinewidth{1.003750pt}%
\definecolor{currentstroke}{rgb}{0.000000,0.000000,0.000000}%
\pgfsetstrokecolor{currentstroke}%
\pgfsetdash{}{0pt}%
\pgfpathmoveto{\pgfqpoint{2.139937in}{1.443291in}}%
\pgfpathcurveto{\pgfqpoint{2.150988in}{1.443291in}}{\pgfqpoint{2.161587in}{1.447682in}}{\pgfqpoint{2.169400in}{1.455495in}}%
\pgfpathcurveto{\pgfqpoint{2.177214in}{1.463309in}}{\pgfqpoint{2.181604in}{1.473908in}}{\pgfqpoint{2.181604in}{1.484958in}}%
\pgfpathcurveto{\pgfqpoint{2.181604in}{1.496008in}}{\pgfqpoint{2.177214in}{1.506607in}}{\pgfqpoint{2.169400in}{1.514421in}}%
\pgfpathcurveto{\pgfqpoint{2.161587in}{1.522235in}}{\pgfqpoint{2.150988in}{1.526625in}}{\pgfqpoint{2.139937in}{1.526625in}}%
\pgfpathcurveto{\pgfqpoint{2.128887in}{1.526625in}}{\pgfqpoint{2.118288in}{1.522235in}}{\pgfqpoint{2.110475in}{1.514421in}}%
\pgfpathcurveto{\pgfqpoint{2.102661in}{1.506607in}}{\pgfqpoint{2.098271in}{1.496008in}}{\pgfqpoint{2.098271in}{1.484958in}}%
\pgfpathcurveto{\pgfqpoint{2.098271in}{1.473908in}}{\pgfqpoint{2.102661in}{1.463309in}}{\pgfqpoint{2.110475in}{1.455495in}}%
\pgfpathcurveto{\pgfqpoint{2.118288in}{1.447682in}}{\pgfqpoint{2.128887in}{1.443291in}}{\pgfqpoint{2.139937in}{1.443291in}}%
\pgfpathclose%
\pgfusepath{stroke,fill}%
\end{pgfscope}%
\begin{pgfscope}%
\pgfpathrectangle{\pgfqpoint{0.375000in}{0.330000in}}{\pgfqpoint{2.325000in}{2.310000in}}%
\pgfusepath{clip}%
\pgfsetbuttcap%
\pgfsetroundjoin%
\definecolor{currentfill}{rgb}{0.000000,0.000000,0.000000}%
\pgfsetfillcolor{currentfill}%
\pgfsetlinewidth{1.003750pt}%
\definecolor{currentstroke}{rgb}{0.000000,0.000000,0.000000}%
\pgfsetstrokecolor{currentstroke}%
\pgfsetdash{}{0pt}%
\pgfpathmoveto{\pgfqpoint{2.139937in}{2.474583in}}%
\pgfpathcurveto{\pgfqpoint{2.150988in}{2.474583in}}{\pgfqpoint{2.161587in}{2.478974in}}{\pgfqpoint{2.169400in}{2.486787in}}%
\pgfpathcurveto{\pgfqpoint{2.177214in}{2.494601in}}{\pgfqpoint{2.181604in}{2.505200in}}{\pgfqpoint{2.181604in}{2.516250in}}%
\pgfpathcurveto{\pgfqpoint{2.181604in}{2.527300in}}{\pgfqpoint{2.177214in}{2.537899in}}{\pgfqpoint{2.169400in}{2.545713in}}%
\pgfpathcurveto{\pgfqpoint{2.161587in}{2.553526in}}{\pgfqpoint{2.150988in}{2.557917in}}{\pgfqpoint{2.139937in}{2.557917in}}%
\pgfpathcurveto{\pgfqpoint{2.128887in}{2.557917in}}{\pgfqpoint{2.118288in}{2.553526in}}{\pgfqpoint{2.110475in}{2.545713in}}%
\pgfpathcurveto{\pgfqpoint{2.102661in}{2.537899in}}{\pgfqpoint{2.098271in}{2.527300in}}{\pgfqpoint{2.098271in}{2.516250in}}%
\pgfpathcurveto{\pgfqpoint{2.098271in}{2.505200in}}{\pgfqpoint{2.102661in}{2.494601in}}{\pgfqpoint{2.110475in}{2.486787in}}%
\pgfpathcurveto{\pgfqpoint{2.118288in}{2.478974in}}{\pgfqpoint{2.128887in}{2.474583in}}{\pgfqpoint{2.139937in}{2.474583in}}%
\pgfpathclose%
\pgfusepath{stroke,fill}%
\end{pgfscope}%
\begin{pgfscope}%
\pgfpathrectangle{\pgfqpoint{0.375000in}{0.330000in}}{\pgfqpoint{2.325000in}{2.310000in}}%
\pgfusepath{clip}%
\pgfsetbuttcap%
\pgfsetroundjoin%
\definecolor{currentfill}{rgb}{0.000000,0.000000,0.000000}%
\pgfsetfillcolor{currentfill}%
\pgfsetlinewidth{1.003750pt}%
\definecolor{currentstroke}{rgb}{0.000000,0.000000,0.000000}%
\pgfsetstrokecolor{currentstroke}%
\pgfsetdash{}{0pt}%
\pgfpathmoveto{\pgfqpoint{2.139937in}{1.443291in}}%
\pgfpathcurveto{\pgfqpoint{2.150988in}{1.443291in}}{\pgfqpoint{2.161587in}{1.447682in}}{\pgfqpoint{2.169400in}{1.455495in}}%
\pgfpathcurveto{\pgfqpoint{2.177214in}{1.463309in}}{\pgfqpoint{2.181604in}{1.473908in}}{\pgfqpoint{2.181604in}{1.484958in}}%
\pgfpathcurveto{\pgfqpoint{2.181604in}{1.496008in}}{\pgfqpoint{2.177214in}{1.506607in}}{\pgfqpoint{2.169400in}{1.514421in}}%
\pgfpathcurveto{\pgfqpoint{2.161587in}{1.522235in}}{\pgfqpoint{2.150988in}{1.526625in}}{\pgfqpoint{2.139937in}{1.526625in}}%
\pgfpathcurveto{\pgfqpoint{2.128887in}{1.526625in}}{\pgfqpoint{2.118288in}{1.522235in}}{\pgfqpoint{2.110475in}{1.514421in}}%
\pgfpathcurveto{\pgfqpoint{2.102661in}{1.506607in}}{\pgfqpoint{2.098271in}{1.496008in}}{\pgfqpoint{2.098271in}{1.484958in}}%
\pgfpathcurveto{\pgfqpoint{2.098271in}{1.473908in}}{\pgfqpoint{2.102661in}{1.463309in}}{\pgfqpoint{2.110475in}{1.455495in}}%
\pgfpathcurveto{\pgfqpoint{2.118288in}{1.447682in}}{\pgfqpoint{2.128887in}{1.443291in}}{\pgfqpoint{2.139937in}{1.443291in}}%
\pgfpathclose%
\pgfusepath{stroke,fill}%
\end{pgfscope}%
\begin{pgfscope}%
\pgfpathrectangle{\pgfqpoint{0.375000in}{0.330000in}}{\pgfqpoint{2.325000in}{2.310000in}}%
\pgfusepath{clip}%
\pgfsetbuttcap%
\pgfsetroundjoin%
\definecolor{currentfill}{rgb}{0.000000,0.000000,0.000000}%
\pgfsetfillcolor{currentfill}%
\pgfsetlinewidth{1.003750pt}%
\definecolor{currentstroke}{rgb}{0.000000,0.000000,0.000000}%
\pgfsetstrokecolor{currentstroke}%
\pgfsetdash{}{0pt}%
\pgfpathmoveto{\pgfqpoint{2.139937in}{2.474583in}}%
\pgfpathcurveto{\pgfqpoint{2.150988in}{2.474583in}}{\pgfqpoint{2.161587in}{2.478974in}}{\pgfqpoint{2.169400in}{2.486787in}}%
\pgfpathcurveto{\pgfqpoint{2.177214in}{2.494601in}}{\pgfqpoint{2.181604in}{2.505200in}}{\pgfqpoint{2.181604in}{2.516250in}}%
\pgfpathcurveto{\pgfqpoint{2.181604in}{2.527300in}}{\pgfqpoint{2.177214in}{2.537899in}}{\pgfqpoint{2.169400in}{2.545713in}}%
\pgfpathcurveto{\pgfqpoint{2.161587in}{2.553526in}}{\pgfqpoint{2.150988in}{2.557917in}}{\pgfqpoint{2.139937in}{2.557917in}}%
\pgfpathcurveto{\pgfqpoint{2.128887in}{2.557917in}}{\pgfqpoint{2.118288in}{2.553526in}}{\pgfqpoint{2.110475in}{2.545713in}}%
\pgfpathcurveto{\pgfqpoint{2.102661in}{2.537899in}}{\pgfqpoint{2.098271in}{2.527300in}}{\pgfqpoint{2.098271in}{2.516250in}}%
\pgfpathcurveto{\pgfqpoint{2.098271in}{2.505200in}}{\pgfqpoint{2.102661in}{2.494601in}}{\pgfqpoint{2.110475in}{2.486787in}}%
\pgfpathcurveto{\pgfqpoint{2.118288in}{2.478974in}}{\pgfqpoint{2.128887in}{2.474583in}}{\pgfqpoint{2.139937in}{2.474583in}}%
\pgfpathclose%
\pgfusepath{stroke,fill}%
\end{pgfscope}%
\begin{pgfscope}%
\pgfpathrectangle{\pgfqpoint{0.375000in}{0.330000in}}{\pgfqpoint{2.325000in}{2.310000in}}%
\pgfusepath{clip}%
\pgfsetbuttcap%
\pgfsetroundjoin%
\definecolor{currentfill}{rgb}{0.000000,0.000000,0.000000}%
\pgfsetfillcolor{currentfill}%
\pgfsetlinewidth{1.003750pt}%
\definecolor{currentstroke}{rgb}{0.000000,0.000000,0.000000}%
\pgfsetstrokecolor{currentstroke}%
\pgfsetdash{}{0pt}%
\pgfpathmoveto{\pgfqpoint{2.139937in}{1.443291in}}%
\pgfpathcurveto{\pgfqpoint{2.150988in}{1.443291in}}{\pgfqpoint{2.161587in}{1.447682in}}{\pgfqpoint{2.169400in}{1.455495in}}%
\pgfpathcurveto{\pgfqpoint{2.177214in}{1.463309in}}{\pgfqpoint{2.181604in}{1.473908in}}{\pgfqpoint{2.181604in}{1.484958in}}%
\pgfpathcurveto{\pgfqpoint{2.181604in}{1.496008in}}{\pgfqpoint{2.177214in}{1.506607in}}{\pgfqpoint{2.169400in}{1.514421in}}%
\pgfpathcurveto{\pgfqpoint{2.161587in}{1.522235in}}{\pgfqpoint{2.150988in}{1.526625in}}{\pgfqpoint{2.139937in}{1.526625in}}%
\pgfpathcurveto{\pgfqpoint{2.128887in}{1.526625in}}{\pgfqpoint{2.118288in}{1.522235in}}{\pgfqpoint{2.110475in}{1.514421in}}%
\pgfpathcurveto{\pgfqpoint{2.102661in}{1.506607in}}{\pgfqpoint{2.098271in}{1.496008in}}{\pgfqpoint{2.098271in}{1.484958in}}%
\pgfpathcurveto{\pgfqpoint{2.098271in}{1.473908in}}{\pgfqpoint{2.102661in}{1.463309in}}{\pgfqpoint{2.110475in}{1.455495in}}%
\pgfpathcurveto{\pgfqpoint{2.118288in}{1.447682in}}{\pgfqpoint{2.128887in}{1.443291in}}{\pgfqpoint{2.139937in}{1.443291in}}%
\pgfpathclose%
\pgfusepath{stroke,fill}%
\end{pgfscope}%
\begin{pgfscope}%
\pgfpathrectangle{\pgfqpoint{0.375000in}{0.330000in}}{\pgfqpoint{2.325000in}{2.310000in}}%
\pgfusepath{clip}%
\pgfsetbuttcap%
\pgfsetroundjoin%
\definecolor{currentfill}{rgb}{0.000000,0.000000,0.000000}%
\pgfsetfillcolor{currentfill}%
\pgfsetlinewidth{1.003750pt}%
\definecolor{currentstroke}{rgb}{0.000000,0.000000,0.000000}%
\pgfsetstrokecolor{currentstroke}%
\pgfsetdash{}{0pt}%
\pgfpathmoveto{\pgfqpoint{2.139937in}{1.443291in}}%
\pgfpathcurveto{\pgfqpoint{2.150988in}{1.443291in}}{\pgfqpoint{2.161587in}{1.447682in}}{\pgfqpoint{2.169400in}{1.455495in}}%
\pgfpathcurveto{\pgfqpoint{2.177214in}{1.463309in}}{\pgfqpoint{2.181604in}{1.473908in}}{\pgfqpoint{2.181604in}{1.484958in}}%
\pgfpathcurveto{\pgfqpoint{2.181604in}{1.496008in}}{\pgfqpoint{2.177214in}{1.506607in}}{\pgfqpoint{2.169400in}{1.514421in}}%
\pgfpathcurveto{\pgfqpoint{2.161587in}{1.522235in}}{\pgfqpoint{2.150988in}{1.526625in}}{\pgfqpoint{2.139937in}{1.526625in}}%
\pgfpathcurveto{\pgfqpoint{2.128887in}{1.526625in}}{\pgfqpoint{2.118288in}{1.522235in}}{\pgfqpoint{2.110475in}{1.514421in}}%
\pgfpathcurveto{\pgfqpoint{2.102661in}{1.506607in}}{\pgfqpoint{2.098271in}{1.496008in}}{\pgfqpoint{2.098271in}{1.484958in}}%
\pgfpathcurveto{\pgfqpoint{2.098271in}{1.473908in}}{\pgfqpoint{2.102661in}{1.463309in}}{\pgfqpoint{2.110475in}{1.455495in}}%
\pgfpathcurveto{\pgfqpoint{2.118288in}{1.447682in}}{\pgfqpoint{2.128887in}{1.443291in}}{\pgfqpoint{2.139937in}{1.443291in}}%
\pgfpathclose%
\pgfusepath{stroke,fill}%
\end{pgfscope}%
\begin{pgfscope}%
\pgfpathrectangle{\pgfqpoint{0.375000in}{0.330000in}}{\pgfqpoint{2.325000in}{2.310000in}}%
\pgfusepath{clip}%
\pgfsetbuttcap%
\pgfsetroundjoin%
\definecolor{currentfill}{rgb}{0.000000,0.000000,0.000000}%
\pgfsetfillcolor{currentfill}%
\pgfsetlinewidth{1.003750pt}%
\definecolor{currentstroke}{rgb}{0.000000,0.000000,0.000000}%
\pgfsetstrokecolor{currentstroke}%
\pgfsetdash{}{0pt}%
\pgfpathmoveto{\pgfqpoint{2.139937in}{1.443291in}}%
\pgfpathcurveto{\pgfqpoint{2.150988in}{1.443291in}}{\pgfqpoint{2.161587in}{1.447682in}}{\pgfqpoint{2.169400in}{1.455495in}}%
\pgfpathcurveto{\pgfqpoint{2.177214in}{1.463309in}}{\pgfqpoint{2.181604in}{1.473908in}}{\pgfqpoint{2.181604in}{1.484958in}}%
\pgfpathcurveto{\pgfqpoint{2.181604in}{1.496008in}}{\pgfqpoint{2.177214in}{1.506607in}}{\pgfqpoint{2.169400in}{1.514421in}}%
\pgfpathcurveto{\pgfqpoint{2.161587in}{1.522235in}}{\pgfqpoint{2.150988in}{1.526625in}}{\pgfqpoint{2.139937in}{1.526625in}}%
\pgfpathcurveto{\pgfqpoint{2.128887in}{1.526625in}}{\pgfqpoint{2.118288in}{1.522235in}}{\pgfqpoint{2.110475in}{1.514421in}}%
\pgfpathcurveto{\pgfqpoint{2.102661in}{1.506607in}}{\pgfqpoint{2.098271in}{1.496008in}}{\pgfqpoint{2.098271in}{1.484958in}}%
\pgfpathcurveto{\pgfqpoint{2.098271in}{1.473908in}}{\pgfqpoint{2.102661in}{1.463309in}}{\pgfqpoint{2.110475in}{1.455495in}}%
\pgfpathcurveto{\pgfqpoint{2.118288in}{1.447682in}}{\pgfqpoint{2.128887in}{1.443291in}}{\pgfqpoint{2.139937in}{1.443291in}}%
\pgfpathclose%
\pgfusepath{stroke,fill}%
\end{pgfscope}%
\begin{pgfscope}%
\pgfpathrectangle{\pgfqpoint{0.375000in}{0.330000in}}{\pgfqpoint{2.325000in}{2.310000in}}%
\pgfusepath{clip}%
\pgfsetbuttcap%
\pgfsetroundjoin%
\definecolor{currentfill}{rgb}{0.000000,0.000000,0.000000}%
\pgfsetfillcolor{currentfill}%
\pgfsetlinewidth{1.003750pt}%
\definecolor{currentstroke}{rgb}{0.000000,0.000000,0.000000}%
\pgfsetstrokecolor{currentstroke}%
\pgfsetdash{}{0pt}%
\pgfpathmoveto{\pgfqpoint{2.139937in}{2.474583in}}%
\pgfpathcurveto{\pgfqpoint{2.150988in}{2.474583in}}{\pgfqpoint{2.161587in}{2.478974in}}{\pgfqpoint{2.169400in}{2.486787in}}%
\pgfpathcurveto{\pgfqpoint{2.177214in}{2.494601in}}{\pgfqpoint{2.181604in}{2.505200in}}{\pgfqpoint{2.181604in}{2.516250in}}%
\pgfpathcurveto{\pgfqpoint{2.181604in}{2.527300in}}{\pgfqpoint{2.177214in}{2.537899in}}{\pgfqpoint{2.169400in}{2.545713in}}%
\pgfpathcurveto{\pgfqpoint{2.161587in}{2.553526in}}{\pgfqpoint{2.150988in}{2.557917in}}{\pgfqpoint{2.139937in}{2.557917in}}%
\pgfpathcurveto{\pgfqpoint{2.128887in}{2.557917in}}{\pgfqpoint{2.118288in}{2.553526in}}{\pgfqpoint{2.110475in}{2.545713in}}%
\pgfpathcurveto{\pgfqpoint{2.102661in}{2.537899in}}{\pgfqpoint{2.098271in}{2.527300in}}{\pgfqpoint{2.098271in}{2.516250in}}%
\pgfpathcurveto{\pgfqpoint{2.098271in}{2.505200in}}{\pgfqpoint{2.102661in}{2.494601in}}{\pgfqpoint{2.110475in}{2.486787in}}%
\pgfpathcurveto{\pgfqpoint{2.118288in}{2.478974in}}{\pgfqpoint{2.128887in}{2.474583in}}{\pgfqpoint{2.139937in}{2.474583in}}%
\pgfpathclose%
\pgfusepath{stroke,fill}%
\end{pgfscope}%
\begin{pgfscope}%
\pgfpathrectangle{\pgfqpoint{0.375000in}{0.330000in}}{\pgfqpoint{2.325000in}{2.310000in}}%
\pgfusepath{clip}%
\pgfsetbuttcap%
\pgfsetroundjoin%
\definecolor{currentfill}{rgb}{0.000000,0.000000,0.000000}%
\pgfsetfillcolor{currentfill}%
\pgfsetlinewidth{1.003750pt}%
\definecolor{currentstroke}{rgb}{0.000000,0.000000,0.000000}%
\pgfsetstrokecolor{currentstroke}%
\pgfsetdash{}{0pt}%
\pgfpathmoveto{\pgfqpoint{2.139937in}{2.474583in}}%
\pgfpathcurveto{\pgfqpoint{2.150988in}{2.474583in}}{\pgfqpoint{2.161587in}{2.478974in}}{\pgfqpoint{2.169400in}{2.486787in}}%
\pgfpathcurveto{\pgfqpoint{2.177214in}{2.494601in}}{\pgfqpoint{2.181604in}{2.505200in}}{\pgfqpoint{2.181604in}{2.516250in}}%
\pgfpathcurveto{\pgfqpoint{2.181604in}{2.527300in}}{\pgfqpoint{2.177214in}{2.537899in}}{\pgfqpoint{2.169400in}{2.545713in}}%
\pgfpathcurveto{\pgfqpoint{2.161587in}{2.553526in}}{\pgfqpoint{2.150988in}{2.557917in}}{\pgfqpoint{2.139937in}{2.557917in}}%
\pgfpathcurveto{\pgfqpoint{2.128887in}{2.557917in}}{\pgfqpoint{2.118288in}{2.553526in}}{\pgfqpoint{2.110475in}{2.545713in}}%
\pgfpathcurveto{\pgfqpoint{2.102661in}{2.537899in}}{\pgfqpoint{2.098271in}{2.527300in}}{\pgfqpoint{2.098271in}{2.516250in}}%
\pgfpathcurveto{\pgfqpoint{2.098271in}{2.505200in}}{\pgfqpoint{2.102661in}{2.494601in}}{\pgfqpoint{2.110475in}{2.486787in}}%
\pgfpathcurveto{\pgfqpoint{2.118288in}{2.478974in}}{\pgfqpoint{2.128887in}{2.474583in}}{\pgfqpoint{2.139937in}{2.474583in}}%
\pgfpathclose%
\pgfusepath{stroke,fill}%
\end{pgfscope}%
\begin{pgfscope}%
\pgfpathrectangle{\pgfqpoint{0.375000in}{0.330000in}}{\pgfqpoint{2.325000in}{2.310000in}}%
\pgfusepath{clip}%
\pgfsetbuttcap%
\pgfsetroundjoin%
\definecolor{currentfill}{rgb}{0.000000,0.000000,0.000000}%
\pgfsetfillcolor{currentfill}%
\pgfsetlinewidth{1.003750pt}%
\definecolor{currentstroke}{rgb}{0.000000,0.000000,0.000000}%
\pgfsetstrokecolor{currentstroke}%
\pgfsetdash{}{0pt}%
\pgfpathmoveto{\pgfqpoint{2.139937in}{1.443291in}}%
\pgfpathcurveto{\pgfqpoint{2.150988in}{1.443291in}}{\pgfqpoint{2.161587in}{1.447682in}}{\pgfqpoint{2.169400in}{1.455495in}}%
\pgfpathcurveto{\pgfqpoint{2.177214in}{1.463309in}}{\pgfqpoint{2.181604in}{1.473908in}}{\pgfqpoint{2.181604in}{1.484958in}}%
\pgfpathcurveto{\pgfqpoint{2.181604in}{1.496008in}}{\pgfqpoint{2.177214in}{1.506607in}}{\pgfqpoint{2.169400in}{1.514421in}}%
\pgfpathcurveto{\pgfqpoint{2.161587in}{1.522235in}}{\pgfqpoint{2.150988in}{1.526625in}}{\pgfqpoint{2.139937in}{1.526625in}}%
\pgfpathcurveto{\pgfqpoint{2.128887in}{1.526625in}}{\pgfqpoint{2.118288in}{1.522235in}}{\pgfqpoint{2.110475in}{1.514421in}}%
\pgfpathcurveto{\pgfqpoint{2.102661in}{1.506607in}}{\pgfqpoint{2.098271in}{1.496008in}}{\pgfqpoint{2.098271in}{1.484958in}}%
\pgfpathcurveto{\pgfqpoint{2.098271in}{1.473908in}}{\pgfqpoint{2.102661in}{1.463309in}}{\pgfqpoint{2.110475in}{1.455495in}}%
\pgfpathcurveto{\pgfqpoint{2.118288in}{1.447682in}}{\pgfqpoint{2.128887in}{1.443291in}}{\pgfqpoint{2.139937in}{1.443291in}}%
\pgfpathclose%
\pgfusepath{stroke,fill}%
\end{pgfscope}%
\begin{pgfscope}%
\pgfpathrectangle{\pgfqpoint{0.375000in}{0.330000in}}{\pgfqpoint{2.325000in}{2.310000in}}%
\pgfusepath{clip}%
\pgfsetbuttcap%
\pgfsetroundjoin%
\definecolor{currentfill}{rgb}{0.000000,0.000000,0.000000}%
\pgfsetfillcolor{currentfill}%
\pgfsetlinewidth{1.003750pt}%
\definecolor{currentstroke}{rgb}{0.000000,0.000000,0.000000}%
\pgfsetstrokecolor{currentstroke}%
\pgfsetdash{}{0pt}%
\pgfpathmoveto{\pgfqpoint{2.139937in}{2.474583in}}%
\pgfpathcurveto{\pgfqpoint{2.150988in}{2.474583in}}{\pgfqpoint{2.161587in}{2.478974in}}{\pgfqpoint{2.169400in}{2.486787in}}%
\pgfpathcurveto{\pgfqpoint{2.177214in}{2.494601in}}{\pgfqpoint{2.181604in}{2.505200in}}{\pgfqpoint{2.181604in}{2.516250in}}%
\pgfpathcurveto{\pgfqpoint{2.181604in}{2.527300in}}{\pgfqpoint{2.177214in}{2.537899in}}{\pgfqpoint{2.169400in}{2.545713in}}%
\pgfpathcurveto{\pgfqpoint{2.161587in}{2.553526in}}{\pgfqpoint{2.150988in}{2.557917in}}{\pgfqpoint{2.139937in}{2.557917in}}%
\pgfpathcurveto{\pgfqpoint{2.128887in}{2.557917in}}{\pgfqpoint{2.118288in}{2.553526in}}{\pgfqpoint{2.110475in}{2.545713in}}%
\pgfpathcurveto{\pgfqpoint{2.102661in}{2.537899in}}{\pgfqpoint{2.098271in}{2.527300in}}{\pgfqpoint{2.098271in}{2.516250in}}%
\pgfpathcurveto{\pgfqpoint{2.098271in}{2.505200in}}{\pgfqpoint{2.102661in}{2.494601in}}{\pgfqpoint{2.110475in}{2.486787in}}%
\pgfpathcurveto{\pgfqpoint{2.118288in}{2.478974in}}{\pgfqpoint{2.128887in}{2.474583in}}{\pgfqpoint{2.139937in}{2.474583in}}%
\pgfpathclose%
\pgfusepath{stroke,fill}%
\end{pgfscope}%
\begin{pgfscope}%
\pgfpathrectangle{\pgfqpoint{0.375000in}{0.330000in}}{\pgfqpoint{2.325000in}{2.310000in}}%
\pgfusepath{clip}%
\pgfsetbuttcap%
\pgfsetroundjoin%
\definecolor{currentfill}{rgb}{0.000000,0.000000,0.000000}%
\pgfsetfillcolor{currentfill}%
\pgfsetlinewidth{1.003750pt}%
\definecolor{currentstroke}{rgb}{0.000000,0.000000,0.000000}%
\pgfsetstrokecolor{currentstroke}%
\pgfsetdash{}{0pt}%
\pgfpathmoveto{\pgfqpoint{2.139937in}{1.443291in}}%
\pgfpathcurveto{\pgfqpoint{2.150988in}{1.443291in}}{\pgfqpoint{2.161587in}{1.447682in}}{\pgfqpoint{2.169400in}{1.455495in}}%
\pgfpathcurveto{\pgfqpoint{2.177214in}{1.463309in}}{\pgfqpoint{2.181604in}{1.473908in}}{\pgfqpoint{2.181604in}{1.484958in}}%
\pgfpathcurveto{\pgfqpoint{2.181604in}{1.496008in}}{\pgfqpoint{2.177214in}{1.506607in}}{\pgfqpoint{2.169400in}{1.514421in}}%
\pgfpathcurveto{\pgfqpoint{2.161587in}{1.522235in}}{\pgfqpoint{2.150988in}{1.526625in}}{\pgfqpoint{2.139937in}{1.526625in}}%
\pgfpathcurveto{\pgfqpoint{2.128887in}{1.526625in}}{\pgfqpoint{2.118288in}{1.522235in}}{\pgfqpoint{2.110475in}{1.514421in}}%
\pgfpathcurveto{\pgfqpoint{2.102661in}{1.506607in}}{\pgfqpoint{2.098271in}{1.496008in}}{\pgfqpoint{2.098271in}{1.484958in}}%
\pgfpathcurveto{\pgfqpoint{2.098271in}{1.473908in}}{\pgfqpoint{2.102661in}{1.463309in}}{\pgfqpoint{2.110475in}{1.455495in}}%
\pgfpathcurveto{\pgfqpoint{2.118288in}{1.447682in}}{\pgfqpoint{2.128887in}{1.443291in}}{\pgfqpoint{2.139937in}{1.443291in}}%
\pgfpathclose%
\pgfusepath{stroke,fill}%
\end{pgfscope}%
\begin{pgfscope}%
\pgfpathrectangle{\pgfqpoint{0.375000in}{0.330000in}}{\pgfqpoint{2.325000in}{2.310000in}}%
\pgfusepath{clip}%
\pgfsetbuttcap%
\pgfsetroundjoin%
\definecolor{currentfill}{rgb}{0.000000,0.000000,0.000000}%
\pgfsetfillcolor{currentfill}%
\pgfsetlinewidth{1.003750pt}%
\definecolor{currentstroke}{rgb}{0.000000,0.000000,0.000000}%
\pgfsetstrokecolor{currentstroke}%
\pgfsetdash{}{0pt}%
\pgfpathmoveto{\pgfqpoint{2.139937in}{2.474583in}}%
\pgfpathcurveto{\pgfqpoint{2.150988in}{2.474583in}}{\pgfqpoint{2.161587in}{2.478974in}}{\pgfqpoint{2.169400in}{2.486787in}}%
\pgfpathcurveto{\pgfqpoint{2.177214in}{2.494601in}}{\pgfqpoint{2.181604in}{2.505200in}}{\pgfqpoint{2.181604in}{2.516250in}}%
\pgfpathcurveto{\pgfqpoint{2.181604in}{2.527300in}}{\pgfqpoint{2.177214in}{2.537899in}}{\pgfqpoint{2.169400in}{2.545713in}}%
\pgfpathcurveto{\pgfqpoint{2.161587in}{2.553526in}}{\pgfqpoint{2.150988in}{2.557917in}}{\pgfqpoint{2.139937in}{2.557917in}}%
\pgfpathcurveto{\pgfqpoint{2.128887in}{2.557917in}}{\pgfqpoint{2.118288in}{2.553526in}}{\pgfqpoint{2.110475in}{2.545713in}}%
\pgfpathcurveto{\pgfqpoint{2.102661in}{2.537899in}}{\pgfqpoint{2.098271in}{2.527300in}}{\pgfqpoint{2.098271in}{2.516250in}}%
\pgfpathcurveto{\pgfqpoint{2.098271in}{2.505200in}}{\pgfqpoint{2.102661in}{2.494601in}}{\pgfqpoint{2.110475in}{2.486787in}}%
\pgfpathcurveto{\pgfqpoint{2.118288in}{2.478974in}}{\pgfqpoint{2.128887in}{2.474583in}}{\pgfqpoint{2.139937in}{2.474583in}}%
\pgfpathclose%
\pgfusepath{stroke,fill}%
\end{pgfscope}%
\begin{pgfscope}%
\pgfpathrectangle{\pgfqpoint{0.375000in}{0.330000in}}{\pgfqpoint{2.325000in}{2.310000in}}%
\pgfusepath{clip}%
\pgfsetbuttcap%
\pgfsetroundjoin%
\definecolor{currentfill}{rgb}{0.000000,0.000000,0.000000}%
\pgfsetfillcolor{currentfill}%
\pgfsetlinewidth{1.003750pt}%
\definecolor{currentstroke}{rgb}{0.000000,0.000000,0.000000}%
\pgfsetstrokecolor{currentstroke}%
\pgfsetdash{}{0pt}%
\pgfpathmoveto{\pgfqpoint{2.139937in}{1.443291in}}%
\pgfpathcurveto{\pgfqpoint{2.150988in}{1.443291in}}{\pgfqpoint{2.161587in}{1.447682in}}{\pgfqpoint{2.169400in}{1.455495in}}%
\pgfpathcurveto{\pgfqpoint{2.177214in}{1.463309in}}{\pgfqpoint{2.181604in}{1.473908in}}{\pgfqpoint{2.181604in}{1.484958in}}%
\pgfpathcurveto{\pgfqpoint{2.181604in}{1.496008in}}{\pgfqpoint{2.177214in}{1.506607in}}{\pgfqpoint{2.169400in}{1.514421in}}%
\pgfpathcurveto{\pgfqpoint{2.161587in}{1.522235in}}{\pgfqpoint{2.150988in}{1.526625in}}{\pgfqpoint{2.139937in}{1.526625in}}%
\pgfpathcurveto{\pgfqpoint{2.128887in}{1.526625in}}{\pgfqpoint{2.118288in}{1.522235in}}{\pgfqpoint{2.110475in}{1.514421in}}%
\pgfpathcurveto{\pgfqpoint{2.102661in}{1.506607in}}{\pgfqpoint{2.098271in}{1.496008in}}{\pgfqpoint{2.098271in}{1.484958in}}%
\pgfpathcurveto{\pgfqpoint{2.098271in}{1.473908in}}{\pgfqpoint{2.102661in}{1.463309in}}{\pgfqpoint{2.110475in}{1.455495in}}%
\pgfpathcurveto{\pgfqpoint{2.118288in}{1.447682in}}{\pgfqpoint{2.128887in}{1.443291in}}{\pgfqpoint{2.139937in}{1.443291in}}%
\pgfpathclose%
\pgfusepath{stroke,fill}%
\end{pgfscope}%
\begin{pgfscope}%
\pgfpathrectangle{\pgfqpoint{0.375000in}{0.330000in}}{\pgfqpoint{2.325000in}{2.310000in}}%
\pgfusepath{clip}%
\pgfsetbuttcap%
\pgfsetroundjoin%
\definecolor{currentfill}{rgb}{0.000000,0.000000,0.000000}%
\pgfsetfillcolor{currentfill}%
\pgfsetlinewidth{1.003750pt}%
\definecolor{currentstroke}{rgb}{0.000000,0.000000,0.000000}%
\pgfsetstrokecolor{currentstroke}%
\pgfsetdash{}{0pt}%
\pgfpathmoveto{\pgfqpoint{2.139937in}{2.474583in}}%
\pgfpathcurveto{\pgfqpoint{2.150988in}{2.474583in}}{\pgfqpoint{2.161587in}{2.478974in}}{\pgfqpoint{2.169400in}{2.486787in}}%
\pgfpathcurveto{\pgfqpoint{2.177214in}{2.494601in}}{\pgfqpoint{2.181604in}{2.505200in}}{\pgfqpoint{2.181604in}{2.516250in}}%
\pgfpathcurveto{\pgfqpoint{2.181604in}{2.527300in}}{\pgfqpoint{2.177214in}{2.537899in}}{\pgfqpoint{2.169400in}{2.545713in}}%
\pgfpathcurveto{\pgfqpoint{2.161587in}{2.553526in}}{\pgfqpoint{2.150988in}{2.557917in}}{\pgfqpoint{2.139937in}{2.557917in}}%
\pgfpathcurveto{\pgfqpoint{2.128887in}{2.557917in}}{\pgfqpoint{2.118288in}{2.553526in}}{\pgfqpoint{2.110475in}{2.545713in}}%
\pgfpathcurveto{\pgfqpoint{2.102661in}{2.537899in}}{\pgfqpoint{2.098271in}{2.527300in}}{\pgfqpoint{2.098271in}{2.516250in}}%
\pgfpathcurveto{\pgfqpoint{2.098271in}{2.505200in}}{\pgfqpoint{2.102661in}{2.494601in}}{\pgfqpoint{2.110475in}{2.486787in}}%
\pgfpathcurveto{\pgfqpoint{2.118288in}{2.478974in}}{\pgfqpoint{2.128887in}{2.474583in}}{\pgfqpoint{2.139937in}{2.474583in}}%
\pgfpathclose%
\pgfusepath{stroke,fill}%
\end{pgfscope}%
\begin{pgfscope}%
\pgfpathrectangle{\pgfqpoint{0.375000in}{0.330000in}}{\pgfqpoint{2.325000in}{2.310000in}}%
\pgfusepath{clip}%
\pgfsetbuttcap%
\pgfsetroundjoin%
\definecolor{currentfill}{rgb}{0.000000,0.000000,0.000000}%
\pgfsetfillcolor{currentfill}%
\pgfsetlinewidth{1.003750pt}%
\definecolor{currentstroke}{rgb}{0.000000,0.000000,0.000000}%
\pgfsetstrokecolor{currentstroke}%
\pgfsetdash{}{0pt}%
\pgfpathmoveto{\pgfqpoint{2.139937in}{1.443291in}}%
\pgfpathcurveto{\pgfqpoint{2.150988in}{1.443291in}}{\pgfqpoint{2.161587in}{1.447682in}}{\pgfqpoint{2.169400in}{1.455495in}}%
\pgfpathcurveto{\pgfqpoint{2.177214in}{1.463309in}}{\pgfqpoint{2.181604in}{1.473908in}}{\pgfqpoint{2.181604in}{1.484958in}}%
\pgfpathcurveto{\pgfqpoint{2.181604in}{1.496008in}}{\pgfqpoint{2.177214in}{1.506607in}}{\pgfqpoint{2.169400in}{1.514421in}}%
\pgfpathcurveto{\pgfqpoint{2.161587in}{1.522235in}}{\pgfqpoint{2.150988in}{1.526625in}}{\pgfqpoint{2.139937in}{1.526625in}}%
\pgfpathcurveto{\pgfqpoint{2.128887in}{1.526625in}}{\pgfqpoint{2.118288in}{1.522235in}}{\pgfqpoint{2.110475in}{1.514421in}}%
\pgfpathcurveto{\pgfqpoint{2.102661in}{1.506607in}}{\pgfqpoint{2.098271in}{1.496008in}}{\pgfqpoint{2.098271in}{1.484958in}}%
\pgfpathcurveto{\pgfqpoint{2.098271in}{1.473908in}}{\pgfqpoint{2.102661in}{1.463309in}}{\pgfqpoint{2.110475in}{1.455495in}}%
\pgfpathcurveto{\pgfqpoint{2.118288in}{1.447682in}}{\pgfqpoint{2.128887in}{1.443291in}}{\pgfqpoint{2.139937in}{1.443291in}}%
\pgfpathclose%
\pgfusepath{stroke,fill}%
\end{pgfscope}%
\begin{pgfscope}%
\pgfpathrectangle{\pgfqpoint{0.375000in}{0.330000in}}{\pgfqpoint{2.325000in}{2.310000in}}%
\pgfusepath{clip}%
\pgfsetbuttcap%
\pgfsetroundjoin%
\definecolor{currentfill}{rgb}{0.000000,0.000000,0.000000}%
\pgfsetfillcolor{currentfill}%
\pgfsetlinewidth{1.003750pt}%
\definecolor{currentstroke}{rgb}{0.000000,0.000000,0.000000}%
\pgfsetstrokecolor{currentstroke}%
\pgfsetdash{}{0pt}%
\pgfpathmoveto{\pgfqpoint{2.139937in}{2.474583in}}%
\pgfpathcurveto{\pgfqpoint{2.150988in}{2.474583in}}{\pgfqpoint{2.161587in}{2.478974in}}{\pgfqpoint{2.169400in}{2.486787in}}%
\pgfpathcurveto{\pgfqpoint{2.177214in}{2.494601in}}{\pgfqpoint{2.181604in}{2.505200in}}{\pgfqpoint{2.181604in}{2.516250in}}%
\pgfpathcurveto{\pgfqpoint{2.181604in}{2.527300in}}{\pgfqpoint{2.177214in}{2.537899in}}{\pgfqpoint{2.169400in}{2.545713in}}%
\pgfpathcurveto{\pgfqpoint{2.161587in}{2.553526in}}{\pgfqpoint{2.150988in}{2.557917in}}{\pgfqpoint{2.139937in}{2.557917in}}%
\pgfpathcurveto{\pgfqpoint{2.128887in}{2.557917in}}{\pgfqpoint{2.118288in}{2.553526in}}{\pgfqpoint{2.110475in}{2.545713in}}%
\pgfpathcurveto{\pgfqpoint{2.102661in}{2.537899in}}{\pgfqpoint{2.098271in}{2.527300in}}{\pgfqpoint{2.098271in}{2.516250in}}%
\pgfpathcurveto{\pgfqpoint{2.098271in}{2.505200in}}{\pgfqpoint{2.102661in}{2.494601in}}{\pgfqpoint{2.110475in}{2.486787in}}%
\pgfpathcurveto{\pgfqpoint{2.118288in}{2.478974in}}{\pgfqpoint{2.128887in}{2.474583in}}{\pgfqpoint{2.139937in}{2.474583in}}%
\pgfpathclose%
\pgfusepath{stroke,fill}%
\end{pgfscope}%
\begin{pgfscope}%
\pgfpathrectangle{\pgfqpoint{0.375000in}{0.330000in}}{\pgfqpoint{2.325000in}{2.310000in}}%
\pgfusepath{clip}%
\pgfsetbuttcap%
\pgfsetroundjoin%
\definecolor{currentfill}{rgb}{0.000000,0.000000,0.000000}%
\pgfsetfillcolor{currentfill}%
\pgfsetlinewidth{1.003750pt}%
\definecolor{currentstroke}{rgb}{0.000000,0.000000,0.000000}%
\pgfsetstrokecolor{currentstroke}%
\pgfsetdash{}{0pt}%
\pgfpathmoveto{\pgfqpoint{2.139937in}{1.443291in}}%
\pgfpathcurveto{\pgfqpoint{2.150988in}{1.443291in}}{\pgfqpoint{2.161587in}{1.447682in}}{\pgfqpoint{2.169400in}{1.455495in}}%
\pgfpathcurveto{\pgfqpoint{2.177214in}{1.463309in}}{\pgfqpoint{2.181604in}{1.473908in}}{\pgfqpoint{2.181604in}{1.484958in}}%
\pgfpathcurveto{\pgfqpoint{2.181604in}{1.496008in}}{\pgfqpoint{2.177214in}{1.506607in}}{\pgfqpoint{2.169400in}{1.514421in}}%
\pgfpathcurveto{\pgfqpoint{2.161587in}{1.522235in}}{\pgfqpoint{2.150988in}{1.526625in}}{\pgfqpoint{2.139937in}{1.526625in}}%
\pgfpathcurveto{\pgfqpoint{2.128887in}{1.526625in}}{\pgfqpoint{2.118288in}{1.522235in}}{\pgfqpoint{2.110475in}{1.514421in}}%
\pgfpathcurveto{\pgfqpoint{2.102661in}{1.506607in}}{\pgfqpoint{2.098271in}{1.496008in}}{\pgfqpoint{2.098271in}{1.484958in}}%
\pgfpathcurveto{\pgfqpoint{2.098271in}{1.473908in}}{\pgfqpoint{2.102661in}{1.463309in}}{\pgfqpoint{2.110475in}{1.455495in}}%
\pgfpathcurveto{\pgfqpoint{2.118288in}{1.447682in}}{\pgfqpoint{2.128887in}{1.443291in}}{\pgfqpoint{2.139937in}{1.443291in}}%
\pgfpathclose%
\pgfusepath{stroke,fill}%
\end{pgfscope}%
\begin{pgfscope}%
\pgfpathrectangle{\pgfqpoint{0.375000in}{0.330000in}}{\pgfqpoint{2.325000in}{2.310000in}}%
\pgfusepath{clip}%
\pgfsetbuttcap%
\pgfsetroundjoin%
\definecolor{currentfill}{rgb}{0.000000,0.000000,0.000000}%
\pgfsetfillcolor{currentfill}%
\pgfsetlinewidth{1.003750pt}%
\definecolor{currentstroke}{rgb}{0.000000,0.000000,0.000000}%
\pgfsetstrokecolor{currentstroke}%
\pgfsetdash{}{0pt}%
\pgfpathmoveto{\pgfqpoint{2.139937in}{2.474583in}}%
\pgfpathcurveto{\pgfqpoint{2.150988in}{2.474583in}}{\pgfqpoint{2.161587in}{2.478974in}}{\pgfqpoint{2.169400in}{2.486787in}}%
\pgfpathcurveto{\pgfqpoint{2.177214in}{2.494601in}}{\pgfqpoint{2.181604in}{2.505200in}}{\pgfqpoint{2.181604in}{2.516250in}}%
\pgfpathcurveto{\pgfqpoint{2.181604in}{2.527300in}}{\pgfqpoint{2.177214in}{2.537899in}}{\pgfqpoint{2.169400in}{2.545713in}}%
\pgfpathcurveto{\pgfqpoint{2.161587in}{2.553526in}}{\pgfqpoint{2.150988in}{2.557917in}}{\pgfqpoint{2.139937in}{2.557917in}}%
\pgfpathcurveto{\pgfqpoint{2.128887in}{2.557917in}}{\pgfqpoint{2.118288in}{2.553526in}}{\pgfqpoint{2.110475in}{2.545713in}}%
\pgfpathcurveto{\pgfqpoint{2.102661in}{2.537899in}}{\pgfqpoint{2.098271in}{2.527300in}}{\pgfqpoint{2.098271in}{2.516250in}}%
\pgfpathcurveto{\pgfqpoint{2.098271in}{2.505200in}}{\pgfqpoint{2.102661in}{2.494601in}}{\pgfqpoint{2.110475in}{2.486787in}}%
\pgfpathcurveto{\pgfqpoint{2.118288in}{2.478974in}}{\pgfqpoint{2.128887in}{2.474583in}}{\pgfqpoint{2.139937in}{2.474583in}}%
\pgfpathclose%
\pgfusepath{stroke,fill}%
\end{pgfscope}%
\begin{pgfscope}%
\pgfpathrectangle{\pgfqpoint{0.375000in}{0.330000in}}{\pgfqpoint{2.325000in}{2.310000in}}%
\pgfusepath{clip}%
\pgfsetbuttcap%
\pgfsetroundjoin%
\definecolor{currentfill}{rgb}{0.000000,0.000000,0.000000}%
\pgfsetfillcolor{currentfill}%
\pgfsetlinewidth{1.003750pt}%
\definecolor{currentstroke}{rgb}{0.000000,0.000000,0.000000}%
\pgfsetstrokecolor{currentstroke}%
\pgfsetdash{}{0pt}%
\pgfpathmoveto{\pgfqpoint{2.139937in}{2.474583in}}%
\pgfpathcurveto{\pgfqpoint{2.150988in}{2.474583in}}{\pgfqpoint{2.161587in}{2.478974in}}{\pgfqpoint{2.169400in}{2.486787in}}%
\pgfpathcurveto{\pgfqpoint{2.177214in}{2.494601in}}{\pgfqpoint{2.181604in}{2.505200in}}{\pgfqpoint{2.181604in}{2.516250in}}%
\pgfpathcurveto{\pgfqpoint{2.181604in}{2.527300in}}{\pgfqpoint{2.177214in}{2.537899in}}{\pgfqpoint{2.169400in}{2.545713in}}%
\pgfpathcurveto{\pgfqpoint{2.161587in}{2.553526in}}{\pgfqpoint{2.150988in}{2.557917in}}{\pgfqpoint{2.139937in}{2.557917in}}%
\pgfpathcurveto{\pgfqpoint{2.128887in}{2.557917in}}{\pgfqpoint{2.118288in}{2.553526in}}{\pgfqpoint{2.110475in}{2.545713in}}%
\pgfpathcurveto{\pgfqpoint{2.102661in}{2.537899in}}{\pgfqpoint{2.098271in}{2.527300in}}{\pgfqpoint{2.098271in}{2.516250in}}%
\pgfpathcurveto{\pgfqpoint{2.098271in}{2.505200in}}{\pgfqpoint{2.102661in}{2.494601in}}{\pgfqpoint{2.110475in}{2.486787in}}%
\pgfpathcurveto{\pgfqpoint{2.118288in}{2.478974in}}{\pgfqpoint{2.128887in}{2.474583in}}{\pgfqpoint{2.139937in}{2.474583in}}%
\pgfpathclose%
\pgfusepath{stroke,fill}%
\end{pgfscope}%
\begin{pgfscope}%
\pgfpathrectangle{\pgfqpoint{0.375000in}{0.330000in}}{\pgfqpoint{2.325000in}{2.310000in}}%
\pgfusepath{clip}%
\pgfsetbuttcap%
\pgfsetroundjoin%
\definecolor{currentfill}{rgb}{0.000000,0.000000,0.000000}%
\pgfsetfillcolor{currentfill}%
\pgfsetlinewidth{1.003750pt}%
\definecolor{currentstroke}{rgb}{0.000000,0.000000,0.000000}%
\pgfsetstrokecolor{currentstroke}%
\pgfsetdash{}{0pt}%
\pgfpathmoveto{\pgfqpoint{2.139937in}{2.474583in}}%
\pgfpathcurveto{\pgfqpoint{2.150988in}{2.474583in}}{\pgfqpoint{2.161587in}{2.478974in}}{\pgfqpoint{2.169400in}{2.486787in}}%
\pgfpathcurveto{\pgfqpoint{2.177214in}{2.494601in}}{\pgfqpoint{2.181604in}{2.505200in}}{\pgfqpoint{2.181604in}{2.516250in}}%
\pgfpathcurveto{\pgfqpoint{2.181604in}{2.527300in}}{\pgfqpoint{2.177214in}{2.537899in}}{\pgfqpoint{2.169400in}{2.545713in}}%
\pgfpathcurveto{\pgfqpoint{2.161587in}{2.553526in}}{\pgfqpoint{2.150988in}{2.557917in}}{\pgfqpoint{2.139937in}{2.557917in}}%
\pgfpathcurveto{\pgfqpoint{2.128887in}{2.557917in}}{\pgfqpoint{2.118288in}{2.553526in}}{\pgfqpoint{2.110475in}{2.545713in}}%
\pgfpathcurveto{\pgfqpoint{2.102661in}{2.537899in}}{\pgfqpoint{2.098271in}{2.527300in}}{\pgfqpoint{2.098271in}{2.516250in}}%
\pgfpathcurveto{\pgfqpoint{2.098271in}{2.505200in}}{\pgfqpoint{2.102661in}{2.494601in}}{\pgfqpoint{2.110475in}{2.486787in}}%
\pgfpathcurveto{\pgfqpoint{2.118288in}{2.478974in}}{\pgfqpoint{2.128887in}{2.474583in}}{\pgfqpoint{2.139937in}{2.474583in}}%
\pgfpathclose%
\pgfusepath{stroke,fill}%
\end{pgfscope}%
\begin{pgfscope}%
\pgfpathrectangle{\pgfqpoint{0.375000in}{0.330000in}}{\pgfqpoint{2.325000in}{2.310000in}}%
\pgfusepath{clip}%
\pgfsetbuttcap%
\pgfsetroundjoin%
\definecolor{currentfill}{rgb}{0.000000,0.000000,0.000000}%
\pgfsetfillcolor{currentfill}%
\pgfsetlinewidth{1.003750pt}%
\definecolor{currentstroke}{rgb}{0.000000,0.000000,0.000000}%
\pgfsetstrokecolor{currentstroke}%
\pgfsetdash{}{0pt}%
\pgfpathmoveto{\pgfqpoint{2.139937in}{2.474583in}}%
\pgfpathcurveto{\pgfqpoint{2.150988in}{2.474583in}}{\pgfqpoint{2.161587in}{2.478974in}}{\pgfqpoint{2.169400in}{2.486787in}}%
\pgfpathcurveto{\pgfqpoint{2.177214in}{2.494601in}}{\pgfqpoint{2.181604in}{2.505200in}}{\pgfqpoint{2.181604in}{2.516250in}}%
\pgfpathcurveto{\pgfqpoint{2.181604in}{2.527300in}}{\pgfqpoint{2.177214in}{2.537899in}}{\pgfqpoint{2.169400in}{2.545713in}}%
\pgfpathcurveto{\pgfqpoint{2.161587in}{2.553526in}}{\pgfqpoint{2.150988in}{2.557917in}}{\pgfqpoint{2.139937in}{2.557917in}}%
\pgfpathcurveto{\pgfqpoint{2.128887in}{2.557917in}}{\pgfqpoint{2.118288in}{2.553526in}}{\pgfqpoint{2.110475in}{2.545713in}}%
\pgfpathcurveto{\pgfqpoint{2.102661in}{2.537899in}}{\pgfqpoint{2.098271in}{2.527300in}}{\pgfqpoint{2.098271in}{2.516250in}}%
\pgfpathcurveto{\pgfqpoint{2.098271in}{2.505200in}}{\pgfqpoint{2.102661in}{2.494601in}}{\pgfqpoint{2.110475in}{2.486787in}}%
\pgfpathcurveto{\pgfqpoint{2.118288in}{2.478974in}}{\pgfqpoint{2.128887in}{2.474583in}}{\pgfqpoint{2.139937in}{2.474583in}}%
\pgfpathclose%
\pgfusepath{stroke,fill}%
\end{pgfscope}%
\begin{pgfscope}%
\pgfpathrectangle{\pgfqpoint{0.375000in}{0.330000in}}{\pgfqpoint{2.325000in}{2.310000in}}%
\pgfusepath{clip}%
\pgfsetbuttcap%
\pgfsetroundjoin%
\definecolor{currentfill}{rgb}{0.000000,0.000000,0.000000}%
\pgfsetfillcolor{currentfill}%
\pgfsetlinewidth{1.003750pt}%
\definecolor{currentstroke}{rgb}{0.000000,0.000000,0.000000}%
\pgfsetstrokecolor{currentstroke}%
\pgfsetdash{}{0pt}%
\pgfpathmoveto{\pgfqpoint{2.139937in}{1.443291in}}%
\pgfpathcurveto{\pgfqpoint{2.150988in}{1.443291in}}{\pgfqpoint{2.161587in}{1.447682in}}{\pgfqpoint{2.169400in}{1.455495in}}%
\pgfpathcurveto{\pgfqpoint{2.177214in}{1.463309in}}{\pgfqpoint{2.181604in}{1.473908in}}{\pgfqpoint{2.181604in}{1.484958in}}%
\pgfpathcurveto{\pgfqpoint{2.181604in}{1.496008in}}{\pgfqpoint{2.177214in}{1.506607in}}{\pgfqpoint{2.169400in}{1.514421in}}%
\pgfpathcurveto{\pgfqpoint{2.161587in}{1.522235in}}{\pgfqpoint{2.150988in}{1.526625in}}{\pgfqpoint{2.139937in}{1.526625in}}%
\pgfpathcurveto{\pgfqpoint{2.128887in}{1.526625in}}{\pgfqpoint{2.118288in}{1.522235in}}{\pgfqpoint{2.110475in}{1.514421in}}%
\pgfpathcurveto{\pgfqpoint{2.102661in}{1.506607in}}{\pgfqpoint{2.098271in}{1.496008in}}{\pgfqpoint{2.098271in}{1.484958in}}%
\pgfpathcurveto{\pgfqpoint{2.098271in}{1.473908in}}{\pgfqpoint{2.102661in}{1.463309in}}{\pgfqpoint{2.110475in}{1.455495in}}%
\pgfpathcurveto{\pgfqpoint{2.118288in}{1.447682in}}{\pgfqpoint{2.128887in}{1.443291in}}{\pgfqpoint{2.139937in}{1.443291in}}%
\pgfpathclose%
\pgfusepath{stroke,fill}%
\end{pgfscope}%
\begin{pgfscope}%
\pgfpathrectangle{\pgfqpoint{0.375000in}{0.330000in}}{\pgfqpoint{2.325000in}{2.310000in}}%
\pgfusepath{clip}%
\pgfsetbuttcap%
\pgfsetroundjoin%
\definecolor{currentfill}{rgb}{0.000000,0.000000,0.000000}%
\pgfsetfillcolor{currentfill}%
\pgfsetlinewidth{1.003750pt}%
\definecolor{currentstroke}{rgb}{0.000000,0.000000,0.000000}%
\pgfsetstrokecolor{currentstroke}%
\pgfsetdash{}{0pt}%
\pgfpathmoveto{\pgfqpoint{2.139937in}{1.443291in}}%
\pgfpathcurveto{\pgfqpoint{2.150988in}{1.443291in}}{\pgfqpoint{2.161587in}{1.447682in}}{\pgfqpoint{2.169400in}{1.455495in}}%
\pgfpathcurveto{\pgfqpoint{2.177214in}{1.463309in}}{\pgfqpoint{2.181604in}{1.473908in}}{\pgfqpoint{2.181604in}{1.484958in}}%
\pgfpathcurveto{\pgfqpoint{2.181604in}{1.496008in}}{\pgfqpoint{2.177214in}{1.506607in}}{\pgfqpoint{2.169400in}{1.514421in}}%
\pgfpathcurveto{\pgfqpoint{2.161587in}{1.522235in}}{\pgfqpoint{2.150988in}{1.526625in}}{\pgfqpoint{2.139937in}{1.526625in}}%
\pgfpathcurveto{\pgfqpoint{2.128887in}{1.526625in}}{\pgfqpoint{2.118288in}{1.522235in}}{\pgfqpoint{2.110475in}{1.514421in}}%
\pgfpathcurveto{\pgfqpoint{2.102661in}{1.506607in}}{\pgfqpoint{2.098271in}{1.496008in}}{\pgfqpoint{2.098271in}{1.484958in}}%
\pgfpathcurveto{\pgfqpoint{2.098271in}{1.473908in}}{\pgfqpoint{2.102661in}{1.463309in}}{\pgfqpoint{2.110475in}{1.455495in}}%
\pgfpathcurveto{\pgfqpoint{2.118288in}{1.447682in}}{\pgfqpoint{2.128887in}{1.443291in}}{\pgfqpoint{2.139937in}{1.443291in}}%
\pgfpathclose%
\pgfusepath{stroke,fill}%
\end{pgfscope}%
\begin{pgfscope}%
\pgfsetbuttcap%
\pgfsetroundjoin%
\definecolor{currentfill}{rgb}{0.000000,0.000000,0.000000}%
\pgfsetfillcolor{currentfill}%
\pgfsetlinewidth{0.803000pt}%
\definecolor{currentstroke}{rgb}{0.000000,0.000000,0.000000}%
\pgfsetstrokecolor{currentstroke}%
\pgfsetdash{}{0pt}%
\pgfsys@defobject{currentmarker}{\pgfqpoint{0.000000in}{-0.048611in}}{\pgfqpoint{0.000000in}{0.000000in}}{%
\pgfpathmoveto{\pgfqpoint{0.000000in}{0.000000in}}%
\pgfpathlineto{\pgfqpoint{0.000000in}{-0.048611in}}%
\pgfusepath{stroke,fill}%
}%
\begin{pgfscope}%
\pgfsys@transformshift{0.459750in}{0.330000in}%
\pgfsys@useobject{currentmarker}{}%
\end{pgfscope}%
\end{pgfscope}%
\begin{pgfscope}%
\definecolor{textcolor}{rgb}{0.000000,0.000000,0.000000}%
\pgfsetstrokecolor{textcolor}%
\pgfsetfillcolor{textcolor}%
\pgftext[x=0.459750in,y=0.232778in,,top]{\color{textcolor}\sffamily\fontsize{10.000000}{12.000000}\selectfont 20}%
\end{pgfscope}%
\begin{pgfscope}%
\pgfsetbuttcap%
\pgfsetroundjoin%
\definecolor{currentfill}{rgb}{0.000000,0.000000,0.000000}%
\pgfsetfillcolor{currentfill}%
\pgfsetlinewidth{0.803000pt}%
\definecolor{currentstroke}{rgb}{0.000000,0.000000,0.000000}%
\pgfsetstrokecolor{currentstroke}%
\pgfsetdash{}{0pt}%
\pgfsys@defobject{currentmarker}{\pgfqpoint{0.000000in}{-0.048611in}}{\pgfqpoint{0.000000in}{0.000000in}}{%
\pgfpathmoveto{\pgfqpoint{0.000000in}{0.000000in}}%
\pgfpathlineto{\pgfqpoint{0.000000in}{-0.048611in}}%
\pgfusepath{stroke,fill}%
}%
\begin{pgfscope}%
\pgfsys@transformshift{1.019812in}{0.330000in}%
\pgfsys@useobject{currentmarker}{}%
\end{pgfscope}%
\end{pgfscope}%
\begin{pgfscope}%
\definecolor{textcolor}{rgb}{0.000000,0.000000,0.000000}%
\pgfsetstrokecolor{textcolor}%
\pgfsetfillcolor{textcolor}%
\pgftext[x=1.019812in,y=0.232778in,,top]{\color{textcolor}\sffamily\fontsize{10.000000}{12.000000}\selectfont 40}%
\end{pgfscope}%
\begin{pgfscope}%
\pgfsetbuttcap%
\pgfsetroundjoin%
\definecolor{currentfill}{rgb}{0.000000,0.000000,0.000000}%
\pgfsetfillcolor{currentfill}%
\pgfsetlinewidth{0.803000pt}%
\definecolor{currentstroke}{rgb}{0.000000,0.000000,0.000000}%
\pgfsetstrokecolor{currentstroke}%
\pgfsetdash{}{0pt}%
\pgfsys@defobject{currentmarker}{\pgfqpoint{0.000000in}{-0.048611in}}{\pgfqpoint{0.000000in}{0.000000in}}{%
\pgfpathmoveto{\pgfqpoint{0.000000in}{0.000000in}}%
\pgfpathlineto{\pgfqpoint{0.000000in}{-0.048611in}}%
\pgfusepath{stroke,fill}%
}%
\begin{pgfscope}%
\pgfsys@transformshift{1.579875in}{0.330000in}%
\pgfsys@useobject{currentmarker}{}%
\end{pgfscope}%
\end{pgfscope}%
\begin{pgfscope}%
\definecolor{textcolor}{rgb}{0.000000,0.000000,0.000000}%
\pgfsetstrokecolor{textcolor}%
\pgfsetfillcolor{textcolor}%
\pgftext[x=1.579875in,y=0.232778in,,top]{\color{textcolor}\sffamily\fontsize{10.000000}{12.000000}\selectfont 60}%
\end{pgfscope}%
\begin{pgfscope}%
\pgfsetbuttcap%
\pgfsetroundjoin%
\definecolor{currentfill}{rgb}{0.000000,0.000000,0.000000}%
\pgfsetfillcolor{currentfill}%
\pgfsetlinewidth{0.803000pt}%
\definecolor{currentstroke}{rgb}{0.000000,0.000000,0.000000}%
\pgfsetstrokecolor{currentstroke}%
\pgfsetdash{}{0pt}%
\pgfsys@defobject{currentmarker}{\pgfqpoint{0.000000in}{-0.048611in}}{\pgfqpoint{0.000000in}{0.000000in}}{%
\pgfpathmoveto{\pgfqpoint{0.000000in}{0.000000in}}%
\pgfpathlineto{\pgfqpoint{0.000000in}{-0.048611in}}%
\pgfusepath{stroke,fill}%
}%
\begin{pgfscope}%
\pgfsys@transformshift{2.139937in}{0.330000in}%
\pgfsys@useobject{currentmarker}{}%
\end{pgfscope}%
\end{pgfscope}%
\begin{pgfscope}%
\definecolor{textcolor}{rgb}{0.000000,0.000000,0.000000}%
\pgfsetstrokecolor{textcolor}%
\pgfsetfillcolor{textcolor}%
\pgftext[x=2.139937in,y=0.232778in,,top]{\color{textcolor}\sffamily\fontsize{10.000000}{12.000000}\selectfont 80}%
\end{pgfscope}%
\begin{pgfscope}%
\pgfsetbuttcap%
\pgfsetroundjoin%
\definecolor{currentfill}{rgb}{0.000000,0.000000,0.000000}%
\pgfsetfillcolor{currentfill}%
\pgfsetlinewidth{0.803000pt}%
\definecolor{currentstroke}{rgb}{0.000000,0.000000,0.000000}%
\pgfsetstrokecolor{currentstroke}%
\pgfsetdash{}{0pt}%
\pgfsys@defobject{currentmarker}{\pgfqpoint{0.000000in}{-0.048611in}}{\pgfqpoint{0.000000in}{0.000000in}}{%
\pgfpathmoveto{\pgfqpoint{0.000000in}{0.000000in}}%
\pgfpathlineto{\pgfqpoint{0.000000in}{-0.048611in}}%
\pgfusepath{stroke,fill}%
}%
\begin{pgfscope}%
\pgfsys@transformshift{2.700000in}{0.330000in}%
\pgfsys@useobject{currentmarker}{}%
\end{pgfscope}%
\end{pgfscope}%
\begin{pgfscope}%
\definecolor{textcolor}{rgb}{0.000000,0.000000,0.000000}%
\pgfsetstrokecolor{textcolor}%
\pgfsetfillcolor{textcolor}%
\pgftext[x=2.700000in,y=0.232778in,,top]{\color{textcolor}\sffamily\fontsize{10.000000}{12.000000}\selectfont 100}%
\end{pgfscope}%
\begin{pgfscope}%
\definecolor{textcolor}{rgb}{0.000000,0.000000,0.000000}%
\pgfsetstrokecolor{textcolor}%
\pgfsetfillcolor{textcolor}%
\pgftext[x=1.537500in,y=0.042809in,,top]{\color{textcolor}\sffamily\fontsize{10.000000}{12.000000}\selectfont \(\displaystyle k\)}%
\end{pgfscope}%
\begin{pgfscope}%
\pgfsetbuttcap%
\pgfsetroundjoin%
\definecolor{currentfill}{rgb}{0.000000,0.000000,0.000000}%
\pgfsetfillcolor{currentfill}%
\pgfsetlinewidth{0.803000pt}%
\definecolor{currentstroke}{rgb}{0.000000,0.000000,0.000000}%
\pgfsetstrokecolor{currentstroke}%
\pgfsetdash{}{0pt}%
\pgfsys@defobject{currentmarker}{\pgfqpoint{-0.048611in}{0.000000in}}{\pgfqpoint{0.000000in}{0.000000in}}{%
\pgfpathmoveto{\pgfqpoint{0.000000in}{0.000000in}}%
\pgfpathlineto{\pgfqpoint{-0.048611in}{0.000000in}}%
\pgfusepath{stroke,fill}%
}%
\begin{pgfscope}%
\pgfsys@transformshift{0.375000in}{0.453666in}%
\pgfsys@useobject{currentmarker}{}%
\end{pgfscope}%
\end{pgfscope}%
\begin{pgfscope}%
\definecolor{textcolor}{rgb}{0.000000,0.000000,0.000000}%
\pgfsetstrokecolor{textcolor}%
\pgfsetfillcolor{textcolor}%
\pgftext[x=0.189413in,y=0.400905in,left,base]{\color{textcolor}\sffamily\fontsize{10.000000}{12.000000}\selectfont 8}%
\end{pgfscope}%
\begin{pgfscope}%
\pgfsetbuttcap%
\pgfsetroundjoin%
\definecolor{currentfill}{rgb}{0.000000,0.000000,0.000000}%
\pgfsetfillcolor{currentfill}%
\pgfsetlinewidth{0.803000pt}%
\definecolor{currentstroke}{rgb}{0.000000,0.000000,0.000000}%
\pgfsetstrokecolor{currentstroke}%
\pgfsetdash{}{0pt}%
\pgfsys@defobject{currentmarker}{\pgfqpoint{-0.048611in}{0.000000in}}{\pgfqpoint{0.000000in}{0.000000in}}{%
\pgfpathmoveto{\pgfqpoint{0.000000in}{0.000000in}}%
\pgfpathlineto{\pgfqpoint{-0.048611in}{0.000000in}}%
\pgfusepath{stroke,fill}%
}%
\begin{pgfscope}%
\pgfsys@transformshift{0.375000in}{1.484958in}%
\pgfsys@useobject{currentmarker}{}%
\end{pgfscope}%
\end{pgfscope}%
\begin{pgfscope}%
\definecolor{textcolor}{rgb}{0.000000,0.000000,0.000000}%
\pgfsetstrokecolor{textcolor}%
\pgfsetfillcolor{textcolor}%
\pgftext[x=0.189413in,y=1.432197in,left,base]{\color{textcolor}\sffamily\fontsize{10.000000}{12.000000}\selectfont 9}%
\end{pgfscope}%
\begin{pgfscope}%
\pgfsetbuttcap%
\pgfsetroundjoin%
\definecolor{currentfill}{rgb}{0.000000,0.000000,0.000000}%
\pgfsetfillcolor{currentfill}%
\pgfsetlinewidth{0.803000pt}%
\definecolor{currentstroke}{rgb}{0.000000,0.000000,0.000000}%
\pgfsetstrokecolor{currentstroke}%
\pgfsetdash{}{0pt}%
\pgfsys@defobject{currentmarker}{\pgfqpoint{-0.048611in}{0.000000in}}{\pgfqpoint{0.000000in}{0.000000in}}{%
\pgfpathmoveto{\pgfqpoint{0.000000in}{0.000000in}}%
\pgfpathlineto{\pgfqpoint{-0.048611in}{0.000000in}}%
\pgfusepath{stroke,fill}%
}%
\begin{pgfscope}%
\pgfsys@transformshift{0.375000in}{2.516250in}%
\pgfsys@useobject{currentmarker}{}%
\end{pgfscope}%
\end{pgfscope}%
\begin{pgfscope}%
\definecolor{textcolor}{rgb}{0.000000,0.000000,0.000000}%
\pgfsetstrokecolor{textcolor}%
\pgfsetfillcolor{textcolor}%
\pgftext[x=0.101047in,y=2.463488in,left,base]{\color{textcolor}\sffamily\fontsize{10.000000}{12.000000}\selectfont 10}%
\end{pgfscope}%
\begin{pgfscope}%
\definecolor{textcolor}{rgb}{0.000000,0.000000,0.000000}%
\pgfsetstrokecolor{textcolor}%
\pgfsetfillcolor{textcolor}%
\pgftext[x=0.045492in,y=1.485000in,,bottom,rotate=90.000000]{\color{textcolor}\sffamily\fontsize{10.000000}{12.000000}\selectfont Number of GMRES Iterations}%
\end{pgfscope}%
\begin{pgfscope}%
\pgfsetrectcap%
\pgfsetmiterjoin%
\pgfsetlinewidth{0.803000pt}%
\definecolor{currentstroke}{rgb}{0.000000,0.000000,0.000000}%
\pgfsetstrokecolor{currentstroke}%
\pgfsetdash{}{0pt}%
\pgfpathmoveto{\pgfqpoint{0.375000in}{0.330000in}}%
\pgfpathlineto{\pgfqpoint{0.375000in}{2.640000in}}%
\pgfusepath{stroke}%
\end{pgfscope}%
\begin{pgfscope}%
\pgfsetrectcap%
\pgfsetmiterjoin%
\pgfsetlinewidth{0.803000pt}%
\definecolor{currentstroke}{rgb}{0.000000,0.000000,0.000000}%
\pgfsetstrokecolor{currentstroke}%
\pgfsetdash{}{0pt}%
\pgfpathmoveto{\pgfqpoint{2.700000in}{0.330000in}}%
\pgfpathlineto{\pgfqpoint{2.700000in}{2.640000in}}%
\pgfusepath{stroke}%
\end{pgfscope}%
\begin{pgfscope}%
\pgfsetrectcap%
\pgfsetmiterjoin%
\pgfsetlinewidth{0.803000pt}%
\definecolor{currentstroke}{rgb}{0.000000,0.000000,0.000000}%
\pgfsetstrokecolor{currentstroke}%
\pgfsetdash{}{0pt}%
\pgfpathmoveto{\pgfqpoint{0.375000in}{0.330000in}}%
\pgfpathlineto{\pgfqpoint{2.700000in}{0.330000in}}%
\pgfusepath{stroke}%
\end{pgfscope}%
\begin{pgfscope}%
\pgfsetrectcap%
\pgfsetmiterjoin%
\pgfsetlinewidth{0.803000pt}%
\definecolor{currentstroke}{rgb}{0.000000,0.000000,0.000000}%
\pgfsetstrokecolor{currentstroke}%
\pgfsetdash{}{0pt}%
\pgfpathmoveto{\pgfqpoint{0.375000in}{2.640000in}}%
\pgfpathlineto{\pgfqpoint{2.700000in}{2.640000in}}%
\pgfusepath{stroke}%
\end{pgfscope}%
\end{pgfpicture}%
\makeatother%
\endgroup%

   \caption[Maximum GMRES iteration counts when $\NLiDRRdtd{\Aso-\Ast} = 0.5\times  k^{-\beta}$ for $\beta = 0.4,0.5,0.6,0.7.$]{Maximum GMRES iteration counts for solving systems with matrix $\AmatoI\Amatt$, where $\nso=\nst=1$ and $\NLiDRRdtd{\Aso-\Ast} = 0.5\times  k^{-\beta}$ for $\beta = 0.4,0.5 ,0.6,0.7.$}\label{fig:linfinityA1}
\end{figure}

    \begin{figure}
      \centering
%% Creator: Matplotlib, PGF backend
%%
%% To include the figure in your LaTeX document, write
%%   \input{<filename>.pgf}
%%
%% Make sure the required packages are loaded in your preamble
%%   \usepackage{pgf}
%%
%% Figures using additional raster images can only be included by \input if
%% they are in the same directory as the main LaTeX file. For loading figures
%% from other directories you can use the `import` package
%%   \usepackage{import}
%% and then include the figures with
%%   \import{<path to file>}{<filename>.pgf}
%%
%% Matplotlib used the following preamble
%%   \usepackage{fontspec}
%%   \setmainfont{DejaVuSerif.ttf}[Path=/home/owen/progs/firedrake-complex/firedrake/lib/python3.5/site-packages/matplotlib/mpl-data/fonts/ttf/]
%%   \setsansfont{DejaVuSans.ttf}[Path=/home/owen/progs/firedrake-complex/firedrake/lib/python3.5/site-packages/matplotlib/mpl-data/fonts/ttf/]
%%   \setmonofont{DejaVuSansMono.ttf}[Path=/home/owen/progs/firedrake-complex/firedrake/lib/python3.5/site-packages/matplotlib/mpl-data/fonts/ttf/]
%%
\begingroup%
\makeatletter%
\begin{pgfpicture}%
\pgfpathrectangle{\pgfpointorigin}{\pgfqpoint{3.000000in}{3.000000in}}%
\pgfusepath{use as bounding box, clip}%
\begin{pgfscope}%
\pgfsetbuttcap%
\pgfsetmiterjoin%
\definecolor{currentfill}{rgb}{1.000000,1.000000,1.000000}%
\pgfsetfillcolor{currentfill}%
\pgfsetlinewidth{0.000000pt}%
\definecolor{currentstroke}{rgb}{1.000000,1.000000,1.000000}%
\pgfsetstrokecolor{currentstroke}%
\pgfsetdash{}{0pt}%
\pgfpathmoveto{\pgfqpoint{0.000000in}{0.000000in}}%
\pgfpathlineto{\pgfqpoint{3.000000in}{0.000000in}}%
\pgfpathlineto{\pgfqpoint{3.000000in}{3.000000in}}%
\pgfpathlineto{\pgfqpoint{0.000000in}{3.000000in}}%
\pgfpathclose%
\pgfusepath{fill}%
\end{pgfscope}%
\begin{pgfscope}%
\pgfsetbuttcap%
\pgfsetmiterjoin%
\definecolor{currentfill}{rgb}{1.000000,1.000000,1.000000}%
\pgfsetfillcolor{currentfill}%
\pgfsetlinewidth{0.000000pt}%
\definecolor{currentstroke}{rgb}{0.000000,0.000000,0.000000}%
\pgfsetstrokecolor{currentstroke}%
\pgfsetstrokeopacity{0.000000}%
\pgfsetdash{}{0pt}%
\pgfpathmoveto{\pgfqpoint{0.375000in}{0.330000in}}%
\pgfpathlineto{\pgfqpoint{2.700000in}{0.330000in}}%
\pgfpathlineto{\pgfqpoint{2.700000in}{2.640000in}}%
\pgfpathlineto{\pgfqpoint{0.375000in}{2.640000in}}%
\pgfpathclose%
\pgfusepath{fill}%
\end{pgfscope}%
\begin{pgfscope}%
\pgfpathrectangle{\pgfqpoint{0.375000in}{0.330000in}}{\pgfqpoint{2.325000in}{2.310000in}}%
\pgfusepath{clip}%
\pgfsetbuttcap%
\pgfsetroundjoin%
\definecolor{currentfill}{rgb}{0.000000,0.000000,0.000000}%
\pgfsetfillcolor{currentfill}%
\pgfsetlinewidth{1.003750pt}%
\definecolor{currentstroke}{rgb}{0.000000,0.000000,0.000000}%
\pgfsetstrokecolor{currentstroke}%
\pgfsetdash{}{0pt}%
\pgfpathmoveto{\pgfqpoint{0.459750in}{0.417935in}}%
\pgfpathcurveto{\pgfqpoint{0.470800in}{0.417935in}}{\pgfqpoint{0.481399in}{0.422325in}}{\pgfqpoint{0.489213in}{0.430139in}}%
\pgfpathcurveto{\pgfqpoint{0.497026in}{0.437952in}}{\pgfqpoint{0.501417in}{0.448551in}}{\pgfqpoint{0.501417in}{0.459601in}}%
\pgfpathcurveto{\pgfqpoint{0.501417in}{0.470652in}}{\pgfqpoint{0.497026in}{0.481251in}}{\pgfqpoint{0.489213in}{0.489064in}}%
\pgfpathcurveto{\pgfqpoint{0.481399in}{0.496878in}}{\pgfqpoint{0.470800in}{0.501268in}}{\pgfqpoint{0.459750in}{0.501268in}}%
\pgfpathcurveto{\pgfqpoint{0.448700in}{0.501268in}}{\pgfqpoint{0.438101in}{0.496878in}}{\pgfqpoint{0.430287in}{0.489064in}}%
\pgfpathcurveto{\pgfqpoint{0.422474in}{0.481251in}}{\pgfqpoint{0.418083in}{0.470652in}}{\pgfqpoint{0.418083in}{0.459601in}}%
\pgfpathcurveto{\pgfqpoint{0.418083in}{0.448551in}}{\pgfqpoint{0.422474in}{0.437952in}}{\pgfqpoint{0.430287in}{0.430139in}}%
\pgfpathcurveto{\pgfqpoint{0.438101in}{0.422325in}}{\pgfqpoint{0.448700in}{0.417935in}}{\pgfqpoint{0.459750in}{0.417935in}}%
\pgfpathclose%
\pgfusepath{stroke,fill}%
\end{pgfscope}%
\begin{pgfscope}%
\pgfpathrectangle{\pgfqpoint{0.375000in}{0.330000in}}{\pgfqpoint{2.325000in}{2.310000in}}%
\pgfusepath{clip}%
\pgfsetbuttcap%
\pgfsetroundjoin%
\definecolor{currentfill}{rgb}{0.000000,0.000000,0.000000}%
\pgfsetfillcolor{currentfill}%
\pgfsetlinewidth{1.003750pt}%
\definecolor{currentstroke}{rgb}{0.000000,0.000000,0.000000}%
\pgfsetstrokecolor{currentstroke}%
\pgfsetdash{}{0pt}%
\pgfpathmoveto{\pgfqpoint{0.459750in}{0.411867in}}%
\pgfpathcurveto{\pgfqpoint{0.470800in}{0.411867in}}{\pgfqpoint{0.481399in}{0.416257in}}{\pgfqpoint{0.489213in}{0.424071in}}%
\pgfpathcurveto{\pgfqpoint{0.497026in}{0.431884in}}{\pgfqpoint{0.501417in}{0.442483in}}{\pgfqpoint{0.501417in}{0.453533in}}%
\pgfpathcurveto{\pgfqpoint{0.501417in}{0.464583in}}{\pgfqpoint{0.497026in}{0.475182in}}{\pgfqpoint{0.489213in}{0.482996in}}%
\pgfpathcurveto{\pgfqpoint{0.481399in}{0.490810in}}{\pgfqpoint{0.470800in}{0.495200in}}{\pgfqpoint{0.459750in}{0.495200in}}%
\pgfpathcurveto{\pgfqpoint{0.448700in}{0.495200in}}{\pgfqpoint{0.438101in}{0.490810in}}{\pgfqpoint{0.430287in}{0.482996in}}%
\pgfpathcurveto{\pgfqpoint{0.422474in}{0.475182in}}{\pgfqpoint{0.418083in}{0.464583in}}{\pgfqpoint{0.418083in}{0.453533in}}%
\pgfpathcurveto{\pgfqpoint{0.418083in}{0.442483in}}{\pgfqpoint{0.422474in}{0.431884in}}{\pgfqpoint{0.430287in}{0.424071in}}%
\pgfpathcurveto{\pgfqpoint{0.438101in}{0.416257in}}{\pgfqpoint{0.448700in}{0.411867in}}{\pgfqpoint{0.459750in}{0.411867in}}%
\pgfpathclose%
\pgfusepath{stroke,fill}%
\end{pgfscope}%
\begin{pgfscope}%
\pgfpathrectangle{\pgfqpoint{0.375000in}{0.330000in}}{\pgfqpoint{2.325000in}{2.310000in}}%
\pgfusepath{clip}%
\pgfsetbuttcap%
\pgfsetroundjoin%
\definecolor{currentfill}{rgb}{0.000000,0.000000,0.000000}%
\pgfsetfillcolor{currentfill}%
\pgfsetlinewidth{1.003750pt}%
\definecolor{currentstroke}{rgb}{0.000000,0.000000,0.000000}%
\pgfsetstrokecolor{currentstroke}%
\pgfsetdash{}{0pt}%
\pgfpathmoveto{\pgfqpoint{0.459750in}{0.405799in}}%
\pgfpathcurveto{\pgfqpoint{0.470800in}{0.405799in}}{\pgfqpoint{0.481399in}{0.410189in}}{\pgfqpoint{0.489213in}{0.418002in}}%
\pgfpathcurveto{\pgfqpoint{0.497026in}{0.425816in}}{\pgfqpoint{0.501417in}{0.436415in}}{\pgfqpoint{0.501417in}{0.447465in}}%
\pgfpathcurveto{\pgfqpoint{0.501417in}{0.458515in}}{\pgfqpoint{0.497026in}{0.469114in}}{\pgfqpoint{0.489213in}{0.476928in}}%
\pgfpathcurveto{\pgfqpoint{0.481399in}{0.484742in}}{\pgfqpoint{0.470800in}{0.489132in}}{\pgfqpoint{0.459750in}{0.489132in}}%
\pgfpathcurveto{\pgfqpoint{0.448700in}{0.489132in}}{\pgfqpoint{0.438101in}{0.484742in}}{\pgfqpoint{0.430287in}{0.476928in}}%
\pgfpathcurveto{\pgfqpoint{0.422474in}{0.469114in}}{\pgfqpoint{0.418083in}{0.458515in}}{\pgfqpoint{0.418083in}{0.447465in}}%
\pgfpathcurveto{\pgfqpoint{0.418083in}{0.436415in}}{\pgfqpoint{0.422474in}{0.425816in}}{\pgfqpoint{0.430287in}{0.418002in}}%
\pgfpathcurveto{\pgfqpoint{0.438101in}{0.410189in}}{\pgfqpoint{0.448700in}{0.405799in}}{\pgfqpoint{0.459750in}{0.405799in}}%
\pgfpathclose%
\pgfusepath{stroke,fill}%
\end{pgfscope}%
\begin{pgfscope}%
\pgfpathrectangle{\pgfqpoint{0.375000in}{0.330000in}}{\pgfqpoint{2.325000in}{2.310000in}}%
\pgfusepath{clip}%
\pgfsetbuttcap%
\pgfsetroundjoin%
\definecolor{currentfill}{rgb}{0.000000,0.000000,0.000000}%
\pgfsetfillcolor{currentfill}%
\pgfsetlinewidth{1.003750pt}%
\definecolor{currentstroke}{rgb}{0.000000,0.000000,0.000000}%
\pgfsetstrokecolor{currentstroke}%
\pgfsetdash{}{0pt}%
\pgfpathmoveto{\pgfqpoint{0.459750in}{0.411867in}}%
\pgfpathcurveto{\pgfqpoint{0.470800in}{0.411867in}}{\pgfqpoint{0.481399in}{0.416257in}}{\pgfqpoint{0.489213in}{0.424071in}}%
\pgfpathcurveto{\pgfqpoint{0.497026in}{0.431884in}}{\pgfqpoint{0.501417in}{0.442483in}}{\pgfqpoint{0.501417in}{0.453533in}}%
\pgfpathcurveto{\pgfqpoint{0.501417in}{0.464583in}}{\pgfqpoint{0.497026in}{0.475182in}}{\pgfqpoint{0.489213in}{0.482996in}}%
\pgfpathcurveto{\pgfqpoint{0.481399in}{0.490810in}}{\pgfqpoint{0.470800in}{0.495200in}}{\pgfqpoint{0.459750in}{0.495200in}}%
\pgfpathcurveto{\pgfqpoint{0.448700in}{0.495200in}}{\pgfqpoint{0.438101in}{0.490810in}}{\pgfqpoint{0.430287in}{0.482996in}}%
\pgfpathcurveto{\pgfqpoint{0.422474in}{0.475182in}}{\pgfqpoint{0.418083in}{0.464583in}}{\pgfqpoint{0.418083in}{0.453533in}}%
\pgfpathcurveto{\pgfqpoint{0.418083in}{0.442483in}}{\pgfqpoint{0.422474in}{0.431884in}}{\pgfqpoint{0.430287in}{0.424071in}}%
\pgfpathcurveto{\pgfqpoint{0.438101in}{0.416257in}}{\pgfqpoint{0.448700in}{0.411867in}}{\pgfqpoint{0.459750in}{0.411867in}}%
\pgfpathclose%
\pgfusepath{stroke,fill}%
\end{pgfscope}%
\begin{pgfscope}%
\pgfpathrectangle{\pgfqpoint{0.375000in}{0.330000in}}{\pgfqpoint{2.325000in}{2.310000in}}%
\pgfusepath{clip}%
\pgfsetbuttcap%
\pgfsetroundjoin%
\definecolor{currentfill}{rgb}{0.000000,0.000000,0.000000}%
\pgfsetfillcolor{currentfill}%
\pgfsetlinewidth{1.003750pt}%
\definecolor{currentstroke}{rgb}{0.000000,0.000000,0.000000}%
\pgfsetstrokecolor{currentstroke}%
\pgfsetdash{}{0pt}%
\pgfpathmoveto{\pgfqpoint{0.459750in}{0.424003in}}%
\pgfpathcurveto{\pgfqpoint{0.470800in}{0.424003in}}{\pgfqpoint{0.481399in}{0.428393in}}{\pgfqpoint{0.489213in}{0.436207in}}%
\pgfpathcurveto{\pgfqpoint{0.497026in}{0.444020in}}{\pgfqpoint{0.501417in}{0.454619in}}{\pgfqpoint{0.501417in}{0.465670in}}%
\pgfpathcurveto{\pgfqpoint{0.501417in}{0.476720in}}{\pgfqpoint{0.497026in}{0.487319in}}{\pgfqpoint{0.489213in}{0.495132in}}%
\pgfpathcurveto{\pgfqpoint{0.481399in}{0.502946in}}{\pgfqpoint{0.470800in}{0.507336in}}{\pgfqpoint{0.459750in}{0.507336in}}%
\pgfpathcurveto{\pgfqpoint{0.448700in}{0.507336in}}{\pgfqpoint{0.438101in}{0.502946in}}{\pgfqpoint{0.430287in}{0.495132in}}%
\pgfpathcurveto{\pgfqpoint{0.422474in}{0.487319in}}{\pgfqpoint{0.418083in}{0.476720in}}{\pgfqpoint{0.418083in}{0.465670in}}%
\pgfpathcurveto{\pgfqpoint{0.418083in}{0.454619in}}{\pgfqpoint{0.422474in}{0.444020in}}{\pgfqpoint{0.430287in}{0.436207in}}%
\pgfpathcurveto{\pgfqpoint{0.438101in}{0.428393in}}{\pgfqpoint{0.448700in}{0.424003in}}{\pgfqpoint{0.459750in}{0.424003in}}%
\pgfpathclose%
\pgfusepath{stroke,fill}%
\end{pgfscope}%
\begin{pgfscope}%
\pgfpathrectangle{\pgfqpoint{0.375000in}{0.330000in}}{\pgfqpoint{2.325000in}{2.310000in}}%
\pgfusepath{clip}%
\pgfsetbuttcap%
\pgfsetroundjoin%
\definecolor{currentfill}{rgb}{0.000000,0.000000,0.000000}%
\pgfsetfillcolor{currentfill}%
\pgfsetlinewidth{1.003750pt}%
\definecolor{currentstroke}{rgb}{0.000000,0.000000,0.000000}%
\pgfsetstrokecolor{currentstroke}%
\pgfsetdash{}{0pt}%
\pgfpathmoveto{\pgfqpoint{0.459750in}{0.411867in}}%
\pgfpathcurveto{\pgfqpoint{0.470800in}{0.411867in}}{\pgfqpoint{0.481399in}{0.416257in}}{\pgfqpoint{0.489213in}{0.424071in}}%
\pgfpathcurveto{\pgfqpoint{0.497026in}{0.431884in}}{\pgfqpoint{0.501417in}{0.442483in}}{\pgfqpoint{0.501417in}{0.453533in}}%
\pgfpathcurveto{\pgfqpoint{0.501417in}{0.464583in}}{\pgfqpoint{0.497026in}{0.475182in}}{\pgfqpoint{0.489213in}{0.482996in}}%
\pgfpathcurveto{\pgfqpoint{0.481399in}{0.490810in}}{\pgfqpoint{0.470800in}{0.495200in}}{\pgfqpoint{0.459750in}{0.495200in}}%
\pgfpathcurveto{\pgfqpoint{0.448700in}{0.495200in}}{\pgfqpoint{0.438101in}{0.490810in}}{\pgfqpoint{0.430287in}{0.482996in}}%
\pgfpathcurveto{\pgfqpoint{0.422474in}{0.475182in}}{\pgfqpoint{0.418083in}{0.464583in}}{\pgfqpoint{0.418083in}{0.453533in}}%
\pgfpathcurveto{\pgfqpoint{0.418083in}{0.442483in}}{\pgfqpoint{0.422474in}{0.431884in}}{\pgfqpoint{0.430287in}{0.424071in}}%
\pgfpathcurveto{\pgfqpoint{0.438101in}{0.416257in}}{\pgfqpoint{0.448700in}{0.411867in}}{\pgfqpoint{0.459750in}{0.411867in}}%
\pgfpathclose%
\pgfusepath{stroke,fill}%
\end{pgfscope}%
\begin{pgfscope}%
\pgfpathrectangle{\pgfqpoint{0.375000in}{0.330000in}}{\pgfqpoint{2.325000in}{2.310000in}}%
\pgfusepath{clip}%
\pgfsetbuttcap%
\pgfsetroundjoin%
\definecolor{currentfill}{rgb}{0.000000,0.000000,0.000000}%
\pgfsetfillcolor{currentfill}%
\pgfsetlinewidth{1.003750pt}%
\definecolor{currentstroke}{rgb}{0.000000,0.000000,0.000000}%
\pgfsetstrokecolor{currentstroke}%
\pgfsetdash{}{0pt}%
\pgfpathmoveto{\pgfqpoint{0.459750in}{0.405799in}}%
\pgfpathcurveto{\pgfqpoint{0.470800in}{0.405799in}}{\pgfqpoint{0.481399in}{0.410189in}}{\pgfqpoint{0.489213in}{0.418002in}}%
\pgfpathcurveto{\pgfqpoint{0.497026in}{0.425816in}}{\pgfqpoint{0.501417in}{0.436415in}}{\pgfqpoint{0.501417in}{0.447465in}}%
\pgfpathcurveto{\pgfqpoint{0.501417in}{0.458515in}}{\pgfqpoint{0.497026in}{0.469114in}}{\pgfqpoint{0.489213in}{0.476928in}}%
\pgfpathcurveto{\pgfqpoint{0.481399in}{0.484742in}}{\pgfqpoint{0.470800in}{0.489132in}}{\pgfqpoint{0.459750in}{0.489132in}}%
\pgfpathcurveto{\pgfqpoint{0.448700in}{0.489132in}}{\pgfqpoint{0.438101in}{0.484742in}}{\pgfqpoint{0.430287in}{0.476928in}}%
\pgfpathcurveto{\pgfqpoint{0.422474in}{0.469114in}}{\pgfqpoint{0.418083in}{0.458515in}}{\pgfqpoint{0.418083in}{0.447465in}}%
\pgfpathcurveto{\pgfqpoint{0.418083in}{0.436415in}}{\pgfqpoint{0.422474in}{0.425816in}}{\pgfqpoint{0.430287in}{0.418002in}}%
\pgfpathcurveto{\pgfqpoint{0.438101in}{0.410189in}}{\pgfqpoint{0.448700in}{0.405799in}}{\pgfqpoint{0.459750in}{0.405799in}}%
\pgfpathclose%
\pgfusepath{stroke,fill}%
\end{pgfscope}%
\begin{pgfscope}%
\pgfpathrectangle{\pgfqpoint{0.375000in}{0.330000in}}{\pgfqpoint{2.325000in}{2.310000in}}%
\pgfusepath{clip}%
\pgfsetbuttcap%
\pgfsetroundjoin%
\definecolor{currentfill}{rgb}{0.000000,0.000000,0.000000}%
\pgfsetfillcolor{currentfill}%
\pgfsetlinewidth{1.003750pt}%
\definecolor{currentstroke}{rgb}{0.000000,0.000000,0.000000}%
\pgfsetstrokecolor{currentstroke}%
\pgfsetdash{}{0pt}%
\pgfpathmoveto{\pgfqpoint{0.459750in}{0.417935in}}%
\pgfpathcurveto{\pgfqpoint{0.470800in}{0.417935in}}{\pgfqpoint{0.481399in}{0.422325in}}{\pgfqpoint{0.489213in}{0.430139in}}%
\pgfpathcurveto{\pgfqpoint{0.497026in}{0.437952in}}{\pgfqpoint{0.501417in}{0.448551in}}{\pgfqpoint{0.501417in}{0.459601in}}%
\pgfpathcurveto{\pgfqpoint{0.501417in}{0.470652in}}{\pgfqpoint{0.497026in}{0.481251in}}{\pgfqpoint{0.489213in}{0.489064in}}%
\pgfpathcurveto{\pgfqpoint{0.481399in}{0.496878in}}{\pgfqpoint{0.470800in}{0.501268in}}{\pgfqpoint{0.459750in}{0.501268in}}%
\pgfpathcurveto{\pgfqpoint{0.448700in}{0.501268in}}{\pgfqpoint{0.438101in}{0.496878in}}{\pgfqpoint{0.430287in}{0.489064in}}%
\pgfpathcurveto{\pgfqpoint{0.422474in}{0.481251in}}{\pgfqpoint{0.418083in}{0.470652in}}{\pgfqpoint{0.418083in}{0.459601in}}%
\pgfpathcurveto{\pgfqpoint{0.418083in}{0.448551in}}{\pgfqpoint{0.422474in}{0.437952in}}{\pgfqpoint{0.430287in}{0.430139in}}%
\pgfpathcurveto{\pgfqpoint{0.438101in}{0.422325in}}{\pgfqpoint{0.448700in}{0.417935in}}{\pgfqpoint{0.459750in}{0.417935in}}%
\pgfpathclose%
\pgfusepath{stroke,fill}%
\end{pgfscope}%
\begin{pgfscope}%
\pgfpathrectangle{\pgfqpoint{0.375000in}{0.330000in}}{\pgfqpoint{2.325000in}{2.310000in}}%
\pgfusepath{clip}%
\pgfsetbuttcap%
\pgfsetroundjoin%
\definecolor{currentfill}{rgb}{0.000000,0.000000,0.000000}%
\pgfsetfillcolor{currentfill}%
\pgfsetlinewidth{1.003750pt}%
\definecolor{currentstroke}{rgb}{0.000000,0.000000,0.000000}%
\pgfsetstrokecolor{currentstroke}%
\pgfsetdash{}{0pt}%
\pgfpathmoveto{\pgfqpoint{0.459750in}{0.411867in}}%
\pgfpathcurveto{\pgfqpoint{0.470800in}{0.411867in}}{\pgfqpoint{0.481399in}{0.416257in}}{\pgfqpoint{0.489213in}{0.424071in}}%
\pgfpathcurveto{\pgfqpoint{0.497026in}{0.431884in}}{\pgfqpoint{0.501417in}{0.442483in}}{\pgfqpoint{0.501417in}{0.453533in}}%
\pgfpathcurveto{\pgfqpoint{0.501417in}{0.464583in}}{\pgfqpoint{0.497026in}{0.475182in}}{\pgfqpoint{0.489213in}{0.482996in}}%
\pgfpathcurveto{\pgfqpoint{0.481399in}{0.490810in}}{\pgfqpoint{0.470800in}{0.495200in}}{\pgfqpoint{0.459750in}{0.495200in}}%
\pgfpathcurveto{\pgfqpoint{0.448700in}{0.495200in}}{\pgfqpoint{0.438101in}{0.490810in}}{\pgfqpoint{0.430287in}{0.482996in}}%
\pgfpathcurveto{\pgfqpoint{0.422474in}{0.475182in}}{\pgfqpoint{0.418083in}{0.464583in}}{\pgfqpoint{0.418083in}{0.453533in}}%
\pgfpathcurveto{\pgfqpoint{0.418083in}{0.442483in}}{\pgfqpoint{0.422474in}{0.431884in}}{\pgfqpoint{0.430287in}{0.424071in}}%
\pgfpathcurveto{\pgfqpoint{0.438101in}{0.416257in}}{\pgfqpoint{0.448700in}{0.411867in}}{\pgfqpoint{0.459750in}{0.411867in}}%
\pgfpathclose%
\pgfusepath{stroke,fill}%
\end{pgfscope}%
\begin{pgfscope}%
\pgfpathrectangle{\pgfqpoint{0.375000in}{0.330000in}}{\pgfqpoint{2.325000in}{2.310000in}}%
\pgfusepath{clip}%
\pgfsetbuttcap%
\pgfsetroundjoin%
\definecolor{currentfill}{rgb}{0.000000,0.000000,0.000000}%
\pgfsetfillcolor{currentfill}%
\pgfsetlinewidth{1.003750pt}%
\definecolor{currentstroke}{rgb}{0.000000,0.000000,0.000000}%
\pgfsetstrokecolor{currentstroke}%
\pgfsetdash{}{0pt}%
\pgfpathmoveto{\pgfqpoint{0.459750in}{0.417935in}}%
\pgfpathcurveto{\pgfqpoint{0.470800in}{0.417935in}}{\pgfqpoint{0.481399in}{0.422325in}}{\pgfqpoint{0.489213in}{0.430139in}}%
\pgfpathcurveto{\pgfqpoint{0.497026in}{0.437952in}}{\pgfqpoint{0.501417in}{0.448551in}}{\pgfqpoint{0.501417in}{0.459601in}}%
\pgfpathcurveto{\pgfqpoint{0.501417in}{0.470652in}}{\pgfqpoint{0.497026in}{0.481251in}}{\pgfqpoint{0.489213in}{0.489064in}}%
\pgfpathcurveto{\pgfqpoint{0.481399in}{0.496878in}}{\pgfqpoint{0.470800in}{0.501268in}}{\pgfqpoint{0.459750in}{0.501268in}}%
\pgfpathcurveto{\pgfqpoint{0.448700in}{0.501268in}}{\pgfqpoint{0.438101in}{0.496878in}}{\pgfqpoint{0.430287in}{0.489064in}}%
\pgfpathcurveto{\pgfqpoint{0.422474in}{0.481251in}}{\pgfqpoint{0.418083in}{0.470652in}}{\pgfqpoint{0.418083in}{0.459601in}}%
\pgfpathcurveto{\pgfqpoint{0.418083in}{0.448551in}}{\pgfqpoint{0.422474in}{0.437952in}}{\pgfqpoint{0.430287in}{0.430139in}}%
\pgfpathcurveto{\pgfqpoint{0.438101in}{0.422325in}}{\pgfqpoint{0.448700in}{0.417935in}}{\pgfqpoint{0.459750in}{0.417935in}}%
\pgfpathclose%
\pgfusepath{stroke,fill}%
\end{pgfscope}%
\begin{pgfscope}%
\pgfpathrectangle{\pgfqpoint{0.375000in}{0.330000in}}{\pgfqpoint{2.325000in}{2.310000in}}%
\pgfusepath{clip}%
\pgfsetbuttcap%
\pgfsetroundjoin%
\definecolor{currentfill}{rgb}{0.000000,0.000000,0.000000}%
\pgfsetfillcolor{currentfill}%
\pgfsetlinewidth{1.003750pt}%
\definecolor{currentstroke}{rgb}{0.000000,0.000000,0.000000}%
\pgfsetstrokecolor{currentstroke}%
\pgfsetdash{}{0pt}%
\pgfpathmoveto{\pgfqpoint{0.459750in}{0.436139in}}%
\pgfpathcurveto{\pgfqpoint{0.470800in}{0.436139in}}{\pgfqpoint{0.481399in}{0.440529in}}{\pgfqpoint{0.489213in}{0.448343in}}%
\pgfpathcurveto{\pgfqpoint{0.497026in}{0.456157in}}{\pgfqpoint{0.501417in}{0.466756in}}{\pgfqpoint{0.501417in}{0.477806in}}%
\pgfpathcurveto{\pgfqpoint{0.501417in}{0.488856in}}{\pgfqpoint{0.497026in}{0.499455in}}{\pgfqpoint{0.489213in}{0.507269in}}%
\pgfpathcurveto{\pgfqpoint{0.481399in}{0.515082in}}{\pgfqpoint{0.470800in}{0.519472in}}{\pgfqpoint{0.459750in}{0.519472in}}%
\pgfpathcurveto{\pgfqpoint{0.448700in}{0.519472in}}{\pgfqpoint{0.438101in}{0.515082in}}{\pgfqpoint{0.430287in}{0.507269in}}%
\pgfpathcurveto{\pgfqpoint{0.422474in}{0.499455in}}{\pgfqpoint{0.418083in}{0.488856in}}{\pgfqpoint{0.418083in}{0.477806in}}%
\pgfpathcurveto{\pgfqpoint{0.418083in}{0.466756in}}{\pgfqpoint{0.422474in}{0.456157in}}{\pgfqpoint{0.430287in}{0.448343in}}%
\pgfpathcurveto{\pgfqpoint{0.438101in}{0.440529in}}{\pgfqpoint{0.448700in}{0.436139in}}{\pgfqpoint{0.459750in}{0.436139in}}%
\pgfpathclose%
\pgfusepath{stroke,fill}%
\end{pgfscope}%
\begin{pgfscope}%
\pgfpathrectangle{\pgfqpoint{0.375000in}{0.330000in}}{\pgfqpoint{2.325000in}{2.310000in}}%
\pgfusepath{clip}%
\pgfsetbuttcap%
\pgfsetroundjoin%
\definecolor{currentfill}{rgb}{0.000000,0.000000,0.000000}%
\pgfsetfillcolor{currentfill}%
\pgfsetlinewidth{1.003750pt}%
\definecolor{currentstroke}{rgb}{0.000000,0.000000,0.000000}%
\pgfsetstrokecolor{currentstroke}%
\pgfsetdash{}{0pt}%
\pgfpathmoveto{\pgfqpoint{0.459750in}{0.424003in}}%
\pgfpathcurveto{\pgfqpoint{0.470800in}{0.424003in}}{\pgfqpoint{0.481399in}{0.428393in}}{\pgfqpoint{0.489213in}{0.436207in}}%
\pgfpathcurveto{\pgfqpoint{0.497026in}{0.444020in}}{\pgfqpoint{0.501417in}{0.454619in}}{\pgfqpoint{0.501417in}{0.465670in}}%
\pgfpathcurveto{\pgfqpoint{0.501417in}{0.476720in}}{\pgfqpoint{0.497026in}{0.487319in}}{\pgfqpoint{0.489213in}{0.495132in}}%
\pgfpathcurveto{\pgfqpoint{0.481399in}{0.502946in}}{\pgfqpoint{0.470800in}{0.507336in}}{\pgfqpoint{0.459750in}{0.507336in}}%
\pgfpathcurveto{\pgfqpoint{0.448700in}{0.507336in}}{\pgfqpoint{0.438101in}{0.502946in}}{\pgfqpoint{0.430287in}{0.495132in}}%
\pgfpathcurveto{\pgfqpoint{0.422474in}{0.487319in}}{\pgfqpoint{0.418083in}{0.476720in}}{\pgfqpoint{0.418083in}{0.465670in}}%
\pgfpathcurveto{\pgfqpoint{0.418083in}{0.454619in}}{\pgfqpoint{0.422474in}{0.444020in}}{\pgfqpoint{0.430287in}{0.436207in}}%
\pgfpathcurveto{\pgfqpoint{0.438101in}{0.428393in}}{\pgfqpoint{0.448700in}{0.424003in}}{\pgfqpoint{0.459750in}{0.424003in}}%
\pgfpathclose%
\pgfusepath{stroke,fill}%
\end{pgfscope}%
\begin{pgfscope}%
\pgfpathrectangle{\pgfqpoint{0.375000in}{0.330000in}}{\pgfqpoint{2.325000in}{2.310000in}}%
\pgfusepath{clip}%
\pgfsetbuttcap%
\pgfsetroundjoin%
\definecolor{currentfill}{rgb}{0.000000,0.000000,0.000000}%
\pgfsetfillcolor{currentfill}%
\pgfsetlinewidth{1.003750pt}%
\definecolor{currentstroke}{rgb}{0.000000,0.000000,0.000000}%
\pgfsetstrokecolor{currentstroke}%
\pgfsetdash{}{0pt}%
\pgfpathmoveto{\pgfqpoint{0.459750in}{0.411867in}}%
\pgfpathcurveto{\pgfqpoint{0.470800in}{0.411867in}}{\pgfqpoint{0.481399in}{0.416257in}}{\pgfqpoint{0.489213in}{0.424071in}}%
\pgfpathcurveto{\pgfqpoint{0.497026in}{0.431884in}}{\pgfqpoint{0.501417in}{0.442483in}}{\pgfqpoint{0.501417in}{0.453533in}}%
\pgfpathcurveto{\pgfqpoint{0.501417in}{0.464583in}}{\pgfqpoint{0.497026in}{0.475182in}}{\pgfqpoint{0.489213in}{0.482996in}}%
\pgfpathcurveto{\pgfqpoint{0.481399in}{0.490810in}}{\pgfqpoint{0.470800in}{0.495200in}}{\pgfqpoint{0.459750in}{0.495200in}}%
\pgfpathcurveto{\pgfqpoint{0.448700in}{0.495200in}}{\pgfqpoint{0.438101in}{0.490810in}}{\pgfqpoint{0.430287in}{0.482996in}}%
\pgfpathcurveto{\pgfqpoint{0.422474in}{0.475182in}}{\pgfqpoint{0.418083in}{0.464583in}}{\pgfqpoint{0.418083in}{0.453533in}}%
\pgfpathcurveto{\pgfqpoint{0.418083in}{0.442483in}}{\pgfqpoint{0.422474in}{0.431884in}}{\pgfqpoint{0.430287in}{0.424071in}}%
\pgfpathcurveto{\pgfqpoint{0.438101in}{0.416257in}}{\pgfqpoint{0.448700in}{0.411867in}}{\pgfqpoint{0.459750in}{0.411867in}}%
\pgfpathclose%
\pgfusepath{stroke,fill}%
\end{pgfscope}%
\begin{pgfscope}%
\pgfpathrectangle{\pgfqpoint{0.375000in}{0.330000in}}{\pgfqpoint{2.325000in}{2.310000in}}%
\pgfusepath{clip}%
\pgfsetbuttcap%
\pgfsetroundjoin%
\definecolor{currentfill}{rgb}{0.000000,0.000000,0.000000}%
\pgfsetfillcolor{currentfill}%
\pgfsetlinewidth{1.003750pt}%
\definecolor{currentstroke}{rgb}{0.000000,0.000000,0.000000}%
\pgfsetstrokecolor{currentstroke}%
\pgfsetdash{}{0pt}%
\pgfpathmoveto{\pgfqpoint{0.459750in}{0.405799in}}%
\pgfpathcurveto{\pgfqpoint{0.470800in}{0.405799in}}{\pgfqpoint{0.481399in}{0.410189in}}{\pgfqpoint{0.489213in}{0.418002in}}%
\pgfpathcurveto{\pgfqpoint{0.497026in}{0.425816in}}{\pgfqpoint{0.501417in}{0.436415in}}{\pgfqpoint{0.501417in}{0.447465in}}%
\pgfpathcurveto{\pgfqpoint{0.501417in}{0.458515in}}{\pgfqpoint{0.497026in}{0.469114in}}{\pgfqpoint{0.489213in}{0.476928in}}%
\pgfpathcurveto{\pgfqpoint{0.481399in}{0.484742in}}{\pgfqpoint{0.470800in}{0.489132in}}{\pgfqpoint{0.459750in}{0.489132in}}%
\pgfpathcurveto{\pgfqpoint{0.448700in}{0.489132in}}{\pgfqpoint{0.438101in}{0.484742in}}{\pgfqpoint{0.430287in}{0.476928in}}%
\pgfpathcurveto{\pgfqpoint{0.422474in}{0.469114in}}{\pgfqpoint{0.418083in}{0.458515in}}{\pgfqpoint{0.418083in}{0.447465in}}%
\pgfpathcurveto{\pgfqpoint{0.418083in}{0.436415in}}{\pgfqpoint{0.422474in}{0.425816in}}{\pgfqpoint{0.430287in}{0.418002in}}%
\pgfpathcurveto{\pgfqpoint{0.438101in}{0.410189in}}{\pgfqpoint{0.448700in}{0.405799in}}{\pgfqpoint{0.459750in}{0.405799in}}%
\pgfpathclose%
\pgfusepath{stroke,fill}%
\end{pgfscope}%
\begin{pgfscope}%
\pgfpathrectangle{\pgfqpoint{0.375000in}{0.330000in}}{\pgfqpoint{2.325000in}{2.310000in}}%
\pgfusepath{clip}%
\pgfsetbuttcap%
\pgfsetroundjoin%
\definecolor{currentfill}{rgb}{0.000000,0.000000,0.000000}%
\pgfsetfillcolor{currentfill}%
\pgfsetlinewidth{1.003750pt}%
\definecolor{currentstroke}{rgb}{0.000000,0.000000,0.000000}%
\pgfsetstrokecolor{currentstroke}%
\pgfsetdash{}{0pt}%
\pgfpathmoveto{\pgfqpoint{0.459750in}{0.411867in}}%
\pgfpathcurveto{\pgfqpoint{0.470800in}{0.411867in}}{\pgfqpoint{0.481399in}{0.416257in}}{\pgfqpoint{0.489213in}{0.424071in}}%
\pgfpathcurveto{\pgfqpoint{0.497026in}{0.431884in}}{\pgfqpoint{0.501417in}{0.442483in}}{\pgfqpoint{0.501417in}{0.453533in}}%
\pgfpathcurveto{\pgfqpoint{0.501417in}{0.464583in}}{\pgfqpoint{0.497026in}{0.475182in}}{\pgfqpoint{0.489213in}{0.482996in}}%
\pgfpathcurveto{\pgfqpoint{0.481399in}{0.490810in}}{\pgfqpoint{0.470800in}{0.495200in}}{\pgfqpoint{0.459750in}{0.495200in}}%
\pgfpathcurveto{\pgfqpoint{0.448700in}{0.495200in}}{\pgfqpoint{0.438101in}{0.490810in}}{\pgfqpoint{0.430287in}{0.482996in}}%
\pgfpathcurveto{\pgfqpoint{0.422474in}{0.475182in}}{\pgfqpoint{0.418083in}{0.464583in}}{\pgfqpoint{0.418083in}{0.453533in}}%
\pgfpathcurveto{\pgfqpoint{0.418083in}{0.442483in}}{\pgfqpoint{0.422474in}{0.431884in}}{\pgfqpoint{0.430287in}{0.424071in}}%
\pgfpathcurveto{\pgfqpoint{0.438101in}{0.416257in}}{\pgfqpoint{0.448700in}{0.411867in}}{\pgfqpoint{0.459750in}{0.411867in}}%
\pgfpathclose%
\pgfusepath{stroke,fill}%
\end{pgfscope}%
\begin{pgfscope}%
\pgfpathrectangle{\pgfqpoint{0.375000in}{0.330000in}}{\pgfqpoint{2.325000in}{2.310000in}}%
\pgfusepath{clip}%
\pgfsetbuttcap%
\pgfsetroundjoin%
\definecolor{currentfill}{rgb}{0.000000,0.000000,0.000000}%
\pgfsetfillcolor{currentfill}%
\pgfsetlinewidth{1.003750pt}%
\definecolor{currentstroke}{rgb}{0.000000,0.000000,0.000000}%
\pgfsetstrokecolor{currentstroke}%
\pgfsetdash{}{0pt}%
\pgfpathmoveto{\pgfqpoint{0.459750in}{0.405799in}}%
\pgfpathcurveto{\pgfqpoint{0.470800in}{0.405799in}}{\pgfqpoint{0.481399in}{0.410189in}}{\pgfqpoint{0.489213in}{0.418002in}}%
\pgfpathcurveto{\pgfqpoint{0.497026in}{0.425816in}}{\pgfqpoint{0.501417in}{0.436415in}}{\pgfqpoint{0.501417in}{0.447465in}}%
\pgfpathcurveto{\pgfqpoint{0.501417in}{0.458515in}}{\pgfqpoint{0.497026in}{0.469114in}}{\pgfqpoint{0.489213in}{0.476928in}}%
\pgfpathcurveto{\pgfqpoint{0.481399in}{0.484742in}}{\pgfqpoint{0.470800in}{0.489132in}}{\pgfqpoint{0.459750in}{0.489132in}}%
\pgfpathcurveto{\pgfqpoint{0.448700in}{0.489132in}}{\pgfqpoint{0.438101in}{0.484742in}}{\pgfqpoint{0.430287in}{0.476928in}}%
\pgfpathcurveto{\pgfqpoint{0.422474in}{0.469114in}}{\pgfqpoint{0.418083in}{0.458515in}}{\pgfqpoint{0.418083in}{0.447465in}}%
\pgfpathcurveto{\pgfqpoint{0.418083in}{0.436415in}}{\pgfqpoint{0.422474in}{0.425816in}}{\pgfqpoint{0.430287in}{0.418002in}}%
\pgfpathcurveto{\pgfqpoint{0.438101in}{0.410189in}}{\pgfqpoint{0.448700in}{0.405799in}}{\pgfqpoint{0.459750in}{0.405799in}}%
\pgfpathclose%
\pgfusepath{stroke,fill}%
\end{pgfscope}%
\begin{pgfscope}%
\pgfpathrectangle{\pgfqpoint{0.375000in}{0.330000in}}{\pgfqpoint{2.325000in}{2.310000in}}%
\pgfusepath{clip}%
\pgfsetbuttcap%
\pgfsetroundjoin%
\definecolor{currentfill}{rgb}{0.000000,0.000000,0.000000}%
\pgfsetfillcolor{currentfill}%
\pgfsetlinewidth{1.003750pt}%
\definecolor{currentstroke}{rgb}{0.000000,0.000000,0.000000}%
\pgfsetstrokecolor{currentstroke}%
\pgfsetdash{}{0pt}%
\pgfpathmoveto{\pgfqpoint{0.459750in}{0.411867in}}%
\pgfpathcurveto{\pgfqpoint{0.470800in}{0.411867in}}{\pgfqpoint{0.481399in}{0.416257in}}{\pgfqpoint{0.489213in}{0.424071in}}%
\pgfpathcurveto{\pgfqpoint{0.497026in}{0.431884in}}{\pgfqpoint{0.501417in}{0.442483in}}{\pgfqpoint{0.501417in}{0.453533in}}%
\pgfpathcurveto{\pgfqpoint{0.501417in}{0.464583in}}{\pgfqpoint{0.497026in}{0.475182in}}{\pgfqpoint{0.489213in}{0.482996in}}%
\pgfpathcurveto{\pgfqpoint{0.481399in}{0.490810in}}{\pgfqpoint{0.470800in}{0.495200in}}{\pgfqpoint{0.459750in}{0.495200in}}%
\pgfpathcurveto{\pgfqpoint{0.448700in}{0.495200in}}{\pgfqpoint{0.438101in}{0.490810in}}{\pgfqpoint{0.430287in}{0.482996in}}%
\pgfpathcurveto{\pgfqpoint{0.422474in}{0.475182in}}{\pgfqpoint{0.418083in}{0.464583in}}{\pgfqpoint{0.418083in}{0.453533in}}%
\pgfpathcurveto{\pgfqpoint{0.418083in}{0.442483in}}{\pgfqpoint{0.422474in}{0.431884in}}{\pgfqpoint{0.430287in}{0.424071in}}%
\pgfpathcurveto{\pgfqpoint{0.438101in}{0.416257in}}{\pgfqpoint{0.448700in}{0.411867in}}{\pgfqpoint{0.459750in}{0.411867in}}%
\pgfpathclose%
\pgfusepath{stroke,fill}%
\end{pgfscope}%
\begin{pgfscope}%
\pgfpathrectangle{\pgfqpoint{0.375000in}{0.330000in}}{\pgfqpoint{2.325000in}{2.310000in}}%
\pgfusepath{clip}%
\pgfsetbuttcap%
\pgfsetroundjoin%
\definecolor{currentfill}{rgb}{0.000000,0.000000,0.000000}%
\pgfsetfillcolor{currentfill}%
\pgfsetlinewidth{1.003750pt}%
\definecolor{currentstroke}{rgb}{0.000000,0.000000,0.000000}%
\pgfsetstrokecolor{currentstroke}%
\pgfsetdash{}{0pt}%
\pgfpathmoveto{\pgfqpoint{0.459750in}{0.411867in}}%
\pgfpathcurveto{\pgfqpoint{0.470800in}{0.411867in}}{\pgfqpoint{0.481399in}{0.416257in}}{\pgfqpoint{0.489213in}{0.424071in}}%
\pgfpathcurveto{\pgfqpoint{0.497026in}{0.431884in}}{\pgfqpoint{0.501417in}{0.442483in}}{\pgfqpoint{0.501417in}{0.453533in}}%
\pgfpathcurveto{\pgfqpoint{0.501417in}{0.464583in}}{\pgfqpoint{0.497026in}{0.475182in}}{\pgfqpoint{0.489213in}{0.482996in}}%
\pgfpathcurveto{\pgfqpoint{0.481399in}{0.490810in}}{\pgfqpoint{0.470800in}{0.495200in}}{\pgfqpoint{0.459750in}{0.495200in}}%
\pgfpathcurveto{\pgfqpoint{0.448700in}{0.495200in}}{\pgfqpoint{0.438101in}{0.490810in}}{\pgfqpoint{0.430287in}{0.482996in}}%
\pgfpathcurveto{\pgfqpoint{0.422474in}{0.475182in}}{\pgfqpoint{0.418083in}{0.464583in}}{\pgfqpoint{0.418083in}{0.453533in}}%
\pgfpathcurveto{\pgfqpoint{0.418083in}{0.442483in}}{\pgfqpoint{0.422474in}{0.431884in}}{\pgfqpoint{0.430287in}{0.424071in}}%
\pgfpathcurveto{\pgfqpoint{0.438101in}{0.416257in}}{\pgfqpoint{0.448700in}{0.411867in}}{\pgfqpoint{0.459750in}{0.411867in}}%
\pgfpathclose%
\pgfusepath{stroke,fill}%
\end{pgfscope}%
\begin{pgfscope}%
\pgfpathrectangle{\pgfqpoint{0.375000in}{0.330000in}}{\pgfqpoint{2.325000in}{2.310000in}}%
\pgfusepath{clip}%
\pgfsetbuttcap%
\pgfsetroundjoin%
\definecolor{currentfill}{rgb}{0.000000,0.000000,0.000000}%
\pgfsetfillcolor{currentfill}%
\pgfsetlinewidth{1.003750pt}%
\definecolor{currentstroke}{rgb}{0.000000,0.000000,0.000000}%
\pgfsetstrokecolor{currentstroke}%
\pgfsetdash{}{0pt}%
\pgfpathmoveto{\pgfqpoint{0.459750in}{0.424003in}}%
\pgfpathcurveto{\pgfqpoint{0.470800in}{0.424003in}}{\pgfqpoint{0.481399in}{0.428393in}}{\pgfqpoint{0.489213in}{0.436207in}}%
\pgfpathcurveto{\pgfqpoint{0.497026in}{0.444020in}}{\pgfqpoint{0.501417in}{0.454619in}}{\pgfqpoint{0.501417in}{0.465670in}}%
\pgfpathcurveto{\pgfqpoint{0.501417in}{0.476720in}}{\pgfqpoint{0.497026in}{0.487319in}}{\pgfqpoint{0.489213in}{0.495132in}}%
\pgfpathcurveto{\pgfqpoint{0.481399in}{0.502946in}}{\pgfqpoint{0.470800in}{0.507336in}}{\pgfqpoint{0.459750in}{0.507336in}}%
\pgfpathcurveto{\pgfqpoint{0.448700in}{0.507336in}}{\pgfqpoint{0.438101in}{0.502946in}}{\pgfqpoint{0.430287in}{0.495132in}}%
\pgfpathcurveto{\pgfqpoint{0.422474in}{0.487319in}}{\pgfqpoint{0.418083in}{0.476720in}}{\pgfqpoint{0.418083in}{0.465670in}}%
\pgfpathcurveto{\pgfqpoint{0.418083in}{0.454619in}}{\pgfqpoint{0.422474in}{0.444020in}}{\pgfqpoint{0.430287in}{0.436207in}}%
\pgfpathcurveto{\pgfqpoint{0.438101in}{0.428393in}}{\pgfqpoint{0.448700in}{0.424003in}}{\pgfqpoint{0.459750in}{0.424003in}}%
\pgfpathclose%
\pgfusepath{stroke,fill}%
\end{pgfscope}%
\begin{pgfscope}%
\pgfpathrectangle{\pgfqpoint{0.375000in}{0.330000in}}{\pgfqpoint{2.325000in}{2.310000in}}%
\pgfusepath{clip}%
\pgfsetbuttcap%
\pgfsetroundjoin%
\definecolor{currentfill}{rgb}{0.000000,0.000000,0.000000}%
\pgfsetfillcolor{currentfill}%
\pgfsetlinewidth{1.003750pt}%
\definecolor{currentstroke}{rgb}{0.000000,0.000000,0.000000}%
\pgfsetstrokecolor{currentstroke}%
\pgfsetdash{}{0pt}%
\pgfpathmoveto{\pgfqpoint{0.459750in}{0.424003in}}%
\pgfpathcurveto{\pgfqpoint{0.470800in}{0.424003in}}{\pgfqpoint{0.481399in}{0.428393in}}{\pgfqpoint{0.489213in}{0.436207in}}%
\pgfpathcurveto{\pgfqpoint{0.497026in}{0.444020in}}{\pgfqpoint{0.501417in}{0.454619in}}{\pgfqpoint{0.501417in}{0.465670in}}%
\pgfpathcurveto{\pgfqpoint{0.501417in}{0.476720in}}{\pgfqpoint{0.497026in}{0.487319in}}{\pgfqpoint{0.489213in}{0.495132in}}%
\pgfpathcurveto{\pgfqpoint{0.481399in}{0.502946in}}{\pgfqpoint{0.470800in}{0.507336in}}{\pgfqpoint{0.459750in}{0.507336in}}%
\pgfpathcurveto{\pgfqpoint{0.448700in}{0.507336in}}{\pgfqpoint{0.438101in}{0.502946in}}{\pgfqpoint{0.430287in}{0.495132in}}%
\pgfpathcurveto{\pgfqpoint{0.422474in}{0.487319in}}{\pgfqpoint{0.418083in}{0.476720in}}{\pgfqpoint{0.418083in}{0.465670in}}%
\pgfpathcurveto{\pgfqpoint{0.418083in}{0.454619in}}{\pgfqpoint{0.422474in}{0.444020in}}{\pgfqpoint{0.430287in}{0.436207in}}%
\pgfpathcurveto{\pgfqpoint{0.438101in}{0.428393in}}{\pgfqpoint{0.448700in}{0.424003in}}{\pgfqpoint{0.459750in}{0.424003in}}%
\pgfpathclose%
\pgfusepath{stroke,fill}%
\end{pgfscope}%
\begin{pgfscope}%
\pgfpathrectangle{\pgfqpoint{0.375000in}{0.330000in}}{\pgfqpoint{2.325000in}{2.310000in}}%
\pgfusepath{clip}%
\pgfsetbuttcap%
\pgfsetroundjoin%
\definecolor{currentfill}{rgb}{0.000000,0.000000,0.000000}%
\pgfsetfillcolor{currentfill}%
\pgfsetlinewidth{1.003750pt}%
\definecolor{currentstroke}{rgb}{0.000000,0.000000,0.000000}%
\pgfsetstrokecolor{currentstroke}%
\pgfsetdash{}{0pt}%
\pgfpathmoveto{\pgfqpoint{0.459750in}{0.417935in}}%
\pgfpathcurveto{\pgfqpoint{0.470800in}{0.417935in}}{\pgfqpoint{0.481399in}{0.422325in}}{\pgfqpoint{0.489213in}{0.430139in}}%
\pgfpathcurveto{\pgfqpoint{0.497026in}{0.437952in}}{\pgfqpoint{0.501417in}{0.448551in}}{\pgfqpoint{0.501417in}{0.459601in}}%
\pgfpathcurveto{\pgfqpoint{0.501417in}{0.470652in}}{\pgfqpoint{0.497026in}{0.481251in}}{\pgfqpoint{0.489213in}{0.489064in}}%
\pgfpathcurveto{\pgfqpoint{0.481399in}{0.496878in}}{\pgfqpoint{0.470800in}{0.501268in}}{\pgfqpoint{0.459750in}{0.501268in}}%
\pgfpathcurveto{\pgfqpoint{0.448700in}{0.501268in}}{\pgfqpoint{0.438101in}{0.496878in}}{\pgfqpoint{0.430287in}{0.489064in}}%
\pgfpathcurveto{\pgfqpoint{0.422474in}{0.481251in}}{\pgfqpoint{0.418083in}{0.470652in}}{\pgfqpoint{0.418083in}{0.459601in}}%
\pgfpathcurveto{\pgfqpoint{0.418083in}{0.448551in}}{\pgfqpoint{0.422474in}{0.437952in}}{\pgfqpoint{0.430287in}{0.430139in}}%
\pgfpathcurveto{\pgfqpoint{0.438101in}{0.422325in}}{\pgfqpoint{0.448700in}{0.417935in}}{\pgfqpoint{0.459750in}{0.417935in}}%
\pgfpathclose%
\pgfusepath{stroke,fill}%
\end{pgfscope}%
\begin{pgfscope}%
\pgfpathrectangle{\pgfqpoint{0.375000in}{0.330000in}}{\pgfqpoint{2.325000in}{2.310000in}}%
\pgfusepath{clip}%
\pgfsetbuttcap%
\pgfsetroundjoin%
\definecolor{currentfill}{rgb}{0.000000,0.000000,0.000000}%
\pgfsetfillcolor{currentfill}%
\pgfsetlinewidth{1.003750pt}%
\definecolor{currentstroke}{rgb}{0.000000,0.000000,0.000000}%
\pgfsetstrokecolor{currentstroke}%
\pgfsetdash{}{0pt}%
\pgfpathmoveto{\pgfqpoint{0.459750in}{0.411867in}}%
\pgfpathcurveto{\pgfqpoint{0.470800in}{0.411867in}}{\pgfqpoint{0.481399in}{0.416257in}}{\pgfqpoint{0.489213in}{0.424071in}}%
\pgfpathcurveto{\pgfqpoint{0.497026in}{0.431884in}}{\pgfqpoint{0.501417in}{0.442483in}}{\pgfqpoint{0.501417in}{0.453533in}}%
\pgfpathcurveto{\pgfqpoint{0.501417in}{0.464583in}}{\pgfqpoint{0.497026in}{0.475182in}}{\pgfqpoint{0.489213in}{0.482996in}}%
\pgfpathcurveto{\pgfqpoint{0.481399in}{0.490810in}}{\pgfqpoint{0.470800in}{0.495200in}}{\pgfqpoint{0.459750in}{0.495200in}}%
\pgfpathcurveto{\pgfqpoint{0.448700in}{0.495200in}}{\pgfqpoint{0.438101in}{0.490810in}}{\pgfqpoint{0.430287in}{0.482996in}}%
\pgfpathcurveto{\pgfqpoint{0.422474in}{0.475182in}}{\pgfqpoint{0.418083in}{0.464583in}}{\pgfqpoint{0.418083in}{0.453533in}}%
\pgfpathcurveto{\pgfqpoint{0.418083in}{0.442483in}}{\pgfqpoint{0.422474in}{0.431884in}}{\pgfqpoint{0.430287in}{0.424071in}}%
\pgfpathcurveto{\pgfqpoint{0.438101in}{0.416257in}}{\pgfqpoint{0.448700in}{0.411867in}}{\pgfqpoint{0.459750in}{0.411867in}}%
\pgfpathclose%
\pgfusepath{stroke,fill}%
\end{pgfscope}%
\begin{pgfscope}%
\pgfpathrectangle{\pgfqpoint{0.375000in}{0.330000in}}{\pgfqpoint{2.325000in}{2.310000in}}%
\pgfusepath{clip}%
\pgfsetbuttcap%
\pgfsetroundjoin%
\definecolor{currentfill}{rgb}{0.000000,0.000000,0.000000}%
\pgfsetfillcolor{currentfill}%
\pgfsetlinewidth{1.003750pt}%
\definecolor{currentstroke}{rgb}{0.000000,0.000000,0.000000}%
\pgfsetstrokecolor{currentstroke}%
\pgfsetdash{}{0pt}%
\pgfpathmoveto{\pgfqpoint{0.459750in}{0.417935in}}%
\pgfpathcurveto{\pgfqpoint{0.470800in}{0.417935in}}{\pgfqpoint{0.481399in}{0.422325in}}{\pgfqpoint{0.489213in}{0.430139in}}%
\pgfpathcurveto{\pgfqpoint{0.497026in}{0.437952in}}{\pgfqpoint{0.501417in}{0.448551in}}{\pgfqpoint{0.501417in}{0.459601in}}%
\pgfpathcurveto{\pgfqpoint{0.501417in}{0.470652in}}{\pgfqpoint{0.497026in}{0.481251in}}{\pgfqpoint{0.489213in}{0.489064in}}%
\pgfpathcurveto{\pgfqpoint{0.481399in}{0.496878in}}{\pgfqpoint{0.470800in}{0.501268in}}{\pgfqpoint{0.459750in}{0.501268in}}%
\pgfpathcurveto{\pgfqpoint{0.448700in}{0.501268in}}{\pgfqpoint{0.438101in}{0.496878in}}{\pgfqpoint{0.430287in}{0.489064in}}%
\pgfpathcurveto{\pgfqpoint{0.422474in}{0.481251in}}{\pgfqpoint{0.418083in}{0.470652in}}{\pgfqpoint{0.418083in}{0.459601in}}%
\pgfpathcurveto{\pgfqpoint{0.418083in}{0.448551in}}{\pgfqpoint{0.422474in}{0.437952in}}{\pgfqpoint{0.430287in}{0.430139in}}%
\pgfpathcurveto{\pgfqpoint{0.438101in}{0.422325in}}{\pgfqpoint{0.448700in}{0.417935in}}{\pgfqpoint{0.459750in}{0.417935in}}%
\pgfpathclose%
\pgfusepath{stroke,fill}%
\end{pgfscope}%
\begin{pgfscope}%
\pgfpathrectangle{\pgfqpoint{0.375000in}{0.330000in}}{\pgfqpoint{2.325000in}{2.310000in}}%
\pgfusepath{clip}%
\pgfsetbuttcap%
\pgfsetroundjoin%
\definecolor{currentfill}{rgb}{0.000000,0.000000,0.000000}%
\pgfsetfillcolor{currentfill}%
\pgfsetlinewidth{1.003750pt}%
\definecolor{currentstroke}{rgb}{0.000000,0.000000,0.000000}%
\pgfsetstrokecolor{currentstroke}%
\pgfsetdash{}{0pt}%
\pgfpathmoveto{\pgfqpoint{0.459750in}{0.424003in}}%
\pgfpathcurveto{\pgfqpoint{0.470800in}{0.424003in}}{\pgfqpoint{0.481399in}{0.428393in}}{\pgfqpoint{0.489213in}{0.436207in}}%
\pgfpathcurveto{\pgfqpoint{0.497026in}{0.444020in}}{\pgfqpoint{0.501417in}{0.454619in}}{\pgfqpoint{0.501417in}{0.465670in}}%
\pgfpathcurveto{\pgfqpoint{0.501417in}{0.476720in}}{\pgfqpoint{0.497026in}{0.487319in}}{\pgfqpoint{0.489213in}{0.495132in}}%
\pgfpathcurveto{\pgfqpoint{0.481399in}{0.502946in}}{\pgfqpoint{0.470800in}{0.507336in}}{\pgfqpoint{0.459750in}{0.507336in}}%
\pgfpathcurveto{\pgfqpoint{0.448700in}{0.507336in}}{\pgfqpoint{0.438101in}{0.502946in}}{\pgfqpoint{0.430287in}{0.495132in}}%
\pgfpathcurveto{\pgfqpoint{0.422474in}{0.487319in}}{\pgfqpoint{0.418083in}{0.476720in}}{\pgfqpoint{0.418083in}{0.465670in}}%
\pgfpathcurveto{\pgfqpoint{0.418083in}{0.454619in}}{\pgfqpoint{0.422474in}{0.444020in}}{\pgfqpoint{0.430287in}{0.436207in}}%
\pgfpathcurveto{\pgfqpoint{0.438101in}{0.428393in}}{\pgfqpoint{0.448700in}{0.424003in}}{\pgfqpoint{0.459750in}{0.424003in}}%
\pgfpathclose%
\pgfusepath{stroke,fill}%
\end{pgfscope}%
\begin{pgfscope}%
\pgfpathrectangle{\pgfqpoint{0.375000in}{0.330000in}}{\pgfqpoint{2.325000in}{2.310000in}}%
\pgfusepath{clip}%
\pgfsetbuttcap%
\pgfsetroundjoin%
\definecolor{currentfill}{rgb}{0.000000,0.000000,0.000000}%
\pgfsetfillcolor{currentfill}%
\pgfsetlinewidth{1.003750pt}%
\definecolor{currentstroke}{rgb}{0.000000,0.000000,0.000000}%
\pgfsetstrokecolor{currentstroke}%
\pgfsetdash{}{0pt}%
\pgfpathmoveto{\pgfqpoint{0.459750in}{0.411867in}}%
\pgfpathcurveto{\pgfqpoint{0.470800in}{0.411867in}}{\pgfqpoint{0.481399in}{0.416257in}}{\pgfqpoint{0.489213in}{0.424071in}}%
\pgfpathcurveto{\pgfqpoint{0.497026in}{0.431884in}}{\pgfqpoint{0.501417in}{0.442483in}}{\pgfqpoint{0.501417in}{0.453533in}}%
\pgfpathcurveto{\pgfqpoint{0.501417in}{0.464583in}}{\pgfqpoint{0.497026in}{0.475182in}}{\pgfqpoint{0.489213in}{0.482996in}}%
\pgfpathcurveto{\pgfqpoint{0.481399in}{0.490810in}}{\pgfqpoint{0.470800in}{0.495200in}}{\pgfqpoint{0.459750in}{0.495200in}}%
\pgfpathcurveto{\pgfqpoint{0.448700in}{0.495200in}}{\pgfqpoint{0.438101in}{0.490810in}}{\pgfqpoint{0.430287in}{0.482996in}}%
\pgfpathcurveto{\pgfqpoint{0.422474in}{0.475182in}}{\pgfqpoint{0.418083in}{0.464583in}}{\pgfqpoint{0.418083in}{0.453533in}}%
\pgfpathcurveto{\pgfqpoint{0.418083in}{0.442483in}}{\pgfqpoint{0.422474in}{0.431884in}}{\pgfqpoint{0.430287in}{0.424071in}}%
\pgfpathcurveto{\pgfqpoint{0.438101in}{0.416257in}}{\pgfqpoint{0.448700in}{0.411867in}}{\pgfqpoint{0.459750in}{0.411867in}}%
\pgfpathclose%
\pgfusepath{stroke,fill}%
\end{pgfscope}%
\begin{pgfscope}%
\pgfpathrectangle{\pgfqpoint{0.375000in}{0.330000in}}{\pgfqpoint{2.325000in}{2.310000in}}%
\pgfusepath{clip}%
\pgfsetbuttcap%
\pgfsetroundjoin%
\definecolor{currentfill}{rgb}{0.000000,0.000000,0.000000}%
\pgfsetfillcolor{currentfill}%
\pgfsetlinewidth{1.003750pt}%
\definecolor{currentstroke}{rgb}{0.000000,0.000000,0.000000}%
\pgfsetstrokecolor{currentstroke}%
\pgfsetdash{}{0pt}%
\pgfpathmoveto{\pgfqpoint{0.459750in}{0.405799in}}%
\pgfpathcurveto{\pgfqpoint{0.470800in}{0.405799in}}{\pgfqpoint{0.481399in}{0.410189in}}{\pgfqpoint{0.489213in}{0.418002in}}%
\pgfpathcurveto{\pgfqpoint{0.497026in}{0.425816in}}{\pgfqpoint{0.501417in}{0.436415in}}{\pgfqpoint{0.501417in}{0.447465in}}%
\pgfpathcurveto{\pgfqpoint{0.501417in}{0.458515in}}{\pgfqpoint{0.497026in}{0.469114in}}{\pgfqpoint{0.489213in}{0.476928in}}%
\pgfpathcurveto{\pgfqpoint{0.481399in}{0.484742in}}{\pgfqpoint{0.470800in}{0.489132in}}{\pgfqpoint{0.459750in}{0.489132in}}%
\pgfpathcurveto{\pgfqpoint{0.448700in}{0.489132in}}{\pgfqpoint{0.438101in}{0.484742in}}{\pgfqpoint{0.430287in}{0.476928in}}%
\pgfpathcurveto{\pgfqpoint{0.422474in}{0.469114in}}{\pgfqpoint{0.418083in}{0.458515in}}{\pgfqpoint{0.418083in}{0.447465in}}%
\pgfpathcurveto{\pgfqpoint{0.418083in}{0.436415in}}{\pgfqpoint{0.422474in}{0.425816in}}{\pgfqpoint{0.430287in}{0.418002in}}%
\pgfpathcurveto{\pgfqpoint{0.438101in}{0.410189in}}{\pgfqpoint{0.448700in}{0.405799in}}{\pgfqpoint{0.459750in}{0.405799in}}%
\pgfpathclose%
\pgfusepath{stroke,fill}%
\end{pgfscope}%
\begin{pgfscope}%
\pgfpathrectangle{\pgfqpoint{0.375000in}{0.330000in}}{\pgfqpoint{2.325000in}{2.310000in}}%
\pgfusepath{clip}%
\pgfsetbuttcap%
\pgfsetroundjoin%
\definecolor{currentfill}{rgb}{0.000000,0.000000,0.000000}%
\pgfsetfillcolor{currentfill}%
\pgfsetlinewidth{1.003750pt}%
\definecolor{currentstroke}{rgb}{0.000000,0.000000,0.000000}%
\pgfsetstrokecolor{currentstroke}%
\pgfsetdash{}{0pt}%
\pgfpathmoveto{\pgfqpoint{0.459750in}{0.399730in}}%
\pgfpathcurveto{\pgfqpoint{0.470800in}{0.399730in}}{\pgfqpoint{0.481399in}{0.404121in}}{\pgfqpoint{0.489213in}{0.411934in}}%
\pgfpathcurveto{\pgfqpoint{0.497026in}{0.419748in}}{\pgfqpoint{0.501417in}{0.430347in}}{\pgfqpoint{0.501417in}{0.441397in}}%
\pgfpathcurveto{\pgfqpoint{0.501417in}{0.452447in}}{\pgfqpoint{0.497026in}{0.463046in}}{\pgfqpoint{0.489213in}{0.470860in}}%
\pgfpathcurveto{\pgfqpoint{0.481399in}{0.478674in}}{\pgfqpoint{0.470800in}{0.483064in}}{\pgfqpoint{0.459750in}{0.483064in}}%
\pgfpathcurveto{\pgfqpoint{0.448700in}{0.483064in}}{\pgfqpoint{0.438101in}{0.478674in}}{\pgfqpoint{0.430287in}{0.470860in}}%
\pgfpathcurveto{\pgfqpoint{0.422474in}{0.463046in}}{\pgfqpoint{0.418083in}{0.452447in}}{\pgfqpoint{0.418083in}{0.441397in}}%
\pgfpathcurveto{\pgfqpoint{0.418083in}{0.430347in}}{\pgfqpoint{0.422474in}{0.419748in}}{\pgfqpoint{0.430287in}{0.411934in}}%
\pgfpathcurveto{\pgfqpoint{0.438101in}{0.404121in}}{\pgfqpoint{0.448700in}{0.399730in}}{\pgfqpoint{0.459750in}{0.399730in}}%
\pgfpathclose%
\pgfusepath{stroke,fill}%
\end{pgfscope}%
\begin{pgfscope}%
\pgfpathrectangle{\pgfqpoint{0.375000in}{0.330000in}}{\pgfqpoint{2.325000in}{2.310000in}}%
\pgfusepath{clip}%
\pgfsetbuttcap%
\pgfsetroundjoin%
\definecolor{currentfill}{rgb}{0.000000,0.000000,0.000000}%
\pgfsetfillcolor{currentfill}%
\pgfsetlinewidth{1.003750pt}%
\definecolor{currentstroke}{rgb}{0.000000,0.000000,0.000000}%
\pgfsetstrokecolor{currentstroke}%
\pgfsetdash{}{0pt}%
\pgfpathmoveto{\pgfqpoint{0.459750in}{0.411867in}}%
\pgfpathcurveto{\pgfqpoint{0.470800in}{0.411867in}}{\pgfqpoint{0.481399in}{0.416257in}}{\pgfqpoint{0.489213in}{0.424071in}}%
\pgfpathcurveto{\pgfqpoint{0.497026in}{0.431884in}}{\pgfqpoint{0.501417in}{0.442483in}}{\pgfqpoint{0.501417in}{0.453533in}}%
\pgfpathcurveto{\pgfqpoint{0.501417in}{0.464583in}}{\pgfqpoint{0.497026in}{0.475182in}}{\pgfqpoint{0.489213in}{0.482996in}}%
\pgfpathcurveto{\pgfqpoint{0.481399in}{0.490810in}}{\pgfqpoint{0.470800in}{0.495200in}}{\pgfqpoint{0.459750in}{0.495200in}}%
\pgfpathcurveto{\pgfqpoint{0.448700in}{0.495200in}}{\pgfqpoint{0.438101in}{0.490810in}}{\pgfqpoint{0.430287in}{0.482996in}}%
\pgfpathcurveto{\pgfqpoint{0.422474in}{0.475182in}}{\pgfqpoint{0.418083in}{0.464583in}}{\pgfqpoint{0.418083in}{0.453533in}}%
\pgfpathcurveto{\pgfqpoint{0.418083in}{0.442483in}}{\pgfqpoint{0.422474in}{0.431884in}}{\pgfqpoint{0.430287in}{0.424071in}}%
\pgfpathcurveto{\pgfqpoint{0.438101in}{0.416257in}}{\pgfqpoint{0.448700in}{0.411867in}}{\pgfqpoint{0.459750in}{0.411867in}}%
\pgfpathclose%
\pgfusepath{stroke,fill}%
\end{pgfscope}%
\begin{pgfscope}%
\pgfpathrectangle{\pgfqpoint{0.375000in}{0.330000in}}{\pgfqpoint{2.325000in}{2.310000in}}%
\pgfusepath{clip}%
\pgfsetbuttcap%
\pgfsetroundjoin%
\definecolor{currentfill}{rgb}{0.000000,0.000000,0.000000}%
\pgfsetfillcolor{currentfill}%
\pgfsetlinewidth{1.003750pt}%
\definecolor{currentstroke}{rgb}{0.000000,0.000000,0.000000}%
\pgfsetstrokecolor{currentstroke}%
\pgfsetdash{}{0pt}%
\pgfpathmoveto{\pgfqpoint{0.459750in}{0.424003in}}%
\pgfpathcurveto{\pgfqpoint{0.470800in}{0.424003in}}{\pgfqpoint{0.481399in}{0.428393in}}{\pgfqpoint{0.489213in}{0.436207in}}%
\pgfpathcurveto{\pgfqpoint{0.497026in}{0.444020in}}{\pgfqpoint{0.501417in}{0.454619in}}{\pgfqpoint{0.501417in}{0.465670in}}%
\pgfpathcurveto{\pgfqpoint{0.501417in}{0.476720in}}{\pgfqpoint{0.497026in}{0.487319in}}{\pgfqpoint{0.489213in}{0.495132in}}%
\pgfpathcurveto{\pgfqpoint{0.481399in}{0.502946in}}{\pgfqpoint{0.470800in}{0.507336in}}{\pgfqpoint{0.459750in}{0.507336in}}%
\pgfpathcurveto{\pgfqpoint{0.448700in}{0.507336in}}{\pgfqpoint{0.438101in}{0.502946in}}{\pgfqpoint{0.430287in}{0.495132in}}%
\pgfpathcurveto{\pgfqpoint{0.422474in}{0.487319in}}{\pgfqpoint{0.418083in}{0.476720in}}{\pgfqpoint{0.418083in}{0.465670in}}%
\pgfpathcurveto{\pgfqpoint{0.418083in}{0.454619in}}{\pgfqpoint{0.422474in}{0.444020in}}{\pgfqpoint{0.430287in}{0.436207in}}%
\pgfpathcurveto{\pgfqpoint{0.438101in}{0.428393in}}{\pgfqpoint{0.448700in}{0.424003in}}{\pgfqpoint{0.459750in}{0.424003in}}%
\pgfpathclose%
\pgfusepath{stroke,fill}%
\end{pgfscope}%
\begin{pgfscope}%
\pgfpathrectangle{\pgfqpoint{0.375000in}{0.330000in}}{\pgfqpoint{2.325000in}{2.310000in}}%
\pgfusepath{clip}%
\pgfsetbuttcap%
\pgfsetroundjoin%
\definecolor{currentfill}{rgb}{0.000000,0.000000,0.000000}%
\pgfsetfillcolor{currentfill}%
\pgfsetlinewidth{1.003750pt}%
\definecolor{currentstroke}{rgb}{0.000000,0.000000,0.000000}%
\pgfsetstrokecolor{currentstroke}%
\pgfsetdash{}{0pt}%
\pgfpathmoveto{\pgfqpoint{0.459750in}{0.424003in}}%
\pgfpathcurveto{\pgfqpoint{0.470800in}{0.424003in}}{\pgfqpoint{0.481399in}{0.428393in}}{\pgfqpoint{0.489213in}{0.436207in}}%
\pgfpathcurveto{\pgfqpoint{0.497026in}{0.444020in}}{\pgfqpoint{0.501417in}{0.454619in}}{\pgfqpoint{0.501417in}{0.465670in}}%
\pgfpathcurveto{\pgfqpoint{0.501417in}{0.476720in}}{\pgfqpoint{0.497026in}{0.487319in}}{\pgfqpoint{0.489213in}{0.495132in}}%
\pgfpathcurveto{\pgfqpoint{0.481399in}{0.502946in}}{\pgfqpoint{0.470800in}{0.507336in}}{\pgfqpoint{0.459750in}{0.507336in}}%
\pgfpathcurveto{\pgfqpoint{0.448700in}{0.507336in}}{\pgfqpoint{0.438101in}{0.502946in}}{\pgfqpoint{0.430287in}{0.495132in}}%
\pgfpathcurveto{\pgfqpoint{0.422474in}{0.487319in}}{\pgfqpoint{0.418083in}{0.476720in}}{\pgfqpoint{0.418083in}{0.465670in}}%
\pgfpathcurveto{\pgfqpoint{0.418083in}{0.454619in}}{\pgfqpoint{0.422474in}{0.444020in}}{\pgfqpoint{0.430287in}{0.436207in}}%
\pgfpathcurveto{\pgfqpoint{0.438101in}{0.428393in}}{\pgfqpoint{0.448700in}{0.424003in}}{\pgfqpoint{0.459750in}{0.424003in}}%
\pgfpathclose%
\pgfusepath{stroke,fill}%
\end{pgfscope}%
\begin{pgfscope}%
\pgfpathrectangle{\pgfqpoint{0.375000in}{0.330000in}}{\pgfqpoint{2.325000in}{2.310000in}}%
\pgfusepath{clip}%
\pgfsetbuttcap%
\pgfsetroundjoin%
\definecolor{currentfill}{rgb}{0.000000,0.000000,0.000000}%
\pgfsetfillcolor{currentfill}%
\pgfsetlinewidth{1.003750pt}%
\definecolor{currentstroke}{rgb}{0.000000,0.000000,0.000000}%
\pgfsetstrokecolor{currentstroke}%
\pgfsetdash{}{0pt}%
\pgfpathmoveto{\pgfqpoint{0.459750in}{0.424003in}}%
\pgfpathcurveto{\pgfqpoint{0.470800in}{0.424003in}}{\pgfqpoint{0.481399in}{0.428393in}}{\pgfqpoint{0.489213in}{0.436207in}}%
\pgfpathcurveto{\pgfqpoint{0.497026in}{0.444020in}}{\pgfqpoint{0.501417in}{0.454619in}}{\pgfqpoint{0.501417in}{0.465670in}}%
\pgfpathcurveto{\pgfqpoint{0.501417in}{0.476720in}}{\pgfqpoint{0.497026in}{0.487319in}}{\pgfqpoint{0.489213in}{0.495132in}}%
\pgfpathcurveto{\pgfqpoint{0.481399in}{0.502946in}}{\pgfqpoint{0.470800in}{0.507336in}}{\pgfqpoint{0.459750in}{0.507336in}}%
\pgfpathcurveto{\pgfqpoint{0.448700in}{0.507336in}}{\pgfqpoint{0.438101in}{0.502946in}}{\pgfqpoint{0.430287in}{0.495132in}}%
\pgfpathcurveto{\pgfqpoint{0.422474in}{0.487319in}}{\pgfqpoint{0.418083in}{0.476720in}}{\pgfqpoint{0.418083in}{0.465670in}}%
\pgfpathcurveto{\pgfqpoint{0.418083in}{0.454619in}}{\pgfqpoint{0.422474in}{0.444020in}}{\pgfqpoint{0.430287in}{0.436207in}}%
\pgfpathcurveto{\pgfqpoint{0.438101in}{0.428393in}}{\pgfqpoint{0.448700in}{0.424003in}}{\pgfqpoint{0.459750in}{0.424003in}}%
\pgfpathclose%
\pgfusepath{stroke,fill}%
\end{pgfscope}%
\begin{pgfscope}%
\pgfpathrectangle{\pgfqpoint{0.375000in}{0.330000in}}{\pgfqpoint{2.325000in}{2.310000in}}%
\pgfusepath{clip}%
\pgfsetbuttcap%
\pgfsetroundjoin%
\definecolor{currentfill}{rgb}{0.000000,0.000000,0.000000}%
\pgfsetfillcolor{currentfill}%
\pgfsetlinewidth{1.003750pt}%
\definecolor{currentstroke}{rgb}{0.000000,0.000000,0.000000}%
\pgfsetstrokecolor{currentstroke}%
\pgfsetdash{}{0pt}%
\pgfpathmoveto{\pgfqpoint{0.459750in}{0.424003in}}%
\pgfpathcurveto{\pgfqpoint{0.470800in}{0.424003in}}{\pgfqpoint{0.481399in}{0.428393in}}{\pgfqpoint{0.489213in}{0.436207in}}%
\pgfpathcurveto{\pgfqpoint{0.497026in}{0.444020in}}{\pgfqpoint{0.501417in}{0.454619in}}{\pgfqpoint{0.501417in}{0.465670in}}%
\pgfpathcurveto{\pgfqpoint{0.501417in}{0.476720in}}{\pgfqpoint{0.497026in}{0.487319in}}{\pgfqpoint{0.489213in}{0.495132in}}%
\pgfpathcurveto{\pgfqpoint{0.481399in}{0.502946in}}{\pgfqpoint{0.470800in}{0.507336in}}{\pgfqpoint{0.459750in}{0.507336in}}%
\pgfpathcurveto{\pgfqpoint{0.448700in}{0.507336in}}{\pgfqpoint{0.438101in}{0.502946in}}{\pgfqpoint{0.430287in}{0.495132in}}%
\pgfpathcurveto{\pgfqpoint{0.422474in}{0.487319in}}{\pgfqpoint{0.418083in}{0.476720in}}{\pgfqpoint{0.418083in}{0.465670in}}%
\pgfpathcurveto{\pgfqpoint{0.418083in}{0.454619in}}{\pgfqpoint{0.422474in}{0.444020in}}{\pgfqpoint{0.430287in}{0.436207in}}%
\pgfpathcurveto{\pgfqpoint{0.438101in}{0.428393in}}{\pgfqpoint{0.448700in}{0.424003in}}{\pgfqpoint{0.459750in}{0.424003in}}%
\pgfpathclose%
\pgfusepath{stroke,fill}%
\end{pgfscope}%
\begin{pgfscope}%
\pgfpathrectangle{\pgfqpoint{0.375000in}{0.330000in}}{\pgfqpoint{2.325000in}{2.310000in}}%
\pgfusepath{clip}%
\pgfsetbuttcap%
\pgfsetroundjoin%
\definecolor{currentfill}{rgb}{0.000000,0.000000,0.000000}%
\pgfsetfillcolor{currentfill}%
\pgfsetlinewidth{1.003750pt}%
\definecolor{currentstroke}{rgb}{0.000000,0.000000,0.000000}%
\pgfsetstrokecolor{currentstroke}%
\pgfsetdash{}{0pt}%
\pgfpathmoveto{\pgfqpoint{0.459750in}{0.417935in}}%
\pgfpathcurveto{\pgfqpoint{0.470800in}{0.417935in}}{\pgfqpoint{0.481399in}{0.422325in}}{\pgfqpoint{0.489213in}{0.430139in}}%
\pgfpathcurveto{\pgfqpoint{0.497026in}{0.437952in}}{\pgfqpoint{0.501417in}{0.448551in}}{\pgfqpoint{0.501417in}{0.459601in}}%
\pgfpathcurveto{\pgfqpoint{0.501417in}{0.470652in}}{\pgfqpoint{0.497026in}{0.481251in}}{\pgfqpoint{0.489213in}{0.489064in}}%
\pgfpathcurveto{\pgfqpoint{0.481399in}{0.496878in}}{\pgfqpoint{0.470800in}{0.501268in}}{\pgfqpoint{0.459750in}{0.501268in}}%
\pgfpathcurveto{\pgfqpoint{0.448700in}{0.501268in}}{\pgfqpoint{0.438101in}{0.496878in}}{\pgfqpoint{0.430287in}{0.489064in}}%
\pgfpathcurveto{\pgfqpoint{0.422474in}{0.481251in}}{\pgfqpoint{0.418083in}{0.470652in}}{\pgfqpoint{0.418083in}{0.459601in}}%
\pgfpathcurveto{\pgfqpoint{0.418083in}{0.448551in}}{\pgfqpoint{0.422474in}{0.437952in}}{\pgfqpoint{0.430287in}{0.430139in}}%
\pgfpathcurveto{\pgfqpoint{0.438101in}{0.422325in}}{\pgfqpoint{0.448700in}{0.417935in}}{\pgfqpoint{0.459750in}{0.417935in}}%
\pgfpathclose%
\pgfusepath{stroke,fill}%
\end{pgfscope}%
\begin{pgfscope}%
\pgfpathrectangle{\pgfqpoint{0.375000in}{0.330000in}}{\pgfqpoint{2.325000in}{2.310000in}}%
\pgfusepath{clip}%
\pgfsetbuttcap%
\pgfsetroundjoin%
\definecolor{currentfill}{rgb}{0.000000,0.000000,0.000000}%
\pgfsetfillcolor{currentfill}%
\pgfsetlinewidth{1.003750pt}%
\definecolor{currentstroke}{rgb}{0.000000,0.000000,0.000000}%
\pgfsetstrokecolor{currentstroke}%
\pgfsetdash{}{0pt}%
\pgfpathmoveto{\pgfqpoint{0.459750in}{0.424003in}}%
\pgfpathcurveto{\pgfqpoint{0.470800in}{0.424003in}}{\pgfqpoint{0.481399in}{0.428393in}}{\pgfqpoint{0.489213in}{0.436207in}}%
\pgfpathcurveto{\pgfqpoint{0.497026in}{0.444020in}}{\pgfqpoint{0.501417in}{0.454619in}}{\pgfqpoint{0.501417in}{0.465670in}}%
\pgfpathcurveto{\pgfqpoint{0.501417in}{0.476720in}}{\pgfqpoint{0.497026in}{0.487319in}}{\pgfqpoint{0.489213in}{0.495132in}}%
\pgfpathcurveto{\pgfqpoint{0.481399in}{0.502946in}}{\pgfqpoint{0.470800in}{0.507336in}}{\pgfqpoint{0.459750in}{0.507336in}}%
\pgfpathcurveto{\pgfqpoint{0.448700in}{0.507336in}}{\pgfqpoint{0.438101in}{0.502946in}}{\pgfqpoint{0.430287in}{0.495132in}}%
\pgfpathcurveto{\pgfqpoint{0.422474in}{0.487319in}}{\pgfqpoint{0.418083in}{0.476720in}}{\pgfqpoint{0.418083in}{0.465670in}}%
\pgfpathcurveto{\pgfqpoint{0.418083in}{0.454619in}}{\pgfqpoint{0.422474in}{0.444020in}}{\pgfqpoint{0.430287in}{0.436207in}}%
\pgfpathcurveto{\pgfqpoint{0.438101in}{0.428393in}}{\pgfqpoint{0.448700in}{0.424003in}}{\pgfqpoint{0.459750in}{0.424003in}}%
\pgfpathclose%
\pgfusepath{stroke,fill}%
\end{pgfscope}%
\begin{pgfscope}%
\pgfpathrectangle{\pgfqpoint{0.375000in}{0.330000in}}{\pgfqpoint{2.325000in}{2.310000in}}%
\pgfusepath{clip}%
\pgfsetbuttcap%
\pgfsetroundjoin%
\definecolor{currentfill}{rgb}{0.000000,0.000000,0.000000}%
\pgfsetfillcolor{currentfill}%
\pgfsetlinewidth{1.003750pt}%
\definecolor{currentstroke}{rgb}{0.000000,0.000000,0.000000}%
\pgfsetstrokecolor{currentstroke}%
\pgfsetdash{}{0pt}%
\pgfpathmoveto{\pgfqpoint{0.459750in}{0.405799in}}%
\pgfpathcurveto{\pgfqpoint{0.470800in}{0.405799in}}{\pgfqpoint{0.481399in}{0.410189in}}{\pgfqpoint{0.489213in}{0.418002in}}%
\pgfpathcurveto{\pgfqpoint{0.497026in}{0.425816in}}{\pgfqpoint{0.501417in}{0.436415in}}{\pgfqpoint{0.501417in}{0.447465in}}%
\pgfpathcurveto{\pgfqpoint{0.501417in}{0.458515in}}{\pgfqpoint{0.497026in}{0.469114in}}{\pgfqpoint{0.489213in}{0.476928in}}%
\pgfpathcurveto{\pgfqpoint{0.481399in}{0.484742in}}{\pgfqpoint{0.470800in}{0.489132in}}{\pgfqpoint{0.459750in}{0.489132in}}%
\pgfpathcurveto{\pgfqpoint{0.448700in}{0.489132in}}{\pgfqpoint{0.438101in}{0.484742in}}{\pgfqpoint{0.430287in}{0.476928in}}%
\pgfpathcurveto{\pgfqpoint{0.422474in}{0.469114in}}{\pgfqpoint{0.418083in}{0.458515in}}{\pgfqpoint{0.418083in}{0.447465in}}%
\pgfpathcurveto{\pgfqpoint{0.418083in}{0.436415in}}{\pgfqpoint{0.422474in}{0.425816in}}{\pgfqpoint{0.430287in}{0.418002in}}%
\pgfpathcurveto{\pgfqpoint{0.438101in}{0.410189in}}{\pgfqpoint{0.448700in}{0.405799in}}{\pgfqpoint{0.459750in}{0.405799in}}%
\pgfpathclose%
\pgfusepath{stroke,fill}%
\end{pgfscope}%
\begin{pgfscope}%
\pgfpathrectangle{\pgfqpoint{0.375000in}{0.330000in}}{\pgfqpoint{2.325000in}{2.310000in}}%
\pgfusepath{clip}%
\pgfsetbuttcap%
\pgfsetroundjoin%
\definecolor{currentfill}{rgb}{0.000000,0.000000,0.000000}%
\pgfsetfillcolor{currentfill}%
\pgfsetlinewidth{1.003750pt}%
\definecolor{currentstroke}{rgb}{0.000000,0.000000,0.000000}%
\pgfsetstrokecolor{currentstroke}%
\pgfsetdash{}{0pt}%
\pgfpathmoveto{\pgfqpoint{0.459750in}{0.405799in}}%
\pgfpathcurveto{\pgfqpoint{0.470800in}{0.405799in}}{\pgfqpoint{0.481399in}{0.410189in}}{\pgfqpoint{0.489213in}{0.418002in}}%
\pgfpathcurveto{\pgfqpoint{0.497026in}{0.425816in}}{\pgfqpoint{0.501417in}{0.436415in}}{\pgfqpoint{0.501417in}{0.447465in}}%
\pgfpathcurveto{\pgfqpoint{0.501417in}{0.458515in}}{\pgfqpoint{0.497026in}{0.469114in}}{\pgfqpoint{0.489213in}{0.476928in}}%
\pgfpathcurveto{\pgfqpoint{0.481399in}{0.484742in}}{\pgfqpoint{0.470800in}{0.489132in}}{\pgfqpoint{0.459750in}{0.489132in}}%
\pgfpathcurveto{\pgfqpoint{0.448700in}{0.489132in}}{\pgfqpoint{0.438101in}{0.484742in}}{\pgfqpoint{0.430287in}{0.476928in}}%
\pgfpathcurveto{\pgfqpoint{0.422474in}{0.469114in}}{\pgfqpoint{0.418083in}{0.458515in}}{\pgfqpoint{0.418083in}{0.447465in}}%
\pgfpathcurveto{\pgfqpoint{0.418083in}{0.436415in}}{\pgfqpoint{0.422474in}{0.425816in}}{\pgfqpoint{0.430287in}{0.418002in}}%
\pgfpathcurveto{\pgfqpoint{0.438101in}{0.410189in}}{\pgfqpoint{0.448700in}{0.405799in}}{\pgfqpoint{0.459750in}{0.405799in}}%
\pgfpathclose%
\pgfusepath{stroke,fill}%
\end{pgfscope}%
\begin{pgfscope}%
\pgfpathrectangle{\pgfqpoint{0.375000in}{0.330000in}}{\pgfqpoint{2.325000in}{2.310000in}}%
\pgfusepath{clip}%
\pgfsetbuttcap%
\pgfsetroundjoin%
\definecolor{currentfill}{rgb}{0.000000,0.000000,0.000000}%
\pgfsetfillcolor{currentfill}%
\pgfsetlinewidth{1.003750pt}%
\definecolor{currentstroke}{rgb}{0.000000,0.000000,0.000000}%
\pgfsetstrokecolor{currentstroke}%
\pgfsetdash{}{0pt}%
\pgfpathmoveto{\pgfqpoint{0.459750in}{0.411867in}}%
\pgfpathcurveto{\pgfqpoint{0.470800in}{0.411867in}}{\pgfqpoint{0.481399in}{0.416257in}}{\pgfqpoint{0.489213in}{0.424071in}}%
\pgfpathcurveto{\pgfqpoint{0.497026in}{0.431884in}}{\pgfqpoint{0.501417in}{0.442483in}}{\pgfqpoint{0.501417in}{0.453533in}}%
\pgfpathcurveto{\pgfqpoint{0.501417in}{0.464583in}}{\pgfqpoint{0.497026in}{0.475182in}}{\pgfqpoint{0.489213in}{0.482996in}}%
\pgfpathcurveto{\pgfqpoint{0.481399in}{0.490810in}}{\pgfqpoint{0.470800in}{0.495200in}}{\pgfqpoint{0.459750in}{0.495200in}}%
\pgfpathcurveto{\pgfqpoint{0.448700in}{0.495200in}}{\pgfqpoint{0.438101in}{0.490810in}}{\pgfqpoint{0.430287in}{0.482996in}}%
\pgfpathcurveto{\pgfqpoint{0.422474in}{0.475182in}}{\pgfqpoint{0.418083in}{0.464583in}}{\pgfqpoint{0.418083in}{0.453533in}}%
\pgfpathcurveto{\pgfqpoint{0.418083in}{0.442483in}}{\pgfqpoint{0.422474in}{0.431884in}}{\pgfqpoint{0.430287in}{0.424071in}}%
\pgfpathcurveto{\pgfqpoint{0.438101in}{0.416257in}}{\pgfqpoint{0.448700in}{0.411867in}}{\pgfqpoint{0.459750in}{0.411867in}}%
\pgfpathclose%
\pgfusepath{stroke,fill}%
\end{pgfscope}%
\begin{pgfscope}%
\pgfpathrectangle{\pgfqpoint{0.375000in}{0.330000in}}{\pgfqpoint{2.325000in}{2.310000in}}%
\pgfusepath{clip}%
\pgfsetbuttcap%
\pgfsetroundjoin%
\definecolor{currentfill}{rgb}{0.000000,0.000000,0.000000}%
\pgfsetfillcolor{currentfill}%
\pgfsetlinewidth{1.003750pt}%
\definecolor{currentstroke}{rgb}{0.000000,0.000000,0.000000}%
\pgfsetstrokecolor{currentstroke}%
\pgfsetdash{}{0pt}%
\pgfpathmoveto{\pgfqpoint{0.459750in}{0.411867in}}%
\pgfpathcurveto{\pgfqpoint{0.470800in}{0.411867in}}{\pgfqpoint{0.481399in}{0.416257in}}{\pgfqpoint{0.489213in}{0.424071in}}%
\pgfpathcurveto{\pgfqpoint{0.497026in}{0.431884in}}{\pgfqpoint{0.501417in}{0.442483in}}{\pgfqpoint{0.501417in}{0.453533in}}%
\pgfpathcurveto{\pgfqpoint{0.501417in}{0.464583in}}{\pgfqpoint{0.497026in}{0.475182in}}{\pgfqpoint{0.489213in}{0.482996in}}%
\pgfpathcurveto{\pgfqpoint{0.481399in}{0.490810in}}{\pgfqpoint{0.470800in}{0.495200in}}{\pgfqpoint{0.459750in}{0.495200in}}%
\pgfpathcurveto{\pgfqpoint{0.448700in}{0.495200in}}{\pgfqpoint{0.438101in}{0.490810in}}{\pgfqpoint{0.430287in}{0.482996in}}%
\pgfpathcurveto{\pgfqpoint{0.422474in}{0.475182in}}{\pgfqpoint{0.418083in}{0.464583in}}{\pgfqpoint{0.418083in}{0.453533in}}%
\pgfpathcurveto{\pgfqpoint{0.418083in}{0.442483in}}{\pgfqpoint{0.422474in}{0.431884in}}{\pgfqpoint{0.430287in}{0.424071in}}%
\pgfpathcurveto{\pgfqpoint{0.438101in}{0.416257in}}{\pgfqpoint{0.448700in}{0.411867in}}{\pgfqpoint{0.459750in}{0.411867in}}%
\pgfpathclose%
\pgfusepath{stroke,fill}%
\end{pgfscope}%
\begin{pgfscope}%
\pgfpathrectangle{\pgfqpoint{0.375000in}{0.330000in}}{\pgfqpoint{2.325000in}{2.310000in}}%
\pgfusepath{clip}%
\pgfsetbuttcap%
\pgfsetroundjoin%
\definecolor{currentfill}{rgb}{0.000000,0.000000,0.000000}%
\pgfsetfillcolor{currentfill}%
\pgfsetlinewidth{1.003750pt}%
\definecolor{currentstroke}{rgb}{0.000000,0.000000,0.000000}%
\pgfsetstrokecolor{currentstroke}%
\pgfsetdash{}{0pt}%
\pgfpathmoveto{\pgfqpoint{0.459750in}{0.424003in}}%
\pgfpathcurveto{\pgfqpoint{0.470800in}{0.424003in}}{\pgfqpoint{0.481399in}{0.428393in}}{\pgfqpoint{0.489213in}{0.436207in}}%
\pgfpathcurveto{\pgfqpoint{0.497026in}{0.444020in}}{\pgfqpoint{0.501417in}{0.454619in}}{\pgfqpoint{0.501417in}{0.465670in}}%
\pgfpathcurveto{\pgfqpoint{0.501417in}{0.476720in}}{\pgfqpoint{0.497026in}{0.487319in}}{\pgfqpoint{0.489213in}{0.495132in}}%
\pgfpathcurveto{\pgfqpoint{0.481399in}{0.502946in}}{\pgfqpoint{0.470800in}{0.507336in}}{\pgfqpoint{0.459750in}{0.507336in}}%
\pgfpathcurveto{\pgfqpoint{0.448700in}{0.507336in}}{\pgfqpoint{0.438101in}{0.502946in}}{\pgfqpoint{0.430287in}{0.495132in}}%
\pgfpathcurveto{\pgfqpoint{0.422474in}{0.487319in}}{\pgfqpoint{0.418083in}{0.476720in}}{\pgfqpoint{0.418083in}{0.465670in}}%
\pgfpathcurveto{\pgfqpoint{0.418083in}{0.454619in}}{\pgfqpoint{0.422474in}{0.444020in}}{\pgfqpoint{0.430287in}{0.436207in}}%
\pgfpathcurveto{\pgfqpoint{0.438101in}{0.428393in}}{\pgfqpoint{0.448700in}{0.424003in}}{\pgfqpoint{0.459750in}{0.424003in}}%
\pgfpathclose%
\pgfusepath{stroke,fill}%
\end{pgfscope}%
\begin{pgfscope}%
\pgfpathrectangle{\pgfqpoint{0.375000in}{0.330000in}}{\pgfqpoint{2.325000in}{2.310000in}}%
\pgfusepath{clip}%
\pgfsetbuttcap%
\pgfsetroundjoin%
\definecolor{currentfill}{rgb}{0.000000,0.000000,0.000000}%
\pgfsetfillcolor{currentfill}%
\pgfsetlinewidth{1.003750pt}%
\definecolor{currentstroke}{rgb}{0.000000,0.000000,0.000000}%
\pgfsetstrokecolor{currentstroke}%
\pgfsetdash{}{0pt}%
\pgfpathmoveto{\pgfqpoint{0.459750in}{0.424003in}}%
\pgfpathcurveto{\pgfqpoint{0.470800in}{0.424003in}}{\pgfqpoint{0.481399in}{0.428393in}}{\pgfqpoint{0.489213in}{0.436207in}}%
\pgfpathcurveto{\pgfqpoint{0.497026in}{0.444020in}}{\pgfqpoint{0.501417in}{0.454619in}}{\pgfqpoint{0.501417in}{0.465670in}}%
\pgfpathcurveto{\pgfqpoint{0.501417in}{0.476720in}}{\pgfqpoint{0.497026in}{0.487319in}}{\pgfqpoint{0.489213in}{0.495132in}}%
\pgfpathcurveto{\pgfqpoint{0.481399in}{0.502946in}}{\pgfqpoint{0.470800in}{0.507336in}}{\pgfqpoint{0.459750in}{0.507336in}}%
\pgfpathcurveto{\pgfqpoint{0.448700in}{0.507336in}}{\pgfqpoint{0.438101in}{0.502946in}}{\pgfqpoint{0.430287in}{0.495132in}}%
\pgfpathcurveto{\pgfqpoint{0.422474in}{0.487319in}}{\pgfqpoint{0.418083in}{0.476720in}}{\pgfqpoint{0.418083in}{0.465670in}}%
\pgfpathcurveto{\pgfqpoint{0.418083in}{0.454619in}}{\pgfqpoint{0.422474in}{0.444020in}}{\pgfqpoint{0.430287in}{0.436207in}}%
\pgfpathcurveto{\pgfqpoint{0.438101in}{0.428393in}}{\pgfqpoint{0.448700in}{0.424003in}}{\pgfqpoint{0.459750in}{0.424003in}}%
\pgfpathclose%
\pgfusepath{stroke,fill}%
\end{pgfscope}%
\begin{pgfscope}%
\pgfpathrectangle{\pgfqpoint{0.375000in}{0.330000in}}{\pgfqpoint{2.325000in}{2.310000in}}%
\pgfusepath{clip}%
\pgfsetbuttcap%
\pgfsetroundjoin%
\definecolor{currentfill}{rgb}{0.000000,0.000000,0.000000}%
\pgfsetfillcolor{currentfill}%
\pgfsetlinewidth{1.003750pt}%
\definecolor{currentstroke}{rgb}{0.000000,0.000000,0.000000}%
\pgfsetstrokecolor{currentstroke}%
\pgfsetdash{}{0pt}%
\pgfpathmoveto{\pgfqpoint{0.459750in}{0.424003in}}%
\pgfpathcurveto{\pgfqpoint{0.470800in}{0.424003in}}{\pgfqpoint{0.481399in}{0.428393in}}{\pgfqpoint{0.489213in}{0.436207in}}%
\pgfpathcurveto{\pgfqpoint{0.497026in}{0.444020in}}{\pgfqpoint{0.501417in}{0.454619in}}{\pgfqpoint{0.501417in}{0.465670in}}%
\pgfpathcurveto{\pgfqpoint{0.501417in}{0.476720in}}{\pgfqpoint{0.497026in}{0.487319in}}{\pgfqpoint{0.489213in}{0.495132in}}%
\pgfpathcurveto{\pgfqpoint{0.481399in}{0.502946in}}{\pgfqpoint{0.470800in}{0.507336in}}{\pgfqpoint{0.459750in}{0.507336in}}%
\pgfpathcurveto{\pgfqpoint{0.448700in}{0.507336in}}{\pgfqpoint{0.438101in}{0.502946in}}{\pgfqpoint{0.430287in}{0.495132in}}%
\pgfpathcurveto{\pgfqpoint{0.422474in}{0.487319in}}{\pgfqpoint{0.418083in}{0.476720in}}{\pgfqpoint{0.418083in}{0.465670in}}%
\pgfpathcurveto{\pgfqpoint{0.418083in}{0.454619in}}{\pgfqpoint{0.422474in}{0.444020in}}{\pgfqpoint{0.430287in}{0.436207in}}%
\pgfpathcurveto{\pgfqpoint{0.438101in}{0.428393in}}{\pgfqpoint{0.448700in}{0.424003in}}{\pgfqpoint{0.459750in}{0.424003in}}%
\pgfpathclose%
\pgfusepath{stroke,fill}%
\end{pgfscope}%
\begin{pgfscope}%
\pgfpathrectangle{\pgfqpoint{0.375000in}{0.330000in}}{\pgfqpoint{2.325000in}{2.310000in}}%
\pgfusepath{clip}%
\pgfsetbuttcap%
\pgfsetroundjoin%
\definecolor{currentfill}{rgb}{0.000000,0.000000,0.000000}%
\pgfsetfillcolor{currentfill}%
\pgfsetlinewidth{1.003750pt}%
\definecolor{currentstroke}{rgb}{0.000000,0.000000,0.000000}%
\pgfsetstrokecolor{currentstroke}%
\pgfsetdash{}{0pt}%
\pgfpathmoveto{\pgfqpoint{0.459750in}{0.411867in}}%
\pgfpathcurveto{\pgfqpoint{0.470800in}{0.411867in}}{\pgfqpoint{0.481399in}{0.416257in}}{\pgfqpoint{0.489213in}{0.424071in}}%
\pgfpathcurveto{\pgfqpoint{0.497026in}{0.431884in}}{\pgfqpoint{0.501417in}{0.442483in}}{\pgfqpoint{0.501417in}{0.453533in}}%
\pgfpathcurveto{\pgfqpoint{0.501417in}{0.464583in}}{\pgfqpoint{0.497026in}{0.475182in}}{\pgfqpoint{0.489213in}{0.482996in}}%
\pgfpathcurveto{\pgfqpoint{0.481399in}{0.490810in}}{\pgfqpoint{0.470800in}{0.495200in}}{\pgfqpoint{0.459750in}{0.495200in}}%
\pgfpathcurveto{\pgfqpoint{0.448700in}{0.495200in}}{\pgfqpoint{0.438101in}{0.490810in}}{\pgfqpoint{0.430287in}{0.482996in}}%
\pgfpathcurveto{\pgfqpoint{0.422474in}{0.475182in}}{\pgfqpoint{0.418083in}{0.464583in}}{\pgfqpoint{0.418083in}{0.453533in}}%
\pgfpathcurveto{\pgfqpoint{0.418083in}{0.442483in}}{\pgfqpoint{0.422474in}{0.431884in}}{\pgfqpoint{0.430287in}{0.424071in}}%
\pgfpathcurveto{\pgfqpoint{0.438101in}{0.416257in}}{\pgfqpoint{0.448700in}{0.411867in}}{\pgfqpoint{0.459750in}{0.411867in}}%
\pgfpathclose%
\pgfusepath{stroke,fill}%
\end{pgfscope}%
\begin{pgfscope}%
\pgfpathrectangle{\pgfqpoint{0.375000in}{0.330000in}}{\pgfqpoint{2.325000in}{2.310000in}}%
\pgfusepath{clip}%
\pgfsetbuttcap%
\pgfsetroundjoin%
\definecolor{currentfill}{rgb}{0.000000,0.000000,0.000000}%
\pgfsetfillcolor{currentfill}%
\pgfsetlinewidth{1.003750pt}%
\definecolor{currentstroke}{rgb}{0.000000,0.000000,0.000000}%
\pgfsetstrokecolor{currentstroke}%
\pgfsetdash{}{0pt}%
\pgfpathmoveto{\pgfqpoint{0.459750in}{0.411867in}}%
\pgfpathcurveto{\pgfqpoint{0.470800in}{0.411867in}}{\pgfqpoint{0.481399in}{0.416257in}}{\pgfqpoint{0.489213in}{0.424071in}}%
\pgfpathcurveto{\pgfqpoint{0.497026in}{0.431884in}}{\pgfqpoint{0.501417in}{0.442483in}}{\pgfqpoint{0.501417in}{0.453533in}}%
\pgfpathcurveto{\pgfqpoint{0.501417in}{0.464583in}}{\pgfqpoint{0.497026in}{0.475182in}}{\pgfqpoint{0.489213in}{0.482996in}}%
\pgfpathcurveto{\pgfqpoint{0.481399in}{0.490810in}}{\pgfqpoint{0.470800in}{0.495200in}}{\pgfqpoint{0.459750in}{0.495200in}}%
\pgfpathcurveto{\pgfqpoint{0.448700in}{0.495200in}}{\pgfqpoint{0.438101in}{0.490810in}}{\pgfqpoint{0.430287in}{0.482996in}}%
\pgfpathcurveto{\pgfqpoint{0.422474in}{0.475182in}}{\pgfqpoint{0.418083in}{0.464583in}}{\pgfqpoint{0.418083in}{0.453533in}}%
\pgfpathcurveto{\pgfqpoint{0.418083in}{0.442483in}}{\pgfqpoint{0.422474in}{0.431884in}}{\pgfqpoint{0.430287in}{0.424071in}}%
\pgfpathcurveto{\pgfqpoint{0.438101in}{0.416257in}}{\pgfqpoint{0.448700in}{0.411867in}}{\pgfqpoint{0.459750in}{0.411867in}}%
\pgfpathclose%
\pgfusepath{stroke,fill}%
\end{pgfscope}%
\begin{pgfscope}%
\pgfpathrectangle{\pgfqpoint{0.375000in}{0.330000in}}{\pgfqpoint{2.325000in}{2.310000in}}%
\pgfusepath{clip}%
\pgfsetbuttcap%
\pgfsetroundjoin%
\definecolor{currentfill}{rgb}{0.000000,0.000000,0.000000}%
\pgfsetfillcolor{currentfill}%
\pgfsetlinewidth{1.003750pt}%
\definecolor{currentstroke}{rgb}{0.000000,0.000000,0.000000}%
\pgfsetstrokecolor{currentstroke}%
\pgfsetdash{}{0pt}%
\pgfpathmoveto{\pgfqpoint{0.459750in}{0.417935in}}%
\pgfpathcurveto{\pgfqpoint{0.470800in}{0.417935in}}{\pgfqpoint{0.481399in}{0.422325in}}{\pgfqpoint{0.489213in}{0.430139in}}%
\pgfpathcurveto{\pgfqpoint{0.497026in}{0.437952in}}{\pgfqpoint{0.501417in}{0.448551in}}{\pgfqpoint{0.501417in}{0.459601in}}%
\pgfpathcurveto{\pgfqpoint{0.501417in}{0.470652in}}{\pgfqpoint{0.497026in}{0.481251in}}{\pgfqpoint{0.489213in}{0.489064in}}%
\pgfpathcurveto{\pgfqpoint{0.481399in}{0.496878in}}{\pgfqpoint{0.470800in}{0.501268in}}{\pgfqpoint{0.459750in}{0.501268in}}%
\pgfpathcurveto{\pgfqpoint{0.448700in}{0.501268in}}{\pgfqpoint{0.438101in}{0.496878in}}{\pgfqpoint{0.430287in}{0.489064in}}%
\pgfpathcurveto{\pgfqpoint{0.422474in}{0.481251in}}{\pgfqpoint{0.418083in}{0.470652in}}{\pgfqpoint{0.418083in}{0.459601in}}%
\pgfpathcurveto{\pgfqpoint{0.418083in}{0.448551in}}{\pgfqpoint{0.422474in}{0.437952in}}{\pgfqpoint{0.430287in}{0.430139in}}%
\pgfpathcurveto{\pgfqpoint{0.438101in}{0.422325in}}{\pgfqpoint{0.448700in}{0.417935in}}{\pgfqpoint{0.459750in}{0.417935in}}%
\pgfpathclose%
\pgfusepath{stroke,fill}%
\end{pgfscope}%
\begin{pgfscope}%
\pgfpathrectangle{\pgfqpoint{0.375000in}{0.330000in}}{\pgfqpoint{2.325000in}{2.310000in}}%
\pgfusepath{clip}%
\pgfsetbuttcap%
\pgfsetroundjoin%
\definecolor{currentfill}{rgb}{0.000000,0.000000,0.000000}%
\pgfsetfillcolor{currentfill}%
\pgfsetlinewidth{1.003750pt}%
\definecolor{currentstroke}{rgb}{0.000000,0.000000,0.000000}%
\pgfsetstrokecolor{currentstroke}%
\pgfsetdash{}{0pt}%
\pgfpathmoveto{\pgfqpoint{0.459750in}{0.411867in}}%
\pgfpathcurveto{\pgfqpoint{0.470800in}{0.411867in}}{\pgfqpoint{0.481399in}{0.416257in}}{\pgfqpoint{0.489213in}{0.424071in}}%
\pgfpathcurveto{\pgfqpoint{0.497026in}{0.431884in}}{\pgfqpoint{0.501417in}{0.442483in}}{\pgfqpoint{0.501417in}{0.453533in}}%
\pgfpathcurveto{\pgfqpoint{0.501417in}{0.464583in}}{\pgfqpoint{0.497026in}{0.475182in}}{\pgfqpoint{0.489213in}{0.482996in}}%
\pgfpathcurveto{\pgfqpoint{0.481399in}{0.490810in}}{\pgfqpoint{0.470800in}{0.495200in}}{\pgfqpoint{0.459750in}{0.495200in}}%
\pgfpathcurveto{\pgfqpoint{0.448700in}{0.495200in}}{\pgfqpoint{0.438101in}{0.490810in}}{\pgfqpoint{0.430287in}{0.482996in}}%
\pgfpathcurveto{\pgfqpoint{0.422474in}{0.475182in}}{\pgfqpoint{0.418083in}{0.464583in}}{\pgfqpoint{0.418083in}{0.453533in}}%
\pgfpathcurveto{\pgfqpoint{0.418083in}{0.442483in}}{\pgfqpoint{0.422474in}{0.431884in}}{\pgfqpoint{0.430287in}{0.424071in}}%
\pgfpathcurveto{\pgfqpoint{0.438101in}{0.416257in}}{\pgfqpoint{0.448700in}{0.411867in}}{\pgfqpoint{0.459750in}{0.411867in}}%
\pgfpathclose%
\pgfusepath{stroke,fill}%
\end{pgfscope}%
\begin{pgfscope}%
\pgfpathrectangle{\pgfqpoint{0.375000in}{0.330000in}}{\pgfqpoint{2.325000in}{2.310000in}}%
\pgfusepath{clip}%
\pgfsetbuttcap%
\pgfsetroundjoin%
\definecolor{currentfill}{rgb}{0.000000,0.000000,0.000000}%
\pgfsetfillcolor{currentfill}%
\pgfsetlinewidth{1.003750pt}%
\definecolor{currentstroke}{rgb}{0.000000,0.000000,0.000000}%
\pgfsetstrokecolor{currentstroke}%
\pgfsetdash{}{0pt}%
\pgfpathmoveto{\pgfqpoint{0.459750in}{0.411867in}}%
\pgfpathcurveto{\pgfqpoint{0.470800in}{0.411867in}}{\pgfqpoint{0.481399in}{0.416257in}}{\pgfqpoint{0.489213in}{0.424071in}}%
\pgfpathcurveto{\pgfqpoint{0.497026in}{0.431884in}}{\pgfqpoint{0.501417in}{0.442483in}}{\pgfqpoint{0.501417in}{0.453533in}}%
\pgfpathcurveto{\pgfqpoint{0.501417in}{0.464583in}}{\pgfqpoint{0.497026in}{0.475182in}}{\pgfqpoint{0.489213in}{0.482996in}}%
\pgfpathcurveto{\pgfqpoint{0.481399in}{0.490810in}}{\pgfqpoint{0.470800in}{0.495200in}}{\pgfqpoint{0.459750in}{0.495200in}}%
\pgfpathcurveto{\pgfqpoint{0.448700in}{0.495200in}}{\pgfqpoint{0.438101in}{0.490810in}}{\pgfqpoint{0.430287in}{0.482996in}}%
\pgfpathcurveto{\pgfqpoint{0.422474in}{0.475182in}}{\pgfqpoint{0.418083in}{0.464583in}}{\pgfqpoint{0.418083in}{0.453533in}}%
\pgfpathcurveto{\pgfqpoint{0.418083in}{0.442483in}}{\pgfqpoint{0.422474in}{0.431884in}}{\pgfqpoint{0.430287in}{0.424071in}}%
\pgfpathcurveto{\pgfqpoint{0.438101in}{0.416257in}}{\pgfqpoint{0.448700in}{0.411867in}}{\pgfqpoint{0.459750in}{0.411867in}}%
\pgfpathclose%
\pgfusepath{stroke,fill}%
\end{pgfscope}%
\begin{pgfscope}%
\pgfpathrectangle{\pgfqpoint{0.375000in}{0.330000in}}{\pgfqpoint{2.325000in}{2.310000in}}%
\pgfusepath{clip}%
\pgfsetbuttcap%
\pgfsetroundjoin%
\definecolor{currentfill}{rgb}{0.000000,0.000000,0.000000}%
\pgfsetfillcolor{currentfill}%
\pgfsetlinewidth{1.003750pt}%
\definecolor{currentstroke}{rgb}{0.000000,0.000000,0.000000}%
\pgfsetstrokecolor{currentstroke}%
\pgfsetdash{}{0pt}%
\pgfpathmoveto{\pgfqpoint{0.459750in}{0.417935in}}%
\pgfpathcurveto{\pgfqpoint{0.470800in}{0.417935in}}{\pgfqpoint{0.481399in}{0.422325in}}{\pgfqpoint{0.489213in}{0.430139in}}%
\pgfpathcurveto{\pgfqpoint{0.497026in}{0.437952in}}{\pgfqpoint{0.501417in}{0.448551in}}{\pgfqpoint{0.501417in}{0.459601in}}%
\pgfpathcurveto{\pgfqpoint{0.501417in}{0.470652in}}{\pgfqpoint{0.497026in}{0.481251in}}{\pgfqpoint{0.489213in}{0.489064in}}%
\pgfpathcurveto{\pgfqpoint{0.481399in}{0.496878in}}{\pgfqpoint{0.470800in}{0.501268in}}{\pgfqpoint{0.459750in}{0.501268in}}%
\pgfpathcurveto{\pgfqpoint{0.448700in}{0.501268in}}{\pgfqpoint{0.438101in}{0.496878in}}{\pgfqpoint{0.430287in}{0.489064in}}%
\pgfpathcurveto{\pgfqpoint{0.422474in}{0.481251in}}{\pgfqpoint{0.418083in}{0.470652in}}{\pgfqpoint{0.418083in}{0.459601in}}%
\pgfpathcurveto{\pgfqpoint{0.418083in}{0.448551in}}{\pgfqpoint{0.422474in}{0.437952in}}{\pgfqpoint{0.430287in}{0.430139in}}%
\pgfpathcurveto{\pgfqpoint{0.438101in}{0.422325in}}{\pgfqpoint{0.448700in}{0.417935in}}{\pgfqpoint{0.459750in}{0.417935in}}%
\pgfpathclose%
\pgfusepath{stroke,fill}%
\end{pgfscope}%
\begin{pgfscope}%
\pgfpathrectangle{\pgfqpoint{0.375000in}{0.330000in}}{\pgfqpoint{2.325000in}{2.310000in}}%
\pgfusepath{clip}%
\pgfsetbuttcap%
\pgfsetroundjoin%
\definecolor{currentfill}{rgb}{0.000000,0.000000,0.000000}%
\pgfsetfillcolor{currentfill}%
\pgfsetlinewidth{1.003750pt}%
\definecolor{currentstroke}{rgb}{0.000000,0.000000,0.000000}%
\pgfsetstrokecolor{currentstroke}%
\pgfsetdash{}{0pt}%
\pgfpathmoveto{\pgfqpoint{0.459750in}{0.424003in}}%
\pgfpathcurveto{\pgfqpoint{0.470800in}{0.424003in}}{\pgfqpoint{0.481399in}{0.428393in}}{\pgfqpoint{0.489213in}{0.436207in}}%
\pgfpathcurveto{\pgfqpoint{0.497026in}{0.444020in}}{\pgfqpoint{0.501417in}{0.454619in}}{\pgfqpoint{0.501417in}{0.465670in}}%
\pgfpathcurveto{\pgfqpoint{0.501417in}{0.476720in}}{\pgfqpoint{0.497026in}{0.487319in}}{\pgfqpoint{0.489213in}{0.495132in}}%
\pgfpathcurveto{\pgfqpoint{0.481399in}{0.502946in}}{\pgfqpoint{0.470800in}{0.507336in}}{\pgfqpoint{0.459750in}{0.507336in}}%
\pgfpathcurveto{\pgfqpoint{0.448700in}{0.507336in}}{\pgfqpoint{0.438101in}{0.502946in}}{\pgfqpoint{0.430287in}{0.495132in}}%
\pgfpathcurveto{\pgfqpoint{0.422474in}{0.487319in}}{\pgfqpoint{0.418083in}{0.476720in}}{\pgfqpoint{0.418083in}{0.465670in}}%
\pgfpathcurveto{\pgfqpoint{0.418083in}{0.454619in}}{\pgfqpoint{0.422474in}{0.444020in}}{\pgfqpoint{0.430287in}{0.436207in}}%
\pgfpathcurveto{\pgfqpoint{0.438101in}{0.428393in}}{\pgfqpoint{0.448700in}{0.424003in}}{\pgfqpoint{0.459750in}{0.424003in}}%
\pgfpathclose%
\pgfusepath{stroke,fill}%
\end{pgfscope}%
\begin{pgfscope}%
\pgfpathrectangle{\pgfqpoint{0.375000in}{0.330000in}}{\pgfqpoint{2.325000in}{2.310000in}}%
\pgfusepath{clip}%
\pgfsetbuttcap%
\pgfsetroundjoin%
\definecolor{currentfill}{rgb}{0.000000,0.000000,0.000000}%
\pgfsetfillcolor{currentfill}%
\pgfsetlinewidth{1.003750pt}%
\definecolor{currentstroke}{rgb}{0.000000,0.000000,0.000000}%
\pgfsetstrokecolor{currentstroke}%
\pgfsetdash{}{0pt}%
\pgfpathmoveto{\pgfqpoint{0.459750in}{0.405799in}}%
\pgfpathcurveto{\pgfqpoint{0.470800in}{0.405799in}}{\pgfqpoint{0.481399in}{0.410189in}}{\pgfqpoint{0.489213in}{0.418002in}}%
\pgfpathcurveto{\pgfqpoint{0.497026in}{0.425816in}}{\pgfqpoint{0.501417in}{0.436415in}}{\pgfqpoint{0.501417in}{0.447465in}}%
\pgfpathcurveto{\pgfqpoint{0.501417in}{0.458515in}}{\pgfqpoint{0.497026in}{0.469114in}}{\pgfqpoint{0.489213in}{0.476928in}}%
\pgfpathcurveto{\pgfqpoint{0.481399in}{0.484742in}}{\pgfqpoint{0.470800in}{0.489132in}}{\pgfqpoint{0.459750in}{0.489132in}}%
\pgfpathcurveto{\pgfqpoint{0.448700in}{0.489132in}}{\pgfqpoint{0.438101in}{0.484742in}}{\pgfqpoint{0.430287in}{0.476928in}}%
\pgfpathcurveto{\pgfqpoint{0.422474in}{0.469114in}}{\pgfqpoint{0.418083in}{0.458515in}}{\pgfqpoint{0.418083in}{0.447465in}}%
\pgfpathcurveto{\pgfqpoint{0.418083in}{0.436415in}}{\pgfqpoint{0.422474in}{0.425816in}}{\pgfqpoint{0.430287in}{0.418002in}}%
\pgfpathcurveto{\pgfqpoint{0.438101in}{0.410189in}}{\pgfqpoint{0.448700in}{0.405799in}}{\pgfqpoint{0.459750in}{0.405799in}}%
\pgfpathclose%
\pgfusepath{stroke,fill}%
\end{pgfscope}%
\begin{pgfscope}%
\pgfpathrectangle{\pgfqpoint{0.375000in}{0.330000in}}{\pgfqpoint{2.325000in}{2.310000in}}%
\pgfusepath{clip}%
\pgfsetbuttcap%
\pgfsetroundjoin%
\definecolor{currentfill}{rgb}{0.000000,0.000000,0.000000}%
\pgfsetfillcolor{currentfill}%
\pgfsetlinewidth{1.003750pt}%
\definecolor{currentstroke}{rgb}{0.000000,0.000000,0.000000}%
\pgfsetstrokecolor{currentstroke}%
\pgfsetdash{}{0pt}%
\pgfpathmoveto{\pgfqpoint{0.459750in}{0.424003in}}%
\pgfpathcurveto{\pgfqpoint{0.470800in}{0.424003in}}{\pgfqpoint{0.481399in}{0.428393in}}{\pgfqpoint{0.489213in}{0.436207in}}%
\pgfpathcurveto{\pgfqpoint{0.497026in}{0.444020in}}{\pgfqpoint{0.501417in}{0.454619in}}{\pgfqpoint{0.501417in}{0.465670in}}%
\pgfpathcurveto{\pgfqpoint{0.501417in}{0.476720in}}{\pgfqpoint{0.497026in}{0.487319in}}{\pgfqpoint{0.489213in}{0.495132in}}%
\pgfpathcurveto{\pgfqpoint{0.481399in}{0.502946in}}{\pgfqpoint{0.470800in}{0.507336in}}{\pgfqpoint{0.459750in}{0.507336in}}%
\pgfpathcurveto{\pgfqpoint{0.448700in}{0.507336in}}{\pgfqpoint{0.438101in}{0.502946in}}{\pgfqpoint{0.430287in}{0.495132in}}%
\pgfpathcurveto{\pgfqpoint{0.422474in}{0.487319in}}{\pgfqpoint{0.418083in}{0.476720in}}{\pgfqpoint{0.418083in}{0.465670in}}%
\pgfpathcurveto{\pgfqpoint{0.418083in}{0.454619in}}{\pgfqpoint{0.422474in}{0.444020in}}{\pgfqpoint{0.430287in}{0.436207in}}%
\pgfpathcurveto{\pgfqpoint{0.438101in}{0.428393in}}{\pgfqpoint{0.448700in}{0.424003in}}{\pgfqpoint{0.459750in}{0.424003in}}%
\pgfpathclose%
\pgfusepath{stroke,fill}%
\end{pgfscope}%
\begin{pgfscope}%
\pgfpathrectangle{\pgfqpoint{0.375000in}{0.330000in}}{\pgfqpoint{2.325000in}{2.310000in}}%
\pgfusepath{clip}%
\pgfsetbuttcap%
\pgfsetroundjoin%
\definecolor{currentfill}{rgb}{0.000000,0.000000,0.000000}%
\pgfsetfillcolor{currentfill}%
\pgfsetlinewidth{1.003750pt}%
\definecolor{currentstroke}{rgb}{0.000000,0.000000,0.000000}%
\pgfsetstrokecolor{currentstroke}%
\pgfsetdash{}{0pt}%
\pgfpathmoveto{\pgfqpoint{0.459750in}{0.411867in}}%
\pgfpathcurveto{\pgfqpoint{0.470800in}{0.411867in}}{\pgfqpoint{0.481399in}{0.416257in}}{\pgfqpoint{0.489213in}{0.424071in}}%
\pgfpathcurveto{\pgfqpoint{0.497026in}{0.431884in}}{\pgfqpoint{0.501417in}{0.442483in}}{\pgfqpoint{0.501417in}{0.453533in}}%
\pgfpathcurveto{\pgfqpoint{0.501417in}{0.464583in}}{\pgfqpoint{0.497026in}{0.475182in}}{\pgfqpoint{0.489213in}{0.482996in}}%
\pgfpathcurveto{\pgfqpoint{0.481399in}{0.490810in}}{\pgfqpoint{0.470800in}{0.495200in}}{\pgfqpoint{0.459750in}{0.495200in}}%
\pgfpathcurveto{\pgfqpoint{0.448700in}{0.495200in}}{\pgfqpoint{0.438101in}{0.490810in}}{\pgfqpoint{0.430287in}{0.482996in}}%
\pgfpathcurveto{\pgfqpoint{0.422474in}{0.475182in}}{\pgfqpoint{0.418083in}{0.464583in}}{\pgfqpoint{0.418083in}{0.453533in}}%
\pgfpathcurveto{\pgfqpoint{0.418083in}{0.442483in}}{\pgfqpoint{0.422474in}{0.431884in}}{\pgfqpoint{0.430287in}{0.424071in}}%
\pgfpathcurveto{\pgfqpoint{0.438101in}{0.416257in}}{\pgfqpoint{0.448700in}{0.411867in}}{\pgfqpoint{0.459750in}{0.411867in}}%
\pgfpathclose%
\pgfusepath{stroke,fill}%
\end{pgfscope}%
\begin{pgfscope}%
\pgfpathrectangle{\pgfqpoint{0.375000in}{0.330000in}}{\pgfqpoint{2.325000in}{2.310000in}}%
\pgfusepath{clip}%
\pgfsetbuttcap%
\pgfsetroundjoin%
\definecolor{currentfill}{rgb}{0.000000,0.000000,0.000000}%
\pgfsetfillcolor{currentfill}%
\pgfsetlinewidth{1.003750pt}%
\definecolor{currentstroke}{rgb}{0.000000,0.000000,0.000000}%
\pgfsetstrokecolor{currentstroke}%
\pgfsetdash{}{0pt}%
\pgfpathmoveto{\pgfqpoint{0.459750in}{0.393662in}}%
\pgfpathcurveto{\pgfqpoint{0.470800in}{0.393662in}}{\pgfqpoint{0.481399in}{0.398053in}}{\pgfqpoint{0.489213in}{0.405866in}}%
\pgfpathcurveto{\pgfqpoint{0.497026in}{0.413680in}}{\pgfqpoint{0.501417in}{0.424279in}}{\pgfqpoint{0.501417in}{0.435329in}}%
\pgfpathcurveto{\pgfqpoint{0.501417in}{0.446379in}}{\pgfqpoint{0.497026in}{0.456978in}}{\pgfqpoint{0.489213in}{0.464792in}}%
\pgfpathcurveto{\pgfqpoint{0.481399in}{0.472605in}}{\pgfqpoint{0.470800in}{0.476996in}}{\pgfqpoint{0.459750in}{0.476996in}}%
\pgfpathcurveto{\pgfqpoint{0.448700in}{0.476996in}}{\pgfqpoint{0.438101in}{0.472605in}}{\pgfqpoint{0.430287in}{0.464792in}}%
\pgfpathcurveto{\pgfqpoint{0.422474in}{0.456978in}}{\pgfqpoint{0.418083in}{0.446379in}}{\pgfqpoint{0.418083in}{0.435329in}}%
\pgfpathcurveto{\pgfqpoint{0.418083in}{0.424279in}}{\pgfqpoint{0.422474in}{0.413680in}}{\pgfqpoint{0.430287in}{0.405866in}}%
\pgfpathcurveto{\pgfqpoint{0.438101in}{0.398053in}}{\pgfqpoint{0.448700in}{0.393662in}}{\pgfqpoint{0.459750in}{0.393662in}}%
\pgfpathclose%
\pgfusepath{stroke,fill}%
\end{pgfscope}%
\begin{pgfscope}%
\pgfpathrectangle{\pgfqpoint{0.375000in}{0.330000in}}{\pgfqpoint{2.325000in}{2.310000in}}%
\pgfusepath{clip}%
\pgfsetbuttcap%
\pgfsetroundjoin%
\definecolor{currentfill}{rgb}{0.000000,0.000000,0.000000}%
\pgfsetfillcolor{currentfill}%
\pgfsetlinewidth{1.003750pt}%
\definecolor{currentstroke}{rgb}{0.000000,0.000000,0.000000}%
\pgfsetstrokecolor{currentstroke}%
\pgfsetdash{}{0pt}%
\pgfpathmoveto{\pgfqpoint{0.459750in}{0.430071in}}%
\pgfpathcurveto{\pgfqpoint{0.470800in}{0.430071in}}{\pgfqpoint{0.481399in}{0.434461in}}{\pgfqpoint{0.489213in}{0.442275in}}%
\pgfpathcurveto{\pgfqpoint{0.497026in}{0.450088in}}{\pgfqpoint{0.501417in}{0.460688in}}{\pgfqpoint{0.501417in}{0.471738in}}%
\pgfpathcurveto{\pgfqpoint{0.501417in}{0.482788in}}{\pgfqpoint{0.497026in}{0.493387in}}{\pgfqpoint{0.489213in}{0.501200in}}%
\pgfpathcurveto{\pgfqpoint{0.481399in}{0.509014in}}{\pgfqpoint{0.470800in}{0.513404in}}{\pgfqpoint{0.459750in}{0.513404in}}%
\pgfpathcurveto{\pgfqpoint{0.448700in}{0.513404in}}{\pgfqpoint{0.438101in}{0.509014in}}{\pgfqpoint{0.430287in}{0.501200in}}%
\pgfpathcurveto{\pgfqpoint{0.422474in}{0.493387in}}{\pgfqpoint{0.418083in}{0.482788in}}{\pgfqpoint{0.418083in}{0.471738in}}%
\pgfpathcurveto{\pgfqpoint{0.418083in}{0.460688in}}{\pgfqpoint{0.422474in}{0.450088in}}{\pgfqpoint{0.430287in}{0.442275in}}%
\pgfpathcurveto{\pgfqpoint{0.438101in}{0.434461in}}{\pgfqpoint{0.448700in}{0.430071in}}{\pgfqpoint{0.459750in}{0.430071in}}%
\pgfpathclose%
\pgfusepath{stroke,fill}%
\end{pgfscope}%
\begin{pgfscope}%
\pgfpathrectangle{\pgfqpoint{0.375000in}{0.330000in}}{\pgfqpoint{2.325000in}{2.310000in}}%
\pgfusepath{clip}%
\pgfsetbuttcap%
\pgfsetroundjoin%
\definecolor{currentfill}{rgb}{0.000000,0.000000,0.000000}%
\pgfsetfillcolor{currentfill}%
\pgfsetlinewidth{1.003750pt}%
\definecolor{currentstroke}{rgb}{0.000000,0.000000,0.000000}%
\pgfsetstrokecolor{currentstroke}%
\pgfsetdash{}{0pt}%
\pgfpathmoveto{\pgfqpoint{0.459750in}{0.424003in}}%
\pgfpathcurveto{\pgfqpoint{0.470800in}{0.424003in}}{\pgfqpoint{0.481399in}{0.428393in}}{\pgfqpoint{0.489213in}{0.436207in}}%
\pgfpathcurveto{\pgfqpoint{0.497026in}{0.444020in}}{\pgfqpoint{0.501417in}{0.454619in}}{\pgfqpoint{0.501417in}{0.465670in}}%
\pgfpathcurveto{\pgfqpoint{0.501417in}{0.476720in}}{\pgfqpoint{0.497026in}{0.487319in}}{\pgfqpoint{0.489213in}{0.495132in}}%
\pgfpathcurveto{\pgfqpoint{0.481399in}{0.502946in}}{\pgfqpoint{0.470800in}{0.507336in}}{\pgfqpoint{0.459750in}{0.507336in}}%
\pgfpathcurveto{\pgfqpoint{0.448700in}{0.507336in}}{\pgfqpoint{0.438101in}{0.502946in}}{\pgfqpoint{0.430287in}{0.495132in}}%
\pgfpathcurveto{\pgfqpoint{0.422474in}{0.487319in}}{\pgfqpoint{0.418083in}{0.476720in}}{\pgfqpoint{0.418083in}{0.465670in}}%
\pgfpathcurveto{\pgfqpoint{0.418083in}{0.454619in}}{\pgfqpoint{0.422474in}{0.444020in}}{\pgfqpoint{0.430287in}{0.436207in}}%
\pgfpathcurveto{\pgfqpoint{0.438101in}{0.428393in}}{\pgfqpoint{0.448700in}{0.424003in}}{\pgfqpoint{0.459750in}{0.424003in}}%
\pgfpathclose%
\pgfusepath{stroke,fill}%
\end{pgfscope}%
\begin{pgfscope}%
\pgfpathrectangle{\pgfqpoint{0.375000in}{0.330000in}}{\pgfqpoint{2.325000in}{2.310000in}}%
\pgfusepath{clip}%
\pgfsetbuttcap%
\pgfsetroundjoin%
\definecolor{currentfill}{rgb}{0.000000,0.000000,0.000000}%
\pgfsetfillcolor{currentfill}%
\pgfsetlinewidth{1.003750pt}%
\definecolor{currentstroke}{rgb}{0.000000,0.000000,0.000000}%
\pgfsetstrokecolor{currentstroke}%
\pgfsetdash{}{0pt}%
\pgfpathmoveto{\pgfqpoint{0.459750in}{0.417935in}}%
\pgfpathcurveto{\pgfqpoint{0.470800in}{0.417935in}}{\pgfqpoint{0.481399in}{0.422325in}}{\pgfqpoint{0.489213in}{0.430139in}}%
\pgfpathcurveto{\pgfqpoint{0.497026in}{0.437952in}}{\pgfqpoint{0.501417in}{0.448551in}}{\pgfqpoint{0.501417in}{0.459601in}}%
\pgfpathcurveto{\pgfqpoint{0.501417in}{0.470652in}}{\pgfqpoint{0.497026in}{0.481251in}}{\pgfqpoint{0.489213in}{0.489064in}}%
\pgfpathcurveto{\pgfqpoint{0.481399in}{0.496878in}}{\pgfqpoint{0.470800in}{0.501268in}}{\pgfqpoint{0.459750in}{0.501268in}}%
\pgfpathcurveto{\pgfqpoint{0.448700in}{0.501268in}}{\pgfqpoint{0.438101in}{0.496878in}}{\pgfqpoint{0.430287in}{0.489064in}}%
\pgfpathcurveto{\pgfqpoint{0.422474in}{0.481251in}}{\pgfqpoint{0.418083in}{0.470652in}}{\pgfqpoint{0.418083in}{0.459601in}}%
\pgfpathcurveto{\pgfqpoint{0.418083in}{0.448551in}}{\pgfqpoint{0.422474in}{0.437952in}}{\pgfqpoint{0.430287in}{0.430139in}}%
\pgfpathcurveto{\pgfqpoint{0.438101in}{0.422325in}}{\pgfqpoint{0.448700in}{0.417935in}}{\pgfqpoint{0.459750in}{0.417935in}}%
\pgfpathclose%
\pgfusepath{stroke,fill}%
\end{pgfscope}%
\begin{pgfscope}%
\pgfpathrectangle{\pgfqpoint{0.375000in}{0.330000in}}{\pgfqpoint{2.325000in}{2.310000in}}%
\pgfusepath{clip}%
\pgfsetbuttcap%
\pgfsetroundjoin%
\definecolor{currentfill}{rgb}{0.000000,0.000000,0.000000}%
\pgfsetfillcolor{currentfill}%
\pgfsetlinewidth{1.003750pt}%
\definecolor{currentstroke}{rgb}{0.000000,0.000000,0.000000}%
\pgfsetstrokecolor{currentstroke}%
\pgfsetdash{}{0pt}%
\pgfpathmoveto{\pgfqpoint{0.459750in}{0.411867in}}%
\pgfpathcurveto{\pgfqpoint{0.470800in}{0.411867in}}{\pgfqpoint{0.481399in}{0.416257in}}{\pgfqpoint{0.489213in}{0.424071in}}%
\pgfpathcurveto{\pgfqpoint{0.497026in}{0.431884in}}{\pgfqpoint{0.501417in}{0.442483in}}{\pgfqpoint{0.501417in}{0.453533in}}%
\pgfpathcurveto{\pgfqpoint{0.501417in}{0.464583in}}{\pgfqpoint{0.497026in}{0.475182in}}{\pgfqpoint{0.489213in}{0.482996in}}%
\pgfpathcurveto{\pgfqpoint{0.481399in}{0.490810in}}{\pgfqpoint{0.470800in}{0.495200in}}{\pgfqpoint{0.459750in}{0.495200in}}%
\pgfpathcurveto{\pgfqpoint{0.448700in}{0.495200in}}{\pgfqpoint{0.438101in}{0.490810in}}{\pgfqpoint{0.430287in}{0.482996in}}%
\pgfpathcurveto{\pgfqpoint{0.422474in}{0.475182in}}{\pgfqpoint{0.418083in}{0.464583in}}{\pgfqpoint{0.418083in}{0.453533in}}%
\pgfpathcurveto{\pgfqpoint{0.418083in}{0.442483in}}{\pgfqpoint{0.422474in}{0.431884in}}{\pgfqpoint{0.430287in}{0.424071in}}%
\pgfpathcurveto{\pgfqpoint{0.438101in}{0.416257in}}{\pgfqpoint{0.448700in}{0.411867in}}{\pgfqpoint{0.459750in}{0.411867in}}%
\pgfpathclose%
\pgfusepath{stroke,fill}%
\end{pgfscope}%
\begin{pgfscope}%
\pgfpathrectangle{\pgfqpoint{0.375000in}{0.330000in}}{\pgfqpoint{2.325000in}{2.310000in}}%
\pgfusepath{clip}%
\pgfsetbuttcap%
\pgfsetroundjoin%
\definecolor{currentfill}{rgb}{0.000000,0.000000,0.000000}%
\pgfsetfillcolor{currentfill}%
\pgfsetlinewidth{1.003750pt}%
\definecolor{currentstroke}{rgb}{0.000000,0.000000,0.000000}%
\pgfsetstrokecolor{currentstroke}%
\pgfsetdash{}{0pt}%
\pgfpathmoveto{\pgfqpoint{0.459750in}{0.417935in}}%
\pgfpathcurveto{\pgfqpoint{0.470800in}{0.417935in}}{\pgfqpoint{0.481399in}{0.422325in}}{\pgfqpoint{0.489213in}{0.430139in}}%
\pgfpathcurveto{\pgfqpoint{0.497026in}{0.437952in}}{\pgfqpoint{0.501417in}{0.448551in}}{\pgfqpoint{0.501417in}{0.459601in}}%
\pgfpathcurveto{\pgfqpoint{0.501417in}{0.470652in}}{\pgfqpoint{0.497026in}{0.481251in}}{\pgfqpoint{0.489213in}{0.489064in}}%
\pgfpathcurveto{\pgfqpoint{0.481399in}{0.496878in}}{\pgfqpoint{0.470800in}{0.501268in}}{\pgfqpoint{0.459750in}{0.501268in}}%
\pgfpathcurveto{\pgfqpoint{0.448700in}{0.501268in}}{\pgfqpoint{0.438101in}{0.496878in}}{\pgfqpoint{0.430287in}{0.489064in}}%
\pgfpathcurveto{\pgfqpoint{0.422474in}{0.481251in}}{\pgfqpoint{0.418083in}{0.470652in}}{\pgfqpoint{0.418083in}{0.459601in}}%
\pgfpathcurveto{\pgfqpoint{0.418083in}{0.448551in}}{\pgfqpoint{0.422474in}{0.437952in}}{\pgfqpoint{0.430287in}{0.430139in}}%
\pgfpathcurveto{\pgfqpoint{0.438101in}{0.422325in}}{\pgfqpoint{0.448700in}{0.417935in}}{\pgfqpoint{0.459750in}{0.417935in}}%
\pgfpathclose%
\pgfusepath{stroke,fill}%
\end{pgfscope}%
\begin{pgfscope}%
\pgfpathrectangle{\pgfqpoint{0.375000in}{0.330000in}}{\pgfqpoint{2.325000in}{2.310000in}}%
\pgfusepath{clip}%
\pgfsetbuttcap%
\pgfsetroundjoin%
\definecolor{currentfill}{rgb}{0.000000,0.000000,0.000000}%
\pgfsetfillcolor{currentfill}%
\pgfsetlinewidth{1.003750pt}%
\definecolor{currentstroke}{rgb}{0.000000,0.000000,0.000000}%
\pgfsetstrokecolor{currentstroke}%
\pgfsetdash{}{0pt}%
\pgfpathmoveto{\pgfqpoint{0.459750in}{0.399730in}}%
\pgfpathcurveto{\pgfqpoint{0.470800in}{0.399730in}}{\pgfqpoint{0.481399in}{0.404121in}}{\pgfqpoint{0.489213in}{0.411934in}}%
\pgfpathcurveto{\pgfqpoint{0.497026in}{0.419748in}}{\pgfqpoint{0.501417in}{0.430347in}}{\pgfqpoint{0.501417in}{0.441397in}}%
\pgfpathcurveto{\pgfqpoint{0.501417in}{0.452447in}}{\pgfqpoint{0.497026in}{0.463046in}}{\pgfqpoint{0.489213in}{0.470860in}}%
\pgfpathcurveto{\pgfqpoint{0.481399in}{0.478674in}}{\pgfqpoint{0.470800in}{0.483064in}}{\pgfqpoint{0.459750in}{0.483064in}}%
\pgfpathcurveto{\pgfqpoint{0.448700in}{0.483064in}}{\pgfqpoint{0.438101in}{0.478674in}}{\pgfqpoint{0.430287in}{0.470860in}}%
\pgfpathcurveto{\pgfqpoint{0.422474in}{0.463046in}}{\pgfqpoint{0.418083in}{0.452447in}}{\pgfqpoint{0.418083in}{0.441397in}}%
\pgfpathcurveto{\pgfqpoint{0.418083in}{0.430347in}}{\pgfqpoint{0.422474in}{0.419748in}}{\pgfqpoint{0.430287in}{0.411934in}}%
\pgfpathcurveto{\pgfqpoint{0.438101in}{0.404121in}}{\pgfqpoint{0.448700in}{0.399730in}}{\pgfqpoint{0.459750in}{0.399730in}}%
\pgfpathclose%
\pgfusepath{stroke,fill}%
\end{pgfscope}%
\begin{pgfscope}%
\pgfpathrectangle{\pgfqpoint{0.375000in}{0.330000in}}{\pgfqpoint{2.325000in}{2.310000in}}%
\pgfusepath{clip}%
\pgfsetbuttcap%
\pgfsetroundjoin%
\definecolor{currentfill}{rgb}{0.000000,0.000000,0.000000}%
\pgfsetfillcolor{currentfill}%
\pgfsetlinewidth{1.003750pt}%
\definecolor{currentstroke}{rgb}{0.000000,0.000000,0.000000}%
\pgfsetstrokecolor{currentstroke}%
\pgfsetdash{}{0pt}%
\pgfpathmoveto{\pgfqpoint{0.459750in}{0.417935in}}%
\pgfpathcurveto{\pgfqpoint{0.470800in}{0.417935in}}{\pgfqpoint{0.481399in}{0.422325in}}{\pgfqpoint{0.489213in}{0.430139in}}%
\pgfpathcurveto{\pgfqpoint{0.497026in}{0.437952in}}{\pgfqpoint{0.501417in}{0.448551in}}{\pgfqpoint{0.501417in}{0.459601in}}%
\pgfpathcurveto{\pgfqpoint{0.501417in}{0.470652in}}{\pgfqpoint{0.497026in}{0.481251in}}{\pgfqpoint{0.489213in}{0.489064in}}%
\pgfpathcurveto{\pgfqpoint{0.481399in}{0.496878in}}{\pgfqpoint{0.470800in}{0.501268in}}{\pgfqpoint{0.459750in}{0.501268in}}%
\pgfpathcurveto{\pgfqpoint{0.448700in}{0.501268in}}{\pgfqpoint{0.438101in}{0.496878in}}{\pgfqpoint{0.430287in}{0.489064in}}%
\pgfpathcurveto{\pgfqpoint{0.422474in}{0.481251in}}{\pgfqpoint{0.418083in}{0.470652in}}{\pgfqpoint{0.418083in}{0.459601in}}%
\pgfpathcurveto{\pgfqpoint{0.418083in}{0.448551in}}{\pgfqpoint{0.422474in}{0.437952in}}{\pgfqpoint{0.430287in}{0.430139in}}%
\pgfpathcurveto{\pgfqpoint{0.438101in}{0.422325in}}{\pgfqpoint{0.448700in}{0.417935in}}{\pgfqpoint{0.459750in}{0.417935in}}%
\pgfpathclose%
\pgfusepath{stroke,fill}%
\end{pgfscope}%
\begin{pgfscope}%
\pgfpathrectangle{\pgfqpoint{0.375000in}{0.330000in}}{\pgfqpoint{2.325000in}{2.310000in}}%
\pgfusepath{clip}%
\pgfsetbuttcap%
\pgfsetroundjoin%
\definecolor{currentfill}{rgb}{0.000000,0.000000,0.000000}%
\pgfsetfillcolor{currentfill}%
\pgfsetlinewidth{1.003750pt}%
\definecolor{currentstroke}{rgb}{0.000000,0.000000,0.000000}%
\pgfsetstrokecolor{currentstroke}%
\pgfsetdash{}{0pt}%
\pgfpathmoveto{\pgfqpoint{0.459750in}{0.424003in}}%
\pgfpathcurveto{\pgfqpoint{0.470800in}{0.424003in}}{\pgfqpoint{0.481399in}{0.428393in}}{\pgfqpoint{0.489213in}{0.436207in}}%
\pgfpathcurveto{\pgfqpoint{0.497026in}{0.444020in}}{\pgfqpoint{0.501417in}{0.454619in}}{\pgfqpoint{0.501417in}{0.465670in}}%
\pgfpathcurveto{\pgfqpoint{0.501417in}{0.476720in}}{\pgfqpoint{0.497026in}{0.487319in}}{\pgfqpoint{0.489213in}{0.495132in}}%
\pgfpathcurveto{\pgfqpoint{0.481399in}{0.502946in}}{\pgfqpoint{0.470800in}{0.507336in}}{\pgfqpoint{0.459750in}{0.507336in}}%
\pgfpathcurveto{\pgfqpoint{0.448700in}{0.507336in}}{\pgfqpoint{0.438101in}{0.502946in}}{\pgfqpoint{0.430287in}{0.495132in}}%
\pgfpathcurveto{\pgfqpoint{0.422474in}{0.487319in}}{\pgfqpoint{0.418083in}{0.476720in}}{\pgfqpoint{0.418083in}{0.465670in}}%
\pgfpathcurveto{\pgfqpoint{0.418083in}{0.454619in}}{\pgfqpoint{0.422474in}{0.444020in}}{\pgfqpoint{0.430287in}{0.436207in}}%
\pgfpathcurveto{\pgfqpoint{0.438101in}{0.428393in}}{\pgfqpoint{0.448700in}{0.424003in}}{\pgfqpoint{0.459750in}{0.424003in}}%
\pgfpathclose%
\pgfusepath{stroke,fill}%
\end{pgfscope}%
\begin{pgfscope}%
\pgfpathrectangle{\pgfqpoint{0.375000in}{0.330000in}}{\pgfqpoint{2.325000in}{2.310000in}}%
\pgfusepath{clip}%
\pgfsetbuttcap%
\pgfsetroundjoin%
\definecolor{currentfill}{rgb}{0.000000,0.000000,0.000000}%
\pgfsetfillcolor{currentfill}%
\pgfsetlinewidth{1.003750pt}%
\definecolor{currentstroke}{rgb}{0.000000,0.000000,0.000000}%
\pgfsetstrokecolor{currentstroke}%
\pgfsetdash{}{0pt}%
\pgfpathmoveto{\pgfqpoint{0.459750in}{0.424003in}}%
\pgfpathcurveto{\pgfqpoint{0.470800in}{0.424003in}}{\pgfqpoint{0.481399in}{0.428393in}}{\pgfqpoint{0.489213in}{0.436207in}}%
\pgfpathcurveto{\pgfqpoint{0.497026in}{0.444020in}}{\pgfqpoint{0.501417in}{0.454619in}}{\pgfqpoint{0.501417in}{0.465670in}}%
\pgfpathcurveto{\pgfqpoint{0.501417in}{0.476720in}}{\pgfqpoint{0.497026in}{0.487319in}}{\pgfqpoint{0.489213in}{0.495132in}}%
\pgfpathcurveto{\pgfqpoint{0.481399in}{0.502946in}}{\pgfqpoint{0.470800in}{0.507336in}}{\pgfqpoint{0.459750in}{0.507336in}}%
\pgfpathcurveto{\pgfqpoint{0.448700in}{0.507336in}}{\pgfqpoint{0.438101in}{0.502946in}}{\pgfqpoint{0.430287in}{0.495132in}}%
\pgfpathcurveto{\pgfqpoint{0.422474in}{0.487319in}}{\pgfqpoint{0.418083in}{0.476720in}}{\pgfqpoint{0.418083in}{0.465670in}}%
\pgfpathcurveto{\pgfqpoint{0.418083in}{0.454619in}}{\pgfqpoint{0.422474in}{0.444020in}}{\pgfqpoint{0.430287in}{0.436207in}}%
\pgfpathcurveto{\pgfqpoint{0.438101in}{0.428393in}}{\pgfqpoint{0.448700in}{0.424003in}}{\pgfqpoint{0.459750in}{0.424003in}}%
\pgfpathclose%
\pgfusepath{stroke,fill}%
\end{pgfscope}%
\begin{pgfscope}%
\pgfpathrectangle{\pgfqpoint{0.375000in}{0.330000in}}{\pgfqpoint{2.325000in}{2.310000in}}%
\pgfusepath{clip}%
\pgfsetbuttcap%
\pgfsetroundjoin%
\definecolor{currentfill}{rgb}{0.000000,0.000000,0.000000}%
\pgfsetfillcolor{currentfill}%
\pgfsetlinewidth{1.003750pt}%
\definecolor{currentstroke}{rgb}{0.000000,0.000000,0.000000}%
\pgfsetstrokecolor{currentstroke}%
\pgfsetdash{}{0pt}%
\pgfpathmoveto{\pgfqpoint{0.459750in}{0.411867in}}%
\pgfpathcurveto{\pgfqpoint{0.470800in}{0.411867in}}{\pgfqpoint{0.481399in}{0.416257in}}{\pgfqpoint{0.489213in}{0.424071in}}%
\pgfpathcurveto{\pgfqpoint{0.497026in}{0.431884in}}{\pgfqpoint{0.501417in}{0.442483in}}{\pgfqpoint{0.501417in}{0.453533in}}%
\pgfpathcurveto{\pgfqpoint{0.501417in}{0.464583in}}{\pgfqpoint{0.497026in}{0.475182in}}{\pgfqpoint{0.489213in}{0.482996in}}%
\pgfpathcurveto{\pgfqpoint{0.481399in}{0.490810in}}{\pgfqpoint{0.470800in}{0.495200in}}{\pgfqpoint{0.459750in}{0.495200in}}%
\pgfpathcurveto{\pgfqpoint{0.448700in}{0.495200in}}{\pgfqpoint{0.438101in}{0.490810in}}{\pgfqpoint{0.430287in}{0.482996in}}%
\pgfpathcurveto{\pgfqpoint{0.422474in}{0.475182in}}{\pgfqpoint{0.418083in}{0.464583in}}{\pgfqpoint{0.418083in}{0.453533in}}%
\pgfpathcurveto{\pgfqpoint{0.418083in}{0.442483in}}{\pgfqpoint{0.422474in}{0.431884in}}{\pgfqpoint{0.430287in}{0.424071in}}%
\pgfpathcurveto{\pgfqpoint{0.438101in}{0.416257in}}{\pgfqpoint{0.448700in}{0.411867in}}{\pgfqpoint{0.459750in}{0.411867in}}%
\pgfpathclose%
\pgfusepath{stroke,fill}%
\end{pgfscope}%
\begin{pgfscope}%
\pgfpathrectangle{\pgfqpoint{0.375000in}{0.330000in}}{\pgfqpoint{2.325000in}{2.310000in}}%
\pgfusepath{clip}%
\pgfsetbuttcap%
\pgfsetroundjoin%
\definecolor{currentfill}{rgb}{0.000000,0.000000,0.000000}%
\pgfsetfillcolor{currentfill}%
\pgfsetlinewidth{1.003750pt}%
\definecolor{currentstroke}{rgb}{0.000000,0.000000,0.000000}%
\pgfsetstrokecolor{currentstroke}%
\pgfsetdash{}{0pt}%
\pgfpathmoveto{\pgfqpoint{0.459750in}{0.424003in}}%
\pgfpathcurveto{\pgfqpoint{0.470800in}{0.424003in}}{\pgfqpoint{0.481399in}{0.428393in}}{\pgfqpoint{0.489213in}{0.436207in}}%
\pgfpathcurveto{\pgfqpoint{0.497026in}{0.444020in}}{\pgfqpoint{0.501417in}{0.454619in}}{\pgfqpoint{0.501417in}{0.465670in}}%
\pgfpathcurveto{\pgfqpoint{0.501417in}{0.476720in}}{\pgfqpoint{0.497026in}{0.487319in}}{\pgfqpoint{0.489213in}{0.495132in}}%
\pgfpathcurveto{\pgfqpoint{0.481399in}{0.502946in}}{\pgfqpoint{0.470800in}{0.507336in}}{\pgfqpoint{0.459750in}{0.507336in}}%
\pgfpathcurveto{\pgfqpoint{0.448700in}{0.507336in}}{\pgfqpoint{0.438101in}{0.502946in}}{\pgfqpoint{0.430287in}{0.495132in}}%
\pgfpathcurveto{\pgfqpoint{0.422474in}{0.487319in}}{\pgfqpoint{0.418083in}{0.476720in}}{\pgfqpoint{0.418083in}{0.465670in}}%
\pgfpathcurveto{\pgfqpoint{0.418083in}{0.454619in}}{\pgfqpoint{0.422474in}{0.444020in}}{\pgfqpoint{0.430287in}{0.436207in}}%
\pgfpathcurveto{\pgfqpoint{0.438101in}{0.428393in}}{\pgfqpoint{0.448700in}{0.424003in}}{\pgfqpoint{0.459750in}{0.424003in}}%
\pgfpathclose%
\pgfusepath{stroke,fill}%
\end{pgfscope}%
\begin{pgfscope}%
\pgfpathrectangle{\pgfqpoint{0.375000in}{0.330000in}}{\pgfqpoint{2.325000in}{2.310000in}}%
\pgfusepath{clip}%
\pgfsetbuttcap%
\pgfsetroundjoin%
\definecolor{currentfill}{rgb}{0.000000,0.000000,0.000000}%
\pgfsetfillcolor{currentfill}%
\pgfsetlinewidth{1.003750pt}%
\definecolor{currentstroke}{rgb}{0.000000,0.000000,0.000000}%
\pgfsetstrokecolor{currentstroke}%
\pgfsetdash{}{0pt}%
\pgfpathmoveto{\pgfqpoint{0.459750in}{0.405799in}}%
\pgfpathcurveto{\pgfqpoint{0.470800in}{0.405799in}}{\pgfqpoint{0.481399in}{0.410189in}}{\pgfqpoint{0.489213in}{0.418002in}}%
\pgfpathcurveto{\pgfqpoint{0.497026in}{0.425816in}}{\pgfqpoint{0.501417in}{0.436415in}}{\pgfqpoint{0.501417in}{0.447465in}}%
\pgfpathcurveto{\pgfqpoint{0.501417in}{0.458515in}}{\pgfqpoint{0.497026in}{0.469114in}}{\pgfqpoint{0.489213in}{0.476928in}}%
\pgfpathcurveto{\pgfqpoint{0.481399in}{0.484742in}}{\pgfqpoint{0.470800in}{0.489132in}}{\pgfqpoint{0.459750in}{0.489132in}}%
\pgfpathcurveto{\pgfqpoint{0.448700in}{0.489132in}}{\pgfqpoint{0.438101in}{0.484742in}}{\pgfqpoint{0.430287in}{0.476928in}}%
\pgfpathcurveto{\pgfqpoint{0.422474in}{0.469114in}}{\pgfqpoint{0.418083in}{0.458515in}}{\pgfqpoint{0.418083in}{0.447465in}}%
\pgfpathcurveto{\pgfqpoint{0.418083in}{0.436415in}}{\pgfqpoint{0.422474in}{0.425816in}}{\pgfqpoint{0.430287in}{0.418002in}}%
\pgfpathcurveto{\pgfqpoint{0.438101in}{0.410189in}}{\pgfqpoint{0.448700in}{0.405799in}}{\pgfqpoint{0.459750in}{0.405799in}}%
\pgfpathclose%
\pgfusepath{stroke,fill}%
\end{pgfscope}%
\begin{pgfscope}%
\pgfpathrectangle{\pgfqpoint{0.375000in}{0.330000in}}{\pgfqpoint{2.325000in}{2.310000in}}%
\pgfusepath{clip}%
\pgfsetbuttcap%
\pgfsetroundjoin%
\definecolor{currentfill}{rgb}{0.000000,0.000000,0.000000}%
\pgfsetfillcolor{currentfill}%
\pgfsetlinewidth{1.003750pt}%
\definecolor{currentstroke}{rgb}{0.000000,0.000000,0.000000}%
\pgfsetstrokecolor{currentstroke}%
\pgfsetdash{}{0pt}%
\pgfpathmoveto{\pgfqpoint{0.459750in}{0.405799in}}%
\pgfpathcurveto{\pgfqpoint{0.470800in}{0.405799in}}{\pgfqpoint{0.481399in}{0.410189in}}{\pgfqpoint{0.489213in}{0.418002in}}%
\pgfpathcurveto{\pgfqpoint{0.497026in}{0.425816in}}{\pgfqpoint{0.501417in}{0.436415in}}{\pgfqpoint{0.501417in}{0.447465in}}%
\pgfpathcurveto{\pgfqpoint{0.501417in}{0.458515in}}{\pgfqpoint{0.497026in}{0.469114in}}{\pgfqpoint{0.489213in}{0.476928in}}%
\pgfpathcurveto{\pgfqpoint{0.481399in}{0.484742in}}{\pgfqpoint{0.470800in}{0.489132in}}{\pgfqpoint{0.459750in}{0.489132in}}%
\pgfpathcurveto{\pgfqpoint{0.448700in}{0.489132in}}{\pgfqpoint{0.438101in}{0.484742in}}{\pgfqpoint{0.430287in}{0.476928in}}%
\pgfpathcurveto{\pgfqpoint{0.422474in}{0.469114in}}{\pgfqpoint{0.418083in}{0.458515in}}{\pgfqpoint{0.418083in}{0.447465in}}%
\pgfpathcurveto{\pgfqpoint{0.418083in}{0.436415in}}{\pgfqpoint{0.422474in}{0.425816in}}{\pgfqpoint{0.430287in}{0.418002in}}%
\pgfpathcurveto{\pgfqpoint{0.438101in}{0.410189in}}{\pgfqpoint{0.448700in}{0.405799in}}{\pgfqpoint{0.459750in}{0.405799in}}%
\pgfpathclose%
\pgfusepath{stroke,fill}%
\end{pgfscope}%
\begin{pgfscope}%
\pgfpathrectangle{\pgfqpoint{0.375000in}{0.330000in}}{\pgfqpoint{2.325000in}{2.310000in}}%
\pgfusepath{clip}%
\pgfsetbuttcap%
\pgfsetroundjoin%
\definecolor{currentfill}{rgb}{0.000000,0.000000,0.000000}%
\pgfsetfillcolor{currentfill}%
\pgfsetlinewidth{1.003750pt}%
\definecolor{currentstroke}{rgb}{0.000000,0.000000,0.000000}%
\pgfsetstrokecolor{currentstroke}%
\pgfsetdash{}{0pt}%
\pgfpathmoveto{\pgfqpoint{0.459750in}{0.411867in}}%
\pgfpathcurveto{\pgfqpoint{0.470800in}{0.411867in}}{\pgfqpoint{0.481399in}{0.416257in}}{\pgfqpoint{0.489213in}{0.424071in}}%
\pgfpathcurveto{\pgfqpoint{0.497026in}{0.431884in}}{\pgfqpoint{0.501417in}{0.442483in}}{\pgfqpoint{0.501417in}{0.453533in}}%
\pgfpathcurveto{\pgfqpoint{0.501417in}{0.464583in}}{\pgfqpoint{0.497026in}{0.475182in}}{\pgfqpoint{0.489213in}{0.482996in}}%
\pgfpathcurveto{\pgfqpoint{0.481399in}{0.490810in}}{\pgfqpoint{0.470800in}{0.495200in}}{\pgfqpoint{0.459750in}{0.495200in}}%
\pgfpathcurveto{\pgfqpoint{0.448700in}{0.495200in}}{\pgfqpoint{0.438101in}{0.490810in}}{\pgfqpoint{0.430287in}{0.482996in}}%
\pgfpathcurveto{\pgfqpoint{0.422474in}{0.475182in}}{\pgfqpoint{0.418083in}{0.464583in}}{\pgfqpoint{0.418083in}{0.453533in}}%
\pgfpathcurveto{\pgfqpoint{0.418083in}{0.442483in}}{\pgfqpoint{0.422474in}{0.431884in}}{\pgfqpoint{0.430287in}{0.424071in}}%
\pgfpathcurveto{\pgfqpoint{0.438101in}{0.416257in}}{\pgfqpoint{0.448700in}{0.411867in}}{\pgfqpoint{0.459750in}{0.411867in}}%
\pgfpathclose%
\pgfusepath{stroke,fill}%
\end{pgfscope}%
\begin{pgfscope}%
\pgfpathrectangle{\pgfqpoint{0.375000in}{0.330000in}}{\pgfqpoint{2.325000in}{2.310000in}}%
\pgfusepath{clip}%
\pgfsetbuttcap%
\pgfsetroundjoin%
\definecolor{currentfill}{rgb}{0.000000,0.000000,0.000000}%
\pgfsetfillcolor{currentfill}%
\pgfsetlinewidth{1.003750pt}%
\definecolor{currentstroke}{rgb}{0.000000,0.000000,0.000000}%
\pgfsetstrokecolor{currentstroke}%
\pgfsetdash{}{0pt}%
\pgfpathmoveto{\pgfqpoint{0.459750in}{0.436139in}}%
\pgfpathcurveto{\pgfqpoint{0.470800in}{0.436139in}}{\pgfqpoint{0.481399in}{0.440529in}}{\pgfqpoint{0.489213in}{0.448343in}}%
\pgfpathcurveto{\pgfqpoint{0.497026in}{0.456157in}}{\pgfqpoint{0.501417in}{0.466756in}}{\pgfqpoint{0.501417in}{0.477806in}}%
\pgfpathcurveto{\pgfqpoint{0.501417in}{0.488856in}}{\pgfqpoint{0.497026in}{0.499455in}}{\pgfqpoint{0.489213in}{0.507269in}}%
\pgfpathcurveto{\pgfqpoint{0.481399in}{0.515082in}}{\pgfqpoint{0.470800in}{0.519472in}}{\pgfqpoint{0.459750in}{0.519472in}}%
\pgfpathcurveto{\pgfqpoint{0.448700in}{0.519472in}}{\pgfqpoint{0.438101in}{0.515082in}}{\pgfqpoint{0.430287in}{0.507269in}}%
\pgfpathcurveto{\pgfqpoint{0.422474in}{0.499455in}}{\pgfqpoint{0.418083in}{0.488856in}}{\pgfqpoint{0.418083in}{0.477806in}}%
\pgfpathcurveto{\pgfqpoint{0.418083in}{0.466756in}}{\pgfqpoint{0.422474in}{0.456157in}}{\pgfqpoint{0.430287in}{0.448343in}}%
\pgfpathcurveto{\pgfqpoint{0.438101in}{0.440529in}}{\pgfqpoint{0.448700in}{0.436139in}}{\pgfqpoint{0.459750in}{0.436139in}}%
\pgfpathclose%
\pgfusepath{stroke,fill}%
\end{pgfscope}%
\begin{pgfscope}%
\pgfpathrectangle{\pgfqpoint{0.375000in}{0.330000in}}{\pgfqpoint{2.325000in}{2.310000in}}%
\pgfusepath{clip}%
\pgfsetbuttcap%
\pgfsetroundjoin%
\definecolor{currentfill}{rgb}{0.000000,0.000000,0.000000}%
\pgfsetfillcolor{currentfill}%
\pgfsetlinewidth{1.003750pt}%
\definecolor{currentstroke}{rgb}{0.000000,0.000000,0.000000}%
\pgfsetstrokecolor{currentstroke}%
\pgfsetdash{}{0pt}%
\pgfpathmoveto{\pgfqpoint{0.459750in}{0.417935in}}%
\pgfpathcurveto{\pgfqpoint{0.470800in}{0.417935in}}{\pgfqpoint{0.481399in}{0.422325in}}{\pgfqpoint{0.489213in}{0.430139in}}%
\pgfpathcurveto{\pgfqpoint{0.497026in}{0.437952in}}{\pgfqpoint{0.501417in}{0.448551in}}{\pgfqpoint{0.501417in}{0.459601in}}%
\pgfpathcurveto{\pgfqpoint{0.501417in}{0.470652in}}{\pgfqpoint{0.497026in}{0.481251in}}{\pgfqpoint{0.489213in}{0.489064in}}%
\pgfpathcurveto{\pgfqpoint{0.481399in}{0.496878in}}{\pgfqpoint{0.470800in}{0.501268in}}{\pgfqpoint{0.459750in}{0.501268in}}%
\pgfpathcurveto{\pgfqpoint{0.448700in}{0.501268in}}{\pgfqpoint{0.438101in}{0.496878in}}{\pgfqpoint{0.430287in}{0.489064in}}%
\pgfpathcurveto{\pgfqpoint{0.422474in}{0.481251in}}{\pgfqpoint{0.418083in}{0.470652in}}{\pgfqpoint{0.418083in}{0.459601in}}%
\pgfpathcurveto{\pgfqpoint{0.418083in}{0.448551in}}{\pgfqpoint{0.422474in}{0.437952in}}{\pgfqpoint{0.430287in}{0.430139in}}%
\pgfpathcurveto{\pgfqpoint{0.438101in}{0.422325in}}{\pgfqpoint{0.448700in}{0.417935in}}{\pgfqpoint{0.459750in}{0.417935in}}%
\pgfpathclose%
\pgfusepath{stroke,fill}%
\end{pgfscope}%
\begin{pgfscope}%
\pgfpathrectangle{\pgfqpoint{0.375000in}{0.330000in}}{\pgfqpoint{2.325000in}{2.310000in}}%
\pgfusepath{clip}%
\pgfsetbuttcap%
\pgfsetroundjoin%
\definecolor{currentfill}{rgb}{0.000000,0.000000,0.000000}%
\pgfsetfillcolor{currentfill}%
\pgfsetlinewidth{1.003750pt}%
\definecolor{currentstroke}{rgb}{0.000000,0.000000,0.000000}%
\pgfsetstrokecolor{currentstroke}%
\pgfsetdash{}{0pt}%
\pgfpathmoveto{\pgfqpoint{0.459750in}{0.399730in}}%
\pgfpathcurveto{\pgfqpoint{0.470800in}{0.399730in}}{\pgfqpoint{0.481399in}{0.404121in}}{\pgfqpoint{0.489213in}{0.411934in}}%
\pgfpathcurveto{\pgfqpoint{0.497026in}{0.419748in}}{\pgfqpoint{0.501417in}{0.430347in}}{\pgfqpoint{0.501417in}{0.441397in}}%
\pgfpathcurveto{\pgfqpoint{0.501417in}{0.452447in}}{\pgfqpoint{0.497026in}{0.463046in}}{\pgfqpoint{0.489213in}{0.470860in}}%
\pgfpathcurveto{\pgfqpoint{0.481399in}{0.478674in}}{\pgfqpoint{0.470800in}{0.483064in}}{\pgfqpoint{0.459750in}{0.483064in}}%
\pgfpathcurveto{\pgfqpoint{0.448700in}{0.483064in}}{\pgfqpoint{0.438101in}{0.478674in}}{\pgfqpoint{0.430287in}{0.470860in}}%
\pgfpathcurveto{\pgfqpoint{0.422474in}{0.463046in}}{\pgfqpoint{0.418083in}{0.452447in}}{\pgfqpoint{0.418083in}{0.441397in}}%
\pgfpathcurveto{\pgfqpoint{0.418083in}{0.430347in}}{\pgfqpoint{0.422474in}{0.419748in}}{\pgfqpoint{0.430287in}{0.411934in}}%
\pgfpathcurveto{\pgfqpoint{0.438101in}{0.404121in}}{\pgfqpoint{0.448700in}{0.399730in}}{\pgfqpoint{0.459750in}{0.399730in}}%
\pgfpathclose%
\pgfusepath{stroke,fill}%
\end{pgfscope}%
\begin{pgfscope}%
\pgfpathrectangle{\pgfqpoint{0.375000in}{0.330000in}}{\pgfqpoint{2.325000in}{2.310000in}}%
\pgfusepath{clip}%
\pgfsetbuttcap%
\pgfsetroundjoin%
\definecolor{currentfill}{rgb}{0.000000,0.000000,0.000000}%
\pgfsetfillcolor{currentfill}%
\pgfsetlinewidth{1.003750pt}%
\definecolor{currentstroke}{rgb}{0.000000,0.000000,0.000000}%
\pgfsetstrokecolor{currentstroke}%
\pgfsetdash{}{0pt}%
\pgfpathmoveto{\pgfqpoint{0.459750in}{0.411867in}}%
\pgfpathcurveto{\pgfqpoint{0.470800in}{0.411867in}}{\pgfqpoint{0.481399in}{0.416257in}}{\pgfqpoint{0.489213in}{0.424071in}}%
\pgfpathcurveto{\pgfqpoint{0.497026in}{0.431884in}}{\pgfqpoint{0.501417in}{0.442483in}}{\pgfqpoint{0.501417in}{0.453533in}}%
\pgfpathcurveto{\pgfqpoint{0.501417in}{0.464583in}}{\pgfqpoint{0.497026in}{0.475182in}}{\pgfqpoint{0.489213in}{0.482996in}}%
\pgfpathcurveto{\pgfqpoint{0.481399in}{0.490810in}}{\pgfqpoint{0.470800in}{0.495200in}}{\pgfqpoint{0.459750in}{0.495200in}}%
\pgfpathcurveto{\pgfqpoint{0.448700in}{0.495200in}}{\pgfqpoint{0.438101in}{0.490810in}}{\pgfqpoint{0.430287in}{0.482996in}}%
\pgfpathcurveto{\pgfqpoint{0.422474in}{0.475182in}}{\pgfqpoint{0.418083in}{0.464583in}}{\pgfqpoint{0.418083in}{0.453533in}}%
\pgfpathcurveto{\pgfqpoint{0.418083in}{0.442483in}}{\pgfqpoint{0.422474in}{0.431884in}}{\pgfqpoint{0.430287in}{0.424071in}}%
\pgfpathcurveto{\pgfqpoint{0.438101in}{0.416257in}}{\pgfqpoint{0.448700in}{0.411867in}}{\pgfqpoint{0.459750in}{0.411867in}}%
\pgfpathclose%
\pgfusepath{stroke,fill}%
\end{pgfscope}%
\begin{pgfscope}%
\pgfpathrectangle{\pgfqpoint{0.375000in}{0.330000in}}{\pgfqpoint{2.325000in}{2.310000in}}%
\pgfusepath{clip}%
\pgfsetbuttcap%
\pgfsetroundjoin%
\definecolor{currentfill}{rgb}{0.000000,0.000000,0.000000}%
\pgfsetfillcolor{currentfill}%
\pgfsetlinewidth{1.003750pt}%
\definecolor{currentstroke}{rgb}{0.000000,0.000000,0.000000}%
\pgfsetstrokecolor{currentstroke}%
\pgfsetdash{}{0pt}%
\pgfpathmoveto{\pgfqpoint{0.459750in}{0.430071in}}%
\pgfpathcurveto{\pgfqpoint{0.470800in}{0.430071in}}{\pgfqpoint{0.481399in}{0.434461in}}{\pgfqpoint{0.489213in}{0.442275in}}%
\pgfpathcurveto{\pgfqpoint{0.497026in}{0.450088in}}{\pgfqpoint{0.501417in}{0.460688in}}{\pgfqpoint{0.501417in}{0.471738in}}%
\pgfpathcurveto{\pgfqpoint{0.501417in}{0.482788in}}{\pgfqpoint{0.497026in}{0.493387in}}{\pgfqpoint{0.489213in}{0.501200in}}%
\pgfpathcurveto{\pgfqpoint{0.481399in}{0.509014in}}{\pgfqpoint{0.470800in}{0.513404in}}{\pgfqpoint{0.459750in}{0.513404in}}%
\pgfpathcurveto{\pgfqpoint{0.448700in}{0.513404in}}{\pgfqpoint{0.438101in}{0.509014in}}{\pgfqpoint{0.430287in}{0.501200in}}%
\pgfpathcurveto{\pgfqpoint{0.422474in}{0.493387in}}{\pgfqpoint{0.418083in}{0.482788in}}{\pgfqpoint{0.418083in}{0.471738in}}%
\pgfpathcurveto{\pgfqpoint{0.418083in}{0.460688in}}{\pgfqpoint{0.422474in}{0.450088in}}{\pgfqpoint{0.430287in}{0.442275in}}%
\pgfpathcurveto{\pgfqpoint{0.438101in}{0.434461in}}{\pgfqpoint{0.448700in}{0.430071in}}{\pgfqpoint{0.459750in}{0.430071in}}%
\pgfpathclose%
\pgfusepath{stroke,fill}%
\end{pgfscope}%
\begin{pgfscope}%
\pgfpathrectangle{\pgfqpoint{0.375000in}{0.330000in}}{\pgfqpoint{2.325000in}{2.310000in}}%
\pgfusepath{clip}%
\pgfsetbuttcap%
\pgfsetroundjoin%
\definecolor{currentfill}{rgb}{0.000000,0.000000,0.000000}%
\pgfsetfillcolor{currentfill}%
\pgfsetlinewidth{1.003750pt}%
\definecolor{currentstroke}{rgb}{0.000000,0.000000,0.000000}%
\pgfsetstrokecolor{currentstroke}%
\pgfsetdash{}{0pt}%
\pgfpathmoveto{\pgfqpoint{0.459750in}{0.411867in}}%
\pgfpathcurveto{\pgfqpoint{0.470800in}{0.411867in}}{\pgfqpoint{0.481399in}{0.416257in}}{\pgfqpoint{0.489213in}{0.424071in}}%
\pgfpathcurveto{\pgfqpoint{0.497026in}{0.431884in}}{\pgfqpoint{0.501417in}{0.442483in}}{\pgfqpoint{0.501417in}{0.453533in}}%
\pgfpathcurveto{\pgfqpoint{0.501417in}{0.464583in}}{\pgfqpoint{0.497026in}{0.475182in}}{\pgfqpoint{0.489213in}{0.482996in}}%
\pgfpathcurveto{\pgfqpoint{0.481399in}{0.490810in}}{\pgfqpoint{0.470800in}{0.495200in}}{\pgfqpoint{0.459750in}{0.495200in}}%
\pgfpathcurveto{\pgfqpoint{0.448700in}{0.495200in}}{\pgfqpoint{0.438101in}{0.490810in}}{\pgfqpoint{0.430287in}{0.482996in}}%
\pgfpathcurveto{\pgfqpoint{0.422474in}{0.475182in}}{\pgfqpoint{0.418083in}{0.464583in}}{\pgfqpoint{0.418083in}{0.453533in}}%
\pgfpathcurveto{\pgfqpoint{0.418083in}{0.442483in}}{\pgfqpoint{0.422474in}{0.431884in}}{\pgfqpoint{0.430287in}{0.424071in}}%
\pgfpathcurveto{\pgfqpoint{0.438101in}{0.416257in}}{\pgfqpoint{0.448700in}{0.411867in}}{\pgfqpoint{0.459750in}{0.411867in}}%
\pgfpathclose%
\pgfusepath{stroke,fill}%
\end{pgfscope}%
\begin{pgfscope}%
\pgfpathrectangle{\pgfqpoint{0.375000in}{0.330000in}}{\pgfqpoint{2.325000in}{2.310000in}}%
\pgfusepath{clip}%
\pgfsetbuttcap%
\pgfsetroundjoin%
\definecolor{currentfill}{rgb}{0.000000,0.000000,0.000000}%
\pgfsetfillcolor{currentfill}%
\pgfsetlinewidth{1.003750pt}%
\definecolor{currentstroke}{rgb}{0.000000,0.000000,0.000000}%
\pgfsetstrokecolor{currentstroke}%
\pgfsetdash{}{0pt}%
\pgfpathmoveto{\pgfqpoint{0.459750in}{0.430071in}}%
\pgfpathcurveto{\pgfqpoint{0.470800in}{0.430071in}}{\pgfqpoint{0.481399in}{0.434461in}}{\pgfqpoint{0.489213in}{0.442275in}}%
\pgfpathcurveto{\pgfqpoint{0.497026in}{0.450088in}}{\pgfqpoint{0.501417in}{0.460688in}}{\pgfqpoint{0.501417in}{0.471738in}}%
\pgfpathcurveto{\pgfqpoint{0.501417in}{0.482788in}}{\pgfqpoint{0.497026in}{0.493387in}}{\pgfqpoint{0.489213in}{0.501200in}}%
\pgfpathcurveto{\pgfqpoint{0.481399in}{0.509014in}}{\pgfqpoint{0.470800in}{0.513404in}}{\pgfqpoint{0.459750in}{0.513404in}}%
\pgfpathcurveto{\pgfqpoint{0.448700in}{0.513404in}}{\pgfqpoint{0.438101in}{0.509014in}}{\pgfqpoint{0.430287in}{0.501200in}}%
\pgfpathcurveto{\pgfqpoint{0.422474in}{0.493387in}}{\pgfqpoint{0.418083in}{0.482788in}}{\pgfqpoint{0.418083in}{0.471738in}}%
\pgfpathcurveto{\pgfqpoint{0.418083in}{0.460688in}}{\pgfqpoint{0.422474in}{0.450088in}}{\pgfqpoint{0.430287in}{0.442275in}}%
\pgfpathcurveto{\pgfqpoint{0.438101in}{0.434461in}}{\pgfqpoint{0.448700in}{0.430071in}}{\pgfqpoint{0.459750in}{0.430071in}}%
\pgfpathclose%
\pgfusepath{stroke,fill}%
\end{pgfscope}%
\begin{pgfscope}%
\pgfpathrectangle{\pgfqpoint{0.375000in}{0.330000in}}{\pgfqpoint{2.325000in}{2.310000in}}%
\pgfusepath{clip}%
\pgfsetbuttcap%
\pgfsetroundjoin%
\definecolor{currentfill}{rgb}{0.000000,0.000000,0.000000}%
\pgfsetfillcolor{currentfill}%
\pgfsetlinewidth{1.003750pt}%
\definecolor{currentstroke}{rgb}{0.000000,0.000000,0.000000}%
\pgfsetstrokecolor{currentstroke}%
\pgfsetdash{}{0pt}%
\pgfpathmoveto{\pgfqpoint{0.459750in}{0.424003in}}%
\pgfpathcurveto{\pgfqpoint{0.470800in}{0.424003in}}{\pgfqpoint{0.481399in}{0.428393in}}{\pgfqpoint{0.489213in}{0.436207in}}%
\pgfpathcurveto{\pgfqpoint{0.497026in}{0.444020in}}{\pgfqpoint{0.501417in}{0.454619in}}{\pgfqpoint{0.501417in}{0.465670in}}%
\pgfpathcurveto{\pgfqpoint{0.501417in}{0.476720in}}{\pgfqpoint{0.497026in}{0.487319in}}{\pgfqpoint{0.489213in}{0.495132in}}%
\pgfpathcurveto{\pgfqpoint{0.481399in}{0.502946in}}{\pgfqpoint{0.470800in}{0.507336in}}{\pgfqpoint{0.459750in}{0.507336in}}%
\pgfpathcurveto{\pgfqpoint{0.448700in}{0.507336in}}{\pgfqpoint{0.438101in}{0.502946in}}{\pgfqpoint{0.430287in}{0.495132in}}%
\pgfpathcurveto{\pgfqpoint{0.422474in}{0.487319in}}{\pgfqpoint{0.418083in}{0.476720in}}{\pgfqpoint{0.418083in}{0.465670in}}%
\pgfpathcurveto{\pgfqpoint{0.418083in}{0.454619in}}{\pgfqpoint{0.422474in}{0.444020in}}{\pgfqpoint{0.430287in}{0.436207in}}%
\pgfpathcurveto{\pgfqpoint{0.438101in}{0.428393in}}{\pgfqpoint{0.448700in}{0.424003in}}{\pgfqpoint{0.459750in}{0.424003in}}%
\pgfpathclose%
\pgfusepath{stroke,fill}%
\end{pgfscope}%
\begin{pgfscope}%
\pgfpathrectangle{\pgfqpoint{0.375000in}{0.330000in}}{\pgfqpoint{2.325000in}{2.310000in}}%
\pgfusepath{clip}%
\pgfsetbuttcap%
\pgfsetroundjoin%
\definecolor{currentfill}{rgb}{0.000000,0.000000,0.000000}%
\pgfsetfillcolor{currentfill}%
\pgfsetlinewidth{1.003750pt}%
\definecolor{currentstroke}{rgb}{0.000000,0.000000,0.000000}%
\pgfsetstrokecolor{currentstroke}%
\pgfsetdash{}{0pt}%
\pgfpathmoveto{\pgfqpoint{0.459750in}{0.411867in}}%
\pgfpathcurveto{\pgfqpoint{0.470800in}{0.411867in}}{\pgfqpoint{0.481399in}{0.416257in}}{\pgfqpoint{0.489213in}{0.424071in}}%
\pgfpathcurveto{\pgfqpoint{0.497026in}{0.431884in}}{\pgfqpoint{0.501417in}{0.442483in}}{\pgfqpoint{0.501417in}{0.453533in}}%
\pgfpathcurveto{\pgfqpoint{0.501417in}{0.464583in}}{\pgfqpoint{0.497026in}{0.475182in}}{\pgfqpoint{0.489213in}{0.482996in}}%
\pgfpathcurveto{\pgfqpoint{0.481399in}{0.490810in}}{\pgfqpoint{0.470800in}{0.495200in}}{\pgfqpoint{0.459750in}{0.495200in}}%
\pgfpathcurveto{\pgfqpoint{0.448700in}{0.495200in}}{\pgfqpoint{0.438101in}{0.490810in}}{\pgfqpoint{0.430287in}{0.482996in}}%
\pgfpathcurveto{\pgfqpoint{0.422474in}{0.475182in}}{\pgfqpoint{0.418083in}{0.464583in}}{\pgfqpoint{0.418083in}{0.453533in}}%
\pgfpathcurveto{\pgfqpoint{0.418083in}{0.442483in}}{\pgfqpoint{0.422474in}{0.431884in}}{\pgfqpoint{0.430287in}{0.424071in}}%
\pgfpathcurveto{\pgfqpoint{0.438101in}{0.416257in}}{\pgfqpoint{0.448700in}{0.411867in}}{\pgfqpoint{0.459750in}{0.411867in}}%
\pgfpathclose%
\pgfusepath{stroke,fill}%
\end{pgfscope}%
\begin{pgfscope}%
\pgfpathrectangle{\pgfqpoint{0.375000in}{0.330000in}}{\pgfqpoint{2.325000in}{2.310000in}}%
\pgfusepath{clip}%
\pgfsetbuttcap%
\pgfsetroundjoin%
\definecolor{currentfill}{rgb}{0.000000,0.000000,0.000000}%
\pgfsetfillcolor{currentfill}%
\pgfsetlinewidth{1.003750pt}%
\definecolor{currentstroke}{rgb}{0.000000,0.000000,0.000000}%
\pgfsetstrokecolor{currentstroke}%
\pgfsetdash{}{0pt}%
\pgfpathmoveto{\pgfqpoint{0.459750in}{0.411867in}}%
\pgfpathcurveto{\pgfqpoint{0.470800in}{0.411867in}}{\pgfqpoint{0.481399in}{0.416257in}}{\pgfqpoint{0.489213in}{0.424071in}}%
\pgfpathcurveto{\pgfqpoint{0.497026in}{0.431884in}}{\pgfqpoint{0.501417in}{0.442483in}}{\pgfqpoint{0.501417in}{0.453533in}}%
\pgfpathcurveto{\pgfqpoint{0.501417in}{0.464583in}}{\pgfqpoint{0.497026in}{0.475182in}}{\pgfqpoint{0.489213in}{0.482996in}}%
\pgfpathcurveto{\pgfqpoint{0.481399in}{0.490810in}}{\pgfqpoint{0.470800in}{0.495200in}}{\pgfqpoint{0.459750in}{0.495200in}}%
\pgfpathcurveto{\pgfqpoint{0.448700in}{0.495200in}}{\pgfqpoint{0.438101in}{0.490810in}}{\pgfqpoint{0.430287in}{0.482996in}}%
\pgfpathcurveto{\pgfqpoint{0.422474in}{0.475182in}}{\pgfqpoint{0.418083in}{0.464583in}}{\pgfqpoint{0.418083in}{0.453533in}}%
\pgfpathcurveto{\pgfqpoint{0.418083in}{0.442483in}}{\pgfqpoint{0.422474in}{0.431884in}}{\pgfqpoint{0.430287in}{0.424071in}}%
\pgfpathcurveto{\pgfqpoint{0.438101in}{0.416257in}}{\pgfqpoint{0.448700in}{0.411867in}}{\pgfqpoint{0.459750in}{0.411867in}}%
\pgfpathclose%
\pgfusepath{stroke,fill}%
\end{pgfscope}%
\begin{pgfscope}%
\pgfpathrectangle{\pgfqpoint{0.375000in}{0.330000in}}{\pgfqpoint{2.325000in}{2.310000in}}%
\pgfusepath{clip}%
\pgfsetbuttcap%
\pgfsetroundjoin%
\definecolor{currentfill}{rgb}{0.000000,0.000000,0.000000}%
\pgfsetfillcolor{currentfill}%
\pgfsetlinewidth{1.003750pt}%
\definecolor{currentstroke}{rgb}{0.000000,0.000000,0.000000}%
\pgfsetstrokecolor{currentstroke}%
\pgfsetdash{}{0pt}%
\pgfpathmoveto{\pgfqpoint{0.459750in}{0.424003in}}%
\pgfpathcurveto{\pgfqpoint{0.470800in}{0.424003in}}{\pgfqpoint{0.481399in}{0.428393in}}{\pgfqpoint{0.489213in}{0.436207in}}%
\pgfpathcurveto{\pgfqpoint{0.497026in}{0.444020in}}{\pgfqpoint{0.501417in}{0.454619in}}{\pgfqpoint{0.501417in}{0.465670in}}%
\pgfpathcurveto{\pgfqpoint{0.501417in}{0.476720in}}{\pgfqpoint{0.497026in}{0.487319in}}{\pgfqpoint{0.489213in}{0.495132in}}%
\pgfpathcurveto{\pgfqpoint{0.481399in}{0.502946in}}{\pgfqpoint{0.470800in}{0.507336in}}{\pgfqpoint{0.459750in}{0.507336in}}%
\pgfpathcurveto{\pgfqpoint{0.448700in}{0.507336in}}{\pgfqpoint{0.438101in}{0.502946in}}{\pgfqpoint{0.430287in}{0.495132in}}%
\pgfpathcurveto{\pgfqpoint{0.422474in}{0.487319in}}{\pgfqpoint{0.418083in}{0.476720in}}{\pgfqpoint{0.418083in}{0.465670in}}%
\pgfpathcurveto{\pgfqpoint{0.418083in}{0.454619in}}{\pgfqpoint{0.422474in}{0.444020in}}{\pgfqpoint{0.430287in}{0.436207in}}%
\pgfpathcurveto{\pgfqpoint{0.438101in}{0.428393in}}{\pgfqpoint{0.448700in}{0.424003in}}{\pgfqpoint{0.459750in}{0.424003in}}%
\pgfpathclose%
\pgfusepath{stroke,fill}%
\end{pgfscope}%
\begin{pgfscope}%
\pgfpathrectangle{\pgfqpoint{0.375000in}{0.330000in}}{\pgfqpoint{2.325000in}{2.310000in}}%
\pgfusepath{clip}%
\pgfsetbuttcap%
\pgfsetroundjoin%
\definecolor{currentfill}{rgb}{0.000000,0.000000,0.000000}%
\pgfsetfillcolor{currentfill}%
\pgfsetlinewidth{1.003750pt}%
\definecolor{currentstroke}{rgb}{0.000000,0.000000,0.000000}%
\pgfsetstrokecolor{currentstroke}%
\pgfsetdash{}{0pt}%
\pgfpathmoveto{\pgfqpoint{0.459750in}{0.411867in}}%
\pgfpathcurveto{\pgfqpoint{0.470800in}{0.411867in}}{\pgfqpoint{0.481399in}{0.416257in}}{\pgfqpoint{0.489213in}{0.424071in}}%
\pgfpathcurveto{\pgfqpoint{0.497026in}{0.431884in}}{\pgfqpoint{0.501417in}{0.442483in}}{\pgfqpoint{0.501417in}{0.453533in}}%
\pgfpathcurveto{\pgfqpoint{0.501417in}{0.464583in}}{\pgfqpoint{0.497026in}{0.475182in}}{\pgfqpoint{0.489213in}{0.482996in}}%
\pgfpathcurveto{\pgfqpoint{0.481399in}{0.490810in}}{\pgfqpoint{0.470800in}{0.495200in}}{\pgfqpoint{0.459750in}{0.495200in}}%
\pgfpathcurveto{\pgfqpoint{0.448700in}{0.495200in}}{\pgfqpoint{0.438101in}{0.490810in}}{\pgfqpoint{0.430287in}{0.482996in}}%
\pgfpathcurveto{\pgfqpoint{0.422474in}{0.475182in}}{\pgfqpoint{0.418083in}{0.464583in}}{\pgfqpoint{0.418083in}{0.453533in}}%
\pgfpathcurveto{\pgfqpoint{0.418083in}{0.442483in}}{\pgfqpoint{0.422474in}{0.431884in}}{\pgfqpoint{0.430287in}{0.424071in}}%
\pgfpathcurveto{\pgfqpoint{0.438101in}{0.416257in}}{\pgfqpoint{0.448700in}{0.411867in}}{\pgfqpoint{0.459750in}{0.411867in}}%
\pgfpathclose%
\pgfusepath{stroke,fill}%
\end{pgfscope}%
\begin{pgfscope}%
\pgfpathrectangle{\pgfqpoint{0.375000in}{0.330000in}}{\pgfqpoint{2.325000in}{2.310000in}}%
\pgfusepath{clip}%
\pgfsetbuttcap%
\pgfsetroundjoin%
\definecolor{currentfill}{rgb}{0.000000,0.000000,0.000000}%
\pgfsetfillcolor{currentfill}%
\pgfsetlinewidth{1.003750pt}%
\definecolor{currentstroke}{rgb}{0.000000,0.000000,0.000000}%
\pgfsetstrokecolor{currentstroke}%
\pgfsetdash{}{0pt}%
\pgfpathmoveto{\pgfqpoint{0.459750in}{0.417935in}}%
\pgfpathcurveto{\pgfqpoint{0.470800in}{0.417935in}}{\pgfqpoint{0.481399in}{0.422325in}}{\pgfqpoint{0.489213in}{0.430139in}}%
\pgfpathcurveto{\pgfqpoint{0.497026in}{0.437952in}}{\pgfqpoint{0.501417in}{0.448551in}}{\pgfqpoint{0.501417in}{0.459601in}}%
\pgfpathcurveto{\pgfqpoint{0.501417in}{0.470652in}}{\pgfqpoint{0.497026in}{0.481251in}}{\pgfqpoint{0.489213in}{0.489064in}}%
\pgfpathcurveto{\pgfqpoint{0.481399in}{0.496878in}}{\pgfqpoint{0.470800in}{0.501268in}}{\pgfqpoint{0.459750in}{0.501268in}}%
\pgfpathcurveto{\pgfqpoint{0.448700in}{0.501268in}}{\pgfqpoint{0.438101in}{0.496878in}}{\pgfqpoint{0.430287in}{0.489064in}}%
\pgfpathcurveto{\pgfqpoint{0.422474in}{0.481251in}}{\pgfqpoint{0.418083in}{0.470652in}}{\pgfqpoint{0.418083in}{0.459601in}}%
\pgfpathcurveto{\pgfqpoint{0.418083in}{0.448551in}}{\pgfqpoint{0.422474in}{0.437952in}}{\pgfqpoint{0.430287in}{0.430139in}}%
\pgfpathcurveto{\pgfqpoint{0.438101in}{0.422325in}}{\pgfqpoint{0.448700in}{0.417935in}}{\pgfqpoint{0.459750in}{0.417935in}}%
\pgfpathclose%
\pgfusepath{stroke,fill}%
\end{pgfscope}%
\begin{pgfscope}%
\pgfpathrectangle{\pgfqpoint{0.375000in}{0.330000in}}{\pgfqpoint{2.325000in}{2.310000in}}%
\pgfusepath{clip}%
\pgfsetbuttcap%
\pgfsetroundjoin%
\definecolor{currentfill}{rgb}{0.000000,0.000000,0.000000}%
\pgfsetfillcolor{currentfill}%
\pgfsetlinewidth{1.003750pt}%
\definecolor{currentstroke}{rgb}{0.000000,0.000000,0.000000}%
\pgfsetstrokecolor{currentstroke}%
\pgfsetdash{}{0pt}%
\pgfpathmoveto{\pgfqpoint{0.459750in}{0.417935in}}%
\pgfpathcurveto{\pgfqpoint{0.470800in}{0.417935in}}{\pgfqpoint{0.481399in}{0.422325in}}{\pgfqpoint{0.489213in}{0.430139in}}%
\pgfpathcurveto{\pgfqpoint{0.497026in}{0.437952in}}{\pgfqpoint{0.501417in}{0.448551in}}{\pgfqpoint{0.501417in}{0.459601in}}%
\pgfpathcurveto{\pgfqpoint{0.501417in}{0.470652in}}{\pgfqpoint{0.497026in}{0.481251in}}{\pgfqpoint{0.489213in}{0.489064in}}%
\pgfpathcurveto{\pgfqpoint{0.481399in}{0.496878in}}{\pgfqpoint{0.470800in}{0.501268in}}{\pgfqpoint{0.459750in}{0.501268in}}%
\pgfpathcurveto{\pgfqpoint{0.448700in}{0.501268in}}{\pgfqpoint{0.438101in}{0.496878in}}{\pgfqpoint{0.430287in}{0.489064in}}%
\pgfpathcurveto{\pgfqpoint{0.422474in}{0.481251in}}{\pgfqpoint{0.418083in}{0.470652in}}{\pgfqpoint{0.418083in}{0.459601in}}%
\pgfpathcurveto{\pgfqpoint{0.418083in}{0.448551in}}{\pgfqpoint{0.422474in}{0.437952in}}{\pgfqpoint{0.430287in}{0.430139in}}%
\pgfpathcurveto{\pgfqpoint{0.438101in}{0.422325in}}{\pgfqpoint{0.448700in}{0.417935in}}{\pgfqpoint{0.459750in}{0.417935in}}%
\pgfpathclose%
\pgfusepath{stroke,fill}%
\end{pgfscope}%
\begin{pgfscope}%
\pgfpathrectangle{\pgfqpoint{0.375000in}{0.330000in}}{\pgfqpoint{2.325000in}{2.310000in}}%
\pgfusepath{clip}%
\pgfsetbuttcap%
\pgfsetroundjoin%
\definecolor{currentfill}{rgb}{0.000000,0.000000,0.000000}%
\pgfsetfillcolor{currentfill}%
\pgfsetlinewidth{1.003750pt}%
\definecolor{currentstroke}{rgb}{0.000000,0.000000,0.000000}%
\pgfsetstrokecolor{currentstroke}%
\pgfsetdash{}{0pt}%
\pgfpathmoveto{\pgfqpoint{0.459750in}{0.411867in}}%
\pgfpathcurveto{\pgfqpoint{0.470800in}{0.411867in}}{\pgfqpoint{0.481399in}{0.416257in}}{\pgfqpoint{0.489213in}{0.424071in}}%
\pgfpathcurveto{\pgfqpoint{0.497026in}{0.431884in}}{\pgfqpoint{0.501417in}{0.442483in}}{\pgfqpoint{0.501417in}{0.453533in}}%
\pgfpathcurveto{\pgfqpoint{0.501417in}{0.464583in}}{\pgfqpoint{0.497026in}{0.475182in}}{\pgfqpoint{0.489213in}{0.482996in}}%
\pgfpathcurveto{\pgfqpoint{0.481399in}{0.490810in}}{\pgfqpoint{0.470800in}{0.495200in}}{\pgfqpoint{0.459750in}{0.495200in}}%
\pgfpathcurveto{\pgfqpoint{0.448700in}{0.495200in}}{\pgfqpoint{0.438101in}{0.490810in}}{\pgfqpoint{0.430287in}{0.482996in}}%
\pgfpathcurveto{\pgfqpoint{0.422474in}{0.475182in}}{\pgfqpoint{0.418083in}{0.464583in}}{\pgfqpoint{0.418083in}{0.453533in}}%
\pgfpathcurveto{\pgfqpoint{0.418083in}{0.442483in}}{\pgfqpoint{0.422474in}{0.431884in}}{\pgfqpoint{0.430287in}{0.424071in}}%
\pgfpathcurveto{\pgfqpoint{0.438101in}{0.416257in}}{\pgfqpoint{0.448700in}{0.411867in}}{\pgfqpoint{0.459750in}{0.411867in}}%
\pgfpathclose%
\pgfusepath{stroke,fill}%
\end{pgfscope}%
\begin{pgfscope}%
\pgfpathrectangle{\pgfqpoint{0.375000in}{0.330000in}}{\pgfqpoint{2.325000in}{2.310000in}}%
\pgfusepath{clip}%
\pgfsetbuttcap%
\pgfsetroundjoin%
\definecolor{currentfill}{rgb}{0.000000,0.000000,0.000000}%
\pgfsetfillcolor{currentfill}%
\pgfsetlinewidth{1.003750pt}%
\definecolor{currentstroke}{rgb}{0.000000,0.000000,0.000000}%
\pgfsetstrokecolor{currentstroke}%
\pgfsetdash{}{0pt}%
\pgfpathmoveto{\pgfqpoint{0.459750in}{0.417935in}}%
\pgfpathcurveto{\pgfqpoint{0.470800in}{0.417935in}}{\pgfqpoint{0.481399in}{0.422325in}}{\pgfqpoint{0.489213in}{0.430139in}}%
\pgfpathcurveto{\pgfqpoint{0.497026in}{0.437952in}}{\pgfqpoint{0.501417in}{0.448551in}}{\pgfqpoint{0.501417in}{0.459601in}}%
\pgfpathcurveto{\pgfqpoint{0.501417in}{0.470652in}}{\pgfqpoint{0.497026in}{0.481251in}}{\pgfqpoint{0.489213in}{0.489064in}}%
\pgfpathcurveto{\pgfqpoint{0.481399in}{0.496878in}}{\pgfqpoint{0.470800in}{0.501268in}}{\pgfqpoint{0.459750in}{0.501268in}}%
\pgfpathcurveto{\pgfqpoint{0.448700in}{0.501268in}}{\pgfqpoint{0.438101in}{0.496878in}}{\pgfqpoint{0.430287in}{0.489064in}}%
\pgfpathcurveto{\pgfqpoint{0.422474in}{0.481251in}}{\pgfqpoint{0.418083in}{0.470652in}}{\pgfqpoint{0.418083in}{0.459601in}}%
\pgfpathcurveto{\pgfqpoint{0.418083in}{0.448551in}}{\pgfqpoint{0.422474in}{0.437952in}}{\pgfqpoint{0.430287in}{0.430139in}}%
\pgfpathcurveto{\pgfqpoint{0.438101in}{0.422325in}}{\pgfqpoint{0.448700in}{0.417935in}}{\pgfqpoint{0.459750in}{0.417935in}}%
\pgfpathclose%
\pgfusepath{stroke,fill}%
\end{pgfscope}%
\begin{pgfscope}%
\pgfpathrectangle{\pgfqpoint{0.375000in}{0.330000in}}{\pgfqpoint{2.325000in}{2.310000in}}%
\pgfusepath{clip}%
\pgfsetbuttcap%
\pgfsetroundjoin%
\definecolor{currentfill}{rgb}{0.000000,0.000000,0.000000}%
\pgfsetfillcolor{currentfill}%
\pgfsetlinewidth{1.003750pt}%
\definecolor{currentstroke}{rgb}{0.000000,0.000000,0.000000}%
\pgfsetstrokecolor{currentstroke}%
\pgfsetdash{}{0pt}%
\pgfpathmoveto{\pgfqpoint{0.459750in}{0.430071in}}%
\pgfpathcurveto{\pgfqpoint{0.470800in}{0.430071in}}{\pgfqpoint{0.481399in}{0.434461in}}{\pgfqpoint{0.489213in}{0.442275in}}%
\pgfpathcurveto{\pgfqpoint{0.497026in}{0.450088in}}{\pgfqpoint{0.501417in}{0.460688in}}{\pgfqpoint{0.501417in}{0.471738in}}%
\pgfpathcurveto{\pgfqpoint{0.501417in}{0.482788in}}{\pgfqpoint{0.497026in}{0.493387in}}{\pgfqpoint{0.489213in}{0.501200in}}%
\pgfpathcurveto{\pgfqpoint{0.481399in}{0.509014in}}{\pgfqpoint{0.470800in}{0.513404in}}{\pgfqpoint{0.459750in}{0.513404in}}%
\pgfpathcurveto{\pgfqpoint{0.448700in}{0.513404in}}{\pgfqpoint{0.438101in}{0.509014in}}{\pgfqpoint{0.430287in}{0.501200in}}%
\pgfpathcurveto{\pgfqpoint{0.422474in}{0.493387in}}{\pgfqpoint{0.418083in}{0.482788in}}{\pgfqpoint{0.418083in}{0.471738in}}%
\pgfpathcurveto{\pgfqpoint{0.418083in}{0.460688in}}{\pgfqpoint{0.422474in}{0.450088in}}{\pgfqpoint{0.430287in}{0.442275in}}%
\pgfpathcurveto{\pgfqpoint{0.438101in}{0.434461in}}{\pgfqpoint{0.448700in}{0.430071in}}{\pgfqpoint{0.459750in}{0.430071in}}%
\pgfpathclose%
\pgfusepath{stroke,fill}%
\end{pgfscope}%
\begin{pgfscope}%
\pgfpathrectangle{\pgfqpoint{0.375000in}{0.330000in}}{\pgfqpoint{2.325000in}{2.310000in}}%
\pgfusepath{clip}%
\pgfsetbuttcap%
\pgfsetroundjoin%
\definecolor{currentfill}{rgb}{0.000000,0.000000,0.000000}%
\pgfsetfillcolor{currentfill}%
\pgfsetlinewidth{1.003750pt}%
\definecolor{currentstroke}{rgb}{0.000000,0.000000,0.000000}%
\pgfsetstrokecolor{currentstroke}%
\pgfsetdash{}{0pt}%
\pgfpathmoveto{\pgfqpoint{0.459750in}{0.411867in}}%
\pgfpathcurveto{\pgfqpoint{0.470800in}{0.411867in}}{\pgfqpoint{0.481399in}{0.416257in}}{\pgfqpoint{0.489213in}{0.424071in}}%
\pgfpathcurveto{\pgfqpoint{0.497026in}{0.431884in}}{\pgfqpoint{0.501417in}{0.442483in}}{\pgfqpoint{0.501417in}{0.453533in}}%
\pgfpathcurveto{\pgfqpoint{0.501417in}{0.464583in}}{\pgfqpoint{0.497026in}{0.475182in}}{\pgfqpoint{0.489213in}{0.482996in}}%
\pgfpathcurveto{\pgfqpoint{0.481399in}{0.490810in}}{\pgfqpoint{0.470800in}{0.495200in}}{\pgfqpoint{0.459750in}{0.495200in}}%
\pgfpathcurveto{\pgfqpoint{0.448700in}{0.495200in}}{\pgfqpoint{0.438101in}{0.490810in}}{\pgfqpoint{0.430287in}{0.482996in}}%
\pgfpathcurveto{\pgfqpoint{0.422474in}{0.475182in}}{\pgfqpoint{0.418083in}{0.464583in}}{\pgfqpoint{0.418083in}{0.453533in}}%
\pgfpathcurveto{\pgfqpoint{0.418083in}{0.442483in}}{\pgfqpoint{0.422474in}{0.431884in}}{\pgfqpoint{0.430287in}{0.424071in}}%
\pgfpathcurveto{\pgfqpoint{0.438101in}{0.416257in}}{\pgfqpoint{0.448700in}{0.411867in}}{\pgfqpoint{0.459750in}{0.411867in}}%
\pgfpathclose%
\pgfusepath{stroke,fill}%
\end{pgfscope}%
\begin{pgfscope}%
\pgfpathrectangle{\pgfqpoint{0.375000in}{0.330000in}}{\pgfqpoint{2.325000in}{2.310000in}}%
\pgfusepath{clip}%
\pgfsetbuttcap%
\pgfsetroundjoin%
\definecolor{currentfill}{rgb}{0.000000,0.000000,0.000000}%
\pgfsetfillcolor{currentfill}%
\pgfsetlinewidth{1.003750pt}%
\definecolor{currentstroke}{rgb}{0.000000,0.000000,0.000000}%
\pgfsetstrokecolor{currentstroke}%
\pgfsetdash{}{0pt}%
\pgfpathmoveto{\pgfqpoint{0.459750in}{0.399730in}}%
\pgfpathcurveto{\pgfqpoint{0.470800in}{0.399730in}}{\pgfqpoint{0.481399in}{0.404121in}}{\pgfqpoint{0.489213in}{0.411934in}}%
\pgfpathcurveto{\pgfqpoint{0.497026in}{0.419748in}}{\pgfqpoint{0.501417in}{0.430347in}}{\pgfqpoint{0.501417in}{0.441397in}}%
\pgfpathcurveto{\pgfqpoint{0.501417in}{0.452447in}}{\pgfqpoint{0.497026in}{0.463046in}}{\pgfqpoint{0.489213in}{0.470860in}}%
\pgfpathcurveto{\pgfqpoint{0.481399in}{0.478674in}}{\pgfqpoint{0.470800in}{0.483064in}}{\pgfqpoint{0.459750in}{0.483064in}}%
\pgfpathcurveto{\pgfqpoint{0.448700in}{0.483064in}}{\pgfqpoint{0.438101in}{0.478674in}}{\pgfqpoint{0.430287in}{0.470860in}}%
\pgfpathcurveto{\pgfqpoint{0.422474in}{0.463046in}}{\pgfqpoint{0.418083in}{0.452447in}}{\pgfqpoint{0.418083in}{0.441397in}}%
\pgfpathcurveto{\pgfqpoint{0.418083in}{0.430347in}}{\pgfqpoint{0.422474in}{0.419748in}}{\pgfqpoint{0.430287in}{0.411934in}}%
\pgfpathcurveto{\pgfqpoint{0.438101in}{0.404121in}}{\pgfqpoint{0.448700in}{0.399730in}}{\pgfqpoint{0.459750in}{0.399730in}}%
\pgfpathclose%
\pgfusepath{stroke,fill}%
\end{pgfscope}%
\begin{pgfscope}%
\pgfpathrectangle{\pgfqpoint{0.375000in}{0.330000in}}{\pgfqpoint{2.325000in}{2.310000in}}%
\pgfusepath{clip}%
\pgfsetbuttcap%
\pgfsetroundjoin%
\definecolor{currentfill}{rgb}{0.000000,0.000000,0.000000}%
\pgfsetfillcolor{currentfill}%
\pgfsetlinewidth{1.003750pt}%
\definecolor{currentstroke}{rgb}{0.000000,0.000000,0.000000}%
\pgfsetstrokecolor{currentstroke}%
\pgfsetdash{}{0pt}%
\pgfpathmoveto{\pgfqpoint{0.459750in}{0.411867in}}%
\pgfpathcurveto{\pgfqpoint{0.470800in}{0.411867in}}{\pgfqpoint{0.481399in}{0.416257in}}{\pgfqpoint{0.489213in}{0.424071in}}%
\pgfpathcurveto{\pgfqpoint{0.497026in}{0.431884in}}{\pgfqpoint{0.501417in}{0.442483in}}{\pgfqpoint{0.501417in}{0.453533in}}%
\pgfpathcurveto{\pgfqpoint{0.501417in}{0.464583in}}{\pgfqpoint{0.497026in}{0.475182in}}{\pgfqpoint{0.489213in}{0.482996in}}%
\pgfpathcurveto{\pgfqpoint{0.481399in}{0.490810in}}{\pgfqpoint{0.470800in}{0.495200in}}{\pgfqpoint{0.459750in}{0.495200in}}%
\pgfpathcurveto{\pgfqpoint{0.448700in}{0.495200in}}{\pgfqpoint{0.438101in}{0.490810in}}{\pgfqpoint{0.430287in}{0.482996in}}%
\pgfpathcurveto{\pgfqpoint{0.422474in}{0.475182in}}{\pgfqpoint{0.418083in}{0.464583in}}{\pgfqpoint{0.418083in}{0.453533in}}%
\pgfpathcurveto{\pgfqpoint{0.418083in}{0.442483in}}{\pgfqpoint{0.422474in}{0.431884in}}{\pgfqpoint{0.430287in}{0.424071in}}%
\pgfpathcurveto{\pgfqpoint{0.438101in}{0.416257in}}{\pgfqpoint{0.448700in}{0.411867in}}{\pgfqpoint{0.459750in}{0.411867in}}%
\pgfpathclose%
\pgfusepath{stroke,fill}%
\end{pgfscope}%
\begin{pgfscope}%
\pgfpathrectangle{\pgfqpoint{0.375000in}{0.330000in}}{\pgfqpoint{2.325000in}{2.310000in}}%
\pgfusepath{clip}%
\pgfsetbuttcap%
\pgfsetroundjoin%
\definecolor{currentfill}{rgb}{0.000000,0.000000,0.000000}%
\pgfsetfillcolor{currentfill}%
\pgfsetlinewidth{1.003750pt}%
\definecolor{currentstroke}{rgb}{0.000000,0.000000,0.000000}%
\pgfsetstrokecolor{currentstroke}%
\pgfsetdash{}{0pt}%
\pgfpathmoveto{\pgfqpoint{0.459750in}{0.424003in}}%
\pgfpathcurveto{\pgfqpoint{0.470800in}{0.424003in}}{\pgfqpoint{0.481399in}{0.428393in}}{\pgfqpoint{0.489213in}{0.436207in}}%
\pgfpathcurveto{\pgfqpoint{0.497026in}{0.444020in}}{\pgfqpoint{0.501417in}{0.454619in}}{\pgfqpoint{0.501417in}{0.465670in}}%
\pgfpathcurveto{\pgfqpoint{0.501417in}{0.476720in}}{\pgfqpoint{0.497026in}{0.487319in}}{\pgfqpoint{0.489213in}{0.495132in}}%
\pgfpathcurveto{\pgfqpoint{0.481399in}{0.502946in}}{\pgfqpoint{0.470800in}{0.507336in}}{\pgfqpoint{0.459750in}{0.507336in}}%
\pgfpathcurveto{\pgfqpoint{0.448700in}{0.507336in}}{\pgfqpoint{0.438101in}{0.502946in}}{\pgfqpoint{0.430287in}{0.495132in}}%
\pgfpathcurveto{\pgfqpoint{0.422474in}{0.487319in}}{\pgfqpoint{0.418083in}{0.476720in}}{\pgfqpoint{0.418083in}{0.465670in}}%
\pgfpathcurveto{\pgfqpoint{0.418083in}{0.454619in}}{\pgfqpoint{0.422474in}{0.444020in}}{\pgfqpoint{0.430287in}{0.436207in}}%
\pgfpathcurveto{\pgfqpoint{0.438101in}{0.428393in}}{\pgfqpoint{0.448700in}{0.424003in}}{\pgfqpoint{0.459750in}{0.424003in}}%
\pgfpathclose%
\pgfusepath{stroke,fill}%
\end{pgfscope}%
\begin{pgfscope}%
\pgfpathrectangle{\pgfqpoint{0.375000in}{0.330000in}}{\pgfqpoint{2.325000in}{2.310000in}}%
\pgfusepath{clip}%
\pgfsetbuttcap%
\pgfsetroundjoin%
\definecolor{currentfill}{rgb}{0.000000,0.000000,0.000000}%
\pgfsetfillcolor{currentfill}%
\pgfsetlinewidth{1.003750pt}%
\definecolor{currentstroke}{rgb}{0.000000,0.000000,0.000000}%
\pgfsetstrokecolor{currentstroke}%
\pgfsetdash{}{0pt}%
\pgfpathmoveto{\pgfqpoint{0.459750in}{0.417935in}}%
\pgfpathcurveto{\pgfqpoint{0.470800in}{0.417935in}}{\pgfqpoint{0.481399in}{0.422325in}}{\pgfqpoint{0.489213in}{0.430139in}}%
\pgfpathcurveto{\pgfqpoint{0.497026in}{0.437952in}}{\pgfqpoint{0.501417in}{0.448551in}}{\pgfqpoint{0.501417in}{0.459601in}}%
\pgfpathcurveto{\pgfqpoint{0.501417in}{0.470652in}}{\pgfqpoint{0.497026in}{0.481251in}}{\pgfqpoint{0.489213in}{0.489064in}}%
\pgfpathcurveto{\pgfqpoint{0.481399in}{0.496878in}}{\pgfqpoint{0.470800in}{0.501268in}}{\pgfqpoint{0.459750in}{0.501268in}}%
\pgfpathcurveto{\pgfqpoint{0.448700in}{0.501268in}}{\pgfqpoint{0.438101in}{0.496878in}}{\pgfqpoint{0.430287in}{0.489064in}}%
\pgfpathcurveto{\pgfqpoint{0.422474in}{0.481251in}}{\pgfqpoint{0.418083in}{0.470652in}}{\pgfqpoint{0.418083in}{0.459601in}}%
\pgfpathcurveto{\pgfqpoint{0.418083in}{0.448551in}}{\pgfqpoint{0.422474in}{0.437952in}}{\pgfqpoint{0.430287in}{0.430139in}}%
\pgfpathcurveto{\pgfqpoint{0.438101in}{0.422325in}}{\pgfqpoint{0.448700in}{0.417935in}}{\pgfqpoint{0.459750in}{0.417935in}}%
\pgfpathclose%
\pgfusepath{stroke,fill}%
\end{pgfscope}%
\begin{pgfscope}%
\pgfpathrectangle{\pgfqpoint{0.375000in}{0.330000in}}{\pgfqpoint{2.325000in}{2.310000in}}%
\pgfusepath{clip}%
\pgfsetbuttcap%
\pgfsetroundjoin%
\definecolor{currentfill}{rgb}{0.000000,0.000000,0.000000}%
\pgfsetfillcolor{currentfill}%
\pgfsetlinewidth{1.003750pt}%
\definecolor{currentstroke}{rgb}{0.000000,0.000000,0.000000}%
\pgfsetstrokecolor{currentstroke}%
\pgfsetdash{}{0pt}%
\pgfpathmoveto{\pgfqpoint{0.459750in}{0.424003in}}%
\pgfpathcurveto{\pgfqpoint{0.470800in}{0.424003in}}{\pgfqpoint{0.481399in}{0.428393in}}{\pgfqpoint{0.489213in}{0.436207in}}%
\pgfpathcurveto{\pgfqpoint{0.497026in}{0.444020in}}{\pgfqpoint{0.501417in}{0.454619in}}{\pgfqpoint{0.501417in}{0.465670in}}%
\pgfpathcurveto{\pgfqpoint{0.501417in}{0.476720in}}{\pgfqpoint{0.497026in}{0.487319in}}{\pgfqpoint{0.489213in}{0.495132in}}%
\pgfpathcurveto{\pgfqpoint{0.481399in}{0.502946in}}{\pgfqpoint{0.470800in}{0.507336in}}{\pgfqpoint{0.459750in}{0.507336in}}%
\pgfpathcurveto{\pgfqpoint{0.448700in}{0.507336in}}{\pgfqpoint{0.438101in}{0.502946in}}{\pgfqpoint{0.430287in}{0.495132in}}%
\pgfpathcurveto{\pgfqpoint{0.422474in}{0.487319in}}{\pgfqpoint{0.418083in}{0.476720in}}{\pgfqpoint{0.418083in}{0.465670in}}%
\pgfpathcurveto{\pgfqpoint{0.418083in}{0.454619in}}{\pgfqpoint{0.422474in}{0.444020in}}{\pgfqpoint{0.430287in}{0.436207in}}%
\pgfpathcurveto{\pgfqpoint{0.438101in}{0.428393in}}{\pgfqpoint{0.448700in}{0.424003in}}{\pgfqpoint{0.459750in}{0.424003in}}%
\pgfpathclose%
\pgfusepath{stroke,fill}%
\end{pgfscope}%
\begin{pgfscope}%
\pgfpathrectangle{\pgfqpoint{0.375000in}{0.330000in}}{\pgfqpoint{2.325000in}{2.310000in}}%
\pgfusepath{clip}%
\pgfsetbuttcap%
\pgfsetroundjoin%
\definecolor{currentfill}{rgb}{0.000000,0.000000,0.000000}%
\pgfsetfillcolor{currentfill}%
\pgfsetlinewidth{1.003750pt}%
\definecolor{currentstroke}{rgb}{0.000000,0.000000,0.000000}%
\pgfsetstrokecolor{currentstroke}%
\pgfsetdash{}{0pt}%
\pgfpathmoveto{\pgfqpoint{0.459750in}{0.405799in}}%
\pgfpathcurveto{\pgfqpoint{0.470800in}{0.405799in}}{\pgfqpoint{0.481399in}{0.410189in}}{\pgfqpoint{0.489213in}{0.418002in}}%
\pgfpathcurveto{\pgfqpoint{0.497026in}{0.425816in}}{\pgfqpoint{0.501417in}{0.436415in}}{\pgfqpoint{0.501417in}{0.447465in}}%
\pgfpathcurveto{\pgfqpoint{0.501417in}{0.458515in}}{\pgfqpoint{0.497026in}{0.469114in}}{\pgfqpoint{0.489213in}{0.476928in}}%
\pgfpathcurveto{\pgfqpoint{0.481399in}{0.484742in}}{\pgfqpoint{0.470800in}{0.489132in}}{\pgfqpoint{0.459750in}{0.489132in}}%
\pgfpathcurveto{\pgfqpoint{0.448700in}{0.489132in}}{\pgfqpoint{0.438101in}{0.484742in}}{\pgfqpoint{0.430287in}{0.476928in}}%
\pgfpathcurveto{\pgfqpoint{0.422474in}{0.469114in}}{\pgfqpoint{0.418083in}{0.458515in}}{\pgfqpoint{0.418083in}{0.447465in}}%
\pgfpathcurveto{\pgfqpoint{0.418083in}{0.436415in}}{\pgfqpoint{0.422474in}{0.425816in}}{\pgfqpoint{0.430287in}{0.418002in}}%
\pgfpathcurveto{\pgfqpoint{0.438101in}{0.410189in}}{\pgfqpoint{0.448700in}{0.405799in}}{\pgfqpoint{0.459750in}{0.405799in}}%
\pgfpathclose%
\pgfusepath{stroke,fill}%
\end{pgfscope}%
\begin{pgfscope}%
\pgfpathrectangle{\pgfqpoint{0.375000in}{0.330000in}}{\pgfqpoint{2.325000in}{2.310000in}}%
\pgfusepath{clip}%
\pgfsetbuttcap%
\pgfsetroundjoin%
\definecolor{currentfill}{rgb}{0.000000,0.000000,0.000000}%
\pgfsetfillcolor{currentfill}%
\pgfsetlinewidth{1.003750pt}%
\definecolor{currentstroke}{rgb}{0.000000,0.000000,0.000000}%
\pgfsetstrokecolor{currentstroke}%
\pgfsetdash{}{0pt}%
\pgfpathmoveto{\pgfqpoint{0.459750in}{0.417935in}}%
\pgfpathcurveto{\pgfqpoint{0.470800in}{0.417935in}}{\pgfqpoint{0.481399in}{0.422325in}}{\pgfqpoint{0.489213in}{0.430139in}}%
\pgfpathcurveto{\pgfqpoint{0.497026in}{0.437952in}}{\pgfqpoint{0.501417in}{0.448551in}}{\pgfqpoint{0.501417in}{0.459601in}}%
\pgfpathcurveto{\pgfqpoint{0.501417in}{0.470652in}}{\pgfqpoint{0.497026in}{0.481251in}}{\pgfqpoint{0.489213in}{0.489064in}}%
\pgfpathcurveto{\pgfqpoint{0.481399in}{0.496878in}}{\pgfqpoint{0.470800in}{0.501268in}}{\pgfqpoint{0.459750in}{0.501268in}}%
\pgfpathcurveto{\pgfqpoint{0.448700in}{0.501268in}}{\pgfqpoint{0.438101in}{0.496878in}}{\pgfqpoint{0.430287in}{0.489064in}}%
\pgfpathcurveto{\pgfqpoint{0.422474in}{0.481251in}}{\pgfqpoint{0.418083in}{0.470652in}}{\pgfqpoint{0.418083in}{0.459601in}}%
\pgfpathcurveto{\pgfqpoint{0.418083in}{0.448551in}}{\pgfqpoint{0.422474in}{0.437952in}}{\pgfqpoint{0.430287in}{0.430139in}}%
\pgfpathcurveto{\pgfqpoint{0.438101in}{0.422325in}}{\pgfqpoint{0.448700in}{0.417935in}}{\pgfqpoint{0.459750in}{0.417935in}}%
\pgfpathclose%
\pgfusepath{stroke,fill}%
\end{pgfscope}%
\begin{pgfscope}%
\pgfpathrectangle{\pgfqpoint{0.375000in}{0.330000in}}{\pgfqpoint{2.325000in}{2.310000in}}%
\pgfusepath{clip}%
\pgfsetbuttcap%
\pgfsetroundjoin%
\definecolor{currentfill}{rgb}{0.000000,0.000000,0.000000}%
\pgfsetfillcolor{currentfill}%
\pgfsetlinewidth{1.003750pt}%
\definecolor{currentstroke}{rgb}{0.000000,0.000000,0.000000}%
\pgfsetstrokecolor{currentstroke}%
\pgfsetdash{}{0pt}%
\pgfpathmoveto{\pgfqpoint{0.459750in}{0.417935in}}%
\pgfpathcurveto{\pgfqpoint{0.470800in}{0.417935in}}{\pgfqpoint{0.481399in}{0.422325in}}{\pgfqpoint{0.489213in}{0.430139in}}%
\pgfpathcurveto{\pgfqpoint{0.497026in}{0.437952in}}{\pgfqpoint{0.501417in}{0.448551in}}{\pgfqpoint{0.501417in}{0.459601in}}%
\pgfpathcurveto{\pgfqpoint{0.501417in}{0.470652in}}{\pgfqpoint{0.497026in}{0.481251in}}{\pgfqpoint{0.489213in}{0.489064in}}%
\pgfpathcurveto{\pgfqpoint{0.481399in}{0.496878in}}{\pgfqpoint{0.470800in}{0.501268in}}{\pgfqpoint{0.459750in}{0.501268in}}%
\pgfpathcurveto{\pgfqpoint{0.448700in}{0.501268in}}{\pgfqpoint{0.438101in}{0.496878in}}{\pgfqpoint{0.430287in}{0.489064in}}%
\pgfpathcurveto{\pgfqpoint{0.422474in}{0.481251in}}{\pgfqpoint{0.418083in}{0.470652in}}{\pgfqpoint{0.418083in}{0.459601in}}%
\pgfpathcurveto{\pgfqpoint{0.418083in}{0.448551in}}{\pgfqpoint{0.422474in}{0.437952in}}{\pgfqpoint{0.430287in}{0.430139in}}%
\pgfpathcurveto{\pgfqpoint{0.438101in}{0.422325in}}{\pgfqpoint{0.448700in}{0.417935in}}{\pgfqpoint{0.459750in}{0.417935in}}%
\pgfpathclose%
\pgfusepath{stroke,fill}%
\end{pgfscope}%
\begin{pgfscope}%
\pgfpathrectangle{\pgfqpoint{0.375000in}{0.330000in}}{\pgfqpoint{2.325000in}{2.310000in}}%
\pgfusepath{clip}%
\pgfsetbuttcap%
\pgfsetroundjoin%
\definecolor{currentfill}{rgb}{0.000000,0.000000,0.000000}%
\pgfsetfillcolor{currentfill}%
\pgfsetlinewidth{1.003750pt}%
\definecolor{currentstroke}{rgb}{0.000000,0.000000,0.000000}%
\pgfsetstrokecolor{currentstroke}%
\pgfsetdash{}{0pt}%
\pgfpathmoveto{\pgfqpoint{0.459750in}{0.411867in}}%
\pgfpathcurveto{\pgfqpoint{0.470800in}{0.411867in}}{\pgfqpoint{0.481399in}{0.416257in}}{\pgfqpoint{0.489213in}{0.424071in}}%
\pgfpathcurveto{\pgfqpoint{0.497026in}{0.431884in}}{\pgfqpoint{0.501417in}{0.442483in}}{\pgfqpoint{0.501417in}{0.453533in}}%
\pgfpathcurveto{\pgfqpoint{0.501417in}{0.464583in}}{\pgfqpoint{0.497026in}{0.475182in}}{\pgfqpoint{0.489213in}{0.482996in}}%
\pgfpathcurveto{\pgfqpoint{0.481399in}{0.490810in}}{\pgfqpoint{0.470800in}{0.495200in}}{\pgfqpoint{0.459750in}{0.495200in}}%
\pgfpathcurveto{\pgfqpoint{0.448700in}{0.495200in}}{\pgfqpoint{0.438101in}{0.490810in}}{\pgfqpoint{0.430287in}{0.482996in}}%
\pgfpathcurveto{\pgfqpoint{0.422474in}{0.475182in}}{\pgfqpoint{0.418083in}{0.464583in}}{\pgfqpoint{0.418083in}{0.453533in}}%
\pgfpathcurveto{\pgfqpoint{0.418083in}{0.442483in}}{\pgfqpoint{0.422474in}{0.431884in}}{\pgfqpoint{0.430287in}{0.424071in}}%
\pgfpathcurveto{\pgfqpoint{0.438101in}{0.416257in}}{\pgfqpoint{0.448700in}{0.411867in}}{\pgfqpoint{0.459750in}{0.411867in}}%
\pgfpathclose%
\pgfusepath{stroke,fill}%
\end{pgfscope}%
\begin{pgfscope}%
\pgfpathrectangle{\pgfqpoint{0.375000in}{0.330000in}}{\pgfqpoint{2.325000in}{2.310000in}}%
\pgfusepath{clip}%
\pgfsetbuttcap%
\pgfsetroundjoin%
\definecolor{currentfill}{rgb}{0.000000,0.000000,0.000000}%
\pgfsetfillcolor{currentfill}%
\pgfsetlinewidth{1.003750pt}%
\definecolor{currentstroke}{rgb}{0.000000,0.000000,0.000000}%
\pgfsetstrokecolor{currentstroke}%
\pgfsetdash{}{0pt}%
\pgfpathmoveto{\pgfqpoint{0.459750in}{0.411867in}}%
\pgfpathcurveto{\pgfqpoint{0.470800in}{0.411867in}}{\pgfqpoint{0.481399in}{0.416257in}}{\pgfqpoint{0.489213in}{0.424071in}}%
\pgfpathcurveto{\pgfqpoint{0.497026in}{0.431884in}}{\pgfqpoint{0.501417in}{0.442483in}}{\pgfqpoint{0.501417in}{0.453533in}}%
\pgfpathcurveto{\pgfqpoint{0.501417in}{0.464583in}}{\pgfqpoint{0.497026in}{0.475182in}}{\pgfqpoint{0.489213in}{0.482996in}}%
\pgfpathcurveto{\pgfqpoint{0.481399in}{0.490810in}}{\pgfqpoint{0.470800in}{0.495200in}}{\pgfqpoint{0.459750in}{0.495200in}}%
\pgfpathcurveto{\pgfqpoint{0.448700in}{0.495200in}}{\pgfqpoint{0.438101in}{0.490810in}}{\pgfqpoint{0.430287in}{0.482996in}}%
\pgfpathcurveto{\pgfqpoint{0.422474in}{0.475182in}}{\pgfqpoint{0.418083in}{0.464583in}}{\pgfqpoint{0.418083in}{0.453533in}}%
\pgfpathcurveto{\pgfqpoint{0.418083in}{0.442483in}}{\pgfqpoint{0.422474in}{0.431884in}}{\pgfqpoint{0.430287in}{0.424071in}}%
\pgfpathcurveto{\pgfqpoint{0.438101in}{0.416257in}}{\pgfqpoint{0.448700in}{0.411867in}}{\pgfqpoint{0.459750in}{0.411867in}}%
\pgfpathclose%
\pgfusepath{stroke,fill}%
\end{pgfscope}%
\begin{pgfscope}%
\pgfpathrectangle{\pgfqpoint{0.375000in}{0.330000in}}{\pgfqpoint{2.325000in}{2.310000in}}%
\pgfusepath{clip}%
\pgfsetbuttcap%
\pgfsetroundjoin%
\definecolor{currentfill}{rgb}{0.000000,0.000000,0.000000}%
\pgfsetfillcolor{currentfill}%
\pgfsetlinewidth{1.003750pt}%
\definecolor{currentstroke}{rgb}{0.000000,0.000000,0.000000}%
\pgfsetstrokecolor{currentstroke}%
\pgfsetdash{}{0pt}%
\pgfpathmoveto{\pgfqpoint{0.459750in}{0.430071in}}%
\pgfpathcurveto{\pgfqpoint{0.470800in}{0.430071in}}{\pgfqpoint{0.481399in}{0.434461in}}{\pgfqpoint{0.489213in}{0.442275in}}%
\pgfpathcurveto{\pgfqpoint{0.497026in}{0.450088in}}{\pgfqpoint{0.501417in}{0.460688in}}{\pgfqpoint{0.501417in}{0.471738in}}%
\pgfpathcurveto{\pgfqpoint{0.501417in}{0.482788in}}{\pgfqpoint{0.497026in}{0.493387in}}{\pgfqpoint{0.489213in}{0.501200in}}%
\pgfpathcurveto{\pgfqpoint{0.481399in}{0.509014in}}{\pgfqpoint{0.470800in}{0.513404in}}{\pgfqpoint{0.459750in}{0.513404in}}%
\pgfpathcurveto{\pgfqpoint{0.448700in}{0.513404in}}{\pgfqpoint{0.438101in}{0.509014in}}{\pgfqpoint{0.430287in}{0.501200in}}%
\pgfpathcurveto{\pgfqpoint{0.422474in}{0.493387in}}{\pgfqpoint{0.418083in}{0.482788in}}{\pgfqpoint{0.418083in}{0.471738in}}%
\pgfpathcurveto{\pgfqpoint{0.418083in}{0.460688in}}{\pgfqpoint{0.422474in}{0.450088in}}{\pgfqpoint{0.430287in}{0.442275in}}%
\pgfpathcurveto{\pgfqpoint{0.438101in}{0.434461in}}{\pgfqpoint{0.448700in}{0.430071in}}{\pgfqpoint{0.459750in}{0.430071in}}%
\pgfpathclose%
\pgfusepath{stroke,fill}%
\end{pgfscope}%
\begin{pgfscope}%
\pgfpathrectangle{\pgfqpoint{0.375000in}{0.330000in}}{\pgfqpoint{2.325000in}{2.310000in}}%
\pgfusepath{clip}%
\pgfsetbuttcap%
\pgfsetroundjoin%
\definecolor{currentfill}{rgb}{0.000000,0.000000,0.000000}%
\pgfsetfillcolor{currentfill}%
\pgfsetlinewidth{1.003750pt}%
\definecolor{currentstroke}{rgb}{0.000000,0.000000,0.000000}%
\pgfsetstrokecolor{currentstroke}%
\pgfsetdash{}{0pt}%
\pgfpathmoveto{\pgfqpoint{0.459750in}{0.405799in}}%
\pgfpathcurveto{\pgfqpoint{0.470800in}{0.405799in}}{\pgfqpoint{0.481399in}{0.410189in}}{\pgfqpoint{0.489213in}{0.418002in}}%
\pgfpathcurveto{\pgfqpoint{0.497026in}{0.425816in}}{\pgfqpoint{0.501417in}{0.436415in}}{\pgfqpoint{0.501417in}{0.447465in}}%
\pgfpathcurveto{\pgfqpoint{0.501417in}{0.458515in}}{\pgfqpoint{0.497026in}{0.469114in}}{\pgfqpoint{0.489213in}{0.476928in}}%
\pgfpathcurveto{\pgfqpoint{0.481399in}{0.484742in}}{\pgfqpoint{0.470800in}{0.489132in}}{\pgfqpoint{0.459750in}{0.489132in}}%
\pgfpathcurveto{\pgfqpoint{0.448700in}{0.489132in}}{\pgfqpoint{0.438101in}{0.484742in}}{\pgfqpoint{0.430287in}{0.476928in}}%
\pgfpathcurveto{\pgfqpoint{0.422474in}{0.469114in}}{\pgfqpoint{0.418083in}{0.458515in}}{\pgfqpoint{0.418083in}{0.447465in}}%
\pgfpathcurveto{\pgfqpoint{0.418083in}{0.436415in}}{\pgfqpoint{0.422474in}{0.425816in}}{\pgfqpoint{0.430287in}{0.418002in}}%
\pgfpathcurveto{\pgfqpoint{0.438101in}{0.410189in}}{\pgfqpoint{0.448700in}{0.405799in}}{\pgfqpoint{0.459750in}{0.405799in}}%
\pgfpathclose%
\pgfusepath{stroke,fill}%
\end{pgfscope}%
\begin{pgfscope}%
\pgfpathrectangle{\pgfqpoint{0.375000in}{0.330000in}}{\pgfqpoint{2.325000in}{2.310000in}}%
\pgfusepath{clip}%
\pgfsetbuttcap%
\pgfsetroundjoin%
\definecolor{currentfill}{rgb}{0.000000,0.000000,0.000000}%
\pgfsetfillcolor{currentfill}%
\pgfsetlinewidth{1.003750pt}%
\definecolor{currentstroke}{rgb}{0.000000,0.000000,0.000000}%
\pgfsetstrokecolor{currentstroke}%
\pgfsetdash{}{0pt}%
\pgfpathmoveto{\pgfqpoint{0.459750in}{0.405799in}}%
\pgfpathcurveto{\pgfqpoint{0.470800in}{0.405799in}}{\pgfqpoint{0.481399in}{0.410189in}}{\pgfqpoint{0.489213in}{0.418002in}}%
\pgfpathcurveto{\pgfqpoint{0.497026in}{0.425816in}}{\pgfqpoint{0.501417in}{0.436415in}}{\pgfqpoint{0.501417in}{0.447465in}}%
\pgfpathcurveto{\pgfqpoint{0.501417in}{0.458515in}}{\pgfqpoint{0.497026in}{0.469114in}}{\pgfqpoint{0.489213in}{0.476928in}}%
\pgfpathcurveto{\pgfqpoint{0.481399in}{0.484742in}}{\pgfqpoint{0.470800in}{0.489132in}}{\pgfqpoint{0.459750in}{0.489132in}}%
\pgfpathcurveto{\pgfqpoint{0.448700in}{0.489132in}}{\pgfqpoint{0.438101in}{0.484742in}}{\pgfqpoint{0.430287in}{0.476928in}}%
\pgfpathcurveto{\pgfqpoint{0.422474in}{0.469114in}}{\pgfqpoint{0.418083in}{0.458515in}}{\pgfqpoint{0.418083in}{0.447465in}}%
\pgfpathcurveto{\pgfqpoint{0.418083in}{0.436415in}}{\pgfqpoint{0.422474in}{0.425816in}}{\pgfqpoint{0.430287in}{0.418002in}}%
\pgfpathcurveto{\pgfqpoint{0.438101in}{0.410189in}}{\pgfqpoint{0.448700in}{0.405799in}}{\pgfqpoint{0.459750in}{0.405799in}}%
\pgfpathclose%
\pgfusepath{stroke,fill}%
\end{pgfscope}%
\begin{pgfscope}%
\pgfpathrectangle{\pgfqpoint{0.375000in}{0.330000in}}{\pgfqpoint{2.325000in}{2.310000in}}%
\pgfusepath{clip}%
\pgfsetbuttcap%
\pgfsetroundjoin%
\definecolor{currentfill}{rgb}{0.000000,0.000000,0.000000}%
\pgfsetfillcolor{currentfill}%
\pgfsetlinewidth{1.003750pt}%
\definecolor{currentstroke}{rgb}{0.000000,0.000000,0.000000}%
\pgfsetstrokecolor{currentstroke}%
\pgfsetdash{}{0pt}%
\pgfpathmoveto{\pgfqpoint{0.459750in}{0.411867in}}%
\pgfpathcurveto{\pgfqpoint{0.470800in}{0.411867in}}{\pgfqpoint{0.481399in}{0.416257in}}{\pgfqpoint{0.489213in}{0.424071in}}%
\pgfpathcurveto{\pgfqpoint{0.497026in}{0.431884in}}{\pgfqpoint{0.501417in}{0.442483in}}{\pgfqpoint{0.501417in}{0.453533in}}%
\pgfpathcurveto{\pgfqpoint{0.501417in}{0.464583in}}{\pgfqpoint{0.497026in}{0.475182in}}{\pgfqpoint{0.489213in}{0.482996in}}%
\pgfpathcurveto{\pgfqpoint{0.481399in}{0.490810in}}{\pgfqpoint{0.470800in}{0.495200in}}{\pgfqpoint{0.459750in}{0.495200in}}%
\pgfpathcurveto{\pgfqpoint{0.448700in}{0.495200in}}{\pgfqpoint{0.438101in}{0.490810in}}{\pgfqpoint{0.430287in}{0.482996in}}%
\pgfpathcurveto{\pgfqpoint{0.422474in}{0.475182in}}{\pgfqpoint{0.418083in}{0.464583in}}{\pgfqpoint{0.418083in}{0.453533in}}%
\pgfpathcurveto{\pgfqpoint{0.418083in}{0.442483in}}{\pgfqpoint{0.422474in}{0.431884in}}{\pgfqpoint{0.430287in}{0.424071in}}%
\pgfpathcurveto{\pgfqpoint{0.438101in}{0.416257in}}{\pgfqpoint{0.448700in}{0.411867in}}{\pgfqpoint{0.459750in}{0.411867in}}%
\pgfpathclose%
\pgfusepath{stroke,fill}%
\end{pgfscope}%
\begin{pgfscope}%
\pgfpathrectangle{\pgfqpoint{0.375000in}{0.330000in}}{\pgfqpoint{2.325000in}{2.310000in}}%
\pgfusepath{clip}%
\pgfsetbuttcap%
\pgfsetroundjoin%
\definecolor{currentfill}{rgb}{0.000000,0.000000,0.000000}%
\pgfsetfillcolor{currentfill}%
\pgfsetlinewidth{1.003750pt}%
\definecolor{currentstroke}{rgb}{0.000000,0.000000,0.000000}%
\pgfsetstrokecolor{currentstroke}%
\pgfsetdash{}{0pt}%
\pgfpathmoveto{\pgfqpoint{0.459750in}{0.417935in}}%
\pgfpathcurveto{\pgfqpoint{0.470800in}{0.417935in}}{\pgfqpoint{0.481399in}{0.422325in}}{\pgfqpoint{0.489213in}{0.430139in}}%
\pgfpathcurveto{\pgfqpoint{0.497026in}{0.437952in}}{\pgfqpoint{0.501417in}{0.448551in}}{\pgfqpoint{0.501417in}{0.459601in}}%
\pgfpathcurveto{\pgfqpoint{0.501417in}{0.470652in}}{\pgfqpoint{0.497026in}{0.481251in}}{\pgfqpoint{0.489213in}{0.489064in}}%
\pgfpathcurveto{\pgfqpoint{0.481399in}{0.496878in}}{\pgfqpoint{0.470800in}{0.501268in}}{\pgfqpoint{0.459750in}{0.501268in}}%
\pgfpathcurveto{\pgfqpoint{0.448700in}{0.501268in}}{\pgfqpoint{0.438101in}{0.496878in}}{\pgfqpoint{0.430287in}{0.489064in}}%
\pgfpathcurveto{\pgfqpoint{0.422474in}{0.481251in}}{\pgfqpoint{0.418083in}{0.470652in}}{\pgfqpoint{0.418083in}{0.459601in}}%
\pgfpathcurveto{\pgfqpoint{0.418083in}{0.448551in}}{\pgfqpoint{0.422474in}{0.437952in}}{\pgfqpoint{0.430287in}{0.430139in}}%
\pgfpathcurveto{\pgfqpoint{0.438101in}{0.422325in}}{\pgfqpoint{0.448700in}{0.417935in}}{\pgfqpoint{0.459750in}{0.417935in}}%
\pgfpathclose%
\pgfusepath{stroke,fill}%
\end{pgfscope}%
\begin{pgfscope}%
\pgfpathrectangle{\pgfqpoint{0.375000in}{0.330000in}}{\pgfqpoint{2.325000in}{2.310000in}}%
\pgfusepath{clip}%
\pgfsetbuttcap%
\pgfsetroundjoin%
\definecolor{currentfill}{rgb}{0.000000,0.000000,0.000000}%
\pgfsetfillcolor{currentfill}%
\pgfsetlinewidth{1.003750pt}%
\definecolor{currentstroke}{rgb}{0.000000,0.000000,0.000000}%
\pgfsetstrokecolor{currentstroke}%
\pgfsetdash{}{0pt}%
\pgfpathmoveto{\pgfqpoint{0.459750in}{0.405799in}}%
\pgfpathcurveto{\pgfqpoint{0.470800in}{0.405799in}}{\pgfqpoint{0.481399in}{0.410189in}}{\pgfqpoint{0.489213in}{0.418002in}}%
\pgfpathcurveto{\pgfqpoint{0.497026in}{0.425816in}}{\pgfqpoint{0.501417in}{0.436415in}}{\pgfqpoint{0.501417in}{0.447465in}}%
\pgfpathcurveto{\pgfqpoint{0.501417in}{0.458515in}}{\pgfqpoint{0.497026in}{0.469114in}}{\pgfqpoint{0.489213in}{0.476928in}}%
\pgfpathcurveto{\pgfqpoint{0.481399in}{0.484742in}}{\pgfqpoint{0.470800in}{0.489132in}}{\pgfqpoint{0.459750in}{0.489132in}}%
\pgfpathcurveto{\pgfqpoint{0.448700in}{0.489132in}}{\pgfqpoint{0.438101in}{0.484742in}}{\pgfqpoint{0.430287in}{0.476928in}}%
\pgfpathcurveto{\pgfqpoint{0.422474in}{0.469114in}}{\pgfqpoint{0.418083in}{0.458515in}}{\pgfqpoint{0.418083in}{0.447465in}}%
\pgfpathcurveto{\pgfqpoint{0.418083in}{0.436415in}}{\pgfqpoint{0.422474in}{0.425816in}}{\pgfqpoint{0.430287in}{0.418002in}}%
\pgfpathcurveto{\pgfqpoint{0.438101in}{0.410189in}}{\pgfqpoint{0.448700in}{0.405799in}}{\pgfqpoint{0.459750in}{0.405799in}}%
\pgfpathclose%
\pgfusepath{stroke,fill}%
\end{pgfscope}%
\begin{pgfscope}%
\pgfpathrectangle{\pgfqpoint{0.375000in}{0.330000in}}{\pgfqpoint{2.325000in}{2.310000in}}%
\pgfusepath{clip}%
\pgfsetbuttcap%
\pgfsetroundjoin%
\definecolor{currentfill}{rgb}{0.000000,0.000000,0.000000}%
\pgfsetfillcolor{currentfill}%
\pgfsetlinewidth{1.003750pt}%
\definecolor{currentstroke}{rgb}{0.000000,0.000000,0.000000}%
\pgfsetstrokecolor{currentstroke}%
\pgfsetdash{}{0pt}%
\pgfpathmoveto{\pgfqpoint{1.019812in}{0.630318in}}%
\pgfpathcurveto{\pgfqpoint{1.030863in}{0.630318in}}{\pgfqpoint{1.041462in}{0.634709in}}{\pgfqpoint{1.049275in}{0.642522in}}%
\pgfpathcurveto{\pgfqpoint{1.057089in}{0.650336in}}{\pgfqpoint{1.061479in}{0.660935in}}{\pgfqpoint{1.061479in}{0.671985in}}%
\pgfpathcurveto{\pgfqpoint{1.061479in}{0.683035in}}{\pgfqpoint{1.057089in}{0.693634in}}{\pgfqpoint{1.049275in}{0.701448in}}%
\pgfpathcurveto{\pgfqpoint{1.041462in}{0.709262in}}{\pgfqpoint{1.030863in}{0.713652in}}{\pgfqpoint{1.019812in}{0.713652in}}%
\pgfpathcurveto{\pgfqpoint{1.008762in}{0.713652in}}{\pgfqpoint{0.998163in}{0.709262in}}{\pgfqpoint{0.990350in}{0.701448in}}%
\pgfpathcurveto{\pgfqpoint{0.982536in}{0.693634in}}{\pgfqpoint{0.978146in}{0.683035in}}{\pgfqpoint{0.978146in}{0.671985in}}%
\pgfpathcurveto{\pgfqpoint{0.978146in}{0.660935in}}{\pgfqpoint{0.982536in}{0.650336in}}{\pgfqpoint{0.990350in}{0.642522in}}%
\pgfpathcurveto{\pgfqpoint{0.998163in}{0.634709in}}{\pgfqpoint{1.008762in}{0.630318in}}{\pgfqpoint{1.019812in}{0.630318in}}%
\pgfpathclose%
\pgfusepath{stroke,fill}%
\end{pgfscope}%
\begin{pgfscope}%
\pgfpathrectangle{\pgfqpoint{0.375000in}{0.330000in}}{\pgfqpoint{2.325000in}{2.310000in}}%
\pgfusepath{clip}%
\pgfsetbuttcap%
\pgfsetroundjoin%
\definecolor{currentfill}{rgb}{0.000000,0.000000,0.000000}%
\pgfsetfillcolor{currentfill}%
\pgfsetlinewidth{1.003750pt}%
\definecolor{currentstroke}{rgb}{0.000000,0.000000,0.000000}%
\pgfsetstrokecolor{currentstroke}%
\pgfsetdash{}{0pt}%
\pgfpathmoveto{\pgfqpoint{1.019812in}{0.599978in}}%
\pgfpathcurveto{\pgfqpoint{1.030863in}{0.599978in}}{\pgfqpoint{1.041462in}{0.604368in}}{\pgfqpoint{1.049275in}{0.612182in}}%
\pgfpathcurveto{\pgfqpoint{1.057089in}{0.619995in}}{\pgfqpoint{1.061479in}{0.630594in}}{\pgfqpoint{1.061479in}{0.641645in}}%
\pgfpathcurveto{\pgfqpoint{1.061479in}{0.652695in}}{\pgfqpoint{1.057089in}{0.663294in}}{\pgfqpoint{1.049275in}{0.671107in}}%
\pgfpathcurveto{\pgfqpoint{1.041462in}{0.678921in}}{\pgfqpoint{1.030863in}{0.683311in}}{\pgfqpoint{1.019812in}{0.683311in}}%
\pgfpathcurveto{\pgfqpoint{1.008762in}{0.683311in}}{\pgfqpoint{0.998163in}{0.678921in}}{\pgfqpoint{0.990350in}{0.671107in}}%
\pgfpathcurveto{\pgfqpoint{0.982536in}{0.663294in}}{\pgfqpoint{0.978146in}{0.652695in}}{\pgfqpoint{0.978146in}{0.641645in}}%
\pgfpathcurveto{\pgfqpoint{0.978146in}{0.630594in}}{\pgfqpoint{0.982536in}{0.619995in}}{\pgfqpoint{0.990350in}{0.612182in}}%
\pgfpathcurveto{\pgfqpoint{0.998163in}{0.604368in}}{\pgfqpoint{1.008762in}{0.599978in}}{\pgfqpoint{1.019812in}{0.599978in}}%
\pgfpathclose%
\pgfusepath{stroke,fill}%
\end{pgfscope}%
\begin{pgfscope}%
\pgfpathrectangle{\pgfqpoint{0.375000in}{0.330000in}}{\pgfqpoint{2.325000in}{2.310000in}}%
\pgfusepath{clip}%
\pgfsetbuttcap%
\pgfsetroundjoin%
\definecolor{currentfill}{rgb}{0.000000,0.000000,0.000000}%
\pgfsetfillcolor{currentfill}%
\pgfsetlinewidth{1.003750pt}%
\definecolor{currentstroke}{rgb}{0.000000,0.000000,0.000000}%
\pgfsetstrokecolor{currentstroke}%
\pgfsetdash{}{0pt}%
\pgfpathmoveto{\pgfqpoint{1.019812in}{0.569637in}}%
\pgfpathcurveto{\pgfqpoint{1.030863in}{0.569637in}}{\pgfqpoint{1.041462in}{0.574028in}}{\pgfqpoint{1.049275in}{0.581841in}}%
\pgfpathcurveto{\pgfqpoint{1.057089in}{0.589655in}}{\pgfqpoint{1.061479in}{0.600254in}}{\pgfqpoint{1.061479in}{0.611304in}}%
\pgfpathcurveto{\pgfqpoint{1.061479in}{0.622354in}}{\pgfqpoint{1.057089in}{0.632953in}}{\pgfqpoint{1.049275in}{0.640767in}}%
\pgfpathcurveto{\pgfqpoint{1.041462in}{0.648580in}}{\pgfqpoint{1.030863in}{0.652971in}}{\pgfqpoint{1.019812in}{0.652971in}}%
\pgfpathcurveto{\pgfqpoint{1.008762in}{0.652971in}}{\pgfqpoint{0.998163in}{0.648580in}}{\pgfqpoint{0.990350in}{0.640767in}}%
\pgfpathcurveto{\pgfqpoint{0.982536in}{0.632953in}}{\pgfqpoint{0.978146in}{0.622354in}}{\pgfqpoint{0.978146in}{0.611304in}}%
\pgfpathcurveto{\pgfqpoint{0.978146in}{0.600254in}}{\pgfqpoint{0.982536in}{0.589655in}}{\pgfqpoint{0.990350in}{0.581841in}}%
\pgfpathcurveto{\pgfqpoint{0.998163in}{0.574028in}}{\pgfqpoint{1.008762in}{0.569637in}}{\pgfqpoint{1.019812in}{0.569637in}}%
\pgfpathclose%
\pgfusepath{stroke,fill}%
\end{pgfscope}%
\begin{pgfscope}%
\pgfpathrectangle{\pgfqpoint{0.375000in}{0.330000in}}{\pgfqpoint{2.325000in}{2.310000in}}%
\pgfusepath{clip}%
\pgfsetbuttcap%
\pgfsetroundjoin%
\definecolor{currentfill}{rgb}{0.000000,0.000000,0.000000}%
\pgfsetfillcolor{currentfill}%
\pgfsetlinewidth{1.003750pt}%
\definecolor{currentstroke}{rgb}{0.000000,0.000000,0.000000}%
\pgfsetstrokecolor{currentstroke}%
\pgfsetdash{}{0pt}%
\pgfpathmoveto{\pgfqpoint{1.019812in}{0.636387in}}%
\pgfpathcurveto{\pgfqpoint{1.030863in}{0.636387in}}{\pgfqpoint{1.041462in}{0.640777in}}{\pgfqpoint{1.049275in}{0.648590in}}%
\pgfpathcurveto{\pgfqpoint{1.057089in}{0.656404in}}{\pgfqpoint{1.061479in}{0.667003in}}{\pgfqpoint{1.061479in}{0.678053in}}%
\pgfpathcurveto{\pgfqpoint{1.061479in}{0.689103in}}{\pgfqpoint{1.057089in}{0.699702in}}{\pgfqpoint{1.049275in}{0.707516in}}%
\pgfpathcurveto{\pgfqpoint{1.041462in}{0.715330in}}{\pgfqpoint{1.030863in}{0.719720in}}{\pgfqpoint{1.019812in}{0.719720in}}%
\pgfpathcurveto{\pgfqpoint{1.008762in}{0.719720in}}{\pgfqpoint{0.998163in}{0.715330in}}{\pgfqpoint{0.990350in}{0.707516in}}%
\pgfpathcurveto{\pgfqpoint{0.982536in}{0.699702in}}{\pgfqpoint{0.978146in}{0.689103in}}{\pgfqpoint{0.978146in}{0.678053in}}%
\pgfpathcurveto{\pgfqpoint{0.978146in}{0.667003in}}{\pgfqpoint{0.982536in}{0.656404in}}{\pgfqpoint{0.990350in}{0.648590in}}%
\pgfpathcurveto{\pgfqpoint{0.998163in}{0.640777in}}{\pgfqpoint{1.008762in}{0.636387in}}{\pgfqpoint{1.019812in}{0.636387in}}%
\pgfpathclose%
\pgfusepath{stroke,fill}%
\end{pgfscope}%
\begin{pgfscope}%
\pgfpathrectangle{\pgfqpoint{0.375000in}{0.330000in}}{\pgfqpoint{2.325000in}{2.310000in}}%
\pgfusepath{clip}%
\pgfsetbuttcap%
\pgfsetroundjoin%
\definecolor{currentfill}{rgb}{0.000000,0.000000,0.000000}%
\pgfsetfillcolor{currentfill}%
\pgfsetlinewidth{1.003750pt}%
\definecolor{currentstroke}{rgb}{0.000000,0.000000,0.000000}%
\pgfsetstrokecolor{currentstroke}%
\pgfsetdash{}{0pt}%
\pgfpathmoveto{\pgfqpoint{1.019812in}{0.654591in}}%
\pgfpathcurveto{\pgfqpoint{1.030863in}{0.654591in}}{\pgfqpoint{1.041462in}{0.658981in}}{\pgfqpoint{1.049275in}{0.666795in}}%
\pgfpathcurveto{\pgfqpoint{1.057089in}{0.674608in}}{\pgfqpoint{1.061479in}{0.685207in}}{\pgfqpoint{1.061479in}{0.696258in}}%
\pgfpathcurveto{\pgfqpoint{1.061479in}{0.707308in}}{\pgfqpoint{1.057089in}{0.717907in}}{\pgfqpoint{1.049275in}{0.725720in}}%
\pgfpathcurveto{\pgfqpoint{1.041462in}{0.733534in}}{\pgfqpoint{1.030863in}{0.737924in}}{\pgfqpoint{1.019812in}{0.737924in}}%
\pgfpathcurveto{\pgfqpoint{1.008762in}{0.737924in}}{\pgfqpoint{0.998163in}{0.733534in}}{\pgfqpoint{0.990350in}{0.725720in}}%
\pgfpathcurveto{\pgfqpoint{0.982536in}{0.717907in}}{\pgfqpoint{0.978146in}{0.707308in}}{\pgfqpoint{0.978146in}{0.696258in}}%
\pgfpathcurveto{\pgfqpoint{0.978146in}{0.685207in}}{\pgfqpoint{0.982536in}{0.674608in}}{\pgfqpoint{0.990350in}{0.666795in}}%
\pgfpathcurveto{\pgfqpoint{0.998163in}{0.658981in}}{\pgfqpoint{1.008762in}{0.654591in}}{\pgfqpoint{1.019812in}{0.654591in}}%
\pgfpathclose%
\pgfusepath{stroke,fill}%
\end{pgfscope}%
\begin{pgfscope}%
\pgfpathrectangle{\pgfqpoint{0.375000in}{0.330000in}}{\pgfqpoint{2.325000in}{2.310000in}}%
\pgfusepath{clip}%
\pgfsetbuttcap%
\pgfsetroundjoin%
\definecolor{currentfill}{rgb}{0.000000,0.000000,0.000000}%
\pgfsetfillcolor{currentfill}%
\pgfsetlinewidth{1.003750pt}%
\definecolor{currentstroke}{rgb}{0.000000,0.000000,0.000000}%
\pgfsetstrokecolor{currentstroke}%
\pgfsetdash{}{0pt}%
\pgfpathmoveto{\pgfqpoint{1.019812in}{0.599978in}}%
\pgfpathcurveto{\pgfqpoint{1.030863in}{0.599978in}}{\pgfqpoint{1.041462in}{0.604368in}}{\pgfqpoint{1.049275in}{0.612182in}}%
\pgfpathcurveto{\pgfqpoint{1.057089in}{0.619995in}}{\pgfqpoint{1.061479in}{0.630594in}}{\pgfqpoint{1.061479in}{0.641645in}}%
\pgfpathcurveto{\pgfqpoint{1.061479in}{0.652695in}}{\pgfqpoint{1.057089in}{0.663294in}}{\pgfqpoint{1.049275in}{0.671107in}}%
\pgfpathcurveto{\pgfqpoint{1.041462in}{0.678921in}}{\pgfqpoint{1.030863in}{0.683311in}}{\pgfqpoint{1.019812in}{0.683311in}}%
\pgfpathcurveto{\pgfqpoint{1.008762in}{0.683311in}}{\pgfqpoint{0.998163in}{0.678921in}}{\pgfqpoint{0.990350in}{0.671107in}}%
\pgfpathcurveto{\pgfqpoint{0.982536in}{0.663294in}}{\pgfqpoint{0.978146in}{0.652695in}}{\pgfqpoint{0.978146in}{0.641645in}}%
\pgfpathcurveto{\pgfqpoint{0.978146in}{0.630594in}}{\pgfqpoint{0.982536in}{0.619995in}}{\pgfqpoint{0.990350in}{0.612182in}}%
\pgfpathcurveto{\pgfqpoint{0.998163in}{0.604368in}}{\pgfqpoint{1.008762in}{0.599978in}}{\pgfqpoint{1.019812in}{0.599978in}}%
\pgfpathclose%
\pgfusepath{stroke,fill}%
\end{pgfscope}%
\begin{pgfscope}%
\pgfpathrectangle{\pgfqpoint{0.375000in}{0.330000in}}{\pgfqpoint{2.325000in}{2.310000in}}%
\pgfusepath{clip}%
\pgfsetbuttcap%
\pgfsetroundjoin%
\definecolor{currentfill}{rgb}{0.000000,0.000000,0.000000}%
\pgfsetfillcolor{currentfill}%
\pgfsetlinewidth{1.003750pt}%
\definecolor{currentstroke}{rgb}{0.000000,0.000000,0.000000}%
\pgfsetstrokecolor{currentstroke}%
\pgfsetdash{}{0pt}%
\pgfpathmoveto{\pgfqpoint{1.019812in}{0.624250in}}%
\pgfpathcurveto{\pgfqpoint{1.030863in}{0.624250in}}{\pgfqpoint{1.041462in}{0.628641in}}{\pgfqpoint{1.049275in}{0.636454in}}%
\pgfpathcurveto{\pgfqpoint{1.057089in}{0.644268in}}{\pgfqpoint{1.061479in}{0.654867in}}{\pgfqpoint{1.061479in}{0.665917in}}%
\pgfpathcurveto{\pgfqpoint{1.061479in}{0.676967in}}{\pgfqpoint{1.057089in}{0.687566in}}{\pgfqpoint{1.049275in}{0.695380in}}%
\pgfpathcurveto{\pgfqpoint{1.041462in}{0.703193in}}{\pgfqpoint{1.030863in}{0.707584in}}{\pgfqpoint{1.019812in}{0.707584in}}%
\pgfpathcurveto{\pgfqpoint{1.008762in}{0.707584in}}{\pgfqpoint{0.998163in}{0.703193in}}{\pgfqpoint{0.990350in}{0.695380in}}%
\pgfpathcurveto{\pgfqpoint{0.982536in}{0.687566in}}{\pgfqpoint{0.978146in}{0.676967in}}{\pgfqpoint{0.978146in}{0.665917in}}%
\pgfpathcurveto{\pgfqpoint{0.978146in}{0.654867in}}{\pgfqpoint{0.982536in}{0.644268in}}{\pgfqpoint{0.990350in}{0.636454in}}%
\pgfpathcurveto{\pgfqpoint{0.998163in}{0.628641in}}{\pgfqpoint{1.008762in}{0.624250in}}{\pgfqpoint{1.019812in}{0.624250in}}%
\pgfpathclose%
\pgfusepath{stroke,fill}%
\end{pgfscope}%
\begin{pgfscope}%
\pgfpathrectangle{\pgfqpoint{0.375000in}{0.330000in}}{\pgfqpoint{2.325000in}{2.310000in}}%
\pgfusepath{clip}%
\pgfsetbuttcap%
\pgfsetroundjoin%
\definecolor{currentfill}{rgb}{0.000000,0.000000,0.000000}%
\pgfsetfillcolor{currentfill}%
\pgfsetlinewidth{1.003750pt}%
\definecolor{currentstroke}{rgb}{0.000000,0.000000,0.000000}%
\pgfsetstrokecolor{currentstroke}%
\pgfsetdash{}{0pt}%
\pgfpathmoveto{\pgfqpoint{1.019812in}{0.624250in}}%
\pgfpathcurveto{\pgfqpoint{1.030863in}{0.624250in}}{\pgfqpoint{1.041462in}{0.628641in}}{\pgfqpoint{1.049275in}{0.636454in}}%
\pgfpathcurveto{\pgfqpoint{1.057089in}{0.644268in}}{\pgfqpoint{1.061479in}{0.654867in}}{\pgfqpoint{1.061479in}{0.665917in}}%
\pgfpathcurveto{\pgfqpoint{1.061479in}{0.676967in}}{\pgfqpoint{1.057089in}{0.687566in}}{\pgfqpoint{1.049275in}{0.695380in}}%
\pgfpathcurveto{\pgfqpoint{1.041462in}{0.703193in}}{\pgfqpoint{1.030863in}{0.707584in}}{\pgfqpoint{1.019812in}{0.707584in}}%
\pgfpathcurveto{\pgfqpoint{1.008762in}{0.707584in}}{\pgfqpoint{0.998163in}{0.703193in}}{\pgfqpoint{0.990350in}{0.695380in}}%
\pgfpathcurveto{\pgfqpoint{0.982536in}{0.687566in}}{\pgfqpoint{0.978146in}{0.676967in}}{\pgfqpoint{0.978146in}{0.665917in}}%
\pgfpathcurveto{\pgfqpoint{0.978146in}{0.654867in}}{\pgfqpoint{0.982536in}{0.644268in}}{\pgfqpoint{0.990350in}{0.636454in}}%
\pgfpathcurveto{\pgfqpoint{0.998163in}{0.628641in}}{\pgfqpoint{1.008762in}{0.624250in}}{\pgfqpoint{1.019812in}{0.624250in}}%
\pgfpathclose%
\pgfusepath{stroke,fill}%
\end{pgfscope}%
\begin{pgfscope}%
\pgfpathrectangle{\pgfqpoint{0.375000in}{0.330000in}}{\pgfqpoint{2.325000in}{2.310000in}}%
\pgfusepath{clip}%
\pgfsetbuttcap%
\pgfsetroundjoin%
\definecolor{currentfill}{rgb}{0.000000,0.000000,0.000000}%
\pgfsetfillcolor{currentfill}%
\pgfsetlinewidth{1.003750pt}%
\definecolor{currentstroke}{rgb}{0.000000,0.000000,0.000000}%
\pgfsetstrokecolor{currentstroke}%
\pgfsetdash{}{0pt}%
\pgfpathmoveto{\pgfqpoint{1.019812in}{0.642455in}}%
\pgfpathcurveto{\pgfqpoint{1.030863in}{0.642455in}}{\pgfqpoint{1.041462in}{0.646845in}}{\pgfqpoint{1.049275in}{0.654659in}}%
\pgfpathcurveto{\pgfqpoint{1.057089in}{0.662472in}}{\pgfqpoint{1.061479in}{0.673071in}}{\pgfqpoint{1.061479in}{0.684121in}}%
\pgfpathcurveto{\pgfqpoint{1.061479in}{0.695171in}}{\pgfqpoint{1.057089in}{0.705770in}}{\pgfqpoint{1.049275in}{0.713584in}}%
\pgfpathcurveto{\pgfqpoint{1.041462in}{0.721398in}}{\pgfqpoint{1.030863in}{0.725788in}}{\pgfqpoint{1.019812in}{0.725788in}}%
\pgfpathcurveto{\pgfqpoint{1.008762in}{0.725788in}}{\pgfqpoint{0.998163in}{0.721398in}}{\pgfqpoint{0.990350in}{0.713584in}}%
\pgfpathcurveto{\pgfqpoint{0.982536in}{0.705770in}}{\pgfqpoint{0.978146in}{0.695171in}}{\pgfqpoint{0.978146in}{0.684121in}}%
\pgfpathcurveto{\pgfqpoint{0.978146in}{0.673071in}}{\pgfqpoint{0.982536in}{0.662472in}}{\pgfqpoint{0.990350in}{0.654659in}}%
\pgfpathcurveto{\pgfqpoint{0.998163in}{0.646845in}}{\pgfqpoint{1.008762in}{0.642455in}}{\pgfqpoint{1.019812in}{0.642455in}}%
\pgfpathclose%
\pgfusepath{stroke,fill}%
\end{pgfscope}%
\begin{pgfscope}%
\pgfpathrectangle{\pgfqpoint{0.375000in}{0.330000in}}{\pgfqpoint{2.325000in}{2.310000in}}%
\pgfusepath{clip}%
\pgfsetbuttcap%
\pgfsetroundjoin%
\definecolor{currentfill}{rgb}{0.000000,0.000000,0.000000}%
\pgfsetfillcolor{currentfill}%
\pgfsetlinewidth{1.003750pt}%
\definecolor{currentstroke}{rgb}{0.000000,0.000000,0.000000}%
\pgfsetstrokecolor{currentstroke}%
\pgfsetdash{}{0pt}%
\pgfpathmoveto{\pgfqpoint{1.019812in}{0.642455in}}%
\pgfpathcurveto{\pgfqpoint{1.030863in}{0.642455in}}{\pgfqpoint{1.041462in}{0.646845in}}{\pgfqpoint{1.049275in}{0.654659in}}%
\pgfpathcurveto{\pgfqpoint{1.057089in}{0.662472in}}{\pgfqpoint{1.061479in}{0.673071in}}{\pgfqpoint{1.061479in}{0.684121in}}%
\pgfpathcurveto{\pgfqpoint{1.061479in}{0.695171in}}{\pgfqpoint{1.057089in}{0.705770in}}{\pgfqpoint{1.049275in}{0.713584in}}%
\pgfpathcurveto{\pgfqpoint{1.041462in}{0.721398in}}{\pgfqpoint{1.030863in}{0.725788in}}{\pgfqpoint{1.019812in}{0.725788in}}%
\pgfpathcurveto{\pgfqpoint{1.008762in}{0.725788in}}{\pgfqpoint{0.998163in}{0.721398in}}{\pgfqpoint{0.990350in}{0.713584in}}%
\pgfpathcurveto{\pgfqpoint{0.982536in}{0.705770in}}{\pgfqpoint{0.978146in}{0.695171in}}{\pgfqpoint{0.978146in}{0.684121in}}%
\pgfpathcurveto{\pgfqpoint{0.978146in}{0.673071in}}{\pgfqpoint{0.982536in}{0.662472in}}{\pgfqpoint{0.990350in}{0.654659in}}%
\pgfpathcurveto{\pgfqpoint{0.998163in}{0.646845in}}{\pgfqpoint{1.008762in}{0.642455in}}{\pgfqpoint{1.019812in}{0.642455in}}%
\pgfpathclose%
\pgfusepath{stroke,fill}%
\end{pgfscope}%
\begin{pgfscope}%
\pgfpathrectangle{\pgfqpoint{0.375000in}{0.330000in}}{\pgfqpoint{2.325000in}{2.310000in}}%
\pgfusepath{clip}%
\pgfsetbuttcap%
\pgfsetroundjoin%
\definecolor{currentfill}{rgb}{0.000000,0.000000,0.000000}%
\pgfsetfillcolor{currentfill}%
\pgfsetlinewidth{1.003750pt}%
\definecolor{currentstroke}{rgb}{0.000000,0.000000,0.000000}%
\pgfsetstrokecolor{currentstroke}%
\pgfsetdash{}{0pt}%
\pgfpathmoveto{\pgfqpoint{1.019812in}{0.775953in}}%
\pgfpathcurveto{\pgfqpoint{1.030863in}{0.775953in}}{\pgfqpoint{1.041462in}{0.780343in}}{\pgfqpoint{1.049275in}{0.788157in}}%
\pgfpathcurveto{\pgfqpoint{1.057089in}{0.795971in}}{\pgfqpoint{1.061479in}{0.806570in}}{\pgfqpoint{1.061479in}{0.817620in}}%
\pgfpathcurveto{\pgfqpoint{1.061479in}{0.828670in}}{\pgfqpoint{1.057089in}{0.839269in}}{\pgfqpoint{1.049275in}{0.847082in}}%
\pgfpathcurveto{\pgfqpoint{1.041462in}{0.854896in}}{\pgfqpoint{1.030863in}{0.859286in}}{\pgfqpoint{1.019812in}{0.859286in}}%
\pgfpathcurveto{\pgfqpoint{1.008762in}{0.859286in}}{\pgfqpoint{0.998163in}{0.854896in}}{\pgfqpoint{0.990350in}{0.847082in}}%
\pgfpathcurveto{\pgfqpoint{0.982536in}{0.839269in}}{\pgfqpoint{0.978146in}{0.828670in}}{\pgfqpoint{0.978146in}{0.817620in}}%
\pgfpathcurveto{\pgfqpoint{0.978146in}{0.806570in}}{\pgfqpoint{0.982536in}{0.795971in}}{\pgfqpoint{0.990350in}{0.788157in}}%
\pgfpathcurveto{\pgfqpoint{0.998163in}{0.780343in}}{\pgfqpoint{1.008762in}{0.775953in}}{\pgfqpoint{1.019812in}{0.775953in}}%
\pgfpathclose%
\pgfusepath{stroke,fill}%
\end{pgfscope}%
\begin{pgfscope}%
\pgfpathrectangle{\pgfqpoint{0.375000in}{0.330000in}}{\pgfqpoint{2.325000in}{2.310000in}}%
\pgfusepath{clip}%
\pgfsetbuttcap%
\pgfsetroundjoin%
\definecolor{currentfill}{rgb}{0.000000,0.000000,0.000000}%
\pgfsetfillcolor{currentfill}%
\pgfsetlinewidth{1.003750pt}%
\definecolor{currentstroke}{rgb}{0.000000,0.000000,0.000000}%
\pgfsetstrokecolor{currentstroke}%
\pgfsetdash{}{0pt}%
\pgfpathmoveto{\pgfqpoint{1.019812in}{0.642455in}}%
\pgfpathcurveto{\pgfqpoint{1.030863in}{0.642455in}}{\pgfqpoint{1.041462in}{0.646845in}}{\pgfqpoint{1.049275in}{0.654659in}}%
\pgfpathcurveto{\pgfqpoint{1.057089in}{0.662472in}}{\pgfqpoint{1.061479in}{0.673071in}}{\pgfqpoint{1.061479in}{0.684121in}}%
\pgfpathcurveto{\pgfqpoint{1.061479in}{0.695171in}}{\pgfqpoint{1.057089in}{0.705770in}}{\pgfqpoint{1.049275in}{0.713584in}}%
\pgfpathcurveto{\pgfqpoint{1.041462in}{0.721398in}}{\pgfqpoint{1.030863in}{0.725788in}}{\pgfqpoint{1.019812in}{0.725788in}}%
\pgfpathcurveto{\pgfqpoint{1.008762in}{0.725788in}}{\pgfqpoint{0.998163in}{0.721398in}}{\pgfqpoint{0.990350in}{0.713584in}}%
\pgfpathcurveto{\pgfqpoint{0.982536in}{0.705770in}}{\pgfqpoint{0.978146in}{0.695171in}}{\pgfqpoint{0.978146in}{0.684121in}}%
\pgfpathcurveto{\pgfqpoint{0.978146in}{0.673071in}}{\pgfqpoint{0.982536in}{0.662472in}}{\pgfqpoint{0.990350in}{0.654659in}}%
\pgfpathcurveto{\pgfqpoint{0.998163in}{0.646845in}}{\pgfqpoint{1.008762in}{0.642455in}}{\pgfqpoint{1.019812in}{0.642455in}}%
\pgfpathclose%
\pgfusepath{stroke,fill}%
\end{pgfscope}%
\begin{pgfscope}%
\pgfpathrectangle{\pgfqpoint{0.375000in}{0.330000in}}{\pgfqpoint{2.325000in}{2.310000in}}%
\pgfusepath{clip}%
\pgfsetbuttcap%
\pgfsetroundjoin%
\definecolor{currentfill}{rgb}{0.000000,0.000000,0.000000}%
\pgfsetfillcolor{currentfill}%
\pgfsetlinewidth{1.003750pt}%
\definecolor{currentstroke}{rgb}{0.000000,0.000000,0.000000}%
\pgfsetstrokecolor{currentstroke}%
\pgfsetdash{}{0pt}%
\pgfpathmoveto{\pgfqpoint{1.019812in}{0.630318in}}%
\pgfpathcurveto{\pgfqpoint{1.030863in}{0.630318in}}{\pgfqpoint{1.041462in}{0.634709in}}{\pgfqpoint{1.049275in}{0.642522in}}%
\pgfpathcurveto{\pgfqpoint{1.057089in}{0.650336in}}{\pgfqpoint{1.061479in}{0.660935in}}{\pgfqpoint{1.061479in}{0.671985in}}%
\pgfpathcurveto{\pgfqpoint{1.061479in}{0.683035in}}{\pgfqpoint{1.057089in}{0.693634in}}{\pgfqpoint{1.049275in}{0.701448in}}%
\pgfpathcurveto{\pgfqpoint{1.041462in}{0.709262in}}{\pgfqpoint{1.030863in}{0.713652in}}{\pgfqpoint{1.019812in}{0.713652in}}%
\pgfpathcurveto{\pgfqpoint{1.008762in}{0.713652in}}{\pgfqpoint{0.998163in}{0.709262in}}{\pgfqpoint{0.990350in}{0.701448in}}%
\pgfpathcurveto{\pgfqpoint{0.982536in}{0.693634in}}{\pgfqpoint{0.978146in}{0.683035in}}{\pgfqpoint{0.978146in}{0.671985in}}%
\pgfpathcurveto{\pgfqpoint{0.978146in}{0.660935in}}{\pgfqpoint{0.982536in}{0.650336in}}{\pgfqpoint{0.990350in}{0.642522in}}%
\pgfpathcurveto{\pgfqpoint{0.998163in}{0.634709in}}{\pgfqpoint{1.008762in}{0.630318in}}{\pgfqpoint{1.019812in}{0.630318in}}%
\pgfpathclose%
\pgfusepath{stroke,fill}%
\end{pgfscope}%
\begin{pgfscope}%
\pgfpathrectangle{\pgfqpoint{0.375000in}{0.330000in}}{\pgfqpoint{2.325000in}{2.310000in}}%
\pgfusepath{clip}%
\pgfsetbuttcap%
\pgfsetroundjoin%
\definecolor{currentfill}{rgb}{0.000000,0.000000,0.000000}%
\pgfsetfillcolor{currentfill}%
\pgfsetlinewidth{1.003750pt}%
\definecolor{currentstroke}{rgb}{0.000000,0.000000,0.000000}%
\pgfsetstrokecolor{currentstroke}%
\pgfsetdash{}{0pt}%
\pgfpathmoveto{\pgfqpoint{1.019812in}{0.618182in}}%
\pgfpathcurveto{\pgfqpoint{1.030863in}{0.618182in}}{\pgfqpoint{1.041462in}{0.622573in}}{\pgfqpoint{1.049275in}{0.630386in}}%
\pgfpathcurveto{\pgfqpoint{1.057089in}{0.638200in}}{\pgfqpoint{1.061479in}{0.648799in}}{\pgfqpoint{1.061479in}{0.659849in}}%
\pgfpathcurveto{\pgfqpoint{1.061479in}{0.670899in}}{\pgfqpoint{1.057089in}{0.681498in}}{\pgfqpoint{1.049275in}{0.689312in}}%
\pgfpathcurveto{\pgfqpoint{1.041462in}{0.697125in}}{\pgfqpoint{1.030863in}{0.701516in}}{\pgfqpoint{1.019812in}{0.701516in}}%
\pgfpathcurveto{\pgfqpoint{1.008762in}{0.701516in}}{\pgfqpoint{0.998163in}{0.697125in}}{\pgfqpoint{0.990350in}{0.689312in}}%
\pgfpathcurveto{\pgfqpoint{0.982536in}{0.681498in}}{\pgfqpoint{0.978146in}{0.670899in}}{\pgfqpoint{0.978146in}{0.659849in}}%
\pgfpathcurveto{\pgfqpoint{0.978146in}{0.648799in}}{\pgfqpoint{0.982536in}{0.638200in}}{\pgfqpoint{0.990350in}{0.630386in}}%
\pgfpathcurveto{\pgfqpoint{0.998163in}{0.622573in}}{\pgfqpoint{1.008762in}{0.618182in}}{\pgfqpoint{1.019812in}{0.618182in}}%
\pgfpathclose%
\pgfusepath{stroke,fill}%
\end{pgfscope}%
\begin{pgfscope}%
\pgfpathrectangle{\pgfqpoint{0.375000in}{0.330000in}}{\pgfqpoint{2.325000in}{2.310000in}}%
\pgfusepath{clip}%
\pgfsetbuttcap%
\pgfsetroundjoin%
\definecolor{currentfill}{rgb}{0.000000,0.000000,0.000000}%
\pgfsetfillcolor{currentfill}%
\pgfsetlinewidth{1.003750pt}%
\definecolor{currentstroke}{rgb}{0.000000,0.000000,0.000000}%
\pgfsetstrokecolor{currentstroke}%
\pgfsetdash{}{0pt}%
\pgfpathmoveto{\pgfqpoint{1.019812in}{0.593910in}}%
\pgfpathcurveto{\pgfqpoint{1.030863in}{0.593910in}}{\pgfqpoint{1.041462in}{0.598300in}}{\pgfqpoint{1.049275in}{0.606114in}}%
\pgfpathcurveto{\pgfqpoint{1.057089in}{0.613927in}}{\pgfqpoint{1.061479in}{0.624526in}}{\pgfqpoint{1.061479in}{0.635576in}}%
\pgfpathcurveto{\pgfqpoint{1.061479in}{0.646627in}}{\pgfqpoint{1.057089in}{0.657226in}}{\pgfqpoint{1.049275in}{0.665039in}}%
\pgfpathcurveto{\pgfqpoint{1.041462in}{0.672853in}}{\pgfqpoint{1.030863in}{0.677243in}}{\pgfqpoint{1.019812in}{0.677243in}}%
\pgfpathcurveto{\pgfqpoint{1.008762in}{0.677243in}}{\pgfqpoint{0.998163in}{0.672853in}}{\pgfqpoint{0.990350in}{0.665039in}}%
\pgfpathcurveto{\pgfqpoint{0.982536in}{0.657226in}}{\pgfqpoint{0.978146in}{0.646627in}}{\pgfqpoint{0.978146in}{0.635576in}}%
\pgfpathcurveto{\pgfqpoint{0.978146in}{0.624526in}}{\pgfqpoint{0.982536in}{0.613927in}}{\pgfqpoint{0.990350in}{0.606114in}}%
\pgfpathcurveto{\pgfqpoint{0.998163in}{0.598300in}}{\pgfqpoint{1.008762in}{0.593910in}}{\pgfqpoint{1.019812in}{0.593910in}}%
\pgfpathclose%
\pgfusepath{stroke,fill}%
\end{pgfscope}%
\begin{pgfscope}%
\pgfpathrectangle{\pgfqpoint{0.375000in}{0.330000in}}{\pgfqpoint{2.325000in}{2.310000in}}%
\pgfusepath{clip}%
\pgfsetbuttcap%
\pgfsetroundjoin%
\definecolor{currentfill}{rgb}{0.000000,0.000000,0.000000}%
\pgfsetfillcolor{currentfill}%
\pgfsetlinewidth{1.003750pt}%
\definecolor{currentstroke}{rgb}{0.000000,0.000000,0.000000}%
\pgfsetstrokecolor{currentstroke}%
\pgfsetdash{}{0pt}%
\pgfpathmoveto{\pgfqpoint{1.019812in}{0.599978in}}%
\pgfpathcurveto{\pgfqpoint{1.030863in}{0.599978in}}{\pgfqpoint{1.041462in}{0.604368in}}{\pgfqpoint{1.049275in}{0.612182in}}%
\pgfpathcurveto{\pgfqpoint{1.057089in}{0.619995in}}{\pgfqpoint{1.061479in}{0.630594in}}{\pgfqpoint{1.061479in}{0.641645in}}%
\pgfpathcurveto{\pgfqpoint{1.061479in}{0.652695in}}{\pgfqpoint{1.057089in}{0.663294in}}{\pgfqpoint{1.049275in}{0.671107in}}%
\pgfpathcurveto{\pgfqpoint{1.041462in}{0.678921in}}{\pgfqpoint{1.030863in}{0.683311in}}{\pgfqpoint{1.019812in}{0.683311in}}%
\pgfpathcurveto{\pgfqpoint{1.008762in}{0.683311in}}{\pgfqpoint{0.998163in}{0.678921in}}{\pgfqpoint{0.990350in}{0.671107in}}%
\pgfpathcurveto{\pgfqpoint{0.982536in}{0.663294in}}{\pgfqpoint{0.978146in}{0.652695in}}{\pgfqpoint{0.978146in}{0.641645in}}%
\pgfpathcurveto{\pgfqpoint{0.978146in}{0.630594in}}{\pgfqpoint{0.982536in}{0.619995in}}{\pgfqpoint{0.990350in}{0.612182in}}%
\pgfpathcurveto{\pgfqpoint{0.998163in}{0.604368in}}{\pgfqpoint{1.008762in}{0.599978in}}{\pgfqpoint{1.019812in}{0.599978in}}%
\pgfpathclose%
\pgfusepath{stroke,fill}%
\end{pgfscope}%
\begin{pgfscope}%
\pgfpathrectangle{\pgfqpoint{0.375000in}{0.330000in}}{\pgfqpoint{2.325000in}{2.310000in}}%
\pgfusepath{clip}%
\pgfsetbuttcap%
\pgfsetroundjoin%
\definecolor{currentfill}{rgb}{0.000000,0.000000,0.000000}%
\pgfsetfillcolor{currentfill}%
\pgfsetlinewidth{1.003750pt}%
\definecolor{currentstroke}{rgb}{0.000000,0.000000,0.000000}%
\pgfsetstrokecolor{currentstroke}%
\pgfsetdash{}{0pt}%
\pgfpathmoveto{\pgfqpoint{1.019812in}{0.587842in}}%
\pgfpathcurveto{\pgfqpoint{1.030863in}{0.587842in}}{\pgfqpoint{1.041462in}{0.592232in}}{\pgfqpoint{1.049275in}{0.600046in}}%
\pgfpathcurveto{\pgfqpoint{1.057089in}{0.607859in}}{\pgfqpoint{1.061479in}{0.618458in}}{\pgfqpoint{1.061479in}{0.629508in}}%
\pgfpathcurveto{\pgfqpoint{1.061479in}{0.640559in}}{\pgfqpoint{1.057089in}{0.651158in}}{\pgfqpoint{1.049275in}{0.658971in}}%
\pgfpathcurveto{\pgfqpoint{1.041462in}{0.666785in}}{\pgfqpoint{1.030863in}{0.671175in}}{\pgfqpoint{1.019812in}{0.671175in}}%
\pgfpathcurveto{\pgfqpoint{1.008762in}{0.671175in}}{\pgfqpoint{0.998163in}{0.666785in}}{\pgfqpoint{0.990350in}{0.658971in}}%
\pgfpathcurveto{\pgfqpoint{0.982536in}{0.651158in}}{\pgfqpoint{0.978146in}{0.640559in}}{\pgfqpoint{0.978146in}{0.629508in}}%
\pgfpathcurveto{\pgfqpoint{0.978146in}{0.618458in}}{\pgfqpoint{0.982536in}{0.607859in}}{\pgfqpoint{0.990350in}{0.600046in}}%
\pgfpathcurveto{\pgfqpoint{0.998163in}{0.592232in}}{\pgfqpoint{1.008762in}{0.587842in}}{\pgfqpoint{1.019812in}{0.587842in}}%
\pgfpathclose%
\pgfusepath{stroke,fill}%
\end{pgfscope}%
\begin{pgfscope}%
\pgfpathrectangle{\pgfqpoint{0.375000in}{0.330000in}}{\pgfqpoint{2.325000in}{2.310000in}}%
\pgfusepath{clip}%
\pgfsetbuttcap%
\pgfsetroundjoin%
\definecolor{currentfill}{rgb}{0.000000,0.000000,0.000000}%
\pgfsetfillcolor{currentfill}%
\pgfsetlinewidth{1.003750pt}%
\definecolor{currentstroke}{rgb}{0.000000,0.000000,0.000000}%
\pgfsetstrokecolor{currentstroke}%
\pgfsetdash{}{0pt}%
\pgfpathmoveto{\pgfqpoint{1.019812in}{0.630318in}}%
\pgfpathcurveto{\pgfqpoint{1.030863in}{0.630318in}}{\pgfqpoint{1.041462in}{0.634709in}}{\pgfqpoint{1.049275in}{0.642522in}}%
\pgfpathcurveto{\pgfqpoint{1.057089in}{0.650336in}}{\pgfqpoint{1.061479in}{0.660935in}}{\pgfqpoint{1.061479in}{0.671985in}}%
\pgfpathcurveto{\pgfqpoint{1.061479in}{0.683035in}}{\pgfqpoint{1.057089in}{0.693634in}}{\pgfqpoint{1.049275in}{0.701448in}}%
\pgfpathcurveto{\pgfqpoint{1.041462in}{0.709262in}}{\pgfqpoint{1.030863in}{0.713652in}}{\pgfqpoint{1.019812in}{0.713652in}}%
\pgfpathcurveto{\pgfqpoint{1.008762in}{0.713652in}}{\pgfqpoint{0.998163in}{0.709262in}}{\pgfqpoint{0.990350in}{0.701448in}}%
\pgfpathcurveto{\pgfqpoint{0.982536in}{0.693634in}}{\pgfqpoint{0.978146in}{0.683035in}}{\pgfqpoint{0.978146in}{0.671985in}}%
\pgfpathcurveto{\pgfqpoint{0.978146in}{0.660935in}}{\pgfqpoint{0.982536in}{0.650336in}}{\pgfqpoint{0.990350in}{0.642522in}}%
\pgfpathcurveto{\pgfqpoint{0.998163in}{0.634709in}}{\pgfqpoint{1.008762in}{0.630318in}}{\pgfqpoint{1.019812in}{0.630318in}}%
\pgfpathclose%
\pgfusepath{stroke,fill}%
\end{pgfscope}%
\begin{pgfscope}%
\pgfpathrectangle{\pgfqpoint{0.375000in}{0.330000in}}{\pgfqpoint{2.325000in}{2.310000in}}%
\pgfusepath{clip}%
\pgfsetbuttcap%
\pgfsetroundjoin%
\definecolor{currentfill}{rgb}{0.000000,0.000000,0.000000}%
\pgfsetfillcolor{currentfill}%
\pgfsetlinewidth{1.003750pt}%
\definecolor{currentstroke}{rgb}{0.000000,0.000000,0.000000}%
\pgfsetstrokecolor{currentstroke}%
\pgfsetdash{}{0pt}%
\pgfpathmoveto{\pgfqpoint{1.019812in}{0.672795in}}%
\pgfpathcurveto{\pgfqpoint{1.030863in}{0.672795in}}{\pgfqpoint{1.041462in}{0.677185in}}{\pgfqpoint{1.049275in}{0.684999in}}%
\pgfpathcurveto{\pgfqpoint{1.057089in}{0.692813in}}{\pgfqpoint{1.061479in}{0.703412in}}{\pgfqpoint{1.061479in}{0.714462in}}%
\pgfpathcurveto{\pgfqpoint{1.061479in}{0.725512in}}{\pgfqpoint{1.057089in}{0.736111in}}{\pgfqpoint{1.049275in}{0.743925in}}%
\pgfpathcurveto{\pgfqpoint{1.041462in}{0.751738in}}{\pgfqpoint{1.030863in}{0.756129in}}{\pgfqpoint{1.019812in}{0.756129in}}%
\pgfpathcurveto{\pgfqpoint{1.008762in}{0.756129in}}{\pgfqpoint{0.998163in}{0.751738in}}{\pgfqpoint{0.990350in}{0.743925in}}%
\pgfpathcurveto{\pgfqpoint{0.982536in}{0.736111in}}{\pgfqpoint{0.978146in}{0.725512in}}{\pgfqpoint{0.978146in}{0.714462in}}%
\pgfpathcurveto{\pgfqpoint{0.978146in}{0.703412in}}{\pgfqpoint{0.982536in}{0.692813in}}{\pgfqpoint{0.990350in}{0.684999in}}%
\pgfpathcurveto{\pgfqpoint{0.998163in}{0.677185in}}{\pgfqpoint{1.008762in}{0.672795in}}{\pgfqpoint{1.019812in}{0.672795in}}%
\pgfpathclose%
\pgfusepath{stroke,fill}%
\end{pgfscope}%
\begin{pgfscope}%
\pgfpathrectangle{\pgfqpoint{0.375000in}{0.330000in}}{\pgfqpoint{2.325000in}{2.310000in}}%
\pgfusepath{clip}%
\pgfsetbuttcap%
\pgfsetroundjoin%
\definecolor{currentfill}{rgb}{0.000000,0.000000,0.000000}%
\pgfsetfillcolor{currentfill}%
\pgfsetlinewidth{1.003750pt}%
\definecolor{currentstroke}{rgb}{0.000000,0.000000,0.000000}%
\pgfsetstrokecolor{currentstroke}%
\pgfsetdash{}{0pt}%
\pgfpathmoveto{\pgfqpoint{1.019812in}{0.703136in}}%
\pgfpathcurveto{\pgfqpoint{1.030863in}{0.703136in}}{\pgfqpoint{1.041462in}{0.707526in}}{\pgfqpoint{1.049275in}{0.715340in}}%
\pgfpathcurveto{\pgfqpoint{1.057089in}{0.723153in}}{\pgfqpoint{1.061479in}{0.733752in}}{\pgfqpoint{1.061479in}{0.744802in}}%
\pgfpathcurveto{\pgfqpoint{1.061479in}{0.755853in}}{\pgfqpoint{1.057089in}{0.766452in}}{\pgfqpoint{1.049275in}{0.774265in}}%
\pgfpathcurveto{\pgfqpoint{1.041462in}{0.782079in}}{\pgfqpoint{1.030863in}{0.786469in}}{\pgfqpoint{1.019812in}{0.786469in}}%
\pgfpathcurveto{\pgfqpoint{1.008762in}{0.786469in}}{\pgfqpoint{0.998163in}{0.782079in}}{\pgfqpoint{0.990350in}{0.774265in}}%
\pgfpathcurveto{\pgfqpoint{0.982536in}{0.766452in}}{\pgfqpoint{0.978146in}{0.755853in}}{\pgfqpoint{0.978146in}{0.744802in}}%
\pgfpathcurveto{\pgfqpoint{0.978146in}{0.733752in}}{\pgfqpoint{0.982536in}{0.723153in}}{\pgfqpoint{0.990350in}{0.715340in}}%
\pgfpathcurveto{\pgfqpoint{0.998163in}{0.707526in}}{\pgfqpoint{1.008762in}{0.703136in}}{\pgfqpoint{1.019812in}{0.703136in}}%
\pgfpathclose%
\pgfusepath{stroke,fill}%
\end{pgfscope}%
\begin{pgfscope}%
\pgfpathrectangle{\pgfqpoint{0.375000in}{0.330000in}}{\pgfqpoint{2.325000in}{2.310000in}}%
\pgfusepath{clip}%
\pgfsetbuttcap%
\pgfsetroundjoin%
\definecolor{currentfill}{rgb}{0.000000,0.000000,0.000000}%
\pgfsetfillcolor{currentfill}%
\pgfsetlinewidth{1.003750pt}%
\definecolor{currentstroke}{rgb}{0.000000,0.000000,0.000000}%
\pgfsetstrokecolor{currentstroke}%
\pgfsetdash{}{0pt}%
\pgfpathmoveto{\pgfqpoint{1.019812in}{0.636387in}}%
\pgfpathcurveto{\pgfqpoint{1.030863in}{0.636387in}}{\pgfqpoint{1.041462in}{0.640777in}}{\pgfqpoint{1.049275in}{0.648590in}}%
\pgfpathcurveto{\pgfqpoint{1.057089in}{0.656404in}}{\pgfqpoint{1.061479in}{0.667003in}}{\pgfqpoint{1.061479in}{0.678053in}}%
\pgfpathcurveto{\pgfqpoint{1.061479in}{0.689103in}}{\pgfqpoint{1.057089in}{0.699702in}}{\pgfqpoint{1.049275in}{0.707516in}}%
\pgfpathcurveto{\pgfqpoint{1.041462in}{0.715330in}}{\pgfqpoint{1.030863in}{0.719720in}}{\pgfqpoint{1.019812in}{0.719720in}}%
\pgfpathcurveto{\pgfqpoint{1.008762in}{0.719720in}}{\pgfqpoint{0.998163in}{0.715330in}}{\pgfqpoint{0.990350in}{0.707516in}}%
\pgfpathcurveto{\pgfqpoint{0.982536in}{0.699702in}}{\pgfqpoint{0.978146in}{0.689103in}}{\pgfqpoint{0.978146in}{0.678053in}}%
\pgfpathcurveto{\pgfqpoint{0.978146in}{0.667003in}}{\pgfqpoint{0.982536in}{0.656404in}}{\pgfqpoint{0.990350in}{0.648590in}}%
\pgfpathcurveto{\pgfqpoint{0.998163in}{0.640777in}}{\pgfqpoint{1.008762in}{0.636387in}}{\pgfqpoint{1.019812in}{0.636387in}}%
\pgfpathclose%
\pgfusepath{stroke,fill}%
\end{pgfscope}%
\begin{pgfscope}%
\pgfpathrectangle{\pgfqpoint{0.375000in}{0.330000in}}{\pgfqpoint{2.325000in}{2.310000in}}%
\pgfusepath{clip}%
\pgfsetbuttcap%
\pgfsetroundjoin%
\definecolor{currentfill}{rgb}{0.000000,0.000000,0.000000}%
\pgfsetfillcolor{currentfill}%
\pgfsetlinewidth{1.003750pt}%
\definecolor{currentstroke}{rgb}{0.000000,0.000000,0.000000}%
\pgfsetstrokecolor{currentstroke}%
\pgfsetdash{}{0pt}%
\pgfpathmoveto{\pgfqpoint{1.019812in}{0.612114in}}%
\pgfpathcurveto{\pgfqpoint{1.030863in}{0.612114in}}{\pgfqpoint{1.041462in}{0.616504in}}{\pgfqpoint{1.049275in}{0.624318in}}%
\pgfpathcurveto{\pgfqpoint{1.057089in}{0.632132in}}{\pgfqpoint{1.061479in}{0.642731in}}{\pgfqpoint{1.061479in}{0.653781in}}%
\pgfpathcurveto{\pgfqpoint{1.061479in}{0.664831in}}{\pgfqpoint{1.057089in}{0.675430in}}{\pgfqpoint{1.049275in}{0.683244in}}%
\pgfpathcurveto{\pgfqpoint{1.041462in}{0.691057in}}{\pgfqpoint{1.030863in}{0.695447in}}{\pgfqpoint{1.019812in}{0.695447in}}%
\pgfpathcurveto{\pgfqpoint{1.008762in}{0.695447in}}{\pgfqpoint{0.998163in}{0.691057in}}{\pgfqpoint{0.990350in}{0.683244in}}%
\pgfpathcurveto{\pgfqpoint{0.982536in}{0.675430in}}{\pgfqpoint{0.978146in}{0.664831in}}{\pgfqpoint{0.978146in}{0.653781in}}%
\pgfpathcurveto{\pgfqpoint{0.978146in}{0.642731in}}{\pgfqpoint{0.982536in}{0.632132in}}{\pgfqpoint{0.990350in}{0.624318in}}%
\pgfpathcurveto{\pgfqpoint{0.998163in}{0.616504in}}{\pgfqpoint{1.008762in}{0.612114in}}{\pgfqpoint{1.019812in}{0.612114in}}%
\pgfpathclose%
\pgfusepath{stroke,fill}%
\end{pgfscope}%
\begin{pgfscope}%
\pgfpathrectangle{\pgfqpoint{0.375000in}{0.330000in}}{\pgfqpoint{2.325000in}{2.310000in}}%
\pgfusepath{clip}%
\pgfsetbuttcap%
\pgfsetroundjoin%
\definecolor{currentfill}{rgb}{0.000000,0.000000,0.000000}%
\pgfsetfillcolor{currentfill}%
\pgfsetlinewidth{1.003750pt}%
\definecolor{currentstroke}{rgb}{0.000000,0.000000,0.000000}%
\pgfsetstrokecolor{currentstroke}%
\pgfsetdash{}{0pt}%
\pgfpathmoveto{\pgfqpoint{1.019812in}{0.666727in}}%
\pgfpathcurveto{\pgfqpoint{1.030863in}{0.666727in}}{\pgfqpoint{1.041462in}{0.671117in}}{\pgfqpoint{1.049275in}{0.678931in}}%
\pgfpathcurveto{\pgfqpoint{1.057089in}{0.686745in}}{\pgfqpoint{1.061479in}{0.697344in}}{\pgfqpoint{1.061479in}{0.708394in}}%
\pgfpathcurveto{\pgfqpoint{1.061479in}{0.719444in}}{\pgfqpoint{1.057089in}{0.730043in}}{\pgfqpoint{1.049275in}{0.737857in}}%
\pgfpathcurveto{\pgfqpoint{1.041462in}{0.745670in}}{\pgfqpoint{1.030863in}{0.750060in}}{\pgfqpoint{1.019812in}{0.750060in}}%
\pgfpathcurveto{\pgfqpoint{1.008762in}{0.750060in}}{\pgfqpoint{0.998163in}{0.745670in}}{\pgfqpoint{0.990350in}{0.737857in}}%
\pgfpathcurveto{\pgfqpoint{0.982536in}{0.730043in}}{\pgfqpoint{0.978146in}{0.719444in}}{\pgfqpoint{0.978146in}{0.708394in}}%
\pgfpathcurveto{\pgfqpoint{0.978146in}{0.697344in}}{\pgfqpoint{0.982536in}{0.686745in}}{\pgfqpoint{0.990350in}{0.678931in}}%
\pgfpathcurveto{\pgfqpoint{0.998163in}{0.671117in}}{\pgfqpoint{1.008762in}{0.666727in}}{\pgfqpoint{1.019812in}{0.666727in}}%
\pgfpathclose%
\pgfusepath{stroke,fill}%
\end{pgfscope}%
\begin{pgfscope}%
\pgfpathrectangle{\pgfqpoint{0.375000in}{0.330000in}}{\pgfqpoint{2.325000in}{2.310000in}}%
\pgfusepath{clip}%
\pgfsetbuttcap%
\pgfsetroundjoin%
\definecolor{currentfill}{rgb}{0.000000,0.000000,0.000000}%
\pgfsetfillcolor{currentfill}%
\pgfsetlinewidth{1.003750pt}%
\definecolor{currentstroke}{rgb}{0.000000,0.000000,0.000000}%
\pgfsetstrokecolor{currentstroke}%
\pgfsetdash{}{0pt}%
\pgfpathmoveto{\pgfqpoint{1.019812in}{0.678863in}}%
\pgfpathcurveto{\pgfqpoint{1.030863in}{0.678863in}}{\pgfqpoint{1.041462in}{0.683254in}}{\pgfqpoint{1.049275in}{0.691067in}}%
\pgfpathcurveto{\pgfqpoint{1.057089in}{0.698881in}}{\pgfqpoint{1.061479in}{0.709480in}}{\pgfqpoint{1.061479in}{0.720530in}}%
\pgfpathcurveto{\pgfqpoint{1.061479in}{0.731580in}}{\pgfqpoint{1.057089in}{0.742179in}}{\pgfqpoint{1.049275in}{0.749993in}}%
\pgfpathcurveto{\pgfqpoint{1.041462in}{0.757806in}}{\pgfqpoint{1.030863in}{0.762197in}}{\pgfqpoint{1.019812in}{0.762197in}}%
\pgfpathcurveto{\pgfqpoint{1.008762in}{0.762197in}}{\pgfqpoint{0.998163in}{0.757806in}}{\pgfqpoint{0.990350in}{0.749993in}}%
\pgfpathcurveto{\pgfqpoint{0.982536in}{0.742179in}}{\pgfqpoint{0.978146in}{0.731580in}}{\pgfqpoint{0.978146in}{0.720530in}}%
\pgfpathcurveto{\pgfqpoint{0.978146in}{0.709480in}}{\pgfqpoint{0.982536in}{0.698881in}}{\pgfqpoint{0.990350in}{0.691067in}}%
\pgfpathcurveto{\pgfqpoint{0.998163in}{0.683254in}}{\pgfqpoint{1.008762in}{0.678863in}}{\pgfqpoint{1.019812in}{0.678863in}}%
\pgfpathclose%
\pgfusepath{stroke,fill}%
\end{pgfscope}%
\begin{pgfscope}%
\pgfpathrectangle{\pgfqpoint{0.375000in}{0.330000in}}{\pgfqpoint{2.325000in}{2.310000in}}%
\pgfusepath{clip}%
\pgfsetbuttcap%
\pgfsetroundjoin%
\definecolor{currentfill}{rgb}{0.000000,0.000000,0.000000}%
\pgfsetfillcolor{currentfill}%
\pgfsetlinewidth{1.003750pt}%
\definecolor{currentstroke}{rgb}{0.000000,0.000000,0.000000}%
\pgfsetstrokecolor{currentstroke}%
\pgfsetdash{}{0pt}%
\pgfpathmoveto{\pgfqpoint{1.019812in}{0.612114in}}%
\pgfpathcurveto{\pgfqpoint{1.030863in}{0.612114in}}{\pgfqpoint{1.041462in}{0.616504in}}{\pgfqpoint{1.049275in}{0.624318in}}%
\pgfpathcurveto{\pgfqpoint{1.057089in}{0.632132in}}{\pgfqpoint{1.061479in}{0.642731in}}{\pgfqpoint{1.061479in}{0.653781in}}%
\pgfpathcurveto{\pgfqpoint{1.061479in}{0.664831in}}{\pgfqpoint{1.057089in}{0.675430in}}{\pgfqpoint{1.049275in}{0.683244in}}%
\pgfpathcurveto{\pgfqpoint{1.041462in}{0.691057in}}{\pgfqpoint{1.030863in}{0.695447in}}{\pgfqpoint{1.019812in}{0.695447in}}%
\pgfpathcurveto{\pgfqpoint{1.008762in}{0.695447in}}{\pgfqpoint{0.998163in}{0.691057in}}{\pgfqpoint{0.990350in}{0.683244in}}%
\pgfpathcurveto{\pgfqpoint{0.982536in}{0.675430in}}{\pgfqpoint{0.978146in}{0.664831in}}{\pgfqpoint{0.978146in}{0.653781in}}%
\pgfpathcurveto{\pgfqpoint{0.978146in}{0.642731in}}{\pgfqpoint{0.982536in}{0.632132in}}{\pgfqpoint{0.990350in}{0.624318in}}%
\pgfpathcurveto{\pgfqpoint{0.998163in}{0.616504in}}{\pgfqpoint{1.008762in}{0.612114in}}{\pgfqpoint{1.019812in}{0.612114in}}%
\pgfpathclose%
\pgfusepath{stroke,fill}%
\end{pgfscope}%
\begin{pgfscope}%
\pgfpathrectangle{\pgfqpoint{0.375000in}{0.330000in}}{\pgfqpoint{2.325000in}{2.310000in}}%
\pgfusepath{clip}%
\pgfsetbuttcap%
\pgfsetroundjoin%
\definecolor{currentfill}{rgb}{0.000000,0.000000,0.000000}%
\pgfsetfillcolor{currentfill}%
\pgfsetlinewidth{1.003750pt}%
\definecolor{currentstroke}{rgb}{0.000000,0.000000,0.000000}%
\pgfsetstrokecolor{currentstroke}%
\pgfsetdash{}{0pt}%
\pgfpathmoveto{\pgfqpoint{1.019812in}{0.575706in}}%
\pgfpathcurveto{\pgfqpoint{1.030863in}{0.575706in}}{\pgfqpoint{1.041462in}{0.580096in}}{\pgfqpoint{1.049275in}{0.587909in}}%
\pgfpathcurveto{\pgfqpoint{1.057089in}{0.595723in}}{\pgfqpoint{1.061479in}{0.606322in}}{\pgfqpoint{1.061479in}{0.617372in}}%
\pgfpathcurveto{\pgfqpoint{1.061479in}{0.628422in}}{\pgfqpoint{1.057089in}{0.639021in}}{\pgfqpoint{1.049275in}{0.646835in}}%
\pgfpathcurveto{\pgfqpoint{1.041462in}{0.654649in}}{\pgfqpoint{1.030863in}{0.659039in}}{\pgfqpoint{1.019812in}{0.659039in}}%
\pgfpathcurveto{\pgfqpoint{1.008762in}{0.659039in}}{\pgfqpoint{0.998163in}{0.654649in}}{\pgfqpoint{0.990350in}{0.646835in}}%
\pgfpathcurveto{\pgfqpoint{0.982536in}{0.639021in}}{\pgfqpoint{0.978146in}{0.628422in}}{\pgfqpoint{0.978146in}{0.617372in}}%
\pgfpathcurveto{\pgfqpoint{0.978146in}{0.606322in}}{\pgfqpoint{0.982536in}{0.595723in}}{\pgfqpoint{0.990350in}{0.587909in}}%
\pgfpathcurveto{\pgfqpoint{0.998163in}{0.580096in}}{\pgfqpoint{1.008762in}{0.575706in}}{\pgfqpoint{1.019812in}{0.575706in}}%
\pgfpathclose%
\pgfusepath{stroke,fill}%
\end{pgfscope}%
\begin{pgfscope}%
\pgfpathrectangle{\pgfqpoint{0.375000in}{0.330000in}}{\pgfqpoint{2.325000in}{2.310000in}}%
\pgfusepath{clip}%
\pgfsetbuttcap%
\pgfsetroundjoin%
\definecolor{currentfill}{rgb}{0.000000,0.000000,0.000000}%
\pgfsetfillcolor{currentfill}%
\pgfsetlinewidth{1.003750pt}%
\definecolor{currentstroke}{rgb}{0.000000,0.000000,0.000000}%
\pgfsetstrokecolor{currentstroke}%
\pgfsetdash{}{0pt}%
\pgfpathmoveto{\pgfqpoint{1.019812in}{0.575706in}}%
\pgfpathcurveto{\pgfqpoint{1.030863in}{0.575706in}}{\pgfqpoint{1.041462in}{0.580096in}}{\pgfqpoint{1.049275in}{0.587909in}}%
\pgfpathcurveto{\pgfqpoint{1.057089in}{0.595723in}}{\pgfqpoint{1.061479in}{0.606322in}}{\pgfqpoint{1.061479in}{0.617372in}}%
\pgfpathcurveto{\pgfqpoint{1.061479in}{0.628422in}}{\pgfqpoint{1.057089in}{0.639021in}}{\pgfqpoint{1.049275in}{0.646835in}}%
\pgfpathcurveto{\pgfqpoint{1.041462in}{0.654649in}}{\pgfqpoint{1.030863in}{0.659039in}}{\pgfqpoint{1.019812in}{0.659039in}}%
\pgfpathcurveto{\pgfqpoint{1.008762in}{0.659039in}}{\pgfqpoint{0.998163in}{0.654649in}}{\pgfqpoint{0.990350in}{0.646835in}}%
\pgfpathcurveto{\pgfqpoint{0.982536in}{0.639021in}}{\pgfqpoint{0.978146in}{0.628422in}}{\pgfqpoint{0.978146in}{0.617372in}}%
\pgfpathcurveto{\pgfqpoint{0.978146in}{0.606322in}}{\pgfqpoint{0.982536in}{0.595723in}}{\pgfqpoint{0.990350in}{0.587909in}}%
\pgfpathcurveto{\pgfqpoint{0.998163in}{0.580096in}}{\pgfqpoint{1.008762in}{0.575706in}}{\pgfqpoint{1.019812in}{0.575706in}}%
\pgfpathclose%
\pgfusepath{stroke,fill}%
\end{pgfscope}%
\begin{pgfscope}%
\pgfpathrectangle{\pgfqpoint{0.375000in}{0.330000in}}{\pgfqpoint{2.325000in}{2.310000in}}%
\pgfusepath{clip}%
\pgfsetbuttcap%
\pgfsetroundjoin%
\definecolor{currentfill}{rgb}{0.000000,0.000000,0.000000}%
\pgfsetfillcolor{currentfill}%
\pgfsetlinewidth{1.003750pt}%
\definecolor{currentstroke}{rgb}{0.000000,0.000000,0.000000}%
\pgfsetstrokecolor{currentstroke}%
\pgfsetdash{}{0pt}%
\pgfpathmoveto{\pgfqpoint{1.019812in}{0.630318in}}%
\pgfpathcurveto{\pgfqpoint{1.030863in}{0.630318in}}{\pgfqpoint{1.041462in}{0.634709in}}{\pgfqpoint{1.049275in}{0.642522in}}%
\pgfpathcurveto{\pgfqpoint{1.057089in}{0.650336in}}{\pgfqpoint{1.061479in}{0.660935in}}{\pgfqpoint{1.061479in}{0.671985in}}%
\pgfpathcurveto{\pgfqpoint{1.061479in}{0.683035in}}{\pgfqpoint{1.057089in}{0.693634in}}{\pgfqpoint{1.049275in}{0.701448in}}%
\pgfpathcurveto{\pgfqpoint{1.041462in}{0.709262in}}{\pgfqpoint{1.030863in}{0.713652in}}{\pgfqpoint{1.019812in}{0.713652in}}%
\pgfpathcurveto{\pgfqpoint{1.008762in}{0.713652in}}{\pgfqpoint{0.998163in}{0.709262in}}{\pgfqpoint{0.990350in}{0.701448in}}%
\pgfpathcurveto{\pgfqpoint{0.982536in}{0.693634in}}{\pgfqpoint{0.978146in}{0.683035in}}{\pgfqpoint{0.978146in}{0.671985in}}%
\pgfpathcurveto{\pgfqpoint{0.978146in}{0.660935in}}{\pgfqpoint{0.982536in}{0.650336in}}{\pgfqpoint{0.990350in}{0.642522in}}%
\pgfpathcurveto{\pgfqpoint{0.998163in}{0.634709in}}{\pgfqpoint{1.008762in}{0.630318in}}{\pgfqpoint{1.019812in}{0.630318in}}%
\pgfpathclose%
\pgfusepath{stroke,fill}%
\end{pgfscope}%
\begin{pgfscope}%
\pgfpathrectangle{\pgfqpoint{0.375000in}{0.330000in}}{\pgfqpoint{2.325000in}{2.310000in}}%
\pgfusepath{clip}%
\pgfsetbuttcap%
\pgfsetroundjoin%
\definecolor{currentfill}{rgb}{0.000000,0.000000,0.000000}%
\pgfsetfillcolor{currentfill}%
\pgfsetlinewidth{1.003750pt}%
\definecolor{currentstroke}{rgb}{0.000000,0.000000,0.000000}%
\pgfsetstrokecolor{currentstroke}%
\pgfsetdash{}{0pt}%
\pgfpathmoveto{\pgfqpoint{1.019812in}{0.666727in}}%
\pgfpathcurveto{\pgfqpoint{1.030863in}{0.666727in}}{\pgfqpoint{1.041462in}{0.671117in}}{\pgfqpoint{1.049275in}{0.678931in}}%
\pgfpathcurveto{\pgfqpoint{1.057089in}{0.686745in}}{\pgfqpoint{1.061479in}{0.697344in}}{\pgfqpoint{1.061479in}{0.708394in}}%
\pgfpathcurveto{\pgfqpoint{1.061479in}{0.719444in}}{\pgfqpoint{1.057089in}{0.730043in}}{\pgfqpoint{1.049275in}{0.737857in}}%
\pgfpathcurveto{\pgfqpoint{1.041462in}{0.745670in}}{\pgfqpoint{1.030863in}{0.750060in}}{\pgfqpoint{1.019812in}{0.750060in}}%
\pgfpathcurveto{\pgfqpoint{1.008762in}{0.750060in}}{\pgfqpoint{0.998163in}{0.745670in}}{\pgfqpoint{0.990350in}{0.737857in}}%
\pgfpathcurveto{\pgfqpoint{0.982536in}{0.730043in}}{\pgfqpoint{0.978146in}{0.719444in}}{\pgfqpoint{0.978146in}{0.708394in}}%
\pgfpathcurveto{\pgfqpoint{0.978146in}{0.697344in}}{\pgfqpoint{0.982536in}{0.686745in}}{\pgfqpoint{0.990350in}{0.678931in}}%
\pgfpathcurveto{\pgfqpoint{0.998163in}{0.671117in}}{\pgfqpoint{1.008762in}{0.666727in}}{\pgfqpoint{1.019812in}{0.666727in}}%
\pgfpathclose%
\pgfusepath{stroke,fill}%
\end{pgfscope}%
\begin{pgfscope}%
\pgfpathrectangle{\pgfqpoint{0.375000in}{0.330000in}}{\pgfqpoint{2.325000in}{2.310000in}}%
\pgfusepath{clip}%
\pgfsetbuttcap%
\pgfsetroundjoin%
\definecolor{currentfill}{rgb}{0.000000,0.000000,0.000000}%
\pgfsetfillcolor{currentfill}%
\pgfsetlinewidth{1.003750pt}%
\definecolor{currentstroke}{rgb}{0.000000,0.000000,0.000000}%
\pgfsetstrokecolor{currentstroke}%
\pgfsetdash{}{0pt}%
\pgfpathmoveto{\pgfqpoint{1.019812in}{0.684931in}}%
\pgfpathcurveto{\pgfqpoint{1.030863in}{0.684931in}}{\pgfqpoint{1.041462in}{0.689322in}}{\pgfqpoint{1.049275in}{0.697135in}}%
\pgfpathcurveto{\pgfqpoint{1.057089in}{0.704949in}}{\pgfqpoint{1.061479in}{0.715548in}}{\pgfqpoint{1.061479in}{0.726598in}}%
\pgfpathcurveto{\pgfqpoint{1.061479in}{0.737648in}}{\pgfqpoint{1.057089in}{0.748247in}}{\pgfqpoint{1.049275in}{0.756061in}}%
\pgfpathcurveto{\pgfqpoint{1.041462in}{0.763874in}}{\pgfqpoint{1.030863in}{0.768265in}}{\pgfqpoint{1.019812in}{0.768265in}}%
\pgfpathcurveto{\pgfqpoint{1.008762in}{0.768265in}}{\pgfqpoint{0.998163in}{0.763874in}}{\pgfqpoint{0.990350in}{0.756061in}}%
\pgfpathcurveto{\pgfqpoint{0.982536in}{0.748247in}}{\pgfqpoint{0.978146in}{0.737648in}}{\pgfqpoint{0.978146in}{0.726598in}}%
\pgfpathcurveto{\pgfqpoint{0.978146in}{0.715548in}}{\pgfqpoint{0.982536in}{0.704949in}}{\pgfqpoint{0.990350in}{0.697135in}}%
\pgfpathcurveto{\pgfqpoint{0.998163in}{0.689322in}}{\pgfqpoint{1.008762in}{0.684931in}}{\pgfqpoint{1.019812in}{0.684931in}}%
\pgfpathclose%
\pgfusepath{stroke,fill}%
\end{pgfscope}%
\begin{pgfscope}%
\pgfpathrectangle{\pgfqpoint{0.375000in}{0.330000in}}{\pgfqpoint{2.325000in}{2.310000in}}%
\pgfusepath{clip}%
\pgfsetbuttcap%
\pgfsetroundjoin%
\definecolor{currentfill}{rgb}{0.000000,0.000000,0.000000}%
\pgfsetfillcolor{currentfill}%
\pgfsetlinewidth{1.003750pt}%
\definecolor{currentstroke}{rgb}{0.000000,0.000000,0.000000}%
\pgfsetstrokecolor{currentstroke}%
\pgfsetdash{}{0pt}%
\pgfpathmoveto{\pgfqpoint{1.019812in}{0.642455in}}%
\pgfpathcurveto{\pgfqpoint{1.030863in}{0.642455in}}{\pgfqpoint{1.041462in}{0.646845in}}{\pgfqpoint{1.049275in}{0.654659in}}%
\pgfpathcurveto{\pgfqpoint{1.057089in}{0.662472in}}{\pgfqpoint{1.061479in}{0.673071in}}{\pgfqpoint{1.061479in}{0.684121in}}%
\pgfpathcurveto{\pgfqpoint{1.061479in}{0.695171in}}{\pgfqpoint{1.057089in}{0.705770in}}{\pgfqpoint{1.049275in}{0.713584in}}%
\pgfpathcurveto{\pgfqpoint{1.041462in}{0.721398in}}{\pgfqpoint{1.030863in}{0.725788in}}{\pgfqpoint{1.019812in}{0.725788in}}%
\pgfpathcurveto{\pgfqpoint{1.008762in}{0.725788in}}{\pgfqpoint{0.998163in}{0.721398in}}{\pgfqpoint{0.990350in}{0.713584in}}%
\pgfpathcurveto{\pgfqpoint{0.982536in}{0.705770in}}{\pgfqpoint{0.978146in}{0.695171in}}{\pgfqpoint{0.978146in}{0.684121in}}%
\pgfpathcurveto{\pgfqpoint{0.978146in}{0.673071in}}{\pgfqpoint{0.982536in}{0.662472in}}{\pgfqpoint{0.990350in}{0.654659in}}%
\pgfpathcurveto{\pgfqpoint{0.998163in}{0.646845in}}{\pgfqpoint{1.008762in}{0.642455in}}{\pgfqpoint{1.019812in}{0.642455in}}%
\pgfpathclose%
\pgfusepath{stroke,fill}%
\end{pgfscope}%
\begin{pgfscope}%
\pgfpathrectangle{\pgfqpoint{0.375000in}{0.330000in}}{\pgfqpoint{2.325000in}{2.310000in}}%
\pgfusepath{clip}%
\pgfsetbuttcap%
\pgfsetroundjoin%
\definecolor{currentfill}{rgb}{0.000000,0.000000,0.000000}%
\pgfsetfillcolor{currentfill}%
\pgfsetlinewidth{1.003750pt}%
\definecolor{currentstroke}{rgb}{0.000000,0.000000,0.000000}%
\pgfsetstrokecolor{currentstroke}%
\pgfsetdash{}{0pt}%
\pgfpathmoveto{\pgfqpoint{1.019812in}{0.691000in}}%
\pgfpathcurveto{\pgfqpoint{1.030863in}{0.691000in}}{\pgfqpoint{1.041462in}{0.695390in}}{\pgfqpoint{1.049275in}{0.703203in}}%
\pgfpathcurveto{\pgfqpoint{1.057089in}{0.711017in}}{\pgfqpoint{1.061479in}{0.721616in}}{\pgfqpoint{1.061479in}{0.732666in}}%
\pgfpathcurveto{\pgfqpoint{1.061479in}{0.743716in}}{\pgfqpoint{1.057089in}{0.754315in}}{\pgfqpoint{1.049275in}{0.762129in}}%
\pgfpathcurveto{\pgfqpoint{1.041462in}{0.769943in}}{\pgfqpoint{1.030863in}{0.774333in}}{\pgfqpoint{1.019812in}{0.774333in}}%
\pgfpathcurveto{\pgfqpoint{1.008762in}{0.774333in}}{\pgfqpoint{0.998163in}{0.769943in}}{\pgfqpoint{0.990350in}{0.762129in}}%
\pgfpathcurveto{\pgfqpoint{0.982536in}{0.754315in}}{\pgfqpoint{0.978146in}{0.743716in}}{\pgfqpoint{0.978146in}{0.732666in}}%
\pgfpathcurveto{\pgfqpoint{0.978146in}{0.721616in}}{\pgfqpoint{0.982536in}{0.711017in}}{\pgfqpoint{0.990350in}{0.703203in}}%
\pgfpathcurveto{\pgfqpoint{0.998163in}{0.695390in}}{\pgfqpoint{1.008762in}{0.691000in}}{\pgfqpoint{1.019812in}{0.691000in}}%
\pgfpathclose%
\pgfusepath{stroke,fill}%
\end{pgfscope}%
\begin{pgfscope}%
\pgfpathrectangle{\pgfqpoint{0.375000in}{0.330000in}}{\pgfqpoint{2.325000in}{2.310000in}}%
\pgfusepath{clip}%
\pgfsetbuttcap%
\pgfsetroundjoin%
\definecolor{currentfill}{rgb}{0.000000,0.000000,0.000000}%
\pgfsetfillcolor{currentfill}%
\pgfsetlinewidth{1.003750pt}%
\definecolor{currentstroke}{rgb}{0.000000,0.000000,0.000000}%
\pgfsetstrokecolor{currentstroke}%
\pgfsetdash{}{0pt}%
\pgfpathmoveto{\pgfqpoint{1.019812in}{0.624250in}}%
\pgfpathcurveto{\pgfqpoint{1.030863in}{0.624250in}}{\pgfqpoint{1.041462in}{0.628641in}}{\pgfqpoint{1.049275in}{0.636454in}}%
\pgfpathcurveto{\pgfqpoint{1.057089in}{0.644268in}}{\pgfqpoint{1.061479in}{0.654867in}}{\pgfqpoint{1.061479in}{0.665917in}}%
\pgfpathcurveto{\pgfqpoint{1.061479in}{0.676967in}}{\pgfqpoint{1.057089in}{0.687566in}}{\pgfqpoint{1.049275in}{0.695380in}}%
\pgfpathcurveto{\pgfqpoint{1.041462in}{0.703193in}}{\pgfqpoint{1.030863in}{0.707584in}}{\pgfqpoint{1.019812in}{0.707584in}}%
\pgfpathcurveto{\pgfqpoint{1.008762in}{0.707584in}}{\pgfqpoint{0.998163in}{0.703193in}}{\pgfqpoint{0.990350in}{0.695380in}}%
\pgfpathcurveto{\pgfqpoint{0.982536in}{0.687566in}}{\pgfqpoint{0.978146in}{0.676967in}}{\pgfqpoint{0.978146in}{0.665917in}}%
\pgfpathcurveto{\pgfqpoint{0.978146in}{0.654867in}}{\pgfqpoint{0.982536in}{0.644268in}}{\pgfqpoint{0.990350in}{0.636454in}}%
\pgfpathcurveto{\pgfqpoint{0.998163in}{0.628641in}}{\pgfqpoint{1.008762in}{0.624250in}}{\pgfqpoint{1.019812in}{0.624250in}}%
\pgfpathclose%
\pgfusepath{stroke,fill}%
\end{pgfscope}%
\begin{pgfscope}%
\pgfpathrectangle{\pgfqpoint{0.375000in}{0.330000in}}{\pgfqpoint{2.325000in}{2.310000in}}%
\pgfusepath{clip}%
\pgfsetbuttcap%
\pgfsetroundjoin%
\definecolor{currentfill}{rgb}{0.000000,0.000000,0.000000}%
\pgfsetfillcolor{currentfill}%
\pgfsetlinewidth{1.003750pt}%
\definecolor{currentstroke}{rgb}{0.000000,0.000000,0.000000}%
\pgfsetstrokecolor{currentstroke}%
\pgfsetdash{}{0pt}%
\pgfpathmoveto{\pgfqpoint{1.019812in}{0.624250in}}%
\pgfpathcurveto{\pgfqpoint{1.030863in}{0.624250in}}{\pgfqpoint{1.041462in}{0.628641in}}{\pgfqpoint{1.049275in}{0.636454in}}%
\pgfpathcurveto{\pgfqpoint{1.057089in}{0.644268in}}{\pgfqpoint{1.061479in}{0.654867in}}{\pgfqpoint{1.061479in}{0.665917in}}%
\pgfpathcurveto{\pgfqpoint{1.061479in}{0.676967in}}{\pgfqpoint{1.057089in}{0.687566in}}{\pgfqpoint{1.049275in}{0.695380in}}%
\pgfpathcurveto{\pgfqpoint{1.041462in}{0.703193in}}{\pgfqpoint{1.030863in}{0.707584in}}{\pgfqpoint{1.019812in}{0.707584in}}%
\pgfpathcurveto{\pgfqpoint{1.008762in}{0.707584in}}{\pgfqpoint{0.998163in}{0.703193in}}{\pgfqpoint{0.990350in}{0.695380in}}%
\pgfpathcurveto{\pgfqpoint{0.982536in}{0.687566in}}{\pgfqpoint{0.978146in}{0.676967in}}{\pgfqpoint{0.978146in}{0.665917in}}%
\pgfpathcurveto{\pgfqpoint{0.978146in}{0.654867in}}{\pgfqpoint{0.982536in}{0.644268in}}{\pgfqpoint{0.990350in}{0.636454in}}%
\pgfpathcurveto{\pgfqpoint{0.998163in}{0.628641in}}{\pgfqpoint{1.008762in}{0.624250in}}{\pgfqpoint{1.019812in}{0.624250in}}%
\pgfpathclose%
\pgfusepath{stroke,fill}%
\end{pgfscope}%
\begin{pgfscope}%
\pgfpathrectangle{\pgfqpoint{0.375000in}{0.330000in}}{\pgfqpoint{2.325000in}{2.310000in}}%
\pgfusepath{clip}%
\pgfsetbuttcap%
\pgfsetroundjoin%
\definecolor{currentfill}{rgb}{0.000000,0.000000,0.000000}%
\pgfsetfillcolor{currentfill}%
\pgfsetlinewidth{1.003750pt}%
\definecolor{currentstroke}{rgb}{0.000000,0.000000,0.000000}%
\pgfsetstrokecolor{currentstroke}%
\pgfsetdash{}{0pt}%
\pgfpathmoveto{\pgfqpoint{1.019812in}{0.587842in}}%
\pgfpathcurveto{\pgfqpoint{1.030863in}{0.587842in}}{\pgfqpoint{1.041462in}{0.592232in}}{\pgfqpoint{1.049275in}{0.600046in}}%
\pgfpathcurveto{\pgfqpoint{1.057089in}{0.607859in}}{\pgfqpoint{1.061479in}{0.618458in}}{\pgfqpoint{1.061479in}{0.629508in}}%
\pgfpathcurveto{\pgfqpoint{1.061479in}{0.640559in}}{\pgfqpoint{1.057089in}{0.651158in}}{\pgfqpoint{1.049275in}{0.658971in}}%
\pgfpathcurveto{\pgfqpoint{1.041462in}{0.666785in}}{\pgfqpoint{1.030863in}{0.671175in}}{\pgfqpoint{1.019812in}{0.671175in}}%
\pgfpathcurveto{\pgfqpoint{1.008762in}{0.671175in}}{\pgfqpoint{0.998163in}{0.666785in}}{\pgfqpoint{0.990350in}{0.658971in}}%
\pgfpathcurveto{\pgfqpoint{0.982536in}{0.651158in}}{\pgfqpoint{0.978146in}{0.640559in}}{\pgfqpoint{0.978146in}{0.629508in}}%
\pgfpathcurveto{\pgfqpoint{0.978146in}{0.618458in}}{\pgfqpoint{0.982536in}{0.607859in}}{\pgfqpoint{0.990350in}{0.600046in}}%
\pgfpathcurveto{\pgfqpoint{0.998163in}{0.592232in}}{\pgfqpoint{1.008762in}{0.587842in}}{\pgfqpoint{1.019812in}{0.587842in}}%
\pgfpathclose%
\pgfusepath{stroke,fill}%
\end{pgfscope}%
\begin{pgfscope}%
\pgfpathrectangle{\pgfqpoint{0.375000in}{0.330000in}}{\pgfqpoint{2.325000in}{2.310000in}}%
\pgfusepath{clip}%
\pgfsetbuttcap%
\pgfsetroundjoin%
\definecolor{currentfill}{rgb}{0.000000,0.000000,0.000000}%
\pgfsetfillcolor{currentfill}%
\pgfsetlinewidth{1.003750pt}%
\definecolor{currentstroke}{rgb}{0.000000,0.000000,0.000000}%
\pgfsetstrokecolor{currentstroke}%
\pgfsetdash{}{0pt}%
\pgfpathmoveto{\pgfqpoint{1.019812in}{0.593910in}}%
\pgfpathcurveto{\pgfqpoint{1.030863in}{0.593910in}}{\pgfqpoint{1.041462in}{0.598300in}}{\pgfqpoint{1.049275in}{0.606114in}}%
\pgfpathcurveto{\pgfqpoint{1.057089in}{0.613927in}}{\pgfqpoint{1.061479in}{0.624526in}}{\pgfqpoint{1.061479in}{0.635576in}}%
\pgfpathcurveto{\pgfqpoint{1.061479in}{0.646627in}}{\pgfqpoint{1.057089in}{0.657226in}}{\pgfqpoint{1.049275in}{0.665039in}}%
\pgfpathcurveto{\pgfqpoint{1.041462in}{0.672853in}}{\pgfqpoint{1.030863in}{0.677243in}}{\pgfqpoint{1.019812in}{0.677243in}}%
\pgfpathcurveto{\pgfqpoint{1.008762in}{0.677243in}}{\pgfqpoint{0.998163in}{0.672853in}}{\pgfqpoint{0.990350in}{0.665039in}}%
\pgfpathcurveto{\pgfqpoint{0.982536in}{0.657226in}}{\pgfqpoint{0.978146in}{0.646627in}}{\pgfqpoint{0.978146in}{0.635576in}}%
\pgfpathcurveto{\pgfqpoint{0.978146in}{0.624526in}}{\pgfqpoint{0.982536in}{0.613927in}}{\pgfqpoint{0.990350in}{0.606114in}}%
\pgfpathcurveto{\pgfqpoint{0.998163in}{0.598300in}}{\pgfqpoint{1.008762in}{0.593910in}}{\pgfqpoint{1.019812in}{0.593910in}}%
\pgfpathclose%
\pgfusepath{stroke,fill}%
\end{pgfscope}%
\begin{pgfscope}%
\pgfpathrectangle{\pgfqpoint{0.375000in}{0.330000in}}{\pgfqpoint{2.325000in}{2.310000in}}%
\pgfusepath{clip}%
\pgfsetbuttcap%
\pgfsetroundjoin%
\definecolor{currentfill}{rgb}{0.000000,0.000000,0.000000}%
\pgfsetfillcolor{currentfill}%
\pgfsetlinewidth{1.003750pt}%
\definecolor{currentstroke}{rgb}{0.000000,0.000000,0.000000}%
\pgfsetstrokecolor{currentstroke}%
\pgfsetdash{}{0pt}%
\pgfpathmoveto{\pgfqpoint{1.019812in}{0.599978in}}%
\pgfpathcurveto{\pgfqpoint{1.030863in}{0.599978in}}{\pgfqpoint{1.041462in}{0.604368in}}{\pgfqpoint{1.049275in}{0.612182in}}%
\pgfpathcurveto{\pgfqpoint{1.057089in}{0.619995in}}{\pgfqpoint{1.061479in}{0.630594in}}{\pgfqpoint{1.061479in}{0.641645in}}%
\pgfpathcurveto{\pgfqpoint{1.061479in}{0.652695in}}{\pgfqpoint{1.057089in}{0.663294in}}{\pgfqpoint{1.049275in}{0.671107in}}%
\pgfpathcurveto{\pgfqpoint{1.041462in}{0.678921in}}{\pgfqpoint{1.030863in}{0.683311in}}{\pgfqpoint{1.019812in}{0.683311in}}%
\pgfpathcurveto{\pgfqpoint{1.008762in}{0.683311in}}{\pgfqpoint{0.998163in}{0.678921in}}{\pgfqpoint{0.990350in}{0.671107in}}%
\pgfpathcurveto{\pgfqpoint{0.982536in}{0.663294in}}{\pgfqpoint{0.978146in}{0.652695in}}{\pgfqpoint{0.978146in}{0.641645in}}%
\pgfpathcurveto{\pgfqpoint{0.978146in}{0.630594in}}{\pgfqpoint{0.982536in}{0.619995in}}{\pgfqpoint{0.990350in}{0.612182in}}%
\pgfpathcurveto{\pgfqpoint{0.998163in}{0.604368in}}{\pgfqpoint{1.008762in}{0.599978in}}{\pgfqpoint{1.019812in}{0.599978in}}%
\pgfpathclose%
\pgfusepath{stroke,fill}%
\end{pgfscope}%
\begin{pgfscope}%
\pgfpathrectangle{\pgfqpoint{0.375000in}{0.330000in}}{\pgfqpoint{2.325000in}{2.310000in}}%
\pgfusepath{clip}%
\pgfsetbuttcap%
\pgfsetroundjoin%
\definecolor{currentfill}{rgb}{0.000000,0.000000,0.000000}%
\pgfsetfillcolor{currentfill}%
\pgfsetlinewidth{1.003750pt}%
\definecolor{currentstroke}{rgb}{0.000000,0.000000,0.000000}%
\pgfsetstrokecolor{currentstroke}%
\pgfsetdash{}{0pt}%
\pgfpathmoveto{\pgfqpoint{1.019812in}{0.642455in}}%
\pgfpathcurveto{\pgfqpoint{1.030863in}{0.642455in}}{\pgfqpoint{1.041462in}{0.646845in}}{\pgfqpoint{1.049275in}{0.654659in}}%
\pgfpathcurveto{\pgfqpoint{1.057089in}{0.662472in}}{\pgfqpoint{1.061479in}{0.673071in}}{\pgfqpoint{1.061479in}{0.684121in}}%
\pgfpathcurveto{\pgfqpoint{1.061479in}{0.695171in}}{\pgfqpoint{1.057089in}{0.705770in}}{\pgfqpoint{1.049275in}{0.713584in}}%
\pgfpathcurveto{\pgfqpoint{1.041462in}{0.721398in}}{\pgfqpoint{1.030863in}{0.725788in}}{\pgfqpoint{1.019812in}{0.725788in}}%
\pgfpathcurveto{\pgfqpoint{1.008762in}{0.725788in}}{\pgfqpoint{0.998163in}{0.721398in}}{\pgfqpoint{0.990350in}{0.713584in}}%
\pgfpathcurveto{\pgfqpoint{0.982536in}{0.705770in}}{\pgfqpoint{0.978146in}{0.695171in}}{\pgfqpoint{0.978146in}{0.684121in}}%
\pgfpathcurveto{\pgfqpoint{0.978146in}{0.673071in}}{\pgfqpoint{0.982536in}{0.662472in}}{\pgfqpoint{0.990350in}{0.654659in}}%
\pgfpathcurveto{\pgfqpoint{0.998163in}{0.646845in}}{\pgfqpoint{1.008762in}{0.642455in}}{\pgfqpoint{1.019812in}{0.642455in}}%
\pgfpathclose%
\pgfusepath{stroke,fill}%
\end{pgfscope}%
\begin{pgfscope}%
\pgfpathrectangle{\pgfqpoint{0.375000in}{0.330000in}}{\pgfqpoint{2.325000in}{2.310000in}}%
\pgfusepath{clip}%
\pgfsetbuttcap%
\pgfsetroundjoin%
\definecolor{currentfill}{rgb}{0.000000,0.000000,0.000000}%
\pgfsetfillcolor{currentfill}%
\pgfsetlinewidth{1.003750pt}%
\definecolor{currentstroke}{rgb}{0.000000,0.000000,0.000000}%
\pgfsetstrokecolor{currentstroke}%
\pgfsetdash{}{0pt}%
\pgfpathmoveto{\pgfqpoint{1.019812in}{0.660659in}}%
\pgfpathcurveto{\pgfqpoint{1.030863in}{0.660659in}}{\pgfqpoint{1.041462in}{0.665049in}}{\pgfqpoint{1.049275in}{0.672863in}}%
\pgfpathcurveto{\pgfqpoint{1.057089in}{0.680676in}}{\pgfqpoint{1.061479in}{0.691276in}}{\pgfqpoint{1.061479in}{0.702326in}}%
\pgfpathcurveto{\pgfqpoint{1.061479in}{0.713376in}}{\pgfqpoint{1.057089in}{0.723975in}}{\pgfqpoint{1.049275in}{0.731788in}}%
\pgfpathcurveto{\pgfqpoint{1.041462in}{0.739602in}}{\pgfqpoint{1.030863in}{0.743992in}}{\pgfqpoint{1.019812in}{0.743992in}}%
\pgfpathcurveto{\pgfqpoint{1.008762in}{0.743992in}}{\pgfqpoint{0.998163in}{0.739602in}}{\pgfqpoint{0.990350in}{0.731788in}}%
\pgfpathcurveto{\pgfqpoint{0.982536in}{0.723975in}}{\pgfqpoint{0.978146in}{0.713376in}}{\pgfqpoint{0.978146in}{0.702326in}}%
\pgfpathcurveto{\pgfqpoint{0.978146in}{0.691276in}}{\pgfqpoint{0.982536in}{0.680676in}}{\pgfqpoint{0.990350in}{0.672863in}}%
\pgfpathcurveto{\pgfqpoint{0.998163in}{0.665049in}}{\pgfqpoint{1.008762in}{0.660659in}}{\pgfqpoint{1.019812in}{0.660659in}}%
\pgfpathclose%
\pgfusepath{stroke,fill}%
\end{pgfscope}%
\begin{pgfscope}%
\pgfpathrectangle{\pgfqpoint{0.375000in}{0.330000in}}{\pgfqpoint{2.325000in}{2.310000in}}%
\pgfusepath{clip}%
\pgfsetbuttcap%
\pgfsetroundjoin%
\definecolor{currentfill}{rgb}{0.000000,0.000000,0.000000}%
\pgfsetfillcolor{currentfill}%
\pgfsetlinewidth{1.003750pt}%
\definecolor{currentstroke}{rgb}{0.000000,0.000000,0.000000}%
\pgfsetstrokecolor{currentstroke}%
\pgfsetdash{}{0pt}%
\pgfpathmoveto{\pgfqpoint{1.019812in}{0.672795in}}%
\pgfpathcurveto{\pgfqpoint{1.030863in}{0.672795in}}{\pgfqpoint{1.041462in}{0.677185in}}{\pgfqpoint{1.049275in}{0.684999in}}%
\pgfpathcurveto{\pgfqpoint{1.057089in}{0.692813in}}{\pgfqpoint{1.061479in}{0.703412in}}{\pgfqpoint{1.061479in}{0.714462in}}%
\pgfpathcurveto{\pgfqpoint{1.061479in}{0.725512in}}{\pgfqpoint{1.057089in}{0.736111in}}{\pgfqpoint{1.049275in}{0.743925in}}%
\pgfpathcurveto{\pgfqpoint{1.041462in}{0.751738in}}{\pgfqpoint{1.030863in}{0.756129in}}{\pgfqpoint{1.019812in}{0.756129in}}%
\pgfpathcurveto{\pgfqpoint{1.008762in}{0.756129in}}{\pgfqpoint{0.998163in}{0.751738in}}{\pgfqpoint{0.990350in}{0.743925in}}%
\pgfpathcurveto{\pgfqpoint{0.982536in}{0.736111in}}{\pgfqpoint{0.978146in}{0.725512in}}{\pgfqpoint{0.978146in}{0.714462in}}%
\pgfpathcurveto{\pgfqpoint{0.978146in}{0.703412in}}{\pgfqpoint{0.982536in}{0.692813in}}{\pgfqpoint{0.990350in}{0.684999in}}%
\pgfpathcurveto{\pgfqpoint{0.998163in}{0.677185in}}{\pgfqpoint{1.008762in}{0.672795in}}{\pgfqpoint{1.019812in}{0.672795in}}%
\pgfpathclose%
\pgfusepath{stroke,fill}%
\end{pgfscope}%
\begin{pgfscope}%
\pgfpathrectangle{\pgfqpoint{0.375000in}{0.330000in}}{\pgfqpoint{2.325000in}{2.310000in}}%
\pgfusepath{clip}%
\pgfsetbuttcap%
\pgfsetroundjoin%
\definecolor{currentfill}{rgb}{0.000000,0.000000,0.000000}%
\pgfsetfillcolor{currentfill}%
\pgfsetlinewidth{1.003750pt}%
\definecolor{currentstroke}{rgb}{0.000000,0.000000,0.000000}%
\pgfsetstrokecolor{currentstroke}%
\pgfsetdash{}{0pt}%
\pgfpathmoveto{\pgfqpoint{1.019812in}{0.648523in}}%
\pgfpathcurveto{\pgfqpoint{1.030863in}{0.648523in}}{\pgfqpoint{1.041462in}{0.652913in}}{\pgfqpoint{1.049275in}{0.660727in}}%
\pgfpathcurveto{\pgfqpoint{1.057089in}{0.668540in}}{\pgfqpoint{1.061479in}{0.679139in}}{\pgfqpoint{1.061479in}{0.690189in}}%
\pgfpathcurveto{\pgfqpoint{1.061479in}{0.701240in}}{\pgfqpoint{1.057089in}{0.711839in}}{\pgfqpoint{1.049275in}{0.719652in}}%
\pgfpathcurveto{\pgfqpoint{1.041462in}{0.727466in}}{\pgfqpoint{1.030863in}{0.731856in}}{\pgfqpoint{1.019812in}{0.731856in}}%
\pgfpathcurveto{\pgfqpoint{1.008762in}{0.731856in}}{\pgfqpoint{0.998163in}{0.727466in}}{\pgfqpoint{0.990350in}{0.719652in}}%
\pgfpathcurveto{\pgfqpoint{0.982536in}{0.711839in}}{\pgfqpoint{0.978146in}{0.701240in}}{\pgfqpoint{0.978146in}{0.690189in}}%
\pgfpathcurveto{\pgfqpoint{0.978146in}{0.679139in}}{\pgfqpoint{0.982536in}{0.668540in}}{\pgfqpoint{0.990350in}{0.660727in}}%
\pgfpathcurveto{\pgfqpoint{0.998163in}{0.652913in}}{\pgfqpoint{1.008762in}{0.648523in}}{\pgfqpoint{1.019812in}{0.648523in}}%
\pgfpathclose%
\pgfusepath{stroke,fill}%
\end{pgfscope}%
\begin{pgfscope}%
\pgfpathrectangle{\pgfqpoint{0.375000in}{0.330000in}}{\pgfqpoint{2.325000in}{2.310000in}}%
\pgfusepath{clip}%
\pgfsetbuttcap%
\pgfsetroundjoin%
\definecolor{currentfill}{rgb}{0.000000,0.000000,0.000000}%
\pgfsetfillcolor{currentfill}%
\pgfsetlinewidth{1.003750pt}%
\definecolor{currentstroke}{rgb}{0.000000,0.000000,0.000000}%
\pgfsetstrokecolor{currentstroke}%
\pgfsetdash{}{0pt}%
\pgfpathmoveto{\pgfqpoint{1.019812in}{0.612114in}}%
\pgfpathcurveto{\pgfqpoint{1.030863in}{0.612114in}}{\pgfqpoint{1.041462in}{0.616504in}}{\pgfqpoint{1.049275in}{0.624318in}}%
\pgfpathcurveto{\pgfqpoint{1.057089in}{0.632132in}}{\pgfqpoint{1.061479in}{0.642731in}}{\pgfqpoint{1.061479in}{0.653781in}}%
\pgfpathcurveto{\pgfqpoint{1.061479in}{0.664831in}}{\pgfqpoint{1.057089in}{0.675430in}}{\pgfqpoint{1.049275in}{0.683244in}}%
\pgfpathcurveto{\pgfqpoint{1.041462in}{0.691057in}}{\pgfqpoint{1.030863in}{0.695447in}}{\pgfqpoint{1.019812in}{0.695447in}}%
\pgfpathcurveto{\pgfqpoint{1.008762in}{0.695447in}}{\pgfqpoint{0.998163in}{0.691057in}}{\pgfqpoint{0.990350in}{0.683244in}}%
\pgfpathcurveto{\pgfqpoint{0.982536in}{0.675430in}}{\pgfqpoint{0.978146in}{0.664831in}}{\pgfqpoint{0.978146in}{0.653781in}}%
\pgfpathcurveto{\pgfqpoint{0.978146in}{0.642731in}}{\pgfqpoint{0.982536in}{0.632132in}}{\pgfqpoint{0.990350in}{0.624318in}}%
\pgfpathcurveto{\pgfqpoint{0.998163in}{0.616504in}}{\pgfqpoint{1.008762in}{0.612114in}}{\pgfqpoint{1.019812in}{0.612114in}}%
\pgfpathclose%
\pgfusepath{stroke,fill}%
\end{pgfscope}%
\begin{pgfscope}%
\pgfpathrectangle{\pgfqpoint{0.375000in}{0.330000in}}{\pgfqpoint{2.325000in}{2.310000in}}%
\pgfusepath{clip}%
\pgfsetbuttcap%
\pgfsetroundjoin%
\definecolor{currentfill}{rgb}{0.000000,0.000000,0.000000}%
\pgfsetfillcolor{currentfill}%
\pgfsetlinewidth{1.003750pt}%
\definecolor{currentstroke}{rgb}{0.000000,0.000000,0.000000}%
\pgfsetstrokecolor{currentstroke}%
\pgfsetdash{}{0pt}%
\pgfpathmoveto{\pgfqpoint{1.019812in}{0.599978in}}%
\pgfpathcurveto{\pgfqpoint{1.030863in}{0.599978in}}{\pgfqpoint{1.041462in}{0.604368in}}{\pgfqpoint{1.049275in}{0.612182in}}%
\pgfpathcurveto{\pgfqpoint{1.057089in}{0.619995in}}{\pgfqpoint{1.061479in}{0.630594in}}{\pgfqpoint{1.061479in}{0.641645in}}%
\pgfpathcurveto{\pgfqpoint{1.061479in}{0.652695in}}{\pgfqpoint{1.057089in}{0.663294in}}{\pgfqpoint{1.049275in}{0.671107in}}%
\pgfpathcurveto{\pgfqpoint{1.041462in}{0.678921in}}{\pgfqpoint{1.030863in}{0.683311in}}{\pgfqpoint{1.019812in}{0.683311in}}%
\pgfpathcurveto{\pgfqpoint{1.008762in}{0.683311in}}{\pgfqpoint{0.998163in}{0.678921in}}{\pgfqpoint{0.990350in}{0.671107in}}%
\pgfpathcurveto{\pgfqpoint{0.982536in}{0.663294in}}{\pgfqpoint{0.978146in}{0.652695in}}{\pgfqpoint{0.978146in}{0.641645in}}%
\pgfpathcurveto{\pgfqpoint{0.978146in}{0.630594in}}{\pgfqpoint{0.982536in}{0.619995in}}{\pgfqpoint{0.990350in}{0.612182in}}%
\pgfpathcurveto{\pgfqpoint{0.998163in}{0.604368in}}{\pgfqpoint{1.008762in}{0.599978in}}{\pgfqpoint{1.019812in}{0.599978in}}%
\pgfpathclose%
\pgfusepath{stroke,fill}%
\end{pgfscope}%
\begin{pgfscope}%
\pgfpathrectangle{\pgfqpoint{0.375000in}{0.330000in}}{\pgfqpoint{2.325000in}{2.310000in}}%
\pgfusepath{clip}%
\pgfsetbuttcap%
\pgfsetroundjoin%
\definecolor{currentfill}{rgb}{0.000000,0.000000,0.000000}%
\pgfsetfillcolor{currentfill}%
\pgfsetlinewidth{1.003750pt}%
\definecolor{currentstroke}{rgb}{0.000000,0.000000,0.000000}%
\pgfsetstrokecolor{currentstroke}%
\pgfsetdash{}{0pt}%
\pgfpathmoveto{\pgfqpoint{1.019812in}{0.630318in}}%
\pgfpathcurveto{\pgfqpoint{1.030863in}{0.630318in}}{\pgfqpoint{1.041462in}{0.634709in}}{\pgfqpoint{1.049275in}{0.642522in}}%
\pgfpathcurveto{\pgfqpoint{1.057089in}{0.650336in}}{\pgfqpoint{1.061479in}{0.660935in}}{\pgfqpoint{1.061479in}{0.671985in}}%
\pgfpathcurveto{\pgfqpoint{1.061479in}{0.683035in}}{\pgfqpoint{1.057089in}{0.693634in}}{\pgfqpoint{1.049275in}{0.701448in}}%
\pgfpathcurveto{\pgfqpoint{1.041462in}{0.709262in}}{\pgfqpoint{1.030863in}{0.713652in}}{\pgfqpoint{1.019812in}{0.713652in}}%
\pgfpathcurveto{\pgfqpoint{1.008762in}{0.713652in}}{\pgfqpoint{0.998163in}{0.709262in}}{\pgfqpoint{0.990350in}{0.701448in}}%
\pgfpathcurveto{\pgfqpoint{0.982536in}{0.693634in}}{\pgfqpoint{0.978146in}{0.683035in}}{\pgfqpoint{0.978146in}{0.671985in}}%
\pgfpathcurveto{\pgfqpoint{0.978146in}{0.660935in}}{\pgfqpoint{0.982536in}{0.650336in}}{\pgfqpoint{0.990350in}{0.642522in}}%
\pgfpathcurveto{\pgfqpoint{0.998163in}{0.634709in}}{\pgfqpoint{1.008762in}{0.630318in}}{\pgfqpoint{1.019812in}{0.630318in}}%
\pgfpathclose%
\pgfusepath{stroke,fill}%
\end{pgfscope}%
\begin{pgfscope}%
\pgfpathrectangle{\pgfqpoint{0.375000in}{0.330000in}}{\pgfqpoint{2.325000in}{2.310000in}}%
\pgfusepath{clip}%
\pgfsetbuttcap%
\pgfsetroundjoin%
\definecolor{currentfill}{rgb}{0.000000,0.000000,0.000000}%
\pgfsetfillcolor{currentfill}%
\pgfsetlinewidth{1.003750pt}%
\definecolor{currentstroke}{rgb}{0.000000,0.000000,0.000000}%
\pgfsetstrokecolor{currentstroke}%
\pgfsetdash{}{0pt}%
\pgfpathmoveto{\pgfqpoint{1.019812in}{0.593910in}}%
\pgfpathcurveto{\pgfqpoint{1.030863in}{0.593910in}}{\pgfqpoint{1.041462in}{0.598300in}}{\pgfqpoint{1.049275in}{0.606114in}}%
\pgfpathcurveto{\pgfqpoint{1.057089in}{0.613927in}}{\pgfqpoint{1.061479in}{0.624526in}}{\pgfqpoint{1.061479in}{0.635576in}}%
\pgfpathcurveto{\pgfqpoint{1.061479in}{0.646627in}}{\pgfqpoint{1.057089in}{0.657226in}}{\pgfqpoint{1.049275in}{0.665039in}}%
\pgfpathcurveto{\pgfqpoint{1.041462in}{0.672853in}}{\pgfqpoint{1.030863in}{0.677243in}}{\pgfqpoint{1.019812in}{0.677243in}}%
\pgfpathcurveto{\pgfqpoint{1.008762in}{0.677243in}}{\pgfqpoint{0.998163in}{0.672853in}}{\pgfqpoint{0.990350in}{0.665039in}}%
\pgfpathcurveto{\pgfqpoint{0.982536in}{0.657226in}}{\pgfqpoint{0.978146in}{0.646627in}}{\pgfqpoint{0.978146in}{0.635576in}}%
\pgfpathcurveto{\pgfqpoint{0.978146in}{0.624526in}}{\pgfqpoint{0.982536in}{0.613927in}}{\pgfqpoint{0.990350in}{0.606114in}}%
\pgfpathcurveto{\pgfqpoint{0.998163in}{0.598300in}}{\pgfqpoint{1.008762in}{0.593910in}}{\pgfqpoint{1.019812in}{0.593910in}}%
\pgfpathclose%
\pgfusepath{stroke,fill}%
\end{pgfscope}%
\begin{pgfscope}%
\pgfpathrectangle{\pgfqpoint{0.375000in}{0.330000in}}{\pgfqpoint{2.325000in}{2.310000in}}%
\pgfusepath{clip}%
\pgfsetbuttcap%
\pgfsetroundjoin%
\definecolor{currentfill}{rgb}{0.000000,0.000000,0.000000}%
\pgfsetfillcolor{currentfill}%
\pgfsetlinewidth{1.003750pt}%
\definecolor{currentstroke}{rgb}{0.000000,0.000000,0.000000}%
\pgfsetstrokecolor{currentstroke}%
\pgfsetdash{}{0pt}%
\pgfpathmoveto{\pgfqpoint{1.019812in}{0.630318in}}%
\pgfpathcurveto{\pgfqpoint{1.030863in}{0.630318in}}{\pgfqpoint{1.041462in}{0.634709in}}{\pgfqpoint{1.049275in}{0.642522in}}%
\pgfpathcurveto{\pgfqpoint{1.057089in}{0.650336in}}{\pgfqpoint{1.061479in}{0.660935in}}{\pgfqpoint{1.061479in}{0.671985in}}%
\pgfpathcurveto{\pgfqpoint{1.061479in}{0.683035in}}{\pgfqpoint{1.057089in}{0.693634in}}{\pgfqpoint{1.049275in}{0.701448in}}%
\pgfpathcurveto{\pgfqpoint{1.041462in}{0.709262in}}{\pgfqpoint{1.030863in}{0.713652in}}{\pgfqpoint{1.019812in}{0.713652in}}%
\pgfpathcurveto{\pgfqpoint{1.008762in}{0.713652in}}{\pgfqpoint{0.998163in}{0.709262in}}{\pgfqpoint{0.990350in}{0.701448in}}%
\pgfpathcurveto{\pgfqpoint{0.982536in}{0.693634in}}{\pgfqpoint{0.978146in}{0.683035in}}{\pgfqpoint{0.978146in}{0.671985in}}%
\pgfpathcurveto{\pgfqpoint{0.978146in}{0.660935in}}{\pgfqpoint{0.982536in}{0.650336in}}{\pgfqpoint{0.990350in}{0.642522in}}%
\pgfpathcurveto{\pgfqpoint{0.998163in}{0.634709in}}{\pgfqpoint{1.008762in}{0.630318in}}{\pgfqpoint{1.019812in}{0.630318in}}%
\pgfpathclose%
\pgfusepath{stroke,fill}%
\end{pgfscope}%
\begin{pgfscope}%
\pgfpathrectangle{\pgfqpoint{0.375000in}{0.330000in}}{\pgfqpoint{2.325000in}{2.310000in}}%
\pgfusepath{clip}%
\pgfsetbuttcap%
\pgfsetroundjoin%
\definecolor{currentfill}{rgb}{0.000000,0.000000,0.000000}%
\pgfsetfillcolor{currentfill}%
\pgfsetlinewidth{1.003750pt}%
\definecolor{currentstroke}{rgb}{0.000000,0.000000,0.000000}%
\pgfsetstrokecolor{currentstroke}%
\pgfsetdash{}{0pt}%
\pgfpathmoveto{\pgfqpoint{1.019812in}{0.612114in}}%
\pgfpathcurveto{\pgfqpoint{1.030863in}{0.612114in}}{\pgfqpoint{1.041462in}{0.616504in}}{\pgfqpoint{1.049275in}{0.624318in}}%
\pgfpathcurveto{\pgfqpoint{1.057089in}{0.632132in}}{\pgfqpoint{1.061479in}{0.642731in}}{\pgfqpoint{1.061479in}{0.653781in}}%
\pgfpathcurveto{\pgfqpoint{1.061479in}{0.664831in}}{\pgfqpoint{1.057089in}{0.675430in}}{\pgfqpoint{1.049275in}{0.683244in}}%
\pgfpathcurveto{\pgfqpoint{1.041462in}{0.691057in}}{\pgfqpoint{1.030863in}{0.695447in}}{\pgfqpoint{1.019812in}{0.695447in}}%
\pgfpathcurveto{\pgfqpoint{1.008762in}{0.695447in}}{\pgfqpoint{0.998163in}{0.691057in}}{\pgfqpoint{0.990350in}{0.683244in}}%
\pgfpathcurveto{\pgfqpoint{0.982536in}{0.675430in}}{\pgfqpoint{0.978146in}{0.664831in}}{\pgfqpoint{0.978146in}{0.653781in}}%
\pgfpathcurveto{\pgfqpoint{0.978146in}{0.642731in}}{\pgfqpoint{0.982536in}{0.632132in}}{\pgfqpoint{0.990350in}{0.624318in}}%
\pgfpathcurveto{\pgfqpoint{0.998163in}{0.616504in}}{\pgfqpoint{1.008762in}{0.612114in}}{\pgfqpoint{1.019812in}{0.612114in}}%
\pgfpathclose%
\pgfusepath{stroke,fill}%
\end{pgfscope}%
\begin{pgfscope}%
\pgfpathrectangle{\pgfqpoint{0.375000in}{0.330000in}}{\pgfqpoint{2.325000in}{2.310000in}}%
\pgfusepath{clip}%
\pgfsetbuttcap%
\pgfsetroundjoin%
\definecolor{currentfill}{rgb}{0.000000,0.000000,0.000000}%
\pgfsetfillcolor{currentfill}%
\pgfsetlinewidth{1.003750pt}%
\definecolor{currentstroke}{rgb}{0.000000,0.000000,0.000000}%
\pgfsetstrokecolor{currentstroke}%
\pgfsetdash{}{0pt}%
\pgfpathmoveto{\pgfqpoint{1.019812in}{0.672795in}}%
\pgfpathcurveto{\pgfqpoint{1.030863in}{0.672795in}}{\pgfqpoint{1.041462in}{0.677185in}}{\pgfqpoint{1.049275in}{0.684999in}}%
\pgfpathcurveto{\pgfqpoint{1.057089in}{0.692813in}}{\pgfqpoint{1.061479in}{0.703412in}}{\pgfqpoint{1.061479in}{0.714462in}}%
\pgfpathcurveto{\pgfqpoint{1.061479in}{0.725512in}}{\pgfqpoint{1.057089in}{0.736111in}}{\pgfqpoint{1.049275in}{0.743925in}}%
\pgfpathcurveto{\pgfqpoint{1.041462in}{0.751738in}}{\pgfqpoint{1.030863in}{0.756129in}}{\pgfqpoint{1.019812in}{0.756129in}}%
\pgfpathcurveto{\pgfqpoint{1.008762in}{0.756129in}}{\pgfqpoint{0.998163in}{0.751738in}}{\pgfqpoint{0.990350in}{0.743925in}}%
\pgfpathcurveto{\pgfqpoint{0.982536in}{0.736111in}}{\pgfqpoint{0.978146in}{0.725512in}}{\pgfqpoint{0.978146in}{0.714462in}}%
\pgfpathcurveto{\pgfqpoint{0.978146in}{0.703412in}}{\pgfqpoint{0.982536in}{0.692813in}}{\pgfqpoint{0.990350in}{0.684999in}}%
\pgfpathcurveto{\pgfqpoint{0.998163in}{0.677185in}}{\pgfqpoint{1.008762in}{0.672795in}}{\pgfqpoint{1.019812in}{0.672795in}}%
\pgfpathclose%
\pgfusepath{stroke,fill}%
\end{pgfscope}%
\begin{pgfscope}%
\pgfpathrectangle{\pgfqpoint{0.375000in}{0.330000in}}{\pgfqpoint{2.325000in}{2.310000in}}%
\pgfusepath{clip}%
\pgfsetbuttcap%
\pgfsetroundjoin%
\definecolor{currentfill}{rgb}{0.000000,0.000000,0.000000}%
\pgfsetfillcolor{currentfill}%
\pgfsetlinewidth{1.003750pt}%
\definecolor{currentstroke}{rgb}{0.000000,0.000000,0.000000}%
\pgfsetstrokecolor{currentstroke}%
\pgfsetdash{}{0pt}%
\pgfpathmoveto{\pgfqpoint{1.019812in}{0.587842in}}%
\pgfpathcurveto{\pgfqpoint{1.030863in}{0.587842in}}{\pgfqpoint{1.041462in}{0.592232in}}{\pgfqpoint{1.049275in}{0.600046in}}%
\pgfpathcurveto{\pgfqpoint{1.057089in}{0.607859in}}{\pgfqpoint{1.061479in}{0.618458in}}{\pgfqpoint{1.061479in}{0.629508in}}%
\pgfpathcurveto{\pgfqpoint{1.061479in}{0.640559in}}{\pgfqpoint{1.057089in}{0.651158in}}{\pgfqpoint{1.049275in}{0.658971in}}%
\pgfpathcurveto{\pgfqpoint{1.041462in}{0.666785in}}{\pgfqpoint{1.030863in}{0.671175in}}{\pgfqpoint{1.019812in}{0.671175in}}%
\pgfpathcurveto{\pgfqpoint{1.008762in}{0.671175in}}{\pgfqpoint{0.998163in}{0.666785in}}{\pgfqpoint{0.990350in}{0.658971in}}%
\pgfpathcurveto{\pgfqpoint{0.982536in}{0.651158in}}{\pgfqpoint{0.978146in}{0.640559in}}{\pgfqpoint{0.978146in}{0.629508in}}%
\pgfpathcurveto{\pgfqpoint{0.978146in}{0.618458in}}{\pgfqpoint{0.982536in}{0.607859in}}{\pgfqpoint{0.990350in}{0.600046in}}%
\pgfpathcurveto{\pgfqpoint{0.998163in}{0.592232in}}{\pgfqpoint{1.008762in}{0.587842in}}{\pgfqpoint{1.019812in}{0.587842in}}%
\pgfpathclose%
\pgfusepath{stroke,fill}%
\end{pgfscope}%
\begin{pgfscope}%
\pgfpathrectangle{\pgfqpoint{0.375000in}{0.330000in}}{\pgfqpoint{2.325000in}{2.310000in}}%
\pgfusepath{clip}%
\pgfsetbuttcap%
\pgfsetroundjoin%
\definecolor{currentfill}{rgb}{0.000000,0.000000,0.000000}%
\pgfsetfillcolor{currentfill}%
\pgfsetlinewidth{1.003750pt}%
\definecolor{currentstroke}{rgb}{0.000000,0.000000,0.000000}%
\pgfsetstrokecolor{currentstroke}%
\pgfsetdash{}{0pt}%
\pgfpathmoveto{\pgfqpoint{1.019812in}{0.636387in}}%
\pgfpathcurveto{\pgfqpoint{1.030863in}{0.636387in}}{\pgfqpoint{1.041462in}{0.640777in}}{\pgfqpoint{1.049275in}{0.648590in}}%
\pgfpathcurveto{\pgfqpoint{1.057089in}{0.656404in}}{\pgfqpoint{1.061479in}{0.667003in}}{\pgfqpoint{1.061479in}{0.678053in}}%
\pgfpathcurveto{\pgfqpoint{1.061479in}{0.689103in}}{\pgfqpoint{1.057089in}{0.699702in}}{\pgfqpoint{1.049275in}{0.707516in}}%
\pgfpathcurveto{\pgfqpoint{1.041462in}{0.715330in}}{\pgfqpoint{1.030863in}{0.719720in}}{\pgfqpoint{1.019812in}{0.719720in}}%
\pgfpathcurveto{\pgfqpoint{1.008762in}{0.719720in}}{\pgfqpoint{0.998163in}{0.715330in}}{\pgfqpoint{0.990350in}{0.707516in}}%
\pgfpathcurveto{\pgfqpoint{0.982536in}{0.699702in}}{\pgfqpoint{0.978146in}{0.689103in}}{\pgfqpoint{0.978146in}{0.678053in}}%
\pgfpathcurveto{\pgfqpoint{0.978146in}{0.667003in}}{\pgfqpoint{0.982536in}{0.656404in}}{\pgfqpoint{0.990350in}{0.648590in}}%
\pgfpathcurveto{\pgfqpoint{0.998163in}{0.640777in}}{\pgfqpoint{1.008762in}{0.636387in}}{\pgfqpoint{1.019812in}{0.636387in}}%
\pgfpathclose%
\pgfusepath{stroke,fill}%
\end{pgfscope}%
\begin{pgfscope}%
\pgfpathrectangle{\pgfqpoint{0.375000in}{0.330000in}}{\pgfqpoint{2.325000in}{2.310000in}}%
\pgfusepath{clip}%
\pgfsetbuttcap%
\pgfsetroundjoin%
\definecolor{currentfill}{rgb}{0.000000,0.000000,0.000000}%
\pgfsetfillcolor{currentfill}%
\pgfsetlinewidth{1.003750pt}%
\definecolor{currentstroke}{rgb}{0.000000,0.000000,0.000000}%
\pgfsetstrokecolor{currentstroke}%
\pgfsetdash{}{0pt}%
\pgfpathmoveto{\pgfqpoint{1.019812in}{0.612114in}}%
\pgfpathcurveto{\pgfqpoint{1.030863in}{0.612114in}}{\pgfqpoint{1.041462in}{0.616504in}}{\pgfqpoint{1.049275in}{0.624318in}}%
\pgfpathcurveto{\pgfqpoint{1.057089in}{0.632132in}}{\pgfqpoint{1.061479in}{0.642731in}}{\pgfqpoint{1.061479in}{0.653781in}}%
\pgfpathcurveto{\pgfqpoint{1.061479in}{0.664831in}}{\pgfqpoint{1.057089in}{0.675430in}}{\pgfqpoint{1.049275in}{0.683244in}}%
\pgfpathcurveto{\pgfqpoint{1.041462in}{0.691057in}}{\pgfqpoint{1.030863in}{0.695447in}}{\pgfqpoint{1.019812in}{0.695447in}}%
\pgfpathcurveto{\pgfqpoint{1.008762in}{0.695447in}}{\pgfqpoint{0.998163in}{0.691057in}}{\pgfqpoint{0.990350in}{0.683244in}}%
\pgfpathcurveto{\pgfqpoint{0.982536in}{0.675430in}}{\pgfqpoint{0.978146in}{0.664831in}}{\pgfqpoint{0.978146in}{0.653781in}}%
\pgfpathcurveto{\pgfqpoint{0.978146in}{0.642731in}}{\pgfqpoint{0.982536in}{0.632132in}}{\pgfqpoint{0.990350in}{0.624318in}}%
\pgfpathcurveto{\pgfqpoint{0.998163in}{0.616504in}}{\pgfqpoint{1.008762in}{0.612114in}}{\pgfqpoint{1.019812in}{0.612114in}}%
\pgfpathclose%
\pgfusepath{stroke,fill}%
\end{pgfscope}%
\begin{pgfscope}%
\pgfpathrectangle{\pgfqpoint{0.375000in}{0.330000in}}{\pgfqpoint{2.325000in}{2.310000in}}%
\pgfusepath{clip}%
\pgfsetbuttcap%
\pgfsetroundjoin%
\definecolor{currentfill}{rgb}{0.000000,0.000000,0.000000}%
\pgfsetfillcolor{currentfill}%
\pgfsetlinewidth{1.003750pt}%
\definecolor{currentstroke}{rgb}{0.000000,0.000000,0.000000}%
\pgfsetstrokecolor{currentstroke}%
\pgfsetdash{}{0pt}%
\pgfpathmoveto{\pgfqpoint{1.019812in}{0.557501in}}%
\pgfpathcurveto{\pgfqpoint{1.030863in}{0.557501in}}{\pgfqpoint{1.041462in}{0.561891in}}{\pgfqpoint{1.049275in}{0.569705in}}%
\pgfpathcurveto{\pgfqpoint{1.057089in}{0.577519in}}{\pgfqpoint{1.061479in}{0.588118in}}{\pgfqpoint{1.061479in}{0.599168in}}%
\pgfpathcurveto{\pgfqpoint{1.061479in}{0.610218in}}{\pgfqpoint{1.057089in}{0.620817in}}{\pgfqpoint{1.049275in}{0.628631in}}%
\pgfpathcurveto{\pgfqpoint{1.041462in}{0.636444in}}{\pgfqpoint{1.030863in}{0.640835in}}{\pgfqpoint{1.019812in}{0.640835in}}%
\pgfpathcurveto{\pgfqpoint{1.008762in}{0.640835in}}{\pgfqpoint{0.998163in}{0.636444in}}{\pgfqpoint{0.990350in}{0.628631in}}%
\pgfpathcurveto{\pgfqpoint{0.982536in}{0.620817in}}{\pgfqpoint{0.978146in}{0.610218in}}{\pgfqpoint{0.978146in}{0.599168in}}%
\pgfpathcurveto{\pgfqpoint{0.978146in}{0.588118in}}{\pgfqpoint{0.982536in}{0.577519in}}{\pgfqpoint{0.990350in}{0.569705in}}%
\pgfpathcurveto{\pgfqpoint{0.998163in}{0.561891in}}{\pgfqpoint{1.008762in}{0.557501in}}{\pgfqpoint{1.019812in}{0.557501in}}%
\pgfpathclose%
\pgfusepath{stroke,fill}%
\end{pgfscope}%
\begin{pgfscope}%
\pgfpathrectangle{\pgfqpoint{0.375000in}{0.330000in}}{\pgfqpoint{2.325000in}{2.310000in}}%
\pgfusepath{clip}%
\pgfsetbuttcap%
\pgfsetroundjoin%
\definecolor{currentfill}{rgb}{0.000000,0.000000,0.000000}%
\pgfsetfillcolor{currentfill}%
\pgfsetlinewidth{1.003750pt}%
\definecolor{currentstroke}{rgb}{0.000000,0.000000,0.000000}%
\pgfsetstrokecolor{currentstroke}%
\pgfsetdash{}{0pt}%
\pgfpathmoveto{\pgfqpoint{1.019812in}{0.739544in}}%
\pgfpathcurveto{\pgfqpoint{1.030863in}{0.739544in}}{\pgfqpoint{1.041462in}{0.743935in}}{\pgfqpoint{1.049275in}{0.751748in}}%
\pgfpathcurveto{\pgfqpoint{1.057089in}{0.759562in}}{\pgfqpoint{1.061479in}{0.770161in}}{\pgfqpoint{1.061479in}{0.781211in}}%
\pgfpathcurveto{\pgfqpoint{1.061479in}{0.792261in}}{\pgfqpoint{1.057089in}{0.802860in}}{\pgfqpoint{1.049275in}{0.810674in}}%
\pgfpathcurveto{\pgfqpoint{1.041462in}{0.818487in}}{\pgfqpoint{1.030863in}{0.822878in}}{\pgfqpoint{1.019812in}{0.822878in}}%
\pgfpathcurveto{\pgfqpoint{1.008762in}{0.822878in}}{\pgfqpoint{0.998163in}{0.818487in}}{\pgfqpoint{0.990350in}{0.810674in}}%
\pgfpathcurveto{\pgfqpoint{0.982536in}{0.802860in}}{\pgfqpoint{0.978146in}{0.792261in}}{\pgfqpoint{0.978146in}{0.781211in}}%
\pgfpathcurveto{\pgfqpoint{0.978146in}{0.770161in}}{\pgfqpoint{0.982536in}{0.759562in}}{\pgfqpoint{0.990350in}{0.751748in}}%
\pgfpathcurveto{\pgfqpoint{0.998163in}{0.743935in}}{\pgfqpoint{1.008762in}{0.739544in}}{\pgfqpoint{1.019812in}{0.739544in}}%
\pgfpathclose%
\pgfusepath{stroke,fill}%
\end{pgfscope}%
\begin{pgfscope}%
\pgfpathrectangle{\pgfqpoint{0.375000in}{0.330000in}}{\pgfqpoint{2.325000in}{2.310000in}}%
\pgfusepath{clip}%
\pgfsetbuttcap%
\pgfsetroundjoin%
\definecolor{currentfill}{rgb}{0.000000,0.000000,0.000000}%
\pgfsetfillcolor{currentfill}%
\pgfsetlinewidth{1.003750pt}%
\definecolor{currentstroke}{rgb}{0.000000,0.000000,0.000000}%
\pgfsetstrokecolor{currentstroke}%
\pgfsetdash{}{0pt}%
\pgfpathmoveto{\pgfqpoint{1.019812in}{0.684931in}}%
\pgfpathcurveto{\pgfqpoint{1.030863in}{0.684931in}}{\pgfqpoint{1.041462in}{0.689322in}}{\pgfqpoint{1.049275in}{0.697135in}}%
\pgfpathcurveto{\pgfqpoint{1.057089in}{0.704949in}}{\pgfqpoint{1.061479in}{0.715548in}}{\pgfqpoint{1.061479in}{0.726598in}}%
\pgfpathcurveto{\pgfqpoint{1.061479in}{0.737648in}}{\pgfqpoint{1.057089in}{0.748247in}}{\pgfqpoint{1.049275in}{0.756061in}}%
\pgfpathcurveto{\pgfqpoint{1.041462in}{0.763874in}}{\pgfqpoint{1.030863in}{0.768265in}}{\pgfqpoint{1.019812in}{0.768265in}}%
\pgfpathcurveto{\pgfqpoint{1.008762in}{0.768265in}}{\pgfqpoint{0.998163in}{0.763874in}}{\pgfqpoint{0.990350in}{0.756061in}}%
\pgfpathcurveto{\pgfqpoint{0.982536in}{0.748247in}}{\pgfqpoint{0.978146in}{0.737648in}}{\pgfqpoint{0.978146in}{0.726598in}}%
\pgfpathcurveto{\pgfqpoint{0.978146in}{0.715548in}}{\pgfqpoint{0.982536in}{0.704949in}}{\pgfqpoint{0.990350in}{0.697135in}}%
\pgfpathcurveto{\pgfqpoint{0.998163in}{0.689322in}}{\pgfqpoint{1.008762in}{0.684931in}}{\pgfqpoint{1.019812in}{0.684931in}}%
\pgfpathclose%
\pgfusepath{stroke,fill}%
\end{pgfscope}%
\begin{pgfscope}%
\pgfpathrectangle{\pgfqpoint{0.375000in}{0.330000in}}{\pgfqpoint{2.325000in}{2.310000in}}%
\pgfusepath{clip}%
\pgfsetbuttcap%
\pgfsetroundjoin%
\definecolor{currentfill}{rgb}{0.000000,0.000000,0.000000}%
\pgfsetfillcolor{currentfill}%
\pgfsetlinewidth{1.003750pt}%
\definecolor{currentstroke}{rgb}{0.000000,0.000000,0.000000}%
\pgfsetstrokecolor{currentstroke}%
\pgfsetdash{}{0pt}%
\pgfpathmoveto{\pgfqpoint{1.019812in}{0.606046in}}%
\pgfpathcurveto{\pgfqpoint{1.030863in}{0.606046in}}{\pgfqpoint{1.041462in}{0.610436in}}{\pgfqpoint{1.049275in}{0.618250in}}%
\pgfpathcurveto{\pgfqpoint{1.057089in}{0.626064in}}{\pgfqpoint{1.061479in}{0.636663in}}{\pgfqpoint{1.061479in}{0.647713in}}%
\pgfpathcurveto{\pgfqpoint{1.061479in}{0.658763in}}{\pgfqpoint{1.057089in}{0.669362in}}{\pgfqpoint{1.049275in}{0.677175in}}%
\pgfpathcurveto{\pgfqpoint{1.041462in}{0.684989in}}{\pgfqpoint{1.030863in}{0.689379in}}{\pgfqpoint{1.019812in}{0.689379in}}%
\pgfpathcurveto{\pgfqpoint{1.008762in}{0.689379in}}{\pgfqpoint{0.998163in}{0.684989in}}{\pgfqpoint{0.990350in}{0.677175in}}%
\pgfpathcurveto{\pgfqpoint{0.982536in}{0.669362in}}{\pgfqpoint{0.978146in}{0.658763in}}{\pgfqpoint{0.978146in}{0.647713in}}%
\pgfpathcurveto{\pgfqpoint{0.978146in}{0.636663in}}{\pgfqpoint{0.982536in}{0.626064in}}{\pgfqpoint{0.990350in}{0.618250in}}%
\pgfpathcurveto{\pgfqpoint{0.998163in}{0.610436in}}{\pgfqpoint{1.008762in}{0.606046in}}{\pgfqpoint{1.019812in}{0.606046in}}%
\pgfpathclose%
\pgfusepath{stroke,fill}%
\end{pgfscope}%
\begin{pgfscope}%
\pgfpathrectangle{\pgfqpoint{0.375000in}{0.330000in}}{\pgfqpoint{2.325000in}{2.310000in}}%
\pgfusepath{clip}%
\pgfsetbuttcap%
\pgfsetroundjoin%
\definecolor{currentfill}{rgb}{0.000000,0.000000,0.000000}%
\pgfsetfillcolor{currentfill}%
\pgfsetlinewidth{1.003750pt}%
\definecolor{currentstroke}{rgb}{0.000000,0.000000,0.000000}%
\pgfsetstrokecolor{currentstroke}%
\pgfsetdash{}{0pt}%
\pgfpathmoveto{\pgfqpoint{1.019812in}{0.630318in}}%
\pgfpathcurveto{\pgfqpoint{1.030863in}{0.630318in}}{\pgfqpoint{1.041462in}{0.634709in}}{\pgfqpoint{1.049275in}{0.642522in}}%
\pgfpathcurveto{\pgfqpoint{1.057089in}{0.650336in}}{\pgfqpoint{1.061479in}{0.660935in}}{\pgfqpoint{1.061479in}{0.671985in}}%
\pgfpathcurveto{\pgfqpoint{1.061479in}{0.683035in}}{\pgfqpoint{1.057089in}{0.693634in}}{\pgfqpoint{1.049275in}{0.701448in}}%
\pgfpathcurveto{\pgfqpoint{1.041462in}{0.709262in}}{\pgfqpoint{1.030863in}{0.713652in}}{\pgfqpoint{1.019812in}{0.713652in}}%
\pgfpathcurveto{\pgfqpoint{1.008762in}{0.713652in}}{\pgfqpoint{0.998163in}{0.709262in}}{\pgfqpoint{0.990350in}{0.701448in}}%
\pgfpathcurveto{\pgfqpoint{0.982536in}{0.693634in}}{\pgfqpoint{0.978146in}{0.683035in}}{\pgfqpoint{0.978146in}{0.671985in}}%
\pgfpathcurveto{\pgfqpoint{0.978146in}{0.660935in}}{\pgfqpoint{0.982536in}{0.650336in}}{\pgfqpoint{0.990350in}{0.642522in}}%
\pgfpathcurveto{\pgfqpoint{0.998163in}{0.634709in}}{\pgfqpoint{1.008762in}{0.630318in}}{\pgfqpoint{1.019812in}{0.630318in}}%
\pgfpathclose%
\pgfusepath{stroke,fill}%
\end{pgfscope}%
\begin{pgfscope}%
\pgfpathrectangle{\pgfqpoint{0.375000in}{0.330000in}}{\pgfqpoint{2.325000in}{2.310000in}}%
\pgfusepath{clip}%
\pgfsetbuttcap%
\pgfsetroundjoin%
\definecolor{currentfill}{rgb}{0.000000,0.000000,0.000000}%
\pgfsetfillcolor{currentfill}%
\pgfsetlinewidth{1.003750pt}%
\definecolor{currentstroke}{rgb}{0.000000,0.000000,0.000000}%
\pgfsetstrokecolor{currentstroke}%
\pgfsetdash{}{0pt}%
\pgfpathmoveto{\pgfqpoint{1.019812in}{0.599978in}}%
\pgfpathcurveto{\pgfqpoint{1.030863in}{0.599978in}}{\pgfqpoint{1.041462in}{0.604368in}}{\pgfqpoint{1.049275in}{0.612182in}}%
\pgfpathcurveto{\pgfqpoint{1.057089in}{0.619995in}}{\pgfqpoint{1.061479in}{0.630594in}}{\pgfqpoint{1.061479in}{0.641645in}}%
\pgfpathcurveto{\pgfqpoint{1.061479in}{0.652695in}}{\pgfqpoint{1.057089in}{0.663294in}}{\pgfqpoint{1.049275in}{0.671107in}}%
\pgfpathcurveto{\pgfqpoint{1.041462in}{0.678921in}}{\pgfqpoint{1.030863in}{0.683311in}}{\pgfqpoint{1.019812in}{0.683311in}}%
\pgfpathcurveto{\pgfqpoint{1.008762in}{0.683311in}}{\pgfqpoint{0.998163in}{0.678921in}}{\pgfqpoint{0.990350in}{0.671107in}}%
\pgfpathcurveto{\pgfqpoint{0.982536in}{0.663294in}}{\pgfqpoint{0.978146in}{0.652695in}}{\pgfqpoint{0.978146in}{0.641645in}}%
\pgfpathcurveto{\pgfqpoint{0.978146in}{0.630594in}}{\pgfqpoint{0.982536in}{0.619995in}}{\pgfqpoint{0.990350in}{0.612182in}}%
\pgfpathcurveto{\pgfqpoint{0.998163in}{0.604368in}}{\pgfqpoint{1.008762in}{0.599978in}}{\pgfqpoint{1.019812in}{0.599978in}}%
\pgfpathclose%
\pgfusepath{stroke,fill}%
\end{pgfscope}%
\begin{pgfscope}%
\pgfpathrectangle{\pgfqpoint{0.375000in}{0.330000in}}{\pgfqpoint{2.325000in}{2.310000in}}%
\pgfusepath{clip}%
\pgfsetbuttcap%
\pgfsetroundjoin%
\definecolor{currentfill}{rgb}{0.000000,0.000000,0.000000}%
\pgfsetfillcolor{currentfill}%
\pgfsetlinewidth{1.003750pt}%
\definecolor{currentstroke}{rgb}{0.000000,0.000000,0.000000}%
\pgfsetstrokecolor{currentstroke}%
\pgfsetdash{}{0pt}%
\pgfpathmoveto{\pgfqpoint{1.019812in}{0.533229in}}%
\pgfpathcurveto{\pgfqpoint{1.030863in}{0.533229in}}{\pgfqpoint{1.041462in}{0.537619in}}{\pgfqpoint{1.049275in}{0.545433in}}%
\pgfpathcurveto{\pgfqpoint{1.057089in}{0.553246in}}{\pgfqpoint{1.061479in}{0.563845in}}{\pgfqpoint{1.061479in}{0.574895in}}%
\pgfpathcurveto{\pgfqpoint{1.061479in}{0.585946in}}{\pgfqpoint{1.057089in}{0.596545in}}{\pgfqpoint{1.049275in}{0.604358in}}%
\pgfpathcurveto{\pgfqpoint{1.041462in}{0.612172in}}{\pgfqpoint{1.030863in}{0.616562in}}{\pgfqpoint{1.019812in}{0.616562in}}%
\pgfpathcurveto{\pgfqpoint{1.008762in}{0.616562in}}{\pgfqpoint{0.998163in}{0.612172in}}{\pgfqpoint{0.990350in}{0.604358in}}%
\pgfpathcurveto{\pgfqpoint{0.982536in}{0.596545in}}{\pgfqpoint{0.978146in}{0.585946in}}{\pgfqpoint{0.978146in}{0.574895in}}%
\pgfpathcurveto{\pgfqpoint{0.978146in}{0.563845in}}{\pgfqpoint{0.982536in}{0.553246in}}{\pgfqpoint{0.990350in}{0.545433in}}%
\pgfpathcurveto{\pgfqpoint{0.998163in}{0.537619in}}{\pgfqpoint{1.008762in}{0.533229in}}{\pgfqpoint{1.019812in}{0.533229in}}%
\pgfpathclose%
\pgfusepath{stroke,fill}%
\end{pgfscope}%
\begin{pgfscope}%
\pgfpathrectangle{\pgfqpoint{0.375000in}{0.330000in}}{\pgfqpoint{2.325000in}{2.310000in}}%
\pgfusepath{clip}%
\pgfsetbuttcap%
\pgfsetroundjoin%
\definecolor{currentfill}{rgb}{0.000000,0.000000,0.000000}%
\pgfsetfillcolor{currentfill}%
\pgfsetlinewidth{1.003750pt}%
\definecolor{currentstroke}{rgb}{0.000000,0.000000,0.000000}%
\pgfsetstrokecolor{currentstroke}%
\pgfsetdash{}{0pt}%
\pgfpathmoveto{\pgfqpoint{1.019812in}{0.630318in}}%
\pgfpathcurveto{\pgfqpoint{1.030863in}{0.630318in}}{\pgfqpoint{1.041462in}{0.634709in}}{\pgfqpoint{1.049275in}{0.642522in}}%
\pgfpathcurveto{\pgfqpoint{1.057089in}{0.650336in}}{\pgfqpoint{1.061479in}{0.660935in}}{\pgfqpoint{1.061479in}{0.671985in}}%
\pgfpathcurveto{\pgfqpoint{1.061479in}{0.683035in}}{\pgfqpoint{1.057089in}{0.693634in}}{\pgfqpoint{1.049275in}{0.701448in}}%
\pgfpathcurveto{\pgfqpoint{1.041462in}{0.709262in}}{\pgfqpoint{1.030863in}{0.713652in}}{\pgfqpoint{1.019812in}{0.713652in}}%
\pgfpathcurveto{\pgfqpoint{1.008762in}{0.713652in}}{\pgfqpoint{0.998163in}{0.709262in}}{\pgfqpoint{0.990350in}{0.701448in}}%
\pgfpathcurveto{\pgfqpoint{0.982536in}{0.693634in}}{\pgfqpoint{0.978146in}{0.683035in}}{\pgfqpoint{0.978146in}{0.671985in}}%
\pgfpathcurveto{\pgfqpoint{0.978146in}{0.660935in}}{\pgfqpoint{0.982536in}{0.650336in}}{\pgfqpoint{0.990350in}{0.642522in}}%
\pgfpathcurveto{\pgfqpoint{0.998163in}{0.634709in}}{\pgfqpoint{1.008762in}{0.630318in}}{\pgfqpoint{1.019812in}{0.630318in}}%
\pgfpathclose%
\pgfusepath{stroke,fill}%
\end{pgfscope}%
\begin{pgfscope}%
\pgfpathrectangle{\pgfqpoint{0.375000in}{0.330000in}}{\pgfqpoint{2.325000in}{2.310000in}}%
\pgfusepath{clip}%
\pgfsetbuttcap%
\pgfsetroundjoin%
\definecolor{currentfill}{rgb}{0.000000,0.000000,0.000000}%
\pgfsetfillcolor{currentfill}%
\pgfsetlinewidth{1.003750pt}%
\definecolor{currentstroke}{rgb}{0.000000,0.000000,0.000000}%
\pgfsetstrokecolor{currentstroke}%
\pgfsetdash{}{0pt}%
\pgfpathmoveto{\pgfqpoint{1.019812in}{0.709204in}}%
\pgfpathcurveto{\pgfqpoint{1.030863in}{0.709204in}}{\pgfqpoint{1.041462in}{0.713594in}}{\pgfqpoint{1.049275in}{0.721408in}}%
\pgfpathcurveto{\pgfqpoint{1.057089in}{0.729221in}}{\pgfqpoint{1.061479in}{0.739820in}}{\pgfqpoint{1.061479in}{0.750870in}}%
\pgfpathcurveto{\pgfqpoint{1.061479in}{0.761921in}}{\pgfqpoint{1.057089in}{0.772520in}}{\pgfqpoint{1.049275in}{0.780333in}}%
\pgfpathcurveto{\pgfqpoint{1.041462in}{0.788147in}}{\pgfqpoint{1.030863in}{0.792537in}}{\pgfqpoint{1.019812in}{0.792537in}}%
\pgfpathcurveto{\pgfqpoint{1.008762in}{0.792537in}}{\pgfqpoint{0.998163in}{0.788147in}}{\pgfqpoint{0.990350in}{0.780333in}}%
\pgfpathcurveto{\pgfqpoint{0.982536in}{0.772520in}}{\pgfqpoint{0.978146in}{0.761921in}}{\pgfqpoint{0.978146in}{0.750870in}}%
\pgfpathcurveto{\pgfqpoint{0.978146in}{0.739820in}}{\pgfqpoint{0.982536in}{0.729221in}}{\pgfqpoint{0.990350in}{0.721408in}}%
\pgfpathcurveto{\pgfqpoint{0.998163in}{0.713594in}}{\pgfqpoint{1.008762in}{0.709204in}}{\pgfqpoint{1.019812in}{0.709204in}}%
\pgfpathclose%
\pgfusepath{stroke,fill}%
\end{pgfscope}%
\begin{pgfscope}%
\pgfpathrectangle{\pgfqpoint{0.375000in}{0.330000in}}{\pgfqpoint{2.325000in}{2.310000in}}%
\pgfusepath{clip}%
\pgfsetbuttcap%
\pgfsetroundjoin%
\definecolor{currentfill}{rgb}{0.000000,0.000000,0.000000}%
\pgfsetfillcolor{currentfill}%
\pgfsetlinewidth{1.003750pt}%
\definecolor{currentstroke}{rgb}{0.000000,0.000000,0.000000}%
\pgfsetstrokecolor{currentstroke}%
\pgfsetdash{}{0pt}%
\pgfpathmoveto{\pgfqpoint{1.019812in}{0.636387in}}%
\pgfpathcurveto{\pgfqpoint{1.030863in}{0.636387in}}{\pgfqpoint{1.041462in}{0.640777in}}{\pgfqpoint{1.049275in}{0.648590in}}%
\pgfpathcurveto{\pgfqpoint{1.057089in}{0.656404in}}{\pgfqpoint{1.061479in}{0.667003in}}{\pgfqpoint{1.061479in}{0.678053in}}%
\pgfpathcurveto{\pgfqpoint{1.061479in}{0.689103in}}{\pgfqpoint{1.057089in}{0.699702in}}{\pgfqpoint{1.049275in}{0.707516in}}%
\pgfpathcurveto{\pgfqpoint{1.041462in}{0.715330in}}{\pgfqpoint{1.030863in}{0.719720in}}{\pgfqpoint{1.019812in}{0.719720in}}%
\pgfpathcurveto{\pgfqpoint{1.008762in}{0.719720in}}{\pgfqpoint{0.998163in}{0.715330in}}{\pgfqpoint{0.990350in}{0.707516in}}%
\pgfpathcurveto{\pgfqpoint{0.982536in}{0.699702in}}{\pgfqpoint{0.978146in}{0.689103in}}{\pgfqpoint{0.978146in}{0.678053in}}%
\pgfpathcurveto{\pgfqpoint{0.978146in}{0.667003in}}{\pgfqpoint{0.982536in}{0.656404in}}{\pgfqpoint{0.990350in}{0.648590in}}%
\pgfpathcurveto{\pgfqpoint{0.998163in}{0.640777in}}{\pgfqpoint{1.008762in}{0.636387in}}{\pgfqpoint{1.019812in}{0.636387in}}%
\pgfpathclose%
\pgfusepath{stroke,fill}%
\end{pgfscope}%
\begin{pgfscope}%
\pgfpathrectangle{\pgfqpoint{0.375000in}{0.330000in}}{\pgfqpoint{2.325000in}{2.310000in}}%
\pgfusepath{clip}%
\pgfsetbuttcap%
\pgfsetroundjoin%
\definecolor{currentfill}{rgb}{0.000000,0.000000,0.000000}%
\pgfsetfillcolor{currentfill}%
\pgfsetlinewidth{1.003750pt}%
\definecolor{currentstroke}{rgb}{0.000000,0.000000,0.000000}%
\pgfsetstrokecolor{currentstroke}%
\pgfsetdash{}{0pt}%
\pgfpathmoveto{\pgfqpoint{1.019812in}{0.581774in}}%
\pgfpathcurveto{\pgfqpoint{1.030863in}{0.581774in}}{\pgfqpoint{1.041462in}{0.586164in}}{\pgfqpoint{1.049275in}{0.593977in}}%
\pgfpathcurveto{\pgfqpoint{1.057089in}{0.601791in}}{\pgfqpoint{1.061479in}{0.612390in}}{\pgfqpoint{1.061479in}{0.623440in}}%
\pgfpathcurveto{\pgfqpoint{1.061479in}{0.634490in}}{\pgfqpoint{1.057089in}{0.645089in}}{\pgfqpoint{1.049275in}{0.652903in}}%
\pgfpathcurveto{\pgfqpoint{1.041462in}{0.660717in}}{\pgfqpoint{1.030863in}{0.665107in}}{\pgfqpoint{1.019812in}{0.665107in}}%
\pgfpathcurveto{\pgfqpoint{1.008762in}{0.665107in}}{\pgfqpoint{0.998163in}{0.660717in}}{\pgfqpoint{0.990350in}{0.652903in}}%
\pgfpathcurveto{\pgfqpoint{0.982536in}{0.645089in}}{\pgfqpoint{0.978146in}{0.634490in}}{\pgfqpoint{0.978146in}{0.623440in}}%
\pgfpathcurveto{\pgfqpoint{0.978146in}{0.612390in}}{\pgfqpoint{0.982536in}{0.601791in}}{\pgfqpoint{0.990350in}{0.593977in}}%
\pgfpathcurveto{\pgfqpoint{0.998163in}{0.586164in}}{\pgfqpoint{1.008762in}{0.581774in}}{\pgfqpoint{1.019812in}{0.581774in}}%
\pgfpathclose%
\pgfusepath{stroke,fill}%
\end{pgfscope}%
\begin{pgfscope}%
\pgfpathrectangle{\pgfqpoint{0.375000in}{0.330000in}}{\pgfqpoint{2.325000in}{2.310000in}}%
\pgfusepath{clip}%
\pgfsetbuttcap%
\pgfsetroundjoin%
\definecolor{currentfill}{rgb}{0.000000,0.000000,0.000000}%
\pgfsetfillcolor{currentfill}%
\pgfsetlinewidth{1.003750pt}%
\definecolor{currentstroke}{rgb}{0.000000,0.000000,0.000000}%
\pgfsetstrokecolor{currentstroke}%
\pgfsetdash{}{0pt}%
\pgfpathmoveto{\pgfqpoint{1.019812in}{0.666727in}}%
\pgfpathcurveto{\pgfqpoint{1.030863in}{0.666727in}}{\pgfqpoint{1.041462in}{0.671117in}}{\pgfqpoint{1.049275in}{0.678931in}}%
\pgfpathcurveto{\pgfqpoint{1.057089in}{0.686745in}}{\pgfqpoint{1.061479in}{0.697344in}}{\pgfqpoint{1.061479in}{0.708394in}}%
\pgfpathcurveto{\pgfqpoint{1.061479in}{0.719444in}}{\pgfqpoint{1.057089in}{0.730043in}}{\pgfqpoint{1.049275in}{0.737857in}}%
\pgfpathcurveto{\pgfqpoint{1.041462in}{0.745670in}}{\pgfqpoint{1.030863in}{0.750060in}}{\pgfqpoint{1.019812in}{0.750060in}}%
\pgfpathcurveto{\pgfqpoint{1.008762in}{0.750060in}}{\pgfqpoint{0.998163in}{0.745670in}}{\pgfqpoint{0.990350in}{0.737857in}}%
\pgfpathcurveto{\pgfqpoint{0.982536in}{0.730043in}}{\pgfqpoint{0.978146in}{0.719444in}}{\pgfqpoint{0.978146in}{0.708394in}}%
\pgfpathcurveto{\pgfqpoint{0.978146in}{0.697344in}}{\pgfqpoint{0.982536in}{0.686745in}}{\pgfqpoint{0.990350in}{0.678931in}}%
\pgfpathcurveto{\pgfqpoint{0.998163in}{0.671117in}}{\pgfqpoint{1.008762in}{0.666727in}}{\pgfqpoint{1.019812in}{0.666727in}}%
\pgfpathclose%
\pgfusepath{stroke,fill}%
\end{pgfscope}%
\begin{pgfscope}%
\pgfpathrectangle{\pgfqpoint{0.375000in}{0.330000in}}{\pgfqpoint{2.325000in}{2.310000in}}%
\pgfusepath{clip}%
\pgfsetbuttcap%
\pgfsetroundjoin%
\definecolor{currentfill}{rgb}{0.000000,0.000000,0.000000}%
\pgfsetfillcolor{currentfill}%
\pgfsetlinewidth{1.003750pt}%
\definecolor{currentstroke}{rgb}{0.000000,0.000000,0.000000}%
\pgfsetstrokecolor{currentstroke}%
\pgfsetdash{}{0pt}%
\pgfpathmoveto{\pgfqpoint{1.019812in}{0.606046in}}%
\pgfpathcurveto{\pgfqpoint{1.030863in}{0.606046in}}{\pgfqpoint{1.041462in}{0.610436in}}{\pgfqpoint{1.049275in}{0.618250in}}%
\pgfpathcurveto{\pgfqpoint{1.057089in}{0.626064in}}{\pgfqpoint{1.061479in}{0.636663in}}{\pgfqpoint{1.061479in}{0.647713in}}%
\pgfpathcurveto{\pgfqpoint{1.061479in}{0.658763in}}{\pgfqpoint{1.057089in}{0.669362in}}{\pgfqpoint{1.049275in}{0.677175in}}%
\pgfpathcurveto{\pgfqpoint{1.041462in}{0.684989in}}{\pgfqpoint{1.030863in}{0.689379in}}{\pgfqpoint{1.019812in}{0.689379in}}%
\pgfpathcurveto{\pgfqpoint{1.008762in}{0.689379in}}{\pgfqpoint{0.998163in}{0.684989in}}{\pgfqpoint{0.990350in}{0.677175in}}%
\pgfpathcurveto{\pgfqpoint{0.982536in}{0.669362in}}{\pgfqpoint{0.978146in}{0.658763in}}{\pgfqpoint{0.978146in}{0.647713in}}%
\pgfpathcurveto{\pgfqpoint{0.978146in}{0.636663in}}{\pgfqpoint{0.982536in}{0.626064in}}{\pgfqpoint{0.990350in}{0.618250in}}%
\pgfpathcurveto{\pgfqpoint{0.998163in}{0.610436in}}{\pgfqpoint{1.008762in}{0.606046in}}{\pgfqpoint{1.019812in}{0.606046in}}%
\pgfpathclose%
\pgfusepath{stroke,fill}%
\end{pgfscope}%
\begin{pgfscope}%
\pgfpathrectangle{\pgfqpoint{0.375000in}{0.330000in}}{\pgfqpoint{2.325000in}{2.310000in}}%
\pgfusepath{clip}%
\pgfsetbuttcap%
\pgfsetroundjoin%
\definecolor{currentfill}{rgb}{0.000000,0.000000,0.000000}%
\pgfsetfillcolor{currentfill}%
\pgfsetlinewidth{1.003750pt}%
\definecolor{currentstroke}{rgb}{0.000000,0.000000,0.000000}%
\pgfsetstrokecolor{currentstroke}%
\pgfsetdash{}{0pt}%
\pgfpathmoveto{\pgfqpoint{1.019812in}{0.599978in}}%
\pgfpathcurveto{\pgfqpoint{1.030863in}{0.599978in}}{\pgfqpoint{1.041462in}{0.604368in}}{\pgfqpoint{1.049275in}{0.612182in}}%
\pgfpathcurveto{\pgfqpoint{1.057089in}{0.619995in}}{\pgfqpoint{1.061479in}{0.630594in}}{\pgfqpoint{1.061479in}{0.641645in}}%
\pgfpathcurveto{\pgfqpoint{1.061479in}{0.652695in}}{\pgfqpoint{1.057089in}{0.663294in}}{\pgfqpoint{1.049275in}{0.671107in}}%
\pgfpathcurveto{\pgfqpoint{1.041462in}{0.678921in}}{\pgfqpoint{1.030863in}{0.683311in}}{\pgfqpoint{1.019812in}{0.683311in}}%
\pgfpathcurveto{\pgfqpoint{1.008762in}{0.683311in}}{\pgfqpoint{0.998163in}{0.678921in}}{\pgfqpoint{0.990350in}{0.671107in}}%
\pgfpathcurveto{\pgfqpoint{0.982536in}{0.663294in}}{\pgfqpoint{0.978146in}{0.652695in}}{\pgfqpoint{0.978146in}{0.641645in}}%
\pgfpathcurveto{\pgfqpoint{0.978146in}{0.630594in}}{\pgfqpoint{0.982536in}{0.619995in}}{\pgfqpoint{0.990350in}{0.612182in}}%
\pgfpathcurveto{\pgfqpoint{0.998163in}{0.604368in}}{\pgfqpoint{1.008762in}{0.599978in}}{\pgfqpoint{1.019812in}{0.599978in}}%
\pgfpathclose%
\pgfusepath{stroke,fill}%
\end{pgfscope}%
\begin{pgfscope}%
\pgfpathrectangle{\pgfqpoint{0.375000in}{0.330000in}}{\pgfqpoint{2.325000in}{2.310000in}}%
\pgfusepath{clip}%
\pgfsetbuttcap%
\pgfsetroundjoin%
\definecolor{currentfill}{rgb}{0.000000,0.000000,0.000000}%
\pgfsetfillcolor{currentfill}%
\pgfsetlinewidth{1.003750pt}%
\definecolor{currentstroke}{rgb}{0.000000,0.000000,0.000000}%
\pgfsetstrokecolor{currentstroke}%
\pgfsetdash{}{0pt}%
\pgfpathmoveto{\pgfqpoint{1.019812in}{0.599978in}}%
\pgfpathcurveto{\pgfqpoint{1.030863in}{0.599978in}}{\pgfqpoint{1.041462in}{0.604368in}}{\pgfqpoint{1.049275in}{0.612182in}}%
\pgfpathcurveto{\pgfqpoint{1.057089in}{0.619995in}}{\pgfqpoint{1.061479in}{0.630594in}}{\pgfqpoint{1.061479in}{0.641645in}}%
\pgfpathcurveto{\pgfqpoint{1.061479in}{0.652695in}}{\pgfqpoint{1.057089in}{0.663294in}}{\pgfqpoint{1.049275in}{0.671107in}}%
\pgfpathcurveto{\pgfqpoint{1.041462in}{0.678921in}}{\pgfqpoint{1.030863in}{0.683311in}}{\pgfqpoint{1.019812in}{0.683311in}}%
\pgfpathcurveto{\pgfqpoint{1.008762in}{0.683311in}}{\pgfqpoint{0.998163in}{0.678921in}}{\pgfqpoint{0.990350in}{0.671107in}}%
\pgfpathcurveto{\pgfqpoint{0.982536in}{0.663294in}}{\pgfqpoint{0.978146in}{0.652695in}}{\pgfqpoint{0.978146in}{0.641645in}}%
\pgfpathcurveto{\pgfqpoint{0.978146in}{0.630594in}}{\pgfqpoint{0.982536in}{0.619995in}}{\pgfqpoint{0.990350in}{0.612182in}}%
\pgfpathcurveto{\pgfqpoint{0.998163in}{0.604368in}}{\pgfqpoint{1.008762in}{0.599978in}}{\pgfqpoint{1.019812in}{0.599978in}}%
\pgfpathclose%
\pgfusepath{stroke,fill}%
\end{pgfscope}%
\begin{pgfscope}%
\pgfpathrectangle{\pgfqpoint{0.375000in}{0.330000in}}{\pgfqpoint{2.325000in}{2.310000in}}%
\pgfusepath{clip}%
\pgfsetbuttcap%
\pgfsetroundjoin%
\definecolor{currentfill}{rgb}{0.000000,0.000000,0.000000}%
\pgfsetfillcolor{currentfill}%
\pgfsetlinewidth{1.003750pt}%
\definecolor{currentstroke}{rgb}{0.000000,0.000000,0.000000}%
\pgfsetstrokecolor{currentstroke}%
\pgfsetdash{}{0pt}%
\pgfpathmoveto{\pgfqpoint{1.019812in}{0.709204in}}%
\pgfpathcurveto{\pgfqpoint{1.030863in}{0.709204in}}{\pgfqpoint{1.041462in}{0.713594in}}{\pgfqpoint{1.049275in}{0.721408in}}%
\pgfpathcurveto{\pgfqpoint{1.057089in}{0.729221in}}{\pgfqpoint{1.061479in}{0.739820in}}{\pgfqpoint{1.061479in}{0.750870in}}%
\pgfpathcurveto{\pgfqpoint{1.061479in}{0.761921in}}{\pgfqpoint{1.057089in}{0.772520in}}{\pgfqpoint{1.049275in}{0.780333in}}%
\pgfpathcurveto{\pgfqpoint{1.041462in}{0.788147in}}{\pgfqpoint{1.030863in}{0.792537in}}{\pgfqpoint{1.019812in}{0.792537in}}%
\pgfpathcurveto{\pgfqpoint{1.008762in}{0.792537in}}{\pgfqpoint{0.998163in}{0.788147in}}{\pgfqpoint{0.990350in}{0.780333in}}%
\pgfpathcurveto{\pgfqpoint{0.982536in}{0.772520in}}{\pgfqpoint{0.978146in}{0.761921in}}{\pgfqpoint{0.978146in}{0.750870in}}%
\pgfpathcurveto{\pgfqpoint{0.978146in}{0.739820in}}{\pgfqpoint{0.982536in}{0.729221in}}{\pgfqpoint{0.990350in}{0.721408in}}%
\pgfpathcurveto{\pgfqpoint{0.998163in}{0.713594in}}{\pgfqpoint{1.008762in}{0.709204in}}{\pgfqpoint{1.019812in}{0.709204in}}%
\pgfpathclose%
\pgfusepath{stroke,fill}%
\end{pgfscope}%
\begin{pgfscope}%
\pgfpathrectangle{\pgfqpoint{0.375000in}{0.330000in}}{\pgfqpoint{2.325000in}{2.310000in}}%
\pgfusepath{clip}%
\pgfsetbuttcap%
\pgfsetroundjoin%
\definecolor{currentfill}{rgb}{0.000000,0.000000,0.000000}%
\pgfsetfillcolor{currentfill}%
\pgfsetlinewidth{1.003750pt}%
\definecolor{currentstroke}{rgb}{0.000000,0.000000,0.000000}%
\pgfsetstrokecolor{currentstroke}%
\pgfsetdash{}{0pt}%
\pgfpathmoveto{\pgfqpoint{1.019812in}{0.636387in}}%
\pgfpathcurveto{\pgfqpoint{1.030863in}{0.636387in}}{\pgfqpoint{1.041462in}{0.640777in}}{\pgfqpoint{1.049275in}{0.648590in}}%
\pgfpathcurveto{\pgfqpoint{1.057089in}{0.656404in}}{\pgfqpoint{1.061479in}{0.667003in}}{\pgfqpoint{1.061479in}{0.678053in}}%
\pgfpathcurveto{\pgfqpoint{1.061479in}{0.689103in}}{\pgfqpoint{1.057089in}{0.699702in}}{\pgfqpoint{1.049275in}{0.707516in}}%
\pgfpathcurveto{\pgfqpoint{1.041462in}{0.715330in}}{\pgfqpoint{1.030863in}{0.719720in}}{\pgfqpoint{1.019812in}{0.719720in}}%
\pgfpathcurveto{\pgfqpoint{1.008762in}{0.719720in}}{\pgfqpoint{0.998163in}{0.715330in}}{\pgfqpoint{0.990350in}{0.707516in}}%
\pgfpathcurveto{\pgfqpoint{0.982536in}{0.699702in}}{\pgfqpoint{0.978146in}{0.689103in}}{\pgfqpoint{0.978146in}{0.678053in}}%
\pgfpathcurveto{\pgfqpoint{0.978146in}{0.667003in}}{\pgfqpoint{0.982536in}{0.656404in}}{\pgfqpoint{0.990350in}{0.648590in}}%
\pgfpathcurveto{\pgfqpoint{0.998163in}{0.640777in}}{\pgfqpoint{1.008762in}{0.636387in}}{\pgfqpoint{1.019812in}{0.636387in}}%
\pgfpathclose%
\pgfusepath{stroke,fill}%
\end{pgfscope}%
\begin{pgfscope}%
\pgfpathrectangle{\pgfqpoint{0.375000in}{0.330000in}}{\pgfqpoint{2.325000in}{2.310000in}}%
\pgfusepath{clip}%
\pgfsetbuttcap%
\pgfsetroundjoin%
\definecolor{currentfill}{rgb}{0.000000,0.000000,0.000000}%
\pgfsetfillcolor{currentfill}%
\pgfsetlinewidth{1.003750pt}%
\definecolor{currentstroke}{rgb}{0.000000,0.000000,0.000000}%
\pgfsetstrokecolor{currentstroke}%
\pgfsetdash{}{0pt}%
\pgfpathmoveto{\pgfqpoint{1.019812in}{0.569637in}}%
\pgfpathcurveto{\pgfqpoint{1.030863in}{0.569637in}}{\pgfqpoint{1.041462in}{0.574028in}}{\pgfqpoint{1.049275in}{0.581841in}}%
\pgfpathcurveto{\pgfqpoint{1.057089in}{0.589655in}}{\pgfqpoint{1.061479in}{0.600254in}}{\pgfqpoint{1.061479in}{0.611304in}}%
\pgfpathcurveto{\pgfqpoint{1.061479in}{0.622354in}}{\pgfqpoint{1.057089in}{0.632953in}}{\pgfqpoint{1.049275in}{0.640767in}}%
\pgfpathcurveto{\pgfqpoint{1.041462in}{0.648580in}}{\pgfqpoint{1.030863in}{0.652971in}}{\pgfqpoint{1.019812in}{0.652971in}}%
\pgfpathcurveto{\pgfqpoint{1.008762in}{0.652971in}}{\pgfqpoint{0.998163in}{0.648580in}}{\pgfqpoint{0.990350in}{0.640767in}}%
\pgfpathcurveto{\pgfqpoint{0.982536in}{0.632953in}}{\pgfqpoint{0.978146in}{0.622354in}}{\pgfqpoint{0.978146in}{0.611304in}}%
\pgfpathcurveto{\pgfqpoint{0.978146in}{0.600254in}}{\pgfqpoint{0.982536in}{0.589655in}}{\pgfqpoint{0.990350in}{0.581841in}}%
\pgfpathcurveto{\pgfqpoint{0.998163in}{0.574028in}}{\pgfqpoint{1.008762in}{0.569637in}}{\pgfqpoint{1.019812in}{0.569637in}}%
\pgfpathclose%
\pgfusepath{stroke,fill}%
\end{pgfscope}%
\begin{pgfscope}%
\pgfpathrectangle{\pgfqpoint{0.375000in}{0.330000in}}{\pgfqpoint{2.325000in}{2.310000in}}%
\pgfusepath{clip}%
\pgfsetbuttcap%
\pgfsetroundjoin%
\definecolor{currentfill}{rgb}{0.000000,0.000000,0.000000}%
\pgfsetfillcolor{currentfill}%
\pgfsetlinewidth{1.003750pt}%
\definecolor{currentstroke}{rgb}{0.000000,0.000000,0.000000}%
\pgfsetstrokecolor{currentstroke}%
\pgfsetdash{}{0pt}%
\pgfpathmoveto{\pgfqpoint{1.019812in}{0.612114in}}%
\pgfpathcurveto{\pgfqpoint{1.030863in}{0.612114in}}{\pgfqpoint{1.041462in}{0.616504in}}{\pgfqpoint{1.049275in}{0.624318in}}%
\pgfpathcurveto{\pgfqpoint{1.057089in}{0.632132in}}{\pgfqpoint{1.061479in}{0.642731in}}{\pgfqpoint{1.061479in}{0.653781in}}%
\pgfpathcurveto{\pgfqpoint{1.061479in}{0.664831in}}{\pgfqpoint{1.057089in}{0.675430in}}{\pgfqpoint{1.049275in}{0.683244in}}%
\pgfpathcurveto{\pgfqpoint{1.041462in}{0.691057in}}{\pgfqpoint{1.030863in}{0.695447in}}{\pgfqpoint{1.019812in}{0.695447in}}%
\pgfpathcurveto{\pgfqpoint{1.008762in}{0.695447in}}{\pgfqpoint{0.998163in}{0.691057in}}{\pgfqpoint{0.990350in}{0.683244in}}%
\pgfpathcurveto{\pgfqpoint{0.982536in}{0.675430in}}{\pgfqpoint{0.978146in}{0.664831in}}{\pgfqpoint{0.978146in}{0.653781in}}%
\pgfpathcurveto{\pgfqpoint{0.978146in}{0.642731in}}{\pgfqpoint{0.982536in}{0.632132in}}{\pgfqpoint{0.990350in}{0.624318in}}%
\pgfpathcurveto{\pgfqpoint{0.998163in}{0.616504in}}{\pgfqpoint{1.008762in}{0.612114in}}{\pgfqpoint{1.019812in}{0.612114in}}%
\pgfpathclose%
\pgfusepath{stroke,fill}%
\end{pgfscope}%
\begin{pgfscope}%
\pgfpathrectangle{\pgfqpoint{0.375000in}{0.330000in}}{\pgfqpoint{2.325000in}{2.310000in}}%
\pgfusepath{clip}%
\pgfsetbuttcap%
\pgfsetroundjoin%
\definecolor{currentfill}{rgb}{0.000000,0.000000,0.000000}%
\pgfsetfillcolor{currentfill}%
\pgfsetlinewidth{1.003750pt}%
\definecolor{currentstroke}{rgb}{0.000000,0.000000,0.000000}%
\pgfsetstrokecolor{currentstroke}%
\pgfsetdash{}{0pt}%
\pgfpathmoveto{\pgfqpoint{1.019812in}{0.697068in}}%
\pgfpathcurveto{\pgfqpoint{1.030863in}{0.697068in}}{\pgfqpoint{1.041462in}{0.701458in}}{\pgfqpoint{1.049275in}{0.709272in}}%
\pgfpathcurveto{\pgfqpoint{1.057089in}{0.717085in}}{\pgfqpoint{1.061479in}{0.727684in}}{\pgfqpoint{1.061479in}{0.738734in}}%
\pgfpathcurveto{\pgfqpoint{1.061479in}{0.749784in}}{\pgfqpoint{1.057089in}{0.760383in}}{\pgfqpoint{1.049275in}{0.768197in}}%
\pgfpathcurveto{\pgfqpoint{1.041462in}{0.776011in}}{\pgfqpoint{1.030863in}{0.780401in}}{\pgfqpoint{1.019812in}{0.780401in}}%
\pgfpathcurveto{\pgfqpoint{1.008762in}{0.780401in}}{\pgfqpoint{0.998163in}{0.776011in}}{\pgfqpoint{0.990350in}{0.768197in}}%
\pgfpathcurveto{\pgfqpoint{0.982536in}{0.760383in}}{\pgfqpoint{0.978146in}{0.749784in}}{\pgfqpoint{0.978146in}{0.738734in}}%
\pgfpathcurveto{\pgfqpoint{0.978146in}{0.727684in}}{\pgfqpoint{0.982536in}{0.717085in}}{\pgfqpoint{0.990350in}{0.709272in}}%
\pgfpathcurveto{\pgfqpoint{0.998163in}{0.701458in}}{\pgfqpoint{1.008762in}{0.697068in}}{\pgfqpoint{1.019812in}{0.697068in}}%
\pgfpathclose%
\pgfusepath{stroke,fill}%
\end{pgfscope}%
\begin{pgfscope}%
\pgfpathrectangle{\pgfqpoint{0.375000in}{0.330000in}}{\pgfqpoint{2.325000in}{2.310000in}}%
\pgfusepath{clip}%
\pgfsetbuttcap%
\pgfsetroundjoin%
\definecolor{currentfill}{rgb}{0.000000,0.000000,0.000000}%
\pgfsetfillcolor{currentfill}%
\pgfsetlinewidth{1.003750pt}%
\definecolor{currentstroke}{rgb}{0.000000,0.000000,0.000000}%
\pgfsetstrokecolor{currentstroke}%
\pgfsetdash{}{0pt}%
\pgfpathmoveto{\pgfqpoint{1.019812in}{0.618182in}}%
\pgfpathcurveto{\pgfqpoint{1.030863in}{0.618182in}}{\pgfqpoint{1.041462in}{0.622573in}}{\pgfqpoint{1.049275in}{0.630386in}}%
\pgfpathcurveto{\pgfqpoint{1.057089in}{0.638200in}}{\pgfqpoint{1.061479in}{0.648799in}}{\pgfqpoint{1.061479in}{0.659849in}}%
\pgfpathcurveto{\pgfqpoint{1.061479in}{0.670899in}}{\pgfqpoint{1.057089in}{0.681498in}}{\pgfqpoint{1.049275in}{0.689312in}}%
\pgfpathcurveto{\pgfqpoint{1.041462in}{0.697125in}}{\pgfqpoint{1.030863in}{0.701516in}}{\pgfqpoint{1.019812in}{0.701516in}}%
\pgfpathcurveto{\pgfqpoint{1.008762in}{0.701516in}}{\pgfqpoint{0.998163in}{0.697125in}}{\pgfqpoint{0.990350in}{0.689312in}}%
\pgfpathcurveto{\pgfqpoint{0.982536in}{0.681498in}}{\pgfqpoint{0.978146in}{0.670899in}}{\pgfqpoint{0.978146in}{0.659849in}}%
\pgfpathcurveto{\pgfqpoint{0.978146in}{0.648799in}}{\pgfqpoint{0.982536in}{0.638200in}}{\pgfqpoint{0.990350in}{0.630386in}}%
\pgfpathcurveto{\pgfqpoint{0.998163in}{0.622573in}}{\pgfqpoint{1.008762in}{0.618182in}}{\pgfqpoint{1.019812in}{0.618182in}}%
\pgfpathclose%
\pgfusepath{stroke,fill}%
\end{pgfscope}%
\begin{pgfscope}%
\pgfpathrectangle{\pgfqpoint{0.375000in}{0.330000in}}{\pgfqpoint{2.325000in}{2.310000in}}%
\pgfusepath{clip}%
\pgfsetbuttcap%
\pgfsetroundjoin%
\definecolor{currentfill}{rgb}{0.000000,0.000000,0.000000}%
\pgfsetfillcolor{currentfill}%
\pgfsetlinewidth{1.003750pt}%
\definecolor{currentstroke}{rgb}{0.000000,0.000000,0.000000}%
\pgfsetstrokecolor{currentstroke}%
\pgfsetdash{}{0pt}%
\pgfpathmoveto{\pgfqpoint{1.019812in}{0.666727in}}%
\pgfpathcurveto{\pgfqpoint{1.030863in}{0.666727in}}{\pgfqpoint{1.041462in}{0.671117in}}{\pgfqpoint{1.049275in}{0.678931in}}%
\pgfpathcurveto{\pgfqpoint{1.057089in}{0.686745in}}{\pgfqpoint{1.061479in}{0.697344in}}{\pgfqpoint{1.061479in}{0.708394in}}%
\pgfpathcurveto{\pgfqpoint{1.061479in}{0.719444in}}{\pgfqpoint{1.057089in}{0.730043in}}{\pgfqpoint{1.049275in}{0.737857in}}%
\pgfpathcurveto{\pgfqpoint{1.041462in}{0.745670in}}{\pgfqpoint{1.030863in}{0.750060in}}{\pgfqpoint{1.019812in}{0.750060in}}%
\pgfpathcurveto{\pgfqpoint{1.008762in}{0.750060in}}{\pgfqpoint{0.998163in}{0.745670in}}{\pgfqpoint{0.990350in}{0.737857in}}%
\pgfpathcurveto{\pgfqpoint{0.982536in}{0.730043in}}{\pgfqpoint{0.978146in}{0.719444in}}{\pgfqpoint{0.978146in}{0.708394in}}%
\pgfpathcurveto{\pgfqpoint{0.978146in}{0.697344in}}{\pgfqpoint{0.982536in}{0.686745in}}{\pgfqpoint{0.990350in}{0.678931in}}%
\pgfpathcurveto{\pgfqpoint{0.998163in}{0.671117in}}{\pgfqpoint{1.008762in}{0.666727in}}{\pgfqpoint{1.019812in}{0.666727in}}%
\pgfpathclose%
\pgfusepath{stroke,fill}%
\end{pgfscope}%
\begin{pgfscope}%
\pgfpathrectangle{\pgfqpoint{0.375000in}{0.330000in}}{\pgfqpoint{2.325000in}{2.310000in}}%
\pgfusepath{clip}%
\pgfsetbuttcap%
\pgfsetroundjoin%
\definecolor{currentfill}{rgb}{0.000000,0.000000,0.000000}%
\pgfsetfillcolor{currentfill}%
\pgfsetlinewidth{1.003750pt}%
\definecolor{currentstroke}{rgb}{0.000000,0.000000,0.000000}%
\pgfsetstrokecolor{currentstroke}%
\pgfsetdash{}{0pt}%
\pgfpathmoveto{\pgfqpoint{1.019812in}{0.642455in}}%
\pgfpathcurveto{\pgfqpoint{1.030863in}{0.642455in}}{\pgfqpoint{1.041462in}{0.646845in}}{\pgfqpoint{1.049275in}{0.654659in}}%
\pgfpathcurveto{\pgfqpoint{1.057089in}{0.662472in}}{\pgfqpoint{1.061479in}{0.673071in}}{\pgfqpoint{1.061479in}{0.684121in}}%
\pgfpathcurveto{\pgfqpoint{1.061479in}{0.695171in}}{\pgfqpoint{1.057089in}{0.705770in}}{\pgfqpoint{1.049275in}{0.713584in}}%
\pgfpathcurveto{\pgfqpoint{1.041462in}{0.721398in}}{\pgfqpoint{1.030863in}{0.725788in}}{\pgfqpoint{1.019812in}{0.725788in}}%
\pgfpathcurveto{\pgfqpoint{1.008762in}{0.725788in}}{\pgfqpoint{0.998163in}{0.721398in}}{\pgfqpoint{0.990350in}{0.713584in}}%
\pgfpathcurveto{\pgfqpoint{0.982536in}{0.705770in}}{\pgfqpoint{0.978146in}{0.695171in}}{\pgfqpoint{0.978146in}{0.684121in}}%
\pgfpathcurveto{\pgfqpoint{0.978146in}{0.673071in}}{\pgfqpoint{0.982536in}{0.662472in}}{\pgfqpoint{0.990350in}{0.654659in}}%
\pgfpathcurveto{\pgfqpoint{0.998163in}{0.646845in}}{\pgfqpoint{1.008762in}{0.642455in}}{\pgfqpoint{1.019812in}{0.642455in}}%
\pgfpathclose%
\pgfusepath{stroke,fill}%
\end{pgfscope}%
\begin{pgfscope}%
\pgfpathrectangle{\pgfqpoint{0.375000in}{0.330000in}}{\pgfqpoint{2.325000in}{2.310000in}}%
\pgfusepath{clip}%
\pgfsetbuttcap%
\pgfsetroundjoin%
\definecolor{currentfill}{rgb}{0.000000,0.000000,0.000000}%
\pgfsetfillcolor{currentfill}%
\pgfsetlinewidth{1.003750pt}%
\definecolor{currentstroke}{rgb}{0.000000,0.000000,0.000000}%
\pgfsetstrokecolor{currentstroke}%
\pgfsetdash{}{0pt}%
\pgfpathmoveto{\pgfqpoint{1.019812in}{0.612114in}}%
\pgfpathcurveto{\pgfqpoint{1.030863in}{0.612114in}}{\pgfqpoint{1.041462in}{0.616504in}}{\pgfqpoint{1.049275in}{0.624318in}}%
\pgfpathcurveto{\pgfqpoint{1.057089in}{0.632132in}}{\pgfqpoint{1.061479in}{0.642731in}}{\pgfqpoint{1.061479in}{0.653781in}}%
\pgfpathcurveto{\pgfqpoint{1.061479in}{0.664831in}}{\pgfqpoint{1.057089in}{0.675430in}}{\pgfqpoint{1.049275in}{0.683244in}}%
\pgfpathcurveto{\pgfqpoint{1.041462in}{0.691057in}}{\pgfqpoint{1.030863in}{0.695447in}}{\pgfqpoint{1.019812in}{0.695447in}}%
\pgfpathcurveto{\pgfqpoint{1.008762in}{0.695447in}}{\pgfqpoint{0.998163in}{0.691057in}}{\pgfqpoint{0.990350in}{0.683244in}}%
\pgfpathcurveto{\pgfqpoint{0.982536in}{0.675430in}}{\pgfqpoint{0.978146in}{0.664831in}}{\pgfqpoint{0.978146in}{0.653781in}}%
\pgfpathcurveto{\pgfqpoint{0.978146in}{0.642731in}}{\pgfqpoint{0.982536in}{0.632132in}}{\pgfqpoint{0.990350in}{0.624318in}}%
\pgfpathcurveto{\pgfqpoint{0.998163in}{0.616504in}}{\pgfqpoint{1.008762in}{0.612114in}}{\pgfqpoint{1.019812in}{0.612114in}}%
\pgfpathclose%
\pgfusepath{stroke,fill}%
\end{pgfscope}%
\begin{pgfscope}%
\pgfpathrectangle{\pgfqpoint{0.375000in}{0.330000in}}{\pgfqpoint{2.325000in}{2.310000in}}%
\pgfusepath{clip}%
\pgfsetbuttcap%
\pgfsetroundjoin%
\definecolor{currentfill}{rgb}{0.000000,0.000000,0.000000}%
\pgfsetfillcolor{currentfill}%
\pgfsetlinewidth{1.003750pt}%
\definecolor{currentstroke}{rgb}{0.000000,0.000000,0.000000}%
\pgfsetstrokecolor{currentstroke}%
\pgfsetdash{}{0pt}%
\pgfpathmoveto{\pgfqpoint{1.019812in}{0.606046in}}%
\pgfpathcurveto{\pgfqpoint{1.030863in}{0.606046in}}{\pgfqpoint{1.041462in}{0.610436in}}{\pgfqpoint{1.049275in}{0.618250in}}%
\pgfpathcurveto{\pgfqpoint{1.057089in}{0.626064in}}{\pgfqpoint{1.061479in}{0.636663in}}{\pgfqpoint{1.061479in}{0.647713in}}%
\pgfpathcurveto{\pgfqpoint{1.061479in}{0.658763in}}{\pgfqpoint{1.057089in}{0.669362in}}{\pgfqpoint{1.049275in}{0.677175in}}%
\pgfpathcurveto{\pgfqpoint{1.041462in}{0.684989in}}{\pgfqpoint{1.030863in}{0.689379in}}{\pgfqpoint{1.019812in}{0.689379in}}%
\pgfpathcurveto{\pgfqpoint{1.008762in}{0.689379in}}{\pgfqpoint{0.998163in}{0.684989in}}{\pgfqpoint{0.990350in}{0.677175in}}%
\pgfpathcurveto{\pgfqpoint{0.982536in}{0.669362in}}{\pgfqpoint{0.978146in}{0.658763in}}{\pgfqpoint{0.978146in}{0.647713in}}%
\pgfpathcurveto{\pgfqpoint{0.978146in}{0.636663in}}{\pgfqpoint{0.982536in}{0.626064in}}{\pgfqpoint{0.990350in}{0.618250in}}%
\pgfpathcurveto{\pgfqpoint{0.998163in}{0.610436in}}{\pgfqpoint{1.008762in}{0.606046in}}{\pgfqpoint{1.019812in}{0.606046in}}%
\pgfpathclose%
\pgfusepath{stroke,fill}%
\end{pgfscope}%
\begin{pgfscope}%
\pgfpathrectangle{\pgfqpoint{0.375000in}{0.330000in}}{\pgfqpoint{2.325000in}{2.310000in}}%
\pgfusepath{clip}%
\pgfsetbuttcap%
\pgfsetroundjoin%
\definecolor{currentfill}{rgb}{0.000000,0.000000,0.000000}%
\pgfsetfillcolor{currentfill}%
\pgfsetlinewidth{1.003750pt}%
\definecolor{currentstroke}{rgb}{0.000000,0.000000,0.000000}%
\pgfsetstrokecolor{currentstroke}%
\pgfsetdash{}{0pt}%
\pgfpathmoveto{\pgfqpoint{1.019812in}{0.666727in}}%
\pgfpathcurveto{\pgfqpoint{1.030863in}{0.666727in}}{\pgfqpoint{1.041462in}{0.671117in}}{\pgfqpoint{1.049275in}{0.678931in}}%
\pgfpathcurveto{\pgfqpoint{1.057089in}{0.686745in}}{\pgfqpoint{1.061479in}{0.697344in}}{\pgfqpoint{1.061479in}{0.708394in}}%
\pgfpathcurveto{\pgfqpoint{1.061479in}{0.719444in}}{\pgfqpoint{1.057089in}{0.730043in}}{\pgfqpoint{1.049275in}{0.737857in}}%
\pgfpathcurveto{\pgfqpoint{1.041462in}{0.745670in}}{\pgfqpoint{1.030863in}{0.750060in}}{\pgfqpoint{1.019812in}{0.750060in}}%
\pgfpathcurveto{\pgfqpoint{1.008762in}{0.750060in}}{\pgfqpoint{0.998163in}{0.745670in}}{\pgfqpoint{0.990350in}{0.737857in}}%
\pgfpathcurveto{\pgfqpoint{0.982536in}{0.730043in}}{\pgfqpoint{0.978146in}{0.719444in}}{\pgfqpoint{0.978146in}{0.708394in}}%
\pgfpathcurveto{\pgfqpoint{0.978146in}{0.697344in}}{\pgfqpoint{0.982536in}{0.686745in}}{\pgfqpoint{0.990350in}{0.678931in}}%
\pgfpathcurveto{\pgfqpoint{0.998163in}{0.671117in}}{\pgfqpoint{1.008762in}{0.666727in}}{\pgfqpoint{1.019812in}{0.666727in}}%
\pgfpathclose%
\pgfusepath{stroke,fill}%
\end{pgfscope}%
\begin{pgfscope}%
\pgfpathrectangle{\pgfqpoint{0.375000in}{0.330000in}}{\pgfqpoint{2.325000in}{2.310000in}}%
\pgfusepath{clip}%
\pgfsetbuttcap%
\pgfsetroundjoin%
\definecolor{currentfill}{rgb}{0.000000,0.000000,0.000000}%
\pgfsetfillcolor{currentfill}%
\pgfsetlinewidth{1.003750pt}%
\definecolor{currentstroke}{rgb}{0.000000,0.000000,0.000000}%
\pgfsetstrokecolor{currentstroke}%
\pgfsetdash{}{0pt}%
\pgfpathmoveto{\pgfqpoint{1.019812in}{0.587842in}}%
\pgfpathcurveto{\pgfqpoint{1.030863in}{0.587842in}}{\pgfqpoint{1.041462in}{0.592232in}}{\pgfqpoint{1.049275in}{0.600046in}}%
\pgfpathcurveto{\pgfqpoint{1.057089in}{0.607859in}}{\pgfqpoint{1.061479in}{0.618458in}}{\pgfqpoint{1.061479in}{0.629508in}}%
\pgfpathcurveto{\pgfqpoint{1.061479in}{0.640559in}}{\pgfqpoint{1.057089in}{0.651158in}}{\pgfqpoint{1.049275in}{0.658971in}}%
\pgfpathcurveto{\pgfqpoint{1.041462in}{0.666785in}}{\pgfqpoint{1.030863in}{0.671175in}}{\pgfqpoint{1.019812in}{0.671175in}}%
\pgfpathcurveto{\pgfqpoint{1.008762in}{0.671175in}}{\pgfqpoint{0.998163in}{0.666785in}}{\pgfqpoint{0.990350in}{0.658971in}}%
\pgfpathcurveto{\pgfqpoint{0.982536in}{0.651158in}}{\pgfqpoint{0.978146in}{0.640559in}}{\pgfqpoint{0.978146in}{0.629508in}}%
\pgfpathcurveto{\pgfqpoint{0.978146in}{0.618458in}}{\pgfqpoint{0.982536in}{0.607859in}}{\pgfqpoint{0.990350in}{0.600046in}}%
\pgfpathcurveto{\pgfqpoint{0.998163in}{0.592232in}}{\pgfqpoint{1.008762in}{0.587842in}}{\pgfqpoint{1.019812in}{0.587842in}}%
\pgfpathclose%
\pgfusepath{stroke,fill}%
\end{pgfscope}%
\begin{pgfscope}%
\pgfpathrectangle{\pgfqpoint{0.375000in}{0.330000in}}{\pgfqpoint{2.325000in}{2.310000in}}%
\pgfusepath{clip}%
\pgfsetbuttcap%
\pgfsetroundjoin%
\definecolor{currentfill}{rgb}{0.000000,0.000000,0.000000}%
\pgfsetfillcolor{currentfill}%
\pgfsetlinewidth{1.003750pt}%
\definecolor{currentstroke}{rgb}{0.000000,0.000000,0.000000}%
\pgfsetstrokecolor{currentstroke}%
\pgfsetdash{}{0pt}%
\pgfpathmoveto{\pgfqpoint{1.019812in}{0.642455in}}%
\pgfpathcurveto{\pgfqpoint{1.030863in}{0.642455in}}{\pgfqpoint{1.041462in}{0.646845in}}{\pgfqpoint{1.049275in}{0.654659in}}%
\pgfpathcurveto{\pgfqpoint{1.057089in}{0.662472in}}{\pgfqpoint{1.061479in}{0.673071in}}{\pgfqpoint{1.061479in}{0.684121in}}%
\pgfpathcurveto{\pgfqpoint{1.061479in}{0.695171in}}{\pgfqpoint{1.057089in}{0.705770in}}{\pgfqpoint{1.049275in}{0.713584in}}%
\pgfpathcurveto{\pgfqpoint{1.041462in}{0.721398in}}{\pgfqpoint{1.030863in}{0.725788in}}{\pgfqpoint{1.019812in}{0.725788in}}%
\pgfpathcurveto{\pgfqpoint{1.008762in}{0.725788in}}{\pgfqpoint{0.998163in}{0.721398in}}{\pgfqpoint{0.990350in}{0.713584in}}%
\pgfpathcurveto{\pgfqpoint{0.982536in}{0.705770in}}{\pgfqpoint{0.978146in}{0.695171in}}{\pgfqpoint{0.978146in}{0.684121in}}%
\pgfpathcurveto{\pgfqpoint{0.978146in}{0.673071in}}{\pgfqpoint{0.982536in}{0.662472in}}{\pgfqpoint{0.990350in}{0.654659in}}%
\pgfpathcurveto{\pgfqpoint{0.998163in}{0.646845in}}{\pgfqpoint{1.008762in}{0.642455in}}{\pgfqpoint{1.019812in}{0.642455in}}%
\pgfpathclose%
\pgfusepath{stroke,fill}%
\end{pgfscope}%
\begin{pgfscope}%
\pgfpathrectangle{\pgfqpoint{0.375000in}{0.330000in}}{\pgfqpoint{2.325000in}{2.310000in}}%
\pgfusepath{clip}%
\pgfsetbuttcap%
\pgfsetroundjoin%
\definecolor{currentfill}{rgb}{0.000000,0.000000,0.000000}%
\pgfsetfillcolor{currentfill}%
\pgfsetlinewidth{1.003750pt}%
\definecolor{currentstroke}{rgb}{0.000000,0.000000,0.000000}%
\pgfsetstrokecolor{currentstroke}%
\pgfsetdash{}{0pt}%
\pgfpathmoveto{\pgfqpoint{1.019812in}{0.618182in}}%
\pgfpathcurveto{\pgfqpoint{1.030863in}{0.618182in}}{\pgfqpoint{1.041462in}{0.622573in}}{\pgfqpoint{1.049275in}{0.630386in}}%
\pgfpathcurveto{\pgfqpoint{1.057089in}{0.638200in}}{\pgfqpoint{1.061479in}{0.648799in}}{\pgfqpoint{1.061479in}{0.659849in}}%
\pgfpathcurveto{\pgfqpoint{1.061479in}{0.670899in}}{\pgfqpoint{1.057089in}{0.681498in}}{\pgfqpoint{1.049275in}{0.689312in}}%
\pgfpathcurveto{\pgfqpoint{1.041462in}{0.697125in}}{\pgfqpoint{1.030863in}{0.701516in}}{\pgfqpoint{1.019812in}{0.701516in}}%
\pgfpathcurveto{\pgfqpoint{1.008762in}{0.701516in}}{\pgfqpoint{0.998163in}{0.697125in}}{\pgfqpoint{0.990350in}{0.689312in}}%
\pgfpathcurveto{\pgfqpoint{0.982536in}{0.681498in}}{\pgfqpoint{0.978146in}{0.670899in}}{\pgfqpoint{0.978146in}{0.659849in}}%
\pgfpathcurveto{\pgfqpoint{0.978146in}{0.648799in}}{\pgfqpoint{0.982536in}{0.638200in}}{\pgfqpoint{0.990350in}{0.630386in}}%
\pgfpathcurveto{\pgfqpoint{0.998163in}{0.622573in}}{\pgfqpoint{1.008762in}{0.618182in}}{\pgfqpoint{1.019812in}{0.618182in}}%
\pgfpathclose%
\pgfusepath{stroke,fill}%
\end{pgfscope}%
\begin{pgfscope}%
\pgfpathrectangle{\pgfqpoint{0.375000in}{0.330000in}}{\pgfqpoint{2.325000in}{2.310000in}}%
\pgfusepath{clip}%
\pgfsetbuttcap%
\pgfsetroundjoin%
\definecolor{currentfill}{rgb}{0.000000,0.000000,0.000000}%
\pgfsetfillcolor{currentfill}%
\pgfsetlinewidth{1.003750pt}%
\definecolor{currentstroke}{rgb}{0.000000,0.000000,0.000000}%
\pgfsetstrokecolor{currentstroke}%
\pgfsetdash{}{0pt}%
\pgfpathmoveto{\pgfqpoint{1.019812in}{0.636387in}}%
\pgfpathcurveto{\pgfqpoint{1.030863in}{0.636387in}}{\pgfqpoint{1.041462in}{0.640777in}}{\pgfqpoint{1.049275in}{0.648590in}}%
\pgfpathcurveto{\pgfqpoint{1.057089in}{0.656404in}}{\pgfqpoint{1.061479in}{0.667003in}}{\pgfqpoint{1.061479in}{0.678053in}}%
\pgfpathcurveto{\pgfqpoint{1.061479in}{0.689103in}}{\pgfqpoint{1.057089in}{0.699702in}}{\pgfqpoint{1.049275in}{0.707516in}}%
\pgfpathcurveto{\pgfqpoint{1.041462in}{0.715330in}}{\pgfqpoint{1.030863in}{0.719720in}}{\pgfqpoint{1.019812in}{0.719720in}}%
\pgfpathcurveto{\pgfqpoint{1.008762in}{0.719720in}}{\pgfqpoint{0.998163in}{0.715330in}}{\pgfqpoint{0.990350in}{0.707516in}}%
\pgfpathcurveto{\pgfqpoint{0.982536in}{0.699702in}}{\pgfqpoint{0.978146in}{0.689103in}}{\pgfqpoint{0.978146in}{0.678053in}}%
\pgfpathcurveto{\pgfqpoint{0.978146in}{0.667003in}}{\pgfqpoint{0.982536in}{0.656404in}}{\pgfqpoint{0.990350in}{0.648590in}}%
\pgfpathcurveto{\pgfqpoint{0.998163in}{0.640777in}}{\pgfqpoint{1.008762in}{0.636387in}}{\pgfqpoint{1.019812in}{0.636387in}}%
\pgfpathclose%
\pgfusepath{stroke,fill}%
\end{pgfscope}%
\begin{pgfscope}%
\pgfpathrectangle{\pgfqpoint{0.375000in}{0.330000in}}{\pgfqpoint{2.325000in}{2.310000in}}%
\pgfusepath{clip}%
\pgfsetbuttcap%
\pgfsetroundjoin%
\definecolor{currentfill}{rgb}{0.000000,0.000000,0.000000}%
\pgfsetfillcolor{currentfill}%
\pgfsetlinewidth{1.003750pt}%
\definecolor{currentstroke}{rgb}{0.000000,0.000000,0.000000}%
\pgfsetstrokecolor{currentstroke}%
\pgfsetdash{}{0pt}%
\pgfpathmoveto{\pgfqpoint{1.019812in}{0.636387in}}%
\pgfpathcurveto{\pgfqpoint{1.030863in}{0.636387in}}{\pgfqpoint{1.041462in}{0.640777in}}{\pgfqpoint{1.049275in}{0.648590in}}%
\pgfpathcurveto{\pgfqpoint{1.057089in}{0.656404in}}{\pgfqpoint{1.061479in}{0.667003in}}{\pgfqpoint{1.061479in}{0.678053in}}%
\pgfpathcurveto{\pgfqpoint{1.061479in}{0.689103in}}{\pgfqpoint{1.057089in}{0.699702in}}{\pgfqpoint{1.049275in}{0.707516in}}%
\pgfpathcurveto{\pgfqpoint{1.041462in}{0.715330in}}{\pgfqpoint{1.030863in}{0.719720in}}{\pgfqpoint{1.019812in}{0.719720in}}%
\pgfpathcurveto{\pgfqpoint{1.008762in}{0.719720in}}{\pgfqpoint{0.998163in}{0.715330in}}{\pgfqpoint{0.990350in}{0.707516in}}%
\pgfpathcurveto{\pgfqpoint{0.982536in}{0.699702in}}{\pgfqpoint{0.978146in}{0.689103in}}{\pgfqpoint{0.978146in}{0.678053in}}%
\pgfpathcurveto{\pgfqpoint{0.978146in}{0.667003in}}{\pgfqpoint{0.982536in}{0.656404in}}{\pgfqpoint{0.990350in}{0.648590in}}%
\pgfpathcurveto{\pgfqpoint{0.998163in}{0.640777in}}{\pgfqpoint{1.008762in}{0.636387in}}{\pgfqpoint{1.019812in}{0.636387in}}%
\pgfpathclose%
\pgfusepath{stroke,fill}%
\end{pgfscope}%
\begin{pgfscope}%
\pgfpathrectangle{\pgfqpoint{0.375000in}{0.330000in}}{\pgfqpoint{2.325000in}{2.310000in}}%
\pgfusepath{clip}%
\pgfsetbuttcap%
\pgfsetroundjoin%
\definecolor{currentfill}{rgb}{0.000000,0.000000,0.000000}%
\pgfsetfillcolor{currentfill}%
\pgfsetlinewidth{1.003750pt}%
\definecolor{currentstroke}{rgb}{0.000000,0.000000,0.000000}%
\pgfsetstrokecolor{currentstroke}%
\pgfsetdash{}{0pt}%
\pgfpathmoveto{\pgfqpoint{1.019812in}{0.697068in}}%
\pgfpathcurveto{\pgfqpoint{1.030863in}{0.697068in}}{\pgfqpoint{1.041462in}{0.701458in}}{\pgfqpoint{1.049275in}{0.709272in}}%
\pgfpathcurveto{\pgfqpoint{1.057089in}{0.717085in}}{\pgfqpoint{1.061479in}{0.727684in}}{\pgfqpoint{1.061479in}{0.738734in}}%
\pgfpathcurveto{\pgfqpoint{1.061479in}{0.749784in}}{\pgfqpoint{1.057089in}{0.760383in}}{\pgfqpoint{1.049275in}{0.768197in}}%
\pgfpathcurveto{\pgfqpoint{1.041462in}{0.776011in}}{\pgfqpoint{1.030863in}{0.780401in}}{\pgfqpoint{1.019812in}{0.780401in}}%
\pgfpathcurveto{\pgfqpoint{1.008762in}{0.780401in}}{\pgfqpoint{0.998163in}{0.776011in}}{\pgfqpoint{0.990350in}{0.768197in}}%
\pgfpathcurveto{\pgfqpoint{0.982536in}{0.760383in}}{\pgfqpoint{0.978146in}{0.749784in}}{\pgfqpoint{0.978146in}{0.738734in}}%
\pgfpathcurveto{\pgfqpoint{0.978146in}{0.727684in}}{\pgfqpoint{0.982536in}{0.717085in}}{\pgfqpoint{0.990350in}{0.709272in}}%
\pgfpathcurveto{\pgfqpoint{0.998163in}{0.701458in}}{\pgfqpoint{1.008762in}{0.697068in}}{\pgfqpoint{1.019812in}{0.697068in}}%
\pgfpathclose%
\pgfusepath{stroke,fill}%
\end{pgfscope}%
\begin{pgfscope}%
\pgfpathrectangle{\pgfqpoint{0.375000in}{0.330000in}}{\pgfqpoint{2.325000in}{2.310000in}}%
\pgfusepath{clip}%
\pgfsetbuttcap%
\pgfsetroundjoin%
\definecolor{currentfill}{rgb}{0.000000,0.000000,0.000000}%
\pgfsetfillcolor{currentfill}%
\pgfsetlinewidth{1.003750pt}%
\definecolor{currentstroke}{rgb}{0.000000,0.000000,0.000000}%
\pgfsetstrokecolor{currentstroke}%
\pgfsetdash{}{0pt}%
\pgfpathmoveto{\pgfqpoint{1.019812in}{0.618182in}}%
\pgfpathcurveto{\pgfqpoint{1.030863in}{0.618182in}}{\pgfqpoint{1.041462in}{0.622573in}}{\pgfqpoint{1.049275in}{0.630386in}}%
\pgfpathcurveto{\pgfqpoint{1.057089in}{0.638200in}}{\pgfqpoint{1.061479in}{0.648799in}}{\pgfqpoint{1.061479in}{0.659849in}}%
\pgfpathcurveto{\pgfqpoint{1.061479in}{0.670899in}}{\pgfqpoint{1.057089in}{0.681498in}}{\pgfqpoint{1.049275in}{0.689312in}}%
\pgfpathcurveto{\pgfqpoint{1.041462in}{0.697125in}}{\pgfqpoint{1.030863in}{0.701516in}}{\pgfqpoint{1.019812in}{0.701516in}}%
\pgfpathcurveto{\pgfqpoint{1.008762in}{0.701516in}}{\pgfqpoint{0.998163in}{0.697125in}}{\pgfqpoint{0.990350in}{0.689312in}}%
\pgfpathcurveto{\pgfqpoint{0.982536in}{0.681498in}}{\pgfqpoint{0.978146in}{0.670899in}}{\pgfqpoint{0.978146in}{0.659849in}}%
\pgfpathcurveto{\pgfqpoint{0.978146in}{0.648799in}}{\pgfqpoint{0.982536in}{0.638200in}}{\pgfqpoint{0.990350in}{0.630386in}}%
\pgfpathcurveto{\pgfqpoint{0.998163in}{0.622573in}}{\pgfqpoint{1.008762in}{0.618182in}}{\pgfqpoint{1.019812in}{0.618182in}}%
\pgfpathclose%
\pgfusepath{stroke,fill}%
\end{pgfscope}%
\begin{pgfscope}%
\pgfpathrectangle{\pgfqpoint{0.375000in}{0.330000in}}{\pgfqpoint{2.325000in}{2.310000in}}%
\pgfusepath{clip}%
\pgfsetbuttcap%
\pgfsetroundjoin%
\definecolor{currentfill}{rgb}{0.000000,0.000000,0.000000}%
\pgfsetfillcolor{currentfill}%
\pgfsetlinewidth{1.003750pt}%
\definecolor{currentstroke}{rgb}{0.000000,0.000000,0.000000}%
\pgfsetstrokecolor{currentstroke}%
\pgfsetdash{}{0pt}%
\pgfpathmoveto{\pgfqpoint{1.019812in}{0.587842in}}%
\pgfpathcurveto{\pgfqpoint{1.030863in}{0.587842in}}{\pgfqpoint{1.041462in}{0.592232in}}{\pgfqpoint{1.049275in}{0.600046in}}%
\pgfpathcurveto{\pgfqpoint{1.057089in}{0.607859in}}{\pgfqpoint{1.061479in}{0.618458in}}{\pgfqpoint{1.061479in}{0.629508in}}%
\pgfpathcurveto{\pgfqpoint{1.061479in}{0.640559in}}{\pgfqpoint{1.057089in}{0.651158in}}{\pgfqpoint{1.049275in}{0.658971in}}%
\pgfpathcurveto{\pgfqpoint{1.041462in}{0.666785in}}{\pgfqpoint{1.030863in}{0.671175in}}{\pgfqpoint{1.019812in}{0.671175in}}%
\pgfpathcurveto{\pgfqpoint{1.008762in}{0.671175in}}{\pgfqpoint{0.998163in}{0.666785in}}{\pgfqpoint{0.990350in}{0.658971in}}%
\pgfpathcurveto{\pgfqpoint{0.982536in}{0.651158in}}{\pgfqpoint{0.978146in}{0.640559in}}{\pgfqpoint{0.978146in}{0.629508in}}%
\pgfpathcurveto{\pgfqpoint{0.978146in}{0.618458in}}{\pgfqpoint{0.982536in}{0.607859in}}{\pgfqpoint{0.990350in}{0.600046in}}%
\pgfpathcurveto{\pgfqpoint{0.998163in}{0.592232in}}{\pgfqpoint{1.008762in}{0.587842in}}{\pgfqpoint{1.019812in}{0.587842in}}%
\pgfpathclose%
\pgfusepath{stroke,fill}%
\end{pgfscope}%
\begin{pgfscope}%
\pgfpathrectangle{\pgfqpoint{0.375000in}{0.330000in}}{\pgfqpoint{2.325000in}{2.310000in}}%
\pgfusepath{clip}%
\pgfsetbuttcap%
\pgfsetroundjoin%
\definecolor{currentfill}{rgb}{0.000000,0.000000,0.000000}%
\pgfsetfillcolor{currentfill}%
\pgfsetlinewidth{1.003750pt}%
\definecolor{currentstroke}{rgb}{0.000000,0.000000,0.000000}%
\pgfsetstrokecolor{currentstroke}%
\pgfsetdash{}{0pt}%
\pgfpathmoveto{\pgfqpoint{1.019812in}{0.593910in}}%
\pgfpathcurveto{\pgfqpoint{1.030863in}{0.593910in}}{\pgfqpoint{1.041462in}{0.598300in}}{\pgfqpoint{1.049275in}{0.606114in}}%
\pgfpathcurveto{\pgfqpoint{1.057089in}{0.613927in}}{\pgfqpoint{1.061479in}{0.624526in}}{\pgfqpoint{1.061479in}{0.635576in}}%
\pgfpathcurveto{\pgfqpoint{1.061479in}{0.646627in}}{\pgfqpoint{1.057089in}{0.657226in}}{\pgfqpoint{1.049275in}{0.665039in}}%
\pgfpathcurveto{\pgfqpoint{1.041462in}{0.672853in}}{\pgfqpoint{1.030863in}{0.677243in}}{\pgfqpoint{1.019812in}{0.677243in}}%
\pgfpathcurveto{\pgfqpoint{1.008762in}{0.677243in}}{\pgfqpoint{0.998163in}{0.672853in}}{\pgfqpoint{0.990350in}{0.665039in}}%
\pgfpathcurveto{\pgfqpoint{0.982536in}{0.657226in}}{\pgfqpoint{0.978146in}{0.646627in}}{\pgfqpoint{0.978146in}{0.635576in}}%
\pgfpathcurveto{\pgfqpoint{0.978146in}{0.624526in}}{\pgfqpoint{0.982536in}{0.613927in}}{\pgfqpoint{0.990350in}{0.606114in}}%
\pgfpathcurveto{\pgfqpoint{0.998163in}{0.598300in}}{\pgfqpoint{1.008762in}{0.593910in}}{\pgfqpoint{1.019812in}{0.593910in}}%
\pgfpathclose%
\pgfusepath{stroke,fill}%
\end{pgfscope}%
\begin{pgfscope}%
\pgfpathrectangle{\pgfqpoint{0.375000in}{0.330000in}}{\pgfqpoint{2.325000in}{2.310000in}}%
\pgfusepath{clip}%
\pgfsetbuttcap%
\pgfsetroundjoin%
\definecolor{currentfill}{rgb}{0.000000,0.000000,0.000000}%
\pgfsetfillcolor{currentfill}%
\pgfsetlinewidth{1.003750pt}%
\definecolor{currentstroke}{rgb}{0.000000,0.000000,0.000000}%
\pgfsetstrokecolor{currentstroke}%
\pgfsetdash{}{0pt}%
\pgfpathmoveto{\pgfqpoint{1.019812in}{0.648523in}}%
\pgfpathcurveto{\pgfqpoint{1.030863in}{0.648523in}}{\pgfqpoint{1.041462in}{0.652913in}}{\pgfqpoint{1.049275in}{0.660727in}}%
\pgfpathcurveto{\pgfqpoint{1.057089in}{0.668540in}}{\pgfqpoint{1.061479in}{0.679139in}}{\pgfqpoint{1.061479in}{0.690189in}}%
\pgfpathcurveto{\pgfqpoint{1.061479in}{0.701240in}}{\pgfqpoint{1.057089in}{0.711839in}}{\pgfqpoint{1.049275in}{0.719652in}}%
\pgfpathcurveto{\pgfqpoint{1.041462in}{0.727466in}}{\pgfqpoint{1.030863in}{0.731856in}}{\pgfqpoint{1.019812in}{0.731856in}}%
\pgfpathcurveto{\pgfqpoint{1.008762in}{0.731856in}}{\pgfqpoint{0.998163in}{0.727466in}}{\pgfqpoint{0.990350in}{0.719652in}}%
\pgfpathcurveto{\pgfqpoint{0.982536in}{0.711839in}}{\pgfqpoint{0.978146in}{0.701240in}}{\pgfqpoint{0.978146in}{0.690189in}}%
\pgfpathcurveto{\pgfqpoint{0.978146in}{0.679139in}}{\pgfqpoint{0.982536in}{0.668540in}}{\pgfqpoint{0.990350in}{0.660727in}}%
\pgfpathcurveto{\pgfqpoint{0.998163in}{0.652913in}}{\pgfqpoint{1.008762in}{0.648523in}}{\pgfqpoint{1.019812in}{0.648523in}}%
\pgfpathclose%
\pgfusepath{stroke,fill}%
\end{pgfscope}%
\begin{pgfscope}%
\pgfpathrectangle{\pgfqpoint{0.375000in}{0.330000in}}{\pgfqpoint{2.325000in}{2.310000in}}%
\pgfusepath{clip}%
\pgfsetbuttcap%
\pgfsetroundjoin%
\definecolor{currentfill}{rgb}{0.000000,0.000000,0.000000}%
\pgfsetfillcolor{currentfill}%
\pgfsetlinewidth{1.003750pt}%
\definecolor{currentstroke}{rgb}{0.000000,0.000000,0.000000}%
\pgfsetstrokecolor{currentstroke}%
\pgfsetdash{}{0pt}%
\pgfpathmoveto{\pgfqpoint{1.019812in}{0.606046in}}%
\pgfpathcurveto{\pgfqpoint{1.030863in}{0.606046in}}{\pgfqpoint{1.041462in}{0.610436in}}{\pgfqpoint{1.049275in}{0.618250in}}%
\pgfpathcurveto{\pgfqpoint{1.057089in}{0.626064in}}{\pgfqpoint{1.061479in}{0.636663in}}{\pgfqpoint{1.061479in}{0.647713in}}%
\pgfpathcurveto{\pgfqpoint{1.061479in}{0.658763in}}{\pgfqpoint{1.057089in}{0.669362in}}{\pgfqpoint{1.049275in}{0.677175in}}%
\pgfpathcurveto{\pgfqpoint{1.041462in}{0.684989in}}{\pgfqpoint{1.030863in}{0.689379in}}{\pgfqpoint{1.019812in}{0.689379in}}%
\pgfpathcurveto{\pgfqpoint{1.008762in}{0.689379in}}{\pgfqpoint{0.998163in}{0.684989in}}{\pgfqpoint{0.990350in}{0.677175in}}%
\pgfpathcurveto{\pgfqpoint{0.982536in}{0.669362in}}{\pgfqpoint{0.978146in}{0.658763in}}{\pgfqpoint{0.978146in}{0.647713in}}%
\pgfpathcurveto{\pgfqpoint{0.978146in}{0.636663in}}{\pgfqpoint{0.982536in}{0.626064in}}{\pgfqpoint{0.990350in}{0.618250in}}%
\pgfpathcurveto{\pgfqpoint{0.998163in}{0.610436in}}{\pgfqpoint{1.008762in}{0.606046in}}{\pgfqpoint{1.019812in}{0.606046in}}%
\pgfpathclose%
\pgfusepath{stroke,fill}%
\end{pgfscope}%
\begin{pgfscope}%
\pgfpathrectangle{\pgfqpoint{0.375000in}{0.330000in}}{\pgfqpoint{2.325000in}{2.310000in}}%
\pgfusepath{clip}%
\pgfsetbuttcap%
\pgfsetroundjoin%
\definecolor{currentfill}{rgb}{0.000000,0.000000,0.000000}%
\pgfsetfillcolor{currentfill}%
\pgfsetlinewidth{1.003750pt}%
\definecolor{currentstroke}{rgb}{0.000000,0.000000,0.000000}%
\pgfsetstrokecolor{currentstroke}%
\pgfsetdash{}{0pt}%
\pgfpathmoveto{\pgfqpoint{1.019812in}{0.642455in}}%
\pgfpathcurveto{\pgfqpoint{1.030863in}{0.642455in}}{\pgfqpoint{1.041462in}{0.646845in}}{\pgfqpoint{1.049275in}{0.654659in}}%
\pgfpathcurveto{\pgfqpoint{1.057089in}{0.662472in}}{\pgfqpoint{1.061479in}{0.673071in}}{\pgfqpoint{1.061479in}{0.684121in}}%
\pgfpathcurveto{\pgfqpoint{1.061479in}{0.695171in}}{\pgfqpoint{1.057089in}{0.705770in}}{\pgfqpoint{1.049275in}{0.713584in}}%
\pgfpathcurveto{\pgfqpoint{1.041462in}{0.721398in}}{\pgfqpoint{1.030863in}{0.725788in}}{\pgfqpoint{1.019812in}{0.725788in}}%
\pgfpathcurveto{\pgfqpoint{1.008762in}{0.725788in}}{\pgfqpoint{0.998163in}{0.721398in}}{\pgfqpoint{0.990350in}{0.713584in}}%
\pgfpathcurveto{\pgfqpoint{0.982536in}{0.705770in}}{\pgfqpoint{0.978146in}{0.695171in}}{\pgfqpoint{0.978146in}{0.684121in}}%
\pgfpathcurveto{\pgfqpoint{0.978146in}{0.673071in}}{\pgfqpoint{0.982536in}{0.662472in}}{\pgfqpoint{0.990350in}{0.654659in}}%
\pgfpathcurveto{\pgfqpoint{0.998163in}{0.646845in}}{\pgfqpoint{1.008762in}{0.642455in}}{\pgfqpoint{1.019812in}{0.642455in}}%
\pgfpathclose%
\pgfusepath{stroke,fill}%
\end{pgfscope}%
\begin{pgfscope}%
\pgfpathrectangle{\pgfqpoint{0.375000in}{0.330000in}}{\pgfqpoint{2.325000in}{2.310000in}}%
\pgfusepath{clip}%
\pgfsetbuttcap%
\pgfsetroundjoin%
\definecolor{currentfill}{rgb}{0.000000,0.000000,0.000000}%
\pgfsetfillcolor{currentfill}%
\pgfsetlinewidth{1.003750pt}%
\definecolor{currentstroke}{rgb}{0.000000,0.000000,0.000000}%
\pgfsetstrokecolor{currentstroke}%
\pgfsetdash{}{0pt}%
\pgfpathmoveto{\pgfqpoint{1.019812in}{0.581774in}}%
\pgfpathcurveto{\pgfqpoint{1.030863in}{0.581774in}}{\pgfqpoint{1.041462in}{0.586164in}}{\pgfqpoint{1.049275in}{0.593977in}}%
\pgfpathcurveto{\pgfqpoint{1.057089in}{0.601791in}}{\pgfqpoint{1.061479in}{0.612390in}}{\pgfqpoint{1.061479in}{0.623440in}}%
\pgfpathcurveto{\pgfqpoint{1.061479in}{0.634490in}}{\pgfqpoint{1.057089in}{0.645089in}}{\pgfqpoint{1.049275in}{0.652903in}}%
\pgfpathcurveto{\pgfqpoint{1.041462in}{0.660717in}}{\pgfqpoint{1.030863in}{0.665107in}}{\pgfqpoint{1.019812in}{0.665107in}}%
\pgfpathcurveto{\pgfqpoint{1.008762in}{0.665107in}}{\pgfqpoint{0.998163in}{0.660717in}}{\pgfqpoint{0.990350in}{0.652903in}}%
\pgfpathcurveto{\pgfqpoint{0.982536in}{0.645089in}}{\pgfqpoint{0.978146in}{0.634490in}}{\pgfqpoint{0.978146in}{0.623440in}}%
\pgfpathcurveto{\pgfqpoint{0.978146in}{0.612390in}}{\pgfqpoint{0.982536in}{0.601791in}}{\pgfqpoint{0.990350in}{0.593977in}}%
\pgfpathcurveto{\pgfqpoint{0.998163in}{0.586164in}}{\pgfqpoint{1.008762in}{0.581774in}}{\pgfqpoint{1.019812in}{0.581774in}}%
\pgfpathclose%
\pgfusepath{stroke,fill}%
\end{pgfscope}%
\begin{pgfscope}%
\pgfpathrectangle{\pgfqpoint{0.375000in}{0.330000in}}{\pgfqpoint{2.325000in}{2.310000in}}%
\pgfusepath{clip}%
\pgfsetbuttcap%
\pgfsetroundjoin%
\definecolor{currentfill}{rgb}{0.000000,0.000000,0.000000}%
\pgfsetfillcolor{currentfill}%
\pgfsetlinewidth{1.003750pt}%
\definecolor{currentstroke}{rgb}{0.000000,0.000000,0.000000}%
\pgfsetstrokecolor{currentstroke}%
\pgfsetdash{}{0pt}%
\pgfpathmoveto{\pgfqpoint{1.019812in}{0.660659in}}%
\pgfpathcurveto{\pgfqpoint{1.030863in}{0.660659in}}{\pgfqpoint{1.041462in}{0.665049in}}{\pgfqpoint{1.049275in}{0.672863in}}%
\pgfpathcurveto{\pgfqpoint{1.057089in}{0.680676in}}{\pgfqpoint{1.061479in}{0.691276in}}{\pgfqpoint{1.061479in}{0.702326in}}%
\pgfpathcurveto{\pgfqpoint{1.061479in}{0.713376in}}{\pgfqpoint{1.057089in}{0.723975in}}{\pgfqpoint{1.049275in}{0.731788in}}%
\pgfpathcurveto{\pgfqpoint{1.041462in}{0.739602in}}{\pgfqpoint{1.030863in}{0.743992in}}{\pgfqpoint{1.019812in}{0.743992in}}%
\pgfpathcurveto{\pgfqpoint{1.008762in}{0.743992in}}{\pgfqpoint{0.998163in}{0.739602in}}{\pgfqpoint{0.990350in}{0.731788in}}%
\pgfpathcurveto{\pgfqpoint{0.982536in}{0.723975in}}{\pgfqpoint{0.978146in}{0.713376in}}{\pgfqpoint{0.978146in}{0.702326in}}%
\pgfpathcurveto{\pgfqpoint{0.978146in}{0.691276in}}{\pgfqpoint{0.982536in}{0.680676in}}{\pgfqpoint{0.990350in}{0.672863in}}%
\pgfpathcurveto{\pgfqpoint{0.998163in}{0.665049in}}{\pgfqpoint{1.008762in}{0.660659in}}{\pgfqpoint{1.019812in}{0.660659in}}%
\pgfpathclose%
\pgfusepath{stroke,fill}%
\end{pgfscope}%
\begin{pgfscope}%
\pgfpathrectangle{\pgfqpoint{0.375000in}{0.330000in}}{\pgfqpoint{2.325000in}{2.310000in}}%
\pgfusepath{clip}%
\pgfsetbuttcap%
\pgfsetroundjoin%
\definecolor{currentfill}{rgb}{0.000000,0.000000,0.000000}%
\pgfsetfillcolor{currentfill}%
\pgfsetlinewidth{1.003750pt}%
\definecolor{currentstroke}{rgb}{0.000000,0.000000,0.000000}%
\pgfsetstrokecolor{currentstroke}%
\pgfsetdash{}{0pt}%
\pgfpathmoveto{\pgfqpoint{1.019812in}{0.636387in}}%
\pgfpathcurveto{\pgfqpoint{1.030863in}{0.636387in}}{\pgfqpoint{1.041462in}{0.640777in}}{\pgfqpoint{1.049275in}{0.648590in}}%
\pgfpathcurveto{\pgfqpoint{1.057089in}{0.656404in}}{\pgfqpoint{1.061479in}{0.667003in}}{\pgfqpoint{1.061479in}{0.678053in}}%
\pgfpathcurveto{\pgfqpoint{1.061479in}{0.689103in}}{\pgfqpoint{1.057089in}{0.699702in}}{\pgfqpoint{1.049275in}{0.707516in}}%
\pgfpathcurveto{\pgfqpoint{1.041462in}{0.715330in}}{\pgfqpoint{1.030863in}{0.719720in}}{\pgfqpoint{1.019812in}{0.719720in}}%
\pgfpathcurveto{\pgfqpoint{1.008762in}{0.719720in}}{\pgfqpoint{0.998163in}{0.715330in}}{\pgfqpoint{0.990350in}{0.707516in}}%
\pgfpathcurveto{\pgfqpoint{0.982536in}{0.699702in}}{\pgfqpoint{0.978146in}{0.689103in}}{\pgfqpoint{0.978146in}{0.678053in}}%
\pgfpathcurveto{\pgfqpoint{0.978146in}{0.667003in}}{\pgfqpoint{0.982536in}{0.656404in}}{\pgfqpoint{0.990350in}{0.648590in}}%
\pgfpathcurveto{\pgfqpoint{0.998163in}{0.640777in}}{\pgfqpoint{1.008762in}{0.636387in}}{\pgfqpoint{1.019812in}{0.636387in}}%
\pgfpathclose%
\pgfusepath{stroke,fill}%
\end{pgfscope}%
\begin{pgfscope}%
\pgfpathrectangle{\pgfqpoint{0.375000in}{0.330000in}}{\pgfqpoint{2.325000in}{2.310000in}}%
\pgfusepath{clip}%
\pgfsetbuttcap%
\pgfsetroundjoin%
\definecolor{currentfill}{rgb}{0.000000,0.000000,0.000000}%
\pgfsetfillcolor{currentfill}%
\pgfsetlinewidth{1.003750pt}%
\definecolor{currentstroke}{rgb}{0.000000,0.000000,0.000000}%
\pgfsetstrokecolor{currentstroke}%
\pgfsetdash{}{0pt}%
\pgfpathmoveto{\pgfqpoint{1.019812in}{0.636387in}}%
\pgfpathcurveto{\pgfqpoint{1.030863in}{0.636387in}}{\pgfqpoint{1.041462in}{0.640777in}}{\pgfqpoint{1.049275in}{0.648590in}}%
\pgfpathcurveto{\pgfqpoint{1.057089in}{0.656404in}}{\pgfqpoint{1.061479in}{0.667003in}}{\pgfqpoint{1.061479in}{0.678053in}}%
\pgfpathcurveto{\pgfqpoint{1.061479in}{0.689103in}}{\pgfqpoint{1.057089in}{0.699702in}}{\pgfqpoint{1.049275in}{0.707516in}}%
\pgfpathcurveto{\pgfqpoint{1.041462in}{0.715330in}}{\pgfqpoint{1.030863in}{0.719720in}}{\pgfqpoint{1.019812in}{0.719720in}}%
\pgfpathcurveto{\pgfqpoint{1.008762in}{0.719720in}}{\pgfqpoint{0.998163in}{0.715330in}}{\pgfqpoint{0.990350in}{0.707516in}}%
\pgfpathcurveto{\pgfqpoint{0.982536in}{0.699702in}}{\pgfqpoint{0.978146in}{0.689103in}}{\pgfqpoint{0.978146in}{0.678053in}}%
\pgfpathcurveto{\pgfqpoint{0.978146in}{0.667003in}}{\pgfqpoint{0.982536in}{0.656404in}}{\pgfqpoint{0.990350in}{0.648590in}}%
\pgfpathcurveto{\pgfqpoint{0.998163in}{0.640777in}}{\pgfqpoint{1.008762in}{0.636387in}}{\pgfqpoint{1.019812in}{0.636387in}}%
\pgfpathclose%
\pgfusepath{stroke,fill}%
\end{pgfscope}%
\begin{pgfscope}%
\pgfpathrectangle{\pgfqpoint{0.375000in}{0.330000in}}{\pgfqpoint{2.325000in}{2.310000in}}%
\pgfusepath{clip}%
\pgfsetbuttcap%
\pgfsetroundjoin%
\definecolor{currentfill}{rgb}{0.000000,0.000000,0.000000}%
\pgfsetfillcolor{currentfill}%
\pgfsetlinewidth{1.003750pt}%
\definecolor{currentstroke}{rgb}{0.000000,0.000000,0.000000}%
\pgfsetstrokecolor{currentstroke}%
\pgfsetdash{}{0pt}%
\pgfpathmoveto{\pgfqpoint{1.019812in}{0.606046in}}%
\pgfpathcurveto{\pgfqpoint{1.030863in}{0.606046in}}{\pgfqpoint{1.041462in}{0.610436in}}{\pgfqpoint{1.049275in}{0.618250in}}%
\pgfpathcurveto{\pgfqpoint{1.057089in}{0.626064in}}{\pgfqpoint{1.061479in}{0.636663in}}{\pgfqpoint{1.061479in}{0.647713in}}%
\pgfpathcurveto{\pgfqpoint{1.061479in}{0.658763in}}{\pgfqpoint{1.057089in}{0.669362in}}{\pgfqpoint{1.049275in}{0.677175in}}%
\pgfpathcurveto{\pgfqpoint{1.041462in}{0.684989in}}{\pgfqpoint{1.030863in}{0.689379in}}{\pgfqpoint{1.019812in}{0.689379in}}%
\pgfpathcurveto{\pgfqpoint{1.008762in}{0.689379in}}{\pgfqpoint{0.998163in}{0.684989in}}{\pgfqpoint{0.990350in}{0.677175in}}%
\pgfpathcurveto{\pgfqpoint{0.982536in}{0.669362in}}{\pgfqpoint{0.978146in}{0.658763in}}{\pgfqpoint{0.978146in}{0.647713in}}%
\pgfpathcurveto{\pgfqpoint{0.978146in}{0.636663in}}{\pgfqpoint{0.982536in}{0.626064in}}{\pgfqpoint{0.990350in}{0.618250in}}%
\pgfpathcurveto{\pgfqpoint{0.998163in}{0.610436in}}{\pgfqpoint{1.008762in}{0.606046in}}{\pgfqpoint{1.019812in}{0.606046in}}%
\pgfpathclose%
\pgfusepath{stroke,fill}%
\end{pgfscope}%
\begin{pgfscope}%
\pgfpathrectangle{\pgfqpoint{0.375000in}{0.330000in}}{\pgfqpoint{2.325000in}{2.310000in}}%
\pgfusepath{clip}%
\pgfsetbuttcap%
\pgfsetroundjoin%
\definecolor{currentfill}{rgb}{0.000000,0.000000,0.000000}%
\pgfsetfillcolor{currentfill}%
\pgfsetlinewidth{1.003750pt}%
\definecolor{currentstroke}{rgb}{0.000000,0.000000,0.000000}%
\pgfsetstrokecolor{currentstroke}%
\pgfsetdash{}{0pt}%
\pgfpathmoveto{\pgfqpoint{1.019812in}{0.672795in}}%
\pgfpathcurveto{\pgfqpoint{1.030863in}{0.672795in}}{\pgfqpoint{1.041462in}{0.677185in}}{\pgfqpoint{1.049275in}{0.684999in}}%
\pgfpathcurveto{\pgfqpoint{1.057089in}{0.692813in}}{\pgfqpoint{1.061479in}{0.703412in}}{\pgfqpoint{1.061479in}{0.714462in}}%
\pgfpathcurveto{\pgfqpoint{1.061479in}{0.725512in}}{\pgfqpoint{1.057089in}{0.736111in}}{\pgfqpoint{1.049275in}{0.743925in}}%
\pgfpathcurveto{\pgfqpoint{1.041462in}{0.751738in}}{\pgfqpoint{1.030863in}{0.756129in}}{\pgfqpoint{1.019812in}{0.756129in}}%
\pgfpathcurveto{\pgfqpoint{1.008762in}{0.756129in}}{\pgfqpoint{0.998163in}{0.751738in}}{\pgfqpoint{0.990350in}{0.743925in}}%
\pgfpathcurveto{\pgfqpoint{0.982536in}{0.736111in}}{\pgfqpoint{0.978146in}{0.725512in}}{\pgfqpoint{0.978146in}{0.714462in}}%
\pgfpathcurveto{\pgfqpoint{0.978146in}{0.703412in}}{\pgfqpoint{0.982536in}{0.692813in}}{\pgfqpoint{0.990350in}{0.684999in}}%
\pgfpathcurveto{\pgfqpoint{0.998163in}{0.677185in}}{\pgfqpoint{1.008762in}{0.672795in}}{\pgfqpoint{1.019812in}{0.672795in}}%
\pgfpathclose%
\pgfusepath{stroke,fill}%
\end{pgfscope}%
\begin{pgfscope}%
\pgfpathrectangle{\pgfqpoint{0.375000in}{0.330000in}}{\pgfqpoint{2.325000in}{2.310000in}}%
\pgfusepath{clip}%
\pgfsetbuttcap%
\pgfsetroundjoin%
\definecolor{currentfill}{rgb}{0.000000,0.000000,0.000000}%
\pgfsetfillcolor{currentfill}%
\pgfsetlinewidth{1.003750pt}%
\definecolor{currentstroke}{rgb}{0.000000,0.000000,0.000000}%
\pgfsetstrokecolor{currentstroke}%
\pgfsetdash{}{0pt}%
\pgfpathmoveto{\pgfqpoint{1.019812in}{0.593910in}}%
\pgfpathcurveto{\pgfqpoint{1.030863in}{0.593910in}}{\pgfqpoint{1.041462in}{0.598300in}}{\pgfqpoint{1.049275in}{0.606114in}}%
\pgfpathcurveto{\pgfqpoint{1.057089in}{0.613927in}}{\pgfqpoint{1.061479in}{0.624526in}}{\pgfqpoint{1.061479in}{0.635576in}}%
\pgfpathcurveto{\pgfqpoint{1.061479in}{0.646627in}}{\pgfqpoint{1.057089in}{0.657226in}}{\pgfqpoint{1.049275in}{0.665039in}}%
\pgfpathcurveto{\pgfqpoint{1.041462in}{0.672853in}}{\pgfqpoint{1.030863in}{0.677243in}}{\pgfqpoint{1.019812in}{0.677243in}}%
\pgfpathcurveto{\pgfqpoint{1.008762in}{0.677243in}}{\pgfqpoint{0.998163in}{0.672853in}}{\pgfqpoint{0.990350in}{0.665039in}}%
\pgfpathcurveto{\pgfqpoint{0.982536in}{0.657226in}}{\pgfqpoint{0.978146in}{0.646627in}}{\pgfqpoint{0.978146in}{0.635576in}}%
\pgfpathcurveto{\pgfqpoint{0.978146in}{0.624526in}}{\pgfqpoint{0.982536in}{0.613927in}}{\pgfqpoint{0.990350in}{0.606114in}}%
\pgfpathcurveto{\pgfqpoint{0.998163in}{0.598300in}}{\pgfqpoint{1.008762in}{0.593910in}}{\pgfqpoint{1.019812in}{0.593910in}}%
\pgfpathclose%
\pgfusepath{stroke,fill}%
\end{pgfscope}%
\begin{pgfscope}%
\pgfpathrectangle{\pgfqpoint{0.375000in}{0.330000in}}{\pgfqpoint{2.325000in}{2.310000in}}%
\pgfusepath{clip}%
\pgfsetbuttcap%
\pgfsetroundjoin%
\definecolor{currentfill}{rgb}{0.000000,0.000000,0.000000}%
\pgfsetfillcolor{currentfill}%
\pgfsetlinewidth{1.003750pt}%
\definecolor{currentstroke}{rgb}{0.000000,0.000000,0.000000}%
\pgfsetstrokecolor{currentstroke}%
\pgfsetdash{}{0pt}%
\pgfpathmoveto{\pgfqpoint{1.019812in}{0.599978in}}%
\pgfpathcurveto{\pgfqpoint{1.030863in}{0.599978in}}{\pgfqpoint{1.041462in}{0.604368in}}{\pgfqpoint{1.049275in}{0.612182in}}%
\pgfpathcurveto{\pgfqpoint{1.057089in}{0.619995in}}{\pgfqpoint{1.061479in}{0.630594in}}{\pgfqpoint{1.061479in}{0.641645in}}%
\pgfpathcurveto{\pgfqpoint{1.061479in}{0.652695in}}{\pgfqpoint{1.057089in}{0.663294in}}{\pgfqpoint{1.049275in}{0.671107in}}%
\pgfpathcurveto{\pgfqpoint{1.041462in}{0.678921in}}{\pgfqpoint{1.030863in}{0.683311in}}{\pgfqpoint{1.019812in}{0.683311in}}%
\pgfpathcurveto{\pgfqpoint{1.008762in}{0.683311in}}{\pgfqpoint{0.998163in}{0.678921in}}{\pgfqpoint{0.990350in}{0.671107in}}%
\pgfpathcurveto{\pgfqpoint{0.982536in}{0.663294in}}{\pgfqpoint{0.978146in}{0.652695in}}{\pgfqpoint{0.978146in}{0.641645in}}%
\pgfpathcurveto{\pgfqpoint{0.978146in}{0.630594in}}{\pgfqpoint{0.982536in}{0.619995in}}{\pgfqpoint{0.990350in}{0.612182in}}%
\pgfpathcurveto{\pgfqpoint{0.998163in}{0.604368in}}{\pgfqpoint{1.008762in}{0.599978in}}{\pgfqpoint{1.019812in}{0.599978in}}%
\pgfpathclose%
\pgfusepath{stroke,fill}%
\end{pgfscope}%
\begin{pgfscope}%
\pgfpathrectangle{\pgfqpoint{0.375000in}{0.330000in}}{\pgfqpoint{2.325000in}{2.310000in}}%
\pgfusepath{clip}%
\pgfsetbuttcap%
\pgfsetroundjoin%
\definecolor{currentfill}{rgb}{0.000000,0.000000,0.000000}%
\pgfsetfillcolor{currentfill}%
\pgfsetlinewidth{1.003750pt}%
\definecolor{currentstroke}{rgb}{0.000000,0.000000,0.000000}%
\pgfsetstrokecolor{currentstroke}%
\pgfsetdash{}{0pt}%
\pgfpathmoveto{\pgfqpoint{1.019812in}{0.624250in}}%
\pgfpathcurveto{\pgfqpoint{1.030863in}{0.624250in}}{\pgfqpoint{1.041462in}{0.628641in}}{\pgfqpoint{1.049275in}{0.636454in}}%
\pgfpathcurveto{\pgfqpoint{1.057089in}{0.644268in}}{\pgfqpoint{1.061479in}{0.654867in}}{\pgfqpoint{1.061479in}{0.665917in}}%
\pgfpathcurveto{\pgfqpoint{1.061479in}{0.676967in}}{\pgfqpoint{1.057089in}{0.687566in}}{\pgfqpoint{1.049275in}{0.695380in}}%
\pgfpathcurveto{\pgfqpoint{1.041462in}{0.703193in}}{\pgfqpoint{1.030863in}{0.707584in}}{\pgfqpoint{1.019812in}{0.707584in}}%
\pgfpathcurveto{\pgfqpoint{1.008762in}{0.707584in}}{\pgfqpoint{0.998163in}{0.703193in}}{\pgfqpoint{0.990350in}{0.695380in}}%
\pgfpathcurveto{\pgfqpoint{0.982536in}{0.687566in}}{\pgfqpoint{0.978146in}{0.676967in}}{\pgfqpoint{0.978146in}{0.665917in}}%
\pgfpathcurveto{\pgfqpoint{0.978146in}{0.654867in}}{\pgfqpoint{0.982536in}{0.644268in}}{\pgfqpoint{0.990350in}{0.636454in}}%
\pgfpathcurveto{\pgfqpoint{0.998163in}{0.628641in}}{\pgfqpoint{1.008762in}{0.624250in}}{\pgfqpoint{1.019812in}{0.624250in}}%
\pgfpathclose%
\pgfusepath{stroke,fill}%
\end{pgfscope}%
\begin{pgfscope}%
\pgfpathrectangle{\pgfqpoint{0.375000in}{0.330000in}}{\pgfqpoint{2.325000in}{2.310000in}}%
\pgfusepath{clip}%
\pgfsetbuttcap%
\pgfsetroundjoin%
\definecolor{currentfill}{rgb}{0.000000,0.000000,0.000000}%
\pgfsetfillcolor{currentfill}%
\pgfsetlinewidth{1.003750pt}%
\definecolor{currentstroke}{rgb}{0.000000,0.000000,0.000000}%
\pgfsetstrokecolor{currentstroke}%
\pgfsetdash{}{0pt}%
\pgfpathmoveto{\pgfqpoint{1.019812in}{0.630318in}}%
\pgfpathcurveto{\pgfqpoint{1.030863in}{0.630318in}}{\pgfqpoint{1.041462in}{0.634709in}}{\pgfqpoint{1.049275in}{0.642522in}}%
\pgfpathcurveto{\pgfqpoint{1.057089in}{0.650336in}}{\pgfqpoint{1.061479in}{0.660935in}}{\pgfqpoint{1.061479in}{0.671985in}}%
\pgfpathcurveto{\pgfqpoint{1.061479in}{0.683035in}}{\pgfqpoint{1.057089in}{0.693634in}}{\pgfqpoint{1.049275in}{0.701448in}}%
\pgfpathcurveto{\pgfqpoint{1.041462in}{0.709262in}}{\pgfqpoint{1.030863in}{0.713652in}}{\pgfqpoint{1.019812in}{0.713652in}}%
\pgfpathcurveto{\pgfqpoint{1.008762in}{0.713652in}}{\pgfqpoint{0.998163in}{0.709262in}}{\pgfqpoint{0.990350in}{0.701448in}}%
\pgfpathcurveto{\pgfqpoint{0.982536in}{0.693634in}}{\pgfqpoint{0.978146in}{0.683035in}}{\pgfqpoint{0.978146in}{0.671985in}}%
\pgfpathcurveto{\pgfqpoint{0.978146in}{0.660935in}}{\pgfqpoint{0.982536in}{0.650336in}}{\pgfqpoint{0.990350in}{0.642522in}}%
\pgfpathcurveto{\pgfqpoint{0.998163in}{0.634709in}}{\pgfqpoint{1.008762in}{0.630318in}}{\pgfqpoint{1.019812in}{0.630318in}}%
\pgfpathclose%
\pgfusepath{stroke,fill}%
\end{pgfscope}%
\begin{pgfscope}%
\pgfpathrectangle{\pgfqpoint{0.375000in}{0.330000in}}{\pgfqpoint{2.325000in}{2.310000in}}%
\pgfusepath{clip}%
\pgfsetbuttcap%
\pgfsetroundjoin%
\definecolor{currentfill}{rgb}{0.000000,0.000000,0.000000}%
\pgfsetfillcolor{currentfill}%
\pgfsetlinewidth{1.003750pt}%
\definecolor{currentstroke}{rgb}{0.000000,0.000000,0.000000}%
\pgfsetstrokecolor{currentstroke}%
\pgfsetdash{}{0pt}%
\pgfpathmoveto{\pgfqpoint{1.019812in}{0.569637in}}%
\pgfpathcurveto{\pgfqpoint{1.030863in}{0.569637in}}{\pgfqpoint{1.041462in}{0.574028in}}{\pgfqpoint{1.049275in}{0.581841in}}%
\pgfpathcurveto{\pgfqpoint{1.057089in}{0.589655in}}{\pgfqpoint{1.061479in}{0.600254in}}{\pgfqpoint{1.061479in}{0.611304in}}%
\pgfpathcurveto{\pgfqpoint{1.061479in}{0.622354in}}{\pgfqpoint{1.057089in}{0.632953in}}{\pgfqpoint{1.049275in}{0.640767in}}%
\pgfpathcurveto{\pgfqpoint{1.041462in}{0.648580in}}{\pgfqpoint{1.030863in}{0.652971in}}{\pgfqpoint{1.019812in}{0.652971in}}%
\pgfpathcurveto{\pgfqpoint{1.008762in}{0.652971in}}{\pgfqpoint{0.998163in}{0.648580in}}{\pgfqpoint{0.990350in}{0.640767in}}%
\pgfpathcurveto{\pgfqpoint{0.982536in}{0.632953in}}{\pgfqpoint{0.978146in}{0.622354in}}{\pgfqpoint{0.978146in}{0.611304in}}%
\pgfpathcurveto{\pgfqpoint{0.978146in}{0.600254in}}{\pgfqpoint{0.982536in}{0.589655in}}{\pgfqpoint{0.990350in}{0.581841in}}%
\pgfpathcurveto{\pgfqpoint{0.998163in}{0.574028in}}{\pgfqpoint{1.008762in}{0.569637in}}{\pgfqpoint{1.019812in}{0.569637in}}%
\pgfpathclose%
\pgfusepath{stroke,fill}%
\end{pgfscope}%
\begin{pgfscope}%
\pgfpathrectangle{\pgfqpoint{0.375000in}{0.330000in}}{\pgfqpoint{2.325000in}{2.310000in}}%
\pgfusepath{clip}%
\pgfsetbuttcap%
\pgfsetroundjoin%
\definecolor{currentfill}{rgb}{0.000000,0.000000,0.000000}%
\pgfsetfillcolor{currentfill}%
\pgfsetlinewidth{1.003750pt}%
\definecolor{currentstroke}{rgb}{0.000000,0.000000,0.000000}%
\pgfsetstrokecolor{currentstroke}%
\pgfsetdash{}{0pt}%
\pgfpathmoveto{\pgfqpoint{1.579875in}{1.000473in}}%
\pgfpathcurveto{\pgfqpoint{1.590925in}{1.000473in}}{\pgfqpoint{1.601524in}{1.004863in}}{\pgfqpoint{1.609338in}{1.012677in}}%
\pgfpathcurveto{\pgfqpoint{1.617151in}{1.020490in}}{\pgfqpoint{1.621542in}{1.031089in}}{\pgfqpoint{1.621542in}{1.042140in}}%
\pgfpathcurveto{\pgfqpoint{1.621542in}{1.053190in}}{\pgfqpoint{1.617151in}{1.063789in}}{\pgfqpoint{1.609338in}{1.071602in}}%
\pgfpathcurveto{\pgfqpoint{1.601524in}{1.079416in}}{\pgfqpoint{1.590925in}{1.083806in}}{\pgfqpoint{1.579875in}{1.083806in}}%
\pgfpathcurveto{\pgfqpoint{1.568825in}{1.083806in}}{\pgfqpoint{1.558226in}{1.079416in}}{\pgfqpoint{1.550412in}{1.071602in}}%
\pgfpathcurveto{\pgfqpoint{1.542599in}{1.063789in}}{\pgfqpoint{1.538208in}{1.053190in}}{\pgfqpoint{1.538208in}{1.042140in}}%
\pgfpathcurveto{\pgfqpoint{1.538208in}{1.031089in}}{\pgfqpoint{1.542599in}{1.020490in}}{\pgfqpoint{1.550412in}{1.012677in}}%
\pgfpathcurveto{\pgfqpoint{1.558226in}{1.004863in}}{\pgfqpoint{1.568825in}{1.000473in}}{\pgfqpoint{1.579875in}{1.000473in}}%
\pgfpathclose%
\pgfusepath{stroke,fill}%
\end{pgfscope}%
\begin{pgfscope}%
\pgfpathrectangle{\pgfqpoint{0.375000in}{0.330000in}}{\pgfqpoint{2.325000in}{2.310000in}}%
\pgfusepath{clip}%
\pgfsetbuttcap%
\pgfsetroundjoin%
\definecolor{currentfill}{rgb}{0.000000,0.000000,0.000000}%
\pgfsetfillcolor{currentfill}%
\pgfsetlinewidth{1.003750pt}%
\definecolor{currentstroke}{rgb}{0.000000,0.000000,0.000000}%
\pgfsetstrokecolor{currentstroke}%
\pgfsetdash{}{0pt}%
\pgfpathmoveto{\pgfqpoint{1.579875in}{0.982269in}}%
\pgfpathcurveto{\pgfqpoint{1.590925in}{0.982269in}}{\pgfqpoint{1.601524in}{0.986659in}}{\pgfqpoint{1.609338in}{0.994472in}}%
\pgfpathcurveto{\pgfqpoint{1.617151in}{1.002286in}}{\pgfqpoint{1.621542in}{1.012885in}}{\pgfqpoint{1.621542in}{1.023935in}}%
\pgfpathcurveto{\pgfqpoint{1.621542in}{1.034985in}}{\pgfqpoint{1.617151in}{1.045584in}}{\pgfqpoint{1.609338in}{1.053398in}}%
\pgfpathcurveto{\pgfqpoint{1.601524in}{1.061212in}}{\pgfqpoint{1.590925in}{1.065602in}}{\pgfqpoint{1.579875in}{1.065602in}}%
\pgfpathcurveto{\pgfqpoint{1.568825in}{1.065602in}}{\pgfqpoint{1.558226in}{1.061212in}}{\pgfqpoint{1.550412in}{1.053398in}}%
\pgfpathcurveto{\pgfqpoint{1.542599in}{1.045584in}}{\pgfqpoint{1.538208in}{1.034985in}}{\pgfqpoint{1.538208in}{1.023935in}}%
\pgfpathcurveto{\pgfqpoint{1.538208in}{1.012885in}}{\pgfqpoint{1.542599in}{1.002286in}}{\pgfqpoint{1.550412in}{0.994472in}}%
\pgfpathcurveto{\pgfqpoint{1.558226in}{0.986659in}}{\pgfqpoint{1.568825in}{0.982269in}}{\pgfqpoint{1.579875in}{0.982269in}}%
\pgfpathclose%
\pgfusepath{stroke,fill}%
\end{pgfscope}%
\begin{pgfscope}%
\pgfpathrectangle{\pgfqpoint{0.375000in}{0.330000in}}{\pgfqpoint{2.325000in}{2.310000in}}%
\pgfusepath{clip}%
\pgfsetbuttcap%
\pgfsetroundjoin%
\definecolor{currentfill}{rgb}{0.000000,0.000000,0.000000}%
\pgfsetfillcolor{currentfill}%
\pgfsetlinewidth{1.003750pt}%
\definecolor{currentstroke}{rgb}{0.000000,0.000000,0.000000}%
\pgfsetstrokecolor{currentstroke}%
\pgfsetdash{}{0pt}%
\pgfpathmoveto{\pgfqpoint{1.579875in}{0.957996in}}%
\pgfpathcurveto{\pgfqpoint{1.590925in}{0.957996in}}{\pgfqpoint{1.601524in}{0.962386in}}{\pgfqpoint{1.609338in}{0.970200in}}%
\pgfpathcurveto{\pgfqpoint{1.617151in}{0.978014in}}{\pgfqpoint{1.621542in}{0.988613in}}{\pgfqpoint{1.621542in}{0.999663in}}%
\pgfpathcurveto{\pgfqpoint{1.621542in}{1.010713in}}{\pgfqpoint{1.617151in}{1.021312in}}{\pgfqpoint{1.609338in}{1.029126in}}%
\pgfpathcurveto{\pgfqpoint{1.601524in}{1.036939in}}{\pgfqpoint{1.590925in}{1.041329in}}{\pgfqpoint{1.579875in}{1.041329in}}%
\pgfpathcurveto{\pgfqpoint{1.568825in}{1.041329in}}{\pgfqpoint{1.558226in}{1.036939in}}{\pgfqpoint{1.550412in}{1.029126in}}%
\pgfpathcurveto{\pgfqpoint{1.542599in}{1.021312in}}{\pgfqpoint{1.538208in}{1.010713in}}{\pgfqpoint{1.538208in}{0.999663in}}%
\pgfpathcurveto{\pgfqpoint{1.538208in}{0.988613in}}{\pgfqpoint{1.542599in}{0.978014in}}{\pgfqpoint{1.550412in}{0.970200in}}%
\pgfpathcurveto{\pgfqpoint{1.558226in}{0.962386in}}{\pgfqpoint{1.568825in}{0.957996in}}{\pgfqpoint{1.579875in}{0.957996in}}%
\pgfpathclose%
\pgfusepath{stroke,fill}%
\end{pgfscope}%
\begin{pgfscope}%
\pgfpathrectangle{\pgfqpoint{0.375000in}{0.330000in}}{\pgfqpoint{2.325000in}{2.310000in}}%
\pgfusepath{clip}%
\pgfsetbuttcap%
\pgfsetroundjoin%
\definecolor{currentfill}{rgb}{0.000000,0.000000,0.000000}%
\pgfsetfillcolor{currentfill}%
\pgfsetlinewidth{1.003750pt}%
\definecolor{currentstroke}{rgb}{0.000000,0.000000,0.000000}%
\pgfsetstrokecolor{currentstroke}%
\pgfsetdash{}{0pt}%
\pgfpathmoveto{\pgfqpoint{1.579875in}{1.079358in}}%
\pgfpathcurveto{\pgfqpoint{1.590925in}{1.079358in}}{\pgfqpoint{1.601524in}{1.083749in}}{\pgfqpoint{1.609338in}{1.091562in}}%
\pgfpathcurveto{\pgfqpoint{1.617151in}{1.099376in}}{\pgfqpoint{1.621542in}{1.109975in}}{\pgfqpoint{1.621542in}{1.121025in}}%
\pgfpathcurveto{\pgfqpoint{1.621542in}{1.132075in}}{\pgfqpoint{1.617151in}{1.142674in}}{\pgfqpoint{1.609338in}{1.150488in}}%
\pgfpathcurveto{\pgfqpoint{1.601524in}{1.158301in}}{\pgfqpoint{1.590925in}{1.162692in}}{\pgfqpoint{1.579875in}{1.162692in}}%
\pgfpathcurveto{\pgfqpoint{1.568825in}{1.162692in}}{\pgfqpoint{1.558226in}{1.158301in}}{\pgfqpoint{1.550412in}{1.150488in}}%
\pgfpathcurveto{\pgfqpoint{1.542599in}{1.142674in}}{\pgfqpoint{1.538208in}{1.132075in}}{\pgfqpoint{1.538208in}{1.121025in}}%
\pgfpathcurveto{\pgfqpoint{1.538208in}{1.109975in}}{\pgfqpoint{1.542599in}{1.099376in}}{\pgfqpoint{1.550412in}{1.091562in}}%
\pgfpathcurveto{\pgfqpoint{1.558226in}{1.083749in}}{\pgfqpoint{1.568825in}{1.079358in}}{\pgfqpoint{1.579875in}{1.079358in}}%
\pgfpathclose%
\pgfusepath{stroke,fill}%
\end{pgfscope}%
\begin{pgfscope}%
\pgfpathrectangle{\pgfqpoint{0.375000in}{0.330000in}}{\pgfqpoint{2.325000in}{2.310000in}}%
\pgfusepath{clip}%
\pgfsetbuttcap%
\pgfsetroundjoin%
\definecolor{currentfill}{rgb}{0.000000,0.000000,0.000000}%
\pgfsetfillcolor{currentfill}%
\pgfsetlinewidth{1.003750pt}%
\definecolor{currentstroke}{rgb}{0.000000,0.000000,0.000000}%
\pgfsetstrokecolor{currentstroke}%
\pgfsetdash{}{0pt}%
\pgfpathmoveto{\pgfqpoint{1.579875in}{1.000473in}}%
\pgfpathcurveto{\pgfqpoint{1.590925in}{1.000473in}}{\pgfqpoint{1.601524in}{1.004863in}}{\pgfqpoint{1.609338in}{1.012677in}}%
\pgfpathcurveto{\pgfqpoint{1.617151in}{1.020490in}}{\pgfqpoint{1.621542in}{1.031089in}}{\pgfqpoint{1.621542in}{1.042140in}}%
\pgfpathcurveto{\pgfqpoint{1.621542in}{1.053190in}}{\pgfqpoint{1.617151in}{1.063789in}}{\pgfqpoint{1.609338in}{1.071602in}}%
\pgfpathcurveto{\pgfqpoint{1.601524in}{1.079416in}}{\pgfqpoint{1.590925in}{1.083806in}}{\pgfqpoint{1.579875in}{1.083806in}}%
\pgfpathcurveto{\pgfqpoint{1.568825in}{1.083806in}}{\pgfqpoint{1.558226in}{1.079416in}}{\pgfqpoint{1.550412in}{1.071602in}}%
\pgfpathcurveto{\pgfqpoint{1.542599in}{1.063789in}}{\pgfqpoint{1.538208in}{1.053190in}}{\pgfqpoint{1.538208in}{1.042140in}}%
\pgfpathcurveto{\pgfqpoint{1.538208in}{1.031089in}}{\pgfqpoint{1.542599in}{1.020490in}}{\pgfqpoint{1.550412in}{1.012677in}}%
\pgfpathcurveto{\pgfqpoint{1.558226in}{1.004863in}}{\pgfqpoint{1.568825in}{1.000473in}}{\pgfqpoint{1.579875in}{1.000473in}}%
\pgfpathclose%
\pgfusepath{stroke,fill}%
\end{pgfscope}%
\begin{pgfscope}%
\pgfpathrectangle{\pgfqpoint{0.375000in}{0.330000in}}{\pgfqpoint{2.325000in}{2.310000in}}%
\pgfusepath{clip}%
\pgfsetbuttcap%
\pgfsetroundjoin%
\definecolor{currentfill}{rgb}{0.000000,0.000000,0.000000}%
\pgfsetfillcolor{currentfill}%
\pgfsetlinewidth{1.003750pt}%
\definecolor{currentstroke}{rgb}{0.000000,0.000000,0.000000}%
\pgfsetstrokecolor{currentstroke}%
\pgfsetdash{}{0pt}%
\pgfpathmoveto{\pgfqpoint{1.579875in}{0.964064in}}%
\pgfpathcurveto{\pgfqpoint{1.590925in}{0.964064in}}{\pgfqpoint{1.601524in}{0.968455in}}{\pgfqpoint{1.609338in}{0.976268in}}%
\pgfpathcurveto{\pgfqpoint{1.617151in}{0.984082in}}{\pgfqpoint{1.621542in}{0.994681in}}{\pgfqpoint{1.621542in}{1.005731in}}%
\pgfpathcurveto{\pgfqpoint{1.621542in}{1.016781in}}{\pgfqpoint{1.617151in}{1.027380in}}{\pgfqpoint{1.609338in}{1.035194in}}%
\pgfpathcurveto{\pgfqpoint{1.601524in}{1.043007in}}{\pgfqpoint{1.590925in}{1.047398in}}{\pgfqpoint{1.579875in}{1.047398in}}%
\pgfpathcurveto{\pgfqpoint{1.568825in}{1.047398in}}{\pgfqpoint{1.558226in}{1.043007in}}{\pgfqpoint{1.550412in}{1.035194in}}%
\pgfpathcurveto{\pgfqpoint{1.542599in}{1.027380in}}{\pgfqpoint{1.538208in}{1.016781in}}{\pgfqpoint{1.538208in}{1.005731in}}%
\pgfpathcurveto{\pgfqpoint{1.538208in}{0.994681in}}{\pgfqpoint{1.542599in}{0.984082in}}{\pgfqpoint{1.550412in}{0.976268in}}%
\pgfpathcurveto{\pgfqpoint{1.558226in}{0.968455in}}{\pgfqpoint{1.568825in}{0.964064in}}{\pgfqpoint{1.579875in}{0.964064in}}%
\pgfpathclose%
\pgfusepath{stroke,fill}%
\end{pgfscope}%
\begin{pgfscope}%
\pgfpathrectangle{\pgfqpoint{0.375000in}{0.330000in}}{\pgfqpoint{2.325000in}{2.310000in}}%
\pgfusepath{clip}%
\pgfsetbuttcap%
\pgfsetroundjoin%
\definecolor{currentfill}{rgb}{0.000000,0.000000,0.000000}%
\pgfsetfillcolor{currentfill}%
\pgfsetlinewidth{1.003750pt}%
\definecolor{currentstroke}{rgb}{0.000000,0.000000,0.000000}%
\pgfsetstrokecolor{currentstroke}%
\pgfsetdash{}{0pt}%
\pgfpathmoveto{\pgfqpoint{1.579875in}{1.206788in}}%
\pgfpathcurveto{\pgfqpoint{1.590925in}{1.206788in}}{\pgfqpoint{1.601524in}{1.211179in}}{\pgfqpoint{1.609338in}{1.218992in}}%
\pgfpathcurveto{\pgfqpoint{1.617151in}{1.226806in}}{\pgfqpoint{1.621542in}{1.237405in}}{\pgfqpoint{1.621542in}{1.248455in}}%
\pgfpathcurveto{\pgfqpoint{1.621542in}{1.259505in}}{\pgfqpoint{1.617151in}{1.270104in}}{\pgfqpoint{1.609338in}{1.277918in}}%
\pgfpathcurveto{\pgfqpoint{1.601524in}{1.285732in}}{\pgfqpoint{1.590925in}{1.290122in}}{\pgfqpoint{1.579875in}{1.290122in}}%
\pgfpathcurveto{\pgfqpoint{1.568825in}{1.290122in}}{\pgfqpoint{1.558226in}{1.285732in}}{\pgfqpoint{1.550412in}{1.277918in}}%
\pgfpathcurveto{\pgfqpoint{1.542599in}{1.270104in}}{\pgfqpoint{1.538208in}{1.259505in}}{\pgfqpoint{1.538208in}{1.248455in}}%
\pgfpathcurveto{\pgfqpoint{1.538208in}{1.237405in}}{\pgfqpoint{1.542599in}{1.226806in}}{\pgfqpoint{1.550412in}{1.218992in}}%
\pgfpathcurveto{\pgfqpoint{1.558226in}{1.211179in}}{\pgfqpoint{1.568825in}{1.206788in}}{\pgfqpoint{1.579875in}{1.206788in}}%
\pgfpathclose%
\pgfusepath{stroke,fill}%
\end{pgfscope}%
\begin{pgfscope}%
\pgfpathrectangle{\pgfqpoint{0.375000in}{0.330000in}}{\pgfqpoint{2.325000in}{2.310000in}}%
\pgfusepath{clip}%
\pgfsetbuttcap%
\pgfsetroundjoin%
\definecolor{currentfill}{rgb}{0.000000,0.000000,0.000000}%
\pgfsetfillcolor{currentfill}%
\pgfsetlinewidth{1.003750pt}%
\definecolor{currentstroke}{rgb}{0.000000,0.000000,0.000000}%
\pgfsetstrokecolor{currentstroke}%
\pgfsetdash{}{0pt}%
\pgfpathmoveto{\pgfqpoint{1.579875in}{0.970132in}}%
\pgfpathcurveto{\pgfqpoint{1.590925in}{0.970132in}}{\pgfqpoint{1.601524in}{0.974523in}}{\pgfqpoint{1.609338in}{0.982336in}}%
\pgfpathcurveto{\pgfqpoint{1.617151in}{0.990150in}}{\pgfqpoint{1.621542in}{1.000749in}}{\pgfqpoint{1.621542in}{1.011799in}}%
\pgfpathcurveto{\pgfqpoint{1.621542in}{1.022849in}}{\pgfqpoint{1.617151in}{1.033448in}}{\pgfqpoint{1.609338in}{1.041262in}}%
\pgfpathcurveto{\pgfqpoint{1.601524in}{1.049075in}}{\pgfqpoint{1.590925in}{1.053466in}}{\pgfqpoint{1.579875in}{1.053466in}}%
\pgfpathcurveto{\pgfqpoint{1.568825in}{1.053466in}}{\pgfqpoint{1.558226in}{1.049075in}}{\pgfqpoint{1.550412in}{1.041262in}}%
\pgfpathcurveto{\pgfqpoint{1.542599in}{1.033448in}}{\pgfqpoint{1.538208in}{1.022849in}}{\pgfqpoint{1.538208in}{1.011799in}}%
\pgfpathcurveto{\pgfqpoint{1.538208in}{1.000749in}}{\pgfqpoint{1.542599in}{0.990150in}}{\pgfqpoint{1.550412in}{0.982336in}}%
\pgfpathcurveto{\pgfqpoint{1.558226in}{0.974523in}}{\pgfqpoint{1.568825in}{0.970132in}}{\pgfqpoint{1.579875in}{0.970132in}}%
\pgfpathclose%
\pgfusepath{stroke,fill}%
\end{pgfscope}%
\begin{pgfscope}%
\pgfpathrectangle{\pgfqpoint{0.375000in}{0.330000in}}{\pgfqpoint{2.325000in}{2.310000in}}%
\pgfusepath{clip}%
\pgfsetbuttcap%
\pgfsetroundjoin%
\definecolor{currentfill}{rgb}{0.000000,0.000000,0.000000}%
\pgfsetfillcolor{currentfill}%
\pgfsetlinewidth{1.003750pt}%
\definecolor{currentstroke}{rgb}{0.000000,0.000000,0.000000}%
\pgfsetstrokecolor{currentstroke}%
\pgfsetdash{}{0pt}%
\pgfpathmoveto{\pgfqpoint{1.579875in}{0.988337in}}%
\pgfpathcurveto{\pgfqpoint{1.590925in}{0.988337in}}{\pgfqpoint{1.601524in}{0.992727in}}{\pgfqpoint{1.609338in}{1.000541in}}%
\pgfpathcurveto{\pgfqpoint{1.617151in}{1.008354in}}{\pgfqpoint{1.621542in}{1.018953in}}{\pgfqpoint{1.621542in}{1.030003in}}%
\pgfpathcurveto{\pgfqpoint{1.621542in}{1.041053in}}{\pgfqpoint{1.617151in}{1.051653in}}{\pgfqpoint{1.609338in}{1.059466in}}%
\pgfpathcurveto{\pgfqpoint{1.601524in}{1.067280in}}{\pgfqpoint{1.590925in}{1.071670in}}{\pgfqpoint{1.579875in}{1.071670in}}%
\pgfpathcurveto{\pgfqpoint{1.568825in}{1.071670in}}{\pgfqpoint{1.558226in}{1.067280in}}{\pgfqpoint{1.550412in}{1.059466in}}%
\pgfpathcurveto{\pgfqpoint{1.542599in}{1.051653in}}{\pgfqpoint{1.538208in}{1.041053in}}{\pgfqpoint{1.538208in}{1.030003in}}%
\pgfpathcurveto{\pgfqpoint{1.538208in}{1.018953in}}{\pgfqpoint{1.542599in}{1.008354in}}{\pgfqpoint{1.550412in}{1.000541in}}%
\pgfpathcurveto{\pgfqpoint{1.558226in}{0.992727in}}{\pgfqpoint{1.568825in}{0.988337in}}{\pgfqpoint{1.579875in}{0.988337in}}%
\pgfpathclose%
\pgfusepath{stroke,fill}%
\end{pgfscope}%
\begin{pgfscope}%
\pgfpathrectangle{\pgfqpoint{0.375000in}{0.330000in}}{\pgfqpoint{2.325000in}{2.310000in}}%
\pgfusepath{clip}%
\pgfsetbuttcap%
\pgfsetroundjoin%
\definecolor{currentfill}{rgb}{0.000000,0.000000,0.000000}%
\pgfsetfillcolor{currentfill}%
\pgfsetlinewidth{1.003750pt}%
\definecolor{currentstroke}{rgb}{0.000000,0.000000,0.000000}%
\pgfsetstrokecolor{currentstroke}%
\pgfsetdash{}{0pt}%
\pgfpathmoveto{\pgfqpoint{1.579875in}{0.970132in}}%
\pgfpathcurveto{\pgfqpoint{1.590925in}{0.970132in}}{\pgfqpoint{1.601524in}{0.974523in}}{\pgfqpoint{1.609338in}{0.982336in}}%
\pgfpathcurveto{\pgfqpoint{1.617151in}{0.990150in}}{\pgfqpoint{1.621542in}{1.000749in}}{\pgfqpoint{1.621542in}{1.011799in}}%
\pgfpathcurveto{\pgfqpoint{1.621542in}{1.022849in}}{\pgfqpoint{1.617151in}{1.033448in}}{\pgfqpoint{1.609338in}{1.041262in}}%
\pgfpathcurveto{\pgfqpoint{1.601524in}{1.049075in}}{\pgfqpoint{1.590925in}{1.053466in}}{\pgfqpoint{1.579875in}{1.053466in}}%
\pgfpathcurveto{\pgfqpoint{1.568825in}{1.053466in}}{\pgfqpoint{1.558226in}{1.049075in}}{\pgfqpoint{1.550412in}{1.041262in}}%
\pgfpathcurveto{\pgfqpoint{1.542599in}{1.033448in}}{\pgfqpoint{1.538208in}{1.022849in}}{\pgfqpoint{1.538208in}{1.011799in}}%
\pgfpathcurveto{\pgfqpoint{1.538208in}{1.000749in}}{\pgfqpoint{1.542599in}{0.990150in}}{\pgfqpoint{1.550412in}{0.982336in}}%
\pgfpathcurveto{\pgfqpoint{1.558226in}{0.974523in}}{\pgfqpoint{1.568825in}{0.970132in}}{\pgfqpoint{1.579875in}{0.970132in}}%
\pgfpathclose%
\pgfusepath{stroke,fill}%
\end{pgfscope}%
\begin{pgfscope}%
\pgfpathrectangle{\pgfqpoint{0.375000in}{0.330000in}}{\pgfqpoint{2.325000in}{2.310000in}}%
\pgfusepath{clip}%
\pgfsetbuttcap%
\pgfsetroundjoin%
\definecolor{currentfill}{rgb}{0.000000,0.000000,0.000000}%
\pgfsetfillcolor{currentfill}%
\pgfsetlinewidth{1.003750pt}%
\definecolor{currentstroke}{rgb}{0.000000,0.000000,0.000000}%
\pgfsetstrokecolor{currentstroke}%
\pgfsetdash{}{0pt}%
\pgfpathmoveto{\pgfqpoint{1.579875in}{1.176448in}}%
\pgfpathcurveto{\pgfqpoint{1.590925in}{1.176448in}}{\pgfqpoint{1.601524in}{1.180838in}}{\pgfqpoint{1.609338in}{1.188652in}}%
\pgfpathcurveto{\pgfqpoint{1.617151in}{1.196465in}}{\pgfqpoint{1.621542in}{1.207065in}}{\pgfqpoint{1.621542in}{1.218115in}}%
\pgfpathcurveto{\pgfqpoint{1.621542in}{1.229165in}}{\pgfqpoint{1.617151in}{1.239764in}}{\pgfqpoint{1.609338in}{1.247577in}}%
\pgfpathcurveto{\pgfqpoint{1.601524in}{1.255391in}}{\pgfqpoint{1.590925in}{1.259781in}}{\pgfqpoint{1.579875in}{1.259781in}}%
\pgfpathcurveto{\pgfqpoint{1.568825in}{1.259781in}}{\pgfqpoint{1.558226in}{1.255391in}}{\pgfqpoint{1.550412in}{1.247577in}}%
\pgfpathcurveto{\pgfqpoint{1.542599in}{1.239764in}}{\pgfqpoint{1.538208in}{1.229165in}}{\pgfqpoint{1.538208in}{1.218115in}}%
\pgfpathcurveto{\pgfqpoint{1.538208in}{1.207065in}}{\pgfqpoint{1.542599in}{1.196465in}}{\pgfqpoint{1.550412in}{1.188652in}}%
\pgfpathcurveto{\pgfqpoint{1.558226in}{1.180838in}}{\pgfqpoint{1.568825in}{1.176448in}}{\pgfqpoint{1.579875in}{1.176448in}}%
\pgfpathclose%
\pgfusepath{stroke,fill}%
\end{pgfscope}%
\begin{pgfscope}%
\pgfpathrectangle{\pgfqpoint{0.375000in}{0.330000in}}{\pgfqpoint{2.325000in}{2.310000in}}%
\pgfusepath{clip}%
\pgfsetbuttcap%
\pgfsetroundjoin%
\definecolor{currentfill}{rgb}{0.000000,0.000000,0.000000}%
\pgfsetfillcolor{currentfill}%
\pgfsetlinewidth{1.003750pt}%
\definecolor{currentstroke}{rgb}{0.000000,0.000000,0.000000}%
\pgfsetstrokecolor{currentstroke}%
\pgfsetdash{}{0pt}%
\pgfpathmoveto{\pgfqpoint{1.579875in}{1.188584in}}%
\pgfpathcurveto{\pgfqpoint{1.590925in}{1.188584in}}{\pgfqpoint{1.601524in}{1.192974in}}{\pgfqpoint{1.609338in}{1.200788in}}%
\pgfpathcurveto{\pgfqpoint{1.617151in}{1.208602in}}{\pgfqpoint{1.621542in}{1.219201in}}{\pgfqpoint{1.621542in}{1.230251in}}%
\pgfpathcurveto{\pgfqpoint{1.621542in}{1.241301in}}{\pgfqpoint{1.617151in}{1.251900in}}{\pgfqpoint{1.609338in}{1.259714in}}%
\pgfpathcurveto{\pgfqpoint{1.601524in}{1.267527in}}{\pgfqpoint{1.590925in}{1.271918in}}{\pgfqpoint{1.579875in}{1.271918in}}%
\pgfpathcurveto{\pgfqpoint{1.568825in}{1.271918in}}{\pgfqpoint{1.558226in}{1.267527in}}{\pgfqpoint{1.550412in}{1.259714in}}%
\pgfpathcurveto{\pgfqpoint{1.542599in}{1.251900in}}{\pgfqpoint{1.538208in}{1.241301in}}{\pgfqpoint{1.538208in}{1.230251in}}%
\pgfpathcurveto{\pgfqpoint{1.538208in}{1.219201in}}{\pgfqpoint{1.542599in}{1.208602in}}{\pgfqpoint{1.550412in}{1.200788in}}%
\pgfpathcurveto{\pgfqpoint{1.558226in}{1.192974in}}{\pgfqpoint{1.568825in}{1.188584in}}{\pgfqpoint{1.579875in}{1.188584in}}%
\pgfpathclose%
\pgfusepath{stroke,fill}%
\end{pgfscope}%
\begin{pgfscope}%
\pgfpathrectangle{\pgfqpoint{0.375000in}{0.330000in}}{\pgfqpoint{2.325000in}{2.310000in}}%
\pgfusepath{clip}%
\pgfsetbuttcap%
\pgfsetroundjoin%
\definecolor{currentfill}{rgb}{0.000000,0.000000,0.000000}%
\pgfsetfillcolor{currentfill}%
\pgfsetlinewidth{1.003750pt}%
\definecolor{currentstroke}{rgb}{0.000000,0.000000,0.000000}%
\pgfsetstrokecolor{currentstroke}%
\pgfsetdash{}{0pt}%
\pgfpathmoveto{\pgfqpoint{1.579875in}{1.170380in}}%
\pgfpathcurveto{\pgfqpoint{1.590925in}{1.170380in}}{\pgfqpoint{1.601524in}{1.174770in}}{\pgfqpoint{1.609338in}{1.182584in}}%
\pgfpathcurveto{\pgfqpoint{1.617151in}{1.190397in}}{\pgfqpoint{1.621542in}{1.200996in}}{\pgfqpoint{1.621542in}{1.212047in}}%
\pgfpathcurveto{\pgfqpoint{1.621542in}{1.223097in}}{\pgfqpoint{1.617151in}{1.233696in}}{\pgfqpoint{1.609338in}{1.241509in}}%
\pgfpathcurveto{\pgfqpoint{1.601524in}{1.249323in}}{\pgfqpoint{1.590925in}{1.253713in}}{\pgfqpoint{1.579875in}{1.253713in}}%
\pgfpathcurveto{\pgfqpoint{1.568825in}{1.253713in}}{\pgfqpoint{1.558226in}{1.249323in}}{\pgfqpoint{1.550412in}{1.241509in}}%
\pgfpathcurveto{\pgfqpoint{1.542599in}{1.233696in}}{\pgfqpoint{1.538208in}{1.223097in}}{\pgfqpoint{1.538208in}{1.212047in}}%
\pgfpathcurveto{\pgfqpoint{1.538208in}{1.200996in}}{\pgfqpoint{1.542599in}{1.190397in}}{\pgfqpoint{1.550412in}{1.182584in}}%
\pgfpathcurveto{\pgfqpoint{1.558226in}{1.174770in}}{\pgfqpoint{1.568825in}{1.170380in}}{\pgfqpoint{1.579875in}{1.170380in}}%
\pgfpathclose%
\pgfusepath{stroke,fill}%
\end{pgfscope}%
\begin{pgfscope}%
\pgfpathrectangle{\pgfqpoint{0.375000in}{0.330000in}}{\pgfqpoint{2.325000in}{2.310000in}}%
\pgfusepath{clip}%
\pgfsetbuttcap%
\pgfsetroundjoin%
\definecolor{currentfill}{rgb}{0.000000,0.000000,0.000000}%
\pgfsetfillcolor{currentfill}%
\pgfsetlinewidth{1.003750pt}%
\definecolor{currentstroke}{rgb}{0.000000,0.000000,0.000000}%
\pgfsetstrokecolor{currentstroke}%
\pgfsetdash{}{0pt}%
\pgfpathmoveto{\pgfqpoint{1.579875in}{0.970132in}}%
\pgfpathcurveto{\pgfqpoint{1.590925in}{0.970132in}}{\pgfqpoint{1.601524in}{0.974523in}}{\pgfqpoint{1.609338in}{0.982336in}}%
\pgfpathcurveto{\pgfqpoint{1.617151in}{0.990150in}}{\pgfqpoint{1.621542in}{1.000749in}}{\pgfqpoint{1.621542in}{1.011799in}}%
\pgfpathcurveto{\pgfqpoint{1.621542in}{1.022849in}}{\pgfqpoint{1.617151in}{1.033448in}}{\pgfqpoint{1.609338in}{1.041262in}}%
\pgfpathcurveto{\pgfqpoint{1.601524in}{1.049075in}}{\pgfqpoint{1.590925in}{1.053466in}}{\pgfqpoint{1.579875in}{1.053466in}}%
\pgfpathcurveto{\pgfqpoint{1.568825in}{1.053466in}}{\pgfqpoint{1.558226in}{1.049075in}}{\pgfqpoint{1.550412in}{1.041262in}}%
\pgfpathcurveto{\pgfqpoint{1.542599in}{1.033448in}}{\pgfqpoint{1.538208in}{1.022849in}}{\pgfqpoint{1.538208in}{1.011799in}}%
\pgfpathcurveto{\pgfqpoint{1.538208in}{1.000749in}}{\pgfqpoint{1.542599in}{0.990150in}}{\pgfqpoint{1.550412in}{0.982336in}}%
\pgfpathcurveto{\pgfqpoint{1.558226in}{0.974523in}}{\pgfqpoint{1.568825in}{0.970132in}}{\pgfqpoint{1.579875in}{0.970132in}}%
\pgfpathclose%
\pgfusepath{stroke,fill}%
\end{pgfscope}%
\begin{pgfscope}%
\pgfpathrectangle{\pgfqpoint{0.375000in}{0.330000in}}{\pgfqpoint{2.325000in}{2.310000in}}%
\pgfusepath{clip}%
\pgfsetbuttcap%
\pgfsetroundjoin%
\definecolor{currentfill}{rgb}{0.000000,0.000000,0.000000}%
\pgfsetfillcolor{currentfill}%
\pgfsetlinewidth{1.003750pt}%
\definecolor{currentstroke}{rgb}{0.000000,0.000000,0.000000}%
\pgfsetstrokecolor{currentstroke}%
\pgfsetdash{}{0pt}%
\pgfpathmoveto{\pgfqpoint{1.579875in}{1.140039in}}%
\pgfpathcurveto{\pgfqpoint{1.590925in}{1.140039in}}{\pgfqpoint{1.601524in}{1.144430in}}{\pgfqpoint{1.609338in}{1.152243in}}%
\pgfpathcurveto{\pgfqpoint{1.617151in}{1.160057in}}{\pgfqpoint{1.621542in}{1.170656in}}{\pgfqpoint{1.621542in}{1.181706in}}%
\pgfpathcurveto{\pgfqpoint{1.621542in}{1.192756in}}{\pgfqpoint{1.617151in}{1.203355in}}{\pgfqpoint{1.609338in}{1.211169in}}%
\pgfpathcurveto{\pgfqpoint{1.601524in}{1.218982in}}{\pgfqpoint{1.590925in}{1.223373in}}{\pgfqpoint{1.579875in}{1.223373in}}%
\pgfpathcurveto{\pgfqpoint{1.568825in}{1.223373in}}{\pgfqpoint{1.558226in}{1.218982in}}{\pgfqpoint{1.550412in}{1.211169in}}%
\pgfpathcurveto{\pgfqpoint{1.542599in}{1.203355in}}{\pgfqpoint{1.538208in}{1.192756in}}{\pgfqpoint{1.538208in}{1.181706in}}%
\pgfpathcurveto{\pgfqpoint{1.538208in}{1.170656in}}{\pgfqpoint{1.542599in}{1.160057in}}{\pgfqpoint{1.550412in}{1.152243in}}%
\pgfpathcurveto{\pgfqpoint{1.558226in}{1.144430in}}{\pgfqpoint{1.568825in}{1.140039in}}{\pgfqpoint{1.579875in}{1.140039in}}%
\pgfpathclose%
\pgfusepath{stroke,fill}%
\end{pgfscope}%
\begin{pgfscope}%
\pgfpathrectangle{\pgfqpoint{0.375000in}{0.330000in}}{\pgfqpoint{2.325000in}{2.310000in}}%
\pgfusepath{clip}%
\pgfsetbuttcap%
\pgfsetroundjoin%
\definecolor{currentfill}{rgb}{0.000000,0.000000,0.000000}%
\pgfsetfillcolor{currentfill}%
\pgfsetlinewidth{1.003750pt}%
\definecolor{currentstroke}{rgb}{0.000000,0.000000,0.000000}%
\pgfsetstrokecolor{currentstroke}%
\pgfsetdash{}{0pt}%
\pgfpathmoveto{\pgfqpoint{1.579875in}{1.067222in}}%
\pgfpathcurveto{\pgfqpoint{1.590925in}{1.067222in}}{\pgfqpoint{1.601524in}{1.071612in}}{\pgfqpoint{1.609338in}{1.079426in}}%
\pgfpathcurveto{\pgfqpoint{1.617151in}{1.087240in}}{\pgfqpoint{1.621542in}{1.097839in}}{\pgfqpoint{1.621542in}{1.108889in}}%
\pgfpathcurveto{\pgfqpoint{1.621542in}{1.119939in}}{\pgfqpoint{1.617151in}{1.130538in}}{\pgfqpoint{1.609338in}{1.138352in}}%
\pgfpathcurveto{\pgfqpoint{1.601524in}{1.146165in}}{\pgfqpoint{1.590925in}{1.150555in}}{\pgfqpoint{1.579875in}{1.150555in}}%
\pgfpathcurveto{\pgfqpoint{1.568825in}{1.150555in}}{\pgfqpoint{1.558226in}{1.146165in}}{\pgfqpoint{1.550412in}{1.138352in}}%
\pgfpathcurveto{\pgfqpoint{1.542599in}{1.130538in}}{\pgfqpoint{1.538208in}{1.119939in}}{\pgfqpoint{1.538208in}{1.108889in}}%
\pgfpathcurveto{\pgfqpoint{1.538208in}{1.097839in}}{\pgfqpoint{1.542599in}{1.087240in}}{\pgfqpoint{1.550412in}{1.079426in}}%
\pgfpathcurveto{\pgfqpoint{1.558226in}{1.071612in}}{\pgfqpoint{1.568825in}{1.067222in}}{\pgfqpoint{1.579875in}{1.067222in}}%
\pgfpathclose%
\pgfusepath{stroke,fill}%
\end{pgfscope}%
\begin{pgfscope}%
\pgfpathrectangle{\pgfqpoint{0.375000in}{0.330000in}}{\pgfqpoint{2.325000in}{2.310000in}}%
\pgfusepath{clip}%
\pgfsetbuttcap%
\pgfsetroundjoin%
\definecolor{currentfill}{rgb}{0.000000,0.000000,0.000000}%
\pgfsetfillcolor{currentfill}%
\pgfsetlinewidth{1.003750pt}%
\definecolor{currentstroke}{rgb}{0.000000,0.000000,0.000000}%
\pgfsetstrokecolor{currentstroke}%
\pgfsetdash{}{0pt}%
\pgfpathmoveto{\pgfqpoint{1.579875in}{1.012609in}}%
\pgfpathcurveto{\pgfqpoint{1.590925in}{1.012609in}}{\pgfqpoint{1.601524in}{1.016999in}}{\pgfqpoint{1.609338in}{1.024813in}}%
\pgfpathcurveto{\pgfqpoint{1.617151in}{1.032627in}}{\pgfqpoint{1.621542in}{1.043226in}}{\pgfqpoint{1.621542in}{1.054276in}}%
\pgfpathcurveto{\pgfqpoint{1.621542in}{1.065326in}}{\pgfqpoint{1.617151in}{1.075925in}}{\pgfqpoint{1.609338in}{1.083739in}}%
\pgfpathcurveto{\pgfqpoint{1.601524in}{1.091552in}}{\pgfqpoint{1.590925in}{1.095942in}}{\pgfqpoint{1.579875in}{1.095942in}}%
\pgfpathcurveto{\pgfqpoint{1.568825in}{1.095942in}}{\pgfqpoint{1.558226in}{1.091552in}}{\pgfqpoint{1.550412in}{1.083739in}}%
\pgfpathcurveto{\pgfqpoint{1.542599in}{1.075925in}}{\pgfqpoint{1.538208in}{1.065326in}}{\pgfqpoint{1.538208in}{1.054276in}}%
\pgfpathcurveto{\pgfqpoint{1.538208in}{1.043226in}}{\pgfqpoint{1.542599in}{1.032627in}}{\pgfqpoint{1.550412in}{1.024813in}}%
\pgfpathcurveto{\pgfqpoint{1.558226in}{1.016999in}}{\pgfqpoint{1.568825in}{1.012609in}}{\pgfqpoint{1.579875in}{1.012609in}}%
\pgfpathclose%
\pgfusepath{stroke,fill}%
\end{pgfscope}%
\begin{pgfscope}%
\pgfpathrectangle{\pgfqpoint{0.375000in}{0.330000in}}{\pgfqpoint{2.325000in}{2.310000in}}%
\pgfusepath{clip}%
\pgfsetbuttcap%
\pgfsetroundjoin%
\definecolor{currentfill}{rgb}{0.000000,0.000000,0.000000}%
\pgfsetfillcolor{currentfill}%
\pgfsetlinewidth{1.003750pt}%
\definecolor{currentstroke}{rgb}{0.000000,0.000000,0.000000}%
\pgfsetstrokecolor{currentstroke}%
\pgfsetdash{}{0pt}%
\pgfpathmoveto{\pgfqpoint{1.579875in}{1.103631in}}%
\pgfpathcurveto{\pgfqpoint{1.590925in}{1.103631in}}{\pgfqpoint{1.601524in}{1.108021in}}{\pgfqpoint{1.609338in}{1.115835in}}%
\pgfpathcurveto{\pgfqpoint{1.617151in}{1.123648in}}{\pgfqpoint{1.621542in}{1.134247in}}{\pgfqpoint{1.621542in}{1.145297in}}%
\pgfpathcurveto{\pgfqpoint{1.621542in}{1.156347in}}{\pgfqpoint{1.617151in}{1.166947in}}{\pgfqpoint{1.609338in}{1.174760in}}%
\pgfpathcurveto{\pgfqpoint{1.601524in}{1.182574in}}{\pgfqpoint{1.590925in}{1.186964in}}{\pgfqpoint{1.579875in}{1.186964in}}%
\pgfpathcurveto{\pgfqpoint{1.568825in}{1.186964in}}{\pgfqpoint{1.558226in}{1.182574in}}{\pgfqpoint{1.550412in}{1.174760in}}%
\pgfpathcurveto{\pgfqpoint{1.542599in}{1.166947in}}{\pgfqpoint{1.538208in}{1.156347in}}{\pgfqpoint{1.538208in}{1.145297in}}%
\pgfpathcurveto{\pgfqpoint{1.538208in}{1.134247in}}{\pgfqpoint{1.542599in}{1.123648in}}{\pgfqpoint{1.550412in}{1.115835in}}%
\pgfpathcurveto{\pgfqpoint{1.558226in}{1.108021in}}{\pgfqpoint{1.568825in}{1.103631in}}{\pgfqpoint{1.579875in}{1.103631in}}%
\pgfpathclose%
\pgfusepath{stroke,fill}%
\end{pgfscope}%
\begin{pgfscope}%
\pgfpathrectangle{\pgfqpoint{0.375000in}{0.330000in}}{\pgfqpoint{2.325000in}{2.310000in}}%
\pgfusepath{clip}%
\pgfsetbuttcap%
\pgfsetroundjoin%
\definecolor{currentfill}{rgb}{0.000000,0.000000,0.000000}%
\pgfsetfillcolor{currentfill}%
\pgfsetlinewidth{1.003750pt}%
\definecolor{currentstroke}{rgb}{0.000000,0.000000,0.000000}%
\pgfsetstrokecolor{currentstroke}%
\pgfsetdash{}{0pt}%
\pgfpathmoveto{\pgfqpoint{1.579875in}{1.133971in}}%
\pgfpathcurveto{\pgfqpoint{1.590925in}{1.133971in}}{\pgfqpoint{1.601524in}{1.138361in}}{\pgfqpoint{1.609338in}{1.146175in}}%
\pgfpathcurveto{\pgfqpoint{1.617151in}{1.153989in}}{\pgfqpoint{1.621542in}{1.164588in}}{\pgfqpoint{1.621542in}{1.175638in}}%
\pgfpathcurveto{\pgfqpoint{1.621542in}{1.186688in}}{\pgfqpoint{1.617151in}{1.197287in}}{\pgfqpoint{1.609338in}{1.205101in}}%
\pgfpathcurveto{\pgfqpoint{1.601524in}{1.212914in}}{\pgfqpoint{1.590925in}{1.217305in}}{\pgfqpoint{1.579875in}{1.217305in}}%
\pgfpathcurveto{\pgfqpoint{1.568825in}{1.217305in}}{\pgfqpoint{1.558226in}{1.212914in}}{\pgfqpoint{1.550412in}{1.205101in}}%
\pgfpathcurveto{\pgfqpoint{1.542599in}{1.197287in}}{\pgfqpoint{1.538208in}{1.186688in}}{\pgfqpoint{1.538208in}{1.175638in}}%
\pgfpathcurveto{\pgfqpoint{1.538208in}{1.164588in}}{\pgfqpoint{1.542599in}{1.153989in}}{\pgfqpoint{1.550412in}{1.146175in}}%
\pgfpathcurveto{\pgfqpoint{1.558226in}{1.138361in}}{\pgfqpoint{1.568825in}{1.133971in}}{\pgfqpoint{1.579875in}{1.133971in}}%
\pgfpathclose%
\pgfusepath{stroke,fill}%
\end{pgfscope}%
\begin{pgfscope}%
\pgfpathrectangle{\pgfqpoint{0.375000in}{0.330000in}}{\pgfqpoint{2.325000in}{2.310000in}}%
\pgfusepath{clip}%
\pgfsetbuttcap%
\pgfsetroundjoin%
\definecolor{currentfill}{rgb}{0.000000,0.000000,0.000000}%
\pgfsetfillcolor{currentfill}%
\pgfsetlinewidth{1.003750pt}%
\definecolor{currentstroke}{rgb}{0.000000,0.000000,0.000000}%
\pgfsetstrokecolor{currentstroke}%
\pgfsetdash{}{0pt}%
\pgfpathmoveto{\pgfqpoint{1.579875in}{0.994405in}}%
\pgfpathcurveto{\pgfqpoint{1.590925in}{0.994405in}}{\pgfqpoint{1.601524in}{0.998795in}}{\pgfqpoint{1.609338in}{1.006609in}}%
\pgfpathcurveto{\pgfqpoint{1.617151in}{1.014422in}}{\pgfqpoint{1.621542in}{1.025021in}}{\pgfqpoint{1.621542in}{1.036071in}}%
\pgfpathcurveto{\pgfqpoint{1.621542in}{1.047122in}}{\pgfqpoint{1.617151in}{1.057721in}}{\pgfqpoint{1.609338in}{1.065534in}}%
\pgfpathcurveto{\pgfqpoint{1.601524in}{1.073348in}}{\pgfqpoint{1.590925in}{1.077738in}}{\pgfqpoint{1.579875in}{1.077738in}}%
\pgfpathcurveto{\pgfqpoint{1.568825in}{1.077738in}}{\pgfqpoint{1.558226in}{1.073348in}}{\pgfqpoint{1.550412in}{1.065534in}}%
\pgfpathcurveto{\pgfqpoint{1.542599in}{1.057721in}}{\pgfqpoint{1.538208in}{1.047122in}}{\pgfqpoint{1.538208in}{1.036071in}}%
\pgfpathcurveto{\pgfqpoint{1.538208in}{1.025021in}}{\pgfqpoint{1.542599in}{1.014422in}}{\pgfqpoint{1.550412in}{1.006609in}}%
\pgfpathcurveto{\pgfqpoint{1.558226in}{0.998795in}}{\pgfqpoint{1.568825in}{0.994405in}}{\pgfqpoint{1.579875in}{0.994405in}}%
\pgfpathclose%
\pgfusepath{stroke,fill}%
\end{pgfscope}%
\begin{pgfscope}%
\pgfpathrectangle{\pgfqpoint{0.375000in}{0.330000in}}{\pgfqpoint{2.325000in}{2.310000in}}%
\pgfusepath{clip}%
\pgfsetbuttcap%
\pgfsetroundjoin%
\definecolor{currentfill}{rgb}{0.000000,0.000000,0.000000}%
\pgfsetfillcolor{currentfill}%
\pgfsetlinewidth{1.003750pt}%
\definecolor{currentstroke}{rgb}{0.000000,0.000000,0.000000}%
\pgfsetstrokecolor{currentstroke}%
\pgfsetdash{}{0pt}%
\pgfpathmoveto{\pgfqpoint{1.579875in}{0.951928in}}%
\pgfpathcurveto{\pgfqpoint{1.590925in}{0.951928in}}{\pgfqpoint{1.601524in}{0.956318in}}{\pgfqpoint{1.609338in}{0.964132in}}%
\pgfpathcurveto{\pgfqpoint{1.617151in}{0.971946in}}{\pgfqpoint{1.621542in}{0.982545in}}{\pgfqpoint{1.621542in}{0.993595in}}%
\pgfpathcurveto{\pgfqpoint{1.621542in}{1.004645in}}{\pgfqpoint{1.617151in}{1.015244in}}{\pgfqpoint{1.609338in}{1.023058in}}%
\pgfpathcurveto{\pgfqpoint{1.601524in}{1.030871in}}{\pgfqpoint{1.590925in}{1.035261in}}{\pgfqpoint{1.579875in}{1.035261in}}%
\pgfpathcurveto{\pgfqpoint{1.568825in}{1.035261in}}{\pgfqpoint{1.558226in}{1.030871in}}{\pgfqpoint{1.550412in}{1.023058in}}%
\pgfpathcurveto{\pgfqpoint{1.542599in}{1.015244in}}{\pgfqpoint{1.538208in}{1.004645in}}{\pgfqpoint{1.538208in}{0.993595in}}%
\pgfpathcurveto{\pgfqpoint{1.538208in}{0.982545in}}{\pgfqpoint{1.542599in}{0.971946in}}{\pgfqpoint{1.550412in}{0.964132in}}%
\pgfpathcurveto{\pgfqpoint{1.558226in}{0.956318in}}{\pgfqpoint{1.568825in}{0.951928in}}{\pgfqpoint{1.579875in}{0.951928in}}%
\pgfpathclose%
\pgfusepath{stroke,fill}%
\end{pgfscope}%
\begin{pgfscope}%
\pgfpathrectangle{\pgfqpoint{0.375000in}{0.330000in}}{\pgfqpoint{2.325000in}{2.310000in}}%
\pgfusepath{clip}%
\pgfsetbuttcap%
\pgfsetroundjoin%
\definecolor{currentfill}{rgb}{0.000000,0.000000,0.000000}%
\pgfsetfillcolor{currentfill}%
\pgfsetlinewidth{1.003750pt}%
\definecolor{currentstroke}{rgb}{0.000000,0.000000,0.000000}%
\pgfsetstrokecolor{currentstroke}%
\pgfsetdash{}{0pt}%
\pgfpathmoveto{\pgfqpoint{1.579875in}{0.982269in}}%
\pgfpathcurveto{\pgfqpoint{1.590925in}{0.982269in}}{\pgfqpoint{1.601524in}{0.986659in}}{\pgfqpoint{1.609338in}{0.994472in}}%
\pgfpathcurveto{\pgfqpoint{1.617151in}{1.002286in}}{\pgfqpoint{1.621542in}{1.012885in}}{\pgfqpoint{1.621542in}{1.023935in}}%
\pgfpathcurveto{\pgfqpoint{1.621542in}{1.034985in}}{\pgfqpoint{1.617151in}{1.045584in}}{\pgfqpoint{1.609338in}{1.053398in}}%
\pgfpathcurveto{\pgfqpoint{1.601524in}{1.061212in}}{\pgfqpoint{1.590925in}{1.065602in}}{\pgfqpoint{1.579875in}{1.065602in}}%
\pgfpathcurveto{\pgfqpoint{1.568825in}{1.065602in}}{\pgfqpoint{1.558226in}{1.061212in}}{\pgfqpoint{1.550412in}{1.053398in}}%
\pgfpathcurveto{\pgfqpoint{1.542599in}{1.045584in}}{\pgfqpoint{1.538208in}{1.034985in}}{\pgfqpoint{1.538208in}{1.023935in}}%
\pgfpathcurveto{\pgfqpoint{1.538208in}{1.012885in}}{\pgfqpoint{1.542599in}{1.002286in}}{\pgfqpoint{1.550412in}{0.994472in}}%
\pgfpathcurveto{\pgfqpoint{1.558226in}{0.986659in}}{\pgfqpoint{1.568825in}{0.982269in}}{\pgfqpoint{1.579875in}{0.982269in}}%
\pgfpathclose%
\pgfusepath{stroke,fill}%
\end{pgfscope}%
\begin{pgfscope}%
\pgfpathrectangle{\pgfqpoint{0.375000in}{0.330000in}}{\pgfqpoint{2.325000in}{2.310000in}}%
\pgfusepath{clip}%
\pgfsetbuttcap%
\pgfsetroundjoin%
\definecolor{currentfill}{rgb}{0.000000,0.000000,0.000000}%
\pgfsetfillcolor{currentfill}%
\pgfsetlinewidth{1.003750pt}%
\definecolor{currentstroke}{rgb}{0.000000,0.000000,0.000000}%
\pgfsetstrokecolor{currentstroke}%
\pgfsetdash{}{0pt}%
\pgfpathmoveto{\pgfqpoint{1.579875in}{1.085426in}}%
\pgfpathcurveto{\pgfqpoint{1.590925in}{1.085426in}}{\pgfqpoint{1.601524in}{1.089817in}}{\pgfqpoint{1.609338in}{1.097630in}}%
\pgfpathcurveto{\pgfqpoint{1.617151in}{1.105444in}}{\pgfqpoint{1.621542in}{1.116043in}}{\pgfqpoint{1.621542in}{1.127093in}}%
\pgfpathcurveto{\pgfqpoint{1.621542in}{1.138143in}}{\pgfqpoint{1.617151in}{1.148742in}}{\pgfqpoint{1.609338in}{1.156556in}}%
\pgfpathcurveto{\pgfqpoint{1.601524in}{1.164369in}}{\pgfqpoint{1.590925in}{1.168760in}}{\pgfqpoint{1.579875in}{1.168760in}}%
\pgfpathcurveto{\pgfqpoint{1.568825in}{1.168760in}}{\pgfqpoint{1.558226in}{1.164369in}}{\pgfqpoint{1.550412in}{1.156556in}}%
\pgfpathcurveto{\pgfqpoint{1.542599in}{1.148742in}}{\pgfqpoint{1.538208in}{1.138143in}}{\pgfqpoint{1.538208in}{1.127093in}}%
\pgfpathcurveto{\pgfqpoint{1.538208in}{1.116043in}}{\pgfqpoint{1.542599in}{1.105444in}}{\pgfqpoint{1.550412in}{1.097630in}}%
\pgfpathcurveto{\pgfqpoint{1.558226in}{1.089817in}}{\pgfqpoint{1.568825in}{1.085426in}}{\pgfqpoint{1.579875in}{1.085426in}}%
\pgfpathclose%
\pgfusepath{stroke,fill}%
\end{pgfscope}%
\begin{pgfscope}%
\pgfpathrectangle{\pgfqpoint{0.375000in}{0.330000in}}{\pgfqpoint{2.325000in}{2.310000in}}%
\pgfusepath{clip}%
\pgfsetbuttcap%
\pgfsetroundjoin%
\definecolor{currentfill}{rgb}{0.000000,0.000000,0.000000}%
\pgfsetfillcolor{currentfill}%
\pgfsetlinewidth{1.003750pt}%
\definecolor{currentstroke}{rgb}{0.000000,0.000000,0.000000}%
\pgfsetstrokecolor{currentstroke}%
\pgfsetdash{}{0pt}%
\pgfpathmoveto{\pgfqpoint{1.579875in}{1.055086in}}%
\pgfpathcurveto{\pgfqpoint{1.590925in}{1.055086in}}{\pgfqpoint{1.601524in}{1.059476in}}{\pgfqpoint{1.609338in}{1.067290in}}%
\pgfpathcurveto{\pgfqpoint{1.617151in}{1.075103in}}{\pgfqpoint{1.621542in}{1.085702in}}{\pgfqpoint{1.621542in}{1.096753in}}%
\pgfpathcurveto{\pgfqpoint{1.621542in}{1.107803in}}{\pgfqpoint{1.617151in}{1.118402in}}{\pgfqpoint{1.609338in}{1.126215in}}%
\pgfpathcurveto{\pgfqpoint{1.601524in}{1.134029in}}{\pgfqpoint{1.590925in}{1.138419in}}{\pgfqpoint{1.579875in}{1.138419in}}%
\pgfpathcurveto{\pgfqpoint{1.568825in}{1.138419in}}{\pgfqpoint{1.558226in}{1.134029in}}{\pgfqpoint{1.550412in}{1.126215in}}%
\pgfpathcurveto{\pgfqpoint{1.542599in}{1.118402in}}{\pgfqpoint{1.538208in}{1.107803in}}{\pgfqpoint{1.538208in}{1.096753in}}%
\pgfpathcurveto{\pgfqpoint{1.538208in}{1.085702in}}{\pgfqpoint{1.542599in}{1.075103in}}{\pgfqpoint{1.550412in}{1.067290in}}%
\pgfpathcurveto{\pgfqpoint{1.558226in}{1.059476in}}{\pgfqpoint{1.568825in}{1.055086in}}{\pgfqpoint{1.579875in}{1.055086in}}%
\pgfpathclose%
\pgfusepath{stroke,fill}%
\end{pgfscope}%
\begin{pgfscope}%
\pgfpathrectangle{\pgfqpoint{0.375000in}{0.330000in}}{\pgfqpoint{2.325000in}{2.310000in}}%
\pgfusepath{clip}%
\pgfsetbuttcap%
\pgfsetroundjoin%
\definecolor{currentfill}{rgb}{0.000000,0.000000,0.000000}%
\pgfsetfillcolor{currentfill}%
\pgfsetlinewidth{1.003750pt}%
\definecolor{currentstroke}{rgb}{0.000000,0.000000,0.000000}%
\pgfsetstrokecolor{currentstroke}%
\pgfsetdash{}{0pt}%
\pgfpathmoveto{\pgfqpoint{1.579875in}{0.982269in}}%
\pgfpathcurveto{\pgfqpoint{1.590925in}{0.982269in}}{\pgfqpoint{1.601524in}{0.986659in}}{\pgfqpoint{1.609338in}{0.994472in}}%
\pgfpathcurveto{\pgfqpoint{1.617151in}{1.002286in}}{\pgfqpoint{1.621542in}{1.012885in}}{\pgfqpoint{1.621542in}{1.023935in}}%
\pgfpathcurveto{\pgfqpoint{1.621542in}{1.034985in}}{\pgfqpoint{1.617151in}{1.045584in}}{\pgfqpoint{1.609338in}{1.053398in}}%
\pgfpathcurveto{\pgfqpoint{1.601524in}{1.061212in}}{\pgfqpoint{1.590925in}{1.065602in}}{\pgfqpoint{1.579875in}{1.065602in}}%
\pgfpathcurveto{\pgfqpoint{1.568825in}{1.065602in}}{\pgfqpoint{1.558226in}{1.061212in}}{\pgfqpoint{1.550412in}{1.053398in}}%
\pgfpathcurveto{\pgfqpoint{1.542599in}{1.045584in}}{\pgfqpoint{1.538208in}{1.034985in}}{\pgfqpoint{1.538208in}{1.023935in}}%
\pgfpathcurveto{\pgfqpoint{1.538208in}{1.012885in}}{\pgfqpoint{1.542599in}{1.002286in}}{\pgfqpoint{1.550412in}{0.994472in}}%
\pgfpathcurveto{\pgfqpoint{1.558226in}{0.986659in}}{\pgfqpoint{1.568825in}{0.982269in}}{\pgfqpoint{1.579875in}{0.982269in}}%
\pgfpathclose%
\pgfusepath{stroke,fill}%
\end{pgfscope}%
\begin{pgfscope}%
\pgfpathrectangle{\pgfqpoint{0.375000in}{0.330000in}}{\pgfqpoint{2.325000in}{2.310000in}}%
\pgfusepath{clip}%
\pgfsetbuttcap%
\pgfsetroundjoin%
\definecolor{currentfill}{rgb}{0.000000,0.000000,0.000000}%
\pgfsetfillcolor{currentfill}%
\pgfsetlinewidth{1.003750pt}%
\definecolor{currentstroke}{rgb}{0.000000,0.000000,0.000000}%
\pgfsetstrokecolor{currentstroke}%
\pgfsetdash{}{0pt}%
\pgfpathmoveto{\pgfqpoint{1.579875in}{1.000473in}}%
\pgfpathcurveto{\pgfqpoint{1.590925in}{1.000473in}}{\pgfqpoint{1.601524in}{1.004863in}}{\pgfqpoint{1.609338in}{1.012677in}}%
\pgfpathcurveto{\pgfqpoint{1.617151in}{1.020490in}}{\pgfqpoint{1.621542in}{1.031089in}}{\pgfqpoint{1.621542in}{1.042140in}}%
\pgfpathcurveto{\pgfqpoint{1.621542in}{1.053190in}}{\pgfqpoint{1.617151in}{1.063789in}}{\pgfqpoint{1.609338in}{1.071602in}}%
\pgfpathcurveto{\pgfqpoint{1.601524in}{1.079416in}}{\pgfqpoint{1.590925in}{1.083806in}}{\pgfqpoint{1.579875in}{1.083806in}}%
\pgfpathcurveto{\pgfqpoint{1.568825in}{1.083806in}}{\pgfqpoint{1.558226in}{1.079416in}}{\pgfqpoint{1.550412in}{1.071602in}}%
\pgfpathcurveto{\pgfqpoint{1.542599in}{1.063789in}}{\pgfqpoint{1.538208in}{1.053190in}}{\pgfqpoint{1.538208in}{1.042140in}}%
\pgfpathcurveto{\pgfqpoint{1.538208in}{1.031089in}}{\pgfqpoint{1.542599in}{1.020490in}}{\pgfqpoint{1.550412in}{1.012677in}}%
\pgfpathcurveto{\pgfqpoint{1.558226in}{1.004863in}}{\pgfqpoint{1.568825in}{1.000473in}}{\pgfqpoint{1.579875in}{1.000473in}}%
\pgfpathclose%
\pgfusepath{stroke,fill}%
\end{pgfscope}%
\begin{pgfscope}%
\pgfpathrectangle{\pgfqpoint{0.375000in}{0.330000in}}{\pgfqpoint{2.325000in}{2.310000in}}%
\pgfusepath{clip}%
\pgfsetbuttcap%
\pgfsetroundjoin%
\definecolor{currentfill}{rgb}{0.000000,0.000000,0.000000}%
\pgfsetfillcolor{currentfill}%
\pgfsetlinewidth{1.003750pt}%
\definecolor{currentstroke}{rgb}{0.000000,0.000000,0.000000}%
\pgfsetstrokecolor{currentstroke}%
\pgfsetdash{}{0pt}%
\pgfpathmoveto{\pgfqpoint{1.579875in}{1.042950in}}%
\pgfpathcurveto{\pgfqpoint{1.590925in}{1.042950in}}{\pgfqpoint{1.601524in}{1.047340in}}{\pgfqpoint{1.609338in}{1.055154in}}%
\pgfpathcurveto{\pgfqpoint{1.617151in}{1.062967in}}{\pgfqpoint{1.621542in}{1.073566in}}{\pgfqpoint{1.621542in}{1.084616in}}%
\pgfpathcurveto{\pgfqpoint{1.621542in}{1.095666in}}{\pgfqpoint{1.617151in}{1.106265in}}{\pgfqpoint{1.609338in}{1.114079in}}%
\pgfpathcurveto{\pgfqpoint{1.601524in}{1.121893in}}{\pgfqpoint{1.590925in}{1.126283in}}{\pgfqpoint{1.579875in}{1.126283in}}%
\pgfpathcurveto{\pgfqpoint{1.568825in}{1.126283in}}{\pgfqpoint{1.558226in}{1.121893in}}{\pgfqpoint{1.550412in}{1.114079in}}%
\pgfpathcurveto{\pgfqpoint{1.542599in}{1.106265in}}{\pgfqpoint{1.538208in}{1.095666in}}{\pgfqpoint{1.538208in}{1.084616in}}%
\pgfpathcurveto{\pgfqpoint{1.538208in}{1.073566in}}{\pgfqpoint{1.542599in}{1.062967in}}{\pgfqpoint{1.550412in}{1.055154in}}%
\pgfpathcurveto{\pgfqpoint{1.558226in}{1.047340in}}{\pgfqpoint{1.568825in}{1.042950in}}{\pgfqpoint{1.579875in}{1.042950in}}%
\pgfpathclose%
\pgfusepath{stroke,fill}%
\end{pgfscope}%
\begin{pgfscope}%
\pgfpathrectangle{\pgfqpoint{0.375000in}{0.330000in}}{\pgfqpoint{2.325000in}{2.310000in}}%
\pgfusepath{clip}%
\pgfsetbuttcap%
\pgfsetroundjoin%
\definecolor{currentfill}{rgb}{0.000000,0.000000,0.000000}%
\pgfsetfillcolor{currentfill}%
\pgfsetlinewidth{1.003750pt}%
\definecolor{currentstroke}{rgb}{0.000000,0.000000,0.000000}%
\pgfsetstrokecolor{currentstroke}%
\pgfsetdash{}{0pt}%
\pgfpathmoveto{\pgfqpoint{1.579875in}{0.988337in}}%
\pgfpathcurveto{\pgfqpoint{1.590925in}{0.988337in}}{\pgfqpoint{1.601524in}{0.992727in}}{\pgfqpoint{1.609338in}{1.000541in}}%
\pgfpathcurveto{\pgfqpoint{1.617151in}{1.008354in}}{\pgfqpoint{1.621542in}{1.018953in}}{\pgfqpoint{1.621542in}{1.030003in}}%
\pgfpathcurveto{\pgfqpoint{1.621542in}{1.041053in}}{\pgfqpoint{1.617151in}{1.051653in}}{\pgfqpoint{1.609338in}{1.059466in}}%
\pgfpathcurveto{\pgfqpoint{1.601524in}{1.067280in}}{\pgfqpoint{1.590925in}{1.071670in}}{\pgfqpoint{1.579875in}{1.071670in}}%
\pgfpathcurveto{\pgfqpoint{1.568825in}{1.071670in}}{\pgfqpoint{1.558226in}{1.067280in}}{\pgfqpoint{1.550412in}{1.059466in}}%
\pgfpathcurveto{\pgfqpoint{1.542599in}{1.051653in}}{\pgfqpoint{1.538208in}{1.041053in}}{\pgfqpoint{1.538208in}{1.030003in}}%
\pgfpathcurveto{\pgfqpoint{1.538208in}{1.018953in}}{\pgfqpoint{1.542599in}{1.008354in}}{\pgfqpoint{1.550412in}{1.000541in}}%
\pgfpathcurveto{\pgfqpoint{1.558226in}{0.992727in}}{\pgfqpoint{1.568825in}{0.988337in}}{\pgfqpoint{1.579875in}{0.988337in}}%
\pgfpathclose%
\pgfusepath{stroke,fill}%
\end{pgfscope}%
\begin{pgfscope}%
\pgfpathrectangle{\pgfqpoint{0.375000in}{0.330000in}}{\pgfqpoint{2.325000in}{2.310000in}}%
\pgfusepath{clip}%
\pgfsetbuttcap%
\pgfsetroundjoin%
\definecolor{currentfill}{rgb}{0.000000,0.000000,0.000000}%
\pgfsetfillcolor{currentfill}%
\pgfsetlinewidth{1.003750pt}%
\definecolor{currentstroke}{rgb}{0.000000,0.000000,0.000000}%
\pgfsetstrokecolor{currentstroke}%
\pgfsetdash{}{0pt}%
\pgfpathmoveto{\pgfqpoint{1.579875in}{1.079358in}}%
\pgfpathcurveto{\pgfqpoint{1.590925in}{1.079358in}}{\pgfqpoint{1.601524in}{1.083749in}}{\pgfqpoint{1.609338in}{1.091562in}}%
\pgfpathcurveto{\pgfqpoint{1.617151in}{1.099376in}}{\pgfqpoint{1.621542in}{1.109975in}}{\pgfqpoint{1.621542in}{1.121025in}}%
\pgfpathcurveto{\pgfqpoint{1.621542in}{1.132075in}}{\pgfqpoint{1.617151in}{1.142674in}}{\pgfqpoint{1.609338in}{1.150488in}}%
\pgfpathcurveto{\pgfqpoint{1.601524in}{1.158301in}}{\pgfqpoint{1.590925in}{1.162692in}}{\pgfqpoint{1.579875in}{1.162692in}}%
\pgfpathcurveto{\pgfqpoint{1.568825in}{1.162692in}}{\pgfqpoint{1.558226in}{1.158301in}}{\pgfqpoint{1.550412in}{1.150488in}}%
\pgfpathcurveto{\pgfqpoint{1.542599in}{1.142674in}}{\pgfqpoint{1.538208in}{1.132075in}}{\pgfqpoint{1.538208in}{1.121025in}}%
\pgfpathcurveto{\pgfqpoint{1.538208in}{1.109975in}}{\pgfqpoint{1.542599in}{1.099376in}}{\pgfqpoint{1.550412in}{1.091562in}}%
\pgfpathcurveto{\pgfqpoint{1.558226in}{1.083749in}}{\pgfqpoint{1.568825in}{1.079358in}}{\pgfqpoint{1.579875in}{1.079358in}}%
\pgfpathclose%
\pgfusepath{stroke,fill}%
\end{pgfscope}%
\begin{pgfscope}%
\pgfpathrectangle{\pgfqpoint{0.375000in}{0.330000in}}{\pgfqpoint{2.325000in}{2.310000in}}%
\pgfusepath{clip}%
\pgfsetbuttcap%
\pgfsetroundjoin%
\definecolor{currentfill}{rgb}{0.000000,0.000000,0.000000}%
\pgfsetfillcolor{currentfill}%
\pgfsetlinewidth{1.003750pt}%
\definecolor{currentstroke}{rgb}{0.000000,0.000000,0.000000}%
\pgfsetstrokecolor{currentstroke}%
\pgfsetdash{}{0pt}%
\pgfpathmoveto{\pgfqpoint{1.579875in}{1.055086in}}%
\pgfpathcurveto{\pgfqpoint{1.590925in}{1.055086in}}{\pgfqpoint{1.601524in}{1.059476in}}{\pgfqpoint{1.609338in}{1.067290in}}%
\pgfpathcurveto{\pgfqpoint{1.617151in}{1.075103in}}{\pgfqpoint{1.621542in}{1.085702in}}{\pgfqpoint{1.621542in}{1.096753in}}%
\pgfpathcurveto{\pgfqpoint{1.621542in}{1.107803in}}{\pgfqpoint{1.617151in}{1.118402in}}{\pgfqpoint{1.609338in}{1.126215in}}%
\pgfpathcurveto{\pgfqpoint{1.601524in}{1.134029in}}{\pgfqpoint{1.590925in}{1.138419in}}{\pgfqpoint{1.579875in}{1.138419in}}%
\pgfpathcurveto{\pgfqpoint{1.568825in}{1.138419in}}{\pgfqpoint{1.558226in}{1.134029in}}{\pgfqpoint{1.550412in}{1.126215in}}%
\pgfpathcurveto{\pgfqpoint{1.542599in}{1.118402in}}{\pgfqpoint{1.538208in}{1.107803in}}{\pgfqpoint{1.538208in}{1.096753in}}%
\pgfpathcurveto{\pgfqpoint{1.538208in}{1.085702in}}{\pgfqpoint{1.542599in}{1.075103in}}{\pgfqpoint{1.550412in}{1.067290in}}%
\pgfpathcurveto{\pgfqpoint{1.558226in}{1.059476in}}{\pgfqpoint{1.568825in}{1.055086in}}{\pgfqpoint{1.579875in}{1.055086in}}%
\pgfpathclose%
\pgfusepath{stroke,fill}%
\end{pgfscope}%
\begin{pgfscope}%
\pgfpathrectangle{\pgfqpoint{0.375000in}{0.330000in}}{\pgfqpoint{2.325000in}{2.310000in}}%
\pgfusepath{clip}%
\pgfsetbuttcap%
\pgfsetroundjoin%
\definecolor{currentfill}{rgb}{0.000000,0.000000,0.000000}%
\pgfsetfillcolor{currentfill}%
\pgfsetlinewidth{1.003750pt}%
\definecolor{currentstroke}{rgb}{0.000000,0.000000,0.000000}%
\pgfsetstrokecolor{currentstroke}%
\pgfsetdash{}{0pt}%
\pgfpathmoveto{\pgfqpoint{1.579875in}{0.988337in}}%
\pgfpathcurveto{\pgfqpoint{1.590925in}{0.988337in}}{\pgfqpoint{1.601524in}{0.992727in}}{\pgfqpoint{1.609338in}{1.000541in}}%
\pgfpathcurveto{\pgfqpoint{1.617151in}{1.008354in}}{\pgfqpoint{1.621542in}{1.018953in}}{\pgfqpoint{1.621542in}{1.030003in}}%
\pgfpathcurveto{\pgfqpoint{1.621542in}{1.041053in}}{\pgfqpoint{1.617151in}{1.051653in}}{\pgfqpoint{1.609338in}{1.059466in}}%
\pgfpathcurveto{\pgfqpoint{1.601524in}{1.067280in}}{\pgfqpoint{1.590925in}{1.071670in}}{\pgfqpoint{1.579875in}{1.071670in}}%
\pgfpathcurveto{\pgfqpoint{1.568825in}{1.071670in}}{\pgfqpoint{1.558226in}{1.067280in}}{\pgfqpoint{1.550412in}{1.059466in}}%
\pgfpathcurveto{\pgfqpoint{1.542599in}{1.051653in}}{\pgfqpoint{1.538208in}{1.041053in}}{\pgfqpoint{1.538208in}{1.030003in}}%
\pgfpathcurveto{\pgfqpoint{1.538208in}{1.018953in}}{\pgfqpoint{1.542599in}{1.008354in}}{\pgfqpoint{1.550412in}{1.000541in}}%
\pgfpathcurveto{\pgfqpoint{1.558226in}{0.992727in}}{\pgfqpoint{1.568825in}{0.988337in}}{\pgfqpoint{1.579875in}{0.988337in}}%
\pgfpathclose%
\pgfusepath{stroke,fill}%
\end{pgfscope}%
\begin{pgfscope}%
\pgfpathrectangle{\pgfqpoint{0.375000in}{0.330000in}}{\pgfqpoint{2.325000in}{2.310000in}}%
\pgfusepath{clip}%
\pgfsetbuttcap%
\pgfsetroundjoin%
\definecolor{currentfill}{rgb}{0.000000,0.000000,0.000000}%
\pgfsetfillcolor{currentfill}%
\pgfsetlinewidth{1.003750pt}%
\definecolor{currentstroke}{rgb}{0.000000,0.000000,0.000000}%
\pgfsetstrokecolor{currentstroke}%
\pgfsetdash{}{0pt}%
\pgfpathmoveto{\pgfqpoint{1.579875in}{1.121835in}}%
\pgfpathcurveto{\pgfqpoint{1.590925in}{1.121835in}}{\pgfqpoint{1.601524in}{1.126225in}}{\pgfqpoint{1.609338in}{1.134039in}}%
\pgfpathcurveto{\pgfqpoint{1.617151in}{1.141853in}}{\pgfqpoint{1.621542in}{1.152452in}}{\pgfqpoint{1.621542in}{1.163502in}}%
\pgfpathcurveto{\pgfqpoint{1.621542in}{1.174552in}}{\pgfqpoint{1.617151in}{1.185151in}}{\pgfqpoint{1.609338in}{1.192964in}}%
\pgfpathcurveto{\pgfqpoint{1.601524in}{1.200778in}}{\pgfqpoint{1.590925in}{1.205168in}}{\pgfqpoint{1.579875in}{1.205168in}}%
\pgfpathcurveto{\pgfqpoint{1.568825in}{1.205168in}}{\pgfqpoint{1.558226in}{1.200778in}}{\pgfqpoint{1.550412in}{1.192964in}}%
\pgfpathcurveto{\pgfqpoint{1.542599in}{1.185151in}}{\pgfqpoint{1.538208in}{1.174552in}}{\pgfqpoint{1.538208in}{1.163502in}}%
\pgfpathcurveto{\pgfqpoint{1.538208in}{1.152452in}}{\pgfqpoint{1.542599in}{1.141853in}}{\pgfqpoint{1.550412in}{1.134039in}}%
\pgfpathcurveto{\pgfqpoint{1.558226in}{1.126225in}}{\pgfqpoint{1.568825in}{1.121835in}}{\pgfqpoint{1.579875in}{1.121835in}}%
\pgfpathclose%
\pgfusepath{stroke,fill}%
\end{pgfscope}%
\begin{pgfscope}%
\pgfpathrectangle{\pgfqpoint{0.375000in}{0.330000in}}{\pgfqpoint{2.325000in}{2.310000in}}%
\pgfusepath{clip}%
\pgfsetbuttcap%
\pgfsetroundjoin%
\definecolor{currentfill}{rgb}{0.000000,0.000000,0.000000}%
\pgfsetfillcolor{currentfill}%
\pgfsetlinewidth{1.003750pt}%
\definecolor{currentstroke}{rgb}{0.000000,0.000000,0.000000}%
\pgfsetstrokecolor{currentstroke}%
\pgfsetdash{}{0pt}%
\pgfpathmoveto{\pgfqpoint{1.579875in}{1.073290in}}%
\pgfpathcurveto{\pgfqpoint{1.590925in}{1.073290in}}{\pgfqpoint{1.601524in}{1.077680in}}{\pgfqpoint{1.609338in}{1.085494in}}%
\pgfpathcurveto{\pgfqpoint{1.617151in}{1.093308in}}{\pgfqpoint{1.621542in}{1.103907in}}{\pgfqpoint{1.621542in}{1.114957in}}%
\pgfpathcurveto{\pgfqpoint{1.621542in}{1.126007in}}{\pgfqpoint{1.617151in}{1.136606in}}{\pgfqpoint{1.609338in}{1.144420in}}%
\pgfpathcurveto{\pgfqpoint{1.601524in}{1.152233in}}{\pgfqpoint{1.590925in}{1.156624in}}{\pgfqpoint{1.579875in}{1.156624in}}%
\pgfpathcurveto{\pgfqpoint{1.568825in}{1.156624in}}{\pgfqpoint{1.558226in}{1.152233in}}{\pgfqpoint{1.550412in}{1.144420in}}%
\pgfpathcurveto{\pgfqpoint{1.542599in}{1.136606in}}{\pgfqpoint{1.538208in}{1.126007in}}{\pgfqpoint{1.538208in}{1.114957in}}%
\pgfpathcurveto{\pgfqpoint{1.538208in}{1.103907in}}{\pgfqpoint{1.542599in}{1.093308in}}{\pgfqpoint{1.550412in}{1.085494in}}%
\pgfpathcurveto{\pgfqpoint{1.558226in}{1.077680in}}{\pgfqpoint{1.568825in}{1.073290in}}{\pgfqpoint{1.579875in}{1.073290in}}%
\pgfpathclose%
\pgfusepath{stroke,fill}%
\end{pgfscope}%
\begin{pgfscope}%
\pgfpathrectangle{\pgfqpoint{0.375000in}{0.330000in}}{\pgfqpoint{2.325000in}{2.310000in}}%
\pgfusepath{clip}%
\pgfsetbuttcap%
\pgfsetroundjoin%
\definecolor{currentfill}{rgb}{0.000000,0.000000,0.000000}%
\pgfsetfillcolor{currentfill}%
\pgfsetlinewidth{1.003750pt}%
\definecolor{currentstroke}{rgb}{0.000000,0.000000,0.000000}%
\pgfsetstrokecolor{currentstroke}%
\pgfsetdash{}{0pt}%
\pgfpathmoveto{\pgfqpoint{1.579875in}{1.073290in}}%
\pgfpathcurveto{\pgfqpoint{1.590925in}{1.073290in}}{\pgfqpoint{1.601524in}{1.077680in}}{\pgfqpoint{1.609338in}{1.085494in}}%
\pgfpathcurveto{\pgfqpoint{1.617151in}{1.093308in}}{\pgfqpoint{1.621542in}{1.103907in}}{\pgfqpoint{1.621542in}{1.114957in}}%
\pgfpathcurveto{\pgfqpoint{1.621542in}{1.126007in}}{\pgfqpoint{1.617151in}{1.136606in}}{\pgfqpoint{1.609338in}{1.144420in}}%
\pgfpathcurveto{\pgfqpoint{1.601524in}{1.152233in}}{\pgfqpoint{1.590925in}{1.156624in}}{\pgfqpoint{1.579875in}{1.156624in}}%
\pgfpathcurveto{\pgfqpoint{1.568825in}{1.156624in}}{\pgfqpoint{1.558226in}{1.152233in}}{\pgfqpoint{1.550412in}{1.144420in}}%
\pgfpathcurveto{\pgfqpoint{1.542599in}{1.136606in}}{\pgfqpoint{1.538208in}{1.126007in}}{\pgfqpoint{1.538208in}{1.114957in}}%
\pgfpathcurveto{\pgfqpoint{1.538208in}{1.103907in}}{\pgfqpoint{1.542599in}{1.093308in}}{\pgfqpoint{1.550412in}{1.085494in}}%
\pgfpathcurveto{\pgfqpoint{1.558226in}{1.077680in}}{\pgfqpoint{1.568825in}{1.073290in}}{\pgfqpoint{1.579875in}{1.073290in}}%
\pgfpathclose%
\pgfusepath{stroke,fill}%
\end{pgfscope}%
\begin{pgfscope}%
\pgfpathrectangle{\pgfqpoint{0.375000in}{0.330000in}}{\pgfqpoint{2.325000in}{2.310000in}}%
\pgfusepath{clip}%
\pgfsetbuttcap%
\pgfsetroundjoin%
\definecolor{currentfill}{rgb}{0.000000,0.000000,0.000000}%
\pgfsetfillcolor{currentfill}%
\pgfsetlinewidth{1.003750pt}%
\definecolor{currentstroke}{rgb}{0.000000,0.000000,0.000000}%
\pgfsetstrokecolor{currentstroke}%
\pgfsetdash{}{0pt}%
\pgfpathmoveto{\pgfqpoint{1.579875in}{1.109699in}}%
\pgfpathcurveto{\pgfqpoint{1.590925in}{1.109699in}}{\pgfqpoint{1.601524in}{1.114089in}}{\pgfqpoint{1.609338in}{1.121903in}}%
\pgfpathcurveto{\pgfqpoint{1.617151in}{1.129716in}}{\pgfqpoint{1.621542in}{1.140315in}}{\pgfqpoint{1.621542in}{1.151365in}}%
\pgfpathcurveto{\pgfqpoint{1.621542in}{1.162416in}}{\pgfqpoint{1.617151in}{1.173015in}}{\pgfqpoint{1.609338in}{1.180828in}}%
\pgfpathcurveto{\pgfqpoint{1.601524in}{1.188642in}}{\pgfqpoint{1.590925in}{1.193032in}}{\pgfqpoint{1.579875in}{1.193032in}}%
\pgfpathcurveto{\pgfqpoint{1.568825in}{1.193032in}}{\pgfqpoint{1.558226in}{1.188642in}}{\pgfqpoint{1.550412in}{1.180828in}}%
\pgfpathcurveto{\pgfqpoint{1.542599in}{1.173015in}}{\pgfqpoint{1.538208in}{1.162416in}}{\pgfqpoint{1.538208in}{1.151365in}}%
\pgfpathcurveto{\pgfqpoint{1.538208in}{1.140315in}}{\pgfqpoint{1.542599in}{1.129716in}}{\pgfqpoint{1.550412in}{1.121903in}}%
\pgfpathcurveto{\pgfqpoint{1.558226in}{1.114089in}}{\pgfqpoint{1.568825in}{1.109699in}}{\pgfqpoint{1.579875in}{1.109699in}}%
\pgfpathclose%
\pgfusepath{stroke,fill}%
\end{pgfscope}%
\begin{pgfscope}%
\pgfpathrectangle{\pgfqpoint{0.375000in}{0.330000in}}{\pgfqpoint{2.325000in}{2.310000in}}%
\pgfusepath{clip}%
\pgfsetbuttcap%
\pgfsetroundjoin%
\definecolor{currentfill}{rgb}{0.000000,0.000000,0.000000}%
\pgfsetfillcolor{currentfill}%
\pgfsetlinewidth{1.003750pt}%
\definecolor{currentstroke}{rgb}{0.000000,0.000000,0.000000}%
\pgfsetstrokecolor{currentstroke}%
\pgfsetdash{}{0pt}%
\pgfpathmoveto{\pgfqpoint{1.579875in}{1.000473in}}%
\pgfpathcurveto{\pgfqpoint{1.590925in}{1.000473in}}{\pgfqpoint{1.601524in}{1.004863in}}{\pgfqpoint{1.609338in}{1.012677in}}%
\pgfpathcurveto{\pgfqpoint{1.617151in}{1.020490in}}{\pgfqpoint{1.621542in}{1.031089in}}{\pgfqpoint{1.621542in}{1.042140in}}%
\pgfpathcurveto{\pgfqpoint{1.621542in}{1.053190in}}{\pgfqpoint{1.617151in}{1.063789in}}{\pgfqpoint{1.609338in}{1.071602in}}%
\pgfpathcurveto{\pgfqpoint{1.601524in}{1.079416in}}{\pgfqpoint{1.590925in}{1.083806in}}{\pgfqpoint{1.579875in}{1.083806in}}%
\pgfpathcurveto{\pgfqpoint{1.568825in}{1.083806in}}{\pgfqpoint{1.558226in}{1.079416in}}{\pgfqpoint{1.550412in}{1.071602in}}%
\pgfpathcurveto{\pgfqpoint{1.542599in}{1.063789in}}{\pgfqpoint{1.538208in}{1.053190in}}{\pgfqpoint{1.538208in}{1.042140in}}%
\pgfpathcurveto{\pgfqpoint{1.538208in}{1.031089in}}{\pgfqpoint{1.542599in}{1.020490in}}{\pgfqpoint{1.550412in}{1.012677in}}%
\pgfpathcurveto{\pgfqpoint{1.558226in}{1.004863in}}{\pgfqpoint{1.568825in}{1.000473in}}{\pgfqpoint{1.579875in}{1.000473in}}%
\pgfpathclose%
\pgfusepath{stroke,fill}%
\end{pgfscope}%
\begin{pgfscope}%
\pgfpathrectangle{\pgfqpoint{0.375000in}{0.330000in}}{\pgfqpoint{2.325000in}{2.310000in}}%
\pgfusepath{clip}%
\pgfsetbuttcap%
\pgfsetroundjoin%
\definecolor{currentfill}{rgb}{0.000000,0.000000,0.000000}%
\pgfsetfillcolor{currentfill}%
\pgfsetlinewidth{1.003750pt}%
\definecolor{currentstroke}{rgb}{0.000000,0.000000,0.000000}%
\pgfsetstrokecolor{currentstroke}%
\pgfsetdash{}{0pt}%
\pgfpathmoveto{\pgfqpoint{1.579875in}{1.085426in}}%
\pgfpathcurveto{\pgfqpoint{1.590925in}{1.085426in}}{\pgfqpoint{1.601524in}{1.089817in}}{\pgfqpoint{1.609338in}{1.097630in}}%
\pgfpathcurveto{\pgfqpoint{1.617151in}{1.105444in}}{\pgfqpoint{1.621542in}{1.116043in}}{\pgfqpoint{1.621542in}{1.127093in}}%
\pgfpathcurveto{\pgfqpoint{1.621542in}{1.138143in}}{\pgfqpoint{1.617151in}{1.148742in}}{\pgfqpoint{1.609338in}{1.156556in}}%
\pgfpathcurveto{\pgfqpoint{1.601524in}{1.164369in}}{\pgfqpoint{1.590925in}{1.168760in}}{\pgfqpoint{1.579875in}{1.168760in}}%
\pgfpathcurveto{\pgfqpoint{1.568825in}{1.168760in}}{\pgfqpoint{1.558226in}{1.164369in}}{\pgfqpoint{1.550412in}{1.156556in}}%
\pgfpathcurveto{\pgfqpoint{1.542599in}{1.148742in}}{\pgfqpoint{1.538208in}{1.138143in}}{\pgfqpoint{1.538208in}{1.127093in}}%
\pgfpathcurveto{\pgfqpoint{1.538208in}{1.116043in}}{\pgfqpoint{1.542599in}{1.105444in}}{\pgfqpoint{1.550412in}{1.097630in}}%
\pgfpathcurveto{\pgfqpoint{1.558226in}{1.089817in}}{\pgfqpoint{1.568825in}{1.085426in}}{\pgfqpoint{1.579875in}{1.085426in}}%
\pgfpathclose%
\pgfusepath{stroke,fill}%
\end{pgfscope}%
\begin{pgfscope}%
\pgfpathrectangle{\pgfqpoint{0.375000in}{0.330000in}}{\pgfqpoint{2.325000in}{2.310000in}}%
\pgfusepath{clip}%
\pgfsetbuttcap%
\pgfsetroundjoin%
\definecolor{currentfill}{rgb}{0.000000,0.000000,0.000000}%
\pgfsetfillcolor{currentfill}%
\pgfsetlinewidth{1.003750pt}%
\definecolor{currentstroke}{rgb}{0.000000,0.000000,0.000000}%
\pgfsetstrokecolor{currentstroke}%
\pgfsetdash{}{0pt}%
\pgfpathmoveto{\pgfqpoint{1.579875in}{0.970132in}}%
\pgfpathcurveto{\pgfqpoint{1.590925in}{0.970132in}}{\pgfqpoint{1.601524in}{0.974523in}}{\pgfqpoint{1.609338in}{0.982336in}}%
\pgfpathcurveto{\pgfqpoint{1.617151in}{0.990150in}}{\pgfqpoint{1.621542in}{1.000749in}}{\pgfqpoint{1.621542in}{1.011799in}}%
\pgfpathcurveto{\pgfqpoint{1.621542in}{1.022849in}}{\pgfqpoint{1.617151in}{1.033448in}}{\pgfqpoint{1.609338in}{1.041262in}}%
\pgfpathcurveto{\pgfqpoint{1.601524in}{1.049075in}}{\pgfqpoint{1.590925in}{1.053466in}}{\pgfqpoint{1.579875in}{1.053466in}}%
\pgfpathcurveto{\pgfqpoint{1.568825in}{1.053466in}}{\pgfqpoint{1.558226in}{1.049075in}}{\pgfqpoint{1.550412in}{1.041262in}}%
\pgfpathcurveto{\pgfqpoint{1.542599in}{1.033448in}}{\pgfqpoint{1.538208in}{1.022849in}}{\pgfqpoint{1.538208in}{1.011799in}}%
\pgfpathcurveto{\pgfqpoint{1.538208in}{1.000749in}}{\pgfqpoint{1.542599in}{0.990150in}}{\pgfqpoint{1.550412in}{0.982336in}}%
\pgfpathcurveto{\pgfqpoint{1.558226in}{0.974523in}}{\pgfqpoint{1.568825in}{0.970132in}}{\pgfqpoint{1.579875in}{0.970132in}}%
\pgfpathclose%
\pgfusepath{stroke,fill}%
\end{pgfscope}%
\begin{pgfscope}%
\pgfpathrectangle{\pgfqpoint{0.375000in}{0.330000in}}{\pgfqpoint{2.325000in}{2.310000in}}%
\pgfusepath{clip}%
\pgfsetbuttcap%
\pgfsetroundjoin%
\definecolor{currentfill}{rgb}{0.000000,0.000000,0.000000}%
\pgfsetfillcolor{currentfill}%
\pgfsetlinewidth{1.003750pt}%
\definecolor{currentstroke}{rgb}{0.000000,0.000000,0.000000}%
\pgfsetstrokecolor{currentstroke}%
\pgfsetdash{}{0pt}%
\pgfpathmoveto{\pgfqpoint{1.579875in}{0.994405in}}%
\pgfpathcurveto{\pgfqpoint{1.590925in}{0.994405in}}{\pgfqpoint{1.601524in}{0.998795in}}{\pgfqpoint{1.609338in}{1.006609in}}%
\pgfpathcurveto{\pgfqpoint{1.617151in}{1.014422in}}{\pgfqpoint{1.621542in}{1.025021in}}{\pgfqpoint{1.621542in}{1.036071in}}%
\pgfpathcurveto{\pgfqpoint{1.621542in}{1.047122in}}{\pgfqpoint{1.617151in}{1.057721in}}{\pgfqpoint{1.609338in}{1.065534in}}%
\pgfpathcurveto{\pgfqpoint{1.601524in}{1.073348in}}{\pgfqpoint{1.590925in}{1.077738in}}{\pgfqpoint{1.579875in}{1.077738in}}%
\pgfpathcurveto{\pgfqpoint{1.568825in}{1.077738in}}{\pgfqpoint{1.558226in}{1.073348in}}{\pgfqpoint{1.550412in}{1.065534in}}%
\pgfpathcurveto{\pgfqpoint{1.542599in}{1.057721in}}{\pgfqpoint{1.538208in}{1.047122in}}{\pgfqpoint{1.538208in}{1.036071in}}%
\pgfpathcurveto{\pgfqpoint{1.538208in}{1.025021in}}{\pgfqpoint{1.542599in}{1.014422in}}{\pgfqpoint{1.550412in}{1.006609in}}%
\pgfpathcurveto{\pgfqpoint{1.558226in}{0.998795in}}{\pgfqpoint{1.568825in}{0.994405in}}{\pgfqpoint{1.579875in}{0.994405in}}%
\pgfpathclose%
\pgfusepath{stroke,fill}%
\end{pgfscope}%
\begin{pgfscope}%
\pgfpathrectangle{\pgfqpoint{0.375000in}{0.330000in}}{\pgfqpoint{2.325000in}{2.310000in}}%
\pgfusepath{clip}%
\pgfsetbuttcap%
\pgfsetroundjoin%
\definecolor{currentfill}{rgb}{0.000000,0.000000,0.000000}%
\pgfsetfillcolor{currentfill}%
\pgfsetlinewidth{1.003750pt}%
\definecolor{currentstroke}{rgb}{0.000000,0.000000,0.000000}%
\pgfsetstrokecolor{currentstroke}%
\pgfsetdash{}{0pt}%
\pgfpathmoveto{\pgfqpoint{1.579875in}{1.055086in}}%
\pgfpathcurveto{\pgfqpoint{1.590925in}{1.055086in}}{\pgfqpoint{1.601524in}{1.059476in}}{\pgfqpoint{1.609338in}{1.067290in}}%
\pgfpathcurveto{\pgfqpoint{1.617151in}{1.075103in}}{\pgfqpoint{1.621542in}{1.085702in}}{\pgfqpoint{1.621542in}{1.096753in}}%
\pgfpathcurveto{\pgfqpoint{1.621542in}{1.107803in}}{\pgfqpoint{1.617151in}{1.118402in}}{\pgfqpoint{1.609338in}{1.126215in}}%
\pgfpathcurveto{\pgfqpoint{1.601524in}{1.134029in}}{\pgfqpoint{1.590925in}{1.138419in}}{\pgfqpoint{1.579875in}{1.138419in}}%
\pgfpathcurveto{\pgfqpoint{1.568825in}{1.138419in}}{\pgfqpoint{1.558226in}{1.134029in}}{\pgfqpoint{1.550412in}{1.126215in}}%
\pgfpathcurveto{\pgfqpoint{1.542599in}{1.118402in}}{\pgfqpoint{1.538208in}{1.107803in}}{\pgfqpoint{1.538208in}{1.096753in}}%
\pgfpathcurveto{\pgfqpoint{1.538208in}{1.085702in}}{\pgfqpoint{1.542599in}{1.075103in}}{\pgfqpoint{1.550412in}{1.067290in}}%
\pgfpathcurveto{\pgfqpoint{1.558226in}{1.059476in}}{\pgfqpoint{1.568825in}{1.055086in}}{\pgfqpoint{1.579875in}{1.055086in}}%
\pgfpathclose%
\pgfusepath{stroke,fill}%
\end{pgfscope}%
\begin{pgfscope}%
\pgfpathrectangle{\pgfqpoint{0.375000in}{0.330000in}}{\pgfqpoint{2.325000in}{2.310000in}}%
\pgfusepath{clip}%
\pgfsetbuttcap%
\pgfsetroundjoin%
\definecolor{currentfill}{rgb}{0.000000,0.000000,0.000000}%
\pgfsetfillcolor{currentfill}%
\pgfsetlinewidth{1.003750pt}%
\definecolor{currentstroke}{rgb}{0.000000,0.000000,0.000000}%
\pgfsetstrokecolor{currentstroke}%
\pgfsetdash{}{0pt}%
\pgfpathmoveto{\pgfqpoint{1.579875in}{1.042950in}}%
\pgfpathcurveto{\pgfqpoint{1.590925in}{1.042950in}}{\pgfqpoint{1.601524in}{1.047340in}}{\pgfqpoint{1.609338in}{1.055154in}}%
\pgfpathcurveto{\pgfqpoint{1.617151in}{1.062967in}}{\pgfqpoint{1.621542in}{1.073566in}}{\pgfqpoint{1.621542in}{1.084616in}}%
\pgfpathcurveto{\pgfqpoint{1.621542in}{1.095666in}}{\pgfqpoint{1.617151in}{1.106265in}}{\pgfqpoint{1.609338in}{1.114079in}}%
\pgfpathcurveto{\pgfqpoint{1.601524in}{1.121893in}}{\pgfqpoint{1.590925in}{1.126283in}}{\pgfqpoint{1.579875in}{1.126283in}}%
\pgfpathcurveto{\pgfqpoint{1.568825in}{1.126283in}}{\pgfqpoint{1.558226in}{1.121893in}}{\pgfqpoint{1.550412in}{1.114079in}}%
\pgfpathcurveto{\pgfqpoint{1.542599in}{1.106265in}}{\pgfqpoint{1.538208in}{1.095666in}}{\pgfqpoint{1.538208in}{1.084616in}}%
\pgfpathcurveto{\pgfqpoint{1.538208in}{1.073566in}}{\pgfqpoint{1.542599in}{1.062967in}}{\pgfqpoint{1.550412in}{1.055154in}}%
\pgfpathcurveto{\pgfqpoint{1.558226in}{1.047340in}}{\pgfqpoint{1.568825in}{1.042950in}}{\pgfqpoint{1.579875in}{1.042950in}}%
\pgfpathclose%
\pgfusepath{stroke,fill}%
\end{pgfscope}%
\begin{pgfscope}%
\pgfpathrectangle{\pgfqpoint{0.375000in}{0.330000in}}{\pgfqpoint{2.325000in}{2.310000in}}%
\pgfusepath{clip}%
\pgfsetbuttcap%
\pgfsetroundjoin%
\definecolor{currentfill}{rgb}{0.000000,0.000000,0.000000}%
\pgfsetfillcolor{currentfill}%
\pgfsetlinewidth{1.003750pt}%
\definecolor{currentstroke}{rgb}{0.000000,0.000000,0.000000}%
\pgfsetstrokecolor{currentstroke}%
\pgfsetdash{}{0pt}%
\pgfpathmoveto{\pgfqpoint{1.579875in}{1.073290in}}%
\pgfpathcurveto{\pgfqpoint{1.590925in}{1.073290in}}{\pgfqpoint{1.601524in}{1.077680in}}{\pgfqpoint{1.609338in}{1.085494in}}%
\pgfpathcurveto{\pgfqpoint{1.617151in}{1.093308in}}{\pgfqpoint{1.621542in}{1.103907in}}{\pgfqpoint{1.621542in}{1.114957in}}%
\pgfpathcurveto{\pgfqpoint{1.621542in}{1.126007in}}{\pgfqpoint{1.617151in}{1.136606in}}{\pgfqpoint{1.609338in}{1.144420in}}%
\pgfpathcurveto{\pgfqpoint{1.601524in}{1.152233in}}{\pgfqpoint{1.590925in}{1.156624in}}{\pgfqpoint{1.579875in}{1.156624in}}%
\pgfpathcurveto{\pgfqpoint{1.568825in}{1.156624in}}{\pgfqpoint{1.558226in}{1.152233in}}{\pgfqpoint{1.550412in}{1.144420in}}%
\pgfpathcurveto{\pgfqpoint{1.542599in}{1.136606in}}{\pgfqpoint{1.538208in}{1.126007in}}{\pgfqpoint{1.538208in}{1.114957in}}%
\pgfpathcurveto{\pgfqpoint{1.538208in}{1.103907in}}{\pgfqpoint{1.542599in}{1.093308in}}{\pgfqpoint{1.550412in}{1.085494in}}%
\pgfpathcurveto{\pgfqpoint{1.558226in}{1.077680in}}{\pgfqpoint{1.568825in}{1.073290in}}{\pgfqpoint{1.579875in}{1.073290in}}%
\pgfpathclose%
\pgfusepath{stroke,fill}%
\end{pgfscope}%
\begin{pgfscope}%
\pgfpathrectangle{\pgfqpoint{0.375000in}{0.330000in}}{\pgfqpoint{2.325000in}{2.310000in}}%
\pgfusepath{clip}%
\pgfsetbuttcap%
\pgfsetroundjoin%
\definecolor{currentfill}{rgb}{0.000000,0.000000,0.000000}%
\pgfsetfillcolor{currentfill}%
\pgfsetlinewidth{1.003750pt}%
\definecolor{currentstroke}{rgb}{0.000000,0.000000,0.000000}%
\pgfsetstrokecolor{currentstroke}%
\pgfsetdash{}{0pt}%
\pgfpathmoveto{\pgfqpoint{1.579875in}{0.970132in}}%
\pgfpathcurveto{\pgfqpoint{1.590925in}{0.970132in}}{\pgfqpoint{1.601524in}{0.974523in}}{\pgfqpoint{1.609338in}{0.982336in}}%
\pgfpathcurveto{\pgfqpoint{1.617151in}{0.990150in}}{\pgfqpoint{1.621542in}{1.000749in}}{\pgfqpoint{1.621542in}{1.011799in}}%
\pgfpathcurveto{\pgfqpoint{1.621542in}{1.022849in}}{\pgfqpoint{1.617151in}{1.033448in}}{\pgfqpoint{1.609338in}{1.041262in}}%
\pgfpathcurveto{\pgfqpoint{1.601524in}{1.049075in}}{\pgfqpoint{1.590925in}{1.053466in}}{\pgfqpoint{1.579875in}{1.053466in}}%
\pgfpathcurveto{\pgfqpoint{1.568825in}{1.053466in}}{\pgfqpoint{1.558226in}{1.049075in}}{\pgfqpoint{1.550412in}{1.041262in}}%
\pgfpathcurveto{\pgfqpoint{1.542599in}{1.033448in}}{\pgfqpoint{1.538208in}{1.022849in}}{\pgfqpoint{1.538208in}{1.011799in}}%
\pgfpathcurveto{\pgfqpoint{1.538208in}{1.000749in}}{\pgfqpoint{1.542599in}{0.990150in}}{\pgfqpoint{1.550412in}{0.982336in}}%
\pgfpathcurveto{\pgfqpoint{1.558226in}{0.974523in}}{\pgfqpoint{1.568825in}{0.970132in}}{\pgfqpoint{1.579875in}{0.970132in}}%
\pgfpathclose%
\pgfusepath{stroke,fill}%
\end{pgfscope}%
\begin{pgfscope}%
\pgfpathrectangle{\pgfqpoint{0.375000in}{0.330000in}}{\pgfqpoint{2.325000in}{2.310000in}}%
\pgfusepath{clip}%
\pgfsetbuttcap%
\pgfsetroundjoin%
\definecolor{currentfill}{rgb}{0.000000,0.000000,0.000000}%
\pgfsetfillcolor{currentfill}%
\pgfsetlinewidth{1.003750pt}%
\definecolor{currentstroke}{rgb}{0.000000,0.000000,0.000000}%
\pgfsetstrokecolor{currentstroke}%
\pgfsetdash{}{0pt}%
\pgfpathmoveto{\pgfqpoint{1.579875in}{1.055086in}}%
\pgfpathcurveto{\pgfqpoint{1.590925in}{1.055086in}}{\pgfqpoint{1.601524in}{1.059476in}}{\pgfqpoint{1.609338in}{1.067290in}}%
\pgfpathcurveto{\pgfqpoint{1.617151in}{1.075103in}}{\pgfqpoint{1.621542in}{1.085702in}}{\pgfqpoint{1.621542in}{1.096753in}}%
\pgfpathcurveto{\pgfqpoint{1.621542in}{1.107803in}}{\pgfqpoint{1.617151in}{1.118402in}}{\pgfqpoint{1.609338in}{1.126215in}}%
\pgfpathcurveto{\pgfqpoint{1.601524in}{1.134029in}}{\pgfqpoint{1.590925in}{1.138419in}}{\pgfqpoint{1.579875in}{1.138419in}}%
\pgfpathcurveto{\pgfqpoint{1.568825in}{1.138419in}}{\pgfqpoint{1.558226in}{1.134029in}}{\pgfqpoint{1.550412in}{1.126215in}}%
\pgfpathcurveto{\pgfqpoint{1.542599in}{1.118402in}}{\pgfqpoint{1.538208in}{1.107803in}}{\pgfqpoint{1.538208in}{1.096753in}}%
\pgfpathcurveto{\pgfqpoint{1.538208in}{1.085702in}}{\pgfqpoint{1.542599in}{1.075103in}}{\pgfqpoint{1.550412in}{1.067290in}}%
\pgfpathcurveto{\pgfqpoint{1.558226in}{1.059476in}}{\pgfqpoint{1.568825in}{1.055086in}}{\pgfqpoint{1.579875in}{1.055086in}}%
\pgfpathclose%
\pgfusepath{stroke,fill}%
\end{pgfscope}%
\begin{pgfscope}%
\pgfpathrectangle{\pgfqpoint{0.375000in}{0.330000in}}{\pgfqpoint{2.325000in}{2.310000in}}%
\pgfusepath{clip}%
\pgfsetbuttcap%
\pgfsetroundjoin%
\definecolor{currentfill}{rgb}{0.000000,0.000000,0.000000}%
\pgfsetfillcolor{currentfill}%
\pgfsetlinewidth{1.003750pt}%
\definecolor{currentstroke}{rgb}{0.000000,0.000000,0.000000}%
\pgfsetstrokecolor{currentstroke}%
\pgfsetdash{}{0pt}%
\pgfpathmoveto{\pgfqpoint{1.579875in}{0.994405in}}%
\pgfpathcurveto{\pgfqpoint{1.590925in}{0.994405in}}{\pgfqpoint{1.601524in}{0.998795in}}{\pgfqpoint{1.609338in}{1.006609in}}%
\pgfpathcurveto{\pgfqpoint{1.617151in}{1.014422in}}{\pgfqpoint{1.621542in}{1.025021in}}{\pgfqpoint{1.621542in}{1.036071in}}%
\pgfpathcurveto{\pgfqpoint{1.621542in}{1.047122in}}{\pgfqpoint{1.617151in}{1.057721in}}{\pgfqpoint{1.609338in}{1.065534in}}%
\pgfpathcurveto{\pgfqpoint{1.601524in}{1.073348in}}{\pgfqpoint{1.590925in}{1.077738in}}{\pgfqpoint{1.579875in}{1.077738in}}%
\pgfpathcurveto{\pgfqpoint{1.568825in}{1.077738in}}{\pgfqpoint{1.558226in}{1.073348in}}{\pgfqpoint{1.550412in}{1.065534in}}%
\pgfpathcurveto{\pgfqpoint{1.542599in}{1.057721in}}{\pgfqpoint{1.538208in}{1.047122in}}{\pgfqpoint{1.538208in}{1.036071in}}%
\pgfpathcurveto{\pgfqpoint{1.538208in}{1.025021in}}{\pgfqpoint{1.542599in}{1.014422in}}{\pgfqpoint{1.550412in}{1.006609in}}%
\pgfpathcurveto{\pgfqpoint{1.558226in}{0.998795in}}{\pgfqpoint{1.568825in}{0.994405in}}{\pgfqpoint{1.579875in}{0.994405in}}%
\pgfpathclose%
\pgfusepath{stroke,fill}%
\end{pgfscope}%
\begin{pgfscope}%
\pgfpathrectangle{\pgfqpoint{0.375000in}{0.330000in}}{\pgfqpoint{2.325000in}{2.310000in}}%
\pgfusepath{clip}%
\pgfsetbuttcap%
\pgfsetroundjoin%
\definecolor{currentfill}{rgb}{0.000000,0.000000,0.000000}%
\pgfsetfillcolor{currentfill}%
\pgfsetlinewidth{1.003750pt}%
\definecolor{currentstroke}{rgb}{0.000000,0.000000,0.000000}%
\pgfsetstrokecolor{currentstroke}%
\pgfsetdash{}{0pt}%
\pgfpathmoveto{\pgfqpoint{1.579875in}{1.024745in}}%
\pgfpathcurveto{\pgfqpoint{1.590925in}{1.024745in}}{\pgfqpoint{1.601524in}{1.029136in}}{\pgfqpoint{1.609338in}{1.036949in}}%
\pgfpathcurveto{\pgfqpoint{1.617151in}{1.044763in}}{\pgfqpoint{1.621542in}{1.055362in}}{\pgfqpoint{1.621542in}{1.066412in}}%
\pgfpathcurveto{\pgfqpoint{1.621542in}{1.077462in}}{\pgfqpoint{1.617151in}{1.088061in}}{\pgfqpoint{1.609338in}{1.095875in}}%
\pgfpathcurveto{\pgfqpoint{1.601524in}{1.103688in}}{\pgfqpoint{1.590925in}{1.108079in}}{\pgfqpoint{1.579875in}{1.108079in}}%
\pgfpathcurveto{\pgfqpoint{1.568825in}{1.108079in}}{\pgfqpoint{1.558226in}{1.103688in}}{\pgfqpoint{1.550412in}{1.095875in}}%
\pgfpathcurveto{\pgfqpoint{1.542599in}{1.088061in}}{\pgfqpoint{1.538208in}{1.077462in}}{\pgfqpoint{1.538208in}{1.066412in}}%
\pgfpathcurveto{\pgfqpoint{1.538208in}{1.055362in}}{\pgfqpoint{1.542599in}{1.044763in}}{\pgfqpoint{1.550412in}{1.036949in}}%
\pgfpathcurveto{\pgfqpoint{1.558226in}{1.029136in}}{\pgfqpoint{1.568825in}{1.024745in}}{\pgfqpoint{1.579875in}{1.024745in}}%
\pgfpathclose%
\pgfusepath{stroke,fill}%
\end{pgfscope}%
\begin{pgfscope}%
\pgfpathrectangle{\pgfqpoint{0.375000in}{0.330000in}}{\pgfqpoint{2.325000in}{2.310000in}}%
\pgfusepath{clip}%
\pgfsetbuttcap%
\pgfsetroundjoin%
\definecolor{currentfill}{rgb}{0.000000,0.000000,0.000000}%
\pgfsetfillcolor{currentfill}%
\pgfsetlinewidth{1.003750pt}%
\definecolor{currentstroke}{rgb}{0.000000,0.000000,0.000000}%
\pgfsetstrokecolor{currentstroke}%
\pgfsetdash{}{0pt}%
\pgfpathmoveto{\pgfqpoint{1.579875in}{1.073290in}}%
\pgfpathcurveto{\pgfqpoint{1.590925in}{1.073290in}}{\pgfqpoint{1.601524in}{1.077680in}}{\pgfqpoint{1.609338in}{1.085494in}}%
\pgfpathcurveto{\pgfqpoint{1.617151in}{1.093308in}}{\pgfqpoint{1.621542in}{1.103907in}}{\pgfqpoint{1.621542in}{1.114957in}}%
\pgfpathcurveto{\pgfqpoint{1.621542in}{1.126007in}}{\pgfqpoint{1.617151in}{1.136606in}}{\pgfqpoint{1.609338in}{1.144420in}}%
\pgfpathcurveto{\pgfqpoint{1.601524in}{1.152233in}}{\pgfqpoint{1.590925in}{1.156624in}}{\pgfqpoint{1.579875in}{1.156624in}}%
\pgfpathcurveto{\pgfqpoint{1.568825in}{1.156624in}}{\pgfqpoint{1.558226in}{1.152233in}}{\pgfqpoint{1.550412in}{1.144420in}}%
\pgfpathcurveto{\pgfqpoint{1.542599in}{1.136606in}}{\pgfqpoint{1.538208in}{1.126007in}}{\pgfqpoint{1.538208in}{1.114957in}}%
\pgfpathcurveto{\pgfqpoint{1.538208in}{1.103907in}}{\pgfqpoint{1.542599in}{1.093308in}}{\pgfqpoint{1.550412in}{1.085494in}}%
\pgfpathcurveto{\pgfqpoint{1.558226in}{1.077680in}}{\pgfqpoint{1.568825in}{1.073290in}}{\pgfqpoint{1.579875in}{1.073290in}}%
\pgfpathclose%
\pgfusepath{stroke,fill}%
\end{pgfscope}%
\begin{pgfscope}%
\pgfpathrectangle{\pgfqpoint{0.375000in}{0.330000in}}{\pgfqpoint{2.325000in}{2.310000in}}%
\pgfusepath{clip}%
\pgfsetbuttcap%
\pgfsetroundjoin%
\definecolor{currentfill}{rgb}{0.000000,0.000000,0.000000}%
\pgfsetfillcolor{currentfill}%
\pgfsetlinewidth{1.003750pt}%
\definecolor{currentstroke}{rgb}{0.000000,0.000000,0.000000}%
\pgfsetstrokecolor{currentstroke}%
\pgfsetdash{}{0pt}%
\pgfpathmoveto{\pgfqpoint{1.579875in}{1.049018in}}%
\pgfpathcurveto{\pgfqpoint{1.590925in}{1.049018in}}{\pgfqpoint{1.601524in}{1.053408in}}{\pgfqpoint{1.609338in}{1.061222in}}%
\pgfpathcurveto{\pgfqpoint{1.617151in}{1.069035in}}{\pgfqpoint{1.621542in}{1.079634in}}{\pgfqpoint{1.621542in}{1.090684in}}%
\pgfpathcurveto{\pgfqpoint{1.621542in}{1.101735in}}{\pgfqpoint{1.617151in}{1.112334in}}{\pgfqpoint{1.609338in}{1.120147in}}%
\pgfpathcurveto{\pgfqpoint{1.601524in}{1.127961in}}{\pgfqpoint{1.590925in}{1.132351in}}{\pgfqpoint{1.579875in}{1.132351in}}%
\pgfpathcurveto{\pgfqpoint{1.568825in}{1.132351in}}{\pgfqpoint{1.558226in}{1.127961in}}{\pgfqpoint{1.550412in}{1.120147in}}%
\pgfpathcurveto{\pgfqpoint{1.542599in}{1.112334in}}{\pgfqpoint{1.538208in}{1.101735in}}{\pgfqpoint{1.538208in}{1.090684in}}%
\pgfpathcurveto{\pgfqpoint{1.538208in}{1.079634in}}{\pgfqpoint{1.542599in}{1.069035in}}{\pgfqpoint{1.550412in}{1.061222in}}%
\pgfpathcurveto{\pgfqpoint{1.558226in}{1.053408in}}{\pgfqpoint{1.568825in}{1.049018in}}{\pgfqpoint{1.579875in}{1.049018in}}%
\pgfpathclose%
\pgfusepath{stroke,fill}%
\end{pgfscope}%
\begin{pgfscope}%
\pgfpathrectangle{\pgfqpoint{0.375000in}{0.330000in}}{\pgfqpoint{2.325000in}{2.310000in}}%
\pgfusepath{clip}%
\pgfsetbuttcap%
\pgfsetroundjoin%
\definecolor{currentfill}{rgb}{0.000000,0.000000,0.000000}%
\pgfsetfillcolor{currentfill}%
\pgfsetlinewidth{1.003750pt}%
\definecolor{currentstroke}{rgb}{0.000000,0.000000,0.000000}%
\pgfsetstrokecolor{currentstroke}%
\pgfsetdash{}{0pt}%
\pgfpathmoveto{\pgfqpoint{1.579875in}{0.982269in}}%
\pgfpathcurveto{\pgfqpoint{1.590925in}{0.982269in}}{\pgfqpoint{1.601524in}{0.986659in}}{\pgfqpoint{1.609338in}{0.994472in}}%
\pgfpathcurveto{\pgfqpoint{1.617151in}{1.002286in}}{\pgfqpoint{1.621542in}{1.012885in}}{\pgfqpoint{1.621542in}{1.023935in}}%
\pgfpathcurveto{\pgfqpoint{1.621542in}{1.034985in}}{\pgfqpoint{1.617151in}{1.045584in}}{\pgfqpoint{1.609338in}{1.053398in}}%
\pgfpathcurveto{\pgfqpoint{1.601524in}{1.061212in}}{\pgfqpoint{1.590925in}{1.065602in}}{\pgfqpoint{1.579875in}{1.065602in}}%
\pgfpathcurveto{\pgfqpoint{1.568825in}{1.065602in}}{\pgfqpoint{1.558226in}{1.061212in}}{\pgfqpoint{1.550412in}{1.053398in}}%
\pgfpathcurveto{\pgfqpoint{1.542599in}{1.045584in}}{\pgfqpoint{1.538208in}{1.034985in}}{\pgfqpoint{1.538208in}{1.023935in}}%
\pgfpathcurveto{\pgfqpoint{1.538208in}{1.012885in}}{\pgfqpoint{1.542599in}{1.002286in}}{\pgfqpoint{1.550412in}{0.994472in}}%
\pgfpathcurveto{\pgfqpoint{1.558226in}{0.986659in}}{\pgfqpoint{1.568825in}{0.982269in}}{\pgfqpoint{1.579875in}{0.982269in}}%
\pgfpathclose%
\pgfusepath{stroke,fill}%
\end{pgfscope}%
\begin{pgfscope}%
\pgfpathrectangle{\pgfqpoint{0.375000in}{0.330000in}}{\pgfqpoint{2.325000in}{2.310000in}}%
\pgfusepath{clip}%
\pgfsetbuttcap%
\pgfsetroundjoin%
\definecolor{currentfill}{rgb}{0.000000,0.000000,0.000000}%
\pgfsetfillcolor{currentfill}%
\pgfsetlinewidth{1.003750pt}%
\definecolor{currentstroke}{rgb}{0.000000,0.000000,0.000000}%
\pgfsetstrokecolor{currentstroke}%
\pgfsetdash{}{0pt}%
\pgfpathmoveto{\pgfqpoint{1.579875in}{1.224993in}}%
\pgfpathcurveto{\pgfqpoint{1.590925in}{1.224993in}}{\pgfqpoint{1.601524in}{1.229383in}}{\pgfqpoint{1.609338in}{1.237197in}}%
\pgfpathcurveto{\pgfqpoint{1.617151in}{1.245010in}}{\pgfqpoint{1.621542in}{1.255609in}}{\pgfqpoint{1.621542in}{1.266659in}}%
\pgfpathcurveto{\pgfqpoint{1.621542in}{1.277710in}}{\pgfqpoint{1.617151in}{1.288309in}}{\pgfqpoint{1.609338in}{1.296122in}}%
\pgfpathcurveto{\pgfqpoint{1.601524in}{1.303936in}}{\pgfqpoint{1.590925in}{1.308326in}}{\pgfqpoint{1.579875in}{1.308326in}}%
\pgfpathcurveto{\pgfqpoint{1.568825in}{1.308326in}}{\pgfqpoint{1.558226in}{1.303936in}}{\pgfqpoint{1.550412in}{1.296122in}}%
\pgfpathcurveto{\pgfqpoint{1.542599in}{1.288309in}}{\pgfqpoint{1.538208in}{1.277710in}}{\pgfqpoint{1.538208in}{1.266659in}}%
\pgfpathcurveto{\pgfqpoint{1.538208in}{1.255609in}}{\pgfqpoint{1.542599in}{1.245010in}}{\pgfqpoint{1.550412in}{1.237197in}}%
\pgfpathcurveto{\pgfqpoint{1.558226in}{1.229383in}}{\pgfqpoint{1.568825in}{1.224993in}}{\pgfqpoint{1.579875in}{1.224993in}}%
\pgfpathclose%
\pgfusepath{stroke,fill}%
\end{pgfscope}%
\begin{pgfscope}%
\pgfpathrectangle{\pgfqpoint{0.375000in}{0.330000in}}{\pgfqpoint{2.325000in}{2.310000in}}%
\pgfusepath{clip}%
\pgfsetbuttcap%
\pgfsetroundjoin%
\definecolor{currentfill}{rgb}{0.000000,0.000000,0.000000}%
\pgfsetfillcolor{currentfill}%
\pgfsetlinewidth{1.003750pt}%
\definecolor{currentstroke}{rgb}{0.000000,0.000000,0.000000}%
\pgfsetstrokecolor{currentstroke}%
\pgfsetdash{}{0pt}%
\pgfpathmoveto{\pgfqpoint{1.579875in}{1.085426in}}%
\pgfpathcurveto{\pgfqpoint{1.590925in}{1.085426in}}{\pgfqpoint{1.601524in}{1.089817in}}{\pgfqpoint{1.609338in}{1.097630in}}%
\pgfpathcurveto{\pgfqpoint{1.617151in}{1.105444in}}{\pgfqpoint{1.621542in}{1.116043in}}{\pgfqpoint{1.621542in}{1.127093in}}%
\pgfpathcurveto{\pgfqpoint{1.621542in}{1.138143in}}{\pgfqpoint{1.617151in}{1.148742in}}{\pgfqpoint{1.609338in}{1.156556in}}%
\pgfpathcurveto{\pgfqpoint{1.601524in}{1.164369in}}{\pgfqpoint{1.590925in}{1.168760in}}{\pgfqpoint{1.579875in}{1.168760in}}%
\pgfpathcurveto{\pgfqpoint{1.568825in}{1.168760in}}{\pgfqpoint{1.558226in}{1.164369in}}{\pgfqpoint{1.550412in}{1.156556in}}%
\pgfpathcurveto{\pgfqpoint{1.542599in}{1.148742in}}{\pgfqpoint{1.538208in}{1.138143in}}{\pgfqpoint{1.538208in}{1.127093in}}%
\pgfpathcurveto{\pgfqpoint{1.538208in}{1.116043in}}{\pgfqpoint{1.542599in}{1.105444in}}{\pgfqpoint{1.550412in}{1.097630in}}%
\pgfpathcurveto{\pgfqpoint{1.558226in}{1.089817in}}{\pgfqpoint{1.568825in}{1.085426in}}{\pgfqpoint{1.579875in}{1.085426in}}%
\pgfpathclose%
\pgfusepath{stroke,fill}%
\end{pgfscope}%
\begin{pgfscope}%
\pgfpathrectangle{\pgfqpoint{0.375000in}{0.330000in}}{\pgfqpoint{2.325000in}{2.310000in}}%
\pgfusepath{clip}%
\pgfsetbuttcap%
\pgfsetroundjoin%
\definecolor{currentfill}{rgb}{0.000000,0.000000,0.000000}%
\pgfsetfillcolor{currentfill}%
\pgfsetlinewidth{1.003750pt}%
\definecolor{currentstroke}{rgb}{0.000000,0.000000,0.000000}%
\pgfsetstrokecolor{currentstroke}%
\pgfsetdash{}{0pt}%
\pgfpathmoveto{\pgfqpoint{1.579875in}{0.933724in}}%
\pgfpathcurveto{\pgfqpoint{1.590925in}{0.933724in}}{\pgfqpoint{1.601524in}{0.938114in}}{\pgfqpoint{1.609338in}{0.945928in}}%
\pgfpathcurveto{\pgfqpoint{1.617151in}{0.953741in}}{\pgfqpoint{1.621542in}{0.964340in}}{\pgfqpoint{1.621542in}{0.975390in}}%
\pgfpathcurveto{\pgfqpoint{1.621542in}{0.986441in}}{\pgfqpoint{1.617151in}{0.997040in}}{\pgfqpoint{1.609338in}{1.004853in}}%
\pgfpathcurveto{\pgfqpoint{1.601524in}{1.012667in}}{\pgfqpoint{1.590925in}{1.017057in}}{\pgfqpoint{1.579875in}{1.017057in}}%
\pgfpathcurveto{\pgfqpoint{1.568825in}{1.017057in}}{\pgfqpoint{1.558226in}{1.012667in}}{\pgfqpoint{1.550412in}{1.004853in}}%
\pgfpathcurveto{\pgfqpoint{1.542599in}{0.997040in}}{\pgfqpoint{1.538208in}{0.986441in}}{\pgfqpoint{1.538208in}{0.975390in}}%
\pgfpathcurveto{\pgfqpoint{1.538208in}{0.964340in}}{\pgfqpoint{1.542599in}{0.953741in}}{\pgfqpoint{1.550412in}{0.945928in}}%
\pgfpathcurveto{\pgfqpoint{1.558226in}{0.938114in}}{\pgfqpoint{1.568825in}{0.933724in}}{\pgfqpoint{1.579875in}{0.933724in}}%
\pgfpathclose%
\pgfusepath{stroke,fill}%
\end{pgfscope}%
\begin{pgfscope}%
\pgfpathrectangle{\pgfqpoint{0.375000in}{0.330000in}}{\pgfqpoint{2.325000in}{2.310000in}}%
\pgfusepath{clip}%
\pgfsetbuttcap%
\pgfsetroundjoin%
\definecolor{currentfill}{rgb}{0.000000,0.000000,0.000000}%
\pgfsetfillcolor{currentfill}%
\pgfsetlinewidth{1.003750pt}%
\definecolor{currentstroke}{rgb}{0.000000,0.000000,0.000000}%
\pgfsetstrokecolor{currentstroke}%
\pgfsetdash{}{0pt}%
\pgfpathmoveto{\pgfqpoint{1.579875in}{1.158244in}}%
\pgfpathcurveto{\pgfqpoint{1.590925in}{1.158244in}}{\pgfqpoint{1.601524in}{1.162634in}}{\pgfqpoint{1.609338in}{1.170448in}}%
\pgfpathcurveto{\pgfqpoint{1.617151in}{1.178261in}}{\pgfqpoint{1.621542in}{1.188860in}}{\pgfqpoint{1.621542in}{1.199910in}}%
\pgfpathcurveto{\pgfqpoint{1.621542in}{1.210960in}}{\pgfqpoint{1.617151in}{1.221559in}}{\pgfqpoint{1.609338in}{1.229373in}}%
\pgfpathcurveto{\pgfqpoint{1.601524in}{1.237187in}}{\pgfqpoint{1.590925in}{1.241577in}}{\pgfqpoint{1.579875in}{1.241577in}}%
\pgfpathcurveto{\pgfqpoint{1.568825in}{1.241577in}}{\pgfqpoint{1.558226in}{1.237187in}}{\pgfqpoint{1.550412in}{1.229373in}}%
\pgfpathcurveto{\pgfqpoint{1.542599in}{1.221559in}}{\pgfqpoint{1.538208in}{1.210960in}}{\pgfqpoint{1.538208in}{1.199910in}}%
\pgfpathcurveto{\pgfqpoint{1.538208in}{1.188860in}}{\pgfqpoint{1.542599in}{1.178261in}}{\pgfqpoint{1.550412in}{1.170448in}}%
\pgfpathcurveto{\pgfqpoint{1.558226in}{1.162634in}}{\pgfqpoint{1.568825in}{1.158244in}}{\pgfqpoint{1.579875in}{1.158244in}}%
\pgfpathclose%
\pgfusepath{stroke,fill}%
\end{pgfscope}%
\begin{pgfscope}%
\pgfpathrectangle{\pgfqpoint{0.375000in}{0.330000in}}{\pgfqpoint{2.325000in}{2.310000in}}%
\pgfusepath{clip}%
\pgfsetbuttcap%
\pgfsetroundjoin%
\definecolor{currentfill}{rgb}{0.000000,0.000000,0.000000}%
\pgfsetfillcolor{currentfill}%
\pgfsetlinewidth{1.003750pt}%
\definecolor{currentstroke}{rgb}{0.000000,0.000000,0.000000}%
\pgfsetstrokecolor{currentstroke}%
\pgfsetdash{}{0pt}%
\pgfpathmoveto{\pgfqpoint{1.579875in}{1.024745in}}%
\pgfpathcurveto{\pgfqpoint{1.590925in}{1.024745in}}{\pgfqpoint{1.601524in}{1.029136in}}{\pgfqpoint{1.609338in}{1.036949in}}%
\pgfpathcurveto{\pgfqpoint{1.617151in}{1.044763in}}{\pgfqpoint{1.621542in}{1.055362in}}{\pgfqpoint{1.621542in}{1.066412in}}%
\pgfpathcurveto{\pgfqpoint{1.621542in}{1.077462in}}{\pgfqpoint{1.617151in}{1.088061in}}{\pgfqpoint{1.609338in}{1.095875in}}%
\pgfpathcurveto{\pgfqpoint{1.601524in}{1.103688in}}{\pgfqpoint{1.590925in}{1.108079in}}{\pgfqpoint{1.579875in}{1.108079in}}%
\pgfpathcurveto{\pgfqpoint{1.568825in}{1.108079in}}{\pgfqpoint{1.558226in}{1.103688in}}{\pgfqpoint{1.550412in}{1.095875in}}%
\pgfpathcurveto{\pgfqpoint{1.542599in}{1.088061in}}{\pgfqpoint{1.538208in}{1.077462in}}{\pgfqpoint{1.538208in}{1.066412in}}%
\pgfpathcurveto{\pgfqpoint{1.538208in}{1.055362in}}{\pgfqpoint{1.542599in}{1.044763in}}{\pgfqpoint{1.550412in}{1.036949in}}%
\pgfpathcurveto{\pgfqpoint{1.558226in}{1.029136in}}{\pgfqpoint{1.568825in}{1.024745in}}{\pgfqpoint{1.579875in}{1.024745in}}%
\pgfpathclose%
\pgfusepath{stroke,fill}%
\end{pgfscope}%
\begin{pgfscope}%
\pgfpathrectangle{\pgfqpoint{0.375000in}{0.330000in}}{\pgfqpoint{2.325000in}{2.310000in}}%
\pgfusepath{clip}%
\pgfsetbuttcap%
\pgfsetroundjoin%
\definecolor{currentfill}{rgb}{0.000000,0.000000,0.000000}%
\pgfsetfillcolor{currentfill}%
\pgfsetlinewidth{1.003750pt}%
\definecolor{currentstroke}{rgb}{0.000000,0.000000,0.000000}%
\pgfsetstrokecolor{currentstroke}%
\pgfsetdash{}{0pt}%
\pgfpathmoveto{\pgfqpoint{1.579875in}{1.176448in}}%
\pgfpathcurveto{\pgfqpoint{1.590925in}{1.176448in}}{\pgfqpoint{1.601524in}{1.180838in}}{\pgfqpoint{1.609338in}{1.188652in}}%
\pgfpathcurveto{\pgfqpoint{1.617151in}{1.196465in}}{\pgfqpoint{1.621542in}{1.207065in}}{\pgfqpoint{1.621542in}{1.218115in}}%
\pgfpathcurveto{\pgfqpoint{1.621542in}{1.229165in}}{\pgfqpoint{1.617151in}{1.239764in}}{\pgfqpoint{1.609338in}{1.247577in}}%
\pgfpathcurveto{\pgfqpoint{1.601524in}{1.255391in}}{\pgfqpoint{1.590925in}{1.259781in}}{\pgfqpoint{1.579875in}{1.259781in}}%
\pgfpathcurveto{\pgfqpoint{1.568825in}{1.259781in}}{\pgfqpoint{1.558226in}{1.255391in}}{\pgfqpoint{1.550412in}{1.247577in}}%
\pgfpathcurveto{\pgfqpoint{1.542599in}{1.239764in}}{\pgfqpoint{1.538208in}{1.229165in}}{\pgfqpoint{1.538208in}{1.218115in}}%
\pgfpathcurveto{\pgfqpoint{1.538208in}{1.207065in}}{\pgfqpoint{1.542599in}{1.196465in}}{\pgfqpoint{1.550412in}{1.188652in}}%
\pgfpathcurveto{\pgfqpoint{1.558226in}{1.180838in}}{\pgfqpoint{1.568825in}{1.176448in}}{\pgfqpoint{1.579875in}{1.176448in}}%
\pgfpathclose%
\pgfusepath{stroke,fill}%
\end{pgfscope}%
\begin{pgfscope}%
\pgfpathrectangle{\pgfqpoint{0.375000in}{0.330000in}}{\pgfqpoint{2.325000in}{2.310000in}}%
\pgfusepath{clip}%
\pgfsetbuttcap%
\pgfsetroundjoin%
\definecolor{currentfill}{rgb}{0.000000,0.000000,0.000000}%
\pgfsetfillcolor{currentfill}%
\pgfsetlinewidth{1.003750pt}%
\definecolor{currentstroke}{rgb}{0.000000,0.000000,0.000000}%
\pgfsetstrokecolor{currentstroke}%
\pgfsetdash{}{0pt}%
\pgfpathmoveto{\pgfqpoint{1.579875in}{1.170380in}}%
\pgfpathcurveto{\pgfqpoint{1.590925in}{1.170380in}}{\pgfqpoint{1.601524in}{1.174770in}}{\pgfqpoint{1.609338in}{1.182584in}}%
\pgfpathcurveto{\pgfqpoint{1.617151in}{1.190397in}}{\pgfqpoint{1.621542in}{1.200996in}}{\pgfqpoint{1.621542in}{1.212047in}}%
\pgfpathcurveto{\pgfqpoint{1.621542in}{1.223097in}}{\pgfqpoint{1.617151in}{1.233696in}}{\pgfqpoint{1.609338in}{1.241509in}}%
\pgfpathcurveto{\pgfqpoint{1.601524in}{1.249323in}}{\pgfqpoint{1.590925in}{1.253713in}}{\pgfqpoint{1.579875in}{1.253713in}}%
\pgfpathcurveto{\pgfqpoint{1.568825in}{1.253713in}}{\pgfqpoint{1.558226in}{1.249323in}}{\pgfqpoint{1.550412in}{1.241509in}}%
\pgfpathcurveto{\pgfqpoint{1.542599in}{1.233696in}}{\pgfqpoint{1.538208in}{1.223097in}}{\pgfqpoint{1.538208in}{1.212047in}}%
\pgfpathcurveto{\pgfqpoint{1.538208in}{1.200996in}}{\pgfqpoint{1.542599in}{1.190397in}}{\pgfqpoint{1.550412in}{1.182584in}}%
\pgfpathcurveto{\pgfqpoint{1.558226in}{1.174770in}}{\pgfqpoint{1.568825in}{1.170380in}}{\pgfqpoint{1.579875in}{1.170380in}}%
\pgfpathclose%
\pgfusepath{stroke,fill}%
\end{pgfscope}%
\begin{pgfscope}%
\pgfpathrectangle{\pgfqpoint{0.375000in}{0.330000in}}{\pgfqpoint{2.325000in}{2.310000in}}%
\pgfusepath{clip}%
\pgfsetbuttcap%
\pgfsetroundjoin%
\definecolor{currentfill}{rgb}{0.000000,0.000000,0.000000}%
\pgfsetfillcolor{currentfill}%
\pgfsetlinewidth{1.003750pt}%
\definecolor{currentstroke}{rgb}{0.000000,0.000000,0.000000}%
\pgfsetstrokecolor{currentstroke}%
\pgfsetdash{}{0pt}%
\pgfpathmoveto{\pgfqpoint{1.579875in}{0.976200in}}%
\pgfpathcurveto{\pgfqpoint{1.590925in}{0.976200in}}{\pgfqpoint{1.601524in}{0.980591in}}{\pgfqpoint{1.609338in}{0.988404in}}%
\pgfpathcurveto{\pgfqpoint{1.617151in}{0.996218in}}{\pgfqpoint{1.621542in}{1.006817in}}{\pgfqpoint{1.621542in}{1.017867in}}%
\pgfpathcurveto{\pgfqpoint{1.621542in}{1.028917in}}{\pgfqpoint{1.617151in}{1.039516in}}{\pgfqpoint{1.609338in}{1.047330in}}%
\pgfpathcurveto{\pgfqpoint{1.601524in}{1.055144in}}{\pgfqpoint{1.590925in}{1.059534in}}{\pgfqpoint{1.579875in}{1.059534in}}%
\pgfpathcurveto{\pgfqpoint{1.568825in}{1.059534in}}{\pgfqpoint{1.558226in}{1.055144in}}{\pgfqpoint{1.550412in}{1.047330in}}%
\pgfpathcurveto{\pgfqpoint{1.542599in}{1.039516in}}{\pgfqpoint{1.538208in}{1.028917in}}{\pgfqpoint{1.538208in}{1.017867in}}%
\pgfpathcurveto{\pgfqpoint{1.538208in}{1.006817in}}{\pgfqpoint{1.542599in}{0.996218in}}{\pgfqpoint{1.550412in}{0.988404in}}%
\pgfpathcurveto{\pgfqpoint{1.558226in}{0.980591in}}{\pgfqpoint{1.568825in}{0.976200in}}{\pgfqpoint{1.579875in}{0.976200in}}%
\pgfpathclose%
\pgfusepath{stroke,fill}%
\end{pgfscope}%
\begin{pgfscope}%
\pgfpathrectangle{\pgfqpoint{0.375000in}{0.330000in}}{\pgfqpoint{2.325000in}{2.310000in}}%
\pgfusepath{clip}%
\pgfsetbuttcap%
\pgfsetroundjoin%
\definecolor{currentfill}{rgb}{0.000000,0.000000,0.000000}%
\pgfsetfillcolor{currentfill}%
\pgfsetlinewidth{1.003750pt}%
\definecolor{currentstroke}{rgb}{0.000000,0.000000,0.000000}%
\pgfsetstrokecolor{currentstroke}%
\pgfsetdash{}{0pt}%
\pgfpathmoveto{\pgfqpoint{1.579875in}{0.957996in}}%
\pgfpathcurveto{\pgfqpoint{1.590925in}{0.957996in}}{\pgfqpoint{1.601524in}{0.962386in}}{\pgfqpoint{1.609338in}{0.970200in}}%
\pgfpathcurveto{\pgfqpoint{1.617151in}{0.978014in}}{\pgfqpoint{1.621542in}{0.988613in}}{\pgfqpoint{1.621542in}{0.999663in}}%
\pgfpathcurveto{\pgfqpoint{1.621542in}{1.010713in}}{\pgfqpoint{1.617151in}{1.021312in}}{\pgfqpoint{1.609338in}{1.029126in}}%
\pgfpathcurveto{\pgfqpoint{1.601524in}{1.036939in}}{\pgfqpoint{1.590925in}{1.041329in}}{\pgfqpoint{1.579875in}{1.041329in}}%
\pgfpathcurveto{\pgfqpoint{1.568825in}{1.041329in}}{\pgfqpoint{1.558226in}{1.036939in}}{\pgfqpoint{1.550412in}{1.029126in}}%
\pgfpathcurveto{\pgfqpoint{1.542599in}{1.021312in}}{\pgfqpoint{1.538208in}{1.010713in}}{\pgfqpoint{1.538208in}{0.999663in}}%
\pgfpathcurveto{\pgfqpoint{1.538208in}{0.988613in}}{\pgfqpoint{1.542599in}{0.978014in}}{\pgfqpoint{1.550412in}{0.970200in}}%
\pgfpathcurveto{\pgfqpoint{1.558226in}{0.962386in}}{\pgfqpoint{1.568825in}{0.957996in}}{\pgfqpoint{1.579875in}{0.957996in}}%
\pgfpathclose%
\pgfusepath{stroke,fill}%
\end{pgfscope}%
\begin{pgfscope}%
\pgfpathrectangle{\pgfqpoint{0.375000in}{0.330000in}}{\pgfqpoint{2.325000in}{2.310000in}}%
\pgfusepath{clip}%
\pgfsetbuttcap%
\pgfsetroundjoin%
\definecolor{currentfill}{rgb}{0.000000,0.000000,0.000000}%
\pgfsetfillcolor{currentfill}%
\pgfsetlinewidth{1.003750pt}%
\definecolor{currentstroke}{rgb}{0.000000,0.000000,0.000000}%
\pgfsetstrokecolor{currentstroke}%
\pgfsetdash{}{0pt}%
\pgfpathmoveto{\pgfqpoint{1.579875in}{1.152176in}}%
\pgfpathcurveto{\pgfqpoint{1.590925in}{1.152176in}}{\pgfqpoint{1.601524in}{1.156566in}}{\pgfqpoint{1.609338in}{1.164379in}}%
\pgfpathcurveto{\pgfqpoint{1.617151in}{1.172193in}}{\pgfqpoint{1.621542in}{1.182792in}}{\pgfqpoint{1.621542in}{1.193842in}}%
\pgfpathcurveto{\pgfqpoint{1.621542in}{1.204892in}}{\pgfqpoint{1.617151in}{1.215491in}}{\pgfqpoint{1.609338in}{1.223305in}}%
\pgfpathcurveto{\pgfqpoint{1.601524in}{1.231119in}}{\pgfqpoint{1.590925in}{1.235509in}}{\pgfqpoint{1.579875in}{1.235509in}}%
\pgfpathcurveto{\pgfqpoint{1.568825in}{1.235509in}}{\pgfqpoint{1.558226in}{1.231119in}}{\pgfqpoint{1.550412in}{1.223305in}}%
\pgfpathcurveto{\pgfqpoint{1.542599in}{1.215491in}}{\pgfqpoint{1.538208in}{1.204892in}}{\pgfqpoint{1.538208in}{1.193842in}}%
\pgfpathcurveto{\pgfqpoint{1.538208in}{1.182792in}}{\pgfqpoint{1.542599in}{1.172193in}}{\pgfqpoint{1.550412in}{1.164379in}}%
\pgfpathcurveto{\pgfqpoint{1.558226in}{1.156566in}}{\pgfqpoint{1.568825in}{1.152176in}}{\pgfqpoint{1.579875in}{1.152176in}}%
\pgfpathclose%
\pgfusepath{stroke,fill}%
\end{pgfscope}%
\begin{pgfscope}%
\pgfpathrectangle{\pgfqpoint{0.375000in}{0.330000in}}{\pgfqpoint{2.325000in}{2.310000in}}%
\pgfusepath{clip}%
\pgfsetbuttcap%
\pgfsetroundjoin%
\definecolor{currentfill}{rgb}{0.000000,0.000000,0.000000}%
\pgfsetfillcolor{currentfill}%
\pgfsetlinewidth{1.003750pt}%
\definecolor{currentstroke}{rgb}{0.000000,0.000000,0.000000}%
\pgfsetstrokecolor{currentstroke}%
\pgfsetdash{}{0pt}%
\pgfpathmoveto{\pgfqpoint{1.579875in}{1.267470in}}%
\pgfpathcurveto{\pgfqpoint{1.590925in}{1.267470in}}{\pgfqpoint{1.601524in}{1.271860in}}{\pgfqpoint{1.609338in}{1.279673in}}%
\pgfpathcurveto{\pgfqpoint{1.617151in}{1.287487in}}{\pgfqpoint{1.621542in}{1.298086in}}{\pgfqpoint{1.621542in}{1.309136in}}%
\pgfpathcurveto{\pgfqpoint{1.621542in}{1.320186in}}{\pgfqpoint{1.617151in}{1.330785in}}{\pgfqpoint{1.609338in}{1.338599in}}%
\pgfpathcurveto{\pgfqpoint{1.601524in}{1.346413in}}{\pgfqpoint{1.590925in}{1.350803in}}{\pgfqpoint{1.579875in}{1.350803in}}%
\pgfpathcurveto{\pgfqpoint{1.568825in}{1.350803in}}{\pgfqpoint{1.558226in}{1.346413in}}{\pgfqpoint{1.550412in}{1.338599in}}%
\pgfpathcurveto{\pgfqpoint{1.542599in}{1.330785in}}{\pgfqpoint{1.538208in}{1.320186in}}{\pgfqpoint{1.538208in}{1.309136in}}%
\pgfpathcurveto{\pgfqpoint{1.538208in}{1.298086in}}{\pgfqpoint{1.542599in}{1.287487in}}{\pgfqpoint{1.550412in}{1.279673in}}%
\pgfpathcurveto{\pgfqpoint{1.558226in}{1.271860in}}{\pgfqpoint{1.568825in}{1.267470in}}{\pgfqpoint{1.579875in}{1.267470in}}%
\pgfpathclose%
\pgfusepath{stroke,fill}%
\end{pgfscope}%
\begin{pgfscope}%
\pgfpathrectangle{\pgfqpoint{0.375000in}{0.330000in}}{\pgfqpoint{2.325000in}{2.310000in}}%
\pgfusepath{clip}%
\pgfsetbuttcap%
\pgfsetroundjoin%
\definecolor{currentfill}{rgb}{0.000000,0.000000,0.000000}%
\pgfsetfillcolor{currentfill}%
\pgfsetlinewidth{1.003750pt}%
\definecolor{currentstroke}{rgb}{0.000000,0.000000,0.000000}%
\pgfsetstrokecolor{currentstroke}%
\pgfsetdash{}{0pt}%
\pgfpathmoveto{\pgfqpoint{1.579875in}{0.976200in}}%
\pgfpathcurveto{\pgfqpoint{1.590925in}{0.976200in}}{\pgfqpoint{1.601524in}{0.980591in}}{\pgfqpoint{1.609338in}{0.988404in}}%
\pgfpathcurveto{\pgfqpoint{1.617151in}{0.996218in}}{\pgfqpoint{1.621542in}{1.006817in}}{\pgfqpoint{1.621542in}{1.017867in}}%
\pgfpathcurveto{\pgfqpoint{1.621542in}{1.028917in}}{\pgfqpoint{1.617151in}{1.039516in}}{\pgfqpoint{1.609338in}{1.047330in}}%
\pgfpathcurveto{\pgfqpoint{1.601524in}{1.055144in}}{\pgfqpoint{1.590925in}{1.059534in}}{\pgfqpoint{1.579875in}{1.059534in}}%
\pgfpathcurveto{\pgfqpoint{1.568825in}{1.059534in}}{\pgfqpoint{1.558226in}{1.055144in}}{\pgfqpoint{1.550412in}{1.047330in}}%
\pgfpathcurveto{\pgfqpoint{1.542599in}{1.039516in}}{\pgfqpoint{1.538208in}{1.028917in}}{\pgfqpoint{1.538208in}{1.017867in}}%
\pgfpathcurveto{\pgfqpoint{1.538208in}{1.006817in}}{\pgfqpoint{1.542599in}{0.996218in}}{\pgfqpoint{1.550412in}{0.988404in}}%
\pgfpathcurveto{\pgfqpoint{1.558226in}{0.980591in}}{\pgfqpoint{1.568825in}{0.976200in}}{\pgfqpoint{1.579875in}{0.976200in}}%
\pgfpathclose%
\pgfusepath{stroke,fill}%
\end{pgfscope}%
\begin{pgfscope}%
\pgfpathrectangle{\pgfqpoint{0.375000in}{0.330000in}}{\pgfqpoint{2.325000in}{2.310000in}}%
\pgfusepath{clip}%
\pgfsetbuttcap%
\pgfsetroundjoin%
\definecolor{currentfill}{rgb}{0.000000,0.000000,0.000000}%
\pgfsetfillcolor{currentfill}%
\pgfsetlinewidth{1.003750pt}%
\definecolor{currentstroke}{rgb}{0.000000,0.000000,0.000000}%
\pgfsetstrokecolor{currentstroke}%
\pgfsetdash{}{0pt}%
\pgfpathmoveto{\pgfqpoint{1.579875in}{0.951928in}}%
\pgfpathcurveto{\pgfqpoint{1.590925in}{0.951928in}}{\pgfqpoint{1.601524in}{0.956318in}}{\pgfqpoint{1.609338in}{0.964132in}}%
\pgfpathcurveto{\pgfqpoint{1.617151in}{0.971946in}}{\pgfqpoint{1.621542in}{0.982545in}}{\pgfqpoint{1.621542in}{0.993595in}}%
\pgfpathcurveto{\pgfqpoint{1.621542in}{1.004645in}}{\pgfqpoint{1.617151in}{1.015244in}}{\pgfqpoint{1.609338in}{1.023058in}}%
\pgfpathcurveto{\pgfqpoint{1.601524in}{1.030871in}}{\pgfqpoint{1.590925in}{1.035261in}}{\pgfqpoint{1.579875in}{1.035261in}}%
\pgfpathcurveto{\pgfqpoint{1.568825in}{1.035261in}}{\pgfqpoint{1.558226in}{1.030871in}}{\pgfqpoint{1.550412in}{1.023058in}}%
\pgfpathcurveto{\pgfqpoint{1.542599in}{1.015244in}}{\pgfqpoint{1.538208in}{1.004645in}}{\pgfqpoint{1.538208in}{0.993595in}}%
\pgfpathcurveto{\pgfqpoint{1.538208in}{0.982545in}}{\pgfqpoint{1.542599in}{0.971946in}}{\pgfqpoint{1.550412in}{0.964132in}}%
\pgfpathcurveto{\pgfqpoint{1.558226in}{0.956318in}}{\pgfqpoint{1.568825in}{0.951928in}}{\pgfqpoint{1.579875in}{0.951928in}}%
\pgfpathclose%
\pgfusepath{stroke,fill}%
\end{pgfscope}%
\begin{pgfscope}%
\pgfpathrectangle{\pgfqpoint{0.375000in}{0.330000in}}{\pgfqpoint{2.325000in}{2.310000in}}%
\pgfusepath{clip}%
\pgfsetbuttcap%
\pgfsetroundjoin%
\definecolor{currentfill}{rgb}{0.000000,0.000000,0.000000}%
\pgfsetfillcolor{currentfill}%
\pgfsetlinewidth{1.003750pt}%
\definecolor{currentstroke}{rgb}{0.000000,0.000000,0.000000}%
\pgfsetstrokecolor{currentstroke}%
\pgfsetdash{}{0pt}%
\pgfpathmoveto{\pgfqpoint{1.579875in}{1.346355in}}%
\pgfpathcurveto{\pgfqpoint{1.590925in}{1.346355in}}{\pgfqpoint{1.601524in}{1.350745in}}{\pgfqpoint{1.609338in}{1.358559in}}%
\pgfpathcurveto{\pgfqpoint{1.617151in}{1.366372in}}{\pgfqpoint{1.621542in}{1.376971in}}{\pgfqpoint{1.621542in}{1.388022in}}%
\pgfpathcurveto{\pgfqpoint{1.621542in}{1.399072in}}{\pgfqpoint{1.617151in}{1.409671in}}{\pgfqpoint{1.609338in}{1.417484in}}%
\pgfpathcurveto{\pgfqpoint{1.601524in}{1.425298in}}{\pgfqpoint{1.590925in}{1.429688in}}{\pgfqpoint{1.579875in}{1.429688in}}%
\pgfpathcurveto{\pgfqpoint{1.568825in}{1.429688in}}{\pgfqpoint{1.558226in}{1.425298in}}{\pgfqpoint{1.550412in}{1.417484in}}%
\pgfpathcurveto{\pgfqpoint{1.542599in}{1.409671in}}{\pgfqpoint{1.538208in}{1.399072in}}{\pgfqpoint{1.538208in}{1.388022in}}%
\pgfpathcurveto{\pgfqpoint{1.538208in}{1.376971in}}{\pgfqpoint{1.542599in}{1.366372in}}{\pgfqpoint{1.550412in}{1.358559in}}%
\pgfpathcurveto{\pgfqpoint{1.558226in}{1.350745in}}{\pgfqpoint{1.568825in}{1.346355in}}{\pgfqpoint{1.579875in}{1.346355in}}%
\pgfpathclose%
\pgfusepath{stroke,fill}%
\end{pgfscope}%
\begin{pgfscope}%
\pgfpathrectangle{\pgfqpoint{0.375000in}{0.330000in}}{\pgfqpoint{2.325000in}{2.310000in}}%
\pgfusepath{clip}%
\pgfsetbuttcap%
\pgfsetroundjoin%
\definecolor{currentfill}{rgb}{0.000000,0.000000,0.000000}%
\pgfsetfillcolor{currentfill}%
\pgfsetlinewidth{1.003750pt}%
\definecolor{currentstroke}{rgb}{0.000000,0.000000,0.000000}%
\pgfsetstrokecolor{currentstroke}%
\pgfsetdash{}{0pt}%
\pgfpathmoveto{\pgfqpoint{1.579875in}{0.994405in}}%
\pgfpathcurveto{\pgfqpoint{1.590925in}{0.994405in}}{\pgfqpoint{1.601524in}{0.998795in}}{\pgfqpoint{1.609338in}{1.006609in}}%
\pgfpathcurveto{\pgfqpoint{1.617151in}{1.014422in}}{\pgfqpoint{1.621542in}{1.025021in}}{\pgfqpoint{1.621542in}{1.036071in}}%
\pgfpathcurveto{\pgfqpoint{1.621542in}{1.047122in}}{\pgfqpoint{1.617151in}{1.057721in}}{\pgfqpoint{1.609338in}{1.065534in}}%
\pgfpathcurveto{\pgfqpoint{1.601524in}{1.073348in}}{\pgfqpoint{1.590925in}{1.077738in}}{\pgfqpoint{1.579875in}{1.077738in}}%
\pgfpathcurveto{\pgfqpoint{1.568825in}{1.077738in}}{\pgfqpoint{1.558226in}{1.073348in}}{\pgfqpoint{1.550412in}{1.065534in}}%
\pgfpathcurveto{\pgfqpoint{1.542599in}{1.057721in}}{\pgfqpoint{1.538208in}{1.047122in}}{\pgfqpoint{1.538208in}{1.036071in}}%
\pgfpathcurveto{\pgfqpoint{1.538208in}{1.025021in}}{\pgfqpoint{1.542599in}{1.014422in}}{\pgfqpoint{1.550412in}{1.006609in}}%
\pgfpathcurveto{\pgfqpoint{1.558226in}{0.998795in}}{\pgfqpoint{1.568825in}{0.994405in}}{\pgfqpoint{1.579875in}{0.994405in}}%
\pgfpathclose%
\pgfusepath{stroke,fill}%
\end{pgfscope}%
\begin{pgfscope}%
\pgfpathrectangle{\pgfqpoint{0.375000in}{0.330000in}}{\pgfqpoint{2.325000in}{2.310000in}}%
\pgfusepath{clip}%
\pgfsetbuttcap%
\pgfsetroundjoin%
\definecolor{currentfill}{rgb}{0.000000,0.000000,0.000000}%
\pgfsetfillcolor{currentfill}%
\pgfsetlinewidth{1.003750pt}%
\definecolor{currentstroke}{rgb}{0.000000,0.000000,0.000000}%
\pgfsetstrokecolor{currentstroke}%
\pgfsetdash{}{0pt}%
\pgfpathmoveto{\pgfqpoint{1.579875in}{1.097563in}}%
\pgfpathcurveto{\pgfqpoint{1.590925in}{1.097563in}}{\pgfqpoint{1.601524in}{1.101953in}}{\pgfqpoint{1.609338in}{1.109766in}}%
\pgfpathcurveto{\pgfqpoint{1.617151in}{1.117580in}}{\pgfqpoint{1.621542in}{1.128179in}}{\pgfqpoint{1.621542in}{1.139229in}}%
\pgfpathcurveto{\pgfqpoint{1.621542in}{1.150279in}}{\pgfqpoint{1.617151in}{1.160878in}}{\pgfqpoint{1.609338in}{1.168692in}}%
\pgfpathcurveto{\pgfqpoint{1.601524in}{1.176506in}}{\pgfqpoint{1.590925in}{1.180896in}}{\pgfqpoint{1.579875in}{1.180896in}}%
\pgfpathcurveto{\pgfqpoint{1.568825in}{1.180896in}}{\pgfqpoint{1.558226in}{1.176506in}}{\pgfqpoint{1.550412in}{1.168692in}}%
\pgfpathcurveto{\pgfqpoint{1.542599in}{1.160878in}}{\pgfqpoint{1.538208in}{1.150279in}}{\pgfqpoint{1.538208in}{1.139229in}}%
\pgfpathcurveto{\pgfqpoint{1.538208in}{1.128179in}}{\pgfqpoint{1.542599in}{1.117580in}}{\pgfqpoint{1.550412in}{1.109766in}}%
\pgfpathcurveto{\pgfqpoint{1.558226in}{1.101953in}}{\pgfqpoint{1.568825in}{1.097563in}}{\pgfqpoint{1.579875in}{1.097563in}}%
\pgfpathclose%
\pgfusepath{stroke,fill}%
\end{pgfscope}%
\begin{pgfscope}%
\pgfpathrectangle{\pgfqpoint{0.375000in}{0.330000in}}{\pgfqpoint{2.325000in}{2.310000in}}%
\pgfusepath{clip}%
\pgfsetbuttcap%
\pgfsetroundjoin%
\definecolor{currentfill}{rgb}{0.000000,0.000000,0.000000}%
\pgfsetfillcolor{currentfill}%
\pgfsetlinewidth{1.003750pt}%
\definecolor{currentstroke}{rgb}{0.000000,0.000000,0.000000}%
\pgfsetstrokecolor{currentstroke}%
\pgfsetdash{}{0pt}%
\pgfpathmoveto{\pgfqpoint{1.579875in}{1.012609in}}%
\pgfpathcurveto{\pgfqpoint{1.590925in}{1.012609in}}{\pgfqpoint{1.601524in}{1.016999in}}{\pgfqpoint{1.609338in}{1.024813in}}%
\pgfpathcurveto{\pgfqpoint{1.617151in}{1.032627in}}{\pgfqpoint{1.621542in}{1.043226in}}{\pgfqpoint{1.621542in}{1.054276in}}%
\pgfpathcurveto{\pgfqpoint{1.621542in}{1.065326in}}{\pgfqpoint{1.617151in}{1.075925in}}{\pgfqpoint{1.609338in}{1.083739in}}%
\pgfpathcurveto{\pgfqpoint{1.601524in}{1.091552in}}{\pgfqpoint{1.590925in}{1.095942in}}{\pgfqpoint{1.579875in}{1.095942in}}%
\pgfpathcurveto{\pgfqpoint{1.568825in}{1.095942in}}{\pgfqpoint{1.558226in}{1.091552in}}{\pgfqpoint{1.550412in}{1.083739in}}%
\pgfpathcurveto{\pgfqpoint{1.542599in}{1.075925in}}{\pgfqpoint{1.538208in}{1.065326in}}{\pgfqpoint{1.538208in}{1.054276in}}%
\pgfpathcurveto{\pgfqpoint{1.538208in}{1.043226in}}{\pgfqpoint{1.542599in}{1.032627in}}{\pgfqpoint{1.550412in}{1.024813in}}%
\pgfpathcurveto{\pgfqpoint{1.558226in}{1.016999in}}{\pgfqpoint{1.568825in}{1.012609in}}{\pgfqpoint{1.579875in}{1.012609in}}%
\pgfpathclose%
\pgfusepath{stroke,fill}%
\end{pgfscope}%
\begin{pgfscope}%
\pgfpathrectangle{\pgfqpoint{0.375000in}{0.330000in}}{\pgfqpoint{2.325000in}{2.310000in}}%
\pgfusepath{clip}%
\pgfsetbuttcap%
\pgfsetroundjoin%
\definecolor{currentfill}{rgb}{0.000000,0.000000,0.000000}%
\pgfsetfillcolor{currentfill}%
\pgfsetlinewidth{1.003750pt}%
\definecolor{currentstroke}{rgb}{0.000000,0.000000,0.000000}%
\pgfsetstrokecolor{currentstroke}%
\pgfsetdash{}{0pt}%
\pgfpathmoveto{\pgfqpoint{1.579875in}{1.073290in}}%
\pgfpathcurveto{\pgfqpoint{1.590925in}{1.073290in}}{\pgfqpoint{1.601524in}{1.077680in}}{\pgfqpoint{1.609338in}{1.085494in}}%
\pgfpathcurveto{\pgfqpoint{1.617151in}{1.093308in}}{\pgfqpoint{1.621542in}{1.103907in}}{\pgfqpoint{1.621542in}{1.114957in}}%
\pgfpathcurveto{\pgfqpoint{1.621542in}{1.126007in}}{\pgfqpoint{1.617151in}{1.136606in}}{\pgfqpoint{1.609338in}{1.144420in}}%
\pgfpathcurveto{\pgfqpoint{1.601524in}{1.152233in}}{\pgfqpoint{1.590925in}{1.156624in}}{\pgfqpoint{1.579875in}{1.156624in}}%
\pgfpathcurveto{\pgfqpoint{1.568825in}{1.156624in}}{\pgfqpoint{1.558226in}{1.152233in}}{\pgfqpoint{1.550412in}{1.144420in}}%
\pgfpathcurveto{\pgfqpoint{1.542599in}{1.136606in}}{\pgfqpoint{1.538208in}{1.126007in}}{\pgfqpoint{1.538208in}{1.114957in}}%
\pgfpathcurveto{\pgfqpoint{1.538208in}{1.103907in}}{\pgfqpoint{1.542599in}{1.093308in}}{\pgfqpoint{1.550412in}{1.085494in}}%
\pgfpathcurveto{\pgfqpoint{1.558226in}{1.077680in}}{\pgfqpoint{1.568825in}{1.073290in}}{\pgfqpoint{1.579875in}{1.073290in}}%
\pgfpathclose%
\pgfusepath{stroke,fill}%
\end{pgfscope}%
\begin{pgfscope}%
\pgfpathrectangle{\pgfqpoint{0.375000in}{0.330000in}}{\pgfqpoint{2.325000in}{2.310000in}}%
\pgfusepath{clip}%
\pgfsetbuttcap%
\pgfsetroundjoin%
\definecolor{currentfill}{rgb}{0.000000,0.000000,0.000000}%
\pgfsetfillcolor{currentfill}%
\pgfsetlinewidth{1.003750pt}%
\definecolor{currentstroke}{rgb}{0.000000,0.000000,0.000000}%
\pgfsetstrokecolor{currentstroke}%
\pgfsetdash{}{0pt}%
\pgfpathmoveto{\pgfqpoint{1.579875in}{1.073290in}}%
\pgfpathcurveto{\pgfqpoint{1.590925in}{1.073290in}}{\pgfqpoint{1.601524in}{1.077680in}}{\pgfqpoint{1.609338in}{1.085494in}}%
\pgfpathcurveto{\pgfqpoint{1.617151in}{1.093308in}}{\pgfqpoint{1.621542in}{1.103907in}}{\pgfqpoint{1.621542in}{1.114957in}}%
\pgfpathcurveto{\pgfqpoint{1.621542in}{1.126007in}}{\pgfqpoint{1.617151in}{1.136606in}}{\pgfqpoint{1.609338in}{1.144420in}}%
\pgfpathcurveto{\pgfqpoint{1.601524in}{1.152233in}}{\pgfqpoint{1.590925in}{1.156624in}}{\pgfqpoint{1.579875in}{1.156624in}}%
\pgfpathcurveto{\pgfqpoint{1.568825in}{1.156624in}}{\pgfqpoint{1.558226in}{1.152233in}}{\pgfqpoint{1.550412in}{1.144420in}}%
\pgfpathcurveto{\pgfqpoint{1.542599in}{1.136606in}}{\pgfqpoint{1.538208in}{1.126007in}}{\pgfqpoint{1.538208in}{1.114957in}}%
\pgfpathcurveto{\pgfqpoint{1.538208in}{1.103907in}}{\pgfqpoint{1.542599in}{1.093308in}}{\pgfqpoint{1.550412in}{1.085494in}}%
\pgfpathcurveto{\pgfqpoint{1.558226in}{1.077680in}}{\pgfqpoint{1.568825in}{1.073290in}}{\pgfqpoint{1.579875in}{1.073290in}}%
\pgfpathclose%
\pgfusepath{stroke,fill}%
\end{pgfscope}%
\begin{pgfscope}%
\pgfpathrectangle{\pgfqpoint{0.375000in}{0.330000in}}{\pgfqpoint{2.325000in}{2.310000in}}%
\pgfusepath{clip}%
\pgfsetbuttcap%
\pgfsetroundjoin%
\definecolor{currentfill}{rgb}{0.000000,0.000000,0.000000}%
\pgfsetfillcolor{currentfill}%
\pgfsetlinewidth{1.003750pt}%
\definecolor{currentstroke}{rgb}{0.000000,0.000000,0.000000}%
\pgfsetstrokecolor{currentstroke}%
\pgfsetdash{}{0pt}%
\pgfpathmoveto{\pgfqpoint{1.579875in}{1.006541in}}%
\pgfpathcurveto{\pgfqpoint{1.590925in}{1.006541in}}{\pgfqpoint{1.601524in}{1.010931in}}{\pgfqpoint{1.609338in}{1.018745in}}%
\pgfpathcurveto{\pgfqpoint{1.617151in}{1.026559in}}{\pgfqpoint{1.621542in}{1.037158in}}{\pgfqpoint{1.621542in}{1.048208in}}%
\pgfpathcurveto{\pgfqpoint{1.621542in}{1.059258in}}{\pgfqpoint{1.617151in}{1.069857in}}{\pgfqpoint{1.609338in}{1.077670in}}%
\pgfpathcurveto{\pgfqpoint{1.601524in}{1.085484in}}{\pgfqpoint{1.590925in}{1.089874in}}{\pgfqpoint{1.579875in}{1.089874in}}%
\pgfpathcurveto{\pgfqpoint{1.568825in}{1.089874in}}{\pgfqpoint{1.558226in}{1.085484in}}{\pgfqpoint{1.550412in}{1.077670in}}%
\pgfpathcurveto{\pgfqpoint{1.542599in}{1.069857in}}{\pgfqpoint{1.538208in}{1.059258in}}{\pgfqpoint{1.538208in}{1.048208in}}%
\pgfpathcurveto{\pgfqpoint{1.538208in}{1.037158in}}{\pgfqpoint{1.542599in}{1.026559in}}{\pgfqpoint{1.550412in}{1.018745in}}%
\pgfpathcurveto{\pgfqpoint{1.558226in}{1.010931in}}{\pgfqpoint{1.568825in}{1.006541in}}{\pgfqpoint{1.579875in}{1.006541in}}%
\pgfpathclose%
\pgfusepath{stroke,fill}%
\end{pgfscope}%
\begin{pgfscope}%
\pgfpathrectangle{\pgfqpoint{0.375000in}{0.330000in}}{\pgfqpoint{2.325000in}{2.310000in}}%
\pgfusepath{clip}%
\pgfsetbuttcap%
\pgfsetroundjoin%
\definecolor{currentfill}{rgb}{0.000000,0.000000,0.000000}%
\pgfsetfillcolor{currentfill}%
\pgfsetlinewidth{1.003750pt}%
\definecolor{currentstroke}{rgb}{0.000000,0.000000,0.000000}%
\pgfsetstrokecolor{currentstroke}%
\pgfsetdash{}{0pt}%
\pgfpathmoveto{\pgfqpoint{1.579875in}{1.012609in}}%
\pgfpathcurveto{\pgfqpoint{1.590925in}{1.012609in}}{\pgfqpoint{1.601524in}{1.016999in}}{\pgfqpoint{1.609338in}{1.024813in}}%
\pgfpathcurveto{\pgfqpoint{1.617151in}{1.032627in}}{\pgfqpoint{1.621542in}{1.043226in}}{\pgfqpoint{1.621542in}{1.054276in}}%
\pgfpathcurveto{\pgfqpoint{1.621542in}{1.065326in}}{\pgfqpoint{1.617151in}{1.075925in}}{\pgfqpoint{1.609338in}{1.083739in}}%
\pgfpathcurveto{\pgfqpoint{1.601524in}{1.091552in}}{\pgfqpoint{1.590925in}{1.095942in}}{\pgfqpoint{1.579875in}{1.095942in}}%
\pgfpathcurveto{\pgfqpoint{1.568825in}{1.095942in}}{\pgfqpoint{1.558226in}{1.091552in}}{\pgfqpoint{1.550412in}{1.083739in}}%
\pgfpathcurveto{\pgfqpoint{1.542599in}{1.075925in}}{\pgfqpoint{1.538208in}{1.065326in}}{\pgfqpoint{1.538208in}{1.054276in}}%
\pgfpathcurveto{\pgfqpoint{1.538208in}{1.043226in}}{\pgfqpoint{1.542599in}{1.032627in}}{\pgfqpoint{1.550412in}{1.024813in}}%
\pgfpathcurveto{\pgfqpoint{1.558226in}{1.016999in}}{\pgfqpoint{1.568825in}{1.012609in}}{\pgfqpoint{1.579875in}{1.012609in}}%
\pgfpathclose%
\pgfusepath{stroke,fill}%
\end{pgfscope}%
\begin{pgfscope}%
\pgfpathrectangle{\pgfqpoint{0.375000in}{0.330000in}}{\pgfqpoint{2.325000in}{2.310000in}}%
\pgfusepath{clip}%
\pgfsetbuttcap%
\pgfsetroundjoin%
\definecolor{currentfill}{rgb}{0.000000,0.000000,0.000000}%
\pgfsetfillcolor{currentfill}%
\pgfsetlinewidth{1.003750pt}%
\definecolor{currentstroke}{rgb}{0.000000,0.000000,0.000000}%
\pgfsetstrokecolor{currentstroke}%
\pgfsetdash{}{0pt}%
\pgfpathmoveto{\pgfqpoint{1.579875in}{1.055086in}}%
\pgfpathcurveto{\pgfqpoint{1.590925in}{1.055086in}}{\pgfqpoint{1.601524in}{1.059476in}}{\pgfqpoint{1.609338in}{1.067290in}}%
\pgfpathcurveto{\pgfqpoint{1.617151in}{1.075103in}}{\pgfqpoint{1.621542in}{1.085702in}}{\pgfqpoint{1.621542in}{1.096753in}}%
\pgfpathcurveto{\pgfqpoint{1.621542in}{1.107803in}}{\pgfqpoint{1.617151in}{1.118402in}}{\pgfqpoint{1.609338in}{1.126215in}}%
\pgfpathcurveto{\pgfqpoint{1.601524in}{1.134029in}}{\pgfqpoint{1.590925in}{1.138419in}}{\pgfqpoint{1.579875in}{1.138419in}}%
\pgfpathcurveto{\pgfqpoint{1.568825in}{1.138419in}}{\pgfqpoint{1.558226in}{1.134029in}}{\pgfqpoint{1.550412in}{1.126215in}}%
\pgfpathcurveto{\pgfqpoint{1.542599in}{1.118402in}}{\pgfqpoint{1.538208in}{1.107803in}}{\pgfqpoint{1.538208in}{1.096753in}}%
\pgfpathcurveto{\pgfqpoint{1.538208in}{1.085702in}}{\pgfqpoint{1.542599in}{1.075103in}}{\pgfqpoint{1.550412in}{1.067290in}}%
\pgfpathcurveto{\pgfqpoint{1.558226in}{1.059476in}}{\pgfqpoint{1.568825in}{1.055086in}}{\pgfqpoint{1.579875in}{1.055086in}}%
\pgfpathclose%
\pgfusepath{stroke,fill}%
\end{pgfscope}%
\begin{pgfscope}%
\pgfpathrectangle{\pgfqpoint{0.375000in}{0.330000in}}{\pgfqpoint{2.325000in}{2.310000in}}%
\pgfusepath{clip}%
\pgfsetbuttcap%
\pgfsetroundjoin%
\definecolor{currentfill}{rgb}{0.000000,0.000000,0.000000}%
\pgfsetfillcolor{currentfill}%
\pgfsetlinewidth{1.003750pt}%
\definecolor{currentstroke}{rgb}{0.000000,0.000000,0.000000}%
\pgfsetstrokecolor{currentstroke}%
\pgfsetdash{}{0pt}%
\pgfpathmoveto{\pgfqpoint{1.579875in}{1.109699in}}%
\pgfpathcurveto{\pgfqpoint{1.590925in}{1.109699in}}{\pgfqpoint{1.601524in}{1.114089in}}{\pgfqpoint{1.609338in}{1.121903in}}%
\pgfpathcurveto{\pgfqpoint{1.617151in}{1.129716in}}{\pgfqpoint{1.621542in}{1.140315in}}{\pgfqpoint{1.621542in}{1.151365in}}%
\pgfpathcurveto{\pgfqpoint{1.621542in}{1.162416in}}{\pgfqpoint{1.617151in}{1.173015in}}{\pgfqpoint{1.609338in}{1.180828in}}%
\pgfpathcurveto{\pgfqpoint{1.601524in}{1.188642in}}{\pgfqpoint{1.590925in}{1.193032in}}{\pgfqpoint{1.579875in}{1.193032in}}%
\pgfpathcurveto{\pgfqpoint{1.568825in}{1.193032in}}{\pgfqpoint{1.558226in}{1.188642in}}{\pgfqpoint{1.550412in}{1.180828in}}%
\pgfpathcurveto{\pgfqpoint{1.542599in}{1.173015in}}{\pgfqpoint{1.538208in}{1.162416in}}{\pgfqpoint{1.538208in}{1.151365in}}%
\pgfpathcurveto{\pgfqpoint{1.538208in}{1.140315in}}{\pgfqpoint{1.542599in}{1.129716in}}{\pgfqpoint{1.550412in}{1.121903in}}%
\pgfpathcurveto{\pgfqpoint{1.558226in}{1.114089in}}{\pgfqpoint{1.568825in}{1.109699in}}{\pgfqpoint{1.579875in}{1.109699in}}%
\pgfpathclose%
\pgfusepath{stroke,fill}%
\end{pgfscope}%
\begin{pgfscope}%
\pgfpathrectangle{\pgfqpoint{0.375000in}{0.330000in}}{\pgfqpoint{2.325000in}{2.310000in}}%
\pgfusepath{clip}%
\pgfsetbuttcap%
\pgfsetroundjoin%
\definecolor{currentfill}{rgb}{0.000000,0.000000,0.000000}%
\pgfsetfillcolor{currentfill}%
\pgfsetlinewidth{1.003750pt}%
\definecolor{currentstroke}{rgb}{0.000000,0.000000,0.000000}%
\pgfsetstrokecolor{currentstroke}%
\pgfsetdash{}{0pt}%
\pgfpathmoveto{\pgfqpoint{1.579875in}{1.042950in}}%
\pgfpathcurveto{\pgfqpoint{1.590925in}{1.042950in}}{\pgfqpoint{1.601524in}{1.047340in}}{\pgfqpoint{1.609338in}{1.055154in}}%
\pgfpathcurveto{\pgfqpoint{1.617151in}{1.062967in}}{\pgfqpoint{1.621542in}{1.073566in}}{\pgfqpoint{1.621542in}{1.084616in}}%
\pgfpathcurveto{\pgfqpoint{1.621542in}{1.095666in}}{\pgfqpoint{1.617151in}{1.106265in}}{\pgfqpoint{1.609338in}{1.114079in}}%
\pgfpathcurveto{\pgfqpoint{1.601524in}{1.121893in}}{\pgfqpoint{1.590925in}{1.126283in}}{\pgfqpoint{1.579875in}{1.126283in}}%
\pgfpathcurveto{\pgfqpoint{1.568825in}{1.126283in}}{\pgfqpoint{1.558226in}{1.121893in}}{\pgfqpoint{1.550412in}{1.114079in}}%
\pgfpathcurveto{\pgfqpoint{1.542599in}{1.106265in}}{\pgfqpoint{1.538208in}{1.095666in}}{\pgfqpoint{1.538208in}{1.084616in}}%
\pgfpathcurveto{\pgfqpoint{1.538208in}{1.073566in}}{\pgfqpoint{1.542599in}{1.062967in}}{\pgfqpoint{1.550412in}{1.055154in}}%
\pgfpathcurveto{\pgfqpoint{1.558226in}{1.047340in}}{\pgfqpoint{1.568825in}{1.042950in}}{\pgfqpoint{1.579875in}{1.042950in}}%
\pgfpathclose%
\pgfusepath{stroke,fill}%
\end{pgfscope}%
\begin{pgfscope}%
\pgfpathrectangle{\pgfqpoint{0.375000in}{0.330000in}}{\pgfqpoint{2.325000in}{2.310000in}}%
\pgfusepath{clip}%
\pgfsetbuttcap%
\pgfsetroundjoin%
\definecolor{currentfill}{rgb}{0.000000,0.000000,0.000000}%
\pgfsetfillcolor{currentfill}%
\pgfsetlinewidth{1.003750pt}%
\definecolor{currentstroke}{rgb}{0.000000,0.000000,0.000000}%
\pgfsetstrokecolor{currentstroke}%
\pgfsetdash{}{0pt}%
\pgfpathmoveto{\pgfqpoint{1.579875in}{1.091494in}}%
\pgfpathcurveto{\pgfqpoint{1.590925in}{1.091494in}}{\pgfqpoint{1.601524in}{1.095885in}}{\pgfqpoint{1.609338in}{1.103698in}}%
\pgfpathcurveto{\pgfqpoint{1.617151in}{1.111512in}}{\pgfqpoint{1.621542in}{1.122111in}}{\pgfqpoint{1.621542in}{1.133161in}}%
\pgfpathcurveto{\pgfqpoint{1.621542in}{1.144211in}}{\pgfqpoint{1.617151in}{1.154810in}}{\pgfqpoint{1.609338in}{1.162624in}}%
\pgfpathcurveto{\pgfqpoint{1.601524in}{1.170438in}}{\pgfqpoint{1.590925in}{1.174828in}}{\pgfqpoint{1.579875in}{1.174828in}}%
\pgfpathcurveto{\pgfqpoint{1.568825in}{1.174828in}}{\pgfqpoint{1.558226in}{1.170438in}}{\pgfqpoint{1.550412in}{1.162624in}}%
\pgfpathcurveto{\pgfqpoint{1.542599in}{1.154810in}}{\pgfqpoint{1.538208in}{1.144211in}}{\pgfqpoint{1.538208in}{1.133161in}}%
\pgfpathcurveto{\pgfqpoint{1.538208in}{1.122111in}}{\pgfqpoint{1.542599in}{1.111512in}}{\pgfqpoint{1.550412in}{1.103698in}}%
\pgfpathcurveto{\pgfqpoint{1.558226in}{1.095885in}}{\pgfqpoint{1.568825in}{1.091494in}}{\pgfqpoint{1.579875in}{1.091494in}}%
\pgfpathclose%
\pgfusepath{stroke,fill}%
\end{pgfscope}%
\begin{pgfscope}%
\pgfpathrectangle{\pgfqpoint{0.375000in}{0.330000in}}{\pgfqpoint{2.325000in}{2.310000in}}%
\pgfusepath{clip}%
\pgfsetbuttcap%
\pgfsetroundjoin%
\definecolor{currentfill}{rgb}{0.000000,0.000000,0.000000}%
\pgfsetfillcolor{currentfill}%
\pgfsetlinewidth{1.003750pt}%
\definecolor{currentstroke}{rgb}{0.000000,0.000000,0.000000}%
\pgfsetstrokecolor{currentstroke}%
\pgfsetdash{}{0pt}%
\pgfpathmoveto{\pgfqpoint{1.579875in}{1.085426in}}%
\pgfpathcurveto{\pgfqpoint{1.590925in}{1.085426in}}{\pgfqpoint{1.601524in}{1.089817in}}{\pgfqpoint{1.609338in}{1.097630in}}%
\pgfpathcurveto{\pgfqpoint{1.617151in}{1.105444in}}{\pgfqpoint{1.621542in}{1.116043in}}{\pgfqpoint{1.621542in}{1.127093in}}%
\pgfpathcurveto{\pgfqpoint{1.621542in}{1.138143in}}{\pgfqpoint{1.617151in}{1.148742in}}{\pgfqpoint{1.609338in}{1.156556in}}%
\pgfpathcurveto{\pgfqpoint{1.601524in}{1.164369in}}{\pgfqpoint{1.590925in}{1.168760in}}{\pgfqpoint{1.579875in}{1.168760in}}%
\pgfpathcurveto{\pgfqpoint{1.568825in}{1.168760in}}{\pgfqpoint{1.558226in}{1.164369in}}{\pgfqpoint{1.550412in}{1.156556in}}%
\pgfpathcurveto{\pgfqpoint{1.542599in}{1.148742in}}{\pgfqpoint{1.538208in}{1.138143in}}{\pgfqpoint{1.538208in}{1.127093in}}%
\pgfpathcurveto{\pgfqpoint{1.538208in}{1.116043in}}{\pgfqpoint{1.542599in}{1.105444in}}{\pgfqpoint{1.550412in}{1.097630in}}%
\pgfpathcurveto{\pgfqpoint{1.558226in}{1.089817in}}{\pgfqpoint{1.568825in}{1.085426in}}{\pgfqpoint{1.579875in}{1.085426in}}%
\pgfpathclose%
\pgfusepath{stroke,fill}%
\end{pgfscope}%
\begin{pgfscope}%
\pgfpathrectangle{\pgfqpoint{0.375000in}{0.330000in}}{\pgfqpoint{2.325000in}{2.310000in}}%
\pgfusepath{clip}%
\pgfsetbuttcap%
\pgfsetroundjoin%
\definecolor{currentfill}{rgb}{0.000000,0.000000,0.000000}%
\pgfsetfillcolor{currentfill}%
\pgfsetlinewidth{1.003750pt}%
\definecolor{currentstroke}{rgb}{0.000000,0.000000,0.000000}%
\pgfsetstrokecolor{currentstroke}%
\pgfsetdash{}{0pt}%
\pgfpathmoveto{\pgfqpoint{1.579875in}{1.042950in}}%
\pgfpathcurveto{\pgfqpoint{1.590925in}{1.042950in}}{\pgfqpoint{1.601524in}{1.047340in}}{\pgfqpoint{1.609338in}{1.055154in}}%
\pgfpathcurveto{\pgfqpoint{1.617151in}{1.062967in}}{\pgfqpoint{1.621542in}{1.073566in}}{\pgfqpoint{1.621542in}{1.084616in}}%
\pgfpathcurveto{\pgfqpoint{1.621542in}{1.095666in}}{\pgfqpoint{1.617151in}{1.106265in}}{\pgfqpoint{1.609338in}{1.114079in}}%
\pgfpathcurveto{\pgfqpoint{1.601524in}{1.121893in}}{\pgfqpoint{1.590925in}{1.126283in}}{\pgfqpoint{1.579875in}{1.126283in}}%
\pgfpathcurveto{\pgfqpoint{1.568825in}{1.126283in}}{\pgfqpoint{1.558226in}{1.121893in}}{\pgfqpoint{1.550412in}{1.114079in}}%
\pgfpathcurveto{\pgfqpoint{1.542599in}{1.106265in}}{\pgfqpoint{1.538208in}{1.095666in}}{\pgfqpoint{1.538208in}{1.084616in}}%
\pgfpathcurveto{\pgfqpoint{1.538208in}{1.073566in}}{\pgfqpoint{1.542599in}{1.062967in}}{\pgfqpoint{1.550412in}{1.055154in}}%
\pgfpathcurveto{\pgfqpoint{1.558226in}{1.047340in}}{\pgfqpoint{1.568825in}{1.042950in}}{\pgfqpoint{1.579875in}{1.042950in}}%
\pgfpathclose%
\pgfusepath{stroke,fill}%
\end{pgfscope}%
\begin{pgfscope}%
\pgfpathrectangle{\pgfqpoint{0.375000in}{0.330000in}}{\pgfqpoint{2.325000in}{2.310000in}}%
\pgfusepath{clip}%
\pgfsetbuttcap%
\pgfsetroundjoin%
\definecolor{currentfill}{rgb}{0.000000,0.000000,0.000000}%
\pgfsetfillcolor{currentfill}%
\pgfsetlinewidth{1.003750pt}%
\definecolor{currentstroke}{rgb}{0.000000,0.000000,0.000000}%
\pgfsetstrokecolor{currentstroke}%
\pgfsetdash{}{0pt}%
\pgfpathmoveto{\pgfqpoint{1.579875in}{0.970132in}}%
\pgfpathcurveto{\pgfqpoint{1.590925in}{0.970132in}}{\pgfqpoint{1.601524in}{0.974523in}}{\pgfqpoint{1.609338in}{0.982336in}}%
\pgfpathcurveto{\pgfqpoint{1.617151in}{0.990150in}}{\pgfqpoint{1.621542in}{1.000749in}}{\pgfqpoint{1.621542in}{1.011799in}}%
\pgfpathcurveto{\pgfqpoint{1.621542in}{1.022849in}}{\pgfqpoint{1.617151in}{1.033448in}}{\pgfqpoint{1.609338in}{1.041262in}}%
\pgfpathcurveto{\pgfqpoint{1.601524in}{1.049075in}}{\pgfqpoint{1.590925in}{1.053466in}}{\pgfqpoint{1.579875in}{1.053466in}}%
\pgfpathcurveto{\pgfqpoint{1.568825in}{1.053466in}}{\pgfqpoint{1.558226in}{1.049075in}}{\pgfqpoint{1.550412in}{1.041262in}}%
\pgfpathcurveto{\pgfqpoint{1.542599in}{1.033448in}}{\pgfqpoint{1.538208in}{1.022849in}}{\pgfqpoint{1.538208in}{1.011799in}}%
\pgfpathcurveto{\pgfqpoint{1.538208in}{1.000749in}}{\pgfqpoint{1.542599in}{0.990150in}}{\pgfqpoint{1.550412in}{0.982336in}}%
\pgfpathcurveto{\pgfqpoint{1.558226in}{0.974523in}}{\pgfqpoint{1.568825in}{0.970132in}}{\pgfqpoint{1.579875in}{0.970132in}}%
\pgfpathclose%
\pgfusepath{stroke,fill}%
\end{pgfscope}%
\begin{pgfscope}%
\pgfpathrectangle{\pgfqpoint{0.375000in}{0.330000in}}{\pgfqpoint{2.325000in}{2.310000in}}%
\pgfusepath{clip}%
\pgfsetbuttcap%
\pgfsetroundjoin%
\definecolor{currentfill}{rgb}{0.000000,0.000000,0.000000}%
\pgfsetfillcolor{currentfill}%
\pgfsetlinewidth{1.003750pt}%
\definecolor{currentstroke}{rgb}{0.000000,0.000000,0.000000}%
\pgfsetstrokecolor{currentstroke}%
\pgfsetdash{}{0pt}%
\pgfpathmoveto{\pgfqpoint{1.579875in}{1.042950in}}%
\pgfpathcurveto{\pgfqpoint{1.590925in}{1.042950in}}{\pgfqpoint{1.601524in}{1.047340in}}{\pgfqpoint{1.609338in}{1.055154in}}%
\pgfpathcurveto{\pgfqpoint{1.617151in}{1.062967in}}{\pgfqpoint{1.621542in}{1.073566in}}{\pgfqpoint{1.621542in}{1.084616in}}%
\pgfpathcurveto{\pgfqpoint{1.621542in}{1.095666in}}{\pgfqpoint{1.617151in}{1.106265in}}{\pgfqpoint{1.609338in}{1.114079in}}%
\pgfpathcurveto{\pgfqpoint{1.601524in}{1.121893in}}{\pgfqpoint{1.590925in}{1.126283in}}{\pgfqpoint{1.579875in}{1.126283in}}%
\pgfpathcurveto{\pgfqpoint{1.568825in}{1.126283in}}{\pgfqpoint{1.558226in}{1.121893in}}{\pgfqpoint{1.550412in}{1.114079in}}%
\pgfpathcurveto{\pgfqpoint{1.542599in}{1.106265in}}{\pgfqpoint{1.538208in}{1.095666in}}{\pgfqpoint{1.538208in}{1.084616in}}%
\pgfpathcurveto{\pgfqpoint{1.538208in}{1.073566in}}{\pgfqpoint{1.542599in}{1.062967in}}{\pgfqpoint{1.550412in}{1.055154in}}%
\pgfpathcurveto{\pgfqpoint{1.558226in}{1.047340in}}{\pgfqpoint{1.568825in}{1.042950in}}{\pgfqpoint{1.579875in}{1.042950in}}%
\pgfpathclose%
\pgfusepath{stroke,fill}%
\end{pgfscope}%
\begin{pgfscope}%
\pgfpathrectangle{\pgfqpoint{0.375000in}{0.330000in}}{\pgfqpoint{2.325000in}{2.310000in}}%
\pgfusepath{clip}%
\pgfsetbuttcap%
\pgfsetroundjoin%
\definecolor{currentfill}{rgb}{0.000000,0.000000,0.000000}%
\pgfsetfillcolor{currentfill}%
\pgfsetlinewidth{1.003750pt}%
\definecolor{currentstroke}{rgb}{0.000000,0.000000,0.000000}%
\pgfsetstrokecolor{currentstroke}%
\pgfsetdash{}{0pt}%
\pgfpathmoveto{\pgfqpoint{1.579875in}{1.164312in}}%
\pgfpathcurveto{\pgfqpoint{1.590925in}{1.164312in}}{\pgfqpoint{1.601524in}{1.168702in}}{\pgfqpoint{1.609338in}{1.176516in}}%
\pgfpathcurveto{\pgfqpoint{1.617151in}{1.184329in}}{\pgfqpoint{1.621542in}{1.194928in}}{\pgfqpoint{1.621542in}{1.205978in}}%
\pgfpathcurveto{\pgfqpoint{1.621542in}{1.217029in}}{\pgfqpoint{1.617151in}{1.227628in}}{\pgfqpoint{1.609338in}{1.235441in}}%
\pgfpathcurveto{\pgfqpoint{1.601524in}{1.243255in}}{\pgfqpoint{1.590925in}{1.247645in}}{\pgfqpoint{1.579875in}{1.247645in}}%
\pgfpathcurveto{\pgfqpoint{1.568825in}{1.247645in}}{\pgfqpoint{1.558226in}{1.243255in}}{\pgfqpoint{1.550412in}{1.235441in}}%
\pgfpathcurveto{\pgfqpoint{1.542599in}{1.227628in}}{\pgfqpoint{1.538208in}{1.217029in}}{\pgfqpoint{1.538208in}{1.205978in}}%
\pgfpathcurveto{\pgfqpoint{1.538208in}{1.194928in}}{\pgfqpoint{1.542599in}{1.184329in}}{\pgfqpoint{1.550412in}{1.176516in}}%
\pgfpathcurveto{\pgfqpoint{1.558226in}{1.168702in}}{\pgfqpoint{1.568825in}{1.164312in}}{\pgfqpoint{1.579875in}{1.164312in}}%
\pgfpathclose%
\pgfusepath{stroke,fill}%
\end{pgfscope}%
\begin{pgfscope}%
\pgfpathrectangle{\pgfqpoint{0.375000in}{0.330000in}}{\pgfqpoint{2.325000in}{2.310000in}}%
\pgfusepath{clip}%
\pgfsetbuttcap%
\pgfsetroundjoin%
\definecolor{currentfill}{rgb}{0.000000,0.000000,0.000000}%
\pgfsetfillcolor{currentfill}%
\pgfsetlinewidth{1.003750pt}%
\definecolor{currentstroke}{rgb}{0.000000,0.000000,0.000000}%
\pgfsetstrokecolor{currentstroke}%
\pgfsetdash{}{0pt}%
\pgfpathmoveto{\pgfqpoint{1.579875in}{1.109699in}}%
\pgfpathcurveto{\pgfqpoint{1.590925in}{1.109699in}}{\pgfqpoint{1.601524in}{1.114089in}}{\pgfqpoint{1.609338in}{1.121903in}}%
\pgfpathcurveto{\pgfqpoint{1.617151in}{1.129716in}}{\pgfqpoint{1.621542in}{1.140315in}}{\pgfqpoint{1.621542in}{1.151365in}}%
\pgfpathcurveto{\pgfqpoint{1.621542in}{1.162416in}}{\pgfqpoint{1.617151in}{1.173015in}}{\pgfqpoint{1.609338in}{1.180828in}}%
\pgfpathcurveto{\pgfqpoint{1.601524in}{1.188642in}}{\pgfqpoint{1.590925in}{1.193032in}}{\pgfqpoint{1.579875in}{1.193032in}}%
\pgfpathcurveto{\pgfqpoint{1.568825in}{1.193032in}}{\pgfqpoint{1.558226in}{1.188642in}}{\pgfqpoint{1.550412in}{1.180828in}}%
\pgfpathcurveto{\pgfqpoint{1.542599in}{1.173015in}}{\pgfqpoint{1.538208in}{1.162416in}}{\pgfqpoint{1.538208in}{1.151365in}}%
\pgfpathcurveto{\pgfqpoint{1.538208in}{1.140315in}}{\pgfqpoint{1.542599in}{1.129716in}}{\pgfqpoint{1.550412in}{1.121903in}}%
\pgfpathcurveto{\pgfqpoint{1.558226in}{1.114089in}}{\pgfqpoint{1.568825in}{1.109699in}}{\pgfqpoint{1.579875in}{1.109699in}}%
\pgfpathclose%
\pgfusepath{stroke,fill}%
\end{pgfscope}%
\begin{pgfscope}%
\pgfpathrectangle{\pgfqpoint{0.375000in}{0.330000in}}{\pgfqpoint{2.325000in}{2.310000in}}%
\pgfusepath{clip}%
\pgfsetbuttcap%
\pgfsetroundjoin%
\definecolor{currentfill}{rgb}{0.000000,0.000000,0.000000}%
\pgfsetfillcolor{currentfill}%
\pgfsetlinewidth{1.003750pt}%
\definecolor{currentstroke}{rgb}{0.000000,0.000000,0.000000}%
\pgfsetstrokecolor{currentstroke}%
\pgfsetdash{}{0pt}%
\pgfpathmoveto{\pgfqpoint{1.579875in}{0.939792in}}%
\pgfpathcurveto{\pgfqpoint{1.590925in}{0.939792in}}{\pgfqpoint{1.601524in}{0.944182in}}{\pgfqpoint{1.609338in}{0.951996in}}%
\pgfpathcurveto{\pgfqpoint{1.617151in}{0.959809in}}{\pgfqpoint{1.621542in}{0.970408in}}{\pgfqpoint{1.621542in}{0.981459in}}%
\pgfpathcurveto{\pgfqpoint{1.621542in}{0.992509in}}{\pgfqpoint{1.617151in}{1.003108in}}{\pgfqpoint{1.609338in}{1.010921in}}%
\pgfpathcurveto{\pgfqpoint{1.601524in}{1.018735in}}{\pgfqpoint{1.590925in}{1.023125in}}{\pgfqpoint{1.579875in}{1.023125in}}%
\pgfpathcurveto{\pgfqpoint{1.568825in}{1.023125in}}{\pgfqpoint{1.558226in}{1.018735in}}{\pgfqpoint{1.550412in}{1.010921in}}%
\pgfpathcurveto{\pgfqpoint{1.542599in}{1.003108in}}{\pgfqpoint{1.538208in}{0.992509in}}{\pgfqpoint{1.538208in}{0.981459in}}%
\pgfpathcurveto{\pgfqpoint{1.538208in}{0.970408in}}{\pgfqpoint{1.542599in}{0.959809in}}{\pgfqpoint{1.550412in}{0.951996in}}%
\pgfpathcurveto{\pgfqpoint{1.558226in}{0.944182in}}{\pgfqpoint{1.568825in}{0.939792in}}{\pgfqpoint{1.579875in}{0.939792in}}%
\pgfpathclose%
\pgfusepath{stroke,fill}%
\end{pgfscope}%
\begin{pgfscope}%
\pgfpathrectangle{\pgfqpoint{0.375000in}{0.330000in}}{\pgfqpoint{2.325000in}{2.310000in}}%
\pgfusepath{clip}%
\pgfsetbuttcap%
\pgfsetroundjoin%
\definecolor{currentfill}{rgb}{0.000000,0.000000,0.000000}%
\pgfsetfillcolor{currentfill}%
\pgfsetlinewidth{1.003750pt}%
\definecolor{currentstroke}{rgb}{0.000000,0.000000,0.000000}%
\pgfsetstrokecolor{currentstroke}%
\pgfsetdash{}{0pt}%
\pgfpathmoveto{\pgfqpoint{1.579875in}{1.146107in}}%
\pgfpathcurveto{\pgfqpoint{1.590925in}{1.146107in}}{\pgfqpoint{1.601524in}{1.150498in}}{\pgfqpoint{1.609338in}{1.158311in}}%
\pgfpathcurveto{\pgfqpoint{1.617151in}{1.166125in}}{\pgfqpoint{1.621542in}{1.176724in}}{\pgfqpoint{1.621542in}{1.187774in}}%
\pgfpathcurveto{\pgfqpoint{1.621542in}{1.198824in}}{\pgfqpoint{1.617151in}{1.209423in}}{\pgfqpoint{1.609338in}{1.217237in}}%
\pgfpathcurveto{\pgfqpoint{1.601524in}{1.225051in}}{\pgfqpoint{1.590925in}{1.229441in}}{\pgfqpoint{1.579875in}{1.229441in}}%
\pgfpathcurveto{\pgfqpoint{1.568825in}{1.229441in}}{\pgfqpoint{1.558226in}{1.225051in}}{\pgfqpoint{1.550412in}{1.217237in}}%
\pgfpathcurveto{\pgfqpoint{1.542599in}{1.209423in}}{\pgfqpoint{1.538208in}{1.198824in}}{\pgfqpoint{1.538208in}{1.187774in}}%
\pgfpathcurveto{\pgfqpoint{1.538208in}{1.176724in}}{\pgfqpoint{1.542599in}{1.166125in}}{\pgfqpoint{1.550412in}{1.158311in}}%
\pgfpathcurveto{\pgfqpoint{1.558226in}{1.150498in}}{\pgfqpoint{1.568825in}{1.146107in}}{\pgfqpoint{1.579875in}{1.146107in}}%
\pgfpathclose%
\pgfusepath{stroke,fill}%
\end{pgfscope}%
\begin{pgfscope}%
\pgfpathrectangle{\pgfqpoint{0.375000in}{0.330000in}}{\pgfqpoint{2.325000in}{2.310000in}}%
\pgfusepath{clip}%
\pgfsetbuttcap%
\pgfsetroundjoin%
\definecolor{currentfill}{rgb}{0.000000,0.000000,0.000000}%
\pgfsetfillcolor{currentfill}%
\pgfsetlinewidth{1.003750pt}%
\definecolor{currentstroke}{rgb}{0.000000,0.000000,0.000000}%
\pgfsetstrokecolor{currentstroke}%
\pgfsetdash{}{0pt}%
\pgfpathmoveto{\pgfqpoint{1.579875in}{1.158244in}}%
\pgfpathcurveto{\pgfqpoint{1.590925in}{1.158244in}}{\pgfqpoint{1.601524in}{1.162634in}}{\pgfqpoint{1.609338in}{1.170448in}}%
\pgfpathcurveto{\pgfqpoint{1.617151in}{1.178261in}}{\pgfqpoint{1.621542in}{1.188860in}}{\pgfqpoint{1.621542in}{1.199910in}}%
\pgfpathcurveto{\pgfqpoint{1.621542in}{1.210960in}}{\pgfqpoint{1.617151in}{1.221559in}}{\pgfqpoint{1.609338in}{1.229373in}}%
\pgfpathcurveto{\pgfqpoint{1.601524in}{1.237187in}}{\pgfqpoint{1.590925in}{1.241577in}}{\pgfqpoint{1.579875in}{1.241577in}}%
\pgfpathcurveto{\pgfqpoint{1.568825in}{1.241577in}}{\pgfqpoint{1.558226in}{1.237187in}}{\pgfqpoint{1.550412in}{1.229373in}}%
\pgfpathcurveto{\pgfqpoint{1.542599in}{1.221559in}}{\pgfqpoint{1.538208in}{1.210960in}}{\pgfqpoint{1.538208in}{1.199910in}}%
\pgfpathcurveto{\pgfqpoint{1.538208in}{1.188860in}}{\pgfqpoint{1.542599in}{1.178261in}}{\pgfqpoint{1.550412in}{1.170448in}}%
\pgfpathcurveto{\pgfqpoint{1.558226in}{1.162634in}}{\pgfqpoint{1.568825in}{1.158244in}}{\pgfqpoint{1.579875in}{1.158244in}}%
\pgfpathclose%
\pgfusepath{stroke,fill}%
\end{pgfscope}%
\begin{pgfscope}%
\pgfpathrectangle{\pgfqpoint{0.375000in}{0.330000in}}{\pgfqpoint{2.325000in}{2.310000in}}%
\pgfusepath{clip}%
\pgfsetbuttcap%
\pgfsetroundjoin%
\definecolor{currentfill}{rgb}{0.000000,0.000000,0.000000}%
\pgfsetfillcolor{currentfill}%
\pgfsetlinewidth{1.003750pt}%
\definecolor{currentstroke}{rgb}{0.000000,0.000000,0.000000}%
\pgfsetstrokecolor{currentstroke}%
\pgfsetdash{}{0pt}%
\pgfpathmoveto{\pgfqpoint{1.579875in}{1.079358in}}%
\pgfpathcurveto{\pgfqpoint{1.590925in}{1.079358in}}{\pgfqpoint{1.601524in}{1.083749in}}{\pgfqpoint{1.609338in}{1.091562in}}%
\pgfpathcurveto{\pgfqpoint{1.617151in}{1.099376in}}{\pgfqpoint{1.621542in}{1.109975in}}{\pgfqpoint{1.621542in}{1.121025in}}%
\pgfpathcurveto{\pgfqpoint{1.621542in}{1.132075in}}{\pgfqpoint{1.617151in}{1.142674in}}{\pgfqpoint{1.609338in}{1.150488in}}%
\pgfpathcurveto{\pgfqpoint{1.601524in}{1.158301in}}{\pgfqpoint{1.590925in}{1.162692in}}{\pgfqpoint{1.579875in}{1.162692in}}%
\pgfpathcurveto{\pgfqpoint{1.568825in}{1.162692in}}{\pgfqpoint{1.558226in}{1.158301in}}{\pgfqpoint{1.550412in}{1.150488in}}%
\pgfpathcurveto{\pgfqpoint{1.542599in}{1.142674in}}{\pgfqpoint{1.538208in}{1.132075in}}{\pgfqpoint{1.538208in}{1.121025in}}%
\pgfpathcurveto{\pgfqpoint{1.538208in}{1.109975in}}{\pgfqpoint{1.542599in}{1.099376in}}{\pgfqpoint{1.550412in}{1.091562in}}%
\pgfpathcurveto{\pgfqpoint{1.558226in}{1.083749in}}{\pgfqpoint{1.568825in}{1.079358in}}{\pgfqpoint{1.579875in}{1.079358in}}%
\pgfpathclose%
\pgfusepath{stroke,fill}%
\end{pgfscope}%
\begin{pgfscope}%
\pgfpathrectangle{\pgfqpoint{0.375000in}{0.330000in}}{\pgfqpoint{2.325000in}{2.310000in}}%
\pgfusepath{clip}%
\pgfsetbuttcap%
\pgfsetroundjoin%
\definecolor{currentfill}{rgb}{0.000000,0.000000,0.000000}%
\pgfsetfillcolor{currentfill}%
\pgfsetlinewidth{1.003750pt}%
\definecolor{currentstroke}{rgb}{0.000000,0.000000,0.000000}%
\pgfsetstrokecolor{currentstroke}%
\pgfsetdash{}{0pt}%
\pgfpathmoveto{\pgfqpoint{1.579875in}{0.964064in}}%
\pgfpathcurveto{\pgfqpoint{1.590925in}{0.964064in}}{\pgfqpoint{1.601524in}{0.968455in}}{\pgfqpoint{1.609338in}{0.976268in}}%
\pgfpathcurveto{\pgfqpoint{1.617151in}{0.984082in}}{\pgfqpoint{1.621542in}{0.994681in}}{\pgfqpoint{1.621542in}{1.005731in}}%
\pgfpathcurveto{\pgfqpoint{1.621542in}{1.016781in}}{\pgfqpoint{1.617151in}{1.027380in}}{\pgfqpoint{1.609338in}{1.035194in}}%
\pgfpathcurveto{\pgfqpoint{1.601524in}{1.043007in}}{\pgfqpoint{1.590925in}{1.047398in}}{\pgfqpoint{1.579875in}{1.047398in}}%
\pgfpathcurveto{\pgfqpoint{1.568825in}{1.047398in}}{\pgfqpoint{1.558226in}{1.043007in}}{\pgfqpoint{1.550412in}{1.035194in}}%
\pgfpathcurveto{\pgfqpoint{1.542599in}{1.027380in}}{\pgfqpoint{1.538208in}{1.016781in}}{\pgfqpoint{1.538208in}{1.005731in}}%
\pgfpathcurveto{\pgfqpoint{1.538208in}{0.994681in}}{\pgfqpoint{1.542599in}{0.984082in}}{\pgfqpoint{1.550412in}{0.976268in}}%
\pgfpathcurveto{\pgfqpoint{1.558226in}{0.968455in}}{\pgfqpoint{1.568825in}{0.964064in}}{\pgfqpoint{1.579875in}{0.964064in}}%
\pgfpathclose%
\pgfusepath{stroke,fill}%
\end{pgfscope}%
\begin{pgfscope}%
\pgfpathrectangle{\pgfqpoint{0.375000in}{0.330000in}}{\pgfqpoint{2.325000in}{2.310000in}}%
\pgfusepath{clip}%
\pgfsetbuttcap%
\pgfsetroundjoin%
\definecolor{currentfill}{rgb}{0.000000,0.000000,0.000000}%
\pgfsetfillcolor{currentfill}%
\pgfsetlinewidth{1.003750pt}%
\definecolor{currentstroke}{rgb}{0.000000,0.000000,0.000000}%
\pgfsetstrokecolor{currentstroke}%
\pgfsetdash{}{0pt}%
\pgfpathmoveto{\pgfqpoint{1.579875in}{1.049018in}}%
\pgfpathcurveto{\pgfqpoint{1.590925in}{1.049018in}}{\pgfqpoint{1.601524in}{1.053408in}}{\pgfqpoint{1.609338in}{1.061222in}}%
\pgfpathcurveto{\pgfqpoint{1.617151in}{1.069035in}}{\pgfqpoint{1.621542in}{1.079634in}}{\pgfqpoint{1.621542in}{1.090684in}}%
\pgfpathcurveto{\pgfqpoint{1.621542in}{1.101735in}}{\pgfqpoint{1.617151in}{1.112334in}}{\pgfqpoint{1.609338in}{1.120147in}}%
\pgfpathcurveto{\pgfqpoint{1.601524in}{1.127961in}}{\pgfqpoint{1.590925in}{1.132351in}}{\pgfqpoint{1.579875in}{1.132351in}}%
\pgfpathcurveto{\pgfqpoint{1.568825in}{1.132351in}}{\pgfqpoint{1.558226in}{1.127961in}}{\pgfqpoint{1.550412in}{1.120147in}}%
\pgfpathcurveto{\pgfqpoint{1.542599in}{1.112334in}}{\pgfqpoint{1.538208in}{1.101735in}}{\pgfqpoint{1.538208in}{1.090684in}}%
\pgfpathcurveto{\pgfqpoint{1.538208in}{1.079634in}}{\pgfqpoint{1.542599in}{1.069035in}}{\pgfqpoint{1.550412in}{1.061222in}}%
\pgfpathcurveto{\pgfqpoint{1.558226in}{1.053408in}}{\pgfqpoint{1.568825in}{1.049018in}}{\pgfqpoint{1.579875in}{1.049018in}}%
\pgfpathclose%
\pgfusepath{stroke,fill}%
\end{pgfscope}%
\begin{pgfscope}%
\pgfpathrectangle{\pgfqpoint{0.375000in}{0.330000in}}{\pgfqpoint{2.325000in}{2.310000in}}%
\pgfusepath{clip}%
\pgfsetbuttcap%
\pgfsetroundjoin%
\definecolor{currentfill}{rgb}{0.000000,0.000000,0.000000}%
\pgfsetfillcolor{currentfill}%
\pgfsetlinewidth{1.003750pt}%
\definecolor{currentstroke}{rgb}{0.000000,0.000000,0.000000}%
\pgfsetstrokecolor{currentstroke}%
\pgfsetdash{}{0pt}%
\pgfpathmoveto{\pgfqpoint{1.579875in}{1.097563in}}%
\pgfpathcurveto{\pgfqpoint{1.590925in}{1.097563in}}{\pgfqpoint{1.601524in}{1.101953in}}{\pgfqpoint{1.609338in}{1.109766in}}%
\pgfpathcurveto{\pgfqpoint{1.617151in}{1.117580in}}{\pgfqpoint{1.621542in}{1.128179in}}{\pgfqpoint{1.621542in}{1.139229in}}%
\pgfpathcurveto{\pgfqpoint{1.621542in}{1.150279in}}{\pgfqpoint{1.617151in}{1.160878in}}{\pgfqpoint{1.609338in}{1.168692in}}%
\pgfpathcurveto{\pgfqpoint{1.601524in}{1.176506in}}{\pgfqpoint{1.590925in}{1.180896in}}{\pgfqpoint{1.579875in}{1.180896in}}%
\pgfpathcurveto{\pgfqpoint{1.568825in}{1.180896in}}{\pgfqpoint{1.558226in}{1.176506in}}{\pgfqpoint{1.550412in}{1.168692in}}%
\pgfpathcurveto{\pgfqpoint{1.542599in}{1.160878in}}{\pgfqpoint{1.538208in}{1.150279in}}{\pgfqpoint{1.538208in}{1.139229in}}%
\pgfpathcurveto{\pgfqpoint{1.538208in}{1.128179in}}{\pgfqpoint{1.542599in}{1.117580in}}{\pgfqpoint{1.550412in}{1.109766in}}%
\pgfpathcurveto{\pgfqpoint{1.558226in}{1.101953in}}{\pgfqpoint{1.568825in}{1.097563in}}{\pgfqpoint{1.579875in}{1.097563in}}%
\pgfpathclose%
\pgfusepath{stroke,fill}%
\end{pgfscope}%
\begin{pgfscope}%
\pgfpathrectangle{\pgfqpoint{0.375000in}{0.330000in}}{\pgfqpoint{2.325000in}{2.310000in}}%
\pgfusepath{clip}%
\pgfsetbuttcap%
\pgfsetroundjoin%
\definecolor{currentfill}{rgb}{0.000000,0.000000,0.000000}%
\pgfsetfillcolor{currentfill}%
\pgfsetlinewidth{1.003750pt}%
\definecolor{currentstroke}{rgb}{0.000000,0.000000,0.000000}%
\pgfsetstrokecolor{currentstroke}%
\pgfsetdash{}{0pt}%
\pgfpathmoveto{\pgfqpoint{1.579875in}{0.994405in}}%
\pgfpathcurveto{\pgfqpoint{1.590925in}{0.994405in}}{\pgfqpoint{1.601524in}{0.998795in}}{\pgfqpoint{1.609338in}{1.006609in}}%
\pgfpathcurveto{\pgfqpoint{1.617151in}{1.014422in}}{\pgfqpoint{1.621542in}{1.025021in}}{\pgfqpoint{1.621542in}{1.036071in}}%
\pgfpathcurveto{\pgfqpoint{1.621542in}{1.047122in}}{\pgfqpoint{1.617151in}{1.057721in}}{\pgfqpoint{1.609338in}{1.065534in}}%
\pgfpathcurveto{\pgfqpoint{1.601524in}{1.073348in}}{\pgfqpoint{1.590925in}{1.077738in}}{\pgfqpoint{1.579875in}{1.077738in}}%
\pgfpathcurveto{\pgfqpoint{1.568825in}{1.077738in}}{\pgfqpoint{1.558226in}{1.073348in}}{\pgfqpoint{1.550412in}{1.065534in}}%
\pgfpathcurveto{\pgfqpoint{1.542599in}{1.057721in}}{\pgfqpoint{1.538208in}{1.047122in}}{\pgfqpoint{1.538208in}{1.036071in}}%
\pgfpathcurveto{\pgfqpoint{1.538208in}{1.025021in}}{\pgfqpoint{1.542599in}{1.014422in}}{\pgfqpoint{1.550412in}{1.006609in}}%
\pgfpathcurveto{\pgfqpoint{1.558226in}{0.998795in}}{\pgfqpoint{1.568825in}{0.994405in}}{\pgfqpoint{1.579875in}{0.994405in}}%
\pgfpathclose%
\pgfusepath{stroke,fill}%
\end{pgfscope}%
\begin{pgfscope}%
\pgfpathrectangle{\pgfqpoint{0.375000in}{0.330000in}}{\pgfqpoint{2.325000in}{2.310000in}}%
\pgfusepath{clip}%
\pgfsetbuttcap%
\pgfsetroundjoin%
\definecolor{currentfill}{rgb}{0.000000,0.000000,0.000000}%
\pgfsetfillcolor{currentfill}%
\pgfsetlinewidth{1.003750pt}%
\definecolor{currentstroke}{rgb}{0.000000,0.000000,0.000000}%
\pgfsetstrokecolor{currentstroke}%
\pgfsetdash{}{0pt}%
\pgfpathmoveto{\pgfqpoint{1.579875in}{1.182516in}}%
\pgfpathcurveto{\pgfqpoint{1.590925in}{1.182516in}}{\pgfqpoint{1.601524in}{1.186906in}}{\pgfqpoint{1.609338in}{1.194720in}}%
\pgfpathcurveto{\pgfqpoint{1.617151in}{1.202534in}}{\pgfqpoint{1.621542in}{1.213133in}}{\pgfqpoint{1.621542in}{1.224183in}}%
\pgfpathcurveto{\pgfqpoint{1.621542in}{1.235233in}}{\pgfqpoint{1.617151in}{1.245832in}}{\pgfqpoint{1.609338in}{1.253646in}}%
\pgfpathcurveto{\pgfqpoint{1.601524in}{1.261459in}}{\pgfqpoint{1.590925in}{1.265849in}}{\pgfqpoint{1.579875in}{1.265849in}}%
\pgfpathcurveto{\pgfqpoint{1.568825in}{1.265849in}}{\pgfqpoint{1.558226in}{1.261459in}}{\pgfqpoint{1.550412in}{1.253646in}}%
\pgfpathcurveto{\pgfqpoint{1.542599in}{1.245832in}}{\pgfqpoint{1.538208in}{1.235233in}}{\pgfqpoint{1.538208in}{1.224183in}}%
\pgfpathcurveto{\pgfqpoint{1.538208in}{1.213133in}}{\pgfqpoint{1.542599in}{1.202534in}}{\pgfqpoint{1.550412in}{1.194720in}}%
\pgfpathcurveto{\pgfqpoint{1.558226in}{1.186906in}}{\pgfqpoint{1.568825in}{1.182516in}}{\pgfqpoint{1.579875in}{1.182516in}}%
\pgfpathclose%
\pgfusepath{stroke,fill}%
\end{pgfscope}%
\begin{pgfscope}%
\pgfpathrectangle{\pgfqpoint{0.375000in}{0.330000in}}{\pgfqpoint{2.325000in}{2.310000in}}%
\pgfusepath{clip}%
\pgfsetbuttcap%
\pgfsetroundjoin%
\definecolor{currentfill}{rgb}{0.000000,0.000000,0.000000}%
\pgfsetfillcolor{currentfill}%
\pgfsetlinewidth{1.003750pt}%
\definecolor{currentstroke}{rgb}{0.000000,0.000000,0.000000}%
\pgfsetstrokecolor{currentstroke}%
\pgfsetdash{}{0pt}%
\pgfpathmoveto{\pgfqpoint{1.579875in}{0.945860in}}%
\pgfpathcurveto{\pgfqpoint{1.590925in}{0.945860in}}{\pgfqpoint{1.601524in}{0.950250in}}{\pgfqpoint{1.609338in}{0.958064in}}%
\pgfpathcurveto{\pgfqpoint{1.617151in}{0.965877in}}{\pgfqpoint{1.621542in}{0.976476in}}{\pgfqpoint{1.621542in}{0.987527in}}%
\pgfpathcurveto{\pgfqpoint{1.621542in}{0.998577in}}{\pgfqpoint{1.617151in}{1.009176in}}{\pgfqpoint{1.609338in}{1.016989in}}%
\pgfpathcurveto{\pgfqpoint{1.601524in}{1.024803in}}{\pgfqpoint{1.590925in}{1.029193in}}{\pgfqpoint{1.579875in}{1.029193in}}%
\pgfpathcurveto{\pgfqpoint{1.568825in}{1.029193in}}{\pgfqpoint{1.558226in}{1.024803in}}{\pgfqpoint{1.550412in}{1.016989in}}%
\pgfpathcurveto{\pgfqpoint{1.542599in}{1.009176in}}{\pgfqpoint{1.538208in}{0.998577in}}{\pgfqpoint{1.538208in}{0.987527in}}%
\pgfpathcurveto{\pgfqpoint{1.538208in}{0.976476in}}{\pgfqpoint{1.542599in}{0.965877in}}{\pgfqpoint{1.550412in}{0.958064in}}%
\pgfpathcurveto{\pgfqpoint{1.558226in}{0.950250in}}{\pgfqpoint{1.568825in}{0.945860in}}{\pgfqpoint{1.579875in}{0.945860in}}%
\pgfpathclose%
\pgfusepath{stroke,fill}%
\end{pgfscope}%
\begin{pgfscope}%
\pgfpathrectangle{\pgfqpoint{0.375000in}{0.330000in}}{\pgfqpoint{2.325000in}{2.310000in}}%
\pgfusepath{clip}%
\pgfsetbuttcap%
\pgfsetroundjoin%
\definecolor{currentfill}{rgb}{0.000000,0.000000,0.000000}%
\pgfsetfillcolor{currentfill}%
\pgfsetlinewidth{1.003750pt}%
\definecolor{currentstroke}{rgb}{0.000000,0.000000,0.000000}%
\pgfsetstrokecolor{currentstroke}%
\pgfsetdash{}{0pt}%
\pgfpathmoveto{\pgfqpoint{1.579875in}{1.042950in}}%
\pgfpathcurveto{\pgfqpoint{1.590925in}{1.042950in}}{\pgfqpoint{1.601524in}{1.047340in}}{\pgfqpoint{1.609338in}{1.055154in}}%
\pgfpathcurveto{\pgfqpoint{1.617151in}{1.062967in}}{\pgfqpoint{1.621542in}{1.073566in}}{\pgfqpoint{1.621542in}{1.084616in}}%
\pgfpathcurveto{\pgfqpoint{1.621542in}{1.095666in}}{\pgfqpoint{1.617151in}{1.106265in}}{\pgfqpoint{1.609338in}{1.114079in}}%
\pgfpathcurveto{\pgfqpoint{1.601524in}{1.121893in}}{\pgfqpoint{1.590925in}{1.126283in}}{\pgfqpoint{1.579875in}{1.126283in}}%
\pgfpathcurveto{\pgfqpoint{1.568825in}{1.126283in}}{\pgfqpoint{1.558226in}{1.121893in}}{\pgfqpoint{1.550412in}{1.114079in}}%
\pgfpathcurveto{\pgfqpoint{1.542599in}{1.106265in}}{\pgfqpoint{1.538208in}{1.095666in}}{\pgfqpoint{1.538208in}{1.084616in}}%
\pgfpathcurveto{\pgfqpoint{1.538208in}{1.073566in}}{\pgfqpoint{1.542599in}{1.062967in}}{\pgfqpoint{1.550412in}{1.055154in}}%
\pgfpathcurveto{\pgfqpoint{1.558226in}{1.047340in}}{\pgfqpoint{1.568825in}{1.042950in}}{\pgfqpoint{1.579875in}{1.042950in}}%
\pgfpathclose%
\pgfusepath{stroke,fill}%
\end{pgfscope}%
\begin{pgfscope}%
\pgfpathrectangle{\pgfqpoint{0.375000in}{0.330000in}}{\pgfqpoint{2.325000in}{2.310000in}}%
\pgfusepath{clip}%
\pgfsetbuttcap%
\pgfsetroundjoin%
\definecolor{currentfill}{rgb}{0.000000,0.000000,0.000000}%
\pgfsetfillcolor{currentfill}%
\pgfsetlinewidth{1.003750pt}%
\definecolor{currentstroke}{rgb}{0.000000,0.000000,0.000000}%
\pgfsetstrokecolor{currentstroke}%
\pgfsetdash{}{0pt}%
\pgfpathmoveto{\pgfqpoint{1.579875in}{1.049018in}}%
\pgfpathcurveto{\pgfqpoint{1.590925in}{1.049018in}}{\pgfqpoint{1.601524in}{1.053408in}}{\pgfqpoint{1.609338in}{1.061222in}}%
\pgfpathcurveto{\pgfqpoint{1.617151in}{1.069035in}}{\pgfqpoint{1.621542in}{1.079634in}}{\pgfqpoint{1.621542in}{1.090684in}}%
\pgfpathcurveto{\pgfqpoint{1.621542in}{1.101735in}}{\pgfqpoint{1.617151in}{1.112334in}}{\pgfqpoint{1.609338in}{1.120147in}}%
\pgfpathcurveto{\pgfqpoint{1.601524in}{1.127961in}}{\pgfqpoint{1.590925in}{1.132351in}}{\pgfqpoint{1.579875in}{1.132351in}}%
\pgfpathcurveto{\pgfqpoint{1.568825in}{1.132351in}}{\pgfqpoint{1.558226in}{1.127961in}}{\pgfqpoint{1.550412in}{1.120147in}}%
\pgfpathcurveto{\pgfqpoint{1.542599in}{1.112334in}}{\pgfqpoint{1.538208in}{1.101735in}}{\pgfqpoint{1.538208in}{1.090684in}}%
\pgfpathcurveto{\pgfqpoint{1.538208in}{1.079634in}}{\pgfqpoint{1.542599in}{1.069035in}}{\pgfqpoint{1.550412in}{1.061222in}}%
\pgfpathcurveto{\pgfqpoint{1.558226in}{1.053408in}}{\pgfqpoint{1.568825in}{1.049018in}}{\pgfqpoint{1.579875in}{1.049018in}}%
\pgfpathclose%
\pgfusepath{stroke,fill}%
\end{pgfscope}%
\begin{pgfscope}%
\pgfpathrectangle{\pgfqpoint{0.375000in}{0.330000in}}{\pgfqpoint{2.325000in}{2.310000in}}%
\pgfusepath{clip}%
\pgfsetbuttcap%
\pgfsetroundjoin%
\definecolor{currentfill}{rgb}{0.000000,0.000000,0.000000}%
\pgfsetfillcolor{currentfill}%
\pgfsetlinewidth{1.003750pt}%
\definecolor{currentstroke}{rgb}{0.000000,0.000000,0.000000}%
\pgfsetstrokecolor{currentstroke}%
\pgfsetdash{}{0pt}%
\pgfpathmoveto{\pgfqpoint{1.579875in}{1.085426in}}%
\pgfpathcurveto{\pgfqpoint{1.590925in}{1.085426in}}{\pgfqpoint{1.601524in}{1.089817in}}{\pgfqpoint{1.609338in}{1.097630in}}%
\pgfpathcurveto{\pgfqpoint{1.617151in}{1.105444in}}{\pgfqpoint{1.621542in}{1.116043in}}{\pgfqpoint{1.621542in}{1.127093in}}%
\pgfpathcurveto{\pgfqpoint{1.621542in}{1.138143in}}{\pgfqpoint{1.617151in}{1.148742in}}{\pgfqpoint{1.609338in}{1.156556in}}%
\pgfpathcurveto{\pgfqpoint{1.601524in}{1.164369in}}{\pgfqpoint{1.590925in}{1.168760in}}{\pgfqpoint{1.579875in}{1.168760in}}%
\pgfpathcurveto{\pgfqpoint{1.568825in}{1.168760in}}{\pgfqpoint{1.558226in}{1.164369in}}{\pgfqpoint{1.550412in}{1.156556in}}%
\pgfpathcurveto{\pgfqpoint{1.542599in}{1.148742in}}{\pgfqpoint{1.538208in}{1.138143in}}{\pgfqpoint{1.538208in}{1.127093in}}%
\pgfpathcurveto{\pgfqpoint{1.538208in}{1.116043in}}{\pgfqpoint{1.542599in}{1.105444in}}{\pgfqpoint{1.550412in}{1.097630in}}%
\pgfpathcurveto{\pgfqpoint{1.558226in}{1.089817in}}{\pgfqpoint{1.568825in}{1.085426in}}{\pgfqpoint{1.579875in}{1.085426in}}%
\pgfpathclose%
\pgfusepath{stroke,fill}%
\end{pgfscope}%
\begin{pgfscope}%
\pgfpathrectangle{\pgfqpoint{0.375000in}{0.330000in}}{\pgfqpoint{2.325000in}{2.310000in}}%
\pgfusepath{clip}%
\pgfsetbuttcap%
\pgfsetroundjoin%
\definecolor{currentfill}{rgb}{0.000000,0.000000,0.000000}%
\pgfsetfillcolor{currentfill}%
\pgfsetlinewidth{1.003750pt}%
\definecolor{currentstroke}{rgb}{0.000000,0.000000,0.000000}%
\pgfsetstrokecolor{currentstroke}%
\pgfsetdash{}{0pt}%
\pgfpathmoveto{\pgfqpoint{1.579875in}{1.097563in}}%
\pgfpathcurveto{\pgfqpoint{1.590925in}{1.097563in}}{\pgfqpoint{1.601524in}{1.101953in}}{\pgfqpoint{1.609338in}{1.109766in}}%
\pgfpathcurveto{\pgfqpoint{1.617151in}{1.117580in}}{\pgfqpoint{1.621542in}{1.128179in}}{\pgfqpoint{1.621542in}{1.139229in}}%
\pgfpathcurveto{\pgfqpoint{1.621542in}{1.150279in}}{\pgfqpoint{1.617151in}{1.160878in}}{\pgfqpoint{1.609338in}{1.168692in}}%
\pgfpathcurveto{\pgfqpoint{1.601524in}{1.176506in}}{\pgfqpoint{1.590925in}{1.180896in}}{\pgfqpoint{1.579875in}{1.180896in}}%
\pgfpathcurveto{\pgfqpoint{1.568825in}{1.180896in}}{\pgfqpoint{1.558226in}{1.176506in}}{\pgfqpoint{1.550412in}{1.168692in}}%
\pgfpathcurveto{\pgfqpoint{1.542599in}{1.160878in}}{\pgfqpoint{1.538208in}{1.150279in}}{\pgfqpoint{1.538208in}{1.139229in}}%
\pgfpathcurveto{\pgfqpoint{1.538208in}{1.128179in}}{\pgfqpoint{1.542599in}{1.117580in}}{\pgfqpoint{1.550412in}{1.109766in}}%
\pgfpathcurveto{\pgfqpoint{1.558226in}{1.101953in}}{\pgfqpoint{1.568825in}{1.097563in}}{\pgfqpoint{1.579875in}{1.097563in}}%
\pgfpathclose%
\pgfusepath{stroke,fill}%
\end{pgfscope}%
\begin{pgfscope}%
\pgfpathrectangle{\pgfqpoint{0.375000in}{0.330000in}}{\pgfqpoint{2.325000in}{2.310000in}}%
\pgfusepath{clip}%
\pgfsetbuttcap%
\pgfsetroundjoin%
\definecolor{currentfill}{rgb}{0.000000,0.000000,0.000000}%
\pgfsetfillcolor{currentfill}%
\pgfsetlinewidth{1.003750pt}%
\definecolor{currentstroke}{rgb}{0.000000,0.000000,0.000000}%
\pgfsetstrokecolor{currentstroke}%
\pgfsetdash{}{0pt}%
\pgfpathmoveto{\pgfqpoint{1.579875in}{1.182516in}}%
\pgfpathcurveto{\pgfqpoint{1.590925in}{1.182516in}}{\pgfqpoint{1.601524in}{1.186906in}}{\pgfqpoint{1.609338in}{1.194720in}}%
\pgfpathcurveto{\pgfqpoint{1.617151in}{1.202534in}}{\pgfqpoint{1.621542in}{1.213133in}}{\pgfqpoint{1.621542in}{1.224183in}}%
\pgfpathcurveto{\pgfqpoint{1.621542in}{1.235233in}}{\pgfqpoint{1.617151in}{1.245832in}}{\pgfqpoint{1.609338in}{1.253646in}}%
\pgfpathcurveto{\pgfqpoint{1.601524in}{1.261459in}}{\pgfqpoint{1.590925in}{1.265849in}}{\pgfqpoint{1.579875in}{1.265849in}}%
\pgfpathcurveto{\pgfqpoint{1.568825in}{1.265849in}}{\pgfqpoint{1.558226in}{1.261459in}}{\pgfqpoint{1.550412in}{1.253646in}}%
\pgfpathcurveto{\pgfqpoint{1.542599in}{1.245832in}}{\pgfqpoint{1.538208in}{1.235233in}}{\pgfqpoint{1.538208in}{1.224183in}}%
\pgfpathcurveto{\pgfqpoint{1.538208in}{1.213133in}}{\pgfqpoint{1.542599in}{1.202534in}}{\pgfqpoint{1.550412in}{1.194720in}}%
\pgfpathcurveto{\pgfqpoint{1.558226in}{1.186906in}}{\pgfqpoint{1.568825in}{1.182516in}}{\pgfqpoint{1.579875in}{1.182516in}}%
\pgfpathclose%
\pgfusepath{stroke,fill}%
\end{pgfscope}%
\begin{pgfscope}%
\pgfpathrectangle{\pgfqpoint{0.375000in}{0.330000in}}{\pgfqpoint{2.325000in}{2.310000in}}%
\pgfusepath{clip}%
\pgfsetbuttcap%
\pgfsetroundjoin%
\definecolor{currentfill}{rgb}{0.000000,0.000000,0.000000}%
\pgfsetfillcolor{currentfill}%
\pgfsetlinewidth{1.003750pt}%
\definecolor{currentstroke}{rgb}{0.000000,0.000000,0.000000}%
\pgfsetstrokecolor{currentstroke}%
\pgfsetdash{}{0pt}%
\pgfpathmoveto{\pgfqpoint{1.579875in}{1.006541in}}%
\pgfpathcurveto{\pgfqpoint{1.590925in}{1.006541in}}{\pgfqpoint{1.601524in}{1.010931in}}{\pgfqpoint{1.609338in}{1.018745in}}%
\pgfpathcurveto{\pgfqpoint{1.617151in}{1.026559in}}{\pgfqpoint{1.621542in}{1.037158in}}{\pgfqpoint{1.621542in}{1.048208in}}%
\pgfpathcurveto{\pgfqpoint{1.621542in}{1.059258in}}{\pgfqpoint{1.617151in}{1.069857in}}{\pgfqpoint{1.609338in}{1.077670in}}%
\pgfpathcurveto{\pgfqpoint{1.601524in}{1.085484in}}{\pgfqpoint{1.590925in}{1.089874in}}{\pgfqpoint{1.579875in}{1.089874in}}%
\pgfpathcurveto{\pgfqpoint{1.568825in}{1.089874in}}{\pgfqpoint{1.558226in}{1.085484in}}{\pgfqpoint{1.550412in}{1.077670in}}%
\pgfpathcurveto{\pgfqpoint{1.542599in}{1.069857in}}{\pgfqpoint{1.538208in}{1.059258in}}{\pgfqpoint{1.538208in}{1.048208in}}%
\pgfpathcurveto{\pgfqpoint{1.538208in}{1.037158in}}{\pgfqpoint{1.542599in}{1.026559in}}{\pgfqpoint{1.550412in}{1.018745in}}%
\pgfpathcurveto{\pgfqpoint{1.558226in}{1.010931in}}{\pgfqpoint{1.568825in}{1.006541in}}{\pgfqpoint{1.579875in}{1.006541in}}%
\pgfpathclose%
\pgfusepath{stroke,fill}%
\end{pgfscope}%
\begin{pgfscope}%
\pgfpathrectangle{\pgfqpoint{0.375000in}{0.330000in}}{\pgfqpoint{2.325000in}{2.310000in}}%
\pgfusepath{clip}%
\pgfsetbuttcap%
\pgfsetroundjoin%
\definecolor{currentfill}{rgb}{0.000000,0.000000,0.000000}%
\pgfsetfillcolor{currentfill}%
\pgfsetlinewidth{1.003750pt}%
\definecolor{currentstroke}{rgb}{0.000000,0.000000,0.000000}%
\pgfsetstrokecolor{currentstroke}%
\pgfsetdash{}{0pt}%
\pgfpathmoveto{\pgfqpoint{1.579875in}{0.982269in}}%
\pgfpathcurveto{\pgfqpoint{1.590925in}{0.982269in}}{\pgfqpoint{1.601524in}{0.986659in}}{\pgfqpoint{1.609338in}{0.994472in}}%
\pgfpathcurveto{\pgfqpoint{1.617151in}{1.002286in}}{\pgfqpoint{1.621542in}{1.012885in}}{\pgfqpoint{1.621542in}{1.023935in}}%
\pgfpathcurveto{\pgfqpoint{1.621542in}{1.034985in}}{\pgfqpoint{1.617151in}{1.045584in}}{\pgfqpoint{1.609338in}{1.053398in}}%
\pgfpathcurveto{\pgfqpoint{1.601524in}{1.061212in}}{\pgfqpoint{1.590925in}{1.065602in}}{\pgfqpoint{1.579875in}{1.065602in}}%
\pgfpathcurveto{\pgfqpoint{1.568825in}{1.065602in}}{\pgfqpoint{1.558226in}{1.061212in}}{\pgfqpoint{1.550412in}{1.053398in}}%
\pgfpathcurveto{\pgfqpoint{1.542599in}{1.045584in}}{\pgfqpoint{1.538208in}{1.034985in}}{\pgfqpoint{1.538208in}{1.023935in}}%
\pgfpathcurveto{\pgfqpoint{1.538208in}{1.012885in}}{\pgfqpoint{1.542599in}{1.002286in}}{\pgfqpoint{1.550412in}{0.994472in}}%
\pgfpathcurveto{\pgfqpoint{1.558226in}{0.986659in}}{\pgfqpoint{1.568825in}{0.982269in}}{\pgfqpoint{1.579875in}{0.982269in}}%
\pgfpathclose%
\pgfusepath{stroke,fill}%
\end{pgfscope}%
\begin{pgfscope}%
\pgfpathrectangle{\pgfqpoint{0.375000in}{0.330000in}}{\pgfqpoint{2.325000in}{2.310000in}}%
\pgfusepath{clip}%
\pgfsetbuttcap%
\pgfsetroundjoin%
\definecolor{currentfill}{rgb}{0.000000,0.000000,0.000000}%
\pgfsetfillcolor{currentfill}%
\pgfsetlinewidth{1.003750pt}%
\definecolor{currentstroke}{rgb}{0.000000,0.000000,0.000000}%
\pgfsetstrokecolor{currentstroke}%
\pgfsetdash{}{0pt}%
\pgfpathmoveto{\pgfqpoint{1.579875in}{1.055086in}}%
\pgfpathcurveto{\pgfqpoint{1.590925in}{1.055086in}}{\pgfqpoint{1.601524in}{1.059476in}}{\pgfqpoint{1.609338in}{1.067290in}}%
\pgfpathcurveto{\pgfqpoint{1.617151in}{1.075103in}}{\pgfqpoint{1.621542in}{1.085702in}}{\pgfqpoint{1.621542in}{1.096753in}}%
\pgfpathcurveto{\pgfqpoint{1.621542in}{1.107803in}}{\pgfqpoint{1.617151in}{1.118402in}}{\pgfqpoint{1.609338in}{1.126215in}}%
\pgfpathcurveto{\pgfqpoint{1.601524in}{1.134029in}}{\pgfqpoint{1.590925in}{1.138419in}}{\pgfqpoint{1.579875in}{1.138419in}}%
\pgfpathcurveto{\pgfqpoint{1.568825in}{1.138419in}}{\pgfqpoint{1.558226in}{1.134029in}}{\pgfqpoint{1.550412in}{1.126215in}}%
\pgfpathcurveto{\pgfqpoint{1.542599in}{1.118402in}}{\pgfqpoint{1.538208in}{1.107803in}}{\pgfqpoint{1.538208in}{1.096753in}}%
\pgfpathcurveto{\pgfqpoint{1.538208in}{1.085702in}}{\pgfqpoint{1.542599in}{1.075103in}}{\pgfqpoint{1.550412in}{1.067290in}}%
\pgfpathcurveto{\pgfqpoint{1.558226in}{1.059476in}}{\pgfqpoint{1.568825in}{1.055086in}}{\pgfqpoint{1.579875in}{1.055086in}}%
\pgfpathclose%
\pgfusepath{stroke,fill}%
\end{pgfscope}%
\begin{pgfscope}%
\pgfpathrectangle{\pgfqpoint{0.375000in}{0.330000in}}{\pgfqpoint{2.325000in}{2.310000in}}%
\pgfusepath{clip}%
\pgfsetbuttcap%
\pgfsetroundjoin%
\definecolor{currentfill}{rgb}{0.000000,0.000000,0.000000}%
\pgfsetfillcolor{currentfill}%
\pgfsetlinewidth{1.003750pt}%
\definecolor{currentstroke}{rgb}{0.000000,0.000000,0.000000}%
\pgfsetstrokecolor{currentstroke}%
\pgfsetdash{}{0pt}%
\pgfpathmoveto{\pgfqpoint{1.579875in}{0.994405in}}%
\pgfpathcurveto{\pgfqpoint{1.590925in}{0.994405in}}{\pgfqpoint{1.601524in}{0.998795in}}{\pgfqpoint{1.609338in}{1.006609in}}%
\pgfpathcurveto{\pgfqpoint{1.617151in}{1.014422in}}{\pgfqpoint{1.621542in}{1.025021in}}{\pgfqpoint{1.621542in}{1.036071in}}%
\pgfpathcurveto{\pgfqpoint{1.621542in}{1.047122in}}{\pgfqpoint{1.617151in}{1.057721in}}{\pgfqpoint{1.609338in}{1.065534in}}%
\pgfpathcurveto{\pgfqpoint{1.601524in}{1.073348in}}{\pgfqpoint{1.590925in}{1.077738in}}{\pgfqpoint{1.579875in}{1.077738in}}%
\pgfpathcurveto{\pgfqpoint{1.568825in}{1.077738in}}{\pgfqpoint{1.558226in}{1.073348in}}{\pgfqpoint{1.550412in}{1.065534in}}%
\pgfpathcurveto{\pgfqpoint{1.542599in}{1.057721in}}{\pgfqpoint{1.538208in}{1.047122in}}{\pgfqpoint{1.538208in}{1.036071in}}%
\pgfpathcurveto{\pgfqpoint{1.538208in}{1.025021in}}{\pgfqpoint{1.542599in}{1.014422in}}{\pgfqpoint{1.550412in}{1.006609in}}%
\pgfpathcurveto{\pgfqpoint{1.558226in}{0.998795in}}{\pgfqpoint{1.568825in}{0.994405in}}{\pgfqpoint{1.579875in}{0.994405in}}%
\pgfpathclose%
\pgfusepath{stroke,fill}%
\end{pgfscope}%
\begin{pgfscope}%
\pgfpathrectangle{\pgfqpoint{0.375000in}{0.330000in}}{\pgfqpoint{2.325000in}{2.310000in}}%
\pgfusepath{clip}%
\pgfsetbuttcap%
\pgfsetroundjoin%
\definecolor{currentfill}{rgb}{0.000000,0.000000,0.000000}%
\pgfsetfillcolor{currentfill}%
\pgfsetlinewidth{1.003750pt}%
\definecolor{currentstroke}{rgb}{0.000000,0.000000,0.000000}%
\pgfsetstrokecolor{currentstroke}%
\pgfsetdash{}{0pt}%
\pgfpathmoveto{\pgfqpoint{1.579875in}{0.945860in}}%
\pgfpathcurveto{\pgfqpoint{1.590925in}{0.945860in}}{\pgfqpoint{1.601524in}{0.950250in}}{\pgfqpoint{1.609338in}{0.958064in}}%
\pgfpathcurveto{\pgfqpoint{1.617151in}{0.965877in}}{\pgfqpoint{1.621542in}{0.976476in}}{\pgfqpoint{1.621542in}{0.987527in}}%
\pgfpathcurveto{\pgfqpoint{1.621542in}{0.998577in}}{\pgfqpoint{1.617151in}{1.009176in}}{\pgfqpoint{1.609338in}{1.016989in}}%
\pgfpathcurveto{\pgfqpoint{1.601524in}{1.024803in}}{\pgfqpoint{1.590925in}{1.029193in}}{\pgfqpoint{1.579875in}{1.029193in}}%
\pgfpathcurveto{\pgfqpoint{1.568825in}{1.029193in}}{\pgfqpoint{1.558226in}{1.024803in}}{\pgfqpoint{1.550412in}{1.016989in}}%
\pgfpathcurveto{\pgfqpoint{1.542599in}{1.009176in}}{\pgfqpoint{1.538208in}{0.998577in}}{\pgfqpoint{1.538208in}{0.987527in}}%
\pgfpathcurveto{\pgfqpoint{1.538208in}{0.976476in}}{\pgfqpoint{1.542599in}{0.965877in}}{\pgfqpoint{1.550412in}{0.958064in}}%
\pgfpathcurveto{\pgfqpoint{1.558226in}{0.950250in}}{\pgfqpoint{1.568825in}{0.945860in}}{\pgfqpoint{1.579875in}{0.945860in}}%
\pgfpathclose%
\pgfusepath{stroke,fill}%
\end{pgfscope}%
\begin{pgfscope}%
\pgfpathrectangle{\pgfqpoint{0.375000in}{0.330000in}}{\pgfqpoint{2.325000in}{2.310000in}}%
\pgfusepath{clip}%
\pgfsetbuttcap%
\pgfsetroundjoin%
\definecolor{currentfill}{rgb}{0.000000,0.000000,0.000000}%
\pgfsetfillcolor{currentfill}%
\pgfsetlinewidth{1.003750pt}%
\definecolor{currentstroke}{rgb}{0.000000,0.000000,0.000000}%
\pgfsetstrokecolor{currentstroke}%
\pgfsetdash{}{0pt}%
\pgfpathmoveto{\pgfqpoint{2.139937in}{2.001710in}}%
\pgfpathcurveto{\pgfqpoint{2.150988in}{2.001710in}}{\pgfqpoint{2.161587in}{2.006101in}}{\pgfqpoint{2.169400in}{2.013914in}}%
\pgfpathcurveto{\pgfqpoint{2.177214in}{2.021728in}}{\pgfqpoint{2.181604in}{2.032327in}}{\pgfqpoint{2.181604in}{2.043377in}}%
\pgfpathcurveto{\pgfqpoint{2.181604in}{2.054427in}}{\pgfqpoint{2.177214in}{2.065026in}}{\pgfqpoint{2.169400in}{2.072840in}}%
\pgfpathcurveto{\pgfqpoint{2.161587in}{2.080653in}}{\pgfqpoint{2.150988in}{2.085044in}}{\pgfqpoint{2.139937in}{2.085044in}}%
\pgfpathcurveto{\pgfqpoint{2.128887in}{2.085044in}}{\pgfqpoint{2.118288in}{2.080653in}}{\pgfqpoint{2.110475in}{2.072840in}}%
\pgfpathcurveto{\pgfqpoint{2.102661in}{2.065026in}}{\pgfqpoint{2.098271in}{2.054427in}}{\pgfqpoint{2.098271in}{2.043377in}}%
\pgfpathcurveto{\pgfqpoint{2.098271in}{2.032327in}}{\pgfqpoint{2.102661in}{2.021728in}}{\pgfqpoint{2.110475in}{2.013914in}}%
\pgfpathcurveto{\pgfqpoint{2.118288in}{2.006101in}}{\pgfqpoint{2.128887in}{2.001710in}}{\pgfqpoint{2.139937in}{2.001710in}}%
\pgfpathclose%
\pgfusepath{stroke,fill}%
\end{pgfscope}%
\begin{pgfscope}%
\pgfpathrectangle{\pgfqpoint{0.375000in}{0.330000in}}{\pgfqpoint{2.325000in}{2.310000in}}%
\pgfusepath{clip}%
\pgfsetbuttcap%
\pgfsetroundjoin%
\definecolor{currentfill}{rgb}{0.000000,0.000000,0.000000}%
\pgfsetfillcolor{currentfill}%
\pgfsetlinewidth{1.003750pt}%
\definecolor{currentstroke}{rgb}{0.000000,0.000000,0.000000}%
\pgfsetstrokecolor{currentstroke}%
\pgfsetdash{}{0pt}%
\pgfpathmoveto{\pgfqpoint{2.139937in}{1.947097in}}%
\pgfpathcurveto{\pgfqpoint{2.150988in}{1.947097in}}{\pgfqpoint{2.161587in}{1.951488in}}{\pgfqpoint{2.169400in}{1.959301in}}%
\pgfpathcurveto{\pgfqpoint{2.177214in}{1.967115in}}{\pgfqpoint{2.181604in}{1.977714in}}{\pgfqpoint{2.181604in}{1.988764in}}%
\pgfpathcurveto{\pgfqpoint{2.181604in}{1.999814in}}{\pgfqpoint{2.177214in}{2.010413in}}{\pgfqpoint{2.169400in}{2.018227in}}%
\pgfpathcurveto{\pgfqpoint{2.161587in}{2.026040in}}{\pgfqpoint{2.150988in}{2.030431in}}{\pgfqpoint{2.139937in}{2.030431in}}%
\pgfpathcurveto{\pgfqpoint{2.128887in}{2.030431in}}{\pgfqpoint{2.118288in}{2.026040in}}{\pgfqpoint{2.110475in}{2.018227in}}%
\pgfpathcurveto{\pgfqpoint{2.102661in}{2.010413in}}{\pgfqpoint{2.098271in}{1.999814in}}{\pgfqpoint{2.098271in}{1.988764in}}%
\pgfpathcurveto{\pgfqpoint{2.098271in}{1.977714in}}{\pgfqpoint{2.102661in}{1.967115in}}{\pgfqpoint{2.110475in}{1.959301in}}%
\pgfpathcurveto{\pgfqpoint{2.118288in}{1.951488in}}{\pgfqpoint{2.128887in}{1.947097in}}{\pgfqpoint{2.139937in}{1.947097in}}%
\pgfpathclose%
\pgfusepath{stroke,fill}%
\end{pgfscope}%
\begin{pgfscope}%
\pgfpathrectangle{\pgfqpoint{0.375000in}{0.330000in}}{\pgfqpoint{2.325000in}{2.310000in}}%
\pgfusepath{clip}%
\pgfsetbuttcap%
\pgfsetroundjoin%
\definecolor{currentfill}{rgb}{0.000000,0.000000,0.000000}%
\pgfsetfillcolor{currentfill}%
\pgfsetlinewidth{1.003750pt}%
\definecolor{currentstroke}{rgb}{0.000000,0.000000,0.000000}%
\pgfsetstrokecolor{currentstroke}%
\pgfsetdash{}{0pt}%
\pgfpathmoveto{\pgfqpoint{2.139937in}{1.868212in}}%
\pgfpathcurveto{\pgfqpoint{2.150988in}{1.868212in}}{\pgfqpoint{2.161587in}{1.872602in}}{\pgfqpoint{2.169400in}{1.880416in}}%
\pgfpathcurveto{\pgfqpoint{2.177214in}{1.888230in}}{\pgfqpoint{2.181604in}{1.898829in}}{\pgfqpoint{2.181604in}{1.909879in}}%
\pgfpathcurveto{\pgfqpoint{2.181604in}{1.920929in}}{\pgfqpoint{2.177214in}{1.931528in}}{\pgfqpoint{2.169400in}{1.939341in}}%
\pgfpathcurveto{\pgfqpoint{2.161587in}{1.947155in}}{\pgfqpoint{2.150988in}{1.951545in}}{\pgfqpoint{2.139937in}{1.951545in}}%
\pgfpathcurveto{\pgfqpoint{2.128887in}{1.951545in}}{\pgfqpoint{2.118288in}{1.947155in}}{\pgfqpoint{2.110475in}{1.939341in}}%
\pgfpathcurveto{\pgfqpoint{2.102661in}{1.931528in}}{\pgfqpoint{2.098271in}{1.920929in}}{\pgfqpoint{2.098271in}{1.909879in}}%
\pgfpathcurveto{\pgfqpoint{2.098271in}{1.898829in}}{\pgfqpoint{2.102661in}{1.888230in}}{\pgfqpoint{2.110475in}{1.880416in}}%
\pgfpathcurveto{\pgfqpoint{2.118288in}{1.872602in}}{\pgfqpoint{2.128887in}{1.868212in}}{\pgfqpoint{2.139937in}{1.868212in}}%
\pgfpathclose%
\pgfusepath{stroke,fill}%
\end{pgfscope}%
\begin{pgfscope}%
\pgfpathrectangle{\pgfqpoint{0.375000in}{0.330000in}}{\pgfqpoint{2.325000in}{2.310000in}}%
\pgfusepath{clip}%
\pgfsetbuttcap%
\pgfsetroundjoin%
\definecolor{currentfill}{rgb}{0.000000,0.000000,0.000000}%
\pgfsetfillcolor{currentfill}%
\pgfsetlinewidth{1.003750pt}%
\definecolor{currentstroke}{rgb}{0.000000,0.000000,0.000000}%
\pgfsetstrokecolor{currentstroke}%
\pgfsetdash{}{0pt}%
\pgfpathmoveto{\pgfqpoint{2.139937in}{1.953165in}}%
\pgfpathcurveto{\pgfqpoint{2.150988in}{1.953165in}}{\pgfqpoint{2.161587in}{1.957556in}}{\pgfqpoint{2.169400in}{1.965369in}}%
\pgfpathcurveto{\pgfqpoint{2.177214in}{1.973183in}}{\pgfqpoint{2.181604in}{1.983782in}}{\pgfqpoint{2.181604in}{1.994832in}}%
\pgfpathcurveto{\pgfqpoint{2.181604in}{2.005882in}}{\pgfqpoint{2.177214in}{2.016481in}}{\pgfqpoint{2.169400in}{2.024295in}}%
\pgfpathcurveto{\pgfqpoint{2.161587in}{2.032109in}}{\pgfqpoint{2.150988in}{2.036499in}}{\pgfqpoint{2.139937in}{2.036499in}}%
\pgfpathcurveto{\pgfqpoint{2.128887in}{2.036499in}}{\pgfqpoint{2.118288in}{2.032109in}}{\pgfqpoint{2.110475in}{2.024295in}}%
\pgfpathcurveto{\pgfqpoint{2.102661in}{2.016481in}}{\pgfqpoint{2.098271in}{2.005882in}}{\pgfqpoint{2.098271in}{1.994832in}}%
\pgfpathcurveto{\pgfqpoint{2.098271in}{1.983782in}}{\pgfqpoint{2.102661in}{1.973183in}}{\pgfqpoint{2.110475in}{1.965369in}}%
\pgfpathcurveto{\pgfqpoint{2.118288in}{1.957556in}}{\pgfqpoint{2.128887in}{1.953165in}}{\pgfqpoint{2.139937in}{1.953165in}}%
\pgfpathclose%
\pgfusepath{stroke,fill}%
\end{pgfscope}%
\begin{pgfscope}%
\pgfpathrectangle{\pgfqpoint{0.375000in}{0.330000in}}{\pgfqpoint{2.325000in}{2.310000in}}%
\pgfusepath{clip}%
\pgfsetbuttcap%
\pgfsetroundjoin%
\definecolor{currentfill}{rgb}{0.000000,0.000000,0.000000}%
\pgfsetfillcolor{currentfill}%
\pgfsetlinewidth{1.003750pt}%
\definecolor{currentstroke}{rgb}{0.000000,0.000000,0.000000}%
\pgfsetstrokecolor{currentstroke}%
\pgfsetdash{}{0pt}%
\pgfpathmoveto{\pgfqpoint{2.139937in}{1.953165in}}%
\pgfpathcurveto{\pgfqpoint{2.150988in}{1.953165in}}{\pgfqpoint{2.161587in}{1.957556in}}{\pgfqpoint{2.169400in}{1.965369in}}%
\pgfpathcurveto{\pgfqpoint{2.177214in}{1.973183in}}{\pgfqpoint{2.181604in}{1.983782in}}{\pgfqpoint{2.181604in}{1.994832in}}%
\pgfpathcurveto{\pgfqpoint{2.181604in}{2.005882in}}{\pgfqpoint{2.177214in}{2.016481in}}{\pgfqpoint{2.169400in}{2.024295in}}%
\pgfpathcurveto{\pgfqpoint{2.161587in}{2.032109in}}{\pgfqpoint{2.150988in}{2.036499in}}{\pgfqpoint{2.139937in}{2.036499in}}%
\pgfpathcurveto{\pgfqpoint{2.128887in}{2.036499in}}{\pgfqpoint{2.118288in}{2.032109in}}{\pgfqpoint{2.110475in}{2.024295in}}%
\pgfpathcurveto{\pgfqpoint{2.102661in}{2.016481in}}{\pgfqpoint{2.098271in}{2.005882in}}{\pgfqpoint{2.098271in}{1.994832in}}%
\pgfpathcurveto{\pgfqpoint{2.098271in}{1.983782in}}{\pgfqpoint{2.102661in}{1.973183in}}{\pgfqpoint{2.110475in}{1.965369in}}%
\pgfpathcurveto{\pgfqpoint{2.118288in}{1.957556in}}{\pgfqpoint{2.128887in}{1.953165in}}{\pgfqpoint{2.139937in}{1.953165in}}%
\pgfpathclose%
\pgfusepath{stroke,fill}%
\end{pgfscope}%
\begin{pgfscope}%
\pgfpathrectangle{\pgfqpoint{0.375000in}{0.330000in}}{\pgfqpoint{2.325000in}{2.310000in}}%
\pgfusepath{clip}%
\pgfsetbuttcap%
\pgfsetroundjoin%
\definecolor{currentfill}{rgb}{0.000000,0.000000,0.000000}%
\pgfsetfillcolor{currentfill}%
\pgfsetlinewidth{1.003750pt}%
\definecolor{currentstroke}{rgb}{0.000000,0.000000,0.000000}%
\pgfsetstrokecolor{currentstroke}%
\pgfsetdash{}{0pt}%
\pgfpathmoveto{\pgfqpoint{2.139937in}{1.977438in}}%
\pgfpathcurveto{\pgfqpoint{2.150988in}{1.977438in}}{\pgfqpoint{2.161587in}{1.981828in}}{\pgfqpoint{2.169400in}{1.989642in}}%
\pgfpathcurveto{\pgfqpoint{2.177214in}{1.997455in}}{\pgfqpoint{2.181604in}{2.008054in}}{\pgfqpoint{2.181604in}{2.019105in}}%
\pgfpathcurveto{\pgfqpoint{2.181604in}{2.030155in}}{\pgfqpoint{2.177214in}{2.040754in}}{\pgfqpoint{2.169400in}{2.048567in}}%
\pgfpathcurveto{\pgfqpoint{2.161587in}{2.056381in}}{\pgfqpoint{2.150988in}{2.060771in}}{\pgfqpoint{2.139937in}{2.060771in}}%
\pgfpathcurveto{\pgfqpoint{2.128887in}{2.060771in}}{\pgfqpoint{2.118288in}{2.056381in}}{\pgfqpoint{2.110475in}{2.048567in}}%
\pgfpathcurveto{\pgfqpoint{2.102661in}{2.040754in}}{\pgfqpoint{2.098271in}{2.030155in}}{\pgfqpoint{2.098271in}{2.019105in}}%
\pgfpathcurveto{\pgfqpoint{2.098271in}{2.008054in}}{\pgfqpoint{2.102661in}{1.997455in}}{\pgfqpoint{2.110475in}{1.989642in}}%
\pgfpathcurveto{\pgfqpoint{2.118288in}{1.981828in}}{\pgfqpoint{2.128887in}{1.977438in}}{\pgfqpoint{2.139937in}{1.977438in}}%
\pgfpathclose%
\pgfusepath{stroke,fill}%
\end{pgfscope}%
\begin{pgfscope}%
\pgfpathrectangle{\pgfqpoint{0.375000in}{0.330000in}}{\pgfqpoint{2.325000in}{2.310000in}}%
\pgfusepath{clip}%
\pgfsetbuttcap%
\pgfsetroundjoin%
\definecolor{currentfill}{rgb}{0.000000,0.000000,0.000000}%
\pgfsetfillcolor{currentfill}%
\pgfsetlinewidth{1.003750pt}%
\definecolor{currentstroke}{rgb}{0.000000,0.000000,0.000000}%
\pgfsetstrokecolor{currentstroke}%
\pgfsetdash{}{0pt}%
\pgfpathmoveto{\pgfqpoint{2.139937in}{2.183754in}}%
\pgfpathcurveto{\pgfqpoint{2.150988in}{2.183754in}}{\pgfqpoint{2.161587in}{2.188144in}}{\pgfqpoint{2.169400in}{2.195957in}}%
\pgfpathcurveto{\pgfqpoint{2.177214in}{2.203771in}}{\pgfqpoint{2.181604in}{2.214370in}}{\pgfqpoint{2.181604in}{2.225420in}}%
\pgfpathcurveto{\pgfqpoint{2.181604in}{2.236470in}}{\pgfqpoint{2.177214in}{2.247069in}}{\pgfqpoint{2.169400in}{2.254883in}}%
\pgfpathcurveto{\pgfqpoint{2.161587in}{2.262697in}}{\pgfqpoint{2.150988in}{2.267087in}}{\pgfqpoint{2.139937in}{2.267087in}}%
\pgfpathcurveto{\pgfqpoint{2.128887in}{2.267087in}}{\pgfqpoint{2.118288in}{2.262697in}}{\pgfqpoint{2.110475in}{2.254883in}}%
\pgfpathcurveto{\pgfqpoint{2.102661in}{2.247069in}}{\pgfqpoint{2.098271in}{2.236470in}}{\pgfqpoint{2.098271in}{2.225420in}}%
\pgfpathcurveto{\pgfqpoint{2.098271in}{2.214370in}}{\pgfqpoint{2.102661in}{2.203771in}}{\pgfqpoint{2.110475in}{2.195957in}}%
\pgfpathcurveto{\pgfqpoint{2.118288in}{2.188144in}}{\pgfqpoint{2.128887in}{2.183754in}}{\pgfqpoint{2.139937in}{2.183754in}}%
\pgfpathclose%
\pgfusepath{stroke,fill}%
\end{pgfscope}%
\begin{pgfscope}%
\pgfpathrectangle{\pgfqpoint{0.375000in}{0.330000in}}{\pgfqpoint{2.325000in}{2.310000in}}%
\pgfusepath{clip}%
\pgfsetbuttcap%
\pgfsetroundjoin%
\definecolor{currentfill}{rgb}{0.000000,0.000000,0.000000}%
\pgfsetfillcolor{currentfill}%
\pgfsetlinewidth{1.003750pt}%
\definecolor{currentstroke}{rgb}{0.000000,0.000000,0.000000}%
\pgfsetstrokecolor{currentstroke}%
\pgfsetdash{}{0pt}%
\pgfpathmoveto{\pgfqpoint{2.139937in}{1.965302in}}%
\pgfpathcurveto{\pgfqpoint{2.150988in}{1.965302in}}{\pgfqpoint{2.161587in}{1.969692in}}{\pgfqpoint{2.169400in}{1.977506in}}%
\pgfpathcurveto{\pgfqpoint{2.177214in}{1.985319in}}{\pgfqpoint{2.181604in}{1.995918in}}{\pgfqpoint{2.181604in}{2.006968in}}%
\pgfpathcurveto{\pgfqpoint{2.181604in}{2.018019in}}{\pgfqpoint{2.177214in}{2.028618in}}{\pgfqpoint{2.169400in}{2.036431in}}%
\pgfpathcurveto{\pgfqpoint{2.161587in}{2.044245in}}{\pgfqpoint{2.150988in}{2.048635in}}{\pgfqpoint{2.139937in}{2.048635in}}%
\pgfpathcurveto{\pgfqpoint{2.128887in}{2.048635in}}{\pgfqpoint{2.118288in}{2.044245in}}{\pgfqpoint{2.110475in}{2.036431in}}%
\pgfpathcurveto{\pgfqpoint{2.102661in}{2.028618in}}{\pgfqpoint{2.098271in}{2.018019in}}{\pgfqpoint{2.098271in}{2.006968in}}%
\pgfpathcurveto{\pgfqpoint{2.098271in}{1.995918in}}{\pgfqpoint{2.102661in}{1.985319in}}{\pgfqpoint{2.110475in}{1.977506in}}%
\pgfpathcurveto{\pgfqpoint{2.118288in}{1.969692in}}{\pgfqpoint{2.128887in}{1.965302in}}{\pgfqpoint{2.139937in}{1.965302in}}%
\pgfpathclose%
\pgfusepath{stroke,fill}%
\end{pgfscope}%
\begin{pgfscope}%
\pgfpathrectangle{\pgfqpoint{0.375000in}{0.330000in}}{\pgfqpoint{2.325000in}{2.310000in}}%
\pgfusepath{clip}%
\pgfsetbuttcap%
\pgfsetroundjoin%
\definecolor{currentfill}{rgb}{0.000000,0.000000,0.000000}%
\pgfsetfillcolor{currentfill}%
\pgfsetlinewidth{1.003750pt}%
\definecolor{currentstroke}{rgb}{0.000000,0.000000,0.000000}%
\pgfsetstrokecolor{currentstroke}%
\pgfsetdash{}{0pt}%
\pgfpathmoveto{\pgfqpoint{2.139937in}{1.886416in}}%
\pgfpathcurveto{\pgfqpoint{2.150988in}{1.886416in}}{\pgfqpoint{2.161587in}{1.890807in}}{\pgfqpoint{2.169400in}{1.898620in}}%
\pgfpathcurveto{\pgfqpoint{2.177214in}{1.906434in}}{\pgfqpoint{2.181604in}{1.917033in}}{\pgfqpoint{2.181604in}{1.928083in}}%
\pgfpathcurveto{\pgfqpoint{2.181604in}{1.939133in}}{\pgfqpoint{2.177214in}{1.949732in}}{\pgfqpoint{2.169400in}{1.957546in}}%
\pgfpathcurveto{\pgfqpoint{2.161587in}{1.965359in}}{\pgfqpoint{2.150988in}{1.969750in}}{\pgfqpoint{2.139937in}{1.969750in}}%
\pgfpathcurveto{\pgfqpoint{2.128887in}{1.969750in}}{\pgfqpoint{2.118288in}{1.965359in}}{\pgfqpoint{2.110475in}{1.957546in}}%
\pgfpathcurveto{\pgfqpoint{2.102661in}{1.949732in}}{\pgfqpoint{2.098271in}{1.939133in}}{\pgfqpoint{2.098271in}{1.928083in}}%
\pgfpathcurveto{\pgfqpoint{2.098271in}{1.917033in}}{\pgfqpoint{2.102661in}{1.906434in}}{\pgfqpoint{2.110475in}{1.898620in}}%
\pgfpathcurveto{\pgfqpoint{2.118288in}{1.890807in}}{\pgfqpoint{2.128887in}{1.886416in}}{\pgfqpoint{2.139937in}{1.886416in}}%
\pgfpathclose%
\pgfusepath{stroke,fill}%
\end{pgfscope}%
\begin{pgfscope}%
\pgfpathrectangle{\pgfqpoint{0.375000in}{0.330000in}}{\pgfqpoint{2.325000in}{2.310000in}}%
\pgfusepath{clip}%
\pgfsetbuttcap%
\pgfsetroundjoin%
\definecolor{currentfill}{rgb}{0.000000,0.000000,0.000000}%
\pgfsetfillcolor{currentfill}%
\pgfsetlinewidth{1.003750pt}%
\definecolor{currentstroke}{rgb}{0.000000,0.000000,0.000000}%
\pgfsetstrokecolor{currentstroke}%
\pgfsetdash{}{0pt}%
\pgfpathmoveto{\pgfqpoint{2.139937in}{2.001710in}}%
\pgfpathcurveto{\pgfqpoint{2.150988in}{2.001710in}}{\pgfqpoint{2.161587in}{2.006101in}}{\pgfqpoint{2.169400in}{2.013914in}}%
\pgfpathcurveto{\pgfqpoint{2.177214in}{2.021728in}}{\pgfqpoint{2.181604in}{2.032327in}}{\pgfqpoint{2.181604in}{2.043377in}}%
\pgfpathcurveto{\pgfqpoint{2.181604in}{2.054427in}}{\pgfqpoint{2.177214in}{2.065026in}}{\pgfqpoint{2.169400in}{2.072840in}}%
\pgfpathcurveto{\pgfqpoint{2.161587in}{2.080653in}}{\pgfqpoint{2.150988in}{2.085044in}}{\pgfqpoint{2.139937in}{2.085044in}}%
\pgfpathcurveto{\pgfqpoint{2.128887in}{2.085044in}}{\pgfqpoint{2.118288in}{2.080653in}}{\pgfqpoint{2.110475in}{2.072840in}}%
\pgfpathcurveto{\pgfqpoint{2.102661in}{2.065026in}}{\pgfqpoint{2.098271in}{2.054427in}}{\pgfqpoint{2.098271in}{2.043377in}}%
\pgfpathcurveto{\pgfqpoint{2.098271in}{2.032327in}}{\pgfqpoint{2.102661in}{2.021728in}}{\pgfqpoint{2.110475in}{2.013914in}}%
\pgfpathcurveto{\pgfqpoint{2.118288in}{2.006101in}}{\pgfqpoint{2.128887in}{2.001710in}}{\pgfqpoint{2.139937in}{2.001710in}}%
\pgfpathclose%
\pgfusepath{stroke,fill}%
\end{pgfscope}%
\begin{pgfscope}%
\pgfpathrectangle{\pgfqpoint{0.375000in}{0.330000in}}{\pgfqpoint{2.325000in}{2.310000in}}%
\pgfusepath{clip}%
\pgfsetbuttcap%
\pgfsetroundjoin%
\definecolor{currentfill}{rgb}{0.000000,0.000000,0.000000}%
\pgfsetfillcolor{currentfill}%
\pgfsetlinewidth{1.003750pt}%
\definecolor{currentstroke}{rgb}{0.000000,0.000000,0.000000}%
\pgfsetstrokecolor{currentstroke}%
\pgfsetdash{}{0pt}%
\pgfpathmoveto{\pgfqpoint{2.139937in}{2.068460in}}%
\pgfpathcurveto{\pgfqpoint{2.150988in}{2.068460in}}{\pgfqpoint{2.161587in}{2.072850in}}{\pgfqpoint{2.169400in}{2.080663in}}%
\pgfpathcurveto{\pgfqpoint{2.177214in}{2.088477in}}{\pgfqpoint{2.181604in}{2.099076in}}{\pgfqpoint{2.181604in}{2.110126in}}%
\pgfpathcurveto{\pgfqpoint{2.181604in}{2.121176in}}{\pgfqpoint{2.177214in}{2.131775in}}{\pgfqpoint{2.169400in}{2.139589in}}%
\pgfpathcurveto{\pgfqpoint{2.161587in}{2.147403in}}{\pgfqpoint{2.150988in}{2.151793in}}{\pgfqpoint{2.139937in}{2.151793in}}%
\pgfpathcurveto{\pgfqpoint{2.128887in}{2.151793in}}{\pgfqpoint{2.118288in}{2.147403in}}{\pgfqpoint{2.110475in}{2.139589in}}%
\pgfpathcurveto{\pgfqpoint{2.102661in}{2.131775in}}{\pgfqpoint{2.098271in}{2.121176in}}{\pgfqpoint{2.098271in}{2.110126in}}%
\pgfpathcurveto{\pgfqpoint{2.098271in}{2.099076in}}{\pgfqpoint{2.102661in}{2.088477in}}{\pgfqpoint{2.110475in}{2.080663in}}%
\pgfpathcurveto{\pgfqpoint{2.118288in}{2.072850in}}{\pgfqpoint{2.128887in}{2.068460in}}{\pgfqpoint{2.139937in}{2.068460in}}%
\pgfpathclose%
\pgfusepath{stroke,fill}%
\end{pgfscope}%
\begin{pgfscope}%
\pgfpathrectangle{\pgfqpoint{0.375000in}{0.330000in}}{\pgfqpoint{2.325000in}{2.310000in}}%
\pgfusepath{clip}%
\pgfsetbuttcap%
\pgfsetroundjoin%
\definecolor{currentfill}{rgb}{0.000000,0.000000,0.000000}%
\pgfsetfillcolor{currentfill}%
\pgfsetlinewidth{1.003750pt}%
\definecolor{currentstroke}{rgb}{0.000000,0.000000,0.000000}%
\pgfsetstrokecolor{currentstroke}%
\pgfsetdash{}{0pt}%
\pgfpathmoveto{\pgfqpoint{2.139937in}{1.843940in}}%
\pgfpathcurveto{\pgfqpoint{2.150988in}{1.843940in}}{\pgfqpoint{2.161587in}{1.848330in}}{\pgfqpoint{2.169400in}{1.856143in}}%
\pgfpathcurveto{\pgfqpoint{2.177214in}{1.863957in}}{\pgfqpoint{2.181604in}{1.874556in}}{\pgfqpoint{2.181604in}{1.885606in}}%
\pgfpathcurveto{\pgfqpoint{2.181604in}{1.896656in}}{\pgfqpoint{2.177214in}{1.907255in}}{\pgfqpoint{2.169400in}{1.915069in}}%
\pgfpathcurveto{\pgfqpoint{2.161587in}{1.922883in}}{\pgfqpoint{2.150988in}{1.927273in}}{\pgfqpoint{2.139937in}{1.927273in}}%
\pgfpathcurveto{\pgfqpoint{2.128887in}{1.927273in}}{\pgfqpoint{2.118288in}{1.922883in}}{\pgfqpoint{2.110475in}{1.915069in}}%
\pgfpathcurveto{\pgfqpoint{2.102661in}{1.907255in}}{\pgfqpoint{2.098271in}{1.896656in}}{\pgfqpoint{2.098271in}{1.885606in}}%
\pgfpathcurveto{\pgfqpoint{2.098271in}{1.874556in}}{\pgfqpoint{2.102661in}{1.863957in}}{\pgfqpoint{2.110475in}{1.856143in}}%
\pgfpathcurveto{\pgfqpoint{2.118288in}{1.848330in}}{\pgfqpoint{2.128887in}{1.843940in}}{\pgfqpoint{2.139937in}{1.843940in}}%
\pgfpathclose%
\pgfusepath{stroke,fill}%
\end{pgfscope}%
\begin{pgfscope}%
\pgfpathrectangle{\pgfqpoint{0.375000in}{0.330000in}}{\pgfqpoint{2.325000in}{2.310000in}}%
\pgfusepath{clip}%
\pgfsetbuttcap%
\pgfsetroundjoin%
\definecolor{currentfill}{rgb}{0.000000,0.000000,0.000000}%
\pgfsetfillcolor{currentfill}%
\pgfsetlinewidth{1.003750pt}%
\definecolor{currentstroke}{rgb}{0.000000,0.000000,0.000000}%
\pgfsetstrokecolor{currentstroke}%
\pgfsetdash{}{0pt}%
\pgfpathmoveto{\pgfqpoint{2.139937in}{2.110936in}}%
\pgfpathcurveto{\pgfqpoint{2.150988in}{2.110936in}}{\pgfqpoint{2.161587in}{2.115327in}}{\pgfqpoint{2.169400in}{2.123140in}}%
\pgfpathcurveto{\pgfqpoint{2.177214in}{2.130954in}}{\pgfqpoint{2.181604in}{2.141553in}}{\pgfqpoint{2.181604in}{2.152603in}}%
\pgfpathcurveto{\pgfqpoint{2.181604in}{2.163653in}}{\pgfqpoint{2.177214in}{2.174252in}}{\pgfqpoint{2.169400in}{2.182066in}}%
\pgfpathcurveto{\pgfqpoint{2.161587in}{2.189879in}}{\pgfqpoint{2.150988in}{2.194270in}}{\pgfqpoint{2.139937in}{2.194270in}}%
\pgfpathcurveto{\pgfqpoint{2.128887in}{2.194270in}}{\pgfqpoint{2.118288in}{2.189879in}}{\pgfqpoint{2.110475in}{2.182066in}}%
\pgfpathcurveto{\pgfqpoint{2.102661in}{2.174252in}}{\pgfqpoint{2.098271in}{2.163653in}}{\pgfqpoint{2.098271in}{2.152603in}}%
\pgfpathcurveto{\pgfqpoint{2.098271in}{2.141553in}}{\pgfqpoint{2.102661in}{2.130954in}}{\pgfqpoint{2.110475in}{2.123140in}}%
\pgfpathcurveto{\pgfqpoint{2.118288in}{2.115327in}}{\pgfqpoint{2.128887in}{2.110936in}}{\pgfqpoint{2.139937in}{2.110936in}}%
\pgfpathclose%
\pgfusepath{stroke,fill}%
\end{pgfscope}%
\begin{pgfscope}%
\pgfpathrectangle{\pgfqpoint{0.375000in}{0.330000in}}{\pgfqpoint{2.325000in}{2.310000in}}%
\pgfusepath{clip}%
\pgfsetbuttcap%
\pgfsetroundjoin%
\definecolor{currentfill}{rgb}{0.000000,0.000000,0.000000}%
\pgfsetfillcolor{currentfill}%
\pgfsetlinewidth{1.003750pt}%
\definecolor{currentstroke}{rgb}{0.000000,0.000000,0.000000}%
\pgfsetstrokecolor{currentstroke}%
\pgfsetdash{}{0pt}%
\pgfpathmoveto{\pgfqpoint{2.139937in}{2.153413in}}%
\pgfpathcurveto{\pgfqpoint{2.150988in}{2.153413in}}{\pgfqpoint{2.161587in}{2.157803in}}{\pgfqpoint{2.169400in}{2.165617in}}%
\pgfpathcurveto{\pgfqpoint{2.177214in}{2.173430in}}{\pgfqpoint{2.181604in}{2.184030in}}{\pgfqpoint{2.181604in}{2.195080in}}%
\pgfpathcurveto{\pgfqpoint{2.181604in}{2.206130in}}{\pgfqpoint{2.177214in}{2.216729in}}{\pgfqpoint{2.169400in}{2.224542in}}%
\pgfpathcurveto{\pgfqpoint{2.161587in}{2.232356in}}{\pgfqpoint{2.150988in}{2.236746in}}{\pgfqpoint{2.139937in}{2.236746in}}%
\pgfpathcurveto{\pgfqpoint{2.128887in}{2.236746in}}{\pgfqpoint{2.118288in}{2.232356in}}{\pgfqpoint{2.110475in}{2.224542in}}%
\pgfpathcurveto{\pgfqpoint{2.102661in}{2.216729in}}{\pgfqpoint{2.098271in}{2.206130in}}{\pgfqpoint{2.098271in}{2.195080in}}%
\pgfpathcurveto{\pgfqpoint{2.098271in}{2.184030in}}{\pgfqpoint{2.102661in}{2.173430in}}{\pgfqpoint{2.110475in}{2.165617in}}%
\pgfpathcurveto{\pgfqpoint{2.118288in}{2.157803in}}{\pgfqpoint{2.128887in}{2.153413in}}{\pgfqpoint{2.139937in}{2.153413in}}%
\pgfpathclose%
\pgfusepath{stroke,fill}%
\end{pgfscope}%
\begin{pgfscope}%
\pgfpathrectangle{\pgfqpoint{0.375000in}{0.330000in}}{\pgfqpoint{2.325000in}{2.310000in}}%
\pgfusepath{clip}%
\pgfsetbuttcap%
\pgfsetroundjoin%
\definecolor{currentfill}{rgb}{0.000000,0.000000,0.000000}%
\pgfsetfillcolor{currentfill}%
\pgfsetlinewidth{1.003750pt}%
\definecolor{currentstroke}{rgb}{0.000000,0.000000,0.000000}%
\pgfsetstrokecolor{currentstroke}%
\pgfsetdash{}{0pt}%
\pgfpathmoveto{\pgfqpoint{2.139937in}{2.062391in}}%
\pgfpathcurveto{\pgfqpoint{2.150988in}{2.062391in}}{\pgfqpoint{2.161587in}{2.066782in}}{\pgfqpoint{2.169400in}{2.074595in}}%
\pgfpathcurveto{\pgfqpoint{2.177214in}{2.082409in}}{\pgfqpoint{2.181604in}{2.093008in}}{\pgfqpoint{2.181604in}{2.104058in}}%
\pgfpathcurveto{\pgfqpoint{2.181604in}{2.115108in}}{\pgfqpoint{2.177214in}{2.125707in}}{\pgfqpoint{2.169400in}{2.133521in}}%
\pgfpathcurveto{\pgfqpoint{2.161587in}{2.141334in}}{\pgfqpoint{2.150988in}{2.145725in}}{\pgfqpoint{2.139937in}{2.145725in}}%
\pgfpathcurveto{\pgfqpoint{2.128887in}{2.145725in}}{\pgfqpoint{2.118288in}{2.141334in}}{\pgfqpoint{2.110475in}{2.133521in}}%
\pgfpathcurveto{\pgfqpoint{2.102661in}{2.125707in}}{\pgfqpoint{2.098271in}{2.115108in}}{\pgfqpoint{2.098271in}{2.104058in}}%
\pgfpathcurveto{\pgfqpoint{2.098271in}{2.093008in}}{\pgfqpoint{2.102661in}{2.082409in}}{\pgfqpoint{2.110475in}{2.074595in}}%
\pgfpathcurveto{\pgfqpoint{2.118288in}{2.066782in}}{\pgfqpoint{2.128887in}{2.062391in}}{\pgfqpoint{2.139937in}{2.062391in}}%
\pgfpathclose%
\pgfusepath{stroke,fill}%
\end{pgfscope}%
\begin{pgfscope}%
\pgfpathrectangle{\pgfqpoint{0.375000in}{0.330000in}}{\pgfqpoint{2.325000in}{2.310000in}}%
\pgfusepath{clip}%
\pgfsetbuttcap%
\pgfsetroundjoin%
\definecolor{currentfill}{rgb}{0.000000,0.000000,0.000000}%
\pgfsetfillcolor{currentfill}%
\pgfsetlinewidth{1.003750pt}%
\definecolor{currentstroke}{rgb}{0.000000,0.000000,0.000000}%
\pgfsetstrokecolor{currentstroke}%
\pgfsetdash{}{0pt}%
\pgfpathmoveto{\pgfqpoint{2.139937in}{2.335456in}}%
\pgfpathcurveto{\pgfqpoint{2.150988in}{2.335456in}}{\pgfqpoint{2.161587in}{2.339846in}}{\pgfqpoint{2.169400in}{2.347660in}}%
\pgfpathcurveto{\pgfqpoint{2.177214in}{2.355474in}}{\pgfqpoint{2.181604in}{2.366073in}}{\pgfqpoint{2.181604in}{2.377123in}}%
\pgfpathcurveto{\pgfqpoint{2.181604in}{2.388173in}}{\pgfqpoint{2.177214in}{2.398772in}}{\pgfqpoint{2.169400in}{2.406586in}}%
\pgfpathcurveto{\pgfqpoint{2.161587in}{2.414399in}}{\pgfqpoint{2.150988in}{2.418789in}}{\pgfqpoint{2.139937in}{2.418789in}}%
\pgfpathcurveto{\pgfqpoint{2.128887in}{2.418789in}}{\pgfqpoint{2.118288in}{2.414399in}}{\pgfqpoint{2.110475in}{2.406586in}}%
\pgfpathcurveto{\pgfqpoint{2.102661in}{2.398772in}}{\pgfqpoint{2.098271in}{2.388173in}}{\pgfqpoint{2.098271in}{2.377123in}}%
\pgfpathcurveto{\pgfqpoint{2.098271in}{2.366073in}}{\pgfqpoint{2.102661in}{2.355474in}}{\pgfqpoint{2.110475in}{2.347660in}}%
\pgfpathcurveto{\pgfqpoint{2.118288in}{2.339846in}}{\pgfqpoint{2.128887in}{2.335456in}}{\pgfqpoint{2.139937in}{2.335456in}}%
\pgfpathclose%
\pgfusepath{stroke,fill}%
\end{pgfscope}%
\begin{pgfscope}%
\pgfpathrectangle{\pgfqpoint{0.375000in}{0.330000in}}{\pgfqpoint{2.325000in}{2.310000in}}%
\pgfusepath{clip}%
\pgfsetbuttcap%
\pgfsetroundjoin%
\definecolor{currentfill}{rgb}{0.000000,0.000000,0.000000}%
\pgfsetfillcolor{currentfill}%
\pgfsetlinewidth{1.003750pt}%
\definecolor{currentstroke}{rgb}{0.000000,0.000000,0.000000}%
\pgfsetstrokecolor{currentstroke}%
\pgfsetdash{}{0pt}%
\pgfpathmoveto{\pgfqpoint{2.139937in}{1.904621in}}%
\pgfpathcurveto{\pgfqpoint{2.150988in}{1.904621in}}{\pgfqpoint{2.161587in}{1.909011in}}{\pgfqpoint{2.169400in}{1.916825in}}%
\pgfpathcurveto{\pgfqpoint{2.177214in}{1.924638in}}{\pgfqpoint{2.181604in}{1.935237in}}{\pgfqpoint{2.181604in}{1.946287in}}%
\pgfpathcurveto{\pgfqpoint{2.181604in}{1.957337in}}{\pgfqpoint{2.177214in}{1.967936in}}{\pgfqpoint{2.169400in}{1.975750in}}%
\pgfpathcurveto{\pgfqpoint{2.161587in}{1.983564in}}{\pgfqpoint{2.150988in}{1.987954in}}{\pgfqpoint{2.139937in}{1.987954in}}%
\pgfpathcurveto{\pgfqpoint{2.128887in}{1.987954in}}{\pgfqpoint{2.118288in}{1.983564in}}{\pgfqpoint{2.110475in}{1.975750in}}%
\pgfpathcurveto{\pgfqpoint{2.102661in}{1.967936in}}{\pgfqpoint{2.098271in}{1.957337in}}{\pgfqpoint{2.098271in}{1.946287in}}%
\pgfpathcurveto{\pgfqpoint{2.098271in}{1.935237in}}{\pgfqpoint{2.102661in}{1.924638in}}{\pgfqpoint{2.110475in}{1.916825in}}%
\pgfpathcurveto{\pgfqpoint{2.118288in}{1.909011in}}{\pgfqpoint{2.128887in}{1.904621in}}{\pgfqpoint{2.139937in}{1.904621in}}%
\pgfpathclose%
\pgfusepath{stroke,fill}%
\end{pgfscope}%
\begin{pgfscope}%
\pgfpathrectangle{\pgfqpoint{0.375000in}{0.330000in}}{\pgfqpoint{2.325000in}{2.310000in}}%
\pgfusepath{clip}%
\pgfsetbuttcap%
\pgfsetroundjoin%
\definecolor{currentfill}{rgb}{0.000000,0.000000,0.000000}%
\pgfsetfillcolor{currentfill}%
\pgfsetlinewidth{1.003750pt}%
\definecolor{currentstroke}{rgb}{0.000000,0.000000,0.000000}%
\pgfsetstrokecolor{currentstroke}%
\pgfsetdash{}{0pt}%
\pgfpathmoveto{\pgfqpoint{2.139937in}{2.353660in}}%
\pgfpathcurveto{\pgfqpoint{2.150988in}{2.353660in}}{\pgfqpoint{2.161587in}{2.358051in}}{\pgfqpoint{2.169400in}{2.365864in}}%
\pgfpathcurveto{\pgfqpoint{2.177214in}{2.373678in}}{\pgfqpoint{2.181604in}{2.384277in}}{\pgfqpoint{2.181604in}{2.395327in}}%
\pgfpathcurveto{\pgfqpoint{2.181604in}{2.406377in}}{\pgfqpoint{2.177214in}{2.416976in}}{\pgfqpoint{2.169400in}{2.424790in}}%
\pgfpathcurveto{\pgfqpoint{2.161587in}{2.432604in}}{\pgfqpoint{2.150988in}{2.436994in}}{\pgfqpoint{2.139937in}{2.436994in}}%
\pgfpathcurveto{\pgfqpoint{2.128887in}{2.436994in}}{\pgfqpoint{2.118288in}{2.432604in}}{\pgfqpoint{2.110475in}{2.424790in}}%
\pgfpathcurveto{\pgfqpoint{2.102661in}{2.416976in}}{\pgfqpoint{2.098271in}{2.406377in}}{\pgfqpoint{2.098271in}{2.395327in}}%
\pgfpathcurveto{\pgfqpoint{2.098271in}{2.384277in}}{\pgfqpoint{2.102661in}{2.373678in}}{\pgfqpoint{2.110475in}{2.365864in}}%
\pgfpathcurveto{\pgfqpoint{2.118288in}{2.358051in}}{\pgfqpoint{2.128887in}{2.353660in}}{\pgfqpoint{2.139937in}{2.353660in}}%
\pgfpathclose%
\pgfusepath{stroke,fill}%
\end{pgfscope}%
\begin{pgfscope}%
\pgfpathrectangle{\pgfqpoint{0.375000in}{0.330000in}}{\pgfqpoint{2.325000in}{2.310000in}}%
\pgfusepath{clip}%
\pgfsetbuttcap%
\pgfsetroundjoin%
\definecolor{currentfill}{rgb}{0.000000,0.000000,0.000000}%
\pgfsetfillcolor{currentfill}%
\pgfsetlinewidth{1.003750pt}%
\definecolor{currentstroke}{rgb}{0.000000,0.000000,0.000000}%
\pgfsetstrokecolor{currentstroke}%
\pgfsetdash{}{0pt}%
\pgfpathmoveto{\pgfqpoint{2.139937in}{2.032051in}}%
\pgfpathcurveto{\pgfqpoint{2.150988in}{2.032051in}}{\pgfqpoint{2.161587in}{2.036441in}}{\pgfqpoint{2.169400in}{2.044255in}}%
\pgfpathcurveto{\pgfqpoint{2.177214in}{2.052068in}}{\pgfqpoint{2.181604in}{2.062667in}}{\pgfqpoint{2.181604in}{2.073718in}}%
\pgfpathcurveto{\pgfqpoint{2.181604in}{2.084768in}}{\pgfqpoint{2.177214in}{2.095367in}}{\pgfqpoint{2.169400in}{2.103180in}}%
\pgfpathcurveto{\pgfqpoint{2.161587in}{2.110994in}}{\pgfqpoint{2.150988in}{2.115384in}}{\pgfqpoint{2.139937in}{2.115384in}}%
\pgfpathcurveto{\pgfqpoint{2.128887in}{2.115384in}}{\pgfqpoint{2.118288in}{2.110994in}}{\pgfqpoint{2.110475in}{2.103180in}}%
\pgfpathcurveto{\pgfqpoint{2.102661in}{2.095367in}}{\pgfqpoint{2.098271in}{2.084768in}}{\pgfqpoint{2.098271in}{2.073718in}}%
\pgfpathcurveto{\pgfqpoint{2.098271in}{2.062667in}}{\pgfqpoint{2.102661in}{2.052068in}}{\pgfqpoint{2.110475in}{2.044255in}}%
\pgfpathcurveto{\pgfqpoint{2.118288in}{2.036441in}}{\pgfqpoint{2.128887in}{2.032051in}}{\pgfqpoint{2.139937in}{2.032051in}}%
\pgfpathclose%
\pgfusepath{stroke,fill}%
\end{pgfscope}%
\begin{pgfscope}%
\pgfpathrectangle{\pgfqpoint{0.375000in}{0.330000in}}{\pgfqpoint{2.325000in}{2.310000in}}%
\pgfusepath{clip}%
\pgfsetbuttcap%
\pgfsetroundjoin%
\definecolor{currentfill}{rgb}{0.000000,0.000000,0.000000}%
\pgfsetfillcolor{currentfill}%
\pgfsetlinewidth{1.003750pt}%
\definecolor{currentstroke}{rgb}{0.000000,0.000000,0.000000}%
\pgfsetstrokecolor{currentstroke}%
\pgfsetdash{}{0pt}%
\pgfpathmoveto{\pgfqpoint{2.139937in}{2.074528in}}%
\pgfpathcurveto{\pgfqpoint{2.150988in}{2.074528in}}{\pgfqpoint{2.161587in}{2.078918in}}{\pgfqpoint{2.169400in}{2.086731in}}%
\pgfpathcurveto{\pgfqpoint{2.177214in}{2.094545in}}{\pgfqpoint{2.181604in}{2.105144in}}{\pgfqpoint{2.181604in}{2.116194in}}%
\pgfpathcurveto{\pgfqpoint{2.181604in}{2.127244in}}{\pgfqpoint{2.177214in}{2.137843in}}{\pgfqpoint{2.169400in}{2.145657in}}%
\pgfpathcurveto{\pgfqpoint{2.161587in}{2.153471in}}{\pgfqpoint{2.150988in}{2.157861in}}{\pgfqpoint{2.139937in}{2.157861in}}%
\pgfpathcurveto{\pgfqpoint{2.128887in}{2.157861in}}{\pgfqpoint{2.118288in}{2.153471in}}{\pgfqpoint{2.110475in}{2.145657in}}%
\pgfpathcurveto{\pgfqpoint{2.102661in}{2.137843in}}{\pgfqpoint{2.098271in}{2.127244in}}{\pgfqpoint{2.098271in}{2.116194in}}%
\pgfpathcurveto{\pgfqpoint{2.098271in}{2.105144in}}{\pgfqpoint{2.102661in}{2.094545in}}{\pgfqpoint{2.110475in}{2.086731in}}%
\pgfpathcurveto{\pgfqpoint{2.118288in}{2.078918in}}{\pgfqpoint{2.128887in}{2.074528in}}{\pgfqpoint{2.139937in}{2.074528in}}%
\pgfpathclose%
\pgfusepath{stroke,fill}%
\end{pgfscope}%
\begin{pgfscope}%
\pgfpathrectangle{\pgfqpoint{0.375000in}{0.330000in}}{\pgfqpoint{2.325000in}{2.310000in}}%
\pgfusepath{clip}%
\pgfsetbuttcap%
\pgfsetroundjoin%
\definecolor{currentfill}{rgb}{0.000000,0.000000,0.000000}%
\pgfsetfillcolor{currentfill}%
\pgfsetlinewidth{1.003750pt}%
\definecolor{currentstroke}{rgb}{0.000000,0.000000,0.000000}%
\pgfsetstrokecolor{currentstroke}%
\pgfsetdash{}{0pt}%
\pgfpathmoveto{\pgfqpoint{2.139937in}{1.928893in}}%
\pgfpathcurveto{\pgfqpoint{2.150988in}{1.928893in}}{\pgfqpoint{2.161587in}{1.933283in}}{\pgfqpoint{2.169400in}{1.941097in}}%
\pgfpathcurveto{\pgfqpoint{2.177214in}{1.948911in}}{\pgfqpoint{2.181604in}{1.959510in}}{\pgfqpoint{2.181604in}{1.970560in}}%
\pgfpathcurveto{\pgfqpoint{2.181604in}{1.981610in}}{\pgfqpoint{2.177214in}{1.992209in}}{\pgfqpoint{2.169400in}{2.000023in}}%
\pgfpathcurveto{\pgfqpoint{2.161587in}{2.007836in}}{\pgfqpoint{2.150988in}{2.012226in}}{\pgfqpoint{2.139937in}{2.012226in}}%
\pgfpathcurveto{\pgfqpoint{2.128887in}{2.012226in}}{\pgfqpoint{2.118288in}{2.007836in}}{\pgfqpoint{2.110475in}{2.000023in}}%
\pgfpathcurveto{\pgfqpoint{2.102661in}{1.992209in}}{\pgfqpoint{2.098271in}{1.981610in}}{\pgfqpoint{2.098271in}{1.970560in}}%
\pgfpathcurveto{\pgfqpoint{2.098271in}{1.959510in}}{\pgfqpoint{2.102661in}{1.948911in}}{\pgfqpoint{2.110475in}{1.941097in}}%
\pgfpathcurveto{\pgfqpoint{2.118288in}{1.933283in}}{\pgfqpoint{2.128887in}{1.928893in}}{\pgfqpoint{2.139937in}{1.928893in}}%
\pgfpathclose%
\pgfusepath{stroke,fill}%
\end{pgfscope}%
\begin{pgfscope}%
\pgfpathrectangle{\pgfqpoint{0.375000in}{0.330000in}}{\pgfqpoint{2.325000in}{2.310000in}}%
\pgfusepath{clip}%
\pgfsetbuttcap%
\pgfsetroundjoin%
\definecolor{currentfill}{rgb}{0.000000,0.000000,0.000000}%
\pgfsetfillcolor{currentfill}%
\pgfsetlinewidth{1.003750pt}%
\definecolor{currentstroke}{rgb}{0.000000,0.000000,0.000000}%
\pgfsetstrokecolor{currentstroke}%
\pgfsetdash{}{0pt}%
\pgfpathmoveto{\pgfqpoint{2.139937in}{1.856076in}}%
\pgfpathcurveto{\pgfqpoint{2.150988in}{1.856076in}}{\pgfqpoint{2.161587in}{1.860466in}}{\pgfqpoint{2.169400in}{1.868280in}}%
\pgfpathcurveto{\pgfqpoint{2.177214in}{1.876093in}}{\pgfqpoint{2.181604in}{1.886692in}}{\pgfqpoint{2.181604in}{1.897742in}}%
\pgfpathcurveto{\pgfqpoint{2.181604in}{1.908793in}}{\pgfqpoint{2.177214in}{1.919392in}}{\pgfqpoint{2.169400in}{1.927205in}}%
\pgfpathcurveto{\pgfqpoint{2.161587in}{1.935019in}}{\pgfqpoint{2.150988in}{1.939409in}}{\pgfqpoint{2.139937in}{1.939409in}}%
\pgfpathcurveto{\pgfqpoint{2.128887in}{1.939409in}}{\pgfqpoint{2.118288in}{1.935019in}}{\pgfqpoint{2.110475in}{1.927205in}}%
\pgfpathcurveto{\pgfqpoint{2.102661in}{1.919392in}}{\pgfqpoint{2.098271in}{1.908793in}}{\pgfqpoint{2.098271in}{1.897742in}}%
\pgfpathcurveto{\pgfqpoint{2.098271in}{1.886692in}}{\pgfqpoint{2.102661in}{1.876093in}}{\pgfqpoint{2.110475in}{1.868280in}}%
\pgfpathcurveto{\pgfqpoint{2.118288in}{1.860466in}}{\pgfqpoint{2.128887in}{1.856076in}}{\pgfqpoint{2.139937in}{1.856076in}}%
\pgfpathclose%
\pgfusepath{stroke,fill}%
\end{pgfscope}%
\begin{pgfscope}%
\pgfpathrectangle{\pgfqpoint{0.375000in}{0.330000in}}{\pgfqpoint{2.325000in}{2.310000in}}%
\pgfusepath{clip}%
\pgfsetbuttcap%
\pgfsetroundjoin%
\definecolor{currentfill}{rgb}{0.000000,0.000000,0.000000}%
\pgfsetfillcolor{currentfill}%
\pgfsetlinewidth{1.003750pt}%
\definecolor{currentstroke}{rgb}{0.000000,0.000000,0.000000}%
\pgfsetstrokecolor{currentstroke}%
\pgfsetdash{}{0pt}%
\pgfpathmoveto{\pgfqpoint{2.139937in}{2.080596in}}%
\pgfpathcurveto{\pgfqpoint{2.150988in}{2.080596in}}{\pgfqpoint{2.161587in}{2.084986in}}{\pgfqpoint{2.169400in}{2.092800in}}%
\pgfpathcurveto{\pgfqpoint{2.177214in}{2.100613in}}{\pgfqpoint{2.181604in}{2.111212in}}{\pgfqpoint{2.181604in}{2.122262in}}%
\pgfpathcurveto{\pgfqpoint{2.181604in}{2.133313in}}{\pgfqpoint{2.177214in}{2.143912in}}{\pgfqpoint{2.169400in}{2.151725in}}%
\pgfpathcurveto{\pgfqpoint{2.161587in}{2.159539in}}{\pgfqpoint{2.150988in}{2.163929in}}{\pgfqpoint{2.139937in}{2.163929in}}%
\pgfpathcurveto{\pgfqpoint{2.128887in}{2.163929in}}{\pgfqpoint{2.118288in}{2.159539in}}{\pgfqpoint{2.110475in}{2.151725in}}%
\pgfpathcurveto{\pgfqpoint{2.102661in}{2.143912in}}{\pgfqpoint{2.098271in}{2.133313in}}{\pgfqpoint{2.098271in}{2.122262in}}%
\pgfpathcurveto{\pgfqpoint{2.098271in}{2.111212in}}{\pgfqpoint{2.102661in}{2.100613in}}{\pgfqpoint{2.110475in}{2.092800in}}%
\pgfpathcurveto{\pgfqpoint{2.118288in}{2.084986in}}{\pgfqpoint{2.128887in}{2.080596in}}{\pgfqpoint{2.139937in}{2.080596in}}%
\pgfpathclose%
\pgfusepath{stroke,fill}%
\end{pgfscope}%
\begin{pgfscope}%
\pgfpathrectangle{\pgfqpoint{0.375000in}{0.330000in}}{\pgfqpoint{2.325000in}{2.310000in}}%
\pgfusepath{clip}%
\pgfsetbuttcap%
\pgfsetroundjoin%
\definecolor{currentfill}{rgb}{0.000000,0.000000,0.000000}%
\pgfsetfillcolor{currentfill}%
\pgfsetlinewidth{1.003750pt}%
\definecolor{currentstroke}{rgb}{0.000000,0.000000,0.000000}%
\pgfsetstrokecolor{currentstroke}%
\pgfsetdash{}{0pt}%
\pgfpathmoveto{\pgfqpoint{2.139937in}{2.068460in}}%
\pgfpathcurveto{\pgfqpoint{2.150988in}{2.068460in}}{\pgfqpoint{2.161587in}{2.072850in}}{\pgfqpoint{2.169400in}{2.080663in}}%
\pgfpathcurveto{\pgfqpoint{2.177214in}{2.088477in}}{\pgfqpoint{2.181604in}{2.099076in}}{\pgfqpoint{2.181604in}{2.110126in}}%
\pgfpathcurveto{\pgfqpoint{2.181604in}{2.121176in}}{\pgfqpoint{2.177214in}{2.131775in}}{\pgfqpoint{2.169400in}{2.139589in}}%
\pgfpathcurveto{\pgfqpoint{2.161587in}{2.147403in}}{\pgfqpoint{2.150988in}{2.151793in}}{\pgfqpoint{2.139937in}{2.151793in}}%
\pgfpathcurveto{\pgfqpoint{2.128887in}{2.151793in}}{\pgfqpoint{2.118288in}{2.147403in}}{\pgfqpoint{2.110475in}{2.139589in}}%
\pgfpathcurveto{\pgfqpoint{2.102661in}{2.131775in}}{\pgfqpoint{2.098271in}{2.121176in}}{\pgfqpoint{2.098271in}{2.110126in}}%
\pgfpathcurveto{\pgfqpoint{2.098271in}{2.099076in}}{\pgfqpoint{2.102661in}{2.088477in}}{\pgfqpoint{2.110475in}{2.080663in}}%
\pgfpathcurveto{\pgfqpoint{2.118288in}{2.072850in}}{\pgfqpoint{2.128887in}{2.068460in}}{\pgfqpoint{2.139937in}{2.068460in}}%
\pgfpathclose%
\pgfusepath{stroke,fill}%
\end{pgfscope}%
\begin{pgfscope}%
\pgfpathrectangle{\pgfqpoint{0.375000in}{0.330000in}}{\pgfqpoint{2.325000in}{2.310000in}}%
\pgfusepath{clip}%
\pgfsetbuttcap%
\pgfsetroundjoin%
\definecolor{currentfill}{rgb}{0.000000,0.000000,0.000000}%
\pgfsetfillcolor{currentfill}%
\pgfsetlinewidth{1.003750pt}%
\definecolor{currentstroke}{rgb}{0.000000,0.000000,0.000000}%
\pgfsetstrokecolor{currentstroke}%
\pgfsetdash{}{0pt}%
\pgfpathmoveto{\pgfqpoint{2.139937in}{1.916757in}}%
\pgfpathcurveto{\pgfqpoint{2.150988in}{1.916757in}}{\pgfqpoint{2.161587in}{1.921147in}}{\pgfqpoint{2.169400in}{1.928961in}}%
\pgfpathcurveto{\pgfqpoint{2.177214in}{1.936774in}}{\pgfqpoint{2.181604in}{1.947373in}}{\pgfqpoint{2.181604in}{1.958424in}}%
\pgfpathcurveto{\pgfqpoint{2.181604in}{1.969474in}}{\pgfqpoint{2.177214in}{1.980073in}}{\pgfqpoint{2.169400in}{1.987886in}}%
\pgfpathcurveto{\pgfqpoint{2.161587in}{1.995700in}}{\pgfqpoint{2.150988in}{2.000090in}}{\pgfqpoint{2.139937in}{2.000090in}}%
\pgfpathcurveto{\pgfqpoint{2.128887in}{2.000090in}}{\pgfqpoint{2.118288in}{1.995700in}}{\pgfqpoint{2.110475in}{1.987886in}}%
\pgfpathcurveto{\pgfqpoint{2.102661in}{1.980073in}}{\pgfqpoint{2.098271in}{1.969474in}}{\pgfqpoint{2.098271in}{1.958424in}}%
\pgfpathcurveto{\pgfqpoint{2.098271in}{1.947373in}}{\pgfqpoint{2.102661in}{1.936774in}}{\pgfqpoint{2.110475in}{1.928961in}}%
\pgfpathcurveto{\pgfqpoint{2.118288in}{1.921147in}}{\pgfqpoint{2.128887in}{1.916757in}}{\pgfqpoint{2.139937in}{1.916757in}}%
\pgfpathclose%
\pgfusepath{stroke,fill}%
\end{pgfscope}%
\begin{pgfscope}%
\pgfpathrectangle{\pgfqpoint{0.375000in}{0.330000in}}{\pgfqpoint{2.325000in}{2.310000in}}%
\pgfusepath{clip}%
\pgfsetbuttcap%
\pgfsetroundjoin%
\definecolor{currentfill}{rgb}{0.000000,0.000000,0.000000}%
\pgfsetfillcolor{currentfill}%
\pgfsetlinewidth{1.003750pt}%
\definecolor{currentstroke}{rgb}{0.000000,0.000000,0.000000}%
\pgfsetstrokecolor{currentstroke}%
\pgfsetdash{}{0pt}%
\pgfpathmoveto{\pgfqpoint{2.139937in}{1.801463in}}%
\pgfpathcurveto{\pgfqpoint{2.150988in}{1.801463in}}{\pgfqpoint{2.161587in}{1.805853in}}{\pgfqpoint{2.169400in}{1.813667in}}%
\pgfpathcurveto{\pgfqpoint{2.177214in}{1.821480in}}{\pgfqpoint{2.181604in}{1.832079in}}{\pgfqpoint{2.181604in}{1.843130in}}%
\pgfpathcurveto{\pgfqpoint{2.181604in}{1.854180in}}{\pgfqpoint{2.177214in}{1.864779in}}{\pgfqpoint{2.169400in}{1.872592in}}%
\pgfpathcurveto{\pgfqpoint{2.161587in}{1.880406in}}{\pgfqpoint{2.150988in}{1.884796in}}{\pgfqpoint{2.139937in}{1.884796in}}%
\pgfpathcurveto{\pgfqpoint{2.128887in}{1.884796in}}{\pgfqpoint{2.118288in}{1.880406in}}{\pgfqpoint{2.110475in}{1.872592in}}%
\pgfpathcurveto{\pgfqpoint{2.102661in}{1.864779in}}{\pgfqpoint{2.098271in}{1.854180in}}{\pgfqpoint{2.098271in}{1.843130in}}%
\pgfpathcurveto{\pgfqpoint{2.098271in}{1.832079in}}{\pgfqpoint{2.102661in}{1.821480in}}{\pgfqpoint{2.110475in}{1.813667in}}%
\pgfpathcurveto{\pgfqpoint{2.118288in}{1.805853in}}{\pgfqpoint{2.128887in}{1.801463in}}{\pgfqpoint{2.139937in}{1.801463in}}%
\pgfpathclose%
\pgfusepath{stroke,fill}%
\end{pgfscope}%
\begin{pgfscope}%
\pgfpathrectangle{\pgfqpoint{0.375000in}{0.330000in}}{\pgfqpoint{2.325000in}{2.310000in}}%
\pgfusepath{clip}%
\pgfsetbuttcap%
\pgfsetroundjoin%
\definecolor{currentfill}{rgb}{0.000000,0.000000,0.000000}%
\pgfsetfillcolor{currentfill}%
\pgfsetlinewidth{1.003750pt}%
\definecolor{currentstroke}{rgb}{0.000000,0.000000,0.000000}%
\pgfsetstrokecolor{currentstroke}%
\pgfsetdash{}{0pt}%
\pgfpathmoveto{\pgfqpoint{2.139937in}{2.025983in}}%
\pgfpathcurveto{\pgfqpoint{2.150988in}{2.025983in}}{\pgfqpoint{2.161587in}{2.030373in}}{\pgfqpoint{2.169400in}{2.038187in}}%
\pgfpathcurveto{\pgfqpoint{2.177214in}{2.046000in}}{\pgfqpoint{2.181604in}{2.056599in}}{\pgfqpoint{2.181604in}{2.067649in}}%
\pgfpathcurveto{\pgfqpoint{2.181604in}{2.078700in}}{\pgfqpoint{2.177214in}{2.089299in}}{\pgfqpoint{2.169400in}{2.097112in}}%
\pgfpathcurveto{\pgfqpoint{2.161587in}{2.104926in}}{\pgfqpoint{2.150988in}{2.109316in}}{\pgfqpoint{2.139937in}{2.109316in}}%
\pgfpathcurveto{\pgfqpoint{2.128887in}{2.109316in}}{\pgfqpoint{2.118288in}{2.104926in}}{\pgfqpoint{2.110475in}{2.097112in}}%
\pgfpathcurveto{\pgfqpoint{2.102661in}{2.089299in}}{\pgfqpoint{2.098271in}{2.078700in}}{\pgfqpoint{2.098271in}{2.067649in}}%
\pgfpathcurveto{\pgfqpoint{2.098271in}{2.056599in}}{\pgfqpoint{2.102661in}{2.046000in}}{\pgfqpoint{2.110475in}{2.038187in}}%
\pgfpathcurveto{\pgfqpoint{2.118288in}{2.030373in}}{\pgfqpoint{2.128887in}{2.025983in}}{\pgfqpoint{2.139937in}{2.025983in}}%
\pgfpathclose%
\pgfusepath{stroke,fill}%
\end{pgfscope}%
\begin{pgfscope}%
\pgfpathrectangle{\pgfqpoint{0.375000in}{0.330000in}}{\pgfqpoint{2.325000in}{2.310000in}}%
\pgfusepath{clip}%
\pgfsetbuttcap%
\pgfsetroundjoin%
\definecolor{currentfill}{rgb}{0.000000,0.000000,0.000000}%
\pgfsetfillcolor{currentfill}%
\pgfsetlinewidth{1.003750pt}%
\definecolor{currentstroke}{rgb}{0.000000,0.000000,0.000000}%
\pgfsetstrokecolor{currentstroke}%
\pgfsetdash{}{0pt}%
\pgfpathmoveto{\pgfqpoint{2.139937in}{1.868212in}}%
\pgfpathcurveto{\pgfqpoint{2.150988in}{1.868212in}}{\pgfqpoint{2.161587in}{1.872602in}}{\pgfqpoint{2.169400in}{1.880416in}}%
\pgfpathcurveto{\pgfqpoint{2.177214in}{1.888230in}}{\pgfqpoint{2.181604in}{1.898829in}}{\pgfqpoint{2.181604in}{1.909879in}}%
\pgfpathcurveto{\pgfqpoint{2.181604in}{1.920929in}}{\pgfqpoint{2.177214in}{1.931528in}}{\pgfqpoint{2.169400in}{1.939341in}}%
\pgfpathcurveto{\pgfqpoint{2.161587in}{1.947155in}}{\pgfqpoint{2.150988in}{1.951545in}}{\pgfqpoint{2.139937in}{1.951545in}}%
\pgfpathcurveto{\pgfqpoint{2.128887in}{1.951545in}}{\pgfqpoint{2.118288in}{1.947155in}}{\pgfqpoint{2.110475in}{1.939341in}}%
\pgfpathcurveto{\pgfqpoint{2.102661in}{1.931528in}}{\pgfqpoint{2.098271in}{1.920929in}}{\pgfqpoint{2.098271in}{1.909879in}}%
\pgfpathcurveto{\pgfqpoint{2.098271in}{1.898829in}}{\pgfqpoint{2.102661in}{1.888230in}}{\pgfqpoint{2.110475in}{1.880416in}}%
\pgfpathcurveto{\pgfqpoint{2.118288in}{1.872602in}}{\pgfqpoint{2.128887in}{1.868212in}}{\pgfqpoint{2.139937in}{1.868212in}}%
\pgfpathclose%
\pgfusepath{stroke,fill}%
\end{pgfscope}%
\begin{pgfscope}%
\pgfpathrectangle{\pgfqpoint{0.375000in}{0.330000in}}{\pgfqpoint{2.325000in}{2.310000in}}%
\pgfusepath{clip}%
\pgfsetbuttcap%
\pgfsetroundjoin%
\definecolor{currentfill}{rgb}{0.000000,0.000000,0.000000}%
\pgfsetfillcolor{currentfill}%
\pgfsetlinewidth{1.003750pt}%
\definecolor{currentstroke}{rgb}{0.000000,0.000000,0.000000}%
\pgfsetstrokecolor{currentstroke}%
\pgfsetdash{}{0pt}%
\pgfpathmoveto{\pgfqpoint{2.139937in}{2.062391in}}%
\pgfpathcurveto{\pgfqpoint{2.150988in}{2.062391in}}{\pgfqpoint{2.161587in}{2.066782in}}{\pgfqpoint{2.169400in}{2.074595in}}%
\pgfpathcurveto{\pgfqpoint{2.177214in}{2.082409in}}{\pgfqpoint{2.181604in}{2.093008in}}{\pgfqpoint{2.181604in}{2.104058in}}%
\pgfpathcurveto{\pgfqpoint{2.181604in}{2.115108in}}{\pgfqpoint{2.177214in}{2.125707in}}{\pgfqpoint{2.169400in}{2.133521in}}%
\pgfpathcurveto{\pgfqpoint{2.161587in}{2.141334in}}{\pgfqpoint{2.150988in}{2.145725in}}{\pgfqpoint{2.139937in}{2.145725in}}%
\pgfpathcurveto{\pgfqpoint{2.128887in}{2.145725in}}{\pgfqpoint{2.118288in}{2.141334in}}{\pgfqpoint{2.110475in}{2.133521in}}%
\pgfpathcurveto{\pgfqpoint{2.102661in}{2.125707in}}{\pgfqpoint{2.098271in}{2.115108in}}{\pgfqpoint{2.098271in}{2.104058in}}%
\pgfpathcurveto{\pgfqpoint{2.098271in}{2.093008in}}{\pgfqpoint{2.102661in}{2.082409in}}{\pgfqpoint{2.110475in}{2.074595in}}%
\pgfpathcurveto{\pgfqpoint{2.118288in}{2.066782in}}{\pgfqpoint{2.128887in}{2.062391in}}{\pgfqpoint{2.139937in}{2.062391in}}%
\pgfpathclose%
\pgfusepath{stroke,fill}%
\end{pgfscope}%
\begin{pgfscope}%
\pgfpathrectangle{\pgfqpoint{0.375000in}{0.330000in}}{\pgfqpoint{2.325000in}{2.310000in}}%
\pgfusepath{clip}%
\pgfsetbuttcap%
\pgfsetroundjoin%
\definecolor{currentfill}{rgb}{0.000000,0.000000,0.000000}%
\pgfsetfillcolor{currentfill}%
\pgfsetlinewidth{1.003750pt}%
\definecolor{currentstroke}{rgb}{0.000000,0.000000,0.000000}%
\pgfsetstrokecolor{currentstroke}%
\pgfsetdash{}{0pt}%
\pgfpathmoveto{\pgfqpoint{2.139937in}{2.068460in}}%
\pgfpathcurveto{\pgfqpoint{2.150988in}{2.068460in}}{\pgfqpoint{2.161587in}{2.072850in}}{\pgfqpoint{2.169400in}{2.080663in}}%
\pgfpathcurveto{\pgfqpoint{2.177214in}{2.088477in}}{\pgfqpoint{2.181604in}{2.099076in}}{\pgfqpoint{2.181604in}{2.110126in}}%
\pgfpathcurveto{\pgfqpoint{2.181604in}{2.121176in}}{\pgfqpoint{2.177214in}{2.131775in}}{\pgfqpoint{2.169400in}{2.139589in}}%
\pgfpathcurveto{\pgfqpoint{2.161587in}{2.147403in}}{\pgfqpoint{2.150988in}{2.151793in}}{\pgfqpoint{2.139937in}{2.151793in}}%
\pgfpathcurveto{\pgfqpoint{2.128887in}{2.151793in}}{\pgfqpoint{2.118288in}{2.147403in}}{\pgfqpoint{2.110475in}{2.139589in}}%
\pgfpathcurveto{\pgfqpoint{2.102661in}{2.131775in}}{\pgfqpoint{2.098271in}{2.121176in}}{\pgfqpoint{2.098271in}{2.110126in}}%
\pgfpathcurveto{\pgfqpoint{2.098271in}{2.099076in}}{\pgfqpoint{2.102661in}{2.088477in}}{\pgfqpoint{2.110475in}{2.080663in}}%
\pgfpathcurveto{\pgfqpoint{2.118288in}{2.072850in}}{\pgfqpoint{2.128887in}{2.068460in}}{\pgfqpoint{2.139937in}{2.068460in}}%
\pgfpathclose%
\pgfusepath{stroke,fill}%
\end{pgfscope}%
\begin{pgfscope}%
\pgfpathrectangle{\pgfqpoint{0.375000in}{0.330000in}}{\pgfqpoint{2.325000in}{2.310000in}}%
\pgfusepath{clip}%
\pgfsetbuttcap%
\pgfsetroundjoin%
\definecolor{currentfill}{rgb}{0.000000,0.000000,0.000000}%
\pgfsetfillcolor{currentfill}%
\pgfsetlinewidth{1.003750pt}%
\definecolor{currentstroke}{rgb}{0.000000,0.000000,0.000000}%
\pgfsetstrokecolor{currentstroke}%
\pgfsetdash{}{0pt}%
\pgfpathmoveto{\pgfqpoint{2.139937in}{1.904621in}}%
\pgfpathcurveto{\pgfqpoint{2.150988in}{1.904621in}}{\pgfqpoint{2.161587in}{1.909011in}}{\pgfqpoint{2.169400in}{1.916825in}}%
\pgfpathcurveto{\pgfqpoint{2.177214in}{1.924638in}}{\pgfqpoint{2.181604in}{1.935237in}}{\pgfqpoint{2.181604in}{1.946287in}}%
\pgfpathcurveto{\pgfqpoint{2.181604in}{1.957337in}}{\pgfqpoint{2.177214in}{1.967936in}}{\pgfqpoint{2.169400in}{1.975750in}}%
\pgfpathcurveto{\pgfqpoint{2.161587in}{1.983564in}}{\pgfqpoint{2.150988in}{1.987954in}}{\pgfqpoint{2.139937in}{1.987954in}}%
\pgfpathcurveto{\pgfqpoint{2.128887in}{1.987954in}}{\pgfqpoint{2.118288in}{1.983564in}}{\pgfqpoint{2.110475in}{1.975750in}}%
\pgfpathcurveto{\pgfqpoint{2.102661in}{1.967936in}}{\pgfqpoint{2.098271in}{1.957337in}}{\pgfqpoint{2.098271in}{1.946287in}}%
\pgfpathcurveto{\pgfqpoint{2.098271in}{1.935237in}}{\pgfqpoint{2.102661in}{1.924638in}}{\pgfqpoint{2.110475in}{1.916825in}}%
\pgfpathcurveto{\pgfqpoint{2.118288in}{1.909011in}}{\pgfqpoint{2.128887in}{1.904621in}}{\pgfqpoint{2.139937in}{1.904621in}}%
\pgfpathclose%
\pgfusepath{stroke,fill}%
\end{pgfscope}%
\begin{pgfscope}%
\pgfpathrectangle{\pgfqpoint{0.375000in}{0.330000in}}{\pgfqpoint{2.325000in}{2.310000in}}%
\pgfusepath{clip}%
\pgfsetbuttcap%
\pgfsetroundjoin%
\definecolor{currentfill}{rgb}{0.000000,0.000000,0.000000}%
\pgfsetfillcolor{currentfill}%
\pgfsetlinewidth{1.003750pt}%
\definecolor{currentstroke}{rgb}{0.000000,0.000000,0.000000}%
\pgfsetstrokecolor{currentstroke}%
\pgfsetdash{}{0pt}%
\pgfpathmoveto{\pgfqpoint{2.139937in}{2.165549in}}%
\pgfpathcurveto{\pgfqpoint{2.150988in}{2.165549in}}{\pgfqpoint{2.161587in}{2.169939in}}{\pgfqpoint{2.169400in}{2.177753in}}%
\pgfpathcurveto{\pgfqpoint{2.177214in}{2.185567in}}{\pgfqpoint{2.181604in}{2.196166in}}{\pgfqpoint{2.181604in}{2.207216in}}%
\pgfpathcurveto{\pgfqpoint{2.181604in}{2.218266in}}{\pgfqpoint{2.177214in}{2.228865in}}{\pgfqpoint{2.169400in}{2.236679in}}%
\pgfpathcurveto{\pgfqpoint{2.161587in}{2.244492in}}{\pgfqpoint{2.150988in}{2.248883in}}{\pgfqpoint{2.139937in}{2.248883in}}%
\pgfpathcurveto{\pgfqpoint{2.128887in}{2.248883in}}{\pgfqpoint{2.118288in}{2.244492in}}{\pgfqpoint{2.110475in}{2.236679in}}%
\pgfpathcurveto{\pgfqpoint{2.102661in}{2.228865in}}{\pgfqpoint{2.098271in}{2.218266in}}{\pgfqpoint{2.098271in}{2.207216in}}%
\pgfpathcurveto{\pgfqpoint{2.098271in}{2.196166in}}{\pgfqpoint{2.102661in}{2.185567in}}{\pgfqpoint{2.110475in}{2.177753in}}%
\pgfpathcurveto{\pgfqpoint{2.118288in}{2.169939in}}{\pgfqpoint{2.128887in}{2.165549in}}{\pgfqpoint{2.139937in}{2.165549in}}%
\pgfpathclose%
\pgfusepath{stroke,fill}%
\end{pgfscope}%
\begin{pgfscope}%
\pgfpathrectangle{\pgfqpoint{0.375000in}{0.330000in}}{\pgfqpoint{2.325000in}{2.310000in}}%
\pgfusepath{clip}%
\pgfsetbuttcap%
\pgfsetroundjoin%
\definecolor{currentfill}{rgb}{0.000000,0.000000,0.000000}%
\pgfsetfillcolor{currentfill}%
\pgfsetlinewidth{1.003750pt}%
\definecolor{currentstroke}{rgb}{0.000000,0.000000,0.000000}%
\pgfsetstrokecolor{currentstroke}%
\pgfsetdash{}{0pt}%
\pgfpathmoveto{\pgfqpoint{2.139937in}{2.104868in}}%
\pgfpathcurveto{\pgfqpoint{2.150988in}{2.104868in}}{\pgfqpoint{2.161587in}{2.109258in}}{\pgfqpoint{2.169400in}{2.117072in}}%
\pgfpathcurveto{\pgfqpoint{2.177214in}{2.124886in}}{\pgfqpoint{2.181604in}{2.135485in}}{\pgfqpoint{2.181604in}{2.146535in}}%
\pgfpathcurveto{\pgfqpoint{2.181604in}{2.157585in}}{\pgfqpoint{2.177214in}{2.168184in}}{\pgfqpoint{2.169400in}{2.175998in}}%
\pgfpathcurveto{\pgfqpoint{2.161587in}{2.183811in}}{\pgfqpoint{2.150988in}{2.188201in}}{\pgfqpoint{2.139937in}{2.188201in}}%
\pgfpathcurveto{\pgfqpoint{2.128887in}{2.188201in}}{\pgfqpoint{2.118288in}{2.183811in}}{\pgfqpoint{2.110475in}{2.175998in}}%
\pgfpathcurveto{\pgfqpoint{2.102661in}{2.168184in}}{\pgfqpoint{2.098271in}{2.157585in}}{\pgfqpoint{2.098271in}{2.146535in}}%
\pgfpathcurveto{\pgfqpoint{2.098271in}{2.135485in}}{\pgfqpoint{2.102661in}{2.124886in}}{\pgfqpoint{2.110475in}{2.117072in}}%
\pgfpathcurveto{\pgfqpoint{2.118288in}{2.109258in}}{\pgfqpoint{2.128887in}{2.104868in}}{\pgfqpoint{2.139937in}{2.104868in}}%
\pgfpathclose%
\pgfusepath{stroke,fill}%
\end{pgfscope}%
\begin{pgfscope}%
\pgfpathrectangle{\pgfqpoint{0.375000in}{0.330000in}}{\pgfqpoint{2.325000in}{2.310000in}}%
\pgfusepath{clip}%
\pgfsetbuttcap%
\pgfsetroundjoin%
\definecolor{currentfill}{rgb}{0.000000,0.000000,0.000000}%
\pgfsetfillcolor{currentfill}%
\pgfsetlinewidth{1.003750pt}%
\definecolor{currentstroke}{rgb}{0.000000,0.000000,0.000000}%
\pgfsetstrokecolor{currentstroke}%
\pgfsetdash{}{0pt}%
\pgfpathmoveto{\pgfqpoint{2.139937in}{2.195890in}}%
\pgfpathcurveto{\pgfqpoint{2.150988in}{2.195890in}}{\pgfqpoint{2.161587in}{2.200280in}}{\pgfqpoint{2.169400in}{2.208094in}}%
\pgfpathcurveto{\pgfqpoint{2.177214in}{2.215907in}}{\pgfqpoint{2.181604in}{2.226506in}}{\pgfqpoint{2.181604in}{2.237556in}}%
\pgfpathcurveto{\pgfqpoint{2.181604in}{2.248607in}}{\pgfqpoint{2.177214in}{2.259206in}}{\pgfqpoint{2.169400in}{2.267019in}}%
\pgfpathcurveto{\pgfqpoint{2.161587in}{2.274833in}}{\pgfqpoint{2.150988in}{2.279223in}}{\pgfqpoint{2.139937in}{2.279223in}}%
\pgfpathcurveto{\pgfqpoint{2.128887in}{2.279223in}}{\pgfqpoint{2.118288in}{2.274833in}}{\pgfqpoint{2.110475in}{2.267019in}}%
\pgfpathcurveto{\pgfqpoint{2.102661in}{2.259206in}}{\pgfqpoint{2.098271in}{2.248607in}}{\pgfqpoint{2.098271in}{2.237556in}}%
\pgfpathcurveto{\pgfqpoint{2.098271in}{2.226506in}}{\pgfqpoint{2.102661in}{2.215907in}}{\pgfqpoint{2.110475in}{2.208094in}}%
\pgfpathcurveto{\pgfqpoint{2.118288in}{2.200280in}}{\pgfqpoint{2.128887in}{2.195890in}}{\pgfqpoint{2.139937in}{2.195890in}}%
\pgfpathclose%
\pgfusepath{stroke,fill}%
\end{pgfscope}%
\begin{pgfscope}%
\pgfpathrectangle{\pgfqpoint{0.375000in}{0.330000in}}{\pgfqpoint{2.325000in}{2.310000in}}%
\pgfusepath{clip}%
\pgfsetbuttcap%
\pgfsetroundjoin%
\definecolor{currentfill}{rgb}{0.000000,0.000000,0.000000}%
\pgfsetfillcolor{currentfill}%
\pgfsetlinewidth{1.003750pt}%
\definecolor{currentstroke}{rgb}{0.000000,0.000000,0.000000}%
\pgfsetstrokecolor{currentstroke}%
\pgfsetdash{}{0pt}%
\pgfpathmoveto{\pgfqpoint{2.139937in}{2.104868in}}%
\pgfpathcurveto{\pgfqpoint{2.150988in}{2.104868in}}{\pgfqpoint{2.161587in}{2.109258in}}{\pgfqpoint{2.169400in}{2.117072in}}%
\pgfpathcurveto{\pgfqpoint{2.177214in}{2.124886in}}{\pgfqpoint{2.181604in}{2.135485in}}{\pgfqpoint{2.181604in}{2.146535in}}%
\pgfpathcurveto{\pgfqpoint{2.181604in}{2.157585in}}{\pgfqpoint{2.177214in}{2.168184in}}{\pgfqpoint{2.169400in}{2.175998in}}%
\pgfpathcurveto{\pgfqpoint{2.161587in}{2.183811in}}{\pgfqpoint{2.150988in}{2.188201in}}{\pgfqpoint{2.139937in}{2.188201in}}%
\pgfpathcurveto{\pgfqpoint{2.128887in}{2.188201in}}{\pgfqpoint{2.118288in}{2.183811in}}{\pgfqpoint{2.110475in}{2.175998in}}%
\pgfpathcurveto{\pgfqpoint{2.102661in}{2.168184in}}{\pgfqpoint{2.098271in}{2.157585in}}{\pgfqpoint{2.098271in}{2.146535in}}%
\pgfpathcurveto{\pgfqpoint{2.098271in}{2.135485in}}{\pgfqpoint{2.102661in}{2.124886in}}{\pgfqpoint{2.110475in}{2.117072in}}%
\pgfpathcurveto{\pgfqpoint{2.118288in}{2.109258in}}{\pgfqpoint{2.128887in}{2.104868in}}{\pgfqpoint{2.139937in}{2.104868in}}%
\pgfpathclose%
\pgfusepath{stroke,fill}%
\end{pgfscope}%
\begin{pgfscope}%
\pgfpathrectangle{\pgfqpoint{0.375000in}{0.330000in}}{\pgfqpoint{2.325000in}{2.310000in}}%
\pgfusepath{clip}%
\pgfsetbuttcap%
\pgfsetroundjoin%
\definecolor{currentfill}{rgb}{0.000000,0.000000,0.000000}%
\pgfsetfillcolor{currentfill}%
\pgfsetlinewidth{1.003750pt}%
\definecolor{currentstroke}{rgb}{0.000000,0.000000,0.000000}%
\pgfsetstrokecolor{currentstroke}%
\pgfsetdash{}{0pt}%
\pgfpathmoveto{\pgfqpoint{2.139937in}{1.953165in}}%
\pgfpathcurveto{\pgfqpoint{2.150988in}{1.953165in}}{\pgfqpoint{2.161587in}{1.957556in}}{\pgfqpoint{2.169400in}{1.965369in}}%
\pgfpathcurveto{\pgfqpoint{2.177214in}{1.973183in}}{\pgfqpoint{2.181604in}{1.983782in}}{\pgfqpoint{2.181604in}{1.994832in}}%
\pgfpathcurveto{\pgfqpoint{2.181604in}{2.005882in}}{\pgfqpoint{2.177214in}{2.016481in}}{\pgfqpoint{2.169400in}{2.024295in}}%
\pgfpathcurveto{\pgfqpoint{2.161587in}{2.032109in}}{\pgfqpoint{2.150988in}{2.036499in}}{\pgfqpoint{2.139937in}{2.036499in}}%
\pgfpathcurveto{\pgfqpoint{2.128887in}{2.036499in}}{\pgfqpoint{2.118288in}{2.032109in}}{\pgfqpoint{2.110475in}{2.024295in}}%
\pgfpathcurveto{\pgfqpoint{2.102661in}{2.016481in}}{\pgfqpoint{2.098271in}{2.005882in}}{\pgfqpoint{2.098271in}{1.994832in}}%
\pgfpathcurveto{\pgfqpoint{2.098271in}{1.983782in}}{\pgfqpoint{2.102661in}{1.973183in}}{\pgfqpoint{2.110475in}{1.965369in}}%
\pgfpathcurveto{\pgfqpoint{2.118288in}{1.957556in}}{\pgfqpoint{2.128887in}{1.953165in}}{\pgfqpoint{2.139937in}{1.953165in}}%
\pgfpathclose%
\pgfusepath{stroke,fill}%
\end{pgfscope}%
\begin{pgfscope}%
\pgfpathrectangle{\pgfqpoint{0.375000in}{0.330000in}}{\pgfqpoint{2.325000in}{2.310000in}}%
\pgfusepath{clip}%
\pgfsetbuttcap%
\pgfsetroundjoin%
\definecolor{currentfill}{rgb}{0.000000,0.000000,0.000000}%
\pgfsetfillcolor{currentfill}%
\pgfsetlinewidth{1.003750pt}%
\definecolor{currentstroke}{rgb}{0.000000,0.000000,0.000000}%
\pgfsetstrokecolor{currentstroke}%
\pgfsetdash{}{0pt}%
\pgfpathmoveto{\pgfqpoint{2.139937in}{2.062391in}}%
\pgfpathcurveto{\pgfqpoint{2.150988in}{2.062391in}}{\pgfqpoint{2.161587in}{2.066782in}}{\pgfqpoint{2.169400in}{2.074595in}}%
\pgfpathcurveto{\pgfqpoint{2.177214in}{2.082409in}}{\pgfqpoint{2.181604in}{2.093008in}}{\pgfqpoint{2.181604in}{2.104058in}}%
\pgfpathcurveto{\pgfqpoint{2.181604in}{2.115108in}}{\pgfqpoint{2.177214in}{2.125707in}}{\pgfqpoint{2.169400in}{2.133521in}}%
\pgfpathcurveto{\pgfqpoint{2.161587in}{2.141334in}}{\pgfqpoint{2.150988in}{2.145725in}}{\pgfqpoint{2.139937in}{2.145725in}}%
\pgfpathcurveto{\pgfqpoint{2.128887in}{2.145725in}}{\pgfqpoint{2.118288in}{2.141334in}}{\pgfqpoint{2.110475in}{2.133521in}}%
\pgfpathcurveto{\pgfqpoint{2.102661in}{2.125707in}}{\pgfqpoint{2.098271in}{2.115108in}}{\pgfqpoint{2.098271in}{2.104058in}}%
\pgfpathcurveto{\pgfqpoint{2.098271in}{2.093008in}}{\pgfqpoint{2.102661in}{2.082409in}}{\pgfqpoint{2.110475in}{2.074595in}}%
\pgfpathcurveto{\pgfqpoint{2.118288in}{2.066782in}}{\pgfqpoint{2.128887in}{2.062391in}}{\pgfqpoint{2.139937in}{2.062391in}}%
\pgfpathclose%
\pgfusepath{stroke,fill}%
\end{pgfscope}%
\begin{pgfscope}%
\pgfpathrectangle{\pgfqpoint{0.375000in}{0.330000in}}{\pgfqpoint{2.325000in}{2.310000in}}%
\pgfusepath{clip}%
\pgfsetbuttcap%
\pgfsetroundjoin%
\definecolor{currentfill}{rgb}{0.000000,0.000000,0.000000}%
\pgfsetfillcolor{currentfill}%
\pgfsetlinewidth{1.003750pt}%
\definecolor{currentstroke}{rgb}{0.000000,0.000000,0.000000}%
\pgfsetstrokecolor{currentstroke}%
\pgfsetdash{}{0pt}%
\pgfpathmoveto{\pgfqpoint{2.139937in}{1.843940in}}%
\pgfpathcurveto{\pgfqpoint{2.150988in}{1.843940in}}{\pgfqpoint{2.161587in}{1.848330in}}{\pgfqpoint{2.169400in}{1.856143in}}%
\pgfpathcurveto{\pgfqpoint{2.177214in}{1.863957in}}{\pgfqpoint{2.181604in}{1.874556in}}{\pgfqpoint{2.181604in}{1.885606in}}%
\pgfpathcurveto{\pgfqpoint{2.181604in}{1.896656in}}{\pgfqpoint{2.177214in}{1.907255in}}{\pgfqpoint{2.169400in}{1.915069in}}%
\pgfpathcurveto{\pgfqpoint{2.161587in}{1.922883in}}{\pgfqpoint{2.150988in}{1.927273in}}{\pgfqpoint{2.139937in}{1.927273in}}%
\pgfpathcurveto{\pgfqpoint{2.128887in}{1.927273in}}{\pgfqpoint{2.118288in}{1.922883in}}{\pgfqpoint{2.110475in}{1.915069in}}%
\pgfpathcurveto{\pgfqpoint{2.102661in}{1.907255in}}{\pgfqpoint{2.098271in}{1.896656in}}{\pgfqpoint{2.098271in}{1.885606in}}%
\pgfpathcurveto{\pgfqpoint{2.098271in}{1.874556in}}{\pgfqpoint{2.102661in}{1.863957in}}{\pgfqpoint{2.110475in}{1.856143in}}%
\pgfpathcurveto{\pgfqpoint{2.118288in}{1.848330in}}{\pgfqpoint{2.128887in}{1.843940in}}{\pgfqpoint{2.139937in}{1.843940in}}%
\pgfpathclose%
\pgfusepath{stroke,fill}%
\end{pgfscope}%
\begin{pgfscope}%
\pgfpathrectangle{\pgfqpoint{0.375000in}{0.330000in}}{\pgfqpoint{2.325000in}{2.310000in}}%
\pgfusepath{clip}%
\pgfsetbuttcap%
\pgfsetroundjoin%
\definecolor{currentfill}{rgb}{0.000000,0.000000,0.000000}%
\pgfsetfillcolor{currentfill}%
\pgfsetlinewidth{1.003750pt}%
\definecolor{currentstroke}{rgb}{0.000000,0.000000,0.000000}%
\pgfsetstrokecolor{currentstroke}%
\pgfsetdash{}{0pt}%
\pgfpathmoveto{\pgfqpoint{2.139937in}{1.941029in}}%
\pgfpathcurveto{\pgfqpoint{2.150988in}{1.941029in}}{\pgfqpoint{2.161587in}{1.945420in}}{\pgfqpoint{2.169400in}{1.953233in}}%
\pgfpathcurveto{\pgfqpoint{2.177214in}{1.961047in}}{\pgfqpoint{2.181604in}{1.971646in}}{\pgfqpoint{2.181604in}{1.982696in}}%
\pgfpathcurveto{\pgfqpoint{2.181604in}{1.993746in}}{\pgfqpoint{2.177214in}{2.004345in}}{\pgfqpoint{2.169400in}{2.012159in}}%
\pgfpathcurveto{\pgfqpoint{2.161587in}{2.019972in}}{\pgfqpoint{2.150988in}{2.024363in}}{\pgfqpoint{2.139937in}{2.024363in}}%
\pgfpathcurveto{\pgfqpoint{2.128887in}{2.024363in}}{\pgfqpoint{2.118288in}{2.019972in}}{\pgfqpoint{2.110475in}{2.012159in}}%
\pgfpathcurveto{\pgfqpoint{2.102661in}{2.004345in}}{\pgfqpoint{2.098271in}{1.993746in}}{\pgfqpoint{2.098271in}{1.982696in}}%
\pgfpathcurveto{\pgfqpoint{2.098271in}{1.971646in}}{\pgfqpoint{2.102661in}{1.961047in}}{\pgfqpoint{2.110475in}{1.953233in}}%
\pgfpathcurveto{\pgfqpoint{2.118288in}{1.945420in}}{\pgfqpoint{2.128887in}{1.941029in}}{\pgfqpoint{2.139937in}{1.941029in}}%
\pgfpathclose%
\pgfusepath{stroke,fill}%
\end{pgfscope}%
\begin{pgfscope}%
\pgfpathrectangle{\pgfqpoint{0.375000in}{0.330000in}}{\pgfqpoint{2.325000in}{2.310000in}}%
\pgfusepath{clip}%
\pgfsetbuttcap%
\pgfsetroundjoin%
\definecolor{currentfill}{rgb}{0.000000,0.000000,0.000000}%
\pgfsetfillcolor{currentfill}%
\pgfsetlinewidth{1.003750pt}%
\definecolor{currentstroke}{rgb}{0.000000,0.000000,0.000000}%
\pgfsetstrokecolor{currentstroke}%
\pgfsetdash{}{0pt}%
\pgfpathmoveto{\pgfqpoint{2.139937in}{1.971370in}}%
\pgfpathcurveto{\pgfqpoint{2.150988in}{1.971370in}}{\pgfqpoint{2.161587in}{1.975760in}}{\pgfqpoint{2.169400in}{1.983574in}}%
\pgfpathcurveto{\pgfqpoint{2.177214in}{1.991387in}}{\pgfqpoint{2.181604in}{2.001986in}}{\pgfqpoint{2.181604in}{2.013036in}}%
\pgfpathcurveto{\pgfqpoint{2.181604in}{2.024087in}}{\pgfqpoint{2.177214in}{2.034686in}}{\pgfqpoint{2.169400in}{2.042499in}}%
\pgfpathcurveto{\pgfqpoint{2.161587in}{2.050313in}}{\pgfqpoint{2.150988in}{2.054703in}}{\pgfqpoint{2.139937in}{2.054703in}}%
\pgfpathcurveto{\pgfqpoint{2.128887in}{2.054703in}}{\pgfqpoint{2.118288in}{2.050313in}}{\pgfqpoint{2.110475in}{2.042499in}}%
\pgfpathcurveto{\pgfqpoint{2.102661in}{2.034686in}}{\pgfqpoint{2.098271in}{2.024087in}}{\pgfqpoint{2.098271in}{2.013036in}}%
\pgfpathcurveto{\pgfqpoint{2.098271in}{2.001986in}}{\pgfqpoint{2.102661in}{1.991387in}}{\pgfqpoint{2.110475in}{1.983574in}}%
\pgfpathcurveto{\pgfqpoint{2.118288in}{1.975760in}}{\pgfqpoint{2.128887in}{1.971370in}}{\pgfqpoint{2.139937in}{1.971370in}}%
\pgfpathclose%
\pgfusepath{stroke,fill}%
\end{pgfscope}%
\begin{pgfscope}%
\pgfpathrectangle{\pgfqpoint{0.375000in}{0.330000in}}{\pgfqpoint{2.325000in}{2.310000in}}%
\pgfusepath{clip}%
\pgfsetbuttcap%
\pgfsetroundjoin%
\definecolor{currentfill}{rgb}{0.000000,0.000000,0.000000}%
\pgfsetfillcolor{currentfill}%
\pgfsetlinewidth{1.003750pt}%
\definecolor{currentstroke}{rgb}{0.000000,0.000000,0.000000}%
\pgfsetstrokecolor{currentstroke}%
\pgfsetdash{}{0pt}%
\pgfpathmoveto{\pgfqpoint{2.139937in}{1.983506in}}%
\pgfpathcurveto{\pgfqpoint{2.150988in}{1.983506in}}{\pgfqpoint{2.161587in}{1.987896in}}{\pgfqpoint{2.169400in}{1.995710in}}%
\pgfpathcurveto{\pgfqpoint{2.177214in}{2.003524in}}{\pgfqpoint{2.181604in}{2.014123in}}{\pgfqpoint{2.181604in}{2.025173in}}%
\pgfpathcurveto{\pgfqpoint{2.181604in}{2.036223in}}{\pgfqpoint{2.177214in}{2.046822in}}{\pgfqpoint{2.169400in}{2.054635in}}%
\pgfpathcurveto{\pgfqpoint{2.161587in}{2.062449in}}{\pgfqpoint{2.150988in}{2.066839in}}{\pgfqpoint{2.139937in}{2.066839in}}%
\pgfpathcurveto{\pgfqpoint{2.128887in}{2.066839in}}{\pgfqpoint{2.118288in}{2.062449in}}{\pgfqpoint{2.110475in}{2.054635in}}%
\pgfpathcurveto{\pgfqpoint{2.102661in}{2.046822in}}{\pgfqpoint{2.098271in}{2.036223in}}{\pgfqpoint{2.098271in}{2.025173in}}%
\pgfpathcurveto{\pgfqpoint{2.098271in}{2.014123in}}{\pgfqpoint{2.102661in}{2.003524in}}{\pgfqpoint{2.110475in}{1.995710in}}%
\pgfpathcurveto{\pgfqpoint{2.118288in}{1.987896in}}{\pgfqpoint{2.128887in}{1.983506in}}{\pgfqpoint{2.139937in}{1.983506in}}%
\pgfpathclose%
\pgfusepath{stroke,fill}%
\end{pgfscope}%
\begin{pgfscope}%
\pgfpathrectangle{\pgfqpoint{0.375000in}{0.330000in}}{\pgfqpoint{2.325000in}{2.310000in}}%
\pgfusepath{clip}%
\pgfsetbuttcap%
\pgfsetroundjoin%
\definecolor{currentfill}{rgb}{0.000000,0.000000,0.000000}%
\pgfsetfillcolor{currentfill}%
\pgfsetlinewidth{1.003750pt}%
\definecolor{currentstroke}{rgb}{0.000000,0.000000,0.000000}%
\pgfsetstrokecolor{currentstroke}%
\pgfsetdash{}{0pt}%
\pgfpathmoveto{\pgfqpoint{2.139937in}{1.989574in}}%
\pgfpathcurveto{\pgfqpoint{2.150988in}{1.989574in}}{\pgfqpoint{2.161587in}{1.993964in}}{\pgfqpoint{2.169400in}{2.001778in}}%
\pgfpathcurveto{\pgfqpoint{2.177214in}{2.009592in}}{\pgfqpoint{2.181604in}{2.020191in}}{\pgfqpoint{2.181604in}{2.031241in}}%
\pgfpathcurveto{\pgfqpoint{2.181604in}{2.042291in}}{\pgfqpoint{2.177214in}{2.052890in}}{\pgfqpoint{2.169400in}{2.060704in}}%
\pgfpathcurveto{\pgfqpoint{2.161587in}{2.068517in}}{\pgfqpoint{2.150988in}{2.072907in}}{\pgfqpoint{2.139937in}{2.072907in}}%
\pgfpathcurveto{\pgfqpoint{2.128887in}{2.072907in}}{\pgfqpoint{2.118288in}{2.068517in}}{\pgfqpoint{2.110475in}{2.060704in}}%
\pgfpathcurveto{\pgfqpoint{2.102661in}{2.052890in}}{\pgfqpoint{2.098271in}{2.042291in}}{\pgfqpoint{2.098271in}{2.031241in}}%
\pgfpathcurveto{\pgfqpoint{2.098271in}{2.020191in}}{\pgfqpoint{2.102661in}{2.009592in}}{\pgfqpoint{2.110475in}{2.001778in}}%
\pgfpathcurveto{\pgfqpoint{2.118288in}{1.993964in}}{\pgfqpoint{2.128887in}{1.989574in}}{\pgfqpoint{2.139937in}{1.989574in}}%
\pgfpathclose%
\pgfusepath{stroke,fill}%
\end{pgfscope}%
\begin{pgfscope}%
\pgfpathrectangle{\pgfqpoint{0.375000in}{0.330000in}}{\pgfqpoint{2.325000in}{2.310000in}}%
\pgfusepath{clip}%
\pgfsetbuttcap%
\pgfsetroundjoin%
\definecolor{currentfill}{rgb}{0.000000,0.000000,0.000000}%
\pgfsetfillcolor{currentfill}%
\pgfsetlinewidth{1.003750pt}%
\definecolor{currentstroke}{rgb}{0.000000,0.000000,0.000000}%
\pgfsetstrokecolor{currentstroke}%
\pgfsetdash{}{0pt}%
\pgfpathmoveto{\pgfqpoint{2.139937in}{1.989574in}}%
\pgfpathcurveto{\pgfqpoint{2.150988in}{1.989574in}}{\pgfqpoint{2.161587in}{1.993964in}}{\pgfqpoint{2.169400in}{2.001778in}}%
\pgfpathcurveto{\pgfqpoint{2.177214in}{2.009592in}}{\pgfqpoint{2.181604in}{2.020191in}}{\pgfqpoint{2.181604in}{2.031241in}}%
\pgfpathcurveto{\pgfqpoint{2.181604in}{2.042291in}}{\pgfqpoint{2.177214in}{2.052890in}}{\pgfqpoint{2.169400in}{2.060704in}}%
\pgfpathcurveto{\pgfqpoint{2.161587in}{2.068517in}}{\pgfqpoint{2.150988in}{2.072907in}}{\pgfqpoint{2.139937in}{2.072907in}}%
\pgfpathcurveto{\pgfqpoint{2.128887in}{2.072907in}}{\pgfqpoint{2.118288in}{2.068517in}}{\pgfqpoint{2.110475in}{2.060704in}}%
\pgfpathcurveto{\pgfqpoint{2.102661in}{2.052890in}}{\pgfqpoint{2.098271in}{2.042291in}}{\pgfqpoint{2.098271in}{2.031241in}}%
\pgfpathcurveto{\pgfqpoint{2.098271in}{2.020191in}}{\pgfqpoint{2.102661in}{2.009592in}}{\pgfqpoint{2.110475in}{2.001778in}}%
\pgfpathcurveto{\pgfqpoint{2.118288in}{1.993964in}}{\pgfqpoint{2.128887in}{1.989574in}}{\pgfqpoint{2.139937in}{1.989574in}}%
\pgfpathclose%
\pgfusepath{stroke,fill}%
\end{pgfscope}%
\begin{pgfscope}%
\pgfpathrectangle{\pgfqpoint{0.375000in}{0.330000in}}{\pgfqpoint{2.325000in}{2.310000in}}%
\pgfusepath{clip}%
\pgfsetbuttcap%
\pgfsetroundjoin%
\definecolor{currentfill}{rgb}{0.000000,0.000000,0.000000}%
\pgfsetfillcolor{currentfill}%
\pgfsetlinewidth{1.003750pt}%
\definecolor{currentstroke}{rgb}{0.000000,0.000000,0.000000}%
\pgfsetstrokecolor{currentstroke}%
\pgfsetdash{}{0pt}%
\pgfpathmoveto{\pgfqpoint{2.139937in}{2.183754in}}%
\pgfpathcurveto{\pgfqpoint{2.150988in}{2.183754in}}{\pgfqpoint{2.161587in}{2.188144in}}{\pgfqpoint{2.169400in}{2.195957in}}%
\pgfpathcurveto{\pgfqpoint{2.177214in}{2.203771in}}{\pgfqpoint{2.181604in}{2.214370in}}{\pgfqpoint{2.181604in}{2.225420in}}%
\pgfpathcurveto{\pgfqpoint{2.181604in}{2.236470in}}{\pgfqpoint{2.177214in}{2.247069in}}{\pgfqpoint{2.169400in}{2.254883in}}%
\pgfpathcurveto{\pgfqpoint{2.161587in}{2.262697in}}{\pgfqpoint{2.150988in}{2.267087in}}{\pgfqpoint{2.139937in}{2.267087in}}%
\pgfpathcurveto{\pgfqpoint{2.128887in}{2.267087in}}{\pgfqpoint{2.118288in}{2.262697in}}{\pgfqpoint{2.110475in}{2.254883in}}%
\pgfpathcurveto{\pgfqpoint{2.102661in}{2.247069in}}{\pgfqpoint{2.098271in}{2.236470in}}{\pgfqpoint{2.098271in}{2.225420in}}%
\pgfpathcurveto{\pgfqpoint{2.098271in}{2.214370in}}{\pgfqpoint{2.102661in}{2.203771in}}{\pgfqpoint{2.110475in}{2.195957in}}%
\pgfpathcurveto{\pgfqpoint{2.118288in}{2.188144in}}{\pgfqpoint{2.128887in}{2.183754in}}{\pgfqpoint{2.139937in}{2.183754in}}%
\pgfpathclose%
\pgfusepath{stroke,fill}%
\end{pgfscope}%
\begin{pgfscope}%
\pgfpathrectangle{\pgfqpoint{0.375000in}{0.330000in}}{\pgfqpoint{2.325000in}{2.310000in}}%
\pgfusepath{clip}%
\pgfsetbuttcap%
\pgfsetroundjoin%
\definecolor{currentfill}{rgb}{0.000000,0.000000,0.000000}%
\pgfsetfillcolor{currentfill}%
\pgfsetlinewidth{1.003750pt}%
\definecolor{currentstroke}{rgb}{0.000000,0.000000,0.000000}%
\pgfsetstrokecolor{currentstroke}%
\pgfsetdash{}{0pt}%
\pgfpathmoveto{\pgfqpoint{2.139937in}{1.922825in}}%
\pgfpathcurveto{\pgfqpoint{2.150988in}{1.922825in}}{\pgfqpoint{2.161587in}{1.927215in}}{\pgfqpoint{2.169400in}{1.935029in}}%
\pgfpathcurveto{\pgfqpoint{2.177214in}{1.942842in}}{\pgfqpoint{2.181604in}{1.953442in}}{\pgfqpoint{2.181604in}{1.964492in}}%
\pgfpathcurveto{\pgfqpoint{2.181604in}{1.975542in}}{\pgfqpoint{2.177214in}{1.986141in}}{\pgfqpoint{2.169400in}{1.993954in}}%
\pgfpathcurveto{\pgfqpoint{2.161587in}{2.001768in}}{\pgfqpoint{2.150988in}{2.006158in}}{\pgfqpoint{2.139937in}{2.006158in}}%
\pgfpathcurveto{\pgfqpoint{2.128887in}{2.006158in}}{\pgfqpoint{2.118288in}{2.001768in}}{\pgfqpoint{2.110475in}{1.993954in}}%
\pgfpathcurveto{\pgfqpoint{2.102661in}{1.986141in}}{\pgfqpoint{2.098271in}{1.975542in}}{\pgfqpoint{2.098271in}{1.964492in}}%
\pgfpathcurveto{\pgfqpoint{2.098271in}{1.953442in}}{\pgfqpoint{2.102661in}{1.942842in}}{\pgfqpoint{2.110475in}{1.935029in}}%
\pgfpathcurveto{\pgfqpoint{2.118288in}{1.927215in}}{\pgfqpoint{2.128887in}{1.922825in}}{\pgfqpoint{2.139937in}{1.922825in}}%
\pgfpathclose%
\pgfusepath{stroke,fill}%
\end{pgfscope}%
\begin{pgfscope}%
\pgfpathrectangle{\pgfqpoint{0.375000in}{0.330000in}}{\pgfqpoint{2.325000in}{2.310000in}}%
\pgfusepath{clip}%
\pgfsetbuttcap%
\pgfsetroundjoin%
\definecolor{currentfill}{rgb}{0.000000,0.000000,0.000000}%
\pgfsetfillcolor{currentfill}%
\pgfsetlinewidth{1.003750pt}%
\definecolor{currentstroke}{rgb}{0.000000,0.000000,0.000000}%
\pgfsetstrokecolor{currentstroke}%
\pgfsetdash{}{0pt}%
\pgfpathmoveto{\pgfqpoint{2.139937in}{1.850008in}}%
\pgfpathcurveto{\pgfqpoint{2.150988in}{1.850008in}}{\pgfqpoint{2.161587in}{1.854398in}}{\pgfqpoint{2.169400in}{1.862212in}}%
\pgfpathcurveto{\pgfqpoint{2.177214in}{1.870025in}}{\pgfqpoint{2.181604in}{1.880624in}}{\pgfqpoint{2.181604in}{1.891674in}}%
\pgfpathcurveto{\pgfqpoint{2.181604in}{1.902724in}}{\pgfqpoint{2.177214in}{1.913324in}}{\pgfqpoint{2.169400in}{1.921137in}}%
\pgfpathcurveto{\pgfqpoint{2.161587in}{1.928951in}}{\pgfqpoint{2.150988in}{1.933341in}}{\pgfqpoint{2.139937in}{1.933341in}}%
\pgfpathcurveto{\pgfqpoint{2.128887in}{1.933341in}}{\pgfqpoint{2.118288in}{1.928951in}}{\pgfqpoint{2.110475in}{1.921137in}}%
\pgfpathcurveto{\pgfqpoint{2.102661in}{1.913324in}}{\pgfqpoint{2.098271in}{1.902724in}}{\pgfqpoint{2.098271in}{1.891674in}}%
\pgfpathcurveto{\pgfqpoint{2.098271in}{1.880624in}}{\pgfqpoint{2.102661in}{1.870025in}}{\pgfqpoint{2.110475in}{1.862212in}}%
\pgfpathcurveto{\pgfqpoint{2.118288in}{1.854398in}}{\pgfqpoint{2.128887in}{1.850008in}}{\pgfqpoint{2.139937in}{1.850008in}}%
\pgfpathclose%
\pgfusepath{stroke,fill}%
\end{pgfscope}%
\begin{pgfscope}%
\pgfpathrectangle{\pgfqpoint{0.375000in}{0.330000in}}{\pgfqpoint{2.325000in}{2.310000in}}%
\pgfusepath{clip}%
\pgfsetbuttcap%
\pgfsetroundjoin%
\definecolor{currentfill}{rgb}{0.000000,0.000000,0.000000}%
\pgfsetfillcolor{currentfill}%
\pgfsetlinewidth{1.003750pt}%
\definecolor{currentstroke}{rgb}{0.000000,0.000000,0.000000}%
\pgfsetstrokecolor{currentstroke}%
\pgfsetdash{}{0pt}%
\pgfpathmoveto{\pgfqpoint{2.139937in}{2.195890in}}%
\pgfpathcurveto{\pgfqpoint{2.150988in}{2.195890in}}{\pgfqpoint{2.161587in}{2.200280in}}{\pgfqpoint{2.169400in}{2.208094in}}%
\pgfpathcurveto{\pgfqpoint{2.177214in}{2.215907in}}{\pgfqpoint{2.181604in}{2.226506in}}{\pgfqpoint{2.181604in}{2.237556in}}%
\pgfpathcurveto{\pgfqpoint{2.181604in}{2.248607in}}{\pgfqpoint{2.177214in}{2.259206in}}{\pgfqpoint{2.169400in}{2.267019in}}%
\pgfpathcurveto{\pgfqpoint{2.161587in}{2.274833in}}{\pgfqpoint{2.150988in}{2.279223in}}{\pgfqpoint{2.139937in}{2.279223in}}%
\pgfpathcurveto{\pgfqpoint{2.128887in}{2.279223in}}{\pgfqpoint{2.118288in}{2.274833in}}{\pgfqpoint{2.110475in}{2.267019in}}%
\pgfpathcurveto{\pgfqpoint{2.102661in}{2.259206in}}{\pgfqpoint{2.098271in}{2.248607in}}{\pgfqpoint{2.098271in}{2.237556in}}%
\pgfpathcurveto{\pgfqpoint{2.098271in}{2.226506in}}{\pgfqpoint{2.102661in}{2.215907in}}{\pgfqpoint{2.110475in}{2.208094in}}%
\pgfpathcurveto{\pgfqpoint{2.118288in}{2.200280in}}{\pgfqpoint{2.128887in}{2.195890in}}{\pgfqpoint{2.139937in}{2.195890in}}%
\pgfpathclose%
\pgfusepath{stroke,fill}%
\end{pgfscope}%
\begin{pgfscope}%
\pgfpathrectangle{\pgfqpoint{0.375000in}{0.330000in}}{\pgfqpoint{2.325000in}{2.310000in}}%
\pgfusepath{clip}%
\pgfsetbuttcap%
\pgfsetroundjoin%
\definecolor{currentfill}{rgb}{0.000000,0.000000,0.000000}%
\pgfsetfillcolor{currentfill}%
\pgfsetlinewidth{1.003750pt}%
\definecolor{currentstroke}{rgb}{0.000000,0.000000,0.000000}%
\pgfsetstrokecolor{currentstroke}%
\pgfsetdash{}{0pt}%
\pgfpathmoveto{\pgfqpoint{2.139937in}{1.953165in}}%
\pgfpathcurveto{\pgfqpoint{2.150988in}{1.953165in}}{\pgfqpoint{2.161587in}{1.957556in}}{\pgfqpoint{2.169400in}{1.965369in}}%
\pgfpathcurveto{\pgfqpoint{2.177214in}{1.973183in}}{\pgfqpoint{2.181604in}{1.983782in}}{\pgfqpoint{2.181604in}{1.994832in}}%
\pgfpathcurveto{\pgfqpoint{2.181604in}{2.005882in}}{\pgfqpoint{2.177214in}{2.016481in}}{\pgfqpoint{2.169400in}{2.024295in}}%
\pgfpathcurveto{\pgfqpoint{2.161587in}{2.032109in}}{\pgfqpoint{2.150988in}{2.036499in}}{\pgfqpoint{2.139937in}{2.036499in}}%
\pgfpathcurveto{\pgfqpoint{2.128887in}{2.036499in}}{\pgfqpoint{2.118288in}{2.032109in}}{\pgfqpoint{2.110475in}{2.024295in}}%
\pgfpathcurveto{\pgfqpoint{2.102661in}{2.016481in}}{\pgfqpoint{2.098271in}{2.005882in}}{\pgfqpoint{2.098271in}{1.994832in}}%
\pgfpathcurveto{\pgfqpoint{2.098271in}{1.983782in}}{\pgfqpoint{2.102661in}{1.973183in}}{\pgfqpoint{2.110475in}{1.965369in}}%
\pgfpathcurveto{\pgfqpoint{2.118288in}{1.957556in}}{\pgfqpoint{2.128887in}{1.953165in}}{\pgfqpoint{2.139937in}{1.953165in}}%
\pgfpathclose%
\pgfusepath{stroke,fill}%
\end{pgfscope}%
\begin{pgfscope}%
\pgfpathrectangle{\pgfqpoint{0.375000in}{0.330000in}}{\pgfqpoint{2.325000in}{2.310000in}}%
\pgfusepath{clip}%
\pgfsetbuttcap%
\pgfsetroundjoin%
\definecolor{currentfill}{rgb}{0.000000,0.000000,0.000000}%
\pgfsetfillcolor{currentfill}%
\pgfsetlinewidth{1.003750pt}%
\definecolor{currentstroke}{rgb}{0.000000,0.000000,0.000000}%
\pgfsetstrokecolor{currentstroke}%
\pgfsetdash{}{0pt}%
\pgfpathmoveto{\pgfqpoint{2.139937in}{1.904621in}}%
\pgfpathcurveto{\pgfqpoint{2.150988in}{1.904621in}}{\pgfqpoint{2.161587in}{1.909011in}}{\pgfqpoint{2.169400in}{1.916825in}}%
\pgfpathcurveto{\pgfqpoint{2.177214in}{1.924638in}}{\pgfqpoint{2.181604in}{1.935237in}}{\pgfqpoint{2.181604in}{1.946287in}}%
\pgfpathcurveto{\pgfqpoint{2.181604in}{1.957337in}}{\pgfqpoint{2.177214in}{1.967936in}}{\pgfqpoint{2.169400in}{1.975750in}}%
\pgfpathcurveto{\pgfqpoint{2.161587in}{1.983564in}}{\pgfqpoint{2.150988in}{1.987954in}}{\pgfqpoint{2.139937in}{1.987954in}}%
\pgfpathcurveto{\pgfqpoint{2.128887in}{1.987954in}}{\pgfqpoint{2.118288in}{1.983564in}}{\pgfqpoint{2.110475in}{1.975750in}}%
\pgfpathcurveto{\pgfqpoint{2.102661in}{1.967936in}}{\pgfqpoint{2.098271in}{1.957337in}}{\pgfqpoint{2.098271in}{1.946287in}}%
\pgfpathcurveto{\pgfqpoint{2.098271in}{1.935237in}}{\pgfqpoint{2.102661in}{1.924638in}}{\pgfqpoint{2.110475in}{1.916825in}}%
\pgfpathcurveto{\pgfqpoint{2.118288in}{1.909011in}}{\pgfqpoint{2.128887in}{1.904621in}}{\pgfqpoint{2.139937in}{1.904621in}}%
\pgfpathclose%
\pgfusepath{stroke,fill}%
\end{pgfscope}%
\begin{pgfscope}%
\pgfpathrectangle{\pgfqpoint{0.375000in}{0.330000in}}{\pgfqpoint{2.325000in}{2.310000in}}%
\pgfusepath{clip}%
\pgfsetbuttcap%
\pgfsetroundjoin%
\definecolor{currentfill}{rgb}{0.000000,0.000000,0.000000}%
\pgfsetfillcolor{currentfill}%
\pgfsetlinewidth{1.003750pt}%
\definecolor{currentstroke}{rgb}{0.000000,0.000000,0.000000}%
\pgfsetstrokecolor{currentstroke}%
\pgfsetdash{}{0pt}%
\pgfpathmoveto{\pgfqpoint{2.139937in}{2.347592in}}%
\pgfpathcurveto{\pgfqpoint{2.150988in}{2.347592in}}{\pgfqpoint{2.161587in}{2.351983in}}{\pgfqpoint{2.169400in}{2.359796in}}%
\pgfpathcurveto{\pgfqpoint{2.177214in}{2.367610in}}{\pgfqpoint{2.181604in}{2.378209in}}{\pgfqpoint{2.181604in}{2.389259in}}%
\pgfpathcurveto{\pgfqpoint{2.181604in}{2.400309in}}{\pgfqpoint{2.177214in}{2.410908in}}{\pgfqpoint{2.169400in}{2.418722in}}%
\pgfpathcurveto{\pgfqpoint{2.161587in}{2.426535in}}{\pgfqpoint{2.150988in}{2.430926in}}{\pgfqpoint{2.139937in}{2.430926in}}%
\pgfpathcurveto{\pgfqpoint{2.128887in}{2.430926in}}{\pgfqpoint{2.118288in}{2.426535in}}{\pgfqpoint{2.110475in}{2.418722in}}%
\pgfpathcurveto{\pgfqpoint{2.102661in}{2.410908in}}{\pgfqpoint{2.098271in}{2.400309in}}{\pgfqpoint{2.098271in}{2.389259in}}%
\pgfpathcurveto{\pgfqpoint{2.098271in}{2.378209in}}{\pgfqpoint{2.102661in}{2.367610in}}{\pgfqpoint{2.110475in}{2.359796in}}%
\pgfpathcurveto{\pgfqpoint{2.118288in}{2.351983in}}{\pgfqpoint{2.128887in}{2.347592in}}{\pgfqpoint{2.139937in}{2.347592in}}%
\pgfpathclose%
\pgfusepath{stroke,fill}%
\end{pgfscope}%
\begin{pgfscope}%
\pgfpathrectangle{\pgfqpoint{0.375000in}{0.330000in}}{\pgfqpoint{2.325000in}{2.310000in}}%
\pgfusepath{clip}%
\pgfsetbuttcap%
\pgfsetroundjoin%
\definecolor{currentfill}{rgb}{0.000000,0.000000,0.000000}%
\pgfsetfillcolor{currentfill}%
\pgfsetlinewidth{1.003750pt}%
\definecolor{currentstroke}{rgb}{0.000000,0.000000,0.000000}%
\pgfsetstrokecolor{currentstroke}%
\pgfsetdash{}{0pt}%
\pgfpathmoveto{\pgfqpoint{2.139937in}{2.032051in}}%
\pgfpathcurveto{\pgfqpoint{2.150988in}{2.032051in}}{\pgfqpoint{2.161587in}{2.036441in}}{\pgfqpoint{2.169400in}{2.044255in}}%
\pgfpathcurveto{\pgfqpoint{2.177214in}{2.052068in}}{\pgfqpoint{2.181604in}{2.062667in}}{\pgfqpoint{2.181604in}{2.073718in}}%
\pgfpathcurveto{\pgfqpoint{2.181604in}{2.084768in}}{\pgfqpoint{2.177214in}{2.095367in}}{\pgfqpoint{2.169400in}{2.103180in}}%
\pgfpathcurveto{\pgfqpoint{2.161587in}{2.110994in}}{\pgfqpoint{2.150988in}{2.115384in}}{\pgfqpoint{2.139937in}{2.115384in}}%
\pgfpathcurveto{\pgfqpoint{2.128887in}{2.115384in}}{\pgfqpoint{2.118288in}{2.110994in}}{\pgfqpoint{2.110475in}{2.103180in}}%
\pgfpathcurveto{\pgfqpoint{2.102661in}{2.095367in}}{\pgfqpoint{2.098271in}{2.084768in}}{\pgfqpoint{2.098271in}{2.073718in}}%
\pgfpathcurveto{\pgfqpoint{2.098271in}{2.062667in}}{\pgfqpoint{2.102661in}{2.052068in}}{\pgfqpoint{2.110475in}{2.044255in}}%
\pgfpathcurveto{\pgfqpoint{2.118288in}{2.036441in}}{\pgfqpoint{2.128887in}{2.032051in}}{\pgfqpoint{2.139937in}{2.032051in}}%
\pgfpathclose%
\pgfusepath{stroke,fill}%
\end{pgfscope}%
\begin{pgfscope}%
\pgfpathrectangle{\pgfqpoint{0.375000in}{0.330000in}}{\pgfqpoint{2.325000in}{2.310000in}}%
\pgfusepath{clip}%
\pgfsetbuttcap%
\pgfsetroundjoin%
\definecolor{currentfill}{rgb}{0.000000,0.000000,0.000000}%
\pgfsetfillcolor{currentfill}%
\pgfsetlinewidth{1.003750pt}%
\definecolor{currentstroke}{rgb}{0.000000,0.000000,0.000000}%
\pgfsetstrokecolor{currentstroke}%
\pgfsetdash{}{0pt}%
\pgfpathmoveto{\pgfqpoint{2.139937in}{1.922825in}}%
\pgfpathcurveto{\pgfqpoint{2.150988in}{1.922825in}}{\pgfqpoint{2.161587in}{1.927215in}}{\pgfqpoint{2.169400in}{1.935029in}}%
\pgfpathcurveto{\pgfqpoint{2.177214in}{1.942842in}}{\pgfqpoint{2.181604in}{1.953442in}}{\pgfqpoint{2.181604in}{1.964492in}}%
\pgfpathcurveto{\pgfqpoint{2.181604in}{1.975542in}}{\pgfqpoint{2.177214in}{1.986141in}}{\pgfqpoint{2.169400in}{1.993954in}}%
\pgfpathcurveto{\pgfqpoint{2.161587in}{2.001768in}}{\pgfqpoint{2.150988in}{2.006158in}}{\pgfqpoint{2.139937in}{2.006158in}}%
\pgfpathcurveto{\pgfqpoint{2.128887in}{2.006158in}}{\pgfqpoint{2.118288in}{2.001768in}}{\pgfqpoint{2.110475in}{1.993954in}}%
\pgfpathcurveto{\pgfqpoint{2.102661in}{1.986141in}}{\pgfqpoint{2.098271in}{1.975542in}}{\pgfqpoint{2.098271in}{1.964492in}}%
\pgfpathcurveto{\pgfqpoint{2.098271in}{1.953442in}}{\pgfqpoint{2.102661in}{1.942842in}}{\pgfqpoint{2.110475in}{1.935029in}}%
\pgfpathcurveto{\pgfqpoint{2.118288in}{1.927215in}}{\pgfqpoint{2.128887in}{1.922825in}}{\pgfqpoint{2.139937in}{1.922825in}}%
\pgfpathclose%
\pgfusepath{stroke,fill}%
\end{pgfscope}%
\begin{pgfscope}%
\pgfpathrectangle{\pgfqpoint{0.375000in}{0.330000in}}{\pgfqpoint{2.325000in}{2.310000in}}%
\pgfusepath{clip}%
\pgfsetbuttcap%
\pgfsetroundjoin%
\definecolor{currentfill}{rgb}{0.000000,0.000000,0.000000}%
\pgfsetfillcolor{currentfill}%
\pgfsetlinewidth{1.003750pt}%
\definecolor{currentstroke}{rgb}{0.000000,0.000000,0.000000}%
\pgfsetstrokecolor{currentstroke}%
\pgfsetdash{}{0pt}%
\pgfpathmoveto{\pgfqpoint{2.139937in}{2.068460in}}%
\pgfpathcurveto{\pgfqpoint{2.150988in}{2.068460in}}{\pgfqpoint{2.161587in}{2.072850in}}{\pgfqpoint{2.169400in}{2.080663in}}%
\pgfpathcurveto{\pgfqpoint{2.177214in}{2.088477in}}{\pgfqpoint{2.181604in}{2.099076in}}{\pgfqpoint{2.181604in}{2.110126in}}%
\pgfpathcurveto{\pgfqpoint{2.181604in}{2.121176in}}{\pgfqpoint{2.177214in}{2.131775in}}{\pgfqpoint{2.169400in}{2.139589in}}%
\pgfpathcurveto{\pgfqpoint{2.161587in}{2.147403in}}{\pgfqpoint{2.150988in}{2.151793in}}{\pgfqpoint{2.139937in}{2.151793in}}%
\pgfpathcurveto{\pgfqpoint{2.128887in}{2.151793in}}{\pgfqpoint{2.118288in}{2.147403in}}{\pgfqpoint{2.110475in}{2.139589in}}%
\pgfpathcurveto{\pgfqpoint{2.102661in}{2.131775in}}{\pgfqpoint{2.098271in}{2.121176in}}{\pgfqpoint{2.098271in}{2.110126in}}%
\pgfpathcurveto{\pgfqpoint{2.098271in}{2.099076in}}{\pgfqpoint{2.102661in}{2.088477in}}{\pgfqpoint{2.110475in}{2.080663in}}%
\pgfpathcurveto{\pgfqpoint{2.118288in}{2.072850in}}{\pgfqpoint{2.128887in}{2.068460in}}{\pgfqpoint{2.139937in}{2.068460in}}%
\pgfpathclose%
\pgfusepath{stroke,fill}%
\end{pgfscope}%
\begin{pgfscope}%
\pgfpathrectangle{\pgfqpoint{0.375000in}{0.330000in}}{\pgfqpoint{2.325000in}{2.310000in}}%
\pgfusepath{clip}%
\pgfsetbuttcap%
\pgfsetroundjoin%
\definecolor{currentfill}{rgb}{0.000000,0.000000,0.000000}%
\pgfsetfillcolor{currentfill}%
\pgfsetlinewidth{1.003750pt}%
\definecolor{currentstroke}{rgb}{0.000000,0.000000,0.000000}%
\pgfsetstrokecolor{currentstroke}%
\pgfsetdash{}{0pt}%
\pgfpathmoveto{\pgfqpoint{2.139937in}{2.044187in}}%
\pgfpathcurveto{\pgfqpoint{2.150988in}{2.044187in}}{\pgfqpoint{2.161587in}{2.048577in}}{\pgfqpoint{2.169400in}{2.056391in}}%
\pgfpathcurveto{\pgfqpoint{2.177214in}{2.064205in}}{\pgfqpoint{2.181604in}{2.074804in}}{\pgfqpoint{2.181604in}{2.085854in}}%
\pgfpathcurveto{\pgfqpoint{2.181604in}{2.096904in}}{\pgfqpoint{2.177214in}{2.107503in}}{\pgfqpoint{2.169400in}{2.115317in}}%
\pgfpathcurveto{\pgfqpoint{2.161587in}{2.123130in}}{\pgfqpoint{2.150988in}{2.127520in}}{\pgfqpoint{2.139937in}{2.127520in}}%
\pgfpathcurveto{\pgfqpoint{2.128887in}{2.127520in}}{\pgfqpoint{2.118288in}{2.123130in}}{\pgfqpoint{2.110475in}{2.115317in}}%
\pgfpathcurveto{\pgfqpoint{2.102661in}{2.107503in}}{\pgfqpoint{2.098271in}{2.096904in}}{\pgfqpoint{2.098271in}{2.085854in}}%
\pgfpathcurveto{\pgfqpoint{2.098271in}{2.074804in}}{\pgfqpoint{2.102661in}{2.064205in}}{\pgfqpoint{2.110475in}{2.056391in}}%
\pgfpathcurveto{\pgfqpoint{2.118288in}{2.048577in}}{\pgfqpoint{2.128887in}{2.044187in}}{\pgfqpoint{2.139937in}{2.044187in}}%
\pgfpathclose%
\pgfusepath{stroke,fill}%
\end{pgfscope}%
\begin{pgfscope}%
\pgfpathrectangle{\pgfqpoint{0.375000in}{0.330000in}}{\pgfqpoint{2.325000in}{2.310000in}}%
\pgfusepath{clip}%
\pgfsetbuttcap%
\pgfsetroundjoin%
\definecolor{currentfill}{rgb}{0.000000,0.000000,0.000000}%
\pgfsetfillcolor{currentfill}%
\pgfsetlinewidth{1.003750pt}%
\definecolor{currentstroke}{rgb}{0.000000,0.000000,0.000000}%
\pgfsetstrokecolor{currentstroke}%
\pgfsetdash{}{0pt}%
\pgfpathmoveto{\pgfqpoint{2.139937in}{2.426478in}}%
\pgfpathcurveto{\pgfqpoint{2.150988in}{2.426478in}}{\pgfqpoint{2.161587in}{2.430868in}}{\pgfqpoint{2.169400in}{2.438682in}}%
\pgfpathcurveto{\pgfqpoint{2.177214in}{2.446495in}}{\pgfqpoint{2.181604in}{2.457094in}}{\pgfqpoint{2.181604in}{2.468144in}}%
\pgfpathcurveto{\pgfqpoint{2.181604in}{2.479195in}}{\pgfqpoint{2.177214in}{2.489794in}}{\pgfqpoint{2.169400in}{2.497607in}}%
\pgfpathcurveto{\pgfqpoint{2.161587in}{2.505421in}}{\pgfqpoint{2.150988in}{2.509811in}}{\pgfqpoint{2.139937in}{2.509811in}}%
\pgfpathcurveto{\pgfqpoint{2.128887in}{2.509811in}}{\pgfqpoint{2.118288in}{2.505421in}}{\pgfqpoint{2.110475in}{2.497607in}}%
\pgfpathcurveto{\pgfqpoint{2.102661in}{2.489794in}}{\pgfqpoint{2.098271in}{2.479195in}}{\pgfqpoint{2.098271in}{2.468144in}}%
\pgfpathcurveto{\pgfqpoint{2.098271in}{2.457094in}}{\pgfqpoint{2.102661in}{2.446495in}}{\pgfqpoint{2.110475in}{2.438682in}}%
\pgfpathcurveto{\pgfqpoint{2.118288in}{2.430868in}}{\pgfqpoint{2.128887in}{2.426478in}}{\pgfqpoint{2.139937in}{2.426478in}}%
\pgfpathclose%
\pgfusepath{stroke,fill}%
\end{pgfscope}%
\begin{pgfscope}%
\pgfpathrectangle{\pgfqpoint{0.375000in}{0.330000in}}{\pgfqpoint{2.325000in}{2.310000in}}%
\pgfusepath{clip}%
\pgfsetbuttcap%
\pgfsetroundjoin%
\definecolor{currentfill}{rgb}{0.000000,0.000000,0.000000}%
\pgfsetfillcolor{currentfill}%
\pgfsetlinewidth{1.003750pt}%
\definecolor{currentstroke}{rgb}{0.000000,0.000000,0.000000}%
\pgfsetstrokecolor{currentstroke}%
\pgfsetdash{}{0pt}%
\pgfpathmoveto{\pgfqpoint{2.139937in}{2.189822in}}%
\pgfpathcurveto{\pgfqpoint{2.150988in}{2.189822in}}{\pgfqpoint{2.161587in}{2.194212in}}{\pgfqpoint{2.169400in}{2.202026in}}%
\pgfpathcurveto{\pgfqpoint{2.177214in}{2.209839in}}{\pgfqpoint{2.181604in}{2.220438in}}{\pgfqpoint{2.181604in}{2.231488in}}%
\pgfpathcurveto{\pgfqpoint{2.181604in}{2.242538in}}{\pgfqpoint{2.177214in}{2.253137in}}{\pgfqpoint{2.169400in}{2.260951in}}%
\pgfpathcurveto{\pgfqpoint{2.161587in}{2.268765in}}{\pgfqpoint{2.150988in}{2.273155in}}{\pgfqpoint{2.139937in}{2.273155in}}%
\pgfpathcurveto{\pgfqpoint{2.128887in}{2.273155in}}{\pgfqpoint{2.118288in}{2.268765in}}{\pgfqpoint{2.110475in}{2.260951in}}%
\pgfpathcurveto{\pgfqpoint{2.102661in}{2.253137in}}{\pgfqpoint{2.098271in}{2.242538in}}{\pgfqpoint{2.098271in}{2.231488in}}%
\pgfpathcurveto{\pgfqpoint{2.098271in}{2.220438in}}{\pgfqpoint{2.102661in}{2.209839in}}{\pgfqpoint{2.110475in}{2.202026in}}%
\pgfpathcurveto{\pgfqpoint{2.118288in}{2.194212in}}{\pgfqpoint{2.128887in}{2.189822in}}{\pgfqpoint{2.139937in}{2.189822in}}%
\pgfpathclose%
\pgfusepath{stroke,fill}%
\end{pgfscope}%
\begin{pgfscope}%
\pgfpathrectangle{\pgfqpoint{0.375000in}{0.330000in}}{\pgfqpoint{2.325000in}{2.310000in}}%
\pgfusepath{clip}%
\pgfsetbuttcap%
\pgfsetroundjoin%
\definecolor{currentfill}{rgb}{0.000000,0.000000,0.000000}%
\pgfsetfillcolor{currentfill}%
\pgfsetlinewidth{1.003750pt}%
\definecolor{currentstroke}{rgb}{0.000000,0.000000,0.000000}%
\pgfsetstrokecolor{currentstroke}%
\pgfsetdash{}{0pt}%
\pgfpathmoveto{\pgfqpoint{2.139937in}{2.019915in}}%
\pgfpathcurveto{\pgfqpoint{2.150988in}{2.019915in}}{\pgfqpoint{2.161587in}{2.024305in}}{\pgfqpoint{2.169400in}{2.032119in}}%
\pgfpathcurveto{\pgfqpoint{2.177214in}{2.039932in}}{\pgfqpoint{2.181604in}{2.050531in}}{\pgfqpoint{2.181604in}{2.061581in}}%
\pgfpathcurveto{\pgfqpoint{2.181604in}{2.072631in}}{\pgfqpoint{2.177214in}{2.083230in}}{\pgfqpoint{2.169400in}{2.091044in}}%
\pgfpathcurveto{\pgfqpoint{2.161587in}{2.098858in}}{\pgfqpoint{2.150988in}{2.103248in}}{\pgfqpoint{2.139937in}{2.103248in}}%
\pgfpathcurveto{\pgfqpoint{2.128887in}{2.103248in}}{\pgfqpoint{2.118288in}{2.098858in}}{\pgfqpoint{2.110475in}{2.091044in}}%
\pgfpathcurveto{\pgfqpoint{2.102661in}{2.083230in}}{\pgfqpoint{2.098271in}{2.072631in}}{\pgfqpoint{2.098271in}{2.061581in}}%
\pgfpathcurveto{\pgfqpoint{2.098271in}{2.050531in}}{\pgfqpoint{2.102661in}{2.039932in}}{\pgfqpoint{2.110475in}{2.032119in}}%
\pgfpathcurveto{\pgfqpoint{2.118288in}{2.024305in}}{\pgfqpoint{2.128887in}{2.019915in}}{\pgfqpoint{2.139937in}{2.019915in}}%
\pgfpathclose%
\pgfusepath{stroke,fill}%
\end{pgfscope}%
\begin{pgfscope}%
\pgfpathrectangle{\pgfqpoint{0.375000in}{0.330000in}}{\pgfqpoint{2.325000in}{2.310000in}}%
\pgfusepath{clip}%
\pgfsetbuttcap%
\pgfsetroundjoin%
\definecolor{currentfill}{rgb}{0.000000,0.000000,0.000000}%
\pgfsetfillcolor{currentfill}%
\pgfsetlinewidth{1.003750pt}%
\definecolor{currentstroke}{rgb}{0.000000,0.000000,0.000000}%
\pgfsetstrokecolor{currentstroke}%
\pgfsetdash{}{0pt}%
\pgfpathmoveto{\pgfqpoint{2.139937in}{1.922825in}}%
\pgfpathcurveto{\pgfqpoint{2.150988in}{1.922825in}}{\pgfqpoint{2.161587in}{1.927215in}}{\pgfqpoint{2.169400in}{1.935029in}}%
\pgfpathcurveto{\pgfqpoint{2.177214in}{1.942842in}}{\pgfqpoint{2.181604in}{1.953442in}}{\pgfqpoint{2.181604in}{1.964492in}}%
\pgfpathcurveto{\pgfqpoint{2.181604in}{1.975542in}}{\pgfqpoint{2.177214in}{1.986141in}}{\pgfqpoint{2.169400in}{1.993954in}}%
\pgfpathcurveto{\pgfqpoint{2.161587in}{2.001768in}}{\pgfqpoint{2.150988in}{2.006158in}}{\pgfqpoint{2.139937in}{2.006158in}}%
\pgfpathcurveto{\pgfqpoint{2.128887in}{2.006158in}}{\pgfqpoint{2.118288in}{2.001768in}}{\pgfqpoint{2.110475in}{1.993954in}}%
\pgfpathcurveto{\pgfqpoint{2.102661in}{1.986141in}}{\pgfqpoint{2.098271in}{1.975542in}}{\pgfqpoint{2.098271in}{1.964492in}}%
\pgfpathcurveto{\pgfqpoint{2.098271in}{1.953442in}}{\pgfqpoint{2.102661in}{1.942842in}}{\pgfqpoint{2.110475in}{1.935029in}}%
\pgfpathcurveto{\pgfqpoint{2.118288in}{1.927215in}}{\pgfqpoint{2.128887in}{1.922825in}}{\pgfqpoint{2.139937in}{1.922825in}}%
\pgfpathclose%
\pgfusepath{stroke,fill}%
\end{pgfscope}%
\begin{pgfscope}%
\pgfpathrectangle{\pgfqpoint{0.375000in}{0.330000in}}{\pgfqpoint{2.325000in}{2.310000in}}%
\pgfusepath{clip}%
\pgfsetbuttcap%
\pgfsetroundjoin%
\definecolor{currentfill}{rgb}{0.000000,0.000000,0.000000}%
\pgfsetfillcolor{currentfill}%
\pgfsetlinewidth{1.003750pt}%
\definecolor{currentstroke}{rgb}{0.000000,0.000000,0.000000}%
\pgfsetstrokecolor{currentstroke}%
\pgfsetdash{}{0pt}%
\pgfpathmoveto{\pgfqpoint{2.139937in}{2.147345in}}%
\pgfpathcurveto{\pgfqpoint{2.150988in}{2.147345in}}{\pgfqpoint{2.161587in}{2.151735in}}{\pgfqpoint{2.169400in}{2.159549in}}%
\pgfpathcurveto{\pgfqpoint{2.177214in}{2.167362in}}{\pgfqpoint{2.181604in}{2.177961in}}{\pgfqpoint{2.181604in}{2.189012in}}%
\pgfpathcurveto{\pgfqpoint{2.181604in}{2.200062in}}{\pgfqpoint{2.177214in}{2.210661in}}{\pgfqpoint{2.169400in}{2.218474in}}%
\pgfpathcurveto{\pgfqpoint{2.161587in}{2.226288in}}{\pgfqpoint{2.150988in}{2.230678in}}{\pgfqpoint{2.139937in}{2.230678in}}%
\pgfpathcurveto{\pgfqpoint{2.128887in}{2.230678in}}{\pgfqpoint{2.118288in}{2.226288in}}{\pgfqpoint{2.110475in}{2.218474in}}%
\pgfpathcurveto{\pgfqpoint{2.102661in}{2.210661in}}{\pgfqpoint{2.098271in}{2.200062in}}{\pgfqpoint{2.098271in}{2.189012in}}%
\pgfpathcurveto{\pgfqpoint{2.098271in}{2.177961in}}{\pgfqpoint{2.102661in}{2.167362in}}{\pgfqpoint{2.110475in}{2.159549in}}%
\pgfpathcurveto{\pgfqpoint{2.118288in}{2.151735in}}{\pgfqpoint{2.128887in}{2.147345in}}{\pgfqpoint{2.139937in}{2.147345in}}%
\pgfpathclose%
\pgfusepath{stroke,fill}%
\end{pgfscope}%
\begin{pgfscope}%
\pgfpathrectangle{\pgfqpoint{0.375000in}{0.330000in}}{\pgfqpoint{2.325000in}{2.310000in}}%
\pgfusepath{clip}%
\pgfsetbuttcap%
\pgfsetroundjoin%
\definecolor{currentfill}{rgb}{0.000000,0.000000,0.000000}%
\pgfsetfillcolor{currentfill}%
\pgfsetlinewidth{1.003750pt}%
\definecolor{currentstroke}{rgb}{0.000000,0.000000,0.000000}%
\pgfsetstrokecolor{currentstroke}%
\pgfsetdash{}{0pt}%
\pgfpathmoveto{\pgfqpoint{2.139937in}{2.475023in}}%
\pgfpathcurveto{\pgfqpoint{2.150988in}{2.475023in}}{\pgfqpoint{2.161587in}{2.479413in}}{\pgfqpoint{2.169400in}{2.487226in}}%
\pgfpathcurveto{\pgfqpoint{2.177214in}{2.495040in}}{\pgfqpoint{2.181604in}{2.505639in}}{\pgfqpoint{2.181604in}{2.516689in}}%
\pgfpathcurveto{\pgfqpoint{2.181604in}{2.527739in}}{\pgfqpoint{2.177214in}{2.538338in}}{\pgfqpoint{2.169400in}{2.546152in}}%
\pgfpathcurveto{\pgfqpoint{2.161587in}{2.553966in}}{\pgfqpoint{2.150988in}{2.558356in}}{\pgfqpoint{2.139937in}{2.558356in}}%
\pgfpathcurveto{\pgfqpoint{2.128887in}{2.558356in}}{\pgfqpoint{2.118288in}{2.553966in}}{\pgfqpoint{2.110475in}{2.546152in}}%
\pgfpathcurveto{\pgfqpoint{2.102661in}{2.538338in}}{\pgfqpoint{2.098271in}{2.527739in}}{\pgfqpoint{2.098271in}{2.516689in}}%
\pgfpathcurveto{\pgfqpoint{2.098271in}{2.505639in}}{\pgfqpoint{2.102661in}{2.495040in}}{\pgfqpoint{2.110475in}{2.487226in}}%
\pgfpathcurveto{\pgfqpoint{2.118288in}{2.479413in}}{\pgfqpoint{2.128887in}{2.475023in}}{\pgfqpoint{2.139937in}{2.475023in}}%
\pgfpathclose%
\pgfusepath{stroke,fill}%
\end{pgfscope}%
\begin{pgfscope}%
\pgfpathrectangle{\pgfqpoint{0.375000in}{0.330000in}}{\pgfqpoint{2.325000in}{2.310000in}}%
\pgfusepath{clip}%
\pgfsetbuttcap%
\pgfsetroundjoin%
\definecolor{currentfill}{rgb}{0.000000,0.000000,0.000000}%
\pgfsetfillcolor{currentfill}%
\pgfsetlinewidth{1.003750pt}%
\definecolor{currentstroke}{rgb}{0.000000,0.000000,0.000000}%
\pgfsetstrokecolor{currentstroke}%
\pgfsetdash{}{0pt}%
\pgfpathmoveto{\pgfqpoint{2.139937in}{2.001710in}}%
\pgfpathcurveto{\pgfqpoint{2.150988in}{2.001710in}}{\pgfqpoint{2.161587in}{2.006101in}}{\pgfqpoint{2.169400in}{2.013914in}}%
\pgfpathcurveto{\pgfqpoint{2.177214in}{2.021728in}}{\pgfqpoint{2.181604in}{2.032327in}}{\pgfqpoint{2.181604in}{2.043377in}}%
\pgfpathcurveto{\pgfqpoint{2.181604in}{2.054427in}}{\pgfqpoint{2.177214in}{2.065026in}}{\pgfqpoint{2.169400in}{2.072840in}}%
\pgfpathcurveto{\pgfqpoint{2.161587in}{2.080653in}}{\pgfqpoint{2.150988in}{2.085044in}}{\pgfqpoint{2.139937in}{2.085044in}}%
\pgfpathcurveto{\pgfqpoint{2.128887in}{2.085044in}}{\pgfqpoint{2.118288in}{2.080653in}}{\pgfqpoint{2.110475in}{2.072840in}}%
\pgfpathcurveto{\pgfqpoint{2.102661in}{2.065026in}}{\pgfqpoint{2.098271in}{2.054427in}}{\pgfqpoint{2.098271in}{2.043377in}}%
\pgfpathcurveto{\pgfqpoint{2.098271in}{2.032327in}}{\pgfqpoint{2.102661in}{2.021728in}}{\pgfqpoint{2.110475in}{2.013914in}}%
\pgfpathcurveto{\pgfqpoint{2.118288in}{2.006101in}}{\pgfqpoint{2.128887in}{2.001710in}}{\pgfqpoint{2.139937in}{2.001710in}}%
\pgfpathclose%
\pgfusepath{stroke,fill}%
\end{pgfscope}%
\begin{pgfscope}%
\pgfpathrectangle{\pgfqpoint{0.375000in}{0.330000in}}{\pgfqpoint{2.325000in}{2.310000in}}%
\pgfusepath{clip}%
\pgfsetbuttcap%
\pgfsetroundjoin%
\definecolor{currentfill}{rgb}{0.000000,0.000000,0.000000}%
\pgfsetfillcolor{currentfill}%
\pgfsetlinewidth{1.003750pt}%
\definecolor{currentstroke}{rgb}{0.000000,0.000000,0.000000}%
\pgfsetstrokecolor{currentstroke}%
\pgfsetdash{}{0pt}%
\pgfpathmoveto{\pgfqpoint{2.139937in}{1.874280in}}%
\pgfpathcurveto{\pgfqpoint{2.150988in}{1.874280in}}{\pgfqpoint{2.161587in}{1.878670in}}{\pgfqpoint{2.169400in}{1.886484in}}%
\pgfpathcurveto{\pgfqpoint{2.177214in}{1.894298in}}{\pgfqpoint{2.181604in}{1.904897in}}{\pgfqpoint{2.181604in}{1.915947in}}%
\pgfpathcurveto{\pgfqpoint{2.181604in}{1.926997in}}{\pgfqpoint{2.177214in}{1.937596in}}{\pgfqpoint{2.169400in}{1.945410in}}%
\pgfpathcurveto{\pgfqpoint{2.161587in}{1.953223in}}{\pgfqpoint{2.150988in}{1.957613in}}{\pgfqpoint{2.139937in}{1.957613in}}%
\pgfpathcurveto{\pgfqpoint{2.128887in}{1.957613in}}{\pgfqpoint{2.118288in}{1.953223in}}{\pgfqpoint{2.110475in}{1.945410in}}%
\pgfpathcurveto{\pgfqpoint{2.102661in}{1.937596in}}{\pgfqpoint{2.098271in}{1.926997in}}{\pgfqpoint{2.098271in}{1.915947in}}%
\pgfpathcurveto{\pgfqpoint{2.098271in}{1.904897in}}{\pgfqpoint{2.102661in}{1.894298in}}{\pgfqpoint{2.110475in}{1.886484in}}%
\pgfpathcurveto{\pgfqpoint{2.118288in}{1.878670in}}{\pgfqpoint{2.128887in}{1.874280in}}{\pgfqpoint{2.139937in}{1.874280in}}%
\pgfpathclose%
\pgfusepath{stroke,fill}%
\end{pgfscope}%
\begin{pgfscope}%
\pgfpathrectangle{\pgfqpoint{0.375000in}{0.330000in}}{\pgfqpoint{2.325000in}{2.310000in}}%
\pgfusepath{clip}%
\pgfsetbuttcap%
\pgfsetroundjoin%
\definecolor{currentfill}{rgb}{0.000000,0.000000,0.000000}%
\pgfsetfillcolor{currentfill}%
\pgfsetlinewidth{1.003750pt}%
\definecolor{currentstroke}{rgb}{0.000000,0.000000,0.000000}%
\pgfsetstrokecolor{currentstroke}%
\pgfsetdash{}{0pt}%
\pgfpathmoveto{\pgfqpoint{2.139937in}{2.019915in}}%
\pgfpathcurveto{\pgfqpoint{2.150988in}{2.019915in}}{\pgfqpoint{2.161587in}{2.024305in}}{\pgfqpoint{2.169400in}{2.032119in}}%
\pgfpathcurveto{\pgfqpoint{2.177214in}{2.039932in}}{\pgfqpoint{2.181604in}{2.050531in}}{\pgfqpoint{2.181604in}{2.061581in}}%
\pgfpathcurveto{\pgfqpoint{2.181604in}{2.072631in}}{\pgfqpoint{2.177214in}{2.083230in}}{\pgfqpoint{2.169400in}{2.091044in}}%
\pgfpathcurveto{\pgfqpoint{2.161587in}{2.098858in}}{\pgfqpoint{2.150988in}{2.103248in}}{\pgfqpoint{2.139937in}{2.103248in}}%
\pgfpathcurveto{\pgfqpoint{2.128887in}{2.103248in}}{\pgfqpoint{2.118288in}{2.098858in}}{\pgfqpoint{2.110475in}{2.091044in}}%
\pgfpathcurveto{\pgfqpoint{2.102661in}{2.083230in}}{\pgfqpoint{2.098271in}{2.072631in}}{\pgfqpoint{2.098271in}{2.061581in}}%
\pgfpathcurveto{\pgfqpoint{2.098271in}{2.050531in}}{\pgfqpoint{2.102661in}{2.039932in}}{\pgfqpoint{2.110475in}{2.032119in}}%
\pgfpathcurveto{\pgfqpoint{2.118288in}{2.024305in}}{\pgfqpoint{2.128887in}{2.019915in}}{\pgfqpoint{2.139937in}{2.019915in}}%
\pgfpathclose%
\pgfusepath{stroke,fill}%
\end{pgfscope}%
\begin{pgfscope}%
\pgfpathrectangle{\pgfqpoint{0.375000in}{0.330000in}}{\pgfqpoint{2.325000in}{2.310000in}}%
\pgfusepath{clip}%
\pgfsetbuttcap%
\pgfsetroundjoin%
\definecolor{currentfill}{rgb}{0.000000,0.000000,0.000000}%
\pgfsetfillcolor{currentfill}%
\pgfsetlinewidth{1.003750pt}%
\definecolor{currentstroke}{rgb}{0.000000,0.000000,0.000000}%
\pgfsetstrokecolor{currentstroke}%
\pgfsetdash{}{0pt}%
\pgfpathmoveto{\pgfqpoint{2.139937in}{1.971370in}}%
\pgfpathcurveto{\pgfqpoint{2.150988in}{1.971370in}}{\pgfqpoint{2.161587in}{1.975760in}}{\pgfqpoint{2.169400in}{1.983574in}}%
\pgfpathcurveto{\pgfqpoint{2.177214in}{1.991387in}}{\pgfqpoint{2.181604in}{2.001986in}}{\pgfqpoint{2.181604in}{2.013036in}}%
\pgfpathcurveto{\pgfqpoint{2.181604in}{2.024087in}}{\pgfqpoint{2.177214in}{2.034686in}}{\pgfqpoint{2.169400in}{2.042499in}}%
\pgfpathcurveto{\pgfqpoint{2.161587in}{2.050313in}}{\pgfqpoint{2.150988in}{2.054703in}}{\pgfqpoint{2.139937in}{2.054703in}}%
\pgfpathcurveto{\pgfqpoint{2.128887in}{2.054703in}}{\pgfqpoint{2.118288in}{2.050313in}}{\pgfqpoint{2.110475in}{2.042499in}}%
\pgfpathcurveto{\pgfqpoint{2.102661in}{2.034686in}}{\pgfqpoint{2.098271in}{2.024087in}}{\pgfqpoint{2.098271in}{2.013036in}}%
\pgfpathcurveto{\pgfqpoint{2.098271in}{2.001986in}}{\pgfqpoint{2.102661in}{1.991387in}}{\pgfqpoint{2.110475in}{1.983574in}}%
\pgfpathcurveto{\pgfqpoint{2.118288in}{1.975760in}}{\pgfqpoint{2.128887in}{1.971370in}}{\pgfqpoint{2.139937in}{1.971370in}}%
\pgfpathclose%
\pgfusepath{stroke,fill}%
\end{pgfscope}%
\begin{pgfscope}%
\pgfpathrectangle{\pgfqpoint{0.375000in}{0.330000in}}{\pgfqpoint{2.325000in}{2.310000in}}%
\pgfusepath{clip}%
\pgfsetbuttcap%
\pgfsetroundjoin%
\definecolor{currentfill}{rgb}{0.000000,0.000000,0.000000}%
\pgfsetfillcolor{currentfill}%
\pgfsetlinewidth{1.003750pt}%
\definecolor{currentstroke}{rgb}{0.000000,0.000000,0.000000}%
\pgfsetstrokecolor{currentstroke}%
\pgfsetdash{}{0pt}%
\pgfpathmoveto{\pgfqpoint{2.139937in}{2.232298in}}%
\pgfpathcurveto{\pgfqpoint{2.150988in}{2.232298in}}{\pgfqpoint{2.161587in}{2.236689in}}{\pgfqpoint{2.169400in}{2.244502in}}%
\pgfpathcurveto{\pgfqpoint{2.177214in}{2.252316in}}{\pgfqpoint{2.181604in}{2.262915in}}{\pgfqpoint{2.181604in}{2.273965in}}%
\pgfpathcurveto{\pgfqpoint{2.181604in}{2.285015in}}{\pgfqpoint{2.177214in}{2.295614in}}{\pgfqpoint{2.169400in}{2.303428in}}%
\pgfpathcurveto{\pgfqpoint{2.161587in}{2.311241in}}{\pgfqpoint{2.150988in}{2.315632in}}{\pgfqpoint{2.139937in}{2.315632in}}%
\pgfpathcurveto{\pgfqpoint{2.128887in}{2.315632in}}{\pgfqpoint{2.118288in}{2.311241in}}{\pgfqpoint{2.110475in}{2.303428in}}%
\pgfpathcurveto{\pgfqpoint{2.102661in}{2.295614in}}{\pgfqpoint{2.098271in}{2.285015in}}{\pgfqpoint{2.098271in}{2.273965in}}%
\pgfpathcurveto{\pgfqpoint{2.098271in}{2.262915in}}{\pgfqpoint{2.102661in}{2.252316in}}{\pgfqpoint{2.110475in}{2.244502in}}%
\pgfpathcurveto{\pgfqpoint{2.118288in}{2.236689in}}{\pgfqpoint{2.128887in}{2.232298in}}{\pgfqpoint{2.139937in}{2.232298in}}%
\pgfpathclose%
\pgfusepath{stroke,fill}%
\end{pgfscope}%
\begin{pgfscope}%
\pgfpathrectangle{\pgfqpoint{0.375000in}{0.330000in}}{\pgfqpoint{2.325000in}{2.310000in}}%
\pgfusepath{clip}%
\pgfsetbuttcap%
\pgfsetroundjoin%
\definecolor{currentfill}{rgb}{0.000000,0.000000,0.000000}%
\pgfsetfillcolor{currentfill}%
\pgfsetlinewidth{1.003750pt}%
\definecolor{currentstroke}{rgb}{0.000000,0.000000,0.000000}%
\pgfsetstrokecolor{currentstroke}%
\pgfsetdash{}{0pt}%
\pgfpathmoveto{\pgfqpoint{2.139937in}{2.013847in}}%
\pgfpathcurveto{\pgfqpoint{2.150988in}{2.013847in}}{\pgfqpoint{2.161587in}{2.018237in}}{\pgfqpoint{2.169400in}{2.026050in}}%
\pgfpathcurveto{\pgfqpoint{2.177214in}{2.033864in}}{\pgfqpoint{2.181604in}{2.044463in}}{\pgfqpoint{2.181604in}{2.055513in}}%
\pgfpathcurveto{\pgfqpoint{2.181604in}{2.066563in}}{\pgfqpoint{2.177214in}{2.077162in}}{\pgfqpoint{2.169400in}{2.084976in}}%
\pgfpathcurveto{\pgfqpoint{2.161587in}{2.092790in}}{\pgfqpoint{2.150988in}{2.097180in}}{\pgfqpoint{2.139937in}{2.097180in}}%
\pgfpathcurveto{\pgfqpoint{2.128887in}{2.097180in}}{\pgfqpoint{2.118288in}{2.092790in}}{\pgfqpoint{2.110475in}{2.084976in}}%
\pgfpathcurveto{\pgfqpoint{2.102661in}{2.077162in}}{\pgfqpoint{2.098271in}{2.066563in}}{\pgfqpoint{2.098271in}{2.055513in}}%
\pgfpathcurveto{\pgfqpoint{2.098271in}{2.044463in}}{\pgfqpoint{2.102661in}{2.033864in}}{\pgfqpoint{2.110475in}{2.026050in}}%
\pgfpathcurveto{\pgfqpoint{2.118288in}{2.018237in}}{\pgfqpoint{2.128887in}{2.013847in}}{\pgfqpoint{2.139937in}{2.013847in}}%
\pgfpathclose%
\pgfusepath{stroke,fill}%
\end{pgfscope}%
\begin{pgfscope}%
\pgfpathrectangle{\pgfqpoint{0.375000in}{0.330000in}}{\pgfqpoint{2.325000in}{2.310000in}}%
\pgfusepath{clip}%
\pgfsetbuttcap%
\pgfsetroundjoin%
\definecolor{currentfill}{rgb}{0.000000,0.000000,0.000000}%
\pgfsetfillcolor{currentfill}%
\pgfsetlinewidth{1.003750pt}%
\definecolor{currentstroke}{rgb}{0.000000,0.000000,0.000000}%
\pgfsetstrokecolor{currentstroke}%
\pgfsetdash{}{0pt}%
\pgfpathmoveto{\pgfqpoint{2.139937in}{2.038119in}}%
\pgfpathcurveto{\pgfqpoint{2.150988in}{2.038119in}}{\pgfqpoint{2.161587in}{2.042509in}}{\pgfqpoint{2.169400in}{2.050323in}}%
\pgfpathcurveto{\pgfqpoint{2.177214in}{2.058136in}}{\pgfqpoint{2.181604in}{2.068736in}}{\pgfqpoint{2.181604in}{2.079786in}}%
\pgfpathcurveto{\pgfqpoint{2.181604in}{2.090836in}}{\pgfqpoint{2.177214in}{2.101435in}}{\pgfqpoint{2.169400in}{2.109248in}}%
\pgfpathcurveto{\pgfqpoint{2.161587in}{2.117062in}}{\pgfqpoint{2.150988in}{2.121452in}}{\pgfqpoint{2.139937in}{2.121452in}}%
\pgfpathcurveto{\pgfqpoint{2.128887in}{2.121452in}}{\pgfqpoint{2.118288in}{2.117062in}}{\pgfqpoint{2.110475in}{2.109248in}}%
\pgfpathcurveto{\pgfqpoint{2.102661in}{2.101435in}}{\pgfqpoint{2.098271in}{2.090836in}}{\pgfqpoint{2.098271in}{2.079786in}}%
\pgfpathcurveto{\pgfqpoint{2.098271in}{2.068736in}}{\pgfqpoint{2.102661in}{2.058136in}}{\pgfqpoint{2.110475in}{2.050323in}}%
\pgfpathcurveto{\pgfqpoint{2.118288in}{2.042509in}}{\pgfqpoint{2.128887in}{2.038119in}}{\pgfqpoint{2.139937in}{2.038119in}}%
\pgfpathclose%
\pgfusepath{stroke,fill}%
\end{pgfscope}%
\begin{pgfscope}%
\pgfpathrectangle{\pgfqpoint{0.375000in}{0.330000in}}{\pgfqpoint{2.325000in}{2.310000in}}%
\pgfusepath{clip}%
\pgfsetbuttcap%
\pgfsetroundjoin%
\definecolor{currentfill}{rgb}{0.000000,0.000000,0.000000}%
\pgfsetfillcolor{currentfill}%
\pgfsetlinewidth{1.003750pt}%
\definecolor{currentstroke}{rgb}{0.000000,0.000000,0.000000}%
\pgfsetstrokecolor{currentstroke}%
\pgfsetdash{}{0pt}%
\pgfpathmoveto{\pgfqpoint{2.139937in}{2.262639in}}%
\pgfpathcurveto{\pgfqpoint{2.150988in}{2.262639in}}{\pgfqpoint{2.161587in}{2.267029in}}{\pgfqpoint{2.169400in}{2.274843in}}%
\pgfpathcurveto{\pgfqpoint{2.177214in}{2.282656in}}{\pgfqpoint{2.181604in}{2.293255in}}{\pgfqpoint{2.181604in}{2.304306in}}%
\pgfpathcurveto{\pgfqpoint{2.181604in}{2.315356in}}{\pgfqpoint{2.177214in}{2.325955in}}{\pgfqpoint{2.169400in}{2.333768in}}%
\pgfpathcurveto{\pgfqpoint{2.161587in}{2.341582in}}{\pgfqpoint{2.150988in}{2.345972in}}{\pgfqpoint{2.139937in}{2.345972in}}%
\pgfpathcurveto{\pgfqpoint{2.128887in}{2.345972in}}{\pgfqpoint{2.118288in}{2.341582in}}{\pgfqpoint{2.110475in}{2.333768in}}%
\pgfpathcurveto{\pgfqpoint{2.102661in}{2.325955in}}{\pgfqpoint{2.098271in}{2.315356in}}{\pgfqpoint{2.098271in}{2.304306in}}%
\pgfpathcurveto{\pgfqpoint{2.098271in}{2.293255in}}{\pgfqpoint{2.102661in}{2.282656in}}{\pgfqpoint{2.110475in}{2.274843in}}%
\pgfpathcurveto{\pgfqpoint{2.118288in}{2.267029in}}{\pgfqpoint{2.128887in}{2.262639in}}{\pgfqpoint{2.139937in}{2.262639in}}%
\pgfpathclose%
\pgfusepath{stroke,fill}%
\end{pgfscope}%
\begin{pgfscope}%
\pgfpathrectangle{\pgfqpoint{0.375000in}{0.330000in}}{\pgfqpoint{2.325000in}{2.310000in}}%
\pgfusepath{clip}%
\pgfsetbuttcap%
\pgfsetroundjoin%
\definecolor{currentfill}{rgb}{0.000000,0.000000,0.000000}%
\pgfsetfillcolor{currentfill}%
\pgfsetlinewidth{1.003750pt}%
\definecolor{currentstroke}{rgb}{0.000000,0.000000,0.000000}%
\pgfsetstrokecolor{currentstroke}%
\pgfsetdash{}{0pt}%
\pgfpathmoveto{\pgfqpoint{2.139937in}{2.056323in}}%
\pgfpathcurveto{\pgfqpoint{2.150988in}{2.056323in}}{\pgfqpoint{2.161587in}{2.060714in}}{\pgfqpoint{2.169400in}{2.068527in}}%
\pgfpathcurveto{\pgfqpoint{2.177214in}{2.076341in}}{\pgfqpoint{2.181604in}{2.086940in}}{\pgfqpoint{2.181604in}{2.097990in}}%
\pgfpathcurveto{\pgfqpoint{2.181604in}{2.109040in}}{\pgfqpoint{2.177214in}{2.119639in}}{\pgfqpoint{2.169400in}{2.127453in}}%
\pgfpathcurveto{\pgfqpoint{2.161587in}{2.135266in}}{\pgfqpoint{2.150988in}{2.139657in}}{\pgfqpoint{2.139937in}{2.139657in}}%
\pgfpathcurveto{\pgfqpoint{2.128887in}{2.139657in}}{\pgfqpoint{2.118288in}{2.135266in}}{\pgfqpoint{2.110475in}{2.127453in}}%
\pgfpathcurveto{\pgfqpoint{2.102661in}{2.119639in}}{\pgfqpoint{2.098271in}{2.109040in}}{\pgfqpoint{2.098271in}{2.097990in}}%
\pgfpathcurveto{\pgfqpoint{2.098271in}{2.086940in}}{\pgfqpoint{2.102661in}{2.076341in}}{\pgfqpoint{2.110475in}{2.068527in}}%
\pgfpathcurveto{\pgfqpoint{2.118288in}{2.060714in}}{\pgfqpoint{2.128887in}{2.056323in}}{\pgfqpoint{2.139937in}{2.056323in}}%
\pgfpathclose%
\pgfusepath{stroke,fill}%
\end{pgfscope}%
\begin{pgfscope}%
\pgfpathrectangle{\pgfqpoint{0.375000in}{0.330000in}}{\pgfqpoint{2.325000in}{2.310000in}}%
\pgfusepath{clip}%
\pgfsetbuttcap%
\pgfsetroundjoin%
\definecolor{currentfill}{rgb}{0.000000,0.000000,0.000000}%
\pgfsetfillcolor{currentfill}%
\pgfsetlinewidth{1.003750pt}%
\definecolor{currentstroke}{rgb}{0.000000,0.000000,0.000000}%
\pgfsetstrokecolor{currentstroke}%
\pgfsetdash{}{0pt}%
\pgfpathmoveto{\pgfqpoint{2.139937in}{2.450750in}}%
\pgfpathcurveto{\pgfqpoint{2.150988in}{2.450750in}}{\pgfqpoint{2.161587in}{2.455140in}}{\pgfqpoint{2.169400in}{2.462954in}}%
\pgfpathcurveto{\pgfqpoint{2.177214in}{2.470768in}}{\pgfqpoint{2.181604in}{2.481367in}}{\pgfqpoint{2.181604in}{2.492417in}}%
\pgfpathcurveto{\pgfqpoint{2.181604in}{2.503467in}}{\pgfqpoint{2.177214in}{2.514066in}}{\pgfqpoint{2.169400in}{2.521880in}}%
\pgfpathcurveto{\pgfqpoint{2.161587in}{2.529693in}}{\pgfqpoint{2.150988in}{2.534083in}}{\pgfqpoint{2.139937in}{2.534083in}}%
\pgfpathcurveto{\pgfqpoint{2.128887in}{2.534083in}}{\pgfqpoint{2.118288in}{2.529693in}}{\pgfqpoint{2.110475in}{2.521880in}}%
\pgfpathcurveto{\pgfqpoint{2.102661in}{2.514066in}}{\pgfqpoint{2.098271in}{2.503467in}}{\pgfqpoint{2.098271in}{2.492417in}}%
\pgfpathcurveto{\pgfqpoint{2.098271in}{2.481367in}}{\pgfqpoint{2.102661in}{2.470768in}}{\pgfqpoint{2.110475in}{2.462954in}}%
\pgfpathcurveto{\pgfqpoint{2.118288in}{2.455140in}}{\pgfqpoint{2.128887in}{2.450750in}}{\pgfqpoint{2.139937in}{2.450750in}}%
\pgfpathclose%
\pgfusepath{stroke,fill}%
\end{pgfscope}%
\begin{pgfscope}%
\pgfpathrectangle{\pgfqpoint{0.375000in}{0.330000in}}{\pgfqpoint{2.325000in}{2.310000in}}%
\pgfusepath{clip}%
\pgfsetbuttcap%
\pgfsetroundjoin%
\definecolor{currentfill}{rgb}{0.000000,0.000000,0.000000}%
\pgfsetfillcolor{currentfill}%
\pgfsetlinewidth{1.003750pt}%
\definecolor{currentstroke}{rgb}{0.000000,0.000000,0.000000}%
\pgfsetstrokecolor{currentstroke}%
\pgfsetdash{}{0pt}%
\pgfpathmoveto{\pgfqpoint{2.139937in}{1.959234in}}%
\pgfpathcurveto{\pgfqpoint{2.150988in}{1.959234in}}{\pgfqpoint{2.161587in}{1.963624in}}{\pgfqpoint{2.169400in}{1.971437in}}%
\pgfpathcurveto{\pgfqpoint{2.177214in}{1.979251in}}{\pgfqpoint{2.181604in}{1.989850in}}{\pgfqpoint{2.181604in}{2.000900in}}%
\pgfpathcurveto{\pgfqpoint{2.181604in}{2.011950in}}{\pgfqpoint{2.177214in}{2.022549in}}{\pgfqpoint{2.169400in}{2.030363in}}%
\pgfpathcurveto{\pgfqpoint{2.161587in}{2.038177in}}{\pgfqpoint{2.150988in}{2.042567in}}{\pgfqpoint{2.139937in}{2.042567in}}%
\pgfpathcurveto{\pgfqpoint{2.128887in}{2.042567in}}{\pgfqpoint{2.118288in}{2.038177in}}{\pgfqpoint{2.110475in}{2.030363in}}%
\pgfpathcurveto{\pgfqpoint{2.102661in}{2.022549in}}{\pgfqpoint{2.098271in}{2.011950in}}{\pgfqpoint{2.098271in}{2.000900in}}%
\pgfpathcurveto{\pgfqpoint{2.098271in}{1.989850in}}{\pgfqpoint{2.102661in}{1.979251in}}{\pgfqpoint{2.110475in}{1.971437in}}%
\pgfpathcurveto{\pgfqpoint{2.118288in}{1.963624in}}{\pgfqpoint{2.128887in}{1.959234in}}{\pgfqpoint{2.139937in}{1.959234in}}%
\pgfpathclose%
\pgfusepath{stroke,fill}%
\end{pgfscope}%
\begin{pgfscope}%
\pgfpathrectangle{\pgfqpoint{0.375000in}{0.330000in}}{\pgfqpoint{2.325000in}{2.310000in}}%
\pgfusepath{clip}%
\pgfsetbuttcap%
\pgfsetroundjoin%
\definecolor{currentfill}{rgb}{0.000000,0.000000,0.000000}%
\pgfsetfillcolor{currentfill}%
\pgfsetlinewidth{1.003750pt}%
\definecolor{currentstroke}{rgb}{0.000000,0.000000,0.000000}%
\pgfsetstrokecolor{currentstroke}%
\pgfsetdash{}{0pt}%
\pgfpathmoveto{\pgfqpoint{2.139937in}{1.947097in}}%
\pgfpathcurveto{\pgfqpoint{2.150988in}{1.947097in}}{\pgfqpoint{2.161587in}{1.951488in}}{\pgfqpoint{2.169400in}{1.959301in}}%
\pgfpathcurveto{\pgfqpoint{2.177214in}{1.967115in}}{\pgfqpoint{2.181604in}{1.977714in}}{\pgfqpoint{2.181604in}{1.988764in}}%
\pgfpathcurveto{\pgfqpoint{2.181604in}{1.999814in}}{\pgfqpoint{2.177214in}{2.010413in}}{\pgfqpoint{2.169400in}{2.018227in}}%
\pgfpathcurveto{\pgfqpoint{2.161587in}{2.026040in}}{\pgfqpoint{2.150988in}{2.030431in}}{\pgfqpoint{2.139937in}{2.030431in}}%
\pgfpathcurveto{\pgfqpoint{2.128887in}{2.030431in}}{\pgfqpoint{2.118288in}{2.026040in}}{\pgfqpoint{2.110475in}{2.018227in}}%
\pgfpathcurveto{\pgfqpoint{2.102661in}{2.010413in}}{\pgfqpoint{2.098271in}{1.999814in}}{\pgfqpoint{2.098271in}{1.988764in}}%
\pgfpathcurveto{\pgfqpoint{2.098271in}{1.977714in}}{\pgfqpoint{2.102661in}{1.967115in}}{\pgfqpoint{2.110475in}{1.959301in}}%
\pgfpathcurveto{\pgfqpoint{2.118288in}{1.951488in}}{\pgfqpoint{2.128887in}{1.947097in}}{\pgfqpoint{2.139937in}{1.947097in}}%
\pgfpathclose%
\pgfusepath{stroke,fill}%
\end{pgfscope}%
\begin{pgfscope}%
\pgfpathrectangle{\pgfqpoint{0.375000in}{0.330000in}}{\pgfqpoint{2.325000in}{2.310000in}}%
\pgfusepath{clip}%
\pgfsetbuttcap%
\pgfsetroundjoin%
\definecolor{currentfill}{rgb}{0.000000,0.000000,0.000000}%
\pgfsetfillcolor{currentfill}%
\pgfsetlinewidth{1.003750pt}%
\definecolor{currentstroke}{rgb}{0.000000,0.000000,0.000000}%
\pgfsetstrokecolor{currentstroke}%
\pgfsetdash{}{0pt}%
\pgfpathmoveto{\pgfqpoint{2.139937in}{2.098800in}}%
\pgfpathcurveto{\pgfqpoint{2.150988in}{2.098800in}}{\pgfqpoint{2.161587in}{2.103190in}}{\pgfqpoint{2.169400in}{2.111004in}}%
\pgfpathcurveto{\pgfqpoint{2.177214in}{2.118818in}}{\pgfqpoint{2.181604in}{2.129417in}}{\pgfqpoint{2.181604in}{2.140467in}}%
\pgfpathcurveto{\pgfqpoint{2.181604in}{2.151517in}}{\pgfqpoint{2.177214in}{2.162116in}}{\pgfqpoint{2.169400in}{2.169929in}}%
\pgfpathcurveto{\pgfqpoint{2.161587in}{2.177743in}}{\pgfqpoint{2.150988in}{2.182133in}}{\pgfqpoint{2.139937in}{2.182133in}}%
\pgfpathcurveto{\pgfqpoint{2.128887in}{2.182133in}}{\pgfqpoint{2.118288in}{2.177743in}}{\pgfqpoint{2.110475in}{2.169929in}}%
\pgfpathcurveto{\pgfqpoint{2.102661in}{2.162116in}}{\pgfqpoint{2.098271in}{2.151517in}}{\pgfqpoint{2.098271in}{2.140467in}}%
\pgfpathcurveto{\pgfqpoint{2.098271in}{2.129417in}}{\pgfqpoint{2.102661in}{2.118818in}}{\pgfqpoint{2.110475in}{2.111004in}}%
\pgfpathcurveto{\pgfqpoint{2.118288in}{2.103190in}}{\pgfqpoint{2.128887in}{2.098800in}}{\pgfqpoint{2.139937in}{2.098800in}}%
\pgfpathclose%
\pgfusepath{stroke,fill}%
\end{pgfscope}%
\begin{pgfscope}%
\pgfpathrectangle{\pgfqpoint{0.375000in}{0.330000in}}{\pgfqpoint{2.325000in}{2.310000in}}%
\pgfusepath{clip}%
\pgfsetbuttcap%
\pgfsetroundjoin%
\definecolor{currentfill}{rgb}{0.000000,0.000000,0.000000}%
\pgfsetfillcolor{currentfill}%
\pgfsetlinewidth{1.003750pt}%
\definecolor{currentstroke}{rgb}{0.000000,0.000000,0.000000}%
\pgfsetstrokecolor{currentstroke}%
\pgfsetdash{}{0pt}%
\pgfpathmoveto{\pgfqpoint{2.139937in}{2.038119in}}%
\pgfpathcurveto{\pgfqpoint{2.150988in}{2.038119in}}{\pgfqpoint{2.161587in}{2.042509in}}{\pgfqpoint{2.169400in}{2.050323in}}%
\pgfpathcurveto{\pgfqpoint{2.177214in}{2.058136in}}{\pgfqpoint{2.181604in}{2.068736in}}{\pgfqpoint{2.181604in}{2.079786in}}%
\pgfpathcurveto{\pgfqpoint{2.181604in}{2.090836in}}{\pgfqpoint{2.177214in}{2.101435in}}{\pgfqpoint{2.169400in}{2.109248in}}%
\pgfpathcurveto{\pgfqpoint{2.161587in}{2.117062in}}{\pgfqpoint{2.150988in}{2.121452in}}{\pgfqpoint{2.139937in}{2.121452in}}%
\pgfpathcurveto{\pgfqpoint{2.128887in}{2.121452in}}{\pgfqpoint{2.118288in}{2.117062in}}{\pgfqpoint{2.110475in}{2.109248in}}%
\pgfpathcurveto{\pgfqpoint{2.102661in}{2.101435in}}{\pgfqpoint{2.098271in}{2.090836in}}{\pgfqpoint{2.098271in}{2.079786in}}%
\pgfpathcurveto{\pgfqpoint{2.098271in}{2.068736in}}{\pgfqpoint{2.102661in}{2.058136in}}{\pgfqpoint{2.110475in}{2.050323in}}%
\pgfpathcurveto{\pgfqpoint{2.118288in}{2.042509in}}{\pgfqpoint{2.128887in}{2.038119in}}{\pgfqpoint{2.139937in}{2.038119in}}%
\pgfpathclose%
\pgfusepath{stroke,fill}%
\end{pgfscope}%
\begin{pgfscope}%
\pgfpathrectangle{\pgfqpoint{0.375000in}{0.330000in}}{\pgfqpoint{2.325000in}{2.310000in}}%
\pgfusepath{clip}%
\pgfsetbuttcap%
\pgfsetroundjoin%
\definecolor{currentfill}{rgb}{0.000000,0.000000,0.000000}%
\pgfsetfillcolor{currentfill}%
\pgfsetlinewidth{1.003750pt}%
\definecolor{currentstroke}{rgb}{0.000000,0.000000,0.000000}%
\pgfsetstrokecolor{currentstroke}%
\pgfsetdash{}{0pt}%
\pgfpathmoveto{\pgfqpoint{2.139937in}{2.390069in}}%
\pgfpathcurveto{\pgfqpoint{2.150988in}{2.390069in}}{\pgfqpoint{2.161587in}{2.394459in}}{\pgfqpoint{2.169400in}{2.402273in}}%
\pgfpathcurveto{\pgfqpoint{2.177214in}{2.410087in}}{\pgfqpoint{2.181604in}{2.420686in}}{\pgfqpoint{2.181604in}{2.431736in}}%
\pgfpathcurveto{\pgfqpoint{2.181604in}{2.442786in}}{\pgfqpoint{2.177214in}{2.453385in}}{\pgfqpoint{2.169400in}{2.461199in}}%
\pgfpathcurveto{\pgfqpoint{2.161587in}{2.469012in}}{\pgfqpoint{2.150988in}{2.473402in}}{\pgfqpoint{2.139937in}{2.473402in}}%
\pgfpathcurveto{\pgfqpoint{2.128887in}{2.473402in}}{\pgfqpoint{2.118288in}{2.469012in}}{\pgfqpoint{2.110475in}{2.461199in}}%
\pgfpathcurveto{\pgfqpoint{2.102661in}{2.453385in}}{\pgfqpoint{2.098271in}{2.442786in}}{\pgfqpoint{2.098271in}{2.431736in}}%
\pgfpathcurveto{\pgfqpoint{2.098271in}{2.420686in}}{\pgfqpoint{2.102661in}{2.410087in}}{\pgfqpoint{2.110475in}{2.402273in}}%
\pgfpathcurveto{\pgfqpoint{2.118288in}{2.394459in}}{\pgfqpoint{2.128887in}{2.390069in}}{\pgfqpoint{2.139937in}{2.390069in}}%
\pgfpathclose%
\pgfusepath{stroke,fill}%
\end{pgfscope}%
\begin{pgfscope}%
\pgfpathrectangle{\pgfqpoint{0.375000in}{0.330000in}}{\pgfqpoint{2.325000in}{2.310000in}}%
\pgfusepath{clip}%
\pgfsetbuttcap%
\pgfsetroundjoin%
\definecolor{currentfill}{rgb}{0.000000,0.000000,0.000000}%
\pgfsetfillcolor{currentfill}%
\pgfsetlinewidth{1.003750pt}%
\definecolor{currentstroke}{rgb}{0.000000,0.000000,0.000000}%
\pgfsetstrokecolor{currentstroke}%
\pgfsetdash{}{0pt}%
\pgfpathmoveto{\pgfqpoint{2.139937in}{1.959234in}}%
\pgfpathcurveto{\pgfqpoint{2.150988in}{1.959234in}}{\pgfqpoint{2.161587in}{1.963624in}}{\pgfqpoint{2.169400in}{1.971437in}}%
\pgfpathcurveto{\pgfqpoint{2.177214in}{1.979251in}}{\pgfqpoint{2.181604in}{1.989850in}}{\pgfqpoint{2.181604in}{2.000900in}}%
\pgfpathcurveto{\pgfqpoint{2.181604in}{2.011950in}}{\pgfqpoint{2.177214in}{2.022549in}}{\pgfqpoint{2.169400in}{2.030363in}}%
\pgfpathcurveto{\pgfqpoint{2.161587in}{2.038177in}}{\pgfqpoint{2.150988in}{2.042567in}}{\pgfqpoint{2.139937in}{2.042567in}}%
\pgfpathcurveto{\pgfqpoint{2.128887in}{2.042567in}}{\pgfqpoint{2.118288in}{2.038177in}}{\pgfqpoint{2.110475in}{2.030363in}}%
\pgfpathcurveto{\pgfqpoint{2.102661in}{2.022549in}}{\pgfqpoint{2.098271in}{2.011950in}}{\pgfqpoint{2.098271in}{2.000900in}}%
\pgfpathcurveto{\pgfqpoint{2.098271in}{1.989850in}}{\pgfqpoint{2.102661in}{1.979251in}}{\pgfqpoint{2.110475in}{1.971437in}}%
\pgfpathcurveto{\pgfqpoint{2.118288in}{1.963624in}}{\pgfqpoint{2.128887in}{1.959234in}}{\pgfqpoint{2.139937in}{1.959234in}}%
\pgfpathclose%
\pgfusepath{stroke,fill}%
\end{pgfscope}%
\begin{pgfscope}%
\pgfpathrectangle{\pgfqpoint{0.375000in}{0.330000in}}{\pgfqpoint{2.325000in}{2.310000in}}%
\pgfusepath{clip}%
\pgfsetbuttcap%
\pgfsetroundjoin%
\definecolor{currentfill}{rgb}{0.000000,0.000000,0.000000}%
\pgfsetfillcolor{currentfill}%
\pgfsetlinewidth{1.003750pt}%
\definecolor{currentstroke}{rgb}{0.000000,0.000000,0.000000}%
\pgfsetstrokecolor{currentstroke}%
\pgfsetdash{}{0pt}%
\pgfpathmoveto{\pgfqpoint{2.139937in}{1.898553in}}%
\pgfpathcurveto{\pgfqpoint{2.150988in}{1.898553in}}{\pgfqpoint{2.161587in}{1.902943in}}{\pgfqpoint{2.169400in}{1.910756in}}%
\pgfpathcurveto{\pgfqpoint{2.177214in}{1.918570in}}{\pgfqpoint{2.181604in}{1.929169in}}{\pgfqpoint{2.181604in}{1.940219in}}%
\pgfpathcurveto{\pgfqpoint{2.181604in}{1.951269in}}{\pgfqpoint{2.177214in}{1.961868in}}{\pgfqpoint{2.169400in}{1.969682in}}%
\pgfpathcurveto{\pgfqpoint{2.161587in}{1.977496in}}{\pgfqpoint{2.150988in}{1.981886in}}{\pgfqpoint{2.139937in}{1.981886in}}%
\pgfpathcurveto{\pgfqpoint{2.128887in}{1.981886in}}{\pgfqpoint{2.118288in}{1.977496in}}{\pgfqpoint{2.110475in}{1.969682in}}%
\pgfpathcurveto{\pgfqpoint{2.102661in}{1.961868in}}{\pgfqpoint{2.098271in}{1.951269in}}{\pgfqpoint{2.098271in}{1.940219in}}%
\pgfpathcurveto{\pgfqpoint{2.098271in}{1.929169in}}{\pgfqpoint{2.102661in}{1.918570in}}{\pgfqpoint{2.110475in}{1.910756in}}%
\pgfpathcurveto{\pgfqpoint{2.118288in}{1.902943in}}{\pgfqpoint{2.128887in}{1.898553in}}{\pgfqpoint{2.139937in}{1.898553in}}%
\pgfpathclose%
\pgfusepath{stroke,fill}%
\end{pgfscope}%
\begin{pgfscope}%
\pgfpathrectangle{\pgfqpoint{0.375000in}{0.330000in}}{\pgfqpoint{2.325000in}{2.310000in}}%
\pgfusepath{clip}%
\pgfsetbuttcap%
\pgfsetroundjoin%
\definecolor{currentfill}{rgb}{0.000000,0.000000,0.000000}%
\pgfsetfillcolor{currentfill}%
\pgfsetlinewidth{1.003750pt}%
\definecolor{currentstroke}{rgb}{0.000000,0.000000,0.000000}%
\pgfsetstrokecolor{currentstroke}%
\pgfsetdash{}{0pt}%
\pgfpathmoveto{\pgfqpoint{2.139937in}{2.208026in}}%
\pgfpathcurveto{\pgfqpoint{2.150988in}{2.208026in}}{\pgfqpoint{2.161587in}{2.212416in}}{\pgfqpoint{2.169400in}{2.220230in}}%
\pgfpathcurveto{\pgfqpoint{2.177214in}{2.228043in}}{\pgfqpoint{2.181604in}{2.238642in}}{\pgfqpoint{2.181604in}{2.249693in}}%
\pgfpathcurveto{\pgfqpoint{2.181604in}{2.260743in}}{\pgfqpoint{2.177214in}{2.271342in}}{\pgfqpoint{2.169400in}{2.279155in}}%
\pgfpathcurveto{\pgfqpoint{2.161587in}{2.286969in}}{\pgfqpoint{2.150988in}{2.291359in}}{\pgfqpoint{2.139937in}{2.291359in}}%
\pgfpathcurveto{\pgfqpoint{2.128887in}{2.291359in}}{\pgfqpoint{2.118288in}{2.286969in}}{\pgfqpoint{2.110475in}{2.279155in}}%
\pgfpathcurveto{\pgfqpoint{2.102661in}{2.271342in}}{\pgfqpoint{2.098271in}{2.260743in}}{\pgfqpoint{2.098271in}{2.249693in}}%
\pgfpathcurveto{\pgfqpoint{2.098271in}{2.238642in}}{\pgfqpoint{2.102661in}{2.228043in}}{\pgfqpoint{2.110475in}{2.220230in}}%
\pgfpathcurveto{\pgfqpoint{2.118288in}{2.212416in}}{\pgfqpoint{2.128887in}{2.208026in}}{\pgfqpoint{2.139937in}{2.208026in}}%
\pgfpathclose%
\pgfusepath{stroke,fill}%
\end{pgfscope}%
\begin{pgfscope}%
\pgfpathrectangle{\pgfqpoint{0.375000in}{0.330000in}}{\pgfqpoint{2.325000in}{2.310000in}}%
\pgfusepath{clip}%
\pgfsetbuttcap%
\pgfsetroundjoin%
\definecolor{currentfill}{rgb}{0.000000,0.000000,0.000000}%
\pgfsetfillcolor{currentfill}%
\pgfsetlinewidth{1.003750pt}%
\definecolor{currentstroke}{rgb}{0.000000,0.000000,0.000000}%
\pgfsetstrokecolor{currentstroke}%
\pgfsetdash{}{0pt}%
\pgfpathmoveto{\pgfqpoint{2.139937in}{1.904621in}}%
\pgfpathcurveto{\pgfqpoint{2.150988in}{1.904621in}}{\pgfqpoint{2.161587in}{1.909011in}}{\pgfqpoint{2.169400in}{1.916825in}}%
\pgfpathcurveto{\pgfqpoint{2.177214in}{1.924638in}}{\pgfqpoint{2.181604in}{1.935237in}}{\pgfqpoint{2.181604in}{1.946287in}}%
\pgfpathcurveto{\pgfqpoint{2.181604in}{1.957337in}}{\pgfqpoint{2.177214in}{1.967936in}}{\pgfqpoint{2.169400in}{1.975750in}}%
\pgfpathcurveto{\pgfqpoint{2.161587in}{1.983564in}}{\pgfqpoint{2.150988in}{1.987954in}}{\pgfqpoint{2.139937in}{1.987954in}}%
\pgfpathcurveto{\pgfqpoint{2.128887in}{1.987954in}}{\pgfqpoint{2.118288in}{1.983564in}}{\pgfqpoint{2.110475in}{1.975750in}}%
\pgfpathcurveto{\pgfqpoint{2.102661in}{1.967936in}}{\pgfqpoint{2.098271in}{1.957337in}}{\pgfqpoint{2.098271in}{1.946287in}}%
\pgfpathcurveto{\pgfqpoint{2.098271in}{1.935237in}}{\pgfqpoint{2.102661in}{1.924638in}}{\pgfqpoint{2.110475in}{1.916825in}}%
\pgfpathcurveto{\pgfqpoint{2.118288in}{1.909011in}}{\pgfqpoint{2.128887in}{1.904621in}}{\pgfqpoint{2.139937in}{1.904621in}}%
\pgfpathclose%
\pgfusepath{stroke,fill}%
\end{pgfscope}%
\begin{pgfscope}%
\pgfpathrectangle{\pgfqpoint{0.375000in}{0.330000in}}{\pgfqpoint{2.325000in}{2.310000in}}%
\pgfusepath{clip}%
\pgfsetbuttcap%
\pgfsetroundjoin%
\definecolor{currentfill}{rgb}{0.000000,0.000000,0.000000}%
\pgfsetfillcolor{currentfill}%
\pgfsetlinewidth{1.003750pt}%
\definecolor{currentstroke}{rgb}{0.000000,0.000000,0.000000}%
\pgfsetstrokecolor{currentstroke}%
\pgfsetdash{}{0pt}%
\pgfpathmoveto{\pgfqpoint{2.139937in}{2.384001in}}%
\pgfpathcurveto{\pgfqpoint{2.150988in}{2.384001in}}{\pgfqpoint{2.161587in}{2.388391in}}{\pgfqpoint{2.169400in}{2.396205in}}%
\pgfpathcurveto{\pgfqpoint{2.177214in}{2.404019in}}{\pgfqpoint{2.181604in}{2.414618in}}{\pgfqpoint{2.181604in}{2.425668in}}%
\pgfpathcurveto{\pgfqpoint{2.181604in}{2.436718in}}{\pgfqpoint{2.177214in}{2.447317in}}{\pgfqpoint{2.169400in}{2.455130in}}%
\pgfpathcurveto{\pgfqpoint{2.161587in}{2.462944in}}{\pgfqpoint{2.150988in}{2.467334in}}{\pgfqpoint{2.139937in}{2.467334in}}%
\pgfpathcurveto{\pgfqpoint{2.128887in}{2.467334in}}{\pgfqpoint{2.118288in}{2.462944in}}{\pgfqpoint{2.110475in}{2.455130in}}%
\pgfpathcurveto{\pgfqpoint{2.102661in}{2.447317in}}{\pgfqpoint{2.098271in}{2.436718in}}{\pgfqpoint{2.098271in}{2.425668in}}%
\pgfpathcurveto{\pgfqpoint{2.098271in}{2.414618in}}{\pgfqpoint{2.102661in}{2.404019in}}{\pgfqpoint{2.110475in}{2.396205in}}%
\pgfpathcurveto{\pgfqpoint{2.118288in}{2.388391in}}{\pgfqpoint{2.128887in}{2.384001in}}{\pgfqpoint{2.139937in}{2.384001in}}%
\pgfpathclose%
\pgfusepath{stroke,fill}%
\end{pgfscope}%
\begin{pgfscope}%
\pgfpathrectangle{\pgfqpoint{0.375000in}{0.330000in}}{\pgfqpoint{2.325000in}{2.310000in}}%
\pgfusepath{clip}%
\pgfsetbuttcap%
\pgfsetroundjoin%
\definecolor{currentfill}{rgb}{0.000000,0.000000,0.000000}%
\pgfsetfillcolor{currentfill}%
\pgfsetlinewidth{1.003750pt}%
\definecolor{currentstroke}{rgb}{0.000000,0.000000,0.000000}%
\pgfsetstrokecolor{currentstroke}%
\pgfsetdash{}{0pt}%
\pgfpathmoveto{\pgfqpoint{2.139937in}{1.892484in}}%
\pgfpathcurveto{\pgfqpoint{2.150988in}{1.892484in}}{\pgfqpoint{2.161587in}{1.896875in}}{\pgfqpoint{2.169400in}{1.904688in}}%
\pgfpathcurveto{\pgfqpoint{2.177214in}{1.912502in}}{\pgfqpoint{2.181604in}{1.923101in}}{\pgfqpoint{2.181604in}{1.934151in}}%
\pgfpathcurveto{\pgfqpoint{2.181604in}{1.945201in}}{\pgfqpoint{2.177214in}{1.955800in}}{\pgfqpoint{2.169400in}{1.963614in}}%
\pgfpathcurveto{\pgfqpoint{2.161587in}{1.971428in}}{\pgfqpoint{2.150988in}{1.975818in}}{\pgfqpoint{2.139937in}{1.975818in}}%
\pgfpathcurveto{\pgfqpoint{2.128887in}{1.975818in}}{\pgfqpoint{2.118288in}{1.971428in}}{\pgfqpoint{2.110475in}{1.963614in}}%
\pgfpathcurveto{\pgfqpoint{2.102661in}{1.955800in}}{\pgfqpoint{2.098271in}{1.945201in}}{\pgfqpoint{2.098271in}{1.934151in}}%
\pgfpathcurveto{\pgfqpoint{2.098271in}{1.923101in}}{\pgfqpoint{2.102661in}{1.912502in}}{\pgfqpoint{2.110475in}{1.904688in}}%
\pgfpathcurveto{\pgfqpoint{2.118288in}{1.896875in}}{\pgfqpoint{2.128887in}{1.892484in}}{\pgfqpoint{2.139937in}{1.892484in}}%
\pgfpathclose%
\pgfusepath{stroke,fill}%
\end{pgfscope}%
\begin{pgfscope}%
\pgfpathrectangle{\pgfqpoint{0.375000in}{0.330000in}}{\pgfqpoint{2.325000in}{2.310000in}}%
\pgfusepath{clip}%
\pgfsetbuttcap%
\pgfsetroundjoin%
\definecolor{currentfill}{rgb}{0.000000,0.000000,0.000000}%
\pgfsetfillcolor{currentfill}%
\pgfsetlinewidth{1.003750pt}%
\definecolor{currentstroke}{rgb}{0.000000,0.000000,0.000000}%
\pgfsetstrokecolor{currentstroke}%
\pgfsetdash{}{0pt}%
\pgfpathmoveto{\pgfqpoint{2.139937in}{2.141277in}}%
\pgfpathcurveto{\pgfqpoint{2.150988in}{2.141277in}}{\pgfqpoint{2.161587in}{2.145667in}}{\pgfqpoint{2.169400in}{2.153481in}}%
\pgfpathcurveto{\pgfqpoint{2.177214in}{2.161294in}}{\pgfqpoint{2.181604in}{2.171893in}}{\pgfqpoint{2.181604in}{2.182943in}}%
\pgfpathcurveto{\pgfqpoint{2.181604in}{2.193994in}}{\pgfqpoint{2.177214in}{2.204593in}}{\pgfqpoint{2.169400in}{2.212406in}}%
\pgfpathcurveto{\pgfqpoint{2.161587in}{2.220220in}}{\pgfqpoint{2.150988in}{2.224610in}}{\pgfqpoint{2.139937in}{2.224610in}}%
\pgfpathcurveto{\pgfqpoint{2.128887in}{2.224610in}}{\pgfqpoint{2.118288in}{2.220220in}}{\pgfqpoint{2.110475in}{2.212406in}}%
\pgfpathcurveto{\pgfqpoint{2.102661in}{2.204593in}}{\pgfqpoint{2.098271in}{2.193994in}}{\pgfqpoint{2.098271in}{2.182943in}}%
\pgfpathcurveto{\pgfqpoint{2.098271in}{2.171893in}}{\pgfqpoint{2.102661in}{2.161294in}}{\pgfqpoint{2.110475in}{2.153481in}}%
\pgfpathcurveto{\pgfqpoint{2.118288in}{2.145667in}}{\pgfqpoint{2.128887in}{2.141277in}}{\pgfqpoint{2.139937in}{2.141277in}}%
\pgfpathclose%
\pgfusepath{stroke,fill}%
\end{pgfscope}%
\begin{pgfscope}%
\pgfpathrectangle{\pgfqpoint{0.375000in}{0.330000in}}{\pgfqpoint{2.325000in}{2.310000in}}%
\pgfusepath{clip}%
\pgfsetbuttcap%
\pgfsetroundjoin%
\definecolor{currentfill}{rgb}{0.000000,0.000000,0.000000}%
\pgfsetfillcolor{currentfill}%
\pgfsetlinewidth{1.003750pt}%
\definecolor{currentstroke}{rgb}{0.000000,0.000000,0.000000}%
\pgfsetstrokecolor{currentstroke}%
\pgfsetdash{}{0pt}%
\pgfpathmoveto{\pgfqpoint{2.139937in}{1.977438in}}%
\pgfpathcurveto{\pgfqpoint{2.150988in}{1.977438in}}{\pgfqpoint{2.161587in}{1.981828in}}{\pgfqpoint{2.169400in}{1.989642in}}%
\pgfpathcurveto{\pgfqpoint{2.177214in}{1.997455in}}{\pgfqpoint{2.181604in}{2.008054in}}{\pgfqpoint{2.181604in}{2.019105in}}%
\pgfpathcurveto{\pgfqpoint{2.181604in}{2.030155in}}{\pgfqpoint{2.177214in}{2.040754in}}{\pgfqpoint{2.169400in}{2.048567in}}%
\pgfpathcurveto{\pgfqpoint{2.161587in}{2.056381in}}{\pgfqpoint{2.150988in}{2.060771in}}{\pgfqpoint{2.139937in}{2.060771in}}%
\pgfpathcurveto{\pgfqpoint{2.128887in}{2.060771in}}{\pgfqpoint{2.118288in}{2.056381in}}{\pgfqpoint{2.110475in}{2.048567in}}%
\pgfpathcurveto{\pgfqpoint{2.102661in}{2.040754in}}{\pgfqpoint{2.098271in}{2.030155in}}{\pgfqpoint{2.098271in}{2.019105in}}%
\pgfpathcurveto{\pgfqpoint{2.098271in}{2.008054in}}{\pgfqpoint{2.102661in}{1.997455in}}{\pgfqpoint{2.110475in}{1.989642in}}%
\pgfpathcurveto{\pgfqpoint{2.118288in}{1.981828in}}{\pgfqpoint{2.128887in}{1.977438in}}{\pgfqpoint{2.139937in}{1.977438in}}%
\pgfpathclose%
\pgfusepath{stroke,fill}%
\end{pgfscope}%
\begin{pgfscope}%
\pgfpathrectangle{\pgfqpoint{0.375000in}{0.330000in}}{\pgfqpoint{2.325000in}{2.310000in}}%
\pgfusepath{clip}%
\pgfsetbuttcap%
\pgfsetroundjoin%
\definecolor{currentfill}{rgb}{0.000000,0.000000,0.000000}%
\pgfsetfillcolor{currentfill}%
\pgfsetlinewidth{1.003750pt}%
\definecolor{currentstroke}{rgb}{0.000000,0.000000,0.000000}%
\pgfsetstrokecolor{currentstroke}%
\pgfsetdash{}{0pt}%
\pgfpathmoveto{\pgfqpoint{2.139937in}{2.038119in}}%
\pgfpathcurveto{\pgfqpoint{2.150988in}{2.038119in}}{\pgfqpoint{2.161587in}{2.042509in}}{\pgfqpoint{2.169400in}{2.050323in}}%
\pgfpathcurveto{\pgfqpoint{2.177214in}{2.058136in}}{\pgfqpoint{2.181604in}{2.068736in}}{\pgfqpoint{2.181604in}{2.079786in}}%
\pgfpathcurveto{\pgfqpoint{2.181604in}{2.090836in}}{\pgfqpoint{2.177214in}{2.101435in}}{\pgfqpoint{2.169400in}{2.109248in}}%
\pgfpathcurveto{\pgfqpoint{2.161587in}{2.117062in}}{\pgfqpoint{2.150988in}{2.121452in}}{\pgfqpoint{2.139937in}{2.121452in}}%
\pgfpathcurveto{\pgfqpoint{2.128887in}{2.121452in}}{\pgfqpoint{2.118288in}{2.117062in}}{\pgfqpoint{2.110475in}{2.109248in}}%
\pgfpathcurveto{\pgfqpoint{2.102661in}{2.101435in}}{\pgfqpoint{2.098271in}{2.090836in}}{\pgfqpoint{2.098271in}{2.079786in}}%
\pgfpathcurveto{\pgfqpoint{2.098271in}{2.068736in}}{\pgfqpoint{2.102661in}{2.058136in}}{\pgfqpoint{2.110475in}{2.050323in}}%
\pgfpathcurveto{\pgfqpoint{2.118288in}{2.042509in}}{\pgfqpoint{2.128887in}{2.038119in}}{\pgfqpoint{2.139937in}{2.038119in}}%
\pgfpathclose%
\pgfusepath{stroke,fill}%
\end{pgfscope}%
\begin{pgfscope}%
\pgfpathrectangle{\pgfqpoint{0.375000in}{0.330000in}}{\pgfqpoint{2.325000in}{2.310000in}}%
\pgfusepath{clip}%
\pgfsetbuttcap%
\pgfsetroundjoin%
\definecolor{currentfill}{rgb}{0.000000,0.000000,0.000000}%
\pgfsetfillcolor{currentfill}%
\pgfsetlinewidth{1.003750pt}%
\definecolor{currentstroke}{rgb}{0.000000,0.000000,0.000000}%
\pgfsetstrokecolor{currentstroke}%
\pgfsetdash{}{0pt}%
\pgfpathmoveto{\pgfqpoint{2.139937in}{2.001710in}}%
\pgfpathcurveto{\pgfqpoint{2.150988in}{2.001710in}}{\pgfqpoint{2.161587in}{2.006101in}}{\pgfqpoint{2.169400in}{2.013914in}}%
\pgfpathcurveto{\pgfqpoint{2.177214in}{2.021728in}}{\pgfqpoint{2.181604in}{2.032327in}}{\pgfqpoint{2.181604in}{2.043377in}}%
\pgfpathcurveto{\pgfqpoint{2.181604in}{2.054427in}}{\pgfqpoint{2.177214in}{2.065026in}}{\pgfqpoint{2.169400in}{2.072840in}}%
\pgfpathcurveto{\pgfqpoint{2.161587in}{2.080653in}}{\pgfqpoint{2.150988in}{2.085044in}}{\pgfqpoint{2.139937in}{2.085044in}}%
\pgfpathcurveto{\pgfqpoint{2.128887in}{2.085044in}}{\pgfqpoint{2.118288in}{2.080653in}}{\pgfqpoint{2.110475in}{2.072840in}}%
\pgfpathcurveto{\pgfqpoint{2.102661in}{2.065026in}}{\pgfqpoint{2.098271in}{2.054427in}}{\pgfqpoint{2.098271in}{2.043377in}}%
\pgfpathcurveto{\pgfqpoint{2.098271in}{2.032327in}}{\pgfqpoint{2.102661in}{2.021728in}}{\pgfqpoint{2.110475in}{2.013914in}}%
\pgfpathcurveto{\pgfqpoint{2.118288in}{2.006101in}}{\pgfqpoint{2.128887in}{2.001710in}}{\pgfqpoint{2.139937in}{2.001710in}}%
\pgfpathclose%
\pgfusepath{stroke,fill}%
\end{pgfscope}%
\begin{pgfscope}%
\pgfpathrectangle{\pgfqpoint{0.375000in}{0.330000in}}{\pgfqpoint{2.325000in}{2.310000in}}%
\pgfusepath{clip}%
\pgfsetbuttcap%
\pgfsetroundjoin%
\definecolor{currentfill}{rgb}{0.000000,0.000000,0.000000}%
\pgfsetfillcolor{currentfill}%
\pgfsetlinewidth{1.003750pt}%
\definecolor{currentstroke}{rgb}{0.000000,0.000000,0.000000}%
\pgfsetstrokecolor{currentstroke}%
\pgfsetdash{}{0pt}%
\pgfpathmoveto{\pgfqpoint{2.139937in}{2.019915in}}%
\pgfpathcurveto{\pgfqpoint{2.150988in}{2.019915in}}{\pgfqpoint{2.161587in}{2.024305in}}{\pgfqpoint{2.169400in}{2.032119in}}%
\pgfpathcurveto{\pgfqpoint{2.177214in}{2.039932in}}{\pgfqpoint{2.181604in}{2.050531in}}{\pgfqpoint{2.181604in}{2.061581in}}%
\pgfpathcurveto{\pgfqpoint{2.181604in}{2.072631in}}{\pgfqpoint{2.177214in}{2.083230in}}{\pgfqpoint{2.169400in}{2.091044in}}%
\pgfpathcurveto{\pgfqpoint{2.161587in}{2.098858in}}{\pgfqpoint{2.150988in}{2.103248in}}{\pgfqpoint{2.139937in}{2.103248in}}%
\pgfpathcurveto{\pgfqpoint{2.128887in}{2.103248in}}{\pgfqpoint{2.118288in}{2.098858in}}{\pgfqpoint{2.110475in}{2.091044in}}%
\pgfpathcurveto{\pgfqpoint{2.102661in}{2.083230in}}{\pgfqpoint{2.098271in}{2.072631in}}{\pgfqpoint{2.098271in}{2.061581in}}%
\pgfpathcurveto{\pgfqpoint{2.098271in}{2.050531in}}{\pgfqpoint{2.102661in}{2.039932in}}{\pgfqpoint{2.110475in}{2.032119in}}%
\pgfpathcurveto{\pgfqpoint{2.118288in}{2.024305in}}{\pgfqpoint{2.128887in}{2.019915in}}{\pgfqpoint{2.139937in}{2.019915in}}%
\pgfpathclose%
\pgfusepath{stroke,fill}%
\end{pgfscope}%
\begin{pgfscope}%
\pgfpathrectangle{\pgfqpoint{0.375000in}{0.330000in}}{\pgfqpoint{2.325000in}{2.310000in}}%
\pgfusepath{clip}%
\pgfsetbuttcap%
\pgfsetroundjoin%
\definecolor{currentfill}{rgb}{0.000000,0.000000,0.000000}%
\pgfsetfillcolor{currentfill}%
\pgfsetlinewidth{1.003750pt}%
\definecolor{currentstroke}{rgb}{0.000000,0.000000,0.000000}%
\pgfsetstrokecolor{currentstroke}%
\pgfsetdash{}{0pt}%
\pgfpathmoveto{\pgfqpoint{2.139937in}{2.232298in}}%
\pgfpathcurveto{\pgfqpoint{2.150988in}{2.232298in}}{\pgfqpoint{2.161587in}{2.236689in}}{\pgfqpoint{2.169400in}{2.244502in}}%
\pgfpathcurveto{\pgfqpoint{2.177214in}{2.252316in}}{\pgfqpoint{2.181604in}{2.262915in}}{\pgfqpoint{2.181604in}{2.273965in}}%
\pgfpathcurveto{\pgfqpoint{2.181604in}{2.285015in}}{\pgfqpoint{2.177214in}{2.295614in}}{\pgfqpoint{2.169400in}{2.303428in}}%
\pgfpathcurveto{\pgfqpoint{2.161587in}{2.311241in}}{\pgfqpoint{2.150988in}{2.315632in}}{\pgfqpoint{2.139937in}{2.315632in}}%
\pgfpathcurveto{\pgfqpoint{2.128887in}{2.315632in}}{\pgfqpoint{2.118288in}{2.311241in}}{\pgfqpoint{2.110475in}{2.303428in}}%
\pgfpathcurveto{\pgfqpoint{2.102661in}{2.295614in}}{\pgfqpoint{2.098271in}{2.285015in}}{\pgfqpoint{2.098271in}{2.273965in}}%
\pgfpathcurveto{\pgfqpoint{2.098271in}{2.262915in}}{\pgfqpoint{2.102661in}{2.252316in}}{\pgfqpoint{2.110475in}{2.244502in}}%
\pgfpathcurveto{\pgfqpoint{2.118288in}{2.236689in}}{\pgfqpoint{2.128887in}{2.232298in}}{\pgfqpoint{2.139937in}{2.232298in}}%
\pgfpathclose%
\pgfusepath{stroke,fill}%
\end{pgfscope}%
\begin{pgfscope}%
\pgfpathrectangle{\pgfqpoint{0.375000in}{0.330000in}}{\pgfqpoint{2.325000in}{2.310000in}}%
\pgfusepath{clip}%
\pgfsetbuttcap%
\pgfsetroundjoin%
\definecolor{currentfill}{rgb}{0.000000,0.000000,0.000000}%
\pgfsetfillcolor{currentfill}%
\pgfsetlinewidth{1.003750pt}%
\definecolor{currentstroke}{rgb}{0.000000,0.000000,0.000000}%
\pgfsetstrokecolor{currentstroke}%
\pgfsetdash{}{0pt}%
\pgfpathmoveto{\pgfqpoint{2.139937in}{1.934961in}}%
\pgfpathcurveto{\pgfqpoint{2.150988in}{1.934961in}}{\pgfqpoint{2.161587in}{1.939351in}}{\pgfqpoint{2.169400in}{1.947165in}}%
\pgfpathcurveto{\pgfqpoint{2.177214in}{1.954979in}}{\pgfqpoint{2.181604in}{1.965578in}}{\pgfqpoint{2.181604in}{1.976628in}}%
\pgfpathcurveto{\pgfqpoint{2.181604in}{1.987678in}}{\pgfqpoint{2.177214in}{1.998277in}}{\pgfqpoint{2.169400in}{2.006091in}}%
\pgfpathcurveto{\pgfqpoint{2.161587in}{2.013904in}}{\pgfqpoint{2.150988in}{2.018295in}}{\pgfqpoint{2.139937in}{2.018295in}}%
\pgfpathcurveto{\pgfqpoint{2.128887in}{2.018295in}}{\pgfqpoint{2.118288in}{2.013904in}}{\pgfqpoint{2.110475in}{2.006091in}}%
\pgfpathcurveto{\pgfqpoint{2.102661in}{1.998277in}}{\pgfqpoint{2.098271in}{1.987678in}}{\pgfqpoint{2.098271in}{1.976628in}}%
\pgfpathcurveto{\pgfqpoint{2.098271in}{1.965578in}}{\pgfqpoint{2.102661in}{1.954979in}}{\pgfqpoint{2.110475in}{1.947165in}}%
\pgfpathcurveto{\pgfqpoint{2.118288in}{1.939351in}}{\pgfqpoint{2.128887in}{1.934961in}}{\pgfqpoint{2.139937in}{1.934961in}}%
\pgfpathclose%
\pgfusepath{stroke,fill}%
\end{pgfscope}%
\begin{pgfscope}%
\pgfpathrectangle{\pgfqpoint{0.375000in}{0.330000in}}{\pgfqpoint{2.325000in}{2.310000in}}%
\pgfusepath{clip}%
\pgfsetbuttcap%
\pgfsetroundjoin%
\definecolor{currentfill}{rgb}{0.000000,0.000000,0.000000}%
\pgfsetfillcolor{currentfill}%
\pgfsetlinewidth{1.003750pt}%
\definecolor{currentstroke}{rgb}{0.000000,0.000000,0.000000}%
\pgfsetstrokecolor{currentstroke}%
\pgfsetdash{}{0pt}%
\pgfpathmoveto{\pgfqpoint{2.139937in}{2.305116in}}%
\pgfpathcurveto{\pgfqpoint{2.150988in}{2.305116in}}{\pgfqpoint{2.161587in}{2.309506in}}{\pgfqpoint{2.169400in}{2.317320in}}%
\pgfpathcurveto{\pgfqpoint{2.177214in}{2.325133in}}{\pgfqpoint{2.181604in}{2.335732in}}{\pgfqpoint{2.181604in}{2.346782in}}%
\pgfpathcurveto{\pgfqpoint{2.181604in}{2.357832in}}{\pgfqpoint{2.177214in}{2.368431in}}{\pgfqpoint{2.169400in}{2.376245in}}%
\pgfpathcurveto{\pgfqpoint{2.161587in}{2.384059in}}{\pgfqpoint{2.150988in}{2.388449in}}{\pgfqpoint{2.139937in}{2.388449in}}%
\pgfpathcurveto{\pgfqpoint{2.128887in}{2.388449in}}{\pgfqpoint{2.118288in}{2.384059in}}{\pgfqpoint{2.110475in}{2.376245in}}%
\pgfpathcurveto{\pgfqpoint{2.102661in}{2.368431in}}{\pgfqpoint{2.098271in}{2.357832in}}{\pgfqpoint{2.098271in}{2.346782in}}%
\pgfpathcurveto{\pgfqpoint{2.098271in}{2.335732in}}{\pgfqpoint{2.102661in}{2.325133in}}{\pgfqpoint{2.110475in}{2.317320in}}%
\pgfpathcurveto{\pgfqpoint{2.118288in}{2.309506in}}{\pgfqpoint{2.128887in}{2.305116in}}{\pgfqpoint{2.139937in}{2.305116in}}%
\pgfpathclose%
\pgfusepath{stroke,fill}%
\end{pgfscope}%
\begin{pgfscope}%
\pgfpathrectangle{\pgfqpoint{0.375000in}{0.330000in}}{\pgfqpoint{2.325000in}{2.310000in}}%
\pgfusepath{clip}%
\pgfsetbuttcap%
\pgfsetroundjoin%
\definecolor{currentfill}{rgb}{0.000000,0.000000,0.000000}%
\pgfsetfillcolor{currentfill}%
\pgfsetlinewidth{1.003750pt}%
\definecolor{currentstroke}{rgb}{0.000000,0.000000,0.000000}%
\pgfsetstrokecolor{currentstroke}%
\pgfsetdash{}{0pt}%
\pgfpathmoveto{\pgfqpoint{2.139937in}{2.032051in}}%
\pgfpathcurveto{\pgfqpoint{2.150988in}{2.032051in}}{\pgfqpoint{2.161587in}{2.036441in}}{\pgfqpoint{2.169400in}{2.044255in}}%
\pgfpathcurveto{\pgfqpoint{2.177214in}{2.052068in}}{\pgfqpoint{2.181604in}{2.062667in}}{\pgfqpoint{2.181604in}{2.073718in}}%
\pgfpathcurveto{\pgfqpoint{2.181604in}{2.084768in}}{\pgfqpoint{2.177214in}{2.095367in}}{\pgfqpoint{2.169400in}{2.103180in}}%
\pgfpathcurveto{\pgfqpoint{2.161587in}{2.110994in}}{\pgfqpoint{2.150988in}{2.115384in}}{\pgfqpoint{2.139937in}{2.115384in}}%
\pgfpathcurveto{\pgfqpoint{2.128887in}{2.115384in}}{\pgfqpoint{2.118288in}{2.110994in}}{\pgfqpoint{2.110475in}{2.103180in}}%
\pgfpathcurveto{\pgfqpoint{2.102661in}{2.095367in}}{\pgfqpoint{2.098271in}{2.084768in}}{\pgfqpoint{2.098271in}{2.073718in}}%
\pgfpathcurveto{\pgfqpoint{2.098271in}{2.062667in}}{\pgfqpoint{2.102661in}{2.052068in}}{\pgfqpoint{2.110475in}{2.044255in}}%
\pgfpathcurveto{\pgfqpoint{2.118288in}{2.036441in}}{\pgfqpoint{2.128887in}{2.032051in}}{\pgfqpoint{2.139937in}{2.032051in}}%
\pgfpathclose%
\pgfusepath{stroke,fill}%
\end{pgfscope}%
\begin{pgfscope}%
\pgfpathrectangle{\pgfqpoint{0.375000in}{0.330000in}}{\pgfqpoint{2.325000in}{2.310000in}}%
\pgfusepath{clip}%
\pgfsetbuttcap%
\pgfsetroundjoin%
\definecolor{currentfill}{rgb}{0.000000,0.000000,0.000000}%
\pgfsetfillcolor{currentfill}%
\pgfsetlinewidth{1.003750pt}%
\definecolor{currentstroke}{rgb}{0.000000,0.000000,0.000000}%
\pgfsetstrokecolor{currentstroke}%
\pgfsetdash{}{0pt}%
\pgfpathmoveto{\pgfqpoint{2.139937in}{2.238366in}}%
\pgfpathcurveto{\pgfqpoint{2.150988in}{2.238366in}}{\pgfqpoint{2.161587in}{2.242757in}}{\pgfqpoint{2.169400in}{2.250570in}}%
\pgfpathcurveto{\pgfqpoint{2.177214in}{2.258384in}}{\pgfqpoint{2.181604in}{2.268983in}}{\pgfqpoint{2.181604in}{2.280033in}}%
\pgfpathcurveto{\pgfqpoint{2.181604in}{2.291083in}}{\pgfqpoint{2.177214in}{2.301682in}}{\pgfqpoint{2.169400in}{2.309496in}}%
\pgfpathcurveto{\pgfqpoint{2.161587in}{2.317310in}}{\pgfqpoint{2.150988in}{2.321700in}}{\pgfqpoint{2.139937in}{2.321700in}}%
\pgfpathcurveto{\pgfqpoint{2.128887in}{2.321700in}}{\pgfqpoint{2.118288in}{2.317310in}}{\pgfqpoint{2.110475in}{2.309496in}}%
\pgfpathcurveto{\pgfqpoint{2.102661in}{2.301682in}}{\pgfqpoint{2.098271in}{2.291083in}}{\pgfqpoint{2.098271in}{2.280033in}}%
\pgfpathcurveto{\pgfqpoint{2.098271in}{2.268983in}}{\pgfqpoint{2.102661in}{2.258384in}}{\pgfqpoint{2.110475in}{2.250570in}}%
\pgfpathcurveto{\pgfqpoint{2.118288in}{2.242757in}}{\pgfqpoint{2.128887in}{2.238366in}}{\pgfqpoint{2.139937in}{2.238366in}}%
\pgfpathclose%
\pgfusepath{stroke,fill}%
\end{pgfscope}%
\begin{pgfscope}%
\pgfpathrectangle{\pgfqpoint{0.375000in}{0.330000in}}{\pgfqpoint{2.325000in}{2.310000in}}%
\pgfusepath{clip}%
\pgfsetbuttcap%
\pgfsetroundjoin%
\definecolor{currentfill}{rgb}{0.000000,0.000000,0.000000}%
\pgfsetfillcolor{currentfill}%
\pgfsetlinewidth{1.003750pt}%
\definecolor{currentstroke}{rgb}{0.000000,0.000000,0.000000}%
\pgfsetstrokecolor{currentstroke}%
\pgfsetdash{}{0pt}%
\pgfpathmoveto{\pgfqpoint{2.139937in}{1.916757in}}%
\pgfpathcurveto{\pgfqpoint{2.150988in}{1.916757in}}{\pgfqpoint{2.161587in}{1.921147in}}{\pgfqpoint{2.169400in}{1.928961in}}%
\pgfpathcurveto{\pgfqpoint{2.177214in}{1.936774in}}{\pgfqpoint{2.181604in}{1.947373in}}{\pgfqpoint{2.181604in}{1.958424in}}%
\pgfpathcurveto{\pgfqpoint{2.181604in}{1.969474in}}{\pgfqpoint{2.177214in}{1.980073in}}{\pgfqpoint{2.169400in}{1.987886in}}%
\pgfpathcurveto{\pgfqpoint{2.161587in}{1.995700in}}{\pgfqpoint{2.150988in}{2.000090in}}{\pgfqpoint{2.139937in}{2.000090in}}%
\pgfpathcurveto{\pgfqpoint{2.128887in}{2.000090in}}{\pgfqpoint{2.118288in}{1.995700in}}{\pgfqpoint{2.110475in}{1.987886in}}%
\pgfpathcurveto{\pgfqpoint{2.102661in}{1.980073in}}{\pgfqpoint{2.098271in}{1.969474in}}{\pgfqpoint{2.098271in}{1.958424in}}%
\pgfpathcurveto{\pgfqpoint{2.098271in}{1.947373in}}{\pgfqpoint{2.102661in}{1.936774in}}{\pgfqpoint{2.110475in}{1.928961in}}%
\pgfpathcurveto{\pgfqpoint{2.118288in}{1.921147in}}{\pgfqpoint{2.128887in}{1.916757in}}{\pgfqpoint{2.139937in}{1.916757in}}%
\pgfpathclose%
\pgfusepath{stroke,fill}%
\end{pgfscope}%
\begin{pgfscope}%
\pgfpathrectangle{\pgfqpoint{0.375000in}{0.330000in}}{\pgfqpoint{2.325000in}{2.310000in}}%
\pgfusepath{clip}%
\pgfsetbuttcap%
\pgfsetroundjoin%
\definecolor{currentfill}{rgb}{0.000000,0.000000,0.000000}%
\pgfsetfillcolor{currentfill}%
\pgfsetlinewidth{1.003750pt}%
\definecolor{currentstroke}{rgb}{0.000000,0.000000,0.000000}%
\pgfsetstrokecolor{currentstroke}%
\pgfsetdash{}{0pt}%
\pgfpathmoveto{\pgfqpoint{2.139937in}{2.214094in}}%
\pgfpathcurveto{\pgfqpoint{2.150988in}{2.214094in}}{\pgfqpoint{2.161587in}{2.218484in}}{\pgfqpoint{2.169400in}{2.226298in}}%
\pgfpathcurveto{\pgfqpoint{2.177214in}{2.234112in}}{\pgfqpoint{2.181604in}{2.244711in}}{\pgfqpoint{2.181604in}{2.255761in}}%
\pgfpathcurveto{\pgfqpoint{2.181604in}{2.266811in}}{\pgfqpoint{2.177214in}{2.277410in}}{\pgfqpoint{2.169400in}{2.285223in}}%
\pgfpathcurveto{\pgfqpoint{2.161587in}{2.293037in}}{\pgfqpoint{2.150988in}{2.297427in}}{\pgfqpoint{2.139937in}{2.297427in}}%
\pgfpathcurveto{\pgfqpoint{2.128887in}{2.297427in}}{\pgfqpoint{2.118288in}{2.293037in}}{\pgfqpoint{2.110475in}{2.285223in}}%
\pgfpathcurveto{\pgfqpoint{2.102661in}{2.277410in}}{\pgfqpoint{2.098271in}{2.266811in}}{\pgfqpoint{2.098271in}{2.255761in}}%
\pgfpathcurveto{\pgfqpoint{2.098271in}{2.244711in}}{\pgfqpoint{2.102661in}{2.234112in}}{\pgfqpoint{2.110475in}{2.226298in}}%
\pgfpathcurveto{\pgfqpoint{2.118288in}{2.218484in}}{\pgfqpoint{2.128887in}{2.214094in}}{\pgfqpoint{2.139937in}{2.214094in}}%
\pgfpathclose%
\pgfusepath{stroke,fill}%
\end{pgfscope}%
\begin{pgfscope}%
\pgfpathrectangle{\pgfqpoint{0.375000in}{0.330000in}}{\pgfqpoint{2.325000in}{2.310000in}}%
\pgfusepath{clip}%
\pgfsetbuttcap%
\pgfsetroundjoin%
\definecolor{currentfill}{rgb}{0.000000,0.000000,0.000000}%
\pgfsetfillcolor{currentfill}%
\pgfsetlinewidth{1.003750pt}%
\definecolor{currentstroke}{rgb}{0.000000,0.000000,0.000000}%
\pgfsetstrokecolor{currentstroke}%
\pgfsetdash{}{0pt}%
\pgfpathmoveto{\pgfqpoint{2.139937in}{1.916757in}}%
\pgfpathcurveto{\pgfqpoint{2.150988in}{1.916757in}}{\pgfqpoint{2.161587in}{1.921147in}}{\pgfqpoint{2.169400in}{1.928961in}}%
\pgfpathcurveto{\pgfqpoint{2.177214in}{1.936774in}}{\pgfqpoint{2.181604in}{1.947373in}}{\pgfqpoint{2.181604in}{1.958424in}}%
\pgfpathcurveto{\pgfqpoint{2.181604in}{1.969474in}}{\pgfqpoint{2.177214in}{1.980073in}}{\pgfqpoint{2.169400in}{1.987886in}}%
\pgfpathcurveto{\pgfqpoint{2.161587in}{1.995700in}}{\pgfqpoint{2.150988in}{2.000090in}}{\pgfqpoint{2.139937in}{2.000090in}}%
\pgfpathcurveto{\pgfqpoint{2.128887in}{2.000090in}}{\pgfqpoint{2.118288in}{1.995700in}}{\pgfqpoint{2.110475in}{1.987886in}}%
\pgfpathcurveto{\pgfqpoint{2.102661in}{1.980073in}}{\pgfqpoint{2.098271in}{1.969474in}}{\pgfqpoint{2.098271in}{1.958424in}}%
\pgfpathcurveto{\pgfqpoint{2.098271in}{1.947373in}}{\pgfqpoint{2.102661in}{1.936774in}}{\pgfqpoint{2.110475in}{1.928961in}}%
\pgfpathcurveto{\pgfqpoint{2.118288in}{1.921147in}}{\pgfqpoint{2.128887in}{1.916757in}}{\pgfqpoint{2.139937in}{1.916757in}}%
\pgfpathclose%
\pgfusepath{stroke,fill}%
\end{pgfscope}%
\begin{pgfscope}%
\pgfpathrectangle{\pgfqpoint{0.375000in}{0.330000in}}{\pgfqpoint{2.325000in}{2.310000in}}%
\pgfusepath{clip}%
\pgfsetbuttcap%
\pgfsetroundjoin%
\definecolor{currentfill}{rgb}{0.000000,0.000000,0.000000}%
\pgfsetfillcolor{currentfill}%
\pgfsetlinewidth{1.003750pt}%
\definecolor{currentstroke}{rgb}{0.000000,0.000000,0.000000}%
\pgfsetstrokecolor{currentstroke}%
\pgfsetdash{}{0pt}%
\pgfpathmoveto{\pgfqpoint{2.139937in}{1.965302in}}%
\pgfpathcurveto{\pgfqpoint{2.150988in}{1.965302in}}{\pgfqpoint{2.161587in}{1.969692in}}{\pgfqpoint{2.169400in}{1.977506in}}%
\pgfpathcurveto{\pgfqpoint{2.177214in}{1.985319in}}{\pgfqpoint{2.181604in}{1.995918in}}{\pgfqpoint{2.181604in}{2.006968in}}%
\pgfpathcurveto{\pgfqpoint{2.181604in}{2.018019in}}{\pgfqpoint{2.177214in}{2.028618in}}{\pgfqpoint{2.169400in}{2.036431in}}%
\pgfpathcurveto{\pgfqpoint{2.161587in}{2.044245in}}{\pgfqpoint{2.150988in}{2.048635in}}{\pgfqpoint{2.139937in}{2.048635in}}%
\pgfpathcurveto{\pgfqpoint{2.128887in}{2.048635in}}{\pgfqpoint{2.118288in}{2.044245in}}{\pgfqpoint{2.110475in}{2.036431in}}%
\pgfpathcurveto{\pgfqpoint{2.102661in}{2.028618in}}{\pgfqpoint{2.098271in}{2.018019in}}{\pgfqpoint{2.098271in}{2.006968in}}%
\pgfpathcurveto{\pgfqpoint{2.098271in}{1.995918in}}{\pgfqpoint{2.102661in}{1.985319in}}{\pgfqpoint{2.110475in}{1.977506in}}%
\pgfpathcurveto{\pgfqpoint{2.118288in}{1.969692in}}{\pgfqpoint{2.128887in}{1.965302in}}{\pgfqpoint{2.139937in}{1.965302in}}%
\pgfpathclose%
\pgfusepath{stroke,fill}%
\end{pgfscope}%
\begin{pgfscope}%
\pgfpathrectangle{\pgfqpoint{0.375000in}{0.330000in}}{\pgfqpoint{2.325000in}{2.310000in}}%
\pgfusepath{clip}%
\pgfsetbuttcap%
\pgfsetroundjoin%
\definecolor{currentfill}{rgb}{0.000000,0.000000,0.000000}%
\pgfsetfillcolor{currentfill}%
\pgfsetlinewidth{1.003750pt}%
\definecolor{currentstroke}{rgb}{0.000000,0.000000,0.000000}%
\pgfsetstrokecolor{currentstroke}%
\pgfsetdash{}{0pt}%
\pgfpathmoveto{\pgfqpoint{2.139937in}{1.995642in}}%
\pgfpathcurveto{\pgfqpoint{2.150988in}{1.995642in}}{\pgfqpoint{2.161587in}{2.000033in}}{\pgfqpoint{2.169400in}{2.007846in}}%
\pgfpathcurveto{\pgfqpoint{2.177214in}{2.015660in}}{\pgfqpoint{2.181604in}{2.026259in}}{\pgfqpoint{2.181604in}{2.037309in}}%
\pgfpathcurveto{\pgfqpoint{2.181604in}{2.048359in}}{\pgfqpoint{2.177214in}{2.058958in}}{\pgfqpoint{2.169400in}{2.066772in}}%
\pgfpathcurveto{\pgfqpoint{2.161587in}{2.074585in}}{\pgfqpoint{2.150988in}{2.078976in}}{\pgfqpoint{2.139937in}{2.078976in}}%
\pgfpathcurveto{\pgfqpoint{2.128887in}{2.078976in}}{\pgfqpoint{2.118288in}{2.074585in}}{\pgfqpoint{2.110475in}{2.066772in}}%
\pgfpathcurveto{\pgfqpoint{2.102661in}{2.058958in}}{\pgfqpoint{2.098271in}{2.048359in}}{\pgfqpoint{2.098271in}{2.037309in}}%
\pgfpathcurveto{\pgfqpoint{2.098271in}{2.026259in}}{\pgfqpoint{2.102661in}{2.015660in}}{\pgfqpoint{2.110475in}{2.007846in}}%
\pgfpathcurveto{\pgfqpoint{2.118288in}{2.000033in}}{\pgfqpoint{2.128887in}{1.995642in}}{\pgfqpoint{2.139937in}{1.995642in}}%
\pgfpathclose%
\pgfusepath{stroke,fill}%
\end{pgfscope}%
\begin{pgfscope}%
\pgfpathrectangle{\pgfqpoint{0.375000in}{0.330000in}}{\pgfqpoint{2.325000in}{2.310000in}}%
\pgfusepath{clip}%
\pgfsetbuttcap%
\pgfsetroundjoin%
\definecolor{currentfill}{rgb}{0.000000,0.000000,0.000000}%
\pgfsetfillcolor{currentfill}%
\pgfsetlinewidth{1.003750pt}%
\definecolor{currentstroke}{rgb}{0.000000,0.000000,0.000000}%
\pgfsetstrokecolor{currentstroke}%
\pgfsetdash{}{0pt}%
\pgfpathmoveto{\pgfqpoint{2.139937in}{1.892484in}}%
\pgfpathcurveto{\pgfqpoint{2.150988in}{1.892484in}}{\pgfqpoint{2.161587in}{1.896875in}}{\pgfqpoint{2.169400in}{1.904688in}}%
\pgfpathcurveto{\pgfqpoint{2.177214in}{1.912502in}}{\pgfqpoint{2.181604in}{1.923101in}}{\pgfqpoint{2.181604in}{1.934151in}}%
\pgfpathcurveto{\pgfqpoint{2.181604in}{1.945201in}}{\pgfqpoint{2.177214in}{1.955800in}}{\pgfqpoint{2.169400in}{1.963614in}}%
\pgfpathcurveto{\pgfqpoint{2.161587in}{1.971428in}}{\pgfqpoint{2.150988in}{1.975818in}}{\pgfqpoint{2.139937in}{1.975818in}}%
\pgfpathcurveto{\pgfqpoint{2.128887in}{1.975818in}}{\pgfqpoint{2.118288in}{1.971428in}}{\pgfqpoint{2.110475in}{1.963614in}}%
\pgfpathcurveto{\pgfqpoint{2.102661in}{1.955800in}}{\pgfqpoint{2.098271in}{1.945201in}}{\pgfqpoint{2.098271in}{1.934151in}}%
\pgfpathcurveto{\pgfqpoint{2.098271in}{1.923101in}}{\pgfqpoint{2.102661in}{1.912502in}}{\pgfqpoint{2.110475in}{1.904688in}}%
\pgfpathcurveto{\pgfqpoint{2.118288in}{1.896875in}}{\pgfqpoint{2.128887in}{1.892484in}}{\pgfqpoint{2.139937in}{1.892484in}}%
\pgfpathclose%
\pgfusepath{stroke,fill}%
\end{pgfscope}%
\begin{pgfscope}%
\pgfpathrectangle{\pgfqpoint{0.375000in}{0.330000in}}{\pgfqpoint{2.325000in}{2.310000in}}%
\pgfusepath{clip}%
\pgfsetbuttcap%
\pgfsetroundjoin%
\definecolor{currentfill}{rgb}{0.000000,0.000000,0.000000}%
\pgfsetfillcolor{currentfill}%
\pgfsetlinewidth{1.003750pt}%
\definecolor{currentstroke}{rgb}{0.000000,0.000000,0.000000}%
\pgfsetstrokecolor{currentstroke}%
\pgfsetdash{}{0pt}%
\pgfpathmoveto{\pgfqpoint{2.139937in}{2.086664in}}%
\pgfpathcurveto{\pgfqpoint{2.150988in}{2.086664in}}{\pgfqpoint{2.161587in}{2.091054in}}{\pgfqpoint{2.169400in}{2.098868in}}%
\pgfpathcurveto{\pgfqpoint{2.177214in}{2.106681in}}{\pgfqpoint{2.181604in}{2.117280in}}{\pgfqpoint{2.181604in}{2.128330in}}%
\pgfpathcurveto{\pgfqpoint{2.181604in}{2.139381in}}{\pgfqpoint{2.177214in}{2.149980in}}{\pgfqpoint{2.169400in}{2.157793in}}%
\pgfpathcurveto{\pgfqpoint{2.161587in}{2.165607in}}{\pgfqpoint{2.150988in}{2.169997in}}{\pgfqpoint{2.139937in}{2.169997in}}%
\pgfpathcurveto{\pgfqpoint{2.128887in}{2.169997in}}{\pgfqpoint{2.118288in}{2.165607in}}{\pgfqpoint{2.110475in}{2.157793in}}%
\pgfpathcurveto{\pgfqpoint{2.102661in}{2.149980in}}{\pgfqpoint{2.098271in}{2.139381in}}{\pgfqpoint{2.098271in}{2.128330in}}%
\pgfpathcurveto{\pgfqpoint{2.098271in}{2.117280in}}{\pgfqpoint{2.102661in}{2.106681in}}{\pgfqpoint{2.110475in}{2.098868in}}%
\pgfpathcurveto{\pgfqpoint{2.118288in}{2.091054in}}{\pgfqpoint{2.128887in}{2.086664in}}{\pgfqpoint{2.139937in}{2.086664in}}%
\pgfpathclose%
\pgfusepath{stroke,fill}%
\end{pgfscope}%
\begin{pgfscope}%
\pgfpathrectangle{\pgfqpoint{0.375000in}{0.330000in}}{\pgfqpoint{2.325000in}{2.310000in}}%
\pgfusepath{clip}%
\pgfsetbuttcap%
\pgfsetroundjoin%
\definecolor{currentfill}{rgb}{0.000000,0.000000,0.000000}%
\pgfsetfillcolor{currentfill}%
\pgfsetlinewidth{1.003750pt}%
\definecolor{currentstroke}{rgb}{0.000000,0.000000,0.000000}%
\pgfsetstrokecolor{currentstroke}%
\pgfsetdash{}{0pt}%
\pgfpathmoveto{\pgfqpoint{2.139937in}{2.056323in}}%
\pgfpathcurveto{\pgfqpoint{2.150988in}{2.056323in}}{\pgfqpoint{2.161587in}{2.060714in}}{\pgfqpoint{2.169400in}{2.068527in}}%
\pgfpathcurveto{\pgfqpoint{2.177214in}{2.076341in}}{\pgfqpoint{2.181604in}{2.086940in}}{\pgfqpoint{2.181604in}{2.097990in}}%
\pgfpathcurveto{\pgfqpoint{2.181604in}{2.109040in}}{\pgfqpoint{2.177214in}{2.119639in}}{\pgfqpoint{2.169400in}{2.127453in}}%
\pgfpathcurveto{\pgfqpoint{2.161587in}{2.135266in}}{\pgfqpoint{2.150988in}{2.139657in}}{\pgfqpoint{2.139937in}{2.139657in}}%
\pgfpathcurveto{\pgfqpoint{2.128887in}{2.139657in}}{\pgfqpoint{2.118288in}{2.135266in}}{\pgfqpoint{2.110475in}{2.127453in}}%
\pgfpathcurveto{\pgfqpoint{2.102661in}{2.119639in}}{\pgfqpoint{2.098271in}{2.109040in}}{\pgfqpoint{2.098271in}{2.097990in}}%
\pgfpathcurveto{\pgfqpoint{2.098271in}{2.086940in}}{\pgfqpoint{2.102661in}{2.076341in}}{\pgfqpoint{2.110475in}{2.068527in}}%
\pgfpathcurveto{\pgfqpoint{2.118288in}{2.060714in}}{\pgfqpoint{2.128887in}{2.056323in}}{\pgfqpoint{2.139937in}{2.056323in}}%
\pgfpathclose%
\pgfusepath{stroke,fill}%
\end{pgfscope}%
\begin{pgfscope}%
\pgfpathrectangle{\pgfqpoint{0.375000in}{0.330000in}}{\pgfqpoint{2.325000in}{2.310000in}}%
\pgfusepath{clip}%
\pgfsetbuttcap%
\pgfsetroundjoin%
\definecolor{currentfill}{rgb}{0.000000,0.000000,0.000000}%
\pgfsetfillcolor{currentfill}%
\pgfsetlinewidth{1.003750pt}%
\definecolor{currentstroke}{rgb}{0.000000,0.000000,0.000000}%
\pgfsetstrokecolor{currentstroke}%
\pgfsetdash{}{0pt}%
\pgfpathmoveto{\pgfqpoint{2.139937in}{1.886416in}}%
\pgfpathcurveto{\pgfqpoint{2.150988in}{1.886416in}}{\pgfqpoint{2.161587in}{1.890807in}}{\pgfqpoint{2.169400in}{1.898620in}}%
\pgfpathcurveto{\pgfqpoint{2.177214in}{1.906434in}}{\pgfqpoint{2.181604in}{1.917033in}}{\pgfqpoint{2.181604in}{1.928083in}}%
\pgfpathcurveto{\pgfqpoint{2.181604in}{1.939133in}}{\pgfqpoint{2.177214in}{1.949732in}}{\pgfqpoint{2.169400in}{1.957546in}}%
\pgfpathcurveto{\pgfqpoint{2.161587in}{1.965359in}}{\pgfqpoint{2.150988in}{1.969750in}}{\pgfqpoint{2.139937in}{1.969750in}}%
\pgfpathcurveto{\pgfqpoint{2.128887in}{1.969750in}}{\pgfqpoint{2.118288in}{1.965359in}}{\pgfqpoint{2.110475in}{1.957546in}}%
\pgfpathcurveto{\pgfqpoint{2.102661in}{1.949732in}}{\pgfqpoint{2.098271in}{1.939133in}}{\pgfqpoint{2.098271in}{1.928083in}}%
\pgfpathcurveto{\pgfqpoint{2.098271in}{1.917033in}}{\pgfqpoint{2.102661in}{1.906434in}}{\pgfqpoint{2.110475in}{1.898620in}}%
\pgfpathcurveto{\pgfqpoint{2.118288in}{1.890807in}}{\pgfqpoint{2.128887in}{1.886416in}}{\pgfqpoint{2.139937in}{1.886416in}}%
\pgfpathclose%
\pgfusepath{stroke,fill}%
\end{pgfscope}%
\begin{pgfscope}%
\pgfsetbuttcap%
\pgfsetroundjoin%
\definecolor{currentfill}{rgb}{0.000000,0.000000,0.000000}%
\pgfsetfillcolor{currentfill}%
\pgfsetlinewidth{0.803000pt}%
\definecolor{currentstroke}{rgb}{0.000000,0.000000,0.000000}%
\pgfsetstrokecolor{currentstroke}%
\pgfsetdash{}{0pt}%
\pgfsys@defobject{currentmarker}{\pgfqpoint{0.000000in}{-0.048611in}}{\pgfqpoint{0.000000in}{0.000000in}}{%
\pgfpathmoveto{\pgfqpoint{0.000000in}{0.000000in}}%
\pgfpathlineto{\pgfqpoint{0.000000in}{-0.048611in}}%
\pgfusepath{stroke,fill}%
}%
\begin{pgfscope}%
\pgfsys@transformshift{0.459750in}{0.330000in}%
\pgfsys@useobject{currentmarker}{}%
\end{pgfscope}%
\end{pgfscope}%
\begin{pgfscope}%
\definecolor{textcolor}{rgb}{0.000000,0.000000,0.000000}%
\pgfsetstrokecolor{textcolor}%
\pgfsetfillcolor{textcolor}%
\pgftext[x=0.459750in,y=0.232778in,,top]{\color{textcolor}\sffamily\fontsize{10.000000}{12.000000}\selectfont 20}%
\end{pgfscope}%
\begin{pgfscope}%
\pgfsetbuttcap%
\pgfsetroundjoin%
\definecolor{currentfill}{rgb}{0.000000,0.000000,0.000000}%
\pgfsetfillcolor{currentfill}%
\pgfsetlinewidth{0.803000pt}%
\definecolor{currentstroke}{rgb}{0.000000,0.000000,0.000000}%
\pgfsetstrokecolor{currentstroke}%
\pgfsetdash{}{0pt}%
\pgfsys@defobject{currentmarker}{\pgfqpoint{0.000000in}{-0.048611in}}{\pgfqpoint{0.000000in}{0.000000in}}{%
\pgfpathmoveto{\pgfqpoint{0.000000in}{0.000000in}}%
\pgfpathlineto{\pgfqpoint{0.000000in}{-0.048611in}}%
\pgfusepath{stroke,fill}%
}%
\begin{pgfscope}%
\pgfsys@transformshift{1.019812in}{0.330000in}%
\pgfsys@useobject{currentmarker}{}%
\end{pgfscope}%
\end{pgfscope}%
\begin{pgfscope}%
\definecolor{textcolor}{rgb}{0.000000,0.000000,0.000000}%
\pgfsetstrokecolor{textcolor}%
\pgfsetfillcolor{textcolor}%
\pgftext[x=1.019812in,y=0.232778in,,top]{\color{textcolor}\sffamily\fontsize{10.000000}{12.000000}\selectfont 40}%
\end{pgfscope}%
\begin{pgfscope}%
\pgfsetbuttcap%
\pgfsetroundjoin%
\definecolor{currentfill}{rgb}{0.000000,0.000000,0.000000}%
\pgfsetfillcolor{currentfill}%
\pgfsetlinewidth{0.803000pt}%
\definecolor{currentstroke}{rgb}{0.000000,0.000000,0.000000}%
\pgfsetstrokecolor{currentstroke}%
\pgfsetdash{}{0pt}%
\pgfsys@defobject{currentmarker}{\pgfqpoint{0.000000in}{-0.048611in}}{\pgfqpoint{0.000000in}{0.000000in}}{%
\pgfpathmoveto{\pgfqpoint{0.000000in}{0.000000in}}%
\pgfpathlineto{\pgfqpoint{0.000000in}{-0.048611in}}%
\pgfusepath{stroke,fill}%
}%
\begin{pgfscope}%
\pgfsys@transformshift{1.579875in}{0.330000in}%
\pgfsys@useobject{currentmarker}{}%
\end{pgfscope}%
\end{pgfscope}%
\begin{pgfscope}%
\definecolor{textcolor}{rgb}{0.000000,0.000000,0.000000}%
\pgfsetstrokecolor{textcolor}%
\pgfsetfillcolor{textcolor}%
\pgftext[x=1.579875in,y=0.232778in,,top]{\color{textcolor}\sffamily\fontsize{10.000000}{12.000000}\selectfont 60}%
\end{pgfscope}%
\begin{pgfscope}%
\pgfsetbuttcap%
\pgfsetroundjoin%
\definecolor{currentfill}{rgb}{0.000000,0.000000,0.000000}%
\pgfsetfillcolor{currentfill}%
\pgfsetlinewidth{0.803000pt}%
\definecolor{currentstroke}{rgb}{0.000000,0.000000,0.000000}%
\pgfsetstrokecolor{currentstroke}%
\pgfsetdash{}{0pt}%
\pgfsys@defobject{currentmarker}{\pgfqpoint{0.000000in}{-0.048611in}}{\pgfqpoint{0.000000in}{0.000000in}}{%
\pgfpathmoveto{\pgfqpoint{0.000000in}{0.000000in}}%
\pgfpathlineto{\pgfqpoint{0.000000in}{-0.048611in}}%
\pgfusepath{stroke,fill}%
}%
\begin{pgfscope}%
\pgfsys@transformshift{2.139937in}{0.330000in}%
\pgfsys@useobject{currentmarker}{}%
\end{pgfscope}%
\end{pgfscope}%
\begin{pgfscope}%
\definecolor{textcolor}{rgb}{0.000000,0.000000,0.000000}%
\pgfsetstrokecolor{textcolor}%
\pgfsetfillcolor{textcolor}%
\pgftext[x=2.139937in,y=0.232778in,,top]{\color{textcolor}\sffamily\fontsize{10.000000}{12.000000}\selectfont 80}%
\end{pgfscope}%
\begin{pgfscope}%
\pgfsetbuttcap%
\pgfsetroundjoin%
\definecolor{currentfill}{rgb}{0.000000,0.000000,0.000000}%
\pgfsetfillcolor{currentfill}%
\pgfsetlinewidth{0.803000pt}%
\definecolor{currentstroke}{rgb}{0.000000,0.000000,0.000000}%
\pgfsetstrokecolor{currentstroke}%
\pgfsetdash{}{0pt}%
\pgfsys@defobject{currentmarker}{\pgfqpoint{0.000000in}{-0.048611in}}{\pgfqpoint{0.000000in}{0.000000in}}{%
\pgfpathmoveto{\pgfqpoint{0.000000in}{0.000000in}}%
\pgfpathlineto{\pgfqpoint{0.000000in}{-0.048611in}}%
\pgfusepath{stroke,fill}%
}%
\begin{pgfscope}%
\pgfsys@transformshift{2.700000in}{0.330000in}%
\pgfsys@useobject{currentmarker}{}%
\end{pgfscope}%
\end{pgfscope}%
\begin{pgfscope}%
\definecolor{textcolor}{rgb}{0.000000,0.000000,0.000000}%
\pgfsetstrokecolor{textcolor}%
\pgfsetfillcolor{textcolor}%
\pgftext[x=2.700000in,y=0.232778in,,top]{\color{textcolor}\sffamily\fontsize{10.000000}{12.000000}\selectfont 100}%
\end{pgfscope}%
\begin{pgfscope}%
\definecolor{textcolor}{rgb}{0.000000,0.000000,0.000000}%
\pgfsetstrokecolor{textcolor}%
\pgfsetfillcolor{textcolor}%
\pgftext[x=1.537500in,y=0.042809in,,top]{\color{textcolor}\sffamily\fontsize{10.000000}{12.000000}\selectfont \(\displaystyle k\)}%
\end{pgfscope}%
\begin{pgfscope}%
\pgfsetbuttcap%
\pgfsetroundjoin%
\definecolor{currentfill}{rgb}{0.000000,0.000000,0.000000}%
\pgfsetfillcolor{currentfill}%
\pgfsetlinewidth{0.803000pt}%
\definecolor{currentstroke}{rgb}{0.000000,0.000000,0.000000}%
\pgfsetstrokecolor{currentstroke}%
\pgfsetdash{}{0pt}%
\pgfsys@defobject{currentmarker}{\pgfqpoint{-0.048611in}{0.000000in}}{\pgfqpoint{0.000000in}{0.000000in}}{%
\pgfpathmoveto{\pgfqpoint{0.000000in}{0.000000in}}%
\pgfpathlineto{\pgfqpoint{-0.048611in}{0.000000in}}%
\pgfusepath{stroke,fill}%
}%
\begin{pgfscope}%
\pgfsys@transformshift{0.375000in}{0.326103in}%
\pgfsys@useobject{currentmarker}{}%
\end{pgfscope}%
\end{pgfscope}%
\begin{pgfscope}%
\definecolor{textcolor}{rgb}{0.000000,0.000000,0.000000}%
\pgfsetstrokecolor{textcolor}%
\pgfsetfillcolor{textcolor}%
\pgftext[x=0.189413in,y=0.273342in,left,base]{\color{textcolor}\sffamily\fontsize{10.000000}{12.000000}\selectfont 0}%
\end{pgfscope}%
\begin{pgfscope}%
\pgfsetbuttcap%
\pgfsetroundjoin%
\definecolor{currentfill}{rgb}{0.000000,0.000000,0.000000}%
\pgfsetfillcolor{currentfill}%
\pgfsetlinewidth{0.803000pt}%
\definecolor{currentstroke}{rgb}{0.000000,0.000000,0.000000}%
\pgfsetstrokecolor{currentstroke}%
\pgfsetdash{}{0pt}%
\pgfsys@defobject{currentmarker}{\pgfqpoint{-0.048611in}{0.000000in}}{\pgfqpoint{0.000000in}{0.000000in}}{%
\pgfpathmoveto{\pgfqpoint{0.000000in}{0.000000in}}%
\pgfpathlineto{\pgfqpoint{-0.048611in}{0.000000in}}%
\pgfusepath{stroke,fill}%
}%
\begin{pgfscope}%
\pgfsys@transformshift{0.375000in}{0.568827in}%
\pgfsys@useobject{currentmarker}{}%
\end{pgfscope}%
\end{pgfscope}%
\begin{pgfscope}%
\definecolor{textcolor}{rgb}{0.000000,0.000000,0.000000}%
\pgfsetstrokecolor{textcolor}%
\pgfsetfillcolor{textcolor}%
\pgftext[x=0.101047in,y=0.516066in,left,base]{\color{textcolor}\sffamily\fontsize{10.000000}{12.000000}\selectfont 40}%
\end{pgfscope}%
\begin{pgfscope}%
\pgfsetbuttcap%
\pgfsetroundjoin%
\definecolor{currentfill}{rgb}{0.000000,0.000000,0.000000}%
\pgfsetfillcolor{currentfill}%
\pgfsetlinewidth{0.803000pt}%
\definecolor{currentstroke}{rgb}{0.000000,0.000000,0.000000}%
\pgfsetstrokecolor{currentstroke}%
\pgfsetdash{}{0pt}%
\pgfsys@defobject{currentmarker}{\pgfqpoint{-0.048611in}{0.000000in}}{\pgfqpoint{0.000000in}{0.000000in}}{%
\pgfpathmoveto{\pgfqpoint{0.000000in}{0.000000in}}%
\pgfpathlineto{\pgfqpoint{-0.048611in}{0.000000in}}%
\pgfusepath{stroke,fill}%
}%
\begin{pgfscope}%
\pgfsys@transformshift{0.375000in}{0.811552in}%
\pgfsys@useobject{currentmarker}{}%
\end{pgfscope}%
\end{pgfscope}%
\begin{pgfscope}%
\definecolor{textcolor}{rgb}{0.000000,0.000000,0.000000}%
\pgfsetstrokecolor{textcolor}%
\pgfsetfillcolor{textcolor}%
\pgftext[x=0.101047in,y=0.758790in,left,base]{\color{textcolor}\sffamily\fontsize{10.000000}{12.000000}\selectfont 80}%
\end{pgfscope}%
\begin{pgfscope}%
\pgfsetbuttcap%
\pgfsetroundjoin%
\definecolor{currentfill}{rgb}{0.000000,0.000000,0.000000}%
\pgfsetfillcolor{currentfill}%
\pgfsetlinewidth{0.803000pt}%
\definecolor{currentstroke}{rgb}{0.000000,0.000000,0.000000}%
\pgfsetstrokecolor{currentstroke}%
\pgfsetdash{}{0pt}%
\pgfsys@defobject{currentmarker}{\pgfqpoint{-0.048611in}{0.000000in}}{\pgfqpoint{0.000000in}{0.000000in}}{%
\pgfpathmoveto{\pgfqpoint{0.000000in}{0.000000in}}%
\pgfpathlineto{\pgfqpoint{-0.048611in}{0.000000in}}%
\pgfusepath{stroke,fill}%
}%
\begin{pgfscope}%
\pgfsys@transformshift{0.375000in}{1.054276in}%
\pgfsys@useobject{currentmarker}{}%
\end{pgfscope}%
\end{pgfscope}%
\begin{pgfscope}%
\definecolor{textcolor}{rgb}{0.000000,0.000000,0.000000}%
\pgfsetstrokecolor{textcolor}%
\pgfsetfillcolor{textcolor}%
\pgftext[x=0.012682in,y=1.001514in,left,base]{\color{textcolor}\sffamily\fontsize{10.000000}{12.000000}\selectfont 120}%
\end{pgfscope}%
\begin{pgfscope}%
\pgfsetbuttcap%
\pgfsetroundjoin%
\definecolor{currentfill}{rgb}{0.000000,0.000000,0.000000}%
\pgfsetfillcolor{currentfill}%
\pgfsetlinewidth{0.803000pt}%
\definecolor{currentstroke}{rgb}{0.000000,0.000000,0.000000}%
\pgfsetstrokecolor{currentstroke}%
\pgfsetdash{}{0pt}%
\pgfsys@defobject{currentmarker}{\pgfqpoint{-0.048611in}{0.000000in}}{\pgfqpoint{0.000000in}{0.000000in}}{%
\pgfpathmoveto{\pgfqpoint{0.000000in}{0.000000in}}%
\pgfpathlineto{\pgfqpoint{-0.048611in}{0.000000in}}%
\pgfusepath{stroke,fill}%
}%
\begin{pgfscope}%
\pgfsys@transformshift{0.375000in}{1.297000in}%
\pgfsys@useobject{currentmarker}{}%
\end{pgfscope}%
\end{pgfscope}%
\begin{pgfscope}%
\definecolor{textcolor}{rgb}{0.000000,0.000000,0.000000}%
\pgfsetstrokecolor{textcolor}%
\pgfsetfillcolor{textcolor}%
\pgftext[x=0.012682in,y=1.244238in,left,base]{\color{textcolor}\sffamily\fontsize{10.000000}{12.000000}\selectfont 160}%
\end{pgfscope}%
\begin{pgfscope}%
\pgfsetbuttcap%
\pgfsetroundjoin%
\definecolor{currentfill}{rgb}{0.000000,0.000000,0.000000}%
\pgfsetfillcolor{currentfill}%
\pgfsetlinewidth{0.803000pt}%
\definecolor{currentstroke}{rgb}{0.000000,0.000000,0.000000}%
\pgfsetstrokecolor{currentstroke}%
\pgfsetdash{}{0pt}%
\pgfsys@defobject{currentmarker}{\pgfqpoint{-0.048611in}{0.000000in}}{\pgfqpoint{0.000000in}{0.000000in}}{%
\pgfpathmoveto{\pgfqpoint{0.000000in}{0.000000in}}%
\pgfpathlineto{\pgfqpoint{-0.048611in}{0.000000in}}%
\pgfusepath{stroke,fill}%
}%
\begin{pgfscope}%
\pgfsys@transformshift{0.375000in}{1.539724in}%
\pgfsys@useobject{currentmarker}{}%
\end{pgfscope}%
\end{pgfscope}%
\begin{pgfscope}%
\definecolor{textcolor}{rgb}{0.000000,0.000000,0.000000}%
\pgfsetstrokecolor{textcolor}%
\pgfsetfillcolor{textcolor}%
\pgftext[x=0.012682in,y=1.486963in,left,base]{\color{textcolor}\sffamily\fontsize{10.000000}{12.000000}\selectfont 200}%
\end{pgfscope}%
\begin{pgfscope}%
\pgfsetbuttcap%
\pgfsetroundjoin%
\definecolor{currentfill}{rgb}{0.000000,0.000000,0.000000}%
\pgfsetfillcolor{currentfill}%
\pgfsetlinewidth{0.803000pt}%
\definecolor{currentstroke}{rgb}{0.000000,0.000000,0.000000}%
\pgfsetstrokecolor{currentstroke}%
\pgfsetdash{}{0pt}%
\pgfsys@defobject{currentmarker}{\pgfqpoint{-0.048611in}{0.000000in}}{\pgfqpoint{0.000000in}{0.000000in}}{%
\pgfpathmoveto{\pgfqpoint{0.000000in}{0.000000in}}%
\pgfpathlineto{\pgfqpoint{-0.048611in}{0.000000in}}%
\pgfusepath{stroke,fill}%
}%
\begin{pgfscope}%
\pgfsys@transformshift{0.375000in}{1.782448in}%
\pgfsys@useobject{currentmarker}{}%
\end{pgfscope}%
\end{pgfscope}%
\begin{pgfscope}%
\definecolor{textcolor}{rgb}{0.000000,0.000000,0.000000}%
\pgfsetstrokecolor{textcolor}%
\pgfsetfillcolor{textcolor}%
\pgftext[x=0.012682in,y=1.729687in,left,base]{\color{textcolor}\sffamily\fontsize{10.000000}{12.000000}\selectfont 240}%
\end{pgfscope}%
\begin{pgfscope}%
\pgfsetbuttcap%
\pgfsetroundjoin%
\definecolor{currentfill}{rgb}{0.000000,0.000000,0.000000}%
\pgfsetfillcolor{currentfill}%
\pgfsetlinewidth{0.803000pt}%
\definecolor{currentstroke}{rgb}{0.000000,0.000000,0.000000}%
\pgfsetstrokecolor{currentstroke}%
\pgfsetdash{}{0pt}%
\pgfsys@defobject{currentmarker}{\pgfqpoint{-0.048611in}{0.000000in}}{\pgfqpoint{0.000000in}{0.000000in}}{%
\pgfpathmoveto{\pgfqpoint{0.000000in}{0.000000in}}%
\pgfpathlineto{\pgfqpoint{-0.048611in}{0.000000in}}%
\pgfusepath{stroke,fill}%
}%
\begin{pgfscope}%
\pgfsys@transformshift{0.375000in}{2.025173in}%
\pgfsys@useobject{currentmarker}{}%
\end{pgfscope}%
\end{pgfscope}%
\begin{pgfscope}%
\definecolor{textcolor}{rgb}{0.000000,0.000000,0.000000}%
\pgfsetstrokecolor{textcolor}%
\pgfsetfillcolor{textcolor}%
\pgftext[x=0.012682in,y=1.972411in,left,base]{\color{textcolor}\sffamily\fontsize{10.000000}{12.000000}\selectfont 280}%
\end{pgfscope}%
\begin{pgfscope}%
\pgfsetbuttcap%
\pgfsetroundjoin%
\definecolor{currentfill}{rgb}{0.000000,0.000000,0.000000}%
\pgfsetfillcolor{currentfill}%
\pgfsetlinewidth{0.803000pt}%
\definecolor{currentstroke}{rgb}{0.000000,0.000000,0.000000}%
\pgfsetstrokecolor{currentstroke}%
\pgfsetdash{}{0pt}%
\pgfsys@defobject{currentmarker}{\pgfqpoint{-0.048611in}{0.000000in}}{\pgfqpoint{0.000000in}{0.000000in}}{%
\pgfpathmoveto{\pgfqpoint{0.000000in}{0.000000in}}%
\pgfpathlineto{\pgfqpoint{-0.048611in}{0.000000in}}%
\pgfusepath{stroke,fill}%
}%
\begin{pgfscope}%
\pgfsys@transformshift{0.375000in}{2.267897in}%
\pgfsys@useobject{currentmarker}{}%
\end{pgfscope}%
\end{pgfscope}%
\begin{pgfscope}%
\definecolor{textcolor}{rgb}{0.000000,0.000000,0.000000}%
\pgfsetstrokecolor{textcolor}%
\pgfsetfillcolor{textcolor}%
\pgftext[x=0.012682in,y=2.215135in,left,base]{\color{textcolor}\sffamily\fontsize{10.000000}{12.000000}\selectfont 320}%
\end{pgfscope}%
\begin{pgfscope}%
\pgfsetbuttcap%
\pgfsetroundjoin%
\definecolor{currentfill}{rgb}{0.000000,0.000000,0.000000}%
\pgfsetfillcolor{currentfill}%
\pgfsetlinewidth{0.803000pt}%
\definecolor{currentstroke}{rgb}{0.000000,0.000000,0.000000}%
\pgfsetstrokecolor{currentstroke}%
\pgfsetdash{}{0pt}%
\pgfsys@defobject{currentmarker}{\pgfqpoint{-0.048611in}{0.000000in}}{\pgfqpoint{0.000000in}{0.000000in}}{%
\pgfpathmoveto{\pgfqpoint{0.000000in}{0.000000in}}%
\pgfpathlineto{\pgfqpoint{-0.048611in}{0.000000in}}%
\pgfusepath{stroke,fill}%
}%
\begin{pgfscope}%
\pgfsys@transformshift{0.375000in}{2.510621in}%
\pgfsys@useobject{currentmarker}{}%
\end{pgfscope}%
\end{pgfscope}%
\begin{pgfscope}%
\definecolor{textcolor}{rgb}{0.000000,0.000000,0.000000}%
\pgfsetstrokecolor{textcolor}%
\pgfsetfillcolor{textcolor}%
\pgftext[x=0.012682in,y=2.457860in,left,base]{\color{textcolor}\sffamily\fontsize{10.000000}{12.000000}\selectfont 360}%
\end{pgfscope}%
\begin{pgfscope}%
\definecolor{textcolor}{rgb}{0.000000,0.000000,0.000000}%
\pgfsetstrokecolor{textcolor}%
\pgfsetfillcolor{textcolor}%
\pgftext[x=-0.042874in,y=1.485000in,,bottom,rotate=90.000000]{\color{textcolor}\sffamily\fontsize{10.000000}{12.000000}\selectfont Number of GMRES Iterations}%
\end{pgfscope}%
\begin{pgfscope}%
\pgfsetrectcap%
\pgfsetmiterjoin%
\pgfsetlinewidth{0.803000pt}%
\definecolor{currentstroke}{rgb}{0.000000,0.000000,0.000000}%
\pgfsetstrokecolor{currentstroke}%
\pgfsetdash{}{0pt}%
\pgfpathmoveto{\pgfqpoint{0.375000in}{0.330000in}}%
\pgfpathlineto{\pgfqpoint{0.375000in}{2.640000in}}%
\pgfusepath{stroke}%
\end{pgfscope}%
\begin{pgfscope}%
\pgfsetrectcap%
\pgfsetmiterjoin%
\pgfsetlinewidth{0.803000pt}%
\definecolor{currentstroke}{rgb}{0.000000,0.000000,0.000000}%
\pgfsetstrokecolor{currentstroke}%
\pgfsetdash{}{0pt}%
\pgfpathmoveto{\pgfqpoint{2.700000in}{0.330000in}}%
\pgfpathlineto{\pgfqpoint{2.700000in}{2.640000in}}%
\pgfusepath{stroke}%
\end{pgfscope}%
\begin{pgfscope}%
\pgfsetrectcap%
\pgfsetmiterjoin%
\pgfsetlinewidth{0.803000pt}%
\definecolor{currentstroke}{rgb}{0.000000,0.000000,0.000000}%
\pgfsetstrokecolor{currentstroke}%
\pgfsetdash{}{0pt}%
\pgfpathmoveto{\pgfqpoint{0.375000in}{0.330000in}}%
\pgfpathlineto{\pgfqpoint{2.700000in}{0.330000in}}%
\pgfusepath{stroke}%
\end{pgfscope}%
\begin{pgfscope}%
\pgfsetrectcap%
\pgfsetmiterjoin%
\pgfsetlinewidth{0.803000pt}%
\definecolor{currentstroke}{rgb}{0.000000,0.000000,0.000000}%
\pgfsetstrokecolor{currentstroke}%
\pgfsetdash{}{0pt}%
\pgfpathmoveto{\pgfqpoint{0.375000in}{2.640000in}}%
\pgfpathlineto{\pgfqpoint{2.700000in}{2.640000in}}%
\pgfusepath{stroke}%
\end{pgfscope}%
\end{pgfpicture}%
\makeatother%
\endgroup%

    \caption[Maximum GMRES iteration counts when $\NLiDRRdtd{\Aso-\Ast} = 0.5\times  k^{-\beta}$ for $\beta = 0.8,0.9,1.$]{Maximum GMRES iteration counts for solving systems with matrix $\AmatoI\Amatt$, where $\nso=\nst=1$ and $\NLiDRRdtd{\Aso-\Ast} = 0.5\times  k^{-\beta}$ for $\beta = 0.8,0.9,1.$}\label{fig:linfinityA2}
\end{figure}

    \begin{figure}
      \centering
%% Creator: Matplotlib, PGF backend
%%
%% To include the figure in your LaTeX document, write
%%   \input{<filename>.pgf}
%%
%% Make sure the required packages are loaded in your preamble
%%   \usepackage{pgf}
%%
%% Figures using additional raster images can only be included by \input if
%% they are in the same directory as the main LaTeX file. For loading figures
%% from other directories you can use the `import` package
%%   \usepackage{import}
%% and then include the figures with
%%   \import{<path to file>}{<filename>.pgf}
%%
%% Matplotlib used the following preamble
%%   \usepackage{fontspec}
%%   \setmainfont{DejaVuSerif.ttf}[Path=/home/owen/progs/firedrake-complex/firedrake/lib/python3.5/site-packages/matplotlib/mpl-data/fonts/ttf/]
%%   \setsansfont{DejaVuSans.ttf}[Path=/home/owen/progs/firedrake-complex/firedrake/lib/python3.5/site-packages/matplotlib/mpl-data/fonts/ttf/]
%%   \setmonofont{DejaVuSansMono.ttf}[Path=/home/owen/progs/firedrake-complex/firedrake/lib/python3.5/site-packages/matplotlib/mpl-data/fonts/ttf/]
%%
\begingroup%
\makeatletter%
\begin{pgfpicture}%
\pgfpathrectangle{\pgfpointorigin}{\pgfqpoint{5.000000in}{2.500000in}}%
\pgfusepath{use as bounding box, clip}%
\begin{pgfscope}%
\pgfsetbuttcap%
\pgfsetmiterjoin%
\definecolor{currentfill}{rgb}{1.000000,1.000000,1.000000}%
\pgfsetfillcolor{currentfill}%
\pgfsetlinewidth{0.000000pt}%
\definecolor{currentstroke}{rgb}{1.000000,1.000000,1.000000}%
\pgfsetstrokecolor{currentstroke}%
\pgfsetdash{}{0pt}%
\pgfpathmoveto{\pgfqpoint{0.000000in}{0.000000in}}%
\pgfpathlineto{\pgfqpoint{5.000000in}{0.000000in}}%
\pgfpathlineto{\pgfqpoint{5.000000in}{2.500000in}}%
\pgfpathlineto{\pgfqpoint{0.000000in}{2.500000in}}%
\pgfpathclose%
\pgfusepath{fill}%
\end{pgfscope}%
\begin{pgfscope}%
\pgfsetbuttcap%
\pgfsetmiterjoin%
\definecolor{currentfill}{rgb}{1.000000,1.000000,1.000000}%
\pgfsetfillcolor{currentfill}%
\pgfsetlinewidth{0.000000pt}%
\definecolor{currentstroke}{rgb}{0.000000,0.000000,0.000000}%
\pgfsetstrokecolor{currentstroke}%
\pgfsetstrokeopacity{0.000000}%
\pgfsetdash{}{0pt}%
\pgfpathmoveto{\pgfqpoint{0.625000in}{0.275000in}}%
\pgfpathlineto{\pgfqpoint{4.500000in}{0.275000in}}%
\pgfpathlineto{\pgfqpoint{4.500000in}{2.200000in}}%
\pgfpathlineto{\pgfqpoint{0.625000in}{2.200000in}}%
\pgfpathclose%
\pgfusepath{fill}%
\end{pgfscope}%
\begin{pgfscope}%
\pgfsetbuttcap%
\pgfsetroundjoin%
\definecolor{currentfill}{rgb}{0.000000,0.000000,0.000000}%
\pgfsetfillcolor{currentfill}%
\pgfsetlinewidth{0.803000pt}%
\definecolor{currentstroke}{rgb}{0.000000,0.000000,0.000000}%
\pgfsetstrokecolor{currentstroke}%
\pgfsetdash{}{0pt}%
\pgfsys@defobject{currentmarker}{\pgfqpoint{0.000000in}{-0.048611in}}{\pgfqpoint{0.000000in}{0.000000in}}{%
\pgfpathmoveto{\pgfqpoint{0.000000in}{0.000000in}}%
\pgfpathlineto{\pgfqpoint{0.000000in}{-0.048611in}}%
\pgfusepath{stroke,fill}%
}%
\begin{pgfscope}%
\pgfsys@transformshift{0.801136in}{0.275000in}%
\pgfsys@useobject{currentmarker}{}%
\end{pgfscope}%
\end{pgfscope}%
\begin{pgfscope}%
\definecolor{textcolor}{rgb}{0.000000,0.000000,0.000000}%
\pgfsetstrokecolor{textcolor}%
\pgfsetfillcolor{textcolor}%
\pgftext[x=0.801136in,y=0.177778in,,top]{\color{textcolor}\sffamily\fontsize{10.000000}{12.000000}\selectfont 20}%
\end{pgfscope}%
\begin{pgfscope}%
\pgfsetbuttcap%
\pgfsetroundjoin%
\definecolor{currentfill}{rgb}{0.000000,0.000000,0.000000}%
\pgfsetfillcolor{currentfill}%
\pgfsetlinewidth{0.803000pt}%
\definecolor{currentstroke}{rgb}{0.000000,0.000000,0.000000}%
\pgfsetstrokecolor{currentstroke}%
\pgfsetdash{}{0pt}%
\pgfsys@defobject{currentmarker}{\pgfqpoint{0.000000in}{-0.048611in}}{\pgfqpoint{0.000000in}{0.000000in}}{%
\pgfpathmoveto{\pgfqpoint{0.000000in}{0.000000in}}%
\pgfpathlineto{\pgfqpoint{0.000000in}{-0.048611in}}%
\pgfusepath{stroke,fill}%
}%
\begin{pgfscope}%
\pgfsys@transformshift{1.975379in}{0.275000in}%
\pgfsys@useobject{currentmarker}{}%
\end{pgfscope}%
\end{pgfscope}%
\begin{pgfscope}%
\definecolor{textcolor}{rgb}{0.000000,0.000000,0.000000}%
\pgfsetstrokecolor{textcolor}%
\pgfsetfillcolor{textcolor}%
\pgftext[x=1.975379in,y=0.177778in,,top]{\color{textcolor}\sffamily\fontsize{10.000000}{12.000000}\selectfont 40}%
\end{pgfscope}%
\begin{pgfscope}%
\pgfsetbuttcap%
\pgfsetroundjoin%
\definecolor{currentfill}{rgb}{0.000000,0.000000,0.000000}%
\pgfsetfillcolor{currentfill}%
\pgfsetlinewidth{0.803000pt}%
\definecolor{currentstroke}{rgb}{0.000000,0.000000,0.000000}%
\pgfsetstrokecolor{currentstroke}%
\pgfsetdash{}{0pt}%
\pgfsys@defobject{currentmarker}{\pgfqpoint{0.000000in}{-0.048611in}}{\pgfqpoint{0.000000in}{0.000000in}}{%
\pgfpathmoveto{\pgfqpoint{0.000000in}{0.000000in}}%
\pgfpathlineto{\pgfqpoint{0.000000in}{-0.048611in}}%
\pgfusepath{stroke,fill}%
}%
\begin{pgfscope}%
\pgfsys@transformshift{3.149621in}{0.275000in}%
\pgfsys@useobject{currentmarker}{}%
\end{pgfscope}%
\end{pgfscope}%
\begin{pgfscope}%
\definecolor{textcolor}{rgb}{0.000000,0.000000,0.000000}%
\pgfsetstrokecolor{textcolor}%
\pgfsetfillcolor{textcolor}%
\pgftext[x=3.149621in,y=0.177778in,,top]{\color{textcolor}\sffamily\fontsize{10.000000}{12.000000}\selectfont 60}%
\end{pgfscope}%
\begin{pgfscope}%
\pgfsetbuttcap%
\pgfsetroundjoin%
\definecolor{currentfill}{rgb}{0.000000,0.000000,0.000000}%
\pgfsetfillcolor{currentfill}%
\pgfsetlinewidth{0.803000pt}%
\definecolor{currentstroke}{rgb}{0.000000,0.000000,0.000000}%
\pgfsetstrokecolor{currentstroke}%
\pgfsetdash{}{0pt}%
\pgfsys@defobject{currentmarker}{\pgfqpoint{0.000000in}{-0.048611in}}{\pgfqpoint{0.000000in}{0.000000in}}{%
\pgfpathmoveto{\pgfqpoint{0.000000in}{0.000000in}}%
\pgfpathlineto{\pgfqpoint{0.000000in}{-0.048611in}}%
\pgfusepath{stroke,fill}%
}%
\begin{pgfscope}%
\pgfsys@transformshift{4.323864in}{0.275000in}%
\pgfsys@useobject{currentmarker}{}%
\end{pgfscope}%
\end{pgfscope}%
\begin{pgfscope}%
\definecolor{textcolor}{rgb}{0.000000,0.000000,0.000000}%
\pgfsetstrokecolor{textcolor}%
\pgfsetfillcolor{textcolor}%
\pgftext[x=4.323864in,y=0.177778in,,top]{\color{textcolor}\sffamily\fontsize{10.000000}{12.000000}\selectfont 80}%
\end{pgfscope}%
\begin{pgfscope}%
\definecolor{textcolor}{rgb}{0.000000,0.000000,0.000000}%
\pgfsetstrokecolor{textcolor}%
\pgfsetfillcolor{textcolor}%
\pgftext[x=2.562500in,y=-0.012191in,,top]{\color{textcolor}\sffamily\fontsize{10.000000}{12.000000}\selectfont \(\displaystyle k\)}%
\end{pgfscope}%
\begin{pgfscope}%
\pgfsetbuttcap%
\pgfsetroundjoin%
\definecolor{currentfill}{rgb}{0.000000,0.000000,0.000000}%
\pgfsetfillcolor{currentfill}%
\pgfsetlinewidth{0.803000pt}%
\definecolor{currentstroke}{rgb}{0.000000,0.000000,0.000000}%
\pgfsetstrokecolor{currentstroke}%
\pgfsetdash{}{0pt}%
\pgfsys@defobject{currentmarker}{\pgfqpoint{-0.048611in}{0.000000in}}{\pgfqpoint{0.000000in}{0.000000in}}{%
\pgfpathmoveto{\pgfqpoint{0.000000in}{0.000000in}}%
\pgfpathlineto{\pgfqpoint{-0.048611in}{0.000000in}}%
\pgfusepath{stroke,fill}%
}%
\begin{pgfscope}%
\pgfsys@transformshift{0.625000in}{0.467500in}%
\pgfsys@useobject{currentmarker}{}%
\end{pgfscope}%
\end{pgfscope}%
\begin{pgfscope}%
\definecolor{textcolor}{rgb}{0.000000,0.000000,0.000000}%
\pgfsetstrokecolor{textcolor}%
\pgfsetfillcolor{textcolor}%
\pgftext[x=0.351047in,y=0.414738in,left,base]{\color{textcolor}\sffamily\fontsize{10.000000}{12.000000}\selectfont 20}%
\end{pgfscope}%
\begin{pgfscope}%
\pgfsetbuttcap%
\pgfsetroundjoin%
\definecolor{currentfill}{rgb}{0.000000,0.000000,0.000000}%
\pgfsetfillcolor{currentfill}%
\pgfsetlinewidth{0.803000pt}%
\definecolor{currentstroke}{rgb}{0.000000,0.000000,0.000000}%
\pgfsetstrokecolor{currentstroke}%
\pgfsetdash{}{0pt}%
\pgfsys@defobject{currentmarker}{\pgfqpoint{-0.048611in}{0.000000in}}{\pgfqpoint{0.000000in}{0.000000in}}{%
\pgfpathmoveto{\pgfqpoint{0.000000in}{0.000000in}}%
\pgfpathlineto{\pgfqpoint{-0.048611in}{0.000000in}}%
\pgfusepath{stroke,fill}%
}%
\begin{pgfscope}%
\pgfsys@transformshift{0.625000in}{0.700833in}%
\pgfsys@useobject{currentmarker}{}%
\end{pgfscope}%
\end{pgfscope}%
\begin{pgfscope}%
\definecolor{textcolor}{rgb}{0.000000,0.000000,0.000000}%
\pgfsetstrokecolor{textcolor}%
\pgfsetfillcolor{textcolor}%
\pgftext[x=0.351047in,y=0.648072in,left,base]{\color{textcolor}\sffamily\fontsize{10.000000}{12.000000}\selectfont 40}%
\end{pgfscope}%
\begin{pgfscope}%
\pgfsetbuttcap%
\pgfsetroundjoin%
\definecolor{currentfill}{rgb}{0.000000,0.000000,0.000000}%
\pgfsetfillcolor{currentfill}%
\pgfsetlinewidth{0.803000pt}%
\definecolor{currentstroke}{rgb}{0.000000,0.000000,0.000000}%
\pgfsetstrokecolor{currentstroke}%
\pgfsetdash{}{0pt}%
\pgfsys@defobject{currentmarker}{\pgfqpoint{-0.048611in}{0.000000in}}{\pgfqpoint{0.000000in}{0.000000in}}{%
\pgfpathmoveto{\pgfqpoint{0.000000in}{0.000000in}}%
\pgfpathlineto{\pgfqpoint{-0.048611in}{0.000000in}}%
\pgfusepath{stroke,fill}%
}%
\begin{pgfscope}%
\pgfsys@transformshift{0.625000in}{0.934167in}%
\pgfsys@useobject{currentmarker}{}%
\end{pgfscope}%
\end{pgfscope}%
\begin{pgfscope}%
\definecolor{textcolor}{rgb}{0.000000,0.000000,0.000000}%
\pgfsetstrokecolor{textcolor}%
\pgfsetfillcolor{textcolor}%
\pgftext[x=0.351047in,y=0.881405in,left,base]{\color{textcolor}\sffamily\fontsize{10.000000}{12.000000}\selectfont 60}%
\end{pgfscope}%
\begin{pgfscope}%
\pgfsetbuttcap%
\pgfsetroundjoin%
\definecolor{currentfill}{rgb}{0.000000,0.000000,0.000000}%
\pgfsetfillcolor{currentfill}%
\pgfsetlinewidth{0.803000pt}%
\definecolor{currentstroke}{rgb}{0.000000,0.000000,0.000000}%
\pgfsetstrokecolor{currentstroke}%
\pgfsetdash{}{0pt}%
\pgfsys@defobject{currentmarker}{\pgfqpoint{-0.048611in}{0.000000in}}{\pgfqpoint{0.000000in}{0.000000in}}{%
\pgfpathmoveto{\pgfqpoint{0.000000in}{0.000000in}}%
\pgfpathlineto{\pgfqpoint{-0.048611in}{0.000000in}}%
\pgfusepath{stroke,fill}%
}%
\begin{pgfscope}%
\pgfsys@transformshift{0.625000in}{1.167500in}%
\pgfsys@useobject{currentmarker}{}%
\end{pgfscope}%
\end{pgfscope}%
\begin{pgfscope}%
\definecolor{textcolor}{rgb}{0.000000,0.000000,0.000000}%
\pgfsetstrokecolor{textcolor}%
\pgfsetfillcolor{textcolor}%
\pgftext[x=0.351047in,y=1.114738in,left,base]{\color{textcolor}\sffamily\fontsize{10.000000}{12.000000}\selectfont 80}%
\end{pgfscope}%
\begin{pgfscope}%
\pgfsetbuttcap%
\pgfsetroundjoin%
\definecolor{currentfill}{rgb}{0.000000,0.000000,0.000000}%
\pgfsetfillcolor{currentfill}%
\pgfsetlinewidth{0.803000pt}%
\definecolor{currentstroke}{rgb}{0.000000,0.000000,0.000000}%
\pgfsetstrokecolor{currentstroke}%
\pgfsetdash{}{0pt}%
\pgfsys@defobject{currentmarker}{\pgfqpoint{-0.048611in}{0.000000in}}{\pgfqpoint{0.000000in}{0.000000in}}{%
\pgfpathmoveto{\pgfqpoint{0.000000in}{0.000000in}}%
\pgfpathlineto{\pgfqpoint{-0.048611in}{0.000000in}}%
\pgfusepath{stroke,fill}%
}%
\begin{pgfscope}%
\pgfsys@transformshift{0.625000in}{1.400833in}%
\pgfsys@useobject{currentmarker}{}%
\end{pgfscope}%
\end{pgfscope}%
\begin{pgfscope}%
\definecolor{textcolor}{rgb}{0.000000,0.000000,0.000000}%
\pgfsetstrokecolor{textcolor}%
\pgfsetfillcolor{textcolor}%
\pgftext[x=0.262682in,y=1.348072in,left,base]{\color{textcolor}\sffamily\fontsize{10.000000}{12.000000}\selectfont 100}%
\end{pgfscope}%
\begin{pgfscope}%
\pgfsetbuttcap%
\pgfsetroundjoin%
\definecolor{currentfill}{rgb}{0.000000,0.000000,0.000000}%
\pgfsetfillcolor{currentfill}%
\pgfsetlinewidth{0.803000pt}%
\definecolor{currentstroke}{rgb}{0.000000,0.000000,0.000000}%
\pgfsetstrokecolor{currentstroke}%
\pgfsetdash{}{0pt}%
\pgfsys@defobject{currentmarker}{\pgfqpoint{-0.048611in}{0.000000in}}{\pgfqpoint{0.000000in}{0.000000in}}{%
\pgfpathmoveto{\pgfqpoint{0.000000in}{0.000000in}}%
\pgfpathlineto{\pgfqpoint{-0.048611in}{0.000000in}}%
\pgfusepath{stroke,fill}%
}%
\begin{pgfscope}%
\pgfsys@transformshift{0.625000in}{1.634167in}%
\pgfsys@useobject{currentmarker}{}%
\end{pgfscope}%
\end{pgfscope}%
\begin{pgfscope}%
\definecolor{textcolor}{rgb}{0.000000,0.000000,0.000000}%
\pgfsetstrokecolor{textcolor}%
\pgfsetfillcolor{textcolor}%
\pgftext[x=0.262682in,y=1.581405in,left,base]{\color{textcolor}\sffamily\fontsize{10.000000}{12.000000}\selectfont 120}%
\end{pgfscope}%
\begin{pgfscope}%
\pgfsetbuttcap%
\pgfsetroundjoin%
\definecolor{currentfill}{rgb}{0.000000,0.000000,0.000000}%
\pgfsetfillcolor{currentfill}%
\pgfsetlinewidth{0.803000pt}%
\definecolor{currentstroke}{rgb}{0.000000,0.000000,0.000000}%
\pgfsetstrokecolor{currentstroke}%
\pgfsetdash{}{0pt}%
\pgfsys@defobject{currentmarker}{\pgfqpoint{-0.048611in}{0.000000in}}{\pgfqpoint{0.000000in}{0.000000in}}{%
\pgfpathmoveto{\pgfqpoint{0.000000in}{0.000000in}}%
\pgfpathlineto{\pgfqpoint{-0.048611in}{0.000000in}}%
\pgfusepath{stroke,fill}%
}%
\begin{pgfscope}%
\pgfsys@transformshift{0.625000in}{1.867500in}%
\pgfsys@useobject{currentmarker}{}%
\end{pgfscope}%
\end{pgfscope}%
\begin{pgfscope}%
\definecolor{textcolor}{rgb}{0.000000,0.000000,0.000000}%
\pgfsetstrokecolor{textcolor}%
\pgfsetfillcolor{textcolor}%
\pgftext[x=0.262682in,y=1.814738in,left,base]{\color{textcolor}\sffamily\fontsize{10.000000}{12.000000}\selectfont 140}%
\end{pgfscope}%
\begin{pgfscope}%
\pgfsetbuttcap%
\pgfsetroundjoin%
\definecolor{currentfill}{rgb}{0.000000,0.000000,0.000000}%
\pgfsetfillcolor{currentfill}%
\pgfsetlinewidth{0.803000pt}%
\definecolor{currentstroke}{rgb}{0.000000,0.000000,0.000000}%
\pgfsetstrokecolor{currentstroke}%
\pgfsetdash{}{0pt}%
\pgfsys@defobject{currentmarker}{\pgfqpoint{-0.048611in}{0.000000in}}{\pgfqpoint{0.000000in}{0.000000in}}{%
\pgfpathmoveto{\pgfqpoint{0.000000in}{0.000000in}}%
\pgfpathlineto{\pgfqpoint{-0.048611in}{0.000000in}}%
\pgfusepath{stroke,fill}%
}%
\begin{pgfscope}%
\pgfsys@transformshift{0.625000in}{2.100833in}%
\pgfsys@useobject{currentmarker}{}%
\end{pgfscope}%
\end{pgfscope}%
\begin{pgfscope}%
\definecolor{textcolor}{rgb}{0.000000,0.000000,0.000000}%
\pgfsetstrokecolor{textcolor}%
\pgfsetfillcolor{textcolor}%
\pgftext[x=0.262682in,y=2.048072in,left,base]{\color{textcolor}\sffamily\fontsize{10.000000}{12.000000}\selectfont 160}%
\end{pgfscope}%
\begin{pgfscope}%
\definecolor{textcolor}{rgb}{0.000000,0.000000,0.000000}%
\pgfsetstrokecolor{textcolor}%
\pgfsetfillcolor{textcolor}%
\pgftext[x=0.207126in,y=1.237500in,,bottom,rotate=90.000000]{\color{textcolor}\sffamily\fontsize{10.000000}{12.000000}\selectfont Number of GMRES Iterations}%
\end{pgfscope}%
\begin{pgfscope}%
\pgfpathrectangle{\pgfqpoint{0.625000in}{0.275000in}}{\pgfqpoint{3.875000in}{1.925000in}}%
\pgfusepath{clip}%
\pgfsetbuttcap%
\pgfsetroundjoin%
\definecolor{currentfill}{rgb}{0.000000,0.000000,0.000000}%
\pgfsetfillcolor{currentfill}%
\pgfsetlinewidth{1.003750pt}%
\definecolor{currentstroke}{rgb}{0.000000,0.000000,0.000000}%
\pgfsetstrokecolor{currentstroke}%
\pgfsetdash{}{0pt}%
\pgfsys@defobject{currentmarker}{\pgfqpoint{-0.041667in}{-0.041667in}}{\pgfqpoint{0.041667in}{0.041667in}}{%
\pgfpathmoveto{\pgfqpoint{0.000000in}{-0.041667in}}%
\pgfpathcurveto{\pgfqpoint{0.011050in}{-0.041667in}}{\pgfqpoint{0.021649in}{-0.037276in}}{\pgfqpoint{0.029463in}{-0.029463in}}%
\pgfpathcurveto{\pgfqpoint{0.037276in}{-0.021649in}}{\pgfqpoint{0.041667in}{-0.011050in}}{\pgfqpoint{0.041667in}{0.000000in}}%
\pgfpathcurveto{\pgfqpoint{0.041667in}{0.011050in}}{\pgfqpoint{0.037276in}{0.021649in}}{\pgfqpoint{0.029463in}{0.029463in}}%
\pgfpathcurveto{\pgfqpoint{0.021649in}{0.037276in}}{\pgfqpoint{0.011050in}{0.041667in}}{\pgfqpoint{0.000000in}{0.041667in}}%
\pgfpathcurveto{\pgfqpoint{-0.011050in}{0.041667in}}{\pgfqpoint{-0.021649in}{0.037276in}}{\pgfqpoint{-0.029463in}{0.029463in}}%
\pgfpathcurveto{\pgfqpoint{-0.037276in}{0.021649in}}{\pgfqpoint{-0.041667in}{0.011050in}}{\pgfqpoint{-0.041667in}{0.000000in}}%
\pgfpathcurveto{\pgfqpoint{-0.041667in}{-0.011050in}}{\pgfqpoint{-0.037276in}{-0.021649in}}{\pgfqpoint{-0.029463in}{-0.029463in}}%
\pgfpathcurveto{\pgfqpoint{-0.021649in}{-0.037276in}}{\pgfqpoint{-0.011050in}{-0.041667in}}{\pgfqpoint{0.000000in}{-0.041667in}}%
\pgfpathclose%
\pgfusepath{stroke,fill}%
}%
\begin{pgfscope}%
\pgfsys@transformshift{0.801136in}{0.479167in}%
\pgfsys@useobject{currentmarker}{}%
\end{pgfscope}%
\begin{pgfscope}%
\pgfsys@transformshift{0.801136in}{0.490833in}%
\pgfsys@useobject{currentmarker}{}%
\end{pgfscope}%
\begin{pgfscope}%
\pgfsys@transformshift{0.801136in}{0.502500in}%
\pgfsys@useobject{currentmarker}{}%
\end{pgfscope}%
\begin{pgfscope}%
\pgfsys@transformshift{0.801136in}{0.514167in}%
\pgfsys@useobject{currentmarker}{}%
\end{pgfscope}%
\begin{pgfscope}%
\pgfsys@transformshift{0.801136in}{0.537500in}%
\pgfsys@useobject{currentmarker}{}%
\end{pgfscope}%
\end{pgfscope}%
\begin{pgfscope}%
\pgfpathrectangle{\pgfqpoint{0.625000in}{0.275000in}}{\pgfqpoint{3.875000in}{1.925000in}}%
\pgfusepath{clip}%
\pgfsetbuttcap%
\pgfsetroundjoin%
\definecolor{currentfill}{rgb}{0.000000,0.000000,0.000000}%
\pgfsetfillcolor{currentfill}%
\pgfsetlinewidth{1.003750pt}%
\definecolor{currentstroke}{rgb}{0.000000,0.000000,0.000000}%
\pgfsetstrokecolor{currentstroke}%
\pgfsetdash{}{0pt}%
\pgfsys@defobject{currentmarker}{\pgfqpoint{-0.041667in}{-0.041667in}}{\pgfqpoint{0.041667in}{0.041667in}}{%
\pgfpathmoveto{\pgfqpoint{0.000000in}{-0.041667in}}%
\pgfpathcurveto{\pgfqpoint{0.011050in}{-0.041667in}}{\pgfqpoint{0.021649in}{-0.037276in}}{\pgfqpoint{0.029463in}{-0.029463in}}%
\pgfpathcurveto{\pgfqpoint{0.037276in}{-0.021649in}}{\pgfqpoint{0.041667in}{-0.011050in}}{\pgfqpoint{0.041667in}{0.000000in}}%
\pgfpathcurveto{\pgfqpoint{0.041667in}{0.011050in}}{\pgfqpoint{0.037276in}{0.021649in}}{\pgfqpoint{0.029463in}{0.029463in}}%
\pgfpathcurveto{\pgfqpoint{0.021649in}{0.037276in}}{\pgfqpoint{0.011050in}{0.041667in}}{\pgfqpoint{0.000000in}{0.041667in}}%
\pgfpathcurveto{\pgfqpoint{-0.011050in}{0.041667in}}{\pgfqpoint{-0.021649in}{0.037276in}}{\pgfqpoint{-0.029463in}{0.029463in}}%
\pgfpathcurveto{\pgfqpoint{-0.037276in}{0.021649in}}{\pgfqpoint{-0.041667in}{0.011050in}}{\pgfqpoint{-0.041667in}{0.000000in}}%
\pgfpathcurveto{\pgfqpoint{-0.041667in}{-0.011050in}}{\pgfqpoint{-0.037276in}{-0.021649in}}{\pgfqpoint{-0.029463in}{-0.029463in}}%
\pgfpathcurveto{\pgfqpoint{-0.021649in}{-0.037276in}}{\pgfqpoint{-0.011050in}{-0.041667in}}{\pgfqpoint{0.000000in}{-0.041667in}}%
\pgfpathclose%
\pgfusepath{stroke,fill}%
}%
\begin{pgfscope}%
\pgfsys@transformshift{1.975379in}{0.677500in}%
\pgfsys@useobject{currentmarker}{}%
\end{pgfscope}%
\begin{pgfscope}%
\pgfsys@transformshift{1.975379in}{0.689167in}%
\pgfsys@useobject{currentmarker}{}%
\end{pgfscope}%
\begin{pgfscope}%
\pgfsys@transformshift{1.975379in}{0.700833in}%
\pgfsys@useobject{currentmarker}{}%
\end{pgfscope}%
\begin{pgfscope}%
\pgfsys@transformshift{1.975379in}{0.712500in}%
\pgfsys@useobject{currentmarker}{}%
\end{pgfscope}%
\begin{pgfscope}%
\pgfsys@transformshift{1.975379in}{0.724167in}%
\pgfsys@useobject{currentmarker}{}%
\end{pgfscope}%
\begin{pgfscope}%
\pgfsys@transformshift{1.975379in}{0.735833in}%
\pgfsys@useobject{currentmarker}{}%
\end{pgfscope}%
\begin{pgfscope}%
\pgfsys@transformshift{1.975379in}{0.747500in}%
\pgfsys@useobject{currentmarker}{}%
\end{pgfscope}%
\begin{pgfscope}%
\pgfsys@transformshift{1.975379in}{0.759167in}%
\pgfsys@useobject{currentmarker}{}%
\end{pgfscope}%
\begin{pgfscope}%
\pgfsys@transformshift{1.975379in}{0.770833in}%
\pgfsys@useobject{currentmarker}{}%
\end{pgfscope}%
\begin{pgfscope}%
\pgfsys@transformshift{1.975379in}{0.782500in}%
\pgfsys@useobject{currentmarker}{}%
\end{pgfscope}%
\begin{pgfscope}%
\pgfsys@transformshift{1.975379in}{0.794167in}%
\pgfsys@useobject{currentmarker}{}%
\end{pgfscope}%
\begin{pgfscope}%
\pgfsys@transformshift{1.975379in}{0.805833in}%
\pgfsys@useobject{currentmarker}{}%
\end{pgfscope}%
\begin{pgfscope}%
\pgfsys@transformshift{1.975379in}{0.817500in}%
\pgfsys@useobject{currentmarker}{}%
\end{pgfscope}%
\begin{pgfscope}%
\pgfsys@transformshift{1.975379in}{0.829167in}%
\pgfsys@useobject{currentmarker}{}%
\end{pgfscope}%
\end{pgfscope}%
\begin{pgfscope}%
\pgfpathrectangle{\pgfqpoint{0.625000in}{0.275000in}}{\pgfqpoint{3.875000in}{1.925000in}}%
\pgfusepath{clip}%
\pgfsetbuttcap%
\pgfsetroundjoin%
\definecolor{currentfill}{rgb}{0.000000,0.000000,0.000000}%
\pgfsetfillcolor{currentfill}%
\pgfsetlinewidth{1.003750pt}%
\definecolor{currentstroke}{rgb}{0.000000,0.000000,0.000000}%
\pgfsetstrokecolor{currentstroke}%
\pgfsetdash{}{0pt}%
\pgfsys@defobject{currentmarker}{\pgfqpoint{-0.041667in}{-0.041667in}}{\pgfqpoint{0.041667in}{0.041667in}}{%
\pgfpathmoveto{\pgfqpoint{0.000000in}{-0.041667in}}%
\pgfpathcurveto{\pgfqpoint{0.011050in}{-0.041667in}}{\pgfqpoint{0.021649in}{-0.037276in}}{\pgfqpoint{0.029463in}{-0.029463in}}%
\pgfpathcurveto{\pgfqpoint{0.037276in}{-0.021649in}}{\pgfqpoint{0.041667in}{-0.011050in}}{\pgfqpoint{0.041667in}{0.000000in}}%
\pgfpathcurveto{\pgfqpoint{0.041667in}{0.011050in}}{\pgfqpoint{0.037276in}{0.021649in}}{\pgfqpoint{0.029463in}{0.029463in}}%
\pgfpathcurveto{\pgfqpoint{0.021649in}{0.037276in}}{\pgfqpoint{0.011050in}{0.041667in}}{\pgfqpoint{0.000000in}{0.041667in}}%
\pgfpathcurveto{\pgfqpoint{-0.011050in}{0.041667in}}{\pgfqpoint{-0.021649in}{0.037276in}}{\pgfqpoint{-0.029463in}{0.029463in}}%
\pgfpathcurveto{\pgfqpoint{-0.037276in}{0.021649in}}{\pgfqpoint{-0.041667in}{0.011050in}}{\pgfqpoint{-0.041667in}{0.000000in}}%
\pgfpathcurveto{\pgfqpoint{-0.041667in}{-0.011050in}}{\pgfqpoint{-0.037276in}{-0.021649in}}{\pgfqpoint{-0.029463in}{-0.029463in}}%
\pgfpathcurveto{\pgfqpoint{-0.021649in}{-0.037276in}}{\pgfqpoint{-0.011050in}{-0.041667in}}{\pgfqpoint{0.000000in}{-0.041667in}}%
\pgfpathclose%
\pgfusepath{stroke,fill}%
}%
\begin{pgfscope}%
\pgfsys@transformshift{3.149621in}{1.190833in}%
\pgfsys@useobject{currentmarker}{}%
\end{pgfscope}%
\begin{pgfscope}%
\pgfsys@transformshift{3.149621in}{1.202500in}%
\pgfsys@useobject{currentmarker}{}%
\end{pgfscope}%
\begin{pgfscope}%
\pgfsys@transformshift{3.149621in}{1.214167in}%
\pgfsys@useobject{currentmarker}{}%
\end{pgfscope}%
\begin{pgfscope}%
\pgfsys@transformshift{3.149621in}{1.225833in}%
\pgfsys@useobject{currentmarker}{}%
\end{pgfscope}%
\begin{pgfscope}%
\pgfsys@transformshift{3.149621in}{1.237500in}%
\pgfsys@useobject{currentmarker}{}%
\end{pgfscope}%
\begin{pgfscope}%
\pgfsys@transformshift{3.149621in}{1.249167in}%
\pgfsys@useobject{currentmarker}{}%
\end{pgfscope}%
\begin{pgfscope}%
\pgfsys@transformshift{3.149621in}{1.260833in}%
\pgfsys@useobject{currentmarker}{}%
\end{pgfscope}%
\begin{pgfscope}%
\pgfsys@transformshift{3.149621in}{1.272500in}%
\pgfsys@useobject{currentmarker}{}%
\end{pgfscope}%
\begin{pgfscope}%
\pgfsys@transformshift{3.149621in}{1.284167in}%
\pgfsys@useobject{currentmarker}{}%
\end{pgfscope}%
\begin{pgfscope}%
\pgfsys@transformshift{3.149621in}{1.295833in}%
\pgfsys@useobject{currentmarker}{}%
\end{pgfscope}%
\begin{pgfscope}%
\pgfsys@transformshift{3.149621in}{1.307500in}%
\pgfsys@useobject{currentmarker}{}%
\end{pgfscope}%
\begin{pgfscope}%
\pgfsys@transformshift{3.149621in}{1.319167in}%
\pgfsys@useobject{currentmarker}{}%
\end{pgfscope}%
\begin{pgfscope}%
\pgfsys@transformshift{3.149621in}{1.330833in}%
\pgfsys@useobject{currentmarker}{}%
\end{pgfscope}%
\begin{pgfscope}%
\pgfsys@transformshift{3.149621in}{1.342500in}%
\pgfsys@useobject{currentmarker}{}%
\end{pgfscope}%
\begin{pgfscope}%
\pgfsys@transformshift{3.149621in}{1.354167in}%
\pgfsys@useobject{currentmarker}{}%
\end{pgfscope}%
\begin{pgfscope}%
\pgfsys@transformshift{3.149621in}{1.365833in}%
\pgfsys@useobject{currentmarker}{}%
\end{pgfscope}%
\begin{pgfscope}%
\pgfsys@transformshift{3.149621in}{1.377500in}%
\pgfsys@useobject{currentmarker}{}%
\end{pgfscope}%
\begin{pgfscope}%
\pgfsys@transformshift{3.149621in}{1.412500in}%
\pgfsys@useobject{currentmarker}{}%
\end{pgfscope}%
\begin{pgfscope}%
\pgfsys@transformshift{3.149621in}{1.482500in}%
\pgfsys@useobject{currentmarker}{}%
\end{pgfscope}%
\end{pgfscope}%
\begin{pgfscope}%
\pgfpathrectangle{\pgfqpoint{0.625000in}{0.275000in}}{\pgfqpoint{3.875000in}{1.925000in}}%
\pgfusepath{clip}%
\pgfsetbuttcap%
\pgfsetroundjoin%
\definecolor{currentfill}{rgb}{0.000000,0.000000,0.000000}%
\pgfsetfillcolor{currentfill}%
\pgfsetlinewidth{1.003750pt}%
\definecolor{currentstroke}{rgb}{0.000000,0.000000,0.000000}%
\pgfsetstrokecolor{currentstroke}%
\pgfsetdash{}{0pt}%
\pgfsys@defobject{currentmarker}{\pgfqpoint{-0.041667in}{-0.041667in}}{\pgfqpoint{0.041667in}{0.041667in}}{%
\pgfpathmoveto{\pgfqpoint{0.000000in}{-0.041667in}}%
\pgfpathcurveto{\pgfqpoint{0.011050in}{-0.041667in}}{\pgfqpoint{0.021649in}{-0.037276in}}{\pgfqpoint{0.029463in}{-0.029463in}}%
\pgfpathcurveto{\pgfqpoint{0.037276in}{-0.021649in}}{\pgfqpoint{0.041667in}{-0.011050in}}{\pgfqpoint{0.041667in}{0.000000in}}%
\pgfpathcurveto{\pgfqpoint{0.041667in}{0.011050in}}{\pgfqpoint{0.037276in}{0.021649in}}{\pgfqpoint{0.029463in}{0.029463in}}%
\pgfpathcurveto{\pgfqpoint{0.021649in}{0.037276in}}{\pgfqpoint{0.011050in}{0.041667in}}{\pgfqpoint{0.000000in}{0.041667in}}%
\pgfpathcurveto{\pgfqpoint{-0.011050in}{0.041667in}}{\pgfqpoint{-0.021649in}{0.037276in}}{\pgfqpoint{-0.029463in}{0.029463in}}%
\pgfpathcurveto{\pgfqpoint{-0.037276in}{0.021649in}}{\pgfqpoint{-0.041667in}{0.011050in}}{\pgfqpoint{-0.041667in}{0.000000in}}%
\pgfpathcurveto{\pgfqpoint{-0.041667in}{-0.011050in}}{\pgfqpoint{-0.037276in}{-0.021649in}}{\pgfqpoint{-0.029463in}{-0.029463in}}%
\pgfpathcurveto{\pgfqpoint{-0.021649in}{-0.037276in}}{\pgfqpoint{-0.011050in}{-0.041667in}}{\pgfqpoint{0.000000in}{-0.041667in}}%
\pgfpathclose%
\pgfusepath{stroke,fill}%
}%
\begin{pgfscope}%
\pgfsys@transformshift{4.323864in}{1.575833in}%
\pgfsys@useobject{currentmarker}{}%
\end{pgfscope}%
\begin{pgfscope}%
\pgfsys@transformshift{4.323864in}{1.587500in}%
\pgfsys@useobject{currentmarker}{}%
\end{pgfscope}%
\begin{pgfscope}%
\pgfsys@transformshift{4.323864in}{1.599167in}%
\pgfsys@useobject{currentmarker}{}%
\end{pgfscope}%
\begin{pgfscope}%
\pgfsys@transformshift{4.323864in}{1.610833in}%
\pgfsys@useobject{currentmarker}{}%
\end{pgfscope}%
\begin{pgfscope}%
\pgfsys@transformshift{4.323864in}{1.622500in}%
\pgfsys@useobject{currentmarker}{}%
\end{pgfscope}%
\begin{pgfscope}%
\pgfsys@transformshift{4.323864in}{1.634167in}%
\pgfsys@useobject{currentmarker}{}%
\end{pgfscope}%
\begin{pgfscope}%
\pgfsys@transformshift{4.323864in}{1.645833in}%
\pgfsys@useobject{currentmarker}{}%
\end{pgfscope}%
\begin{pgfscope}%
\pgfsys@transformshift{4.323864in}{1.657500in}%
\pgfsys@useobject{currentmarker}{}%
\end{pgfscope}%
\begin{pgfscope}%
\pgfsys@transformshift{4.323864in}{1.669167in}%
\pgfsys@useobject{currentmarker}{}%
\end{pgfscope}%
\begin{pgfscope}%
\pgfsys@transformshift{4.323864in}{1.680833in}%
\pgfsys@useobject{currentmarker}{}%
\end{pgfscope}%
\begin{pgfscope}%
\pgfsys@transformshift{4.323864in}{1.692500in}%
\pgfsys@useobject{currentmarker}{}%
\end{pgfscope}%
\begin{pgfscope}%
\pgfsys@transformshift{4.323864in}{1.704167in}%
\pgfsys@useobject{currentmarker}{}%
\end{pgfscope}%
\begin{pgfscope}%
\pgfsys@transformshift{4.323864in}{1.715833in}%
\pgfsys@useobject{currentmarker}{}%
\end{pgfscope}%
\begin{pgfscope}%
\pgfsys@transformshift{4.323864in}{1.727500in}%
\pgfsys@useobject{currentmarker}{}%
\end{pgfscope}%
\begin{pgfscope}%
\pgfsys@transformshift{4.323864in}{1.739167in}%
\pgfsys@useobject{currentmarker}{}%
\end{pgfscope}%
\begin{pgfscope}%
\pgfsys@transformshift{4.323864in}{1.750833in}%
\pgfsys@useobject{currentmarker}{}%
\end{pgfscope}%
\begin{pgfscope}%
\pgfsys@transformshift{4.323864in}{1.762500in}%
\pgfsys@useobject{currentmarker}{}%
\end{pgfscope}%
\begin{pgfscope}%
\pgfsys@transformshift{4.323864in}{1.774167in}%
\pgfsys@useobject{currentmarker}{}%
\end{pgfscope}%
\begin{pgfscope}%
\pgfsys@transformshift{4.323864in}{1.785833in}%
\pgfsys@useobject{currentmarker}{}%
\end{pgfscope}%
\begin{pgfscope}%
\pgfsys@transformshift{4.323864in}{1.820833in}%
\pgfsys@useobject{currentmarker}{}%
\end{pgfscope}%
\begin{pgfscope}%
\pgfsys@transformshift{4.323864in}{1.844167in}%
\pgfsys@useobject{currentmarker}{}%
\end{pgfscope}%
\begin{pgfscope}%
\pgfsys@transformshift{4.323864in}{1.855833in}%
\pgfsys@useobject{currentmarker}{}%
\end{pgfscope}%
\begin{pgfscope}%
\pgfsys@transformshift{4.323864in}{1.879167in}%
\pgfsys@useobject{currentmarker}{}%
\end{pgfscope}%
\begin{pgfscope}%
\pgfsys@transformshift{4.323864in}{1.902500in}%
\pgfsys@useobject{currentmarker}{}%
\end{pgfscope}%
\begin{pgfscope}%
\pgfsys@transformshift{4.323864in}{1.914167in}%
\pgfsys@useobject{currentmarker}{}%
\end{pgfscope}%
\begin{pgfscope}%
\pgfsys@transformshift{4.323864in}{1.925833in}%
\pgfsys@useobject{currentmarker}{}%
\end{pgfscope}%
\begin{pgfscope}%
\pgfsys@transformshift{4.323864in}{2.112500in}%
\pgfsys@useobject{currentmarker}{}%
\end{pgfscope}%
\end{pgfscope}%
\begin{pgfscope}%
\pgfpathrectangle{\pgfqpoint{0.625000in}{0.275000in}}{\pgfqpoint{3.875000in}{1.925000in}}%
\pgfusepath{clip}%
\pgfsetbuttcap%
\pgfsetmiterjoin%
\definecolor{currentfill}{rgb}{0.000000,0.000000,0.000000}%
\pgfsetfillcolor{currentfill}%
\pgfsetlinewidth{1.003750pt}%
\definecolor{currentstroke}{rgb}{0.000000,0.000000,0.000000}%
\pgfsetstrokecolor{currentstroke}%
\pgfsetdash{}{0pt}%
\pgfsys@defobject{currentmarker}{\pgfqpoint{-0.041667in}{-0.041667in}}{\pgfqpoint{0.041667in}{0.041667in}}{%
\pgfpathmoveto{\pgfqpoint{-0.000000in}{-0.041667in}}%
\pgfpathlineto{\pgfqpoint{0.041667in}{0.041667in}}%
\pgfpathlineto{\pgfqpoint{-0.041667in}{0.041667in}}%
\pgfpathclose%
\pgfusepath{stroke,fill}%
}%
\begin{pgfscope}%
\pgfsys@transformshift{0.801136in}{0.432500in}%
\pgfsys@useobject{currentmarker}{}%
\end{pgfscope}%
\begin{pgfscope}%
\pgfsys@transformshift{0.801136in}{0.444167in}%
\pgfsys@useobject{currentmarker}{}%
\end{pgfscope}%
\begin{pgfscope}%
\pgfsys@transformshift{0.801136in}{0.455833in}%
\pgfsys@useobject{currentmarker}{}%
\end{pgfscope}%
\begin{pgfscope}%
\pgfsys@transformshift{0.801136in}{0.467500in}%
\pgfsys@useobject{currentmarker}{}%
\end{pgfscope}%
\end{pgfscope}%
\begin{pgfscope}%
\pgfpathrectangle{\pgfqpoint{0.625000in}{0.275000in}}{\pgfqpoint{3.875000in}{1.925000in}}%
\pgfusepath{clip}%
\pgfsetbuttcap%
\pgfsetmiterjoin%
\definecolor{currentfill}{rgb}{0.000000,0.000000,0.000000}%
\pgfsetfillcolor{currentfill}%
\pgfsetlinewidth{1.003750pt}%
\definecolor{currentstroke}{rgb}{0.000000,0.000000,0.000000}%
\pgfsetstrokecolor{currentstroke}%
\pgfsetdash{}{0pt}%
\pgfsys@defobject{currentmarker}{\pgfqpoint{-0.041667in}{-0.041667in}}{\pgfqpoint{0.041667in}{0.041667in}}{%
\pgfpathmoveto{\pgfqpoint{-0.000000in}{-0.041667in}}%
\pgfpathlineto{\pgfqpoint{0.041667in}{0.041667in}}%
\pgfpathlineto{\pgfqpoint{-0.041667in}{0.041667in}}%
\pgfpathclose%
\pgfusepath{stroke,fill}%
}%
\begin{pgfscope}%
\pgfsys@transformshift{1.975379in}{0.525833in}%
\pgfsys@useobject{currentmarker}{}%
\end{pgfscope}%
\begin{pgfscope}%
\pgfsys@transformshift{1.975379in}{0.537500in}%
\pgfsys@useobject{currentmarker}{}%
\end{pgfscope}%
\begin{pgfscope}%
\pgfsys@transformshift{1.975379in}{0.549167in}%
\pgfsys@useobject{currentmarker}{}%
\end{pgfscope}%
\begin{pgfscope}%
\pgfsys@transformshift{1.975379in}{0.560833in}%
\pgfsys@useobject{currentmarker}{}%
\end{pgfscope}%
\begin{pgfscope}%
\pgfsys@transformshift{1.975379in}{0.572500in}%
\pgfsys@useobject{currentmarker}{}%
\end{pgfscope}%
\begin{pgfscope}%
\pgfsys@transformshift{1.975379in}{0.584167in}%
\pgfsys@useobject{currentmarker}{}%
\end{pgfscope}%
\begin{pgfscope}%
\pgfsys@transformshift{1.975379in}{0.595833in}%
\pgfsys@useobject{currentmarker}{}%
\end{pgfscope}%
\end{pgfscope}%
\begin{pgfscope}%
\pgfpathrectangle{\pgfqpoint{0.625000in}{0.275000in}}{\pgfqpoint{3.875000in}{1.925000in}}%
\pgfusepath{clip}%
\pgfsetbuttcap%
\pgfsetmiterjoin%
\definecolor{currentfill}{rgb}{0.000000,0.000000,0.000000}%
\pgfsetfillcolor{currentfill}%
\pgfsetlinewidth{1.003750pt}%
\definecolor{currentstroke}{rgb}{0.000000,0.000000,0.000000}%
\pgfsetstrokecolor{currentstroke}%
\pgfsetdash{}{0pt}%
\pgfsys@defobject{currentmarker}{\pgfqpoint{-0.041667in}{-0.041667in}}{\pgfqpoint{0.041667in}{0.041667in}}{%
\pgfpathmoveto{\pgfqpoint{-0.000000in}{-0.041667in}}%
\pgfpathlineto{\pgfqpoint{0.041667in}{0.041667in}}%
\pgfpathlineto{\pgfqpoint{-0.041667in}{0.041667in}}%
\pgfpathclose%
\pgfusepath{stroke,fill}%
}%
\begin{pgfscope}%
\pgfsys@transformshift{3.149621in}{0.735833in}%
\pgfsys@useobject{currentmarker}{}%
\end{pgfscope}%
\begin{pgfscope}%
\pgfsys@transformshift{3.149621in}{0.747500in}%
\pgfsys@useobject{currentmarker}{}%
\end{pgfscope}%
\begin{pgfscope}%
\pgfsys@transformshift{3.149621in}{0.759167in}%
\pgfsys@useobject{currentmarker}{}%
\end{pgfscope}%
\begin{pgfscope}%
\pgfsys@transformshift{3.149621in}{0.770833in}%
\pgfsys@useobject{currentmarker}{}%
\end{pgfscope}%
\begin{pgfscope}%
\pgfsys@transformshift{3.149621in}{0.782500in}%
\pgfsys@useobject{currentmarker}{}%
\end{pgfscope}%
\begin{pgfscope}%
\pgfsys@transformshift{3.149621in}{0.794167in}%
\pgfsys@useobject{currentmarker}{}%
\end{pgfscope}%
\begin{pgfscope}%
\pgfsys@transformshift{3.149621in}{0.805833in}%
\pgfsys@useobject{currentmarker}{}%
\end{pgfscope}%
\begin{pgfscope}%
\pgfsys@transformshift{3.149621in}{0.817500in}%
\pgfsys@useobject{currentmarker}{}%
\end{pgfscope}%
\begin{pgfscope}%
\pgfsys@transformshift{3.149621in}{0.829167in}%
\pgfsys@useobject{currentmarker}{}%
\end{pgfscope}%
\end{pgfscope}%
\begin{pgfscope}%
\pgfpathrectangle{\pgfqpoint{0.625000in}{0.275000in}}{\pgfqpoint{3.875000in}{1.925000in}}%
\pgfusepath{clip}%
\pgfsetbuttcap%
\pgfsetmiterjoin%
\definecolor{currentfill}{rgb}{0.000000,0.000000,0.000000}%
\pgfsetfillcolor{currentfill}%
\pgfsetlinewidth{1.003750pt}%
\definecolor{currentstroke}{rgb}{0.000000,0.000000,0.000000}%
\pgfsetstrokecolor{currentstroke}%
\pgfsetdash{}{0pt}%
\pgfsys@defobject{currentmarker}{\pgfqpoint{-0.041667in}{-0.041667in}}{\pgfqpoint{0.041667in}{0.041667in}}{%
\pgfpathmoveto{\pgfqpoint{-0.000000in}{-0.041667in}}%
\pgfpathlineto{\pgfqpoint{0.041667in}{0.041667in}}%
\pgfpathlineto{\pgfqpoint{-0.041667in}{0.041667in}}%
\pgfpathclose%
\pgfusepath{stroke,fill}%
}%
\begin{pgfscope}%
\pgfsys@transformshift{4.323864in}{0.934167in}%
\pgfsys@useobject{currentmarker}{}%
\end{pgfscope}%
\begin{pgfscope}%
\pgfsys@transformshift{4.323864in}{0.957500in}%
\pgfsys@useobject{currentmarker}{}%
\end{pgfscope}%
\begin{pgfscope}%
\pgfsys@transformshift{4.323864in}{0.969167in}%
\pgfsys@useobject{currentmarker}{}%
\end{pgfscope}%
\begin{pgfscope}%
\pgfsys@transformshift{4.323864in}{0.980833in}%
\pgfsys@useobject{currentmarker}{}%
\end{pgfscope}%
\begin{pgfscope}%
\pgfsys@transformshift{4.323864in}{0.992500in}%
\pgfsys@useobject{currentmarker}{}%
\end{pgfscope}%
\begin{pgfscope}%
\pgfsys@transformshift{4.323864in}{1.004167in}%
\pgfsys@useobject{currentmarker}{}%
\end{pgfscope}%
\begin{pgfscope}%
\pgfsys@transformshift{4.323864in}{1.015833in}%
\pgfsys@useobject{currentmarker}{}%
\end{pgfscope}%
\begin{pgfscope}%
\pgfsys@transformshift{4.323864in}{1.027500in}%
\pgfsys@useobject{currentmarker}{}%
\end{pgfscope}%
\begin{pgfscope}%
\pgfsys@transformshift{4.323864in}{1.039167in}%
\pgfsys@useobject{currentmarker}{}%
\end{pgfscope}%
\begin{pgfscope}%
\pgfsys@transformshift{4.323864in}{1.050833in}%
\pgfsys@useobject{currentmarker}{}%
\end{pgfscope}%
\begin{pgfscope}%
\pgfsys@transformshift{4.323864in}{1.062500in}%
\pgfsys@useobject{currentmarker}{}%
\end{pgfscope}%
\begin{pgfscope}%
\pgfsys@transformshift{4.323864in}{1.074167in}%
\pgfsys@useobject{currentmarker}{}%
\end{pgfscope}%
\begin{pgfscope}%
\pgfsys@transformshift{4.323864in}{1.085833in}%
\pgfsys@useobject{currentmarker}{}%
\end{pgfscope}%
\begin{pgfscope}%
\pgfsys@transformshift{4.323864in}{1.097500in}%
\pgfsys@useobject{currentmarker}{}%
\end{pgfscope}%
\begin{pgfscope}%
\pgfsys@transformshift{4.323864in}{1.109167in}%
\pgfsys@useobject{currentmarker}{}%
\end{pgfscope}%
\begin{pgfscope}%
\pgfsys@transformshift{4.323864in}{1.120833in}%
\pgfsys@useobject{currentmarker}{}%
\end{pgfscope}%
\end{pgfscope}%
\begin{pgfscope}%
\pgfpathrectangle{\pgfqpoint{0.625000in}{0.275000in}}{\pgfqpoint{3.875000in}{1.925000in}}%
\pgfusepath{clip}%
\pgfsetbuttcap%
\pgfsetmiterjoin%
\definecolor{currentfill}{rgb}{0.000000,0.000000,0.000000}%
\pgfsetfillcolor{currentfill}%
\pgfsetlinewidth{1.003750pt}%
\definecolor{currentstroke}{rgb}{0.000000,0.000000,0.000000}%
\pgfsetstrokecolor{currentstroke}%
\pgfsetdash{}{0pt}%
\pgfsys@defobject{currentmarker}{\pgfqpoint{-0.041667in}{-0.041667in}}{\pgfqpoint{0.041667in}{0.041667in}}{%
\pgfpathmoveto{\pgfqpoint{-0.041667in}{-0.041667in}}%
\pgfpathlineto{\pgfqpoint{0.041667in}{-0.041667in}}%
\pgfpathlineto{\pgfqpoint{0.041667in}{0.041667in}}%
\pgfpathlineto{\pgfqpoint{-0.041667in}{0.041667in}}%
\pgfpathclose%
\pgfusepath{stroke,fill}%
}%
\begin{pgfscope}%
\pgfsys@transformshift{0.801136in}{0.397500in}%
\pgfsys@useobject{currentmarker}{}%
\end{pgfscope}%
\begin{pgfscope}%
\pgfsys@transformshift{0.801136in}{0.409167in}%
\pgfsys@useobject{currentmarker}{}%
\end{pgfscope}%
\begin{pgfscope}%
\pgfsys@transformshift{0.801136in}{0.420833in}%
\pgfsys@useobject{currentmarker}{}%
\end{pgfscope}%
\end{pgfscope}%
\begin{pgfscope}%
\pgfpathrectangle{\pgfqpoint{0.625000in}{0.275000in}}{\pgfqpoint{3.875000in}{1.925000in}}%
\pgfusepath{clip}%
\pgfsetbuttcap%
\pgfsetmiterjoin%
\definecolor{currentfill}{rgb}{0.000000,0.000000,0.000000}%
\pgfsetfillcolor{currentfill}%
\pgfsetlinewidth{1.003750pt}%
\definecolor{currentstroke}{rgb}{0.000000,0.000000,0.000000}%
\pgfsetstrokecolor{currentstroke}%
\pgfsetdash{}{0pt}%
\pgfsys@defobject{currentmarker}{\pgfqpoint{-0.041667in}{-0.041667in}}{\pgfqpoint{0.041667in}{0.041667in}}{%
\pgfpathmoveto{\pgfqpoint{-0.041667in}{-0.041667in}}%
\pgfpathlineto{\pgfqpoint{0.041667in}{-0.041667in}}%
\pgfpathlineto{\pgfqpoint{0.041667in}{0.041667in}}%
\pgfpathlineto{\pgfqpoint{-0.041667in}{0.041667in}}%
\pgfpathclose%
\pgfusepath{stroke,fill}%
}%
\begin{pgfscope}%
\pgfsys@transformshift{1.975379in}{0.455833in}%
\pgfsys@useobject{currentmarker}{}%
\end{pgfscope}%
\begin{pgfscope}%
\pgfsys@transformshift{1.975379in}{0.467500in}%
\pgfsys@useobject{currentmarker}{}%
\end{pgfscope}%
\begin{pgfscope}%
\pgfsys@transformshift{1.975379in}{0.479167in}%
\pgfsys@useobject{currentmarker}{}%
\end{pgfscope}%
\end{pgfscope}%
\begin{pgfscope}%
\pgfpathrectangle{\pgfqpoint{0.625000in}{0.275000in}}{\pgfqpoint{3.875000in}{1.925000in}}%
\pgfusepath{clip}%
\pgfsetbuttcap%
\pgfsetmiterjoin%
\definecolor{currentfill}{rgb}{0.000000,0.000000,0.000000}%
\pgfsetfillcolor{currentfill}%
\pgfsetlinewidth{1.003750pt}%
\definecolor{currentstroke}{rgb}{0.000000,0.000000,0.000000}%
\pgfsetstrokecolor{currentstroke}%
\pgfsetdash{}{0pt}%
\pgfsys@defobject{currentmarker}{\pgfqpoint{-0.041667in}{-0.041667in}}{\pgfqpoint{0.041667in}{0.041667in}}{%
\pgfpathmoveto{\pgfqpoint{-0.041667in}{-0.041667in}}%
\pgfpathlineto{\pgfqpoint{0.041667in}{-0.041667in}}%
\pgfpathlineto{\pgfqpoint{0.041667in}{0.041667in}}%
\pgfpathlineto{\pgfqpoint{-0.041667in}{0.041667in}}%
\pgfpathclose%
\pgfusepath{stroke,fill}%
}%
\begin{pgfscope}%
\pgfsys@transformshift{3.149621in}{0.537500in}%
\pgfsys@useobject{currentmarker}{}%
\end{pgfscope}%
\begin{pgfscope}%
\pgfsys@transformshift{3.149621in}{0.549167in}%
\pgfsys@useobject{currentmarker}{}%
\end{pgfscope}%
\begin{pgfscope}%
\pgfsys@transformshift{3.149621in}{0.560833in}%
\pgfsys@useobject{currentmarker}{}%
\end{pgfscope}%
\begin{pgfscope}%
\pgfsys@transformshift{3.149621in}{0.572500in}%
\pgfsys@useobject{currentmarker}{}%
\end{pgfscope}%
\end{pgfscope}%
\begin{pgfscope}%
\pgfpathrectangle{\pgfqpoint{0.625000in}{0.275000in}}{\pgfqpoint{3.875000in}{1.925000in}}%
\pgfusepath{clip}%
\pgfsetbuttcap%
\pgfsetmiterjoin%
\definecolor{currentfill}{rgb}{0.000000,0.000000,0.000000}%
\pgfsetfillcolor{currentfill}%
\pgfsetlinewidth{1.003750pt}%
\definecolor{currentstroke}{rgb}{0.000000,0.000000,0.000000}%
\pgfsetstrokecolor{currentstroke}%
\pgfsetdash{}{0pt}%
\pgfsys@defobject{currentmarker}{\pgfqpoint{-0.041667in}{-0.041667in}}{\pgfqpoint{0.041667in}{0.041667in}}{%
\pgfpathmoveto{\pgfqpoint{-0.041667in}{-0.041667in}}%
\pgfpathlineto{\pgfqpoint{0.041667in}{-0.041667in}}%
\pgfpathlineto{\pgfqpoint{0.041667in}{0.041667in}}%
\pgfpathlineto{\pgfqpoint{-0.041667in}{0.041667in}}%
\pgfpathclose%
\pgfusepath{stroke,fill}%
}%
\begin{pgfscope}%
\pgfsys@transformshift{4.323864in}{0.619167in}%
\pgfsys@useobject{currentmarker}{}%
\end{pgfscope}%
\begin{pgfscope}%
\pgfsys@transformshift{4.323864in}{0.630833in}%
\pgfsys@useobject{currentmarker}{}%
\end{pgfscope}%
\begin{pgfscope}%
\pgfsys@transformshift{4.323864in}{0.642500in}%
\pgfsys@useobject{currentmarker}{}%
\end{pgfscope}%
\begin{pgfscope}%
\pgfsys@transformshift{4.323864in}{0.654167in}%
\pgfsys@useobject{currentmarker}{}%
\end{pgfscope}%
\begin{pgfscope}%
\pgfsys@transformshift{4.323864in}{0.665833in}%
\pgfsys@useobject{currentmarker}{}%
\end{pgfscope}%
\begin{pgfscope}%
\pgfsys@transformshift{4.323864in}{0.677500in}%
\pgfsys@useobject{currentmarker}{}%
\end{pgfscope}%
\end{pgfscope}%
\begin{pgfscope}%
\pgfpathrectangle{\pgfqpoint{0.625000in}{0.275000in}}{\pgfqpoint{3.875000in}{1.925000in}}%
\pgfusepath{clip}%
\pgfsetbuttcap%
\pgfsetmiterjoin%
\definecolor{currentfill}{rgb}{0.000000,0.000000,0.000000}%
\pgfsetfillcolor{currentfill}%
\pgfsetlinewidth{1.003750pt}%
\definecolor{currentstroke}{rgb}{0.000000,0.000000,0.000000}%
\pgfsetstrokecolor{currentstroke}%
\pgfsetdash{}{0pt}%
\pgfsys@defobject{currentmarker}{\pgfqpoint{-0.035355in}{-0.058926in}}{\pgfqpoint{0.035355in}{0.058926in}}{%
\pgfpathmoveto{\pgfqpoint{-0.000000in}{-0.058926in}}%
\pgfpathlineto{\pgfqpoint{0.035355in}{0.000000in}}%
\pgfpathlineto{\pgfqpoint{0.000000in}{0.058926in}}%
\pgfpathlineto{\pgfqpoint{-0.035355in}{0.000000in}}%
\pgfpathclose%
\pgfusepath{stroke,fill}%
}%
\begin{pgfscope}%
\pgfsys@transformshift{0.801136in}{0.362500in}%
\pgfsys@useobject{currentmarker}{}%
\end{pgfscope}%
\begin{pgfscope}%
\pgfsys@transformshift{0.801136in}{0.374167in}%
\pgfsys@useobject{currentmarker}{}%
\end{pgfscope}%
\begin{pgfscope}%
\pgfsys@transformshift{0.801136in}{0.385833in}%
\pgfsys@useobject{currentmarker}{}%
\end{pgfscope}%
\end{pgfscope}%
\begin{pgfscope}%
\pgfpathrectangle{\pgfqpoint{0.625000in}{0.275000in}}{\pgfqpoint{3.875000in}{1.925000in}}%
\pgfusepath{clip}%
\pgfsetbuttcap%
\pgfsetmiterjoin%
\definecolor{currentfill}{rgb}{0.000000,0.000000,0.000000}%
\pgfsetfillcolor{currentfill}%
\pgfsetlinewidth{1.003750pt}%
\definecolor{currentstroke}{rgb}{0.000000,0.000000,0.000000}%
\pgfsetstrokecolor{currentstroke}%
\pgfsetdash{}{0pt}%
\pgfsys@defobject{currentmarker}{\pgfqpoint{-0.035355in}{-0.058926in}}{\pgfqpoint{0.035355in}{0.058926in}}{%
\pgfpathmoveto{\pgfqpoint{-0.000000in}{-0.058926in}}%
\pgfpathlineto{\pgfqpoint{0.035355in}{0.000000in}}%
\pgfpathlineto{\pgfqpoint{0.000000in}{0.058926in}}%
\pgfpathlineto{\pgfqpoint{-0.035355in}{0.000000in}}%
\pgfpathclose%
\pgfusepath{stroke,fill}%
}%
\begin{pgfscope}%
\pgfsys@transformshift{1.975379in}{0.397500in}%
\pgfsys@useobject{currentmarker}{}%
\end{pgfscope}%
\begin{pgfscope}%
\pgfsys@transformshift{1.975379in}{0.409167in}%
\pgfsys@useobject{currentmarker}{}%
\end{pgfscope}%
\begin{pgfscope}%
\pgfsys@transformshift{1.975379in}{0.420833in}%
\pgfsys@useobject{currentmarker}{}%
\end{pgfscope}%
\end{pgfscope}%
\begin{pgfscope}%
\pgfpathrectangle{\pgfqpoint{0.625000in}{0.275000in}}{\pgfqpoint{3.875000in}{1.925000in}}%
\pgfusepath{clip}%
\pgfsetbuttcap%
\pgfsetmiterjoin%
\definecolor{currentfill}{rgb}{0.000000,0.000000,0.000000}%
\pgfsetfillcolor{currentfill}%
\pgfsetlinewidth{1.003750pt}%
\definecolor{currentstroke}{rgb}{0.000000,0.000000,0.000000}%
\pgfsetstrokecolor{currentstroke}%
\pgfsetdash{}{0pt}%
\pgfsys@defobject{currentmarker}{\pgfqpoint{-0.035355in}{-0.058926in}}{\pgfqpoint{0.035355in}{0.058926in}}{%
\pgfpathmoveto{\pgfqpoint{-0.000000in}{-0.058926in}}%
\pgfpathlineto{\pgfqpoint{0.035355in}{0.000000in}}%
\pgfpathlineto{\pgfqpoint{0.000000in}{0.058926in}}%
\pgfpathlineto{\pgfqpoint{-0.035355in}{0.000000in}}%
\pgfpathclose%
\pgfusepath{stroke,fill}%
}%
\begin{pgfscope}%
\pgfsys@transformshift{3.149621in}{0.444167in}%
\pgfsys@useobject{currentmarker}{}%
\end{pgfscope}%
\begin{pgfscope}%
\pgfsys@transformshift{3.149621in}{0.455833in}%
\pgfsys@useobject{currentmarker}{}%
\end{pgfscope}%
\begin{pgfscope}%
\pgfsys@transformshift{3.149621in}{0.467500in}%
\pgfsys@useobject{currentmarker}{}%
\end{pgfscope}%
\end{pgfscope}%
\begin{pgfscope}%
\pgfpathrectangle{\pgfqpoint{0.625000in}{0.275000in}}{\pgfqpoint{3.875000in}{1.925000in}}%
\pgfusepath{clip}%
\pgfsetbuttcap%
\pgfsetmiterjoin%
\definecolor{currentfill}{rgb}{0.000000,0.000000,0.000000}%
\pgfsetfillcolor{currentfill}%
\pgfsetlinewidth{1.003750pt}%
\definecolor{currentstroke}{rgb}{0.000000,0.000000,0.000000}%
\pgfsetstrokecolor{currentstroke}%
\pgfsetdash{}{0pt}%
\pgfsys@defobject{currentmarker}{\pgfqpoint{-0.035355in}{-0.058926in}}{\pgfqpoint{0.035355in}{0.058926in}}{%
\pgfpathmoveto{\pgfqpoint{-0.000000in}{-0.058926in}}%
\pgfpathlineto{\pgfqpoint{0.035355in}{0.000000in}}%
\pgfpathlineto{\pgfqpoint{0.000000in}{0.058926in}}%
\pgfpathlineto{\pgfqpoint{-0.035355in}{0.000000in}}%
\pgfpathclose%
\pgfusepath{stroke,fill}%
}%
\begin{pgfscope}%
\pgfsys@transformshift{4.323864in}{0.467500in}%
\pgfsys@useobject{currentmarker}{}%
\end{pgfscope}%
\begin{pgfscope}%
\pgfsys@transformshift{4.323864in}{0.479167in}%
\pgfsys@useobject{currentmarker}{}%
\end{pgfscope}%
\begin{pgfscope}%
\pgfsys@transformshift{4.323864in}{0.490833in}%
\pgfsys@useobject{currentmarker}{}%
\end{pgfscope}%
\begin{pgfscope}%
\pgfsys@transformshift{4.323864in}{0.502500in}%
\pgfsys@useobject{currentmarker}{}%
\end{pgfscope}%
\end{pgfscope}%
\begin{pgfscope}%
\pgfsetrectcap%
\pgfsetmiterjoin%
\pgfsetlinewidth{0.803000pt}%
\definecolor{currentstroke}{rgb}{0.000000,0.000000,0.000000}%
\pgfsetstrokecolor{currentstroke}%
\pgfsetdash{}{0pt}%
\pgfpathmoveto{\pgfqpoint{0.625000in}{0.275000in}}%
\pgfpathlineto{\pgfqpoint{0.625000in}{2.200000in}}%
\pgfusepath{stroke}%
\end{pgfscope}%
\begin{pgfscope}%
\pgfsetrectcap%
\pgfsetmiterjoin%
\pgfsetlinewidth{0.803000pt}%
\definecolor{currentstroke}{rgb}{0.000000,0.000000,0.000000}%
\pgfsetstrokecolor{currentstroke}%
\pgfsetdash{}{0pt}%
\pgfpathmoveto{\pgfqpoint{4.500000in}{0.275000in}}%
\pgfpathlineto{\pgfqpoint{4.500000in}{2.200000in}}%
\pgfusepath{stroke}%
\end{pgfscope}%
\begin{pgfscope}%
\pgfsetrectcap%
\pgfsetmiterjoin%
\pgfsetlinewidth{0.803000pt}%
\definecolor{currentstroke}{rgb}{0.000000,0.000000,0.000000}%
\pgfsetstrokecolor{currentstroke}%
\pgfsetdash{}{0pt}%
\pgfpathmoveto{\pgfqpoint{0.625000in}{0.275000in}}%
\pgfpathlineto{\pgfqpoint{4.500000in}{0.275000in}}%
\pgfusepath{stroke}%
\end{pgfscope}%
\begin{pgfscope}%
\pgfsetrectcap%
\pgfsetmiterjoin%
\pgfsetlinewidth{0.803000pt}%
\definecolor{currentstroke}{rgb}{0.000000,0.000000,0.000000}%
\pgfsetstrokecolor{currentstroke}%
\pgfsetdash{}{0pt}%
\pgfpathmoveto{\pgfqpoint{0.625000in}{2.200000in}}%
\pgfpathlineto{\pgfqpoint{4.500000in}{2.200000in}}%
\pgfusepath{stroke}%
\end{pgfscope}%
\end{pgfpicture}%
\makeatother%
\endgroup%

  \caption[Maximum GMRES iteration counts when $\NLiDRR{\nso-\nst} = 0.5\times  k^{-\beta}$ for $\beta = 0,0.1,0.2,0.3.$]{Maximum GMRES iteration counts for solving systems with matrix $\AmatoI\Amatt$, where $\Aso=\Ast=1$ and $\NLiDRR{\nso-\nst} = 0.5\times  k^{-\beta}$ for $\beta = 0,0.1,0.2,0.3.$}\label{fig:linfinityn0}
    \end{figure}
    
    \begin{figure}
      \centering
%% Creator: Matplotlib, PGF backend
%%
%% To include the figure in your LaTeX document, write
%%   \input{<filename>.pgf}
%%
%% Make sure the required packages are loaded in your preamble
%%   \usepackage{pgf}
%%
%% Figures using additional raster images can only be included by \input if
%% they are in the same directory as the main LaTeX file. For loading figures
%% from other directories you can use the `import` package
%%   \usepackage{import}
%% and then include the figures with
%%   \import{<path to file>}{<filename>.pgf}
%%
%% Matplotlib used the following preamble
%%   \usepackage{fontspec}
%%   \setmainfont{DejaVuSerif.ttf}[Path=/home/owen/progs/firedrake-complex/firedrake/lib/python3.5/site-packages/matplotlib/mpl-data/fonts/ttf/]
%%   \setsansfont{DejaVuSans.ttf}[Path=/home/owen/progs/firedrake-complex/firedrake/lib/python3.5/site-packages/matplotlib/mpl-data/fonts/ttf/]
%%   \setmonofont{DejaVuSansMono.ttf}[Path=/home/owen/progs/firedrake-complex/firedrake/lib/python3.5/site-packages/matplotlib/mpl-data/fonts/ttf/]
%%
\begingroup%
\makeatletter%
\begin{pgfpicture}%
\pgfpathrectangle{\pgfpointorigin}{\pgfqpoint{5.500000in}{5.500000in}}%
\pgfusepath{use as bounding box, clip}%
\begin{pgfscope}%
\pgfsetbuttcap%
\pgfsetmiterjoin%
\definecolor{currentfill}{rgb}{1.000000,1.000000,1.000000}%
\pgfsetfillcolor{currentfill}%
\pgfsetlinewidth{0.000000pt}%
\definecolor{currentstroke}{rgb}{1.000000,1.000000,1.000000}%
\pgfsetstrokecolor{currentstroke}%
\pgfsetdash{}{0pt}%
\pgfpathmoveto{\pgfqpoint{0.000000in}{0.000000in}}%
\pgfpathlineto{\pgfqpoint{5.500000in}{0.000000in}}%
\pgfpathlineto{\pgfqpoint{5.500000in}{5.500000in}}%
\pgfpathlineto{\pgfqpoint{0.000000in}{5.500000in}}%
\pgfpathclose%
\pgfusepath{fill}%
\end{pgfscope}%
\begin{pgfscope}%
\pgfsetbuttcap%
\pgfsetmiterjoin%
\definecolor{currentfill}{rgb}{1.000000,1.000000,1.000000}%
\pgfsetfillcolor{currentfill}%
\pgfsetlinewidth{0.000000pt}%
\definecolor{currentstroke}{rgb}{0.000000,0.000000,0.000000}%
\pgfsetstrokecolor{currentstroke}%
\pgfsetstrokeopacity{0.000000}%
\pgfsetdash{}{0pt}%
\pgfpathmoveto{\pgfqpoint{0.687500in}{0.605000in}}%
\pgfpathlineto{\pgfqpoint{4.950000in}{0.605000in}}%
\pgfpathlineto{\pgfqpoint{4.950000in}{4.840000in}}%
\pgfpathlineto{\pgfqpoint{0.687500in}{4.840000in}}%
\pgfpathclose%
\pgfusepath{fill}%
\end{pgfscope}%
\begin{pgfscope}%
\pgfsetbuttcap%
\pgfsetroundjoin%
\definecolor{currentfill}{rgb}{0.000000,0.000000,0.000000}%
\pgfsetfillcolor{currentfill}%
\pgfsetlinewidth{0.803000pt}%
\definecolor{currentstroke}{rgb}{0.000000,0.000000,0.000000}%
\pgfsetstrokecolor{currentstroke}%
\pgfsetdash{}{0pt}%
\pgfsys@defobject{currentmarker}{\pgfqpoint{0.000000in}{-0.048611in}}{\pgfqpoint{0.000000in}{0.000000in}}{%
\pgfpathmoveto{\pgfqpoint{0.000000in}{0.000000in}}%
\pgfpathlineto{\pgfqpoint{0.000000in}{-0.048611in}}%
\pgfusepath{stroke,fill}%
}%
\begin{pgfscope}%
\pgfsys@transformshift{0.881250in}{0.605000in}%
\pgfsys@useobject{currentmarker}{}%
\end{pgfscope}%
\end{pgfscope}%
\begin{pgfscope}%
\definecolor{textcolor}{rgb}{0.000000,0.000000,0.000000}%
\pgfsetstrokecolor{textcolor}%
\pgfsetfillcolor{textcolor}%
\pgftext[x=0.881250in,y=0.507778in,,top]{\color{textcolor}\sffamily\fontsize{10.000000}{12.000000}\selectfont \(\displaystyle 20\)}%
\end{pgfscope}%
\begin{pgfscope}%
\pgfsetbuttcap%
\pgfsetroundjoin%
\definecolor{currentfill}{rgb}{0.000000,0.000000,0.000000}%
\pgfsetfillcolor{currentfill}%
\pgfsetlinewidth{0.803000pt}%
\definecolor{currentstroke}{rgb}{0.000000,0.000000,0.000000}%
\pgfsetstrokecolor{currentstroke}%
\pgfsetdash{}{0pt}%
\pgfsys@defobject{currentmarker}{\pgfqpoint{0.000000in}{-0.048611in}}{\pgfqpoint{0.000000in}{0.000000in}}{%
\pgfpathmoveto{\pgfqpoint{0.000000in}{0.000000in}}%
\pgfpathlineto{\pgfqpoint{0.000000in}{-0.048611in}}%
\pgfusepath{stroke,fill}%
}%
\begin{pgfscope}%
\pgfsys@transformshift{2.172917in}{0.605000in}%
\pgfsys@useobject{currentmarker}{}%
\end{pgfscope}%
\end{pgfscope}%
\begin{pgfscope}%
\definecolor{textcolor}{rgb}{0.000000,0.000000,0.000000}%
\pgfsetstrokecolor{textcolor}%
\pgfsetfillcolor{textcolor}%
\pgftext[x=2.172917in,y=0.507778in,,top]{\color{textcolor}\sffamily\fontsize{10.000000}{12.000000}\selectfont \(\displaystyle 40\)}%
\end{pgfscope}%
\begin{pgfscope}%
\pgfsetbuttcap%
\pgfsetroundjoin%
\definecolor{currentfill}{rgb}{0.000000,0.000000,0.000000}%
\pgfsetfillcolor{currentfill}%
\pgfsetlinewidth{0.803000pt}%
\definecolor{currentstroke}{rgb}{0.000000,0.000000,0.000000}%
\pgfsetstrokecolor{currentstroke}%
\pgfsetdash{}{0pt}%
\pgfsys@defobject{currentmarker}{\pgfqpoint{0.000000in}{-0.048611in}}{\pgfqpoint{0.000000in}{0.000000in}}{%
\pgfpathmoveto{\pgfqpoint{0.000000in}{0.000000in}}%
\pgfpathlineto{\pgfqpoint{0.000000in}{-0.048611in}}%
\pgfusepath{stroke,fill}%
}%
\begin{pgfscope}%
\pgfsys@transformshift{3.464583in}{0.605000in}%
\pgfsys@useobject{currentmarker}{}%
\end{pgfscope}%
\end{pgfscope}%
\begin{pgfscope}%
\definecolor{textcolor}{rgb}{0.000000,0.000000,0.000000}%
\pgfsetstrokecolor{textcolor}%
\pgfsetfillcolor{textcolor}%
\pgftext[x=3.464583in,y=0.507778in,,top]{\color{textcolor}\sffamily\fontsize{10.000000}{12.000000}\selectfont \(\displaystyle 60\)}%
\end{pgfscope}%
\begin{pgfscope}%
\pgfsetbuttcap%
\pgfsetroundjoin%
\definecolor{currentfill}{rgb}{0.000000,0.000000,0.000000}%
\pgfsetfillcolor{currentfill}%
\pgfsetlinewidth{0.803000pt}%
\definecolor{currentstroke}{rgb}{0.000000,0.000000,0.000000}%
\pgfsetstrokecolor{currentstroke}%
\pgfsetdash{}{0pt}%
\pgfsys@defobject{currentmarker}{\pgfqpoint{0.000000in}{-0.048611in}}{\pgfqpoint{0.000000in}{0.000000in}}{%
\pgfpathmoveto{\pgfqpoint{0.000000in}{0.000000in}}%
\pgfpathlineto{\pgfqpoint{0.000000in}{-0.048611in}}%
\pgfusepath{stroke,fill}%
}%
\begin{pgfscope}%
\pgfsys@transformshift{4.756250in}{0.605000in}%
\pgfsys@useobject{currentmarker}{}%
\end{pgfscope}%
\end{pgfscope}%
\begin{pgfscope}%
\definecolor{textcolor}{rgb}{0.000000,0.000000,0.000000}%
\pgfsetstrokecolor{textcolor}%
\pgfsetfillcolor{textcolor}%
\pgftext[x=4.756250in,y=0.507778in,,top]{\color{textcolor}\sffamily\fontsize{10.000000}{12.000000}\selectfont \(\displaystyle 80\)}%
\end{pgfscope}%
\begin{pgfscope}%
\definecolor{textcolor}{rgb}{0.000000,0.000000,0.000000}%
\pgfsetstrokecolor{textcolor}%
\pgfsetfillcolor{textcolor}%
\pgftext[x=2.818750in,y=0.317809in,,top]{\color{textcolor}\sffamily\fontsize{10.000000}{12.000000}\selectfont \(\displaystyle k\)}%
\end{pgfscope}%
\begin{pgfscope}%
\pgfsetbuttcap%
\pgfsetroundjoin%
\definecolor{currentfill}{rgb}{0.000000,0.000000,0.000000}%
\pgfsetfillcolor{currentfill}%
\pgfsetlinewidth{0.803000pt}%
\definecolor{currentstroke}{rgb}{0.000000,0.000000,0.000000}%
\pgfsetstrokecolor{currentstroke}%
\pgfsetdash{}{0pt}%
\pgfsys@defobject{currentmarker}{\pgfqpoint{-0.048611in}{0.000000in}}{\pgfqpoint{0.000000in}{0.000000in}}{%
\pgfpathmoveto{\pgfqpoint{0.000000in}{0.000000in}}%
\pgfpathlineto{\pgfqpoint{-0.048611in}{0.000000in}}%
\pgfusepath{stroke,fill}%
}%
\begin{pgfscope}%
\pgfsys@transformshift{0.687500in}{0.797500in}%
\pgfsys@useobject{currentmarker}{}%
\end{pgfscope}%
\end{pgfscope}%
\begin{pgfscope}%
\definecolor{textcolor}{rgb}{0.000000,0.000000,0.000000}%
\pgfsetstrokecolor{textcolor}%
\pgfsetfillcolor{textcolor}%
\pgftext[x=0.520833in,y=0.744738in,left,base]{\color{textcolor}\sffamily\fontsize{10.000000}{12.000000}\selectfont \(\displaystyle 7\)}%
\end{pgfscope}%
\begin{pgfscope}%
\pgfsetbuttcap%
\pgfsetroundjoin%
\definecolor{currentfill}{rgb}{0.000000,0.000000,0.000000}%
\pgfsetfillcolor{currentfill}%
\pgfsetlinewidth{0.803000pt}%
\definecolor{currentstroke}{rgb}{0.000000,0.000000,0.000000}%
\pgfsetstrokecolor{currentstroke}%
\pgfsetdash{}{0pt}%
\pgfsys@defobject{currentmarker}{\pgfqpoint{-0.048611in}{0.000000in}}{\pgfqpoint{0.000000in}{0.000000in}}{%
\pgfpathmoveto{\pgfqpoint{0.000000in}{0.000000in}}%
\pgfpathlineto{\pgfqpoint{-0.048611in}{0.000000in}}%
\pgfusepath{stroke,fill}%
}%
\begin{pgfscope}%
\pgfsys@transformshift{0.687500in}{1.225278in}%
\pgfsys@useobject{currentmarker}{}%
\end{pgfscope}%
\end{pgfscope}%
\begin{pgfscope}%
\definecolor{textcolor}{rgb}{0.000000,0.000000,0.000000}%
\pgfsetstrokecolor{textcolor}%
\pgfsetfillcolor{textcolor}%
\pgftext[x=0.520833in,y=1.172516in,left,base]{\color{textcolor}\sffamily\fontsize{10.000000}{12.000000}\selectfont \(\displaystyle 8\)}%
\end{pgfscope}%
\begin{pgfscope}%
\pgfsetbuttcap%
\pgfsetroundjoin%
\definecolor{currentfill}{rgb}{0.000000,0.000000,0.000000}%
\pgfsetfillcolor{currentfill}%
\pgfsetlinewidth{0.803000pt}%
\definecolor{currentstroke}{rgb}{0.000000,0.000000,0.000000}%
\pgfsetstrokecolor{currentstroke}%
\pgfsetdash{}{0pt}%
\pgfsys@defobject{currentmarker}{\pgfqpoint{-0.048611in}{0.000000in}}{\pgfqpoint{0.000000in}{0.000000in}}{%
\pgfpathmoveto{\pgfqpoint{0.000000in}{0.000000in}}%
\pgfpathlineto{\pgfqpoint{-0.048611in}{0.000000in}}%
\pgfusepath{stroke,fill}%
}%
\begin{pgfscope}%
\pgfsys@transformshift{0.687500in}{1.653056in}%
\pgfsys@useobject{currentmarker}{}%
\end{pgfscope}%
\end{pgfscope}%
\begin{pgfscope}%
\definecolor{textcolor}{rgb}{0.000000,0.000000,0.000000}%
\pgfsetstrokecolor{textcolor}%
\pgfsetfillcolor{textcolor}%
\pgftext[x=0.520833in,y=1.600294in,left,base]{\color{textcolor}\sffamily\fontsize{10.000000}{12.000000}\selectfont \(\displaystyle 9\)}%
\end{pgfscope}%
\begin{pgfscope}%
\pgfsetbuttcap%
\pgfsetroundjoin%
\definecolor{currentfill}{rgb}{0.000000,0.000000,0.000000}%
\pgfsetfillcolor{currentfill}%
\pgfsetlinewidth{0.803000pt}%
\definecolor{currentstroke}{rgb}{0.000000,0.000000,0.000000}%
\pgfsetstrokecolor{currentstroke}%
\pgfsetdash{}{0pt}%
\pgfsys@defobject{currentmarker}{\pgfqpoint{-0.048611in}{0.000000in}}{\pgfqpoint{0.000000in}{0.000000in}}{%
\pgfpathmoveto{\pgfqpoint{0.000000in}{0.000000in}}%
\pgfpathlineto{\pgfqpoint{-0.048611in}{0.000000in}}%
\pgfusepath{stroke,fill}%
}%
\begin{pgfscope}%
\pgfsys@transformshift{0.687500in}{2.080833in}%
\pgfsys@useobject{currentmarker}{}%
\end{pgfscope}%
\end{pgfscope}%
\begin{pgfscope}%
\definecolor{textcolor}{rgb}{0.000000,0.000000,0.000000}%
\pgfsetstrokecolor{textcolor}%
\pgfsetfillcolor{textcolor}%
\pgftext[x=0.451388in,y=2.028072in,left,base]{\color{textcolor}\sffamily\fontsize{10.000000}{12.000000}\selectfont \(\displaystyle 10\)}%
\end{pgfscope}%
\begin{pgfscope}%
\pgfsetbuttcap%
\pgfsetroundjoin%
\definecolor{currentfill}{rgb}{0.000000,0.000000,0.000000}%
\pgfsetfillcolor{currentfill}%
\pgfsetlinewidth{0.803000pt}%
\definecolor{currentstroke}{rgb}{0.000000,0.000000,0.000000}%
\pgfsetstrokecolor{currentstroke}%
\pgfsetdash{}{0pt}%
\pgfsys@defobject{currentmarker}{\pgfqpoint{-0.048611in}{0.000000in}}{\pgfqpoint{0.000000in}{0.000000in}}{%
\pgfpathmoveto{\pgfqpoint{0.000000in}{0.000000in}}%
\pgfpathlineto{\pgfqpoint{-0.048611in}{0.000000in}}%
\pgfusepath{stroke,fill}%
}%
\begin{pgfscope}%
\pgfsys@transformshift{0.687500in}{2.508611in}%
\pgfsys@useobject{currentmarker}{}%
\end{pgfscope}%
\end{pgfscope}%
\begin{pgfscope}%
\definecolor{textcolor}{rgb}{0.000000,0.000000,0.000000}%
\pgfsetstrokecolor{textcolor}%
\pgfsetfillcolor{textcolor}%
\pgftext[x=0.451388in,y=2.455850in,left,base]{\color{textcolor}\sffamily\fontsize{10.000000}{12.000000}\selectfont \(\displaystyle 11\)}%
\end{pgfscope}%
\begin{pgfscope}%
\pgfsetbuttcap%
\pgfsetroundjoin%
\definecolor{currentfill}{rgb}{0.000000,0.000000,0.000000}%
\pgfsetfillcolor{currentfill}%
\pgfsetlinewidth{0.803000pt}%
\definecolor{currentstroke}{rgb}{0.000000,0.000000,0.000000}%
\pgfsetstrokecolor{currentstroke}%
\pgfsetdash{}{0pt}%
\pgfsys@defobject{currentmarker}{\pgfqpoint{-0.048611in}{0.000000in}}{\pgfqpoint{0.000000in}{0.000000in}}{%
\pgfpathmoveto{\pgfqpoint{0.000000in}{0.000000in}}%
\pgfpathlineto{\pgfqpoint{-0.048611in}{0.000000in}}%
\pgfusepath{stroke,fill}%
}%
\begin{pgfscope}%
\pgfsys@transformshift{0.687500in}{2.936389in}%
\pgfsys@useobject{currentmarker}{}%
\end{pgfscope}%
\end{pgfscope}%
\begin{pgfscope}%
\definecolor{textcolor}{rgb}{0.000000,0.000000,0.000000}%
\pgfsetstrokecolor{textcolor}%
\pgfsetfillcolor{textcolor}%
\pgftext[x=0.451388in,y=2.883627in,left,base]{\color{textcolor}\sffamily\fontsize{10.000000}{12.000000}\selectfont \(\displaystyle 12\)}%
\end{pgfscope}%
\begin{pgfscope}%
\pgfsetbuttcap%
\pgfsetroundjoin%
\definecolor{currentfill}{rgb}{0.000000,0.000000,0.000000}%
\pgfsetfillcolor{currentfill}%
\pgfsetlinewidth{0.803000pt}%
\definecolor{currentstroke}{rgb}{0.000000,0.000000,0.000000}%
\pgfsetstrokecolor{currentstroke}%
\pgfsetdash{}{0pt}%
\pgfsys@defobject{currentmarker}{\pgfqpoint{-0.048611in}{0.000000in}}{\pgfqpoint{0.000000in}{0.000000in}}{%
\pgfpathmoveto{\pgfqpoint{0.000000in}{0.000000in}}%
\pgfpathlineto{\pgfqpoint{-0.048611in}{0.000000in}}%
\pgfusepath{stroke,fill}%
}%
\begin{pgfscope}%
\pgfsys@transformshift{0.687500in}{3.364167in}%
\pgfsys@useobject{currentmarker}{}%
\end{pgfscope}%
\end{pgfscope}%
\begin{pgfscope}%
\definecolor{textcolor}{rgb}{0.000000,0.000000,0.000000}%
\pgfsetstrokecolor{textcolor}%
\pgfsetfillcolor{textcolor}%
\pgftext[x=0.451388in,y=3.311405in,left,base]{\color{textcolor}\sffamily\fontsize{10.000000}{12.000000}\selectfont \(\displaystyle 13\)}%
\end{pgfscope}%
\begin{pgfscope}%
\pgfsetbuttcap%
\pgfsetroundjoin%
\definecolor{currentfill}{rgb}{0.000000,0.000000,0.000000}%
\pgfsetfillcolor{currentfill}%
\pgfsetlinewidth{0.803000pt}%
\definecolor{currentstroke}{rgb}{0.000000,0.000000,0.000000}%
\pgfsetstrokecolor{currentstroke}%
\pgfsetdash{}{0pt}%
\pgfsys@defobject{currentmarker}{\pgfqpoint{-0.048611in}{0.000000in}}{\pgfqpoint{0.000000in}{0.000000in}}{%
\pgfpathmoveto{\pgfqpoint{0.000000in}{0.000000in}}%
\pgfpathlineto{\pgfqpoint{-0.048611in}{0.000000in}}%
\pgfusepath{stroke,fill}%
}%
\begin{pgfscope}%
\pgfsys@transformshift{0.687500in}{3.791944in}%
\pgfsys@useobject{currentmarker}{}%
\end{pgfscope}%
\end{pgfscope}%
\begin{pgfscope}%
\definecolor{textcolor}{rgb}{0.000000,0.000000,0.000000}%
\pgfsetstrokecolor{textcolor}%
\pgfsetfillcolor{textcolor}%
\pgftext[x=0.451388in,y=3.739183in,left,base]{\color{textcolor}\sffamily\fontsize{10.000000}{12.000000}\selectfont \(\displaystyle 14\)}%
\end{pgfscope}%
\begin{pgfscope}%
\pgfsetbuttcap%
\pgfsetroundjoin%
\definecolor{currentfill}{rgb}{0.000000,0.000000,0.000000}%
\pgfsetfillcolor{currentfill}%
\pgfsetlinewidth{0.803000pt}%
\definecolor{currentstroke}{rgb}{0.000000,0.000000,0.000000}%
\pgfsetstrokecolor{currentstroke}%
\pgfsetdash{}{0pt}%
\pgfsys@defobject{currentmarker}{\pgfqpoint{-0.048611in}{0.000000in}}{\pgfqpoint{0.000000in}{0.000000in}}{%
\pgfpathmoveto{\pgfqpoint{0.000000in}{0.000000in}}%
\pgfpathlineto{\pgfqpoint{-0.048611in}{0.000000in}}%
\pgfusepath{stroke,fill}%
}%
\begin{pgfscope}%
\pgfsys@transformshift{0.687500in}{4.219722in}%
\pgfsys@useobject{currentmarker}{}%
\end{pgfscope}%
\end{pgfscope}%
\begin{pgfscope}%
\definecolor{textcolor}{rgb}{0.000000,0.000000,0.000000}%
\pgfsetstrokecolor{textcolor}%
\pgfsetfillcolor{textcolor}%
\pgftext[x=0.451388in,y=4.166961in,left,base]{\color{textcolor}\sffamily\fontsize{10.000000}{12.000000}\selectfont \(\displaystyle 15\)}%
\end{pgfscope}%
\begin{pgfscope}%
\pgfsetbuttcap%
\pgfsetroundjoin%
\definecolor{currentfill}{rgb}{0.000000,0.000000,0.000000}%
\pgfsetfillcolor{currentfill}%
\pgfsetlinewidth{0.803000pt}%
\definecolor{currentstroke}{rgb}{0.000000,0.000000,0.000000}%
\pgfsetstrokecolor{currentstroke}%
\pgfsetdash{}{0pt}%
\pgfsys@defobject{currentmarker}{\pgfqpoint{-0.048611in}{0.000000in}}{\pgfqpoint{0.000000in}{0.000000in}}{%
\pgfpathmoveto{\pgfqpoint{0.000000in}{0.000000in}}%
\pgfpathlineto{\pgfqpoint{-0.048611in}{0.000000in}}%
\pgfusepath{stroke,fill}%
}%
\begin{pgfscope}%
\pgfsys@transformshift{0.687500in}{4.647500in}%
\pgfsys@useobject{currentmarker}{}%
\end{pgfscope}%
\end{pgfscope}%
\begin{pgfscope}%
\definecolor{textcolor}{rgb}{0.000000,0.000000,0.000000}%
\pgfsetstrokecolor{textcolor}%
\pgfsetfillcolor{textcolor}%
\pgftext[x=0.451388in,y=4.594738in,left,base]{\color{textcolor}\sffamily\fontsize{10.000000}{12.000000}\selectfont \(\displaystyle 16\)}%
\end{pgfscope}%
\begin{pgfscope}%
\definecolor{textcolor}{rgb}{0.000000,0.000000,0.000000}%
\pgfsetstrokecolor{textcolor}%
\pgfsetfillcolor{textcolor}%
\pgftext[x=0.395833in,y=2.722500in,,bottom,rotate=90.000000]{\color{textcolor}\sffamily\fontsize{10.000000}{12.000000}\selectfont Maximum Number of GMRES Iterations}%
\end{pgfscope}%
\begin{pgfscope}%
\pgfpathrectangle{\pgfqpoint{0.687500in}{0.605000in}}{\pgfqpoint{4.262500in}{4.235000in}}%
\pgfusepath{clip}%
\pgfsetbuttcap%
\pgfsetroundjoin%
\pgfsetlinewidth{1.505625pt}%
\definecolor{currentstroke}{rgb}{0.843137,0.000000,0.000000}%
\pgfsetstrokecolor{currentstroke}%
\pgfsetdash{{5.550000pt}{2.400000pt}}{0.000000pt}%
\pgfpathmoveto{\pgfqpoint{0.881250in}{2.508611in}}%
\pgfpathlineto{\pgfqpoint{2.172917in}{3.364167in}}%
\pgfpathlineto{\pgfqpoint{3.464583in}{3.791944in}}%
\pgfpathlineto{\pgfqpoint{4.756250in}{4.647500in}}%
\pgfusepath{stroke}%
\end{pgfscope}%
\begin{pgfscope}%
\pgfpathrectangle{\pgfqpoint{0.687500in}{0.605000in}}{\pgfqpoint{4.262500in}{4.235000in}}%
\pgfusepath{clip}%
\pgfsetbuttcap%
\pgfsetroundjoin%
\definecolor{currentfill}{rgb}{0.843137,0.000000,0.000000}%
\pgfsetfillcolor{currentfill}%
\pgfsetlinewidth{1.003750pt}%
\definecolor{currentstroke}{rgb}{0.843137,0.000000,0.000000}%
\pgfsetstrokecolor{currentstroke}%
\pgfsetdash{}{0pt}%
\pgfsys@defobject{currentmarker}{\pgfqpoint{-0.041667in}{-0.041667in}}{\pgfqpoint{0.041667in}{0.041667in}}{%
\pgfpathmoveto{\pgfqpoint{0.000000in}{-0.041667in}}%
\pgfpathcurveto{\pgfqpoint{0.011050in}{-0.041667in}}{\pgfqpoint{0.021649in}{-0.037276in}}{\pgfqpoint{0.029463in}{-0.029463in}}%
\pgfpathcurveto{\pgfqpoint{0.037276in}{-0.021649in}}{\pgfqpoint{0.041667in}{-0.011050in}}{\pgfqpoint{0.041667in}{0.000000in}}%
\pgfpathcurveto{\pgfqpoint{0.041667in}{0.011050in}}{\pgfqpoint{0.037276in}{0.021649in}}{\pgfqpoint{0.029463in}{0.029463in}}%
\pgfpathcurveto{\pgfqpoint{0.021649in}{0.037276in}}{\pgfqpoint{0.011050in}{0.041667in}}{\pgfqpoint{0.000000in}{0.041667in}}%
\pgfpathcurveto{\pgfqpoint{-0.011050in}{0.041667in}}{\pgfqpoint{-0.021649in}{0.037276in}}{\pgfqpoint{-0.029463in}{0.029463in}}%
\pgfpathcurveto{\pgfqpoint{-0.037276in}{0.021649in}}{\pgfqpoint{-0.041667in}{0.011050in}}{\pgfqpoint{-0.041667in}{0.000000in}}%
\pgfpathcurveto{\pgfqpoint{-0.041667in}{-0.011050in}}{\pgfqpoint{-0.037276in}{-0.021649in}}{\pgfqpoint{-0.029463in}{-0.029463in}}%
\pgfpathcurveto{\pgfqpoint{-0.021649in}{-0.037276in}}{\pgfqpoint{-0.011050in}{-0.041667in}}{\pgfqpoint{0.000000in}{-0.041667in}}%
\pgfpathclose%
\pgfusepath{stroke,fill}%
}%
\begin{pgfscope}%
\pgfsys@transformshift{0.881250in}{2.508611in}%
\pgfsys@useobject{currentmarker}{}%
\end{pgfscope}%
\begin{pgfscope}%
\pgfsys@transformshift{2.172917in}{3.364167in}%
\pgfsys@useobject{currentmarker}{}%
\end{pgfscope}%
\begin{pgfscope}%
\pgfsys@transformshift{3.464583in}{3.791944in}%
\pgfsys@useobject{currentmarker}{}%
\end{pgfscope}%
\begin{pgfscope}%
\pgfsys@transformshift{4.756250in}{4.647500in}%
\pgfsys@useobject{currentmarker}{}%
\end{pgfscope}%
\end{pgfscope}%
\begin{pgfscope}%
\pgfpathrectangle{\pgfqpoint{0.687500in}{0.605000in}}{\pgfqpoint{4.262500in}{4.235000in}}%
\pgfusepath{clip}%
\pgfsetbuttcap%
\pgfsetroundjoin%
\pgfsetlinewidth{1.505625pt}%
\definecolor{currentstroke}{rgb}{0.549020,0.235294,1.000000}%
\pgfsetstrokecolor{currentstroke}%
\pgfsetdash{{5.550000pt}{2.400000pt}}{0.000000pt}%
\pgfpathmoveto{\pgfqpoint{0.881250in}{2.080833in}}%
\pgfpathlineto{\pgfqpoint{2.172917in}{2.080833in}}%
\pgfpathlineto{\pgfqpoint{3.464583in}{2.508611in}}%
\pgfpathlineto{\pgfqpoint{4.756250in}{2.936389in}}%
\pgfusepath{stroke}%
\end{pgfscope}%
\begin{pgfscope}%
\pgfpathrectangle{\pgfqpoint{0.687500in}{0.605000in}}{\pgfqpoint{4.262500in}{4.235000in}}%
\pgfusepath{clip}%
\pgfsetbuttcap%
\pgfsetmiterjoin%
\definecolor{currentfill}{rgb}{0.549020,0.235294,1.000000}%
\pgfsetfillcolor{currentfill}%
\pgfsetlinewidth{1.003750pt}%
\definecolor{currentstroke}{rgb}{0.549020,0.235294,1.000000}%
\pgfsetstrokecolor{currentstroke}%
\pgfsetdash{}{0pt}%
\pgfsys@defobject{currentmarker}{\pgfqpoint{-0.041667in}{-0.041667in}}{\pgfqpoint{0.041667in}{0.041667in}}{%
\pgfpathmoveto{\pgfqpoint{0.000000in}{0.041667in}}%
\pgfpathlineto{\pgfqpoint{-0.041667in}{-0.041667in}}%
\pgfpathlineto{\pgfqpoint{0.041667in}{-0.041667in}}%
\pgfpathclose%
\pgfusepath{stroke,fill}%
}%
\begin{pgfscope}%
\pgfsys@transformshift{0.881250in}{2.080833in}%
\pgfsys@useobject{currentmarker}{}%
\end{pgfscope}%
\begin{pgfscope}%
\pgfsys@transformshift{2.172917in}{2.080833in}%
\pgfsys@useobject{currentmarker}{}%
\end{pgfscope}%
\begin{pgfscope}%
\pgfsys@transformshift{3.464583in}{2.508611in}%
\pgfsys@useobject{currentmarker}{}%
\end{pgfscope}%
\begin{pgfscope}%
\pgfsys@transformshift{4.756250in}{2.936389in}%
\pgfsys@useobject{currentmarker}{}%
\end{pgfscope}%
\end{pgfscope}%
\begin{pgfscope}%
\pgfpathrectangle{\pgfqpoint{0.687500in}{0.605000in}}{\pgfqpoint{4.262500in}{4.235000in}}%
\pgfusepath{clip}%
\pgfsetbuttcap%
\pgfsetroundjoin%
\pgfsetlinewidth{1.505625pt}%
\definecolor{currentstroke}{rgb}{0.007843,0.533333,0.000000}%
\pgfsetstrokecolor{currentstroke}%
\pgfsetdash{{5.550000pt}{2.400000pt}}{0.000000pt}%
\pgfpathmoveto{\pgfqpoint{0.881250in}{1.225278in}}%
\pgfpathlineto{\pgfqpoint{2.172917in}{1.653056in}}%
\pgfpathlineto{\pgfqpoint{3.464583in}{1.653056in}}%
\pgfpathlineto{\pgfqpoint{4.756250in}{1.653056in}}%
\pgfusepath{stroke}%
\end{pgfscope}%
\begin{pgfscope}%
\pgfpathrectangle{\pgfqpoint{0.687500in}{0.605000in}}{\pgfqpoint{4.262500in}{4.235000in}}%
\pgfusepath{clip}%
\pgfsetbuttcap%
\pgfsetmiterjoin%
\definecolor{currentfill}{rgb}{0.007843,0.533333,0.000000}%
\pgfsetfillcolor{currentfill}%
\pgfsetlinewidth{1.003750pt}%
\definecolor{currentstroke}{rgb}{0.007843,0.533333,0.000000}%
\pgfsetstrokecolor{currentstroke}%
\pgfsetdash{}{0pt}%
\pgfsys@defobject{currentmarker}{\pgfqpoint{-0.041667in}{-0.041667in}}{\pgfqpoint{0.041667in}{0.041667in}}{%
\pgfpathmoveto{\pgfqpoint{-0.000000in}{-0.041667in}}%
\pgfpathlineto{\pgfqpoint{0.041667in}{0.041667in}}%
\pgfpathlineto{\pgfqpoint{-0.041667in}{0.041667in}}%
\pgfpathclose%
\pgfusepath{stroke,fill}%
}%
\begin{pgfscope}%
\pgfsys@transformshift{0.881250in}{1.225278in}%
\pgfsys@useobject{currentmarker}{}%
\end{pgfscope}%
\begin{pgfscope}%
\pgfsys@transformshift{2.172917in}{1.653056in}%
\pgfsys@useobject{currentmarker}{}%
\end{pgfscope}%
\begin{pgfscope}%
\pgfsys@transformshift{3.464583in}{1.653056in}%
\pgfsys@useobject{currentmarker}{}%
\end{pgfscope}%
\begin{pgfscope}%
\pgfsys@transformshift{4.756250in}{1.653056in}%
\pgfsys@useobject{currentmarker}{}%
\end{pgfscope}%
\end{pgfscope}%
\begin{pgfscope}%
\pgfpathrectangle{\pgfqpoint{0.687500in}{0.605000in}}{\pgfqpoint{4.262500in}{4.235000in}}%
\pgfusepath{clip}%
\pgfsetbuttcap%
\pgfsetroundjoin%
\pgfsetlinewidth{1.505625pt}%
\definecolor{currentstroke}{rgb}{0.000000,0.674510,0.780392}%
\pgfsetstrokecolor{currentstroke}%
\pgfsetdash{{5.550000pt}{2.400000pt}}{0.000000pt}%
\pgfpathmoveto{\pgfqpoint{0.881250in}{0.797500in}}%
\pgfpathlineto{\pgfqpoint{2.172917in}{1.225278in}}%
\pgfpathlineto{\pgfqpoint{3.464583in}{1.225278in}}%
\pgfpathlineto{\pgfqpoint{4.756250in}{1.225278in}}%
\pgfusepath{stroke}%
\end{pgfscope}%
\begin{pgfscope}%
\pgfpathrectangle{\pgfqpoint{0.687500in}{0.605000in}}{\pgfqpoint{4.262500in}{4.235000in}}%
\pgfusepath{clip}%
\pgfsetbuttcap%
\pgfsetmiterjoin%
\definecolor{currentfill}{rgb}{0.000000,0.674510,0.780392}%
\pgfsetfillcolor{currentfill}%
\pgfsetlinewidth{1.003750pt}%
\definecolor{currentstroke}{rgb}{0.000000,0.674510,0.780392}%
\pgfsetstrokecolor{currentstroke}%
\pgfsetdash{}{0pt}%
\pgfsys@defobject{currentmarker}{\pgfqpoint{-0.041667in}{-0.041667in}}{\pgfqpoint{0.041667in}{0.041667in}}{%
\pgfpathmoveto{\pgfqpoint{0.041667in}{-0.000000in}}%
\pgfpathlineto{\pgfqpoint{-0.041667in}{0.041667in}}%
\pgfpathlineto{\pgfqpoint{-0.041667in}{-0.041667in}}%
\pgfpathclose%
\pgfusepath{stroke,fill}%
}%
\begin{pgfscope}%
\pgfsys@transformshift{0.881250in}{0.797500in}%
\pgfsys@useobject{currentmarker}{}%
\end{pgfscope}%
\begin{pgfscope}%
\pgfsys@transformshift{2.172917in}{1.225278in}%
\pgfsys@useobject{currentmarker}{}%
\end{pgfscope}%
\begin{pgfscope}%
\pgfsys@transformshift{3.464583in}{1.225278in}%
\pgfsys@useobject{currentmarker}{}%
\end{pgfscope}%
\begin{pgfscope}%
\pgfsys@transformshift{4.756250in}{1.225278in}%
\pgfsys@useobject{currentmarker}{}%
\end{pgfscope}%
\end{pgfscope}%
\begin{pgfscope}%
\pgfsetrectcap%
\pgfsetmiterjoin%
\pgfsetlinewidth{0.803000pt}%
\definecolor{currentstroke}{rgb}{0.000000,0.000000,0.000000}%
\pgfsetstrokecolor{currentstroke}%
\pgfsetdash{}{0pt}%
\pgfpathmoveto{\pgfqpoint{0.687500in}{0.605000in}}%
\pgfpathlineto{\pgfqpoint{0.687500in}{4.840000in}}%
\pgfusepath{stroke}%
\end{pgfscope}%
\begin{pgfscope}%
\pgfsetrectcap%
\pgfsetmiterjoin%
\pgfsetlinewidth{0.000000pt}%
\definecolor{currentstroke}{rgb}{0.000000,0.000000,0.000000}%
\pgfsetstrokecolor{currentstroke}%
\pgfsetstrokeopacity{0.000000}%
\pgfsetdash{}{0pt}%
\pgfpathmoveto{\pgfqpoint{4.950000in}{0.605000in}}%
\pgfpathlineto{\pgfqpoint{4.950000in}{4.840000in}}%
\pgfusepath{}%
\end{pgfscope}%
\begin{pgfscope}%
\pgfsetrectcap%
\pgfsetmiterjoin%
\pgfsetlinewidth{0.803000pt}%
\definecolor{currentstroke}{rgb}{0.000000,0.000000,0.000000}%
\pgfsetstrokecolor{currentstroke}%
\pgfsetdash{}{0pt}%
\pgfpathmoveto{\pgfqpoint{0.687500in}{0.605000in}}%
\pgfpathlineto{\pgfqpoint{4.950000in}{0.605000in}}%
\pgfusepath{stroke}%
\end{pgfscope}%
\begin{pgfscope}%
\pgfsetrectcap%
\pgfsetmiterjoin%
\pgfsetlinewidth{0.000000pt}%
\definecolor{currentstroke}{rgb}{0.000000,0.000000,0.000000}%
\pgfsetstrokecolor{currentstroke}%
\pgfsetstrokeopacity{0.000000}%
\pgfsetdash{}{0pt}%
\pgfpathmoveto{\pgfqpoint{0.687500in}{4.840000in}}%
\pgfpathlineto{\pgfqpoint{4.950000in}{4.840000in}}%
\pgfusepath{}%
\end{pgfscope}%
\begin{pgfscope}%
\pgfsetbuttcap%
\pgfsetmiterjoin%
\definecolor{currentfill}{rgb}{1.000000,1.000000,1.000000}%
\pgfsetfillcolor{currentfill}%
\pgfsetfillopacity{0.800000}%
\pgfsetlinewidth{1.003750pt}%
\definecolor{currentstroke}{rgb}{0.800000,0.800000,0.800000}%
\pgfsetstrokecolor{currentstroke}%
\pgfsetstrokeopacity{0.800000}%
\pgfsetdash{}{0pt}%
\pgfpathmoveto{\pgfqpoint{0.784722in}{3.913460in}}%
\pgfpathlineto{\pgfqpoint{1.682540in}{3.913460in}}%
\pgfpathquadraticcurveto{\pgfqpoint{1.710317in}{3.913460in}}{\pgfqpoint{1.710317in}{3.941238in}}%
\pgfpathlineto{\pgfqpoint{1.710317in}{4.742778in}}%
\pgfpathquadraticcurveto{\pgfqpoint{1.710317in}{4.770556in}}{\pgfqpoint{1.682540in}{4.770556in}}%
\pgfpathlineto{\pgfqpoint{0.784722in}{4.770556in}}%
\pgfpathquadraticcurveto{\pgfqpoint{0.756944in}{4.770556in}}{\pgfqpoint{0.756944in}{4.742778in}}%
\pgfpathlineto{\pgfqpoint{0.756944in}{3.941238in}}%
\pgfpathquadraticcurveto{\pgfqpoint{0.756944in}{3.913460in}}{\pgfqpoint{0.784722in}{3.913460in}}%
\pgfpathclose%
\pgfusepath{stroke,fill}%
\end{pgfscope}%
\begin{pgfscope}%
\pgfsetbuttcap%
\pgfsetroundjoin%
\pgfsetlinewidth{1.505625pt}%
\definecolor{currentstroke}{rgb}{0.843137,0.000000,0.000000}%
\pgfsetstrokecolor{currentstroke}%
\pgfsetdash{{5.550000pt}{2.400000pt}}{0.000000pt}%
\pgfpathmoveto{\pgfqpoint{0.812500in}{4.658088in}}%
\pgfpathlineto{\pgfqpoint{1.090278in}{4.658088in}}%
\pgfusepath{stroke}%
\end{pgfscope}%
\begin{pgfscope}%
\pgfsetbuttcap%
\pgfsetroundjoin%
\definecolor{currentfill}{rgb}{0.843137,0.000000,0.000000}%
\pgfsetfillcolor{currentfill}%
\pgfsetlinewidth{1.003750pt}%
\definecolor{currentstroke}{rgb}{0.843137,0.000000,0.000000}%
\pgfsetstrokecolor{currentstroke}%
\pgfsetdash{}{0pt}%
\pgfsys@defobject{currentmarker}{\pgfqpoint{-0.041667in}{-0.041667in}}{\pgfqpoint{0.041667in}{0.041667in}}{%
\pgfpathmoveto{\pgfqpoint{0.000000in}{-0.041667in}}%
\pgfpathcurveto{\pgfqpoint{0.011050in}{-0.041667in}}{\pgfqpoint{0.021649in}{-0.037276in}}{\pgfqpoint{0.029463in}{-0.029463in}}%
\pgfpathcurveto{\pgfqpoint{0.037276in}{-0.021649in}}{\pgfqpoint{0.041667in}{-0.011050in}}{\pgfqpoint{0.041667in}{0.000000in}}%
\pgfpathcurveto{\pgfqpoint{0.041667in}{0.011050in}}{\pgfqpoint{0.037276in}{0.021649in}}{\pgfqpoint{0.029463in}{0.029463in}}%
\pgfpathcurveto{\pgfqpoint{0.021649in}{0.037276in}}{\pgfqpoint{0.011050in}{0.041667in}}{\pgfqpoint{0.000000in}{0.041667in}}%
\pgfpathcurveto{\pgfqpoint{-0.011050in}{0.041667in}}{\pgfqpoint{-0.021649in}{0.037276in}}{\pgfqpoint{-0.029463in}{0.029463in}}%
\pgfpathcurveto{\pgfqpoint{-0.037276in}{0.021649in}}{\pgfqpoint{-0.041667in}{0.011050in}}{\pgfqpoint{-0.041667in}{0.000000in}}%
\pgfpathcurveto{\pgfqpoint{-0.041667in}{-0.011050in}}{\pgfqpoint{-0.037276in}{-0.021649in}}{\pgfqpoint{-0.029463in}{-0.029463in}}%
\pgfpathcurveto{\pgfqpoint{-0.021649in}{-0.037276in}}{\pgfqpoint{-0.011050in}{-0.041667in}}{\pgfqpoint{0.000000in}{-0.041667in}}%
\pgfpathclose%
\pgfusepath{stroke,fill}%
}%
\begin{pgfscope}%
\pgfsys@transformshift{0.951389in}{4.658088in}%
\pgfsys@useobject{currentmarker}{}%
\end{pgfscope}%
\end{pgfscope}%
\begin{pgfscope}%
\definecolor{textcolor}{rgb}{0.000000,0.000000,0.000000}%
\pgfsetstrokecolor{textcolor}%
\pgfsetfillcolor{textcolor}%
\pgftext[x=1.201389in,y=4.609477in,left,base]{\color{textcolor}\sffamily\fontsize{10.000000}{12.000000}\selectfont \(\displaystyle \beta = \)0.4}%
\end{pgfscope}%
\begin{pgfscope}%
\pgfsetbuttcap%
\pgfsetroundjoin%
\pgfsetlinewidth{1.505625pt}%
\definecolor{currentstroke}{rgb}{0.549020,0.235294,1.000000}%
\pgfsetstrokecolor{currentstroke}%
\pgfsetdash{{5.550000pt}{2.400000pt}}{0.000000pt}%
\pgfpathmoveto{\pgfqpoint{0.812500in}{4.454231in}}%
\pgfpathlineto{\pgfqpoint{1.090278in}{4.454231in}}%
\pgfusepath{stroke}%
\end{pgfscope}%
\begin{pgfscope}%
\pgfsetbuttcap%
\pgfsetmiterjoin%
\definecolor{currentfill}{rgb}{0.549020,0.235294,1.000000}%
\pgfsetfillcolor{currentfill}%
\pgfsetlinewidth{1.003750pt}%
\definecolor{currentstroke}{rgb}{0.549020,0.235294,1.000000}%
\pgfsetstrokecolor{currentstroke}%
\pgfsetdash{}{0pt}%
\pgfsys@defobject{currentmarker}{\pgfqpoint{-0.041667in}{-0.041667in}}{\pgfqpoint{0.041667in}{0.041667in}}{%
\pgfpathmoveto{\pgfqpoint{0.000000in}{0.041667in}}%
\pgfpathlineto{\pgfqpoint{-0.041667in}{-0.041667in}}%
\pgfpathlineto{\pgfqpoint{0.041667in}{-0.041667in}}%
\pgfpathclose%
\pgfusepath{stroke,fill}%
}%
\begin{pgfscope}%
\pgfsys@transformshift{0.951389in}{4.454231in}%
\pgfsys@useobject{currentmarker}{}%
\end{pgfscope}%
\end{pgfscope}%
\begin{pgfscope}%
\definecolor{textcolor}{rgb}{0.000000,0.000000,0.000000}%
\pgfsetstrokecolor{textcolor}%
\pgfsetfillcolor{textcolor}%
\pgftext[x=1.201389in,y=4.405620in,left,base]{\color{textcolor}\sffamily\fontsize{10.000000}{12.000000}\selectfont \(\displaystyle \beta = \)0.5}%
\end{pgfscope}%
\begin{pgfscope}%
\pgfsetbuttcap%
\pgfsetroundjoin%
\pgfsetlinewidth{1.505625pt}%
\definecolor{currentstroke}{rgb}{0.007843,0.533333,0.000000}%
\pgfsetstrokecolor{currentstroke}%
\pgfsetdash{{5.550000pt}{2.400000pt}}{0.000000pt}%
\pgfpathmoveto{\pgfqpoint{0.812500in}{4.250374in}}%
\pgfpathlineto{\pgfqpoint{1.090278in}{4.250374in}}%
\pgfusepath{stroke}%
\end{pgfscope}%
\begin{pgfscope}%
\pgfsetbuttcap%
\pgfsetmiterjoin%
\definecolor{currentfill}{rgb}{0.007843,0.533333,0.000000}%
\pgfsetfillcolor{currentfill}%
\pgfsetlinewidth{1.003750pt}%
\definecolor{currentstroke}{rgb}{0.007843,0.533333,0.000000}%
\pgfsetstrokecolor{currentstroke}%
\pgfsetdash{}{0pt}%
\pgfsys@defobject{currentmarker}{\pgfqpoint{-0.041667in}{-0.041667in}}{\pgfqpoint{0.041667in}{0.041667in}}{%
\pgfpathmoveto{\pgfqpoint{-0.000000in}{-0.041667in}}%
\pgfpathlineto{\pgfqpoint{0.041667in}{0.041667in}}%
\pgfpathlineto{\pgfqpoint{-0.041667in}{0.041667in}}%
\pgfpathclose%
\pgfusepath{stroke,fill}%
}%
\begin{pgfscope}%
\pgfsys@transformshift{0.951389in}{4.250374in}%
\pgfsys@useobject{currentmarker}{}%
\end{pgfscope}%
\end{pgfscope}%
\begin{pgfscope}%
\definecolor{textcolor}{rgb}{0.000000,0.000000,0.000000}%
\pgfsetstrokecolor{textcolor}%
\pgfsetfillcolor{textcolor}%
\pgftext[x=1.201389in,y=4.201763in,left,base]{\color{textcolor}\sffamily\fontsize{10.000000}{12.000000}\selectfont \(\displaystyle \beta = \)0.6}%
\end{pgfscope}%
\begin{pgfscope}%
\pgfsetbuttcap%
\pgfsetroundjoin%
\pgfsetlinewidth{1.505625pt}%
\definecolor{currentstroke}{rgb}{0.000000,0.674510,0.780392}%
\pgfsetstrokecolor{currentstroke}%
\pgfsetdash{{5.550000pt}{2.400000pt}}{0.000000pt}%
\pgfpathmoveto{\pgfqpoint{0.812500in}{4.046516in}}%
\pgfpathlineto{\pgfqpoint{1.090278in}{4.046516in}}%
\pgfusepath{stroke}%
\end{pgfscope}%
\begin{pgfscope}%
\pgfsetbuttcap%
\pgfsetmiterjoin%
\definecolor{currentfill}{rgb}{0.000000,0.674510,0.780392}%
\pgfsetfillcolor{currentfill}%
\pgfsetlinewidth{1.003750pt}%
\definecolor{currentstroke}{rgb}{0.000000,0.674510,0.780392}%
\pgfsetstrokecolor{currentstroke}%
\pgfsetdash{}{0pt}%
\pgfsys@defobject{currentmarker}{\pgfqpoint{-0.041667in}{-0.041667in}}{\pgfqpoint{0.041667in}{0.041667in}}{%
\pgfpathmoveto{\pgfqpoint{0.041667in}{-0.000000in}}%
\pgfpathlineto{\pgfqpoint{-0.041667in}{0.041667in}}%
\pgfpathlineto{\pgfqpoint{-0.041667in}{-0.041667in}}%
\pgfpathclose%
\pgfusepath{stroke,fill}%
}%
\begin{pgfscope}%
\pgfsys@transformshift{0.951389in}{4.046516in}%
\pgfsys@useobject{currentmarker}{}%
\end{pgfscope}%
\end{pgfscope}%
\begin{pgfscope}%
\definecolor{textcolor}{rgb}{0.000000,0.000000,0.000000}%
\pgfsetstrokecolor{textcolor}%
\pgfsetfillcolor{textcolor}%
\pgftext[x=1.201389in,y=3.997905in,left,base]{\color{textcolor}\sffamily\fontsize{10.000000}{12.000000}\selectfont \(\displaystyle \beta = \)0.7}%
\end{pgfscope}%
\end{pgfpicture}%
\makeatother%
\endgroup%

   \caption[Maximum GMRES iteration counts when $\NLiDRR{\nso-\nst} = 0.5\times  k^{-\beta}$ for $\beta = 0.4,0.5,0.6,0.7.$]{Maximum GMRES iteration counts for solving systems with matrix $\AmatoI\Amatt$, where $\Aso=\Ast=1$ and $\NLiDRR{\nso-\nst} = 0.5\times  k^{-\beta}$ for $\beta = 0.4,0.5,0.6,0.7.$}\label{fig:linfinityn1}
\end{figure}

    \begin{figure}
      \centering
%% Creator: Matplotlib, PGF backend
%%
%% To include the figure in your LaTeX document, write
%%   \input{<filename>.pgf}
%%
%% Make sure the required packages are loaded in your preamble
%%   \usepackage{pgf}
%%
%% Figures using additional raster images can only be included by \input if
%% they are in the same directory as the main LaTeX file. For loading figures
%% from other directories you can use the `import` package
%%   \usepackage{import}
%% and then include the figures with
%%   \import{<path to file>}{<filename>.pgf}
%%
%% Matplotlib used the following preamble
%%   \usepackage{fontspec}
%%   \setmainfont{DejaVuSerif.ttf}[Path=/home/owen/progs/firedrake-complex/firedrake/lib/python3.5/site-packages/matplotlib/mpl-data/fonts/ttf/]
%%   \setsansfont{DejaVuSans.ttf}[Path=/home/owen/progs/firedrake-complex/firedrake/lib/python3.5/site-packages/matplotlib/mpl-data/fonts/ttf/]
%%   \setmonofont{DejaVuSansMono.ttf}[Path=/home/owen/progs/firedrake-complex/firedrake/lib/python3.5/site-packages/matplotlib/mpl-data/fonts/ttf/]
%%
\begingroup%
\makeatletter%
\begin{pgfpicture}%
\pgfpathrectangle{\pgfpointorigin}{\pgfqpoint{3.000000in}{3.000000in}}%
\pgfusepath{use as bounding box, clip}%
\begin{pgfscope}%
\pgfsetbuttcap%
\pgfsetmiterjoin%
\definecolor{currentfill}{rgb}{1.000000,1.000000,1.000000}%
\pgfsetfillcolor{currentfill}%
\pgfsetlinewidth{0.000000pt}%
\definecolor{currentstroke}{rgb}{1.000000,1.000000,1.000000}%
\pgfsetstrokecolor{currentstroke}%
\pgfsetdash{}{0pt}%
\pgfpathmoveto{\pgfqpoint{0.000000in}{0.000000in}}%
\pgfpathlineto{\pgfqpoint{3.000000in}{0.000000in}}%
\pgfpathlineto{\pgfqpoint{3.000000in}{3.000000in}}%
\pgfpathlineto{\pgfqpoint{0.000000in}{3.000000in}}%
\pgfpathclose%
\pgfusepath{fill}%
\end{pgfscope}%
\begin{pgfscope}%
\pgfsetbuttcap%
\pgfsetmiterjoin%
\definecolor{currentfill}{rgb}{1.000000,1.000000,1.000000}%
\pgfsetfillcolor{currentfill}%
\pgfsetlinewidth{0.000000pt}%
\definecolor{currentstroke}{rgb}{0.000000,0.000000,0.000000}%
\pgfsetstrokecolor{currentstroke}%
\pgfsetstrokeopacity{0.000000}%
\pgfsetdash{}{0pt}%
\pgfpathmoveto{\pgfqpoint{0.375000in}{0.330000in}}%
\pgfpathlineto{\pgfqpoint{2.700000in}{0.330000in}}%
\pgfpathlineto{\pgfqpoint{2.700000in}{2.640000in}}%
\pgfpathlineto{\pgfqpoint{0.375000in}{2.640000in}}%
\pgfpathclose%
\pgfusepath{fill}%
\end{pgfscope}%
\begin{pgfscope}%
\pgfpathrectangle{\pgfqpoint{0.375000in}{0.330000in}}{\pgfqpoint{2.325000in}{2.310000in}}%
\pgfusepath{clip}%
\pgfsetbuttcap%
\pgfsetroundjoin%
\definecolor{currentfill}{rgb}{0.000000,0.000000,0.000000}%
\pgfsetfillcolor{currentfill}%
\pgfsetlinewidth{1.003750pt}%
\definecolor{currentstroke}{rgb}{0.000000,0.000000,0.000000}%
\pgfsetstrokecolor{currentstroke}%
\pgfsetdash{}{0pt}%
\pgfpathmoveto{\pgfqpoint{0.480841in}{0.446310in}}%
\pgfpathcurveto{\pgfqpoint{0.491891in}{0.446310in}}{\pgfqpoint{0.502490in}{0.450700in}}{\pgfqpoint{0.510303in}{0.458514in}}%
\pgfpathcurveto{\pgfqpoint{0.518117in}{0.466328in}}{\pgfqpoint{0.522507in}{0.476927in}}{\pgfqpoint{0.522507in}{0.487977in}}%
\pgfpathcurveto{\pgfqpoint{0.522507in}{0.499027in}}{\pgfqpoint{0.518117in}{0.509626in}}{\pgfqpoint{0.510303in}{0.517440in}}%
\pgfpathcurveto{\pgfqpoint{0.502490in}{0.525253in}}{\pgfqpoint{0.491891in}{0.529644in}}{\pgfqpoint{0.480841in}{0.529644in}}%
\pgfpathcurveto{\pgfqpoint{0.469790in}{0.529644in}}{\pgfqpoint{0.459191in}{0.525253in}}{\pgfqpoint{0.451378in}{0.517440in}}%
\pgfpathcurveto{\pgfqpoint{0.443564in}{0.509626in}}{\pgfqpoint{0.439174in}{0.499027in}}{\pgfqpoint{0.439174in}{0.487977in}}%
\pgfpathcurveto{\pgfqpoint{0.439174in}{0.476927in}}{\pgfqpoint{0.443564in}{0.466328in}}{\pgfqpoint{0.451378in}{0.458514in}}%
\pgfpathcurveto{\pgfqpoint{0.459191in}{0.450700in}}{\pgfqpoint{0.469790in}{0.446310in}}{\pgfqpoint{0.480841in}{0.446310in}}%
\pgfpathclose%
\pgfusepath{stroke,fill}%
\end{pgfscope}%
\begin{pgfscope}%
\pgfpathrectangle{\pgfqpoint{0.375000in}{0.330000in}}{\pgfqpoint{2.325000in}{2.310000in}}%
\pgfusepath{clip}%
\pgfsetbuttcap%
\pgfsetroundjoin%
\definecolor{currentfill}{rgb}{0.000000,0.000000,0.000000}%
\pgfsetfillcolor{currentfill}%
\pgfsetlinewidth{1.003750pt}%
\definecolor{currentstroke}{rgb}{0.000000,0.000000,0.000000}%
\pgfsetstrokecolor{currentstroke}%
\pgfsetdash{}{0pt}%
\pgfpathmoveto{\pgfqpoint{0.480841in}{0.446310in}}%
\pgfpathcurveto{\pgfqpoint{0.491891in}{0.446310in}}{\pgfqpoint{0.502490in}{0.450700in}}{\pgfqpoint{0.510303in}{0.458514in}}%
\pgfpathcurveto{\pgfqpoint{0.518117in}{0.466328in}}{\pgfqpoint{0.522507in}{0.476927in}}{\pgfqpoint{0.522507in}{0.487977in}}%
\pgfpathcurveto{\pgfqpoint{0.522507in}{0.499027in}}{\pgfqpoint{0.518117in}{0.509626in}}{\pgfqpoint{0.510303in}{0.517440in}}%
\pgfpathcurveto{\pgfqpoint{0.502490in}{0.525253in}}{\pgfqpoint{0.491891in}{0.529644in}}{\pgfqpoint{0.480841in}{0.529644in}}%
\pgfpathcurveto{\pgfqpoint{0.469790in}{0.529644in}}{\pgfqpoint{0.459191in}{0.525253in}}{\pgfqpoint{0.451378in}{0.517440in}}%
\pgfpathcurveto{\pgfqpoint{0.443564in}{0.509626in}}{\pgfqpoint{0.439174in}{0.499027in}}{\pgfqpoint{0.439174in}{0.487977in}}%
\pgfpathcurveto{\pgfqpoint{0.439174in}{0.476927in}}{\pgfqpoint{0.443564in}{0.466328in}}{\pgfqpoint{0.451378in}{0.458514in}}%
\pgfpathcurveto{\pgfqpoint{0.459191in}{0.450700in}}{\pgfqpoint{0.469790in}{0.446310in}}{\pgfqpoint{0.480841in}{0.446310in}}%
\pgfpathclose%
\pgfusepath{stroke,fill}%
\end{pgfscope}%
\begin{pgfscope}%
\pgfpathrectangle{\pgfqpoint{0.375000in}{0.330000in}}{\pgfqpoint{2.325000in}{2.310000in}}%
\pgfusepath{clip}%
\pgfsetbuttcap%
\pgfsetroundjoin%
\definecolor{currentfill}{rgb}{0.000000,0.000000,0.000000}%
\pgfsetfillcolor{currentfill}%
\pgfsetlinewidth{1.003750pt}%
\definecolor{currentstroke}{rgb}{0.000000,0.000000,0.000000}%
\pgfsetstrokecolor{currentstroke}%
\pgfsetdash{}{0pt}%
\pgfpathmoveto{\pgfqpoint{0.480841in}{0.446310in}}%
\pgfpathcurveto{\pgfqpoint{0.491891in}{0.446310in}}{\pgfqpoint{0.502490in}{0.450700in}}{\pgfqpoint{0.510303in}{0.458514in}}%
\pgfpathcurveto{\pgfqpoint{0.518117in}{0.466328in}}{\pgfqpoint{0.522507in}{0.476927in}}{\pgfqpoint{0.522507in}{0.487977in}}%
\pgfpathcurveto{\pgfqpoint{0.522507in}{0.499027in}}{\pgfqpoint{0.518117in}{0.509626in}}{\pgfqpoint{0.510303in}{0.517440in}}%
\pgfpathcurveto{\pgfqpoint{0.502490in}{0.525253in}}{\pgfqpoint{0.491891in}{0.529644in}}{\pgfqpoint{0.480841in}{0.529644in}}%
\pgfpathcurveto{\pgfqpoint{0.469790in}{0.529644in}}{\pgfqpoint{0.459191in}{0.525253in}}{\pgfqpoint{0.451378in}{0.517440in}}%
\pgfpathcurveto{\pgfqpoint{0.443564in}{0.509626in}}{\pgfqpoint{0.439174in}{0.499027in}}{\pgfqpoint{0.439174in}{0.487977in}}%
\pgfpathcurveto{\pgfqpoint{0.439174in}{0.476927in}}{\pgfqpoint{0.443564in}{0.466328in}}{\pgfqpoint{0.451378in}{0.458514in}}%
\pgfpathcurveto{\pgfqpoint{0.459191in}{0.450700in}}{\pgfqpoint{0.469790in}{0.446310in}}{\pgfqpoint{0.480841in}{0.446310in}}%
\pgfpathclose%
\pgfusepath{stroke,fill}%
\end{pgfscope}%
\begin{pgfscope}%
\pgfpathrectangle{\pgfqpoint{0.375000in}{0.330000in}}{\pgfqpoint{2.325000in}{2.310000in}}%
\pgfusepath{clip}%
\pgfsetbuttcap%
\pgfsetroundjoin%
\definecolor{currentfill}{rgb}{0.000000,0.000000,0.000000}%
\pgfsetfillcolor{currentfill}%
\pgfsetlinewidth{1.003750pt}%
\definecolor{currentstroke}{rgb}{0.000000,0.000000,0.000000}%
\pgfsetstrokecolor{currentstroke}%
\pgfsetdash{}{0pt}%
\pgfpathmoveto{\pgfqpoint{0.480841in}{0.446310in}}%
\pgfpathcurveto{\pgfqpoint{0.491891in}{0.446310in}}{\pgfqpoint{0.502490in}{0.450700in}}{\pgfqpoint{0.510303in}{0.458514in}}%
\pgfpathcurveto{\pgfqpoint{0.518117in}{0.466328in}}{\pgfqpoint{0.522507in}{0.476927in}}{\pgfqpoint{0.522507in}{0.487977in}}%
\pgfpathcurveto{\pgfqpoint{0.522507in}{0.499027in}}{\pgfqpoint{0.518117in}{0.509626in}}{\pgfqpoint{0.510303in}{0.517440in}}%
\pgfpathcurveto{\pgfqpoint{0.502490in}{0.525253in}}{\pgfqpoint{0.491891in}{0.529644in}}{\pgfqpoint{0.480841in}{0.529644in}}%
\pgfpathcurveto{\pgfqpoint{0.469790in}{0.529644in}}{\pgfqpoint{0.459191in}{0.525253in}}{\pgfqpoint{0.451378in}{0.517440in}}%
\pgfpathcurveto{\pgfqpoint{0.443564in}{0.509626in}}{\pgfqpoint{0.439174in}{0.499027in}}{\pgfqpoint{0.439174in}{0.487977in}}%
\pgfpathcurveto{\pgfqpoint{0.439174in}{0.476927in}}{\pgfqpoint{0.443564in}{0.466328in}}{\pgfqpoint{0.451378in}{0.458514in}}%
\pgfpathcurveto{\pgfqpoint{0.459191in}{0.450700in}}{\pgfqpoint{0.469790in}{0.446310in}}{\pgfqpoint{0.480841in}{0.446310in}}%
\pgfpathclose%
\pgfusepath{stroke,fill}%
\end{pgfscope}%
\begin{pgfscope}%
\pgfpathrectangle{\pgfqpoint{0.375000in}{0.330000in}}{\pgfqpoint{2.325000in}{2.310000in}}%
\pgfusepath{clip}%
\pgfsetbuttcap%
\pgfsetroundjoin%
\definecolor{currentfill}{rgb}{0.000000,0.000000,0.000000}%
\pgfsetfillcolor{currentfill}%
\pgfsetlinewidth{1.003750pt}%
\definecolor{currentstroke}{rgb}{0.000000,0.000000,0.000000}%
\pgfsetstrokecolor{currentstroke}%
\pgfsetdash{}{0pt}%
\pgfpathmoveto{\pgfqpoint{0.480841in}{0.446310in}}%
\pgfpathcurveto{\pgfqpoint{0.491891in}{0.446310in}}{\pgfqpoint{0.502490in}{0.450700in}}{\pgfqpoint{0.510303in}{0.458514in}}%
\pgfpathcurveto{\pgfqpoint{0.518117in}{0.466328in}}{\pgfqpoint{0.522507in}{0.476927in}}{\pgfqpoint{0.522507in}{0.487977in}}%
\pgfpathcurveto{\pgfqpoint{0.522507in}{0.499027in}}{\pgfqpoint{0.518117in}{0.509626in}}{\pgfqpoint{0.510303in}{0.517440in}}%
\pgfpathcurveto{\pgfqpoint{0.502490in}{0.525253in}}{\pgfqpoint{0.491891in}{0.529644in}}{\pgfqpoint{0.480841in}{0.529644in}}%
\pgfpathcurveto{\pgfqpoint{0.469790in}{0.529644in}}{\pgfqpoint{0.459191in}{0.525253in}}{\pgfqpoint{0.451378in}{0.517440in}}%
\pgfpathcurveto{\pgfqpoint{0.443564in}{0.509626in}}{\pgfqpoint{0.439174in}{0.499027in}}{\pgfqpoint{0.439174in}{0.487977in}}%
\pgfpathcurveto{\pgfqpoint{0.439174in}{0.476927in}}{\pgfqpoint{0.443564in}{0.466328in}}{\pgfqpoint{0.451378in}{0.458514in}}%
\pgfpathcurveto{\pgfqpoint{0.459191in}{0.450700in}}{\pgfqpoint{0.469790in}{0.446310in}}{\pgfqpoint{0.480841in}{0.446310in}}%
\pgfpathclose%
\pgfusepath{stroke,fill}%
\end{pgfscope}%
\begin{pgfscope}%
\pgfpathrectangle{\pgfqpoint{0.375000in}{0.330000in}}{\pgfqpoint{2.325000in}{2.310000in}}%
\pgfusepath{clip}%
\pgfsetbuttcap%
\pgfsetroundjoin%
\definecolor{currentfill}{rgb}{0.000000,0.000000,0.000000}%
\pgfsetfillcolor{currentfill}%
\pgfsetlinewidth{1.003750pt}%
\definecolor{currentstroke}{rgb}{0.000000,0.000000,0.000000}%
\pgfsetstrokecolor{currentstroke}%
\pgfsetdash{}{0pt}%
\pgfpathmoveto{\pgfqpoint{0.480841in}{0.446310in}}%
\pgfpathcurveto{\pgfqpoint{0.491891in}{0.446310in}}{\pgfqpoint{0.502490in}{0.450700in}}{\pgfqpoint{0.510303in}{0.458514in}}%
\pgfpathcurveto{\pgfqpoint{0.518117in}{0.466328in}}{\pgfqpoint{0.522507in}{0.476927in}}{\pgfqpoint{0.522507in}{0.487977in}}%
\pgfpathcurveto{\pgfqpoint{0.522507in}{0.499027in}}{\pgfqpoint{0.518117in}{0.509626in}}{\pgfqpoint{0.510303in}{0.517440in}}%
\pgfpathcurveto{\pgfqpoint{0.502490in}{0.525253in}}{\pgfqpoint{0.491891in}{0.529644in}}{\pgfqpoint{0.480841in}{0.529644in}}%
\pgfpathcurveto{\pgfqpoint{0.469790in}{0.529644in}}{\pgfqpoint{0.459191in}{0.525253in}}{\pgfqpoint{0.451378in}{0.517440in}}%
\pgfpathcurveto{\pgfqpoint{0.443564in}{0.509626in}}{\pgfqpoint{0.439174in}{0.499027in}}{\pgfqpoint{0.439174in}{0.487977in}}%
\pgfpathcurveto{\pgfqpoint{0.439174in}{0.476927in}}{\pgfqpoint{0.443564in}{0.466328in}}{\pgfqpoint{0.451378in}{0.458514in}}%
\pgfpathcurveto{\pgfqpoint{0.459191in}{0.450700in}}{\pgfqpoint{0.469790in}{0.446310in}}{\pgfqpoint{0.480841in}{0.446310in}}%
\pgfpathclose%
\pgfusepath{stroke,fill}%
\end{pgfscope}%
\begin{pgfscope}%
\pgfpathrectangle{\pgfqpoint{0.375000in}{0.330000in}}{\pgfqpoint{2.325000in}{2.310000in}}%
\pgfusepath{clip}%
\pgfsetbuttcap%
\pgfsetroundjoin%
\definecolor{currentfill}{rgb}{0.000000,0.000000,0.000000}%
\pgfsetfillcolor{currentfill}%
\pgfsetlinewidth{1.003750pt}%
\definecolor{currentstroke}{rgb}{0.000000,0.000000,0.000000}%
\pgfsetstrokecolor{currentstroke}%
\pgfsetdash{}{0pt}%
\pgfpathmoveto{\pgfqpoint{0.480841in}{0.446310in}}%
\pgfpathcurveto{\pgfqpoint{0.491891in}{0.446310in}}{\pgfqpoint{0.502490in}{0.450700in}}{\pgfqpoint{0.510303in}{0.458514in}}%
\pgfpathcurveto{\pgfqpoint{0.518117in}{0.466328in}}{\pgfqpoint{0.522507in}{0.476927in}}{\pgfqpoint{0.522507in}{0.487977in}}%
\pgfpathcurveto{\pgfqpoint{0.522507in}{0.499027in}}{\pgfqpoint{0.518117in}{0.509626in}}{\pgfqpoint{0.510303in}{0.517440in}}%
\pgfpathcurveto{\pgfqpoint{0.502490in}{0.525253in}}{\pgfqpoint{0.491891in}{0.529644in}}{\pgfqpoint{0.480841in}{0.529644in}}%
\pgfpathcurveto{\pgfqpoint{0.469790in}{0.529644in}}{\pgfqpoint{0.459191in}{0.525253in}}{\pgfqpoint{0.451378in}{0.517440in}}%
\pgfpathcurveto{\pgfqpoint{0.443564in}{0.509626in}}{\pgfqpoint{0.439174in}{0.499027in}}{\pgfqpoint{0.439174in}{0.487977in}}%
\pgfpathcurveto{\pgfqpoint{0.439174in}{0.476927in}}{\pgfqpoint{0.443564in}{0.466328in}}{\pgfqpoint{0.451378in}{0.458514in}}%
\pgfpathcurveto{\pgfqpoint{0.459191in}{0.450700in}}{\pgfqpoint{0.469790in}{0.446310in}}{\pgfqpoint{0.480841in}{0.446310in}}%
\pgfpathclose%
\pgfusepath{stroke,fill}%
\end{pgfscope}%
\begin{pgfscope}%
\pgfpathrectangle{\pgfqpoint{0.375000in}{0.330000in}}{\pgfqpoint{2.325000in}{2.310000in}}%
\pgfusepath{clip}%
\pgfsetbuttcap%
\pgfsetroundjoin%
\definecolor{currentfill}{rgb}{0.000000,0.000000,0.000000}%
\pgfsetfillcolor{currentfill}%
\pgfsetlinewidth{1.003750pt}%
\definecolor{currentstroke}{rgb}{0.000000,0.000000,0.000000}%
\pgfsetstrokecolor{currentstroke}%
\pgfsetdash{}{0pt}%
\pgfpathmoveto{\pgfqpoint{0.480841in}{0.446310in}}%
\pgfpathcurveto{\pgfqpoint{0.491891in}{0.446310in}}{\pgfqpoint{0.502490in}{0.450700in}}{\pgfqpoint{0.510303in}{0.458514in}}%
\pgfpathcurveto{\pgfqpoint{0.518117in}{0.466328in}}{\pgfqpoint{0.522507in}{0.476927in}}{\pgfqpoint{0.522507in}{0.487977in}}%
\pgfpathcurveto{\pgfqpoint{0.522507in}{0.499027in}}{\pgfqpoint{0.518117in}{0.509626in}}{\pgfqpoint{0.510303in}{0.517440in}}%
\pgfpathcurveto{\pgfqpoint{0.502490in}{0.525253in}}{\pgfqpoint{0.491891in}{0.529644in}}{\pgfqpoint{0.480841in}{0.529644in}}%
\pgfpathcurveto{\pgfqpoint{0.469790in}{0.529644in}}{\pgfqpoint{0.459191in}{0.525253in}}{\pgfqpoint{0.451378in}{0.517440in}}%
\pgfpathcurveto{\pgfqpoint{0.443564in}{0.509626in}}{\pgfqpoint{0.439174in}{0.499027in}}{\pgfqpoint{0.439174in}{0.487977in}}%
\pgfpathcurveto{\pgfqpoint{0.439174in}{0.476927in}}{\pgfqpoint{0.443564in}{0.466328in}}{\pgfqpoint{0.451378in}{0.458514in}}%
\pgfpathcurveto{\pgfqpoint{0.459191in}{0.450700in}}{\pgfqpoint{0.469790in}{0.446310in}}{\pgfqpoint{0.480841in}{0.446310in}}%
\pgfpathclose%
\pgfusepath{stroke,fill}%
\end{pgfscope}%
\begin{pgfscope}%
\pgfpathrectangle{\pgfqpoint{0.375000in}{0.330000in}}{\pgfqpoint{2.325000in}{2.310000in}}%
\pgfusepath{clip}%
\pgfsetbuttcap%
\pgfsetroundjoin%
\definecolor{currentfill}{rgb}{0.000000,0.000000,0.000000}%
\pgfsetfillcolor{currentfill}%
\pgfsetlinewidth{1.003750pt}%
\definecolor{currentstroke}{rgb}{0.000000,0.000000,0.000000}%
\pgfsetstrokecolor{currentstroke}%
\pgfsetdash{}{0pt}%
\pgfpathmoveto{\pgfqpoint{0.480841in}{0.446310in}}%
\pgfpathcurveto{\pgfqpoint{0.491891in}{0.446310in}}{\pgfqpoint{0.502490in}{0.450700in}}{\pgfqpoint{0.510303in}{0.458514in}}%
\pgfpathcurveto{\pgfqpoint{0.518117in}{0.466328in}}{\pgfqpoint{0.522507in}{0.476927in}}{\pgfqpoint{0.522507in}{0.487977in}}%
\pgfpathcurveto{\pgfqpoint{0.522507in}{0.499027in}}{\pgfqpoint{0.518117in}{0.509626in}}{\pgfqpoint{0.510303in}{0.517440in}}%
\pgfpathcurveto{\pgfqpoint{0.502490in}{0.525253in}}{\pgfqpoint{0.491891in}{0.529644in}}{\pgfqpoint{0.480841in}{0.529644in}}%
\pgfpathcurveto{\pgfqpoint{0.469790in}{0.529644in}}{\pgfqpoint{0.459191in}{0.525253in}}{\pgfqpoint{0.451378in}{0.517440in}}%
\pgfpathcurveto{\pgfqpoint{0.443564in}{0.509626in}}{\pgfqpoint{0.439174in}{0.499027in}}{\pgfqpoint{0.439174in}{0.487977in}}%
\pgfpathcurveto{\pgfqpoint{0.439174in}{0.476927in}}{\pgfqpoint{0.443564in}{0.466328in}}{\pgfqpoint{0.451378in}{0.458514in}}%
\pgfpathcurveto{\pgfqpoint{0.459191in}{0.450700in}}{\pgfqpoint{0.469790in}{0.446310in}}{\pgfqpoint{0.480841in}{0.446310in}}%
\pgfpathclose%
\pgfusepath{stroke,fill}%
\end{pgfscope}%
\begin{pgfscope}%
\pgfpathrectangle{\pgfqpoint{0.375000in}{0.330000in}}{\pgfqpoint{2.325000in}{2.310000in}}%
\pgfusepath{clip}%
\pgfsetbuttcap%
\pgfsetroundjoin%
\definecolor{currentfill}{rgb}{0.000000,0.000000,0.000000}%
\pgfsetfillcolor{currentfill}%
\pgfsetlinewidth{1.003750pt}%
\definecolor{currentstroke}{rgb}{0.000000,0.000000,0.000000}%
\pgfsetstrokecolor{currentstroke}%
\pgfsetdash{}{0pt}%
\pgfpathmoveto{\pgfqpoint{0.480841in}{0.446310in}}%
\pgfpathcurveto{\pgfqpoint{0.491891in}{0.446310in}}{\pgfqpoint{0.502490in}{0.450700in}}{\pgfqpoint{0.510303in}{0.458514in}}%
\pgfpathcurveto{\pgfqpoint{0.518117in}{0.466328in}}{\pgfqpoint{0.522507in}{0.476927in}}{\pgfqpoint{0.522507in}{0.487977in}}%
\pgfpathcurveto{\pgfqpoint{0.522507in}{0.499027in}}{\pgfqpoint{0.518117in}{0.509626in}}{\pgfqpoint{0.510303in}{0.517440in}}%
\pgfpathcurveto{\pgfqpoint{0.502490in}{0.525253in}}{\pgfqpoint{0.491891in}{0.529644in}}{\pgfqpoint{0.480841in}{0.529644in}}%
\pgfpathcurveto{\pgfqpoint{0.469790in}{0.529644in}}{\pgfqpoint{0.459191in}{0.525253in}}{\pgfqpoint{0.451378in}{0.517440in}}%
\pgfpathcurveto{\pgfqpoint{0.443564in}{0.509626in}}{\pgfqpoint{0.439174in}{0.499027in}}{\pgfqpoint{0.439174in}{0.487977in}}%
\pgfpathcurveto{\pgfqpoint{0.439174in}{0.476927in}}{\pgfqpoint{0.443564in}{0.466328in}}{\pgfqpoint{0.451378in}{0.458514in}}%
\pgfpathcurveto{\pgfqpoint{0.459191in}{0.450700in}}{\pgfqpoint{0.469790in}{0.446310in}}{\pgfqpoint{0.480841in}{0.446310in}}%
\pgfpathclose%
\pgfusepath{stroke,fill}%
\end{pgfscope}%
\begin{pgfscope}%
\pgfpathrectangle{\pgfqpoint{0.375000in}{0.330000in}}{\pgfqpoint{2.325000in}{2.310000in}}%
\pgfusepath{clip}%
\pgfsetbuttcap%
\pgfsetroundjoin%
\definecolor{currentfill}{rgb}{0.000000,0.000000,0.000000}%
\pgfsetfillcolor{currentfill}%
\pgfsetlinewidth{1.003750pt}%
\definecolor{currentstroke}{rgb}{0.000000,0.000000,0.000000}%
\pgfsetstrokecolor{currentstroke}%
\pgfsetdash{}{0pt}%
\pgfpathmoveto{\pgfqpoint{0.480841in}{0.498341in}}%
\pgfpathcurveto{\pgfqpoint{0.491891in}{0.498341in}}{\pgfqpoint{0.502490in}{0.502731in}}{\pgfqpoint{0.510303in}{0.510545in}}%
\pgfpathcurveto{\pgfqpoint{0.518117in}{0.518359in}}{\pgfqpoint{0.522507in}{0.528958in}}{\pgfqpoint{0.522507in}{0.540008in}}%
\pgfpathcurveto{\pgfqpoint{0.522507in}{0.551058in}}{\pgfqpoint{0.518117in}{0.561657in}}{\pgfqpoint{0.510303in}{0.569470in}}%
\pgfpathcurveto{\pgfqpoint{0.502490in}{0.577284in}}{\pgfqpoint{0.491891in}{0.581674in}}{\pgfqpoint{0.480841in}{0.581674in}}%
\pgfpathcurveto{\pgfqpoint{0.469790in}{0.581674in}}{\pgfqpoint{0.459191in}{0.577284in}}{\pgfqpoint{0.451378in}{0.569470in}}%
\pgfpathcurveto{\pgfqpoint{0.443564in}{0.561657in}}{\pgfqpoint{0.439174in}{0.551058in}}{\pgfqpoint{0.439174in}{0.540008in}}%
\pgfpathcurveto{\pgfqpoint{0.439174in}{0.528958in}}{\pgfqpoint{0.443564in}{0.518359in}}{\pgfqpoint{0.451378in}{0.510545in}}%
\pgfpathcurveto{\pgfqpoint{0.459191in}{0.502731in}}{\pgfqpoint{0.469790in}{0.498341in}}{\pgfqpoint{0.480841in}{0.498341in}}%
\pgfpathclose%
\pgfusepath{stroke,fill}%
\end{pgfscope}%
\begin{pgfscope}%
\pgfpathrectangle{\pgfqpoint{0.375000in}{0.330000in}}{\pgfqpoint{2.325000in}{2.310000in}}%
\pgfusepath{clip}%
\pgfsetbuttcap%
\pgfsetroundjoin%
\definecolor{currentfill}{rgb}{0.000000,0.000000,0.000000}%
\pgfsetfillcolor{currentfill}%
\pgfsetlinewidth{1.003750pt}%
\definecolor{currentstroke}{rgb}{0.000000,0.000000,0.000000}%
\pgfsetstrokecolor{currentstroke}%
\pgfsetdash{}{0pt}%
\pgfpathmoveto{\pgfqpoint{0.480841in}{0.446310in}}%
\pgfpathcurveto{\pgfqpoint{0.491891in}{0.446310in}}{\pgfqpoint{0.502490in}{0.450700in}}{\pgfqpoint{0.510303in}{0.458514in}}%
\pgfpathcurveto{\pgfqpoint{0.518117in}{0.466328in}}{\pgfqpoint{0.522507in}{0.476927in}}{\pgfqpoint{0.522507in}{0.487977in}}%
\pgfpathcurveto{\pgfqpoint{0.522507in}{0.499027in}}{\pgfqpoint{0.518117in}{0.509626in}}{\pgfqpoint{0.510303in}{0.517440in}}%
\pgfpathcurveto{\pgfqpoint{0.502490in}{0.525253in}}{\pgfqpoint{0.491891in}{0.529644in}}{\pgfqpoint{0.480841in}{0.529644in}}%
\pgfpathcurveto{\pgfqpoint{0.469790in}{0.529644in}}{\pgfqpoint{0.459191in}{0.525253in}}{\pgfqpoint{0.451378in}{0.517440in}}%
\pgfpathcurveto{\pgfqpoint{0.443564in}{0.509626in}}{\pgfqpoint{0.439174in}{0.499027in}}{\pgfqpoint{0.439174in}{0.487977in}}%
\pgfpathcurveto{\pgfqpoint{0.439174in}{0.476927in}}{\pgfqpoint{0.443564in}{0.466328in}}{\pgfqpoint{0.451378in}{0.458514in}}%
\pgfpathcurveto{\pgfqpoint{0.459191in}{0.450700in}}{\pgfqpoint{0.469790in}{0.446310in}}{\pgfqpoint{0.480841in}{0.446310in}}%
\pgfpathclose%
\pgfusepath{stroke,fill}%
\end{pgfscope}%
\begin{pgfscope}%
\pgfpathrectangle{\pgfqpoint{0.375000in}{0.330000in}}{\pgfqpoint{2.325000in}{2.310000in}}%
\pgfusepath{clip}%
\pgfsetbuttcap%
\pgfsetroundjoin%
\definecolor{currentfill}{rgb}{0.000000,0.000000,0.000000}%
\pgfsetfillcolor{currentfill}%
\pgfsetlinewidth{1.003750pt}%
\definecolor{currentstroke}{rgb}{0.000000,0.000000,0.000000}%
\pgfsetstrokecolor{currentstroke}%
\pgfsetdash{}{0pt}%
\pgfpathmoveto{\pgfqpoint{0.480841in}{0.446310in}}%
\pgfpathcurveto{\pgfqpoint{0.491891in}{0.446310in}}{\pgfqpoint{0.502490in}{0.450700in}}{\pgfqpoint{0.510303in}{0.458514in}}%
\pgfpathcurveto{\pgfqpoint{0.518117in}{0.466328in}}{\pgfqpoint{0.522507in}{0.476927in}}{\pgfqpoint{0.522507in}{0.487977in}}%
\pgfpathcurveto{\pgfqpoint{0.522507in}{0.499027in}}{\pgfqpoint{0.518117in}{0.509626in}}{\pgfqpoint{0.510303in}{0.517440in}}%
\pgfpathcurveto{\pgfqpoint{0.502490in}{0.525253in}}{\pgfqpoint{0.491891in}{0.529644in}}{\pgfqpoint{0.480841in}{0.529644in}}%
\pgfpathcurveto{\pgfqpoint{0.469790in}{0.529644in}}{\pgfqpoint{0.459191in}{0.525253in}}{\pgfqpoint{0.451378in}{0.517440in}}%
\pgfpathcurveto{\pgfqpoint{0.443564in}{0.509626in}}{\pgfqpoint{0.439174in}{0.499027in}}{\pgfqpoint{0.439174in}{0.487977in}}%
\pgfpathcurveto{\pgfqpoint{0.439174in}{0.476927in}}{\pgfqpoint{0.443564in}{0.466328in}}{\pgfqpoint{0.451378in}{0.458514in}}%
\pgfpathcurveto{\pgfqpoint{0.459191in}{0.450700in}}{\pgfqpoint{0.469790in}{0.446310in}}{\pgfqpoint{0.480841in}{0.446310in}}%
\pgfpathclose%
\pgfusepath{stroke,fill}%
\end{pgfscope}%
\begin{pgfscope}%
\pgfpathrectangle{\pgfqpoint{0.375000in}{0.330000in}}{\pgfqpoint{2.325000in}{2.310000in}}%
\pgfusepath{clip}%
\pgfsetbuttcap%
\pgfsetroundjoin%
\definecolor{currentfill}{rgb}{0.000000,0.000000,0.000000}%
\pgfsetfillcolor{currentfill}%
\pgfsetlinewidth{1.003750pt}%
\definecolor{currentstroke}{rgb}{0.000000,0.000000,0.000000}%
\pgfsetstrokecolor{currentstroke}%
\pgfsetdash{}{0pt}%
\pgfpathmoveto{\pgfqpoint{0.480841in}{0.446310in}}%
\pgfpathcurveto{\pgfqpoint{0.491891in}{0.446310in}}{\pgfqpoint{0.502490in}{0.450700in}}{\pgfqpoint{0.510303in}{0.458514in}}%
\pgfpathcurveto{\pgfqpoint{0.518117in}{0.466328in}}{\pgfqpoint{0.522507in}{0.476927in}}{\pgfqpoint{0.522507in}{0.487977in}}%
\pgfpathcurveto{\pgfqpoint{0.522507in}{0.499027in}}{\pgfqpoint{0.518117in}{0.509626in}}{\pgfqpoint{0.510303in}{0.517440in}}%
\pgfpathcurveto{\pgfqpoint{0.502490in}{0.525253in}}{\pgfqpoint{0.491891in}{0.529644in}}{\pgfqpoint{0.480841in}{0.529644in}}%
\pgfpathcurveto{\pgfqpoint{0.469790in}{0.529644in}}{\pgfqpoint{0.459191in}{0.525253in}}{\pgfqpoint{0.451378in}{0.517440in}}%
\pgfpathcurveto{\pgfqpoint{0.443564in}{0.509626in}}{\pgfqpoint{0.439174in}{0.499027in}}{\pgfqpoint{0.439174in}{0.487977in}}%
\pgfpathcurveto{\pgfqpoint{0.439174in}{0.476927in}}{\pgfqpoint{0.443564in}{0.466328in}}{\pgfqpoint{0.451378in}{0.458514in}}%
\pgfpathcurveto{\pgfqpoint{0.459191in}{0.450700in}}{\pgfqpoint{0.469790in}{0.446310in}}{\pgfqpoint{0.480841in}{0.446310in}}%
\pgfpathclose%
\pgfusepath{stroke,fill}%
\end{pgfscope}%
\begin{pgfscope}%
\pgfpathrectangle{\pgfqpoint{0.375000in}{0.330000in}}{\pgfqpoint{2.325000in}{2.310000in}}%
\pgfusepath{clip}%
\pgfsetbuttcap%
\pgfsetroundjoin%
\definecolor{currentfill}{rgb}{0.000000,0.000000,0.000000}%
\pgfsetfillcolor{currentfill}%
\pgfsetlinewidth{1.003750pt}%
\definecolor{currentstroke}{rgb}{0.000000,0.000000,0.000000}%
\pgfsetstrokecolor{currentstroke}%
\pgfsetdash{}{0pt}%
\pgfpathmoveto{\pgfqpoint{0.480841in}{0.446310in}}%
\pgfpathcurveto{\pgfqpoint{0.491891in}{0.446310in}}{\pgfqpoint{0.502490in}{0.450700in}}{\pgfqpoint{0.510303in}{0.458514in}}%
\pgfpathcurveto{\pgfqpoint{0.518117in}{0.466328in}}{\pgfqpoint{0.522507in}{0.476927in}}{\pgfqpoint{0.522507in}{0.487977in}}%
\pgfpathcurveto{\pgfqpoint{0.522507in}{0.499027in}}{\pgfqpoint{0.518117in}{0.509626in}}{\pgfqpoint{0.510303in}{0.517440in}}%
\pgfpathcurveto{\pgfqpoint{0.502490in}{0.525253in}}{\pgfqpoint{0.491891in}{0.529644in}}{\pgfqpoint{0.480841in}{0.529644in}}%
\pgfpathcurveto{\pgfqpoint{0.469790in}{0.529644in}}{\pgfqpoint{0.459191in}{0.525253in}}{\pgfqpoint{0.451378in}{0.517440in}}%
\pgfpathcurveto{\pgfqpoint{0.443564in}{0.509626in}}{\pgfqpoint{0.439174in}{0.499027in}}{\pgfqpoint{0.439174in}{0.487977in}}%
\pgfpathcurveto{\pgfqpoint{0.439174in}{0.476927in}}{\pgfqpoint{0.443564in}{0.466328in}}{\pgfqpoint{0.451378in}{0.458514in}}%
\pgfpathcurveto{\pgfqpoint{0.459191in}{0.450700in}}{\pgfqpoint{0.469790in}{0.446310in}}{\pgfqpoint{0.480841in}{0.446310in}}%
\pgfpathclose%
\pgfusepath{stroke,fill}%
\end{pgfscope}%
\begin{pgfscope}%
\pgfpathrectangle{\pgfqpoint{0.375000in}{0.330000in}}{\pgfqpoint{2.325000in}{2.310000in}}%
\pgfusepath{clip}%
\pgfsetbuttcap%
\pgfsetroundjoin%
\definecolor{currentfill}{rgb}{0.000000,0.000000,0.000000}%
\pgfsetfillcolor{currentfill}%
\pgfsetlinewidth{1.003750pt}%
\definecolor{currentstroke}{rgb}{0.000000,0.000000,0.000000}%
\pgfsetstrokecolor{currentstroke}%
\pgfsetdash{}{0pt}%
\pgfpathmoveto{\pgfqpoint{0.480841in}{0.498341in}}%
\pgfpathcurveto{\pgfqpoint{0.491891in}{0.498341in}}{\pgfqpoint{0.502490in}{0.502731in}}{\pgfqpoint{0.510303in}{0.510545in}}%
\pgfpathcurveto{\pgfqpoint{0.518117in}{0.518359in}}{\pgfqpoint{0.522507in}{0.528958in}}{\pgfqpoint{0.522507in}{0.540008in}}%
\pgfpathcurveto{\pgfqpoint{0.522507in}{0.551058in}}{\pgfqpoint{0.518117in}{0.561657in}}{\pgfqpoint{0.510303in}{0.569470in}}%
\pgfpathcurveto{\pgfqpoint{0.502490in}{0.577284in}}{\pgfqpoint{0.491891in}{0.581674in}}{\pgfqpoint{0.480841in}{0.581674in}}%
\pgfpathcurveto{\pgfqpoint{0.469790in}{0.581674in}}{\pgfqpoint{0.459191in}{0.577284in}}{\pgfqpoint{0.451378in}{0.569470in}}%
\pgfpathcurveto{\pgfqpoint{0.443564in}{0.561657in}}{\pgfqpoint{0.439174in}{0.551058in}}{\pgfqpoint{0.439174in}{0.540008in}}%
\pgfpathcurveto{\pgfqpoint{0.439174in}{0.528958in}}{\pgfqpoint{0.443564in}{0.518359in}}{\pgfqpoint{0.451378in}{0.510545in}}%
\pgfpathcurveto{\pgfqpoint{0.459191in}{0.502731in}}{\pgfqpoint{0.469790in}{0.498341in}}{\pgfqpoint{0.480841in}{0.498341in}}%
\pgfpathclose%
\pgfusepath{stroke,fill}%
\end{pgfscope}%
\begin{pgfscope}%
\pgfpathrectangle{\pgfqpoint{0.375000in}{0.330000in}}{\pgfqpoint{2.325000in}{2.310000in}}%
\pgfusepath{clip}%
\pgfsetbuttcap%
\pgfsetroundjoin%
\definecolor{currentfill}{rgb}{0.000000,0.000000,0.000000}%
\pgfsetfillcolor{currentfill}%
\pgfsetlinewidth{1.003750pt}%
\definecolor{currentstroke}{rgb}{0.000000,0.000000,0.000000}%
\pgfsetstrokecolor{currentstroke}%
\pgfsetdash{}{0pt}%
\pgfpathmoveto{\pgfqpoint{0.480841in}{0.446310in}}%
\pgfpathcurveto{\pgfqpoint{0.491891in}{0.446310in}}{\pgfqpoint{0.502490in}{0.450700in}}{\pgfqpoint{0.510303in}{0.458514in}}%
\pgfpathcurveto{\pgfqpoint{0.518117in}{0.466328in}}{\pgfqpoint{0.522507in}{0.476927in}}{\pgfqpoint{0.522507in}{0.487977in}}%
\pgfpathcurveto{\pgfqpoint{0.522507in}{0.499027in}}{\pgfqpoint{0.518117in}{0.509626in}}{\pgfqpoint{0.510303in}{0.517440in}}%
\pgfpathcurveto{\pgfqpoint{0.502490in}{0.525253in}}{\pgfqpoint{0.491891in}{0.529644in}}{\pgfqpoint{0.480841in}{0.529644in}}%
\pgfpathcurveto{\pgfqpoint{0.469790in}{0.529644in}}{\pgfqpoint{0.459191in}{0.525253in}}{\pgfqpoint{0.451378in}{0.517440in}}%
\pgfpathcurveto{\pgfqpoint{0.443564in}{0.509626in}}{\pgfqpoint{0.439174in}{0.499027in}}{\pgfqpoint{0.439174in}{0.487977in}}%
\pgfpathcurveto{\pgfqpoint{0.439174in}{0.476927in}}{\pgfqpoint{0.443564in}{0.466328in}}{\pgfqpoint{0.451378in}{0.458514in}}%
\pgfpathcurveto{\pgfqpoint{0.459191in}{0.450700in}}{\pgfqpoint{0.469790in}{0.446310in}}{\pgfqpoint{0.480841in}{0.446310in}}%
\pgfpathclose%
\pgfusepath{stroke,fill}%
\end{pgfscope}%
\begin{pgfscope}%
\pgfpathrectangle{\pgfqpoint{0.375000in}{0.330000in}}{\pgfqpoint{2.325000in}{2.310000in}}%
\pgfusepath{clip}%
\pgfsetbuttcap%
\pgfsetroundjoin%
\definecolor{currentfill}{rgb}{0.000000,0.000000,0.000000}%
\pgfsetfillcolor{currentfill}%
\pgfsetlinewidth{1.003750pt}%
\definecolor{currentstroke}{rgb}{0.000000,0.000000,0.000000}%
\pgfsetstrokecolor{currentstroke}%
\pgfsetdash{}{0pt}%
\pgfpathmoveto{\pgfqpoint{0.480841in}{0.394279in}}%
\pgfpathcurveto{\pgfqpoint{0.491891in}{0.394279in}}{\pgfqpoint{0.502490in}{0.398670in}}{\pgfqpoint{0.510303in}{0.406483in}}%
\pgfpathcurveto{\pgfqpoint{0.518117in}{0.414297in}}{\pgfqpoint{0.522507in}{0.424896in}}{\pgfqpoint{0.522507in}{0.435946in}}%
\pgfpathcurveto{\pgfqpoint{0.522507in}{0.446996in}}{\pgfqpoint{0.518117in}{0.457595in}}{\pgfqpoint{0.510303in}{0.465409in}}%
\pgfpathcurveto{\pgfqpoint{0.502490in}{0.473222in}}{\pgfqpoint{0.491891in}{0.477613in}}{\pgfqpoint{0.480841in}{0.477613in}}%
\pgfpathcurveto{\pgfqpoint{0.469790in}{0.477613in}}{\pgfqpoint{0.459191in}{0.473222in}}{\pgfqpoint{0.451378in}{0.465409in}}%
\pgfpathcurveto{\pgfqpoint{0.443564in}{0.457595in}}{\pgfqpoint{0.439174in}{0.446996in}}{\pgfqpoint{0.439174in}{0.435946in}}%
\pgfpathcurveto{\pgfqpoint{0.439174in}{0.424896in}}{\pgfqpoint{0.443564in}{0.414297in}}{\pgfqpoint{0.451378in}{0.406483in}}%
\pgfpathcurveto{\pgfqpoint{0.459191in}{0.398670in}}{\pgfqpoint{0.469790in}{0.394279in}}{\pgfqpoint{0.480841in}{0.394279in}}%
\pgfpathclose%
\pgfusepath{stroke,fill}%
\end{pgfscope}%
\begin{pgfscope}%
\pgfpathrectangle{\pgfqpoint{0.375000in}{0.330000in}}{\pgfqpoint{2.325000in}{2.310000in}}%
\pgfusepath{clip}%
\pgfsetbuttcap%
\pgfsetroundjoin%
\definecolor{currentfill}{rgb}{0.000000,0.000000,0.000000}%
\pgfsetfillcolor{currentfill}%
\pgfsetlinewidth{1.003750pt}%
\definecolor{currentstroke}{rgb}{0.000000,0.000000,0.000000}%
\pgfsetstrokecolor{currentstroke}%
\pgfsetdash{}{0pt}%
\pgfpathmoveto{\pgfqpoint{0.480841in}{0.394279in}}%
\pgfpathcurveto{\pgfqpoint{0.491891in}{0.394279in}}{\pgfqpoint{0.502490in}{0.398670in}}{\pgfqpoint{0.510303in}{0.406483in}}%
\pgfpathcurveto{\pgfqpoint{0.518117in}{0.414297in}}{\pgfqpoint{0.522507in}{0.424896in}}{\pgfqpoint{0.522507in}{0.435946in}}%
\pgfpathcurveto{\pgfqpoint{0.522507in}{0.446996in}}{\pgfqpoint{0.518117in}{0.457595in}}{\pgfqpoint{0.510303in}{0.465409in}}%
\pgfpathcurveto{\pgfqpoint{0.502490in}{0.473222in}}{\pgfqpoint{0.491891in}{0.477613in}}{\pgfqpoint{0.480841in}{0.477613in}}%
\pgfpathcurveto{\pgfqpoint{0.469790in}{0.477613in}}{\pgfqpoint{0.459191in}{0.473222in}}{\pgfqpoint{0.451378in}{0.465409in}}%
\pgfpathcurveto{\pgfqpoint{0.443564in}{0.457595in}}{\pgfqpoint{0.439174in}{0.446996in}}{\pgfqpoint{0.439174in}{0.435946in}}%
\pgfpathcurveto{\pgfqpoint{0.439174in}{0.424896in}}{\pgfqpoint{0.443564in}{0.414297in}}{\pgfqpoint{0.451378in}{0.406483in}}%
\pgfpathcurveto{\pgfqpoint{0.459191in}{0.398670in}}{\pgfqpoint{0.469790in}{0.394279in}}{\pgfqpoint{0.480841in}{0.394279in}}%
\pgfpathclose%
\pgfusepath{stroke,fill}%
\end{pgfscope}%
\begin{pgfscope}%
\pgfpathrectangle{\pgfqpoint{0.375000in}{0.330000in}}{\pgfqpoint{2.325000in}{2.310000in}}%
\pgfusepath{clip}%
\pgfsetbuttcap%
\pgfsetroundjoin%
\definecolor{currentfill}{rgb}{0.000000,0.000000,0.000000}%
\pgfsetfillcolor{currentfill}%
\pgfsetlinewidth{1.003750pt}%
\definecolor{currentstroke}{rgb}{0.000000,0.000000,0.000000}%
\pgfsetstrokecolor{currentstroke}%
\pgfsetdash{}{0pt}%
\pgfpathmoveto{\pgfqpoint{0.480841in}{0.446310in}}%
\pgfpathcurveto{\pgfqpoint{0.491891in}{0.446310in}}{\pgfqpoint{0.502490in}{0.450700in}}{\pgfqpoint{0.510303in}{0.458514in}}%
\pgfpathcurveto{\pgfqpoint{0.518117in}{0.466328in}}{\pgfqpoint{0.522507in}{0.476927in}}{\pgfqpoint{0.522507in}{0.487977in}}%
\pgfpathcurveto{\pgfqpoint{0.522507in}{0.499027in}}{\pgfqpoint{0.518117in}{0.509626in}}{\pgfqpoint{0.510303in}{0.517440in}}%
\pgfpathcurveto{\pgfqpoint{0.502490in}{0.525253in}}{\pgfqpoint{0.491891in}{0.529644in}}{\pgfqpoint{0.480841in}{0.529644in}}%
\pgfpathcurveto{\pgfqpoint{0.469790in}{0.529644in}}{\pgfqpoint{0.459191in}{0.525253in}}{\pgfqpoint{0.451378in}{0.517440in}}%
\pgfpathcurveto{\pgfqpoint{0.443564in}{0.509626in}}{\pgfqpoint{0.439174in}{0.499027in}}{\pgfqpoint{0.439174in}{0.487977in}}%
\pgfpathcurveto{\pgfqpoint{0.439174in}{0.476927in}}{\pgfqpoint{0.443564in}{0.466328in}}{\pgfqpoint{0.451378in}{0.458514in}}%
\pgfpathcurveto{\pgfqpoint{0.459191in}{0.450700in}}{\pgfqpoint{0.469790in}{0.446310in}}{\pgfqpoint{0.480841in}{0.446310in}}%
\pgfpathclose%
\pgfusepath{stroke,fill}%
\end{pgfscope}%
\begin{pgfscope}%
\pgfpathrectangle{\pgfqpoint{0.375000in}{0.330000in}}{\pgfqpoint{2.325000in}{2.310000in}}%
\pgfusepath{clip}%
\pgfsetbuttcap%
\pgfsetroundjoin%
\definecolor{currentfill}{rgb}{0.000000,0.000000,0.000000}%
\pgfsetfillcolor{currentfill}%
\pgfsetlinewidth{1.003750pt}%
\definecolor{currentstroke}{rgb}{0.000000,0.000000,0.000000}%
\pgfsetstrokecolor{currentstroke}%
\pgfsetdash{}{0pt}%
\pgfpathmoveto{\pgfqpoint{0.480841in}{0.394279in}}%
\pgfpathcurveto{\pgfqpoint{0.491891in}{0.394279in}}{\pgfqpoint{0.502490in}{0.398670in}}{\pgfqpoint{0.510303in}{0.406483in}}%
\pgfpathcurveto{\pgfqpoint{0.518117in}{0.414297in}}{\pgfqpoint{0.522507in}{0.424896in}}{\pgfqpoint{0.522507in}{0.435946in}}%
\pgfpathcurveto{\pgfqpoint{0.522507in}{0.446996in}}{\pgfqpoint{0.518117in}{0.457595in}}{\pgfqpoint{0.510303in}{0.465409in}}%
\pgfpathcurveto{\pgfqpoint{0.502490in}{0.473222in}}{\pgfqpoint{0.491891in}{0.477613in}}{\pgfqpoint{0.480841in}{0.477613in}}%
\pgfpathcurveto{\pgfqpoint{0.469790in}{0.477613in}}{\pgfqpoint{0.459191in}{0.473222in}}{\pgfqpoint{0.451378in}{0.465409in}}%
\pgfpathcurveto{\pgfqpoint{0.443564in}{0.457595in}}{\pgfqpoint{0.439174in}{0.446996in}}{\pgfqpoint{0.439174in}{0.435946in}}%
\pgfpathcurveto{\pgfqpoint{0.439174in}{0.424896in}}{\pgfqpoint{0.443564in}{0.414297in}}{\pgfqpoint{0.451378in}{0.406483in}}%
\pgfpathcurveto{\pgfqpoint{0.459191in}{0.398670in}}{\pgfqpoint{0.469790in}{0.394279in}}{\pgfqpoint{0.480841in}{0.394279in}}%
\pgfpathclose%
\pgfusepath{stroke,fill}%
\end{pgfscope}%
\begin{pgfscope}%
\pgfpathrectangle{\pgfqpoint{0.375000in}{0.330000in}}{\pgfqpoint{2.325000in}{2.310000in}}%
\pgfusepath{clip}%
\pgfsetbuttcap%
\pgfsetroundjoin%
\definecolor{currentfill}{rgb}{0.000000,0.000000,0.000000}%
\pgfsetfillcolor{currentfill}%
\pgfsetlinewidth{1.003750pt}%
\definecolor{currentstroke}{rgb}{0.000000,0.000000,0.000000}%
\pgfsetstrokecolor{currentstroke}%
\pgfsetdash{}{0pt}%
\pgfpathmoveto{\pgfqpoint{0.480841in}{0.446310in}}%
\pgfpathcurveto{\pgfqpoint{0.491891in}{0.446310in}}{\pgfqpoint{0.502490in}{0.450700in}}{\pgfqpoint{0.510303in}{0.458514in}}%
\pgfpathcurveto{\pgfqpoint{0.518117in}{0.466328in}}{\pgfqpoint{0.522507in}{0.476927in}}{\pgfqpoint{0.522507in}{0.487977in}}%
\pgfpathcurveto{\pgfqpoint{0.522507in}{0.499027in}}{\pgfqpoint{0.518117in}{0.509626in}}{\pgfqpoint{0.510303in}{0.517440in}}%
\pgfpathcurveto{\pgfqpoint{0.502490in}{0.525253in}}{\pgfqpoint{0.491891in}{0.529644in}}{\pgfqpoint{0.480841in}{0.529644in}}%
\pgfpathcurveto{\pgfqpoint{0.469790in}{0.529644in}}{\pgfqpoint{0.459191in}{0.525253in}}{\pgfqpoint{0.451378in}{0.517440in}}%
\pgfpathcurveto{\pgfqpoint{0.443564in}{0.509626in}}{\pgfqpoint{0.439174in}{0.499027in}}{\pgfqpoint{0.439174in}{0.487977in}}%
\pgfpathcurveto{\pgfqpoint{0.439174in}{0.476927in}}{\pgfqpoint{0.443564in}{0.466328in}}{\pgfqpoint{0.451378in}{0.458514in}}%
\pgfpathcurveto{\pgfqpoint{0.459191in}{0.450700in}}{\pgfqpoint{0.469790in}{0.446310in}}{\pgfqpoint{0.480841in}{0.446310in}}%
\pgfpathclose%
\pgfusepath{stroke,fill}%
\end{pgfscope}%
\begin{pgfscope}%
\pgfpathrectangle{\pgfqpoint{0.375000in}{0.330000in}}{\pgfqpoint{2.325000in}{2.310000in}}%
\pgfusepath{clip}%
\pgfsetbuttcap%
\pgfsetroundjoin%
\definecolor{currentfill}{rgb}{0.000000,0.000000,0.000000}%
\pgfsetfillcolor{currentfill}%
\pgfsetlinewidth{1.003750pt}%
\definecolor{currentstroke}{rgb}{0.000000,0.000000,0.000000}%
\pgfsetstrokecolor{currentstroke}%
\pgfsetdash{}{0pt}%
\pgfpathmoveto{\pgfqpoint{0.480841in}{0.446310in}}%
\pgfpathcurveto{\pgfqpoint{0.491891in}{0.446310in}}{\pgfqpoint{0.502490in}{0.450700in}}{\pgfqpoint{0.510303in}{0.458514in}}%
\pgfpathcurveto{\pgfqpoint{0.518117in}{0.466328in}}{\pgfqpoint{0.522507in}{0.476927in}}{\pgfqpoint{0.522507in}{0.487977in}}%
\pgfpathcurveto{\pgfqpoint{0.522507in}{0.499027in}}{\pgfqpoint{0.518117in}{0.509626in}}{\pgfqpoint{0.510303in}{0.517440in}}%
\pgfpathcurveto{\pgfqpoint{0.502490in}{0.525253in}}{\pgfqpoint{0.491891in}{0.529644in}}{\pgfqpoint{0.480841in}{0.529644in}}%
\pgfpathcurveto{\pgfqpoint{0.469790in}{0.529644in}}{\pgfqpoint{0.459191in}{0.525253in}}{\pgfqpoint{0.451378in}{0.517440in}}%
\pgfpathcurveto{\pgfqpoint{0.443564in}{0.509626in}}{\pgfqpoint{0.439174in}{0.499027in}}{\pgfqpoint{0.439174in}{0.487977in}}%
\pgfpathcurveto{\pgfqpoint{0.439174in}{0.476927in}}{\pgfqpoint{0.443564in}{0.466328in}}{\pgfqpoint{0.451378in}{0.458514in}}%
\pgfpathcurveto{\pgfqpoint{0.459191in}{0.450700in}}{\pgfqpoint{0.469790in}{0.446310in}}{\pgfqpoint{0.480841in}{0.446310in}}%
\pgfpathclose%
\pgfusepath{stroke,fill}%
\end{pgfscope}%
\begin{pgfscope}%
\pgfpathrectangle{\pgfqpoint{0.375000in}{0.330000in}}{\pgfqpoint{2.325000in}{2.310000in}}%
\pgfusepath{clip}%
\pgfsetbuttcap%
\pgfsetroundjoin%
\definecolor{currentfill}{rgb}{0.000000,0.000000,0.000000}%
\pgfsetfillcolor{currentfill}%
\pgfsetlinewidth{1.003750pt}%
\definecolor{currentstroke}{rgb}{0.000000,0.000000,0.000000}%
\pgfsetstrokecolor{currentstroke}%
\pgfsetdash{}{0pt}%
\pgfpathmoveto{\pgfqpoint{0.480841in}{0.446310in}}%
\pgfpathcurveto{\pgfqpoint{0.491891in}{0.446310in}}{\pgfqpoint{0.502490in}{0.450700in}}{\pgfqpoint{0.510303in}{0.458514in}}%
\pgfpathcurveto{\pgfqpoint{0.518117in}{0.466328in}}{\pgfqpoint{0.522507in}{0.476927in}}{\pgfqpoint{0.522507in}{0.487977in}}%
\pgfpathcurveto{\pgfqpoint{0.522507in}{0.499027in}}{\pgfqpoint{0.518117in}{0.509626in}}{\pgfqpoint{0.510303in}{0.517440in}}%
\pgfpathcurveto{\pgfqpoint{0.502490in}{0.525253in}}{\pgfqpoint{0.491891in}{0.529644in}}{\pgfqpoint{0.480841in}{0.529644in}}%
\pgfpathcurveto{\pgfqpoint{0.469790in}{0.529644in}}{\pgfqpoint{0.459191in}{0.525253in}}{\pgfqpoint{0.451378in}{0.517440in}}%
\pgfpathcurveto{\pgfqpoint{0.443564in}{0.509626in}}{\pgfqpoint{0.439174in}{0.499027in}}{\pgfqpoint{0.439174in}{0.487977in}}%
\pgfpathcurveto{\pgfqpoint{0.439174in}{0.476927in}}{\pgfqpoint{0.443564in}{0.466328in}}{\pgfqpoint{0.451378in}{0.458514in}}%
\pgfpathcurveto{\pgfqpoint{0.459191in}{0.450700in}}{\pgfqpoint{0.469790in}{0.446310in}}{\pgfqpoint{0.480841in}{0.446310in}}%
\pgfpathclose%
\pgfusepath{stroke,fill}%
\end{pgfscope}%
\begin{pgfscope}%
\pgfpathrectangle{\pgfqpoint{0.375000in}{0.330000in}}{\pgfqpoint{2.325000in}{2.310000in}}%
\pgfusepath{clip}%
\pgfsetbuttcap%
\pgfsetroundjoin%
\definecolor{currentfill}{rgb}{0.000000,0.000000,0.000000}%
\pgfsetfillcolor{currentfill}%
\pgfsetlinewidth{1.003750pt}%
\definecolor{currentstroke}{rgb}{0.000000,0.000000,0.000000}%
\pgfsetstrokecolor{currentstroke}%
\pgfsetdash{}{0pt}%
\pgfpathmoveto{\pgfqpoint{0.480841in}{0.446310in}}%
\pgfpathcurveto{\pgfqpoint{0.491891in}{0.446310in}}{\pgfqpoint{0.502490in}{0.450700in}}{\pgfqpoint{0.510303in}{0.458514in}}%
\pgfpathcurveto{\pgfqpoint{0.518117in}{0.466328in}}{\pgfqpoint{0.522507in}{0.476927in}}{\pgfqpoint{0.522507in}{0.487977in}}%
\pgfpathcurveto{\pgfqpoint{0.522507in}{0.499027in}}{\pgfqpoint{0.518117in}{0.509626in}}{\pgfqpoint{0.510303in}{0.517440in}}%
\pgfpathcurveto{\pgfqpoint{0.502490in}{0.525253in}}{\pgfqpoint{0.491891in}{0.529644in}}{\pgfqpoint{0.480841in}{0.529644in}}%
\pgfpathcurveto{\pgfqpoint{0.469790in}{0.529644in}}{\pgfqpoint{0.459191in}{0.525253in}}{\pgfqpoint{0.451378in}{0.517440in}}%
\pgfpathcurveto{\pgfqpoint{0.443564in}{0.509626in}}{\pgfqpoint{0.439174in}{0.499027in}}{\pgfqpoint{0.439174in}{0.487977in}}%
\pgfpathcurveto{\pgfqpoint{0.439174in}{0.476927in}}{\pgfqpoint{0.443564in}{0.466328in}}{\pgfqpoint{0.451378in}{0.458514in}}%
\pgfpathcurveto{\pgfqpoint{0.459191in}{0.450700in}}{\pgfqpoint{0.469790in}{0.446310in}}{\pgfqpoint{0.480841in}{0.446310in}}%
\pgfpathclose%
\pgfusepath{stroke,fill}%
\end{pgfscope}%
\begin{pgfscope}%
\pgfpathrectangle{\pgfqpoint{0.375000in}{0.330000in}}{\pgfqpoint{2.325000in}{2.310000in}}%
\pgfusepath{clip}%
\pgfsetbuttcap%
\pgfsetroundjoin%
\definecolor{currentfill}{rgb}{0.000000,0.000000,0.000000}%
\pgfsetfillcolor{currentfill}%
\pgfsetlinewidth{1.003750pt}%
\definecolor{currentstroke}{rgb}{0.000000,0.000000,0.000000}%
\pgfsetstrokecolor{currentstroke}%
\pgfsetdash{}{0pt}%
\pgfpathmoveto{\pgfqpoint{0.480841in}{0.446310in}}%
\pgfpathcurveto{\pgfqpoint{0.491891in}{0.446310in}}{\pgfqpoint{0.502490in}{0.450700in}}{\pgfqpoint{0.510303in}{0.458514in}}%
\pgfpathcurveto{\pgfqpoint{0.518117in}{0.466328in}}{\pgfqpoint{0.522507in}{0.476927in}}{\pgfqpoint{0.522507in}{0.487977in}}%
\pgfpathcurveto{\pgfqpoint{0.522507in}{0.499027in}}{\pgfqpoint{0.518117in}{0.509626in}}{\pgfqpoint{0.510303in}{0.517440in}}%
\pgfpathcurveto{\pgfqpoint{0.502490in}{0.525253in}}{\pgfqpoint{0.491891in}{0.529644in}}{\pgfqpoint{0.480841in}{0.529644in}}%
\pgfpathcurveto{\pgfqpoint{0.469790in}{0.529644in}}{\pgfqpoint{0.459191in}{0.525253in}}{\pgfqpoint{0.451378in}{0.517440in}}%
\pgfpathcurveto{\pgfqpoint{0.443564in}{0.509626in}}{\pgfqpoint{0.439174in}{0.499027in}}{\pgfqpoint{0.439174in}{0.487977in}}%
\pgfpathcurveto{\pgfqpoint{0.439174in}{0.476927in}}{\pgfqpoint{0.443564in}{0.466328in}}{\pgfqpoint{0.451378in}{0.458514in}}%
\pgfpathcurveto{\pgfqpoint{0.459191in}{0.450700in}}{\pgfqpoint{0.469790in}{0.446310in}}{\pgfqpoint{0.480841in}{0.446310in}}%
\pgfpathclose%
\pgfusepath{stroke,fill}%
\end{pgfscope}%
\begin{pgfscope}%
\pgfpathrectangle{\pgfqpoint{0.375000in}{0.330000in}}{\pgfqpoint{2.325000in}{2.310000in}}%
\pgfusepath{clip}%
\pgfsetbuttcap%
\pgfsetroundjoin%
\definecolor{currentfill}{rgb}{0.000000,0.000000,0.000000}%
\pgfsetfillcolor{currentfill}%
\pgfsetlinewidth{1.003750pt}%
\definecolor{currentstroke}{rgb}{0.000000,0.000000,0.000000}%
\pgfsetstrokecolor{currentstroke}%
\pgfsetdash{}{0pt}%
\pgfpathmoveto{\pgfqpoint{0.480841in}{0.446310in}}%
\pgfpathcurveto{\pgfqpoint{0.491891in}{0.446310in}}{\pgfqpoint{0.502490in}{0.450700in}}{\pgfqpoint{0.510303in}{0.458514in}}%
\pgfpathcurveto{\pgfqpoint{0.518117in}{0.466328in}}{\pgfqpoint{0.522507in}{0.476927in}}{\pgfqpoint{0.522507in}{0.487977in}}%
\pgfpathcurveto{\pgfqpoint{0.522507in}{0.499027in}}{\pgfqpoint{0.518117in}{0.509626in}}{\pgfqpoint{0.510303in}{0.517440in}}%
\pgfpathcurveto{\pgfqpoint{0.502490in}{0.525253in}}{\pgfqpoint{0.491891in}{0.529644in}}{\pgfqpoint{0.480841in}{0.529644in}}%
\pgfpathcurveto{\pgfqpoint{0.469790in}{0.529644in}}{\pgfqpoint{0.459191in}{0.525253in}}{\pgfqpoint{0.451378in}{0.517440in}}%
\pgfpathcurveto{\pgfqpoint{0.443564in}{0.509626in}}{\pgfqpoint{0.439174in}{0.499027in}}{\pgfqpoint{0.439174in}{0.487977in}}%
\pgfpathcurveto{\pgfqpoint{0.439174in}{0.476927in}}{\pgfqpoint{0.443564in}{0.466328in}}{\pgfqpoint{0.451378in}{0.458514in}}%
\pgfpathcurveto{\pgfqpoint{0.459191in}{0.450700in}}{\pgfqpoint{0.469790in}{0.446310in}}{\pgfqpoint{0.480841in}{0.446310in}}%
\pgfpathclose%
\pgfusepath{stroke,fill}%
\end{pgfscope}%
\begin{pgfscope}%
\pgfpathrectangle{\pgfqpoint{0.375000in}{0.330000in}}{\pgfqpoint{2.325000in}{2.310000in}}%
\pgfusepath{clip}%
\pgfsetbuttcap%
\pgfsetroundjoin%
\definecolor{currentfill}{rgb}{0.000000,0.000000,0.000000}%
\pgfsetfillcolor{currentfill}%
\pgfsetlinewidth{1.003750pt}%
\definecolor{currentstroke}{rgb}{0.000000,0.000000,0.000000}%
\pgfsetstrokecolor{currentstroke}%
\pgfsetdash{}{0pt}%
\pgfpathmoveto{\pgfqpoint{0.480841in}{0.446310in}}%
\pgfpathcurveto{\pgfqpoint{0.491891in}{0.446310in}}{\pgfqpoint{0.502490in}{0.450700in}}{\pgfqpoint{0.510303in}{0.458514in}}%
\pgfpathcurveto{\pgfqpoint{0.518117in}{0.466328in}}{\pgfqpoint{0.522507in}{0.476927in}}{\pgfqpoint{0.522507in}{0.487977in}}%
\pgfpathcurveto{\pgfqpoint{0.522507in}{0.499027in}}{\pgfqpoint{0.518117in}{0.509626in}}{\pgfqpoint{0.510303in}{0.517440in}}%
\pgfpathcurveto{\pgfqpoint{0.502490in}{0.525253in}}{\pgfqpoint{0.491891in}{0.529644in}}{\pgfqpoint{0.480841in}{0.529644in}}%
\pgfpathcurveto{\pgfqpoint{0.469790in}{0.529644in}}{\pgfqpoint{0.459191in}{0.525253in}}{\pgfqpoint{0.451378in}{0.517440in}}%
\pgfpathcurveto{\pgfqpoint{0.443564in}{0.509626in}}{\pgfqpoint{0.439174in}{0.499027in}}{\pgfqpoint{0.439174in}{0.487977in}}%
\pgfpathcurveto{\pgfqpoint{0.439174in}{0.476927in}}{\pgfqpoint{0.443564in}{0.466328in}}{\pgfqpoint{0.451378in}{0.458514in}}%
\pgfpathcurveto{\pgfqpoint{0.459191in}{0.450700in}}{\pgfqpoint{0.469790in}{0.446310in}}{\pgfqpoint{0.480841in}{0.446310in}}%
\pgfpathclose%
\pgfusepath{stroke,fill}%
\end{pgfscope}%
\begin{pgfscope}%
\pgfpathrectangle{\pgfqpoint{0.375000in}{0.330000in}}{\pgfqpoint{2.325000in}{2.310000in}}%
\pgfusepath{clip}%
\pgfsetbuttcap%
\pgfsetroundjoin%
\definecolor{currentfill}{rgb}{0.000000,0.000000,0.000000}%
\pgfsetfillcolor{currentfill}%
\pgfsetlinewidth{1.003750pt}%
\definecolor{currentstroke}{rgb}{0.000000,0.000000,0.000000}%
\pgfsetstrokecolor{currentstroke}%
\pgfsetdash{}{0pt}%
\pgfpathmoveto{\pgfqpoint{0.480841in}{0.446310in}}%
\pgfpathcurveto{\pgfqpoint{0.491891in}{0.446310in}}{\pgfqpoint{0.502490in}{0.450700in}}{\pgfqpoint{0.510303in}{0.458514in}}%
\pgfpathcurveto{\pgfqpoint{0.518117in}{0.466328in}}{\pgfqpoint{0.522507in}{0.476927in}}{\pgfqpoint{0.522507in}{0.487977in}}%
\pgfpathcurveto{\pgfqpoint{0.522507in}{0.499027in}}{\pgfqpoint{0.518117in}{0.509626in}}{\pgfqpoint{0.510303in}{0.517440in}}%
\pgfpathcurveto{\pgfqpoint{0.502490in}{0.525253in}}{\pgfqpoint{0.491891in}{0.529644in}}{\pgfqpoint{0.480841in}{0.529644in}}%
\pgfpathcurveto{\pgfqpoint{0.469790in}{0.529644in}}{\pgfqpoint{0.459191in}{0.525253in}}{\pgfqpoint{0.451378in}{0.517440in}}%
\pgfpathcurveto{\pgfqpoint{0.443564in}{0.509626in}}{\pgfqpoint{0.439174in}{0.499027in}}{\pgfqpoint{0.439174in}{0.487977in}}%
\pgfpathcurveto{\pgfqpoint{0.439174in}{0.476927in}}{\pgfqpoint{0.443564in}{0.466328in}}{\pgfqpoint{0.451378in}{0.458514in}}%
\pgfpathcurveto{\pgfqpoint{0.459191in}{0.450700in}}{\pgfqpoint{0.469790in}{0.446310in}}{\pgfqpoint{0.480841in}{0.446310in}}%
\pgfpathclose%
\pgfusepath{stroke,fill}%
\end{pgfscope}%
\begin{pgfscope}%
\pgfpathrectangle{\pgfqpoint{0.375000in}{0.330000in}}{\pgfqpoint{2.325000in}{2.310000in}}%
\pgfusepath{clip}%
\pgfsetbuttcap%
\pgfsetroundjoin%
\definecolor{currentfill}{rgb}{0.000000,0.000000,0.000000}%
\pgfsetfillcolor{currentfill}%
\pgfsetlinewidth{1.003750pt}%
\definecolor{currentstroke}{rgb}{0.000000,0.000000,0.000000}%
\pgfsetstrokecolor{currentstroke}%
\pgfsetdash{}{0pt}%
\pgfpathmoveto{\pgfqpoint{0.480841in}{0.446310in}}%
\pgfpathcurveto{\pgfqpoint{0.491891in}{0.446310in}}{\pgfqpoint{0.502490in}{0.450700in}}{\pgfqpoint{0.510303in}{0.458514in}}%
\pgfpathcurveto{\pgfqpoint{0.518117in}{0.466328in}}{\pgfqpoint{0.522507in}{0.476927in}}{\pgfqpoint{0.522507in}{0.487977in}}%
\pgfpathcurveto{\pgfqpoint{0.522507in}{0.499027in}}{\pgfqpoint{0.518117in}{0.509626in}}{\pgfqpoint{0.510303in}{0.517440in}}%
\pgfpathcurveto{\pgfqpoint{0.502490in}{0.525253in}}{\pgfqpoint{0.491891in}{0.529644in}}{\pgfqpoint{0.480841in}{0.529644in}}%
\pgfpathcurveto{\pgfqpoint{0.469790in}{0.529644in}}{\pgfqpoint{0.459191in}{0.525253in}}{\pgfqpoint{0.451378in}{0.517440in}}%
\pgfpathcurveto{\pgfqpoint{0.443564in}{0.509626in}}{\pgfqpoint{0.439174in}{0.499027in}}{\pgfqpoint{0.439174in}{0.487977in}}%
\pgfpathcurveto{\pgfqpoint{0.439174in}{0.476927in}}{\pgfqpoint{0.443564in}{0.466328in}}{\pgfqpoint{0.451378in}{0.458514in}}%
\pgfpathcurveto{\pgfqpoint{0.459191in}{0.450700in}}{\pgfqpoint{0.469790in}{0.446310in}}{\pgfqpoint{0.480841in}{0.446310in}}%
\pgfpathclose%
\pgfusepath{stroke,fill}%
\end{pgfscope}%
\begin{pgfscope}%
\pgfpathrectangle{\pgfqpoint{0.375000in}{0.330000in}}{\pgfqpoint{2.325000in}{2.310000in}}%
\pgfusepath{clip}%
\pgfsetbuttcap%
\pgfsetroundjoin%
\definecolor{currentfill}{rgb}{0.000000,0.000000,0.000000}%
\pgfsetfillcolor{currentfill}%
\pgfsetlinewidth{1.003750pt}%
\definecolor{currentstroke}{rgb}{0.000000,0.000000,0.000000}%
\pgfsetstrokecolor{currentstroke}%
\pgfsetdash{}{0pt}%
\pgfpathmoveto{\pgfqpoint{0.480841in}{0.446310in}}%
\pgfpathcurveto{\pgfqpoint{0.491891in}{0.446310in}}{\pgfqpoint{0.502490in}{0.450700in}}{\pgfqpoint{0.510303in}{0.458514in}}%
\pgfpathcurveto{\pgfqpoint{0.518117in}{0.466328in}}{\pgfqpoint{0.522507in}{0.476927in}}{\pgfqpoint{0.522507in}{0.487977in}}%
\pgfpathcurveto{\pgfqpoint{0.522507in}{0.499027in}}{\pgfqpoint{0.518117in}{0.509626in}}{\pgfqpoint{0.510303in}{0.517440in}}%
\pgfpathcurveto{\pgfqpoint{0.502490in}{0.525253in}}{\pgfqpoint{0.491891in}{0.529644in}}{\pgfqpoint{0.480841in}{0.529644in}}%
\pgfpathcurveto{\pgfqpoint{0.469790in}{0.529644in}}{\pgfqpoint{0.459191in}{0.525253in}}{\pgfqpoint{0.451378in}{0.517440in}}%
\pgfpathcurveto{\pgfqpoint{0.443564in}{0.509626in}}{\pgfqpoint{0.439174in}{0.499027in}}{\pgfqpoint{0.439174in}{0.487977in}}%
\pgfpathcurveto{\pgfqpoint{0.439174in}{0.476927in}}{\pgfqpoint{0.443564in}{0.466328in}}{\pgfqpoint{0.451378in}{0.458514in}}%
\pgfpathcurveto{\pgfqpoint{0.459191in}{0.450700in}}{\pgfqpoint{0.469790in}{0.446310in}}{\pgfqpoint{0.480841in}{0.446310in}}%
\pgfpathclose%
\pgfusepath{stroke,fill}%
\end{pgfscope}%
\begin{pgfscope}%
\pgfpathrectangle{\pgfqpoint{0.375000in}{0.330000in}}{\pgfqpoint{2.325000in}{2.310000in}}%
\pgfusepath{clip}%
\pgfsetbuttcap%
\pgfsetroundjoin%
\definecolor{currentfill}{rgb}{0.000000,0.000000,0.000000}%
\pgfsetfillcolor{currentfill}%
\pgfsetlinewidth{1.003750pt}%
\definecolor{currentstroke}{rgb}{0.000000,0.000000,0.000000}%
\pgfsetstrokecolor{currentstroke}%
\pgfsetdash{}{0pt}%
\pgfpathmoveto{\pgfqpoint{0.480841in}{0.446310in}}%
\pgfpathcurveto{\pgfqpoint{0.491891in}{0.446310in}}{\pgfqpoint{0.502490in}{0.450700in}}{\pgfqpoint{0.510303in}{0.458514in}}%
\pgfpathcurveto{\pgfqpoint{0.518117in}{0.466328in}}{\pgfqpoint{0.522507in}{0.476927in}}{\pgfqpoint{0.522507in}{0.487977in}}%
\pgfpathcurveto{\pgfqpoint{0.522507in}{0.499027in}}{\pgfqpoint{0.518117in}{0.509626in}}{\pgfqpoint{0.510303in}{0.517440in}}%
\pgfpathcurveto{\pgfqpoint{0.502490in}{0.525253in}}{\pgfqpoint{0.491891in}{0.529644in}}{\pgfqpoint{0.480841in}{0.529644in}}%
\pgfpathcurveto{\pgfqpoint{0.469790in}{0.529644in}}{\pgfqpoint{0.459191in}{0.525253in}}{\pgfqpoint{0.451378in}{0.517440in}}%
\pgfpathcurveto{\pgfqpoint{0.443564in}{0.509626in}}{\pgfqpoint{0.439174in}{0.499027in}}{\pgfqpoint{0.439174in}{0.487977in}}%
\pgfpathcurveto{\pgfqpoint{0.439174in}{0.476927in}}{\pgfqpoint{0.443564in}{0.466328in}}{\pgfqpoint{0.451378in}{0.458514in}}%
\pgfpathcurveto{\pgfqpoint{0.459191in}{0.450700in}}{\pgfqpoint{0.469790in}{0.446310in}}{\pgfqpoint{0.480841in}{0.446310in}}%
\pgfpathclose%
\pgfusepath{stroke,fill}%
\end{pgfscope}%
\begin{pgfscope}%
\pgfpathrectangle{\pgfqpoint{0.375000in}{0.330000in}}{\pgfqpoint{2.325000in}{2.310000in}}%
\pgfusepath{clip}%
\pgfsetbuttcap%
\pgfsetroundjoin%
\definecolor{currentfill}{rgb}{0.000000,0.000000,0.000000}%
\pgfsetfillcolor{currentfill}%
\pgfsetlinewidth{1.003750pt}%
\definecolor{currentstroke}{rgb}{0.000000,0.000000,0.000000}%
\pgfsetstrokecolor{currentstroke}%
\pgfsetdash{}{0pt}%
\pgfpathmoveto{\pgfqpoint{0.480841in}{0.446310in}}%
\pgfpathcurveto{\pgfqpoint{0.491891in}{0.446310in}}{\pgfqpoint{0.502490in}{0.450700in}}{\pgfqpoint{0.510303in}{0.458514in}}%
\pgfpathcurveto{\pgfqpoint{0.518117in}{0.466328in}}{\pgfqpoint{0.522507in}{0.476927in}}{\pgfqpoint{0.522507in}{0.487977in}}%
\pgfpathcurveto{\pgfqpoint{0.522507in}{0.499027in}}{\pgfqpoint{0.518117in}{0.509626in}}{\pgfqpoint{0.510303in}{0.517440in}}%
\pgfpathcurveto{\pgfqpoint{0.502490in}{0.525253in}}{\pgfqpoint{0.491891in}{0.529644in}}{\pgfqpoint{0.480841in}{0.529644in}}%
\pgfpathcurveto{\pgfqpoint{0.469790in}{0.529644in}}{\pgfqpoint{0.459191in}{0.525253in}}{\pgfqpoint{0.451378in}{0.517440in}}%
\pgfpathcurveto{\pgfqpoint{0.443564in}{0.509626in}}{\pgfqpoint{0.439174in}{0.499027in}}{\pgfqpoint{0.439174in}{0.487977in}}%
\pgfpathcurveto{\pgfqpoint{0.439174in}{0.476927in}}{\pgfqpoint{0.443564in}{0.466328in}}{\pgfqpoint{0.451378in}{0.458514in}}%
\pgfpathcurveto{\pgfqpoint{0.459191in}{0.450700in}}{\pgfqpoint{0.469790in}{0.446310in}}{\pgfqpoint{0.480841in}{0.446310in}}%
\pgfpathclose%
\pgfusepath{stroke,fill}%
\end{pgfscope}%
\begin{pgfscope}%
\pgfpathrectangle{\pgfqpoint{0.375000in}{0.330000in}}{\pgfqpoint{2.325000in}{2.310000in}}%
\pgfusepath{clip}%
\pgfsetbuttcap%
\pgfsetroundjoin%
\definecolor{currentfill}{rgb}{0.000000,0.000000,0.000000}%
\pgfsetfillcolor{currentfill}%
\pgfsetlinewidth{1.003750pt}%
\definecolor{currentstroke}{rgb}{0.000000,0.000000,0.000000}%
\pgfsetstrokecolor{currentstroke}%
\pgfsetdash{}{0pt}%
\pgfpathmoveto{\pgfqpoint{0.480841in}{0.446310in}}%
\pgfpathcurveto{\pgfqpoint{0.491891in}{0.446310in}}{\pgfqpoint{0.502490in}{0.450700in}}{\pgfqpoint{0.510303in}{0.458514in}}%
\pgfpathcurveto{\pgfqpoint{0.518117in}{0.466328in}}{\pgfqpoint{0.522507in}{0.476927in}}{\pgfqpoint{0.522507in}{0.487977in}}%
\pgfpathcurveto{\pgfqpoint{0.522507in}{0.499027in}}{\pgfqpoint{0.518117in}{0.509626in}}{\pgfqpoint{0.510303in}{0.517440in}}%
\pgfpathcurveto{\pgfqpoint{0.502490in}{0.525253in}}{\pgfqpoint{0.491891in}{0.529644in}}{\pgfqpoint{0.480841in}{0.529644in}}%
\pgfpathcurveto{\pgfqpoint{0.469790in}{0.529644in}}{\pgfqpoint{0.459191in}{0.525253in}}{\pgfqpoint{0.451378in}{0.517440in}}%
\pgfpathcurveto{\pgfqpoint{0.443564in}{0.509626in}}{\pgfqpoint{0.439174in}{0.499027in}}{\pgfqpoint{0.439174in}{0.487977in}}%
\pgfpathcurveto{\pgfqpoint{0.439174in}{0.476927in}}{\pgfqpoint{0.443564in}{0.466328in}}{\pgfqpoint{0.451378in}{0.458514in}}%
\pgfpathcurveto{\pgfqpoint{0.459191in}{0.450700in}}{\pgfqpoint{0.469790in}{0.446310in}}{\pgfqpoint{0.480841in}{0.446310in}}%
\pgfpathclose%
\pgfusepath{stroke,fill}%
\end{pgfscope}%
\begin{pgfscope}%
\pgfpathrectangle{\pgfqpoint{0.375000in}{0.330000in}}{\pgfqpoint{2.325000in}{2.310000in}}%
\pgfusepath{clip}%
\pgfsetbuttcap%
\pgfsetroundjoin%
\definecolor{currentfill}{rgb}{0.000000,0.000000,0.000000}%
\pgfsetfillcolor{currentfill}%
\pgfsetlinewidth{1.003750pt}%
\definecolor{currentstroke}{rgb}{0.000000,0.000000,0.000000}%
\pgfsetstrokecolor{currentstroke}%
\pgfsetdash{}{0pt}%
\pgfpathmoveto{\pgfqpoint{0.480841in}{0.446310in}}%
\pgfpathcurveto{\pgfqpoint{0.491891in}{0.446310in}}{\pgfqpoint{0.502490in}{0.450700in}}{\pgfqpoint{0.510303in}{0.458514in}}%
\pgfpathcurveto{\pgfqpoint{0.518117in}{0.466328in}}{\pgfqpoint{0.522507in}{0.476927in}}{\pgfqpoint{0.522507in}{0.487977in}}%
\pgfpathcurveto{\pgfqpoint{0.522507in}{0.499027in}}{\pgfqpoint{0.518117in}{0.509626in}}{\pgfqpoint{0.510303in}{0.517440in}}%
\pgfpathcurveto{\pgfqpoint{0.502490in}{0.525253in}}{\pgfqpoint{0.491891in}{0.529644in}}{\pgfqpoint{0.480841in}{0.529644in}}%
\pgfpathcurveto{\pgfqpoint{0.469790in}{0.529644in}}{\pgfqpoint{0.459191in}{0.525253in}}{\pgfqpoint{0.451378in}{0.517440in}}%
\pgfpathcurveto{\pgfqpoint{0.443564in}{0.509626in}}{\pgfqpoint{0.439174in}{0.499027in}}{\pgfqpoint{0.439174in}{0.487977in}}%
\pgfpathcurveto{\pgfqpoint{0.439174in}{0.476927in}}{\pgfqpoint{0.443564in}{0.466328in}}{\pgfqpoint{0.451378in}{0.458514in}}%
\pgfpathcurveto{\pgfqpoint{0.459191in}{0.450700in}}{\pgfqpoint{0.469790in}{0.446310in}}{\pgfqpoint{0.480841in}{0.446310in}}%
\pgfpathclose%
\pgfusepath{stroke,fill}%
\end{pgfscope}%
\begin{pgfscope}%
\pgfpathrectangle{\pgfqpoint{0.375000in}{0.330000in}}{\pgfqpoint{2.325000in}{2.310000in}}%
\pgfusepath{clip}%
\pgfsetbuttcap%
\pgfsetroundjoin%
\definecolor{currentfill}{rgb}{0.000000,0.000000,0.000000}%
\pgfsetfillcolor{currentfill}%
\pgfsetlinewidth{1.003750pt}%
\definecolor{currentstroke}{rgb}{0.000000,0.000000,0.000000}%
\pgfsetstrokecolor{currentstroke}%
\pgfsetdash{}{0pt}%
\pgfpathmoveto{\pgfqpoint{0.480841in}{0.446310in}}%
\pgfpathcurveto{\pgfqpoint{0.491891in}{0.446310in}}{\pgfqpoint{0.502490in}{0.450700in}}{\pgfqpoint{0.510303in}{0.458514in}}%
\pgfpathcurveto{\pgfqpoint{0.518117in}{0.466328in}}{\pgfqpoint{0.522507in}{0.476927in}}{\pgfqpoint{0.522507in}{0.487977in}}%
\pgfpathcurveto{\pgfqpoint{0.522507in}{0.499027in}}{\pgfqpoint{0.518117in}{0.509626in}}{\pgfqpoint{0.510303in}{0.517440in}}%
\pgfpathcurveto{\pgfqpoint{0.502490in}{0.525253in}}{\pgfqpoint{0.491891in}{0.529644in}}{\pgfqpoint{0.480841in}{0.529644in}}%
\pgfpathcurveto{\pgfqpoint{0.469790in}{0.529644in}}{\pgfqpoint{0.459191in}{0.525253in}}{\pgfqpoint{0.451378in}{0.517440in}}%
\pgfpathcurveto{\pgfqpoint{0.443564in}{0.509626in}}{\pgfqpoint{0.439174in}{0.499027in}}{\pgfqpoint{0.439174in}{0.487977in}}%
\pgfpathcurveto{\pgfqpoint{0.439174in}{0.476927in}}{\pgfqpoint{0.443564in}{0.466328in}}{\pgfqpoint{0.451378in}{0.458514in}}%
\pgfpathcurveto{\pgfqpoint{0.459191in}{0.450700in}}{\pgfqpoint{0.469790in}{0.446310in}}{\pgfqpoint{0.480841in}{0.446310in}}%
\pgfpathclose%
\pgfusepath{stroke,fill}%
\end{pgfscope}%
\begin{pgfscope}%
\pgfpathrectangle{\pgfqpoint{0.375000in}{0.330000in}}{\pgfqpoint{2.325000in}{2.310000in}}%
\pgfusepath{clip}%
\pgfsetbuttcap%
\pgfsetroundjoin%
\definecolor{currentfill}{rgb}{0.000000,0.000000,0.000000}%
\pgfsetfillcolor{currentfill}%
\pgfsetlinewidth{1.003750pt}%
\definecolor{currentstroke}{rgb}{0.000000,0.000000,0.000000}%
\pgfsetstrokecolor{currentstroke}%
\pgfsetdash{}{0pt}%
\pgfpathmoveto{\pgfqpoint{0.480841in}{0.446310in}}%
\pgfpathcurveto{\pgfqpoint{0.491891in}{0.446310in}}{\pgfqpoint{0.502490in}{0.450700in}}{\pgfqpoint{0.510303in}{0.458514in}}%
\pgfpathcurveto{\pgfqpoint{0.518117in}{0.466328in}}{\pgfqpoint{0.522507in}{0.476927in}}{\pgfqpoint{0.522507in}{0.487977in}}%
\pgfpathcurveto{\pgfqpoint{0.522507in}{0.499027in}}{\pgfqpoint{0.518117in}{0.509626in}}{\pgfqpoint{0.510303in}{0.517440in}}%
\pgfpathcurveto{\pgfqpoint{0.502490in}{0.525253in}}{\pgfqpoint{0.491891in}{0.529644in}}{\pgfqpoint{0.480841in}{0.529644in}}%
\pgfpathcurveto{\pgfqpoint{0.469790in}{0.529644in}}{\pgfqpoint{0.459191in}{0.525253in}}{\pgfqpoint{0.451378in}{0.517440in}}%
\pgfpathcurveto{\pgfqpoint{0.443564in}{0.509626in}}{\pgfqpoint{0.439174in}{0.499027in}}{\pgfqpoint{0.439174in}{0.487977in}}%
\pgfpathcurveto{\pgfqpoint{0.439174in}{0.476927in}}{\pgfqpoint{0.443564in}{0.466328in}}{\pgfqpoint{0.451378in}{0.458514in}}%
\pgfpathcurveto{\pgfqpoint{0.459191in}{0.450700in}}{\pgfqpoint{0.469790in}{0.446310in}}{\pgfqpoint{0.480841in}{0.446310in}}%
\pgfpathclose%
\pgfusepath{stroke,fill}%
\end{pgfscope}%
\begin{pgfscope}%
\pgfpathrectangle{\pgfqpoint{0.375000in}{0.330000in}}{\pgfqpoint{2.325000in}{2.310000in}}%
\pgfusepath{clip}%
\pgfsetbuttcap%
\pgfsetroundjoin%
\definecolor{currentfill}{rgb}{0.000000,0.000000,0.000000}%
\pgfsetfillcolor{currentfill}%
\pgfsetlinewidth{1.003750pt}%
\definecolor{currentstroke}{rgb}{0.000000,0.000000,0.000000}%
\pgfsetstrokecolor{currentstroke}%
\pgfsetdash{}{0pt}%
\pgfpathmoveto{\pgfqpoint{0.480841in}{0.446310in}}%
\pgfpathcurveto{\pgfqpoint{0.491891in}{0.446310in}}{\pgfqpoint{0.502490in}{0.450700in}}{\pgfqpoint{0.510303in}{0.458514in}}%
\pgfpathcurveto{\pgfqpoint{0.518117in}{0.466328in}}{\pgfqpoint{0.522507in}{0.476927in}}{\pgfqpoint{0.522507in}{0.487977in}}%
\pgfpathcurveto{\pgfqpoint{0.522507in}{0.499027in}}{\pgfqpoint{0.518117in}{0.509626in}}{\pgfqpoint{0.510303in}{0.517440in}}%
\pgfpathcurveto{\pgfqpoint{0.502490in}{0.525253in}}{\pgfqpoint{0.491891in}{0.529644in}}{\pgfqpoint{0.480841in}{0.529644in}}%
\pgfpathcurveto{\pgfqpoint{0.469790in}{0.529644in}}{\pgfqpoint{0.459191in}{0.525253in}}{\pgfqpoint{0.451378in}{0.517440in}}%
\pgfpathcurveto{\pgfqpoint{0.443564in}{0.509626in}}{\pgfqpoint{0.439174in}{0.499027in}}{\pgfqpoint{0.439174in}{0.487977in}}%
\pgfpathcurveto{\pgfqpoint{0.439174in}{0.476927in}}{\pgfqpoint{0.443564in}{0.466328in}}{\pgfqpoint{0.451378in}{0.458514in}}%
\pgfpathcurveto{\pgfqpoint{0.459191in}{0.450700in}}{\pgfqpoint{0.469790in}{0.446310in}}{\pgfqpoint{0.480841in}{0.446310in}}%
\pgfpathclose%
\pgfusepath{stroke,fill}%
\end{pgfscope}%
\begin{pgfscope}%
\pgfpathrectangle{\pgfqpoint{0.375000in}{0.330000in}}{\pgfqpoint{2.325000in}{2.310000in}}%
\pgfusepath{clip}%
\pgfsetbuttcap%
\pgfsetroundjoin%
\definecolor{currentfill}{rgb}{0.000000,0.000000,0.000000}%
\pgfsetfillcolor{currentfill}%
\pgfsetlinewidth{1.003750pt}%
\definecolor{currentstroke}{rgb}{0.000000,0.000000,0.000000}%
\pgfsetstrokecolor{currentstroke}%
\pgfsetdash{}{0pt}%
\pgfpathmoveto{\pgfqpoint{0.480841in}{0.446310in}}%
\pgfpathcurveto{\pgfqpoint{0.491891in}{0.446310in}}{\pgfqpoint{0.502490in}{0.450700in}}{\pgfqpoint{0.510303in}{0.458514in}}%
\pgfpathcurveto{\pgfqpoint{0.518117in}{0.466328in}}{\pgfqpoint{0.522507in}{0.476927in}}{\pgfqpoint{0.522507in}{0.487977in}}%
\pgfpathcurveto{\pgfqpoint{0.522507in}{0.499027in}}{\pgfqpoint{0.518117in}{0.509626in}}{\pgfqpoint{0.510303in}{0.517440in}}%
\pgfpathcurveto{\pgfqpoint{0.502490in}{0.525253in}}{\pgfqpoint{0.491891in}{0.529644in}}{\pgfqpoint{0.480841in}{0.529644in}}%
\pgfpathcurveto{\pgfqpoint{0.469790in}{0.529644in}}{\pgfqpoint{0.459191in}{0.525253in}}{\pgfqpoint{0.451378in}{0.517440in}}%
\pgfpathcurveto{\pgfqpoint{0.443564in}{0.509626in}}{\pgfqpoint{0.439174in}{0.499027in}}{\pgfqpoint{0.439174in}{0.487977in}}%
\pgfpathcurveto{\pgfqpoint{0.439174in}{0.476927in}}{\pgfqpoint{0.443564in}{0.466328in}}{\pgfqpoint{0.451378in}{0.458514in}}%
\pgfpathcurveto{\pgfqpoint{0.459191in}{0.450700in}}{\pgfqpoint{0.469790in}{0.446310in}}{\pgfqpoint{0.480841in}{0.446310in}}%
\pgfpathclose%
\pgfusepath{stroke,fill}%
\end{pgfscope}%
\begin{pgfscope}%
\pgfpathrectangle{\pgfqpoint{0.375000in}{0.330000in}}{\pgfqpoint{2.325000in}{2.310000in}}%
\pgfusepath{clip}%
\pgfsetbuttcap%
\pgfsetroundjoin%
\definecolor{currentfill}{rgb}{0.000000,0.000000,0.000000}%
\pgfsetfillcolor{currentfill}%
\pgfsetlinewidth{1.003750pt}%
\definecolor{currentstroke}{rgb}{0.000000,0.000000,0.000000}%
\pgfsetstrokecolor{currentstroke}%
\pgfsetdash{}{0pt}%
\pgfpathmoveto{\pgfqpoint{0.480841in}{0.446310in}}%
\pgfpathcurveto{\pgfqpoint{0.491891in}{0.446310in}}{\pgfqpoint{0.502490in}{0.450700in}}{\pgfqpoint{0.510303in}{0.458514in}}%
\pgfpathcurveto{\pgfqpoint{0.518117in}{0.466328in}}{\pgfqpoint{0.522507in}{0.476927in}}{\pgfqpoint{0.522507in}{0.487977in}}%
\pgfpathcurveto{\pgfqpoint{0.522507in}{0.499027in}}{\pgfqpoint{0.518117in}{0.509626in}}{\pgfqpoint{0.510303in}{0.517440in}}%
\pgfpathcurveto{\pgfqpoint{0.502490in}{0.525253in}}{\pgfqpoint{0.491891in}{0.529644in}}{\pgfqpoint{0.480841in}{0.529644in}}%
\pgfpathcurveto{\pgfqpoint{0.469790in}{0.529644in}}{\pgfqpoint{0.459191in}{0.525253in}}{\pgfqpoint{0.451378in}{0.517440in}}%
\pgfpathcurveto{\pgfqpoint{0.443564in}{0.509626in}}{\pgfqpoint{0.439174in}{0.499027in}}{\pgfqpoint{0.439174in}{0.487977in}}%
\pgfpathcurveto{\pgfqpoint{0.439174in}{0.476927in}}{\pgfqpoint{0.443564in}{0.466328in}}{\pgfqpoint{0.451378in}{0.458514in}}%
\pgfpathcurveto{\pgfqpoint{0.459191in}{0.450700in}}{\pgfqpoint{0.469790in}{0.446310in}}{\pgfqpoint{0.480841in}{0.446310in}}%
\pgfpathclose%
\pgfusepath{stroke,fill}%
\end{pgfscope}%
\begin{pgfscope}%
\pgfpathrectangle{\pgfqpoint{0.375000in}{0.330000in}}{\pgfqpoint{2.325000in}{2.310000in}}%
\pgfusepath{clip}%
\pgfsetbuttcap%
\pgfsetroundjoin%
\definecolor{currentfill}{rgb}{0.000000,0.000000,0.000000}%
\pgfsetfillcolor{currentfill}%
\pgfsetlinewidth{1.003750pt}%
\definecolor{currentstroke}{rgb}{0.000000,0.000000,0.000000}%
\pgfsetstrokecolor{currentstroke}%
\pgfsetdash{}{0pt}%
\pgfpathmoveto{\pgfqpoint{0.480841in}{0.446310in}}%
\pgfpathcurveto{\pgfqpoint{0.491891in}{0.446310in}}{\pgfqpoint{0.502490in}{0.450700in}}{\pgfqpoint{0.510303in}{0.458514in}}%
\pgfpathcurveto{\pgfqpoint{0.518117in}{0.466328in}}{\pgfqpoint{0.522507in}{0.476927in}}{\pgfqpoint{0.522507in}{0.487977in}}%
\pgfpathcurveto{\pgfqpoint{0.522507in}{0.499027in}}{\pgfqpoint{0.518117in}{0.509626in}}{\pgfqpoint{0.510303in}{0.517440in}}%
\pgfpathcurveto{\pgfqpoint{0.502490in}{0.525253in}}{\pgfqpoint{0.491891in}{0.529644in}}{\pgfqpoint{0.480841in}{0.529644in}}%
\pgfpathcurveto{\pgfqpoint{0.469790in}{0.529644in}}{\pgfqpoint{0.459191in}{0.525253in}}{\pgfqpoint{0.451378in}{0.517440in}}%
\pgfpathcurveto{\pgfqpoint{0.443564in}{0.509626in}}{\pgfqpoint{0.439174in}{0.499027in}}{\pgfqpoint{0.439174in}{0.487977in}}%
\pgfpathcurveto{\pgfqpoint{0.439174in}{0.476927in}}{\pgfqpoint{0.443564in}{0.466328in}}{\pgfqpoint{0.451378in}{0.458514in}}%
\pgfpathcurveto{\pgfqpoint{0.459191in}{0.450700in}}{\pgfqpoint{0.469790in}{0.446310in}}{\pgfqpoint{0.480841in}{0.446310in}}%
\pgfpathclose%
\pgfusepath{stroke,fill}%
\end{pgfscope}%
\begin{pgfscope}%
\pgfpathrectangle{\pgfqpoint{0.375000in}{0.330000in}}{\pgfqpoint{2.325000in}{2.310000in}}%
\pgfusepath{clip}%
\pgfsetbuttcap%
\pgfsetroundjoin%
\definecolor{currentfill}{rgb}{0.000000,0.000000,0.000000}%
\pgfsetfillcolor{currentfill}%
\pgfsetlinewidth{1.003750pt}%
\definecolor{currentstroke}{rgb}{0.000000,0.000000,0.000000}%
\pgfsetstrokecolor{currentstroke}%
\pgfsetdash{}{0pt}%
\pgfpathmoveto{\pgfqpoint{0.480841in}{0.446310in}}%
\pgfpathcurveto{\pgfqpoint{0.491891in}{0.446310in}}{\pgfqpoint{0.502490in}{0.450700in}}{\pgfqpoint{0.510303in}{0.458514in}}%
\pgfpathcurveto{\pgfqpoint{0.518117in}{0.466328in}}{\pgfqpoint{0.522507in}{0.476927in}}{\pgfqpoint{0.522507in}{0.487977in}}%
\pgfpathcurveto{\pgfqpoint{0.522507in}{0.499027in}}{\pgfqpoint{0.518117in}{0.509626in}}{\pgfqpoint{0.510303in}{0.517440in}}%
\pgfpathcurveto{\pgfqpoint{0.502490in}{0.525253in}}{\pgfqpoint{0.491891in}{0.529644in}}{\pgfqpoint{0.480841in}{0.529644in}}%
\pgfpathcurveto{\pgfqpoint{0.469790in}{0.529644in}}{\pgfqpoint{0.459191in}{0.525253in}}{\pgfqpoint{0.451378in}{0.517440in}}%
\pgfpathcurveto{\pgfqpoint{0.443564in}{0.509626in}}{\pgfqpoint{0.439174in}{0.499027in}}{\pgfqpoint{0.439174in}{0.487977in}}%
\pgfpathcurveto{\pgfqpoint{0.439174in}{0.476927in}}{\pgfqpoint{0.443564in}{0.466328in}}{\pgfqpoint{0.451378in}{0.458514in}}%
\pgfpathcurveto{\pgfqpoint{0.459191in}{0.450700in}}{\pgfqpoint{0.469790in}{0.446310in}}{\pgfqpoint{0.480841in}{0.446310in}}%
\pgfpathclose%
\pgfusepath{stroke,fill}%
\end{pgfscope}%
\begin{pgfscope}%
\pgfpathrectangle{\pgfqpoint{0.375000in}{0.330000in}}{\pgfqpoint{2.325000in}{2.310000in}}%
\pgfusepath{clip}%
\pgfsetbuttcap%
\pgfsetroundjoin%
\definecolor{currentfill}{rgb}{0.000000,0.000000,0.000000}%
\pgfsetfillcolor{currentfill}%
\pgfsetlinewidth{1.003750pt}%
\definecolor{currentstroke}{rgb}{0.000000,0.000000,0.000000}%
\pgfsetstrokecolor{currentstroke}%
\pgfsetdash{}{0pt}%
\pgfpathmoveto{\pgfqpoint{0.480841in}{0.446310in}}%
\pgfpathcurveto{\pgfqpoint{0.491891in}{0.446310in}}{\pgfqpoint{0.502490in}{0.450700in}}{\pgfqpoint{0.510303in}{0.458514in}}%
\pgfpathcurveto{\pgfqpoint{0.518117in}{0.466328in}}{\pgfqpoint{0.522507in}{0.476927in}}{\pgfqpoint{0.522507in}{0.487977in}}%
\pgfpathcurveto{\pgfqpoint{0.522507in}{0.499027in}}{\pgfqpoint{0.518117in}{0.509626in}}{\pgfqpoint{0.510303in}{0.517440in}}%
\pgfpathcurveto{\pgfqpoint{0.502490in}{0.525253in}}{\pgfqpoint{0.491891in}{0.529644in}}{\pgfqpoint{0.480841in}{0.529644in}}%
\pgfpathcurveto{\pgfqpoint{0.469790in}{0.529644in}}{\pgfqpoint{0.459191in}{0.525253in}}{\pgfqpoint{0.451378in}{0.517440in}}%
\pgfpathcurveto{\pgfqpoint{0.443564in}{0.509626in}}{\pgfqpoint{0.439174in}{0.499027in}}{\pgfqpoint{0.439174in}{0.487977in}}%
\pgfpathcurveto{\pgfqpoint{0.439174in}{0.476927in}}{\pgfqpoint{0.443564in}{0.466328in}}{\pgfqpoint{0.451378in}{0.458514in}}%
\pgfpathcurveto{\pgfqpoint{0.459191in}{0.450700in}}{\pgfqpoint{0.469790in}{0.446310in}}{\pgfqpoint{0.480841in}{0.446310in}}%
\pgfpathclose%
\pgfusepath{stroke,fill}%
\end{pgfscope}%
\begin{pgfscope}%
\pgfpathrectangle{\pgfqpoint{0.375000in}{0.330000in}}{\pgfqpoint{2.325000in}{2.310000in}}%
\pgfusepath{clip}%
\pgfsetbuttcap%
\pgfsetroundjoin%
\definecolor{currentfill}{rgb}{0.000000,0.000000,0.000000}%
\pgfsetfillcolor{currentfill}%
\pgfsetlinewidth{1.003750pt}%
\definecolor{currentstroke}{rgb}{0.000000,0.000000,0.000000}%
\pgfsetstrokecolor{currentstroke}%
\pgfsetdash{}{0pt}%
\pgfpathmoveto{\pgfqpoint{0.480841in}{0.446310in}}%
\pgfpathcurveto{\pgfqpoint{0.491891in}{0.446310in}}{\pgfqpoint{0.502490in}{0.450700in}}{\pgfqpoint{0.510303in}{0.458514in}}%
\pgfpathcurveto{\pgfqpoint{0.518117in}{0.466328in}}{\pgfqpoint{0.522507in}{0.476927in}}{\pgfqpoint{0.522507in}{0.487977in}}%
\pgfpathcurveto{\pgfqpoint{0.522507in}{0.499027in}}{\pgfqpoint{0.518117in}{0.509626in}}{\pgfqpoint{0.510303in}{0.517440in}}%
\pgfpathcurveto{\pgfqpoint{0.502490in}{0.525253in}}{\pgfqpoint{0.491891in}{0.529644in}}{\pgfqpoint{0.480841in}{0.529644in}}%
\pgfpathcurveto{\pgfqpoint{0.469790in}{0.529644in}}{\pgfqpoint{0.459191in}{0.525253in}}{\pgfqpoint{0.451378in}{0.517440in}}%
\pgfpathcurveto{\pgfqpoint{0.443564in}{0.509626in}}{\pgfqpoint{0.439174in}{0.499027in}}{\pgfqpoint{0.439174in}{0.487977in}}%
\pgfpathcurveto{\pgfqpoint{0.439174in}{0.476927in}}{\pgfqpoint{0.443564in}{0.466328in}}{\pgfqpoint{0.451378in}{0.458514in}}%
\pgfpathcurveto{\pgfqpoint{0.459191in}{0.450700in}}{\pgfqpoint{0.469790in}{0.446310in}}{\pgfqpoint{0.480841in}{0.446310in}}%
\pgfpathclose%
\pgfusepath{stroke,fill}%
\end{pgfscope}%
\begin{pgfscope}%
\pgfpathrectangle{\pgfqpoint{0.375000in}{0.330000in}}{\pgfqpoint{2.325000in}{2.310000in}}%
\pgfusepath{clip}%
\pgfsetbuttcap%
\pgfsetroundjoin%
\definecolor{currentfill}{rgb}{0.000000,0.000000,0.000000}%
\pgfsetfillcolor{currentfill}%
\pgfsetlinewidth{1.003750pt}%
\definecolor{currentstroke}{rgb}{0.000000,0.000000,0.000000}%
\pgfsetstrokecolor{currentstroke}%
\pgfsetdash{}{0pt}%
\pgfpathmoveto{\pgfqpoint{0.480841in}{0.446310in}}%
\pgfpathcurveto{\pgfqpoint{0.491891in}{0.446310in}}{\pgfqpoint{0.502490in}{0.450700in}}{\pgfqpoint{0.510303in}{0.458514in}}%
\pgfpathcurveto{\pgfqpoint{0.518117in}{0.466328in}}{\pgfqpoint{0.522507in}{0.476927in}}{\pgfqpoint{0.522507in}{0.487977in}}%
\pgfpathcurveto{\pgfqpoint{0.522507in}{0.499027in}}{\pgfqpoint{0.518117in}{0.509626in}}{\pgfqpoint{0.510303in}{0.517440in}}%
\pgfpathcurveto{\pgfqpoint{0.502490in}{0.525253in}}{\pgfqpoint{0.491891in}{0.529644in}}{\pgfqpoint{0.480841in}{0.529644in}}%
\pgfpathcurveto{\pgfqpoint{0.469790in}{0.529644in}}{\pgfqpoint{0.459191in}{0.525253in}}{\pgfqpoint{0.451378in}{0.517440in}}%
\pgfpathcurveto{\pgfqpoint{0.443564in}{0.509626in}}{\pgfqpoint{0.439174in}{0.499027in}}{\pgfqpoint{0.439174in}{0.487977in}}%
\pgfpathcurveto{\pgfqpoint{0.439174in}{0.476927in}}{\pgfqpoint{0.443564in}{0.466328in}}{\pgfqpoint{0.451378in}{0.458514in}}%
\pgfpathcurveto{\pgfqpoint{0.459191in}{0.450700in}}{\pgfqpoint{0.469790in}{0.446310in}}{\pgfqpoint{0.480841in}{0.446310in}}%
\pgfpathclose%
\pgfusepath{stroke,fill}%
\end{pgfscope}%
\begin{pgfscope}%
\pgfpathrectangle{\pgfqpoint{0.375000in}{0.330000in}}{\pgfqpoint{2.325000in}{2.310000in}}%
\pgfusepath{clip}%
\pgfsetbuttcap%
\pgfsetroundjoin%
\definecolor{currentfill}{rgb}{0.000000,0.000000,0.000000}%
\pgfsetfillcolor{currentfill}%
\pgfsetlinewidth{1.003750pt}%
\definecolor{currentstroke}{rgb}{0.000000,0.000000,0.000000}%
\pgfsetstrokecolor{currentstroke}%
\pgfsetdash{}{0pt}%
\pgfpathmoveto{\pgfqpoint{0.480841in}{0.446310in}}%
\pgfpathcurveto{\pgfqpoint{0.491891in}{0.446310in}}{\pgfqpoint{0.502490in}{0.450700in}}{\pgfqpoint{0.510303in}{0.458514in}}%
\pgfpathcurveto{\pgfqpoint{0.518117in}{0.466328in}}{\pgfqpoint{0.522507in}{0.476927in}}{\pgfqpoint{0.522507in}{0.487977in}}%
\pgfpathcurveto{\pgfqpoint{0.522507in}{0.499027in}}{\pgfqpoint{0.518117in}{0.509626in}}{\pgfqpoint{0.510303in}{0.517440in}}%
\pgfpathcurveto{\pgfqpoint{0.502490in}{0.525253in}}{\pgfqpoint{0.491891in}{0.529644in}}{\pgfqpoint{0.480841in}{0.529644in}}%
\pgfpathcurveto{\pgfqpoint{0.469790in}{0.529644in}}{\pgfqpoint{0.459191in}{0.525253in}}{\pgfqpoint{0.451378in}{0.517440in}}%
\pgfpathcurveto{\pgfqpoint{0.443564in}{0.509626in}}{\pgfqpoint{0.439174in}{0.499027in}}{\pgfqpoint{0.439174in}{0.487977in}}%
\pgfpathcurveto{\pgfqpoint{0.439174in}{0.476927in}}{\pgfqpoint{0.443564in}{0.466328in}}{\pgfqpoint{0.451378in}{0.458514in}}%
\pgfpathcurveto{\pgfqpoint{0.459191in}{0.450700in}}{\pgfqpoint{0.469790in}{0.446310in}}{\pgfqpoint{0.480841in}{0.446310in}}%
\pgfpathclose%
\pgfusepath{stroke,fill}%
\end{pgfscope}%
\begin{pgfscope}%
\pgfpathrectangle{\pgfqpoint{0.375000in}{0.330000in}}{\pgfqpoint{2.325000in}{2.310000in}}%
\pgfusepath{clip}%
\pgfsetbuttcap%
\pgfsetroundjoin%
\definecolor{currentfill}{rgb}{0.000000,0.000000,0.000000}%
\pgfsetfillcolor{currentfill}%
\pgfsetlinewidth{1.003750pt}%
\definecolor{currentstroke}{rgb}{0.000000,0.000000,0.000000}%
\pgfsetstrokecolor{currentstroke}%
\pgfsetdash{}{0pt}%
\pgfpathmoveto{\pgfqpoint{0.480841in}{0.446310in}}%
\pgfpathcurveto{\pgfqpoint{0.491891in}{0.446310in}}{\pgfqpoint{0.502490in}{0.450700in}}{\pgfqpoint{0.510303in}{0.458514in}}%
\pgfpathcurveto{\pgfqpoint{0.518117in}{0.466328in}}{\pgfqpoint{0.522507in}{0.476927in}}{\pgfqpoint{0.522507in}{0.487977in}}%
\pgfpathcurveto{\pgfqpoint{0.522507in}{0.499027in}}{\pgfqpoint{0.518117in}{0.509626in}}{\pgfqpoint{0.510303in}{0.517440in}}%
\pgfpathcurveto{\pgfqpoint{0.502490in}{0.525253in}}{\pgfqpoint{0.491891in}{0.529644in}}{\pgfqpoint{0.480841in}{0.529644in}}%
\pgfpathcurveto{\pgfqpoint{0.469790in}{0.529644in}}{\pgfqpoint{0.459191in}{0.525253in}}{\pgfqpoint{0.451378in}{0.517440in}}%
\pgfpathcurveto{\pgfqpoint{0.443564in}{0.509626in}}{\pgfqpoint{0.439174in}{0.499027in}}{\pgfqpoint{0.439174in}{0.487977in}}%
\pgfpathcurveto{\pgfqpoint{0.439174in}{0.476927in}}{\pgfqpoint{0.443564in}{0.466328in}}{\pgfqpoint{0.451378in}{0.458514in}}%
\pgfpathcurveto{\pgfqpoint{0.459191in}{0.450700in}}{\pgfqpoint{0.469790in}{0.446310in}}{\pgfqpoint{0.480841in}{0.446310in}}%
\pgfpathclose%
\pgfusepath{stroke,fill}%
\end{pgfscope}%
\begin{pgfscope}%
\pgfpathrectangle{\pgfqpoint{0.375000in}{0.330000in}}{\pgfqpoint{2.325000in}{2.310000in}}%
\pgfusepath{clip}%
\pgfsetbuttcap%
\pgfsetroundjoin%
\definecolor{currentfill}{rgb}{0.000000,0.000000,0.000000}%
\pgfsetfillcolor{currentfill}%
\pgfsetlinewidth{1.003750pt}%
\definecolor{currentstroke}{rgb}{0.000000,0.000000,0.000000}%
\pgfsetstrokecolor{currentstroke}%
\pgfsetdash{}{0pt}%
\pgfpathmoveto{\pgfqpoint{0.480841in}{0.446310in}}%
\pgfpathcurveto{\pgfqpoint{0.491891in}{0.446310in}}{\pgfqpoint{0.502490in}{0.450700in}}{\pgfqpoint{0.510303in}{0.458514in}}%
\pgfpathcurveto{\pgfqpoint{0.518117in}{0.466328in}}{\pgfqpoint{0.522507in}{0.476927in}}{\pgfqpoint{0.522507in}{0.487977in}}%
\pgfpathcurveto{\pgfqpoint{0.522507in}{0.499027in}}{\pgfqpoint{0.518117in}{0.509626in}}{\pgfqpoint{0.510303in}{0.517440in}}%
\pgfpathcurveto{\pgfqpoint{0.502490in}{0.525253in}}{\pgfqpoint{0.491891in}{0.529644in}}{\pgfqpoint{0.480841in}{0.529644in}}%
\pgfpathcurveto{\pgfqpoint{0.469790in}{0.529644in}}{\pgfqpoint{0.459191in}{0.525253in}}{\pgfqpoint{0.451378in}{0.517440in}}%
\pgfpathcurveto{\pgfqpoint{0.443564in}{0.509626in}}{\pgfqpoint{0.439174in}{0.499027in}}{\pgfqpoint{0.439174in}{0.487977in}}%
\pgfpathcurveto{\pgfqpoint{0.439174in}{0.476927in}}{\pgfqpoint{0.443564in}{0.466328in}}{\pgfqpoint{0.451378in}{0.458514in}}%
\pgfpathcurveto{\pgfqpoint{0.459191in}{0.450700in}}{\pgfqpoint{0.469790in}{0.446310in}}{\pgfqpoint{0.480841in}{0.446310in}}%
\pgfpathclose%
\pgfusepath{stroke,fill}%
\end{pgfscope}%
\begin{pgfscope}%
\pgfpathrectangle{\pgfqpoint{0.375000in}{0.330000in}}{\pgfqpoint{2.325000in}{2.310000in}}%
\pgfusepath{clip}%
\pgfsetbuttcap%
\pgfsetroundjoin%
\definecolor{currentfill}{rgb}{0.000000,0.000000,0.000000}%
\pgfsetfillcolor{currentfill}%
\pgfsetlinewidth{1.003750pt}%
\definecolor{currentstroke}{rgb}{0.000000,0.000000,0.000000}%
\pgfsetstrokecolor{currentstroke}%
\pgfsetdash{}{0pt}%
\pgfpathmoveto{\pgfqpoint{0.480841in}{0.498341in}}%
\pgfpathcurveto{\pgfqpoint{0.491891in}{0.498341in}}{\pgfqpoint{0.502490in}{0.502731in}}{\pgfqpoint{0.510303in}{0.510545in}}%
\pgfpathcurveto{\pgfqpoint{0.518117in}{0.518359in}}{\pgfqpoint{0.522507in}{0.528958in}}{\pgfqpoint{0.522507in}{0.540008in}}%
\pgfpathcurveto{\pgfqpoint{0.522507in}{0.551058in}}{\pgfqpoint{0.518117in}{0.561657in}}{\pgfqpoint{0.510303in}{0.569470in}}%
\pgfpathcurveto{\pgfqpoint{0.502490in}{0.577284in}}{\pgfqpoint{0.491891in}{0.581674in}}{\pgfqpoint{0.480841in}{0.581674in}}%
\pgfpathcurveto{\pgfqpoint{0.469790in}{0.581674in}}{\pgfqpoint{0.459191in}{0.577284in}}{\pgfqpoint{0.451378in}{0.569470in}}%
\pgfpathcurveto{\pgfqpoint{0.443564in}{0.561657in}}{\pgfqpoint{0.439174in}{0.551058in}}{\pgfqpoint{0.439174in}{0.540008in}}%
\pgfpathcurveto{\pgfqpoint{0.439174in}{0.528958in}}{\pgfqpoint{0.443564in}{0.518359in}}{\pgfqpoint{0.451378in}{0.510545in}}%
\pgfpathcurveto{\pgfqpoint{0.459191in}{0.502731in}}{\pgfqpoint{0.469790in}{0.498341in}}{\pgfqpoint{0.480841in}{0.498341in}}%
\pgfpathclose%
\pgfusepath{stroke,fill}%
\end{pgfscope}%
\begin{pgfscope}%
\pgfpathrectangle{\pgfqpoint{0.375000in}{0.330000in}}{\pgfqpoint{2.325000in}{2.310000in}}%
\pgfusepath{clip}%
\pgfsetbuttcap%
\pgfsetroundjoin%
\definecolor{currentfill}{rgb}{0.000000,0.000000,0.000000}%
\pgfsetfillcolor{currentfill}%
\pgfsetlinewidth{1.003750pt}%
\definecolor{currentstroke}{rgb}{0.000000,0.000000,0.000000}%
\pgfsetstrokecolor{currentstroke}%
\pgfsetdash{}{0pt}%
\pgfpathmoveto{\pgfqpoint{0.480841in}{0.446310in}}%
\pgfpathcurveto{\pgfqpoint{0.491891in}{0.446310in}}{\pgfqpoint{0.502490in}{0.450700in}}{\pgfqpoint{0.510303in}{0.458514in}}%
\pgfpathcurveto{\pgfqpoint{0.518117in}{0.466328in}}{\pgfqpoint{0.522507in}{0.476927in}}{\pgfqpoint{0.522507in}{0.487977in}}%
\pgfpathcurveto{\pgfqpoint{0.522507in}{0.499027in}}{\pgfqpoint{0.518117in}{0.509626in}}{\pgfqpoint{0.510303in}{0.517440in}}%
\pgfpathcurveto{\pgfqpoint{0.502490in}{0.525253in}}{\pgfqpoint{0.491891in}{0.529644in}}{\pgfqpoint{0.480841in}{0.529644in}}%
\pgfpathcurveto{\pgfqpoint{0.469790in}{0.529644in}}{\pgfqpoint{0.459191in}{0.525253in}}{\pgfqpoint{0.451378in}{0.517440in}}%
\pgfpathcurveto{\pgfqpoint{0.443564in}{0.509626in}}{\pgfqpoint{0.439174in}{0.499027in}}{\pgfqpoint{0.439174in}{0.487977in}}%
\pgfpathcurveto{\pgfqpoint{0.439174in}{0.476927in}}{\pgfqpoint{0.443564in}{0.466328in}}{\pgfqpoint{0.451378in}{0.458514in}}%
\pgfpathcurveto{\pgfqpoint{0.459191in}{0.450700in}}{\pgfqpoint{0.469790in}{0.446310in}}{\pgfqpoint{0.480841in}{0.446310in}}%
\pgfpathclose%
\pgfusepath{stroke,fill}%
\end{pgfscope}%
\begin{pgfscope}%
\pgfpathrectangle{\pgfqpoint{0.375000in}{0.330000in}}{\pgfqpoint{2.325000in}{2.310000in}}%
\pgfusepath{clip}%
\pgfsetbuttcap%
\pgfsetroundjoin%
\definecolor{currentfill}{rgb}{0.000000,0.000000,0.000000}%
\pgfsetfillcolor{currentfill}%
\pgfsetlinewidth{1.003750pt}%
\definecolor{currentstroke}{rgb}{0.000000,0.000000,0.000000}%
\pgfsetstrokecolor{currentstroke}%
\pgfsetdash{}{0pt}%
\pgfpathmoveto{\pgfqpoint{0.480841in}{0.498341in}}%
\pgfpathcurveto{\pgfqpoint{0.491891in}{0.498341in}}{\pgfqpoint{0.502490in}{0.502731in}}{\pgfqpoint{0.510303in}{0.510545in}}%
\pgfpathcurveto{\pgfqpoint{0.518117in}{0.518359in}}{\pgfqpoint{0.522507in}{0.528958in}}{\pgfqpoint{0.522507in}{0.540008in}}%
\pgfpathcurveto{\pgfqpoint{0.522507in}{0.551058in}}{\pgfqpoint{0.518117in}{0.561657in}}{\pgfqpoint{0.510303in}{0.569470in}}%
\pgfpathcurveto{\pgfqpoint{0.502490in}{0.577284in}}{\pgfqpoint{0.491891in}{0.581674in}}{\pgfqpoint{0.480841in}{0.581674in}}%
\pgfpathcurveto{\pgfqpoint{0.469790in}{0.581674in}}{\pgfqpoint{0.459191in}{0.577284in}}{\pgfqpoint{0.451378in}{0.569470in}}%
\pgfpathcurveto{\pgfqpoint{0.443564in}{0.561657in}}{\pgfqpoint{0.439174in}{0.551058in}}{\pgfqpoint{0.439174in}{0.540008in}}%
\pgfpathcurveto{\pgfqpoint{0.439174in}{0.528958in}}{\pgfqpoint{0.443564in}{0.518359in}}{\pgfqpoint{0.451378in}{0.510545in}}%
\pgfpathcurveto{\pgfqpoint{0.459191in}{0.502731in}}{\pgfqpoint{0.469790in}{0.498341in}}{\pgfqpoint{0.480841in}{0.498341in}}%
\pgfpathclose%
\pgfusepath{stroke,fill}%
\end{pgfscope}%
\begin{pgfscope}%
\pgfpathrectangle{\pgfqpoint{0.375000in}{0.330000in}}{\pgfqpoint{2.325000in}{2.310000in}}%
\pgfusepath{clip}%
\pgfsetbuttcap%
\pgfsetroundjoin%
\definecolor{currentfill}{rgb}{0.000000,0.000000,0.000000}%
\pgfsetfillcolor{currentfill}%
\pgfsetlinewidth{1.003750pt}%
\definecolor{currentstroke}{rgb}{0.000000,0.000000,0.000000}%
\pgfsetstrokecolor{currentstroke}%
\pgfsetdash{}{0pt}%
\pgfpathmoveto{\pgfqpoint{0.480841in}{0.446310in}}%
\pgfpathcurveto{\pgfqpoint{0.491891in}{0.446310in}}{\pgfqpoint{0.502490in}{0.450700in}}{\pgfqpoint{0.510303in}{0.458514in}}%
\pgfpathcurveto{\pgfqpoint{0.518117in}{0.466328in}}{\pgfqpoint{0.522507in}{0.476927in}}{\pgfqpoint{0.522507in}{0.487977in}}%
\pgfpathcurveto{\pgfqpoint{0.522507in}{0.499027in}}{\pgfqpoint{0.518117in}{0.509626in}}{\pgfqpoint{0.510303in}{0.517440in}}%
\pgfpathcurveto{\pgfqpoint{0.502490in}{0.525253in}}{\pgfqpoint{0.491891in}{0.529644in}}{\pgfqpoint{0.480841in}{0.529644in}}%
\pgfpathcurveto{\pgfqpoint{0.469790in}{0.529644in}}{\pgfqpoint{0.459191in}{0.525253in}}{\pgfqpoint{0.451378in}{0.517440in}}%
\pgfpathcurveto{\pgfqpoint{0.443564in}{0.509626in}}{\pgfqpoint{0.439174in}{0.499027in}}{\pgfqpoint{0.439174in}{0.487977in}}%
\pgfpathcurveto{\pgfqpoint{0.439174in}{0.476927in}}{\pgfqpoint{0.443564in}{0.466328in}}{\pgfqpoint{0.451378in}{0.458514in}}%
\pgfpathcurveto{\pgfqpoint{0.459191in}{0.450700in}}{\pgfqpoint{0.469790in}{0.446310in}}{\pgfqpoint{0.480841in}{0.446310in}}%
\pgfpathclose%
\pgfusepath{stroke,fill}%
\end{pgfscope}%
\begin{pgfscope}%
\pgfpathrectangle{\pgfqpoint{0.375000in}{0.330000in}}{\pgfqpoint{2.325000in}{2.310000in}}%
\pgfusepath{clip}%
\pgfsetbuttcap%
\pgfsetroundjoin%
\definecolor{currentfill}{rgb}{0.000000,0.000000,0.000000}%
\pgfsetfillcolor{currentfill}%
\pgfsetlinewidth{1.003750pt}%
\definecolor{currentstroke}{rgb}{0.000000,0.000000,0.000000}%
\pgfsetstrokecolor{currentstroke}%
\pgfsetdash{}{0pt}%
\pgfpathmoveto{\pgfqpoint{0.480841in}{0.446310in}}%
\pgfpathcurveto{\pgfqpoint{0.491891in}{0.446310in}}{\pgfqpoint{0.502490in}{0.450700in}}{\pgfqpoint{0.510303in}{0.458514in}}%
\pgfpathcurveto{\pgfqpoint{0.518117in}{0.466328in}}{\pgfqpoint{0.522507in}{0.476927in}}{\pgfqpoint{0.522507in}{0.487977in}}%
\pgfpathcurveto{\pgfqpoint{0.522507in}{0.499027in}}{\pgfqpoint{0.518117in}{0.509626in}}{\pgfqpoint{0.510303in}{0.517440in}}%
\pgfpathcurveto{\pgfqpoint{0.502490in}{0.525253in}}{\pgfqpoint{0.491891in}{0.529644in}}{\pgfqpoint{0.480841in}{0.529644in}}%
\pgfpathcurveto{\pgfqpoint{0.469790in}{0.529644in}}{\pgfqpoint{0.459191in}{0.525253in}}{\pgfqpoint{0.451378in}{0.517440in}}%
\pgfpathcurveto{\pgfqpoint{0.443564in}{0.509626in}}{\pgfqpoint{0.439174in}{0.499027in}}{\pgfqpoint{0.439174in}{0.487977in}}%
\pgfpathcurveto{\pgfqpoint{0.439174in}{0.476927in}}{\pgfqpoint{0.443564in}{0.466328in}}{\pgfqpoint{0.451378in}{0.458514in}}%
\pgfpathcurveto{\pgfqpoint{0.459191in}{0.450700in}}{\pgfqpoint{0.469790in}{0.446310in}}{\pgfqpoint{0.480841in}{0.446310in}}%
\pgfpathclose%
\pgfusepath{stroke,fill}%
\end{pgfscope}%
\begin{pgfscope}%
\pgfpathrectangle{\pgfqpoint{0.375000in}{0.330000in}}{\pgfqpoint{2.325000in}{2.310000in}}%
\pgfusepath{clip}%
\pgfsetbuttcap%
\pgfsetroundjoin%
\definecolor{currentfill}{rgb}{0.000000,0.000000,0.000000}%
\pgfsetfillcolor{currentfill}%
\pgfsetlinewidth{1.003750pt}%
\definecolor{currentstroke}{rgb}{0.000000,0.000000,0.000000}%
\pgfsetstrokecolor{currentstroke}%
\pgfsetdash{}{0pt}%
\pgfpathmoveto{\pgfqpoint{0.480841in}{0.446310in}}%
\pgfpathcurveto{\pgfqpoint{0.491891in}{0.446310in}}{\pgfqpoint{0.502490in}{0.450700in}}{\pgfqpoint{0.510303in}{0.458514in}}%
\pgfpathcurveto{\pgfqpoint{0.518117in}{0.466328in}}{\pgfqpoint{0.522507in}{0.476927in}}{\pgfqpoint{0.522507in}{0.487977in}}%
\pgfpathcurveto{\pgfqpoint{0.522507in}{0.499027in}}{\pgfqpoint{0.518117in}{0.509626in}}{\pgfqpoint{0.510303in}{0.517440in}}%
\pgfpathcurveto{\pgfqpoint{0.502490in}{0.525253in}}{\pgfqpoint{0.491891in}{0.529644in}}{\pgfqpoint{0.480841in}{0.529644in}}%
\pgfpathcurveto{\pgfqpoint{0.469790in}{0.529644in}}{\pgfqpoint{0.459191in}{0.525253in}}{\pgfqpoint{0.451378in}{0.517440in}}%
\pgfpathcurveto{\pgfqpoint{0.443564in}{0.509626in}}{\pgfqpoint{0.439174in}{0.499027in}}{\pgfqpoint{0.439174in}{0.487977in}}%
\pgfpathcurveto{\pgfqpoint{0.439174in}{0.476927in}}{\pgfqpoint{0.443564in}{0.466328in}}{\pgfqpoint{0.451378in}{0.458514in}}%
\pgfpathcurveto{\pgfqpoint{0.459191in}{0.450700in}}{\pgfqpoint{0.469790in}{0.446310in}}{\pgfqpoint{0.480841in}{0.446310in}}%
\pgfpathclose%
\pgfusepath{stroke,fill}%
\end{pgfscope}%
\begin{pgfscope}%
\pgfpathrectangle{\pgfqpoint{0.375000in}{0.330000in}}{\pgfqpoint{2.325000in}{2.310000in}}%
\pgfusepath{clip}%
\pgfsetbuttcap%
\pgfsetroundjoin%
\definecolor{currentfill}{rgb}{0.000000,0.000000,0.000000}%
\pgfsetfillcolor{currentfill}%
\pgfsetlinewidth{1.003750pt}%
\definecolor{currentstroke}{rgb}{0.000000,0.000000,0.000000}%
\pgfsetstrokecolor{currentstroke}%
\pgfsetdash{}{0pt}%
\pgfpathmoveto{\pgfqpoint{0.480841in}{0.498341in}}%
\pgfpathcurveto{\pgfqpoint{0.491891in}{0.498341in}}{\pgfqpoint{0.502490in}{0.502731in}}{\pgfqpoint{0.510303in}{0.510545in}}%
\pgfpathcurveto{\pgfqpoint{0.518117in}{0.518359in}}{\pgfqpoint{0.522507in}{0.528958in}}{\pgfqpoint{0.522507in}{0.540008in}}%
\pgfpathcurveto{\pgfqpoint{0.522507in}{0.551058in}}{\pgfqpoint{0.518117in}{0.561657in}}{\pgfqpoint{0.510303in}{0.569470in}}%
\pgfpathcurveto{\pgfqpoint{0.502490in}{0.577284in}}{\pgfqpoint{0.491891in}{0.581674in}}{\pgfqpoint{0.480841in}{0.581674in}}%
\pgfpathcurveto{\pgfqpoint{0.469790in}{0.581674in}}{\pgfqpoint{0.459191in}{0.577284in}}{\pgfqpoint{0.451378in}{0.569470in}}%
\pgfpathcurveto{\pgfqpoint{0.443564in}{0.561657in}}{\pgfqpoint{0.439174in}{0.551058in}}{\pgfqpoint{0.439174in}{0.540008in}}%
\pgfpathcurveto{\pgfqpoint{0.439174in}{0.528958in}}{\pgfqpoint{0.443564in}{0.518359in}}{\pgfqpoint{0.451378in}{0.510545in}}%
\pgfpathcurveto{\pgfqpoint{0.459191in}{0.502731in}}{\pgfqpoint{0.469790in}{0.498341in}}{\pgfqpoint{0.480841in}{0.498341in}}%
\pgfpathclose%
\pgfusepath{stroke,fill}%
\end{pgfscope}%
\begin{pgfscope}%
\pgfpathrectangle{\pgfqpoint{0.375000in}{0.330000in}}{\pgfqpoint{2.325000in}{2.310000in}}%
\pgfusepath{clip}%
\pgfsetbuttcap%
\pgfsetroundjoin%
\definecolor{currentfill}{rgb}{0.000000,0.000000,0.000000}%
\pgfsetfillcolor{currentfill}%
\pgfsetlinewidth{1.003750pt}%
\definecolor{currentstroke}{rgb}{0.000000,0.000000,0.000000}%
\pgfsetstrokecolor{currentstroke}%
\pgfsetdash{}{0pt}%
\pgfpathmoveto{\pgfqpoint{0.480841in}{0.446310in}}%
\pgfpathcurveto{\pgfqpoint{0.491891in}{0.446310in}}{\pgfqpoint{0.502490in}{0.450700in}}{\pgfqpoint{0.510303in}{0.458514in}}%
\pgfpathcurveto{\pgfqpoint{0.518117in}{0.466328in}}{\pgfqpoint{0.522507in}{0.476927in}}{\pgfqpoint{0.522507in}{0.487977in}}%
\pgfpathcurveto{\pgfqpoint{0.522507in}{0.499027in}}{\pgfqpoint{0.518117in}{0.509626in}}{\pgfqpoint{0.510303in}{0.517440in}}%
\pgfpathcurveto{\pgfqpoint{0.502490in}{0.525253in}}{\pgfqpoint{0.491891in}{0.529644in}}{\pgfqpoint{0.480841in}{0.529644in}}%
\pgfpathcurveto{\pgfqpoint{0.469790in}{0.529644in}}{\pgfqpoint{0.459191in}{0.525253in}}{\pgfqpoint{0.451378in}{0.517440in}}%
\pgfpathcurveto{\pgfqpoint{0.443564in}{0.509626in}}{\pgfqpoint{0.439174in}{0.499027in}}{\pgfqpoint{0.439174in}{0.487977in}}%
\pgfpathcurveto{\pgfqpoint{0.439174in}{0.476927in}}{\pgfqpoint{0.443564in}{0.466328in}}{\pgfqpoint{0.451378in}{0.458514in}}%
\pgfpathcurveto{\pgfqpoint{0.459191in}{0.450700in}}{\pgfqpoint{0.469790in}{0.446310in}}{\pgfqpoint{0.480841in}{0.446310in}}%
\pgfpathclose%
\pgfusepath{stroke,fill}%
\end{pgfscope}%
\begin{pgfscope}%
\pgfpathrectangle{\pgfqpoint{0.375000in}{0.330000in}}{\pgfqpoint{2.325000in}{2.310000in}}%
\pgfusepath{clip}%
\pgfsetbuttcap%
\pgfsetroundjoin%
\definecolor{currentfill}{rgb}{0.000000,0.000000,0.000000}%
\pgfsetfillcolor{currentfill}%
\pgfsetlinewidth{1.003750pt}%
\definecolor{currentstroke}{rgb}{0.000000,0.000000,0.000000}%
\pgfsetstrokecolor{currentstroke}%
\pgfsetdash{}{0pt}%
\pgfpathmoveto{\pgfqpoint{0.480841in}{0.446310in}}%
\pgfpathcurveto{\pgfqpoint{0.491891in}{0.446310in}}{\pgfqpoint{0.502490in}{0.450700in}}{\pgfqpoint{0.510303in}{0.458514in}}%
\pgfpathcurveto{\pgfqpoint{0.518117in}{0.466328in}}{\pgfqpoint{0.522507in}{0.476927in}}{\pgfqpoint{0.522507in}{0.487977in}}%
\pgfpathcurveto{\pgfqpoint{0.522507in}{0.499027in}}{\pgfqpoint{0.518117in}{0.509626in}}{\pgfqpoint{0.510303in}{0.517440in}}%
\pgfpathcurveto{\pgfqpoint{0.502490in}{0.525253in}}{\pgfqpoint{0.491891in}{0.529644in}}{\pgfqpoint{0.480841in}{0.529644in}}%
\pgfpathcurveto{\pgfqpoint{0.469790in}{0.529644in}}{\pgfqpoint{0.459191in}{0.525253in}}{\pgfqpoint{0.451378in}{0.517440in}}%
\pgfpathcurveto{\pgfqpoint{0.443564in}{0.509626in}}{\pgfqpoint{0.439174in}{0.499027in}}{\pgfqpoint{0.439174in}{0.487977in}}%
\pgfpathcurveto{\pgfqpoint{0.439174in}{0.476927in}}{\pgfqpoint{0.443564in}{0.466328in}}{\pgfqpoint{0.451378in}{0.458514in}}%
\pgfpathcurveto{\pgfqpoint{0.459191in}{0.450700in}}{\pgfqpoint{0.469790in}{0.446310in}}{\pgfqpoint{0.480841in}{0.446310in}}%
\pgfpathclose%
\pgfusepath{stroke,fill}%
\end{pgfscope}%
\begin{pgfscope}%
\pgfpathrectangle{\pgfqpoint{0.375000in}{0.330000in}}{\pgfqpoint{2.325000in}{2.310000in}}%
\pgfusepath{clip}%
\pgfsetbuttcap%
\pgfsetroundjoin%
\definecolor{currentfill}{rgb}{0.000000,0.000000,0.000000}%
\pgfsetfillcolor{currentfill}%
\pgfsetlinewidth{1.003750pt}%
\definecolor{currentstroke}{rgb}{0.000000,0.000000,0.000000}%
\pgfsetstrokecolor{currentstroke}%
\pgfsetdash{}{0pt}%
\pgfpathmoveto{\pgfqpoint{0.480841in}{0.446310in}}%
\pgfpathcurveto{\pgfqpoint{0.491891in}{0.446310in}}{\pgfqpoint{0.502490in}{0.450700in}}{\pgfqpoint{0.510303in}{0.458514in}}%
\pgfpathcurveto{\pgfqpoint{0.518117in}{0.466328in}}{\pgfqpoint{0.522507in}{0.476927in}}{\pgfqpoint{0.522507in}{0.487977in}}%
\pgfpathcurveto{\pgfqpoint{0.522507in}{0.499027in}}{\pgfqpoint{0.518117in}{0.509626in}}{\pgfqpoint{0.510303in}{0.517440in}}%
\pgfpathcurveto{\pgfqpoint{0.502490in}{0.525253in}}{\pgfqpoint{0.491891in}{0.529644in}}{\pgfqpoint{0.480841in}{0.529644in}}%
\pgfpathcurveto{\pgfqpoint{0.469790in}{0.529644in}}{\pgfqpoint{0.459191in}{0.525253in}}{\pgfqpoint{0.451378in}{0.517440in}}%
\pgfpathcurveto{\pgfqpoint{0.443564in}{0.509626in}}{\pgfqpoint{0.439174in}{0.499027in}}{\pgfqpoint{0.439174in}{0.487977in}}%
\pgfpathcurveto{\pgfqpoint{0.439174in}{0.476927in}}{\pgfqpoint{0.443564in}{0.466328in}}{\pgfqpoint{0.451378in}{0.458514in}}%
\pgfpathcurveto{\pgfqpoint{0.459191in}{0.450700in}}{\pgfqpoint{0.469790in}{0.446310in}}{\pgfqpoint{0.480841in}{0.446310in}}%
\pgfpathclose%
\pgfusepath{stroke,fill}%
\end{pgfscope}%
\begin{pgfscope}%
\pgfpathrectangle{\pgfqpoint{0.375000in}{0.330000in}}{\pgfqpoint{2.325000in}{2.310000in}}%
\pgfusepath{clip}%
\pgfsetbuttcap%
\pgfsetroundjoin%
\definecolor{currentfill}{rgb}{0.000000,0.000000,0.000000}%
\pgfsetfillcolor{currentfill}%
\pgfsetlinewidth{1.003750pt}%
\definecolor{currentstroke}{rgb}{0.000000,0.000000,0.000000}%
\pgfsetstrokecolor{currentstroke}%
\pgfsetdash{}{0pt}%
\pgfpathmoveto{\pgfqpoint{0.480841in}{0.394279in}}%
\pgfpathcurveto{\pgfqpoint{0.491891in}{0.394279in}}{\pgfqpoint{0.502490in}{0.398670in}}{\pgfqpoint{0.510303in}{0.406483in}}%
\pgfpathcurveto{\pgfqpoint{0.518117in}{0.414297in}}{\pgfqpoint{0.522507in}{0.424896in}}{\pgfqpoint{0.522507in}{0.435946in}}%
\pgfpathcurveto{\pgfqpoint{0.522507in}{0.446996in}}{\pgfqpoint{0.518117in}{0.457595in}}{\pgfqpoint{0.510303in}{0.465409in}}%
\pgfpathcurveto{\pgfqpoint{0.502490in}{0.473222in}}{\pgfqpoint{0.491891in}{0.477613in}}{\pgfqpoint{0.480841in}{0.477613in}}%
\pgfpathcurveto{\pgfqpoint{0.469790in}{0.477613in}}{\pgfqpoint{0.459191in}{0.473222in}}{\pgfqpoint{0.451378in}{0.465409in}}%
\pgfpathcurveto{\pgfqpoint{0.443564in}{0.457595in}}{\pgfqpoint{0.439174in}{0.446996in}}{\pgfqpoint{0.439174in}{0.435946in}}%
\pgfpathcurveto{\pgfqpoint{0.439174in}{0.424896in}}{\pgfqpoint{0.443564in}{0.414297in}}{\pgfqpoint{0.451378in}{0.406483in}}%
\pgfpathcurveto{\pgfqpoint{0.459191in}{0.398670in}}{\pgfqpoint{0.469790in}{0.394279in}}{\pgfqpoint{0.480841in}{0.394279in}}%
\pgfpathclose%
\pgfusepath{stroke,fill}%
\end{pgfscope}%
\begin{pgfscope}%
\pgfpathrectangle{\pgfqpoint{0.375000in}{0.330000in}}{\pgfqpoint{2.325000in}{2.310000in}}%
\pgfusepath{clip}%
\pgfsetbuttcap%
\pgfsetroundjoin%
\definecolor{currentfill}{rgb}{0.000000,0.000000,0.000000}%
\pgfsetfillcolor{currentfill}%
\pgfsetlinewidth{1.003750pt}%
\definecolor{currentstroke}{rgb}{0.000000,0.000000,0.000000}%
\pgfsetstrokecolor{currentstroke}%
\pgfsetdash{}{0pt}%
\pgfpathmoveto{\pgfqpoint{0.480841in}{0.446310in}}%
\pgfpathcurveto{\pgfqpoint{0.491891in}{0.446310in}}{\pgfqpoint{0.502490in}{0.450700in}}{\pgfqpoint{0.510303in}{0.458514in}}%
\pgfpathcurveto{\pgfqpoint{0.518117in}{0.466328in}}{\pgfqpoint{0.522507in}{0.476927in}}{\pgfqpoint{0.522507in}{0.487977in}}%
\pgfpathcurveto{\pgfqpoint{0.522507in}{0.499027in}}{\pgfqpoint{0.518117in}{0.509626in}}{\pgfqpoint{0.510303in}{0.517440in}}%
\pgfpathcurveto{\pgfqpoint{0.502490in}{0.525253in}}{\pgfqpoint{0.491891in}{0.529644in}}{\pgfqpoint{0.480841in}{0.529644in}}%
\pgfpathcurveto{\pgfqpoint{0.469790in}{0.529644in}}{\pgfqpoint{0.459191in}{0.525253in}}{\pgfqpoint{0.451378in}{0.517440in}}%
\pgfpathcurveto{\pgfqpoint{0.443564in}{0.509626in}}{\pgfqpoint{0.439174in}{0.499027in}}{\pgfqpoint{0.439174in}{0.487977in}}%
\pgfpathcurveto{\pgfqpoint{0.439174in}{0.476927in}}{\pgfqpoint{0.443564in}{0.466328in}}{\pgfqpoint{0.451378in}{0.458514in}}%
\pgfpathcurveto{\pgfqpoint{0.459191in}{0.450700in}}{\pgfqpoint{0.469790in}{0.446310in}}{\pgfqpoint{0.480841in}{0.446310in}}%
\pgfpathclose%
\pgfusepath{stroke,fill}%
\end{pgfscope}%
\begin{pgfscope}%
\pgfpathrectangle{\pgfqpoint{0.375000in}{0.330000in}}{\pgfqpoint{2.325000in}{2.310000in}}%
\pgfusepath{clip}%
\pgfsetbuttcap%
\pgfsetroundjoin%
\definecolor{currentfill}{rgb}{0.000000,0.000000,0.000000}%
\pgfsetfillcolor{currentfill}%
\pgfsetlinewidth{1.003750pt}%
\definecolor{currentstroke}{rgb}{0.000000,0.000000,0.000000}%
\pgfsetstrokecolor{currentstroke}%
\pgfsetdash{}{0pt}%
\pgfpathmoveto{\pgfqpoint{0.480841in}{0.446310in}}%
\pgfpathcurveto{\pgfqpoint{0.491891in}{0.446310in}}{\pgfqpoint{0.502490in}{0.450700in}}{\pgfqpoint{0.510303in}{0.458514in}}%
\pgfpathcurveto{\pgfqpoint{0.518117in}{0.466328in}}{\pgfqpoint{0.522507in}{0.476927in}}{\pgfqpoint{0.522507in}{0.487977in}}%
\pgfpathcurveto{\pgfqpoint{0.522507in}{0.499027in}}{\pgfqpoint{0.518117in}{0.509626in}}{\pgfqpoint{0.510303in}{0.517440in}}%
\pgfpathcurveto{\pgfqpoint{0.502490in}{0.525253in}}{\pgfqpoint{0.491891in}{0.529644in}}{\pgfqpoint{0.480841in}{0.529644in}}%
\pgfpathcurveto{\pgfqpoint{0.469790in}{0.529644in}}{\pgfqpoint{0.459191in}{0.525253in}}{\pgfqpoint{0.451378in}{0.517440in}}%
\pgfpathcurveto{\pgfqpoint{0.443564in}{0.509626in}}{\pgfqpoint{0.439174in}{0.499027in}}{\pgfqpoint{0.439174in}{0.487977in}}%
\pgfpathcurveto{\pgfqpoint{0.439174in}{0.476927in}}{\pgfqpoint{0.443564in}{0.466328in}}{\pgfqpoint{0.451378in}{0.458514in}}%
\pgfpathcurveto{\pgfqpoint{0.459191in}{0.450700in}}{\pgfqpoint{0.469790in}{0.446310in}}{\pgfqpoint{0.480841in}{0.446310in}}%
\pgfpathclose%
\pgfusepath{stroke,fill}%
\end{pgfscope}%
\begin{pgfscope}%
\pgfpathrectangle{\pgfqpoint{0.375000in}{0.330000in}}{\pgfqpoint{2.325000in}{2.310000in}}%
\pgfusepath{clip}%
\pgfsetbuttcap%
\pgfsetroundjoin%
\definecolor{currentfill}{rgb}{0.000000,0.000000,0.000000}%
\pgfsetfillcolor{currentfill}%
\pgfsetlinewidth{1.003750pt}%
\definecolor{currentstroke}{rgb}{0.000000,0.000000,0.000000}%
\pgfsetstrokecolor{currentstroke}%
\pgfsetdash{}{0pt}%
\pgfpathmoveto{\pgfqpoint{0.480841in}{0.446310in}}%
\pgfpathcurveto{\pgfqpoint{0.491891in}{0.446310in}}{\pgfqpoint{0.502490in}{0.450700in}}{\pgfqpoint{0.510303in}{0.458514in}}%
\pgfpathcurveto{\pgfqpoint{0.518117in}{0.466328in}}{\pgfqpoint{0.522507in}{0.476927in}}{\pgfqpoint{0.522507in}{0.487977in}}%
\pgfpathcurveto{\pgfqpoint{0.522507in}{0.499027in}}{\pgfqpoint{0.518117in}{0.509626in}}{\pgfqpoint{0.510303in}{0.517440in}}%
\pgfpathcurveto{\pgfqpoint{0.502490in}{0.525253in}}{\pgfqpoint{0.491891in}{0.529644in}}{\pgfqpoint{0.480841in}{0.529644in}}%
\pgfpathcurveto{\pgfqpoint{0.469790in}{0.529644in}}{\pgfqpoint{0.459191in}{0.525253in}}{\pgfqpoint{0.451378in}{0.517440in}}%
\pgfpathcurveto{\pgfqpoint{0.443564in}{0.509626in}}{\pgfqpoint{0.439174in}{0.499027in}}{\pgfqpoint{0.439174in}{0.487977in}}%
\pgfpathcurveto{\pgfqpoint{0.439174in}{0.476927in}}{\pgfqpoint{0.443564in}{0.466328in}}{\pgfqpoint{0.451378in}{0.458514in}}%
\pgfpathcurveto{\pgfqpoint{0.459191in}{0.450700in}}{\pgfqpoint{0.469790in}{0.446310in}}{\pgfqpoint{0.480841in}{0.446310in}}%
\pgfpathclose%
\pgfusepath{stroke,fill}%
\end{pgfscope}%
\begin{pgfscope}%
\pgfpathrectangle{\pgfqpoint{0.375000in}{0.330000in}}{\pgfqpoint{2.325000in}{2.310000in}}%
\pgfusepath{clip}%
\pgfsetbuttcap%
\pgfsetroundjoin%
\definecolor{currentfill}{rgb}{0.000000,0.000000,0.000000}%
\pgfsetfillcolor{currentfill}%
\pgfsetlinewidth{1.003750pt}%
\definecolor{currentstroke}{rgb}{0.000000,0.000000,0.000000}%
\pgfsetstrokecolor{currentstroke}%
\pgfsetdash{}{0pt}%
\pgfpathmoveto{\pgfqpoint{0.480841in}{0.498341in}}%
\pgfpathcurveto{\pgfqpoint{0.491891in}{0.498341in}}{\pgfqpoint{0.502490in}{0.502731in}}{\pgfqpoint{0.510303in}{0.510545in}}%
\pgfpathcurveto{\pgfqpoint{0.518117in}{0.518359in}}{\pgfqpoint{0.522507in}{0.528958in}}{\pgfqpoint{0.522507in}{0.540008in}}%
\pgfpathcurveto{\pgfqpoint{0.522507in}{0.551058in}}{\pgfqpoint{0.518117in}{0.561657in}}{\pgfqpoint{0.510303in}{0.569470in}}%
\pgfpathcurveto{\pgfqpoint{0.502490in}{0.577284in}}{\pgfqpoint{0.491891in}{0.581674in}}{\pgfqpoint{0.480841in}{0.581674in}}%
\pgfpathcurveto{\pgfqpoint{0.469790in}{0.581674in}}{\pgfqpoint{0.459191in}{0.577284in}}{\pgfqpoint{0.451378in}{0.569470in}}%
\pgfpathcurveto{\pgfqpoint{0.443564in}{0.561657in}}{\pgfqpoint{0.439174in}{0.551058in}}{\pgfqpoint{0.439174in}{0.540008in}}%
\pgfpathcurveto{\pgfqpoint{0.439174in}{0.528958in}}{\pgfqpoint{0.443564in}{0.518359in}}{\pgfqpoint{0.451378in}{0.510545in}}%
\pgfpathcurveto{\pgfqpoint{0.459191in}{0.502731in}}{\pgfqpoint{0.469790in}{0.498341in}}{\pgfqpoint{0.480841in}{0.498341in}}%
\pgfpathclose%
\pgfusepath{stroke,fill}%
\end{pgfscope}%
\begin{pgfscope}%
\pgfpathrectangle{\pgfqpoint{0.375000in}{0.330000in}}{\pgfqpoint{2.325000in}{2.310000in}}%
\pgfusepath{clip}%
\pgfsetbuttcap%
\pgfsetroundjoin%
\definecolor{currentfill}{rgb}{0.000000,0.000000,0.000000}%
\pgfsetfillcolor{currentfill}%
\pgfsetlinewidth{1.003750pt}%
\definecolor{currentstroke}{rgb}{0.000000,0.000000,0.000000}%
\pgfsetstrokecolor{currentstroke}%
\pgfsetdash{}{0pt}%
\pgfpathmoveto{\pgfqpoint{0.480841in}{0.446310in}}%
\pgfpathcurveto{\pgfqpoint{0.491891in}{0.446310in}}{\pgfqpoint{0.502490in}{0.450700in}}{\pgfqpoint{0.510303in}{0.458514in}}%
\pgfpathcurveto{\pgfqpoint{0.518117in}{0.466328in}}{\pgfqpoint{0.522507in}{0.476927in}}{\pgfqpoint{0.522507in}{0.487977in}}%
\pgfpathcurveto{\pgfqpoint{0.522507in}{0.499027in}}{\pgfqpoint{0.518117in}{0.509626in}}{\pgfqpoint{0.510303in}{0.517440in}}%
\pgfpathcurveto{\pgfqpoint{0.502490in}{0.525253in}}{\pgfqpoint{0.491891in}{0.529644in}}{\pgfqpoint{0.480841in}{0.529644in}}%
\pgfpathcurveto{\pgfqpoint{0.469790in}{0.529644in}}{\pgfqpoint{0.459191in}{0.525253in}}{\pgfqpoint{0.451378in}{0.517440in}}%
\pgfpathcurveto{\pgfqpoint{0.443564in}{0.509626in}}{\pgfqpoint{0.439174in}{0.499027in}}{\pgfqpoint{0.439174in}{0.487977in}}%
\pgfpathcurveto{\pgfqpoint{0.439174in}{0.476927in}}{\pgfqpoint{0.443564in}{0.466328in}}{\pgfqpoint{0.451378in}{0.458514in}}%
\pgfpathcurveto{\pgfqpoint{0.459191in}{0.450700in}}{\pgfqpoint{0.469790in}{0.446310in}}{\pgfqpoint{0.480841in}{0.446310in}}%
\pgfpathclose%
\pgfusepath{stroke,fill}%
\end{pgfscope}%
\begin{pgfscope}%
\pgfpathrectangle{\pgfqpoint{0.375000in}{0.330000in}}{\pgfqpoint{2.325000in}{2.310000in}}%
\pgfusepath{clip}%
\pgfsetbuttcap%
\pgfsetroundjoin%
\definecolor{currentfill}{rgb}{0.000000,0.000000,0.000000}%
\pgfsetfillcolor{currentfill}%
\pgfsetlinewidth{1.003750pt}%
\definecolor{currentstroke}{rgb}{0.000000,0.000000,0.000000}%
\pgfsetstrokecolor{currentstroke}%
\pgfsetdash{}{0pt}%
\pgfpathmoveto{\pgfqpoint{0.480841in}{0.446310in}}%
\pgfpathcurveto{\pgfqpoint{0.491891in}{0.446310in}}{\pgfqpoint{0.502490in}{0.450700in}}{\pgfqpoint{0.510303in}{0.458514in}}%
\pgfpathcurveto{\pgfqpoint{0.518117in}{0.466328in}}{\pgfqpoint{0.522507in}{0.476927in}}{\pgfqpoint{0.522507in}{0.487977in}}%
\pgfpathcurveto{\pgfqpoint{0.522507in}{0.499027in}}{\pgfqpoint{0.518117in}{0.509626in}}{\pgfqpoint{0.510303in}{0.517440in}}%
\pgfpathcurveto{\pgfqpoint{0.502490in}{0.525253in}}{\pgfqpoint{0.491891in}{0.529644in}}{\pgfqpoint{0.480841in}{0.529644in}}%
\pgfpathcurveto{\pgfqpoint{0.469790in}{0.529644in}}{\pgfqpoint{0.459191in}{0.525253in}}{\pgfqpoint{0.451378in}{0.517440in}}%
\pgfpathcurveto{\pgfqpoint{0.443564in}{0.509626in}}{\pgfqpoint{0.439174in}{0.499027in}}{\pgfqpoint{0.439174in}{0.487977in}}%
\pgfpathcurveto{\pgfqpoint{0.439174in}{0.476927in}}{\pgfqpoint{0.443564in}{0.466328in}}{\pgfqpoint{0.451378in}{0.458514in}}%
\pgfpathcurveto{\pgfqpoint{0.459191in}{0.450700in}}{\pgfqpoint{0.469790in}{0.446310in}}{\pgfqpoint{0.480841in}{0.446310in}}%
\pgfpathclose%
\pgfusepath{stroke,fill}%
\end{pgfscope}%
\begin{pgfscope}%
\pgfpathrectangle{\pgfqpoint{0.375000in}{0.330000in}}{\pgfqpoint{2.325000in}{2.310000in}}%
\pgfusepath{clip}%
\pgfsetbuttcap%
\pgfsetroundjoin%
\definecolor{currentfill}{rgb}{0.000000,0.000000,0.000000}%
\pgfsetfillcolor{currentfill}%
\pgfsetlinewidth{1.003750pt}%
\definecolor{currentstroke}{rgb}{0.000000,0.000000,0.000000}%
\pgfsetstrokecolor{currentstroke}%
\pgfsetdash{}{0pt}%
\pgfpathmoveto{\pgfqpoint{0.480841in}{0.446310in}}%
\pgfpathcurveto{\pgfqpoint{0.491891in}{0.446310in}}{\pgfqpoint{0.502490in}{0.450700in}}{\pgfqpoint{0.510303in}{0.458514in}}%
\pgfpathcurveto{\pgfqpoint{0.518117in}{0.466328in}}{\pgfqpoint{0.522507in}{0.476927in}}{\pgfqpoint{0.522507in}{0.487977in}}%
\pgfpathcurveto{\pgfqpoint{0.522507in}{0.499027in}}{\pgfqpoint{0.518117in}{0.509626in}}{\pgfqpoint{0.510303in}{0.517440in}}%
\pgfpathcurveto{\pgfqpoint{0.502490in}{0.525253in}}{\pgfqpoint{0.491891in}{0.529644in}}{\pgfqpoint{0.480841in}{0.529644in}}%
\pgfpathcurveto{\pgfqpoint{0.469790in}{0.529644in}}{\pgfqpoint{0.459191in}{0.525253in}}{\pgfqpoint{0.451378in}{0.517440in}}%
\pgfpathcurveto{\pgfqpoint{0.443564in}{0.509626in}}{\pgfqpoint{0.439174in}{0.499027in}}{\pgfqpoint{0.439174in}{0.487977in}}%
\pgfpathcurveto{\pgfqpoint{0.439174in}{0.476927in}}{\pgfqpoint{0.443564in}{0.466328in}}{\pgfqpoint{0.451378in}{0.458514in}}%
\pgfpathcurveto{\pgfqpoint{0.459191in}{0.450700in}}{\pgfqpoint{0.469790in}{0.446310in}}{\pgfqpoint{0.480841in}{0.446310in}}%
\pgfpathclose%
\pgfusepath{stroke,fill}%
\end{pgfscope}%
\begin{pgfscope}%
\pgfpathrectangle{\pgfqpoint{0.375000in}{0.330000in}}{\pgfqpoint{2.325000in}{2.310000in}}%
\pgfusepath{clip}%
\pgfsetbuttcap%
\pgfsetroundjoin%
\definecolor{currentfill}{rgb}{0.000000,0.000000,0.000000}%
\pgfsetfillcolor{currentfill}%
\pgfsetlinewidth{1.003750pt}%
\definecolor{currentstroke}{rgb}{0.000000,0.000000,0.000000}%
\pgfsetstrokecolor{currentstroke}%
\pgfsetdash{}{0pt}%
\pgfpathmoveto{\pgfqpoint{0.480841in}{0.446310in}}%
\pgfpathcurveto{\pgfqpoint{0.491891in}{0.446310in}}{\pgfqpoint{0.502490in}{0.450700in}}{\pgfqpoint{0.510303in}{0.458514in}}%
\pgfpathcurveto{\pgfqpoint{0.518117in}{0.466328in}}{\pgfqpoint{0.522507in}{0.476927in}}{\pgfqpoint{0.522507in}{0.487977in}}%
\pgfpathcurveto{\pgfqpoint{0.522507in}{0.499027in}}{\pgfqpoint{0.518117in}{0.509626in}}{\pgfqpoint{0.510303in}{0.517440in}}%
\pgfpathcurveto{\pgfqpoint{0.502490in}{0.525253in}}{\pgfqpoint{0.491891in}{0.529644in}}{\pgfqpoint{0.480841in}{0.529644in}}%
\pgfpathcurveto{\pgfqpoint{0.469790in}{0.529644in}}{\pgfqpoint{0.459191in}{0.525253in}}{\pgfqpoint{0.451378in}{0.517440in}}%
\pgfpathcurveto{\pgfqpoint{0.443564in}{0.509626in}}{\pgfqpoint{0.439174in}{0.499027in}}{\pgfqpoint{0.439174in}{0.487977in}}%
\pgfpathcurveto{\pgfqpoint{0.439174in}{0.476927in}}{\pgfqpoint{0.443564in}{0.466328in}}{\pgfqpoint{0.451378in}{0.458514in}}%
\pgfpathcurveto{\pgfqpoint{0.459191in}{0.450700in}}{\pgfqpoint{0.469790in}{0.446310in}}{\pgfqpoint{0.480841in}{0.446310in}}%
\pgfpathclose%
\pgfusepath{stroke,fill}%
\end{pgfscope}%
\begin{pgfscope}%
\pgfpathrectangle{\pgfqpoint{0.375000in}{0.330000in}}{\pgfqpoint{2.325000in}{2.310000in}}%
\pgfusepath{clip}%
\pgfsetbuttcap%
\pgfsetroundjoin%
\definecolor{currentfill}{rgb}{0.000000,0.000000,0.000000}%
\pgfsetfillcolor{currentfill}%
\pgfsetlinewidth{1.003750pt}%
\definecolor{currentstroke}{rgb}{0.000000,0.000000,0.000000}%
\pgfsetstrokecolor{currentstroke}%
\pgfsetdash{}{0pt}%
\pgfpathmoveto{\pgfqpoint{0.480841in}{0.446310in}}%
\pgfpathcurveto{\pgfqpoint{0.491891in}{0.446310in}}{\pgfqpoint{0.502490in}{0.450700in}}{\pgfqpoint{0.510303in}{0.458514in}}%
\pgfpathcurveto{\pgfqpoint{0.518117in}{0.466328in}}{\pgfqpoint{0.522507in}{0.476927in}}{\pgfqpoint{0.522507in}{0.487977in}}%
\pgfpathcurveto{\pgfqpoint{0.522507in}{0.499027in}}{\pgfqpoint{0.518117in}{0.509626in}}{\pgfqpoint{0.510303in}{0.517440in}}%
\pgfpathcurveto{\pgfqpoint{0.502490in}{0.525253in}}{\pgfqpoint{0.491891in}{0.529644in}}{\pgfqpoint{0.480841in}{0.529644in}}%
\pgfpathcurveto{\pgfqpoint{0.469790in}{0.529644in}}{\pgfqpoint{0.459191in}{0.525253in}}{\pgfqpoint{0.451378in}{0.517440in}}%
\pgfpathcurveto{\pgfqpoint{0.443564in}{0.509626in}}{\pgfqpoint{0.439174in}{0.499027in}}{\pgfqpoint{0.439174in}{0.487977in}}%
\pgfpathcurveto{\pgfqpoint{0.439174in}{0.476927in}}{\pgfqpoint{0.443564in}{0.466328in}}{\pgfqpoint{0.451378in}{0.458514in}}%
\pgfpathcurveto{\pgfqpoint{0.459191in}{0.450700in}}{\pgfqpoint{0.469790in}{0.446310in}}{\pgfqpoint{0.480841in}{0.446310in}}%
\pgfpathclose%
\pgfusepath{stroke,fill}%
\end{pgfscope}%
\begin{pgfscope}%
\pgfpathrectangle{\pgfqpoint{0.375000in}{0.330000in}}{\pgfqpoint{2.325000in}{2.310000in}}%
\pgfusepath{clip}%
\pgfsetbuttcap%
\pgfsetroundjoin%
\definecolor{currentfill}{rgb}{0.000000,0.000000,0.000000}%
\pgfsetfillcolor{currentfill}%
\pgfsetlinewidth{1.003750pt}%
\definecolor{currentstroke}{rgb}{0.000000,0.000000,0.000000}%
\pgfsetstrokecolor{currentstroke}%
\pgfsetdash{}{0pt}%
\pgfpathmoveto{\pgfqpoint{0.480841in}{0.446310in}}%
\pgfpathcurveto{\pgfqpoint{0.491891in}{0.446310in}}{\pgfqpoint{0.502490in}{0.450700in}}{\pgfqpoint{0.510303in}{0.458514in}}%
\pgfpathcurveto{\pgfqpoint{0.518117in}{0.466328in}}{\pgfqpoint{0.522507in}{0.476927in}}{\pgfqpoint{0.522507in}{0.487977in}}%
\pgfpathcurveto{\pgfqpoint{0.522507in}{0.499027in}}{\pgfqpoint{0.518117in}{0.509626in}}{\pgfqpoint{0.510303in}{0.517440in}}%
\pgfpathcurveto{\pgfqpoint{0.502490in}{0.525253in}}{\pgfqpoint{0.491891in}{0.529644in}}{\pgfqpoint{0.480841in}{0.529644in}}%
\pgfpathcurveto{\pgfqpoint{0.469790in}{0.529644in}}{\pgfqpoint{0.459191in}{0.525253in}}{\pgfqpoint{0.451378in}{0.517440in}}%
\pgfpathcurveto{\pgfqpoint{0.443564in}{0.509626in}}{\pgfqpoint{0.439174in}{0.499027in}}{\pgfqpoint{0.439174in}{0.487977in}}%
\pgfpathcurveto{\pgfqpoint{0.439174in}{0.476927in}}{\pgfqpoint{0.443564in}{0.466328in}}{\pgfqpoint{0.451378in}{0.458514in}}%
\pgfpathcurveto{\pgfqpoint{0.459191in}{0.450700in}}{\pgfqpoint{0.469790in}{0.446310in}}{\pgfqpoint{0.480841in}{0.446310in}}%
\pgfpathclose%
\pgfusepath{stroke,fill}%
\end{pgfscope}%
\begin{pgfscope}%
\pgfpathrectangle{\pgfqpoint{0.375000in}{0.330000in}}{\pgfqpoint{2.325000in}{2.310000in}}%
\pgfusepath{clip}%
\pgfsetbuttcap%
\pgfsetroundjoin%
\definecolor{currentfill}{rgb}{0.000000,0.000000,0.000000}%
\pgfsetfillcolor{currentfill}%
\pgfsetlinewidth{1.003750pt}%
\definecolor{currentstroke}{rgb}{0.000000,0.000000,0.000000}%
\pgfsetstrokecolor{currentstroke}%
\pgfsetdash{}{0pt}%
\pgfpathmoveto{\pgfqpoint{0.480841in}{0.446310in}}%
\pgfpathcurveto{\pgfqpoint{0.491891in}{0.446310in}}{\pgfqpoint{0.502490in}{0.450700in}}{\pgfqpoint{0.510303in}{0.458514in}}%
\pgfpathcurveto{\pgfqpoint{0.518117in}{0.466328in}}{\pgfqpoint{0.522507in}{0.476927in}}{\pgfqpoint{0.522507in}{0.487977in}}%
\pgfpathcurveto{\pgfqpoint{0.522507in}{0.499027in}}{\pgfqpoint{0.518117in}{0.509626in}}{\pgfqpoint{0.510303in}{0.517440in}}%
\pgfpathcurveto{\pgfqpoint{0.502490in}{0.525253in}}{\pgfqpoint{0.491891in}{0.529644in}}{\pgfqpoint{0.480841in}{0.529644in}}%
\pgfpathcurveto{\pgfqpoint{0.469790in}{0.529644in}}{\pgfqpoint{0.459191in}{0.525253in}}{\pgfqpoint{0.451378in}{0.517440in}}%
\pgfpathcurveto{\pgfqpoint{0.443564in}{0.509626in}}{\pgfqpoint{0.439174in}{0.499027in}}{\pgfqpoint{0.439174in}{0.487977in}}%
\pgfpathcurveto{\pgfqpoint{0.439174in}{0.476927in}}{\pgfqpoint{0.443564in}{0.466328in}}{\pgfqpoint{0.451378in}{0.458514in}}%
\pgfpathcurveto{\pgfqpoint{0.459191in}{0.450700in}}{\pgfqpoint{0.469790in}{0.446310in}}{\pgfqpoint{0.480841in}{0.446310in}}%
\pgfpathclose%
\pgfusepath{stroke,fill}%
\end{pgfscope}%
\begin{pgfscope}%
\pgfpathrectangle{\pgfqpoint{0.375000in}{0.330000in}}{\pgfqpoint{2.325000in}{2.310000in}}%
\pgfusepath{clip}%
\pgfsetbuttcap%
\pgfsetroundjoin%
\definecolor{currentfill}{rgb}{0.000000,0.000000,0.000000}%
\pgfsetfillcolor{currentfill}%
\pgfsetlinewidth{1.003750pt}%
\definecolor{currentstroke}{rgb}{0.000000,0.000000,0.000000}%
\pgfsetstrokecolor{currentstroke}%
\pgfsetdash{}{0pt}%
\pgfpathmoveto{\pgfqpoint{0.480841in}{0.446310in}}%
\pgfpathcurveto{\pgfqpoint{0.491891in}{0.446310in}}{\pgfqpoint{0.502490in}{0.450700in}}{\pgfqpoint{0.510303in}{0.458514in}}%
\pgfpathcurveto{\pgfqpoint{0.518117in}{0.466328in}}{\pgfqpoint{0.522507in}{0.476927in}}{\pgfqpoint{0.522507in}{0.487977in}}%
\pgfpathcurveto{\pgfqpoint{0.522507in}{0.499027in}}{\pgfqpoint{0.518117in}{0.509626in}}{\pgfqpoint{0.510303in}{0.517440in}}%
\pgfpathcurveto{\pgfqpoint{0.502490in}{0.525253in}}{\pgfqpoint{0.491891in}{0.529644in}}{\pgfqpoint{0.480841in}{0.529644in}}%
\pgfpathcurveto{\pgfqpoint{0.469790in}{0.529644in}}{\pgfqpoint{0.459191in}{0.525253in}}{\pgfqpoint{0.451378in}{0.517440in}}%
\pgfpathcurveto{\pgfqpoint{0.443564in}{0.509626in}}{\pgfqpoint{0.439174in}{0.499027in}}{\pgfqpoint{0.439174in}{0.487977in}}%
\pgfpathcurveto{\pgfqpoint{0.439174in}{0.476927in}}{\pgfqpoint{0.443564in}{0.466328in}}{\pgfqpoint{0.451378in}{0.458514in}}%
\pgfpathcurveto{\pgfqpoint{0.459191in}{0.450700in}}{\pgfqpoint{0.469790in}{0.446310in}}{\pgfqpoint{0.480841in}{0.446310in}}%
\pgfpathclose%
\pgfusepath{stroke,fill}%
\end{pgfscope}%
\begin{pgfscope}%
\pgfpathrectangle{\pgfqpoint{0.375000in}{0.330000in}}{\pgfqpoint{2.325000in}{2.310000in}}%
\pgfusepath{clip}%
\pgfsetbuttcap%
\pgfsetroundjoin%
\definecolor{currentfill}{rgb}{0.000000,0.000000,0.000000}%
\pgfsetfillcolor{currentfill}%
\pgfsetlinewidth{1.003750pt}%
\definecolor{currentstroke}{rgb}{0.000000,0.000000,0.000000}%
\pgfsetstrokecolor{currentstroke}%
\pgfsetdash{}{0pt}%
\pgfpathmoveto{\pgfqpoint{0.480841in}{0.446310in}}%
\pgfpathcurveto{\pgfqpoint{0.491891in}{0.446310in}}{\pgfqpoint{0.502490in}{0.450700in}}{\pgfqpoint{0.510303in}{0.458514in}}%
\pgfpathcurveto{\pgfqpoint{0.518117in}{0.466328in}}{\pgfqpoint{0.522507in}{0.476927in}}{\pgfqpoint{0.522507in}{0.487977in}}%
\pgfpathcurveto{\pgfqpoint{0.522507in}{0.499027in}}{\pgfqpoint{0.518117in}{0.509626in}}{\pgfqpoint{0.510303in}{0.517440in}}%
\pgfpathcurveto{\pgfqpoint{0.502490in}{0.525253in}}{\pgfqpoint{0.491891in}{0.529644in}}{\pgfqpoint{0.480841in}{0.529644in}}%
\pgfpathcurveto{\pgfqpoint{0.469790in}{0.529644in}}{\pgfqpoint{0.459191in}{0.525253in}}{\pgfqpoint{0.451378in}{0.517440in}}%
\pgfpathcurveto{\pgfqpoint{0.443564in}{0.509626in}}{\pgfqpoint{0.439174in}{0.499027in}}{\pgfqpoint{0.439174in}{0.487977in}}%
\pgfpathcurveto{\pgfqpoint{0.439174in}{0.476927in}}{\pgfqpoint{0.443564in}{0.466328in}}{\pgfqpoint{0.451378in}{0.458514in}}%
\pgfpathcurveto{\pgfqpoint{0.459191in}{0.450700in}}{\pgfqpoint{0.469790in}{0.446310in}}{\pgfqpoint{0.480841in}{0.446310in}}%
\pgfpathclose%
\pgfusepath{stroke,fill}%
\end{pgfscope}%
\begin{pgfscope}%
\pgfpathrectangle{\pgfqpoint{0.375000in}{0.330000in}}{\pgfqpoint{2.325000in}{2.310000in}}%
\pgfusepath{clip}%
\pgfsetbuttcap%
\pgfsetroundjoin%
\definecolor{currentfill}{rgb}{0.000000,0.000000,0.000000}%
\pgfsetfillcolor{currentfill}%
\pgfsetlinewidth{1.003750pt}%
\definecolor{currentstroke}{rgb}{0.000000,0.000000,0.000000}%
\pgfsetstrokecolor{currentstroke}%
\pgfsetdash{}{0pt}%
\pgfpathmoveto{\pgfqpoint{0.480841in}{0.446310in}}%
\pgfpathcurveto{\pgfqpoint{0.491891in}{0.446310in}}{\pgfqpoint{0.502490in}{0.450700in}}{\pgfqpoint{0.510303in}{0.458514in}}%
\pgfpathcurveto{\pgfqpoint{0.518117in}{0.466328in}}{\pgfqpoint{0.522507in}{0.476927in}}{\pgfqpoint{0.522507in}{0.487977in}}%
\pgfpathcurveto{\pgfqpoint{0.522507in}{0.499027in}}{\pgfqpoint{0.518117in}{0.509626in}}{\pgfqpoint{0.510303in}{0.517440in}}%
\pgfpathcurveto{\pgfqpoint{0.502490in}{0.525253in}}{\pgfqpoint{0.491891in}{0.529644in}}{\pgfqpoint{0.480841in}{0.529644in}}%
\pgfpathcurveto{\pgfqpoint{0.469790in}{0.529644in}}{\pgfqpoint{0.459191in}{0.525253in}}{\pgfqpoint{0.451378in}{0.517440in}}%
\pgfpathcurveto{\pgfqpoint{0.443564in}{0.509626in}}{\pgfqpoint{0.439174in}{0.499027in}}{\pgfqpoint{0.439174in}{0.487977in}}%
\pgfpathcurveto{\pgfqpoint{0.439174in}{0.476927in}}{\pgfqpoint{0.443564in}{0.466328in}}{\pgfqpoint{0.451378in}{0.458514in}}%
\pgfpathcurveto{\pgfqpoint{0.459191in}{0.450700in}}{\pgfqpoint{0.469790in}{0.446310in}}{\pgfqpoint{0.480841in}{0.446310in}}%
\pgfpathclose%
\pgfusepath{stroke,fill}%
\end{pgfscope}%
\begin{pgfscope}%
\pgfpathrectangle{\pgfqpoint{0.375000in}{0.330000in}}{\pgfqpoint{2.325000in}{2.310000in}}%
\pgfusepath{clip}%
\pgfsetbuttcap%
\pgfsetroundjoin%
\definecolor{currentfill}{rgb}{0.000000,0.000000,0.000000}%
\pgfsetfillcolor{currentfill}%
\pgfsetlinewidth{1.003750pt}%
\definecolor{currentstroke}{rgb}{0.000000,0.000000,0.000000}%
\pgfsetstrokecolor{currentstroke}%
\pgfsetdash{}{0pt}%
\pgfpathmoveto{\pgfqpoint{0.480841in}{0.446310in}}%
\pgfpathcurveto{\pgfqpoint{0.491891in}{0.446310in}}{\pgfqpoint{0.502490in}{0.450700in}}{\pgfqpoint{0.510303in}{0.458514in}}%
\pgfpathcurveto{\pgfqpoint{0.518117in}{0.466328in}}{\pgfqpoint{0.522507in}{0.476927in}}{\pgfqpoint{0.522507in}{0.487977in}}%
\pgfpathcurveto{\pgfqpoint{0.522507in}{0.499027in}}{\pgfqpoint{0.518117in}{0.509626in}}{\pgfqpoint{0.510303in}{0.517440in}}%
\pgfpathcurveto{\pgfqpoint{0.502490in}{0.525253in}}{\pgfqpoint{0.491891in}{0.529644in}}{\pgfqpoint{0.480841in}{0.529644in}}%
\pgfpathcurveto{\pgfqpoint{0.469790in}{0.529644in}}{\pgfqpoint{0.459191in}{0.525253in}}{\pgfqpoint{0.451378in}{0.517440in}}%
\pgfpathcurveto{\pgfqpoint{0.443564in}{0.509626in}}{\pgfqpoint{0.439174in}{0.499027in}}{\pgfqpoint{0.439174in}{0.487977in}}%
\pgfpathcurveto{\pgfqpoint{0.439174in}{0.476927in}}{\pgfqpoint{0.443564in}{0.466328in}}{\pgfqpoint{0.451378in}{0.458514in}}%
\pgfpathcurveto{\pgfqpoint{0.459191in}{0.450700in}}{\pgfqpoint{0.469790in}{0.446310in}}{\pgfqpoint{0.480841in}{0.446310in}}%
\pgfpathclose%
\pgfusepath{stroke,fill}%
\end{pgfscope}%
\begin{pgfscope}%
\pgfpathrectangle{\pgfqpoint{0.375000in}{0.330000in}}{\pgfqpoint{2.325000in}{2.310000in}}%
\pgfusepath{clip}%
\pgfsetbuttcap%
\pgfsetroundjoin%
\definecolor{currentfill}{rgb}{0.000000,0.000000,0.000000}%
\pgfsetfillcolor{currentfill}%
\pgfsetlinewidth{1.003750pt}%
\definecolor{currentstroke}{rgb}{0.000000,0.000000,0.000000}%
\pgfsetstrokecolor{currentstroke}%
\pgfsetdash{}{0pt}%
\pgfpathmoveto{\pgfqpoint{0.480841in}{0.446310in}}%
\pgfpathcurveto{\pgfqpoint{0.491891in}{0.446310in}}{\pgfqpoint{0.502490in}{0.450700in}}{\pgfqpoint{0.510303in}{0.458514in}}%
\pgfpathcurveto{\pgfqpoint{0.518117in}{0.466328in}}{\pgfqpoint{0.522507in}{0.476927in}}{\pgfqpoint{0.522507in}{0.487977in}}%
\pgfpathcurveto{\pgfqpoint{0.522507in}{0.499027in}}{\pgfqpoint{0.518117in}{0.509626in}}{\pgfqpoint{0.510303in}{0.517440in}}%
\pgfpathcurveto{\pgfqpoint{0.502490in}{0.525253in}}{\pgfqpoint{0.491891in}{0.529644in}}{\pgfqpoint{0.480841in}{0.529644in}}%
\pgfpathcurveto{\pgfqpoint{0.469790in}{0.529644in}}{\pgfqpoint{0.459191in}{0.525253in}}{\pgfqpoint{0.451378in}{0.517440in}}%
\pgfpathcurveto{\pgfqpoint{0.443564in}{0.509626in}}{\pgfqpoint{0.439174in}{0.499027in}}{\pgfqpoint{0.439174in}{0.487977in}}%
\pgfpathcurveto{\pgfqpoint{0.439174in}{0.476927in}}{\pgfqpoint{0.443564in}{0.466328in}}{\pgfqpoint{0.451378in}{0.458514in}}%
\pgfpathcurveto{\pgfqpoint{0.459191in}{0.450700in}}{\pgfqpoint{0.469790in}{0.446310in}}{\pgfqpoint{0.480841in}{0.446310in}}%
\pgfpathclose%
\pgfusepath{stroke,fill}%
\end{pgfscope}%
\begin{pgfscope}%
\pgfpathrectangle{\pgfqpoint{0.375000in}{0.330000in}}{\pgfqpoint{2.325000in}{2.310000in}}%
\pgfusepath{clip}%
\pgfsetbuttcap%
\pgfsetroundjoin%
\definecolor{currentfill}{rgb}{0.000000,0.000000,0.000000}%
\pgfsetfillcolor{currentfill}%
\pgfsetlinewidth{1.003750pt}%
\definecolor{currentstroke}{rgb}{0.000000,0.000000,0.000000}%
\pgfsetstrokecolor{currentstroke}%
\pgfsetdash{}{0pt}%
\pgfpathmoveto{\pgfqpoint{0.480841in}{0.446310in}}%
\pgfpathcurveto{\pgfqpoint{0.491891in}{0.446310in}}{\pgfqpoint{0.502490in}{0.450700in}}{\pgfqpoint{0.510303in}{0.458514in}}%
\pgfpathcurveto{\pgfqpoint{0.518117in}{0.466328in}}{\pgfqpoint{0.522507in}{0.476927in}}{\pgfqpoint{0.522507in}{0.487977in}}%
\pgfpathcurveto{\pgfqpoint{0.522507in}{0.499027in}}{\pgfqpoint{0.518117in}{0.509626in}}{\pgfqpoint{0.510303in}{0.517440in}}%
\pgfpathcurveto{\pgfqpoint{0.502490in}{0.525253in}}{\pgfqpoint{0.491891in}{0.529644in}}{\pgfqpoint{0.480841in}{0.529644in}}%
\pgfpathcurveto{\pgfqpoint{0.469790in}{0.529644in}}{\pgfqpoint{0.459191in}{0.525253in}}{\pgfqpoint{0.451378in}{0.517440in}}%
\pgfpathcurveto{\pgfqpoint{0.443564in}{0.509626in}}{\pgfqpoint{0.439174in}{0.499027in}}{\pgfqpoint{0.439174in}{0.487977in}}%
\pgfpathcurveto{\pgfqpoint{0.439174in}{0.476927in}}{\pgfqpoint{0.443564in}{0.466328in}}{\pgfqpoint{0.451378in}{0.458514in}}%
\pgfpathcurveto{\pgfqpoint{0.459191in}{0.450700in}}{\pgfqpoint{0.469790in}{0.446310in}}{\pgfqpoint{0.480841in}{0.446310in}}%
\pgfpathclose%
\pgfusepath{stroke,fill}%
\end{pgfscope}%
\begin{pgfscope}%
\pgfpathrectangle{\pgfqpoint{0.375000in}{0.330000in}}{\pgfqpoint{2.325000in}{2.310000in}}%
\pgfusepath{clip}%
\pgfsetbuttcap%
\pgfsetroundjoin%
\definecolor{currentfill}{rgb}{0.000000,0.000000,0.000000}%
\pgfsetfillcolor{currentfill}%
\pgfsetlinewidth{1.003750pt}%
\definecolor{currentstroke}{rgb}{0.000000,0.000000,0.000000}%
\pgfsetstrokecolor{currentstroke}%
\pgfsetdash{}{0pt}%
\pgfpathmoveto{\pgfqpoint{0.480841in}{0.446310in}}%
\pgfpathcurveto{\pgfqpoint{0.491891in}{0.446310in}}{\pgfqpoint{0.502490in}{0.450700in}}{\pgfqpoint{0.510303in}{0.458514in}}%
\pgfpathcurveto{\pgfqpoint{0.518117in}{0.466328in}}{\pgfqpoint{0.522507in}{0.476927in}}{\pgfqpoint{0.522507in}{0.487977in}}%
\pgfpathcurveto{\pgfqpoint{0.522507in}{0.499027in}}{\pgfqpoint{0.518117in}{0.509626in}}{\pgfqpoint{0.510303in}{0.517440in}}%
\pgfpathcurveto{\pgfqpoint{0.502490in}{0.525253in}}{\pgfqpoint{0.491891in}{0.529644in}}{\pgfqpoint{0.480841in}{0.529644in}}%
\pgfpathcurveto{\pgfqpoint{0.469790in}{0.529644in}}{\pgfqpoint{0.459191in}{0.525253in}}{\pgfqpoint{0.451378in}{0.517440in}}%
\pgfpathcurveto{\pgfqpoint{0.443564in}{0.509626in}}{\pgfqpoint{0.439174in}{0.499027in}}{\pgfqpoint{0.439174in}{0.487977in}}%
\pgfpathcurveto{\pgfqpoint{0.439174in}{0.476927in}}{\pgfqpoint{0.443564in}{0.466328in}}{\pgfqpoint{0.451378in}{0.458514in}}%
\pgfpathcurveto{\pgfqpoint{0.459191in}{0.450700in}}{\pgfqpoint{0.469790in}{0.446310in}}{\pgfqpoint{0.480841in}{0.446310in}}%
\pgfpathclose%
\pgfusepath{stroke,fill}%
\end{pgfscope}%
\begin{pgfscope}%
\pgfpathrectangle{\pgfqpoint{0.375000in}{0.330000in}}{\pgfqpoint{2.325000in}{2.310000in}}%
\pgfusepath{clip}%
\pgfsetbuttcap%
\pgfsetroundjoin%
\definecolor{currentfill}{rgb}{0.000000,0.000000,0.000000}%
\pgfsetfillcolor{currentfill}%
\pgfsetlinewidth{1.003750pt}%
\definecolor{currentstroke}{rgb}{0.000000,0.000000,0.000000}%
\pgfsetstrokecolor{currentstroke}%
\pgfsetdash{}{0pt}%
\pgfpathmoveto{\pgfqpoint{0.480841in}{0.446310in}}%
\pgfpathcurveto{\pgfqpoint{0.491891in}{0.446310in}}{\pgfqpoint{0.502490in}{0.450700in}}{\pgfqpoint{0.510303in}{0.458514in}}%
\pgfpathcurveto{\pgfqpoint{0.518117in}{0.466328in}}{\pgfqpoint{0.522507in}{0.476927in}}{\pgfqpoint{0.522507in}{0.487977in}}%
\pgfpathcurveto{\pgfqpoint{0.522507in}{0.499027in}}{\pgfqpoint{0.518117in}{0.509626in}}{\pgfqpoint{0.510303in}{0.517440in}}%
\pgfpathcurveto{\pgfqpoint{0.502490in}{0.525253in}}{\pgfqpoint{0.491891in}{0.529644in}}{\pgfqpoint{0.480841in}{0.529644in}}%
\pgfpathcurveto{\pgfqpoint{0.469790in}{0.529644in}}{\pgfqpoint{0.459191in}{0.525253in}}{\pgfqpoint{0.451378in}{0.517440in}}%
\pgfpathcurveto{\pgfqpoint{0.443564in}{0.509626in}}{\pgfqpoint{0.439174in}{0.499027in}}{\pgfqpoint{0.439174in}{0.487977in}}%
\pgfpathcurveto{\pgfqpoint{0.439174in}{0.476927in}}{\pgfqpoint{0.443564in}{0.466328in}}{\pgfqpoint{0.451378in}{0.458514in}}%
\pgfpathcurveto{\pgfqpoint{0.459191in}{0.450700in}}{\pgfqpoint{0.469790in}{0.446310in}}{\pgfqpoint{0.480841in}{0.446310in}}%
\pgfpathclose%
\pgfusepath{stroke,fill}%
\end{pgfscope}%
\begin{pgfscope}%
\pgfpathrectangle{\pgfqpoint{0.375000in}{0.330000in}}{\pgfqpoint{2.325000in}{2.310000in}}%
\pgfusepath{clip}%
\pgfsetbuttcap%
\pgfsetroundjoin%
\definecolor{currentfill}{rgb}{0.000000,0.000000,0.000000}%
\pgfsetfillcolor{currentfill}%
\pgfsetlinewidth{1.003750pt}%
\definecolor{currentstroke}{rgb}{0.000000,0.000000,0.000000}%
\pgfsetstrokecolor{currentstroke}%
\pgfsetdash{}{0pt}%
\pgfpathmoveto{\pgfqpoint{0.480841in}{0.446310in}}%
\pgfpathcurveto{\pgfqpoint{0.491891in}{0.446310in}}{\pgfqpoint{0.502490in}{0.450700in}}{\pgfqpoint{0.510303in}{0.458514in}}%
\pgfpathcurveto{\pgfqpoint{0.518117in}{0.466328in}}{\pgfqpoint{0.522507in}{0.476927in}}{\pgfqpoint{0.522507in}{0.487977in}}%
\pgfpathcurveto{\pgfqpoint{0.522507in}{0.499027in}}{\pgfqpoint{0.518117in}{0.509626in}}{\pgfqpoint{0.510303in}{0.517440in}}%
\pgfpathcurveto{\pgfqpoint{0.502490in}{0.525253in}}{\pgfqpoint{0.491891in}{0.529644in}}{\pgfqpoint{0.480841in}{0.529644in}}%
\pgfpathcurveto{\pgfqpoint{0.469790in}{0.529644in}}{\pgfqpoint{0.459191in}{0.525253in}}{\pgfqpoint{0.451378in}{0.517440in}}%
\pgfpathcurveto{\pgfqpoint{0.443564in}{0.509626in}}{\pgfqpoint{0.439174in}{0.499027in}}{\pgfqpoint{0.439174in}{0.487977in}}%
\pgfpathcurveto{\pgfqpoint{0.439174in}{0.476927in}}{\pgfqpoint{0.443564in}{0.466328in}}{\pgfqpoint{0.451378in}{0.458514in}}%
\pgfpathcurveto{\pgfqpoint{0.459191in}{0.450700in}}{\pgfqpoint{0.469790in}{0.446310in}}{\pgfqpoint{0.480841in}{0.446310in}}%
\pgfpathclose%
\pgfusepath{stroke,fill}%
\end{pgfscope}%
\begin{pgfscope}%
\pgfpathrectangle{\pgfqpoint{0.375000in}{0.330000in}}{\pgfqpoint{2.325000in}{2.310000in}}%
\pgfusepath{clip}%
\pgfsetbuttcap%
\pgfsetroundjoin%
\definecolor{currentfill}{rgb}{0.000000,0.000000,0.000000}%
\pgfsetfillcolor{currentfill}%
\pgfsetlinewidth{1.003750pt}%
\definecolor{currentstroke}{rgb}{0.000000,0.000000,0.000000}%
\pgfsetstrokecolor{currentstroke}%
\pgfsetdash{}{0pt}%
\pgfpathmoveto{\pgfqpoint{0.480841in}{0.394279in}}%
\pgfpathcurveto{\pgfqpoint{0.491891in}{0.394279in}}{\pgfqpoint{0.502490in}{0.398670in}}{\pgfqpoint{0.510303in}{0.406483in}}%
\pgfpathcurveto{\pgfqpoint{0.518117in}{0.414297in}}{\pgfqpoint{0.522507in}{0.424896in}}{\pgfqpoint{0.522507in}{0.435946in}}%
\pgfpathcurveto{\pgfqpoint{0.522507in}{0.446996in}}{\pgfqpoint{0.518117in}{0.457595in}}{\pgfqpoint{0.510303in}{0.465409in}}%
\pgfpathcurveto{\pgfqpoint{0.502490in}{0.473222in}}{\pgfqpoint{0.491891in}{0.477613in}}{\pgfqpoint{0.480841in}{0.477613in}}%
\pgfpathcurveto{\pgfqpoint{0.469790in}{0.477613in}}{\pgfqpoint{0.459191in}{0.473222in}}{\pgfqpoint{0.451378in}{0.465409in}}%
\pgfpathcurveto{\pgfqpoint{0.443564in}{0.457595in}}{\pgfqpoint{0.439174in}{0.446996in}}{\pgfqpoint{0.439174in}{0.435946in}}%
\pgfpathcurveto{\pgfqpoint{0.439174in}{0.424896in}}{\pgfqpoint{0.443564in}{0.414297in}}{\pgfqpoint{0.451378in}{0.406483in}}%
\pgfpathcurveto{\pgfqpoint{0.459191in}{0.398670in}}{\pgfqpoint{0.469790in}{0.394279in}}{\pgfqpoint{0.480841in}{0.394279in}}%
\pgfpathclose%
\pgfusepath{stroke,fill}%
\end{pgfscope}%
\begin{pgfscope}%
\pgfpathrectangle{\pgfqpoint{0.375000in}{0.330000in}}{\pgfqpoint{2.325000in}{2.310000in}}%
\pgfusepath{clip}%
\pgfsetbuttcap%
\pgfsetroundjoin%
\definecolor{currentfill}{rgb}{0.000000,0.000000,0.000000}%
\pgfsetfillcolor{currentfill}%
\pgfsetlinewidth{1.003750pt}%
\definecolor{currentstroke}{rgb}{0.000000,0.000000,0.000000}%
\pgfsetstrokecolor{currentstroke}%
\pgfsetdash{}{0pt}%
\pgfpathmoveto{\pgfqpoint{0.480841in}{0.446310in}}%
\pgfpathcurveto{\pgfqpoint{0.491891in}{0.446310in}}{\pgfqpoint{0.502490in}{0.450700in}}{\pgfqpoint{0.510303in}{0.458514in}}%
\pgfpathcurveto{\pgfqpoint{0.518117in}{0.466328in}}{\pgfqpoint{0.522507in}{0.476927in}}{\pgfqpoint{0.522507in}{0.487977in}}%
\pgfpathcurveto{\pgfqpoint{0.522507in}{0.499027in}}{\pgfqpoint{0.518117in}{0.509626in}}{\pgfqpoint{0.510303in}{0.517440in}}%
\pgfpathcurveto{\pgfqpoint{0.502490in}{0.525253in}}{\pgfqpoint{0.491891in}{0.529644in}}{\pgfqpoint{0.480841in}{0.529644in}}%
\pgfpathcurveto{\pgfqpoint{0.469790in}{0.529644in}}{\pgfqpoint{0.459191in}{0.525253in}}{\pgfqpoint{0.451378in}{0.517440in}}%
\pgfpathcurveto{\pgfqpoint{0.443564in}{0.509626in}}{\pgfqpoint{0.439174in}{0.499027in}}{\pgfqpoint{0.439174in}{0.487977in}}%
\pgfpathcurveto{\pgfqpoint{0.439174in}{0.476927in}}{\pgfqpoint{0.443564in}{0.466328in}}{\pgfqpoint{0.451378in}{0.458514in}}%
\pgfpathcurveto{\pgfqpoint{0.459191in}{0.450700in}}{\pgfqpoint{0.469790in}{0.446310in}}{\pgfqpoint{0.480841in}{0.446310in}}%
\pgfpathclose%
\pgfusepath{stroke,fill}%
\end{pgfscope}%
\begin{pgfscope}%
\pgfpathrectangle{\pgfqpoint{0.375000in}{0.330000in}}{\pgfqpoint{2.325000in}{2.310000in}}%
\pgfusepath{clip}%
\pgfsetbuttcap%
\pgfsetroundjoin%
\definecolor{currentfill}{rgb}{0.000000,0.000000,0.000000}%
\pgfsetfillcolor{currentfill}%
\pgfsetlinewidth{1.003750pt}%
\definecolor{currentstroke}{rgb}{0.000000,0.000000,0.000000}%
\pgfsetstrokecolor{currentstroke}%
\pgfsetdash{}{0pt}%
\pgfpathmoveto{\pgfqpoint{0.480841in}{0.446310in}}%
\pgfpathcurveto{\pgfqpoint{0.491891in}{0.446310in}}{\pgfqpoint{0.502490in}{0.450700in}}{\pgfqpoint{0.510303in}{0.458514in}}%
\pgfpathcurveto{\pgfqpoint{0.518117in}{0.466328in}}{\pgfqpoint{0.522507in}{0.476927in}}{\pgfqpoint{0.522507in}{0.487977in}}%
\pgfpathcurveto{\pgfqpoint{0.522507in}{0.499027in}}{\pgfqpoint{0.518117in}{0.509626in}}{\pgfqpoint{0.510303in}{0.517440in}}%
\pgfpathcurveto{\pgfqpoint{0.502490in}{0.525253in}}{\pgfqpoint{0.491891in}{0.529644in}}{\pgfqpoint{0.480841in}{0.529644in}}%
\pgfpathcurveto{\pgfqpoint{0.469790in}{0.529644in}}{\pgfqpoint{0.459191in}{0.525253in}}{\pgfqpoint{0.451378in}{0.517440in}}%
\pgfpathcurveto{\pgfqpoint{0.443564in}{0.509626in}}{\pgfqpoint{0.439174in}{0.499027in}}{\pgfqpoint{0.439174in}{0.487977in}}%
\pgfpathcurveto{\pgfqpoint{0.439174in}{0.476927in}}{\pgfqpoint{0.443564in}{0.466328in}}{\pgfqpoint{0.451378in}{0.458514in}}%
\pgfpathcurveto{\pgfqpoint{0.459191in}{0.450700in}}{\pgfqpoint{0.469790in}{0.446310in}}{\pgfqpoint{0.480841in}{0.446310in}}%
\pgfpathclose%
\pgfusepath{stroke,fill}%
\end{pgfscope}%
\begin{pgfscope}%
\pgfpathrectangle{\pgfqpoint{0.375000in}{0.330000in}}{\pgfqpoint{2.325000in}{2.310000in}}%
\pgfusepath{clip}%
\pgfsetbuttcap%
\pgfsetroundjoin%
\definecolor{currentfill}{rgb}{0.000000,0.000000,0.000000}%
\pgfsetfillcolor{currentfill}%
\pgfsetlinewidth{1.003750pt}%
\definecolor{currentstroke}{rgb}{0.000000,0.000000,0.000000}%
\pgfsetstrokecolor{currentstroke}%
\pgfsetdash{}{0pt}%
\pgfpathmoveto{\pgfqpoint{0.480841in}{0.446310in}}%
\pgfpathcurveto{\pgfqpoint{0.491891in}{0.446310in}}{\pgfqpoint{0.502490in}{0.450700in}}{\pgfqpoint{0.510303in}{0.458514in}}%
\pgfpathcurveto{\pgfqpoint{0.518117in}{0.466328in}}{\pgfqpoint{0.522507in}{0.476927in}}{\pgfqpoint{0.522507in}{0.487977in}}%
\pgfpathcurveto{\pgfqpoint{0.522507in}{0.499027in}}{\pgfqpoint{0.518117in}{0.509626in}}{\pgfqpoint{0.510303in}{0.517440in}}%
\pgfpathcurveto{\pgfqpoint{0.502490in}{0.525253in}}{\pgfqpoint{0.491891in}{0.529644in}}{\pgfqpoint{0.480841in}{0.529644in}}%
\pgfpathcurveto{\pgfqpoint{0.469790in}{0.529644in}}{\pgfqpoint{0.459191in}{0.525253in}}{\pgfqpoint{0.451378in}{0.517440in}}%
\pgfpathcurveto{\pgfqpoint{0.443564in}{0.509626in}}{\pgfqpoint{0.439174in}{0.499027in}}{\pgfqpoint{0.439174in}{0.487977in}}%
\pgfpathcurveto{\pgfqpoint{0.439174in}{0.476927in}}{\pgfqpoint{0.443564in}{0.466328in}}{\pgfqpoint{0.451378in}{0.458514in}}%
\pgfpathcurveto{\pgfqpoint{0.459191in}{0.450700in}}{\pgfqpoint{0.469790in}{0.446310in}}{\pgfqpoint{0.480841in}{0.446310in}}%
\pgfpathclose%
\pgfusepath{stroke,fill}%
\end{pgfscope}%
\begin{pgfscope}%
\pgfpathrectangle{\pgfqpoint{0.375000in}{0.330000in}}{\pgfqpoint{2.325000in}{2.310000in}}%
\pgfusepath{clip}%
\pgfsetbuttcap%
\pgfsetroundjoin%
\definecolor{currentfill}{rgb}{0.000000,0.000000,0.000000}%
\pgfsetfillcolor{currentfill}%
\pgfsetlinewidth{1.003750pt}%
\definecolor{currentstroke}{rgb}{0.000000,0.000000,0.000000}%
\pgfsetstrokecolor{currentstroke}%
\pgfsetdash{}{0pt}%
\pgfpathmoveto{\pgfqpoint{0.480841in}{0.446310in}}%
\pgfpathcurveto{\pgfqpoint{0.491891in}{0.446310in}}{\pgfqpoint{0.502490in}{0.450700in}}{\pgfqpoint{0.510303in}{0.458514in}}%
\pgfpathcurveto{\pgfqpoint{0.518117in}{0.466328in}}{\pgfqpoint{0.522507in}{0.476927in}}{\pgfqpoint{0.522507in}{0.487977in}}%
\pgfpathcurveto{\pgfqpoint{0.522507in}{0.499027in}}{\pgfqpoint{0.518117in}{0.509626in}}{\pgfqpoint{0.510303in}{0.517440in}}%
\pgfpathcurveto{\pgfqpoint{0.502490in}{0.525253in}}{\pgfqpoint{0.491891in}{0.529644in}}{\pgfqpoint{0.480841in}{0.529644in}}%
\pgfpathcurveto{\pgfqpoint{0.469790in}{0.529644in}}{\pgfqpoint{0.459191in}{0.525253in}}{\pgfqpoint{0.451378in}{0.517440in}}%
\pgfpathcurveto{\pgfqpoint{0.443564in}{0.509626in}}{\pgfqpoint{0.439174in}{0.499027in}}{\pgfqpoint{0.439174in}{0.487977in}}%
\pgfpathcurveto{\pgfqpoint{0.439174in}{0.476927in}}{\pgfqpoint{0.443564in}{0.466328in}}{\pgfqpoint{0.451378in}{0.458514in}}%
\pgfpathcurveto{\pgfqpoint{0.459191in}{0.450700in}}{\pgfqpoint{0.469790in}{0.446310in}}{\pgfqpoint{0.480841in}{0.446310in}}%
\pgfpathclose%
\pgfusepath{stroke,fill}%
\end{pgfscope}%
\begin{pgfscope}%
\pgfpathrectangle{\pgfqpoint{0.375000in}{0.330000in}}{\pgfqpoint{2.325000in}{2.310000in}}%
\pgfusepath{clip}%
\pgfsetbuttcap%
\pgfsetroundjoin%
\definecolor{currentfill}{rgb}{0.000000,0.000000,0.000000}%
\pgfsetfillcolor{currentfill}%
\pgfsetlinewidth{1.003750pt}%
\definecolor{currentstroke}{rgb}{0.000000,0.000000,0.000000}%
\pgfsetstrokecolor{currentstroke}%
\pgfsetdash{}{0pt}%
\pgfpathmoveto{\pgfqpoint{0.480841in}{0.446310in}}%
\pgfpathcurveto{\pgfqpoint{0.491891in}{0.446310in}}{\pgfqpoint{0.502490in}{0.450700in}}{\pgfqpoint{0.510303in}{0.458514in}}%
\pgfpathcurveto{\pgfqpoint{0.518117in}{0.466328in}}{\pgfqpoint{0.522507in}{0.476927in}}{\pgfqpoint{0.522507in}{0.487977in}}%
\pgfpathcurveto{\pgfqpoint{0.522507in}{0.499027in}}{\pgfqpoint{0.518117in}{0.509626in}}{\pgfqpoint{0.510303in}{0.517440in}}%
\pgfpathcurveto{\pgfqpoint{0.502490in}{0.525253in}}{\pgfqpoint{0.491891in}{0.529644in}}{\pgfqpoint{0.480841in}{0.529644in}}%
\pgfpathcurveto{\pgfqpoint{0.469790in}{0.529644in}}{\pgfqpoint{0.459191in}{0.525253in}}{\pgfqpoint{0.451378in}{0.517440in}}%
\pgfpathcurveto{\pgfqpoint{0.443564in}{0.509626in}}{\pgfqpoint{0.439174in}{0.499027in}}{\pgfqpoint{0.439174in}{0.487977in}}%
\pgfpathcurveto{\pgfqpoint{0.439174in}{0.476927in}}{\pgfqpoint{0.443564in}{0.466328in}}{\pgfqpoint{0.451378in}{0.458514in}}%
\pgfpathcurveto{\pgfqpoint{0.459191in}{0.450700in}}{\pgfqpoint{0.469790in}{0.446310in}}{\pgfqpoint{0.480841in}{0.446310in}}%
\pgfpathclose%
\pgfusepath{stroke,fill}%
\end{pgfscope}%
\begin{pgfscope}%
\pgfpathrectangle{\pgfqpoint{0.375000in}{0.330000in}}{\pgfqpoint{2.325000in}{2.310000in}}%
\pgfusepath{clip}%
\pgfsetbuttcap%
\pgfsetroundjoin%
\definecolor{currentfill}{rgb}{0.000000,0.000000,0.000000}%
\pgfsetfillcolor{currentfill}%
\pgfsetlinewidth{1.003750pt}%
\definecolor{currentstroke}{rgb}{0.000000,0.000000,0.000000}%
\pgfsetstrokecolor{currentstroke}%
\pgfsetdash{}{0pt}%
\pgfpathmoveto{\pgfqpoint{0.480841in}{0.446310in}}%
\pgfpathcurveto{\pgfqpoint{0.491891in}{0.446310in}}{\pgfqpoint{0.502490in}{0.450700in}}{\pgfqpoint{0.510303in}{0.458514in}}%
\pgfpathcurveto{\pgfqpoint{0.518117in}{0.466328in}}{\pgfqpoint{0.522507in}{0.476927in}}{\pgfqpoint{0.522507in}{0.487977in}}%
\pgfpathcurveto{\pgfqpoint{0.522507in}{0.499027in}}{\pgfqpoint{0.518117in}{0.509626in}}{\pgfqpoint{0.510303in}{0.517440in}}%
\pgfpathcurveto{\pgfqpoint{0.502490in}{0.525253in}}{\pgfqpoint{0.491891in}{0.529644in}}{\pgfqpoint{0.480841in}{0.529644in}}%
\pgfpathcurveto{\pgfqpoint{0.469790in}{0.529644in}}{\pgfqpoint{0.459191in}{0.525253in}}{\pgfqpoint{0.451378in}{0.517440in}}%
\pgfpathcurveto{\pgfqpoint{0.443564in}{0.509626in}}{\pgfqpoint{0.439174in}{0.499027in}}{\pgfqpoint{0.439174in}{0.487977in}}%
\pgfpathcurveto{\pgfqpoint{0.439174in}{0.476927in}}{\pgfqpoint{0.443564in}{0.466328in}}{\pgfqpoint{0.451378in}{0.458514in}}%
\pgfpathcurveto{\pgfqpoint{0.459191in}{0.450700in}}{\pgfqpoint{0.469790in}{0.446310in}}{\pgfqpoint{0.480841in}{0.446310in}}%
\pgfpathclose%
\pgfusepath{stroke,fill}%
\end{pgfscope}%
\begin{pgfscope}%
\pgfpathrectangle{\pgfqpoint{0.375000in}{0.330000in}}{\pgfqpoint{2.325000in}{2.310000in}}%
\pgfusepath{clip}%
\pgfsetbuttcap%
\pgfsetroundjoin%
\definecolor{currentfill}{rgb}{0.000000,0.000000,0.000000}%
\pgfsetfillcolor{currentfill}%
\pgfsetlinewidth{1.003750pt}%
\definecolor{currentstroke}{rgb}{0.000000,0.000000,0.000000}%
\pgfsetstrokecolor{currentstroke}%
\pgfsetdash{}{0pt}%
\pgfpathmoveto{\pgfqpoint{0.480841in}{0.446310in}}%
\pgfpathcurveto{\pgfqpoint{0.491891in}{0.446310in}}{\pgfqpoint{0.502490in}{0.450700in}}{\pgfqpoint{0.510303in}{0.458514in}}%
\pgfpathcurveto{\pgfqpoint{0.518117in}{0.466328in}}{\pgfqpoint{0.522507in}{0.476927in}}{\pgfqpoint{0.522507in}{0.487977in}}%
\pgfpathcurveto{\pgfqpoint{0.522507in}{0.499027in}}{\pgfqpoint{0.518117in}{0.509626in}}{\pgfqpoint{0.510303in}{0.517440in}}%
\pgfpathcurveto{\pgfqpoint{0.502490in}{0.525253in}}{\pgfqpoint{0.491891in}{0.529644in}}{\pgfqpoint{0.480841in}{0.529644in}}%
\pgfpathcurveto{\pgfqpoint{0.469790in}{0.529644in}}{\pgfqpoint{0.459191in}{0.525253in}}{\pgfqpoint{0.451378in}{0.517440in}}%
\pgfpathcurveto{\pgfqpoint{0.443564in}{0.509626in}}{\pgfqpoint{0.439174in}{0.499027in}}{\pgfqpoint{0.439174in}{0.487977in}}%
\pgfpathcurveto{\pgfqpoint{0.439174in}{0.476927in}}{\pgfqpoint{0.443564in}{0.466328in}}{\pgfqpoint{0.451378in}{0.458514in}}%
\pgfpathcurveto{\pgfqpoint{0.459191in}{0.450700in}}{\pgfqpoint{0.469790in}{0.446310in}}{\pgfqpoint{0.480841in}{0.446310in}}%
\pgfpathclose%
\pgfusepath{stroke,fill}%
\end{pgfscope}%
\begin{pgfscope}%
\pgfpathrectangle{\pgfqpoint{0.375000in}{0.330000in}}{\pgfqpoint{2.325000in}{2.310000in}}%
\pgfusepath{clip}%
\pgfsetbuttcap%
\pgfsetroundjoin%
\definecolor{currentfill}{rgb}{0.000000,0.000000,0.000000}%
\pgfsetfillcolor{currentfill}%
\pgfsetlinewidth{1.003750pt}%
\definecolor{currentstroke}{rgb}{0.000000,0.000000,0.000000}%
\pgfsetstrokecolor{currentstroke}%
\pgfsetdash{}{0pt}%
\pgfpathmoveto{\pgfqpoint{0.480841in}{0.446310in}}%
\pgfpathcurveto{\pgfqpoint{0.491891in}{0.446310in}}{\pgfqpoint{0.502490in}{0.450700in}}{\pgfqpoint{0.510303in}{0.458514in}}%
\pgfpathcurveto{\pgfqpoint{0.518117in}{0.466328in}}{\pgfqpoint{0.522507in}{0.476927in}}{\pgfqpoint{0.522507in}{0.487977in}}%
\pgfpathcurveto{\pgfqpoint{0.522507in}{0.499027in}}{\pgfqpoint{0.518117in}{0.509626in}}{\pgfqpoint{0.510303in}{0.517440in}}%
\pgfpathcurveto{\pgfqpoint{0.502490in}{0.525253in}}{\pgfqpoint{0.491891in}{0.529644in}}{\pgfqpoint{0.480841in}{0.529644in}}%
\pgfpathcurveto{\pgfqpoint{0.469790in}{0.529644in}}{\pgfqpoint{0.459191in}{0.525253in}}{\pgfqpoint{0.451378in}{0.517440in}}%
\pgfpathcurveto{\pgfqpoint{0.443564in}{0.509626in}}{\pgfqpoint{0.439174in}{0.499027in}}{\pgfqpoint{0.439174in}{0.487977in}}%
\pgfpathcurveto{\pgfqpoint{0.439174in}{0.476927in}}{\pgfqpoint{0.443564in}{0.466328in}}{\pgfqpoint{0.451378in}{0.458514in}}%
\pgfpathcurveto{\pgfqpoint{0.459191in}{0.450700in}}{\pgfqpoint{0.469790in}{0.446310in}}{\pgfqpoint{0.480841in}{0.446310in}}%
\pgfpathclose%
\pgfusepath{stroke,fill}%
\end{pgfscope}%
\begin{pgfscope}%
\pgfpathrectangle{\pgfqpoint{0.375000in}{0.330000in}}{\pgfqpoint{2.325000in}{2.310000in}}%
\pgfusepath{clip}%
\pgfsetbuttcap%
\pgfsetroundjoin%
\definecolor{currentfill}{rgb}{0.000000,0.000000,0.000000}%
\pgfsetfillcolor{currentfill}%
\pgfsetlinewidth{1.003750pt}%
\definecolor{currentstroke}{rgb}{0.000000,0.000000,0.000000}%
\pgfsetstrokecolor{currentstroke}%
\pgfsetdash{}{0pt}%
\pgfpathmoveto{\pgfqpoint{0.480841in}{0.446310in}}%
\pgfpathcurveto{\pgfqpoint{0.491891in}{0.446310in}}{\pgfqpoint{0.502490in}{0.450700in}}{\pgfqpoint{0.510303in}{0.458514in}}%
\pgfpathcurveto{\pgfqpoint{0.518117in}{0.466328in}}{\pgfqpoint{0.522507in}{0.476927in}}{\pgfqpoint{0.522507in}{0.487977in}}%
\pgfpathcurveto{\pgfqpoint{0.522507in}{0.499027in}}{\pgfqpoint{0.518117in}{0.509626in}}{\pgfqpoint{0.510303in}{0.517440in}}%
\pgfpathcurveto{\pgfqpoint{0.502490in}{0.525253in}}{\pgfqpoint{0.491891in}{0.529644in}}{\pgfqpoint{0.480841in}{0.529644in}}%
\pgfpathcurveto{\pgfqpoint{0.469790in}{0.529644in}}{\pgfqpoint{0.459191in}{0.525253in}}{\pgfqpoint{0.451378in}{0.517440in}}%
\pgfpathcurveto{\pgfqpoint{0.443564in}{0.509626in}}{\pgfqpoint{0.439174in}{0.499027in}}{\pgfqpoint{0.439174in}{0.487977in}}%
\pgfpathcurveto{\pgfqpoint{0.439174in}{0.476927in}}{\pgfqpoint{0.443564in}{0.466328in}}{\pgfqpoint{0.451378in}{0.458514in}}%
\pgfpathcurveto{\pgfqpoint{0.459191in}{0.450700in}}{\pgfqpoint{0.469790in}{0.446310in}}{\pgfqpoint{0.480841in}{0.446310in}}%
\pgfpathclose%
\pgfusepath{stroke,fill}%
\end{pgfscope}%
\begin{pgfscope}%
\pgfpathrectangle{\pgfqpoint{0.375000in}{0.330000in}}{\pgfqpoint{2.325000in}{2.310000in}}%
\pgfusepath{clip}%
\pgfsetbuttcap%
\pgfsetroundjoin%
\definecolor{currentfill}{rgb}{0.000000,0.000000,0.000000}%
\pgfsetfillcolor{currentfill}%
\pgfsetlinewidth{1.003750pt}%
\definecolor{currentstroke}{rgb}{0.000000,0.000000,0.000000}%
\pgfsetstrokecolor{currentstroke}%
\pgfsetdash{}{0pt}%
\pgfpathmoveto{\pgfqpoint{0.480841in}{0.446310in}}%
\pgfpathcurveto{\pgfqpoint{0.491891in}{0.446310in}}{\pgfqpoint{0.502490in}{0.450700in}}{\pgfqpoint{0.510303in}{0.458514in}}%
\pgfpathcurveto{\pgfqpoint{0.518117in}{0.466328in}}{\pgfqpoint{0.522507in}{0.476927in}}{\pgfqpoint{0.522507in}{0.487977in}}%
\pgfpathcurveto{\pgfqpoint{0.522507in}{0.499027in}}{\pgfqpoint{0.518117in}{0.509626in}}{\pgfqpoint{0.510303in}{0.517440in}}%
\pgfpathcurveto{\pgfqpoint{0.502490in}{0.525253in}}{\pgfqpoint{0.491891in}{0.529644in}}{\pgfqpoint{0.480841in}{0.529644in}}%
\pgfpathcurveto{\pgfqpoint{0.469790in}{0.529644in}}{\pgfqpoint{0.459191in}{0.525253in}}{\pgfqpoint{0.451378in}{0.517440in}}%
\pgfpathcurveto{\pgfqpoint{0.443564in}{0.509626in}}{\pgfqpoint{0.439174in}{0.499027in}}{\pgfqpoint{0.439174in}{0.487977in}}%
\pgfpathcurveto{\pgfqpoint{0.439174in}{0.476927in}}{\pgfqpoint{0.443564in}{0.466328in}}{\pgfqpoint{0.451378in}{0.458514in}}%
\pgfpathcurveto{\pgfqpoint{0.459191in}{0.450700in}}{\pgfqpoint{0.469790in}{0.446310in}}{\pgfqpoint{0.480841in}{0.446310in}}%
\pgfpathclose%
\pgfusepath{stroke,fill}%
\end{pgfscope}%
\begin{pgfscope}%
\pgfpathrectangle{\pgfqpoint{0.375000in}{0.330000in}}{\pgfqpoint{2.325000in}{2.310000in}}%
\pgfusepath{clip}%
\pgfsetbuttcap%
\pgfsetroundjoin%
\definecolor{currentfill}{rgb}{0.000000,0.000000,0.000000}%
\pgfsetfillcolor{currentfill}%
\pgfsetlinewidth{1.003750pt}%
\definecolor{currentstroke}{rgb}{0.000000,0.000000,0.000000}%
\pgfsetstrokecolor{currentstroke}%
\pgfsetdash{}{0pt}%
\pgfpathmoveto{\pgfqpoint{0.480841in}{0.446310in}}%
\pgfpathcurveto{\pgfqpoint{0.491891in}{0.446310in}}{\pgfqpoint{0.502490in}{0.450700in}}{\pgfqpoint{0.510303in}{0.458514in}}%
\pgfpathcurveto{\pgfqpoint{0.518117in}{0.466328in}}{\pgfqpoint{0.522507in}{0.476927in}}{\pgfqpoint{0.522507in}{0.487977in}}%
\pgfpathcurveto{\pgfqpoint{0.522507in}{0.499027in}}{\pgfqpoint{0.518117in}{0.509626in}}{\pgfqpoint{0.510303in}{0.517440in}}%
\pgfpathcurveto{\pgfqpoint{0.502490in}{0.525253in}}{\pgfqpoint{0.491891in}{0.529644in}}{\pgfqpoint{0.480841in}{0.529644in}}%
\pgfpathcurveto{\pgfqpoint{0.469790in}{0.529644in}}{\pgfqpoint{0.459191in}{0.525253in}}{\pgfqpoint{0.451378in}{0.517440in}}%
\pgfpathcurveto{\pgfqpoint{0.443564in}{0.509626in}}{\pgfqpoint{0.439174in}{0.499027in}}{\pgfqpoint{0.439174in}{0.487977in}}%
\pgfpathcurveto{\pgfqpoint{0.439174in}{0.476927in}}{\pgfqpoint{0.443564in}{0.466328in}}{\pgfqpoint{0.451378in}{0.458514in}}%
\pgfpathcurveto{\pgfqpoint{0.459191in}{0.450700in}}{\pgfqpoint{0.469790in}{0.446310in}}{\pgfqpoint{0.480841in}{0.446310in}}%
\pgfpathclose%
\pgfusepath{stroke,fill}%
\end{pgfscope}%
\begin{pgfscope}%
\pgfpathrectangle{\pgfqpoint{0.375000in}{0.330000in}}{\pgfqpoint{2.325000in}{2.310000in}}%
\pgfusepath{clip}%
\pgfsetbuttcap%
\pgfsetroundjoin%
\definecolor{currentfill}{rgb}{0.000000,0.000000,0.000000}%
\pgfsetfillcolor{currentfill}%
\pgfsetlinewidth{1.003750pt}%
\definecolor{currentstroke}{rgb}{0.000000,0.000000,0.000000}%
\pgfsetstrokecolor{currentstroke}%
\pgfsetdash{}{0pt}%
\pgfpathmoveto{\pgfqpoint{0.480841in}{0.446310in}}%
\pgfpathcurveto{\pgfqpoint{0.491891in}{0.446310in}}{\pgfqpoint{0.502490in}{0.450700in}}{\pgfqpoint{0.510303in}{0.458514in}}%
\pgfpathcurveto{\pgfqpoint{0.518117in}{0.466328in}}{\pgfqpoint{0.522507in}{0.476927in}}{\pgfqpoint{0.522507in}{0.487977in}}%
\pgfpathcurveto{\pgfqpoint{0.522507in}{0.499027in}}{\pgfqpoint{0.518117in}{0.509626in}}{\pgfqpoint{0.510303in}{0.517440in}}%
\pgfpathcurveto{\pgfqpoint{0.502490in}{0.525253in}}{\pgfqpoint{0.491891in}{0.529644in}}{\pgfqpoint{0.480841in}{0.529644in}}%
\pgfpathcurveto{\pgfqpoint{0.469790in}{0.529644in}}{\pgfqpoint{0.459191in}{0.525253in}}{\pgfqpoint{0.451378in}{0.517440in}}%
\pgfpathcurveto{\pgfqpoint{0.443564in}{0.509626in}}{\pgfqpoint{0.439174in}{0.499027in}}{\pgfqpoint{0.439174in}{0.487977in}}%
\pgfpathcurveto{\pgfqpoint{0.439174in}{0.476927in}}{\pgfqpoint{0.443564in}{0.466328in}}{\pgfqpoint{0.451378in}{0.458514in}}%
\pgfpathcurveto{\pgfqpoint{0.459191in}{0.450700in}}{\pgfqpoint{0.469790in}{0.446310in}}{\pgfqpoint{0.480841in}{0.446310in}}%
\pgfpathclose%
\pgfusepath{stroke,fill}%
\end{pgfscope}%
\begin{pgfscope}%
\pgfpathrectangle{\pgfqpoint{0.375000in}{0.330000in}}{\pgfqpoint{2.325000in}{2.310000in}}%
\pgfusepath{clip}%
\pgfsetbuttcap%
\pgfsetroundjoin%
\definecolor{currentfill}{rgb}{0.000000,0.000000,0.000000}%
\pgfsetfillcolor{currentfill}%
\pgfsetlinewidth{1.003750pt}%
\definecolor{currentstroke}{rgb}{0.000000,0.000000,0.000000}%
\pgfsetstrokecolor{currentstroke}%
\pgfsetdash{}{0pt}%
\pgfpathmoveto{\pgfqpoint{0.480841in}{0.446310in}}%
\pgfpathcurveto{\pgfqpoint{0.491891in}{0.446310in}}{\pgfqpoint{0.502490in}{0.450700in}}{\pgfqpoint{0.510303in}{0.458514in}}%
\pgfpathcurveto{\pgfqpoint{0.518117in}{0.466328in}}{\pgfqpoint{0.522507in}{0.476927in}}{\pgfqpoint{0.522507in}{0.487977in}}%
\pgfpathcurveto{\pgfqpoint{0.522507in}{0.499027in}}{\pgfqpoint{0.518117in}{0.509626in}}{\pgfqpoint{0.510303in}{0.517440in}}%
\pgfpathcurveto{\pgfqpoint{0.502490in}{0.525253in}}{\pgfqpoint{0.491891in}{0.529644in}}{\pgfqpoint{0.480841in}{0.529644in}}%
\pgfpathcurveto{\pgfqpoint{0.469790in}{0.529644in}}{\pgfqpoint{0.459191in}{0.525253in}}{\pgfqpoint{0.451378in}{0.517440in}}%
\pgfpathcurveto{\pgfqpoint{0.443564in}{0.509626in}}{\pgfqpoint{0.439174in}{0.499027in}}{\pgfqpoint{0.439174in}{0.487977in}}%
\pgfpathcurveto{\pgfqpoint{0.439174in}{0.476927in}}{\pgfqpoint{0.443564in}{0.466328in}}{\pgfqpoint{0.451378in}{0.458514in}}%
\pgfpathcurveto{\pgfqpoint{0.459191in}{0.450700in}}{\pgfqpoint{0.469790in}{0.446310in}}{\pgfqpoint{0.480841in}{0.446310in}}%
\pgfpathclose%
\pgfusepath{stroke,fill}%
\end{pgfscope}%
\begin{pgfscope}%
\pgfpathrectangle{\pgfqpoint{0.375000in}{0.330000in}}{\pgfqpoint{2.325000in}{2.310000in}}%
\pgfusepath{clip}%
\pgfsetbuttcap%
\pgfsetroundjoin%
\definecolor{currentfill}{rgb}{0.000000,0.000000,0.000000}%
\pgfsetfillcolor{currentfill}%
\pgfsetlinewidth{1.003750pt}%
\definecolor{currentstroke}{rgb}{0.000000,0.000000,0.000000}%
\pgfsetstrokecolor{currentstroke}%
\pgfsetdash{}{0pt}%
\pgfpathmoveto{\pgfqpoint{0.480841in}{0.446310in}}%
\pgfpathcurveto{\pgfqpoint{0.491891in}{0.446310in}}{\pgfqpoint{0.502490in}{0.450700in}}{\pgfqpoint{0.510303in}{0.458514in}}%
\pgfpathcurveto{\pgfqpoint{0.518117in}{0.466328in}}{\pgfqpoint{0.522507in}{0.476927in}}{\pgfqpoint{0.522507in}{0.487977in}}%
\pgfpathcurveto{\pgfqpoint{0.522507in}{0.499027in}}{\pgfqpoint{0.518117in}{0.509626in}}{\pgfqpoint{0.510303in}{0.517440in}}%
\pgfpathcurveto{\pgfqpoint{0.502490in}{0.525253in}}{\pgfqpoint{0.491891in}{0.529644in}}{\pgfqpoint{0.480841in}{0.529644in}}%
\pgfpathcurveto{\pgfqpoint{0.469790in}{0.529644in}}{\pgfqpoint{0.459191in}{0.525253in}}{\pgfqpoint{0.451378in}{0.517440in}}%
\pgfpathcurveto{\pgfqpoint{0.443564in}{0.509626in}}{\pgfqpoint{0.439174in}{0.499027in}}{\pgfqpoint{0.439174in}{0.487977in}}%
\pgfpathcurveto{\pgfqpoint{0.439174in}{0.476927in}}{\pgfqpoint{0.443564in}{0.466328in}}{\pgfqpoint{0.451378in}{0.458514in}}%
\pgfpathcurveto{\pgfqpoint{0.459191in}{0.450700in}}{\pgfqpoint{0.469790in}{0.446310in}}{\pgfqpoint{0.480841in}{0.446310in}}%
\pgfpathclose%
\pgfusepath{stroke,fill}%
\end{pgfscope}%
\begin{pgfscope}%
\pgfpathrectangle{\pgfqpoint{0.375000in}{0.330000in}}{\pgfqpoint{2.325000in}{2.310000in}}%
\pgfusepath{clip}%
\pgfsetbuttcap%
\pgfsetroundjoin%
\definecolor{currentfill}{rgb}{0.000000,0.000000,0.000000}%
\pgfsetfillcolor{currentfill}%
\pgfsetlinewidth{1.003750pt}%
\definecolor{currentstroke}{rgb}{0.000000,0.000000,0.000000}%
\pgfsetstrokecolor{currentstroke}%
\pgfsetdash{}{0pt}%
\pgfpathmoveto{\pgfqpoint{0.480841in}{0.394279in}}%
\pgfpathcurveto{\pgfqpoint{0.491891in}{0.394279in}}{\pgfqpoint{0.502490in}{0.398670in}}{\pgfqpoint{0.510303in}{0.406483in}}%
\pgfpathcurveto{\pgfqpoint{0.518117in}{0.414297in}}{\pgfqpoint{0.522507in}{0.424896in}}{\pgfqpoint{0.522507in}{0.435946in}}%
\pgfpathcurveto{\pgfqpoint{0.522507in}{0.446996in}}{\pgfqpoint{0.518117in}{0.457595in}}{\pgfqpoint{0.510303in}{0.465409in}}%
\pgfpathcurveto{\pgfqpoint{0.502490in}{0.473222in}}{\pgfqpoint{0.491891in}{0.477613in}}{\pgfqpoint{0.480841in}{0.477613in}}%
\pgfpathcurveto{\pgfqpoint{0.469790in}{0.477613in}}{\pgfqpoint{0.459191in}{0.473222in}}{\pgfqpoint{0.451378in}{0.465409in}}%
\pgfpathcurveto{\pgfqpoint{0.443564in}{0.457595in}}{\pgfqpoint{0.439174in}{0.446996in}}{\pgfqpoint{0.439174in}{0.435946in}}%
\pgfpathcurveto{\pgfqpoint{0.439174in}{0.424896in}}{\pgfqpoint{0.443564in}{0.414297in}}{\pgfqpoint{0.451378in}{0.406483in}}%
\pgfpathcurveto{\pgfqpoint{0.459191in}{0.398670in}}{\pgfqpoint{0.469790in}{0.394279in}}{\pgfqpoint{0.480841in}{0.394279in}}%
\pgfpathclose%
\pgfusepath{stroke,fill}%
\end{pgfscope}%
\begin{pgfscope}%
\pgfpathrectangle{\pgfqpoint{0.375000in}{0.330000in}}{\pgfqpoint{2.325000in}{2.310000in}}%
\pgfusepath{clip}%
\pgfsetbuttcap%
\pgfsetroundjoin%
\definecolor{currentfill}{rgb}{0.000000,0.000000,0.000000}%
\pgfsetfillcolor{currentfill}%
\pgfsetlinewidth{1.003750pt}%
\definecolor{currentstroke}{rgb}{0.000000,0.000000,0.000000}%
\pgfsetstrokecolor{currentstroke}%
\pgfsetdash{}{0pt}%
\pgfpathmoveto{\pgfqpoint{0.480841in}{0.446310in}}%
\pgfpathcurveto{\pgfqpoint{0.491891in}{0.446310in}}{\pgfqpoint{0.502490in}{0.450700in}}{\pgfqpoint{0.510303in}{0.458514in}}%
\pgfpathcurveto{\pgfqpoint{0.518117in}{0.466328in}}{\pgfqpoint{0.522507in}{0.476927in}}{\pgfqpoint{0.522507in}{0.487977in}}%
\pgfpathcurveto{\pgfqpoint{0.522507in}{0.499027in}}{\pgfqpoint{0.518117in}{0.509626in}}{\pgfqpoint{0.510303in}{0.517440in}}%
\pgfpathcurveto{\pgfqpoint{0.502490in}{0.525253in}}{\pgfqpoint{0.491891in}{0.529644in}}{\pgfqpoint{0.480841in}{0.529644in}}%
\pgfpathcurveto{\pgfqpoint{0.469790in}{0.529644in}}{\pgfqpoint{0.459191in}{0.525253in}}{\pgfqpoint{0.451378in}{0.517440in}}%
\pgfpathcurveto{\pgfqpoint{0.443564in}{0.509626in}}{\pgfqpoint{0.439174in}{0.499027in}}{\pgfqpoint{0.439174in}{0.487977in}}%
\pgfpathcurveto{\pgfqpoint{0.439174in}{0.476927in}}{\pgfqpoint{0.443564in}{0.466328in}}{\pgfqpoint{0.451378in}{0.458514in}}%
\pgfpathcurveto{\pgfqpoint{0.459191in}{0.450700in}}{\pgfqpoint{0.469790in}{0.446310in}}{\pgfqpoint{0.480841in}{0.446310in}}%
\pgfpathclose%
\pgfusepath{stroke,fill}%
\end{pgfscope}%
\begin{pgfscope}%
\pgfpathrectangle{\pgfqpoint{0.375000in}{0.330000in}}{\pgfqpoint{2.325000in}{2.310000in}}%
\pgfusepath{clip}%
\pgfsetbuttcap%
\pgfsetroundjoin%
\definecolor{currentfill}{rgb}{0.000000,0.000000,0.000000}%
\pgfsetfillcolor{currentfill}%
\pgfsetlinewidth{1.003750pt}%
\definecolor{currentstroke}{rgb}{0.000000,0.000000,0.000000}%
\pgfsetstrokecolor{currentstroke}%
\pgfsetdash{}{0pt}%
\pgfpathmoveto{\pgfqpoint{0.480841in}{0.446310in}}%
\pgfpathcurveto{\pgfqpoint{0.491891in}{0.446310in}}{\pgfqpoint{0.502490in}{0.450700in}}{\pgfqpoint{0.510303in}{0.458514in}}%
\pgfpathcurveto{\pgfqpoint{0.518117in}{0.466328in}}{\pgfqpoint{0.522507in}{0.476927in}}{\pgfqpoint{0.522507in}{0.487977in}}%
\pgfpathcurveto{\pgfqpoint{0.522507in}{0.499027in}}{\pgfqpoint{0.518117in}{0.509626in}}{\pgfqpoint{0.510303in}{0.517440in}}%
\pgfpathcurveto{\pgfqpoint{0.502490in}{0.525253in}}{\pgfqpoint{0.491891in}{0.529644in}}{\pgfqpoint{0.480841in}{0.529644in}}%
\pgfpathcurveto{\pgfqpoint{0.469790in}{0.529644in}}{\pgfqpoint{0.459191in}{0.525253in}}{\pgfqpoint{0.451378in}{0.517440in}}%
\pgfpathcurveto{\pgfqpoint{0.443564in}{0.509626in}}{\pgfqpoint{0.439174in}{0.499027in}}{\pgfqpoint{0.439174in}{0.487977in}}%
\pgfpathcurveto{\pgfqpoint{0.439174in}{0.476927in}}{\pgfqpoint{0.443564in}{0.466328in}}{\pgfqpoint{0.451378in}{0.458514in}}%
\pgfpathcurveto{\pgfqpoint{0.459191in}{0.450700in}}{\pgfqpoint{0.469790in}{0.446310in}}{\pgfqpoint{0.480841in}{0.446310in}}%
\pgfpathclose%
\pgfusepath{stroke,fill}%
\end{pgfscope}%
\begin{pgfscope}%
\pgfpathrectangle{\pgfqpoint{0.375000in}{0.330000in}}{\pgfqpoint{2.325000in}{2.310000in}}%
\pgfusepath{clip}%
\pgfsetbuttcap%
\pgfsetroundjoin%
\definecolor{currentfill}{rgb}{0.000000,0.000000,0.000000}%
\pgfsetfillcolor{currentfill}%
\pgfsetlinewidth{1.003750pt}%
\definecolor{currentstroke}{rgb}{0.000000,0.000000,0.000000}%
\pgfsetstrokecolor{currentstroke}%
\pgfsetdash{}{0pt}%
\pgfpathmoveto{\pgfqpoint{0.480841in}{0.394279in}}%
\pgfpathcurveto{\pgfqpoint{0.491891in}{0.394279in}}{\pgfqpoint{0.502490in}{0.398670in}}{\pgfqpoint{0.510303in}{0.406483in}}%
\pgfpathcurveto{\pgfqpoint{0.518117in}{0.414297in}}{\pgfqpoint{0.522507in}{0.424896in}}{\pgfqpoint{0.522507in}{0.435946in}}%
\pgfpathcurveto{\pgfqpoint{0.522507in}{0.446996in}}{\pgfqpoint{0.518117in}{0.457595in}}{\pgfqpoint{0.510303in}{0.465409in}}%
\pgfpathcurveto{\pgfqpoint{0.502490in}{0.473222in}}{\pgfqpoint{0.491891in}{0.477613in}}{\pgfqpoint{0.480841in}{0.477613in}}%
\pgfpathcurveto{\pgfqpoint{0.469790in}{0.477613in}}{\pgfqpoint{0.459191in}{0.473222in}}{\pgfqpoint{0.451378in}{0.465409in}}%
\pgfpathcurveto{\pgfqpoint{0.443564in}{0.457595in}}{\pgfqpoint{0.439174in}{0.446996in}}{\pgfqpoint{0.439174in}{0.435946in}}%
\pgfpathcurveto{\pgfqpoint{0.439174in}{0.424896in}}{\pgfqpoint{0.443564in}{0.414297in}}{\pgfqpoint{0.451378in}{0.406483in}}%
\pgfpathcurveto{\pgfqpoint{0.459191in}{0.398670in}}{\pgfqpoint{0.469790in}{0.394279in}}{\pgfqpoint{0.480841in}{0.394279in}}%
\pgfpathclose%
\pgfusepath{stroke,fill}%
\end{pgfscope}%
\begin{pgfscope}%
\pgfpathrectangle{\pgfqpoint{0.375000in}{0.330000in}}{\pgfqpoint{2.325000in}{2.310000in}}%
\pgfusepath{clip}%
\pgfsetbuttcap%
\pgfsetroundjoin%
\definecolor{currentfill}{rgb}{0.000000,0.000000,0.000000}%
\pgfsetfillcolor{currentfill}%
\pgfsetlinewidth{1.003750pt}%
\definecolor{currentstroke}{rgb}{0.000000,0.000000,0.000000}%
\pgfsetstrokecolor{currentstroke}%
\pgfsetdash{}{0pt}%
\pgfpathmoveto{\pgfqpoint{0.480841in}{0.446310in}}%
\pgfpathcurveto{\pgfqpoint{0.491891in}{0.446310in}}{\pgfqpoint{0.502490in}{0.450700in}}{\pgfqpoint{0.510303in}{0.458514in}}%
\pgfpathcurveto{\pgfqpoint{0.518117in}{0.466328in}}{\pgfqpoint{0.522507in}{0.476927in}}{\pgfqpoint{0.522507in}{0.487977in}}%
\pgfpathcurveto{\pgfqpoint{0.522507in}{0.499027in}}{\pgfqpoint{0.518117in}{0.509626in}}{\pgfqpoint{0.510303in}{0.517440in}}%
\pgfpathcurveto{\pgfqpoint{0.502490in}{0.525253in}}{\pgfqpoint{0.491891in}{0.529644in}}{\pgfqpoint{0.480841in}{0.529644in}}%
\pgfpathcurveto{\pgfqpoint{0.469790in}{0.529644in}}{\pgfqpoint{0.459191in}{0.525253in}}{\pgfqpoint{0.451378in}{0.517440in}}%
\pgfpathcurveto{\pgfqpoint{0.443564in}{0.509626in}}{\pgfqpoint{0.439174in}{0.499027in}}{\pgfqpoint{0.439174in}{0.487977in}}%
\pgfpathcurveto{\pgfqpoint{0.439174in}{0.476927in}}{\pgfqpoint{0.443564in}{0.466328in}}{\pgfqpoint{0.451378in}{0.458514in}}%
\pgfpathcurveto{\pgfqpoint{0.459191in}{0.450700in}}{\pgfqpoint{0.469790in}{0.446310in}}{\pgfqpoint{0.480841in}{0.446310in}}%
\pgfpathclose%
\pgfusepath{stroke,fill}%
\end{pgfscope}%
\begin{pgfscope}%
\pgfpathrectangle{\pgfqpoint{0.375000in}{0.330000in}}{\pgfqpoint{2.325000in}{2.310000in}}%
\pgfusepath{clip}%
\pgfsetbuttcap%
\pgfsetroundjoin%
\definecolor{currentfill}{rgb}{0.000000,0.000000,0.000000}%
\pgfsetfillcolor{currentfill}%
\pgfsetlinewidth{1.003750pt}%
\definecolor{currentstroke}{rgb}{0.000000,0.000000,0.000000}%
\pgfsetstrokecolor{currentstroke}%
\pgfsetdash{}{0pt}%
\pgfpathmoveto{\pgfqpoint{0.480841in}{0.446310in}}%
\pgfpathcurveto{\pgfqpoint{0.491891in}{0.446310in}}{\pgfqpoint{0.502490in}{0.450700in}}{\pgfqpoint{0.510303in}{0.458514in}}%
\pgfpathcurveto{\pgfqpoint{0.518117in}{0.466328in}}{\pgfqpoint{0.522507in}{0.476927in}}{\pgfqpoint{0.522507in}{0.487977in}}%
\pgfpathcurveto{\pgfqpoint{0.522507in}{0.499027in}}{\pgfqpoint{0.518117in}{0.509626in}}{\pgfqpoint{0.510303in}{0.517440in}}%
\pgfpathcurveto{\pgfqpoint{0.502490in}{0.525253in}}{\pgfqpoint{0.491891in}{0.529644in}}{\pgfqpoint{0.480841in}{0.529644in}}%
\pgfpathcurveto{\pgfqpoint{0.469790in}{0.529644in}}{\pgfqpoint{0.459191in}{0.525253in}}{\pgfqpoint{0.451378in}{0.517440in}}%
\pgfpathcurveto{\pgfqpoint{0.443564in}{0.509626in}}{\pgfqpoint{0.439174in}{0.499027in}}{\pgfqpoint{0.439174in}{0.487977in}}%
\pgfpathcurveto{\pgfqpoint{0.439174in}{0.476927in}}{\pgfqpoint{0.443564in}{0.466328in}}{\pgfqpoint{0.451378in}{0.458514in}}%
\pgfpathcurveto{\pgfqpoint{0.459191in}{0.450700in}}{\pgfqpoint{0.469790in}{0.446310in}}{\pgfqpoint{0.480841in}{0.446310in}}%
\pgfpathclose%
\pgfusepath{stroke,fill}%
\end{pgfscope}%
\begin{pgfscope}%
\pgfpathrectangle{\pgfqpoint{0.375000in}{0.330000in}}{\pgfqpoint{2.325000in}{2.310000in}}%
\pgfusepath{clip}%
\pgfsetbuttcap%
\pgfsetroundjoin%
\definecolor{currentfill}{rgb}{0.000000,0.000000,0.000000}%
\pgfsetfillcolor{currentfill}%
\pgfsetlinewidth{1.003750pt}%
\definecolor{currentstroke}{rgb}{0.000000,0.000000,0.000000}%
\pgfsetstrokecolor{currentstroke}%
\pgfsetdash{}{0pt}%
\pgfpathmoveto{\pgfqpoint{1.005694in}{0.706464in}}%
\pgfpathcurveto{\pgfqpoint{1.016744in}{0.706464in}}{\pgfqpoint{1.027343in}{0.710855in}}{\pgfqpoint{1.035157in}{0.718668in}}%
\pgfpathcurveto{\pgfqpoint{1.042971in}{0.726482in}}{\pgfqpoint{1.047361in}{0.737081in}}{\pgfqpoint{1.047361in}{0.748131in}}%
\pgfpathcurveto{\pgfqpoint{1.047361in}{0.759181in}}{\pgfqpoint{1.042971in}{0.769780in}}{\pgfqpoint{1.035157in}{0.777594in}}%
\pgfpathcurveto{\pgfqpoint{1.027343in}{0.785407in}}{\pgfqpoint{1.016744in}{0.789798in}}{\pgfqpoint{1.005694in}{0.789798in}}%
\pgfpathcurveto{\pgfqpoint{0.994644in}{0.789798in}}{\pgfqpoint{0.984045in}{0.785407in}}{\pgfqpoint{0.976232in}{0.777594in}}%
\pgfpathcurveto{\pgfqpoint{0.968418in}{0.769780in}}{\pgfqpoint{0.964028in}{0.759181in}}{\pgfqpoint{0.964028in}{0.748131in}}%
\pgfpathcurveto{\pgfqpoint{0.964028in}{0.737081in}}{\pgfqpoint{0.968418in}{0.726482in}}{\pgfqpoint{0.976232in}{0.718668in}}%
\pgfpathcurveto{\pgfqpoint{0.984045in}{0.710855in}}{\pgfqpoint{0.994644in}{0.706464in}}{\pgfqpoint{1.005694in}{0.706464in}}%
\pgfpathclose%
\pgfusepath{stroke,fill}%
\end{pgfscope}%
\begin{pgfscope}%
\pgfpathrectangle{\pgfqpoint{0.375000in}{0.330000in}}{\pgfqpoint{2.325000in}{2.310000in}}%
\pgfusepath{clip}%
\pgfsetbuttcap%
\pgfsetroundjoin%
\definecolor{currentfill}{rgb}{0.000000,0.000000,0.000000}%
\pgfsetfillcolor{currentfill}%
\pgfsetlinewidth{1.003750pt}%
\definecolor{currentstroke}{rgb}{0.000000,0.000000,0.000000}%
\pgfsetstrokecolor{currentstroke}%
\pgfsetdash{}{0pt}%
\pgfpathmoveto{\pgfqpoint{1.005694in}{0.758495in}}%
\pgfpathcurveto{\pgfqpoint{1.016744in}{0.758495in}}{\pgfqpoint{1.027343in}{0.762885in}}{\pgfqpoint{1.035157in}{0.770699in}}%
\pgfpathcurveto{\pgfqpoint{1.042971in}{0.778513in}}{\pgfqpoint{1.047361in}{0.789112in}}{\pgfqpoint{1.047361in}{0.800162in}}%
\pgfpathcurveto{\pgfqpoint{1.047361in}{0.811212in}}{\pgfqpoint{1.042971in}{0.821811in}}{\pgfqpoint{1.035157in}{0.829625in}}%
\pgfpathcurveto{\pgfqpoint{1.027343in}{0.837438in}}{\pgfqpoint{1.016744in}{0.841828in}}{\pgfqpoint{1.005694in}{0.841828in}}%
\pgfpathcurveto{\pgfqpoint{0.994644in}{0.841828in}}{\pgfqpoint{0.984045in}{0.837438in}}{\pgfqpoint{0.976232in}{0.829625in}}%
\pgfpathcurveto{\pgfqpoint{0.968418in}{0.821811in}}{\pgfqpoint{0.964028in}{0.811212in}}{\pgfqpoint{0.964028in}{0.800162in}}%
\pgfpathcurveto{\pgfqpoint{0.964028in}{0.789112in}}{\pgfqpoint{0.968418in}{0.778513in}}{\pgfqpoint{0.976232in}{0.770699in}}%
\pgfpathcurveto{\pgfqpoint{0.984045in}{0.762885in}}{\pgfqpoint{0.994644in}{0.758495in}}{\pgfqpoint{1.005694in}{0.758495in}}%
\pgfpathclose%
\pgfusepath{stroke,fill}%
\end{pgfscope}%
\begin{pgfscope}%
\pgfpathrectangle{\pgfqpoint{0.375000in}{0.330000in}}{\pgfqpoint{2.325000in}{2.310000in}}%
\pgfusepath{clip}%
\pgfsetbuttcap%
\pgfsetroundjoin%
\definecolor{currentfill}{rgb}{0.000000,0.000000,0.000000}%
\pgfsetfillcolor{currentfill}%
\pgfsetlinewidth{1.003750pt}%
\definecolor{currentstroke}{rgb}{0.000000,0.000000,0.000000}%
\pgfsetstrokecolor{currentstroke}%
\pgfsetdash{}{0pt}%
\pgfpathmoveto{\pgfqpoint{1.005694in}{0.706464in}}%
\pgfpathcurveto{\pgfqpoint{1.016744in}{0.706464in}}{\pgfqpoint{1.027343in}{0.710855in}}{\pgfqpoint{1.035157in}{0.718668in}}%
\pgfpathcurveto{\pgfqpoint{1.042971in}{0.726482in}}{\pgfqpoint{1.047361in}{0.737081in}}{\pgfqpoint{1.047361in}{0.748131in}}%
\pgfpathcurveto{\pgfqpoint{1.047361in}{0.759181in}}{\pgfqpoint{1.042971in}{0.769780in}}{\pgfqpoint{1.035157in}{0.777594in}}%
\pgfpathcurveto{\pgfqpoint{1.027343in}{0.785407in}}{\pgfqpoint{1.016744in}{0.789798in}}{\pgfqpoint{1.005694in}{0.789798in}}%
\pgfpathcurveto{\pgfqpoint{0.994644in}{0.789798in}}{\pgfqpoint{0.984045in}{0.785407in}}{\pgfqpoint{0.976232in}{0.777594in}}%
\pgfpathcurveto{\pgfqpoint{0.968418in}{0.769780in}}{\pgfqpoint{0.964028in}{0.759181in}}{\pgfqpoint{0.964028in}{0.748131in}}%
\pgfpathcurveto{\pgfqpoint{0.964028in}{0.737081in}}{\pgfqpoint{0.968418in}{0.726482in}}{\pgfqpoint{0.976232in}{0.718668in}}%
\pgfpathcurveto{\pgfqpoint{0.984045in}{0.710855in}}{\pgfqpoint{0.994644in}{0.706464in}}{\pgfqpoint{1.005694in}{0.706464in}}%
\pgfpathclose%
\pgfusepath{stroke,fill}%
\end{pgfscope}%
\begin{pgfscope}%
\pgfpathrectangle{\pgfqpoint{0.375000in}{0.330000in}}{\pgfqpoint{2.325000in}{2.310000in}}%
\pgfusepath{clip}%
\pgfsetbuttcap%
\pgfsetroundjoin%
\definecolor{currentfill}{rgb}{0.000000,0.000000,0.000000}%
\pgfsetfillcolor{currentfill}%
\pgfsetlinewidth{1.003750pt}%
\definecolor{currentstroke}{rgb}{0.000000,0.000000,0.000000}%
\pgfsetstrokecolor{currentstroke}%
\pgfsetdash{}{0pt}%
\pgfpathmoveto{\pgfqpoint{1.005694in}{0.706464in}}%
\pgfpathcurveto{\pgfqpoint{1.016744in}{0.706464in}}{\pgfqpoint{1.027343in}{0.710855in}}{\pgfqpoint{1.035157in}{0.718668in}}%
\pgfpathcurveto{\pgfqpoint{1.042971in}{0.726482in}}{\pgfqpoint{1.047361in}{0.737081in}}{\pgfqpoint{1.047361in}{0.748131in}}%
\pgfpathcurveto{\pgfqpoint{1.047361in}{0.759181in}}{\pgfqpoint{1.042971in}{0.769780in}}{\pgfqpoint{1.035157in}{0.777594in}}%
\pgfpathcurveto{\pgfqpoint{1.027343in}{0.785407in}}{\pgfqpoint{1.016744in}{0.789798in}}{\pgfqpoint{1.005694in}{0.789798in}}%
\pgfpathcurveto{\pgfqpoint{0.994644in}{0.789798in}}{\pgfqpoint{0.984045in}{0.785407in}}{\pgfqpoint{0.976232in}{0.777594in}}%
\pgfpathcurveto{\pgfqpoint{0.968418in}{0.769780in}}{\pgfqpoint{0.964028in}{0.759181in}}{\pgfqpoint{0.964028in}{0.748131in}}%
\pgfpathcurveto{\pgfqpoint{0.964028in}{0.737081in}}{\pgfqpoint{0.968418in}{0.726482in}}{\pgfqpoint{0.976232in}{0.718668in}}%
\pgfpathcurveto{\pgfqpoint{0.984045in}{0.710855in}}{\pgfqpoint{0.994644in}{0.706464in}}{\pgfqpoint{1.005694in}{0.706464in}}%
\pgfpathclose%
\pgfusepath{stroke,fill}%
\end{pgfscope}%
\begin{pgfscope}%
\pgfpathrectangle{\pgfqpoint{0.375000in}{0.330000in}}{\pgfqpoint{2.325000in}{2.310000in}}%
\pgfusepath{clip}%
\pgfsetbuttcap%
\pgfsetroundjoin%
\definecolor{currentfill}{rgb}{0.000000,0.000000,0.000000}%
\pgfsetfillcolor{currentfill}%
\pgfsetlinewidth{1.003750pt}%
\definecolor{currentstroke}{rgb}{0.000000,0.000000,0.000000}%
\pgfsetstrokecolor{currentstroke}%
\pgfsetdash{}{0pt}%
\pgfpathmoveto{\pgfqpoint{1.005694in}{0.758495in}}%
\pgfpathcurveto{\pgfqpoint{1.016744in}{0.758495in}}{\pgfqpoint{1.027343in}{0.762885in}}{\pgfqpoint{1.035157in}{0.770699in}}%
\pgfpathcurveto{\pgfqpoint{1.042971in}{0.778513in}}{\pgfqpoint{1.047361in}{0.789112in}}{\pgfqpoint{1.047361in}{0.800162in}}%
\pgfpathcurveto{\pgfqpoint{1.047361in}{0.811212in}}{\pgfqpoint{1.042971in}{0.821811in}}{\pgfqpoint{1.035157in}{0.829625in}}%
\pgfpathcurveto{\pgfqpoint{1.027343in}{0.837438in}}{\pgfqpoint{1.016744in}{0.841828in}}{\pgfqpoint{1.005694in}{0.841828in}}%
\pgfpathcurveto{\pgfqpoint{0.994644in}{0.841828in}}{\pgfqpoint{0.984045in}{0.837438in}}{\pgfqpoint{0.976232in}{0.829625in}}%
\pgfpathcurveto{\pgfqpoint{0.968418in}{0.821811in}}{\pgfqpoint{0.964028in}{0.811212in}}{\pgfqpoint{0.964028in}{0.800162in}}%
\pgfpathcurveto{\pgfqpoint{0.964028in}{0.789112in}}{\pgfqpoint{0.968418in}{0.778513in}}{\pgfqpoint{0.976232in}{0.770699in}}%
\pgfpathcurveto{\pgfqpoint{0.984045in}{0.762885in}}{\pgfqpoint{0.994644in}{0.758495in}}{\pgfqpoint{1.005694in}{0.758495in}}%
\pgfpathclose%
\pgfusepath{stroke,fill}%
\end{pgfscope}%
\begin{pgfscope}%
\pgfpathrectangle{\pgfqpoint{0.375000in}{0.330000in}}{\pgfqpoint{2.325000in}{2.310000in}}%
\pgfusepath{clip}%
\pgfsetbuttcap%
\pgfsetroundjoin%
\definecolor{currentfill}{rgb}{0.000000,0.000000,0.000000}%
\pgfsetfillcolor{currentfill}%
\pgfsetlinewidth{1.003750pt}%
\definecolor{currentstroke}{rgb}{0.000000,0.000000,0.000000}%
\pgfsetstrokecolor{currentstroke}%
\pgfsetdash{}{0pt}%
\pgfpathmoveto{\pgfqpoint{1.005694in}{0.758495in}}%
\pgfpathcurveto{\pgfqpoint{1.016744in}{0.758495in}}{\pgfqpoint{1.027343in}{0.762885in}}{\pgfqpoint{1.035157in}{0.770699in}}%
\pgfpathcurveto{\pgfqpoint{1.042971in}{0.778513in}}{\pgfqpoint{1.047361in}{0.789112in}}{\pgfqpoint{1.047361in}{0.800162in}}%
\pgfpathcurveto{\pgfqpoint{1.047361in}{0.811212in}}{\pgfqpoint{1.042971in}{0.821811in}}{\pgfqpoint{1.035157in}{0.829625in}}%
\pgfpathcurveto{\pgfqpoint{1.027343in}{0.837438in}}{\pgfqpoint{1.016744in}{0.841828in}}{\pgfqpoint{1.005694in}{0.841828in}}%
\pgfpathcurveto{\pgfqpoint{0.994644in}{0.841828in}}{\pgfqpoint{0.984045in}{0.837438in}}{\pgfqpoint{0.976232in}{0.829625in}}%
\pgfpathcurveto{\pgfqpoint{0.968418in}{0.821811in}}{\pgfqpoint{0.964028in}{0.811212in}}{\pgfqpoint{0.964028in}{0.800162in}}%
\pgfpathcurveto{\pgfqpoint{0.964028in}{0.789112in}}{\pgfqpoint{0.968418in}{0.778513in}}{\pgfqpoint{0.976232in}{0.770699in}}%
\pgfpathcurveto{\pgfqpoint{0.984045in}{0.762885in}}{\pgfqpoint{0.994644in}{0.758495in}}{\pgfqpoint{1.005694in}{0.758495in}}%
\pgfpathclose%
\pgfusepath{stroke,fill}%
\end{pgfscope}%
\begin{pgfscope}%
\pgfpathrectangle{\pgfqpoint{0.375000in}{0.330000in}}{\pgfqpoint{2.325000in}{2.310000in}}%
\pgfusepath{clip}%
\pgfsetbuttcap%
\pgfsetroundjoin%
\definecolor{currentfill}{rgb}{0.000000,0.000000,0.000000}%
\pgfsetfillcolor{currentfill}%
\pgfsetlinewidth{1.003750pt}%
\definecolor{currentstroke}{rgb}{0.000000,0.000000,0.000000}%
\pgfsetstrokecolor{currentstroke}%
\pgfsetdash{}{0pt}%
\pgfpathmoveto{\pgfqpoint{1.005694in}{0.706464in}}%
\pgfpathcurveto{\pgfqpoint{1.016744in}{0.706464in}}{\pgfqpoint{1.027343in}{0.710855in}}{\pgfqpoint{1.035157in}{0.718668in}}%
\pgfpathcurveto{\pgfqpoint{1.042971in}{0.726482in}}{\pgfqpoint{1.047361in}{0.737081in}}{\pgfqpoint{1.047361in}{0.748131in}}%
\pgfpathcurveto{\pgfqpoint{1.047361in}{0.759181in}}{\pgfqpoint{1.042971in}{0.769780in}}{\pgfqpoint{1.035157in}{0.777594in}}%
\pgfpathcurveto{\pgfqpoint{1.027343in}{0.785407in}}{\pgfqpoint{1.016744in}{0.789798in}}{\pgfqpoint{1.005694in}{0.789798in}}%
\pgfpathcurveto{\pgfqpoint{0.994644in}{0.789798in}}{\pgfqpoint{0.984045in}{0.785407in}}{\pgfqpoint{0.976232in}{0.777594in}}%
\pgfpathcurveto{\pgfqpoint{0.968418in}{0.769780in}}{\pgfqpoint{0.964028in}{0.759181in}}{\pgfqpoint{0.964028in}{0.748131in}}%
\pgfpathcurveto{\pgfqpoint{0.964028in}{0.737081in}}{\pgfqpoint{0.968418in}{0.726482in}}{\pgfqpoint{0.976232in}{0.718668in}}%
\pgfpathcurveto{\pgfqpoint{0.984045in}{0.710855in}}{\pgfqpoint{0.994644in}{0.706464in}}{\pgfqpoint{1.005694in}{0.706464in}}%
\pgfpathclose%
\pgfusepath{stroke,fill}%
\end{pgfscope}%
\begin{pgfscope}%
\pgfpathrectangle{\pgfqpoint{0.375000in}{0.330000in}}{\pgfqpoint{2.325000in}{2.310000in}}%
\pgfusepath{clip}%
\pgfsetbuttcap%
\pgfsetroundjoin%
\definecolor{currentfill}{rgb}{0.000000,0.000000,0.000000}%
\pgfsetfillcolor{currentfill}%
\pgfsetlinewidth{1.003750pt}%
\definecolor{currentstroke}{rgb}{0.000000,0.000000,0.000000}%
\pgfsetstrokecolor{currentstroke}%
\pgfsetdash{}{0pt}%
\pgfpathmoveto{\pgfqpoint{1.005694in}{0.654433in}}%
\pgfpathcurveto{\pgfqpoint{1.016744in}{0.654433in}}{\pgfqpoint{1.027343in}{0.658824in}}{\pgfqpoint{1.035157in}{0.666637in}}%
\pgfpathcurveto{\pgfqpoint{1.042971in}{0.674451in}}{\pgfqpoint{1.047361in}{0.685050in}}{\pgfqpoint{1.047361in}{0.696100in}}%
\pgfpathcurveto{\pgfqpoint{1.047361in}{0.707150in}}{\pgfqpoint{1.042971in}{0.717749in}}{\pgfqpoint{1.035157in}{0.725563in}}%
\pgfpathcurveto{\pgfqpoint{1.027343in}{0.733377in}}{\pgfqpoint{1.016744in}{0.737767in}}{\pgfqpoint{1.005694in}{0.737767in}}%
\pgfpathcurveto{\pgfqpoint{0.994644in}{0.737767in}}{\pgfqpoint{0.984045in}{0.733377in}}{\pgfqpoint{0.976232in}{0.725563in}}%
\pgfpathcurveto{\pgfqpoint{0.968418in}{0.717749in}}{\pgfqpoint{0.964028in}{0.707150in}}{\pgfqpoint{0.964028in}{0.696100in}}%
\pgfpathcurveto{\pgfqpoint{0.964028in}{0.685050in}}{\pgfqpoint{0.968418in}{0.674451in}}{\pgfqpoint{0.976232in}{0.666637in}}%
\pgfpathcurveto{\pgfqpoint{0.984045in}{0.658824in}}{\pgfqpoint{0.994644in}{0.654433in}}{\pgfqpoint{1.005694in}{0.654433in}}%
\pgfpathclose%
\pgfusepath{stroke,fill}%
\end{pgfscope}%
\begin{pgfscope}%
\pgfpathrectangle{\pgfqpoint{0.375000in}{0.330000in}}{\pgfqpoint{2.325000in}{2.310000in}}%
\pgfusepath{clip}%
\pgfsetbuttcap%
\pgfsetroundjoin%
\definecolor{currentfill}{rgb}{0.000000,0.000000,0.000000}%
\pgfsetfillcolor{currentfill}%
\pgfsetlinewidth{1.003750pt}%
\definecolor{currentstroke}{rgb}{0.000000,0.000000,0.000000}%
\pgfsetstrokecolor{currentstroke}%
\pgfsetdash{}{0pt}%
\pgfpathmoveto{\pgfqpoint{1.005694in}{0.706464in}}%
\pgfpathcurveto{\pgfqpoint{1.016744in}{0.706464in}}{\pgfqpoint{1.027343in}{0.710855in}}{\pgfqpoint{1.035157in}{0.718668in}}%
\pgfpathcurveto{\pgfqpoint{1.042971in}{0.726482in}}{\pgfqpoint{1.047361in}{0.737081in}}{\pgfqpoint{1.047361in}{0.748131in}}%
\pgfpathcurveto{\pgfqpoint{1.047361in}{0.759181in}}{\pgfqpoint{1.042971in}{0.769780in}}{\pgfqpoint{1.035157in}{0.777594in}}%
\pgfpathcurveto{\pgfqpoint{1.027343in}{0.785407in}}{\pgfqpoint{1.016744in}{0.789798in}}{\pgfqpoint{1.005694in}{0.789798in}}%
\pgfpathcurveto{\pgfqpoint{0.994644in}{0.789798in}}{\pgfqpoint{0.984045in}{0.785407in}}{\pgfqpoint{0.976232in}{0.777594in}}%
\pgfpathcurveto{\pgfqpoint{0.968418in}{0.769780in}}{\pgfqpoint{0.964028in}{0.759181in}}{\pgfqpoint{0.964028in}{0.748131in}}%
\pgfpathcurveto{\pgfqpoint{0.964028in}{0.737081in}}{\pgfqpoint{0.968418in}{0.726482in}}{\pgfqpoint{0.976232in}{0.718668in}}%
\pgfpathcurveto{\pgfqpoint{0.984045in}{0.710855in}}{\pgfqpoint{0.994644in}{0.706464in}}{\pgfqpoint{1.005694in}{0.706464in}}%
\pgfpathclose%
\pgfusepath{stroke,fill}%
\end{pgfscope}%
\begin{pgfscope}%
\pgfpathrectangle{\pgfqpoint{0.375000in}{0.330000in}}{\pgfqpoint{2.325000in}{2.310000in}}%
\pgfusepath{clip}%
\pgfsetbuttcap%
\pgfsetroundjoin%
\definecolor{currentfill}{rgb}{0.000000,0.000000,0.000000}%
\pgfsetfillcolor{currentfill}%
\pgfsetlinewidth{1.003750pt}%
\definecolor{currentstroke}{rgb}{0.000000,0.000000,0.000000}%
\pgfsetstrokecolor{currentstroke}%
\pgfsetdash{}{0pt}%
\pgfpathmoveto{\pgfqpoint{1.005694in}{0.706464in}}%
\pgfpathcurveto{\pgfqpoint{1.016744in}{0.706464in}}{\pgfqpoint{1.027343in}{0.710855in}}{\pgfqpoint{1.035157in}{0.718668in}}%
\pgfpathcurveto{\pgfqpoint{1.042971in}{0.726482in}}{\pgfqpoint{1.047361in}{0.737081in}}{\pgfqpoint{1.047361in}{0.748131in}}%
\pgfpathcurveto{\pgfqpoint{1.047361in}{0.759181in}}{\pgfqpoint{1.042971in}{0.769780in}}{\pgfqpoint{1.035157in}{0.777594in}}%
\pgfpathcurveto{\pgfqpoint{1.027343in}{0.785407in}}{\pgfqpoint{1.016744in}{0.789798in}}{\pgfqpoint{1.005694in}{0.789798in}}%
\pgfpathcurveto{\pgfqpoint{0.994644in}{0.789798in}}{\pgfqpoint{0.984045in}{0.785407in}}{\pgfqpoint{0.976232in}{0.777594in}}%
\pgfpathcurveto{\pgfqpoint{0.968418in}{0.769780in}}{\pgfqpoint{0.964028in}{0.759181in}}{\pgfqpoint{0.964028in}{0.748131in}}%
\pgfpathcurveto{\pgfqpoint{0.964028in}{0.737081in}}{\pgfqpoint{0.968418in}{0.726482in}}{\pgfqpoint{0.976232in}{0.718668in}}%
\pgfpathcurveto{\pgfqpoint{0.984045in}{0.710855in}}{\pgfqpoint{0.994644in}{0.706464in}}{\pgfqpoint{1.005694in}{0.706464in}}%
\pgfpathclose%
\pgfusepath{stroke,fill}%
\end{pgfscope}%
\begin{pgfscope}%
\pgfpathrectangle{\pgfqpoint{0.375000in}{0.330000in}}{\pgfqpoint{2.325000in}{2.310000in}}%
\pgfusepath{clip}%
\pgfsetbuttcap%
\pgfsetroundjoin%
\definecolor{currentfill}{rgb}{0.000000,0.000000,0.000000}%
\pgfsetfillcolor{currentfill}%
\pgfsetlinewidth{1.003750pt}%
\definecolor{currentstroke}{rgb}{0.000000,0.000000,0.000000}%
\pgfsetstrokecolor{currentstroke}%
\pgfsetdash{}{0pt}%
\pgfpathmoveto{\pgfqpoint{1.005694in}{0.758495in}}%
\pgfpathcurveto{\pgfqpoint{1.016744in}{0.758495in}}{\pgfqpoint{1.027343in}{0.762885in}}{\pgfqpoint{1.035157in}{0.770699in}}%
\pgfpathcurveto{\pgfqpoint{1.042971in}{0.778513in}}{\pgfqpoint{1.047361in}{0.789112in}}{\pgfqpoint{1.047361in}{0.800162in}}%
\pgfpathcurveto{\pgfqpoint{1.047361in}{0.811212in}}{\pgfqpoint{1.042971in}{0.821811in}}{\pgfqpoint{1.035157in}{0.829625in}}%
\pgfpathcurveto{\pgfqpoint{1.027343in}{0.837438in}}{\pgfqpoint{1.016744in}{0.841828in}}{\pgfqpoint{1.005694in}{0.841828in}}%
\pgfpathcurveto{\pgfqpoint{0.994644in}{0.841828in}}{\pgfqpoint{0.984045in}{0.837438in}}{\pgfqpoint{0.976232in}{0.829625in}}%
\pgfpathcurveto{\pgfqpoint{0.968418in}{0.821811in}}{\pgfqpoint{0.964028in}{0.811212in}}{\pgfqpoint{0.964028in}{0.800162in}}%
\pgfpathcurveto{\pgfqpoint{0.964028in}{0.789112in}}{\pgfqpoint{0.968418in}{0.778513in}}{\pgfqpoint{0.976232in}{0.770699in}}%
\pgfpathcurveto{\pgfqpoint{0.984045in}{0.762885in}}{\pgfqpoint{0.994644in}{0.758495in}}{\pgfqpoint{1.005694in}{0.758495in}}%
\pgfpathclose%
\pgfusepath{stroke,fill}%
\end{pgfscope}%
\begin{pgfscope}%
\pgfpathrectangle{\pgfqpoint{0.375000in}{0.330000in}}{\pgfqpoint{2.325000in}{2.310000in}}%
\pgfusepath{clip}%
\pgfsetbuttcap%
\pgfsetroundjoin%
\definecolor{currentfill}{rgb}{0.000000,0.000000,0.000000}%
\pgfsetfillcolor{currentfill}%
\pgfsetlinewidth{1.003750pt}%
\definecolor{currentstroke}{rgb}{0.000000,0.000000,0.000000}%
\pgfsetstrokecolor{currentstroke}%
\pgfsetdash{}{0pt}%
\pgfpathmoveto{\pgfqpoint{1.005694in}{0.758495in}}%
\pgfpathcurveto{\pgfqpoint{1.016744in}{0.758495in}}{\pgfqpoint{1.027343in}{0.762885in}}{\pgfqpoint{1.035157in}{0.770699in}}%
\pgfpathcurveto{\pgfqpoint{1.042971in}{0.778513in}}{\pgfqpoint{1.047361in}{0.789112in}}{\pgfqpoint{1.047361in}{0.800162in}}%
\pgfpathcurveto{\pgfqpoint{1.047361in}{0.811212in}}{\pgfqpoint{1.042971in}{0.821811in}}{\pgfqpoint{1.035157in}{0.829625in}}%
\pgfpathcurveto{\pgfqpoint{1.027343in}{0.837438in}}{\pgfqpoint{1.016744in}{0.841828in}}{\pgfqpoint{1.005694in}{0.841828in}}%
\pgfpathcurveto{\pgfqpoint{0.994644in}{0.841828in}}{\pgfqpoint{0.984045in}{0.837438in}}{\pgfqpoint{0.976232in}{0.829625in}}%
\pgfpathcurveto{\pgfqpoint{0.968418in}{0.821811in}}{\pgfqpoint{0.964028in}{0.811212in}}{\pgfqpoint{0.964028in}{0.800162in}}%
\pgfpathcurveto{\pgfqpoint{0.964028in}{0.789112in}}{\pgfqpoint{0.968418in}{0.778513in}}{\pgfqpoint{0.976232in}{0.770699in}}%
\pgfpathcurveto{\pgfqpoint{0.984045in}{0.762885in}}{\pgfqpoint{0.994644in}{0.758495in}}{\pgfqpoint{1.005694in}{0.758495in}}%
\pgfpathclose%
\pgfusepath{stroke,fill}%
\end{pgfscope}%
\begin{pgfscope}%
\pgfpathrectangle{\pgfqpoint{0.375000in}{0.330000in}}{\pgfqpoint{2.325000in}{2.310000in}}%
\pgfusepath{clip}%
\pgfsetbuttcap%
\pgfsetroundjoin%
\definecolor{currentfill}{rgb}{0.000000,0.000000,0.000000}%
\pgfsetfillcolor{currentfill}%
\pgfsetlinewidth{1.003750pt}%
\definecolor{currentstroke}{rgb}{0.000000,0.000000,0.000000}%
\pgfsetstrokecolor{currentstroke}%
\pgfsetdash{}{0pt}%
\pgfpathmoveto{\pgfqpoint{1.005694in}{0.706464in}}%
\pgfpathcurveto{\pgfqpoint{1.016744in}{0.706464in}}{\pgfqpoint{1.027343in}{0.710855in}}{\pgfqpoint{1.035157in}{0.718668in}}%
\pgfpathcurveto{\pgfqpoint{1.042971in}{0.726482in}}{\pgfqpoint{1.047361in}{0.737081in}}{\pgfqpoint{1.047361in}{0.748131in}}%
\pgfpathcurveto{\pgfqpoint{1.047361in}{0.759181in}}{\pgfqpoint{1.042971in}{0.769780in}}{\pgfqpoint{1.035157in}{0.777594in}}%
\pgfpathcurveto{\pgfqpoint{1.027343in}{0.785407in}}{\pgfqpoint{1.016744in}{0.789798in}}{\pgfqpoint{1.005694in}{0.789798in}}%
\pgfpathcurveto{\pgfqpoint{0.994644in}{0.789798in}}{\pgfqpoint{0.984045in}{0.785407in}}{\pgfqpoint{0.976232in}{0.777594in}}%
\pgfpathcurveto{\pgfqpoint{0.968418in}{0.769780in}}{\pgfqpoint{0.964028in}{0.759181in}}{\pgfqpoint{0.964028in}{0.748131in}}%
\pgfpathcurveto{\pgfqpoint{0.964028in}{0.737081in}}{\pgfqpoint{0.968418in}{0.726482in}}{\pgfqpoint{0.976232in}{0.718668in}}%
\pgfpathcurveto{\pgfqpoint{0.984045in}{0.710855in}}{\pgfqpoint{0.994644in}{0.706464in}}{\pgfqpoint{1.005694in}{0.706464in}}%
\pgfpathclose%
\pgfusepath{stroke,fill}%
\end{pgfscope}%
\begin{pgfscope}%
\pgfpathrectangle{\pgfqpoint{0.375000in}{0.330000in}}{\pgfqpoint{2.325000in}{2.310000in}}%
\pgfusepath{clip}%
\pgfsetbuttcap%
\pgfsetroundjoin%
\definecolor{currentfill}{rgb}{0.000000,0.000000,0.000000}%
\pgfsetfillcolor{currentfill}%
\pgfsetlinewidth{1.003750pt}%
\definecolor{currentstroke}{rgb}{0.000000,0.000000,0.000000}%
\pgfsetstrokecolor{currentstroke}%
\pgfsetdash{}{0pt}%
\pgfpathmoveto{\pgfqpoint{1.005694in}{0.706464in}}%
\pgfpathcurveto{\pgfqpoint{1.016744in}{0.706464in}}{\pgfqpoint{1.027343in}{0.710855in}}{\pgfqpoint{1.035157in}{0.718668in}}%
\pgfpathcurveto{\pgfqpoint{1.042971in}{0.726482in}}{\pgfqpoint{1.047361in}{0.737081in}}{\pgfqpoint{1.047361in}{0.748131in}}%
\pgfpathcurveto{\pgfqpoint{1.047361in}{0.759181in}}{\pgfqpoint{1.042971in}{0.769780in}}{\pgfqpoint{1.035157in}{0.777594in}}%
\pgfpathcurveto{\pgfqpoint{1.027343in}{0.785407in}}{\pgfqpoint{1.016744in}{0.789798in}}{\pgfqpoint{1.005694in}{0.789798in}}%
\pgfpathcurveto{\pgfqpoint{0.994644in}{0.789798in}}{\pgfqpoint{0.984045in}{0.785407in}}{\pgfqpoint{0.976232in}{0.777594in}}%
\pgfpathcurveto{\pgfqpoint{0.968418in}{0.769780in}}{\pgfqpoint{0.964028in}{0.759181in}}{\pgfqpoint{0.964028in}{0.748131in}}%
\pgfpathcurveto{\pgfqpoint{0.964028in}{0.737081in}}{\pgfqpoint{0.968418in}{0.726482in}}{\pgfqpoint{0.976232in}{0.718668in}}%
\pgfpathcurveto{\pgfqpoint{0.984045in}{0.710855in}}{\pgfqpoint{0.994644in}{0.706464in}}{\pgfqpoint{1.005694in}{0.706464in}}%
\pgfpathclose%
\pgfusepath{stroke,fill}%
\end{pgfscope}%
\begin{pgfscope}%
\pgfpathrectangle{\pgfqpoint{0.375000in}{0.330000in}}{\pgfqpoint{2.325000in}{2.310000in}}%
\pgfusepath{clip}%
\pgfsetbuttcap%
\pgfsetroundjoin%
\definecolor{currentfill}{rgb}{0.000000,0.000000,0.000000}%
\pgfsetfillcolor{currentfill}%
\pgfsetlinewidth{1.003750pt}%
\definecolor{currentstroke}{rgb}{0.000000,0.000000,0.000000}%
\pgfsetstrokecolor{currentstroke}%
\pgfsetdash{}{0pt}%
\pgfpathmoveto{\pgfqpoint{1.005694in}{0.706464in}}%
\pgfpathcurveto{\pgfqpoint{1.016744in}{0.706464in}}{\pgfqpoint{1.027343in}{0.710855in}}{\pgfqpoint{1.035157in}{0.718668in}}%
\pgfpathcurveto{\pgfqpoint{1.042971in}{0.726482in}}{\pgfqpoint{1.047361in}{0.737081in}}{\pgfqpoint{1.047361in}{0.748131in}}%
\pgfpathcurveto{\pgfqpoint{1.047361in}{0.759181in}}{\pgfqpoint{1.042971in}{0.769780in}}{\pgfqpoint{1.035157in}{0.777594in}}%
\pgfpathcurveto{\pgfqpoint{1.027343in}{0.785407in}}{\pgfqpoint{1.016744in}{0.789798in}}{\pgfqpoint{1.005694in}{0.789798in}}%
\pgfpathcurveto{\pgfqpoint{0.994644in}{0.789798in}}{\pgfqpoint{0.984045in}{0.785407in}}{\pgfqpoint{0.976232in}{0.777594in}}%
\pgfpathcurveto{\pgfqpoint{0.968418in}{0.769780in}}{\pgfqpoint{0.964028in}{0.759181in}}{\pgfqpoint{0.964028in}{0.748131in}}%
\pgfpathcurveto{\pgfqpoint{0.964028in}{0.737081in}}{\pgfqpoint{0.968418in}{0.726482in}}{\pgfqpoint{0.976232in}{0.718668in}}%
\pgfpathcurveto{\pgfqpoint{0.984045in}{0.710855in}}{\pgfqpoint{0.994644in}{0.706464in}}{\pgfqpoint{1.005694in}{0.706464in}}%
\pgfpathclose%
\pgfusepath{stroke,fill}%
\end{pgfscope}%
\begin{pgfscope}%
\pgfpathrectangle{\pgfqpoint{0.375000in}{0.330000in}}{\pgfqpoint{2.325000in}{2.310000in}}%
\pgfusepath{clip}%
\pgfsetbuttcap%
\pgfsetroundjoin%
\definecolor{currentfill}{rgb}{0.000000,0.000000,0.000000}%
\pgfsetfillcolor{currentfill}%
\pgfsetlinewidth{1.003750pt}%
\definecolor{currentstroke}{rgb}{0.000000,0.000000,0.000000}%
\pgfsetstrokecolor{currentstroke}%
\pgfsetdash{}{0pt}%
\pgfpathmoveto{\pgfqpoint{1.005694in}{0.758495in}}%
\pgfpathcurveto{\pgfqpoint{1.016744in}{0.758495in}}{\pgfqpoint{1.027343in}{0.762885in}}{\pgfqpoint{1.035157in}{0.770699in}}%
\pgfpathcurveto{\pgfqpoint{1.042971in}{0.778513in}}{\pgfqpoint{1.047361in}{0.789112in}}{\pgfqpoint{1.047361in}{0.800162in}}%
\pgfpathcurveto{\pgfqpoint{1.047361in}{0.811212in}}{\pgfqpoint{1.042971in}{0.821811in}}{\pgfqpoint{1.035157in}{0.829625in}}%
\pgfpathcurveto{\pgfqpoint{1.027343in}{0.837438in}}{\pgfqpoint{1.016744in}{0.841828in}}{\pgfqpoint{1.005694in}{0.841828in}}%
\pgfpathcurveto{\pgfqpoint{0.994644in}{0.841828in}}{\pgfqpoint{0.984045in}{0.837438in}}{\pgfqpoint{0.976232in}{0.829625in}}%
\pgfpathcurveto{\pgfqpoint{0.968418in}{0.821811in}}{\pgfqpoint{0.964028in}{0.811212in}}{\pgfqpoint{0.964028in}{0.800162in}}%
\pgfpathcurveto{\pgfqpoint{0.964028in}{0.789112in}}{\pgfqpoint{0.968418in}{0.778513in}}{\pgfqpoint{0.976232in}{0.770699in}}%
\pgfpathcurveto{\pgfqpoint{0.984045in}{0.762885in}}{\pgfqpoint{0.994644in}{0.758495in}}{\pgfqpoint{1.005694in}{0.758495in}}%
\pgfpathclose%
\pgfusepath{stroke,fill}%
\end{pgfscope}%
\begin{pgfscope}%
\pgfpathrectangle{\pgfqpoint{0.375000in}{0.330000in}}{\pgfqpoint{2.325000in}{2.310000in}}%
\pgfusepath{clip}%
\pgfsetbuttcap%
\pgfsetroundjoin%
\definecolor{currentfill}{rgb}{0.000000,0.000000,0.000000}%
\pgfsetfillcolor{currentfill}%
\pgfsetlinewidth{1.003750pt}%
\definecolor{currentstroke}{rgb}{0.000000,0.000000,0.000000}%
\pgfsetstrokecolor{currentstroke}%
\pgfsetdash{}{0pt}%
\pgfpathmoveto{\pgfqpoint{1.005694in}{0.706464in}}%
\pgfpathcurveto{\pgfqpoint{1.016744in}{0.706464in}}{\pgfqpoint{1.027343in}{0.710855in}}{\pgfqpoint{1.035157in}{0.718668in}}%
\pgfpathcurveto{\pgfqpoint{1.042971in}{0.726482in}}{\pgfqpoint{1.047361in}{0.737081in}}{\pgfqpoint{1.047361in}{0.748131in}}%
\pgfpathcurveto{\pgfqpoint{1.047361in}{0.759181in}}{\pgfqpoint{1.042971in}{0.769780in}}{\pgfqpoint{1.035157in}{0.777594in}}%
\pgfpathcurveto{\pgfqpoint{1.027343in}{0.785407in}}{\pgfqpoint{1.016744in}{0.789798in}}{\pgfqpoint{1.005694in}{0.789798in}}%
\pgfpathcurveto{\pgfqpoint{0.994644in}{0.789798in}}{\pgfqpoint{0.984045in}{0.785407in}}{\pgfqpoint{0.976232in}{0.777594in}}%
\pgfpathcurveto{\pgfqpoint{0.968418in}{0.769780in}}{\pgfqpoint{0.964028in}{0.759181in}}{\pgfqpoint{0.964028in}{0.748131in}}%
\pgfpathcurveto{\pgfqpoint{0.964028in}{0.737081in}}{\pgfqpoint{0.968418in}{0.726482in}}{\pgfqpoint{0.976232in}{0.718668in}}%
\pgfpathcurveto{\pgfqpoint{0.984045in}{0.710855in}}{\pgfqpoint{0.994644in}{0.706464in}}{\pgfqpoint{1.005694in}{0.706464in}}%
\pgfpathclose%
\pgfusepath{stroke,fill}%
\end{pgfscope}%
\begin{pgfscope}%
\pgfpathrectangle{\pgfqpoint{0.375000in}{0.330000in}}{\pgfqpoint{2.325000in}{2.310000in}}%
\pgfusepath{clip}%
\pgfsetbuttcap%
\pgfsetroundjoin%
\definecolor{currentfill}{rgb}{0.000000,0.000000,0.000000}%
\pgfsetfillcolor{currentfill}%
\pgfsetlinewidth{1.003750pt}%
\definecolor{currentstroke}{rgb}{0.000000,0.000000,0.000000}%
\pgfsetstrokecolor{currentstroke}%
\pgfsetdash{}{0pt}%
\pgfpathmoveto{\pgfqpoint{1.005694in}{0.654433in}}%
\pgfpathcurveto{\pgfqpoint{1.016744in}{0.654433in}}{\pgfqpoint{1.027343in}{0.658824in}}{\pgfqpoint{1.035157in}{0.666637in}}%
\pgfpathcurveto{\pgfqpoint{1.042971in}{0.674451in}}{\pgfqpoint{1.047361in}{0.685050in}}{\pgfqpoint{1.047361in}{0.696100in}}%
\pgfpathcurveto{\pgfqpoint{1.047361in}{0.707150in}}{\pgfqpoint{1.042971in}{0.717749in}}{\pgfqpoint{1.035157in}{0.725563in}}%
\pgfpathcurveto{\pgfqpoint{1.027343in}{0.733377in}}{\pgfqpoint{1.016744in}{0.737767in}}{\pgfqpoint{1.005694in}{0.737767in}}%
\pgfpathcurveto{\pgfqpoint{0.994644in}{0.737767in}}{\pgfqpoint{0.984045in}{0.733377in}}{\pgfqpoint{0.976232in}{0.725563in}}%
\pgfpathcurveto{\pgfqpoint{0.968418in}{0.717749in}}{\pgfqpoint{0.964028in}{0.707150in}}{\pgfqpoint{0.964028in}{0.696100in}}%
\pgfpathcurveto{\pgfqpoint{0.964028in}{0.685050in}}{\pgfqpoint{0.968418in}{0.674451in}}{\pgfqpoint{0.976232in}{0.666637in}}%
\pgfpathcurveto{\pgfqpoint{0.984045in}{0.658824in}}{\pgfqpoint{0.994644in}{0.654433in}}{\pgfqpoint{1.005694in}{0.654433in}}%
\pgfpathclose%
\pgfusepath{stroke,fill}%
\end{pgfscope}%
\begin{pgfscope}%
\pgfpathrectangle{\pgfqpoint{0.375000in}{0.330000in}}{\pgfqpoint{2.325000in}{2.310000in}}%
\pgfusepath{clip}%
\pgfsetbuttcap%
\pgfsetroundjoin%
\definecolor{currentfill}{rgb}{0.000000,0.000000,0.000000}%
\pgfsetfillcolor{currentfill}%
\pgfsetlinewidth{1.003750pt}%
\definecolor{currentstroke}{rgb}{0.000000,0.000000,0.000000}%
\pgfsetstrokecolor{currentstroke}%
\pgfsetdash{}{0pt}%
\pgfpathmoveto{\pgfqpoint{1.005694in}{0.706464in}}%
\pgfpathcurveto{\pgfqpoint{1.016744in}{0.706464in}}{\pgfqpoint{1.027343in}{0.710855in}}{\pgfqpoint{1.035157in}{0.718668in}}%
\pgfpathcurveto{\pgfqpoint{1.042971in}{0.726482in}}{\pgfqpoint{1.047361in}{0.737081in}}{\pgfqpoint{1.047361in}{0.748131in}}%
\pgfpathcurveto{\pgfqpoint{1.047361in}{0.759181in}}{\pgfqpoint{1.042971in}{0.769780in}}{\pgfqpoint{1.035157in}{0.777594in}}%
\pgfpathcurveto{\pgfqpoint{1.027343in}{0.785407in}}{\pgfqpoint{1.016744in}{0.789798in}}{\pgfqpoint{1.005694in}{0.789798in}}%
\pgfpathcurveto{\pgfqpoint{0.994644in}{0.789798in}}{\pgfqpoint{0.984045in}{0.785407in}}{\pgfqpoint{0.976232in}{0.777594in}}%
\pgfpathcurveto{\pgfqpoint{0.968418in}{0.769780in}}{\pgfqpoint{0.964028in}{0.759181in}}{\pgfqpoint{0.964028in}{0.748131in}}%
\pgfpathcurveto{\pgfqpoint{0.964028in}{0.737081in}}{\pgfqpoint{0.968418in}{0.726482in}}{\pgfqpoint{0.976232in}{0.718668in}}%
\pgfpathcurveto{\pgfqpoint{0.984045in}{0.710855in}}{\pgfqpoint{0.994644in}{0.706464in}}{\pgfqpoint{1.005694in}{0.706464in}}%
\pgfpathclose%
\pgfusepath{stroke,fill}%
\end{pgfscope}%
\begin{pgfscope}%
\pgfpathrectangle{\pgfqpoint{0.375000in}{0.330000in}}{\pgfqpoint{2.325000in}{2.310000in}}%
\pgfusepath{clip}%
\pgfsetbuttcap%
\pgfsetroundjoin%
\definecolor{currentfill}{rgb}{0.000000,0.000000,0.000000}%
\pgfsetfillcolor{currentfill}%
\pgfsetlinewidth{1.003750pt}%
\definecolor{currentstroke}{rgb}{0.000000,0.000000,0.000000}%
\pgfsetstrokecolor{currentstroke}%
\pgfsetdash{}{0pt}%
\pgfpathmoveto{\pgfqpoint{1.005694in}{0.758495in}}%
\pgfpathcurveto{\pgfqpoint{1.016744in}{0.758495in}}{\pgfqpoint{1.027343in}{0.762885in}}{\pgfqpoint{1.035157in}{0.770699in}}%
\pgfpathcurveto{\pgfqpoint{1.042971in}{0.778513in}}{\pgfqpoint{1.047361in}{0.789112in}}{\pgfqpoint{1.047361in}{0.800162in}}%
\pgfpathcurveto{\pgfqpoint{1.047361in}{0.811212in}}{\pgfqpoint{1.042971in}{0.821811in}}{\pgfqpoint{1.035157in}{0.829625in}}%
\pgfpathcurveto{\pgfqpoint{1.027343in}{0.837438in}}{\pgfqpoint{1.016744in}{0.841828in}}{\pgfqpoint{1.005694in}{0.841828in}}%
\pgfpathcurveto{\pgfqpoint{0.994644in}{0.841828in}}{\pgfqpoint{0.984045in}{0.837438in}}{\pgfqpoint{0.976232in}{0.829625in}}%
\pgfpathcurveto{\pgfqpoint{0.968418in}{0.821811in}}{\pgfqpoint{0.964028in}{0.811212in}}{\pgfqpoint{0.964028in}{0.800162in}}%
\pgfpathcurveto{\pgfqpoint{0.964028in}{0.789112in}}{\pgfqpoint{0.968418in}{0.778513in}}{\pgfqpoint{0.976232in}{0.770699in}}%
\pgfpathcurveto{\pgfqpoint{0.984045in}{0.762885in}}{\pgfqpoint{0.994644in}{0.758495in}}{\pgfqpoint{1.005694in}{0.758495in}}%
\pgfpathclose%
\pgfusepath{stroke,fill}%
\end{pgfscope}%
\begin{pgfscope}%
\pgfpathrectangle{\pgfqpoint{0.375000in}{0.330000in}}{\pgfqpoint{2.325000in}{2.310000in}}%
\pgfusepath{clip}%
\pgfsetbuttcap%
\pgfsetroundjoin%
\definecolor{currentfill}{rgb}{0.000000,0.000000,0.000000}%
\pgfsetfillcolor{currentfill}%
\pgfsetlinewidth{1.003750pt}%
\definecolor{currentstroke}{rgb}{0.000000,0.000000,0.000000}%
\pgfsetstrokecolor{currentstroke}%
\pgfsetdash{}{0pt}%
\pgfpathmoveto{\pgfqpoint{1.005694in}{0.706464in}}%
\pgfpathcurveto{\pgfqpoint{1.016744in}{0.706464in}}{\pgfqpoint{1.027343in}{0.710855in}}{\pgfqpoint{1.035157in}{0.718668in}}%
\pgfpathcurveto{\pgfqpoint{1.042971in}{0.726482in}}{\pgfqpoint{1.047361in}{0.737081in}}{\pgfqpoint{1.047361in}{0.748131in}}%
\pgfpathcurveto{\pgfqpoint{1.047361in}{0.759181in}}{\pgfqpoint{1.042971in}{0.769780in}}{\pgfqpoint{1.035157in}{0.777594in}}%
\pgfpathcurveto{\pgfqpoint{1.027343in}{0.785407in}}{\pgfqpoint{1.016744in}{0.789798in}}{\pgfqpoint{1.005694in}{0.789798in}}%
\pgfpathcurveto{\pgfqpoint{0.994644in}{0.789798in}}{\pgfqpoint{0.984045in}{0.785407in}}{\pgfqpoint{0.976232in}{0.777594in}}%
\pgfpathcurveto{\pgfqpoint{0.968418in}{0.769780in}}{\pgfqpoint{0.964028in}{0.759181in}}{\pgfqpoint{0.964028in}{0.748131in}}%
\pgfpathcurveto{\pgfqpoint{0.964028in}{0.737081in}}{\pgfqpoint{0.968418in}{0.726482in}}{\pgfqpoint{0.976232in}{0.718668in}}%
\pgfpathcurveto{\pgfqpoint{0.984045in}{0.710855in}}{\pgfqpoint{0.994644in}{0.706464in}}{\pgfqpoint{1.005694in}{0.706464in}}%
\pgfpathclose%
\pgfusepath{stroke,fill}%
\end{pgfscope}%
\begin{pgfscope}%
\pgfpathrectangle{\pgfqpoint{0.375000in}{0.330000in}}{\pgfqpoint{2.325000in}{2.310000in}}%
\pgfusepath{clip}%
\pgfsetbuttcap%
\pgfsetroundjoin%
\definecolor{currentfill}{rgb}{0.000000,0.000000,0.000000}%
\pgfsetfillcolor{currentfill}%
\pgfsetlinewidth{1.003750pt}%
\definecolor{currentstroke}{rgb}{0.000000,0.000000,0.000000}%
\pgfsetstrokecolor{currentstroke}%
\pgfsetdash{}{0pt}%
\pgfpathmoveto{\pgfqpoint{1.005694in}{0.706464in}}%
\pgfpathcurveto{\pgfqpoint{1.016744in}{0.706464in}}{\pgfqpoint{1.027343in}{0.710855in}}{\pgfqpoint{1.035157in}{0.718668in}}%
\pgfpathcurveto{\pgfqpoint{1.042971in}{0.726482in}}{\pgfqpoint{1.047361in}{0.737081in}}{\pgfqpoint{1.047361in}{0.748131in}}%
\pgfpathcurveto{\pgfqpoint{1.047361in}{0.759181in}}{\pgfqpoint{1.042971in}{0.769780in}}{\pgfqpoint{1.035157in}{0.777594in}}%
\pgfpathcurveto{\pgfqpoint{1.027343in}{0.785407in}}{\pgfqpoint{1.016744in}{0.789798in}}{\pgfqpoint{1.005694in}{0.789798in}}%
\pgfpathcurveto{\pgfqpoint{0.994644in}{0.789798in}}{\pgfqpoint{0.984045in}{0.785407in}}{\pgfqpoint{0.976232in}{0.777594in}}%
\pgfpathcurveto{\pgfqpoint{0.968418in}{0.769780in}}{\pgfqpoint{0.964028in}{0.759181in}}{\pgfqpoint{0.964028in}{0.748131in}}%
\pgfpathcurveto{\pgfqpoint{0.964028in}{0.737081in}}{\pgfqpoint{0.968418in}{0.726482in}}{\pgfqpoint{0.976232in}{0.718668in}}%
\pgfpathcurveto{\pgfqpoint{0.984045in}{0.710855in}}{\pgfqpoint{0.994644in}{0.706464in}}{\pgfqpoint{1.005694in}{0.706464in}}%
\pgfpathclose%
\pgfusepath{stroke,fill}%
\end{pgfscope}%
\begin{pgfscope}%
\pgfpathrectangle{\pgfqpoint{0.375000in}{0.330000in}}{\pgfqpoint{2.325000in}{2.310000in}}%
\pgfusepath{clip}%
\pgfsetbuttcap%
\pgfsetroundjoin%
\definecolor{currentfill}{rgb}{0.000000,0.000000,0.000000}%
\pgfsetfillcolor{currentfill}%
\pgfsetlinewidth{1.003750pt}%
\definecolor{currentstroke}{rgb}{0.000000,0.000000,0.000000}%
\pgfsetstrokecolor{currentstroke}%
\pgfsetdash{}{0pt}%
\pgfpathmoveto{\pgfqpoint{1.005694in}{0.758495in}}%
\pgfpathcurveto{\pgfqpoint{1.016744in}{0.758495in}}{\pgfqpoint{1.027343in}{0.762885in}}{\pgfqpoint{1.035157in}{0.770699in}}%
\pgfpathcurveto{\pgfqpoint{1.042971in}{0.778513in}}{\pgfqpoint{1.047361in}{0.789112in}}{\pgfqpoint{1.047361in}{0.800162in}}%
\pgfpathcurveto{\pgfqpoint{1.047361in}{0.811212in}}{\pgfqpoint{1.042971in}{0.821811in}}{\pgfqpoint{1.035157in}{0.829625in}}%
\pgfpathcurveto{\pgfqpoint{1.027343in}{0.837438in}}{\pgfqpoint{1.016744in}{0.841828in}}{\pgfqpoint{1.005694in}{0.841828in}}%
\pgfpathcurveto{\pgfqpoint{0.994644in}{0.841828in}}{\pgfqpoint{0.984045in}{0.837438in}}{\pgfqpoint{0.976232in}{0.829625in}}%
\pgfpathcurveto{\pgfqpoint{0.968418in}{0.821811in}}{\pgfqpoint{0.964028in}{0.811212in}}{\pgfqpoint{0.964028in}{0.800162in}}%
\pgfpathcurveto{\pgfqpoint{0.964028in}{0.789112in}}{\pgfqpoint{0.968418in}{0.778513in}}{\pgfqpoint{0.976232in}{0.770699in}}%
\pgfpathcurveto{\pgfqpoint{0.984045in}{0.762885in}}{\pgfqpoint{0.994644in}{0.758495in}}{\pgfqpoint{1.005694in}{0.758495in}}%
\pgfpathclose%
\pgfusepath{stroke,fill}%
\end{pgfscope}%
\begin{pgfscope}%
\pgfpathrectangle{\pgfqpoint{0.375000in}{0.330000in}}{\pgfqpoint{2.325000in}{2.310000in}}%
\pgfusepath{clip}%
\pgfsetbuttcap%
\pgfsetroundjoin%
\definecolor{currentfill}{rgb}{0.000000,0.000000,0.000000}%
\pgfsetfillcolor{currentfill}%
\pgfsetlinewidth{1.003750pt}%
\definecolor{currentstroke}{rgb}{0.000000,0.000000,0.000000}%
\pgfsetstrokecolor{currentstroke}%
\pgfsetdash{}{0pt}%
\pgfpathmoveto{\pgfqpoint{1.005694in}{0.706464in}}%
\pgfpathcurveto{\pgfqpoint{1.016744in}{0.706464in}}{\pgfqpoint{1.027343in}{0.710855in}}{\pgfqpoint{1.035157in}{0.718668in}}%
\pgfpathcurveto{\pgfqpoint{1.042971in}{0.726482in}}{\pgfqpoint{1.047361in}{0.737081in}}{\pgfqpoint{1.047361in}{0.748131in}}%
\pgfpathcurveto{\pgfqpoint{1.047361in}{0.759181in}}{\pgfqpoint{1.042971in}{0.769780in}}{\pgfqpoint{1.035157in}{0.777594in}}%
\pgfpathcurveto{\pgfqpoint{1.027343in}{0.785407in}}{\pgfqpoint{1.016744in}{0.789798in}}{\pgfqpoint{1.005694in}{0.789798in}}%
\pgfpathcurveto{\pgfqpoint{0.994644in}{0.789798in}}{\pgfqpoint{0.984045in}{0.785407in}}{\pgfqpoint{0.976232in}{0.777594in}}%
\pgfpathcurveto{\pgfqpoint{0.968418in}{0.769780in}}{\pgfqpoint{0.964028in}{0.759181in}}{\pgfqpoint{0.964028in}{0.748131in}}%
\pgfpathcurveto{\pgfqpoint{0.964028in}{0.737081in}}{\pgfqpoint{0.968418in}{0.726482in}}{\pgfqpoint{0.976232in}{0.718668in}}%
\pgfpathcurveto{\pgfqpoint{0.984045in}{0.710855in}}{\pgfqpoint{0.994644in}{0.706464in}}{\pgfqpoint{1.005694in}{0.706464in}}%
\pgfpathclose%
\pgfusepath{stroke,fill}%
\end{pgfscope}%
\begin{pgfscope}%
\pgfpathrectangle{\pgfqpoint{0.375000in}{0.330000in}}{\pgfqpoint{2.325000in}{2.310000in}}%
\pgfusepath{clip}%
\pgfsetbuttcap%
\pgfsetroundjoin%
\definecolor{currentfill}{rgb}{0.000000,0.000000,0.000000}%
\pgfsetfillcolor{currentfill}%
\pgfsetlinewidth{1.003750pt}%
\definecolor{currentstroke}{rgb}{0.000000,0.000000,0.000000}%
\pgfsetstrokecolor{currentstroke}%
\pgfsetdash{}{0pt}%
\pgfpathmoveto{\pgfqpoint{1.005694in}{0.706464in}}%
\pgfpathcurveto{\pgfqpoint{1.016744in}{0.706464in}}{\pgfqpoint{1.027343in}{0.710855in}}{\pgfqpoint{1.035157in}{0.718668in}}%
\pgfpathcurveto{\pgfqpoint{1.042971in}{0.726482in}}{\pgfqpoint{1.047361in}{0.737081in}}{\pgfqpoint{1.047361in}{0.748131in}}%
\pgfpathcurveto{\pgfqpoint{1.047361in}{0.759181in}}{\pgfqpoint{1.042971in}{0.769780in}}{\pgfqpoint{1.035157in}{0.777594in}}%
\pgfpathcurveto{\pgfqpoint{1.027343in}{0.785407in}}{\pgfqpoint{1.016744in}{0.789798in}}{\pgfqpoint{1.005694in}{0.789798in}}%
\pgfpathcurveto{\pgfqpoint{0.994644in}{0.789798in}}{\pgfqpoint{0.984045in}{0.785407in}}{\pgfqpoint{0.976232in}{0.777594in}}%
\pgfpathcurveto{\pgfqpoint{0.968418in}{0.769780in}}{\pgfqpoint{0.964028in}{0.759181in}}{\pgfqpoint{0.964028in}{0.748131in}}%
\pgfpathcurveto{\pgfqpoint{0.964028in}{0.737081in}}{\pgfqpoint{0.968418in}{0.726482in}}{\pgfqpoint{0.976232in}{0.718668in}}%
\pgfpathcurveto{\pgfqpoint{0.984045in}{0.710855in}}{\pgfqpoint{0.994644in}{0.706464in}}{\pgfqpoint{1.005694in}{0.706464in}}%
\pgfpathclose%
\pgfusepath{stroke,fill}%
\end{pgfscope}%
\begin{pgfscope}%
\pgfpathrectangle{\pgfqpoint{0.375000in}{0.330000in}}{\pgfqpoint{2.325000in}{2.310000in}}%
\pgfusepath{clip}%
\pgfsetbuttcap%
\pgfsetroundjoin%
\definecolor{currentfill}{rgb}{0.000000,0.000000,0.000000}%
\pgfsetfillcolor{currentfill}%
\pgfsetlinewidth{1.003750pt}%
\definecolor{currentstroke}{rgb}{0.000000,0.000000,0.000000}%
\pgfsetstrokecolor{currentstroke}%
\pgfsetdash{}{0pt}%
\pgfpathmoveto{\pgfqpoint{1.005694in}{0.758495in}}%
\pgfpathcurveto{\pgfqpoint{1.016744in}{0.758495in}}{\pgfqpoint{1.027343in}{0.762885in}}{\pgfqpoint{1.035157in}{0.770699in}}%
\pgfpathcurveto{\pgfqpoint{1.042971in}{0.778513in}}{\pgfqpoint{1.047361in}{0.789112in}}{\pgfqpoint{1.047361in}{0.800162in}}%
\pgfpathcurveto{\pgfqpoint{1.047361in}{0.811212in}}{\pgfqpoint{1.042971in}{0.821811in}}{\pgfqpoint{1.035157in}{0.829625in}}%
\pgfpathcurveto{\pgfqpoint{1.027343in}{0.837438in}}{\pgfqpoint{1.016744in}{0.841828in}}{\pgfqpoint{1.005694in}{0.841828in}}%
\pgfpathcurveto{\pgfqpoint{0.994644in}{0.841828in}}{\pgfqpoint{0.984045in}{0.837438in}}{\pgfqpoint{0.976232in}{0.829625in}}%
\pgfpathcurveto{\pgfqpoint{0.968418in}{0.821811in}}{\pgfqpoint{0.964028in}{0.811212in}}{\pgfqpoint{0.964028in}{0.800162in}}%
\pgfpathcurveto{\pgfqpoint{0.964028in}{0.789112in}}{\pgfqpoint{0.968418in}{0.778513in}}{\pgfqpoint{0.976232in}{0.770699in}}%
\pgfpathcurveto{\pgfqpoint{0.984045in}{0.762885in}}{\pgfqpoint{0.994644in}{0.758495in}}{\pgfqpoint{1.005694in}{0.758495in}}%
\pgfpathclose%
\pgfusepath{stroke,fill}%
\end{pgfscope}%
\begin{pgfscope}%
\pgfpathrectangle{\pgfqpoint{0.375000in}{0.330000in}}{\pgfqpoint{2.325000in}{2.310000in}}%
\pgfusepath{clip}%
\pgfsetbuttcap%
\pgfsetroundjoin%
\definecolor{currentfill}{rgb}{0.000000,0.000000,0.000000}%
\pgfsetfillcolor{currentfill}%
\pgfsetlinewidth{1.003750pt}%
\definecolor{currentstroke}{rgb}{0.000000,0.000000,0.000000}%
\pgfsetstrokecolor{currentstroke}%
\pgfsetdash{}{0pt}%
\pgfpathmoveto{\pgfqpoint{1.005694in}{0.810526in}}%
\pgfpathcurveto{\pgfqpoint{1.016744in}{0.810526in}}{\pgfqpoint{1.027343in}{0.814916in}}{\pgfqpoint{1.035157in}{0.822730in}}%
\pgfpathcurveto{\pgfqpoint{1.042971in}{0.830543in}}{\pgfqpoint{1.047361in}{0.841143in}}{\pgfqpoint{1.047361in}{0.852193in}}%
\pgfpathcurveto{\pgfqpoint{1.047361in}{0.863243in}}{\pgfqpoint{1.042971in}{0.873842in}}{\pgfqpoint{1.035157in}{0.881655in}}%
\pgfpathcurveto{\pgfqpoint{1.027343in}{0.889469in}}{\pgfqpoint{1.016744in}{0.893859in}}{\pgfqpoint{1.005694in}{0.893859in}}%
\pgfpathcurveto{\pgfqpoint{0.994644in}{0.893859in}}{\pgfqpoint{0.984045in}{0.889469in}}{\pgfqpoint{0.976232in}{0.881655in}}%
\pgfpathcurveto{\pgfqpoint{0.968418in}{0.873842in}}{\pgfqpoint{0.964028in}{0.863243in}}{\pgfqpoint{0.964028in}{0.852193in}}%
\pgfpathcurveto{\pgfqpoint{0.964028in}{0.841143in}}{\pgfqpoint{0.968418in}{0.830543in}}{\pgfqpoint{0.976232in}{0.822730in}}%
\pgfpathcurveto{\pgfqpoint{0.984045in}{0.814916in}}{\pgfqpoint{0.994644in}{0.810526in}}{\pgfqpoint{1.005694in}{0.810526in}}%
\pgfpathclose%
\pgfusepath{stroke,fill}%
\end{pgfscope}%
\begin{pgfscope}%
\pgfpathrectangle{\pgfqpoint{0.375000in}{0.330000in}}{\pgfqpoint{2.325000in}{2.310000in}}%
\pgfusepath{clip}%
\pgfsetbuttcap%
\pgfsetroundjoin%
\definecolor{currentfill}{rgb}{0.000000,0.000000,0.000000}%
\pgfsetfillcolor{currentfill}%
\pgfsetlinewidth{1.003750pt}%
\definecolor{currentstroke}{rgb}{0.000000,0.000000,0.000000}%
\pgfsetstrokecolor{currentstroke}%
\pgfsetdash{}{0pt}%
\pgfpathmoveto{\pgfqpoint{1.005694in}{0.758495in}}%
\pgfpathcurveto{\pgfqpoint{1.016744in}{0.758495in}}{\pgfqpoint{1.027343in}{0.762885in}}{\pgfqpoint{1.035157in}{0.770699in}}%
\pgfpathcurveto{\pgfqpoint{1.042971in}{0.778513in}}{\pgfqpoint{1.047361in}{0.789112in}}{\pgfqpoint{1.047361in}{0.800162in}}%
\pgfpathcurveto{\pgfqpoint{1.047361in}{0.811212in}}{\pgfqpoint{1.042971in}{0.821811in}}{\pgfqpoint{1.035157in}{0.829625in}}%
\pgfpathcurveto{\pgfqpoint{1.027343in}{0.837438in}}{\pgfqpoint{1.016744in}{0.841828in}}{\pgfqpoint{1.005694in}{0.841828in}}%
\pgfpathcurveto{\pgfqpoint{0.994644in}{0.841828in}}{\pgfqpoint{0.984045in}{0.837438in}}{\pgfqpoint{0.976232in}{0.829625in}}%
\pgfpathcurveto{\pgfqpoint{0.968418in}{0.821811in}}{\pgfqpoint{0.964028in}{0.811212in}}{\pgfqpoint{0.964028in}{0.800162in}}%
\pgfpathcurveto{\pgfqpoint{0.964028in}{0.789112in}}{\pgfqpoint{0.968418in}{0.778513in}}{\pgfqpoint{0.976232in}{0.770699in}}%
\pgfpathcurveto{\pgfqpoint{0.984045in}{0.762885in}}{\pgfqpoint{0.994644in}{0.758495in}}{\pgfqpoint{1.005694in}{0.758495in}}%
\pgfpathclose%
\pgfusepath{stroke,fill}%
\end{pgfscope}%
\begin{pgfscope}%
\pgfpathrectangle{\pgfqpoint{0.375000in}{0.330000in}}{\pgfqpoint{2.325000in}{2.310000in}}%
\pgfusepath{clip}%
\pgfsetbuttcap%
\pgfsetroundjoin%
\definecolor{currentfill}{rgb}{0.000000,0.000000,0.000000}%
\pgfsetfillcolor{currentfill}%
\pgfsetlinewidth{1.003750pt}%
\definecolor{currentstroke}{rgb}{0.000000,0.000000,0.000000}%
\pgfsetstrokecolor{currentstroke}%
\pgfsetdash{}{0pt}%
\pgfpathmoveto{\pgfqpoint{1.005694in}{0.706464in}}%
\pgfpathcurveto{\pgfqpoint{1.016744in}{0.706464in}}{\pgfqpoint{1.027343in}{0.710855in}}{\pgfqpoint{1.035157in}{0.718668in}}%
\pgfpathcurveto{\pgfqpoint{1.042971in}{0.726482in}}{\pgfqpoint{1.047361in}{0.737081in}}{\pgfqpoint{1.047361in}{0.748131in}}%
\pgfpathcurveto{\pgfqpoint{1.047361in}{0.759181in}}{\pgfqpoint{1.042971in}{0.769780in}}{\pgfqpoint{1.035157in}{0.777594in}}%
\pgfpathcurveto{\pgfqpoint{1.027343in}{0.785407in}}{\pgfqpoint{1.016744in}{0.789798in}}{\pgfqpoint{1.005694in}{0.789798in}}%
\pgfpathcurveto{\pgfqpoint{0.994644in}{0.789798in}}{\pgfqpoint{0.984045in}{0.785407in}}{\pgfqpoint{0.976232in}{0.777594in}}%
\pgfpathcurveto{\pgfqpoint{0.968418in}{0.769780in}}{\pgfqpoint{0.964028in}{0.759181in}}{\pgfqpoint{0.964028in}{0.748131in}}%
\pgfpathcurveto{\pgfqpoint{0.964028in}{0.737081in}}{\pgfqpoint{0.968418in}{0.726482in}}{\pgfqpoint{0.976232in}{0.718668in}}%
\pgfpathcurveto{\pgfqpoint{0.984045in}{0.710855in}}{\pgfqpoint{0.994644in}{0.706464in}}{\pgfqpoint{1.005694in}{0.706464in}}%
\pgfpathclose%
\pgfusepath{stroke,fill}%
\end{pgfscope}%
\begin{pgfscope}%
\pgfpathrectangle{\pgfqpoint{0.375000in}{0.330000in}}{\pgfqpoint{2.325000in}{2.310000in}}%
\pgfusepath{clip}%
\pgfsetbuttcap%
\pgfsetroundjoin%
\definecolor{currentfill}{rgb}{0.000000,0.000000,0.000000}%
\pgfsetfillcolor{currentfill}%
\pgfsetlinewidth{1.003750pt}%
\definecolor{currentstroke}{rgb}{0.000000,0.000000,0.000000}%
\pgfsetstrokecolor{currentstroke}%
\pgfsetdash{}{0pt}%
\pgfpathmoveto{\pgfqpoint{1.005694in}{0.758495in}}%
\pgfpathcurveto{\pgfqpoint{1.016744in}{0.758495in}}{\pgfqpoint{1.027343in}{0.762885in}}{\pgfqpoint{1.035157in}{0.770699in}}%
\pgfpathcurveto{\pgfqpoint{1.042971in}{0.778513in}}{\pgfqpoint{1.047361in}{0.789112in}}{\pgfqpoint{1.047361in}{0.800162in}}%
\pgfpathcurveto{\pgfqpoint{1.047361in}{0.811212in}}{\pgfqpoint{1.042971in}{0.821811in}}{\pgfqpoint{1.035157in}{0.829625in}}%
\pgfpathcurveto{\pgfqpoint{1.027343in}{0.837438in}}{\pgfqpoint{1.016744in}{0.841828in}}{\pgfqpoint{1.005694in}{0.841828in}}%
\pgfpathcurveto{\pgfqpoint{0.994644in}{0.841828in}}{\pgfqpoint{0.984045in}{0.837438in}}{\pgfqpoint{0.976232in}{0.829625in}}%
\pgfpathcurveto{\pgfqpoint{0.968418in}{0.821811in}}{\pgfqpoint{0.964028in}{0.811212in}}{\pgfqpoint{0.964028in}{0.800162in}}%
\pgfpathcurveto{\pgfqpoint{0.964028in}{0.789112in}}{\pgfqpoint{0.968418in}{0.778513in}}{\pgfqpoint{0.976232in}{0.770699in}}%
\pgfpathcurveto{\pgfqpoint{0.984045in}{0.762885in}}{\pgfqpoint{0.994644in}{0.758495in}}{\pgfqpoint{1.005694in}{0.758495in}}%
\pgfpathclose%
\pgfusepath{stroke,fill}%
\end{pgfscope}%
\begin{pgfscope}%
\pgfpathrectangle{\pgfqpoint{0.375000in}{0.330000in}}{\pgfqpoint{2.325000in}{2.310000in}}%
\pgfusepath{clip}%
\pgfsetbuttcap%
\pgfsetroundjoin%
\definecolor{currentfill}{rgb}{0.000000,0.000000,0.000000}%
\pgfsetfillcolor{currentfill}%
\pgfsetlinewidth{1.003750pt}%
\definecolor{currentstroke}{rgb}{0.000000,0.000000,0.000000}%
\pgfsetstrokecolor{currentstroke}%
\pgfsetdash{}{0pt}%
\pgfpathmoveto{\pgfqpoint{1.005694in}{0.706464in}}%
\pgfpathcurveto{\pgfqpoint{1.016744in}{0.706464in}}{\pgfqpoint{1.027343in}{0.710855in}}{\pgfqpoint{1.035157in}{0.718668in}}%
\pgfpathcurveto{\pgfqpoint{1.042971in}{0.726482in}}{\pgfqpoint{1.047361in}{0.737081in}}{\pgfqpoint{1.047361in}{0.748131in}}%
\pgfpathcurveto{\pgfqpoint{1.047361in}{0.759181in}}{\pgfqpoint{1.042971in}{0.769780in}}{\pgfqpoint{1.035157in}{0.777594in}}%
\pgfpathcurveto{\pgfqpoint{1.027343in}{0.785407in}}{\pgfqpoint{1.016744in}{0.789798in}}{\pgfqpoint{1.005694in}{0.789798in}}%
\pgfpathcurveto{\pgfqpoint{0.994644in}{0.789798in}}{\pgfqpoint{0.984045in}{0.785407in}}{\pgfqpoint{0.976232in}{0.777594in}}%
\pgfpathcurveto{\pgfqpoint{0.968418in}{0.769780in}}{\pgfqpoint{0.964028in}{0.759181in}}{\pgfqpoint{0.964028in}{0.748131in}}%
\pgfpathcurveto{\pgfqpoint{0.964028in}{0.737081in}}{\pgfqpoint{0.968418in}{0.726482in}}{\pgfqpoint{0.976232in}{0.718668in}}%
\pgfpathcurveto{\pgfqpoint{0.984045in}{0.710855in}}{\pgfqpoint{0.994644in}{0.706464in}}{\pgfqpoint{1.005694in}{0.706464in}}%
\pgfpathclose%
\pgfusepath{stroke,fill}%
\end{pgfscope}%
\begin{pgfscope}%
\pgfpathrectangle{\pgfqpoint{0.375000in}{0.330000in}}{\pgfqpoint{2.325000in}{2.310000in}}%
\pgfusepath{clip}%
\pgfsetbuttcap%
\pgfsetroundjoin%
\definecolor{currentfill}{rgb}{0.000000,0.000000,0.000000}%
\pgfsetfillcolor{currentfill}%
\pgfsetlinewidth{1.003750pt}%
\definecolor{currentstroke}{rgb}{0.000000,0.000000,0.000000}%
\pgfsetstrokecolor{currentstroke}%
\pgfsetdash{}{0pt}%
\pgfpathmoveto{\pgfqpoint{1.005694in}{0.758495in}}%
\pgfpathcurveto{\pgfqpoint{1.016744in}{0.758495in}}{\pgfqpoint{1.027343in}{0.762885in}}{\pgfqpoint{1.035157in}{0.770699in}}%
\pgfpathcurveto{\pgfqpoint{1.042971in}{0.778513in}}{\pgfqpoint{1.047361in}{0.789112in}}{\pgfqpoint{1.047361in}{0.800162in}}%
\pgfpathcurveto{\pgfqpoint{1.047361in}{0.811212in}}{\pgfqpoint{1.042971in}{0.821811in}}{\pgfqpoint{1.035157in}{0.829625in}}%
\pgfpathcurveto{\pgfqpoint{1.027343in}{0.837438in}}{\pgfqpoint{1.016744in}{0.841828in}}{\pgfqpoint{1.005694in}{0.841828in}}%
\pgfpathcurveto{\pgfqpoint{0.994644in}{0.841828in}}{\pgfqpoint{0.984045in}{0.837438in}}{\pgfqpoint{0.976232in}{0.829625in}}%
\pgfpathcurveto{\pgfqpoint{0.968418in}{0.821811in}}{\pgfqpoint{0.964028in}{0.811212in}}{\pgfqpoint{0.964028in}{0.800162in}}%
\pgfpathcurveto{\pgfqpoint{0.964028in}{0.789112in}}{\pgfqpoint{0.968418in}{0.778513in}}{\pgfqpoint{0.976232in}{0.770699in}}%
\pgfpathcurveto{\pgfqpoint{0.984045in}{0.762885in}}{\pgfqpoint{0.994644in}{0.758495in}}{\pgfqpoint{1.005694in}{0.758495in}}%
\pgfpathclose%
\pgfusepath{stroke,fill}%
\end{pgfscope}%
\begin{pgfscope}%
\pgfpathrectangle{\pgfqpoint{0.375000in}{0.330000in}}{\pgfqpoint{2.325000in}{2.310000in}}%
\pgfusepath{clip}%
\pgfsetbuttcap%
\pgfsetroundjoin%
\definecolor{currentfill}{rgb}{0.000000,0.000000,0.000000}%
\pgfsetfillcolor{currentfill}%
\pgfsetlinewidth{1.003750pt}%
\definecolor{currentstroke}{rgb}{0.000000,0.000000,0.000000}%
\pgfsetstrokecolor{currentstroke}%
\pgfsetdash{}{0pt}%
\pgfpathmoveto{\pgfqpoint{1.005694in}{0.758495in}}%
\pgfpathcurveto{\pgfqpoint{1.016744in}{0.758495in}}{\pgfqpoint{1.027343in}{0.762885in}}{\pgfqpoint{1.035157in}{0.770699in}}%
\pgfpathcurveto{\pgfqpoint{1.042971in}{0.778513in}}{\pgfqpoint{1.047361in}{0.789112in}}{\pgfqpoint{1.047361in}{0.800162in}}%
\pgfpathcurveto{\pgfqpoint{1.047361in}{0.811212in}}{\pgfqpoint{1.042971in}{0.821811in}}{\pgfqpoint{1.035157in}{0.829625in}}%
\pgfpathcurveto{\pgfqpoint{1.027343in}{0.837438in}}{\pgfqpoint{1.016744in}{0.841828in}}{\pgfqpoint{1.005694in}{0.841828in}}%
\pgfpathcurveto{\pgfqpoint{0.994644in}{0.841828in}}{\pgfqpoint{0.984045in}{0.837438in}}{\pgfqpoint{0.976232in}{0.829625in}}%
\pgfpathcurveto{\pgfqpoint{0.968418in}{0.821811in}}{\pgfqpoint{0.964028in}{0.811212in}}{\pgfqpoint{0.964028in}{0.800162in}}%
\pgfpathcurveto{\pgfqpoint{0.964028in}{0.789112in}}{\pgfqpoint{0.968418in}{0.778513in}}{\pgfqpoint{0.976232in}{0.770699in}}%
\pgfpathcurveto{\pgfqpoint{0.984045in}{0.762885in}}{\pgfqpoint{0.994644in}{0.758495in}}{\pgfqpoint{1.005694in}{0.758495in}}%
\pgfpathclose%
\pgfusepath{stroke,fill}%
\end{pgfscope}%
\begin{pgfscope}%
\pgfpathrectangle{\pgfqpoint{0.375000in}{0.330000in}}{\pgfqpoint{2.325000in}{2.310000in}}%
\pgfusepath{clip}%
\pgfsetbuttcap%
\pgfsetroundjoin%
\definecolor{currentfill}{rgb}{0.000000,0.000000,0.000000}%
\pgfsetfillcolor{currentfill}%
\pgfsetlinewidth{1.003750pt}%
\definecolor{currentstroke}{rgb}{0.000000,0.000000,0.000000}%
\pgfsetstrokecolor{currentstroke}%
\pgfsetdash{}{0pt}%
\pgfpathmoveto{\pgfqpoint{1.005694in}{0.758495in}}%
\pgfpathcurveto{\pgfqpoint{1.016744in}{0.758495in}}{\pgfqpoint{1.027343in}{0.762885in}}{\pgfqpoint{1.035157in}{0.770699in}}%
\pgfpathcurveto{\pgfqpoint{1.042971in}{0.778513in}}{\pgfqpoint{1.047361in}{0.789112in}}{\pgfqpoint{1.047361in}{0.800162in}}%
\pgfpathcurveto{\pgfqpoint{1.047361in}{0.811212in}}{\pgfqpoint{1.042971in}{0.821811in}}{\pgfqpoint{1.035157in}{0.829625in}}%
\pgfpathcurveto{\pgfqpoint{1.027343in}{0.837438in}}{\pgfqpoint{1.016744in}{0.841828in}}{\pgfqpoint{1.005694in}{0.841828in}}%
\pgfpathcurveto{\pgfqpoint{0.994644in}{0.841828in}}{\pgfqpoint{0.984045in}{0.837438in}}{\pgfqpoint{0.976232in}{0.829625in}}%
\pgfpathcurveto{\pgfqpoint{0.968418in}{0.821811in}}{\pgfqpoint{0.964028in}{0.811212in}}{\pgfqpoint{0.964028in}{0.800162in}}%
\pgfpathcurveto{\pgfqpoint{0.964028in}{0.789112in}}{\pgfqpoint{0.968418in}{0.778513in}}{\pgfqpoint{0.976232in}{0.770699in}}%
\pgfpathcurveto{\pgfqpoint{0.984045in}{0.762885in}}{\pgfqpoint{0.994644in}{0.758495in}}{\pgfqpoint{1.005694in}{0.758495in}}%
\pgfpathclose%
\pgfusepath{stroke,fill}%
\end{pgfscope}%
\begin{pgfscope}%
\pgfpathrectangle{\pgfqpoint{0.375000in}{0.330000in}}{\pgfqpoint{2.325000in}{2.310000in}}%
\pgfusepath{clip}%
\pgfsetbuttcap%
\pgfsetroundjoin%
\definecolor{currentfill}{rgb}{0.000000,0.000000,0.000000}%
\pgfsetfillcolor{currentfill}%
\pgfsetlinewidth{1.003750pt}%
\definecolor{currentstroke}{rgb}{0.000000,0.000000,0.000000}%
\pgfsetstrokecolor{currentstroke}%
\pgfsetdash{}{0pt}%
\pgfpathmoveto{\pgfqpoint{1.005694in}{0.706464in}}%
\pgfpathcurveto{\pgfqpoint{1.016744in}{0.706464in}}{\pgfqpoint{1.027343in}{0.710855in}}{\pgfqpoint{1.035157in}{0.718668in}}%
\pgfpathcurveto{\pgfqpoint{1.042971in}{0.726482in}}{\pgfqpoint{1.047361in}{0.737081in}}{\pgfqpoint{1.047361in}{0.748131in}}%
\pgfpathcurveto{\pgfqpoint{1.047361in}{0.759181in}}{\pgfqpoint{1.042971in}{0.769780in}}{\pgfqpoint{1.035157in}{0.777594in}}%
\pgfpathcurveto{\pgfqpoint{1.027343in}{0.785407in}}{\pgfqpoint{1.016744in}{0.789798in}}{\pgfqpoint{1.005694in}{0.789798in}}%
\pgfpathcurveto{\pgfqpoint{0.994644in}{0.789798in}}{\pgfqpoint{0.984045in}{0.785407in}}{\pgfqpoint{0.976232in}{0.777594in}}%
\pgfpathcurveto{\pgfqpoint{0.968418in}{0.769780in}}{\pgfqpoint{0.964028in}{0.759181in}}{\pgfqpoint{0.964028in}{0.748131in}}%
\pgfpathcurveto{\pgfqpoint{0.964028in}{0.737081in}}{\pgfqpoint{0.968418in}{0.726482in}}{\pgfqpoint{0.976232in}{0.718668in}}%
\pgfpathcurveto{\pgfqpoint{0.984045in}{0.710855in}}{\pgfqpoint{0.994644in}{0.706464in}}{\pgfqpoint{1.005694in}{0.706464in}}%
\pgfpathclose%
\pgfusepath{stroke,fill}%
\end{pgfscope}%
\begin{pgfscope}%
\pgfpathrectangle{\pgfqpoint{0.375000in}{0.330000in}}{\pgfqpoint{2.325000in}{2.310000in}}%
\pgfusepath{clip}%
\pgfsetbuttcap%
\pgfsetroundjoin%
\definecolor{currentfill}{rgb}{0.000000,0.000000,0.000000}%
\pgfsetfillcolor{currentfill}%
\pgfsetlinewidth{1.003750pt}%
\definecolor{currentstroke}{rgb}{0.000000,0.000000,0.000000}%
\pgfsetstrokecolor{currentstroke}%
\pgfsetdash{}{0pt}%
\pgfpathmoveto{\pgfqpoint{1.005694in}{0.706464in}}%
\pgfpathcurveto{\pgfqpoint{1.016744in}{0.706464in}}{\pgfqpoint{1.027343in}{0.710855in}}{\pgfqpoint{1.035157in}{0.718668in}}%
\pgfpathcurveto{\pgfqpoint{1.042971in}{0.726482in}}{\pgfqpoint{1.047361in}{0.737081in}}{\pgfqpoint{1.047361in}{0.748131in}}%
\pgfpathcurveto{\pgfqpoint{1.047361in}{0.759181in}}{\pgfqpoint{1.042971in}{0.769780in}}{\pgfqpoint{1.035157in}{0.777594in}}%
\pgfpathcurveto{\pgfqpoint{1.027343in}{0.785407in}}{\pgfqpoint{1.016744in}{0.789798in}}{\pgfqpoint{1.005694in}{0.789798in}}%
\pgfpathcurveto{\pgfqpoint{0.994644in}{0.789798in}}{\pgfqpoint{0.984045in}{0.785407in}}{\pgfqpoint{0.976232in}{0.777594in}}%
\pgfpathcurveto{\pgfqpoint{0.968418in}{0.769780in}}{\pgfqpoint{0.964028in}{0.759181in}}{\pgfqpoint{0.964028in}{0.748131in}}%
\pgfpathcurveto{\pgfqpoint{0.964028in}{0.737081in}}{\pgfqpoint{0.968418in}{0.726482in}}{\pgfqpoint{0.976232in}{0.718668in}}%
\pgfpathcurveto{\pgfqpoint{0.984045in}{0.710855in}}{\pgfqpoint{0.994644in}{0.706464in}}{\pgfqpoint{1.005694in}{0.706464in}}%
\pgfpathclose%
\pgfusepath{stroke,fill}%
\end{pgfscope}%
\begin{pgfscope}%
\pgfpathrectangle{\pgfqpoint{0.375000in}{0.330000in}}{\pgfqpoint{2.325000in}{2.310000in}}%
\pgfusepath{clip}%
\pgfsetbuttcap%
\pgfsetroundjoin%
\definecolor{currentfill}{rgb}{0.000000,0.000000,0.000000}%
\pgfsetfillcolor{currentfill}%
\pgfsetlinewidth{1.003750pt}%
\definecolor{currentstroke}{rgb}{0.000000,0.000000,0.000000}%
\pgfsetstrokecolor{currentstroke}%
\pgfsetdash{}{0pt}%
\pgfpathmoveto{\pgfqpoint{1.005694in}{0.706464in}}%
\pgfpathcurveto{\pgfqpoint{1.016744in}{0.706464in}}{\pgfqpoint{1.027343in}{0.710855in}}{\pgfqpoint{1.035157in}{0.718668in}}%
\pgfpathcurveto{\pgfqpoint{1.042971in}{0.726482in}}{\pgfqpoint{1.047361in}{0.737081in}}{\pgfqpoint{1.047361in}{0.748131in}}%
\pgfpathcurveto{\pgfqpoint{1.047361in}{0.759181in}}{\pgfqpoint{1.042971in}{0.769780in}}{\pgfqpoint{1.035157in}{0.777594in}}%
\pgfpathcurveto{\pgfqpoint{1.027343in}{0.785407in}}{\pgfqpoint{1.016744in}{0.789798in}}{\pgfqpoint{1.005694in}{0.789798in}}%
\pgfpathcurveto{\pgfqpoint{0.994644in}{0.789798in}}{\pgfqpoint{0.984045in}{0.785407in}}{\pgfqpoint{0.976232in}{0.777594in}}%
\pgfpathcurveto{\pgfqpoint{0.968418in}{0.769780in}}{\pgfqpoint{0.964028in}{0.759181in}}{\pgfqpoint{0.964028in}{0.748131in}}%
\pgfpathcurveto{\pgfqpoint{0.964028in}{0.737081in}}{\pgfqpoint{0.968418in}{0.726482in}}{\pgfqpoint{0.976232in}{0.718668in}}%
\pgfpathcurveto{\pgfqpoint{0.984045in}{0.710855in}}{\pgfqpoint{0.994644in}{0.706464in}}{\pgfqpoint{1.005694in}{0.706464in}}%
\pgfpathclose%
\pgfusepath{stroke,fill}%
\end{pgfscope}%
\begin{pgfscope}%
\pgfpathrectangle{\pgfqpoint{0.375000in}{0.330000in}}{\pgfqpoint{2.325000in}{2.310000in}}%
\pgfusepath{clip}%
\pgfsetbuttcap%
\pgfsetroundjoin%
\definecolor{currentfill}{rgb}{0.000000,0.000000,0.000000}%
\pgfsetfillcolor{currentfill}%
\pgfsetlinewidth{1.003750pt}%
\definecolor{currentstroke}{rgb}{0.000000,0.000000,0.000000}%
\pgfsetstrokecolor{currentstroke}%
\pgfsetdash{}{0pt}%
\pgfpathmoveto{\pgfqpoint{1.005694in}{0.706464in}}%
\pgfpathcurveto{\pgfqpoint{1.016744in}{0.706464in}}{\pgfqpoint{1.027343in}{0.710855in}}{\pgfqpoint{1.035157in}{0.718668in}}%
\pgfpathcurveto{\pgfqpoint{1.042971in}{0.726482in}}{\pgfqpoint{1.047361in}{0.737081in}}{\pgfqpoint{1.047361in}{0.748131in}}%
\pgfpathcurveto{\pgfqpoint{1.047361in}{0.759181in}}{\pgfqpoint{1.042971in}{0.769780in}}{\pgfqpoint{1.035157in}{0.777594in}}%
\pgfpathcurveto{\pgfqpoint{1.027343in}{0.785407in}}{\pgfqpoint{1.016744in}{0.789798in}}{\pgfqpoint{1.005694in}{0.789798in}}%
\pgfpathcurveto{\pgfqpoint{0.994644in}{0.789798in}}{\pgfqpoint{0.984045in}{0.785407in}}{\pgfqpoint{0.976232in}{0.777594in}}%
\pgfpathcurveto{\pgfqpoint{0.968418in}{0.769780in}}{\pgfqpoint{0.964028in}{0.759181in}}{\pgfqpoint{0.964028in}{0.748131in}}%
\pgfpathcurveto{\pgfqpoint{0.964028in}{0.737081in}}{\pgfqpoint{0.968418in}{0.726482in}}{\pgfqpoint{0.976232in}{0.718668in}}%
\pgfpathcurveto{\pgfqpoint{0.984045in}{0.710855in}}{\pgfqpoint{0.994644in}{0.706464in}}{\pgfqpoint{1.005694in}{0.706464in}}%
\pgfpathclose%
\pgfusepath{stroke,fill}%
\end{pgfscope}%
\begin{pgfscope}%
\pgfpathrectangle{\pgfqpoint{0.375000in}{0.330000in}}{\pgfqpoint{2.325000in}{2.310000in}}%
\pgfusepath{clip}%
\pgfsetbuttcap%
\pgfsetroundjoin%
\definecolor{currentfill}{rgb}{0.000000,0.000000,0.000000}%
\pgfsetfillcolor{currentfill}%
\pgfsetlinewidth{1.003750pt}%
\definecolor{currentstroke}{rgb}{0.000000,0.000000,0.000000}%
\pgfsetstrokecolor{currentstroke}%
\pgfsetdash{}{0pt}%
\pgfpathmoveto{\pgfqpoint{1.005694in}{0.758495in}}%
\pgfpathcurveto{\pgfqpoint{1.016744in}{0.758495in}}{\pgfqpoint{1.027343in}{0.762885in}}{\pgfqpoint{1.035157in}{0.770699in}}%
\pgfpathcurveto{\pgfqpoint{1.042971in}{0.778513in}}{\pgfqpoint{1.047361in}{0.789112in}}{\pgfqpoint{1.047361in}{0.800162in}}%
\pgfpathcurveto{\pgfqpoint{1.047361in}{0.811212in}}{\pgfqpoint{1.042971in}{0.821811in}}{\pgfqpoint{1.035157in}{0.829625in}}%
\pgfpathcurveto{\pgfqpoint{1.027343in}{0.837438in}}{\pgfqpoint{1.016744in}{0.841828in}}{\pgfqpoint{1.005694in}{0.841828in}}%
\pgfpathcurveto{\pgfqpoint{0.994644in}{0.841828in}}{\pgfqpoint{0.984045in}{0.837438in}}{\pgfqpoint{0.976232in}{0.829625in}}%
\pgfpathcurveto{\pgfqpoint{0.968418in}{0.821811in}}{\pgfqpoint{0.964028in}{0.811212in}}{\pgfqpoint{0.964028in}{0.800162in}}%
\pgfpathcurveto{\pgfqpoint{0.964028in}{0.789112in}}{\pgfqpoint{0.968418in}{0.778513in}}{\pgfqpoint{0.976232in}{0.770699in}}%
\pgfpathcurveto{\pgfqpoint{0.984045in}{0.762885in}}{\pgfqpoint{0.994644in}{0.758495in}}{\pgfqpoint{1.005694in}{0.758495in}}%
\pgfpathclose%
\pgfusepath{stroke,fill}%
\end{pgfscope}%
\begin{pgfscope}%
\pgfpathrectangle{\pgfqpoint{0.375000in}{0.330000in}}{\pgfqpoint{2.325000in}{2.310000in}}%
\pgfusepath{clip}%
\pgfsetbuttcap%
\pgfsetroundjoin%
\definecolor{currentfill}{rgb}{0.000000,0.000000,0.000000}%
\pgfsetfillcolor{currentfill}%
\pgfsetlinewidth{1.003750pt}%
\definecolor{currentstroke}{rgb}{0.000000,0.000000,0.000000}%
\pgfsetstrokecolor{currentstroke}%
\pgfsetdash{}{0pt}%
\pgfpathmoveto{\pgfqpoint{1.005694in}{0.758495in}}%
\pgfpathcurveto{\pgfqpoint{1.016744in}{0.758495in}}{\pgfqpoint{1.027343in}{0.762885in}}{\pgfqpoint{1.035157in}{0.770699in}}%
\pgfpathcurveto{\pgfqpoint{1.042971in}{0.778513in}}{\pgfqpoint{1.047361in}{0.789112in}}{\pgfqpoint{1.047361in}{0.800162in}}%
\pgfpathcurveto{\pgfqpoint{1.047361in}{0.811212in}}{\pgfqpoint{1.042971in}{0.821811in}}{\pgfqpoint{1.035157in}{0.829625in}}%
\pgfpathcurveto{\pgfqpoint{1.027343in}{0.837438in}}{\pgfqpoint{1.016744in}{0.841828in}}{\pgfqpoint{1.005694in}{0.841828in}}%
\pgfpathcurveto{\pgfqpoint{0.994644in}{0.841828in}}{\pgfqpoint{0.984045in}{0.837438in}}{\pgfqpoint{0.976232in}{0.829625in}}%
\pgfpathcurveto{\pgfqpoint{0.968418in}{0.821811in}}{\pgfqpoint{0.964028in}{0.811212in}}{\pgfqpoint{0.964028in}{0.800162in}}%
\pgfpathcurveto{\pgfqpoint{0.964028in}{0.789112in}}{\pgfqpoint{0.968418in}{0.778513in}}{\pgfqpoint{0.976232in}{0.770699in}}%
\pgfpathcurveto{\pgfqpoint{0.984045in}{0.762885in}}{\pgfqpoint{0.994644in}{0.758495in}}{\pgfqpoint{1.005694in}{0.758495in}}%
\pgfpathclose%
\pgfusepath{stroke,fill}%
\end{pgfscope}%
\begin{pgfscope}%
\pgfpathrectangle{\pgfqpoint{0.375000in}{0.330000in}}{\pgfqpoint{2.325000in}{2.310000in}}%
\pgfusepath{clip}%
\pgfsetbuttcap%
\pgfsetroundjoin%
\definecolor{currentfill}{rgb}{0.000000,0.000000,0.000000}%
\pgfsetfillcolor{currentfill}%
\pgfsetlinewidth{1.003750pt}%
\definecolor{currentstroke}{rgb}{0.000000,0.000000,0.000000}%
\pgfsetstrokecolor{currentstroke}%
\pgfsetdash{}{0pt}%
\pgfpathmoveto{\pgfqpoint{1.005694in}{0.758495in}}%
\pgfpathcurveto{\pgfqpoint{1.016744in}{0.758495in}}{\pgfqpoint{1.027343in}{0.762885in}}{\pgfqpoint{1.035157in}{0.770699in}}%
\pgfpathcurveto{\pgfqpoint{1.042971in}{0.778513in}}{\pgfqpoint{1.047361in}{0.789112in}}{\pgfqpoint{1.047361in}{0.800162in}}%
\pgfpathcurveto{\pgfqpoint{1.047361in}{0.811212in}}{\pgfqpoint{1.042971in}{0.821811in}}{\pgfqpoint{1.035157in}{0.829625in}}%
\pgfpathcurveto{\pgfqpoint{1.027343in}{0.837438in}}{\pgfqpoint{1.016744in}{0.841828in}}{\pgfqpoint{1.005694in}{0.841828in}}%
\pgfpathcurveto{\pgfqpoint{0.994644in}{0.841828in}}{\pgfqpoint{0.984045in}{0.837438in}}{\pgfqpoint{0.976232in}{0.829625in}}%
\pgfpathcurveto{\pgfqpoint{0.968418in}{0.821811in}}{\pgfqpoint{0.964028in}{0.811212in}}{\pgfqpoint{0.964028in}{0.800162in}}%
\pgfpathcurveto{\pgfqpoint{0.964028in}{0.789112in}}{\pgfqpoint{0.968418in}{0.778513in}}{\pgfqpoint{0.976232in}{0.770699in}}%
\pgfpathcurveto{\pgfqpoint{0.984045in}{0.762885in}}{\pgfqpoint{0.994644in}{0.758495in}}{\pgfqpoint{1.005694in}{0.758495in}}%
\pgfpathclose%
\pgfusepath{stroke,fill}%
\end{pgfscope}%
\begin{pgfscope}%
\pgfpathrectangle{\pgfqpoint{0.375000in}{0.330000in}}{\pgfqpoint{2.325000in}{2.310000in}}%
\pgfusepath{clip}%
\pgfsetbuttcap%
\pgfsetroundjoin%
\definecolor{currentfill}{rgb}{0.000000,0.000000,0.000000}%
\pgfsetfillcolor{currentfill}%
\pgfsetlinewidth{1.003750pt}%
\definecolor{currentstroke}{rgb}{0.000000,0.000000,0.000000}%
\pgfsetstrokecolor{currentstroke}%
\pgfsetdash{}{0pt}%
\pgfpathmoveto{\pgfqpoint{1.005694in}{0.706464in}}%
\pgfpathcurveto{\pgfqpoint{1.016744in}{0.706464in}}{\pgfqpoint{1.027343in}{0.710855in}}{\pgfqpoint{1.035157in}{0.718668in}}%
\pgfpathcurveto{\pgfqpoint{1.042971in}{0.726482in}}{\pgfqpoint{1.047361in}{0.737081in}}{\pgfqpoint{1.047361in}{0.748131in}}%
\pgfpathcurveto{\pgfqpoint{1.047361in}{0.759181in}}{\pgfqpoint{1.042971in}{0.769780in}}{\pgfqpoint{1.035157in}{0.777594in}}%
\pgfpathcurveto{\pgfqpoint{1.027343in}{0.785407in}}{\pgfqpoint{1.016744in}{0.789798in}}{\pgfqpoint{1.005694in}{0.789798in}}%
\pgfpathcurveto{\pgfqpoint{0.994644in}{0.789798in}}{\pgfqpoint{0.984045in}{0.785407in}}{\pgfqpoint{0.976232in}{0.777594in}}%
\pgfpathcurveto{\pgfqpoint{0.968418in}{0.769780in}}{\pgfqpoint{0.964028in}{0.759181in}}{\pgfqpoint{0.964028in}{0.748131in}}%
\pgfpathcurveto{\pgfqpoint{0.964028in}{0.737081in}}{\pgfqpoint{0.968418in}{0.726482in}}{\pgfqpoint{0.976232in}{0.718668in}}%
\pgfpathcurveto{\pgfqpoint{0.984045in}{0.710855in}}{\pgfqpoint{0.994644in}{0.706464in}}{\pgfqpoint{1.005694in}{0.706464in}}%
\pgfpathclose%
\pgfusepath{stroke,fill}%
\end{pgfscope}%
\begin{pgfscope}%
\pgfpathrectangle{\pgfqpoint{0.375000in}{0.330000in}}{\pgfqpoint{2.325000in}{2.310000in}}%
\pgfusepath{clip}%
\pgfsetbuttcap%
\pgfsetroundjoin%
\definecolor{currentfill}{rgb}{0.000000,0.000000,0.000000}%
\pgfsetfillcolor{currentfill}%
\pgfsetlinewidth{1.003750pt}%
\definecolor{currentstroke}{rgb}{0.000000,0.000000,0.000000}%
\pgfsetstrokecolor{currentstroke}%
\pgfsetdash{}{0pt}%
\pgfpathmoveto{\pgfqpoint{1.005694in}{0.706464in}}%
\pgfpathcurveto{\pgfqpoint{1.016744in}{0.706464in}}{\pgfqpoint{1.027343in}{0.710855in}}{\pgfqpoint{1.035157in}{0.718668in}}%
\pgfpathcurveto{\pgfqpoint{1.042971in}{0.726482in}}{\pgfqpoint{1.047361in}{0.737081in}}{\pgfqpoint{1.047361in}{0.748131in}}%
\pgfpathcurveto{\pgfqpoint{1.047361in}{0.759181in}}{\pgfqpoint{1.042971in}{0.769780in}}{\pgfqpoint{1.035157in}{0.777594in}}%
\pgfpathcurveto{\pgfqpoint{1.027343in}{0.785407in}}{\pgfqpoint{1.016744in}{0.789798in}}{\pgfqpoint{1.005694in}{0.789798in}}%
\pgfpathcurveto{\pgfqpoint{0.994644in}{0.789798in}}{\pgfqpoint{0.984045in}{0.785407in}}{\pgfqpoint{0.976232in}{0.777594in}}%
\pgfpathcurveto{\pgfqpoint{0.968418in}{0.769780in}}{\pgfqpoint{0.964028in}{0.759181in}}{\pgfqpoint{0.964028in}{0.748131in}}%
\pgfpathcurveto{\pgfqpoint{0.964028in}{0.737081in}}{\pgfqpoint{0.968418in}{0.726482in}}{\pgfqpoint{0.976232in}{0.718668in}}%
\pgfpathcurveto{\pgfqpoint{0.984045in}{0.710855in}}{\pgfqpoint{0.994644in}{0.706464in}}{\pgfqpoint{1.005694in}{0.706464in}}%
\pgfpathclose%
\pgfusepath{stroke,fill}%
\end{pgfscope}%
\begin{pgfscope}%
\pgfpathrectangle{\pgfqpoint{0.375000in}{0.330000in}}{\pgfqpoint{2.325000in}{2.310000in}}%
\pgfusepath{clip}%
\pgfsetbuttcap%
\pgfsetroundjoin%
\definecolor{currentfill}{rgb}{0.000000,0.000000,0.000000}%
\pgfsetfillcolor{currentfill}%
\pgfsetlinewidth{1.003750pt}%
\definecolor{currentstroke}{rgb}{0.000000,0.000000,0.000000}%
\pgfsetstrokecolor{currentstroke}%
\pgfsetdash{}{0pt}%
\pgfpathmoveto{\pgfqpoint{1.005694in}{0.706464in}}%
\pgfpathcurveto{\pgfqpoint{1.016744in}{0.706464in}}{\pgfqpoint{1.027343in}{0.710855in}}{\pgfqpoint{1.035157in}{0.718668in}}%
\pgfpathcurveto{\pgfqpoint{1.042971in}{0.726482in}}{\pgfqpoint{1.047361in}{0.737081in}}{\pgfqpoint{1.047361in}{0.748131in}}%
\pgfpathcurveto{\pgfqpoint{1.047361in}{0.759181in}}{\pgfqpoint{1.042971in}{0.769780in}}{\pgfqpoint{1.035157in}{0.777594in}}%
\pgfpathcurveto{\pgfqpoint{1.027343in}{0.785407in}}{\pgfqpoint{1.016744in}{0.789798in}}{\pgfqpoint{1.005694in}{0.789798in}}%
\pgfpathcurveto{\pgfqpoint{0.994644in}{0.789798in}}{\pgfqpoint{0.984045in}{0.785407in}}{\pgfqpoint{0.976232in}{0.777594in}}%
\pgfpathcurveto{\pgfqpoint{0.968418in}{0.769780in}}{\pgfqpoint{0.964028in}{0.759181in}}{\pgfqpoint{0.964028in}{0.748131in}}%
\pgfpathcurveto{\pgfqpoint{0.964028in}{0.737081in}}{\pgfqpoint{0.968418in}{0.726482in}}{\pgfqpoint{0.976232in}{0.718668in}}%
\pgfpathcurveto{\pgfqpoint{0.984045in}{0.710855in}}{\pgfqpoint{0.994644in}{0.706464in}}{\pgfqpoint{1.005694in}{0.706464in}}%
\pgfpathclose%
\pgfusepath{stroke,fill}%
\end{pgfscope}%
\begin{pgfscope}%
\pgfpathrectangle{\pgfqpoint{0.375000in}{0.330000in}}{\pgfqpoint{2.325000in}{2.310000in}}%
\pgfusepath{clip}%
\pgfsetbuttcap%
\pgfsetroundjoin%
\definecolor{currentfill}{rgb}{0.000000,0.000000,0.000000}%
\pgfsetfillcolor{currentfill}%
\pgfsetlinewidth{1.003750pt}%
\definecolor{currentstroke}{rgb}{0.000000,0.000000,0.000000}%
\pgfsetstrokecolor{currentstroke}%
\pgfsetdash{}{0pt}%
\pgfpathmoveto{\pgfqpoint{1.005694in}{0.706464in}}%
\pgfpathcurveto{\pgfqpoint{1.016744in}{0.706464in}}{\pgfqpoint{1.027343in}{0.710855in}}{\pgfqpoint{1.035157in}{0.718668in}}%
\pgfpathcurveto{\pgfqpoint{1.042971in}{0.726482in}}{\pgfqpoint{1.047361in}{0.737081in}}{\pgfqpoint{1.047361in}{0.748131in}}%
\pgfpathcurveto{\pgfqpoint{1.047361in}{0.759181in}}{\pgfqpoint{1.042971in}{0.769780in}}{\pgfqpoint{1.035157in}{0.777594in}}%
\pgfpathcurveto{\pgfqpoint{1.027343in}{0.785407in}}{\pgfqpoint{1.016744in}{0.789798in}}{\pgfqpoint{1.005694in}{0.789798in}}%
\pgfpathcurveto{\pgfqpoint{0.994644in}{0.789798in}}{\pgfqpoint{0.984045in}{0.785407in}}{\pgfqpoint{0.976232in}{0.777594in}}%
\pgfpathcurveto{\pgfqpoint{0.968418in}{0.769780in}}{\pgfqpoint{0.964028in}{0.759181in}}{\pgfqpoint{0.964028in}{0.748131in}}%
\pgfpathcurveto{\pgfqpoint{0.964028in}{0.737081in}}{\pgfqpoint{0.968418in}{0.726482in}}{\pgfqpoint{0.976232in}{0.718668in}}%
\pgfpathcurveto{\pgfqpoint{0.984045in}{0.710855in}}{\pgfqpoint{0.994644in}{0.706464in}}{\pgfqpoint{1.005694in}{0.706464in}}%
\pgfpathclose%
\pgfusepath{stroke,fill}%
\end{pgfscope}%
\begin{pgfscope}%
\pgfpathrectangle{\pgfqpoint{0.375000in}{0.330000in}}{\pgfqpoint{2.325000in}{2.310000in}}%
\pgfusepath{clip}%
\pgfsetbuttcap%
\pgfsetroundjoin%
\definecolor{currentfill}{rgb}{0.000000,0.000000,0.000000}%
\pgfsetfillcolor{currentfill}%
\pgfsetlinewidth{1.003750pt}%
\definecolor{currentstroke}{rgb}{0.000000,0.000000,0.000000}%
\pgfsetstrokecolor{currentstroke}%
\pgfsetdash{}{0pt}%
\pgfpathmoveto{\pgfqpoint{1.005694in}{0.706464in}}%
\pgfpathcurveto{\pgfqpoint{1.016744in}{0.706464in}}{\pgfqpoint{1.027343in}{0.710855in}}{\pgfqpoint{1.035157in}{0.718668in}}%
\pgfpathcurveto{\pgfqpoint{1.042971in}{0.726482in}}{\pgfqpoint{1.047361in}{0.737081in}}{\pgfqpoint{1.047361in}{0.748131in}}%
\pgfpathcurveto{\pgfqpoint{1.047361in}{0.759181in}}{\pgfqpoint{1.042971in}{0.769780in}}{\pgfqpoint{1.035157in}{0.777594in}}%
\pgfpathcurveto{\pgfqpoint{1.027343in}{0.785407in}}{\pgfqpoint{1.016744in}{0.789798in}}{\pgfqpoint{1.005694in}{0.789798in}}%
\pgfpathcurveto{\pgfqpoint{0.994644in}{0.789798in}}{\pgfqpoint{0.984045in}{0.785407in}}{\pgfqpoint{0.976232in}{0.777594in}}%
\pgfpathcurveto{\pgfqpoint{0.968418in}{0.769780in}}{\pgfqpoint{0.964028in}{0.759181in}}{\pgfqpoint{0.964028in}{0.748131in}}%
\pgfpathcurveto{\pgfqpoint{0.964028in}{0.737081in}}{\pgfqpoint{0.968418in}{0.726482in}}{\pgfqpoint{0.976232in}{0.718668in}}%
\pgfpathcurveto{\pgfqpoint{0.984045in}{0.710855in}}{\pgfqpoint{0.994644in}{0.706464in}}{\pgfqpoint{1.005694in}{0.706464in}}%
\pgfpathclose%
\pgfusepath{stroke,fill}%
\end{pgfscope}%
\begin{pgfscope}%
\pgfpathrectangle{\pgfqpoint{0.375000in}{0.330000in}}{\pgfqpoint{2.325000in}{2.310000in}}%
\pgfusepath{clip}%
\pgfsetbuttcap%
\pgfsetroundjoin%
\definecolor{currentfill}{rgb}{0.000000,0.000000,0.000000}%
\pgfsetfillcolor{currentfill}%
\pgfsetlinewidth{1.003750pt}%
\definecolor{currentstroke}{rgb}{0.000000,0.000000,0.000000}%
\pgfsetstrokecolor{currentstroke}%
\pgfsetdash{}{0pt}%
\pgfpathmoveto{\pgfqpoint{1.005694in}{0.706464in}}%
\pgfpathcurveto{\pgfqpoint{1.016744in}{0.706464in}}{\pgfqpoint{1.027343in}{0.710855in}}{\pgfqpoint{1.035157in}{0.718668in}}%
\pgfpathcurveto{\pgfqpoint{1.042971in}{0.726482in}}{\pgfqpoint{1.047361in}{0.737081in}}{\pgfqpoint{1.047361in}{0.748131in}}%
\pgfpathcurveto{\pgfqpoint{1.047361in}{0.759181in}}{\pgfqpoint{1.042971in}{0.769780in}}{\pgfqpoint{1.035157in}{0.777594in}}%
\pgfpathcurveto{\pgfqpoint{1.027343in}{0.785407in}}{\pgfqpoint{1.016744in}{0.789798in}}{\pgfqpoint{1.005694in}{0.789798in}}%
\pgfpathcurveto{\pgfqpoint{0.994644in}{0.789798in}}{\pgfqpoint{0.984045in}{0.785407in}}{\pgfqpoint{0.976232in}{0.777594in}}%
\pgfpathcurveto{\pgfqpoint{0.968418in}{0.769780in}}{\pgfqpoint{0.964028in}{0.759181in}}{\pgfqpoint{0.964028in}{0.748131in}}%
\pgfpathcurveto{\pgfqpoint{0.964028in}{0.737081in}}{\pgfqpoint{0.968418in}{0.726482in}}{\pgfqpoint{0.976232in}{0.718668in}}%
\pgfpathcurveto{\pgfqpoint{0.984045in}{0.710855in}}{\pgfqpoint{0.994644in}{0.706464in}}{\pgfqpoint{1.005694in}{0.706464in}}%
\pgfpathclose%
\pgfusepath{stroke,fill}%
\end{pgfscope}%
\begin{pgfscope}%
\pgfpathrectangle{\pgfqpoint{0.375000in}{0.330000in}}{\pgfqpoint{2.325000in}{2.310000in}}%
\pgfusepath{clip}%
\pgfsetbuttcap%
\pgfsetroundjoin%
\definecolor{currentfill}{rgb}{0.000000,0.000000,0.000000}%
\pgfsetfillcolor{currentfill}%
\pgfsetlinewidth{1.003750pt}%
\definecolor{currentstroke}{rgb}{0.000000,0.000000,0.000000}%
\pgfsetstrokecolor{currentstroke}%
\pgfsetdash{}{0pt}%
\pgfpathmoveto{\pgfqpoint{1.005694in}{0.758495in}}%
\pgfpathcurveto{\pgfqpoint{1.016744in}{0.758495in}}{\pgfqpoint{1.027343in}{0.762885in}}{\pgfqpoint{1.035157in}{0.770699in}}%
\pgfpathcurveto{\pgfqpoint{1.042971in}{0.778513in}}{\pgfqpoint{1.047361in}{0.789112in}}{\pgfqpoint{1.047361in}{0.800162in}}%
\pgfpathcurveto{\pgfqpoint{1.047361in}{0.811212in}}{\pgfqpoint{1.042971in}{0.821811in}}{\pgfqpoint{1.035157in}{0.829625in}}%
\pgfpathcurveto{\pgfqpoint{1.027343in}{0.837438in}}{\pgfqpoint{1.016744in}{0.841828in}}{\pgfqpoint{1.005694in}{0.841828in}}%
\pgfpathcurveto{\pgfqpoint{0.994644in}{0.841828in}}{\pgfqpoint{0.984045in}{0.837438in}}{\pgfqpoint{0.976232in}{0.829625in}}%
\pgfpathcurveto{\pgfqpoint{0.968418in}{0.821811in}}{\pgfqpoint{0.964028in}{0.811212in}}{\pgfqpoint{0.964028in}{0.800162in}}%
\pgfpathcurveto{\pgfqpoint{0.964028in}{0.789112in}}{\pgfqpoint{0.968418in}{0.778513in}}{\pgfqpoint{0.976232in}{0.770699in}}%
\pgfpathcurveto{\pgfqpoint{0.984045in}{0.762885in}}{\pgfqpoint{0.994644in}{0.758495in}}{\pgfqpoint{1.005694in}{0.758495in}}%
\pgfpathclose%
\pgfusepath{stroke,fill}%
\end{pgfscope}%
\begin{pgfscope}%
\pgfpathrectangle{\pgfqpoint{0.375000in}{0.330000in}}{\pgfqpoint{2.325000in}{2.310000in}}%
\pgfusepath{clip}%
\pgfsetbuttcap%
\pgfsetroundjoin%
\definecolor{currentfill}{rgb}{0.000000,0.000000,0.000000}%
\pgfsetfillcolor{currentfill}%
\pgfsetlinewidth{1.003750pt}%
\definecolor{currentstroke}{rgb}{0.000000,0.000000,0.000000}%
\pgfsetstrokecolor{currentstroke}%
\pgfsetdash{}{0pt}%
\pgfpathmoveto{\pgfqpoint{1.005694in}{0.758495in}}%
\pgfpathcurveto{\pgfqpoint{1.016744in}{0.758495in}}{\pgfqpoint{1.027343in}{0.762885in}}{\pgfqpoint{1.035157in}{0.770699in}}%
\pgfpathcurveto{\pgfqpoint{1.042971in}{0.778513in}}{\pgfqpoint{1.047361in}{0.789112in}}{\pgfqpoint{1.047361in}{0.800162in}}%
\pgfpathcurveto{\pgfqpoint{1.047361in}{0.811212in}}{\pgfqpoint{1.042971in}{0.821811in}}{\pgfqpoint{1.035157in}{0.829625in}}%
\pgfpathcurveto{\pgfqpoint{1.027343in}{0.837438in}}{\pgfqpoint{1.016744in}{0.841828in}}{\pgfqpoint{1.005694in}{0.841828in}}%
\pgfpathcurveto{\pgfqpoint{0.994644in}{0.841828in}}{\pgfqpoint{0.984045in}{0.837438in}}{\pgfqpoint{0.976232in}{0.829625in}}%
\pgfpathcurveto{\pgfqpoint{0.968418in}{0.821811in}}{\pgfqpoint{0.964028in}{0.811212in}}{\pgfqpoint{0.964028in}{0.800162in}}%
\pgfpathcurveto{\pgfqpoint{0.964028in}{0.789112in}}{\pgfqpoint{0.968418in}{0.778513in}}{\pgfqpoint{0.976232in}{0.770699in}}%
\pgfpathcurveto{\pgfqpoint{0.984045in}{0.762885in}}{\pgfqpoint{0.994644in}{0.758495in}}{\pgfqpoint{1.005694in}{0.758495in}}%
\pgfpathclose%
\pgfusepath{stroke,fill}%
\end{pgfscope}%
\begin{pgfscope}%
\pgfpathrectangle{\pgfqpoint{0.375000in}{0.330000in}}{\pgfqpoint{2.325000in}{2.310000in}}%
\pgfusepath{clip}%
\pgfsetbuttcap%
\pgfsetroundjoin%
\definecolor{currentfill}{rgb}{0.000000,0.000000,0.000000}%
\pgfsetfillcolor{currentfill}%
\pgfsetlinewidth{1.003750pt}%
\definecolor{currentstroke}{rgb}{0.000000,0.000000,0.000000}%
\pgfsetstrokecolor{currentstroke}%
\pgfsetdash{}{0pt}%
\pgfpathmoveto{\pgfqpoint{1.005694in}{0.706464in}}%
\pgfpathcurveto{\pgfqpoint{1.016744in}{0.706464in}}{\pgfqpoint{1.027343in}{0.710855in}}{\pgfqpoint{1.035157in}{0.718668in}}%
\pgfpathcurveto{\pgfqpoint{1.042971in}{0.726482in}}{\pgfqpoint{1.047361in}{0.737081in}}{\pgfqpoint{1.047361in}{0.748131in}}%
\pgfpathcurveto{\pgfqpoint{1.047361in}{0.759181in}}{\pgfqpoint{1.042971in}{0.769780in}}{\pgfqpoint{1.035157in}{0.777594in}}%
\pgfpathcurveto{\pgfqpoint{1.027343in}{0.785407in}}{\pgfqpoint{1.016744in}{0.789798in}}{\pgfqpoint{1.005694in}{0.789798in}}%
\pgfpathcurveto{\pgfqpoint{0.994644in}{0.789798in}}{\pgfqpoint{0.984045in}{0.785407in}}{\pgfqpoint{0.976232in}{0.777594in}}%
\pgfpathcurveto{\pgfqpoint{0.968418in}{0.769780in}}{\pgfqpoint{0.964028in}{0.759181in}}{\pgfqpoint{0.964028in}{0.748131in}}%
\pgfpathcurveto{\pgfqpoint{0.964028in}{0.737081in}}{\pgfqpoint{0.968418in}{0.726482in}}{\pgfqpoint{0.976232in}{0.718668in}}%
\pgfpathcurveto{\pgfqpoint{0.984045in}{0.710855in}}{\pgfqpoint{0.994644in}{0.706464in}}{\pgfqpoint{1.005694in}{0.706464in}}%
\pgfpathclose%
\pgfusepath{stroke,fill}%
\end{pgfscope}%
\begin{pgfscope}%
\pgfpathrectangle{\pgfqpoint{0.375000in}{0.330000in}}{\pgfqpoint{2.325000in}{2.310000in}}%
\pgfusepath{clip}%
\pgfsetbuttcap%
\pgfsetroundjoin%
\definecolor{currentfill}{rgb}{0.000000,0.000000,0.000000}%
\pgfsetfillcolor{currentfill}%
\pgfsetlinewidth{1.003750pt}%
\definecolor{currentstroke}{rgb}{0.000000,0.000000,0.000000}%
\pgfsetstrokecolor{currentstroke}%
\pgfsetdash{}{0pt}%
\pgfpathmoveto{\pgfqpoint{1.005694in}{0.758495in}}%
\pgfpathcurveto{\pgfqpoint{1.016744in}{0.758495in}}{\pgfqpoint{1.027343in}{0.762885in}}{\pgfqpoint{1.035157in}{0.770699in}}%
\pgfpathcurveto{\pgfqpoint{1.042971in}{0.778513in}}{\pgfqpoint{1.047361in}{0.789112in}}{\pgfqpoint{1.047361in}{0.800162in}}%
\pgfpathcurveto{\pgfqpoint{1.047361in}{0.811212in}}{\pgfqpoint{1.042971in}{0.821811in}}{\pgfqpoint{1.035157in}{0.829625in}}%
\pgfpathcurveto{\pgfqpoint{1.027343in}{0.837438in}}{\pgfqpoint{1.016744in}{0.841828in}}{\pgfqpoint{1.005694in}{0.841828in}}%
\pgfpathcurveto{\pgfqpoint{0.994644in}{0.841828in}}{\pgfqpoint{0.984045in}{0.837438in}}{\pgfqpoint{0.976232in}{0.829625in}}%
\pgfpathcurveto{\pgfqpoint{0.968418in}{0.821811in}}{\pgfqpoint{0.964028in}{0.811212in}}{\pgfqpoint{0.964028in}{0.800162in}}%
\pgfpathcurveto{\pgfqpoint{0.964028in}{0.789112in}}{\pgfqpoint{0.968418in}{0.778513in}}{\pgfqpoint{0.976232in}{0.770699in}}%
\pgfpathcurveto{\pgfqpoint{0.984045in}{0.762885in}}{\pgfqpoint{0.994644in}{0.758495in}}{\pgfqpoint{1.005694in}{0.758495in}}%
\pgfpathclose%
\pgfusepath{stroke,fill}%
\end{pgfscope}%
\begin{pgfscope}%
\pgfpathrectangle{\pgfqpoint{0.375000in}{0.330000in}}{\pgfqpoint{2.325000in}{2.310000in}}%
\pgfusepath{clip}%
\pgfsetbuttcap%
\pgfsetroundjoin%
\definecolor{currentfill}{rgb}{0.000000,0.000000,0.000000}%
\pgfsetfillcolor{currentfill}%
\pgfsetlinewidth{1.003750pt}%
\definecolor{currentstroke}{rgb}{0.000000,0.000000,0.000000}%
\pgfsetstrokecolor{currentstroke}%
\pgfsetdash{}{0pt}%
\pgfpathmoveto{\pgfqpoint{1.005694in}{0.706464in}}%
\pgfpathcurveto{\pgfqpoint{1.016744in}{0.706464in}}{\pgfqpoint{1.027343in}{0.710855in}}{\pgfqpoint{1.035157in}{0.718668in}}%
\pgfpathcurveto{\pgfqpoint{1.042971in}{0.726482in}}{\pgfqpoint{1.047361in}{0.737081in}}{\pgfqpoint{1.047361in}{0.748131in}}%
\pgfpathcurveto{\pgfqpoint{1.047361in}{0.759181in}}{\pgfqpoint{1.042971in}{0.769780in}}{\pgfqpoint{1.035157in}{0.777594in}}%
\pgfpathcurveto{\pgfqpoint{1.027343in}{0.785407in}}{\pgfqpoint{1.016744in}{0.789798in}}{\pgfqpoint{1.005694in}{0.789798in}}%
\pgfpathcurveto{\pgfqpoint{0.994644in}{0.789798in}}{\pgfqpoint{0.984045in}{0.785407in}}{\pgfqpoint{0.976232in}{0.777594in}}%
\pgfpathcurveto{\pgfqpoint{0.968418in}{0.769780in}}{\pgfqpoint{0.964028in}{0.759181in}}{\pgfqpoint{0.964028in}{0.748131in}}%
\pgfpathcurveto{\pgfqpoint{0.964028in}{0.737081in}}{\pgfqpoint{0.968418in}{0.726482in}}{\pgfqpoint{0.976232in}{0.718668in}}%
\pgfpathcurveto{\pgfqpoint{0.984045in}{0.710855in}}{\pgfqpoint{0.994644in}{0.706464in}}{\pgfqpoint{1.005694in}{0.706464in}}%
\pgfpathclose%
\pgfusepath{stroke,fill}%
\end{pgfscope}%
\begin{pgfscope}%
\pgfpathrectangle{\pgfqpoint{0.375000in}{0.330000in}}{\pgfqpoint{2.325000in}{2.310000in}}%
\pgfusepath{clip}%
\pgfsetbuttcap%
\pgfsetroundjoin%
\definecolor{currentfill}{rgb}{0.000000,0.000000,0.000000}%
\pgfsetfillcolor{currentfill}%
\pgfsetlinewidth{1.003750pt}%
\definecolor{currentstroke}{rgb}{0.000000,0.000000,0.000000}%
\pgfsetstrokecolor{currentstroke}%
\pgfsetdash{}{0pt}%
\pgfpathmoveto{\pgfqpoint{1.005694in}{0.758495in}}%
\pgfpathcurveto{\pgfqpoint{1.016744in}{0.758495in}}{\pgfqpoint{1.027343in}{0.762885in}}{\pgfqpoint{1.035157in}{0.770699in}}%
\pgfpathcurveto{\pgfqpoint{1.042971in}{0.778513in}}{\pgfqpoint{1.047361in}{0.789112in}}{\pgfqpoint{1.047361in}{0.800162in}}%
\pgfpathcurveto{\pgfqpoint{1.047361in}{0.811212in}}{\pgfqpoint{1.042971in}{0.821811in}}{\pgfqpoint{1.035157in}{0.829625in}}%
\pgfpathcurveto{\pgfqpoint{1.027343in}{0.837438in}}{\pgfqpoint{1.016744in}{0.841828in}}{\pgfqpoint{1.005694in}{0.841828in}}%
\pgfpathcurveto{\pgfqpoint{0.994644in}{0.841828in}}{\pgfqpoint{0.984045in}{0.837438in}}{\pgfqpoint{0.976232in}{0.829625in}}%
\pgfpathcurveto{\pgfqpoint{0.968418in}{0.821811in}}{\pgfqpoint{0.964028in}{0.811212in}}{\pgfqpoint{0.964028in}{0.800162in}}%
\pgfpathcurveto{\pgfqpoint{0.964028in}{0.789112in}}{\pgfqpoint{0.968418in}{0.778513in}}{\pgfqpoint{0.976232in}{0.770699in}}%
\pgfpathcurveto{\pgfqpoint{0.984045in}{0.762885in}}{\pgfqpoint{0.994644in}{0.758495in}}{\pgfqpoint{1.005694in}{0.758495in}}%
\pgfpathclose%
\pgfusepath{stroke,fill}%
\end{pgfscope}%
\begin{pgfscope}%
\pgfpathrectangle{\pgfqpoint{0.375000in}{0.330000in}}{\pgfqpoint{2.325000in}{2.310000in}}%
\pgfusepath{clip}%
\pgfsetbuttcap%
\pgfsetroundjoin%
\definecolor{currentfill}{rgb}{0.000000,0.000000,0.000000}%
\pgfsetfillcolor{currentfill}%
\pgfsetlinewidth{1.003750pt}%
\definecolor{currentstroke}{rgb}{0.000000,0.000000,0.000000}%
\pgfsetstrokecolor{currentstroke}%
\pgfsetdash{}{0pt}%
\pgfpathmoveto{\pgfqpoint{1.005694in}{0.706464in}}%
\pgfpathcurveto{\pgfqpoint{1.016744in}{0.706464in}}{\pgfqpoint{1.027343in}{0.710855in}}{\pgfqpoint{1.035157in}{0.718668in}}%
\pgfpathcurveto{\pgfqpoint{1.042971in}{0.726482in}}{\pgfqpoint{1.047361in}{0.737081in}}{\pgfqpoint{1.047361in}{0.748131in}}%
\pgfpathcurveto{\pgfqpoint{1.047361in}{0.759181in}}{\pgfqpoint{1.042971in}{0.769780in}}{\pgfqpoint{1.035157in}{0.777594in}}%
\pgfpathcurveto{\pgfqpoint{1.027343in}{0.785407in}}{\pgfqpoint{1.016744in}{0.789798in}}{\pgfqpoint{1.005694in}{0.789798in}}%
\pgfpathcurveto{\pgfqpoint{0.994644in}{0.789798in}}{\pgfqpoint{0.984045in}{0.785407in}}{\pgfqpoint{0.976232in}{0.777594in}}%
\pgfpathcurveto{\pgfqpoint{0.968418in}{0.769780in}}{\pgfqpoint{0.964028in}{0.759181in}}{\pgfqpoint{0.964028in}{0.748131in}}%
\pgfpathcurveto{\pgfqpoint{0.964028in}{0.737081in}}{\pgfqpoint{0.968418in}{0.726482in}}{\pgfqpoint{0.976232in}{0.718668in}}%
\pgfpathcurveto{\pgfqpoint{0.984045in}{0.710855in}}{\pgfqpoint{0.994644in}{0.706464in}}{\pgfqpoint{1.005694in}{0.706464in}}%
\pgfpathclose%
\pgfusepath{stroke,fill}%
\end{pgfscope}%
\begin{pgfscope}%
\pgfpathrectangle{\pgfqpoint{0.375000in}{0.330000in}}{\pgfqpoint{2.325000in}{2.310000in}}%
\pgfusepath{clip}%
\pgfsetbuttcap%
\pgfsetroundjoin%
\definecolor{currentfill}{rgb}{0.000000,0.000000,0.000000}%
\pgfsetfillcolor{currentfill}%
\pgfsetlinewidth{1.003750pt}%
\definecolor{currentstroke}{rgb}{0.000000,0.000000,0.000000}%
\pgfsetstrokecolor{currentstroke}%
\pgfsetdash{}{0pt}%
\pgfpathmoveto{\pgfqpoint{1.005694in}{0.758495in}}%
\pgfpathcurveto{\pgfqpoint{1.016744in}{0.758495in}}{\pgfqpoint{1.027343in}{0.762885in}}{\pgfqpoint{1.035157in}{0.770699in}}%
\pgfpathcurveto{\pgfqpoint{1.042971in}{0.778513in}}{\pgfqpoint{1.047361in}{0.789112in}}{\pgfqpoint{1.047361in}{0.800162in}}%
\pgfpathcurveto{\pgfqpoint{1.047361in}{0.811212in}}{\pgfqpoint{1.042971in}{0.821811in}}{\pgfqpoint{1.035157in}{0.829625in}}%
\pgfpathcurveto{\pgfqpoint{1.027343in}{0.837438in}}{\pgfqpoint{1.016744in}{0.841828in}}{\pgfqpoint{1.005694in}{0.841828in}}%
\pgfpathcurveto{\pgfqpoint{0.994644in}{0.841828in}}{\pgfqpoint{0.984045in}{0.837438in}}{\pgfqpoint{0.976232in}{0.829625in}}%
\pgfpathcurveto{\pgfqpoint{0.968418in}{0.821811in}}{\pgfqpoint{0.964028in}{0.811212in}}{\pgfqpoint{0.964028in}{0.800162in}}%
\pgfpathcurveto{\pgfqpoint{0.964028in}{0.789112in}}{\pgfqpoint{0.968418in}{0.778513in}}{\pgfqpoint{0.976232in}{0.770699in}}%
\pgfpathcurveto{\pgfqpoint{0.984045in}{0.762885in}}{\pgfqpoint{0.994644in}{0.758495in}}{\pgfqpoint{1.005694in}{0.758495in}}%
\pgfpathclose%
\pgfusepath{stroke,fill}%
\end{pgfscope}%
\begin{pgfscope}%
\pgfpathrectangle{\pgfqpoint{0.375000in}{0.330000in}}{\pgfqpoint{2.325000in}{2.310000in}}%
\pgfusepath{clip}%
\pgfsetbuttcap%
\pgfsetroundjoin%
\definecolor{currentfill}{rgb}{0.000000,0.000000,0.000000}%
\pgfsetfillcolor{currentfill}%
\pgfsetlinewidth{1.003750pt}%
\definecolor{currentstroke}{rgb}{0.000000,0.000000,0.000000}%
\pgfsetstrokecolor{currentstroke}%
\pgfsetdash{}{0pt}%
\pgfpathmoveto{\pgfqpoint{1.005694in}{0.758495in}}%
\pgfpathcurveto{\pgfqpoint{1.016744in}{0.758495in}}{\pgfqpoint{1.027343in}{0.762885in}}{\pgfqpoint{1.035157in}{0.770699in}}%
\pgfpathcurveto{\pgfqpoint{1.042971in}{0.778513in}}{\pgfqpoint{1.047361in}{0.789112in}}{\pgfqpoint{1.047361in}{0.800162in}}%
\pgfpathcurveto{\pgfqpoint{1.047361in}{0.811212in}}{\pgfqpoint{1.042971in}{0.821811in}}{\pgfqpoint{1.035157in}{0.829625in}}%
\pgfpathcurveto{\pgfqpoint{1.027343in}{0.837438in}}{\pgfqpoint{1.016744in}{0.841828in}}{\pgfqpoint{1.005694in}{0.841828in}}%
\pgfpathcurveto{\pgfqpoint{0.994644in}{0.841828in}}{\pgfqpoint{0.984045in}{0.837438in}}{\pgfqpoint{0.976232in}{0.829625in}}%
\pgfpathcurveto{\pgfqpoint{0.968418in}{0.821811in}}{\pgfqpoint{0.964028in}{0.811212in}}{\pgfqpoint{0.964028in}{0.800162in}}%
\pgfpathcurveto{\pgfqpoint{0.964028in}{0.789112in}}{\pgfqpoint{0.968418in}{0.778513in}}{\pgfqpoint{0.976232in}{0.770699in}}%
\pgfpathcurveto{\pgfqpoint{0.984045in}{0.762885in}}{\pgfqpoint{0.994644in}{0.758495in}}{\pgfqpoint{1.005694in}{0.758495in}}%
\pgfpathclose%
\pgfusepath{stroke,fill}%
\end{pgfscope}%
\begin{pgfscope}%
\pgfpathrectangle{\pgfqpoint{0.375000in}{0.330000in}}{\pgfqpoint{2.325000in}{2.310000in}}%
\pgfusepath{clip}%
\pgfsetbuttcap%
\pgfsetroundjoin%
\definecolor{currentfill}{rgb}{0.000000,0.000000,0.000000}%
\pgfsetfillcolor{currentfill}%
\pgfsetlinewidth{1.003750pt}%
\definecolor{currentstroke}{rgb}{0.000000,0.000000,0.000000}%
\pgfsetstrokecolor{currentstroke}%
\pgfsetdash{}{0pt}%
\pgfpathmoveto{\pgfqpoint{1.005694in}{0.758495in}}%
\pgfpathcurveto{\pgfqpoint{1.016744in}{0.758495in}}{\pgfqpoint{1.027343in}{0.762885in}}{\pgfqpoint{1.035157in}{0.770699in}}%
\pgfpathcurveto{\pgfqpoint{1.042971in}{0.778513in}}{\pgfqpoint{1.047361in}{0.789112in}}{\pgfqpoint{1.047361in}{0.800162in}}%
\pgfpathcurveto{\pgfqpoint{1.047361in}{0.811212in}}{\pgfqpoint{1.042971in}{0.821811in}}{\pgfqpoint{1.035157in}{0.829625in}}%
\pgfpathcurveto{\pgfqpoint{1.027343in}{0.837438in}}{\pgfqpoint{1.016744in}{0.841828in}}{\pgfqpoint{1.005694in}{0.841828in}}%
\pgfpathcurveto{\pgfqpoint{0.994644in}{0.841828in}}{\pgfqpoint{0.984045in}{0.837438in}}{\pgfqpoint{0.976232in}{0.829625in}}%
\pgfpathcurveto{\pgfqpoint{0.968418in}{0.821811in}}{\pgfqpoint{0.964028in}{0.811212in}}{\pgfqpoint{0.964028in}{0.800162in}}%
\pgfpathcurveto{\pgfqpoint{0.964028in}{0.789112in}}{\pgfqpoint{0.968418in}{0.778513in}}{\pgfqpoint{0.976232in}{0.770699in}}%
\pgfpathcurveto{\pgfqpoint{0.984045in}{0.762885in}}{\pgfqpoint{0.994644in}{0.758495in}}{\pgfqpoint{1.005694in}{0.758495in}}%
\pgfpathclose%
\pgfusepath{stroke,fill}%
\end{pgfscope}%
\begin{pgfscope}%
\pgfpathrectangle{\pgfqpoint{0.375000in}{0.330000in}}{\pgfqpoint{2.325000in}{2.310000in}}%
\pgfusepath{clip}%
\pgfsetbuttcap%
\pgfsetroundjoin%
\definecolor{currentfill}{rgb}{0.000000,0.000000,0.000000}%
\pgfsetfillcolor{currentfill}%
\pgfsetlinewidth{1.003750pt}%
\definecolor{currentstroke}{rgb}{0.000000,0.000000,0.000000}%
\pgfsetstrokecolor{currentstroke}%
\pgfsetdash{}{0pt}%
\pgfpathmoveto{\pgfqpoint{1.005694in}{0.706464in}}%
\pgfpathcurveto{\pgfqpoint{1.016744in}{0.706464in}}{\pgfqpoint{1.027343in}{0.710855in}}{\pgfqpoint{1.035157in}{0.718668in}}%
\pgfpathcurveto{\pgfqpoint{1.042971in}{0.726482in}}{\pgfqpoint{1.047361in}{0.737081in}}{\pgfqpoint{1.047361in}{0.748131in}}%
\pgfpathcurveto{\pgfqpoint{1.047361in}{0.759181in}}{\pgfqpoint{1.042971in}{0.769780in}}{\pgfqpoint{1.035157in}{0.777594in}}%
\pgfpathcurveto{\pgfqpoint{1.027343in}{0.785407in}}{\pgfqpoint{1.016744in}{0.789798in}}{\pgfqpoint{1.005694in}{0.789798in}}%
\pgfpathcurveto{\pgfqpoint{0.994644in}{0.789798in}}{\pgfqpoint{0.984045in}{0.785407in}}{\pgfqpoint{0.976232in}{0.777594in}}%
\pgfpathcurveto{\pgfqpoint{0.968418in}{0.769780in}}{\pgfqpoint{0.964028in}{0.759181in}}{\pgfqpoint{0.964028in}{0.748131in}}%
\pgfpathcurveto{\pgfqpoint{0.964028in}{0.737081in}}{\pgfqpoint{0.968418in}{0.726482in}}{\pgfqpoint{0.976232in}{0.718668in}}%
\pgfpathcurveto{\pgfqpoint{0.984045in}{0.710855in}}{\pgfqpoint{0.994644in}{0.706464in}}{\pgfqpoint{1.005694in}{0.706464in}}%
\pgfpathclose%
\pgfusepath{stroke,fill}%
\end{pgfscope}%
\begin{pgfscope}%
\pgfpathrectangle{\pgfqpoint{0.375000in}{0.330000in}}{\pgfqpoint{2.325000in}{2.310000in}}%
\pgfusepath{clip}%
\pgfsetbuttcap%
\pgfsetroundjoin%
\definecolor{currentfill}{rgb}{0.000000,0.000000,0.000000}%
\pgfsetfillcolor{currentfill}%
\pgfsetlinewidth{1.003750pt}%
\definecolor{currentstroke}{rgb}{0.000000,0.000000,0.000000}%
\pgfsetstrokecolor{currentstroke}%
\pgfsetdash{}{0pt}%
\pgfpathmoveto{\pgfqpoint{1.005694in}{0.706464in}}%
\pgfpathcurveto{\pgfqpoint{1.016744in}{0.706464in}}{\pgfqpoint{1.027343in}{0.710855in}}{\pgfqpoint{1.035157in}{0.718668in}}%
\pgfpathcurveto{\pgfqpoint{1.042971in}{0.726482in}}{\pgfqpoint{1.047361in}{0.737081in}}{\pgfqpoint{1.047361in}{0.748131in}}%
\pgfpathcurveto{\pgfqpoint{1.047361in}{0.759181in}}{\pgfqpoint{1.042971in}{0.769780in}}{\pgfqpoint{1.035157in}{0.777594in}}%
\pgfpathcurveto{\pgfqpoint{1.027343in}{0.785407in}}{\pgfqpoint{1.016744in}{0.789798in}}{\pgfqpoint{1.005694in}{0.789798in}}%
\pgfpathcurveto{\pgfqpoint{0.994644in}{0.789798in}}{\pgfqpoint{0.984045in}{0.785407in}}{\pgfqpoint{0.976232in}{0.777594in}}%
\pgfpathcurveto{\pgfqpoint{0.968418in}{0.769780in}}{\pgfqpoint{0.964028in}{0.759181in}}{\pgfqpoint{0.964028in}{0.748131in}}%
\pgfpathcurveto{\pgfqpoint{0.964028in}{0.737081in}}{\pgfqpoint{0.968418in}{0.726482in}}{\pgfqpoint{0.976232in}{0.718668in}}%
\pgfpathcurveto{\pgfqpoint{0.984045in}{0.710855in}}{\pgfqpoint{0.994644in}{0.706464in}}{\pgfqpoint{1.005694in}{0.706464in}}%
\pgfpathclose%
\pgfusepath{stroke,fill}%
\end{pgfscope}%
\begin{pgfscope}%
\pgfpathrectangle{\pgfqpoint{0.375000in}{0.330000in}}{\pgfqpoint{2.325000in}{2.310000in}}%
\pgfusepath{clip}%
\pgfsetbuttcap%
\pgfsetroundjoin%
\definecolor{currentfill}{rgb}{0.000000,0.000000,0.000000}%
\pgfsetfillcolor{currentfill}%
\pgfsetlinewidth{1.003750pt}%
\definecolor{currentstroke}{rgb}{0.000000,0.000000,0.000000}%
\pgfsetstrokecolor{currentstroke}%
\pgfsetdash{}{0pt}%
\pgfpathmoveto{\pgfqpoint{1.005694in}{0.706464in}}%
\pgfpathcurveto{\pgfqpoint{1.016744in}{0.706464in}}{\pgfqpoint{1.027343in}{0.710855in}}{\pgfqpoint{1.035157in}{0.718668in}}%
\pgfpathcurveto{\pgfqpoint{1.042971in}{0.726482in}}{\pgfqpoint{1.047361in}{0.737081in}}{\pgfqpoint{1.047361in}{0.748131in}}%
\pgfpathcurveto{\pgfqpoint{1.047361in}{0.759181in}}{\pgfqpoint{1.042971in}{0.769780in}}{\pgfqpoint{1.035157in}{0.777594in}}%
\pgfpathcurveto{\pgfqpoint{1.027343in}{0.785407in}}{\pgfqpoint{1.016744in}{0.789798in}}{\pgfqpoint{1.005694in}{0.789798in}}%
\pgfpathcurveto{\pgfqpoint{0.994644in}{0.789798in}}{\pgfqpoint{0.984045in}{0.785407in}}{\pgfqpoint{0.976232in}{0.777594in}}%
\pgfpathcurveto{\pgfqpoint{0.968418in}{0.769780in}}{\pgfqpoint{0.964028in}{0.759181in}}{\pgfqpoint{0.964028in}{0.748131in}}%
\pgfpathcurveto{\pgfqpoint{0.964028in}{0.737081in}}{\pgfqpoint{0.968418in}{0.726482in}}{\pgfqpoint{0.976232in}{0.718668in}}%
\pgfpathcurveto{\pgfqpoint{0.984045in}{0.710855in}}{\pgfqpoint{0.994644in}{0.706464in}}{\pgfqpoint{1.005694in}{0.706464in}}%
\pgfpathclose%
\pgfusepath{stroke,fill}%
\end{pgfscope}%
\begin{pgfscope}%
\pgfpathrectangle{\pgfqpoint{0.375000in}{0.330000in}}{\pgfqpoint{2.325000in}{2.310000in}}%
\pgfusepath{clip}%
\pgfsetbuttcap%
\pgfsetroundjoin%
\definecolor{currentfill}{rgb}{0.000000,0.000000,0.000000}%
\pgfsetfillcolor{currentfill}%
\pgfsetlinewidth{1.003750pt}%
\definecolor{currentstroke}{rgb}{0.000000,0.000000,0.000000}%
\pgfsetstrokecolor{currentstroke}%
\pgfsetdash{}{0pt}%
\pgfpathmoveto{\pgfqpoint{1.005694in}{0.706464in}}%
\pgfpathcurveto{\pgfqpoint{1.016744in}{0.706464in}}{\pgfqpoint{1.027343in}{0.710855in}}{\pgfqpoint{1.035157in}{0.718668in}}%
\pgfpathcurveto{\pgfqpoint{1.042971in}{0.726482in}}{\pgfqpoint{1.047361in}{0.737081in}}{\pgfqpoint{1.047361in}{0.748131in}}%
\pgfpathcurveto{\pgfqpoint{1.047361in}{0.759181in}}{\pgfqpoint{1.042971in}{0.769780in}}{\pgfqpoint{1.035157in}{0.777594in}}%
\pgfpathcurveto{\pgfqpoint{1.027343in}{0.785407in}}{\pgfqpoint{1.016744in}{0.789798in}}{\pgfqpoint{1.005694in}{0.789798in}}%
\pgfpathcurveto{\pgfqpoint{0.994644in}{0.789798in}}{\pgfqpoint{0.984045in}{0.785407in}}{\pgfqpoint{0.976232in}{0.777594in}}%
\pgfpathcurveto{\pgfqpoint{0.968418in}{0.769780in}}{\pgfqpoint{0.964028in}{0.759181in}}{\pgfqpoint{0.964028in}{0.748131in}}%
\pgfpathcurveto{\pgfqpoint{0.964028in}{0.737081in}}{\pgfqpoint{0.968418in}{0.726482in}}{\pgfqpoint{0.976232in}{0.718668in}}%
\pgfpathcurveto{\pgfqpoint{0.984045in}{0.710855in}}{\pgfqpoint{0.994644in}{0.706464in}}{\pgfqpoint{1.005694in}{0.706464in}}%
\pgfpathclose%
\pgfusepath{stroke,fill}%
\end{pgfscope}%
\begin{pgfscope}%
\pgfpathrectangle{\pgfqpoint{0.375000in}{0.330000in}}{\pgfqpoint{2.325000in}{2.310000in}}%
\pgfusepath{clip}%
\pgfsetbuttcap%
\pgfsetroundjoin%
\definecolor{currentfill}{rgb}{0.000000,0.000000,0.000000}%
\pgfsetfillcolor{currentfill}%
\pgfsetlinewidth{1.003750pt}%
\definecolor{currentstroke}{rgb}{0.000000,0.000000,0.000000}%
\pgfsetstrokecolor{currentstroke}%
\pgfsetdash{}{0pt}%
\pgfpathmoveto{\pgfqpoint{1.005694in}{0.706464in}}%
\pgfpathcurveto{\pgfqpoint{1.016744in}{0.706464in}}{\pgfqpoint{1.027343in}{0.710855in}}{\pgfqpoint{1.035157in}{0.718668in}}%
\pgfpathcurveto{\pgfqpoint{1.042971in}{0.726482in}}{\pgfqpoint{1.047361in}{0.737081in}}{\pgfqpoint{1.047361in}{0.748131in}}%
\pgfpathcurveto{\pgfqpoint{1.047361in}{0.759181in}}{\pgfqpoint{1.042971in}{0.769780in}}{\pgfqpoint{1.035157in}{0.777594in}}%
\pgfpathcurveto{\pgfqpoint{1.027343in}{0.785407in}}{\pgfqpoint{1.016744in}{0.789798in}}{\pgfqpoint{1.005694in}{0.789798in}}%
\pgfpathcurveto{\pgfqpoint{0.994644in}{0.789798in}}{\pgfqpoint{0.984045in}{0.785407in}}{\pgfqpoint{0.976232in}{0.777594in}}%
\pgfpathcurveto{\pgfqpoint{0.968418in}{0.769780in}}{\pgfqpoint{0.964028in}{0.759181in}}{\pgfqpoint{0.964028in}{0.748131in}}%
\pgfpathcurveto{\pgfqpoint{0.964028in}{0.737081in}}{\pgfqpoint{0.968418in}{0.726482in}}{\pgfqpoint{0.976232in}{0.718668in}}%
\pgfpathcurveto{\pgfqpoint{0.984045in}{0.710855in}}{\pgfqpoint{0.994644in}{0.706464in}}{\pgfqpoint{1.005694in}{0.706464in}}%
\pgfpathclose%
\pgfusepath{stroke,fill}%
\end{pgfscope}%
\begin{pgfscope}%
\pgfpathrectangle{\pgfqpoint{0.375000in}{0.330000in}}{\pgfqpoint{2.325000in}{2.310000in}}%
\pgfusepath{clip}%
\pgfsetbuttcap%
\pgfsetroundjoin%
\definecolor{currentfill}{rgb}{0.000000,0.000000,0.000000}%
\pgfsetfillcolor{currentfill}%
\pgfsetlinewidth{1.003750pt}%
\definecolor{currentstroke}{rgb}{0.000000,0.000000,0.000000}%
\pgfsetstrokecolor{currentstroke}%
\pgfsetdash{}{0pt}%
\pgfpathmoveto{\pgfqpoint{1.005694in}{0.758495in}}%
\pgfpathcurveto{\pgfqpoint{1.016744in}{0.758495in}}{\pgfqpoint{1.027343in}{0.762885in}}{\pgfqpoint{1.035157in}{0.770699in}}%
\pgfpathcurveto{\pgfqpoint{1.042971in}{0.778513in}}{\pgfqpoint{1.047361in}{0.789112in}}{\pgfqpoint{1.047361in}{0.800162in}}%
\pgfpathcurveto{\pgfqpoint{1.047361in}{0.811212in}}{\pgfqpoint{1.042971in}{0.821811in}}{\pgfqpoint{1.035157in}{0.829625in}}%
\pgfpathcurveto{\pgfqpoint{1.027343in}{0.837438in}}{\pgfqpoint{1.016744in}{0.841828in}}{\pgfqpoint{1.005694in}{0.841828in}}%
\pgfpathcurveto{\pgfqpoint{0.994644in}{0.841828in}}{\pgfqpoint{0.984045in}{0.837438in}}{\pgfqpoint{0.976232in}{0.829625in}}%
\pgfpathcurveto{\pgfqpoint{0.968418in}{0.821811in}}{\pgfqpoint{0.964028in}{0.811212in}}{\pgfqpoint{0.964028in}{0.800162in}}%
\pgfpathcurveto{\pgfqpoint{0.964028in}{0.789112in}}{\pgfqpoint{0.968418in}{0.778513in}}{\pgfqpoint{0.976232in}{0.770699in}}%
\pgfpathcurveto{\pgfqpoint{0.984045in}{0.762885in}}{\pgfqpoint{0.994644in}{0.758495in}}{\pgfqpoint{1.005694in}{0.758495in}}%
\pgfpathclose%
\pgfusepath{stroke,fill}%
\end{pgfscope}%
\begin{pgfscope}%
\pgfpathrectangle{\pgfqpoint{0.375000in}{0.330000in}}{\pgfqpoint{2.325000in}{2.310000in}}%
\pgfusepath{clip}%
\pgfsetbuttcap%
\pgfsetroundjoin%
\definecolor{currentfill}{rgb}{0.000000,0.000000,0.000000}%
\pgfsetfillcolor{currentfill}%
\pgfsetlinewidth{1.003750pt}%
\definecolor{currentstroke}{rgb}{0.000000,0.000000,0.000000}%
\pgfsetstrokecolor{currentstroke}%
\pgfsetdash{}{0pt}%
\pgfpathmoveto{\pgfqpoint{1.005694in}{0.706464in}}%
\pgfpathcurveto{\pgfqpoint{1.016744in}{0.706464in}}{\pgfqpoint{1.027343in}{0.710855in}}{\pgfqpoint{1.035157in}{0.718668in}}%
\pgfpathcurveto{\pgfqpoint{1.042971in}{0.726482in}}{\pgfqpoint{1.047361in}{0.737081in}}{\pgfqpoint{1.047361in}{0.748131in}}%
\pgfpathcurveto{\pgfqpoint{1.047361in}{0.759181in}}{\pgfqpoint{1.042971in}{0.769780in}}{\pgfqpoint{1.035157in}{0.777594in}}%
\pgfpathcurveto{\pgfqpoint{1.027343in}{0.785407in}}{\pgfqpoint{1.016744in}{0.789798in}}{\pgfqpoint{1.005694in}{0.789798in}}%
\pgfpathcurveto{\pgfqpoint{0.994644in}{0.789798in}}{\pgfqpoint{0.984045in}{0.785407in}}{\pgfqpoint{0.976232in}{0.777594in}}%
\pgfpathcurveto{\pgfqpoint{0.968418in}{0.769780in}}{\pgfqpoint{0.964028in}{0.759181in}}{\pgfqpoint{0.964028in}{0.748131in}}%
\pgfpathcurveto{\pgfqpoint{0.964028in}{0.737081in}}{\pgfqpoint{0.968418in}{0.726482in}}{\pgfqpoint{0.976232in}{0.718668in}}%
\pgfpathcurveto{\pgfqpoint{0.984045in}{0.710855in}}{\pgfqpoint{0.994644in}{0.706464in}}{\pgfqpoint{1.005694in}{0.706464in}}%
\pgfpathclose%
\pgfusepath{stroke,fill}%
\end{pgfscope}%
\begin{pgfscope}%
\pgfpathrectangle{\pgfqpoint{0.375000in}{0.330000in}}{\pgfqpoint{2.325000in}{2.310000in}}%
\pgfusepath{clip}%
\pgfsetbuttcap%
\pgfsetroundjoin%
\definecolor{currentfill}{rgb}{0.000000,0.000000,0.000000}%
\pgfsetfillcolor{currentfill}%
\pgfsetlinewidth{1.003750pt}%
\definecolor{currentstroke}{rgb}{0.000000,0.000000,0.000000}%
\pgfsetstrokecolor{currentstroke}%
\pgfsetdash{}{0pt}%
\pgfpathmoveto{\pgfqpoint{1.005694in}{0.758495in}}%
\pgfpathcurveto{\pgfqpoint{1.016744in}{0.758495in}}{\pgfqpoint{1.027343in}{0.762885in}}{\pgfqpoint{1.035157in}{0.770699in}}%
\pgfpathcurveto{\pgfqpoint{1.042971in}{0.778513in}}{\pgfqpoint{1.047361in}{0.789112in}}{\pgfqpoint{1.047361in}{0.800162in}}%
\pgfpathcurveto{\pgfqpoint{1.047361in}{0.811212in}}{\pgfqpoint{1.042971in}{0.821811in}}{\pgfqpoint{1.035157in}{0.829625in}}%
\pgfpathcurveto{\pgfqpoint{1.027343in}{0.837438in}}{\pgfqpoint{1.016744in}{0.841828in}}{\pgfqpoint{1.005694in}{0.841828in}}%
\pgfpathcurveto{\pgfqpoint{0.994644in}{0.841828in}}{\pgfqpoint{0.984045in}{0.837438in}}{\pgfqpoint{0.976232in}{0.829625in}}%
\pgfpathcurveto{\pgfqpoint{0.968418in}{0.821811in}}{\pgfqpoint{0.964028in}{0.811212in}}{\pgfqpoint{0.964028in}{0.800162in}}%
\pgfpathcurveto{\pgfqpoint{0.964028in}{0.789112in}}{\pgfqpoint{0.968418in}{0.778513in}}{\pgfqpoint{0.976232in}{0.770699in}}%
\pgfpathcurveto{\pgfqpoint{0.984045in}{0.762885in}}{\pgfqpoint{0.994644in}{0.758495in}}{\pgfqpoint{1.005694in}{0.758495in}}%
\pgfpathclose%
\pgfusepath{stroke,fill}%
\end{pgfscope}%
\begin{pgfscope}%
\pgfpathrectangle{\pgfqpoint{0.375000in}{0.330000in}}{\pgfqpoint{2.325000in}{2.310000in}}%
\pgfusepath{clip}%
\pgfsetbuttcap%
\pgfsetroundjoin%
\definecolor{currentfill}{rgb}{0.000000,0.000000,0.000000}%
\pgfsetfillcolor{currentfill}%
\pgfsetlinewidth{1.003750pt}%
\definecolor{currentstroke}{rgb}{0.000000,0.000000,0.000000}%
\pgfsetstrokecolor{currentstroke}%
\pgfsetdash{}{0pt}%
\pgfpathmoveto{\pgfqpoint{1.005694in}{0.706464in}}%
\pgfpathcurveto{\pgfqpoint{1.016744in}{0.706464in}}{\pgfqpoint{1.027343in}{0.710855in}}{\pgfqpoint{1.035157in}{0.718668in}}%
\pgfpathcurveto{\pgfqpoint{1.042971in}{0.726482in}}{\pgfqpoint{1.047361in}{0.737081in}}{\pgfqpoint{1.047361in}{0.748131in}}%
\pgfpathcurveto{\pgfqpoint{1.047361in}{0.759181in}}{\pgfqpoint{1.042971in}{0.769780in}}{\pgfqpoint{1.035157in}{0.777594in}}%
\pgfpathcurveto{\pgfqpoint{1.027343in}{0.785407in}}{\pgfqpoint{1.016744in}{0.789798in}}{\pgfqpoint{1.005694in}{0.789798in}}%
\pgfpathcurveto{\pgfqpoint{0.994644in}{0.789798in}}{\pgfqpoint{0.984045in}{0.785407in}}{\pgfqpoint{0.976232in}{0.777594in}}%
\pgfpathcurveto{\pgfqpoint{0.968418in}{0.769780in}}{\pgfqpoint{0.964028in}{0.759181in}}{\pgfqpoint{0.964028in}{0.748131in}}%
\pgfpathcurveto{\pgfqpoint{0.964028in}{0.737081in}}{\pgfqpoint{0.968418in}{0.726482in}}{\pgfqpoint{0.976232in}{0.718668in}}%
\pgfpathcurveto{\pgfqpoint{0.984045in}{0.710855in}}{\pgfqpoint{0.994644in}{0.706464in}}{\pgfqpoint{1.005694in}{0.706464in}}%
\pgfpathclose%
\pgfusepath{stroke,fill}%
\end{pgfscope}%
\begin{pgfscope}%
\pgfpathrectangle{\pgfqpoint{0.375000in}{0.330000in}}{\pgfqpoint{2.325000in}{2.310000in}}%
\pgfusepath{clip}%
\pgfsetbuttcap%
\pgfsetroundjoin%
\definecolor{currentfill}{rgb}{0.000000,0.000000,0.000000}%
\pgfsetfillcolor{currentfill}%
\pgfsetlinewidth{1.003750pt}%
\definecolor{currentstroke}{rgb}{0.000000,0.000000,0.000000}%
\pgfsetstrokecolor{currentstroke}%
\pgfsetdash{}{0pt}%
\pgfpathmoveto{\pgfqpoint{1.005694in}{0.758495in}}%
\pgfpathcurveto{\pgfqpoint{1.016744in}{0.758495in}}{\pgfqpoint{1.027343in}{0.762885in}}{\pgfqpoint{1.035157in}{0.770699in}}%
\pgfpathcurveto{\pgfqpoint{1.042971in}{0.778513in}}{\pgfqpoint{1.047361in}{0.789112in}}{\pgfqpoint{1.047361in}{0.800162in}}%
\pgfpathcurveto{\pgfqpoint{1.047361in}{0.811212in}}{\pgfqpoint{1.042971in}{0.821811in}}{\pgfqpoint{1.035157in}{0.829625in}}%
\pgfpathcurveto{\pgfqpoint{1.027343in}{0.837438in}}{\pgfqpoint{1.016744in}{0.841828in}}{\pgfqpoint{1.005694in}{0.841828in}}%
\pgfpathcurveto{\pgfqpoint{0.994644in}{0.841828in}}{\pgfqpoint{0.984045in}{0.837438in}}{\pgfqpoint{0.976232in}{0.829625in}}%
\pgfpathcurveto{\pgfqpoint{0.968418in}{0.821811in}}{\pgfqpoint{0.964028in}{0.811212in}}{\pgfqpoint{0.964028in}{0.800162in}}%
\pgfpathcurveto{\pgfqpoint{0.964028in}{0.789112in}}{\pgfqpoint{0.968418in}{0.778513in}}{\pgfqpoint{0.976232in}{0.770699in}}%
\pgfpathcurveto{\pgfqpoint{0.984045in}{0.762885in}}{\pgfqpoint{0.994644in}{0.758495in}}{\pgfqpoint{1.005694in}{0.758495in}}%
\pgfpathclose%
\pgfusepath{stroke,fill}%
\end{pgfscope}%
\begin{pgfscope}%
\pgfpathrectangle{\pgfqpoint{0.375000in}{0.330000in}}{\pgfqpoint{2.325000in}{2.310000in}}%
\pgfusepath{clip}%
\pgfsetbuttcap%
\pgfsetroundjoin%
\definecolor{currentfill}{rgb}{0.000000,0.000000,0.000000}%
\pgfsetfillcolor{currentfill}%
\pgfsetlinewidth{1.003750pt}%
\definecolor{currentstroke}{rgb}{0.000000,0.000000,0.000000}%
\pgfsetstrokecolor{currentstroke}%
\pgfsetdash{}{0pt}%
\pgfpathmoveto{\pgfqpoint{1.005694in}{0.706464in}}%
\pgfpathcurveto{\pgfqpoint{1.016744in}{0.706464in}}{\pgfqpoint{1.027343in}{0.710855in}}{\pgfqpoint{1.035157in}{0.718668in}}%
\pgfpathcurveto{\pgfqpoint{1.042971in}{0.726482in}}{\pgfqpoint{1.047361in}{0.737081in}}{\pgfqpoint{1.047361in}{0.748131in}}%
\pgfpathcurveto{\pgfqpoint{1.047361in}{0.759181in}}{\pgfqpoint{1.042971in}{0.769780in}}{\pgfqpoint{1.035157in}{0.777594in}}%
\pgfpathcurveto{\pgfqpoint{1.027343in}{0.785407in}}{\pgfqpoint{1.016744in}{0.789798in}}{\pgfqpoint{1.005694in}{0.789798in}}%
\pgfpathcurveto{\pgfqpoint{0.994644in}{0.789798in}}{\pgfqpoint{0.984045in}{0.785407in}}{\pgfqpoint{0.976232in}{0.777594in}}%
\pgfpathcurveto{\pgfqpoint{0.968418in}{0.769780in}}{\pgfqpoint{0.964028in}{0.759181in}}{\pgfqpoint{0.964028in}{0.748131in}}%
\pgfpathcurveto{\pgfqpoint{0.964028in}{0.737081in}}{\pgfqpoint{0.968418in}{0.726482in}}{\pgfqpoint{0.976232in}{0.718668in}}%
\pgfpathcurveto{\pgfqpoint{0.984045in}{0.710855in}}{\pgfqpoint{0.994644in}{0.706464in}}{\pgfqpoint{1.005694in}{0.706464in}}%
\pgfpathclose%
\pgfusepath{stroke,fill}%
\end{pgfscope}%
\begin{pgfscope}%
\pgfpathrectangle{\pgfqpoint{0.375000in}{0.330000in}}{\pgfqpoint{2.325000in}{2.310000in}}%
\pgfusepath{clip}%
\pgfsetbuttcap%
\pgfsetroundjoin%
\definecolor{currentfill}{rgb}{0.000000,0.000000,0.000000}%
\pgfsetfillcolor{currentfill}%
\pgfsetlinewidth{1.003750pt}%
\definecolor{currentstroke}{rgb}{0.000000,0.000000,0.000000}%
\pgfsetstrokecolor{currentstroke}%
\pgfsetdash{}{0pt}%
\pgfpathmoveto{\pgfqpoint{1.005694in}{0.758495in}}%
\pgfpathcurveto{\pgfqpoint{1.016744in}{0.758495in}}{\pgfqpoint{1.027343in}{0.762885in}}{\pgfqpoint{1.035157in}{0.770699in}}%
\pgfpathcurveto{\pgfqpoint{1.042971in}{0.778513in}}{\pgfqpoint{1.047361in}{0.789112in}}{\pgfqpoint{1.047361in}{0.800162in}}%
\pgfpathcurveto{\pgfqpoint{1.047361in}{0.811212in}}{\pgfqpoint{1.042971in}{0.821811in}}{\pgfqpoint{1.035157in}{0.829625in}}%
\pgfpathcurveto{\pgfqpoint{1.027343in}{0.837438in}}{\pgfqpoint{1.016744in}{0.841828in}}{\pgfqpoint{1.005694in}{0.841828in}}%
\pgfpathcurveto{\pgfqpoint{0.994644in}{0.841828in}}{\pgfqpoint{0.984045in}{0.837438in}}{\pgfqpoint{0.976232in}{0.829625in}}%
\pgfpathcurveto{\pgfqpoint{0.968418in}{0.821811in}}{\pgfqpoint{0.964028in}{0.811212in}}{\pgfqpoint{0.964028in}{0.800162in}}%
\pgfpathcurveto{\pgfqpoint{0.964028in}{0.789112in}}{\pgfqpoint{0.968418in}{0.778513in}}{\pgfqpoint{0.976232in}{0.770699in}}%
\pgfpathcurveto{\pgfqpoint{0.984045in}{0.762885in}}{\pgfqpoint{0.994644in}{0.758495in}}{\pgfqpoint{1.005694in}{0.758495in}}%
\pgfpathclose%
\pgfusepath{stroke,fill}%
\end{pgfscope}%
\begin{pgfscope}%
\pgfpathrectangle{\pgfqpoint{0.375000in}{0.330000in}}{\pgfqpoint{2.325000in}{2.310000in}}%
\pgfusepath{clip}%
\pgfsetbuttcap%
\pgfsetroundjoin%
\definecolor{currentfill}{rgb}{0.000000,0.000000,0.000000}%
\pgfsetfillcolor{currentfill}%
\pgfsetlinewidth{1.003750pt}%
\definecolor{currentstroke}{rgb}{0.000000,0.000000,0.000000}%
\pgfsetstrokecolor{currentstroke}%
\pgfsetdash{}{0pt}%
\pgfpathmoveto{\pgfqpoint{1.005694in}{0.758495in}}%
\pgfpathcurveto{\pgfqpoint{1.016744in}{0.758495in}}{\pgfqpoint{1.027343in}{0.762885in}}{\pgfqpoint{1.035157in}{0.770699in}}%
\pgfpathcurveto{\pgfqpoint{1.042971in}{0.778513in}}{\pgfqpoint{1.047361in}{0.789112in}}{\pgfqpoint{1.047361in}{0.800162in}}%
\pgfpathcurveto{\pgfqpoint{1.047361in}{0.811212in}}{\pgfqpoint{1.042971in}{0.821811in}}{\pgfqpoint{1.035157in}{0.829625in}}%
\pgfpathcurveto{\pgfqpoint{1.027343in}{0.837438in}}{\pgfqpoint{1.016744in}{0.841828in}}{\pgfqpoint{1.005694in}{0.841828in}}%
\pgfpathcurveto{\pgfqpoint{0.994644in}{0.841828in}}{\pgfqpoint{0.984045in}{0.837438in}}{\pgfqpoint{0.976232in}{0.829625in}}%
\pgfpathcurveto{\pgfqpoint{0.968418in}{0.821811in}}{\pgfqpoint{0.964028in}{0.811212in}}{\pgfqpoint{0.964028in}{0.800162in}}%
\pgfpathcurveto{\pgfqpoint{0.964028in}{0.789112in}}{\pgfqpoint{0.968418in}{0.778513in}}{\pgfqpoint{0.976232in}{0.770699in}}%
\pgfpathcurveto{\pgfqpoint{0.984045in}{0.762885in}}{\pgfqpoint{0.994644in}{0.758495in}}{\pgfqpoint{1.005694in}{0.758495in}}%
\pgfpathclose%
\pgfusepath{stroke,fill}%
\end{pgfscope}%
\begin{pgfscope}%
\pgfpathrectangle{\pgfqpoint{0.375000in}{0.330000in}}{\pgfqpoint{2.325000in}{2.310000in}}%
\pgfusepath{clip}%
\pgfsetbuttcap%
\pgfsetroundjoin%
\definecolor{currentfill}{rgb}{0.000000,0.000000,0.000000}%
\pgfsetfillcolor{currentfill}%
\pgfsetlinewidth{1.003750pt}%
\definecolor{currentstroke}{rgb}{0.000000,0.000000,0.000000}%
\pgfsetstrokecolor{currentstroke}%
\pgfsetdash{}{0pt}%
\pgfpathmoveto{\pgfqpoint{1.005694in}{0.758495in}}%
\pgfpathcurveto{\pgfqpoint{1.016744in}{0.758495in}}{\pgfqpoint{1.027343in}{0.762885in}}{\pgfqpoint{1.035157in}{0.770699in}}%
\pgfpathcurveto{\pgfqpoint{1.042971in}{0.778513in}}{\pgfqpoint{1.047361in}{0.789112in}}{\pgfqpoint{1.047361in}{0.800162in}}%
\pgfpathcurveto{\pgfqpoint{1.047361in}{0.811212in}}{\pgfqpoint{1.042971in}{0.821811in}}{\pgfqpoint{1.035157in}{0.829625in}}%
\pgfpathcurveto{\pgfqpoint{1.027343in}{0.837438in}}{\pgfqpoint{1.016744in}{0.841828in}}{\pgfqpoint{1.005694in}{0.841828in}}%
\pgfpathcurveto{\pgfqpoint{0.994644in}{0.841828in}}{\pgfqpoint{0.984045in}{0.837438in}}{\pgfqpoint{0.976232in}{0.829625in}}%
\pgfpathcurveto{\pgfqpoint{0.968418in}{0.821811in}}{\pgfqpoint{0.964028in}{0.811212in}}{\pgfqpoint{0.964028in}{0.800162in}}%
\pgfpathcurveto{\pgfqpoint{0.964028in}{0.789112in}}{\pgfqpoint{0.968418in}{0.778513in}}{\pgfqpoint{0.976232in}{0.770699in}}%
\pgfpathcurveto{\pgfqpoint{0.984045in}{0.762885in}}{\pgfqpoint{0.994644in}{0.758495in}}{\pgfqpoint{1.005694in}{0.758495in}}%
\pgfpathclose%
\pgfusepath{stroke,fill}%
\end{pgfscope}%
\begin{pgfscope}%
\pgfpathrectangle{\pgfqpoint{0.375000in}{0.330000in}}{\pgfqpoint{2.325000in}{2.310000in}}%
\pgfusepath{clip}%
\pgfsetbuttcap%
\pgfsetroundjoin%
\definecolor{currentfill}{rgb}{0.000000,0.000000,0.000000}%
\pgfsetfillcolor{currentfill}%
\pgfsetlinewidth{1.003750pt}%
\definecolor{currentstroke}{rgb}{0.000000,0.000000,0.000000}%
\pgfsetstrokecolor{currentstroke}%
\pgfsetdash{}{0pt}%
\pgfpathmoveto{\pgfqpoint{1.005694in}{0.758495in}}%
\pgfpathcurveto{\pgfqpoint{1.016744in}{0.758495in}}{\pgfqpoint{1.027343in}{0.762885in}}{\pgfqpoint{1.035157in}{0.770699in}}%
\pgfpathcurveto{\pgfqpoint{1.042971in}{0.778513in}}{\pgfqpoint{1.047361in}{0.789112in}}{\pgfqpoint{1.047361in}{0.800162in}}%
\pgfpathcurveto{\pgfqpoint{1.047361in}{0.811212in}}{\pgfqpoint{1.042971in}{0.821811in}}{\pgfqpoint{1.035157in}{0.829625in}}%
\pgfpathcurveto{\pgfqpoint{1.027343in}{0.837438in}}{\pgfqpoint{1.016744in}{0.841828in}}{\pgfqpoint{1.005694in}{0.841828in}}%
\pgfpathcurveto{\pgfqpoint{0.994644in}{0.841828in}}{\pgfqpoint{0.984045in}{0.837438in}}{\pgfqpoint{0.976232in}{0.829625in}}%
\pgfpathcurveto{\pgfqpoint{0.968418in}{0.821811in}}{\pgfqpoint{0.964028in}{0.811212in}}{\pgfqpoint{0.964028in}{0.800162in}}%
\pgfpathcurveto{\pgfqpoint{0.964028in}{0.789112in}}{\pgfqpoint{0.968418in}{0.778513in}}{\pgfqpoint{0.976232in}{0.770699in}}%
\pgfpathcurveto{\pgfqpoint{0.984045in}{0.762885in}}{\pgfqpoint{0.994644in}{0.758495in}}{\pgfqpoint{1.005694in}{0.758495in}}%
\pgfpathclose%
\pgfusepath{stroke,fill}%
\end{pgfscope}%
\begin{pgfscope}%
\pgfpathrectangle{\pgfqpoint{0.375000in}{0.330000in}}{\pgfqpoint{2.325000in}{2.310000in}}%
\pgfusepath{clip}%
\pgfsetbuttcap%
\pgfsetroundjoin%
\definecolor{currentfill}{rgb}{0.000000,0.000000,0.000000}%
\pgfsetfillcolor{currentfill}%
\pgfsetlinewidth{1.003750pt}%
\definecolor{currentstroke}{rgb}{0.000000,0.000000,0.000000}%
\pgfsetstrokecolor{currentstroke}%
\pgfsetdash{}{0pt}%
\pgfpathmoveto{\pgfqpoint{1.005694in}{0.758495in}}%
\pgfpathcurveto{\pgfqpoint{1.016744in}{0.758495in}}{\pgfqpoint{1.027343in}{0.762885in}}{\pgfqpoint{1.035157in}{0.770699in}}%
\pgfpathcurveto{\pgfqpoint{1.042971in}{0.778513in}}{\pgfqpoint{1.047361in}{0.789112in}}{\pgfqpoint{1.047361in}{0.800162in}}%
\pgfpathcurveto{\pgfqpoint{1.047361in}{0.811212in}}{\pgfqpoint{1.042971in}{0.821811in}}{\pgfqpoint{1.035157in}{0.829625in}}%
\pgfpathcurveto{\pgfqpoint{1.027343in}{0.837438in}}{\pgfqpoint{1.016744in}{0.841828in}}{\pgfqpoint{1.005694in}{0.841828in}}%
\pgfpathcurveto{\pgfqpoint{0.994644in}{0.841828in}}{\pgfqpoint{0.984045in}{0.837438in}}{\pgfqpoint{0.976232in}{0.829625in}}%
\pgfpathcurveto{\pgfqpoint{0.968418in}{0.821811in}}{\pgfqpoint{0.964028in}{0.811212in}}{\pgfqpoint{0.964028in}{0.800162in}}%
\pgfpathcurveto{\pgfqpoint{0.964028in}{0.789112in}}{\pgfqpoint{0.968418in}{0.778513in}}{\pgfqpoint{0.976232in}{0.770699in}}%
\pgfpathcurveto{\pgfqpoint{0.984045in}{0.762885in}}{\pgfqpoint{0.994644in}{0.758495in}}{\pgfqpoint{1.005694in}{0.758495in}}%
\pgfpathclose%
\pgfusepath{stroke,fill}%
\end{pgfscope}%
\begin{pgfscope}%
\pgfpathrectangle{\pgfqpoint{0.375000in}{0.330000in}}{\pgfqpoint{2.325000in}{2.310000in}}%
\pgfusepath{clip}%
\pgfsetbuttcap%
\pgfsetroundjoin%
\definecolor{currentfill}{rgb}{0.000000,0.000000,0.000000}%
\pgfsetfillcolor{currentfill}%
\pgfsetlinewidth{1.003750pt}%
\definecolor{currentstroke}{rgb}{0.000000,0.000000,0.000000}%
\pgfsetstrokecolor{currentstroke}%
\pgfsetdash{}{0pt}%
\pgfpathmoveto{\pgfqpoint{1.005694in}{0.758495in}}%
\pgfpathcurveto{\pgfqpoint{1.016744in}{0.758495in}}{\pgfqpoint{1.027343in}{0.762885in}}{\pgfqpoint{1.035157in}{0.770699in}}%
\pgfpathcurveto{\pgfqpoint{1.042971in}{0.778513in}}{\pgfqpoint{1.047361in}{0.789112in}}{\pgfqpoint{1.047361in}{0.800162in}}%
\pgfpathcurveto{\pgfqpoint{1.047361in}{0.811212in}}{\pgfqpoint{1.042971in}{0.821811in}}{\pgfqpoint{1.035157in}{0.829625in}}%
\pgfpathcurveto{\pgfqpoint{1.027343in}{0.837438in}}{\pgfqpoint{1.016744in}{0.841828in}}{\pgfqpoint{1.005694in}{0.841828in}}%
\pgfpathcurveto{\pgfqpoint{0.994644in}{0.841828in}}{\pgfqpoint{0.984045in}{0.837438in}}{\pgfqpoint{0.976232in}{0.829625in}}%
\pgfpathcurveto{\pgfqpoint{0.968418in}{0.821811in}}{\pgfqpoint{0.964028in}{0.811212in}}{\pgfqpoint{0.964028in}{0.800162in}}%
\pgfpathcurveto{\pgfqpoint{0.964028in}{0.789112in}}{\pgfqpoint{0.968418in}{0.778513in}}{\pgfqpoint{0.976232in}{0.770699in}}%
\pgfpathcurveto{\pgfqpoint{0.984045in}{0.762885in}}{\pgfqpoint{0.994644in}{0.758495in}}{\pgfqpoint{1.005694in}{0.758495in}}%
\pgfpathclose%
\pgfusepath{stroke,fill}%
\end{pgfscope}%
\begin{pgfscope}%
\pgfpathrectangle{\pgfqpoint{0.375000in}{0.330000in}}{\pgfqpoint{2.325000in}{2.310000in}}%
\pgfusepath{clip}%
\pgfsetbuttcap%
\pgfsetroundjoin%
\definecolor{currentfill}{rgb}{0.000000,0.000000,0.000000}%
\pgfsetfillcolor{currentfill}%
\pgfsetlinewidth{1.003750pt}%
\definecolor{currentstroke}{rgb}{0.000000,0.000000,0.000000}%
\pgfsetstrokecolor{currentstroke}%
\pgfsetdash{}{0pt}%
\pgfpathmoveto{\pgfqpoint{1.005694in}{0.706464in}}%
\pgfpathcurveto{\pgfqpoint{1.016744in}{0.706464in}}{\pgfqpoint{1.027343in}{0.710855in}}{\pgfqpoint{1.035157in}{0.718668in}}%
\pgfpathcurveto{\pgfqpoint{1.042971in}{0.726482in}}{\pgfqpoint{1.047361in}{0.737081in}}{\pgfqpoint{1.047361in}{0.748131in}}%
\pgfpathcurveto{\pgfqpoint{1.047361in}{0.759181in}}{\pgfqpoint{1.042971in}{0.769780in}}{\pgfqpoint{1.035157in}{0.777594in}}%
\pgfpathcurveto{\pgfqpoint{1.027343in}{0.785407in}}{\pgfqpoint{1.016744in}{0.789798in}}{\pgfqpoint{1.005694in}{0.789798in}}%
\pgfpathcurveto{\pgfqpoint{0.994644in}{0.789798in}}{\pgfqpoint{0.984045in}{0.785407in}}{\pgfqpoint{0.976232in}{0.777594in}}%
\pgfpathcurveto{\pgfqpoint{0.968418in}{0.769780in}}{\pgfqpoint{0.964028in}{0.759181in}}{\pgfqpoint{0.964028in}{0.748131in}}%
\pgfpathcurveto{\pgfqpoint{0.964028in}{0.737081in}}{\pgfqpoint{0.968418in}{0.726482in}}{\pgfqpoint{0.976232in}{0.718668in}}%
\pgfpathcurveto{\pgfqpoint{0.984045in}{0.710855in}}{\pgfqpoint{0.994644in}{0.706464in}}{\pgfqpoint{1.005694in}{0.706464in}}%
\pgfpathclose%
\pgfusepath{stroke,fill}%
\end{pgfscope}%
\begin{pgfscope}%
\pgfpathrectangle{\pgfqpoint{0.375000in}{0.330000in}}{\pgfqpoint{2.325000in}{2.310000in}}%
\pgfusepath{clip}%
\pgfsetbuttcap%
\pgfsetroundjoin%
\definecolor{currentfill}{rgb}{0.000000,0.000000,0.000000}%
\pgfsetfillcolor{currentfill}%
\pgfsetlinewidth{1.003750pt}%
\definecolor{currentstroke}{rgb}{0.000000,0.000000,0.000000}%
\pgfsetstrokecolor{currentstroke}%
\pgfsetdash{}{0pt}%
\pgfpathmoveto{\pgfqpoint{1.005694in}{0.706464in}}%
\pgfpathcurveto{\pgfqpoint{1.016744in}{0.706464in}}{\pgfqpoint{1.027343in}{0.710855in}}{\pgfqpoint{1.035157in}{0.718668in}}%
\pgfpathcurveto{\pgfqpoint{1.042971in}{0.726482in}}{\pgfqpoint{1.047361in}{0.737081in}}{\pgfqpoint{1.047361in}{0.748131in}}%
\pgfpathcurveto{\pgfqpoint{1.047361in}{0.759181in}}{\pgfqpoint{1.042971in}{0.769780in}}{\pgfqpoint{1.035157in}{0.777594in}}%
\pgfpathcurveto{\pgfqpoint{1.027343in}{0.785407in}}{\pgfqpoint{1.016744in}{0.789798in}}{\pgfqpoint{1.005694in}{0.789798in}}%
\pgfpathcurveto{\pgfqpoint{0.994644in}{0.789798in}}{\pgfqpoint{0.984045in}{0.785407in}}{\pgfqpoint{0.976232in}{0.777594in}}%
\pgfpathcurveto{\pgfqpoint{0.968418in}{0.769780in}}{\pgfqpoint{0.964028in}{0.759181in}}{\pgfqpoint{0.964028in}{0.748131in}}%
\pgfpathcurveto{\pgfqpoint{0.964028in}{0.737081in}}{\pgfqpoint{0.968418in}{0.726482in}}{\pgfqpoint{0.976232in}{0.718668in}}%
\pgfpathcurveto{\pgfqpoint{0.984045in}{0.710855in}}{\pgfqpoint{0.994644in}{0.706464in}}{\pgfqpoint{1.005694in}{0.706464in}}%
\pgfpathclose%
\pgfusepath{stroke,fill}%
\end{pgfscope}%
\begin{pgfscope}%
\pgfpathrectangle{\pgfqpoint{0.375000in}{0.330000in}}{\pgfqpoint{2.325000in}{2.310000in}}%
\pgfusepath{clip}%
\pgfsetbuttcap%
\pgfsetroundjoin%
\definecolor{currentfill}{rgb}{0.000000,0.000000,0.000000}%
\pgfsetfillcolor{currentfill}%
\pgfsetlinewidth{1.003750pt}%
\definecolor{currentstroke}{rgb}{0.000000,0.000000,0.000000}%
\pgfsetstrokecolor{currentstroke}%
\pgfsetdash{}{0pt}%
\pgfpathmoveto{\pgfqpoint{1.005694in}{0.706464in}}%
\pgfpathcurveto{\pgfqpoint{1.016744in}{0.706464in}}{\pgfqpoint{1.027343in}{0.710855in}}{\pgfqpoint{1.035157in}{0.718668in}}%
\pgfpathcurveto{\pgfqpoint{1.042971in}{0.726482in}}{\pgfqpoint{1.047361in}{0.737081in}}{\pgfqpoint{1.047361in}{0.748131in}}%
\pgfpathcurveto{\pgfqpoint{1.047361in}{0.759181in}}{\pgfqpoint{1.042971in}{0.769780in}}{\pgfqpoint{1.035157in}{0.777594in}}%
\pgfpathcurveto{\pgfqpoint{1.027343in}{0.785407in}}{\pgfqpoint{1.016744in}{0.789798in}}{\pgfqpoint{1.005694in}{0.789798in}}%
\pgfpathcurveto{\pgfqpoint{0.994644in}{0.789798in}}{\pgfqpoint{0.984045in}{0.785407in}}{\pgfqpoint{0.976232in}{0.777594in}}%
\pgfpathcurveto{\pgfqpoint{0.968418in}{0.769780in}}{\pgfqpoint{0.964028in}{0.759181in}}{\pgfqpoint{0.964028in}{0.748131in}}%
\pgfpathcurveto{\pgfqpoint{0.964028in}{0.737081in}}{\pgfqpoint{0.968418in}{0.726482in}}{\pgfqpoint{0.976232in}{0.718668in}}%
\pgfpathcurveto{\pgfqpoint{0.984045in}{0.710855in}}{\pgfqpoint{0.994644in}{0.706464in}}{\pgfqpoint{1.005694in}{0.706464in}}%
\pgfpathclose%
\pgfusepath{stroke,fill}%
\end{pgfscope}%
\begin{pgfscope}%
\pgfpathrectangle{\pgfqpoint{0.375000in}{0.330000in}}{\pgfqpoint{2.325000in}{2.310000in}}%
\pgfusepath{clip}%
\pgfsetbuttcap%
\pgfsetroundjoin%
\definecolor{currentfill}{rgb}{0.000000,0.000000,0.000000}%
\pgfsetfillcolor{currentfill}%
\pgfsetlinewidth{1.003750pt}%
\definecolor{currentstroke}{rgb}{0.000000,0.000000,0.000000}%
\pgfsetstrokecolor{currentstroke}%
\pgfsetdash{}{0pt}%
\pgfpathmoveto{\pgfqpoint{1.005694in}{0.758495in}}%
\pgfpathcurveto{\pgfqpoint{1.016744in}{0.758495in}}{\pgfqpoint{1.027343in}{0.762885in}}{\pgfqpoint{1.035157in}{0.770699in}}%
\pgfpathcurveto{\pgfqpoint{1.042971in}{0.778513in}}{\pgfqpoint{1.047361in}{0.789112in}}{\pgfqpoint{1.047361in}{0.800162in}}%
\pgfpathcurveto{\pgfqpoint{1.047361in}{0.811212in}}{\pgfqpoint{1.042971in}{0.821811in}}{\pgfqpoint{1.035157in}{0.829625in}}%
\pgfpathcurveto{\pgfqpoint{1.027343in}{0.837438in}}{\pgfqpoint{1.016744in}{0.841828in}}{\pgfqpoint{1.005694in}{0.841828in}}%
\pgfpathcurveto{\pgfqpoint{0.994644in}{0.841828in}}{\pgfqpoint{0.984045in}{0.837438in}}{\pgfqpoint{0.976232in}{0.829625in}}%
\pgfpathcurveto{\pgfqpoint{0.968418in}{0.821811in}}{\pgfqpoint{0.964028in}{0.811212in}}{\pgfqpoint{0.964028in}{0.800162in}}%
\pgfpathcurveto{\pgfqpoint{0.964028in}{0.789112in}}{\pgfqpoint{0.968418in}{0.778513in}}{\pgfqpoint{0.976232in}{0.770699in}}%
\pgfpathcurveto{\pgfqpoint{0.984045in}{0.762885in}}{\pgfqpoint{0.994644in}{0.758495in}}{\pgfqpoint{1.005694in}{0.758495in}}%
\pgfpathclose%
\pgfusepath{stroke,fill}%
\end{pgfscope}%
\begin{pgfscope}%
\pgfpathrectangle{\pgfqpoint{0.375000in}{0.330000in}}{\pgfqpoint{2.325000in}{2.310000in}}%
\pgfusepath{clip}%
\pgfsetbuttcap%
\pgfsetroundjoin%
\definecolor{currentfill}{rgb}{0.000000,0.000000,0.000000}%
\pgfsetfillcolor{currentfill}%
\pgfsetlinewidth{1.003750pt}%
\definecolor{currentstroke}{rgb}{0.000000,0.000000,0.000000}%
\pgfsetstrokecolor{currentstroke}%
\pgfsetdash{}{0pt}%
\pgfpathmoveto{\pgfqpoint{1.005694in}{0.758495in}}%
\pgfpathcurveto{\pgfqpoint{1.016744in}{0.758495in}}{\pgfqpoint{1.027343in}{0.762885in}}{\pgfqpoint{1.035157in}{0.770699in}}%
\pgfpathcurveto{\pgfqpoint{1.042971in}{0.778513in}}{\pgfqpoint{1.047361in}{0.789112in}}{\pgfqpoint{1.047361in}{0.800162in}}%
\pgfpathcurveto{\pgfqpoint{1.047361in}{0.811212in}}{\pgfqpoint{1.042971in}{0.821811in}}{\pgfqpoint{1.035157in}{0.829625in}}%
\pgfpathcurveto{\pgfqpoint{1.027343in}{0.837438in}}{\pgfqpoint{1.016744in}{0.841828in}}{\pgfqpoint{1.005694in}{0.841828in}}%
\pgfpathcurveto{\pgfqpoint{0.994644in}{0.841828in}}{\pgfqpoint{0.984045in}{0.837438in}}{\pgfqpoint{0.976232in}{0.829625in}}%
\pgfpathcurveto{\pgfqpoint{0.968418in}{0.821811in}}{\pgfqpoint{0.964028in}{0.811212in}}{\pgfqpoint{0.964028in}{0.800162in}}%
\pgfpathcurveto{\pgfqpoint{0.964028in}{0.789112in}}{\pgfqpoint{0.968418in}{0.778513in}}{\pgfqpoint{0.976232in}{0.770699in}}%
\pgfpathcurveto{\pgfqpoint{0.984045in}{0.762885in}}{\pgfqpoint{0.994644in}{0.758495in}}{\pgfqpoint{1.005694in}{0.758495in}}%
\pgfpathclose%
\pgfusepath{stroke,fill}%
\end{pgfscope}%
\begin{pgfscope}%
\pgfpathrectangle{\pgfqpoint{0.375000in}{0.330000in}}{\pgfqpoint{2.325000in}{2.310000in}}%
\pgfusepath{clip}%
\pgfsetbuttcap%
\pgfsetroundjoin%
\definecolor{currentfill}{rgb}{0.000000,0.000000,0.000000}%
\pgfsetfillcolor{currentfill}%
\pgfsetlinewidth{1.003750pt}%
\definecolor{currentstroke}{rgb}{0.000000,0.000000,0.000000}%
\pgfsetstrokecolor{currentstroke}%
\pgfsetdash{}{0pt}%
\pgfpathmoveto{\pgfqpoint{1.005694in}{0.654433in}}%
\pgfpathcurveto{\pgfqpoint{1.016744in}{0.654433in}}{\pgfqpoint{1.027343in}{0.658824in}}{\pgfqpoint{1.035157in}{0.666637in}}%
\pgfpathcurveto{\pgfqpoint{1.042971in}{0.674451in}}{\pgfqpoint{1.047361in}{0.685050in}}{\pgfqpoint{1.047361in}{0.696100in}}%
\pgfpathcurveto{\pgfqpoint{1.047361in}{0.707150in}}{\pgfqpoint{1.042971in}{0.717749in}}{\pgfqpoint{1.035157in}{0.725563in}}%
\pgfpathcurveto{\pgfqpoint{1.027343in}{0.733377in}}{\pgfqpoint{1.016744in}{0.737767in}}{\pgfqpoint{1.005694in}{0.737767in}}%
\pgfpathcurveto{\pgfqpoint{0.994644in}{0.737767in}}{\pgfqpoint{0.984045in}{0.733377in}}{\pgfqpoint{0.976232in}{0.725563in}}%
\pgfpathcurveto{\pgfqpoint{0.968418in}{0.717749in}}{\pgfqpoint{0.964028in}{0.707150in}}{\pgfqpoint{0.964028in}{0.696100in}}%
\pgfpathcurveto{\pgfqpoint{0.964028in}{0.685050in}}{\pgfqpoint{0.968418in}{0.674451in}}{\pgfqpoint{0.976232in}{0.666637in}}%
\pgfpathcurveto{\pgfqpoint{0.984045in}{0.658824in}}{\pgfqpoint{0.994644in}{0.654433in}}{\pgfqpoint{1.005694in}{0.654433in}}%
\pgfpathclose%
\pgfusepath{stroke,fill}%
\end{pgfscope}%
\begin{pgfscope}%
\pgfpathrectangle{\pgfqpoint{0.375000in}{0.330000in}}{\pgfqpoint{2.325000in}{2.310000in}}%
\pgfusepath{clip}%
\pgfsetbuttcap%
\pgfsetroundjoin%
\definecolor{currentfill}{rgb}{0.000000,0.000000,0.000000}%
\pgfsetfillcolor{currentfill}%
\pgfsetlinewidth{1.003750pt}%
\definecolor{currentstroke}{rgb}{0.000000,0.000000,0.000000}%
\pgfsetstrokecolor{currentstroke}%
\pgfsetdash{}{0pt}%
\pgfpathmoveto{\pgfqpoint{1.005694in}{0.654433in}}%
\pgfpathcurveto{\pgfqpoint{1.016744in}{0.654433in}}{\pgfqpoint{1.027343in}{0.658824in}}{\pgfqpoint{1.035157in}{0.666637in}}%
\pgfpathcurveto{\pgfqpoint{1.042971in}{0.674451in}}{\pgfqpoint{1.047361in}{0.685050in}}{\pgfqpoint{1.047361in}{0.696100in}}%
\pgfpathcurveto{\pgfqpoint{1.047361in}{0.707150in}}{\pgfqpoint{1.042971in}{0.717749in}}{\pgfqpoint{1.035157in}{0.725563in}}%
\pgfpathcurveto{\pgfqpoint{1.027343in}{0.733377in}}{\pgfqpoint{1.016744in}{0.737767in}}{\pgfqpoint{1.005694in}{0.737767in}}%
\pgfpathcurveto{\pgfqpoint{0.994644in}{0.737767in}}{\pgfqpoint{0.984045in}{0.733377in}}{\pgfqpoint{0.976232in}{0.725563in}}%
\pgfpathcurveto{\pgfqpoint{0.968418in}{0.717749in}}{\pgfqpoint{0.964028in}{0.707150in}}{\pgfqpoint{0.964028in}{0.696100in}}%
\pgfpathcurveto{\pgfqpoint{0.964028in}{0.685050in}}{\pgfqpoint{0.968418in}{0.674451in}}{\pgfqpoint{0.976232in}{0.666637in}}%
\pgfpathcurveto{\pgfqpoint{0.984045in}{0.658824in}}{\pgfqpoint{0.994644in}{0.654433in}}{\pgfqpoint{1.005694in}{0.654433in}}%
\pgfpathclose%
\pgfusepath{stroke,fill}%
\end{pgfscope}%
\begin{pgfscope}%
\pgfpathrectangle{\pgfqpoint{0.375000in}{0.330000in}}{\pgfqpoint{2.325000in}{2.310000in}}%
\pgfusepath{clip}%
\pgfsetbuttcap%
\pgfsetroundjoin%
\definecolor{currentfill}{rgb}{0.000000,0.000000,0.000000}%
\pgfsetfillcolor{currentfill}%
\pgfsetlinewidth{1.003750pt}%
\definecolor{currentstroke}{rgb}{0.000000,0.000000,0.000000}%
\pgfsetstrokecolor{currentstroke}%
\pgfsetdash{}{0pt}%
\pgfpathmoveto{\pgfqpoint{1.005694in}{0.758495in}}%
\pgfpathcurveto{\pgfqpoint{1.016744in}{0.758495in}}{\pgfqpoint{1.027343in}{0.762885in}}{\pgfqpoint{1.035157in}{0.770699in}}%
\pgfpathcurveto{\pgfqpoint{1.042971in}{0.778513in}}{\pgfqpoint{1.047361in}{0.789112in}}{\pgfqpoint{1.047361in}{0.800162in}}%
\pgfpathcurveto{\pgfqpoint{1.047361in}{0.811212in}}{\pgfqpoint{1.042971in}{0.821811in}}{\pgfqpoint{1.035157in}{0.829625in}}%
\pgfpathcurveto{\pgfqpoint{1.027343in}{0.837438in}}{\pgfqpoint{1.016744in}{0.841828in}}{\pgfqpoint{1.005694in}{0.841828in}}%
\pgfpathcurveto{\pgfqpoint{0.994644in}{0.841828in}}{\pgfqpoint{0.984045in}{0.837438in}}{\pgfqpoint{0.976232in}{0.829625in}}%
\pgfpathcurveto{\pgfqpoint{0.968418in}{0.821811in}}{\pgfqpoint{0.964028in}{0.811212in}}{\pgfqpoint{0.964028in}{0.800162in}}%
\pgfpathcurveto{\pgfqpoint{0.964028in}{0.789112in}}{\pgfqpoint{0.968418in}{0.778513in}}{\pgfqpoint{0.976232in}{0.770699in}}%
\pgfpathcurveto{\pgfqpoint{0.984045in}{0.762885in}}{\pgfqpoint{0.994644in}{0.758495in}}{\pgfqpoint{1.005694in}{0.758495in}}%
\pgfpathclose%
\pgfusepath{stroke,fill}%
\end{pgfscope}%
\begin{pgfscope}%
\pgfpathrectangle{\pgfqpoint{0.375000in}{0.330000in}}{\pgfqpoint{2.325000in}{2.310000in}}%
\pgfusepath{clip}%
\pgfsetbuttcap%
\pgfsetroundjoin%
\definecolor{currentfill}{rgb}{0.000000,0.000000,0.000000}%
\pgfsetfillcolor{currentfill}%
\pgfsetlinewidth{1.003750pt}%
\definecolor{currentstroke}{rgb}{0.000000,0.000000,0.000000}%
\pgfsetstrokecolor{currentstroke}%
\pgfsetdash{}{0pt}%
\pgfpathmoveto{\pgfqpoint{1.005694in}{0.706464in}}%
\pgfpathcurveto{\pgfqpoint{1.016744in}{0.706464in}}{\pgfqpoint{1.027343in}{0.710855in}}{\pgfqpoint{1.035157in}{0.718668in}}%
\pgfpathcurveto{\pgfqpoint{1.042971in}{0.726482in}}{\pgfqpoint{1.047361in}{0.737081in}}{\pgfqpoint{1.047361in}{0.748131in}}%
\pgfpathcurveto{\pgfqpoint{1.047361in}{0.759181in}}{\pgfqpoint{1.042971in}{0.769780in}}{\pgfqpoint{1.035157in}{0.777594in}}%
\pgfpathcurveto{\pgfqpoint{1.027343in}{0.785407in}}{\pgfqpoint{1.016744in}{0.789798in}}{\pgfqpoint{1.005694in}{0.789798in}}%
\pgfpathcurveto{\pgfqpoint{0.994644in}{0.789798in}}{\pgfqpoint{0.984045in}{0.785407in}}{\pgfqpoint{0.976232in}{0.777594in}}%
\pgfpathcurveto{\pgfqpoint{0.968418in}{0.769780in}}{\pgfqpoint{0.964028in}{0.759181in}}{\pgfqpoint{0.964028in}{0.748131in}}%
\pgfpathcurveto{\pgfqpoint{0.964028in}{0.737081in}}{\pgfqpoint{0.968418in}{0.726482in}}{\pgfqpoint{0.976232in}{0.718668in}}%
\pgfpathcurveto{\pgfqpoint{0.984045in}{0.710855in}}{\pgfqpoint{0.994644in}{0.706464in}}{\pgfqpoint{1.005694in}{0.706464in}}%
\pgfpathclose%
\pgfusepath{stroke,fill}%
\end{pgfscope}%
\begin{pgfscope}%
\pgfpathrectangle{\pgfqpoint{0.375000in}{0.330000in}}{\pgfqpoint{2.325000in}{2.310000in}}%
\pgfusepath{clip}%
\pgfsetbuttcap%
\pgfsetroundjoin%
\definecolor{currentfill}{rgb}{0.000000,0.000000,0.000000}%
\pgfsetfillcolor{currentfill}%
\pgfsetlinewidth{1.003750pt}%
\definecolor{currentstroke}{rgb}{0.000000,0.000000,0.000000}%
\pgfsetstrokecolor{currentstroke}%
\pgfsetdash{}{0pt}%
\pgfpathmoveto{\pgfqpoint{1.005694in}{0.758495in}}%
\pgfpathcurveto{\pgfqpoint{1.016744in}{0.758495in}}{\pgfqpoint{1.027343in}{0.762885in}}{\pgfqpoint{1.035157in}{0.770699in}}%
\pgfpathcurveto{\pgfqpoint{1.042971in}{0.778513in}}{\pgfqpoint{1.047361in}{0.789112in}}{\pgfqpoint{1.047361in}{0.800162in}}%
\pgfpathcurveto{\pgfqpoint{1.047361in}{0.811212in}}{\pgfqpoint{1.042971in}{0.821811in}}{\pgfqpoint{1.035157in}{0.829625in}}%
\pgfpathcurveto{\pgfqpoint{1.027343in}{0.837438in}}{\pgfqpoint{1.016744in}{0.841828in}}{\pgfqpoint{1.005694in}{0.841828in}}%
\pgfpathcurveto{\pgfqpoint{0.994644in}{0.841828in}}{\pgfqpoint{0.984045in}{0.837438in}}{\pgfqpoint{0.976232in}{0.829625in}}%
\pgfpathcurveto{\pgfqpoint{0.968418in}{0.821811in}}{\pgfqpoint{0.964028in}{0.811212in}}{\pgfqpoint{0.964028in}{0.800162in}}%
\pgfpathcurveto{\pgfqpoint{0.964028in}{0.789112in}}{\pgfqpoint{0.968418in}{0.778513in}}{\pgfqpoint{0.976232in}{0.770699in}}%
\pgfpathcurveto{\pgfqpoint{0.984045in}{0.762885in}}{\pgfqpoint{0.994644in}{0.758495in}}{\pgfqpoint{1.005694in}{0.758495in}}%
\pgfpathclose%
\pgfusepath{stroke,fill}%
\end{pgfscope}%
\begin{pgfscope}%
\pgfpathrectangle{\pgfqpoint{0.375000in}{0.330000in}}{\pgfqpoint{2.325000in}{2.310000in}}%
\pgfusepath{clip}%
\pgfsetbuttcap%
\pgfsetroundjoin%
\definecolor{currentfill}{rgb}{0.000000,0.000000,0.000000}%
\pgfsetfillcolor{currentfill}%
\pgfsetlinewidth{1.003750pt}%
\definecolor{currentstroke}{rgb}{0.000000,0.000000,0.000000}%
\pgfsetstrokecolor{currentstroke}%
\pgfsetdash{}{0pt}%
\pgfpathmoveto{\pgfqpoint{1.005694in}{0.706464in}}%
\pgfpathcurveto{\pgfqpoint{1.016744in}{0.706464in}}{\pgfqpoint{1.027343in}{0.710855in}}{\pgfqpoint{1.035157in}{0.718668in}}%
\pgfpathcurveto{\pgfqpoint{1.042971in}{0.726482in}}{\pgfqpoint{1.047361in}{0.737081in}}{\pgfqpoint{1.047361in}{0.748131in}}%
\pgfpathcurveto{\pgfqpoint{1.047361in}{0.759181in}}{\pgfqpoint{1.042971in}{0.769780in}}{\pgfqpoint{1.035157in}{0.777594in}}%
\pgfpathcurveto{\pgfqpoint{1.027343in}{0.785407in}}{\pgfqpoint{1.016744in}{0.789798in}}{\pgfqpoint{1.005694in}{0.789798in}}%
\pgfpathcurveto{\pgfqpoint{0.994644in}{0.789798in}}{\pgfqpoint{0.984045in}{0.785407in}}{\pgfqpoint{0.976232in}{0.777594in}}%
\pgfpathcurveto{\pgfqpoint{0.968418in}{0.769780in}}{\pgfqpoint{0.964028in}{0.759181in}}{\pgfqpoint{0.964028in}{0.748131in}}%
\pgfpathcurveto{\pgfqpoint{0.964028in}{0.737081in}}{\pgfqpoint{0.968418in}{0.726482in}}{\pgfqpoint{0.976232in}{0.718668in}}%
\pgfpathcurveto{\pgfqpoint{0.984045in}{0.710855in}}{\pgfqpoint{0.994644in}{0.706464in}}{\pgfqpoint{1.005694in}{0.706464in}}%
\pgfpathclose%
\pgfusepath{stroke,fill}%
\end{pgfscope}%
\begin{pgfscope}%
\pgfpathrectangle{\pgfqpoint{0.375000in}{0.330000in}}{\pgfqpoint{2.325000in}{2.310000in}}%
\pgfusepath{clip}%
\pgfsetbuttcap%
\pgfsetroundjoin%
\definecolor{currentfill}{rgb}{0.000000,0.000000,0.000000}%
\pgfsetfillcolor{currentfill}%
\pgfsetlinewidth{1.003750pt}%
\definecolor{currentstroke}{rgb}{0.000000,0.000000,0.000000}%
\pgfsetstrokecolor{currentstroke}%
\pgfsetdash{}{0pt}%
\pgfpathmoveto{\pgfqpoint{1.005694in}{0.706464in}}%
\pgfpathcurveto{\pgfqpoint{1.016744in}{0.706464in}}{\pgfqpoint{1.027343in}{0.710855in}}{\pgfqpoint{1.035157in}{0.718668in}}%
\pgfpathcurveto{\pgfqpoint{1.042971in}{0.726482in}}{\pgfqpoint{1.047361in}{0.737081in}}{\pgfqpoint{1.047361in}{0.748131in}}%
\pgfpathcurveto{\pgfqpoint{1.047361in}{0.759181in}}{\pgfqpoint{1.042971in}{0.769780in}}{\pgfqpoint{1.035157in}{0.777594in}}%
\pgfpathcurveto{\pgfqpoint{1.027343in}{0.785407in}}{\pgfqpoint{1.016744in}{0.789798in}}{\pgfqpoint{1.005694in}{0.789798in}}%
\pgfpathcurveto{\pgfqpoint{0.994644in}{0.789798in}}{\pgfqpoint{0.984045in}{0.785407in}}{\pgfqpoint{0.976232in}{0.777594in}}%
\pgfpathcurveto{\pgfqpoint{0.968418in}{0.769780in}}{\pgfqpoint{0.964028in}{0.759181in}}{\pgfqpoint{0.964028in}{0.748131in}}%
\pgfpathcurveto{\pgfqpoint{0.964028in}{0.737081in}}{\pgfqpoint{0.968418in}{0.726482in}}{\pgfqpoint{0.976232in}{0.718668in}}%
\pgfpathcurveto{\pgfqpoint{0.984045in}{0.710855in}}{\pgfqpoint{0.994644in}{0.706464in}}{\pgfqpoint{1.005694in}{0.706464in}}%
\pgfpathclose%
\pgfusepath{stroke,fill}%
\end{pgfscope}%
\begin{pgfscope}%
\pgfpathrectangle{\pgfqpoint{0.375000in}{0.330000in}}{\pgfqpoint{2.325000in}{2.310000in}}%
\pgfusepath{clip}%
\pgfsetbuttcap%
\pgfsetroundjoin%
\definecolor{currentfill}{rgb}{0.000000,0.000000,0.000000}%
\pgfsetfillcolor{currentfill}%
\pgfsetlinewidth{1.003750pt}%
\definecolor{currentstroke}{rgb}{0.000000,0.000000,0.000000}%
\pgfsetstrokecolor{currentstroke}%
\pgfsetdash{}{0pt}%
\pgfpathmoveto{\pgfqpoint{1.005694in}{0.706464in}}%
\pgfpathcurveto{\pgfqpoint{1.016744in}{0.706464in}}{\pgfqpoint{1.027343in}{0.710855in}}{\pgfqpoint{1.035157in}{0.718668in}}%
\pgfpathcurveto{\pgfqpoint{1.042971in}{0.726482in}}{\pgfqpoint{1.047361in}{0.737081in}}{\pgfqpoint{1.047361in}{0.748131in}}%
\pgfpathcurveto{\pgfqpoint{1.047361in}{0.759181in}}{\pgfqpoint{1.042971in}{0.769780in}}{\pgfqpoint{1.035157in}{0.777594in}}%
\pgfpathcurveto{\pgfqpoint{1.027343in}{0.785407in}}{\pgfqpoint{1.016744in}{0.789798in}}{\pgfqpoint{1.005694in}{0.789798in}}%
\pgfpathcurveto{\pgfqpoint{0.994644in}{0.789798in}}{\pgfqpoint{0.984045in}{0.785407in}}{\pgfqpoint{0.976232in}{0.777594in}}%
\pgfpathcurveto{\pgfqpoint{0.968418in}{0.769780in}}{\pgfqpoint{0.964028in}{0.759181in}}{\pgfqpoint{0.964028in}{0.748131in}}%
\pgfpathcurveto{\pgfqpoint{0.964028in}{0.737081in}}{\pgfqpoint{0.968418in}{0.726482in}}{\pgfqpoint{0.976232in}{0.718668in}}%
\pgfpathcurveto{\pgfqpoint{0.984045in}{0.710855in}}{\pgfqpoint{0.994644in}{0.706464in}}{\pgfqpoint{1.005694in}{0.706464in}}%
\pgfpathclose%
\pgfusepath{stroke,fill}%
\end{pgfscope}%
\begin{pgfscope}%
\pgfpathrectangle{\pgfqpoint{0.375000in}{0.330000in}}{\pgfqpoint{2.325000in}{2.310000in}}%
\pgfusepath{clip}%
\pgfsetbuttcap%
\pgfsetroundjoin%
\definecolor{currentfill}{rgb}{0.000000,0.000000,0.000000}%
\pgfsetfillcolor{currentfill}%
\pgfsetlinewidth{1.003750pt}%
\definecolor{currentstroke}{rgb}{0.000000,0.000000,0.000000}%
\pgfsetstrokecolor{currentstroke}%
\pgfsetdash{}{0pt}%
\pgfpathmoveto{\pgfqpoint{1.005694in}{0.758495in}}%
\pgfpathcurveto{\pgfqpoint{1.016744in}{0.758495in}}{\pgfqpoint{1.027343in}{0.762885in}}{\pgfqpoint{1.035157in}{0.770699in}}%
\pgfpathcurveto{\pgfqpoint{1.042971in}{0.778513in}}{\pgfqpoint{1.047361in}{0.789112in}}{\pgfqpoint{1.047361in}{0.800162in}}%
\pgfpathcurveto{\pgfqpoint{1.047361in}{0.811212in}}{\pgfqpoint{1.042971in}{0.821811in}}{\pgfqpoint{1.035157in}{0.829625in}}%
\pgfpathcurveto{\pgfqpoint{1.027343in}{0.837438in}}{\pgfqpoint{1.016744in}{0.841828in}}{\pgfqpoint{1.005694in}{0.841828in}}%
\pgfpathcurveto{\pgfqpoint{0.994644in}{0.841828in}}{\pgfqpoint{0.984045in}{0.837438in}}{\pgfqpoint{0.976232in}{0.829625in}}%
\pgfpathcurveto{\pgfqpoint{0.968418in}{0.821811in}}{\pgfqpoint{0.964028in}{0.811212in}}{\pgfqpoint{0.964028in}{0.800162in}}%
\pgfpathcurveto{\pgfqpoint{0.964028in}{0.789112in}}{\pgfqpoint{0.968418in}{0.778513in}}{\pgfqpoint{0.976232in}{0.770699in}}%
\pgfpathcurveto{\pgfqpoint{0.984045in}{0.762885in}}{\pgfqpoint{0.994644in}{0.758495in}}{\pgfqpoint{1.005694in}{0.758495in}}%
\pgfpathclose%
\pgfusepath{stroke,fill}%
\end{pgfscope}%
\begin{pgfscope}%
\pgfpathrectangle{\pgfqpoint{0.375000in}{0.330000in}}{\pgfqpoint{2.325000in}{2.310000in}}%
\pgfusepath{clip}%
\pgfsetbuttcap%
\pgfsetroundjoin%
\definecolor{currentfill}{rgb}{0.000000,0.000000,0.000000}%
\pgfsetfillcolor{currentfill}%
\pgfsetlinewidth{1.003750pt}%
\definecolor{currentstroke}{rgb}{0.000000,0.000000,0.000000}%
\pgfsetstrokecolor{currentstroke}%
\pgfsetdash{}{0pt}%
\pgfpathmoveto{\pgfqpoint{1.005694in}{0.706464in}}%
\pgfpathcurveto{\pgfqpoint{1.016744in}{0.706464in}}{\pgfqpoint{1.027343in}{0.710855in}}{\pgfqpoint{1.035157in}{0.718668in}}%
\pgfpathcurveto{\pgfqpoint{1.042971in}{0.726482in}}{\pgfqpoint{1.047361in}{0.737081in}}{\pgfqpoint{1.047361in}{0.748131in}}%
\pgfpathcurveto{\pgfqpoint{1.047361in}{0.759181in}}{\pgfqpoint{1.042971in}{0.769780in}}{\pgfqpoint{1.035157in}{0.777594in}}%
\pgfpathcurveto{\pgfqpoint{1.027343in}{0.785407in}}{\pgfqpoint{1.016744in}{0.789798in}}{\pgfqpoint{1.005694in}{0.789798in}}%
\pgfpathcurveto{\pgfqpoint{0.994644in}{0.789798in}}{\pgfqpoint{0.984045in}{0.785407in}}{\pgfqpoint{0.976232in}{0.777594in}}%
\pgfpathcurveto{\pgfqpoint{0.968418in}{0.769780in}}{\pgfqpoint{0.964028in}{0.759181in}}{\pgfqpoint{0.964028in}{0.748131in}}%
\pgfpathcurveto{\pgfqpoint{0.964028in}{0.737081in}}{\pgfqpoint{0.968418in}{0.726482in}}{\pgfqpoint{0.976232in}{0.718668in}}%
\pgfpathcurveto{\pgfqpoint{0.984045in}{0.710855in}}{\pgfqpoint{0.994644in}{0.706464in}}{\pgfqpoint{1.005694in}{0.706464in}}%
\pgfpathclose%
\pgfusepath{stroke,fill}%
\end{pgfscope}%
\begin{pgfscope}%
\pgfpathrectangle{\pgfqpoint{0.375000in}{0.330000in}}{\pgfqpoint{2.325000in}{2.310000in}}%
\pgfusepath{clip}%
\pgfsetbuttcap%
\pgfsetroundjoin%
\definecolor{currentfill}{rgb}{0.000000,0.000000,0.000000}%
\pgfsetfillcolor{currentfill}%
\pgfsetlinewidth{1.003750pt}%
\definecolor{currentstroke}{rgb}{0.000000,0.000000,0.000000}%
\pgfsetstrokecolor{currentstroke}%
\pgfsetdash{}{0pt}%
\pgfpathmoveto{\pgfqpoint{1.005694in}{0.654433in}}%
\pgfpathcurveto{\pgfqpoint{1.016744in}{0.654433in}}{\pgfqpoint{1.027343in}{0.658824in}}{\pgfqpoint{1.035157in}{0.666637in}}%
\pgfpathcurveto{\pgfqpoint{1.042971in}{0.674451in}}{\pgfqpoint{1.047361in}{0.685050in}}{\pgfqpoint{1.047361in}{0.696100in}}%
\pgfpathcurveto{\pgfqpoint{1.047361in}{0.707150in}}{\pgfqpoint{1.042971in}{0.717749in}}{\pgfqpoint{1.035157in}{0.725563in}}%
\pgfpathcurveto{\pgfqpoint{1.027343in}{0.733377in}}{\pgfqpoint{1.016744in}{0.737767in}}{\pgfqpoint{1.005694in}{0.737767in}}%
\pgfpathcurveto{\pgfqpoint{0.994644in}{0.737767in}}{\pgfqpoint{0.984045in}{0.733377in}}{\pgfqpoint{0.976232in}{0.725563in}}%
\pgfpathcurveto{\pgfqpoint{0.968418in}{0.717749in}}{\pgfqpoint{0.964028in}{0.707150in}}{\pgfqpoint{0.964028in}{0.696100in}}%
\pgfpathcurveto{\pgfqpoint{0.964028in}{0.685050in}}{\pgfqpoint{0.968418in}{0.674451in}}{\pgfqpoint{0.976232in}{0.666637in}}%
\pgfpathcurveto{\pgfqpoint{0.984045in}{0.658824in}}{\pgfqpoint{0.994644in}{0.654433in}}{\pgfqpoint{1.005694in}{0.654433in}}%
\pgfpathclose%
\pgfusepath{stroke,fill}%
\end{pgfscope}%
\begin{pgfscope}%
\pgfpathrectangle{\pgfqpoint{0.375000in}{0.330000in}}{\pgfqpoint{2.325000in}{2.310000in}}%
\pgfusepath{clip}%
\pgfsetbuttcap%
\pgfsetroundjoin%
\definecolor{currentfill}{rgb}{0.000000,0.000000,0.000000}%
\pgfsetfillcolor{currentfill}%
\pgfsetlinewidth{1.003750pt}%
\definecolor{currentstroke}{rgb}{0.000000,0.000000,0.000000}%
\pgfsetstrokecolor{currentstroke}%
\pgfsetdash{}{0pt}%
\pgfpathmoveto{\pgfqpoint{1.005694in}{0.706464in}}%
\pgfpathcurveto{\pgfqpoint{1.016744in}{0.706464in}}{\pgfqpoint{1.027343in}{0.710855in}}{\pgfqpoint{1.035157in}{0.718668in}}%
\pgfpathcurveto{\pgfqpoint{1.042971in}{0.726482in}}{\pgfqpoint{1.047361in}{0.737081in}}{\pgfqpoint{1.047361in}{0.748131in}}%
\pgfpathcurveto{\pgfqpoint{1.047361in}{0.759181in}}{\pgfqpoint{1.042971in}{0.769780in}}{\pgfqpoint{1.035157in}{0.777594in}}%
\pgfpathcurveto{\pgfqpoint{1.027343in}{0.785407in}}{\pgfqpoint{1.016744in}{0.789798in}}{\pgfqpoint{1.005694in}{0.789798in}}%
\pgfpathcurveto{\pgfqpoint{0.994644in}{0.789798in}}{\pgfqpoint{0.984045in}{0.785407in}}{\pgfqpoint{0.976232in}{0.777594in}}%
\pgfpathcurveto{\pgfqpoint{0.968418in}{0.769780in}}{\pgfqpoint{0.964028in}{0.759181in}}{\pgfqpoint{0.964028in}{0.748131in}}%
\pgfpathcurveto{\pgfqpoint{0.964028in}{0.737081in}}{\pgfqpoint{0.968418in}{0.726482in}}{\pgfqpoint{0.976232in}{0.718668in}}%
\pgfpathcurveto{\pgfqpoint{0.984045in}{0.710855in}}{\pgfqpoint{0.994644in}{0.706464in}}{\pgfqpoint{1.005694in}{0.706464in}}%
\pgfpathclose%
\pgfusepath{stroke,fill}%
\end{pgfscope}%
\begin{pgfscope}%
\pgfpathrectangle{\pgfqpoint{0.375000in}{0.330000in}}{\pgfqpoint{2.325000in}{2.310000in}}%
\pgfusepath{clip}%
\pgfsetbuttcap%
\pgfsetroundjoin%
\definecolor{currentfill}{rgb}{0.000000,0.000000,0.000000}%
\pgfsetfillcolor{currentfill}%
\pgfsetlinewidth{1.003750pt}%
\definecolor{currentstroke}{rgb}{0.000000,0.000000,0.000000}%
\pgfsetstrokecolor{currentstroke}%
\pgfsetdash{}{0pt}%
\pgfpathmoveto{\pgfqpoint{1.005694in}{0.758495in}}%
\pgfpathcurveto{\pgfqpoint{1.016744in}{0.758495in}}{\pgfqpoint{1.027343in}{0.762885in}}{\pgfqpoint{1.035157in}{0.770699in}}%
\pgfpathcurveto{\pgfqpoint{1.042971in}{0.778513in}}{\pgfqpoint{1.047361in}{0.789112in}}{\pgfqpoint{1.047361in}{0.800162in}}%
\pgfpathcurveto{\pgfqpoint{1.047361in}{0.811212in}}{\pgfqpoint{1.042971in}{0.821811in}}{\pgfqpoint{1.035157in}{0.829625in}}%
\pgfpathcurveto{\pgfqpoint{1.027343in}{0.837438in}}{\pgfqpoint{1.016744in}{0.841828in}}{\pgfqpoint{1.005694in}{0.841828in}}%
\pgfpathcurveto{\pgfqpoint{0.994644in}{0.841828in}}{\pgfqpoint{0.984045in}{0.837438in}}{\pgfqpoint{0.976232in}{0.829625in}}%
\pgfpathcurveto{\pgfqpoint{0.968418in}{0.821811in}}{\pgfqpoint{0.964028in}{0.811212in}}{\pgfqpoint{0.964028in}{0.800162in}}%
\pgfpathcurveto{\pgfqpoint{0.964028in}{0.789112in}}{\pgfqpoint{0.968418in}{0.778513in}}{\pgfqpoint{0.976232in}{0.770699in}}%
\pgfpathcurveto{\pgfqpoint{0.984045in}{0.762885in}}{\pgfqpoint{0.994644in}{0.758495in}}{\pgfqpoint{1.005694in}{0.758495in}}%
\pgfpathclose%
\pgfusepath{stroke,fill}%
\end{pgfscope}%
\begin{pgfscope}%
\pgfpathrectangle{\pgfqpoint{0.375000in}{0.330000in}}{\pgfqpoint{2.325000in}{2.310000in}}%
\pgfusepath{clip}%
\pgfsetbuttcap%
\pgfsetroundjoin%
\definecolor{currentfill}{rgb}{0.000000,0.000000,0.000000}%
\pgfsetfillcolor{currentfill}%
\pgfsetlinewidth{1.003750pt}%
\definecolor{currentstroke}{rgb}{0.000000,0.000000,0.000000}%
\pgfsetstrokecolor{currentstroke}%
\pgfsetdash{}{0pt}%
\pgfpathmoveto{\pgfqpoint{1.005694in}{0.758495in}}%
\pgfpathcurveto{\pgfqpoint{1.016744in}{0.758495in}}{\pgfqpoint{1.027343in}{0.762885in}}{\pgfqpoint{1.035157in}{0.770699in}}%
\pgfpathcurveto{\pgfqpoint{1.042971in}{0.778513in}}{\pgfqpoint{1.047361in}{0.789112in}}{\pgfqpoint{1.047361in}{0.800162in}}%
\pgfpathcurveto{\pgfqpoint{1.047361in}{0.811212in}}{\pgfqpoint{1.042971in}{0.821811in}}{\pgfqpoint{1.035157in}{0.829625in}}%
\pgfpathcurveto{\pgfqpoint{1.027343in}{0.837438in}}{\pgfqpoint{1.016744in}{0.841828in}}{\pgfqpoint{1.005694in}{0.841828in}}%
\pgfpathcurveto{\pgfqpoint{0.994644in}{0.841828in}}{\pgfqpoint{0.984045in}{0.837438in}}{\pgfqpoint{0.976232in}{0.829625in}}%
\pgfpathcurveto{\pgfqpoint{0.968418in}{0.821811in}}{\pgfqpoint{0.964028in}{0.811212in}}{\pgfqpoint{0.964028in}{0.800162in}}%
\pgfpathcurveto{\pgfqpoint{0.964028in}{0.789112in}}{\pgfqpoint{0.968418in}{0.778513in}}{\pgfqpoint{0.976232in}{0.770699in}}%
\pgfpathcurveto{\pgfqpoint{0.984045in}{0.762885in}}{\pgfqpoint{0.994644in}{0.758495in}}{\pgfqpoint{1.005694in}{0.758495in}}%
\pgfpathclose%
\pgfusepath{stroke,fill}%
\end{pgfscope}%
\begin{pgfscope}%
\pgfpathrectangle{\pgfqpoint{0.375000in}{0.330000in}}{\pgfqpoint{2.325000in}{2.310000in}}%
\pgfusepath{clip}%
\pgfsetbuttcap%
\pgfsetroundjoin%
\definecolor{currentfill}{rgb}{0.000000,0.000000,0.000000}%
\pgfsetfillcolor{currentfill}%
\pgfsetlinewidth{1.003750pt}%
\definecolor{currentstroke}{rgb}{0.000000,0.000000,0.000000}%
\pgfsetstrokecolor{currentstroke}%
\pgfsetdash{}{0pt}%
\pgfpathmoveto{\pgfqpoint{1.005694in}{0.758495in}}%
\pgfpathcurveto{\pgfqpoint{1.016744in}{0.758495in}}{\pgfqpoint{1.027343in}{0.762885in}}{\pgfqpoint{1.035157in}{0.770699in}}%
\pgfpathcurveto{\pgfqpoint{1.042971in}{0.778513in}}{\pgfqpoint{1.047361in}{0.789112in}}{\pgfqpoint{1.047361in}{0.800162in}}%
\pgfpathcurveto{\pgfqpoint{1.047361in}{0.811212in}}{\pgfqpoint{1.042971in}{0.821811in}}{\pgfqpoint{1.035157in}{0.829625in}}%
\pgfpathcurveto{\pgfqpoint{1.027343in}{0.837438in}}{\pgfqpoint{1.016744in}{0.841828in}}{\pgfqpoint{1.005694in}{0.841828in}}%
\pgfpathcurveto{\pgfqpoint{0.994644in}{0.841828in}}{\pgfqpoint{0.984045in}{0.837438in}}{\pgfqpoint{0.976232in}{0.829625in}}%
\pgfpathcurveto{\pgfqpoint{0.968418in}{0.821811in}}{\pgfqpoint{0.964028in}{0.811212in}}{\pgfqpoint{0.964028in}{0.800162in}}%
\pgfpathcurveto{\pgfqpoint{0.964028in}{0.789112in}}{\pgfqpoint{0.968418in}{0.778513in}}{\pgfqpoint{0.976232in}{0.770699in}}%
\pgfpathcurveto{\pgfqpoint{0.984045in}{0.762885in}}{\pgfqpoint{0.994644in}{0.758495in}}{\pgfqpoint{1.005694in}{0.758495in}}%
\pgfpathclose%
\pgfusepath{stroke,fill}%
\end{pgfscope}%
\begin{pgfscope}%
\pgfpathrectangle{\pgfqpoint{0.375000in}{0.330000in}}{\pgfqpoint{2.325000in}{2.310000in}}%
\pgfusepath{clip}%
\pgfsetbuttcap%
\pgfsetroundjoin%
\definecolor{currentfill}{rgb}{0.000000,0.000000,0.000000}%
\pgfsetfillcolor{currentfill}%
\pgfsetlinewidth{1.003750pt}%
\definecolor{currentstroke}{rgb}{0.000000,0.000000,0.000000}%
\pgfsetstrokecolor{currentstroke}%
\pgfsetdash{}{0pt}%
\pgfpathmoveto{\pgfqpoint{1.005694in}{0.706464in}}%
\pgfpathcurveto{\pgfqpoint{1.016744in}{0.706464in}}{\pgfqpoint{1.027343in}{0.710855in}}{\pgfqpoint{1.035157in}{0.718668in}}%
\pgfpathcurveto{\pgfqpoint{1.042971in}{0.726482in}}{\pgfqpoint{1.047361in}{0.737081in}}{\pgfqpoint{1.047361in}{0.748131in}}%
\pgfpathcurveto{\pgfqpoint{1.047361in}{0.759181in}}{\pgfqpoint{1.042971in}{0.769780in}}{\pgfqpoint{1.035157in}{0.777594in}}%
\pgfpathcurveto{\pgfqpoint{1.027343in}{0.785407in}}{\pgfqpoint{1.016744in}{0.789798in}}{\pgfqpoint{1.005694in}{0.789798in}}%
\pgfpathcurveto{\pgfqpoint{0.994644in}{0.789798in}}{\pgfqpoint{0.984045in}{0.785407in}}{\pgfqpoint{0.976232in}{0.777594in}}%
\pgfpathcurveto{\pgfqpoint{0.968418in}{0.769780in}}{\pgfqpoint{0.964028in}{0.759181in}}{\pgfqpoint{0.964028in}{0.748131in}}%
\pgfpathcurveto{\pgfqpoint{0.964028in}{0.737081in}}{\pgfqpoint{0.968418in}{0.726482in}}{\pgfqpoint{0.976232in}{0.718668in}}%
\pgfpathcurveto{\pgfqpoint{0.984045in}{0.710855in}}{\pgfqpoint{0.994644in}{0.706464in}}{\pgfqpoint{1.005694in}{0.706464in}}%
\pgfpathclose%
\pgfusepath{stroke,fill}%
\end{pgfscope}%
\begin{pgfscope}%
\pgfpathrectangle{\pgfqpoint{0.375000in}{0.330000in}}{\pgfqpoint{2.325000in}{2.310000in}}%
\pgfusepath{clip}%
\pgfsetbuttcap%
\pgfsetroundjoin%
\definecolor{currentfill}{rgb}{0.000000,0.000000,0.000000}%
\pgfsetfillcolor{currentfill}%
\pgfsetlinewidth{1.003750pt}%
\definecolor{currentstroke}{rgb}{0.000000,0.000000,0.000000}%
\pgfsetstrokecolor{currentstroke}%
\pgfsetdash{}{0pt}%
\pgfpathmoveto{\pgfqpoint{1.005694in}{0.706464in}}%
\pgfpathcurveto{\pgfqpoint{1.016744in}{0.706464in}}{\pgfqpoint{1.027343in}{0.710855in}}{\pgfqpoint{1.035157in}{0.718668in}}%
\pgfpathcurveto{\pgfqpoint{1.042971in}{0.726482in}}{\pgfqpoint{1.047361in}{0.737081in}}{\pgfqpoint{1.047361in}{0.748131in}}%
\pgfpathcurveto{\pgfqpoint{1.047361in}{0.759181in}}{\pgfqpoint{1.042971in}{0.769780in}}{\pgfqpoint{1.035157in}{0.777594in}}%
\pgfpathcurveto{\pgfqpoint{1.027343in}{0.785407in}}{\pgfqpoint{1.016744in}{0.789798in}}{\pgfqpoint{1.005694in}{0.789798in}}%
\pgfpathcurveto{\pgfqpoint{0.994644in}{0.789798in}}{\pgfqpoint{0.984045in}{0.785407in}}{\pgfqpoint{0.976232in}{0.777594in}}%
\pgfpathcurveto{\pgfqpoint{0.968418in}{0.769780in}}{\pgfqpoint{0.964028in}{0.759181in}}{\pgfqpoint{0.964028in}{0.748131in}}%
\pgfpathcurveto{\pgfqpoint{0.964028in}{0.737081in}}{\pgfqpoint{0.968418in}{0.726482in}}{\pgfqpoint{0.976232in}{0.718668in}}%
\pgfpathcurveto{\pgfqpoint{0.984045in}{0.710855in}}{\pgfqpoint{0.994644in}{0.706464in}}{\pgfqpoint{1.005694in}{0.706464in}}%
\pgfpathclose%
\pgfusepath{stroke,fill}%
\end{pgfscope}%
\begin{pgfscope}%
\pgfpathrectangle{\pgfqpoint{0.375000in}{0.330000in}}{\pgfqpoint{2.325000in}{2.310000in}}%
\pgfusepath{clip}%
\pgfsetbuttcap%
\pgfsetroundjoin%
\definecolor{currentfill}{rgb}{0.000000,0.000000,0.000000}%
\pgfsetfillcolor{currentfill}%
\pgfsetlinewidth{1.003750pt}%
\definecolor{currentstroke}{rgb}{0.000000,0.000000,0.000000}%
\pgfsetstrokecolor{currentstroke}%
\pgfsetdash{}{0pt}%
\pgfpathmoveto{\pgfqpoint{1.005694in}{0.706464in}}%
\pgfpathcurveto{\pgfqpoint{1.016744in}{0.706464in}}{\pgfqpoint{1.027343in}{0.710855in}}{\pgfqpoint{1.035157in}{0.718668in}}%
\pgfpathcurveto{\pgfqpoint{1.042971in}{0.726482in}}{\pgfqpoint{1.047361in}{0.737081in}}{\pgfqpoint{1.047361in}{0.748131in}}%
\pgfpathcurveto{\pgfqpoint{1.047361in}{0.759181in}}{\pgfqpoint{1.042971in}{0.769780in}}{\pgfqpoint{1.035157in}{0.777594in}}%
\pgfpathcurveto{\pgfqpoint{1.027343in}{0.785407in}}{\pgfqpoint{1.016744in}{0.789798in}}{\pgfqpoint{1.005694in}{0.789798in}}%
\pgfpathcurveto{\pgfqpoint{0.994644in}{0.789798in}}{\pgfqpoint{0.984045in}{0.785407in}}{\pgfqpoint{0.976232in}{0.777594in}}%
\pgfpathcurveto{\pgfqpoint{0.968418in}{0.769780in}}{\pgfqpoint{0.964028in}{0.759181in}}{\pgfqpoint{0.964028in}{0.748131in}}%
\pgfpathcurveto{\pgfqpoint{0.964028in}{0.737081in}}{\pgfqpoint{0.968418in}{0.726482in}}{\pgfqpoint{0.976232in}{0.718668in}}%
\pgfpathcurveto{\pgfqpoint{0.984045in}{0.710855in}}{\pgfqpoint{0.994644in}{0.706464in}}{\pgfqpoint{1.005694in}{0.706464in}}%
\pgfpathclose%
\pgfusepath{stroke,fill}%
\end{pgfscope}%
\begin{pgfscope}%
\pgfpathrectangle{\pgfqpoint{0.375000in}{0.330000in}}{\pgfqpoint{2.325000in}{2.310000in}}%
\pgfusepath{clip}%
\pgfsetbuttcap%
\pgfsetroundjoin%
\definecolor{currentfill}{rgb}{0.000000,0.000000,0.000000}%
\pgfsetfillcolor{currentfill}%
\pgfsetlinewidth{1.003750pt}%
\definecolor{currentstroke}{rgb}{0.000000,0.000000,0.000000}%
\pgfsetstrokecolor{currentstroke}%
\pgfsetdash{}{0pt}%
\pgfpathmoveto{\pgfqpoint{1.005694in}{0.706464in}}%
\pgfpathcurveto{\pgfqpoint{1.016744in}{0.706464in}}{\pgfqpoint{1.027343in}{0.710855in}}{\pgfqpoint{1.035157in}{0.718668in}}%
\pgfpathcurveto{\pgfqpoint{1.042971in}{0.726482in}}{\pgfqpoint{1.047361in}{0.737081in}}{\pgfqpoint{1.047361in}{0.748131in}}%
\pgfpathcurveto{\pgfqpoint{1.047361in}{0.759181in}}{\pgfqpoint{1.042971in}{0.769780in}}{\pgfqpoint{1.035157in}{0.777594in}}%
\pgfpathcurveto{\pgfqpoint{1.027343in}{0.785407in}}{\pgfqpoint{1.016744in}{0.789798in}}{\pgfqpoint{1.005694in}{0.789798in}}%
\pgfpathcurveto{\pgfqpoint{0.994644in}{0.789798in}}{\pgfqpoint{0.984045in}{0.785407in}}{\pgfqpoint{0.976232in}{0.777594in}}%
\pgfpathcurveto{\pgfqpoint{0.968418in}{0.769780in}}{\pgfqpoint{0.964028in}{0.759181in}}{\pgfqpoint{0.964028in}{0.748131in}}%
\pgfpathcurveto{\pgfqpoint{0.964028in}{0.737081in}}{\pgfqpoint{0.968418in}{0.726482in}}{\pgfqpoint{0.976232in}{0.718668in}}%
\pgfpathcurveto{\pgfqpoint{0.984045in}{0.710855in}}{\pgfqpoint{0.994644in}{0.706464in}}{\pgfqpoint{1.005694in}{0.706464in}}%
\pgfpathclose%
\pgfusepath{stroke,fill}%
\end{pgfscope}%
\begin{pgfscope}%
\pgfpathrectangle{\pgfqpoint{0.375000in}{0.330000in}}{\pgfqpoint{2.325000in}{2.310000in}}%
\pgfusepath{clip}%
\pgfsetbuttcap%
\pgfsetroundjoin%
\definecolor{currentfill}{rgb}{0.000000,0.000000,0.000000}%
\pgfsetfillcolor{currentfill}%
\pgfsetlinewidth{1.003750pt}%
\definecolor{currentstroke}{rgb}{0.000000,0.000000,0.000000}%
\pgfsetstrokecolor{currentstroke}%
\pgfsetdash{}{0pt}%
\pgfpathmoveto{\pgfqpoint{1.005694in}{0.758495in}}%
\pgfpathcurveto{\pgfqpoint{1.016744in}{0.758495in}}{\pgfqpoint{1.027343in}{0.762885in}}{\pgfqpoint{1.035157in}{0.770699in}}%
\pgfpathcurveto{\pgfqpoint{1.042971in}{0.778513in}}{\pgfqpoint{1.047361in}{0.789112in}}{\pgfqpoint{1.047361in}{0.800162in}}%
\pgfpathcurveto{\pgfqpoint{1.047361in}{0.811212in}}{\pgfqpoint{1.042971in}{0.821811in}}{\pgfqpoint{1.035157in}{0.829625in}}%
\pgfpathcurveto{\pgfqpoint{1.027343in}{0.837438in}}{\pgfqpoint{1.016744in}{0.841828in}}{\pgfqpoint{1.005694in}{0.841828in}}%
\pgfpathcurveto{\pgfqpoint{0.994644in}{0.841828in}}{\pgfqpoint{0.984045in}{0.837438in}}{\pgfqpoint{0.976232in}{0.829625in}}%
\pgfpathcurveto{\pgfqpoint{0.968418in}{0.821811in}}{\pgfqpoint{0.964028in}{0.811212in}}{\pgfqpoint{0.964028in}{0.800162in}}%
\pgfpathcurveto{\pgfqpoint{0.964028in}{0.789112in}}{\pgfqpoint{0.968418in}{0.778513in}}{\pgfqpoint{0.976232in}{0.770699in}}%
\pgfpathcurveto{\pgfqpoint{0.984045in}{0.762885in}}{\pgfqpoint{0.994644in}{0.758495in}}{\pgfqpoint{1.005694in}{0.758495in}}%
\pgfpathclose%
\pgfusepath{stroke,fill}%
\end{pgfscope}%
\begin{pgfscope}%
\pgfpathrectangle{\pgfqpoint{0.375000in}{0.330000in}}{\pgfqpoint{2.325000in}{2.310000in}}%
\pgfusepath{clip}%
\pgfsetbuttcap%
\pgfsetroundjoin%
\definecolor{currentfill}{rgb}{0.000000,0.000000,0.000000}%
\pgfsetfillcolor{currentfill}%
\pgfsetlinewidth{1.003750pt}%
\definecolor{currentstroke}{rgb}{0.000000,0.000000,0.000000}%
\pgfsetstrokecolor{currentstroke}%
\pgfsetdash{}{0pt}%
\pgfpathmoveto{\pgfqpoint{1.005694in}{0.706464in}}%
\pgfpathcurveto{\pgfqpoint{1.016744in}{0.706464in}}{\pgfqpoint{1.027343in}{0.710855in}}{\pgfqpoint{1.035157in}{0.718668in}}%
\pgfpathcurveto{\pgfqpoint{1.042971in}{0.726482in}}{\pgfqpoint{1.047361in}{0.737081in}}{\pgfqpoint{1.047361in}{0.748131in}}%
\pgfpathcurveto{\pgfqpoint{1.047361in}{0.759181in}}{\pgfqpoint{1.042971in}{0.769780in}}{\pgfqpoint{1.035157in}{0.777594in}}%
\pgfpathcurveto{\pgfqpoint{1.027343in}{0.785407in}}{\pgfqpoint{1.016744in}{0.789798in}}{\pgfqpoint{1.005694in}{0.789798in}}%
\pgfpathcurveto{\pgfqpoint{0.994644in}{0.789798in}}{\pgfqpoint{0.984045in}{0.785407in}}{\pgfqpoint{0.976232in}{0.777594in}}%
\pgfpathcurveto{\pgfqpoint{0.968418in}{0.769780in}}{\pgfqpoint{0.964028in}{0.759181in}}{\pgfqpoint{0.964028in}{0.748131in}}%
\pgfpathcurveto{\pgfqpoint{0.964028in}{0.737081in}}{\pgfqpoint{0.968418in}{0.726482in}}{\pgfqpoint{0.976232in}{0.718668in}}%
\pgfpathcurveto{\pgfqpoint{0.984045in}{0.710855in}}{\pgfqpoint{0.994644in}{0.706464in}}{\pgfqpoint{1.005694in}{0.706464in}}%
\pgfpathclose%
\pgfusepath{stroke,fill}%
\end{pgfscope}%
\begin{pgfscope}%
\pgfpathrectangle{\pgfqpoint{0.375000in}{0.330000in}}{\pgfqpoint{2.325000in}{2.310000in}}%
\pgfusepath{clip}%
\pgfsetbuttcap%
\pgfsetroundjoin%
\definecolor{currentfill}{rgb}{0.000000,0.000000,0.000000}%
\pgfsetfillcolor{currentfill}%
\pgfsetlinewidth{1.003750pt}%
\definecolor{currentstroke}{rgb}{0.000000,0.000000,0.000000}%
\pgfsetstrokecolor{currentstroke}%
\pgfsetdash{}{0pt}%
\pgfpathmoveto{\pgfqpoint{1.530548in}{1.122711in}}%
\pgfpathcurveto{\pgfqpoint{1.541598in}{1.122711in}}{\pgfqpoint{1.552197in}{1.127101in}}{\pgfqpoint{1.560011in}{1.134915in}}%
\pgfpathcurveto{\pgfqpoint{1.567825in}{1.142728in}}{\pgfqpoint{1.572215in}{1.153327in}}{\pgfqpoint{1.572215in}{1.164378in}}%
\pgfpathcurveto{\pgfqpoint{1.572215in}{1.175428in}}{\pgfqpoint{1.567825in}{1.186027in}}{\pgfqpoint{1.560011in}{1.193840in}}%
\pgfpathcurveto{\pgfqpoint{1.552197in}{1.201654in}}{\pgfqpoint{1.541598in}{1.206044in}}{\pgfqpoint{1.530548in}{1.206044in}}%
\pgfpathcurveto{\pgfqpoint{1.519498in}{1.206044in}}{\pgfqpoint{1.508899in}{1.201654in}}{\pgfqpoint{1.501085in}{1.193840in}}%
\pgfpathcurveto{\pgfqpoint{1.493272in}{1.186027in}}{\pgfqpoint{1.488881in}{1.175428in}}{\pgfqpoint{1.488881in}{1.164378in}}%
\pgfpathcurveto{\pgfqpoint{1.488881in}{1.153327in}}{\pgfqpoint{1.493272in}{1.142728in}}{\pgfqpoint{1.501085in}{1.134915in}}%
\pgfpathcurveto{\pgfqpoint{1.508899in}{1.127101in}}{\pgfqpoint{1.519498in}{1.122711in}}{\pgfqpoint{1.530548in}{1.122711in}}%
\pgfpathclose%
\pgfusepath{stroke,fill}%
\end{pgfscope}%
\begin{pgfscope}%
\pgfpathrectangle{\pgfqpoint{0.375000in}{0.330000in}}{\pgfqpoint{2.325000in}{2.310000in}}%
\pgfusepath{clip}%
\pgfsetbuttcap%
\pgfsetroundjoin%
\definecolor{currentfill}{rgb}{0.000000,0.000000,0.000000}%
\pgfsetfillcolor{currentfill}%
\pgfsetlinewidth{1.003750pt}%
\definecolor{currentstroke}{rgb}{0.000000,0.000000,0.000000}%
\pgfsetstrokecolor{currentstroke}%
\pgfsetdash{}{0pt}%
\pgfpathmoveto{\pgfqpoint{1.530548in}{1.070680in}}%
\pgfpathcurveto{\pgfqpoint{1.541598in}{1.070680in}}{\pgfqpoint{1.552197in}{1.075070in}}{\pgfqpoint{1.560011in}{1.082884in}}%
\pgfpathcurveto{\pgfqpoint{1.567825in}{1.090698in}}{\pgfqpoint{1.572215in}{1.101297in}}{\pgfqpoint{1.572215in}{1.112347in}}%
\pgfpathcurveto{\pgfqpoint{1.572215in}{1.123397in}}{\pgfqpoint{1.567825in}{1.133996in}}{\pgfqpoint{1.560011in}{1.141810in}}%
\pgfpathcurveto{\pgfqpoint{1.552197in}{1.149623in}}{\pgfqpoint{1.541598in}{1.154013in}}{\pgfqpoint{1.530548in}{1.154013in}}%
\pgfpathcurveto{\pgfqpoint{1.519498in}{1.154013in}}{\pgfqpoint{1.508899in}{1.149623in}}{\pgfqpoint{1.501085in}{1.141810in}}%
\pgfpathcurveto{\pgfqpoint{1.493272in}{1.133996in}}{\pgfqpoint{1.488881in}{1.123397in}}{\pgfqpoint{1.488881in}{1.112347in}}%
\pgfpathcurveto{\pgfqpoint{1.488881in}{1.101297in}}{\pgfqpoint{1.493272in}{1.090698in}}{\pgfqpoint{1.501085in}{1.082884in}}%
\pgfpathcurveto{\pgfqpoint{1.508899in}{1.075070in}}{\pgfqpoint{1.519498in}{1.070680in}}{\pgfqpoint{1.530548in}{1.070680in}}%
\pgfpathclose%
\pgfusepath{stroke,fill}%
\end{pgfscope}%
\begin{pgfscope}%
\pgfpathrectangle{\pgfqpoint{0.375000in}{0.330000in}}{\pgfqpoint{2.325000in}{2.310000in}}%
\pgfusepath{clip}%
\pgfsetbuttcap%
\pgfsetroundjoin%
\definecolor{currentfill}{rgb}{0.000000,0.000000,0.000000}%
\pgfsetfillcolor{currentfill}%
\pgfsetlinewidth{1.003750pt}%
\definecolor{currentstroke}{rgb}{0.000000,0.000000,0.000000}%
\pgfsetstrokecolor{currentstroke}%
\pgfsetdash{}{0pt}%
\pgfpathmoveto{\pgfqpoint{1.530548in}{1.018649in}}%
\pgfpathcurveto{\pgfqpoint{1.541598in}{1.018649in}}{\pgfqpoint{1.552197in}{1.023040in}}{\pgfqpoint{1.560011in}{1.030853in}}%
\pgfpathcurveto{\pgfqpoint{1.567825in}{1.038667in}}{\pgfqpoint{1.572215in}{1.049266in}}{\pgfqpoint{1.572215in}{1.060316in}}%
\pgfpathcurveto{\pgfqpoint{1.572215in}{1.071366in}}{\pgfqpoint{1.567825in}{1.081965in}}{\pgfqpoint{1.560011in}{1.089779in}}%
\pgfpathcurveto{\pgfqpoint{1.552197in}{1.097592in}}{\pgfqpoint{1.541598in}{1.101983in}}{\pgfqpoint{1.530548in}{1.101983in}}%
\pgfpathcurveto{\pgfqpoint{1.519498in}{1.101983in}}{\pgfqpoint{1.508899in}{1.097592in}}{\pgfqpoint{1.501085in}{1.089779in}}%
\pgfpathcurveto{\pgfqpoint{1.493272in}{1.081965in}}{\pgfqpoint{1.488881in}{1.071366in}}{\pgfqpoint{1.488881in}{1.060316in}}%
\pgfpathcurveto{\pgfqpoint{1.488881in}{1.049266in}}{\pgfqpoint{1.493272in}{1.038667in}}{\pgfqpoint{1.501085in}{1.030853in}}%
\pgfpathcurveto{\pgfqpoint{1.508899in}{1.023040in}}{\pgfqpoint{1.519498in}{1.018649in}}{\pgfqpoint{1.530548in}{1.018649in}}%
\pgfpathclose%
\pgfusepath{stroke,fill}%
\end{pgfscope}%
\begin{pgfscope}%
\pgfpathrectangle{\pgfqpoint{0.375000in}{0.330000in}}{\pgfqpoint{2.325000in}{2.310000in}}%
\pgfusepath{clip}%
\pgfsetbuttcap%
\pgfsetroundjoin%
\definecolor{currentfill}{rgb}{0.000000,0.000000,0.000000}%
\pgfsetfillcolor{currentfill}%
\pgfsetlinewidth{1.003750pt}%
\definecolor{currentstroke}{rgb}{0.000000,0.000000,0.000000}%
\pgfsetstrokecolor{currentstroke}%
\pgfsetdash{}{0pt}%
\pgfpathmoveto{\pgfqpoint{1.530548in}{1.018649in}}%
\pgfpathcurveto{\pgfqpoint{1.541598in}{1.018649in}}{\pgfqpoint{1.552197in}{1.023040in}}{\pgfqpoint{1.560011in}{1.030853in}}%
\pgfpathcurveto{\pgfqpoint{1.567825in}{1.038667in}}{\pgfqpoint{1.572215in}{1.049266in}}{\pgfqpoint{1.572215in}{1.060316in}}%
\pgfpathcurveto{\pgfqpoint{1.572215in}{1.071366in}}{\pgfqpoint{1.567825in}{1.081965in}}{\pgfqpoint{1.560011in}{1.089779in}}%
\pgfpathcurveto{\pgfqpoint{1.552197in}{1.097592in}}{\pgfqpoint{1.541598in}{1.101983in}}{\pgfqpoint{1.530548in}{1.101983in}}%
\pgfpathcurveto{\pgfqpoint{1.519498in}{1.101983in}}{\pgfqpoint{1.508899in}{1.097592in}}{\pgfqpoint{1.501085in}{1.089779in}}%
\pgfpathcurveto{\pgfqpoint{1.493272in}{1.081965in}}{\pgfqpoint{1.488881in}{1.071366in}}{\pgfqpoint{1.488881in}{1.060316in}}%
\pgfpathcurveto{\pgfqpoint{1.488881in}{1.049266in}}{\pgfqpoint{1.493272in}{1.038667in}}{\pgfqpoint{1.501085in}{1.030853in}}%
\pgfpathcurveto{\pgfqpoint{1.508899in}{1.023040in}}{\pgfqpoint{1.519498in}{1.018649in}}{\pgfqpoint{1.530548in}{1.018649in}}%
\pgfpathclose%
\pgfusepath{stroke,fill}%
\end{pgfscope}%
\begin{pgfscope}%
\pgfpathrectangle{\pgfqpoint{0.375000in}{0.330000in}}{\pgfqpoint{2.325000in}{2.310000in}}%
\pgfusepath{clip}%
\pgfsetbuttcap%
\pgfsetroundjoin%
\definecolor{currentfill}{rgb}{0.000000,0.000000,0.000000}%
\pgfsetfillcolor{currentfill}%
\pgfsetlinewidth{1.003750pt}%
\definecolor{currentstroke}{rgb}{0.000000,0.000000,0.000000}%
\pgfsetstrokecolor{currentstroke}%
\pgfsetdash{}{0pt}%
\pgfpathmoveto{\pgfqpoint{1.530548in}{1.122711in}}%
\pgfpathcurveto{\pgfqpoint{1.541598in}{1.122711in}}{\pgfqpoint{1.552197in}{1.127101in}}{\pgfqpoint{1.560011in}{1.134915in}}%
\pgfpathcurveto{\pgfqpoint{1.567825in}{1.142728in}}{\pgfqpoint{1.572215in}{1.153327in}}{\pgfqpoint{1.572215in}{1.164378in}}%
\pgfpathcurveto{\pgfqpoint{1.572215in}{1.175428in}}{\pgfqpoint{1.567825in}{1.186027in}}{\pgfqpoint{1.560011in}{1.193840in}}%
\pgfpathcurveto{\pgfqpoint{1.552197in}{1.201654in}}{\pgfqpoint{1.541598in}{1.206044in}}{\pgfqpoint{1.530548in}{1.206044in}}%
\pgfpathcurveto{\pgfqpoint{1.519498in}{1.206044in}}{\pgfqpoint{1.508899in}{1.201654in}}{\pgfqpoint{1.501085in}{1.193840in}}%
\pgfpathcurveto{\pgfqpoint{1.493272in}{1.186027in}}{\pgfqpoint{1.488881in}{1.175428in}}{\pgfqpoint{1.488881in}{1.164378in}}%
\pgfpathcurveto{\pgfqpoint{1.488881in}{1.153327in}}{\pgfqpoint{1.493272in}{1.142728in}}{\pgfqpoint{1.501085in}{1.134915in}}%
\pgfpathcurveto{\pgfqpoint{1.508899in}{1.127101in}}{\pgfqpoint{1.519498in}{1.122711in}}{\pgfqpoint{1.530548in}{1.122711in}}%
\pgfpathclose%
\pgfusepath{stroke,fill}%
\end{pgfscope}%
\begin{pgfscope}%
\pgfpathrectangle{\pgfqpoint{0.375000in}{0.330000in}}{\pgfqpoint{2.325000in}{2.310000in}}%
\pgfusepath{clip}%
\pgfsetbuttcap%
\pgfsetroundjoin%
\definecolor{currentfill}{rgb}{0.000000,0.000000,0.000000}%
\pgfsetfillcolor{currentfill}%
\pgfsetlinewidth{1.003750pt}%
\definecolor{currentstroke}{rgb}{0.000000,0.000000,0.000000}%
\pgfsetstrokecolor{currentstroke}%
\pgfsetdash{}{0pt}%
\pgfpathmoveto{\pgfqpoint{1.530548in}{1.070680in}}%
\pgfpathcurveto{\pgfqpoint{1.541598in}{1.070680in}}{\pgfqpoint{1.552197in}{1.075070in}}{\pgfqpoint{1.560011in}{1.082884in}}%
\pgfpathcurveto{\pgfqpoint{1.567825in}{1.090698in}}{\pgfqpoint{1.572215in}{1.101297in}}{\pgfqpoint{1.572215in}{1.112347in}}%
\pgfpathcurveto{\pgfqpoint{1.572215in}{1.123397in}}{\pgfqpoint{1.567825in}{1.133996in}}{\pgfqpoint{1.560011in}{1.141810in}}%
\pgfpathcurveto{\pgfqpoint{1.552197in}{1.149623in}}{\pgfqpoint{1.541598in}{1.154013in}}{\pgfqpoint{1.530548in}{1.154013in}}%
\pgfpathcurveto{\pgfqpoint{1.519498in}{1.154013in}}{\pgfqpoint{1.508899in}{1.149623in}}{\pgfqpoint{1.501085in}{1.141810in}}%
\pgfpathcurveto{\pgfqpoint{1.493272in}{1.133996in}}{\pgfqpoint{1.488881in}{1.123397in}}{\pgfqpoint{1.488881in}{1.112347in}}%
\pgfpathcurveto{\pgfqpoint{1.488881in}{1.101297in}}{\pgfqpoint{1.493272in}{1.090698in}}{\pgfqpoint{1.501085in}{1.082884in}}%
\pgfpathcurveto{\pgfqpoint{1.508899in}{1.075070in}}{\pgfqpoint{1.519498in}{1.070680in}}{\pgfqpoint{1.530548in}{1.070680in}}%
\pgfpathclose%
\pgfusepath{stroke,fill}%
\end{pgfscope}%
\begin{pgfscope}%
\pgfpathrectangle{\pgfqpoint{0.375000in}{0.330000in}}{\pgfqpoint{2.325000in}{2.310000in}}%
\pgfusepath{clip}%
\pgfsetbuttcap%
\pgfsetroundjoin%
\definecolor{currentfill}{rgb}{0.000000,0.000000,0.000000}%
\pgfsetfillcolor{currentfill}%
\pgfsetlinewidth{1.003750pt}%
\definecolor{currentstroke}{rgb}{0.000000,0.000000,0.000000}%
\pgfsetstrokecolor{currentstroke}%
\pgfsetdash{}{0pt}%
\pgfpathmoveto{\pgfqpoint{1.530548in}{1.070680in}}%
\pgfpathcurveto{\pgfqpoint{1.541598in}{1.070680in}}{\pgfqpoint{1.552197in}{1.075070in}}{\pgfqpoint{1.560011in}{1.082884in}}%
\pgfpathcurveto{\pgfqpoint{1.567825in}{1.090698in}}{\pgfqpoint{1.572215in}{1.101297in}}{\pgfqpoint{1.572215in}{1.112347in}}%
\pgfpathcurveto{\pgfqpoint{1.572215in}{1.123397in}}{\pgfqpoint{1.567825in}{1.133996in}}{\pgfqpoint{1.560011in}{1.141810in}}%
\pgfpathcurveto{\pgfqpoint{1.552197in}{1.149623in}}{\pgfqpoint{1.541598in}{1.154013in}}{\pgfqpoint{1.530548in}{1.154013in}}%
\pgfpathcurveto{\pgfqpoint{1.519498in}{1.154013in}}{\pgfqpoint{1.508899in}{1.149623in}}{\pgfqpoint{1.501085in}{1.141810in}}%
\pgfpathcurveto{\pgfqpoint{1.493272in}{1.133996in}}{\pgfqpoint{1.488881in}{1.123397in}}{\pgfqpoint{1.488881in}{1.112347in}}%
\pgfpathcurveto{\pgfqpoint{1.488881in}{1.101297in}}{\pgfqpoint{1.493272in}{1.090698in}}{\pgfqpoint{1.501085in}{1.082884in}}%
\pgfpathcurveto{\pgfqpoint{1.508899in}{1.075070in}}{\pgfqpoint{1.519498in}{1.070680in}}{\pgfqpoint{1.530548in}{1.070680in}}%
\pgfpathclose%
\pgfusepath{stroke,fill}%
\end{pgfscope}%
\begin{pgfscope}%
\pgfpathrectangle{\pgfqpoint{0.375000in}{0.330000in}}{\pgfqpoint{2.325000in}{2.310000in}}%
\pgfusepath{clip}%
\pgfsetbuttcap%
\pgfsetroundjoin%
\definecolor{currentfill}{rgb}{0.000000,0.000000,0.000000}%
\pgfsetfillcolor{currentfill}%
\pgfsetlinewidth{1.003750pt}%
\definecolor{currentstroke}{rgb}{0.000000,0.000000,0.000000}%
\pgfsetstrokecolor{currentstroke}%
\pgfsetdash{}{0pt}%
\pgfpathmoveto{\pgfqpoint{1.530548in}{1.070680in}}%
\pgfpathcurveto{\pgfqpoint{1.541598in}{1.070680in}}{\pgfqpoint{1.552197in}{1.075070in}}{\pgfqpoint{1.560011in}{1.082884in}}%
\pgfpathcurveto{\pgfqpoint{1.567825in}{1.090698in}}{\pgfqpoint{1.572215in}{1.101297in}}{\pgfqpoint{1.572215in}{1.112347in}}%
\pgfpathcurveto{\pgfqpoint{1.572215in}{1.123397in}}{\pgfqpoint{1.567825in}{1.133996in}}{\pgfqpoint{1.560011in}{1.141810in}}%
\pgfpathcurveto{\pgfqpoint{1.552197in}{1.149623in}}{\pgfqpoint{1.541598in}{1.154013in}}{\pgfqpoint{1.530548in}{1.154013in}}%
\pgfpathcurveto{\pgfqpoint{1.519498in}{1.154013in}}{\pgfqpoint{1.508899in}{1.149623in}}{\pgfqpoint{1.501085in}{1.141810in}}%
\pgfpathcurveto{\pgfqpoint{1.493272in}{1.133996in}}{\pgfqpoint{1.488881in}{1.123397in}}{\pgfqpoint{1.488881in}{1.112347in}}%
\pgfpathcurveto{\pgfqpoint{1.488881in}{1.101297in}}{\pgfqpoint{1.493272in}{1.090698in}}{\pgfqpoint{1.501085in}{1.082884in}}%
\pgfpathcurveto{\pgfqpoint{1.508899in}{1.075070in}}{\pgfqpoint{1.519498in}{1.070680in}}{\pgfqpoint{1.530548in}{1.070680in}}%
\pgfpathclose%
\pgfusepath{stroke,fill}%
\end{pgfscope}%
\begin{pgfscope}%
\pgfpathrectangle{\pgfqpoint{0.375000in}{0.330000in}}{\pgfqpoint{2.325000in}{2.310000in}}%
\pgfusepath{clip}%
\pgfsetbuttcap%
\pgfsetroundjoin%
\definecolor{currentfill}{rgb}{0.000000,0.000000,0.000000}%
\pgfsetfillcolor{currentfill}%
\pgfsetlinewidth{1.003750pt}%
\definecolor{currentstroke}{rgb}{0.000000,0.000000,0.000000}%
\pgfsetstrokecolor{currentstroke}%
\pgfsetdash{}{0pt}%
\pgfpathmoveto{\pgfqpoint{1.530548in}{1.122711in}}%
\pgfpathcurveto{\pgfqpoint{1.541598in}{1.122711in}}{\pgfqpoint{1.552197in}{1.127101in}}{\pgfqpoint{1.560011in}{1.134915in}}%
\pgfpathcurveto{\pgfqpoint{1.567825in}{1.142728in}}{\pgfqpoint{1.572215in}{1.153327in}}{\pgfqpoint{1.572215in}{1.164378in}}%
\pgfpathcurveto{\pgfqpoint{1.572215in}{1.175428in}}{\pgfqpoint{1.567825in}{1.186027in}}{\pgfqpoint{1.560011in}{1.193840in}}%
\pgfpathcurveto{\pgfqpoint{1.552197in}{1.201654in}}{\pgfqpoint{1.541598in}{1.206044in}}{\pgfqpoint{1.530548in}{1.206044in}}%
\pgfpathcurveto{\pgfqpoint{1.519498in}{1.206044in}}{\pgfqpoint{1.508899in}{1.201654in}}{\pgfqpoint{1.501085in}{1.193840in}}%
\pgfpathcurveto{\pgfqpoint{1.493272in}{1.186027in}}{\pgfqpoint{1.488881in}{1.175428in}}{\pgfqpoint{1.488881in}{1.164378in}}%
\pgfpathcurveto{\pgfqpoint{1.488881in}{1.153327in}}{\pgfqpoint{1.493272in}{1.142728in}}{\pgfqpoint{1.501085in}{1.134915in}}%
\pgfpathcurveto{\pgfqpoint{1.508899in}{1.127101in}}{\pgfqpoint{1.519498in}{1.122711in}}{\pgfqpoint{1.530548in}{1.122711in}}%
\pgfpathclose%
\pgfusepath{stroke,fill}%
\end{pgfscope}%
\begin{pgfscope}%
\pgfpathrectangle{\pgfqpoint{0.375000in}{0.330000in}}{\pgfqpoint{2.325000in}{2.310000in}}%
\pgfusepath{clip}%
\pgfsetbuttcap%
\pgfsetroundjoin%
\definecolor{currentfill}{rgb}{0.000000,0.000000,0.000000}%
\pgfsetfillcolor{currentfill}%
\pgfsetlinewidth{1.003750pt}%
\definecolor{currentstroke}{rgb}{0.000000,0.000000,0.000000}%
\pgfsetstrokecolor{currentstroke}%
\pgfsetdash{}{0pt}%
\pgfpathmoveto{\pgfqpoint{1.530548in}{1.070680in}}%
\pgfpathcurveto{\pgfqpoint{1.541598in}{1.070680in}}{\pgfqpoint{1.552197in}{1.075070in}}{\pgfqpoint{1.560011in}{1.082884in}}%
\pgfpathcurveto{\pgfqpoint{1.567825in}{1.090698in}}{\pgfqpoint{1.572215in}{1.101297in}}{\pgfqpoint{1.572215in}{1.112347in}}%
\pgfpathcurveto{\pgfqpoint{1.572215in}{1.123397in}}{\pgfqpoint{1.567825in}{1.133996in}}{\pgfqpoint{1.560011in}{1.141810in}}%
\pgfpathcurveto{\pgfqpoint{1.552197in}{1.149623in}}{\pgfqpoint{1.541598in}{1.154013in}}{\pgfqpoint{1.530548in}{1.154013in}}%
\pgfpathcurveto{\pgfqpoint{1.519498in}{1.154013in}}{\pgfqpoint{1.508899in}{1.149623in}}{\pgfqpoint{1.501085in}{1.141810in}}%
\pgfpathcurveto{\pgfqpoint{1.493272in}{1.133996in}}{\pgfqpoint{1.488881in}{1.123397in}}{\pgfqpoint{1.488881in}{1.112347in}}%
\pgfpathcurveto{\pgfqpoint{1.488881in}{1.101297in}}{\pgfqpoint{1.493272in}{1.090698in}}{\pgfqpoint{1.501085in}{1.082884in}}%
\pgfpathcurveto{\pgfqpoint{1.508899in}{1.075070in}}{\pgfqpoint{1.519498in}{1.070680in}}{\pgfqpoint{1.530548in}{1.070680in}}%
\pgfpathclose%
\pgfusepath{stroke,fill}%
\end{pgfscope}%
\begin{pgfscope}%
\pgfpathrectangle{\pgfqpoint{0.375000in}{0.330000in}}{\pgfqpoint{2.325000in}{2.310000in}}%
\pgfusepath{clip}%
\pgfsetbuttcap%
\pgfsetroundjoin%
\definecolor{currentfill}{rgb}{0.000000,0.000000,0.000000}%
\pgfsetfillcolor{currentfill}%
\pgfsetlinewidth{1.003750pt}%
\definecolor{currentstroke}{rgb}{0.000000,0.000000,0.000000}%
\pgfsetstrokecolor{currentstroke}%
\pgfsetdash{}{0pt}%
\pgfpathmoveto{\pgfqpoint{1.530548in}{1.122711in}}%
\pgfpathcurveto{\pgfqpoint{1.541598in}{1.122711in}}{\pgfqpoint{1.552197in}{1.127101in}}{\pgfqpoint{1.560011in}{1.134915in}}%
\pgfpathcurveto{\pgfqpoint{1.567825in}{1.142728in}}{\pgfqpoint{1.572215in}{1.153327in}}{\pgfqpoint{1.572215in}{1.164378in}}%
\pgfpathcurveto{\pgfqpoint{1.572215in}{1.175428in}}{\pgfqpoint{1.567825in}{1.186027in}}{\pgfqpoint{1.560011in}{1.193840in}}%
\pgfpathcurveto{\pgfqpoint{1.552197in}{1.201654in}}{\pgfqpoint{1.541598in}{1.206044in}}{\pgfqpoint{1.530548in}{1.206044in}}%
\pgfpathcurveto{\pgfqpoint{1.519498in}{1.206044in}}{\pgfqpoint{1.508899in}{1.201654in}}{\pgfqpoint{1.501085in}{1.193840in}}%
\pgfpathcurveto{\pgfqpoint{1.493272in}{1.186027in}}{\pgfqpoint{1.488881in}{1.175428in}}{\pgfqpoint{1.488881in}{1.164378in}}%
\pgfpathcurveto{\pgfqpoint{1.488881in}{1.153327in}}{\pgfqpoint{1.493272in}{1.142728in}}{\pgfqpoint{1.501085in}{1.134915in}}%
\pgfpathcurveto{\pgfqpoint{1.508899in}{1.127101in}}{\pgfqpoint{1.519498in}{1.122711in}}{\pgfqpoint{1.530548in}{1.122711in}}%
\pgfpathclose%
\pgfusepath{stroke,fill}%
\end{pgfscope}%
\begin{pgfscope}%
\pgfpathrectangle{\pgfqpoint{0.375000in}{0.330000in}}{\pgfqpoint{2.325000in}{2.310000in}}%
\pgfusepath{clip}%
\pgfsetbuttcap%
\pgfsetroundjoin%
\definecolor{currentfill}{rgb}{0.000000,0.000000,0.000000}%
\pgfsetfillcolor{currentfill}%
\pgfsetlinewidth{1.003750pt}%
\definecolor{currentstroke}{rgb}{0.000000,0.000000,0.000000}%
\pgfsetstrokecolor{currentstroke}%
\pgfsetdash{}{0pt}%
\pgfpathmoveto{\pgfqpoint{1.530548in}{1.070680in}}%
\pgfpathcurveto{\pgfqpoint{1.541598in}{1.070680in}}{\pgfqpoint{1.552197in}{1.075070in}}{\pgfqpoint{1.560011in}{1.082884in}}%
\pgfpathcurveto{\pgfqpoint{1.567825in}{1.090698in}}{\pgfqpoint{1.572215in}{1.101297in}}{\pgfqpoint{1.572215in}{1.112347in}}%
\pgfpathcurveto{\pgfqpoint{1.572215in}{1.123397in}}{\pgfqpoint{1.567825in}{1.133996in}}{\pgfqpoint{1.560011in}{1.141810in}}%
\pgfpathcurveto{\pgfqpoint{1.552197in}{1.149623in}}{\pgfqpoint{1.541598in}{1.154013in}}{\pgfqpoint{1.530548in}{1.154013in}}%
\pgfpathcurveto{\pgfqpoint{1.519498in}{1.154013in}}{\pgfqpoint{1.508899in}{1.149623in}}{\pgfqpoint{1.501085in}{1.141810in}}%
\pgfpathcurveto{\pgfqpoint{1.493272in}{1.133996in}}{\pgfqpoint{1.488881in}{1.123397in}}{\pgfqpoint{1.488881in}{1.112347in}}%
\pgfpathcurveto{\pgfqpoint{1.488881in}{1.101297in}}{\pgfqpoint{1.493272in}{1.090698in}}{\pgfqpoint{1.501085in}{1.082884in}}%
\pgfpathcurveto{\pgfqpoint{1.508899in}{1.075070in}}{\pgfqpoint{1.519498in}{1.070680in}}{\pgfqpoint{1.530548in}{1.070680in}}%
\pgfpathclose%
\pgfusepath{stroke,fill}%
\end{pgfscope}%
\begin{pgfscope}%
\pgfpathrectangle{\pgfqpoint{0.375000in}{0.330000in}}{\pgfqpoint{2.325000in}{2.310000in}}%
\pgfusepath{clip}%
\pgfsetbuttcap%
\pgfsetroundjoin%
\definecolor{currentfill}{rgb}{0.000000,0.000000,0.000000}%
\pgfsetfillcolor{currentfill}%
\pgfsetlinewidth{1.003750pt}%
\definecolor{currentstroke}{rgb}{0.000000,0.000000,0.000000}%
\pgfsetstrokecolor{currentstroke}%
\pgfsetdash{}{0pt}%
\pgfpathmoveto{\pgfqpoint{1.530548in}{1.018649in}}%
\pgfpathcurveto{\pgfqpoint{1.541598in}{1.018649in}}{\pgfqpoint{1.552197in}{1.023040in}}{\pgfqpoint{1.560011in}{1.030853in}}%
\pgfpathcurveto{\pgfqpoint{1.567825in}{1.038667in}}{\pgfqpoint{1.572215in}{1.049266in}}{\pgfqpoint{1.572215in}{1.060316in}}%
\pgfpathcurveto{\pgfqpoint{1.572215in}{1.071366in}}{\pgfqpoint{1.567825in}{1.081965in}}{\pgfqpoint{1.560011in}{1.089779in}}%
\pgfpathcurveto{\pgfqpoint{1.552197in}{1.097592in}}{\pgfqpoint{1.541598in}{1.101983in}}{\pgfqpoint{1.530548in}{1.101983in}}%
\pgfpathcurveto{\pgfqpoint{1.519498in}{1.101983in}}{\pgfqpoint{1.508899in}{1.097592in}}{\pgfqpoint{1.501085in}{1.089779in}}%
\pgfpathcurveto{\pgfqpoint{1.493272in}{1.081965in}}{\pgfqpoint{1.488881in}{1.071366in}}{\pgfqpoint{1.488881in}{1.060316in}}%
\pgfpathcurveto{\pgfqpoint{1.488881in}{1.049266in}}{\pgfqpoint{1.493272in}{1.038667in}}{\pgfqpoint{1.501085in}{1.030853in}}%
\pgfpathcurveto{\pgfqpoint{1.508899in}{1.023040in}}{\pgfqpoint{1.519498in}{1.018649in}}{\pgfqpoint{1.530548in}{1.018649in}}%
\pgfpathclose%
\pgfusepath{stroke,fill}%
\end{pgfscope}%
\begin{pgfscope}%
\pgfpathrectangle{\pgfqpoint{0.375000in}{0.330000in}}{\pgfqpoint{2.325000in}{2.310000in}}%
\pgfusepath{clip}%
\pgfsetbuttcap%
\pgfsetroundjoin%
\definecolor{currentfill}{rgb}{0.000000,0.000000,0.000000}%
\pgfsetfillcolor{currentfill}%
\pgfsetlinewidth{1.003750pt}%
\definecolor{currentstroke}{rgb}{0.000000,0.000000,0.000000}%
\pgfsetstrokecolor{currentstroke}%
\pgfsetdash{}{0pt}%
\pgfpathmoveto{\pgfqpoint{1.530548in}{1.070680in}}%
\pgfpathcurveto{\pgfqpoint{1.541598in}{1.070680in}}{\pgfqpoint{1.552197in}{1.075070in}}{\pgfqpoint{1.560011in}{1.082884in}}%
\pgfpathcurveto{\pgfqpoint{1.567825in}{1.090698in}}{\pgfqpoint{1.572215in}{1.101297in}}{\pgfqpoint{1.572215in}{1.112347in}}%
\pgfpathcurveto{\pgfqpoint{1.572215in}{1.123397in}}{\pgfqpoint{1.567825in}{1.133996in}}{\pgfqpoint{1.560011in}{1.141810in}}%
\pgfpathcurveto{\pgfqpoint{1.552197in}{1.149623in}}{\pgfqpoint{1.541598in}{1.154013in}}{\pgfqpoint{1.530548in}{1.154013in}}%
\pgfpathcurveto{\pgfqpoint{1.519498in}{1.154013in}}{\pgfqpoint{1.508899in}{1.149623in}}{\pgfqpoint{1.501085in}{1.141810in}}%
\pgfpathcurveto{\pgfqpoint{1.493272in}{1.133996in}}{\pgfqpoint{1.488881in}{1.123397in}}{\pgfqpoint{1.488881in}{1.112347in}}%
\pgfpathcurveto{\pgfqpoint{1.488881in}{1.101297in}}{\pgfqpoint{1.493272in}{1.090698in}}{\pgfqpoint{1.501085in}{1.082884in}}%
\pgfpathcurveto{\pgfqpoint{1.508899in}{1.075070in}}{\pgfqpoint{1.519498in}{1.070680in}}{\pgfqpoint{1.530548in}{1.070680in}}%
\pgfpathclose%
\pgfusepath{stroke,fill}%
\end{pgfscope}%
\begin{pgfscope}%
\pgfpathrectangle{\pgfqpoint{0.375000in}{0.330000in}}{\pgfqpoint{2.325000in}{2.310000in}}%
\pgfusepath{clip}%
\pgfsetbuttcap%
\pgfsetroundjoin%
\definecolor{currentfill}{rgb}{0.000000,0.000000,0.000000}%
\pgfsetfillcolor{currentfill}%
\pgfsetlinewidth{1.003750pt}%
\definecolor{currentstroke}{rgb}{0.000000,0.000000,0.000000}%
\pgfsetstrokecolor{currentstroke}%
\pgfsetdash{}{0pt}%
\pgfpathmoveto{\pgfqpoint{1.530548in}{1.070680in}}%
\pgfpathcurveto{\pgfqpoint{1.541598in}{1.070680in}}{\pgfqpoint{1.552197in}{1.075070in}}{\pgfqpoint{1.560011in}{1.082884in}}%
\pgfpathcurveto{\pgfqpoint{1.567825in}{1.090698in}}{\pgfqpoint{1.572215in}{1.101297in}}{\pgfqpoint{1.572215in}{1.112347in}}%
\pgfpathcurveto{\pgfqpoint{1.572215in}{1.123397in}}{\pgfqpoint{1.567825in}{1.133996in}}{\pgfqpoint{1.560011in}{1.141810in}}%
\pgfpathcurveto{\pgfqpoint{1.552197in}{1.149623in}}{\pgfqpoint{1.541598in}{1.154013in}}{\pgfqpoint{1.530548in}{1.154013in}}%
\pgfpathcurveto{\pgfqpoint{1.519498in}{1.154013in}}{\pgfqpoint{1.508899in}{1.149623in}}{\pgfqpoint{1.501085in}{1.141810in}}%
\pgfpathcurveto{\pgfqpoint{1.493272in}{1.133996in}}{\pgfqpoint{1.488881in}{1.123397in}}{\pgfqpoint{1.488881in}{1.112347in}}%
\pgfpathcurveto{\pgfqpoint{1.488881in}{1.101297in}}{\pgfqpoint{1.493272in}{1.090698in}}{\pgfqpoint{1.501085in}{1.082884in}}%
\pgfpathcurveto{\pgfqpoint{1.508899in}{1.075070in}}{\pgfqpoint{1.519498in}{1.070680in}}{\pgfqpoint{1.530548in}{1.070680in}}%
\pgfpathclose%
\pgfusepath{stroke,fill}%
\end{pgfscope}%
\begin{pgfscope}%
\pgfpathrectangle{\pgfqpoint{0.375000in}{0.330000in}}{\pgfqpoint{2.325000in}{2.310000in}}%
\pgfusepath{clip}%
\pgfsetbuttcap%
\pgfsetroundjoin%
\definecolor{currentfill}{rgb}{0.000000,0.000000,0.000000}%
\pgfsetfillcolor{currentfill}%
\pgfsetlinewidth{1.003750pt}%
\definecolor{currentstroke}{rgb}{0.000000,0.000000,0.000000}%
\pgfsetstrokecolor{currentstroke}%
\pgfsetdash{}{0pt}%
\pgfpathmoveto{\pgfqpoint{1.530548in}{1.070680in}}%
\pgfpathcurveto{\pgfqpoint{1.541598in}{1.070680in}}{\pgfqpoint{1.552197in}{1.075070in}}{\pgfqpoint{1.560011in}{1.082884in}}%
\pgfpathcurveto{\pgfqpoint{1.567825in}{1.090698in}}{\pgfqpoint{1.572215in}{1.101297in}}{\pgfqpoint{1.572215in}{1.112347in}}%
\pgfpathcurveto{\pgfqpoint{1.572215in}{1.123397in}}{\pgfqpoint{1.567825in}{1.133996in}}{\pgfqpoint{1.560011in}{1.141810in}}%
\pgfpathcurveto{\pgfqpoint{1.552197in}{1.149623in}}{\pgfqpoint{1.541598in}{1.154013in}}{\pgfqpoint{1.530548in}{1.154013in}}%
\pgfpathcurveto{\pgfqpoint{1.519498in}{1.154013in}}{\pgfqpoint{1.508899in}{1.149623in}}{\pgfqpoint{1.501085in}{1.141810in}}%
\pgfpathcurveto{\pgfqpoint{1.493272in}{1.133996in}}{\pgfqpoint{1.488881in}{1.123397in}}{\pgfqpoint{1.488881in}{1.112347in}}%
\pgfpathcurveto{\pgfqpoint{1.488881in}{1.101297in}}{\pgfqpoint{1.493272in}{1.090698in}}{\pgfqpoint{1.501085in}{1.082884in}}%
\pgfpathcurveto{\pgfqpoint{1.508899in}{1.075070in}}{\pgfqpoint{1.519498in}{1.070680in}}{\pgfqpoint{1.530548in}{1.070680in}}%
\pgfpathclose%
\pgfusepath{stroke,fill}%
\end{pgfscope}%
\begin{pgfscope}%
\pgfpathrectangle{\pgfqpoint{0.375000in}{0.330000in}}{\pgfqpoint{2.325000in}{2.310000in}}%
\pgfusepath{clip}%
\pgfsetbuttcap%
\pgfsetroundjoin%
\definecolor{currentfill}{rgb}{0.000000,0.000000,0.000000}%
\pgfsetfillcolor{currentfill}%
\pgfsetlinewidth{1.003750pt}%
\definecolor{currentstroke}{rgb}{0.000000,0.000000,0.000000}%
\pgfsetstrokecolor{currentstroke}%
\pgfsetdash{}{0pt}%
\pgfpathmoveto{\pgfqpoint{1.530548in}{1.070680in}}%
\pgfpathcurveto{\pgfqpoint{1.541598in}{1.070680in}}{\pgfqpoint{1.552197in}{1.075070in}}{\pgfqpoint{1.560011in}{1.082884in}}%
\pgfpathcurveto{\pgfqpoint{1.567825in}{1.090698in}}{\pgfqpoint{1.572215in}{1.101297in}}{\pgfqpoint{1.572215in}{1.112347in}}%
\pgfpathcurveto{\pgfqpoint{1.572215in}{1.123397in}}{\pgfqpoint{1.567825in}{1.133996in}}{\pgfqpoint{1.560011in}{1.141810in}}%
\pgfpathcurveto{\pgfqpoint{1.552197in}{1.149623in}}{\pgfqpoint{1.541598in}{1.154013in}}{\pgfqpoint{1.530548in}{1.154013in}}%
\pgfpathcurveto{\pgfqpoint{1.519498in}{1.154013in}}{\pgfqpoint{1.508899in}{1.149623in}}{\pgfqpoint{1.501085in}{1.141810in}}%
\pgfpathcurveto{\pgfqpoint{1.493272in}{1.133996in}}{\pgfqpoint{1.488881in}{1.123397in}}{\pgfqpoint{1.488881in}{1.112347in}}%
\pgfpathcurveto{\pgfqpoint{1.488881in}{1.101297in}}{\pgfqpoint{1.493272in}{1.090698in}}{\pgfqpoint{1.501085in}{1.082884in}}%
\pgfpathcurveto{\pgfqpoint{1.508899in}{1.075070in}}{\pgfqpoint{1.519498in}{1.070680in}}{\pgfqpoint{1.530548in}{1.070680in}}%
\pgfpathclose%
\pgfusepath{stroke,fill}%
\end{pgfscope}%
\begin{pgfscope}%
\pgfpathrectangle{\pgfqpoint{0.375000in}{0.330000in}}{\pgfqpoint{2.325000in}{2.310000in}}%
\pgfusepath{clip}%
\pgfsetbuttcap%
\pgfsetroundjoin%
\definecolor{currentfill}{rgb}{0.000000,0.000000,0.000000}%
\pgfsetfillcolor{currentfill}%
\pgfsetlinewidth{1.003750pt}%
\definecolor{currentstroke}{rgb}{0.000000,0.000000,0.000000}%
\pgfsetstrokecolor{currentstroke}%
\pgfsetdash{}{0pt}%
\pgfpathmoveto{\pgfqpoint{1.530548in}{1.070680in}}%
\pgfpathcurveto{\pgfqpoint{1.541598in}{1.070680in}}{\pgfqpoint{1.552197in}{1.075070in}}{\pgfqpoint{1.560011in}{1.082884in}}%
\pgfpathcurveto{\pgfqpoint{1.567825in}{1.090698in}}{\pgfqpoint{1.572215in}{1.101297in}}{\pgfqpoint{1.572215in}{1.112347in}}%
\pgfpathcurveto{\pgfqpoint{1.572215in}{1.123397in}}{\pgfqpoint{1.567825in}{1.133996in}}{\pgfqpoint{1.560011in}{1.141810in}}%
\pgfpathcurveto{\pgfqpoint{1.552197in}{1.149623in}}{\pgfqpoint{1.541598in}{1.154013in}}{\pgfqpoint{1.530548in}{1.154013in}}%
\pgfpathcurveto{\pgfqpoint{1.519498in}{1.154013in}}{\pgfqpoint{1.508899in}{1.149623in}}{\pgfqpoint{1.501085in}{1.141810in}}%
\pgfpathcurveto{\pgfqpoint{1.493272in}{1.133996in}}{\pgfqpoint{1.488881in}{1.123397in}}{\pgfqpoint{1.488881in}{1.112347in}}%
\pgfpathcurveto{\pgfqpoint{1.488881in}{1.101297in}}{\pgfqpoint{1.493272in}{1.090698in}}{\pgfqpoint{1.501085in}{1.082884in}}%
\pgfpathcurveto{\pgfqpoint{1.508899in}{1.075070in}}{\pgfqpoint{1.519498in}{1.070680in}}{\pgfqpoint{1.530548in}{1.070680in}}%
\pgfpathclose%
\pgfusepath{stroke,fill}%
\end{pgfscope}%
\begin{pgfscope}%
\pgfpathrectangle{\pgfqpoint{0.375000in}{0.330000in}}{\pgfqpoint{2.325000in}{2.310000in}}%
\pgfusepath{clip}%
\pgfsetbuttcap%
\pgfsetroundjoin%
\definecolor{currentfill}{rgb}{0.000000,0.000000,0.000000}%
\pgfsetfillcolor{currentfill}%
\pgfsetlinewidth{1.003750pt}%
\definecolor{currentstroke}{rgb}{0.000000,0.000000,0.000000}%
\pgfsetstrokecolor{currentstroke}%
\pgfsetdash{}{0pt}%
\pgfpathmoveto{\pgfqpoint{1.530548in}{1.070680in}}%
\pgfpathcurveto{\pgfqpoint{1.541598in}{1.070680in}}{\pgfqpoint{1.552197in}{1.075070in}}{\pgfqpoint{1.560011in}{1.082884in}}%
\pgfpathcurveto{\pgfqpoint{1.567825in}{1.090698in}}{\pgfqpoint{1.572215in}{1.101297in}}{\pgfqpoint{1.572215in}{1.112347in}}%
\pgfpathcurveto{\pgfqpoint{1.572215in}{1.123397in}}{\pgfqpoint{1.567825in}{1.133996in}}{\pgfqpoint{1.560011in}{1.141810in}}%
\pgfpathcurveto{\pgfqpoint{1.552197in}{1.149623in}}{\pgfqpoint{1.541598in}{1.154013in}}{\pgfqpoint{1.530548in}{1.154013in}}%
\pgfpathcurveto{\pgfqpoint{1.519498in}{1.154013in}}{\pgfqpoint{1.508899in}{1.149623in}}{\pgfqpoint{1.501085in}{1.141810in}}%
\pgfpathcurveto{\pgfqpoint{1.493272in}{1.133996in}}{\pgfqpoint{1.488881in}{1.123397in}}{\pgfqpoint{1.488881in}{1.112347in}}%
\pgfpathcurveto{\pgfqpoint{1.488881in}{1.101297in}}{\pgfqpoint{1.493272in}{1.090698in}}{\pgfqpoint{1.501085in}{1.082884in}}%
\pgfpathcurveto{\pgfqpoint{1.508899in}{1.075070in}}{\pgfqpoint{1.519498in}{1.070680in}}{\pgfqpoint{1.530548in}{1.070680in}}%
\pgfpathclose%
\pgfusepath{stroke,fill}%
\end{pgfscope}%
\begin{pgfscope}%
\pgfpathrectangle{\pgfqpoint{0.375000in}{0.330000in}}{\pgfqpoint{2.325000in}{2.310000in}}%
\pgfusepath{clip}%
\pgfsetbuttcap%
\pgfsetroundjoin%
\definecolor{currentfill}{rgb}{0.000000,0.000000,0.000000}%
\pgfsetfillcolor{currentfill}%
\pgfsetlinewidth{1.003750pt}%
\definecolor{currentstroke}{rgb}{0.000000,0.000000,0.000000}%
\pgfsetstrokecolor{currentstroke}%
\pgfsetdash{}{0pt}%
\pgfpathmoveto{\pgfqpoint{1.530548in}{1.018649in}}%
\pgfpathcurveto{\pgfqpoint{1.541598in}{1.018649in}}{\pgfqpoint{1.552197in}{1.023040in}}{\pgfqpoint{1.560011in}{1.030853in}}%
\pgfpathcurveto{\pgfqpoint{1.567825in}{1.038667in}}{\pgfqpoint{1.572215in}{1.049266in}}{\pgfqpoint{1.572215in}{1.060316in}}%
\pgfpathcurveto{\pgfqpoint{1.572215in}{1.071366in}}{\pgfqpoint{1.567825in}{1.081965in}}{\pgfqpoint{1.560011in}{1.089779in}}%
\pgfpathcurveto{\pgfqpoint{1.552197in}{1.097592in}}{\pgfqpoint{1.541598in}{1.101983in}}{\pgfqpoint{1.530548in}{1.101983in}}%
\pgfpathcurveto{\pgfqpoint{1.519498in}{1.101983in}}{\pgfqpoint{1.508899in}{1.097592in}}{\pgfqpoint{1.501085in}{1.089779in}}%
\pgfpathcurveto{\pgfqpoint{1.493272in}{1.081965in}}{\pgfqpoint{1.488881in}{1.071366in}}{\pgfqpoint{1.488881in}{1.060316in}}%
\pgfpathcurveto{\pgfqpoint{1.488881in}{1.049266in}}{\pgfqpoint{1.493272in}{1.038667in}}{\pgfqpoint{1.501085in}{1.030853in}}%
\pgfpathcurveto{\pgfqpoint{1.508899in}{1.023040in}}{\pgfqpoint{1.519498in}{1.018649in}}{\pgfqpoint{1.530548in}{1.018649in}}%
\pgfpathclose%
\pgfusepath{stroke,fill}%
\end{pgfscope}%
\begin{pgfscope}%
\pgfpathrectangle{\pgfqpoint{0.375000in}{0.330000in}}{\pgfqpoint{2.325000in}{2.310000in}}%
\pgfusepath{clip}%
\pgfsetbuttcap%
\pgfsetroundjoin%
\definecolor{currentfill}{rgb}{0.000000,0.000000,0.000000}%
\pgfsetfillcolor{currentfill}%
\pgfsetlinewidth{1.003750pt}%
\definecolor{currentstroke}{rgb}{0.000000,0.000000,0.000000}%
\pgfsetstrokecolor{currentstroke}%
\pgfsetdash{}{0pt}%
\pgfpathmoveto{\pgfqpoint{1.530548in}{1.018649in}}%
\pgfpathcurveto{\pgfqpoint{1.541598in}{1.018649in}}{\pgfqpoint{1.552197in}{1.023040in}}{\pgfqpoint{1.560011in}{1.030853in}}%
\pgfpathcurveto{\pgfqpoint{1.567825in}{1.038667in}}{\pgfqpoint{1.572215in}{1.049266in}}{\pgfqpoint{1.572215in}{1.060316in}}%
\pgfpathcurveto{\pgfqpoint{1.572215in}{1.071366in}}{\pgfqpoint{1.567825in}{1.081965in}}{\pgfqpoint{1.560011in}{1.089779in}}%
\pgfpathcurveto{\pgfqpoint{1.552197in}{1.097592in}}{\pgfqpoint{1.541598in}{1.101983in}}{\pgfqpoint{1.530548in}{1.101983in}}%
\pgfpathcurveto{\pgfqpoint{1.519498in}{1.101983in}}{\pgfqpoint{1.508899in}{1.097592in}}{\pgfqpoint{1.501085in}{1.089779in}}%
\pgfpathcurveto{\pgfqpoint{1.493272in}{1.081965in}}{\pgfqpoint{1.488881in}{1.071366in}}{\pgfqpoint{1.488881in}{1.060316in}}%
\pgfpathcurveto{\pgfqpoint{1.488881in}{1.049266in}}{\pgfqpoint{1.493272in}{1.038667in}}{\pgfqpoint{1.501085in}{1.030853in}}%
\pgfpathcurveto{\pgfqpoint{1.508899in}{1.023040in}}{\pgfqpoint{1.519498in}{1.018649in}}{\pgfqpoint{1.530548in}{1.018649in}}%
\pgfpathclose%
\pgfusepath{stroke,fill}%
\end{pgfscope}%
\begin{pgfscope}%
\pgfpathrectangle{\pgfqpoint{0.375000in}{0.330000in}}{\pgfqpoint{2.325000in}{2.310000in}}%
\pgfusepath{clip}%
\pgfsetbuttcap%
\pgfsetroundjoin%
\definecolor{currentfill}{rgb}{0.000000,0.000000,0.000000}%
\pgfsetfillcolor{currentfill}%
\pgfsetlinewidth{1.003750pt}%
\definecolor{currentstroke}{rgb}{0.000000,0.000000,0.000000}%
\pgfsetstrokecolor{currentstroke}%
\pgfsetdash{}{0pt}%
\pgfpathmoveto{\pgfqpoint{1.530548in}{1.018649in}}%
\pgfpathcurveto{\pgfqpoint{1.541598in}{1.018649in}}{\pgfqpoint{1.552197in}{1.023040in}}{\pgfqpoint{1.560011in}{1.030853in}}%
\pgfpathcurveto{\pgfqpoint{1.567825in}{1.038667in}}{\pgfqpoint{1.572215in}{1.049266in}}{\pgfqpoint{1.572215in}{1.060316in}}%
\pgfpathcurveto{\pgfqpoint{1.572215in}{1.071366in}}{\pgfqpoint{1.567825in}{1.081965in}}{\pgfqpoint{1.560011in}{1.089779in}}%
\pgfpathcurveto{\pgfqpoint{1.552197in}{1.097592in}}{\pgfqpoint{1.541598in}{1.101983in}}{\pgfqpoint{1.530548in}{1.101983in}}%
\pgfpathcurveto{\pgfqpoint{1.519498in}{1.101983in}}{\pgfqpoint{1.508899in}{1.097592in}}{\pgfqpoint{1.501085in}{1.089779in}}%
\pgfpathcurveto{\pgfqpoint{1.493272in}{1.081965in}}{\pgfqpoint{1.488881in}{1.071366in}}{\pgfqpoint{1.488881in}{1.060316in}}%
\pgfpathcurveto{\pgfqpoint{1.488881in}{1.049266in}}{\pgfqpoint{1.493272in}{1.038667in}}{\pgfqpoint{1.501085in}{1.030853in}}%
\pgfpathcurveto{\pgfqpoint{1.508899in}{1.023040in}}{\pgfqpoint{1.519498in}{1.018649in}}{\pgfqpoint{1.530548in}{1.018649in}}%
\pgfpathclose%
\pgfusepath{stroke,fill}%
\end{pgfscope}%
\begin{pgfscope}%
\pgfpathrectangle{\pgfqpoint{0.375000in}{0.330000in}}{\pgfqpoint{2.325000in}{2.310000in}}%
\pgfusepath{clip}%
\pgfsetbuttcap%
\pgfsetroundjoin%
\definecolor{currentfill}{rgb}{0.000000,0.000000,0.000000}%
\pgfsetfillcolor{currentfill}%
\pgfsetlinewidth{1.003750pt}%
\definecolor{currentstroke}{rgb}{0.000000,0.000000,0.000000}%
\pgfsetstrokecolor{currentstroke}%
\pgfsetdash{}{0pt}%
\pgfpathmoveto{\pgfqpoint{1.530548in}{1.174742in}}%
\pgfpathcurveto{\pgfqpoint{1.541598in}{1.174742in}}{\pgfqpoint{1.552197in}{1.179132in}}{\pgfqpoint{1.560011in}{1.186946in}}%
\pgfpathcurveto{\pgfqpoint{1.567825in}{1.194759in}}{\pgfqpoint{1.572215in}{1.205358in}}{\pgfqpoint{1.572215in}{1.216408in}}%
\pgfpathcurveto{\pgfqpoint{1.572215in}{1.227459in}}{\pgfqpoint{1.567825in}{1.238058in}}{\pgfqpoint{1.560011in}{1.245871in}}%
\pgfpathcurveto{\pgfqpoint{1.552197in}{1.253685in}}{\pgfqpoint{1.541598in}{1.258075in}}{\pgfqpoint{1.530548in}{1.258075in}}%
\pgfpathcurveto{\pgfqpoint{1.519498in}{1.258075in}}{\pgfqpoint{1.508899in}{1.253685in}}{\pgfqpoint{1.501085in}{1.245871in}}%
\pgfpathcurveto{\pgfqpoint{1.493272in}{1.238058in}}{\pgfqpoint{1.488881in}{1.227459in}}{\pgfqpoint{1.488881in}{1.216408in}}%
\pgfpathcurveto{\pgfqpoint{1.488881in}{1.205358in}}{\pgfqpoint{1.493272in}{1.194759in}}{\pgfqpoint{1.501085in}{1.186946in}}%
\pgfpathcurveto{\pgfqpoint{1.508899in}{1.179132in}}{\pgfqpoint{1.519498in}{1.174742in}}{\pgfqpoint{1.530548in}{1.174742in}}%
\pgfpathclose%
\pgfusepath{stroke,fill}%
\end{pgfscope}%
\begin{pgfscope}%
\pgfpathrectangle{\pgfqpoint{0.375000in}{0.330000in}}{\pgfqpoint{2.325000in}{2.310000in}}%
\pgfusepath{clip}%
\pgfsetbuttcap%
\pgfsetroundjoin%
\definecolor{currentfill}{rgb}{0.000000,0.000000,0.000000}%
\pgfsetfillcolor{currentfill}%
\pgfsetlinewidth{1.003750pt}%
\definecolor{currentstroke}{rgb}{0.000000,0.000000,0.000000}%
\pgfsetstrokecolor{currentstroke}%
\pgfsetdash{}{0pt}%
\pgfpathmoveto{\pgfqpoint{1.530548in}{1.070680in}}%
\pgfpathcurveto{\pgfqpoint{1.541598in}{1.070680in}}{\pgfqpoint{1.552197in}{1.075070in}}{\pgfqpoint{1.560011in}{1.082884in}}%
\pgfpathcurveto{\pgfqpoint{1.567825in}{1.090698in}}{\pgfqpoint{1.572215in}{1.101297in}}{\pgfqpoint{1.572215in}{1.112347in}}%
\pgfpathcurveto{\pgfqpoint{1.572215in}{1.123397in}}{\pgfqpoint{1.567825in}{1.133996in}}{\pgfqpoint{1.560011in}{1.141810in}}%
\pgfpathcurveto{\pgfqpoint{1.552197in}{1.149623in}}{\pgfqpoint{1.541598in}{1.154013in}}{\pgfqpoint{1.530548in}{1.154013in}}%
\pgfpathcurveto{\pgfqpoint{1.519498in}{1.154013in}}{\pgfqpoint{1.508899in}{1.149623in}}{\pgfqpoint{1.501085in}{1.141810in}}%
\pgfpathcurveto{\pgfqpoint{1.493272in}{1.133996in}}{\pgfqpoint{1.488881in}{1.123397in}}{\pgfqpoint{1.488881in}{1.112347in}}%
\pgfpathcurveto{\pgfqpoint{1.488881in}{1.101297in}}{\pgfqpoint{1.493272in}{1.090698in}}{\pgfqpoint{1.501085in}{1.082884in}}%
\pgfpathcurveto{\pgfqpoint{1.508899in}{1.075070in}}{\pgfqpoint{1.519498in}{1.070680in}}{\pgfqpoint{1.530548in}{1.070680in}}%
\pgfpathclose%
\pgfusepath{stroke,fill}%
\end{pgfscope}%
\begin{pgfscope}%
\pgfpathrectangle{\pgfqpoint{0.375000in}{0.330000in}}{\pgfqpoint{2.325000in}{2.310000in}}%
\pgfusepath{clip}%
\pgfsetbuttcap%
\pgfsetroundjoin%
\definecolor{currentfill}{rgb}{0.000000,0.000000,0.000000}%
\pgfsetfillcolor{currentfill}%
\pgfsetlinewidth{1.003750pt}%
\definecolor{currentstroke}{rgb}{0.000000,0.000000,0.000000}%
\pgfsetstrokecolor{currentstroke}%
\pgfsetdash{}{0pt}%
\pgfpathmoveto{\pgfqpoint{1.530548in}{1.070680in}}%
\pgfpathcurveto{\pgfqpoint{1.541598in}{1.070680in}}{\pgfqpoint{1.552197in}{1.075070in}}{\pgfqpoint{1.560011in}{1.082884in}}%
\pgfpathcurveto{\pgfqpoint{1.567825in}{1.090698in}}{\pgfqpoint{1.572215in}{1.101297in}}{\pgfqpoint{1.572215in}{1.112347in}}%
\pgfpathcurveto{\pgfqpoint{1.572215in}{1.123397in}}{\pgfqpoint{1.567825in}{1.133996in}}{\pgfqpoint{1.560011in}{1.141810in}}%
\pgfpathcurveto{\pgfqpoint{1.552197in}{1.149623in}}{\pgfqpoint{1.541598in}{1.154013in}}{\pgfqpoint{1.530548in}{1.154013in}}%
\pgfpathcurveto{\pgfqpoint{1.519498in}{1.154013in}}{\pgfqpoint{1.508899in}{1.149623in}}{\pgfqpoint{1.501085in}{1.141810in}}%
\pgfpathcurveto{\pgfqpoint{1.493272in}{1.133996in}}{\pgfqpoint{1.488881in}{1.123397in}}{\pgfqpoint{1.488881in}{1.112347in}}%
\pgfpathcurveto{\pgfqpoint{1.488881in}{1.101297in}}{\pgfqpoint{1.493272in}{1.090698in}}{\pgfqpoint{1.501085in}{1.082884in}}%
\pgfpathcurveto{\pgfqpoint{1.508899in}{1.075070in}}{\pgfqpoint{1.519498in}{1.070680in}}{\pgfqpoint{1.530548in}{1.070680in}}%
\pgfpathclose%
\pgfusepath{stroke,fill}%
\end{pgfscope}%
\begin{pgfscope}%
\pgfpathrectangle{\pgfqpoint{0.375000in}{0.330000in}}{\pgfqpoint{2.325000in}{2.310000in}}%
\pgfusepath{clip}%
\pgfsetbuttcap%
\pgfsetroundjoin%
\definecolor{currentfill}{rgb}{0.000000,0.000000,0.000000}%
\pgfsetfillcolor{currentfill}%
\pgfsetlinewidth{1.003750pt}%
\definecolor{currentstroke}{rgb}{0.000000,0.000000,0.000000}%
\pgfsetstrokecolor{currentstroke}%
\pgfsetdash{}{0pt}%
\pgfpathmoveto{\pgfqpoint{1.530548in}{1.070680in}}%
\pgfpathcurveto{\pgfqpoint{1.541598in}{1.070680in}}{\pgfqpoint{1.552197in}{1.075070in}}{\pgfqpoint{1.560011in}{1.082884in}}%
\pgfpathcurveto{\pgfqpoint{1.567825in}{1.090698in}}{\pgfqpoint{1.572215in}{1.101297in}}{\pgfqpoint{1.572215in}{1.112347in}}%
\pgfpathcurveto{\pgfqpoint{1.572215in}{1.123397in}}{\pgfqpoint{1.567825in}{1.133996in}}{\pgfqpoint{1.560011in}{1.141810in}}%
\pgfpathcurveto{\pgfqpoint{1.552197in}{1.149623in}}{\pgfqpoint{1.541598in}{1.154013in}}{\pgfqpoint{1.530548in}{1.154013in}}%
\pgfpathcurveto{\pgfqpoint{1.519498in}{1.154013in}}{\pgfqpoint{1.508899in}{1.149623in}}{\pgfqpoint{1.501085in}{1.141810in}}%
\pgfpathcurveto{\pgfqpoint{1.493272in}{1.133996in}}{\pgfqpoint{1.488881in}{1.123397in}}{\pgfqpoint{1.488881in}{1.112347in}}%
\pgfpathcurveto{\pgfqpoint{1.488881in}{1.101297in}}{\pgfqpoint{1.493272in}{1.090698in}}{\pgfqpoint{1.501085in}{1.082884in}}%
\pgfpathcurveto{\pgfqpoint{1.508899in}{1.075070in}}{\pgfqpoint{1.519498in}{1.070680in}}{\pgfqpoint{1.530548in}{1.070680in}}%
\pgfpathclose%
\pgfusepath{stroke,fill}%
\end{pgfscope}%
\begin{pgfscope}%
\pgfpathrectangle{\pgfqpoint{0.375000in}{0.330000in}}{\pgfqpoint{2.325000in}{2.310000in}}%
\pgfusepath{clip}%
\pgfsetbuttcap%
\pgfsetroundjoin%
\definecolor{currentfill}{rgb}{0.000000,0.000000,0.000000}%
\pgfsetfillcolor{currentfill}%
\pgfsetlinewidth{1.003750pt}%
\definecolor{currentstroke}{rgb}{0.000000,0.000000,0.000000}%
\pgfsetstrokecolor{currentstroke}%
\pgfsetdash{}{0pt}%
\pgfpathmoveto{\pgfqpoint{1.530548in}{1.070680in}}%
\pgfpathcurveto{\pgfqpoint{1.541598in}{1.070680in}}{\pgfqpoint{1.552197in}{1.075070in}}{\pgfqpoint{1.560011in}{1.082884in}}%
\pgfpathcurveto{\pgfqpoint{1.567825in}{1.090698in}}{\pgfqpoint{1.572215in}{1.101297in}}{\pgfqpoint{1.572215in}{1.112347in}}%
\pgfpathcurveto{\pgfqpoint{1.572215in}{1.123397in}}{\pgfqpoint{1.567825in}{1.133996in}}{\pgfqpoint{1.560011in}{1.141810in}}%
\pgfpathcurveto{\pgfqpoint{1.552197in}{1.149623in}}{\pgfqpoint{1.541598in}{1.154013in}}{\pgfqpoint{1.530548in}{1.154013in}}%
\pgfpathcurveto{\pgfqpoint{1.519498in}{1.154013in}}{\pgfqpoint{1.508899in}{1.149623in}}{\pgfqpoint{1.501085in}{1.141810in}}%
\pgfpathcurveto{\pgfqpoint{1.493272in}{1.133996in}}{\pgfqpoint{1.488881in}{1.123397in}}{\pgfqpoint{1.488881in}{1.112347in}}%
\pgfpathcurveto{\pgfqpoint{1.488881in}{1.101297in}}{\pgfqpoint{1.493272in}{1.090698in}}{\pgfqpoint{1.501085in}{1.082884in}}%
\pgfpathcurveto{\pgfqpoint{1.508899in}{1.075070in}}{\pgfqpoint{1.519498in}{1.070680in}}{\pgfqpoint{1.530548in}{1.070680in}}%
\pgfpathclose%
\pgfusepath{stroke,fill}%
\end{pgfscope}%
\begin{pgfscope}%
\pgfpathrectangle{\pgfqpoint{0.375000in}{0.330000in}}{\pgfqpoint{2.325000in}{2.310000in}}%
\pgfusepath{clip}%
\pgfsetbuttcap%
\pgfsetroundjoin%
\definecolor{currentfill}{rgb}{0.000000,0.000000,0.000000}%
\pgfsetfillcolor{currentfill}%
\pgfsetlinewidth{1.003750pt}%
\definecolor{currentstroke}{rgb}{0.000000,0.000000,0.000000}%
\pgfsetstrokecolor{currentstroke}%
\pgfsetdash{}{0pt}%
\pgfpathmoveto{\pgfqpoint{1.530548in}{1.070680in}}%
\pgfpathcurveto{\pgfqpoint{1.541598in}{1.070680in}}{\pgfqpoint{1.552197in}{1.075070in}}{\pgfqpoint{1.560011in}{1.082884in}}%
\pgfpathcurveto{\pgfqpoint{1.567825in}{1.090698in}}{\pgfqpoint{1.572215in}{1.101297in}}{\pgfqpoint{1.572215in}{1.112347in}}%
\pgfpathcurveto{\pgfqpoint{1.572215in}{1.123397in}}{\pgfqpoint{1.567825in}{1.133996in}}{\pgfqpoint{1.560011in}{1.141810in}}%
\pgfpathcurveto{\pgfqpoint{1.552197in}{1.149623in}}{\pgfqpoint{1.541598in}{1.154013in}}{\pgfqpoint{1.530548in}{1.154013in}}%
\pgfpathcurveto{\pgfqpoint{1.519498in}{1.154013in}}{\pgfqpoint{1.508899in}{1.149623in}}{\pgfqpoint{1.501085in}{1.141810in}}%
\pgfpathcurveto{\pgfqpoint{1.493272in}{1.133996in}}{\pgfqpoint{1.488881in}{1.123397in}}{\pgfqpoint{1.488881in}{1.112347in}}%
\pgfpathcurveto{\pgfqpoint{1.488881in}{1.101297in}}{\pgfqpoint{1.493272in}{1.090698in}}{\pgfqpoint{1.501085in}{1.082884in}}%
\pgfpathcurveto{\pgfqpoint{1.508899in}{1.075070in}}{\pgfqpoint{1.519498in}{1.070680in}}{\pgfqpoint{1.530548in}{1.070680in}}%
\pgfpathclose%
\pgfusepath{stroke,fill}%
\end{pgfscope}%
\begin{pgfscope}%
\pgfpathrectangle{\pgfqpoint{0.375000in}{0.330000in}}{\pgfqpoint{2.325000in}{2.310000in}}%
\pgfusepath{clip}%
\pgfsetbuttcap%
\pgfsetroundjoin%
\definecolor{currentfill}{rgb}{0.000000,0.000000,0.000000}%
\pgfsetfillcolor{currentfill}%
\pgfsetlinewidth{1.003750pt}%
\definecolor{currentstroke}{rgb}{0.000000,0.000000,0.000000}%
\pgfsetstrokecolor{currentstroke}%
\pgfsetdash{}{0pt}%
\pgfpathmoveto{\pgfqpoint{1.530548in}{1.070680in}}%
\pgfpathcurveto{\pgfqpoint{1.541598in}{1.070680in}}{\pgfqpoint{1.552197in}{1.075070in}}{\pgfqpoint{1.560011in}{1.082884in}}%
\pgfpathcurveto{\pgfqpoint{1.567825in}{1.090698in}}{\pgfqpoint{1.572215in}{1.101297in}}{\pgfqpoint{1.572215in}{1.112347in}}%
\pgfpathcurveto{\pgfqpoint{1.572215in}{1.123397in}}{\pgfqpoint{1.567825in}{1.133996in}}{\pgfqpoint{1.560011in}{1.141810in}}%
\pgfpathcurveto{\pgfqpoint{1.552197in}{1.149623in}}{\pgfqpoint{1.541598in}{1.154013in}}{\pgfqpoint{1.530548in}{1.154013in}}%
\pgfpathcurveto{\pgfqpoint{1.519498in}{1.154013in}}{\pgfqpoint{1.508899in}{1.149623in}}{\pgfqpoint{1.501085in}{1.141810in}}%
\pgfpathcurveto{\pgfqpoint{1.493272in}{1.133996in}}{\pgfqpoint{1.488881in}{1.123397in}}{\pgfqpoint{1.488881in}{1.112347in}}%
\pgfpathcurveto{\pgfqpoint{1.488881in}{1.101297in}}{\pgfqpoint{1.493272in}{1.090698in}}{\pgfqpoint{1.501085in}{1.082884in}}%
\pgfpathcurveto{\pgfqpoint{1.508899in}{1.075070in}}{\pgfqpoint{1.519498in}{1.070680in}}{\pgfqpoint{1.530548in}{1.070680in}}%
\pgfpathclose%
\pgfusepath{stroke,fill}%
\end{pgfscope}%
\begin{pgfscope}%
\pgfpathrectangle{\pgfqpoint{0.375000in}{0.330000in}}{\pgfqpoint{2.325000in}{2.310000in}}%
\pgfusepath{clip}%
\pgfsetbuttcap%
\pgfsetroundjoin%
\definecolor{currentfill}{rgb}{0.000000,0.000000,0.000000}%
\pgfsetfillcolor{currentfill}%
\pgfsetlinewidth{1.003750pt}%
\definecolor{currentstroke}{rgb}{0.000000,0.000000,0.000000}%
\pgfsetstrokecolor{currentstroke}%
\pgfsetdash{}{0pt}%
\pgfpathmoveto{\pgfqpoint{1.530548in}{1.122711in}}%
\pgfpathcurveto{\pgfqpoint{1.541598in}{1.122711in}}{\pgfqpoint{1.552197in}{1.127101in}}{\pgfqpoint{1.560011in}{1.134915in}}%
\pgfpathcurveto{\pgfqpoint{1.567825in}{1.142728in}}{\pgfqpoint{1.572215in}{1.153327in}}{\pgfqpoint{1.572215in}{1.164378in}}%
\pgfpathcurveto{\pgfqpoint{1.572215in}{1.175428in}}{\pgfqpoint{1.567825in}{1.186027in}}{\pgfqpoint{1.560011in}{1.193840in}}%
\pgfpathcurveto{\pgfqpoint{1.552197in}{1.201654in}}{\pgfqpoint{1.541598in}{1.206044in}}{\pgfqpoint{1.530548in}{1.206044in}}%
\pgfpathcurveto{\pgfqpoint{1.519498in}{1.206044in}}{\pgfqpoint{1.508899in}{1.201654in}}{\pgfqpoint{1.501085in}{1.193840in}}%
\pgfpathcurveto{\pgfqpoint{1.493272in}{1.186027in}}{\pgfqpoint{1.488881in}{1.175428in}}{\pgfqpoint{1.488881in}{1.164378in}}%
\pgfpathcurveto{\pgfqpoint{1.488881in}{1.153327in}}{\pgfqpoint{1.493272in}{1.142728in}}{\pgfqpoint{1.501085in}{1.134915in}}%
\pgfpathcurveto{\pgfqpoint{1.508899in}{1.127101in}}{\pgfqpoint{1.519498in}{1.122711in}}{\pgfqpoint{1.530548in}{1.122711in}}%
\pgfpathclose%
\pgfusepath{stroke,fill}%
\end{pgfscope}%
\begin{pgfscope}%
\pgfpathrectangle{\pgfqpoint{0.375000in}{0.330000in}}{\pgfqpoint{2.325000in}{2.310000in}}%
\pgfusepath{clip}%
\pgfsetbuttcap%
\pgfsetroundjoin%
\definecolor{currentfill}{rgb}{0.000000,0.000000,0.000000}%
\pgfsetfillcolor{currentfill}%
\pgfsetlinewidth{1.003750pt}%
\definecolor{currentstroke}{rgb}{0.000000,0.000000,0.000000}%
\pgfsetstrokecolor{currentstroke}%
\pgfsetdash{}{0pt}%
\pgfpathmoveto{\pgfqpoint{1.530548in}{1.122711in}}%
\pgfpathcurveto{\pgfqpoint{1.541598in}{1.122711in}}{\pgfqpoint{1.552197in}{1.127101in}}{\pgfqpoint{1.560011in}{1.134915in}}%
\pgfpathcurveto{\pgfqpoint{1.567825in}{1.142728in}}{\pgfqpoint{1.572215in}{1.153327in}}{\pgfqpoint{1.572215in}{1.164378in}}%
\pgfpathcurveto{\pgfqpoint{1.572215in}{1.175428in}}{\pgfqpoint{1.567825in}{1.186027in}}{\pgfqpoint{1.560011in}{1.193840in}}%
\pgfpathcurveto{\pgfqpoint{1.552197in}{1.201654in}}{\pgfqpoint{1.541598in}{1.206044in}}{\pgfqpoint{1.530548in}{1.206044in}}%
\pgfpathcurveto{\pgfqpoint{1.519498in}{1.206044in}}{\pgfqpoint{1.508899in}{1.201654in}}{\pgfqpoint{1.501085in}{1.193840in}}%
\pgfpathcurveto{\pgfqpoint{1.493272in}{1.186027in}}{\pgfqpoint{1.488881in}{1.175428in}}{\pgfqpoint{1.488881in}{1.164378in}}%
\pgfpathcurveto{\pgfqpoint{1.488881in}{1.153327in}}{\pgfqpoint{1.493272in}{1.142728in}}{\pgfqpoint{1.501085in}{1.134915in}}%
\pgfpathcurveto{\pgfqpoint{1.508899in}{1.127101in}}{\pgfqpoint{1.519498in}{1.122711in}}{\pgfqpoint{1.530548in}{1.122711in}}%
\pgfpathclose%
\pgfusepath{stroke,fill}%
\end{pgfscope}%
\begin{pgfscope}%
\pgfpathrectangle{\pgfqpoint{0.375000in}{0.330000in}}{\pgfqpoint{2.325000in}{2.310000in}}%
\pgfusepath{clip}%
\pgfsetbuttcap%
\pgfsetroundjoin%
\definecolor{currentfill}{rgb}{0.000000,0.000000,0.000000}%
\pgfsetfillcolor{currentfill}%
\pgfsetlinewidth{1.003750pt}%
\definecolor{currentstroke}{rgb}{0.000000,0.000000,0.000000}%
\pgfsetstrokecolor{currentstroke}%
\pgfsetdash{}{0pt}%
\pgfpathmoveto{\pgfqpoint{1.530548in}{1.070680in}}%
\pgfpathcurveto{\pgfqpoint{1.541598in}{1.070680in}}{\pgfqpoint{1.552197in}{1.075070in}}{\pgfqpoint{1.560011in}{1.082884in}}%
\pgfpathcurveto{\pgfqpoint{1.567825in}{1.090698in}}{\pgfqpoint{1.572215in}{1.101297in}}{\pgfqpoint{1.572215in}{1.112347in}}%
\pgfpathcurveto{\pgfqpoint{1.572215in}{1.123397in}}{\pgfqpoint{1.567825in}{1.133996in}}{\pgfqpoint{1.560011in}{1.141810in}}%
\pgfpathcurveto{\pgfqpoint{1.552197in}{1.149623in}}{\pgfqpoint{1.541598in}{1.154013in}}{\pgfqpoint{1.530548in}{1.154013in}}%
\pgfpathcurveto{\pgfqpoint{1.519498in}{1.154013in}}{\pgfqpoint{1.508899in}{1.149623in}}{\pgfqpoint{1.501085in}{1.141810in}}%
\pgfpathcurveto{\pgfqpoint{1.493272in}{1.133996in}}{\pgfqpoint{1.488881in}{1.123397in}}{\pgfqpoint{1.488881in}{1.112347in}}%
\pgfpathcurveto{\pgfqpoint{1.488881in}{1.101297in}}{\pgfqpoint{1.493272in}{1.090698in}}{\pgfqpoint{1.501085in}{1.082884in}}%
\pgfpathcurveto{\pgfqpoint{1.508899in}{1.075070in}}{\pgfqpoint{1.519498in}{1.070680in}}{\pgfqpoint{1.530548in}{1.070680in}}%
\pgfpathclose%
\pgfusepath{stroke,fill}%
\end{pgfscope}%
\begin{pgfscope}%
\pgfpathrectangle{\pgfqpoint{0.375000in}{0.330000in}}{\pgfqpoint{2.325000in}{2.310000in}}%
\pgfusepath{clip}%
\pgfsetbuttcap%
\pgfsetroundjoin%
\definecolor{currentfill}{rgb}{0.000000,0.000000,0.000000}%
\pgfsetfillcolor{currentfill}%
\pgfsetlinewidth{1.003750pt}%
\definecolor{currentstroke}{rgb}{0.000000,0.000000,0.000000}%
\pgfsetstrokecolor{currentstroke}%
\pgfsetdash{}{0pt}%
\pgfpathmoveto{\pgfqpoint{1.530548in}{1.070680in}}%
\pgfpathcurveto{\pgfqpoint{1.541598in}{1.070680in}}{\pgfqpoint{1.552197in}{1.075070in}}{\pgfqpoint{1.560011in}{1.082884in}}%
\pgfpathcurveto{\pgfqpoint{1.567825in}{1.090698in}}{\pgfqpoint{1.572215in}{1.101297in}}{\pgfqpoint{1.572215in}{1.112347in}}%
\pgfpathcurveto{\pgfqpoint{1.572215in}{1.123397in}}{\pgfqpoint{1.567825in}{1.133996in}}{\pgfqpoint{1.560011in}{1.141810in}}%
\pgfpathcurveto{\pgfqpoint{1.552197in}{1.149623in}}{\pgfqpoint{1.541598in}{1.154013in}}{\pgfqpoint{1.530548in}{1.154013in}}%
\pgfpathcurveto{\pgfqpoint{1.519498in}{1.154013in}}{\pgfqpoint{1.508899in}{1.149623in}}{\pgfqpoint{1.501085in}{1.141810in}}%
\pgfpathcurveto{\pgfqpoint{1.493272in}{1.133996in}}{\pgfqpoint{1.488881in}{1.123397in}}{\pgfqpoint{1.488881in}{1.112347in}}%
\pgfpathcurveto{\pgfqpoint{1.488881in}{1.101297in}}{\pgfqpoint{1.493272in}{1.090698in}}{\pgfqpoint{1.501085in}{1.082884in}}%
\pgfpathcurveto{\pgfqpoint{1.508899in}{1.075070in}}{\pgfqpoint{1.519498in}{1.070680in}}{\pgfqpoint{1.530548in}{1.070680in}}%
\pgfpathclose%
\pgfusepath{stroke,fill}%
\end{pgfscope}%
\begin{pgfscope}%
\pgfpathrectangle{\pgfqpoint{0.375000in}{0.330000in}}{\pgfqpoint{2.325000in}{2.310000in}}%
\pgfusepath{clip}%
\pgfsetbuttcap%
\pgfsetroundjoin%
\definecolor{currentfill}{rgb}{0.000000,0.000000,0.000000}%
\pgfsetfillcolor{currentfill}%
\pgfsetlinewidth{1.003750pt}%
\definecolor{currentstroke}{rgb}{0.000000,0.000000,0.000000}%
\pgfsetstrokecolor{currentstroke}%
\pgfsetdash{}{0pt}%
\pgfpathmoveto{\pgfqpoint{1.530548in}{1.070680in}}%
\pgfpathcurveto{\pgfqpoint{1.541598in}{1.070680in}}{\pgfqpoint{1.552197in}{1.075070in}}{\pgfqpoint{1.560011in}{1.082884in}}%
\pgfpathcurveto{\pgfqpoint{1.567825in}{1.090698in}}{\pgfqpoint{1.572215in}{1.101297in}}{\pgfqpoint{1.572215in}{1.112347in}}%
\pgfpathcurveto{\pgfqpoint{1.572215in}{1.123397in}}{\pgfqpoint{1.567825in}{1.133996in}}{\pgfqpoint{1.560011in}{1.141810in}}%
\pgfpathcurveto{\pgfqpoint{1.552197in}{1.149623in}}{\pgfqpoint{1.541598in}{1.154013in}}{\pgfqpoint{1.530548in}{1.154013in}}%
\pgfpathcurveto{\pgfqpoint{1.519498in}{1.154013in}}{\pgfqpoint{1.508899in}{1.149623in}}{\pgfqpoint{1.501085in}{1.141810in}}%
\pgfpathcurveto{\pgfqpoint{1.493272in}{1.133996in}}{\pgfqpoint{1.488881in}{1.123397in}}{\pgfqpoint{1.488881in}{1.112347in}}%
\pgfpathcurveto{\pgfqpoint{1.488881in}{1.101297in}}{\pgfqpoint{1.493272in}{1.090698in}}{\pgfqpoint{1.501085in}{1.082884in}}%
\pgfpathcurveto{\pgfqpoint{1.508899in}{1.075070in}}{\pgfqpoint{1.519498in}{1.070680in}}{\pgfqpoint{1.530548in}{1.070680in}}%
\pgfpathclose%
\pgfusepath{stroke,fill}%
\end{pgfscope}%
\begin{pgfscope}%
\pgfpathrectangle{\pgfqpoint{0.375000in}{0.330000in}}{\pgfqpoint{2.325000in}{2.310000in}}%
\pgfusepath{clip}%
\pgfsetbuttcap%
\pgfsetroundjoin%
\definecolor{currentfill}{rgb}{0.000000,0.000000,0.000000}%
\pgfsetfillcolor{currentfill}%
\pgfsetlinewidth{1.003750pt}%
\definecolor{currentstroke}{rgb}{0.000000,0.000000,0.000000}%
\pgfsetstrokecolor{currentstroke}%
\pgfsetdash{}{0pt}%
\pgfpathmoveto{\pgfqpoint{1.530548in}{1.018649in}}%
\pgfpathcurveto{\pgfqpoint{1.541598in}{1.018649in}}{\pgfqpoint{1.552197in}{1.023040in}}{\pgfqpoint{1.560011in}{1.030853in}}%
\pgfpathcurveto{\pgfqpoint{1.567825in}{1.038667in}}{\pgfqpoint{1.572215in}{1.049266in}}{\pgfqpoint{1.572215in}{1.060316in}}%
\pgfpathcurveto{\pgfqpoint{1.572215in}{1.071366in}}{\pgfqpoint{1.567825in}{1.081965in}}{\pgfqpoint{1.560011in}{1.089779in}}%
\pgfpathcurveto{\pgfqpoint{1.552197in}{1.097592in}}{\pgfqpoint{1.541598in}{1.101983in}}{\pgfqpoint{1.530548in}{1.101983in}}%
\pgfpathcurveto{\pgfqpoint{1.519498in}{1.101983in}}{\pgfqpoint{1.508899in}{1.097592in}}{\pgfqpoint{1.501085in}{1.089779in}}%
\pgfpathcurveto{\pgfqpoint{1.493272in}{1.081965in}}{\pgfqpoint{1.488881in}{1.071366in}}{\pgfqpoint{1.488881in}{1.060316in}}%
\pgfpathcurveto{\pgfqpoint{1.488881in}{1.049266in}}{\pgfqpoint{1.493272in}{1.038667in}}{\pgfqpoint{1.501085in}{1.030853in}}%
\pgfpathcurveto{\pgfqpoint{1.508899in}{1.023040in}}{\pgfqpoint{1.519498in}{1.018649in}}{\pgfqpoint{1.530548in}{1.018649in}}%
\pgfpathclose%
\pgfusepath{stroke,fill}%
\end{pgfscope}%
\begin{pgfscope}%
\pgfpathrectangle{\pgfqpoint{0.375000in}{0.330000in}}{\pgfqpoint{2.325000in}{2.310000in}}%
\pgfusepath{clip}%
\pgfsetbuttcap%
\pgfsetroundjoin%
\definecolor{currentfill}{rgb}{0.000000,0.000000,0.000000}%
\pgfsetfillcolor{currentfill}%
\pgfsetlinewidth{1.003750pt}%
\definecolor{currentstroke}{rgb}{0.000000,0.000000,0.000000}%
\pgfsetstrokecolor{currentstroke}%
\pgfsetdash{}{0pt}%
\pgfpathmoveto{\pgfqpoint{1.530548in}{1.122711in}}%
\pgfpathcurveto{\pgfqpoint{1.541598in}{1.122711in}}{\pgfqpoint{1.552197in}{1.127101in}}{\pgfqpoint{1.560011in}{1.134915in}}%
\pgfpathcurveto{\pgfqpoint{1.567825in}{1.142728in}}{\pgfqpoint{1.572215in}{1.153327in}}{\pgfqpoint{1.572215in}{1.164378in}}%
\pgfpathcurveto{\pgfqpoint{1.572215in}{1.175428in}}{\pgfqpoint{1.567825in}{1.186027in}}{\pgfqpoint{1.560011in}{1.193840in}}%
\pgfpathcurveto{\pgfqpoint{1.552197in}{1.201654in}}{\pgfqpoint{1.541598in}{1.206044in}}{\pgfqpoint{1.530548in}{1.206044in}}%
\pgfpathcurveto{\pgfqpoint{1.519498in}{1.206044in}}{\pgfqpoint{1.508899in}{1.201654in}}{\pgfqpoint{1.501085in}{1.193840in}}%
\pgfpathcurveto{\pgfqpoint{1.493272in}{1.186027in}}{\pgfqpoint{1.488881in}{1.175428in}}{\pgfqpoint{1.488881in}{1.164378in}}%
\pgfpathcurveto{\pgfqpoint{1.488881in}{1.153327in}}{\pgfqpoint{1.493272in}{1.142728in}}{\pgfqpoint{1.501085in}{1.134915in}}%
\pgfpathcurveto{\pgfqpoint{1.508899in}{1.127101in}}{\pgfqpoint{1.519498in}{1.122711in}}{\pgfqpoint{1.530548in}{1.122711in}}%
\pgfpathclose%
\pgfusepath{stroke,fill}%
\end{pgfscope}%
\begin{pgfscope}%
\pgfpathrectangle{\pgfqpoint{0.375000in}{0.330000in}}{\pgfqpoint{2.325000in}{2.310000in}}%
\pgfusepath{clip}%
\pgfsetbuttcap%
\pgfsetroundjoin%
\definecolor{currentfill}{rgb}{0.000000,0.000000,0.000000}%
\pgfsetfillcolor{currentfill}%
\pgfsetlinewidth{1.003750pt}%
\definecolor{currentstroke}{rgb}{0.000000,0.000000,0.000000}%
\pgfsetstrokecolor{currentstroke}%
\pgfsetdash{}{0pt}%
\pgfpathmoveto{\pgfqpoint{1.530548in}{1.070680in}}%
\pgfpathcurveto{\pgfqpoint{1.541598in}{1.070680in}}{\pgfqpoint{1.552197in}{1.075070in}}{\pgfqpoint{1.560011in}{1.082884in}}%
\pgfpathcurveto{\pgfqpoint{1.567825in}{1.090698in}}{\pgfqpoint{1.572215in}{1.101297in}}{\pgfqpoint{1.572215in}{1.112347in}}%
\pgfpathcurveto{\pgfqpoint{1.572215in}{1.123397in}}{\pgfqpoint{1.567825in}{1.133996in}}{\pgfqpoint{1.560011in}{1.141810in}}%
\pgfpathcurveto{\pgfqpoint{1.552197in}{1.149623in}}{\pgfqpoint{1.541598in}{1.154013in}}{\pgfqpoint{1.530548in}{1.154013in}}%
\pgfpathcurveto{\pgfqpoint{1.519498in}{1.154013in}}{\pgfqpoint{1.508899in}{1.149623in}}{\pgfqpoint{1.501085in}{1.141810in}}%
\pgfpathcurveto{\pgfqpoint{1.493272in}{1.133996in}}{\pgfqpoint{1.488881in}{1.123397in}}{\pgfqpoint{1.488881in}{1.112347in}}%
\pgfpathcurveto{\pgfqpoint{1.488881in}{1.101297in}}{\pgfqpoint{1.493272in}{1.090698in}}{\pgfqpoint{1.501085in}{1.082884in}}%
\pgfpathcurveto{\pgfqpoint{1.508899in}{1.075070in}}{\pgfqpoint{1.519498in}{1.070680in}}{\pgfqpoint{1.530548in}{1.070680in}}%
\pgfpathclose%
\pgfusepath{stroke,fill}%
\end{pgfscope}%
\begin{pgfscope}%
\pgfpathrectangle{\pgfqpoint{0.375000in}{0.330000in}}{\pgfqpoint{2.325000in}{2.310000in}}%
\pgfusepath{clip}%
\pgfsetbuttcap%
\pgfsetroundjoin%
\definecolor{currentfill}{rgb}{0.000000,0.000000,0.000000}%
\pgfsetfillcolor{currentfill}%
\pgfsetlinewidth{1.003750pt}%
\definecolor{currentstroke}{rgb}{0.000000,0.000000,0.000000}%
\pgfsetstrokecolor{currentstroke}%
\pgfsetdash{}{0pt}%
\pgfpathmoveto{\pgfqpoint{1.530548in}{1.122711in}}%
\pgfpathcurveto{\pgfqpoint{1.541598in}{1.122711in}}{\pgfqpoint{1.552197in}{1.127101in}}{\pgfqpoint{1.560011in}{1.134915in}}%
\pgfpathcurveto{\pgfqpoint{1.567825in}{1.142728in}}{\pgfqpoint{1.572215in}{1.153327in}}{\pgfqpoint{1.572215in}{1.164378in}}%
\pgfpathcurveto{\pgfqpoint{1.572215in}{1.175428in}}{\pgfqpoint{1.567825in}{1.186027in}}{\pgfqpoint{1.560011in}{1.193840in}}%
\pgfpathcurveto{\pgfqpoint{1.552197in}{1.201654in}}{\pgfqpoint{1.541598in}{1.206044in}}{\pgfqpoint{1.530548in}{1.206044in}}%
\pgfpathcurveto{\pgfqpoint{1.519498in}{1.206044in}}{\pgfqpoint{1.508899in}{1.201654in}}{\pgfqpoint{1.501085in}{1.193840in}}%
\pgfpathcurveto{\pgfqpoint{1.493272in}{1.186027in}}{\pgfqpoint{1.488881in}{1.175428in}}{\pgfqpoint{1.488881in}{1.164378in}}%
\pgfpathcurveto{\pgfqpoint{1.488881in}{1.153327in}}{\pgfqpoint{1.493272in}{1.142728in}}{\pgfqpoint{1.501085in}{1.134915in}}%
\pgfpathcurveto{\pgfqpoint{1.508899in}{1.127101in}}{\pgfqpoint{1.519498in}{1.122711in}}{\pgfqpoint{1.530548in}{1.122711in}}%
\pgfpathclose%
\pgfusepath{stroke,fill}%
\end{pgfscope}%
\begin{pgfscope}%
\pgfpathrectangle{\pgfqpoint{0.375000in}{0.330000in}}{\pgfqpoint{2.325000in}{2.310000in}}%
\pgfusepath{clip}%
\pgfsetbuttcap%
\pgfsetroundjoin%
\definecolor{currentfill}{rgb}{0.000000,0.000000,0.000000}%
\pgfsetfillcolor{currentfill}%
\pgfsetlinewidth{1.003750pt}%
\definecolor{currentstroke}{rgb}{0.000000,0.000000,0.000000}%
\pgfsetstrokecolor{currentstroke}%
\pgfsetdash{}{0pt}%
\pgfpathmoveto{\pgfqpoint{1.530548in}{1.070680in}}%
\pgfpathcurveto{\pgfqpoint{1.541598in}{1.070680in}}{\pgfqpoint{1.552197in}{1.075070in}}{\pgfqpoint{1.560011in}{1.082884in}}%
\pgfpathcurveto{\pgfqpoint{1.567825in}{1.090698in}}{\pgfqpoint{1.572215in}{1.101297in}}{\pgfqpoint{1.572215in}{1.112347in}}%
\pgfpathcurveto{\pgfqpoint{1.572215in}{1.123397in}}{\pgfqpoint{1.567825in}{1.133996in}}{\pgfqpoint{1.560011in}{1.141810in}}%
\pgfpathcurveto{\pgfqpoint{1.552197in}{1.149623in}}{\pgfqpoint{1.541598in}{1.154013in}}{\pgfqpoint{1.530548in}{1.154013in}}%
\pgfpathcurveto{\pgfqpoint{1.519498in}{1.154013in}}{\pgfqpoint{1.508899in}{1.149623in}}{\pgfqpoint{1.501085in}{1.141810in}}%
\pgfpathcurveto{\pgfqpoint{1.493272in}{1.133996in}}{\pgfqpoint{1.488881in}{1.123397in}}{\pgfqpoint{1.488881in}{1.112347in}}%
\pgfpathcurveto{\pgfqpoint{1.488881in}{1.101297in}}{\pgfqpoint{1.493272in}{1.090698in}}{\pgfqpoint{1.501085in}{1.082884in}}%
\pgfpathcurveto{\pgfqpoint{1.508899in}{1.075070in}}{\pgfqpoint{1.519498in}{1.070680in}}{\pgfqpoint{1.530548in}{1.070680in}}%
\pgfpathclose%
\pgfusepath{stroke,fill}%
\end{pgfscope}%
\begin{pgfscope}%
\pgfpathrectangle{\pgfqpoint{0.375000in}{0.330000in}}{\pgfqpoint{2.325000in}{2.310000in}}%
\pgfusepath{clip}%
\pgfsetbuttcap%
\pgfsetroundjoin%
\definecolor{currentfill}{rgb}{0.000000,0.000000,0.000000}%
\pgfsetfillcolor{currentfill}%
\pgfsetlinewidth{1.003750pt}%
\definecolor{currentstroke}{rgb}{0.000000,0.000000,0.000000}%
\pgfsetstrokecolor{currentstroke}%
\pgfsetdash{}{0pt}%
\pgfpathmoveto{\pgfqpoint{1.530548in}{1.174742in}}%
\pgfpathcurveto{\pgfqpoint{1.541598in}{1.174742in}}{\pgfqpoint{1.552197in}{1.179132in}}{\pgfqpoint{1.560011in}{1.186946in}}%
\pgfpathcurveto{\pgfqpoint{1.567825in}{1.194759in}}{\pgfqpoint{1.572215in}{1.205358in}}{\pgfqpoint{1.572215in}{1.216408in}}%
\pgfpathcurveto{\pgfqpoint{1.572215in}{1.227459in}}{\pgfqpoint{1.567825in}{1.238058in}}{\pgfqpoint{1.560011in}{1.245871in}}%
\pgfpathcurveto{\pgfqpoint{1.552197in}{1.253685in}}{\pgfqpoint{1.541598in}{1.258075in}}{\pgfqpoint{1.530548in}{1.258075in}}%
\pgfpathcurveto{\pgfqpoint{1.519498in}{1.258075in}}{\pgfqpoint{1.508899in}{1.253685in}}{\pgfqpoint{1.501085in}{1.245871in}}%
\pgfpathcurveto{\pgfqpoint{1.493272in}{1.238058in}}{\pgfqpoint{1.488881in}{1.227459in}}{\pgfqpoint{1.488881in}{1.216408in}}%
\pgfpathcurveto{\pgfqpoint{1.488881in}{1.205358in}}{\pgfqpoint{1.493272in}{1.194759in}}{\pgfqpoint{1.501085in}{1.186946in}}%
\pgfpathcurveto{\pgfqpoint{1.508899in}{1.179132in}}{\pgfqpoint{1.519498in}{1.174742in}}{\pgfqpoint{1.530548in}{1.174742in}}%
\pgfpathclose%
\pgfusepath{stroke,fill}%
\end{pgfscope}%
\begin{pgfscope}%
\pgfpathrectangle{\pgfqpoint{0.375000in}{0.330000in}}{\pgfqpoint{2.325000in}{2.310000in}}%
\pgfusepath{clip}%
\pgfsetbuttcap%
\pgfsetroundjoin%
\definecolor{currentfill}{rgb}{0.000000,0.000000,0.000000}%
\pgfsetfillcolor{currentfill}%
\pgfsetlinewidth{1.003750pt}%
\definecolor{currentstroke}{rgb}{0.000000,0.000000,0.000000}%
\pgfsetstrokecolor{currentstroke}%
\pgfsetdash{}{0pt}%
\pgfpathmoveto{\pgfqpoint{1.530548in}{1.070680in}}%
\pgfpathcurveto{\pgfqpoint{1.541598in}{1.070680in}}{\pgfqpoint{1.552197in}{1.075070in}}{\pgfqpoint{1.560011in}{1.082884in}}%
\pgfpathcurveto{\pgfqpoint{1.567825in}{1.090698in}}{\pgfqpoint{1.572215in}{1.101297in}}{\pgfqpoint{1.572215in}{1.112347in}}%
\pgfpathcurveto{\pgfqpoint{1.572215in}{1.123397in}}{\pgfqpoint{1.567825in}{1.133996in}}{\pgfqpoint{1.560011in}{1.141810in}}%
\pgfpathcurveto{\pgfqpoint{1.552197in}{1.149623in}}{\pgfqpoint{1.541598in}{1.154013in}}{\pgfqpoint{1.530548in}{1.154013in}}%
\pgfpathcurveto{\pgfqpoint{1.519498in}{1.154013in}}{\pgfqpoint{1.508899in}{1.149623in}}{\pgfqpoint{1.501085in}{1.141810in}}%
\pgfpathcurveto{\pgfqpoint{1.493272in}{1.133996in}}{\pgfqpoint{1.488881in}{1.123397in}}{\pgfqpoint{1.488881in}{1.112347in}}%
\pgfpathcurveto{\pgfqpoint{1.488881in}{1.101297in}}{\pgfqpoint{1.493272in}{1.090698in}}{\pgfqpoint{1.501085in}{1.082884in}}%
\pgfpathcurveto{\pgfqpoint{1.508899in}{1.075070in}}{\pgfqpoint{1.519498in}{1.070680in}}{\pgfqpoint{1.530548in}{1.070680in}}%
\pgfpathclose%
\pgfusepath{stroke,fill}%
\end{pgfscope}%
\begin{pgfscope}%
\pgfpathrectangle{\pgfqpoint{0.375000in}{0.330000in}}{\pgfqpoint{2.325000in}{2.310000in}}%
\pgfusepath{clip}%
\pgfsetbuttcap%
\pgfsetroundjoin%
\definecolor{currentfill}{rgb}{0.000000,0.000000,0.000000}%
\pgfsetfillcolor{currentfill}%
\pgfsetlinewidth{1.003750pt}%
\definecolor{currentstroke}{rgb}{0.000000,0.000000,0.000000}%
\pgfsetstrokecolor{currentstroke}%
\pgfsetdash{}{0pt}%
\pgfpathmoveto{\pgfqpoint{1.530548in}{1.070680in}}%
\pgfpathcurveto{\pgfqpoint{1.541598in}{1.070680in}}{\pgfqpoint{1.552197in}{1.075070in}}{\pgfqpoint{1.560011in}{1.082884in}}%
\pgfpathcurveto{\pgfqpoint{1.567825in}{1.090698in}}{\pgfqpoint{1.572215in}{1.101297in}}{\pgfqpoint{1.572215in}{1.112347in}}%
\pgfpathcurveto{\pgfqpoint{1.572215in}{1.123397in}}{\pgfqpoint{1.567825in}{1.133996in}}{\pgfqpoint{1.560011in}{1.141810in}}%
\pgfpathcurveto{\pgfqpoint{1.552197in}{1.149623in}}{\pgfqpoint{1.541598in}{1.154013in}}{\pgfqpoint{1.530548in}{1.154013in}}%
\pgfpathcurveto{\pgfqpoint{1.519498in}{1.154013in}}{\pgfqpoint{1.508899in}{1.149623in}}{\pgfqpoint{1.501085in}{1.141810in}}%
\pgfpathcurveto{\pgfqpoint{1.493272in}{1.133996in}}{\pgfqpoint{1.488881in}{1.123397in}}{\pgfqpoint{1.488881in}{1.112347in}}%
\pgfpathcurveto{\pgfqpoint{1.488881in}{1.101297in}}{\pgfqpoint{1.493272in}{1.090698in}}{\pgfqpoint{1.501085in}{1.082884in}}%
\pgfpathcurveto{\pgfqpoint{1.508899in}{1.075070in}}{\pgfqpoint{1.519498in}{1.070680in}}{\pgfqpoint{1.530548in}{1.070680in}}%
\pgfpathclose%
\pgfusepath{stroke,fill}%
\end{pgfscope}%
\begin{pgfscope}%
\pgfpathrectangle{\pgfqpoint{0.375000in}{0.330000in}}{\pgfqpoint{2.325000in}{2.310000in}}%
\pgfusepath{clip}%
\pgfsetbuttcap%
\pgfsetroundjoin%
\definecolor{currentfill}{rgb}{0.000000,0.000000,0.000000}%
\pgfsetfillcolor{currentfill}%
\pgfsetlinewidth{1.003750pt}%
\definecolor{currentstroke}{rgb}{0.000000,0.000000,0.000000}%
\pgfsetstrokecolor{currentstroke}%
\pgfsetdash{}{0pt}%
\pgfpathmoveto{\pgfqpoint{1.530548in}{1.018649in}}%
\pgfpathcurveto{\pgfqpoint{1.541598in}{1.018649in}}{\pgfqpoint{1.552197in}{1.023040in}}{\pgfqpoint{1.560011in}{1.030853in}}%
\pgfpathcurveto{\pgfqpoint{1.567825in}{1.038667in}}{\pgfqpoint{1.572215in}{1.049266in}}{\pgfqpoint{1.572215in}{1.060316in}}%
\pgfpathcurveto{\pgfqpoint{1.572215in}{1.071366in}}{\pgfqpoint{1.567825in}{1.081965in}}{\pgfqpoint{1.560011in}{1.089779in}}%
\pgfpathcurveto{\pgfqpoint{1.552197in}{1.097592in}}{\pgfqpoint{1.541598in}{1.101983in}}{\pgfqpoint{1.530548in}{1.101983in}}%
\pgfpathcurveto{\pgfqpoint{1.519498in}{1.101983in}}{\pgfqpoint{1.508899in}{1.097592in}}{\pgfqpoint{1.501085in}{1.089779in}}%
\pgfpathcurveto{\pgfqpoint{1.493272in}{1.081965in}}{\pgfqpoint{1.488881in}{1.071366in}}{\pgfqpoint{1.488881in}{1.060316in}}%
\pgfpathcurveto{\pgfqpoint{1.488881in}{1.049266in}}{\pgfqpoint{1.493272in}{1.038667in}}{\pgfqpoint{1.501085in}{1.030853in}}%
\pgfpathcurveto{\pgfqpoint{1.508899in}{1.023040in}}{\pgfqpoint{1.519498in}{1.018649in}}{\pgfqpoint{1.530548in}{1.018649in}}%
\pgfpathclose%
\pgfusepath{stroke,fill}%
\end{pgfscope}%
\begin{pgfscope}%
\pgfpathrectangle{\pgfqpoint{0.375000in}{0.330000in}}{\pgfqpoint{2.325000in}{2.310000in}}%
\pgfusepath{clip}%
\pgfsetbuttcap%
\pgfsetroundjoin%
\definecolor{currentfill}{rgb}{0.000000,0.000000,0.000000}%
\pgfsetfillcolor{currentfill}%
\pgfsetlinewidth{1.003750pt}%
\definecolor{currentstroke}{rgb}{0.000000,0.000000,0.000000}%
\pgfsetstrokecolor{currentstroke}%
\pgfsetdash{}{0pt}%
\pgfpathmoveto{\pgfqpoint{1.530548in}{1.070680in}}%
\pgfpathcurveto{\pgfqpoint{1.541598in}{1.070680in}}{\pgfqpoint{1.552197in}{1.075070in}}{\pgfqpoint{1.560011in}{1.082884in}}%
\pgfpathcurveto{\pgfqpoint{1.567825in}{1.090698in}}{\pgfqpoint{1.572215in}{1.101297in}}{\pgfqpoint{1.572215in}{1.112347in}}%
\pgfpathcurveto{\pgfqpoint{1.572215in}{1.123397in}}{\pgfqpoint{1.567825in}{1.133996in}}{\pgfqpoint{1.560011in}{1.141810in}}%
\pgfpathcurveto{\pgfqpoint{1.552197in}{1.149623in}}{\pgfqpoint{1.541598in}{1.154013in}}{\pgfqpoint{1.530548in}{1.154013in}}%
\pgfpathcurveto{\pgfqpoint{1.519498in}{1.154013in}}{\pgfqpoint{1.508899in}{1.149623in}}{\pgfqpoint{1.501085in}{1.141810in}}%
\pgfpathcurveto{\pgfqpoint{1.493272in}{1.133996in}}{\pgfqpoint{1.488881in}{1.123397in}}{\pgfqpoint{1.488881in}{1.112347in}}%
\pgfpathcurveto{\pgfqpoint{1.488881in}{1.101297in}}{\pgfqpoint{1.493272in}{1.090698in}}{\pgfqpoint{1.501085in}{1.082884in}}%
\pgfpathcurveto{\pgfqpoint{1.508899in}{1.075070in}}{\pgfqpoint{1.519498in}{1.070680in}}{\pgfqpoint{1.530548in}{1.070680in}}%
\pgfpathclose%
\pgfusepath{stroke,fill}%
\end{pgfscope}%
\begin{pgfscope}%
\pgfpathrectangle{\pgfqpoint{0.375000in}{0.330000in}}{\pgfqpoint{2.325000in}{2.310000in}}%
\pgfusepath{clip}%
\pgfsetbuttcap%
\pgfsetroundjoin%
\definecolor{currentfill}{rgb}{0.000000,0.000000,0.000000}%
\pgfsetfillcolor{currentfill}%
\pgfsetlinewidth{1.003750pt}%
\definecolor{currentstroke}{rgb}{0.000000,0.000000,0.000000}%
\pgfsetstrokecolor{currentstroke}%
\pgfsetdash{}{0pt}%
\pgfpathmoveto{\pgfqpoint{1.530548in}{1.070680in}}%
\pgfpathcurveto{\pgfqpoint{1.541598in}{1.070680in}}{\pgfqpoint{1.552197in}{1.075070in}}{\pgfqpoint{1.560011in}{1.082884in}}%
\pgfpathcurveto{\pgfqpoint{1.567825in}{1.090698in}}{\pgfqpoint{1.572215in}{1.101297in}}{\pgfqpoint{1.572215in}{1.112347in}}%
\pgfpathcurveto{\pgfqpoint{1.572215in}{1.123397in}}{\pgfqpoint{1.567825in}{1.133996in}}{\pgfqpoint{1.560011in}{1.141810in}}%
\pgfpathcurveto{\pgfqpoint{1.552197in}{1.149623in}}{\pgfqpoint{1.541598in}{1.154013in}}{\pgfqpoint{1.530548in}{1.154013in}}%
\pgfpathcurveto{\pgfqpoint{1.519498in}{1.154013in}}{\pgfqpoint{1.508899in}{1.149623in}}{\pgfqpoint{1.501085in}{1.141810in}}%
\pgfpathcurveto{\pgfqpoint{1.493272in}{1.133996in}}{\pgfqpoint{1.488881in}{1.123397in}}{\pgfqpoint{1.488881in}{1.112347in}}%
\pgfpathcurveto{\pgfqpoint{1.488881in}{1.101297in}}{\pgfqpoint{1.493272in}{1.090698in}}{\pgfqpoint{1.501085in}{1.082884in}}%
\pgfpathcurveto{\pgfqpoint{1.508899in}{1.075070in}}{\pgfqpoint{1.519498in}{1.070680in}}{\pgfqpoint{1.530548in}{1.070680in}}%
\pgfpathclose%
\pgfusepath{stroke,fill}%
\end{pgfscope}%
\begin{pgfscope}%
\pgfpathrectangle{\pgfqpoint{0.375000in}{0.330000in}}{\pgfqpoint{2.325000in}{2.310000in}}%
\pgfusepath{clip}%
\pgfsetbuttcap%
\pgfsetroundjoin%
\definecolor{currentfill}{rgb}{0.000000,0.000000,0.000000}%
\pgfsetfillcolor{currentfill}%
\pgfsetlinewidth{1.003750pt}%
\definecolor{currentstroke}{rgb}{0.000000,0.000000,0.000000}%
\pgfsetstrokecolor{currentstroke}%
\pgfsetdash{}{0pt}%
\pgfpathmoveto{\pgfqpoint{1.530548in}{1.122711in}}%
\pgfpathcurveto{\pgfqpoint{1.541598in}{1.122711in}}{\pgfqpoint{1.552197in}{1.127101in}}{\pgfqpoint{1.560011in}{1.134915in}}%
\pgfpathcurveto{\pgfqpoint{1.567825in}{1.142728in}}{\pgfqpoint{1.572215in}{1.153327in}}{\pgfqpoint{1.572215in}{1.164378in}}%
\pgfpathcurveto{\pgfqpoint{1.572215in}{1.175428in}}{\pgfqpoint{1.567825in}{1.186027in}}{\pgfqpoint{1.560011in}{1.193840in}}%
\pgfpathcurveto{\pgfqpoint{1.552197in}{1.201654in}}{\pgfqpoint{1.541598in}{1.206044in}}{\pgfqpoint{1.530548in}{1.206044in}}%
\pgfpathcurveto{\pgfqpoint{1.519498in}{1.206044in}}{\pgfqpoint{1.508899in}{1.201654in}}{\pgfqpoint{1.501085in}{1.193840in}}%
\pgfpathcurveto{\pgfqpoint{1.493272in}{1.186027in}}{\pgfqpoint{1.488881in}{1.175428in}}{\pgfqpoint{1.488881in}{1.164378in}}%
\pgfpathcurveto{\pgfqpoint{1.488881in}{1.153327in}}{\pgfqpoint{1.493272in}{1.142728in}}{\pgfqpoint{1.501085in}{1.134915in}}%
\pgfpathcurveto{\pgfqpoint{1.508899in}{1.127101in}}{\pgfqpoint{1.519498in}{1.122711in}}{\pgfqpoint{1.530548in}{1.122711in}}%
\pgfpathclose%
\pgfusepath{stroke,fill}%
\end{pgfscope}%
\begin{pgfscope}%
\pgfpathrectangle{\pgfqpoint{0.375000in}{0.330000in}}{\pgfqpoint{2.325000in}{2.310000in}}%
\pgfusepath{clip}%
\pgfsetbuttcap%
\pgfsetroundjoin%
\definecolor{currentfill}{rgb}{0.000000,0.000000,0.000000}%
\pgfsetfillcolor{currentfill}%
\pgfsetlinewidth{1.003750pt}%
\definecolor{currentstroke}{rgb}{0.000000,0.000000,0.000000}%
\pgfsetstrokecolor{currentstroke}%
\pgfsetdash{}{0pt}%
\pgfpathmoveto{\pgfqpoint{1.530548in}{1.122711in}}%
\pgfpathcurveto{\pgfqpoint{1.541598in}{1.122711in}}{\pgfqpoint{1.552197in}{1.127101in}}{\pgfqpoint{1.560011in}{1.134915in}}%
\pgfpathcurveto{\pgfqpoint{1.567825in}{1.142728in}}{\pgfqpoint{1.572215in}{1.153327in}}{\pgfqpoint{1.572215in}{1.164378in}}%
\pgfpathcurveto{\pgfqpoint{1.572215in}{1.175428in}}{\pgfqpoint{1.567825in}{1.186027in}}{\pgfqpoint{1.560011in}{1.193840in}}%
\pgfpathcurveto{\pgfqpoint{1.552197in}{1.201654in}}{\pgfqpoint{1.541598in}{1.206044in}}{\pgfqpoint{1.530548in}{1.206044in}}%
\pgfpathcurveto{\pgfqpoint{1.519498in}{1.206044in}}{\pgfqpoint{1.508899in}{1.201654in}}{\pgfqpoint{1.501085in}{1.193840in}}%
\pgfpathcurveto{\pgfqpoint{1.493272in}{1.186027in}}{\pgfqpoint{1.488881in}{1.175428in}}{\pgfqpoint{1.488881in}{1.164378in}}%
\pgfpathcurveto{\pgfqpoint{1.488881in}{1.153327in}}{\pgfqpoint{1.493272in}{1.142728in}}{\pgfqpoint{1.501085in}{1.134915in}}%
\pgfpathcurveto{\pgfqpoint{1.508899in}{1.127101in}}{\pgfqpoint{1.519498in}{1.122711in}}{\pgfqpoint{1.530548in}{1.122711in}}%
\pgfpathclose%
\pgfusepath{stroke,fill}%
\end{pgfscope}%
\begin{pgfscope}%
\pgfpathrectangle{\pgfqpoint{0.375000in}{0.330000in}}{\pgfqpoint{2.325000in}{2.310000in}}%
\pgfusepath{clip}%
\pgfsetbuttcap%
\pgfsetroundjoin%
\definecolor{currentfill}{rgb}{0.000000,0.000000,0.000000}%
\pgfsetfillcolor{currentfill}%
\pgfsetlinewidth{1.003750pt}%
\definecolor{currentstroke}{rgb}{0.000000,0.000000,0.000000}%
\pgfsetstrokecolor{currentstroke}%
\pgfsetdash{}{0pt}%
\pgfpathmoveto{\pgfqpoint{1.530548in}{1.122711in}}%
\pgfpathcurveto{\pgfqpoint{1.541598in}{1.122711in}}{\pgfqpoint{1.552197in}{1.127101in}}{\pgfqpoint{1.560011in}{1.134915in}}%
\pgfpathcurveto{\pgfqpoint{1.567825in}{1.142728in}}{\pgfqpoint{1.572215in}{1.153327in}}{\pgfqpoint{1.572215in}{1.164378in}}%
\pgfpathcurveto{\pgfqpoint{1.572215in}{1.175428in}}{\pgfqpoint{1.567825in}{1.186027in}}{\pgfqpoint{1.560011in}{1.193840in}}%
\pgfpathcurveto{\pgfqpoint{1.552197in}{1.201654in}}{\pgfqpoint{1.541598in}{1.206044in}}{\pgfqpoint{1.530548in}{1.206044in}}%
\pgfpathcurveto{\pgfqpoint{1.519498in}{1.206044in}}{\pgfqpoint{1.508899in}{1.201654in}}{\pgfqpoint{1.501085in}{1.193840in}}%
\pgfpathcurveto{\pgfqpoint{1.493272in}{1.186027in}}{\pgfqpoint{1.488881in}{1.175428in}}{\pgfqpoint{1.488881in}{1.164378in}}%
\pgfpathcurveto{\pgfqpoint{1.488881in}{1.153327in}}{\pgfqpoint{1.493272in}{1.142728in}}{\pgfqpoint{1.501085in}{1.134915in}}%
\pgfpathcurveto{\pgfqpoint{1.508899in}{1.127101in}}{\pgfqpoint{1.519498in}{1.122711in}}{\pgfqpoint{1.530548in}{1.122711in}}%
\pgfpathclose%
\pgfusepath{stroke,fill}%
\end{pgfscope}%
\begin{pgfscope}%
\pgfpathrectangle{\pgfqpoint{0.375000in}{0.330000in}}{\pgfqpoint{2.325000in}{2.310000in}}%
\pgfusepath{clip}%
\pgfsetbuttcap%
\pgfsetroundjoin%
\definecolor{currentfill}{rgb}{0.000000,0.000000,0.000000}%
\pgfsetfillcolor{currentfill}%
\pgfsetlinewidth{1.003750pt}%
\definecolor{currentstroke}{rgb}{0.000000,0.000000,0.000000}%
\pgfsetstrokecolor{currentstroke}%
\pgfsetdash{}{0pt}%
\pgfpathmoveto{\pgfqpoint{1.530548in}{1.122711in}}%
\pgfpathcurveto{\pgfqpoint{1.541598in}{1.122711in}}{\pgfqpoint{1.552197in}{1.127101in}}{\pgfqpoint{1.560011in}{1.134915in}}%
\pgfpathcurveto{\pgfqpoint{1.567825in}{1.142728in}}{\pgfqpoint{1.572215in}{1.153327in}}{\pgfqpoint{1.572215in}{1.164378in}}%
\pgfpathcurveto{\pgfqpoint{1.572215in}{1.175428in}}{\pgfqpoint{1.567825in}{1.186027in}}{\pgfqpoint{1.560011in}{1.193840in}}%
\pgfpathcurveto{\pgfqpoint{1.552197in}{1.201654in}}{\pgfqpoint{1.541598in}{1.206044in}}{\pgfqpoint{1.530548in}{1.206044in}}%
\pgfpathcurveto{\pgfqpoint{1.519498in}{1.206044in}}{\pgfqpoint{1.508899in}{1.201654in}}{\pgfqpoint{1.501085in}{1.193840in}}%
\pgfpathcurveto{\pgfqpoint{1.493272in}{1.186027in}}{\pgfqpoint{1.488881in}{1.175428in}}{\pgfqpoint{1.488881in}{1.164378in}}%
\pgfpathcurveto{\pgfqpoint{1.488881in}{1.153327in}}{\pgfqpoint{1.493272in}{1.142728in}}{\pgfqpoint{1.501085in}{1.134915in}}%
\pgfpathcurveto{\pgfqpoint{1.508899in}{1.127101in}}{\pgfqpoint{1.519498in}{1.122711in}}{\pgfqpoint{1.530548in}{1.122711in}}%
\pgfpathclose%
\pgfusepath{stroke,fill}%
\end{pgfscope}%
\begin{pgfscope}%
\pgfpathrectangle{\pgfqpoint{0.375000in}{0.330000in}}{\pgfqpoint{2.325000in}{2.310000in}}%
\pgfusepath{clip}%
\pgfsetbuttcap%
\pgfsetroundjoin%
\definecolor{currentfill}{rgb}{0.000000,0.000000,0.000000}%
\pgfsetfillcolor{currentfill}%
\pgfsetlinewidth{1.003750pt}%
\definecolor{currentstroke}{rgb}{0.000000,0.000000,0.000000}%
\pgfsetstrokecolor{currentstroke}%
\pgfsetdash{}{0pt}%
\pgfpathmoveto{\pgfqpoint{1.530548in}{1.122711in}}%
\pgfpathcurveto{\pgfqpoint{1.541598in}{1.122711in}}{\pgfqpoint{1.552197in}{1.127101in}}{\pgfqpoint{1.560011in}{1.134915in}}%
\pgfpathcurveto{\pgfqpoint{1.567825in}{1.142728in}}{\pgfqpoint{1.572215in}{1.153327in}}{\pgfqpoint{1.572215in}{1.164378in}}%
\pgfpathcurveto{\pgfqpoint{1.572215in}{1.175428in}}{\pgfqpoint{1.567825in}{1.186027in}}{\pgfqpoint{1.560011in}{1.193840in}}%
\pgfpathcurveto{\pgfqpoint{1.552197in}{1.201654in}}{\pgfqpoint{1.541598in}{1.206044in}}{\pgfqpoint{1.530548in}{1.206044in}}%
\pgfpathcurveto{\pgfqpoint{1.519498in}{1.206044in}}{\pgfqpoint{1.508899in}{1.201654in}}{\pgfqpoint{1.501085in}{1.193840in}}%
\pgfpathcurveto{\pgfqpoint{1.493272in}{1.186027in}}{\pgfqpoint{1.488881in}{1.175428in}}{\pgfqpoint{1.488881in}{1.164378in}}%
\pgfpathcurveto{\pgfqpoint{1.488881in}{1.153327in}}{\pgfqpoint{1.493272in}{1.142728in}}{\pgfqpoint{1.501085in}{1.134915in}}%
\pgfpathcurveto{\pgfqpoint{1.508899in}{1.127101in}}{\pgfqpoint{1.519498in}{1.122711in}}{\pgfqpoint{1.530548in}{1.122711in}}%
\pgfpathclose%
\pgfusepath{stroke,fill}%
\end{pgfscope}%
\begin{pgfscope}%
\pgfpathrectangle{\pgfqpoint{0.375000in}{0.330000in}}{\pgfqpoint{2.325000in}{2.310000in}}%
\pgfusepath{clip}%
\pgfsetbuttcap%
\pgfsetroundjoin%
\definecolor{currentfill}{rgb}{0.000000,0.000000,0.000000}%
\pgfsetfillcolor{currentfill}%
\pgfsetlinewidth{1.003750pt}%
\definecolor{currentstroke}{rgb}{0.000000,0.000000,0.000000}%
\pgfsetstrokecolor{currentstroke}%
\pgfsetdash{}{0pt}%
\pgfpathmoveto{\pgfqpoint{1.530548in}{1.122711in}}%
\pgfpathcurveto{\pgfqpoint{1.541598in}{1.122711in}}{\pgfqpoint{1.552197in}{1.127101in}}{\pgfqpoint{1.560011in}{1.134915in}}%
\pgfpathcurveto{\pgfqpoint{1.567825in}{1.142728in}}{\pgfqpoint{1.572215in}{1.153327in}}{\pgfqpoint{1.572215in}{1.164378in}}%
\pgfpathcurveto{\pgfqpoint{1.572215in}{1.175428in}}{\pgfqpoint{1.567825in}{1.186027in}}{\pgfqpoint{1.560011in}{1.193840in}}%
\pgfpathcurveto{\pgfqpoint{1.552197in}{1.201654in}}{\pgfqpoint{1.541598in}{1.206044in}}{\pgfqpoint{1.530548in}{1.206044in}}%
\pgfpathcurveto{\pgfqpoint{1.519498in}{1.206044in}}{\pgfqpoint{1.508899in}{1.201654in}}{\pgfqpoint{1.501085in}{1.193840in}}%
\pgfpathcurveto{\pgfqpoint{1.493272in}{1.186027in}}{\pgfqpoint{1.488881in}{1.175428in}}{\pgfqpoint{1.488881in}{1.164378in}}%
\pgfpathcurveto{\pgfqpoint{1.488881in}{1.153327in}}{\pgfqpoint{1.493272in}{1.142728in}}{\pgfqpoint{1.501085in}{1.134915in}}%
\pgfpathcurveto{\pgfqpoint{1.508899in}{1.127101in}}{\pgfqpoint{1.519498in}{1.122711in}}{\pgfqpoint{1.530548in}{1.122711in}}%
\pgfpathclose%
\pgfusepath{stroke,fill}%
\end{pgfscope}%
\begin{pgfscope}%
\pgfpathrectangle{\pgfqpoint{0.375000in}{0.330000in}}{\pgfqpoint{2.325000in}{2.310000in}}%
\pgfusepath{clip}%
\pgfsetbuttcap%
\pgfsetroundjoin%
\definecolor{currentfill}{rgb}{0.000000,0.000000,0.000000}%
\pgfsetfillcolor{currentfill}%
\pgfsetlinewidth{1.003750pt}%
\definecolor{currentstroke}{rgb}{0.000000,0.000000,0.000000}%
\pgfsetstrokecolor{currentstroke}%
\pgfsetdash{}{0pt}%
\pgfpathmoveto{\pgfqpoint{1.530548in}{1.070680in}}%
\pgfpathcurveto{\pgfqpoint{1.541598in}{1.070680in}}{\pgfqpoint{1.552197in}{1.075070in}}{\pgfqpoint{1.560011in}{1.082884in}}%
\pgfpathcurveto{\pgfqpoint{1.567825in}{1.090698in}}{\pgfqpoint{1.572215in}{1.101297in}}{\pgfqpoint{1.572215in}{1.112347in}}%
\pgfpathcurveto{\pgfqpoint{1.572215in}{1.123397in}}{\pgfqpoint{1.567825in}{1.133996in}}{\pgfqpoint{1.560011in}{1.141810in}}%
\pgfpathcurveto{\pgfqpoint{1.552197in}{1.149623in}}{\pgfqpoint{1.541598in}{1.154013in}}{\pgfqpoint{1.530548in}{1.154013in}}%
\pgfpathcurveto{\pgfqpoint{1.519498in}{1.154013in}}{\pgfqpoint{1.508899in}{1.149623in}}{\pgfqpoint{1.501085in}{1.141810in}}%
\pgfpathcurveto{\pgfqpoint{1.493272in}{1.133996in}}{\pgfqpoint{1.488881in}{1.123397in}}{\pgfqpoint{1.488881in}{1.112347in}}%
\pgfpathcurveto{\pgfqpoint{1.488881in}{1.101297in}}{\pgfqpoint{1.493272in}{1.090698in}}{\pgfqpoint{1.501085in}{1.082884in}}%
\pgfpathcurveto{\pgfqpoint{1.508899in}{1.075070in}}{\pgfqpoint{1.519498in}{1.070680in}}{\pgfqpoint{1.530548in}{1.070680in}}%
\pgfpathclose%
\pgfusepath{stroke,fill}%
\end{pgfscope}%
\begin{pgfscope}%
\pgfpathrectangle{\pgfqpoint{0.375000in}{0.330000in}}{\pgfqpoint{2.325000in}{2.310000in}}%
\pgfusepath{clip}%
\pgfsetbuttcap%
\pgfsetroundjoin%
\definecolor{currentfill}{rgb}{0.000000,0.000000,0.000000}%
\pgfsetfillcolor{currentfill}%
\pgfsetlinewidth{1.003750pt}%
\definecolor{currentstroke}{rgb}{0.000000,0.000000,0.000000}%
\pgfsetstrokecolor{currentstroke}%
\pgfsetdash{}{0pt}%
\pgfpathmoveto{\pgfqpoint{1.530548in}{1.070680in}}%
\pgfpathcurveto{\pgfqpoint{1.541598in}{1.070680in}}{\pgfqpoint{1.552197in}{1.075070in}}{\pgfqpoint{1.560011in}{1.082884in}}%
\pgfpathcurveto{\pgfqpoint{1.567825in}{1.090698in}}{\pgfqpoint{1.572215in}{1.101297in}}{\pgfqpoint{1.572215in}{1.112347in}}%
\pgfpathcurveto{\pgfqpoint{1.572215in}{1.123397in}}{\pgfqpoint{1.567825in}{1.133996in}}{\pgfqpoint{1.560011in}{1.141810in}}%
\pgfpathcurveto{\pgfqpoint{1.552197in}{1.149623in}}{\pgfqpoint{1.541598in}{1.154013in}}{\pgfqpoint{1.530548in}{1.154013in}}%
\pgfpathcurveto{\pgfqpoint{1.519498in}{1.154013in}}{\pgfqpoint{1.508899in}{1.149623in}}{\pgfqpoint{1.501085in}{1.141810in}}%
\pgfpathcurveto{\pgfqpoint{1.493272in}{1.133996in}}{\pgfqpoint{1.488881in}{1.123397in}}{\pgfqpoint{1.488881in}{1.112347in}}%
\pgfpathcurveto{\pgfqpoint{1.488881in}{1.101297in}}{\pgfqpoint{1.493272in}{1.090698in}}{\pgfqpoint{1.501085in}{1.082884in}}%
\pgfpathcurveto{\pgfqpoint{1.508899in}{1.075070in}}{\pgfqpoint{1.519498in}{1.070680in}}{\pgfqpoint{1.530548in}{1.070680in}}%
\pgfpathclose%
\pgfusepath{stroke,fill}%
\end{pgfscope}%
\begin{pgfscope}%
\pgfpathrectangle{\pgfqpoint{0.375000in}{0.330000in}}{\pgfqpoint{2.325000in}{2.310000in}}%
\pgfusepath{clip}%
\pgfsetbuttcap%
\pgfsetroundjoin%
\definecolor{currentfill}{rgb}{0.000000,0.000000,0.000000}%
\pgfsetfillcolor{currentfill}%
\pgfsetlinewidth{1.003750pt}%
\definecolor{currentstroke}{rgb}{0.000000,0.000000,0.000000}%
\pgfsetstrokecolor{currentstroke}%
\pgfsetdash{}{0pt}%
\pgfpathmoveto{\pgfqpoint{1.530548in}{1.070680in}}%
\pgfpathcurveto{\pgfqpoint{1.541598in}{1.070680in}}{\pgfqpoint{1.552197in}{1.075070in}}{\pgfqpoint{1.560011in}{1.082884in}}%
\pgfpathcurveto{\pgfqpoint{1.567825in}{1.090698in}}{\pgfqpoint{1.572215in}{1.101297in}}{\pgfqpoint{1.572215in}{1.112347in}}%
\pgfpathcurveto{\pgfqpoint{1.572215in}{1.123397in}}{\pgfqpoint{1.567825in}{1.133996in}}{\pgfqpoint{1.560011in}{1.141810in}}%
\pgfpathcurveto{\pgfqpoint{1.552197in}{1.149623in}}{\pgfqpoint{1.541598in}{1.154013in}}{\pgfqpoint{1.530548in}{1.154013in}}%
\pgfpathcurveto{\pgfqpoint{1.519498in}{1.154013in}}{\pgfqpoint{1.508899in}{1.149623in}}{\pgfqpoint{1.501085in}{1.141810in}}%
\pgfpathcurveto{\pgfqpoint{1.493272in}{1.133996in}}{\pgfqpoint{1.488881in}{1.123397in}}{\pgfqpoint{1.488881in}{1.112347in}}%
\pgfpathcurveto{\pgfqpoint{1.488881in}{1.101297in}}{\pgfqpoint{1.493272in}{1.090698in}}{\pgfqpoint{1.501085in}{1.082884in}}%
\pgfpathcurveto{\pgfqpoint{1.508899in}{1.075070in}}{\pgfqpoint{1.519498in}{1.070680in}}{\pgfqpoint{1.530548in}{1.070680in}}%
\pgfpathclose%
\pgfusepath{stroke,fill}%
\end{pgfscope}%
\begin{pgfscope}%
\pgfpathrectangle{\pgfqpoint{0.375000in}{0.330000in}}{\pgfqpoint{2.325000in}{2.310000in}}%
\pgfusepath{clip}%
\pgfsetbuttcap%
\pgfsetroundjoin%
\definecolor{currentfill}{rgb}{0.000000,0.000000,0.000000}%
\pgfsetfillcolor{currentfill}%
\pgfsetlinewidth{1.003750pt}%
\definecolor{currentstroke}{rgb}{0.000000,0.000000,0.000000}%
\pgfsetstrokecolor{currentstroke}%
\pgfsetdash{}{0pt}%
\pgfpathmoveto{\pgfqpoint{1.530548in}{1.070680in}}%
\pgfpathcurveto{\pgfqpoint{1.541598in}{1.070680in}}{\pgfqpoint{1.552197in}{1.075070in}}{\pgfqpoint{1.560011in}{1.082884in}}%
\pgfpathcurveto{\pgfqpoint{1.567825in}{1.090698in}}{\pgfqpoint{1.572215in}{1.101297in}}{\pgfqpoint{1.572215in}{1.112347in}}%
\pgfpathcurveto{\pgfqpoint{1.572215in}{1.123397in}}{\pgfqpoint{1.567825in}{1.133996in}}{\pgfqpoint{1.560011in}{1.141810in}}%
\pgfpathcurveto{\pgfqpoint{1.552197in}{1.149623in}}{\pgfqpoint{1.541598in}{1.154013in}}{\pgfqpoint{1.530548in}{1.154013in}}%
\pgfpathcurveto{\pgfqpoint{1.519498in}{1.154013in}}{\pgfqpoint{1.508899in}{1.149623in}}{\pgfqpoint{1.501085in}{1.141810in}}%
\pgfpathcurveto{\pgfqpoint{1.493272in}{1.133996in}}{\pgfqpoint{1.488881in}{1.123397in}}{\pgfqpoint{1.488881in}{1.112347in}}%
\pgfpathcurveto{\pgfqpoint{1.488881in}{1.101297in}}{\pgfqpoint{1.493272in}{1.090698in}}{\pgfqpoint{1.501085in}{1.082884in}}%
\pgfpathcurveto{\pgfqpoint{1.508899in}{1.075070in}}{\pgfqpoint{1.519498in}{1.070680in}}{\pgfqpoint{1.530548in}{1.070680in}}%
\pgfpathclose%
\pgfusepath{stroke,fill}%
\end{pgfscope}%
\begin{pgfscope}%
\pgfpathrectangle{\pgfqpoint{0.375000in}{0.330000in}}{\pgfqpoint{2.325000in}{2.310000in}}%
\pgfusepath{clip}%
\pgfsetbuttcap%
\pgfsetroundjoin%
\definecolor{currentfill}{rgb}{0.000000,0.000000,0.000000}%
\pgfsetfillcolor{currentfill}%
\pgfsetlinewidth{1.003750pt}%
\definecolor{currentstroke}{rgb}{0.000000,0.000000,0.000000}%
\pgfsetstrokecolor{currentstroke}%
\pgfsetdash{}{0pt}%
\pgfpathmoveto{\pgfqpoint{1.530548in}{1.070680in}}%
\pgfpathcurveto{\pgfqpoint{1.541598in}{1.070680in}}{\pgfqpoint{1.552197in}{1.075070in}}{\pgfqpoint{1.560011in}{1.082884in}}%
\pgfpathcurveto{\pgfqpoint{1.567825in}{1.090698in}}{\pgfqpoint{1.572215in}{1.101297in}}{\pgfqpoint{1.572215in}{1.112347in}}%
\pgfpathcurveto{\pgfqpoint{1.572215in}{1.123397in}}{\pgfqpoint{1.567825in}{1.133996in}}{\pgfqpoint{1.560011in}{1.141810in}}%
\pgfpathcurveto{\pgfqpoint{1.552197in}{1.149623in}}{\pgfqpoint{1.541598in}{1.154013in}}{\pgfqpoint{1.530548in}{1.154013in}}%
\pgfpathcurveto{\pgfqpoint{1.519498in}{1.154013in}}{\pgfqpoint{1.508899in}{1.149623in}}{\pgfqpoint{1.501085in}{1.141810in}}%
\pgfpathcurveto{\pgfqpoint{1.493272in}{1.133996in}}{\pgfqpoint{1.488881in}{1.123397in}}{\pgfqpoint{1.488881in}{1.112347in}}%
\pgfpathcurveto{\pgfqpoint{1.488881in}{1.101297in}}{\pgfqpoint{1.493272in}{1.090698in}}{\pgfqpoint{1.501085in}{1.082884in}}%
\pgfpathcurveto{\pgfqpoint{1.508899in}{1.075070in}}{\pgfqpoint{1.519498in}{1.070680in}}{\pgfqpoint{1.530548in}{1.070680in}}%
\pgfpathclose%
\pgfusepath{stroke,fill}%
\end{pgfscope}%
\begin{pgfscope}%
\pgfpathrectangle{\pgfqpoint{0.375000in}{0.330000in}}{\pgfqpoint{2.325000in}{2.310000in}}%
\pgfusepath{clip}%
\pgfsetbuttcap%
\pgfsetroundjoin%
\definecolor{currentfill}{rgb}{0.000000,0.000000,0.000000}%
\pgfsetfillcolor{currentfill}%
\pgfsetlinewidth{1.003750pt}%
\definecolor{currentstroke}{rgb}{0.000000,0.000000,0.000000}%
\pgfsetstrokecolor{currentstroke}%
\pgfsetdash{}{0pt}%
\pgfpathmoveto{\pgfqpoint{1.530548in}{1.070680in}}%
\pgfpathcurveto{\pgfqpoint{1.541598in}{1.070680in}}{\pgfqpoint{1.552197in}{1.075070in}}{\pgfqpoint{1.560011in}{1.082884in}}%
\pgfpathcurveto{\pgfqpoint{1.567825in}{1.090698in}}{\pgfqpoint{1.572215in}{1.101297in}}{\pgfqpoint{1.572215in}{1.112347in}}%
\pgfpathcurveto{\pgfqpoint{1.572215in}{1.123397in}}{\pgfqpoint{1.567825in}{1.133996in}}{\pgfqpoint{1.560011in}{1.141810in}}%
\pgfpathcurveto{\pgfqpoint{1.552197in}{1.149623in}}{\pgfqpoint{1.541598in}{1.154013in}}{\pgfqpoint{1.530548in}{1.154013in}}%
\pgfpathcurveto{\pgfqpoint{1.519498in}{1.154013in}}{\pgfqpoint{1.508899in}{1.149623in}}{\pgfqpoint{1.501085in}{1.141810in}}%
\pgfpathcurveto{\pgfqpoint{1.493272in}{1.133996in}}{\pgfqpoint{1.488881in}{1.123397in}}{\pgfqpoint{1.488881in}{1.112347in}}%
\pgfpathcurveto{\pgfqpoint{1.488881in}{1.101297in}}{\pgfqpoint{1.493272in}{1.090698in}}{\pgfqpoint{1.501085in}{1.082884in}}%
\pgfpathcurveto{\pgfqpoint{1.508899in}{1.075070in}}{\pgfqpoint{1.519498in}{1.070680in}}{\pgfqpoint{1.530548in}{1.070680in}}%
\pgfpathclose%
\pgfusepath{stroke,fill}%
\end{pgfscope}%
\begin{pgfscope}%
\pgfpathrectangle{\pgfqpoint{0.375000in}{0.330000in}}{\pgfqpoint{2.325000in}{2.310000in}}%
\pgfusepath{clip}%
\pgfsetbuttcap%
\pgfsetroundjoin%
\definecolor{currentfill}{rgb}{0.000000,0.000000,0.000000}%
\pgfsetfillcolor{currentfill}%
\pgfsetlinewidth{1.003750pt}%
\definecolor{currentstroke}{rgb}{0.000000,0.000000,0.000000}%
\pgfsetstrokecolor{currentstroke}%
\pgfsetdash{}{0pt}%
\pgfpathmoveto{\pgfqpoint{1.530548in}{1.018649in}}%
\pgfpathcurveto{\pgfqpoint{1.541598in}{1.018649in}}{\pgfqpoint{1.552197in}{1.023040in}}{\pgfqpoint{1.560011in}{1.030853in}}%
\pgfpathcurveto{\pgfqpoint{1.567825in}{1.038667in}}{\pgfqpoint{1.572215in}{1.049266in}}{\pgfqpoint{1.572215in}{1.060316in}}%
\pgfpathcurveto{\pgfqpoint{1.572215in}{1.071366in}}{\pgfqpoint{1.567825in}{1.081965in}}{\pgfqpoint{1.560011in}{1.089779in}}%
\pgfpathcurveto{\pgfqpoint{1.552197in}{1.097592in}}{\pgfqpoint{1.541598in}{1.101983in}}{\pgfqpoint{1.530548in}{1.101983in}}%
\pgfpathcurveto{\pgfqpoint{1.519498in}{1.101983in}}{\pgfqpoint{1.508899in}{1.097592in}}{\pgfqpoint{1.501085in}{1.089779in}}%
\pgfpathcurveto{\pgfqpoint{1.493272in}{1.081965in}}{\pgfqpoint{1.488881in}{1.071366in}}{\pgfqpoint{1.488881in}{1.060316in}}%
\pgfpathcurveto{\pgfqpoint{1.488881in}{1.049266in}}{\pgfqpoint{1.493272in}{1.038667in}}{\pgfqpoint{1.501085in}{1.030853in}}%
\pgfpathcurveto{\pgfqpoint{1.508899in}{1.023040in}}{\pgfqpoint{1.519498in}{1.018649in}}{\pgfqpoint{1.530548in}{1.018649in}}%
\pgfpathclose%
\pgfusepath{stroke,fill}%
\end{pgfscope}%
\begin{pgfscope}%
\pgfpathrectangle{\pgfqpoint{0.375000in}{0.330000in}}{\pgfqpoint{2.325000in}{2.310000in}}%
\pgfusepath{clip}%
\pgfsetbuttcap%
\pgfsetroundjoin%
\definecolor{currentfill}{rgb}{0.000000,0.000000,0.000000}%
\pgfsetfillcolor{currentfill}%
\pgfsetlinewidth{1.003750pt}%
\definecolor{currentstroke}{rgb}{0.000000,0.000000,0.000000}%
\pgfsetstrokecolor{currentstroke}%
\pgfsetdash{}{0pt}%
\pgfpathmoveto{\pgfqpoint{1.530548in}{1.070680in}}%
\pgfpathcurveto{\pgfqpoint{1.541598in}{1.070680in}}{\pgfqpoint{1.552197in}{1.075070in}}{\pgfqpoint{1.560011in}{1.082884in}}%
\pgfpathcurveto{\pgfqpoint{1.567825in}{1.090698in}}{\pgfqpoint{1.572215in}{1.101297in}}{\pgfqpoint{1.572215in}{1.112347in}}%
\pgfpathcurveto{\pgfqpoint{1.572215in}{1.123397in}}{\pgfqpoint{1.567825in}{1.133996in}}{\pgfqpoint{1.560011in}{1.141810in}}%
\pgfpathcurveto{\pgfqpoint{1.552197in}{1.149623in}}{\pgfqpoint{1.541598in}{1.154013in}}{\pgfqpoint{1.530548in}{1.154013in}}%
\pgfpathcurveto{\pgfqpoint{1.519498in}{1.154013in}}{\pgfqpoint{1.508899in}{1.149623in}}{\pgfqpoint{1.501085in}{1.141810in}}%
\pgfpathcurveto{\pgfqpoint{1.493272in}{1.133996in}}{\pgfqpoint{1.488881in}{1.123397in}}{\pgfqpoint{1.488881in}{1.112347in}}%
\pgfpathcurveto{\pgfqpoint{1.488881in}{1.101297in}}{\pgfqpoint{1.493272in}{1.090698in}}{\pgfqpoint{1.501085in}{1.082884in}}%
\pgfpathcurveto{\pgfqpoint{1.508899in}{1.075070in}}{\pgfqpoint{1.519498in}{1.070680in}}{\pgfqpoint{1.530548in}{1.070680in}}%
\pgfpathclose%
\pgfusepath{stroke,fill}%
\end{pgfscope}%
\begin{pgfscope}%
\pgfpathrectangle{\pgfqpoint{0.375000in}{0.330000in}}{\pgfqpoint{2.325000in}{2.310000in}}%
\pgfusepath{clip}%
\pgfsetbuttcap%
\pgfsetroundjoin%
\definecolor{currentfill}{rgb}{0.000000,0.000000,0.000000}%
\pgfsetfillcolor{currentfill}%
\pgfsetlinewidth{1.003750pt}%
\definecolor{currentstroke}{rgb}{0.000000,0.000000,0.000000}%
\pgfsetstrokecolor{currentstroke}%
\pgfsetdash{}{0pt}%
\pgfpathmoveto{\pgfqpoint{1.530548in}{1.070680in}}%
\pgfpathcurveto{\pgfqpoint{1.541598in}{1.070680in}}{\pgfqpoint{1.552197in}{1.075070in}}{\pgfqpoint{1.560011in}{1.082884in}}%
\pgfpathcurveto{\pgfqpoint{1.567825in}{1.090698in}}{\pgfqpoint{1.572215in}{1.101297in}}{\pgfqpoint{1.572215in}{1.112347in}}%
\pgfpathcurveto{\pgfqpoint{1.572215in}{1.123397in}}{\pgfqpoint{1.567825in}{1.133996in}}{\pgfqpoint{1.560011in}{1.141810in}}%
\pgfpathcurveto{\pgfqpoint{1.552197in}{1.149623in}}{\pgfqpoint{1.541598in}{1.154013in}}{\pgfqpoint{1.530548in}{1.154013in}}%
\pgfpathcurveto{\pgfqpoint{1.519498in}{1.154013in}}{\pgfqpoint{1.508899in}{1.149623in}}{\pgfqpoint{1.501085in}{1.141810in}}%
\pgfpathcurveto{\pgfqpoint{1.493272in}{1.133996in}}{\pgfqpoint{1.488881in}{1.123397in}}{\pgfqpoint{1.488881in}{1.112347in}}%
\pgfpathcurveto{\pgfqpoint{1.488881in}{1.101297in}}{\pgfqpoint{1.493272in}{1.090698in}}{\pgfqpoint{1.501085in}{1.082884in}}%
\pgfpathcurveto{\pgfqpoint{1.508899in}{1.075070in}}{\pgfqpoint{1.519498in}{1.070680in}}{\pgfqpoint{1.530548in}{1.070680in}}%
\pgfpathclose%
\pgfusepath{stroke,fill}%
\end{pgfscope}%
\begin{pgfscope}%
\pgfpathrectangle{\pgfqpoint{0.375000in}{0.330000in}}{\pgfqpoint{2.325000in}{2.310000in}}%
\pgfusepath{clip}%
\pgfsetbuttcap%
\pgfsetroundjoin%
\definecolor{currentfill}{rgb}{0.000000,0.000000,0.000000}%
\pgfsetfillcolor{currentfill}%
\pgfsetlinewidth{1.003750pt}%
\definecolor{currentstroke}{rgb}{0.000000,0.000000,0.000000}%
\pgfsetstrokecolor{currentstroke}%
\pgfsetdash{}{0pt}%
\pgfpathmoveto{\pgfqpoint{1.530548in}{1.018649in}}%
\pgfpathcurveto{\pgfqpoint{1.541598in}{1.018649in}}{\pgfqpoint{1.552197in}{1.023040in}}{\pgfqpoint{1.560011in}{1.030853in}}%
\pgfpathcurveto{\pgfqpoint{1.567825in}{1.038667in}}{\pgfqpoint{1.572215in}{1.049266in}}{\pgfqpoint{1.572215in}{1.060316in}}%
\pgfpathcurveto{\pgfqpoint{1.572215in}{1.071366in}}{\pgfqpoint{1.567825in}{1.081965in}}{\pgfqpoint{1.560011in}{1.089779in}}%
\pgfpathcurveto{\pgfqpoint{1.552197in}{1.097592in}}{\pgfqpoint{1.541598in}{1.101983in}}{\pgfqpoint{1.530548in}{1.101983in}}%
\pgfpathcurveto{\pgfqpoint{1.519498in}{1.101983in}}{\pgfqpoint{1.508899in}{1.097592in}}{\pgfqpoint{1.501085in}{1.089779in}}%
\pgfpathcurveto{\pgfqpoint{1.493272in}{1.081965in}}{\pgfqpoint{1.488881in}{1.071366in}}{\pgfqpoint{1.488881in}{1.060316in}}%
\pgfpathcurveto{\pgfqpoint{1.488881in}{1.049266in}}{\pgfqpoint{1.493272in}{1.038667in}}{\pgfqpoint{1.501085in}{1.030853in}}%
\pgfpathcurveto{\pgfqpoint{1.508899in}{1.023040in}}{\pgfqpoint{1.519498in}{1.018649in}}{\pgfqpoint{1.530548in}{1.018649in}}%
\pgfpathclose%
\pgfusepath{stroke,fill}%
\end{pgfscope}%
\begin{pgfscope}%
\pgfpathrectangle{\pgfqpoint{0.375000in}{0.330000in}}{\pgfqpoint{2.325000in}{2.310000in}}%
\pgfusepath{clip}%
\pgfsetbuttcap%
\pgfsetroundjoin%
\definecolor{currentfill}{rgb}{0.000000,0.000000,0.000000}%
\pgfsetfillcolor{currentfill}%
\pgfsetlinewidth{1.003750pt}%
\definecolor{currentstroke}{rgb}{0.000000,0.000000,0.000000}%
\pgfsetstrokecolor{currentstroke}%
\pgfsetdash{}{0pt}%
\pgfpathmoveto{\pgfqpoint{1.530548in}{1.018649in}}%
\pgfpathcurveto{\pgfqpoint{1.541598in}{1.018649in}}{\pgfqpoint{1.552197in}{1.023040in}}{\pgfqpoint{1.560011in}{1.030853in}}%
\pgfpathcurveto{\pgfqpoint{1.567825in}{1.038667in}}{\pgfqpoint{1.572215in}{1.049266in}}{\pgfqpoint{1.572215in}{1.060316in}}%
\pgfpathcurveto{\pgfqpoint{1.572215in}{1.071366in}}{\pgfqpoint{1.567825in}{1.081965in}}{\pgfqpoint{1.560011in}{1.089779in}}%
\pgfpathcurveto{\pgfqpoint{1.552197in}{1.097592in}}{\pgfqpoint{1.541598in}{1.101983in}}{\pgfqpoint{1.530548in}{1.101983in}}%
\pgfpathcurveto{\pgfqpoint{1.519498in}{1.101983in}}{\pgfqpoint{1.508899in}{1.097592in}}{\pgfqpoint{1.501085in}{1.089779in}}%
\pgfpathcurveto{\pgfqpoint{1.493272in}{1.081965in}}{\pgfqpoint{1.488881in}{1.071366in}}{\pgfqpoint{1.488881in}{1.060316in}}%
\pgfpathcurveto{\pgfqpoint{1.488881in}{1.049266in}}{\pgfqpoint{1.493272in}{1.038667in}}{\pgfqpoint{1.501085in}{1.030853in}}%
\pgfpathcurveto{\pgfqpoint{1.508899in}{1.023040in}}{\pgfqpoint{1.519498in}{1.018649in}}{\pgfqpoint{1.530548in}{1.018649in}}%
\pgfpathclose%
\pgfusepath{stroke,fill}%
\end{pgfscope}%
\begin{pgfscope}%
\pgfpathrectangle{\pgfqpoint{0.375000in}{0.330000in}}{\pgfqpoint{2.325000in}{2.310000in}}%
\pgfusepath{clip}%
\pgfsetbuttcap%
\pgfsetroundjoin%
\definecolor{currentfill}{rgb}{0.000000,0.000000,0.000000}%
\pgfsetfillcolor{currentfill}%
\pgfsetlinewidth{1.003750pt}%
\definecolor{currentstroke}{rgb}{0.000000,0.000000,0.000000}%
\pgfsetstrokecolor{currentstroke}%
\pgfsetdash{}{0pt}%
\pgfpathmoveto{\pgfqpoint{1.530548in}{1.174742in}}%
\pgfpathcurveto{\pgfqpoint{1.541598in}{1.174742in}}{\pgfqpoint{1.552197in}{1.179132in}}{\pgfqpoint{1.560011in}{1.186946in}}%
\pgfpathcurveto{\pgfqpoint{1.567825in}{1.194759in}}{\pgfqpoint{1.572215in}{1.205358in}}{\pgfqpoint{1.572215in}{1.216408in}}%
\pgfpathcurveto{\pgfqpoint{1.572215in}{1.227459in}}{\pgfqpoint{1.567825in}{1.238058in}}{\pgfqpoint{1.560011in}{1.245871in}}%
\pgfpathcurveto{\pgfqpoint{1.552197in}{1.253685in}}{\pgfqpoint{1.541598in}{1.258075in}}{\pgfqpoint{1.530548in}{1.258075in}}%
\pgfpathcurveto{\pgfqpoint{1.519498in}{1.258075in}}{\pgfqpoint{1.508899in}{1.253685in}}{\pgfqpoint{1.501085in}{1.245871in}}%
\pgfpathcurveto{\pgfqpoint{1.493272in}{1.238058in}}{\pgfqpoint{1.488881in}{1.227459in}}{\pgfqpoint{1.488881in}{1.216408in}}%
\pgfpathcurveto{\pgfqpoint{1.488881in}{1.205358in}}{\pgfqpoint{1.493272in}{1.194759in}}{\pgfqpoint{1.501085in}{1.186946in}}%
\pgfpathcurveto{\pgfqpoint{1.508899in}{1.179132in}}{\pgfqpoint{1.519498in}{1.174742in}}{\pgfqpoint{1.530548in}{1.174742in}}%
\pgfpathclose%
\pgfusepath{stroke,fill}%
\end{pgfscope}%
\begin{pgfscope}%
\pgfpathrectangle{\pgfqpoint{0.375000in}{0.330000in}}{\pgfqpoint{2.325000in}{2.310000in}}%
\pgfusepath{clip}%
\pgfsetbuttcap%
\pgfsetroundjoin%
\definecolor{currentfill}{rgb}{0.000000,0.000000,0.000000}%
\pgfsetfillcolor{currentfill}%
\pgfsetlinewidth{1.003750pt}%
\definecolor{currentstroke}{rgb}{0.000000,0.000000,0.000000}%
\pgfsetstrokecolor{currentstroke}%
\pgfsetdash{}{0pt}%
\pgfpathmoveto{\pgfqpoint{1.530548in}{1.070680in}}%
\pgfpathcurveto{\pgfqpoint{1.541598in}{1.070680in}}{\pgfqpoint{1.552197in}{1.075070in}}{\pgfqpoint{1.560011in}{1.082884in}}%
\pgfpathcurveto{\pgfqpoint{1.567825in}{1.090698in}}{\pgfqpoint{1.572215in}{1.101297in}}{\pgfqpoint{1.572215in}{1.112347in}}%
\pgfpathcurveto{\pgfqpoint{1.572215in}{1.123397in}}{\pgfqpoint{1.567825in}{1.133996in}}{\pgfqpoint{1.560011in}{1.141810in}}%
\pgfpathcurveto{\pgfqpoint{1.552197in}{1.149623in}}{\pgfqpoint{1.541598in}{1.154013in}}{\pgfqpoint{1.530548in}{1.154013in}}%
\pgfpathcurveto{\pgfqpoint{1.519498in}{1.154013in}}{\pgfqpoint{1.508899in}{1.149623in}}{\pgfqpoint{1.501085in}{1.141810in}}%
\pgfpathcurveto{\pgfqpoint{1.493272in}{1.133996in}}{\pgfqpoint{1.488881in}{1.123397in}}{\pgfqpoint{1.488881in}{1.112347in}}%
\pgfpathcurveto{\pgfqpoint{1.488881in}{1.101297in}}{\pgfqpoint{1.493272in}{1.090698in}}{\pgfqpoint{1.501085in}{1.082884in}}%
\pgfpathcurveto{\pgfqpoint{1.508899in}{1.075070in}}{\pgfqpoint{1.519498in}{1.070680in}}{\pgfqpoint{1.530548in}{1.070680in}}%
\pgfpathclose%
\pgfusepath{stroke,fill}%
\end{pgfscope}%
\begin{pgfscope}%
\pgfpathrectangle{\pgfqpoint{0.375000in}{0.330000in}}{\pgfqpoint{2.325000in}{2.310000in}}%
\pgfusepath{clip}%
\pgfsetbuttcap%
\pgfsetroundjoin%
\definecolor{currentfill}{rgb}{0.000000,0.000000,0.000000}%
\pgfsetfillcolor{currentfill}%
\pgfsetlinewidth{1.003750pt}%
\definecolor{currentstroke}{rgb}{0.000000,0.000000,0.000000}%
\pgfsetstrokecolor{currentstroke}%
\pgfsetdash{}{0pt}%
\pgfpathmoveto{\pgfqpoint{1.530548in}{1.018649in}}%
\pgfpathcurveto{\pgfqpoint{1.541598in}{1.018649in}}{\pgfqpoint{1.552197in}{1.023040in}}{\pgfqpoint{1.560011in}{1.030853in}}%
\pgfpathcurveto{\pgfqpoint{1.567825in}{1.038667in}}{\pgfqpoint{1.572215in}{1.049266in}}{\pgfqpoint{1.572215in}{1.060316in}}%
\pgfpathcurveto{\pgfqpoint{1.572215in}{1.071366in}}{\pgfqpoint{1.567825in}{1.081965in}}{\pgfqpoint{1.560011in}{1.089779in}}%
\pgfpathcurveto{\pgfqpoint{1.552197in}{1.097592in}}{\pgfqpoint{1.541598in}{1.101983in}}{\pgfqpoint{1.530548in}{1.101983in}}%
\pgfpathcurveto{\pgfqpoint{1.519498in}{1.101983in}}{\pgfqpoint{1.508899in}{1.097592in}}{\pgfqpoint{1.501085in}{1.089779in}}%
\pgfpathcurveto{\pgfqpoint{1.493272in}{1.081965in}}{\pgfqpoint{1.488881in}{1.071366in}}{\pgfqpoint{1.488881in}{1.060316in}}%
\pgfpathcurveto{\pgfqpoint{1.488881in}{1.049266in}}{\pgfqpoint{1.493272in}{1.038667in}}{\pgfqpoint{1.501085in}{1.030853in}}%
\pgfpathcurveto{\pgfqpoint{1.508899in}{1.023040in}}{\pgfqpoint{1.519498in}{1.018649in}}{\pgfqpoint{1.530548in}{1.018649in}}%
\pgfpathclose%
\pgfusepath{stroke,fill}%
\end{pgfscope}%
\begin{pgfscope}%
\pgfpathrectangle{\pgfqpoint{0.375000in}{0.330000in}}{\pgfqpoint{2.325000in}{2.310000in}}%
\pgfusepath{clip}%
\pgfsetbuttcap%
\pgfsetroundjoin%
\definecolor{currentfill}{rgb}{0.000000,0.000000,0.000000}%
\pgfsetfillcolor{currentfill}%
\pgfsetlinewidth{1.003750pt}%
\definecolor{currentstroke}{rgb}{0.000000,0.000000,0.000000}%
\pgfsetstrokecolor{currentstroke}%
\pgfsetdash{}{0pt}%
\pgfpathmoveto{\pgfqpoint{1.530548in}{1.018649in}}%
\pgfpathcurveto{\pgfqpoint{1.541598in}{1.018649in}}{\pgfqpoint{1.552197in}{1.023040in}}{\pgfqpoint{1.560011in}{1.030853in}}%
\pgfpathcurveto{\pgfqpoint{1.567825in}{1.038667in}}{\pgfqpoint{1.572215in}{1.049266in}}{\pgfqpoint{1.572215in}{1.060316in}}%
\pgfpathcurveto{\pgfqpoint{1.572215in}{1.071366in}}{\pgfqpoint{1.567825in}{1.081965in}}{\pgfqpoint{1.560011in}{1.089779in}}%
\pgfpathcurveto{\pgfqpoint{1.552197in}{1.097592in}}{\pgfqpoint{1.541598in}{1.101983in}}{\pgfqpoint{1.530548in}{1.101983in}}%
\pgfpathcurveto{\pgfqpoint{1.519498in}{1.101983in}}{\pgfqpoint{1.508899in}{1.097592in}}{\pgfqpoint{1.501085in}{1.089779in}}%
\pgfpathcurveto{\pgfqpoint{1.493272in}{1.081965in}}{\pgfqpoint{1.488881in}{1.071366in}}{\pgfqpoint{1.488881in}{1.060316in}}%
\pgfpathcurveto{\pgfqpoint{1.488881in}{1.049266in}}{\pgfqpoint{1.493272in}{1.038667in}}{\pgfqpoint{1.501085in}{1.030853in}}%
\pgfpathcurveto{\pgfqpoint{1.508899in}{1.023040in}}{\pgfqpoint{1.519498in}{1.018649in}}{\pgfqpoint{1.530548in}{1.018649in}}%
\pgfpathclose%
\pgfusepath{stroke,fill}%
\end{pgfscope}%
\begin{pgfscope}%
\pgfpathrectangle{\pgfqpoint{0.375000in}{0.330000in}}{\pgfqpoint{2.325000in}{2.310000in}}%
\pgfusepath{clip}%
\pgfsetbuttcap%
\pgfsetroundjoin%
\definecolor{currentfill}{rgb}{0.000000,0.000000,0.000000}%
\pgfsetfillcolor{currentfill}%
\pgfsetlinewidth{1.003750pt}%
\definecolor{currentstroke}{rgb}{0.000000,0.000000,0.000000}%
\pgfsetstrokecolor{currentstroke}%
\pgfsetdash{}{0pt}%
\pgfpathmoveto{\pgfqpoint{1.530548in}{1.174742in}}%
\pgfpathcurveto{\pgfqpoint{1.541598in}{1.174742in}}{\pgfqpoint{1.552197in}{1.179132in}}{\pgfqpoint{1.560011in}{1.186946in}}%
\pgfpathcurveto{\pgfqpoint{1.567825in}{1.194759in}}{\pgfqpoint{1.572215in}{1.205358in}}{\pgfqpoint{1.572215in}{1.216408in}}%
\pgfpathcurveto{\pgfqpoint{1.572215in}{1.227459in}}{\pgfqpoint{1.567825in}{1.238058in}}{\pgfqpoint{1.560011in}{1.245871in}}%
\pgfpathcurveto{\pgfqpoint{1.552197in}{1.253685in}}{\pgfqpoint{1.541598in}{1.258075in}}{\pgfqpoint{1.530548in}{1.258075in}}%
\pgfpathcurveto{\pgfqpoint{1.519498in}{1.258075in}}{\pgfqpoint{1.508899in}{1.253685in}}{\pgfqpoint{1.501085in}{1.245871in}}%
\pgfpathcurveto{\pgfqpoint{1.493272in}{1.238058in}}{\pgfqpoint{1.488881in}{1.227459in}}{\pgfqpoint{1.488881in}{1.216408in}}%
\pgfpathcurveto{\pgfqpoint{1.488881in}{1.205358in}}{\pgfqpoint{1.493272in}{1.194759in}}{\pgfqpoint{1.501085in}{1.186946in}}%
\pgfpathcurveto{\pgfqpoint{1.508899in}{1.179132in}}{\pgfqpoint{1.519498in}{1.174742in}}{\pgfqpoint{1.530548in}{1.174742in}}%
\pgfpathclose%
\pgfusepath{stroke,fill}%
\end{pgfscope}%
\begin{pgfscope}%
\pgfpathrectangle{\pgfqpoint{0.375000in}{0.330000in}}{\pgfqpoint{2.325000in}{2.310000in}}%
\pgfusepath{clip}%
\pgfsetbuttcap%
\pgfsetroundjoin%
\definecolor{currentfill}{rgb}{0.000000,0.000000,0.000000}%
\pgfsetfillcolor{currentfill}%
\pgfsetlinewidth{1.003750pt}%
\definecolor{currentstroke}{rgb}{0.000000,0.000000,0.000000}%
\pgfsetstrokecolor{currentstroke}%
\pgfsetdash{}{0pt}%
\pgfpathmoveto{\pgfqpoint{1.530548in}{1.018649in}}%
\pgfpathcurveto{\pgfqpoint{1.541598in}{1.018649in}}{\pgfqpoint{1.552197in}{1.023040in}}{\pgfqpoint{1.560011in}{1.030853in}}%
\pgfpathcurveto{\pgfqpoint{1.567825in}{1.038667in}}{\pgfqpoint{1.572215in}{1.049266in}}{\pgfqpoint{1.572215in}{1.060316in}}%
\pgfpathcurveto{\pgfqpoint{1.572215in}{1.071366in}}{\pgfqpoint{1.567825in}{1.081965in}}{\pgfqpoint{1.560011in}{1.089779in}}%
\pgfpathcurveto{\pgfqpoint{1.552197in}{1.097592in}}{\pgfqpoint{1.541598in}{1.101983in}}{\pgfqpoint{1.530548in}{1.101983in}}%
\pgfpathcurveto{\pgfqpoint{1.519498in}{1.101983in}}{\pgfqpoint{1.508899in}{1.097592in}}{\pgfqpoint{1.501085in}{1.089779in}}%
\pgfpathcurveto{\pgfqpoint{1.493272in}{1.081965in}}{\pgfqpoint{1.488881in}{1.071366in}}{\pgfqpoint{1.488881in}{1.060316in}}%
\pgfpathcurveto{\pgfqpoint{1.488881in}{1.049266in}}{\pgfqpoint{1.493272in}{1.038667in}}{\pgfqpoint{1.501085in}{1.030853in}}%
\pgfpathcurveto{\pgfqpoint{1.508899in}{1.023040in}}{\pgfqpoint{1.519498in}{1.018649in}}{\pgfqpoint{1.530548in}{1.018649in}}%
\pgfpathclose%
\pgfusepath{stroke,fill}%
\end{pgfscope}%
\begin{pgfscope}%
\pgfpathrectangle{\pgfqpoint{0.375000in}{0.330000in}}{\pgfqpoint{2.325000in}{2.310000in}}%
\pgfusepath{clip}%
\pgfsetbuttcap%
\pgfsetroundjoin%
\definecolor{currentfill}{rgb}{0.000000,0.000000,0.000000}%
\pgfsetfillcolor{currentfill}%
\pgfsetlinewidth{1.003750pt}%
\definecolor{currentstroke}{rgb}{0.000000,0.000000,0.000000}%
\pgfsetstrokecolor{currentstroke}%
\pgfsetdash{}{0pt}%
\pgfpathmoveto{\pgfqpoint{1.530548in}{1.122711in}}%
\pgfpathcurveto{\pgfqpoint{1.541598in}{1.122711in}}{\pgfqpoint{1.552197in}{1.127101in}}{\pgfqpoint{1.560011in}{1.134915in}}%
\pgfpathcurveto{\pgfqpoint{1.567825in}{1.142728in}}{\pgfqpoint{1.572215in}{1.153327in}}{\pgfqpoint{1.572215in}{1.164378in}}%
\pgfpathcurveto{\pgfqpoint{1.572215in}{1.175428in}}{\pgfqpoint{1.567825in}{1.186027in}}{\pgfqpoint{1.560011in}{1.193840in}}%
\pgfpathcurveto{\pgfqpoint{1.552197in}{1.201654in}}{\pgfqpoint{1.541598in}{1.206044in}}{\pgfqpoint{1.530548in}{1.206044in}}%
\pgfpathcurveto{\pgfqpoint{1.519498in}{1.206044in}}{\pgfqpoint{1.508899in}{1.201654in}}{\pgfqpoint{1.501085in}{1.193840in}}%
\pgfpathcurveto{\pgfqpoint{1.493272in}{1.186027in}}{\pgfqpoint{1.488881in}{1.175428in}}{\pgfqpoint{1.488881in}{1.164378in}}%
\pgfpathcurveto{\pgfqpoint{1.488881in}{1.153327in}}{\pgfqpoint{1.493272in}{1.142728in}}{\pgfqpoint{1.501085in}{1.134915in}}%
\pgfpathcurveto{\pgfqpoint{1.508899in}{1.127101in}}{\pgfqpoint{1.519498in}{1.122711in}}{\pgfqpoint{1.530548in}{1.122711in}}%
\pgfpathclose%
\pgfusepath{stroke,fill}%
\end{pgfscope}%
\begin{pgfscope}%
\pgfpathrectangle{\pgfqpoint{0.375000in}{0.330000in}}{\pgfqpoint{2.325000in}{2.310000in}}%
\pgfusepath{clip}%
\pgfsetbuttcap%
\pgfsetroundjoin%
\definecolor{currentfill}{rgb}{0.000000,0.000000,0.000000}%
\pgfsetfillcolor{currentfill}%
\pgfsetlinewidth{1.003750pt}%
\definecolor{currentstroke}{rgb}{0.000000,0.000000,0.000000}%
\pgfsetstrokecolor{currentstroke}%
\pgfsetdash{}{0pt}%
\pgfpathmoveto{\pgfqpoint{1.530548in}{1.070680in}}%
\pgfpathcurveto{\pgfqpoint{1.541598in}{1.070680in}}{\pgfqpoint{1.552197in}{1.075070in}}{\pgfqpoint{1.560011in}{1.082884in}}%
\pgfpathcurveto{\pgfqpoint{1.567825in}{1.090698in}}{\pgfqpoint{1.572215in}{1.101297in}}{\pgfqpoint{1.572215in}{1.112347in}}%
\pgfpathcurveto{\pgfqpoint{1.572215in}{1.123397in}}{\pgfqpoint{1.567825in}{1.133996in}}{\pgfqpoint{1.560011in}{1.141810in}}%
\pgfpathcurveto{\pgfqpoint{1.552197in}{1.149623in}}{\pgfqpoint{1.541598in}{1.154013in}}{\pgfqpoint{1.530548in}{1.154013in}}%
\pgfpathcurveto{\pgfqpoint{1.519498in}{1.154013in}}{\pgfqpoint{1.508899in}{1.149623in}}{\pgfqpoint{1.501085in}{1.141810in}}%
\pgfpathcurveto{\pgfqpoint{1.493272in}{1.133996in}}{\pgfqpoint{1.488881in}{1.123397in}}{\pgfqpoint{1.488881in}{1.112347in}}%
\pgfpathcurveto{\pgfqpoint{1.488881in}{1.101297in}}{\pgfqpoint{1.493272in}{1.090698in}}{\pgfqpoint{1.501085in}{1.082884in}}%
\pgfpathcurveto{\pgfqpoint{1.508899in}{1.075070in}}{\pgfqpoint{1.519498in}{1.070680in}}{\pgfqpoint{1.530548in}{1.070680in}}%
\pgfpathclose%
\pgfusepath{stroke,fill}%
\end{pgfscope}%
\begin{pgfscope}%
\pgfpathrectangle{\pgfqpoint{0.375000in}{0.330000in}}{\pgfqpoint{2.325000in}{2.310000in}}%
\pgfusepath{clip}%
\pgfsetbuttcap%
\pgfsetroundjoin%
\definecolor{currentfill}{rgb}{0.000000,0.000000,0.000000}%
\pgfsetfillcolor{currentfill}%
\pgfsetlinewidth{1.003750pt}%
\definecolor{currentstroke}{rgb}{0.000000,0.000000,0.000000}%
\pgfsetstrokecolor{currentstroke}%
\pgfsetdash{}{0pt}%
\pgfpathmoveto{\pgfqpoint{1.530548in}{1.018649in}}%
\pgfpathcurveto{\pgfqpoint{1.541598in}{1.018649in}}{\pgfqpoint{1.552197in}{1.023040in}}{\pgfqpoint{1.560011in}{1.030853in}}%
\pgfpathcurveto{\pgfqpoint{1.567825in}{1.038667in}}{\pgfqpoint{1.572215in}{1.049266in}}{\pgfqpoint{1.572215in}{1.060316in}}%
\pgfpathcurveto{\pgfqpoint{1.572215in}{1.071366in}}{\pgfqpoint{1.567825in}{1.081965in}}{\pgfqpoint{1.560011in}{1.089779in}}%
\pgfpathcurveto{\pgfqpoint{1.552197in}{1.097592in}}{\pgfqpoint{1.541598in}{1.101983in}}{\pgfqpoint{1.530548in}{1.101983in}}%
\pgfpathcurveto{\pgfqpoint{1.519498in}{1.101983in}}{\pgfqpoint{1.508899in}{1.097592in}}{\pgfqpoint{1.501085in}{1.089779in}}%
\pgfpathcurveto{\pgfqpoint{1.493272in}{1.081965in}}{\pgfqpoint{1.488881in}{1.071366in}}{\pgfqpoint{1.488881in}{1.060316in}}%
\pgfpathcurveto{\pgfqpoint{1.488881in}{1.049266in}}{\pgfqpoint{1.493272in}{1.038667in}}{\pgfqpoint{1.501085in}{1.030853in}}%
\pgfpathcurveto{\pgfqpoint{1.508899in}{1.023040in}}{\pgfqpoint{1.519498in}{1.018649in}}{\pgfqpoint{1.530548in}{1.018649in}}%
\pgfpathclose%
\pgfusepath{stroke,fill}%
\end{pgfscope}%
\begin{pgfscope}%
\pgfpathrectangle{\pgfqpoint{0.375000in}{0.330000in}}{\pgfqpoint{2.325000in}{2.310000in}}%
\pgfusepath{clip}%
\pgfsetbuttcap%
\pgfsetroundjoin%
\definecolor{currentfill}{rgb}{0.000000,0.000000,0.000000}%
\pgfsetfillcolor{currentfill}%
\pgfsetlinewidth{1.003750pt}%
\definecolor{currentstroke}{rgb}{0.000000,0.000000,0.000000}%
\pgfsetstrokecolor{currentstroke}%
\pgfsetdash{}{0pt}%
\pgfpathmoveto{\pgfqpoint{1.530548in}{1.070680in}}%
\pgfpathcurveto{\pgfqpoint{1.541598in}{1.070680in}}{\pgfqpoint{1.552197in}{1.075070in}}{\pgfqpoint{1.560011in}{1.082884in}}%
\pgfpathcurveto{\pgfqpoint{1.567825in}{1.090698in}}{\pgfqpoint{1.572215in}{1.101297in}}{\pgfqpoint{1.572215in}{1.112347in}}%
\pgfpathcurveto{\pgfqpoint{1.572215in}{1.123397in}}{\pgfqpoint{1.567825in}{1.133996in}}{\pgfqpoint{1.560011in}{1.141810in}}%
\pgfpathcurveto{\pgfqpoint{1.552197in}{1.149623in}}{\pgfqpoint{1.541598in}{1.154013in}}{\pgfqpoint{1.530548in}{1.154013in}}%
\pgfpathcurveto{\pgfqpoint{1.519498in}{1.154013in}}{\pgfqpoint{1.508899in}{1.149623in}}{\pgfqpoint{1.501085in}{1.141810in}}%
\pgfpathcurveto{\pgfqpoint{1.493272in}{1.133996in}}{\pgfqpoint{1.488881in}{1.123397in}}{\pgfqpoint{1.488881in}{1.112347in}}%
\pgfpathcurveto{\pgfqpoint{1.488881in}{1.101297in}}{\pgfqpoint{1.493272in}{1.090698in}}{\pgfqpoint{1.501085in}{1.082884in}}%
\pgfpathcurveto{\pgfqpoint{1.508899in}{1.075070in}}{\pgfqpoint{1.519498in}{1.070680in}}{\pgfqpoint{1.530548in}{1.070680in}}%
\pgfpathclose%
\pgfusepath{stroke,fill}%
\end{pgfscope}%
\begin{pgfscope}%
\pgfpathrectangle{\pgfqpoint{0.375000in}{0.330000in}}{\pgfqpoint{2.325000in}{2.310000in}}%
\pgfusepath{clip}%
\pgfsetbuttcap%
\pgfsetroundjoin%
\definecolor{currentfill}{rgb}{0.000000,0.000000,0.000000}%
\pgfsetfillcolor{currentfill}%
\pgfsetlinewidth{1.003750pt}%
\definecolor{currentstroke}{rgb}{0.000000,0.000000,0.000000}%
\pgfsetstrokecolor{currentstroke}%
\pgfsetdash{}{0pt}%
\pgfpathmoveto{\pgfqpoint{1.530548in}{1.122711in}}%
\pgfpathcurveto{\pgfqpoint{1.541598in}{1.122711in}}{\pgfqpoint{1.552197in}{1.127101in}}{\pgfqpoint{1.560011in}{1.134915in}}%
\pgfpathcurveto{\pgfqpoint{1.567825in}{1.142728in}}{\pgfqpoint{1.572215in}{1.153327in}}{\pgfqpoint{1.572215in}{1.164378in}}%
\pgfpathcurveto{\pgfqpoint{1.572215in}{1.175428in}}{\pgfqpoint{1.567825in}{1.186027in}}{\pgfqpoint{1.560011in}{1.193840in}}%
\pgfpathcurveto{\pgfqpoint{1.552197in}{1.201654in}}{\pgfqpoint{1.541598in}{1.206044in}}{\pgfqpoint{1.530548in}{1.206044in}}%
\pgfpathcurveto{\pgfqpoint{1.519498in}{1.206044in}}{\pgfqpoint{1.508899in}{1.201654in}}{\pgfqpoint{1.501085in}{1.193840in}}%
\pgfpathcurveto{\pgfqpoint{1.493272in}{1.186027in}}{\pgfqpoint{1.488881in}{1.175428in}}{\pgfqpoint{1.488881in}{1.164378in}}%
\pgfpathcurveto{\pgfqpoint{1.488881in}{1.153327in}}{\pgfqpoint{1.493272in}{1.142728in}}{\pgfqpoint{1.501085in}{1.134915in}}%
\pgfpathcurveto{\pgfqpoint{1.508899in}{1.127101in}}{\pgfqpoint{1.519498in}{1.122711in}}{\pgfqpoint{1.530548in}{1.122711in}}%
\pgfpathclose%
\pgfusepath{stroke,fill}%
\end{pgfscope}%
\begin{pgfscope}%
\pgfpathrectangle{\pgfqpoint{0.375000in}{0.330000in}}{\pgfqpoint{2.325000in}{2.310000in}}%
\pgfusepath{clip}%
\pgfsetbuttcap%
\pgfsetroundjoin%
\definecolor{currentfill}{rgb}{0.000000,0.000000,0.000000}%
\pgfsetfillcolor{currentfill}%
\pgfsetlinewidth{1.003750pt}%
\definecolor{currentstroke}{rgb}{0.000000,0.000000,0.000000}%
\pgfsetstrokecolor{currentstroke}%
\pgfsetdash{}{0pt}%
\pgfpathmoveto{\pgfqpoint{1.530548in}{1.070680in}}%
\pgfpathcurveto{\pgfqpoint{1.541598in}{1.070680in}}{\pgfqpoint{1.552197in}{1.075070in}}{\pgfqpoint{1.560011in}{1.082884in}}%
\pgfpathcurveto{\pgfqpoint{1.567825in}{1.090698in}}{\pgfqpoint{1.572215in}{1.101297in}}{\pgfqpoint{1.572215in}{1.112347in}}%
\pgfpathcurveto{\pgfqpoint{1.572215in}{1.123397in}}{\pgfqpoint{1.567825in}{1.133996in}}{\pgfqpoint{1.560011in}{1.141810in}}%
\pgfpathcurveto{\pgfqpoint{1.552197in}{1.149623in}}{\pgfqpoint{1.541598in}{1.154013in}}{\pgfqpoint{1.530548in}{1.154013in}}%
\pgfpathcurveto{\pgfqpoint{1.519498in}{1.154013in}}{\pgfqpoint{1.508899in}{1.149623in}}{\pgfqpoint{1.501085in}{1.141810in}}%
\pgfpathcurveto{\pgfqpoint{1.493272in}{1.133996in}}{\pgfqpoint{1.488881in}{1.123397in}}{\pgfqpoint{1.488881in}{1.112347in}}%
\pgfpathcurveto{\pgfqpoint{1.488881in}{1.101297in}}{\pgfqpoint{1.493272in}{1.090698in}}{\pgfqpoint{1.501085in}{1.082884in}}%
\pgfpathcurveto{\pgfqpoint{1.508899in}{1.075070in}}{\pgfqpoint{1.519498in}{1.070680in}}{\pgfqpoint{1.530548in}{1.070680in}}%
\pgfpathclose%
\pgfusepath{stroke,fill}%
\end{pgfscope}%
\begin{pgfscope}%
\pgfpathrectangle{\pgfqpoint{0.375000in}{0.330000in}}{\pgfqpoint{2.325000in}{2.310000in}}%
\pgfusepath{clip}%
\pgfsetbuttcap%
\pgfsetroundjoin%
\definecolor{currentfill}{rgb}{0.000000,0.000000,0.000000}%
\pgfsetfillcolor{currentfill}%
\pgfsetlinewidth{1.003750pt}%
\definecolor{currentstroke}{rgb}{0.000000,0.000000,0.000000}%
\pgfsetstrokecolor{currentstroke}%
\pgfsetdash{}{0pt}%
\pgfpathmoveto{\pgfqpoint{1.530548in}{1.070680in}}%
\pgfpathcurveto{\pgfqpoint{1.541598in}{1.070680in}}{\pgfqpoint{1.552197in}{1.075070in}}{\pgfqpoint{1.560011in}{1.082884in}}%
\pgfpathcurveto{\pgfqpoint{1.567825in}{1.090698in}}{\pgfqpoint{1.572215in}{1.101297in}}{\pgfqpoint{1.572215in}{1.112347in}}%
\pgfpathcurveto{\pgfqpoint{1.572215in}{1.123397in}}{\pgfqpoint{1.567825in}{1.133996in}}{\pgfqpoint{1.560011in}{1.141810in}}%
\pgfpathcurveto{\pgfqpoint{1.552197in}{1.149623in}}{\pgfqpoint{1.541598in}{1.154013in}}{\pgfqpoint{1.530548in}{1.154013in}}%
\pgfpathcurveto{\pgfqpoint{1.519498in}{1.154013in}}{\pgfqpoint{1.508899in}{1.149623in}}{\pgfqpoint{1.501085in}{1.141810in}}%
\pgfpathcurveto{\pgfqpoint{1.493272in}{1.133996in}}{\pgfqpoint{1.488881in}{1.123397in}}{\pgfqpoint{1.488881in}{1.112347in}}%
\pgfpathcurveto{\pgfqpoint{1.488881in}{1.101297in}}{\pgfqpoint{1.493272in}{1.090698in}}{\pgfqpoint{1.501085in}{1.082884in}}%
\pgfpathcurveto{\pgfqpoint{1.508899in}{1.075070in}}{\pgfqpoint{1.519498in}{1.070680in}}{\pgfqpoint{1.530548in}{1.070680in}}%
\pgfpathclose%
\pgfusepath{stroke,fill}%
\end{pgfscope}%
\begin{pgfscope}%
\pgfpathrectangle{\pgfqpoint{0.375000in}{0.330000in}}{\pgfqpoint{2.325000in}{2.310000in}}%
\pgfusepath{clip}%
\pgfsetbuttcap%
\pgfsetroundjoin%
\definecolor{currentfill}{rgb}{0.000000,0.000000,0.000000}%
\pgfsetfillcolor{currentfill}%
\pgfsetlinewidth{1.003750pt}%
\definecolor{currentstroke}{rgb}{0.000000,0.000000,0.000000}%
\pgfsetstrokecolor{currentstroke}%
\pgfsetdash{}{0pt}%
\pgfpathmoveto{\pgfqpoint{1.530548in}{1.070680in}}%
\pgfpathcurveto{\pgfqpoint{1.541598in}{1.070680in}}{\pgfqpoint{1.552197in}{1.075070in}}{\pgfqpoint{1.560011in}{1.082884in}}%
\pgfpathcurveto{\pgfqpoint{1.567825in}{1.090698in}}{\pgfqpoint{1.572215in}{1.101297in}}{\pgfqpoint{1.572215in}{1.112347in}}%
\pgfpathcurveto{\pgfqpoint{1.572215in}{1.123397in}}{\pgfqpoint{1.567825in}{1.133996in}}{\pgfqpoint{1.560011in}{1.141810in}}%
\pgfpathcurveto{\pgfqpoint{1.552197in}{1.149623in}}{\pgfqpoint{1.541598in}{1.154013in}}{\pgfqpoint{1.530548in}{1.154013in}}%
\pgfpathcurveto{\pgfqpoint{1.519498in}{1.154013in}}{\pgfqpoint{1.508899in}{1.149623in}}{\pgfqpoint{1.501085in}{1.141810in}}%
\pgfpathcurveto{\pgfqpoint{1.493272in}{1.133996in}}{\pgfqpoint{1.488881in}{1.123397in}}{\pgfqpoint{1.488881in}{1.112347in}}%
\pgfpathcurveto{\pgfqpoint{1.488881in}{1.101297in}}{\pgfqpoint{1.493272in}{1.090698in}}{\pgfqpoint{1.501085in}{1.082884in}}%
\pgfpathcurveto{\pgfqpoint{1.508899in}{1.075070in}}{\pgfqpoint{1.519498in}{1.070680in}}{\pgfqpoint{1.530548in}{1.070680in}}%
\pgfpathclose%
\pgfusepath{stroke,fill}%
\end{pgfscope}%
\begin{pgfscope}%
\pgfpathrectangle{\pgfqpoint{0.375000in}{0.330000in}}{\pgfqpoint{2.325000in}{2.310000in}}%
\pgfusepath{clip}%
\pgfsetbuttcap%
\pgfsetroundjoin%
\definecolor{currentfill}{rgb}{0.000000,0.000000,0.000000}%
\pgfsetfillcolor{currentfill}%
\pgfsetlinewidth{1.003750pt}%
\definecolor{currentstroke}{rgb}{0.000000,0.000000,0.000000}%
\pgfsetstrokecolor{currentstroke}%
\pgfsetdash{}{0pt}%
\pgfpathmoveto{\pgfqpoint{1.530548in}{1.070680in}}%
\pgfpathcurveto{\pgfqpoint{1.541598in}{1.070680in}}{\pgfqpoint{1.552197in}{1.075070in}}{\pgfqpoint{1.560011in}{1.082884in}}%
\pgfpathcurveto{\pgfqpoint{1.567825in}{1.090698in}}{\pgfqpoint{1.572215in}{1.101297in}}{\pgfqpoint{1.572215in}{1.112347in}}%
\pgfpathcurveto{\pgfqpoint{1.572215in}{1.123397in}}{\pgfqpoint{1.567825in}{1.133996in}}{\pgfqpoint{1.560011in}{1.141810in}}%
\pgfpathcurveto{\pgfqpoint{1.552197in}{1.149623in}}{\pgfqpoint{1.541598in}{1.154013in}}{\pgfqpoint{1.530548in}{1.154013in}}%
\pgfpathcurveto{\pgfqpoint{1.519498in}{1.154013in}}{\pgfqpoint{1.508899in}{1.149623in}}{\pgfqpoint{1.501085in}{1.141810in}}%
\pgfpathcurveto{\pgfqpoint{1.493272in}{1.133996in}}{\pgfqpoint{1.488881in}{1.123397in}}{\pgfqpoint{1.488881in}{1.112347in}}%
\pgfpathcurveto{\pgfqpoint{1.488881in}{1.101297in}}{\pgfqpoint{1.493272in}{1.090698in}}{\pgfqpoint{1.501085in}{1.082884in}}%
\pgfpathcurveto{\pgfqpoint{1.508899in}{1.075070in}}{\pgfqpoint{1.519498in}{1.070680in}}{\pgfqpoint{1.530548in}{1.070680in}}%
\pgfpathclose%
\pgfusepath{stroke,fill}%
\end{pgfscope}%
\begin{pgfscope}%
\pgfpathrectangle{\pgfqpoint{0.375000in}{0.330000in}}{\pgfqpoint{2.325000in}{2.310000in}}%
\pgfusepath{clip}%
\pgfsetbuttcap%
\pgfsetroundjoin%
\definecolor{currentfill}{rgb}{0.000000,0.000000,0.000000}%
\pgfsetfillcolor{currentfill}%
\pgfsetlinewidth{1.003750pt}%
\definecolor{currentstroke}{rgb}{0.000000,0.000000,0.000000}%
\pgfsetstrokecolor{currentstroke}%
\pgfsetdash{}{0pt}%
\pgfpathmoveto{\pgfqpoint{1.530548in}{1.122711in}}%
\pgfpathcurveto{\pgfqpoint{1.541598in}{1.122711in}}{\pgfqpoint{1.552197in}{1.127101in}}{\pgfqpoint{1.560011in}{1.134915in}}%
\pgfpathcurveto{\pgfqpoint{1.567825in}{1.142728in}}{\pgfqpoint{1.572215in}{1.153327in}}{\pgfqpoint{1.572215in}{1.164378in}}%
\pgfpathcurveto{\pgfqpoint{1.572215in}{1.175428in}}{\pgfqpoint{1.567825in}{1.186027in}}{\pgfqpoint{1.560011in}{1.193840in}}%
\pgfpathcurveto{\pgfqpoint{1.552197in}{1.201654in}}{\pgfqpoint{1.541598in}{1.206044in}}{\pgfqpoint{1.530548in}{1.206044in}}%
\pgfpathcurveto{\pgfqpoint{1.519498in}{1.206044in}}{\pgfqpoint{1.508899in}{1.201654in}}{\pgfqpoint{1.501085in}{1.193840in}}%
\pgfpathcurveto{\pgfqpoint{1.493272in}{1.186027in}}{\pgfqpoint{1.488881in}{1.175428in}}{\pgfqpoint{1.488881in}{1.164378in}}%
\pgfpathcurveto{\pgfqpoint{1.488881in}{1.153327in}}{\pgfqpoint{1.493272in}{1.142728in}}{\pgfqpoint{1.501085in}{1.134915in}}%
\pgfpathcurveto{\pgfqpoint{1.508899in}{1.127101in}}{\pgfqpoint{1.519498in}{1.122711in}}{\pgfqpoint{1.530548in}{1.122711in}}%
\pgfpathclose%
\pgfusepath{stroke,fill}%
\end{pgfscope}%
\begin{pgfscope}%
\pgfpathrectangle{\pgfqpoint{0.375000in}{0.330000in}}{\pgfqpoint{2.325000in}{2.310000in}}%
\pgfusepath{clip}%
\pgfsetbuttcap%
\pgfsetroundjoin%
\definecolor{currentfill}{rgb}{0.000000,0.000000,0.000000}%
\pgfsetfillcolor{currentfill}%
\pgfsetlinewidth{1.003750pt}%
\definecolor{currentstroke}{rgb}{0.000000,0.000000,0.000000}%
\pgfsetstrokecolor{currentstroke}%
\pgfsetdash{}{0pt}%
\pgfpathmoveto{\pgfqpoint{1.530548in}{1.122711in}}%
\pgfpathcurveto{\pgfqpoint{1.541598in}{1.122711in}}{\pgfqpoint{1.552197in}{1.127101in}}{\pgfqpoint{1.560011in}{1.134915in}}%
\pgfpathcurveto{\pgfqpoint{1.567825in}{1.142728in}}{\pgfqpoint{1.572215in}{1.153327in}}{\pgfqpoint{1.572215in}{1.164378in}}%
\pgfpathcurveto{\pgfqpoint{1.572215in}{1.175428in}}{\pgfqpoint{1.567825in}{1.186027in}}{\pgfqpoint{1.560011in}{1.193840in}}%
\pgfpathcurveto{\pgfqpoint{1.552197in}{1.201654in}}{\pgfqpoint{1.541598in}{1.206044in}}{\pgfqpoint{1.530548in}{1.206044in}}%
\pgfpathcurveto{\pgfqpoint{1.519498in}{1.206044in}}{\pgfqpoint{1.508899in}{1.201654in}}{\pgfqpoint{1.501085in}{1.193840in}}%
\pgfpathcurveto{\pgfqpoint{1.493272in}{1.186027in}}{\pgfqpoint{1.488881in}{1.175428in}}{\pgfqpoint{1.488881in}{1.164378in}}%
\pgfpathcurveto{\pgfqpoint{1.488881in}{1.153327in}}{\pgfqpoint{1.493272in}{1.142728in}}{\pgfqpoint{1.501085in}{1.134915in}}%
\pgfpathcurveto{\pgfqpoint{1.508899in}{1.127101in}}{\pgfqpoint{1.519498in}{1.122711in}}{\pgfqpoint{1.530548in}{1.122711in}}%
\pgfpathclose%
\pgfusepath{stroke,fill}%
\end{pgfscope}%
\begin{pgfscope}%
\pgfpathrectangle{\pgfqpoint{0.375000in}{0.330000in}}{\pgfqpoint{2.325000in}{2.310000in}}%
\pgfusepath{clip}%
\pgfsetbuttcap%
\pgfsetroundjoin%
\definecolor{currentfill}{rgb}{0.000000,0.000000,0.000000}%
\pgfsetfillcolor{currentfill}%
\pgfsetlinewidth{1.003750pt}%
\definecolor{currentstroke}{rgb}{0.000000,0.000000,0.000000}%
\pgfsetstrokecolor{currentstroke}%
\pgfsetdash{}{0pt}%
\pgfpathmoveto{\pgfqpoint{1.530548in}{1.070680in}}%
\pgfpathcurveto{\pgfqpoint{1.541598in}{1.070680in}}{\pgfqpoint{1.552197in}{1.075070in}}{\pgfqpoint{1.560011in}{1.082884in}}%
\pgfpathcurveto{\pgfqpoint{1.567825in}{1.090698in}}{\pgfqpoint{1.572215in}{1.101297in}}{\pgfqpoint{1.572215in}{1.112347in}}%
\pgfpathcurveto{\pgfqpoint{1.572215in}{1.123397in}}{\pgfqpoint{1.567825in}{1.133996in}}{\pgfqpoint{1.560011in}{1.141810in}}%
\pgfpathcurveto{\pgfqpoint{1.552197in}{1.149623in}}{\pgfqpoint{1.541598in}{1.154013in}}{\pgfqpoint{1.530548in}{1.154013in}}%
\pgfpathcurveto{\pgfqpoint{1.519498in}{1.154013in}}{\pgfqpoint{1.508899in}{1.149623in}}{\pgfqpoint{1.501085in}{1.141810in}}%
\pgfpathcurveto{\pgfqpoint{1.493272in}{1.133996in}}{\pgfqpoint{1.488881in}{1.123397in}}{\pgfqpoint{1.488881in}{1.112347in}}%
\pgfpathcurveto{\pgfqpoint{1.488881in}{1.101297in}}{\pgfqpoint{1.493272in}{1.090698in}}{\pgfqpoint{1.501085in}{1.082884in}}%
\pgfpathcurveto{\pgfqpoint{1.508899in}{1.075070in}}{\pgfqpoint{1.519498in}{1.070680in}}{\pgfqpoint{1.530548in}{1.070680in}}%
\pgfpathclose%
\pgfusepath{stroke,fill}%
\end{pgfscope}%
\begin{pgfscope}%
\pgfpathrectangle{\pgfqpoint{0.375000in}{0.330000in}}{\pgfqpoint{2.325000in}{2.310000in}}%
\pgfusepath{clip}%
\pgfsetbuttcap%
\pgfsetroundjoin%
\definecolor{currentfill}{rgb}{0.000000,0.000000,0.000000}%
\pgfsetfillcolor{currentfill}%
\pgfsetlinewidth{1.003750pt}%
\definecolor{currentstroke}{rgb}{0.000000,0.000000,0.000000}%
\pgfsetstrokecolor{currentstroke}%
\pgfsetdash{}{0pt}%
\pgfpathmoveto{\pgfqpoint{1.530548in}{1.070680in}}%
\pgfpathcurveto{\pgfqpoint{1.541598in}{1.070680in}}{\pgfqpoint{1.552197in}{1.075070in}}{\pgfqpoint{1.560011in}{1.082884in}}%
\pgfpathcurveto{\pgfqpoint{1.567825in}{1.090698in}}{\pgfqpoint{1.572215in}{1.101297in}}{\pgfqpoint{1.572215in}{1.112347in}}%
\pgfpathcurveto{\pgfqpoint{1.572215in}{1.123397in}}{\pgfqpoint{1.567825in}{1.133996in}}{\pgfqpoint{1.560011in}{1.141810in}}%
\pgfpathcurveto{\pgfqpoint{1.552197in}{1.149623in}}{\pgfqpoint{1.541598in}{1.154013in}}{\pgfqpoint{1.530548in}{1.154013in}}%
\pgfpathcurveto{\pgfqpoint{1.519498in}{1.154013in}}{\pgfqpoint{1.508899in}{1.149623in}}{\pgfqpoint{1.501085in}{1.141810in}}%
\pgfpathcurveto{\pgfqpoint{1.493272in}{1.133996in}}{\pgfqpoint{1.488881in}{1.123397in}}{\pgfqpoint{1.488881in}{1.112347in}}%
\pgfpathcurveto{\pgfqpoint{1.488881in}{1.101297in}}{\pgfqpoint{1.493272in}{1.090698in}}{\pgfqpoint{1.501085in}{1.082884in}}%
\pgfpathcurveto{\pgfqpoint{1.508899in}{1.075070in}}{\pgfqpoint{1.519498in}{1.070680in}}{\pgfqpoint{1.530548in}{1.070680in}}%
\pgfpathclose%
\pgfusepath{stroke,fill}%
\end{pgfscope}%
\begin{pgfscope}%
\pgfpathrectangle{\pgfqpoint{0.375000in}{0.330000in}}{\pgfqpoint{2.325000in}{2.310000in}}%
\pgfusepath{clip}%
\pgfsetbuttcap%
\pgfsetroundjoin%
\definecolor{currentfill}{rgb}{0.000000,0.000000,0.000000}%
\pgfsetfillcolor{currentfill}%
\pgfsetlinewidth{1.003750pt}%
\definecolor{currentstroke}{rgb}{0.000000,0.000000,0.000000}%
\pgfsetstrokecolor{currentstroke}%
\pgfsetdash{}{0pt}%
\pgfpathmoveto{\pgfqpoint{1.530548in}{1.070680in}}%
\pgfpathcurveto{\pgfqpoint{1.541598in}{1.070680in}}{\pgfqpoint{1.552197in}{1.075070in}}{\pgfqpoint{1.560011in}{1.082884in}}%
\pgfpathcurveto{\pgfqpoint{1.567825in}{1.090698in}}{\pgfqpoint{1.572215in}{1.101297in}}{\pgfqpoint{1.572215in}{1.112347in}}%
\pgfpathcurveto{\pgfqpoint{1.572215in}{1.123397in}}{\pgfqpoint{1.567825in}{1.133996in}}{\pgfqpoint{1.560011in}{1.141810in}}%
\pgfpathcurveto{\pgfqpoint{1.552197in}{1.149623in}}{\pgfqpoint{1.541598in}{1.154013in}}{\pgfqpoint{1.530548in}{1.154013in}}%
\pgfpathcurveto{\pgfqpoint{1.519498in}{1.154013in}}{\pgfqpoint{1.508899in}{1.149623in}}{\pgfqpoint{1.501085in}{1.141810in}}%
\pgfpathcurveto{\pgfqpoint{1.493272in}{1.133996in}}{\pgfqpoint{1.488881in}{1.123397in}}{\pgfqpoint{1.488881in}{1.112347in}}%
\pgfpathcurveto{\pgfqpoint{1.488881in}{1.101297in}}{\pgfqpoint{1.493272in}{1.090698in}}{\pgfqpoint{1.501085in}{1.082884in}}%
\pgfpathcurveto{\pgfqpoint{1.508899in}{1.075070in}}{\pgfqpoint{1.519498in}{1.070680in}}{\pgfqpoint{1.530548in}{1.070680in}}%
\pgfpathclose%
\pgfusepath{stroke,fill}%
\end{pgfscope}%
\begin{pgfscope}%
\pgfpathrectangle{\pgfqpoint{0.375000in}{0.330000in}}{\pgfqpoint{2.325000in}{2.310000in}}%
\pgfusepath{clip}%
\pgfsetbuttcap%
\pgfsetroundjoin%
\definecolor{currentfill}{rgb}{0.000000,0.000000,0.000000}%
\pgfsetfillcolor{currentfill}%
\pgfsetlinewidth{1.003750pt}%
\definecolor{currentstroke}{rgb}{0.000000,0.000000,0.000000}%
\pgfsetstrokecolor{currentstroke}%
\pgfsetdash{}{0pt}%
\pgfpathmoveto{\pgfqpoint{1.530548in}{1.018649in}}%
\pgfpathcurveto{\pgfqpoint{1.541598in}{1.018649in}}{\pgfqpoint{1.552197in}{1.023040in}}{\pgfqpoint{1.560011in}{1.030853in}}%
\pgfpathcurveto{\pgfqpoint{1.567825in}{1.038667in}}{\pgfqpoint{1.572215in}{1.049266in}}{\pgfqpoint{1.572215in}{1.060316in}}%
\pgfpathcurveto{\pgfqpoint{1.572215in}{1.071366in}}{\pgfqpoint{1.567825in}{1.081965in}}{\pgfqpoint{1.560011in}{1.089779in}}%
\pgfpathcurveto{\pgfqpoint{1.552197in}{1.097592in}}{\pgfqpoint{1.541598in}{1.101983in}}{\pgfqpoint{1.530548in}{1.101983in}}%
\pgfpathcurveto{\pgfqpoint{1.519498in}{1.101983in}}{\pgfqpoint{1.508899in}{1.097592in}}{\pgfqpoint{1.501085in}{1.089779in}}%
\pgfpathcurveto{\pgfqpoint{1.493272in}{1.081965in}}{\pgfqpoint{1.488881in}{1.071366in}}{\pgfqpoint{1.488881in}{1.060316in}}%
\pgfpathcurveto{\pgfqpoint{1.488881in}{1.049266in}}{\pgfqpoint{1.493272in}{1.038667in}}{\pgfqpoint{1.501085in}{1.030853in}}%
\pgfpathcurveto{\pgfqpoint{1.508899in}{1.023040in}}{\pgfqpoint{1.519498in}{1.018649in}}{\pgfqpoint{1.530548in}{1.018649in}}%
\pgfpathclose%
\pgfusepath{stroke,fill}%
\end{pgfscope}%
\begin{pgfscope}%
\pgfpathrectangle{\pgfqpoint{0.375000in}{0.330000in}}{\pgfqpoint{2.325000in}{2.310000in}}%
\pgfusepath{clip}%
\pgfsetbuttcap%
\pgfsetroundjoin%
\definecolor{currentfill}{rgb}{0.000000,0.000000,0.000000}%
\pgfsetfillcolor{currentfill}%
\pgfsetlinewidth{1.003750pt}%
\definecolor{currentstroke}{rgb}{0.000000,0.000000,0.000000}%
\pgfsetstrokecolor{currentstroke}%
\pgfsetdash{}{0pt}%
\pgfpathmoveto{\pgfqpoint{1.530548in}{1.070680in}}%
\pgfpathcurveto{\pgfqpoint{1.541598in}{1.070680in}}{\pgfqpoint{1.552197in}{1.075070in}}{\pgfqpoint{1.560011in}{1.082884in}}%
\pgfpathcurveto{\pgfqpoint{1.567825in}{1.090698in}}{\pgfqpoint{1.572215in}{1.101297in}}{\pgfqpoint{1.572215in}{1.112347in}}%
\pgfpathcurveto{\pgfqpoint{1.572215in}{1.123397in}}{\pgfqpoint{1.567825in}{1.133996in}}{\pgfqpoint{1.560011in}{1.141810in}}%
\pgfpathcurveto{\pgfqpoint{1.552197in}{1.149623in}}{\pgfqpoint{1.541598in}{1.154013in}}{\pgfqpoint{1.530548in}{1.154013in}}%
\pgfpathcurveto{\pgfqpoint{1.519498in}{1.154013in}}{\pgfqpoint{1.508899in}{1.149623in}}{\pgfqpoint{1.501085in}{1.141810in}}%
\pgfpathcurveto{\pgfqpoint{1.493272in}{1.133996in}}{\pgfqpoint{1.488881in}{1.123397in}}{\pgfqpoint{1.488881in}{1.112347in}}%
\pgfpathcurveto{\pgfqpoint{1.488881in}{1.101297in}}{\pgfqpoint{1.493272in}{1.090698in}}{\pgfqpoint{1.501085in}{1.082884in}}%
\pgfpathcurveto{\pgfqpoint{1.508899in}{1.075070in}}{\pgfqpoint{1.519498in}{1.070680in}}{\pgfqpoint{1.530548in}{1.070680in}}%
\pgfpathclose%
\pgfusepath{stroke,fill}%
\end{pgfscope}%
\begin{pgfscope}%
\pgfpathrectangle{\pgfqpoint{0.375000in}{0.330000in}}{\pgfqpoint{2.325000in}{2.310000in}}%
\pgfusepath{clip}%
\pgfsetbuttcap%
\pgfsetroundjoin%
\definecolor{currentfill}{rgb}{0.000000,0.000000,0.000000}%
\pgfsetfillcolor{currentfill}%
\pgfsetlinewidth{1.003750pt}%
\definecolor{currentstroke}{rgb}{0.000000,0.000000,0.000000}%
\pgfsetstrokecolor{currentstroke}%
\pgfsetdash{}{0pt}%
\pgfpathmoveto{\pgfqpoint{1.530548in}{1.070680in}}%
\pgfpathcurveto{\pgfqpoint{1.541598in}{1.070680in}}{\pgfqpoint{1.552197in}{1.075070in}}{\pgfqpoint{1.560011in}{1.082884in}}%
\pgfpathcurveto{\pgfqpoint{1.567825in}{1.090698in}}{\pgfqpoint{1.572215in}{1.101297in}}{\pgfqpoint{1.572215in}{1.112347in}}%
\pgfpathcurveto{\pgfqpoint{1.572215in}{1.123397in}}{\pgfqpoint{1.567825in}{1.133996in}}{\pgfqpoint{1.560011in}{1.141810in}}%
\pgfpathcurveto{\pgfqpoint{1.552197in}{1.149623in}}{\pgfqpoint{1.541598in}{1.154013in}}{\pgfqpoint{1.530548in}{1.154013in}}%
\pgfpathcurveto{\pgfqpoint{1.519498in}{1.154013in}}{\pgfqpoint{1.508899in}{1.149623in}}{\pgfqpoint{1.501085in}{1.141810in}}%
\pgfpathcurveto{\pgfqpoint{1.493272in}{1.133996in}}{\pgfqpoint{1.488881in}{1.123397in}}{\pgfqpoint{1.488881in}{1.112347in}}%
\pgfpathcurveto{\pgfqpoint{1.488881in}{1.101297in}}{\pgfqpoint{1.493272in}{1.090698in}}{\pgfqpoint{1.501085in}{1.082884in}}%
\pgfpathcurveto{\pgfqpoint{1.508899in}{1.075070in}}{\pgfqpoint{1.519498in}{1.070680in}}{\pgfqpoint{1.530548in}{1.070680in}}%
\pgfpathclose%
\pgfusepath{stroke,fill}%
\end{pgfscope}%
\begin{pgfscope}%
\pgfpathrectangle{\pgfqpoint{0.375000in}{0.330000in}}{\pgfqpoint{2.325000in}{2.310000in}}%
\pgfusepath{clip}%
\pgfsetbuttcap%
\pgfsetroundjoin%
\definecolor{currentfill}{rgb}{0.000000,0.000000,0.000000}%
\pgfsetfillcolor{currentfill}%
\pgfsetlinewidth{1.003750pt}%
\definecolor{currentstroke}{rgb}{0.000000,0.000000,0.000000}%
\pgfsetstrokecolor{currentstroke}%
\pgfsetdash{}{0pt}%
\pgfpathmoveto{\pgfqpoint{1.530548in}{1.070680in}}%
\pgfpathcurveto{\pgfqpoint{1.541598in}{1.070680in}}{\pgfqpoint{1.552197in}{1.075070in}}{\pgfqpoint{1.560011in}{1.082884in}}%
\pgfpathcurveto{\pgfqpoint{1.567825in}{1.090698in}}{\pgfqpoint{1.572215in}{1.101297in}}{\pgfqpoint{1.572215in}{1.112347in}}%
\pgfpathcurveto{\pgfqpoint{1.572215in}{1.123397in}}{\pgfqpoint{1.567825in}{1.133996in}}{\pgfqpoint{1.560011in}{1.141810in}}%
\pgfpathcurveto{\pgfqpoint{1.552197in}{1.149623in}}{\pgfqpoint{1.541598in}{1.154013in}}{\pgfqpoint{1.530548in}{1.154013in}}%
\pgfpathcurveto{\pgfqpoint{1.519498in}{1.154013in}}{\pgfqpoint{1.508899in}{1.149623in}}{\pgfqpoint{1.501085in}{1.141810in}}%
\pgfpathcurveto{\pgfqpoint{1.493272in}{1.133996in}}{\pgfqpoint{1.488881in}{1.123397in}}{\pgfqpoint{1.488881in}{1.112347in}}%
\pgfpathcurveto{\pgfqpoint{1.488881in}{1.101297in}}{\pgfqpoint{1.493272in}{1.090698in}}{\pgfqpoint{1.501085in}{1.082884in}}%
\pgfpathcurveto{\pgfqpoint{1.508899in}{1.075070in}}{\pgfqpoint{1.519498in}{1.070680in}}{\pgfqpoint{1.530548in}{1.070680in}}%
\pgfpathclose%
\pgfusepath{stroke,fill}%
\end{pgfscope}%
\begin{pgfscope}%
\pgfpathrectangle{\pgfqpoint{0.375000in}{0.330000in}}{\pgfqpoint{2.325000in}{2.310000in}}%
\pgfusepath{clip}%
\pgfsetbuttcap%
\pgfsetroundjoin%
\definecolor{currentfill}{rgb}{0.000000,0.000000,0.000000}%
\pgfsetfillcolor{currentfill}%
\pgfsetlinewidth{1.003750pt}%
\definecolor{currentstroke}{rgb}{0.000000,0.000000,0.000000}%
\pgfsetstrokecolor{currentstroke}%
\pgfsetdash{}{0pt}%
\pgfpathmoveto{\pgfqpoint{1.530548in}{1.070680in}}%
\pgfpathcurveto{\pgfqpoint{1.541598in}{1.070680in}}{\pgfqpoint{1.552197in}{1.075070in}}{\pgfqpoint{1.560011in}{1.082884in}}%
\pgfpathcurveto{\pgfqpoint{1.567825in}{1.090698in}}{\pgfqpoint{1.572215in}{1.101297in}}{\pgfqpoint{1.572215in}{1.112347in}}%
\pgfpathcurveto{\pgfqpoint{1.572215in}{1.123397in}}{\pgfqpoint{1.567825in}{1.133996in}}{\pgfqpoint{1.560011in}{1.141810in}}%
\pgfpathcurveto{\pgfqpoint{1.552197in}{1.149623in}}{\pgfqpoint{1.541598in}{1.154013in}}{\pgfqpoint{1.530548in}{1.154013in}}%
\pgfpathcurveto{\pgfqpoint{1.519498in}{1.154013in}}{\pgfqpoint{1.508899in}{1.149623in}}{\pgfqpoint{1.501085in}{1.141810in}}%
\pgfpathcurveto{\pgfqpoint{1.493272in}{1.133996in}}{\pgfqpoint{1.488881in}{1.123397in}}{\pgfqpoint{1.488881in}{1.112347in}}%
\pgfpathcurveto{\pgfqpoint{1.488881in}{1.101297in}}{\pgfqpoint{1.493272in}{1.090698in}}{\pgfqpoint{1.501085in}{1.082884in}}%
\pgfpathcurveto{\pgfqpoint{1.508899in}{1.075070in}}{\pgfqpoint{1.519498in}{1.070680in}}{\pgfqpoint{1.530548in}{1.070680in}}%
\pgfpathclose%
\pgfusepath{stroke,fill}%
\end{pgfscope}%
\begin{pgfscope}%
\pgfpathrectangle{\pgfqpoint{0.375000in}{0.330000in}}{\pgfqpoint{2.325000in}{2.310000in}}%
\pgfusepath{clip}%
\pgfsetbuttcap%
\pgfsetroundjoin%
\definecolor{currentfill}{rgb}{0.000000,0.000000,0.000000}%
\pgfsetfillcolor{currentfill}%
\pgfsetlinewidth{1.003750pt}%
\definecolor{currentstroke}{rgb}{0.000000,0.000000,0.000000}%
\pgfsetstrokecolor{currentstroke}%
\pgfsetdash{}{0pt}%
\pgfpathmoveto{\pgfqpoint{1.530548in}{1.070680in}}%
\pgfpathcurveto{\pgfqpoint{1.541598in}{1.070680in}}{\pgfqpoint{1.552197in}{1.075070in}}{\pgfqpoint{1.560011in}{1.082884in}}%
\pgfpathcurveto{\pgfqpoint{1.567825in}{1.090698in}}{\pgfqpoint{1.572215in}{1.101297in}}{\pgfqpoint{1.572215in}{1.112347in}}%
\pgfpathcurveto{\pgfqpoint{1.572215in}{1.123397in}}{\pgfqpoint{1.567825in}{1.133996in}}{\pgfqpoint{1.560011in}{1.141810in}}%
\pgfpathcurveto{\pgfqpoint{1.552197in}{1.149623in}}{\pgfqpoint{1.541598in}{1.154013in}}{\pgfqpoint{1.530548in}{1.154013in}}%
\pgfpathcurveto{\pgfqpoint{1.519498in}{1.154013in}}{\pgfqpoint{1.508899in}{1.149623in}}{\pgfqpoint{1.501085in}{1.141810in}}%
\pgfpathcurveto{\pgfqpoint{1.493272in}{1.133996in}}{\pgfqpoint{1.488881in}{1.123397in}}{\pgfqpoint{1.488881in}{1.112347in}}%
\pgfpathcurveto{\pgfqpoint{1.488881in}{1.101297in}}{\pgfqpoint{1.493272in}{1.090698in}}{\pgfqpoint{1.501085in}{1.082884in}}%
\pgfpathcurveto{\pgfqpoint{1.508899in}{1.075070in}}{\pgfqpoint{1.519498in}{1.070680in}}{\pgfqpoint{1.530548in}{1.070680in}}%
\pgfpathclose%
\pgfusepath{stroke,fill}%
\end{pgfscope}%
\begin{pgfscope}%
\pgfpathrectangle{\pgfqpoint{0.375000in}{0.330000in}}{\pgfqpoint{2.325000in}{2.310000in}}%
\pgfusepath{clip}%
\pgfsetbuttcap%
\pgfsetroundjoin%
\definecolor{currentfill}{rgb}{0.000000,0.000000,0.000000}%
\pgfsetfillcolor{currentfill}%
\pgfsetlinewidth{1.003750pt}%
\definecolor{currentstroke}{rgb}{0.000000,0.000000,0.000000}%
\pgfsetstrokecolor{currentstroke}%
\pgfsetdash{}{0pt}%
\pgfpathmoveto{\pgfqpoint{1.530548in}{1.070680in}}%
\pgfpathcurveto{\pgfqpoint{1.541598in}{1.070680in}}{\pgfqpoint{1.552197in}{1.075070in}}{\pgfqpoint{1.560011in}{1.082884in}}%
\pgfpathcurveto{\pgfqpoint{1.567825in}{1.090698in}}{\pgfqpoint{1.572215in}{1.101297in}}{\pgfqpoint{1.572215in}{1.112347in}}%
\pgfpathcurveto{\pgfqpoint{1.572215in}{1.123397in}}{\pgfqpoint{1.567825in}{1.133996in}}{\pgfqpoint{1.560011in}{1.141810in}}%
\pgfpathcurveto{\pgfqpoint{1.552197in}{1.149623in}}{\pgfqpoint{1.541598in}{1.154013in}}{\pgfqpoint{1.530548in}{1.154013in}}%
\pgfpathcurveto{\pgfqpoint{1.519498in}{1.154013in}}{\pgfqpoint{1.508899in}{1.149623in}}{\pgfqpoint{1.501085in}{1.141810in}}%
\pgfpathcurveto{\pgfqpoint{1.493272in}{1.133996in}}{\pgfqpoint{1.488881in}{1.123397in}}{\pgfqpoint{1.488881in}{1.112347in}}%
\pgfpathcurveto{\pgfqpoint{1.488881in}{1.101297in}}{\pgfqpoint{1.493272in}{1.090698in}}{\pgfqpoint{1.501085in}{1.082884in}}%
\pgfpathcurveto{\pgfqpoint{1.508899in}{1.075070in}}{\pgfqpoint{1.519498in}{1.070680in}}{\pgfqpoint{1.530548in}{1.070680in}}%
\pgfpathclose%
\pgfusepath{stroke,fill}%
\end{pgfscope}%
\begin{pgfscope}%
\pgfpathrectangle{\pgfqpoint{0.375000in}{0.330000in}}{\pgfqpoint{2.325000in}{2.310000in}}%
\pgfusepath{clip}%
\pgfsetbuttcap%
\pgfsetroundjoin%
\definecolor{currentfill}{rgb}{0.000000,0.000000,0.000000}%
\pgfsetfillcolor{currentfill}%
\pgfsetlinewidth{1.003750pt}%
\definecolor{currentstroke}{rgb}{0.000000,0.000000,0.000000}%
\pgfsetstrokecolor{currentstroke}%
\pgfsetdash{}{0pt}%
\pgfpathmoveto{\pgfqpoint{1.530548in}{1.018649in}}%
\pgfpathcurveto{\pgfqpoint{1.541598in}{1.018649in}}{\pgfqpoint{1.552197in}{1.023040in}}{\pgfqpoint{1.560011in}{1.030853in}}%
\pgfpathcurveto{\pgfqpoint{1.567825in}{1.038667in}}{\pgfqpoint{1.572215in}{1.049266in}}{\pgfqpoint{1.572215in}{1.060316in}}%
\pgfpathcurveto{\pgfqpoint{1.572215in}{1.071366in}}{\pgfqpoint{1.567825in}{1.081965in}}{\pgfqpoint{1.560011in}{1.089779in}}%
\pgfpathcurveto{\pgfqpoint{1.552197in}{1.097592in}}{\pgfqpoint{1.541598in}{1.101983in}}{\pgfqpoint{1.530548in}{1.101983in}}%
\pgfpathcurveto{\pgfqpoint{1.519498in}{1.101983in}}{\pgfqpoint{1.508899in}{1.097592in}}{\pgfqpoint{1.501085in}{1.089779in}}%
\pgfpathcurveto{\pgfqpoint{1.493272in}{1.081965in}}{\pgfqpoint{1.488881in}{1.071366in}}{\pgfqpoint{1.488881in}{1.060316in}}%
\pgfpathcurveto{\pgfqpoint{1.488881in}{1.049266in}}{\pgfqpoint{1.493272in}{1.038667in}}{\pgfqpoint{1.501085in}{1.030853in}}%
\pgfpathcurveto{\pgfqpoint{1.508899in}{1.023040in}}{\pgfqpoint{1.519498in}{1.018649in}}{\pgfqpoint{1.530548in}{1.018649in}}%
\pgfpathclose%
\pgfusepath{stroke,fill}%
\end{pgfscope}%
\begin{pgfscope}%
\pgfpathrectangle{\pgfqpoint{0.375000in}{0.330000in}}{\pgfqpoint{2.325000in}{2.310000in}}%
\pgfusepath{clip}%
\pgfsetbuttcap%
\pgfsetroundjoin%
\definecolor{currentfill}{rgb}{0.000000,0.000000,0.000000}%
\pgfsetfillcolor{currentfill}%
\pgfsetlinewidth{1.003750pt}%
\definecolor{currentstroke}{rgb}{0.000000,0.000000,0.000000}%
\pgfsetstrokecolor{currentstroke}%
\pgfsetdash{}{0pt}%
\pgfpathmoveto{\pgfqpoint{1.530548in}{1.070680in}}%
\pgfpathcurveto{\pgfqpoint{1.541598in}{1.070680in}}{\pgfqpoint{1.552197in}{1.075070in}}{\pgfqpoint{1.560011in}{1.082884in}}%
\pgfpathcurveto{\pgfqpoint{1.567825in}{1.090698in}}{\pgfqpoint{1.572215in}{1.101297in}}{\pgfqpoint{1.572215in}{1.112347in}}%
\pgfpathcurveto{\pgfqpoint{1.572215in}{1.123397in}}{\pgfqpoint{1.567825in}{1.133996in}}{\pgfqpoint{1.560011in}{1.141810in}}%
\pgfpathcurveto{\pgfqpoint{1.552197in}{1.149623in}}{\pgfqpoint{1.541598in}{1.154013in}}{\pgfqpoint{1.530548in}{1.154013in}}%
\pgfpathcurveto{\pgfqpoint{1.519498in}{1.154013in}}{\pgfqpoint{1.508899in}{1.149623in}}{\pgfqpoint{1.501085in}{1.141810in}}%
\pgfpathcurveto{\pgfqpoint{1.493272in}{1.133996in}}{\pgfqpoint{1.488881in}{1.123397in}}{\pgfqpoint{1.488881in}{1.112347in}}%
\pgfpathcurveto{\pgfqpoint{1.488881in}{1.101297in}}{\pgfqpoint{1.493272in}{1.090698in}}{\pgfqpoint{1.501085in}{1.082884in}}%
\pgfpathcurveto{\pgfqpoint{1.508899in}{1.075070in}}{\pgfqpoint{1.519498in}{1.070680in}}{\pgfqpoint{1.530548in}{1.070680in}}%
\pgfpathclose%
\pgfusepath{stroke,fill}%
\end{pgfscope}%
\begin{pgfscope}%
\pgfpathrectangle{\pgfqpoint{0.375000in}{0.330000in}}{\pgfqpoint{2.325000in}{2.310000in}}%
\pgfusepath{clip}%
\pgfsetbuttcap%
\pgfsetroundjoin%
\definecolor{currentfill}{rgb}{0.000000,0.000000,0.000000}%
\pgfsetfillcolor{currentfill}%
\pgfsetlinewidth{1.003750pt}%
\definecolor{currentstroke}{rgb}{0.000000,0.000000,0.000000}%
\pgfsetstrokecolor{currentstroke}%
\pgfsetdash{}{0pt}%
\pgfpathmoveto{\pgfqpoint{1.530548in}{1.122711in}}%
\pgfpathcurveto{\pgfqpoint{1.541598in}{1.122711in}}{\pgfqpoint{1.552197in}{1.127101in}}{\pgfqpoint{1.560011in}{1.134915in}}%
\pgfpathcurveto{\pgfqpoint{1.567825in}{1.142728in}}{\pgfqpoint{1.572215in}{1.153327in}}{\pgfqpoint{1.572215in}{1.164378in}}%
\pgfpathcurveto{\pgfqpoint{1.572215in}{1.175428in}}{\pgfqpoint{1.567825in}{1.186027in}}{\pgfqpoint{1.560011in}{1.193840in}}%
\pgfpathcurveto{\pgfqpoint{1.552197in}{1.201654in}}{\pgfqpoint{1.541598in}{1.206044in}}{\pgfqpoint{1.530548in}{1.206044in}}%
\pgfpathcurveto{\pgfqpoint{1.519498in}{1.206044in}}{\pgfqpoint{1.508899in}{1.201654in}}{\pgfqpoint{1.501085in}{1.193840in}}%
\pgfpathcurveto{\pgfqpoint{1.493272in}{1.186027in}}{\pgfqpoint{1.488881in}{1.175428in}}{\pgfqpoint{1.488881in}{1.164378in}}%
\pgfpathcurveto{\pgfqpoint{1.488881in}{1.153327in}}{\pgfqpoint{1.493272in}{1.142728in}}{\pgfqpoint{1.501085in}{1.134915in}}%
\pgfpathcurveto{\pgfqpoint{1.508899in}{1.127101in}}{\pgfqpoint{1.519498in}{1.122711in}}{\pgfqpoint{1.530548in}{1.122711in}}%
\pgfpathclose%
\pgfusepath{stroke,fill}%
\end{pgfscope}%
\begin{pgfscope}%
\pgfpathrectangle{\pgfqpoint{0.375000in}{0.330000in}}{\pgfqpoint{2.325000in}{2.310000in}}%
\pgfusepath{clip}%
\pgfsetbuttcap%
\pgfsetroundjoin%
\definecolor{currentfill}{rgb}{0.000000,0.000000,0.000000}%
\pgfsetfillcolor{currentfill}%
\pgfsetlinewidth{1.003750pt}%
\definecolor{currentstroke}{rgb}{0.000000,0.000000,0.000000}%
\pgfsetstrokecolor{currentstroke}%
\pgfsetdash{}{0pt}%
\pgfpathmoveto{\pgfqpoint{1.530548in}{1.070680in}}%
\pgfpathcurveto{\pgfqpoint{1.541598in}{1.070680in}}{\pgfqpoint{1.552197in}{1.075070in}}{\pgfqpoint{1.560011in}{1.082884in}}%
\pgfpathcurveto{\pgfqpoint{1.567825in}{1.090698in}}{\pgfqpoint{1.572215in}{1.101297in}}{\pgfqpoint{1.572215in}{1.112347in}}%
\pgfpathcurveto{\pgfqpoint{1.572215in}{1.123397in}}{\pgfqpoint{1.567825in}{1.133996in}}{\pgfqpoint{1.560011in}{1.141810in}}%
\pgfpathcurveto{\pgfqpoint{1.552197in}{1.149623in}}{\pgfqpoint{1.541598in}{1.154013in}}{\pgfqpoint{1.530548in}{1.154013in}}%
\pgfpathcurveto{\pgfqpoint{1.519498in}{1.154013in}}{\pgfqpoint{1.508899in}{1.149623in}}{\pgfqpoint{1.501085in}{1.141810in}}%
\pgfpathcurveto{\pgfqpoint{1.493272in}{1.133996in}}{\pgfqpoint{1.488881in}{1.123397in}}{\pgfqpoint{1.488881in}{1.112347in}}%
\pgfpathcurveto{\pgfqpoint{1.488881in}{1.101297in}}{\pgfqpoint{1.493272in}{1.090698in}}{\pgfqpoint{1.501085in}{1.082884in}}%
\pgfpathcurveto{\pgfqpoint{1.508899in}{1.075070in}}{\pgfqpoint{1.519498in}{1.070680in}}{\pgfqpoint{1.530548in}{1.070680in}}%
\pgfpathclose%
\pgfusepath{stroke,fill}%
\end{pgfscope}%
\begin{pgfscope}%
\pgfpathrectangle{\pgfqpoint{0.375000in}{0.330000in}}{\pgfqpoint{2.325000in}{2.310000in}}%
\pgfusepath{clip}%
\pgfsetbuttcap%
\pgfsetroundjoin%
\definecolor{currentfill}{rgb}{0.000000,0.000000,0.000000}%
\pgfsetfillcolor{currentfill}%
\pgfsetlinewidth{1.003750pt}%
\definecolor{currentstroke}{rgb}{0.000000,0.000000,0.000000}%
\pgfsetstrokecolor{currentstroke}%
\pgfsetdash{}{0pt}%
\pgfpathmoveto{\pgfqpoint{1.530548in}{1.070680in}}%
\pgfpathcurveto{\pgfqpoint{1.541598in}{1.070680in}}{\pgfqpoint{1.552197in}{1.075070in}}{\pgfqpoint{1.560011in}{1.082884in}}%
\pgfpathcurveto{\pgfqpoint{1.567825in}{1.090698in}}{\pgfqpoint{1.572215in}{1.101297in}}{\pgfqpoint{1.572215in}{1.112347in}}%
\pgfpathcurveto{\pgfqpoint{1.572215in}{1.123397in}}{\pgfqpoint{1.567825in}{1.133996in}}{\pgfqpoint{1.560011in}{1.141810in}}%
\pgfpathcurveto{\pgfqpoint{1.552197in}{1.149623in}}{\pgfqpoint{1.541598in}{1.154013in}}{\pgfqpoint{1.530548in}{1.154013in}}%
\pgfpathcurveto{\pgfqpoint{1.519498in}{1.154013in}}{\pgfqpoint{1.508899in}{1.149623in}}{\pgfqpoint{1.501085in}{1.141810in}}%
\pgfpathcurveto{\pgfqpoint{1.493272in}{1.133996in}}{\pgfqpoint{1.488881in}{1.123397in}}{\pgfqpoint{1.488881in}{1.112347in}}%
\pgfpathcurveto{\pgfqpoint{1.488881in}{1.101297in}}{\pgfqpoint{1.493272in}{1.090698in}}{\pgfqpoint{1.501085in}{1.082884in}}%
\pgfpathcurveto{\pgfqpoint{1.508899in}{1.075070in}}{\pgfqpoint{1.519498in}{1.070680in}}{\pgfqpoint{1.530548in}{1.070680in}}%
\pgfpathclose%
\pgfusepath{stroke,fill}%
\end{pgfscope}%
\begin{pgfscope}%
\pgfpathrectangle{\pgfqpoint{0.375000in}{0.330000in}}{\pgfqpoint{2.325000in}{2.310000in}}%
\pgfusepath{clip}%
\pgfsetbuttcap%
\pgfsetroundjoin%
\definecolor{currentfill}{rgb}{0.000000,0.000000,0.000000}%
\pgfsetfillcolor{currentfill}%
\pgfsetlinewidth{1.003750pt}%
\definecolor{currentstroke}{rgb}{0.000000,0.000000,0.000000}%
\pgfsetstrokecolor{currentstroke}%
\pgfsetdash{}{0pt}%
\pgfpathmoveto{\pgfqpoint{1.530548in}{1.122711in}}%
\pgfpathcurveto{\pgfqpoint{1.541598in}{1.122711in}}{\pgfqpoint{1.552197in}{1.127101in}}{\pgfqpoint{1.560011in}{1.134915in}}%
\pgfpathcurveto{\pgfqpoint{1.567825in}{1.142728in}}{\pgfqpoint{1.572215in}{1.153327in}}{\pgfqpoint{1.572215in}{1.164378in}}%
\pgfpathcurveto{\pgfqpoint{1.572215in}{1.175428in}}{\pgfqpoint{1.567825in}{1.186027in}}{\pgfqpoint{1.560011in}{1.193840in}}%
\pgfpathcurveto{\pgfqpoint{1.552197in}{1.201654in}}{\pgfqpoint{1.541598in}{1.206044in}}{\pgfqpoint{1.530548in}{1.206044in}}%
\pgfpathcurveto{\pgfqpoint{1.519498in}{1.206044in}}{\pgfqpoint{1.508899in}{1.201654in}}{\pgfqpoint{1.501085in}{1.193840in}}%
\pgfpathcurveto{\pgfqpoint{1.493272in}{1.186027in}}{\pgfqpoint{1.488881in}{1.175428in}}{\pgfqpoint{1.488881in}{1.164378in}}%
\pgfpathcurveto{\pgfqpoint{1.488881in}{1.153327in}}{\pgfqpoint{1.493272in}{1.142728in}}{\pgfqpoint{1.501085in}{1.134915in}}%
\pgfpathcurveto{\pgfqpoint{1.508899in}{1.127101in}}{\pgfqpoint{1.519498in}{1.122711in}}{\pgfqpoint{1.530548in}{1.122711in}}%
\pgfpathclose%
\pgfusepath{stroke,fill}%
\end{pgfscope}%
\begin{pgfscope}%
\pgfpathrectangle{\pgfqpoint{0.375000in}{0.330000in}}{\pgfqpoint{2.325000in}{2.310000in}}%
\pgfusepath{clip}%
\pgfsetbuttcap%
\pgfsetroundjoin%
\definecolor{currentfill}{rgb}{0.000000,0.000000,0.000000}%
\pgfsetfillcolor{currentfill}%
\pgfsetlinewidth{1.003750pt}%
\definecolor{currentstroke}{rgb}{0.000000,0.000000,0.000000}%
\pgfsetstrokecolor{currentstroke}%
\pgfsetdash{}{0pt}%
\pgfpathmoveto{\pgfqpoint{1.530548in}{1.070680in}}%
\pgfpathcurveto{\pgfqpoint{1.541598in}{1.070680in}}{\pgfqpoint{1.552197in}{1.075070in}}{\pgfqpoint{1.560011in}{1.082884in}}%
\pgfpathcurveto{\pgfqpoint{1.567825in}{1.090698in}}{\pgfqpoint{1.572215in}{1.101297in}}{\pgfqpoint{1.572215in}{1.112347in}}%
\pgfpathcurveto{\pgfqpoint{1.572215in}{1.123397in}}{\pgfqpoint{1.567825in}{1.133996in}}{\pgfqpoint{1.560011in}{1.141810in}}%
\pgfpathcurveto{\pgfqpoint{1.552197in}{1.149623in}}{\pgfqpoint{1.541598in}{1.154013in}}{\pgfqpoint{1.530548in}{1.154013in}}%
\pgfpathcurveto{\pgfqpoint{1.519498in}{1.154013in}}{\pgfqpoint{1.508899in}{1.149623in}}{\pgfqpoint{1.501085in}{1.141810in}}%
\pgfpathcurveto{\pgfqpoint{1.493272in}{1.133996in}}{\pgfqpoint{1.488881in}{1.123397in}}{\pgfqpoint{1.488881in}{1.112347in}}%
\pgfpathcurveto{\pgfqpoint{1.488881in}{1.101297in}}{\pgfqpoint{1.493272in}{1.090698in}}{\pgfqpoint{1.501085in}{1.082884in}}%
\pgfpathcurveto{\pgfqpoint{1.508899in}{1.075070in}}{\pgfqpoint{1.519498in}{1.070680in}}{\pgfqpoint{1.530548in}{1.070680in}}%
\pgfpathclose%
\pgfusepath{stroke,fill}%
\end{pgfscope}%
\begin{pgfscope}%
\pgfpathrectangle{\pgfqpoint{0.375000in}{0.330000in}}{\pgfqpoint{2.325000in}{2.310000in}}%
\pgfusepath{clip}%
\pgfsetbuttcap%
\pgfsetroundjoin%
\definecolor{currentfill}{rgb}{0.000000,0.000000,0.000000}%
\pgfsetfillcolor{currentfill}%
\pgfsetlinewidth{1.003750pt}%
\definecolor{currentstroke}{rgb}{0.000000,0.000000,0.000000}%
\pgfsetstrokecolor{currentstroke}%
\pgfsetdash{}{0pt}%
\pgfpathmoveto{\pgfqpoint{1.530548in}{1.070680in}}%
\pgfpathcurveto{\pgfqpoint{1.541598in}{1.070680in}}{\pgfqpoint{1.552197in}{1.075070in}}{\pgfqpoint{1.560011in}{1.082884in}}%
\pgfpathcurveto{\pgfqpoint{1.567825in}{1.090698in}}{\pgfqpoint{1.572215in}{1.101297in}}{\pgfqpoint{1.572215in}{1.112347in}}%
\pgfpathcurveto{\pgfqpoint{1.572215in}{1.123397in}}{\pgfqpoint{1.567825in}{1.133996in}}{\pgfqpoint{1.560011in}{1.141810in}}%
\pgfpathcurveto{\pgfqpoint{1.552197in}{1.149623in}}{\pgfqpoint{1.541598in}{1.154013in}}{\pgfqpoint{1.530548in}{1.154013in}}%
\pgfpathcurveto{\pgfqpoint{1.519498in}{1.154013in}}{\pgfqpoint{1.508899in}{1.149623in}}{\pgfqpoint{1.501085in}{1.141810in}}%
\pgfpathcurveto{\pgfqpoint{1.493272in}{1.133996in}}{\pgfqpoint{1.488881in}{1.123397in}}{\pgfqpoint{1.488881in}{1.112347in}}%
\pgfpathcurveto{\pgfqpoint{1.488881in}{1.101297in}}{\pgfqpoint{1.493272in}{1.090698in}}{\pgfqpoint{1.501085in}{1.082884in}}%
\pgfpathcurveto{\pgfqpoint{1.508899in}{1.075070in}}{\pgfqpoint{1.519498in}{1.070680in}}{\pgfqpoint{1.530548in}{1.070680in}}%
\pgfpathclose%
\pgfusepath{stroke,fill}%
\end{pgfscope}%
\begin{pgfscope}%
\pgfpathrectangle{\pgfqpoint{0.375000in}{0.330000in}}{\pgfqpoint{2.325000in}{2.310000in}}%
\pgfusepath{clip}%
\pgfsetbuttcap%
\pgfsetroundjoin%
\definecolor{currentfill}{rgb}{0.000000,0.000000,0.000000}%
\pgfsetfillcolor{currentfill}%
\pgfsetlinewidth{1.003750pt}%
\definecolor{currentstroke}{rgb}{0.000000,0.000000,0.000000}%
\pgfsetstrokecolor{currentstroke}%
\pgfsetdash{}{0pt}%
\pgfpathmoveto{\pgfqpoint{1.530548in}{1.070680in}}%
\pgfpathcurveto{\pgfqpoint{1.541598in}{1.070680in}}{\pgfqpoint{1.552197in}{1.075070in}}{\pgfqpoint{1.560011in}{1.082884in}}%
\pgfpathcurveto{\pgfqpoint{1.567825in}{1.090698in}}{\pgfqpoint{1.572215in}{1.101297in}}{\pgfqpoint{1.572215in}{1.112347in}}%
\pgfpathcurveto{\pgfqpoint{1.572215in}{1.123397in}}{\pgfqpoint{1.567825in}{1.133996in}}{\pgfqpoint{1.560011in}{1.141810in}}%
\pgfpathcurveto{\pgfqpoint{1.552197in}{1.149623in}}{\pgfqpoint{1.541598in}{1.154013in}}{\pgfqpoint{1.530548in}{1.154013in}}%
\pgfpathcurveto{\pgfqpoint{1.519498in}{1.154013in}}{\pgfqpoint{1.508899in}{1.149623in}}{\pgfqpoint{1.501085in}{1.141810in}}%
\pgfpathcurveto{\pgfqpoint{1.493272in}{1.133996in}}{\pgfqpoint{1.488881in}{1.123397in}}{\pgfqpoint{1.488881in}{1.112347in}}%
\pgfpathcurveto{\pgfqpoint{1.488881in}{1.101297in}}{\pgfqpoint{1.493272in}{1.090698in}}{\pgfqpoint{1.501085in}{1.082884in}}%
\pgfpathcurveto{\pgfqpoint{1.508899in}{1.075070in}}{\pgfqpoint{1.519498in}{1.070680in}}{\pgfqpoint{1.530548in}{1.070680in}}%
\pgfpathclose%
\pgfusepath{stroke,fill}%
\end{pgfscope}%
\begin{pgfscope}%
\pgfpathrectangle{\pgfqpoint{0.375000in}{0.330000in}}{\pgfqpoint{2.325000in}{2.310000in}}%
\pgfusepath{clip}%
\pgfsetbuttcap%
\pgfsetroundjoin%
\definecolor{currentfill}{rgb}{0.000000,0.000000,0.000000}%
\pgfsetfillcolor{currentfill}%
\pgfsetlinewidth{1.003750pt}%
\definecolor{currentstroke}{rgb}{0.000000,0.000000,0.000000}%
\pgfsetstrokecolor{currentstroke}%
\pgfsetdash{}{0pt}%
\pgfpathmoveto{\pgfqpoint{1.530548in}{1.122711in}}%
\pgfpathcurveto{\pgfqpoint{1.541598in}{1.122711in}}{\pgfqpoint{1.552197in}{1.127101in}}{\pgfqpoint{1.560011in}{1.134915in}}%
\pgfpathcurveto{\pgfqpoint{1.567825in}{1.142728in}}{\pgfqpoint{1.572215in}{1.153327in}}{\pgfqpoint{1.572215in}{1.164378in}}%
\pgfpathcurveto{\pgfqpoint{1.572215in}{1.175428in}}{\pgfqpoint{1.567825in}{1.186027in}}{\pgfqpoint{1.560011in}{1.193840in}}%
\pgfpathcurveto{\pgfqpoint{1.552197in}{1.201654in}}{\pgfqpoint{1.541598in}{1.206044in}}{\pgfqpoint{1.530548in}{1.206044in}}%
\pgfpathcurveto{\pgfqpoint{1.519498in}{1.206044in}}{\pgfqpoint{1.508899in}{1.201654in}}{\pgfqpoint{1.501085in}{1.193840in}}%
\pgfpathcurveto{\pgfqpoint{1.493272in}{1.186027in}}{\pgfqpoint{1.488881in}{1.175428in}}{\pgfqpoint{1.488881in}{1.164378in}}%
\pgfpathcurveto{\pgfqpoint{1.488881in}{1.153327in}}{\pgfqpoint{1.493272in}{1.142728in}}{\pgfqpoint{1.501085in}{1.134915in}}%
\pgfpathcurveto{\pgfqpoint{1.508899in}{1.127101in}}{\pgfqpoint{1.519498in}{1.122711in}}{\pgfqpoint{1.530548in}{1.122711in}}%
\pgfpathclose%
\pgfusepath{stroke,fill}%
\end{pgfscope}%
\begin{pgfscope}%
\pgfpathrectangle{\pgfqpoint{0.375000in}{0.330000in}}{\pgfqpoint{2.325000in}{2.310000in}}%
\pgfusepath{clip}%
\pgfsetbuttcap%
\pgfsetroundjoin%
\definecolor{currentfill}{rgb}{0.000000,0.000000,0.000000}%
\pgfsetfillcolor{currentfill}%
\pgfsetlinewidth{1.003750pt}%
\definecolor{currentstroke}{rgb}{0.000000,0.000000,0.000000}%
\pgfsetstrokecolor{currentstroke}%
\pgfsetdash{}{0pt}%
\pgfpathmoveto{\pgfqpoint{1.530548in}{1.122711in}}%
\pgfpathcurveto{\pgfqpoint{1.541598in}{1.122711in}}{\pgfqpoint{1.552197in}{1.127101in}}{\pgfqpoint{1.560011in}{1.134915in}}%
\pgfpathcurveto{\pgfqpoint{1.567825in}{1.142728in}}{\pgfqpoint{1.572215in}{1.153327in}}{\pgfqpoint{1.572215in}{1.164378in}}%
\pgfpathcurveto{\pgfqpoint{1.572215in}{1.175428in}}{\pgfqpoint{1.567825in}{1.186027in}}{\pgfqpoint{1.560011in}{1.193840in}}%
\pgfpathcurveto{\pgfqpoint{1.552197in}{1.201654in}}{\pgfqpoint{1.541598in}{1.206044in}}{\pgfqpoint{1.530548in}{1.206044in}}%
\pgfpathcurveto{\pgfqpoint{1.519498in}{1.206044in}}{\pgfqpoint{1.508899in}{1.201654in}}{\pgfqpoint{1.501085in}{1.193840in}}%
\pgfpathcurveto{\pgfqpoint{1.493272in}{1.186027in}}{\pgfqpoint{1.488881in}{1.175428in}}{\pgfqpoint{1.488881in}{1.164378in}}%
\pgfpathcurveto{\pgfqpoint{1.488881in}{1.153327in}}{\pgfqpoint{1.493272in}{1.142728in}}{\pgfqpoint{1.501085in}{1.134915in}}%
\pgfpathcurveto{\pgfqpoint{1.508899in}{1.127101in}}{\pgfqpoint{1.519498in}{1.122711in}}{\pgfqpoint{1.530548in}{1.122711in}}%
\pgfpathclose%
\pgfusepath{stroke,fill}%
\end{pgfscope}%
\begin{pgfscope}%
\pgfpathrectangle{\pgfqpoint{0.375000in}{0.330000in}}{\pgfqpoint{2.325000in}{2.310000in}}%
\pgfusepath{clip}%
\pgfsetbuttcap%
\pgfsetroundjoin%
\definecolor{currentfill}{rgb}{0.000000,0.000000,0.000000}%
\pgfsetfillcolor{currentfill}%
\pgfsetlinewidth{1.003750pt}%
\definecolor{currentstroke}{rgb}{0.000000,0.000000,0.000000}%
\pgfsetstrokecolor{currentstroke}%
\pgfsetdash{}{0pt}%
\pgfpathmoveto{\pgfqpoint{2.055402in}{1.486927in}}%
\pgfpathcurveto{\pgfqpoint{2.066452in}{1.486927in}}{\pgfqpoint{2.077051in}{1.491317in}}{\pgfqpoint{2.084865in}{1.499131in}}%
\pgfpathcurveto{\pgfqpoint{2.092678in}{1.506944in}}{\pgfqpoint{2.097069in}{1.517543in}}{\pgfqpoint{2.097069in}{1.528593in}}%
\pgfpathcurveto{\pgfqpoint{2.097069in}{1.539644in}}{\pgfqpoint{2.092678in}{1.550243in}}{\pgfqpoint{2.084865in}{1.558056in}}%
\pgfpathcurveto{\pgfqpoint{2.077051in}{1.565870in}}{\pgfqpoint{2.066452in}{1.570260in}}{\pgfqpoint{2.055402in}{1.570260in}}%
\pgfpathcurveto{\pgfqpoint{2.044352in}{1.570260in}}{\pgfqpoint{2.033753in}{1.565870in}}{\pgfqpoint{2.025939in}{1.558056in}}%
\pgfpathcurveto{\pgfqpoint{2.018125in}{1.550243in}}{\pgfqpoint{2.013735in}{1.539644in}}{\pgfqpoint{2.013735in}{1.528593in}}%
\pgfpathcurveto{\pgfqpoint{2.013735in}{1.517543in}}{\pgfqpoint{2.018125in}{1.506944in}}{\pgfqpoint{2.025939in}{1.499131in}}%
\pgfpathcurveto{\pgfqpoint{2.033753in}{1.491317in}}{\pgfqpoint{2.044352in}{1.486927in}}{\pgfqpoint{2.055402in}{1.486927in}}%
\pgfpathclose%
\pgfusepath{stroke,fill}%
\end{pgfscope}%
\begin{pgfscope}%
\pgfpathrectangle{\pgfqpoint{0.375000in}{0.330000in}}{\pgfqpoint{2.325000in}{2.310000in}}%
\pgfusepath{clip}%
\pgfsetbuttcap%
\pgfsetroundjoin%
\definecolor{currentfill}{rgb}{0.000000,0.000000,0.000000}%
\pgfsetfillcolor{currentfill}%
\pgfsetlinewidth{1.003750pt}%
\definecolor{currentstroke}{rgb}{0.000000,0.000000,0.000000}%
\pgfsetstrokecolor{currentstroke}%
\pgfsetdash{}{0pt}%
\pgfpathmoveto{\pgfqpoint{2.055402in}{1.486927in}}%
\pgfpathcurveto{\pgfqpoint{2.066452in}{1.486927in}}{\pgfqpoint{2.077051in}{1.491317in}}{\pgfqpoint{2.084865in}{1.499131in}}%
\pgfpathcurveto{\pgfqpoint{2.092678in}{1.506944in}}{\pgfqpoint{2.097069in}{1.517543in}}{\pgfqpoint{2.097069in}{1.528593in}}%
\pgfpathcurveto{\pgfqpoint{2.097069in}{1.539644in}}{\pgfqpoint{2.092678in}{1.550243in}}{\pgfqpoint{2.084865in}{1.558056in}}%
\pgfpathcurveto{\pgfqpoint{2.077051in}{1.565870in}}{\pgfqpoint{2.066452in}{1.570260in}}{\pgfqpoint{2.055402in}{1.570260in}}%
\pgfpathcurveto{\pgfqpoint{2.044352in}{1.570260in}}{\pgfqpoint{2.033753in}{1.565870in}}{\pgfqpoint{2.025939in}{1.558056in}}%
\pgfpathcurveto{\pgfqpoint{2.018125in}{1.550243in}}{\pgfqpoint{2.013735in}{1.539644in}}{\pgfqpoint{2.013735in}{1.528593in}}%
\pgfpathcurveto{\pgfqpoint{2.013735in}{1.517543in}}{\pgfqpoint{2.018125in}{1.506944in}}{\pgfqpoint{2.025939in}{1.499131in}}%
\pgfpathcurveto{\pgfqpoint{2.033753in}{1.491317in}}{\pgfqpoint{2.044352in}{1.486927in}}{\pgfqpoint{2.055402in}{1.486927in}}%
\pgfpathclose%
\pgfusepath{stroke,fill}%
\end{pgfscope}%
\begin{pgfscope}%
\pgfpathrectangle{\pgfqpoint{0.375000in}{0.330000in}}{\pgfqpoint{2.325000in}{2.310000in}}%
\pgfusepath{clip}%
\pgfsetbuttcap%
\pgfsetroundjoin%
\definecolor{currentfill}{rgb}{0.000000,0.000000,0.000000}%
\pgfsetfillcolor{currentfill}%
\pgfsetlinewidth{1.003750pt}%
\definecolor{currentstroke}{rgb}{0.000000,0.000000,0.000000}%
\pgfsetstrokecolor{currentstroke}%
\pgfsetdash{}{0pt}%
\pgfpathmoveto{\pgfqpoint{2.055402in}{1.486927in}}%
\pgfpathcurveto{\pgfqpoint{2.066452in}{1.486927in}}{\pgfqpoint{2.077051in}{1.491317in}}{\pgfqpoint{2.084865in}{1.499131in}}%
\pgfpathcurveto{\pgfqpoint{2.092678in}{1.506944in}}{\pgfqpoint{2.097069in}{1.517543in}}{\pgfqpoint{2.097069in}{1.528593in}}%
\pgfpathcurveto{\pgfqpoint{2.097069in}{1.539644in}}{\pgfqpoint{2.092678in}{1.550243in}}{\pgfqpoint{2.084865in}{1.558056in}}%
\pgfpathcurveto{\pgfqpoint{2.077051in}{1.565870in}}{\pgfqpoint{2.066452in}{1.570260in}}{\pgfqpoint{2.055402in}{1.570260in}}%
\pgfpathcurveto{\pgfqpoint{2.044352in}{1.570260in}}{\pgfqpoint{2.033753in}{1.565870in}}{\pgfqpoint{2.025939in}{1.558056in}}%
\pgfpathcurveto{\pgfqpoint{2.018125in}{1.550243in}}{\pgfqpoint{2.013735in}{1.539644in}}{\pgfqpoint{2.013735in}{1.528593in}}%
\pgfpathcurveto{\pgfqpoint{2.013735in}{1.517543in}}{\pgfqpoint{2.018125in}{1.506944in}}{\pgfqpoint{2.025939in}{1.499131in}}%
\pgfpathcurveto{\pgfqpoint{2.033753in}{1.491317in}}{\pgfqpoint{2.044352in}{1.486927in}}{\pgfqpoint{2.055402in}{1.486927in}}%
\pgfpathclose%
\pgfusepath{stroke,fill}%
\end{pgfscope}%
\begin{pgfscope}%
\pgfpathrectangle{\pgfqpoint{0.375000in}{0.330000in}}{\pgfqpoint{2.325000in}{2.310000in}}%
\pgfusepath{clip}%
\pgfsetbuttcap%
\pgfsetroundjoin%
\definecolor{currentfill}{rgb}{0.000000,0.000000,0.000000}%
\pgfsetfillcolor{currentfill}%
\pgfsetlinewidth{1.003750pt}%
\definecolor{currentstroke}{rgb}{0.000000,0.000000,0.000000}%
\pgfsetstrokecolor{currentstroke}%
\pgfsetdash{}{0pt}%
\pgfpathmoveto{\pgfqpoint{2.055402in}{1.434896in}}%
\pgfpathcurveto{\pgfqpoint{2.066452in}{1.434896in}}{\pgfqpoint{2.077051in}{1.439286in}}{\pgfqpoint{2.084865in}{1.447100in}}%
\pgfpathcurveto{\pgfqpoint{2.092678in}{1.454913in}}{\pgfqpoint{2.097069in}{1.465512in}}{\pgfqpoint{2.097069in}{1.476563in}}%
\pgfpathcurveto{\pgfqpoint{2.097069in}{1.487613in}}{\pgfqpoint{2.092678in}{1.498212in}}{\pgfqpoint{2.084865in}{1.506025in}}%
\pgfpathcurveto{\pgfqpoint{2.077051in}{1.513839in}}{\pgfqpoint{2.066452in}{1.518229in}}{\pgfqpoint{2.055402in}{1.518229in}}%
\pgfpathcurveto{\pgfqpoint{2.044352in}{1.518229in}}{\pgfqpoint{2.033753in}{1.513839in}}{\pgfqpoint{2.025939in}{1.506025in}}%
\pgfpathcurveto{\pgfqpoint{2.018125in}{1.498212in}}{\pgfqpoint{2.013735in}{1.487613in}}{\pgfqpoint{2.013735in}{1.476563in}}%
\pgfpathcurveto{\pgfqpoint{2.013735in}{1.465512in}}{\pgfqpoint{2.018125in}{1.454913in}}{\pgfqpoint{2.025939in}{1.447100in}}%
\pgfpathcurveto{\pgfqpoint{2.033753in}{1.439286in}}{\pgfqpoint{2.044352in}{1.434896in}}{\pgfqpoint{2.055402in}{1.434896in}}%
\pgfpathclose%
\pgfusepath{stroke,fill}%
\end{pgfscope}%
\begin{pgfscope}%
\pgfpathrectangle{\pgfqpoint{0.375000in}{0.330000in}}{\pgfqpoint{2.325000in}{2.310000in}}%
\pgfusepath{clip}%
\pgfsetbuttcap%
\pgfsetroundjoin%
\definecolor{currentfill}{rgb}{0.000000,0.000000,0.000000}%
\pgfsetfillcolor{currentfill}%
\pgfsetlinewidth{1.003750pt}%
\definecolor{currentstroke}{rgb}{0.000000,0.000000,0.000000}%
\pgfsetstrokecolor{currentstroke}%
\pgfsetdash{}{0pt}%
\pgfpathmoveto{\pgfqpoint{2.055402in}{1.382865in}}%
\pgfpathcurveto{\pgfqpoint{2.066452in}{1.382865in}}{\pgfqpoint{2.077051in}{1.387255in}}{\pgfqpoint{2.084865in}{1.395069in}}%
\pgfpathcurveto{\pgfqpoint{2.092678in}{1.402883in}}{\pgfqpoint{2.097069in}{1.413482in}}{\pgfqpoint{2.097069in}{1.424532in}}%
\pgfpathcurveto{\pgfqpoint{2.097069in}{1.435582in}}{\pgfqpoint{2.092678in}{1.446181in}}{\pgfqpoint{2.084865in}{1.453995in}}%
\pgfpathcurveto{\pgfqpoint{2.077051in}{1.461808in}}{\pgfqpoint{2.066452in}{1.466198in}}{\pgfqpoint{2.055402in}{1.466198in}}%
\pgfpathcurveto{\pgfqpoint{2.044352in}{1.466198in}}{\pgfqpoint{2.033753in}{1.461808in}}{\pgfqpoint{2.025939in}{1.453995in}}%
\pgfpathcurveto{\pgfqpoint{2.018125in}{1.446181in}}{\pgfqpoint{2.013735in}{1.435582in}}{\pgfqpoint{2.013735in}{1.424532in}}%
\pgfpathcurveto{\pgfqpoint{2.013735in}{1.413482in}}{\pgfqpoint{2.018125in}{1.402883in}}{\pgfqpoint{2.025939in}{1.395069in}}%
\pgfpathcurveto{\pgfqpoint{2.033753in}{1.387255in}}{\pgfqpoint{2.044352in}{1.382865in}}{\pgfqpoint{2.055402in}{1.382865in}}%
\pgfpathclose%
\pgfusepath{stroke,fill}%
\end{pgfscope}%
\begin{pgfscope}%
\pgfpathrectangle{\pgfqpoint{0.375000in}{0.330000in}}{\pgfqpoint{2.325000in}{2.310000in}}%
\pgfusepath{clip}%
\pgfsetbuttcap%
\pgfsetroundjoin%
\definecolor{currentfill}{rgb}{0.000000,0.000000,0.000000}%
\pgfsetfillcolor{currentfill}%
\pgfsetlinewidth{1.003750pt}%
\definecolor{currentstroke}{rgb}{0.000000,0.000000,0.000000}%
\pgfsetstrokecolor{currentstroke}%
\pgfsetdash{}{0pt}%
\pgfpathmoveto{\pgfqpoint{2.055402in}{1.486927in}}%
\pgfpathcurveto{\pgfqpoint{2.066452in}{1.486927in}}{\pgfqpoint{2.077051in}{1.491317in}}{\pgfqpoint{2.084865in}{1.499131in}}%
\pgfpathcurveto{\pgfqpoint{2.092678in}{1.506944in}}{\pgfqpoint{2.097069in}{1.517543in}}{\pgfqpoint{2.097069in}{1.528593in}}%
\pgfpathcurveto{\pgfqpoint{2.097069in}{1.539644in}}{\pgfqpoint{2.092678in}{1.550243in}}{\pgfqpoint{2.084865in}{1.558056in}}%
\pgfpathcurveto{\pgfqpoint{2.077051in}{1.565870in}}{\pgfqpoint{2.066452in}{1.570260in}}{\pgfqpoint{2.055402in}{1.570260in}}%
\pgfpathcurveto{\pgfqpoint{2.044352in}{1.570260in}}{\pgfqpoint{2.033753in}{1.565870in}}{\pgfqpoint{2.025939in}{1.558056in}}%
\pgfpathcurveto{\pgfqpoint{2.018125in}{1.550243in}}{\pgfqpoint{2.013735in}{1.539644in}}{\pgfqpoint{2.013735in}{1.528593in}}%
\pgfpathcurveto{\pgfqpoint{2.013735in}{1.517543in}}{\pgfqpoint{2.018125in}{1.506944in}}{\pgfqpoint{2.025939in}{1.499131in}}%
\pgfpathcurveto{\pgfqpoint{2.033753in}{1.491317in}}{\pgfqpoint{2.044352in}{1.486927in}}{\pgfqpoint{2.055402in}{1.486927in}}%
\pgfpathclose%
\pgfusepath{stroke,fill}%
\end{pgfscope}%
\begin{pgfscope}%
\pgfpathrectangle{\pgfqpoint{0.375000in}{0.330000in}}{\pgfqpoint{2.325000in}{2.310000in}}%
\pgfusepath{clip}%
\pgfsetbuttcap%
\pgfsetroundjoin%
\definecolor{currentfill}{rgb}{0.000000,0.000000,0.000000}%
\pgfsetfillcolor{currentfill}%
\pgfsetlinewidth{1.003750pt}%
\definecolor{currentstroke}{rgb}{0.000000,0.000000,0.000000}%
\pgfsetstrokecolor{currentstroke}%
\pgfsetdash{}{0pt}%
\pgfpathmoveto{\pgfqpoint{2.055402in}{1.486927in}}%
\pgfpathcurveto{\pgfqpoint{2.066452in}{1.486927in}}{\pgfqpoint{2.077051in}{1.491317in}}{\pgfqpoint{2.084865in}{1.499131in}}%
\pgfpathcurveto{\pgfqpoint{2.092678in}{1.506944in}}{\pgfqpoint{2.097069in}{1.517543in}}{\pgfqpoint{2.097069in}{1.528593in}}%
\pgfpathcurveto{\pgfqpoint{2.097069in}{1.539644in}}{\pgfqpoint{2.092678in}{1.550243in}}{\pgfqpoint{2.084865in}{1.558056in}}%
\pgfpathcurveto{\pgfqpoint{2.077051in}{1.565870in}}{\pgfqpoint{2.066452in}{1.570260in}}{\pgfqpoint{2.055402in}{1.570260in}}%
\pgfpathcurveto{\pgfqpoint{2.044352in}{1.570260in}}{\pgfqpoint{2.033753in}{1.565870in}}{\pgfqpoint{2.025939in}{1.558056in}}%
\pgfpathcurveto{\pgfqpoint{2.018125in}{1.550243in}}{\pgfqpoint{2.013735in}{1.539644in}}{\pgfqpoint{2.013735in}{1.528593in}}%
\pgfpathcurveto{\pgfqpoint{2.013735in}{1.517543in}}{\pgfqpoint{2.018125in}{1.506944in}}{\pgfqpoint{2.025939in}{1.499131in}}%
\pgfpathcurveto{\pgfqpoint{2.033753in}{1.491317in}}{\pgfqpoint{2.044352in}{1.486927in}}{\pgfqpoint{2.055402in}{1.486927in}}%
\pgfpathclose%
\pgfusepath{stroke,fill}%
\end{pgfscope}%
\begin{pgfscope}%
\pgfpathrectangle{\pgfqpoint{0.375000in}{0.330000in}}{\pgfqpoint{2.325000in}{2.310000in}}%
\pgfusepath{clip}%
\pgfsetbuttcap%
\pgfsetroundjoin%
\definecolor{currentfill}{rgb}{0.000000,0.000000,0.000000}%
\pgfsetfillcolor{currentfill}%
\pgfsetlinewidth{1.003750pt}%
\definecolor{currentstroke}{rgb}{0.000000,0.000000,0.000000}%
\pgfsetstrokecolor{currentstroke}%
\pgfsetdash{}{0pt}%
\pgfpathmoveto{\pgfqpoint{2.055402in}{1.486927in}}%
\pgfpathcurveto{\pgfqpoint{2.066452in}{1.486927in}}{\pgfqpoint{2.077051in}{1.491317in}}{\pgfqpoint{2.084865in}{1.499131in}}%
\pgfpathcurveto{\pgfqpoint{2.092678in}{1.506944in}}{\pgfqpoint{2.097069in}{1.517543in}}{\pgfqpoint{2.097069in}{1.528593in}}%
\pgfpathcurveto{\pgfqpoint{2.097069in}{1.539644in}}{\pgfqpoint{2.092678in}{1.550243in}}{\pgfqpoint{2.084865in}{1.558056in}}%
\pgfpathcurveto{\pgfqpoint{2.077051in}{1.565870in}}{\pgfqpoint{2.066452in}{1.570260in}}{\pgfqpoint{2.055402in}{1.570260in}}%
\pgfpathcurveto{\pgfqpoint{2.044352in}{1.570260in}}{\pgfqpoint{2.033753in}{1.565870in}}{\pgfqpoint{2.025939in}{1.558056in}}%
\pgfpathcurveto{\pgfqpoint{2.018125in}{1.550243in}}{\pgfqpoint{2.013735in}{1.539644in}}{\pgfqpoint{2.013735in}{1.528593in}}%
\pgfpathcurveto{\pgfqpoint{2.013735in}{1.517543in}}{\pgfqpoint{2.018125in}{1.506944in}}{\pgfqpoint{2.025939in}{1.499131in}}%
\pgfpathcurveto{\pgfqpoint{2.033753in}{1.491317in}}{\pgfqpoint{2.044352in}{1.486927in}}{\pgfqpoint{2.055402in}{1.486927in}}%
\pgfpathclose%
\pgfusepath{stroke,fill}%
\end{pgfscope}%
\begin{pgfscope}%
\pgfpathrectangle{\pgfqpoint{0.375000in}{0.330000in}}{\pgfqpoint{2.325000in}{2.310000in}}%
\pgfusepath{clip}%
\pgfsetbuttcap%
\pgfsetroundjoin%
\definecolor{currentfill}{rgb}{0.000000,0.000000,0.000000}%
\pgfsetfillcolor{currentfill}%
\pgfsetlinewidth{1.003750pt}%
\definecolor{currentstroke}{rgb}{0.000000,0.000000,0.000000}%
\pgfsetstrokecolor{currentstroke}%
\pgfsetdash{}{0pt}%
\pgfpathmoveto{\pgfqpoint{2.055402in}{1.538958in}}%
\pgfpathcurveto{\pgfqpoint{2.066452in}{1.538958in}}{\pgfqpoint{2.077051in}{1.543348in}}{\pgfqpoint{2.084865in}{1.551161in}}%
\pgfpathcurveto{\pgfqpoint{2.092678in}{1.558975in}}{\pgfqpoint{2.097069in}{1.569574in}}{\pgfqpoint{2.097069in}{1.580624in}}%
\pgfpathcurveto{\pgfqpoint{2.097069in}{1.591674in}}{\pgfqpoint{2.092678in}{1.602273in}}{\pgfqpoint{2.084865in}{1.610087in}}%
\pgfpathcurveto{\pgfqpoint{2.077051in}{1.617901in}}{\pgfqpoint{2.066452in}{1.622291in}}{\pgfqpoint{2.055402in}{1.622291in}}%
\pgfpathcurveto{\pgfqpoint{2.044352in}{1.622291in}}{\pgfqpoint{2.033753in}{1.617901in}}{\pgfqpoint{2.025939in}{1.610087in}}%
\pgfpathcurveto{\pgfqpoint{2.018125in}{1.602273in}}{\pgfqpoint{2.013735in}{1.591674in}}{\pgfqpoint{2.013735in}{1.580624in}}%
\pgfpathcurveto{\pgfqpoint{2.013735in}{1.569574in}}{\pgfqpoint{2.018125in}{1.558975in}}{\pgfqpoint{2.025939in}{1.551161in}}%
\pgfpathcurveto{\pgfqpoint{2.033753in}{1.543348in}}{\pgfqpoint{2.044352in}{1.538958in}}{\pgfqpoint{2.055402in}{1.538958in}}%
\pgfpathclose%
\pgfusepath{stroke,fill}%
\end{pgfscope}%
\begin{pgfscope}%
\pgfpathrectangle{\pgfqpoint{0.375000in}{0.330000in}}{\pgfqpoint{2.325000in}{2.310000in}}%
\pgfusepath{clip}%
\pgfsetbuttcap%
\pgfsetroundjoin%
\definecolor{currentfill}{rgb}{0.000000,0.000000,0.000000}%
\pgfsetfillcolor{currentfill}%
\pgfsetlinewidth{1.003750pt}%
\definecolor{currentstroke}{rgb}{0.000000,0.000000,0.000000}%
\pgfsetstrokecolor{currentstroke}%
\pgfsetdash{}{0pt}%
\pgfpathmoveto{\pgfqpoint{2.055402in}{1.486927in}}%
\pgfpathcurveto{\pgfqpoint{2.066452in}{1.486927in}}{\pgfqpoint{2.077051in}{1.491317in}}{\pgfqpoint{2.084865in}{1.499131in}}%
\pgfpathcurveto{\pgfqpoint{2.092678in}{1.506944in}}{\pgfqpoint{2.097069in}{1.517543in}}{\pgfqpoint{2.097069in}{1.528593in}}%
\pgfpathcurveto{\pgfqpoint{2.097069in}{1.539644in}}{\pgfqpoint{2.092678in}{1.550243in}}{\pgfqpoint{2.084865in}{1.558056in}}%
\pgfpathcurveto{\pgfqpoint{2.077051in}{1.565870in}}{\pgfqpoint{2.066452in}{1.570260in}}{\pgfqpoint{2.055402in}{1.570260in}}%
\pgfpathcurveto{\pgfqpoint{2.044352in}{1.570260in}}{\pgfqpoint{2.033753in}{1.565870in}}{\pgfqpoint{2.025939in}{1.558056in}}%
\pgfpathcurveto{\pgfqpoint{2.018125in}{1.550243in}}{\pgfqpoint{2.013735in}{1.539644in}}{\pgfqpoint{2.013735in}{1.528593in}}%
\pgfpathcurveto{\pgfqpoint{2.013735in}{1.517543in}}{\pgfqpoint{2.018125in}{1.506944in}}{\pgfqpoint{2.025939in}{1.499131in}}%
\pgfpathcurveto{\pgfqpoint{2.033753in}{1.491317in}}{\pgfqpoint{2.044352in}{1.486927in}}{\pgfqpoint{2.055402in}{1.486927in}}%
\pgfpathclose%
\pgfusepath{stroke,fill}%
\end{pgfscope}%
\begin{pgfscope}%
\pgfpathrectangle{\pgfqpoint{0.375000in}{0.330000in}}{\pgfqpoint{2.325000in}{2.310000in}}%
\pgfusepath{clip}%
\pgfsetbuttcap%
\pgfsetroundjoin%
\definecolor{currentfill}{rgb}{0.000000,0.000000,0.000000}%
\pgfsetfillcolor{currentfill}%
\pgfsetlinewidth{1.003750pt}%
\definecolor{currentstroke}{rgb}{0.000000,0.000000,0.000000}%
\pgfsetstrokecolor{currentstroke}%
\pgfsetdash{}{0pt}%
\pgfpathmoveto{\pgfqpoint{2.055402in}{1.486927in}}%
\pgfpathcurveto{\pgfqpoint{2.066452in}{1.486927in}}{\pgfqpoint{2.077051in}{1.491317in}}{\pgfqpoint{2.084865in}{1.499131in}}%
\pgfpathcurveto{\pgfqpoint{2.092678in}{1.506944in}}{\pgfqpoint{2.097069in}{1.517543in}}{\pgfqpoint{2.097069in}{1.528593in}}%
\pgfpathcurveto{\pgfqpoint{2.097069in}{1.539644in}}{\pgfqpoint{2.092678in}{1.550243in}}{\pgfqpoint{2.084865in}{1.558056in}}%
\pgfpathcurveto{\pgfqpoint{2.077051in}{1.565870in}}{\pgfqpoint{2.066452in}{1.570260in}}{\pgfqpoint{2.055402in}{1.570260in}}%
\pgfpathcurveto{\pgfqpoint{2.044352in}{1.570260in}}{\pgfqpoint{2.033753in}{1.565870in}}{\pgfqpoint{2.025939in}{1.558056in}}%
\pgfpathcurveto{\pgfqpoint{2.018125in}{1.550243in}}{\pgfqpoint{2.013735in}{1.539644in}}{\pgfqpoint{2.013735in}{1.528593in}}%
\pgfpathcurveto{\pgfqpoint{2.013735in}{1.517543in}}{\pgfqpoint{2.018125in}{1.506944in}}{\pgfqpoint{2.025939in}{1.499131in}}%
\pgfpathcurveto{\pgfqpoint{2.033753in}{1.491317in}}{\pgfqpoint{2.044352in}{1.486927in}}{\pgfqpoint{2.055402in}{1.486927in}}%
\pgfpathclose%
\pgfusepath{stroke,fill}%
\end{pgfscope}%
\begin{pgfscope}%
\pgfpathrectangle{\pgfqpoint{0.375000in}{0.330000in}}{\pgfqpoint{2.325000in}{2.310000in}}%
\pgfusepath{clip}%
\pgfsetbuttcap%
\pgfsetroundjoin%
\definecolor{currentfill}{rgb}{0.000000,0.000000,0.000000}%
\pgfsetfillcolor{currentfill}%
\pgfsetlinewidth{1.003750pt}%
\definecolor{currentstroke}{rgb}{0.000000,0.000000,0.000000}%
\pgfsetstrokecolor{currentstroke}%
\pgfsetdash{}{0pt}%
\pgfpathmoveto{\pgfqpoint{2.055402in}{1.538958in}}%
\pgfpathcurveto{\pgfqpoint{2.066452in}{1.538958in}}{\pgfqpoint{2.077051in}{1.543348in}}{\pgfqpoint{2.084865in}{1.551161in}}%
\pgfpathcurveto{\pgfqpoint{2.092678in}{1.558975in}}{\pgfqpoint{2.097069in}{1.569574in}}{\pgfqpoint{2.097069in}{1.580624in}}%
\pgfpathcurveto{\pgfqpoint{2.097069in}{1.591674in}}{\pgfqpoint{2.092678in}{1.602273in}}{\pgfqpoint{2.084865in}{1.610087in}}%
\pgfpathcurveto{\pgfqpoint{2.077051in}{1.617901in}}{\pgfqpoint{2.066452in}{1.622291in}}{\pgfqpoint{2.055402in}{1.622291in}}%
\pgfpathcurveto{\pgfqpoint{2.044352in}{1.622291in}}{\pgfqpoint{2.033753in}{1.617901in}}{\pgfqpoint{2.025939in}{1.610087in}}%
\pgfpathcurveto{\pgfqpoint{2.018125in}{1.602273in}}{\pgfqpoint{2.013735in}{1.591674in}}{\pgfqpoint{2.013735in}{1.580624in}}%
\pgfpathcurveto{\pgfqpoint{2.013735in}{1.569574in}}{\pgfqpoint{2.018125in}{1.558975in}}{\pgfqpoint{2.025939in}{1.551161in}}%
\pgfpathcurveto{\pgfqpoint{2.033753in}{1.543348in}}{\pgfqpoint{2.044352in}{1.538958in}}{\pgfqpoint{2.055402in}{1.538958in}}%
\pgfpathclose%
\pgfusepath{stroke,fill}%
\end{pgfscope}%
\begin{pgfscope}%
\pgfpathrectangle{\pgfqpoint{0.375000in}{0.330000in}}{\pgfqpoint{2.325000in}{2.310000in}}%
\pgfusepath{clip}%
\pgfsetbuttcap%
\pgfsetroundjoin%
\definecolor{currentfill}{rgb}{0.000000,0.000000,0.000000}%
\pgfsetfillcolor{currentfill}%
\pgfsetlinewidth{1.003750pt}%
\definecolor{currentstroke}{rgb}{0.000000,0.000000,0.000000}%
\pgfsetstrokecolor{currentstroke}%
\pgfsetdash{}{0pt}%
\pgfpathmoveto{\pgfqpoint{2.055402in}{1.538958in}}%
\pgfpathcurveto{\pgfqpoint{2.066452in}{1.538958in}}{\pgfqpoint{2.077051in}{1.543348in}}{\pgfqpoint{2.084865in}{1.551161in}}%
\pgfpathcurveto{\pgfqpoint{2.092678in}{1.558975in}}{\pgfqpoint{2.097069in}{1.569574in}}{\pgfqpoint{2.097069in}{1.580624in}}%
\pgfpathcurveto{\pgfqpoint{2.097069in}{1.591674in}}{\pgfqpoint{2.092678in}{1.602273in}}{\pgfqpoint{2.084865in}{1.610087in}}%
\pgfpathcurveto{\pgfqpoint{2.077051in}{1.617901in}}{\pgfqpoint{2.066452in}{1.622291in}}{\pgfqpoint{2.055402in}{1.622291in}}%
\pgfpathcurveto{\pgfqpoint{2.044352in}{1.622291in}}{\pgfqpoint{2.033753in}{1.617901in}}{\pgfqpoint{2.025939in}{1.610087in}}%
\pgfpathcurveto{\pgfqpoint{2.018125in}{1.602273in}}{\pgfqpoint{2.013735in}{1.591674in}}{\pgfqpoint{2.013735in}{1.580624in}}%
\pgfpathcurveto{\pgfqpoint{2.013735in}{1.569574in}}{\pgfqpoint{2.018125in}{1.558975in}}{\pgfqpoint{2.025939in}{1.551161in}}%
\pgfpathcurveto{\pgfqpoint{2.033753in}{1.543348in}}{\pgfqpoint{2.044352in}{1.538958in}}{\pgfqpoint{2.055402in}{1.538958in}}%
\pgfpathclose%
\pgfusepath{stroke,fill}%
\end{pgfscope}%
\begin{pgfscope}%
\pgfpathrectangle{\pgfqpoint{0.375000in}{0.330000in}}{\pgfqpoint{2.325000in}{2.310000in}}%
\pgfusepath{clip}%
\pgfsetbuttcap%
\pgfsetroundjoin%
\definecolor{currentfill}{rgb}{0.000000,0.000000,0.000000}%
\pgfsetfillcolor{currentfill}%
\pgfsetlinewidth{1.003750pt}%
\definecolor{currentstroke}{rgb}{0.000000,0.000000,0.000000}%
\pgfsetstrokecolor{currentstroke}%
\pgfsetdash{}{0pt}%
\pgfpathmoveto{\pgfqpoint{2.055402in}{1.486927in}}%
\pgfpathcurveto{\pgfqpoint{2.066452in}{1.486927in}}{\pgfqpoint{2.077051in}{1.491317in}}{\pgfqpoint{2.084865in}{1.499131in}}%
\pgfpathcurveto{\pgfqpoint{2.092678in}{1.506944in}}{\pgfqpoint{2.097069in}{1.517543in}}{\pgfqpoint{2.097069in}{1.528593in}}%
\pgfpathcurveto{\pgfqpoint{2.097069in}{1.539644in}}{\pgfqpoint{2.092678in}{1.550243in}}{\pgfqpoint{2.084865in}{1.558056in}}%
\pgfpathcurveto{\pgfqpoint{2.077051in}{1.565870in}}{\pgfqpoint{2.066452in}{1.570260in}}{\pgfqpoint{2.055402in}{1.570260in}}%
\pgfpathcurveto{\pgfqpoint{2.044352in}{1.570260in}}{\pgfqpoint{2.033753in}{1.565870in}}{\pgfqpoint{2.025939in}{1.558056in}}%
\pgfpathcurveto{\pgfqpoint{2.018125in}{1.550243in}}{\pgfqpoint{2.013735in}{1.539644in}}{\pgfqpoint{2.013735in}{1.528593in}}%
\pgfpathcurveto{\pgfqpoint{2.013735in}{1.517543in}}{\pgfqpoint{2.018125in}{1.506944in}}{\pgfqpoint{2.025939in}{1.499131in}}%
\pgfpathcurveto{\pgfqpoint{2.033753in}{1.491317in}}{\pgfqpoint{2.044352in}{1.486927in}}{\pgfqpoint{2.055402in}{1.486927in}}%
\pgfpathclose%
\pgfusepath{stroke,fill}%
\end{pgfscope}%
\begin{pgfscope}%
\pgfpathrectangle{\pgfqpoint{0.375000in}{0.330000in}}{\pgfqpoint{2.325000in}{2.310000in}}%
\pgfusepath{clip}%
\pgfsetbuttcap%
\pgfsetroundjoin%
\definecolor{currentfill}{rgb}{0.000000,0.000000,0.000000}%
\pgfsetfillcolor{currentfill}%
\pgfsetlinewidth{1.003750pt}%
\definecolor{currentstroke}{rgb}{0.000000,0.000000,0.000000}%
\pgfsetstrokecolor{currentstroke}%
\pgfsetdash{}{0pt}%
\pgfpathmoveto{\pgfqpoint{2.055402in}{1.434896in}}%
\pgfpathcurveto{\pgfqpoint{2.066452in}{1.434896in}}{\pgfqpoint{2.077051in}{1.439286in}}{\pgfqpoint{2.084865in}{1.447100in}}%
\pgfpathcurveto{\pgfqpoint{2.092678in}{1.454913in}}{\pgfqpoint{2.097069in}{1.465512in}}{\pgfqpoint{2.097069in}{1.476563in}}%
\pgfpathcurveto{\pgfqpoint{2.097069in}{1.487613in}}{\pgfqpoint{2.092678in}{1.498212in}}{\pgfqpoint{2.084865in}{1.506025in}}%
\pgfpathcurveto{\pgfqpoint{2.077051in}{1.513839in}}{\pgfqpoint{2.066452in}{1.518229in}}{\pgfqpoint{2.055402in}{1.518229in}}%
\pgfpathcurveto{\pgfqpoint{2.044352in}{1.518229in}}{\pgfqpoint{2.033753in}{1.513839in}}{\pgfqpoint{2.025939in}{1.506025in}}%
\pgfpathcurveto{\pgfqpoint{2.018125in}{1.498212in}}{\pgfqpoint{2.013735in}{1.487613in}}{\pgfqpoint{2.013735in}{1.476563in}}%
\pgfpathcurveto{\pgfqpoint{2.013735in}{1.465512in}}{\pgfqpoint{2.018125in}{1.454913in}}{\pgfqpoint{2.025939in}{1.447100in}}%
\pgfpathcurveto{\pgfqpoint{2.033753in}{1.439286in}}{\pgfqpoint{2.044352in}{1.434896in}}{\pgfqpoint{2.055402in}{1.434896in}}%
\pgfpathclose%
\pgfusepath{stroke,fill}%
\end{pgfscope}%
\begin{pgfscope}%
\pgfpathrectangle{\pgfqpoint{0.375000in}{0.330000in}}{\pgfqpoint{2.325000in}{2.310000in}}%
\pgfusepath{clip}%
\pgfsetbuttcap%
\pgfsetroundjoin%
\definecolor{currentfill}{rgb}{0.000000,0.000000,0.000000}%
\pgfsetfillcolor{currentfill}%
\pgfsetlinewidth{1.003750pt}%
\definecolor{currentstroke}{rgb}{0.000000,0.000000,0.000000}%
\pgfsetstrokecolor{currentstroke}%
\pgfsetdash{}{0pt}%
\pgfpathmoveto{\pgfqpoint{2.055402in}{1.486927in}}%
\pgfpathcurveto{\pgfqpoint{2.066452in}{1.486927in}}{\pgfqpoint{2.077051in}{1.491317in}}{\pgfqpoint{2.084865in}{1.499131in}}%
\pgfpathcurveto{\pgfqpoint{2.092678in}{1.506944in}}{\pgfqpoint{2.097069in}{1.517543in}}{\pgfqpoint{2.097069in}{1.528593in}}%
\pgfpathcurveto{\pgfqpoint{2.097069in}{1.539644in}}{\pgfqpoint{2.092678in}{1.550243in}}{\pgfqpoint{2.084865in}{1.558056in}}%
\pgfpathcurveto{\pgfqpoint{2.077051in}{1.565870in}}{\pgfqpoint{2.066452in}{1.570260in}}{\pgfqpoint{2.055402in}{1.570260in}}%
\pgfpathcurveto{\pgfqpoint{2.044352in}{1.570260in}}{\pgfqpoint{2.033753in}{1.565870in}}{\pgfqpoint{2.025939in}{1.558056in}}%
\pgfpathcurveto{\pgfqpoint{2.018125in}{1.550243in}}{\pgfqpoint{2.013735in}{1.539644in}}{\pgfqpoint{2.013735in}{1.528593in}}%
\pgfpathcurveto{\pgfqpoint{2.013735in}{1.517543in}}{\pgfqpoint{2.018125in}{1.506944in}}{\pgfqpoint{2.025939in}{1.499131in}}%
\pgfpathcurveto{\pgfqpoint{2.033753in}{1.491317in}}{\pgfqpoint{2.044352in}{1.486927in}}{\pgfqpoint{2.055402in}{1.486927in}}%
\pgfpathclose%
\pgfusepath{stroke,fill}%
\end{pgfscope}%
\begin{pgfscope}%
\pgfpathrectangle{\pgfqpoint{0.375000in}{0.330000in}}{\pgfqpoint{2.325000in}{2.310000in}}%
\pgfusepath{clip}%
\pgfsetbuttcap%
\pgfsetroundjoin%
\definecolor{currentfill}{rgb}{0.000000,0.000000,0.000000}%
\pgfsetfillcolor{currentfill}%
\pgfsetlinewidth{1.003750pt}%
\definecolor{currentstroke}{rgb}{0.000000,0.000000,0.000000}%
\pgfsetstrokecolor{currentstroke}%
\pgfsetdash{}{0pt}%
\pgfpathmoveto{\pgfqpoint{2.055402in}{1.486927in}}%
\pgfpathcurveto{\pgfqpoint{2.066452in}{1.486927in}}{\pgfqpoint{2.077051in}{1.491317in}}{\pgfqpoint{2.084865in}{1.499131in}}%
\pgfpathcurveto{\pgfqpoint{2.092678in}{1.506944in}}{\pgfqpoint{2.097069in}{1.517543in}}{\pgfqpoint{2.097069in}{1.528593in}}%
\pgfpathcurveto{\pgfqpoint{2.097069in}{1.539644in}}{\pgfqpoint{2.092678in}{1.550243in}}{\pgfqpoint{2.084865in}{1.558056in}}%
\pgfpathcurveto{\pgfqpoint{2.077051in}{1.565870in}}{\pgfqpoint{2.066452in}{1.570260in}}{\pgfqpoint{2.055402in}{1.570260in}}%
\pgfpathcurveto{\pgfqpoint{2.044352in}{1.570260in}}{\pgfqpoint{2.033753in}{1.565870in}}{\pgfqpoint{2.025939in}{1.558056in}}%
\pgfpathcurveto{\pgfqpoint{2.018125in}{1.550243in}}{\pgfqpoint{2.013735in}{1.539644in}}{\pgfqpoint{2.013735in}{1.528593in}}%
\pgfpathcurveto{\pgfqpoint{2.013735in}{1.517543in}}{\pgfqpoint{2.018125in}{1.506944in}}{\pgfqpoint{2.025939in}{1.499131in}}%
\pgfpathcurveto{\pgfqpoint{2.033753in}{1.491317in}}{\pgfqpoint{2.044352in}{1.486927in}}{\pgfqpoint{2.055402in}{1.486927in}}%
\pgfpathclose%
\pgfusepath{stroke,fill}%
\end{pgfscope}%
\begin{pgfscope}%
\pgfpathrectangle{\pgfqpoint{0.375000in}{0.330000in}}{\pgfqpoint{2.325000in}{2.310000in}}%
\pgfusepath{clip}%
\pgfsetbuttcap%
\pgfsetroundjoin%
\definecolor{currentfill}{rgb}{0.000000,0.000000,0.000000}%
\pgfsetfillcolor{currentfill}%
\pgfsetlinewidth{1.003750pt}%
\definecolor{currentstroke}{rgb}{0.000000,0.000000,0.000000}%
\pgfsetstrokecolor{currentstroke}%
\pgfsetdash{}{0pt}%
\pgfpathmoveto{\pgfqpoint{2.055402in}{1.538958in}}%
\pgfpathcurveto{\pgfqpoint{2.066452in}{1.538958in}}{\pgfqpoint{2.077051in}{1.543348in}}{\pgfqpoint{2.084865in}{1.551161in}}%
\pgfpathcurveto{\pgfqpoint{2.092678in}{1.558975in}}{\pgfqpoint{2.097069in}{1.569574in}}{\pgfqpoint{2.097069in}{1.580624in}}%
\pgfpathcurveto{\pgfqpoint{2.097069in}{1.591674in}}{\pgfqpoint{2.092678in}{1.602273in}}{\pgfqpoint{2.084865in}{1.610087in}}%
\pgfpathcurveto{\pgfqpoint{2.077051in}{1.617901in}}{\pgfqpoint{2.066452in}{1.622291in}}{\pgfqpoint{2.055402in}{1.622291in}}%
\pgfpathcurveto{\pgfqpoint{2.044352in}{1.622291in}}{\pgfqpoint{2.033753in}{1.617901in}}{\pgfqpoint{2.025939in}{1.610087in}}%
\pgfpathcurveto{\pgfqpoint{2.018125in}{1.602273in}}{\pgfqpoint{2.013735in}{1.591674in}}{\pgfqpoint{2.013735in}{1.580624in}}%
\pgfpathcurveto{\pgfqpoint{2.013735in}{1.569574in}}{\pgfqpoint{2.018125in}{1.558975in}}{\pgfqpoint{2.025939in}{1.551161in}}%
\pgfpathcurveto{\pgfqpoint{2.033753in}{1.543348in}}{\pgfqpoint{2.044352in}{1.538958in}}{\pgfqpoint{2.055402in}{1.538958in}}%
\pgfpathclose%
\pgfusepath{stroke,fill}%
\end{pgfscope}%
\begin{pgfscope}%
\pgfpathrectangle{\pgfqpoint{0.375000in}{0.330000in}}{\pgfqpoint{2.325000in}{2.310000in}}%
\pgfusepath{clip}%
\pgfsetbuttcap%
\pgfsetroundjoin%
\definecolor{currentfill}{rgb}{0.000000,0.000000,0.000000}%
\pgfsetfillcolor{currentfill}%
\pgfsetlinewidth{1.003750pt}%
\definecolor{currentstroke}{rgb}{0.000000,0.000000,0.000000}%
\pgfsetstrokecolor{currentstroke}%
\pgfsetdash{}{0pt}%
\pgfpathmoveto{\pgfqpoint{2.055402in}{1.486927in}}%
\pgfpathcurveto{\pgfqpoint{2.066452in}{1.486927in}}{\pgfqpoint{2.077051in}{1.491317in}}{\pgfqpoint{2.084865in}{1.499131in}}%
\pgfpathcurveto{\pgfqpoint{2.092678in}{1.506944in}}{\pgfqpoint{2.097069in}{1.517543in}}{\pgfqpoint{2.097069in}{1.528593in}}%
\pgfpathcurveto{\pgfqpoint{2.097069in}{1.539644in}}{\pgfqpoint{2.092678in}{1.550243in}}{\pgfqpoint{2.084865in}{1.558056in}}%
\pgfpathcurveto{\pgfqpoint{2.077051in}{1.565870in}}{\pgfqpoint{2.066452in}{1.570260in}}{\pgfqpoint{2.055402in}{1.570260in}}%
\pgfpathcurveto{\pgfqpoint{2.044352in}{1.570260in}}{\pgfqpoint{2.033753in}{1.565870in}}{\pgfqpoint{2.025939in}{1.558056in}}%
\pgfpathcurveto{\pgfqpoint{2.018125in}{1.550243in}}{\pgfqpoint{2.013735in}{1.539644in}}{\pgfqpoint{2.013735in}{1.528593in}}%
\pgfpathcurveto{\pgfqpoint{2.013735in}{1.517543in}}{\pgfqpoint{2.018125in}{1.506944in}}{\pgfqpoint{2.025939in}{1.499131in}}%
\pgfpathcurveto{\pgfqpoint{2.033753in}{1.491317in}}{\pgfqpoint{2.044352in}{1.486927in}}{\pgfqpoint{2.055402in}{1.486927in}}%
\pgfpathclose%
\pgfusepath{stroke,fill}%
\end{pgfscope}%
\begin{pgfscope}%
\pgfpathrectangle{\pgfqpoint{0.375000in}{0.330000in}}{\pgfqpoint{2.325000in}{2.310000in}}%
\pgfusepath{clip}%
\pgfsetbuttcap%
\pgfsetroundjoin%
\definecolor{currentfill}{rgb}{0.000000,0.000000,0.000000}%
\pgfsetfillcolor{currentfill}%
\pgfsetlinewidth{1.003750pt}%
\definecolor{currentstroke}{rgb}{0.000000,0.000000,0.000000}%
\pgfsetstrokecolor{currentstroke}%
\pgfsetdash{}{0pt}%
\pgfpathmoveto{\pgfqpoint{2.055402in}{1.486927in}}%
\pgfpathcurveto{\pgfqpoint{2.066452in}{1.486927in}}{\pgfqpoint{2.077051in}{1.491317in}}{\pgfqpoint{2.084865in}{1.499131in}}%
\pgfpathcurveto{\pgfqpoint{2.092678in}{1.506944in}}{\pgfqpoint{2.097069in}{1.517543in}}{\pgfqpoint{2.097069in}{1.528593in}}%
\pgfpathcurveto{\pgfqpoint{2.097069in}{1.539644in}}{\pgfqpoint{2.092678in}{1.550243in}}{\pgfqpoint{2.084865in}{1.558056in}}%
\pgfpathcurveto{\pgfqpoint{2.077051in}{1.565870in}}{\pgfqpoint{2.066452in}{1.570260in}}{\pgfqpoint{2.055402in}{1.570260in}}%
\pgfpathcurveto{\pgfqpoint{2.044352in}{1.570260in}}{\pgfqpoint{2.033753in}{1.565870in}}{\pgfqpoint{2.025939in}{1.558056in}}%
\pgfpathcurveto{\pgfqpoint{2.018125in}{1.550243in}}{\pgfqpoint{2.013735in}{1.539644in}}{\pgfqpoint{2.013735in}{1.528593in}}%
\pgfpathcurveto{\pgfqpoint{2.013735in}{1.517543in}}{\pgfqpoint{2.018125in}{1.506944in}}{\pgfqpoint{2.025939in}{1.499131in}}%
\pgfpathcurveto{\pgfqpoint{2.033753in}{1.491317in}}{\pgfqpoint{2.044352in}{1.486927in}}{\pgfqpoint{2.055402in}{1.486927in}}%
\pgfpathclose%
\pgfusepath{stroke,fill}%
\end{pgfscope}%
\begin{pgfscope}%
\pgfpathrectangle{\pgfqpoint{0.375000in}{0.330000in}}{\pgfqpoint{2.325000in}{2.310000in}}%
\pgfusepath{clip}%
\pgfsetbuttcap%
\pgfsetroundjoin%
\definecolor{currentfill}{rgb}{0.000000,0.000000,0.000000}%
\pgfsetfillcolor{currentfill}%
\pgfsetlinewidth{1.003750pt}%
\definecolor{currentstroke}{rgb}{0.000000,0.000000,0.000000}%
\pgfsetstrokecolor{currentstroke}%
\pgfsetdash{}{0pt}%
\pgfpathmoveto{\pgfqpoint{2.055402in}{1.486927in}}%
\pgfpathcurveto{\pgfqpoint{2.066452in}{1.486927in}}{\pgfqpoint{2.077051in}{1.491317in}}{\pgfqpoint{2.084865in}{1.499131in}}%
\pgfpathcurveto{\pgfqpoint{2.092678in}{1.506944in}}{\pgfqpoint{2.097069in}{1.517543in}}{\pgfqpoint{2.097069in}{1.528593in}}%
\pgfpathcurveto{\pgfqpoint{2.097069in}{1.539644in}}{\pgfqpoint{2.092678in}{1.550243in}}{\pgfqpoint{2.084865in}{1.558056in}}%
\pgfpathcurveto{\pgfqpoint{2.077051in}{1.565870in}}{\pgfqpoint{2.066452in}{1.570260in}}{\pgfqpoint{2.055402in}{1.570260in}}%
\pgfpathcurveto{\pgfqpoint{2.044352in}{1.570260in}}{\pgfqpoint{2.033753in}{1.565870in}}{\pgfqpoint{2.025939in}{1.558056in}}%
\pgfpathcurveto{\pgfqpoint{2.018125in}{1.550243in}}{\pgfqpoint{2.013735in}{1.539644in}}{\pgfqpoint{2.013735in}{1.528593in}}%
\pgfpathcurveto{\pgfqpoint{2.013735in}{1.517543in}}{\pgfqpoint{2.018125in}{1.506944in}}{\pgfqpoint{2.025939in}{1.499131in}}%
\pgfpathcurveto{\pgfqpoint{2.033753in}{1.491317in}}{\pgfqpoint{2.044352in}{1.486927in}}{\pgfqpoint{2.055402in}{1.486927in}}%
\pgfpathclose%
\pgfusepath{stroke,fill}%
\end{pgfscope}%
\begin{pgfscope}%
\pgfpathrectangle{\pgfqpoint{0.375000in}{0.330000in}}{\pgfqpoint{2.325000in}{2.310000in}}%
\pgfusepath{clip}%
\pgfsetbuttcap%
\pgfsetroundjoin%
\definecolor{currentfill}{rgb}{0.000000,0.000000,0.000000}%
\pgfsetfillcolor{currentfill}%
\pgfsetlinewidth{1.003750pt}%
\definecolor{currentstroke}{rgb}{0.000000,0.000000,0.000000}%
\pgfsetstrokecolor{currentstroke}%
\pgfsetdash{}{0pt}%
\pgfpathmoveto{\pgfqpoint{2.055402in}{1.486927in}}%
\pgfpathcurveto{\pgfqpoint{2.066452in}{1.486927in}}{\pgfqpoint{2.077051in}{1.491317in}}{\pgfqpoint{2.084865in}{1.499131in}}%
\pgfpathcurveto{\pgfqpoint{2.092678in}{1.506944in}}{\pgfqpoint{2.097069in}{1.517543in}}{\pgfqpoint{2.097069in}{1.528593in}}%
\pgfpathcurveto{\pgfqpoint{2.097069in}{1.539644in}}{\pgfqpoint{2.092678in}{1.550243in}}{\pgfqpoint{2.084865in}{1.558056in}}%
\pgfpathcurveto{\pgfqpoint{2.077051in}{1.565870in}}{\pgfqpoint{2.066452in}{1.570260in}}{\pgfqpoint{2.055402in}{1.570260in}}%
\pgfpathcurveto{\pgfqpoint{2.044352in}{1.570260in}}{\pgfqpoint{2.033753in}{1.565870in}}{\pgfqpoint{2.025939in}{1.558056in}}%
\pgfpathcurveto{\pgfqpoint{2.018125in}{1.550243in}}{\pgfqpoint{2.013735in}{1.539644in}}{\pgfqpoint{2.013735in}{1.528593in}}%
\pgfpathcurveto{\pgfqpoint{2.013735in}{1.517543in}}{\pgfqpoint{2.018125in}{1.506944in}}{\pgfqpoint{2.025939in}{1.499131in}}%
\pgfpathcurveto{\pgfqpoint{2.033753in}{1.491317in}}{\pgfqpoint{2.044352in}{1.486927in}}{\pgfqpoint{2.055402in}{1.486927in}}%
\pgfpathclose%
\pgfusepath{stroke,fill}%
\end{pgfscope}%
\begin{pgfscope}%
\pgfpathrectangle{\pgfqpoint{0.375000in}{0.330000in}}{\pgfqpoint{2.325000in}{2.310000in}}%
\pgfusepath{clip}%
\pgfsetbuttcap%
\pgfsetroundjoin%
\definecolor{currentfill}{rgb}{0.000000,0.000000,0.000000}%
\pgfsetfillcolor{currentfill}%
\pgfsetlinewidth{1.003750pt}%
\definecolor{currentstroke}{rgb}{0.000000,0.000000,0.000000}%
\pgfsetstrokecolor{currentstroke}%
\pgfsetdash{}{0pt}%
\pgfpathmoveto{\pgfqpoint{2.055402in}{1.590988in}}%
\pgfpathcurveto{\pgfqpoint{2.066452in}{1.590988in}}{\pgfqpoint{2.077051in}{1.595379in}}{\pgfqpoint{2.084865in}{1.603192in}}%
\pgfpathcurveto{\pgfqpoint{2.092678in}{1.611006in}}{\pgfqpoint{2.097069in}{1.621605in}}{\pgfqpoint{2.097069in}{1.632655in}}%
\pgfpathcurveto{\pgfqpoint{2.097069in}{1.643705in}}{\pgfqpoint{2.092678in}{1.654304in}}{\pgfqpoint{2.084865in}{1.662118in}}%
\pgfpathcurveto{\pgfqpoint{2.077051in}{1.669931in}}{\pgfqpoint{2.066452in}{1.674322in}}{\pgfqpoint{2.055402in}{1.674322in}}%
\pgfpathcurveto{\pgfqpoint{2.044352in}{1.674322in}}{\pgfqpoint{2.033753in}{1.669931in}}{\pgfqpoint{2.025939in}{1.662118in}}%
\pgfpathcurveto{\pgfqpoint{2.018125in}{1.654304in}}{\pgfqpoint{2.013735in}{1.643705in}}{\pgfqpoint{2.013735in}{1.632655in}}%
\pgfpathcurveto{\pgfqpoint{2.013735in}{1.621605in}}{\pgfqpoint{2.018125in}{1.611006in}}{\pgfqpoint{2.025939in}{1.603192in}}%
\pgfpathcurveto{\pgfqpoint{2.033753in}{1.595379in}}{\pgfqpoint{2.044352in}{1.590988in}}{\pgfqpoint{2.055402in}{1.590988in}}%
\pgfpathclose%
\pgfusepath{stroke,fill}%
\end{pgfscope}%
\begin{pgfscope}%
\pgfpathrectangle{\pgfqpoint{0.375000in}{0.330000in}}{\pgfqpoint{2.325000in}{2.310000in}}%
\pgfusepath{clip}%
\pgfsetbuttcap%
\pgfsetroundjoin%
\definecolor{currentfill}{rgb}{0.000000,0.000000,0.000000}%
\pgfsetfillcolor{currentfill}%
\pgfsetlinewidth{1.003750pt}%
\definecolor{currentstroke}{rgb}{0.000000,0.000000,0.000000}%
\pgfsetstrokecolor{currentstroke}%
\pgfsetdash{}{0pt}%
\pgfpathmoveto{\pgfqpoint{2.055402in}{1.538958in}}%
\pgfpathcurveto{\pgfqpoint{2.066452in}{1.538958in}}{\pgfqpoint{2.077051in}{1.543348in}}{\pgfqpoint{2.084865in}{1.551161in}}%
\pgfpathcurveto{\pgfqpoint{2.092678in}{1.558975in}}{\pgfqpoint{2.097069in}{1.569574in}}{\pgfqpoint{2.097069in}{1.580624in}}%
\pgfpathcurveto{\pgfqpoint{2.097069in}{1.591674in}}{\pgfqpoint{2.092678in}{1.602273in}}{\pgfqpoint{2.084865in}{1.610087in}}%
\pgfpathcurveto{\pgfqpoint{2.077051in}{1.617901in}}{\pgfqpoint{2.066452in}{1.622291in}}{\pgfqpoint{2.055402in}{1.622291in}}%
\pgfpathcurveto{\pgfqpoint{2.044352in}{1.622291in}}{\pgfqpoint{2.033753in}{1.617901in}}{\pgfqpoint{2.025939in}{1.610087in}}%
\pgfpathcurveto{\pgfqpoint{2.018125in}{1.602273in}}{\pgfqpoint{2.013735in}{1.591674in}}{\pgfqpoint{2.013735in}{1.580624in}}%
\pgfpathcurveto{\pgfqpoint{2.013735in}{1.569574in}}{\pgfqpoint{2.018125in}{1.558975in}}{\pgfqpoint{2.025939in}{1.551161in}}%
\pgfpathcurveto{\pgfqpoint{2.033753in}{1.543348in}}{\pgfqpoint{2.044352in}{1.538958in}}{\pgfqpoint{2.055402in}{1.538958in}}%
\pgfpathclose%
\pgfusepath{stroke,fill}%
\end{pgfscope}%
\begin{pgfscope}%
\pgfpathrectangle{\pgfqpoint{0.375000in}{0.330000in}}{\pgfqpoint{2.325000in}{2.310000in}}%
\pgfusepath{clip}%
\pgfsetbuttcap%
\pgfsetroundjoin%
\definecolor{currentfill}{rgb}{0.000000,0.000000,0.000000}%
\pgfsetfillcolor{currentfill}%
\pgfsetlinewidth{1.003750pt}%
\definecolor{currentstroke}{rgb}{0.000000,0.000000,0.000000}%
\pgfsetstrokecolor{currentstroke}%
\pgfsetdash{}{0pt}%
\pgfpathmoveto{\pgfqpoint{2.055402in}{1.486927in}}%
\pgfpathcurveto{\pgfqpoint{2.066452in}{1.486927in}}{\pgfqpoint{2.077051in}{1.491317in}}{\pgfqpoint{2.084865in}{1.499131in}}%
\pgfpathcurveto{\pgfqpoint{2.092678in}{1.506944in}}{\pgfqpoint{2.097069in}{1.517543in}}{\pgfqpoint{2.097069in}{1.528593in}}%
\pgfpathcurveto{\pgfqpoint{2.097069in}{1.539644in}}{\pgfqpoint{2.092678in}{1.550243in}}{\pgfqpoint{2.084865in}{1.558056in}}%
\pgfpathcurveto{\pgfqpoint{2.077051in}{1.565870in}}{\pgfqpoint{2.066452in}{1.570260in}}{\pgfqpoint{2.055402in}{1.570260in}}%
\pgfpathcurveto{\pgfqpoint{2.044352in}{1.570260in}}{\pgfqpoint{2.033753in}{1.565870in}}{\pgfqpoint{2.025939in}{1.558056in}}%
\pgfpathcurveto{\pgfqpoint{2.018125in}{1.550243in}}{\pgfqpoint{2.013735in}{1.539644in}}{\pgfqpoint{2.013735in}{1.528593in}}%
\pgfpathcurveto{\pgfqpoint{2.013735in}{1.517543in}}{\pgfqpoint{2.018125in}{1.506944in}}{\pgfqpoint{2.025939in}{1.499131in}}%
\pgfpathcurveto{\pgfqpoint{2.033753in}{1.491317in}}{\pgfqpoint{2.044352in}{1.486927in}}{\pgfqpoint{2.055402in}{1.486927in}}%
\pgfpathclose%
\pgfusepath{stroke,fill}%
\end{pgfscope}%
\begin{pgfscope}%
\pgfpathrectangle{\pgfqpoint{0.375000in}{0.330000in}}{\pgfqpoint{2.325000in}{2.310000in}}%
\pgfusepath{clip}%
\pgfsetbuttcap%
\pgfsetroundjoin%
\definecolor{currentfill}{rgb}{0.000000,0.000000,0.000000}%
\pgfsetfillcolor{currentfill}%
\pgfsetlinewidth{1.003750pt}%
\definecolor{currentstroke}{rgb}{0.000000,0.000000,0.000000}%
\pgfsetstrokecolor{currentstroke}%
\pgfsetdash{}{0pt}%
\pgfpathmoveto{\pgfqpoint{2.055402in}{1.486927in}}%
\pgfpathcurveto{\pgfqpoint{2.066452in}{1.486927in}}{\pgfqpoint{2.077051in}{1.491317in}}{\pgfqpoint{2.084865in}{1.499131in}}%
\pgfpathcurveto{\pgfqpoint{2.092678in}{1.506944in}}{\pgfqpoint{2.097069in}{1.517543in}}{\pgfqpoint{2.097069in}{1.528593in}}%
\pgfpathcurveto{\pgfqpoint{2.097069in}{1.539644in}}{\pgfqpoint{2.092678in}{1.550243in}}{\pgfqpoint{2.084865in}{1.558056in}}%
\pgfpathcurveto{\pgfqpoint{2.077051in}{1.565870in}}{\pgfqpoint{2.066452in}{1.570260in}}{\pgfqpoint{2.055402in}{1.570260in}}%
\pgfpathcurveto{\pgfqpoint{2.044352in}{1.570260in}}{\pgfqpoint{2.033753in}{1.565870in}}{\pgfqpoint{2.025939in}{1.558056in}}%
\pgfpathcurveto{\pgfqpoint{2.018125in}{1.550243in}}{\pgfqpoint{2.013735in}{1.539644in}}{\pgfqpoint{2.013735in}{1.528593in}}%
\pgfpathcurveto{\pgfqpoint{2.013735in}{1.517543in}}{\pgfqpoint{2.018125in}{1.506944in}}{\pgfqpoint{2.025939in}{1.499131in}}%
\pgfpathcurveto{\pgfqpoint{2.033753in}{1.491317in}}{\pgfqpoint{2.044352in}{1.486927in}}{\pgfqpoint{2.055402in}{1.486927in}}%
\pgfpathclose%
\pgfusepath{stroke,fill}%
\end{pgfscope}%
\begin{pgfscope}%
\pgfpathrectangle{\pgfqpoint{0.375000in}{0.330000in}}{\pgfqpoint{2.325000in}{2.310000in}}%
\pgfusepath{clip}%
\pgfsetbuttcap%
\pgfsetroundjoin%
\definecolor{currentfill}{rgb}{0.000000,0.000000,0.000000}%
\pgfsetfillcolor{currentfill}%
\pgfsetlinewidth{1.003750pt}%
\definecolor{currentstroke}{rgb}{0.000000,0.000000,0.000000}%
\pgfsetstrokecolor{currentstroke}%
\pgfsetdash{}{0pt}%
\pgfpathmoveto{\pgfqpoint{2.055402in}{1.486927in}}%
\pgfpathcurveto{\pgfqpoint{2.066452in}{1.486927in}}{\pgfqpoint{2.077051in}{1.491317in}}{\pgfqpoint{2.084865in}{1.499131in}}%
\pgfpathcurveto{\pgfqpoint{2.092678in}{1.506944in}}{\pgfqpoint{2.097069in}{1.517543in}}{\pgfqpoint{2.097069in}{1.528593in}}%
\pgfpathcurveto{\pgfqpoint{2.097069in}{1.539644in}}{\pgfqpoint{2.092678in}{1.550243in}}{\pgfqpoint{2.084865in}{1.558056in}}%
\pgfpathcurveto{\pgfqpoint{2.077051in}{1.565870in}}{\pgfqpoint{2.066452in}{1.570260in}}{\pgfqpoint{2.055402in}{1.570260in}}%
\pgfpathcurveto{\pgfqpoint{2.044352in}{1.570260in}}{\pgfqpoint{2.033753in}{1.565870in}}{\pgfqpoint{2.025939in}{1.558056in}}%
\pgfpathcurveto{\pgfqpoint{2.018125in}{1.550243in}}{\pgfqpoint{2.013735in}{1.539644in}}{\pgfqpoint{2.013735in}{1.528593in}}%
\pgfpathcurveto{\pgfqpoint{2.013735in}{1.517543in}}{\pgfqpoint{2.018125in}{1.506944in}}{\pgfqpoint{2.025939in}{1.499131in}}%
\pgfpathcurveto{\pgfqpoint{2.033753in}{1.491317in}}{\pgfqpoint{2.044352in}{1.486927in}}{\pgfqpoint{2.055402in}{1.486927in}}%
\pgfpathclose%
\pgfusepath{stroke,fill}%
\end{pgfscope}%
\begin{pgfscope}%
\pgfpathrectangle{\pgfqpoint{0.375000in}{0.330000in}}{\pgfqpoint{2.325000in}{2.310000in}}%
\pgfusepath{clip}%
\pgfsetbuttcap%
\pgfsetroundjoin%
\definecolor{currentfill}{rgb}{0.000000,0.000000,0.000000}%
\pgfsetfillcolor{currentfill}%
\pgfsetlinewidth{1.003750pt}%
\definecolor{currentstroke}{rgb}{0.000000,0.000000,0.000000}%
\pgfsetstrokecolor{currentstroke}%
\pgfsetdash{}{0pt}%
\pgfpathmoveto{\pgfqpoint{2.055402in}{1.538958in}}%
\pgfpathcurveto{\pgfqpoint{2.066452in}{1.538958in}}{\pgfqpoint{2.077051in}{1.543348in}}{\pgfqpoint{2.084865in}{1.551161in}}%
\pgfpathcurveto{\pgfqpoint{2.092678in}{1.558975in}}{\pgfqpoint{2.097069in}{1.569574in}}{\pgfqpoint{2.097069in}{1.580624in}}%
\pgfpathcurveto{\pgfqpoint{2.097069in}{1.591674in}}{\pgfqpoint{2.092678in}{1.602273in}}{\pgfqpoint{2.084865in}{1.610087in}}%
\pgfpathcurveto{\pgfqpoint{2.077051in}{1.617901in}}{\pgfqpoint{2.066452in}{1.622291in}}{\pgfqpoint{2.055402in}{1.622291in}}%
\pgfpathcurveto{\pgfqpoint{2.044352in}{1.622291in}}{\pgfqpoint{2.033753in}{1.617901in}}{\pgfqpoint{2.025939in}{1.610087in}}%
\pgfpathcurveto{\pgfqpoint{2.018125in}{1.602273in}}{\pgfqpoint{2.013735in}{1.591674in}}{\pgfqpoint{2.013735in}{1.580624in}}%
\pgfpathcurveto{\pgfqpoint{2.013735in}{1.569574in}}{\pgfqpoint{2.018125in}{1.558975in}}{\pgfqpoint{2.025939in}{1.551161in}}%
\pgfpathcurveto{\pgfqpoint{2.033753in}{1.543348in}}{\pgfqpoint{2.044352in}{1.538958in}}{\pgfqpoint{2.055402in}{1.538958in}}%
\pgfpathclose%
\pgfusepath{stroke,fill}%
\end{pgfscope}%
\begin{pgfscope}%
\pgfpathrectangle{\pgfqpoint{0.375000in}{0.330000in}}{\pgfqpoint{2.325000in}{2.310000in}}%
\pgfusepath{clip}%
\pgfsetbuttcap%
\pgfsetroundjoin%
\definecolor{currentfill}{rgb}{0.000000,0.000000,0.000000}%
\pgfsetfillcolor{currentfill}%
\pgfsetlinewidth{1.003750pt}%
\definecolor{currentstroke}{rgb}{0.000000,0.000000,0.000000}%
\pgfsetstrokecolor{currentstroke}%
\pgfsetdash{}{0pt}%
\pgfpathmoveto{\pgfqpoint{2.055402in}{1.486927in}}%
\pgfpathcurveto{\pgfqpoint{2.066452in}{1.486927in}}{\pgfqpoint{2.077051in}{1.491317in}}{\pgfqpoint{2.084865in}{1.499131in}}%
\pgfpathcurveto{\pgfqpoint{2.092678in}{1.506944in}}{\pgfqpoint{2.097069in}{1.517543in}}{\pgfqpoint{2.097069in}{1.528593in}}%
\pgfpathcurveto{\pgfqpoint{2.097069in}{1.539644in}}{\pgfqpoint{2.092678in}{1.550243in}}{\pgfqpoint{2.084865in}{1.558056in}}%
\pgfpathcurveto{\pgfqpoint{2.077051in}{1.565870in}}{\pgfqpoint{2.066452in}{1.570260in}}{\pgfqpoint{2.055402in}{1.570260in}}%
\pgfpathcurveto{\pgfqpoint{2.044352in}{1.570260in}}{\pgfqpoint{2.033753in}{1.565870in}}{\pgfqpoint{2.025939in}{1.558056in}}%
\pgfpathcurveto{\pgfqpoint{2.018125in}{1.550243in}}{\pgfqpoint{2.013735in}{1.539644in}}{\pgfqpoint{2.013735in}{1.528593in}}%
\pgfpathcurveto{\pgfqpoint{2.013735in}{1.517543in}}{\pgfqpoint{2.018125in}{1.506944in}}{\pgfqpoint{2.025939in}{1.499131in}}%
\pgfpathcurveto{\pgfqpoint{2.033753in}{1.491317in}}{\pgfqpoint{2.044352in}{1.486927in}}{\pgfqpoint{2.055402in}{1.486927in}}%
\pgfpathclose%
\pgfusepath{stroke,fill}%
\end{pgfscope}%
\begin{pgfscope}%
\pgfpathrectangle{\pgfqpoint{0.375000in}{0.330000in}}{\pgfqpoint{2.325000in}{2.310000in}}%
\pgfusepath{clip}%
\pgfsetbuttcap%
\pgfsetroundjoin%
\definecolor{currentfill}{rgb}{0.000000,0.000000,0.000000}%
\pgfsetfillcolor{currentfill}%
\pgfsetlinewidth{1.003750pt}%
\definecolor{currentstroke}{rgb}{0.000000,0.000000,0.000000}%
\pgfsetstrokecolor{currentstroke}%
\pgfsetdash{}{0pt}%
\pgfpathmoveto{\pgfqpoint{2.055402in}{1.434896in}}%
\pgfpathcurveto{\pgfqpoint{2.066452in}{1.434896in}}{\pgfqpoint{2.077051in}{1.439286in}}{\pgfqpoint{2.084865in}{1.447100in}}%
\pgfpathcurveto{\pgfqpoint{2.092678in}{1.454913in}}{\pgfqpoint{2.097069in}{1.465512in}}{\pgfqpoint{2.097069in}{1.476563in}}%
\pgfpathcurveto{\pgfqpoint{2.097069in}{1.487613in}}{\pgfqpoint{2.092678in}{1.498212in}}{\pgfqpoint{2.084865in}{1.506025in}}%
\pgfpathcurveto{\pgfqpoint{2.077051in}{1.513839in}}{\pgfqpoint{2.066452in}{1.518229in}}{\pgfqpoint{2.055402in}{1.518229in}}%
\pgfpathcurveto{\pgfqpoint{2.044352in}{1.518229in}}{\pgfqpoint{2.033753in}{1.513839in}}{\pgfqpoint{2.025939in}{1.506025in}}%
\pgfpathcurveto{\pgfqpoint{2.018125in}{1.498212in}}{\pgfqpoint{2.013735in}{1.487613in}}{\pgfqpoint{2.013735in}{1.476563in}}%
\pgfpathcurveto{\pgfqpoint{2.013735in}{1.465512in}}{\pgfqpoint{2.018125in}{1.454913in}}{\pgfqpoint{2.025939in}{1.447100in}}%
\pgfpathcurveto{\pgfqpoint{2.033753in}{1.439286in}}{\pgfqpoint{2.044352in}{1.434896in}}{\pgfqpoint{2.055402in}{1.434896in}}%
\pgfpathclose%
\pgfusepath{stroke,fill}%
\end{pgfscope}%
\begin{pgfscope}%
\pgfpathrectangle{\pgfqpoint{0.375000in}{0.330000in}}{\pgfqpoint{2.325000in}{2.310000in}}%
\pgfusepath{clip}%
\pgfsetbuttcap%
\pgfsetroundjoin%
\definecolor{currentfill}{rgb}{0.000000,0.000000,0.000000}%
\pgfsetfillcolor{currentfill}%
\pgfsetlinewidth{1.003750pt}%
\definecolor{currentstroke}{rgb}{0.000000,0.000000,0.000000}%
\pgfsetstrokecolor{currentstroke}%
\pgfsetdash{}{0pt}%
\pgfpathmoveto{\pgfqpoint{2.055402in}{1.434896in}}%
\pgfpathcurveto{\pgfqpoint{2.066452in}{1.434896in}}{\pgfqpoint{2.077051in}{1.439286in}}{\pgfqpoint{2.084865in}{1.447100in}}%
\pgfpathcurveto{\pgfqpoint{2.092678in}{1.454913in}}{\pgfqpoint{2.097069in}{1.465512in}}{\pgfqpoint{2.097069in}{1.476563in}}%
\pgfpathcurveto{\pgfqpoint{2.097069in}{1.487613in}}{\pgfqpoint{2.092678in}{1.498212in}}{\pgfqpoint{2.084865in}{1.506025in}}%
\pgfpathcurveto{\pgfqpoint{2.077051in}{1.513839in}}{\pgfqpoint{2.066452in}{1.518229in}}{\pgfqpoint{2.055402in}{1.518229in}}%
\pgfpathcurveto{\pgfqpoint{2.044352in}{1.518229in}}{\pgfqpoint{2.033753in}{1.513839in}}{\pgfqpoint{2.025939in}{1.506025in}}%
\pgfpathcurveto{\pgfqpoint{2.018125in}{1.498212in}}{\pgfqpoint{2.013735in}{1.487613in}}{\pgfqpoint{2.013735in}{1.476563in}}%
\pgfpathcurveto{\pgfqpoint{2.013735in}{1.465512in}}{\pgfqpoint{2.018125in}{1.454913in}}{\pgfqpoint{2.025939in}{1.447100in}}%
\pgfpathcurveto{\pgfqpoint{2.033753in}{1.439286in}}{\pgfqpoint{2.044352in}{1.434896in}}{\pgfqpoint{2.055402in}{1.434896in}}%
\pgfpathclose%
\pgfusepath{stroke,fill}%
\end{pgfscope}%
\begin{pgfscope}%
\pgfpathrectangle{\pgfqpoint{0.375000in}{0.330000in}}{\pgfqpoint{2.325000in}{2.310000in}}%
\pgfusepath{clip}%
\pgfsetbuttcap%
\pgfsetroundjoin%
\definecolor{currentfill}{rgb}{0.000000,0.000000,0.000000}%
\pgfsetfillcolor{currentfill}%
\pgfsetlinewidth{1.003750pt}%
\definecolor{currentstroke}{rgb}{0.000000,0.000000,0.000000}%
\pgfsetstrokecolor{currentstroke}%
\pgfsetdash{}{0pt}%
\pgfpathmoveto{\pgfqpoint{2.055402in}{1.486927in}}%
\pgfpathcurveto{\pgfqpoint{2.066452in}{1.486927in}}{\pgfqpoint{2.077051in}{1.491317in}}{\pgfqpoint{2.084865in}{1.499131in}}%
\pgfpathcurveto{\pgfqpoint{2.092678in}{1.506944in}}{\pgfqpoint{2.097069in}{1.517543in}}{\pgfqpoint{2.097069in}{1.528593in}}%
\pgfpathcurveto{\pgfqpoint{2.097069in}{1.539644in}}{\pgfqpoint{2.092678in}{1.550243in}}{\pgfqpoint{2.084865in}{1.558056in}}%
\pgfpathcurveto{\pgfqpoint{2.077051in}{1.565870in}}{\pgfqpoint{2.066452in}{1.570260in}}{\pgfqpoint{2.055402in}{1.570260in}}%
\pgfpathcurveto{\pgfqpoint{2.044352in}{1.570260in}}{\pgfqpoint{2.033753in}{1.565870in}}{\pgfqpoint{2.025939in}{1.558056in}}%
\pgfpathcurveto{\pgfqpoint{2.018125in}{1.550243in}}{\pgfqpoint{2.013735in}{1.539644in}}{\pgfqpoint{2.013735in}{1.528593in}}%
\pgfpathcurveto{\pgfqpoint{2.013735in}{1.517543in}}{\pgfqpoint{2.018125in}{1.506944in}}{\pgfqpoint{2.025939in}{1.499131in}}%
\pgfpathcurveto{\pgfqpoint{2.033753in}{1.491317in}}{\pgfqpoint{2.044352in}{1.486927in}}{\pgfqpoint{2.055402in}{1.486927in}}%
\pgfpathclose%
\pgfusepath{stroke,fill}%
\end{pgfscope}%
\begin{pgfscope}%
\pgfpathrectangle{\pgfqpoint{0.375000in}{0.330000in}}{\pgfqpoint{2.325000in}{2.310000in}}%
\pgfusepath{clip}%
\pgfsetbuttcap%
\pgfsetroundjoin%
\definecolor{currentfill}{rgb}{0.000000,0.000000,0.000000}%
\pgfsetfillcolor{currentfill}%
\pgfsetlinewidth{1.003750pt}%
\definecolor{currentstroke}{rgb}{0.000000,0.000000,0.000000}%
\pgfsetstrokecolor{currentstroke}%
\pgfsetdash{}{0pt}%
\pgfpathmoveto{\pgfqpoint{2.055402in}{1.538958in}}%
\pgfpathcurveto{\pgfqpoint{2.066452in}{1.538958in}}{\pgfqpoint{2.077051in}{1.543348in}}{\pgfqpoint{2.084865in}{1.551161in}}%
\pgfpathcurveto{\pgfqpoint{2.092678in}{1.558975in}}{\pgfqpoint{2.097069in}{1.569574in}}{\pgfqpoint{2.097069in}{1.580624in}}%
\pgfpathcurveto{\pgfqpoint{2.097069in}{1.591674in}}{\pgfqpoint{2.092678in}{1.602273in}}{\pgfqpoint{2.084865in}{1.610087in}}%
\pgfpathcurveto{\pgfqpoint{2.077051in}{1.617901in}}{\pgfqpoint{2.066452in}{1.622291in}}{\pgfqpoint{2.055402in}{1.622291in}}%
\pgfpathcurveto{\pgfqpoint{2.044352in}{1.622291in}}{\pgfqpoint{2.033753in}{1.617901in}}{\pgfqpoint{2.025939in}{1.610087in}}%
\pgfpathcurveto{\pgfqpoint{2.018125in}{1.602273in}}{\pgfqpoint{2.013735in}{1.591674in}}{\pgfqpoint{2.013735in}{1.580624in}}%
\pgfpathcurveto{\pgfqpoint{2.013735in}{1.569574in}}{\pgfqpoint{2.018125in}{1.558975in}}{\pgfqpoint{2.025939in}{1.551161in}}%
\pgfpathcurveto{\pgfqpoint{2.033753in}{1.543348in}}{\pgfqpoint{2.044352in}{1.538958in}}{\pgfqpoint{2.055402in}{1.538958in}}%
\pgfpathclose%
\pgfusepath{stroke,fill}%
\end{pgfscope}%
\begin{pgfscope}%
\pgfpathrectangle{\pgfqpoint{0.375000in}{0.330000in}}{\pgfqpoint{2.325000in}{2.310000in}}%
\pgfusepath{clip}%
\pgfsetbuttcap%
\pgfsetroundjoin%
\definecolor{currentfill}{rgb}{0.000000,0.000000,0.000000}%
\pgfsetfillcolor{currentfill}%
\pgfsetlinewidth{1.003750pt}%
\definecolor{currentstroke}{rgb}{0.000000,0.000000,0.000000}%
\pgfsetstrokecolor{currentstroke}%
\pgfsetdash{}{0pt}%
\pgfpathmoveto{\pgfqpoint{2.055402in}{1.486927in}}%
\pgfpathcurveto{\pgfqpoint{2.066452in}{1.486927in}}{\pgfqpoint{2.077051in}{1.491317in}}{\pgfqpoint{2.084865in}{1.499131in}}%
\pgfpathcurveto{\pgfqpoint{2.092678in}{1.506944in}}{\pgfqpoint{2.097069in}{1.517543in}}{\pgfqpoint{2.097069in}{1.528593in}}%
\pgfpathcurveto{\pgfqpoint{2.097069in}{1.539644in}}{\pgfqpoint{2.092678in}{1.550243in}}{\pgfqpoint{2.084865in}{1.558056in}}%
\pgfpathcurveto{\pgfqpoint{2.077051in}{1.565870in}}{\pgfqpoint{2.066452in}{1.570260in}}{\pgfqpoint{2.055402in}{1.570260in}}%
\pgfpathcurveto{\pgfqpoint{2.044352in}{1.570260in}}{\pgfqpoint{2.033753in}{1.565870in}}{\pgfqpoint{2.025939in}{1.558056in}}%
\pgfpathcurveto{\pgfqpoint{2.018125in}{1.550243in}}{\pgfqpoint{2.013735in}{1.539644in}}{\pgfqpoint{2.013735in}{1.528593in}}%
\pgfpathcurveto{\pgfqpoint{2.013735in}{1.517543in}}{\pgfqpoint{2.018125in}{1.506944in}}{\pgfqpoint{2.025939in}{1.499131in}}%
\pgfpathcurveto{\pgfqpoint{2.033753in}{1.491317in}}{\pgfqpoint{2.044352in}{1.486927in}}{\pgfqpoint{2.055402in}{1.486927in}}%
\pgfpathclose%
\pgfusepath{stroke,fill}%
\end{pgfscope}%
\begin{pgfscope}%
\pgfpathrectangle{\pgfqpoint{0.375000in}{0.330000in}}{\pgfqpoint{2.325000in}{2.310000in}}%
\pgfusepath{clip}%
\pgfsetbuttcap%
\pgfsetroundjoin%
\definecolor{currentfill}{rgb}{0.000000,0.000000,0.000000}%
\pgfsetfillcolor{currentfill}%
\pgfsetlinewidth{1.003750pt}%
\definecolor{currentstroke}{rgb}{0.000000,0.000000,0.000000}%
\pgfsetstrokecolor{currentstroke}%
\pgfsetdash{}{0pt}%
\pgfpathmoveto{\pgfqpoint{2.055402in}{1.486927in}}%
\pgfpathcurveto{\pgfqpoint{2.066452in}{1.486927in}}{\pgfqpoint{2.077051in}{1.491317in}}{\pgfqpoint{2.084865in}{1.499131in}}%
\pgfpathcurveto{\pgfqpoint{2.092678in}{1.506944in}}{\pgfqpoint{2.097069in}{1.517543in}}{\pgfqpoint{2.097069in}{1.528593in}}%
\pgfpathcurveto{\pgfqpoint{2.097069in}{1.539644in}}{\pgfqpoint{2.092678in}{1.550243in}}{\pgfqpoint{2.084865in}{1.558056in}}%
\pgfpathcurveto{\pgfqpoint{2.077051in}{1.565870in}}{\pgfqpoint{2.066452in}{1.570260in}}{\pgfqpoint{2.055402in}{1.570260in}}%
\pgfpathcurveto{\pgfqpoint{2.044352in}{1.570260in}}{\pgfqpoint{2.033753in}{1.565870in}}{\pgfqpoint{2.025939in}{1.558056in}}%
\pgfpathcurveto{\pgfqpoint{2.018125in}{1.550243in}}{\pgfqpoint{2.013735in}{1.539644in}}{\pgfqpoint{2.013735in}{1.528593in}}%
\pgfpathcurveto{\pgfqpoint{2.013735in}{1.517543in}}{\pgfqpoint{2.018125in}{1.506944in}}{\pgfqpoint{2.025939in}{1.499131in}}%
\pgfpathcurveto{\pgfqpoint{2.033753in}{1.491317in}}{\pgfqpoint{2.044352in}{1.486927in}}{\pgfqpoint{2.055402in}{1.486927in}}%
\pgfpathclose%
\pgfusepath{stroke,fill}%
\end{pgfscope}%
\begin{pgfscope}%
\pgfpathrectangle{\pgfqpoint{0.375000in}{0.330000in}}{\pgfqpoint{2.325000in}{2.310000in}}%
\pgfusepath{clip}%
\pgfsetbuttcap%
\pgfsetroundjoin%
\definecolor{currentfill}{rgb}{0.000000,0.000000,0.000000}%
\pgfsetfillcolor{currentfill}%
\pgfsetlinewidth{1.003750pt}%
\definecolor{currentstroke}{rgb}{0.000000,0.000000,0.000000}%
\pgfsetstrokecolor{currentstroke}%
\pgfsetdash{}{0pt}%
\pgfpathmoveto{\pgfqpoint{2.055402in}{1.538958in}}%
\pgfpathcurveto{\pgfqpoint{2.066452in}{1.538958in}}{\pgfqpoint{2.077051in}{1.543348in}}{\pgfqpoint{2.084865in}{1.551161in}}%
\pgfpathcurveto{\pgfqpoint{2.092678in}{1.558975in}}{\pgfqpoint{2.097069in}{1.569574in}}{\pgfqpoint{2.097069in}{1.580624in}}%
\pgfpathcurveto{\pgfqpoint{2.097069in}{1.591674in}}{\pgfqpoint{2.092678in}{1.602273in}}{\pgfqpoint{2.084865in}{1.610087in}}%
\pgfpathcurveto{\pgfqpoint{2.077051in}{1.617901in}}{\pgfqpoint{2.066452in}{1.622291in}}{\pgfqpoint{2.055402in}{1.622291in}}%
\pgfpathcurveto{\pgfqpoint{2.044352in}{1.622291in}}{\pgfqpoint{2.033753in}{1.617901in}}{\pgfqpoint{2.025939in}{1.610087in}}%
\pgfpathcurveto{\pgfqpoint{2.018125in}{1.602273in}}{\pgfqpoint{2.013735in}{1.591674in}}{\pgfqpoint{2.013735in}{1.580624in}}%
\pgfpathcurveto{\pgfqpoint{2.013735in}{1.569574in}}{\pgfqpoint{2.018125in}{1.558975in}}{\pgfqpoint{2.025939in}{1.551161in}}%
\pgfpathcurveto{\pgfqpoint{2.033753in}{1.543348in}}{\pgfqpoint{2.044352in}{1.538958in}}{\pgfqpoint{2.055402in}{1.538958in}}%
\pgfpathclose%
\pgfusepath{stroke,fill}%
\end{pgfscope}%
\begin{pgfscope}%
\pgfpathrectangle{\pgfqpoint{0.375000in}{0.330000in}}{\pgfqpoint{2.325000in}{2.310000in}}%
\pgfusepath{clip}%
\pgfsetbuttcap%
\pgfsetroundjoin%
\definecolor{currentfill}{rgb}{0.000000,0.000000,0.000000}%
\pgfsetfillcolor{currentfill}%
\pgfsetlinewidth{1.003750pt}%
\definecolor{currentstroke}{rgb}{0.000000,0.000000,0.000000}%
\pgfsetstrokecolor{currentstroke}%
\pgfsetdash{}{0pt}%
\pgfpathmoveto{\pgfqpoint{2.055402in}{1.538958in}}%
\pgfpathcurveto{\pgfqpoint{2.066452in}{1.538958in}}{\pgfqpoint{2.077051in}{1.543348in}}{\pgfqpoint{2.084865in}{1.551161in}}%
\pgfpathcurveto{\pgfqpoint{2.092678in}{1.558975in}}{\pgfqpoint{2.097069in}{1.569574in}}{\pgfqpoint{2.097069in}{1.580624in}}%
\pgfpathcurveto{\pgfqpoint{2.097069in}{1.591674in}}{\pgfqpoint{2.092678in}{1.602273in}}{\pgfqpoint{2.084865in}{1.610087in}}%
\pgfpathcurveto{\pgfqpoint{2.077051in}{1.617901in}}{\pgfqpoint{2.066452in}{1.622291in}}{\pgfqpoint{2.055402in}{1.622291in}}%
\pgfpathcurveto{\pgfqpoint{2.044352in}{1.622291in}}{\pgfqpoint{2.033753in}{1.617901in}}{\pgfqpoint{2.025939in}{1.610087in}}%
\pgfpathcurveto{\pgfqpoint{2.018125in}{1.602273in}}{\pgfqpoint{2.013735in}{1.591674in}}{\pgfqpoint{2.013735in}{1.580624in}}%
\pgfpathcurveto{\pgfqpoint{2.013735in}{1.569574in}}{\pgfqpoint{2.018125in}{1.558975in}}{\pgfqpoint{2.025939in}{1.551161in}}%
\pgfpathcurveto{\pgfqpoint{2.033753in}{1.543348in}}{\pgfqpoint{2.044352in}{1.538958in}}{\pgfqpoint{2.055402in}{1.538958in}}%
\pgfpathclose%
\pgfusepath{stroke,fill}%
\end{pgfscope}%
\begin{pgfscope}%
\pgfpathrectangle{\pgfqpoint{0.375000in}{0.330000in}}{\pgfqpoint{2.325000in}{2.310000in}}%
\pgfusepath{clip}%
\pgfsetbuttcap%
\pgfsetroundjoin%
\definecolor{currentfill}{rgb}{0.000000,0.000000,0.000000}%
\pgfsetfillcolor{currentfill}%
\pgfsetlinewidth{1.003750pt}%
\definecolor{currentstroke}{rgb}{0.000000,0.000000,0.000000}%
\pgfsetstrokecolor{currentstroke}%
\pgfsetdash{}{0pt}%
\pgfpathmoveto{\pgfqpoint{2.055402in}{1.538958in}}%
\pgfpathcurveto{\pgfqpoint{2.066452in}{1.538958in}}{\pgfqpoint{2.077051in}{1.543348in}}{\pgfqpoint{2.084865in}{1.551161in}}%
\pgfpathcurveto{\pgfqpoint{2.092678in}{1.558975in}}{\pgfqpoint{2.097069in}{1.569574in}}{\pgfqpoint{2.097069in}{1.580624in}}%
\pgfpathcurveto{\pgfqpoint{2.097069in}{1.591674in}}{\pgfqpoint{2.092678in}{1.602273in}}{\pgfqpoint{2.084865in}{1.610087in}}%
\pgfpathcurveto{\pgfqpoint{2.077051in}{1.617901in}}{\pgfqpoint{2.066452in}{1.622291in}}{\pgfqpoint{2.055402in}{1.622291in}}%
\pgfpathcurveto{\pgfqpoint{2.044352in}{1.622291in}}{\pgfqpoint{2.033753in}{1.617901in}}{\pgfqpoint{2.025939in}{1.610087in}}%
\pgfpathcurveto{\pgfqpoint{2.018125in}{1.602273in}}{\pgfqpoint{2.013735in}{1.591674in}}{\pgfqpoint{2.013735in}{1.580624in}}%
\pgfpathcurveto{\pgfqpoint{2.013735in}{1.569574in}}{\pgfqpoint{2.018125in}{1.558975in}}{\pgfqpoint{2.025939in}{1.551161in}}%
\pgfpathcurveto{\pgfqpoint{2.033753in}{1.543348in}}{\pgfqpoint{2.044352in}{1.538958in}}{\pgfqpoint{2.055402in}{1.538958in}}%
\pgfpathclose%
\pgfusepath{stroke,fill}%
\end{pgfscope}%
\begin{pgfscope}%
\pgfpathrectangle{\pgfqpoint{0.375000in}{0.330000in}}{\pgfqpoint{2.325000in}{2.310000in}}%
\pgfusepath{clip}%
\pgfsetbuttcap%
\pgfsetroundjoin%
\definecolor{currentfill}{rgb}{0.000000,0.000000,0.000000}%
\pgfsetfillcolor{currentfill}%
\pgfsetlinewidth{1.003750pt}%
\definecolor{currentstroke}{rgb}{0.000000,0.000000,0.000000}%
\pgfsetstrokecolor{currentstroke}%
\pgfsetdash{}{0pt}%
\pgfpathmoveto{\pgfqpoint{2.055402in}{1.434896in}}%
\pgfpathcurveto{\pgfqpoint{2.066452in}{1.434896in}}{\pgfqpoint{2.077051in}{1.439286in}}{\pgfqpoint{2.084865in}{1.447100in}}%
\pgfpathcurveto{\pgfqpoint{2.092678in}{1.454913in}}{\pgfqpoint{2.097069in}{1.465512in}}{\pgfqpoint{2.097069in}{1.476563in}}%
\pgfpathcurveto{\pgfqpoint{2.097069in}{1.487613in}}{\pgfqpoint{2.092678in}{1.498212in}}{\pgfqpoint{2.084865in}{1.506025in}}%
\pgfpathcurveto{\pgfqpoint{2.077051in}{1.513839in}}{\pgfqpoint{2.066452in}{1.518229in}}{\pgfqpoint{2.055402in}{1.518229in}}%
\pgfpathcurveto{\pgfqpoint{2.044352in}{1.518229in}}{\pgfqpoint{2.033753in}{1.513839in}}{\pgfqpoint{2.025939in}{1.506025in}}%
\pgfpathcurveto{\pgfqpoint{2.018125in}{1.498212in}}{\pgfqpoint{2.013735in}{1.487613in}}{\pgfqpoint{2.013735in}{1.476563in}}%
\pgfpathcurveto{\pgfqpoint{2.013735in}{1.465512in}}{\pgfqpoint{2.018125in}{1.454913in}}{\pgfqpoint{2.025939in}{1.447100in}}%
\pgfpathcurveto{\pgfqpoint{2.033753in}{1.439286in}}{\pgfqpoint{2.044352in}{1.434896in}}{\pgfqpoint{2.055402in}{1.434896in}}%
\pgfpathclose%
\pgfusepath{stroke,fill}%
\end{pgfscope}%
\begin{pgfscope}%
\pgfpathrectangle{\pgfqpoint{0.375000in}{0.330000in}}{\pgfqpoint{2.325000in}{2.310000in}}%
\pgfusepath{clip}%
\pgfsetbuttcap%
\pgfsetroundjoin%
\definecolor{currentfill}{rgb}{0.000000,0.000000,0.000000}%
\pgfsetfillcolor{currentfill}%
\pgfsetlinewidth{1.003750pt}%
\definecolor{currentstroke}{rgb}{0.000000,0.000000,0.000000}%
\pgfsetstrokecolor{currentstroke}%
\pgfsetdash{}{0pt}%
\pgfpathmoveto{\pgfqpoint{2.055402in}{1.590988in}}%
\pgfpathcurveto{\pgfqpoint{2.066452in}{1.590988in}}{\pgfqpoint{2.077051in}{1.595379in}}{\pgfqpoint{2.084865in}{1.603192in}}%
\pgfpathcurveto{\pgfqpoint{2.092678in}{1.611006in}}{\pgfqpoint{2.097069in}{1.621605in}}{\pgfqpoint{2.097069in}{1.632655in}}%
\pgfpathcurveto{\pgfqpoint{2.097069in}{1.643705in}}{\pgfqpoint{2.092678in}{1.654304in}}{\pgfqpoint{2.084865in}{1.662118in}}%
\pgfpathcurveto{\pgfqpoint{2.077051in}{1.669931in}}{\pgfqpoint{2.066452in}{1.674322in}}{\pgfqpoint{2.055402in}{1.674322in}}%
\pgfpathcurveto{\pgfqpoint{2.044352in}{1.674322in}}{\pgfqpoint{2.033753in}{1.669931in}}{\pgfqpoint{2.025939in}{1.662118in}}%
\pgfpathcurveto{\pgfqpoint{2.018125in}{1.654304in}}{\pgfqpoint{2.013735in}{1.643705in}}{\pgfqpoint{2.013735in}{1.632655in}}%
\pgfpathcurveto{\pgfqpoint{2.013735in}{1.621605in}}{\pgfqpoint{2.018125in}{1.611006in}}{\pgfqpoint{2.025939in}{1.603192in}}%
\pgfpathcurveto{\pgfqpoint{2.033753in}{1.595379in}}{\pgfqpoint{2.044352in}{1.590988in}}{\pgfqpoint{2.055402in}{1.590988in}}%
\pgfpathclose%
\pgfusepath{stroke,fill}%
\end{pgfscope}%
\begin{pgfscope}%
\pgfpathrectangle{\pgfqpoint{0.375000in}{0.330000in}}{\pgfqpoint{2.325000in}{2.310000in}}%
\pgfusepath{clip}%
\pgfsetbuttcap%
\pgfsetroundjoin%
\definecolor{currentfill}{rgb}{0.000000,0.000000,0.000000}%
\pgfsetfillcolor{currentfill}%
\pgfsetlinewidth{1.003750pt}%
\definecolor{currentstroke}{rgb}{0.000000,0.000000,0.000000}%
\pgfsetstrokecolor{currentstroke}%
\pgfsetdash{}{0pt}%
\pgfpathmoveto{\pgfqpoint{2.055402in}{1.486927in}}%
\pgfpathcurveto{\pgfqpoint{2.066452in}{1.486927in}}{\pgfqpoint{2.077051in}{1.491317in}}{\pgfqpoint{2.084865in}{1.499131in}}%
\pgfpathcurveto{\pgfqpoint{2.092678in}{1.506944in}}{\pgfqpoint{2.097069in}{1.517543in}}{\pgfqpoint{2.097069in}{1.528593in}}%
\pgfpathcurveto{\pgfqpoint{2.097069in}{1.539644in}}{\pgfqpoint{2.092678in}{1.550243in}}{\pgfqpoint{2.084865in}{1.558056in}}%
\pgfpathcurveto{\pgfqpoint{2.077051in}{1.565870in}}{\pgfqpoint{2.066452in}{1.570260in}}{\pgfqpoint{2.055402in}{1.570260in}}%
\pgfpathcurveto{\pgfqpoint{2.044352in}{1.570260in}}{\pgfqpoint{2.033753in}{1.565870in}}{\pgfqpoint{2.025939in}{1.558056in}}%
\pgfpathcurveto{\pgfqpoint{2.018125in}{1.550243in}}{\pgfqpoint{2.013735in}{1.539644in}}{\pgfqpoint{2.013735in}{1.528593in}}%
\pgfpathcurveto{\pgfqpoint{2.013735in}{1.517543in}}{\pgfqpoint{2.018125in}{1.506944in}}{\pgfqpoint{2.025939in}{1.499131in}}%
\pgfpathcurveto{\pgfqpoint{2.033753in}{1.491317in}}{\pgfqpoint{2.044352in}{1.486927in}}{\pgfqpoint{2.055402in}{1.486927in}}%
\pgfpathclose%
\pgfusepath{stroke,fill}%
\end{pgfscope}%
\begin{pgfscope}%
\pgfpathrectangle{\pgfqpoint{0.375000in}{0.330000in}}{\pgfqpoint{2.325000in}{2.310000in}}%
\pgfusepath{clip}%
\pgfsetbuttcap%
\pgfsetroundjoin%
\definecolor{currentfill}{rgb}{0.000000,0.000000,0.000000}%
\pgfsetfillcolor{currentfill}%
\pgfsetlinewidth{1.003750pt}%
\definecolor{currentstroke}{rgb}{0.000000,0.000000,0.000000}%
\pgfsetstrokecolor{currentstroke}%
\pgfsetdash{}{0pt}%
\pgfpathmoveto{\pgfqpoint{2.055402in}{1.434896in}}%
\pgfpathcurveto{\pgfqpoint{2.066452in}{1.434896in}}{\pgfqpoint{2.077051in}{1.439286in}}{\pgfqpoint{2.084865in}{1.447100in}}%
\pgfpathcurveto{\pgfqpoint{2.092678in}{1.454913in}}{\pgfqpoint{2.097069in}{1.465512in}}{\pgfqpoint{2.097069in}{1.476563in}}%
\pgfpathcurveto{\pgfqpoint{2.097069in}{1.487613in}}{\pgfqpoint{2.092678in}{1.498212in}}{\pgfqpoint{2.084865in}{1.506025in}}%
\pgfpathcurveto{\pgfqpoint{2.077051in}{1.513839in}}{\pgfqpoint{2.066452in}{1.518229in}}{\pgfqpoint{2.055402in}{1.518229in}}%
\pgfpathcurveto{\pgfqpoint{2.044352in}{1.518229in}}{\pgfqpoint{2.033753in}{1.513839in}}{\pgfqpoint{2.025939in}{1.506025in}}%
\pgfpathcurveto{\pgfqpoint{2.018125in}{1.498212in}}{\pgfqpoint{2.013735in}{1.487613in}}{\pgfqpoint{2.013735in}{1.476563in}}%
\pgfpathcurveto{\pgfqpoint{2.013735in}{1.465512in}}{\pgfqpoint{2.018125in}{1.454913in}}{\pgfqpoint{2.025939in}{1.447100in}}%
\pgfpathcurveto{\pgfqpoint{2.033753in}{1.439286in}}{\pgfqpoint{2.044352in}{1.434896in}}{\pgfqpoint{2.055402in}{1.434896in}}%
\pgfpathclose%
\pgfusepath{stroke,fill}%
\end{pgfscope}%
\begin{pgfscope}%
\pgfpathrectangle{\pgfqpoint{0.375000in}{0.330000in}}{\pgfqpoint{2.325000in}{2.310000in}}%
\pgfusepath{clip}%
\pgfsetbuttcap%
\pgfsetroundjoin%
\definecolor{currentfill}{rgb}{0.000000,0.000000,0.000000}%
\pgfsetfillcolor{currentfill}%
\pgfsetlinewidth{1.003750pt}%
\definecolor{currentstroke}{rgb}{0.000000,0.000000,0.000000}%
\pgfsetstrokecolor{currentstroke}%
\pgfsetdash{}{0pt}%
\pgfpathmoveto{\pgfqpoint{2.055402in}{1.486927in}}%
\pgfpathcurveto{\pgfqpoint{2.066452in}{1.486927in}}{\pgfqpoint{2.077051in}{1.491317in}}{\pgfqpoint{2.084865in}{1.499131in}}%
\pgfpathcurveto{\pgfqpoint{2.092678in}{1.506944in}}{\pgfqpoint{2.097069in}{1.517543in}}{\pgfqpoint{2.097069in}{1.528593in}}%
\pgfpathcurveto{\pgfqpoint{2.097069in}{1.539644in}}{\pgfqpoint{2.092678in}{1.550243in}}{\pgfqpoint{2.084865in}{1.558056in}}%
\pgfpathcurveto{\pgfqpoint{2.077051in}{1.565870in}}{\pgfqpoint{2.066452in}{1.570260in}}{\pgfqpoint{2.055402in}{1.570260in}}%
\pgfpathcurveto{\pgfqpoint{2.044352in}{1.570260in}}{\pgfqpoint{2.033753in}{1.565870in}}{\pgfqpoint{2.025939in}{1.558056in}}%
\pgfpathcurveto{\pgfqpoint{2.018125in}{1.550243in}}{\pgfqpoint{2.013735in}{1.539644in}}{\pgfqpoint{2.013735in}{1.528593in}}%
\pgfpathcurveto{\pgfqpoint{2.013735in}{1.517543in}}{\pgfqpoint{2.018125in}{1.506944in}}{\pgfqpoint{2.025939in}{1.499131in}}%
\pgfpathcurveto{\pgfqpoint{2.033753in}{1.491317in}}{\pgfqpoint{2.044352in}{1.486927in}}{\pgfqpoint{2.055402in}{1.486927in}}%
\pgfpathclose%
\pgfusepath{stroke,fill}%
\end{pgfscope}%
\begin{pgfscope}%
\pgfpathrectangle{\pgfqpoint{0.375000in}{0.330000in}}{\pgfqpoint{2.325000in}{2.310000in}}%
\pgfusepath{clip}%
\pgfsetbuttcap%
\pgfsetroundjoin%
\definecolor{currentfill}{rgb}{0.000000,0.000000,0.000000}%
\pgfsetfillcolor{currentfill}%
\pgfsetlinewidth{1.003750pt}%
\definecolor{currentstroke}{rgb}{0.000000,0.000000,0.000000}%
\pgfsetstrokecolor{currentstroke}%
\pgfsetdash{}{0pt}%
\pgfpathmoveto{\pgfqpoint{2.055402in}{1.382865in}}%
\pgfpathcurveto{\pgfqpoint{2.066452in}{1.382865in}}{\pgfqpoint{2.077051in}{1.387255in}}{\pgfqpoint{2.084865in}{1.395069in}}%
\pgfpathcurveto{\pgfqpoint{2.092678in}{1.402883in}}{\pgfqpoint{2.097069in}{1.413482in}}{\pgfqpoint{2.097069in}{1.424532in}}%
\pgfpathcurveto{\pgfqpoint{2.097069in}{1.435582in}}{\pgfqpoint{2.092678in}{1.446181in}}{\pgfqpoint{2.084865in}{1.453995in}}%
\pgfpathcurveto{\pgfqpoint{2.077051in}{1.461808in}}{\pgfqpoint{2.066452in}{1.466198in}}{\pgfqpoint{2.055402in}{1.466198in}}%
\pgfpathcurveto{\pgfqpoint{2.044352in}{1.466198in}}{\pgfqpoint{2.033753in}{1.461808in}}{\pgfqpoint{2.025939in}{1.453995in}}%
\pgfpathcurveto{\pgfqpoint{2.018125in}{1.446181in}}{\pgfqpoint{2.013735in}{1.435582in}}{\pgfqpoint{2.013735in}{1.424532in}}%
\pgfpathcurveto{\pgfqpoint{2.013735in}{1.413482in}}{\pgfqpoint{2.018125in}{1.402883in}}{\pgfqpoint{2.025939in}{1.395069in}}%
\pgfpathcurveto{\pgfqpoint{2.033753in}{1.387255in}}{\pgfqpoint{2.044352in}{1.382865in}}{\pgfqpoint{2.055402in}{1.382865in}}%
\pgfpathclose%
\pgfusepath{stroke,fill}%
\end{pgfscope}%
\begin{pgfscope}%
\pgfpathrectangle{\pgfqpoint{0.375000in}{0.330000in}}{\pgfqpoint{2.325000in}{2.310000in}}%
\pgfusepath{clip}%
\pgfsetbuttcap%
\pgfsetroundjoin%
\definecolor{currentfill}{rgb}{0.000000,0.000000,0.000000}%
\pgfsetfillcolor{currentfill}%
\pgfsetlinewidth{1.003750pt}%
\definecolor{currentstroke}{rgb}{0.000000,0.000000,0.000000}%
\pgfsetstrokecolor{currentstroke}%
\pgfsetdash{}{0pt}%
\pgfpathmoveto{\pgfqpoint{2.055402in}{1.538958in}}%
\pgfpathcurveto{\pgfqpoint{2.066452in}{1.538958in}}{\pgfqpoint{2.077051in}{1.543348in}}{\pgfqpoint{2.084865in}{1.551161in}}%
\pgfpathcurveto{\pgfqpoint{2.092678in}{1.558975in}}{\pgfqpoint{2.097069in}{1.569574in}}{\pgfqpoint{2.097069in}{1.580624in}}%
\pgfpathcurveto{\pgfqpoint{2.097069in}{1.591674in}}{\pgfqpoint{2.092678in}{1.602273in}}{\pgfqpoint{2.084865in}{1.610087in}}%
\pgfpathcurveto{\pgfqpoint{2.077051in}{1.617901in}}{\pgfqpoint{2.066452in}{1.622291in}}{\pgfqpoint{2.055402in}{1.622291in}}%
\pgfpathcurveto{\pgfqpoint{2.044352in}{1.622291in}}{\pgfqpoint{2.033753in}{1.617901in}}{\pgfqpoint{2.025939in}{1.610087in}}%
\pgfpathcurveto{\pgfqpoint{2.018125in}{1.602273in}}{\pgfqpoint{2.013735in}{1.591674in}}{\pgfqpoint{2.013735in}{1.580624in}}%
\pgfpathcurveto{\pgfqpoint{2.013735in}{1.569574in}}{\pgfqpoint{2.018125in}{1.558975in}}{\pgfqpoint{2.025939in}{1.551161in}}%
\pgfpathcurveto{\pgfqpoint{2.033753in}{1.543348in}}{\pgfqpoint{2.044352in}{1.538958in}}{\pgfqpoint{2.055402in}{1.538958in}}%
\pgfpathclose%
\pgfusepath{stroke,fill}%
\end{pgfscope}%
\begin{pgfscope}%
\pgfpathrectangle{\pgfqpoint{0.375000in}{0.330000in}}{\pgfqpoint{2.325000in}{2.310000in}}%
\pgfusepath{clip}%
\pgfsetbuttcap%
\pgfsetroundjoin%
\definecolor{currentfill}{rgb}{0.000000,0.000000,0.000000}%
\pgfsetfillcolor{currentfill}%
\pgfsetlinewidth{1.003750pt}%
\definecolor{currentstroke}{rgb}{0.000000,0.000000,0.000000}%
\pgfsetstrokecolor{currentstroke}%
\pgfsetdash{}{0pt}%
\pgfpathmoveto{\pgfqpoint{2.055402in}{1.434896in}}%
\pgfpathcurveto{\pgfqpoint{2.066452in}{1.434896in}}{\pgfqpoint{2.077051in}{1.439286in}}{\pgfqpoint{2.084865in}{1.447100in}}%
\pgfpathcurveto{\pgfqpoint{2.092678in}{1.454913in}}{\pgfqpoint{2.097069in}{1.465512in}}{\pgfqpoint{2.097069in}{1.476563in}}%
\pgfpathcurveto{\pgfqpoint{2.097069in}{1.487613in}}{\pgfqpoint{2.092678in}{1.498212in}}{\pgfqpoint{2.084865in}{1.506025in}}%
\pgfpathcurveto{\pgfqpoint{2.077051in}{1.513839in}}{\pgfqpoint{2.066452in}{1.518229in}}{\pgfqpoint{2.055402in}{1.518229in}}%
\pgfpathcurveto{\pgfqpoint{2.044352in}{1.518229in}}{\pgfqpoint{2.033753in}{1.513839in}}{\pgfqpoint{2.025939in}{1.506025in}}%
\pgfpathcurveto{\pgfqpoint{2.018125in}{1.498212in}}{\pgfqpoint{2.013735in}{1.487613in}}{\pgfqpoint{2.013735in}{1.476563in}}%
\pgfpathcurveto{\pgfqpoint{2.013735in}{1.465512in}}{\pgfqpoint{2.018125in}{1.454913in}}{\pgfqpoint{2.025939in}{1.447100in}}%
\pgfpathcurveto{\pgfqpoint{2.033753in}{1.439286in}}{\pgfqpoint{2.044352in}{1.434896in}}{\pgfqpoint{2.055402in}{1.434896in}}%
\pgfpathclose%
\pgfusepath{stroke,fill}%
\end{pgfscope}%
\begin{pgfscope}%
\pgfpathrectangle{\pgfqpoint{0.375000in}{0.330000in}}{\pgfqpoint{2.325000in}{2.310000in}}%
\pgfusepath{clip}%
\pgfsetbuttcap%
\pgfsetroundjoin%
\definecolor{currentfill}{rgb}{0.000000,0.000000,0.000000}%
\pgfsetfillcolor{currentfill}%
\pgfsetlinewidth{1.003750pt}%
\definecolor{currentstroke}{rgb}{0.000000,0.000000,0.000000}%
\pgfsetstrokecolor{currentstroke}%
\pgfsetdash{}{0pt}%
\pgfpathmoveto{\pgfqpoint{2.055402in}{1.486927in}}%
\pgfpathcurveto{\pgfqpoint{2.066452in}{1.486927in}}{\pgfqpoint{2.077051in}{1.491317in}}{\pgfqpoint{2.084865in}{1.499131in}}%
\pgfpathcurveto{\pgfqpoint{2.092678in}{1.506944in}}{\pgfqpoint{2.097069in}{1.517543in}}{\pgfqpoint{2.097069in}{1.528593in}}%
\pgfpathcurveto{\pgfqpoint{2.097069in}{1.539644in}}{\pgfqpoint{2.092678in}{1.550243in}}{\pgfqpoint{2.084865in}{1.558056in}}%
\pgfpathcurveto{\pgfqpoint{2.077051in}{1.565870in}}{\pgfqpoint{2.066452in}{1.570260in}}{\pgfqpoint{2.055402in}{1.570260in}}%
\pgfpathcurveto{\pgfqpoint{2.044352in}{1.570260in}}{\pgfqpoint{2.033753in}{1.565870in}}{\pgfqpoint{2.025939in}{1.558056in}}%
\pgfpathcurveto{\pgfqpoint{2.018125in}{1.550243in}}{\pgfqpoint{2.013735in}{1.539644in}}{\pgfqpoint{2.013735in}{1.528593in}}%
\pgfpathcurveto{\pgfqpoint{2.013735in}{1.517543in}}{\pgfqpoint{2.018125in}{1.506944in}}{\pgfqpoint{2.025939in}{1.499131in}}%
\pgfpathcurveto{\pgfqpoint{2.033753in}{1.491317in}}{\pgfqpoint{2.044352in}{1.486927in}}{\pgfqpoint{2.055402in}{1.486927in}}%
\pgfpathclose%
\pgfusepath{stroke,fill}%
\end{pgfscope}%
\begin{pgfscope}%
\pgfpathrectangle{\pgfqpoint{0.375000in}{0.330000in}}{\pgfqpoint{2.325000in}{2.310000in}}%
\pgfusepath{clip}%
\pgfsetbuttcap%
\pgfsetroundjoin%
\definecolor{currentfill}{rgb}{0.000000,0.000000,0.000000}%
\pgfsetfillcolor{currentfill}%
\pgfsetlinewidth{1.003750pt}%
\definecolor{currentstroke}{rgb}{0.000000,0.000000,0.000000}%
\pgfsetstrokecolor{currentstroke}%
\pgfsetdash{}{0pt}%
\pgfpathmoveto{\pgfqpoint{2.055402in}{1.538958in}}%
\pgfpathcurveto{\pgfqpoint{2.066452in}{1.538958in}}{\pgfqpoint{2.077051in}{1.543348in}}{\pgfqpoint{2.084865in}{1.551161in}}%
\pgfpathcurveto{\pgfqpoint{2.092678in}{1.558975in}}{\pgfqpoint{2.097069in}{1.569574in}}{\pgfqpoint{2.097069in}{1.580624in}}%
\pgfpathcurveto{\pgfqpoint{2.097069in}{1.591674in}}{\pgfqpoint{2.092678in}{1.602273in}}{\pgfqpoint{2.084865in}{1.610087in}}%
\pgfpathcurveto{\pgfqpoint{2.077051in}{1.617901in}}{\pgfqpoint{2.066452in}{1.622291in}}{\pgfqpoint{2.055402in}{1.622291in}}%
\pgfpathcurveto{\pgfqpoint{2.044352in}{1.622291in}}{\pgfqpoint{2.033753in}{1.617901in}}{\pgfqpoint{2.025939in}{1.610087in}}%
\pgfpathcurveto{\pgfqpoint{2.018125in}{1.602273in}}{\pgfqpoint{2.013735in}{1.591674in}}{\pgfqpoint{2.013735in}{1.580624in}}%
\pgfpathcurveto{\pgfqpoint{2.013735in}{1.569574in}}{\pgfqpoint{2.018125in}{1.558975in}}{\pgfqpoint{2.025939in}{1.551161in}}%
\pgfpathcurveto{\pgfqpoint{2.033753in}{1.543348in}}{\pgfqpoint{2.044352in}{1.538958in}}{\pgfqpoint{2.055402in}{1.538958in}}%
\pgfpathclose%
\pgfusepath{stroke,fill}%
\end{pgfscope}%
\begin{pgfscope}%
\pgfpathrectangle{\pgfqpoint{0.375000in}{0.330000in}}{\pgfqpoint{2.325000in}{2.310000in}}%
\pgfusepath{clip}%
\pgfsetbuttcap%
\pgfsetroundjoin%
\definecolor{currentfill}{rgb}{0.000000,0.000000,0.000000}%
\pgfsetfillcolor{currentfill}%
\pgfsetlinewidth{1.003750pt}%
\definecolor{currentstroke}{rgb}{0.000000,0.000000,0.000000}%
\pgfsetstrokecolor{currentstroke}%
\pgfsetdash{}{0pt}%
\pgfpathmoveto{\pgfqpoint{2.055402in}{1.486927in}}%
\pgfpathcurveto{\pgfqpoint{2.066452in}{1.486927in}}{\pgfqpoint{2.077051in}{1.491317in}}{\pgfqpoint{2.084865in}{1.499131in}}%
\pgfpathcurveto{\pgfqpoint{2.092678in}{1.506944in}}{\pgfqpoint{2.097069in}{1.517543in}}{\pgfqpoint{2.097069in}{1.528593in}}%
\pgfpathcurveto{\pgfqpoint{2.097069in}{1.539644in}}{\pgfqpoint{2.092678in}{1.550243in}}{\pgfqpoint{2.084865in}{1.558056in}}%
\pgfpathcurveto{\pgfqpoint{2.077051in}{1.565870in}}{\pgfqpoint{2.066452in}{1.570260in}}{\pgfqpoint{2.055402in}{1.570260in}}%
\pgfpathcurveto{\pgfqpoint{2.044352in}{1.570260in}}{\pgfqpoint{2.033753in}{1.565870in}}{\pgfqpoint{2.025939in}{1.558056in}}%
\pgfpathcurveto{\pgfqpoint{2.018125in}{1.550243in}}{\pgfqpoint{2.013735in}{1.539644in}}{\pgfqpoint{2.013735in}{1.528593in}}%
\pgfpathcurveto{\pgfqpoint{2.013735in}{1.517543in}}{\pgfqpoint{2.018125in}{1.506944in}}{\pgfqpoint{2.025939in}{1.499131in}}%
\pgfpathcurveto{\pgfqpoint{2.033753in}{1.491317in}}{\pgfqpoint{2.044352in}{1.486927in}}{\pgfqpoint{2.055402in}{1.486927in}}%
\pgfpathclose%
\pgfusepath{stroke,fill}%
\end{pgfscope}%
\begin{pgfscope}%
\pgfpathrectangle{\pgfqpoint{0.375000in}{0.330000in}}{\pgfqpoint{2.325000in}{2.310000in}}%
\pgfusepath{clip}%
\pgfsetbuttcap%
\pgfsetroundjoin%
\definecolor{currentfill}{rgb}{0.000000,0.000000,0.000000}%
\pgfsetfillcolor{currentfill}%
\pgfsetlinewidth{1.003750pt}%
\definecolor{currentstroke}{rgb}{0.000000,0.000000,0.000000}%
\pgfsetstrokecolor{currentstroke}%
\pgfsetdash{}{0pt}%
\pgfpathmoveto{\pgfqpoint{2.055402in}{1.538958in}}%
\pgfpathcurveto{\pgfqpoint{2.066452in}{1.538958in}}{\pgfqpoint{2.077051in}{1.543348in}}{\pgfqpoint{2.084865in}{1.551161in}}%
\pgfpathcurveto{\pgfqpoint{2.092678in}{1.558975in}}{\pgfqpoint{2.097069in}{1.569574in}}{\pgfqpoint{2.097069in}{1.580624in}}%
\pgfpathcurveto{\pgfqpoint{2.097069in}{1.591674in}}{\pgfqpoint{2.092678in}{1.602273in}}{\pgfqpoint{2.084865in}{1.610087in}}%
\pgfpathcurveto{\pgfqpoint{2.077051in}{1.617901in}}{\pgfqpoint{2.066452in}{1.622291in}}{\pgfqpoint{2.055402in}{1.622291in}}%
\pgfpathcurveto{\pgfqpoint{2.044352in}{1.622291in}}{\pgfqpoint{2.033753in}{1.617901in}}{\pgfqpoint{2.025939in}{1.610087in}}%
\pgfpathcurveto{\pgfqpoint{2.018125in}{1.602273in}}{\pgfqpoint{2.013735in}{1.591674in}}{\pgfqpoint{2.013735in}{1.580624in}}%
\pgfpathcurveto{\pgfqpoint{2.013735in}{1.569574in}}{\pgfqpoint{2.018125in}{1.558975in}}{\pgfqpoint{2.025939in}{1.551161in}}%
\pgfpathcurveto{\pgfqpoint{2.033753in}{1.543348in}}{\pgfqpoint{2.044352in}{1.538958in}}{\pgfqpoint{2.055402in}{1.538958in}}%
\pgfpathclose%
\pgfusepath{stroke,fill}%
\end{pgfscope}%
\begin{pgfscope}%
\pgfpathrectangle{\pgfqpoint{0.375000in}{0.330000in}}{\pgfqpoint{2.325000in}{2.310000in}}%
\pgfusepath{clip}%
\pgfsetbuttcap%
\pgfsetroundjoin%
\definecolor{currentfill}{rgb}{0.000000,0.000000,0.000000}%
\pgfsetfillcolor{currentfill}%
\pgfsetlinewidth{1.003750pt}%
\definecolor{currentstroke}{rgb}{0.000000,0.000000,0.000000}%
\pgfsetstrokecolor{currentstroke}%
\pgfsetdash{}{0pt}%
\pgfpathmoveto{\pgfqpoint{2.055402in}{1.486927in}}%
\pgfpathcurveto{\pgfqpoint{2.066452in}{1.486927in}}{\pgfqpoint{2.077051in}{1.491317in}}{\pgfqpoint{2.084865in}{1.499131in}}%
\pgfpathcurveto{\pgfqpoint{2.092678in}{1.506944in}}{\pgfqpoint{2.097069in}{1.517543in}}{\pgfqpoint{2.097069in}{1.528593in}}%
\pgfpathcurveto{\pgfqpoint{2.097069in}{1.539644in}}{\pgfqpoint{2.092678in}{1.550243in}}{\pgfqpoint{2.084865in}{1.558056in}}%
\pgfpathcurveto{\pgfqpoint{2.077051in}{1.565870in}}{\pgfqpoint{2.066452in}{1.570260in}}{\pgfqpoint{2.055402in}{1.570260in}}%
\pgfpathcurveto{\pgfqpoint{2.044352in}{1.570260in}}{\pgfqpoint{2.033753in}{1.565870in}}{\pgfqpoint{2.025939in}{1.558056in}}%
\pgfpathcurveto{\pgfqpoint{2.018125in}{1.550243in}}{\pgfqpoint{2.013735in}{1.539644in}}{\pgfqpoint{2.013735in}{1.528593in}}%
\pgfpathcurveto{\pgfqpoint{2.013735in}{1.517543in}}{\pgfqpoint{2.018125in}{1.506944in}}{\pgfqpoint{2.025939in}{1.499131in}}%
\pgfpathcurveto{\pgfqpoint{2.033753in}{1.491317in}}{\pgfqpoint{2.044352in}{1.486927in}}{\pgfqpoint{2.055402in}{1.486927in}}%
\pgfpathclose%
\pgfusepath{stroke,fill}%
\end{pgfscope}%
\begin{pgfscope}%
\pgfpathrectangle{\pgfqpoint{0.375000in}{0.330000in}}{\pgfqpoint{2.325000in}{2.310000in}}%
\pgfusepath{clip}%
\pgfsetbuttcap%
\pgfsetroundjoin%
\definecolor{currentfill}{rgb}{0.000000,0.000000,0.000000}%
\pgfsetfillcolor{currentfill}%
\pgfsetlinewidth{1.003750pt}%
\definecolor{currentstroke}{rgb}{0.000000,0.000000,0.000000}%
\pgfsetstrokecolor{currentstroke}%
\pgfsetdash{}{0pt}%
\pgfpathmoveto{\pgfqpoint{2.055402in}{1.434896in}}%
\pgfpathcurveto{\pgfqpoint{2.066452in}{1.434896in}}{\pgfqpoint{2.077051in}{1.439286in}}{\pgfqpoint{2.084865in}{1.447100in}}%
\pgfpathcurveto{\pgfqpoint{2.092678in}{1.454913in}}{\pgfqpoint{2.097069in}{1.465512in}}{\pgfqpoint{2.097069in}{1.476563in}}%
\pgfpathcurveto{\pgfqpoint{2.097069in}{1.487613in}}{\pgfqpoint{2.092678in}{1.498212in}}{\pgfqpoint{2.084865in}{1.506025in}}%
\pgfpathcurveto{\pgfqpoint{2.077051in}{1.513839in}}{\pgfqpoint{2.066452in}{1.518229in}}{\pgfqpoint{2.055402in}{1.518229in}}%
\pgfpathcurveto{\pgfqpoint{2.044352in}{1.518229in}}{\pgfqpoint{2.033753in}{1.513839in}}{\pgfqpoint{2.025939in}{1.506025in}}%
\pgfpathcurveto{\pgfqpoint{2.018125in}{1.498212in}}{\pgfqpoint{2.013735in}{1.487613in}}{\pgfqpoint{2.013735in}{1.476563in}}%
\pgfpathcurveto{\pgfqpoint{2.013735in}{1.465512in}}{\pgfqpoint{2.018125in}{1.454913in}}{\pgfqpoint{2.025939in}{1.447100in}}%
\pgfpathcurveto{\pgfqpoint{2.033753in}{1.439286in}}{\pgfqpoint{2.044352in}{1.434896in}}{\pgfqpoint{2.055402in}{1.434896in}}%
\pgfpathclose%
\pgfusepath{stroke,fill}%
\end{pgfscope}%
\begin{pgfscope}%
\pgfpathrectangle{\pgfqpoint{0.375000in}{0.330000in}}{\pgfqpoint{2.325000in}{2.310000in}}%
\pgfusepath{clip}%
\pgfsetbuttcap%
\pgfsetroundjoin%
\definecolor{currentfill}{rgb}{0.000000,0.000000,0.000000}%
\pgfsetfillcolor{currentfill}%
\pgfsetlinewidth{1.003750pt}%
\definecolor{currentstroke}{rgb}{0.000000,0.000000,0.000000}%
\pgfsetstrokecolor{currentstroke}%
\pgfsetdash{}{0pt}%
\pgfpathmoveto{\pgfqpoint{2.055402in}{1.382865in}}%
\pgfpathcurveto{\pgfqpoint{2.066452in}{1.382865in}}{\pgfqpoint{2.077051in}{1.387255in}}{\pgfqpoint{2.084865in}{1.395069in}}%
\pgfpathcurveto{\pgfqpoint{2.092678in}{1.402883in}}{\pgfqpoint{2.097069in}{1.413482in}}{\pgfqpoint{2.097069in}{1.424532in}}%
\pgfpathcurveto{\pgfqpoint{2.097069in}{1.435582in}}{\pgfqpoint{2.092678in}{1.446181in}}{\pgfqpoint{2.084865in}{1.453995in}}%
\pgfpathcurveto{\pgfqpoint{2.077051in}{1.461808in}}{\pgfqpoint{2.066452in}{1.466198in}}{\pgfqpoint{2.055402in}{1.466198in}}%
\pgfpathcurveto{\pgfqpoint{2.044352in}{1.466198in}}{\pgfqpoint{2.033753in}{1.461808in}}{\pgfqpoint{2.025939in}{1.453995in}}%
\pgfpathcurveto{\pgfqpoint{2.018125in}{1.446181in}}{\pgfqpoint{2.013735in}{1.435582in}}{\pgfqpoint{2.013735in}{1.424532in}}%
\pgfpathcurveto{\pgfqpoint{2.013735in}{1.413482in}}{\pgfqpoint{2.018125in}{1.402883in}}{\pgfqpoint{2.025939in}{1.395069in}}%
\pgfpathcurveto{\pgfqpoint{2.033753in}{1.387255in}}{\pgfqpoint{2.044352in}{1.382865in}}{\pgfqpoint{2.055402in}{1.382865in}}%
\pgfpathclose%
\pgfusepath{stroke,fill}%
\end{pgfscope}%
\begin{pgfscope}%
\pgfpathrectangle{\pgfqpoint{0.375000in}{0.330000in}}{\pgfqpoint{2.325000in}{2.310000in}}%
\pgfusepath{clip}%
\pgfsetbuttcap%
\pgfsetroundjoin%
\definecolor{currentfill}{rgb}{0.000000,0.000000,0.000000}%
\pgfsetfillcolor{currentfill}%
\pgfsetlinewidth{1.003750pt}%
\definecolor{currentstroke}{rgb}{0.000000,0.000000,0.000000}%
\pgfsetstrokecolor{currentstroke}%
\pgfsetdash{}{0pt}%
\pgfpathmoveto{\pgfqpoint{2.055402in}{1.434896in}}%
\pgfpathcurveto{\pgfqpoint{2.066452in}{1.434896in}}{\pgfqpoint{2.077051in}{1.439286in}}{\pgfqpoint{2.084865in}{1.447100in}}%
\pgfpathcurveto{\pgfqpoint{2.092678in}{1.454913in}}{\pgfqpoint{2.097069in}{1.465512in}}{\pgfqpoint{2.097069in}{1.476563in}}%
\pgfpathcurveto{\pgfqpoint{2.097069in}{1.487613in}}{\pgfqpoint{2.092678in}{1.498212in}}{\pgfqpoint{2.084865in}{1.506025in}}%
\pgfpathcurveto{\pgfqpoint{2.077051in}{1.513839in}}{\pgfqpoint{2.066452in}{1.518229in}}{\pgfqpoint{2.055402in}{1.518229in}}%
\pgfpathcurveto{\pgfqpoint{2.044352in}{1.518229in}}{\pgfqpoint{2.033753in}{1.513839in}}{\pgfqpoint{2.025939in}{1.506025in}}%
\pgfpathcurveto{\pgfqpoint{2.018125in}{1.498212in}}{\pgfqpoint{2.013735in}{1.487613in}}{\pgfqpoint{2.013735in}{1.476563in}}%
\pgfpathcurveto{\pgfqpoint{2.013735in}{1.465512in}}{\pgfqpoint{2.018125in}{1.454913in}}{\pgfqpoint{2.025939in}{1.447100in}}%
\pgfpathcurveto{\pgfqpoint{2.033753in}{1.439286in}}{\pgfqpoint{2.044352in}{1.434896in}}{\pgfqpoint{2.055402in}{1.434896in}}%
\pgfpathclose%
\pgfusepath{stroke,fill}%
\end{pgfscope}%
\begin{pgfscope}%
\pgfpathrectangle{\pgfqpoint{0.375000in}{0.330000in}}{\pgfqpoint{2.325000in}{2.310000in}}%
\pgfusepath{clip}%
\pgfsetbuttcap%
\pgfsetroundjoin%
\definecolor{currentfill}{rgb}{0.000000,0.000000,0.000000}%
\pgfsetfillcolor{currentfill}%
\pgfsetlinewidth{1.003750pt}%
\definecolor{currentstroke}{rgb}{0.000000,0.000000,0.000000}%
\pgfsetstrokecolor{currentstroke}%
\pgfsetdash{}{0pt}%
\pgfpathmoveto{\pgfqpoint{2.055402in}{1.486927in}}%
\pgfpathcurveto{\pgfqpoint{2.066452in}{1.486927in}}{\pgfqpoint{2.077051in}{1.491317in}}{\pgfqpoint{2.084865in}{1.499131in}}%
\pgfpathcurveto{\pgfqpoint{2.092678in}{1.506944in}}{\pgfqpoint{2.097069in}{1.517543in}}{\pgfqpoint{2.097069in}{1.528593in}}%
\pgfpathcurveto{\pgfqpoint{2.097069in}{1.539644in}}{\pgfqpoint{2.092678in}{1.550243in}}{\pgfqpoint{2.084865in}{1.558056in}}%
\pgfpathcurveto{\pgfqpoint{2.077051in}{1.565870in}}{\pgfqpoint{2.066452in}{1.570260in}}{\pgfqpoint{2.055402in}{1.570260in}}%
\pgfpathcurveto{\pgfqpoint{2.044352in}{1.570260in}}{\pgfqpoint{2.033753in}{1.565870in}}{\pgfqpoint{2.025939in}{1.558056in}}%
\pgfpathcurveto{\pgfqpoint{2.018125in}{1.550243in}}{\pgfqpoint{2.013735in}{1.539644in}}{\pgfqpoint{2.013735in}{1.528593in}}%
\pgfpathcurveto{\pgfqpoint{2.013735in}{1.517543in}}{\pgfqpoint{2.018125in}{1.506944in}}{\pgfqpoint{2.025939in}{1.499131in}}%
\pgfpathcurveto{\pgfqpoint{2.033753in}{1.491317in}}{\pgfqpoint{2.044352in}{1.486927in}}{\pgfqpoint{2.055402in}{1.486927in}}%
\pgfpathclose%
\pgfusepath{stroke,fill}%
\end{pgfscope}%
\begin{pgfscope}%
\pgfpathrectangle{\pgfqpoint{0.375000in}{0.330000in}}{\pgfqpoint{2.325000in}{2.310000in}}%
\pgfusepath{clip}%
\pgfsetbuttcap%
\pgfsetroundjoin%
\definecolor{currentfill}{rgb}{0.000000,0.000000,0.000000}%
\pgfsetfillcolor{currentfill}%
\pgfsetlinewidth{1.003750pt}%
\definecolor{currentstroke}{rgb}{0.000000,0.000000,0.000000}%
\pgfsetstrokecolor{currentstroke}%
\pgfsetdash{}{0pt}%
\pgfpathmoveto{\pgfqpoint{2.055402in}{1.486927in}}%
\pgfpathcurveto{\pgfqpoint{2.066452in}{1.486927in}}{\pgfqpoint{2.077051in}{1.491317in}}{\pgfqpoint{2.084865in}{1.499131in}}%
\pgfpathcurveto{\pgfqpoint{2.092678in}{1.506944in}}{\pgfqpoint{2.097069in}{1.517543in}}{\pgfqpoint{2.097069in}{1.528593in}}%
\pgfpathcurveto{\pgfqpoint{2.097069in}{1.539644in}}{\pgfqpoint{2.092678in}{1.550243in}}{\pgfqpoint{2.084865in}{1.558056in}}%
\pgfpathcurveto{\pgfqpoint{2.077051in}{1.565870in}}{\pgfqpoint{2.066452in}{1.570260in}}{\pgfqpoint{2.055402in}{1.570260in}}%
\pgfpathcurveto{\pgfqpoint{2.044352in}{1.570260in}}{\pgfqpoint{2.033753in}{1.565870in}}{\pgfqpoint{2.025939in}{1.558056in}}%
\pgfpathcurveto{\pgfqpoint{2.018125in}{1.550243in}}{\pgfqpoint{2.013735in}{1.539644in}}{\pgfqpoint{2.013735in}{1.528593in}}%
\pgfpathcurveto{\pgfqpoint{2.013735in}{1.517543in}}{\pgfqpoint{2.018125in}{1.506944in}}{\pgfqpoint{2.025939in}{1.499131in}}%
\pgfpathcurveto{\pgfqpoint{2.033753in}{1.491317in}}{\pgfqpoint{2.044352in}{1.486927in}}{\pgfqpoint{2.055402in}{1.486927in}}%
\pgfpathclose%
\pgfusepath{stroke,fill}%
\end{pgfscope}%
\begin{pgfscope}%
\pgfpathrectangle{\pgfqpoint{0.375000in}{0.330000in}}{\pgfqpoint{2.325000in}{2.310000in}}%
\pgfusepath{clip}%
\pgfsetbuttcap%
\pgfsetroundjoin%
\definecolor{currentfill}{rgb}{0.000000,0.000000,0.000000}%
\pgfsetfillcolor{currentfill}%
\pgfsetlinewidth{1.003750pt}%
\definecolor{currentstroke}{rgb}{0.000000,0.000000,0.000000}%
\pgfsetstrokecolor{currentstroke}%
\pgfsetdash{}{0pt}%
\pgfpathmoveto{\pgfqpoint{2.055402in}{1.538958in}}%
\pgfpathcurveto{\pgfqpoint{2.066452in}{1.538958in}}{\pgfqpoint{2.077051in}{1.543348in}}{\pgfqpoint{2.084865in}{1.551161in}}%
\pgfpathcurveto{\pgfqpoint{2.092678in}{1.558975in}}{\pgfqpoint{2.097069in}{1.569574in}}{\pgfqpoint{2.097069in}{1.580624in}}%
\pgfpathcurveto{\pgfqpoint{2.097069in}{1.591674in}}{\pgfqpoint{2.092678in}{1.602273in}}{\pgfqpoint{2.084865in}{1.610087in}}%
\pgfpathcurveto{\pgfqpoint{2.077051in}{1.617901in}}{\pgfqpoint{2.066452in}{1.622291in}}{\pgfqpoint{2.055402in}{1.622291in}}%
\pgfpathcurveto{\pgfqpoint{2.044352in}{1.622291in}}{\pgfqpoint{2.033753in}{1.617901in}}{\pgfqpoint{2.025939in}{1.610087in}}%
\pgfpathcurveto{\pgfqpoint{2.018125in}{1.602273in}}{\pgfqpoint{2.013735in}{1.591674in}}{\pgfqpoint{2.013735in}{1.580624in}}%
\pgfpathcurveto{\pgfqpoint{2.013735in}{1.569574in}}{\pgfqpoint{2.018125in}{1.558975in}}{\pgfqpoint{2.025939in}{1.551161in}}%
\pgfpathcurveto{\pgfqpoint{2.033753in}{1.543348in}}{\pgfqpoint{2.044352in}{1.538958in}}{\pgfqpoint{2.055402in}{1.538958in}}%
\pgfpathclose%
\pgfusepath{stroke,fill}%
\end{pgfscope}%
\begin{pgfscope}%
\pgfpathrectangle{\pgfqpoint{0.375000in}{0.330000in}}{\pgfqpoint{2.325000in}{2.310000in}}%
\pgfusepath{clip}%
\pgfsetbuttcap%
\pgfsetroundjoin%
\definecolor{currentfill}{rgb}{0.000000,0.000000,0.000000}%
\pgfsetfillcolor{currentfill}%
\pgfsetlinewidth{1.003750pt}%
\definecolor{currentstroke}{rgb}{0.000000,0.000000,0.000000}%
\pgfsetstrokecolor{currentstroke}%
\pgfsetdash{}{0pt}%
\pgfpathmoveto{\pgfqpoint{2.055402in}{1.486927in}}%
\pgfpathcurveto{\pgfqpoint{2.066452in}{1.486927in}}{\pgfqpoint{2.077051in}{1.491317in}}{\pgfqpoint{2.084865in}{1.499131in}}%
\pgfpathcurveto{\pgfqpoint{2.092678in}{1.506944in}}{\pgfqpoint{2.097069in}{1.517543in}}{\pgfqpoint{2.097069in}{1.528593in}}%
\pgfpathcurveto{\pgfqpoint{2.097069in}{1.539644in}}{\pgfqpoint{2.092678in}{1.550243in}}{\pgfqpoint{2.084865in}{1.558056in}}%
\pgfpathcurveto{\pgfqpoint{2.077051in}{1.565870in}}{\pgfqpoint{2.066452in}{1.570260in}}{\pgfqpoint{2.055402in}{1.570260in}}%
\pgfpathcurveto{\pgfqpoint{2.044352in}{1.570260in}}{\pgfqpoint{2.033753in}{1.565870in}}{\pgfqpoint{2.025939in}{1.558056in}}%
\pgfpathcurveto{\pgfqpoint{2.018125in}{1.550243in}}{\pgfqpoint{2.013735in}{1.539644in}}{\pgfqpoint{2.013735in}{1.528593in}}%
\pgfpathcurveto{\pgfqpoint{2.013735in}{1.517543in}}{\pgfqpoint{2.018125in}{1.506944in}}{\pgfqpoint{2.025939in}{1.499131in}}%
\pgfpathcurveto{\pgfqpoint{2.033753in}{1.491317in}}{\pgfqpoint{2.044352in}{1.486927in}}{\pgfqpoint{2.055402in}{1.486927in}}%
\pgfpathclose%
\pgfusepath{stroke,fill}%
\end{pgfscope}%
\begin{pgfscope}%
\pgfpathrectangle{\pgfqpoint{0.375000in}{0.330000in}}{\pgfqpoint{2.325000in}{2.310000in}}%
\pgfusepath{clip}%
\pgfsetbuttcap%
\pgfsetroundjoin%
\definecolor{currentfill}{rgb}{0.000000,0.000000,0.000000}%
\pgfsetfillcolor{currentfill}%
\pgfsetlinewidth{1.003750pt}%
\definecolor{currentstroke}{rgb}{0.000000,0.000000,0.000000}%
\pgfsetstrokecolor{currentstroke}%
\pgfsetdash{}{0pt}%
\pgfpathmoveto{\pgfqpoint{2.055402in}{1.486927in}}%
\pgfpathcurveto{\pgfqpoint{2.066452in}{1.486927in}}{\pgfqpoint{2.077051in}{1.491317in}}{\pgfqpoint{2.084865in}{1.499131in}}%
\pgfpathcurveto{\pgfqpoint{2.092678in}{1.506944in}}{\pgfqpoint{2.097069in}{1.517543in}}{\pgfqpoint{2.097069in}{1.528593in}}%
\pgfpathcurveto{\pgfqpoint{2.097069in}{1.539644in}}{\pgfqpoint{2.092678in}{1.550243in}}{\pgfqpoint{2.084865in}{1.558056in}}%
\pgfpathcurveto{\pgfqpoint{2.077051in}{1.565870in}}{\pgfqpoint{2.066452in}{1.570260in}}{\pgfqpoint{2.055402in}{1.570260in}}%
\pgfpathcurveto{\pgfqpoint{2.044352in}{1.570260in}}{\pgfqpoint{2.033753in}{1.565870in}}{\pgfqpoint{2.025939in}{1.558056in}}%
\pgfpathcurveto{\pgfqpoint{2.018125in}{1.550243in}}{\pgfqpoint{2.013735in}{1.539644in}}{\pgfqpoint{2.013735in}{1.528593in}}%
\pgfpathcurveto{\pgfqpoint{2.013735in}{1.517543in}}{\pgfqpoint{2.018125in}{1.506944in}}{\pgfqpoint{2.025939in}{1.499131in}}%
\pgfpathcurveto{\pgfqpoint{2.033753in}{1.491317in}}{\pgfqpoint{2.044352in}{1.486927in}}{\pgfqpoint{2.055402in}{1.486927in}}%
\pgfpathclose%
\pgfusepath{stroke,fill}%
\end{pgfscope}%
\begin{pgfscope}%
\pgfpathrectangle{\pgfqpoint{0.375000in}{0.330000in}}{\pgfqpoint{2.325000in}{2.310000in}}%
\pgfusepath{clip}%
\pgfsetbuttcap%
\pgfsetroundjoin%
\definecolor{currentfill}{rgb}{0.000000,0.000000,0.000000}%
\pgfsetfillcolor{currentfill}%
\pgfsetlinewidth{1.003750pt}%
\definecolor{currentstroke}{rgb}{0.000000,0.000000,0.000000}%
\pgfsetstrokecolor{currentstroke}%
\pgfsetdash{}{0pt}%
\pgfpathmoveto{\pgfqpoint{2.055402in}{1.486927in}}%
\pgfpathcurveto{\pgfqpoint{2.066452in}{1.486927in}}{\pgfqpoint{2.077051in}{1.491317in}}{\pgfqpoint{2.084865in}{1.499131in}}%
\pgfpathcurveto{\pgfqpoint{2.092678in}{1.506944in}}{\pgfqpoint{2.097069in}{1.517543in}}{\pgfqpoint{2.097069in}{1.528593in}}%
\pgfpathcurveto{\pgfqpoint{2.097069in}{1.539644in}}{\pgfqpoint{2.092678in}{1.550243in}}{\pgfqpoint{2.084865in}{1.558056in}}%
\pgfpathcurveto{\pgfqpoint{2.077051in}{1.565870in}}{\pgfqpoint{2.066452in}{1.570260in}}{\pgfqpoint{2.055402in}{1.570260in}}%
\pgfpathcurveto{\pgfqpoint{2.044352in}{1.570260in}}{\pgfqpoint{2.033753in}{1.565870in}}{\pgfqpoint{2.025939in}{1.558056in}}%
\pgfpathcurveto{\pgfqpoint{2.018125in}{1.550243in}}{\pgfqpoint{2.013735in}{1.539644in}}{\pgfqpoint{2.013735in}{1.528593in}}%
\pgfpathcurveto{\pgfqpoint{2.013735in}{1.517543in}}{\pgfqpoint{2.018125in}{1.506944in}}{\pgfqpoint{2.025939in}{1.499131in}}%
\pgfpathcurveto{\pgfqpoint{2.033753in}{1.491317in}}{\pgfqpoint{2.044352in}{1.486927in}}{\pgfqpoint{2.055402in}{1.486927in}}%
\pgfpathclose%
\pgfusepath{stroke,fill}%
\end{pgfscope}%
\begin{pgfscope}%
\pgfpathrectangle{\pgfqpoint{0.375000in}{0.330000in}}{\pgfqpoint{2.325000in}{2.310000in}}%
\pgfusepath{clip}%
\pgfsetbuttcap%
\pgfsetroundjoin%
\definecolor{currentfill}{rgb}{0.000000,0.000000,0.000000}%
\pgfsetfillcolor{currentfill}%
\pgfsetlinewidth{1.003750pt}%
\definecolor{currentstroke}{rgb}{0.000000,0.000000,0.000000}%
\pgfsetstrokecolor{currentstroke}%
\pgfsetdash{}{0pt}%
\pgfpathmoveto{\pgfqpoint{2.055402in}{1.382865in}}%
\pgfpathcurveto{\pgfqpoint{2.066452in}{1.382865in}}{\pgfqpoint{2.077051in}{1.387255in}}{\pgfqpoint{2.084865in}{1.395069in}}%
\pgfpathcurveto{\pgfqpoint{2.092678in}{1.402883in}}{\pgfqpoint{2.097069in}{1.413482in}}{\pgfqpoint{2.097069in}{1.424532in}}%
\pgfpathcurveto{\pgfqpoint{2.097069in}{1.435582in}}{\pgfqpoint{2.092678in}{1.446181in}}{\pgfqpoint{2.084865in}{1.453995in}}%
\pgfpathcurveto{\pgfqpoint{2.077051in}{1.461808in}}{\pgfqpoint{2.066452in}{1.466198in}}{\pgfqpoint{2.055402in}{1.466198in}}%
\pgfpathcurveto{\pgfqpoint{2.044352in}{1.466198in}}{\pgfqpoint{2.033753in}{1.461808in}}{\pgfqpoint{2.025939in}{1.453995in}}%
\pgfpathcurveto{\pgfqpoint{2.018125in}{1.446181in}}{\pgfqpoint{2.013735in}{1.435582in}}{\pgfqpoint{2.013735in}{1.424532in}}%
\pgfpathcurveto{\pgfqpoint{2.013735in}{1.413482in}}{\pgfqpoint{2.018125in}{1.402883in}}{\pgfqpoint{2.025939in}{1.395069in}}%
\pgfpathcurveto{\pgfqpoint{2.033753in}{1.387255in}}{\pgfqpoint{2.044352in}{1.382865in}}{\pgfqpoint{2.055402in}{1.382865in}}%
\pgfpathclose%
\pgfusepath{stroke,fill}%
\end{pgfscope}%
\begin{pgfscope}%
\pgfpathrectangle{\pgfqpoint{0.375000in}{0.330000in}}{\pgfqpoint{2.325000in}{2.310000in}}%
\pgfusepath{clip}%
\pgfsetbuttcap%
\pgfsetroundjoin%
\definecolor{currentfill}{rgb}{0.000000,0.000000,0.000000}%
\pgfsetfillcolor{currentfill}%
\pgfsetlinewidth{1.003750pt}%
\definecolor{currentstroke}{rgb}{0.000000,0.000000,0.000000}%
\pgfsetstrokecolor{currentstroke}%
\pgfsetdash{}{0pt}%
\pgfpathmoveto{\pgfqpoint{2.055402in}{1.486927in}}%
\pgfpathcurveto{\pgfqpoint{2.066452in}{1.486927in}}{\pgfqpoint{2.077051in}{1.491317in}}{\pgfqpoint{2.084865in}{1.499131in}}%
\pgfpathcurveto{\pgfqpoint{2.092678in}{1.506944in}}{\pgfqpoint{2.097069in}{1.517543in}}{\pgfqpoint{2.097069in}{1.528593in}}%
\pgfpathcurveto{\pgfqpoint{2.097069in}{1.539644in}}{\pgfqpoint{2.092678in}{1.550243in}}{\pgfqpoint{2.084865in}{1.558056in}}%
\pgfpathcurveto{\pgfqpoint{2.077051in}{1.565870in}}{\pgfqpoint{2.066452in}{1.570260in}}{\pgfqpoint{2.055402in}{1.570260in}}%
\pgfpathcurveto{\pgfqpoint{2.044352in}{1.570260in}}{\pgfqpoint{2.033753in}{1.565870in}}{\pgfqpoint{2.025939in}{1.558056in}}%
\pgfpathcurveto{\pgfqpoint{2.018125in}{1.550243in}}{\pgfqpoint{2.013735in}{1.539644in}}{\pgfqpoint{2.013735in}{1.528593in}}%
\pgfpathcurveto{\pgfqpoint{2.013735in}{1.517543in}}{\pgfqpoint{2.018125in}{1.506944in}}{\pgfqpoint{2.025939in}{1.499131in}}%
\pgfpathcurveto{\pgfqpoint{2.033753in}{1.491317in}}{\pgfqpoint{2.044352in}{1.486927in}}{\pgfqpoint{2.055402in}{1.486927in}}%
\pgfpathclose%
\pgfusepath{stroke,fill}%
\end{pgfscope}%
\begin{pgfscope}%
\pgfpathrectangle{\pgfqpoint{0.375000in}{0.330000in}}{\pgfqpoint{2.325000in}{2.310000in}}%
\pgfusepath{clip}%
\pgfsetbuttcap%
\pgfsetroundjoin%
\definecolor{currentfill}{rgb}{0.000000,0.000000,0.000000}%
\pgfsetfillcolor{currentfill}%
\pgfsetlinewidth{1.003750pt}%
\definecolor{currentstroke}{rgb}{0.000000,0.000000,0.000000}%
\pgfsetstrokecolor{currentstroke}%
\pgfsetdash{}{0pt}%
\pgfpathmoveto{\pgfqpoint{2.055402in}{1.538958in}}%
\pgfpathcurveto{\pgfqpoint{2.066452in}{1.538958in}}{\pgfqpoint{2.077051in}{1.543348in}}{\pgfqpoint{2.084865in}{1.551161in}}%
\pgfpathcurveto{\pgfqpoint{2.092678in}{1.558975in}}{\pgfqpoint{2.097069in}{1.569574in}}{\pgfqpoint{2.097069in}{1.580624in}}%
\pgfpathcurveto{\pgfqpoint{2.097069in}{1.591674in}}{\pgfqpoint{2.092678in}{1.602273in}}{\pgfqpoint{2.084865in}{1.610087in}}%
\pgfpathcurveto{\pgfqpoint{2.077051in}{1.617901in}}{\pgfqpoint{2.066452in}{1.622291in}}{\pgfqpoint{2.055402in}{1.622291in}}%
\pgfpathcurveto{\pgfqpoint{2.044352in}{1.622291in}}{\pgfqpoint{2.033753in}{1.617901in}}{\pgfqpoint{2.025939in}{1.610087in}}%
\pgfpathcurveto{\pgfqpoint{2.018125in}{1.602273in}}{\pgfqpoint{2.013735in}{1.591674in}}{\pgfqpoint{2.013735in}{1.580624in}}%
\pgfpathcurveto{\pgfqpoint{2.013735in}{1.569574in}}{\pgfqpoint{2.018125in}{1.558975in}}{\pgfqpoint{2.025939in}{1.551161in}}%
\pgfpathcurveto{\pgfqpoint{2.033753in}{1.543348in}}{\pgfqpoint{2.044352in}{1.538958in}}{\pgfqpoint{2.055402in}{1.538958in}}%
\pgfpathclose%
\pgfusepath{stroke,fill}%
\end{pgfscope}%
\begin{pgfscope}%
\pgfpathrectangle{\pgfqpoint{0.375000in}{0.330000in}}{\pgfqpoint{2.325000in}{2.310000in}}%
\pgfusepath{clip}%
\pgfsetbuttcap%
\pgfsetroundjoin%
\definecolor{currentfill}{rgb}{0.000000,0.000000,0.000000}%
\pgfsetfillcolor{currentfill}%
\pgfsetlinewidth{1.003750pt}%
\definecolor{currentstroke}{rgb}{0.000000,0.000000,0.000000}%
\pgfsetstrokecolor{currentstroke}%
\pgfsetdash{}{0pt}%
\pgfpathmoveto{\pgfqpoint{2.055402in}{1.538958in}}%
\pgfpathcurveto{\pgfqpoint{2.066452in}{1.538958in}}{\pgfqpoint{2.077051in}{1.543348in}}{\pgfqpoint{2.084865in}{1.551161in}}%
\pgfpathcurveto{\pgfqpoint{2.092678in}{1.558975in}}{\pgfqpoint{2.097069in}{1.569574in}}{\pgfqpoint{2.097069in}{1.580624in}}%
\pgfpathcurveto{\pgfqpoint{2.097069in}{1.591674in}}{\pgfqpoint{2.092678in}{1.602273in}}{\pgfqpoint{2.084865in}{1.610087in}}%
\pgfpathcurveto{\pgfqpoint{2.077051in}{1.617901in}}{\pgfqpoint{2.066452in}{1.622291in}}{\pgfqpoint{2.055402in}{1.622291in}}%
\pgfpathcurveto{\pgfqpoint{2.044352in}{1.622291in}}{\pgfqpoint{2.033753in}{1.617901in}}{\pgfqpoint{2.025939in}{1.610087in}}%
\pgfpathcurveto{\pgfqpoint{2.018125in}{1.602273in}}{\pgfqpoint{2.013735in}{1.591674in}}{\pgfqpoint{2.013735in}{1.580624in}}%
\pgfpathcurveto{\pgfqpoint{2.013735in}{1.569574in}}{\pgfqpoint{2.018125in}{1.558975in}}{\pgfqpoint{2.025939in}{1.551161in}}%
\pgfpathcurveto{\pgfqpoint{2.033753in}{1.543348in}}{\pgfqpoint{2.044352in}{1.538958in}}{\pgfqpoint{2.055402in}{1.538958in}}%
\pgfpathclose%
\pgfusepath{stroke,fill}%
\end{pgfscope}%
\begin{pgfscope}%
\pgfpathrectangle{\pgfqpoint{0.375000in}{0.330000in}}{\pgfqpoint{2.325000in}{2.310000in}}%
\pgfusepath{clip}%
\pgfsetbuttcap%
\pgfsetroundjoin%
\definecolor{currentfill}{rgb}{0.000000,0.000000,0.000000}%
\pgfsetfillcolor{currentfill}%
\pgfsetlinewidth{1.003750pt}%
\definecolor{currentstroke}{rgb}{0.000000,0.000000,0.000000}%
\pgfsetstrokecolor{currentstroke}%
\pgfsetdash{}{0pt}%
\pgfpathmoveto{\pgfqpoint{2.055402in}{1.434896in}}%
\pgfpathcurveto{\pgfqpoint{2.066452in}{1.434896in}}{\pgfqpoint{2.077051in}{1.439286in}}{\pgfqpoint{2.084865in}{1.447100in}}%
\pgfpathcurveto{\pgfqpoint{2.092678in}{1.454913in}}{\pgfqpoint{2.097069in}{1.465512in}}{\pgfqpoint{2.097069in}{1.476563in}}%
\pgfpathcurveto{\pgfqpoint{2.097069in}{1.487613in}}{\pgfqpoint{2.092678in}{1.498212in}}{\pgfqpoint{2.084865in}{1.506025in}}%
\pgfpathcurveto{\pgfqpoint{2.077051in}{1.513839in}}{\pgfqpoint{2.066452in}{1.518229in}}{\pgfqpoint{2.055402in}{1.518229in}}%
\pgfpathcurveto{\pgfqpoint{2.044352in}{1.518229in}}{\pgfqpoint{2.033753in}{1.513839in}}{\pgfqpoint{2.025939in}{1.506025in}}%
\pgfpathcurveto{\pgfqpoint{2.018125in}{1.498212in}}{\pgfqpoint{2.013735in}{1.487613in}}{\pgfqpoint{2.013735in}{1.476563in}}%
\pgfpathcurveto{\pgfqpoint{2.013735in}{1.465512in}}{\pgfqpoint{2.018125in}{1.454913in}}{\pgfqpoint{2.025939in}{1.447100in}}%
\pgfpathcurveto{\pgfqpoint{2.033753in}{1.439286in}}{\pgfqpoint{2.044352in}{1.434896in}}{\pgfqpoint{2.055402in}{1.434896in}}%
\pgfpathclose%
\pgfusepath{stroke,fill}%
\end{pgfscope}%
\begin{pgfscope}%
\pgfpathrectangle{\pgfqpoint{0.375000in}{0.330000in}}{\pgfqpoint{2.325000in}{2.310000in}}%
\pgfusepath{clip}%
\pgfsetbuttcap%
\pgfsetroundjoin%
\definecolor{currentfill}{rgb}{0.000000,0.000000,0.000000}%
\pgfsetfillcolor{currentfill}%
\pgfsetlinewidth{1.003750pt}%
\definecolor{currentstroke}{rgb}{0.000000,0.000000,0.000000}%
\pgfsetstrokecolor{currentstroke}%
\pgfsetdash{}{0pt}%
\pgfpathmoveto{\pgfqpoint{2.055402in}{1.486927in}}%
\pgfpathcurveto{\pgfqpoint{2.066452in}{1.486927in}}{\pgfqpoint{2.077051in}{1.491317in}}{\pgfqpoint{2.084865in}{1.499131in}}%
\pgfpathcurveto{\pgfqpoint{2.092678in}{1.506944in}}{\pgfqpoint{2.097069in}{1.517543in}}{\pgfqpoint{2.097069in}{1.528593in}}%
\pgfpathcurveto{\pgfqpoint{2.097069in}{1.539644in}}{\pgfqpoint{2.092678in}{1.550243in}}{\pgfqpoint{2.084865in}{1.558056in}}%
\pgfpathcurveto{\pgfqpoint{2.077051in}{1.565870in}}{\pgfqpoint{2.066452in}{1.570260in}}{\pgfqpoint{2.055402in}{1.570260in}}%
\pgfpathcurveto{\pgfqpoint{2.044352in}{1.570260in}}{\pgfqpoint{2.033753in}{1.565870in}}{\pgfqpoint{2.025939in}{1.558056in}}%
\pgfpathcurveto{\pgfqpoint{2.018125in}{1.550243in}}{\pgfqpoint{2.013735in}{1.539644in}}{\pgfqpoint{2.013735in}{1.528593in}}%
\pgfpathcurveto{\pgfqpoint{2.013735in}{1.517543in}}{\pgfqpoint{2.018125in}{1.506944in}}{\pgfqpoint{2.025939in}{1.499131in}}%
\pgfpathcurveto{\pgfqpoint{2.033753in}{1.491317in}}{\pgfqpoint{2.044352in}{1.486927in}}{\pgfqpoint{2.055402in}{1.486927in}}%
\pgfpathclose%
\pgfusepath{stroke,fill}%
\end{pgfscope}%
\begin{pgfscope}%
\pgfpathrectangle{\pgfqpoint{0.375000in}{0.330000in}}{\pgfqpoint{2.325000in}{2.310000in}}%
\pgfusepath{clip}%
\pgfsetbuttcap%
\pgfsetroundjoin%
\definecolor{currentfill}{rgb}{0.000000,0.000000,0.000000}%
\pgfsetfillcolor{currentfill}%
\pgfsetlinewidth{1.003750pt}%
\definecolor{currentstroke}{rgb}{0.000000,0.000000,0.000000}%
\pgfsetstrokecolor{currentstroke}%
\pgfsetdash{}{0pt}%
\pgfpathmoveto{\pgfqpoint{2.055402in}{1.590988in}}%
\pgfpathcurveto{\pgfqpoint{2.066452in}{1.590988in}}{\pgfqpoint{2.077051in}{1.595379in}}{\pgfqpoint{2.084865in}{1.603192in}}%
\pgfpathcurveto{\pgfqpoint{2.092678in}{1.611006in}}{\pgfqpoint{2.097069in}{1.621605in}}{\pgfqpoint{2.097069in}{1.632655in}}%
\pgfpathcurveto{\pgfqpoint{2.097069in}{1.643705in}}{\pgfqpoint{2.092678in}{1.654304in}}{\pgfqpoint{2.084865in}{1.662118in}}%
\pgfpathcurveto{\pgfqpoint{2.077051in}{1.669931in}}{\pgfqpoint{2.066452in}{1.674322in}}{\pgfqpoint{2.055402in}{1.674322in}}%
\pgfpathcurveto{\pgfqpoint{2.044352in}{1.674322in}}{\pgfqpoint{2.033753in}{1.669931in}}{\pgfqpoint{2.025939in}{1.662118in}}%
\pgfpathcurveto{\pgfqpoint{2.018125in}{1.654304in}}{\pgfqpoint{2.013735in}{1.643705in}}{\pgfqpoint{2.013735in}{1.632655in}}%
\pgfpathcurveto{\pgfqpoint{2.013735in}{1.621605in}}{\pgfqpoint{2.018125in}{1.611006in}}{\pgfqpoint{2.025939in}{1.603192in}}%
\pgfpathcurveto{\pgfqpoint{2.033753in}{1.595379in}}{\pgfqpoint{2.044352in}{1.590988in}}{\pgfqpoint{2.055402in}{1.590988in}}%
\pgfpathclose%
\pgfusepath{stroke,fill}%
\end{pgfscope}%
\begin{pgfscope}%
\pgfpathrectangle{\pgfqpoint{0.375000in}{0.330000in}}{\pgfqpoint{2.325000in}{2.310000in}}%
\pgfusepath{clip}%
\pgfsetbuttcap%
\pgfsetroundjoin%
\definecolor{currentfill}{rgb}{0.000000,0.000000,0.000000}%
\pgfsetfillcolor{currentfill}%
\pgfsetlinewidth{1.003750pt}%
\definecolor{currentstroke}{rgb}{0.000000,0.000000,0.000000}%
\pgfsetstrokecolor{currentstroke}%
\pgfsetdash{}{0pt}%
\pgfpathmoveto{\pgfqpoint{2.055402in}{1.434896in}}%
\pgfpathcurveto{\pgfqpoint{2.066452in}{1.434896in}}{\pgfqpoint{2.077051in}{1.439286in}}{\pgfqpoint{2.084865in}{1.447100in}}%
\pgfpathcurveto{\pgfqpoint{2.092678in}{1.454913in}}{\pgfqpoint{2.097069in}{1.465512in}}{\pgfqpoint{2.097069in}{1.476563in}}%
\pgfpathcurveto{\pgfqpoint{2.097069in}{1.487613in}}{\pgfqpoint{2.092678in}{1.498212in}}{\pgfqpoint{2.084865in}{1.506025in}}%
\pgfpathcurveto{\pgfqpoint{2.077051in}{1.513839in}}{\pgfqpoint{2.066452in}{1.518229in}}{\pgfqpoint{2.055402in}{1.518229in}}%
\pgfpathcurveto{\pgfqpoint{2.044352in}{1.518229in}}{\pgfqpoint{2.033753in}{1.513839in}}{\pgfqpoint{2.025939in}{1.506025in}}%
\pgfpathcurveto{\pgfqpoint{2.018125in}{1.498212in}}{\pgfqpoint{2.013735in}{1.487613in}}{\pgfqpoint{2.013735in}{1.476563in}}%
\pgfpathcurveto{\pgfqpoint{2.013735in}{1.465512in}}{\pgfqpoint{2.018125in}{1.454913in}}{\pgfqpoint{2.025939in}{1.447100in}}%
\pgfpathcurveto{\pgfqpoint{2.033753in}{1.439286in}}{\pgfqpoint{2.044352in}{1.434896in}}{\pgfqpoint{2.055402in}{1.434896in}}%
\pgfpathclose%
\pgfusepath{stroke,fill}%
\end{pgfscope}%
\begin{pgfscope}%
\pgfpathrectangle{\pgfqpoint{0.375000in}{0.330000in}}{\pgfqpoint{2.325000in}{2.310000in}}%
\pgfusepath{clip}%
\pgfsetbuttcap%
\pgfsetroundjoin%
\definecolor{currentfill}{rgb}{0.000000,0.000000,0.000000}%
\pgfsetfillcolor{currentfill}%
\pgfsetlinewidth{1.003750pt}%
\definecolor{currentstroke}{rgb}{0.000000,0.000000,0.000000}%
\pgfsetstrokecolor{currentstroke}%
\pgfsetdash{}{0pt}%
\pgfpathmoveto{\pgfqpoint{2.055402in}{1.538958in}}%
\pgfpathcurveto{\pgfqpoint{2.066452in}{1.538958in}}{\pgfqpoint{2.077051in}{1.543348in}}{\pgfqpoint{2.084865in}{1.551161in}}%
\pgfpathcurveto{\pgfqpoint{2.092678in}{1.558975in}}{\pgfqpoint{2.097069in}{1.569574in}}{\pgfqpoint{2.097069in}{1.580624in}}%
\pgfpathcurveto{\pgfqpoint{2.097069in}{1.591674in}}{\pgfqpoint{2.092678in}{1.602273in}}{\pgfqpoint{2.084865in}{1.610087in}}%
\pgfpathcurveto{\pgfqpoint{2.077051in}{1.617901in}}{\pgfqpoint{2.066452in}{1.622291in}}{\pgfqpoint{2.055402in}{1.622291in}}%
\pgfpathcurveto{\pgfqpoint{2.044352in}{1.622291in}}{\pgfqpoint{2.033753in}{1.617901in}}{\pgfqpoint{2.025939in}{1.610087in}}%
\pgfpathcurveto{\pgfqpoint{2.018125in}{1.602273in}}{\pgfqpoint{2.013735in}{1.591674in}}{\pgfqpoint{2.013735in}{1.580624in}}%
\pgfpathcurveto{\pgfqpoint{2.013735in}{1.569574in}}{\pgfqpoint{2.018125in}{1.558975in}}{\pgfqpoint{2.025939in}{1.551161in}}%
\pgfpathcurveto{\pgfqpoint{2.033753in}{1.543348in}}{\pgfqpoint{2.044352in}{1.538958in}}{\pgfqpoint{2.055402in}{1.538958in}}%
\pgfpathclose%
\pgfusepath{stroke,fill}%
\end{pgfscope}%
\begin{pgfscope}%
\pgfpathrectangle{\pgfqpoint{0.375000in}{0.330000in}}{\pgfqpoint{2.325000in}{2.310000in}}%
\pgfusepath{clip}%
\pgfsetbuttcap%
\pgfsetroundjoin%
\definecolor{currentfill}{rgb}{0.000000,0.000000,0.000000}%
\pgfsetfillcolor{currentfill}%
\pgfsetlinewidth{1.003750pt}%
\definecolor{currentstroke}{rgb}{0.000000,0.000000,0.000000}%
\pgfsetstrokecolor{currentstroke}%
\pgfsetdash{}{0pt}%
\pgfpathmoveto{\pgfqpoint{2.055402in}{1.486927in}}%
\pgfpathcurveto{\pgfqpoint{2.066452in}{1.486927in}}{\pgfqpoint{2.077051in}{1.491317in}}{\pgfqpoint{2.084865in}{1.499131in}}%
\pgfpathcurveto{\pgfqpoint{2.092678in}{1.506944in}}{\pgfqpoint{2.097069in}{1.517543in}}{\pgfqpoint{2.097069in}{1.528593in}}%
\pgfpathcurveto{\pgfqpoint{2.097069in}{1.539644in}}{\pgfqpoint{2.092678in}{1.550243in}}{\pgfqpoint{2.084865in}{1.558056in}}%
\pgfpathcurveto{\pgfqpoint{2.077051in}{1.565870in}}{\pgfqpoint{2.066452in}{1.570260in}}{\pgfqpoint{2.055402in}{1.570260in}}%
\pgfpathcurveto{\pgfqpoint{2.044352in}{1.570260in}}{\pgfqpoint{2.033753in}{1.565870in}}{\pgfqpoint{2.025939in}{1.558056in}}%
\pgfpathcurveto{\pgfqpoint{2.018125in}{1.550243in}}{\pgfqpoint{2.013735in}{1.539644in}}{\pgfqpoint{2.013735in}{1.528593in}}%
\pgfpathcurveto{\pgfqpoint{2.013735in}{1.517543in}}{\pgfqpoint{2.018125in}{1.506944in}}{\pgfqpoint{2.025939in}{1.499131in}}%
\pgfpathcurveto{\pgfqpoint{2.033753in}{1.491317in}}{\pgfqpoint{2.044352in}{1.486927in}}{\pgfqpoint{2.055402in}{1.486927in}}%
\pgfpathclose%
\pgfusepath{stroke,fill}%
\end{pgfscope}%
\begin{pgfscope}%
\pgfpathrectangle{\pgfqpoint{0.375000in}{0.330000in}}{\pgfqpoint{2.325000in}{2.310000in}}%
\pgfusepath{clip}%
\pgfsetbuttcap%
\pgfsetroundjoin%
\definecolor{currentfill}{rgb}{0.000000,0.000000,0.000000}%
\pgfsetfillcolor{currentfill}%
\pgfsetlinewidth{1.003750pt}%
\definecolor{currentstroke}{rgb}{0.000000,0.000000,0.000000}%
\pgfsetstrokecolor{currentstroke}%
\pgfsetdash{}{0pt}%
\pgfpathmoveto{\pgfqpoint{2.055402in}{1.486927in}}%
\pgfpathcurveto{\pgfqpoint{2.066452in}{1.486927in}}{\pgfqpoint{2.077051in}{1.491317in}}{\pgfqpoint{2.084865in}{1.499131in}}%
\pgfpathcurveto{\pgfqpoint{2.092678in}{1.506944in}}{\pgfqpoint{2.097069in}{1.517543in}}{\pgfqpoint{2.097069in}{1.528593in}}%
\pgfpathcurveto{\pgfqpoint{2.097069in}{1.539644in}}{\pgfqpoint{2.092678in}{1.550243in}}{\pgfqpoint{2.084865in}{1.558056in}}%
\pgfpathcurveto{\pgfqpoint{2.077051in}{1.565870in}}{\pgfqpoint{2.066452in}{1.570260in}}{\pgfqpoint{2.055402in}{1.570260in}}%
\pgfpathcurveto{\pgfqpoint{2.044352in}{1.570260in}}{\pgfqpoint{2.033753in}{1.565870in}}{\pgfqpoint{2.025939in}{1.558056in}}%
\pgfpathcurveto{\pgfqpoint{2.018125in}{1.550243in}}{\pgfqpoint{2.013735in}{1.539644in}}{\pgfqpoint{2.013735in}{1.528593in}}%
\pgfpathcurveto{\pgfqpoint{2.013735in}{1.517543in}}{\pgfqpoint{2.018125in}{1.506944in}}{\pgfqpoint{2.025939in}{1.499131in}}%
\pgfpathcurveto{\pgfqpoint{2.033753in}{1.491317in}}{\pgfqpoint{2.044352in}{1.486927in}}{\pgfqpoint{2.055402in}{1.486927in}}%
\pgfpathclose%
\pgfusepath{stroke,fill}%
\end{pgfscope}%
\begin{pgfscope}%
\pgfpathrectangle{\pgfqpoint{0.375000in}{0.330000in}}{\pgfqpoint{2.325000in}{2.310000in}}%
\pgfusepath{clip}%
\pgfsetbuttcap%
\pgfsetroundjoin%
\definecolor{currentfill}{rgb}{0.000000,0.000000,0.000000}%
\pgfsetfillcolor{currentfill}%
\pgfsetlinewidth{1.003750pt}%
\definecolor{currentstroke}{rgb}{0.000000,0.000000,0.000000}%
\pgfsetstrokecolor{currentstroke}%
\pgfsetdash{}{0pt}%
\pgfpathmoveto{\pgfqpoint{2.055402in}{1.486927in}}%
\pgfpathcurveto{\pgfqpoint{2.066452in}{1.486927in}}{\pgfqpoint{2.077051in}{1.491317in}}{\pgfqpoint{2.084865in}{1.499131in}}%
\pgfpathcurveto{\pgfqpoint{2.092678in}{1.506944in}}{\pgfqpoint{2.097069in}{1.517543in}}{\pgfqpoint{2.097069in}{1.528593in}}%
\pgfpathcurveto{\pgfqpoint{2.097069in}{1.539644in}}{\pgfqpoint{2.092678in}{1.550243in}}{\pgfqpoint{2.084865in}{1.558056in}}%
\pgfpathcurveto{\pgfqpoint{2.077051in}{1.565870in}}{\pgfqpoint{2.066452in}{1.570260in}}{\pgfqpoint{2.055402in}{1.570260in}}%
\pgfpathcurveto{\pgfqpoint{2.044352in}{1.570260in}}{\pgfqpoint{2.033753in}{1.565870in}}{\pgfqpoint{2.025939in}{1.558056in}}%
\pgfpathcurveto{\pgfqpoint{2.018125in}{1.550243in}}{\pgfqpoint{2.013735in}{1.539644in}}{\pgfqpoint{2.013735in}{1.528593in}}%
\pgfpathcurveto{\pgfqpoint{2.013735in}{1.517543in}}{\pgfqpoint{2.018125in}{1.506944in}}{\pgfqpoint{2.025939in}{1.499131in}}%
\pgfpathcurveto{\pgfqpoint{2.033753in}{1.491317in}}{\pgfqpoint{2.044352in}{1.486927in}}{\pgfqpoint{2.055402in}{1.486927in}}%
\pgfpathclose%
\pgfusepath{stroke,fill}%
\end{pgfscope}%
\begin{pgfscope}%
\pgfpathrectangle{\pgfqpoint{0.375000in}{0.330000in}}{\pgfqpoint{2.325000in}{2.310000in}}%
\pgfusepath{clip}%
\pgfsetbuttcap%
\pgfsetroundjoin%
\definecolor{currentfill}{rgb}{0.000000,0.000000,0.000000}%
\pgfsetfillcolor{currentfill}%
\pgfsetlinewidth{1.003750pt}%
\definecolor{currentstroke}{rgb}{0.000000,0.000000,0.000000}%
\pgfsetstrokecolor{currentstroke}%
\pgfsetdash{}{0pt}%
\pgfpathmoveto{\pgfqpoint{2.055402in}{1.486927in}}%
\pgfpathcurveto{\pgfqpoint{2.066452in}{1.486927in}}{\pgfqpoint{2.077051in}{1.491317in}}{\pgfqpoint{2.084865in}{1.499131in}}%
\pgfpathcurveto{\pgfqpoint{2.092678in}{1.506944in}}{\pgfqpoint{2.097069in}{1.517543in}}{\pgfqpoint{2.097069in}{1.528593in}}%
\pgfpathcurveto{\pgfqpoint{2.097069in}{1.539644in}}{\pgfqpoint{2.092678in}{1.550243in}}{\pgfqpoint{2.084865in}{1.558056in}}%
\pgfpathcurveto{\pgfqpoint{2.077051in}{1.565870in}}{\pgfqpoint{2.066452in}{1.570260in}}{\pgfqpoint{2.055402in}{1.570260in}}%
\pgfpathcurveto{\pgfqpoint{2.044352in}{1.570260in}}{\pgfqpoint{2.033753in}{1.565870in}}{\pgfqpoint{2.025939in}{1.558056in}}%
\pgfpathcurveto{\pgfqpoint{2.018125in}{1.550243in}}{\pgfqpoint{2.013735in}{1.539644in}}{\pgfqpoint{2.013735in}{1.528593in}}%
\pgfpathcurveto{\pgfqpoint{2.013735in}{1.517543in}}{\pgfqpoint{2.018125in}{1.506944in}}{\pgfqpoint{2.025939in}{1.499131in}}%
\pgfpathcurveto{\pgfqpoint{2.033753in}{1.491317in}}{\pgfqpoint{2.044352in}{1.486927in}}{\pgfqpoint{2.055402in}{1.486927in}}%
\pgfpathclose%
\pgfusepath{stroke,fill}%
\end{pgfscope}%
\begin{pgfscope}%
\pgfpathrectangle{\pgfqpoint{0.375000in}{0.330000in}}{\pgfqpoint{2.325000in}{2.310000in}}%
\pgfusepath{clip}%
\pgfsetbuttcap%
\pgfsetroundjoin%
\definecolor{currentfill}{rgb}{0.000000,0.000000,0.000000}%
\pgfsetfillcolor{currentfill}%
\pgfsetlinewidth{1.003750pt}%
\definecolor{currentstroke}{rgb}{0.000000,0.000000,0.000000}%
\pgfsetstrokecolor{currentstroke}%
\pgfsetdash{}{0pt}%
\pgfpathmoveto{\pgfqpoint{2.055402in}{1.434896in}}%
\pgfpathcurveto{\pgfqpoint{2.066452in}{1.434896in}}{\pgfqpoint{2.077051in}{1.439286in}}{\pgfqpoint{2.084865in}{1.447100in}}%
\pgfpathcurveto{\pgfqpoint{2.092678in}{1.454913in}}{\pgfqpoint{2.097069in}{1.465512in}}{\pgfqpoint{2.097069in}{1.476563in}}%
\pgfpathcurveto{\pgfqpoint{2.097069in}{1.487613in}}{\pgfqpoint{2.092678in}{1.498212in}}{\pgfqpoint{2.084865in}{1.506025in}}%
\pgfpathcurveto{\pgfqpoint{2.077051in}{1.513839in}}{\pgfqpoint{2.066452in}{1.518229in}}{\pgfqpoint{2.055402in}{1.518229in}}%
\pgfpathcurveto{\pgfqpoint{2.044352in}{1.518229in}}{\pgfqpoint{2.033753in}{1.513839in}}{\pgfqpoint{2.025939in}{1.506025in}}%
\pgfpathcurveto{\pgfqpoint{2.018125in}{1.498212in}}{\pgfqpoint{2.013735in}{1.487613in}}{\pgfqpoint{2.013735in}{1.476563in}}%
\pgfpathcurveto{\pgfqpoint{2.013735in}{1.465512in}}{\pgfqpoint{2.018125in}{1.454913in}}{\pgfqpoint{2.025939in}{1.447100in}}%
\pgfpathcurveto{\pgfqpoint{2.033753in}{1.439286in}}{\pgfqpoint{2.044352in}{1.434896in}}{\pgfqpoint{2.055402in}{1.434896in}}%
\pgfpathclose%
\pgfusepath{stroke,fill}%
\end{pgfscope}%
\begin{pgfscope}%
\pgfpathrectangle{\pgfqpoint{0.375000in}{0.330000in}}{\pgfqpoint{2.325000in}{2.310000in}}%
\pgfusepath{clip}%
\pgfsetbuttcap%
\pgfsetroundjoin%
\definecolor{currentfill}{rgb}{0.000000,0.000000,0.000000}%
\pgfsetfillcolor{currentfill}%
\pgfsetlinewidth{1.003750pt}%
\definecolor{currentstroke}{rgb}{0.000000,0.000000,0.000000}%
\pgfsetstrokecolor{currentstroke}%
\pgfsetdash{}{0pt}%
\pgfpathmoveto{\pgfqpoint{2.055402in}{1.486927in}}%
\pgfpathcurveto{\pgfqpoint{2.066452in}{1.486927in}}{\pgfqpoint{2.077051in}{1.491317in}}{\pgfqpoint{2.084865in}{1.499131in}}%
\pgfpathcurveto{\pgfqpoint{2.092678in}{1.506944in}}{\pgfqpoint{2.097069in}{1.517543in}}{\pgfqpoint{2.097069in}{1.528593in}}%
\pgfpathcurveto{\pgfqpoint{2.097069in}{1.539644in}}{\pgfqpoint{2.092678in}{1.550243in}}{\pgfqpoint{2.084865in}{1.558056in}}%
\pgfpathcurveto{\pgfqpoint{2.077051in}{1.565870in}}{\pgfqpoint{2.066452in}{1.570260in}}{\pgfqpoint{2.055402in}{1.570260in}}%
\pgfpathcurveto{\pgfqpoint{2.044352in}{1.570260in}}{\pgfqpoint{2.033753in}{1.565870in}}{\pgfqpoint{2.025939in}{1.558056in}}%
\pgfpathcurveto{\pgfqpoint{2.018125in}{1.550243in}}{\pgfqpoint{2.013735in}{1.539644in}}{\pgfqpoint{2.013735in}{1.528593in}}%
\pgfpathcurveto{\pgfqpoint{2.013735in}{1.517543in}}{\pgfqpoint{2.018125in}{1.506944in}}{\pgfqpoint{2.025939in}{1.499131in}}%
\pgfpathcurveto{\pgfqpoint{2.033753in}{1.491317in}}{\pgfqpoint{2.044352in}{1.486927in}}{\pgfqpoint{2.055402in}{1.486927in}}%
\pgfpathclose%
\pgfusepath{stroke,fill}%
\end{pgfscope}%
\begin{pgfscope}%
\pgfpathrectangle{\pgfqpoint{0.375000in}{0.330000in}}{\pgfqpoint{2.325000in}{2.310000in}}%
\pgfusepath{clip}%
\pgfsetbuttcap%
\pgfsetroundjoin%
\definecolor{currentfill}{rgb}{0.000000,0.000000,0.000000}%
\pgfsetfillcolor{currentfill}%
\pgfsetlinewidth{1.003750pt}%
\definecolor{currentstroke}{rgb}{0.000000,0.000000,0.000000}%
\pgfsetstrokecolor{currentstroke}%
\pgfsetdash{}{0pt}%
\pgfpathmoveto{\pgfqpoint{2.055402in}{1.538958in}}%
\pgfpathcurveto{\pgfqpoint{2.066452in}{1.538958in}}{\pgfqpoint{2.077051in}{1.543348in}}{\pgfqpoint{2.084865in}{1.551161in}}%
\pgfpathcurveto{\pgfqpoint{2.092678in}{1.558975in}}{\pgfqpoint{2.097069in}{1.569574in}}{\pgfqpoint{2.097069in}{1.580624in}}%
\pgfpathcurveto{\pgfqpoint{2.097069in}{1.591674in}}{\pgfqpoint{2.092678in}{1.602273in}}{\pgfqpoint{2.084865in}{1.610087in}}%
\pgfpathcurveto{\pgfqpoint{2.077051in}{1.617901in}}{\pgfqpoint{2.066452in}{1.622291in}}{\pgfqpoint{2.055402in}{1.622291in}}%
\pgfpathcurveto{\pgfqpoint{2.044352in}{1.622291in}}{\pgfqpoint{2.033753in}{1.617901in}}{\pgfqpoint{2.025939in}{1.610087in}}%
\pgfpathcurveto{\pgfqpoint{2.018125in}{1.602273in}}{\pgfqpoint{2.013735in}{1.591674in}}{\pgfqpoint{2.013735in}{1.580624in}}%
\pgfpathcurveto{\pgfqpoint{2.013735in}{1.569574in}}{\pgfqpoint{2.018125in}{1.558975in}}{\pgfqpoint{2.025939in}{1.551161in}}%
\pgfpathcurveto{\pgfqpoint{2.033753in}{1.543348in}}{\pgfqpoint{2.044352in}{1.538958in}}{\pgfqpoint{2.055402in}{1.538958in}}%
\pgfpathclose%
\pgfusepath{stroke,fill}%
\end{pgfscope}%
\begin{pgfscope}%
\pgfpathrectangle{\pgfqpoint{0.375000in}{0.330000in}}{\pgfqpoint{2.325000in}{2.310000in}}%
\pgfusepath{clip}%
\pgfsetbuttcap%
\pgfsetroundjoin%
\definecolor{currentfill}{rgb}{0.000000,0.000000,0.000000}%
\pgfsetfillcolor{currentfill}%
\pgfsetlinewidth{1.003750pt}%
\definecolor{currentstroke}{rgb}{0.000000,0.000000,0.000000}%
\pgfsetstrokecolor{currentstroke}%
\pgfsetdash{}{0pt}%
\pgfpathmoveto{\pgfqpoint{2.055402in}{1.434896in}}%
\pgfpathcurveto{\pgfqpoint{2.066452in}{1.434896in}}{\pgfqpoint{2.077051in}{1.439286in}}{\pgfqpoint{2.084865in}{1.447100in}}%
\pgfpathcurveto{\pgfqpoint{2.092678in}{1.454913in}}{\pgfqpoint{2.097069in}{1.465512in}}{\pgfqpoint{2.097069in}{1.476563in}}%
\pgfpathcurveto{\pgfqpoint{2.097069in}{1.487613in}}{\pgfqpoint{2.092678in}{1.498212in}}{\pgfqpoint{2.084865in}{1.506025in}}%
\pgfpathcurveto{\pgfqpoint{2.077051in}{1.513839in}}{\pgfqpoint{2.066452in}{1.518229in}}{\pgfqpoint{2.055402in}{1.518229in}}%
\pgfpathcurveto{\pgfqpoint{2.044352in}{1.518229in}}{\pgfqpoint{2.033753in}{1.513839in}}{\pgfqpoint{2.025939in}{1.506025in}}%
\pgfpathcurveto{\pgfqpoint{2.018125in}{1.498212in}}{\pgfqpoint{2.013735in}{1.487613in}}{\pgfqpoint{2.013735in}{1.476563in}}%
\pgfpathcurveto{\pgfqpoint{2.013735in}{1.465512in}}{\pgfqpoint{2.018125in}{1.454913in}}{\pgfqpoint{2.025939in}{1.447100in}}%
\pgfpathcurveto{\pgfqpoint{2.033753in}{1.439286in}}{\pgfqpoint{2.044352in}{1.434896in}}{\pgfqpoint{2.055402in}{1.434896in}}%
\pgfpathclose%
\pgfusepath{stroke,fill}%
\end{pgfscope}%
\begin{pgfscope}%
\pgfpathrectangle{\pgfqpoint{0.375000in}{0.330000in}}{\pgfqpoint{2.325000in}{2.310000in}}%
\pgfusepath{clip}%
\pgfsetbuttcap%
\pgfsetroundjoin%
\definecolor{currentfill}{rgb}{0.000000,0.000000,0.000000}%
\pgfsetfillcolor{currentfill}%
\pgfsetlinewidth{1.003750pt}%
\definecolor{currentstroke}{rgb}{0.000000,0.000000,0.000000}%
\pgfsetstrokecolor{currentstroke}%
\pgfsetdash{}{0pt}%
\pgfpathmoveto{\pgfqpoint{2.055402in}{1.486927in}}%
\pgfpathcurveto{\pgfqpoint{2.066452in}{1.486927in}}{\pgfqpoint{2.077051in}{1.491317in}}{\pgfqpoint{2.084865in}{1.499131in}}%
\pgfpathcurveto{\pgfqpoint{2.092678in}{1.506944in}}{\pgfqpoint{2.097069in}{1.517543in}}{\pgfqpoint{2.097069in}{1.528593in}}%
\pgfpathcurveto{\pgfqpoint{2.097069in}{1.539644in}}{\pgfqpoint{2.092678in}{1.550243in}}{\pgfqpoint{2.084865in}{1.558056in}}%
\pgfpathcurveto{\pgfqpoint{2.077051in}{1.565870in}}{\pgfqpoint{2.066452in}{1.570260in}}{\pgfqpoint{2.055402in}{1.570260in}}%
\pgfpathcurveto{\pgfqpoint{2.044352in}{1.570260in}}{\pgfqpoint{2.033753in}{1.565870in}}{\pgfqpoint{2.025939in}{1.558056in}}%
\pgfpathcurveto{\pgfqpoint{2.018125in}{1.550243in}}{\pgfqpoint{2.013735in}{1.539644in}}{\pgfqpoint{2.013735in}{1.528593in}}%
\pgfpathcurveto{\pgfqpoint{2.013735in}{1.517543in}}{\pgfqpoint{2.018125in}{1.506944in}}{\pgfqpoint{2.025939in}{1.499131in}}%
\pgfpathcurveto{\pgfqpoint{2.033753in}{1.491317in}}{\pgfqpoint{2.044352in}{1.486927in}}{\pgfqpoint{2.055402in}{1.486927in}}%
\pgfpathclose%
\pgfusepath{stroke,fill}%
\end{pgfscope}%
\begin{pgfscope}%
\pgfpathrectangle{\pgfqpoint{0.375000in}{0.330000in}}{\pgfqpoint{2.325000in}{2.310000in}}%
\pgfusepath{clip}%
\pgfsetbuttcap%
\pgfsetroundjoin%
\definecolor{currentfill}{rgb}{0.000000,0.000000,0.000000}%
\pgfsetfillcolor{currentfill}%
\pgfsetlinewidth{1.003750pt}%
\definecolor{currentstroke}{rgb}{0.000000,0.000000,0.000000}%
\pgfsetstrokecolor{currentstroke}%
\pgfsetdash{}{0pt}%
\pgfpathmoveto{\pgfqpoint{2.055402in}{1.486927in}}%
\pgfpathcurveto{\pgfqpoint{2.066452in}{1.486927in}}{\pgfqpoint{2.077051in}{1.491317in}}{\pgfqpoint{2.084865in}{1.499131in}}%
\pgfpathcurveto{\pgfqpoint{2.092678in}{1.506944in}}{\pgfqpoint{2.097069in}{1.517543in}}{\pgfqpoint{2.097069in}{1.528593in}}%
\pgfpathcurveto{\pgfqpoint{2.097069in}{1.539644in}}{\pgfqpoint{2.092678in}{1.550243in}}{\pgfqpoint{2.084865in}{1.558056in}}%
\pgfpathcurveto{\pgfqpoint{2.077051in}{1.565870in}}{\pgfqpoint{2.066452in}{1.570260in}}{\pgfqpoint{2.055402in}{1.570260in}}%
\pgfpathcurveto{\pgfqpoint{2.044352in}{1.570260in}}{\pgfqpoint{2.033753in}{1.565870in}}{\pgfqpoint{2.025939in}{1.558056in}}%
\pgfpathcurveto{\pgfqpoint{2.018125in}{1.550243in}}{\pgfqpoint{2.013735in}{1.539644in}}{\pgfqpoint{2.013735in}{1.528593in}}%
\pgfpathcurveto{\pgfqpoint{2.013735in}{1.517543in}}{\pgfqpoint{2.018125in}{1.506944in}}{\pgfqpoint{2.025939in}{1.499131in}}%
\pgfpathcurveto{\pgfqpoint{2.033753in}{1.491317in}}{\pgfqpoint{2.044352in}{1.486927in}}{\pgfqpoint{2.055402in}{1.486927in}}%
\pgfpathclose%
\pgfusepath{stroke,fill}%
\end{pgfscope}%
\begin{pgfscope}%
\pgfpathrectangle{\pgfqpoint{0.375000in}{0.330000in}}{\pgfqpoint{2.325000in}{2.310000in}}%
\pgfusepath{clip}%
\pgfsetbuttcap%
\pgfsetroundjoin%
\definecolor{currentfill}{rgb}{0.000000,0.000000,0.000000}%
\pgfsetfillcolor{currentfill}%
\pgfsetlinewidth{1.003750pt}%
\definecolor{currentstroke}{rgb}{0.000000,0.000000,0.000000}%
\pgfsetstrokecolor{currentstroke}%
\pgfsetdash{}{0pt}%
\pgfpathmoveto{\pgfqpoint{2.055402in}{1.486927in}}%
\pgfpathcurveto{\pgfqpoint{2.066452in}{1.486927in}}{\pgfqpoint{2.077051in}{1.491317in}}{\pgfqpoint{2.084865in}{1.499131in}}%
\pgfpathcurveto{\pgfqpoint{2.092678in}{1.506944in}}{\pgfqpoint{2.097069in}{1.517543in}}{\pgfqpoint{2.097069in}{1.528593in}}%
\pgfpathcurveto{\pgfqpoint{2.097069in}{1.539644in}}{\pgfqpoint{2.092678in}{1.550243in}}{\pgfqpoint{2.084865in}{1.558056in}}%
\pgfpathcurveto{\pgfqpoint{2.077051in}{1.565870in}}{\pgfqpoint{2.066452in}{1.570260in}}{\pgfqpoint{2.055402in}{1.570260in}}%
\pgfpathcurveto{\pgfqpoint{2.044352in}{1.570260in}}{\pgfqpoint{2.033753in}{1.565870in}}{\pgfqpoint{2.025939in}{1.558056in}}%
\pgfpathcurveto{\pgfqpoint{2.018125in}{1.550243in}}{\pgfqpoint{2.013735in}{1.539644in}}{\pgfqpoint{2.013735in}{1.528593in}}%
\pgfpathcurveto{\pgfqpoint{2.013735in}{1.517543in}}{\pgfqpoint{2.018125in}{1.506944in}}{\pgfqpoint{2.025939in}{1.499131in}}%
\pgfpathcurveto{\pgfqpoint{2.033753in}{1.491317in}}{\pgfqpoint{2.044352in}{1.486927in}}{\pgfqpoint{2.055402in}{1.486927in}}%
\pgfpathclose%
\pgfusepath{stroke,fill}%
\end{pgfscope}%
\begin{pgfscope}%
\pgfpathrectangle{\pgfqpoint{0.375000in}{0.330000in}}{\pgfqpoint{2.325000in}{2.310000in}}%
\pgfusepath{clip}%
\pgfsetbuttcap%
\pgfsetroundjoin%
\definecolor{currentfill}{rgb}{0.000000,0.000000,0.000000}%
\pgfsetfillcolor{currentfill}%
\pgfsetlinewidth{1.003750pt}%
\definecolor{currentstroke}{rgb}{0.000000,0.000000,0.000000}%
\pgfsetstrokecolor{currentstroke}%
\pgfsetdash{}{0pt}%
\pgfpathmoveto{\pgfqpoint{2.055402in}{1.486927in}}%
\pgfpathcurveto{\pgfqpoint{2.066452in}{1.486927in}}{\pgfqpoint{2.077051in}{1.491317in}}{\pgfqpoint{2.084865in}{1.499131in}}%
\pgfpathcurveto{\pgfqpoint{2.092678in}{1.506944in}}{\pgfqpoint{2.097069in}{1.517543in}}{\pgfqpoint{2.097069in}{1.528593in}}%
\pgfpathcurveto{\pgfqpoint{2.097069in}{1.539644in}}{\pgfqpoint{2.092678in}{1.550243in}}{\pgfqpoint{2.084865in}{1.558056in}}%
\pgfpathcurveto{\pgfqpoint{2.077051in}{1.565870in}}{\pgfqpoint{2.066452in}{1.570260in}}{\pgfqpoint{2.055402in}{1.570260in}}%
\pgfpathcurveto{\pgfqpoint{2.044352in}{1.570260in}}{\pgfqpoint{2.033753in}{1.565870in}}{\pgfqpoint{2.025939in}{1.558056in}}%
\pgfpathcurveto{\pgfqpoint{2.018125in}{1.550243in}}{\pgfqpoint{2.013735in}{1.539644in}}{\pgfqpoint{2.013735in}{1.528593in}}%
\pgfpathcurveto{\pgfqpoint{2.013735in}{1.517543in}}{\pgfqpoint{2.018125in}{1.506944in}}{\pgfqpoint{2.025939in}{1.499131in}}%
\pgfpathcurveto{\pgfqpoint{2.033753in}{1.491317in}}{\pgfqpoint{2.044352in}{1.486927in}}{\pgfqpoint{2.055402in}{1.486927in}}%
\pgfpathclose%
\pgfusepath{stroke,fill}%
\end{pgfscope}%
\begin{pgfscope}%
\pgfpathrectangle{\pgfqpoint{0.375000in}{0.330000in}}{\pgfqpoint{2.325000in}{2.310000in}}%
\pgfusepath{clip}%
\pgfsetbuttcap%
\pgfsetroundjoin%
\definecolor{currentfill}{rgb}{0.000000,0.000000,0.000000}%
\pgfsetfillcolor{currentfill}%
\pgfsetlinewidth{1.003750pt}%
\definecolor{currentstroke}{rgb}{0.000000,0.000000,0.000000}%
\pgfsetstrokecolor{currentstroke}%
\pgfsetdash{}{0pt}%
\pgfpathmoveto{\pgfqpoint{2.055402in}{1.434896in}}%
\pgfpathcurveto{\pgfqpoint{2.066452in}{1.434896in}}{\pgfqpoint{2.077051in}{1.439286in}}{\pgfqpoint{2.084865in}{1.447100in}}%
\pgfpathcurveto{\pgfqpoint{2.092678in}{1.454913in}}{\pgfqpoint{2.097069in}{1.465512in}}{\pgfqpoint{2.097069in}{1.476563in}}%
\pgfpathcurveto{\pgfqpoint{2.097069in}{1.487613in}}{\pgfqpoint{2.092678in}{1.498212in}}{\pgfqpoint{2.084865in}{1.506025in}}%
\pgfpathcurveto{\pgfqpoint{2.077051in}{1.513839in}}{\pgfqpoint{2.066452in}{1.518229in}}{\pgfqpoint{2.055402in}{1.518229in}}%
\pgfpathcurveto{\pgfqpoint{2.044352in}{1.518229in}}{\pgfqpoint{2.033753in}{1.513839in}}{\pgfqpoint{2.025939in}{1.506025in}}%
\pgfpathcurveto{\pgfqpoint{2.018125in}{1.498212in}}{\pgfqpoint{2.013735in}{1.487613in}}{\pgfqpoint{2.013735in}{1.476563in}}%
\pgfpathcurveto{\pgfqpoint{2.013735in}{1.465512in}}{\pgfqpoint{2.018125in}{1.454913in}}{\pgfqpoint{2.025939in}{1.447100in}}%
\pgfpathcurveto{\pgfqpoint{2.033753in}{1.439286in}}{\pgfqpoint{2.044352in}{1.434896in}}{\pgfqpoint{2.055402in}{1.434896in}}%
\pgfpathclose%
\pgfusepath{stroke,fill}%
\end{pgfscope}%
\begin{pgfscope}%
\pgfpathrectangle{\pgfqpoint{0.375000in}{0.330000in}}{\pgfqpoint{2.325000in}{2.310000in}}%
\pgfusepath{clip}%
\pgfsetbuttcap%
\pgfsetroundjoin%
\definecolor{currentfill}{rgb}{0.000000,0.000000,0.000000}%
\pgfsetfillcolor{currentfill}%
\pgfsetlinewidth{1.003750pt}%
\definecolor{currentstroke}{rgb}{0.000000,0.000000,0.000000}%
\pgfsetstrokecolor{currentstroke}%
\pgfsetdash{}{0pt}%
\pgfpathmoveto{\pgfqpoint{2.055402in}{1.538958in}}%
\pgfpathcurveto{\pgfqpoint{2.066452in}{1.538958in}}{\pgfqpoint{2.077051in}{1.543348in}}{\pgfqpoint{2.084865in}{1.551161in}}%
\pgfpathcurveto{\pgfqpoint{2.092678in}{1.558975in}}{\pgfqpoint{2.097069in}{1.569574in}}{\pgfqpoint{2.097069in}{1.580624in}}%
\pgfpathcurveto{\pgfqpoint{2.097069in}{1.591674in}}{\pgfqpoint{2.092678in}{1.602273in}}{\pgfqpoint{2.084865in}{1.610087in}}%
\pgfpathcurveto{\pgfqpoint{2.077051in}{1.617901in}}{\pgfqpoint{2.066452in}{1.622291in}}{\pgfqpoint{2.055402in}{1.622291in}}%
\pgfpathcurveto{\pgfqpoint{2.044352in}{1.622291in}}{\pgfqpoint{2.033753in}{1.617901in}}{\pgfqpoint{2.025939in}{1.610087in}}%
\pgfpathcurveto{\pgfqpoint{2.018125in}{1.602273in}}{\pgfqpoint{2.013735in}{1.591674in}}{\pgfqpoint{2.013735in}{1.580624in}}%
\pgfpathcurveto{\pgfqpoint{2.013735in}{1.569574in}}{\pgfqpoint{2.018125in}{1.558975in}}{\pgfqpoint{2.025939in}{1.551161in}}%
\pgfpathcurveto{\pgfqpoint{2.033753in}{1.543348in}}{\pgfqpoint{2.044352in}{1.538958in}}{\pgfqpoint{2.055402in}{1.538958in}}%
\pgfpathclose%
\pgfusepath{stroke,fill}%
\end{pgfscope}%
\begin{pgfscope}%
\pgfpathrectangle{\pgfqpoint{0.375000in}{0.330000in}}{\pgfqpoint{2.325000in}{2.310000in}}%
\pgfusepath{clip}%
\pgfsetbuttcap%
\pgfsetroundjoin%
\definecolor{currentfill}{rgb}{0.000000,0.000000,0.000000}%
\pgfsetfillcolor{currentfill}%
\pgfsetlinewidth{1.003750pt}%
\definecolor{currentstroke}{rgb}{0.000000,0.000000,0.000000}%
\pgfsetstrokecolor{currentstroke}%
\pgfsetdash{}{0pt}%
\pgfpathmoveto{\pgfqpoint{2.055402in}{1.486927in}}%
\pgfpathcurveto{\pgfqpoint{2.066452in}{1.486927in}}{\pgfqpoint{2.077051in}{1.491317in}}{\pgfqpoint{2.084865in}{1.499131in}}%
\pgfpathcurveto{\pgfqpoint{2.092678in}{1.506944in}}{\pgfqpoint{2.097069in}{1.517543in}}{\pgfqpoint{2.097069in}{1.528593in}}%
\pgfpathcurveto{\pgfqpoint{2.097069in}{1.539644in}}{\pgfqpoint{2.092678in}{1.550243in}}{\pgfqpoint{2.084865in}{1.558056in}}%
\pgfpathcurveto{\pgfqpoint{2.077051in}{1.565870in}}{\pgfqpoint{2.066452in}{1.570260in}}{\pgfqpoint{2.055402in}{1.570260in}}%
\pgfpathcurveto{\pgfqpoint{2.044352in}{1.570260in}}{\pgfqpoint{2.033753in}{1.565870in}}{\pgfqpoint{2.025939in}{1.558056in}}%
\pgfpathcurveto{\pgfqpoint{2.018125in}{1.550243in}}{\pgfqpoint{2.013735in}{1.539644in}}{\pgfqpoint{2.013735in}{1.528593in}}%
\pgfpathcurveto{\pgfqpoint{2.013735in}{1.517543in}}{\pgfqpoint{2.018125in}{1.506944in}}{\pgfqpoint{2.025939in}{1.499131in}}%
\pgfpathcurveto{\pgfqpoint{2.033753in}{1.491317in}}{\pgfqpoint{2.044352in}{1.486927in}}{\pgfqpoint{2.055402in}{1.486927in}}%
\pgfpathclose%
\pgfusepath{stroke,fill}%
\end{pgfscope}%
\begin{pgfscope}%
\pgfpathrectangle{\pgfqpoint{0.375000in}{0.330000in}}{\pgfqpoint{2.325000in}{2.310000in}}%
\pgfusepath{clip}%
\pgfsetbuttcap%
\pgfsetroundjoin%
\definecolor{currentfill}{rgb}{0.000000,0.000000,0.000000}%
\pgfsetfillcolor{currentfill}%
\pgfsetlinewidth{1.003750pt}%
\definecolor{currentstroke}{rgb}{0.000000,0.000000,0.000000}%
\pgfsetstrokecolor{currentstroke}%
\pgfsetdash{}{0pt}%
\pgfpathmoveto{\pgfqpoint{2.055402in}{1.434896in}}%
\pgfpathcurveto{\pgfqpoint{2.066452in}{1.434896in}}{\pgfqpoint{2.077051in}{1.439286in}}{\pgfqpoint{2.084865in}{1.447100in}}%
\pgfpathcurveto{\pgfqpoint{2.092678in}{1.454913in}}{\pgfqpoint{2.097069in}{1.465512in}}{\pgfqpoint{2.097069in}{1.476563in}}%
\pgfpathcurveto{\pgfqpoint{2.097069in}{1.487613in}}{\pgfqpoint{2.092678in}{1.498212in}}{\pgfqpoint{2.084865in}{1.506025in}}%
\pgfpathcurveto{\pgfqpoint{2.077051in}{1.513839in}}{\pgfqpoint{2.066452in}{1.518229in}}{\pgfqpoint{2.055402in}{1.518229in}}%
\pgfpathcurveto{\pgfqpoint{2.044352in}{1.518229in}}{\pgfqpoint{2.033753in}{1.513839in}}{\pgfqpoint{2.025939in}{1.506025in}}%
\pgfpathcurveto{\pgfqpoint{2.018125in}{1.498212in}}{\pgfqpoint{2.013735in}{1.487613in}}{\pgfqpoint{2.013735in}{1.476563in}}%
\pgfpathcurveto{\pgfqpoint{2.013735in}{1.465512in}}{\pgfqpoint{2.018125in}{1.454913in}}{\pgfqpoint{2.025939in}{1.447100in}}%
\pgfpathcurveto{\pgfqpoint{2.033753in}{1.439286in}}{\pgfqpoint{2.044352in}{1.434896in}}{\pgfqpoint{2.055402in}{1.434896in}}%
\pgfpathclose%
\pgfusepath{stroke,fill}%
\end{pgfscope}%
\begin{pgfscope}%
\pgfpathrectangle{\pgfqpoint{0.375000in}{0.330000in}}{\pgfqpoint{2.325000in}{2.310000in}}%
\pgfusepath{clip}%
\pgfsetbuttcap%
\pgfsetroundjoin%
\definecolor{currentfill}{rgb}{0.000000,0.000000,0.000000}%
\pgfsetfillcolor{currentfill}%
\pgfsetlinewidth{1.003750pt}%
\definecolor{currentstroke}{rgb}{0.000000,0.000000,0.000000}%
\pgfsetstrokecolor{currentstroke}%
\pgfsetdash{}{0pt}%
\pgfpathmoveto{\pgfqpoint{2.055402in}{1.538958in}}%
\pgfpathcurveto{\pgfqpoint{2.066452in}{1.538958in}}{\pgfqpoint{2.077051in}{1.543348in}}{\pgfqpoint{2.084865in}{1.551161in}}%
\pgfpathcurveto{\pgfqpoint{2.092678in}{1.558975in}}{\pgfqpoint{2.097069in}{1.569574in}}{\pgfqpoint{2.097069in}{1.580624in}}%
\pgfpathcurveto{\pgfqpoint{2.097069in}{1.591674in}}{\pgfqpoint{2.092678in}{1.602273in}}{\pgfqpoint{2.084865in}{1.610087in}}%
\pgfpathcurveto{\pgfqpoint{2.077051in}{1.617901in}}{\pgfqpoint{2.066452in}{1.622291in}}{\pgfqpoint{2.055402in}{1.622291in}}%
\pgfpathcurveto{\pgfqpoint{2.044352in}{1.622291in}}{\pgfqpoint{2.033753in}{1.617901in}}{\pgfqpoint{2.025939in}{1.610087in}}%
\pgfpathcurveto{\pgfqpoint{2.018125in}{1.602273in}}{\pgfqpoint{2.013735in}{1.591674in}}{\pgfqpoint{2.013735in}{1.580624in}}%
\pgfpathcurveto{\pgfqpoint{2.013735in}{1.569574in}}{\pgfqpoint{2.018125in}{1.558975in}}{\pgfqpoint{2.025939in}{1.551161in}}%
\pgfpathcurveto{\pgfqpoint{2.033753in}{1.543348in}}{\pgfqpoint{2.044352in}{1.538958in}}{\pgfqpoint{2.055402in}{1.538958in}}%
\pgfpathclose%
\pgfusepath{stroke,fill}%
\end{pgfscope}%
\begin{pgfscope}%
\pgfpathrectangle{\pgfqpoint{0.375000in}{0.330000in}}{\pgfqpoint{2.325000in}{2.310000in}}%
\pgfusepath{clip}%
\pgfsetbuttcap%
\pgfsetroundjoin%
\definecolor{currentfill}{rgb}{0.000000,0.000000,0.000000}%
\pgfsetfillcolor{currentfill}%
\pgfsetlinewidth{1.003750pt}%
\definecolor{currentstroke}{rgb}{0.000000,0.000000,0.000000}%
\pgfsetstrokecolor{currentstroke}%
\pgfsetdash{}{0pt}%
\pgfpathmoveto{\pgfqpoint{2.055402in}{1.538958in}}%
\pgfpathcurveto{\pgfqpoint{2.066452in}{1.538958in}}{\pgfqpoint{2.077051in}{1.543348in}}{\pgfqpoint{2.084865in}{1.551161in}}%
\pgfpathcurveto{\pgfqpoint{2.092678in}{1.558975in}}{\pgfqpoint{2.097069in}{1.569574in}}{\pgfqpoint{2.097069in}{1.580624in}}%
\pgfpathcurveto{\pgfqpoint{2.097069in}{1.591674in}}{\pgfqpoint{2.092678in}{1.602273in}}{\pgfqpoint{2.084865in}{1.610087in}}%
\pgfpathcurveto{\pgfqpoint{2.077051in}{1.617901in}}{\pgfqpoint{2.066452in}{1.622291in}}{\pgfqpoint{2.055402in}{1.622291in}}%
\pgfpathcurveto{\pgfqpoint{2.044352in}{1.622291in}}{\pgfqpoint{2.033753in}{1.617901in}}{\pgfqpoint{2.025939in}{1.610087in}}%
\pgfpathcurveto{\pgfqpoint{2.018125in}{1.602273in}}{\pgfqpoint{2.013735in}{1.591674in}}{\pgfqpoint{2.013735in}{1.580624in}}%
\pgfpathcurveto{\pgfqpoint{2.013735in}{1.569574in}}{\pgfqpoint{2.018125in}{1.558975in}}{\pgfqpoint{2.025939in}{1.551161in}}%
\pgfpathcurveto{\pgfqpoint{2.033753in}{1.543348in}}{\pgfqpoint{2.044352in}{1.538958in}}{\pgfqpoint{2.055402in}{1.538958in}}%
\pgfpathclose%
\pgfusepath{stroke,fill}%
\end{pgfscope}%
\begin{pgfscope}%
\pgfpathrectangle{\pgfqpoint{0.375000in}{0.330000in}}{\pgfqpoint{2.325000in}{2.310000in}}%
\pgfusepath{clip}%
\pgfsetbuttcap%
\pgfsetroundjoin%
\definecolor{currentfill}{rgb}{0.000000,0.000000,0.000000}%
\pgfsetfillcolor{currentfill}%
\pgfsetlinewidth{1.003750pt}%
\definecolor{currentstroke}{rgb}{0.000000,0.000000,0.000000}%
\pgfsetstrokecolor{currentstroke}%
\pgfsetdash{}{0pt}%
\pgfpathmoveto{\pgfqpoint{2.055402in}{1.590988in}}%
\pgfpathcurveto{\pgfqpoint{2.066452in}{1.590988in}}{\pgfqpoint{2.077051in}{1.595379in}}{\pgfqpoint{2.084865in}{1.603192in}}%
\pgfpathcurveto{\pgfqpoint{2.092678in}{1.611006in}}{\pgfqpoint{2.097069in}{1.621605in}}{\pgfqpoint{2.097069in}{1.632655in}}%
\pgfpathcurveto{\pgfqpoint{2.097069in}{1.643705in}}{\pgfqpoint{2.092678in}{1.654304in}}{\pgfqpoint{2.084865in}{1.662118in}}%
\pgfpathcurveto{\pgfqpoint{2.077051in}{1.669931in}}{\pgfqpoint{2.066452in}{1.674322in}}{\pgfqpoint{2.055402in}{1.674322in}}%
\pgfpathcurveto{\pgfqpoint{2.044352in}{1.674322in}}{\pgfqpoint{2.033753in}{1.669931in}}{\pgfqpoint{2.025939in}{1.662118in}}%
\pgfpathcurveto{\pgfqpoint{2.018125in}{1.654304in}}{\pgfqpoint{2.013735in}{1.643705in}}{\pgfqpoint{2.013735in}{1.632655in}}%
\pgfpathcurveto{\pgfqpoint{2.013735in}{1.621605in}}{\pgfqpoint{2.018125in}{1.611006in}}{\pgfqpoint{2.025939in}{1.603192in}}%
\pgfpathcurveto{\pgfqpoint{2.033753in}{1.595379in}}{\pgfqpoint{2.044352in}{1.590988in}}{\pgfqpoint{2.055402in}{1.590988in}}%
\pgfpathclose%
\pgfusepath{stroke,fill}%
\end{pgfscope}%
\begin{pgfscope}%
\pgfpathrectangle{\pgfqpoint{0.375000in}{0.330000in}}{\pgfqpoint{2.325000in}{2.310000in}}%
\pgfusepath{clip}%
\pgfsetbuttcap%
\pgfsetroundjoin%
\definecolor{currentfill}{rgb}{0.000000,0.000000,0.000000}%
\pgfsetfillcolor{currentfill}%
\pgfsetlinewidth{1.003750pt}%
\definecolor{currentstroke}{rgb}{0.000000,0.000000,0.000000}%
\pgfsetstrokecolor{currentstroke}%
\pgfsetdash{}{0pt}%
\pgfpathmoveto{\pgfqpoint{2.055402in}{1.486927in}}%
\pgfpathcurveto{\pgfqpoint{2.066452in}{1.486927in}}{\pgfqpoint{2.077051in}{1.491317in}}{\pgfqpoint{2.084865in}{1.499131in}}%
\pgfpathcurveto{\pgfqpoint{2.092678in}{1.506944in}}{\pgfqpoint{2.097069in}{1.517543in}}{\pgfqpoint{2.097069in}{1.528593in}}%
\pgfpathcurveto{\pgfqpoint{2.097069in}{1.539644in}}{\pgfqpoint{2.092678in}{1.550243in}}{\pgfqpoint{2.084865in}{1.558056in}}%
\pgfpathcurveto{\pgfqpoint{2.077051in}{1.565870in}}{\pgfqpoint{2.066452in}{1.570260in}}{\pgfqpoint{2.055402in}{1.570260in}}%
\pgfpathcurveto{\pgfqpoint{2.044352in}{1.570260in}}{\pgfqpoint{2.033753in}{1.565870in}}{\pgfqpoint{2.025939in}{1.558056in}}%
\pgfpathcurveto{\pgfqpoint{2.018125in}{1.550243in}}{\pgfqpoint{2.013735in}{1.539644in}}{\pgfqpoint{2.013735in}{1.528593in}}%
\pgfpathcurveto{\pgfqpoint{2.013735in}{1.517543in}}{\pgfqpoint{2.018125in}{1.506944in}}{\pgfqpoint{2.025939in}{1.499131in}}%
\pgfpathcurveto{\pgfqpoint{2.033753in}{1.491317in}}{\pgfqpoint{2.044352in}{1.486927in}}{\pgfqpoint{2.055402in}{1.486927in}}%
\pgfpathclose%
\pgfusepath{stroke,fill}%
\end{pgfscope}%
\begin{pgfscope}%
\pgfpathrectangle{\pgfqpoint{0.375000in}{0.330000in}}{\pgfqpoint{2.325000in}{2.310000in}}%
\pgfusepath{clip}%
\pgfsetbuttcap%
\pgfsetroundjoin%
\definecolor{currentfill}{rgb}{0.000000,0.000000,0.000000}%
\pgfsetfillcolor{currentfill}%
\pgfsetlinewidth{1.003750pt}%
\definecolor{currentstroke}{rgb}{0.000000,0.000000,0.000000}%
\pgfsetstrokecolor{currentstroke}%
\pgfsetdash{}{0pt}%
\pgfpathmoveto{\pgfqpoint{2.055402in}{1.434896in}}%
\pgfpathcurveto{\pgfqpoint{2.066452in}{1.434896in}}{\pgfqpoint{2.077051in}{1.439286in}}{\pgfqpoint{2.084865in}{1.447100in}}%
\pgfpathcurveto{\pgfqpoint{2.092678in}{1.454913in}}{\pgfqpoint{2.097069in}{1.465512in}}{\pgfqpoint{2.097069in}{1.476563in}}%
\pgfpathcurveto{\pgfqpoint{2.097069in}{1.487613in}}{\pgfqpoint{2.092678in}{1.498212in}}{\pgfqpoint{2.084865in}{1.506025in}}%
\pgfpathcurveto{\pgfqpoint{2.077051in}{1.513839in}}{\pgfqpoint{2.066452in}{1.518229in}}{\pgfqpoint{2.055402in}{1.518229in}}%
\pgfpathcurveto{\pgfqpoint{2.044352in}{1.518229in}}{\pgfqpoint{2.033753in}{1.513839in}}{\pgfqpoint{2.025939in}{1.506025in}}%
\pgfpathcurveto{\pgfqpoint{2.018125in}{1.498212in}}{\pgfqpoint{2.013735in}{1.487613in}}{\pgfqpoint{2.013735in}{1.476563in}}%
\pgfpathcurveto{\pgfqpoint{2.013735in}{1.465512in}}{\pgfqpoint{2.018125in}{1.454913in}}{\pgfqpoint{2.025939in}{1.447100in}}%
\pgfpathcurveto{\pgfqpoint{2.033753in}{1.439286in}}{\pgfqpoint{2.044352in}{1.434896in}}{\pgfqpoint{2.055402in}{1.434896in}}%
\pgfpathclose%
\pgfusepath{stroke,fill}%
\end{pgfscope}%
\begin{pgfscope}%
\pgfpathrectangle{\pgfqpoint{0.375000in}{0.330000in}}{\pgfqpoint{2.325000in}{2.310000in}}%
\pgfusepath{clip}%
\pgfsetbuttcap%
\pgfsetroundjoin%
\definecolor{currentfill}{rgb}{0.000000,0.000000,0.000000}%
\pgfsetfillcolor{currentfill}%
\pgfsetlinewidth{1.003750pt}%
\definecolor{currentstroke}{rgb}{0.000000,0.000000,0.000000}%
\pgfsetstrokecolor{currentstroke}%
\pgfsetdash{}{0pt}%
\pgfpathmoveto{\pgfqpoint{2.055402in}{1.538958in}}%
\pgfpathcurveto{\pgfqpoint{2.066452in}{1.538958in}}{\pgfqpoint{2.077051in}{1.543348in}}{\pgfqpoint{2.084865in}{1.551161in}}%
\pgfpathcurveto{\pgfqpoint{2.092678in}{1.558975in}}{\pgfqpoint{2.097069in}{1.569574in}}{\pgfqpoint{2.097069in}{1.580624in}}%
\pgfpathcurveto{\pgfqpoint{2.097069in}{1.591674in}}{\pgfqpoint{2.092678in}{1.602273in}}{\pgfqpoint{2.084865in}{1.610087in}}%
\pgfpathcurveto{\pgfqpoint{2.077051in}{1.617901in}}{\pgfqpoint{2.066452in}{1.622291in}}{\pgfqpoint{2.055402in}{1.622291in}}%
\pgfpathcurveto{\pgfqpoint{2.044352in}{1.622291in}}{\pgfqpoint{2.033753in}{1.617901in}}{\pgfqpoint{2.025939in}{1.610087in}}%
\pgfpathcurveto{\pgfqpoint{2.018125in}{1.602273in}}{\pgfqpoint{2.013735in}{1.591674in}}{\pgfqpoint{2.013735in}{1.580624in}}%
\pgfpathcurveto{\pgfqpoint{2.013735in}{1.569574in}}{\pgfqpoint{2.018125in}{1.558975in}}{\pgfqpoint{2.025939in}{1.551161in}}%
\pgfpathcurveto{\pgfqpoint{2.033753in}{1.543348in}}{\pgfqpoint{2.044352in}{1.538958in}}{\pgfqpoint{2.055402in}{1.538958in}}%
\pgfpathclose%
\pgfusepath{stroke,fill}%
\end{pgfscope}%
\begin{pgfscope}%
\pgfpathrectangle{\pgfqpoint{0.375000in}{0.330000in}}{\pgfqpoint{2.325000in}{2.310000in}}%
\pgfusepath{clip}%
\pgfsetbuttcap%
\pgfsetroundjoin%
\definecolor{currentfill}{rgb}{0.000000,0.000000,0.000000}%
\pgfsetfillcolor{currentfill}%
\pgfsetlinewidth{1.003750pt}%
\definecolor{currentstroke}{rgb}{0.000000,0.000000,0.000000}%
\pgfsetstrokecolor{currentstroke}%
\pgfsetdash{}{0pt}%
\pgfpathmoveto{\pgfqpoint{2.055402in}{1.538958in}}%
\pgfpathcurveto{\pgfqpoint{2.066452in}{1.538958in}}{\pgfqpoint{2.077051in}{1.543348in}}{\pgfqpoint{2.084865in}{1.551161in}}%
\pgfpathcurveto{\pgfqpoint{2.092678in}{1.558975in}}{\pgfqpoint{2.097069in}{1.569574in}}{\pgfqpoint{2.097069in}{1.580624in}}%
\pgfpathcurveto{\pgfqpoint{2.097069in}{1.591674in}}{\pgfqpoint{2.092678in}{1.602273in}}{\pgfqpoint{2.084865in}{1.610087in}}%
\pgfpathcurveto{\pgfqpoint{2.077051in}{1.617901in}}{\pgfqpoint{2.066452in}{1.622291in}}{\pgfqpoint{2.055402in}{1.622291in}}%
\pgfpathcurveto{\pgfqpoint{2.044352in}{1.622291in}}{\pgfqpoint{2.033753in}{1.617901in}}{\pgfqpoint{2.025939in}{1.610087in}}%
\pgfpathcurveto{\pgfqpoint{2.018125in}{1.602273in}}{\pgfqpoint{2.013735in}{1.591674in}}{\pgfqpoint{2.013735in}{1.580624in}}%
\pgfpathcurveto{\pgfqpoint{2.013735in}{1.569574in}}{\pgfqpoint{2.018125in}{1.558975in}}{\pgfqpoint{2.025939in}{1.551161in}}%
\pgfpathcurveto{\pgfqpoint{2.033753in}{1.543348in}}{\pgfqpoint{2.044352in}{1.538958in}}{\pgfqpoint{2.055402in}{1.538958in}}%
\pgfpathclose%
\pgfusepath{stroke,fill}%
\end{pgfscope}%
\begin{pgfscope}%
\pgfpathrectangle{\pgfqpoint{0.375000in}{0.330000in}}{\pgfqpoint{2.325000in}{2.310000in}}%
\pgfusepath{clip}%
\pgfsetbuttcap%
\pgfsetroundjoin%
\definecolor{currentfill}{rgb}{0.000000,0.000000,0.000000}%
\pgfsetfillcolor{currentfill}%
\pgfsetlinewidth{1.003750pt}%
\definecolor{currentstroke}{rgb}{0.000000,0.000000,0.000000}%
\pgfsetstrokecolor{currentstroke}%
\pgfsetdash{}{0pt}%
\pgfpathmoveto{\pgfqpoint{2.055402in}{1.486927in}}%
\pgfpathcurveto{\pgfqpoint{2.066452in}{1.486927in}}{\pgfqpoint{2.077051in}{1.491317in}}{\pgfqpoint{2.084865in}{1.499131in}}%
\pgfpathcurveto{\pgfqpoint{2.092678in}{1.506944in}}{\pgfqpoint{2.097069in}{1.517543in}}{\pgfqpoint{2.097069in}{1.528593in}}%
\pgfpathcurveto{\pgfqpoint{2.097069in}{1.539644in}}{\pgfqpoint{2.092678in}{1.550243in}}{\pgfqpoint{2.084865in}{1.558056in}}%
\pgfpathcurveto{\pgfqpoint{2.077051in}{1.565870in}}{\pgfqpoint{2.066452in}{1.570260in}}{\pgfqpoint{2.055402in}{1.570260in}}%
\pgfpathcurveto{\pgfqpoint{2.044352in}{1.570260in}}{\pgfqpoint{2.033753in}{1.565870in}}{\pgfqpoint{2.025939in}{1.558056in}}%
\pgfpathcurveto{\pgfqpoint{2.018125in}{1.550243in}}{\pgfqpoint{2.013735in}{1.539644in}}{\pgfqpoint{2.013735in}{1.528593in}}%
\pgfpathcurveto{\pgfqpoint{2.013735in}{1.517543in}}{\pgfqpoint{2.018125in}{1.506944in}}{\pgfqpoint{2.025939in}{1.499131in}}%
\pgfpathcurveto{\pgfqpoint{2.033753in}{1.491317in}}{\pgfqpoint{2.044352in}{1.486927in}}{\pgfqpoint{2.055402in}{1.486927in}}%
\pgfpathclose%
\pgfusepath{stroke,fill}%
\end{pgfscope}%
\begin{pgfscope}%
\pgfpathrectangle{\pgfqpoint{0.375000in}{0.330000in}}{\pgfqpoint{2.325000in}{2.310000in}}%
\pgfusepath{clip}%
\pgfsetbuttcap%
\pgfsetroundjoin%
\definecolor{currentfill}{rgb}{0.000000,0.000000,0.000000}%
\pgfsetfillcolor{currentfill}%
\pgfsetlinewidth{1.003750pt}%
\definecolor{currentstroke}{rgb}{0.000000,0.000000,0.000000}%
\pgfsetstrokecolor{currentstroke}%
\pgfsetdash{}{0pt}%
\pgfpathmoveto{\pgfqpoint{2.055402in}{1.434896in}}%
\pgfpathcurveto{\pgfqpoint{2.066452in}{1.434896in}}{\pgfqpoint{2.077051in}{1.439286in}}{\pgfqpoint{2.084865in}{1.447100in}}%
\pgfpathcurveto{\pgfqpoint{2.092678in}{1.454913in}}{\pgfqpoint{2.097069in}{1.465512in}}{\pgfqpoint{2.097069in}{1.476563in}}%
\pgfpathcurveto{\pgfqpoint{2.097069in}{1.487613in}}{\pgfqpoint{2.092678in}{1.498212in}}{\pgfqpoint{2.084865in}{1.506025in}}%
\pgfpathcurveto{\pgfqpoint{2.077051in}{1.513839in}}{\pgfqpoint{2.066452in}{1.518229in}}{\pgfqpoint{2.055402in}{1.518229in}}%
\pgfpathcurveto{\pgfqpoint{2.044352in}{1.518229in}}{\pgfqpoint{2.033753in}{1.513839in}}{\pgfqpoint{2.025939in}{1.506025in}}%
\pgfpathcurveto{\pgfqpoint{2.018125in}{1.498212in}}{\pgfqpoint{2.013735in}{1.487613in}}{\pgfqpoint{2.013735in}{1.476563in}}%
\pgfpathcurveto{\pgfqpoint{2.013735in}{1.465512in}}{\pgfqpoint{2.018125in}{1.454913in}}{\pgfqpoint{2.025939in}{1.447100in}}%
\pgfpathcurveto{\pgfqpoint{2.033753in}{1.439286in}}{\pgfqpoint{2.044352in}{1.434896in}}{\pgfqpoint{2.055402in}{1.434896in}}%
\pgfpathclose%
\pgfusepath{stroke,fill}%
\end{pgfscope}%
\begin{pgfscope}%
\pgfpathrectangle{\pgfqpoint{0.375000in}{0.330000in}}{\pgfqpoint{2.325000in}{2.310000in}}%
\pgfusepath{clip}%
\pgfsetbuttcap%
\pgfsetroundjoin%
\definecolor{currentfill}{rgb}{0.000000,0.000000,0.000000}%
\pgfsetfillcolor{currentfill}%
\pgfsetlinewidth{1.003750pt}%
\definecolor{currentstroke}{rgb}{0.000000,0.000000,0.000000}%
\pgfsetstrokecolor{currentstroke}%
\pgfsetdash{}{0pt}%
\pgfpathmoveto{\pgfqpoint{2.055402in}{1.538958in}}%
\pgfpathcurveto{\pgfqpoint{2.066452in}{1.538958in}}{\pgfqpoint{2.077051in}{1.543348in}}{\pgfqpoint{2.084865in}{1.551161in}}%
\pgfpathcurveto{\pgfqpoint{2.092678in}{1.558975in}}{\pgfqpoint{2.097069in}{1.569574in}}{\pgfqpoint{2.097069in}{1.580624in}}%
\pgfpathcurveto{\pgfqpoint{2.097069in}{1.591674in}}{\pgfqpoint{2.092678in}{1.602273in}}{\pgfqpoint{2.084865in}{1.610087in}}%
\pgfpathcurveto{\pgfqpoint{2.077051in}{1.617901in}}{\pgfqpoint{2.066452in}{1.622291in}}{\pgfqpoint{2.055402in}{1.622291in}}%
\pgfpathcurveto{\pgfqpoint{2.044352in}{1.622291in}}{\pgfqpoint{2.033753in}{1.617901in}}{\pgfqpoint{2.025939in}{1.610087in}}%
\pgfpathcurveto{\pgfqpoint{2.018125in}{1.602273in}}{\pgfqpoint{2.013735in}{1.591674in}}{\pgfqpoint{2.013735in}{1.580624in}}%
\pgfpathcurveto{\pgfqpoint{2.013735in}{1.569574in}}{\pgfqpoint{2.018125in}{1.558975in}}{\pgfqpoint{2.025939in}{1.551161in}}%
\pgfpathcurveto{\pgfqpoint{2.033753in}{1.543348in}}{\pgfqpoint{2.044352in}{1.538958in}}{\pgfqpoint{2.055402in}{1.538958in}}%
\pgfpathclose%
\pgfusepath{stroke,fill}%
\end{pgfscope}%
\begin{pgfscope}%
\pgfpathrectangle{\pgfqpoint{0.375000in}{0.330000in}}{\pgfqpoint{2.325000in}{2.310000in}}%
\pgfusepath{clip}%
\pgfsetbuttcap%
\pgfsetroundjoin%
\definecolor{currentfill}{rgb}{0.000000,0.000000,0.000000}%
\pgfsetfillcolor{currentfill}%
\pgfsetlinewidth{1.003750pt}%
\definecolor{currentstroke}{rgb}{0.000000,0.000000,0.000000}%
\pgfsetstrokecolor{currentstroke}%
\pgfsetdash{}{0pt}%
\pgfpathmoveto{\pgfqpoint{2.055402in}{1.486927in}}%
\pgfpathcurveto{\pgfqpoint{2.066452in}{1.486927in}}{\pgfqpoint{2.077051in}{1.491317in}}{\pgfqpoint{2.084865in}{1.499131in}}%
\pgfpathcurveto{\pgfqpoint{2.092678in}{1.506944in}}{\pgfqpoint{2.097069in}{1.517543in}}{\pgfqpoint{2.097069in}{1.528593in}}%
\pgfpathcurveto{\pgfqpoint{2.097069in}{1.539644in}}{\pgfqpoint{2.092678in}{1.550243in}}{\pgfqpoint{2.084865in}{1.558056in}}%
\pgfpathcurveto{\pgfqpoint{2.077051in}{1.565870in}}{\pgfqpoint{2.066452in}{1.570260in}}{\pgfqpoint{2.055402in}{1.570260in}}%
\pgfpathcurveto{\pgfqpoint{2.044352in}{1.570260in}}{\pgfqpoint{2.033753in}{1.565870in}}{\pgfqpoint{2.025939in}{1.558056in}}%
\pgfpathcurveto{\pgfqpoint{2.018125in}{1.550243in}}{\pgfqpoint{2.013735in}{1.539644in}}{\pgfqpoint{2.013735in}{1.528593in}}%
\pgfpathcurveto{\pgfqpoint{2.013735in}{1.517543in}}{\pgfqpoint{2.018125in}{1.506944in}}{\pgfqpoint{2.025939in}{1.499131in}}%
\pgfpathcurveto{\pgfqpoint{2.033753in}{1.491317in}}{\pgfqpoint{2.044352in}{1.486927in}}{\pgfqpoint{2.055402in}{1.486927in}}%
\pgfpathclose%
\pgfusepath{stroke,fill}%
\end{pgfscope}%
\begin{pgfscope}%
\pgfpathrectangle{\pgfqpoint{0.375000in}{0.330000in}}{\pgfqpoint{2.325000in}{2.310000in}}%
\pgfusepath{clip}%
\pgfsetbuttcap%
\pgfsetroundjoin%
\definecolor{currentfill}{rgb}{0.000000,0.000000,0.000000}%
\pgfsetfillcolor{currentfill}%
\pgfsetlinewidth{1.003750pt}%
\definecolor{currentstroke}{rgb}{0.000000,0.000000,0.000000}%
\pgfsetstrokecolor{currentstroke}%
\pgfsetdash{}{0pt}%
\pgfpathmoveto{\pgfqpoint{2.055402in}{1.538958in}}%
\pgfpathcurveto{\pgfqpoint{2.066452in}{1.538958in}}{\pgfqpoint{2.077051in}{1.543348in}}{\pgfqpoint{2.084865in}{1.551161in}}%
\pgfpathcurveto{\pgfqpoint{2.092678in}{1.558975in}}{\pgfqpoint{2.097069in}{1.569574in}}{\pgfqpoint{2.097069in}{1.580624in}}%
\pgfpathcurveto{\pgfqpoint{2.097069in}{1.591674in}}{\pgfqpoint{2.092678in}{1.602273in}}{\pgfqpoint{2.084865in}{1.610087in}}%
\pgfpathcurveto{\pgfqpoint{2.077051in}{1.617901in}}{\pgfqpoint{2.066452in}{1.622291in}}{\pgfqpoint{2.055402in}{1.622291in}}%
\pgfpathcurveto{\pgfqpoint{2.044352in}{1.622291in}}{\pgfqpoint{2.033753in}{1.617901in}}{\pgfqpoint{2.025939in}{1.610087in}}%
\pgfpathcurveto{\pgfqpoint{2.018125in}{1.602273in}}{\pgfqpoint{2.013735in}{1.591674in}}{\pgfqpoint{2.013735in}{1.580624in}}%
\pgfpathcurveto{\pgfqpoint{2.013735in}{1.569574in}}{\pgfqpoint{2.018125in}{1.558975in}}{\pgfqpoint{2.025939in}{1.551161in}}%
\pgfpathcurveto{\pgfqpoint{2.033753in}{1.543348in}}{\pgfqpoint{2.044352in}{1.538958in}}{\pgfqpoint{2.055402in}{1.538958in}}%
\pgfpathclose%
\pgfusepath{stroke,fill}%
\end{pgfscope}%
\begin{pgfscope}%
\pgfpathrectangle{\pgfqpoint{0.375000in}{0.330000in}}{\pgfqpoint{2.325000in}{2.310000in}}%
\pgfusepath{clip}%
\pgfsetbuttcap%
\pgfsetroundjoin%
\definecolor{currentfill}{rgb}{0.000000,0.000000,0.000000}%
\pgfsetfillcolor{currentfill}%
\pgfsetlinewidth{1.003750pt}%
\definecolor{currentstroke}{rgb}{0.000000,0.000000,0.000000}%
\pgfsetstrokecolor{currentstroke}%
\pgfsetdash{}{0pt}%
\pgfpathmoveto{\pgfqpoint{2.055402in}{1.538958in}}%
\pgfpathcurveto{\pgfqpoint{2.066452in}{1.538958in}}{\pgfqpoint{2.077051in}{1.543348in}}{\pgfqpoint{2.084865in}{1.551161in}}%
\pgfpathcurveto{\pgfqpoint{2.092678in}{1.558975in}}{\pgfqpoint{2.097069in}{1.569574in}}{\pgfqpoint{2.097069in}{1.580624in}}%
\pgfpathcurveto{\pgfqpoint{2.097069in}{1.591674in}}{\pgfqpoint{2.092678in}{1.602273in}}{\pgfqpoint{2.084865in}{1.610087in}}%
\pgfpathcurveto{\pgfqpoint{2.077051in}{1.617901in}}{\pgfqpoint{2.066452in}{1.622291in}}{\pgfqpoint{2.055402in}{1.622291in}}%
\pgfpathcurveto{\pgfqpoint{2.044352in}{1.622291in}}{\pgfqpoint{2.033753in}{1.617901in}}{\pgfqpoint{2.025939in}{1.610087in}}%
\pgfpathcurveto{\pgfqpoint{2.018125in}{1.602273in}}{\pgfqpoint{2.013735in}{1.591674in}}{\pgfqpoint{2.013735in}{1.580624in}}%
\pgfpathcurveto{\pgfqpoint{2.013735in}{1.569574in}}{\pgfqpoint{2.018125in}{1.558975in}}{\pgfqpoint{2.025939in}{1.551161in}}%
\pgfpathcurveto{\pgfqpoint{2.033753in}{1.543348in}}{\pgfqpoint{2.044352in}{1.538958in}}{\pgfqpoint{2.055402in}{1.538958in}}%
\pgfpathclose%
\pgfusepath{stroke,fill}%
\end{pgfscope}%
\begin{pgfscope}%
\pgfpathrectangle{\pgfqpoint{0.375000in}{0.330000in}}{\pgfqpoint{2.325000in}{2.310000in}}%
\pgfusepath{clip}%
\pgfsetbuttcap%
\pgfsetroundjoin%
\definecolor{currentfill}{rgb}{0.000000,0.000000,0.000000}%
\pgfsetfillcolor{currentfill}%
\pgfsetlinewidth{1.003750pt}%
\definecolor{currentstroke}{rgb}{0.000000,0.000000,0.000000}%
\pgfsetstrokecolor{currentstroke}%
\pgfsetdash{}{0pt}%
\pgfpathmoveto{\pgfqpoint{2.055402in}{1.486927in}}%
\pgfpathcurveto{\pgfqpoint{2.066452in}{1.486927in}}{\pgfqpoint{2.077051in}{1.491317in}}{\pgfqpoint{2.084865in}{1.499131in}}%
\pgfpathcurveto{\pgfqpoint{2.092678in}{1.506944in}}{\pgfqpoint{2.097069in}{1.517543in}}{\pgfqpoint{2.097069in}{1.528593in}}%
\pgfpathcurveto{\pgfqpoint{2.097069in}{1.539644in}}{\pgfqpoint{2.092678in}{1.550243in}}{\pgfqpoint{2.084865in}{1.558056in}}%
\pgfpathcurveto{\pgfqpoint{2.077051in}{1.565870in}}{\pgfqpoint{2.066452in}{1.570260in}}{\pgfqpoint{2.055402in}{1.570260in}}%
\pgfpathcurveto{\pgfqpoint{2.044352in}{1.570260in}}{\pgfqpoint{2.033753in}{1.565870in}}{\pgfqpoint{2.025939in}{1.558056in}}%
\pgfpathcurveto{\pgfqpoint{2.018125in}{1.550243in}}{\pgfqpoint{2.013735in}{1.539644in}}{\pgfqpoint{2.013735in}{1.528593in}}%
\pgfpathcurveto{\pgfqpoint{2.013735in}{1.517543in}}{\pgfqpoint{2.018125in}{1.506944in}}{\pgfqpoint{2.025939in}{1.499131in}}%
\pgfpathcurveto{\pgfqpoint{2.033753in}{1.491317in}}{\pgfqpoint{2.044352in}{1.486927in}}{\pgfqpoint{2.055402in}{1.486927in}}%
\pgfpathclose%
\pgfusepath{stroke,fill}%
\end{pgfscope}%
\begin{pgfscope}%
\pgfpathrectangle{\pgfqpoint{0.375000in}{0.330000in}}{\pgfqpoint{2.325000in}{2.310000in}}%
\pgfusepath{clip}%
\pgfsetbuttcap%
\pgfsetroundjoin%
\definecolor{currentfill}{rgb}{0.000000,0.000000,0.000000}%
\pgfsetfillcolor{currentfill}%
\pgfsetlinewidth{1.003750pt}%
\definecolor{currentstroke}{rgb}{0.000000,0.000000,0.000000}%
\pgfsetstrokecolor{currentstroke}%
\pgfsetdash{}{0pt}%
\pgfpathmoveto{\pgfqpoint{2.055402in}{1.590988in}}%
\pgfpathcurveto{\pgfqpoint{2.066452in}{1.590988in}}{\pgfqpoint{2.077051in}{1.595379in}}{\pgfqpoint{2.084865in}{1.603192in}}%
\pgfpathcurveto{\pgfqpoint{2.092678in}{1.611006in}}{\pgfqpoint{2.097069in}{1.621605in}}{\pgfqpoint{2.097069in}{1.632655in}}%
\pgfpathcurveto{\pgfqpoint{2.097069in}{1.643705in}}{\pgfqpoint{2.092678in}{1.654304in}}{\pgfqpoint{2.084865in}{1.662118in}}%
\pgfpathcurveto{\pgfqpoint{2.077051in}{1.669931in}}{\pgfqpoint{2.066452in}{1.674322in}}{\pgfqpoint{2.055402in}{1.674322in}}%
\pgfpathcurveto{\pgfqpoint{2.044352in}{1.674322in}}{\pgfqpoint{2.033753in}{1.669931in}}{\pgfqpoint{2.025939in}{1.662118in}}%
\pgfpathcurveto{\pgfqpoint{2.018125in}{1.654304in}}{\pgfqpoint{2.013735in}{1.643705in}}{\pgfqpoint{2.013735in}{1.632655in}}%
\pgfpathcurveto{\pgfqpoint{2.013735in}{1.621605in}}{\pgfqpoint{2.018125in}{1.611006in}}{\pgfqpoint{2.025939in}{1.603192in}}%
\pgfpathcurveto{\pgfqpoint{2.033753in}{1.595379in}}{\pgfqpoint{2.044352in}{1.590988in}}{\pgfqpoint{2.055402in}{1.590988in}}%
\pgfpathclose%
\pgfusepath{stroke,fill}%
\end{pgfscope}%
\begin{pgfscope}%
\pgfpathrectangle{\pgfqpoint{0.375000in}{0.330000in}}{\pgfqpoint{2.325000in}{2.310000in}}%
\pgfusepath{clip}%
\pgfsetbuttcap%
\pgfsetroundjoin%
\definecolor{currentfill}{rgb}{0.000000,0.000000,0.000000}%
\pgfsetfillcolor{currentfill}%
\pgfsetlinewidth{1.003750pt}%
\definecolor{currentstroke}{rgb}{0.000000,0.000000,0.000000}%
\pgfsetstrokecolor{currentstroke}%
\pgfsetdash{}{0pt}%
\pgfpathmoveto{\pgfqpoint{2.580256in}{2.371451in}}%
\pgfpathcurveto{\pgfqpoint{2.591306in}{2.371451in}}{\pgfqpoint{2.601905in}{2.375841in}}{\pgfqpoint{2.609718in}{2.383655in}}%
\pgfpathcurveto{\pgfqpoint{2.617532in}{2.391468in}}{\pgfqpoint{2.621922in}{2.402067in}}{\pgfqpoint{2.621922in}{2.413117in}}%
\pgfpathcurveto{\pgfqpoint{2.621922in}{2.424168in}}{\pgfqpoint{2.617532in}{2.434767in}}{\pgfqpoint{2.609718in}{2.442580in}}%
\pgfpathcurveto{\pgfqpoint{2.601905in}{2.450394in}}{\pgfqpoint{2.591306in}{2.454784in}}{\pgfqpoint{2.580256in}{2.454784in}}%
\pgfpathcurveto{\pgfqpoint{2.569206in}{2.454784in}}{\pgfqpoint{2.558607in}{2.450394in}}{\pgfqpoint{2.550793in}{2.442580in}}%
\pgfpathcurveto{\pgfqpoint{2.542979in}{2.434767in}}{\pgfqpoint{2.538589in}{2.424168in}}{\pgfqpoint{2.538589in}{2.413117in}}%
\pgfpathcurveto{\pgfqpoint{2.538589in}{2.402067in}}{\pgfqpoint{2.542979in}{2.391468in}}{\pgfqpoint{2.550793in}{2.383655in}}%
\pgfpathcurveto{\pgfqpoint{2.558607in}{2.375841in}}{\pgfqpoint{2.569206in}{2.371451in}}{\pgfqpoint{2.580256in}{2.371451in}}%
\pgfpathclose%
\pgfusepath{stroke,fill}%
\end{pgfscope}%
\begin{pgfscope}%
\pgfpathrectangle{\pgfqpoint{0.375000in}{0.330000in}}{\pgfqpoint{2.325000in}{2.310000in}}%
\pgfusepath{clip}%
\pgfsetbuttcap%
\pgfsetroundjoin%
\definecolor{currentfill}{rgb}{0.000000,0.000000,0.000000}%
\pgfsetfillcolor{currentfill}%
\pgfsetlinewidth{1.003750pt}%
\definecolor{currentstroke}{rgb}{0.000000,0.000000,0.000000}%
\pgfsetstrokecolor{currentstroke}%
\pgfsetdash{}{0pt}%
\pgfpathmoveto{\pgfqpoint{2.580256in}{2.319420in}}%
\pgfpathcurveto{\pgfqpoint{2.591306in}{2.319420in}}{\pgfqpoint{2.601905in}{2.323810in}}{\pgfqpoint{2.609718in}{2.331624in}}%
\pgfpathcurveto{\pgfqpoint{2.617532in}{2.339437in}}{\pgfqpoint{2.621922in}{2.350037in}}{\pgfqpoint{2.621922in}{2.361087in}}%
\pgfpathcurveto{\pgfqpoint{2.621922in}{2.372137in}}{\pgfqpoint{2.617532in}{2.382736in}}{\pgfqpoint{2.609718in}{2.390549in}}%
\pgfpathcurveto{\pgfqpoint{2.601905in}{2.398363in}}{\pgfqpoint{2.591306in}{2.402753in}}{\pgfqpoint{2.580256in}{2.402753in}}%
\pgfpathcurveto{\pgfqpoint{2.569206in}{2.402753in}}{\pgfqpoint{2.558607in}{2.398363in}}{\pgfqpoint{2.550793in}{2.390549in}}%
\pgfpathcurveto{\pgfqpoint{2.542979in}{2.382736in}}{\pgfqpoint{2.538589in}{2.372137in}}{\pgfqpoint{2.538589in}{2.361087in}}%
\pgfpathcurveto{\pgfqpoint{2.538589in}{2.350037in}}{\pgfqpoint{2.542979in}{2.339437in}}{\pgfqpoint{2.550793in}{2.331624in}}%
\pgfpathcurveto{\pgfqpoint{2.558607in}{2.323810in}}{\pgfqpoint{2.569206in}{2.319420in}}{\pgfqpoint{2.580256in}{2.319420in}}%
\pgfpathclose%
\pgfusepath{stroke,fill}%
\end{pgfscope}%
\begin{pgfscope}%
\pgfpathrectangle{\pgfqpoint{0.375000in}{0.330000in}}{\pgfqpoint{2.325000in}{2.310000in}}%
\pgfusepath{clip}%
\pgfsetbuttcap%
\pgfsetroundjoin%
\definecolor{currentfill}{rgb}{0.000000,0.000000,0.000000}%
\pgfsetfillcolor{currentfill}%
\pgfsetlinewidth{1.003750pt}%
\definecolor{currentstroke}{rgb}{0.000000,0.000000,0.000000}%
\pgfsetstrokecolor{currentstroke}%
\pgfsetdash{}{0pt}%
\pgfpathmoveto{\pgfqpoint{2.580256in}{2.267389in}}%
\pgfpathcurveto{\pgfqpoint{2.591306in}{2.267389in}}{\pgfqpoint{2.601905in}{2.271779in}}{\pgfqpoint{2.609718in}{2.279593in}}%
\pgfpathcurveto{\pgfqpoint{2.617532in}{2.287407in}}{\pgfqpoint{2.621922in}{2.298006in}}{\pgfqpoint{2.621922in}{2.309056in}}%
\pgfpathcurveto{\pgfqpoint{2.621922in}{2.320106in}}{\pgfqpoint{2.617532in}{2.330705in}}{\pgfqpoint{2.609718in}{2.338519in}}%
\pgfpathcurveto{\pgfqpoint{2.601905in}{2.346332in}}{\pgfqpoint{2.591306in}{2.350722in}}{\pgfqpoint{2.580256in}{2.350722in}}%
\pgfpathcurveto{\pgfqpoint{2.569206in}{2.350722in}}{\pgfqpoint{2.558607in}{2.346332in}}{\pgfqpoint{2.550793in}{2.338519in}}%
\pgfpathcurveto{\pgfqpoint{2.542979in}{2.330705in}}{\pgfqpoint{2.538589in}{2.320106in}}{\pgfqpoint{2.538589in}{2.309056in}}%
\pgfpathcurveto{\pgfqpoint{2.538589in}{2.298006in}}{\pgfqpoint{2.542979in}{2.287407in}}{\pgfqpoint{2.550793in}{2.279593in}}%
\pgfpathcurveto{\pgfqpoint{2.558607in}{2.271779in}}{\pgfqpoint{2.569206in}{2.267389in}}{\pgfqpoint{2.580256in}{2.267389in}}%
\pgfpathclose%
\pgfusepath{stroke,fill}%
\end{pgfscope}%
\begin{pgfscope}%
\pgfpathrectangle{\pgfqpoint{0.375000in}{0.330000in}}{\pgfqpoint{2.325000in}{2.310000in}}%
\pgfusepath{clip}%
\pgfsetbuttcap%
\pgfsetroundjoin%
\definecolor{currentfill}{rgb}{0.000000,0.000000,0.000000}%
\pgfsetfillcolor{currentfill}%
\pgfsetlinewidth{1.003750pt}%
\definecolor{currentstroke}{rgb}{0.000000,0.000000,0.000000}%
\pgfsetstrokecolor{currentstroke}%
\pgfsetdash{}{0pt}%
\pgfpathmoveto{\pgfqpoint{2.580256in}{2.267389in}}%
\pgfpathcurveto{\pgfqpoint{2.591306in}{2.267389in}}{\pgfqpoint{2.601905in}{2.271779in}}{\pgfqpoint{2.609718in}{2.279593in}}%
\pgfpathcurveto{\pgfqpoint{2.617532in}{2.287407in}}{\pgfqpoint{2.621922in}{2.298006in}}{\pgfqpoint{2.621922in}{2.309056in}}%
\pgfpathcurveto{\pgfqpoint{2.621922in}{2.320106in}}{\pgfqpoint{2.617532in}{2.330705in}}{\pgfqpoint{2.609718in}{2.338519in}}%
\pgfpathcurveto{\pgfqpoint{2.601905in}{2.346332in}}{\pgfqpoint{2.591306in}{2.350722in}}{\pgfqpoint{2.580256in}{2.350722in}}%
\pgfpathcurveto{\pgfqpoint{2.569206in}{2.350722in}}{\pgfqpoint{2.558607in}{2.346332in}}{\pgfqpoint{2.550793in}{2.338519in}}%
\pgfpathcurveto{\pgfqpoint{2.542979in}{2.330705in}}{\pgfqpoint{2.538589in}{2.320106in}}{\pgfqpoint{2.538589in}{2.309056in}}%
\pgfpathcurveto{\pgfqpoint{2.538589in}{2.298006in}}{\pgfqpoint{2.542979in}{2.287407in}}{\pgfqpoint{2.550793in}{2.279593in}}%
\pgfpathcurveto{\pgfqpoint{2.558607in}{2.271779in}}{\pgfqpoint{2.569206in}{2.267389in}}{\pgfqpoint{2.580256in}{2.267389in}}%
\pgfpathclose%
\pgfusepath{stroke,fill}%
\end{pgfscope}%
\begin{pgfscope}%
\pgfpathrectangle{\pgfqpoint{0.375000in}{0.330000in}}{\pgfqpoint{2.325000in}{2.310000in}}%
\pgfusepath{clip}%
\pgfsetbuttcap%
\pgfsetroundjoin%
\definecolor{currentfill}{rgb}{0.000000,0.000000,0.000000}%
\pgfsetfillcolor{currentfill}%
\pgfsetlinewidth{1.003750pt}%
\definecolor{currentstroke}{rgb}{0.000000,0.000000,0.000000}%
\pgfsetstrokecolor{currentstroke}%
\pgfsetdash{}{0pt}%
\pgfpathmoveto{\pgfqpoint{2.580256in}{2.319420in}}%
\pgfpathcurveto{\pgfqpoint{2.591306in}{2.319420in}}{\pgfqpoint{2.601905in}{2.323810in}}{\pgfqpoint{2.609718in}{2.331624in}}%
\pgfpathcurveto{\pgfqpoint{2.617532in}{2.339437in}}{\pgfqpoint{2.621922in}{2.350037in}}{\pgfqpoint{2.621922in}{2.361087in}}%
\pgfpathcurveto{\pgfqpoint{2.621922in}{2.372137in}}{\pgfqpoint{2.617532in}{2.382736in}}{\pgfqpoint{2.609718in}{2.390549in}}%
\pgfpathcurveto{\pgfqpoint{2.601905in}{2.398363in}}{\pgfqpoint{2.591306in}{2.402753in}}{\pgfqpoint{2.580256in}{2.402753in}}%
\pgfpathcurveto{\pgfqpoint{2.569206in}{2.402753in}}{\pgfqpoint{2.558607in}{2.398363in}}{\pgfqpoint{2.550793in}{2.390549in}}%
\pgfpathcurveto{\pgfqpoint{2.542979in}{2.382736in}}{\pgfqpoint{2.538589in}{2.372137in}}{\pgfqpoint{2.538589in}{2.361087in}}%
\pgfpathcurveto{\pgfqpoint{2.538589in}{2.350037in}}{\pgfqpoint{2.542979in}{2.339437in}}{\pgfqpoint{2.550793in}{2.331624in}}%
\pgfpathcurveto{\pgfqpoint{2.558607in}{2.323810in}}{\pgfqpoint{2.569206in}{2.319420in}}{\pgfqpoint{2.580256in}{2.319420in}}%
\pgfpathclose%
\pgfusepath{stroke,fill}%
\end{pgfscope}%
\begin{pgfscope}%
\pgfpathrectangle{\pgfqpoint{0.375000in}{0.330000in}}{\pgfqpoint{2.325000in}{2.310000in}}%
\pgfusepath{clip}%
\pgfsetbuttcap%
\pgfsetroundjoin%
\definecolor{currentfill}{rgb}{0.000000,0.000000,0.000000}%
\pgfsetfillcolor{currentfill}%
\pgfsetlinewidth{1.003750pt}%
\definecolor{currentstroke}{rgb}{0.000000,0.000000,0.000000}%
\pgfsetstrokecolor{currentstroke}%
\pgfsetdash{}{0pt}%
\pgfpathmoveto{\pgfqpoint{2.580256in}{2.267389in}}%
\pgfpathcurveto{\pgfqpoint{2.591306in}{2.267389in}}{\pgfqpoint{2.601905in}{2.271779in}}{\pgfqpoint{2.609718in}{2.279593in}}%
\pgfpathcurveto{\pgfqpoint{2.617532in}{2.287407in}}{\pgfqpoint{2.621922in}{2.298006in}}{\pgfqpoint{2.621922in}{2.309056in}}%
\pgfpathcurveto{\pgfqpoint{2.621922in}{2.320106in}}{\pgfqpoint{2.617532in}{2.330705in}}{\pgfqpoint{2.609718in}{2.338519in}}%
\pgfpathcurveto{\pgfqpoint{2.601905in}{2.346332in}}{\pgfqpoint{2.591306in}{2.350722in}}{\pgfqpoint{2.580256in}{2.350722in}}%
\pgfpathcurveto{\pgfqpoint{2.569206in}{2.350722in}}{\pgfqpoint{2.558607in}{2.346332in}}{\pgfqpoint{2.550793in}{2.338519in}}%
\pgfpathcurveto{\pgfqpoint{2.542979in}{2.330705in}}{\pgfqpoint{2.538589in}{2.320106in}}{\pgfqpoint{2.538589in}{2.309056in}}%
\pgfpathcurveto{\pgfqpoint{2.538589in}{2.298006in}}{\pgfqpoint{2.542979in}{2.287407in}}{\pgfqpoint{2.550793in}{2.279593in}}%
\pgfpathcurveto{\pgfqpoint{2.558607in}{2.271779in}}{\pgfqpoint{2.569206in}{2.267389in}}{\pgfqpoint{2.580256in}{2.267389in}}%
\pgfpathclose%
\pgfusepath{stroke,fill}%
\end{pgfscope}%
\begin{pgfscope}%
\pgfpathrectangle{\pgfqpoint{0.375000in}{0.330000in}}{\pgfqpoint{2.325000in}{2.310000in}}%
\pgfusepath{clip}%
\pgfsetbuttcap%
\pgfsetroundjoin%
\definecolor{currentfill}{rgb}{0.000000,0.000000,0.000000}%
\pgfsetfillcolor{currentfill}%
\pgfsetlinewidth{1.003750pt}%
\definecolor{currentstroke}{rgb}{0.000000,0.000000,0.000000}%
\pgfsetstrokecolor{currentstroke}%
\pgfsetdash{}{0pt}%
\pgfpathmoveto{\pgfqpoint{2.580256in}{2.267389in}}%
\pgfpathcurveto{\pgfqpoint{2.591306in}{2.267389in}}{\pgfqpoint{2.601905in}{2.271779in}}{\pgfqpoint{2.609718in}{2.279593in}}%
\pgfpathcurveto{\pgfqpoint{2.617532in}{2.287407in}}{\pgfqpoint{2.621922in}{2.298006in}}{\pgfqpoint{2.621922in}{2.309056in}}%
\pgfpathcurveto{\pgfqpoint{2.621922in}{2.320106in}}{\pgfqpoint{2.617532in}{2.330705in}}{\pgfqpoint{2.609718in}{2.338519in}}%
\pgfpathcurveto{\pgfqpoint{2.601905in}{2.346332in}}{\pgfqpoint{2.591306in}{2.350722in}}{\pgfqpoint{2.580256in}{2.350722in}}%
\pgfpathcurveto{\pgfqpoint{2.569206in}{2.350722in}}{\pgfqpoint{2.558607in}{2.346332in}}{\pgfqpoint{2.550793in}{2.338519in}}%
\pgfpathcurveto{\pgfqpoint{2.542979in}{2.330705in}}{\pgfqpoint{2.538589in}{2.320106in}}{\pgfqpoint{2.538589in}{2.309056in}}%
\pgfpathcurveto{\pgfqpoint{2.538589in}{2.298006in}}{\pgfqpoint{2.542979in}{2.287407in}}{\pgfqpoint{2.550793in}{2.279593in}}%
\pgfpathcurveto{\pgfqpoint{2.558607in}{2.271779in}}{\pgfqpoint{2.569206in}{2.267389in}}{\pgfqpoint{2.580256in}{2.267389in}}%
\pgfpathclose%
\pgfusepath{stroke,fill}%
\end{pgfscope}%
\begin{pgfscope}%
\pgfpathrectangle{\pgfqpoint{0.375000in}{0.330000in}}{\pgfqpoint{2.325000in}{2.310000in}}%
\pgfusepath{clip}%
\pgfsetbuttcap%
\pgfsetroundjoin%
\definecolor{currentfill}{rgb}{0.000000,0.000000,0.000000}%
\pgfsetfillcolor{currentfill}%
\pgfsetlinewidth{1.003750pt}%
\definecolor{currentstroke}{rgb}{0.000000,0.000000,0.000000}%
\pgfsetstrokecolor{currentstroke}%
\pgfsetdash{}{0pt}%
\pgfpathmoveto{\pgfqpoint{2.580256in}{2.215358in}}%
\pgfpathcurveto{\pgfqpoint{2.591306in}{2.215358in}}{\pgfqpoint{2.601905in}{2.219749in}}{\pgfqpoint{2.609718in}{2.227562in}}%
\pgfpathcurveto{\pgfqpoint{2.617532in}{2.235376in}}{\pgfqpoint{2.621922in}{2.245975in}}{\pgfqpoint{2.621922in}{2.257025in}}%
\pgfpathcurveto{\pgfqpoint{2.621922in}{2.268075in}}{\pgfqpoint{2.617532in}{2.278674in}}{\pgfqpoint{2.609718in}{2.286488in}}%
\pgfpathcurveto{\pgfqpoint{2.601905in}{2.294301in}}{\pgfqpoint{2.591306in}{2.298692in}}{\pgfqpoint{2.580256in}{2.298692in}}%
\pgfpathcurveto{\pgfqpoint{2.569206in}{2.298692in}}{\pgfqpoint{2.558607in}{2.294301in}}{\pgfqpoint{2.550793in}{2.286488in}}%
\pgfpathcurveto{\pgfqpoint{2.542979in}{2.278674in}}{\pgfqpoint{2.538589in}{2.268075in}}{\pgfqpoint{2.538589in}{2.257025in}}%
\pgfpathcurveto{\pgfqpoint{2.538589in}{2.245975in}}{\pgfqpoint{2.542979in}{2.235376in}}{\pgfqpoint{2.550793in}{2.227562in}}%
\pgfpathcurveto{\pgfqpoint{2.558607in}{2.219749in}}{\pgfqpoint{2.569206in}{2.215358in}}{\pgfqpoint{2.580256in}{2.215358in}}%
\pgfpathclose%
\pgfusepath{stroke,fill}%
\end{pgfscope}%
\begin{pgfscope}%
\pgfpathrectangle{\pgfqpoint{0.375000in}{0.330000in}}{\pgfqpoint{2.325000in}{2.310000in}}%
\pgfusepath{clip}%
\pgfsetbuttcap%
\pgfsetroundjoin%
\definecolor{currentfill}{rgb}{0.000000,0.000000,0.000000}%
\pgfsetfillcolor{currentfill}%
\pgfsetlinewidth{1.003750pt}%
\definecolor{currentstroke}{rgb}{0.000000,0.000000,0.000000}%
\pgfsetstrokecolor{currentstroke}%
\pgfsetdash{}{0pt}%
\pgfpathmoveto{\pgfqpoint{2.580256in}{2.215358in}}%
\pgfpathcurveto{\pgfqpoint{2.591306in}{2.215358in}}{\pgfqpoint{2.601905in}{2.219749in}}{\pgfqpoint{2.609718in}{2.227562in}}%
\pgfpathcurveto{\pgfqpoint{2.617532in}{2.235376in}}{\pgfqpoint{2.621922in}{2.245975in}}{\pgfqpoint{2.621922in}{2.257025in}}%
\pgfpathcurveto{\pgfqpoint{2.621922in}{2.268075in}}{\pgfqpoint{2.617532in}{2.278674in}}{\pgfqpoint{2.609718in}{2.286488in}}%
\pgfpathcurveto{\pgfqpoint{2.601905in}{2.294301in}}{\pgfqpoint{2.591306in}{2.298692in}}{\pgfqpoint{2.580256in}{2.298692in}}%
\pgfpathcurveto{\pgfqpoint{2.569206in}{2.298692in}}{\pgfqpoint{2.558607in}{2.294301in}}{\pgfqpoint{2.550793in}{2.286488in}}%
\pgfpathcurveto{\pgfqpoint{2.542979in}{2.278674in}}{\pgfqpoint{2.538589in}{2.268075in}}{\pgfqpoint{2.538589in}{2.257025in}}%
\pgfpathcurveto{\pgfqpoint{2.538589in}{2.245975in}}{\pgfqpoint{2.542979in}{2.235376in}}{\pgfqpoint{2.550793in}{2.227562in}}%
\pgfpathcurveto{\pgfqpoint{2.558607in}{2.219749in}}{\pgfqpoint{2.569206in}{2.215358in}}{\pgfqpoint{2.580256in}{2.215358in}}%
\pgfpathclose%
\pgfusepath{stroke,fill}%
\end{pgfscope}%
\begin{pgfscope}%
\pgfpathrectangle{\pgfqpoint{0.375000in}{0.330000in}}{\pgfqpoint{2.325000in}{2.310000in}}%
\pgfusepath{clip}%
\pgfsetbuttcap%
\pgfsetroundjoin%
\definecolor{currentfill}{rgb}{0.000000,0.000000,0.000000}%
\pgfsetfillcolor{currentfill}%
\pgfsetlinewidth{1.003750pt}%
\definecolor{currentstroke}{rgb}{0.000000,0.000000,0.000000}%
\pgfsetstrokecolor{currentstroke}%
\pgfsetdash{}{0pt}%
\pgfpathmoveto{\pgfqpoint{2.580256in}{2.319420in}}%
\pgfpathcurveto{\pgfqpoint{2.591306in}{2.319420in}}{\pgfqpoint{2.601905in}{2.323810in}}{\pgfqpoint{2.609718in}{2.331624in}}%
\pgfpathcurveto{\pgfqpoint{2.617532in}{2.339437in}}{\pgfqpoint{2.621922in}{2.350037in}}{\pgfqpoint{2.621922in}{2.361087in}}%
\pgfpathcurveto{\pgfqpoint{2.621922in}{2.372137in}}{\pgfqpoint{2.617532in}{2.382736in}}{\pgfqpoint{2.609718in}{2.390549in}}%
\pgfpathcurveto{\pgfqpoint{2.601905in}{2.398363in}}{\pgfqpoint{2.591306in}{2.402753in}}{\pgfqpoint{2.580256in}{2.402753in}}%
\pgfpathcurveto{\pgfqpoint{2.569206in}{2.402753in}}{\pgfqpoint{2.558607in}{2.398363in}}{\pgfqpoint{2.550793in}{2.390549in}}%
\pgfpathcurveto{\pgfqpoint{2.542979in}{2.382736in}}{\pgfqpoint{2.538589in}{2.372137in}}{\pgfqpoint{2.538589in}{2.361087in}}%
\pgfpathcurveto{\pgfqpoint{2.538589in}{2.350037in}}{\pgfqpoint{2.542979in}{2.339437in}}{\pgfqpoint{2.550793in}{2.331624in}}%
\pgfpathcurveto{\pgfqpoint{2.558607in}{2.323810in}}{\pgfqpoint{2.569206in}{2.319420in}}{\pgfqpoint{2.580256in}{2.319420in}}%
\pgfpathclose%
\pgfusepath{stroke,fill}%
\end{pgfscope}%
\begin{pgfscope}%
\pgfpathrectangle{\pgfqpoint{0.375000in}{0.330000in}}{\pgfqpoint{2.325000in}{2.310000in}}%
\pgfusepath{clip}%
\pgfsetbuttcap%
\pgfsetroundjoin%
\definecolor{currentfill}{rgb}{0.000000,0.000000,0.000000}%
\pgfsetfillcolor{currentfill}%
\pgfsetlinewidth{1.003750pt}%
\definecolor{currentstroke}{rgb}{0.000000,0.000000,0.000000}%
\pgfsetstrokecolor{currentstroke}%
\pgfsetdash{}{0pt}%
\pgfpathmoveto{\pgfqpoint{2.580256in}{2.319420in}}%
\pgfpathcurveto{\pgfqpoint{2.591306in}{2.319420in}}{\pgfqpoint{2.601905in}{2.323810in}}{\pgfqpoint{2.609718in}{2.331624in}}%
\pgfpathcurveto{\pgfqpoint{2.617532in}{2.339437in}}{\pgfqpoint{2.621922in}{2.350037in}}{\pgfqpoint{2.621922in}{2.361087in}}%
\pgfpathcurveto{\pgfqpoint{2.621922in}{2.372137in}}{\pgfqpoint{2.617532in}{2.382736in}}{\pgfqpoint{2.609718in}{2.390549in}}%
\pgfpathcurveto{\pgfqpoint{2.601905in}{2.398363in}}{\pgfqpoint{2.591306in}{2.402753in}}{\pgfqpoint{2.580256in}{2.402753in}}%
\pgfpathcurveto{\pgfqpoint{2.569206in}{2.402753in}}{\pgfqpoint{2.558607in}{2.398363in}}{\pgfqpoint{2.550793in}{2.390549in}}%
\pgfpathcurveto{\pgfqpoint{2.542979in}{2.382736in}}{\pgfqpoint{2.538589in}{2.372137in}}{\pgfqpoint{2.538589in}{2.361087in}}%
\pgfpathcurveto{\pgfqpoint{2.538589in}{2.350037in}}{\pgfqpoint{2.542979in}{2.339437in}}{\pgfqpoint{2.550793in}{2.331624in}}%
\pgfpathcurveto{\pgfqpoint{2.558607in}{2.323810in}}{\pgfqpoint{2.569206in}{2.319420in}}{\pgfqpoint{2.580256in}{2.319420in}}%
\pgfpathclose%
\pgfusepath{stroke,fill}%
\end{pgfscope}%
\begin{pgfscope}%
\pgfpathrectangle{\pgfqpoint{0.375000in}{0.330000in}}{\pgfqpoint{2.325000in}{2.310000in}}%
\pgfusepath{clip}%
\pgfsetbuttcap%
\pgfsetroundjoin%
\definecolor{currentfill}{rgb}{0.000000,0.000000,0.000000}%
\pgfsetfillcolor{currentfill}%
\pgfsetlinewidth{1.003750pt}%
\definecolor{currentstroke}{rgb}{0.000000,0.000000,0.000000}%
\pgfsetstrokecolor{currentstroke}%
\pgfsetdash{}{0pt}%
\pgfpathmoveto{\pgfqpoint{2.580256in}{2.319420in}}%
\pgfpathcurveto{\pgfqpoint{2.591306in}{2.319420in}}{\pgfqpoint{2.601905in}{2.323810in}}{\pgfqpoint{2.609718in}{2.331624in}}%
\pgfpathcurveto{\pgfqpoint{2.617532in}{2.339437in}}{\pgfqpoint{2.621922in}{2.350037in}}{\pgfqpoint{2.621922in}{2.361087in}}%
\pgfpathcurveto{\pgfqpoint{2.621922in}{2.372137in}}{\pgfqpoint{2.617532in}{2.382736in}}{\pgfqpoint{2.609718in}{2.390549in}}%
\pgfpathcurveto{\pgfqpoint{2.601905in}{2.398363in}}{\pgfqpoint{2.591306in}{2.402753in}}{\pgfqpoint{2.580256in}{2.402753in}}%
\pgfpathcurveto{\pgfqpoint{2.569206in}{2.402753in}}{\pgfqpoint{2.558607in}{2.398363in}}{\pgfqpoint{2.550793in}{2.390549in}}%
\pgfpathcurveto{\pgfqpoint{2.542979in}{2.382736in}}{\pgfqpoint{2.538589in}{2.372137in}}{\pgfqpoint{2.538589in}{2.361087in}}%
\pgfpathcurveto{\pgfqpoint{2.538589in}{2.350037in}}{\pgfqpoint{2.542979in}{2.339437in}}{\pgfqpoint{2.550793in}{2.331624in}}%
\pgfpathcurveto{\pgfqpoint{2.558607in}{2.323810in}}{\pgfqpoint{2.569206in}{2.319420in}}{\pgfqpoint{2.580256in}{2.319420in}}%
\pgfpathclose%
\pgfusepath{stroke,fill}%
\end{pgfscope}%
\begin{pgfscope}%
\pgfpathrectangle{\pgfqpoint{0.375000in}{0.330000in}}{\pgfqpoint{2.325000in}{2.310000in}}%
\pgfusepath{clip}%
\pgfsetbuttcap%
\pgfsetroundjoin%
\definecolor{currentfill}{rgb}{0.000000,0.000000,0.000000}%
\pgfsetfillcolor{currentfill}%
\pgfsetlinewidth{1.003750pt}%
\definecolor{currentstroke}{rgb}{0.000000,0.000000,0.000000}%
\pgfsetstrokecolor{currentstroke}%
\pgfsetdash{}{0pt}%
\pgfpathmoveto{\pgfqpoint{2.580256in}{2.267389in}}%
\pgfpathcurveto{\pgfqpoint{2.591306in}{2.267389in}}{\pgfqpoint{2.601905in}{2.271779in}}{\pgfqpoint{2.609718in}{2.279593in}}%
\pgfpathcurveto{\pgfqpoint{2.617532in}{2.287407in}}{\pgfqpoint{2.621922in}{2.298006in}}{\pgfqpoint{2.621922in}{2.309056in}}%
\pgfpathcurveto{\pgfqpoint{2.621922in}{2.320106in}}{\pgfqpoint{2.617532in}{2.330705in}}{\pgfqpoint{2.609718in}{2.338519in}}%
\pgfpathcurveto{\pgfqpoint{2.601905in}{2.346332in}}{\pgfqpoint{2.591306in}{2.350722in}}{\pgfqpoint{2.580256in}{2.350722in}}%
\pgfpathcurveto{\pgfqpoint{2.569206in}{2.350722in}}{\pgfqpoint{2.558607in}{2.346332in}}{\pgfqpoint{2.550793in}{2.338519in}}%
\pgfpathcurveto{\pgfqpoint{2.542979in}{2.330705in}}{\pgfqpoint{2.538589in}{2.320106in}}{\pgfqpoint{2.538589in}{2.309056in}}%
\pgfpathcurveto{\pgfqpoint{2.538589in}{2.298006in}}{\pgfqpoint{2.542979in}{2.287407in}}{\pgfqpoint{2.550793in}{2.279593in}}%
\pgfpathcurveto{\pgfqpoint{2.558607in}{2.271779in}}{\pgfqpoint{2.569206in}{2.267389in}}{\pgfqpoint{2.580256in}{2.267389in}}%
\pgfpathclose%
\pgfusepath{stroke,fill}%
\end{pgfscope}%
\begin{pgfscope}%
\pgfpathrectangle{\pgfqpoint{0.375000in}{0.330000in}}{\pgfqpoint{2.325000in}{2.310000in}}%
\pgfusepath{clip}%
\pgfsetbuttcap%
\pgfsetroundjoin%
\definecolor{currentfill}{rgb}{0.000000,0.000000,0.000000}%
\pgfsetfillcolor{currentfill}%
\pgfsetlinewidth{1.003750pt}%
\definecolor{currentstroke}{rgb}{0.000000,0.000000,0.000000}%
\pgfsetstrokecolor{currentstroke}%
\pgfsetdash{}{0pt}%
\pgfpathmoveto{\pgfqpoint{2.580256in}{2.267389in}}%
\pgfpathcurveto{\pgfqpoint{2.591306in}{2.267389in}}{\pgfqpoint{2.601905in}{2.271779in}}{\pgfqpoint{2.609718in}{2.279593in}}%
\pgfpathcurveto{\pgfqpoint{2.617532in}{2.287407in}}{\pgfqpoint{2.621922in}{2.298006in}}{\pgfqpoint{2.621922in}{2.309056in}}%
\pgfpathcurveto{\pgfqpoint{2.621922in}{2.320106in}}{\pgfqpoint{2.617532in}{2.330705in}}{\pgfqpoint{2.609718in}{2.338519in}}%
\pgfpathcurveto{\pgfqpoint{2.601905in}{2.346332in}}{\pgfqpoint{2.591306in}{2.350722in}}{\pgfqpoint{2.580256in}{2.350722in}}%
\pgfpathcurveto{\pgfqpoint{2.569206in}{2.350722in}}{\pgfqpoint{2.558607in}{2.346332in}}{\pgfqpoint{2.550793in}{2.338519in}}%
\pgfpathcurveto{\pgfqpoint{2.542979in}{2.330705in}}{\pgfqpoint{2.538589in}{2.320106in}}{\pgfqpoint{2.538589in}{2.309056in}}%
\pgfpathcurveto{\pgfqpoint{2.538589in}{2.298006in}}{\pgfqpoint{2.542979in}{2.287407in}}{\pgfqpoint{2.550793in}{2.279593in}}%
\pgfpathcurveto{\pgfqpoint{2.558607in}{2.271779in}}{\pgfqpoint{2.569206in}{2.267389in}}{\pgfqpoint{2.580256in}{2.267389in}}%
\pgfpathclose%
\pgfusepath{stroke,fill}%
\end{pgfscope}%
\begin{pgfscope}%
\pgfpathrectangle{\pgfqpoint{0.375000in}{0.330000in}}{\pgfqpoint{2.325000in}{2.310000in}}%
\pgfusepath{clip}%
\pgfsetbuttcap%
\pgfsetroundjoin%
\definecolor{currentfill}{rgb}{0.000000,0.000000,0.000000}%
\pgfsetfillcolor{currentfill}%
\pgfsetlinewidth{1.003750pt}%
\definecolor{currentstroke}{rgb}{0.000000,0.000000,0.000000}%
\pgfsetstrokecolor{currentstroke}%
\pgfsetdash{}{0pt}%
\pgfpathmoveto{\pgfqpoint{2.580256in}{2.267389in}}%
\pgfpathcurveto{\pgfqpoint{2.591306in}{2.267389in}}{\pgfqpoint{2.601905in}{2.271779in}}{\pgfqpoint{2.609718in}{2.279593in}}%
\pgfpathcurveto{\pgfqpoint{2.617532in}{2.287407in}}{\pgfqpoint{2.621922in}{2.298006in}}{\pgfqpoint{2.621922in}{2.309056in}}%
\pgfpathcurveto{\pgfqpoint{2.621922in}{2.320106in}}{\pgfqpoint{2.617532in}{2.330705in}}{\pgfqpoint{2.609718in}{2.338519in}}%
\pgfpathcurveto{\pgfqpoint{2.601905in}{2.346332in}}{\pgfqpoint{2.591306in}{2.350722in}}{\pgfqpoint{2.580256in}{2.350722in}}%
\pgfpathcurveto{\pgfqpoint{2.569206in}{2.350722in}}{\pgfqpoint{2.558607in}{2.346332in}}{\pgfqpoint{2.550793in}{2.338519in}}%
\pgfpathcurveto{\pgfqpoint{2.542979in}{2.330705in}}{\pgfqpoint{2.538589in}{2.320106in}}{\pgfqpoint{2.538589in}{2.309056in}}%
\pgfpathcurveto{\pgfqpoint{2.538589in}{2.298006in}}{\pgfqpoint{2.542979in}{2.287407in}}{\pgfqpoint{2.550793in}{2.279593in}}%
\pgfpathcurveto{\pgfqpoint{2.558607in}{2.271779in}}{\pgfqpoint{2.569206in}{2.267389in}}{\pgfqpoint{2.580256in}{2.267389in}}%
\pgfpathclose%
\pgfusepath{stroke,fill}%
\end{pgfscope}%
\begin{pgfscope}%
\pgfpathrectangle{\pgfqpoint{0.375000in}{0.330000in}}{\pgfqpoint{2.325000in}{2.310000in}}%
\pgfusepath{clip}%
\pgfsetbuttcap%
\pgfsetroundjoin%
\definecolor{currentfill}{rgb}{0.000000,0.000000,0.000000}%
\pgfsetfillcolor{currentfill}%
\pgfsetlinewidth{1.003750pt}%
\definecolor{currentstroke}{rgb}{0.000000,0.000000,0.000000}%
\pgfsetstrokecolor{currentstroke}%
\pgfsetdash{}{0pt}%
\pgfpathmoveto{\pgfqpoint{2.580256in}{2.319420in}}%
\pgfpathcurveto{\pgfqpoint{2.591306in}{2.319420in}}{\pgfqpoint{2.601905in}{2.323810in}}{\pgfqpoint{2.609718in}{2.331624in}}%
\pgfpathcurveto{\pgfqpoint{2.617532in}{2.339437in}}{\pgfqpoint{2.621922in}{2.350037in}}{\pgfqpoint{2.621922in}{2.361087in}}%
\pgfpathcurveto{\pgfqpoint{2.621922in}{2.372137in}}{\pgfqpoint{2.617532in}{2.382736in}}{\pgfqpoint{2.609718in}{2.390549in}}%
\pgfpathcurveto{\pgfqpoint{2.601905in}{2.398363in}}{\pgfqpoint{2.591306in}{2.402753in}}{\pgfqpoint{2.580256in}{2.402753in}}%
\pgfpathcurveto{\pgfqpoint{2.569206in}{2.402753in}}{\pgfqpoint{2.558607in}{2.398363in}}{\pgfqpoint{2.550793in}{2.390549in}}%
\pgfpathcurveto{\pgfqpoint{2.542979in}{2.382736in}}{\pgfqpoint{2.538589in}{2.372137in}}{\pgfqpoint{2.538589in}{2.361087in}}%
\pgfpathcurveto{\pgfqpoint{2.538589in}{2.350037in}}{\pgfqpoint{2.542979in}{2.339437in}}{\pgfqpoint{2.550793in}{2.331624in}}%
\pgfpathcurveto{\pgfqpoint{2.558607in}{2.323810in}}{\pgfqpoint{2.569206in}{2.319420in}}{\pgfqpoint{2.580256in}{2.319420in}}%
\pgfpathclose%
\pgfusepath{stroke,fill}%
\end{pgfscope}%
\begin{pgfscope}%
\pgfpathrectangle{\pgfqpoint{0.375000in}{0.330000in}}{\pgfqpoint{2.325000in}{2.310000in}}%
\pgfusepath{clip}%
\pgfsetbuttcap%
\pgfsetroundjoin%
\definecolor{currentfill}{rgb}{0.000000,0.000000,0.000000}%
\pgfsetfillcolor{currentfill}%
\pgfsetlinewidth{1.003750pt}%
\definecolor{currentstroke}{rgb}{0.000000,0.000000,0.000000}%
\pgfsetstrokecolor{currentstroke}%
\pgfsetdash{}{0pt}%
\pgfpathmoveto{\pgfqpoint{2.580256in}{2.371451in}}%
\pgfpathcurveto{\pgfqpoint{2.591306in}{2.371451in}}{\pgfqpoint{2.601905in}{2.375841in}}{\pgfqpoint{2.609718in}{2.383655in}}%
\pgfpathcurveto{\pgfqpoint{2.617532in}{2.391468in}}{\pgfqpoint{2.621922in}{2.402067in}}{\pgfqpoint{2.621922in}{2.413117in}}%
\pgfpathcurveto{\pgfqpoint{2.621922in}{2.424168in}}{\pgfqpoint{2.617532in}{2.434767in}}{\pgfqpoint{2.609718in}{2.442580in}}%
\pgfpathcurveto{\pgfqpoint{2.601905in}{2.450394in}}{\pgfqpoint{2.591306in}{2.454784in}}{\pgfqpoint{2.580256in}{2.454784in}}%
\pgfpathcurveto{\pgfqpoint{2.569206in}{2.454784in}}{\pgfqpoint{2.558607in}{2.450394in}}{\pgfqpoint{2.550793in}{2.442580in}}%
\pgfpathcurveto{\pgfqpoint{2.542979in}{2.434767in}}{\pgfqpoint{2.538589in}{2.424168in}}{\pgfqpoint{2.538589in}{2.413117in}}%
\pgfpathcurveto{\pgfqpoint{2.538589in}{2.402067in}}{\pgfqpoint{2.542979in}{2.391468in}}{\pgfqpoint{2.550793in}{2.383655in}}%
\pgfpathcurveto{\pgfqpoint{2.558607in}{2.375841in}}{\pgfqpoint{2.569206in}{2.371451in}}{\pgfqpoint{2.580256in}{2.371451in}}%
\pgfpathclose%
\pgfusepath{stroke,fill}%
\end{pgfscope}%
\begin{pgfscope}%
\pgfpathrectangle{\pgfqpoint{0.375000in}{0.330000in}}{\pgfqpoint{2.325000in}{2.310000in}}%
\pgfusepath{clip}%
\pgfsetbuttcap%
\pgfsetroundjoin%
\definecolor{currentfill}{rgb}{0.000000,0.000000,0.000000}%
\pgfsetfillcolor{currentfill}%
\pgfsetlinewidth{1.003750pt}%
\definecolor{currentstroke}{rgb}{0.000000,0.000000,0.000000}%
\pgfsetstrokecolor{currentstroke}%
\pgfsetdash{}{0pt}%
\pgfpathmoveto{\pgfqpoint{2.580256in}{2.319420in}}%
\pgfpathcurveto{\pgfqpoint{2.591306in}{2.319420in}}{\pgfqpoint{2.601905in}{2.323810in}}{\pgfqpoint{2.609718in}{2.331624in}}%
\pgfpathcurveto{\pgfqpoint{2.617532in}{2.339437in}}{\pgfqpoint{2.621922in}{2.350037in}}{\pgfqpoint{2.621922in}{2.361087in}}%
\pgfpathcurveto{\pgfqpoint{2.621922in}{2.372137in}}{\pgfqpoint{2.617532in}{2.382736in}}{\pgfqpoint{2.609718in}{2.390549in}}%
\pgfpathcurveto{\pgfqpoint{2.601905in}{2.398363in}}{\pgfqpoint{2.591306in}{2.402753in}}{\pgfqpoint{2.580256in}{2.402753in}}%
\pgfpathcurveto{\pgfqpoint{2.569206in}{2.402753in}}{\pgfqpoint{2.558607in}{2.398363in}}{\pgfqpoint{2.550793in}{2.390549in}}%
\pgfpathcurveto{\pgfqpoint{2.542979in}{2.382736in}}{\pgfqpoint{2.538589in}{2.372137in}}{\pgfqpoint{2.538589in}{2.361087in}}%
\pgfpathcurveto{\pgfqpoint{2.538589in}{2.350037in}}{\pgfqpoint{2.542979in}{2.339437in}}{\pgfqpoint{2.550793in}{2.331624in}}%
\pgfpathcurveto{\pgfqpoint{2.558607in}{2.323810in}}{\pgfqpoint{2.569206in}{2.319420in}}{\pgfqpoint{2.580256in}{2.319420in}}%
\pgfpathclose%
\pgfusepath{stroke,fill}%
\end{pgfscope}%
\begin{pgfscope}%
\pgfpathrectangle{\pgfqpoint{0.375000in}{0.330000in}}{\pgfqpoint{2.325000in}{2.310000in}}%
\pgfusepath{clip}%
\pgfsetbuttcap%
\pgfsetroundjoin%
\definecolor{currentfill}{rgb}{0.000000,0.000000,0.000000}%
\pgfsetfillcolor{currentfill}%
\pgfsetlinewidth{1.003750pt}%
\definecolor{currentstroke}{rgb}{0.000000,0.000000,0.000000}%
\pgfsetstrokecolor{currentstroke}%
\pgfsetdash{}{0pt}%
\pgfpathmoveto{\pgfqpoint{2.580256in}{2.319420in}}%
\pgfpathcurveto{\pgfqpoint{2.591306in}{2.319420in}}{\pgfqpoint{2.601905in}{2.323810in}}{\pgfqpoint{2.609718in}{2.331624in}}%
\pgfpathcurveto{\pgfqpoint{2.617532in}{2.339437in}}{\pgfqpoint{2.621922in}{2.350037in}}{\pgfqpoint{2.621922in}{2.361087in}}%
\pgfpathcurveto{\pgfqpoint{2.621922in}{2.372137in}}{\pgfqpoint{2.617532in}{2.382736in}}{\pgfqpoint{2.609718in}{2.390549in}}%
\pgfpathcurveto{\pgfqpoint{2.601905in}{2.398363in}}{\pgfqpoint{2.591306in}{2.402753in}}{\pgfqpoint{2.580256in}{2.402753in}}%
\pgfpathcurveto{\pgfqpoint{2.569206in}{2.402753in}}{\pgfqpoint{2.558607in}{2.398363in}}{\pgfqpoint{2.550793in}{2.390549in}}%
\pgfpathcurveto{\pgfqpoint{2.542979in}{2.382736in}}{\pgfqpoint{2.538589in}{2.372137in}}{\pgfqpoint{2.538589in}{2.361087in}}%
\pgfpathcurveto{\pgfqpoint{2.538589in}{2.350037in}}{\pgfqpoint{2.542979in}{2.339437in}}{\pgfqpoint{2.550793in}{2.331624in}}%
\pgfpathcurveto{\pgfqpoint{2.558607in}{2.323810in}}{\pgfqpoint{2.569206in}{2.319420in}}{\pgfqpoint{2.580256in}{2.319420in}}%
\pgfpathclose%
\pgfusepath{stroke,fill}%
\end{pgfscope}%
\begin{pgfscope}%
\pgfpathrectangle{\pgfqpoint{0.375000in}{0.330000in}}{\pgfqpoint{2.325000in}{2.310000in}}%
\pgfusepath{clip}%
\pgfsetbuttcap%
\pgfsetroundjoin%
\definecolor{currentfill}{rgb}{0.000000,0.000000,0.000000}%
\pgfsetfillcolor{currentfill}%
\pgfsetlinewidth{1.003750pt}%
\definecolor{currentstroke}{rgb}{0.000000,0.000000,0.000000}%
\pgfsetstrokecolor{currentstroke}%
\pgfsetdash{}{0pt}%
\pgfpathmoveto{\pgfqpoint{2.580256in}{2.319420in}}%
\pgfpathcurveto{\pgfqpoint{2.591306in}{2.319420in}}{\pgfqpoint{2.601905in}{2.323810in}}{\pgfqpoint{2.609718in}{2.331624in}}%
\pgfpathcurveto{\pgfqpoint{2.617532in}{2.339437in}}{\pgfqpoint{2.621922in}{2.350037in}}{\pgfqpoint{2.621922in}{2.361087in}}%
\pgfpathcurveto{\pgfqpoint{2.621922in}{2.372137in}}{\pgfqpoint{2.617532in}{2.382736in}}{\pgfqpoint{2.609718in}{2.390549in}}%
\pgfpathcurveto{\pgfqpoint{2.601905in}{2.398363in}}{\pgfqpoint{2.591306in}{2.402753in}}{\pgfqpoint{2.580256in}{2.402753in}}%
\pgfpathcurveto{\pgfqpoint{2.569206in}{2.402753in}}{\pgfqpoint{2.558607in}{2.398363in}}{\pgfqpoint{2.550793in}{2.390549in}}%
\pgfpathcurveto{\pgfqpoint{2.542979in}{2.382736in}}{\pgfqpoint{2.538589in}{2.372137in}}{\pgfqpoint{2.538589in}{2.361087in}}%
\pgfpathcurveto{\pgfqpoint{2.538589in}{2.350037in}}{\pgfqpoint{2.542979in}{2.339437in}}{\pgfqpoint{2.550793in}{2.331624in}}%
\pgfpathcurveto{\pgfqpoint{2.558607in}{2.323810in}}{\pgfqpoint{2.569206in}{2.319420in}}{\pgfqpoint{2.580256in}{2.319420in}}%
\pgfpathclose%
\pgfusepath{stroke,fill}%
\end{pgfscope}%
\begin{pgfscope}%
\pgfpathrectangle{\pgfqpoint{0.375000in}{0.330000in}}{\pgfqpoint{2.325000in}{2.310000in}}%
\pgfusepath{clip}%
\pgfsetbuttcap%
\pgfsetroundjoin%
\definecolor{currentfill}{rgb}{0.000000,0.000000,0.000000}%
\pgfsetfillcolor{currentfill}%
\pgfsetlinewidth{1.003750pt}%
\definecolor{currentstroke}{rgb}{0.000000,0.000000,0.000000}%
\pgfsetstrokecolor{currentstroke}%
\pgfsetdash{}{0pt}%
\pgfpathmoveto{\pgfqpoint{2.580256in}{2.371451in}}%
\pgfpathcurveto{\pgfqpoint{2.591306in}{2.371451in}}{\pgfqpoint{2.601905in}{2.375841in}}{\pgfqpoint{2.609718in}{2.383655in}}%
\pgfpathcurveto{\pgfqpoint{2.617532in}{2.391468in}}{\pgfqpoint{2.621922in}{2.402067in}}{\pgfqpoint{2.621922in}{2.413117in}}%
\pgfpathcurveto{\pgfqpoint{2.621922in}{2.424168in}}{\pgfqpoint{2.617532in}{2.434767in}}{\pgfqpoint{2.609718in}{2.442580in}}%
\pgfpathcurveto{\pgfqpoint{2.601905in}{2.450394in}}{\pgfqpoint{2.591306in}{2.454784in}}{\pgfqpoint{2.580256in}{2.454784in}}%
\pgfpathcurveto{\pgfqpoint{2.569206in}{2.454784in}}{\pgfqpoint{2.558607in}{2.450394in}}{\pgfqpoint{2.550793in}{2.442580in}}%
\pgfpathcurveto{\pgfqpoint{2.542979in}{2.434767in}}{\pgfqpoint{2.538589in}{2.424168in}}{\pgfqpoint{2.538589in}{2.413117in}}%
\pgfpathcurveto{\pgfqpoint{2.538589in}{2.402067in}}{\pgfqpoint{2.542979in}{2.391468in}}{\pgfqpoint{2.550793in}{2.383655in}}%
\pgfpathcurveto{\pgfqpoint{2.558607in}{2.375841in}}{\pgfqpoint{2.569206in}{2.371451in}}{\pgfqpoint{2.580256in}{2.371451in}}%
\pgfpathclose%
\pgfusepath{stroke,fill}%
\end{pgfscope}%
\begin{pgfscope}%
\pgfpathrectangle{\pgfqpoint{0.375000in}{0.330000in}}{\pgfqpoint{2.325000in}{2.310000in}}%
\pgfusepath{clip}%
\pgfsetbuttcap%
\pgfsetroundjoin%
\definecolor{currentfill}{rgb}{0.000000,0.000000,0.000000}%
\pgfsetfillcolor{currentfill}%
\pgfsetlinewidth{1.003750pt}%
\definecolor{currentstroke}{rgb}{0.000000,0.000000,0.000000}%
\pgfsetstrokecolor{currentstroke}%
\pgfsetdash{}{0pt}%
\pgfpathmoveto{\pgfqpoint{2.580256in}{2.371451in}}%
\pgfpathcurveto{\pgfqpoint{2.591306in}{2.371451in}}{\pgfqpoint{2.601905in}{2.375841in}}{\pgfqpoint{2.609718in}{2.383655in}}%
\pgfpathcurveto{\pgfqpoint{2.617532in}{2.391468in}}{\pgfqpoint{2.621922in}{2.402067in}}{\pgfqpoint{2.621922in}{2.413117in}}%
\pgfpathcurveto{\pgfqpoint{2.621922in}{2.424168in}}{\pgfqpoint{2.617532in}{2.434767in}}{\pgfqpoint{2.609718in}{2.442580in}}%
\pgfpathcurveto{\pgfqpoint{2.601905in}{2.450394in}}{\pgfqpoint{2.591306in}{2.454784in}}{\pgfqpoint{2.580256in}{2.454784in}}%
\pgfpathcurveto{\pgfqpoint{2.569206in}{2.454784in}}{\pgfqpoint{2.558607in}{2.450394in}}{\pgfqpoint{2.550793in}{2.442580in}}%
\pgfpathcurveto{\pgfqpoint{2.542979in}{2.434767in}}{\pgfqpoint{2.538589in}{2.424168in}}{\pgfqpoint{2.538589in}{2.413117in}}%
\pgfpathcurveto{\pgfqpoint{2.538589in}{2.402067in}}{\pgfqpoint{2.542979in}{2.391468in}}{\pgfqpoint{2.550793in}{2.383655in}}%
\pgfpathcurveto{\pgfqpoint{2.558607in}{2.375841in}}{\pgfqpoint{2.569206in}{2.371451in}}{\pgfqpoint{2.580256in}{2.371451in}}%
\pgfpathclose%
\pgfusepath{stroke,fill}%
\end{pgfscope}%
\begin{pgfscope}%
\pgfpathrectangle{\pgfqpoint{0.375000in}{0.330000in}}{\pgfqpoint{2.325000in}{2.310000in}}%
\pgfusepath{clip}%
\pgfsetbuttcap%
\pgfsetroundjoin%
\definecolor{currentfill}{rgb}{0.000000,0.000000,0.000000}%
\pgfsetfillcolor{currentfill}%
\pgfsetlinewidth{1.003750pt}%
\definecolor{currentstroke}{rgb}{0.000000,0.000000,0.000000}%
\pgfsetstrokecolor{currentstroke}%
\pgfsetdash{}{0pt}%
\pgfpathmoveto{\pgfqpoint{2.580256in}{2.267389in}}%
\pgfpathcurveto{\pgfqpoint{2.591306in}{2.267389in}}{\pgfqpoint{2.601905in}{2.271779in}}{\pgfqpoint{2.609718in}{2.279593in}}%
\pgfpathcurveto{\pgfqpoint{2.617532in}{2.287407in}}{\pgfqpoint{2.621922in}{2.298006in}}{\pgfqpoint{2.621922in}{2.309056in}}%
\pgfpathcurveto{\pgfqpoint{2.621922in}{2.320106in}}{\pgfqpoint{2.617532in}{2.330705in}}{\pgfqpoint{2.609718in}{2.338519in}}%
\pgfpathcurveto{\pgfqpoint{2.601905in}{2.346332in}}{\pgfqpoint{2.591306in}{2.350722in}}{\pgfqpoint{2.580256in}{2.350722in}}%
\pgfpathcurveto{\pgfqpoint{2.569206in}{2.350722in}}{\pgfqpoint{2.558607in}{2.346332in}}{\pgfqpoint{2.550793in}{2.338519in}}%
\pgfpathcurveto{\pgfqpoint{2.542979in}{2.330705in}}{\pgfqpoint{2.538589in}{2.320106in}}{\pgfqpoint{2.538589in}{2.309056in}}%
\pgfpathcurveto{\pgfqpoint{2.538589in}{2.298006in}}{\pgfqpoint{2.542979in}{2.287407in}}{\pgfqpoint{2.550793in}{2.279593in}}%
\pgfpathcurveto{\pgfqpoint{2.558607in}{2.271779in}}{\pgfqpoint{2.569206in}{2.267389in}}{\pgfqpoint{2.580256in}{2.267389in}}%
\pgfpathclose%
\pgfusepath{stroke,fill}%
\end{pgfscope}%
\begin{pgfscope}%
\pgfpathrectangle{\pgfqpoint{0.375000in}{0.330000in}}{\pgfqpoint{2.325000in}{2.310000in}}%
\pgfusepath{clip}%
\pgfsetbuttcap%
\pgfsetroundjoin%
\definecolor{currentfill}{rgb}{0.000000,0.000000,0.000000}%
\pgfsetfillcolor{currentfill}%
\pgfsetlinewidth{1.003750pt}%
\definecolor{currentstroke}{rgb}{0.000000,0.000000,0.000000}%
\pgfsetstrokecolor{currentstroke}%
\pgfsetdash{}{0pt}%
\pgfpathmoveto{\pgfqpoint{2.580256in}{2.267389in}}%
\pgfpathcurveto{\pgfqpoint{2.591306in}{2.267389in}}{\pgfqpoint{2.601905in}{2.271779in}}{\pgfqpoint{2.609718in}{2.279593in}}%
\pgfpathcurveto{\pgfqpoint{2.617532in}{2.287407in}}{\pgfqpoint{2.621922in}{2.298006in}}{\pgfqpoint{2.621922in}{2.309056in}}%
\pgfpathcurveto{\pgfqpoint{2.621922in}{2.320106in}}{\pgfqpoint{2.617532in}{2.330705in}}{\pgfqpoint{2.609718in}{2.338519in}}%
\pgfpathcurveto{\pgfqpoint{2.601905in}{2.346332in}}{\pgfqpoint{2.591306in}{2.350722in}}{\pgfqpoint{2.580256in}{2.350722in}}%
\pgfpathcurveto{\pgfqpoint{2.569206in}{2.350722in}}{\pgfqpoint{2.558607in}{2.346332in}}{\pgfqpoint{2.550793in}{2.338519in}}%
\pgfpathcurveto{\pgfqpoint{2.542979in}{2.330705in}}{\pgfqpoint{2.538589in}{2.320106in}}{\pgfqpoint{2.538589in}{2.309056in}}%
\pgfpathcurveto{\pgfqpoint{2.538589in}{2.298006in}}{\pgfqpoint{2.542979in}{2.287407in}}{\pgfqpoint{2.550793in}{2.279593in}}%
\pgfpathcurveto{\pgfqpoint{2.558607in}{2.271779in}}{\pgfqpoint{2.569206in}{2.267389in}}{\pgfqpoint{2.580256in}{2.267389in}}%
\pgfpathclose%
\pgfusepath{stroke,fill}%
\end{pgfscope}%
\begin{pgfscope}%
\pgfpathrectangle{\pgfqpoint{0.375000in}{0.330000in}}{\pgfqpoint{2.325000in}{2.310000in}}%
\pgfusepath{clip}%
\pgfsetbuttcap%
\pgfsetroundjoin%
\definecolor{currentfill}{rgb}{0.000000,0.000000,0.000000}%
\pgfsetfillcolor{currentfill}%
\pgfsetlinewidth{1.003750pt}%
\definecolor{currentstroke}{rgb}{0.000000,0.000000,0.000000}%
\pgfsetstrokecolor{currentstroke}%
\pgfsetdash{}{0pt}%
\pgfpathmoveto{\pgfqpoint{2.580256in}{2.267389in}}%
\pgfpathcurveto{\pgfqpoint{2.591306in}{2.267389in}}{\pgfqpoint{2.601905in}{2.271779in}}{\pgfqpoint{2.609718in}{2.279593in}}%
\pgfpathcurveto{\pgfqpoint{2.617532in}{2.287407in}}{\pgfqpoint{2.621922in}{2.298006in}}{\pgfqpoint{2.621922in}{2.309056in}}%
\pgfpathcurveto{\pgfqpoint{2.621922in}{2.320106in}}{\pgfqpoint{2.617532in}{2.330705in}}{\pgfqpoint{2.609718in}{2.338519in}}%
\pgfpathcurveto{\pgfqpoint{2.601905in}{2.346332in}}{\pgfqpoint{2.591306in}{2.350722in}}{\pgfqpoint{2.580256in}{2.350722in}}%
\pgfpathcurveto{\pgfqpoint{2.569206in}{2.350722in}}{\pgfqpoint{2.558607in}{2.346332in}}{\pgfqpoint{2.550793in}{2.338519in}}%
\pgfpathcurveto{\pgfqpoint{2.542979in}{2.330705in}}{\pgfqpoint{2.538589in}{2.320106in}}{\pgfqpoint{2.538589in}{2.309056in}}%
\pgfpathcurveto{\pgfqpoint{2.538589in}{2.298006in}}{\pgfqpoint{2.542979in}{2.287407in}}{\pgfqpoint{2.550793in}{2.279593in}}%
\pgfpathcurveto{\pgfqpoint{2.558607in}{2.271779in}}{\pgfqpoint{2.569206in}{2.267389in}}{\pgfqpoint{2.580256in}{2.267389in}}%
\pgfpathclose%
\pgfusepath{stroke,fill}%
\end{pgfscope}%
\begin{pgfscope}%
\pgfpathrectangle{\pgfqpoint{0.375000in}{0.330000in}}{\pgfqpoint{2.325000in}{2.310000in}}%
\pgfusepath{clip}%
\pgfsetbuttcap%
\pgfsetroundjoin%
\definecolor{currentfill}{rgb}{0.000000,0.000000,0.000000}%
\pgfsetfillcolor{currentfill}%
\pgfsetlinewidth{1.003750pt}%
\definecolor{currentstroke}{rgb}{0.000000,0.000000,0.000000}%
\pgfsetstrokecolor{currentstroke}%
\pgfsetdash{}{0pt}%
\pgfpathmoveto{\pgfqpoint{2.580256in}{2.267389in}}%
\pgfpathcurveto{\pgfqpoint{2.591306in}{2.267389in}}{\pgfqpoint{2.601905in}{2.271779in}}{\pgfqpoint{2.609718in}{2.279593in}}%
\pgfpathcurveto{\pgfqpoint{2.617532in}{2.287407in}}{\pgfqpoint{2.621922in}{2.298006in}}{\pgfqpoint{2.621922in}{2.309056in}}%
\pgfpathcurveto{\pgfqpoint{2.621922in}{2.320106in}}{\pgfqpoint{2.617532in}{2.330705in}}{\pgfqpoint{2.609718in}{2.338519in}}%
\pgfpathcurveto{\pgfqpoint{2.601905in}{2.346332in}}{\pgfqpoint{2.591306in}{2.350722in}}{\pgfqpoint{2.580256in}{2.350722in}}%
\pgfpathcurveto{\pgfqpoint{2.569206in}{2.350722in}}{\pgfqpoint{2.558607in}{2.346332in}}{\pgfqpoint{2.550793in}{2.338519in}}%
\pgfpathcurveto{\pgfqpoint{2.542979in}{2.330705in}}{\pgfqpoint{2.538589in}{2.320106in}}{\pgfqpoint{2.538589in}{2.309056in}}%
\pgfpathcurveto{\pgfqpoint{2.538589in}{2.298006in}}{\pgfqpoint{2.542979in}{2.287407in}}{\pgfqpoint{2.550793in}{2.279593in}}%
\pgfpathcurveto{\pgfqpoint{2.558607in}{2.271779in}}{\pgfqpoint{2.569206in}{2.267389in}}{\pgfqpoint{2.580256in}{2.267389in}}%
\pgfpathclose%
\pgfusepath{stroke,fill}%
\end{pgfscope}%
\begin{pgfscope}%
\pgfpathrectangle{\pgfqpoint{0.375000in}{0.330000in}}{\pgfqpoint{2.325000in}{2.310000in}}%
\pgfusepath{clip}%
\pgfsetbuttcap%
\pgfsetroundjoin%
\definecolor{currentfill}{rgb}{0.000000,0.000000,0.000000}%
\pgfsetfillcolor{currentfill}%
\pgfsetlinewidth{1.003750pt}%
\definecolor{currentstroke}{rgb}{0.000000,0.000000,0.000000}%
\pgfsetstrokecolor{currentstroke}%
\pgfsetdash{}{0pt}%
\pgfpathmoveto{\pgfqpoint{2.580256in}{2.319420in}}%
\pgfpathcurveto{\pgfqpoint{2.591306in}{2.319420in}}{\pgfqpoint{2.601905in}{2.323810in}}{\pgfqpoint{2.609718in}{2.331624in}}%
\pgfpathcurveto{\pgfqpoint{2.617532in}{2.339437in}}{\pgfqpoint{2.621922in}{2.350037in}}{\pgfqpoint{2.621922in}{2.361087in}}%
\pgfpathcurveto{\pgfqpoint{2.621922in}{2.372137in}}{\pgfqpoint{2.617532in}{2.382736in}}{\pgfqpoint{2.609718in}{2.390549in}}%
\pgfpathcurveto{\pgfqpoint{2.601905in}{2.398363in}}{\pgfqpoint{2.591306in}{2.402753in}}{\pgfqpoint{2.580256in}{2.402753in}}%
\pgfpathcurveto{\pgfqpoint{2.569206in}{2.402753in}}{\pgfqpoint{2.558607in}{2.398363in}}{\pgfqpoint{2.550793in}{2.390549in}}%
\pgfpathcurveto{\pgfqpoint{2.542979in}{2.382736in}}{\pgfqpoint{2.538589in}{2.372137in}}{\pgfqpoint{2.538589in}{2.361087in}}%
\pgfpathcurveto{\pgfqpoint{2.538589in}{2.350037in}}{\pgfqpoint{2.542979in}{2.339437in}}{\pgfqpoint{2.550793in}{2.331624in}}%
\pgfpathcurveto{\pgfqpoint{2.558607in}{2.323810in}}{\pgfqpoint{2.569206in}{2.319420in}}{\pgfqpoint{2.580256in}{2.319420in}}%
\pgfpathclose%
\pgfusepath{stroke,fill}%
\end{pgfscope}%
\begin{pgfscope}%
\pgfpathrectangle{\pgfqpoint{0.375000in}{0.330000in}}{\pgfqpoint{2.325000in}{2.310000in}}%
\pgfusepath{clip}%
\pgfsetbuttcap%
\pgfsetroundjoin%
\definecolor{currentfill}{rgb}{0.000000,0.000000,0.000000}%
\pgfsetfillcolor{currentfill}%
\pgfsetlinewidth{1.003750pt}%
\definecolor{currentstroke}{rgb}{0.000000,0.000000,0.000000}%
\pgfsetstrokecolor{currentstroke}%
\pgfsetdash{}{0pt}%
\pgfpathmoveto{\pgfqpoint{2.580256in}{2.319420in}}%
\pgfpathcurveto{\pgfqpoint{2.591306in}{2.319420in}}{\pgfqpoint{2.601905in}{2.323810in}}{\pgfqpoint{2.609718in}{2.331624in}}%
\pgfpathcurveto{\pgfqpoint{2.617532in}{2.339437in}}{\pgfqpoint{2.621922in}{2.350037in}}{\pgfqpoint{2.621922in}{2.361087in}}%
\pgfpathcurveto{\pgfqpoint{2.621922in}{2.372137in}}{\pgfqpoint{2.617532in}{2.382736in}}{\pgfqpoint{2.609718in}{2.390549in}}%
\pgfpathcurveto{\pgfqpoint{2.601905in}{2.398363in}}{\pgfqpoint{2.591306in}{2.402753in}}{\pgfqpoint{2.580256in}{2.402753in}}%
\pgfpathcurveto{\pgfqpoint{2.569206in}{2.402753in}}{\pgfqpoint{2.558607in}{2.398363in}}{\pgfqpoint{2.550793in}{2.390549in}}%
\pgfpathcurveto{\pgfqpoint{2.542979in}{2.382736in}}{\pgfqpoint{2.538589in}{2.372137in}}{\pgfqpoint{2.538589in}{2.361087in}}%
\pgfpathcurveto{\pgfqpoint{2.538589in}{2.350037in}}{\pgfqpoint{2.542979in}{2.339437in}}{\pgfqpoint{2.550793in}{2.331624in}}%
\pgfpathcurveto{\pgfqpoint{2.558607in}{2.323810in}}{\pgfqpoint{2.569206in}{2.319420in}}{\pgfqpoint{2.580256in}{2.319420in}}%
\pgfpathclose%
\pgfusepath{stroke,fill}%
\end{pgfscope}%
\begin{pgfscope}%
\pgfpathrectangle{\pgfqpoint{0.375000in}{0.330000in}}{\pgfqpoint{2.325000in}{2.310000in}}%
\pgfusepath{clip}%
\pgfsetbuttcap%
\pgfsetroundjoin%
\definecolor{currentfill}{rgb}{0.000000,0.000000,0.000000}%
\pgfsetfillcolor{currentfill}%
\pgfsetlinewidth{1.003750pt}%
\definecolor{currentstroke}{rgb}{0.000000,0.000000,0.000000}%
\pgfsetstrokecolor{currentstroke}%
\pgfsetdash{}{0pt}%
\pgfpathmoveto{\pgfqpoint{2.580256in}{2.319420in}}%
\pgfpathcurveto{\pgfqpoint{2.591306in}{2.319420in}}{\pgfqpoint{2.601905in}{2.323810in}}{\pgfqpoint{2.609718in}{2.331624in}}%
\pgfpathcurveto{\pgfqpoint{2.617532in}{2.339437in}}{\pgfqpoint{2.621922in}{2.350037in}}{\pgfqpoint{2.621922in}{2.361087in}}%
\pgfpathcurveto{\pgfqpoint{2.621922in}{2.372137in}}{\pgfqpoint{2.617532in}{2.382736in}}{\pgfqpoint{2.609718in}{2.390549in}}%
\pgfpathcurveto{\pgfqpoint{2.601905in}{2.398363in}}{\pgfqpoint{2.591306in}{2.402753in}}{\pgfqpoint{2.580256in}{2.402753in}}%
\pgfpathcurveto{\pgfqpoint{2.569206in}{2.402753in}}{\pgfqpoint{2.558607in}{2.398363in}}{\pgfqpoint{2.550793in}{2.390549in}}%
\pgfpathcurveto{\pgfqpoint{2.542979in}{2.382736in}}{\pgfqpoint{2.538589in}{2.372137in}}{\pgfqpoint{2.538589in}{2.361087in}}%
\pgfpathcurveto{\pgfqpoint{2.538589in}{2.350037in}}{\pgfqpoint{2.542979in}{2.339437in}}{\pgfqpoint{2.550793in}{2.331624in}}%
\pgfpathcurveto{\pgfqpoint{2.558607in}{2.323810in}}{\pgfqpoint{2.569206in}{2.319420in}}{\pgfqpoint{2.580256in}{2.319420in}}%
\pgfpathclose%
\pgfusepath{stroke,fill}%
\end{pgfscope}%
\begin{pgfscope}%
\pgfpathrectangle{\pgfqpoint{0.375000in}{0.330000in}}{\pgfqpoint{2.325000in}{2.310000in}}%
\pgfusepath{clip}%
\pgfsetbuttcap%
\pgfsetroundjoin%
\definecolor{currentfill}{rgb}{0.000000,0.000000,0.000000}%
\pgfsetfillcolor{currentfill}%
\pgfsetlinewidth{1.003750pt}%
\definecolor{currentstroke}{rgb}{0.000000,0.000000,0.000000}%
\pgfsetstrokecolor{currentstroke}%
\pgfsetdash{}{0pt}%
\pgfpathmoveto{\pgfqpoint{2.580256in}{2.319420in}}%
\pgfpathcurveto{\pgfqpoint{2.591306in}{2.319420in}}{\pgfqpoint{2.601905in}{2.323810in}}{\pgfqpoint{2.609718in}{2.331624in}}%
\pgfpathcurveto{\pgfqpoint{2.617532in}{2.339437in}}{\pgfqpoint{2.621922in}{2.350037in}}{\pgfqpoint{2.621922in}{2.361087in}}%
\pgfpathcurveto{\pgfqpoint{2.621922in}{2.372137in}}{\pgfqpoint{2.617532in}{2.382736in}}{\pgfqpoint{2.609718in}{2.390549in}}%
\pgfpathcurveto{\pgfqpoint{2.601905in}{2.398363in}}{\pgfqpoint{2.591306in}{2.402753in}}{\pgfqpoint{2.580256in}{2.402753in}}%
\pgfpathcurveto{\pgfqpoint{2.569206in}{2.402753in}}{\pgfqpoint{2.558607in}{2.398363in}}{\pgfqpoint{2.550793in}{2.390549in}}%
\pgfpathcurveto{\pgfqpoint{2.542979in}{2.382736in}}{\pgfqpoint{2.538589in}{2.372137in}}{\pgfqpoint{2.538589in}{2.361087in}}%
\pgfpathcurveto{\pgfqpoint{2.538589in}{2.350037in}}{\pgfqpoint{2.542979in}{2.339437in}}{\pgfqpoint{2.550793in}{2.331624in}}%
\pgfpathcurveto{\pgfqpoint{2.558607in}{2.323810in}}{\pgfqpoint{2.569206in}{2.319420in}}{\pgfqpoint{2.580256in}{2.319420in}}%
\pgfpathclose%
\pgfusepath{stroke,fill}%
\end{pgfscope}%
\begin{pgfscope}%
\pgfpathrectangle{\pgfqpoint{0.375000in}{0.330000in}}{\pgfqpoint{2.325000in}{2.310000in}}%
\pgfusepath{clip}%
\pgfsetbuttcap%
\pgfsetroundjoin%
\definecolor{currentfill}{rgb}{0.000000,0.000000,0.000000}%
\pgfsetfillcolor{currentfill}%
\pgfsetlinewidth{1.003750pt}%
\definecolor{currentstroke}{rgb}{0.000000,0.000000,0.000000}%
\pgfsetstrokecolor{currentstroke}%
\pgfsetdash{}{0pt}%
\pgfpathmoveto{\pgfqpoint{2.580256in}{2.319420in}}%
\pgfpathcurveto{\pgfqpoint{2.591306in}{2.319420in}}{\pgfqpoint{2.601905in}{2.323810in}}{\pgfqpoint{2.609718in}{2.331624in}}%
\pgfpathcurveto{\pgfqpoint{2.617532in}{2.339437in}}{\pgfqpoint{2.621922in}{2.350037in}}{\pgfqpoint{2.621922in}{2.361087in}}%
\pgfpathcurveto{\pgfqpoint{2.621922in}{2.372137in}}{\pgfqpoint{2.617532in}{2.382736in}}{\pgfqpoint{2.609718in}{2.390549in}}%
\pgfpathcurveto{\pgfqpoint{2.601905in}{2.398363in}}{\pgfqpoint{2.591306in}{2.402753in}}{\pgfqpoint{2.580256in}{2.402753in}}%
\pgfpathcurveto{\pgfqpoint{2.569206in}{2.402753in}}{\pgfqpoint{2.558607in}{2.398363in}}{\pgfqpoint{2.550793in}{2.390549in}}%
\pgfpathcurveto{\pgfqpoint{2.542979in}{2.382736in}}{\pgfqpoint{2.538589in}{2.372137in}}{\pgfqpoint{2.538589in}{2.361087in}}%
\pgfpathcurveto{\pgfqpoint{2.538589in}{2.350037in}}{\pgfqpoint{2.542979in}{2.339437in}}{\pgfqpoint{2.550793in}{2.331624in}}%
\pgfpathcurveto{\pgfqpoint{2.558607in}{2.323810in}}{\pgfqpoint{2.569206in}{2.319420in}}{\pgfqpoint{2.580256in}{2.319420in}}%
\pgfpathclose%
\pgfusepath{stroke,fill}%
\end{pgfscope}%
\begin{pgfscope}%
\pgfpathrectangle{\pgfqpoint{0.375000in}{0.330000in}}{\pgfqpoint{2.325000in}{2.310000in}}%
\pgfusepath{clip}%
\pgfsetbuttcap%
\pgfsetroundjoin%
\definecolor{currentfill}{rgb}{0.000000,0.000000,0.000000}%
\pgfsetfillcolor{currentfill}%
\pgfsetlinewidth{1.003750pt}%
\definecolor{currentstroke}{rgb}{0.000000,0.000000,0.000000}%
\pgfsetstrokecolor{currentstroke}%
\pgfsetdash{}{0pt}%
\pgfpathmoveto{\pgfqpoint{2.580256in}{2.267389in}}%
\pgfpathcurveto{\pgfqpoint{2.591306in}{2.267389in}}{\pgfqpoint{2.601905in}{2.271779in}}{\pgfqpoint{2.609718in}{2.279593in}}%
\pgfpathcurveto{\pgfqpoint{2.617532in}{2.287407in}}{\pgfqpoint{2.621922in}{2.298006in}}{\pgfqpoint{2.621922in}{2.309056in}}%
\pgfpathcurveto{\pgfqpoint{2.621922in}{2.320106in}}{\pgfqpoint{2.617532in}{2.330705in}}{\pgfqpoint{2.609718in}{2.338519in}}%
\pgfpathcurveto{\pgfqpoint{2.601905in}{2.346332in}}{\pgfqpoint{2.591306in}{2.350722in}}{\pgfqpoint{2.580256in}{2.350722in}}%
\pgfpathcurveto{\pgfqpoint{2.569206in}{2.350722in}}{\pgfqpoint{2.558607in}{2.346332in}}{\pgfqpoint{2.550793in}{2.338519in}}%
\pgfpathcurveto{\pgfqpoint{2.542979in}{2.330705in}}{\pgfqpoint{2.538589in}{2.320106in}}{\pgfqpoint{2.538589in}{2.309056in}}%
\pgfpathcurveto{\pgfqpoint{2.538589in}{2.298006in}}{\pgfqpoint{2.542979in}{2.287407in}}{\pgfqpoint{2.550793in}{2.279593in}}%
\pgfpathcurveto{\pgfqpoint{2.558607in}{2.271779in}}{\pgfqpoint{2.569206in}{2.267389in}}{\pgfqpoint{2.580256in}{2.267389in}}%
\pgfpathclose%
\pgfusepath{stroke,fill}%
\end{pgfscope}%
\begin{pgfscope}%
\pgfpathrectangle{\pgfqpoint{0.375000in}{0.330000in}}{\pgfqpoint{2.325000in}{2.310000in}}%
\pgfusepath{clip}%
\pgfsetbuttcap%
\pgfsetroundjoin%
\definecolor{currentfill}{rgb}{0.000000,0.000000,0.000000}%
\pgfsetfillcolor{currentfill}%
\pgfsetlinewidth{1.003750pt}%
\definecolor{currentstroke}{rgb}{0.000000,0.000000,0.000000}%
\pgfsetstrokecolor{currentstroke}%
\pgfsetdash{}{0pt}%
\pgfpathmoveto{\pgfqpoint{2.580256in}{2.371451in}}%
\pgfpathcurveto{\pgfqpoint{2.591306in}{2.371451in}}{\pgfqpoint{2.601905in}{2.375841in}}{\pgfqpoint{2.609718in}{2.383655in}}%
\pgfpathcurveto{\pgfqpoint{2.617532in}{2.391468in}}{\pgfqpoint{2.621922in}{2.402067in}}{\pgfqpoint{2.621922in}{2.413117in}}%
\pgfpathcurveto{\pgfqpoint{2.621922in}{2.424168in}}{\pgfqpoint{2.617532in}{2.434767in}}{\pgfqpoint{2.609718in}{2.442580in}}%
\pgfpathcurveto{\pgfqpoint{2.601905in}{2.450394in}}{\pgfqpoint{2.591306in}{2.454784in}}{\pgfqpoint{2.580256in}{2.454784in}}%
\pgfpathcurveto{\pgfqpoint{2.569206in}{2.454784in}}{\pgfqpoint{2.558607in}{2.450394in}}{\pgfqpoint{2.550793in}{2.442580in}}%
\pgfpathcurveto{\pgfqpoint{2.542979in}{2.434767in}}{\pgfqpoint{2.538589in}{2.424168in}}{\pgfqpoint{2.538589in}{2.413117in}}%
\pgfpathcurveto{\pgfqpoint{2.538589in}{2.402067in}}{\pgfqpoint{2.542979in}{2.391468in}}{\pgfqpoint{2.550793in}{2.383655in}}%
\pgfpathcurveto{\pgfqpoint{2.558607in}{2.375841in}}{\pgfqpoint{2.569206in}{2.371451in}}{\pgfqpoint{2.580256in}{2.371451in}}%
\pgfpathclose%
\pgfusepath{stroke,fill}%
\end{pgfscope}%
\begin{pgfscope}%
\pgfpathrectangle{\pgfqpoint{0.375000in}{0.330000in}}{\pgfqpoint{2.325000in}{2.310000in}}%
\pgfusepath{clip}%
\pgfsetbuttcap%
\pgfsetroundjoin%
\definecolor{currentfill}{rgb}{0.000000,0.000000,0.000000}%
\pgfsetfillcolor{currentfill}%
\pgfsetlinewidth{1.003750pt}%
\definecolor{currentstroke}{rgb}{0.000000,0.000000,0.000000}%
\pgfsetstrokecolor{currentstroke}%
\pgfsetdash{}{0pt}%
\pgfpathmoveto{\pgfqpoint{2.580256in}{2.267389in}}%
\pgfpathcurveto{\pgfqpoint{2.591306in}{2.267389in}}{\pgfqpoint{2.601905in}{2.271779in}}{\pgfqpoint{2.609718in}{2.279593in}}%
\pgfpathcurveto{\pgfqpoint{2.617532in}{2.287407in}}{\pgfqpoint{2.621922in}{2.298006in}}{\pgfqpoint{2.621922in}{2.309056in}}%
\pgfpathcurveto{\pgfqpoint{2.621922in}{2.320106in}}{\pgfqpoint{2.617532in}{2.330705in}}{\pgfqpoint{2.609718in}{2.338519in}}%
\pgfpathcurveto{\pgfqpoint{2.601905in}{2.346332in}}{\pgfqpoint{2.591306in}{2.350722in}}{\pgfqpoint{2.580256in}{2.350722in}}%
\pgfpathcurveto{\pgfqpoint{2.569206in}{2.350722in}}{\pgfqpoint{2.558607in}{2.346332in}}{\pgfqpoint{2.550793in}{2.338519in}}%
\pgfpathcurveto{\pgfqpoint{2.542979in}{2.330705in}}{\pgfqpoint{2.538589in}{2.320106in}}{\pgfqpoint{2.538589in}{2.309056in}}%
\pgfpathcurveto{\pgfqpoint{2.538589in}{2.298006in}}{\pgfqpoint{2.542979in}{2.287407in}}{\pgfqpoint{2.550793in}{2.279593in}}%
\pgfpathcurveto{\pgfqpoint{2.558607in}{2.271779in}}{\pgfqpoint{2.569206in}{2.267389in}}{\pgfqpoint{2.580256in}{2.267389in}}%
\pgfpathclose%
\pgfusepath{stroke,fill}%
\end{pgfscope}%
\begin{pgfscope}%
\pgfpathrectangle{\pgfqpoint{0.375000in}{0.330000in}}{\pgfqpoint{2.325000in}{2.310000in}}%
\pgfusepath{clip}%
\pgfsetbuttcap%
\pgfsetroundjoin%
\definecolor{currentfill}{rgb}{0.000000,0.000000,0.000000}%
\pgfsetfillcolor{currentfill}%
\pgfsetlinewidth{1.003750pt}%
\definecolor{currentstroke}{rgb}{0.000000,0.000000,0.000000}%
\pgfsetstrokecolor{currentstroke}%
\pgfsetdash{}{0pt}%
\pgfpathmoveto{\pgfqpoint{2.580256in}{2.319420in}}%
\pgfpathcurveto{\pgfqpoint{2.591306in}{2.319420in}}{\pgfqpoint{2.601905in}{2.323810in}}{\pgfqpoint{2.609718in}{2.331624in}}%
\pgfpathcurveto{\pgfqpoint{2.617532in}{2.339437in}}{\pgfqpoint{2.621922in}{2.350037in}}{\pgfqpoint{2.621922in}{2.361087in}}%
\pgfpathcurveto{\pgfqpoint{2.621922in}{2.372137in}}{\pgfqpoint{2.617532in}{2.382736in}}{\pgfqpoint{2.609718in}{2.390549in}}%
\pgfpathcurveto{\pgfqpoint{2.601905in}{2.398363in}}{\pgfqpoint{2.591306in}{2.402753in}}{\pgfqpoint{2.580256in}{2.402753in}}%
\pgfpathcurveto{\pgfqpoint{2.569206in}{2.402753in}}{\pgfqpoint{2.558607in}{2.398363in}}{\pgfqpoint{2.550793in}{2.390549in}}%
\pgfpathcurveto{\pgfqpoint{2.542979in}{2.382736in}}{\pgfqpoint{2.538589in}{2.372137in}}{\pgfqpoint{2.538589in}{2.361087in}}%
\pgfpathcurveto{\pgfqpoint{2.538589in}{2.350037in}}{\pgfqpoint{2.542979in}{2.339437in}}{\pgfqpoint{2.550793in}{2.331624in}}%
\pgfpathcurveto{\pgfqpoint{2.558607in}{2.323810in}}{\pgfqpoint{2.569206in}{2.319420in}}{\pgfqpoint{2.580256in}{2.319420in}}%
\pgfpathclose%
\pgfusepath{stroke,fill}%
\end{pgfscope}%
\begin{pgfscope}%
\pgfpathrectangle{\pgfqpoint{0.375000in}{0.330000in}}{\pgfqpoint{2.325000in}{2.310000in}}%
\pgfusepath{clip}%
\pgfsetbuttcap%
\pgfsetroundjoin%
\definecolor{currentfill}{rgb}{0.000000,0.000000,0.000000}%
\pgfsetfillcolor{currentfill}%
\pgfsetlinewidth{1.003750pt}%
\definecolor{currentstroke}{rgb}{0.000000,0.000000,0.000000}%
\pgfsetstrokecolor{currentstroke}%
\pgfsetdash{}{0pt}%
\pgfpathmoveto{\pgfqpoint{2.580256in}{2.319420in}}%
\pgfpathcurveto{\pgfqpoint{2.591306in}{2.319420in}}{\pgfqpoint{2.601905in}{2.323810in}}{\pgfqpoint{2.609718in}{2.331624in}}%
\pgfpathcurveto{\pgfqpoint{2.617532in}{2.339437in}}{\pgfqpoint{2.621922in}{2.350037in}}{\pgfqpoint{2.621922in}{2.361087in}}%
\pgfpathcurveto{\pgfqpoint{2.621922in}{2.372137in}}{\pgfqpoint{2.617532in}{2.382736in}}{\pgfqpoint{2.609718in}{2.390549in}}%
\pgfpathcurveto{\pgfqpoint{2.601905in}{2.398363in}}{\pgfqpoint{2.591306in}{2.402753in}}{\pgfqpoint{2.580256in}{2.402753in}}%
\pgfpathcurveto{\pgfqpoint{2.569206in}{2.402753in}}{\pgfqpoint{2.558607in}{2.398363in}}{\pgfqpoint{2.550793in}{2.390549in}}%
\pgfpathcurveto{\pgfqpoint{2.542979in}{2.382736in}}{\pgfqpoint{2.538589in}{2.372137in}}{\pgfqpoint{2.538589in}{2.361087in}}%
\pgfpathcurveto{\pgfqpoint{2.538589in}{2.350037in}}{\pgfqpoint{2.542979in}{2.339437in}}{\pgfqpoint{2.550793in}{2.331624in}}%
\pgfpathcurveto{\pgfqpoint{2.558607in}{2.323810in}}{\pgfqpoint{2.569206in}{2.319420in}}{\pgfqpoint{2.580256in}{2.319420in}}%
\pgfpathclose%
\pgfusepath{stroke,fill}%
\end{pgfscope}%
\begin{pgfscope}%
\pgfpathrectangle{\pgfqpoint{0.375000in}{0.330000in}}{\pgfqpoint{2.325000in}{2.310000in}}%
\pgfusepath{clip}%
\pgfsetbuttcap%
\pgfsetroundjoin%
\definecolor{currentfill}{rgb}{0.000000,0.000000,0.000000}%
\pgfsetfillcolor{currentfill}%
\pgfsetlinewidth{1.003750pt}%
\definecolor{currentstroke}{rgb}{0.000000,0.000000,0.000000}%
\pgfsetstrokecolor{currentstroke}%
\pgfsetdash{}{0pt}%
\pgfpathmoveto{\pgfqpoint{2.580256in}{2.319420in}}%
\pgfpathcurveto{\pgfqpoint{2.591306in}{2.319420in}}{\pgfqpoint{2.601905in}{2.323810in}}{\pgfqpoint{2.609718in}{2.331624in}}%
\pgfpathcurveto{\pgfqpoint{2.617532in}{2.339437in}}{\pgfqpoint{2.621922in}{2.350037in}}{\pgfqpoint{2.621922in}{2.361087in}}%
\pgfpathcurveto{\pgfqpoint{2.621922in}{2.372137in}}{\pgfqpoint{2.617532in}{2.382736in}}{\pgfqpoint{2.609718in}{2.390549in}}%
\pgfpathcurveto{\pgfqpoint{2.601905in}{2.398363in}}{\pgfqpoint{2.591306in}{2.402753in}}{\pgfqpoint{2.580256in}{2.402753in}}%
\pgfpathcurveto{\pgfqpoint{2.569206in}{2.402753in}}{\pgfqpoint{2.558607in}{2.398363in}}{\pgfqpoint{2.550793in}{2.390549in}}%
\pgfpathcurveto{\pgfqpoint{2.542979in}{2.382736in}}{\pgfqpoint{2.538589in}{2.372137in}}{\pgfqpoint{2.538589in}{2.361087in}}%
\pgfpathcurveto{\pgfqpoint{2.538589in}{2.350037in}}{\pgfqpoint{2.542979in}{2.339437in}}{\pgfqpoint{2.550793in}{2.331624in}}%
\pgfpathcurveto{\pgfqpoint{2.558607in}{2.323810in}}{\pgfqpoint{2.569206in}{2.319420in}}{\pgfqpoint{2.580256in}{2.319420in}}%
\pgfpathclose%
\pgfusepath{stroke,fill}%
\end{pgfscope}%
\begin{pgfscope}%
\pgfpathrectangle{\pgfqpoint{0.375000in}{0.330000in}}{\pgfqpoint{2.325000in}{2.310000in}}%
\pgfusepath{clip}%
\pgfsetbuttcap%
\pgfsetroundjoin%
\definecolor{currentfill}{rgb}{0.000000,0.000000,0.000000}%
\pgfsetfillcolor{currentfill}%
\pgfsetlinewidth{1.003750pt}%
\definecolor{currentstroke}{rgb}{0.000000,0.000000,0.000000}%
\pgfsetstrokecolor{currentstroke}%
\pgfsetdash{}{0pt}%
\pgfpathmoveto{\pgfqpoint{2.580256in}{2.371451in}}%
\pgfpathcurveto{\pgfqpoint{2.591306in}{2.371451in}}{\pgfqpoint{2.601905in}{2.375841in}}{\pgfqpoint{2.609718in}{2.383655in}}%
\pgfpathcurveto{\pgfqpoint{2.617532in}{2.391468in}}{\pgfqpoint{2.621922in}{2.402067in}}{\pgfqpoint{2.621922in}{2.413117in}}%
\pgfpathcurveto{\pgfqpoint{2.621922in}{2.424168in}}{\pgfqpoint{2.617532in}{2.434767in}}{\pgfqpoint{2.609718in}{2.442580in}}%
\pgfpathcurveto{\pgfqpoint{2.601905in}{2.450394in}}{\pgfqpoint{2.591306in}{2.454784in}}{\pgfqpoint{2.580256in}{2.454784in}}%
\pgfpathcurveto{\pgfqpoint{2.569206in}{2.454784in}}{\pgfqpoint{2.558607in}{2.450394in}}{\pgfqpoint{2.550793in}{2.442580in}}%
\pgfpathcurveto{\pgfqpoint{2.542979in}{2.434767in}}{\pgfqpoint{2.538589in}{2.424168in}}{\pgfqpoint{2.538589in}{2.413117in}}%
\pgfpathcurveto{\pgfqpoint{2.538589in}{2.402067in}}{\pgfqpoint{2.542979in}{2.391468in}}{\pgfqpoint{2.550793in}{2.383655in}}%
\pgfpathcurveto{\pgfqpoint{2.558607in}{2.375841in}}{\pgfqpoint{2.569206in}{2.371451in}}{\pgfqpoint{2.580256in}{2.371451in}}%
\pgfpathclose%
\pgfusepath{stroke,fill}%
\end{pgfscope}%
\begin{pgfscope}%
\pgfpathrectangle{\pgfqpoint{0.375000in}{0.330000in}}{\pgfqpoint{2.325000in}{2.310000in}}%
\pgfusepath{clip}%
\pgfsetbuttcap%
\pgfsetroundjoin%
\definecolor{currentfill}{rgb}{0.000000,0.000000,0.000000}%
\pgfsetfillcolor{currentfill}%
\pgfsetlinewidth{1.003750pt}%
\definecolor{currentstroke}{rgb}{0.000000,0.000000,0.000000}%
\pgfsetstrokecolor{currentstroke}%
\pgfsetdash{}{0pt}%
\pgfpathmoveto{\pgfqpoint{2.580256in}{2.215358in}}%
\pgfpathcurveto{\pgfqpoint{2.591306in}{2.215358in}}{\pgfqpoint{2.601905in}{2.219749in}}{\pgfqpoint{2.609718in}{2.227562in}}%
\pgfpathcurveto{\pgfqpoint{2.617532in}{2.235376in}}{\pgfqpoint{2.621922in}{2.245975in}}{\pgfqpoint{2.621922in}{2.257025in}}%
\pgfpathcurveto{\pgfqpoint{2.621922in}{2.268075in}}{\pgfqpoint{2.617532in}{2.278674in}}{\pgfqpoint{2.609718in}{2.286488in}}%
\pgfpathcurveto{\pgfqpoint{2.601905in}{2.294301in}}{\pgfqpoint{2.591306in}{2.298692in}}{\pgfqpoint{2.580256in}{2.298692in}}%
\pgfpathcurveto{\pgfqpoint{2.569206in}{2.298692in}}{\pgfqpoint{2.558607in}{2.294301in}}{\pgfqpoint{2.550793in}{2.286488in}}%
\pgfpathcurveto{\pgfqpoint{2.542979in}{2.278674in}}{\pgfqpoint{2.538589in}{2.268075in}}{\pgfqpoint{2.538589in}{2.257025in}}%
\pgfpathcurveto{\pgfqpoint{2.538589in}{2.245975in}}{\pgfqpoint{2.542979in}{2.235376in}}{\pgfqpoint{2.550793in}{2.227562in}}%
\pgfpathcurveto{\pgfqpoint{2.558607in}{2.219749in}}{\pgfqpoint{2.569206in}{2.215358in}}{\pgfqpoint{2.580256in}{2.215358in}}%
\pgfpathclose%
\pgfusepath{stroke,fill}%
\end{pgfscope}%
\begin{pgfscope}%
\pgfpathrectangle{\pgfqpoint{0.375000in}{0.330000in}}{\pgfqpoint{2.325000in}{2.310000in}}%
\pgfusepath{clip}%
\pgfsetbuttcap%
\pgfsetroundjoin%
\definecolor{currentfill}{rgb}{0.000000,0.000000,0.000000}%
\pgfsetfillcolor{currentfill}%
\pgfsetlinewidth{1.003750pt}%
\definecolor{currentstroke}{rgb}{0.000000,0.000000,0.000000}%
\pgfsetstrokecolor{currentstroke}%
\pgfsetdash{}{0pt}%
\pgfpathmoveto{\pgfqpoint{2.580256in}{2.267389in}}%
\pgfpathcurveto{\pgfqpoint{2.591306in}{2.267389in}}{\pgfqpoint{2.601905in}{2.271779in}}{\pgfqpoint{2.609718in}{2.279593in}}%
\pgfpathcurveto{\pgfqpoint{2.617532in}{2.287407in}}{\pgfqpoint{2.621922in}{2.298006in}}{\pgfqpoint{2.621922in}{2.309056in}}%
\pgfpathcurveto{\pgfqpoint{2.621922in}{2.320106in}}{\pgfqpoint{2.617532in}{2.330705in}}{\pgfqpoint{2.609718in}{2.338519in}}%
\pgfpathcurveto{\pgfqpoint{2.601905in}{2.346332in}}{\pgfqpoint{2.591306in}{2.350722in}}{\pgfqpoint{2.580256in}{2.350722in}}%
\pgfpathcurveto{\pgfqpoint{2.569206in}{2.350722in}}{\pgfqpoint{2.558607in}{2.346332in}}{\pgfqpoint{2.550793in}{2.338519in}}%
\pgfpathcurveto{\pgfqpoint{2.542979in}{2.330705in}}{\pgfqpoint{2.538589in}{2.320106in}}{\pgfqpoint{2.538589in}{2.309056in}}%
\pgfpathcurveto{\pgfqpoint{2.538589in}{2.298006in}}{\pgfqpoint{2.542979in}{2.287407in}}{\pgfqpoint{2.550793in}{2.279593in}}%
\pgfpathcurveto{\pgfqpoint{2.558607in}{2.271779in}}{\pgfqpoint{2.569206in}{2.267389in}}{\pgfqpoint{2.580256in}{2.267389in}}%
\pgfpathclose%
\pgfusepath{stroke,fill}%
\end{pgfscope}%
\begin{pgfscope}%
\pgfpathrectangle{\pgfqpoint{0.375000in}{0.330000in}}{\pgfqpoint{2.325000in}{2.310000in}}%
\pgfusepath{clip}%
\pgfsetbuttcap%
\pgfsetroundjoin%
\definecolor{currentfill}{rgb}{0.000000,0.000000,0.000000}%
\pgfsetfillcolor{currentfill}%
\pgfsetlinewidth{1.003750pt}%
\definecolor{currentstroke}{rgb}{0.000000,0.000000,0.000000}%
\pgfsetstrokecolor{currentstroke}%
\pgfsetdash{}{0pt}%
\pgfpathmoveto{\pgfqpoint{2.580256in}{2.319420in}}%
\pgfpathcurveto{\pgfqpoint{2.591306in}{2.319420in}}{\pgfqpoint{2.601905in}{2.323810in}}{\pgfqpoint{2.609718in}{2.331624in}}%
\pgfpathcurveto{\pgfqpoint{2.617532in}{2.339437in}}{\pgfqpoint{2.621922in}{2.350037in}}{\pgfqpoint{2.621922in}{2.361087in}}%
\pgfpathcurveto{\pgfqpoint{2.621922in}{2.372137in}}{\pgfqpoint{2.617532in}{2.382736in}}{\pgfqpoint{2.609718in}{2.390549in}}%
\pgfpathcurveto{\pgfqpoint{2.601905in}{2.398363in}}{\pgfqpoint{2.591306in}{2.402753in}}{\pgfqpoint{2.580256in}{2.402753in}}%
\pgfpathcurveto{\pgfqpoint{2.569206in}{2.402753in}}{\pgfqpoint{2.558607in}{2.398363in}}{\pgfqpoint{2.550793in}{2.390549in}}%
\pgfpathcurveto{\pgfqpoint{2.542979in}{2.382736in}}{\pgfqpoint{2.538589in}{2.372137in}}{\pgfqpoint{2.538589in}{2.361087in}}%
\pgfpathcurveto{\pgfqpoint{2.538589in}{2.350037in}}{\pgfqpoint{2.542979in}{2.339437in}}{\pgfqpoint{2.550793in}{2.331624in}}%
\pgfpathcurveto{\pgfqpoint{2.558607in}{2.323810in}}{\pgfqpoint{2.569206in}{2.319420in}}{\pgfqpoint{2.580256in}{2.319420in}}%
\pgfpathclose%
\pgfusepath{stroke,fill}%
\end{pgfscope}%
\begin{pgfscope}%
\pgfpathrectangle{\pgfqpoint{0.375000in}{0.330000in}}{\pgfqpoint{2.325000in}{2.310000in}}%
\pgfusepath{clip}%
\pgfsetbuttcap%
\pgfsetroundjoin%
\definecolor{currentfill}{rgb}{0.000000,0.000000,0.000000}%
\pgfsetfillcolor{currentfill}%
\pgfsetlinewidth{1.003750pt}%
\definecolor{currentstroke}{rgb}{0.000000,0.000000,0.000000}%
\pgfsetstrokecolor{currentstroke}%
\pgfsetdash{}{0pt}%
\pgfpathmoveto{\pgfqpoint{2.580256in}{2.215358in}}%
\pgfpathcurveto{\pgfqpoint{2.591306in}{2.215358in}}{\pgfqpoint{2.601905in}{2.219749in}}{\pgfqpoint{2.609718in}{2.227562in}}%
\pgfpathcurveto{\pgfqpoint{2.617532in}{2.235376in}}{\pgfqpoint{2.621922in}{2.245975in}}{\pgfqpoint{2.621922in}{2.257025in}}%
\pgfpathcurveto{\pgfqpoint{2.621922in}{2.268075in}}{\pgfqpoint{2.617532in}{2.278674in}}{\pgfqpoint{2.609718in}{2.286488in}}%
\pgfpathcurveto{\pgfqpoint{2.601905in}{2.294301in}}{\pgfqpoint{2.591306in}{2.298692in}}{\pgfqpoint{2.580256in}{2.298692in}}%
\pgfpathcurveto{\pgfqpoint{2.569206in}{2.298692in}}{\pgfqpoint{2.558607in}{2.294301in}}{\pgfqpoint{2.550793in}{2.286488in}}%
\pgfpathcurveto{\pgfqpoint{2.542979in}{2.278674in}}{\pgfqpoint{2.538589in}{2.268075in}}{\pgfqpoint{2.538589in}{2.257025in}}%
\pgfpathcurveto{\pgfqpoint{2.538589in}{2.245975in}}{\pgfqpoint{2.542979in}{2.235376in}}{\pgfqpoint{2.550793in}{2.227562in}}%
\pgfpathcurveto{\pgfqpoint{2.558607in}{2.219749in}}{\pgfqpoint{2.569206in}{2.215358in}}{\pgfqpoint{2.580256in}{2.215358in}}%
\pgfpathclose%
\pgfusepath{stroke,fill}%
\end{pgfscope}%
\begin{pgfscope}%
\pgfpathrectangle{\pgfqpoint{0.375000in}{0.330000in}}{\pgfqpoint{2.325000in}{2.310000in}}%
\pgfusepath{clip}%
\pgfsetbuttcap%
\pgfsetroundjoin%
\definecolor{currentfill}{rgb}{0.000000,0.000000,0.000000}%
\pgfsetfillcolor{currentfill}%
\pgfsetlinewidth{1.003750pt}%
\definecolor{currentstroke}{rgb}{0.000000,0.000000,0.000000}%
\pgfsetstrokecolor{currentstroke}%
\pgfsetdash{}{0pt}%
\pgfpathmoveto{\pgfqpoint{2.580256in}{2.371451in}}%
\pgfpathcurveto{\pgfqpoint{2.591306in}{2.371451in}}{\pgfqpoint{2.601905in}{2.375841in}}{\pgfqpoint{2.609718in}{2.383655in}}%
\pgfpathcurveto{\pgfqpoint{2.617532in}{2.391468in}}{\pgfqpoint{2.621922in}{2.402067in}}{\pgfqpoint{2.621922in}{2.413117in}}%
\pgfpathcurveto{\pgfqpoint{2.621922in}{2.424168in}}{\pgfqpoint{2.617532in}{2.434767in}}{\pgfqpoint{2.609718in}{2.442580in}}%
\pgfpathcurveto{\pgfqpoint{2.601905in}{2.450394in}}{\pgfqpoint{2.591306in}{2.454784in}}{\pgfqpoint{2.580256in}{2.454784in}}%
\pgfpathcurveto{\pgfqpoint{2.569206in}{2.454784in}}{\pgfqpoint{2.558607in}{2.450394in}}{\pgfqpoint{2.550793in}{2.442580in}}%
\pgfpathcurveto{\pgfqpoint{2.542979in}{2.434767in}}{\pgfqpoint{2.538589in}{2.424168in}}{\pgfqpoint{2.538589in}{2.413117in}}%
\pgfpathcurveto{\pgfqpoint{2.538589in}{2.402067in}}{\pgfqpoint{2.542979in}{2.391468in}}{\pgfqpoint{2.550793in}{2.383655in}}%
\pgfpathcurveto{\pgfqpoint{2.558607in}{2.375841in}}{\pgfqpoint{2.569206in}{2.371451in}}{\pgfqpoint{2.580256in}{2.371451in}}%
\pgfpathclose%
\pgfusepath{stroke,fill}%
\end{pgfscope}%
\begin{pgfscope}%
\pgfpathrectangle{\pgfqpoint{0.375000in}{0.330000in}}{\pgfqpoint{2.325000in}{2.310000in}}%
\pgfusepath{clip}%
\pgfsetbuttcap%
\pgfsetroundjoin%
\definecolor{currentfill}{rgb}{0.000000,0.000000,0.000000}%
\pgfsetfillcolor{currentfill}%
\pgfsetlinewidth{1.003750pt}%
\definecolor{currentstroke}{rgb}{0.000000,0.000000,0.000000}%
\pgfsetstrokecolor{currentstroke}%
\pgfsetdash{}{0pt}%
\pgfpathmoveto{\pgfqpoint{2.580256in}{2.319420in}}%
\pgfpathcurveto{\pgfqpoint{2.591306in}{2.319420in}}{\pgfqpoint{2.601905in}{2.323810in}}{\pgfqpoint{2.609718in}{2.331624in}}%
\pgfpathcurveto{\pgfqpoint{2.617532in}{2.339437in}}{\pgfqpoint{2.621922in}{2.350037in}}{\pgfqpoint{2.621922in}{2.361087in}}%
\pgfpathcurveto{\pgfqpoint{2.621922in}{2.372137in}}{\pgfqpoint{2.617532in}{2.382736in}}{\pgfqpoint{2.609718in}{2.390549in}}%
\pgfpathcurveto{\pgfqpoint{2.601905in}{2.398363in}}{\pgfqpoint{2.591306in}{2.402753in}}{\pgfqpoint{2.580256in}{2.402753in}}%
\pgfpathcurveto{\pgfqpoint{2.569206in}{2.402753in}}{\pgfqpoint{2.558607in}{2.398363in}}{\pgfqpoint{2.550793in}{2.390549in}}%
\pgfpathcurveto{\pgfqpoint{2.542979in}{2.382736in}}{\pgfqpoint{2.538589in}{2.372137in}}{\pgfqpoint{2.538589in}{2.361087in}}%
\pgfpathcurveto{\pgfqpoint{2.538589in}{2.350037in}}{\pgfqpoint{2.542979in}{2.339437in}}{\pgfqpoint{2.550793in}{2.331624in}}%
\pgfpathcurveto{\pgfqpoint{2.558607in}{2.323810in}}{\pgfqpoint{2.569206in}{2.319420in}}{\pgfqpoint{2.580256in}{2.319420in}}%
\pgfpathclose%
\pgfusepath{stroke,fill}%
\end{pgfscope}%
\begin{pgfscope}%
\pgfpathrectangle{\pgfqpoint{0.375000in}{0.330000in}}{\pgfqpoint{2.325000in}{2.310000in}}%
\pgfusepath{clip}%
\pgfsetbuttcap%
\pgfsetroundjoin%
\definecolor{currentfill}{rgb}{0.000000,0.000000,0.000000}%
\pgfsetfillcolor{currentfill}%
\pgfsetlinewidth{1.003750pt}%
\definecolor{currentstroke}{rgb}{0.000000,0.000000,0.000000}%
\pgfsetstrokecolor{currentstroke}%
\pgfsetdash{}{0pt}%
\pgfpathmoveto{\pgfqpoint{2.580256in}{2.319420in}}%
\pgfpathcurveto{\pgfqpoint{2.591306in}{2.319420in}}{\pgfqpoint{2.601905in}{2.323810in}}{\pgfqpoint{2.609718in}{2.331624in}}%
\pgfpathcurveto{\pgfqpoint{2.617532in}{2.339437in}}{\pgfqpoint{2.621922in}{2.350037in}}{\pgfqpoint{2.621922in}{2.361087in}}%
\pgfpathcurveto{\pgfqpoint{2.621922in}{2.372137in}}{\pgfqpoint{2.617532in}{2.382736in}}{\pgfqpoint{2.609718in}{2.390549in}}%
\pgfpathcurveto{\pgfqpoint{2.601905in}{2.398363in}}{\pgfqpoint{2.591306in}{2.402753in}}{\pgfqpoint{2.580256in}{2.402753in}}%
\pgfpathcurveto{\pgfqpoint{2.569206in}{2.402753in}}{\pgfqpoint{2.558607in}{2.398363in}}{\pgfqpoint{2.550793in}{2.390549in}}%
\pgfpathcurveto{\pgfqpoint{2.542979in}{2.382736in}}{\pgfqpoint{2.538589in}{2.372137in}}{\pgfqpoint{2.538589in}{2.361087in}}%
\pgfpathcurveto{\pgfqpoint{2.538589in}{2.350037in}}{\pgfqpoint{2.542979in}{2.339437in}}{\pgfqpoint{2.550793in}{2.331624in}}%
\pgfpathcurveto{\pgfqpoint{2.558607in}{2.323810in}}{\pgfqpoint{2.569206in}{2.319420in}}{\pgfqpoint{2.580256in}{2.319420in}}%
\pgfpathclose%
\pgfusepath{stroke,fill}%
\end{pgfscope}%
\begin{pgfscope}%
\pgfpathrectangle{\pgfqpoint{0.375000in}{0.330000in}}{\pgfqpoint{2.325000in}{2.310000in}}%
\pgfusepath{clip}%
\pgfsetbuttcap%
\pgfsetroundjoin%
\definecolor{currentfill}{rgb}{0.000000,0.000000,0.000000}%
\pgfsetfillcolor{currentfill}%
\pgfsetlinewidth{1.003750pt}%
\definecolor{currentstroke}{rgb}{0.000000,0.000000,0.000000}%
\pgfsetstrokecolor{currentstroke}%
\pgfsetdash{}{0pt}%
\pgfpathmoveto{\pgfqpoint{2.580256in}{2.371451in}}%
\pgfpathcurveto{\pgfqpoint{2.591306in}{2.371451in}}{\pgfqpoint{2.601905in}{2.375841in}}{\pgfqpoint{2.609718in}{2.383655in}}%
\pgfpathcurveto{\pgfqpoint{2.617532in}{2.391468in}}{\pgfqpoint{2.621922in}{2.402067in}}{\pgfqpoint{2.621922in}{2.413117in}}%
\pgfpathcurveto{\pgfqpoint{2.621922in}{2.424168in}}{\pgfqpoint{2.617532in}{2.434767in}}{\pgfqpoint{2.609718in}{2.442580in}}%
\pgfpathcurveto{\pgfqpoint{2.601905in}{2.450394in}}{\pgfqpoint{2.591306in}{2.454784in}}{\pgfqpoint{2.580256in}{2.454784in}}%
\pgfpathcurveto{\pgfqpoint{2.569206in}{2.454784in}}{\pgfqpoint{2.558607in}{2.450394in}}{\pgfqpoint{2.550793in}{2.442580in}}%
\pgfpathcurveto{\pgfqpoint{2.542979in}{2.434767in}}{\pgfqpoint{2.538589in}{2.424168in}}{\pgfqpoint{2.538589in}{2.413117in}}%
\pgfpathcurveto{\pgfqpoint{2.538589in}{2.402067in}}{\pgfqpoint{2.542979in}{2.391468in}}{\pgfqpoint{2.550793in}{2.383655in}}%
\pgfpathcurveto{\pgfqpoint{2.558607in}{2.375841in}}{\pgfqpoint{2.569206in}{2.371451in}}{\pgfqpoint{2.580256in}{2.371451in}}%
\pgfpathclose%
\pgfusepath{stroke,fill}%
\end{pgfscope}%
\begin{pgfscope}%
\pgfpathrectangle{\pgfqpoint{0.375000in}{0.330000in}}{\pgfqpoint{2.325000in}{2.310000in}}%
\pgfusepath{clip}%
\pgfsetbuttcap%
\pgfsetroundjoin%
\definecolor{currentfill}{rgb}{0.000000,0.000000,0.000000}%
\pgfsetfillcolor{currentfill}%
\pgfsetlinewidth{1.003750pt}%
\definecolor{currentstroke}{rgb}{0.000000,0.000000,0.000000}%
\pgfsetstrokecolor{currentstroke}%
\pgfsetdash{}{0pt}%
\pgfpathmoveto{\pgfqpoint{2.580256in}{2.319420in}}%
\pgfpathcurveto{\pgfqpoint{2.591306in}{2.319420in}}{\pgfqpoint{2.601905in}{2.323810in}}{\pgfqpoint{2.609718in}{2.331624in}}%
\pgfpathcurveto{\pgfqpoint{2.617532in}{2.339437in}}{\pgfqpoint{2.621922in}{2.350037in}}{\pgfqpoint{2.621922in}{2.361087in}}%
\pgfpathcurveto{\pgfqpoint{2.621922in}{2.372137in}}{\pgfqpoint{2.617532in}{2.382736in}}{\pgfqpoint{2.609718in}{2.390549in}}%
\pgfpathcurveto{\pgfqpoint{2.601905in}{2.398363in}}{\pgfqpoint{2.591306in}{2.402753in}}{\pgfqpoint{2.580256in}{2.402753in}}%
\pgfpathcurveto{\pgfqpoint{2.569206in}{2.402753in}}{\pgfqpoint{2.558607in}{2.398363in}}{\pgfqpoint{2.550793in}{2.390549in}}%
\pgfpathcurveto{\pgfqpoint{2.542979in}{2.382736in}}{\pgfqpoint{2.538589in}{2.372137in}}{\pgfqpoint{2.538589in}{2.361087in}}%
\pgfpathcurveto{\pgfqpoint{2.538589in}{2.350037in}}{\pgfqpoint{2.542979in}{2.339437in}}{\pgfqpoint{2.550793in}{2.331624in}}%
\pgfpathcurveto{\pgfqpoint{2.558607in}{2.323810in}}{\pgfqpoint{2.569206in}{2.319420in}}{\pgfqpoint{2.580256in}{2.319420in}}%
\pgfpathclose%
\pgfusepath{stroke,fill}%
\end{pgfscope}%
\begin{pgfscope}%
\pgfpathrectangle{\pgfqpoint{0.375000in}{0.330000in}}{\pgfqpoint{2.325000in}{2.310000in}}%
\pgfusepath{clip}%
\pgfsetbuttcap%
\pgfsetroundjoin%
\definecolor{currentfill}{rgb}{0.000000,0.000000,0.000000}%
\pgfsetfillcolor{currentfill}%
\pgfsetlinewidth{1.003750pt}%
\definecolor{currentstroke}{rgb}{0.000000,0.000000,0.000000}%
\pgfsetstrokecolor{currentstroke}%
\pgfsetdash{}{0pt}%
\pgfpathmoveto{\pgfqpoint{2.580256in}{2.319420in}}%
\pgfpathcurveto{\pgfqpoint{2.591306in}{2.319420in}}{\pgfqpoint{2.601905in}{2.323810in}}{\pgfqpoint{2.609718in}{2.331624in}}%
\pgfpathcurveto{\pgfqpoint{2.617532in}{2.339437in}}{\pgfqpoint{2.621922in}{2.350037in}}{\pgfqpoint{2.621922in}{2.361087in}}%
\pgfpathcurveto{\pgfqpoint{2.621922in}{2.372137in}}{\pgfqpoint{2.617532in}{2.382736in}}{\pgfqpoint{2.609718in}{2.390549in}}%
\pgfpathcurveto{\pgfqpoint{2.601905in}{2.398363in}}{\pgfqpoint{2.591306in}{2.402753in}}{\pgfqpoint{2.580256in}{2.402753in}}%
\pgfpathcurveto{\pgfqpoint{2.569206in}{2.402753in}}{\pgfqpoint{2.558607in}{2.398363in}}{\pgfqpoint{2.550793in}{2.390549in}}%
\pgfpathcurveto{\pgfqpoint{2.542979in}{2.382736in}}{\pgfqpoint{2.538589in}{2.372137in}}{\pgfqpoint{2.538589in}{2.361087in}}%
\pgfpathcurveto{\pgfqpoint{2.538589in}{2.350037in}}{\pgfqpoint{2.542979in}{2.339437in}}{\pgfqpoint{2.550793in}{2.331624in}}%
\pgfpathcurveto{\pgfqpoint{2.558607in}{2.323810in}}{\pgfqpoint{2.569206in}{2.319420in}}{\pgfqpoint{2.580256in}{2.319420in}}%
\pgfpathclose%
\pgfusepath{stroke,fill}%
\end{pgfscope}%
\begin{pgfscope}%
\pgfpathrectangle{\pgfqpoint{0.375000in}{0.330000in}}{\pgfqpoint{2.325000in}{2.310000in}}%
\pgfusepath{clip}%
\pgfsetbuttcap%
\pgfsetroundjoin%
\definecolor{currentfill}{rgb}{0.000000,0.000000,0.000000}%
\pgfsetfillcolor{currentfill}%
\pgfsetlinewidth{1.003750pt}%
\definecolor{currentstroke}{rgb}{0.000000,0.000000,0.000000}%
\pgfsetstrokecolor{currentstroke}%
\pgfsetdash{}{0pt}%
\pgfpathmoveto{\pgfqpoint{2.580256in}{2.319420in}}%
\pgfpathcurveto{\pgfqpoint{2.591306in}{2.319420in}}{\pgfqpoint{2.601905in}{2.323810in}}{\pgfqpoint{2.609718in}{2.331624in}}%
\pgfpathcurveto{\pgfqpoint{2.617532in}{2.339437in}}{\pgfqpoint{2.621922in}{2.350037in}}{\pgfqpoint{2.621922in}{2.361087in}}%
\pgfpathcurveto{\pgfqpoint{2.621922in}{2.372137in}}{\pgfqpoint{2.617532in}{2.382736in}}{\pgfqpoint{2.609718in}{2.390549in}}%
\pgfpathcurveto{\pgfqpoint{2.601905in}{2.398363in}}{\pgfqpoint{2.591306in}{2.402753in}}{\pgfqpoint{2.580256in}{2.402753in}}%
\pgfpathcurveto{\pgfqpoint{2.569206in}{2.402753in}}{\pgfqpoint{2.558607in}{2.398363in}}{\pgfqpoint{2.550793in}{2.390549in}}%
\pgfpathcurveto{\pgfqpoint{2.542979in}{2.382736in}}{\pgfqpoint{2.538589in}{2.372137in}}{\pgfqpoint{2.538589in}{2.361087in}}%
\pgfpathcurveto{\pgfqpoint{2.538589in}{2.350037in}}{\pgfqpoint{2.542979in}{2.339437in}}{\pgfqpoint{2.550793in}{2.331624in}}%
\pgfpathcurveto{\pgfqpoint{2.558607in}{2.323810in}}{\pgfqpoint{2.569206in}{2.319420in}}{\pgfqpoint{2.580256in}{2.319420in}}%
\pgfpathclose%
\pgfusepath{stroke,fill}%
\end{pgfscope}%
\begin{pgfscope}%
\pgfpathrectangle{\pgfqpoint{0.375000in}{0.330000in}}{\pgfqpoint{2.325000in}{2.310000in}}%
\pgfusepath{clip}%
\pgfsetbuttcap%
\pgfsetroundjoin%
\definecolor{currentfill}{rgb}{0.000000,0.000000,0.000000}%
\pgfsetfillcolor{currentfill}%
\pgfsetlinewidth{1.003750pt}%
\definecolor{currentstroke}{rgb}{0.000000,0.000000,0.000000}%
\pgfsetstrokecolor{currentstroke}%
\pgfsetdash{}{0pt}%
\pgfpathmoveto{\pgfqpoint{2.580256in}{2.215358in}}%
\pgfpathcurveto{\pgfqpoint{2.591306in}{2.215358in}}{\pgfqpoint{2.601905in}{2.219749in}}{\pgfqpoint{2.609718in}{2.227562in}}%
\pgfpathcurveto{\pgfqpoint{2.617532in}{2.235376in}}{\pgfqpoint{2.621922in}{2.245975in}}{\pgfqpoint{2.621922in}{2.257025in}}%
\pgfpathcurveto{\pgfqpoint{2.621922in}{2.268075in}}{\pgfqpoint{2.617532in}{2.278674in}}{\pgfqpoint{2.609718in}{2.286488in}}%
\pgfpathcurveto{\pgfqpoint{2.601905in}{2.294301in}}{\pgfqpoint{2.591306in}{2.298692in}}{\pgfqpoint{2.580256in}{2.298692in}}%
\pgfpathcurveto{\pgfqpoint{2.569206in}{2.298692in}}{\pgfqpoint{2.558607in}{2.294301in}}{\pgfqpoint{2.550793in}{2.286488in}}%
\pgfpathcurveto{\pgfqpoint{2.542979in}{2.278674in}}{\pgfqpoint{2.538589in}{2.268075in}}{\pgfqpoint{2.538589in}{2.257025in}}%
\pgfpathcurveto{\pgfqpoint{2.538589in}{2.245975in}}{\pgfqpoint{2.542979in}{2.235376in}}{\pgfqpoint{2.550793in}{2.227562in}}%
\pgfpathcurveto{\pgfqpoint{2.558607in}{2.219749in}}{\pgfqpoint{2.569206in}{2.215358in}}{\pgfqpoint{2.580256in}{2.215358in}}%
\pgfpathclose%
\pgfusepath{stroke,fill}%
\end{pgfscope}%
\begin{pgfscope}%
\pgfpathrectangle{\pgfqpoint{0.375000in}{0.330000in}}{\pgfqpoint{2.325000in}{2.310000in}}%
\pgfusepath{clip}%
\pgfsetbuttcap%
\pgfsetroundjoin%
\definecolor{currentfill}{rgb}{0.000000,0.000000,0.000000}%
\pgfsetfillcolor{currentfill}%
\pgfsetlinewidth{1.003750pt}%
\definecolor{currentstroke}{rgb}{0.000000,0.000000,0.000000}%
\pgfsetstrokecolor{currentstroke}%
\pgfsetdash{}{0pt}%
\pgfpathmoveto{\pgfqpoint{2.580256in}{2.215358in}}%
\pgfpathcurveto{\pgfqpoint{2.591306in}{2.215358in}}{\pgfqpoint{2.601905in}{2.219749in}}{\pgfqpoint{2.609718in}{2.227562in}}%
\pgfpathcurveto{\pgfqpoint{2.617532in}{2.235376in}}{\pgfqpoint{2.621922in}{2.245975in}}{\pgfqpoint{2.621922in}{2.257025in}}%
\pgfpathcurveto{\pgfqpoint{2.621922in}{2.268075in}}{\pgfqpoint{2.617532in}{2.278674in}}{\pgfqpoint{2.609718in}{2.286488in}}%
\pgfpathcurveto{\pgfqpoint{2.601905in}{2.294301in}}{\pgfqpoint{2.591306in}{2.298692in}}{\pgfqpoint{2.580256in}{2.298692in}}%
\pgfpathcurveto{\pgfqpoint{2.569206in}{2.298692in}}{\pgfqpoint{2.558607in}{2.294301in}}{\pgfqpoint{2.550793in}{2.286488in}}%
\pgfpathcurveto{\pgfqpoint{2.542979in}{2.278674in}}{\pgfqpoint{2.538589in}{2.268075in}}{\pgfqpoint{2.538589in}{2.257025in}}%
\pgfpathcurveto{\pgfqpoint{2.538589in}{2.245975in}}{\pgfqpoint{2.542979in}{2.235376in}}{\pgfqpoint{2.550793in}{2.227562in}}%
\pgfpathcurveto{\pgfqpoint{2.558607in}{2.219749in}}{\pgfqpoint{2.569206in}{2.215358in}}{\pgfqpoint{2.580256in}{2.215358in}}%
\pgfpathclose%
\pgfusepath{stroke,fill}%
\end{pgfscope}%
\begin{pgfscope}%
\pgfpathrectangle{\pgfqpoint{0.375000in}{0.330000in}}{\pgfqpoint{2.325000in}{2.310000in}}%
\pgfusepath{clip}%
\pgfsetbuttcap%
\pgfsetroundjoin%
\definecolor{currentfill}{rgb}{0.000000,0.000000,0.000000}%
\pgfsetfillcolor{currentfill}%
\pgfsetlinewidth{1.003750pt}%
\definecolor{currentstroke}{rgb}{0.000000,0.000000,0.000000}%
\pgfsetstrokecolor{currentstroke}%
\pgfsetdash{}{0pt}%
\pgfpathmoveto{\pgfqpoint{2.580256in}{2.267389in}}%
\pgfpathcurveto{\pgfqpoint{2.591306in}{2.267389in}}{\pgfqpoint{2.601905in}{2.271779in}}{\pgfqpoint{2.609718in}{2.279593in}}%
\pgfpathcurveto{\pgfqpoint{2.617532in}{2.287407in}}{\pgfqpoint{2.621922in}{2.298006in}}{\pgfqpoint{2.621922in}{2.309056in}}%
\pgfpathcurveto{\pgfqpoint{2.621922in}{2.320106in}}{\pgfqpoint{2.617532in}{2.330705in}}{\pgfqpoint{2.609718in}{2.338519in}}%
\pgfpathcurveto{\pgfqpoint{2.601905in}{2.346332in}}{\pgfqpoint{2.591306in}{2.350722in}}{\pgfqpoint{2.580256in}{2.350722in}}%
\pgfpathcurveto{\pgfqpoint{2.569206in}{2.350722in}}{\pgfqpoint{2.558607in}{2.346332in}}{\pgfqpoint{2.550793in}{2.338519in}}%
\pgfpathcurveto{\pgfqpoint{2.542979in}{2.330705in}}{\pgfqpoint{2.538589in}{2.320106in}}{\pgfqpoint{2.538589in}{2.309056in}}%
\pgfpathcurveto{\pgfqpoint{2.538589in}{2.298006in}}{\pgfqpoint{2.542979in}{2.287407in}}{\pgfqpoint{2.550793in}{2.279593in}}%
\pgfpathcurveto{\pgfqpoint{2.558607in}{2.271779in}}{\pgfqpoint{2.569206in}{2.267389in}}{\pgfqpoint{2.580256in}{2.267389in}}%
\pgfpathclose%
\pgfusepath{stroke,fill}%
\end{pgfscope}%
\begin{pgfscope}%
\pgfpathrectangle{\pgfqpoint{0.375000in}{0.330000in}}{\pgfqpoint{2.325000in}{2.310000in}}%
\pgfusepath{clip}%
\pgfsetbuttcap%
\pgfsetroundjoin%
\definecolor{currentfill}{rgb}{0.000000,0.000000,0.000000}%
\pgfsetfillcolor{currentfill}%
\pgfsetlinewidth{1.003750pt}%
\definecolor{currentstroke}{rgb}{0.000000,0.000000,0.000000}%
\pgfsetstrokecolor{currentstroke}%
\pgfsetdash{}{0pt}%
\pgfpathmoveto{\pgfqpoint{2.580256in}{2.371451in}}%
\pgfpathcurveto{\pgfqpoint{2.591306in}{2.371451in}}{\pgfqpoint{2.601905in}{2.375841in}}{\pgfqpoint{2.609718in}{2.383655in}}%
\pgfpathcurveto{\pgfqpoint{2.617532in}{2.391468in}}{\pgfqpoint{2.621922in}{2.402067in}}{\pgfqpoint{2.621922in}{2.413117in}}%
\pgfpathcurveto{\pgfqpoint{2.621922in}{2.424168in}}{\pgfqpoint{2.617532in}{2.434767in}}{\pgfqpoint{2.609718in}{2.442580in}}%
\pgfpathcurveto{\pgfqpoint{2.601905in}{2.450394in}}{\pgfqpoint{2.591306in}{2.454784in}}{\pgfqpoint{2.580256in}{2.454784in}}%
\pgfpathcurveto{\pgfqpoint{2.569206in}{2.454784in}}{\pgfqpoint{2.558607in}{2.450394in}}{\pgfqpoint{2.550793in}{2.442580in}}%
\pgfpathcurveto{\pgfqpoint{2.542979in}{2.434767in}}{\pgfqpoint{2.538589in}{2.424168in}}{\pgfqpoint{2.538589in}{2.413117in}}%
\pgfpathcurveto{\pgfqpoint{2.538589in}{2.402067in}}{\pgfqpoint{2.542979in}{2.391468in}}{\pgfqpoint{2.550793in}{2.383655in}}%
\pgfpathcurveto{\pgfqpoint{2.558607in}{2.375841in}}{\pgfqpoint{2.569206in}{2.371451in}}{\pgfqpoint{2.580256in}{2.371451in}}%
\pgfpathclose%
\pgfusepath{stroke,fill}%
\end{pgfscope}%
\begin{pgfscope}%
\pgfpathrectangle{\pgfqpoint{0.375000in}{0.330000in}}{\pgfqpoint{2.325000in}{2.310000in}}%
\pgfusepath{clip}%
\pgfsetbuttcap%
\pgfsetroundjoin%
\definecolor{currentfill}{rgb}{0.000000,0.000000,0.000000}%
\pgfsetfillcolor{currentfill}%
\pgfsetlinewidth{1.003750pt}%
\definecolor{currentstroke}{rgb}{0.000000,0.000000,0.000000}%
\pgfsetstrokecolor{currentstroke}%
\pgfsetdash{}{0pt}%
\pgfpathmoveto{\pgfqpoint{2.580256in}{2.215358in}}%
\pgfpathcurveto{\pgfqpoint{2.591306in}{2.215358in}}{\pgfqpoint{2.601905in}{2.219749in}}{\pgfqpoint{2.609718in}{2.227562in}}%
\pgfpathcurveto{\pgfqpoint{2.617532in}{2.235376in}}{\pgfqpoint{2.621922in}{2.245975in}}{\pgfqpoint{2.621922in}{2.257025in}}%
\pgfpathcurveto{\pgfqpoint{2.621922in}{2.268075in}}{\pgfqpoint{2.617532in}{2.278674in}}{\pgfqpoint{2.609718in}{2.286488in}}%
\pgfpathcurveto{\pgfqpoint{2.601905in}{2.294301in}}{\pgfqpoint{2.591306in}{2.298692in}}{\pgfqpoint{2.580256in}{2.298692in}}%
\pgfpathcurveto{\pgfqpoint{2.569206in}{2.298692in}}{\pgfqpoint{2.558607in}{2.294301in}}{\pgfqpoint{2.550793in}{2.286488in}}%
\pgfpathcurveto{\pgfqpoint{2.542979in}{2.278674in}}{\pgfqpoint{2.538589in}{2.268075in}}{\pgfqpoint{2.538589in}{2.257025in}}%
\pgfpathcurveto{\pgfqpoint{2.538589in}{2.245975in}}{\pgfqpoint{2.542979in}{2.235376in}}{\pgfqpoint{2.550793in}{2.227562in}}%
\pgfpathcurveto{\pgfqpoint{2.558607in}{2.219749in}}{\pgfqpoint{2.569206in}{2.215358in}}{\pgfqpoint{2.580256in}{2.215358in}}%
\pgfpathclose%
\pgfusepath{stroke,fill}%
\end{pgfscope}%
\begin{pgfscope}%
\pgfpathrectangle{\pgfqpoint{0.375000in}{0.330000in}}{\pgfqpoint{2.325000in}{2.310000in}}%
\pgfusepath{clip}%
\pgfsetbuttcap%
\pgfsetroundjoin%
\definecolor{currentfill}{rgb}{0.000000,0.000000,0.000000}%
\pgfsetfillcolor{currentfill}%
\pgfsetlinewidth{1.003750pt}%
\definecolor{currentstroke}{rgb}{0.000000,0.000000,0.000000}%
\pgfsetstrokecolor{currentstroke}%
\pgfsetdash{}{0pt}%
\pgfpathmoveto{\pgfqpoint{2.580256in}{2.475512in}}%
\pgfpathcurveto{\pgfqpoint{2.591306in}{2.475512in}}{\pgfqpoint{2.601905in}{2.479903in}}{\pgfqpoint{2.609718in}{2.487716in}}%
\pgfpathcurveto{\pgfqpoint{2.617532in}{2.495530in}}{\pgfqpoint{2.621922in}{2.506129in}}{\pgfqpoint{2.621922in}{2.517179in}}%
\pgfpathcurveto{\pgfqpoint{2.621922in}{2.528229in}}{\pgfqpoint{2.617532in}{2.538828in}}{\pgfqpoint{2.609718in}{2.546642in}}%
\pgfpathcurveto{\pgfqpoint{2.601905in}{2.554456in}}{\pgfqpoint{2.591306in}{2.558846in}}{\pgfqpoint{2.580256in}{2.558846in}}%
\pgfpathcurveto{\pgfqpoint{2.569206in}{2.558846in}}{\pgfqpoint{2.558607in}{2.554456in}}{\pgfqpoint{2.550793in}{2.546642in}}%
\pgfpathcurveto{\pgfqpoint{2.542979in}{2.538828in}}{\pgfqpoint{2.538589in}{2.528229in}}{\pgfqpoint{2.538589in}{2.517179in}}%
\pgfpathcurveto{\pgfqpoint{2.538589in}{2.506129in}}{\pgfqpoint{2.542979in}{2.495530in}}{\pgfqpoint{2.550793in}{2.487716in}}%
\pgfpathcurveto{\pgfqpoint{2.558607in}{2.479903in}}{\pgfqpoint{2.569206in}{2.475512in}}{\pgfqpoint{2.580256in}{2.475512in}}%
\pgfpathclose%
\pgfusepath{stroke,fill}%
\end{pgfscope}%
\begin{pgfscope}%
\pgfpathrectangle{\pgfqpoint{0.375000in}{0.330000in}}{\pgfqpoint{2.325000in}{2.310000in}}%
\pgfusepath{clip}%
\pgfsetbuttcap%
\pgfsetroundjoin%
\definecolor{currentfill}{rgb}{0.000000,0.000000,0.000000}%
\pgfsetfillcolor{currentfill}%
\pgfsetlinewidth{1.003750pt}%
\definecolor{currentstroke}{rgb}{0.000000,0.000000,0.000000}%
\pgfsetstrokecolor{currentstroke}%
\pgfsetdash{}{0pt}%
\pgfpathmoveto{\pgfqpoint{2.580256in}{2.319420in}}%
\pgfpathcurveto{\pgfqpoint{2.591306in}{2.319420in}}{\pgfqpoint{2.601905in}{2.323810in}}{\pgfqpoint{2.609718in}{2.331624in}}%
\pgfpathcurveto{\pgfqpoint{2.617532in}{2.339437in}}{\pgfqpoint{2.621922in}{2.350037in}}{\pgfqpoint{2.621922in}{2.361087in}}%
\pgfpathcurveto{\pgfqpoint{2.621922in}{2.372137in}}{\pgfqpoint{2.617532in}{2.382736in}}{\pgfqpoint{2.609718in}{2.390549in}}%
\pgfpathcurveto{\pgfqpoint{2.601905in}{2.398363in}}{\pgfqpoint{2.591306in}{2.402753in}}{\pgfqpoint{2.580256in}{2.402753in}}%
\pgfpathcurveto{\pgfqpoint{2.569206in}{2.402753in}}{\pgfqpoint{2.558607in}{2.398363in}}{\pgfqpoint{2.550793in}{2.390549in}}%
\pgfpathcurveto{\pgfqpoint{2.542979in}{2.382736in}}{\pgfqpoint{2.538589in}{2.372137in}}{\pgfqpoint{2.538589in}{2.361087in}}%
\pgfpathcurveto{\pgfqpoint{2.538589in}{2.350037in}}{\pgfqpoint{2.542979in}{2.339437in}}{\pgfqpoint{2.550793in}{2.331624in}}%
\pgfpathcurveto{\pgfqpoint{2.558607in}{2.323810in}}{\pgfqpoint{2.569206in}{2.319420in}}{\pgfqpoint{2.580256in}{2.319420in}}%
\pgfpathclose%
\pgfusepath{stroke,fill}%
\end{pgfscope}%
\begin{pgfscope}%
\pgfpathrectangle{\pgfqpoint{0.375000in}{0.330000in}}{\pgfqpoint{2.325000in}{2.310000in}}%
\pgfusepath{clip}%
\pgfsetbuttcap%
\pgfsetroundjoin%
\definecolor{currentfill}{rgb}{0.000000,0.000000,0.000000}%
\pgfsetfillcolor{currentfill}%
\pgfsetlinewidth{1.003750pt}%
\definecolor{currentstroke}{rgb}{0.000000,0.000000,0.000000}%
\pgfsetstrokecolor{currentstroke}%
\pgfsetdash{}{0pt}%
\pgfpathmoveto{\pgfqpoint{2.580256in}{2.371451in}}%
\pgfpathcurveto{\pgfqpoint{2.591306in}{2.371451in}}{\pgfqpoint{2.601905in}{2.375841in}}{\pgfqpoint{2.609718in}{2.383655in}}%
\pgfpathcurveto{\pgfqpoint{2.617532in}{2.391468in}}{\pgfqpoint{2.621922in}{2.402067in}}{\pgfqpoint{2.621922in}{2.413117in}}%
\pgfpathcurveto{\pgfqpoint{2.621922in}{2.424168in}}{\pgfqpoint{2.617532in}{2.434767in}}{\pgfqpoint{2.609718in}{2.442580in}}%
\pgfpathcurveto{\pgfqpoint{2.601905in}{2.450394in}}{\pgfqpoint{2.591306in}{2.454784in}}{\pgfqpoint{2.580256in}{2.454784in}}%
\pgfpathcurveto{\pgfqpoint{2.569206in}{2.454784in}}{\pgfqpoint{2.558607in}{2.450394in}}{\pgfqpoint{2.550793in}{2.442580in}}%
\pgfpathcurveto{\pgfqpoint{2.542979in}{2.434767in}}{\pgfqpoint{2.538589in}{2.424168in}}{\pgfqpoint{2.538589in}{2.413117in}}%
\pgfpathcurveto{\pgfqpoint{2.538589in}{2.402067in}}{\pgfqpoint{2.542979in}{2.391468in}}{\pgfqpoint{2.550793in}{2.383655in}}%
\pgfpathcurveto{\pgfqpoint{2.558607in}{2.375841in}}{\pgfqpoint{2.569206in}{2.371451in}}{\pgfqpoint{2.580256in}{2.371451in}}%
\pgfpathclose%
\pgfusepath{stroke,fill}%
\end{pgfscope}%
\begin{pgfscope}%
\pgfpathrectangle{\pgfqpoint{0.375000in}{0.330000in}}{\pgfqpoint{2.325000in}{2.310000in}}%
\pgfusepath{clip}%
\pgfsetbuttcap%
\pgfsetroundjoin%
\definecolor{currentfill}{rgb}{0.000000,0.000000,0.000000}%
\pgfsetfillcolor{currentfill}%
\pgfsetlinewidth{1.003750pt}%
\definecolor{currentstroke}{rgb}{0.000000,0.000000,0.000000}%
\pgfsetstrokecolor{currentstroke}%
\pgfsetdash{}{0pt}%
\pgfpathmoveto{\pgfqpoint{2.580256in}{2.319420in}}%
\pgfpathcurveto{\pgfqpoint{2.591306in}{2.319420in}}{\pgfqpoint{2.601905in}{2.323810in}}{\pgfqpoint{2.609718in}{2.331624in}}%
\pgfpathcurveto{\pgfqpoint{2.617532in}{2.339437in}}{\pgfqpoint{2.621922in}{2.350037in}}{\pgfqpoint{2.621922in}{2.361087in}}%
\pgfpathcurveto{\pgfqpoint{2.621922in}{2.372137in}}{\pgfqpoint{2.617532in}{2.382736in}}{\pgfqpoint{2.609718in}{2.390549in}}%
\pgfpathcurveto{\pgfqpoint{2.601905in}{2.398363in}}{\pgfqpoint{2.591306in}{2.402753in}}{\pgfqpoint{2.580256in}{2.402753in}}%
\pgfpathcurveto{\pgfqpoint{2.569206in}{2.402753in}}{\pgfqpoint{2.558607in}{2.398363in}}{\pgfqpoint{2.550793in}{2.390549in}}%
\pgfpathcurveto{\pgfqpoint{2.542979in}{2.382736in}}{\pgfqpoint{2.538589in}{2.372137in}}{\pgfqpoint{2.538589in}{2.361087in}}%
\pgfpathcurveto{\pgfqpoint{2.538589in}{2.350037in}}{\pgfqpoint{2.542979in}{2.339437in}}{\pgfqpoint{2.550793in}{2.331624in}}%
\pgfpathcurveto{\pgfqpoint{2.558607in}{2.323810in}}{\pgfqpoint{2.569206in}{2.319420in}}{\pgfqpoint{2.580256in}{2.319420in}}%
\pgfpathclose%
\pgfusepath{stroke,fill}%
\end{pgfscope}%
\begin{pgfscope}%
\pgfpathrectangle{\pgfqpoint{0.375000in}{0.330000in}}{\pgfqpoint{2.325000in}{2.310000in}}%
\pgfusepath{clip}%
\pgfsetbuttcap%
\pgfsetroundjoin%
\definecolor{currentfill}{rgb}{0.000000,0.000000,0.000000}%
\pgfsetfillcolor{currentfill}%
\pgfsetlinewidth{1.003750pt}%
\definecolor{currentstroke}{rgb}{0.000000,0.000000,0.000000}%
\pgfsetstrokecolor{currentstroke}%
\pgfsetdash{}{0pt}%
\pgfpathmoveto{\pgfqpoint{2.580256in}{2.371451in}}%
\pgfpathcurveto{\pgfqpoint{2.591306in}{2.371451in}}{\pgfqpoint{2.601905in}{2.375841in}}{\pgfqpoint{2.609718in}{2.383655in}}%
\pgfpathcurveto{\pgfqpoint{2.617532in}{2.391468in}}{\pgfqpoint{2.621922in}{2.402067in}}{\pgfqpoint{2.621922in}{2.413117in}}%
\pgfpathcurveto{\pgfqpoint{2.621922in}{2.424168in}}{\pgfqpoint{2.617532in}{2.434767in}}{\pgfqpoint{2.609718in}{2.442580in}}%
\pgfpathcurveto{\pgfqpoint{2.601905in}{2.450394in}}{\pgfqpoint{2.591306in}{2.454784in}}{\pgfqpoint{2.580256in}{2.454784in}}%
\pgfpathcurveto{\pgfqpoint{2.569206in}{2.454784in}}{\pgfqpoint{2.558607in}{2.450394in}}{\pgfqpoint{2.550793in}{2.442580in}}%
\pgfpathcurveto{\pgfqpoint{2.542979in}{2.434767in}}{\pgfqpoint{2.538589in}{2.424168in}}{\pgfqpoint{2.538589in}{2.413117in}}%
\pgfpathcurveto{\pgfqpoint{2.538589in}{2.402067in}}{\pgfqpoint{2.542979in}{2.391468in}}{\pgfqpoint{2.550793in}{2.383655in}}%
\pgfpathcurveto{\pgfqpoint{2.558607in}{2.375841in}}{\pgfqpoint{2.569206in}{2.371451in}}{\pgfqpoint{2.580256in}{2.371451in}}%
\pgfpathclose%
\pgfusepath{stroke,fill}%
\end{pgfscope}%
\begin{pgfscope}%
\pgfpathrectangle{\pgfqpoint{0.375000in}{0.330000in}}{\pgfqpoint{2.325000in}{2.310000in}}%
\pgfusepath{clip}%
\pgfsetbuttcap%
\pgfsetroundjoin%
\definecolor{currentfill}{rgb}{0.000000,0.000000,0.000000}%
\pgfsetfillcolor{currentfill}%
\pgfsetlinewidth{1.003750pt}%
\definecolor{currentstroke}{rgb}{0.000000,0.000000,0.000000}%
\pgfsetstrokecolor{currentstroke}%
\pgfsetdash{}{0pt}%
\pgfpathmoveto{\pgfqpoint{2.580256in}{2.319420in}}%
\pgfpathcurveto{\pgfqpoint{2.591306in}{2.319420in}}{\pgfqpoint{2.601905in}{2.323810in}}{\pgfqpoint{2.609718in}{2.331624in}}%
\pgfpathcurveto{\pgfqpoint{2.617532in}{2.339437in}}{\pgfqpoint{2.621922in}{2.350037in}}{\pgfqpoint{2.621922in}{2.361087in}}%
\pgfpathcurveto{\pgfqpoint{2.621922in}{2.372137in}}{\pgfqpoint{2.617532in}{2.382736in}}{\pgfqpoint{2.609718in}{2.390549in}}%
\pgfpathcurveto{\pgfqpoint{2.601905in}{2.398363in}}{\pgfqpoint{2.591306in}{2.402753in}}{\pgfqpoint{2.580256in}{2.402753in}}%
\pgfpathcurveto{\pgfqpoint{2.569206in}{2.402753in}}{\pgfqpoint{2.558607in}{2.398363in}}{\pgfqpoint{2.550793in}{2.390549in}}%
\pgfpathcurveto{\pgfqpoint{2.542979in}{2.382736in}}{\pgfqpoint{2.538589in}{2.372137in}}{\pgfqpoint{2.538589in}{2.361087in}}%
\pgfpathcurveto{\pgfqpoint{2.538589in}{2.350037in}}{\pgfqpoint{2.542979in}{2.339437in}}{\pgfqpoint{2.550793in}{2.331624in}}%
\pgfpathcurveto{\pgfqpoint{2.558607in}{2.323810in}}{\pgfqpoint{2.569206in}{2.319420in}}{\pgfqpoint{2.580256in}{2.319420in}}%
\pgfpathclose%
\pgfusepath{stroke,fill}%
\end{pgfscope}%
\begin{pgfscope}%
\pgfpathrectangle{\pgfqpoint{0.375000in}{0.330000in}}{\pgfqpoint{2.325000in}{2.310000in}}%
\pgfusepath{clip}%
\pgfsetbuttcap%
\pgfsetroundjoin%
\definecolor{currentfill}{rgb}{0.000000,0.000000,0.000000}%
\pgfsetfillcolor{currentfill}%
\pgfsetlinewidth{1.003750pt}%
\definecolor{currentstroke}{rgb}{0.000000,0.000000,0.000000}%
\pgfsetstrokecolor{currentstroke}%
\pgfsetdash{}{0pt}%
\pgfpathmoveto{\pgfqpoint{2.580256in}{2.319420in}}%
\pgfpathcurveto{\pgfqpoint{2.591306in}{2.319420in}}{\pgfqpoint{2.601905in}{2.323810in}}{\pgfqpoint{2.609718in}{2.331624in}}%
\pgfpathcurveto{\pgfqpoint{2.617532in}{2.339437in}}{\pgfqpoint{2.621922in}{2.350037in}}{\pgfqpoint{2.621922in}{2.361087in}}%
\pgfpathcurveto{\pgfqpoint{2.621922in}{2.372137in}}{\pgfqpoint{2.617532in}{2.382736in}}{\pgfqpoint{2.609718in}{2.390549in}}%
\pgfpathcurveto{\pgfqpoint{2.601905in}{2.398363in}}{\pgfqpoint{2.591306in}{2.402753in}}{\pgfqpoint{2.580256in}{2.402753in}}%
\pgfpathcurveto{\pgfqpoint{2.569206in}{2.402753in}}{\pgfqpoint{2.558607in}{2.398363in}}{\pgfqpoint{2.550793in}{2.390549in}}%
\pgfpathcurveto{\pgfqpoint{2.542979in}{2.382736in}}{\pgfqpoint{2.538589in}{2.372137in}}{\pgfqpoint{2.538589in}{2.361087in}}%
\pgfpathcurveto{\pgfqpoint{2.538589in}{2.350037in}}{\pgfqpoint{2.542979in}{2.339437in}}{\pgfqpoint{2.550793in}{2.331624in}}%
\pgfpathcurveto{\pgfqpoint{2.558607in}{2.323810in}}{\pgfqpoint{2.569206in}{2.319420in}}{\pgfqpoint{2.580256in}{2.319420in}}%
\pgfpathclose%
\pgfusepath{stroke,fill}%
\end{pgfscope}%
\begin{pgfscope}%
\pgfpathrectangle{\pgfqpoint{0.375000in}{0.330000in}}{\pgfqpoint{2.325000in}{2.310000in}}%
\pgfusepath{clip}%
\pgfsetbuttcap%
\pgfsetroundjoin%
\definecolor{currentfill}{rgb}{0.000000,0.000000,0.000000}%
\pgfsetfillcolor{currentfill}%
\pgfsetlinewidth{1.003750pt}%
\definecolor{currentstroke}{rgb}{0.000000,0.000000,0.000000}%
\pgfsetstrokecolor{currentstroke}%
\pgfsetdash{}{0pt}%
\pgfpathmoveto{\pgfqpoint{2.580256in}{2.319420in}}%
\pgfpathcurveto{\pgfqpoint{2.591306in}{2.319420in}}{\pgfqpoint{2.601905in}{2.323810in}}{\pgfqpoint{2.609718in}{2.331624in}}%
\pgfpathcurveto{\pgfqpoint{2.617532in}{2.339437in}}{\pgfqpoint{2.621922in}{2.350037in}}{\pgfqpoint{2.621922in}{2.361087in}}%
\pgfpathcurveto{\pgfqpoint{2.621922in}{2.372137in}}{\pgfqpoint{2.617532in}{2.382736in}}{\pgfqpoint{2.609718in}{2.390549in}}%
\pgfpathcurveto{\pgfqpoint{2.601905in}{2.398363in}}{\pgfqpoint{2.591306in}{2.402753in}}{\pgfqpoint{2.580256in}{2.402753in}}%
\pgfpathcurveto{\pgfqpoint{2.569206in}{2.402753in}}{\pgfqpoint{2.558607in}{2.398363in}}{\pgfqpoint{2.550793in}{2.390549in}}%
\pgfpathcurveto{\pgfqpoint{2.542979in}{2.382736in}}{\pgfqpoint{2.538589in}{2.372137in}}{\pgfqpoint{2.538589in}{2.361087in}}%
\pgfpathcurveto{\pgfqpoint{2.538589in}{2.350037in}}{\pgfqpoint{2.542979in}{2.339437in}}{\pgfqpoint{2.550793in}{2.331624in}}%
\pgfpathcurveto{\pgfqpoint{2.558607in}{2.323810in}}{\pgfqpoint{2.569206in}{2.319420in}}{\pgfqpoint{2.580256in}{2.319420in}}%
\pgfpathclose%
\pgfusepath{stroke,fill}%
\end{pgfscope}%
\begin{pgfscope}%
\pgfpathrectangle{\pgfqpoint{0.375000in}{0.330000in}}{\pgfqpoint{2.325000in}{2.310000in}}%
\pgfusepath{clip}%
\pgfsetbuttcap%
\pgfsetroundjoin%
\definecolor{currentfill}{rgb}{0.000000,0.000000,0.000000}%
\pgfsetfillcolor{currentfill}%
\pgfsetlinewidth{1.003750pt}%
\definecolor{currentstroke}{rgb}{0.000000,0.000000,0.000000}%
\pgfsetstrokecolor{currentstroke}%
\pgfsetdash{}{0pt}%
\pgfpathmoveto{\pgfqpoint{2.580256in}{2.267389in}}%
\pgfpathcurveto{\pgfqpoint{2.591306in}{2.267389in}}{\pgfqpoint{2.601905in}{2.271779in}}{\pgfqpoint{2.609718in}{2.279593in}}%
\pgfpathcurveto{\pgfqpoint{2.617532in}{2.287407in}}{\pgfqpoint{2.621922in}{2.298006in}}{\pgfqpoint{2.621922in}{2.309056in}}%
\pgfpathcurveto{\pgfqpoint{2.621922in}{2.320106in}}{\pgfqpoint{2.617532in}{2.330705in}}{\pgfqpoint{2.609718in}{2.338519in}}%
\pgfpathcurveto{\pgfqpoint{2.601905in}{2.346332in}}{\pgfqpoint{2.591306in}{2.350722in}}{\pgfqpoint{2.580256in}{2.350722in}}%
\pgfpathcurveto{\pgfqpoint{2.569206in}{2.350722in}}{\pgfqpoint{2.558607in}{2.346332in}}{\pgfqpoint{2.550793in}{2.338519in}}%
\pgfpathcurveto{\pgfqpoint{2.542979in}{2.330705in}}{\pgfqpoint{2.538589in}{2.320106in}}{\pgfqpoint{2.538589in}{2.309056in}}%
\pgfpathcurveto{\pgfqpoint{2.538589in}{2.298006in}}{\pgfqpoint{2.542979in}{2.287407in}}{\pgfqpoint{2.550793in}{2.279593in}}%
\pgfpathcurveto{\pgfqpoint{2.558607in}{2.271779in}}{\pgfqpoint{2.569206in}{2.267389in}}{\pgfqpoint{2.580256in}{2.267389in}}%
\pgfpathclose%
\pgfusepath{stroke,fill}%
\end{pgfscope}%
\begin{pgfscope}%
\pgfpathrectangle{\pgfqpoint{0.375000in}{0.330000in}}{\pgfqpoint{2.325000in}{2.310000in}}%
\pgfusepath{clip}%
\pgfsetbuttcap%
\pgfsetroundjoin%
\definecolor{currentfill}{rgb}{0.000000,0.000000,0.000000}%
\pgfsetfillcolor{currentfill}%
\pgfsetlinewidth{1.003750pt}%
\definecolor{currentstroke}{rgb}{0.000000,0.000000,0.000000}%
\pgfsetstrokecolor{currentstroke}%
\pgfsetdash{}{0pt}%
\pgfpathmoveto{\pgfqpoint{2.580256in}{2.319420in}}%
\pgfpathcurveto{\pgfqpoint{2.591306in}{2.319420in}}{\pgfqpoint{2.601905in}{2.323810in}}{\pgfqpoint{2.609718in}{2.331624in}}%
\pgfpathcurveto{\pgfqpoint{2.617532in}{2.339437in}}{\pgfqpoint{2.621922in}{2.350037in}}{\pgfqpoint{2.621922in}{2.361087in}}%
\pgfpathcurveto{\pgfqpoint{2.621922in}{2.372137in}}{\pgfqpoint{2.617532in}{2.382736in}}{\pgfqpoint{2.609718in}{2.390549in}}%
\pgfpathcurveto{\pgfqpoint{2.601905in}{2.398363in}}{\pgfqpoint{2.591306in}{2.402753in}}{\pgfqpoint{2.580256in}{2.402753in}}%
\pgfpathcurveto{\pgfqpoint{2.569206in}{2.402753in}}{\pgfqpoint{2.558607in}{2.398363in}}{\pgfqpoint{2.550793in}{2.390549in}}%
\pgfpathcurveto{\pgfqpoint{2.542979in}{2.382736in}}{\pgfqpoint{2.538589in}{2.372137in}}{\pgfqpoint{2.538589in}{2.361087in}}%
\pgfpathcurveto{\pgfqpoint{2.538589in}{2.350037in}}{\pgfqpoint{2.542979in}{2.339437in}}{\pgfqpoint{2.550793in}{2.331624in}}%
\pgfpathcurveto{\pgfqpoint{2.558607in}{2.323810in}}{\pgfqpoint{2.569206in}{2.319420in}}{\pgfqpoint{2.580256in}{2.319420in}}%
\pgfpathclose%
\pgfusepath{stroke,fill}%
\end{pgfscope}%
\begin{pgfscope}%
\pgfpathrectangle{\pgfqpoint{0.375000in}{0.330000in}}{\pgfqpoint{2.325000in}{2.310000in}}%
\pgfusepath{clip}%
\pgfsetbuttcap%
\pgfsetroundjoin%
\definecolor{currentfill}{rgb}{0.000000,0.000000,0.000000}%
\pgfsetfillcolor{currentfill}%
\pgfsetlinewidth{1.003750pt}%
\definecolor{currentstroke}{rgb}{0.000000,0.000000,0.000000}%
\pgfsetstrokecolor{currentstroke}%
\pgfsetdash{}{0pt}%
\pgfpathmoveto{\pgfqpoint{2.580256in}{2.319420in}}%
\pgfpathcurveto{\pgfqpoint{2.591306in}{2.319420in}}{\pgfqpoint{2.601905in}{2.323810in}}{\pgfqpoint{2.609718in}{2.331624in}}%
\pgfpathcurveto{\pgfqpoint{2.617532in}{2.339437in}}{\pgfqpoint{2.621922in}{2.350037in}}{\pgfqpoint{2.621922in}{2.361087in}}%
\pgfpathcurveto{\pgfqpoint{2.621922in}{2.372137in}}{\pgfqpoint{2.617532in}{2.382736in}}{\pgfqpoint{2.609718in}{2.390549in}}%
\pgfpathcurveto{\pgfqpoint{2.601905in}{2.398363in}}{\pgfqpoint{2.591306in}{2.402753in}}{\pgfqpoint{2.580256in}{2.402753in}}%
\pgfpathcurveto{\pgfqpoint{2.569206in}{2.402753in}}{\pgfqpoint{2.558607in}{2.398363in}}{\pgfqpoint{2.550793in}{2.390549in}}%
\pgfpathcurveto{\pgfqpoint{2.542979in}{2.382736in}}{\pgfqpoint{2.538589in}{2.372137in}}{\pgfqpoint{2.538589in}{2.361087in}}%
\pgfpathcurveto{\pgfqpoint{2.538589in}{2.350037in}}{\pgfqpoint{2.542979in}{2.339437in}}{\pgfqpoint{2.550793in}{2.331624in}}%
\pgfpathcurveto{\pgfqpoint{2.558607in}{2.323810in}}{\pgfqpoint{2.569206in}{2.319420in}}{\pgfqpoint{2.580256in}{2.319420in}}%
\pgfpathclose%
\pgfusepath{stroke,fill}%
\end{pgfscope}%
\begin{pgfscope}%
\pgfpathrectangle{\pgfqpoint{0.375000in}{0.330000in}}{\pgfqpoint{2.325000in}{2.310000in}}%
\pgfusepath{clip}%
\pgfsetbuttcap%
\pgfsetroundjoin%
\definecolor{currentfill}{rgb}{0.000000,0.000000,0.000000}%
\pgfsetfillcolor{currentfill}%
\pgfsetlinewidth{1.003750pt}%
\definecolor{currentstroke}{rgb}{0.000000,0.000000,0.000000}%
\pgfsetstrokecolor{currentstroke}%
\pgfsetdash{}{0pt}%
\pgfpathmoveto{\pgfqpoint{2.580256in}{2.267389in}}%
\pgfpathcurveto{\pgfqpoint{2.591306in}{2.267389in}}{\pgfqpoint{2.601905in}{2.271779in}}{\pgfqpoint{2.609718in}{2.279593in}}%
\pgfpathcurveto{\pgfqpoint{2.617532in}{2.287407in}}{\pgfqpoint{2.621922in}{2.298006in}}{\pgfqpoint{2.621922in}{2.309056in}}%
\pgfpathcurveto{\pgfqpoint{2.621922in}{2.320106in}}{\pgfqpoint{2.617532in}{2.330705in}}{\pgfqpoint{2.609718in}{2.338519in}}%
\pgfpathcurveto{\pgfqpoint{2.601905in}{2.346332in}}{\pgfqpoint{2.591306in}{2.350722in}}{\pgfqpoint{2.580256in}{2.350722in}}%
\pgfpathcurveto{\pgfqpoint{2.569206in}{2.350722in}}{\pgfqpoint{2.558607in}{2.346332in}}{\pgfqpoint{2.550793in}{2.338519in}}%
\pgfpathcurveto{\pgfqpoint{2.542979in}{2.330705in}}{\pgfqpoint{2.538589in}{2.320106in}}{\pgfqpoint{2.538589in}{2.309056in}}%
\pgfpathcurveto{\pgfqpoint{2.538589in}{2.298006in}}{\pgfqpoint{2.542979in}{2.287407in}}{\pgfqpoint{2.550793in}{2.279593in}}%
\pgfpathcurveto{\pgfqpoint{2.558607in}{2.271779in}}{\pgfqpoint{2.569206in}{2.267389in}}{\pgfqpoint{2.580256in}{2.267389in}}%
\pgfpathclose%
\pgfusepath{stroke,fill}%
\end{pgfscope}%
\begin{pgfscope}%
\pgfpathrectangle{\pgfqpoint{0.375000in}{0.330000in}}{\pgfqpoint{2.325000in}{2.310000in}}%
\pgfusepath{clip}%
\pgfsetbuttcap%
\pgfsetroundjoin%
\definecolor{currentfill}{rgb}{0.000000,0.000000,0.000000}%
\pgfsetfillcolor{currentfill}%
\pgfsetlinewidth{1.003750pt}%
\definecolor{currentstroke}{rgb}{0.000000,0.000000,0.000000}%
\pgfsetstrokecolor{currentstroke}%
\pgfsetdash{}{0pt}%
\pgfpathmoveto{\pgfqpoint{2.580256in}{2.319420in}}%
\pgfpathcurveto{\pgfqpoint{2.591306in}{2.319420in}}{\pgfqpoint{2.601905in}{2.323810in}}{\pgfqpoint{2.609718in}{2.331624in}}%
\pgfpathcurveto{\pgfqpoint{2.617532in}{2.339437in}}{\pgfqpoint{2.621922in}{2.350037in}}{\pgfqpoint{2.621922in}{2.361087in}}%
\pgfpathcurveto{\pgfqpoint{2.621922in}{2.372137in}}{\pgfqpoint{2.617532in}{2.382736in}}{\pgfqpoint{2.609718in}{2.390549in}}%
\pgfpathcurveto{\pgfqpoint{2.601905in}{2.398363in}}{\pgfqpoint{2.591306in}{2.402753in}}{\pgfqpoint{2.580256in}{2.402753in}}%
\pgfpathcurveto{\pgfqpoint{2.569206in}{2.402753in}}{\pgfqpoint{2.558607in}{2.398363in}}{\pgfqpoint{2.550793in}{2.390549in}}%
\pgfpathcurveto{\pgfqpoint{2.542979in}{2.382736in}}{\pgfqpoint{2.538589in}{2.372137in}}{\pgfqpoint{2.538589in}{2.361087in}}%
\pgfpathcurveto{\pgfqpoint{2.538589in}{2.350037in}}{\pgfqpoint{2.542979in}{2.339437in}}{\pgfqpoint{2.550793in}{2.331624in}}%
\pgfpathcurveto{\pgfqpoint{2.558607in}{2.323810in}}{\pgfqpoint{2.569206in}{2.319420in}}{\pgfqpoint{2.580256in}{2.319420in}}%
\pgfpathclose%
\pgfusepath{stroke,fill}%
\end{pgfscope}%
\begin{pgfscope}%
\pgfpathrectangle{\pgfqpoint{0.375000in}{0.330000in}}{\pgfqpoint{2.325000in}{2.310000in}}%
\pgfusepath{clip}%
\pgfsetbuttcap%
\pgfsetroundjoin%
\definecolor{currentfill}{rgb}{0.000000,0.000000,0.000000}%
\pgfsetfillcolor{currentfill}%
\pgfsetlinewidth{1.003750pt}%
\definecolor{currentstroke}{rgb}{0.000000,0.000000,0.000000}%
\pgfsetstrokecolor{currentstroke}%
\pgfsetdash{}{0pt}%
\pgfpathmoveto{\pgfqpoint{2.580256in}{2.371451in}}%
\pgfpathcurveto{\pgfqpoint{2.591306in}{2.371451in}}{\pgfqpoint{2.601905in}{2.375841in}}{\pgfqpoint{2.609718in}{2.383655in}}%
\pgfpathcurveto{\pgfqpoint{2.617532in}{2.391468in}}{\pgfqpoint{2.621922in}{2.402067in}}{\pgfqpoint{2.621922in}{2.413117in}}%
\pgfpathcurveto{\pgfqpoint{2.621922in}{2.424168in}}{\pgfqpoint{2.617532in}{2.434767in}}{\pgfqpoint{2.609718in}{2.442580in}}%
\pgfpathcurveto{\pgfqpoint{2.601905in}{2.450394in}}{\pgfqpoint{2.591306in}{2.454784in}}{\pgfqpoint{2.580256in}{2.454784in}}%
\pgfpathcurveto{\pgfqpoint{2.569206in}{2.454784in}}{\pgfqpoint{2.558607in}{2.450394in}}{\pgfqpoint{2.550793in}{2.442580in}}%
\pgfpathcurveto{\pgfqpoint{2.542979in}{2.434767in}}{\pgfqpoint{2.538589in}{2.424168in}}{\pgfqpoint{2.538589in}{2.413117in}}%
\pgfpathcurveto{\pgfqpoint{2.538589in}{2.402067in}}{\pgfqpoint{2.542979in}{2.391468in}}{\pgfqpoint{2.550793in}{2.383655in}}%
\pgfpathcurveto{\pgfqpoint{2.558607in}{2.375841in}}{\pgfqpoint{2.569206in}{2.371451in}}{\pgfqpoint{2.580256in}{2.371451in}}%
\pgfpathclose%
\pgfusepath{stroke,fill}%
\end{pgfscope}%
\begin{pgfscope}%
\pgfpathrectangle{\pgfqpoint{0.375000in}{0.330000in}}{\pgfqpoint{2.325000in}{2.310000in}}%
\pgfusepath{clip}%
\pgfsetbuttcap%
\pgfsetroundjoin%
\definecolor{currentfill}{rgb}{0.000000,0.000000,0.000000}%
\pgfsetfillcolor{currentfill}%
\pgfsetlinewidth{1.003750pt}%
\definecolor{currentstroke}{rgb}{0.000000,0.000000,0.000000}%
\pgfsetstrokecolor{currentstroke}%
\pgfsetdash{}{0pt}%
\pgfpathmoveto{\pgfqpoint{2.580256in}{2.267389in}}%
\pgfpathcurveto{\pgfqpoint{2.591306in}{2.267389in}}{\pgfqpoint{2.601905in}{2.271779in}}{\pgfqpoint{2.609718in}{2.279593in}}%
\pgfpathcurveto{\pgfqpoint{2.617532in}{2.287407in}}{\pgfqpoint{2.621922in}{2.298006in}}{\pgfqpoint{2.621922in}{2.309056in}}%
\pgfpathcurveto{\pgfqpoint{2.621922in}{2.320106in}}{\pgfqpoint{2.617532in}{2.330705in}}{\pgfqpoint{2.609718in}{2.338519in}}%
\pgfpathcurveto{\pgfqpoint{2.601905in}{2.346332in}}{\pgfqpoint{2.591306in}{2.350722in}}{\pgfqpoint{2.580256in}{2.350722in}}%
\pgfpathcurveto{\pgfqpoint{2.569206in}{2.350722in}}{\pgfqpoint{2.558607in}{2.346332in}}{\pgfqpoint{2.550793in}{2.338519in}}%
\pgfpathcurveto{\pgfqpoint{2.542979in}{2.330705in}}{\pgfqpoint{2.538589in}{2.320106in}}{\pgfqpoint{2.538589in}{2.309056in}}%
\pgfpathcurveto{\pgfqpoint{2.538589in}{2.298006in}}{\pgfqpoint{2.542979in}{2.287407in}}{\pgfqpoint{2.550793in}{2.279593in}}%
\pgfpathcurveto{\pgfqpoint{2.558607in}{2.271779in}}{\pgfqpoint{2.569206in}{2.267389in}}{\pgfqpoint{2.580256in}{2.267389in}}%
\pgfpathclose%
\pgfusepath{stroke,fill}%
\end{pgfscope}%
\begin{pgfscope}%
\pgfpathrectangle{\pgfqpoint{0.375000in}{0.330000in}}{\pgfqpoint{2.325000in}{2.310000in}}%
\pgfusepath{clip}%
\pgfsetbuttcap%
\pgfsetroundjoin%
\definecolor{currentfill}{rgb}{0.000000,0.000000,0.000000}%
\pgfsetfillcolor{currentfill}%
\pgfsetlinewidth{1.003750pt}%
\definecolor{currentstroke}{rgb}{0.000000,0.000000,0.000000}%
\pgfsetstrokecolor{currentstroke}%
\pgfsetdash{}{0pt}%
\pgfpathmoveto{\pgfqpoint{2.580256in}{2.267389in}}%
\pgfpathcurveto{\pgfqpoint{2.591306in}{2.267389in}}{\pgfqpoint{2.601905in}{2.271779in}}{\pgfqpoint{2.609718in}{2.279593in}}%
\pgfpathcurveto{\pgfqpoint{2.617532in}{2.287407in}}{\pgfqpoint{2.621922in}{2.298006in}}{\pgfqpoint{2.621922in}{2.309056in}}%
\pgfpathcurveto{\pgfqpoint{2.621922in}{2.320106in}}{\pgfqpoint{2.617532in}{2.330705in}}{\pgfqpoint{2.609718in}{2.338519in}}%
\pgfpathcurveto{\pgfqpoint{2.601905in}{2.346332in}}{\pgfqpoint{2.591306in}{2.350722in}}{\pgfqpoint{2.580256in}{2.350722in}}%
\pgfpathcurveto{\pgfqpoint{2.569206in}{2.350722in}}{\pgfqpoint{2.558607in}{2.346332in}}{\pgfqpoint{2.550793in}{2.338519in}}%
\pgfpathcurveto{\pgfqpoint{2.542979in}{2.330705in}}{\pgfqpoint{2.538589in}{2.320106in}}{\pgfqpoint{2.538589in}{2.309056in}}%
\pgfpathcurveto{\pgfqpoint{2.538589in}{2.298006in}}{\pgfqpoint{2.542979in}{2.287407in}}{\pgfqpoint{2.550793in}{2.279593in}}%
\pgfpathcurveto{\pgfqpoint{2.558607in}{2.271779in}}{\pgfqpoint{2.569206in}{2.267389in}}{\pgfqpoint{2.580256in}{2.267389in}}%
\pgfpathclose%
\pgfusepath{stroke,fill}%
\end{pgfscope}%
\begin{pgfscope}%
\pgfpathrectangle{\pgfqpoint{0.375000in}{0.330000in}}{\pgfqpoint{2.325000in}{2.310000in}}%
\pgfusepath{clip}%
\pgfsetbuttcap%
\pgfsetroundjoin%
\definecolor{currentfill}{rgb}{0.000000,0.000000,0.000000}%
\pgfsetfillcolor{currentfill}%
\pgfsetlinewidth{1.003750pt}%
\definecolor{currentstroke}{rgb}{0.000000,0.000000,0.000000}%
\pgfsetstrokecolor{currentstroke}%
\pgfsetdash{}{0pt}%
\pgfpathmoveto{\pgfqpoint{2.580256in}{2.319420in}}%
\pgfpathcurveto{\pgfqpoint{2.591306in}{2.319420in}}{\pgfqpoint{2.601905in}{2.323810in}}{\pgfqpoint{2.609718in}{2.331624in}}%
\pgfpathcurveto{\pgfqpoint{2.617532in}{2.339437in}}{\pgfqpoint{2.621922in}{2.350037in}}{\pgfqpoint{2.621922in}{2.361087in}}%
\pgfpathcurveto{\pgfqpoint{2.621922in}{2.372137in}}{\pgfqpoint{2.617532in}{2.382736in}}{\pgfqpoint{2.609718in}{2.390549in}}%
\pgfpathcurveto{\pgfqpoint{2.601905in}{2.398363in}}{\pgfqpoint{2.591306in}{2.402753in}}{\pgfqpoint{2.580256in}{2.402753in}}%
\pgfpathcurveto{\pgfqpoint{2.569206in}{2.402753in}}{\pgfqpoint{2.558607in}{2.398363in}}{\pgfqpoint{2.550793in}{2.390549in}}%
\pgfpathcurveto{\pgfqpoint{2.542979in}{2.382736in}}{\pgfqpoint{2.538589in}{2.372137in}}{\pgfqpoint{2.538589in}{2.361087in}}%
\pgfpathcurveto{\pgfqpoint{2.538589in}{2.350037in}}{\pgfqpoint{2.542979in}{2.339437in}}{\pgfqpoint{2.550793in}{2.331624in}}%
\pgfpathcurveto{\pgfqpoint{2.558607in}{2.323810in}}{\pgfqpoint{2.569206in}{2.319420in}}{\pgfqpoint{2.580256in}{2.319420in}}%
\pgfpathclose%
\pgfusepath{stroke,fill}%
\end{pgfscope}%
\begin{pgfscope}%
\pgfpathrectangle{\pgfqpoint{0.375000in}{0.330000in}}{\pgfqpoint{2.325000in}{2.310000in}}%
\pgfusepath{clip}%
\pgfsetbuttcap%
\pgfsetroundjoin%
\definecolor{currentfill}{rgb}{0.000000,0.000000,0.000000}%
\pgfsetfillcolor{currentfill}%
\pgfsetlinewidth{1.003750pt}%
\definecolor{currentstroke}{rgb}{0.000000,0.000000,0.000000}%
\pgfsetstrokecolor{currentstroke}%
\pgfsetdash{}{0pt}%
\pgfpathmoveto{\pgfqpoint{2.580256in}{2.371451in}}%
\pgfpathcurveto{\pgfqpoint{2.591306in}{2.371451in}}{\pgfqpoint{2.601905in}{2.375841in}}{\pgfqpoint{2.609718in}{2.383655in}}%
\pgfpathcurveto{\pgfqpoint{2.617532in}{2.391468in}}{\pgfqpoint{2.621922in}{2.402067in}}{\pgfqpoint{2.621922in}{2.413117in}}%
\pgfpathcurveto{\pgfqpoint{2.621922in}{2.424168in}}{\pgfqpoint{2.617532in}{2.434767in}}{\pgfqpoint{2.609718in}{2.442580in}}%
\pgfpathcurveto{\pgfqpoint{2.601905in}{2.450394in}}{\pgfqpoint{2.591306in}{2.454784in}}{\pgfqpoint{2.580256in}{2.454784in}}%
\pgfpathcurveto{\pgfqpoint{2.569206in}{2.454784in}}{\pgfqpoint{2.558607in}{2.450394in}}{\pgfqpoint{2.550793in}{2.442580in}}%
\pgfpathcurveto{\pgfqpoint{2.542979in}{2.434767in}}{\pgfqpoint{2.538589in}{2.424168in}}{\pgfqpoint{2.538589in}{2.413117in}}%
\pgfpathcurveto{\pgfqpoint{2.538589in}{2.402067in}}{\pgfqpoint{2.542979in}{2.391468in}}{\pgfqpoint{2.550793in}{2.383655in}}%
\pgfpathcurveto{\pgfqpoint{2.558607in}{2.375841in}}{\pgfqpoint{2.569206in}{2.371451in}}{\pgfqpoint{2.580256in}{2.371451in}}%
\pgfpathclose%
\pgfusepath{stroke,fill}%
\end{pgfscope}%
\begin{pgfscope}%
\pgfpathrectangle{\pgfqpoint{0.375000in}{0.330000in}}{\pgfqpoint{2.325000in}{2.310000in}}%
\pgfusepath{clip}%
\pgfsetbuttcap%
\pgfsetroundjoin%
\definecolor{currentfill}{rgb}{0.000000,0.000000,0.000000}%
\pgfsetfillcolor{currentfill}%
\pgfsetlinewidth{1.003750pt}%
\definecolor{currentstroke}{rgb}{0.000000,0.000000,0.000000}%
\pgfsetstrokecolor{currentstroke}%
\pgfsetdash{}{0pt}%
\pgfpathmoveto{\pgfqpoint{2.580256in}{2.319420in}}%
\pgfpathcurveto{\pgfqpoint{2.591306in}{2.319420in}}{\pgfqpoint{2.601905in}{2.323810in}}{\pgfqpoint{2.609718in}{2.331624in}}%
\pgfpathcurveto{\pgfqpoint{2.617532in}{2.339437in}}{\pgfqpoint{2.621922in}{2.350037in}}{\pgfqpoint{2.621922in}{2.361087in}}%
\pgfpathcurveto{\pgfqpoint{2.621922in}{2.372137in}}{\pgfqpoint{2.617532in}{2.382736in}}{\pgfqpoint{2.609718in}{2.390549in}}%
\pgfpathcurveto{\pgfqpoint{2.601905in}{2.398363in}}{\pgfqpoint{2.591306in}{2.402753in}}{\pgfqpoint{2.580256in}{2.402753in}}%
\pgfpathcurveto{\pgfqpoint{2.569206in}{2.402753in}}{\pgfqpoint{2.558607in}{2.398363in}}{\pgfqpoint{2.550793in}{2.390549in}}%
\pgfpathcurveto{\pgfqpoint{2.542979in}{2.382736in}}{\pgfqpoint{2.538589in}{2.372137in}}{\pgfqpoint{2.538589in}{2.361087in}}%
\pgfpathcurveto{\pgfqpoint{2.538589in}{2.350037in}}{\pgfqpoint{2.542979in}{2.339437in}}{\pgfqpoint{2.550793in}{2.331624in}}%
\pgfpathcurveto{\pgfqpoint{2.558607in}{2.323810in}}{\pgfqpoint{2.569206in}{2.319420in}}{\pgfqpoint{2.580256in}{2.319420in}}%
\pgfpathclose%
\pgfusepath{stroke,fill}%
\end{pgfscope}%
\begin{pgfscope}%
\pgfpathrectangle{\pgfqpoint{0.375000in}{0.330000in}}{\pgfqpoint{2.325000in}{2.310000in}}%
\pgfusepath{clip}%
\pgfsetbuttcap%
\pgfsetroundjoin%
\definecolor{currentfill}{rgb}{0.000000,0.000000,0.000000}%
\pgfsetfillcolor{currentfill}%
\pgfsetlinewidth{1.003750pt}%
\definecolor{currentstroke}{rgb}{0.000000,0.000000,0.000000}%
\pgfsetstrokecolor{currentstroke}%
\pgfsetdash{}{0pt}%
\pgfpathmoveto{\pgfqpoint{2.580256in}{2.215358in}}%
\pgfpathcurveto{\pgfqpoint{2.591306in}{2.215358in}}{\pgfqpoint{2.601905in}{2.219749in}}{\pgfqpoint{2.609718in}{2.227562in}}%
\pgfpathcurveto{\pgfqpoint{2.617532in}{2.235376in}}{\pgfqpoint{2.621922in}{2.245975in}}{\pgfqpoint{2.621922in}{2.257025in}}%
\pgfpathcurveto{\pgfqpoint{2.621922in}{2.268075in}}{\pgfqpoint{2.617532in}{2.278674in}}{\pgfqpoint{2.609718in}{2.286488in}}%
\pgfpathcurveto{\pgfqpoint{2.601905in}{2.294301in}}{\pgfqpoint{2.591306in}{2.298692in}}{\pgfqpoint{2.580256in}{2.298692in}}%
\pgfpathcurveto{\pgfqpoint{2.569206in}{2.298692in}}{\pgfqpoint{2.558607in}{2.294301in}}{\pgfqpoint{2.550793in}{2.286488in}}%
\pgfpathcurveto{\pgfqpoint{2.542979in}{2.278674in}}{\pgfqpoint{2.538589in}{2.268075in}}{\pgfqpoint{2.538589in}{2.257025in}}%
\pgfpathcurveto{\pgfqpoint{2.538589in}{2.245975in}}{\pgfqpoint{2.542979in}{2.235376in}}{\pgfqpoint{2.550793in}{2.227562in}}%
\pgfpathcurveto{\pgfqpoint{2.558607in}{2.219749in}}{\pgfqpoint{2.569206in}{2.215358in}}{\pgfqpoint{2.580256in}{2.215358in}}%
\pgfpathclose%
\pgfusepath{stroke,fill}%
\end{pgfscope}%
\begin{pgfscope}%
\pgfpathrectangle{\pgfqpoint{0.375000in}{0.330000in}}{\pgfqpoint{2.325000in}{2.310000in}}%
\pgfusepath{clip}%
\pgfsetbuttcap%
\pgfsetroundjoin%
\definecolor{currentfill}{rgb}{0.000000,0.000000,0.000000}%
\pgfsetfillcolor{currentfill}%
\pgfsetlinewidth{1.003750pt}%
\definecolor{currentstroke}{rgb}{0.000000,0.000000,0.000000}%
\pgfsetstrokecolor{currentstroke}%
\pgfsetdash{}{0pt}%
\pgfpathmoveto{\pgfqpoint{2.580256in}{2.215358in}}%
\pgfpathcurveto{\pgfqpoint{2.591306in}{2.215358in}}{\pgfqpoint{2.601905in}{2.219749in}}{\pgfqpoint{2.609718in}{2.227562in}}%
\pgfpathcurveto{\pgfqpoint{2.617532in}{2.235376in}}{\pgfqpoint{2.621922in}{2.245975in}}{\pgfqpoint{2.621922in}{2.257025in}}%
\pgfpathcurveto{\pgfqpoint{2.621922in}{2.268075in}}{\pgfqpoint{2.617532in}{2.278674in}}{\pgfqpoint{2.609718in}{2.286488in}}%
\pgfpathcurveto{\pgfqpoint{2.601905in}{2.294301in}}{\pgfqpoint{2.591306in}{2.298692in}}{\pgfqpoint{2.580256in}{2.298692in}}%
\pgfpathcurveto{\pgfqpoint{2.569206in}{2.298692in}}{\pgfqpoint{2.558607in}{2.294301in}}{\pgfqpoint{2.550793in}{2.286488in}}%
\pgfpathcurveto{\pgfqpoint{2.542979in}{2.278674in}}{\pgfqpoint{2.538589in}{2.268075in}}{\pgfqpoint{2.538589in}{2.257025in}}%
\pgfpathcurveto{\pgfqpoint{2.538589in}{2.245975in}}{\pgfqpoint{2.542979in}{2.235376in}}{\pgfqpoint{2.550793in}{2.227562in}}%
\pgfpathcurveto{\pgfqpoint{2.558607in}{2.219749in}}{\pgfqpoint{2.569206in}{2.215358in}}{\pgfqpoint{2.580256in}{2.215358in}}%
\pgfpathclose%
\pgfusepath{stroke,fill}%
\end{pgfscope}%
\begin{pgfscope}%
\pgfpathrectangle{\pgfqpoint{0.375000in}{0.330000in}}{\pgfqpoint{2.325000in}{2.310000in}}%
\pgfusepath{clip}%
\pgfsetbuttcap%
\pgfsetroundjoin%
\definecolor{currentfill}{rgb}{0.000000,0.000000,0.000000}%
\pgfsetfillcolor{currentfill}%
\pgfsetlinewidth{1.003750pt}%
\definecolor{currentstroke}{rgb}{0.000000,0.000000,0.000000}%
\pgfsetstrokecolor{currentstroke}%
\pgfsetdash{}{0pt}%
\pgfpathmoveto{\pgfqpoint{2.580256in}{2.267389in}}%
\pgfpathcurveto{\pgfqpoint{2.591306in}{2.267389in}}{\pgfqpoint{2.601905in}{2.271779in}}{\pgfqpoint{2.609718in}{2.279593in}}%
\pgfpathcurveto{\pgfqpoint{2.617532in}{2.287407in}}{\pgfqpoint{2.621922in}{2.298006in}}{\pgfqpoint{2.621922in}{2.309056in}}%
\pgfpathcurveto{\pgfqpoint{2.621922in}{2.320106in}}{\pgfqpoint{2.617532in}{2.330705in}}{\pgfqpoint{2.609718in}{2.338519in}}%
\pgfpathcurveto{\pgfqpoint{2.601905in}{2.346332in}}{\pgfqpoint{2.591306in}{2.350722in}}{\pgfqpoint{2.580256in}{2.350722in}}%
\pgfpathcurveto{\pgfqpoint{2.569206in}{2.350722in}}{\pgfqpoint{2.558607in}{2.346332in}}{\pgfqpoint{2.550793in}{2.338519in}}%
\pgfpathcurveto{\pgfqpoint{2.542979in}{2.330705in}}{\pgfqpoint{2.538589in}{2.320106in}}{\pgfqpoint{2.538589in}{2.309056in}}%
\pgfpathcurveto{\pgfqpoint{2.538589in}{2.298006in}}{\pgfqpoint{2.542979in}{2.287407in}}{\pgfqpoint{2.550793in}{2.279593in}}%
\pgfpathcurveto{\pgfqpoint{2.558607in}{2.271779in}}{\pgfqpoint{2.569206in}{2.267389in}}{\pgfqpoint{2.580256in}{2.267389in}}%
\pgfpathclose%
\pgfusepath{stroke,fill}%
\end{pgfscope}%
\begin{pgfscope}%
\pgfpathrectangle{\pgfqpoint{0.375000in}{0.330000in}}{\pgfqpoint{2.325000in}{2.310000in}}%
\pgfusepath{clip}%
\pgfsetbuttcap%
\pgfsetroundjoin%
\definecolor{currentfill}{rgb}{0.000000,0.000000,0.000000}%
\pgfsetfillcolor{currentfill}%
\pgfsetlinewidth{1.003750pt}%
\definecolor{currentstroke}{rgb}{0.000000,0.000000,0.000000}%
\pgfsetstrokecolor{currentstroke}%
\pgfsetdash{}{0pt}%
\pgfpathmoveto{\pgfqpoint{2.580256in}{2.267389in}}%
\pgfpathcurveto{\pgfqpoint{2.591306in}{2.267389in}}{\pgfqpoint{2.601905in}{2.271779in}}{\pgfqpoint{2.609718in}{2.279593in}}%
\pgfpathcurveto{\pgfqpoint{2.617532in}{2.287407in}}{\pgfqpoint{2.621922in}{2.298006in}}{\pgfqpoint{2.621922in}{2.309056in}}%
\pgfpathcurveto{\pgfqpoint{2.621922in}{2.320106in}}{\pgfqpoint{2.617532in}{2.330705in}}{\pgfqpoint{2.609718in}{2.338519in}}%
\pgfpathcurveto{\pgfqpoint{2.601905in}{2.346332in}}{\pgfqpoint{2.591306in}{2.350722in}}{\pgfqpoint{2.580256in}{2.350722in}}%
\pgfpathcurveto{\pgfqpoint{2.569206in}{2.350722in}}{\pgfqpoint{2.558607in}{2.346332in}}{\pgfqpoint{2.550793in}{2.338519in}}%
\pgfpathcurveto{\pgfqpoint{2.542979in}{2.330705in}}{\pgfqpoint{2.538589in}{2.320106in}}{\pgfqpoint{2.538589in}{2.309056in}}%
\pgfpathcurveto{\pgfqpoint{2.538589in}{2.298006in}}{\pgfqpoint{2.542979in}{2.287407in}}{\pgfqpoint{2.550793in}{2.279593in}}%
\pgfpathcurveto{\pgfqpoint{2.558607in}{2.271779in}}{\pgfqpoint{2.569206in}{2.267389in}}{\pgfqpoint{2.580256in}{2.267389in}}%
\pgfpathclose%
\pgfusepath{stroke,fill}%
\end{pgfscope}%
\begin{pgfscope}%
\pgfpathrectangle{\pgfqpoint{0.375000in}{0.330000in}}{\pgfqpoint{2.325000in}{2.310000in}}%
\pgfusepath{clip}%
\pgfsetbuttcap%
\pgfsetroundjoin%
\definecolor{currentfill}{rgb}{0.000000,0.000000,0.000000}%
\pgfsetfillcolor{currentfill}%
\pgfsetlinewidth{1.003750pt}%
\definecolor{currentstroke}{rgb}{0.000000,0.000000,0.000000}%
\pgfsetstrokecolor{currentstroke}%
\pgfsetdash{}{0pt}%
\pgfpathmoveto{\pgfqpoint{2.580256in}{2.215358in}}%
\pgfpathcurveto{\pgfqpoint{2.591306in}{2.215358in}}{\pgfqpoint{2.601905in}{2.219749in}}{\pgfqpoint{2.609718in}{2.227562in}}%
\pgfpathcurveto{\pgfqpoint{2.617532in}{2.235376in}}{\pgfqpoint{2.621922in}{2.245975in}}{\pgfqpoint{2.621922in}{2.257025in}}%
\pgfpathcurveto{\pgfqpoint{2.621922in}{2.268075in}}{\pgfqpoint{2.617532in}{2.278674in}}{\pgfqpoint{2.609718in}{2.286488in}}%
\pgfpathcurveto{\pgfqpoint{2.601905in}{2.294301in}}{\pgfqpoint{2.591306in}{2.298692in}}{\pgfqpoint{2.580256in}{2.298692in}}%
\pgfpathcurveto{\pgfqpoint{2.569206in}{2.298692in}}{\pgfqpoint{2.558607in}{2.294301in}}{\pgfqpoint{2.550793in}{2.286488in}}%
\pgfpathcurveto{\pgfqpoint{2.542979in}{2.278674in}}{\pgfqpoint{2.538589in}{2.268075in}}{\pgfqpoint{2.538589in}{2.257025in}}%
\pgfpathcurveto{\pgfqpoint{2.538589in}{2.245975in}}{\pgfqpoint{2.542979in}{2.235376in}}{\pgfqpoint{2.550793in}{2.227562in}}%
\pgfpathcurveto{\pgfqpoint{2.558607in}{2.219749in}}{\pgfqpoint{2.569206in}{2.215358in}}{\pgfqpoint{2.580256in}{2.215358in}}%
\pgfpathclose%
\pgfusepath{stroke,fill}%
\end{pgfscope}%
\begin{pgfscope}%
\pgfpathrectangle{\pgfqpoint{0.375000in}{0.330000in}}{\pgfqpoint{2.325000in}{2.310000in}}%
\pgfusepath{clip}%
\pgfsetbuttcap%
\pgfsetroundjoin%
\definecolor{currentfill}{rgb}{0.000000,0.000000,0.000000}%
\pgfsetfillcolor{currentfill}%
\pgfsetlinewidth{1.003750pt}%
\definecolor{currentstroke}{rgb}{0.000000,0.000000,0.000000}%
\pgfsetstrokecolor{currentstroke}%
\pgfsetdash{}{0pt}%
\pgfpathmoveto{\pgfqpoint{2.580256in}{2.371451in}}%
\pgfpathcurveto{\pgfqpoint{2.591306in}{2.371451in}}{\pgfqpoint{2.601905in}{2.375841in}}{\pgfqpoint{2.609718in}{2.383655in}}%
\pgfpathcurveto{\pgfqpoint{2.617532in}{2.391468in}}{\pgfqpoint{2.621922in}{2.402067in}}{\pgfqpoint{2.621922in}{2.413117in}}%
\pgfpathcurveto{\pgfqpoint{2.621922in}{2.424168in}}{\pgfqpoint{2.617532in}{2.434767in}}{\pgfqpoint{2.609718in}{2.442580in}}%
\pgfpathcurveto{\pgfqpoint{2.601905in}{2.450394in}}{\pgfqpoint{2.591306in}{2.454784in}}{\pgfqpoint{2.580256in}{2.454784in}}%
\pgfpathcurveto{\pgfqpoint{2.569206in}{2.454784in}}{\pgfqpoint{2.558607in}{2.450394in}}{\pgfqpoint{2.550793in}{2.442580in}}%
\pgfpathcurveto{\pgfqpoint{2.542979in}{2.434767in}}{\pgfqpoint{2.538589in}{2.424168in}}{\pgfqpoint{2.538589in}{2.413117in}}%
\pgfpathcurveto{\pgfqpoint{2.538589in}{2.402067in}}{\pgfqpoint{2.542979in}{2.391468in}}{\pgfqpoint{2.550793in}{2.383655in}}%
\pgfpathcurveto{\pgfqpoint{2.558607in}{2.375841in}}{\pgfqpoint{2.569206in}{2.371451in}}{\pgfqpoint{2.580256in}{2.371451in}}%
\pgfpathclose%
\pgfusepath{stroke,fill}%
\end{pgfscope}%
\begin{pgfscope}%
\pgfpathrectangle{\pgfqpoint{0.375000in}{0.330000in}}{\pgfqpoint{2.325000in}{2.310000in}}%
\pgfusepath{clip}%
\pgfsetbuttcap%
\pgfsetroundjoin%
\definecolor{currentfill}{rgb}{0.000000,0.000000,0.000000}%
\pgfsetfillcolor{currentfill}%
\pgfsetlinewidth{1.003750pt}%
\definecolor{currentstroke}{rgb}{0.000000,0.000000,0.000000}%
\pgfsetstrokecolor{currentstroke}%
\pgfsetdash{}{0pt}%
\pgfpathmoveto{\pgfqpoint{2.580256in}{2.371451in}}%
\pgfpathcurveto{\pgfqpoint{2.591306in}{2.371451in}}{\pgfqpoint{2.601905in}{2.375841in}}{\pgfqpoint{2.609718in}{2.383655in}}%
\pgfpathcurveto{\pgfqpoint{2.617532in}{2.391468in}}{\pgfqpoint{2.621922in}{2.402067in}}{\pgfqpoint{2.621922in}{2.413117in}}%
\pgfpathcurveto{\pgfqpoint{2.621922in}{2.424168in}}{\pgfqpoint{2.617532in}{2.434767in}}{\pgfqpoint{2.609718in}{2.442580in}}%
\pgfpathcurveto{\pgfqpoint{2.601905in}{2.450394in}}{\pgfqpoint{2.591306in}{2.454784in}}{\pgfqpoint{2.580256in}{2.454784in}}%
\pgfpathcurveto{\pgfqpoint{2.569206in}{2.454784in}}{\pgfqpoint{2.558607in}{2.450394in}}{\pgfqpoint{2.550793in}{2.442580in}}%
\pgfpathcurveto{\pgfqpoint{2.542979in}{2.434767in}}{\pgfqpoint{2.538589in}{2.424168in}}{\pgfqpoint{2.538589in}{2.413117in}}%
\pgfpathcurveto{\pgfqpoint{2.538589in}{2.402067in}}{\pgfqpoint{2.542979in}{2.391468in}}{\pgfqpoint{2.550793in}{2.383655in}}%
\pgfpathcurveto{\pgfqpoint{2.558607in}{2.375841in}}{\pgfqpoint{2.569206in}{2.371451in}}{\pgfqpoint{2.580256in}{2.371451in}}%
\pgfpathclose%
\pgfusepath{stroke,fill}%
\end{pgfscope}%
\begin{pgfscope}%
\pgfpathrectangle{\pgfqpoint{0.375000in}{0.330000in}}{\pgfqpoint{2.325000in}{2.310000in}}%
\pgfusepath{clip}%
\pgfsetbuttcap%
\pgfsetroundjoin%
\definecolor{currentfill}{rgb}{0.000000,0.000000,0.000000}%
\pgfsetfillcolor{currentfill}%
\pgfsetlinewidth{1.003750pt}%
\definecolor{currentstroke}{rgb}{0.000000,0.000000,0.000000}%
\pgfsetstrokecolor{currentstroke}%
\pgfsetdash{}{0pt}%
\pgfpathmoveto{\pgfqpoint{2.580256in}{2.267389in}}%
\pgfpathcurveto{\pgfqpoint{2.591306in}{2.267389in}}{\pgfqpoint{2.601905in}{2.271779in}}{\pgfqpoint{2.609718in}{2.279593in}}%
\pgfpathcurveto{\pgfqpoint{2.617532in}{2.287407in}}{\pgfqpoint{2.621922in}{2.298006in}}{\pgfqpoint{2.621922in}{2.309056in}}%
\pgfpathcurveto{\pgfqpoint{2.621922in}{2.320106in}}{\pgfqpoint{2.617532in}{2.330705in}}{\pgfqpoint{2.609718in}{2.338519in}}%
\pgfpathcurveto{\pgfqpoint{2.601905in}{2.346332in}}{\pgfqpoint{2.591306in}{2.350722in}}{\pgfqpoint{2.580256in}{2.350722in}}%
\pgfpathcurveto{\pgfqpoint{2.569206in}{2.350722in}}{\pgfqpoint{2.558607in}{2.346332in}}{\pgfqpoint{2.550793in}{2.338519in}}%
\pgfpathcurveto{\pgfqpoint{2.542979in}{2.330705in}}{\pgfqpoint{2.538589in}{2.320106in}}{\pgfqpoint{2.538589in}{2.309056in}}%
\pgfpathcurveto{\pgfqpoint{2.538589in}{2.298006in}}{\pgfqpoint{2.542979in}{2.287407in}}{\pgfqpoint{2.550793in}{2.279593in}}%
\pgfpathcurveto{\pgfqpoint{2.558607in}{2.271779in}}{\pgfqpoint{2.569206in}{2.267389in}}{\pgfqpoint{2.580256in}{2.267389in}}%
\pgfpathclose%
\pgfusepath{stroke,fill}%
\end{pgfscope}%
\begin{pgfscope}%
\pgfpathrectangle{\pgfqpoint{0.375000in}{0.330000in}}{\pgfqpoint{2.325000in}{2.310000in}}%
\pgfusepath{clip}%
\pgfsetbuttcap%
\pgfsetroundjoin%
\definecolor{currentfill}{rgb}{0.000000,0.000000,0.000000}%
\pgfsetfillcolor{currentfill}%
\pgfsetlinewidth{1.003750pt}%
\definecolor{currentstroke}{rgb}{0.000000,0.000000,0.000000}%
\pgfsetstrokecolor{currentstroke}%
\pgfsetdash{}{0pt}%
\pgfpathmoveto{\pgfqpoint{2.580256in}{2.267389in}}%
\pgfpathcurveto{\pgfqpoint{2.591306in}{2.267389in}}{\pgfqpoint{2.601905in}{2.271779in}}{\pgfqpoint{2.609718in}{2.279593in}}%
\pgfpathcurveto{\pgfqpoint{2.617532in}{2.287407in}}{\pgfqpoint{2.621922in}{2.298006in}}{\pgfqpoint{2.621922in}{2.309056in}}%
\pgfpathcurveto{\pgfqpoint{2.621922in}{2.320106in}}{\pgfqpoint{2.617532in}{2.330705in}}{\pgfqpoint{2.609718in}{2.338519in}}%
\pgfpathcurveto{\pgfqpoint{2.601905in}{2.346332in}}{\pgfqpoint{2.591306in}{2.350722in}}{\pgfqpoint{2.580256in}{2.350722in}}%
\pgfpathcurveto{\pgfqpoint{2.569206in}{2.350722in}}{\pgfqpoint{2.558607in}{2.346332in}}{\pgfqpoint{2.550793in}{2.338519in}}%
\pgfpathcurveto{\pgfqpoint{2.542979in}{2.330705in}}{\pgfqpoint{2.538589in}{2.320106in}}{\pgfqpoint{2.538589in}{2.309056in}}%
\pgfpathcurveto{\pgfqpoint{2.538589in}{2.298006in}}{\pgfqpoint{2.542979in}{2.287407in}}{\pgfqpoint{2.550793in}{2.279593in}}%
\pgfpathcurveto{\pgfqpoint{2.558607in}{2.271779in}}{\pgfqpoint{2.569206in}{2.267389in}}{\pgfqpoint{2.580256in}{2.267389in}}%
\pgfpathclose%
\pgfusepath{stroke,fill}%
\end{pgfscope}%
\begin{pgfscope}%
\pgfpathrectangle{\pgfqpoint{0.375000in}{0.330000in}}{\pgfqpoint{2.325000in}{2.310000in}}%
\pgfusepath{clip}%
\pgfsetbuttcap%
\pgfsetroundjoin%
\definecolor{currentfill}{rgb}{0.000000,0.000000,0.000000}%
\pgfsetfillcolor{currentfill}%
\pgfsetlinewidth{1.003750pt}%
\definecolor{currentstroke}{rgb}{0.000000,0.000000,0.000000}%
\pgfsetstrokecolor{currentstroke}%
\pgfsetdash{}{0pt}%
\pgfpathmoveto{\pgfqpoint{2.580256in}{2.267389in}}%
\pgfpathcurveto{\pgfqpoint{2.591306in}{2.267389in}}{\pgfqpoint{2.601905in}{2.271779in}}{\pgfqpoint{2.609718in}{2.279593in}}%
\pgfpathcurveto{\pgfqpoint{2.617532in}{2.287407in}}{\pgfqpoint{2.621922in}{2.298006in}}{\pgfqpoint{2.621922in}{2.309056in}}%
\pgfpathcurveto{\pgfqpoint{2.621922in}{2.320106in}}{\pgfqpoint{2.617532in}{2.330705in}}{\pgfqpoint{2.609718in}{2.338519in}}%
\pgfpathcurveto{\pgfqpoint{2.601905in}{2.346332in}}{\pgfqpoint{2.591306in}{2.350722in}}{\pgfqpoint{2.580256in}{2.350722in}}%
\pgfpathcurveto{\pgfqpoint{2.569206in}{2.350722in}}{\pgfqpoint{2.558607in}{2.346332in}}{\pgfqpoint{2.550793in}{2.338519in}}%
\pgfpathcurveto{\pgfqpoint{2.542979in}{2.330705in}}{\pgfqpoint{2.538589in}{2.320106in}}{\pgfqpoint{2.538589in}{2.309056in}}%
\pgfpathcurveto{\pgfqpoint{2.538589in}{2.298006in}}{\pgfqpoint{2.542979in}{2.287407in}}{\pgfqpoint{2.550793in}{2.279593in}}%
\pgfpathcurveto{\pgfqpoint{2.558607in}{2.271779in}}{\pgfqpoint{2.569206in}{2.267389in}}{\pgfqpoint{2.580256in}{2.267389in}}%
\pgfpathclose%
\pgfusepath{stroke,fill}%
\end{pgfscope}%
\begin{pgfscope}%
\pgfpathrectangle{\pgfqpoint{0.375000in}{0.330000in}}{\pgfqpoint{2.325000in}{2.310000in}}%
\pgfusepath{clip}%
\pgfsetbuttcap%
\pgfsetroundjoin%
\definecolor{currentfill}{rgb}{0.000000,0.000000,0.000000}%
\pgfsetfillcolor{currentfill}%
\pgfsetlinewidth{1.003750pt}%
\definecolor{currentstroke}{rgb}{0.000000,0.000000,0.000000}%
\pgfsetstrokecolor{currentstroke}%
\pgfsetdash{}{0pt}%
\pgfpathmoveto{\pgfqpoint{2.580256in}{2.215358in}}%
\pgfpathcurveto{\pgfqpoint{2.591306in}{2.215358in}}{\pgfqpoint{2.601905in}{2.219749in}}{\pgfqpoint{2.609718in}{2.227562in}}%
\pgfpathcurveto{\pgfqpoint{2.617532in}{2.235376in}}{\pgfqpoint{2.621922in}{2.245975in}}{\pgfqpoint{2.621922in}{2.257025in}}%
\pgfpathcurveto{\pgfqpoint{2.621922in}{2.268075in}}{\pgfqpoint{2.617532in}{2.278674in}}{\pgfqpoint{2.609718in}{2.286488in}}%
\pgfpathcurveto{\pgfqpoint{2.601905in}{2.294301in}}{\pgfqpoint{2.591306in}{2.298692in}}{\pgfqpoint{2.580256in}{2.298692in}}%
\pgfpathcurveto{\pgfqpoint{2.569206in}{2.298692in}}{\pgfqpoint{2.558607in}{2.294301in}}{\pgfqpoint{2.550793in}{2.286488in}}%
\pgfpathcurveto{\pgfqpoint{2.542979in}{2.278674in}}{\pgfqpoint{2.538589in}{2.268075in}}{\pgfqpoint{2.538589in}{2.257025in}}%
\pgfpathcurveto{\pgfqpoint{2.538589in}{2.245975in}}{\pgfqpoint{2.542979in}{2.235376in}}{\pgfqpoint{2.550793in}{2.227562in}}%
\pgfpathcurveto{\pgfqpoint{2.558607in}{2.219749in}}{\pgfqpoint{2.569206in}{2.215358in}}{\pgfqpoint{2.580256in}{2.215358in}}%
\pgfpathclose%
\pgfusepath{stroke,fill}%
\end{pgfscope}%
\begin{pgfscope}%
\pgfpathrectangle{\pgfqpoint{0.375000in}{0.330000in}}{\pgfqpoint{2.325000in}{2.310000in}}%
\pgfusepath{clip}%
\pgfsetbuttcap%
\pgfsetroundjoin%
\definecolor{currentfill}{rgb}{0.000000,0.000000,0.000000}%
\pgfsetfillcolor{currentfill}%
\pgfsetlinewidth{1.003750pt}%
\definecolor{currentstroke}{rgb}{0.000000,0.000000,0.000000}%
\pgfsetstrokecolor{currentstroke}%
\pgfsetdash{}{0pt}%
\pgfpathmoveto{\pgfqpoint{2.580256in}{2.319420in}}%
\pgfpathcurveto{\pgfqpoint{2.591306in}{2.319420in}}{\pgfqpoint{2.601905in}{2.323810in}}{\pgfqpoint{2.609718in}{2.331624in}}%
\pgfpathcurveto{\pgfqpoint{2.617532in}{2.339437in}}{\pgfqpoint{2.621922in}{2.350037in}}{\pgfqpoint{2.621922in}{2.361087in}}%
\pgfpathcurveto{\pgfqpoint{2.621922in}{2.372137in}}{\pgfqpoint{2.617532in}{2.382736in}}{\pgfqpoint{2.609718in}{2.390549in}}%
\pgfpathcurveto{\pgfqpoint{2.601905in}{2.398363in}}{\pgfqpoint{2.591306in}{2.402753in}}{\pgfqpoint{2.580256in}{2.402753in}}%
\pgfpathcurveto{\pgfqpoint{2.569206in}{2.402753in}}{\pgfqpoint{2.558607in}{2.398363in}}{\pgfqpoint{2.550793in}{2.390549in}}%
\pgfpathcurveto{\pgfqpoint{2.542979in}{2.382736in}}{\pgfqpoint{2.538589in}{2.372137in}}{\pgfqpoint{2.538589in}{2.361087in}}%
\pgfpathcurveto{\pgfqpoint{2.538589in}{2.350037in}}{\pgfqpoint{2.542979in}{2.339437in}}{\pgfqpoint{2.550793in}{2.331624in}}%
\pgfpathcurveto{\pgfqpoint{2.558607in}{2.323810in}}{\pgfqpoint{2.569206in}{2.319420in}}{\pgfqpoint{2.580256in}{2.319420in}}%
\pgfpathclose%
\pgfusepath{stroke,fill}%
\end{pgfscope}%
\begin{pgfscope}%
\pgfpathrectangle{\pgfqpoint{0.375000in}{0.330000in}}{\pgfqpoint{2.325000in}{2.310000in}}%
\pgfusepath{clip}%
\pgfsetbuttcap%
\pgfsetroundjoin%
\definecolor{currentfill}{rgb}{0.000000,0.000000,0.000000}%
\pgfsetfillcolor{currentfill}%
\pgfsetlinewidth{1.003750pt}%
\definecolor{currentstroke}{rgb}{0.000000,0.000000,0.000000}%
\pgfsetstrokecolor{currentstroke}%
\pgfsetdash{}{0pt}%
\pgfpathmoveto{\pgfqpoint{2.580256in}{2.267389in}}%
\pgfpathcurveto{\pgfqpoint{2.591306in}{2.267389in}}{\pgfqpoint{2.601905in}{2.271779in}}{\pgfqpoint{2.609718in}{2.279593in}}%
\pgfpathcurveto{\pgfqpoint{2.617532in}{2.287407in}}{\pgfqpoint{2.621922in}{2.298006in}}{\pgfqpoint{2.621922in}{2.309056in}}%
\pgfpathcurveto{\pgfqpoint{2.621922in}{2.320106in}}{\pgfqpoint{2.617532in}{2.330705in}}{\pgfqpoint{2.609718in}{2.338519in}}%
\pgfpathcurveto{\pgfqpoint{2.601905in}{2.346332in}}{\pgfqpoint{2.591306in}{2.350722in}}{\pgfqpoint{2.580256in}{2.350722in}}%
\pgfpathcurveto{\pgfqpoint{2.569206in}{2.350722in}}{\pgfqpoint{2.558607in}{2.346332in}}{\pgfqpoint{2.550793in}{2.338519in}}%
\pgfpathcurveto{\pgfqpoint{2.542979in}{2.330705in}}{\pgfqpoint{2.538589in}{2.320106in}}{\pgfqpoint{2.538589in}{2.309056in}}%
\pgfpathcurveto{\pgfqpoint{2.538589in}{2.298006in}}{\pgfqpoint{2.542979in}{2.287407in}}{\pgfqpoint{2.550793in}{2.279593in}}%
\pgfpathcurveto{\pgfqpoint{2.558607in}{2.271779in}}{\pgfqpoint{2.569206in}{2.267389in}}{\pgfqpoint{2.580256in}{2.267389in}}%
\pgfpathclose%
\pgfusepath{stroke,fill}%
\end{pgfscope}%
\begin{pgfscope}%
\pgfpathrectangle{\pgfqpoint{0.375000in}{0.330000in}}{\pgfqpoint{2.325000in}{2.310000in}}%
\pgfusepath{clip}%
\pgfsetbuttcap%
\pgfsetroundjoin%
\definecolor{currentfill}{rgb}{0.000000,0.000000,0.000000}%
\pgfsetfillcolor{currentfill}%
\pgfsetlinewidth{1.003750pt}%
\definecolor{currentstroke}{rgb}{0.000000,0.000000,0.000000}%
\pgfsetstrokecolor{currentstroke}%
\pgfsetdash{}{0pt}%
\pgfpathmoveto{\pgfqpoint{2.580256in}{2.267389in}}%
\pgfpathcurveto{\pgfqpoint{2.591306in}{2.267389in}}{\pgfqpoint{2.601905in}{2.271779in}}{\pgfqpoint{2.609718in}{2.279593in}}%
\pgfpathcurveto{\pgfqpoint{2.617532in}{2.287407in}}{\pgfqpoint{2.621922in}{2.298006in}}{\pgfqpoint{2.621922in}{2.309056in}}%
\pgfpathcurveto{\pgfqpoint{2.621922in}{2.320106in}}{\pgfqpoint{2.617532in}{2.330705in}}{\pgfqpoint{2.609718in}{2.338519in}}%
\pgfpathcurveto{\pgfqpoint{2.601905in}{2.346332in}}{\pgfqpoint{2.591306in}{2.350722in}}{\pgfqpoint{2.580256in}{2.350722in}}%
\pgfpathcurveto{\pgfqpoint{2.569206in}{2.350722in}}{\pgfqpoint{2.558607in}{2.346332in}}{\pgfqpoint{2.550793in}{2.338519in}}%
\pgfpathcurveto{\pgfqpoint{2.542979in}{2.330705in}}{\pgfqpoint{2.538589in}{2.320106in}}{\pgfqpoint{2.538589in}{2.309056in}}%
\pgfpathcurveto{\pgfqpoint{2.538589in}{2.298006in}}{\pgfqpoint{2.542979in}{2.287407in}}{\pgfqpoint{2.550793in}{2.279593in}}%
\pgfpathcurveto{\pgfqpoint{2.558607in}{2.271779in}}{\pgfqpoint{2.569206in}{2.267389in}}{\pgfqpoint{2.580256in}{2.267389in}}%
\pgfpathclose%
\pgfusepath{stroke,fill}%
\end{pgfscope}%
\begin{pgfscope}%
\pgfpathrectangle{\pgfqpoint{0.375000in}{0.330000in}}{\pgfqpoint{2.325000in}{2.310000in}}%
\pgfusepath{clip}%
\pgfsetbuttcap%
\pgfsetroundjoin%
\definecolor{currentfill}{rgb}{0.000000,0.000000,0.000000}%
\pgfsetfillcolor{currentfill}%
\pgfsetlinewidth{1.003750pt}%
\definecolor{currentstroke}{rgb}{0.000000,0.000000,0.000000}%
\pgfsetstrokecolor{currentstroke}%
\pgfsetdash{}{0pt}%
\pgfpathmoveto{\pgfqpoint{2.580256in}{2.319420in}}%
\pgfpathcurveto{\pgfqpoint{2.591306in}{2.319420in}}{\pgfqpoint{2.601905in}{2.323810in}}{\pgfqpoint{2.609718in}{2.331624in}}%
\pgfpathcurveto{\pgfqpoint{2.617532in}{2.339437in}}{\pgfqpoint{2.621922in}{2.350037in}}{\pgfqpoint{2.621922in}{2.361087in}}%
\pgfpathcurveto{\pgfqpoint{2.621922in}{2.372137in}}{\pgfqpoint{2.617532in}{2.382736in}}{\pgfqpoint{2.609718in}{2.390549in}}%
\pgfpathcurveto{\pgfqpoint{2.601905in}{2.398363in}}{\pgfqpoint{2.591306in}{2.402753in}}{\pgfqpoint{2.580256in}{2.402753in}}%
\pgfpathcurveto{\pgfqpoint{2.569206in}{2.402753in}}{\pgfqpoint{2.558607in}{2.398363in}}{\pgfqpoint{2.550793in}{2.390549in}}%
\pgfpathcurveto{\pgfqpoint{2.542979in}{2.382736in}}{\pgfqpoint{2.538589in}{2.372137in}}{\pgfqpoint{2.538589in}{2.361087in}}%
\pgfpathcurveto{\pgfqpoint{2.538589in}{2.350037in}}{\pgfqpoint{2.542979in}{2.339437in}}{\pgfqpoint{2.550793in}{2.331624in}}%
\pgfpathcurveto{\pgfqpoint{2.558607in}{2.323810in}}{\pgfqpoint{2.569206in}{2.319420in}}{\pgfqpoint{2.580256in}{2.319420in}}%
\pgfpathclose%
\pgfusepath{stroke,fill}%
\end{pgfscope}%
\begin{pgfscope}%
\pgfpathrectangle{\pgfqpoint{0.375000in}{0.330000in}}{\pgfqpoint{2.325000in}{2.310000in}}%
\pgfusepath{clip}%
\pgfsetbuttcap%
\pgfsetroundjoin%
\definecolor{currentfill}{rgb}{0.000000,0.000000,0.000000}%
\pgfsetfillcolor{currentfill}%
\pgfsetlinewidth{1.003750pt}%
\definecolor{currentstroke}{rgb}{0.000000,0.000000,0.000000}%
\pgfsetstrokecolor{currentstroke}%
\pgfsetdash{}{0pt}%
\pgfpathmoveto{\pgfqpoint{2.580256in}{2.267389in}}%
\pgfpathcurveto{\pgfqpoint{2.591306in}{2.267389in}}{\pgfqpoint{2.601905in}{2.271779in}}{\pgfqpoint{2.609718in}{2.279593in}}%
\pgfpathcurveto{\pgfqpoint{2.617532in}{2.287407in}}{\pgfqpoint{2.621922in}{2.298006in}}{\pgfqpoint{2.621922in}{2.309056in}}%
\pgfpathcurveto{\pgfqpoint{2.621922in}{2.320106in}}{\pgfqpoint{2.617532in}{2.330705in}}{\pgfqpoint{2.609718in}{2.338519in}}%
\pgfpathcurveto{\pgfqpoint{2.601905in}{2.346332in}}{\pgfqpoint{2.591306in}{2.350722in}}{\pgfqpoint{2.580256in}{2.350722in}}%
\pgfpathcurveto{\pgfqpoint{2.569206in}{2.350722in}}{\pgfqpoint{2.558607in}{2.346332in}}{\pgfqpoint{2.550793in}{2.338519in}}%
\pgfpathcurveto{\pgfqpoint{2.542979in}{2.330705in}}{\pgfqpoint{2.538589in}{2.320106in}}{\pgfqpoint{2.538589in}{2.309056in}}%
\pgfpathcurveto{\pgfqpoint{2.538589in}{2.298006in}}{\pgfqpoint{2.542979in}{2.287407in}}{\pgfqpoint{2.550793in}{2.279593in}}%
\pgfpathcurveto{\pgfqpoint{2.558607in}{2.271779in}}{\pgfqpoint{2.569206in}{2.267389in}}{\pgfqpoint{2.580256in}{2.267389in}}%
\pgfpathclose%
\pgfusepath{stroke,fill}%
\end{pgfscope}%
\begin{pgfscope}%
\pgfpathrectangle{\pgfqpoint{0.375000in}{0.330000in}}{\pgfqpoint{2.325000in}{2.310000in}}%
\pgfusepath{clip}%
\pgfsetbuttcap%
\pgfsetroundjoin%
\definecolor{currentfill}{rgb}{0.000000,0.000000,0.000000}%
\pgfsetfillcolor{currentfill}%
\pgfsetlinewidth{1.003750pt}%
\definecolor{currentstroke}{rgb}{0.000000,0.000000,0.000000}%
\pgfsetstrokecolor{currentstroke}%
\pgfsetdash{}{0pt}%
\pgfpathmoveto{\pgfqpoint{2.580256in}{2.267389in}}%
\pgfpathcurveto{\pgfqpoint{2.591306in}{2.267389in}}{\pgfqpoint{2.601905in}{2.271779in}}{\pgfqpoint{2.609718in}{2.279593in}}%
\pgfpathcurveto{\pgfqpoint{2.617532in}{2.287407in}}{\pgfqpoint{2.621922in}{2.298006in}}{\pgfqpoint{2.621922in}{2.309056in}}%
\pgfpathcurveto{\pgfqpoint{2.621922in}{2.320106in}}{\pgfqpoint{2.617532in}{2.330705in}}{\pgfqpoint{2.609718in}{2.338519in}}%
\pgfpathcurveto{\pgfqpoint{2.601905in}{2.346332in}}{\pgfqpoint{2.591306in}{2.350722in}}{\pgfqpoint{2.580256in}{2.350722in}}%
\pgfpathcurveto{\pgfqpoint{2.569206in}{2.350722in}}{\pgfqpoint{2.558607in}{2.346332in}}{\pgfqpoint{2.550793in}{2.338519in}}%
\pgfpathcurveto{\pgfqpoint{2.542979in}{2.330705in}}{\pgfqpoint{2.538589in}{2.320106in}}{\pgfqpoint{2.538589in}{2.309056in}}%
\pgfpathcurveto{\pgfqpoint{2.538589in}{2.298006in}}{\pgfqpoint{2.542979in}{2.287407in}}{\pgfqpoint{2.550793in}{2.279593in}}%
\pgfpathcurveto{\pgfqpoint{2.558607in}{2.271779in}}{\pgfqpoint{2.569206in}{2.267389in}}{\pgfqpoint{2.580256in}{2.267389in}}%
\pgfpathclose%
\pgfusepath{stroke,fill}%
\end{pgfscope}%
\begin{pgfscope}%
\pgfpathrectangle{\pgfqpoint{0.375000in}{0.330000in}}{\pgfqpoint{2.325000in}{2.310000in}}%
\pgfusepath{clip}%
\pgfsetbuttcap%
\pgfsetroundjoin%
\definecolor{currentfill}{rgb}{0.000000,0.000000,0.000000}%
\pgfsetfillcolor{currentfill}%
\pgfsetlinewidth{1.003750pt}%
\definecolor{currentstroke}{rgb}{0.000000,0.000000,0.000000}%
\pgfsetstrokecolor{currentstroke}%
\pgfsetdash{}{0pt}%
\pgfpathmoveto{\pgfqpoint{2.580256in}{2.319420in}}%
\pgfpathcurveto{\pgfqpoint{2.591306in}{2.319420in}}{\pgfqpoint{2.601905in}{2.323810in}}{\pgfqpoint{2.609718in}{2.331624in}}%
\pgfpathcurveto{\pgfqpoint{2.617532in}{2.339437in}}{\pgfqpoint{2.621922in}{2.350037in}}{\pgfqpoint{2.621922in}{2.361087in}}%
\pgfpathcurveto{\pgfqpoint{2.621922in}{2.372137in}}{\pgfqpoint{2.617532in}{2.382736in}}{\pgfqpoint{2.609718in}{2.390549in}}%
\pgfpathcurveto{\pgfqpoint{2.601905in}{2.398363in}}{\pgfqpoint{2.591306in}{2.402753in}}{\pgfqpoint{2.580256in}{2.402753in}}%
\pgfpathcurveto{\pgfqpoint{2.569206in}{2.402753in}}{\pgfqpoint{2.558607in}{2.398363in}}{\pgfqpoint{2.550793in}{2.390549in}}%
\pgfpathcurveto{\pgfqpoint{2.542979in}{2.382736in}}{\pgfqpoint{2.538589in}{2.372137in}}{\pgfqpoint{2.538589in}{2.361087in}}%
\pgfpathcurveto{\pgfqpoint{2.538589in}{2.350037in}}{\pgfqpoint{2.542979in}{2.339437in}}{\pgfqpoint{2.550793in}{2.331624in}}%
\pgfpathcurveto{\pgfqpoint{2.558607in}{2.323810in}}{\pgfqpoint{2.569206in}{2.319420in}}{\pgfqpoint{2.580256in}{2.319420in}}%
\pgfpathclose%
\pgfusepath{stroke,fill}%
\end{pgfscope}%
\begin{pgfscope}%
\pgfpathrectangle{\pgfqpoint{0.375000in}{0.330000in}}{\pgfqpoint{2.325000in}{2.310000in}}%
\pgfusepath{clip}%
\pgfsetbuttcap%
\pgfsetroundjoin%
\definecolor{currentfill}{rgb}{0.000000,0.000000,0.000000}%
\pgfsetfillcolor{currentfill}%
\pgfsetlinewidth{1.003750pt}%
\definecolor{currentstroke}{rgb}{0.000000,0.000000,0.000000}%
\pgfsetstrokecolor{currentstroke}%
\pgfsetdash{}{0pt}%
\pgfpathmoveto{\pgfqpoint{2.580256in}{2.371451in}}%
\pgfpathcurveto{\pgfqpoint{2.591306in}{2.371451in}}{\pgfqpoint{2.601905in}{2.375841in}}{\pgfqpoint{2.609718in}{2.383655in}}%
\pgfpathcurveto{\pgfqpoint{2.617532in}{2.391468in}}{\pgfqpoint{2.621922in}{2.402067in}}{\pgfqpoint{2.621922in}{2.413117in}}%
\pgfpathcurveto{\pgfqpoint{2.621922in}{2.424168in}}{\pgfqpoint{2.617532in}{2.434767in}}{\pgfqpoint{2.609718in}{2.442580in}}%
\pgfpathcurveto{\pgfqpoint{2.601905in}{2.450394in}}{\pgfqpoint{2.591306in}{2.454784in}}{\pgfqpoint{2.580256in}{2.454784in}}%
\pgfpathcurveto{\pgfqpoint{2.569206in}{2.454784in}}{\pgfqpoint{2.558607in}{2.450394in}}{\pgfqpoint{2.550793in}{2.442580in}}%
\pgfpathcurveto{\pgfqpoint{2.542979in}{2.434767in}}{\pgfqpoint{2.538589in}{2.424168in}}{\pgfqpoint{2.538589in}{2.413117in}}%
\pgfpathcurveto{\pgfqpoint{2.538589in}{2.402067in}}{\pgfqpoint{2.542979in}{2.391468in}}{\pgfqpoint{2.550793in}{2.383655in}}%
\pgfpathcurveto{\pgfqpoint{2.558607in}{2.375841in}}{\pgfqpoint{2.569206in}{2.371451in}}{\pgfqpoint{2.580256in}{2.371451in}}%
\pgfpathclose%
\pgfusepath{stroke,fill}%
\end{pgfscope}%
\begin{pgfscope}%
\pgfpathrectangle{\pgfqpoint{0.375000in}{0.330000in}}{\pgfqpoint{2.325000in}{2.310000in}}%
\pgfusepath{clip}%
\pgfsetbuttcap%
\pgfsetroundjoin%
\definecolor{currentfill}{rgb}{0.000000,0.000000,0.000000}%
\pgfsetfillcolor{currentfill}%
\pgfsetlinewidth{1.003750pt}%
\definecolor{currentstroke}{rgb}{0.000000,0.000000,0.000000}%
\pgfsetstrokecolor{currentstroke}%
\pgfsetdash{}{0pt}%
\pgfpathmoveto{\pgfqpoint{2.580256in}{2.319420in}}%
\pgfpathcurveto{\pgfqpoint{2.591306in}{2.319420in}}{\pgfqpoint{2.601905in}{2.323810in}}{\pgfqpoint{2.609718in}{2.331624in}}%
\pgfpathcurveto{\pgfqpoint{2.617532in}{2.339437in}}{\pgfqpoint{2.621922in}{2.350037in}}{\pgfqpoint{2.621922in}{2.361087in}}%
\pgfpathcurveto{\pgfqpoint{2.621922in}{2.372137in}}{\pgfqpoint{2.617532in}{2.382736in}}{\pgfqpoint{2.609718in}{2.390549in}}%
\pgfpathcurveto{\pgfqpoint{2.601905in}{2.398363in}}{\pgfqpoint{2.591306in}{2.402753in}}{\pgfqpoint{2.580256in}{2.402753in}}%
\pgfpathcurveto{\pgfqpoint{2.569206in}{2.402753in}}{\pgfqpoint{2.558607in}{2.398363in}}{\pgfqpoint{2.550793in}{2.390549in}}%
\pgfpathcurveto{\pgfqpoint{2.542979in}{2.382736in}}{\pgfqpoint{2.538589in}{2.372137in}}{\pgfqpoint{2.538589in}{2.361087in}}%
\pgfpathcurveto{\pgfqpoint{2.538589in}{2.350037in}}{\pgfqpoint{2.542979in}{2.339437in}}{\pgfqpoint{2.550793in}{2.331624in}}%
\pgfpathcurveto{\pgfqpoint{2.558607in}{2.323810in}}{\pgfqpoint{2.569206in}{2.319420in}}{\pgfqpoint{2.580256in}{2.319420in}}%
\pgfpathclose%
\pgfusepath{stroke,fill}%
\end{pgfscope}%
\begin{pgfscope}%
\pgfpathrectangle{\pgfqpoint{0.375000in}{0.330000in}}{\pgfqpoint{2.325000in}{2.310000in}}%
\pgfusepath{clip}%
\pgfsetbuttcap%
\pgfsetroundjoin%
\definecolor{currentfill}{rgb}{0.000000,0.000000,0.000000}%
\pgfsetfillcolor{currentfill}%
\pgfsetlinewidth{1.003750pt}%
\definecolor{currentstroke}{rgb}{0.000000,0.000000,0.000000}%
\pgfsetstrokecolor{currentstroke}%
\pgfsetdash{}{0pt}%
\pgfpathmoveto{\pgfqpoint{2.580256in}{2.319420in}}%
\pgfpathcurveto{\pgfqpoint{2.591306in}{2.319420in}}{\pgfqpoint{2.601905in}{2.323810in}}{\pgfqpoint{2.609718in}{2.331624in}}%
\pgfpathcurveto{\pgfqpoint{2.617532in}{2.339437in}}{\pgfqpoint{2.621922in}{2.350037in}}{\pgfqpoint{2.621922in}{2.361087in}}%
\pgfpathcurveto{\pgfqpoint{2.621922in}{2.372137in}}{\pgfqpoint{2.617532in}{2.382736in}}{\pgfqpoint{2.609718in}{2.390549in}}%
\pgfpathcurveto{\pgfqpoint{2.601905in}{2.398363in}}{\pgfqpoint{2.591306in}{2.402753in}}{\pgfqpoint{2.580256in}{2.402753in}}%
\pgfpathcurveto{\pgfqpoint{2.569206in}{2.402753in}}{\pgfqpoint{2.558607in}{2.398363in}}{\pgfqpoint{2.550793in}{2.390549in}}%
\pgfpathcurveto{\pgfqpoint{2.542979in}{2.382736in}}{\pgfqpoint{2.538589in}{2.372137in}}{\pgfqpoint{2.538589in}{2.361087in}}%
\pgfpathcurveto{\pgfqpoint{2.538589in}{2.350037in}}{\pgfqpoint{2.542979in}{2.339437in}}{\pgfqpoint{2.550793in}{2.331624in}}%
\pgfpathcurveto{\pgfqpoint{2.558607in}{2.323810in}}{\pgfqpoint{2.569206in}{2.319420in}}{\pgfqpoint{2.580256in}{2.319420in}}%
\pgfpathclose%
\pgfusepath{stroke,fill}%
\end{pgfscope}%
\begin{pgfscope}%
\pgfpathrectangle{\pgfqpoint{0.375000in}{0.330000in}}{\pgfqpoint{2.325000in}{2.310000in}}%
\pgfusepath{clip}%
\pgfsetbuttcap%
\pgfsetroundjoin%
\definecolor{currentfill}{rgb}{0.000000,0.000000,0.000000}%
\pgfsetfillcolor{currentfill}%
\pgfsetlinewidth{1.003750pt}%
\definecolor{currentstroke}{rgb}{0.000000,0.000000,0.000000}%
\pgfsetstrokecolor{currentstroke}%
\pgfsetdash{}{0pt}%
\pgfpathmoveto{\pgfqpoint{2.580256in}{2.319420in}}%
\pgfpathcurveto{\pgfqpoint{2.591306in}{2.319420in}}{\pgfqpoint{2.601905in}{2.323810in}}{\pgfqpoint{2.609718in}{2.331624in}}%
\pgfpathcurveto{\pgfqpoint{2.617532in}{2.339437in}}{\pgfqpoint{2.621922in}{2.350037in}}{\pgfqpoint{2.621922in}{2.361087in}}%
\pgfpathcurveto{\pgfqpoint{2.621922in}{2.372137in}}{\pgfqpoint{2.617532in}{2.382736in}}{\pgfqpoint{2.609718in}{2.390549in}}%
\pgfpathcurveto{\pgfqpoint{2.601905in}{2.398363in}}{\pgfqpoint{2.591306in}{2.402753in}}{\pgfqpoint{2.580256in}{2.402753in}}%
\pgfpathcurveto{\pgfqpoint{2.569206in}{2.402753in}}{\pgfqpoint{2.558607in}{2.398363in}}{\pgfqpoint{2.550793in}{2.390549in}}%
\pgfpathcurveto{\pgfqpoint{2.542979in}{2.382736in}}{\pgfqpoint{2.538589in}{2.372137in}}{\pgfqpoint{2.538589in}{2.361087in}}%
\pgfpathcurveto{\pgfqpoint{2.538589in}{2.350037in}}{\pgfqpoint{2.542979in}{2.339437in}}{\pgfqpoint{2.550793in}{2.331624in}}%
\pgfpathcurveto{\pgfqpoint{2.558607in}{2.323810in}}{\pgfqpoint{2.569206in}{2.319420in}}{\pgfqpoint{2.580256in}{2.319420in}}%
\pgfpathclose%
\pgfusepath{stroke,fill}%
\end{pgfscope}%
\begin{pgfscope}%
\pgfpathrectangle{\pgfqpoint{0.375000in}{0.330000in}}{\pgfqpoint{2.325000in}{2.310000in}}%
\pgfusepath{clip}%
\pgfsetbuttcap%
\pgfsetroundjoin%
\definecolor{currentfill}{rgb}{0.000000,0.000000,0.000000}%
\pgfsetfillcolor{currentfill}%
\pgfsetlinewidth{1.003750pt}%
\definecolor{currentstroke}{rgb}{0.000000,0.000000,0.000000}%
\pgfsetstrokecolor{currentstroke}%
\pgfsetdash{}{0pt}%
\pgfpathmoveto{\pgfqpoint{2.580256in}{2.319420in}}%
\pgfpathcurveto{\pgfqpoint{2.591306in}{2.319420in}}{\pgfqpoint{2.601905in}{2.323810in}}{\pgfqpoint{2.609718in}{2.331624in}}%
\pgfpathcurveto{\pgfqpoint{2.617532in}{2.339437in}}{\pgfqpoint{2.621922in}{2.350037in}}{\pgfqpoint{2.621922in}{2.361087in}}%
\pgfpathcurveto{\pgfqpoint{2.621922in}{2.372137in}}{\pgfqpoint{2.617532in}{2.382736in}}{\pgfqpoint{2.609718in}{2.390549in}}%
\pgfpathcurveto{\pgfqpoint{2.601905in}{2.398363in}}{\pgfqpoint{2.591306in}{2.402753in}}{\pgfqpoint{2.580256in}{2.402753in}}%
\pgfpathcurveto{\pgfqpoint{2.569206in}{2.402753in}}{\pgfqpoint{2.558607in}{2.398363in}}{\pgfqpoint{2.550793in}{2.390549in}}%
\pgfpathcurveto{\pgfqpoint{2.542979in}{2.382736in}}{\pgfqpoint{2.538589in}{2.372137in}}{\pgfqpoint{2.538589in}{2.361087in}}%
\pgfpathcurveto{\pgfqpoint{2.538589in}{2.350037in}}{\pgfqpoint{2.542979in}{2.339437in}}{\pgfqpoint{2.550793in}{2.331624in}}%
\pgfpathcurveto{\pgfqpoint{2.558607in}{2.323810in}}{\pgfqpoint{2.569206in}{2.319420in}}{\pgfqpoint{2.580256in}{2.319420in}}%
\pgfpathclose%
\pgfusepath{stroke,fill}%
\end{pgfscope}%
\begin{pgfscope}%
\pgfpathrectangle{\pgfqpoint{0.375000in}{0.330000in}}{\pgfqpoint{2.325000in}{2.310000in}}%
\pgfusepath{clip}%
\pgfsetbuttcap%
\pgfsetroundjoin%
\definecolor{currentfill}{rgb}{0.000000,0.000000,0.000000}%
\pgfsetfillcolor{currentfill}%
\pgfsetlinewidth{1.003750pt}%
\definecolor{currentstroke}{rgb}{0.000000,0.000000,0.000000}%
\pgfsetstrokecolor{currentstroke}%
\pgfsetdash{}{0pt}%
\pgfpathmoveto{\pgfqpoint{2.580256in}{2.319420in}}%
\pgfpathcurveto{\pgfqpoint{2.591306in}{2.319420in}}{\pgfqpoint{2.601905in}{2.323810in}}{\pgfqpoint{2.609718in}{2.331624in}}%
\pgfpathcurveto{\pgfqpoint{2.617532in}{2.339437in}}{\pgfqpoint{2.621922in}{2.350037in}}{\pgfqpoint{2.621922in}{2.361087in}}%
\pgfpathcurveto{\pgfqpoint{2.621922in}{2.372137in}}{\pgfqpoint{2.617532in}{2.382736in}}{\pgfqpoint{2.609718in}{2.390549in}}%
\pgfpathcurveto{\pgfqpoint{2.601905in}{2.398363in}}{\pgfqpoint{2.591306in}{2.402753in}}{\pgfqpoint{2.580256in}{2.402753in}}%
\pgfpathcurveto{\pgfqpoint{2.569206in}{2.402753in}}{\pgfqpoint{2.558607in}{2.398363in}}{\pgfqpoint{2.550793in}{2.390549in}}%
\pgfpathcurveto{\pgfqpoint{2.542979in}{2.382736in}}{\pgfqpoint{2.538589in}{2.372137in}}{\pgfqpoint{2.538589in}{2.361087in}}%
\pgfpathcurveto{\pgfqpoint{2.538589in}{2.350037in}}{\pgfqpoint{2.542979in}{2.339437in}}{\pgfqpoint{2.550793in}{2.331624in}}%
\pgfpathcurveto{\pgfqpoint{2.558607in}{2.323810in}}{\pgfqpoint{2.569206in}{2.319420in}}{\pgfqpoint{2.580256in}{2.319420in}}%
\pgfpathclose%
\pgfusepath{stroke,fill}%
\end{pgfscope}%
\begin{pgfscope}%
\pgfpathrectangle{\pgfqpoint{0.375000in}{0.330000in}}{\pgfqpoint{2.325000in}{2.310000in}}%
\pgfusepath{clip}%
\pgfsetbuttcap%
\pgfsetroundjoin%
\definecolor{currentfill}{rgb}{0.000000,0.000000,0.000000}%
\pgfsetfillcolor{currentfill}%
\pgfsetlinewidth{1.003750pt}%
\definecolor{currentstroke}{rgb}{0.000000,0.000000,0.000000}%
\pgfsetstrokecolor{currentstroke}%
\pgfsetdash{}{0pt}%
\pgfpathmoveto{\pgfqpoint{2.580256in}{2.267389in}}%
\pgfpathcurveto{\pgfqpoint{2.591306in}{2.267389in}}{\pgfqpoint{2.601905in}{2.271779in}}{\pgfqpoint{2.609718in}{2.279593in}}%
\pgfpathcurveto{\pgfqpoint{2.617532in}{2.287407in}}{\pgfqpoint{2.621922in}{2.298006in}}{\pgfqpoint{2.621922in}{2.309056in}}%
\pgfpathcurveto{\pgfqpoint{2.621922in}{2.320106in}}{\pgfqpoint{2.617532in}{2.330705in}}{\pgfqpoint{2.609718in}{2.338519in}}%
\pgfpathcurveto{\pgfqpoint{2.601905in}{2.346332in}}{\pgfqpoint{2.591306in}{2.350722in}}{\pgfqpoint{2.580256in}{2.350722in}}%
\pgfpathcurveto{\pgfqpoint{2.569206in}{2.350722in}}{\pgfqpoint{2.558607in}{2.346332in}}{\pgfqpoint{2.550793in}{2.338519in}}%
\pgfpathcurveto{\pgfqpoint{2.542979in}{2.330705in}}{\pgfqpoint{2.538589in}{2.320106in}}{\pgfqpoint{2.538589in}{2.309056in}}%
\pgfpathcurveto{\pgfqpoint{2.538589in}{2.298006in}}{\pgfqpoint{2.542979in}{2.287407in}}{\pgfqpoint{2.550793in}{2.279593in}}%
\pgfpathcurveto{\pgfqpoint{2.558607in}{2.271779in}}{\pgfqpoint{2.569206in}{2.267389in}}{\pgfqpoint{2.580256in}{2.267389in}}%
\pgfpathclose%
\pgfusepath{stroke,fill}%
\end{pgfscope}%
\begin{pgfscope}%
\pgfpathrectangle{\pgfqpoint{0.375000in}{0.330000in}}{\pgfqpoint{2.325000in}{2.310000in}}%
\pgfusepath{clip}%
\pgfsetbuttcap%
\pgfsetroundjoin%
\definecolor{currentfill}{rgb}{0.000000,0.000000,0.000000}%
\pgfsetfillcolor{currentfill}%
\pgfsetlinewidth{1.003750pt}%
\definecolor{currentstroke}{rgb}{0.000000,0.000000,0.000000}%
\pgfsetstrokecolor{currentstroke}%
\pgfsetdash{}{0pt}%
\pgfpathmoveto{\pgfqpoint{2.580256in}{2.267389in}}%
\pgfpathcurveto{\pgfqpoint{2.591306in}{2.267389in}}{\pgfqpoint{2.601905in}{2.271779in}}{\pgfqpoint{2.609718in}{2.279593in}}%
\pgfpathcurveto{\pgfqpoint{2.617532in}{2.287407in}}{\pgfqpoint{2.621922in}{2.298006in}}{\pgfqpoint{2.621922in}{2.309056in}}%
\pgfpathcurveto{\pgfqpoint{2.621922in}{2.320106in}}{\pgfqpoint{2.617532in}{2.330705in}}{\pgfqpoint{2.609718in}{2.338519in}}%
\pgfpathcurveto{\pgfqpoint{2.601905in}{2.346332in}}{\pgfqpoint{2.591306in}{2.350722in}}{\pgfqpoint{2.580256in}{2.350722in}}%
\pgfpathcurveto{\pgfqpoint{2.569206in}{2.350722in}}{\pgfqpoint{2.558607in}{2.346332in}}{\pgfqpoint{2.550793in}{2.338519in}}%
\pgfpathcurveto{\pgfqpoint{2.542979in}{2.330705in}}{\pgfqpoint{2.538589in}{2.320106in}}{\pgfqpoint{2.538589in}{2.309056in}}%
\pgfpathcurveto{\pgfqpoint{2.538589in}{2.298006in}}{\pgfqpoint{2.542979in}{2.287407in}}{\pgfqpoint{2.550793in}{2.279593in}}%
\pgfpathcurveto{\pgfqpoint{2.558607in}{2.271779in}}{\pgfqpoint{2.569206in}{2.267389in}}{\pgfqpoint{2.580256in}{2.267389in}}%
\pgfpathclose%
\pgfusepath{stroke,fill}%
\end{pgfscope}%
\begin{pgfscope}%
\pgfpathrectangle{\pgfqpoint{0.375000in}{0.330000in}}{\pgfqpoint{2.325000in}{2.310000in}}%
\pgfusepath{clip}%
\pgfsetbuttcap%
\pgfsetroundjoin%
\definecolor{currentfill}{rgb}{0.000000,0.000000,0.000000}%
\pgfsetfillcolor{currentfill}%
\pgfsetlinewidth{1.003750pt}%
\definecolor{currentstroke}{rgb}{0.000000,0.000000,0.000000}%
\pgfsetstrokecolor{currentstroke}%
\pgfsetdash{}{0pt}%
\pgfpathmoveto{\pgfqpoint{2.580256in}{2.267389in}}%
\pgfpathcurveto{\pgfqpoint{2.591306in}{2.267389in}}{\pgfqpoint{2.601905in}{2.271779in}}{\pgfqpoint{2.609718in}{2.279593in}}%
\pgfpathcurveto{\pgfqpoint{2.617532in}{2.287407in}}{\pgfqpoint{2.621922in}{2.298006in}}{\pgfqpoint{2.621922in}{2.309056in}}%
\pgfpathcurveto{\pgfqpoint{2.621922in}{2.320106in}}{\pgfqpoint{2.617532in}{2.330705in}}{\pgfqpoint{2.609718in}{2.338519in}}%
\pgfpathcurveto{\pgfqpoint{2.601905in}{2.346332in}}{\pgfqpoint{2.591306in}{2.350722in}}{\pgfqpoint{2.580256in}{2.350722in}}%
\pgfpathcurveto{\pgfqpoint{2.569206in}{2.350722in}}{\pgfqpoint{2.558607in}{2.346332in}}{\pgfqpoint{2.550793in}{2.338519in}}%
\pgfpathcurveto{\pgfqpoint{2.542979in}{2.330705in}}{\pgfqpoint{2.538589in}{2.320106in}}{\pgfqpoint{2.538589in}{2.309056in}}%
\pgfpathcurveto{\pgfqpoint{2.538589in}{2.298006in}}{\pgfqpoint{2.542979in}{2.287407in}}{\pgfqpoint{2.550793in}{2.279593in}}%
\pgfpathcurveto{\pgfqpoint{2.558607in}{2.271779in}}{\pgfqpoint{2.569206in}{2.267389in}}{\pgfqpoint{2.580256in}{2.267389in}}%
\pgfpathclose%
\pgfusepath{stroke,fill}%
\end{pgfscope}%
\begin{pgfscope}%
\pgfsetbuttcap%
\pgfsetroundjoin%
\definecolor{currentfill}{rgb}{0.000000,0.000000,0.000000}%
\pgfsetfillcolor{currentfill}%
\pgfsetlinewidth{0.803000pt}%
\definecolor{currentstroke}{rgb}{0.000000,0.000000,0.000000}%
\pgfsetstrokecolor{currentstroke}%
\pgfsetdash{}{0pt}%
\pgfsys@defobject{currentmarker}{\pgfqpoint{0.000000in}{-0.048611in}}{\pgfqpoint{0.000000in}{0.000000in}}{%
\pgfpathmoveto{\pgfqpoint{0.000000in}{0.000000in}}%
\pgfpathlineto{\pgfqpoint{0.000000in}{-0.048611in}}%
\pgfusepath{stroke,fill}%
}%
\begin{pgfscope}%
\pgfsys@transformshift{0.480841in}{0.330000in}%
\pgfsys@useobject{currentmarker}{}%
\end{pgfscope}%
\end{pgfscope}%
\begin{pgfscope}%
\definecolor{textcolor}{rgb}{0.000000,0.000000,0.000000}%
\pgfsetstrokecolor{textcolor}%
\pgfsetfillcolor{textcolor}%
\pgftext[x=0.480841in,y=0.232778in,,top]{\color{textcolor}\sffamily\fontsize{10.000000}{12.000000}\selectfont 20}%
\end{pgfscope}%
\begin{pgfscope}%
\pgfsetbuttcap%
\pgfsetroundjoin%
\definecolor{currentfill}{rgb}{0.000000,0.000000,0.000000}%
\pgfsetfillcolor{currentfill}%
\pgfsetlinewidth{0.803000pt}%
\definecolor{currentstroke}{rgb}{0.000000,0.000000,0.000000}%
\pgfsetstrokecolor{currentstroke}%
\pgfsetdash{}{0pt}%
\pgfsys@defobject{currentmarker}{\pgfqpoint{0.000000in}{-0.048611in}}{\pgfqpoint{0.000000in}{0.000000in}}{%
\pgfpathmoveto{\pgfqpoint{0.000000in}{0.000000in}}%
\pgfpathlineto{\pgfqpoint{0.000000in}{-0.048611in}}%
\pgfusepath{stroke,fill}%
}%
\begin{pgfscope}%
\pgfsys@transformshift{1.005694in}{0.330000in}%
\pgfsys@useobject{currentmarker}{}%
\end{pgfscope}%
\end{pgfscope}%
\begin{pgfscope}%
\definecolor{textcolor}{rgb}{0.000000,0.000000,0.000000}%
\pgfsetstrokecolor{textcolor}%
\pgfsetfillcolor{textcolor}%
\pgftext[x=1.005694in,y=0.232778in,,top]{\color{textcolor}\sffamily\fontsize{10.000000}{12.000000}\selectfont 40}%
\end{pgfscope}%
\begin{pgfscope}%
\pgfsetbuttcap%
\pgfsetroundjoin%
\definecolor{currentfill}{rgb}{0.000000,0.000000,0.000000}%
\pgfsetfillcolor{currentfill}%
\pgfsetlinewidth{0.803000pt}%
\definecolor{currentstroke}{rgb}{0.000000,0.000000,0.000000}%
\pgfsetstrokecolor{currentstroke}%
\pgfsetdash{}{0pt}%
\pgfsys@defobject{currentmarker}{\pgfqpoint{0.000000in}{-0.048611in}}{\pgfqpoint{0.000000in}{0.000000in}}{%
\pgfpathmoveto{\pgfqpoint{0.000000in}{0.000000in}}%
\pgfpathlineto{\pgfqpoint{0.000000in}{-0.048611in}}%
\pgfusepath{stroke,fill}%
}%
\begin{pgfscope}%
\pgfsys@transformshift{1.530548in}{0.330000in}%
\pgfsys@useobject{currentmarker}{}%
\end{pgfscope}%
\end{pgfscope}%
\begin{pgfscope}%
\definecolor{textcolor}{rgb}{0.000000,0.000000,0.000000}%
\pgfsetstrokecolor{textcolor}%
\pgfsetfillcolor{textcolor}%
\pgftext[x=1.530548in,y=0.232778in,,top]{\color{textcolor}\sffamily\fontsize{10.000000}{12.000000}\selectfont 60}%
\end{pgfscope}%
\begin{pgfscope}%
\pgfsetbuttcap%
\pgfsetroundjoin%
\definecolor{currentfill}{rgb}{0.000000,0.000000,0.000000}%
\pgfsetfillcolor{currentfill}%
\pgfsetlinewidth{0.803000pt}%
\definecolor{currentstroke}{rgb}{0.000000,0.000000,0.000000}%
\pgfsetstrokecolor{currentstroke}%
\pgfsetdash{}{0pt}%
\pgfsys@defobject{currentmarker}{\pgfqpoint{0.000000in}{-0.048611in}}{\pgfqpoint{0.000000in}{0.000000in}}{%
\pgfpathmoveto{\pgfqpoint{0.000000in}{0.000000in}}%
\pgfpathlineto{\pgfqpoint{0.000000in}{-0.048611in}}%
\pgfusepath{stroke,fill}%
}%
\begin{pgfscope}%
\pgfsys@transformshift{2.055402in}{0.330000in}%
\pgfsys@useobject{currentmarker}{}%
\end{pgfscope}%
\end{pgfscope}%
\begin{pgfscope}%
\definecolor{textcolor}{rgb}{0.000000,0.000000,0.000000}%
\pgfsetstrokecolor{textcolor}%
\pgfsetfillcolor{textcolor}%
\pgftext[x=2.055402in,y=0.232778in,,top]{\color{textcolor}\sffamily\fontsize{10.000000}{12.000000}\selectfont 80}%
\end{pgfscope}%
\begin{pgfscope}%
\pgfsetbuttcap%
\pgfsetroundjoin%
\definecolor{currentfill}{rgb}{0.000000,0.000000,0.000000}%
\pgfsetfillcolor{currentfill}%
\pgfsetlinewidth{0.803000pt}%
\definecolor{currentstroke}{rgb}{0.000000,0.000000,0.000000}%
\pgfsetstrokecolor{currentstroke}%
\pgfsetdash{}{0pt}%
\pgfsys@defobject{currentmarker}{\pgfqpoint{0.000000in}{-0.048611in}}{\pgfqpoint{0.000000in}{0.000000in}}{%
\pgfpathmoveto{\pgfqpoint{0.000000in}{0.000000in}}%
\pgfpathlineto{\pgfqpoint{0.000000in}{-0.048611in}}%
\pgfusepath{stroke,fill}%
}%
\begin{pgfscope}%
\pgfsys@transformshift{2.580256in}{0.330000in}%
\pgfsys@useobject{currentmarker}{}%
\end{pgfscope}%
\end{pgfscope}%
\begin{pgfscope}%
\definecolor{textcolor}{rgb}{0.000000,0.000000,0.000000}%
\pgfsetstrokecolor{textcolor}%
\pgfsetfillcolor{textcolor}%
\pgftext[x=2.580256in,y=0.232778in,,top]{\color{textcolor}\sffamily\fontsize{10.000000}{12.000000}\selectfont 100}%
\end{pgfscope}%
\begin{pgfscope}%
\definecolor{textcolor}{rgb}{0.000000,0.000000,0.000000}%
\pgfsetstrokecolor{textcolor}%
\pgfsetfillcolor{textcolor}%
\pgftext[x=1.537500in,y=0.042809in,,top]{\color{textcolor}\sffamily\fontsize{10.000000}{12.000000}\selectfont \(\displaystyle k\)}%
\end{pgfscope}%
\begin{pgfscope}%
\pgfsetbuttcap%
\pgfsetroundjoin%
\definecolor{currentfill}{rgb}{0.000000,0.000000,0.000000}%
\pgfsetfillcolor{currentfill}%
\pgfsetlinewidth{0.803000pt}%
\definecolor{currentstroke}{rgb}{0.000000,0.000000,0.000000}%
\pgfsetstrokecolor{currentstroke}%
\pgfsetdash{}{0pt}%
\pgfsys@defobject{currentmarker}{\pgfqpoint{-0.048611in}{0.000000in}}{\pgfqpoint{0.000000in}{0.000000in}}{%
\pgfpathmoveto{\pgfqpoint{0.000000in}{0.000000in}}%
\pgfpathlineto{\pgfqpoint{-0.048611in}{0.000000in}}%
\pgfusepath{stroke,fill}%
}%
\begin{pgfscope}%
\pgfsys@transformshift{0.375000in}{0.383915in}%
\pgfsys@useobject{currentmarker}{}%
\end{pgfscope}%
\end{pgfscope}%
\begin{pgfscope}%
\definecolor{textcolor}{rgb}{0.000000,0.000000,0.000000}%
\pgfsetstrokecolor{textcolor}%
\pgfsetfillcolor{textcolor}%
\pgftext[x=0.101047in,y=0.331154in,left,base]{\color{textcolor}\sffamily\fontsize{10.000000}{12.000000}\selectfont 10}%
\end{pgfscope}%
\begin{pgfscope}%
\pgfsetbuttcap%
\pgfsetroundjoin%
\definecolor{currentfill}{rgb}{0.000000,0.000000,0.000000}%
\pgfsetfillcolor{currentfill}%
\pgfsetlinewidth{0.803000pt}%
\definecolor{currentstroke}{rgb}{0.000000,0.000000,0.000000}%
\pgfsetstrokecolor{currentstroke}%
\pgfsetdash{}{0pt}%
\pgfsys@defobject{currentmarker}{\pgfqpoint{-0.048611in}{0.000000in}}{\pgfqpoint{0.000000in}{0.000000in}}{%
\pgfpathmoveto{\pgfqpoint{0.000000in}{0.000000in}}%
\pgfpathlineto{\pgfqpoint{-0.048611in}{0.000000in}}%
\pgfusepath{stroke,fill}%
}%
\begin{pgfscope}%
\pgfsys@transformshift{0.375000in}{0.644069in}%
\pgfsys@useobject{currentmarker}{}%
\end{pgfscope}%
\end{pgfscope}%
\begin{pgfscope}%
\definecolor{textcolor}{rgb}{0.000000,0.000000,0.000000}%
\pgfsetstrokecolor{textcolor}%
\pgfsetfillcolor{textcolor}%
\pgftext[x=0.101047in,y=0.591308in,left,base]{\color{textcolor}\sffamily\fontsize{10.000000}{12.000000}\selectfont 15}%
\end{pgfscope}%
\begin{pgfscope}%
\pgfsetbuttcap%
\pgfsetroundjoin%
\definecolor{currentfill}{rgb}{0.000000,0.000000,0.000000}%
\pgfsetfillcolor{currentfill}%
\pgfsetlinewidth{0.803000pt}%
\definecolor{currentstroke}{rgb}{0.000000,0.000000,0.000000}%
\pgfsetstrokecolor{currentstroke}%
\pgfsetdash{}{0pt}%
\pgfsys@defobject{currentmarker}{\pgfqpoint{-0.048611in}{0.000000in}}{\pgfqpoint{0.000000in}{0.000000in}}{%
\pgfpathmoveto{\pgfqpoint{0.000000in}{0.000000in}}%
\pgfpathlineto{\pgfqpoint{-0.048611in}{0.000000in}}%
\pgfusepath{stroke,fill}%
}%
\begin{pgfscope}%
\pgfsys@transformshift{0.375000in}{0.904223in}%
\pgfsys@useobject{currentmarker}{}%
\end{pgfscope}%
\end{pgfscope}%
\begin{pgfscope}%
\definecolor{textcolor}{rgb}{0.000000,0.000000,0.000000}%
\pgfsetstrokecolor{textcolor}%
\pgfsetfillcolor{textcolor}%
\pgftext[x=0.101047in,y=0.851462in,left,base]{\color{textcolor}\sffamily\fontsize{10.000000}{12.000000}\selectfont 20}%
\end{pgfscope}%
\begin{pgfscope}%
\pgfsetbuttcap%
\pgfsetroundjoin%
\definecolor{currentfill}{rgb}{0.000000,0.000000,0.000000}%
\pgfsetfillcolor{currentfill}%
\pgfsetlinewidth{0.803000pt}%
\definecolor{currentstroke}{rgb}{0.000000,0.000000,0.000000}%
\pgfsetstrokecolor{currentstroke}%
\pgfsetdash{}{0pt}%
\pgfsys@defobject{currentmarker}{\pgfqpoint{-0.048611in}{0.000000in}}{\pgfqpoint{0.000000in}{0.000000in}}{%
\pgfpathmoveto{\pgfqpoint{0.000000in}{0.000000in}}%
\pgfpathlineto{\pgfqpoint{-0.048611in}{0.000000in}}%
\pgfusepath{stroke,fill}%
}%
\begin{pgfscope}%
\pgfsys@transformshift{0.375000in}{1.164378in}%
\pgfsys@useobject{currentmarker}{}%
\end{pgfscope}%
\end{pgfscope}%
\begin{pgfscope}%
\definecolor{textcolor}{rgb}{0.000000,0.000000,0.000000}%
\pgfsetstrokecolor{textcolor}%
\pgfsetfillcolor{textcolor}%
\pgftext[x=0.101047in,y=1.111616in,left,base]{\color{textcolor}\sffamily\fontsize{10.000000}{12.000000}\selectfont 25}%
\end{pgfscope}%
\begin{pgfscope}%
\pgfsetbuttcap%
\pgfsetroundjoin%
\definecolor{currentfill}{rgb}{0.000000,0.000000,0.000000}%
\pgfsetfillcolor{currentfill}%
\pgfsetlinewidth{0.803000pt}%
\definecolor{currentstroke}{rgb}{0.000000,0.000000,0.000000}%
\pgfsetstrokecolor{currentstroke}%
\pgfsetdash{}{0pt}%
\pgfsys@defobject{currentmarker}{\pgfqpoint{-0.048611in}{0.000000in}}{\pgfqpoint{0.000000in}{0.000000in}}{%
\pgfpathmoveto{\pgfqpoint{0.000000in}{0.000000in}}%
\pgfpathlineto{\pgfqpoint{-0.048611in}{0.000000in}}%
\pgfusepath{stroke,fill}%
}%
\begin{pgfscope}%
\pgfsys@transformshift{0.375000in}{1.424532in}%
\pgfsys@useobject{currentmarker}{}%
\end{pgfscope}%
\end{pgfscope}%
\begin{pgfscope}%
\definecolor{textcolor}{rgb}{0.000000,0.000000,0.000000}%
\pgfsetstrokecolor{textcolor}%
\pgfsetfillcolor{textcolor}%
\pgftext[x=0.101047in,y=1.371770in,left,base]{\color{textcolor}\sffamily\fontsize{10.000000}{12.000000}\selectfont 30}%
\end{pgfscope}%
\begin{pgfscope}%
\pgfsetbuttcap%
\pgfsetroundjoin%
\definecolor{currentfill}{rgb}{0.000000,0.000000,0.000000}%
\pgfsetfillcolor{currentfill}%
\pgfsetlinewidth{0.803000pt}%
\definecolor{currentstroke}{rgb}{0.000000,0.000000,0.000000}%
\pgfsetstrokecolor{currentstroke}%
\pgfsetdash{}{0pt}%
\pgfsys@defobject{currentmarker}{\pgfqpoint{-0.048611in}{0.000000in}}{\pgfqpoint{0.000000in}{0.000000in}}{%
\pgfpathmoveto{\pgfqpoint{0.000000in}{0.000000in}}%
\pgfpathlineto{\pgfqpoint{-0.048611in}{0.000000in}}%
\pgfusepath{stroke,fill}%
}%
\begin{pgfscope}%
\pgfsys@transformshift{0.375000in}{1.684686in}%
\pgfsys@useobject{currentmarker}{}%
\end{pgfscope}%
\end{pgfscope}%
\begin{pgfscope}%
\definecolor{textcolor}{rgb}{0.000000,0.000000,0.000000}%
\pgfsetstrokecolor{textcolor}%
\pgfsetfillcolor{textcolor}%
\pgftext[x=0.101047in,y=1.631924in,left,base]{\color{textcolor}\sffamily\fontsize{10.000000}{12.000000}\selectfont 35}%
\end{pgfscope}%
\begin{pgfscope}%
\pgfsetbuttcap%
\pgfsetroundjoin%
\definecolor{currentfill}{rgb}{0.000000,0.000000,0.000000}%
\pgfsetfillcolor{currentfill}%
\pgfsetlinewidth{0.803000pt}%
\definecolor{currentstroke}{rgb}{0.000000,0.000000,0.000000}%
\pgfsetstrokecolor{currentstroke}%
\pgfsetdash{}{0pt}%
\pgfsys@defobject{currentmarker}{\pgfqpoint{-0.048611in}{0.000000in}}{\pgfqpoint{0.000000in}{0.000000in}}{%
\pgfpathmoveto{\pgfqpoint{0.000000in}{0.000000in}}%
\pgfpathlineto{\pgfqpoint{-0.048611in}{0.000000in}}%
\pgfusepath{stroke,fill}%
}%
\begin{pgfscope}%
\pgfsys@transformshift{0.375000in}{1.944840in}%
\pgfsys@useobject{currentmarker}{}%
\end{pgfscope}%
\end{pgfscope}%
\begin{pgfscope}%
\definecolor{textcolor}{rgb}{0.000000,0.000000,0.000000}%
\pgfsetstrokecolor{textcolor}%
\pgfsetfillcolor{textcolor}%
\pgftext[x=0.101047in,y=1.892078in,left,base]{\color{textcolor}\sffamily\fontsize{10.000000}{12.000000}\selectfont 40}%
\end{pgfscope}%
\begin{pgfscope}%
\pgfsetbuttcap%
\pgfsetroundjoin%
\definecolor{currentfill}{rgb}{0.000000,0.000000,0.000000}%
\pgfsetfillcolor{currentfill}%
\pgfsetlinewidth{0.803000pt}%
\definecolor{currentstroke}{rgb}{0.000000,0.000000,0.000000}%
\pgfsetstrokecolor{currentstroke}%
\pgfsetdash{}{0pt}%
\pgfsys@defobject{currentmarker}{\pgfqpoint{-0.048611in}{0.000000in}}{\pgfqpoint{0.000000in}{0.000000in}}{%
\pgfpathmoveto{\pgfqpoint{0.000000in}{0.000000in}}%
\pgfpathlineto{\pgfqpoint{-0.048611in}{0.000000in}}%
\pgfusepath{stroke,fill}%
}%
\begin{pgfscope}%
\pgfsys@transformshift{0.375000in}{2.204994in}%
\pgfsys@useobject{currentmarker}{}%
\end{pgfscope}%
\end{pgfscope}%
\begin{pgfscope}%
\definecolor{textcolor}{rgb}{0.000000,0.000000,0.000000}%
\pgfsetstrokecolor{textcolor}%
\pgfsetfillcolor{textcolor}%
\pgftext[x=0.101047in,y=2.152233in,left,base]{\color{textcolor}\sffamily\fontsize{10.000000}{12.000000}\selectfont 45}%
\end{pgfscope}%
\begin{pgfscope}%
\pgfsetbuttcap%
\pgfsetroundjoin%
\definecolor{currentfill}{rgb}{0.000000,0.000000,0.000000}%
\pgfsetfillcolor{currentfill}%
\pgfsetlinewidth{0.803000pt}%
\definecolor{currentstroke}{rgb}{0.000000,0.000000,0.000000}%
\pgfsetstrokecolor{currentstroke}%
\pgfsetdash{}{0pt}%
\pgfsys@defobject{currentmarker}{\pgfqpoint{-0.048611in}{0.000000in}}{\pgfqpoint{0.000000in}{0.000000in}}{%
\pgfpathmoveto{\pgfqpoint{0.000000in}{0.000000in}}%
\pgfpathlineto{\pgfqpoint{-0.048611in}{0.000000in}}%
\pgfusepath{stroke,fill}%
}%
\begin{pgfscope}%
\pgfsys@transformshift{0.375000in}{2.465148in}%
\pgfsys@useobject{currentmarker}{}%
\end{pgfscope}%
\end{pgfscope}%
\begin{pgfscope}%
\definecolor{textcolor}{rgb}{0.000000,0.000000,0.000000}%
\pgfsetstrokecolor{textcolor}%
\pgfsetfillcolor{textcolor}%
\pgftext[x=0.101047in,y=2.412387in,left,base]{\color{textcolor}\sffamily\fontsize{10.000000}{12.000000}\selectfont 50}%
\end{pgfscope}%
\begin{pgfscope}%
\definecolor{textcolor}{rgb}{0.000000,0.000000,0.000000}%
\pgfsetstrokecolor{textcolor}%
\pgfsetfillcolor{textcolor}%
\pgftext[x=0.045492in,y=1.485000in,,bottom,rotate=90.000000]{\color{textcolor}\sffamily\fontsize{10.000000}{12.000000}\selectfont Number of GMRES Iterations}%
\end{pgfscope}%
\begin{pgfscope}%
\pgfsetrectcap%
\pgfsetmiterjoin%
\pgfsetlinewidth{0.803000pt}%
\definecolor{currentstroke}{rgb}{0.000000,0.000000,0.000000}%
\pgfsetstrokecolor{currentstroke}%
\pgfsetdash{}{0pt}%
\pgfpathmoveto{\pgfqpoint{0.375000in}{0.330000in}}%
\pgfpathlineto{\pgfqpoint{0.375000in}{2.640000in}}%
\pgfusepath{stroke}%
\end{pgfscope}%
\begin{pgfscope}%
\pgfsetrectcap%
\pgfsetmiterjoin%
\pgfsetlinewidth{0.803000pt}%
\definecolor{currentstroke}{rgb}{0.000000,0.000000,0.000000}%
\pgfsetstrokecolor{currentstroke}%
\pgfsetdash{}{0pt}%
\pgfpathmoveto{\pgfqpoint{2.700000in}{0.330000in}}%
\pgfpathlineto{\pgfqpoint{2.700000in}{2.640000in}}%
\pgfusepath{stroke}%
\end{pgfscope}%
\begin{pgfscope}%
\pgfsetrectcap%
\pgfsetmiterjoin%
\pgfsetlinewidth{0.803000pt}%
\definecolor{currentstroke}{rgb}{0.000000,0.000000,0.000000}%
\pgfsetstrokecolor{currentstroke}%
\pgfsetdash{}{0pt}%
\pgfpathmoveto{\pgfqpoint{0.375000in}{0.330000in}}%
\pgfpathlineto{\pgfqpoint{2.700000in}{0.330000in}}%
\pgfusepath{stroke}%
\end{pgfscope}%
\begin{pgfscope}%
\pgfsetrectcap%
\pgfsetmiterjoin%
\pgfsetlinewidth{0.803000pt}%
\definecolor{currentstroke}{rgb}{0.000000,0.000000,0.000000}%
\pgfsetstrokecolor{currentstroke}%
\pgfsetdash{}{0pt}%
\pgfpathmoveto{\pgfqpoint{0.375000in}{2.640000in}}%
\pgfpathlineto{\pgfqpoint{2.700000in}{2.640000in}}%
\pgfusepath{stroke}%
\end{pgfscope}%
\end{pgfpicture}%
\makeatother%
\endgroup%

   \caption[Maximum GMRES iteration counts when $\NLiDRR{\nso-\nst} = 0.5\times  k^{-\beta}$ for $\beta = 0.8,0.9,1.$]{Maximum GMRES iteration counts for solving systems with matrix $\AmatoI\Amatt$, where $\Aso=\Ast=1$ and $\NLiDRR{\nso-\nst} = 0.5\times  k^{-\beta}$ for $\beta = 0.8,0.9,1.$}\label{fig:linfinityn2}
\end{figure}
  
%  \ednote{Euan---I had to redo these computations, as there was a bug in my code, so we now don't see \emph{really} rapid growth in the number of GMRES iterations as $k$ inreases. I think the previous code may have may the variations in $n$ too large, meaning we got really rapid growth.}


%Say that these are for the TEDP defined in \cref{def:TEDP}.

\section[Proofs of Theorems \MakeLowercase{\ref{cor:1}, \ref{cor:1a}, and \ref{thm:2}} and Lemma \MakeLowercase{\ref{lem:sharp}}]{Proofs of Theorems \ref{cor:1}, \ref{cor:1a}, and \ref{thm:2} and \newline Lemma \ref{lem:sharp}}\label{sec:3}

%then the weighted norm $\|\cdot\|_{\Dmat_k}$ is given by 
%\beq\label{eq:Dk3}
%\N{\vvec}_{\Dmat_k}^2\de   \N{v_h}^2_{\HokD}=\big( \Dmat_k \vvec,\vvec\big)_2,
%\eeq
%for
%$v_h =\sum_i v_i \phi_i$.



%For a proof of \cref{lem:normequiv}, see\footnote{In \cite[Chapter V, Lemma 2.5]{Br:07} the assumption is made that the meshes underlying $\Vhp$ are \emph{uniform}. However, the definition of uniformity in \cite[Chapter 2, Definition 5.1(4)]{Br:07} is the same as the more standard definition of quasi-uniformity in \cref{def:quasiuniform}.} \cite[Chapter V, Lemma 2.5]{Br:07}.

%% \bpf[Sketch proof of \cref{lem:normequiv}]\opntodo{can omit this if can find a good reference. One possibility . Need to check basis scaling business.}
%% The inequalities in \cref{eq:normequiv1} follow from writing $\|v_h\|_{\LtD}$ as a sum of integrals over elements of the mesh, and then mapping to the reference element \ednote{Euan to discuss with Ivan}.
%% %\beqs
%% %\N{v_h}^2_{L^2(\O
%% %\eeqs
%% Then, \cref{eq:normequiv2} follows from \cref{eq:normequiv1} and the inequalities
%% \beqs
%% \N{v_h}_{L^2(D)}\lesssim \N{\nabla v_h}_{L^2(D)}\lesssim \frac{1}{h} \N{v_h}_{L^2(D)},
%% \eeqs
%% the first of which follows from the Poincar\'e inequality, since $v_h \in \HozDD$
%% (see, e.g., \cite[Proposition 5.3.4]{BrSc:00}), the second of which follows from a standard inverse estimate (see, e.g., \cite[Theorem 4.5.11]{BrSc:00}).
%% \epf




\subsection{Proof of the main ingredient of the proofs of \cref{cor:1,cor:1a}}\label{sec:proofs}
As the first step towards proving \cref{cor:1,cor:1a}, we prove \cref{lem:normequiv}, concerning norm equivalences of finite-element functions.

\bpf[Proof of \cref{lem:normequiv}]
\label{page:normequivpf}
We first show \cref{eq:normequiv1} by direct computation, before concluding \cref{eq:normequiv2} from \cref{eq:normequiv1} and a standard inverse inequality. Throughout this proof, when we use $\sim$ the hidden constants are independent of $\tau \in \Th,$ the mesh size $h$, and $\vh \in \Vhp,$ but may depend on $p.$

For any $\vh \in \Vhp$ we have (letting $\bnj$ denote a node of $\Th$)
\begin{align}
  \NLtD{\vh}^2 &= \sum_{\tau \in \Th} \int_\tau \abs{\vh}^2\label{eq:twiddlenumber1}\\
  &\sim \sum_{\tau \in \Th} \abs{\tau} \sum_{\bnj \in \tau} \abs{\vh(\bnj)}^2 \label{eq:twiddlenumber2}\\
  &\sim h^d \sum_{\tau \in \Th} \sum_{\bnj \in \tau} \abs{\vh(\bnj)}^2, \text{ by quasi-uniformity,}\nonumber\\
  &\sim h^2 \Nt{\uvec},
\end{align}
i.e., \cref{eq:normequiv1}. The expression \cref{eq:twiddlenumber2} follows from \cref{eq:twiddlenumber1} because the terms defined on $\tau$ are equivalent norms on $\tau$ of functions in $\Vhp.$

To show \cref{eq:normequiv2} we recall the standard inverse inequality (see, e.g., \cite[Theorem 4.5.11 and Remark 4.5.20]{BrSc:08})
\beq\label{eq:twiddleinverse}
\NHoD{\vh} \lesssim h^{-1} \NLtD{\vh}.
\eeq
By combining \cref{eq:twiddleinverse} and the right-hand side of \cref{eq:normequiv1}, we obtain \cref{eq:normequiv2}.
\epf


The main part of the proofs of \cref{cor:1,cor:1a} is the following \lcnamecref{thm:1}.

\begin{theorem}[Main ingredient of the answer to \cref{it:nbpcq1}]\label{thm:1}
Let $\kz \geq 0,$ $k \geq \kz,$ and assume $\Dm$, $\Aso$, and $\nso$ satisfy \cref{cond:1nbpc}, and assume that $h$ and $p$ satisfy \cref{cond:2}. 
Let the $k$- and $h$-independent constants $\mpm$ and $\splus$ be given as in \cref{lem:normequiv}.
Then there exist $\Co, \Ct>0$, independent of $h$ and $k$ (but dependent on $\Dm, \Aso, \nso$, $p$, and $\kz$) such that
\begin{align}\nonumber
&\max\Big\{
\NDmatk{\Imat - (\Amat^{(1)})^{-1}\Amat^{(2)}}, 
\N{\Imat -\Amat^{(2)} (\Amat^{(1)})^{-1}}_{(\Dmat_k)^{-1}}
\Big\}\\
&\hspace{3cm} 
\leq C_1 \,k \,
\NLiDop{\Aso-\Ast} + C_2 \, k \, \NLiDRR{\nso-\nst}
\label{eq:main1}
\end{align}
and 
\begin{align}\nonumber
&\max\Big\{
\N{\Imat - (\Amat^{(1)})^{-1}\Amat^{(2)}}_2, 
\N{\Imat -\Amat^{(2)} (\Amat^{(1)})^{-1}}_2
\Big\}\\
&\hspace{0cm} 
\leq C_1 \,\left(\frac{s_+}{m_-}\right) \,\frac{1}{h} \,
\NLiDop{\Aso-\Ast} + C_2 \, \left(\frac{m_+}{m_-} \right)k \, \NLiDRR{\nso-\nst}.
\label{eq:main1a}
\end{align}
\end{theorem}

The proof of \cref{thm:1} is given after the following two lemmas, that are the heart of the proof of \cref{thm:1}.

\ble[Bounds on $(\Amato)^{-1} \Mmat_{n}$]\label{lem:keylemma1}
Assume that \cref{cond:1nbpc} holds, and assume that part (i) of \cref{cond:2} holds. Then, for $n\in \LiDRR$,
\beq\label{eq:keybound1}
\max\Big\{\big\| (\Amato)^{-1} \Mmat_{n} \big\|_{\Dmat_k}, \,\,
\big\|  \Mmat_{n}(\Amato)^{-1} \big\|_{(\Dmat_k)^{-1}}
\Big\}\leq 
C_2
%\frac{m_+}{m_-} \left[ C_{\rm FEM1}^{(1)} + C_{\rm bound}^{(1)}\right] 
\frac{\NLiDRR{n}}{k}
\eeq
and 
\beq\label{eq:keybound1a}
\max\Big\{\big\| (\Amato)^{-1} \Mmat_{n} \big\|_2, \,\,
\big\|  \Mmat_{n}(\Amato)^{-1} \big\|_2 
\Big\}\leq 
C_2 
%\frac{m_+}{m_-} \left[ C_{\rm FEM1}^{(1)} + C_{\rm bound}^{(1)}\right] 
\left(\frac{m_+}{m_-}\right) \frac{\NLiDRR{n}}{k},
\eeq
where
\beq\label{eq:C2}
C_2\de%\frac{m_+}{m_-} 
%\left[ 
C_{\rm FEM1}^{(1)} + C_{\rm bound}^{(1)}.%\right].
\eeq
\ele

The proof of \cref{lem:keylemma1} is on \cpageref{page:lemkeylemma1proof} below.

\ble[Bounds on $(\Amato)^{-1} \Smat_A$]\label{lem:keylemma2}
Assume that \cref{cond:1nbpc} holds, and assume that part (ii) of \cref{cond:2} holds. Then, for $A\in \LiDRRdtd$,
\beq\label{eq:keybound2}
\max\Big\{\big\| (\Amato)^{-1} \Smat_A \big\|_{(\Dmat_k)^{-1}}, \,\,
\big\| \Smat_A (\Amato)^{-1} \big\|_{\Dmat_k}\Big\} \leq C_1\, k\NLiDop{A}
\eeq
and
\beq\label{eq:keybound2a}
\max\Big\{\big\| (\Amato)^{-1} \Smat_A \big\|_2, \,\,
\big\| \Smat_A (\Amato)^{-1} \big\|_2\Big\} \leq C_1\,\left(\frac{s_+}{m_-}\right) \frac{1}{h}\NLiDop{A},
\eeq
%\begin{align}\nonumber
%&\max\Big\{\big\| (\Amato)^{-1} \Smat_A \big\|_2, \,\,
%\big\| \Smat_A (\Amato)^{-1} \big\|_2\Big\}\nonumber \\
%&\hspace{2cm}
% \leq \frac{s_+}{s_-} \left[ C_{\rm FEM2}^{(1)} + 
% \frac{1}{\min\big\{\Asomin,\nsomin\big\}}\left( \frac{1}{k_0} + 2 C^{(1)}_{\rm bound}\nsomax  \right) \right]k\N{A}_{L^\infty(D)}\label{eq:keybound2}
%% + C_{\rm bound}^{(1)}\right) \frac{\N{n}_{L^\infty(D)}}{k}.
%\end{align}
where
\beq\label{eq:C1nbpc}
C_1\de%\frac{s_+}{s_-} 
\left[ C_{\rm FEM2}^{(1)} + 
 \frac{1}{\min\big\{\Asomin,\nsomin\big\}}\left( \frac{1}{k_0} + 2 C^{(1)}_{\rm bound}\nsomax  \right) \right].
\eeq
\ele

The proof of \cref{lem:keylemma2} is on \cpageref{page:lemkeylemma2proof} below.

\bpf[Proof of \cref{thm:1} using \cref{lem:keylemma1,lem:keylemma2}]
Using the definition of the matrices $\Amatj, \SmatA$, and $\Mmatn$ in \cref{eq:matrixAjdef} and \cref{eq:matrixSjdef}, we have
\begin{align}\nonumber
\Imat - (\Amato)^{-1}\Amatt = (\Amato)^{-1}\big(\Amato-\Amatt\big) &=  (\Amato)^{-1}\left( \Smat_{A^{(1)}} - \Smat_{A^{(2)}} - k^2 \big(\Mmat_{n^{(1)}}-\Mmat_{n^{(2)}}\big)\right)\\
&= (\Amato)^{-1}\left( \Smat_{A^{(1)}-A^{(2)}} - k^2 \Mmat_{n^{(1)}-n^{(2)}}\right),\label{eq:idea1}
\end{align}
and similarly 
\beq\label{eq:idea2}
\Imat -\Amatt  (\Amato)^{-1}= \left( \Smat_{A^{(1)}-A^{(2)}} - k^2 \Mmat_{n^{(1)}-n^{(2)}}\right)(\Amato)^{-1}.
\eeq
The bounds in  \cref{eq:main1} on $\NDk{\Imat - (\Amato)^{-1}\Amatt}$ and  $\NDkI{\Imat - \Amatt(\Amato)^{-1}}$ then follow from using the bounds \cref{eq:keybound1,eq:keybound2} in \cref{eq:idea1,eq:idea2}. The bounds in \Cref{eq:main1a} on $\Nt{\Imat - (\Amato)^{-1}\Amatt}$ and  $\Nt{\Imat - \Amatt(\Amato)^{-1}}$ follow completely analagously, except we use the bounds \cref{eq:keybound1a,eq:keybound2a} instead of the bounds \cref{eq:keybound1,eq:keybound2}.
%
%, and $C_1$, $C_2$ in \cref{eq:main1} are given explicitly by
%\beq\label{eq:C1nbpc}
%C_1\de%\frac{s_+}{s_-} 
%\left[ C_{\rm FEM2}^{(1)} + 
% \frac{1}{\min\big\{\Asomin,\nsomin\big\}}\left( \frac{1}{k_0} + 2 C^{(1)}_{\rm bound}\nsomax  \right) \right] \,\,\tand\,\,
%\quad C_2\de  %+ \frac{m_+}{m_-} 
% \left[ C_{\rm FEM1}^{(1)} + C_{\rm bound}^{(1)}\right].
%\eeq
\epf

\subsubsection{Proofs of \cref{lem:keylemma1,lem:keylemma2}}

The proofs of \cref{lem:keylemma1,lem:keylemma2} require the concept of the \emph{adjoint} sesquilinear form to $\aG(\cdot,\cdot)$.

\begin{definition}[The adjoint sesquilinear form $\aGs(\cdot,\cdot)$]\label{def:adjoint}
Let $D$, $A$, and $n$ be as in \cref{prob:vgen}. The adjoint sesquilinear form, $\aGs(\cdot,\cdot)$, to $\aG(\cdot,\cdot)$ defined in \cref{eq:agen} is given by
\beq\label{eq:EDPadjoint}
\aGs(w,v) \de \int_{D} 
\Big((A \grad w)\cdot\grad \vb
 - k^2 n w\vb\Big) -  \DPGI{\trI w}{\T(\trI v)}.
\eeq
\end{definition}

\noi It is then straightforward to check that
\beq\label{eq:A*}
\Amat^\dagger \de \Smat_A -k^2 \Mmat_n - \Nmat^\dagger
\eeq
(where $^\dagger$ denotes conjugate transpose) is the Galerkin matrix for the sesquilinear form $\aGs(\cdot,\cdot)$; i.e.~$(\Amat^\dagger)_{ij} = \aGs(\phi_j, \phi_i)$.

The following \lcnamecref{lem:adjoint} shows that if $w$ solves an adjoint Helmholtz problem, then $\wbar$ solves a (standard) Helmholtz problem with a related right-hand side.

\ble[Link between variational problems involving $\aG(\cdot,\cdot)$ and $\aGs(\cdot,\cdot)$]\label{lem:adjoint}

\

\noindent If the source term $\LG$ is as in \cref{prob:vgen}, $w$ is the solution to \cref{prob:vgen}, the boundary operator $\T$ satisfies
\beq\label{eq:DPconj}
\DPGI{\T \psi}{\phibar} = \DPGI{\T \phi}{\psibar} \tfa \phi,\psi \in \HhGI,
\eeq
and
\beq\label{eq:adjoint1}
\aGs(w,v)= \LG(v) \quad\tfa v\in \HozDD,
\eeq
then $\overline{w}$ satisfies
\beq\label{eq:adjoint2}
\aG(\overline{w},v)= \overline{\LG(\overline{v})} \quad\tfa v\in \HozDD.
\eeq
\ele

\bpf[Proof of \cref{lem:adjoint}]
From \cref{eq:adjoint1} we have that 
\beqs
\overline{\aGs(w,\overline{v})}= \overline{\LG(\overline{v})} \quad\tfa v\in \HozDD.
\eeqs
Using the definition of $\aGs(\cdot,\cdot)$ and the property \cref{eq:DPconj} in the left-hand side of this last equation, we find \cref{eq:adjoint2}.
\epf

\bco[\Cref{eq:adjoint2} holds for \cref{prob:vedp,prob:vtedp}]\label{cor:adjoint}
If \cref{prob:vgen} is chosen to represent either \cref{prob:vedp} or \cref{prob:vtedp}, then \cref{eq:adjoint2} holds.
\eco

\bpf[Proof of \cref{cor:adjoint}]
The only thing we need to check is that \cref{eq:DPconj} holds for both \cref{prob:vedp,prob:vtedp}. For \cref{prob:vtedp}, when $\T = ik$, the proof is straightforward. For \cref{prob:vedp} when $\T = \DtN$ we need the following property of the DtN map $\DtN$:
\beq\label{eq:DtN}
\DPGR{\DtN\psi}{\phibar} = \DPGR{\DtN \phi}{\psibar} \quad\tfa \phi,\psi \in H^{1/2}(\GR).
\eeq
This property follows from the fact that, if $\uo$ and $\ut$ are solutions of the homogeneous Helmholtz equation $\Delta u +k^2 u=0$ in $\RRd\setminus \overline{\BR}$, both satisfying the Sommerfeld radiation condition \cref{eq:src}, then
\beqs
\int_{\GR} (\gamma \uo)\, \dn \utb = \int_{\GR} (\gamma \ut)\,\dn \uob;
\eeqs
which follows from Green's theorem and, e.g., \cite[Lemma 4.10]{Sp:15}.
\epf

We can now prove \cref{lem:keylemma1,lem:keylemma2}.

\bpf[Proof of \cref{lem:keylemma1}]
\label{page:lemkeylemma1proof}
We first concentrate on proving \cref{eq:keybound1}.
Given $\fvec \in \CC^N$ and $n\in \LiDRR$, we create a variational problem whose Galerkin discretisation leads to the equation $\Amato \tbu = \Mmat_n\,\fvec$.
Indeed, let $\widetilde{f} \de \sum_j f_j \phi_j\in \HozDD$. Define $\widetilde{u}$ to be the solution of the variational problem 
\beq\label{eq:411}
a^{(1)}(\widetilde{u},v)= \IPLtD{n\widetilde{f}}{v} \quad\text{ for all } v\in \HozDD,
\eeq
and let $\tu_h$ be the solution of the finite-element approximation of \cref{eq:411}, i.e.,
\beq\label{eq:41}
a^{(1)}(\tu_h,v_h)= \IPLtD{n\widetilde{f}}{v_h} \quad\text{ for all } v_h\in \Vhp,
\eeq
and let $\tbu$ be the vector of nodal values of $\tu_h$. The definition of $\widetilde{f}$ then implies that \cref{eq:41} is equivalent to the linear system $\Amato \tbu = \Mmat_{n}\,\fvec$, and so to obtain a bound on $\|(\Amato)^{-1}\Mmat_n\|_{\Dmat_k}$ we need to bound $\|\tbu\|_{\Dmat_k}$ in terms of $\|\fvec\|_{\Dmat_k}$. (Recall $\fvec \in \CCN$ was arbitrary.) Because of the definition 
of $\|\cdot\|_{\Dmat_k}$ in \cref{eq:Dk}, this bound is equivalent to bounding $\|\tu_h\|_{\HokD}$ in terms of $\|\widetilde{f}\|_{\HokD}$.

%First observe that the bound \cref{eq:bound3} from Part (i) of \cref{cond:2} holds for the solution of the variational problem
%\beqs%\label{eq:411}
%a^{(1)}(u,v)= (n\phi_j,v)_{L^2(\Omega)} \quad\text{ for all } v\in H^1(\Omega),
%\eeqs
%and hence, by linearity, it also holds for the solution $\widetilde{u}$ of the variational problem \cref{eq:411}.

Using %the bounds in \cref{eq:normequiv1}, 
the triangle inequality and the bounds \cref{eq:bound1} and \cref{eq:bound3} from \cref{cond:2,cond:1nbpc} respectively, we find
%Note that the hypotheses imply that the bound on the solution operator 
%\cref{eq:bound_unif} holds (by \cref{cor:uniform}), and also that if $h k\sqrt{|k^2-\eps|} \leq C_1$ then quasi-optimality \cref{eq:qoeps_lemma} holds (by \cref{lem:qo}).
%Starting with \cref{eq:equiv} we then have 
\begin{align}
%m_- h^{d/2}k \N{\tbu}_2 \leq k\N{\tu_h}_{\LtD}\leq  
\N{\tu_h}_{\HokD} \leq
\N{\tu-\tu_h}_{\HokD} + \N{\tu}_{\HokD} \label{eq:mainevent1}
& \leq C^{(1)}_{\rm FEM1}\NLtD{n\ftilde} + C^{(1)}_{\rm bound}\NLtD{n\ftilde} \\ 
& \leq \mleft(C^{(1)}_{\rm FEM1} + C^{(1)}_{\rm bound}\mright)\NLiDRR{n}\NLtD{\ftilde} \label{eq:mainevent1a} \\
& \leq\big(C^{(1)}_{\rm FEM1}+  C^{(1)}_{\rm bound}\big)\NLiDRR{n}\frac{\big\|\widetilde{f}\big\|_{\HokD}}{k};\nonumber
%& \leq\big(C^{(1)}_{\rm FEM1}+  C^{(1)}_{\rm bound}\big)\N{n}_{L^\infty(D)} m_+ h^{d/2} \N{\fvec}_2,
\end{align}
the bound on $\|(\Amato)^{-1}\Mmat_n\|_{\Dmat_k}$ in \cref{eq:keybound1} then follows from the definition of $\|\cdot\|_{\Dmat_k}$ in \cref{eq:Dk} and the definition of $C_2$ \cref{eq:C2}.

To prove the bound on $\|\Mmat_n(\Amato)^{-1}\|_{(\Dmat_k)^{-1}}$ in \cref{eq:keybound1}, first observe that the definitions of $\|\cdot\|_{\Dmat_k}$ and $\|\cdot\|_{(\Dmat_k)^{-1}}$ in \cref{eq:Dk} imply that, for any matrix $\Cmat \in \CCNtN$ and for any $\vvec\in \CC^N$,
\beq\label{eq:A380-0}
\frac{
\big\|\matrixC \vvec \big\|_{(\Dmat_k)^{-1}}
}{
\big\|\vvec\|_{(\Dmat_k)^{-1}}
} = 
\frac{
\big\|\matrixC^\dagger \wvec \big\|_{\Dmat_k}
}{
\big\|\wvec\|_{\Dmat_k}
}
\eeq
where $\wvec \de (\Dmat_k)^{1/2}\vvec$, and where $\matrixC^\dagger$ is the conjugate transpose of $\matrixC$ (i.e.~the adjoint with respect to $(\cdot,\cdot)_2$).
Therefore, since $\Mmat_n$ is a real, symmetric matrix,
\beqs
\frac{
\big\|\Mmat_n (\Amato)^{-1}\vvec\big\|_{(\Dmat_k)^{-1}}
}{
\N{\vvec}_{(\Dmat_k)^{-1}}
}
=
\frac{\NDk{\mleft(\AmatoI\Mmatn\mright)^\dagger \wvec}}{\NDk{\wvec}}
= 
\frac{
\big\|((\Amato)^\dagger)^{-1}\Mmat_n\wvec\big\|_{\Dmat_k}
}{
\N{\wvec}_{\Dmat_k}
},
 \eeqs
 so that 
\beq\label{eq:A380} 
 \big\|\Mmat_n (\Amato)^{-1}\big\|_{(\Dmat_k)^{-1}}=\big\|((\Amato)^\dagger)^{-1}\Mmat_n\big\|_{\Dmat_k}.
 \eeq 
Recall from the text below \cref{eq:A*} that $(\Amato)^\dagger$ is the Galerkin matrix corresponding to the variational problem \cref{eq:adjoint1} -- the adjoint problem. \cref{lem:adjoint} implies that if the EDP %with coefficients $A^{(1)}$ and $n^{(1)}$ 
satisfies \cref{cond:1nbpc,cond:2}, then so does the adjoint problem. Therefore, the argument above leading to the bound on $\|(\Amato)^{-1}\Mmat_n\|_{\Dmat_k}$ under \cref{cond:1nbpc} and Part (i) of \cref{cond:2} proves the same bound on $\|((\Amato)^\dagger)^{-1}\Mmat_n\|_{\Dmat_k}$, and then, using \cref{eq:A380}, also on $\big\|\Mmat_n(\Amato)^{-1}\big\|_{(\Dmat_k)^{-1}}$.

To prove the bound on  $\|(\Amato)^{-1}\Mmat_n\|_{2}$ in \cref{eq:keybound1a}, we use the bounds 
\beqs
m_- h^{d/2} k \N{\tbu}_2 \leq k \N{\widetilde{u}_h}_{\LtD} \leq \N{\widetilde{u}_h}_{\HokD}
\,\tand\,
\big\|\widetilde{f}\big\|_{\LtD} \leq m_+ h^{d/2}\N{\fvec}_2,
\eeqs
on either side of the inequality \cref{eq:mainevent1}, with these bounds coming from \cref{eq:normequiv1}. The proof of the bound on 
$\|\Mmat_n((\Amato)^\dagger)^{-1}\|_{2}$ in \cref{eq:keybound1a} follows in a similar way to above, using the fact that 
$\|\Mmat_n (\Amato)^{-1}\|_2=\|((\Amato)^\dagger)^{-1}\Mmat_n\|_2$ (compare to \cref{eq:A380}).
%, namely the variational problem \cref{eq:EDPvar} with the operator $T_R$ in $a^{(1)}(\cdot,\cdot)$ replaced by $\overline{T_R}$ (corresponding to the $-\ri k$ in the radiation condition \cref{eq:src} being changed to $+i k$).
%
%Now, if $u$ is the solution of the adjoint problem with data $\LE(v)$, then $\overline{u}$ is the solution of the original problem with data $\overline{\LE(\overline{v})}$; 
%
%in particular if $\LE(v)$ is as in \cref{eq:EDPa}, then the $L^2$ data of the adjoint problem is just $\overline{f}$. Therefore, if the EDP satisfies \cref{cond:1nbpc,cond:2}, then so does its adjoint, and
% the bound in \cref{eq:keybound1} on $\|(\Amato)^{-1}\Mmat_n\|_{2}$ also holds for $\|((\Amato)^\dagger)^{-1}\Mmat_n\|_{2}$.
\epf

The proof of \cref{lem:keylemma2} uses the following \lcnamecref{lem:H1}, which one can prove using the G\aa rding inequality \cref{eq:gardingbrief}; see \cite[Lemma 5.1]{GrPeSp:19}.

\ble[Bound for data in $\HozDDs$]\label{lem:H1}
%With the sesquilinear form $a(\cdot,\cdot)$ defined by \cref{eq:EDPa} with $A=\Aso$ and $n=\nso$, 
Given $\LGtilde\in \HozDDs$, let $\widetilde{u}$ be the solution of the variational problem
\beqs
\text{ find } \,\,\widetilde{u} \in H^1_{0,D}(D) \,\,\tst \,\,
a^{(1)}(\widetilde{u},v)=\LGtilde(v) \,\, \tfa v\in H^1_{0,D}(D).
\eeqs
If \cref{cond:1nbpc} holds, then $\widetilde{u}$ exists, is unique, and satisfies the bound
\beq\label{eq:bound2}
\N{\widetilde{u}}_{\HokD} \leq \frac{1}{\min\{\Asomin,\nsomin\}}\left( 1 + 2 C^{(1)}_{\rm bound}\nsomax  k\right) \big\|\LGtilde\big\|_{(\HokD)^*}
\eeq
for all $k\geq k_0$.
\ele
%Observe that, similar to \cref{rem:yesitis}, \cref{eq:bound2} is a $k$-independent bound, due to the norm $\NHokDs{\LE}$ on the right-hand side.


\bpf[Proof of \cref{lem:keylemma2}]
\label{page:lemkeylemma2proof}
In a similar way to the proof of \cref{lem:keylemma1}, given $\fvec \in \CC^N$ and $A\in \LiDRRdtd$, let $\widetilde{f} \de \sum_j f_j \phi_j$ and observe that $\widetilde{f} \in \HozDD$. Define $\widetilde{u}$ to be the solution of the variational problem 
\beq\label{eq:411a}
a^{(1)}(\widetilde{u},v)= \LGtilde(v) \quad\text{ for all } v\in \HozDD,
\quad\text{ where } \quad
 \LGtilde(v) \de \IPLtD{(A\nabla\widetilde{f}}{\nabla v}.
\eeq
Observe that the definition of the norms $\|\cdot\|_{(\HokD)^*}$ and $\|\cdot\|_{\HokD}$ \cref{eq:weightednorm} and the Cauchy-Schwarz inequality imply that
\begin{align}
\big\| \LGtilde\big\|_{(\HokD)^*}&\leq \big\|A\nabla \widetilde{f}\big\|_{\LtD}\nonumber\\
&\leq \NLiDop{A} \big\|\nabla \widetilde{f}\big\|_{\LtD}\label{eq:Fbounda}\\
&\leq \NLiDop{A} \big\| \widetilde{f}\big\|_{\HokD}.\label{eq:Fbound}
\end{align}
Let $\tu_h$ be the solution of the finite element approximation of \cref{eq:411a}, i.e.,
\beq\label{eq:41a}
a^{(1)}(\tu_h,v_h)= \LGtilde(v_h) \quad\text{ for all } v_h\in \Vhp,
\eeq
and let $\tbu$ be the vector of nodal values of $\tu_h$. The definition of $\widetilde{f}$ then implies that \cref{eq:41a} is equivalent to $\Amato \tbu = \Smat_A\,\fvec$. 

Similar to the proof of \cref{lem:keylemma1},
using the triangle inequality, the bound \cref{eq:bound4} from \cref{cond:2}, the bound \cref{eq:bound2} from \cref{lem:H1}, the bound \cref{eq:Fbound}, and the definition of $C_1$ \cref{eq:C1nbpc},
we find
%Note that the hypotheses imply that the bound on the solution operator 
%\cref{eq:bound_unif} holds (by \cref{cor:uniform}), and also that if $h k\sqrt{|k^2-\eps|} \leq C_1$ then quasi-optimality \cref{eq:qoeps_lemma} holds (by \cref{lem:qo}).
%Starting with \cref{eq:equiv} we then have 
\begin{align}\nonumber 
%s_- h^{(d-2)/2} \N{\tbu}_2 &\leq \N{\nabla \tu_h}_{\LtD}\leq  
\N{\tu_h}_{\HokD} &\leq
\N{\tu-\tu_h}_{\HokD} + \N{\tu}_{\HokD},\nonumber \\ \nonumber
& \leq \left[ C^{(1)}_{\rm FEM2} k + 
\frac{1}{\min\{\Asomin,\nsomin\}}\left( 1 + 2 C^{(1)}_{\rm bound}\nsomax k  \right) 
\right]\big\|\LGtilde\big\|_{(\HokD)^*},\\
&\leq C_1 \, k\, 
%\left[C^{(1)}_{\rm FEM2} k + \frac{1}{\min\{\Asomin,\nsomin\}}\left( 1 + 2 C^{(1)}_{\rm bound}\nsomaxk  \right) \right]
\NLiDop{A} \big\|\nabla\widetilde{f}\big\|_{\LtD},\label{eq:mainevent2}\\
&\leq C_1 \, k\, 
%\left[C^{(1)}_{\rm FEM2} k + \frac{1}{\min\{\Asomin,\nsomin\}}\left( 1 + 2 C^{(1)}_{\rm bound}\nsomaxk  \right) \right]
\NLiDop{A} \big\|\widetilde{f}\big\|_{\HokD},\nonumber
%&\leq \left[ C^{(1)}_{\rm FEM2} k + 
%\frac{1}{\min\{\Asomin,\nsomin\}}\left( 1 + 2 C^{(1)}_{\rm bound}\nsomaxk  \right) 
%\right]\big\|A\big\|_{L^\infty(D)}s_+ h^{(d-2)/2} \N{\fvec}_2,
\end{align}
and the bound on $\|(\Amato)^{-1}\Smat_A\|_{\Dmat_k}$ in \cref{eq:keybound2} follows.

The bound on $\|\Smat_A(\Amato)^{-1}\|_{(\Dmat_k)^{-1}}$ follows in a similar way to how we obtained the 
bound on  $\|\Mmat_n(\Amato)^{-1}\|_{(\Dmat_k)^{-1}}$ from the bound on $\|(\Amato)^{-1}\Mmat_n\|_{\Dmat_k}$ in Part (i). Indeed, 
\cref{eq:A380-0} and the fact that $\Smat_A$ is a real, symmetric matrix imply that 
\beq\label{eq:A380-2} 
 \big\|\Smat_A (\Amato)^{-1}\big\|_{(\Dmat_k)^{-1}}=\big\|\big((\Amato)^\dagger\big)^{-1}\Smat_A\big\|_{\Dmat_k}
 \eeq 
%since 
%\beqs
%\big\|\Smat_A(\Amato)^{-1}\big\|_{2}=\big\|(\Smat_A(\Amato)^{-1})^\dagger\big\|_{2}=\big\|((\Amato)^\dagger)^{-1}\Smat_A\big\|_{2},
%\eeqs
(c.f. \cref{eq:A380}),
and then the arguments in the proof of part (i) imply that 
the bound in \cref{eq:keybound2} on $\|(\Amato)^{-1}\Smat_A\|_{\Dmat_k}$ also holds for $\|((\Amato)^\dagger)^{-1}\Smat_A\|_{\Dmat_k}$.

To prove the bound on  $\|(\Amato)^{-1}\Smat_A\|_{2}$ in \cref{eq:keybound2a}, we use the bounds 
\beqs
m_- h^{d/2} k \N{\tbu}_2 \leq k \N{\widetilde{u}_h}_{\LtD} \leq \N{\widetilde{u}_h}_{\HokD}
\,\tand\,
\big\|\nabla \widetilde{f}\big\|_{\LtD} \leq s_+ h^{d/2-1}\N{\fvec}_2,
\eeqs
on either side of the inequality \cref{eq:mainevent2}, with these bounds coming from \cref{eq:normequiv1} and \cref{eq:normequiv2} respectively. The proof of the bound on 
$\|\Smat_A((\Amato)^\dagger)^{-1}\|_{2}$ in \cref{eq:keybound2a} follows in a similar way to above, using \cref{eq:Fbound}.
\epf


%\bre[Analogue of \cref{thm:1} in a weighted norm]\label{rem:weight1}
%The PDE analysis of the Helmholtz equation naturally takes place in the weighted $H^1$ norm $\|\cdot\|_{\HokD}$ defined by \cref{eq:1knorm}. The discrete analogue of this norm is the norm $\|\cdot\|_{\Dmat_k}$ defined by 
%\beq\label{eq:Dk}
%\N{\vvec}_{\Dmat_k}^2\de \big( (\Smat_I + k^2 \Mmat_1)\vvec,\vvec\big)_2 = \N{v_h}^2_{\HokD}
%\eeq
%for
%$v_h =\sum_i v_i \phi_i$. 
%This norm is used, e.g., in recent results about the convergence of domain-decomposition methods %in this norm are proved 
%for the Helmholtz equation \cite{GrSpVa:17}, \cite{GrSpZo:18}, and for the time-harmonic Maxwell equations \cite{BoDoGrSpTo:19}. 
%
%Inspecting the proof of \cref{lem:keylemma}, we see that the bounds \cref{eq:keybound1} and \cref{eq:keybound2} hold with the $\|\cdot\|_2$ norm replaced by the $\|\cdot\|_{\Dmat_k}$ norm and without the terms involving $m_\pm$ and $s_\pm$ on the right-hand side. \cref{thm:1} 
%%(and also \cref{cor:1}) 
%therefore also holds with the $\|\cdot\|_2$ norm replaced by the $\|\cdot\|_{\Dmat_k}$ norm and the constant $C_1$ modified appropriately.
%\ere

\subsection{Proofs of the finite-element results \cref{cor:1,cor:1a}}\label{sec:mainproofs}

We first recall properties of (weighted) GMRES that we will use to prove \cref{cor:1,cor:1a}.

Let 
\beq\label{eq:fov}
W_\Dmat(\matrixC)\de \Big\{ (\matrixC \xvec, \xvec)_{\Dmat} : \xvec \in \CCN, \|\xvec\|_\Dmat=1\Big\};
\eeq
$W_\Dmat(\matrixC)$ is called the \emph{numerical range} or \emph{field of values} of $\matrixC$ (in the $(\cdot,\cdot)_\Dmat$ inner product).

%Recall the so-called ``Elman estimate" for GMRES

\begin{theorem}[Elman estimate for weighted GMRES]\label{thm:GMRES1_intro} 
Let $\matrixC$ be a matrix with $\zerovec\notin W_\Dmat(\matrixC)$. Let $\beta\in[0,\pi/2)$ be defined such that
\beq\label{eq:cosbeta}
\cos \beta \de \frac{\mathrm{dist}\big(\zerovec, W_\Dmat(\matrixC)\big)}{\N{\matrixC}_{\Dmat}}.
\eeq
If the matrix equation $\matrixC \xvec = \by$ is solved using weighted GMRES then, 
for $m\in \mathbb{N}$, the GMRES residual $\rvecm$ %\de \matrixC \xvec_m - \by$ 
satisfies
\beq\label{eq:Elman}
\frac{\N{\rvecm}_{\Dmat}}{\N{\rvecz}_{\Dmat}} \leq \sin^m \beta. %, \quad \text{ where}\quad 
\eeq
\end{theorem}
The bound \cref{eq:Elman} with $\Dmat=\Imat$ was first proved in \cite[Theorem 6.3]{El:82} (see also \cite[Theorem 3.3]{EiElSc:83}) and was written in the above form in \cite[Equation 1.2]{BeGoTy:06}. The bound \cref{eq:Elman} (for arbitrary Hermitian positive-definite $\Dmat$) was stated implicitly (without proof) in \cite[p. 247]{CaWi:92} and proved in \cite[Theorem 5.1]{GrSpVa:17}. % (see also \cite[Remark 5.2]{GrSpVa:17}). 



\cref{thm:GMRES1_intro} has the following \lcnamecref{cor:GMRES_intro}, and the proofs of \cref{cor:1,cor:1a} follow from combining this with \cref{thm:1}.

\begin{corollary}
\label{cor:GMRES_intro} 
If $\|\Imat - \matrixC \|_\Dmat \leq \alpha < 1$, then, with $\beta$ defined as in \cref{eq:cosbeta},
\beqs
\cos \beta \geq \frac{1-\alpha}{1+\alpha}\eeqs
and
\beq\label{eq:gmressin}
\sin \beta \leq \frac{2 \sqrt{\alpha}}{(1+\alpha)^2}.
\eeq
\end{corollary}

\bpf[Proof of \cref{cor:1}]
\label{page:cor1proof}
This follows from \cref{thm:1} by applying \cref{cor:GMRES_intro} first with $\matrixC= (\Amato)^{-1} \Amatt$, $\Dmat=\Dmat_k$, and $\alpha=1/2$, and then with $\matrixC= \Amatt(\Amato)^{-1} $, $\Dmat=(\Dmat_k)^{-1}$, and $\alpha=1/2$.
\epf

\

\bpf[Proof of \cref{cor:1a}]
\label{page:cor1aproof}
This follows from \cref{thm:1} by applying \cref{cor:GMRES_intro} first with $\matrixC= (\Amato)^{-1} \Amatt$, $\Dmat=\Imat$, and $\alpha=1/2$, and then with $\matrixC= \Amatt(\Amato)^{-1} $, $\Dmat=\Imat$, and $\alpha=1/2$.
\epf


\bre[The improvement of the Elman estimate \cref{eq:Elman} in \cite{BeGoTy:06}]
A stronger result than \cref{eq:Elman} is given for standard (unweighted) GMRES in \cite[Theorem 2.1]{BeGoTy:06}, and then converted to a result about weighted GMRES in \cite[Theorem 5.3]{BoDoGrSpTo:19}; indeed, the convergence factor $\sin \beta$ is replaced by a function of $\beta$ strictly less than $\sin\beta$ for $\beta\in (0,\pi/2)$. Using this stronger result, however, does not improve the $k$-dependence of \cref{cor:1}.
\ere


%\section{Proof of }\label{sec:proofPDE}

\subsection{Proofs of the PDE results \cref{thm:2,lem:sharp}}\label{sec:pdeproofs}

\bpf[Proof of \cref{thm:2}]
\label{page:thm2proof}
%We first prove the upper bound \cref{eq:PDEbound}.
Because we assumed \cref{cond:1nbpc} holds for the EDP (\cref{prob:vedp}), $u^{(1)}$ and $u^{(2)}$ exist, are unique, satisfy $a^{(1)}(u^{(1)}, v) = \LG(v)$  and $a^{(2)}(u^{(2)}, v) = \LG(v)$ for all $v \in \HozDD$, respectively, where $\LG$ is given by \cref{eq:Ledp}. Subtracting these equations, we find that $u^{(1)}- u^{(2)}$ satisfies the variational problem
\beq\label{eq:vp1}
a^{(1)}(u^{(1)}-u^{(2)},v) = \LGtilde(v) \quad\tfa v\in H^1_{0,D}(D)
\eeq
where
\beqs
 \LGtilde(v)\de \int_{D} \left((\Ast-\Aso) \nabla u^{(2)}\right) \cdot\overline{\nabla v} + k^2 (\nso-\nst) u^{(2)}\overline{v}.
\eeqs
Now, by the Cauchy-Schwarz inequality and the definition of the norm $\|\cdot\|_{\HokD}$ (see \cref{eq:weightednorm}), we have
\begin{align*}
| \LGtilde(v)| &\leq \NLiDop{\Aso-\Ast} \big\|\nabla u^{(2)}\big\|_{L^2(D)}
\N{\nabla v}_{L^2(D)} 
\\& \hspace{5cm}+ k^2 
\NLiDRR{\nso-\nst} \big\| u^{(2)}\big\|_{L^2(D)}
\N{v}_{L^2(D)}\\
&\leq\max\Big\{\NLiDop{\Aso-\Ast}\,,\, \NLiDRR{\nso-\nst}\Big\}
\big\| u^{(2)}\big\|_{\HokD} \N{v}_{\HokD}
\end{align*}
(by Cauchy--Schwarz in $\RR^2$). Therefore, by the definition of the norm $\|\cdot\|_{(\HokD)^*}$
\beqs
\big\|\LGtilde\big\|_{(\HokD)^*}\leq \max\set{\NLiDop{\Aso-\Ast},\NLiDRR{\nso-\nst}}.
\big\| u^{(2)}\big\|_{\HokD}.
\eeqs
Since \cref{cond:1nbpc} holds, we can then apply \cref{lem:H1}, i.e.~the bound \cref{eq:bound2}, to the solution of the variational problem \cref{eq:vp1}  to find that 
\begin{align*}
\frac{\big\| u^{(1)} - u^{(2)}\big\|_{\HokD}}
{\big\| u^{(2)}\big\|_{\HokD}, 
}
 \leq 
\,&\frac{1}{\min\big\{\Asomin,\nsomin\big\}}\left( 1 + 2 C^{(1)}_{\rm bound}\nsomax  k\right)
\\
&\quad \mleft(\max\set{\NLiDop{\Aso-\Ast},\NLiDRR{\nso-\nst}}\mright),
\end{align*}
and then the result \cref{eq:PDEbound} follows with 
\beq\label{eq:C3}
C_3\de \frac{1}{\min\big\{\Asomin,\nsomin\big\}}\left( \frac{1}{k_0} + 2 C^{(1)}_{\rm bound}\nsomax  \right).
\eeq
\epf

\bpf[Proof of \cref{lem:sharp}]
\label{page:lemsharpproof}
We actually prove the stronger result that given any function $c(k)$ such that $c(k)>0$ for all $k>0$, there exist 
$f, \nso,$ and $ \nst$ (with $\nso\not= \nst$) with
\beq\label{eq:nck}
\NLiDRR{\nso-\nst} \sim c(k)
\eeq
such that the corresponding solutions $u^{(1)}$ and $u^{(2)}$ of \cref{prob:vedp} with $\Aso = \Ast= I$ exist, are unique, and satisfy \cref{eq:sharp1}. 

The heart of the proof is the equation
\beq\label{eq:obs1}
(\Delta + k^2) \big(e^{i k r}\chi(r)\big) =  e^{i k r}\mleft(\Delta \chi(r) + 2ik\frac{\partial \chi}{\partial r}(r) + i k \frac{d-1}{r} \chi(r)\mright)=: -\widetilde{f}(r),
\eeq
where $\chi(r)$ is chosen to have $\supp \chi \subset D$. Observe that \cref{eq:obs1} is the Helmholtz operator applied to a circular wave $e^{ikr}$, with the added factor $\chi$ which can be chosen to have compact support. The equation \cref{eq:obs1} can be proved using the formula for the Laplacian in $d$-dimensional spherical coordinates
\beq\label{eq:sphericallaplacian}
\Delta \chi = \frac{1}{r^{d-1}} \frac{\partial}{\partial r}\mleft(r^{d-1} \frac{\partial \chi}{\partial r} \mright) + \frac1{r^2} \LapBel \chi,
\eeq
where $\LapBel$ is the Laplace--Beltrami operator on the $d-1$-dimensional sphere (see, e.g., \cite[Equations (17.23) and (17.25)]{RiHoBe:97} for \cref{eq:sphericallaplacian} in $d=2$ and $3.$). Observe that $e^{ikr} \chi(r)$ has $\LapBel e^{ikr} \chi(r) = 0.$

We expect that \cref{eq:obs1} will be key in the proof of the sharpness of \cref{eq:PDEbound}, for the following reasons. Observe that \cref{eq:obs1} proves the sharpness of the nontrapping resolvent estimate \cref{eq:bound1}, since $\NLtD{\ftilde}\sim k$ and $\NHokD{e^{ikr}\chi(r)}\sim k$  and hence $\NHokD{e^{ikr}\chi(r)} \sim \NLtD{\ftilde}$ (see, e.g., \cite[Lemma 3.10]{ChMo:08},  \cite[Lemma 4.12]{Sp:14}).

Also, recall that  the nontrapping resolvent estimate \cref{eq:bound1} was used in the proof of the PDE bound \cref{eq:PDEbound} applied to $\uso-\ust.$ Therefore we expect that if we set things up so that
\beq\label{eq:sharpdiff}
\uso-\ust = e^{ikr}\chi(r),
\eeq
then  combining \cref{eq:obs1} and \cref{eq:sharpdiff} will show the sharpness of the PDE bound \cref{eq:PDEbound}. Moreover, the function $e^{ikr} \chi(r)$ was used to prove the sharpness of resolvent estimates in \cite[Discussion on p. 1445 and Lemma 3.10]{ChMo:08} and \cite[Lemma 4.12]{Sp:14}, and so we can expect it will also be effective for proving sharpness in our setting.

We now set things up so that \cref{eq:sharpdiff} holds. We define $\nso = 1$ and
\beq\label{eq:fiddlyntdone}
\nst = \nso + c(k) \chitilde(r),
\eeq
for some function $\chitilde(r)$ such that $\widetilde{\chi}\in C^{\infty}(D)$, $\widetilde{\chi}\not = 1$ (so that $\nst\not = \nso$), $\supp \, \widetilde{\chi} \compcont D$, and $\NLiDRR{\chitilde} = 1$ (so that $\NLiDRR{\nso-\nst} = c(k)$).   As above, let $\chi=\chi(r)$ with $\chi \in C^{\infty}(D)$ and $\supp \,\chi \compcont D$. We will specify $\chitilde$ and $\chi$ in more detail later.

Let $\ftilde(r)$ be as defined in \cref{eq:obs1}, and define
\beq\label{eq:obs3}
u^{(2)}(\bx)\de -\frac{1}{k^2 c(k)}\frac{\widetilde{f}(r)}{\widetilde{\chi}(r)}
\eeq
and
\beq\label{eq:fiddlyf}
f(\bx)\de -\big(\Delta +k^2 \nst(\bx)\big) u^{(2)}(\bx).
\eeq
I.e., $\ust$ solves \cref{prob:vedp} with coefficients $\Ast = I$ and $\nst$ given by \cref{eq:fiddlyntdone}, and right-hand side $f.$ We will define $\chi$ and $\chitilde$ below in such a way that $\ust \in \HoD$ and $f \in \LtD.$ In particular, we choose $\chi$ and $\chitilde$ so that if $\chitilde=0,$ then $\chi = 0$. Since $\ftilde$ depends on $\chi,$ this relation means we understand the right-hand side of \cref{eq:obs3} to be zero if $\chitilde$ is zero. In addition, since $\widetilde{\chi}(r)$ has compact support and $\ftilde$ depends on $\chi,$ we need to tie both the support of $\widetilde{\chi}$ and how fast $\widetilde{\chi}$ vanishes in a neighbourhood of its support to the definition of $\chi$ for both $u^{(2)}$ and $f$ to be well defined. As the final part of the setup, let $\uso$ solve
\beqs
\mleft(\Delta + k^2\mright) \uso = -f.
\eeqs
I.e., $\uso$ solves \cref{prob:vedp} with coefficients $\Aso = I$ and $\nso =1$ and right-hand side $f$.

Now observe that by construction (since $\nst$ is given by \cref{eq:fiddlyntdone})

\begin{align*}
  \mleft(\Delta + k^2\mright) \mleft(\uso-\ust\mright) &= \mleft(\Delta + k^2\mright)\uso - \mleft(\Delta + k^2 \nst - k^2\mleft(\nst - 1\mright)\mright)\ust\\
  &= -f + f + k^2\mleft(\nst-1\mright)\ust\\
   &= k^2 \mleft(\nst-1\mright)\ust\\
  &= k^2 c(k) \chitilde \frac{-1}{k^2 c(k)} \frac{\ftilde}{\chitilde}\\
  &= -\ftilde\\
  &= \mleft(\Delta + k^2 \mright)\mleft(e^{ikr}\chi(r)\mright).
\end{align*}
Therefore, by uniqueness of the solution of \cref{prob:vedp} (with constant coefficients)
 \beq\label{eq:obs4}
u^{(1)}(\bx)- u^{(2)}(\bx) = e^{i k r}\chi(r).
\eeq
Therefore, by \cref{eq:obs4} and the properties of $e^{ikr} \chi(r)$ discussed above,  we have
\beq\label{eq:pdenumber1}
\big\|u^{(1)}-u^{(2)}\big\|_{L^2(D)} \sim 1
\quad \tand \quad
\big\|u^{(1)}-u^{(2)}\big\|_{\HokD} \sim k.
\eeq
Furthermore, the definitions of $u^{(2)}$ and $\widetilde{f}$ imply that
\beq\label{eq:pdenumber2}
\big\| u^{(2)}\big\|_{L^2(D)} \sim \frac{1}{k\, c(k)} \quad\tand \quad 
\big\| u^{(2)}\big\|_{\HokD} \sim \frac{1}{c(k)},
\eeq
and, since $\|\nso- \nst\|_{L^\infty(D)} = c(k)$, by combining \cref{eq:pdenumber1,eq:pdenumber2}, we see \cref{eq:nck} holds, as required.

Therefore, to complete the proof, we only need to show that there exists a choice of $\chi$ and $\widetilde{\chi}$ for which $u^{(2)}$ and $f$ defined by \cref{eq:obs3,eq:fiddlyf} are 
in $H^{1}(D)$ and $\LtD$ respectively (in fact, we prove that they are in $W^{1,\infty}(D)$ and $L^\infty(D)$ respectively).
%well-defined. 
Because $\chi$ and $\widetilde{\chi}$ are in $C^\infty(D)$ and we choose $\chi$ and $\chitilde$ so that if $\chitilde=0,$ then $\chi=0$, the only issue is what happens at the boundary of the support of $\widetilde{\chi}$, where $u^{(2)}$ has the potential to be singular.
Since $\clos{\Dm} \subset \BR$, there exist $0<R_1<R_2<R$ such that $\overline{\Dm} \subset B_{R_2}\setminus B_{R_1} \subset \BR$. Let $\supp \chi = B_{R_2}\setminus B_{R_1}$ and let $\chi$ vanish to order $m$ at $r= R_1$ and $r=R_2$; i.e.~$\chi(r) \sim (r-R_1)^m$ as $r \downarrow R_1$ and 
$\chi(r) \sim (R_2-r)^m$ as $r \uparrow R_2$. The definition of $\widetilde{f}$ \cref{eq:obs1} then implies that $\widetilde{f}$ vanishes to order $m-2$. Let $\widetilde{\chi}(r)$ vanish to order $\mtilde$ at $r= R_1$ and $r=R_2$. 
We now claim that if $m >\mtilde+4$, then $u^{(2)}\in W^{1,\infty}(D)$ and $f$ $\in L^\infty(D)$. Indeed,  
straightforward calculation from \cref{eq:obs3} shows that  $u^{(2)}(r) \sim (r-R_1)^{m-\mtilde-2}$, $\nabla u^{(2)}(r) \sim (r-R_1)^{m-\mtilde-3}$, and $\Delta u^{(2)}(r) \sim (r-R_1)^{m-\mtilde-4}$ as $r \downarrow R_1$, with analogous behaviour at $r=R_2$.
The assumption 
$m >\mtilde+4$ therefore implies that $u^{(2)}$, $\nabla u ^{(2)}$, and $\Delta u^{(2)}$ vanish (and hence are finite) at $r=R_1$ and $r=R_2$.
\epf

\bre[Why doesn't \cref{lem:sharp} cover the case $\Aso\neq  \Ast$?]
When $\nj\de1$, $j=1,2,$ $\Aso\de I$, and $\Ast\de I + c(k)\widetilde{\chi}$, the variational problem \cref{eq:vp1} implies that 
\beq\label{eq:obs2}
\Delta \big( u^{(1)} - u^{(2)}\big) + k^2 \big( u^{(1)} - u^{(2)}\big) = c(k)\nabla\cdot \big(\widetilde{\chi}\nabla u^{(2)}\big).
\eeq
It is now much harder than in \cref{eq:obs2} to set things up so that $ u^{(1)}(\bx) - u^{(2)}(\bx)=e^{i kr}\chi(r)$ (so that one can then use \cref{eq:obs1}).
\ere

%\section{Proof of \cref{lem:2}}

\section[Extension to weaker norms]{Extension of the nearby preconditioning results to weaker norms}\label{sec:weaknorm}
Recall from \cref{sec:num,sec:main} that GMRES applied to $\AmatoI \Amatt$ converges in a $k$-indepen\-dent number of iterations if $k\NLiDRR{\nso-\nst}$ is sufficiently small (with an analagous result for $\Aso-\Ast$). This result (and the related numerics) shows that $1/k$ may be a sharp threshold when we consider the maximum norm of the difference between $\nso$ and $\nst$. However, this result does not say anything if $\nso-\nst$ is merely small in some integral norm. For example if $\nso$ and $\nst$ (defined on the unit square) are given by
\beq\label{eq:noweak}
\nso(\bx) =
\begin{dcases}
  1 &\tif \bx_1 \leq \half\\
  2  &\tif \bx_1 > \half
  \end{dcases}
\eeq
and
\beq\label{eq:ntweak}
\nst(\bx) =
\begin{dcases}
  1 &\tif \bx_1 \leq \half+\alpha\\
  2  &\tif \bx_1 > \half+\alpha
  \end{dcases}
\eeq
for some $0 < \alpha < 1/2,$ then $\NLiDRR{\nso-\nst} = 1$ for all $\alpha$, but one would expect that for small $\alpha$ the corresponding solutions of \cref{prob:edp} would satisfy $\uso \approx \ust.$ In addition, one might expect that GMRES applied to $\AmatoI\Amatt$ would converge in a $k$-independent number of iterations. Therefore, in this \lcnamecref{sec:weaknorm} we seek to obtain analogues of \cref{thm:1,cor:1,cor:1a} with the difference in $\nso-\nst$ and $\Aso-\Ast$ measured in weaker norms than the $L^\infty$ norm.

The (realistic) best-case result we could obtain would be that GMRES applied to $\AmatoI\Amatt$ converges in a $k$-independent number of iterations if $\NLoDRR{\nso-\nst} \lesssim 1/k$. This result is `best' in the sense that it depends optimally on $k$; recall the discussion in \cref{rem:physical1k} that $1/k$ is the length scale governing the behaviour of Helmholtz problems. In addition, we measure $\nso-\nst$ in the $L^\infty$ norm as above, we are able to control the magnitude of $\nso-\nst$, but not the spatial variability; if $\nso-\nst \neq 0$ only on a set of small (but nonzero) measure, and $\nso-\nst=1$ on this small set, then $\NLiDRR{\nso-\nst} = 1$, regardless of the measure of the set. In contrast, the $L^1$ norm allows us to control both the magnitude of $\nso-\nst$ and the measure of the sets on which it is nonzero.

We will give numerical results indicating that this theoretical best-case result can be achieved (our numerical results actually indicate that we can obtain $k$-independent convergence when $\NLqDRR{\nso-\nst}\sim 1/k$ for any $1 \leq q < \infty$). We will also provide theory results that are, to our knowledge, the best one can prove, although they are sub-optimal in both $q$ and the dependence on $k.$


\subsection{Theory in weaker norms}\label{sec:weakertheory}
Before we prove results analogous to \cref{cor:1,cor:1a} in weaker norms (using a result analogous to \cref{thm:1} in weaker norms), we first recap why the terms  $\NLiDop{\Aso-\Ast}$ and $\NLiDRR{\nso-\nst}$ appear in \cref{thm:1}. These terms appear in \cref{thm:1} because the terms $\NLiDRR{n}$ and $\NLiDop{A}$ appear in \cref{lem:keylemma1,lem:keylemma2}, respectively. These terms appear in these \lcnamecrefs{lem:keylemma1} because in \cref{eq:mainevent1a,eq:Fbounda} we use the bounds
\beq\label{eq:keynbound}
\NLtD{n\ftilde} \leq \NLiDRR{n}\NLtD{\ftilde}
\eeq
and
\beq\label{eq:keyAbound}
\NLtD{A \grad \ftilde} \leq \NLiDop{A}\NLtD{\grad \ftilde}
\eeq
respectively, for an arbitrary function $\ftilde \in \Vhp,$ and these bounds are carried through the rest of the proof.

However, we observe that we have the following generalisation of H\"older's inequality: If $q,s > 2$ such that $1/2 = 1/q+1/s,$ then
\beq\label{eq:genholder}
\NLtD{\vo\vt} \leq \NLqD{\vo}\NLsD{\vt}.
\eeq

If we instead use \cref{eq:genholder} to bound \cref{eq:keynbound,eq:keyAbound} we obtain
\beq\label{eq:keynbound2}
\NLtD{n\ftilde} \leq \NLqDRR{n}\NLsD{\ftilde}
\eeq
and
\beq\label{eq:keyAbound2}
\NLtD{A\grad\ftilde} \leq \NLqDop{A}\NLsD{\grad\ftilde}.
\eeq

As $\ftilde \in \Vhp$, we can apply an inverse inequality to bound $\NLsD{\ftilde}$ by $\NLtD{\ftilde}$. The required inverse inequality is (see \cite[Theorem 4.5.11 and Remark 4.5.20]{BrSc:08}
\beq\label{eq:inverses}
\NLsD{\ftilde} \leq \Cinvs h^{d\mleft(\frac1{s} - \half\mright)} \NLtD{\ftilde}.
\eeq
If we then apply \cref{eq:inverses} to \cref{eq:keynbound2,eq:keyAbound2} we obtain
\beq\label{eq:keynboundfinal}
\NLtD{n\ftilde} \leq \Cinvs \NLqDRR{n} h^{d\mleft(\frac1{s} - \half\mright)} \NLtD{\ftilde} = \Cinvs \NLqDRR{n} h^{-\frac{d}q} \NLtD{\ftilde}
\eeq
and
\beq\label{eq:keyAboundfinal}
\NLtD{A\grad\ftilde} \leq \Cinvs \NLqDop{A} h^{d\mleft(\frac1{s} - \half\mright)} \NLtD{\grad\ftilde} = \Cinvs \NLqDop{A} h^{-\frac{d}q} \NLtD{\grad\ftilde}.
\eeq

Replacing \cref{eq:mainevent1a,eq:Fbounda} with \cref{eq:keynboundfinal,eq:keyAboundfinal} in the proofs of \cref{lem:keylemma1,lem:keylemma2}, and proceeding as in those proofs, we can obtain the following \lcnamecrefs{cor:1alt}, the analogues of \cref{cor:1,cor:1a}.

\bth[Alternative answer to \cref{it:nbpcq1}: $k$-independent weighted GMRES iterations]\label{cor:1alt}

\

\noindent Let the assumptions of \cref{cor:1} hold.  Given $q >2$, there exist $\Cotilde, \Cttilde>0$, independent of $h$ and $k$ (but dependent on $d, \Dm, \Aso, \nso$, $p$, $q$, and $\kz$) such that if 
% there exists $C_2>0$, independent of $h$ and $k$ (but dependent on $\Dm, \Aso, \nso$, $p$, and $k_0$) and given explicitly in \cref{eq:C2} below,
% such that if 
\beq\label{eq:condalt}
\Cotilde kh^{-\frac{d}{q}} \NLqDop{\Aso-\Ast} +\Cttilde  kh^{-\frac{d}{q}} \NLqDRR{\nso-\nst},
\leq \frac{1}{2}
\eeq
then \emph{both} weighted GMRES working in $\|\cdot\|_{\Dmat_k}$ (and the associated inner product) applied to \cref{eq:pcsystem1} \emph{and} weighted GMRES working in $\|\cdot\|_{(\Dmat_k)^{-1}}$ (and the associated inner product) applied to \cref{eq:pcsystem2}  converge in a $k$-independent number of iterations.
\enth

\bth[Alternative answer to \cref{it:nbpcq1}: $k$-independent (unweighted) GMRES iterations]\label{cor:1aalt}

\

\noindent Let the assumptions of \cref{cor:1a} hold.  Given $q >2$, there exist $\Cotilde, \Cttilde>0$, independent of $h$ and $k$ (but dependent on $d, \Dm, \Aso, \nso$, $p$, $q$, and $\kz$) such that if
% there exists $C_2>0$, independent of $h$ and $k$ (but dependent on $\Dm, \Aso, \nso$, $p$, and $k_0$) and given explicitly in \cref{eq:C2} below,
% such that if 
\beq\label{eq:condaalt}
\Cotilde \mleft(\frac{\splus}{\mminus}\mright) h^{-\frac{d}{q}-1} \NLqDop{\Aso-\Ast} + \Cttilde \mleft(\frac{\mplus}{\mminus}\mright) kh^{-\frac{d}{q}} \NLqDRR{\nso-\nst} \leq \half,
\eeq
then standard GMRES (working in the Euclidean norm and inner product) applied to either of the equations \cref{eq:pcsystem1} or \cref{eq:pcsystem2}
%\beqs
%(\Amat^{(1)})^{-1}\Amat^{(2)}\uvec = \fvec\quad\text{ or } \quad\Amat^{(2)}(\Amat^{(1)})^{-1}\vvec = \fvec
%\eeqs
 converges in a $k$-independent number of iterations.
 \enth

 A sketch proof of \cref{cor:1alt,cor:1aalt} is on \cpageref{page:cor1altcor1aaltproof} below.

\bre[Trade off between the type of norm and powers of $h$ and $k$]
Observe that in \cref{cor:1alt,cor:1aalt} there is a trade-off between the norm that one uses to measure $\nso-\nst$ (or $\Aso-\Ast$) and the restriction on the magnitude of this norm. E.g., the condition on $\nso-\nst$ in both \cref{cor:1alt,cor:1aalt} can be summarised as
\beq\label{eq:altsufficientlysmall}
\NLqDRR{\nso-\nst} k h^{-\frac{d}{q}} \text{ is sufficiently small}.
\eeq
with analogous conditions on $\Aso-\Ast.$ Observe that as $q \downarrow 2,$ we measure $\nso-\nst$ in a weaker norm, but the condition \cref{eq:altsufficientlysmall} becomes more restrictive; the power of $h$ increases. Conversely, as $q \uparrow \infty,$ we measure $\nso-\nst$ in a stronger norm, but the condition \cref{eq:altsufficientlysmall} becomes less restrictive; the power of $h$ decreases. (Also observe that in the $q\uparrow\infty$ limit we recover the condition \cref{eq:sufficientlysmall} we previously proved for $\NLiDRR{\nso-\nst}.$
\ere

\bre[\Cref{cor:1,cor:1a} are a special case of \cref{cor:1alt,cor:1aalt}]
Observe that in the case $q=\infty$ \cref{cor:1alt,cor:1aalt} become our previous results in the $L^\infty$ norm, \cref{cor:1,cor:1a}.
\ere

The numerical experiments in \cref{sec:weakernumerics} below suggest that, at least in certain cases, a sufficient condition for nearby preconditioning to be effective is
\beq\label{eq:experimentalsufficientlysmall}
\NLqDRR{\no-\nt} k \quad\text{is sufficiently small},
\eeq
for \emph{any} $q \geq 1$, and moreover \cref{eq:experimentalsufficientlysmall} appears sharp in its $k$-dependence. (This requirement would fit with our previous observation about $1/k$ being the length scale below which perturbations cannot be seen---see \cref{rem:physical1k} above.) However, we do not say that \cref{eq:experimentalsufficientlysmall} is sufficient for all cases; recall that for transmission problems, very small perturbations in $n$ can lead to very different behaviour in the solution $u$ if $k$ is a quasi-resonance for $\no$ or $\nt$; see the discussion at the end of \cref{sec:wpdisc} above.


\subsubsection{Proof of \cref{cor:1alt,cor:1aalt}}

We first state analogues of \cref{lem:keylemma1,lem:keylemma2} in weaker norms; these \lcnamecrefs{lem:keylemma1} are the key to the proofs of \cref{cor:1alt,cor:1aalt} above. The essence of the proofs of \cref{lem:keylemma1a,lem:keylemma2a} are the discussion at the start of \cref{sec:weakertheory}.

\ble[Alternative bounds on $(\Amato)^{-1} \Mmat_{n}$]\label{lem:keylemma1a}
Under the assumptions of \cref{lem:keylemma1}, for $n\in \LiDRR$ and for any $q > 2$,
\beq\label{eq:keybound12}
\max\set{\NDk{\AmatoI \Mmatn},\NDkI{\Mmatn\AmatoI}} \leq \Cttilde h^{-\frac{d}{q}} \frac{\NLqDRR{n}}k
\eeq
and 
\beq\label{eq:keybound1a2}
\max\set{\Nt{\AmatoI \Mmatn},\Nt{\Mmatn\AmatoI}} \leq \Cttilde\mleft(\frac{\mplus}{\mminus}\mright) h^{-\frac{d}q} \frac{\NLqDRR{n}}k,
\eeq
where
\beq\label{eq:C2tilde}
\Cttilde\de%\frac{m_+}{m_-} 
%\left[ 
\Cinvs\Ct,
\eeq
where $\Ct$ is defined by \cref{eq:C2} and $1/s = 1/2 - 1/q.$
\ele

\ble[Alternative bounds on $(\Amato)^{-1} \Smat_A$]\label{lem:keylemma2a}
Under the assumptions of \cref{lem:keylemma2}, for $A\in L^\infty(D,\RR^{d\times d})$ and for any $q > 2$
\beq\label{eq:keybound22}
\max\set{\NDk{\AmatoI \SmatA},\NDkI{\SmatA\AmatoI}} \leq \Cotilde h^{-\frac{d}q}k \NLqDop{A}
\eeq
and
\beq\label{eq:keybound2a2}
\max\set{\Nt{\AmatoI \SmatA},\Nt{\SmatA\AmatoI}} \leq \Cotilde\mleft(\frac{\splus}{\mminus}\mright) h^{-\frac{d}q-1} \NLqDRR{A},
\eeq
%\begin{align}\nonumber
%&\max\Big\{\big\| (\Amato)^{-1} \Smat_A \big\|_2, \,\,
%\big\| \Smat_A (\Amato)^{-1} \big\|_2\Big\}\nonumber \\
%&\hspace{2cm}
% \leq \frac{s_+}{s_-} \left[ C_{\rm FEM2}^{(1)} + 
% \frac{1}{\min\big\{\Asomin,\nsomin\big\}}\left( \frac{1}{k_0} + 2 C^{(1)}_{\rm bound}\nsomax  \right) \right]k\N{A}_{L^\infty(D)}\label{eq:keybound2}
%% + C_{\rm bound}^{(1)}\right) \frac{\N{n}_{L^\infty(D)}}{k}.
%\end{align}
where
\beq\label{eq:C1tildenbpc}
\Cotilde \de \Cinvs\Co,
\eeq
where $\Co$ is given by \cref{eq:C1nbpc} and $1/s = 1/2 - 1/q.$
\ele

The proofs of \cref{lem:keylemma1a,lem:keylemma2a} are virtually identical to the proofs of \cref{lem:keylemma1,lem:keylemma2}, with the modifications for $L^q$ norms detailed at the beginning of \cref{sec:weakertheory}.

\bre[Reduction to \cref{lem:keylemma1,lem:keylemma2}]
Observe that in the case $s=2$ and $q=\infty$ \cref{lem:keylemma1a,lem:keylemma2a} reduce to our previous results \cref{lem:keylemma1,lem:keylemma2}.
\ere

We can use \cref{lem:keylemma1a,lem:keylemma2a} in place of \cref{lem:keylemma1,lem:keylemma2} to obtain the following analogue of \cref{thm:1} in weaker norms.

\begin{theorem}[Alternative main ingredient to answer to \cref{it:nbpcq1}]\label{thm:1alt}
If all the assumptions of \namecref{thm:1} \ref{thm:1} hold, then there exist $\Cotilde, \Cttilde>0$, independent of $h$ and $k$ (but dependent on $d, \Dm, \Aso, \nso$, $p$, $q$, and $\kz$) such that
\begin{align}\nonumber
&\max\set{\NDk{\Imat - \AmatoI\Amatt},\NDkI{\Imat -\Amatt\AmatoI}}\\
&\hspace{3cm} 
\leq \Cotilde kh^{-\frac{d}q} \NLqDop{\Aso-\Ast} + \Cttilde  kh^{-\frac{d}q}  \NLqDRR{\nso-\nst}
\label{eq:main1alt}
\end{align}
and 
\begin{align}\nonumber
&\max\set{\Nt{\Imat - \AmatoI\Amatt}, \Nt{\Imat -\Amatt\AmatoI}}\\
&\hspace{0cm}
\leq \Cotilde \mleft(\frac{\splus}{\mminus}\mright) h^{-\frac{d}q-1}\NLqDop{\Aso-\Ast} + \Cttilde \mleft(\frac{\mplus}{\mminus}\mright) kh^{-\frac{d}q}\NLqDRR{\nso-\nst}.
\label{eq:main1aalt}
\end{align}
\end{theorem}

The proof of \cref{thm:1alt} is identical to the proof of \cref{thm:1}, with \cref{lem:keylemma1,lem:keylemma2} replaced by \cref{lem:keylemma1a,lem:keylemma2a}.

%We can now use \cref{thm:1alt} to obtain the following analogues to \cref{cor:1,cor:1a} in weaker norms.

\bpf[Sketch proof of \cref{cor:1alt,cor:1aalt}]
\label{page:cor1altcor1aaltproof}
The proofs of \cref{cor:1alt,cor:1aalt} are completely analagous to the proofs of \cref{cor:1,cor:1a}, with the exception that we use \cref{thm:1alt} in place of \cref{thm:1}.
\epf


\subsection{Numerics in weaker norms}\label{sec:weakernumerics}
For our computations, we use the computational setup as in \cref{app:compsetup}, with $f$ and $\gI$ corresponding to a plane wave passing through homogeneous media.  We let $\Aso=\Ast=I,$ and we define $\nso$ and $\nst$ by \cref{eq:noweak,eq:ntweak}. For $\alpha = 0.2k^{-\beta},$ $\beta = 0,0.1,\ldots,0.9,1$ and for $k=10,20,\ldots,100$ we used GMRES to solve $\AmatoI\Amatt = \AmatoI \fvec$ (for $\fvec$ given by the Helmholtz problem), and we record the number of GMRES iterations taken to achieve convergence.

Our results in \cref{fig:l1low,fig:l1med,fig:l1high} (also displayed in \cref{tab:l1}) indicate the following conclusions for $\NLqDRR{\nso-\nst} \sim 0.1/k^{-\beta}$, for all $1 \leq q < \infty$:
\bit
\item For $\beta \in (0,0.6)$ there is clear growth of the number of GMRES iterations with $k$,
\item For $\beta = 1$ there is clear boundedness of the number of GMRES iterations with $k$, and
  \item for $\beta \in (0.7,0.9)$ it is unclear if the number of GMRES iterations grows with $k.$
    \eit
We note that the results in \cref{fig:l1low,fig:l1med,fig:l1high} are the analogues of those in \cref{fig:linfinityn0,fig:linfinityn1,fig:linfinityn2}.

If we compare our numerical results with the theory results in \cref{cor:1aalt}, we see that the theory (if $h \sim k^{-3/2}$ and $d=2$, as in our computational experiments) predicts that the number of iterations will remain bounded if $\NLqDRR{\nso-\nst} k^{1+3/q}$ is sufficiently small, for any $q > 2.$ Our computed results indicate that this result is not sharp. The computed results indicate that if $\NLqDRR{\nso-\nst} \sim k^{-1}$ for any $q \geq 1,$ then the number of GMRES iterations is bounded as $k$ increases. Observe again that the `best case' $1/k$ condition is only predicted by the theory in the $q\rightarrow \infty$ limit.

\begin{figure}
%% Creator: Matplotlib, PGF backend
%%
%% To include the figure in your LaTeX document, write
%%   \input{<filename>.pgf}
%%
%% Make sure the required packages are loaded in your preamble
%%   \usepackage{pgf}
%%
%% Figures using additional raster images can only be included by \input if
%% they are in the same directory as the main LaTeX file. For loading figures
%% from other directories you can use the `import` package
%%   \usepackage{import}
%% and then include the figures with
%%   \import{<path to file>}{<filename>.pgf}
%%
%% Matplotlib used the following preamble
%%   \usepackage{fontspec}
%%   \setmainfont{DejaVuSerif.ttf}[Path=/home/owen/progs/firedrake-complex/firedrake/lib/python3.5/site-packages/matplotlib/mpl-data/fonts/ttf/]
%%   \setsansfont{DejaVuSans.ttf}[Path=/home/owen/progs/firedrake-complex/firedrake/lib/python3.5/site-packages/matplotlib/mpl-data/fonts/ttf/]
%%   \setmonofont{DejaVuSansMono.ttf}[Path=/home/owen/progs/firedrake-complex/firedrake/lib/python3.5/site-packages/matplotlib/mpl-data/fonts/ttf/]
%%
\begingroup%
\makeatletter%
\begin{pgfpicture}%
\pgfpathrectangle{\pgfpointorigin}{\pgfqpoint{6.400000in}{4.800000in}}%
\pgfusepath{use as bounding box, clip}%
\begin{pgfscope}%
\pgfsetbuttcap%
\pgfsetmiterjoin%
\definecolor{currentfill}{rgb}{1.000000,1.000000,1.000000}%
\pgfsetfillcolor{currentfill}%
\pgfsetlinewidth{0.000000pt}%
\definecolor{currentstroke}{rgb}{1.000000,1.000000,1.000000}%
\pgfsetstrokecolor{currentstroke}%
\pgfsetdash{}{0pt}%
\pgfpathmoveto{\pgfqpoint{0.000000in}{0.000000in}}%
\pgfpathlineto{\pgfqpoint{6.400000in}{0.000000in}}%
\pgfpathlineto{\pgfqpoint{6.400000in}{4.800000in}}%
\pgfpathlineto{\pgfqpoint{0.000000in}{4.800000in}}%
\pgfpathclose%
\pgfusepath{fill}%
\end{pgfscope}%
\begin{pgfscope}%
\pgfsetbuttcap%
\pgfsetmiterjoin%
\definecolor{currentfill}{rgb}{1.000000,1.000000,1.000000}%
\pgfsetfillcolor{currentfill}%
\pgfsetlinewidth{0.000000pt}%
\definecolor{currentstroke}{rgb}{0.000000,0.000000,0.000000}%
\pgfsetstrokecolor{currentstroke}%
\pgfsetstrokeopacity{0.000000}%
\pgfsetdash{}{0pt}%
\pgfpathmoveto{\pgfqpoint{0.800000in}{0.528000in}}%
\pgfpathlineto{\pgfqpoint{5.760000in}{0.528000in}}%
\pgfpathlineto{\pgfqpoint{5.760000in}{4.224000in}}%
\pgfpathlineto{\pgfqpoint{0.800000in}{4.224000in}}%
\pgfpathclose%
\pgfusepath{fill}%
\end{pgfscope}%
\begin{pgfscope}%
\pgfsetbuttcap%
\pgfsetroundjoin%
\definecolor{currentfill}{rgb}{0.000000,0.000000,0.000000}%
\pgfsetfillcolor{currentfill}%
\pgfsetlinewidth{0.803000pt}%
\definecolor{currentstroke}{rgb}{0.000000,0.000000,0.000000}%
\pgfsetstrokecolor{currentstroke}%
\pgfsetdash{}{0pt}%
\pgfsys@defobject{currentmarker}{\pgfqpoint{0.000000in}{-0.048611in}}{\pgfqpoint{0.000000in}{0.000000in}}{%
\pgfpathmoveto{\pgfqpoint{0.000000in}{0.000000in}}%
\pgfpathlineto{\pgfqpoint{0.000000in}{-0.048611in}}%
\pgfusepath{stroke,fill}%
}%
\begin{pgfscope}%
\pgfsys@transformshift{1.250909in}{0.528000in}%
\pgfsys@useobject{currentmarker}{}%
\end{pgfscope}%
\end{pgfscope}%
\begin{pgfscope}%
\definecolor{textcolor}{rgb}{0.000000,0.000000,0.000000}%
\pgfsetstrokecolor{textcolor}%
\pgfsetfillcolor{textcolor}%
\pgftext[x=1.250909in,y=0.430778in,,top]{\color{textcolor}\sffamily\fontsize{10.000000}{12.000000}\selectfont 10}%
\end{pgfscope}%
\begin{pgfscope}%
\pgfsetbuttcap%
\pgfsetroundjoin%
\definecolor{currentfill}{rgb}{0.000000,0.000000,0.000000}%
\pgfsetfillcolor{currentfill}%
\pgfsetlinewidth{0.803000pt}%
\definecolor{currentstroke}{rgb}{0.000000,0.000000,0.000000}%
\pgfsetstrokecolor{currentstroke}%
\pgfsetdash{}{0pt}%
\pgfsys@defobject{currentmarker}{\pgfqpoint{0.000000in}{-0.048611in}}{\pgfqpoint{0.000000in}{0.000000in}}{%
\pgfpathmoveto{\pgfqpoint{0.000000in}{0.000000in}}%
\pgfpathlineto{\pgfqpoint{0.000000in}{-0.048611in}}%
\pgfusepath{stroke,fill}%
}%
\begin{pgfscope}%
\pgfsys@transformshift{1.701818in}{0.528000in}%
\pgfsys@useobject{currentmarker}{}%
\end{pgfscope}%
\end{pgfscope}%
\begin{pgfscope}%
\definecolor{textcolor}{rgb}{0.000000,0.000000,0.000000}%
\pgfsetstrokecolor{textcolor}%
\pgfsetfillcolor{textcolor}%
\pgftext[x=1.701818in,y=0.430778in,,top]{\color{textcolor}\sffamily\fontsize{10.000000}{12.000000}\selectfont 20}%
\end{pgfscope}%
\begin{pgfscope}%
\pgfsetbuttcap%
\pgfsetroundjoin%
\definecolor{currentfill}{rgb}{0.000000,0.000000,0.000000}%
\pgfsetfillcolor{currentfill}%
\pgfsetlinewidth{0.803000pt}%
\definecolor{currentstroke}{rgb}{0.000000,0.000000,0.000000}%
\pgfsetstrokecolor{currentstroke}%
\pgfsetdash{}{0pt}%
\pgfsys@defobject{currentmarker}{\pgfqpoint{0.000000in}{-0.048611in}}{\pgfqpoint{0.000000in}{0.000000in}}{%
\pgfpathmoveto{\pgfqpoint{0.000000in}{0.000000in}}%
\pgfpathlineto{\pgfqpoint{0.000000in}{-0.048611in}}%
\pgfusepath{stroke,fill}%
}%
\begin{pgfscope}%
\pgfsys@transformshift{2.152727in}{0.528000in}%
\pgfsys@useobject{currentmarker}{}%
\end{pgfscope}%
\end{pgfscope}%
\begin{pgfscope}%
\definecolor{textcolor}{rgb}{0.000000,0.000000,0.000000}%
\pgfsetstrokecolor{textcolor}%
\pgfsetfillcolor{textcolor}%
\pgftext[x=2.152727in,y=0.430778in,,top]{\color{textcolor}\sffamily\fontsize{10.000000}{12.000000}\selectfont 30}%
\end{pgfscope}%
\begin{pgfscope}%
\pgfsetbuttcap%
\pgfsetroundjoin%
\definecolor{currentfill}{rgb}{0.000000,0.000000,0.000000}%
\pgfsetfillcolor{currentfill}%
\pgfsetlinewidth{0.803000pt}%
\definecolor{currentstroke}{rgb}{0.000000,0.000000,0.000000}%
\pgfsetstrokecolor{currentstroke}%
\pgfsetdash{}{0pt}%
\pgfsys@defobject{currentmarker}{\pgfqpoint{0.000000in}{-0.048611in}}{\pgfqpoint{0.000000in}{0.000000in}}{%
\pgfpathmoveto{\pgfqpoint{0.000000in}{0.000000in}}%
\pgfpathlineto{\pgfqpoint{0.000000in}{-0.048611in}}%
\pgfusepath{stroke,fill}%
}%
\begin{pgfscope}%
\pgfsys@transformshift{2.603636in}{0.528000in}%
\pgfsys@useobject{currentmarker}{}%
\end{pgfscope}%
\end{pgfscope}%
\begin{pgfscope}%
\definecolor{textcolor}{rgb}{0.000000,0.000000,0.000000}%
\pgfsetstrokecolor{textcolor}%
\pgfsetfillcolor{textcolor}%
\pgftext[x=2.603636in,y=0.430778in,,top]{\color{textcolor}\sffamily\fontsize{10.000000}{12.000000}\selectfont 40}%
\end{pgfscope}%
\begin{pgfscope}%
\pgfsetbuttcap%
\pgfsetroundjoin%
\definecolor{currentfill}{rgb}{0.000000,0.000000,0.000000}%
\pgfsetfillcolor{currentfill}%
\pgfsetlinewidth{0.803000pt}%
\definecolor{currentstroke}{rgb}{0.000000,0.000000,0.000000}%
\pgfsetstrokecolor{currentstroke}%
\pgfsetdash{}{0pt}%
\pgfsys@defobject{currentmarker}{\pgfqpoint{0.000000in}{-0.048611in}}{\pgfqpoint{0.000000in}{0.000000in}}{%
\pgfpathmoveto{\pgfqpoint{0.000000in}{0.000000in}}%
\pgfpathlineto{\pgfqpoint{0.000000in}{-0.048611in}}%
\pgfusepath{stroke,fill}%
}%
\begin{pgfscope}%
\pgfsys@transformshift{3.054545in}{0.528000in}%
\pgfsys@useobject{currentmarker}{}%
\end{pgfscope}%
\end{pgfscope}%
\begin{pgfscope}%
\definecolor{textcolor}{rgb}{0.000000,0.000000,0.000000}%
\pgfsetstrokecolor{textcolor}%
\pgfsetfillcolor{textcolor}%
\pgftext[x=3.054545in,y=0.430778in,,top]{\color{textcolor}\sffamily\fontsize{10.000000}{12.000000}\selectfont 50}%
\end{pgfscope}%
\begin{pgfscope}%
\pgfsetbuttcap%
\pgfsetroundjoin%
\definecolor{currentfill}{rgb}{0.000000,0.000000,0.000000}%
\pgfsetfillcolor{currentfill}%
\pgfsetlinewidth{0.803000pt}%
\definecolor{currentstroke}{rgb}{0.000000,0.000000,0.000000}%
\pgfsetstrokecolor{currentstroke}%
\pgfsetdash{}{0pt}%
\pgfsys@defobject{currentmarker}{\pgfqpoint{0.000000in}{-0.048611in}}{\pgfqpoint{0.000000in}{0.000000in}}{%
\pgfpathmoveto{\pgfqpoint{0.000000in}{0.000000in}}%
\pgfpathlineto{\pgfqpoint{0.000000in}{-0.048611in}}%
\pgfusepath{stroke,fill}%
}%
\begin{pgfscope}%
\pgfsys@transformshift{3.505455in}{0.528000in}%
\pgfsys@useobject{currentmarker}{}%
\end{pgfscope}%
\end{pgfscope}%
\begin{pgfscope}%
\definecolor{textcolor}{rgb}{0.000000,0.000000,0.000000}%
\pgfsetstrokecolor{textcolor}%
\pgfsetfillcolor{textcolor}%
\pgftext[x=3.505455in,y=0.430778in,,top]{\color{textcolor}\sffamily\fontsize{10.000000}{12.000000}\selectfont 60}%
\end{pgfscope}%
\begin{pgfscope}%
\pgfsetbuttcap%
\pgfsetroundjoin%
\definecolor{currentfill}{rgb}{0.000000,0.000000,0.000000}%
\pgfsetfillcolor{currentfill}%
\pgfsetlinewidth{0.803000pt}%
\definecolor{currentstroke}{rgb}{0.000000,0.000000,0.000000}%
\pgfsetstrokecolor{currentstroke}%
\pgfsetdash{}{0pt}%
\pgfsys@defobject{currentmarker}{\pgfqpoint{0.000000in}{-0.048611in}}{\pgfqpoint{0.000000in}{0.000000in}}{%
\pgfpathmoveto{\pgfqpoint{0.000000in}{0.000000in}}%
\pgfpathlineto{\pgfqpoint{0.000000in}{-0.048611in}}%
\pgfusepath{stroke,fill}%
}%
\begin{pgfscope}%
\pgfsys@transformshift{3.956364in}{0.528000in}%
\pgfsys@useobject{currentmarker}{}%
\end{pgfscope}%
\end{pgfscope}%
\begin{pgfscope}%
\definecolor{textcolor}{rgb}{0.000000,0.000000,0.000000}%
\pgfsetstrokecolor{textcolor}%
\pgfsetfillcolor{textcolor}%
\pgftext[x=3.956364in,y=0.430778in,,top]{\color{textcolor}\sffamily\fontsize{10.000000}{12.000000}\selectfont 70}%
\end{pgfscope}%
\begin{pgfscope}%
\pgfsetbuttcap%
\pgfsetroundjoin%
\definecolor{currentfill}{rgb}{0.000000,0.000000,0.000000}%
\pgfsetfillcolor{currentfill}%
\pgfsetlinewidth{0.803000pt}%
\definecolor{currentstroke}{rgb}{0.000000,0.000000,0.000000}%
\pgfsetstrokecolor{currentstroke}%
\pgfsetdash{}{0pt}%
\pgfsys@defobject{currentmarker}{\pgfqpoint{0.000000in}{-0.048611in}}{\pgfqpoint{0.000000in}{0.000000in}}{%
\pgfpathmoveto{\pgfqpoint{0.000000in}{0.000000in}}%
\pgfpathlineto{\pgfqpoint{0.000000in}{-0.048611in}}%
\pgfusepath{stroke,fill}%
}%
\begin{pgfscope}%
\pgfsys@transformshift{4.407273in}{0.528000in}%
\pgfsys@useobject{currentmarker}{}%
\end{pgfscope}%
\end{pgfscope}%
\begin{pgfscope}%
\definecolor{textcolor}{rgb}{0.000000,0.000000,0.000000}%
\pgfsetstrokecolor{textcolor}%
\pgfsetfillcolor{textcolor}%
\pgftext[x=4.407273in,y=0.430778in,,top]{\color{textcolor}\sffamily\fontsize{10.000000}{12.000000}\selectfont 80}%
\end{pgfscope}%
\begin{pgfscope}%
\pgfsetbuttcap%
\pgfsetroundjoin%
\definecolor{currentfill}{rgb}{0.000000,0.000000,0.000000}%
\pgfsetfillcolor{currentfill}%
\pgfsetlinewidth{0.803000pt}%
\definecolor{currentstroke}{rgb}{0.000000,0.000000,0.000000}%
\pgfsetstrokecolor{currentstroke}%
\pgfsetdash{}{0pt}%
\pgfsys@defobject{currentmarker}{\pgfqpoint{0.000000in}{-0.048611in}}{\pgfqpoint{0.000000in}{0.000000in}}{%
\pgfpathmoveto{\pgfqpoint{0.000000in}{0.000000in}}%
\pgfpathlineto{\pgfqpoint{0.000000in}{-0.048611in}}%
\pgfusepath{stroke,fill}%
}%
\begin{pgfscope}%
\pgfsys@transformshift{4.858182in}{0.528000in}%
\pgfsys@useobject{currentmarker}{}%
\end{pgfscope}%
\end{pgfscope}%
\begin{pgfscope}%
\definecolor{textcolor}{rgb}{0.000000,0.000000,0.000000}%
\pgfsetstrokecolor{textcolor}%
\pgfsetfillcolor{textcolor}%
\pgftext[x=4.858182in,y=0.430778in,,top]{\color{textcolor}\sffamily\fontsize{10.000000}{12.000000}\selectfont 90}%
\end{pgfscope}%
\begin{pgfscope}%
\pgfsetbuttcap%
\pgfsetroundjoin%
\definecolor{currentfill}{rgb}{0.000000,0.000000,0.000000}%
\pgfsetfillcolor{currentfill}%
\pgfsetlinewidth{0.803000pt}%
\definecolor{currentstroke}{rgb}{0.000000,0.000000,0.000000}%
\pgfsetstrokecolor{currentstroke}%
\pgfsetdash{}{0pt}%
\pgfsys@defobject{currentmarker}{\pgfqpoint{0.000000in}{-0.048611in}}{\pgfqpoint{0.000000in}{0.000000in}}{%
\pgfpathmoveto{\pgfqpoint{0.000000in}{0.000000in}}%
\pgfpathlineto{\pgfqpoint{0.000000in}{-0.048611in}}%
\pgfusepath{stroke,fill}%
}%
\begin{pgfscope}%
\pgfsys@transformshift{5.309091in}{0.528000in}%
\pgfsys@useobject{currentmarker}{}%
\end{pgfscope}%
\end{pgfscope}%
\begin{pgfscope}%
\definecolor{textcolor}{rgb}{0.000000,0.000000,0.000000}%
\pgfsetstrokecolor{textcolor}%
\pgfsetfillcolor{textcolor}%
\pgftext[x=5.309091in,y=0.430778in,,top]{\color{textcolor}\sffamily\fontsize{10.000000}{12.000000}\selectfont 100}%
\end{pgfscope}%
\begin{pgfscope}%
\definecolor{textcolor}{rgb}{0.000000,0.000000,0.000000}%
\pgfsetstrokecolor{textcolor}%
\pgfsetfillcolor{textcolor}%
\pgftext[x=3.280000in,y=0.240809in,,top]{\color{textcolor}\sffamily\fontsize{10.000000}{12.000000}\selectfont \(\displaystyle k\)}%
\end{pgfscope}%
\begin{pgfscope}%
\pgfsetbuttcap%
\pgfsetroundjoin%
\definecolor{currentfill}{rgb}{0.000000,0.000000,0.000000}%
\pgfsetfillcolor{currentfill}%
\pgfsetlinewidth{0.803000pt}%
\definecolor{currentstroke}{rgb}{0.000000,0.000000,0.000000}%
\pgfsetstrokecolor{currentstroke}%
\pgfsetdash{}{0pt}%
\pgfsys@defobject{currentmarker}{\pgfqpoint{-0.048611in}{0.000000in}}{\pgfqpoint{0.000000in}{0.000000in}}{%
\pgfpathmoveto{\pgfqpoint{0.000000in}{0.000000in}}%
\pgfpathlineto{\pgfqpoint{-0.048611in}{0.000000in}}%
\pgfusepath{stroke,fill}%
}%
\begin{pgfscope}%
\pgfsys@transformshift{0.800000in}{0.678442in}%
\pgfsys@useobject{currentmarker}{}%
\end{pgfscope}%
\end{pgfscope}%
\begin{pgfscope}%
\definecolor{textcolor}{rgb}{0.000000,0.000000,0.000000}%
\pgfsetstrokecolor{textcolor}%
\pgfsetfillcolor{textcolor}%
\pgftext[x=0.614413in,y=0.625680in,left,base]{\color{textcolor}\sffamily\fontsize{10.000000}{12.000000}\selectfont 0}%
\end{pgfscope}%
\begin{pgfscope}%
\pgfsetbuttcap%
\pgfsetroundjoin%
\definecolor{currentfill}{rgb}{0.000000,0.000000,0.000000}%
\pgfsetfillcolor{currentfill}%
\pgfsetlinewidth{0.803000pt}%
\definecolor{currentstroke}{rgb}{0.000000,0.000000,0.000000}%
\pgfsetstrokecolor{currentstroke}%
\pgfsetdash{}{0pt}%
\pgfsys@defobject{currentmarker}{\pgfqpoint{-0.048611in}{0.000000in}}{\pgfqpoint{0.000000in}{0.000000in}}{%
\pgfpathmoveto{\pgfqpoint{0.000000in}{0.000000in}}%
\pgfpathlineto{\pgfqpoint{-0.048611in}{0.000000in}}%
\pgfusepath{stroke,fill}%
}%
\begin{pgfscope}%
\pgfsys@transformshift{0.800000in}{1.077492in}%
\pgfsys@useobject{currentmarker}{}%
\end{pgfscope}%
\end{pgfscope}%
\begin{pgfscope}%
\definecolor{textcolor}{rgb}{0.000000,0.000000,0.000000}%
\pgfsetstrokecolor{textcolor}%
\pgfsetfillcolor{textcolor}%
\pgftext[x=0.437682in,y=1.024730in,left,base]{\color{textcolor}\sffamily\fontsize{10.000000}{12.000000}\selectfont 250}%
\end{pgfscope}%
\begin{pgfscope}%
\pgfsetbuttcap%
\pgfsetroundjoin%
\definecolor{currentfill}{rgb}{0.000000,0.000000,0.000000}%
\pgfsetfillcolor{currentfill}%
\pgfsetlinewidth{0.803000pt}%
\definecolor{currentstroke}{rgb}{0.000000,0.000000,0.000000}%
\pgfsetstrokecolor{currentstroke}%
\pgfsetdash{}{0pt}%
\pgfsys@defobject{currentmarker}{\pgfqpoint{-0.048611in}{0.000000in}}{\pgfqpoint{0.000000in}{0.000000in}}{%
\pgfpathmoveto{\pgfqpoint{0.000000in}{0.000000in}}%
\pgfpathlineto{\pgfqpoint{-0.048611in}{0.000000in}}%
\pgfusepath{stroke,fill}%
}%
\begin{pgfscope}%
\pgfsys@transformshift{0.800000in}{1.476542in}%
\pgfsys@useobject{currentmarker}{}%
\end{pgfscope}%
\end{pgfscope}%
\begin{pgfscope}%
\definecolor{textcolor}{rgb}{0.000000,0.000000,0.000000}%
\pgfsetstrokecolor{textcolor}%
\pgfsetfillcolor{textcolor}%
\pgftext[x=0.437682in,y=1.423780in,left,base]{\color{textcolor}\sffamily\fontsize{10.000000}{12.000000}\selectfont 500}%
\end{pgfscope}%
\begin{pgfscope}%
\pgfsetbuttcap%
\pgfsetroundjoin%
\definecolor{currentfill}{rgb}{0.000000,0.000000,0.000000}%
\pgfsetfillcolor{currentfill}%
\pgfsetlinewidth{0.803000pt}%
\definecolor{currentstroke}{rgb}{0.000000,0.000000,0.000000}%
\pgfsetstrokecolor{currentstroke}%
\pgfsetdash{}{0pt}%
\pgfsys@defobject{currentmarker}{\pgfqpoint{-0.048611in}{0.000000in}}{\pgfqpoint{0.000000in}{0.000000in}}{%
\pgfpathmoveto{\pgfqpoint{0.000000in}{0.000000in}}%
\pgfpathlineto{\pgfqpoint{-0.048611in}{0.000000in}}%
\pgfusepath{stroke,fill}%
}%
\begin{pgfscope}%
\pgfsys@transformshift{0.800000in}{1.875591in}%
\pgfsys@useobject{currentmarker}{}%
\end{pgfscope}%
\end{pgfscope}%
\begin{pgfscope}%
\definecolor{textcolor}{rgb}{0.000000,0.000000,0.000000}%
\pgfsetstrokecolor{textcolor}%
\pgfsetfillcolor{textcolor}%
\pgftext[x=0.437682in,y=1.822830in,left,base]{\color{textcolor}\sffamily\fontsize{10.000000}{12.000000}\selectfont 750}%
\end{pgfscope}%
\begin{pgfscope}%
\pgfsetbuttcap%
\pgfsetroundjoin%
\definecolor{currentfill}{rgb}{0.000000,0.000000,0.000000}%
\pgfsetfillcolor{currentfill}%
\pgfsetlinewidth{0.803000pt}%
\definecolor{currentstroke}{rgb}{0.000000,0.000000,0.000000}%
\pgfsetstrokecolor{currentstroke}%
\pgfsetdash{}{0pt}%
\pgfsys@defobject{currentmarker}{\pgfqpoint{-0.048611in}{0.000000in}}{\pgfqpoint{0.000000in}{0.000000in}}{%
\pgfpathmoveto{\pgfqpoint{0.000000in}{0.000000in}}%
\pgfpathlineto{\pgfqpoint{-0.048611in}{0.000000in}}%
\pgfusepath{stroke,fill}%
}%
\begin{pgfscope}%
\pgfsys@transformshift{0.800000in}{2.274641in}%
\pgfsys@useobject{currentmarker}{}%
\end{pgfscope}%
\end{pgfscope}%
\begin{pgfscope}%
\definecolor{textcolor}{rgb}{0.000000,0.000000,0.000000}%
\pgfsetstrokecolor{textcolor}%
\pgfsetfillcolor{textcolor}%
\pgftext[x=0.349316in,y=2.221880in,left,base]{\color{textcolor}\sffamily\fontsize{10.000000}{12.000000}\selectfont 1000}%
\end{pgfscope}%
\begin{pgfscope}%
\pgfsetbuttcap%
\pgfsetroundjoin%
\definecolor{currentfill}{rgb}{0.000000,0.000000,0.000000}%
\pgfsetfillcolor{currentfill}%
\pgfsetlinewidth{0.803000pt}%
\definecolor{currentstroke}{rgb}{0.000000,0.000000,0.000000}%
\pgfsetstrokecolor{currentstroke}%
\pgfsetdash{}{0pt}%
\pgfsys@defobject{currentmarker}{\pgfqpoint{-0.048611in}{0.000000in}}{\pgfqpoint{0.000000in}{0.000000in}}{%
\pgfpathmoveto{\pgfqpoint{0.000000in}{0.000000in}}%
\pgfpathlineto{\pgfqpoint{-0.048611in}{0.000000in}}%
\pgfusepath{stroke,fill}%
}%
\begin{pgfscope}%
\pgfsys@transformshift{0.800000in}{2.673691in}%
\pgfsys@useobject{currentmarker}{}%
\end{pgfscope}%
\end{pgfscope}%
\begin{pgfscope}%
\definecolor{textcolor}{rgb}{0.000000,0.000000,0.000000}%
\pgfsetstrokecolor{textcolor}%
\pgfsetfillcolor{textcolor}%
\pgftext[x=0.349316in,y=2.620930in,left,base]{\color{textcolor}\sffamily\fontsize{10.000000}{12.000000}\selectfont 1250}%
\end{pgfscope}%
\begin{pgfscope}%
\pgfsetbuttcap%
\pgfsetroundjoin%
\definecolor{currentfill}{rgb}{0.000000,0.000000,0.000000}%
\pgfsetfillcolor{currentfill}%
\pgfsetlinewidth{0.803000pt}%
\definecolor{currentstroke}{rgb}{0.000000,0.000000,0.000000}%
\pgfsetstrokecolor{currentstroke}%
\pgfsetdash{}{0pt}%
\pgfsys@defobject{currentmarker}{\pgfqpoint{-0.048611in}{0.000000in}}{\pgfqpoint{0.000000in}{0.000000in}}{%
\pgfpathmoveto{\pgfqpoint{0.000000in}{0.000000in}}%
\pgfpathlineto{\pgfqpoint{-0.048611in}{0.000000in}}%
\pgfusepath{stroke,fill}%
}%
\begin{pgfscope}%
\pgfsys@transformshift{0.800000in}{3.072741in}%
\pgfsys@useobject{currentmarker}{}%
\end{pgfscope}%
\end{pgfscope}%
\begin{pgfscope}%
\definecolor{textcolor}{rgb}{0.000000,0.000000,0.000000}%
\pgfsetstrokecolor{textcolor}%
\pgfsetfillcolor{textcolor}%
\pgftext[x=0.349316in,y=3.019980in,left,base]{\color{textcolor}\sffamily\fontsize{10.000000}{12.000000}\selectfont 1500}%
\end{pgfscope}%
\begin{pgfscope}%
\pgfsetbuttcap%
\pgfsetroundjoin%
\definecolor{currentfill}{rgb}{0.000000,0.000000,0.000000}%
\pgfsetfillcolor{currentfill}%
\pgfsetlinewidth{0.803000pt}%
\definecolor{currentstroke}{rgb}{0.000000,0.000000,0.000000}%
\pgfsetstrokecolor{currentstroke}%
\pgfsetdash{}{0pt}%
\pgfsys@defobject{currentmarker}{\pgfqpoint{-0.048611in}{0.000000in}}{\pgfqpoint{0.000000in}{0.000000in}}{%
\pgfpathmoveto{\pgfqpoint{0.000000in}{0.000000in}}%
\pgfpathlineto{\pgfqpoint{-0.048611in}{0.000000in}}%
\pgfusepath{stroke,fill}%
}%
\begin{pgfscope}%
\pgfsys@transformshift{0.800000in}{3.471791in}%
\pgfsys@useobject{currentmarker}{}%
\end{pgfscope}%
\end{pgfscope}%
\begin{pgfscope}%
\definecolor{textcolor}{rgb}{0.000000,0.000000,0.000000}%
\pgfsetstrokecolor{textcolor}%
\pgfsetfillcolor{textcolor}%
\pgftext[x=0.349316in,y=3.419029in,left,base]{\color{textcolor}\sffamily\fontsize{10.000000}{12.000000}\selectfont 1750}%
\end{pgfscope}%
\begin{pgfscope}%
\pgfsetbuttcap%
\pgfsetroundjoin%
\definecolor{currentfill}{rgb}{0.000000,0.000000,0.000000}%
\pgfsetfillcolor{currentfill}%
\pgfsetlinewidth{0.803000pt}%
\definecolor{currentstroke}{rgb}{0.000000,0.000000,0.000000}%
\pgfsetstrokecolor{currentstroke}%
\pgfsetdash{}{0pt}%
\pgfsys@defobject{currentmarker}{\pgfqpoint{-0.048611in}{0.000000in}}{\pgfqpoint{0.000000in}{0.000000in}}{%
\pgfpathmoveto{\pgfqpoint{0.000000in}{0.000000in}}%
\pgfpathlineto{\pgfqpoint{-0.048611in}{0.000000in}}%
\pgfusepath{stroke,fill}%
}%
\begin{pgfscope}%
\pgfsys@transformshift{0.800000in}{3.870841in}%
\pgfsys@useobject{currentmarker}{}%
\end{pgfscope}%
\end{pgfscope}%
\begin{pgfscope}%
\definecolor{textcolor}{rgb}{0.000000,0.000000,0.000000}%
\pgfsetstrokecolor{textcolor}%
\pgfsetfillcolor{textcolor}%
\pgftext[x=0.349316in,y=3.818079in,left,base]{\color{textcolor}\sffamily\fontsize{10.000000}{12.000000}\selectfont 2000}%
\end{pgfscope}%
\begin{pgfscope}%
\definecolor{textcolor}{rgb}{0.000000,0.000000,0.000000}%
\pgfsetstrokecolor{textcolor}%
\pgfsetfillcolor{textcolor}%
\pgftext[x=0.293761in,y=2.376000in,,bottom,rotate=90.000000]{\color{textcolor}\sffamily\fontsize{10.000000}{12.000000}\selectfont Number of GMRES iterations}%
\end{pgfscope}%
\begin{pgfscope}%
\pgfpathrectangle{\pgfqpoint{0.800000in}{0.528000in}}{\pgfqpoint{4.960000in}{3.696000in}}%
\pgfusepath{clip}%
\pgfsetbuttcap%
\pgfsetroundjoin%
\pgfsetlinewidth{1.505625pt}%
\definecolor{currentstroke}{rgb}{0.000000,0.000000,0.000000}%
\pgfsetstrokecolor{currentstroke}%
\pgfsetdash{{5.550000pt}{2.400000pt}}{0.000000pt}%
\pgfpathmoveto{\pgfqpoint{1.250909in}{0.700789in}}%
\pgfpathlineto{\pgfqpoint{1.701818in}{0.742290in}}%
\pgfpathlineto{\pgfqpoint{2.152727in}{0.868390in}}%
\pgfpathlineto{\pgfqpoint{2.603636in}{1.090261in}}%
\pgfpathlineto{\pgfqpoint{3.054545in}{1.360019in}}%
\pgfpathlineto{\pgfqpoint{3.505455in}{1.679259in}}%
\pgfpathlineto{\pgfqpoint{3.956364in}{2.178869in}}%
\pgfpathlineto{\pgfqpoint{4.407273in}{2.712000in}}%
\pgfpathlineto{\pgfqpoint{4.858182in}{3.384000in}}%
\pgfpathlineto{\pgfqpoint{5.309091in}{4.056000in}}%
\pgfusepath{stroke}%
\end{pgfscope}%
\begin{pgfscope}%
\pgfpathrectangle{\pgfqpoint{0.800000in}{0.528000in}}{\pgfqpoint{4.960000in}{3.696000in}}%
\pgfusepath{clip}%
\pgfsetbuttcap%
\pgfsetroundjoin%
\definecolor{currentfill}{rgb}{0.000000,0.000000,0.000000}%
\pgfsetfillcolor{currentfill}%
\pgfsetlinewidth{1.003750pt}%
\definecolor{currentstroke}{rgb}{0.000000,0.000000,0.000000}%
\pgfsetstrokecolor{currentstroke}%
\pgfsetdash{}{0pt}%
\pgfsys@defobject{currentmarker}{\pgfqpoint{-0.041667in}{-0.041667in}}{\pgfqpoint{0.041667in}{0.041667in}}{%
\pgfpathmoveto{\pgfqpoint{0.000000in}{-0.041667in}}%
\pgfpathcurveto{\pgfqpoint{0.011050in}{-0.041667in}}{\pgfqpoint{0.021649in}{-0.037276in}}{\pgfqpoint{0.029463in}{-0.029463in}}%
\pgfpathcurveto{\pgfqpoint{0.037276in}{-0.021649in}}{\pgfqpoint{0.041667in}{-0.011050in}}{\pgfqpoint{0.041667in}{0.000000in}}%
\pgfpathcurveto{\pgfqpoint{0.041667in}{0.011050in}}{\pgfqpoint{0.037276in}{0.021649in}}{\pgfqpoint{0.029463in}{0.029463in}}%
\pgfpathcurveto{\pgfqpoint{0.021649in}{0.037276in}}{\pgfqpoint{0.011050in}{0.041667in}}{\pgfqpoint{0.000000in}{0.041667in}}%
\pgfpathcurveto{\pgfqpoint{-0.011050in}{0.041667in}}{\pgfqpoint{-0.021649in}{0.037276in}}{\pgfqpoint{-0.029463in}{0.029463in}}%
\pgfpathcurveto{\pgfqpoint{-0.037276in}{0.021649in}}{\pgfqpoint{-0.041667in}{0.011050in}}{\pgfqpoint{-0.041667in}{0.000000in}}%
\pgfpathcurveto{\pgfqpoint{-0.041667in}{-0.011050in}}{\pgfqpoint{-0.037276in}{-0.021649in}}{\pgfqpoint{-0.029463in}{-0.029463in}}%
\pgfpathcurveto{\pgfqpoint{-0.021649in}{-0.037276in}}{\pgfqpoint{-0.011050in}{-0.041667in}}{\pgfqpoint{0.000000in}{-0.041667in}}%
\pgfpathclose%
\pgfusepath{stroke,fill}%
}%
\begin{pgfscope}%
\pgfsys@transformshift{1.250909in}{0.700789in}%
\pgfsys@useobject{currentmarker}{}%
\end{pgfscope}%
\begin{pgfscope}%
\pgfsys@transformshift{1.701818in}{0.742290in}%
\pgfsys@useobject{currentmarker}{}%
\end{pgfscope}%
\begin{pgfscope}%
\pgfsys@transformshift{2.152727in}{0.868390in}%
\pgfsys@useobject{currentmarker}{}%
\end{pgfscope}%
\begin{pgfscope}%
\pgfsys@transformshift{2.603636in}{1.090261in}%
\pgfsys@useobject{currentmarker}{}%
\end{pgfscope}%
\begin{pgfscope}%
\pgfsys@transformshift{3.054545in}{1.360019in}%
\pgfsys@useobject{currentmarker}{}%
\end{pgfscope}%
\begin{pgfscope}%
\pgfsys@transformshift{3.505455in}{1.679259in}%
\pgfsys@useobject{currentmarker}{}%
\end{pgfscope}%
\begin{pgfscope}%
\pgfsys@transformshift{3.956364in}{2.178869in}%
\pgfsys@useobject{currentmarker}{}%
\end{pgfscope}%
\begin{pgfscope}%
\pgfsys@transformshift{4.407273in}{2.712000in}%
\pgfsys@useobject{currentmarker}{}%
\end{pgfscope}%
\begin{pgfscope}%
\pgfsys@transformshift{4.858182in}{3.384000in}%
\pgfsys@useobject{currentmarker}{}%
\end{pgfscope}%
\begin{pgfscope}%
\pgfsys@transformshift{5.309091in}{4.056000in}%
\pgfsys@useobject{currentmarker}{}%
\end{pgfscope}%
\end{pgfscope}%
\begin{pgfscope}%
\pgfpathrectangle{\pgfqpoint{0.800000in}{0.528000in}}{\pgfqpoint{4.960000in}{3.696000in}}%
\pgfusepath{clip}%
\pgfsetbuttcap%
\pgfsetroundjoin%
\pgfsetlinewidth{1.505625pt}%
\definecolor{currentstroke}{rgb}{0.000000,0.000000,0.000000}%
\pgfsetstrokecolor{currentstroke}%
\pgfsetdash{{5.550000pt}{2.400000pt}}{0.000000pt}%
\pgfpathmoveto{\pgfqpoint{1.250909in}{0.699192in}}%
\pgfpathlineto{\pgfqpoint{1.701818in}{0.721539in}}%
\pgfpathlineto{\pgfqpoint{2.152727in}{0.790176in}}%
\pgfpathlineto{\pgfqpoint{2.603636in}{0.913083in}}%
\pgfpathlineto{\pgfqpoint{3.054545in}{1.096646in}}%
\pgfpathlineto{\pgfqpoint{3.505455in}{1.307344in}}%
\pgfpathlineto{\pgfqpoint{3.956364in}{1.620200in}}%
\pgfpathlineto{\pgfqpoint{4.407273in}{1.995306in}}%
\pgfpathlineto{\pgfqpoint{4.858182in}{2.478955in}}%
\pgfpathlineto{\pgfqpoint{5.309091in}{2.901948in}}%
\pgfusepath{stroke}%
\end{pgfscope}%
\begin{pgfscope}%
\pgfpathrectangle{\pgfqpoint{0.800000in}{0.528000in}}{\pgfqpoint{4.960000in}{3.696000in}}%
\pgfusepath{clip}%
\pgfsetbuttcap%
\pgfsetmiterjoin%
\definecolor{currentfill}{rgb}{0.000000,0.000000,0.000000}%
\pgfsetfillcolor{currentfill}%
\pgfsetlinewidth{1.003750pt}%
\definecolor{currentstroke}{rgb}{0.000000,0.000000,0.000000}%
\pgfsetstrokecolor{currentstroke}%
\pgfsetdash{}{0pt}%
\pgfsys@defobject{currentmarker}{\pgfqpoint{-0.041667in}{-0.041667in}}{\pgfqpoint{0.041667in}{0.041667in}}{%
\pgfpathmoveto{\pgfqpoint{-0.000000in}{-0.041667in}}%
\pgfpathlineto{\pgfqpoint{0.041667in}{0.041667in}}%
\pgfpathlineto{\pgfqpoint{-0.041667in}{0.041667in}}%
\pgfpathclose%
\pgfusepath{stroke,fill}%
}%
\begin{pgfscope}%
\pgfsys@transformshift{1.250909in}{0.699192in}%
\pgfsys@useobject{currentmarker}{}%
\end{pgfscope}%
\begin{pgfscope}%
\pgfsys@transformshift{1.701818in}{0.721539in}%
\pgfsys@useobject{currentmarker}{}%
\end{pgfscope}%
\begin{pgfscope}%
\pgfsys@transformshift{2.152727in}{0.790176in}%
\pgfsys@useobject{currentmarker}{}%
\end{pgfscope}%
\begin{pgfscope}%
\pgfsys@transformshift{2.603636in}{0.913083in}%
\pgfsys@useobject{currentmarker}{}%
\end{pgfscope}%
\begin{pgfscope}%
\pgfsys@transformshift{3.054545in}{1.096646in}%
\pgfsys@useobject{currentmarker}{}%
\end{pgfscope}%
\begin{pgfscope}%
\pgfsys@transformshift{3.505455in}{1.307344in}%
\pgfsys@useobject{currentmarker}{}%
\end{pgfscope}%
\begin{pgfscope}%
\pgfsys@transformshift{3.956364in}{1.620200in}%
\pgfsys@useobject{currentmarker}{}%
\end{pgfscope}%
\begin{pgfscope}%
\pgfsys@transformshift{4.407273in}{1.995306in}%
\pgfsys@useobject{currentmarker}{}%
\end{pgfscope}%
\begin{pgfscope}%
\pgfsys@transformshift{4.858182in}{2.478955in}%
\pgfsys@useobject{currentmarker}{}%
\end{pgfscope}%
\begin{pgfscope}%
\pgfsys@transformshift{5.309091in}{2.901948in}%
\pgfsys@useobject{currentmarker}{}%
\end{pgfscope}%
\end{pgfscope}%
\begin{pgfscope}%
\pgfpathrectangle{\pgfqpoint{0.800000in}{0.528000in}}{\pgfqpoint{4.960000in}{3.696000in}}%
\pgfusepath{clip}%
\pgfsetbuttcap%
\pgfsetroundjoin%
\pgfsetlinewidth{1.505625pt}%
\definecolor{currentstroke}{rgb}{0.000000,0.000000,0.000000}%
\pgfsetstrokecolor{currentstroke}%
\pgfsetdash{{5.550000pt}{2.400000pt}}{0.000000pt}%
\pgfpathmoveto{\pgfqpoint{1.250909in}{0.697596in}}%
\pgfpathlineto{\pgfqpoint{1.701818in}{0.713558in}}%
\pgfpathlineto{\pgfqpoint{2.152727in}{0.742290in}}%
\pgfpathlineto{\pgfqpoint{2.603636in}{0.801349in}}%
\pgfpathlineto{\pgfqpoint{3.054545in}{0.892333in}}%
\pgfpathlineto{\pgfqpoint{3.505455in}{0.996086in}}%
\pgfpathlineto{\pgfqpoint{3.956364in}{1.144532in}}%
\pgfpathlineto{\pgfqpoint{4.407273in}{1.347249in}}%
\pgfpathlineto{\pgfqpoint{4.858182in}{1.557948in}}%
\pgfpathlineto{\pgfqpoint{5.309091in}{1.837283in}}%
\pgfusepath{stroke}%
\end{pgfscope}%
\begin{pgfscope}%
\pgfpathrectangle{\pgfqpoint{0.800000in}{0.528000in}}{\pgfqpoint{4.960000in}{3.696000in}}%
\pgfusepath{clip}%
\pgfsetbuttcap%
\pgfsetmiterjoin%
\definecolor{currentfill}{rgb}{0.000000,0.000000,0.000000}%
\pgfsetfillcolor{currentfill}%
\pgfsetlinewidth{1.003750pt}%
\definecolor{currentstroke}{rgb}{0.000000,0.000000,0.000000}%
\pgfsetstrokecolor{currentstroke}%
\pgfsetdash{}{0pt}%
\pgfsys@defobject{currentmarker}{\pgfqpoint{-0.041667in}{-0.041667in}}{\pgfqpoint{0.041667in}{0.041667in}}{%
\pgfpathmoveto{\pgfqpoint{-0.041667in}{-0.041667in}}%
\pgfpathlineto{\pgfqpoint{0.041667in}{-0.041667in}}%
\pgfpathlineto{\pgfqpoint{0.041667in}{0.041667in}}%
\pgfpathlineto{\pgfqpoint{-0.041667in}{0.041667in}}%
\pgfpathclose%
\pgfusepath{stroke,fill}%
}%
\begin{pgfscope}%
\pgfsys@transformshift{1.250909in}{0.697596in}%
\pgfsys@useobject{currentmarker}{}%
\end{pgfscope}%
\begin{pgfscope}%
\pgfsys@transformshift{1.701818in}{0.713558in}%
\pgfsys@useobject{currentmarker}{}%
\end{pgfscope}%
\begin{pgfscope}%
\pgfsys@transformshift{2.152727in}{0.742290in}%
\pgfsys@useobject{currentmarker}{}%
\end{pgfscope}%
\begin{pgfscope}%
\pgfsys@transformshift{2.603636in}{0.801349in}%
\pgfsys@useobject{currentmarker}{}%
\end{pgfscope}%
\begin{pgfscope}%
\pgfsys@transformshift{3.054545in}{0.892333in}%
\pgfsys@useobject{currentmarker}{}%
\end{pgfscope}%
\begin{pgfscope}%
\pgfsys@transformshift{3.505455in}{0.996086in}%
\pgfsys@useobject{currentmarker}{}%
\end{pgfscope}%
\begin{pgfscope}%
\pgfsys@transformshift{3.956364in}{1.144532in}%
\pgfsys@useobject{currentmarker}{}%
\end{pgfscope}%
\begin{pgfscope}%
\pgfsys@transformshift{4.407273in}{1.347249in}%
\pgfsys@useobject{currentmarker}{}%
\end{pgfscope}%
\begin{pgfscope}%
\pgfsys@transformshift{4.858182in}{1.557948in}%
\pgfsys@useobject{currentmarker}{}%
\end{pgfscope}%
\begin{pgfscope}%
\pgfsys@transformshift{5.309091in}{1.837283in}%
\pgfsys@useobject{currentmarker}{}%
\end{pgfscope}%
\end{pgfscope}%
\begin{pgfscope}%
\pgfpathrectangle{\pgfqpoint{0.800000in}{0.528000in}}{\pgfqpoint{4.960000in}{3.696000in}}%
\pgfusepath{clip}%
\pgfsetbuttcap%
\pgfsetroundjoin%
\pgfsetlinewidth{1.505625pt}%
\definecolor{currentstroke}{rgb}{0.000000,0.000000,0.000000}%
\pgfsetstrokecolor{currentstroke}%
\pgfsetdash{{5.550000pt}{2.400000pt}}{0.000000pt}%
\pgfpathmoveto{\pgfqpoint{1.250909in}{0.696000in}}%
\pgfpathlineto{\pgfqpoint{1.701818in}{0.707173in}}%
\pgfpathlineto{\pgfqpoint{2.152727in}{0.718347in}}%
\pgfpathlineto{\pgfqpoint{2.603636in}{0.742290in}}%
\pgfpathlineto{\pgfqpoint{3.054545in}{0.771021in}}%
\pgfpathlineto{\pgfqpoint{3.505455in}{0.815715in}}%
\pgfpathlineto{\pgfqpoint{3.956364in}{0.868390in}}%
\pgfpathlineto{\pgfqpoint{4.407273in}{0.938622in}}%
\pgfpathlineto{\pgfqpoint{4.858182in}{1.012048in}}%
\pgfpathlineto{\pgfqpoint{5.309091in}{1.109416in}}%
\pgfusepath{stroke}%
\end{pgfscope}%
\begin{pgfscope}%
\pgfpathrectangle{\pgfqpoint{0.800000in}{0.528000in}}{\pgfqpoint{4.960000in}{3.696000in}}%
\pgfusepath{clip}%
\pgfsetbuttcap%
\pgfsetmiterjoin%
\definecolor{currentfill}{rgb}{0.000000,0.000000,0.000000}%
\pgfsetfillcolor{currentfill}%
\pgfsetlinewidth{1.003750pt}%
\definecolor{currentstroke}{rgb}{0.000000,0.000000,0.000000}%
\pgfsetstrokecolor{currentstroke}%
\pgfsetdash{}{0pt}%
\pgfsys@defobject{currentmarker}{\pgfqpoint{-0.035355in}{-0.058926in}}{\pgfqpoint{0.035355in}{0.058926in}}{%
\pgfpathmoveto{\pgfqpoint{-0.000000in}{-0.058926in}}%
\pgfpathlineto{\pgfqpoint{0.035355in}{0.000000in}}%
\pgfpathlineto{\pgfqpoint{0.000000in}{0.058926in}}%
\pgfpathlineto{\pgfqpoint{-0.035355in}{0.000000in}}%
\pgfpathclose%
\pgfusepath{stroke,fill}%
}%
\begin{pgfscope}%
\pgfsys@transformshift{1.250909in}{0.696000in}%
\pgfsys@useobject{currentmarker}{}%
\end{pgfscope}%
\begin{pgfscope}%
\pgfsys@transformshift{1.701818in}{0.707173in}%
\pgfsys@useobject{currentmarker}{}%
\end{pgfscope}%
\begin{pgfscope}%
\pgfsys@transformshift{2.152727in}{0.718347in}%
\pgfsys@useobject{currentmarker}{}%
\end{pgfscope}%
\begin{pgfscope}%
\pgfsys@transformshift{2.603636in}{0.742290in}%
\pgfsys@useobject{currentmarker}{}%
\end{pgfscope}%
\begin{pgfscope}%
\pgfsys@transformshift{3.054545in}{0.771021in}%
\pgfsys@useobject{currentmarker}{}%
\end{pgfscope}%
\begin{pgfscope}%
\pgfsys@transformshift{3.505455in}{0.815715in}%
\pgfsys@useobject{currentmarker}{}%
\end{pgfscope}%
\begin{pgfscope}%
\pgfsys@transformshift{3.956364in}{0.868390in}%
\pgfsys@useobject{currentmarker}{}%
\end{pgfscope}%
\begin{pgfscope}%
\pgfsys@transformshift{4.407273in}{0.938622in}%
\pgfsys@useobject{currentmarker}{}%
\end{pgfscope}%
\begin{pgfscope}%
\pgfsys@transformshift{4.858182in}{1.012048in}%
\pgfsys@useobject{currentmarker}{}%
\end{pgfscope}%
\begin{pgfscope}%
\pgfsys@transformshift{5.309091in}{1.109416in}%
\pgfsys@useobject{currentmarker}{}%
\end{pgfscope}%
\end{pgfscope}%
\begin{pgfscope}%
\pgfsetrectcap%
\pgfsetmiterjoin%
\pgfsetlinewidth{0.803000pt}%
\definecolor{currentstroke}{rgb}{0.000000,0.000000,0.000000}%
\pgfsetstrokecolor{currentstroke}%
\pgfsetdash{}{0pt}%
\pgfpathmoveto{\pgfqpoint{0.800000in}{0.528000in}}%
\pgfpathlineto{\pgfqpoint{0.800000in}{4.224000in}}%
\pgfusepath{stroke}%
\end{pgfscope}%
\begin{pgfscope}%
\pgfsetrectcap%
\pgfsetmiterjoin%
\pgfsetlinewidth{0.803000pt}%
\definecolor{currentstroke}{rgb}{0.000000,0.000000,0.000000}%
\pgfsetstrokecolor{currentstroke}%
\pgfsetdash{}{0pt}%
\pgfpathmoveto{\pgfqpoint{5.760000in}{0.528000in}}%
\pgfpathlineto{\pgfqpoint{5.760000in}{4.224000in}}%
\pgfusepath{stroke}%
\end{pgfscope}%
\begin{pgfscope}%
\pgfsetrectcap%
\pgfsetmiterjoin%
\pgfsetlinewidth{0.803000pt}%
\definecolor{currentstroke}{rgb}{0.000000,0.000000,0.000000}%
\pgfsetstrokecolor{currentstroke}%
\pgfsetdash{}{0pt}%
\pgfpathmoveto{\pgfqpoint{0.800000in}{0.528000in}}%
\pgfpathlineto{\pgfqpoint{5.760000in}{0.528000in}}%
\pgfusepath{stroke}%
\end{pgfscope}%
\begin{pgfscope}%
\pgfsetrectcap%
\pgfsetmiterjoin%
\pgfsetlinewidth{0.803000pt}%
\definecolor{currentstroke}{rgb}{0.000000,0.000000,0.000000}%
\pgfsetstrokecolor{currentstroke}%
\pgfsetdash{}{0pt}%
\pgfpathmoveto{\pgfqpoint{0.800000in}{4.224000in}}%
\pgfpathlineto{\pgfqpoint{5.760000in}{4.224000in}}%
\pgfusepath{stroke}%
\end{pgfscope}%
\begin{pgfscope}%
\pgfsetbuttcap%
\pgfsetmiterjoin%
\definecolor{currentfill}{rgb}{1.000000,1.000000,1.000000}%
\pgfsetfillcolor{currentfill}%
\pgfsetfillopacity{0.800000}%
\pgfsetlinewidth{1.003750pt}%
\definecolor{currentstroke}{rgb}{0.800000,0.800000,0.800000}%
\pgfsetstrokecolor{currentstroke}%
\pgfsetstrokeopacity{0.800000}%
\pgfsetdash{}{0pt}%
\pgfpathmoveto{\pgfqpoint{0.897222in}{3.236530in}}%
\pgfpathlineto{\pgfqpoint{2.090461in}{3.236530in}}%
\pgfpathquadraticcurveto{\pgfqpoint{2.118239in}{3.236530in}}{\pgfqpoint{2.118239in}{3.264308in}}%
\pgfpathlineto{\pgfqpoint{2.118239in}{4.126778in}}%
\pgfpathquadraticcurveto{\pgfqpoint{2.118239in}{4.154556in}}{\pgfqpoint{2.090461in}{4.154556in}}%
\pgfpathlineto{\pgfqpoint{0.897222in}{4.154556in}}%
\pgfpathquadraticcurveto{\pgfqpoint{0.869444in}{4.154556in}}{\pgfqpoint{0.869444in}{4.126778in}}%
\pgfpathlineto{\pgfqpoint{0.869444in}{3.264308in}}%
\pgfpathquadraticcurveto{\pgfqpoint{0.869444in}{3.236530in}}{\pgfqpoint{0.897222in}{3.236530in}}%
\pgfpathclose%
\pgfusepath{stroke,fill}%
\end{pgfscope}%
\begin{pgfscope}%
\pgfsetbuttcap%
\pgfsetroundjoin%
\pgfsetlinewidth{1.505625pt}%
\definecolor{currentstroke}{rgb}{0.000000,0.000000,0.000000}%
\pgfsetstrokecolor{currentstroke}%
\pgfsetdash{{5.550000pt}{2.400000pt}}{0.000000pt}%
\pgfpathmoveto{\pgfqpoint{0.925000in}{4.042088in}}%
\pgfpathlineto{\pgfqpoint{1.202778in}{4.042088in}}%
\pgfusepath{stroke}%
\end{pgfscope}%
\begin{pgfscope}%
\pgfsetbuttcap%
\pgfsetroundjoin%
\definecolor{currentfill}{rgb}{0.000000,0.000000,0.000000}%
\pgfsetfillcolor{currentfill}%
\pgfsetlinewidth{1.003750pt}%
\definecolor{currentstroke}{rgb}{0.000000,0.000000,0.000000}%
\pgfsetstrokecolor{currentstroke}%
\pgfsetdash{}{0pt}%
\pgfsys@defobject{currentmarker}{\pgfqpoint{-0.041667in}{-0.041667in}}{\pgfqpoint{0.041667in}{0.041667in}}{%
\pgfpathmoveto{\pgfqpoint{0.000000in}{-0.041667in}}%
\pgfpathcurveto{\pgfqpoint{0.011050in}{-0.041667in}}{\pgfqpoint{0.021649in}{-0.037276in}}{\pgfqpoint{0.029463in}{-0.029463in}}%
\pgfpathcurveto{\pgfqpoint{0.037276in}{-0.021649in}}{\pgfqpoint{0.041667in}{-0.011050in}}{\pgfqpoint{0.041667in}{0.000000in}}%
\pgfpathcurveto{\pgfqpoint{0.041667in}{0.011050in}}{\pgfqpoint{0.037276in}{0.021649in}}{\pgfqpoint{0.029463in}{0.029463in}}%
\pgfpathcurveto{\pgfqpoint{0.021649in}{0.037276in}}{\pgfqpoint{0.011050in}{0.041667in}}{\pgfqpoint{0.000000in}{0.041667in}}%
\pgfpathcurveto{\pgfqpoint{-0.011050in}{0.041667in}}{\pgfqpoint{-0.021649in}{0.037276in}}{\pgfqpoint{-0.029463in}{0.029463in}}%
\pgfpathcurveto{\pgfqpoint{-0.037276in}{0.021649in}}{\pgfqpoint{-0.041667in}{0.011050in}}{\pgfqpoint{-0.041667in}{0.000000in}}%
\pgfpathcurveto{\pgfqpoint{-0.041667in}{-0.011050in}}{\pgfqpoint{-0.037276in}{-0.021649in}}{\pgfqpoint{-0.029463in}{-0.029463in}}%
\pgfpathcurveto{\pgfqpoint{-0.021649in}{-0.037276in}}{\pgfqpoint{-0.011050in}{-0.041667in}}{\pgfqpoint{0.000000in}{-0.041667in}}%
\pgfpathclose%
\pgfusepath{stroke,fill}%
}%
\begin{pgfscope}%
\pgfsys@transformshift{1.063889in}{4.042088in}%
\pgfsys@useobject{currentmarker}{}%
\end{pgfscope}%
\end{pgfscope}%
\begin{pgfscope}%
\definecolor{textcolor}{rgb}{0.000000,0.000000,0.000000}%
\pgfsetstrokecolor{textcolor}%
\pgfsetfillcolor{textcolor}%
\pgftext[x=1.313889in,y=3.993477in,left,base]{\color{textcolor}\sffamily\fontsize{10.000000}{12.000000}\selectfont \(\displaystyle \alpha = 0.2\)}%
\end{pgfscope}%
\begin{pgfscope}%
\pgfsetbuttcap%
\pgfsetroundjoin%
\pgfsetlinewidth{1.505625pt}%
\definecolor{currentstroke}{rgb}{0.000000,0.000000,0.000000}%
\pgfsetstrokecolor{currentstroke}%
\pgfsetdash{{5.550000pt}{2.400000pt}}{0.000000pt}%
\pgfpathmoveto{\pgfqpoint{0.925000in}{3.823753in}}%
\pgfpathlineto{\pgfqpoint{1.202778in}{3.823753in}}%
\pgfusepath{stroke}%
\end{pgfscope}%
\begin{pgfscope}%
\pgfsetbuttcap%
\pgfsetmiterjoin%
\definecolor{currentfill}{rgb}{0.000000,0.000000,0.000000}%
\pgfsetfillcolor{currentfill}%
\pgfsetlinewidth{1.003750pt}%
\definecolor{currentstroke}{rgb}{0.000000,0.000000,0.000000}%
\pgfsetstrokecolor{currentstroke}%
\pgfsetdash{}{0pt}%
\pgfsys@defobject{currentmarker}{\pgfqpoint{-0.041667in}{-0.041667in}}{\pgfqpoint{0.041667in}{0.041667in}}{%
\pgfpathmoveto{\pgfqpoint{-0.000000in}{-0.041667in}}%
\pgfpathlineto{\pgfqpoint{0.041667in}{0.041667in}}%
\pgfpathlineto{\pgfqpoint{-0.041667in}{0.041667in}}%
\pgfpathclose%
\pgfusepath{stroke,fill}%
}%
\begin{pgfscope}%
\pgfsys@transformshift{1.063889in}{3.823753in}%
\pgfsys@useobject{currentmarker}{}%
\end{pgfscope}%
\end{pgfscope}%
\begin{pgfscope}%
\definecolor{textcolor}{rgb}{0.000000,0.000000,0.000000}%
\pgfsetstrokecolor{textcolor}%
\pgfsetfillcolor{textcolor}%
\pgftext[x=1.313889in,y=3.775142in,left,base]{\color{textcolor}\sffamily\fontsize{10.000000}{12.000000}\selectfont \(\displaystyle \alpha = 0.2/k^{0.1}\)}%
\end{pgfscope}%
\begin{pgfscope}%
\pgfsetbuttcap%
\pgfsetroundjoin%
\pgfsetlinewidth{1.505625pt}%
\definecolor{currentstroke}{rgb}{0.000000,0.000000,0.000000}%
\pgfsetstrokecolor{currentstroke}%
\pgfsetdash{{5.550000pt}{2.400000pt}}{0.000000pt}%
\pgfpathmoveto{\pgfqpoint{0.925000in}{3.599586in}}%
\pgfpathlineto{\pgfqpoint{1.202778in}{3.599586in}}%
\pgfusepath{stroke}%
\end{pgfscope}%
\begin{pgfscope}%
\pgfsetbuttcap%
\pgfsetmiterjoin%
\definecolor{currentfill}{rgb}{0.000000,0.000000,0.000000}%
\pgfsetfillcolor{currentfill}%
\pgfsetlinewidth{1.003750pt}%
\definecolor{currentstroke}{rgb}{0.000000,0.000000,0.000000}%
\pgfsetstrokecolor{currentstroke}%
\pgfsetdash{}{0pt}%
\pgfsys@defobject{currentmarker}{\pgfqpoint{-0.041667in}{-0.041667in}}{\pgfqpoint{0.041667in}{0.041667in}}{%
\pgfpathmoveto{\pgfqpoint{-0.041667in}{-0.041667in}}%
\pgfpathlineto{\pgfqpoint{0.041667in}{-0.041667in}}%
\pgfpathlineto{\pgfqpoint{0.041667in}{0.041667in}}%
\pgfpathlineto{\pgfqpoint{-0.041667in}{0.041667in}}%
\pgfpathclose%
\pgfusepath{stroke,fill}%
}%
\begin{pgfscope}%
\pgfsys@transformshift{1.063889in}{3.599586in}%
\pgfsys@useobject{currentmarker}{}%
\end{pgfscope}%
\end{pgfscope}%
\begin{pgfscope}%
\definecolor{textcolor}{rgb}{0.000000,0.000000,0.000000}%
\pgfsetstrokecolor{textcolor}%
\pgfsetfillcolor{textcolor}%
\pgftext[x=1.313889in,y=3.550975in,left,base]{\color{textcolor}\sffamily\fontsize{10.000000}{12.000000}\selectfont \(\displaystyle \alpha = 0.2/k^{0.2}\)}%
\end{pgfscope}%
\begin{pgfscope}%
\pgfsetbuttcap%
\pgfsetroundjoin%
\pgfsetlinewidth{1.505625pt}%
\definecolor{currentstroke}{rgb}{0.000000,0.000000,0.000000}%
\pgfsetstrokecolor{currentstroke}%
\pgfsetdash{{5.550000pt}{2.400000pt}}{0.000000pt}%
\pgfpathmoveto{\pgfqpoint{0.925000in}{3.375419in}}%
\pgfpathlineto{\pgfqpoint{1.202778in}{3.375419in}}%
\pgfusepath{stroke}%
\end{pgfscope}%
\begin{pgfscope}%
\pgfsetbuttcap%
\pgfsetmiterjoin%
\definecolor{currentfill}{rgb}{0.000000,0.000000,0.000000}%
\pgfsetfillcolor{currentfill}%
\pgfsetlinewidth{1.003750pt}%
\definecolor{currentstroke}{rgb}{0.000000,0.000000,0.000000}%
\pgfsetstrokecolor{currentstroke}%
\pgfsetdash{}{0pt}%
\pgfsys@defobject{currentmarker}{\pgfqpoint{-0.035355in}{-0.058926in}}{\pgfqpoint{0.035355in}{0.058926in}}{%
\pgfpathmoveto{\pgfqpoint{-0.000000in}{-0.058926in}}%
\pgfpathlineto{\pgfqpoint{0.035355in}{0.000000in}}%
\pgfpathlineto{\pgfqpoint{0.000000in}{0.058926in}}%
\pgfpathlineto{\pgfqpoint{-0.035355in}{0.000000in}}%
\pgfpathclose%
\pgfusepath{stroke,fill}%
}%
\begin{pgfscope}%
\pgfsys@transformshift{1.063889in}{3.375419in}%
\pgfsys@useobject{currentmarker}{}%
\end{pgfscope}%
\end{pgfscope}%
\begin{pgfscope}%
\definecolor{textcolor}{rgb}{0.000000,0.000000,0.000000}%
\pgfsetstrokecolor{textcolor}%
\pgfsetfillcolor{textcolor}%
\pgftext[x=1.313889in,y=3.326808in,left,base]{\color{textcolor}\sffamily\fontsize{10.000000}{12.000000}\selectfont \(\displaystyle \alpha = 0.2/k^{0.3}\)}%
\end{pgfscope}%
\end{pgfpicture}%
\makeatother%
\endgroup%

  \caption[GMRES iteration counts when $\NLqDRR{\nso-\nst} = 0.2\times k^{-\beta},$ for any $1 \leq q < \infty$ and $\beta = 0,0.1,0.2,0.3$.]{GMRES iteration counts for $\AmatoI\Amatt$ given by \cref{eq:noweak,eq:ntweak}, where $\alpha = 0.2\times k^{-\beta},$ for $\beta = 0,0.1,0.2,0.3.$}\label{fig:l1low}
\end{figure}

\begin{figure}
  %% Creator: Matplotlib, PGF backend
%%
%% To include the figure in your LaTeX document, write
%%   \input{<filename>.pgf}
%%
%% Make sure the required packages are loaded in your preamble
%%   \usepackage{pgf}
%%
%% Figures using additional raster images can only be included by \input if
%% they are in the same directory as the main LaTeX file. For loading figures
%% from other directories you can use the `import` package
%%   \usepackage{import}
%% and then include the figures with
%%   \import{<path to file>}{<filename>.pgf}
%%
%% Matplotlib used the following preamble
%%   \usepackage{fontspec}
%%   \setmainfont{DejaVuSerif.ttf}[Path=/home/owen/progs/firedrake-complex/firedrake/lib/python3.5/site-packages/matplotlib/mpl-data/fonts/ttf/]
%%   \setsansfont{DejaVuSans.ttf}[Path=/home/owen/progs/firedrake-complex/firedrake/lib/python3.5/site-packages/matplotlib/mpl-data/fonts/ttf/]
%%   \setmonofont{DejaVuSansMono.ttf}[Path=/home/owen/progs/firedrake-complex/firedrake/lib/python3.5/site-packages/matplotlib/mpl-data/fonts/ttf/]
%%
\begingroup%
\makeatletter%
\begin{pgfpicture}%
\pgfpathrectangle{\pgfpointorigin}{\pgfqpoint{6.400000in}{4.800000in}}%
\pgfusepath{use as bounding box, clip}%
\begin{pgfscope}%
\pgfsetbuttcap%
\pgfsetmiterjoin%
\definecolor{currentfill}{rgb}{1.000000,1.000000,1.000000}%
\pgfsetfillcolor{currentfill}%
\pgfsetlinewidth{0.000000pt}%
\definecolor{currentstroke}{rgb}{1.000000,1.000000,1.000000}%
\pgfsetstrokecolor{currentstroke}%
\pgfsetdash{}{0pt}%
\pgfpathmoveto{\pgfqpoint{0.000000in}{0.000000in}}%
\pgfpathlineto{\pgfqpoint{6.400000in}{0.000000in}}%
\pgfpathlineto{\pgfqpoint{6.400000in}{4.800000in}}%
\pgfpathlineto{\pgfqpoint{0.000000in}{4.800000in}}%
\pgfpathclose%
\pgfusepath{fill}%
\end{pgfscope}%
\begin{pgfscope}%
\pgfsetbuttcap%
\pgfsetmiterjoin%
\definecolor{currentfill}{rgb}{1.000000,1.000000,1.000000}%
\pgfsetfillcolor{currentfill}%
\pgfsetlinewidth{0.000000pt}%
\definecolor{currentstroke}{rgb}{0.000000,0.000000,0.000000}%
\pgfsetstrokecolor{currentstroke}%
\pgfsetstrokeopacity{0.000000}%
\pgfsetdash{}{0pt}%
\pgfpathmoveto{\pgfqpoint{0.800000in}{0.528000in}}%
\pgfpathlineto{\pgfqpoint{5.760000in}{0.528000in}}%
\pgfpathlineto{\pgfqpoint{5.760000in}{4.224000in}}%
\pgfpathlineto{\pgfqpoint{0.800000in}{4.224000in}}%
\pgfpathclose%
\pgfusepath{fill}%
\end{pgfscope}%
\begin{pgfscope}%
\pgfsetbuttcap%
\pgfsetroundjoin%
\definecolor{currentfill}{rgb}{0.000000,0.000000,0.000000}%
\pgfsetfillcolor{currentfill}%
\pgfsetlinewidth{0.803000pt}%
\definecolor{currentstroke}{rgb}{0.000000,0.000000,0.000000}%
\pgfsetstrokecolor{currentstroke}%
\pgfsetdash{}{0pt}%
\pgfsys@defobject{currentmarker}{\pgfqpoint{0.000000in}{-0.048611in}}{\pgfqpoint{0.000000in}{0.000000in}}{%
\pgfpathmoveto{\pgfqpoint{0.000000in}{0.000000in}}%
\pgfpathlineto{\pgfqpoint{0.000000in}{-0.048611in}}%
\pgfusepath{stroke,fill}%
}%
\begin{pgfscope}%
\pgfsys@transformshift{1.250909in}{0.528000in}%
\pgfsys@useobject{currentmarker}{}%
\end{pgfscope}%
\end{pgfscope}%
\begin{pgfscope}%
\definecolor{textcolor}{rgb}{0.000000,0.000000,0.000000}%
\pgfsetstrokecolor{textcolor}%
\pgfsetfillcolor{textcolor}%
\pgftext[x=1.250909in,y=0.430778in,,top]{\color{textcolor}\sffamily\fontsize{10.000000}{12.000000}\selectfont 10}%
\end{pgfscope}%
\begin{pgfscope}%
\pgfsetbuttcap%
\pgfsetroundjoin%
\definecolor{currentfill}{rgb}{0.000000,0.000000,0.000000}%
\pgfsetfillcolor{currentfill}%
\pgfsetlinewidth{0.803000pt}%
\definecolor{currentstroke}{rgb}{0.000000,0.000000,0.000000}%
\pgfsetstrokecolor{currentstroke}%
\pgfsetdash{}{0pt}%
\pgfsys@defobject{currentmarker}{\pgfqpoint{0.000000in}{-0.048611in}}{\pgfqpoint{0.000000in}{0.000000in}}{%
\pgfpathmoveto{\pgfqpoint{0.000000in}{0.000000in}}%
\pgfpathlineto{\pgfqpoint{0.000000in}{-0.048611in}}%
\pgfusepath{stroke,fill}%
}%
\begin{pgfscope}%
\pgfsys@transformshift{1.701818in}{0.528000in}%
\pgfsys@useobject{currentmarker}{}%
\end{pgfscope}%
\end{pgfscope}%
\begin{pgfscope}%
\definecolor{textcolor}{rgb}{0.000000,0.000000,0.000000}%
\pgfsetstrokecolor{textcolor}%
\pgfsetfillcolor{textcolor}%
\pgftext[x=1.701818in,y=0.430778in,,top]{\color{textcolor}\sffamily\fontsize{10.000000}{12.000000}\selectfont 20}%
\end{pgfscope}%
\begin{pgfscope}%
\pgfsetbuttcap%
\pgfsetroundjoin%
\definecolor{currentfill}{rgb}{0.000000,0.000000,0.000000}%
\pgfsetfillcolor{currentfill}%
\pgfsetlinewidth{0.803000pt}%
\definecolor{currentstroke}{rgb}{0.000000,0.000000,0.000000}%
\pgfsetstrokecolor{currentstroke}%
\pgfsetdash{}{0pt}%
\pgfsys@defobject{currentmarker}{\pgfqpoint{0.000000in}{-0.048611in}}{\pgfqpoint{0.000000in}{0.000000in}}{%
\pgfpathmoveto{\pgfqpoint{0.000000in}{0.000000in}}%
\pgfpathlineto{\pgfqpoint{0.000000in}{-0.048611in}}%
\pgfusepath{stroke,fill}%
}%
\begin{pgfscope}%
\pgfsys@transformshift{2.152727in}{0.528000in}%
\pgfsys@useobject{currentmarker}{}%
\end{pgfscope}%
\end{pgfscope}%
\begin{pgfscope}%
\definecolor{textcolor}{rgb}{0.000000,0.000000,0.000000}%
\pgfsetstrokecolor{textcolor}%
\pgfsetfillcolor{textcolor}%
\pgftext[x=2.152727in,y=0.430778in,,top]{\color{textcolor}\sffamily\fontsize{10.000000}{12.000000}\selectfont 30}%
\end{pgfscope}%
\begin{pgfscope}%
\pgfsetbuttcap%
\pgfsetroundjoin%
\definecolor{currentfill}{rgb}{0.000000,0.000000,0.000000}%
\pgfsetfillcolor{currentfill}%
\pgfsetlinewidth{0.803000pt}%
\definecolor{currentstroke}{rgb}{0.000000,0.000000,0.000000}%
\pgfsetstrokecolor{currentstroke}%
\pgfsetdash{}{0pt}%
\pgfsys@defobject{currentmarker}{\pgfqpoint{0.000000in}{-0.048611in}}{\pgfqpoint{0.000000in}{0.000000in}}{%
\pgfpathmoveto{\pgfqpoint{0.000000in}{0.000000in}}%
\pgfpathlineto{\pgfqpoint{0.000000in}{-0.048611in}}%
\pgfusepath{stroke,fill}%
}%
\begin{pgfscope}%
\pgfsys@transformshift{2.603636in}{0.528000in}%
\pgfsys@useobject{currentmarker}{}%
\end{pgfscope}%
\end{pgfscope}%
\begin{pgfscope}%
\definecolor{textcolor}{rgb}{0.000000,0.000000,0.000000}%
\pgfsetstrokecolor{textcolor}%
\pgfsetfillcolor{textcolor}%
\pgftext[x=2.603636in,y=0.430778in,,top]{\color{textcolor}\sffamily\fontsize{10.000000}{12.000000}\selectfont 40}%
\end{pgfscope}%
\begin{pgfscope}%
\pgfsetbuttcap%
\pgfsetroundjoin%
\definecolor{currentfill}{rgb}{0.000000,0.000000,0.000000}%
\pgfsetfillcolor{currentfill}%
\pgfsetlinewidth{0.803000pt}%
\definecolor{currentstroke}{rgb}{0.000000,0.000000,0.000000}%
\pgfsetstrokecolor{currentstroke}%
\pgfsetdash{}{0pt}%
\pgfsys@defobject{currentmarker}{\pgfqpoint{0.000000in}{-0.048611in}}{\pgfqpoint{0.000000in}{0.000000in}}{%
\pgfpathmoveto{\pgfqpoint{0.000000in}{0.000000in}}%
\pgfpathlineto{\pgfqpoint{0.000000in}{-0.048611in}}%
\pgfusepath{stroke,fill}%
}%
\begin{pgfscope}%
\pgfsys@transformshift{3.054545in}{0.528000in}%
\pgfsys@useobject{currentmarker}{}%
\end{pgfscope}%
\end{pgfscope}%
\begin{pgfscope}%
\definecolor{textcolor}{rgb}{0.000000,0.000000,0.000000}%
\pgfsetstrokecolor{textcolor}%
\pgfsetfillcolor{textcolor}%
\pgftext[x=3.054545in,y=0.430778in,,top]{\color{textcolor}\sffamily\fontsize{10.000000}{12.000000}\selectfont 50}%
\end{pgfscope}%
\begin{pgfscope}%
\pgfsetbuttcap%
\pgfsetroundjoin%
\definecolor{currentfill}{rgb}{0.000000,0.000000,0.000000}%
\pgfsetfillcolor{currentfill}%
\pgfsetlinewidth{0.803000pt}%
\definecolor{currentstroke}{rgb}{0.000000,0.000000,0.000000}%
\pgfsetstrokecolor{currentstroke}%
\pgfsetdash{}{0pt}%
\pgfsys@defobject{currentmarker}{\pgfqpoint{0.000000in}{-0.048611in}}{\pgfqpoint{0.000000in}{0.000000in}}{%
\pgfpathmoveto{\pgfqpoint{0.000000in}{0.000000in}}%
\pgfpathlineto{\pgfqpoint{0.000000in}{-0.048611in}}%
\pgfusepath{stroke,fill}%
}%
\begin{pgfscope}%
\pgfsys@transformshift{3.505455in}{0.528000in}%
\pgfsys@useobject{currentmarker}{}%
\end{pgfscope}%
\end{pgfscope}%
\begin{pgfscope}%
\definecolor{textcolor}{rgb}{0.000000,0.000000,0.000000}%
\pgfsetstrokecolor{textcolor}%
\pgfsetfillcolor{textcolor}%
\pgftext[x=3.505455in,y=0.430778in,,top]{\color{textcolor}\sffamily\fontsize{10.000000}{12.000000}\selectfont 60}%
\end{pgfscope}%
\begin{pgfscope}%
\pgfsetbuttcap%
\pgfsetroundjoin%
\definecolor{currentfill}{rgb}{0.000000,0.000000,0.000000}%
\pgfsetfillcolor{currentfill}%
\pgfsetlinewidth{0.803000pt}%
\definecolor{currentstroke}{rgb}{0.000000,0.000000,0.000000}%
\pgfsetstrokecolor{currentstroke}%
\pgfsetdash{}{0pt}%
\pgfsys@defobject{currentmarker}{\pgfqpoint{0.000000in}{-0.048611in}}{\pgfqpoint{0.000000in}{0.000000in}}{%
\pgfpathmoveto{\pgfqpoint{0.000000in}{0.000000in}}%
\pgfpathlineto{\pgfqpoint{0.000000in}{-0.048611in}}%
\pgfusepath{stroke,fill}%
}%
\begin{pgfscope}%
\pgfsys@transformshift{3.956364in}{0.528000in}%
\pgfsys@useobject{currentmarker}{}%
\end{pgfscope}%
\end{pgfscope}%
\begin{pgfscope}%
\definecolor{textcolor}{rgb}{0.000000,0.000000,0.000000}%
\pgfsetstrokecolor{textcolor}%
\pgfsetfillcolor{textcolor}%
\pgftext[x=3.956364in,y=0.430778in,,top]{\color{textcolor}\sffamily\fontsize{10.000000}{12.000000}\selectfont 70}%
\end{pgfscope}%
\begin{pgfscope}%
\pgfsetbuttcap%
\pgfsetroundjoin%
\definecolor{currentfill}{rgb}{0.000000,0.000000,0.000000}%
\pgfsetfillcolor{currentfill}%
\pgfsetlinewidth{0.803000pt}%
\definecolor{currentstroke}{rgb}{0.000000,0.000000,0.000000}%
\pgfsetstrokecolor{currentstroke}%
\pgfsetdash{}{0pt}%
\pgfsys@defobject{currentmarker}{\pgfqpoint{0.000000in}{-0.048611in}}{\pgfqpoint{0.000000in}{0.000000in}}{%
\pgfpathmoveto{\pgfqpoint{0.000000in}{0.000000in}}%
\pgfpathlineto{\pgfqpoint{0.000000in}{-0.048611in}}%
\pgfusepath{stroke,fill}%
}%
\begin{pgfscope}%
\pgfsys@transformshift{4.407273in}{0.528000in}%
\pgfsys@useobject{currentmarker}{}%
\end{pgfscope}%
\end{pgfscope}%
\begin{pgfscope}%
\definecolor{textcolor}{rgb}{0.000000,0.000000,0.000000}%
\pgfsetstrokecolor{textcolor}%
\pgfsetfillcolor{textcolor}%
\pgftext[x=4.407273in,y=0.430778in,,top]{\color{textcolor}\sffamily\fontsize{10.000000}{12.000000}\selectfont 80}%
\end{pgfscope}%
\begin{pgfscope}%
\pgfsetbuttcap%
\pgfsetroundjoin%
\definecolor{currentfill}{rgb}{0.000000,0.000000,0.000000}%
\pgfsetfillcolor{currentfill}%
\pgfsetlinewidth{0.803000pt}%
\definecolor{currentstroke}{rgb}{0.000000,0.000000,0.000000}%
\pgfsetstrokecolor{currentstroke}%
\pgfsetdash{}{0pt}%
\pgfsys@defobject{currentmarker}{\pgfqpoint{0.000000in}{-0.048611in}}{\pgfqpoint{0.000000in}{0.000000in}}{%
\pgfpathmoveto{\pgfqpoint{0.000000in}{0.000000in}}%
\pgfpathlineto{\pgfqpoint{0.000000in}{-0.048611in}}%
\pgfusepath{stroke,fill}%
}%
\begin{pgfscope}%
\pgfsys@transformshift{4.858182in}{0.528000in}%
\pgfsys@useobject{currentmarker}{}%
\end{pgfscope}%
\end{pgfscope}%
\begin{pgfscope}%
\definecolor{textcolor}{rgb}{0.000000,0.000000,0.000000}%
\pgfsetstrokecolor{textcolor}%
\pgfsetfillcolor{textcolor}%
\pgftext[x=4.858182in,y=0.430778in,,top]{\color{textcolor}\sffamily\fontsize{10.000000}{12.000000}\selectfont 90}%
\end{pgfscope}%
\begin{pgfscope}%
\pgfsetbuttcap%
\pgfsetroundjoin%
\definecolor{currentfill}{rgb}{0.000000,0.000000,0.000000}%
\pgfsetfillcolor{currentfill}%
\pgfsetlinewidth{0.803000pt}%
\definecolor{currentstroke}{rgb}{0.000000,0.000000,0.000000}%
\pgfsetstrokecolor{currentstroke}%
\pgfsetdash{}{0pt}%
\pgfsys@defobject{currentmarker}{\pgfqpoint{0.000000in}{-0.048611in}}{\pgfqpoint{0.000000in}{0.000000in}}{%
\pgfpathmoveto{\pgfqpoint{0.000000in}{0.000000in}}%
\pgfpathlineto{\pgfqpoint{0.000000in}{-0.048611in}}%
\pgfusepath{stroke,fill}%
}%
\begin{pgfscope}%
\pgfsys@transformshift{5.309091in}{0.528000in}%
\pgfsys@useobject{currentmarker}{}%
\end{pgfscope}%
\end{pgfscope}%
\begin{pgfscope}%
\definecolor{textcolor}{rgb}{0.000000,0.000000,0.000000}%
\pgfsetstrokecolor{textcolor}%
\pgfsetfillcolor{textcolor}%
\pgftext[x=5.309091in,y=0.430778in,,top]{\color{textcolor}\sffamily\fontsize{10.000000}{12.000000}\selectfont 100}%
\end{pgfscope}%
\begin{pgfscope}%
\definecolor{textcolor}{rgb}{0.000000,0.000000,0.000000}%
\pgfsetstrokecolor{textcolor}%
\pgfsetfillcolor{textcolor}%
\pgftext[x=3.280000in,y=0.240809in,,top]{\color{textcolor}\sffamily\fontsize{10.000000}{12.000000}\selectfont \(\displaystyle k\)}%
\end{pgfscope}%
\begin{pgfscope}%
\pgfsetbuttcap%
\pgfsetroundjoin%
\definecolor{currentfill}{rgb}{0.000000,0.000000,0.000000}%
\pgfsetfillcolor{currentfill}%
\pgfsetlinewidth{0.803000pt}%
\definecolor{currentstroke}{rgb}{0.000000,0.000000,0.000000}%
\pgfsetstrokecolor{currentstroke}%
\pgfsetdash{}{0pt}%
\pgfsys@defobject{currentmarker}{\pgfqpoint{-0.048611in}{0.000000in}}{\pgfqpoint{0.000000in}{0.000000in}}{%
\pgfpathmoveto{\pgfqpoint{0.000000in}{0.000000in}}%
\pgfpathlineto{\pgfqpoint{-0.048611in}{0.000000in}}%
\pgfusepath{stroke,fill}%
}%
\begin{pgfscope}%
\pgfsys@transformshift{0.800000in}{0.770667in}%
\pgfsys@useobject{currentmarker}{}%
\end{pgfscope}%
\end{pgfscope}%
\begin{pgfscope}%
\definecolor{textcolor}{rgb}{0.000000,0.000000,0.000000}%
\pgfsetstrokecolor{textcolor}%
\pgfsetfillcolor{textcolor}%
\pgftext[x=0.526047in,y=0.717905in,left,base]{\color{textcolor}\sffamily\fontsize{10.000000}{12.000000}\selectfont 10}%
\end{pgfscope}%
\begin{pgfscope}%
\pgfsetbuttcap%
\pgfsetroundjoin%
\definecolor{currentfill}{rgb}{0.000000,0.000000,0.000000}%
\pgfsetfillcolor{currentfill}%
\pgfsetlinewidth{0.803000pt}%
\definecolor{currentstroke}{rgb}{0.000000,0.000000,0.000000}%
\pgfsetstrokecolor{currentstroke}%
\pgfsetdash{}{0pt}%
\pgfsys@defobject{currentmarker}{\pgfqpoint{-0.048611in}{0.000000in}}{\pgfqpoint{0.000000in}{0.000000in}}{%
\pgfpathmoveto{\pgfqpoint{0.000000in}{0.000000in}}%
\pgfpathlineto{\pgfqpoint{-0.048611in}{0.000000in}}%
\pgfusepath{stroke,fill}%
}%
\begin{pgfscope}%
\pgfsys@transformshift{0.800000in}{1.144000in}%
\pgfsys@useobject{currentmarker}{}%
\end{pgfscope}%
\end{pgfscope}%
\begin{pgfscope}%
\definecolor{textcolor}{rgb}{0.000000,0.000000,0.000000}%
\pgfsetstrokecolor{textcolor}%
\pgfsetfillcolor{textcolor}%
\pgftext[x=0.526047in,y=1.091238in,left,base]{\color{textcolor}\sffamily\fontsize{10.000000}{12.000000}\selectfont 20}%
\end{pgfscope}%
\begin{pgfscope}%
\pgfsetbuttcap%
\pgfsetroundjoin%
\definecolor{currentfill}{rgb}{0.000000,0.000000,0.000000}%
\pgfsetfillcolor{currentfill}%
\pgfsetlinewidth{0.803000pt}%
\definecolor{currentstroke}{rgb}{0.000000,0.000000,0.000000}%
\pgfsetstrokecolor{currentstroke}%
\pgfsetdash{}{0pt}%
\pgfsys@defobject{currentmarker}{\pgfqpoint{-0.048611in}{0.000000in}}{\pgfqpoint{0.000000in}{0.000000in}}{%
\pgfpathmoveto{\pgfqpoint{0.000000in}{0.000000in}}%
\pgfpathlineto{\pgfqpoint{-0.048611in}{0.000000in}}%
\pgfusepath{stroke,fill}%
}%
\begin{pgfscope}%
\pgfsys@transformshift{0.800000in}{1.517333in}%
\pgfsys@useobject{currentmarker}{}%
\end{pgfscope}%
\end{pgfscope}%
\begin{pgfscope}%
\definecolor{textcolor}{rgb}{0.000000,0.000000,0.000000}%
\pgfsetstrokecolor{textcolor}%
\pgfsetfillcolor{textcolor}%
\pgftext[x=0.526047in,y=1.464572in,left,base]{\color{textcolor}\sffamily\fontsize{10.000000}{12.000000}\selectfont 30}%
\end{pgfscope}%
\begin{pgfscope}%
\pgfsetbuttcap%
\pgfsetroundjoin%
\definecolor{currentfill}{rgb}{0.000000,0.000000,0.000000}%
\pgfsetfillcolor{currentfill}%
\pgfsetlinewidth{0.803000pt}%
\definecolor{currentstroke}{rgb}{0.000000,0.000000,0.000000}%
\pgfsetstrokecolor{currentstroke}%
\pgfsetdash{}{0pt}%
\pgfsys@defobject{currentmarker}{\pgfqpoint{-0.048611in}{0.000000in}}{\pgfqpoint{0.000000in}{0.000000in}}{%
\pgfpathmoveto{\pgfqpoint{0.000000in}{0.000000in}}%
\pgfpathlineto{\pgfqpoint{-0.048611in}{0.000000in}}%
\pgfusepath{stroke,fill}%
}%
\begin{pgfscope}%
\pgfsys@transformshift{0.800000in}{1.890667in}%
\pgfsys@useobject{currentmarker}{}%
\end{pgfscope}%
\end{pgfscope}%
\begin{pgfscope}%
\definecolor{textcolor}{rgb}{0.000000,0.000000,0.000000}%
\pgfsetstrokecolor{textcolor}%
\pgfsetfillcolor{textcolor}%
\pgftext[x=0.526047in,y=1.837905in,left,base]{\color{textcolor}\sffamily\fontsize{10.000000}{12.000000}\selectfont 40}%
\end{pgfscope}%
\begin{pgfscope}%
\pgfsetbuttcap%
\pgfsetroundjoin%
\definecolor{currentfill}{rgb}{0.000000,0.000000,0.000000}%
\pgfsetfillcolor{currentfill}%
\pgfsetlinewidth{0.803000pt}%
\definecolor{currentstroke}{rgb}{0.000000,0.000000,0.000000}%
\pgfsetstrokecolor{currentstroke}%
\pgfsetdash{}{0pt}%
\pgfsys@defobject{currentmarker}{\pgfqpoint{-0.048611in}{0.000000in}}{\pgfqpoint{0.000000in}{0.000000in}}{%
\pgfpathmoveto{\pgfqpoint{0.000000in}{0.000000in}}%
\pgfpathlineto{\pgfqpoint{-0.048611in}{0.000000in}}%
\pgfusepath{stroke,fill}%
}%
\begin{pgfscope}%
\pgfsys@transformshift{0.800000in}{2.264000in}%
\pgfsys@useobject{currentmarker}{}%
\end{pgfscope}%
\end{pgfscope}%
\begin{pgfscope}%
\definecolor{textcolor}{rgb}{0.000000,0.000000,0.000000}%
\pgfsetstrokecolor{textcolor}%
\pgfsetfillcolor{textcolor}%
\pgftext[x=0.526047in,y=2.211238in,left,base]{\color{textcolor}\sffamily\fontsize{10.000000}{12.000000}\selectfont 50}%
\end{pgfscope}%
\begin{pgfscope}%
\pgfsetbuttcap%
\pgfsetroundjoin%
\definecolor{currentfill}{rgb}{0.000000,0.000000,0.000000}%
\pgfsetfillcolor{currentfill}%
\pgfsetlinewidth{0.803000pt}%
\definecolor{currentstroke}{rgb}{0.000000,0.000000,0.000000}%
\pgfsetstrokecolor{currentstroke}%
\pgfsetdash{}{0pt}%
\pgfsys@defobject{currentmarker}{\pgfqpoint{-0.048611in}{0.000000in}}{\pgfqpoint{0.000000in}{0.000000in}}{%
\pgfpathmoveto{\pgfqpoint{0.000000in}{0.000000in}}%
\pgfpathlineto{\pgfqpoint{-0.048611in}{0.000000in}}%
\pgfusepath{stroke,fill}%
}%
\begin{pgfscope}%
\pgfsys@transformshift{0.800000in}{2.637333in}%
\pgfsys@useobject{currentmarker}{}%
\end{pgfscope}%
\end{pgfscope}%
\begin{pgfscope}%
\definecolor{textcolor}{rgb}{0.000000,0.000000,0.000000}%
\pgfsetstrokecolor{textcolor}%
\pgfsetfillcolor{textcolor}%
\pgftext[x=0.526047in,y=2.584572in,left,base]{\color{textcolor}\sffamily\fontsize{10.000000}{12.000000}\selectfont 60}%
\end{pgfscope}%
\begin{pgfscope}%
\pgfsetbuttcap%
\pgfsetroundjoin%
\definecolor{currentfill}{rgb}{0.000000,0.000000,0.000000}%
\pgfsetfillcolor{currentfill}%
\pgfsetlinewidth{0.803000pt}%
\definecolor{currentstroke}{rgb}{0.000000,0.000000,0.000000}%
\pgfsetstrokecolor{currentstroke}%
\pgfsetdash{}{0pt}%
\pgfsys@defobject{currentmarker}{\pgfqpoint{-0.048611in}{0.000000in}}{\pgfqpoint{0.000000in}{0.000000in}}{%
\pgfpathmoveto{\pgfqpoint{0.000000in}{0.000000in}}%
\pgfpathlineto{\pgfqpoint{-0.048611in}{0.000000in}}%
\pgfusepath{stroke,fill}%
}%
\begin{pgfscope}%
\pgfsys@transformshift{0.800000in}{3.010667in}%
\pgfsys@useobject{currentmarker}{}%
\end{pgfscope}%
\end{pgfscope}%
\begin{pgfscope}%
\definecolor{textcolor}{rgb}{0.000000,0.000000,0.000000}%
\pgfsetstrokecolor{textcolor}%
\pgfsetfillcolor{textcolor}%
\pgftext[x=0.526047in,y=2.957905in,left,base]{\color{textcolor}\sffamily\fontsize{10.000000}{12.000000}\selectfont 70}%
\end{pgfscope}%
\begin{pgfscope}%
\pgfsetbuttcap%
\pgfsetroundjoin%
\definecolor{currentfill}{rgb}{0.000000,0.000000,0.000000}%
\pgfsetfillcolor{currentfill}%
\pgfsetlinewidth{0.803000pt}%
\definecolor{currentstroke}{rgb}{0.000000,0.000000,0.000000}%
\pgfsetstrokecolor{currentstroke}%
\pgfsetdash{}{0pt}%
\pgfsys@defobject{currentmarker}{\pgfqpoint{-0.048611in}{0.000000in}}{\pgfqpoint{0.000000in}{0.000000in}}{%
\pgfpathmoveto{\pgfqpoint{0.000000in}{0.000000in}}%
\pgfpathlineto{\pgfqpoint{-0.048611in}{0.000000in}}%
\pgfusepath{stroke,fill}%
}%
\begin{pgfscope}%
\pgfsys@transformshift{0.800000in}{3.384000in}%
\pgfsys@useobject{currentmarker}{}%
\end{pgfscope}%
\end{pgfscope}%
\begin{pgfscope}%
\definecolor{textcolor}{rgb}{0.000000,0.000000,0.000000}%
\pgfsetstrokecolor{textcolor}%
\pgfsetfillcolor{textcolor}%
\pgftext[x=0.526047in,y=3.331238in,left,base]{\color{textcolor}\sffamily\fontsize{10.000000}{12.000000}\selectfont 80}%
\end{pgfscope}%
\begin{pgfscope}%
\pgfsetbuttcap%
\pgfsetroundjoin%
\definecolor{currentfill}{rgb}{0.000000,0.000000,0.000000}%
\pgfsetfillcolor{currentfill}%
\pgfsetlinewidth{0.803000pt}%
\definecolor{currentstroke}{rgb}{0.000000,0.000000,0.000000}%
\pgfsetstrokecolor{currentstroke}%
\pgfsetdash{}{0pt}%
\pgfsys@defobject{currentmarker}{\pgfqpoint{-0.048611in}{0.000000in}}{\pgfqpoint{0.000000in}{0.000000in}}{%
\pgfpathmoveto{\pgfqpoint{0.000000in}{0.000000in}}%
\pgfpathlineto{\pgfqpoint{-0.048611in}{0.000000in}}%
\pgfusepath{stroke,fill}%
}%
\begin{pgfscope}%
\pgfsys@transformshift{0.800000in}{3.757333in}%
\pgfsys@useobject{currentmarker}{}%
\end{pgfscope}%
\end{pgfscope}%
\begin{pgfscope}%
\definecolor{textcolor}{rgb}{0.000000,0.000000,0.000000}%
\pgfsetstrokecolor{textcolor}%
\pgfsetfillcolor{textcolor}%
\pgftext[x=0.526047in,y=3.704572in,left,base]{\color{textcolor}\sffamily\fontsize{10.000000}{12.000000}\selectfont 90}%
\end{pgfscope}%
\begin{pgfscope}%
\pgfsetbuttcap%
\pgfsetroundjoin%
\definecolor{currentfill}{rgb}{0.000000,0.000000,0.000000}%
\pgfsetfillcolor{currentfill}%
\pgfsetlinewidth{0.803000pt}%
\definecolor{currentstroke}{rgb}{0.000000,0.000000,0.000000}%
\pgfsetstrokecolor{currentstroke}%
\pgfsetdash{}{0pt}%
\pgfsys@defobject{currentmarker}{\pgfqpoint{-0.048611in}{0.000000in}}{\pgfqpoint{0.000000in}{0.000000in}}{%
\pgfpathmoveto{\pgfqpoint{0.000000in}{0.000000in}}%
\pgfpathlineto{\pgfqpoint{-0.048611in}{0.000000in}}%
\pgfusepath{stroke,fill}%
}%
\begin{pgfscope}%
\pgfsys@transformshift{0.800000in}{4.130667in}%
\pgfsys@useobject{currentmarker}{}%
\end{pgfscope}%
\end{pgfscope}%
\begin{pgfscope}%
\definecolor{textcolor}{rgb}{0.000000,0.000000,0.000000}%
\pgfsetstrokecolor{textcolor}%
\pgfsetfillcolor{textcolor}%
\pgftext[x=0.437682in,y=4.077905in,left,base]{\color{textcolor}\sffamily\fontsize{10.000000}{12.000000}\selectfont 100}%
\end{pgfscope}%
\begin{pgfscope}%
\definecolor{textcolor}{rgb}{0.000000,0.000000,0.000000}%
\pgfsetstrokecolor{textcolor}%
\pgfsetfillcolor{textcolor}%
\pgftext[x=0.382126in,y=2.376000in,,bottom,rotate=90.000000]{\color{textcolor}\sffamily\fontsize{10.000000}{12.000000}\selectfont Number of GMRES iterations}%
\end{pgfscope}%
\begin{pgfscope}%
\pgfpathrectangle{\pgfqpoint{0.800000in}{0.528000in}}{\pgfqpoint{4.960000in}{3.696000in}}%
\pgfusepath{clip}%
\pgfsetbuttcap%
\pgfsetroundjoin%
\pgfsetlinewidth{1.505625pt}%
\definecolor{currentstroke}{rgb}{0.000000,0.000000,0.000000}%
\pgfsetstrokecolor{currentstroke}%
\pgfsetdash{{5.550000pt}{2.400000pt}}{0.000000pt}%
\pgfpathmoveto{\pgfqpoint{1.250909in}{0.770667in}}%
\pgfpathlineto{\pgfqpoint{1.701818in}{0.957333in}}%
\pgfpathlineto{\pgfqpoint{2.152727in}{1.144000in}}%
\pgfpathlineto{\pgfqpoint{2.603636in}{1.330667in}}%
\pgfpathlineto{\pgfqpoint{3.054545in}{1.517333in}}%
\pgfpathlineto{\pgfqpoint{3.505455in}{1.965333in}}%
\pgfpathlineto{\pgfqpoint{3.956364in}{2.376000in}}%
\pgfpathlineto{\pgfqpoint{4.407273in}{2.786667in}}%
\pgfpathlineto{\pgfqpoint{4.858182in}{3.421333in}}%
\pgfpathlineto{\pgfqpoint{5.309091in}{4.056000in}}%
\pgfusepath{stroke}%
\end{pgfscope}%
\begin{pgfscope}%
\pgfpathrectangle{\pgfqpoint{0.800000in}{0.528000in}}{\pgfqpoint{4.960000in}{3.696000in}}%
\pgfusepath{clip}%
\pgfsetbuttcap%
\pgfsetroundjoin%
\definecolor{currentfill}{rgb}{0.000000,0.000000,0.000000}%
\pgfsetfillcolor{currentfill}%
\pgfsetlinewidth{1.003750pt}%
\definecolor{currentstroke}{rgb}{0.000000,0.000000,0.000000}%
\pgfsetstrokecolor{currentstroke}%
\pgfsetdash{}{0pt}%
\pgfsys@defobject{currentmarker}{\pgfqpoint{-0.041667in}{-0.041667in}}{\pgfqpoint{0.041667in}{0.041667in}}{%
\pgfpathmoveto{\pgfqpoint{0.000000in}{-0.041667in}}%
\pgfpathcurveto{\pgfqpoint{0.011050in}{-0.041667in}}{\pgfqpoint{0.021649in}{-0.037276in}}{\pgfqpoint{0.029463in}{-0.029463in}}%
\pgfpathcurveto{\pgfqpoint{0.037276in}{-0.021649in}}{\pgfqpoint{0.041667in}{-0.011050in}}{\pgfqpoint{0.041667in}{0.000000in}}%
\pgfpathcurveto{\pgfqpoint{0.041667in}{0.011050in}}{\pgfqpoint{0.037276in}{0.021649in}}{\pgfqpoint{0.029463in}{0.029463in}}%
\pgfpathcurveto{\pgfqpoint{0.021649in}{0.037276in}}{\pgfqpoint{0.011050in}{0.041667in}}{\pgfqpoint{0.000000in}{0.041667in}}%
\pgfpathcurveto{\pgfqpoint{-0.011050in}{0.041667in}}{\pgfqpoint{-0.021649in}{0.037276in}}{\pgfqpoint{-0.029463in}{0.029463in}}%
\pgfpathcurveto{\pgfqpoint{-0.037276in}{0.021649in}}{\pgfqpoint{-0.041667in}{0.011050in}}{\pgfqpoint{-0.041667in}{0.000000in}}%
\pgfpathcurveto{\pgfqpoint{-0.041667in}{-0.011050in}}{\pgfqpoint{-0.037276in}{-0.021649in}}{\pgfqpoint{-0.029463in}{-0.029463in}}%
\pgfpathcurveto{\pgfqpoint{-0.021649in}{-0.037276in}}{\pgfqpoint{-0.011050in}{-0.041667in}}{\pgfqpoint{0.000000in}{-0.041667in}}%
\pgfpathclose%
\pgfusepath{stroke,fill}%
}%
\begin{pgfscope}%
\pgfsys@transformshift{1.250909in}{0.770667in}%
\pgfsys@useobject{currentmarker}{}%
\end{pgfscope}%
\begin{pgfscope}%
\pgfsys@transformshift{1.701818in}{0.957333in}%
\pgfsys@useobject{currentmarker}{}%
\end{pgfscope}%
\begin{pgfscope}%
\pgfsys@transformshift{2.152727in}{1.144000in}%
\pgfsys@useobject{currentmarker}{}%
\end{pgfscope}%
\begin{pgfscope}%
\pgfsys@transformshift{2.603636in}{1.330667in}%
\pgfsys@useobject{currentmarker}{}%
\end{pgfscope}%
\begin{pgfscope}%
\pgfsys@transformshift{3.054545in}{1.517333in}%
\pgfsys@useobject{currentmarker}{}%
\end{pgfscope}%
\begin{pgfscope}%
\pgfsys@transformshift{3.505455in}{1.965333in}%
\pgfsys@useobject{currentmarker}{}%
\end{pgfscope}%
\begin{pgfscope}%
\pgfsys@transformshift{3.956364in}{2.376000in}%
\pgfsys@useobject{currentmarker}{}%
\end{pgfscope}%
\begin{pgfscope}%
\pgfsys@transformshift{4.407273in}{2.786667in}%
\pgfsys@useobject{currentmarker}{}%
\end{pgfscope}%
\begin{pgfscope}%
\pgfsys@transformshift{4.858182in}{3.421333in}%
\pgfsys@useobject{currentmarker}{}%
\end{pgfscope}%
\begin{pgfscope}%
\pgfsys@transformshift{5.309091in}{4.056000in}%
\pgfsys@useobject{currentmarker}{}%
\end{pgfscope}%
\end{pgfscope}%
\begin{pgfscope}%
\pgfpathrectangle{\pgfqpoint{0.800000in}{0.528000in}}{\pgfqpoint{4.960000in}{3.696000in}}%
\pgfusepath{clip}%
\pgfsetbuttcap%
\pgfsetroundjoin%
\pgfsetlinewidth{1.505625pt}%
\definecolor{currentstroke}{rgb}{0.000000,0.000000,0.000000}%
\pgfsetstrokecolor{currentstroke}%
\pgfsetdash{{5.550000pt}{2.400000pt}}{0.000000pt}%
\pgfpathmoveto{\pgfqpoint{1.250909in}{0.770667in}}%
\pgfpathlineto{\pgfqpoint{1.701818in}{0.882667in}}%
\pgfpathlineto{\pgfqpoint{2.152727in}{0.994667in}}%
\pgfpathlineto{\pgfqpoint{2.603636in}{1.106667in}}%
\pgfpathlineto{\pgfqpoint{3.054545in}{1.218667in}}%
\pgfpathlineto{\pgfqpoint{3.505455in}{1.330667in}}%
\pgfpathlineto{\pgfqpoint{3.956364in}{1.442667in}}%
\pgfpathlineto{\pgfqpoint{4.407273in}{1.554667in}}%
\pgfpathlineto{\pgfqpoint{4.858182in}{1.778667in}}%
\pgfpathlineto{\pgfqpoint{5.309091in}{1.928000in}}%
\pgfusepath{stroke}%
\end{pgfscope}%
\begin{pgfscope}%
\pgfpathrectangle{\pgfqpoint{0.800000in}{0.528000in}}{\pgfqpoint{4.960000in}{3.696000in}}%
\pgfusepath{clip}%
\pgfsetbuttcap%
\pgfsetmiterjoin%
\definecolor{currentfill}{rgb}{0.000000,0.000000,0.000000}%
\pgfsetfillcolor{currentfill}%
\pgfsetlinewidth{1.003750pt}%
\definecolor{currentstroke}{rgb}{0.000000,0.000000,0.000000}%
\pgfsetstrokecolor{currentstroke}%
\pgfsetdash{}{0pt}%
\pgfsys@defobject{currentmarker}{\pgfqpoint{-0.041667in}{-0.041667in}}{\pgfqpoint{0.041667in}{0.041667in}}{%
\pgfpathmoveto{\pgfqpoint{-0.000000in}{-0.041667in}}%
\pgfpathlineto{\pgfqpoint{0.041667in}{0.041667in}}%
\pgfpathlineto{\pgfqpoint{-0.041667in}{0.041667in}}%
\pgfpathclose%
\pgfusepath{stroke,fill}%
}%
\begin{pgfscope}%
\pgfsys@transformshift{1.250909in}{0.770667in}%
\pgfsys@useobject{currentmarker}{}%
\end{pgfscope}%
\begin{pgfscope}%
\pgfsys@transformshift{1.701818in}{0.882667in}%
\pgfsys@useobject{currentmarker}{}%
\end{pgfscope}%
\begin{pgfscope}%
\pgfsys@transformshift{2.152727in}{0.994667in}%
\pgfsys@useobject{currentmarker}{}%
\end{pgfscope}%
\begin{pgfscope}%
\pgfsys@transformshift{2.603636in}{1.106667in}%
\pgfsys@useobject{currentmarker}{}%
\end{pgfscope}%
\begin{pgfscope}%
\pgfsys@transformshift{3.054545in}{1.218667in}%
\pgfsys@useobject{currentmarker}{}%
\end{pgfscope}%
\begin{pgfscope}%
\pgfsys@transformshift{3.505455in}{1.330667in}%
\pgfsys@useobject{currentmarker}{}%
\end{pgfscope}%
\begin{pgfscope}%
\pgfsys@transformshift{3.956364in}{1.442667in}%
\pgfsys@useobject{currentmarker}{}%
\end{pgfscope}%
\begin{pgfscope}%
\pgfsys@transformshift{4.407273in}{1.554667in}%
\pgfsys@useobject{currentmarker}{}%
\end{pgfscope}%
\begin{pgfscope}%
\pgfsys@transformshift{4.858182in}{1.778667in}%
\pgfsys@useobject{currentmarker}{}%
\end{pgfscope}%
\begin{pgfscope}%
\pgfsys@transformshift{5.309091in}{1.928000in}%
\pgfsys@useobject{currentmarker}{}%
\end{pgfscope}%
\end{pgfscope}%
\begin{pgfscope}%
\pgfpathrectangle{\pgfqpoint{0.800000in}{0.528000in}}{\pgfqpoint{4.960000in}{3.696000in}}%
\pgfusepath{clip}%
\pgfsetbuttcap%
\pgfsetroundjoin%
\pgfsetlinewidth{1.505625pt}%
\definecolor{currentstroke}{rgb}{0.000000,0.000000,0.000000}%
\pgfsetstrokecolor{currentstroke}%
\pgfsetdash{{5.550000pt}{2.400000pt}}{0.000000pt}%
\pgfpathmoveto{\pgfqpoint{1.250909in}{0.733333in}}%
\pgfpathlineto{\pgfqpoint{1.701818in}{0.808000in}}%
\pgfpathlineto{\pgfqpoint{2.152727in}{0.882667in}}%
\pgfpathlineto{\pgfqpoint{2.603636in}{0.920000in}}%
\pgfpathlineto{\pgfqpoint{3.054545in}{0.994667in}}%
\pgfpathlineto{\pgfqpoint{3.505455in}{1.032000in}}%
\pgfpathlineto{\pgfqpoint{3.956364in}{1.106667in}}%
\pgfpathlineto{\pgfqpoint{4.407273in}{1.106667in}}%
\pgfpathlineto{\pgfqpoint{4.858182in}{1.181333in}}%
\pgfpathlineto{\pgfqpoint{5.309091in}{1.218667in}}%
\pgfusepath{stroke}%
\end{pgfscope}%
\begin{pgfscope}%
\pgfpathrectangle{\pgfqpoint{0.800000in}{0.528000in}}{\pgfqpoint{4.960000in}{3.696000in}}%
\pgfusepath{clip}%
\pgfsetbuttcap%
\pgfsetmiterjoin%
\definecolor{currentfill}{rgb}{0.000000,0.000000,0.000000}%
\pgfsetfillcolor{currentfill}%
\pgfsetlinewidth{1.003750pt}%
\definecolor{currentstroke}{rgb}{0.000000,0.000000,0.000000}%
\pgfsetstrokecolor{currentstroke}%
\pgfsetdash{}{0pt}%
\pgfsys@defobject{currentmarker}{\pgfqpoint{-0.041667in}{-0.041667in}}{\pgfqpoint{0.041667in}{0.041667in}}{%
\pgfpathmoveto{\pgfqpoint{-0.041667in}{-0.041667in}}%
\pgfpathlineto{\pgfqpoint{0.041667in}{-0.041667in}}%
\pgfpathlineto{\pgfqpoint{0.041667in}{0.041667in}}%
\pgfpathlineto{\pgfqpoint{-0.041667in}{0.041667in}}%
\pgfpathclose%
\pgfusepath{stroke,fill}%
}%
\begin{pgfscope}%
\pgfsys@transformshift{1.250909in}{0.733333in}%
\pgfsys@useobject{currentmarker}{}%
\end{pgfscope}%
\begin{pgfscope}%
\pgfsys@transformshift{1.701818in}{0.808000in}%
\pgfsys@useobject{currentmarker}{}%
\end{pgfscope}%
\begin{pgfscope}%
\pgfsys@transformshift{2.152727in}{0.882667in}%
\pgfsys@useobject{currentmarker}{}%
\end{pgfscope}%
\begin{pgfscope}%
\pgfsys@transformshift{2.603636in}{0.920000in}%
\pgfsys@useobject{currentmarker}{}%
\end{pgfscope}%
\begin{pgfscope}%
\pgfsys@transformshift{3.054545in}{0.994667in}%
\pgfsys@useobject{currentmarker}{}%
\end{pgfscope}%
\begin{pgfscope}%
\pgfsys@transformshift{3.505455in}{1.032000in}%
\pgfsys@useobject{currentmarker}{}%
\end{pgfscope}%
\begin{pgfscope}%
\pgfsys@transformshift{3.956364in}{1.106667in}%
\pgfsys@useobject{currentmarker}{}%
\end{pgfscope}%
\begin{pgfscope}%
\pgfsys@transformshift{4.407273in}{1.106667in}%
\pgfsys@useobject{currentmarker}{}%
\end{pgfscope}%
\begin{pgfscope}%
\pgfsys@transformshift{4.858182in}{1.181333in}%
\pgfsys@useobject{currentmarker}{}%
\end{pgfscope}%
\begin{pgfscope}%
\pgfsys@transformshift{5.309091in}{1.218667in}%
\pgfsys@useobject{currentmarker}{}%
\end{pgfscope}%
\end{pgfscope}%
\begin{pgfscope}%
\pgfpathrectangle{\pgfqpoint{0.800000in}{0.528000in}}{\pgfqpoint{4.960000in}{3.696000in}}%
\pgfusepath{clip}%
\pgfsetbuttcap%
\pgfsetroundjoin%
\pgfsetlinewidth{1.505625pt}%
\definecolor{currentstroke}{rgb}{0.000000,0.000000,0.000000}%
\pgfsetstrokecolor{currentstroke}%
\pgfsetdash{{5.550000pt}{2.400000pt}}{0.000000pt}%
\pgfpathmoveto{\pgfqpoint{1.250909in}{0.696000in}}%
\pgfpathlineto{\pgfqpoint{1.701818in}{0.733333in}}%
\pgfpathlineto{\pgfqpoint{2.152727in}{0.770667in}}%
\pgfpathlineto{\pgfqpoint{2.603636in}{0.808000in}}%
\pgfpathlineto{\pgfqpoint{3.054545in}{0.845333in}}%
\pgfpathlineto{\pgfqpoint{3.505455in}{0.882667in}}%
\pgfpathlineto{\pgfqpoint{3.956364in}{0.882667in}}%
\pgfpathlineto{\pgfqpoint{4.407273in}{0.920000in}}%
\pgfpathlineto{\pgfqpoint{4.858182in}{0.920000in}}%
\pgfpathlineto{\pgfqpoint{5.309091in}{0.920000in}}%
\pgfusepath{stroke}%
\end{pgfscope}%
\begin{pgfscope}%
\pgfpathrectangle{\pgfqpoint{0.800000in}{0.528000in}}{\pgfqpoint{4.960000in}{3.696000in}}%
\pgfusepath{clip}%
\pgfsetbuttcap%
\pgfsetmiterjoin%
\definecolor{currentfill}{rgb}{0.000000,0.000000,0.000000}%
\pgfsetfillcolor{currentfill}%
\pgfsetlinewidth{1.003750pt}%
\definecolor{currentstroke}{rgb}{0.000000,0.000000,0.000000}%
\pgfsetstrokecolor{currentstroke}%
\pgfsetdash{}{0pt}%
\pgfsys@defobject{currentmarker}{\pgfqpoint{-0.035355in}{-0.058926in}}{\pgfqpoint{0.035355in}{0.058926in}}{%
\pgfpathmoveto{\pgfqpoint{-0.000000in}{-0.058926in}}%
\pgfpathlineto{\pgfqpoint{0.035355in}{0.000000in}}%
\pgfpathlineto{\pgfqpoint{0.000000in}{0.058926in}}%
\pgfpathlineto{\pgfqpoint{-0.035355in}{0.000000in}}%
\pgfpathclose%
\pgfusepath{stroke,fill}%
}%
\begin{pgfscope}%
\pgfsys@transformshift{1.250909in}{0.696000in}%
\pgfsys@useobject{currentmarker}{}%
\end{pgfscope}%
\begin{pgfscope}%
\pgfsys@transformshift{1.701818in}{0.733333in}%
\pgfsys@useobject{currentmarker}{}%
\end{pgfscope}%
\begin{pgfscope}%
\pgfsys@transformshift{2.152727in}{0.770667in}%
\pgfsys@useobject{currentmarker}{}%
\end{pgfscope}%
\begin{pgfscope}%
\pgfsys@transformshift{2.603636in}{0.808000in}%
\pgfsys@useobject{currentmarker}{}%
\end{pgfscope}%
\begin{pgfscope}%
\pgfsys@transformshift{3.054545in}{0.845333in}%
\pgfsys@useobject{currentmarker}{}%
\end{pgfscope}%
\begin{pgfscope}%
\pgfsys@transformshift{3.505455in}{0.882667in}%
\pgfsys@useobject{currentmarker}{}%
\end{pgfscope}%
\begin{pgfscope}%
\pgfsys@transformshift{3.956364in}{0.882667in}%
\pgfsys@useobject{currentmarker}{}%
\end{pgfscope}%
\begin{pgfscope}%
\pgfsys@transformshift{4.407273in}{0.920000in}%
\pgfsys@useobject{currentmarker}{}%
\end{pgfscope}%
\begin{pgfscope}%
\pgfsys@transformshift{4.858182in}{0.920000in}%
\pgfsys@useobject{currentmarker}{}%
\end{pgfscope}%
\begin{pgfscope}%
\pgfsys@transformshift{5.309091in}{0.920000in}%
\pgfsys@useobject{currentmarker}{}%
\end{pgfscope}%
\end{pgfscope}%
\begin{pgfscope}%
\pgfsetrectcap%
\pgfsetmiterjoin%
\pgfsetlinewidth{0.803000pt}%
\definecolor{currentstroke}{rgb}{0.000000,0.000000,0.000000}%
\pgfsetstrokecolor{currentstroke}%
\pgfsetdash{}{0pt}%
\pgfpathmoveto{\pgfqpoint{0.800000in}{0.528000in}}%
\pgfpathlineto{\pgfqpoint{0.800000in}{4.224000in}}%
\pgfusepath{stroke}%
\end{pgfscope}%
\begin{pgfscope}%
\pgfsetrectcap%
\pgfsetmiterjoin%
\pgfsetlinewidth{0.803000pt}%
\definecolor{currentstroke}{rgb}{0.000000,0.000000,0.000000}%
\pgfsetstrokecolor{currentstroke}%
\pgfsetdash{}{0pt}%
\pgfpathmoveto{\pgfqpoint{5.760000in}{0.528000in}}%
\pgfpathlineto{\pgfqpoint{5.760000in}{4.224000in}}%
\pgfusepath{stroke}%
\end{pgfscope}%
\begin{pgfscope}%
\pgfsetrectcap%
\pgfsetmiterjoin%
\pgfsetlinewidth{0.803000pt}%
\definecolor{currentstroke}{rgb}{0.000000,0.000000,0.000000}%
\pgfsetstrokecolor{currentstroke}%
\pgfsetdash{}{0pt}%
\pgfpathmoveto{\pgfqpoint{0.800000in}{0.528000in}}%
\pgfpathlineto{\pgfqpoint{5.760000in}{0.528000in}}%
\pgfusepath{stroke}%
\end{pgfscope}%
\begin{pgfscope}%
\pgfsetrectcap%
\pgfsetmiterjoin%
\pgfsetlinewidth{0.803000pt}%
\definecolor{currentstroke}{rgb}{0.000000,0.000000,0.000000}%
\pgfsetstrokecolor{currentstroke}%
\pgfsetdash{}{0pt}%
\pgfpathmoveto{\pgfqpoint{0.800000in}{4.224000in}}%
\pgfpathlineto{\pgfqpoint{5.760000in}{4.224000in}}%
\pgfusepath{stroke}%
\end{pgfscope}%
\begin{pgfscope}%
\pgfsetbuttcap%
\pgfsetmiterjoin%
\definecolor{currentfill}{rgb}{1.000000,1.000000,1.000000}%
\pgfsetfillcolor{currentfill}%
\pgfsetfillopacity{0.800000}%
\pgfsetlinewidth{1.003750pt}%
\definecolor{currentstroke}{rgb}{0.800000,0.800000,0.800000}%
\pgfsetstrokecolor{currentstroke}%
\pgfsetstrokeopacity{0.800000}%
\pgfsetdash{}{0pt}%
\pgfpathmoveto{\pgfqpoint{0.897222in}{3.297460in}}%
\pgfpathlineto{\pgfqpoint{1.790209in}{3.297460in}}%
\pgfpathquadraticcurveto{\pgfqpoint{1.817987in}{3.297460in}}{\pgfqpoint{1.817987in}{3.325238in}}%
\pgfpathlineto{\pgfqpoint{1.817987in}{4.126778in}}%
\pgfpathquadraticcurveto{\pgfqpoint{1.817987in}{4.154556in}}{\pgfqpoint{1.790209in}{4.154556in}}%
\pgfpathlineto{\pgfqpoint{0.897222in}{4.154556in}}%
\pgfpathquadraticcurveto{\pgfqpoint{0.869444in}{4.154556in}}{\pgfqpoint{0.869444in}{4.126778in}}%
\pgfpathlineto{\pgfqpoint{0.869444in}{3.325238in}}%
\pgfpathquadraticcurveto{\pgfqpoint{0.869444in}{3.297460in}}{\pgfqpoint{0.897222in}{3.297460in}}%
\pgfpathclose%
\pgfusepath{stroke,fill}%
\end{pgfscope}%
\begin{pgfscope}%
\pgfsetbuttcap%
\pgfsetroundjoin%
\pgfsetlinewidth{1.505625pt}%
\definecolor{currentstroke}{rgb}{0.000000,0.000000,0.000000}%
\pgfsetstrokecolor{currentstroke}%
\pgfsetdash{{5.550000pt}{2.400000pt}}{0.000000pt}%
\pgfpathmoveto{\pgfqpoint{0.925000in}{4.042088in}}%
\pgfpathlineto{\pgfqpoint{1.202778in}{4.042088in}}%
\pgfusepath{stroke}%
\end{pgfscope}%
\begin{pgfscope}%
\pgfsetbuttcap%
\pgfsetroundjoin%
\definecolor{currentfill}{rgb}{0.000000,0.000000,0.000000}%
\pgfsetfillcolor{currentfill}%
\pgfsetlinewidth{1.003750pt}%
\definecolor{currentstroke}{rgb}{0.000000,0.000000,0.000000}%
\pgfsetstrokecolor{currentstroke}%
\pgfsetdash{}{0pt}%
\pgfsys@defobject{currentmarker}{\pgfqpoint{-0.041667in}{-0.041667in}}{\pgfqpoint{0.041667in}{0.041667in}}{%
\pgfpathmoveto{\pgfqpoint{0.000000in}{-0.041667in}}%
\pgfpathcurveto{\pgfqpoint{0.011050in}{-0.041667in}}{\pgfqpoint{0.021649in}{-0.037276in}}{\pgfqpoint{0.029463in}{-0.029463in}}%
\pgfpathcurveto{\pgfqpoint{0.037276in}{-0.021649in}}{\pgfqpoint{0.041667in}{-0.011050in}}{\pgfqpoint{0.041667in}{0.000000in}}%
\pgfpathcurveto{\pgfqpoint{0.041667in}{0.011050in}}{\pgfqpoint{0.037276in}{0.021649in}}{\pgfqpoint{0.029463in}{0.029463in}}%
\pgfpathcurveto{\pgfqpoint{0.021649in}{0.037276in}}{\pgfqpoint{0.011050in}{0.041667in}}{\pgfqpoint{0.000000in}{0.041667in}}%
\pgfpathcurveto{\pgfqpoint{-0.011050in}{0.041667in}}{\pgfqpoint{-0.021649in}{0.037276in}}{\pgfqpoint{-0.029463in}{0.029463in}}%
\pgfpathcurveto{\pgfqpoint{-0.037276in}{0.021649in}}{\pgfqpoint{-0.041667in}{0.011050in}}{\pgfqpoint{-0.041667in}{0.000000in}}%
\pgfpathcurveto{\pgfqpoint{-0.041667in}{-0.011050in}}{\pgfqpoint{-0.037276in}{-0.021649in}}{\pgfqpoint{-0.029463in}{-0.029463in}}%
\pgfpathcurveto{\pgfqpoint{-0.021649in}{-0.037276in}}{\pgfqpoint{-0.011050in}{-0.041667in}}{\pgfqpoint{0.000000in}{-0.041667in}}%
\pgfpathclose%
\pgfusepath{stroke,fill}%
}%
\begin{pgfscope}%
\pgfsys@transformshift{1.063889in}{4.042088in}%
\pgfsys@useobject{currentmarker}{}%
\end{pgfscope}%
\end{pgfscope}%
\begin{pgfscope}%
\definecolor{textcolor}{rgb}{0.000000,0.000000,0.000000}%
\pgfsetstrokecolor{textcolor}%
\pgfsetfillcolor{textcolor}%
\pgftext[x=1.313889in,y=3.993477in,left,base]{\color{textcolor}\sffamily\fontsize{10.000000}{12.000000}\selectfont \(\displaystyle \beta = 0.4\)}%
\end{pgfscope}%
\begin{pgfscope}%
\pgfsetbuttcap%
\pgfsetroundjoin%
\pgfsetlinewidth{1.505625pt}%
\definecolor{currentstroke}{rgb}{0.000000,0.000000,0.000000}%
\pgfsetstrokecolor{currentstroke}%
\pgfsetdash{{5.550000pt}{2.400000pt}}{0.000000pt}%
\pgfpathmoveto{\pgfqpoint{0.925000in}{3.838231in}}%
\pgfpathlineto{\pgfqpoint{1.202778in}{3.838231in}}%
\pgfusepath{stroke}%
\end{pgfscope}%
\begin{pgfscope}%
\pgfsetbuttcap%
\pgfsetmiterjoin%
\definecolor{currentfill}{rgb}{0.000000,0.000000,0.000000}%
\pgfsetfillcolor{currentfill}%
\pgfsetlinewidth{1.003750pt}%
\definecolor{currentstroke}{rgb}{0.000000,0.000000,0.000000}%
\pgfsetstrokecolor{currentstroke}%
\pgfsetdash{}{0pt}%
\pgfsys@defobject{currentmarker}{\pgfqpoint{-0.041667in}{-0.041667in}}{\pgfqpoint{0.041667in}{0.041667in}}{%
\pgfpathmoveto{\pgfqpoint{-0.000000in}{-0.041667in}}%
\pgfpathlineto{\pgfqpoint{0.041667in}{0.041667in}}%
\pgfpathlineto{\pgfqpoint{-0.041667in}{0.041667in}}%
\pgfpathclose%
\pgfusepath{stroke,fill}%
}%
\begin{pgfscope}%
\pgfsys@transformshift{1.063889in}{3.838231in}%
\pgfsys@useobject{currentmarker}{}%
\end{pgfscope}%
\end{pgfscope}%
\begin{pgfscope}%
\definecolor{textcolor}{rgb}{0.000000,0.000000,0.000000}%
\pgfsetstrokecolor{textcolor}%
\pgfsetfillcolor{textcolor}%
\pgftext[x=1.313889in,y=3.789620in,left,base]{\color{textcolor}\sffamily\fontsize{10.000000}{12.000000}\selectfont \(\displaystyle \beta = 0.5\)}%
\end{pgfscope}%
\begin{pgfscope}%
\pgfsetbuttcap%
\pgfsetroundjoin%
\pgfsetlinewidth{1.505625pt}%
\definecolor{currentstroke}{rgb}{0.000000,0.000000,0.000000}%
\pgfsetstrokecolor{currentstroke}%
\pgfsetdash{{5.550000pt}{2.400000pt}}{0.000000pt}%
\pgfpathmoveto{\pgfqpoint{0.925000in}{3.634374in}}%
\pgfpathlineto{\pgfqpoint{1.202778in}{3.634374in}}%
\pgfusepath{stroke}%
\end{pgfscope}%
\begin{pgfscope}%
\pgfsetbuttcap%
\pgfsetmiterjoin%
\definecolor{currentfill}{rgb}{0.000000,0.000000,0.000000}%
\pgfsetfillcolor{currentfill}%
\pgfsetlinewidth{1.003750pt}%
\definecolor{currentstroke}{rgb}{0.000000,0.000000,0.000000}%
\pgfsetstrokecolor{currentstroke}%
\pgfsetdash{}{0pt}%
\pgfsys@defobject{currentmarker}{\pgfqpoint{-0.041667in}{-0.041667in}}{\pgfqpoint{0.041667in}{0.041667in}}{%
\pgfpathmoveto{\pgfqpoint{-0.041667in}{-0.041667in}}%
\pgfpathlineto{\pgfqpoint{0.041667in}{-0.041667in}}%
\pgfpathlineto{\pgfqpoint{0.041667in}{0.041667in}}%
\pgfpathlineto{\pgfqpoint{-0.041667in}{0.041667in}}%
\pgfpathclose%
\pgfusepath{stroke,fill}%
}%
\begin{pgfscope}%
\pgfsys@transformshift{1.063889in}{3.634374in}%
\pgfsys@useobject{currentmarker}{}%
\end{pgfscope}%
\end{pgfscope}%
\begin{pgfscope}%
\definecolor{textcolor}{rgb}{0.000000,0.000000,0.000000}%
\pgfsetstrokecolor{textcolor}%
\pgfsetfillcolor{textcolor}%
\pgftext[x=1.313889in,y=3.585762in,left,base]{\color{textcolor}\sffamily\fontsize{10.000000}{12.000000}\selectfont \(\displaystyle \beta = 0.6\)}%
\end{pgfscope}%
\begin{pgfscope}%
\pgfsetbuttcap%
\pgfsetroundjoin%
\pgfsetlinewidth{1.505625pt}%
\definecolor{currentstroke}{rgb}{0.000000,0.000000,0.000000}%
\pgfsetstrokecolor{currentstroke}%
\pgfsetdash{{5.550000pt}{2.400000pt}}{0.000000pt}%
\pgfpathmoveto{\pgfqpoint{0.925000in}{3.430516in}}%
\pgfpathlineto{\pgfqpoint{1.202778in}{3.430516in}}%
\pgfusepath{stroke}%
\end{pgfscope}%
\begin{pgfscope}%
\pgfsetbuttcap%
\pgfsetmiterjoin%
\definecolor{currentfill}{rgb}{0.000000,0.000000,0.000000}%
\pgfsetfillcolor{currentfill}%
\pgfsetlinewidth{1.003750pt}%
\definecolor{currentstroke}{rgb}{0.000000,0.000000,0.000000}%
\pgfsetstrokecolor{currentstroke}%
\pgfsetdash{}{0pt}%
\pgfsys@defobject{currentmarker}{\pgfqpoint{-0.035355in}{-0.058926in}}{\pgfqpoint{0.035355in}{0.058926in}}{%
\pgfpathmoveto{\pgfqpoint{-0.000000in}{-0.058926in}}%
\pgfpathlineto{\pgfqpoint{0.035355in}{0.000000in}}%
\pgfpathlineto{\pgfqpoint{0.000000in}{0.058926in}}%
\pgfpathlineto{\pgfqpoint{-0.035355in}{0.000000in}}%
\pgfpathclose%
\pgfusepath{stroke,fill}%
}%
\begin{pgfscope}%
\pgfsys@transformshift{1.063889in}{3.430516in}%
\pgfsys@useobject{currentmarker}{}%
\end{pgfscope}%
\end{pgfscope}%
\begin{pgfscope}%
\definecolor{textcolor}{rgb}{0.000000,0.000000,0.000000}%
\pgfsetstrokecolor{textcolor}%
\pgfsetfillcolor{textcolor}%
\pgftext[x=1.313889in,y=3.381905in,left,base]{\color{textcolor}\sffamily\fontsize{10.000000}{12.000000}\selectfont \(\displaystyle \beta = 0.7\)}%
\end{pgfscope}%
\end{pgfpicture}%
\makeatother%
\endgroup%

    \caption[GMRES iteration counts when $\NLqDRR{\nso-\nst} = 0.2\times k^{-\beta},$ for any $1 \leq q < \infty$ and $\beta = 0.4,0.5,0.6,0.7$.]{GMRES iteration counts for $\AmatoI\Amatt$ given by \cref{eq:noweak,eq:ntweak}, where $\alpha = 0.2\times k^{-\beta},$ for $\beta = 0.4,0.5,0.6,0.7.$}\label{fig:l1med}
\end{figure}
    
    \begin{figure}
    %% Creator: Matplotlib, PGF backend
%%
%% To include the figure in your LaTeX document, write
%%   \input{<filename>.pgf}
%%
%% Make sure the required packages are loaded in your preamble
%%   \usepackage{pgf}
%%
%% Figures using additional raster images can only be included by \input if
%% they are in the same directory as the main LaTeX file. For loading figures
%% from other directories you can use the `import` package
%%   \usepackage{import}
%% and then include the figures with
%%   \import{<path to file>}{<filename>.pgf}
%%
%% Matplotlib used the following preamble
%%   \usepackage{fontspec}
%%   \setmainfont{DejaVuSerif.ttf}[Path=/home/owen/progs/firedrake-complex/firedrake/lib/python3.5/site-packages/matplotlib/mpl-data/fonts/ttf/]
%%   \setsansfont{DejaVuSans.ttf}[Path=/home/owen/progs/firedrake-complex/firedrake/lib/python3.5/site-packages/matplotlib/mpl-data/fonts/ttf/]
%%   \setmonofont{DejaVuSansMono.ttf}[Path=/home/owen/progs/firedrake-complex/firedrake/lib/python3.5/site-packages/matplotlib/mpl-data/fonts/ttf/]
%%
\begingroup%
\makeatletter%
\begin{pgfpicture}%
\pgfpathrectangle{\pgfpointorigin}{\pgfqpoint{5.500000in}{5.500000in}}%
\pgfusepath{use as bounding box, clip}%
\begin{pgfscope}%
\pgfsetbuttcap%
\pgfsetmiterjoin%
\definecolor{currentfill}{rgb}{1.000000,1.000000,1.000000}%
\pgfsetfillcolor{currentfill}%
\pgfsetlinewidth{0.000000pt}%
\definecolor{currentstroke}{rgb}{1.000000,1.000000,1.000000}%
\pgfsetstrokecolor{currentstroke}%
\pgfsetdash{}{0pt}%
\pgfpathmoveto{\pgfqpoint{0.000000in}{0.000000in}}%
\pgfpathlineto{\pgfqpoint{5.500000in}{0.000000in}}%
\pgfpathlineto{\pgfqpoint{5.500000in}{5.500000in}}%
\pgfpathlineto{\pgfqpoint{0.000000in}{5.500000in}}%
\pgfpathclose%
\pgfusepath{fill}%
\end{pgfscope}%
\begin{pgfscope}%
\pgfsetbuttcap%
\pgfsetmiterjoin%
\definecolor{currentfill}{rgb}{1.000000,1.000000,1.000000}%
\pgfsetfillcolor{currentfill}%
\pgfsetlinewidth{0.000000pt}%
\definecolor{currentstroke}{rgb}{0.000000,0.000000,0.000000}%
\pgfsetstrokecolor{currentstroke}%
\pgfsetstrokeopacity{0.000000}%
\pgfsetdash{}{0pt}%
\pgfpathmoveto{\pgfqpoint{0.687500in}{0.605000in}}%
\pgfpathlineto{\pgfqpoint{4.950000in}{0.605000in}}%
\pgfpathlineto{\pgfqpoint{4.950000in}{4.840000in}}%
\pgfpathlineto{\pgfqpoint{0.687500in}{4.840000in}}%
\pgfpathclose%
\pgfusepath{fill}%
\end{pgfscope}%
\begin{pgfscope}%
\pgfsetbuttcap%
\pgfsetroundjoin%
\definecolor{currentfill}{rgb}{0.000000,0.000000,0.000000}%
\pgfsetfillcolor{currentfill}%
\pgfsetlinewidth{0.803000pt}%
\definecolor{currentstroke}{rgb}{0.000000,0.000000,0.000000}%
\pgfsetstrokecolor{currentstroke}%
\pgfsetdash{}{0pt}%
\pgfsys@defobject{currentmarker}{\pgfqpoint{0.000000in}{-0.048611in}}{\pgfqpoint{0.000000in}{0.000000in}}{%
\pgfpathmoveto{\pgfqpoint{0.000000in}{0.000000in}}%
\pgfpathlineto{\pgfqpoint{0.000000in}{-0.048611in}}%
\pgfusepath{stroke,fill}%
}%
\begin{pgfscope}%
\pgfsys@transformshift{1.075000in}{0.605000in}%
\pgfsys@useobject{currentmarker}{}%
\end{pgfscope}%
\end{pgfscope}%
\begin{pgfscope}%
\definecolor{textcolor}{rgb}{0.000000,0.000000,0.000000}%
\pgfsetstrokecolor{textcolor}%
\pgfsetfillcolor{textcolor}%
\pgftext[x=1.075000in,y=0.507778in,,top]{\color{textcolor}\sffamily\fontsize{10.000000}{12.000000}\selectfont \(\displaystyle 10\)}%
\end{pgfscope}%
\begin{pgfscope}%
\pgfsetbuttcap%
\pgfsetroundjoin%
\definecolor{currentfill}{rgb}{0.000000,0.000000,0.000000}%
\pgfsetfillcolor{currentfill}%
\pgfsetlinewidth{0.803000pt}%
\definecolor{currentstroke}{rgb}{0.000000,0.000000,0.000000}%
\pgfsetstrokecolor{currentstroke}%
\pgfsetdash{}{0pt}%
\pgfsys@defobject{currentmarker}{\pgfqpoint{0.000000in}{-0.048611in}}{\pgfqpoint{0.000000in}{0.000000in}}{%
\pgfpathmoveto{\pgfqpoint{0.000000in}{0.000000in}}%
\pgfpathlineto{\pgfqpoint{0.000000in}{-0.048611in}}%
\pgfusepath{stroke,fill}%
}%
\begin{pgfscope}%
\pgfsys@transformshift{1.462500in}{0.605000in}%
\pgfsys@useobject{currentmarker}{}%
\end{pgfscope}%
\end{pgfscope}%
\begin{pgfscope}%
\definecolor{textcolor}{rgb}{0.000000,0.000000,0.000000}%
\pgfsetstrokecolor{textcolor}%
\pgfsetfillcolor{textcolor}%
\pgftext[x=1.462500in,y=0.507778in,,top]{\color{textcolor}\sffamily\fontsize{10.000000}{12.000000}\selectfont \(\displaystyle 20\)}%
\end{pgfscope}%
\begin{pgfscope}%
\pgfsetbuttcap%
\pgfsetroundjoin%
\definecolor{currentfill}{rgb}{0.000000,0.000000,0.000000}%
\pgfsetfillcolor{currentfill}%
\pgfsetlinewidth{0.803000pt}%
\definecolor{currentstroke}{rgb}{0.000000,0.000000,0.000000}%
\pgfsetstrokecolor{currentstroke}%
\pgfsetdash{}{0pt}%
\pgfsys@defobject{currentmarker}{\pgfqpoint{0.000000in}{-0.048611in}}{\pgfqpoint{0.000000in}{0.000000in}}{%
\pgfpathmoveto{\pgfqpoint{0.000000in}{0.000000in}}%
\pgfpathlineto{\pgfqpoint{0.000000in}{-0.048611in}}%
\pgfusepath{stroke,fill}%
}%
\begin{pgfscope}%
\pgfsys@transformshift{1.850000in}{0.605000in}%
\pgfsys@useobject{currentmarker}{}%
\end{pgfscope}%
\end{pgfscope}%
\begin{pgfscope}%
\definecolor{textcolor}{rgb}{0.000000,0.000000,0.000000}%
\pgfsetstrokecolor{textcolor}%
\pgfsetfillcolor{textcolor}%
\pgftext[x=1.850000in,y=0.507778in,,top]{\color{textcolor}\sffamily\fontsize{10.000000}{12.000000}\selectfont \(\displaystyle 30\)}%
\end{pgfscope}%
\begin{pgfscope}%
\pgfsetbuttcap%
\pgfsetroundjoin%
\definecolor{currentfill}{rgb}{0.000000,0.000000,0.000000}%
\pgfsetfillcolor{currentfill}%
\pgfsetlinewidth{0.803000pt}%
\definecolor{currentstroke}{rgb}{0.000000,0.000000,0.000000}%
\pgfsetstrokecolor{currentstroke}%
\pgfsetdash{}{0pt}%
\pgfsys@defobject{currentmarker}{\pgfqpoint{0.000000in}{-0.048611in}}{\pgfqpoint{0.000000in}{0.000000in}}{%
\pgfpathmoveto{\pgfqpoint{0.000000in}{0.000000in}}%
\pgfpathlineto{\pgfqpoint{0.000000in}{-0.048611in}}%
\pgfusepath{stroke,fill}%
}%
\begin{pgfscope}%
\pgfsys@transformshift{2.237500in}{0.605000in}%
\pgfsys@useobject{currentmarker}{}%
\end{pgfscope}%
\end{pgfscope}%
\begin{pgfscope}%
\definecolor{textcolor}{rgb}{0.000000,0.000000,0.000000}%
\pgfsetstrokecolor{textcolor}%
\pgfsetfillcolor{textcolor}%
\pgftext[x=2.237500in,y=0.507778in,,top]{\color{textcolor}\sffamily\fontsize{10.000000}{12.000000}\selectfont \(\displaystyle 40\)}%
\end{pgfscope}%
\begin{pgfscope}%
\pgfsetbuttcap%
\pgfsetroundjoin%
\definecolor{currentfill}{rgb}{0.000000,0.000000,0.000000}%
\pgfsetfillcolor{currentfill}%
\pgfsetlinewidth{0.803000pt}%
\definecolor{currentstroke}{rgb}{0.000000,0.000000,0.000000}%
\pgfsetstrokecolor{currentstroke}%
\pgfsetdash{}{0pt}%
\pgfsys@defobject{currentmarker}{\pgfqpoint{0.000000in}{-0.048611in}}{\pgfqpoint{0.000000in}{0.000000in}}{%
\pgfpathmoveto{\pgfqpoint{0.000000in}{0.000000in}}%
\pgfpathlineto{\pgfqpoint{0.000000in}{-0.048611in}}%
\pgfusepath{stroke,fill}%
}%
\begin{pgfscope}%
\pgfsys@transformshift{2.625000in}{0.605000in}%
\pgfsys@useobject{currentmarker}{}%
\end{pgfscope}%
\end{pgfscope}%
\begin{pgfscope}%
\definecolor{textcolor}{rgb}{0.000000,0.000000,0.000000}%
\pgfsetstrokecolor{textcolor}%
\pgfsetfillcolor{textcolor}%
\pgftext[x=2.625000in,y=0.507778in,,top]{\color{textcolor}\sffamily\fontsize{10.000000}{12.000000}\selectfont \(\displaystyle 50\)}%
\end{pgfscope}%
\begin{pgfscope}%
\pgfsetbuttcap%
\pgfsetroundjoin%
\definecolor{currentfill}{rgb}{0.000000,0.000000,0.000000}%
\pgfsetfillcolor{currentfill}%
\pgfsetlinewidth{0.803000pt}%
\definecolor{currentstroke}{rgb}{0.000000,0.000000,0.000000}%
\pgfsetstrokecolor{currentstroke}%
\pgfsetdash{}{0pt}%
\pgfsys@defobject{currentmarker}{\pgfqpoint{0.000000in}{-0.048611in}}{\pgfqpoint{0.000000in}{0.000000in}}{%
\pgfpathmoveto{\pgfqpoint{0.000000in}{0.000000in}}%
\pgfpathlineto{\pgfqpoint{0.000000in}{-0.048611in}}%
\pgfusepath{stroke,fill}%
}%
\begin{pgfscope}%
\pgfsys@transformshift{3.012500in}{0.605000in}%
\pgfsys@useobject{currentmarker}{}%
\end{pgfscope}%
\end{pgfscope}%
\begin{pgfscope}%
\definecolor{textcolor}{rgb}{0.000000,0.000000,0.000000}%
\pgfsetstrokecolor{textcolor}%
\pgfsetfillcolor{textcolor}%
\pgftext[x=3.012500in,y=0.507778in,,top]{\color{textcolor}\sffamily\fontsize{10.000000}{12.000000}\selectfont \(\displaystyle 60\)}%
\end{pgfscope}%
\begin{pgfscope}%
\pgfsetbuttcap%
\pgfsetroundjoin%
\definecolor{currentfill}{rgb}{0.000000,0.000000,0.000000}%
\pgfsetfillcolor{currentfill}%
\pgfsetlinewidth{0.803000pt}%
\definecolor{currentstroke}{rgb}{0.000000,0.000000,0.000000}%
\pgfsetstrokecolor{currentstroke}%
\pgfsetdash{}{0pt}%
\pgfsys@defobject{currentmarker}{\pgfqpoint{0.000000in}{-0.048611in}}{\pgfqpoint{0.000000in}{0.000000in}}{%
\pgfpathmoveto{\pgfqpoint{0.000000in}{0.000000in}}%
\pgfpathlineto{\pgfqpoint{0.000000in}{-0.048611in}}%
\pgfusepath{stroke,fill}%
}%
\begin{pgfscope}%
\pgfsys@transformshift{3.400000in}{0.605000in}%
\pgfsys@useobject{currentmarker}{}%
\end{pgfscope}%
\end{pgfscope}%
\begin{pgfscope}%
\definecolor{textcolor}{rgb}{0.000000,0.000000,0.000000}%
\pgfsetstrokecolor{textcolor}%
\pgfsetfillcolor{textcolor}%
\pgftext[x=3.400000in,y=0.507778in,,top]{\color{textcolor}\sffamily\fontsize{10.000000}{12.000000}\selectfont \(\displaystyle 70\)}%
\end{pgfscope}%
\begin{pgfscope}%
\pgfsetbuttcap%
\pgfsetroundjoin%
\definecolor{currentfill}{rgb}{0.000000,0.000000,0.000000}%
\pgfsetfillcolor{currentfill}%
\pgfsetlinewidth{0.803000pt}%
\definecolor{currentstroke}{rgb}{0.000000,0.000000,0.000000}%
\pgfsetstrokecolor{currentstroke}%
\pgfsetdash{}{0pt}%
\pgfsys@defobject{currentmarker}{\pgfqpoint{0.000000in}{-0.048611in}}{\pgfqpoint{0.000000in}{0.000000in}}{%
\pgfpathmoveto{\pgfqpoint{0.000000in}{0.000000in}}%
\pgfpathlineto{\pgfqpoint{0.000000in}{-0.048611in}}%
\pgfusepath{stroke,fill}%
}%
\begin{pgfscope}%
\pgfsys@transformshift{3.787500in}{0.605000in}%
\pgfsys@useobject{currentmarker}{}%
\end{pgfscope}%
\end{pgfscope}%
\begin{pgfscope}%
\definecolor{textcolor}{rgb}{0.000000,0.000000,0.000000}%
\pgfsetstrokecolor{textcolor}%
\pgfsetfillcolor{textcolor}%
\pgftext[x=3.787500in,y=0.507778in,,top]{\color{textcolor}\sffamily\fontsize{10.000000}{12.000000}\selectfont \(\displaystyle 80\)}%
\end{pgfscope}%
\begin{pgfscope}%
\pgfsetbuttcap%
\pgfsetroundjoin%
\definecolor{currentfill}{rgb}{0.000000,0.000000,0.000000}%
\pgfsetfillcolor{currentfill}%
\pgfsetlinewidth{0.803000pt}%
\definecolor{currentstroke}{rgb}{0.000000,0.000000,0.000000}%
\pgfsetstrokecolor{currentstroke}%
\pgfsetdash{}{0pt}%
\pgfsys@defobject{currentmarker}{\pgfqpoint{0.000000in}{-0.048611in}}{\pgfqpoint{0.000000in}{0.000000in}}{%
\pgfpathmoveto{\pgfqpoint{0.000000in}{0.000000in}}%
\pgfpathlineto{\pgfqpoint{0.000000in}{-0.048611in}}%
\pgfusepath{stroke,fill}%
}%
\begin{pgfscope}%
\pgfsys@transformshift{4.175000in}{0.605000in}%
\pgfsys@useobject{currentmarker}{}%
\end{pgfscope}%
\end{pgfscope}%
\begin{pgfscope}%
\definecolor{textcolor}{rgb}{0.000000,0.000000,0.000000}%
\pgfsetstrokecolor{textcolor}%
\pgfsetfillcolor{textcolor}%
\pgftext[x=4.175000in,y=0.507778in,,top]{\color{textcolor}\sffamily\fontsize{10.000000}{12.000000}\selectfont \(\displaystyle 90\)}%
\end{pgfscope}%
\begin{pgfscope}%
\pgfsetbuttcap%
\pgfsetroundjoin%
\definecolor{currentfill}{rgb}{0.000000,0.000000,0.000000}%
\pgfsetfillcolor{currentfill}%
\pgfsetlinewidth{0.803000pt}%
\definecolor{currentstroke}{rgb}{0.000000,0.000000,0.000000}%
\pgfsetstrokecolor{currentstroke}%
\pgfsetdash{}{0pt}%
\pgfsys@defobject{currentmarker}{\pgfqpoint{0.000000in}{-0.048611in}}{\pgfqpoint{0.000000in}{0.000000in}}{%
\pgfpathmoveto{\pgfqpoint{0.000000in}{0.000000in}}%
\pgfpathlineto{\pgfqpoint{0.000000in}{-0.048611in}}%
\pgfusepath{stroke,fill}%
}%
\begin{pgfscope}%
\pgfsys@transformshift{4.562500in}{0.605000in}%
\pgfsys@useobject{currentmarker}{}%
\end{pgfscope}%
\end{pgfscope}%
\begin{pgfscope}%
\definecolor{textcolor}{rgb}{0.000000,0.000000,0.000000}%
\pgfsetstrokecolor{textcolor}%
\pgfsetfillcolor{textcolor}%
\pgftext[x=4.562500in,y=0.507778in,,top]{\color{textcolor}\sffamily\fontsize{10.000000}{12.000000}\selectfont \(\displaystyle 100\)}%
\end{pgfscope}%
\begin{pgfscope}%
\definecolor{textcolor}{rgb}{0.000000,0.000000,0.000000}%
\pgfsetstrokecolor{textcolor}%
\pgfsetfillcolor{textcolor}%
\pgftext[x=2.818750in,y=0.317809in,,top]{\color{textcolor}\sffamily\fontsize{10.000000}{12.000000}\selectfont \(\displaystyle k\)}%
\end{pgfscope}%
\begin{pgfscope}%
\pgfsetbuttcap%
\pgfsetroundjoin%
\definecolor{currentfill}{rgb}{0.000000,0.000000,0.000000}%
\pgfsetfillcolor{currentfill}%
\pgfsetlinewidth{0.803000pt}%
\definecolor{currentstroke}{rgb}{0.000000,0.000000,0.000000}%
\pgfsetstrokecolor{currentstroke}%
\pgfsetdash{}{0pt}%
\pgfsys@defobject{currentmarker}{\pgfqpoint{-0.048611in}{0.000000in}}{\pgfqpoint{0.000000in}{0.000000in}}{%
\pgfpathmoveto{\pgfqpoint{0.000000in}{0.000000in}}%
\pgfpathlineto{\pgfqpoint{-0.048611in}{0.000000in}}%
\pgfusepath{stroke,fill}%
}%
\begin{pgfscope}%
\pgfsys@transformshift{0.687500in}{0.797500in}%
\pgfsys@useobject{currentmarker}{}%
\end{pgfscope}%
\end{pgfscope}%
\begin{pgfscope}%
\definecolor{textcolor}{rgb}{0.000000,0.000000,0.000000}%
\pgfsetstrokecolor{textcolor}%
\pgfsetfillcolor{textcolor}%
\pgftext[x=0.520833in,y=0.744738in,left,base]{\color{textcolor}\sffamily\fontsize{10.000000}{12.000000}\selectfont \(\displaystyle 6\)}%
\end{pgfscope}%
\begin{pgfscope}%
\pgfsetbuttcap%
\pgfsetroundjoin%
\definecolor{currentfill}{rgb}{0.000000,0.000000,0.000000}%
\pgfsetfillcolor{currentfill}%
\pgfsetlinewidth{0.803000pt}%
\definecolor{currentstroke}{rgb}{0.000000,0.000000,0.000000}%
\pgfsetstrokecolor{currentstroke}%
\pgfsetdash{}{0pt}%
\pgfsys@defobject{currentmarker}{\pgfqpoint{-0.048611in}{0.000000in}}{\pgfqpoint{0.000000in}{0.000000in}}{%
\pgfpathmoveto{\pgfqpoint{0.000000in}{0.000000in}}%
\pgfpathlineto{\pgfqpoint{-0.048611in}{0.000000in}}%
\pgfusepath{stroke,fill}%
}%
\begin{pgfscope}%
\pgfsys@transformshift{0.687500in}{1.760000in}%
\pgfsys@useobject{currentmarker}{}%
\end{pgfscope}%
\end{pgfscope}%
\begin{pgfscope}%
\definecolor{textcolor}{rgb}{0.000000,0.000000,0.000000}%
\pgfsetstrokecolor{textcolor}%
\pgfsetfillcolor{textcolor}%
\pgftext[x=0.520833in,y=1.707238in,left,base]{\color{textcolor}\sffamily\fontsize{10.000000}{12.000000}\selectfont \(\displaystyle 7\)}%
\end{pgfscope}%
\begin{pgfscope}%
\pgfsetbuttcap%
\pgfsetroundjoin%
\definecolor{currentfill}{rgb}{0.000000,0.000000,0.000000}%
\pgfsetfillcolor{currentfill}%
\pgfsetlinewidth{0.803000pt}%
\definecolor{currentstroke}{rgb}{0.000000,0.000000,0.000000}%
\pgfsetstrokecolor{currentstroke}%
\pgfsetdash{}{0pt}%
\pgfsys@defobject{currentmarker}{\pgfqpoint{-0.048611in}{0.000000in}}{\pgfqpoint{0.000000in}{0.000000in}}{%
\pgfpathmoveto{\pgfqpoint{0.000000in}{0.000000in}}%
\pgfpathlineto{\pgfqpoint{-0.048611in}{0.000000in}}%
\pgfusepath{stroke,fill}%
}%
\begin{pgfscope}%
\pgfsys@transformshift{0.687500in}{2.722500in}%
\pgfsys@useobject{currentmarker}{}%
\end{pgfscope}%
\end{pgfscope}%
\begin{pgfscope}%
\definecolor{textcolor}{rgb}{0.000000,0.000000,0.000000}%
\pgfsetstrokecolor{textcolor}%
\pgfsetfillcolor{textcolor}%
\pgftext[x=0.520833in,y=2.669738in,left,base]{\color{textcolor}\sffamily\fontsize{10.000000}{12.000000}\selectfont \(\displaystyle 8\)}%
\end{pgfscope}%
\begin{pgfscope}%
\pgfsetbuttcap%
\pgfsetroundjoin%
\definecolor{currentfill}{rgb}{0.000000,0.000000,0.000000}%
\pgfsetfillcolor{currentfill}%
\pgfsetlinewidth{0.803000pt}%
\definecolor{currentstroke}{rgb}{0.000000,0.000000,0.000000}%
\pgfsetstrokecolor{currentstroke}%
\pgfsetdash{}{0pt}%
\pgfsys@defobject{currentmarker}{\pgfqpoint{-0.048611in}{0.000000in}}{\pgfqpoint{0.000000in}{0.000000in}}{%
\pgfpathmoveto{\pgfqpoint{0.000000in}{0.000000in}}%
\pgfpathlineto{\pgfqpoint{-0.048611in}{0.000000in}}%
\pgfusepath{stroke,fill}%
}%
\begin{pgfscope}%
\pgfsys@transformshift{0.687500in}{3.685000in}%
\pgfsys@useobject{currentmarker}{}%
\end{pgfscope}%
\end{pgfscope}%
\begin{pgfscope}%
\definecolor{textcolor}{rgb}{0.000000,0.000000,0.000000}%
\pgfsetstrokecolor{textcolor}%
\pgfsetfillcolor{textcolor}%
\pgftext[x=0.520833in,y=3.632238in,left,base]{\color{textcolor}\sffamily\fontsize{10.000000}{12.000000}\selectfont \(\displaystyle 9\)}%
\end{pgfscope}%
\begin{pgfscope}%
\pgfsetbuttcap%
\pgfsetroundjoin%
\definecolor{currentfill}{rgb}{0.000000,0.000000,0.000000}%
\pgfsetfillcolor{currentfill}%
\pgfsetlinewidth{0.803000pt}%
\definecolor{currentstroke}{rgb}{0.000000,0.000000,0.000000}%
\pgfsetstrokecolor{currentstroke}%
\pgfsetdash{}{0pt}%
\pgfsys@defobject{currentmarker}{\pgfqpoint{-0.048611in}{0.000000in}}{\pgfqpoint{0.000000in}{0.000000in}}{%
\pgfpathmoveto{\pgfqpoint{0.000000in}{0.000000in}}%
\pgfpathlineto{\pgfqpoint{-0.048611in}{0.000000in}}%
\pgfusepath{stroke,fill}%
}%
\begin{pgfscope}%
\pgfsys@transformshift{0.687500in}{4.647500in}%
\pgfsys@useobject{currentmarker}{}%
\end{pgfscope}%
\end{pgfscope}%
\begin{pgfscope}%
\definecolor{textcolor}{rgb}{0.000000,0.000000,0.000000}%
\pgfsetstrokecolor{textcolor}%
\pgfsetfillcolor{textcolor}%
\pgftext[x=0.451388in,y=4.594738in,left,base]{\color{textcolor}\sffamily\fontsize{10.000000}{12.000000}\selectfont \(\displaystyle 10\)}%
\end{pgfscope}%
\begin{pgfscope}%
\definecolor{textcolor}{rgb}{0.000000,0.000000,0.000000}%
\pgfsetstrokecolor{textcolor}%
\pgfsetfillcolor{textcolor}%
\pgftext[x=0.395833in,y=2.722500in,,bottom,rotate=90.000000]{\color{textcolor}\sffamily\fontsize{10.000000}{12.000000}\selectfont Number of GMRES iterations}%
\end{pgfscope}%
\begin{pgfscope}%
\pgfpathrectangle{\pgfqpoint{0.687500in}{0.605000in}}{\pgfqpoint{4.262500in}{4.235000in}}%
\pgfusepath{clip}%
\pgfsetbuttcap%
\pgfsetroundjoin%
\pgfsetlinewidth{1.505625pt}%
\definecolor{currentstroke}{rgb}{0.843137,0.000000,0.000000}%
\pgfsetstrokecolor{currentstroke}%
\pgfsetdash{{5.550000pt}{2.400000pt}}{0.000000pt}%
\pgfpathmoveto{\pgfqpoint{1.075000in}{2.722500in}}%
\pgfpathlineto{\pgfqpoint{1.462500in}{2.722500in}}%
\pgfpathlineto{\pgfqpoint{1.850000in}{3.685000in}}%
\pgfpathlineto{\pgfqpoint{2.237500in}{3.685000in}}%
\pgfpathlineto{\pgfqpoint{2.625000in}{4.647500in}}%
\pgfpathlineto{\pgfqpoint{3.012500in}{4.647500in}}%
\pgfpathlineto{\pgfqpoint{3.400000in}{4.647500in}}%
\pgfpathlineto{\pgfqpoint{3.787500in}{4.647500in}}%
\pgfpathlineto{\pgfqpoint{4.175000in}{4.647500in}}%
\pgfpathlineto{\pgfqpoint{4.562500in}{4.647500in}}%
\pgfusepath{stroke}%
\end{pgfscope}%
\begin{pgfscope}%
\pgfpathrectangle{\pgfqpoint{0.687500in}{0.605000in}}{\pgfqpoint{4.262500in}{4.235000in}}%
\pgfusepath{clip}%
\pgfsetbuttcap%
\pgfsetroundjoin%
\definecolor{currentfill}{rgb}{0.843137,0.000000,0.000000}%
\pgfsetfillcolor{currentfill}%
\pgfsetlinewidth{1.003750pt}%
\definecolor{currentstroke}{rgb}{0.843137,0.000000,0.000000}%
\pgfsetstrokecolor{currentstroke}%
\pgfsetdash{}{0pt}%
\pgfsys@defobject{currentmarker}{\pgfqpoint{-0.041667in}{-0.041667in}}{\pgfqpoint{0.041667in}{0.041667in}}{%
\pgfpathmoveto{\pgfqpoint{0.000000in}{-0.041667in}}%
\pgfpathcurveto{\pgfqpoint{0.011050in}{-0.041667in}}{\pgfqpoint{0.021649in}{-0.037276in}}{\pgfqpoint{0.029463in}{-0.029463in}}%
\pgfpathcurveto{\pgfqpoint{0.037276in}{-0.021649in}}{\pgfqpoint{0.041667in}{-0.011050in}}{\pgfqpoint{0.041667in}{0.000000in}}%
\pgfpathcurveto{\pgfqpoint{0.041667in}{0.011050in}}{\pgfqpoint{0.037276in}{0.021649in}}{\pgfqpoint{0.029463in}{0.029463in}}%
\pgfpathcurveto{\pgfqpoint{0.021649in}{0.037276in}}{\pgfqpoint{0.011050in}{0.041667in}}{\pgfqpoint{0.000000in}{0.041667in}}%
\pgfpathcurveto{\pgfqpoint{-0.011050in}{0.041667in}}{\pgfqpoint{-0.021649in}{0.037276in}}{\pgfqpoint{-0.029463in}{0.029463in}}%
\pgfpathcurveto{\pgfqpoint{-0.037276in}{0.021649in}}{\pgfqpoint{-0.041667in}{0.011050in}}{\pgfqpoint{-0.041667in}{0.000000in}}%
\pgfpathcurveto{\pgfqpoint{-0.041667in}{-0.011050in}}{\pgfqpoint{-0.037276in}{-0.021649in}}{\pgfqpoint{-0.029463in}{-0.029463in}}%
\pgfpathcurveto{\pgfqpoint{-0.021649in}{-0.037276in}}{\pgfqpoint{-0.011050in}{-0.041667in}}{\pgfqpoint{0.000000in}{-0.041667in}}%
\pgfpathclose%
\pgfusepath{stroke,fill}%
}%
\begin{pgfscope}%
\pgfsys@transformshift{1.075000in}{2.722500in}%
\pgfsys@useobject{currentmarker}{}%
\end{pgfscope}%
\begin{pgfscope}%
\pgfsys@transformshift{1.462500in}{2.722500in}%
\pgfsys@useobject{currentmarker}{}%
\end{pgfscope}%
\begin{pgfscope}%
\pgfsys@transformshift{1.850000in}{3.685000in}%
\pgfsys@useobject{currentmarker}{}%
\end{pgfscope}%
\begin{pgfscope}%
\pgfsys@transformshift{2.237500in}{3.685000in}%
\pgfsys@useobject{currentmarker}{}%
\end{pgfscope}%
\begin{pgfscope}%
\pgfsys@transformshift{2.625000in}{4.647500in}%
\pgfsys@useobject{currentmarker}{}%
\end{pgfscope}%
\begin{pgfscope}%
\pgfsys@transformshift{3.012500in}{4.647500in}%
\pgfsys@useobject{currentmarker}{}%
\end{pgfscope}%
\begin{pgfscope}%
\pgfsys@transformshift{3.400000in}{4.647500in}%
\pgfsys@useobject{currentmarker}{}%
\end{pgfscope}%
\begin{pgfscope}%
\pgfsys@transformshift{3.787500in}{4.647500in}%
\pgfsys@useobject{currentmarker}{}%
\end{pgfscope}%
\begin{pgfscope}%
\pgfsys@transformshift{4.175000in}{4.647500in}%
\pgfsys@useobject{currentmarker}{}%
\end{pgfscope}%
\begin{pgfscope}%
\pgfsys@transformshift{4.562500in}{4.647500in}%
\pgfsys@useobject{currentmarker}{}%
\end{pgfscope}%
\end{pgfscope}%
\begin{pgfscope}%
\pgfpathrectangle{\pgfqpoint{0.687500in}{0.605000in}}{\pgfqpoint{4.262500in}{4.235000in}}%
\pgfusepath{clip}%
\pgfsetbuttcap%
\pgfsetroundjoin%
\pgfsetlinewidth{1.505625pt}%
\definecolor{currentstroke}{rgb}{0.549020,0.235294,1.000000}%
\pgfsetstrokecolor{currentstroke}%
\pgfsetdash{{5.550000pt}{2.400000pt}}{0.000000pt}%
\pgfpathmoveto{\pgfqpoint{1.075000in}{1.760000in}}%
\pgfpathlineto{\pgfqpoint{1.462500in}{1.760000in}}%
\pgfpathlineto{\pgfqpoint{1.850000in}{2.722500in}}%
\pgfpathlineto{\pgfqpoint{2.237500in}{2.722500in}}%
\pgfpathlineto{\pgfqpoint{2.625000in}{2.722500in}}%
\pgfpathlineto{\pgfqpoint{3.012500in}{2.722500in}}%
\pgfpathlineto{\pgfqpoint{3.400000in}{2.722500in}}%
\pgfpathlineto{\pgfqpoint{3.787500in}{2.722500in}}%
\pgfpathlineto{\pgfqpoint{4.175000in}{2.722500in}}%
\pgfpathlineto{\pgfqpoint{4.562500in}{2.722500in}}%
\pgfusepath{stroke}%
\end{pgfscope}%
\begin{pgfscope}%
\pgfpathrectangle{\pgfqpoint{0.687500in}{0.605000in}}{\pgfqpoint{4.262500in}{4.235000in}}%
\pgfusepath{clip}%
\pgfsetbuttcap%
\pgfsetmiterjoin%
\definecolor{currentfill}{rgb}{0.549020,0.235294,1.000000}%
\pgfsetfillcolor{currentfill}%
\pgfsetlinewidth{1.003750pt}%
\definecolor{currentstroke}{rgb}{0.549020,0.235294,1.000000}%
\pgfsetstrokecolor{currentstroke}%
\pgfsetdash{}{0pt}%
\pgfsys@defobject{currentmarker}{\pgfqpoint{-0.041667in}{-0.041667in}}{\pgfqpoint{0.041667in}{0.041667in}}{%
\pgfpathmoveto{\pgfqpoint{0.000000in}{0.041667in}}%
\pgfpathlineto{\pgfqpoint{-0.041667in}{-0.041667in}}%
\pgfpathlineto{\pgfqpoint{0.041667in}{-0.041667in}}%
\pgfpathclose%
\pgfusepath{stroke,fill}%
}%
\begin{pgfscope}%
\pgfsys@transformshift{1.075000in}{1.760000in}%
\pgfsys@useobject{currentmarker}{}%
\end{pgfscope}%
\begin{pgfscope}%
\pgfsys@transformshift{1.462500in}{1.760000in}%
\pgfsys@useobject{currentmarker}{}%
\end{pgfscope}%
\begin{pgfscope}%
\pgfsys@transformshift{1.850000in}{2.722500in}%
\pgfsys@useobject{currentmarker}{}%
\end{pgfscope}%
\begin{pgfscope}%
\pgfsys@transformshift{2.237500in}{2.722500in}%
\pgfsys@useobject{currentmarker}{}%
\end{pgfscope}%
\begin{pgfscope}%
\pgfsys@transformshift{2.625000in}{2.722500in}%
\pgfsys@useobject{currentmarker}{}%
\end{pgfscope}%
\begin{pgfscope}%
\pgfsys@transformshift{3.012500in}{2.722500in}%
\pgfsys@useobject{currentmarker}{}%
\end{pgfscope}%
\begin{pgfscope}%
\pgfsys@transformshift{3.400000in}{2.722500in}%
\pgfsys@useobject{currentmarker}{}%
\end{pgfscope}%
\begin{pgfscope}%
\pgfsys@transformshift{3.787500in}{2.722500in}%
\pgfsys@useobject{currentmarker}{}%
\end{pgfscope}%
\begin{pgfscope}%
\pgfsys@transformshift{4.175000in}{2.722500in}%
\pgfsys@useobject{currentmarker}{}%
\end{pgfscope}%
\begin{pgfscope}%
\pgfsys@transformshift{4.562500in}{2.722500in}%
\pgfsys@useobject{currentmarker}{}%
\end{pgfscope}%
\end{pgfscope}%
\begin{pgfscope}%
\pgfpathrectangle{\pgfqpoint{0.687500in}{0.605000in}}{\pgfqpoint{4.262500in}{4.235000in}}%
\pgfusepath{clip}%
\pgfsetbuttcap%
\pgfsetroundjoin%
\pgfsetlinewidth{1.505625pt}%
\definecolor{currentstroke}{rgb}{0.007843,0.533333,0.000000}%
\pgfsetstrokecolor{currentstroke}%
\pgfsetdash{{5.550000pt}{2.400000pt}}{0.000000pt}%
\pgfpathmoveto{\pgfqpoint{1.075000in}{1.760000in}}%
\pgfpathlineto{\pgfqpoint{1.462500in}{0.797500in}}%
\pgfpathlineto{\pgfqpoint{1.850000in}{1.760000in}}%
\pgfpathlineto{\pgfqpoint{2.237500in}{1.760000in}}%
\pgfpathlineto{\pgfqpoint{2.625000in}{1.760000in}}%
\pgfpathlineto{\pgfqpoint{3.012500in}{1.760000in}}%
\pgfpathlineto{\pgfqpoint{3.400000in}{1.760000in}}%
\pgfpathlineto{\pgfqpoint{3.787500in}{1.760000in}}%
\pgfpathlineto{\pgfqpoint{4.175000in}{1.760000in}}%
\pgfpathlineto{\pgfqpoint{4.562500in}{1.760000in}}%
\pgfusepath{stroke}%
\end{pgfscope}%
\begin{pgfscope}%
\pgfpathrectangle{\pgfqpoint{0.687500in}{0.605000in}}{\pgfqpoint{4.262500in}{4.235000in}}%
\pgfusepath{clip}%
\pgfsetbuttcap%
\pgfsetmiterjoin%
\definecolor{currentfill}{rgb}{0.007843,0.533333,0.000000}%
\pgfsetfillcolor{currentfill}%
\pgfsetlinewidth{1.003750pt}%
\definecolor{currentstroke}{rgb}{0.007843,0.533333,0.000000}%
\pgfsetstrokecolor{currentstroke}%
\pgfsetdash{}{0pt}%
\pgfsys@defobject{currentmarker}{\pgfqpoint{-0.041667in}{-0.041667in}}{\pgfqpoint{0.041667in}{0.041667in}}{%
\pgfpathmoveto{\pgfqpoint{-0.000000in}{-0.041667in}}%
\pgfpathlineto{\pgfqpoint{0.041667in}{0.041667in}}%
\pgfpathlineto{\pgfqpoint{-0.041667in}{0.041667in}}%
\pgfpathclose%
\pgfusepath{stroke,fill}%
}%
\begin{pgfscope}%
\pgfsys@transformshift{1.075000in}{1.760000in}%
\pgfsys@useobject{currentmarker}{}%
\end{pgfscope}%
\begin{pgfscope}%
\pgfsys@transformshift{1.462500in}{0.797500in}%
\pgfsys@useobject{currentmarker}{}%
\end{pgfscope}%
\begin{pgfscope}%
\pgfsys@transformshift{1.850000in}{1.760000in}%
\pgfsys@useobject{currentmarker}{}%
\end{pgfscope}%
\begin{pgfscope}%
\pgfsys@transformshift{2.237500in}{1.760000in}%
\pgfsys@useobject{currentmarker}{}%
\end{pgfscope}%
\begin{pgfscope}%
\pgfsys@transformshift{2.625000in}{1.760000in}%
\pgfsys@useobject{currentmarker}{}%
\end{pgfscope}%
\begin{pgfscope}%
\pgfsys@transformshift{3.012500in}{1.760000in}%
\pgfsys@useobject{currentmarker}{}%
\end{pgfscope}%
\begin{pgfscope}%
\pgfsys@transformshift{3.400000in}{1.760000in}%
\pgfsys@useobject{currentmarker}{}%
\end{pgfscope}%
\begin{pgfscope}%
\pgfsys@transformshift{3.787500in}{1.760000in}%
\pgfsys@useobject{currentmarker}{}%
\end{pgfscope}%
\begin{pgfscope}%
\pgfsys@transformshift{4.175000in}{1.760000in}%
\pgfsys@useobject{currentmarker}{}%
\end{pgfscope}%
\begin{pgfscope}%
\pgfsys@transformshift{4.562500in}{1.760000in}%
\pgfsys@useobject{currentmarker}{}%
\end{pgfscope}%
\end{pgfscope}%
\begin{pgfscope}%
\pgfsetrectcap%
\pgfsetmiterjoin%
\pgfsetlinewidth{0.803000pt}%
\definecolor{currentstroke}{rgb}{0.000000,0.000000,0.000000}%
\pgfsetstrokecolor{currentstroke}%
\pgfsetdash{}{0pt}%
\pgfpathmoveto{\pgfqpoint{0.687500in}{0.605000in}}%
\pgfpathlineto{\pgfqpoint{0.687500in}{4.840000in}}%
\pgfusepath{stroke}%
\end{pgfscope}%
\begin{pgfscope}%
\pgfsetrectcap%
\pgfsetmiterjoin%
\pgfsetlinewidth{0.000000pt}%
\definecolor{currentstroke}{rgb}{0.000000,0.000000,0.000000}%
\pgfsetstrokecolor{currentstroke}%
\pgfsetstrokeopacity{0.000000}%
\pgfsetdash{}{0pt}%
\pgfpathmoveto{\pgfqpoint{4.950000in}{0.605000in}}%
\pgfpathlineto{\pgfqpoint{4.950000in}{4.840000in}}%
\pgfusepath{}%
\end{pgfscope}%
\begin{pgfscope}%
\pgfsetrectcap%
\pgfsetmiterjoin%
\pgfsetlinewidth{0.803000pt}%
\definecolor{currentstroke}{rgb}{0.000000,0.000000,0.000000}%
\pgfsetstrokecolor{currentstroke}%
\pgfsetdash{}{0pt}%
\pgfpathmoveto{\pgfqpoint{0.687500in}{0.605000in}}%
\pgfpathlineto{\pgfqpoint{4.950000in}{0.605000in}}%
\pgfusepath{stroke}%
\end{pgfscope}%
\begin{pgfscope}%
\pgfsetrectcap%
\pgfsetmiterjoin%
\pgfsetlinewidth{0.000000pt}%
\definecolor{currentstroke}{rgb}{0.000000,0.000000,0.000000}%
\pgfsetstrokecolor{currentstroke}%
\pgfsetstrokeopacity{0.000000}%
\pgfsetdash{}{0pt}%
\pgfpathmoveto{\pgfqpoint{0.687500in}{4.840000in}}%
\pgfpathlineto{\pgfqpoint{4.950000in}{4.840000in}}%
\pgfusepath{}%
\end{pgfscope}%
\begin{pgfscope}%
\pgfsetbuttcap%
\pgfsetmiterjoin%
\definecolor{currentfill}{rgb}{1.000000,1.000000,1.000000}%
\pgfsetfillcolor{currentfill}%
\pgfsetfillopacity{0.800000}%
\pgfsetlinewidth{1.003750pt}%
\definecolor{currentstroke}{rgb}{0.800000,0.800000,0.800000}%
\pgfsetstrokecolor{currentstroke}%
\pgfsetstrokeopacity{0.800000}%
\pgfsetdash{}{0pt}%
\pgfpathmoveto{\pgfqpoint{0.784722in}{4.117317in}}%
\pgfpathlineto{\pgfqpoint{1.677709in}{4.117317in}}%
\pgfpathquadraticcurveto{\pgfqpoint{1.705487in}{4.117317in}}{\pgfqpoint{1.705487in}{4.145095in}}%
\pgfpathlineto{\pgfqpoint{1.705487in}{4.742778in}}%
\pgfpathquadraticcurveto{\pgfqpoint{1.705487in}{4.770556in}}{\pgfqpoint{1.677709in}{4.770556in}}%
\pgfpathlineto{\pgfqpoint{0.784722in}{4.770556in}}%
\pgfpathquadraticcurveto{\pgfqpoint{0.756944in}{4.770556in}}{\pgfqpoint{0.756944in}{4.742778in}}%
\pgfpathlineto{\pgfqpoint{0.756944in}{4.145095in}}%
\pgfpathquadraticcurveto{\pgfqpoint{0.756944in}{4.117317in}}{\pgfqpoint{0.784722in}{4.117317in}}%
\pgfpathclose%
\pgfusepath{stroke,fill}%
\end{pgfscope}%
\begin{pgfscope}%
\pgfsetbuttcap%
\pgfsetroundjoin%
\pgfsetlinewidth{1.505625pt}%
\definecolor{currentstroke}{rgb}{0.843137,0.000000,0.000000}%
\pgfsetstrokecolor{currentstroke}%
\pgfsetdash{{5.550000pt}{2.400000pt}}{0.000000pt}%
\pgfpathmoveto{\pgfqpoint{0.812500in}{4.658088in}}%
\pgfpathlineto{\pgfqpoint{1.090278in}{4.658088in}}%
\pgfusepath{stroke}%
\end{pgfscope}%
\begin{pgfscope}%
\pgfsetbuttcap%
\pgfsetroundjoin%
\definecolor{currentfill}{rgb}{0.843137,0.000000,0.000000}%
\pgfsetfillcolor{currentfill}%
\pgfsetlinewidth{1.003750pt}%
\definecolor{currentstroke}{rgb}{0.843137,0.000000,0.000000}%
\pgfsetstrokecolor{currentstroke}%
\pgfsetdash{}{0pt}%
\pgfsys@defobject{currentmarker}{\pgfqpoint{-0.041667in}{-0.041667in}}{\pgfqpoint{0.041667in}{0.041667in}}{%
\pgfpathmoveto{\pgfqpoint{0.000000in}{-0.041667in}}%
\pgfpathcurveto{\pgfqpoint{0.011050in}{-0.041667in}}{\pgfqpoint{0.021649in}{-0.037276in}}{\pgfqpoint{0.029463in}{-0.029463in}}%
\pgfpathcurveto{\pgfqpoint{0.037276in}{-0.021649in}}{\pgfqpoint{0.041667in}{-0.011050in}}{\pgfqpoint{0.041667in}{0.000000in}}%
\pgfpathcurveto{\pgfqpoint{0.041667in}{0.011050in}}{\pgfqpoint{0.037276in}{0.021649in}}{\pgfqpoint{0.029463in}{0.029463in}}%
\pgfpathcurveto{\pgfqpoint{0.021649in}{0.037276in}}{\pgfqpoint{0.011050in}{0.041667in}}{\pgfqpoint{0.000000in}{0.041667in}}%
\pgfpathcurveto{\pgfqpoint{-0.011050in}{0.041667in}}{\pgfqpoint{-0.021649in}{0.037276in}}{\pgfqpoint{-0.029463in}{0.029463in}}%
\pgfpathcurveto{\pgfqpoint{-0.037276in}{0.021649in}}{\pgfqpoint{-0.041667in}{0.011050in}}{\pgfqpoint{-0.041667in}{0.000000in}}%
\pgfpathcurveto{\pgfqpoint{-0.041667in}{-0.011050in}}{\pgfqpoint{-0.037276in}{-0.021649in}}{\pgfqpoint{-0.029463in}{-0.029463in}}%
\pgfpathcurveto{\pgfqpoint{-0.021649in}{-0.037276in}}{\pgfqpoint{-0.011050in}{-0.041667in}}{\pgfqpoint{0.000000in}{-0.041667in}}%
\pgfpathclose%
\pgfusepath{stroke,fill}%
}%
\begin{pgfscope}%
\pgfsys@transformshift{0.951389in}{4.658088in}%
\pgfsys@useobject{currentmarker}{}%
\end{pgfscope}%
\end{pgfscope}%
\begin{pgfscope}%
\definecolor{textcolor}{rgb}{0.000000,0.000000,0.000000}%
\pgfsetstrokecolor{textcolor}%
\pgfsetfillcolor{textcolor}%
\pgftext[x=1.201389in,y=4.609477in,left,base]{\color{textcolor}\sffamily\fontsize{10.000000}{12.000000}\selectfont \(\displaystyle \beta = 0.8\)}%
\end{pgfscope}%
\begin{pgfscope}%
\pgfsetbuttcap%
\pgfsetroundjoin%
\pgfsetlinewidth{1.505625pt}%
\definecolor{currentstroke}{rgb}{0.549020,0.235294,1.000000}%
\pgfsetstrokecolor{currentstroke}%
\pgfsetdash{{5.550000pt}{2.400000pt}}{0.000000pt}%
\pgfpathmoveto{\pgfqpoint{0.812500in}{4.454231in}}%
\pgfpathlineto{\pgfqpoint{1.090278in}{4.454231in}}%
\pgfusepath{stroke}%
\end{pgfscope}%
\begin{pgfscope}%
\pgfsetbuttcap%
\pgfsetmiterjoin%
\definecolor{currentfill}{rgb}{0.549020,0.235294,1.000000}%
\pgfsetfillcolor{currentfill}%
\pgfsetlinewidth{1.003750pt}%
\definecolor{currentstroke}{rgb}{0.549020,0.235294,1.000000}%
\pgfsetstrokecolor{currentstroke}%
\pgfsetdash{}{0pt}%
\pgfsys@defobject{currentmarker}{\pgfqpoint{-0.041667in}{-0.041667in}}{\pgfqpoint{0.041667in}{0.041667in}}{%
\pgfpathmoveto{\pgfqpoint{0.000000in}{0.041667in}}%
\pgfpathlineto{\pgfqpoint{-0.041667in}{-0.041667in}}%
\pgfpathlineto{\pgfqpoint{0.041667in}{-0.041667in}}%
\pgfpathclose%
\pgfusepath{stroke,fill}%
}%
\begin{pgfscope}%
\pgfsys@transformshift{0.951389in}{4.454231in}%
\pgfsys@useobject{currentmarker}{}%
\end{pgfscope}%
\end{pgfscope}%
\begin{pgfscope}%
\definecolor{textcolor}{rgb}{0.000000,0.000000,0.000000}%
\pgfsetstrokecolor{textcolor}%
\pgfsetfillcolor{textcolor}%
\pgftext[x=1.201389in,y=4.405620in,left,base]{\color{textcolor}\sffamily\fontsize{10.000000}{12.000000}\selectfont \(\displaystyle \beta = 0.9\)}%
\end{pgfscope}%
\begin{pgfscope}%
\pgfsetbuttcap%
\pgfsetroundjoin%
\pgfsetlinewidth{1.505625pt}%
\definecolor{currentstroke}{rgb}{0.007843,0.533333,0.000000}%
\pgfsetstrokecolor{currentstroke}%
\pgfsetdash{{5.550000pt}{2.400000pt}}{0.000000pt}%
\pgfpathmoveto{\pgfqpoint{0.812500in}{4.250374in}}%
\pgfpathlineto{\pgfqpoint{1.090278in}{4.250374in}}%
\pgfusepath{stroke}%
\end{pgfscope}%
\begin{pgfscope}%
\pgfsetbuttcap%
\pgfsetmiterjoin%
\definecolor{currentfill}{rgb}{0.007843,0.533333,0.000000}%
\pgfsetfillcolor{currentfill}%
\pgfsetlinewidth{1.003750pt}%
\definecolor{currentstroke}{rgb}{0.007843,0.533333,0.000000}%
\pgfsetstrokecolor{currentstroke}%
\pgfsetdash{}{0pt}%
\pgfsys@defobject{currentmarker}{\pgfqpoint{-0.041667in}{-0.041667in}}{\pgfqpoint{0.041667in}{0.041667in}}{%
\pgfpathmoveto{\pgfqpoint{-0.000000in}{-0.041667in}}%
\pgfpathlineto{\pgfqpoint{0.041667in}{0.041667in}}%
\pgfpathlineto{\pgfqpoint{-0.041667in}{0.041667in}}%
\pgfpathclose%
\pgfusepath{stroke,fill}%
}%
\begin{pgfscope}%
\pgfsys@transformshift{0.951389in}{4.250374in}%
\pgfsys@useobject{currentmarker}{}%
\end{pgfscope}%
\end{pgfscope}%
\begin{pgfscope}%
\definecolor{textcolor}{rgb}{0.000000,0.000000,0.000000}%
\pgfsetstrokecolor{textcolor}%
\pgfsetfillcolor{textcolor}%
\pgftext[x=1.201389in,y=4.201763in,left,base]{\color{textcolor}\sffamily\fontsize{10.000000}{12.000000}\selectfont \(\displaystyle \beta = 1\)}%
\end{pgfscope}%
\end{pgfpicture}%
\makeatother%
\endgroup%

      \caption[GMRES iteration counts when $\NLqDRR{\nso-\nst} = 0.2\times k^{-\beta},$ for any $1 \leq q < \infty$ and $\beta = 0.8,0.9,1$.]{GMRES iteration counts for $\AmatoI\Amatt$ given by \cref{eq:noweak,eq:ntweak}, where $\alpha = 0.2\times k^{-\beta},$ for $\beta = 0.8,0.9,1.$}\label{fig:l1high}
\end{figure}

\begin{table}
  \centering
  \begin{tabular}{Sc Sc Sc Sc Sc Sc Sc Sc Sc ScSc }
\toprule

$\eps$\textbackslash$k$ &  10.0  &  20.0  &  30.0  &  40.0  &  50.0  &  60.0  &  70.0  &  80.0  &  90.0  &  100.0 \\

\midrule

0.0 &     14 &     40 &    119 &    258 &    427 &    627 &    940 &   1274 &   1695 &   2116 \\

0.1 &     13 &     27 &     70 &    147 &    262 &    394 &    590 &    825 &   1128 &   1393 \\

0.2 &     12 &     22 &     40 &     77 &    134 &    199 &    292 &    419 &    551 &    726 \\

0.3 &     11 &     18 &     25 &     40 &     58 &     86 &    119 &    163 &    209 &    270 \\

0.4 &     10 &     15 &     20 &     25 &     30 &     42 &     53 &     64 &     81 &     98 \\

0.5 &     10 &     13 &     16 &     19 &     22 &     25 &     28 &     31 &     37 &     41 \\

0.6 &      9 &     11 &     13 &     14 &     16 &     17 &     19 &     19 &     21 &     22 \\

0.7 &      8 &      9 &     10 &     11 &     12 &     13 &     13 &     14 &     14 &     14 \\

0.8 &      8 &      8 &      9 &      9 &     10 &     10 &     10 &     10 &     10 &     10 \\

0.9 &      7 &      7 &      8 &      8 &      8 &      8 &      8 &      8 &      8 &      8 \\

1.0 &      7 &      6 &      7 &      7 &      7 &      7 &      7 &      7 &      7 &      7 \\

\bottomrule

\end{tabular}


  \caption[GMRES iteration counts when $\NLqDRR{\nso-\nst} = 0.2\times k^{-\beta},$ for any $1 \leq q < \infty$ and $\beta = 0,0.1,\ldots,1$.]{GMRES iteration counts for $\AmatoI\Amatt$ given by \cref{eq:noweak,eq:ntweak}, where $\alpha = 0.2\times k^{-\beta}.$}\label{tab:l1}
  \end{table}



\section[Applying nearby preconditioning to QMC]{Applying nearby preconditioning to a Quasi-Monte-Carlo method for the Helmholtz equation}\label{sec:nbpcqmc}

We now apply nearby preconditioning in the implementation of a Quasi-Monte-Carlo (QMC) method for the Helmholtz equation. We begin with a brief description of QMC methods, before detailing two ways in which we apply nearby preconditioning to these methods. Finally, we give computational results illustrating this application.

\subsection{Brief description of QMC}

QMC methods (or rules) are high-dimensional quadrature rules, designed to give rates of convergence (with respect to the number of integration points) which are superior to those of Monte-Carlo methods, under certain conditions. Suppose one wants to approximate $\EXP{Q},$ where $Q$ is some random variable (later in this \lcnamecref{sec:nbpcqmc}, $Q$ will be a function of the solution $u(\omega)$ of a stochastic Helmholtz equation). By definition, the expectation is
\beq\label{eq:qmcexpdef}
\EXP{Q} = \int_\Omega Q(\omega)\ \ddPPomega.
\eeq

If we now suppose $Q$ depends on the sample space $\Omega$ via a finite set of random variables $\Uo,\ldots,\UJ$, then we can rewrite \cref{eq:qmcexpdef} as
\beq\label{eq:qmcexp2}
\EXP{Q} = \int_\Omega Q\mleft((\Uo(\omega),\ldots,\UJ(\omega)\mright)\, \ddPPomega.
\eeq
If, for example, the $\Uj$ are all independant uniform random variables on $\mleft[-1/2,1/2\mright]$, then \cref{eq:qmcexp2} can be rewritten as
\beq\label{eq:qmcexp3}
\EXP{Q} = \int_{\cube{J}} \hspace{-3em}Q\mleft(\by\mright)\, \dd\lambda(\by),
\eeq
where $\lambda$ denotes Lebesgue measure.

Any quadrature rule, or method for approximating $\EXP{Q}$, can then be seen as a method for approximating the $J$-dimensional integral on the right-hand side of \cref{eq:qmcexp3} and vice-versa. Equal-weight quadrature rules choose points $\byo,\ldots,\byNpoints \in \cube{J}$ and use the approximation
\beqs
\EXP{Q} \approx \frac1{\Npoints}\sum_{l=1}^{\Npoints} Q\mleft(\byl\mright).
\eeqs
Monte-Carlo and Quasi-Monte-Carlo rules correspond to different choices of the points $\byl$. In a Monte-Carlo rule the points are chosen at random in accordance with the associated probability distribution. For example, in the case that the $\Uj$ are $\Unif(-1/2,1/2)$ random variables, the points $\byl$ are chosen according to the Uniform distribution on $\cube{J}$. Observe that Monte-Carlo rules do not need the dependence of $Q$ on $\omega$ to take the form prescribed in \cref{eq:qmcexp2}, indeed, they apply to any random variable.

Quasi-Monte-Carlo rules, in contrast to Monte-Carlo rules, do require the dependence on $\omega$ to be via finitely- or countable-many random variables. This is because QMC rules are high-dimensional quadrature rules (in the simplest case performing quadrature on the high-dim\-en\-sion\-al cube $\cube{J}).$ In pure QMC rules the points $\byl$ are chosen deterministically, unlike Monte-Carlo rules.

The main advantage of QMC rules is that they can exhibit higher rates of convergence compared to Monte-Carlo rules; Monte Carlo rules typically converge with rate $\Npoints^{-1/2}$ (see, e.g., \cite[Section 1.1]{Gi:15}), whereas QMC rules can converge with rates up to $\Npoints^{-1}$  or with even higher rates for higher-order QMC rules, see, e.g., \cite[Penultimate paragraph of Section 1.2]{KuNu:16}.

In applying QMC rules to stochastic PDEs, we assume that the random coefficient ($n$ in our case) is defined via finitely many (or countably many) random variables, as in \cref{eq:qmcexp2} above, and we then use QMC rules to estimate expectations of quantities of interest of the solution $u$, i.e., $Q = Q(u).$ We note that applying QMC rules to stochastic PDEs is a vibrant and active research area. For recent overviews of this field, see \cite{KuNu:16,KuNu:18b} (and the associated tutorial \cite{KuNu:18a}). We note that there is currently no rigorous study of how QMC methods behave for the Helmholtz equation, although we understand some such work is currently underway by Ganesh, Kuo, and Sloan \cite{GaKuSl}.

\subsection{Methods for applying nearby preconditioning to QMC}\label{sec:nbpcqmcnum}
In all of our previous uses of nearby preconditioning, we have fixed $\nso$, the value for which we calculate the preconditioner, and have then used $\Amato$ to precondition $\Amatt$ for different values of $\nst.$ However, the key idea for applying nearby preconditioning to QMC methods for the Helmholtz equation is to choose \emph{a number} of different realisations of $\nso$ and use each realisation of $\nso$ as a preconditioner only for those  realisations of $\nst$ for which $\Amato$ is a \emph{good} preconditioner for $\Amatt.$ We adopt this approach because it is highly unlikely that a single realisation of $\nso$ will be a good preconditioner for every realisation of $\nst.$

Therefore, the algorithms presented in this \lcnamecref{sec:nbpcqmcnum} seek to answer the two questions:
\ben
\item \emph{For} which realisations of $n$ should a preconditioner be \emph{calculated}?
  \item \emph{To} which realisations of $n$ should each preconditioner be \emph{applied}?
\een

We now detail two methods for using nearby preconditioning to speed up QMC methods for the Helmholtz equation. To apply these methods, we use the following model problem: We consider the Interior Impedance Problem in 2-d with $f=1$ and $\gI=0$, $A = I$, and $n$ given by
\beq\label{eq:artificialkl}
n(\omega,\bx) = 1 + \sum_{j=1}^{10} \Uj(\omega) \sqrt{\lambdaj} \psij(\bx),
\eeq
where
\beq\label{eq:artificialkllambdas}
\sqrt{\lambdaj} = j^{-2}
\eeq
and
\beq\label{eq:artificialklfuns}
\psij(\bx) = \cos\mleft(\frac{j\pi}4 x\mright)\cos\mleft(\frac{\mleft(j+1\mright)\pi}4 y\mright).
\eeq
Observe that $\NLiDR{\psij}=1$ for all $j,$ and $\sqrt{\lambdaj} \rightarrow 0$ as $j \rightarrow \infty.$ Also note that $\nmin = 1 - \mleft(\sum_{j=1}^{10} j^{-2}\mright)/2 \approx 0.225.$ This expansion is based on the random-field expansion used in \cite[Section 5.1]{GiGrKuScSl:19}, although the main change we make from \cite{GiGrKuScSl:19} is to introduce the factors $1/4$ in \cref{eq:artificialklfuns}. We introduce this factor to ensure that the oscillations in the medium $n$ are `low frequency' compared to the frequency $k$ of the waves passing through the medium\footnote{The highest `frequency' associated with the oscillations in the medium is $(10+1)\pi/4 \approx 26$, whereas we consider waves with frequencies $k=10,\ldots,60$. Therefore (for $k > 26$) the waves are of a `higher frequency' than the medium. Moreover, we would see if there is any change in the behaviour of our algorithm as the frequency of the waves increases past the `frequency' of the medium. However, we do not see any such change.}. Expansions similar to \cref{eq:artificialkl} are often decribed as `artificial Karhunen--Lo\`eve expansions' due to their similarity with the Karhunen--Lo\`eve expansion of a random field. In a Karhunen--Lo\`eve expansion the $\Uj$ are independent random variables whose distribution is determined by the distribution of the random field, and the $\lambdaj$ and $\psij$ are the eigenvalues and eigenvectors of the covariance operator, see, e.g., \cite[Section 7.4]{LoPoSh:14}.

In the remainder of this \lcnamecref{sec:nbpcqmcnum} we will be using QMC methods to approximate $\EXP{Q(u)}$ (for some quantities of interest $Q$). Observe that this expectation can be written
\beqs
\EXP{Q(u)} = \int_{\Omega} Q(u(\omega)) \,\ddPPomega = \int_{\mleft[-\half,\half\mright]^{10}}Q(u(U_1,\ldots,U_{10})) \,\dd U_1 \cdots \dd U_{10},
\eeqs
where we consider $n$ (and therefore $u$) as depending on each of the Uniform random variables $\Uj$ individually. Therefore, because of this correspondence between $n$ as function on $\Omega,$ and $n$ as a function on $\mleft[-1/2,1/2\mright]^{10}$ we will sometimes instead write $n(\by)$ for $\by \in \cube{10}$, by which we mean
\beqs
n(\by) = 1 + \sum_{j=1}^{10} \by_{j} \sqrt{\lambdaj} \psij.
\eeqs
There is no a priori reason that one must have such an affine dependence of the random field on the randomness in order to apply nearby preconditioning to QMC methods. One could, for example, take $n$ to be a lognormal random field, in which case $n$ would take the form $n(\by) = \exp\mleft(\nz + \sum_j \Nj \sqrt{\lambdaj} \psij\mright)$ where the $\Nj$ are Normal$(0,1)$ random variables. However, in the case of affine dependence there is a `parallelisable' nearby-preconditioning-QMC algorithm which we present below.

We stress that the results in this \lcnamecref{sec:nbpcqmcnum} are strictly numerical; there is no current theory to support these calculations. In particular, we observe in \cref{sec:nbpcqmcnumerics} below that in these experiments, for the QMC error for Helmholtz problems to remain bounded as $k$ increases, one must increase the number of QMC points with $k.$ We again remark that there is currently no theoretical justification for this behaviour.


\paragraph{Terminology} Before we describe the nearby-preconditioning-QMC algorithms in detail, we establish two pieces of terminology that will be of use in describing them. Firstly, we will use the word `point' to refer to a point in the parameter space $\cube{J}$, and use phrases such as `calculate a preconditioner at the point $\by$' as shorthand for `calculate the LU decomposition of the system matrix $\Amat$ corresponding to the finite-element discretisation of the Helmholtz IIP (as described above) with coefficient $n(\by)$'.

We also use the words `nearby' and `nearest' (when referring to QMC points) to mean: nearest in the metric
  \beq\label{eq:dapprox}
\dapprox(\byo,\byt) \de \sum_{j=1}^{J} \sqrt{\lambdaj} \abs{{\byo}_{j} - {\byt}_{j}}.%, \tfor \byo,\byt \in \cube{J}.
\eeq

\paragraph{The approximate metric} The metric $\dapprox$ is an approximation of the metric
\beq\label{eq:dqmc}
\dQMC(\byo,\byt) = \NLiDRR{n(\byo)-n(\byt)},% \tfor \byo,\byt \in \cube{J},
\eeq
i.e., the metric on $\cube{J}$ induced by the spatial $L^\infty$ norm. The metric $\dapprox$ is an approximation of $\dQMC$ in the sense made precise in the following \lcnamecref{lem:approxmetric}.

\ble[$\dapprox$ approximates $\dQMC$]\label{lem:approxmetric}
For all $\byo,\byt \in \cube{J},$
\beqs
\dQMC\mleft(\byo,\byt\mright) \leq \dapprox\mleft(\byo,\byt\mright).
\eeqs
\ele

The proof of \cref{lem:approxmetric} is straightforward and omitted.

Observe further that the structure of $\dapprox$ is similar to that of $\dQMC$ and $\dapprox$ is a weighted $L^1$ metric on $\cube{J}$, with the weights corresponding to the terms in \cref{eq:artificialkl}. Recall that $\sqrt{\lambdaj} \rightarrow 0$ as $j \rightarrow \infty$; therefore the higher dimensions contribute less to the value of $\dapprox$ (or, informally, points are `closer' in higher dimensions, or higher dimensions are `smaller' than lower dimensions).

Ideally, for the purposes of utilising nearby preconditioning, we would use the metric $\dQMC$ when describing the geometry of the QMC points  (and computing the nearest QMC point), since the best rigorous results on the behaviour of nearby preconditioning (in terms of $k$-dependence) are proved in \cref{sec:3} for the spatial $L^\infty$-norm\footnote{Although, in line with the results in \cref{sec:weaknorm}, we could instead use a spatial $L^q$ norm, for some $q \geq 1$ in \cref{eq:dqmc}.}. However, computing $\dQMC$ exactly is, in principle. complicated. In contrast, it is easy to compute with $\dapprox,$ since $\dapprox$ enables one to think of $\cube{J}$ as the high-dimensional rectangle $\mleft[0,\sqrt{\lambdao}\mright]\times\cdots\times\mleft[0,\sqrt{\lambdaJ}\mright]$ equipped with the standard $L^1$ metric. Moreover, as discussed above, $\dapprox$ is an approximation of $\dQMC,$ and therefore we expect that it will induce a similar geometry on $\cube{J}.$

\paragraph{Computational complexity of calculating the nearest point} At various points in the two nearby-preconditioning-QMC algorithms we present below, given a point $\by \in \cube{J}$ and a subset $S$ of $\cube{J}$ we must calculate $\nearest(\by,S) \in \cube{J}$, that is the element of $S$ that is closest to $\by$ in the metric $\dapprox.$ In all of the numerical results we present below, we calculate $\nearest(\by,S)$ by brute force, i.e., we calculate $\dapprox(\by,\bytilde)$ for all $\bytilde \in S,$ and choose the element of $S$ that minimises $\dapprox(\by,\bytilde)$. Since calculating $\dapprox(\by,\bytilde)$ involves $\cO\mleft(J\mright)$ operations, the brute force approach to calculating $\dapprox(\by,\bytilde)$ involves $\cO\mleft(J\abs{S}\mright)$ operations. Clearly, this method of calculating $\nearest(\by,S)$ does not scale in $J,$ the stochastic dimension, although it is computationally feasible for our numerical experiments (with $J=10$) below. See \cref{sec:nbpcfuture} below for a suggestion of an alternative, scalable way to calculate $\nearest(\by,S)$.

\subsubsection{A sequential algorithm}
We first describe a straightforward algorithm that uses nearby preconditioning to speed up a QMC calculation. We call this a `sequential' algorithm because, unlike the `parallel' algorithm that we describe below, it is intrinsically sequential and cannot be parallelised, i.e., finite-element solves for different realisations of the random field $n$ cannot be treated in parallel. Although, when performing the individual finite-element solves, one is not restricted to a single core, i.e., one can use parallelisation for each finite-element solve  if the linear systems $\Amat$ are large enough to warrant this.

An overview of the algorithm is:
\ben
\item Choose a QMC point $\by$ for which to calculate a preconditioner
\item\label[itemstep]{it:nearest} Find the nearest QMC point $\byp$ to $\by$ and attempt a GMRES solve of the problem at $\byp$ using the LU decomposition of the system at $\by$ as a preconditioner.
    \item If GMRES converges quickly (i.e., in fewer than a preset number of iterations), return to \cref{it:nearest}.
\item If GMRES takes too long to converge, recalculate the preconditioner at $\byp,$ set $\by = \byp$, and return to \cref{it:nearest}.
  \een
  The algorithm is written in more formal pseudocode in \cref{alg:seq}.
\begin{algorithm}[h]
\DontPrintSemicolon
\SetKwInOut{Input}{Input}
%\SetKwInOut{Output}{Output}
\SetKwFunction{Nearest}{nearest}

\Input{$\maxGMRES$, $\SQMC$}
\BlankLine
Choose starting point $\bystart$\;
$\bypre \defined \bystart$\;
$\Sremaining \defined \SQMC\setminus\set{\bypre}$\;
Calculate and store preconditioner $\Lmat\Umat = \AmatpreI$\;
$\bycurrent \defined$ \Nearest{$\bypre,\Sremaining$}\;
\While{$\Sremaining \neq \emptyset$}{
\eIf{GMRES applied to $\UmatI\LmatI\Amat\bycurrent = \UmatI\LmatI \bff$ converges in fewer than $\maxGMRES$ iterations}{
$\Sremaining \defined \Sremaining\setminus\set{\bycurrent}$\;
$\bycurrent \defined$ \Nearest{$\bypre,\Sremaining$}\;
}{
$\bypre \defined \bycurrent$\;
Calculate and store preconditioner $\Lmat\Umat = \AmatpreI$\;
}
}
\caption[The sequential nearby-preconditioning-Quasi-Monte-Carlo algorithm]{The sequential nearby-preconditioning-Quasi-Monte-Carlo algorithm\label{alg:seq}. $\maxGMRES$ is the maximum allowed number of GMRES iterations and $\SQMC$ is the set of all QMC points. $\nearest(\bypre,\Sremaining)$ denotes the point in $\SQMC$ nearest to $\bypre$ in the $\dQMC$ metric.}
\end{algorithm}
\subsubsection{A parallel algorithm}

The main disadvantage of the `sequential' algorithm described above is that the points at which preconditioners are calculated are identified as the algorithm progresses. The algorithm cannot be parallelised by sending different collections of QMC points to different processors (as one does not know  a priori which preconditioner to use for each QMC point). Therefore, we now suggest an alternative algorithm that allows one to specify the number of preconditioning points \emph{before} the algorithm begins. The algorithm then calculates which points to use as preconditioning points, before performing the linear solves. Because the preconditioners are known in advance, the solves can be computed in parallel if required. The most complicated part of the algorithm is deciding at which points to calculate the preconditioners, and so we describe this part of the algorithm in more detail here. A more formal pseudocode description of the algorithm is given in \cref{alg:par}.

Suppose we are given a set $\SQMC = \set{\byo,\ldots,\byNQMC}$ of QMC points and a number $\Npretarget$; the target number of preconditioners to compute. The aim of this algorithm is to select (approximately) $\Npretarget$ QMC points that are (approximately) equally spaced with respect to the $\dQMC$ metric defined above. If such a goal is achieved, then one expects that the preconditioning points are best located to minimise the total number of GMRES iterations across the solves for all of the QMC points.

The algorithm contains two key ideas:
\ben
  \item Use a surrogate metric in place of $\dQMC$, and
\item Locate the preconditioning points according to a tensor-product rule.
  \een
  We now describe each of these two ideas in turn, before describing our final algorithm.

\paragraph{Surrogate metric} Whilst the metric $\dQMC$ is the metric in which  nearby preconditioning is analysed (as described in \cref{sec:intronbpc} above), in practice $\dQMC$ is difficult to work with, since the  geometry it induces on $\cube{J}$ is nontrivial as this geometry is dependent on the interaction between the functions $\psij$ in the expansion \cref{eq:artificialkl}. Therefore, we work in an alternative, although related metric HELP

\paragraph{Tensor-product algorithm for locating preconditioning points} We first describe the intuition behind our use of a tensor-product rule to locate the preconditioning points (even though we do not use this intuition in the final algorithm). Once we have described this intution, we will then show how it can be adapted to provide the final algorithm. To understand why we use locate the preconditioning points using a tensor-product rule, we first decribe the heuristic we use. Let us assume we want to cover $\cube{J}$ with `balls' of radius $r$. Observe that these balls are measured in the $\dapprox$ metric, and therefore have a similar geometry to balls on $\cube{J}$ in the $L^1$ metric. Therefore, given the centres $\bcone$ and $\bct$ of two adjacent balls, we will have
\beq\label{eq:centres2r}
\dapprox(\bcone,\bct) = 2r.
\eeq
The question now arises of how we choose $\bcone$ and $\bct$ so that \cref{eq:centres2r} holds. We observe that, by the definition of $\dapprox$, if we choose $\bcone$ and $\bct$ such that
\beqs
\sqrt{\lambdaj}\abs{{\bcone}_{j}-{\bct}_j} = \frac{2r}{J} \tforall j = 1,\ldots,J,
\eeqs
then we will have \cref{eq:centres2r} by construction, because
\beqs
\dapprox(\bcone,\bct) = \sum_{j=1}^J \frac{2r}J = 2r.
\eeqs
Therefore, in dimension $j$ we choose the centres of the balls to be spaced
\beqs
\min\set{\frac{2r}{J\sqrt{\lambdaj}},1}
\eeqs
apart (where we include the minimum so that, for high dimensions, we include at least one centre). That is, in dimension $j$, we take
\beq\label{eq:Nj}
\Nj \de \max\set{1,\frac{J\sqrt{\lambdaj}}{2r}}
\eeq
equally spaced points in the sets $\centresj = \set{c_{j,1},\ldots,c_{j,\Nj}},$ and then we form the centres $\bcone,\ldots,\bcNpre$ by taking all possible tensor products of the points in $\centreso,\ldots,\centresJ,$ giving a total of
\beq\label{eq:Npre}
\Npre = \No \times \cdots \times \NJ
\eeq
preconditioning points.

However, we face three immediate difficulties with  the above approach:
\ben
\item The above procedure assumes we know the radius $r$, and then returns the total number of preconditioning points, and their locations. However, we only know in advance the ideal total number of preconditioning points.
\item There is no guarantee that the numbers of points $\Nj$ calculated above are integers.
  \item There is no guarantee the preconditioning points given by the above procedure are QMC points.
    \een
    These questions are all completely valid, and so we slightly modify the above procedure to deal with them.

    \paragraph{Definition of the parallel algorithm} Recall that we assume that we are given a target number of preconditioners $\Npretarget$. The above procedure (amongst other things) defines a map $\Npreideal:\RRp \rightarrow \RRp$ given by $r \mapsto \Npre,$ where $\Npre$ is defined by \cref{eq:Npre} and the number of preconditioners in each dimension is given by \cref{eq:Nj}.  Therefore we can numerically invert this map (or more precisely, calculate numerically the value $\rideal$ such that $\Npreideal(\rideal) = \Npretarget$). (In our computations, we do this calculation via interval bisection.)

Given we expect that the size of the balls over which nearby preconditioning is effective decreases with $\cO\mleft(1/k\mright)$ (in line with \cref{cor:1}), and the number of QMC points needed to keep the error bounded increases with $k$ (see \cref{sec:nbpcqmcnumerics} below), it is not obvious that we should know $\Npreideal$ in advance. See \cpageref{page:seqandpar} for how we use the sequential algorithm to determine how $\Npreideal$ scales with $k$ for the parallel algorithm. We assume for now that we know $\Npreideal$ and hence $\rideal.$

    Once we know the value of $\rideal,$ we can then calculate the numbers of centres in each dimension $\No(\rideal),\ldots,\NJ(\rideal)$ as above (recalling that the $\Nj(\rideal)$ are not necessarily integers). We then obtain integers $\Npreactualj = \round{\Nj(\rideal)}$, where $\round{\cdot}$ denotes rounding to the nearest integer. (Recall $\Nj(\rideal) \geq 1$ for all $j$ by construction, so $\Npreactualj$ will be a positive integer for all $j.$)

We then take $\Npreactualj$ centres in each dimension and define the sets $\centresj$ with $\Nj = \Npreactualj$, as described above. We then obtain a total of $\Npreactual = \Npreactualo \times \cdots \times \NpreactualJ$ preconditioning points.

These points may not be QMC points. We could simply calculate the preconditioners at these non-QMC points. However we instead replace each calculated centre with its nearest QMC point (calculated using brute-force) and calculate the preconditioners at these QMC points. We denote the set of preconditioning points by $\Spre$. Finally, we calculate the map $\Prenearest:\SQMC\rightarrow\Spre$, i.e., for each QMC point we find its nearest preconditioning point, and use the corresponding preconditioner for the linear solve.

    This algorithm is summarised more formally in \cref{alg:par}.
    
    \bre[Is calculating $\Prenearest$ computationally expensive?]
    We note that calculating the map $\Prenearest:\SQMC\rightarrow \Spre$ is an $\cO\mleft(\NQMC\Npre\mright)$ operation, because for each QMC point we must find the nearest preconditioning point. Given that $\Spre \subseteq \SQMC,$ it is possible that calculating $\Prenearest$ could actually be an $\cO\mleft(\NQMC^2\mright)$ operation.

    However, we expect that $\Npre$ will be small relative to $\NQMC$ (and this is borne out in the numerical experiments summarised in \cref{tab:nbpcqmcpar} below) and therefore we expect $\cO\mleft(\NQMC\Npre\mright) \approx \cO\mleft(\NQMC\mright).$ Hence calculating $\Prenearest$ should not be an expensive computational task.

    A similar line of reasoning shows that calculating the nearest QMC point to each of calculated tensor-product points (as outlined above) should also be an $\cO\mleft(\NQMC\mright)$ task.
    \ere

%% Define
%% \beqs
%% \Npreidealj(r) = \max\set{\frac{J \sqrt{\lambdaj}}{2r},1}.
%% \eeqs
%% The `ideal' total number of QMC points is
%% \beqs
%% \Npreideal(r)=\prod_{j=1}^J  \Npreidealj(r)
%% \eeqs

%% Want to calculate the number of preconditioners $\Npre$, the set
%% \beqs
%% \Spre=\set{\ypreo,\ldots,\ypreNpre}
%% \eeqs
%% of QMC points at which to calculate the preconditioner and the map
%% \beqs
%% \nearestpre:\SQMC\rightarrow\Spre
%% \eeqs
%% taking each QMC point to its nearest (in the induced spatial $L^\infty$ norm) preconditioner, where $\SQMC$ is the set of QMC points.

\begin{algorithm}[h]
\DontPrintSemicolon
\SetKwInOut{Input}{Input}\SetKwInOut{Output}{Output}
\SetKwFunction{Round}{round}

\Input{$\Npretarget \in \NN$}
\Output{The set $\Spre$, the map $\nearestpre:\SQMC\rightarrow\Spre$}
\BlankLine
Solve (numerically) $\Npreideal(\rideal) = \Npretarget$ for $\rideal$\;
\For{j $= 1$ \KwTo $J$}{
Calculate $\Npreactualj =$ \Round{$\Npreidealj(\rideal)$}\;
Define $\Sprej$ to be set of $\Npreactualj$ equally spaced points in $\mleft[-1/2,1/2\mright]$\;
}
Define $\displaystyle\Npre = \prod_{j=1}^J \Npreactualj$\;
Define $\Spre$ by taking all possible tensor products of points in $\Sprej$, and then finding the nearest QMC point to each one\;
\For{l $=1$ \KwTo $\NQMC$}{
Calculate $\nearestpre\mleft(\by^{(l)}\mright)$\;
}
\caption[The main part of the parallel nearby-preconditioning-Quasi-Monte-Carlo algorithm.]{The main part of the parallel nearby-preconditioning-Quasi-Monte-Carlo algorithm. This part of the algorithm determines $\Spre$ and $\nearestpre$. $\Spre$ is the set of preconditioning points, and $\nearestpre:\SQMC\rightarrow\Spre$ maps each QMC points to its nearest preconditioning point.\label{alg:par}}
\end{algorithm}

\subsubsection{Comparing and Constrasting the two algorithms}

We now briefly list the main differences in the two algorithms given above.

\paragraph{Complexity} The sequential algorithm is simple and intuitive to describe, given that it mainly revolves around `finding the nearest point'. However, the parallel algorithm is much more complicated, both in the underlying ideas, but also in its technical definition.

\paragraph{Heuristics} The sequential algorithm has very minimal heuristics; one only needs to specify the maximum number of GMRES iterations and this could be determined, for example, by the memory constraints of the machine one is using. In contrast, for the parallel algorithm one needs a heuristic for how many preconditioning points to choose, as this is not given by the algorithm. (In our numerical experiments below, we obtain this heuristic by using the sequential algorithm for low $k$, and then extrapolating the proportion of preconditioning points used for low values of $k$ to larger values of $k.$

\paragraph{Parallelisability} Unsurprisingly (given the name) the sequential algorithm is inherently serial; one must see whether a given solve converges in the required number of GMRES iterations before knowing whether we must recalculate the preconditioner for subsequent solves. (In principle one could parallelise the algorithm by splitting the QMC points up onto different groups of processors, and then use the sequential algorithm on each group of processors. However, there is no guarantee one would split the QMC points up in a way that grouped nearby points, therefore this approach could lead to a substantial increase in computational work.) In contrast, the parallel algorithm is fully parallelisable; once the preconditioning points and the map $\nearestpre:\SQMC\rightarrow\Spre$ have been calculated, one can send different linear solves to different groups of processors as one chooses. (Although note that, unless one sends all of the QMC points corresponding to a single preconditioner to the \emph{same} group of processors, one may need to calculate the same preconditioner several times, on different groups of processors\footnote{In our code, we split up the points with respect to the order they are generated by the QMC code. This was purely to make the code simpler.} However, the decrease in computational time gained from parallelisation should more than offset this increase in computational effort.)

\paragraph{Choice of preconditioning points} Neither algorithm will necessarily pick the optimal set of preconditioning points (optimal in the sense of the minimal number of preconditioning points needed). In the sequential algorithm, there is no guarantee that this method for exploring the sample space and choosing the preconditioning points will give an optimal collection of preconditioning points. Also, whilst for the parallel algorithm the preconditioning points should fill the parameter space `well' (given the points are chosen a priori to be well spaced according to the $\dapprox$ metric), the number of preconditioning points generated is not exactly $\Npretarget$ due to rounding the `ideal' number of centres in each dimension to the nearest integer. Therefore, even in the parallel case, one may not end up with an optimal set of preconditioning points.

\subsection{Numerical Experiments}\label{sec:nbpcqmcnumerics}
We now describe numerical experiments that demostrate the effectiveness of the above algorithms.         Our main result is that, for a particular QMC model problem, nearby preconditioning gives a substantial speedup, with around 98\% of solves being computed using a previously-calculated LU decomposition.

For the computational setup, including the algorithm we use to generate our QMC points, see \cref{app:compsetup}.

Before we perform our numerical experiments, we need to determine:
\bit
\item How the number of QMC points should scale with $k$, and
  \item How many preconditioners we should choose.
    \eit
    Throughout this \lcnamecref{sec:nbpcqmcnumerics} we use the model problem detailed in \cref{eq:artificialkl,eq:artificialkllambdas,eq:artificialklfuns} above.

\subsubsection{QMC error estimators}
    
    To determine how the number of QMC points should scale with $k$, we first estimate how the QMC error grows as $k$ increases. The QMC rule we use is a randomly shifted QMC rule, we use such a rule because there exists an error estimator for this rule, see \cref{eq:errest} below. Our exposition below follows that in \cite[Section 4.2]{GrKuNuScSl:11}.

    Suppose our QMC points are $\byo,\ldots,\byNQMC$, and the resulting QMC rule is
    \beqs
\QMC{Q} = \frac1{\NQMC}\sum_{l=1}^{\NQMC} Q\mleft(u\mleft(\byl\mright)\mright).
\eeqs
For a `shift' $\shift \in \cube{J}$ we define the shifted QMC rule
\beqs
\QMCshift{Q}{\shift} = \frac1{\NQMC}\sum_{l=1}^{\NQMC} Q\mleft(u\mleft(\byl\oplus\shift\mright)\mright),
\eeqs
where $\by \oplus \shift$ denotes $\by + \shift$ `wrapped around' onto the hypercube $\cube{J}$. (Formally $\by \oplus \shift = \fracoperator{\mleft(\by + \bhalf\mright)+\shift} - \bhalf,$ where $\fracoperator{\cdot}$ denotes the fractional part and $\bhalf$ denotes the $J$-dimensional vector with every entry $1/2.$)

We can then define the randomly-shifted QMC rule (with multiple randomly-chosen shifts $\shifto,\ldots,\shiftNshifts$)
\beqs
\QMCrandshift{Q}{\Nshifts} = \frac1{\Nshifts}\sum_{s=1}^{\Nshifts} \QMCshift{Q}{\shifts} = \frac1{\NQMC\Nshifts}\sum_{s=1}^{\Nshifts}\sum_{l=1}^{\NQMC} Q\mleft(u\mleft(\byl\oplus \shifts\mright)\mright).
\eeqs

Having defined the randomly shifted QMC rule, one can use the standard statistical estimator of the standard deviation of the statistical error in $\QMCrandshift{Q}{\Nshifts}$ \cite[Equation (4.6)]{GrKuNuScSl:11}
\beq\label{eq:errest}
\QMCerror{\NQMC}{\Nshifts} = \mleft(\frac1{\Nshifts\mleft(\Nshifts-1\mright)}\sum_{s=1}^{\Nshifts} \mleft(\QMCshift{Q}{\shifts} - \QMCrandshift{Q}{\Nshifts}\mright)^2\mright)^{\half}.
\eeq
(See \cref{app:complexerror} for proof that $\QMCerror{\NQMC}{\Nshifts}^2$ is an unbiased estimator of the variance of $\QMCrandshift{Q}{\Nshifts}$; recall that it does \emph{not} then follow that $\QMCerror{\NQMC}{\Nshifts}$ is an \emph{unbiased} estimator of the standard deviation of $\QMCrandshift{Q}{\Nshifts}$.)

\subsubsection{$k$-dependence of the number of QMC points}

We first sought to determine how $\QMCerror{\NQMC}{\Nshifts}$ depends on $k.$ We estimated the error $\QMCerror{\NQMC}{\Nshifts}$ for the setup described in \cref{app:compsetup} with $\NQMC = 2048$ and $\Nshifts=20$ (i.e., 40,960 PDE solves in total) for $k = 10,20,30,40,50,60$. We set $h = 0.002$ for all of the computations (as $0.002 \approx 60^{-3/2}$), as then by \cref{thm:fembound} the finite-element error is of the order $h^2k^3 \sim (k/60)^3 \lesssim 1$ for all the values of $k$ we consider\footnote{Observe that we do not let $h$ depend on $k$, in contrast to the rest of this thesis. This decision means we do not have to consider the effect of changing the mesh on the resulting interpolation of the random field $n$, and how this interpolation may affect the overall error. In addition, since $k \leq 60,$ our particular choice of mesh ensures that the finite-element error is small for all the values of $k$ we consider.}. The quantities of interest (QoIs) we considered were:
\bit
\item The integral of $u$ over the whole domain $\mleft[0,1\mright]^2$,
\item The value of $u$ at the origin,
\item The value of $u$ at the top-right corner of the domain, and
\item The $x$-component of $\grad u$ at the top-right corner of the domain.
  \eit
  Observe that these QoIs require a certain amount of regularity of the solution. (The integral is defined for functions in $\LoD$, point evaluation for functions in $\Hfn{}{3/2 + \eps}{D}$ and point evaluation of the gradient for functions in $\Hfn{}{5/2+\eps}{D}$ (in 3-d - the corresponding function spaces are $\Hfn{}{1+\eps}{D}$ and $\Hfn{}{2+\eps}{D}$ in 2-d) for any $\eps > 0.$) Therefore computing for this range of QoIs will give a good insight into the behaviour of QMC applied to the Helmholtz equation\footnote{We can evaluate point values of $\uh$ because $\uh$ is continuous, and we use the constant value of $\grad \uh$ on the upper-rightmost mesh element as a proxy for $\grad \uh((1,1))$; such a use is possible due to the structure of our mesh, see \cref{fig:grid}, and the fact that we use first-order finite elements.}.
%%     That is, we randomly choose $\shifto,\ldots,\shiftNshifts$ points in $\cube{J}$ (the `shifts')rause the standard error estimator
%%     \beqs
%%     \mleft(\frac1{\nu\mleft(\nu-1\mright)} \sum_{s=1}^{\Nshifts} 
%%     \eeqs$h = 0.002$ (relation to $k=60$ - maximum?)

%In \cref{fig:integralCalpha,fig:originCalpha,fig:toprightCalpha,fig:gradienttoprightCalpha} we plot how $C$ and $\alpha$ depend on $k$, for the plots of the QMC error with increasing $\NQMC$ for each value of $k,$ see \cref{app:hhqmcconv}.

\begin{figure}[h]
    \centering
    \begin{subfigure}{\textwidth}
      \centering
%% Creator: Matplotlib, PGF backend
%%
%% To include the figure in your LaTeX document, write
%%   \input{<filename>.pgf}
%%
%% Make sure the required packages are loaded in your preamble
%%   \usepackage{pgf}
%%
%% Figures using additional raster images can only be included by \input if
%% they are in the same directory as the main LaTeX file. For loading figures
%% from other directories you can use the `import` package
%%   \usepackage{import}
%% and then include the figures with
%%   \import{<path to file>}{<filename>.pgf}
%%
%% Matplotlib used the following preamble
%%   \usepackage{fontspec}
%%   \setmainfont{DejaVuSerif.ttf}[Path=/home/owen/progs/firedrake-complex/firedrake/lib/python3.5/site-packages/matplotlib/mpl-data/fonts/ttf/]
%%   \setsansfont{DejaVuSans.ttf}[Path=/home/owen/progs/firedrake-complex/firedrake/lib/python3.5/site-packages/matplotlib/mpl-data/fonts/ttf/]
%%   \setmonofont{DejaVuSansMono.ttf}[Path=/home/owen/progs/firedrake-complex/firedrake/lib/python3.5/site-packages/matplotlib/mpl-data/fonts/ttf/]
%%
\begingroup%
\makeatletter%
\begin{pgfpicture}%
\pgfpathrectangle{\pgfpointorigin}{\pgfqpoint{5.000000in}{4.000000in}}%
\pgfusepath{use as bounding box, clip}%
\begin{pgfscope}%
\pgfsetbuttcap%
\pgfsetmiterjoin%
\definecolor{currentfill}{rgb}{1.000000,1.000000,1.000000}%
\pgfsetfillcolor{currentfill}%
\pgfsetlinewidth{0.000000pt}%
\definecolor{currentstroke}{rgb}{1.000000,1.000000,1.000000}%
\pgfsetstrokecolor{currentstroke}%
\pgfsetdash{}{0pt}%
\pgfpathmoveto{\pgfqpoint{0.000000in}{0.000000in}}%
\pgfpathlineto{\pgfqpoint{5.000000in}{0.000000in}}%
\pgfpathlineto{\pgfqpoint{5.000000in}{4.000000in}}%
\pgfpathlineto{\pgfqpoint{0.000000in}{4.000000in}}%
\pgfpathclose%
\pgfusepath{fill}%
\end{pgfscope}%
\begin{pgfscope}%
\pgfsetbuttcap%
\pgfsetmiterjoin%
\definecolor{currentfill}{rgb}{1.000000,1.000000,1.000000}%
\pgfsetfillcolor{currentfill}%
\pgfsetlinewidth{0.000000pt}%
\definecolor{currentstroke}{rgb}{0.000000,0.000000,0.000000}%
\pgfsetstrokecolor{currentstroke}%
\pgfsetstrokeopacity{0.000000}%
\pgfsetdash{}{0pt}%
\pgfpathmoveto{\pgfqpoint{0.625000in}{0.440000in}}%
\pgfpathlineto{\pgfqpoint{4.500000in}{0.440000in}}%
\pgfpathlineto{\pgfqpoint{4.500000in}{3.520000in}}%
\pgfpathlineto{\pgfqpoint{0.625000in}{3.520000in}}%
\pgfpathclose%
\pgfusepath{fill}%
\end{pgfscope}%
\begin{pgfscope}%
\pgfsetbuttcap%
\pgfsetroundjoin%
\definecolor{currentfill}{rgb}{0.000000,0.000000,0.000000}%
\pgfsetfillcolor{currentfill}%
\pgfsetlinewidth{0.803000pt}%
\definecolor{currentstroke}{rgb}{0.000000,0.000000,0.000000}%
\pgfsetstrokecolor{currentstroke}%
\pgfsetdash{}{0pt}%
\pgfsys@defobject{currentmarker}{\pgfqpoint{0.000000in}{-0.048611in}}{\pgfqpoint{0.000000in}{0.000000in}}{%
\pgfpathmoveto{\pgfqpoint{0.000000in}{0.000000in}}%
\pgfpathlineto{\pgfqpoint{0.000000in}{-0.048611in}}%
\pgfusepath{stroke,fill}%
}%
\begin{pgfscope}%
\pgfsys@transformshift{0.801136in}{0.440000in}%
\pgfsys@useobject{currentmarker}{}%
\end{pgfscope}%
\end{pgfscope}%
\begin{pgfscope}%
\definecolor{textcolor}{rgb}{0.000000,0.000000,0.000000}%
\pgfsetstrokecolor{textcolor}%
\pgfsetfillcolor{textcolor}%
\pgftext[x=0.801136in,y=0.342778in,,top]{\color{textcolor}\sffamily\fontsize{10.000000}{12.000000}\selectfont \(\displaystyle {10^{1}}\)}%
\end{pgfscope}%
\begin{pgfscope}%
\pgfsetbuttcap%
\pgfsetroundjoin%
\definecolor{currentfill}{rgb}{0.000000,0.000000,0.000000}%
\pgfsetfillcolor{currentfill}%
\pgfsetlinewidth{0.602250pt}%
\definecolor{currentstroke}{rgb}{0.000000,0.000000,0.000000}%
\pgfsetstrokecolor{currentstroke}%
\pgfsetdash{}{0pt}%
\pgfsys@defobject{currentmarker}{\pgfqpoint{0.000000in}{-0.027778in}}{\pgfqpoint{0.000000in}{0.000000in}}{%
\pgfpathmoveto{\pgfqpoint{0.000000in}{0.000000in}}%
\pgfpathlineto{\pgfqpoint{0.000000in}{-0.027778in}}%
\pgfusepath{stroke,fill}%
}%
\begin{pgfscope}%
\pgfsys@transformshift{2.163913in}{0.440000in}%
\pgfsys@useobject{currentmarker}{}%
\end{pgfscope}%
\end{pgfscope}%
\begin{pgfscope}%
\definecolor{textcolor}{rgb}{0.000000,0.000000,0.000000}%
\pgfsetstrokecolor{textcolor}%
\pgfsetfillcolor{textcolor}%
\pgftext[x=2.163913in,y=0.365000in,,top]{\color{textcolor}\sffamily\fontsize{10.000000}{12.000000}\selectfont \(\displaystyle {2\times10^{1}}\)}%
\end{pgfscope}%
\begin{pgfscope}%
\pgfsetbuttcap%
\pgfsetroundjoin%
\definecolor{currentfill}{rgb}{0.000000,0.000000,0.000000}%
\pgfsetfillcolor{currentfill}%
\pgfsetlinewidth{0.602250pt}%
\definecolor{currentstroke}{rgb}{0.000000,0.000000,0.000000}%
\pgfsetstrokecolor{currentstroke}%
\pgfsetdash{}{0pt}%
\pgfsys@defobject{currentmarker}{\pgfqpoint{0.000000in}{-0.027778in}}{\pgfqpoint{0.000000in}{0.000000in}}{%
\pgfpathmoveto{\pgfqpoint{0.000000in}{0.000000in}}%
\pgfpathlineto{\pgfqpoint{0.000000in}{-0.027778in}}%
\pgfusepath{stroke,fill}%
}%
\begin{pgfscope}%
\pgfsys@transformshift{2.961087in}{0.440000in}%
\pgfsys@useobject{currentmarker}{}%
\end{pgfscope}%
\end{pgfscope}%
\begin{pgfscope}%
\definecolor{textcolor}{rgb}{0.000000,0.000000,0.000000}%
\pgfsetstrokecolor{textcolor}%
\pgfsetfillcolor{textcolor}%
\pgftext[x=2.961087in,y=0.365000in,,top]{\color{textcolor}\sffamily\fontsize{10.000000}{12.000000}\selectfont \(\displaystyle {3\times10^{1}}\)}%
\end{pgfscope}%
\begin{pgfscope}%
\pgfsetbuttcap%
\pgfsetroundjoin%
\definecolor{currentfill}{rgb}{0.000000,0.000000,0.000000}%
\pgfsetfillcolor{currentfill}%
\pgfsetlinewidth{0.602250pt}%
\definecolor{currentstroke}{rgb}{0.000000,0.000000,0.000000}%
\pgfsetstrokecolor{currentstroke}%
\pgfsetdash{}{0pt}%
\pgfsys@defobject{currentmarker}{\pgfqpoint{0.000000in}{-0.027778in}}{\pgfqpoint{0.000000in}{0.000000in}}{%
\pgfpathmoveto{\pgfqpoint{0.000000in}{0.000000in}}%
\pgfpathlineto{\pgfqpoint{0.000000in}{-0.027778in}}%
\pgfusepath{stroke,fill}%
}%
\begin{pgfscope}%
\pgfsys@transformshift{3.526690in}{0.440000in}%
\pgfsys@useobject{currentmarker}{}%
\end{pgfscope}%
\end{pgfscope}%
\begin{pgfscope}%
\definecolor{textcolor}{rgb}{0.000000,0.000000,0.000000}%
\pgfsetstrokecolor{textcolor}%
\pgfsetfillcolor{textcolor}%
\pgftext[x=3.526690in,y=0.365000in,,top]{\color{textcolor}\sffamily\fontsize{10.000000}{12.000000}\selectfont \(\displaystyle {4\times10^{1}}\)}%
\end{pgfscope}%
\begin{pgfscope}%
\pgfsetbuttcap%
\pgfsetroundjoin%
\definecolor{currentfill}{rgb}{0.000000,0.000000,0.000000}%
\pgfsetfillcolor{currentfill}%
\pgfsetlinewidth{0.602250pt}%
\definecolor{currentstroke}{rgb}{0.000000,0.000000,0.000000}%
\pgfsetstrokecolor{currentstroke}%
\pgfsetdash{}{0pt}%
\pgfsys@defobject{currentmarker}{\pgfqpoint{0.000000in}{-0.027778in}}{\pgfqpoint{0.000000in}{0.000000in}}{%
\pgfpathmoveto{\pgfqpoint{0.000000in}{0.000000in}}%
\pgfpathlineto{\pgfqpoint{0.000000in}{-0.027778in}}%
\pgfusepath{stroke,fill}%
}%
\begin{pgfscope}%
\pgfsys@transformshift{3.965406in}{0.440000in}%
\pgfsys@useobject{currentmarker}{}%
\end{pgfscope}%
\end{pgfscope}%
\begin{pgfscope}%
\pgfsetbuttcap%
\pgfsetroundjoin%
\definecolor{currentfill}{rgb}{0.000000,0.000000,0.000000}%
\pgfsetfillcolor{currentfill}%
\pgfsetlinewidth{0.602250pt}%
\definecolor{currentstroke}{rgb}{0.000000,0.000000,0.000000}%
\pgfsetstrokecolor{currentstroke}%
\pgfsetdash{}{0pt}%
\pgfsys@defobject{currentmarker}{\pgfqpoint{0.000000in}{-0.027778in}}{\pgfqpoint{0.000000in}{0.000000in}}{%
\pgfpathmoveto{\pgfqpoint{0.000000in}{0.000000in}}%
\pgfpathlineto{\pgfqpoint{0.000000in}{-0.027778in}}%
\pgfusepath{stroke,fill}%
}%
\begin{pgfscope}%
\pgfsys@transformshift{4.323864in}{0.440000in}%
\pgfsys@useobject{currentmarker}{}%
\end{pgfscope}%
\end{pgfscope}%
\begin{pgfscope}%
\definecolor{textcolor}{rgb}{0.000000,0.000000,0.000000}%
\pgfsetstrokecolor{textcolor}%
\pgfsetfillcolor{textcolor}%
\pgftext[x=4.323864in,y=0.365000in,,top]{\color{textcolor}\sffamily\fontsize{10.000000}{12.000000}\selectfont \(\displaystyle {6\times10^{1}}\)}%
\end{pgfscope}%
\begin{pgfscope}%
\definecolor{textcolor}{rgb}{0.000000,0.000000,0.000000}%
\pgfsetstrokecolor{textcolor}%
\pgfsetfillcolor{textcolor}%
\pgftext[x=2.562500in,y=0.152809in,,top]{\color{textcolor}\sffamily\fontsize{10.000000}{12.000000}\selectfont k}%
\end{pgfscope}%
\begin{pgfscope}%
\pgfsetbuttcap%
\pgfsetroundjoin%
\definecolor{currentfill}{rgb}{0.000000,0.000000,0.000000}%
\pgfsetfillcolor{currentfill}%
\pgfsetlinewidth{0.803000pt}%
\definecolor{currentstroke}{rgb}{0.000000,0.000000,0.000000}%
\pgfsetstrokecolor{currentstroke}%
\pgfsetdash{}{0pt}%
\pgfsys@defobject{currentmarker}{\pgfqpoint{-0.048611in}{0.000000in}}{\pgfqpoint{0.000000in}{0.000000in}}{%
\pgfpathmoveto{\pgfqpoint{0.000000in}{0.000000in}}%
\pgfpathlineto{\pgfqpoint{-0.048611in}{0.000000in}}%
\pgfusepath{stroke,fill}%
}%
\begin{pgfscope}%
\pgfsys@transformshift{0.625000in}{2.395872in}%
\pgfsys@useobject{currentmarker}{}%
\end{pgfscope}%
\end{pgfscope}%
\begin{pgfscope}%
\definecolor{textcolor}{rgb}{0.000000,0.000000,0.000000}%
\pgfsetstrokecolor{textcolor}%
\pgfsetfillcolor{textcolor}%
\pgftext[x=0.239775in,y=2.343110in,left,base]{\color{textcolor}\sffamily\fontsize{10.000000}{12.000000}\selectfont \(\displaystyle {10^{-3}}\)}%
\end{pgfscope}%
\begin{pgfscope}%
\pgfsetbuttcap%
\pgfsetroundjoin%
\definecolor{currentfill}{rgb}{0.000000,0.000000,0.000000}%
\pgfsetfillcolor{currentfill}%
\pgfsetlinewidth{0.602250pt}%
\definecolor{currentstroke}{rgb}{0.000000,0.000000,0.000000}%
\pgfsetstrokecolor{currentstroke}%
\pgfsetdash{}{0pt}%
\pgfsys@defobject{currentmarker}{\pgfqpoint{-0.027778in}{0.000000in}}{\pgfqpoint{0.000000in}{0.000000in}}{%
\pgfpathmoveto{\pgfqpoint{0.000000in}{0.000000in}}%
\pgfpathlineto{\pgfqpoint{-0.027778in}{0.000000in}}%
\pgfusepath{stroke,fill}%
}%
\begin{pgfscope}%
\pgfsys@transformshift{0.625000in}{0.951144in}%
\pgfsys@useobject{currentmarker}{}%
\end{pgfscope}%
\end{pgfscope}%
\begin{pgfscope}%
\pgfsetbuttcap%
\pgfsetroundjoin%
\definecolor{currentfill}{rgb}{0.000000,0.000000,0.000000}%
\pgfsetfillcolor{currentfill}%
\pgfsetlinewidth{0.602250pt}%
\definecolor{currentstroke}{rgb}{0.000000,0.000000,0.000000}%
\pgfsetstrokecolor{currentstroke}%
\pgfsetdash{}{0pt}%
\pgfsys@defobject{currentmarker}{\pgfqpoint{-0.027778in}{0.000000in}}{\pgfqpoint{0.000000in}{0.000000in}}{%
\pgfpathmoveto{\pgfqpoint{0.000000in}{0.000000in}}%
\pgfpathlineto{\pgfqpoint{-0.027778in}{0.000000in}}%
\pgfusepath{stroke,fill}%
}%
\begin{pgfscope}%
\pgfsys@transformshift{0.625000in}{1.315114in}%
\pgfsys@useobject{currentmarker}{}%
\end{pgfscope}%
\end{pgfscope}%
\begin{pgfscope}%
\pgfsetbuttcap%
\pgfsetroundjoin%
\definecolor{currentfill}{rgb}{0.000000,0.000000,0.000000}%
\pgfsetfillcolor{currentfill}%
\pgfsetlinewidth{0.602250pt}%
\definecolor{currentstroke}{rgb}{0.000000,0.000000,0.000000}%
\pgfsetstrokecolor{currentstroke}%
\pgfsetdash{}{0pt}%
\pgfsys@defobject{currentmarker}{\pgfqpoint{-0.027778in}{0.000000in}}{\pgfqpoint{0.000000in}{0.000000in}}{%
\pgfpathmoveto{\pgfqpoint{0.000000in}{0.000000in}}%
\pgfpathlineto{\pgfqpoint{-0.027778in}{0.000000in}}%
\pgfusepath{stroke,fill}%
}%
\begin{pgfscope}%
\pgfsys@transformshift{0.625000in}{1.573354in}%
\pgfsys@useobject{currentmarker}{}%
\end{pgfscope}%
\end{pgfscope}%
\begin{pgfscope}%
\pgfsetbuttcap%
\pgfsetroundjoin%
\definecolor{currentfill}{rgb}{0.000000,0.000000,0.000000}%
\pgfsetfillcolor{currentfill}%
\pgfsetlinewidth{0.602250pt}%
\definecolor{currentstroke}{rgb}{0.000000,0.000000,0.000000}%
\pgfsetstrokecolor{currentstroke}%
\pgfsetdash{}{0pt}%
\pgfsys@defobject{currentmarker}{\pgfqpoint{-0.027778in}{0.000000in}}{\pgfqpoint{0.000000in}{0.000000in}}{%
\pgfpathmoveto{\pgfqpoint{0.000000in}{0.000000in}}%
\pgfpathlineto{\pgfqpoint{-0.027778in}{0.000000in}}%
\pgfusepath{stroke,fill}%
}%
\begin{pgfscope}%
\pgfsys@transformshift{0.625000in}{1.773661in}%
\pgfsys@useobject{currentmarker}{}%
\end{pgfscope}%
\end{pgfscope}%
\begin{pgfscope}%
\pgfsetbuttcap%
\pgfsetroundjoin%
\definecolor{currentfill}{rgb}{0.000000,0.000000,0.000000}%
\pgfsetfillcolor{currentfill}%
\pgfsetlinewidth{0.602250pt}%
\definecolor{currentstroke}{rgb}{0.000000,0.000000,0.000000}%
\pgfsetstrokecolor{currentstroke}%
\pgfsetdash{}{0pt}%
\pgfsys@defobject{currentmarker}{\pgfqpoint{-0.027778in}{0.000000in}}{\pgfqpoint{0.000000in}{0.000000in}}{%
\pgfpathmoveto{\pgfqpoint{0.000000in}{0.000000in}}%
\pgfpathlineto{\pgfqpoint{-0.027778in}{0.000000in}}%
\pgfusepath{stroke,fill}%
}%
\begin{pgfscope}%
\pgfsys@transformshift{0.625000in}{1.937324in}%
\pgfsys@useobject{currentmarker}{}%
\end{pgfscope}%
\end{pgfscope}%
\begin{pgfscope}%
\pgfsetbuttcap%
\pgfsetroundjoin%
\definecolor{currentfill}{rgb}{0.000000,0.000000,0.000000}%
\pgfsetfillcolor{currentfill}%
\pgfsetlinewidth{0.602250pt}%
\definecolor{currentstroke}{rgb}{0.000000,0.000000,0.000000}%
\pgfsetstrokecolor{currentstroke}%
\pgfsetdash{}{0pt}%
\pgfsys@defobject{currentmarker}{\pgfqpoint{-0.027778in}{0.000000in}}{\pgfqpoint{0.000000in}{0.000000in}}{%
\pgfpathmoveto{\pgfqpoint{0.000000in}{0.000000in}}%
\pgfpathlineto{\pgfqpoint{-0.027778in}{0.000000in}}%
\pgfusepath{stroke,fill}%
}%
\begin{pgfscope}%
\pgfsys@transformshift{0.625000in}{2.075699in}%
\pgfsys@useobject{currentmarker}{}%
\end{pgfscope}%
\end{pgfscope}%
\begin{pgfscope}%
\pgfsetbuttcap%
\pgfsetroundjoin%
\definecolor{currentfill}{rgb}{0.000000,0.000000,0.000000}%
\pgfsetfillcolor{currentfill}%
\pgfsetlinewidth{0.602250pt}%
\definecolor{currentstroke}{rgb}{0.000000,0.000000,0.000000}%
\pgfsetstrokecolor{currentstroke}%
\pgfsetdash{}{0pt}%
\pgfsys@defobject{currentmarker}{\pgfqpoint{-0.027778in}{0.000000in}}{\pgfqpoint{0.000000in}{0.000000in}}{%
\pgfpathmoveto{\pgfqpoint{0.000000in}{0.000000in}}%
\pgfpathlineto{\pgfqpoint{-0.027778in}{0.000000in}}%
\pgfusepath{stroke,fill}%
}%
\begin{pgfscope}%
\pgfsys@transformshift{0.625000in}{2.195565in}%
\pgfsys@useobject{currentmarker}{}%
\end{pgfscope}%
\end{pgfscope}%
\begin{pgfscope}%
\pgfsetbuttcap%
\pgfsetroundjoin%
\definecolor{currentfill}{rgb}{0.000000,0.000000,0.000000}%
\pgfsetfillcolor{currentfill}%
\pgfsetlinewidth{0.602250pt}%
\definecolor{currentstroke}{rgb}{0.000000,0.000000,0.000000}%
\pgfsetstrokecolor{currentstroke}%
\pgfsetdash{}{0pt}%
\pgfsys@defobject{currentmarker}{\pgfqpoint{-0.027778in}{0.000000in}}{\pgfqpoint{0.000000in}{0.000000in}}{%
\pgfpathmoveto{\pgfqpoint{0.000000in}{0.000000in}}%
\pgfpathlineto{\pgfqpoint{-0.027778in}{0.000000in}}%
\pgfusepath{stroke,fill}%
}%
\begin{pgfscope}%
\pgfsys@transformshift{0.625000in}{2.301294in}%
\pgfsys@useobject{currentmarker}{}%
\end{pgfscope}%
\end{pgfscope}%
\begin{pgfscope}%
\pgfsetbuttcap%
\pgfsetroundjoin%
\definecolor{currentfill}{rgb}{0.000000,0.000000,0.000000}%
\pgfsetfillcolor{currentfill}%
\pgfsetlinewidth{0.602250pt}%
\definecolor{currentstroke}{rgb}{0.000000,0.000000,0.000000}%
\pgfsetstrokecolor{currentstroke}%
\pgfsetdash{}{0pt}%
\pgfsys@defobject{currentmarker}{\pgfqpoint{-0.027778in}{0.000000in}}{\pgfqpoint{0.000000in}{0.000000in}}{%
\pgfpathmoveto{\pgfqpoint{0.000000in}{0.000000in}}%
\pgfpathlineto{\pgfqpoint{-0.027778in}{0.000000in}}%
\pgfusepath{stroke,fill}%
}%
\begin{pgfscope}%
\pgfsys@transformshift{0.625000in}{3.018082in}%
\pgfsys@useobject{currentmarker}{}%
\end{pgfscope}%
\end{pgfscope}%
\begin{pgfscope}%
\pgfsetbuttcap%
\pgfsetroundjoin%
\definecolor{currentfill}{rgb}{0.000000,0.000000,0.000000}%
\pgfsetfillcolor{currentfill}%
\pgfsetlinewidth{0.602250pt}%
\definecolor{currentstroke}{rgb}{0.000000,0.000000,0.000000}%
\pgfsetstrokecolor{currentstroke}%
\pgfsetdash{}{0pt}%
\pgfsys@defobject{currentmarker}{\pgfqpoint{-0.027778in}{0.000000in}}{\pgfqpoint{0.000000in}{0.000000in}}{%
\pgfpathmoveto{\pgfqpoint{0.000000in}{0.000000in}}%
\pgfpathlineto{\pgfqpoint{-0.027778in}{0.000000in}}%
\pgfusepath{stroke,fill}%
}%
\begin{pgfscope}%
\pgfsys@transformshift{0.625000in}{3.382052in}%
\pgfsys@useobject{currentmarker}{}%
\end{pgfscope}%
\end{pgfscope}%
\begin{pgfscope}%
\definecolor{textcolor}{rgb}{0.000000,0.000000,0.000000}%
\pgfsetstrokecolor{textcolor}%
\pgfsetfillcolor{textcolor}%
\pgftext[x=0.184220in,y=1.980000in,,bottom,rotate=90.000000]{\color{textcolor}\sffamily\fontsize{10.000000}{12.000000}\selectfont C}%
\end{pgfscope}%
\begin{pgfscope}%
\pgfpathrectangle{\pgfqpoint{0.625000in}{0.440000in}}{\pgfqpoint{3.875000in}{3.080000in}}%
\pgfusepath{clip}%
\pgfsetbuttcap%
\pgfsetroundjoin%
\definecolor{currentfill}{rgb}{0.000000,0.000000,0.000000}%
\pgfsetfillcolor{currentfill}%
\pgfsetlinewidth{1.003750pt}%
\definecolor{currentstroke}{rgb}{0.000000,0.000000,0.000000}%
\pgfsetstrokecolor{currentstroke}%
\pgfsetdash{}{0pt}%
\pgfsys@defobject{currentmarker}{\pgfqpoint{-0.041667in}{-0.041667in}}{\pgfqpoint{0.041667in}{0.041667in}}{%
\pgfpathmoveto{\pgfqpoint{0.000000in}{-0.041667in}}%
\pgfpathcurveto{\pgfqpoint{0.011050in}{-0.041667in}}{\pgfqpoint{0.021649in}{-0.037276in}}{\pgfqpoint{0.029463in}{-0.029463in}}%
\pgfpathcurveto{\pgfqpoint{0.037276in}{-0.021649in}}{\pgfqpoint{0.041667in}{-0.011050in}}{\pgfqpoint{0.041667in}{0.000000in}}%
\pgfpathcurveto{\pgfqpoint{0.041667in}{0.011050in}}{\pgfqpoint{0.037276in}{0.021649in}}{\pgfqpoint{0.029463in}{0.029463in}}%
\pgfpathcurveto{\pgfqpoint{0.021649in}{0.037276in}}{\pgfqpoint{0.011050in}{0.041667in}}{\pgfqpoint{0.000000in}{0.041667in}}%
\pgfpathcurveto{\pgfqpoint{-0.011050in}{0.041667in}}{\pgfqpoint{-0.021649in}{0.037276in}}{\pgfqpoint{-0.029463in}{0.029463in}}%
\pgfpathcurveto{\pgfqpoint{-0.037276in}{0.021649in}}{\pgfqpoint{-0.041667in}{0.011050in}}{\pgfqpoint{-0.041667in}{0.000000in}}%
\pgfpathcurveto{\pgfqpoint{-0.041667in}{-0.011050in}}{\pgfqpoint{-0.037276in}{-0.021649in}}{\pgfqpoint{-0.029463in}{-0.029463in}}%
\pgfpathcurveto{\pgfqpoint{-0.021649in}{-0.037276in}}{\pgfqpoint{-0.011050in}{-0.041667in}}{\pgfqpoint{0.000000in}{-0.041667in}}%
\pgfpathclose%
\pgfusepath{stroke,fill}%
}%
\begin{pgfscope}%
\pgfsys@transformshift{0.801136in}{3.380000in}%
\pgfsys@useobject{currentmarker}{}%
\end{pgfscope}%
\begin{pgfscope}%
\pgfsys@transformshift{2.163913in}{2.654291in}%
\pgfsys@useobject{currentmarker}{}%
\end{pgfscope}%
\begin{pgfscope}%
\pgfsys@transformshift{2.961087in}{2.293438in}%
\pgfsys@useobject{currentmarker}{}%
\end{pgfscope}%
\begin{pgfscope}%
\pgfsys@transformshift{3.526690in}{0.957210in}%
\pgfsys@useobject{currentmarker}{}%
\end{pgfscope}%
\begin{pgfscope}%
\pgfsys@transformshift{3.965406in}{0.901992in}%
\pgfsys@useobject{currentmarker}{}%
\end{pgfscope}%
\begin{pgfscope}%
\pgfsys@transformshift{4.323864in}{0.580000in}%
\pgfsys@useobject{currentmarker}{}%
\end{pgfscope}%
\end{pgfscope}%
\begin{pgfscope}%
\pgfsetrectcap%
\pgfsetmiterjoin%
\pgfsetlinewidth{0.803000pt}%
\definecolor{currentstroke}{rgb}{0.000000,0.000000,0.000000}%
\pgfsetstrokecolor{currentstroke}%
\pgfsetdash{}{0pt}%
\pgfpathmoveto{\pgfqpoint{0.625000in}{0.440000in}}%
\pgfpathlineto{\pgfqpoint{0.625000in}{3.520000in}}%
\pgfusepath{stroke}%
\end{pgfscope}%
\begin{pgfscope}%
\pgfsetrectcap%
\pgfsetmiterjoin%
\pgfsetlinewidth{0.803000pt}%
\definecolor{currentstroke}{rgb}{0.000000,0.000000,0.000000}%
\pgfsetstrokecolor{currentstroke}%
\pgfsetdash{}{0pt}%
\pgfpathmoveto{\pgfqpoint{4.500000in}{0.440000in}}%
\pgfpathlineto{\pgfqpoint{4.500000in}{3.520000in}}%
\pgfusepath{stroke}%
\end{pgfscope}%
\begin{pgfscope}%
\pgfsetrectcap%
\pgfsetmiterjoin%
\pgfsetlinewidth{0.803000pt}%
\definecolor{currentstroke}{rgb}{0.000000,0.000000,0.000000}%
\pgfsetstrokecolor{currentstroke}%
\pgfsetdash{}{0pt}%
\pgfpathmoveto{\pgfqpoint{0.625000in}{0.440000in}}%
\pgfpathlineto{\pgfqpoint{4.500000in}{0.440000in}}%
\pgfusepath{stroke}%
\end{pgfscope}%
\begin{pgfscope}%
\pgfsetrectcap%
\pgfsetmiterjoin%
\pgfsetlinewidth{0.803000pt}%
\definecolor{currentstroke}{rgb}{0.000000,0.000000,0.000000}%
\pgfsetstrokecolor{currentstroke}%
\pgfsetdash{}{0pt}%
\pgfpathmoveto{\pgfqpoint{0.625000in}{3.520000in}}%
\pgfpathlineto{\pgfqpoint{4.500000in}{3.520000in}}%
\pgfusepath{stroke}%
\end{pgfscope}%
\end{pgfpicture}%
\makeatother%
\endgroup%

    \end{subfigure}
    \begin{subfigure}{\textwidth}
                \centering
      %% Creator: Matplotlib, PGF backend
%%
%% To include the figure in your LaTeX document, write
%%   \input{<filename>.pgf}
%%
%% Make sure the required packages are loaded in your preamble
%%   \usepackage{pgf}
%%
%% Figures using additional raster images can only be included by \input if
%% they are in the same directory as the main LaTeX file. For loading figures
%% from other directories you can use the `import` package
%%   \usepackage{import}
%% and then include the figures with
%%   \import{<path to file>}{<filename>.pgf}
%%
%% Matplotlib used the following preamble
%%   \usepackage{fontspec}
%%   \setmainfont{DejaVuSerif.ttf}[Path=/home/owen/progs/firedrake-complex/firedrake/lib/python3.5/site-packages/matplotlib/mpl-data/fonts/ttf/]
%%   \setsansfont{DejaVuSans.ttf}[Path=/home/owen/progs/firedrake-complex/firedrake/lib/python3.5/site-packages/matplotlib/mpl-data/fonts/ttf/]
%%   \setmonofont{DejaVuSansMono.ttf}[Path=/home/owen/progs/firedrake-complex/firedrake/lib/python3.5/site-packages/matplotlib/mpl-data/fonts/ttf/]
%%
\begingroup%
\makeatletter%
\begin{pgfpicture}%
\pgfpathrectangle{\pgfpointorigin}{\pgfqpoint{5.000000in}{4.000000in}}%
\pgfusepath{use as bounding box, clip}%
\begin{pgfscope}%
\pgfsetbuttcap%
\pgfsetmiterjoin%
\definecolor{currentfill}{rgb}{1.000000,1.000000,1.000000}%
\pgfsetfillcolor{currentfill}%
\pgfsetlinewidth{0.000000pt}%
\definecolor{currentstroke}{rgb}{1.000000,1.000000,1.000000}%
\pgfsetstrokecolor{currentstroke}%
\pgfsetdash{}{0pt}%
\pgfpathmoveto{\pgfqpoint{0.000000in}{0.000000in}}%
\pgfpathlineto{\pgfqpoint{5.000000in}{0.000000in}}%
\pgfpathlineto{\pgfqpoint{5.000000in}{4.000000in}}%
\pgfpathlineto{\pgfqpoint{0.000000in}{4.000000in}}%
\pgfpathclose%
\pgfusepath{fill}%
\end{pgfscope}%
\begin{pgfscope}%
\pgfsetbuttcap%
\pgfsetmiterjoin%
\definecolor{currentfill}{rgb}{1.000000,1.000000,1.000000}%
\pgfsetfillcolor{currentfill}%
\pgfsetlinewidth{0.000000pt}%
\definecolor{currentstroke}{rgb}{0.000000,0.000000,0.000000}%
\pgfsetstrokecolor{currentstroke}%
\pgfsetstrokeopacity{0.000000}%
\pgfsetdash{}{0pt}%
\pgfpathmoveto{\pgfqpoint{0.625000in}{0.440000in}}%
\pgfpathlineto{\pgfqpoint{4.500000in}{0.440000in}}%
\pgfpathlineto{\pgfqpoint{4.500000in}{3.520000in}}%
\pgfpathlineto{\pgfqpoint{0.625000in}{3.520000in}}%
\pgfpathclose%
\pgfusepath{fill}%
\end{pgfscope}%
\begin{pgfscope}%
\pgfsetbuttcap%
\pgfsetroundjoin%
\definecolor{currentfill}{rgb}{0.000000,0.000000,0.000000}%
\pgfsetfillcolor{currentfill}%
\pgfsetlinewidth{0.803000pt}%
\definecolor{currentstroke}{rgb}{0.000000,0.000000,0.000000}%
\pgfsetstrokecolor{currentstroke}%
\pgfsetdash{}{0pt}%
\pgfsys@defobject{currentmarker}{\pgfqpoint{0.000000in}{-0.048611in}}{\pgfqpoint{0.000000in}{0.000000in}}{%
\pgfpathmoveto{\pgfqpoint{0.000000in}{0.000000in}}%
\pgfpathlineto{\pgfqpoint{0.000000in}{-0.048611in}}%
\pgfusepath{stroke,fill}%
}%
\begin{pgfscope}%
\pgfsys@transformshift{2.172304in}{0.440000in}%
\pgfsys@useobject{currentmarker}{}%
\end{pgfscope}%
\end{pgfscope}%
\begin{pgfscope}%
\definecolor{textcolor}{rgb}{0.000000,0.000000,0.000000}%
\pgfsetstrokecolor{textcolor}%
\pgfsetfillcolor{textcolor}%
\pgftext[x=2.172304in,y=0.342778in,,top]{\color{textcolor}\sffamily\fontsize{10.000000}{12.000000}\selectfont \(\displaystyle {2.718281828459045^{3}}\)}%
\end{pgfscope}%
\begin{pgfscope}%
\pgfsetbuttcap%
\pgfsetroundjoin%
\definecolor{currentfill}{rgb}{0.000000,0.000000,0.000000}%
\pgfsetfillcolor{currentfill}%
\pgfsetlinewidth{0.803000pt}%
\definecolor{currentstroke}{rgb}{0.000000,0.000000,0.000000}%
\pgfsetstrokecolor{currentstroke}%
\pgfsetdash{}{0pt}%
\pgfsys@defobject{currentmarker}{\pgfqpoint{0.000000in}{-0.048611in}}{\pgfqpoint{0.000000in}{0.000000in}}{%
\pgfpathmoveto{\pgfqpoint{0.000000in}{0.000000in}}%
\pgfpathlineto{\pgfqpoint{0.000000in}{-0.048611in}}%
\pgfusepath{stroke,fill}%
}%
\begin{pgfscope}%
\pgfsys@transformshift{4.138375in}{0.440000in}%
\pgfsys@useobject{currentmarker}{}%
\end{pgfscope}%
\end{pgfscope}%
\begin{pgfscope}%
\definecolor{textcolor}{rgb}{0.000000,0.000000,0.000000}%
\pgfsetstrokecolor{textcolor}%
\pgfsetfillcolor{textcolor}%
\pgftext[x=4.138375in,y=0.342778in,,top]{\color{textcolor}\sffamily\fontsize{10.000000}{12.000000}\selectfont \(\displaystyle {2.718281828459045^{4}}\)}%
\end{pgfscope}%
\begin{pgfscope}%
\definecolor{textcolor}{rgb}{0.000000,0.000000,0.000000}%
\pgfsetstrokecolor{textcolor}%
\pgfsetfillcolor{textcolor}%
\pgftext[x=2.562500in,y=0.152809in,,top]{\color{textcolor}\sffamily\fontsize{10.000000}{12.000000}\selectfont \(\displaystyle k\)}%
\end{pgfscope}%
\begin{pgfscope}%
\pgfsetbuttcap%
\pgfsetroundjoin%
\definecolor{currentfill}{rgb}{0.000000,0.000000,0.000000}%
\pgfsetfillcolor{currentfill}%
\pgfsetlinewidth{0.803000pt}%
\definecolor{currentstroke}{rgb}{0.000000,0.000000,0.000000}%
\pgfsetstrokecolor{currentstroke}%
\pgfsetdash{}{0pt}%
\pgfsys@defobject{currentmarker}{\pgfqpoint{-0.048611in}{0.000000in}}{\pgfqpoint{0.000000in}{0.000000in}}{%
\pgfpathmoveto{\pgfqpoint{0.000000in}{0.000000in}}%
\pgfpathlineto{\pgfqpoint{-0.048611in}{0.000000in}}%
\pgfusepath{stroke,fill}%
}%
\begin{pgfscope}%
\pgfsys@transformshift{0.625000in}{0.567382in}%
\pgfsys@useobject{currentmarker}{}%
\end{pgfscope}%
\end{pgfscope}%
\begin{pgfscope}%
\definecolor{textcolor}{rgb}{0.000000,0.000000,0.000000}%
\pgfsetstrokecolor{textcolor}%
\pgfsetfillcolor{textcolor}%
\pgftext[x=0.218533in,y=0.514621in,left,base]{\color{textcolor}\sffamily\fontsize{10.000000}{12.000000}\selectfont 0.65}%
\end{pgfscope}%
\begin{pgfscope}%
\pgfsetbuttcap%
\pgfsetroundjoin%
\definecolor{currentfill}{rgb}{0.000000,0.000000,0.000000}%
\pgfsetfillcolor{currentfill}%
\pgfsetlinewidth{0.803000pt}%
\definecolor{currentstroke}{rgb}{0.000000,0.000000,0.000000}%
\pgfsetstrokecolor{currentstroke}%
\pgfsetdash{}{0pt}%
\pgfsys@defobject{currentmarker}{\pgfqpoint{-0.048611in}{0.000000in}}{\pgfqpoint{0.000000in}{0.000000in}}{%
\pgfpathmoveto{\pgfqpoint{0.000000in}{0.000000in}}%
\pgfpathlineto{\pgfqpoint{-0.048611in}{0.000000in}}%
\pgfusepath{stroke,fill}%
}%
\begin{pgfscope}%
\pgfsys@transformshift{0.625000in}{0.985227in}%
\pgfsys@useobject{currentmarker}{}%
\end{pgfscope}%
\end{pgfscope}%
\begin{pgfscope}%
\definecolor{textcolor}{rgb}{0.000000,0.000000,0.000000}%
\pgfsetstrokecolor{textcolor}%
\pgfsetfillcolor{textcolor}%
\pgftext[x=0.218533in,y=0.932465in,left,base]{\color{textcolor}\sffamily\fontsize{10.000000}{12.000000}\selectfont 0.70}%
\end{pgfscope}%
\begin{pgfscope}%
\pgfsetbuttcap%
\pgfsetroundjoin%
\definecolor{currentfill}{rgb}{0.000000,0.000000,0.000000}%
\pgfsetfillcolor{currentfill}%
\pgfsetlinewidth{0.803000pt}%
\definecolor{currentstroke}{rgb}{0.000000,0.000000,0.000000}%
\pgfsetstrokecolor{currentstroke}%
\pgfsetdash{}{0pt}%
\pgfsys@defobject{currentmarker}{\pgfqpoint{-0.048611in}{0.000000in}}{\pgfqpoint{0.000000in}{0.000000in}}{%
\pgfpathmoveto{\pgfqpoint{0.000000in}{0.000000in}}%
\pgfpathlineto{\pgfqpoint{-0.048611in}{0.000000in}}%
\pgfusepath{stroke,fill}%
}%
\begin{pgfscope}%
\pgfsys@transformshift{0.625000in}{1.403071in}%
\pgfsys@useobject{currentmarker}{}%
\end{pgfscope}%
\end{pgfscope}%
\begin{pgfscope}%
\definecolor{textcolor}{rgb}{0.000000,0.000000,0.000000}%
\pgfsetstrokecolor{textcolor}%
\pgfsetfillcolor{textcolor}%
\pgftext[x=0.218533in,y=1.350310in,left,base]{\color{textcolor}\sffamily\fontsize{10.000000}{12.000000}\selectfont 0.75}%
\end{pgfscope}%
\begin{pgfscope}%
\pgfsetbuttcap%
\pgfsetroundjoin%
\definecolor{currentfill}{rgb}{0.000000,0.000000,0.000000}%
\pgfsetfillcolor{currentfill}%
\pgfsetlinewidth{0.803000pt}%
\definecolor{currentstroke}{rgb}{0.000000,0.000000,0.000000}%
\pgfsetstrokecolor{currentstroke}%
\pgfsetdash{}{0pt}%
\pgfsys@defobject{currentmarker}{\pgfqpoint{-0.048611in}{0.000000in}}{\pgfqpoint{0.000000in}{0.000000in}}{%
\pgfpathmoveto{\pgfqpoint{0.000000in}{0.000000in}}%
\pgfpathlineto{\pgfqpoint{-0.048611in}{0.000000in}}%
\pgfusepath{stroke,fill}%
}%
\begin{pgfscope}%
\pgfsys@transformshift{0.625000in}{1.820916in}%
\pgfsys@useobject{currentmarker}{}%
\end{pgfscope}%
\end{pgfscope}%
\begin{pgfscope}%
\definecolor{textcolor}{rgb}{0.000000,0.000000,0.000000}%
\pgfsetstrokecolor{textcolor}%
\pgfsetfillcolor{textcolor}%
\pgftext[x=0.218533in,y=1.768154in,left,base]{\color{textcolor}\sffamily\fontsize{10.000000}{12.000000}\selectfont 0.80}%
\end{pgfscope}%
\begin{pgfscope}%
\pgfsetbuttcap%
\pgfsetroundjoin%
\definecolor{currentfill}{rgb}{0.000000,0.000000,0.000000}%
\pgfsetfillcolor{currentfill}%
\pgfsetlinewidth{0.803000pt}%
\definecolor{currentstroke}{rgb}{0.000000,0.000000,0.000000}%
\pgfsetstrokecolor{currentstroke}%
\pgfsetdash{}{0pt}%
\pgfsys@defobject{currentmarker}{\pgfqpoint{-0.048611in}{0.000000in}}{\pgfqpoint{0.000000in}{0.000000in}}{%
\pgfpathmoveto{\pgfqpoint{0.000000in}{0.000000in}}%
\pgfpathlineto{\pgfqpoint{-0.048611in}{0.000000in}}%
\pgfusepath{stroke,fill}%
}%
\begin{pgfscope}%
\pgfsys@transformshift{0.625000in}{2.238760in}%
\pgfsys@useobject{currentmarker}{}%
\end{pgfscope}%
\end{pgfscope}%
\begin{pgfscope}%
\definecolor{textcolor}{rgb}{0.000000,0.000000,0.000000}%
\pgfsetstrokecolor{textcolor}%
\pgfsetfillcolor{textcolor}%
\pgftext[x=0.218533in,y=2.185999in,left,base]{\color{textcolor}\sffamily\fontsize{10.000000}{12.000000}\selectfont 0.85}%
\end{pgfscope}%
\begin{pgfscope}%
\pgfsetbuttcap%
\pgfsetroundjoin%
\definecolor{currentfill}{rgb}{0.000000,0.000000,0.000000}%
\pgfsetfillcolor{currentfill}%
\pgfsetlinewidth{0.803000pt}%
\definecolor{currentstroke}{rgb}{0.000000,0.000000,0.000000}%
\pgfsetstrokecolor{currentstroke}%
\pgfsetdash{}{0pt}%
\pgfsys@defobject{currentmarker}{\pgfqpoint{-0.048611in}{0.000000in}}{\pgfqpoint{0.000000in}{0.000000in}}{%
\pgfpathmoveto{\pgfqpoint{0.000000in}{0.000000in}}%
\pgfpathlineto{\pgfqpoint{-0.048611in}{0.000000in}}%
\pgfusepath{stroke,fill}%
}%
\begin{pgfscope}%
\pgfsys@transformshift{0.625000in}{2.656605in}%
\pgfsys@useobject{currentmarker}{}%
\end{pgfscope}%
\end{pgfscope}%
\begin{pgfscope}%
\definecolor{textcolor}{rgb}{0.000000,0.000000,0.000000}%
\pgfsetstrokecolor{textcolor}%
\pgfsetfillcolor{textcolor}%
\pgftext[x=0.218533in,y=2.603843in,left,base]{\color{textcolor}\sffamily\fontsize{10.000000}{12.000000}\selectfont 0.90}%
\end{pgfscope}%
\begin{pgfscope}%
\pgfsetbuttcap%
\pgfsetroundjoin%
\definecolor{currentfill}{rgb}{0.000000,0.000000,0.000000}%
\pgfsetfillcolor{currentfill}%
\pgfsetlinewidth{0.803000pt}%
\definecolor{currentstroke}{rgb}{0.000000,0.000000,0.000000}%
\pgfsetstrokecolor{currentstroke}%
\pgfsetdash{}{0pt}%
\pgfsys@defobject{currentmarker}{\pgfqpoint{-0.048611in}{0.000000in}}{\pgfqpoint{0.000000in}{0.000000in}}{%
\pgfpathmoveto{\pgfqpoint{0.000000in}{0.000000in}}%
\pgfpathlineto{\pgfqpoint{-0.048611in}{0.000000in}}%
\pgfusepath{stroke,fill}%
}%
\begin{pgfscope}%
\pgfsys@transformshift{0.625000in}{3.074449in}%
\pgfsys@useobject{currentmarker}{}%
\end{pgfscope}%
\end{pgfscope}%
\begin{pgfscope}%
\definecolor{textcolor}{rgb}{0.000000,0.000000,0.000000}%
\pgfsetstrokecolor{textcolor}%
\pgfsetfillcolor{textcolor}%
\pgftext[x=0.218533in,y=3.021687in,left,base]{\color{textcolor}\sffamily\fontsize{10.000000}{12.000000}\selectfont 0.95}%
\end{pgfscope}%
\begin{pgfscope}%
\pgfsetbuttcap%
\pgfsetroundjoin%
\definecolor{currentfill}{rgb}{0.000000,0.000000,0.000000}%
\pgfsetfillcolor{currentfill}%
\pgfsetlinewidth{0.803000pt}%
\definecolor{currentstroke}{rgb}{0.000000,0.000000,0.000000}%
\pgfsetstrokecolor{currentstroke}%
\pgfsetdash{}{0pt}%
\pgfsys@defobject{currentmarker}{\pgfqpoint{-0.048611in}{0.000000in}}{\pgfqpoint{0.000000in}{0.000000in}}{%
\pgfpathmoveto{\pgfqpoint{0.000000in}{0.000000in}}%
\pgfpathlineto{\pgfqpoint{-0.048611in}{0.000000in}}%
\pgfusepath{stroke,fill}%
}%
\begin{pgfscope}%
\pgfsys@transformshift{0.625000in}{3.492293in}%
\pgfsys@useobject{currentmarker}{}%
\end{pgfscope}%
\end{pgfscope}%
\begin{pgfscope}%
\definecolor{textcolor}{rgb}{0.000000,0.000000,0.000000}%
\pgfsetstrokecolor{textcolor}%
\pgfsetfillcolor{textcolor}%
\pgftext[x=0.218533in,y=3.439532in,left,base]{\color{textcolor}\sffamily\fontsize{10.000000}{12.000000}\selectfont 1.00}%
\end{pgfscope}%
\begin{pgfscope}%
\definecolor{textcolor}{rgb}{0.000000,0.000000,0.000000}%
\pgfsetstrokecolor{textcolor}%
\pgfsetfillcolor{textcolor}%
\pgftext[x=0.162977in,y=1.980000in,,bottom,rotate=90.000000]{\color{textcolor}\sffamily\fontsize{10.000000}{12.000000}\selectfont \(\displaystyle \alpha\)}%
\end{pgfscope}%
\begin{pgfscope}%
\pgfpathrectangle{\pgfqpoint{0.625000in}{0.440000in}}{\pgfqpoint{3.875000in}{3.080000in}}%
\pgfusepath{clip}%
\pgfsetbuttcap%
\pgfsetroundjoin%
\pgfsetlinewidth{1.505625pt}%
\definecolor{currentstroke}{rgb}{0.000000,0.000000,0.000000}%
\pgfsetstrokecolor{currentstroke}%
\pgfsetdash{{5.550000pt}{2.400000pt}}{0.000000pt}%
\pgfpathmoveto{\pgfqpoint{0.801136in}{3.287929in}}%
\pgfpathlineto{\pgfqpoint{2.163913in}{2.521471in}}%
\pgfpathlineto{\pgfqpoint{2.961087in}{2.073122in}}%
\pgfpathlineto{\pgfqpoint{3.526690in}{1.755013in}}%
\pgfpathlineto{\pgfqpoint{3.965406in}{1.508269in}}%
\pgfpathlineto{\pgfqpoint{4.323864in}{1.306664in}}%
\pgfusepath{stroke}%
\end{pgfscope}%
\begin{pgfscope}%
\pgfpathrectangle{\pgfqpoint{0.625000in}{0.440000in}}{\pgfqpoint{3.875000in}{3.080000in}}%
\pgfusepath{clip}%
\pgfsetbuttcap%
\pgfsetroundjoin%
\definecolor{currentfill}{rgb}{0.000000,0.000000,0.000000}%
\pgfsetfillcolor{currentfill}%
\pgfsetlinewidth{1.003750pt}%
\definecolor{currentstroke}{rgb}{0.000000,0.000000,0.000000}%
\pgfsetstrokecolor{currentstroke}%
\pgfsetdash{}{0pt}%
\pgfsys@defobject{currentmarker}{\pgfqpoint{-0.041667in}{-0.041667in}}{\pgfqpoint{0.041667in}{0.041667in}}{%
\pgfpathmoveto{\pgfqpoint{0.000000in}{-0.041667in}}%
\pgfpathcurveto{\pgfqpoint{0.011050in}{-0.041667in}}{\pgfqpoint{0.021649in}{-0.037276in}}{\pgfqpoint{0.029463in}{-0.029463in}}%
\pgfpathcurveto{\pgfqpoint{0.037276in}{-0.021649in}}{\pgfqpoint{0.041667in}{-0.011050in}}{\pgfqpoint{0.041667in}{0.000000in}}%
\pgfpathcurveto{\pgfqpoint{0.041667in}{0.011050in}}{\pgfqpoint{0.037276in}{0.021649in}}{\pgfqpoint{0.029463in}{0.029463in}}%
\pgfpathcurveto{\pgfqpoint{0.021649in}{0.037276in}}{\pgfqpoint{0.011050in}{0.041667in}}{\pgfqpoint{0.000000in}{0.041667in}}%
\pgfpathcurveto{\pgfqpoint{-0.011050in}{0.041667in}}{\pgfqpoint{-0.021649in}{0.037276in}}{\pgfqpoint{-0.029463in}{0.029463in}}%
\pgfpathcurveto{\pgfqpoint{-0.037276in}{0.021649in}}{\pgfqpoint{-0.041667in}{0.011050in}}{\pgfqpoint{-0.041667in}{0.000000in}}%
\pgfpathcurveto{\pgfqpoint{-0.041667in}{-0.011050in}}{\pgfqpoint{-0.037276in}{-0.021649in}}{\pgfqpoint{-0.029463in}{-0.029463in}}%
\pgfpathcurveto{\pgfqpoint{-0.021649in}{-0.037276in}}{\pgfqpoint{-0.011050in}{-0.041667in}}{\pgfqpoint{0.000000in}{-0.041667in}}%
\pgfpathclose%
\pgfusepath{stroke,fill}%
}%
\begin{pgfscope}%
\pgfsys@transformshift{0.801136in}{2.658440in}%
\pgfsys@useobject{currentmarker}{}%
\end{pgfscope}%
\begin{pgfscope}%
\pgfsys@transformshift{2.163913in}{2.797698in}%
\pgfsys@useobject{currentmarker}{}%
\end{pgfscope}%
\begin{pgfscope}%
\pgfsys@transformshift{2.961087in}{3.380000in}%
\pgfsys@useobject{currentmarker}{}%
\end{pgfscope}%
\begin{pgfscope}%
\pgfsys@transformshift{3.526690in}{1.575520in}%
\pgfsys@useobject{currentmarker}{}%
\end{pgfscope}%
\begin{pgfscope}%
\pgfsys@transformshift{3.965406in}{1.460810in}%
\pgfsys@useobject{currentmarker}{}%
\end{pgfscope}%
\begin{pgfscope}%
\pgfsys@transformshift{4.323864in}{0.580000in}%
\pgfsys@useobject{currentmarker}{}%
\end{pgfscope}%
\end{pgfscope}%
\begin{pgfscope}%
\pgfsetrectcap%
\pgfsetmiterjoin%
\pgfsetlinewidth{0.803000pt}%
\definecolor{currentstroke}{rgb}{0.000000,0.000000,0.000000}%
\pgfsetstrokecolor{currentstroke}%
\pgfsetdash{}{0pt}%
\pgfpathmoveto{\pgfqpoint{0.625000in}{0.440000in}}%
\pgfpathlineto{\pgfqpoint{0.625000in}{3.520000in}}%
\pgfusepath{stroke}%
\end{pgfscope}%
\begin{pgfscope}%
\pgfsetrectcap%
\pgfsetmiterjoin%
\pgfsetlinewidth{0.803000pt}%
\definecolor{currentstroke}{rgb}{0.000000,0.000000,0.000000}%
\pgfsetstrokecolor{currentstroke}%
\pgfsetdash{}{0pt}%
\pgfpathmoveto{\pgfqpoint{4.500000in}{0.440000in}}%
\pgfpathlineto{\pgfqpoint{4.500000in}{3.520000in}}%
\pgfusepath{stroke}%
\end{pgfscope}%
\begin{pgfscope}%
\pgfsetrectcap%
\pgfsetmiterjoin%
\pgfsetlinewidth{0.803000pt}%
\definecolor{currentstroke}{rgb}{0.000000,0.000000,0.000000}%
\pgfsetstrokecolor{currentstroke}%
\pgfsetdash{}{0pt}%
\pgfpathmoveto{\pgfqpoint{0.625000in}{0.440000in}}%
\pgfpathlineto{\pgfqpoint{4.500000in}{0.440000in}}%
\pgfusepath{stroke}%
\end{pgfscope}%
\begin{pgfscope}%
\pgfsetrectcap%
\pgfsetmiterjoin%
\pgfsetlinewidth{0.803000pt}%
\definecolor{currentstroke}{rgb}{0.000000,0.000000,0.000000}%
\pgfsetstrokecolor{currentstroke}%
\pgfsetdash{}{0pt}%
\pgfpathmoveto{\pgfqpoint{0.625000in}{3.520000in}}%
\pgfpathlineto{\pgfqpoint{4.500000in}{3.520000in}}%
\pgfusepath{stroke}%
\end{pgfscope}%
\begin{pgfscope}%
\pgfsetbuttcap%
\pgfsetmiterjoin%
\definecolor{currentfill}{rgb}{1.000000,1.000000,1.000000}%
\pgfsetfillcolor{currentfill}%
\pgfsetfillopacity{0.800000}%
\pgfsetlinewidth{1.003750pt}%
\definecolor{currentstroke}{rgb}{0.800000,0.800000,0.800000}%
\pgfsetstrokecolor{currentstroke}%
\pgfsetstrokeopacity{0.800000}%
\pgfsetdash{}{0pt}%
\pgfpathmoveto{\pgfqpoint{0.722222in}{0.509444in}}%
\pgfpathlineto{\pgfqpoint{2.807368in}{0.509444in}}%
\pgfpathquadraticcurveto{\pgfqpoint{2.835146in}{0.509444in}}{\pgfqpoint{2.835146in}{0.537222in}}%
\pgfpathlineto{\pgfqpoint{2.835146in}{0.733023in}}%
\pgfpathquadraticcurveto{\pgfqpoint{2.835146in}{0.760801in}}{\pgfqpoint{2.807368in}{0.760801in}}%
\pgfpathlineto{\pgfqpoint{0.722222in}{0.760801in}}%
\pgfpathquadraticcurveto{\pgfqpoint{0.694444in}{0.760801in}}{\pgfqpoint{0.694444in}{0.733023in}}%
\pgfpathlineto{\pgfqpoint{0.694444in}{0.537222in}}%
\pgfpathquadraticcurveto{\pgfqpoint{0.694444in}{0.509444in}}{\pgfqpoint{0.722222in}{0.509444in}}%
\pgfpathclose%
\pgfusepath{stroke,fill}%
\end{pgfscope}%
\begin{pgfscope}%
\pgfsetbuttcap%
\pgfsetroundjoin%
\pgfsetlinewidth{1.505625pt}%
\definecolor{currentstroke}{rgb}{0.000000,0.000000,0.000000}%
\pgfsetstrokecolor{currentstroke}%
\pgfsetdash{{5.550000pt}{2.400000pt}}{0.000000pt}%
\pgfpathmoveto{\pgfqpoint{0.750000in}{0.648333in}}%
\pgfpathlineto{\pgfqpoint{1.027778in}{0.648333in}}%
\pgfusepath{stroke}%
\end{pgfscope}%
\begin{pgfscope}%
\definecolor{textcolor}{rgb}{0.000000,0.000000,0.000000}%
\pgfsetstrokecolor{textcolor}%
\pgfsetfillcolor{textcolor}%
\pgftext[x=1.138889in,y=0.599722in,left,base]{\color{textcolor}\sffamily\fontsize{10.000000}{12.000000}\selectfont \(\displaystyle \alpha = 1.2802 - 0.1323\log(k)\)}%
\end{pgfscope}%
\end{pgfpicture}%
\makeatother%
\endgroup%

    \end{subfigure}
\caption[The computed Quasi-Monte-Carlo convergence rate for $Q(u) =  \int_D u$.]{Plots of the computed values of $C$ (top) and $\alpha$ (bottom) against $k$ in \cref{eq:qmcerrorform} for $Q(u) = \int_D u$. Observe the $x$-axes are on a $\log_{10}$ scale, but $\loge$ is the natural logarithm. \label{fig:integralCalpha}}
\end{figure}

\begin{figure}[h]
    \centering
    \begin{subfigure}{\textwidth}
            \centering
%% Creator: Matplotlib, PGF backend
%%
%% To include the figure in your LaTeX document, write
%%   \input{<filename>.pgf}
%%
%% Make sure the required packages are loaded in your preamble
%%   \usepackage{pgf}
%%
%% Figures using additional raster images can only be included by \input if
%% they are in the same directory as the main LaTeX file. For loading figures
%% from other directories you can use the `import` package
%%   \usepackage{import}
%% and then include the figures with
%%   \import{<path to file>}{<filename>.pgf}
%%
%% Matplotlib used the following preamble
%%   \usepackage{fontspec}
%%   \setmainfont{DejaVuSerif.ttf}[Path=/home/owen/progs/firedrake-complex/firedrake/lib/python3.5/site-packages/matplotlib/mpl-data/fonts/ttf/]
%%   \setsansfont{DejaVuSans.ttf}[Path=/home/owen/progs/firedrake-complex/firedrake/lib/python3.5/site-packages/matplotlib/mpl-data/fonts/ttf/]
%%   \setmonofont{DejaVuSansMono.ttf}[Path=/home/owen/progs/firedrake-complex/firedrake/lib/python3.5/site-packages/matplotlib/mpl-data/fonts/ttf/]
%%
\begingroup%
\makeatletter%
\begin{pgfpicture}%
\pgfpathrectangle{\pgfpointorigin}{\pgfqpoint{5.000000in}{4.000000in}}%
\pgfusepath{use as bounding box, clip}%
\begin{pgfscope}%
\pgfsetbuttcap%
\pgfsetmiterjoin%
\definecolor{currentfill}{rgb}{1.000000,1.000000,1.000000}%
\pgfsetfillcolor{currentfill}%
\pgfsetlinewidth{0.000000pt}%
\definecolor{currentstroke}{rgb}{1.000000,1.000000,1.000000}%
\pgfsetstrokecolor{currentstroke}%
\pgfsetdash{}{0pt}%
\pgfpathmoveto{\pgfqpoint{0.000000in}{0.000000in}}%
\pgfpathlineto{\pgfqpoint{5.000000in}{0.000000in}}%
\pgfpathlineto{\pgfqpoint{5.000000in}{4.000000in}}%
\pgfpathlineto{\pgfqpoint{0.000000in}{4.000000in}}%
\pgfpathclose%
\pgfusepath{fill}%
\end{pgfscope}%
\begin{pgfscope}%
\pgfsetbuttcap%
\pgfsetmiterjoin%
\definecolor{currentfill}{rgb}{1.000000,1.000000,1.000000}%
\pgfsetfillcolor{currentfill}%
\pgfsetlinewidth{0.000000pt}%
\definecolor{currentstroke}{rgb}{0.000000,0.000000,0.000000}%
\pgfsetstrokecolor{currentstroke}%
\pgfsetstrokeopacity{0.000000}%
\pgfsetdash{}{0pt}%
\pgfpathmoveto{\pgfqpoint{0.827140in}{0.582778in}}%
\pgfpathlineto{\pgfqpoint{4.810222in}{0.582778in}}%
\pgfpathlineto{\pgfqpoint{4.810222in}{3.808710in}}%
\pgfpathlineto{\pgfqpoint{0.827140in}{3.808710in}}%
\pgfpathclose%
\pgfusepath{fill}%
\end{pgfscope}%
\begin{pgfscope}%
\pgfsetbuttcap%
\pgfsetroundjoin%
\definecolor{currentfill}{rgb}{0.000000,0.000000,0.000000}%
\pgfsetfillcolor{currentfill}%
\pgfsetlinewidth{0.803000pt}%
\definecolor{currentstroke}{rgb}{0.000000,0.000000,0.000000}%
\pgfsetstrokecolor{currentstroke}%
\pgfsetdash{}{0pt}%
\pgfsys@defobject{currentmarker}{\pgfqpoint{0.000000in}{-0.048611in}}{\pgfqpoint{0.000000in}{0.000000in}}{%
\pgfpathmoveto{\pgfqpoint{0.000000in}{0.000000in}}%
\pgfpathlineto{\pgfqpoint{0.000000in}{-0.048611in}}%
\pgfusepath{stroke,fill}%
}%
\begin{pgfscope}%
\pgfsys@transformshift{1.008189in}{0.582778in}%
\pgfsys@useobject{currentmarker}{}%
\end{pgfscope}%
\end{pgfscope}%
\begin{pgfscope}%
\definecolor{textcolor}{rgb}{0.000000,0.000000,0.000000}%
\pgfsetstrokecolor{textcolor}%
\pgfsetfillcolor{textcolor}%
\pgftext[x=1.008189in,y=0.485556in,,top]{\color{textcolor}\sffamily\fontsize{10.000000}{12.000000}\selectfont \(\displaystyle 10^{1}\)}%
\end{pgfscope}%
\begin{pgfscope}%
\pgfsetbuttcap%
\pgfsetroundjoin%
\definecolor{currentfill}{rgb}{0.000000,0.000000,0.000000}%
\pgfsetfillcolor{currentfill}%
\pgfsetlinewidth{0.602250pt}%
\definecolor{currentstroke}{rgb}{0.000000,0.000000,0.000000}%
\pgfsetstrokecolor{currentstroke}%
\pgfsetdash{}{0pt}%
\pgfsys@defobject{currentmarker}{\pgfqpoint{0.000000in}{-0.027778in}}{\pgfqpoint{0.000000in}{0.000000in}}{%
\pgfpathmoveto{\pgfqpoint{0.000000in}{0.000000in}}%
\pgfpathlineto{\pgfqpoint{0.000000in}{-0.027778in}}%
\pgfusepath{stroke,fill}%
}%
\begin{pgfscope}%
\pgfsys@transformshift{2.408977in}{0.582778in}%
\pgfsys@useobject{currentmarker}{}%
\end{pgfscope}%
\end{pgfscope}%
\begin{pgfscope}%
\definecolor{textcolor}{rgb}{0.000000,0.000000,0.000000}%
\pgfsetstrokecolor{textcolor}%
\pgfsetfillcolor{textcolor}%
\pgftext[x=2.408977in,y=0.507778in,,top]{\color{textcolor}\sffamily\fontsize{10.000000}{12.000000}\selectfont \(\displaystyle 2\times10^{1}\)}%
\end{pgfscope}%
\begin{pgfscope}%
\pgfsetbuttcap%
\pgfsetroundjoin%
\definecolor{currentfill}{rgb}{0.000000,0.000000,0.000000}%
\pgfsetfillcolor{currentfill}%
\pgfsetlinewidth{0.602250pt}%
\definecolor{currentstroke}{rgb}{0.000000,0.000000,0.000000}%
\pgfsetstrokecolor{currentstroke}%
\pgfsetdash{}{0pt}%
\pgfsys@defobject{currentmarker}{\pgfqpoint{0.000000in}{-0.027778in}}{\pgfqpoint{0.000000in}{0.000000in}}{%
\pgfpathmoveto{\pgfqpoint{0.000000in}{0.000000in}}%
\pgfpathlineto{\pgfqpoint{0.000000in}{-0.027778in}}%
\pgfusepath{stroke,fill}%
}%
\begin{pgfscope}%
\pgfsys@transformshift{3.228385in}{0.582778in}%
\pgfsys@useobject{currentmarker}{}%
\end{pgfscope}%
\end{pgfscope}%
\begin{pgfscope}%
\definecolor{textcolor}{rgb}{0.000000,0.000000,0.000000}%
\pgfsetstrokecolor{textcolor}%
\pgfsetfillcolor{textcolor}%
\pgftext[x=3.228385in,y=0.507778in,,top]{\color{textcolor}\sffamily\fontsize{10.000000}{12.000000}\selectfont \(\displaystyle 3\times10^{1}\)}%
\end{pgfscope}%
\begin{pgfscope}%
\pgfsetbuttcap%
\pgfsetroundjoin%
\definecolor{currentfill}{rgb}{0.000000,0.000000,0.000000}%
\pgfsetfillcolor{currentfill}%
\pgfsetlinewidth{0.602250pt}%
\definecolor{currentstroke}{rgb}{0.000000,0.000000,0.000000}%
\pgfsetstrokecolor{currentstroke}%
\pgfsetdash{}{0pt}%
\pgfsys@defobject{currentmarker}{\pgfqpoint{0.000000in}{-0.027778in}}{\pgfqpoint{0.000000in}{0.000000in}}{%
\pgfpathmoveto{\pgfqpoint{0.000000in}{0.000000in}}%
\pgfpathlineto{\pgfqpoint{0.000000in}{-0.027778in}}%
\pgfusepath{stroke,fill}%
}%
\begin{pgfscope}%
\pgfsys@transformshift{3.809764in}{0.582778in}%
\pgfsys@useobject{currentmarker}{}%
\end{pgfscope}%
\end{pgfscope}%
\begin{pgfscope}%
\definecolor{textcolor}{rgb}{0.000000,0.000000,0.000000}%
\pgfsetstrokecolor{textcolor}%
\pgfsetfillcolor{textcolor}%
\pgftext[x=3.809764in,y=0.507778in,,top]{\color{textcolor}\sffamily\fontsize{10.000000}{12.000000}\selectfont \(\displaystyle 4\times10^{1}\)}%
\end{pgfscope}%
\begin{pgfscope}%
\pgfsetbuttcap%
\pgfsetroundjoin%
\definecolor{currentfill}{rgb}{0.000000,0.000000,0.000000}%
\pgfsetfillcolor{currentfill}%
\pgfsetlinewidth{0.602250pt}%
\definecolor{currentstroke}{rgb}{0.000000,0.000000,0.000000}%
\pgfsetstrokecolor{currentstroke}%
\pgfsetdash{}{0pt}%
\pgfsys@defobject{currentmarker}{\pgfqpoint{0.000000in}{-0.027778in}}{\pgfqpoint{0.000000in}{0.000000in}}{%
\pgfpathmoveto{\pgfqpoint{0.000000in}{0.000000in}}%
\pgfpathlineto{\pgfqpoint{0.000000in}{-0.027778in}}%
\pgfusepath{stroke,fill}%
}%
\begin{pgfscope}%
\pgfsys@transformshift{4.260717in}{0.582778in}%
\pgfsys@useobject{currentmarker}{}%
\end{pgfscope}%
\end{pgfscope}%
\begin{pgfscope}%
\pgfsetbuttcap%
\pgfsetroundjoin%
\definecolor{currentfill}{rgb}{0.000000,0.000000,0.000000}%
\pgfsetfillcolor{currentfill}%
\pgfsetlinewidth{0.602250pt}%
\definecolor{currentstroke}{rgb}{0.000000,0.000000,0.000000}%
\pgfsetstrokecolor{currentstroke}%
\pgfsetdash{}{0pt}%
\pgfsys@defobject{currentmarker}{\pgfqpoint{0.000000in}{-0.027778in}}{\pgfqpoint{0.000000in}{0.000000in}}{%
\pgfpathmoveto{\pgfqpoint{0.000000in}{0.000000in}}%
\pgfpathlineto{\pgfqpoint{0.000000in}{-0.027778in}}%
\pgfusepath{stroke,fill}%
}%
\begin{pgfscope}%
\pgfsys@transformshift{4.629172in}{0.582778in}%
\pgfsys@useobject{currentmarker}{}%
\end{pgfscope}%
\end{pgfscope}%
\begin{pgfscope}%
\definecolor{textcolor}{rgb}{0.000000,0.000000,0.000000}%
\pgfsetstrokecolor{textcolor}%
\pgfsetfillcolor{textcolor}%
\pgftext[x=4.629172in,y=0.507778in,,top]{\color{textcolor}\sffamily\fontsize{10.000000}{12.000000}\selectfont \(\displaystyle 6\times10^{1}\)}%
\end{pgfscope}%
\begin{pgfscope}%
\definecolor{textcolor}{rgb}{0.000000,0.000000,0.000000}%
\pgfsetstrokecolor{textcolor}%
\pgfsetfillcolor{textcolor}%
\pgftext[x=2.818681in,y=0.295587in,,top]{\color{textcolor}\sffamily\fontsize{10.000000}{12.000000}\selectfont \(\displaystyle k\)}%
\end{pgfscope}%
\begin{pgfscope}%
\pgfsetbuttcap%
\pgfsetroundjoin%
\definecolor{currentfill}{rgb}{0.000000,0.000000,0.000000}%
\pgfsetfillcolor{currentfill}%
\pgfsetlinewidth{0.803000pt}%
\definecolor{currentstroke}{rgb}{0.000000,0.000000,0.000000}%
\pgfsetstrokecolor{currentstroke}%
\pgfsetdash{}{0pt}%
\pgfsys@defobject{currentmarker}{\pgfqpoint{-0.048611in}{0.000000in}}{\pgfqpoint{0.000000in}{0.000000in}}{%
\pgfpathmoveto{\pgfqpoint{0.000000in}{0.000000in}}%
\pgfpathlineto{\pgfqpoint{-0.048611in}{0.000000in}}%
\pgfusepath{stroke,fill}%
}%
\begin{pgfscope}%
\pgfsys@transformshift{0.827140in}{1.023971in}%
\pgfsys@useobject{currentmarker}{}%
\end{pgfscope}%
\end{pgfscope}%
\begin{pgfscope}%
\definecolor{textcolor}{rgb}{0.000000,0.000000,0.000000}%
\pgfsetstrokecolor{textcolor}%
\pgfsetfillcolor{textcolor}%
\pgftext[x=0.344114in,y=0.971210in,left,base]{\color{textcolor}\sffamily\fontsize{10.000000}{12.000000}\selectfont \(\displaystyle 0.0080\)}%
\end{pgfscope}%
\begin{pgfscope}%
\pgfsetbuttcap%
\pgfsetroundjoin%
\definecolor{currentfill}{rgb}{0.000000,0.000000,0.000000}%
\pgfsetfillcolor{currentfill}%
\pgfsetlinewidth{0.803000pt}%
\definecolor{currentstroke}{rgb}{0.000000,0.000000,0.000000}%
\pgfsetstrokecolor{currentstroke}%
\pgfsetdash{}{0pt}%
\pgfsys@defobject{currentmarker}{\pgfqpoint{-0.048611in}{0.000000in}}{\pgfqpoint{0.000000in}{0.000000in}}{%
\pgfpathmoveto{\pgfqpoint{0.000000in}{0.000000in}}%
\pgfpathlineto{\pgfqpoint{-0.048611in}{0.000000in}}%
\pgfusepath{stroke,fill}%
}%
\begin{pgfscope}%
\pgfsys@transformshift{0.827140in}{1.716566in}%
\pgfsys@useobject{currentmarker}{}%
\end{pgfscope}%
\end{pgfscope}%
\begin{pgfscope}%
\definecolor{textcolor}{rgb}{0.000000,0.000000,0.000000}%
\pgfsetstrokecolor{textcolor}%
\pgfsetfillcolor{textcolor}%
\pgftext[x=0.344114in,y=1.663804in,left,base]{\color{textcolor}\sffamily\fontsize{10.000000}{12.000000}\selectfont \(\displaystyle 0.0085\)}%
\end{pgfscope}%
\begin{pgfscope}%
\pgfsetbuttcap%
\pgfsetroundjoin%
\definecolor{currentfill}{rgb}{0.000000,0.000000,0.000000}%
\pgfsetfillcolor{currentfill}%
\pgfsetlinewidth{0.803000pt}%
\definecolor{currentstroke}{rgb}{0.000000,0.000000,0.000000}%
\pgfsetstrokecolor{currentstroke}%
\pgfsetdash{}{0pt}%
\pgfsys@defobject{currentmarker}{\pgfqpoint{-0.048611in}{0.000000in}}{\pgfqpoint{0.000000in}{0.000000in}}{%
\pgfpathmoveto{\pgfqpoint{0.000000in}{0.000000in}}%
\pgfpathlineto{\pgfqpoint{-0.048611in}{0.000000in}}%
\pgfusepath{stroke,fill}%
}%
\begin{pgfscope}%
\pgfsys@transformshift{0.827140in}{2.409161in}%
\pgfsys@useobject{currentmarker}{}%
\end{pgfscope}%
\end{pgfscope}%
\begin{pgfscope}%
\definecolor{textcolor}{rgb}{0.000000,0.000000,0.000000}%
\pgfsetstrokecolor{textcolor}%
\pgfsetfillcolor{textcolor}%
\pgftext[x=0.344114in,y=2.356399in,left,base]{\color{textcolor}\sffamily\fontsize{10.000000}{12.000000}\selectfont \(\displaystyle 0.0090\)}%
\end{pgfscope}%
\begin{pgfscope}%
\pgfsetbuttcap%
\pgfsetroundjoin%
\definecolor{currentfill}{rgb}{0.000000,0.000000,0.000000}%
\pgfsetfillcolor{currentfill}%
\pgfsetlinewidth{0.803000pt}%
\definecolor{currentstroke}{rgb}{0.000000,0.000000,0.000000}%
\pgfsetstrokecolor{currentstroke}%
\pgfsetdash{}{0pt}%
\pgfsys@defobject{currentmarker}{\pgfqpoint{-0.048611in}{0.000000in}}{\pgfqpoint{0.000000in}{0.000000in}}{%
\pgfpathmoveto{\pgfqpoint{0.000000in}{0.000000in}}%
\pgfpathlineto{\pgfqpoint{-0.048611in}{0.000000in}}%
\pgfusepath{stroke,fill}%
}%
\begin{pgfscope}%
\pgfsys@transformshift{0.827140in}{3.101756in}%
\pgfsys@useobject{currentmarker}{}%
\end{pgfscope}%
\end{pgfscope}%
\begin{pgfscope}%
\definecolor{textcolor}{rgb}{0.000000,0.000000,0.000000}%
\pgfsetstrokecolor{textcolor}%
\pgfsetfillcolor{textcolor}%
\pgftext[x=0.344114in,y=3.048994in,left,base]{\color{textcolor}\sffamily\fontsize{10.000000}{12.000000}\selectfont \(\displaystyle 0.0095\)}%
\end{pgfscope}%
\begin{pgfscope}%
\pgfsetbuttcap%
\pgfsetroundjoin%
\definecolor{currentfill}{rgb}{0.000000,0.000000,0.000000}%
\pgfsetfillcolor{currentfill}%
\pgfsetlinewidth{0.803000pt}%
\definecolor{currentstroke}{rgb}{0.000000,0.000000,0.000000}%
\pgfsetstrokecolor{currentstroke}%
\pgfsetdash{}{0pt}%
\pgfsys@defobject{currentmarker}{\pgfqpoint{-0.048611in}{0.000000in}}{\pgfqpoint{0.000000in}{0.000000in}}{%
\pgfpathmoveto{\pgfqpoint{0.000000in}{0.000000in}}%
\pgfpathlineto{\pgfqpoint{-0.048611in}{0.000000in}}%
\pgfusepath{stroke,fill}%
}%
\begin{pgfscope}%
\pgfsys@transformshift{0.827140in}{3.794350in}%
\pgfsys@useobject{currentmarker}{}%
\end{pgfscope}%
\end{pgfscope}%
\begin{pgfscope}%
\definecolor{textcolor}{rgb}{0.000000,0.000000,0.000000}%
\pgfsetstrokecolor{textcolor}%
\pgfsetfillcolor{textcolor}%
\pgftext[x=0.344114in,y=3.741589in,left,base]{\color{textcolor}\sffamily\fontsize{10.000000}{12.000000}\selectfont \(\displaystyle 0.0100\)}%
\end{pgfscope}%
\begin{pgfscope}%
\definecolor{textcolor}{rgb}{0.000000,0.000000,0.000000}%
\pgfsetstrokecolor{textcolor}%
\pgfsetfillcolor{textcolor}%
\pgftext[x=0.288559in,y=2.195744in,,bottom,rotate=90.000000]{\color{textcolor}\sffamily\fontsize{10.000000}{12.000000}\selectfont \(\displaystyle C\)}%
\end{pgfscope}%
\begin{pgfscope}%
\pgfpathrectangle{\pgfqpoint{0.827140in}{0.582778in}}{\pgfqpoint{3.983082in}{3.225932in}}%
\pgfusepath{clip}%
\pgfsetbuttcap%
\pgfsetroundjoin%
\definecolor{currentfill}{rgb}{0.000000,0.000000,0.000000}%
\pgfsetfillcolor{currentfill}%
\pgfsetlinewidth{1.003750pt}%
\definecolor{currentstroke}{rgb}{0.000000,0.000000,0.000000}%
\pgfsetstrokecolor{currentstroke}%
\pgfsetdash{}{0pt}%
\pgfsys@defobject{currentmarker}{\pgfqpoint{-0.041667in}{-0.041667in}}{\pgfqpoint{0.041667in}{0.041667in}}{%
\pgfpathmoveto{\pgfqpoint{0.000000in}{-0.041667in}}%
\pgfpathcurveto{\pgfqpoint{0.011050in}{-0.041667in}}{\pgfqpoint{0.021649in}{-0.037276in}}{\pgfqpoint{0.029463in}{-0.029463in}}%
\pgfpathcurveto{\pgfqpoint{0.037276in}{-0.021649in}}{\pgfqpoint{0.041667in}{-0.011050in}}{\pgfqpoint{0.041667in}{0.000000in}}%
\pgfpathcurveto{\pgfqpoint{0.041667in}{0.011050in}}{\pgfqpoint{0.037276in}{0.021649in}}{\pgfqpoint{0.029463in}{0.029463in}}%
\pgfpathcurveto{\pgfqpoint{0.021649in}{0.037276in}}{\pgfqpoint{0.011050in}{0.041667in}}{\pgfqpoint{0.000000in}{0.041667in}}%
\pgfpathcurveto{\pgfqpoint{-0.011050in}{0.041667in}}{\pgfqpoint{-0.021649in}{0.037276in}}{\pgfqpoint{-0.029463in}{0.029463in}}%
\pgfpathcurveto{\pgfqpoint{-0.037276in}{0.021649in}}{\pgfqpoint{-0.041667in}{0.011050in}}{\pgfqpoint{-0.041667in}{0.000000in}}%
\pgfpathcurveto{\pgfqpoint{-0.041667in}{-0.011050in}}{\pgfqpoint{-0.037276in}{-0.021649in}}{\pgfqpoint{-0.029463in}{-0.029463in}}%
\pgfpathcurveto{\pgfqpoint{-0.021649in}{-0.037276in}}{\pgfqpoint{-0.011050in}{-0.041667in}}{\pgfqpoint{0.000000in}{-0.041667in}}%
\pgfpathclose%
\pgfusepath{stroke,fill}%
}%
\begin{pgfscope}%
\pgfsys@transformshift{1.008189in}{0.784599in}%
\pgfsys@useobject{currentmarker}{}%
\end{pgfscope}%
\begin{pgfscope}%
\pgfsys@transformshift{2.408977in}{1.228252in}%
\pgfsys@useobject{currentmarker}{}%
\end{pgfscope}%
\begin{pgfscope}%
\pgfsys@transformshift{3.228385in}{0.729411in}%
\pgfsys@useobject{currentmarker}{}%
\end{pgfscope}%
\begin{pgfscope}%
\pgfsys@transformshift{3.809764in}{3.662077in}%
\pgfsys@useobject{currentmarker}{}%
\end{pgfscope}%
\begin{pgfscope}%
\pgfsys@transformshift{4.260717in}{2.077043in}%
\pgfsys@useobject{currentmarker}{}%
\end{pgfscope}%
\begin{pgfscope}%
\pgfsys@transformshift{4.629172in}{1.993444in}%
\pgfsys@useobject{currentmarker}{}%
\end{pgfscope}%
\end{pgfscope}%
\begin{pgfscope}%
\pgfsetrectcap%
\pgfsetmiterjoin%
\pgfsetlinewidth{0.803000pt}%
\definecolor{currentstroke}{rgb}{0.000000,0.000000,0.000000}%
\pgfsetstrokecolor{currentstroke}%
\pgfsetdash{}{0pt}%
\pgfpathmoveto{\pgfqpoint{0.827140in}{0.582778in}}%
\pgfpathlineto{\pgfqpoint{0.827140in}{3.808710in}}%
\pgfusepath{stroke}%
\end{pgfscope}%
\begin{pgfscope}%
\pgfsetrectcap%
\pgfsetmiterjoin%
\pgfsetlinewidth{0.000000pt}%
\definecolor{currentstroke}{rgb}{0.000000,0.000000,0.000000}%
\pgfsetstrokecolor{currentstroke}%
\pgfsetstrokeopacity{0.000000}%
\pgfsetdash{}{0pt}%
\pgfpathmoveto{\pgfqpoint{4.810222in}{0.582778in}}%
\pgfpathlineto{\pgfqpoint{4.810222in}{3.808710in}}%
\pgfusepath{}%
\end{pgfscope}%
\begin{pgfscope}%
\pgfsetrectcap%
\pgfsetmiterjoin%
\pgfsetlinewidth{0.803000pt}%
\definecolor{currentstroke}{rgb}{0.000000,0.000000,0.000000}%
\pgfsetstrokecolor{currentstroke}%
\pgfsetdash{}{0pt}%
\pgfpathmoveto{\pgfqpoint{0.827140in}{0.582778in}}%
\pgfpathlineto{\pgfqpoint{4.810222in}{0.582778in}}%
\pgfusepath{stroke}%
\end{pgfscope}%
\begin{pgfscope}%
\pgfsetrectcap%
\pgfsetmiterjoin%
\pgfsetlinewidth{0.000000pt}%
\definecolor{currentstroke}{rgb}{0.000000,0.000000,0.000000}%
\pgfsetstrokecolor{currentstroke}%
\pgfsetstrokeopacity{0.000000}%
\pgfsetdash{}{0pt}%
\pgfpathmoveto{\pgfqpoint{0.827140in}{3.808710in}}%
\pgfpathlineto{\pgfqpoint{4.810222in}{3.808710in}}%
\pgfusepath{}%
\end{pgfscope}%
\end{pgfpicture}%
\makeatother%
\endgroup%

  \end{subfigure}
    \begin{subfigure}{\textwidth}
                \centering
%% Creator: Matplotlib, PGF backend
%%
%% To include the figure in your LaTeX document, write
%%   \input{<filename>.pgf}
%%
%% Make sure the required packages are loaded in your preamble
%%   \usepackage{pgf}
%%
%% Figures using additional raster images can only be included by \input if
%% they are in the same directory as the main LaTeX file. For loading figures
%% from other directories you can use the `import` package
%%   \usepackage{import}
%% and then include the figures with
%%   \import{<path to file>}{<filename>.pgf}
%%
%% Matplotlib used the following preamble
%%   \usepackage{fontspec}
%%   \setmainfont{DejaVuSerif.ttf}[Path=/home/owen/progs/firedrake-complex/firedrake/lib/python3.5/site-packages/matplotlib/mpl-data/fonts/ttf/]
%%   \setsansfont{DejaVuSans.ttf}[Path=/home/owen/progs/firedrake-complex/firedrake/lib/python3.5/site-packages/matplotlib/mpl-data/fonts/ttf/]
%%   \setmonofont{DejaVuSansMono.ttf}[Path=/home/owen/progs/firedrake-complex/firedrake/lib/python3.5/site-packages/matplotlib/mpl-data/fonts/ttf/]
%%
\begingroup%
\makeatletter%
\begin{pgfpicture}%
\pgfpathrectangle{\pgfpointorigin}{\pgfqpoint{3.000000in}{3.000000in}}%
\pgfusepath{use as bounding box, clip}%
\begin{pgfscope}%
\pgfsetbuttcap%
\pgfsetmiterjoin%
\definecolor{currentfill}{rgb}{1.000000,1.000000,1.000000}%
\pgfsetfillcolor{currentfill}%
\pgfsetlinewidth{0.000000pt}%
\definecolor{currentstroke}{rgb}{1.000000,1.000000,1.000000}%
\pgfsetstrokecolor{currentstroke}%
\pgfsetdash{}{0pt}%
\pgfpathmoveto{\pgfqpoint{0.000000in}{0.000000in}}%
\pgfpathlineto{\pgfqpoint{3.000000in}{0.000000in}}%
\pgfpathlineto{\pgfqpoint{3.000000in}{3.000000in}}%
\pgfpathlineto{\pgfqpoint{0.000000in}{3.000000in}}%
\pgfpathclose%
\pgfusepath{fill}%
\end{pgfscope}%
\begin{pgfscope}%
\pgfsetbuttcap%
\pgfsetmiterjoin%
\definecolor{currentfill}{rgb}{1.000000,1.000000,1.000000}%
\pgfsetfillcolor{currentfill}%
\pgfsetlinewidth{0.000000pt}%
\definecolor{currentstroke}{rgb}{0.000000,0.000000,0.000000}%
\pgfsetstrokecolor{currentstroke}%
\pgfsetstrokeopacity{0.000000}%
\pgfsetdash{}{0pt}%
\pgfpathmoveto{\pgfqpoint{0.375000in}{0.330000in}}%
\pgfpathlineto{\pgfqpoint{2.700000in}{0.330000in}}%
\pgfpathlineto{\pgfqpoint{2.700000in}{2.640000in}}%
\pgfpathlineto{\pgfqpoint{0.375000in}{2.640000in}}%
\pgfpathclose%
\pgfusepath{fill}%
\end{pgfscope}%
\begin{pgfscope}%
\pgfsetbuttcap%
\pgfsetroundjoin%
\definecolor{currentfill}{rgb}{0.000000,0.000000,0.000000}%
\pgfsetfillcolor{currentfill}%
\pgfsetlinewidth{0.803000pt}%
\definecolor{currentstroke}{rgb}{0.000000,0.000000,0.000000}%
\pgfsetstrokecolor{currentstroke}%
\pgfsetdash{}{0pt}%
\pgfsys@defobject{currentmarker}{\pgfqpoint{0.000000in}{-0.048611in}}{\pgfqpoint{0.000000in}{0.000000in}}{%
\pgfpathmoveto{\pgfqpoint{0.000000in}{0.000000in}}%
\pgfpathlineto{\pgfqpoint{0.000000in}{-0.048611in}}%
\pgfusepath{stroke,fill}%
}%
\begin{pgfscope}%
\pgfsys@transformshift{0.480682in}{0.330000in}%
\pgfsys@useobject{currentmarker}{}%
\end{pgfscope}%
\end{pgfscope}%
\begin{pgfscope}%
\definecolor{textcolor}{rgb}{0.000000,0.000000,0.000000}%
\pgfsetstrokecolor{textcolor}%
\pgfsetfillcolor{textcolor}%
\pgftext[x=0.480682in,y=0.232778in,,top]{\color{textcolor}\sffamily\fontsize{10.000000}{12.000000}\selectfont \(\displaystyle {10^{1}}\)}%
\end{pgfscope}%
\begin{pgfscope}%
\pgfsetbuttcap%
\pgfsetroundjoin%
\definecolor{currentfill}{rgb}{0.000000,0.000000,0.000000}%
\pgfsetfillcolor{currentfill}%
\pgfsetlinewidth{0.602250pt}%
\definecolor{currentstroke}{rgb}{0.000000,0.000000,0.000000}%
\pgfsetstrokecolor{currentstroke}%
\pgfsetdash{}{0pt}%
\pgfsys@defobject{currentmarker}{\pgfqpoint{0.000000in}{-0.027778in}}{\pgfqpoint{0.000000in}{0.000000in}}{%
\pgfpathmoveto{\pgfqpoint{0.000000in}{0.000000in}}%
\pgfpathlineto{\pgfqpoint{0.000000in}{-0.027778in}}%
\pgfusepath{stroke,fill}%
}%
\begin{pgfscope}%
\pgfsys@transformshift{1.298348in}{0.330000in}%
\pgfsys@useobject{currentmarker}{}%
\end{pgfscope}%
\end{pgfscope}%
\begin{pgfscope}%
\definecolor{textcolor}{rgb}{0.000000,0.000000,0.000000}%
\pgfsetstrokecolor{textcolor}%
\pgfsetfillcolor{textcolor}%
\pgftext[x=1.298348in,y=0.255000in,,top]{\color{textcolor}\sffamily\fontsize{10.000000}{12.000000}\selectfont \(\displaystyle {2\times10^{1}}\)}%
\end{pgfscope}%
\begin{pgfscope}%
\pgfsetbuttcap%
\pgfsetroundjoin%
\definecolor{currentfill}{rgb}{0.000000,0.000000,0.000000}%
\pgfsetfillcolor{currentfill}%
\pgfsetlinewidth{0.602250pt}%
\definecolor{currentstroke}{rgb}{0.000000,0.000000,0.000000}%
\pgfsetstrokecolor{currentstroke}%
\pgfsetdash{}{0pt}%
\pgfsys@defobject{currentmarker}{\pgfqpoint{0.000000in}{-0.027778in}}{\pgfqpoint{0.000000in}{0.000000in}}{%
\pgfpathmoveto{\pgfqpoint{0.000000in}{0.000000in}}%
\pgfpathlineto{\pgfqpoint{0.000000in}{-0.027778in}}%
\pgfusepath{stroke,fill}%
}%
\begin{pgfscope}%
\pgfsys@transformshift{1.776652in}{0.330000in}%
\pgfsys@useobject{currentmarker}{}%
\end{pgfscope}%
\end{pgfscope}%
\begin{pgfscope}%
\definecolor{textcolor}{rgb}{0.000000,0.000000,0.000000}%
\pgfsetstrokecolor{textcolor}%
\pgfsetfillcolor{textcolor}%
\pgftext[x=1.776652in,y=0.255000in,,top]{\color{textcolor}\sffamily\fontsize{10.000000}{12.000000}\selectfont \(\displaystyle {3\times10^{1}}\)}%
\end{pgfscope}%
\begin{pgfscope}%
\pgfsetbuttcap%
\pgfsetroundjoin%
\definecolor{currentfill}{rgb}{0.000000,0.000000,0.000000}%
\pgfsetfillcolor{currentfill}%
\pgfsetlinewidth{0.602250pt}%
\definecolor{currentstroke}{rgb}{0.000000,0.000000,0.000000}%
\pgfsetstrokecolor{currentstroke}%
\pgfsetdash{}{0pt}%
\pgfsys@defobject{currentmarker}{\pgfqpoint{0.000000in}{-0.027778in}}{\pgfqpoint{0.000000in}{0.000000in}}{%
\pgfpathmoveto{\pgfqpoint{0.000000in}{0.000000in}}%
\pgfpathlineto{\pgfqpoint{0.000000in}{-0.027778in}}%
\pgfusepath{stroke,fill}%
}%
\begin{pgfscope}%
\pgfsys@transformshift{2.116014in}{0.330000in}%
\pgfsys@useobject{currentmarker}{}%
\end{pgfscope}%
\end{pgfscope}%
\begin{pgfscope}%
\definecolor{textcolor}{rgb}{0.000000,0.000000,0.000000}%
\pgfsetstrokecolor{textcolor}%
\pgfsetfillcolor{textcolor}%
\pgftext[x=2.116014in,y=0.255000in,,top]{\color{textcolor}\sffamily\fontsize{10.000000}{12.000000}\selectfont \(\displaystyle {4\times10^{1}}\)}%
\end{pgfscope}%
\begin{pgfscope}%
\pgfsetbuttcap%
\pgfsetroundjoin%
\definecolor{currentfill}{rgb}{0.000000,0.000000,0.000000}%
\pgfsetfillcolor{currentfill}%
\pgfsetlinewidth{0.602250pt}%
\definecolor{currentstroke}{rgb}{0.000000,0.000000,0.000000}%
\pgfsetstrokecolor{currentstroke}%
\pgfsetdash{}{0pt}%
\pgfsys@defobject{currentmarker}{\pgfqpoint{0.000000in}{-0.027778in}}{\pgfqpoint{0.000000in}{0.000000in}}{%
\pgfpathmoveto{\pgfqpoint{0.000000in}{0.000000in}}%
\pgfpathlineto{\pgfqpoint{0.000000in}{-0.027778in}}%
\pgfusepath{stroke,fill}%
}%
\begin{pgfscope}%
\pgfsys@transformshift{2.379244in}{0.330000in}%
\pgfsys@useobject{currentmarker}{}%
\end{pgfscope}%
\end{pgfscope}%
\begin{pgfscope}%
\pgfsetbuttcap%
\pgfsetroundjoin%
\definecolor{currentfill}{rgb}{0.000000,0.000000,0.000000}%
\pgfsetfillcolor{currentfill}%
\pgfsetlinewidth{0.602250pt}%
\definecolor{currentstroke}{rgb}{0.000000,0.000000,0.000000}%
\pgfsetstrokecolor{currentstroke}%
\pgfsetdash{}{0pt}%
\pgfsys@defobject{currentmarker}{\pgfqpoint{0.000000in}{-0.027778in}}{\pgfqpoint{0.000000in}{0.000000in}}{%
\pgfpathmoveto{\pgfqpoint{0.000000in}{0.000000in}}%
\pgfpathlineto{\pgfqpoint{0.000000in}{-0.027778in}}%
\pgfusepath{stroke,fill}%
}%
\begin{pgfscope}%
\pgfsys@transformshift{2.594318in}{0.330000in}%
\pgfsys@useobject{currentmarker}{}%
\end{pgfscope}%
\end{pgfscope}%
\begin{pgfscope}%
\definecolor{textcolor}{rgb}{0.000000,0.000000,0.000000}%
\pgfsetstrokecolor{textcolor}%
\pgfsetfillcolor{textcolor}%
\pgftext[x=2.594318in,y=0.255000in,,top]{\color{textcolor}\sffamily\fontsize{10.000000}{12.000000}\selectfont \(\displaystyle {6\times10^{1}}\)}%
\end{pgfscope}%
\begin{pgfscope}%
\definecolor{textcolor}{rgb}{0.000000,0.000000,0.000000}%
\pgfsetstrokecolor{textcolor}%
\pgfsetfillcolor{textcolor}%
\pgftext[x=1.537500in,y=0.042809in,,top]{\color{textcolor}\sffamily\fontsize{10.000000}{12.000000}\selectfont \(\displaystyle k\)}%
\end{pgfscope}%
\begin{pgfscope}%
\pgfsetbuttcap%
\pgfsetroundjoin%
\definecolor{currentfill}{rgb}{0.000000,0.000000,0.000000}%
\pgfsetfillcolor{currentfill}%
\pgfsetlinewidth{0.803000pt}%
\definecolor{currentstroke}{rgb}{0.000000,0.000000,0.000000}%
\pgfsetstrokecolor{currentstroke}%
\pgfsetdash{}{0pt}%
\pgfsys@defobject{currentmarker}{\pgfqpoint{-0.048611in}{0.000000in}}{\pgfqpoint{0.000000in}{0.000000in}}{%
\pgfpathmoveto{\pgfqpoint{0.000000in}{0.000000in}}%
\pgfpathlineto{\pgfqpoint{-0.048611in}{0.000000in}}%
\pgfusepath{stroke,fill}%
}%
\begin{pgfscope}%
\pgfsys@transformshift{0.375000in}{0.619643in}%
\pgfsys@useobject{currentmarker}{}%
\end{pgfscope}%
\end{pgfscope}%
\begin{pgfscope}%
\definecolor{textcolor}{rgb}{0.000000,0.000000,0.000000}%
\pgfsetstrokecolor{textcolor}%
\pgfsetfillcolor{textcolor}%
\pgftext[x=-0.031467in,y=0.566881in,left,base]{\color{textcolor}\sffamily\fontsize{10.000000}{12.000000}\selectfont 0.70}%
\end{pgfscope}%
\begin{pgfscope}%
\pgfsetbuttcap%
\pgfsetroundjoin%
\definecolor{currentfill}{rgb}{0.000000,0.000000,0.000000}%
\pgfsetfillcolor{currentfill}%
\pgfsetlinewidth{0.803000pt}%
\definecolor{currentstroke}{rgb}{0.000000,0.000000,0.000000}%
\pgfsetstrokecolor{currentstroke}%
\pgfsetdash{}{0pt}%
\pgfsys@defobject{currentmarker}{\pgfqpoint{-0.048611in}{0.000000in}}{\pgfqpoint{0.000000in}{0.000000in}}{%
\pgfpathmoveto{\pgfqpoint{0.000000in}{0.000000in}}%
\pgfpathlineto{\pgfqpoint{-0.048611in}{0.000000in}}%
\pgfusepath{stroke,fill}%
}%
\begin{pgfscope}%
\pgfsys@transformshift{0.375000in}{0.970999in}%
\pgfsys@useobject{currentmarker}{}%
\end{pgfscope}%
\end{pgfscope}%
\begin{pgfscope}%
\definecolor{textcolor}{rgb}{0.000000,0.000000,0.000000}%
\pgfsetstrokecolor{textcolor}%
\pgfsetfillcolor{textcolor}%
\pgftext[x=-0.031467in,y=0.918238in,left,base]{\color{textcolor}\sffamily\fontsize{10.000000}{12.000000}\selectfont 0.75}%
\end{pgfscope}%
\begin{pgfscope}%
\pgfsetbuttcap%
\pgfsetroundjoin%
\definecolor{currentfill}{rgb}{0.000000,0.000000,0.000000}%
\pgfsetfillcolor{currentfill}%
\pgfsetlinewidth{0.803000pt}%
\definecolor{currentstroke}{rgb}{0.000000,0.000000,0.000000}%
\pgfsetstrokecolor{currentstroke}%
\pgfsetdash{}{0pt}%
\pgfsys@defobject{currentmarker}{\pgfqpoint{-0.048611in}{0.000000in}}{\pgfqpoint{0.000000in}{0.000000in}}{%
\pgfpathmoveto{\pgfqpoint{0.000000in}{0.000000in}}%
\pgfpathlineto{\pgfqpoint{-0.048611in}{0.000000in}}%
\pgfusepath{stroke,fill}%
}%
\begin{pgfscope}%
\pgfsys@transformshift{0.375000in}{1.322356in}%
\pgfsys@useobject{currentmarker}{}%
\end{pgfscope}%
\end{pgfscope}%
\begin{pgfscope}%
\definecolor{textcolor}{rgb}{0.000000,0.000000,0.000000}%
\pgfsetstrokecolor{textcolor}%
\pgfsetfillcolor{textcolor}%
\pgftext[x=-0.031467in,y=1.269594in,left,base]{\color{textcolor}\sffamily\fontsize{10.000000}{12.000000}\selectfont 0.80}%
\end{pgfscope}%
\begin{pgfscope}%
\pgfsetbuttcap%
\pgfsetroundjoin%
\definecolor{currentfill}{rgb}{0.000000,0.000000,0.000000}%
\pgfsetfillcolor{currentfill}%
\pgfsetlinewidth{0.803000pt}%
\definecolor{currentstroke}{rgb}{0.000000,0.000000,0.000000}%
\pgfsetstrokecolor{currentstroke}%
\pgfsetdash{}{0pt}%
\pgfsys@defobject{currentmarker}{\pgfqpoint{-0.048611in}{0.000000in}}{\pgfqpoint{0.000000in}{0.000000in}}{%
\pgfpathmoveto{\pgfqpoint{0.000000in}{0.000000in}}%
\pgfpathlineto{\pgfqpoint{-0.048611in}{0.000000in}}%
\pgfusepath{stroke,fill}%
}%
\begin{pgfscope}%
\pgfsys@transformshift{0.375000in}{1.673713in}%
\pgfsys@useobject{currentmarker}{}%
\end{pgfscope}%
\end{pgfscope}%
\begin{pgfscope}%
\definecolor{textcolor}{rgb}{0.000000,0.000000,0.000000}%
\pgfsetstrokecolor{textcolor}%
\pgfsetfillcolor{textcolor}%
\pgftext[x=-0.031467in,y=1.620951in,left,base]{\color{textcolor}\sffamily\fontsize{10.000000}{12.000000}\selectfont 0.85}%
\end{pgfscope}%
\begin{pgfscope}%
\pgfsetbuttcap%
\pgfsetroundjoin%
\definecolor{currentfill}{rgb}{0.000000,0.000000,0.000000}%
\pgfsetfillcolor{currentfill}%
\pgfsetlinewidth{0.803000pt}%
\definecolor{currentstroke}{rgb}{0.000000,0.000000,0.000000}%
\pgfsetstrokecolor{currentstroke}%
\pgfsetdash{}{0pt}%
\pgfsys@defobject{currentmarker}{\pgfqpoint{-0.048611in}{0.000000in}}{\pgfqpoint{0.000000in}{0.000000in}}{%
\pgfpathmoveto{\pgfqpoint{0.000000in}{0.000000in}}%
\pgfpathlineto{\pgfqpoint{-0.048611in}{0.000000in}}%
\pgfusepath{stroke,fill}%
}%
\begin{pgfscope}%
\pgfsys@transformshift{0.375000in}{2.025069in}%
\pgfsys@useobject{currentmarker}{}%
\end{pgfscope}%
\end{pgfscope}%
\begin{pgfscope}%
\definecolor{textcolor}{rgb}{0.000000,0.000000,0.000000}%
\pgfsetstrokecolor{textcolor}%
\pgfsetfillcolor{textcolor}%
\pgftext[x=-0.031467in,y=1.972308in,left,base]{\color{textcolor}\sffamily\fontsize{10.000000}{12.000000}\selectfont 0.90}%
\end{pgfscope}%
\begin{pgfscope}%
\pgfsetbuttcap%
\pgfsetroundjoin%
\definecolor{currentfill}{rgb}{0.000000,0.000000,0.000000}%
\pgfsetfillcolor{currentfill}%
\pgfsetlinewidth{0.803000pt}%
\definecolor{currentstroke}{rgb}{0.000000,0.000000,0.000000}%
\pgfsetstrokecolor{currentstroke}%
\pgfsetdash{}{0pt}%
\pgfsys@defobject{currentmarker}{\pgfqpoint{-0.048611in}{0.000000in}}{\pgfqpoint{0.000000in}{0.000000in}}{%
\pgfpathmoveto{\pgfqpoint{0.000000in}{0.000000in}}%
\pgfpathlineto{\pgfqpoint{-0.048611in}{0.000000in}}%
\pgfusepath{stroke,fill}%
}%
\begin{pgfscope}%
\pgfsys@transformshift{0.375000in}{2.376426in}%
\pgfsys@useobject{currentmarker}{}%
\end{pgfscope}%
\end{pgfscope}%
\begin{pgfscope}%
\definecolor{textcolor}{rgb}{0.000000,0.000000,0.000000}%
\pgfsetstrokecolor{textcolor}%
\pgfsetfillcolor{textcolor}%
\pgftext[x=-0.031467in,y=2.323664in,left,base]{\color{textcolor}\sffamily\fontsize{10.000000}{12.000000}\selectfont 0.95}%
\end{pgfscope}%
\begin{pgfscope}%
\definecolor{textcolor}{rgb}{0.000000,0.000000,0.000000}%
\pgfsetstrokecolor{textcolor}%
\pgfsetfillcolor{textcolor}%
\pgftext[x=-0.087023in,y=1.485000in,,bottom,rotate=90.000000]{\color{textcolor}\sffamily\fontsize{10.000000}{12.000000}\selectfont \(\displaystyle \alpha\)}%
\end{pgfscope}%
\begin{pgfscope}%
\pgfpathrectangle{\pgfqpoint{0.375000in}{0.330000in}}{\pgfqpoint{2.325000in}{2.310000in}}%
\pgfusepath{clip}%
\pgfsetbuttcap%
\pgfsetroundjoin%
\pgfsetlinewidth{1.505625pt}%
\definecolor{currentstroke}{rgb}{0.000000,0.000000,0.000000}%
\pgfsetstrokecolor{currentstroke}%
\pgfsetdash{{5.550000pt}{2.400000pt}}{0.000000pt}%
\pgfpathmoveto{\pgfqpoint{0.480682in}{2.535000in}}%
\pgfpathlineto{\pgfqpoint{1.298348in}{1.722609in}}%
\pgfpathlineto{\pgfqpoint{1.776652in}{1.247391in}}%
\pgfpathlineto{\pgfqpoint{2.116014in}{0.910218in}}%
\pgfpathlineto{\pgfqpoint{2.379244in}{0.648687in}}%
\pgfpathlineto{\pgfqpoint{2.594318in}{0.435000in}}%
\pgfusepath{stroke}%
\end{pgfscope}%
\begin{pgfscope}%
\pgfpathrectangle{\pgfqpoint{0.375000in}{0.330000in}}{\pgfqpoint{2.325000in}{2.310000in}}%
\pgfusepath{clip}%
\pgfsetbuttcap%
\pgfsetroundjoin%
\definecolor{currentfill}{rgb}{0.000000,0.000000,0.000000}%
\pgfsetfillcolor{currentfill}%
\pgfsetlinewidth{1.003750pt}%
\definecolor{currentstroke}{rgb}{0.000000,0.000000,0.000000}%
\pgfsetstrokecolor{currentstroke}%
\pgfsetdash{}{0pt}%
\pgfsys@defobject{currentmarker}{\pgfqpoint{-0.041667in}{-0.041667in}}{\pgfqpoint{0.041667in}{0.041667in}}{%
\pgfpathmoveto{\pgfqpoint{0.000000in}{-0.041667in}}%
\pgfpathcurveto{\pgfqpoint{0.011050in}{-0.041667in}}{\pgfqpoint{0.021649in}{-0.037276in}}{\pgfqpoint{0.029463in}{-0.029463in}}%
\pgfpathcurveto{\pgfqpoint{0.037276in}{-0.021649in}}{\pgfqpoint{0.041667in}{-0.011050in}}{\pgfqpoint{0.041667in}{0.000000in}}%
\pgfpathcurveto{\pgfqpoint{0.041667in}{0.011050in}}{\pgfqpoint{0.037276in}{0.021649in}}{\pgfqpoint{0.029463in}{0.029463in}}%
\pgfpathcurveto{\pgfqpoint{0.021649in}{0.037276in}}{\pgfqpoint{0.011050in}{0.041667in}}{\pgfqpoint{0.000000in}{0.041667in}}%
\pgfpathcurveto{\pgfqpoint{-0.011050in}{0.041667in}}{\pgfqpoint{-0.021649in}{0.037276in}}{\pgfqpoint{-0.029463in}{0.029463in}}%
\pgfpathcurveto{\pgfqpoint{-0.037276in}{0.021649in}}{\pgfqpoint{-0.041667in}{0.011050in}}{\pgfqpoint{-0.041667in}{0.000000in}}%
\pgfpathcurveto{\pgfqpoint{-0.041667in}{-0.011050in}}{\pgfqpoint{-0.037276in}{-0.021649in}}{\pgfqpoint{-0.029463in}{-0.029463in}}%
\pgfpathcurveto{\pgfqpoint{-0.021649in}{-0.037276in}}{\pgfqpoint{-0.011050in}{-0.041667in}}{\pgfqpoint{0.000000in}{-0.041667in}}%
\pgfpathclose%
\pgfusepath{stroke,fill}%
}%
\begin{pgfscope}%
\pgfsys@transformshift{0.480682in}{2.512308in}%
\pgfsys@useobject{currentmarker}{}%
\end{pgfscope}%
\begin{pgfscope}%
\pgfsys@transformshift{1.298348in}{1.821112in}%
\pgfsys@useobject{currentmarker}{}%
\end{pgfscope}%
\begin{pgfscope}%
\pgfsys@transformshift{1.776652in}{1.142383in}%
\pgfsys@useobject{currentmarker}{}%
\end{pgfscope}%
\begin{pgfscope}%
\pgfsys@transformshift{2.116014in}{0.929580in}%
\pgfsys@useobject{currentmarker}{}%
\end{pgfscope}%
\begin{pgfscope}%
\pgfsys@transformshift{2.379244in}{0.634304in}%
\pgfsys@useobject{currentmarker}{}%
\end{pgfscope}%
\begin{pgfscope}%
\pgfsys@transformshift{2.594318in}{0.459218in}%
\pgfsys@useobject{currentmarker}{}%
\end{pgfscope}%
\end{pgfscope}%
\begin{pgfscope}%
\pgfsetrectcap%
\pgfsetmiterjoin%
\pgfsetlinewidth{0.803000pt}%
\definecolor{currentstroke}{rgb}{0.000000,0.000000,0.000000}%
\pgfsetstrokecolor{currentstroke}%
\pgfsetdash{}{0pt}%
\pgfpathmoveto{\pgfqpoint{0.375000in}{0.330000in}}%
\pgfpathlineto{\pgfqpoint{0.375000in}{2.640000in}}%
\pgfusepath{stroke}%
\end{pgfscope}%
\begin{pgfscope}%
\pgfsetrectcap%
\pgfsetmiterjoin%
\pgfsetlinewidth{0.803000pt}%
\definecolor{currentstroke}{rgb}{0.000000,0.000000,0.000000}%
\pgfsetstrokecolor{currentstroke}%
\pgfsetdash{}{0pt}%
\pgfpathmoveto{\pgfqpoint{2.700000in}{0.330000in}}%
\pgfpathlineto{\pgfqpoint{2.700000in}{2.640000in}}%
\pgfusepath{stroke}%
\end{pgfscope}%
\begin{pgfscope}%
\pgfsetrectcap%
\pgfsetmiterjoin%
\pgfsetlinewidth{0.803000pt}%
\definecolor{currentstroke}{rgb}{0.000000,0.000000,0.000000}%
\pgfsetstrokecolor{currentstroke}%
\pgfsetdash{}{0pt}%
\pgfpathmoveto{\pgfqpoint{0.375000in}{0.330000in}}%
\pgfpathlineto{\pgfqpoint{2.700000in}{0.330000in}}%
\pgfusepath{stroke}%
\end{pgfscope}%
\begin{pgfscope}%
\pgfsetrectcap%
\pgfsetmiterjoin%
\pgfsetlinewidth{0.803000pt}%
\definecolor{currentstroke}{rgb}{0.000000,0.000000,0.000000}%
\pgfsetstrokecolor{currentstroke}%
\pgfsetdash{}{0pt}%
\pgfpathmoveto{\pgfqpoint{0.375000in}{2.640000in}}%
\pgfpathlineto{\pgfqpoint{2.700000in}{2.640000in}}%
\pgfusepath{stroke}%
\end{pgfscope}%
\begin{pgfscope}%
\pgfsetbuttcap%
\pgfsetmiterjoin%
\definecolor{currentfill}{rgb}{1.000000,1.000000,1.000000}%
\pgfsetfillcolor{currentfill}%
\pgfsetfillopacity{0.800000}%
\pgfsetlinewidth{1.003750pt}%
\definecolor{currentstroke}{rgb}{0.800000,0.800000,0.800000}%
\pgfsetstrokecolor{currentstroke}%
\pgfsetstrokeopacity{0.800000}%
\pgfsetdash{}{0pt}%
\pgfpathmoveto{\pgfqpoint{0.517632in}{2.319199in}}%
\pgfpathlineto{\pgfqpoint{2.602778in}{2.319199in}}%
\pgfpathquadraticcurveto{\pgfqpoint{2.630556in}{2.319199in}}{\pgfqpoint{2.630556in}{2.346977in}}%
\pgfpathlineto{\pgfqpoint{2.630556in}{2.542778in}}%
\pgfpathquadraticcurveto{\pgfqpoint{2.630556in}{2.570556in}}{\pgfqpoint{2.602778in}{2.570556in}}%
\pgfpathlineto{\pgfqpoint{0.517632in}{2.570556in}}%
\pgfpathquadraticcurveto{\pgfqpoint{0.489854in}{2.570556in}}{\pgfqpoint{0.489854in}{2.542778in}}%
\pgfpathlineto{\pgfqpoint{0.489854in}{2.346977in}}%
\pgfpathquadraticcurveto{\pgfqpoint{0.489854in}{2.319199in}}{\pgfqpoint{0.517632in}{2.319199in}}%
\pgfpathclose%
\pgfusepath{stroke,fill}%
\end{pgfscope}%
\begin{pgfscope}%
\pgfsetbuttcap%
\pgfsetroundjoin%
\pgfsetlinewidth{1.505625pt}%
\definecolor{currentstroke}{rgb}{0.000000,0.000000,0.000000}%
\pgfsetstrokecolor{currentstroke}%
\pgfsetdash{{5.550000pt}{2.400000pt}}{0.000000pt}%
\pgfpathmoveto{\pgfqpoint{0.545410in}{2.458088in}}%
\pgfpathlineto{\pgfqpoint{0.823187in}{2.458088in}}%
\pgfusepath{stroke}%
\end{pgfscope}%
\begin{pgfscope}%
\definecolor{textcolor}{rgb}{0.000000,0.000000,0.000000}%
\pgfsetstrokecolor{textcolor}%
\pgfsetfillcolor{textcolor}%
\pgftext[x=0.934298in,y=2.409477in,left,base]{\color{textcolor}\sffamily\fontsize{10.000000}{12.000000}\selectfont \(\displaystyle \alpha = 1.3566 - 0.3840\log(k)\)}%
\end{pgfscope}%
\end{pgfpicture}%
\makeatother%
\endgroup%

    \end{subfigure}

\caption[The computed Quasi-Monte-Carlo convergence rate for $Q(u) =  u(\bzero)$.]{The computed values of $C$ (top) and $\alpha$ (bottom) against $k$ in \cref{eq:qmcerrorform} for $Q(u) =  u(\bzero)$. Observe the $x$-axes are on a $\log_{10}$ scale, but $\loge$ is the natural logarithm. \label{fig:originCalpha}}
\end{figure}

\begin{figure}[h]
    \centering
    \begin{subfigure}{\textwidth}
            \centering
%% Creator: Matplotlib, PGF backend
%%
%% To include the figure in your LaTeX document, write
%%   \input{<filename>.pgf}
%%
%% Make sure the required packages are loaded in your preamble
%%   \usepackage{pgf}
%%
%% Figures using additional raster images can only be included by \input if
%% they are in the same directory as the main LaTeX file. For loading figures
%% from other directories you can use the `import` package
%%   \usepackage{import}
%% and then include the figures with
%%   \import{<path to file>}{<filename>.pgf}
%%
%% Matplotlib used the following preamble
%%   \usepackage{fontspec}
%%   \setmainfont{DejaVuSerif.ttf}[Path=/home/owen/progs/firedrake-complex/firedrake/lib/python3.5/site-packages/matplotlib/mpl-data/fonts/ttf/]
%%   \setsansfont{DejaVuSans.ttf}[Path=/home/owen/progs/firedrake-complex/firedrake/lib/python3.5/site-packages/matplotlib/mpl-data/fonts/ttf/]
%%   \setmonofont{DejaVuSansMono.ttf}[Path=/home/owen/progs/firedrake-complex/firedrake/lib/python3.5/site-packages/matplotlib/mpl-data/fonts/ttf/]
%%
\begingroup%
\makeatletter%
\begin{pgfpicture}%
\pgfpathrectangle{\pgfpointorigin}{\pgfqpoint{5.000000in}{4.000000in}}%
\pgfusepath{use as bounding box, clip}%
\begin{pgfscope}%
\pgfsetbuttcap%
\pgfsetmiterjoin%
\definecolor{currentfill}{rgb}{1.000000,1.000000,1.000000}%
\pgfsetfillcolor{currentfill}%
\pgfsetlinewidth{0.000000pt}%
\definecolor{currentstroke}{rgb}{1.000000,1.000000,1.000000}%
\pgfsetstrokecolor{currentstroke}%
\pgfsetdash{}{0pt}%
\pgfpathmoveto{\pgfqpoint{0.000000in}{0.000000in}}%
\pgfpathlineto{\pgfqpoint{5.000000in}{0.000000in}}%
\pgfpathlineto{\pgfqpoint{5.000000in}{4.000000in}}%
\pgfpathlineto{\pgfqpoint{0.000000in}{4.000000in}}%
\pgfpathclose%
\pgfusepath{fill}%
\end{pgfscope}%
\begin{pgfscope}%
\pgfsetbuttcap%
\pgfsetmiterjoin%
\definecolor{currentfill}{rgb}{1.000000,1.000000,1.000000}%
\pgfsetfillcolor{currentfill}%
\pgfsetlinewidth{0.000000pt}%
\definecolor{currentstroke}{rgb}{0.000000,0.000000,0.000000}%
\pgfsetstrokecolor{currentstroke}%
\pgfsetstrokeopacity{0.000000}%
\pgfsetdash{}{0pt}%
\pgfpathmoveto{\pgfqpoint{0.629421in}{0.617778in}}%
\pgfpathlineto{\pgfqpoint{4.793685in}{0.617778in}}%
\pgfpathlineto{\pgfqpoint{4.793685in}{3.800000in}}%
\pgfpathlineto{\pgfqpoint{0.629421in}{3.800000in}}%
\pgfpathclose%
\pgfusepath{fill}%
\end{pgfscope}%
\begin{pgfscope}%
\pgfsetbuttcap%
\pgfsetroundjoin%
\definecolor{currentfill}{rgb}{0.000000,0.000000,0.000000}%
\pgfsetfillcolor{currentfill}%
\pgfsetlinewidth{0.803000pt}%
\definecolor{currentstroke}{rgb}{0.000000,0.000000,0.000000}%
\pgfsetstrokecolor{currentstroke}%
\pgfsetdash{}{0pt}%
\pgfsys@defobject{currentmarker}{\pgfqpoint{0.000000in}{-0.048611in}}{\pgfqpoint{0.000000in}{0.000000in}}{%
\pgfpathmoveto{\pgfqpoint{0.000000in}{0.000000in}}%
\pgfpathlineto{\pgfqpoint{0.000000in}{-0.048611in}}%
\pgfusepath{stroke,fill}%
}%
\begin{pgfscope}%
\pgfsys@transformshift{0.818706in}{0.617778in}%
\pgfsys@useobject{currentmarker}{}%
\end{pgfscope}%
\end{pgfscope}%
\begin{pgfscope}%
\definecolor{textcolor}{rgb}{0.000000,0.000000,0.000000}%
\pgfsetstrokecolor{textcolor}%
\pgfsetfillcolor{textcolor}%
\pgftext[x=0.818706in,y=0.520556in,,top]{\color{textcolor}\sffamily\fontsize{11.000000}{13.200000}\selectfont \(\displaystyle 10^{1}\)}%
\end{pgfscope}%
\begin{pgfscope}%
\pgfsetbuttcap%
\pgfsetroundjoin%
\definecolor{currentfill}{rgb}{0.000000,0.000000,0.000000}%
\pgfsetfillcolor{currentfill}%
\pgfsetlinewidth{0.602250pt}%
\definecolor{currentstroke}{rgb}{0.000000,0.000000,0.000000}%
\pgfsetstrokecolor{currentstroke}%
\pgfsetdash{}{0pt}%
\pgfsys@defobject{currentmarker}{\pgfqpoint{0.000000in}{-0.027778in}}{\pgfqpoint{0.000000in}{0.000000in}}{%
\pgfpathmoveto{\pgfqpoint{0.000000in}{0.000000in}}%
\pgfpathlineto{\pgfqpoint{0.000000in}{-0.027778in}}%
\pgfusepath{stroke,fill}%
}%
\begin{pgfscope}%
\pgfsys@transformshift{2.283212in}{0.617778in}%
\pgfsys@useobject{currentmarker}{}%
\end{pgfscope}%
\end{pgfscope}%
\begin{pgfscope}%
\definecolor{textcolor}{rgb}{0.000000,0.000000,0.000000}%
\pgfsetstrokecolor{textcolor}%
\pgfsetfillcolor{textcolor}%
\pgftext[x=2.283212in,y=0.542778in,,top]{\color{textcolor}\sffamily\fontsize{11.000000}{13.200000}\selectfont \(\displaystyle 2\times10^{1}\)}%
\end{pgfscope}%
\begin{pgfscope}%
\pgfsetbuttcap%
\pgfsetroundjoin%
\definecolor{currentfill}{rgb}{0.000000,0.000000,0.000000}%
\pgfsetfillcolor{currentfill}%
\pgfsetlinewidth{0.602250pt}%
\definecolor{currentstroke}{rgb}{0.000000,0.000000,0.000000}%
\pgfsetstrokecolor{currentstroke}%
\pgfsetdash{}{0pt}%
\pgfsys@defobject{currentmarker}{\pgfqpoint{0.000000in}{-0.027778in}}{\pgfqpoint{0.000000in}{0.000000in}}{%
\pgfpathmoveto{\pgfqpoint{0.000000in}{0.000000in}}%
\pgfpathlineto{\pgfqpoint{0.000000in}{-0.027778in}}%
\pgfusepath{stroke,fill}%
}%
\begin{pgfscope}%
\pgfsys@transformshift{3.139894in}{0.617778in}%
\pgfsys@useobject{currentmarker}{}%
\end{pgfscope}%
\end{pgfscope}%
\begin{pgfscope}%
\definecolor{textcolor}{rgb}{0.000000,0.000000,0.000000}%
\pgfsetstrokecolor{textcolor}%
\pgfsetfillcolor{textcolor}%
\pgftext[x=3.139894in,y=0.542778in,,top]{\color{textcolor}\sffamily\fontsize{11.000000}{13.200000}\selectfont \(\displaystyle 3\times10^{1}\)}%
\end{pgfscope}%
\begin{pgfscope}%
\pgfsetbuttcap%
\pgfsetroundjoin%
\definecolor{currentfill}{rgb}{0.000000,0.000000,0.000000}%
\pgfsetfillcolor{currentfill}%
\pgfsetlinewidth{0.602250pt}%
\definecolor{currentstroke}{rgb}{0.000000,0.000000,0.000000}%
\pgfsetstrokecolor{currentstroke}%
\pgfsetdash{}{0pt}%
\pgfsys@defobject{currentmarker}{\pgfqpoint{0.000000in}{-0.027778in}}{\pgfqpoint{0.000000in}{0.000000in}}{%
\pgfpathmoveto{\pgfqpoint{0.000000in}{0.000000in}}%
\pgfpathlineto{\pgfqpoint{0.000000in}{-0.027778in}}%
\pgfusepath{stroke,fill}%
}%
\begin{pgfscope}%
\pgfsys@transformshift{3.747719in}{0.617778in}%
\pgfsys@useobject{currentmarker}{}%
\end{pgfscope}%
\end{pgfscope}%
\begin{pgfscope}%
\definecolor{textcolor}{rgb}{0.000000,0.000000,0.000000}%
\pgfsetstrokecolor{textcolor}%
\pgfsetfillcolor{textcolor}%
\pgftext[x=3.747719in,y=0.542778in,,top]{\color{textcolor}\sffamily\fontsize{11.000000}{13.200000}\selectfont \(\displaystyle 4\times10^{1}\)}%
\end{pgfscope}%
\begin{pgfscope}%
\pgfsetbuttcap%
\pgfsetroundjoin%
\definecolor{currentfill}{rgb}{0.000000,0.000000,0.000000}%
\pgfsetfillcolor{currentfill}%
\pgfsetlinewidth{0.602250pt}%
\definecolor{currentstroke}{rgb}{0.000000,0.000000,0.000000}%
\pgfsetstrokecolor{currentstroke}%
\pgfsetdash{}{0pt}%
\pgfsys@defobject{currentmarker}{\pgfqpoint{0.000000in}{-0.027778in}}{\pgfqpoint{0.000000in}{0.000000in}}{%
\pgfpathmoveto{\pgfqpoint{0.000000in}{0.000000in}}%
\pgfpathlineto{\pgfqpoint{0.000000in}{-0.027778in}}%
\pgfusepath{stroke,fill}%
}%
\begin{pgfscope}%
\pgfsys@transformshift{4.219184in}{0.617778in}%
\pgfsys@useobject{currentmarker}{}%
\end{pgfscope}%
\end{pgfscope}%
\begin{pgfscope}%
\pgfsetbuttcap%
\pgfsetroundjoin%
\definecolor{currentfill}{rgb}{0.000000,0.000000,0.000000}%
\pgfsetfillcolor{currentfill}%
\pgfsetlinewidth{0.602250pt}%
\definecolor{currentstroke}{rgb}{0.000000,0.000000,0.000000}%
\pgfsetstrokecolor{currentstroke}%
\pgfsetdash{}{0pt}%
\pgfsys@defobject{currentmarker}{\pgfqpoint{0.000000in}{-0.027778in}}{\pgfqpoint{0.000000in}{0.000000in}}{%
\pgfpathmoveto{\pgfqpoint{0.000000in}{0.000000in}}%
\pgfpathlineto{\pgfqpoint{0.000000in}{-0.027778in}}%
\pgfusepath{stroke,fill}%
}%
\begin{pgfscope}%
\pgfsys@transformshift{4.604400in}{0.617778in}%
\pgfsys@useobject{currentmarker}{}%
\end{pgfscope}%
\end{pgfscope}%
\begin{pgfscope}%
\definecolor{textcolor}{rgb}{0.000000,0.000000,0.000000}%
\pgfsetstrokecolor{textcolor}%
\pgfsetfillcolor{textcolor}%
\pgftext[x=4.604400in,y=0.542778in,,top]{\color{textcolor}\sffamily\fontsize{11.000000}{13.200000}\selectfont \(\displaystyle 6\times10^{1}\)}%
\end{pgfscope}%
\begin{pgfscope}%
\definecolor{textcolor}{rgb}{0.000000,0.000000,0.000000}%
\pgfsetstrokecolor{textcolor}%
\pgfsetfillcolor{textcolor}%
\pgftext[x=2.711553in,y=0.317146in,,top]{\color{textcolor}\sffamily\fontsize{11.000000}{13.200000}\selectfont \(\displaystyle k\)}%
\end{pgfscope}%
\begin{pgfscope}%
\pgfsetbuttcap%
\pgfsetroundjoin%
\definecolor{currentfill}{rgb}{0.000000,0.000000,0.000000}%
\pgfsetfillcolor{currentfill}%
\pgfsetlinewidth{0.803000pt}%
\definecolor{currentstroke}{rgb}{0.000000,0.000000,0.000000}%
\pgfsetstrokecolor{currentstroke}%
\pgfsetdash{}{0pt}%
\pgfsys@defobject{currentmarker}{\pgfqpoint{-0.048611in}{0.000000in}}{\pgfqpoint{0.000000in}{0.000000in}}{%
\pgfpathmoveto{\pgfqpoint{0.000000in}{0.000000in}}%
\pgfpathlineto{\pgfqpoint{-0.048611in}{0.000000in}}%
\pgfusepath{stroke,fill}%
}%
\begin{pgfscope}%
\pgfsys@transformshift{0.629421in}{0.964204in}%
\pgfsys@useobject{currentmarker}{}%
\end{pgfscope}%
\end{pgfscope}%
\begin{pgfscope}%
\definecolor{textcolor}{rgb}{0.000000,0.000000,0.000000}%
\pgfsetstrokecolor{textcolor}%
\pgfsetfillcolor{textcolor}%
\pgftext[x=0.261828in,y=0.906167in,left,base]{\color{textcolor}\sffamily\fontsize{11.000000}{13.200000}\selectfont \(\displaystyle 0.10\)}%
\end{pgfscope}%
\begin{pgfscope}%
\pgfsetbuttcap%
\pgfsetroundjoin%
\definecolor{currentfill}{rgb}{0.000000,0.000000,0.000000}%
\pgfsetfillcolor{currentfill}%
\pgfsetlinewidth{0.803000pt}%
\definecolor{currentstroke}{rgb}{0.000000,0.000000,0.000000}%
\pgfsetstrokecolor{currentstroke}%
\pgfsetdash{}{0pt}%
\pgfsys@defobject{currentmarker}{\pgfqpoint{-0.048611in}{0.000000in}}{\pgfqpoint{0.000000in}{0.000000in}}{%
\pgfpathmoveto{\pgfqpoint{0.000000in}{0.000000in}}%
\pgfpathlineto{\pgfqpoint{-0.048611in}{0.000000in}}%
\pgfusepath{stroke,fill}%
}%
\begin{pgfscope}%
\pgfsys@transformshift{0.629421in}{1.573444in}%
\pgfsys@useobject{currentmarker}{}%
\end{pgfscope}%
\end{pgfscope}%
\begin{pgfscope}%
\definecolor{textcolor}{rgb}{0.000000,0.000000,0.000000}%
\pgfsetstrokecolor{textcolor}%
\pgfsetfillcolor{textcolor}%
\pgftext[x=0.261828in,y=1.515406in,left,base]{\color{textcolor}\sffamily\fontsize{11.000000}{13.200000}\selectfont \(\displaystyle 0.12\)}%
\end{pgfscope}%
\begin{pgfscope}%
\pgfsetbuttcap%
\pgfsetroundjoin%
\definecolor{currentfill}{rgb}{0.000000,0.000000,0.000000}%
\pgfsetfillcolor{currentfill}%
\pgfsetlinewidth{0.803000pt}%
\definecolor{currentstroke}{rgb}{0.000000,0.000000,0.000000}%
\pgfsetstrokecolor{currentstroke}%
\pgfsetdash{}{0pt}%
\pgfsys@defobject{currentmarker}{\pgfqpoint{-0.048611in}{0.000000in}}{\pgfqpoint{0.000000in}{0.000000in}}{%
\pgfpathmoveto{\pgfqpoint{0.000000in}{0.000000in}}%
\pgfpathlineto{\pgfqpoint{-0.048611in}{0.000000in}}%
\pgfusepath{stroke,fill}%
}%
\begin{pgfscope}%
\pgfsys@transformshift{0.629421in}{2.182683in}%
\pgfsys@useobject{currentmarker}{}%
\end{pgfscope}%
\end{pgfscope}%
\begin{pgfscope}%
\definecolor{textcolor}{rgb}{0.000000,0.000000,0.000000}%
\pgfsetstrokecolor{textcolor}%
\pgfsetfillcolor{textcolor}%
\pgftext[x=0.261828in,y=2.124646in,left,base]{\color{textcolor}\sffamily\fontsize{11.000000}{13.200000}\selectfont \(\displaystyle 0.14\)}%
\end{pgfscope}%
\begin{pgfscope}%
\pgfsetbuttcap%
\pgfsetroundjoin%
\definecolor{currentfill}{rgb}{0.000000,0.000000,0.000000}%
\pgfsetfillcolor{currentfill}%
\pgfsetlinewidth{0.803000pt}%
\definecolor{currentstroke}{rgb}{0.000000,0.000000,0.000000}%
\pgfsetstrokecolor{currentstroke}%
\pgfsetdash{}{0pt}%
\pgfsys@defobject{currentmarker}{\pgfqpoint{-0.048611in}{0.000000in}}{\pgfqpoint{0.000000in}{0.000000in}}{%
\pgfpathmoveto{\pgfqpoint{0.000000in}{0.000000in}}%
\pgfpathlineto{\pgfqpoint{-0.048611in}{0.000000in}}%
\pgfusepath{stroke,fill}%
}%
\begin{pgfscope}%
\pgfsys@transformshift{0.629421in}{2.791923in}%
\pgfsys@useobject{currentmarker}{}%
\end{pgfscope}%
\end{pgfscope}%
\begin{pgfscope}%
\definecolor{textcolor}{rgb}{0.000000,0.000000,0.000000}%
\pgfsetstrokecolor{textcolor}%
\pgfsetfillcolor{textcolor}%
\pgftext[x=0.261828in,y=2.733885in,left,base]{\color{textcolor}\sffamily\fontsize{11.000000}{13.200000}\selectfont \(\displaystyle 0.16\)}%
\end{pgfscope}%
\begin{pgfscope}%
\pgfsetbuttcap%
\pgfsetroundjoin%
\definecolor{currentfill}{rgb}{0.000000,0.000000,0.000000}%
\pgfsetfillcolor{currentfill}%
\pgfsetlinewidth{0.803000pt}%
\definecolor{currentstroke}{rgb}{0.000000,0.000000,0.000000}%
\pgfsetstrokecolor{currentstroke}%
\pgfsetdash{}{0pt}%
\pgfsys@defobject{currentmarker}{\pgfqpoint{-0.048611in}{0.000000in}}{\pgfqpoint{0.000000in}{0.000000in}}{%
\pgfpathmoveto{\pgfqpoint{0.000000in}{0.000000in}}%
\pgfpathlineto{\pgfqpoint{-0.048611in}{0.000000in}}%
\pgfusepath{stroke,fill}%
}%
\begin{pgfscope}%
\pgfsys@transformshift{0.629421in}{3.401163in}%
\pgfsys@useobject{currentmarker}{}%
\end{pgfscope}%
\end{pgfscope}%
\begin{pgfscope}%
\definecolor{textcolor}{rgb}{0.000000,0.000000,0.000000}%
\pgfsetstrokecolor{textcolor}%
\pgfsetfillcolor{textcolor}%
\pgftext[x=0.261828in,y=3.343125in,left,base]{\color{textcolor}\sffamily\fontsize{11.000000}{13.200000}\selectfont \(\displaystyle 0.18\)}%
\end{pgfscope}%
\begin{pgfscope}%
\definecolor{textcolor}{rgb}{0.000000,0.000000,0.000000}%
\pgfsetstrokecolor{textcolor}%
\pgfsetfillcolor{textcolor}%
\pgftext[x=0.206273in,y=2.208889in,,bottom]{\color{textcolor}\sffamily\fontsize{11.000000}{13.200000}\selectfont \(\displaystyle C\)}%
\end{pgfscope}%
\begin{pgfscope}%
\pgfpathrectangle{\pgfqpoint{0.629421in}{0.617778in}}{\pgfqpoint{4.164264in}{3.182222in}}%
\pgfusepath{clip}%
\pgfsetbuttcap%
\pgfsetroundjoin%
\definecolor{currentfill}{rgb}{0.000000,0.000000,0.000000}%
\pgfsetfillcolor{currentfill}%
\pgfsetlinewidth{1.003750pt}%
\definecolor{currentstroke}{rgb}{0.000000,0.000000,0.000000}%
\pgfsetstrokecolor{currentstroke}%
\pgfsetdash{}{0pt}%
\pgfsys@defobject{currentmarker}{\pgfqpoint{-0.041667in}{-0.041667in}}{\pgfqpoint{0.041667in}{0.041667in}}{%
\pgfpathmoveto{\pgfqpoint{0.000000in}{-0.041667in}}%
\pgfpathcurveto{\pgfqpoint{0.011050in}{-0.041667in}}{\pgfqpoint{0.021649in}{-0.037276in}}{\pgfqpoint{0.029463in}{-0.029463in}}%
\pgfpathcurveto{\pgfqpoint{0.037276in}{-0.021649in}}{\pgfqpoint{0.041667in}{-0.011050in}}{\pgfqpoint{0.041667in}{0.000000in}}%
\pgfpathcurveto{\pgfqpoint{0.041667in}{0.011050in}}{\pgfqpoint{0.037276in}{0.021649in}}{\pgfqpoint{0.029463in}{0.029463in}}%
\pgfpathcurveto{\pgfqpoint{0.021649in}{0.037276in}}{\pgfqpoint{0.011050in}{0.041667in}}{\pgfqpoint{0.000000in}{0.041667in}}%
\pgfpathcurveto{\pgfqpoint{-0.011050in}{0.041667in}}{\pgfqpoint{-0.021649in}{0.037276in}}{\pgfqpoint{-0.029463in}{0.029463in}}%
\pgfpathcurveto{\pgfqpoint{-0.037276in}{0.021649in}}{\pgfqpoint{-0.041667in}{0.011050in}}{\pgfqpoint{-0.041667in}{0.000000in}}%
\pgfpathcurveto{\pgfqpoint{-0.041667in}{-0.011050in}}{\pgfqpoint{-0.037276in}{-0.021649in}}{\pgfqpoint{-0.029463in}{-0.029463in}}%
\pgfpathcurveto{\pgfqpoint{-0.021649in}{-0.037276in}}{\pgfqpoint{-0.011050in}{-0.041667in}}{\pgfqpoint{0.000000in}{-0.041667in}}%
\pgfpathclose%
\pgfusepath{stroke,fill}%
}%
\begin{pgfscope}%
\pgfsys@transformshift{0.818706in}{0.762424in}%
\pgfsys@useobject{currentmarker}{}%
\end{pgfscope}%
\begin{pgfscope}%
\pgfsys@transformshift{2.283212in}{1.979947in}%
\pgfsys@useobject{currentmarker}{}%
\end{pgfscope}%
\begin{pgfscope}%
\pgfsys@transformshift{3.139894in}{3.432012in}%
\pgfsys@useobject{currentmarker}{}%
\end{pgfscope}%
\begin{pgfscope}%
\pgfsys@transformshift{3.747719in}{2.830476in}%
\pgfsys@useobject{currentmarker}{}%
\end{pgfscope}%
\begin{pgfscope}%
\pgfsys@transformshift{4.219184in}{3.655354in}%
\pgfsys@useobject{currentmarker}{}%
\end{pgfscope}%
\begin{pgfscope}%
\pgfsys@transformshift{4.604400in}{2.294962in}%
\pgfsys@useobject{currentmarker}{}%
\end{pgfscope}%
\end{pgfscope}%
\begin{pgfscope}%
\pgfsetrectcap%
\pgfsetmiterjoin%
\pgfsetlinewidth{0.803000pt}%
\definecolor{currentstroke}{rgb}{0.000000,0.000000,0.000000}%
\pgfsetstrokecolor{currentstroke}%
\pgfsetdash{}{0pt}%
\pgfpathmoveto{\pgfqpoint{0.629421in}{0.617778in}}%
\pgfpathlineto{\pgfqpoint{0.629421in}{3.800000in}}%
\pgfusepath{stroke}%
\end{pgfscope}%
\begin{pgfscope}%
\pgfsetrectcap%
\pgfsetmiterjoin%
\pgfsetlinewidth{0.000000pt}%
\definecolor{currentstroke}{rgb}{0.000000,0.000000,0.000000}%
\pgfsetstrokecolor{currentstroke}%
\pgfsetstrokeopacity{0.000000}%
\pgfsetdash{}{0pt}%
\pgfpathmoveto{\pgfqpoint{4.793685in}{0.617778in}}%
\pgfpathlineto{\pgfqpoint{4.793685in}{3.800000in}}%
\pgfusepath{}%
\end{pgfscope}%
\begin{pgfscope}%
\pgfsetrectcap%
\pgfsetmiterjoin%
\pgfsetlinewidth{0.803000pt}%
\definecolor{currentstroke}{rgb}{0.000000,0.000000,0.000000}%
\pgfsetstrokecolor{currentstroke}%
\pgfsetdash{}{0pt}%
\pgfpathmoveto{\pgfqpoint{0.629421in}{0.617778in}}%
\pgfpathlineto{\pgfqpoint{4.793685in}{0.617778in}}%
\pgfusepath{stroke}%
\end{pgfscope}%
\begin{pgfscope}%
\pgfsetrectcap%
\pgfsetmiterjoin%
\pgfsetlinewidth{0.000000pt}%
\definecolor{currentstroke}{rgb}{0.000000,0.000000,0.000000}%
\pgfsetstrokecolor{currentstroke}%
\pgfsetstrokeopacity{0.000000}%
\pgfsetdash{}{0pt}%
\pgfpathmoveto{\pgfqpoint{0.629421in}{3.800000in}}%
\pgfpathlineto{\pgfqpoint{4.793685in}{3.800000in}}%
\pgfusepath{}%
\end{pgfscope}%
\end{pgfpicture}%
\makeatother%
\endgroup%

  \end{subfigure}
    \begin{subfigure}{\textwidth}
            \centering
%% Creator: Matplotlib, PGF backend
%%
%% To include the figure in your LaTeX document, write
%%   \input{<filename>.pgf}
%%
%% Make sure the required packages are loaded in your preamble
%%   \usepackage{pgf}
%%
%% Figures using additional raster images can only be included by \input if
%% they are in the same directory as the main LaTeX file. For loading figures
%% from other directories you can use the `import` package
%%   \usepackage{import}
%% and then include the figures with
%%   \import{<path to file>}{<filename>.pgf}
%%
%% Matplotlib used the following preamble
%%   \usepackage{fontspec}
%%   \setmainfont{DejaVuSerif.ttf}[Path=/home/owen/progs/firedrake-complex/firedrake/lib/python3.5/site-packages/matplotlib/mpl-data/fonts/ttf/]
%%   \setsansfont{DejaVuSans.ttf}[Path=/home/owen/progs/firedrake-complex/firedrake/lib/python3.5/site-packages/matplotlib/mpl-data/fonts/ttf/]
%%   \setmonofont{DejaVuSansMono.ttf}[Path=/home/owen/progs/firedrake-complex/firedrake/lib/python3.5/site-packages/matplotlib/mpl-data/fonts/ttf/]
%%
\begingroup%
\makeatletter%
\begin{pgfpicture}%
\pgfpathrectangle{\pgfpointorigin}{\pgfqpoint{3.000000in}{3.000000in}}%
\pgfusepath{use as bounding box, clip}%
\begin{pgfscope}%
\pgfsetbuttcap%
\pgfsetmiterjoin%
\definecolor{currentfill}{rgb}{1.000000,1.000000,1.000000}%
\pgfsetfillcolor{currentfill}%
\pgfsetlinewidth{0.000000pt}%
\definecolor{currentstroke}{rgb}{1.000000,1.000000,1.000000}%
\pgfsetstrokecolor{currentstroke}%
\pgfsetdash{}{0pt}%
\pgfpathmoveto{\pgfqpoint{0.000000in}{0.000000in}}%
\pgfpathlineto{\pgfqpoint{3.000000in}{0.000000in}}%
\pgfpathlineto{\pgfqpoint{3.000000in}{3.000000in}}%
\pgfpathlineto{\pgfqpoint{0.000000in}{3.000000in}}%
\pgfpathclose%
\pgfusepath{fill}%
\end{pgfscope}%
\begin{pgfscope}%
\pgfsetbuttcap%
\pgfsetmiterjoin%
\definecolor{currentfill}{rgb}{1.000000,1.000000,1.000000}%
\pgfsetfillcolor{currentfill}%
\pgfsetlinewidth{0.000000pt}%
\definecolor{currentstroke}{rgb}{0.000000,0.000000,0.000000}%
\pgfsetstrokecolor{currentstroke}%
\pgfsetstrokeopacity{0.000000}%
\pgfsetdash{}{0pt}%
\pgfpathmoveto{\pgfqpoint{0.375000in}{0.330000in}}%
\pgfpathlineto{\pgfqpoint{2.700000in}{0.330000in}}%
\pgfpathlineto{\pgfqpoint{2.700000in}{2.640000in}}%
\pgfpathlineto{\pgfqpoint{0.375000in}{2.640000in}}%
\pgfpathclose%
\pgfusepath{fill}%
\end{pgfscope}%
\begin{pgfscope}%
\pgfsetbuttcap%
\pgfsetroundjoin%
\definecolor{currentfill}{rgb}{0.000000,0.000000,0.000000}%
\pgfsetfillcolor{currentfill}%
\pgfsetlinewidth{0.803000pt}%
\definecolor{currentstroke}{rgb}{0.000000,0.000000,0.000000}%
\pgfsetstrokecolor{currentstroke}%
\pgfsetdash{}{0pt}%
\pgfsys@defobject{currentmarker}{\pgfqpoint{0.000000in}{-0.048611in}}{\pgfqpoint{0.000000in}{0.000000in}}{%
\pgfpathmoveto{\pgfqpoint{0.000000in}{0.000000in}}%
\pgfpathlineto{\pgfqpoint{0.000000in}{-0.048611in}}%
\pgfusepath{stroke,fill}%
}%
\begin{pgfscope}%
\pgfsys@transformshift{0.480682in}{0.330000in}%
\pgfsys@useobject{currentmarker}{}%
\end{pgfscope}%
\end{pgfscope}%
\begin{pgfscope}%
\definecolor{textcolor}{rgb}{0.000000,0.000000,0.000000}%
\pgfsetstrokecolor{textcolor}%
\pgfsetfillcolor{textcolor}%
\pgftext[x=0.480682in,y=0.232778in,,top]{\color{textcolor}\sffamily\fontsize{10.000000}{12.000000}\selectfont \(\displaystyle {10^{1}}\)}%
\end{pgfscope}%
\begin{pgfscope}%
\pgfsetbuttcap%
\pgfsetroundjoin%
\definecolor{currentfill}{rgb}{0.000000,0.000000,0.000000}%
\pgfsetfillcolor{currentfill}%
\pgfsetlinewidth{0.602250pt}%
\definecolor{currentstroke}{rgb}{0.000000,0.000000,0.000000}%
\pgfsetstrokecolor{currentstroke}%
\pgfsetdash{}{0pt}%
\pgfsys@defobject{currentmarker}{\pgfqpoint{0.000000in}{-0.027778in}}{\pgfqpoint{0.000000in}{0.000000in}}{%
\pgfpathmoveto{\pgfqpoint{0.000000in}{0.000000in}}%
\pgfpathlineto{\pgfqpoint{0.000000in}{-0.027778in}}%
\pgfusepath{stroke,fill}%
}%
\begin{pgfscope}%
\pgfsys@transformshift{1.298348in}{0.330000in}%
\pgfsys@useobject{currentmarker}{}%
\end{pgfscope}%
\end{pgfscope}%
\begin{pgfscope}%
\definecolor{textcolor}{rgb}{0.000000,0.000000,0.000000}%
\pgfsetstrokecolor{textcolor}%
\pgfsetfillcolor{textcolor}%
\pgftext[x=1.298348in,y=0.255000in,,top]{\color{textcolor}\sffamily\fontsize{10.000000}{12.000000}\selectfont \(\displaystyle {2\times10^{1}}\)}%
\end{pgfscope}%
\begin{pgfscope}%
\pgfsetbuttcap%
\pgfsetroundjoin%
\definecolor{currentfill}{rgb}{0.000000,0.000000,0.000000}%
\pgfsetfillcolor{currentfill}%
\pgfsetlinewidth{0.602250pt}%
\definecolor{currentstroke}{rgb}{0.000000,0.000000,0.000000}%
\pgfsetstrokecolor{currentstroke}%
\pgfsetdash{}{0pt}%
\pgfsys@defobject{currentmarker}{\pgfqpoint{0.000000in}{-0.027778in}}{\pgfqpoint{0.000000in}{0.000000in}}{%
\pgfpathmoveto{\pgfqpoint{0.000000in}{0.000000in}}%
\pgfpathlineto{\pgfqpoint{0.000000in}{-0.027778in}}%
\pgfusepath{stroke,fill}%
}%
\begin{pgfscope}%
\pgfsys@transformshift{1.776652in}{0.330000in}%
\pgfsys@useobject{currentmarker}{}%
\end{pgfscope}%
\end{pgfscope}%
\begin{pgfscope}%
\definecolor{textcolor}{rgb}{0.000000,0.000000,0.000000}%
\pgfsetstrokecolor{textcolor}%
\pgfsetfillcolor{textcolor}%
\pgftext[x=1.776652in,y=0.255000in,,top]{\color{textcolor}\sffamily\fontsize{10.000000}{12.000000}\selectfont \(\displaystyle {3\times10^{1}}\)}%
\end{pgfscope}%
\begin{pgfscope}%
\pgfsetbuttcap%
\pgfsetroundjoin%
\definecolor{currentfill}{rgb}{0.000000,0.000000,0.000000}%
\pgfsetfillcolor{currentfill}%
\pgfsetlinewidth{0.602250pt}%
\definecolor{currentstroke}{rgb}{0.000000,0.000000,0.000000}%
\pgfsetstrokecolor{currentstroke}%
\pgfsetdash{}{0pt}%
\pgfsys@defobject{currentmarker}{\pgfqpoint{0.000000in}{-0.027778in}}{\pgfqpoint{0.000000in}{0.000000in}}{%
\pgfpathmoveto{\pgfqpoint{0.000000in}{0.000000in}}%
\pgfpathlineto{\pgfqpoint{0.000000in}{-0.027778in}}%
\pgfusepath{stroke,fill}%
}%
\begin{pgfscope}%
\pgfsys@transformshift{2.116014in}{0.330000in}%
\pgfsys@useobject{currentmarker}{}%
\end{pgfscope}%
\end{pgfscope}%
\begin{pgfscope}%
\definecolor{textcolor}{rgb}{0.000000,0.000000,0.000000}%
\pgfsetstrokecolor{textcolor}%
\pgfsetfillcolor{textcolor}%
\pgftext[x=2.116014in,y=0.255000in,,top]{\color{textcolor}\sffamily\fontsize{10.000000}{12.000000}\selectfont \(\displaystyle {4\times10^{1}}\)}%
\end{pgfscope}%
\begin{pgfscope}%
\pgfsetbuttcap%
\pgfsetroundjoin%
\definecolor{currentfill}{rgb}{0.000000,0.000000,0.000000}%
\pgfsetfillcolor{currentfill}%
\pgfsetlinewidth{0.602250pt}%
\definecolor{currentstroke}{rgb}{0.000000,0.000000,0.000000}%
\pgfsetstrokecolor{currentstroke}%
\pgfsetdash{}{0pt}%
\pgfsys@defobject{currentmarker}{\pgfqpoint{0.000000in}{-0.027778in}}{\pgfqpoint{0.000000in}{0.000000in}}{%
\pgfpathmoveto{\pgfqpoint{0.000000in}{0.000000in}}%
\pgfpathlineto{\pgfqpoint{0.000000in}{-0.027778in}}%
\pgfusepath{stroke,fill}%
}%
\begin{pgfscope}%
\pgfsys@transformshift{2.379244in}{0.330000in}%
\pgfsys@useobject{currentmarker}{}%
\end{pgfscope}%
\end{pgfscope}%
\begin{pgfscope}%
\pgfsetbuttcap%
\pgfsetroundjoin%
\definecolor{currentfill}{rgb}{0.000000,0.000000,0.000000}%
\pgfsetfillcolor{currentfill}%
\pgfsetlinewidth{0.602250pt}%
\definecolor{currentstroke}{rgb}{0.000000,0.000000,0.000000}%
\pgfsetstrokecolor{currentstroke}%
\pgfsetdash{}{0pt}%
\pgfsys@defobject{currentmarker}{\pgfqpoint{0.000000in}{-0.027778in}}{\pgfqpoint{0.000000in}{0.000000in}}{%
\pgfpathmoveto{\pgfqpoint{0.000000in}{0.000000in}}%
\pgfpathlineto{\pgfqpoint{0.000000in}{-0.027778in}}%
\pgfusepath{stroke,fill}%
}%
\begin{pgfscope}%
\pgfsys@transformshift{2.594318in}{0.330000in}%
\pgfsys@useobject{currentmarker}{}%
\end{pgfscope}%
\end{pgfscope}%
\begin{pgfscope}%
\definecolor{textcolor}{rgb}{0.000000,0.000000,0.000000}%
\pgfsetstrokecolor{textcolor}%
\pgfsetfillcolor{textcolor}%
\pgftext[x=2.594318in,y=0.255000in,,top]{\color{textcolor}\sffamily\fontsize{10.000000}{12.000000}\selectfont \(\displaystyle {6\times10^{1}}\)}%
\end{pgfscope}%
\begin{pgfscope}%
\definecolor{textcolor}{rgb}{0.000000,0.000000,0.000000}%
\pgfsetstrokecolor{textcolor}%
\pgfsetfillcolor{textcolor}%
\pgftext[x=1.537500in,y=0.042809in,,top]{\color{textcolor}\sffamily\fontsize{10.000000}{12.000000}\selectfont \(\displaystyle k\)}%
\end{pgfscope}%
\begin{pgfscope}%
\pgfsetbuttcap%
\pgfsetroundjoin%
\definecolor{currentfill}{rgb}{0.000000,0.000000,0.000000}%
\pgfsetfillcolor{currentfill}%
\pgfsetlinewidth{0.803000pt}%
\definecolor{currentstroke}{rgb}{0.000000,0.000000,0.000000}%
\pgfsetstrokecolor{currentstroke}%
\pgfsetdash{}{0pt}%
\pgfsys@defobject{currentmarker}{\pgfqpoint{-0.048611in}{0.000000in}}{\pgfqpoint{0.000000in}{0.000000in}}{%
\pgfpathmoveto{\pgfqpoint{0.000000in}{0.000000in}}%
\pgfpathlineto{\pgfqpoint{-0.048611in}{0.000000in}}%
\pgfusepath{stroke,fill}%
}%
\begin{pgfscope}%
\pgfsys@transformshift{0.375000in}{0.385906in}%
\pgfsys@useobject{currentmarker}{}%
\end{pgfscope}%
\end{pgfscope}%
\begin{pgfscope}%
\definecolor{textcolor}{rgb}{0.000000,0.000000,0.000000}%
\pgfsetstrokecolor{textcolor}%
\pgfsetfillcolor{textcolor}%
\pgftext[x=-0.031467in,y=0.333145in,left,base]{\color{textcolor}\sffamily\fontsize{10.000000}{12.000000}\selectfont 0.65}%
\end{pgfscope}%
\begin{pgfscope}%
\pgfsetbuttcap%
\pgfsetroundjoin%
\definecolor{currentfill}{rgb}{0.000000,0.000000,0.000000}%
\pgfsetfillcolor{currentfill}%
\pgfsetlinewidth{0.803000pt}%
\definecolor{currentstroke}{rgb}{0.000000,0.000000,0.000000}%
\pgfsetstrokecolor{currentstroke}%
\pgfsetdash{}{0pt}%
\pgfsys@defobject{currentmarker}{\pgfqpoint{-0.048611in}{0.000000in}}{\pgfqpoint{0.000000in}{0.000000in}}{%
\pgfpathmoveto{\pgfqpoint{0.000000in}{0.000000in}}%
\pgfpathlineto{\pgfqpoint{-0.048611in}{0.000000in}}%
\pgfusepath{stroke,fill}%
}%
\begin{pgfscope}%
\pgfsys@transformshift{0.375000in}{0.675658in}%
\pgfsys@useobject{currentmarker}{}%
\end{pgfscope}%
\end{pgfscope}%
\begin{pgfscope}%
\definecolor{textcolor}{rgb}{0.000000,0.000000,0.000000}%
\pgfsetstrokecolor{textcolor}%
\pgfsetfillcolor{textcolor}%
\pgftext[x=-0.031467in,y=0.622897in,left,base]{\color{textcolor}\sffamily\fontsize{10.000000}{12.000000}\selectfont 0.70}%
\end{pgfscope}%
\begin{pgfscope}%
\pgfsetbuttcap%
\pgfsetroundjoin%
\definecolor{currentfill}{rgb}{0.000000,0.000000,0.000000}%
\pgfsetfillcolor{currentfill}%
\pgfsetlinewidth{0.803000pt}%
\definecolor{currentstroke}{rgb}{0.000000,0.000000,0.000000}%
\pgfsetstrokecolor{currentstroke}%
\pgfsetdash{}{0pt}%
\pgfsys@defobject{currentmarker}{\pgfqpoint{-0.048611in}{0.000000in}}{\pgfqpoint{0.000000in}{0.000000in}}{%
\pgfpathmoveto{\pgfqpoint{0.000000in}{0.000000in}}%
\pgfpathlineto{\pgfqpoint{-0.048611in}{0.000000in}}%
\pgfusepath{stroke,fill}%
}%
\begin{pgfscope}%
\pgfsys@transformshift{0.375000in}{0.965410in}%
\pgfsys@useobject{currentmarker}{}%
\end{pgfscope}%
\end{pgfscope}%
\begin{pgfscope}%
\definecolor{textcolor}{rgb}{0.000000,0.000000,0.000000}%
\pgfsetstrokecolor{textcolor}%
\pgfsetfillcolor{textcolor}%
\pgftext[x=-0.031467in,y=0.912648in,left,base]{\color{textcolor}\sffamily\fontsize{10.000000}{12.000000}\selectfont 0.75}%
\end{pgfscope}%
\begin{pgfscope}%
\pgfsetbuttcap%
\pgfsetroundjoin%
\definecolor{currentfill}{rgb}{0.000000,0.000000,0.000000}%
\pgfsetfillcolor{currentfill}%
\pgfsetlinewidth{0.803000pt}%
\definecolor{currentstroke}{rgb}{0.000000,0.000000,0.000000}%
\pgfsetstrokecolor{currentstroke}%
\pgfsetdash{}{0pt}%
\pgfsys@defobject{currentmarker}{\pgfqpoint{-0.048611in}{0.000000in}}{\pgfqpoint{0.000000in}{0.000000in}}{%
\pgfpathmoveto{\pgfqpoint{0.000000in}{0.000000in}}%
\pgfpathlineto{\pgfqpoint{-0.048611in}{0.000000in}}%
\pgfusepath{stroke,fill}%
}%
\begin{pgfscope}%
\pgfsys@transformshift{0.375000in}{1.255162in}%
\pgfsys@useobject{currentmarker}{}%
\end{pgfscope}%
\end{pgfscope}%
\begin{pgfscope}%
\definecolor{textcolor}{rgb}{0.000000,0.000000,0.000000}%
\pgfsetstrokecolor{textcolor}%
\pgfsetfillcolor{textcolor}%
\pgftext[x=-0.031467in,y=1.202400in,left,base]{\color{textcolor}\sffamily\fontsize{10.000000}{12.000000}\selectfont 0.80}%
\end{pgfscope}%
\begin{pgfscope}%
\pgfsetbuttcap%
\pgfsetroundjoin%
\definecolor{currentfill}{rgb}{0.000000,0.000000,0.000000}%
\pgfsetfillcolor{currentfill}%
\pgfsetlinewidth{0.803000pt}%
\definecolor{currentstroke}{rgb}{0.000000,0.000000,0.000000}%
\pgfsetstrokecolor{currentstroke}%
\pgfsetdash{}{0pt}%
\pgfsys@defobject{currentmarker}{\pgfqpoint{-0.048611in}{0.000000in}}{\pgfqpoint{0.000000in}{0.000000in}}{%
\pgfpathmoveto{\pgfqpoint{0.000000in}{0.000000in}}%
\pgfpathlineto{\pgfqpoint{-0.048611in}{0.000000in}}%
\pgfusepath{stroke,fill}%
}%
\begin{pgfscope}%
\pgfsys@transformshift{0.375000in}{1.544913in}%
\pgfsys@useobject{currentmarker}{}%
\end{pgfscope}%
\end{pgfscope}%
\begin{pgfscope}%
\definecolor{textcolor}{rgb}{0.000000,0.000000,0.000000}%
\pgfsetstrokecolor{textcolor}%
\pgfsetfillcolor{textcolor}%
\pgftext[x=-0.031467in,y=1.492152in,left,base]{\color{textcolor}\sffamily\fontsize{10.000000}{12.000000}\selectfont 0.85}%
\end{pgfscope}%
\begin{pgfscope}%
\pgfsetbuttcap%
\pgfsetroundjoin%
\definecolor{currentfill}{rgb}{0.000000,0.000000,0.000000}%
\pgfsetfillcolor{currentfill}%
\pgfsetlinewidth{0.803000pt}%
\definecolor{currentstroke}{rgb}{0.000000,0.000000,0.000000}%
\pgfsetstrokecolor{currentstroke}%
\pgfsetdash{}{0pt}%
\pgfsys@defobject{currentmarker}{\pgfqpoint{-0.048611in}{0.000000in}}{\pgfqpoint{0.000000in}{0.000000in}}{%
\pgfpathmoveto{\pgfqpoint{0.000000in}{0.000000in}}%
\pgfpathlineto{\pgfqpoint{-0.048611in}{0.000000in}}%
\pgfusepath{stroke,fill}%
}%
\begin{pgfscope}%
\pgfsys@transformshift{0.375000in}{1.834665in}%
\pgfsys@useobject{currentmarker}{}%
\end{pgfscope}%
\end{pgfscope}%
\begin{pgfscope}%
\definecolor{textcolor}{rgb}{0.000000,0.000000,0.000000}%
\pgfsetstrokecolor{textcolor}%
\pgfsetfillcolor{textcolor}%
\pgftext[x=-0.031467in,y=1.781904in,left,base]{\color{textcolor}\sffamily\fontsize{10.000000}{12.000000}\selectfont 0.90}%
\end{pgfscope}%
\begin{pgfscope}%
\pgfsetbuttcap%
\pgfsetroundjoin%
\definecolor{currentfill}{rgb}{0.000000,0.000000,0.000000}%
\pgfsetfillcolor{currentfill}%
\pgfsetlinewidth{0.803000pt}%
\definecolor{currentstroke}{rgb}{0.000000,0.000000,0.000000}%
\pgfsetstrokecolor{currentstroke}%
\pgfsetdash{}{0pt}%
\pgfsys@defobject{currentmarker}{\pgfqpoint{-0.048611in}{0.000000in}}{\pgfqpoint{0.000000in}{0.000000in}}{%
\pgfpathmoveto{\pgfqpoint{0.000000in}{0.000000in}}%
\pgfpathlineto{\pgfqpoint{-0.048611in}{0.000000in}}%
\pgfusepath{stroke,fill}%
}%
\begin{pgfscope}%
\pgfsys@transformshift{0.375000in}{2.124417in}%
\pgfsys@useobject{currentmarker}{}%
\end{pgfscope}%
\end{pgfscope}%
\begin{pgfscope}%
\definecolor{textcolor}{rgb}{0.000000,0.000000,0.000000}%
\pgfsetstrokecolor{textcolor}%
\pgfsetfillcolor{textcolor}%
\pgftext[x=-0.031467in,y=2.071655in,left,base]{\color{textcolor}\sffamily\fontsize{10.000000}{12.000000}\selectfont 0.95}%
\end{pgfscope}%
\begin{pgfscope}%
\pgfsetbuttcap%
\pgfsetroundjoin%
\definecolor{currentfill}{rgb}{0.000000,0.000000,0.000000}%
\pgfsetfillcolor{currentfill}%
\pgfsetlinewidth{0.803000pt}%
\definecolor{currentstroke}{rgb}{0.000000,0.000000,0.000000}%
\pgfsetstrokecolor{currentstroke}%
\pgfsetdash{}{0pt}%
\pgfsys@defobject{currentmarker}{\pgfqpoint{-0.048611in}{0.000000in}}{\pgfqpoint{0.000000in}{0.000000in}}{%
\pgfpathmoveto{\pgfqpoint{0.000000in}{0.000000in}}%
\pgfpathlineto{\pgfqpoint{-0.048611in}{0.000000in}}%
\pgfusepath{stroke,fill}%
}%
\begin{pgfscope}%
\pgfsys@transformshift{0.375000in}{2.414169in}%
\pgfsys@useobject{currentmarker}{}%
\end{pgfscope}%
\end{pgfscope}%
\begin{pgfscope}%
\definecolor{textcolor}{rgb}{0.000000,0.000000,0.000000}%
\pgfsetstrokecolor{textcolor}%
\pgfsetfillcolor{textcolor}%
\pgftext[x=-0.031467in,y=2.361407in,left,base]{\color{textcolor}\sffamily\fontsize{10.000000}{12.000000}\selectfont 1.00}%
\end{pgfscope}%
\begin{pgfscope}%
\definecolor{textcolor}{rgb}{0.000000,0.000000,0.000000}%
\pgfsetstrokecolor{textcolor}%
\pgfsetfillcolor{textcolor}%
\pgftext[x=-0.087023in,y=1.485000in,,bottom,rotate=90.000000]{\color{textcolor}\sffamily\fontsize{10.000000}{12.000000}\selectfont \(\displaystyle \alpha\)}%
\end{pgfscope}%
\begin{pgfscope}%
\pgfpathrectangle{\pgfqpoint{0.375000in}{0.330000in}}{\pgfqpoint{2.325000in}{2.310000in}}%
\pgfusepath{clip}%
\pgfsetbuttcap%
\pgfsetroundjoin%
\pgfsetlinewidth{1.505625pt}%
\definecolor{currentstroke}{rgb}{0.000000,0.000000,0.000000}%
\pgfsetstrokecolor{currentstroke}%
\pgfsetdash{{5.550000pt}{2.400000pt}}{0.000000pt}%
\pgfpathmoveto{\pgfqpoint{0.480682in}{2.535000in}}%
\pgfpathlineto{\pgfqpoint{1.298348in}{1.813714in}}%
\pgfpathlineto{\pgfqpoint{1.776652in}{1.391789in}}%
\pgfpathlineto{\pgfqpoint{2.116014in}{1.092429in}}%
\pgfpathlineto{\pgfqpoint{2.379244in}{0.860227in}}%
\pgfpathlineto{\pgfqpoint{2.594318in}{0.670504in}}%
\pgfusepath{stroke}%
\end{pgfscope}%
\begin{pgfscope}%
\pgfpathrectangle{\pgfqpoint{0.375000in}{0.330000in}}{\pgfqpoint{2.325000in}{2.310000in}}%
\pgfusepath{clip}%
\pgfsetbuttcap%
\pgfsetroundjoin%
\definecolor{currentfill}{rgb}{0.000000,0.000000,0.000000}%
\pgfsetfillcolor{currentfill}%
\pgfsetlinewidth{1.003750pt}%
\definecolor{currentstroke}{rgb}{0.000000,0.000000,0.000000}%
\pgfsetstrokecolor{currentstroke}%
\pgfsetdash{}{0pt}%
\pgfsys@defobject{currentmarker}{\pgfqpoint{-0.041667in}{-0.041667in}}{\pgfqpoint{0.041667in}{0.041667in}}{%
\pgfpathmoveto{\pgfqpoint{0.000000in}{-0.041667in}}%
\pgfpathcurveto{\pgfqpoint{0.011050in}{-0.041667in}}{\pgfqpoint{0.021649in}{-0.037276in}}{\pgfqpoint{0.029463in}{-0.029463in}}%
\pgfpathcurveto{\pgfqpoint{0.037276in}{-0.021649in}}{\pgfqpoint{0.041667in}{-0.011050in}}{\pgfqpoint{0.041667in}{0.000000in}}%
\pgfpathcurveto{\pgfqpoint{0.041667in}{0.011050in}}{\pgfqpoint{0.037276in}{0.021649in}}{\pgfqpoint{0.029463in}{0.029463in}}%
\pgfpathcurveto{\pgfqpoint{0.021649in}{0.037276in}}{\pgfqpoint{0.011050in}{0.041667in}}{\pgfqpoint{0.000000in}{0.041667in}}%
\pgfpathcurveto{\pgfqpoint{-0.011050in}{0.041667in}}{\pgfqpoint{-0.021649in}{0.037276in}}{\pgfqpoint{-0.029463in}{0.029463in}}%
\pgfpathcurveto{\pgfqpoint{-0.037276in}{0.021649in}}{\pgfqpoint{-0.041667in}{0.011050in}}{\pgfqpoint{-0.041667in}{0.000000in}}%
\pgfpathcurveto{\pgfqpoint{-0.041667in}{-0.011050in}}{\pgfqpoint{-0.037276in}{-0.021649in}}{\pgfqpoint{-0.029463in}{-0.029463in}}%
\pgfpathcurveto{\pgfqpoint{-0.021649in}{-0.037276in}}{\pgfqpoint{-0.011050in}{-0.041667in}}{\pgfqpoint{0.000000in}{-0.041667in}}%
\pgfpathclose%
\pgfusepath{stroke,fill}%
}%
\begin{pgfscope}%
\pgfsys@transformshift{0.480682in}{2.496116in}%
\pgfsys@useobject{currentmarker}{}%
\end{pgfscope}%
\begin{pgfscope}%
\pgfsys@transformshift{1.298348in}{1.765386in}%
\pgfsys@useobject{currentmarker}{}%
\end{pgfscope}%
\begin{pgfscope}%
\pgfsys@transformshift{1.776652in}{1.612099in}%
\pgfsys@useobject{currentmarker}{}%
\end{pgfscope}%
\begin{pgfscope}%
\pgfsys@transformshift{2.116014in}{0.874574in}%
\pgfsys@useobject{currentmarker}{}%
\end{pgfscope}%
\begin{pgfscope}%
\pgfsys@transformshift{2.379244in}{1.180488in}%
\pgfsys@useobject{currentmarker}{}%
\end{pgfscope}%
\begin{pgfscope}%
\pgfsys@transformshift{2.594318in}{0.435000in}%
\pgfsys@useobject{currentmarker}{}%
\end{pgfscope}%
\end{pgfscope}%
\begin{pgfscope}%
\pgfsetrectcap%
\pgfsetmiterjoin%
\pgfsetlinewidth{0.803000pt}%
\definecolor{currentstroke}{rgb}{0.000000,0.000000,0.000000}%
\pgfsetstrokecolor{currentstroke}%
\pgfsetdash{}{0pt}%
\pgfpathmoveto{\pgfqpoint{0.375000in}{0.330000in}}%
\pgfpathlineto{\pgfqpoint{0.375000in}{2.640000in}}%
\pgfusepath{stroke}%
\end{pgfscope}%
\begin{pgfscope}%
\pgfsetrectcap%
\pgfsetmiterjoin%
\pgfsetlinewidth{0.803000pt}%
\definecolor{currentstroke}{rgb}{0.000000,0.000000,0.000000}%
\pgfsetstrokecolor{currentstroke}%
\pgfsetdash{}{0pt}%
\pgfpathmoveto{\pgfqpoint{2.700000in}{0.330000in}}%
\pgfpathlineto{\pgfqpoint{2.700000in}{2.640000in}}%
\pgfusepath{stroke}%
\end{pgfscope}%
\begin{pgfscope}%
\pgfsetrectcap%
\pgfsetmiterjoin%
\pgfsetlinewidth{0.803000pt}%
\definecolor{currentstroke}{rgb}{0.000000,0.000000,0.000000}%
\pgfsetstrokecolor{currentstroke}%
\pgfsetdash{}{0pt}%
\pgfpathmoveto{\pgfqpoint{0.375000in}{0.330000in}}%
\pgfpathlineto{\pgfqpoint{2.700000in}{0.330000in}}%
\pgfusepath{stroke}%
\end{pgfscope}%
\begin{pgfscope}%
\pgfsetrectcap%
\pgfsetmiterjoin%
\pgfsetlinewidth{0.803000pt}%
\definecolor{currentstroke}{rgb}{0.000000,0.000000,0.000000}%
\pgfsetstrokecolor{currentstroke}%
\pgfsetdash{}{0pt}%
\pgfpathmoveto{\pgfqpoint{0.375000in}{2.640000in}}%
\pgfpathlineto{\pgfqpoint{2.700000in}{2.640000in}}%
\pgfusepath{stroke}%
\end{pgfscope}%
\begin{pgfscope}%
\pgfsetbuttcap%
\pgfsetmiterjoin%
\definecolor{currentfill}{rgb}{1.000000,1.000000,1.000000}%
\pgfsetfillcolor{currentfill}%
\pgfsetfillopacity{0.800000}%
\pgfsetlinewidth{1.003750pt}%
\definecolor{currentstroke}{rgb}{0.800000,0.800000,0.800000}%
\pgfsetstrokecolor{currentstroke}%
\pgfsetstrokeopacity{0.800000}%
\pgfsetdash{}{0pt}%
\pgfpathmoveto{\pgfqpoint{0.472222in}{0.399444in}}%
\pgfpathlineto{\pgfqpoint{2.557368in}{0.399444in}}%
\pgfpathquadraticcurveto{\pgfqpoint{2.585146in}{0.399444in}}{\pgfqpoint{2.585146in}{0.427222in}}%
\pgfpathlineto{\pgfqpoint{2.585146in}{0.623023in}}%
\pgfpathquadraticcurveto{\pgfqpoint{2.585146in}{0.650801in}}{\pgfqpoint{2.557368in}{0.650801in}}%
\pgfpathlineto{\pgfqpoint{0.472222in}{0.650801in}}%
\pgfpathquadraticcurveto{\pgfqpoint{0.444444in}{0.650801in}}{\pgfqpoint{0.444444in}{0.623023in}}%
\pgfpathlineto{\pgfqpoint{0.444444in}{0.427222in}}%
\pgfpathquadraticcurveto{\pgfqpoint{0.444444in}{0.399444in}}{\pgfqpoint{0.472222in}{0.399444in}}%
\pgfpathclose%
\pgfusepath{stroke,fill}%
\end{pgfscope}%
\begin{pgfscope}%
\pgfsetbuttcap%
\pgfsetroundjoin%
\pgfsetlinewidth{1.505625pt}%
\definecolor{currentstroke}{rgb}{0.000000,0.000000,0.000000}%
\pgfsetstrokecolor{currentstroke}%
\pgfsetdash{{5.550000pt}{2.400000pt}}{0.000000pt}%
\pgfpathmoveto{\pgfqpoint{0.500000in}{0.538333in}}%
\pgfpathlineto{\pgfqpoint{0.777778in}{0.538333in}}%
\pgfusepath{stroke}%
\end{pgfscope}%
\begin{pgfscope}%
\definecolor{textcolor}{rgb}{0.000000,0.000000,0.000000}%
\pgfsetstrokecolor{textcolor}%
\pgfsetfillcolor{textcolor}%
\pgftext[x=0.888889in,y=0.489722in,left,base]{\color{textcolor}\sffamily\fontsize{10.000000}{12.000000}\selectfont \(\displaystyle \alpha = 1.4343 - 0.4134\log(k)\)}%
\end{pgfscope}%
\end{pgfpicture}%
\makeatother%
\endgroup%

    \end{subfigure}
\caption[The computed Quasi-Monte-Carlo convergence rate for $Q(u) =  u(1,1)$.]{The computed values of $C$ (top) and $\alpha$ (bottom) against $k$ in \cref{eq:qmcerrorform} for $Q(u) = u((1,1))$. Observe the $x$-axes are on a $\log_{10}$ scale, but $\loge$ is the natural logarithm.  \label{fig:toprightCalpha}}
\end{figure}

\begin{figure}[h]
    \centering
    \begin{subfigure}{\textwidth}
            \centering
%% Creator: Matplotlib, PGF backend
%%
%% To include the figure in your LaTeX document, write
%%   \input{<filename>.pgf}
%%
%% Make sure the required packages are loaded in your preamble
%%   \usepackage{pgf}
%%
%% Figures using additional raster images can only be included by \input if
%% they are in the same directory as the main LaTeX file. For loading figures
%% from other directories you can use the `import` package
%%   \usepackage{import}
%% and then include the figures with
%%   \import{<path to file>}{<filename>.pgf}
%%
%% Matplotlib used the following preamble
%%   \usepackage{fontspec}
%%   \setmainfont{DejaVuSerif.ttf}[Path=/home/owen/progs/firedrake-complex/firedrake/lib/python3.5/site-packages/matplotlib/mpl-data/fonts/ttf/]
%%   \setsansfont{DejaVuSans.ttf}[Path=/home/owen/progs/firedrake-complex/firedrake/lib/python3.5/site-packages/matplotlib/mpl-data/fonts/ttf/]
%%   \setmonofont{DejaVuSansMono.ttf}[Path=/home/owen/progs/firedrake-complex/firedrake/lib/python3.5/site-packages/matplotlib/mpl-data/fonts/ttf/]
%%
\begingroup%
\makeatletter%
\begin{pgfpicture}%
\pgfpathrectangle{\pgfpointorigin}{\pgfqpoint{5.000000in}{4.000000in}}%
\pgfusepath{use as bounding box, clip}%
\begin{pgfscope}%
\pgfsetbuttcap%
\pgfsetmiterjoin%
\definecolor{currentfill}{rgb}{1.000000,1.000000,1.000000}%
\pgfsetfillcolor{currentfill}%
\pgfsetlinewidth{0.000000pt}%
\definecolor{currentstroke}{rgb}{1.000000,1.000000,1.000000}%
\pgfsetstrokecolor{currentstroke}%
\pgfsetdash{}{0pt}%
\pgfpathmoveto{\pgfqpoint{0.000000in}{0.000000in}}%
\pgfpathlineto{\pgfqpoint{5.000000in}{0.000000in}}%
\pgfpathlineto{\pgfqpoint{5.000000in}{4.000000in}}%
\pgfpathlineto{\pgfqpoint{0.000000in}{4.000000in}}%
\pgfpathclose%
\pgfusepath{fill}%
\end{pgfscope}%
\begin{pgfscope}%
\pgfsetbuttcap%
\pgfsetmiterjoin%
\definecolor{currentfill}{rgb}{1.000000,1.000000,1.000000}%
\pgfsetfillcolor{currentfill}%
\pgfsetlinewidth{0.000000pt}%
\definecolor{currentstroke}{rgb}{0.000000,0.000000,0.000000}%
\pgfsetstrokecolor{currentstroke}%
\pgfsetstrokeopacity{0.000000}%
\pgfsetdash{}{0pt}%
\pgfpathmoveto{\pgfqpoint{0.511963in}{0.582778in}}%
\pgfpathlineto{\pgfqpoint{4.810222in}{0.582778in}}%
\pgfpathlineto{\pgfqpoint{4.810222in}{3.815000in}}%
\pgfpathlineto{\pgfqpoint{0.511963in}{3.815000in}}%
\pgfpathclose%
\pgfusepath{fill}%
\end{pgfscope}%
\begin{pgfscope}%
\pgfsetbuttcap%
\pgfsetroundjoin%
\definecolor{currentfill}{rgb}{0.000000,0.000000,0.000000}%
\pgfsetfillcolor{currentfill}%
\pgfsetlinewidth{0.803000pt}%
\definecolor{currentstroke}{rgb}{0.000000,0.000000,0.000000}%
\pgfsetstrokecolor{currentstroke}%
\pgfsetdash{}{0pt}%
\pgfsys@defobject{currentmarker}{\pgfqpoint{0.000000in}{-0.048611in}}{\pgfqpoint{0.000000in}{0.000000in}}{%
\pgfpathmoveto{\pgfqpoint{0.000000in}{0.000000in}}%
\pgfpathlineto{\pgfqpoint{0.000000in}{-0.048611in}}%
\pgfusepath{stroke,fill}%
}%
\begin{pgfscope}%
\pgfsys@transformshift{0.707338in}{0.582778in}%
\pgfsys@useobject{currentmarker}{}%
\end{pgfscope}%
\end{pgfscope}%
\begin{pgfscope}%
\definecolor{textcolor}{rgb}{0.000000,0.000000,0.000000}%
\pgfsetstrokecolor{textcolor}%
\pgfsetfillcolor{textcolor}%
\pgftext[x=0.707338in,y=0.485556in,,top]{\color{textcolor}\sffamily\fontsize{10.000000}{12.000000}\selectfont \(\displaystyle 10^{1}\)}%
\end{pgfscope}%
\begin{pgfscope}%
\pgfsetbuttcap%
\pgfsetroundjoin%
\definecolor{currentfill}{rgb}{0.000000,0.000000,0.000000}%
\pgfsetfillcolor{currentfill}%
\pgfsetlinewidth{0.602250pt}%
\definecolor{currentstroke}{rgb}{0.000000,0.000000,0.000000}%
\pgfsetstrokecolor{currentstroke}%
\pgfsetdash{}{0pt}%
\pgfsys@defobject{currentmarker}{\pgfqpoint{0.000000in}{-0.027778in}}{\pgfqpoint{0.000000in}{0.000000in}}{%
\pgfpathmoveto{\pgfqpoint{0.000000in}{0.000000in}}%
\pgfpathlineto{\pgfqpoint{0.000000in}{-0.027778in}}%
\pgfusepath{stroke,fill}%
}%
\begin{pgfscope}%
\pgfsys@transformshift{2.218969in}{0.582778in}%
\pgfsys@useobject{currentmarker}{}%
\end{pgfscope}%
\end{pgfscope}%
\begin{pgfscope}%
\definecolor{textcolor}{rgb}{0.000000,0.000000,0.000000}%
\pgfsetstrokecolor{textcolor}%
\pgfsetfillcolor{textcolor}%
\pgftext[x=2.218969in,y=0.507778in,,top]{\color{textcolor}\sffamily\fontsize{10.000000}{12.000000}\selectfont \(\displaystyle 2\times10^{1}\)}%
\end{pgfscope}%
\begin{pgfscope}%
\pgfsetbuttcap%
\pgfsetroundjoin%
\definecolor{currentfill}{rgb}{0.000000,0.000000,0.000000}%
\pgfsetfillcolor{currentfill}%
\pgfsetlinewidth{0.602250pt}%
\definecolor{currentstroke}{rgb}{0.000000,0.000000,0.000000}%
\pgfsetstrokecolor{currentstroke}%
\pgfsetdash{}{0pt}%
\pgfsys@defobject{currentmarker}{\pgfqpoint{0.000000in}{-0.027778in}}{\pgfqpoint{0.000000in}{0.000000in}}{%
\pgfpathmoveto{\pgfqpoint{0.000000in}{0.000000in}}%
\pgfpathlineto{\pgfqpoint{0.000000in}{-0.027778in}}%
\pgfusepath{stroke,fill}%
}%
\begin{pgfscope}%
\pgfsys@transformshift{3.103216in}{0.582778in}%
\pgfsys@useobject{currentmarker}{}%
\end{pgfscope}%
\end{pgfscope}%
\begin{pgfscope}%
\definecolor{textcolor}{rgb}{0.000000,0.000000,0.000000}%
\pgfsetstrokecolor{textcolor}%
\pgfsetfillcolor{textcolor}%
\pgftext[x=3.103216in,y=0.507778in,,top]{\color{textcolor}\sffamily\fontsize{10.000000}{12.000000}\selectfont \(\displaystyle 3\times10^{1}\)}%
\end{pgfscope}%
\begin{pgfscope}%
\pgfsetbuttcap%
\pgfsetroundjoin%
\definecolor{currentfill}{rgb}{0.000000,0.000000,0.000000}%
\pgfsetfillcolor{currentfill}%
\pgfsetlinewidth{0.602250pt}%
\definecolor{currentstroke}{rgb}{0.000000,0.000000,0.000000}%
\pgfsetstrokecolor{currentstroke}%
\pgfsetdash{}{0pt}%
\pgfsys@defobject{currentmarker}{\pgfqpoint{0.000000in}{-0.027778in}}{\pgfqpoint{0.000000in}{0.000000in}}{%
\pgfpathmoveto{\pgfqpoint{0.000000in}{0.000000in}}%
\pgfpathlineto{\pgfqpoint{0.000000in}{-0.027778in}}%
\pgfusepath{stroke,fill}%
}%
\begin{pgfscope}%
\pgfsys@transformshift{3.730599in}{0.582778in}%
\pgfsys@useobject{currentmarker}{}%
\end{pgfscope}%
\end{pgfscope}%
\begin{pgfscope}%
\definecolor{textcolor}{rgb}{0.000000,0.000000,0.000000}%
\pgfsetstrokecolor{textcolor}%
\pgfsetfillcolor{textcolor}%
\pgftext[x=3.730599in,y=0.507778in,,top]{\color{textcolor}\sffamily\fontsize{10.000000}{12.000000}\selectfont \(\displaystyle 4\times10^{1}\)}%
\end{pgfscope}%
\begin{pgfscope}%
\pgfsetbuttcap%
\pgfsetroundjoin%
\definecolor{currentfill}{rgb}{0.000000,0.000000,0.000000}%
\pgfsetfillcolor{currentfill}%
\pgfsetlinewidth{0.602250pt}%
\definecolor{currentstroke}{rgb}{0.000000,0.000000,0.000000}%
\pgfsetstrokecolor{currentstroke}%
\pgfsetdash{}{0pt}%
\pgfsys@defobject{currentmarker}{\pgfqpoint{0.000000in}{-0.027778in}}{\pgfqpoint{0.000000in}{0.000000in}}{%
\pgfpathmoveto{\pgfqpoint{0.000000in}{0.000000in}}%
\pgfpathlineto{\pgfqpoint{0.000000in}{-0.027778in}}%
\pgfusepath{stroke,fill}%
}%
\begin{pgfscope}%
\pgfsys@transformshift{4.217235in}{0.582778in}%
\pgfsys@useobject{currentmarker}{}%
\end{pgfscope}%
\end{pgfscope}%
\begin{pgfscope}%
\pgfsetbuttcap%
\pgfsetroundjoin%
\definecolor{currentfill}{rgb}{0.000000,0.000000,0.000000}%
\pgfsetfillcolor{currentfill}%
\pgfsetlinewidth{0.602250pt}%
\definecolor{currentstroke}{rgb}{0.000000,0.000000,0.000000}%
\pgfsetstrokecolor{currentstroke}%
\pgfsetdash{}{0pt}%
\pgfsys@defobject{currentmarker}{\pgfqpoint{0.000000in}{-0.027778in}}{\pgfqpoint{0.000000in}{0.000000in}}{%
\pgfpathmoveto{\pgfqpoint{0.000000in}{0.000000in}}%
\pgfpathlineto{\pgfqpoint{0.000000in}{-0.027778in}}%
\pgfusepath{stroke,fill}%
}%
\begin{pgfscope}%
\pgfsys@transformshift{4.614846in}{0.582778in}%
\pgfsys@useobject{currentmarker}{}%
\end{pgfscope}%
\end{pgfscope}%
\begin{pgfscope}%
\definecolor{textcolor}{rgb}{0.000000,0.000000,0.000000}%
\pgfsetstrokecolor{textcolor}%
\pgfsetfillcolor{textcolor}%
\pgftext[x=4.614846in,y=0.507778in,,top]{\color{textcolor}\sffamily\fontsize{10.000000}{12.000000}\selectfont \(\displaystyle 6\times10^{1}\)}%
\end{pgfscope}%
\begin{pgfscope}%
\definecolor{textcolor}{rgb}{0.000000,0.000000,0.000000}%
\pgfsetstrokecolor{textcolor}%
\pgfsetfillcolor{textcolor}%
\pgftext[x=2.661092in,y=0.295587in,,top]{\color{textcolor}\sffamily\fontsize{10.000000}{12.000000}\selectfont \(\displaystyle k\)}%
\end{pgfscope}%
\begin{pgfscope}%
\pgfsetbuttcap%
\pgfsetroundjoin%
\definecolor{currentfill}{rgb}{0.000000,0.000000,0.000000}%
\pgfsetfillcolor{currentfill}%
\pgfsetlinewidth{0.803000pt}%
\definecolor{currentstroke}{rgb}{0.000000,0.000000,0.000000}%
\pgfsetstrokecolor{currentstroke}%
\pgfsetdash{}{0pt}%
\pgfsys@defobject{currentmarker}{\pgfqpoint{-0.048611in}{0.000000in}}{\pgfqpoint{0.000000in}{0.000000in}}{%
\pgfpathmoveto{\pgfqpoint{0.000000in}{0.000000in}}%
\pgfpathlineto{\pgfqpoint{-0.048611in}{0.000000in}}%
\pgfusepath{stroke,fill}%
}%
\begin{pgfscope}%
\pgfsys@transformshift{0.511963in}{1.100383in}%
\pgfsys@useobject{currentmarker}{}%
\end{pgfscope}%
\end{pgfscope}%
\begin{pgfscope}%
\definecolor{textcolor}{rgb}{0.000000,0.000000,0.000000}%
\pgfsetstrokecolor{textcolor}%
\pgfsetfillcolor{textcolor}%
\pgftext[x=0.345296in,y=1.047621in,left,base]{\color{textcolor}\sffamily\fontsize{10.000000}{12.000000}\selectfont \(\displaystyle 2\)}%
\end{pgfscope}%
\begin{pgfscope}%
\pgfsetbuttcap%
\pgfsetroundjoin%
\definecolor{currentfill}{rgb}{0.000000,0.000000,0.000000}%
\pgfsetfillcolor{currentfill}%
\pgfsetlinewidth{0.803000pt}%
\definecolor{currentstroke}{rgb}{0.000000,0.000000,0.000000}%
\pgfsetstrokecolor{currentstroke}%
\pgfsetdash{}{0pt}%
\pgfsys@defobject{currentmarker}{\pgfqpoint{-0.048611in}{0.000000in}}{\pgfqpoint{0.000000in}{0.000000in}}{%
\pgfpathmoveto{\pgfqpoint{0.000000in}{0.000000in}}%
\pgfpathlineto{\pgfqpoint{-0.048611in}{0.000000in}}%
\pgfusepath{stroke,fill}%
}%
\begin{pgfscope}%
\pgfsys@transformshift{0.511963in}{1.795343in}%
\pgfsys@useobject{currentmarker}{}%
\end{pgfscope}%
\end{pgfscope}%
\begin{pgfscope}%
\definecolor{textcolor}{rgb}{0.000000,0.000000,0.000000}%
\pgfsetstrokecolor{textcolor}%
\pgfsetfillcolor{textcolor}%
\pgftext[x=0.345296in,y=1.742582in,left,base]{\color{textcolor}\sffamily\fontsize{10.000000}{12.000000}\selectfont \(\displaystyle 4\)}%
\end{pgfscope}%
\begin{pgfscope}%
\pgfsetbuttcap%
\pgfsetroundjoin%
\definecolor{currentfill}{rgb}{0.000000,0.000000,0.000000}%
\pgfsetfillcolor{currentfill}%
\pgfsetlinewidth{0.803000pt}%
\definecolor{currentstroke}{rgb}{0.000000,0.000000,0.000000}%
\pgfsetstrokecolor{currentstroke}%
\pgfsetdash{}{0pt}%
\pgfsys@defobject{currentmarker}{\pgfqpoint{-0.048611in}{0.000000in}}{\pgfqpoint{0.000000in}{0.000000in}}{%
\pgfpathmoveto{\pgfqpoint{0.000000in}{0.000000in}}%
\pgfpathlineto{\pgfqpoint{-0.048611in}{0.000000in}}%
\pgfusepath{stroke,fill}%
}%
\begin{pgfscope}%
\pgfsys@transformshift{0.511963in}{2.490303in}%
\pgfsys@useobject{currentmarker}{}%
\end{pgfscope}%
\end{pgfscope}%
\begin{pgfscope}%
\definecolor{textcolor}{rgb}{0.000000,0.000000,0.000000}%
\pgfsetstrokecolor{textcolor}%
\pgfsetfillcolor{textcolor}%
\pgftext[x=0.345296in,y=2.437542in,left,base]{\color{textcolor}\sffamily\fontsize{10.000000}{12.000000}\selectfont \(\displaystyle 6\)}%
\end{pgfscope}%
\begin{pgfscope}%
\pgfsetbuttcap%
\pgfsetroundjoin%
\definecolor{currentfill}{rgb}{0.000000,0.000000,0.000000}%
\pgfsetfillcolor{currentfill}%
\pgfsetlinewidth{0.803000pt}%
\definecolor{currentstroke}{rgb}{0.000000,0.000000,0.000000}%
\pgfsetstrokecolor{currentstroke}%
\pgfsetdash{}{0pt}%
\pgfsys@defobject{currentmarker}{\pgfqpoint{-0.048611in}{0.000000in}}{\pgfqpoint{0.000000in}{0.000000in}}{%
\pgfpathmoveto{\pgfqpoint{0.000000in}{0.000000in}}%
\pgfpathlineto{\pgfqpoint{-0.048611in}{0.000000in}}%
\pgfusepath{stroke,fill}%
}%
\begin{pgfscope}%
\pgfsys@transformshift{0.511963in}{3.185263in}%
\pgfsys@useobject{currentmarker}{}%
\end{pgfscope}%
\end{pgfscope}%
\begin{pgfscope}%
\definecolor{textcolor}{rgb}{0.000000,0.000000,0.000000}%
\pgfsetstrokecolor{textcolor}%
\pgfsetfillcolor{textcolor}%
\pgftext[x=0.345296in,y=3.132502in,left,base]{\color{textcolor}\sffamily\fontsize{10.000000}{12.000000}\selectfont \(\displaystyle 8\)}%
\end{pgfscope}%
\begin{pgfscope}%
\definecolor{textcolor}{rgb}{0.000000,0.000000,0.000000}%
\pgfsetstrokecolor{textcolor}%
\pgfsetfillcolor{textcolor}%
\pgftext[x=0.289740in,y=2.198889in,,bottom,rotate=90.000000]{\color{textcolor}\sffamily\fontsize{10.000000}{12.000000}\selectfont \(\displaystyle C\)}%
\end{pgfscope}%
\begin{pgfscope}%
\pgfpathrectangle{\pgfqpoint{0.511963in}{0.582778in}}{\pgfqpoint{4.298259in}{3.232222in}}%
\pgfusepath{clip}%
\pgfsetbuttcap%
\pgfsetroundjoin%
\definecolor{currentfill}{rgb}{0.000000,0.000000,0.000000}%
\pgfsetfillcolor{currentfill}%
\pgfsetlinewidth{1.003750pt}%
\definecolor{currentstroke}{rgb}{0.000000,0.000000,0.000000}%
\pgfsetstrokecolor{currentstroke}%
\pgfsetdash{}{0pt}%
\pgfsys@defobject{currentmarker}{\pgfqpoint{-0.041667in}{-0.041667in}}{\pgfqpoint{0.041667in}{0.041667in}}{%
\pgfpathmoveto{\pgfqpoint{0.000000in}{-0.041667in}}%
\pgfpathcurveto{\pgfqpoint{0.011050in}{-0.041667in}}{\pgfqpoint{0.021649in}{-0.037276in}}{\pgfqpoint{0.029463in}{-0.029463in}}%
\pgfpathcurveto{\pgfqpoint{0.037276in}{-0.021649in}}{\pgfqpoint{0.041667in}{-0.011050in}}{\pgfqpoint{0.041667in}{0.000000in}}%
\pgfpathcurveto{\pgfqpoint{0.041667in}{0.011050in}}{\pgfqpoint{0.037276in}{0.021649in}}{\pgfqpoint{0.029463in}{0.029463in}}%
\pgfpathcurveto{\pgfqpoint{0.021649in}{0.037276in}}{\pgfqpoint{0.011050in}{0.041667in}}{\pgfqpoint{0.000000in}{0.041667in}}%
\pgfpathcurveto{\pgfqpoint{-0.011050in}{0.041667in}}{\pgfqpoint{-0.021649in}{0.037276in}}{\pgfqpoint{-0.029463in}{0.029463in}}%
\pgfpathcurveto{\pgfqpoint{-0.037276in}{0.021649in}}{\pgfqpoint{-0.041667in}{0.011050in}}{\pgfqpoint{-0.041667in}{0.000000in}}%
\pgfpathcurveto{\pgfqpoint{-0.041667in}{-0.011050in}}{\pgfqpoint{-0.037276in}{-0.021649in}}{\pgfqpoint{-0.029463in}{-0.029463in}}%
\pgfpathcurveto{\pgfqpoint{-0.021649in}{-0.037276in}}{\pgfqpoint{-0.011050in}{-0.041667in}}{\pgfqpoint{0.000000in}{-0.041667in}}%
\pgfpathclose%
\pgfusepath{stroke,fill}%
}%
\begin{pgfscope}%
\pgfsys@transformshift{0.707338in}{0.729697in}%
\pgfsys@useobject{currentmarker}{}%
\end{pgfscope}%
\begin{pgfscope}%
\pgfsys@transformshift{2.218969in}{1.330412in}%
\pgfsys@useobject{currentmarker}{}%
\end{pgfscope}%
\begin{pgfscope}%
\pgfsys@transformshift{3.103216in}{2.289398in}%
\pgfsys@useobject{currentmarker}{}%
\end{pgfscope}%
\begin{pgfscope}%
\pgfsys@transformshift{3.730599in}{2.640983in}%
\pgfsys@useobject{currentmarker}{}%
\end{pgfscope}%
\begin{pgfscope}%
\pgfsys@transformshift{4.217235in}{3.668081in}%
\pgfsys@useobject{currentmarker}{}%
\end{pgfscope}%
\begin{pgfscope}%
\pgfsys@transformshift{4.614846in}{3.387773in}%
\pgfsys@useobject{currentmarker}{}%
\end{pgfscope}%
\end{pgfscope}%
\begin{pgfscope}%
\pgfsetrectcap%
\pgfsetmiterjoin%
\pgfsetlinewidth{0.803000pt}%
\definecolor{currentstroke}{rgb}{0.000000,0.000000,0.000000}%
\pgfsetstrokecolor{currentstroke}%
\pgfsetdash{}{0pt}%
\pgfpathmoveto{\pgfqpoint{0.511963in}{0.582778in}}%
\pgfpathlineto{\pgfqpoint{0.511963in}{3.815000in}}%
\pgfusepath{stroke}%
\end{pgfscope}%
\begin{pgfscope}%
\pgfsetrectcap%
\pgfsetmiterjoin%
\pgfsetlinewidth{0.803000pt}%
\definecolor{currentstroke}{rgb}{0.000000,0.000000,0.000000}%
\pgfsetstrokecolor{currentstroke}%
\pgfsetdash{}{0pt}%
\pgfpathmoveto{\pgfqpoint{4.810222in}{0.582778in}}%
\pgfpathlineto{\pgfqpoint{4.810222in}{3.815000in}}%
\pgfusepath{stroke}%
\end{pgfscope}%
\begin{pgfscope}%
\pgfsetrectcap%
\pgfsetmiterjoin%
\pgfsetlinewidth{0.803000pt}%
\definecolor{currentstroke}{rgb}{0.000000,0.000000,0.000000}%
\pgfsetstrokecolor{currentstroke}%
\pgfsetdash{}{0pt}%
\pgfpathmoveto{\pgfqpoint{0.511963in}{0.582778in}}%
\pgfpathlineto{\pgfqpoint{4.810222in}{0.582778in}}%
\pgfusepath{stroke}%
\end{pgfscope}%
\begin{pgfscope}%
\pgfsetrectcap%
\pgfsetmiterjoin%
\pgfsetlinewidth{0.803000pt}%
\definecolor{currentstroke}{rgb}{0.000000,0.000000,0.000000}%
\pgfsetstrokecolor{currentstroke}%
\pgfsetdash{}{0pt}%
\pgfpathmoveto{\pgfqpoint{0.511963in}{3.815000in}}%
\pgfpathlineto{\pgfqpoint{4.810222in}{3.815000in}}%
\pgfusepath{stroke}%
\end{pgfscope}%
\end{pgfpicture}%
\makeatother%
\endgroup%

  \end{subfigure}
    \begin{subfigure}{\textwidth}
            \centering 
\input{gradient_top_right-alpha-plot.pgf}
    \end{subfigure}
\caption[The computed Quasi-Monte-Carlo convergence rate for $Q(u) =  \gradu(1,1)$.]{The computed values of $C$ (top) and $\alpha$ (bottom) against $k$ in \cref{eq:qmcerrorform} for $Q(u) = \gradu((1,1))$. Observe the $x$-axes are on a $\log_{10}$ scale, but $\loge$ is the natural logarithm.  \label{fig:gradienttoprightCalpha}}
\end{figure}


Motivated by QMC theory for other applications, e.g., \cite[Equation 4.2]{GrKuNuScSl:11}, we test experimentally the assumption that the QMC error satisfies
\beq\label{eq:qmcerrorform}
\QMCerror{Q}{\Nshifts} = C \NQMC^{-\alpha},
\eeq
for some $C, \alpha > 0.$ Using data for the values of $k$ listed above, \cref{fig:integralCalpha,fig:originCalpha,fig:toprightCalpha,fig:gradienttoprightCalpha} plot the computed values of $C$ and $\alpha$ against $k$. (In \cref{app:hhqmcconv}, we plot the QMC error for increasing $\NQMC$ for each $k \in \set{10,20,30,40,50,60}$ and for each QoI---these plots allow us to determine the values of $C$ and $\alpha$ for each value of $k.$) For the QoIs that are point evaluations (\cref{fig:originCalpha,fig:toprightCalpha}), $C$ appears not to vary very much; thus we assume $C$ is constant in all of the following calculations.

\cref{fig:integralCalpha,fig:originCalpha,fig:toprightCalpha,fig:gradienttoprightCalpha} (bottom panes) show $\alpha$ decreasing at a rate proportional to $\log k$. Therefore we conjecture
\beq\label{eq:alphaform}
\alpha(k) = \alphaz - \alphao\loge(k),
\eeq
for some constants $\alphaz,\alphao > 0.$ (Throughout this \lcnamecref{sec:nbpcqmcnumerics}, $\loge$ denotes the natural logarithm.) We fitted $\alphaz$ and $\alphao$ numerically, and have plotted the resulting line of best fit on \cref{fig:integralCalpha,fig:originCalpha,fig:toprightCalpha,fig:gradienttoprightCalpha}. (Observe that the conjectured form \cref{eq:alphaform} cannot hold for $k$ very large, as then $\alpha(k)$ would be negative, and there would be no convergence as the number of QMC points is increased. Nevertheless, for the range of $k$ we consider in these numerical experiments, the form \cref{eq:alphaform} seems to give a good fit with the data.) The values of $C$ and $\alpha$ for the different QoIs are given in \cref{fig:integralCalpha,fig:originCalpha,fig:toprightCalpha,fig:gradienttoprightCalpha}.

Having understood how the QMC error increases with $k$ for fixed $\NQMC$, we now use this knowledge to determine how one should increase $\NQMC$ with $k$ in order to keep the QMC error bounded. Recalling that we assume $C$ in \cref{eq:qmcerrorform} is constant, if we take
\beq\label{eq:Nform}
\NQMC(k) = \exp\mleft(\Ctilde \alpha(k)^{-1}\mright),
\eeq
for some constant $\Ctilde > 0$, then substituting \cref{eq:Nform} into \cref{eq:qmcerrorform}, we see that the QMC error should remain bounded, with
\beqs
\QMCerror{Q}{\Nshifts} = C \exp\mleft(-\Ctilde\mright).
\eeqs
Observe that, since $\alpha(k)$ decreases as $k$ increases, \cref{eq:Nform} will increase as $k$ increases.

In our numerical experiments with increasing $\NQMC(k)$ below, we set $\Ctilde$ so that $\NQMC(10) = 2048,$ because in our numerical experiments to determine the behaviour of the QMC error, we used $\NQMC = 2048$ (with 20 shifts). Also in our numerical experiments below we take the number of QMC points to be a power of 2, because the lattice rule we use to generate the points is a complete lattice rule if $\NQMC$ is a power of 2 (see \cite{NuREADME}). We choose $\NQMC$ to be a power of 2 by setting $\NQMC(k) = 2^{M(k)},$ where
\beqs
M(k) = \round{\logtwo\mleft(\exp\mleft(\Ctilde \alpha(k)^{-1}\mright)\mright)}.
\eeqs

Based on the results for the QoIs in \cref{tab:qmcalpha} (excluding the results for the QoI being the integral of $u$ and $\grad u((1,1))$, as these seem to display slightly different convergence characteristics), in our numerical experiments below we take $\alpha(k) = 1.38 - 0.19  \loge(k).$ The resulting values of $\NQMC$ are summarised in \cref{tab:qmcpoints}.

\begin{table}[h!]
  \centering
  \begin{tabular}{Sc Sc Sc Sc Sc}
\toprule
{} & $Q = \int_D u$ & $Q = u(\bzero)$ & $Q = u((1,1))$ & $Q = \gradu((1,1))$ \\
\midrule
$\alphaz$ &           1.34 &            1.38 &           1.51 &                1.51 \\
$\alphao$ &           0.16 &            0.19 &           0.21 &                0.21 \\
\bottomrule
\end{tabular}

  \caption{The quantities $\alphaz$ and $\alphao$ for different QoIs, where the QMC error $\approx C \NQMC^{-\mleft(\alphaz - \alphao\loge(k)\mright)}$.}\label{tab:qmcalpha}
  \end{table}


\begin{table}[h]
  \centering
  \begin{tabular}{Sc Sc Sc }

\toprule

$k$ & $\NQMC$ & $\NQMCactual$\\
\midrule

10 &    2048 &          2048 \\

20 &    6184 &          8192 \\

30 &   13885 &         16384 \\

40 &   27164 &         32768 \\

50 &   48971 &         65536 \\

60 &   83577 &         65536 \\

\bottomrule

\end{tabular}


  \caption{The ideal and actual number of QMC points $\NQMC$ used in the numerical experiments summarised in \cref{tab:nbpcqmcseq,tab:nbpcqmcpar}, chosen so that the QMC error is empirically bounded for all $k$.\label{tab:qmcpoints}}
  \end{table}

\subsubsection{Numerical results for nearby preconditioning applied to QMC}

Now that we have an estimate of how the number of QMC points should scale with $k$ in order to keep the QMC error bounded, we apply nearby preconditioning to QMC (with the number of points chosen as in \cref{tab:qmcpoints}) and observe how the computational work of this nearby-preconditioning-QMC (NP-QMC) algorithm scales with $k.$

As outlined above, we combine our sequential- and parallel-NPQMC algorithms:\label{page:seqandpar}
\bit
\item We first use the sequential algorithm for low $k$ (fixing the maximum number of GMRES iterations) and observe how the number of preconditioners (as a proportion of the number of QMC points) changes with $k$. We thus obtain an empirical relationship between $k$ and the proportion of QMC points used to construct preconditioners.
  \item We then use the parallel algorithm (with the above proportion of preconditioners) for higher values of $k.$
    \eit
    We remark that, in principle, one could use the sequential algorithm for all values of $k$, however, this would take an incredibly long time--- we see in \cref{tab:qmcpoints} that for $k=60$ we must perform $2^{17}$ Helmholtz solves; if we performed these solves sequentially, and each solve took 10 seconds, this computation would take over 2 weeks to complete.

    The results for the sequential algorithm are summarised in \cref{tab:nbpcqmcseq}, for $k = 10,\,20,\,30$. The results show that nearby preconditioning is effective, with the number of preconditioners growing (approximately) linearly in $k$, but at a very low percentage of the total number of solves. Also, observe than nearby preconditioning is much more effective than mean-based preconditioning, where we use a single preconditioner, corresponding to the mean of $n$, to precondition all the realisations.

    Performing a linear fit for the percentage of LU-factorisations used in the nearby-preconditioning algorithm, we obtain that the percentage of LU-factorisations grows like $-0.04 + 0.02k$ (see \cref{fig:lu}). This result indicates that although the radius of the balls in which nearby preconditioning is effective decreases with $\cO\mleft(1/k\mright)$, the fact that the number of QMC points increases with $k$ means that a large proportion of the solves are computed using a previously-calculated LU decomposition. Observe that if the number of QMC points remained constant in $k$, we would expect the number of preconditioners to (potentially) increase like $k^J$, because the number of balls of radius $\sim 1/k$ in $\cube{J}$ is $\sim k^J.$

    Based on these sequential results, we then used the parallel algorithm with a target proportion of preconditioners of ($-0.04 + 0.02k$)\%. (Although recall from our discussion above that the actual proportion of preconditioners used can vary due to rounding in the algorithm.) The results of these computations are summarised in \cref{tab:nbpcqmcpar}. We observe that the fraction of preconditioners is approximately $-0.04 + 0.02k$, but the maximum (and average) number of GMRES iterations appears to grow slowly with $k.$ This growth (which did not occur with the sequential algorithm) may be because the placement of the preconditioning points is not optimal with respect to the $\dQMC$ metric; we conjecture that oversampling the number of preconditioners needed (for example, taking a proportion of ($0.05k$)\%) may result in a bounded number of GMRES iterations Nevertheless, we see that nearby preconditioning gives considerable speedup, drastically reducing the number of preconditioners that must be calculated.

    \begin{figure}
      \input{lu-graph.pgf}
      \caption[The number of LU factorisations in the sequential nearby-preconditioning-QMC algorithm as a percentage of the total number of solves.]{The number of LU factorisations in the sequential algorithm as a percentage of the total number of solves.\label{fig:lu}}
      \end{figure}
    \afterpage{% Heard about from https://tex.stackexchange.com/questions/11471/how-to-wrap-text-around-landscape-page
\begin{landscape} % Heard about from https://tex.stackexchange.com/questions/19017/how-to-place-a-table-on-a-new-page-with-landscape-orientation-without-clearing-t/19021#19021 and https://tex.stackexchange.com/questions/25369/how-to-rotate-a-table 
    \begin{table}
  \centering
  \begin{tabular}{Sc Sc Sc Sc Sc Sc}
\toprule

$k$ & \# LU factorisations & \makecell{Total \#\\linear systems} & \makecell{\# LU factorisations$/$\\\# linear systems}(\%) & \makecell{Average \#\\GMRES iterations} & \makecell{Max. \#\\GMRES iterations}\\
\midrule

10 &                    5 &                                2048 &                                               0.24 &                                    7.13 &                                   10 \\

20 &                   39 &                                8192 &                                               0.48 &                                    7.26 &                                   10 \\

30 &                  127 &                               16384 &                                               0.78 &                                    7.47 &                                   10 \\

\bottomrule

\end{tabular}


  \caption[Results for the sequential nearby-preconditioning-Quasi-Monte-Carlo algorithm.]{Results applying our sequential nearby-preconditioning-Quasi-Monte-Carlo algorithm with the maximum number of GMRES iterations $=10$, alongside results for mean-based preconditioning.}\label{tab:nbpcqmcseq}
\end{table}

\begin{table}
  \centering
  \begin{tabular}{Sc Sc Sc Sc Sc Sc}
\toprule

$k$ & \# LU factorisations & \makecell{Total \#\\linear systems} & \makecell{\# LU factorisations$/$\\\# linear systems}(\%) & \makecell{Average \#\\GMRES iterations} & \makecell{Max. \#\\GMRES iterations}\\
\midrule

10 &                    4 &                                2048 &                                               0.20 &                                    6.46 &                                   10 \\

30 &                  199 &                              131072 &                                               0.15 &                                    6.48 &                                   12 \\

40 &                  410 &                              262144 &                                               0.16 &                                    6.88 &                                   14 \\

50 &                 1118 &                              524288 &                                               0.21 &                                    6.97 &                                   14 \\

60 &                 1475 &                             1048576 &                                               0.14 &                                    7.35 &                                   16 \\

\bottomrule

\end{tabular}


  \caption{Results applying our parallel nearby-preconditioning-Quasi-Monte-Carlo algorithm with the target proportion of preconditioners as $(-0.04+0.02k)$\%.}\label{tab:nbpcqmcpar}
\end{table}
\end{landscape}
}

    In conclusion, we see that nearby preconditioning gives a significant speedup when applied to a QMC model problem.%, with around 98\% of solves being computed using a previously-calculated LU decomposition.% We therefore expect that this technique will give significant speed up when applied to other, more realistic problems.
    

    
%% \section{Extension of the results to the truncated exterior Dirichlet problem}\label{sec:TEDP}

%% We now briefly outline how the results in \cref{sec:main} above can be extended to \cref{prob:vtedp}, the Truncated Exterior Dirichlet Problem.

%% %\subsection{Definition of the TEDP and analogues of the results in \cref{sec:3}}

%% %% \paragraph{The impedance boundary $\Gamma_I$.} By comparing \cref{eq:src,eq:ibc}, we see that, in the case $g_I=0$, the TEDP approximates the DtN operator $T_R$ by $\ri k$. Indeed, by using Green's first identity and the definition of the normal derivative (see, e.g., \cite[Lemma 4.3]{Mc:00}), show that the boundary condition on $\Gamma_I$ imposed in the variational problem \cref{prob:vtedp} is 
%% %% %In this BVP, the DtN operator $T_R$ Sommerfeld radiation condition 
%% %% \beq\label{eq:imp}
%% %% \dudnu - \ri k\gamma u = g_I \ton \Gamma_I.
%% %% \eeq
%% %% where $\nu$ is the unit outward-pointing normal vector to $\Omega$ on $\Gamma_I$.

%% \paragraph{Existence and uniqueness of a solution to the TEDP.} The sesquilinear form $\aT(\cdot,\cdot)$ defined in \cref{eq:aT} satisfies the G\aa rding inequality \cref{eq:gardingbrief}, and existence and uniqueness of a solution to the TEDP follow under the same condition on $A$ (piecewise-Lipschitz) as for the EDP, as discussed in \cref{thm:tedp}.%sec:vpGm}; in the case of Lipschitz scalar $A$, these unique-continuation arguments are summarised in \cite[\S2]{GrSa:18}.

%% \paragraph{Finite-element/Galerkin solution.}
%% The Galerkin matrix $\Amat$ is defined exactly as in \cref{eq:matrixAdef}, except that 
%% \beq\label{eq:NTEDP}
%% \big(\Nmat\big)_{ij}\de i k\int_{\Gamma_I}  (\gamma\phi_i) \,\gamma \phi_j.
%% \eeq

%% \paragraph{The adjoint sesquilinear form.} For the TEDP, the adjoint sesquilinear form is given by 
%% \beq\label{eq:TEDPadjoint}
%% a^\dagger(u,v) \de \int_{D} 
%% \Big((A \grad u)\cdot\grad \vb
%%  - k^2 n u\vb\Big) +i k\int_{\Gamma_I} \gamma u\, \overline{\gamma v};
%% \eeq
%% then \cref{eq:A*} holds (with $\Nmat$ now given by \cref{eq:NTEDP}), and the analogue of \cref{lem:adjoint} follows in a straightforward way.


%% \paragraph{The analogues of \cref{cond:1nbpc,cond:2}.}
%% The statement of the TEDP analogues of \cref{cond:1nbpc,cond:2} are the same as for the EDP, apart from the following.
%% \ben
%% \item
%% $\supp \,f$ need not be a subset of $\widetilde{\Omega}$ (i.e.~the support of $f$ can go up to the impedance boundary $\Gamma_I$), and
%% \item the assumption $g_I= 0$ needs to be added to \cref{cond:1nbpc} and Part (i) of \cref{cond:2}.
%% \een
%%  Note that, since $\aT(\cdot,\cdot)$ for the TEDP satisfies the same G\aa rding inequality \cref{eq:gardingbrief} as $a(\cdot,\cdot)$ for the EDP, \cref{lem:H1} holds for the TEDP under the TEDP-analogue of \cref{cond:1nbpc}.

%% \paragraph{The main results \cref{thm:1,cor:1}.}
%% Since \cref{cond:1nbpc,cond:2} are essentially unchanged from the EDP case, \cref{lem:keylemma1,lem:keylemma2} hold for the TEDP, and thus so do \cref{thm:1,cor:1,cor:1a}.

%% \paragraph{The PDE results \cref{thm:2} and \cref{lem:sharp}.}

%% The PDE bound \cref{thm:2} relies only on \cref{lem:H1}, which, as stated above, also holds for the TEDP. Therefore \cref{thm:2} holds for the TEDP under the TEDP-analogue of \cref{cond:1nbpc} described above. The construction in \cref{lem:sharp} to show sharpness of the bound in \cref{thm:1} (at least when $\Aso= \Ast= I$) also holds for the TEDP; this is because one can choose the supports of $\chi$ and $\widetilde{\chi}$ to be contained inside $\widetilde{\Omega}$, and then $u^{(1)}$ and $u^{(2)}$ defined in \cref{lem:sharp} satisfy the impedance boundary condition \cref{eq:imp} on $\Gamma_I$.

%% %% \paragraph{When the TEDP-analogue of \cref{cond:1nbpc} holds.}

%% %% In \cref{sec:cond1hold} we discussed 4 situations (Cases 1-4) where \cref{cond:1nbpc} is proved to hold for the EDP. We now discuss the TEDP-analogues of these.
%% %% %Cases 1, 3, and 4 (there is no proof yet for the TEDP-analogue of Case 2).

%% %% \emph{Cases 1 and 2: $\Aso$, $\nso$, and $\Gamma_I$  are $C^\infty$.} 
%% %% With the rays defined as in the EDP case (by the Melrose--Sj{\"o}strand generalized bicharacteristic flow 
%% %% \cite[\S24.3]{Ho:85}), the TEDP-analogue of nontrapping for the EDP is the assumption that 
%% %% every ray eventually hits the boundary at a \emph{non-diffractive point} (defined in \cite[Page 1037]{BaLeRa:92}). Note that, in the case $\Dm=\emptyset$ $\Aso= I$, and $\nso=1$, every ray eventually hits the boundary at a non-diffractive point by \cite[Lemma 5.3]{BaSpWu:16}.
%% %% Under the additional assumption that $\nso= 1$, \cref{cond:1nbpc} follows from the results of \cite{BaLeRa:92} by combining \cite[Theorem 1.8]{BaSpWu:16} and \cite[Remark 5.6]{BaSpWu:16}, but $C^{(1)}_{\rm bound}$ is not given explicitly.

%% %% \emph{Case 3: $\Dm$ is starshaped with respect to the origin, $\Aso$ and $\nso$ are Lipschitz and satisfy radial monotonicity-like conditions.}
%% %% When $\Gamma_I$ is also starshaped with respect to the origin and $A$ and $n$ satisfy \cref{eq:A1nbpc} and \cref{eq:n1nbpc} respectively (with $\Dp$ replaced by $\Omega$), 
%% %% \cite[Theorem A.6(i)]{GrPeSp:19} proves that
%% %% \cref{cond:1nbpc} holds, with an explicit expression for $C^{(1)}_{\rm bound}$. Analogous results when (a) $2\Aso - (\bx\cdot\nabla)\Aso \geq \mu_1$ and $\nso= 1$,
%% %% and  (b) $\Aso= I$ and  $2\nso + \bx \cdot \nabla \nso \geq \mu_2$, 
%% %% are contained in \cite[Theorem A.6(ii)]{GrPeSp:19} and \cite[Theorem A.6(iii)]{GrPeSp:19} respectively.
%% %% When $A$ is scalar, these results were also proved in \cite[Theorem 1]{BrGaPe:17} and, when $\Aso= I$ and $\Dm=\emptyset$, also in \cite[Theorem 3.2]{GrSa:18}.

%% %% \emph{Case 4: %\item[Case 4:]
%% %%  $\Aso$ and $\nso$ are allowed to be discontinuous.}
%% %% %\een
%% %% \cref{cond:1nbpc} is proved in \cite{CaVo:10} (without an explicit expression for $C^{(1)}_{\rm bound}$) when $\Dm$ is $C^\infty$ and nontrapping, $\Gamma_I$ is $C^\infty$, $\Aso= I $, and $\nso$ is a piecewise-constant, monotonically non-decreasing function, jumping on interfaces that are $C^\infty$ with strictly positive curvature.
%% %% Recall from \cref{cond:1nbpc} that \cite[Theorem 2.7]{GrPeSp:19} proves that \cref{cond:1nbpc} holds for the EDP (with an explicit expression for $C^{(1)}_{\rm bound}$) when $\Dm$ is starshaped with respect to the origin, $A$ and $n$ are $L^\infty$, with $A$ monotonically \emph{non-increasing} in the radial direction, and $n$ monotonically \emph{non-decreasing}. This proof can be extended to the TEDP, with the additional assumption that $\Gamma_I$ is star-shaped with respect to the origin; see the discussion in \cite[Section A.2]{GrPeSp:19}.

%% %\cref{cond:1nbpc} is proved, with an explicit expression for $C^{(1)}_{\rm bound}$, when 

%% %\newpage
%% %
%% %\section*{Questions for Th\'eo}
%% %
%% %\ben
%% %\item At the place marked A on the scanned pages, you seem to use the inequality 
%% %\beq\label{eq:Theo1}
%% %\vert\vert\vert \xi - \cP_h \xi\vert\vert\vert \lesssim h^\alpha \N{u_\phi- \cP_h u_\phi}_{0,\Omega}.
%% %\eeq
%% %\een
%% %
%% %\newpag

%% %% \section*{Owen to do list}
%% %% \ben
%% %% \item Varying  $\|\Aso-\Ast\|_{L^\infty}$ and $\|\nso-\nst\|_{L^\infty}$ in standard GMRES.
%% %% \item Computations where $\|\Aso-\Ast\|_{L^\infty}$ and $\|\nso-\nst\|_{L^\infty}$ are sometimes large; is having the standard deviations of these $\sim 1/k$ good enough for $k$-independent GMRES iterations?
%% %% \item ***on backburner*** Checking under what conditions (if any) Part (ii) \cref{cond:2} holds by running the following experiment:
%% %% %\item Exciting experiments for random $n$ that you told us about last week.
%% %% %\item In the weighted norm, the condition on $A$ is ``$k \|\Aso-\Ast\|_{L^\infty}$ sufficiently small" but in the Euclidean norm the best we have so far is ``$h^{-1} \|\Aso-\Ast\|_{L^\infty}$ sufficiently small". You indicated before that experiments seemed to indicate that ``$k \|\Aso-\Ast\|_{L^\infty}$ sufficiently small" seemed correct for the Euclidean norm too. The next time we meet, can you show me these results please?
%% %% %\item Please run the following numerical experiment.
%% %% \bit
%% %% \item TEDP with $\Omega$ a square/rectangle.
%% %% \item $\Aso$ being at least Lipschitz (but smooth is fine). To keep things simple, just take scalar- (as opposed to matrix-) valued $\Aso$ and don't worry about making it nontrapping.
%% %% \item Smoothness of $\nso$ doesn't really matter, just take smooth in the first instance for simplicity (and also don't worry about nontrapping).
%% %% \item $\Vhp$ piecewise linear.
%% %% \item Linear system $\Amato \uvec = \Smat_{A} \balpha$ for some arbitrary complex-valued vector $\balpha$ and some arbitrary $A\in L^\infty$. (I claim this corresponds to the problem described in Part (ii) of \cref{cond:2},  but please check this!)
%% %% \item For each $\Aso, \nso, \balpha$, solve linear system for increasing values of $k$, first with $h\sim k^{-2}$, and then with $h\sim k^{-3/2}$.
%% %% \item Goal: see if the bound \cref{eq:bound4} holds, using 
%% %% \beqs
%% %% \N{\sum_j \alpha_j (A\nabla \phi_j)}_{\LtD} \quad \text{ as a proxy for } \quad \N{\LE}_{(\HokD)'}.
%% %% \eeqs
%% %% \eit
%% %% %\item Varying  $\|\Aso-\Ast\|_{L^\infty}$ and $\|\nso-\nst\|_{L^\infty}$ in \emph{weighted} GMRES.
%% %% \een

\section{Review of related techniques in the literature}\label{sec:nbpclitreview}
   
Having proved rigorous results on the effectiveness of nearby preconditioning, and also applied it to a UQ algorithm, we now review similar computational techniques (applied to other problems) which can be found in the literature. Whilst the idea of \emph{nearby} preconditioning introduced here is, as far as we are aware, novel, there has been a body of work on the closely-related idea of \emph{mean-based} preconditioning. In mean-based preconditioning a \emph{single} preconditioner is calculated corresponding to the mean of the random coefficient. This is in contrast to nearby preconditioning, where \emph{multiple} preconditioners are calculated, corresponding to each realisation in a particular subset of all the realisations. Mean-based preconditioning has been most extensively studied for the stationary diffusion equation
    \beqs
\grad \cdot \mleft(\kappa\grad u\mright)  = -f,
\eeqs
with a small number of works analysing other PDEs, including two works on the Helmholtz equation. We will first explain the idea of mean-based preconditioning before we review the literature applying it to the stationary diffusion equation and other PDEs, and finally turning our attention to mean-based preconditioning for the Helmholtz equation. In general, the computational and mathematical results in the literature show that mean-based preconditioning  is effective if the variance of the random parameters is small enough, i.e.,  if most of the samples are sufficiently close to the mean.

Mean-based preconditioning was first developed for the stationary diffusion equation in the context of so-called Stochastic Spectral Finite-Element Methods (SSFEMs). In these methods, the random field $a$ is given by a series expansion, such as a Karhunen--Lo\`eve expansion, and the dependence of $u$ on the random parameters is computed using a Polynomial Chaos expansion (see, e.g., \cite[Section 2.4.2]{GhSp:12}. The resulting problem is then discretised in the whole space $D \times \Omega$, where $D$ is the spatial domain and $\Omega$ the probability space. The resulting discrete problems involve very large matrices of the form
\beq\label{eq:sgmatrix}
\Amat \otimes \Gmat,
\eeq
where $\Amat$ is a standard finite-element matrix, $\Gmat$ is a matrix corresponding to the discretisation in $\Omega,$ and $\otimes$ is the Kronecker product. For SSFEMs (and the closely-related stochastic-Galerkin FEMs, see, e.g. \cite{BaTeZo:04}, which also have discretisations of the form \cref{eq:sgmatrix}) a mean-based preconditioner is a matrix of the form
\beq\label{eq:mbsg}
\Amatmean \otimes \ImatOmega,
\eeq
where $\Amatmean$ is the standard finite-element matrix corresponding to the mean of $\kappa$ and $\ImatOmega$ is the identity matrix associated with the discretisation on $\Omega.$ Using a mean-based preconditioner of the form \cref{eq:mbsg} gives considerable computational savings, as only one preconditioner of a standard finite-element matrix needs to be calculated.

When stochastic Galerkin methods are used with so-called `doubly-orthogonal bases' (see, e.g., \cite[Section 3.2]{ErPoSiUl:09}), then the linear system \cref{eq:sgmatrix} decouples into many distinct standard finite-element matrices; mean-based preconditioning has also been investigated in this context (and in the context of stochastic collocation methods (see, e.g., \cite{BaNoTe:07}), where one similarly obtains many different standard finite-element matrices) as will be discussed below.

The main insight gleaned from studies of mean-based preconditioning is that, as stated above, if the variance of $\kappa$ (or any other stochastic coefficients) is sufficiently small, then mean-based preconditioning is effective.

The initial computational work on mean-based preconditioning for the stationary diffusion equation was carried out by Ghanem and Kruger \cite{GhKr:96}, Pellissetti and Ghanem \cite{PeGh:00}, and Keese \cite{Ke:04}, with theory (proving bounds on the eigenvalues of the preconditioned matrices) following from Powell and Elman \cite{PoEl:09} and Ernst, Powell, Silvester, and Ullmann \cite{ErPoSiUl:09}. These eigenvalue bounds are analagous to results in \cref{sec:main} above, as they allow one to infer convergence properties of the iterative method used. All of the above results were for $\kappa$ given by a (real or artificial) Karhunen--Lo\`eve expansion; that is, in the case where $\kappa$ depends linearly on the random parameter. In the case where $\kappa$ is a lognormal random field (and so the dependence is no longer linear), Powell and Ullmann \cite{PoUl:10} declared mean-based preconditioners to be ineffective, and so developed more advanced preconditioners; in contrast, Ullmann, Elman and Ernst \cite{UlElEr:12} transformed a stationary diffusion problem with lognormal coefficient into a stationary convection-diffusion problem with a random coefficient depending linearly on the noise, before proving eigenvalue bounds as before. With a more computational slant, Tipireddy, Phipps, and Ghanem \cite{TiPhGh:10} and Rosseel and Vandewalle \cite{RoVa:10} compared the computational properties of several mean-based preconditioners and Elman, Miller, Phipps, and Tuminaro \cite{ElMiPhTu:11} compared the computational cost of mean-based preconditioners for stochastic Galerkin and stochastic collocation methods.

Seeking to apply mean-based preconditioning to more challenging problems, Powell and Silvester \cite{PoSi:12} performed computational investigations for mean-based preconditioners applied to stochastic Galerkin discretisations of the steady-state Navier--Stokes equations, and Soused\'ik and Elman \cite{SoEl:16} introduced a Gauss--Seidel-type preconditioner, using mean-based ideas, for the steady-state Navier--Stokes equations. Finally, Khan, Powell, and Silvester \cite{KhPoSi:19} applied mean-based preconditioning to stochastic Galerkin discretisations of the equations for nearly-incompressible elasticity.

The works applying mean-based preconditioning to many individual systems (for the stationary diffusion equation) are those of Eiermann, Ernst and Ullmann \cite{EiErUl:07}; Ernst, Powell, Silvester, and Ullmann \cite{ErPoSiUl:09}; and Gordon and Powell \cite{GoPo:12}. \cite{EiErUl:07} contained computational results in a (decoupled) stochastic Galerkin setting; \cite{ErPoSiUl:09} proved eigenvalue bounds in the same setting, and \cite{GoPo:12} proved rigorous eigenvalue bounds in a stochastic collocation setting. All these works assume linear dependence on the noise, and show that mean-based preconditioning works well when the variance is sufficiently small.

We now turn our attention to mean-based preconditioning for the Helmholtz equation. The first work we discuss is the recent work of Wang and Liao \cite{WaLi:19}. They discretise a stochastic Helmholtz problem with $k=10$ and $n$ given by a truncated Karhunen--Lo\`eve expansion (with either 4 terms or 1 term) and use a generalised polynomial chaos (gPC) expansion (see, e.g., \cite{XiKa:02}) for the solution $u$. Whilst they use mean-based preconditioning (in the `Kronecker product' sense) they are more interested in investigating the effect of the number of terms in the gPC expansion on the accuracy of the discrete solution. Nonetheless, they see convergence using the mean-based preconditioner, although more iterations are needed when the random field is `close to' exciting a resonant frequency (see \cite[Example 4.2]{WaLi:19}).

The work most similar to ours is the work of Jin and Cai \cite{JiCa:09}, who use a stochastic Galerkin discretisation with a doubly-orthogonal basis for a stochastic Helmholtz equation, resulting in around 5000 linear systems. They take $k = 225$ and a Karhunen--Lo\`eve expansion with 4 terms for both (scalar-valued) $A$ and $n$. The random variables in the Karhunen--Lo\`eve expansions are $\Unif(-\sqrt{3},\sqrt{3})$ and $\Unif(-45\sqrt{3},45\sqrt{3})$ for $A$ and $n$ respectively. Their mean-based preconditioner is a 1-level additive Schwarz preconditioner, and they compare resuing the preconditioner with reusing the Krylov subspaces (an idea first introduced by Parks, De Sturler, Mackey, Johnson, and Maiti in \cite{PadeMaJoMa:06}), as well as combining both techniques. Intriguingly, they see no additional benefit from reusing the preconditioner, but considerable benefit from recycling the Krylov subspaces. Based on our results in this \lcnamecref{chap:nbpc}, we conjecture that they see no benefit from a single mean-based preconditioner because $k$ is reasonably large, and therefore for most of the realisations, $k\NLiDRR{\EXP{n}-\nsj}$ and $k\NLiDRRdtd{\EXP{A}-\Asj}$ are not sufficiently small, and so there is little-to-no effect on the number of GMRES iterations from mean-based preconditioning. We conjecture that if they had used multiple preconditioners distributed around the stochastic parameter space, they would have seen computational improvements, as described in this \lcnamecref{chap:nbpc}.
    
\section{Probabilistic nearby preconditioning results}\label{sec:nbpcstochastic}

We now briefly overview how one can prove probabilistic results on the effectiveness of nearby preconditioning. All of the results in \cref{sec:intronbpc,sec:num,sec:3,sec:weaknorm} above have been for deterministic (as opposed to stochastic) coefficients $A$ and $n$ (and we then applied these deterministic results to QMC methods for the Helmholtz equation in \cref{sec:nbpcqmc}). Therefore we now turn our attention to obtaining probabilistic results on the effectiveness of nearby preconditioning for stochastic Helmholtz problems, i.e., \cref{prob:msedp,prob:somsedp,prob:svsedp} from \cref{chap:stochastic}. Firstly, in \cref{cor:stonbpcas} below, we prove an `essentially deterministic' result on the effectiveness of nearby preconditioning, before proving probabilistic results on the effectiveness of nearby preconditioning applied to stochastic problems. However, we will see that our efforts to prove probabilistic results are restricted by the applicability of the Elman estimate (\cref{thm:GMRES1_intro} above).

Throughout this \lcnamecref{sec:nbpcstochastic} we consider \cref{prob:msedp} from \cref{chap:stochastic} but with $A=I$, i.e., for simplicity we only consider the case of random $n$, although everything we say could be easily extended to include random $A$. To maintain consistent notation with the rest of this \lcnamecref{chap:nbpc} we will use a superscript ${}^{(2)}$ to refer to the stochastic problem (e.g., the random coefficient will be $\nst(\omega)$, the solution will be $\ust(\omega)$, the matrices arising from the finite-element discretiation will be $\Amatt(\omega),$ etc.). We let $\nso \in \LiDRR$ define a \emph{deterministic} Helmholtz problem. We will use the discretisation of this deterministic Helmholtz problem to precondition the discretisations of the realisations of the stochastic Helmholtz problem. I.e., we will consider the performance of GMRES applied to
\beq\label{eq:stopc}
\AmatoI\Amatt(\omega)\uvec = \AmatoI \fvec.
\eeq
For simplicity, in all that follows we will measure $\no-\nt$ in the $L^{\infty}$ norm, although one could use any of the weaker norms discussed in \cref{sec:weaknorm} above, and obtain analogous results.

\subsection{Probabilistic theory for nearby preconditioning}
\bde[Number of GMRES iterations required for convergence]\label{def:numGMRESitsconv}

\

\noindent Let $\GMRES{\eps}{\nso}{\nst}$ denote the number of iterations required for GMRES in the unweighted norm $\Nt{\cdot}$ with $\Nt{\rvecz} = 1,$ applied to
\beqs
\AmatoI\Amatt  \uvec = \AmatoI \fvec
\eeqs
to converge to within a tolerance $\eps,$ i.e., to achieve
\beqs
\frac{\Nt{\rvecm}}{\Nt{\fvec}} < \eps.
\eeqs
\ede

Note that $\GMRES{\eps}{\nso}{\nst}$ is a random variable, see \cref{lem:randomvariable} below.

If we apply \cref{cor:1a} to the problem \cref{eq:stopc} we can straightforwardly conclude the following \lcnamecref{cor:stonbpcas}.

\bco[Almost-sure nearby preconditioning]\label{cor:stonbpcas}
Let $0 < \eps < 1,$ $\nst:\Omega \rightarrow \LiDRR$ satisfy the assumptions at the start of \cref{sec:hh-results}, $\no, \,\Dm,$ and $f$ be as in \cref{prob:vgen}, and let the assumptions of \cref{cor:1a} hold. Then $\GMRES{\eps}{\nso}{\nst}$ is bounded independently of $k$ almost surely if
\beq\label{eq:nbpcas}
\NLiDRR{\nso-\nst(\omega)} \leq \frac1{2\Ct k}
\eeq
almost surely.
\eco

%Observe that implicit in \cref{cor:stonbpcas} is the fact that $\GMRES{\eps}{\nso}{\nst}$ is a random variable; we sketch a proof of this fact now.

\ble[$\GMRES{\eps}{\no}{\nt}$ is a random variable]\label{lem:randomvariable}
Under the assumptions of \cref{cor:stonbpcas}, $\GMRES{\eps}{\nso}{\nst}$ is a random variable, i.e., $\GMRES{\eps}{\nso}{\nst}:\Omega\rightarrow \RR$ is measurable.
\ele

\bpf[Sketch Proof of \cref{lem:randomvariable}]
All of the operations used in constructing the vectors $\xvecm$ in the GMRES algorithm are measurable functions of $\xvecmmo$ and $\AmatoI\Amatt$ (see, e.g., \cite[Algorithms 11.4.2 and 5.1.3]{GoVa:13}), therefore $\mleft(\rvecm\mright)_{m=1}^N$ is a sequence of random variables, i.e., a stochastic process (see, e.g., \cite[Definition 2.1.4]{Ok:13}). The stopping criterion $\Nt{\rvecm}/\Nt{\fvec} < \eps$ is an exit time for the stochastic process $\xvecm$ from the set $\CCN \setminus \ball{\CCN}{\xvecs}{\eps\Nt{\fvec}},$ where $\xvecs$ is the true solution. Therefore, because we assume $\OFP$ is a complete probability space, it follows from, e.g.,  \cite[Example 7.2.2]{Ok:13} that $\GMRES{\eps}{\no}{\nt}$ is a stopping time (see \cite[Definition 7.2.1]{Ok:13}). Because $\GMRES{\eps}{\no}{\nt}$ is a stopping time, it is measurable with respect to the associated filtration (see, e.g., \cite[Definition 3.2.2]{Ok:13}), and so is measurable with respect to $\cF$; i.e., $\GMRES{\eps}{\no}{\nt}$ is a random variable.
\epf

The numerical results in \cref{sec:num} above can be seen (in part) as confirming \cref{cor:stonbpcas}. Recall that in \cref{sec:num} we let $\nso-\nst$ be a piecewise-constant random field, and we fixed $\alpha = \NLiDRR{\nso-\nst}$ or $\NLiDRRdtd{\Aso-\Ast}$ almost surely. When we fixed $\alpha = 0.5/k$ almost surely (see \cref{fig:linfinityA2,fig:linfinityn2}) we saw that the number of GMRES iterations was bounded independently of $k.$ This behaviour is precisely that given in \cref{cor:stonbpcas}.

\bre[Drawbacks of \cref{cor:stonbpcas}]\label{rem:notideal}
There are two drawbacks of \cref{cor:stonbpcas}:
\ben
\item\label[itemdrawback]{it:notideal1} The condition \cref{eq:nbpcas} must hold almost surely, and
  \item\label[itemdrawback]{it:notideal2} \Cref{cor:stonbpcas} does not give any explicit information on how the distribution of the number of GMRES iterations depends on the distribution of $\NLiDRR{\nso-\nst}.$
    \een
    \Cref{it:notideal1} is not ideal because in many physically realistic problems $\NLiDRR{\no-\nt(\omega)}$ may be unbounded (e.g., if $\nt$ is a lognormal random field) or even if bounded may not satisfy the condition \cref{eq:nbpcas} almost surely.% \Cref{it:notideal2} is not ideal because it means one cannot infer information about the distribution of the number of GMRES iterations from the distribution of$\NLiDRR{\nso-\nst(\omega)}.$
    \ere

    To correct the deficiencies described in \cref{rem:notideal} one would aim to  prove a bound on the number of GMRES iterations depending explicitly on $\NLiDRR{\nso-\nst(\omega)}$, and then use this bound to prove a probabilistic estimate for the number of GMRES iterations. Such a bound is given in \cref{lem:probgmres1} in \cref{app:probnbpc}. However, such a bound will be highly pessimistic, and will impart little useful information. The reason for this lack of information is that the Elman estimate (\cref{cor:GMRES_intro} above) when applied to the nearby-preconditioned system $\AmatoI\Amatt$) only applies when $k\NLiDop{\Aso-\Ast}$ and $k\NLiDRR{\nso-\nst}$ are sufficiently small (as we saw in \cref{cor:1} above). Therefore one can only obtain detailed information on how the number of GMRES iterations depends on $\NLiDop{\Aso-\Ast}$ and $\NLiDRR{\nso-\nst}$ when these quantities are small (informally, when they are $\lesssim 1/k$). In all other cases (again, informally, when these quantities are $\gtrsim 1/k$) the only statement one can make about the convergence of GMRES is that there will be at most $N$ iterations, where $N$ is the number of degrees of freedom (this result is recalled in \cref{cor:gmresguaranteed} below). In summary, current results on GMRES convergence will only allow us to prove what are likely to be very pessimistic bounds on how the number of GMRES iterations for $\AmatoI\Amatt$ depends on $\NLiDop{\Aso-\Ast}$ and $\NLiDRR{\nso-\nst}$. For completeness, we record these results in \cref{app:probnbpc}.






%%  TO HERE BRO

%%     We first define notation for the number of GMRES iterations required for convergence. Because \cref{lem:probgmres1} below is a \emph{deterministic} result, i.e., it does not require $\nst$ to be a random field. Therefore for this \lcnamecref{lem:probgmres1} only, we assume $\nst$ is as given at the beginning of this \lcnamecref{chap:nbpc}.


    


%% \bre[\Cref{thm:probgmres} is pessimistic]\label{rem:pessimistic}
%% Observe that we expect the bound in \cref{thm:probgmres} to be pessimistic, i.e., we expect that \cref{eq:GMRESprob} is not sharp in its dependence on $\alpha$. In particular, we expect \cref{eq:GMRESprob} is not sharp for large values of $R.$ We now show that in most cases, for large values of $R$ the left-hand side of \cref{eq:GMRESprob} is independent of $R$.

%% One can show via elementary calculus that for $\alpha < 1$ $\Gfnname$ achieves its maximum when $\alpha = 1/3$ (assuming that for $\alpha < 1$ the expression involving $\alpha$ in \cref{eq:gdef} is always  at most $N$). Also observe that $\Gfnname$ over the range $\alpha  \in (0,1)$ only depends on $k$ through the dependence of $\alpha$ on $k.$ Therefore, the maximum of $\Gfnname$ over $\alpha \in (0,1)$ is independent of $k.$ Let $\Gfnmaxlo$ denote the value of this maximum. Then, for any $R \in \mleft(\Gfnmaxlo,N\mright)$, the estimate $\PP\mleft(\Gfn{\nso-\nst} \leq R\mright)$ is equal to $\PP\mleft(\alpha<1\mright) = \PP\mleft(\NLiDRR{\nso-\nst} < 1/\mleft(\Ct k\mright)\mright),$ i.e., the lower-bound in \cref{eq:GMRESprob} is \emph{independent} of $R$, for $R \in \mleft(\Gfnmaxlo,N\mright)$  This is almost certainly not sharp - we would expect $\PP\mleft(\GMRES{\eps}{\nso}{\nst} \leq R\mright)$ to increase with $R$. However, because the only rigorous result we have available if $\alpha \geq 1$ is \cref{cor:gmresguaranteed}, we cannot prove a better bound.
%% \ere

\subsection{Numerical probabalistic results for nearby preconditioning}\label{sec:qualgmres}

Notwithstanding the fact that we are limited in the probabilistic results that we can \emph{prove} about nearby preconditioning, we will now see that we \emph{observe} reasonable probabilistic behaviour when we perform numerical experiments. We again recall that (informally) \cref{cor:stonbpcas} states that we obtain almost-surely bounded GMRES iterations if $\NLiDRR{\nso-\nst} \lesssim 1/k.$ A plausible probabalistic analogue of this result would be that we have bounded \emph{average} number of GMRES iterations if the standard deviation of $\NLiDRR{\nso-\nst}$ is of the order $1/k$. We expect this result because the standard deviation of a random variable is a (probabilistic) measure of its variation. In \cref{cor:stonbpcas} we show that the number of GMRES iterations is bounded almost surely if the variation in $\NLiDRR{\nso-\nst}$ is bounded (of the order $1/k$) almost surely. Therefore, it reasonable to assume that the probabilistic analogue of the number of GMRES iterations (the average) is bounded if the probabilistic analogue of the variation in $\NLiDRR{\nso-\nst}$ (the standard deviation) is bounded (of the order $1/k$). We will see exactly this behaviour in our numerical experiments.

In our numerical experiments we use the computational setup described in \cref{app:compsetup}, with $f=1$ and $\gI=0,$ $\Aso=\Ast=I$, $\nso=1,$ and $\NLiDRR{\nso-\nst}$ given by an exponential random variables with standard deviation $\sigma.$  We consider three cases:
\ben
\item\label[itemcase]{it:sigma1} $\displaystyle \sigma  = 1,$
\item\label[itemcase]{it:sigma2} $\displaystyle \sigma  = \frac{1}k,$ and
  \item\label[itemcase]{it:sigma3} $\displaystyle \sigma  = \frac{1}{k^2}$.
    \een

    For each of these cases we calculate
    \beq\label{eq:gmresprob}
    \PP\mleft(\GMRES{\eps}{\no}{\nt} \leq 12\mright).
    \eeq

    Based on the reasoning above we expect that in \cref{it:sigma2} the probability \cref{eq:gmresprob} \emph{is constant} as $k$ increases, and using similar reasoning, we expect that in \cref{it:sigma1} the probability \cref{eq:gmresprob} \emph{decreases} as $k$ increases and in \cref{it:sigma3} the probability \cref{eq:gmresprob} \emph{increases} as $k$ increases.   This is approximately the behaviour we observe in \cref{fig:prob-plot-0.0,fig:prob-plot-1.0,fig:prob-plot-2.0}. This behaviour demonstrates that whilst the theory developed in the rest of this \lcnamecref{chap:nbpc} does not allow us to easily prove useful results about the probabalistic behaviour of nearby preconditioning, the theory does give us \emph{intuition} as to what the probabilistic behavour will be.

%% \begin{figure}[p]
%%   \centering
%%   \begin{subfigure}{\textwidth}
%%     \centering
%% %% Creator: Matplotlib, PGF backend
%%
%% To include the figure in your LaTeX document, write
%%   \input{<filename>.pgf}
%%
%% Make sure the required packages are loaded in your preamble
%%   \usepackage{pgf}
%%
%% Figures using additional raster images can only be included by \input if
%% they are in the same directory as the main LaTeX file. For loading figures
%% from other directories you can use the `import` package
%%   \usepackage{import}
%% and then include the figures with
%%   \import{<path to file>}{<filename>.pgf}
%%
%% Matplotlib used the following preamble
%%   \usepackage{fontspec}
%%   \setmainfont{DejaVuSerif.ttf}[Path=/home/owen/progs/firedrake-complex/firedrake/lib/python3.5/site-packages/matplotlib/mpl-data/fonts/ttf/]
%%   \setsansfont{DejaVuSans.ttf}[Path=/home/owen/progs/firedrake-complex/firedrake/lib/python3.5/site-packages/matplotlib/mpl-data/fonts/ttf/]
%%   \setmonofont{DejaVuSansMono.ttf}[Path=/home/owen/progs/firedrake-complex/firedrake/lib/python3.5/site-packages/matplotlib/mpl-data/fonts/ttf/]
%%
\begingroup%
\makeatletter%
\begin{pgfpicture}%
\pgfpathrectangle{\pgfpointorigin}{\pgfqpoint{6.000000in}{2.500000in}}%
\pgfusepath{use as bounding box, clip}%
\begin{pgfscope}%
\pgfsetbuttcap%
\pgfsetmiterjoin%
\definecolor{currentfill}{rgb}{1.000000,1.000000,1.000000}%
\pgfsetfillcolor{currentfill}%
\pgfsetlinewidth{0.000000pt}%
\definecolor{currentstroke}{rgb}{1.000000,1.000000,1.000000}%
\pgfsetstrokecolor{currentstroke}%
\pgfsetdash{}{0pt}%
\pgfpathmoveto{\pgfqpoint{0.000000in}{0.000000in}}%
\pgfpathlineto{\pgfqpoint{6.000000in}{0.000000in}}%
\pgfpathlineto{\pgfqpoint{6.000000in}{2.500000in}}%
\pgfpathlineto{\pgfqpoint{0.000000in}{2.500000in}}%
\pgfpathclose%
\pgfusepath{fill}%
\end{pgfscope}%
\begin{pgfscope}%
\pgfsetbuttcap%
\pgfsetmiterjoin%
\definecolor{currentfill}{rgb}{1.000000,1.000000,1.000000}%
\pgfsetfillcolor{currentfill}%
\pgfsetlinewidth{0.000000pt}%
\definecolor{currentstroke}{rgb}{0.000000,0.000000,0.000000}%
\pgfsetstrokecolor{currentstroke}%
\pgfsetstrokeopacity{0.000000}%
\pgfsetdash{}{0pt}%
\pgfpathmoveto{\pgfqpoint{0.750000in}{0.275000in}}%
\pgfpathlineto{\pgfqpoint{5.400000in}{0.275000in}}%
\pgfpathlineto{\pgfqpoint{5.400000in}{2.200000in}}%
\pgfpathlineto{\pgfqpoint{0.750000in}{2.200000in}}%
\pgfpathclose%
\pgfusepath{fill}%
\end{pgfscope}%
\begin{pgfscope}%
\pgfsetbuttcap%
\pgfsetroundjoin%
\definecolor{currentfill}{rgb}{0.000000,0.000000,0.000000}%
\pgfsetfillcolor{currentfill}%
\pgfsetlinewidth{0.803000pt}%
\definecolor{currentstroke}{rgb}{0.000000,0.000000,0.000000}%
\pgfsetstrokecolor{currentstroke}%
\pgfsetdash{}{0pt}%
\pgfsys@defobject{currentmarker}{\pgfqpoint{0.000000in}{-0.048611in}}{\pgfqpoint{0.000000in}{0.000000in}}{%
\pgfpathmoveto{\pgfqpoint{0.000000in}{0.000000in}}%
\pgfpathlineto{\pgfqpoint{0.000000in}{-0.048611in}}%
\pgfusepath{stroke,fill}%
}%
\begin{pgfscope}%
\pgfsys@transformshift{0.961364in}{0.275000in}%
\pgfsys@useobject{currentmarker}{}%
\end{pgfscope}%
\end{pgfscope}%
\begin{pgfscope}%
\definecolor{textcolor}{rgb}{0.000000,0.000000,0.000000}%
\pgfsetstrokecolor{textcolor}%
\pgfsetfillcolor{textcolor}%
\pgftext[x=0.961364in,y=0.177778in,,top]{\color{textcolor}\sffamily\fontsize{10.000000}{12.000000}\selectfont \(\displaystyle 10\)}%
\end{pgfscope}%
\begin{pgfscope}%
\pgfsetbuttcap%
\pgfsetroundjoin%
\definecolor{currentfill}{rgb}{0.000000,0.000000,0.000000}%
\pgfsetfillcolor{currentfill}%
\pgfsetlinewidth{0.803000pt}%
\definecolor{currentstroke}{rgb}{0.000000,0.000000,0.000000}%
\pgfsetstrokecolor{currentstroke}%
\pgfsetdash{}{0pt}%
\pgfsys@defobject{currentmarker}{\pgfqpoint{0.000000in}{-0.048611in}}{\pgfqpoint{0.000000in}{0.000000in}}{%
\pgfpathmoveto{\pgfqpoint{0.000000in}{0.000000in}}%
\pgfpathlineto{\pgfqpoint{0.000000in}{-0.048611in}}%
\pgfusepath{stroke,fill}%
}%
\begin{pgfscope}%
\pgfsys@transformshift{1.665909in}{0.275000in}%
\pgfsys@useobject{currentmarker}{}%
\end{pgfscope}%
\end{pgfscope}%
\begin{pgfscope}%
\definecolor{textcolor}{rgb}{0.000000,0.000000,0.000000}%
\pgfsetstrokecolor{textcolor}%
\pgfsetfillcolor{textcolor}%
\pgftext[x=1.665909in,y=0.177778in,,top]{\color{textcolor}\sffamily\fontsize{10.000000}{12.000000}\selectfont \(\displaystyle 15\)}%
\end{pgfscope}%
\begin{pgfscope}%
\pgfsetbuttcap%
\pgfsetroundjoin%
\definecolor{currentfill}{rgb}{0.000000,0.000000,0.000000}%
\pgfsetfillcolor{currentfill}%
\pgfsetlinewidth{0.803000pt}%
\definecolor{currentstroke}{rgb}{0.000000,0.000000,0.000000}%
\pgfsetstrokecolor{currentstroke}%
\pgfsetdash{}{0pt}%
\pgfsys@defobject{currentmarker}{\pgfqpoint{0.000000in}{-0.048611in}}{\pgfqpoint{0.000000in}{0.000000in}}{%
\pgfpathmoveto{\pgfqpoint{0.000000in}{0.000000in}}%
\pgfpathlineto{\pgfqpoint{0.000000in}{-0.048611in}}%
\pgfusepath{stroke,fill}%
}%
\begin{pgfscope}%
\pgfsys@transformshift{2.370455in}{0.275000in}%
\pgfsys@useobject{currentmarker}{}%
\end{pgfscope}%
\end{pgfscope}%
\begin{pgfscope}%
\definecolor{textcolor}{rgb}{0.000000,0.000000,0.000000}%
\pgfsetstrokecolor{textcolor}%
\pgfsetfillcolor{textcolor}%
\pgftext[x=2.370455in,y=0.177778in,,top]{\color{textcolor}\sffamily\fontsize{10.000000}{12.000000}\selectfont \(\displaystyle 20\)}%
\end{pgfscope}%
\begin{pgfscope}%
\pgfsetbuttcap%
\pgfsetroundjoin%
\definecolor{currentfill}{rgb}{0.000000,0.000000,0.000000}%
\pgfsetfillcolor{currentfill}%
\pgfsetlinewidth{0.803000pt}%
\definecolor{currentstroke}{rgb}{0.000000,0.000000,0.000000}%
\pgfsetstrokecolor{currentstroke}%
\pgfsetdash{}{0pt}%
\pgfsys@defobject{currentmarker}{\pgfqpoint{0.000000in}{-0.048611in}}{\pgfqpoint{0.000000in}{0.000000in}}{%
\pgfpathmoveto{\pgfqpoint{0.000000in}{0.000000in}}%
\pgfpathlineto{\pgfqpoint{0.000000in}{-0.048611in}}%
\pgfusepath{stroke,fill}%
}%
\begin{pgfscope}%
\pgfsys@transformshift{3.075000in}{0.275000in}%
\pgfsys@useobject{currentmarker}{}%
\end{pgfscope}%
\end{pgfscope}%
\begin{pgfscope}%
\definecolor{textcolor}{rgb}{0.000000,0.000000,0.000000}%
\pgfsetstrokecolor{textcolor}%
\pgfsetfillcolor{textcolor}%
\pgftext[x=3.075000in,y=0.177778in,,top]{\color{textcolor}\sffamily\fontsize{10.000000}{12.000000}\selectfont \(\displaystyle 25\)}%
\end{pgfscope}%
\begin{pgfscope}%
\pgfsetbuttcap%
\pgfsetroundjoin%
\definecolor{currentfill}{rgb}{0.000000,0.000000,0.000000}%
\pgfsetfillcolor{currentfill}%
\pgfsetlinewidth{0.803000pt}%
\definecolor{currentstroke}{rgb}{0.000000,0.000000,0.000000}%
\pgfsetstrokecolor{currentstroke}%
\pgfsetdash{}{0pt}%
\pgfsys@defobject{currentmarker}{\pgfqpoint{0.000000in}{-0.048611in}}{\pgfqpoint{0.000000in}{0.000000in}}{%
\pgfpathmoveto{\pgfqpoint{0.000000in}{0.000000in}}%
\pgfpathlineto{\pgfqpoint{0.000000in}{-0.048611in}}%
\pgfusepath{stroke,fill}%
}%
\begin{pgfscope}%
\pgfsys@transformshift{3.779545in}{0.275000in}%
\pgfsys@useobject{currentmarker}{}%
\end{pgfscope}%
\end{pgfscope}%
\begin{pgfscope}%
\definecolor{textcolor}{rgb}{0.000000,0.000000,0.000000}%
\pgfsetstrokecolor{textcolor}%
\pgfsetfillcolor{textcolor}%
\pgftext[x=3.779545in,y=0.177778in,,top]{\color{textcolor}\sffamily\fontsize{10.000000}{12.000000}\selectfont \(\displaystyle 30\)}%
\end{pgfscope}%
\begin{pgfscope}%
\pgfsetbuttcap%
\pgfsetroundjoin%
\definecolor{currentfill}{rgb}{0.000000,0.000000,0.000000}%
\pgfsetfillcolor{currentfill}%
\pgfsetlinewidth{0.803000pt}%
\definecolor{currentstroke}{rgb}{0.000000,0.000000,0.000000}%
\pgfsetstrokecolor{currentstroke}%
\pgfsetdash{}{0pt}%
\pgfsys@defobject{currentmarker}{\pgfqpoint{0.000000in}{-0.048611in}}{\pgfqpoint{0.000000in}{0.000000in}}{%
\pgfpathmoveto{\pgfqpoint{0.000000in}{0.000000in}}%
\pgfpathlineto{\pgfqpoint{0.000000in}{-0.048611in}}%
\pgfusepath{stroke,fill}%
}%
\begin{pgfscope}%
\pgfsys@transformshift{4.484091in}{0.275000in}%
\pgfsys@useobject{currentmarker}{}%
\end{pgfscope}%
\end{pgfscope}%
\begin{pgfscope}%
\definecolor{textcolor}{rgb}{0.000000,0.000000,0.000000}%
\pgfsetstrokecolor{textcolor}%
\pgfsetfillcolor{textcolor}%
\pgftext[x=4.484091in,y=0.177778in,,top]{\color{textcolor}\sffamily\fontsize{10.000000}{12.000000}\selectfont \(\displaystyle 35\)}%
\end{pgfscope}%
\begin{pgfscope}%
\pgfsetbuttcap%
\pgfsetroundjoin%
\definecolor{currentfill}{rgb}{0.000000,0.000000,0.000000}%
\pgfsetfillcolor{currentfill}%
\pgfsetlinewidth{0.803000pt}%
\definecolor{currentstroke}{rgb}{0.000000,0.000000,0.000000}%
\pgfsetstrokecolor{currentstroke}%
\pgfsetdash{}{0pt}%
\pgfsys@defobject{currentmarker}{\pgfqpoint{0.000000in}{-0.048611in}}{\pgfqpoint{0.000000in}{0.000000in}}{%
\pgfpathmoveto{\pgfqpoint{0.000000in}{0.000000in}}%
\pgfpathlineto{\pgfqpoint{0.000000in}{-0.048611in}}%
\pgfusepath{stroke,fill}%
}%
\begin{pgfscope}%
\pgfsys@transformshift{5.188636in}{0.275000in}%
\pgfsys@useobject{currentmarker}{}%
\end{pgfscope}%
\end{pgfscope}%
\begin{pgfscope}%
\definecolor{textcolor}{rgb}{0.000000,0.000000,0.000000}%
\pgfsetstrokecolor{textcolor}%
\pgfsetfillcolor{textcolor}%
\pgftext[x=5.188636in,y=0.177778in,,top]{\color{textcolor}\sffamily\fontsize{10.000000}{12.000000}\selectfont \(\displaystyle 40\)}%
\end{pgfscope}%
\begin{pgfscope}%
\definecolor{textcolor}{rgb}{0.000000,0.000000,0.000000}%
\pgfsetstrokecolor{textcolor}%
\pgfsetfillcolor{textcolor}%
\pgftext[x=3.075000in,y=-0.012191in,,top]{\color{textcolor}\sffamily\fontsize{10.000000}{12.000000}\selectfont \(\displaystyle k\)}%
\end{pgfscope}%
\begin{pgfscope}%
\pgfsetbuttcap%
\pgfsetroundjoin%
\definecolor{currentfill}{rgb}{0.000000,0.000000,0.000000}%
\pgfsetfillcolor{currentfill}%
\pgfsetlinewidth{0.803000pt}%
\definecolor{currentstroke}{rgb}{0.000000,0.000000,0.000000}%
\pgfsetstrokecolor{currentstroke}%
\pgfsetdash{}{0pt}%
\pgfsys@defobject{currentmarker}{\pgfqpoint{-0.048611in}{0.000000in}}{\pgfqpoint{0.000000in}{0.000000in}}{%
\pgfpathmoveto{\pgfqpoint{0.000000in}{0.000000in}}%
\pgfpathlineto{\pgfqpoint{-0.048611in}{0.000000in}}%
\pgfusepath{stroke,fill}%
}%
\begin{pgfscope}%
\pgfsys@transformshift{0.750000in}{0.442128in}%
\pgfsys@useobject{currentmarker}{}%
\end{pgfscope}%
\end{pgfscope}%
\begin{pgfscope}%
\definecolor{textcolor}{rgb}{0.000000,0.000000,0.000000}%
\pgfsetstrokecolor{textcolor}%
\pgfsetfillcolor{textcolor}%
\pgftext[x=0.405863in,y=0.389366in,left,base]{\color{textcolor}\sffamily\fontsize{10.000000}{12.000000}\selectfont \(\displaystyle 0.01\)}%
\end{pgfscope}%
\begin{pgfscope}%
\pgfsetbuttcap%
\pgfsetroundjoin%
\definecolor{currentfill}{rgb}{0.000000,0.000000,0.000000}%
\pgfsetfillcolor{currentfill}%
\pgfsetlinewidth{0.803000pt}%
\definecolor{currentstroke}{rgb}{0.000000,0.000000,0.000000}%
\pgfsetstrokecolor{currentstroke}%
\pgfsetdash{}{0pt}%
\pgfsys@defobject{currentmarker}{\pgfqpoint{-0.048611in}{0.000000in}}{\pgfqpoint{0.000000in}{0.000000in}}{%
\pgfpathmoveto{\pgfqpoint{0.000000in}{0.000000in}}%
\pgfpathlineto{\pgfqpoint{-0.048611in}{0.000000in}}%
\pgfusepath{stroke,fill}%
}%
\begin{pgfscope}%
\pgfsys@transformshift{0.750000in}{1.115498in}%
\pgfsys@useobject{currentmarker}{}%
\end{pgfscope}%
\end{pgfscope}%
\begin{pgfscope}%
\definecolor{textcolor}{rgb}{0.000000,0.000000,0.000000}%
\pgfsetstrokecolor{textcolor}%
\pgfsetfillcolor{textcolor}%
\pgftext[x=0.405863in,y=1.062737in,left,base]{\color{textcolor}\sffamily\fontsize{10.000000}{12.000000}\selectfont \(\displaystyle 0.02\)}%
\end{pgfscope}%
\begin{pgfscope}%
\pgfsetbuttcap%
\pgfsetroundjoin%
\definecolor{currentfill}{rgb}{0.000000,0.000000,0.000000}%
\pgfsetfillcolor{currentfill}%
\pgfsetlinewidth{0.803000pt}%
\definecolor{currentstroke}{rgb}{0.000000,0.000000,0.000000}%
\pgfsetstrokecolor{currentstroke}%
\pgfsetdash{}{0pt}%
\pgfsys@defobject{currentmarker}{\pgfqpoint{-0.048611in}{0.000000in}}{\pgfqpoint{0.000000in}{0.000000in}}{%
\pgfpathmoveto{\pgfqpoint{0.000000in}{0.000000in}}%
\pgfpathlineto{\pgfqpoint{-0.048611in}{0.000000in}}%
\pgfusepath{stroke,fill}%
}%
\begin{pgfscope}%
\pgfsys@transformshift{0.750000in}{1.788869in}%
\pgfsys@useobject{currentmarker}{}%
\end{pgfscope}%
\end{pgfscope}%
\begin{pgfscope}%
\definecolor{textcolor}{rgb}{0.000000,0.000000,0.000000}%
\pgfsetstrokecolor{textcolor}%
\pgfsetfillcolor{textcolor}%
\pgftext[x=0.405863in,y=1.736107in,left,base]{\color{textcolor}\sffamily\fontsize{10.000000}{12.000000}\selectfont \(\displaystyle 0.03\)}%
\end{pgfscope}%
\begin{pgfscope}%
\definecolor{textcolor}{rgb}{0.000000,0.000000,0.000000}%
\pgfsetstrokecolor{textcolor}%
\pgfsetfillcolor{textcolor}%
\pgftext[x=0.165901in,y=0.325194in,left,base,rotate=90.000000]{\color{textcolor}\sffamily\fontsize{10.000000}{12.000000}\selectfont Probability that number of}%
\end{pgfscope}%
\begin{pgfscope}%
\definecolor{textcolor}{rgb}{0.000000,0.000000,0.000000}%
\pgfsetstrokecolor{textcolor}%
\pgfsetfillcolor{textcolor}%
\pgftext[x=0.321418in,y=0.160467in,left,base,rotate=90.000000]{\color{textcolor}\sffamily\fontsize{10.000000}{12.000000}\selectfont GMRES iterations is at most 12}%
\end{pgfscope}%
\begin{pgfscope}%
\pgfpathrectangle{\pgfqpoint{0.750000in}{0.275000in}}{\pgfqpoint{4.650000in}{1.925000in}}%
\pgfusepath{clip}%
\pgfsetbuttcap%
\pgfsetroundjoin%
\definecolor{currentfill}{rgb}{0.000000,0.000000,0.000000}%
\pgfsetfillcolor{currentfill}%
\pgfsetlinewidth{1.003750pt}%
\definecolor{currentstroke}{rgb}{0.000000,0.000000,0.000000}%
\pgfsetstrokecolor{currentstroke}%
\pgfsetdash{}{0pt}%
\pgfsys@defobject{currentmarker}{\pgfqpoint{-0.020833in}{-0.020833in}}{\pgfqpoint{0.020833in}{0.020833in}}{%
\pgfpathmoveto{\pgfqpoint{0.000000in}{-0.020833in}}%
\pgfpathcurveto{\pgfqpoint{0.005525in}{-0.020833in}}{\pgfqpoint{0.010825in}{-0.018638in}}{\pgfqpoint{0.014731in}{-0.014731in}}%
\pgfpathcurveto{\pgfqpoint{0.018638in}{-0.010825in}}{\pgfqpoint{0.020833in}{-0.005525in}}{\pgfqpoint{0.020833in}{0.000000in}}%
\pgfpathcurveto{\pgfqpoint{0.020833in}{0.005525in}}{\pgfqpoint{0.018638in}{0.010825in}}{\pgfqpoint{0.014731in}{0.014731in}}%
\pgfpathcurveto{\pgfqpoint{0.010825in}{0.018638in}}{\pgfqpoint{0.005525in}{0.020833in}}{\pgfqpoint{0.000000in}{0.020833in}}%
\pgfpathcurveto{\pgfqpoint{-0.005525in}{0.020833in}}{\pgfqpoint{-0.010825in}{0.018638in}}{\pgfqpoint{-0.014731in}{0.014731in}}%
\pgfpathcurveto{\pgfqpoint{-0.018638in}{0.010825in}}{\pgfqpoint{-0.020833in}{0.005525in}}{\pgfqpoint{-0.020833in}{0.000000in}}%
\pgfpathcurveto{\pgfqpoint{-0.020833in}{-0.005525in}}{\pgfqpoint{-0.018638in}{-0.010825in}}{\pgfqpoint{-0.014731in}{-0.014731in}}%
\pgfpathcurveto{\pgfqpoint{-0.010825in}{-0.018638in}}{\pgfqpoint{-0.005525in}{-0.020833in}}{\pgfqpoint{0.000000in}{-0.020833in}}%
\pgfpathclose%
\pgfusepath{stroke,fill}%
}%
\begin{pgfscope}%
\pgfsys@transformshift{0.961364in}{2.112500in}%
\pgfsys@useobject{currentmarker}{}%
\end{pgfscope}%
\begin{pgfscope}%
\pgfsys@transformshift{1.003636in}{2.045403in}%
\pgfsys@useobject{currentmarker}{}%
\end{pgfscope}%
\begin{pgfscope}%
\pgfsys@transformshift{1.045909in}{1.982041in}%
\pgfsys@useobject{currentmarker}{}%
\end{pgfscope}%
\begin{pgfscope}%
\pgfsys@transformshift{1.088182in}{1.922109in}%
\pgfsys@useobject{currentmarker}{}%
\end{pgfscope}%
\begin{pgfscope}%
\pgfsys@transformshift{1.130455in}{1.865338in}%
\pgfsys@useobject{currentmarker}{}%
\end{pgfscope}%
\begin{pgfscope}%
\pgfsys@transformshift{1.172727in}{1.811483in}%
\pgfsys@useobject{currentmarker}{}%
\end{pgfscope}%
\begin{pgfscope}%
\pgfsys@transformshift{1.215000in}{1.760325in}%
\pgfsys@useobject{currentmarker}{}%
\end{pgfscope}%
\begin{pgfscope}%
\pgfsys@transformshift{1.257273in}{1.711666in}%
\pgfsys@useobject{currentmarker}{}%
\end{pgfscope}%
\begin{pgfscope}%
\pgfsys@transformshift{1.299545in}{1.665329in}%
\pgfsys@useobject{currentmarker}{}%
\end{pgfscope}%
\begin{pgfscope}%
\pgfsys@transformshift{1.341818in}{1.621150in}%
\pgfsys@useobject{currentmarker}{}%
\end{pgfscope}%
\begin{pgfscope}%
\pgfsys@transformshift{1.384091in}{1.578982in}%
\pgfsys@useobject{currentmarker}{}%
\end{pgfscope}%
\begin{pgfscope}%
\pgfsys@transformshift{1.426364in}{1.538692in}%
\pgfsys@useobject{currentmarker}{}%
\end{pgfscope}%
\begin{pgfscope}%
\pgfsys@transformshift{1.468636in}{1.500155in}%
\pgfsys@useobject{currentmarker}{}%
\end{pgfscope}%
\begin{pgfscope}%
\pgfsys@transformshift{1.510909in}{1.463261in}%
\pgfsys@useobject{currentmarker}{}%
\end{pgfscope}%
\begin{pgfscope}%
\pgfsys@transformshift{1.553182in}{1.427907in}%
\pgfsys@useobject{currentmarker}{}%
\end{pgfscope}%
\begin{pgfscope}%
\pgfsys@transformshift{1.595455in}{1.393997in}%
\pgfsys@useobject{currentmarker}{}%
\end{pgfscope}%
\begin{pgfscope}%
\pgfsys@transformshift{1.637727in}{1.361446in}%
\pgfsys@useobject{currentmarker}{}%
\end{pgfscope}%
\begin{pgfscope}%
\pgfsys@transformshift{1.680000in}{1.330173in}%
\pgfsys@useobject{currentmarker}{}%
\end{pgfscope}%
\begin{pgfscope}%
\pgfsys@transformshift{1.722273in}{1.300104in}%
\pgfsys@useobject{currentmarker}{}%
\end{pgfscope}%
\begin{pgfscope}%
\pgfsys@transformshift{1.764545in}{1.271172in}%
\pgfsys@useobject{currentmarker}{}%
\end{pgfscope}%
\begin{pgfscope}%
\pgfsys@transformshift{1.806818in}{1.243312in}%
\pgfsys@useobject{currentmarker}{}%
\end{pgfscope}%
\begin{pgfscope}%
\pgfsys@transformshift{1.849091in}{1.216467in}%
\pgfsys@useobject{currentmarker}{}%
\end{pgfscope}%
\begin{pgfscope}%
\pgfsys@transformshift{1.891364in}{1.190582in}%
\pgfsys@useobject{currentmarker}{}%
\end{pgfscope}%
\begin{pgfscope}%
\pgfsys@transformshift{1.933636in}{1.165606in}%
\pgfsys@useobject{currentmarker}{}%
\end{pgfscope}%
\begin{pgfscope}%
\pgfsys@transformshift{1.975909in}{1.141492in}%
\pgfsys@useobject{currentmarker}{}%
\end{pgfscope}%
\begin{pgfscope}%
\pgfsys@transformshift{2.018182in}{1.118197in}%
\pgfsys@useobject{currentmarker}{}%
\end{pgfscope}%
\begin{pgfscope}%
\pgfsys@transformshift{2.060455in}{1.095679in}%
\pgfsys@useobject{currentmarker}{}%
\end{pgfscope}%
\begin{pgfscope}%
\pgfsys@transformshift{2.102727in}{1.073901in}%
\pgfsys@useobject{currentmarker}{}%
\end{pgfscope}%
\begin{pgfscope}%
\pgfsys@transformshift{2.145000in}{1.052825in}%
\pgfsys@useobject{currentmarker}{}%
\end{pgfscope}%
\begin{pgfscope}%
\pgfsys@transformshift{2.187273in}{1.032420in}%
\pgfsys@useobject{currentmarker}{}%
\end{pgfscope}%
\begin{pgfscope}%
\pgfsys@transformshift{2.229545in}{1.012653in}%
\pgfsys@useobject{currentmarker}{}%
\end{pgfscope}%
\begin{pgfscope}%
\pgfsys@transformshift{2.271818in}{0.993494in}%
\pgfsys@useobject{currentmarker}{}%
\end{pgfscope}%
\begin{pgfscope}%
\pgfsys@transformshift{2.314091in}{0.974917in}%
\pgfsys@useobject{currentmarker}{}%
\end{pgfscope}%
\begin{pgfscope}%
\pgfsys@transformshift{2.356364in}{0.956895in}%
\pgfsys@useobject{currentmarker}{}%
\end{pgfscope}%
\begin{pgfscope}%
\pgfsys@transformshift{2.398636in}{0.939404in}%
\pgfsys@useobject{currentmarker}{}%
\end{pgfscope}%
\begin{pgfscope}%
\pgfsys@transformshift{2.440909in}{0.922420in}%
\pgfsys@useobject{currentmarker}{}%
\end{pgfscope}%
\begin{pgfscope}%
\pgfsys@transformshift{2.483182in}{0.905922in}%
\pgfsys@useobject{currentmarker}{}%
\end{pgfscope}%
\begin{pgfscope}%
\pgfsys@transformshift{2.525455in}{0.889889in}%
\pgfsys@useobject{currentmarker}{}%
\end{pgfscope}%
\begin{pgfscope}%
\pgfsys@transformshift{2.567727in}{0.874302in}%
\pgfsys@useobject{currentmarker}{}%
\end{pgfscope}%
\begin{pgfscope}%
\pgfsys@transformshift{2.610000in}{0.859143in}%
\pgfsys@useobject{currentmarker}{}%
\end{pgfscope}%
\begin{pgfscope}%
\pgfsys@transformshift{2.652273in}{0.844393in}%
\pgfsys@useobject{currentmarker}{}%
\end{pgfscope}%
\begin{pgfscope}%
\pgfsys@transformshift{2.694545in}{0.830037in}%
\pgfsys@useobject{currentmarker}{}%
\end{pgfscope}%
\begin{pgfscope}%
\pgfsys@transformshift{2.736818in}{0.816060in}%
\pgfsys@useobject{currentmarker}{}%
\end{pgfscope}%
\begin{pgfscope}%
\pgfsys@transformshift{2.779091in}{0.802446in}%
\pgfsys@useobject{currentmarker}{}%
\end{pgfscope}%
\begin{pgfscope}%
\pgfsys@transformshift{2.821364in}{0.789181in}%
\pgfsys@useobject{currentmarker}{}%
\end{pgfscope}%
\begin{pgfscope}%
\pgfsys@transformshift{2.863636in}{0.776252in}%
\pgfsys@useobject{currentmarker}{}%
\end{pgfscope}%
\begin{pgfscope}%
\pgfsys@transformshift{2.905909in}{0.763647in}%
\pgfsys@useobject{currentmarker}{}%
\end{pgfscope}%
\begin{pgfscope}%
\pgfsys@transformshift{2.948182in}{0.751353in}%
\pgfsys@useobject{currentmarker}{}%
\end{pgfscope}%
\begin{pgfscope}%
\pgfsys@transformshift{2.990455in}{0.739359in}%
\pgfsys@useobject{currentmarker}{}%
\end{pgfscope}%
\begin{pgfscope}%
\pgfsys@transformshift{3.032727in}{0.727655in}%
\pgfsys@useobject{currentmarker}{}%
\end{pgfscope}%
\begin{pgfscope}%
\pgfsys@transformshift{3.075000in}{0.716230in}%
\pgfsys@useobject{currentmarker}{}%
\end{pgfscope}%
\begin{pgfscope}%
\pgfsys@transformshift{3.117273in}{0.705073in}%
\pgfsys@useobject{currentmarker}{}%
\end{pgfscope}%
\begin{pgfscope}%
\pgfsys@transformshift{3.159545in}{0.694177in}%
\pgfsys@useobject{currentmarker}{}%
\end{pgfscope}%
\begin{pgfscope}%
\pgfsys@transformshift{3.201818in}{0.683531in}%
\pgfsys@useobject{currentmarker}{}%
\end{pgfscope}%
\begin{pgfscope}%
\pgfsys@transformshift{3.244091in}{0.673127in}%
\pgfsys@useobject{currentmarker}{}%
\end{pgfscope}%
\begin{pgfscope}%
\pgfsys@transformshift{3.286364in}{0.662957in}%
\pgfsys@useobject{currentmarker}{}%
\end{pgfscope}%
\begin{pgfscope}%
\pgfsys@transformshift{3.328636in}{0.653013in}%
\pgfsys@useobject{currentmarker}{}%
\end{pgfscope}%
\begin{pgfscope}%
\pgfsys@transformshift{3.370909in}{0.643288in}%
\pgfsys@useobject{currentmarker}{}%
\end{pgfscope}%
\begin{pgfscope}%
\pgfsys@transformshift{3.413182in}{0.633775in}%
\pgfsys@useobject{currentmarker}{}%
\end{pgfscope}%
\begin{pgfscope}%
\pgfsys@transformshift{3.455455in}{0.624466in}%
\pgfsys@useobject{currentmarker}{}%
\end{pgfscope}%
\begin{pgfscope}%
\pgfsys@transformshift{3.497727in}{0.615356in}%
\pgfsys@useobject{currentmarker}{}%
\end{pgfscope}%
\begin{pgfscope}%
\pgfsys@transformshift{3.540000in}{0.606437in}%
\pgfsys@useobject{currentmarker}{}%
\end{pgfscope}%
\begin{pgfscope}%
\pgfsys@transformshift{3.582273in}{0.597705in}%
\pgfsys@useobject{currentmarker}{}%
\end{pgfscope}%
\begin{pgfscope}%
\pgfsys@transformshift{3.624545in}{0.589152in}%
\pgfsys@useobject{currentmarker}{}%
\end{pgfscope}%
\begin{pgfscope}%
\pgfsys@transformshift{3.666818in}{0.580775in}%
\pgfsys@useobject{currentmarker}{}%
\end{pgfscope}%
\begin{pgfscope}%
\pgfsys@transformshift{3.709091in}{0.572566in}%
\pgfsys@useobject{currentmarker}{}%
\end{pgfscope}%
\begin{pgfscope}%
\pgfsys@transformshift{3.751364in}{0.564522in}%
\pgfsys@useobject{currentmarker}{}%
\end{pgfscope}%
\begin{pgfscope}%
\pgfsys@transformshift{3.793636in}{0.556638in}%
\pgfsys@useobject{currentmarker}{}%
\end{pgfscope}%
\begin{pgfscope}%
\pgfsys@transformshift{3.835909in}{0.548908in}%
\pgfsys@useobject{currentmarker}{}%
\end{pgfscope}%
\begin{pgfscope}%
\pgfsys@transformshift{3.878182in}{0.541328in}%
\pgfsys@useobject{currentmarker}{}%
\end{pgfscope}%
\begin{pgfscope}%
\pgfsys@transformshift{3.920455in}{0.533894in}%
\pgfsys@useobject{currentmarker}{}%
\end{pgfscope}%
\begin{pgfscope}%
\pgfsys@transformshift{3.962727in}{0.526602in}%
\pgfsys@useobject{currentmarker}{}%
\end{pgfscope}%
\begin{pgfscope}%
\pgfsys@transformshift{4.005000in}{0.519448in}%
\pgfsys@useobject{currentmarker}{}%
\end{pgfscope}%
\begin{pgfscope}%
\pgfsys@transformshift{4.047273in}{0.512427in}%
\pgfsys@useobject{currentmarker}{}%
\end{pgfscope}%
\begin{pgfscope}%
\pgfsys@transformshift{4.089545in}{0.505536in}%
\pgfsys@useobject{currentmarker}{}%
\end{pgfscope}%
\begin{pgfscope}%
\pgfsys@transformshift{4.131818in}{0.498772in}%
\pgfsys@useobject{currentmarker}{}%
\end{pgfscope}%
\begin{pgfscope}%
\pgfsys@transformshift{4.174091in}{0.492131in}%
\pgfsys@useobject{currentmarker}{}%
\end{pgfscope}%
\begin{pgfscope}%
\pgfsys@transformshift{4.216364in}{0.485610in}%
\pgfsys@useobject{currentmarker}{}%
\end{pgfscope}%
\begin{pgfscope}%
\pgfsys@transformshift{4.258636in}{0.479205in}%
\pgfsys@useobject{currentmarker}{}%
\end{pgfscope}%
\begin{pgfscope}%
\pgfsys@transformshift{4.300909in}{0.472914in}%
\pgfsys@useobject{currentmarker}{}%
\end{pgfscope}%
\begin{pgfscope}%
\pgfsys@transformshift{4.343182in}{0.466733in}%
\pgfsys@useobject{currentmarker}{}%
\end{pgfscope}%
\begin{pgfscope}%
\pgfsys@transformshift{4.385455in}{0.460660in}%
\pgfsys@useobject{currentmarker}{}%
\end{pgfscope}%
\begin{pgfscope}%
\pgfsys@transformshift{4.427727in}{0.454692in}%
\pgfsys@useobject{currentmarker}{}%
\end{pgfscope}%
\begin{pgfscope}%
\pgfsys@transformshift{4.470000in}{0.448825in}%
\pgfsys@useobject{currentmarker}{}%
\end{pgfscope}%
\begin{pgfscope}%
\pgfsys@transformshift{4.512273in}{0.443058in}%
\pgfsys@useobject{currentmarker}{}%
\end{pgfscope}%
\begin{pgfscope}%
\pgfsys@transformshift{4.554545in}{0.437388in}%
\pgfsys@useobject{currentmarker}{}%
\end{pgfscope}%
\begin{pgfscope}%
\pgfsys@transformshift{4.596818in}{0.431813in}%
\pgfsys@useobject{currentmarker}{}%
\end{pgfscope}%
\begin{pgfscope}%
\pgfsys@transformshift{4.639091in}{0.426330in}%
\pgfsys@useobject{currentmarker}{}%
\end{pgfscope}%
\begin{pgfscope}%
\pgfsys@transformshift{4.681364in}{0.420937in}%
\pgfsys@useobject{currentmarker}{}%
\end{pgfscope}%
\begin{pgfscope}%
\pgfsys@transformshift{4.723636in}{0.415631in}%
\pgfsys@useobject{currentmarker}{}%
\end{pgfscope}%
\begin{pgfscope}%
\pgfsys@transformshift{4.765909in}{0.410411in}%
\pgfsys@useobject{currentmarker}{}%
\end{pgfscope}%
\begin{pgfscope}%
\pgfsys@transformshift{4.808182in}{0.405275in}%
\pgfsys@useobject{currentmarker}{}%
\end{pgfscope}%
\begin{pgfscope}%
\pgfsys@transformshift{4.850455in}{0.400220in}%
\pgfsys@useobject{currentmarker}{}%
\end{pgfscope}%
\begin{pgfscope}%
\pgfsys@transformshift{4.892727in}{0.395245in}%
\pgfsys@useobject{currentmarker}{}%
\end{pgfscope}%
\begin{pgfscope}%
\pgfsys@transformshift{4.935000in}{0.390348in}%
\pgfsys@useobject{currentmarker}{}%
\end{pgfscope}%
\begin{pgfscope}%
\pgfsys@transformshift{4.977273in}{0.385527in}%
\pgfsys@useobject{currentmarker}{}%
\end{pgfscope}%
\begin{pgfscope}%
\pgfsys@transformshift{5.019545in}{0.380779in}%
\pgfsys@useobject{currentmarker}{}%
\end{pgfscope}%
\begin{pgfscope}%
\pgfsys@transformshift{5.061818in}{0.376105in}%
\pgfsys@useobject{currentmarker}{}%
\end{pgfscope}%
\begin{pgfscope}%
\pgfsys@transformshift{5.104091in}{0.371501in}%
\pgfsys@useobject{currentmarker}{}%
\end{pgfscope}%
\begin{pgfscope}%
\pgfsys@transformshift{5.146364in}{0.366967in}%
\pgfsys@useobject{currentmarker}{}%
\end{pgfscope}%
\begin{pgfscope}%
\pgfsys@transformshift{5.188636in}{0.362500in}%
\pgfsys@useobject{currentmarker}{}%
\end{pgfscope}%
\end{pgfscope}%
\begin{pgfscope}%
\pgfsetrectcap%
\pgfsetmiterjoin%
\pgfsetlinewidth{0.803000pt}%
\definecolor{currentstroke}{rgb}{0.000000,0.000000,0.000000}%
\pgfsetstrokecolor{currentstroke}%
\pgfsetdash{}{0pt}%
\pgfpathmoveto{\pgfqpoint{0.750000in}{0.275000in}}%
\pgfpathlineto{\pgfqpoint{0.750000in}{2.200000in}}%
\pgfusepath{stroke}%
\end{pgfscope}%
\begin{pgfscope}%
\pgfsetrectcap%
\pgfsetmiterjoin%
\pgfsetlinewidth{0.803000pt}%
\definecolor{currentstroke}{rgb}{0.000000,0.000000,0.000000}%
\pgfsetstrokecolor{currentstroke}%
\pgfsetdash{}{0pt}%
\pgfpathmoveto{\pgfqpoint{5.400000in}{0.275000in}}%
\pgfpathlineto{\pgfqpoint{5.400000in}{2.200000in}}%
\pgfusepath{stroke}%
\end{pgfscope}%
\begin{pgfscope}%
\pgfsetrectcap%
\pgfsetmiterjoin%
\pgfsetlinewidth{0.803000pt}%
\definecolor{currentstroke}{rgb}{0.000000,0.000000,0.000000}%
\pgfsetstrokecolor{currentstroke}%
\pgfsetdash{}{0pt}%
\pgfpathmoveto{\pgfqpoint{0.750000in}{0.275000in}}%
\pgfpathlineto{\pgfqpoint{5.400000in}{0.275000in}}%
\pgfusepath{stroke}%
\end{pgfscope}%
\begin{pgfscope}%
\pgfsetrectcap%
\pgfsetmiterjoin%
\pgfsetlinewidth{0.803000pt}%
\definecolor{currentstroke}{rgb}{0.000000,0.000000,0.000000}%
\pgfsetstrokecolor{currentstroke}%
\pgfsetdash{}{0pt}%
\pgfpathmoveto{\pgfqpoint{0.750000in}{2.200000in}}%
\pgfpathlineto{\pgfqpoint{5.400000in}{2.200000in}}%
\pgfusepath{stroke}%
\end{pgfscope}%
\end{pgfpicture}%
\makeatother%
\endgroup%

%% \caption{The lower bound in \cref{eq:GMRESprob} with $\NLiDR{\no-\nt} \sim \Exp{\sigma}$ with $\sigma = 1.$\label{fig:prob-theory-plot-0.0}}
%% \end{subfigure}

%% \begin{subfigure}{\textwidth}
%%     \centering
%% %% Creator: Matplotlib, PGF backend
%%
%% To include the figure in your LaTeX document, write
%%   \input{<filename>.pgf}
%%
%% Make sure the required packages are loaded in your preamble
%%   \usepackage{pgf}
%%
%% Figures using additional raster images can only be included by \input if
%% they are in the same directory as the main LaTeX file. For loading figures
%% from other directories you can use the `import` package
%%   \usepackage{import}
%% and then include the figures with
%%   \import{<path to file>}{<filename>.pgf}
%%
%% Matplotlib used the following preamble
%%   \usepackage{fontspec}
%%   \setmainfont{DejaVuSerif.ttf}[Path=/home/owen/progs/firedrake-complex/firedrake/lib/python3.5/site-packages/matplotlib/mpl-data/fonts/ttf/]
%%   \setsansfont{DejaVuSans.ttf}[Path=/home/owen/progs/firedrake-complex/firedrake/lib/python3.5/site-packages/matplotlib/mpl-data/fonts/ttf/]
%%   \setmonofont{DejaVuSansMono.ttf}[Path=/home/owen/progs/firedrake-complex/firedrake/lib/python3.5/site-packages/matplotlib/mpl-data/fonts/ttf/]
%%
\begingroup%
\makeatletter%
\begin{pgfpicture}%
\pgfpathrectangle{\pgfpointorigin}{\pgfqpoint{6.400000in}{4.800000in}}%
\pgfusepath{use as bounding box, clip}%
\begin{pgfscope}%
\pgfsetbuttcap%
\pgfsetmiterjoin%
\definecolor{currentfill}{rgb}{1.000000,1.000000,1.000000}%
\pgfsetfillcolor{currentfill}%
\pgfsetlinewidth{0.000000pt}%
\definecolor{currentstroke}{rgb}{1.000000,1.000000,1.000000}%
\pgfsetstrokecolor{currentstroke}%
\pgfsetdash{}{0pt}%
\pgfpathmoveto{\pgfqpoint{0.000000in}{0.000000in}}%
\pgfpathlineto{\pgfqpoint{6.400000in}{0.000000in}}%
\pgfpathlineto{\pgfqpoint{6.400000in}{4.800000in}}%
\pgfpathlineto{\pgfqpoint{0.000000in}{4.800000in}}%
\pgfpathclose%
\pgfusepath{fill}%
\end{pgfscope}%
\begin{pgfscope}%
\pgfsetbuttcap%
\pgfsetmiterjoin%
\definecolor{currentfill}{rgb}{1.000000,1.000000,1.000000}%
\pgfsetfillcolor{currentfill}%
\pgfsetlinewidth{0.000000pt}%
\definecolor{currentstroke}{rgb}{0.000000,0.000000,0.000000}%
\pgfsetstrokecolor{currentstroke}%
\pgfsetstrokeopacity{0.000000}%
\pgfsetdash{}{0pt}%
\pgfpathmoveto{\pgfqpoint{0.800000in}{0.528000in}}%
\pgfpathlineto{\pgfqpoint{5.760000in}{0.528000in}}%
\pgfpathlineto{\pgfqpoint{5.760000in}{4.224000in}}%
\pgfpathlineto{\pgfqpoint{0.800000in}{4.224000in}}%
\pgfpathclose%
\pgfusepath{fill}%
\end{pgfscope}%
\begin{pgfscope}%
\pgfsetbuttcap%
\pgfsetroundjoin%
\definecolor{currentfill}{rgb}{0.000000,0.000000,0.000000}%
\pgfsetfillcolor{currentfill}%
\pgfsetlinewidth{0.803000pt}%
\definecolor{currentstroke}{rgb}{0.000000,0.000000,0.000000}%
\pgfsetstrokecolor{currentstroke}%
\pgfsetdash{}{0pt}%
\pgfsys@defobject{currentmarker}{\pgfqpoint{0.000000in}{-0.048611in}}{\pgfqpoint{0.000000in}{0.000000in}}{%
\pgfpathmoveto{\pgfqpoint{0.000000in}{0.000000in}}%
\pgfpathlineto{\pgfqpoint{0.000000in}{-0.048611in}}%
\pgfusepath{stroke,fill}%
}%
\begin{pgfscope}%
\pgfsys@transformshift{1.025455in}{0.528000in}%
\pgfsys@useobject{currentmarker}{}%
\end{pgfscope}%
\end{pgfscope}%
\begin{pgfscope}%
\definecolor{textcolor}{rgb}{0.000000,0.000000,0.000000}%
\pgfsetstrokecolor{textcolor}%
\pgfsetfillcolor{textcolor}%
\pgftext[x=1.025455in,y=0.430778in,,top]{\color{textcolor}\sffamily\fontsize{10.000000}{12.000000}\selectfont 10}%
\end{pgfscope}%
\begin{pgfscope}%
\pgfsetbuttcap%
\pgfsetroundjoin%
\definecolor{currentfill}{rgb}{0.000000,0.000000,0.000000}%
\pgfsetfillcolor{currentfill}%
\pgfsetlinewidth{0.803000pt}%
\definecolor{currentstroke}{rgb}{0.000000,0.000000,0.000000}%
\pgfsetstrokecolor{currentstroke}%
\pgfsetdash{}{0pt}%
\pgfsys@defobject{currentmarker}{\pgfqpoint{0.000000in}{-0.048611in}}{\pgfqpoint{0.000000in}{0.000000in}}{%
\pgfpathmoveto{\pgfqpoint{0.000000in}{0.000000in}}%
\pgfpathlineto{\pgfqpoint{0.000000in}{-0.048611in}}%
\pgfusepath{stroke,fill}%
}%
\begin{pgfscope}%
\pgfsys@transformshift{1.776970in}{0.528000in}%
\pgfsys@useobject{currentmarker}{}%
\end{pgfscope}%
\end{pgfscope}%
\begin{pgfscope}%
\definecolor{textcolor}{rgb}{0.000000,0.000000,0.000000}%
\pgfsetstrokecolor{textcolor}%
\pgfsetfillcolor{textcolor}%
\pgftext[x=1.776970in,y=0.430778in,,top]{\color{textcolor}\sffamily\fontsize{10.000000}{12.000000}\selectfont 15}%
\end{pgfscope}%
\begin{pgfscope}%
\pgfsetbuttcap%
\pgfsetroundjoin%
\definecolor{currentfill}{rgb}{0.000000,0.000000,0.000000}%
\pgfsetfillcolor{currentfill}%
\pgfsetlinewidth{0.803000pt}%
\definecolor{currentstroke}{rgb}{0.000000,0.000000,0.000000}%
\pgfsetstrokecolor{currentstroke}%
\pgfsetdash{}{0pt}%
\pgfsys@defobject{currentmarker}{\pgfqpoint{0.000000in}{-0.048611in}}{\pgfqpoint{0.000000in}{0.000000in}}{%
\pgfpathmoveto{\pgfqpoint{0.000000in}{0.000000in}}%
\pgfpathlineto{\pgfqpoint{0.000000in}{-0.048611in}}%
\pgfusepath{stroke,fill}%
}%
\begin{pgfscope}%
\pgfsys@transformshift{2.528485in}{0.528000in}%
\pgfsys@useobject{currentmarker}{}%
\end{pgfscope}%
\end{pgfscope}%
\begin{pgfscope}%
\definecolor{textcolor}{rgb}{0.000000,0.000000,0.000000}%
\pgfsetstrokecolor{textcolor}%
\pgfsetfillcolor{textcolor}%
\pgftext[x=2.528485in,y=0.430778in,,top]{\color{textcolor}\sffamily\fontsize{10.000000}{12.000000}\selectfont 20}%
\end{pgfscope}%
\begin{pgfscope}%
\pgfsetbuttcap%
\pgfsetroundjoin%
\definecolor{currentfill}{rgb}{0.000000,0.000000,0.000000}%
\pgfsetfillcolor{currentfill}%
\pgfsetlinewidth{0.803000pt}%
\definecolor{currentstroke}{rgb}{0.000000,0.000000,0.000000}%
\pgfsetstrokecolor{currentstroke}%
\pgfsetdash{}{0pt}%
\pgfsys@defobject{currentmarker}{\pgfqpoint{0.000000in}{-0.048611in}}{\pgfqpoint{0.000000in}{0.000000in}}{%
\pgfpathmoveto{\pgfqpoint{0.000000in}{0.000000in}}%
\pgfpathlineto{\pgfqpoint{0.000000in}{-0.048611in}}%
\pgfusepath{stroke,fill}%
}%
\begin{pgfscope}%
\pgfsys@transformshift{3.280000in}{0.528000in}%
\pgfsys@useobject{currentmarker}{}%
\end{pgfscope}%
\end{pgfscope}%
\begin{pgfscope}%
\definecolor{textcolor}{rgb}{0.000000,0.000000,0.000000}%
\pgfsetstrokecolor{textcolor}%
\pgfsetfillcolor{textcolor}%
\pgftext[x=3.280000in,y=0.430778in,,top]{\color{textcolor}\sffamily\fontsize{10.000000}{12.000000}\selectfont 25}%
\end{pgfscope}%
\begin{pgfscope}%
\pgfsetbuttcap%
\pgfsetroundjoin%
\definecolor{currentfill}{rgb}{0.000000,0.000000,0.000000}%
\pgfsetfillcolor{currentfill}%
\pgfsetlinewidth{0.803000pt}%
\definecolor{currentstroke}{rgb}{0.000000,0.000000,0.000000}%
\pgfsetstrokecolor{currentstroke}%
\pgfsetdash{}{0pt}%
\pgfsys@defobject{currentmarker}{\pgfqpoint{0.000000in}{-0.048611in}}{\pgfqpoint{0.000000in}{0.000000in}}{%
\pgfpathmoveto{\pgfqpoint{0.000000in}{0.000000in}}%
\pgfpathlineto{\pgfqpoint{0.000000in}{-0.048611in}}%
\pgfusepath{stroke,fill}%
}%
\begin{pgfscope}%
\pgfsys@transformshift{4.031515in}{0.528000in}%
\pgfsys@useobject{currentmarker}{}%
\end{pgfscope}%
\end{pgfscope}%
\begin{pgfscope}%
\definecolor{textcolor}{rgb}{0.000000,0.000000,0.000000}%
\pgfsetstrokecolor{textcolor}%
\pgfsetfillcolor{textcolor}%
\pgftext[x=4.031515in,y=0.430778in,,top]{\color{textcolor}\sffamily\fontsize{10.000000}{12.000000}\selectfont 30}%
\end{pgfscope}%
\begin{pgfscope}%
\pgfsetbuttcap%
\pgfsetroundjoin%
\definecolor{currentfill}{rgb}{0.000000,0.000000,0.000000}%
\pgfsetfillcolor{currentfill}%
\pgfsetlinewidth{0.803000pt}%
\definecolor{currentstroke}{rgb}{0.000000,0.000000,0.000000}%
\pgfsetstrokecolor{currentstroke}%
\pgfsetdash{}{0pt}%
\pgfsys@defobject{currentmarker}{\pgfqpoint{0.000000in}{-0.048611in}}{\pgfqpoint{0.000000in}{0.000000in}}{%
\pgfpathmoveto{\pgfqpoint{0.000000in}{0.000000in}}%
\pgfpathlineto{\pgfqpoint{0.000000in}{-0.048611in}}%
\pgfusepath{stroke,fill}%
}%
\begin{pgfscope}%
\pgfsys@transformshift{4.783030in}{0.528000in}%
\pgfsys@useobject{currentmarker}{}%
\end{pgfscope}%
\end{pgfscope}%
\begin{pgfscope}%
\definecolor{textcolor}{rgb}{0.000000,0.000000,0.000000}%
\pgfsetstrokecolor{textcolor}%
\pgfsetfillcolor{textcolor}%
\pgftext[x=4.783030in,y=0.430778in,,top]{\color{textcolor}\sffamily\fontsize{10.000000}{12.000000}\selectfont 35}%
\end{pgfscope}%
\begin{pgfscope}%
\pgfsetbuttcap%
\pgfsetroundjoin%
\definecolor{currentfill}{rgb}{0.000000,0.000000,0.000000}%
\pgfsetfillcolor{currentfill}%
\pgfsetlinewidth{0.803000pt}%
\definecolor{currentstroke}{rgb}{0.000000,0.000000,0.000000}%
\pgfsetstrokecolor{currentstroke}%
\pgfsetdash{}{0pt}%
\pgfsys@defobject{currentmarker}{\pgfqpoint{0.000000in}{-0.048611in}}{\pgfqpoint{0.000000in}{0.000000in}}{%
\pgfpathmoveto{\pgfqpoint{0.000000in}{0.000000in}}%
\pgfpathlineto{\pgfqpoint{0.000000in}{-0.048611in}}%
\pgfusepath{stroke,fill}%
}%
\begin{pgfscope}%
\pgfsys@transformshift{5.534545in}{0.528000in}%
\pgfsys@useobject{currentmarker}{}%
\end{pgfscope}%
\end{pgfscope}%
\begin{pgfscope}%
\definecolor{textcolor}{rgb}{0.000000,0.000000,0.000000}%
\pgfsetstrokecolor{textcolor}%
\pgfsetfillcolor{textcolor}%
\pgftext[x=5.534545in,y=0.430778in,,top]{\color{textcolor}\sffamily\fontsize{10.000000}{12.000000}\selectfont 40}%
\end{pgfscope}%
\begin{pgfscope}%
\definecolor{textcolor}{rgb}{0.000000,0.000000,0.000000}%
\pgfsetstrokecolor{textcolor}%
\pgfsetfillcolor{textcolor}%
\pgftext[x=3.280000in,y=0.240809in,,top]{\color{textcolor}\sffamily\fontsize{10.000000}{12.000000}\selectfont \(\displaystyle k\)}%
\end{pgfscope}%
\begin{pgfscope}%
\pgfsetbuttcap%
\pgfsetroundjoin%
\definecolor{currentfill}{rgb}{0.000000,0.000000,0.000000}%
\pgfsetfillcolor{currentfill}%
\pgfsetlinewidth{0.803000pt}%
\definecolor{currentstroke}{rgb}{0.000000,0.000000,0.000000}%
\pgfsetstrokecolor{currentstroke}%
\pgfsetdash{}{0pt}%
\pgfsys@defobject{currentmarker}{\pgfqpoint{-0.048611in}{0.000000in}}{\pgfqpoint{0.000000in}{0.000000in}}{%
\pgfpathmoveto{\pgfqpoint{0.000000in}{0.000000in}}%
\pgfpathlineto{\pgfqpoint{-0.048611in}{0.000000in}}%
\pgfusepath{stroke,fill}%
}%
\begin{pgfscope}%
\pgfsys@transformshift{0.800000in}{1.059932in}%
\pgfsys@useobject{currentmarker}{}%
\end{pgfscope}%
\end{pgfscope}%
\begin{pgfscope}%
\definecolor{textcolor}{rgb}{0.000000,0.000000,0.000000}%
\pgfsetstrokecolor{textcolor}%
\pgfsetfillcolor{textcolor}%
\pgftext[x=0.481898in,y=1.007170in,left,base]{\color{textcolor}\sffamily\fontsize{10.000000}{12.000000}\selectfont 6.5}%
\end{pgfscope}%
\begin{pgfscope}%
\pgfsetbuttcap%
\pgfsetroundjoin%
\definecolor{currentfill}{rgb}{0.000000,0.000000,0.000000}%
\pgfsetfillcolor{currentfill}%
\pgfsetlinewidth{0.803000pt}%
\definecolor{currentstroke}{rgb}{0.000000,0.000000,0.000000}%
\pgfsetstrokecolor{currentstroke}%
\pgfsetdash{}{0pt}%
\pgfsys@defobject{currentmarker}{\pgfqpoint{-0.048611in}{0.000000in}}{\pgfqpoint{0.000000in}{0.000000in}}{%
\pgfpathmoveto{\pgfqpoint{0.000000in}{0.000000in}}%
\pgfpathlineto{\pgfqpoint{-0.048611in}{0.000000in}}%
\pgfusepath{stroke,fill}%
}%
\begin{pgfscope}%
\pgfsys@transformshift{0.800000in}{2.047100in}%
\pgfsys@useobject{currentmarker}{}%
\end{pgfscope}%
\end{pgfscope}%
\begin{pgfscope}%
\definecolor{textcolor}{rgb}{0.000000,0.000000,0.000000}%
\pgfsetstrokecolor{textcolor}%
\pgfsetfillcolor{textcolor}%
\pgftext[x=0.481898in,y=1.994338in,left,base]{\color{textcolor}\sffamily\fontsize{10.000000}{12.000000}\selectfont 7.0}%
\end{pgfscope}%
\begin{pgfscope}%
\pgfsetbuttcap%
\pgfsetroundjoin%
\definecolor{currentfill}{rgb}{0.000000,0.000000,0.000000}%
\pgfsetfillcolor{currentfill}%
\pgfsetlinewidth{0.803000pt}%
\definecolor{currentstroke}{rgb}{0.000000,0.000000,0.000000}%
\pgfsetstrokecolor{currentstroke}%
\pgfsetdash{}{0pt}%
\pgfsys@defobject{currentmarker}{\pgfqpoint{-0.048611in}{0.000000in}}{\pgfqpoint{0.000000in}{0.000000in}}{%
\pgfpathmoveto{\pgfqpoint{0.000000in}{0.000000in}}%
\pgfpathlineto{\pgfqpoint{-0.048611in}{0.000000in}}%
\pgfusepath{stroke,fill}%
}%
\begin{pgfscope}%
\pgfsys@transformshift{0.800000in}{3.034268in}%
\pgfsys@useobject{currentmarker}{}%
\end{pgfscope}%
\end{pgfscope}%
\begin{pgfscope}%
\definecolor{textcolor}{rgb}{0.000000,0.000000,0.000000}%
\pgfsetstrokecolor{textcolor}%
\pgfsetfillcolor{textcolor}%
\pgftext[x=0.481898in,y=2.981506in,left,base]{\color{textcolor}\sffamily\fontsize{10.000000}{12.000000}\selectfont 7.5}%
\end{pgfscope}%
\begin{pgfscope}%
\pgfsetbuttcap%
\pgfsetroundjoin%
\definecolor{currentfill}{rgb}{0.000000,0.000000,0.000000}%
\pgfsetfillcolor{currentfill}%
\pgfsetlinewidth{0.803000pt}%
\definecolor{currentstroke}{rgb}{0.000000,0.000000,0.000000}%
\pgfsetstrokecolor{currentstroke}%
\pgfsetdash{}{0pt}%
\pgfsys@defobject{currentmarker}{\pgfqpoint{-0.048611in}{0.000000in}}{\pgfqpoint{0.000000in}{0.000000in}}{%
\pgfpathmoveto{\pgfqpoint{0.000000in}{0.000000in}}%
\pgfpathlineto{\pgfqpoint{-0.048611in}{0.000000in}}%
\pgfusepath{stroke,fill}%
}%
\begin{pgfscope}%
\pgfsys@transformshift{0.800000in}{4.021436in}%
\pgfsys@useobject{currentmarker}{}%
\end{pgfscope}%
\end{pgfscope}%
\begin{pgfscope}%
\definecolor{textcolor}{rgb}{0.000000,0.000000,0.000000}%
\pgfsetstrokecolor{textcolor}%
\pgfsetfillcolor{textcolor}%
\pgftext[x=0.481898in,y=3.968674in,left,base]{\color{textcolor}\sffamily\fontsize{10.000000}{12.000000}\selectfont 8.0}%
\end{pgfscope}%
\begin{pgfscope}%
\definecolor{textcolor}{rgb}{0.000000,0.000000,0.000000}%
\pgfsetstrokecolor{textcolor}%
\pgfsetfillcolor{textcolor}%
\pgftext[x=0.426343in,y=2.376000in,,bottom,rotate=90.000000]{\color{textcolor}\sffamily\fontsize{10.000000}{12.000000}\selectfont Probability Number of GMRES iterations is at most 12}%
\end{pgfscope}%
\begin{pgfscope}%
\definecolor{textcolor}{rgb}{0.000000,0.000000,0.000000}%
\pgfsetstrokecolor{textcolor}%
\pgfsetfillcolor{textcolor}%
\pgftext[x=0.800000in,y=4.265667in,left,base]{\color{textcolor}\sffamily\fontsize{10.000000}{12.000000}\selectfont 1e−11+2.983096487e−1}%
\end{pgfscope}%
\begin{pgfscope}%
\pgfpathrectangle{\pgfqpoint{0.800000in}{0.528000in}}{\pgfqpoint{4.960000in}{3.696000in}}%
\pgfusepath{clip}%
\pgfsetbuttcap%
\pgfsetroundjoin%
\definecolor{currentfill}{rgb}{0.121569,0.466667,0.705882}%
\pgfsetfillcolor{currentfill}%
\pgfsetlinewidth{1.003750pt}%
\definecolor{currentstroke}{rgb}{0.121569,0.466667,0.705882}%
\pgfsetstrokecolor{currentstroke}%
\pgfsetdash{}{0pt}%
\pgfsys@defobject{currentmarker}{\pgfqpoint{-0.020833in}{-0.020833in}}{\pgfqpoint{0.020833in}{0.020833in}}{%
\pgfpathmoveto{\pgfqpoint{0.000000in}{-0.020833in}}%
\pgfpathcurveto{\pgfqpoint{0.005525in}{-0.020833in}}{\pgfqpoint{0.010825in}{-0.018638in}}{\pgfqpoint{0.014731in}{-0.014731in}}%
\pgfpathcurveto{\pgfqpoint{0.018638in}{-0.010825in}}{\pgfqpoint{0.020833in}{-0.005525in}}{\pgfqpoint{0.020833in}{0.000000in}}%
\pgfpathcurveto{\pgfqpoint{0.020833in}{0.005525in}}{\pgfqpoint{0.018638in}{0.010825in}}{\pgfqpoint{0.014731in}{0.014731in}}%
\pgfpathcurveto{\pgfqpoint{0.010825in}{0.018638in}}{\pgfqpoint{0.005525in}{0.020833in}}{\pgfqpoint{0.000000in}{0.020833in}}%
\pgfpathcurveto{\pgfqpoint{-0.005525in}{0.020833in}}{\pgfqpoint{-0.010825in}{0.018638in}}{\pgfqpoint{-0.014731in}{0.014731in}}%
\pgfpathcurveto{\pgfqpoint{-0.018638in}{0.010825in}}{\pgfqpoint{-0.020833in}{0.005525in}}{\pgfqpoint{-0.020833in}{0.000000in}}%
\pgfpathcurveto{\pgfqpoint{-0.020833in}{-0.005525in}}{\pgfqpoint{-0.018638in}{-0.010825in}}{\pgfqpoint{-0.014731in}{-0.014731in}}%
\pgfpathcurveto{\pgfqpoint{-0.010825in}{-0.018638in}}{\pgfqpoint{-0.005525in}{-0.020833in}}{\pgfqpoint{0.000000in}{-0.020833in}}%
\pgfpathclose%
\pgfusepath{stroke,fill}%
}%
\begin{pgfscope}%
\pgfsys@transformshift{1.025455in}{2.375996in}%
\pgfsys@useobject{currentmarker}{}%
\end{pgfscope}%
\begin{pgfscope}%
\pgfsys@transformshift{1.070545in}{2.375986in}%
\pgfsys@useobject{currentmarker}{}%
\end{pgfscope}%
\begin{pgfscope}%
\pgfsys@transformshift{1.115636in}{2.375996in}%
\pgfsys@useobject{currentmarker}{}%
\end{pgfscope}%
\begin{pgfscope}%
\pgfsys@transformshift{1.160727in}{2.376006in}%
\pgfsys@useobject{currentmarker}{}%
\end{pgfscope}%
\begin{pgfscope}%
\pgfsys@transformshift{1.205818in}{2.375986in}%
\pgfsys@useobject{currentmarker}{}%
\end{pgfscope}%
\begin{pgfscope}%
\pgfsys@transformshift{1.250909in}{2.375967in}%
\pgfsys@useobject{currentmarker}{}%
\end{pgfscope}%
\begin{pgfscope}%
\pgfsys@transformshift{1.296000in}{2.375996in}%
\pgfsys@useobject{currentmarker}{}%
\end{pgfscope}%
\begin{pgfscope}%
\pgfsys@transformshift{1.341091in}{2.375986in}%
\pgfsys@useobject{currentmarker}{}%
\end{pgfscope}%
\begin{pgfscope}%
\pgfsys@transformshift{1.386182in}{4.055967in}%
\pgfsys@useobject{currentmarker}{}%
\end{pgfscope}%
\begin{pgfscope}%
\pgfsys@transformshift{1.431273in}{4.055967in}%
\pgfsys@useobject{currentmarker}{}%
\end{pgfscope}%
\begin{pgfscope}%
\pgfsys@transformshift{1.476364in}{4.056006in}%
\pgfsys@useobject{currentmarker}{}%
\end{pgfscope}%
\begin{pgfscope}%
\pgfsys@transformshift{1.521455in}{4.055996in}%
\pgfsys@useobject{currentmarker}{}%
\end{pgfscope}%
\begin{pgfscope}%
\pgfsys@transformshift{1.566545in}{4.055996in}%
\pgfsys@useobject{currentmarker}{}%
\end{pgfscope}%
\begin{pgfscope}%
\pgfsys@transformshift{1.611636in}{4.055977in}%
\pgfsys@useobject{currentmarker}{}%
\end{pgfscope}%
\begin{pgfscope}%
\pgfsys@transformshift{1.656727in}{4.055977in}%
\pgfsys@useobject{currentmarker}{}%
\end{pgfscope}%
\begin{pgfscope}%
\pgfsys@transformshift{1.701818in}{4.056006in}%
\pgfsys@useobject{currentmarker}{}%
\end{pgfscope}%
\begin{pgfscope}%
\pgfsys@transformshift{1.746909in}{4.056006in}%
\pgfsys@useobject{currentmarker}{}%
\end{pgfscope}%
\begin{pgfscope}%
\pgfsys@transformshift{1.792000in}{4.055967in}%
\pgfsys@useobject{currentmarker}{}%
\end{pgfscope}%
\begin{pgfscope}%
\pgfsys@transformshift{1.837091in}{4.055996in}%
\pgfsys@useobject{currentmarker}{}%
\end{pgfscope}%
\begin{pgfscope}%
\pgfsys@transformshift{1.882182in}{4.055977in}%
\pgfsys@useobject{currentmarker}{}%
\end{pgfscope}%
\begin{pgfscope}%
\pgfsys@transformshift{1.927273in}{4.055996in}%
\pgfsys@useobject{currentmarker}{}%
\end{pgfscope}%
\begin{pgfscope}%
\pgfsys@transformshift{1.972364in}{4.055967in}%
\pgfsys@useobject{currentmarker}{}%
\end{pgfscope}%
\begin{pgfscope}%
\pgfsys@transformshift{2.017455in}{4.055977in}%
\pgfsys@useobject{currentmarker}{}%
\end{pgfscope}%
\begin{pgfscope}%
\pgfsys@transformshift{2.062545in}{4.055977in}%
\pgfsys@useobject{currentmarker}{}%
\end{pgfscope}%
\begin{pgfscope}%
\pgfsys@transformshift{2.107636in}{4.055996in}%
\pgfsys@useobject{currentmarker}{}%
\end{pgfscope}%
\begin{pgfscope}%
\pgfsys@transformshift{2.152727in}{4.055977in}%
\pgfsys@useobject{currentmarker}{}%
\end{pgfscope}%
\begin{pgfscope}%
\pgfsys@transformshift{2.197818in}{4.055977in}%
\pgfsys@useobject{currentmarker}{}%
\end{pgfscope}%
\begin{pgfscope}%
\pgfsys@transformshift{2.242909in}{4.055967in}%
\pgfsys@useobject{currentmarker}{}%
\end{pgfscope}%
\begin{pgfscope}%
\pgfsys@transformshift{2.288000in}{4.055996in}%
\pgfsys@useobject{currentmarker}{}%
\end{pgfscope}%
\begin{pgfscope}%
\pgfsys@transformshift{2.333091in}{4.056006in}%
\pgfsys@useobject{currentmarker}{}%
\end{pgfscope}%
\begin{pgfscope}%
\pgfsys@transformshift{2.378182in}{4.055967in}%
\pgfsys@useobject{currentmarker}{}%
\end{pgfscope}%
\begin{pgfscope}%
\pgfsys@transformshift{2.423273in}{4.055977in}%
\pgfsys@useobject{currentmarker}{}%
\end{pgfscope}%
\begin{pgfscope}%
\pgfsys@transformshift{2.468364in}{4.055957in}%
\pgfsys@useobject{currentmarker}{}%
\end{pgfscope}%
\begin{pgfscope}%
\pgfsys@transformshift{2.513455in}{4.055977in}%
\pgfsys@useobject{currentmarker}{}%
\end{pgfscope}%
\begin{pgfscope}%
\pgfsys@transformshift{2.558545in}{4.056006in}%
\pgfsys@useobject{currentmarker}{}%
\end{pgfscope}%
\begin{pgfscope}%
\pgfsys@transformshift{2.603636in}{4.055996in}%
\pgfsys@useobject{currentmarker}{}%
\end{pgfscope}%
\begin{pgfscope}%
\pgfsys@transformshift{2.648727in}{4.055977in}%
\pgfsys@useobject{currentmarker}{}%
\end{pgfscope}%
\begin{pgfscope}%
\pgfsys@transformshift{2.693818in}{4.055977in}%
\pgfsys@useobject{currentmarker}{}%
\end{pgfscope}%
\begin{pgfscope}%
\pgfsys@transformshift{2.738909in}{4.055957in}%
\pgfsys@useobject{currentmarker}{}%
\end{pgfscope}%
\begin{pgfscope}%
\pgfsys@transformshift{2.784000in}{4.055996in}%
\pgfsys@useobject{currentmarker}{}%
\end{pgfscope}%
\begin{pgfscope}%
\pgfsys@transformshift{2.829091in}{4.055967in}%
\pgfsys@useobject{currentmarker}{}%
\end{pgfscope}%
\begin{pgfscope}%
\pgfsys@transformshift{2.874182in}{4.055977in}%
\pgfsys@useobject{currentmarker}{}%
\end{pgfscope}%
\begin{pgfscope}%
\pgfsys@transformshift{2.919273in}{4.055977in}%
\pgfsys@useobject{currentmarker}{}%
\end{pgfscope}%
\begin{pgfscope}%
\pgfsys@transformshift{2.964364in}{4.056006in}%
\pgfsys@useobject{currentmarker}{}%
\end{pgfscope}%
\begin{pgfscope}%
\pgfsys@transformshift{3.009455in}{4.055996in}%
\pgfsys@useobject{currentmarker}{}%
\end{pgfscope}%
\begin{pgfscope}%
\pgfsys@transformshift{3.054545in}{4.055996in}%
\pgfsys@useobject{currentmarker}{}%
\end{pgfscope}%
\begin{pgfscope}%
\pgfsys@transformshift{3.099636in}{4.055996in}%
\pgfsys@useobject{currentmarker}{}%
\end{pgfscope}%
\begin{pgfscope}%
\pgfsys@transformshift{3.144727in}{4.055996in}%
\pgfsys@useobject{currentmarker}{}%
\end{pgfscope}%
\begin{pgfscope}%
\pgfsys@transformshift{3.189818in}{0.696016in}%
\pgfsys@useobject{currentmarker}{}%
\end{pgfscope}%
\begin{pgfscope}%
\pgfsys@transformshift{3.234909in}{0.696025in}%
\pgfsys@useobject{currentmarker}{}%
\end{pgfscope}%
\begin{pgfscope}%
\pgfsys@transformshift{3.280000in}{0.696016in}%
\pgfsys@useobject{currentmarker}{}%
\end{pgfscope}%
\begin{pgfscope}%
\pgfsys@transformshift{3.325091in}{0.696025in}%
\pgfsys@useobject{currentmarker}{}%
\end{pgfscope}%
\begin{pgfscope}%
\pgfsys@transformshift{3.370182in}{0.696025in}%
\pgfsys@useobject{currentmarker}{}%
\end{pgfscope}%
\begin{pgfscope}%
\pgfsys@transformshift{3.415273in}{0.696025in}%
\pgfsys@useobject{currentmarker}{}%
\end{pgfscope}%
\begin{pgfscope}%
\pgfsys@transformshift{3.460364in}{0.696016in}%
\pgfsys@useobject{currentmarker}{}%
\end{pgfscope}%
\begin{pgfscope}%
\pgfsys@transformshift{3.505455in}{0.696016in}%
\pgfsys@useobject{currentmarker}{}%
\end{pgfscope}%
\begin{pgfscope}%
\pgfsys@transformshift{3.550545in}{0.696025in}%
\pgfsys@useobject{currentmarker}{}%
\end{pgfscope}%
\begin{pgfscope}%
\pgfsys@transformshift{3.595636in}{0.696025in}%
\pgfsys@useobject{currentmarker}{}%
\end{pgfscope}%
\begin{pgfscope}%
\pgfsys@transformshift{3.640727in}{0.696025in}%
\pgfsys@useobject{currentmarker}{}%
\end{pgfscope}%
\begin{pgfscope}%
\pgfsys@transformshift{3.685818in}{0.696025in}%
\pgfsys@useobject{currentmarker}{}%
\end{pgfscope}%
\begin{pgfscope}%
\pgfsys@transformshift{3.730909in}{0.696016in}%
\pgfsys@useobject{currentmarker}{}%
\end{pgfscope}%
\begin{pgfscope}%
\pgfsys@transformshift{3.776000in}{0.696016in}%
\pgfsys@useobject{currentmarker}{}%
\end{pgfscope}%
\begin{pgfscope}%
\pgfsys@transformshift{3.821091in}{0.696016in}%
\pgfsys@useobject{currentmarker}{}%
\end{pgfscope}%
\begin{pgfscope}%
\pgfsys@transformshift{3.866182in}{0.696025in}%
\pgfsys@useobject{currentmarker}{}%
\end{pgfscope}%
\begin{pgfscope}%
\pgfsys@transformshift{3.911273in}{0.696025in}%
\pgfsys@useobject{currentmarker}{}%
\end{pgfscope}%
\begin{pgfscope}%
\pgfsys@transformshift{3.956364in}{0.696016in}%
\pgfsys@useobject{currentmarker}{}%
\end{pgfscope}%
\begin{pgfscope}%
\pgfsys@transformshift{4.001455in}{0.696025in}%
\pgfsys@useobject{currentmarker}{}%
\end{pgfscope}%
\begin{pgfscope}%
\pgfsys@transformshift{4.046545in}{0.696025in}%
\pgfsys@useobject{currentmarker}{}%
\end{pgfscope}%
\begin{pgfscope}%
\pgfsys@transformshift{4.091636in}{0.696025in}%
\pgfsys@useobject{currentmarker}{}%
\end{pgfscope}%
\begin{pgfscope}%
\pgfsys@transformshift{4.136727in}{0.696025in}%
\pgfsys@useobject{currentmarker}{}%
\end{pgfscope}%
\begin{pgfscope}%
\pgfsys@transformshift{4.181818in}{0.696025in}%
\pgfsys@useobject{currentmarker}{}%
\end{pgfscope}%
\begin{pgfscope}%
\pgfsys@transformshift{4.226909in}{0.696025in}%
\pgfsys@useobject{currentmarker}{}%
\end{pgfscope}%
\begin{pgfscope}%
\pgfsys@transformshift{4.272000in}{0.696016in}%
\pgfsys@useobject{currentmarker}{}%
\end{pgfscope}%
\begin{pgfscope}%
\pgfsys@transformshift{4.317091in}{0.696025in}%
\pgfsys@useobject{currentmarker}{}%
\end{pgfscope}%
\begin{pgfscope}%
\pgfsys@transformshift{4.362182in}{0.696016in}%
\pgfsys@useobject{currentmarker}{}%
\end{pgfscope}%
\begin{pgfscope}%
\pgfsys@transformshift{4.407273in}{0.696025in}%
\pgfsys@useobject{currentmarker}{}%
\end{pgfscope}%
\begin{pgfscope}%
\pgfsys@transformshift{4.452364in}{0.696025in}%
\pgfsys@useobject{currentmarker}{}%
\end{pgfscope}%
\begin{pgfscope}%
\pgfsys@transformshift{4.497455in}{0.696025in}%
\pgfsys@useobject{currentmarker}{}%
\end{pgfscope}%
\begin{pgfscope}%
\pgfsys@transformshift{4.542545in}{0.696025in}%
\pgfsys@useobject{currentmarker}{}%
\end{pgfscope}%
\begin{pgfscope}%
\pgfsys@transformshift{4.587636in}{0.696016in}%
\pgfsys@useobject{currentmarker}{}%
\end{pgfscope}%
\begin{pgfscope}%
\pgfsys@transformshift{4.632727in}{0.696025in}%
\pgfsys@useobject{currentmarker}{}%
\end{pgfscope}%
\begin{pgfscope}%
\pgfsys@transformshift{4.677818in}{0.696016in}%
\pgfsys@useobject{currentmarker}{}%
\end{pgfscope}%
\begin{pgfscope}%
\pgfsys@transformshift{4.722909in}{0.696016in}%
\pgfsys@useobject{currentmarker}{}%
\end{pgfscope}%
\begin{pgfscope}%
\pgfsys@transformshift{4.768000in}{0.696016in}%
\pgfsys@useobject{currentmarker}{}%
\end{pgfscope}%
\begin{pgfscope}%
\pgfsys@transformshift{4.813091in}{0.696016in}%
\pgfsys@useobject{currentmarker}{}%
\end{pgfscope}%
\begin{pgfscope}%
\pgfsys@transformshift{4.858182in}{0.696016in}%
\pgfsys@useobject{currentmarker}{}%
\end{pgfscope}%
\begin{pgfscope}%
\pgfsys@transformshift{4.903273in}{0.696016in}%
\pgfsys@useobject{currentmarker}{}%
\end{pgfscope}%
\begin{pgfscope}%
\pgfsys@transformshift{4.948364in}{0.696016in}%
\pgfsys@useobject{currentmarker}{}%
\end{pgfscope}%
\begin{pgfscope}%
\pgfsys@transformshift{4.993455in}{0.696016in}%
\pgfsys@useobject{currentmarker}{}%
\end{pgfscope}%
\begin{pgfscope}%
\pgfsys@transformshift{5.038545in}{0.696025in}%
\pgfsys@useobject{currentmarker}{}%
\end{pgfscope}%
\begin{pgfscope}%
\pgfsys@transformshift{5.083636in}{0.696025in}%
\pgfsys@useobject{currentmarker}{}%
\end{pgfscope}%
\begin{pgfscope}%
\pgfsys@transformshift{5.128727in}{0.696025in}%
\pgfsys@useobject{currentmarker}{}%
\end{pgfscope}%
\begin{pgfscope}%
\pgfsys@transformshift{5.173818in}{0.696025in}%
\pgfsys@useobject{currentmarker}{}%
\end{pgfscope}%
\begin{pgfscope}%
\pgfsys@transformshift{5.218909in}{0.696016in}%
\pgfsys@useobject{currentmarker}{}%
\end{pgfscope}%
\begin{pgfscope}%
\pgfsys@transformshift{5.264000in}{0.696025in}%
\pgfsys@useobject{currentmarker}{}%
\end{pgfscope}%
\begin{pgfscope}%
\pgfsys@transformshift{5.309091in}{0.696006in}%
\pgfsys@useobject{currentmarker}{}%
\end{pgfscope}%
\begin{pgfscope}%
\pgfsys@transformshift{5.354182in}{0.696016in}%
\pgfsys@useobject{currentmarker}{}%
\end{pgfscope}%
\begin{pgfscope}%
\pgfsys@transformshift{5.399273in}{0.696025in}%
\pgfsys@useobject{currentmarker}{}%
\end{pgfscope}%
\begin{pgfscope}%
\pgfsys@transformshift{5.444364in}{0.696016in}%
\pgfsys@useobject{currentmarker}{}%
\end{pgfscope}%
\begin{pgfscope}%
\pgfsys@transformshift{5.489455in}{0.696016in}%
\pgfsys@useobject{currentmarker}{}%
\end{pgfscope}%
\begin{pgfscope}%
\pgfsys@transformshift{5.534545in}{0.696025in}%
\pgfsys@useobject{currentmarker}{}%
\end{pgfscope}%
\end{pgfscope}%
\begin{pgfscope}%
\pgfsetrectcap%
\pgfsetmiterjoin%
\pgfsetlinewidth{0.803000pt}%
\definecolor{currentstroke}{rgb}{0.000000,0.000000,0.000000}%
\pgfsetstrokecolor{currentstroke}%
\pgfsetdash{}{0pt}%
\pgfpathmoveto{\pgfqpoint{0.800000in}{0.528008in}}%
\pgfpathlineto{\pgfqpoint{0.800000in}{4.224004in}}%
\pgfusepath{stroke}%
\end{pgfscope}%
\begin{pgfscope}%
\pgfsetrectcap%
\pgfsetmiterjoin%
\pgfsetlinewidth{0.803000pt}%
\definecolor{currentstroke}{rgb}{0.000000,0.000000,0.000000}%
\pgfsetstrokecolor{currentstroke}%
\pgfsetdash{}{0pt}%
\pgfpathmoveto{\pgfqpoint{5.760000in}{0.528008in}}%
\pgfpathlineto{\pgfqpoint{5.760000in}{4.224004in}}%
\pgfusepath{stroke}%
\end{pgfscope}%
\begin{pgfscope}%
\pgfsetrectcap%
\pgfsetmiterjoin%
\pgfsetlinewidth{0.803000pt}%
\definecolor{currentstroke}{rgb}{0.000000,0.000000,0.000000}%
\pgfsetstrokecolor{currentstroke}%
\pgfsetdash{}{0pt}%
\pgfpathmoveto{\pgfqpoint{0.800000in}{0.528000in}}%
\pgfpathlineto{\pgfqpoint{5.760000in}{0.528000in}}%
\pgfusepath{stroke}%
\end{pgfscope}%
\begin{pgfscope}%
\pgfsetrectcap%
\pgfsetmiterjoin%
\pgfsetlinewidth{0.803000pt}%
\definecolor{currentstroke}{rgb}{0.000000,0.000000,0.000000}%
\pgfsetstrokecolor{currentstroke}%
\pgfsetdash{}{0pt}%
\pgfpathmoveto{\pgfqpoint{0.800000in}{4.224000in}}%
\pgfpathlineto{\pgfqpoint{5.760000in}{4.224000in}}%
\pgfusepath{stroke}%
\end{pgfscope}%
\end{pgfpicture}%
\makeatother%
\endgroup%

%% \caption{The lower bound in \cref{eq:GMRESprob} with $\NLiDR{\no-\nt} \sim \Exp{\sigma}$ with $\sigma = 1/k.$\label{fig:prob-theory-plot-1.0}}
%% \end{subfigure}

%% \begin{subfigure}{\textwidth}
%%     \centering
%%     %% Creator: Matplotlib, PGF backend
%%
%% To include the figure in your LaTeX document, write
%%   \input{<filename>.pgf}
%%
%% Make sure the required packages are loaded in your preamble
%%   \usepackage{pgf}
%%
%% Figures using additional raster images can only be included by \input if
%% they are in the same directory as the main LaTeX file. For loading figures
%% from other directories you can use the `import` package
%%   \usepackage{import}
%% and then include the figures with
%%   \import{<path to file>}{<filename>.pgf}
%%
%% Matplotlib used the following preamble
%%   \usepackage{fontspec}
%%   \setmainfont{DejaVuSerif.ttf}[Path=/home/owen/progs/firedrake-complex/firedrake/lib/python3.5/site-packages/matplotlib/mpl-data/fonts/ttf/]
%%   \setsansfont{DejaVuSans.ttf}[Path=/home/owen/progs/firedrake-complex/firedrake/lib/python3.5/site-packages/matplotlib/mpl-data/fonts/ttf/]
%%   \setmonofont{DejaVuSansMono.ttf}[Path=/home/owen/progs/firedrake-complex/firedrake/lib/python3.5/site-packages/matplotlib/mpl-data/fonts/ttf/]
%%
\begingroup%
\makeatletter%
\begin{pgfpicture}%
\pgfpathrectangle{\pgfpointorigin}{\pgfqpoint{6.000000in}{2.500000in}}%
\pgfusepath{use as bounding box, clip}%
\begin{pgfscope}%
\pgfsetbuttcap%
\pgfsetmiterjoin%
\definecolor{currentfill}{rgb}{1.000000,1.000000,1.000000}%
\pgfsetfillcolor{currentfill}%
\pgfsetlinewidth{0.000000pt}%
\definecolor{currentstroke}{rgb}{1.000000,1.000000,1.000000}%
\pgfsetstrokecolor{currentstroke}%
\pgfsetdash{}{0pt}%
\pgfpathmoveto{\pgfqpoint{0.000000in}{0.000000in}}%
\pgfpathlineto{\pgfqpoint{6.000000in}{0.000000in}}%
\pgfpathlineto{\pgfqpoint{6.000000in}{2.500000in}}%
\pgfpathlineto{\pgfqpoint{0.000000in}{2.500000in}}%
\pgfpathclose%
\pgfusepath{fill}%
\end{pgfscope}%
\begin{pgfscope}%
\pgfsetbuttcap%
\pgfsetmiterjoin%
\definecolor{currentfill}{rgb}{1.000000,1.000000,1.000000}%
\pgfsetfillcolor{currentfill}%
\pgfsetlinewidth{0.000000pt}%
\definecolor{currentstroke}{rgb}{0.000000,0.000000,0.000000}%
\pgfsetstrokecolor{currentstroke}%
\pgfsetstrokeopacity{0.000000}%
\pgfsetdash{}{0pt}%
\pgfpathmoveto{\pgfqpoint{0.750000in}{0.275000in}}%
\pgfpathlineto{\pgfqpoint{5.400000in}{0.275000in}}%
\pgfpathlineto{\pgfqpoint{5.400000in}{2.200000in}}%
\pgfpathlineto{\pgfqpoint{0.750000in}{2.200000in}}%
\pgfpathclose%
\pgfusepath{fill}%
\end{pgfscope}%
\begin{pgfscope}%
\pgfsetbuttcap%
\pgfsetroundjoin%
\definecolor{currentfill}{rgb}{0.000000,0.000000,0.000000}%
\pgfsetfillcolor{currentfill}%
\pgfsetlinewidth{0.803000pt}%
\definecolor{currentstroke}{rgb}{0.000000,0.000000,0.000000}%
\pgfsetstrokecolor{currentstroke}%
\pgfsetdash{}{0pt}%
\pgfsys@defobject{currentmarker}{\pgfqpoint{0.000000in}{-0.048611in}}{\pgfqpoint{0.000000in}{0.000000in}}{%
\pgfpathmoveto{\pgfqpoint{0.000000in}{0.000000in}}%
\pgfpathlineto{\pgfqpoint{0.000000in}{-0.048611in}}%
\pgfusepath{stroke,fill}%
}%
\begin{pgfscope}%
\pgfsys@transformshift{0.961364in}{0.275000in}%
\pgfsys@useobject{currentmarker}{}%
\end{pgfscope}%
\end{pgfscope}%
\begin{pgfscope}%
\definecolor{textcolor}{rgb}{0.000000,0.000000,0.000000}%
\pgfsetstrokecolor{textcolor}%
\pgfsetfillcolor{textcolor}%
\pgftext[x=0.961364in,y=0.177778in,,top]{\color{textcolor}\sffamily\fontsize{10.000000}{12.000000}\selectfont \(\displaystyle 10\)}%
\end{pgfscope}%
\begin{pgfscope}%
\pgfsetbuttcap%
\pgfsetroundjoin%
\definecolor{currentfill}{rgb}{0.000000,0.000000,0.000000}%
\pgfsetfillcolor{currentfill}%
\pgfsetlinewidth{0.803000pt}%
\definecolor{currentstroke}{rgb}{0.000000,0.000000,0.000000}%
\pgfsetstrokecolor{currentstroke}%
\pgfsetdash{}{0pt}%
\pgfsys@defobject{currentmarker}{\pgfqpoint{0.000000in}{-0.048611in}}{\pgfqpoint{0.000000in}{0.000000in}}{%
\pgfpathmoveto{\pgfqpoint{0.000000in}{0.000000in}}%
\pgfpathlineto{\pgfqpoint{0.000000in}{-0.048611in}}%
\pgfusepath{stroke,fill}%
}%
\begin{pgfscope}%
\pgfsys@transformshift{1.665909in}{0.275000in}%
\pgfsys@useobject{currentmarker}{}%
\end{pgfscope}%
\end{pgfscope}%
\begin{pgfscope}%
\definecolor{textcolor}{rgb}{0.000000,0.000000,0.000000}%
\pgfsetstrokecolor{textcolor}%
\pgfsetfillcolor{textcolor}%
\pgftext[x=1.665909in,y=0.177778in,,top]{\color{textcolor}\sffamily\fontsize{10.000000}{12.000000}\selectfont \(\displaystyle 15\)}%
\end{pgfscope}%
\begin{pgfscope}%
\pgfsetbuttcap%
\pgfsetroundjoin%
\definecolor{currentfill}{rgb}{0.000000,0.000000,0.000000}%
\pgfsetfillcolor{currentfill}%
\pgfsetlinewidth{0.803000pt}%
\definecolor{currentstroke}{rgb}{0.000000,0.000000,0.000000}%
\pgfsetstrokecolor{currentstroke}%
\pgfsetdash{}{0pt}%
\pgfsys@defobject{currentmarker}{\pgfqpoint{0.000000in}{-0.048611in}}{\pgfqpoint{0.000000in}{0.000000in}}{%
\pgfpathmoveto{\pgfqpoint{0.000000in}{0.000000in}}%
\pgfpathlineto{\pgfqpoint{0.000000in}{-0.048611in}}%
\pgfusepath{stroke,fill}%
}%
\begin{pgfscope}%
\pgfsys@transformshift{2.370455in}{0.275000in}%
\pgfsys@useobject{currentmarker}{}%
\end{pgfscope}%
\end{pgfscope}%
\begin{pgfscope}%
\definecolor{textcolor}{rgb}{0.000000,0.000000,0.000000}%
\pgfsetstrokecolor{textcolor}%
\pgfsetfillcolor{textcolor}%
\pgftext[x=2.370455in,y=0.177778in,,top]{\color{textcolor}\sffamily\fontsize{10.000000}{12.000000}\selectfont \(\displaystyle 20\)}%
\end{pgfscope}%
\begin{pgfscope}%
\pgfsetbuttcap%
\pgfsetroundjoin%
\definecolor{currentfill}{rgb}{0.000000,0.000000,0.000000}%
\pgfsetfillcolor{currentfill}%
\pgfsetlinewidth{0.803000pt}%
\definecolor{currentstroke}{rgb}{0.000000,0.000000,0.000000}%
\pgfsetstrokecolor{currentstroke}%
\pgfsetdash{}{0pt}%
\pgfsys@defobject{currentmarker}{\pgfqpoint{0.000000in}{-0.048611in}}{\pgfqpoint{0.000000in}{0.000000in}}{%
\pgfpathmoveto{\pgfqpoint{0.000000in}{0.000000in}}%
\pgfpathlineto{\pgfqpoint{0.000000in}{-0.048611in}}%
\pgfusepath{stroke,fill}%
}%
\begin{pgfscope}%
\pgfsys@transformshift{3.075000in}{0.275000in}%
\pgfsys@useobject{currentmarker}{}%
\end{pgfscope}%
\end{pgfscope}%
\begin{pgfscope}%
\definecolor{textcolor}{rgb}{0.000000,0.000000,0.000000}%
\pgfsetstrokecolor{textcolor}%
\pgfsetfillcolor{textcolor}%
\pgftext[x=3.075000in,y=0.177778in,,top]{\color{textcolor}\sffamily\fontsize{10.000000}{12.000000}\selectfont \(\displaystyle 25\)}%
\end{pgfscope}%
\begin{pgfscope}%
\pgfsetbuttcap%
\pgfsetroundjoin%
\definecolor{currentfill}{rgb}{0.000000,0.000000,0.000000}%
\pgfsetfillcolor{currentfill}%
\pgfsetlinewidth{0.803000pt}%
\definecolor{currentstroke}{rgb}{0.000000,0.000000,0.000000}%
\pgfsetstrokecolor{currentstroke}%
\pgfsetdash{}{0pt}%
\pgfsys@defobject{currentmarker}{\pgfqpoint{0.000000in}{-0.048611in}}{\pgfqpoint{0.000000in}{0.000000in}}{%
\pgfpathmoveto{\pgfqpoint{0.000000in}{0.000000in}}%
\pgfpathlineto{\pgfqpoint{0.000000in}{-0.048611in}}%
\pgfusepath{stroke,fill}%
}%
\begin{pgfscope}%
\pgfsys@transformshift{3.779545in}{0.275000in}%
\pgfsys@useobject{currentmarker}{}%
\end{pgfscope}%
\end{pgfscope}%
\begin{pgfscope}%
\definecolor{textcolor}{rgb}{0.000000,0.000000,0.000000}%
\pgfsetstrokecolor{textcolor}%
\pgfsetfillcolor{textcolor}%
\pgftext[x=3.779545in,y=0.177778in,,top]{\color{textcolor}\sffamily\fontsize{10.000000}{12.000000}\selectfont \(\displaystyle 30\)}%
\end{pgfscope}%
\begin{pgfscope}%
\pgfsetbuttcap%
\pgfsetroundjoin%
\definecolor{currentfill}{rgb}{0.000000,0.000000,0.000000}%
\pgfsetfillcolor{currentfill}%
\pgfsetlinewidth{0.803000pt}%
\definecolor{currentstroke}{rgb}{0.000000,0.000000,0.000000}%
\pgfsetstrokecolor{currentstroke}%
\pgfsetdash{}{0pt}%
\pgfsys@defobject{currentmarker}{\pgfqpoint{0.000000in}{-0.048611in}}{\pgfqpoint{0.000000in}{0.000000in}}{%
\pgfpathmoveto{\pgfqpoint{0.000000in}{0.000000in}}%
\pgfpathlineto{\pgfqpoint{0.000000in}{-0.048611in}}%
\pgfusepath{stroke,fill}%
}%
\begin{pgfscope}%
\pgfsys@transformshift{4.484091in}{0.275000in}%
\pgfsys@useobject{currentmarker}{}%
\end{pgfscope}%
\end{pgfscope}%
\begin{pgfscope}%
\definecolor{textcolor}{rgb}{0.000000,0.000000,0.000000}%
\pgfsetstrokecolor{textcolor}%
\pgfsetfillcolor{textcolor}%
\pgftext[x=4.484091in,y=0.177778in,,top]{\color{textcolor}\sffamily\fontsize{10.000000}{12.000000}\selectfont \(\displaystyle 35\)}%
\end{pgfscope}%
\begin{pgfscope}%
\pgfsetbuttcap%
\pgfsetroundjoin%
\definecolor{currentfill}{rgb}{0.000000,0.000000,0.000000}%
\pgfsetfillcolor{currentfill}%
\pgfsetlinewidth{0.803000pt}%
\definecolor{currentstroke}{rgb}{0.000000,0.000000,0.000000}%
\pgfsetstrokecolor{currentstroke}%
\pgfsetdash{}{0pt}%
\pgfsys@defobject{currentmarker}{\pgfqpoint{0.000000in}{-0.048611in}}{\pgfqpoint{0.000000in}{0.000000in}}{%
\pgfpathmoveto{\pgfqpoint{0.000000in}{0.000000in}}%
\pgfpathlineto{\pgfqpoint{0.000000in}{-0.048611in}}%
\pgfusepath{stroke,fill}%
}%
\begin{pgfscope}%
\pgfsys@transformshift{5.188636in}{0.275000in}%
\pgfsys@useobject{currentmarker}{}%
\end{pgfscope}%
\end{pgfscope}%
\begin{pgfscope}%
\definecolor{textcolor}{rgb}{0.000000,0.000000,0.000000}%
\pgfsetstrokecolor{textcolor}%
\pgfsetfillcolor{textcolor}%
\pgftext[x=5.188636in,y=0.177778in,,top]{\color{textcolor}\sffamily\fontsize{10.000000}{12.000000}\selectfont \(\displaystyle 40\)}%
\end{pgfscope}%
\begin{pgfscope}%
\definecolor{textcolor}{rgb}{0.000000,0.000000,0.000000}%
\pgfsetstrokecolor{textcolor}%
\pgfsetfillcolor{textcolor}%
\pgftext[x=3.075000in,y=-0.012191in,,top]{\color{textcolor}\sffamily\fontsize{10.000000}{12.000000}\selectfont \(\displaystyle k\)}%
\end{pgfscope}%
\begin{pgfscope}%
\pgfsetbuttcap%
\pgfsetroundjoin%
\definecolor{currentfill}{rgb}{0.000000,0.000000,0.000000}%
\pgfsetfillcolor{currentfill}%
\pgfsetlinewidth{0.803000pt}%
\definecolor{currentstroke}{rgb}{0.000000,0.000000,0.000000}%
\pgfsetstrokecolor{currentstroke}%
\pgfsetdash{}{0pt}%
\pgfsys@defobject{currentmarker}{\pgfqpoint{-0.048611in}{0.000000in}}{\pgfqpoint{0.000000in}{0.000000in}}{%
\pgfpathmoveto{\pgfqpoint{0.000000in}{0.000000in}}%
\pgfpathlineto{\pgfqpoint{-0.048611in}{0.000000in}}%
\pgfusepath{stroke,fill}%
}%
\begin{pgfscope}%
\pgfsys@transformshift{0.750000in}{0.298219in}%
\pgfsys@useobject{currentmarker}{}%
\end{pgfscope}%
\end{pgfscope}%
\begin{pgfscope}%
\definecolor{textcolor}{rgb}{0.000000,0.000000,0.000000}%
\pgfsetstrokecolor{textcolor}%
\pgfsetfillcolor{textcolor}%
\pgftext[x=0.405863in,y=0.245457in,left,base]{\color{textcolor}\sffamily\fontsize{10.000000}{12.000000}\selectfont \(\displaystyle 0.97\)}%
\end{pgfscope}%
\begin{pgfscope}%
\pgfsetbuttcap%
\pgfsetroundjoin%
\definecolor{currentfill}{rgb}{0.000000,0.000000,0.000000}%
\pgfsetfillcolor{currentfill}%
\pgfsetlinewidth{0.803000pt}%
\definecolor{currentstroke}{rgb}{0.000000,0.000000,0.000000}%
\pgfsetstrokecolor{currentstroke}%
\pgfsetdash{}{0pt}%
\pgfsys@defobject{currentmarker}{\pgfqpoint{-0.048611in}{0.000000in}}{\pgfqpoint{0.000000in}{0.000000in}}{%
\pgfpathmoveto{\pgfqpoint{0.000000in}{0.000000in}}%
\pgfpathlineto{\pgfqpoint{-0.048611in}{0.000000in}}%
\pgfusepath{stroke,fill}%
}%
\begin{pgfscope}%
\pgfsys@transformshift{0.750000in}{0.902993in}%
\pgfsys@useobject{currentmarker}{}%
\end{pgfscope}%
\end{pgfscope}%
\begin{pgfscope}%
\definecolor{textcolor}{rgb}{0.000000,0.000000,0.000000}%
\pgfsetstrokecolor{textcolor}%
\pgfsetfillcolor{textcolor}%
\pgftext[x=0.405863in,y=0.850232in,left,base]{\color{textcolor}\sffamily\fontsize{10.000000}{12.000000}\selectfont \(\displaystyle 0.98\)}%
\end{pgfscope}%
\begin{pgfscope}%
\pgfsetbuttcap%
\pgfsetroundjoin%
\definecolor{currentfill}{rgb}{0.000000,0.000000,0.000000}%
\pgfsetfillcolor{currentfill}%
\pgfsetlinewidth{0.803000pt}%
\definecolor{currentstroke}{rgb}{0.000000,0.000000,0.000000}%
\pgfsetstrokecolor{currentstroke}%
\pgfsetdash{}{0pt}%
\pgfsys@defobject{currentmarker}{\pgfqpoint{-0.048611in}{0.000000in}}{\pgfqpoint{0.000000in}{0.000000in}}{%
\pgfpathmoveto{\pgfqpoint{0.000000in}{0.000000in}}%
\pgfpathlineto{\pgfqpoint{-0.048611in}{0.000000in}}%
\pgfusepath{stroke,fill}%
}%
\begin{pgfscope}%
\pgfsys@transformshift{0.750000in}{1.507768in}%
\pgfsys@useobject{currentmarker}{}%
\end{pgfscope}%
\end{pgfscope}%
\begin{pgfscope}%
\definecolor{textcolor}{rgb}{0.000000,0.000000,0.000000}%
\pgfsetstrokecolor{textcolor}%
\pgfsetfillcolor{textcolor}%
\pgftext[x=0.405863in,y=1.455006in,left,base]{\color{textcolor}\sffamily\fontsize{10.000000}{12.000000}\selectfont \(\displaystyle 0.99\)}%
\end{pgfscope}%
\begin{pgfscope}%
\pgfsetbuttcap%
\pgfsetroundjoin%
\definecolor{currentfill}{rgb}{0.000000,0.000000,0.000000}%
\pgfsetfillcolor{currentfill}%
\pgfsetlinewidth{0.803000pt}%
\definecolor{currentstroke}{rgb}{0.000000,0.000000,0.000000}%
\pgfsetstrokecolor{currentstroke}%
\pgfsetdash{}{0pt}%
\pgfsys@defobject{currentmarker}{\pgfqpoint{-0.048611in}{0.000000in}}{\pgfqpoint{0.000000in}{0.000000in}}{%
\pgfpathmoveto{\pgfqpoint{0.000000in}{0.000000in}}%
\pgfpathlineto{\pgfqpoint{-0.048611in}{0.000000in}}%
\pgfusepath{stroke,fill}%
}%
\begin{pgfscope}%
\pgfsys@transformshift{0.750000in}{2.112542in}%
\pgfsys@useobject{currentmarker}{}%
\end{pgfscope}%
\end{pgfscope}%
\begin{pgfscope}%
\definecolor{textcolor}{rgb}{0.000000,0.000000,0.000000}%
\pgfsetstrokecolor{textcolor}%
\pgfsetfillcolor{textcolor}%
\pgftext[x=0.405863in,y=2.059781in,left,base]{\color{textcolor}\sffamily\fontsize{10.000000}{12.000000}\selectfont \(\displaystyle 1.00\)}%
\end{pgfscope}%
\begin{pgfscope}%
\definecolor{textcolor}{rgb}{0.000000,0.000000,0.000000}%
\pgfsetstrokecolor{textcolor}%
\pgfsetfillcolor{textcolor}%
\pgftext[x=0.165901in,y=0.325194in,left,base,rotate=90.000000]{\color{textcolor}\sffamily\fontsize{10.000000}{12.000000}\selectfont Probability that number of}%
\end{pgfscope}%
\begin{pgfscope}%
\definecolor{textcolor}{rgb}{0.000000,0.000000,0.000000}%
\pgfsetstrokecolor{textcolor}%
\pgfsetfillcolor{textcolor}%
\pgftext[x=0.321418in,y=0.160467in,left,base,rotate=90.000000]{\color{textcolor}\sffamily\fontsize{10.000000}{12.000000}\selectfont GMRES iterations is at most 12}%
\end{pgfscope}%
\begin{pgfscope}%
\pgfpathrectangle{\pgfqpoint{0.750000in}{0.275000in}}{\pgfqpoint{4.650000in}{1.925000in}}%
\pgfusepath{clip}%
\pgfsetbuttcap%
\pgfsetroundjoin%
\definecolor{currentfill}{rgb}{0.000000,0.000000,0.000000}%
\pgfsetfillcolor{currentfill}%
\pgfsetlinewidth{1.003750pt}%
\definecolor{currentstroke}{rgb}{0.000000,0.000000,0.000000}%
\pgfsetstrokecolor{currentstroke}%
\pgfsetdash{}{0pt}%
\pgfsys@defobject{currentmarker}{\pgfqpoint{-0.020833in}{-0.020833in}}{\pgfqpoint{0.020833in}{0.020833in}}{%
\pgfpathmoveto{\pgfqpoint{0.000000in}{-0.020833in}}%
\pgfpathcurveto{\pgfqpoint{0.005525in}{-0.020833in}}{\pgfqpoint{0.010825in}{-0.018638in}}{\pgfqpoint{0.014731in}{-0.014731in}}%
\pgfpathcurveto{\pgfqpoint{0.018638in}{-0.010825in}}{\pgfqpoint{0.020833in}{-0.005525in}}{\pgfqpoint{0.020833in}{0.000000in}}%
\pgfpathcurveto{\pgfqpoint{0.020833in}{0.005525in}}{\pgfqpoint{0.018638in}{0.010825in}}{\pgfqpoint{0.014731in}{0.014731in}}%
\pgfpathcurveto{\pgfqpoint{0.010825in}{0.018638in}}{\pgfqpoint{0.005525in}{0.020833in}}{\pgfqpoint{0.000000in}{0.020833in}}%
\pgfpathcurveto{\pgfqpoint{-0.005525in}{0.020833in}}{\pgfqpoint{-0.010825in}{0.018638in}}{\pgfqpoint{-0.014731in}{0.014731in}}%
\pgfpathcurveto{\pgfqpoint{-0.018638in}{0.010825in}}{\pgfqpoint{-0.020833in}{0.005525in}}{\pgfqpoint{-0.020833in}{0.000000in}}%
\pgfpathcurveto{\pgfqpoint{-0.020833in}{-0.005525in}}{\pgfqpoint{-0.018638in}{-0.010825in}}{\pgfqpoint{-0.014731in}{-0.014731in}}%
\pgfpathcurveto{\pgfqpoint{-0.010825in}{-0.018638in}}{\pgfqpoint{-0.005525in}{-0.020833in}}{\pgfqpoint{0.000000in}{-0.020833in}}%
\pgfpathclose%
\pgfusepath{stroke,fill}%
}%
\begin{pgfscope}%
\pgfsys@transformshift{0.961364in}{0.362500in}%
\pgfsys@useobject{currentmarker}{}%
\end{pgfscope}%
\begin{pgfscope}%
\pgfsys@transformshift{1.003636in}{0.538950in}%
\pgfsys@useobject{currentmarker}{}%
\end{pgfscope}%
\begin{pgfscope}%
\pgfsys@transformshift{1.045909in}{0.697609in}%
\pgfsys@useobject{currentmarker}{}%
\end{pgfscope}%
\begin{pgfscope}%
\pgfsys@transformshift{1.088182in}{0.840272in}%
\pgfsys@useobject{currentmarker}{}%
\end{pgfscope}%
\begin{pgfscope}%
\pgfsys@transformshift{1.130455in}{0.968550in}%
\pgfsys@useobject{currentmarker}{}%
\end{pgfscope}%
\begin{pgfscope}%
\pgfsys@transformshift{1.172727in}{1.083894in}%
\pgfsys@useobject{currentmarker}{}%
\end{pgfscope}%
\begin{pgfscope}%
\pgfsys@transformshift{1.215000in}{1.187609in}%
\pgfsys@useobject{currentmarker}{}%
\end{pgfscope}%
\begin{pgfscope}%
\pgfsys@transformshift{1.257273in}{1.280866in}%
\pgfsys@useobject{currentmarker}{}%
\end{pgfscope}%
\begin{pgfscope}%
\pgfsys@transformshift{1.299545in}{1.364721in}%
\pgfsys@useobject{currentmarker}{}%
\end{pgfscope}%
\begin{pgfscope}%
\pgfsys@transformshift{1.341818in}{1.440121in}%
\pgfsys@useobject{currentmarker}{}%
\end{pgfscope}%
\begin{pgfscope}%
\pgfsys@transformshift{1.384091in}{1.507919in}%
\pgfsys@useobject{currentmarker}{}%
\end{pgfscope}%
\begin{pgfscope}%
\pgfsys@transformshift{1.426364in}{1.568881in}%
\pgfsys@useobject{currentmarker}{}%
\end{pgfscope}%
\begin{pgfscope}%
\pgfsys@transformshift{1.468636in}{1.623696in}%
\pgfsys@useobject{currentmarker}{}%
\end{pgfscope}%
\begin{pgfscope}%
\pgfsys@transformshift{1.510909in}{1.672985in}%
\pgfsys@useobject{currentmarker}{}%
\end{pgfscope}%
\begin{pgfscope}%
\pgfsys@transformshift{1.553182in}{1.717303in}%
\pgfsys@useobject{currentmarker}{}%
\end{pgfscope}%
\begin{pgfscope}%
\pgfsys@transformshift{1.595455in}{1.757154in}%
\pgfsys@useobject{currentmarker}{}%
\end{pgfscope}%
\begin{pgfscope}%
\pgfsys@transformshift{1.637727in}{1.792986in}%
\pgfsys@useobject{currentmarker}{}%
\end{pgfscope}%
\begin{pgfscope}%
\pgfsys@transformshift{1.680000in}{1.825206in}%
\pgfsys@useobject{currentmarker}{}%
\end{pgfscope}%
\begin{pgfscope}%
\pgfsys@transformshift{1.722273in}{1.854177in}%
\pgfsys@useobject{currentmarker}{}%
\end{pgfscope}%
\begin{pgfscope}%
\pgfsys@transformshift{1.764545in}{1.880227in}%
\pgfsys@useobject{currentmarker}{}%
\end{pgfscope}%
\begin{pgfscope}%
\pgfsys@transformshift{1.806818in}{1.903651in}%
\pgfsys@useobject{currentmarker}{}%
\end{pgfscope}%
\begin{pgfscope}%
\pgfsys@transformshift{1.849091in}{1.924712in}%
\pgfsys@useobject{currentmarker}{}%
\end{pgfscope}%
\begin{pgfscope}%
\pgfsys@transformshift{1.891364in}{1.943650in}%
\pgfsys@useobject{currentmarker}{}%
\end{pgfscope}%
\begin{pgfscope}%
\pgfsys@transformshift{1.933636in}{1.960679in}%
\pgfsys@useobject{currentmarker}{}%
\end{pgfscope}%
\begin{pgfscope}%
\pgfsys@transformshift{1.975909in}{1.975991in}%
\pgfsys@useobject{currentmarker}{}%
\end{pgfscope}%
\begin{pgfscope}%
\pgfsys@transformshift{2.018182in}{1.989759in}%
\pgfsys@useobject{currentmarker}{}%
\end{pgfscope}%
\begin{pgfscope}%
\pgfsys@transformshift{2.060455in}{2.002139in}%
\pgfsys@useobject{currentmarker}{}%
\end{pgfscope}%
\begin{pgfscope}%
\pgfsys@transformshift{2.102727in}{2.013270in}%
\pgfsys@useobject{currentmarker}{}%
\end{pgfscope}%
\begin{pgfscope}%
\pgfsys@transformshift{2.145000in}{2.023280in}%
\pgfsys@useobject{currentmarker}{}%
\end{pgfscope}%
\begin{pgfscope}%
\pgfsys@transformshift{2.187273in}{2.032280in}%
\pgfsys@useobject{currentmarker}{}%
\end{pgfscope}%
\begin{pgfscope}%
\pgfsys@transformshift{2.229545in}{2.040372in}%
\pgfsys@useobject{currentmarker}{}%
\end{pgfscope}%
\begin{pgfscope}%
\pgfsys@transformshift{2.271818in}{2.047649in}%
\pgfsys@useobject{currentmarker}{}%
\end{pgfscope}%
\begin{pgfscope}%
\pgfsys@transformshift{2.314091in}{2.054192in}%
\pgfsys@useobject{currentmarker}{}%
\end{pgfscope}%
\begin{pgfscope}%
\pgfsys@transformshift{2.356364in}{2.060075in}%
\pgfsys@useobject{currentmarker}{}%
\end{pgfscope}%
\begin{pgfscope}%
\pgfsys@transformshift{2.398636in}{2.065365in}%
\pgfsys@useobject{currentmarker}{}%
\end{pgfscope}%
\begin{pgfscope}%
\pgfsys@transformshift{2.440909in}{2.070122in}%
\pgfsys@useobject{currentmarker}{}%
\end{pgfscope}%
\begin{pgfscope}%
\pgfsys@transformshift{2.483182in}{2.074399in}%
\pgfsys@useobject{currentmarker}{}%
\end{pgfscope}%
\begin{pgfscope}%
\pgfsys@transformshift{2.525455in}{2.078245in}%
\pgfsys@useobject{currentmarker}{}%
\end{pgfscope}%
\begin{pgfscope}%
\pgfsys@transformshift{2.567727in}{2.081703in}%
\pgfsys@useobject{currentmarker}{}%
\end{pgfscope}%
\begin{pgfscope}%
\pgfsys@transformshift{2.610000in}{2.084812in}%
\pgfsys@useobject{currentmarker}{}%
\end{pgfscope}%
\begin{pgfscope}%
\pgfsys@transformshift{2.652273in}{2.087608in}%
\pgfsys@useobject{currentmarker}{}%
\end{pgfscope}%
\begin{pgfscope}%
\pgfsys@transformshift{2.694545in}{2.090122in}%
\pgfsys@useobject{currentmarker}{}%
\end{pgfscope}%
\begin{pgfscope}%
\pgfsys@transformshift{2.736818in}{2.092383in}%
\pgfsys@useobject{currentmarker}{}%
\end{pgfscope}%
\begin{pgfscope}%
\pgfsys@transformshift{2.779091in}{2.094415in}%
\pgfsys@useobject{currentmarker}{}%
\end{pgfscope}%
\begin{pgfscope}%
\pgfsys@transformshift{2.821364in}{2.096243in}%
\pgfsys@useobject{currentmarker}{}%
\end{pgfscope}%
\begin{pgfscope}%
\pgfsys@transformshift{2.863636in}{2.097886in}%
\pgfsys@useobject{currentmarker}{}%
\end{pgfscope}%
\begin{pgfscope}%
\pgfsys@transformshift{2.905909in}{2.099364in}%
\pgfsys@useobject{currentmarker}{}%
\end{pgfscope}%
\begin{pgfscope}%
\pgfsys@transformshift{2.948182in}{2.100693in}%
\pgfsys@useobject{currentmarker}{}%
\end{pgfscope}%
\begin{pgfscope}%
\pgfsys@transformshift{2.990455in}{2.101888in}%
\pgfsys@useobject{currentmarker}{}%
\end{pgfscope}%
\begin{pgfscope}%
\pgfsys@transformshift{3.032727in}{2.102962in}%
\pgfsys@useobject{currentmarker}{}%
\end{pgfscope}%
\begin{pgfscope}%
\pgfsys@transformshift{3.075000in}{2.103928in}%
\pgfsys@useobject{currentmarker}{}%
\end{pgfscope}%
\begin{pgfscope}%
\pgfsys@transformshift{3.117273in}{2.104796in}%
\pgfsys@useobject{currentmarker}{}%
\end{pgfscope}%
\begin{pgfscope}%
\pgfsys@transformshift{3.159545in}{2.105577in}%
\pgfsys@useobject{currentmarker}{}%
\end{pgfscope}%
\begin{pgfscope}%
\pgfsys@transformshift{3.201818in}{2.106280in}%
\pgfsys@useobject{currentmarker}{}%
\end{pgfscope}%
\begin{pgfscope}%
\pgfsys@transformshift{3.244091in}{2.106911in}%
\pgfsys@useobject{currentmarker}{}%
\end{pgfscope}%
\begin{pgfscope}%
\pgfsys@transformshift{3.286364in}{2.107479in}%
\pgfsys@useobject{currentmarker}{}%
\end{pgfscope}%
\begin{pgfscope}%
\pgfsys@transformshift{3.328636in}{2.107989in}%
\pgfsys@useobject{currentmarker}{}%
\end{pgfscope}%
\begin{pgfscope}%
\pgfsys@transformshift{3.370909in}{2.108449in}%
\pgfsys@useobject{currentmarker}{}%
\end{pgfscope}%
\begin{pgfscope}%
\pgfsys@transformshift{3.413182in}{2.108861in}%
\pgfsys@useobject{currentmarker}{}%
\end{pgfscope}%
\begin{pgfscope}%
\pgfsys@transformshift{3.455455in}{2.109232in}%
\pgfsys@useobject{currentmarker}{}%
\end{pgfscope}%
\begin{pgfscope}%
\pgfsys@transformshift{3.497727in}{2.109566in}%
\pgfsys@useobject{currentmarker}{}%
\end{pgfscope}%
\begin{pgfscope}%
\pgfsys@transformshift{3.540000in}{2.109866in}%
\pgfsys@useobject{currentmarker}{}%
\end{pgfscope}%
\begin{pgfscope}%
\pgfsys@transformshift{3.582273in}{2.110136in}%
\pgfsys@useobject{currentmarker}{}%
\end{pgfscope}%
\begin{pgfscope}%
\pgfsys@transformshift{3.624545in}{2.110379in}%
\pgfsys@useobject{currentmarker}{}%
\end{pgfscope}%
\begin{pgfscope}%
\pgfsys@transformshift{3.666818in}{2.110597in}%
\pgfsys@useobject{currentmarker}{}%
\end{pgfscope}%
\begin{pgfscope}%
\pgfsys@transformshift{3.709091in}{2.110793in}%
\pgfsys@useobject{currentmarker}{}%
\end{pgfscope}%
\begin{pgfscope}%
\pgfsys@transformshift{3.751364in}{2.110969in}%
\pgfsys@useobject{currentmarker}{}%
\end{pgfscope}%
\begin{pgfscope}%
\pgfsys@transformshift{3.793636in}{2.111128in}%
\pgfsys@useobject{currentmarker}{}%
\end{pgfscope}%
\begin{pgfscope}%
\pgfsys@transformshift{3.835909in}{2.111271in}%
\pgfsys@useobject{currentmarker}{}%
\end{pgfscope}%
\begin{pgfscope}%
\pgfsys@transformshift{3.878182in}{2.111399in}%
\pgfsys@useobject{currentmarker}{}%
\end{pgfscope}%
\begin{pgfscope}%
\pgfsys@transformshift{3.920455in}{2.111514in}%
\pgfsys@useobject{currentmarker}{}%
\end{pgfscope}%
\begin{pgfscope}%
\pgfsys@transformshift{3.962727in}{2.111618in}%
\pgfsys@useobject{currentmarker}{}%
\end{pgfscope}%
\begin{pgfscope}%
\pgfsys@transformshift{4.005000in}{2.111711in}%
\pgfsys@useobject{currentmarker}{}%
\end{pgfscope}%
\begin{pgfscope}%
\pgfsys@transformshift{4.047273in}{2.111795in}%
\pgfsys@useobject{currentmarker}{}%
\end{pgfscope}%
\begin{pgfscope}%
\pgfsys@transformshift{4.089545in}{2.111870in}%
\pgfsys@useobject{currentmarker}{}%
\end{pgfscope}%
\begin{pgfscope}%
\pgfsys@transformshift{4.131818in}{2.111938in}%
\pgfsys@useobject{currentmarker}{}%
\end{pgfscope}%
\begin{pgfscope}%
\pgfsys@transformshift{4.174091in}{2.111999in}%
\pgfsys@useobject{currentmarker}{}%
\end{pgfscope}%
\begin{pgfscope}%
\pgfsys@transformshift{4.216364in}{2.112054in}%
\pgfsys@useobject{currentmarker}{}%
\end{pgfscope}%
\begin{pgfscope}%
\pgfsys@transformshift{4.258636in}{2.112103in}%
\pgfsys@useobject{currentmarker}{}%
\end{pgfscope}%
\begin{pgfscope}%
\pgfsys@transformshift{4.300909in}{2.112147in}%
\pgfsys@useobject{currentmarker}{}%
\end{pgfscope}%
\begin{pgfscope}%
\pgfsys@transformshift{4.343182in}{2.112187in}%
\pgfsys@useobject{currentmarker}{}%
\end{pgfscope}%
\begin{pgfscope}%
\pgfsys@transformshift{4.385455in}{2.112223in}%
\pgfsys@useobject{currentmarker}{}%
\end{pgfscope}%
\begin{pgfscope}%
\pgfsys@transformshift{4.427727in}{2.112255in}%
\pgfsys@useobject{currentmarker}{}%
\end{pgfscope}%
\begin{pgfscope}%
\pgfsys@transformshift{4.470000in}{2.112284in}%
\pgfsys@useobject{currentmarker}{}%
\end{pgfscope}%
\begin{pgfscope}%
\pgfsys@transformshift{4.512273in}{2.112310in}%
\pgfsys@useobject{currentmarker}{}%
\end{pgfscope}%
\begin{pgfscope}%
\pgfsys@transformshift{4.554545in}{2.112334in}%
\pgfsys@useobject{currentmarker}{}%
\end{pgfscope}%
\begin{pgfscope}%
\pgfsys@transformshift{4.596818in}{2.112355in}%
\pgfsys@useobject{currentmarker}{}%
\end{pgfscope}%
\begin{pgfscope}%
\pgfsys@transformshift{4.639091in}{2.112374in}%
\pgfsys@useobject{currentmarker}{}%
\end{pgfscope}%
\begin{pgfscope}%
\pgfsys@transformshift{4.681364in}{2.112391in}%
\pgfsys@useobject{currentmarker}{}%
\end{pgfscope}%
\begin{pgfscope}%
\pgfsys@transformshift{4.723636in}{2.112406in}%
\pgfsys@useobject{currentmarker}{}%
\end{pgfscope}%
\begin{pgfscope}%
\pgfsys@transformshift{4.765909in}{2.112420in}%
\pgfsys@useobject{currentmarker}{}%
\end{pgfscope}%
\begin{pgfscope}%
\pgfsys@transformshift{4.808182in}{2.112432in}%
\pgfsys@useobject{currentmarker}{}%
\end{pgfscope}%
\begin{pgfscope}%
\pgfsys@transformshift{4.850455in}{2.112443in}%
\pgfsys@useobject{currentmarker}{}%
\end{pgfscope}%
\begin{pgfscope}%
\pgfsys@transformshift{4.892727in}{2.112453in}%
\pgfsys@useobject{currentmarker}{}%
\end{pgfscope}%
\begin{pgfscope}%
\pgfsys@transformshift{4.935000in}{2.112462in}%
\pgfsys@useobject{currentmarker}{}%
\end{pgfscope}%
\begin{pgfscope}%
\pgfsys@transformshift{4.977273in}{2.112470in}%
\pgfsys@useobject{currentmarker}{}%
\end{pgfscope}%
\begin{pgfscope}%
\pgfsys@transformshift{5.019545in}{2.112478in}%
\pgfsys@useobject{currentmarker}{}%
\end{pgfscope}%
\begin{pgfscope}%
\pgfsys@transformshift{5.061818in}{2.112484in}%
\pgfsys@useobject{currentmarker}{}%
\end{pgfscope}%
\begin{pgfscope}%
\pgfsys@transformshift{5.104091in}{2.112490in}%
\pgfsys@useobject{currentmarker}{}%
\end{pgfscope}%
\begin{pgfscope}%
\pgfsys@transformshift{5.146364in}{2.112495in}%
\pgfsys@useobject{currentmarker}{}%
\end{pgfscope}%
\begin{pgfscope}%
\pgfsys@transformshift{5.188636in}{2.112500in}%
\pgfsys@useobject{currentmarker}{}%
\end{pgfscope}%
\end{pgfscope}%
\begin{pgfscope}%
\pgfsetrectcap%
\pgfsetmiterjoin%
\pgfsetlinewidth{0.803000pt}%
\definecolor{currentstroke}{rgb}{0.000000,0.000000,0.000000}%
\pgfsetstrokecolor{currentstroke}%
\pgfsetdash{}{0pt}%
\pgfpathmoveto{\pgfqpoint{0.750000in}{0.275000in}}%
\pgfpathlineto{\pgfqpoint{0.750000in}{2.200000in}}%
\pgfusepath{stroke}%
\end{pgfscope}%
\begin{pgfscope}%
\pgfsetrectcap%
\pgfsetmiterjoin%
\pgfsetlinewidth{0.803000pt}%
\definecolor{currentstroke}{rgb}{0.000000,0.000000,0.000000}%
\pgfsetstrokecolor{currentstroke}%
\pgfsetdash{}{0pt}%
\pgfpathmoveto{\pgfqpoint{5.400000in}{0.275000in}}%
\pgfpathlineto{\pgfqpoint{5.400000in}{2.200000in}}%
\pgfusepath{stroke}%
\end{pgfscope}%
\begin{pgfscope}%
\pgfsetrectcap%
\pgfsetmiterjoin%
\pgfsetlinewidth{0.803000pt}%
\definecolor{currentstroke}{rgb}{0.000000,0.000000,0.000000}%
\pgfsetstrokecolor{currentstroke}%
\pgfsetdash{}{0pt}%
\pgfpathmoveto{\pgfqpoint{0.750000in}{0.275000in}}%
\pgfpathlineto{\pgfqpoint{5.400000in}{0.275000in}}%
\pgfusepath{stroke}%
\end{pgfscope}%
\begin{pgfscope}%
\pgfsetrectcap%
\pgfsetmiterjoin%
\pgfsetlinewidth{0.803000pt}%
\definecolor{currentstroke}{rgb}{0.000000,0.000000,0.000000}%
\pgfsetstrokecolor{currentstroke}%
\pgfsetdash{}{0pt}%
\pgfpathmoveto{\pgfqpoint{0.750000in}{2.200000in}}%
\pgfpathlineto{\pgfqpoint{5.400000in}{2.200000in}}%
\pgfusepath{stroke}%
\end{pgfscope}%
\end{pgfpicture}%
\makeatother%
\endgroup%

%%     \caption{The lower bound in \cref{eq:GMRESprob} with $\NLiDR{\no-\nt} \sim \Exp{\sigma}$ with $\sigma = 1/k^2.$\label{fig:prob-theory-plot-2.0}}
%% \end{subfigure}
%% \caption{The lower bound in \cref{eq:GMRESprob} for $R=12$, $\eps = 10^{-5}$, $N = \ceil{k^{3}}$, and $\Ct=0.1,$ for different functional forms of $\NLiDR{\no-\nt}$.}
%% \end{figure}
%% \ednote{Euan---the y axis on \cref{fig:prob-theory-plot-1.0} isn't correct. But basically, the values are all around 0.29, to within $10^{-11}$.}

\begin{figure}[p]
  \centering
  \begin{subfigure}{\textwidth}
    \centering
%% Creator: Matplotlib, PGF backend
%%
%% To include the figure in your LaTeX document, write
%%   \input{<filename>.pgf}
%%
%% Make sure the required packages are loaded in your preamble
%%   \usepackage{pgf}
%%
%% Figures using additional raster images can only be included by \input if
%% they are in the same directory as the main LaTeX file. For loading figures
%% from other directories you can use the `import` package
%%   \usepackage{import}
%% and then include the figures with
%%   \import{<path to file>}{<filename>.pgf}
%%
%% Matplotlib used the following preamble
%%   \usepackage{mleftright}
%%   \usepackage{fontspec}
%%   \setmainfont{DejaVuSerif.ttf}[Path=/home/owen/progs/firedrake-complex/firedrake/lib/python3.5/site-packages/matplotlib/mpl-data/fonts/ttf/]
%%   \setsansfont{DejaVuSans.ttf}[Path=/home/owen/progs/firedrake-complex/firedrake/lib/python3.5/site-packages/matplotlib/mpl-data/fonts/ttf/]
%%   \setmonofont{DejaVuSansMono.ttf}[Path=/home/owen/progs/firedrake-complex/firedrake/lib/python3.5/site-packages/matplotlib/mpl-data/fonts/ttf/]
%%
\begingroup%
\makeatletter%
\begin{pgfpicture}%
\pgfpathrectangle{\pgfpointorigin}{\pgfqpoint{5.570218in}{2.581603in}}%
\pgfusepath{use as bounding box, clip}%
\begin{pgfscope}%
\pgfsetbuttcap%
\pgfsetmiterjoin%
\definecolor{currentfill}{rgb}{1.000000,1.000000,1.000000}%
\pgfsetfillcolor{currentfill}%
\pgfsetlinewidth{0.000000pt}%
\definecolor{currentstroke}{rgb}{1.000000,1.000000,1.000000}%
\pgfsetstrokecolor{currentstroke}%
\pgfsetdash{}{0pt}%
\pgfpathmoveto{\pgfqpoint{0.000000in}{0.000000in}}%
\pgfpathlineto{\pgfqpoint{5.570218in}{0.000000in}}%
\pgfpathlineto{\pgfqpoint{5.570218in}{2.581603in}}%
\pgfpathlineto{\pgfqpoint{0.000000in}{2.581603in}}%
\pgfpathclose%
\pgfusepath{fill}%
\end{pgfscope}%
\begin{pgfscope}%
\pgfsetbuttcap%
\pgfsetmiterjoin%
\definecolor{currentfill}{rgb}{1.000000,1.000000,1.000000}%
\pgfsetfillcolor{currentfill}%
\pgfsetlinewidth{0.000000pt}%
\definecolor{currentstroke}{rgb}{0.000000,0.000000,0.000000}%
\pgfsetstrokecolor{currentstroke}%
\pgfsetstrokeopacity{0.000000}%
\pgfsetdash{}{0pt}%
\pgfpathmoveto{\pgfqpoint{0.785218in}{0.521603in}}%
\pgfpathlineto{\pgfqpoint{5.435218in}{0.521603in}}%
\pgfpathlineto{\pgfqpoint{5.435218in}{2.446603in}}%
\pgfpathlineto{\pgfqpoint{0.785218in}{2.446603in}}%
\pgfpathclose%
\pgfusepath{fill}%
\end{pgfscope}%
\begin{pgfscope}%
\pgfsetbuttcap%
\pgfsetroundjoin%
\definecolor{currentfill}{rgb}{0.000000,0.000000,0.000000}%
\pgfsetfillcolor{currentfill}%
\pgfsetlinewidth{0.803000pt}%
\definecolor{currentstroke}{rgb}{0.000000,0.000000,0.000000}%
\pgfsetstrokecolor{currentstroke}%
\pgfsetdash{}{0pt}%
\pgfsys@defobject{currentmarker}{\pgfqpoint{0.000000in}{-0.048611in}}{\pgfqpoint{0.000000in}{0.000000in}}{%
\pgfpathmoveto{\pgfqpoint{0.000000in}{0.000000in}}%
\pgfpathlineto{\pgfqpoint{0.000000in}{-0.048611in}}%
\pgfusepath{stroke,fill}%
}%
\begin{pgfscope}%
\pgfsys@transformshift{0.996582in}{0.521603in}%
\pgfsys@useobject{currentmarker}{}%
\end{pgfscope}%
\end{pgfscope}%
\begin{pgfscope}%
\definecolor{textcolor}{rgb}{0.000000,0.000000,0.000000}%
\pgfsetstrokecolor{textcolor}%
\pgfsetfillcolor{textcolor}%
\pgftext[x=0.996582in,y=0.424381in,,top]{\color{textcolor}\sffamily\fontsize{10.000000}{12.000000}\selectfont \(\displaystyle 10\)}%
\end{pgfscope}%
\begin{pgfscope}%
\pgfsetbuttcap%
\pgfsetroundjoin%
\definecolor{currentfill}{rgb}{0.000000,0.000000,0.000000}%
\pgfsetfillcolor{currentfill}%
\pgfsetlinewidth{0.803000pt}%
\definecolor{currentstroke}{rgb}{0.000000,0.000000,0.000000}%
\pgfsetstrokecolor{currentstroke}%
\pgfsetdash{}{0pt}%
\pgfsys@defobject{currentmarker}{\pgfqpoint{0.000000in}{-0.048611in}}{\pgfqpoint{0.000000in}{0.000000in}}{%
\pgfpathmoveto{\pgfqpoint{0.000000in}{0.000000in}}%
\pgfpathlineto{\pgfqpoint{0.000000in}{-0.048611in}}%
\pgfusepath{stroke,fill}%
}%
\begin{pgfscope}%
\pgfsys@transformshift{2.405673in}{0.521603in}%
\pgfsys@useobject{currentmarker}{}%
\end{pgfscope}%
\end{pgfscope}%
\begin{pgfscope}%
\definecolor{textcolor}{rgb}{0.000000,0.000000,0.000000}%
\pgfsetstrokecolor{textcolor}%
\pgfsetfillcolor{textcolor}%
\pgftext[x=2.405673in,y=0.424381in,,top]{\color{textcolor}\sffamily\fontsize{10.000000}{12.000000}\selectfont \(\displaystyle 20\)}%
\end{pgfscope}%
\begin{pgfscope}%
\pgfsetbuttcap%
\pgfsetroundjoin%
\definecolor{currentfill}{rgb}{0.000000,0.000000,0.000000}%
\pgfsetfillcolor{currentfill}%
\pgfsetlinewidth{0.803000pt}%
\definecolor{currentstroke}{rgb}{0.000000,0.000000,0.000000}%
\pgfsetstrokecolor{currentstroke}%
\pgfsetdash{}{0pt}%
\pgfsys@defobject{currentmarker}{\pgfqpoint{0.000000in}{-0.048611in}}{\pgfqpoint{0.000000in}{0.000000in}}{%
\pgfpathmoveto{\pgfqpoint{0.000000in}{0.000000in}}%
\pgfpathlineto{\pgfqpoint{0.000000in}{-0.048611in}}%
\pgfusepath{stroke,fill}%
}%
\begin{pgfscope}%
\pgfsys@transformshift{3.814763in}{0.521603in}%
\pgfsys@useobject{currentmarker}{}%
\end{pgfscope}%
\end{pgfscope}%
\begin{pgfscope}%
\definecolor{textcolor}{rgb}{0.000000,0.000000,0.000000}%
\pgfsetstrokecolor{textcolor}%
\pgfsetfillcolor{textcolor}%
\pgftext[x=3.814763in,y=0.424381in,,top]{\color{textcolor}\sffamily\fontsize{10.000000}{12.000000}\selectfont \(\displaystyle 30\)}%
\end{pgfscope}%
\begin{pgfscope}%
\pgfsetbuttcap%
\pgfsetroundjoin%
\definecolor{currentfill}{rgb}{0.000000,0.000000,0.000000}%
\pgfsetfillcolor{currentfill}%
\pgfsetlinewidth{0.803000pt}%
\definecolor{currentstroke}{rgb}{0.000000,0.000000,0.000000}%
\pgfsetstrokecolor{currentstroke}%
\pgfsetdash{}{0pt}%
\pgfsys@defobject{currentmarker}{\pgfqpoint{0.000000in}{-0.048611in}}{\pgfqpoint{0.000000in}{0.000000in}}{%
\pgfpathmoveto{\pgfqpoint{0.000000in}{0.000000in}}%
\pgfpathlineto{\pgfqpoint{0.000000in}{-0.048611in}}%
\pgfusepath{stroke,fill}%
}%
\begin{pgfscope}%
\pgfsys@transformshift{5.223854in}{0.521603in}%
\pgfsys@useobject{currentmarker}{}%
\end{pgfscope}%
\end{pgfscope}%
\begin{pgfscope}%
\definecolor{textcolor}{rgb}{0.000000,0.000000,0.000000}%
\pgfsetstrokecolor{textcolor}%
\pgfsetfillcolor{textcolor}%
\pgftext[x=5.223854in,y=0.424381in,,top]{\color{textcolor}\sffamily\fontsize{10.000000}{12.000000}\selectfont \(\displaystyle 40\)}%
\end{pgfscope}%
\begin{pgfscope}%
\definecolor{textcolor}{rgb}{0.000000,0.000000,0.000000}%
\pgfsetstrokecolor{textcolor}%
\pgfsetfillcolor{textcolor}%
\pgftext[x=3.110218in,y=0.234413in,,top]{\color{textcolor}\sffamily\fontsize{10.000000}{12.000000}\selectfont \(\displaystyle k\)}%
\end{pgfscope}%
\begin{pgfscope}%
\pgfsetbuttcap%
\pgfsetroundjoin%
\definecolor{currentfill}{rgb}{0.000000,0.000000,0.000000}%
\pgfsetfillcolor{currentfill}%
\pgfsetlinewidth{0.803000pt}%
\definecolor{currentstroke}{rgb}{0.000000,0.000000,0.000000}%
\pgfsetstrokecolor{currentstroke}%
\pgfsetdash{}{0pt}%
\pgfsys@defobject{currentmarker}{\pgfqpoint{-0.048611in}{0.000000in}}{\pgfqpoint{0.000000in}{0.000000in}}{%
\pgfpathmoveto{\pgfqpoint{0.000000in}{0.000000in}}%
\pgfpathlineto{\pgfqpoint{-0.048611in}{0.000000in}}%
\pgfusepath{stroke,fill}%
}%
\begin{pgfscope}%
\pgfsys@transformshift{0.785218in}{0.609103in}%
\pgfsys@useobject{currentmarker}{}%
\end{pgfscope}%
\end{pgfscope}%
\begin{pgfscope}%
\definecolor{textcolor}{rgb}{0.000000,0.000000,0.000000}%
\pgfsetstrokecolor{textcolor}%
\pgfsetfillcolor{textcolor}%
\pgftext[x=0.510526in,y=0.556342in,left,base]{\color{textcolor}\sffamily\fontsize{10.000000}{12.000000}\selectfont \(\displaystyle 0.0\)}%
\end{pgfscope}%
\begin{pgfscope}%
\pgfsetbuttcap%
\pgfsetroundjoin%
\definecolor{currentfill}{rgb}{0.000000,0.000000,0.000000}%
\pgfsetfillcolor{currentfill}%
\pgfsetlinewidth{0.803000pt}%
\definecolor{currentstroke}{rgb}{0.000000,0.000000,0.000000}%
\pgfsetstrokecolor{currentstroke}%
\pgfsetdash{}{0pt}%
\pgfsys@defobject{currentmarker}{\pgfqpoint{-0.048611in}{0.000000in}}{\pgfqpoint{0.000000in}{0.000000in}}{%
\pgfpathmoveto{\pgfqpoint{0.000000in}{0.000000in}}%
\pgfpathlineto{\pgfqpoint{-0.048611in}{0.000000in}}%
\pgfusepath{stroke,fill}%
}%
\begin{pgfscope}%
\pgfsys@transformshift{0.785218in}{1.035933in}%
\pgfsys@useobject{currentmarker}{}%
\end{pgfscope}%
\end{pgfscope}%
\begin{pgfscope}%
\definecolor{textcolor}{rgb}{0.000000,0.000000,0.000000}%
\pgfsetstrokecolor{textcolor}%
\pgfsetfillcolor{textcolor}%
\pgftext[x=0.510526in,y=0.983171in,left,base]{\color{textcolor}\sffamily\fontsize{10.000000}{12.000000}\selectfont \(\displaystyle 0.1\)}%
\end{pgfscope}%
\begin{pgfscope}%
\pgfsetbuttcap%
\pgfsetroundjoin%
\definecolor{currentfill}{rgb}{0.000000,0.000000,0.000000}%
\pgfsetfillcolor{currentfill}%
\pgfsetlinewidth{0.803000pt}%
\definecolor{currentstroke}{rgb}{0.000000,0.000000,0.000000}%
\pgfsetstrokecolor{currentstroke}%
\pgfsetdash{}{0pt}%
\pgfsys@defobject{currentmarker}{\pgfqpoint{-0.048611in}{0.000000in}}{\pgfqpoint{0.000000in}{0.000000in}}{%
\pgfpathmoveto{\pgfqpoint{0.000000in}{0.000000in}}%
\pgfpathlineto{\pgfqpoint{-0.048611in}{0.000000in}}%
\pgfusepath{stroke,fill}%
}%
\begin{pgfscope}%
\pgfsys@transformshift{0.785218in}{1.462762in}%
\pgfsys@useobject{currentmarker}{}%
\end{pgfscope}%
\end{pgfscope}%
\begin{pgfscope}%
\definecolor{textcolor}{rgb}{0.000000,0.000000,0.000000}%
\pgfsetstrokecolor{textcolor}%
\pgfsetfillcolor{textcolor}%
\pgftext[x=0.510526in,y=1.410000in,left,base]{\color{textcolor}\sffamily\fontsize{10.000000}{12.000000}\selectfont \(\displaystyle 0.2\)}%
\end{pgfscope}%
\begin{pgfscope}%
\pgfsetbuttcap%
\pgfsetroundjoin%
\definecolor{currentfill}{rgb}{0.000000,0.000000,0.000000}%
\pgfsetfillcolor{currentfill}%
\pgfsetlinewidth{0.803000pt}%
\definecolor{currentstroke}{rgb}{0.000000,0.000000,0.000000}%
\pgfsetstrokecolor{currentstroke}%
\pgfsetdash{}{0pt}%
\pgfsys@defobject{currentmarker}{\pgfqpoint{-0.048611in}{0.000000in}}{\pgfqpoint{0.000000in}{0.000000in}}{%
\pgfpathmoveto{\pgfqpoint{0.000000in}{0.000000in}}%
\pgfpathlineto{\pgfqpoint{-0.048611in}{0.000000in}}%
\pgfusepath{stroke,fill}%
}%
\begin{pgfscope}%
\pgfsys@transformshift{0.785218in}{1.889591in}%
\pgfsys@useobject{currentmarker}{}%
\end{pgfscope}%
\end{pgfscope}%
\begin{pgfscope}%
\definecolor{textcolor}{rgb}{0.000000,0.000000,0.000000}%
\pgfsetstrokecolor{textcolor}%
\pgfsetfillcolor{textcolor}%
\pgftext[x=0.510526in,y=1.836830in,left,base]{\color{textcolor}\sffamily\fontsize{10.000000}{12.000000}\selectfont \(\displaystyle 0.3\)}%
\end{pgfscope}%
\begin{pgfscope}%
\pgfsetbuttcap%
\pgfsetroundjoin%
\definecolor{currentfill}{rgb}{0.000000,0.000000,0.000000}%
\pgfsetfillcolor{currentfill}%
\pgfsetlinewidth{0.803000pt}%
\definecolor{currentstroke}{rgb}{0.000000,0.000000,0.000000}%
\pgfsetstrokecolor{currentstroke}%
\pgfsetdash{}{0pt}%
\pgfsys@defobject{currentmarker}{\pgfqpoint{-0.048611in}{0.000000in}}{\pgfqpoint{0.000000in}{0.000000in}}{%
\pgfpathmoveto{\pgfqpoint{0.000000in}{0.000000in}}%
\pgfpathlineto{\pgfqpoint{-0.048611in}{0.000000in}}%
\pgfusepath{stroke,fill}%
}%
\begin{pgfscope}%
\pgfsys@transformshift{0.785218in}{2.316420in}%
\pgfsys@useobject{currentmarker}{}%
\end{pgfscope}%
\end{pgfscope}%
\begin{pgfscope}%
\definecolor{textcolor}{rgb}{0.000000,0.000000,0.000000}%
\pgfsetstrokecolor{textcolor}%
\pgfsetfillcolor{textcolor}%
\pgftext[x=0.510526in,y=2.263659in,left,base]{\color{textcolor}\sffamily\fontsize{10.000000}{12.000000}\selectfont \(\displaystyle 0.4\)}%
\end{pgfscope}%
\begin{pgfscope}%
\definecolor{textcolor}{rgb}{0.000000,0.000000,0.000000}%
\pgfsetstrokecolor{textcolor}%
\pgfsetfillcolor{textcolor}%
\pgftext[x=0.144134in,y=0.605638in,left,base,rotate=90.000000]{\color{textcolor}\sffamily\fontsize{10.000000}{12.000000}\selectfont Empirical probability that}%
\end{pgfscope}%
\begin{pgfscope}%
\definecolor{textcolor}{rgb}{0.000000,0.000000,0.000000}%
\pgfsetstrokecolor{textcolor}%
\pgfsetfillcolor{textcolor}%
\pgftext[x=0.364691in,y=0.670455in,left,base,rotate=90.000000]{\color{textcolor}\sffamily\fontsize{10.000000}{12.000000}\selectfont \(\displaystyle \mathrm{GMRES}\mleft(\epsilon, n^{(1)} n^{(2)}\mright) \leq 12\)}%
\end{pgfscope}%
\begin{pgfscope}%
\pgfpathrectangle{\pgfqpoint{0.785218in}{0.521603in}}{\pgfqpoint{4.650000in}{1.925000in}}%
\pgfusepath{clip}%
\pgfsetbuttcap%
\pgfsetroundjoin%
\pgfsetlinewidth{1.505625pt}%
\definecolor{currentstroke}{rgb}{0.000000,0.000000,0.000000}%
\pgfsetstrokecolor{currentstroke}%
\pgfsetdash{{5.550000pt}{2.400000pt}}{0.000000pt}%
\pgfpathmoveto{\pgfqpoint{0.996582in}{2.359103in}}%
\pgfpathlineto{\pgfqpoint{2.405673in}{0.732884in}}%
\pgfpathlineto{\pgfqpoint{3.814763in}{0.609103in}}%
\pgfpathlineto{\pgfqpoint{5.223854in}{0.609103in}}%
\pgfusepath{stroke}%
\end{pgfscope}%
\begin{pgfscope}%
\pgfpathrectangle{\pgfqpoint{0.785218in}{0.521603in}}{\pgfqpoint{4.650000in}{1.925000in}}%
\pgfusepath{clip}%
\pgfsetbuttcap%
\pgfsetroundjoin%
\definecolor{currentfill}{rgb}{0.000000,0.000000,0.000000}%
\pgfsetfillcolor{currentfill}%
\pgfsetlinewidth{1.003750pt}%
\definecolor{currentstroke}{rgb}{0.000000,0.000000,0.000000}%
\pgfsetstrokecolor{currentstroke}%
\pgfsetdash{}{0pt}%
\pgfsys@defobject{currentmarker}{\pgfqpoint{-0.041667in}{-0.041667in}}{\pgfqpoint{0.041667in}{0.041667in}}{%
\pgfpathmoveto{\pgfqpoint{0.000000in}{-0.041667in}}%
\pgfpathcurveto{\pgfqpoint{0.011050in}{-0.041667in}}{\pgfqpoint{0.021649in}{-0.037276in}}{\pgfqpoint{0.029463in}{-0.029463in}}%
\pgfpathcurveto{\pgfqpoint{0.037276in}{-0.021649in}}{\pgfqpoint{0.041667in}{-0.011050in}}{\pgfqpoint{0.041667in}{0.000000in}}%
\pgfpathcurveto{\pgfqpoint{0.041667in}{0.011050in}}{\pgfqpoint{0.037276in}{0.021649in}}{\pgfqpoint{0.029463in}{0.029463in}}%
\pgfpathcurveto{\pgfqpoint{0.021649in}{0.037276in}}{\pgfqpoint{0.011050in}{0.041667in}}{\pgfqpoint{0.000000in}{0.041667in}}%
\pgfpathcurveto{\pgfqpoint{-0.011050in}{0.041667in}}{\pgfqpoint{-0.021649in}{0.037276in}}{\pgfqpoint{-0.029463in}{0.029463in}}%
\pgfpathcurveto{\pgfqpoint{-0.037276in}{0.021649in}}{\pgfqpoint{-0.041667in}{0.011050in}}{\pgfqpoint{-0.041667in}{0.000000in}}%
\pgfpathcurveto{\pgfqpoint{-0.041667in}{-0.011050in}}{\pgfqpoint{-0.037276in}{-0.021649in}}{\pgfqpoint{-0.029463in}{-0.029463in}}%
\pgfpathcurveto{\pgfqpoint{-0.021649in}{-0.037276in}}{\pgfqpoint{-0.011050in}{-0.041667in}}{\pgfqpoint{0.000000in}{-0.041667in}}%
\pgfpathclose%
\pgfusepath{stroke,fill}%
}%
\begin{pgfscope}%
\pgfsys@transformshift{0.996582in}{2.359103in}%
\pgfsys@useobject{currentmarker}{}%
\end{pgfscope}%
\begin{pgfscope}%
\pgfsys@transformshift{2.405673in}{0.732884in}%
\pgfsys@useobject{currentmarker}{}%
\end{pgfscope}%
\begin{pgfscope}%
\pgfsys@transformshift{3.814763in}{0.609103in}%
\pgfsys@useobject{currentmarker}{}%
\end{pgfscope}%
\begin{pgfscope}%
\pgfsys@transformshift{5.223854in}{0.609103in}%
\pgfsys@useobject{currentmarker}{}%
\end{pgfscope}%
\end{pgfscope}%
\begin{pgfscope}%
\pgfsetrectcap%
\pgfsetmiterjoin%
\pgfsetlinewidth{0.803000pt}%
\definecolor{currentstroke}{rgb}{0.000000,0.000000,0.000000}%
\pgfsetstrokecolor{currentstroke}%
\pgfsetdash{}{0pt}%
\pgfpathmoveto{\pgfqpoint{0.785218in}{0.521603in}}%
\pgfpathlineto{\pgfqpoint{0.785218in}{2.446603in}}%
\pgfusepath{stroke}%
\end{pgfscope}%
\begin{pgfscope}%
\pgfsetrectcap%
\pgfsetmiterjoin%
\pgfsetlinewidth{0.000000pt}%
\definecolor{currentstroke}{rgb}{0.000000,0.000000,0.000000}%
\pgfsetstrokecolor{currentstroke}%
\pgfsetstrokeopacity{0.000000}%
\pgfsetdash{}{0pt}%
\pgfpathmoveto{\pgfqpoint{5.435218in}{0.521603in}}%
\pgfpathlineto{\pgfqpoint{5.435218in}{2.446603in}}%
\pgfusepath{}%
\end{pgfscope}%
\begin{pgfscope}%
\pgfsetrectcap%
\pgfsetmiterjoin%
\pgfsetlinewidth{0.803000pt}%
\definecolor{currentstroke}{rgb}{0.000000,0.000000,0.000000}%
\pgfsetstrokecolor{currentstroke}%
\pgfsetdash{}{0pt}%
\pgfpathmoveto{\pgfqpoint{0.785218in}{0.521603in}}%
\pgfpathlineto{\pgfqpoint{5.435218in}{0.521603in}}%
\pgfusepath{stroke}%
\end{pgfscope}%
\begin{pgfscope}%
\pgfsetrectcap%
\pgfsetmiterjoin%
\pgfsetlinewidth{0.000000pt}%
\definecolor{currentstroke}{rgb}{0.000000,0.000000,0.000000}%
\pgfsetstrokecolor{currentstroke}%
\pgfsetstrokeopacity{0.000000}%
\pgfsetdash{}{0pt}%
\pgfpathmoveto{\pgfqpoint{0.785218in}{2.446603in}}%
\pgfpathlineto{\pgfqpoint{5.435218in}{2.446603in}}%
\pgfusepath{}%
\end{pgfscope}%
\end{pgfpicture}%
\makeatother%
\endgroup%

\caption{The empirical probability that $\GMRES{\eps}{\no}{\nt}\leq 12$ for $\sigma = 1.$\label{fig:prob-plot-0.0}}
\end{subfigure}

\begin{subfigure}{\textwidth}
    \centering
%% Creator: Matplotlib, PGF backend
%%
%% To include the figure in your LaTeX document, write
%%   \input{<filename>.pgf}
%%
%% Make sure the required packages are loaded in your preamble
%%   \usepackage{pgf}
%%
%% Figures using additional raster images can only be included by \input if
%% they are in the same directory as the main LaTeX file. For loading figures
%% from other directories you can use the `import` package
%%   \usepackage{import}
%% and then include the figures with
%%   \import{<path to file>}{<filename>.pgf}
%%
%% Matplotlib used the following preamble
%%   \usepackage{fontspec}
%%   \setmainfont{DejaVuSerif.ttf}[Path=/home/owen/progs/firedrake-complex/firedrake/lib/python3.5/site-packages/matplotlib/mpl-data/fonts/ttf/]
%%   \setsansfont{DejaVuSans.ttf}[Path=/home/owen/progs/firedrake-complex/firedrake/lib/python3.5/site-packages/matplotlib/mpl-data/fonts/ttf/]
%%   \setmonofont{DejaVuSansMono.ttf}[Path=/home/owen/progs/firedrake-complex/firedrake/lib/python3.5/site-packages/matplotlib/mpl-data/fonts/ttf/]
%%
\begingroup%
\makeatletter%
\begin{pgfpicture}%
\pgfpathrectangle{\pgfpointorigin}{\pgfqpoint{6.400000in}{4.800000in}}%
\pgfusepath{use as bounding box, clip}%
\begin{pgfscope}%
\pgfsetbuttcap%
\pgfsetmiterjoin%
\definecolor{currentfill}{rgb}{1.000000,1.000000,1.000000}%
\pgfsetfillcolor{currentfill}%
\pgfsetlinewidth{0.000000pt}%
\definecolor{currentstroke}{rgb}{1.000000,1.000000,1.000000}%
\pgfsetstrokecolor{currentstroke}%
\pgfsetdash{}{0pt}%
\pgfpathmoveto{\pgfqpoint{0.000000in}{0.000000in}}%
\pgfpathlineto{\pgfqpoint{6.400000in}{0.000000in}}%
\pgfpathlineto{\pgfqpoint{6.400000in}{4.800000in}}%
\pgfpathlineto{\pgfqpoint{0.000000in}{4.800000in}}%
\pgfpathclose%
\pgfusepath{fill}%
\end{pgfscope}%
\begin{pgfscope}%
\pgfsetbuttcap%
\pgfsetmiterjoin%
\definecolor{currentfill}{rgb}{1.000000,1.000000,1.000000}%
\pgfsetfillcolor{currentfill}%
\pgfsetlinewidth{0.000000pt}%
\definecolor{currentstroke}{rgb}{0.000000,0.000000,0.000000}%
\pgfsetstrokecolor{currentstroke}%
\pgfsetstrokeopacity{0.000000}%
\pgfsetdash{}{0pt}%
\pgfpathmoveto{\pgfqpoint{0.800000in}{0.528000in}}%
\pgfpathlineto{\pgfqpoint{5.760000in}{0.528000in}}%
\pgfpathlineto{\pgfqpoint{5.760000in}{4.224000in}}%
\pgfpathlineto{\pgfqpoint{0.800000in}{4.224000in}}%
\pgfpathclose%
\pgfusepath{fill}%
\end{pgfscope}%
\begin{pgfscope}%
\pgfsetbuttcap%
\pgfsetroundjoin%
\definecolor{currentfill}{rgb}{0.000000,0.000000,0.000000}%
\pgfsetfillcolor{currentfill}%
\pgfsetlinewidth{0.803000pt}%
\definecolor{currentstroke}{rgb}{0.000000,0.000000,0.000000}%
\pgfsetstrokecolor{currentstroke}%
\pgfsetdash{}{0pt}%
\pgfsys@defobject{currentmarker}{\pgfqpoint{0.000000in}{-0.048611in}}{\pgfqpoint{0.000000in}{0.000000in}}{%
\pgfpathmoveto{\pgfqpoint{0.000000in}{0.000000in}}%
\pgfpathlineto{\pgfqpoint{0.000000in}{-0.048611in}}%
\pgfusepath{stroke,fill}%
}%
\begin{pgfscope}%
\pgfsys@transformshift{1.025455in}{0.528000in}%
\pgfsys@useobject{currentmarker}{}%
\end{pgfscope}%
\end{pgfscope}%
\begin{pgfscope}%
\definecolor{textcolor}{rgb}{0.000000,0.000000,0.000000}%
\pgfsetstrokecolor{textcolor}%
\pgfsetfillcolor{textcolor}%
\pgftext[x=1.025455in,y=0.430778in,,top]{\color{textcolor}\sffamily\fontsize{10.000000}{12.000000}\selectfont 10}%
\end{pgfscope}%
\begin{pgfscope}%
\pgfsetbuttcap%
\pgfsetroundjoin%
\definecolor{currentfill}{rgb}{0.000000,0.000000,0.000000}%
\pgfsetfillcolor{currentfill}%
\pgfsetlinewidth{0.803000pt}%
\definecolor{currentstroke}{rgb}{0.000000,0.000000,0.000000}%
\pgfsetstrokecolor{currentstroke}%
\pgfsetdash{}{0pt}%
\pgfsys@defobject{currentmarker}{\pgfqpoint{0.000000in}{-0.048611in}}{\pgfqpoint{0.000000in}{0.000000in}}{%
\pgfpathmoveto{\pgfqpoint{0.000000in}{0.000000in}}%
\pgfpathlineto{\pgfqpoint{0.000000in}{-0.048611in}}%
\pgfusepath{stroke,fill}%
}%
\begin{pgfscope}%
\pgfsys@transformshift{2.528485in}{0.528000in}%
\pgfsys@useobject{currentmarker}{}%
\end{pgfscope}%
\end{pgfscope}%
\begin{pgfscope}%
\definecolor{textcolor}{rgb}{0.000000,0.000000,0.000000}%
\pgfsetstrokecolor{textcolor}%
\pgfsetfillcolor{textcolor}%
\pgftext[x=2.528485in,y=0.430778in,,top]{\color{textcolor}\sffamily\fontsize{10.000000}{12.000000}\selectfont 20}%
\end{pgfscope}%
\begin{pgfscope}%
\pgfsetbuttcap%
\pgfsetroundjoin%
\definecolor{currentfill}{rgb}{0.000000,0.000000,0.000000}%
\pgfsetfillcolor{currentfill}%
\pgfsetlinewidth{0.803000pt}%
\definecolor{currentstroke}{rgb}{0.000000,0.000000,0.000000}%
\pgfsetstrokecolor{currentstroke}%
\pgfsetdash{}{0pt}%
\pgfsys@defobject{currentmarker}{\pgfqpoint{0.000000in}{-0.048611in}}{\pgfqpoint{0.000000in}{0.000000in}}{%
\pgfpathmoveto{\pgfqpoint{0.000000in}{0.000000in}}%
\pgfpathlineto{\pgfqpoint{0.000000in}{-0.048611in}}%
\pgfusepath{stroke,fill}%
}%
\begin{pgfscope}%
\pgfsys@transformshift{4.031515in}{0.528000in}%
\pgfsys@useobject{currentmarker}{}%
\end{pgfscope}%
\end{pgfscope}%
\begin{pgfscope}%
\definecolor{textcolor}{rgb}{0.000000,0.000000,0.000000}%
\pgfsetstrokecolor{textcolor}%
\pgfsetfillcolor{textcolor}%
\pgftext[x=4.031515in,y=0.430778in,,top]{\color{textcolor}\sffamily\fontsize{10.000000}{12.000000}\selectfont 30}%
\end{pgfscope}%
\begin{pgfscope}%
\pgfsetbuttcap%
\pgfsetroundjoin%
\definecolor{currentfill}{rgb}{0.000000,0.000000,0.000000}%
\pgfsetfillcolor{currentfill}%
\pgfsetlinewidth{0.803000pt}%
\definecolor{currentstroke}{rgb}{0.000000,0.000000,0.000000}%
\pgfsetstrokecolor{currentstroke}%
\pgfsetdash{}{0pt}%
\pgfsys@defobject{currentmarker}{\pgfqpoint{0.000000in}{-0.048611in}}{\pgfqpoint{0.000000in}{0.000000in}}{%
\pgfpathmoveto{\pgfqpoint{0.000000in}{0.000000in}}%
\pgfpathlineto{\pgfqpoint{0.000000in}{-0.048611in}}%
\pgfusepath{stroke,fill}%
}%
\begin{pgfscope}%
\pgfsys@transformshift{5.534545in}{0.528000in}%
\pgfsys@useobject{currentmarker}{}%
\end{pgfscope}%
\end{pgfscope}%
\begin{pgfscope}%
\definecolor{textcolor}{rgb}{0.000000,0.000000,0.000000}%
\pgfsetstrokecolor{textcolor}%
\pgfsetfillcolor{textcolor}%
\pgftext[x=5.534545in,y=0.430778in,,top]{\color{textcolor}\sffamily\fontsize{10.000000}{12.000000}\selectfont 40}%
\end{pgfscope}%
\begin{pgfscope}%
\definecolor{textcolor}{rgb}{0.000000,0.000000,0.000000}%
\pgfsetstrokecolor{textcolor}%
\pgfsetfillcolor{textcolor}%
\pgftext[x=3.280000in,y=0.240809in,,top]{\color{textcolor}\sffamily\fontsize{10.000000}{12.000000}\selectfont \(\displaystyle k\)}%
\end{pgfscope}%
\begin{pgfscope}%
\pgfsetbuttcap%
\pgfsetroundjoin%
\definecolor{currentfill}{rgb}{0.000000,0.000000,0.000000}%
\pgfsetfillcolor{currentfill}%
\pgfsetlinewidth{0.803000pt}%
\definecolor{currentstroke}{rgb}{0.000000,0.000000,0.000000}%
\pgfsetstrokecolor{currentstroke}%
\pgfsetdash{}{0pt}%
\pgfsys@defobject{currentmarker}{\pgfqpoint{-0.048611in}{0.000000in}}{\pgfqpoint{0.000000in}{0.000000in}}{%
\pgfpathmoveto{\pgfqpoint{0.000000in}{0.000000in}}%
\pgfpathlineto{\pgfqpoint{-0.048611in}{0.000000in}}%
\pgfusepath{stroke,fill}%
}%
\begin{pgfscope}%
\pgfsys@transformshift{0.800000in}{0.696000in}%
\pgfsys@useobject{currentmarker}{}%
\end{pgfscope}%
\end{pgfscope}%
\begin{pgfscope}%
\definecolor{textcolor}{rgb}{0.000000,0.000000,0.000000}%
\pgfsetstrokecolor{textcolor}%
\pgfsetfillcolor{textcolor}%
\pgftext[x=0.216802in,y=0.643238in,left,base]{\color{textcolor}\sffamily\fontsize{10.000000}{12.000000}\selectfont 0.9920}%
\end{pgfscope}%
\begin{pgfscope}%
\pgfsetbuttcap%
\pgfsetroundjoin%
\definecolor{currentfill}{rgb}{0.000000,0.000000,0.000000}%
\pgfsetfillcolor{currentfill}%
\pgfsetlinewidth{0.803000pt}%
\definecolor{currentstroke}{rgb}{0.000000,0.000000,0.000000}%
\pgfsetstrokecolor{currentstroke}%
\pgfsetdash{}{0pt}%
\pgfsys@defobject{currentmarker}{\pgfqpoint{-0.048611in}{0.000000in}}{\pgfqpoint{0.000000in}{0.000000in}}{%
\pgfpathmoveto{\pgfqpoint{0.000000in}{0.000000in}}%
\pgfpathlineto{\pgfqpoint{-0.048611in}{0.000000in}}%
\pgfusepath{stroke,fill}%
}%
\begin{pgfscope}%
\pgfsys@transformshift{0.800000in}{1.080000in}%
\pgfsys@useobject{currentmarker}{}%
\end{pgfscope}%
\end{pgfscope}%
\begin{pgfscope}%
\definecolor{textcolor}{rgb}{0.000000,0.000000,0.000000}%
\pgfsetstrokecolor{textcolor}%
\pgfsetfillcolor{textcolor}%
\pgftext[x=0.216802in,y=1.027238in,left,base]{\color{textcolor}\sffamily\fontsize{10.000000}{12.000000}\selectfont 0.9928}%
\end{pgfscope}%
\begin{pgfscope}%
\pgfsetbuttcap%
\pgfsetroundjoin%
\definecolor{currentfill}{rgb}{0.000000,0.000000,0.000000}%
\pgfsetfillcolor{currentfill}%
\pgfsetlinewidth{0.803000pt}%
\definecolor{currentstroke}{rgb}{0.000000,0.000000,0.000000}%
\pgfsetstrokecolor{currentstroke}%
\pgfsetdash{}{0pt}%
\pgfsys@defobject{currentmarker}{\pgfqpoint{-0.048611in}{0.000000in}}{\pgfqpoint{0.000000in}{0.000000in}}{%
\pgfpathmoveto{\pgfqpoint{0.000000in}{0.000000in}}%
\pgfpathlineto{\pgfqpoint{-0.048611in}{0.000000in}}%
\pgfusepath{stroke,fill}%
}%
\begin{pgfscope}%
\pgfsys@transformshift{0.800000in}{1.464000in}%
\pgfsys@useobject{currentmarker}{}%
\end{pgfscope}%
\end{pgfscope}%
\begin{pgfscope}%
\definecolor{textcolor}{rgb}{0.000000,0.000000,0.000000}%
\pgfsetstrokecolor{textcolor}%
\pgfsetfillcolor{textcolor}%
\pgftext[x=0.216802in,y=1.411238in,left,base]{\color{textcolor}\sffamily\fontsize{10.000000}{12.000000}\selectfont 0.9936}%
\end{pgfscope}%
\begin{pgfscope}%
\pgfsetbuttcap%
\pgfsetroundjoin%
\definecolor{currentfill}{rgb}{0.000000,0.000000,0.000000}%
\pgfsetfillcolor{currentfill}%
\pgfsetlinewidth{0.803000pt}%
\definecolor{currentstroke}{rgb}{0.000000,0.000000,0.000000}%
\pgfsetstrokecolor{currentstroke}%
\pgfsetdash{}{0pt}%
\pgfsys@defobject{currentmarker}{\pgfqpoint{-0.048611in}{0.000000in}}{\pgfqpoint{0.000000in}{0.000000in}}{%
\pgfpathmoveto{\pgfqpoint{0.000000in}{0.000000in}}%
\pgfpathlineto{\pgfqpoint{-0.048611in}{0.000000in}}%
\pgfusepath{stroke,fill}%
}%
\begin{pgfscope}%
\pgfsys@transformshift{0.800000in}{1.848000in}%
\pgfsys@useobject{currentmarker}{}%
\end{pgfscope}%
\end{pgfscope}%
\begin{pgfscope}%
\definecolor{textcolor}{rgb}{0.000000,0.000000,0.000000}%
\pgfsetstrokecolor{textcolor}%
\pgfsetfillcolor{textcolor}%
\pgftext[x=0.216802in,y=1.795238in,left,base]{\color{textcolor}\sffamily\fontsize{10.000000}{12.000000}\selectfont 0.9944}%
\end{pgfscope}%
\begin{pgfscope}%
\pgfsetbuttcap%
\pgfsetroundjoin%
\definecolor{currentfill}{rgb}{0.000000,0.000000,0.000000}%
\pgfsetfillcolor{currentfill}%
\pgfsetlinewidth{0.803000pt}%
\definecolor{currentstroke}{rgb}{0.000000,0.000000,0.000000}%
\pgfsetstrokecolor{currentstroke}%
\pgfsetdash{}{0pt}%
\pgfsys@defobject{currentmarker}{\pgfqpoint{-0.048611in}{0.000000in}}{\pgfqpoint{0.000000in}{0.000000in}}{%
\pgfpathmoveto{\pgfqpoint{0.000000in}{0.000000in}}%
\pgfpathlineto{\pgfqpoint{-0.048611in}{0.000000in}}%
\pgfusepath{stroke,fill}%
}%
\begin{pgfscope}%
\pgfsys@transformshift{0.800000in}{2.232000in}%
\pgfsys@useobject{currentmarker}{}%
\end{pgfscope}%
\end{pgfscope}%
\begin{pgfscope}%
\definecolor{textcolor}{rgb}{0.000000,0.000000,0.000000}%
\pgfsetstrokecolor{textcolor}%
\pgfsetfillcolor{textcolor}%
\pgftext[x=0.216802in,y=2.179238in,left,base]{\color{textcolor}\sffamily\fontsize{10.000000}{12.000000}\selectfont 0.9952}%
\end{pgfscope}%
\begin{pgfscope}%
\pgfsetbuttcap%
\pgfsetroundjoin%
\definecolor{currentfill}{rgb}{0.000000,0.000000,0.000000}%
\pgfsetfillcolor{currentfill}%
\pgfsetlinewidth{0.803000pt}%
\definecolor{currentstroke}{rgb}{0.000000,0.000000,0.000000}%
\pgfsetstrokecolor{currentstroke}%
\pgfsetdash{}{0pt}%
\pgfsys@defobject{currentmarker}{\pgfqpoint{-0.048611in}{0.000000in}}{\pgfqpoint{0.000000in}{0.000000in}}{%
\pgfpathmoveto{\pgfqpoint{0.000000in}{0.000000in}}%
\pgfpathlineto{\pgfqpoint{-0.048611in}{0.000000in}}%
\pgfusepath{stroke,fill}%
}%
\begin{pgfscope}%
\pgfsys@transformshift{0.800000in}{2.616000in}%
\pgfsys@useobject{currentmarker}{}%
\end{pgfscope}%
\end{pgfscope}%
\begin{pgfscope}%
\definecolor{textcolor}{rgb}{0.000000,0.000000,0.000000}%
\pgfsetstrokecolor{textcolor}%
\pgfsetfillcolor{textcolor}%
\pgftext[x=0.216802in,y=2.563238in,left,base]{\color{textcolor}\sffamily\fontsize{10.000000}{12.000000}\selectfont 0.9960}%
\end{pgfscope}%
\begin{pgfscope}%
\pgfsetbuttcap%
\pgfsetroundjoin%
\definecolor{currentfill}{rgb}{0.000000,0.000000,0.000000}%
\pgfsetfillcolor{currentfill}%
\pgfsetlinewidth{0.803000pt}%
\definecolor{currentstroke}{rgb}{0.000000,0.000000,0.000000}%
\pgfsetstrokecolor{currentstroke}%
\pgfsetdash{}{0pt}%
\pgfsys@defobject{currentmarker}{\pgfqpoint{-0.048611in}{0.000000in}}{\pgfqpoint{0.000000in}{0.000000in}}{%
\pgfpathmoveto{\pgfqpoint{0.000000in}{0.000000in}}%
\pgfpathlineto{\pgfqpoint{-0.048611in}{0.000000in}}%
\pgfusepath{stroke,fill}%
}%
\begin{pgfscope}%
\pgfsys@transformshift{0.800000in}{3.000000in}%
\pgfsys@useobject{currentmarker}{}%
\end{pgfscope}%
\end{pgfscope}%
\begin{pgfscope}%
\definecolor{textcolor}{rgb}{0.000000,0.000000,0.000000}%
\pgfsetstrokecolor{textcolor}%
\pgfsetfillcolor{textcolor}%
\pgftext[x=0.216802in,y=2.947238in,left,base]{\color{textcolor}\sffamily\fontsize{10.000000}{12.000000}\selectfont 0.9968}%
\end{pgfscope}%
\begin{pgfscope}%
\pgfsetbuttcap%
\pgfsetroundjoin%
\definecolor{currentfill}{rgb}{0.000000,0.000000,0.000000}%
\pgfsetfillcolor{currentfill}%
\pgfsetlinewidth{0.803000pt}%
\definecolor{currentstroke}{rgb}{0.000000,0.000000,0.000000}%
\pgfsetstrokecolor{currentstroke}%
\pgfsetdash{}{0pt}%
\pgfsys@defobject{currentmarker}{\pgfqpoint{-0.048611in}{0.000000in}}{\pgfqpoint{0.000000in}{0.000000in}}{%
\pgfpathmoveto{\pgfqpoint{0.000000in}{0.000000in}}%
\pgfpathlineto{\pgfqpoint{-0.048611in}{0.000000in}}%
\pgfusepath{stroke,fill}%
}%
\begin{pgfscope}%
\pgfsys@transformshift{0.800000in}{3.384000in}%
\pgfsys@useobject{currentmarker}{}%
\end{pgfscope}%
\end{pgfscope}%
\begin{pgfscope}%
\definecolor{textcolor}{rgb}{0.000000,0.000000,0.000000}%
\pgfsetstrokecolor{textcolor}%
\pgfsetfillcolor{textcolor}%
\pgftext[x=0.216802in,y=3.331238in,left,base]{\color{textcolor}\sffamily\fontsize{10.000000}{12.000000}\selectfont 0.9976}%
\end{pgfscope}%
\begin{pgfscope}%
\pgfsetbuttcap%
\pgfsetroundjoin%
\definecolor{currentfill}{rgb}{0.000000,0.000000,0.000000}%
\pgfsetfillcolor{currentfill}%
\pgfsetlinewidth{0.803000pt}%
\definecolor{currentstroke}{rgb}{0.000000,0.000000,0.000000}%
\pgfsetstrokecolor{currentstroke}%
\pgfsetdash{}{0pt}%
\pgfsys@defobject{currentmarker}{\pgfqpoint{-0.048611in}{0.000000in}}{\pgfqpoint{0.000000in}{0.000000in}}{%
\pgfpathmoveto{\pgfqpoint{0.000000in}{0.000000in}}%
\pgfpathlineto{\pgfqpoint{-0.048611in}{0.000000in}}%
\pgfusepath{stroke,fill}%
}%
\begin{pgfscope}%
\pgfsys@transformshift{0.800000in}{3.768000in}%
\pgfsys@useobject{currentmarker}{}%
\end{pgfscope}%
\end{pgfscope}%
\begin{pgfscope}%
\definecolor{textcolor}{rgb}{0.000000,0.000000,0.000000}%
\pgfsetstrokecolor{textcolor}%
\pgfsetfillcolor{textcolor}%
\pgftext[x=0.216802in,y=3.715238in,left,base]{\color{textcolor}\sffamily\fontsize{10.000000}{12.000000}\selectfont 0.9984}%
\end{pgfscope}%
\begin{pgfscope}%
\pgfsetbuttcap%
\pgfsetroundjoin%
\definecolor{currentfill}{rgb}{0.000000,0.000000,0.000000}%
\pgfsetfillcolor{currentfill}%
\pgfsetlinewidth{0.803000pt}%
\definecolor{currentstroke}{rgb}{0.000000,0.000000,0.000000}%
\pgfsetstrokecolor{currentstroke}%
\pgfsetdash{}{0pt}%
\pgfsys@defobject{currentmarker}{\pgfqpoint{-0.048611in}{0.000000in}}{\pgfqpoint{0.000000in}{0.000000in}}{%
\pgfpathmoveto{\pgfqpoint{0.000000in}{0.000000in}}%
\pgfpathlineto{\pgfqpoint{-0.048611in}{0.000000in}}%
\pgfusepath{stroke,fill}%
}%
\begin{pgfscope}%
\pgfsys@transformshift{0.800000in}{4.152000in}%
\pgfsys@useobject{currentmarker}{}%
\end{pgfscope}%
\end{pgfscope}%
\begin{pgfscope}%
\definecolor{textcolor}{rgb}{0.000000,0.000000,0.000000}%
\pgfsetstrokecolor{textcolor}%
\pgfsetfillcolor{textcolor}%
\pgftext[x=0.216802in,y=4.099238in,left,base]{\color{textcolor}\sffamily\fontsize{10.000000}{12.000000}\selectfont 0.9992}%
\end{pgfscope}%
\begin{pgfscope}%
\definecolor{textcolor}{rgb}{0.000000,0.000000,0.000000}%
\pgfsetstrokecolor{textcolor}%
\pgfsetfillcolor{textcolor}%
\pgftext[x=0.161247in,y=2.376000in,,bottom,rotate=90.000000]{\color{textcolor}\sffamily\fontsize{10.000000}{12.000000}\selectfont Number of GMRES iterations}%
\end{pgfscope}%
\begin{pgfscope}%
\pgfpathrectangle{\pgfqpoint{0.800000in}{0.528000in}}{\pgfqpoint{4.960000in}{3.696000in}}%
\pgfusepath{clip}%
\pgfsetbuttcap%
\pgfsetroundjoin%
\definecolor{currentfill}{rgb}{0.000000,0.000000,0.000000}%
\pgfsetfillcolor{currentfill}%
\pgfsetlinewidth{1.003750pt}%
\definecolor{currentstroke}{rgb}{0.000000,0.000000,0.000000}%
\pgfsetstrokecolor{currentstroke}%
\pgfsetdash{}{0pt}%
\pgfsys@defobject{currentmarker}{\pgfqpoint{-0.041667in}{-0.041667in}}{\pgfqpoint{0.041667in}{0.041667in}}{%
\pgfpathmoveto{\pgfqpoint{0.000000in}{-0.041667in}}%
\pgfpathcurveto{\pgfqpoint{0.011050in}{-0.041667in}}{\pgfqpoint{0.021649in}{-0.037276in}}{\pgfqpoint{0.029463in}{-0.029463in}}%
\pgfpathcurveto{\pgfqpoint{0.037276in}{-0.021649in}}{\pgfqpoint{0.041667in}{-0.011050in}}{\pgfqpoint{0.041667in}{0.000000in}}%
\pgfpathcurveto{\pgfqpoint{0.041667in}{0.011050in}}{\pgfqpoint{0.037276in}{0.021649in}}{\pgfqpoint{0.029463in}{0.029463in}}%
\pgfpathcurveto{\pgfqpoint{0.021649in}{0.037276in}}{\pgfqpoint{0.011050in}{0.041667in}}{\pgfqpoint{0.000000in}{0.041667in}}%
\pgfpathcurveto{\pgfqpoint{-0.011050in}{0.041667in}}{\pgfqpoint{-0.021649in}{0.037276in}}{\pgfqpoint{-0.029463in}{0.029463in}}%
\pgfpathcurveto{\pgfqpoint{-0.037276in}{0.021649in}}{\pgfqpoint{-0.041667in}{0.011050in}}{\pgfqpoint{-0.041667in}{0.000000in}}%
\pgfpathcurveto{\pgfqpoint{-0.041667in}{-0.011050in}}{\pgfqpoint{-0.037276in}{-0.021649in}}{\pgfqpoint{-0.029463in}{-0.029463in}}%
\pgfpathcurveto{\pgfqpoint{-0.021649in}{-0.037276in}}{\pgfqpoint{-0.011050in}{-0.041667in}}{\pgfqpoint{0.000000in}{-0.041667in}}%
\pgfpathclose%
\pgfusepath{stroke,fill}%
}%
\begin{pgfscope}%
\pgfsys@transformshift{1.025455in}{4.056000in}%
\pgfsys@useobject{currentmarker}{}%
\end{pgfscope}%
\end{pgfscope}%
\begin{pgfscope}%
\pgfpathrectangle{\pgfqpoint{0.800000in}{0.528000in}}{\pgfqpoint{4.960000in}{3.696000in}}%
\pgfusepath{clip}%
\pgfsetbuttcap%
\pgfsetroundjoin%
\definecolor{currentfill}{rgb}{0.000000,0.000000,0.000000}%
\pgfsetfillcolor{currentfill}%
\pgfsetlinewidth{1.003750pt}%
\definecolor{currentstroke}{rgb}{0.000000,0.000000,0.000000}%
\pgfsetstrokecolor{currentstroke}%
\pgfsetdash{}{0pt}%
\pgfsys@defobject{currentmarker}{\pgfqpoint{-0.041667in}{-0.041667in}}{\pgfqpoint{0.041667in}{0.041667in}}{%
\pgfpathmoveto{\pgfqpoint{0.000000in}{-0.041667in}}%
\pgfpathcurveto{\pgfqpoint{0.011050in}{-0.041667in}}{\pgfqpoint{0.021649in}{-0.037276in}}{\pgfqpoint{0.029463in}{-0.029463in}}%
\pgfpathcurveto{\pgfqpoint{0.037276in}{-0.021649in}}{\pgfqpoint{0.041667in}{-0.011050in}}{\pgfqpoint{0.041667in}{0.000000in}}%
\pgfpathcurveto{\pgfqpoint{0.041667in}{0.011050in}}{\pgfqpoint{0.037276in}{0.021649in}}{\pgfqpoint{0.029463in}{0.029463in}}%
\pgfpathcurveto{\pgfqpoint{0.021649in}{0.037276in}}{\pgfqpoint{0.011050in}{0.041667in}}{\pgfqpoint{0.000000in}{0.041667in}}%
\pgfpathcurveto{\pgfqpoint{-0.011050in}{0.041667in}}{\pgfqpoint{-0.021649in}{0.037276in}}{\pgfqpoint{-0.029463in}{0.029463in}}%
\pgfpathcurveto{\pgfqpoint{-0.037276in}{0.021649in}}{\pgfqpoint{-0.041667in}{0.011050in}}{\pgfqpoint{-0.041667in}{0.000000in}}%
\pgfpathcurveto{\pgfqpoint{-0.041667in}{-0.011050in}}{\pgfqpoint{-0.037276in}{-0.021649in}}{\pgfqpoint{-0.029463in}{-0.029463in}}%
\pgfpathcurveto{\pgfqpoint{-0.021649in}{-0.037276in}}{\pgfqpoint{-0.011050in}{-0.041667in}}{\pgfqpoint{0.000000in}{-0.041667in}}%
\pgfpathclose%
\pgfusepath{stroke,fill}%
}%
\begin{pgfscope}%
\pgfsys@transformshift{2.528485in}{0.696000in}%
\pgfsys@useobject{currentmarker}{}%
\end{pgfscope}%
\end{pgfscope}%
\begin{pgfscope}%
\pgfpathrectangle{\pgfqpoint{0.800000in}{0.528000in}}{\pgfqpoint{4.960000in}{3.696000in}}%
\pgfusepath{clip}%
\pgfsetbuttcap%
\pgfsetroundjoin%
\definecolor{currentfill}{rgb}{0.000000,0.000000,0.000000}%
\pgfsetfillcolor{currentfill}%
\pgfsetlinewidth{1.003750pt}%
\definecolor{currentstroke}{rgb}{0.000000,0.000000,0.000000}%
\pgfsetstrokecolor{currentstroke}%
\pgfsetdash{}{0pt}%
\pgfsys@defobject{currentmarker}{\pgfqpoint{-0.041667in}{-0.041667in}}{\pgfqpoint{0.041667in}{0.041667in}}{%
\pgfpathmoveto{\pgfqpoint{0.000000in}{-0.041667in}}%
\pgfpathcurveto{\pgfqpoint{0.011050in}{-0.041667in}}{\pgfqpoint{0.021649in}{-0.037276in}}{\pgfqpoint{0.029463in}{-0.029463in}}%
\pgfpathcurveto{\pgfqpoint{0.037276in}{-0.021649in}}{\pgfqpoint{0.041667in}{-0.011050in}}{\pgfqpoint{0.041667in}{0.000000in}}%
\pgfpathcurveto{\pgfqpoint{0.041667in}{0.011050in}}{\pgfqpoint{0.037276in}{0.021649in}}{\pgfqpoint{0.029463in}{0.029463in}}%
\pgfpathcurveto{\pgfqpoint{0.021649in}{0.037276in}}{\pgfqpoint{0.011050in}{0.041667in}}{\pgfqpoint{0.000000in}{0.041667in}}%
\pgfpathcurveto{\pgfqpoint{-0.011050in}{0.041667in}}{\pgfqpoint{-0.021649in}{0.037276in}}{\pgfqpoint{-0.029463in}{0.029463in}}%
\pgfpathcurveto{\pgfqpoint{-0.037276in}{0.021649in}}{\pgfqpoint{-0.041667in}{0.011050in}}{\pgfqpoint{-0.041667in}{0.000000in}}%
\pgfpathcurveto{\pgfqpoint{-0.041667in}{-0.011050in}}{\pgfqpoint{-0.037276in}{-0.021649in}}{\pgfqpoint{-0.029463in}{-0.029463in}}%
\pgfpathcurveto{\pgfqpoint{-0.021649in}{-0.037276in}}{\pgfqpoint{-0.011050in}{-0.041667in}}{\pgfqpoint{0.000000in}{-0.041667in}}%
\pgfpathclose%
\pgfusepath{stroke,fill}%
}%
\begin{pgfscope}%
\pgfsys@transformshift{4.031515in}{0.696000in}%
\pgfsys@useobject{currentmarker}{}%
\end{pgfscope}%
\end{pgfscope}%
\begin{pgfscope}%
\pgfpathrectangle{\pgfqpoint{0.800000in}{0.528000in}}{\pgfqpoint{4.960000in}{3.696000in}}%
\pgfusepath{clip}%
\pgfsetbuttcap%
\pgfsetroundjoin%
\definecolor{currentfill}{rgb}{0.000000,0.000000,0.000000}%
\pgfsetfillcolor{currentfill}%
\pgfsetlinewidth{1.003750pt}%
\definecolor{currentstroke}{rgb}{0.000000,0.000000,0.000000}%
\pgfsetstrokecolor{currentstroke}%
\pgfsetdash{}{0pt}%
\pgfsys@defobject{currentmarker}{\pgfqpoint{-0.041667in}{-0.041667in}}{\pgfqpoint{0.041667in}{0.041667in}}{%
\pgfpathmoveto{\pgfqpoint{0.000000in}{-0.041667in}}%
\pgfpathcurveto{\pgfqpoint{0.011050in}{-0.041667in}}{\pgfqpoint{0.021649in}{-0.037276in}}{\pgfqpoint{0.029463in}{-0.029463in}}%
\pgfpathcurveto{\pgfqpoint{0.037276in}{-0.021649in}}{\pgfqpoint{0.041667in}{-0.011050in}}{\pgfqpoint{0.041667in}{0.000000in}}%
\pgfpathcurveto{\pgfqpoint{0.041667in}{0.011050in}}{\pgfqpoint{0.037276in}{0.021649in}}{\pgfqpoint{0.029463in}{0.029463in}}%
\pgfpathcurveto{\pgfqpoint{0.021649in}{0.037276in}}{\pgfqpoint{0.011050in}{0.041667in}}{\pgfqpoint{0.000000in}{0.041667in}}%
\pgfpathcurveto{\pgfqpoint{-0.011050in}{0.041667in}}{\pgfqpoint{-0.021649in}{0.037276in}}{\pgfqpoint{-0.029463in}{0.029463in}}%
\pgfpathcurveto{\pgfqpoint{-0.037276in}{0.021649in}}{\pgfqpoint{-0.041667in}{0.011050in}}{\pgfqpoint{-0.041667in}{0.000000in}}%
\pgfpathcurveto{\pgfqpoint{-0.041667in}{-0.011050in}}{\pgfqpoint{-0.037276in}{-0.021649in}}{\pgfqpoint{-0.029463in}{-0.029463in}}%
\pgfpathcurveto{\pgfqpoint{-0.021649in}{-0.037276in}}{\pgfqpoint{-0.011050in}{-0.041667in}}{\pgfqpoint{0.000000in}{-0.041667in}}%
\pgfpathclose%
\pgfusepath{stroke,fill}%
}%
\begin{pgfscope}%
\pgfsys@transformshift{5.534545in}{1.656000in}%
\pgfsys@useobject{currentmarker}{}%
\end{pgfscope}%
\end{pgfscope}%
\begin{pgfscope}%
\pgfsetrectcap%
\pgfsetmiterjoin%
\pgfsetlinewidth{0.803000pt}%
\definecolor{currentstroke}{rgb}{0.000000,0.000000,0.000000}%
\pgfsetstrokecolor{currentstroke}%
\pgfsetdash{}{0pt}%
\pgfpathmoveto{\pgfqpoint{0.800000in}{0.528000in}}%
\pgfpathlineto{\pgfqpoint{0.800000in}{4.224000in}}%
\pgfusepath{stroke}%
\end{pgfscope}%
\begin{pgfscope}%
\pgfsetrectcap%
\pgfsetmiterjoin%
\pgfsetlinewidth{0.803000pt}%
\definecolor{currentstroke}{rgb}{0.000000,0.000000,0.000000}%
\pgfsetstrokecolor{currentstroke}%
\pgfsetdash{}{0pt}%
\pgfpathmoveto{\pgfqpoint{5.760000in}{0.528000in}}%
\pgfpathlineto{\pgfqpoint{5.760000in}{4.224000in}}%
\pgfusepath{stroke}%
\end{pgfscope}%
\begin{pgfscope}%
\pgfsetrectcap%
\pgfsetmiterjoin%
\pgfsetlinewidth{0.803000pt}%
\definecolor{currentstroke}{rgb}{0.000000,0.000000,0.000000}%
\pgfsetstrokecolor{currentstroke}%
\pgfsetdash{}{0pt}%
\pgfpathmoveto{\pgfqpoint{0.800000in}{0.528000in}}%
\pgfpathlineto{\pgfqpoint{5.760000in}{0.528000in}}%
\pgfusepath{stroke}%
\end{pgfscope}%
\begin{pgfscope}%
\pgfsetrectcap%
\pgfsetmiterjoin%
\pgfsetlinewidth{0.803000pt}%
\definecolor{currentstroke}{rgb}{0.000000,0.000000,0.000000}%
\pgfsetstrokecolor{currentstroke}%
\pgfsetdash{}{0pt}%
\pgfpathmoveto{\pgfqpoint{0.800000in}{4.224000in}}%
\pgfpathlineto{\pgfqpoint{5.760000in}{4.224000in}}%
\pgfusepath{stroke}%
\end{pgfscope}%
\end{pgfpicture}%
\makeatother%
\endgroup%

\caption{The empirical probability that $\GMRES{\eps}{\no}{\nt}\leq 12$ for $\sigma = 1/k$\label{fig:prob-plot-1.0}}
\end{subfigure}

\begin{subfigure}{\textwidth}
    \centering
%% Creator: Matplotlib, PGF backend
%%
%% To include the figure in your LaTeX document, write
%%   \input{<filename>.pgf}
%%
%% Make sure the required packages are loaded in your preamble
%%   \usepackage{pgf}
%%
%% Figures using additional raster images can only be included by \input if
%% they are in the same directory as the main LaTeX file. For loading figures
%% from other directories you can use the `import` package
%%   \usepackage{import}
%% and then include the figures with
%%   \import{<path to file>}{<filename>.pgf}
%%
%% Matplotlib used the following preamble
%%   \usepackage{mleftright}
%%   \usepackage{fontspec}
%%   \setmainfont{DejaVuSerif.ttf}[Path=/home/owen/progs/firedrake-complex/firedrake/lib/python3.5/site-packages/matplotlib/mpl-data/fonts/ttf/]
%%   \setsansfont{DejaVuSans.ttf}[Path=/home/owen/progs/firedrake-complex/firedrake/lib/python3.5/site-packages/matplotlib/mpl-data/fonts/ttf/]
%%   \setmonofont{DejaVuSansMono.ttf}[Path=/home/owen/progs/firedrake-complex/firedrake/lib/python3.5/site-packages/matplotlib/mpl-data/fonts/ttf/]
%%
\begingroup%
\makeatletter%
\begin{pgfpicture}%
\pgfpathrectangle{\pgfpointorigin}{\pgfqpoint{5.462193in}{2.581603in}}%
\pgfusepath{use as bounding box, clip}%
\begin{pgfscope}%
\pgfsetbuttcap%
\pgfsetmiterjoin%
\definecolor{currentfill}{rgb}{1.000000,1.000000,1.000000}%
\pgfsetfillcolor{currentfill}%
\pgfsetlinewidth{0.000000pt}%
\definecolor{currentstroke}{rgb}{1.000000,1.000000,1.000000}%
\pgfsetstrokecolor{currentstroke}%
\pgfsetdash{}{0pt}%
\pgfpathmoveto{\pgfqpoint{-0.000000in}{0.000000in}}%
\pgfpathlineto{\pgfqpoint{5.462193in}{0.000000in}}%
\pgfpathlineto{\pgfqpoint{5.462193in}{2.581603in}}%
\pgfpathlineto{\pgfqpoint{-0.000000in}{2.581603in}}%
\pgfpathclose%
\pgfusepath{fill}%
\end{pgfscope}%
\begin{pgfscope}%
\pgfsetbuttcap%
\pgfsetmiterjoin%
\definecolor{currentfill}{rgb}{1.000000,1.000000,1.000000}%
\pgfsetfillcolor{currentfill}%
\pgfsetlinewidth{0.000000pt}%
\definecolor{currentstroke}{rgb}{0.000000,0.000000,0.000000}%
\pgfsetstrokecolor{currentstroke}%
\pgfsetstrokeopacity{0.000000}%
\pgfsetdash{}{0pt}%
\pgfpathmoveto{\pgfqpoint{0.677193in}{0.521603in}}%
\pgfpathlineto{\pgfqpoint{5.327193in}{0.521603in}}%
\pgfpathlineto{\pgfqpoint{5.327193in}{2.446603in}}%
\pgfpathlineto{\pgfqpoint{0.677193in}{2.446603in}}%
\pgfpathclose%
\pgfusepath{fill}%
\end{pgfscope}%
\begin{pgfscope}%
\pgfsetbuttcap%
\pgfsetroundjoin%
\definecolor{currentfill}{rgb}{0.000000,0.000000,0.000000}%
\pgfsetfillcolor{currentfill}%
\pgfsetlinewidth{0.803000pt}%
\definecolor{currentstroke}{rgb}{0.000000,0.000000,0.000000}%
\pgfsetstrokecolor{currentstroke}%
\pgfsetdash{}{0pt}%
\pgfsys@defobject{currentmarker}{\pgfqpoint{0.000000in}{-0.048611in}}{\pgfqpoint{0.000000in}{0.000000in}}{%
\pgfpathmoveto{\pgfqpoint{0.000000in}{0.000000in}}%
\pgfpathlineto{\pgfqpoint{0.000000in}{-0.048611in}}%
\pgfusepath{stroke,fill}%
}%
\begin{pgfscope}%
\pgfsys@transformshift{0.888557in}{0.521603in}%
\pgfsys@useobject{currentmarker}{}%
\end{pgfscope}%
\end{pgfscope}%
\begin{pgfscope}%
\definecolor{textcolor}{rgb}{0.000000,0.000000,0.000000}%
\pgfsetstrokecolor{textcolor}%
\pgfsetfillcolor{textcolor}%
\pgftext[x=0.888557in,y=0.424381in,,top]{\color{textcolor}\sffamily\fontsize{10.000000}{12.000000}\selectfont \(\displaystyle 10\)}%
\end{pgfscope}%
\begin{pgfscope}%
\pgfsetbuttcap%
\pgfsetroundjoin%
\definecolor{currentfill}{rgb}{0.000000,0.000000,0.000000}%
\pgfsetfillcolor{currentfill}%
\pgfsetlinewidth{0.803000pt}%
\definecolor{currentstroke}{rgb}{0.000000,0.000000,0.000000}%
\pgfsetstrokecolor{currentstroke}%
\pgfsetdash{}{0pt}%
\pgfsys@defobject{currentmarker}{\pgfqpoint{0.000000in}{-0.048611in}}{\pgfqpoint{0.000000in}{0.000000in}}{%
\pgfpathmoveto{\pgfqpoint{0.000000in}{0.000000in}}%
\pgfpathlineto{\pgfqpoint{0.000000in}{-0.048611in}}%
\pgfusepath{stroke,fill}%
}%
\begin{pgfscope}%
\pgfsys@transformshift{2.297648in}{0.521603in}%
\pgfsys@useobject{currentmarker}{}%
\end{pgfscope}%
\end{pgfscope}%
\begin{pgfscope}%
\definecolor{textcolor}{rgb}{0.000000,0.000000,0.000000}%
\pgfsetstrokecolor{textcolor}%
\pgfsetfillcolor{textcolor}%
\pgftext[x=2.297648in,y=0.424381in,,top]{\color{textcolor}\sffamily\fontsize{10.000000}{12.000000}\selectfont \(\displaystyle 20\)}%
\end{pgfscope}%
\begin{pgfscope}%
\pgfsetbuttcap%
\pgfsetroundjoin%
\definecolor{currentfill}{rgb}{0.000000,0.000000,0.000000}%
\pgfsetfillcolor{currentfill}%
\pgfsetlinewidth{0.803000pt}%
\definecolor{currentstroke}{rgb}{0.000000,0.000000,0.000000}%
\pgfsetstrokecolor{currentstroke}%
\pgfsetdash{}{0pt}%
\pgfsys@defobject{currentmarker}{\pgfqpoint{0.000000in}{-0.048611in}}{\pgfqpoint{0.000000in}{0.000000in}}{%
\pgfpathmoveto{\pgfqpoint{0.000000in}{0.000000in}}%
\pgfpathlineto{\pgfqpoint{0.000000in}{-0.048611in}}%
\pgfusepath{stroke,fill}%
}%
\begin{pgfscope}%
\pgfsys@transformshift{3.706738in}{0.521603in}%
\pgfsys@useobject{currentmarker}{}%
\end{pgfscope}%
\end{pgfscope}%
\begin{pgfscope}%
\definecolor{textcolor}{rgb}{0.000000,0.000000,0.000000}%
\pgfsetstrokecolor{textcolor}%
\pgfsetfillcolor{textcolor}%
\pgftext[x=3.706738in,y=0.424381in,,top]{\color{textcolor}\sffamily\fontsize{10.000000}{12.000000}\selectfont \(\displaystyle 30\)}%
\end{pgfscope}%
\begin{pgfscope}%
\pgfsetbuttcap%
\pgfsetroundjoin%
\definecolor{currentfill}{rgb}{0.000000,0.000000,0.000000}%
\pgfsetfillcolor{currentfill}%
\pgfsetlinewidth{0.803000pt}%
\definecolor{currentstroke}{rgb}{0.000000,0.000000,0.000000}%
\pgfsetstrokecolor{currentstroke}%
\pgfsetdash{}{0pt}%
\pgfsys@defobject{currentmarker}{\pgfqpoint{0.000000in}{-0.048611in}}{\pgfqpoint{0.000000in}{0.000000in}}{%
\pgfpathmoveto{\pgfqpoint{0.000000in}{0.000000in}}%
\pgfpathlineto{\pgfqpoint{0.000000in}{-0.048611in}}%
\pgfusepath{stroke,fill}%
}%
\begin{pgfscope}%
\pgfsys@transformshift{5.115829in}{0.521603in}%
\pgfsys@useobject{currentmarker}{}%
\end{pgfscope}%
\end{pgfscope}%
\begin{pgfscope}%
\definecolor{textcolor}{rgb}{0.000000,0.000000,0.000000}%
\pgfsetstrokecolor{textcolor}%
\pgfsetfillcolor{textcolor}%
\pgftext[x=5.115829in,y=0.424381in,,top]{\color{textcolor}\sffamily\fontsize{10.000000}{12.000000}\selectfont \(\displaystyle 40\)}%
\end{pgfscope}%
\begin{pgfscope}%
\definecolor{textcolor}{rgb}{0.000000,0.000000,0.000000}%
\pgfsetstrokecolor{textcolor}%
\pgfsetfillcolor{textcolor}%
\pgftext[x=3.002193in,y=0.234413in,,top]{\color{textcolor}\sffamily\fontsize{10.000000}{12.000000}\selectfont \(\displaystyle k\)}%
\end{pgfscope}%
\begin{pgfscope}%
\pgfsetbuttcap%
\pgfsetroundjoin%
\definecolor{currentfill}{rgb}{0.000000,0.000000,0.000000}%
\pgfsetfillcolor{currentfill}%
\pgfsetlinewidth{0.803000pt}%
\definecolor{currentstroke}{rgb}{0.000000,0.000000,0.000000}%
\pgfsetstrokecolor{currentstroke}%
\pgfsetdash{}{0pt}%
\pgfsys@defobject{currentmarker}{\pgfqpoint{-0.048611in}{0.000000in}}{\pgfqpoint{0.000000in}{0.000000in}}{%
\pgfpathmoveto{\pgfqpoint{0.000000in}{0.000000in}}%
\pgfpathlineto{\pgfqpoint{-0.048611in}{0.000000in}}%
\pgfusepath{stroke,fill}%
}%
\begin{pgfscope}%
\pgfsys@transformshift{0.677193in}{1.484103in}%
\pgfsys@useobject{currentmarker}{}%
\end{pgfscope}%
\end{pgfscope}%
\begin{pgfscope}%
\definecolor{textcolor}{rgb}{0.000000,0.000000,0.000000}%
\pgfsetstrokecolor{textcolor}%
\pgfsetfillcolor{textcolor}%
\pgftext[x=0.510526in,y=1.431342in,left,base]{\color{textcolor}\sffamily\fontsize{10.000000}{12.000000}\selectfont \(\displaystyle 1\)}%
\end{pgfscope}%
\begin{pgfscope}%
\definecolor{textcolor}{rgb}{0.000000,0.000000,0.000000}%
\pgfsetstrokecolor{textcolor}%
\pgfsetfillcolor{textcolor}%
\pgftext[x=0.144134in,y=0.605638in,left,base,rotate=90.000000]{\color{textcolor}\sffamily\fontsize{10.000000}{12.000000}\selectfont Empirical probability that}%
\end{pgfscope}%
\begin{pgfscope}%
\definecolor{textcolor}{rgb}{0.000000,0.000000,0.000000}%
\pgfsetstrokecolor{textcolor}%
\pgfsetfillcolor{textcolor}%
\pgftext[x=0.364691in,y=0.670455in,left,base,rotate=90.000000]{\color{textcolor}\sffamily\fontsize{10.000000}{12.000000}\selectfont \(\displaystyle \mathrm{GMRES}\mleft(\epsilon, n^{(1)} n^{(2)}\mright) \leq 12\)}%
\end{pgfscope}%
\begin{pgfscope}%
\pgfpathrectangle{\pgfqpoint{0.677193in}{0.521603in}}{\pgfqpoint{4.650000in}{1.925000in}}%
\pgfusepath{clip}%
\pgfsetbuttcap%
\pgfsetroundjoin%
\pgfsetlinewidth{1.505625pt}%
\definecolor{currentstroke}{rgb}{0.000000,0.000000,0.000000}%
\pgfsetstrokecolor{currentstroke}%
\pgfsetdash{{5.550000pt}{2.400000pt}}{0.000000pt}%
\pgfpathmoveto{\pgfqpoint{0.888557in}{1.484103in}}%
\pgfpathlineto{\pgfqpoint{2.297648in}{1.484103in}}%
\pgfpathlineto{\pgfqpoint{3.706738in}{1.484103in}}%
\pgfpathlineto{\pgfqpoint{5.115829in}{1.484103in}}%
\pgfusepath{stroke}%
\end{pgfscope}%
\begin{pgfscope}%
\pgfpathrectangle{\pgfqpoint{0.677193in}{0.521603in}}{\pgfqpoint{4.650000in}{1.925000in}}%
\pgfusepath{clip}%
\pgfsetbuttcap%
\pgfsetroundjoin%
\definecolor{currentfill}{rgb}{0.000000,0.000000,0.000000}%
\pgfsetfillcolor{currentfill}%
\pgfsetlinewidth{1.003750pt}%
\definecolor{currentstroke}{rgb}{0.000000,0.000000,0.000000}%
\pgfsetstrokecolor{currentstroke}%
\pgfsetdash{}{0pt}%
\pgfsys@defobject{currentmarker}{\pgfqpoint{-0.041667in}{-0.041667in}}{\pgfqpoint{0.041667in}{0.041667in}}{%
\pgfpathmoveto{\pgfqpoint{0.000000in}{-0.041667in}}%
\pgfpathcurveto{\pgfqpoint{0.011050in}{-0.041667in}}{\pgfqpoint{0.021649in}{-0.037276in}}{\pgfqpoint{0.029463in}{-0.029463in}}%
\pgfpathcurveto{\pgfqpoint{0.037276in}{-0.021649in}}{\pgfqpoint{0.041667in}{-0.011050in}}{\pgfqpoint{0.041667in}{0.000000in}}%
\pgfpathcurveto{\pgfqpoint{0.041667in}{0.011050in}}{\pgfqpoint{0.037276in}{0.021649in}}{\pgfqpoint{0.029463in}{0.029463in}}%
\pgfpathcurveto{\pgfqpoint{0.021649in}{0.037276in}}{\pgfqpoint{0.011050in}{0.041667in}}{\pgfqpoint{0.000000in}{0.041667in}}%
\pgfpathcurveto{\pgfqpoint{-0.011050in}{0.041667in}}{\pgfqpoint{-0.021649in}{0.037276in}}{\pgfqpoint{-0.029463in}{0.029463in}}%
\pgfpathcurveto{\pgfqpoint{-0.037276in}{0.021649in}}{\pgfqpoint{-0.041667in}{0.011050in}}{\pgfqpoint{-0.041667in}{0.000000in}}%
\pgfpathcurveto{\pgfqpoint{-0.041667in}{-0.011050in}}{\pgfqpoint{-0.037276in}{-0.021649in}}{\pgfqpoint{-0.029463in}{-0.029463in}}%
\pgfpathcurveto{\pgfqpoint{-0.021649in}{-0.037276in}}{\pgfqpoint{-0.011050in}{-0.041667in}}{\pgfqpoint{0.000000in}{-0.041667in}}%
\pgfpathclose%
\pgfusepath{stroke,fill}%
}%
\begin{pgfscope}%
\pgfsys@transformshift{0.888557in}{1.484103in}%
\pgfsys@useobject{currentmarker}{}%
\end{pgfscope}%
\begin{pgfscope}%
\pgfsys@transformshift{2.297648in}{1.484103in}%
\pgfsys@useobject{currentmarker}{}%
\end{pgfscope}%
\begin{pgfscope}%
\pgfsys@transformshift{3.706738in}{1.484103in}%
\pgfsys@useobject{currentmarker}{}%
\end{pgfscope}%
\begin{pgfscope}%
\pgfsys@transformshift{5.115829in}{1.484103in}%
\pgfsys@useobject{currentmarker}{}%
\end{pgfscope}%
\end{pgfscope}%
\begin{pgfscope}%
\pgfsetrectcap%
\pgfsetmiterjoin%
\pgfsetlinewidth{0.803000pt}%
\definecolor{currentstroke}{rgb}{0.000000,0.000000,0.000000}%
\pgfsetstrokecolor{currentstroke}%
\pgfsetdash{}{0pt}%
\pgfpathmoveto{\pgfqpoint{0.677193in}{0.521603in}}%
\pgfpathlineto{\pgfqpoint{0.677193in}{2.446603in}}%
\pgfusepath{stroke}%
\end{pgfscope}%
\begin{pgfscope}%
\pgfsetrectcap%
\pgfsetmiterjoin%
\pgfsetlinewidth{0.000000pt}%
\definecolor{currentstroke}{rgb}{0.000000,0.000000,0.000000}%
\pgfsetstrokecolor{currentstroke}%
\pgfsetstrokeopacity{0.000000}%
\pgfsetdash{}{0pt}%
\pgfpathmoveto{\pgfqpoint{5.327193in}{0.521603in}}%
\pgfpathlineto{\pgfqpoint{5.327193in}{2.446603in}}%
\pgfusepath{}%
\end{pgfscope}%
\begin{pgfscope}%
\pgfsetrectcap%
\pgfsetmiterjoin%
\pgfsetlinewidth{0.803000pt}%
\definecolor{currentstroke}{rgb}{0.000000,0.000000,0.000000}%
\pgfsetstrokecolor{currentstroke}%
\pgfsetdash{}{0pt}%
\pgfpathmoveto{\pgfqpoint{0.677193in}{0.521603in}}%
\pgfpathlineto{\pgfqpoint{5.327193in}{0.521603in}}%
\pgfusepath{stroke}%
\end{pgfscope}%
\begin{pgfscope}%
\pgfsetrectcap%
\pgfsetmiterjoin%
\pgfsetlinewidth{0.000000pt}%
\definecolor{currentstroke}{rgb}{0.000000,0.000000,0.000000}%
\pgfsetstrokecolor{currentstroke}%
\pgfsetstrokeopacity{0.000000}%
\pgfsetdash{}{0pt}%
\pgfpathmoveto{\pgfqpoint{0.677193in}{2.446603in}}%
\pgfpathlineto{\pgfqpoint{5.327193in}{2.446603in}}%
\pgfusepath{}%
\end{pgfscope}%
\end{pgfpicture}%
\makeatother%
\endgroup%

\caption{The empirical probability that $\GMRES{\eps}{\no}{\nt}\leq 12$ for $\sigma = 1/k^2.$\label{fig:prob-plot-2.0}}
\end{subfigure}
\caption[The empirical probability that GMRES applied to a nearby-preconditioned linear system converges in at most 12 iterations.]{The empirical probability (calculated from 1000 realisations) that $\GMRES{\eps}{\no}{\nt}\leq 12$ for $k = 10, 20, 30, 40,$ where $R=12$, $\eps = 10^{-5}$, $\nso=1,$ and $\NLiDR{\no-\nt} \sim \Exp{\sigma}$ for different functional forms of $\sigma.$}
\end{figure}


\chapter{Multi-Level Monte Carlo for the Helmholtz Equation}\label{chap:mlmc}
\chaptermark{Multi-Level Monte Carlo}

\section{The Multi-Level Monte Carlo Complexity Theorem for the Helmholtz equation}\label{sec:comp}

%% In this \lcnamecref{sec:comp} we state and prove an abstract result on the convergence of multi-level Monte Carlo methods, laregly following the proof of \cite[Theorem 1]{ClGiScTe:11}. Our result is a generalisation of \cite[Theorem 1]{ClGiScTe:11} in the following three ways:
%% \ben
%% \item In \cite{ClGiScTe:11} it is assumed that the convergence of the approximate QoIs $\Qhl$, and the cost of producing samples of these QoIs, only depends on the parameter $\hl$ (where, in stochastic PDE applications, $\hl$ is the mesh size for the finite-element discretisation). However, in this work, we assume that the convergence and cost also depend on another parameter $k,$ and we make the dependence of the final computational cost of the MLMC method explicit in $k.$ In our application to the Helmholtz equation, $k$ will be the wavenumber of the problem.
%% \item In \cite{ClGiScTe:11} it is assumed that the approximating QoIS $\Qhl$ exist for all levels $l$. This corresponds to the finite-element solution of the PDE under investigation existing for all mesh sizes $h.$ Whilst this assumption is true for the stationary diffusion equation studied in \cite{ClGiScTe:11}, it is \emph{not} true for the Helmholtz equation that we study here. Therefore we make the additional assumption (\cref{ass:qoie} below) that $\Qhl$ only exists for sufficiently small $\hl.$
%% \item In \cite{ClGiScTe:11} the error $\eps$ incurred in the MLMC method is equally divided between the bias and the variance of the MLMC method (see the Proof of \cref{thm:mlmccomp3}). However, in this work we assume that there is a quantity $\splitting \in (0,1)$ (see \cref{ass:splittingbounds}), possibly dependent on $k$ that allows a vairable `split' of the error between the bias and the variance. Our main use of this is in\optodo{Insert refs once it's done}, where we use this variable splitting to compensate for the fact that to bound the (squared) bias error by $\eps^2/2$ would mean we take $\hL \lesssim k^{-1},$ but to ensure the finite-element solution $\uh$ exists, we must take $\hL \lesssim k^{-3/2}.$
%% \een
%% We now proceed to prove our abstract MLMC convergence result, comtaining the generalisations metioned above.

Let $\OFP$ be a probability space, and let $Q$ be a random variable\footnote{One can think of $Q$ as being $Q(u),$ where $u$ is the solution of some stochastic PDE.} on $\OFP$ such that $\EXP{Q} < \infty.$ We will refer to $Q$ as the \defn{quantity of interest} or QoI. In order to define the multi-level Monte Carlo (MLMC) method for estimating $\EXP{Q},$ we must also define the following quantities, following  \cite[Theorem 1]{ClGiScTe:11}. We assume there exist
\bit
\item A set of levels\footnote{One can think of $\hl$ as the mesh size associated with level $l$.} $\set{\hl}_{l=0}^L$ ($L$ to be chosen) such that $\hl =\frac{\hlmo}s$ for $l \geq 1.$
\item A set of random variables (that may or may not exist)\footnote{One can think of $\Qhtilde$ as $Q(\uh),$ where $\uhl$ is the finite-element solution of the PDE with mesh size $h  $.} $\set{\Qhtilde}_{h \in (0,1)}.$
  \eit

We denote the random variables $\Qhtilde,$ in order to simplify the notation for mesh dependence in what follows.

In order to do things for the Helmholtz equation, we use the following assumption:

\bas[Mesh conditions for existence and uniqueness]
There exists a measurable function $\Cmesh:\Omega\rightarrow \RRp$ and a function $\mesh:\RRp\rightarrow \RRp$ such that $\Qhtilde(\omega)$ exists (and satisfies the error bounds etc. below) if
\beqs
h \leq \Cmesh(\omega)\mesh(k).
\eeqs
Note that $\mesh(k)\rightarrow 0$ as $k\rightarrow \infty.$
\eas

Observe that for a given $k, \omega$ there is no guarantee that $\Qhtilde(\omega)$ exists. Therefore, we follow [Graham, Parkinson, Scheichl] and define
\beq\label{eq:hmaxomega}
\hmaxomega \de \Cmesh(\omega)\mesh(k).
\eeq
We then define
\beqs
\homega \de \max\set{h,\hmaxomega}
\eeqs
and subsequently define
\beqs
\Qh(\omega) \de \Qtildehomega.
\eeqs
That is, the random variable $\Qh$ is (thinking about things in terms of PDEs etc.) the QoI evaluated at the numerical solution, where that solution is taken on a mesh that is the finer of $h$ and $\hmaxomega$. This guarantees the QoI exists, and the error bounds below hold.

\bre[What is $\mesh(k)$?]
If nontrapping, $\mesh(k)=k^{-3/2}$. If trapping, more stringent, nothing proved in literature, but would expect to be similar to results for contant wavespeed.
\ere

With this setup in place, we define the following quantities.

We define the correction operators\optodo{You may be able to save some time computing these - if both $\hl$ and $\hlmo$ are larger that $\hmaxomega$, then the difference between them is zero.} between the levels by $\Yl \de \Qhl - \Qhlmo, l \geq 1,$ $\Yz = \Qhz.$ We let $\Ylhat$ be an unbiased estimator of $\Yl$, i.e., $\EXP{\Ylhat} = \EXP{\Yl}.$ In what follows $\Ylhat$ will be the Monte Carlo estimator
 \beqs
\Ylhat \de \frac1{\Nl}\sum_{i=1}^{\Nl} \Yli,
 \eeqs
 with $\Nl$ to be chosen, where $\Yli$ denotes independent samples of $\Yl$. Finally we are able to define the \defn{multi-level Monte Carlo estimator}
 \beqs
 \QhatMLhL \de \sum_{l=1}^L \Ylhat,
 \eeqs
 where the $\Ylhat$ are independent.

  The following assumptions
  % \lcnamecrefs{ass:coarse}
   will form the backbone of our analysis. They are a generalisation of the assumptions contained in \cite{ClGiScTe:11,ChScTe:13} for the MLMC method, the generalisation being that we assume that the quantities below depend not only on the levels $\hl$ but also on some additional parameter $k>1.$ When this theory is applied to the Helmholtz equation, $k$ will be the wavenumber of the Helmholtz equation.

%% The following assumption (which will be realised in a more concrete setting for the Helmholtz equation) concerns the existence of the approximating QoIs $\Qhl.$

%% \bas[Existence of $\Qhl$]\label{ass:qoie}
%% There exist $\Ccoarse,\coarseexp > 0$ with $\Ccoarse$ independent of $k$ such that if
%% \beqs
%% \hl \leq \Ccoarse k^{-\coarseexp},
%% \eeqs
%% then the QoI $\Qhl$ exists.
%% \eas

\bas[Convergence of numerical method]\label{ass:a}
There exist $\co, \alpha, \sigma> 0$, such that $\co$ is independent of $h$ and $k$, and
\beqs
\abs{\EXP{\Qh-Q}} \leq \co k^\sigma h^{\alpha}.
\eeqs
\eas

\bas[Variance of correction operators]\label{ass:b}
There exist $\ct, \beta, \tau > 0$, such that $\ct$ is independent of $h$ and $k,$ and
\beqs
\Vl \de \VAR{\Yl} \leq \ct k^\tau\hl^{\beta},
\eeqs  where $\VAR{\cdot}$ denotes variance.
\eas

\bas[Cost of one sample]\label{ass:costone}
There exist $\cthtilde, \gamma > 0$ such that $\cthtilde$ is independent of $h$ and $k$, and if $\Qhtilde(\omega)$ exists, then
\beqs
\Cost{\Qhtilde(\omega)} \leq \cthtilde(\omega) h^{-\gamma},
\eeqs
\eas

In order to obtain a nice expression for the cost of computing one sample of $\Qh,$ we require the following assumption on the coarse space:

\bas[Dependence of coarse space on $k$]\label{ass:coarse}
We let
\beqs
\hz = \Ccoarse \mesh(k).
\eeqs
for some chosen constant $\Ccoarse > 0.$
\eas

\ble[Expected cost of one sample]\label{lem:c}
If
\beq\label{eq:cass}
\cthtilde \in \LpO\text{ for some }p \geq 1 \tand 1/\Cmesh \in \LqgammaO \tfor q \text{ the conjugate exponent of } p,
\eeq
then
\beq\label{eq:singlecost}
\EXP{\Cost{\Qh}} \leq \cth h^{-\gamma},
\eeq
where $\cth = \NLoO{\cthtilde} + \Ccoarse^\gamma \NLpO{\cthtilde}\NLqgammaO{\Cmesh^{-1}}^\gamma.$
\ele

\bpf[Proof of \cref{lem:c}]
The proof follows closely that in \cite[Lemma 5.8]{GrPaSc:19}.

We have
\begin{align}
\EXP{\Cost{\Qh}} &= \int_{\Omega} \Cost{\Qhomegatilde(\omega)} \ddPPomega\nonumber\\
&\leq \int_\Omega \cthtilde(\omega) \homega^{-\gamma} \ddPPomega \text{ by \eqref{eq:singlecost}}\nonumber\\
&\leq \int_{\Omega} \cthtilde(\omega) \mleft(h^{-\gamma} +  \mleft(\hmaxomega\mright)^{-\gamma} \mright) \ddPPomega \text{ as } \homega = \max\set{h,\hmaxomega} \leq h + \hmaxomega\nonumber\\
&=\int_{\Omega} \cthtilde(\omega) \mleft(h^{-\gamma} + \Cmesh^{-\gamma} \mesh(k)^{-\gamma}\mright)\ddPPomega \text{ by \eqref{eq:hmaxomega}}\nonumber\\
= h^{-\gamma} \NLoO{\cthtilde} + \mesh(k)^{-\gamma}\EXP{\cthtilde\Cmesh^{-\gamma}}\label{eq:cfinal}
\end{align}
As the assumptions \eqref{eq:cass} hold, the result follows.
\epf
 
% We write $\Vl$ for $\VAR{\Yl}.$
 
 We want to determine the choices of $L$ and $\Nl, l = 0,\ldots,L,$ such that the root-mean-squared eror (RMSE)
 \beqs
 \err{\QhatMLhL} \de \mleft(\EXP{\mleft(\QhatMLhL - \EXP{Q}\mright)^2}\mright)^{\half}
 \eeqs
 satisfies $\err{\QhatMLhL} \leq \eps,$ for some pre-defined $\eps > 0.$

The proof of the main \lcnamecref{thm:mlmccomp} will require the following \lcnamecref{lem:sumbound}.

\ble\label{lem:sumbound}
If $L$ is given by
\beq\label{eq:Ldef}
L = \ceil{\frac1\alpha\log_{s}\mleft(\sqrt{2}\co  \Ccoarse^\alpha k^{\sigma}\mesh(k)^\alpha \eps^{-1}\mright)},
\eeq
then, for $s>1$ and $\delta \in \RR,$ we have the bound
\beq\label{eq:sumbound}
\sum_{l=0}^{L} s^{\delta l} \leq
\begin{cases}
L+1 & \tif \delta = 0,\\
\frac{\mleft(\sqrt{2}\co\mright)^{\frac\delta\alpha}\Ccoarse^{\delta}s^{\delta}}{1-s^{-\delta}}k^{\frac{\delta\sigma}{\alpha}}\mesh(k)^\delta\eps^{-\frac\delta\alpha} &\tif \delta >0\\
\frac{\mleft(\sqrt{2}\co\mright)^{\frac\delta\alpha}\Ccoarse^{\delta}}{1-s^{-\delta}}k^{\frac{\delta\sigma}{\alpha}}\mesh(k)^\delta\eps^{-\frac\delta\alpha}&\tif \delta < 0
\end{cases}
\eeq
\ele

\bpf[Proof of \cref{lem:sumbound}]
The proof follows that in \cite{ClGiScTe:11}. We first observe that, since $L$ is given by \eqref{eq:Ldef}, it follows that
\beq\label{eq:Lbounds}
\frac1\alpha\log_s\mleft(\sqrt{2}\co\Ccoarse^\alpha k^{\sigma}\mesh(k)^\alpha \eps^{-1}\mright) \leq L < \frac1\alpha\log_s\mleft(\sqrt{2}\co\Ccoarse^\alpha k^{\sigma}\mesh(k)^\alpha \eps^{-1}\mright) + 1.
\eeq
Rearranging \eqref{eq:Lbounds}, we obtain the bounds
\beq\label{eq:saLbounds}
\sqrt{2}\co \Ccoarse^\alpha k^{\sigma}\mesh(k)^\alpha\eps^{-1} \leq s^{\alpha L} < \sqrt{2}\co \Ccoarse^\alpha k^{\sigma}\mesh(k)^\alpha\eps^{-1}s^\alpha.
\eeq
If $\delta > 0,$ then we use the right-hand bound in \eqref{eq:saLbounds} to obtain
\beq\label{eq:sdLpos}
s^{\delta L} < \mleft(\sqrt{2}\co\mright)^{\frac\delta\alpha}\Cppw^{\delta}k^{\frac{\delta\sigma}{\alpha}}\mesh(k)^\delta\eps^{-\frac\delta\alpha}s^{\delta},
\eeq
and if $\delta < 0,$ we use the left-hand bound in \eqref{eq:saLbounds} to obtain
\beq\label{eq:sdLneg}
s^{\delta L} \leq \mleft(\sqrt{2}\co\mright)^{\frac\delta\alpha}\Cppw^{\delta}k^{\frac{\delta\sigma}{\alpha}}\mesh(k)^\delta\eps^{-\frac\delta\alpha}.
\eeq
We now observe that, for $\delta \neq 0,$
\begin{align}
\sum_{l=0}^L s^{\delta l} &= \frac{s^{\delta\mleft(L+1\mright)} -1}{s^{\delta}-1}\nonumber\\
&= \frac{s^{\delta L} - s^{-\delta}}{1-s^{-\delta}}\nonumber\\
&\leq \frac{s^{\delta L}}{1-s^{-\delta}},\label{eq:ssumbound}
\end{align}
since $s^{-\delta} > 0,$ as $s >0.$ Combining \eqref{eq:ssumbound} with \eqref{eq:sdLpos} and \eqref{eq:sdLneg}, we obtain \eqref{eq:sumbound} in the cases $\delta \neq 0.$ The case $\delta=0$ is straightforward.
\epf


%
 \paragraph{The nice case, where $k^{-\sigma/\alpha} \lesssim k^{-\coarseexp}.$}
\optodo{Might need todo something with the constants, as we need $\hL$ (as calculated) $< \hz.$ ensuring the constants are monotone is probably sufficient, as it'll just mean `for $\eps$ sufficiently small'.}
 The following theorem describes the computational effort needed to obtain RMSE $\leq \eps$. It is exactly the same as \cite[Theorem 1]{ClGiScTe:11}, but with the dependence on all the parameters explicit.%, and with some additional cases enumerated. %\Cref{thm:mlmccomp} contains more cases than in \cite[Theorem 1]{ClGiScTe:11} because \cite[Theorem 1]{ClGiScTe:11} makes the assumption throughout that $\alpha \geq 1/2\min\set{\beta,\gamma}.$ This assumption does not always hold for the Helmholtz equation (see the cases of a direct solver in 3-D below), however, examining the proof of \cite[Theorem 1]{ClGiScTe:11}  shows that in any given case, one only needs the assumption $\alpha \geq \beta/2$ or the assumption $\alpha \geq \gamma2$, never both at the same time. Therefore, for convenience, we explicitly state when these conditions are needed, and for completeness, we give the results when these conditions are violated. 

  The next two \lcnamecref{ass:powersnice}\optodo{plural} means that the restriction on the coarse space in \cref{ass:coarse} do not come into play,

 \bas[Epsilon sufficiently small]\label{ass:constants}
 Assume
 \beqs
\eps \leq \sqrt{2} \co \Ccoarse^{\alpha}.
 \eeqs
 \eas

 \bas\label{ass:powersnice}
 Suppose
 \beqs
\frac{\sigma}{\alpha} \geq \coarseexp.
 \eeqs
 \eas
 
 \bth[MLMC Complexity Theorem]\label{thm:mlmccomp}
If \cref{ass:constants,ass:powersnice} hold, $L$ is given by
\beq\label{eq:Lcond}
L = \ceil{\frac1\alpha\log_{s}\mleft(\sqrt{2}\co  \Ccoarse^\alpha k^{\sigma-\coarseexp\alpha} \eps^{-1}\mright)},
\eeq
that is,
\beqs
\hL \leq \mleft(\frac{\eps}{\sqrt{2}\co k^\sigma}\mright)^{\frac1\alpha},
\eeqs
and the number of samples on each computational level is given by
\beqs
\Nl = \ceil{\frac2{\eps^{2}} \mleft(\frac{\Vl}{\Cl}\mright)^{\half}\sum_{j=0}^{L} \mleft(\Vj\Cj\mright)^{\half}},
\eeqs
then computational effort $\CMLhL(\eps)$ required to obtain $\err{\QhatMLhL} \leq \eps$ satisfies the bounds
 
 \begin{numcases}{ \CMLhL(\eps) \lesssim}
 k^{\tau + \rho+\coarseexp\mleft(\gamma - \beta\mright)}\eps^{-2}\mleft(\log_s\mleft(\sqrt{2}\co\Cppw^\alpha k^{\sigma-\coarseexp\alpha} \eps^{-1}\mright)+2\alpha\mright)^2 +  k^{\rho +  \frac{\gamma\sigma}\alpha}\eps^{-\frac\gamma\alpha}
 & if $\beta = \gamma$,\label{eq:mlmchheq}\\ 
k^{\tau + \rho+\mleft(\gamma-\beta\mright)\frac\sigma\alpha}\eps^{-2+\mleft(\frac{\beta-\gamma}{\alpha}\mright)}
 +  k^{\rho +  \frac{\gamma\sigma}\alpha}\eps^{-\frac\gamma\alpha} & otherwise.\label{eq:mlmchhoth}
\end{numcases}
 \enth
 \optodo{Need to say why the latter two cases are the same - in one case $\gamma/\alpha$ dominates, and in the other case the other term dominates? (At least in the Cliffe et. al. set up)}
 \bpf[Proof of \cref{thm:mlmccomp}]
 \ednote{This isn't all the details of the proof, but the bits I've skipped over are exactly the same as those in {\cite{ClGiScTe:11}}.}
 
We first decompose the (squared) mean-squared error into the bias error and the sampling error:

\beqs
\errQhatMLhL^2 = \mleft(\EXP{\QhatMLhL} - \EXP{Q}\mright)^2 + \underbrace{\EXP{\mleft(\QhatMLhL - \EXP{\QhatMLhL}\mright)^2}}_{V\de},
\eeqs
the first term is the \emph{bias}, and the second term is the \emph{variance} of the estimator $\QhatMLhL.$ We now proceed to choose the parameters $L$ and $\Nl, l = 0,\ldots,L$ such that we can bound both the bias and the variance by $\eps^2/2.$

We first bound the bias, to do this, we only need to choose $L.$ One can show\ednote{As in {\cite{ClGiScTe:11}}} that the bias is equal to $\abs{\EXP{\QhL - Q}}^2.$ By \cref{ass:constants,ass:powersnice}, we don't need to worry about the coarse mesh restriction\optodo{Make this proper speak}. Therefore a sufficient condition for the bias to be $\leq \eps^2/2$ is (by \cref{ass:a})
\beqs
\co k^\sigma \hL^\alpha \leq \frac{\eps}{\sqrt{2}},
\eeqs
that is
\beq\label{eq:hLcond}
\hL \leq \mleft(\frac{\eps}{\sqrt{2}\co k^\sigma}\mright)^{\frac1\alpha}.
\eeq
\ednote{Observe that if $Q$ is the weighted $H^1$ norm, then we assume (see below for details) $\alpha=2$ and $\sigma=3,$ so we require $\hL \lesssim k^{-\frac32}.$ If we take $Q$ to be the $L^2$ norm, and assume $\alpha=2$ and $\sigma=2,$ then we only require $\hL \lesssim k^{-1}.$}

As $\hL = \hz s^{-L},$ it follows from \eqref{eq:hLcond} that a sufficient condition for the bias to be $\leq \eps^2/2$ is
\beq\label{eq:Lcondpart}
L = \ceil{\frac1\alpha\log_s\mleft(\sqrt{2}\co k^\sigma \hz^\alpha \eps^{-1}\mright)}.
\eeq
As $\hz = \Ccoarse k^{-\coarseexp},$ we can simplify \eqref{eq:Lcondpart} to obtain \eqref{eq:Lcond}.
% \beqs
% L = \ceil{\frac1\alpha\log_s\mleft(\sqrt{2}\co\Ccoarse^\alpha k^{\sigma-\coarseexp\alpha} \eps^{-1}\mright)}.
% \eeqs

We now seek to bound the variance. One can show\ednote{Again, as in \cite{ClGiScTe:11}} the variance $V = \sum_{l=0}^L \Nl^{-1} \Vl,$ and the cost is $\cC = \sum_{l=0}^L \Nl \Cl.$

To find the optimal number of samples per level (the values of $\Nl, l=0,\ldots,L$) we formulate this as an optimisation problem to find $\Nl$ that minimise $\cC$, subject to $V=\eps/2.$ This can be solved using a Lagrange multiplier as in \cite{Gi:15}, and we obtain
\beq\label{eq:Nl}
\Nl = \ceil{\frac2{\eps^{2}} \mleft(\frac{\Vl}{\Cl}\mright)^{\half}\sum_{j=0}^L \mleft(\Vj\Cj\mright)^{\half}}.
\eeq
\optodo{Check this is correct, should it be divided by the sum?}
We now just need to infer the computational complexity for MLMC with $L$ given by \eqref{eq:Lcond} and the $\Nl$ given by \eqref{eq:Nl}.

The computational complexity $\cC$ is given by
\begin{align}
\cC &= \sum_{l=0}^{L} \Cl \Nl\nonumber\\
&\leq \sum_{l=0}^L \Cl \mleft(\frac2{\eps^{2}} \mleft(\frac{\Vl}{\Cl}\mright)^{\half}\sum_{j=0}^L \mleft(\Vj\Cj\mright)^{\half} + 1\mright) \text{ (by \eqref{eq:Nl})}\nonumber\\
&= 2\eps^{-2}\mleft(\sum_{l=0}^L\mleft(\Vl\Cl\mright)^{\half}\mright)^2 + \sum_{l=0}^L \Cl\nonumber\\
&\leq 2 \ct \cth k^{\tau + \rho}\eps^{-2} \mleft(\sum_{l=0}^L \hl^{\frac{\beta-\gamma}2}\mright)^2 + \cth k^\rho \sum_{l=0}^L \hl^{-\gamma} \text{ (by \cref{ass:b,ass:c})}\nonumber\\
&= 2 \ct\cth k^{\tau + \rho}\eps^{-2}\hz^{\beta-\gamma}\mleft(\sum_{l=0}^L s^{-l\mleft(\frac{\beta-\gamma}2\mright)}\mright)^2 + \cth k^\rho \hz^{-\gamma} \sum_{l=0}^L s^{\gamma l} \text{ (by definition of } \hl\text{ )}\nonumber\\
&=2 \ct\cth \Cppw^{\beta-\gamma}k^{\tau + \rho+\coarseexp\mleft(\gamma - \beta\mright)}\eps^{-2}\mleft(\sum_{l=0}^L s^{l\mleft(\frac{\gamma-\beta}2\mright)}\mright)^2 + \cth\Cppw^{-\gamma} k^{\rho + \gamma\coarseexp}  \sum_{l=0}^L s^{\gamma l} \text{ (by definition of } \hz\text{ )}\nonumber\\
&\leq2\ct\cth \Cppw^{\beta-\gamma}k^{\tau + \rho+\coarseexp\mleft(\gamma - \beta\mright)}\eps^{-2}\mleft(\sum_{l=0}^L s^{l\mleft(\frac{\gamma-\beta}2\mright)}\mright)^2 +  \frac{\mleft(\sqrt{2}\co\mright)^{\frac\gamma\alpha}\cth s^{\gamma}}{1-s^{-\gamma}}k^{\rho +  \frac{\gamma\sigma}\alpha}\eps^{-\frac\gamma\alpha} \text{ (since }\gamma>0,\text{ by \cref{lem:sumbound})}.\label{eq:complexitymidway}
\end{align}
To bound the sum in the first part of \eqref{eq:complexitymidway}, we must distinguish three cases based on $\gamma - \beta.$

If $\gamma=\beta,$ then \eqref{eq:complexitymidway} becomes (using \cref{lem:sumbound})
\begin{multline}
2 \ct\cth \Cppw^{\beta-\gamma}k^{\tau + \rho+\coarseexp\mleft(\gamma - \beta\mright)}\eps^{-2}\mleft(L+1\mright)^2 +  \frac{\mleft(\sqrt{2}\co\mright)^{\frac\gamma\alpha}\cth s^{\gamma}}{1-s^{-\gamma}}k^{\rho +  \frac{\gamma\sigma}\alpha}\eps^{-\frac\gamma\alpha}\\
\leq
2\ct\cth \Cppw^{\beta-\gamma}k^{\tau + \rho+\coarseexp\mleft(\gamma - \beta\mright)}\eps^{-2}\mleft(\frac1\alpha\log_s\mleft(\sqrt{2}\co\Cppw^\alpha k^{\sigma-\coarseexp\alpha} \eps^{-1}\mright)+2\mright)^2 +  \frac{\mleft(\sqrt{2}\co\mright)^{\frac\gamma\alpha}\cth s^{\gamma}}{1-s^{-\gamma}}k^{\rho +  \frac{\gamma\sigma}\alpha}\eps^{-\frac\gamma\alpha}\label{eq:gammaequal}
\end{multline}
by \eqref{eq:Lcond}.

If $\gamma > \beta$ then by \cref{lem:sumbound} \eqref{eq:complexitymidway} becomes
\beq
2 \ct\cth \Cppw^{\beta-\gamma}
\frac{\mleft(\sqrt{2}\co\mright)^{\mleft(\frac{\gamma-\beta}{\alpha}\mright)}\Cppw^{\mleft(\gamma-\beta\mright)}s^{\gamma-\beta}}{\mleft(1-s^{\mleft(\frac{\beta-\gamma}2\mright)}\mright)^{2}}k^{\tau + \rho+\mleft(\gamma-\beta\mright)\frac\sigma\alpha}\eps^{-2+\mleft(\frac{\beta-\gamma}{\alpha}\mright)}
 +  \frac{\mleft(\sqrt{2}\co\mright)^{\frac\gamma\alpha}\cth s^{\gamma}}{1-s^{-\gamma}}k^{\rho +  \frac{\gamma\sigma}\alpha}\eps^{-\frac\gamma\alpha}.\label{eq:gammagtr}
\eeq
If $\gamma < \beta,$ then by \cref{lem:sumbound} \eqref{eq:complexitymidway} becomes
\beq
2 \ct\cth \Cppw^{\beta-\gamma}
\frac{\mleft(\sqrt{2}\co\mright)^{\mleft(\frac{\gamma-\beta}{\alpha}\mright)}\Cppw^{\mleft(\gamma-\beta\mright)}}{\mleft(1-s^{\mleft(\frac{\beta-\gamma}2\mright)}\mright)^{2}}k^{\tau + \rho+\mleft(\gamma-\beta\mright)\frac\sigma\alpha}\eps^{-2+\mleft(\frac{\beta-\gamma}{\alpha}\mright)}
 +  \frac{\mleft(\sqrt{2}\co\mright)^{\frac\gamma\alpha}\cth s^{\gamma}}{1-s^{-\gamma}}k^{\rho +  \frac{\gamma\sigma}\alpha}\eps^{-\frac\gamma\alpha},\label{eq:gammaless}
\eeq
the only difference from \eqref{eq:gammagtr} being the loss of the $s^{\gamma-\beta}$ term.

Removing all the terms that are not of interest from \eqref{eq:gammaequal}, \eqref{eq:gammagtr}, and \eqref{eq:gammaless}, we obtain \eqref{eq:mlmchheq} and \eqref{eq:mlmchhoth}.
\epf


The following theorem describes the computational effort needed to obtain RMSE $\leq \eps$. It is exactly the same as \cite[Theorem 1]{ClGiScTe:11}, but with the dependence on all the parameters explicit.%, and with some additional cases enumerated. %\Cref{thm:mlmccomp} contains more cases than in \cite[Theorem 1]{ClGiScTe:11} because \cite[Theorem 1]{ClGiScTe:11} makes the assumption throughout that $\alpha \geq 1/2\min\set{\beta,\gamma}.$ This assumption does not always hold for the Helmholtz equation (see the cases of a direct solver in 3-D below), however, examining the proof of \cite[Theorem 1]{ClGiScTe:11}  shows that in any given case, one only needs the assumption $\alpha \geq \beta/2$ or the assumption $\alpha \geq \gamma2$, never both at the same time. Therefore, for convenience, we explicitly state when these conditions are needed, and for completeness, we give the results when these conditions are violated. 

%% The following \lcnamecref{ass:constants3} will ensure that \cref{ass:qoie} is satisfied.

%%  \bas[$\eps$ sufficiently small]\label{ass:constants3}
%%  Assume
%%  \beqs
%% \eps \leq \sqrt{2} \co \Ccoarse^{\alpha} k^{\sigma-\coarseexp\alpha}.
%%  \eeqs
%%  \eas

\paragraph{Notation}

$\Cl \de \cth\hl^{-\gamma}$

$\cC \de \EXP{\Cost{\QhatMLhL}}$

\bas[Assumptions on $\eps$ and $k$ to make things nicer]\label{ass:epsk}
\beqs
\eps \leq \min\set{\frac{\sqrt{2}\co\Ccoarse^\alpha}{s^{2\alpha}},\frac1{\sqrt{2}\co\Ccoarse^\alpha}},
\eeqs
and
\beqs
k^\sigma \mesh(k)^\alpha \geq 1.
\eeqs
\eas

\bth[MLMC Complexity Theorem]\label{thm:mlmccomp}
If \cref{ass:epsk} holds,  $L$ is given by \eqref{eq:Ldef}
that is,
\beq\label{eq:hLcond}
\hL \leq \mleft(\frac\eps{\sqrt{2}\co k^{\sigma}}\mright)^{\frac1\alpha},
\eeq
and the number of samples on each computational level is given by
\beqs
\Nl = \ceil{\frac2{\eps^{2}} \mleft(\frac{\Vl}{\Cl}\mright)^{\half}\sum_{j=0}^{L} \mleft(\Vj\Cj\mright)^{\half}},
\eeqs
then computational effort $\CMLhL(\eps)$ required to obtain $\err{\QhatMLhL} \leq \eps$ satisfies the bounds
 
 \begin{numcases}{ \CMLhL(\eps) \lesssim}
k^\tau \mleft(\log\mleft(\frac{k^\sigma \mesh(k)^\alpha}\eps\mright)\mright)^2 \eps^{-2} + k^{\frac{\gamma\sigma}\alpha} \eps^{-\frac\gamma\alpha}  & if $\beta = \gamma$,\label{eq:mlmchheq}\\ 
k^{\tau + \mleft(\gamma-\beta\mright)\frac\sigma\alpha} \eps^{-2-\frac{\gamma-\beta}{\alpha}} + k^{\frac{\gamma\sigma}{\alpha}}\eps^{-\frac\gamma\alpha} & otherwise.\label{eq:mlmchhoth}
\end{numcases}
 \enth
 \bpf[Proof of \cref{thm:mlmccomp}]
We first decompose the (squared) mean-squared error into the bias error and the sampling error:

\beqs
\errQhatMLhL^2 = \mleft(\EXP{\QhatMLhL} - \EXP{Q}\mright)^2 + \underbrace{\EXP{\mleft(\QhatMLhL - \EXP{\QhatMLhL}\mright)^2}}_{V\de},
\eeqs
the first term is the \emph{bias}, and the second term is the \emph{variance} of the estimator $\QhatMLhL.$ We now proceed to choose the parameters $L$ and $\Nl, l = 0,\ldots,L$ such that we can bound both the bias and the variance by $\eps^2/2.$

We first bound the bias, to do this, we only need to choose $L.$ One can show that the bias is equal to $\abs{\EXP{\QhL - Q}}^2.$ Therefore a sufficient condition for the bias to be $\leq \eps^2/2$ is (by \cref{ass:a})
\beqs
\co k^\sigma \hL^\alpha \leq \frac{\eps}{\sqrt{2}},
\eeqs
that is, \eqref{eq:hLcond}. As $\hL = \hz s^{-L},$ it follows from \eqref{eq:hLcond} that a sufficient condition for the bias to be $\leq \eps^2/2$ is
\beq\label{eq:Lcondpart}
L = \ceil{\frac1\alpha\log_s\mleft(\sqrt{2}\co k^\sigma \hz^\alpha \eps^{-1}\mright)}.
\eeq
As $\hz = \Ccoarse \mesh(k),$ we can simplify \eqref{eq:Lcondpart} to obtain \eqref{eq:Ldef}.
% \beqs
% L = \ceil{\frac1\alpha\log_s\mleft(\sqrt{2}\co\Ccoarse^\alpha k^{\sigma-\coarseexp\alpha} \eps^{-1}\mright)}.
% \eeqs

We now seek to bound the variance. One can show the variance $V = \sum_{l=0}^L \Nl^{-1} \Vl,$ and the cost is: (following \cite{GrPaSc:19})
\begin{align}
\cC &= \EXP{\Cost{\QhatMLhL}}\nonumber\\
&\leq \sum_{l=0}^L \EXP{\Cost{\Ylhat}}\text{less equal as could save if you've had to over-refine on both levels in $\Ylhat$ - difference will be zero}\nonumber\\
&= \sum_{l=0}^L \sum_{i=1}^{\Nl} \EXP{\Cost{\Yli}}\nonumber\\
&\leq \sum_{l=0}^L \sum_{i=0}^{\Nl} \mleft(\EXP{\Cost{\Qhl}} + \EXP{\Cost{\Qhlmo}}\mright)\nonumber\\
&\leq \sum_{l=0}^L \Nl \mleft(\cth \hl^{-\gamma} + \cth \hlmo^{-\gamma}\mright)\nonumber\\
&=\sum_{l=0}^L \Nl\mleft(1+s^{-\gamma}\mright) \cth \hl^{-\gamma}\nonumber\\
&= \mleft(1+s^{-\gamma}\mright) \sum_{l=0}^L \Nl\Cl\label{eq:Cboundformin}
\end{align}

To find the optimal number of samples per level (the values of $\Nl, l=0,\ldots,L$) we formulate this as an optimisation problem to find $\Nl$ that minimise \eqref{eq:Cboundformin}, subject to $V=\eps/2.$ This can be solved using a Lagrange multiplier as in \cite{Gi:15}, and we obtain
\beq\label{eq:Nl}
\Nl = \ceil{\frac2{\eps^{2}} \mleft(\frac{\Vl}{\Cl}\mright)^{\half}\sum_{j=0}^L \mleft(\Vj\Cj\mright)^{\half}}.
\eeq
We now just need to infer the computational complexity for MLMC with $L$ given by \eqref{eq:Ldef} and the $\Nl$ given by \eqref{eq:Nl}.

The computational complexity $\cC$ is given by
\begin{align}
\cC &\leq \mleft(1+s^{-\gamma}\mright)\sum_{l=0}^{L} \Cl \Nl\nonumber\\
&\leq \mleft(1+s^{-\gamma}\mright)\sum_{l=0}^L \Cl \mleft(\frac2{\eps^{2}} \mleft(\frac{\Vl}{\Cl}\mright)^{\half}\sum_{j=0}^L \mleft(\Vj\Cj\mright)^{\half} + 1\mright) \text{ (by \eqref{eq:Nl})}\nonumber\\
&= 2\eps^{-2}\mleft(1+s^{-\gamma}\mright)\mleft(\sum_{l=0}^L\mleft(\Vl\Cl\mright)^{\half}\mright)^2 + \mleft(1+s^{-\gamma}\mright)\sum_{l=0}^L \Cl\nonumber\\
&= 2 \ct \cth \mleft(1+s^{-\gamma}\mright)k^{\tau}\eps^{-2} \mleft(\sum_{l=0}^L \hl^{\frac{\beta-\gamma}2}\mright)^2 + \cth \mleft(1+s^{-\gamma}\mright) \sum_{l=0}^L \hl^{-\gamma} \text{ (by \cref{ass:b,ass:c})}\nonumber\\
&= 2 \ct\cth \mleft(1+s^{-\gamma}\mright)k^{\tau}\eps^{-2}\hz^{\beta-\gamma}\mleft(\sum_{l=0}^L s^{l\mleft(\frac{\gamma-\beta}2\mright)}\mright)^2 + \cth\mleft(1+s^{-\gamma}\mright) \hz^{-\gamma} \sum_{l=0}^L s^{\gamma l} \text{ (by definition of } \hl\text{ )}\label{eq:complexitymidway}\\
%% &=2 \ct\cth \Cppw^{\beta-\gamma}k^{\tau + \rho+\coarseexp\mleft(\gamma - \beta\mright)}\eps^{-2}\mleft(\sum_{l=0}^L s^{l\mleft(\frac{\gamma-\beta}2\mright)}\mright)^2 + \cth\Cppw^{-\gamma} k^{\rho + \gamma\coarseexp}  \sum_{l=0}^L s^{\gamma l} \text{ (by definition of } \hz\text{ )}\nonumber\\
%% &\leq2\ct\cth \Cppw^{\beta-\gamma}k^{\tau + \rho+\coarseexp\mleft(\gamma - \beta\mright)}\eps^{-2}\mleft(\sum_{l=0}^L s^{l\mleft(\frac{\gamma-\beta}2\mright)}\mright)^2 +  \frac{\mleft(\sqrt{2}\co\mright)^{\frac\gamma\alpha}\cth s^{\gamma}}{1-s^{-\gamma}}k^{\rho +  \frac{\gamma\sigma}\alpha}\eps^{-\frac\gamma\alpha} \text{ (since }\gamma>0,\text{ by \cref{lem:sumbound})}.\label{eq:complexitymidway}
\end{align}

Using \cref{lem:sumbound}, the second term in \eqref{eq:complexitymidway} can be bounded (as $\gamma > 0$) by %(letting \csumdelta \de \mleft(\sqrt{2}\co\mright)^{\frac\delta\alpha} \Ccoarse^\delta / \mleft(1-s^{-\delta}\mright)$)
\beq\label{eq:firstterm}
\frac{\mleft(1+s^{-\gamma}\mright) \cth \hz^{-\gamma} \mleft(\sqrt{2}\co\mright)^{\frac\gamma\alpha} s^\gamma \Ccoarse^\gamma}{1-s^{-\gamma}} k^{\frac{\gamma\sigma}\alpha} \mesh(k)^\gamma \eps^{-\frac\gamma\alpha}
= \frac{\mleft(1+s^{-\gamma}\mright)\cth \mleft(\sqrt{2}\co\mright)^{\frac\gamma\alpha} s^\gamma}{1-s^{-\gamma}} k^{\frac{\gamma\sigma}\alpha} \eps^{-\frac\gamma\alpha}
\eeq

To bound the sum in the first part of \eqref{eq:complexitymidway}, we must distinguish three cases based on $\gamma - \beta.$


If $\gamma=\beta,$ then the first part of \eqref{eq:complexitymidway} becomes (using \cref{lem:sumbound})
\beq
2 \ct\cth \mleft(1+s^{-\gamma}\mright)k^{\tau}\eps^{-2}\mleft(L+1\mright)^2 \leq 2 \ct\cth \mleft(1+s^{-\gamma}\mright)k^{\tau}\eps^{-2}\mleft(\frac1\alpha \log_s \mleft(\frac{\sqrt{2} \co \Ccoarse^\alpha k^\sigma \mesh(k)^\alpha}\eps\mright)+2\mright)^2,
\label{eq:gammaequal}
\eeq
by \eqref{eq:Ldef}. We wish to simplify \eqref{eq:gammaequal}, so that it is of the form $\mathrm{Constant} \times \text{Terms involving } \eps \text{ and } k.$ To achieve this simplification, we use \cref{ass:epsk}. As $k^\sigma \mesh(k)^\alpha \geq 1$ and $\eps \leq \mleft(\sqrt{2} \co \Ccoarse^{\alpha}\mright)/s^{2\alpha},$ it follows that
\beqs
2 \leq \frac1\alpha \log_s \mleft(\frac{\sqrt{2} \co \Ccoarse^\alpha k^\sigma \mesh(k)^\alpha}\eps\mright),
\eeqs
and thus \eqref{eq:gammaequal} can be bounded by
\beq\label{eq:gammaequalpart1}
8 \ct\cth \mleft(1+s^{-\gamma}\mright)k^{\tau}\eps^{-2}\mleft(\frac1\alpha \log_s \mleft(\frac{\sqrt{2} \co \Ccoarse^\alpha k^\sigma \mesh(k)^\alpha}\eps\mright)\mright)^2.
\eeq
As $k^\sigma \mesh(k)^\alpha \geq 1$ and $\eps \leq 1/\mleft(\sqrt{2}\co\Ccoarse^\alpha\mright),$ we can bound \eqref{eq:gammaequalpart1} by (including a change of base in the logarithm)
\beq\label{eq:gammaequalfinal}
\frac{32 \ct\cth \mleft(1+s^{-\gamma}\mright)}{\alpha^2 \mleft(\log(s)\mright)^2} k^\tau \mleft(\log\mleft(\frac{k^\sigma \mesh(k)^\alpha}\eps\mright)\mright)^2.
\eeq

For simplicity in what follows, we define
\beqs
\csumdelta \de \frac{\mleft(\sqrt{2}\co\mright)^{\frac\delta\alpha}\Ccoarse^{\delta}}{1-s^{-\delta}}.
\eeqs

If $\gamma > \beta$ then by \cref{lem:sumbound} the first term in \eqref{eq:complexitymidway} becomes\optodo{Check if the below is right - it seems to be saying the complexity is independent of the coarse mesh}
\beq
\eps^{-2}2\ct\cth \mleft(1+s^{-\gamma}\mright) k^\tau \hz^{\beta-\gamma}\mleft(\csumgammambetat s^{\frac{\gamma-\beta}2} k^{\frac{\gamma-\beta}2\frac\sigma\alpha} \mesh(k)^{\frac{\gamma-\beta}2} \eps^{-\frac{\gamma-\beta}{2\alpha}}\mright)^2 = \Cgammagtrbeta k^{\tau + \mleft(\gamma-\beta\mright)\frac\sigma\alpha} \eps^{-2-\frac{\gamma-\beta}{\alpha}},\label{eq:gammagtr}
\eeq
where
\beqs
\Cgammagtrbeta \de 2\ct\cth\mleft(1+s^{-\gamma}\mright)\csumgammambetat^2 s^{\gamma-\beta} \Ccoarse^{\beta-\gamma}.
\eeqs
If $\gamma < \beta,$ then analagously the first term in \eqref{eq:complexitymidway} is
\beqs
\Cgammalessbeta k^{\tau + \mleft(\gamma-\beta\mright)\frac\sigma\alpha} \eps^{-2-\frac{\gamma-\beta}{\alpha}},
\eeqs
where
\beqs
\Cgammalessbeta \de \frac{\Cgammagtrbeta}{s^{\gamma-\beta}}.
\eeqs

We now combine \eqref{eq:firstterm}, \eqref{eq:gammaequalfinal}, \eqref{eq:gammagtr}, and \eqref{eq:gammaless} and supress all the constants to obtain the result.
%Removing all the terms that are not of interest from \eqref{eq:gammaequal}, \eqref{eq:gammagtr}, and \eqref{eq:gammaless}, we obtain \eqref{eq:mlmchheq} and \eqref{eq:mlmchhoth}.
\epf


%
\paragraph{The nasty case, where $k^{-\sigma/\alpha} \gtrsim k^{-\coarseexp}.$}

 \bas\label{ass:powersnasty}
 Suppose
 \beqs
\frac{\sigma}{\alpha} \leq \coarseexp.
 \eeqs
 \eas

  \bas[Epsilon sufficiently small]\label{ass:constantsnasty}
 Assume
 \beqs
\eps \leq \co \Ccoarse^{\alpha}.
 \eeqs
 \eas

\bth[MLMC Complexity Theorem]\label{thm:mlmccomp2}
Assume \cref{ass:powersnasty,ass:constantsnasty} hold. Assume $k \geq 1.$ If $L$ is given by
\beq\label{eq:Lcond2}
L = \ceil{\frac1\alpha\log_{s}\mleft(\sqrt{2}\co  \Ccoarse^\alpha k^{\sigma-\coarseexp\alpha} \eps^{-1}\mright)},
\eeq
that is,
\beqs
\hL \leq \mleft(\frac{\eps}{\sqrt{2}\co k^\sigma}\mright)^{\frac1\alpha},
\eeqs
and the number of samples on each computational level is given by
\beqs
\Nl = \ceil{\frac2{\eps^{2}} \mleft(\frac{\Vl}{\Cl}\mright)^{\half}\sum_{j=0}^{L} \mleft(\Vj\Cj\mright)^{\half}},
\eeqs
then computational effort $\CMLhL(\eps)$ required to obtain $\err{\QhatMLhL} \leq \eps$ satisfies the bounds
 
 \begin{numcases}{ \CMLhL(\eps) \lesssim}
 k^{\tau + \rho+\coarseexp\mleft(\gamma - \beta\mright)}\eps^{-2}\mleft(\log_s\mleft(\sqrt{2}\co\Cppw^\alpha k^{\sigma-\coarseexp\alpha} \eps^{-1}\mright)+2\alpha\mright)^2 +  k^{\rho +  \frac{\gamma\sigma}\alpha}\eps^{-\frac\gamma\alpha}
 & if $\beta = \gamma$,\label{eq:mlmchheq2}\\ 
k^{\tau + \rho+\mleft(\gamma-\beta\mright)\frac\sigma\alpha}\eps^{-2+\mleft(\frac{\beta-\gamma}{\alpha}\mright)}
 +  k^{\rho +  \frac{\gamma\sigma}\alpha}\eps^{-\frac\gamma\alpha} & otherwise.\label{eq:mlmchhoth2}
\end{numcases}
 \enth
 \optodo{Need to say why the latter two cases are the same - in one case $\gamma/\alpha$ dominates, and in the other case the other term dominates? (At least in the Cliffe et. al. set up)}

 \bpf[Proof of \cref{thm:mlmccomp}]
 \ednote{This isn't all the details of the proof, but the bits I've skipped over are exactly the same as those in {\cite{ClGiScTe:11}}.}
Do the bias-variance decomposition.  
 We now proceed to choose the parameters $L$ and $\Nl, l = 0,\ldots,L$, however, because the dominant term in this case\optodo{show this} is the \emph{Coarse restriction} we tentatively\ednote{I have no idea at this stage whether this will work} We bound the bias by $\eps^2k^{\sigma-\alpha\coarseexp}$ and the variance by $\eps^2\mleft(1-k^{\sigma-\alpha\coarseexp}\mright).$\optodo{This is all fine by the assumptions}

We first bound the bias, to do this, we only need to choose $L.$ One can show\ednote{As in {\cite{ClGiScTe:11}}} that the bias is equal to $\abs{\EXP{\QhL - Q}}^2.$ Therefore a sufficient condition for the bias to be $\leq \eps^2k^{\sigma-\alpha\coarseexp}$ is (by \cref{ass:a})
\beqs
\co k^\sigma \hL^\alpha \leq \eps k^{\frac{\sigma-\alpha\coarseexp}2},
\eeqs
that is
\beq\label{eq:hLcond2}
\hL \leq \mleft(\frac{\eps}{\sqrt{2}\co k^\sigma}\mright)^{\frac1\alpha}.
\eeq
\ednote{Observe that if $Q$ is the weighted $H^1$ norm, then we assume (see below for details) $\alpha=2$ and $\sigma=3,$ so we require $\hL \lesssim k^{-\frac32}.$ If we take $Q$ to be the $L^2$ norm, and assume $\alpha=2$ and $\sigma=2,$ then we only require $\hL \lesssim k^{-1}.$}

As $\hL = \hz s^{-L},$ it follows from \eqref{eq:hLcond2} that a sufficient condition for the bias to be $\leq \eps^2/2$ is
\beq\label{eq:Lcondpart2}
L = \ceil{\frac1\alpha\log_s\mleft(\sqrt{2}\co k^\sigma \hz^\alpha \eps^{-1}\mright)}.
\eeq
As $\hz = \Ccoarse k^{-\coarseexp},$ we can simplify \eqref{eq:Lcondpart2} to obtain \eqref{eq:Lcond2}.
% \beqs
% L = \ceil{\frac1\alpha\log_s\mleft(\sqrt{2}\co\Ccoarse^\alpha k^{\sigma-\coarseexp\alpha} \eps^{-1}\mright)}.
% \eeqs

We now seek to bound the variance. One can show\ednote{Again, as in \cite{ClGiScTe:11}} the variance $V = \sum_{l=0}^L \Nl^{-1} \Vl,$ and the cost is $\cC = \sum_{l=0}^L \Nl \Cl.$

To find the optimal number of samples per level (the values of $\Nl, l=0,\ldots,L$) we formulate this as an optimisation problem to find $\Nl$ that minimise $\cC$, subject to $V=\eps/2.$ This can be solved using a Lagrange multiplier as in \cite{Gi:15}, and we obtain
\beq\label{eq:Nl2}
\Nl = \ceil{\frac2{\eps^{2}} \mleft(\frac{\Vl}{\Cl}\mright)^{\half}\sum_{j=0}^L \mleft(\Vj\Cj\mright)^{\half}}.
\eeq
\optodo{Check this is correct, should it be divided by the sum?}
We now just need to infer the computational complexity for MLMC with $L$ given by \eqref{eq:Lcond2} and the $\Nl$ given by \eqref{eq:Nl2}.

The computational complexity $\cC$ is given by
\begin{align}
\cC &= \sum_{l=0}^{L} \Cl \Nl\nonumber\\
&\leq \sum_{l=0}^L \Cl \mleft(\frac2{\eps^{2}} \mleft(\frac{\Vl}{\Cl}\mright)^{\half}\sum_{j=0}^L \mleft(\Vj\Cj\mright)^{\half} + 1\mright) \text{ (by \eqref{eq:Nl2})}\nonumber\\
&= 2\eps^{-2}\mleft(\sum_{l=0}^L\mleft(\Vl\Cl\mright)^{\half}\mright)^2 + \sum_{l=0}^L \Cl\nonumber\\
&\leq 2 \ct \cth k^{\tau + \rho}\eps^{-2} \mleft(\sum_{l=0}^L \hl^{\frac{\beta-\gamma}2}\mright)^2 + \cth k^\rho \sum_{l=0}^L \hl^{-\gamma} \text{ (by \cref{ass:b,ass:c})}\nonumber\\
&= 2 \ct\cth k^{\tau + \rho}\eps^{-2}\hz^{\beta-\gamma}\mleft(\sum_{l=0}^L s^{-l\mleft(\frac{\beta-\gamma}2\mright)}\mright)^2 + \cth k^\rho \hz^{-\gamma} \sum_{l=0}^L s^{\gamma l} \text{ (by definition of } \hl\text{ )}\nonumber\\
&=2 \ct\cth \Cppw^{\beta-\gamma}k^{\tau + \rho+\coarseexp\mleft(\gamma - \beta\mright)}\eps^{-2}\mleft(\sum_{l=0}^L s^{l\mleft(\frac{\gamma-\beta}2\mright)}\mright)^2 + \cth\Cppw^{-\gamma} k^{\rho + \gamma\coarseexp}  \sum_{l=0}^L s^{\gamma l} \text{ (by definition of } \hz\text{ )}\nonumber\\
&\leq2\ct\cth \Cppw^{\beta-\gamma}k^{\tau + \rho+\coarseexp\mleft(\gamma - \beta\mright)}\eps^{-2}\mleft(\sum_{l=0}^L s^{l\mleft(\frac{\gamma-\beta}2\mright)}\mright)^2 +  \frac{\mleft(\sqrt{2}\co\mright)^{\frac\gamma\alpha}\cth s^{\gamma}}{1-s^{-\gamma}}k^{\rho +  \frac{\gamma\sigma}\alpha}\eps^{-\frac\gamma\alpha} \text{ (since }\gamma>0,\text{ by \cref{lem:sumbound})}.\label{eq:complexitymidway2}
\end{align}
To bound the sum in the first part of \eqref{eq:complexitymidway2}, we must distinguish three cases based on $\gamma - \beta.$

If $\gamma=\beta,$ then \eqref{eq:complexitymidway2} becomes (using \cref{lem:sumbound})
\begin{multline}
2 \ct\cth \Cppw^{\beta-\gamma}k^{\tau + \rho+\coarseexp\mleft(\gamma - \beta\mright)}\eps^{-2}\mleft(L+1\mright)^2 +  \frac{\mleft(\sqrt{2}\co\mright)^{\frac\gamma\alpha}\cth s^{\gamma}}{1-s^{-\gamma}}k^{\rho +  \frac{\gamma\sigma}\alpha}\eps^{-\frac\gamma\alpha}\\
\leq
2\ct\cth \Cppw^{\beta-\gamma}k^{\tau + \rho+\coarseexp\mleft(\gamma - \beta\mright)}\eps^{-2}\mleft(\frac1\alpha\log_s\mleft(\sqrt{2}\co\Cppw^\alpha k^{\sigma-\coarseexp\alpha} \eps^{-1}\mright)+2\mright)^2 +  \frac{\mleft(\sqrt{2}\co\mright)^{\frac\gamma\alpha}\cth s^{\gamma}}{1-s^{-\gamma}}k^{\rho +  \frac{\gamma\sigma}\alpha}\eps^{-\frac\gamma\alpha}\label{eq:gammaequal2}
\end{multline}
by \eqref{eq:Lcond2}.

If $\gamma > \beta$ then by \cref{lem:sumbound} \eqref{eq:complexitymidway2} becomes
\beq
2 \ct\cth \Cppw^{\beta-\gamma}
\frac{\mleft(\sqrt{2}\co\mright)^{\mleft(\frac{\gamma-\beta}{\alpha}\mright)}\Cppw^{\mleft(\gamma-\beta\mright)}s^{\gamma-\beta}}{\mleft(1-s^{\mleft(\frac{\beta-\gamma}2\mright)}\mright)^{2}}k^{\tau + \rho+\mleft(\gamma-\beta\mright)\frac\sigma\alpha}\eps^{-2+\mleft(\frac{\beta-\gamma}{\alpha}\mright)}
 +  \frac{\mleft(\sqrt{2}\co\mright)^{\frac\gamma\alpha}\cth s^{\gamma}}{1-s^{-\gamma}}k^{\rho +  \frac{\gamma\sigma}\alpha}\eps^{-\frac\gamma\alpha}.\label{eq:gammagtr2}
\eeq
If $\gamma < \beta,$ then by \cref{lem:sumbound} \eqref{eq:complexitymidway2} becomes
\beq
2 \ct\cth \Cppw^{\beta-\gamma}
\frac{\mleft(\sqrt{2}\co\mright)^{\mleft(\frac{\gamma-\beta}{\alpha}\mright)}\Cppw^{\mleft(\gamma-\beta\mright)}}{\mleft(1-s^{\mleft(\frac{\beta-\gamma}2\mright)}\mright)^{2}}k^{\tau + \rho+\mleft(\gamma-\beta\mright)\frac\sigma\alpha}\eps^{-2+\mleft(\frac{\beta-\gamma}{\alpha}\mright)}
 +  \frac{\mleft(\sqrt{2}\co\mright)^{\frac\gamma\alpha}\cth s^{\gamma}}{1-s^{-\gamma}}k^{\rho +  \frac{\gamma\sigma}\alpha}\eps^{-\frac\gamma\alpha},\label{eq:gammaless2}
\eeq
the only difference from \eqref{eq:gammagtr2} being the loss of the $s^{\gamma-\beta}$ term.

Removing all the terms that are not of interest from \eqref{eq:gammaequal2}, \eqref{eq:gammagtr2}, and \eqref{eq:gammaless2}, we obtain \eqref{eq:mlmchheq2} and \eqref{eq:mlmchhoth2}.
\epf



\subsection{The lemma in generality}

\ble\label{lem:sumboundnew}
If $L$ is given by
\beq\label{eq:Ldefgen}
L = \ceil{\Lconst\log_{s}\mleft( \func \eps^{-1}\mright)},
\eeq
for some $\func > 0,$ then, for $s>1$ and $\delta \in \RR,$ we have the bound
\beq\label{eq:sumboundgen}
\sum_{l=0}^{L} s^{\delta l} \leq
\begin{cases}
L+1 & \tif \delta = 0,\\
\frac{s^{\delta}}{1-s^{-\delta}}\func^{\delta\Lconst}\eps^{-\delta\Lconst} &\tif \delta >0\\
\frac{1}{1-s^{-\delta}}\func^{\delta\Lconst}\eps^{-\delta\Lconst}&\tif \delta < 0
\end{cases}
\eeq
\beq\label{eq:sumboundLmo}
\sum_{l=0}^{L} s^{\delta l} \leq
\begin{cases}
L & \tif \delta = 0,\\
\frac{s^{\delta}}{1-s^{-\delta}}\func^{\delta\Lconst}\eps^{-\delta\Lconst} &\tif \delta >0\\
\frac{1}{1-s^{-\delta}}\func^{\delta\Lconst}\eps^{-\delta\Lconst}&\tif \delta < 0
\end{cases}
\eeq
\optodo{Tidy}
\ele

\bpf[Proof of \cref{lem:sumboundnew}]
The proof follows that in \cite{ClGiScTe:11}. We first observe that, since $L$ is given by \eqref{eq:Ldefgen}, it follows that
\beq\label{eq:Lboundsgen}
\Lconst\log_s\mleft(\func \eps^{-1}\mright) \leq L < \Lconst\log_s\mleft(\func \eps^{-1}\mright) + 1.
\eeq
Rearranging \eqref{eq:Lboundsgen}, we obtain the bounds
\beq\label{eq:saLboundsgen}
\mleft( \func\eps^{-1}\mright)^{\alpha \Lconst} \leq s^{\alpha L} < \mleft( \func\eps^{-1}\mright)^{\alpha \Lconst}s^\alpha.
\eeq
If $\delta > 0,$ then we use the right-hand bound in \eqref{eq:saLboundsgen} to obtain
\beq\label{eq:sdLposgen}
s^{\delta L} < \func^{\delta\Lconst}\eps^{-\delta\Lconst}s^{\delta},
\eeq
and if $\delta < 0,$ we use the left-hand bound in \eqref{eq:saLboundsgen} to obtain
\beq\label{eq:sdLneg}
s^{\delta L} \leq \func^{\delta\Lconst}\eps^{-\delta\Lconst}.
\eeq
We now observe that, for $\delta \neq 0,$
\begin{align}
\sum_{l=0}^L s^{\delta l} &= \frac{s^{\delta\mleft(L+1\mright)} -1}{s^{\delta}-1}\nonumber\\
&= \frac{s^{\delta L} - s^{-\delta}}{1-s^{-\delta}}\nonumber\\
&\leq \frac{s^{\delta L}}{1-s^{-\delta}},\label{eq:ssumbound}
\end{align}
since $s^{-\delta} > 0,$ as $s >0.$ Combining \eqref{eq:ssumbound} with \eqref{eq:sdLposgen} and \eqref{eq:sdLneg}, we obtain \eqref{eq:sumboundgen} in the cases $\delta \neq 0.$ The case $\delta=0$ is straightforward.



\paragraph{For sum up to $L-1$}
The proof follows that in \cite{ClGiScTe:11}. We first observe that, since $L$ is given by \eqref{eq:Ldefgen}, it follows that
\beq\label{eq:Lmobounds}
\Lconst\log_s\mleft(\func \eps^{-1}\mright)-1 \leq L-1 < \Lconst\log_s\mleft(\func \eps^{-1}\mright).
\eeq
Rearranging \eqref{eq:Lmobounds}, we obtain the bounds
\beq\label{eq:saLmobounds}
\mleft( \func\eps^{-1}\mright)^{\alpha \Lconst}s^{-\alpha} \leq s^{\alpha (L-1)} < \mleft( \func\eps^{-1}\mright)^{\alpha \Lconst}.
\eeq
If $\delta > 0,$ then we use the right-hand bound in \eqref{eq:saLmobounds} to obtain
\beq\label{eq:sdLmopos}
s^{\delta (L-1)} < \func^{\delta\Lconst}\eps^{-\delta\Lconst}
\eeq
and if $\delta < 0,$ we use the left-hand bound in \eqref{eq:saLmobounds} to obtain
\beq\label{eq:sdLmoneg}
s^{\delta (L-1)} \leq \func^{\delta\Lconst}\eps^{-\delta\Lconst}s^{-\delta}.
\eeq
We now observe that, for $\delta \neq 0,$
\begin{align}
\sum_{l=0}^{L-1} s^{\delta l} &= \frac{s^{\delta\mleft(L\mright)} -1}{s^{\delta}-1}\nonumber\\
&= \frac{s^{\delta (L-1)} - s^{-\delta}}{1-s^{-\delta}}\nonumber\\
&\leq \frac{s^{\delta (L-1)}}{1-s^{-\delta}},\label{eq:ssumboundLmo}
\end{align}
since $s^{-\delta} > 0,$ as $s >0.$ Combining \eqref{eq:ssumboundLmo} with \eqref{eq:sdLmopos} and \eqref{eq:sdLmoneg}, we obtain \eqref{eq:sumboundLmo} in the cases $\delta \neq 0.$ The case $\delta=0$ is straightforward.
\epf


\appendix

\chapter{A Simple Example of an Oscillatory Bayesian Posterior}\label{app:osc}
We construct a simple 1-dimensional example to show that the posteriors arising in Bayesian inverse problems, and the objective functions arising in inverse problems, for the Helmholtz equation are typically oscillatory/multi-modal. Whilst such a statement is commonly made in the literature (see, e.g.,\optodo{Find some references}) and observed in practive (see, e.g.,\optodo{Find more references}), we do not know of an explicit construction of such a posterior, and so we give one for a simple, 1-dimensional Helmholtz problem here.

We consider the 1-dimensional homogeneous Helmholtz equation, posed on on the interval $(-L,0)$, where ultimately $L$ will be an unknown to be determined. We impose a Dirichlet boundary condition given by an incoming plane wave and a sound-soft scatterer at $-L,$ and an outgoing impedance boundary condition at 0. That is, we consider the PDE
\beq\label{eq:app1dhh}
u'' + k^2 u = 0 \tin (-L,0),
\eeq
\beq\label{eq:app1dibc}
u(L) = -\exp(ikL) \tat -L
\eeq
\beq\label{eq:app1ddbc}
u'(0) - iku(0) = 0 \tat 0.
\eeq

We now suppose that $L$ is unknown, that we have measured $u(0)$ (with experimental error), and we wish to know the posterior distribution of $L$. A similar argument to the one presented below would also give analagous results for deterministic inverse problems for $L$. We will show that the posterior distribution of $L$ is multi-modal (that is, it has multiple local maxima/minima) and oscillates on a length scale $2\pi/k.$

To proceed, we first derive an expression for $u(0)$ as a function of $L$. We will then use this expression to formulate the Bayesian posterior for $L$, given knowledge of $u(0).$

Solutions of \eqref{eq:app1dhh} are of the form
\beqs
A\exp(ikx) + B\exp(-ikx),
\eeqs
for some (complex) constants $A$ and $B$. The impedance boundary condition \eqref{eq:app1dibc} immediately implies $B=0$. Then applying the Dirichlet boundary condition \eqref{eq:app1ddbc}\optodo{Finish this later, and talk to someone about it.}

Setup:
\bit
\item Want to determine length of interval (proxy for distance to scatterer) in simple 1-D model
\item Incoming wave, gives DBC \@ $L$ - $\exp(ikx)$ for scattered wave
\item IBC \@ 0\optodo{Make sure you get the sign of the derivative correct!}
\item Helmholtz
\item think of varying $L$/consider BIP for $L$, and show that solution/posterior is an oscialltory/multi-modal function of $L$ on wavelength $2\pi/k$  
\eit

Problem:
\bit
\item For simplicity, Uniform prior on $L$\optodo{Make sure big enough to capture truth}
\item Derive expression for $u(0)$
\item Formulate likelihood (up to proportionality)
\item Show $u(0)$ and likelihood are oscillatory functions of $L$ on length $2\pi/k$
\eit


\chapter{Failure of Fredholm theory for the stochastic variational formulation of Helmholtz problems}\label{sec:federico}
%\chaptermark{Failure of Fredholm Theory}
The standard approach to proving existence and uniqueness of a (deterministic) Helmholtz BVP is to  show that the associated sesquilinear form satisfies a G\r{a}rding inequality, and then apply Fredholm theory to deduce that existence and uniqueness are equivalent; see, e.g., \cite[Theorem 4.10]{Mc:00}. This procedure relies on the fact that the inclusion $\HozDDR \hookrightarrow \LtDR$ is compact; see, e.g., \cite[Theorem 3.27]{Mc:00}.

As noted in \cref{sec:otherwork}, the analysis in \cite{FeLiLo:15} of \cref{prob:svsedp} for the Helmholtz Interior Impedance Problem mimics this approach and assums that $\LtOHoD$ is compactly contained  in $\LtOLtD,$ where $D$ is the spatial domain. Here we briefly show $\LtOHoD$ is \emph{not} compactly contained  in $\LtOLtD$ by giving an explicit example of a bounded sequence in  $\LtOHoD$ that has no convergent subsequence in $\LtOLtD.$ Necessary and sufficient conditions for a subset of $L^p\mleft(\mleft[0,T\mright];B\mright),$ for $B$ a Banach space, to be compact, can be found in \cite{Si:86}. In particular, \cite{Si:86} shows that a space $C$ being compactly contained in a space $B$ does not by itself imply $L^2\mleft(\mleft[0,T\mright];C\mright)$ is compactly contained in $L^2\mleft(\mleft[0,T\mright];B\mright).$

\begin{example} \label{ex:federico}
Let $(\Omega,\cF,\PP) = (\mleft[0,1\mright],\cB(\mleft[0,1\mright]),\lambda).$ Let $D$ be a compact subset of $\mathbb{R}^d.$ Since $\LtO$ is separable, it has an orthonormal basis, which we denote by $(f_m)_{m \in \NN}.$ Let $u_m \in  \LtOHoD$ be defined by $ u_m(\omega)(x) \de f_m(\omega), \tfa x \in D,$
i.e., for each value of $\omega,$ $u_m(\omega)$ is a constant function on $D$ and so $\NHoD{u_m(\omega)} = \NLtD{u_m(\omega)}.$ Then
\beqs
\NLtOHoD{u_m}^2 = \int_\Omega \NHoD{u_m(\omega)}^2 \dd\PP(\omega) = \lambda(D)^2\int_\Omega \abs{f_m(\omega)}^2 \dd\PP(\omega)= \NLtO{f_m}^2 \lambda(D)^2,
\eeqs
and so $u_m$ is a bounded sequence in $\LtOHoD.$ However, for $n \neq m,$ we have
\begin{align*}
\NLtOLtD{u_m-u_n}^2 &= \int_\Omega \NLtD{u_m(\omega)-u_n(\omega)}^2 \dd\PP(\omega)\\
&= \lambda(D)^2 \int_\Omega \abs{u_m(\omega) - u_n(\omega)}^2 \dd\PP(\omega) = \lambda(D)^2\NLtO{f_m-f_n}^2 = 2\lambda(D)^2
\end{align*}
%\beqs
%\NLtOLtD{u_m-u_n}^2 = \int_\Omega \NLtD{u_m(\omega)-u_n(\omega)}^2 \dd\PP(\omega) = \lambda(D)^2\NLtO{f_m-f_n}^2 = 2\lambda(D)^2
%\eeqs
if $n \neq m,$ since the $f_m$ form an orthonormal basis for $\LtD.$ Therefore $(u_m)_{m \in \NN}$ is bounded in $\LtOHoD$ but does not have a convergent subsequence in $\LtOLtD,$ and thus the inclusion of $\LtOHoD$ into $\LtOLtD$ cannot be compact.
\end{example}
 	

\chapter{Recap of basic material on measure theory and Bochner spaces}\label{app:mtBs}
Recall that here, and in the rest of this thesis, $\OFP$ is a complete probability space.

\section{Recap of measure theory results}

We first recall some results from measure theory, with our main reference \cite{Bo:07}. Even though \cite{Bo:07} mainly considers maps with image $\RR,$ the results we quote for more general images are straightforward generalisations of the results in \cite{Bo:07}.

\begin{definition}[Measurable map]\label{def:meas}
If $(\M,\FM)$ and $(N,\FN)$ are measurable spaces, we say that $f:\M\rightarrow N$ is measurable (with respect to $(\FM,\FN)$) if $f^{-1}(E) \in \FM$ for all $E \in \FN.$
\end{definition}



\bde[Borel {$\sigma$}-algebra]\label{def:borelsigma}
If $(S,\TS)$ is a topological space, the \defn{Borel $\sigma$-algebra} $\Borel{S}$ on $S$ is the $\sigma$-algebra generated by $\TS.$
\ede


If $V$ is any topological space (including a Hilbert, Banach, metric, or normed vector space) then we will always take the Borel $\sigma$-algebra on $V$ unless stated otherwise.


\ble[Continuous maps are measurable { \cite[Theorem 2.1.2]{Bo:07}}] \label{lem:contmeas}
Any continuous function between two topological spaces is measurable.
\ele





\ble[The composition of a measurable map with a continuous map is measurable {\cite[Text at the top of p. 146]{Bo:07}}]%{\cite[Text at top of p. 146]{Bo:07}}]
Let $(\M,\FM)$ be a measurable space and let $\mleft(S,\TS\mright)$ and $\mleft(T,\TopT\mright)$ be topological spaces. Let $f:M \rightarrow S$ be measurable and let $h : S \rightarrow T$ be continuous. Then $h \circ f$ is measurable.\label{lem:contplusmeas}
\ele

\bde[Product $\sigma$-algebra {\cite[Section IV.11]{Do:94}}]\label{def:prodsigma}
Let $\mleft(\M_1,\FM_1\mright),\ldots,\mleft(\M_m,\FM_m\mright)$ be measurable spaces. The \emph{product $\sigma$-algebra} $\M_1\otimes\cdots\otimes\M_m$ is defined as the $\sigma$-algebra generated by the set of measurable rectangles
$%\label{eq:measrectdef}
\set{R_{1} \times \cdots \times R_m  \st R_{1} \in \FM_1, \ldots, R_m \in \FM_m}.
$
\ede

\ble[Measurability of the Cartesian product of measurable functions]\label{lem:measprod}

\

\noindent Let $\mleft(\M_1,\FM_1\mright),\ldots,\mleft(\M_m,\FM_m\mright)$ be measurable spaces and $h_j:\Omega \rightarrow \M_j,\, j = 1,\ldots,m$ be measurable functions. Then the product map $\Prodf:\Omega \rightarrow \M_1 \times \cdots \times \M_m$ given by
%\beqs
$\Prodf(\omega) \de \mleft(h_1(\omega),\ldots,h_m(\omega)\mright)$
%\eeqs
is measurable with respect to $\mleft(\cF,\FM_1 \otimes \cdots\otimes\FM_m\mright).$
\ele

\bpf[Sketch proof of \cref{lem:measprod}]

Let $\measrectmany{\FM_1,\ldots,\FM_m}$ denote the set of measurable rectangles, as in \cref{def:prodsigma}. Let %Define the set $\cP$ by
%\beqs
$\cP \de \set{C \subseteq \M_1 \times \cdots \times \M_m \st \Prodf^{-1}\mleft(C\mright) \in \cF}.$
%\eeqs
The proof of the lemma consists of the following straightforward steps, whose proofs are omitted:
%\ben
%\item 
(i) Show $\measrectmany{\FM_1,\ldots,\FM_m} \subseteq \cP.$
%\item 
(ii) Show $\cP$ is a $\sigma$-algebra.
%\item 
(iii) Deduce $\FM_1 \otimes \cdots \otimes \FM_m \subseteq \cP$ (since $\FM_1 \otimes \cdots \otimes \FM_m$ is generated by measurable rectangles).
%\item 
(iv) Conclude $\Prodf$ is measurable with respect to $\mleft(\cF,\FM_1\otimes\cdots\otimes\FM_m\mright).$
%\een
\epf

\ble[The product of Borel $\sigma$-algebras is the Borel $\sigma$-algebra of the product {\cite[Lemma 6.2.1 (i)]{Bo:07}}]
Let $H_1,H_2$ be Hausdorff spaces and let $H_2$ have a countable base (e.g.~$H_2$ could be a separable metric space). Then $\Borel{H_1\times H_2} = \Borel{H_1} \otimes \Borel{H_2},$ where $\Borel{H_1\times H_2}$ is the Borel $\sigma$-algebra of the product topology on $H_1\times H_2.$\label{lem:bogachev}
\ele

\section{Recap of results on Bochner spaces}

We now recap the theory of Bochner spaces, using \cite{DiUh:77} as our main reference. In what follows the space $V$ is always a Banach space.

\bde[Simple function]
A function $v:\Omega \rightarrow V$ is \defn{simple} if there exist $v_1,\ldots,v_m \in V$ and $E_1,\ldots,E_m \in \cF$ such that
%\beqs
$v = \sum_{i=1}^m v_i \chi_{E_{i}},$
%\eeqs
where $\chi_{E_{i}}$ is the indicator function on $E_{i}.$
\ede

\bde[Strongly measurable]\label{def:strongmeas}
A function $v:\Omega \rightarrow V$ is \defn{strongly measurable}\footnote{In \cite{DiUh:77} the authors use the term \defn{$\mu$-measurable} instead of \defn{strongly measurable} (where $\mu$ is the measure on the domain of the functions under consideration).} 
if there exists a sequence of simple functions $(\vn)_{n \in \NN}$ such that
%\beqs
$\lim_{n \rightarrow \infty} \NV{\vn - v} = 0$, %\quad
$\PP$-almost everywhere.
%\eeqs
\ede

%Note that \cite{DiUh:77} uses the term \defn{$\mu$-measurable} instead of \defn{strongly measurable} (where $\mu$ is the measure on the domain of the functions under consideration).


\bde[Bochner integrable {\cite[p. 49]{DiUh:77}}]
A strongly measurable function $v:\Omega \rightarrow V$ is called \defn{Bochner integrable} if there exists a sequence of simple functions $(\vn)_{n \in \NN}$ such that
\beqs
\lim_{n \rightarrow \infty} \int_{\Omega} \NV{\vn(\omega) - v(\omega)} \dd\PP(\omega) = 0.
\eeqs
\ede

\bth[Condition for Bochner integrability {\cite[Theorem II.2.2]{DiUh:77}}]\label{thm:bochnercond}

\

\noindent A strongly meas\-ur\-able function $v:\Omega \rightarrow V$ is Bochner integrable if and only if $\int_\Omega \NV{v} \dd\PP < \infty.$
\enth

\bco[Sufficient condition for Bochner integrability]\label{cor:bochnersimple}
Let $p \geq 1.$ If a strongly measurable function $v:\Omega \rightarrow V$ has $\int_\Omega \NV{v}^p \dd\PP < \infty,$ then $v$ is Bochner integrable.
\eco



\bde[Bochner norm]\label{def:bochnernorm}
For a Bochner integrable function $v:\Omega\rightarrow V,$ let
\beqs
\NLpOV{v} \de \mleft(\int_\Omega \NV{v(\omega)}^p \dd\PP(\omega)\mright)^{1/p}, \,1 \leq p < \infty, \,\,\text{and}\,\,\NLiOV{v} \de \esssup_{\omega \in \Omega} \NV{v(\omega)}.
\eeqs
\ede

\bde[Bochner space]\label{def:bochnerspace}
Let $1\leq p \leq \infty.$ Then
\beqs
\LpOV \de \set{v:\Omega\rightarrow V \st v \text{ is Bochner integrable,}\,\NLpOV{v} < \infty}.
\eeqs
\ede

\bde[Complete probability space]
A probability space $\OFP$ is complete if for every $\Eo \in \cF$ with $\PP(\Eo)=0,$ the inclusion $\Et \subseteq \Eo$ implies that $\Et \in \cF.$
\ede

\bde[Separable space]
A topological space is \defn{separable} if it contains a countable, dense subset.
\ede

\bde[$\sigma$-finite]
A probability space $\OFP$ is \defn{$\sigma$-finite} if there exist $E_1,E_2,\ldots \in \cF$ such that $\Omega = \cup_{m=1}^\infty E_m.$
\ede

\bth[Pettis measurability theorem {\cite[Proposition 2.15]{Ry:02}}]\label{thm:pettis}
Let $\OFP$ be complete and $\sigma$-finite. The following are equivalent for a function $v:\Omega \rightarrow V$:
%\ben
%\item 
(i) $v$ is  strongly measurable,
%\item 
(ii) $v$ is measurable and $\PP$-essentially separably valued.
%\een
\enth

\bco[Equivalence of measurable and strongly measurable when the image is separable]
Let $\OFP$ be $\sigma$-finite. If $V$ is a separable Banach space, then a function $v:\Omega\rightarrow V$ is strongly measurable if, and only if, it is measurable.\label{cor:pettis}
\eco

\ble[The composition of a continuous map and a {$\PP$}-essentially separably valued map]
Let $\mleft(S,\TS\mright)$ and $\mleft(T,\TopT\mright)$ be topological spaces. If $f_1:\Omega \rightarrow S$ and $f_2:S\rightarrow T$ are such that $f_1$ is $\PP$-essentially separably valued and $f_2$ is continuous, then $f_2\circ f_1$ is $\PP$-essentially separably valued.\label{lem:esssep}
\ele

\bpf[Proof of \cref{lem:esssep}]
As $f_1$ is $\PP$-essentially separably valued, there exists $E \in \cF$ such that $\PP(E) = 1$ and $f_1(E) \subseteq G \subseteq S,$ where $G$ is separable. As $\ft$ is continuous, $\ft(G)$ is separable \cite[Theorem 16.4(a)]{Wi:70}. Therefore, since $\mleft(\ft \circ\fo\mright)(E) \subseteq \ft(G),$ it follows that $\ft\circ\fo$ is $\PP$-essentially separably valued.
\epf



\ble[The composition of a continuous map and a strongly measurable map]\label{lem:contplusstrong}

\

\noindent If $\Bo$ and $\Bt$ are Banach spaces and there exist $f_1:\Omega \rightarrow \Bo$ and $f_2:\Bo\rightarrow \Bt$ such that $f_1$ is strongly measurable and $f_2$ is continuous, then $f_2\circ f_1$ is strongly measurable.
\ele

\bpf[Proof of \cref{lem:contplusstrong}]
By \cref{thm:pettis}, $\fo$ is both measurable and $\PP$-essentially separably valued. Therefore we can apply \cref{lem:contplusmeas,lem:esssep} to conclude $\ft \circ \fo$ is both measurable and $\PP$-essentially separably valued. Hence by \cref{thm:pettis} $\ft\circ\fo$ is strongly measurable.
\epf

\ble[Zero in all integrals implies zero almost everywhere {\cite[Corollary II.2.5]{DiUh:77}}]\label{lem:gotoae}

\

\noindent If $\alpha$ is  Bochner integrable and  $\int_E \alpha(\omega) \dd\PP(\omega) = 0 $ for each $E \in \cF$ then $\alpha=0$ $\PP$-almost everywhere.
\ele

\ble[Cartesian product of $\PP$-essentially separably valued maps]\label{lem:prodsep}
Let 

\noindent $\mleft(\cCo,\Top{\cCo}\mright),\ldots,\mleft(\cCm,\Top{\cCm}\mright)$ be topological spaces, and let $s_j:\Omega\rightarrow\cCj,\,j=1,\ldots,m$ be $\PP$-essentially separably valued. Define $\cC \de \cCo \times \cdots \times \cCm$ and equip $\cC$ with the product topology. Then the map $f:\Omega\rightarrow \cC$ given by $s(\omega) \de \mleft(s_1(\omega),\ldots,s_m(\omega)\mright) $ is $\PP$-essentially separably valued.
\ele

The proof of \cref{lem:prodsep} is straightforward and omitted.


\chapter{Measurability of series expansions (used in \cref{sec:generating})}\label{app:meas}
Here we collect together results from measure theory that allow us to conclude in \cref{lem:seriesmeas} that the series expansions for $A$ and $n$ in \cref{sec:generating} are measurable. As mentioned in \cref{sec:generating}, the proof that the sum of measurable functions is measurable is standard, but we have not been able to find this result stated in the literature for this particular setting of mappings into a separable subspace of a general normed vector space.
\ble\label{lem:sepsum}
If $U$ is a separable normed vector space, $m \in \NN,$ and $\phi_j:\Omega\rightarrow U,$ $j=1,\ldots,m$  are measurable functions, then $\phi_1+\cdots+\phi_m : \Omega\rightarrow U$ is measurable.
\ele

\bpf[Sketch proof of \cref{lem:sepsum}]
By induction, it is sufficient to show the result for $m=2.$ We let $\Ballspace{U}{r}{v}$ denote the ball of radius $r>0$ about $v \in U$. To show $\phio+\phit$ is measurable, we let $v \in U, r>0$ and we show $\mleft(\phio+\phit\mright)^{-1}\mleft(\Ballspace{U}{r}{v}\mright) \in \cF.$ Let $\QQU$ denote a countable dense subset of $U,$ which exists as $U$ is separable. Let $\QQFF$ denote a countable dense subset of the field $\FF$, which exists as $\FF = \RR$ or $\CC.$

For $s \in \QQFF, q \in \QQU$ let
\beqs
\setsq = \set{\omega \in \Omega \st \NU{\phio(\omega)-\half v - q} < s} \cap \set{\omega \in \Omega \st \NU{\phit(\omega)-\half v + q} < r-s}.
\eeqs
We claim
\beq\label{eq:sumseteq}
\mleft(\phio+\phit\mright)^{-1}\mleft(\Ballspace{U}{r}{v}\mright) = \bigcup_{s \in \QQFF} \bigcup_{q \in \QQU} \setsq,
\eeq
and the result then follows as the right-hand side is an element of the $\sigma$-algebra $\cF.$ To show \eqref{eq:sumseteq}, let $\omega \in \cup_{s \in \QQFF}\cup_{q \in \QQU} \setsq,$ and let $s \in \QQFF, q \in \QQU$ be such that $\omega \in \setsq.$ Then it follows from the triangle inequality that $\omega \in \mleft(\phio+\phit\mright)^{-1}\mleft(\Ballspace{U}{r}{v}\mright).$
Now let $\omega \in \mleft(\phio+\phit\mright)^{-1}\mleft(\Ballspace{U}{r}{v}\mright),$ define $\dw \de r - \NU{\phio(\omega)+\phit(\omega)-v} > 0,$ fix $s \in \QQFF \cap (0,\dw/2),$ and choose $q \in \QQU$ such that $\NU{\phi(\omega) - v/2 -q} < s.$ Then again it follows from the triangle inequality that $\omega \in \setsq,$ and thus \eqref{eq:sumseteq} holds, as required.
\epf

% Corollary: finite sums lying in a separable subspace of a normed vector space are measurable
\bco\label{cor:sepsubsum}
If $V$ is a normed vector space, $U \subseteq V$ is a separable subspace, and $\phi_j:\Omega\rightarrow U,$  $j=1,\ldots,m$  are measurable functions, then $\phi_1+\cdots+\phi_m : \Omega\rightarrow U$ is measurable.
\eco

% Lemma: scalar measurable function multiplied by a vector is measurable (as it's continuous)
\ble\label{lem:scalarmultmeas}
Let $V$ be a normed vector space. If $v \in V$ and $Y:\Omega\rightarrow \FF$ is a measurable function, then $Yv:\Omega\rightarrow V$ is a measurable function.
\ele

\bpf[Proof of \cref{lem:scalarmultmeas}]
The map $\scalarmult:\FF\rightarrow V$ given by $\scalarmult(x) = xv$ is continuous. As $Yv = \scalarmult \circ Y,$ it follows from \cref{lem:contplusmeas} that $Yv$ is measurable.
\epf

% Lemma: Inclusion map is Borel
\ble\label{lem:incborel}
If $V$ is a normed vector space and $U \subseteq V,$ then the inclusion map $\inc:U\rightarrow V$ is measurable.
\ele

\bpf[Proof of \cref{lem:incborel}]
As $\inc$ is continuous, it immediately follows that it is measurable.
\epf

% Corollary: measurable into subspace means measurable into space
\bco\label{cor:meassubmeansmeas}
If $V$ is a normed vector space, $U\subseteq V$ and $\phi:\Omega\rightarrow U$ is measurable, then $\phi:\Omega\rightarrow V$ is measurable.
\eco

\bpf[Proof of \cref{cor:meassubmeansmeas}]
This is immediate from \cref{lem:incborel} and \cref{lem:contplusmeas}.
\epf

% Lemma: the $\FF$-span  of a finite set of functions is a separable subspace if $\FF$ is separable
\ble\label{lem:spansep}
If $V$ is a normed vector space, $m \in \NN,$ and $\phi_1,\ldots,\phi_m \in V$ for $j = 1,\ldots,m$ then $\spanset{\phi_1,\ldots,\phi_m}$ is a separable subspace of $V.$
\ele

\bpf[Sketch Proof of \cref{lem:spansep}]
As $\FF = \RR$ or $\CC,$ it has a separable subset $\QQFF.$ Since a finite product of countable sets is countable, the set
\beqs
\set{\Ballspace{V}{1/n}{q_1\phi_1 + \cdots + q_m \phi_m} \st n \in \NN, q_1,\ldots,q_m \in \QQFF}
\eeqs
is a countable base for the topology on $\spanset{\phi_1,\ldots,\phi_m}$ induced by the norm $\NV{\cdot}.$ 
\epf

\ble\label{lem:summeas}
If $V$ is a normed vector space, $m \in \NN,$ and for $j = 1,\ldots,m$, $\phi_j\in V$ and $Y_j : \Omega \rightarrow \FF$ are measurable, then the function $\phi:\Omega\rightarrow V$ given by
\beqs
\phi(\omega) = \phi_0 + \sum_{j=1}^m Y_j(\omega)\phi_j
\eeqs
is measurable.
\ele

\bpf[Proof of \cref{lem:summeas}]
The subspace $U=\spanset{\phi_0,\phi_1,\ldots,\phi_m}$ is separable by \cref{lem:spansep}, and it is clear that the image of $\phi$ lies in $U.$ By \cref{lem:scalarmultmeas} and \cref{cor:sepsubsum}, $\phi:\Omega\rightarrow U$ is measurable, and therefore $\phi:\Omega\rightarrow V$ is measurable by \cref{cor:meassubmeansmeas}.
\epf

We now prove that almost-surely convergent sequences of measurable functions are measurable, and we then apply this result to the partial sums in the definitions of $A$ and $n$ in \eqref{eq:nseries}. %In order to prove this, from this point onwards we make the assumption that the space $V$ is complete. This assumption is satisfied by the spaces involved in the definitions of $A$ and $n$ ($\WoiDRRRdtd$ and $\WoiDRRR$ respectively).

We will use the following theorem to establish that the almost-sure limit of a sequence of measurable functions is measurable.

\bth[{\cite[Theorem 4.2.2]{Du:02}}]\label{thm:dudley}
Let $(\metsp,\metric)$ be a metric space. Suppose the functions $\zeta_j:\Omega \rightarrow \metsp$ are measurable, for all $j \in \NN.$ If the limit
\beqs
\zeta(\omega) = \lim_{j \rightarrow \infty} \zeta_j(\omega)
\eeqs
exists for every $\omega \in \Omega,$ then the function $\zeta:\Omega \rightarrow \metsp$ is measurable.
\enth

\bco\label{lem:paelimitmeas}
Let $(\metsp,\metric)$ be a metric space. Suppose the functions $\zeta_m:\Omega \rightarrow \metsp$ are measurable, for all $m \in \NN.$ If the limit
\beq\label{eq:zetalimit}
\lim_{m \rightarrow \infty} \zeta_m(\omega)
\eeq
exists almost surely, then there exists a measurable function $\zeta:\Omega \rightarrow \metsp$ such that $\zeta(\omega) = \lim_{m \rightarrow \infty} \zeta_m(\omega)$ whenever the limit exists.
\eco

\bpf[Proof of \cref{lem:paelimitmeas}]
Following \cite{El:11}, we define $\Omegatilde = \set{\omega \in \Omega \st \text{ \eqref{eq:zetalimit} exists}}.$ Then, for $m \in \NN$ define $\zetatilde_m:\Omega\rightarrow \metsp$ by
\beqs
\zetatilde_m(\omega) =
\begin{cases}
\zeta_m(\omega) & \tif \omega \in \Omegatilde\\
0 & \tif \omega \not\in\Omegatilde
\end{cases}
\eeqs
Observe that, by construction, the limit $\zetatilde(\omega) = \lim_{m\rightarrow\infty} \zetatilde_m(\omega)$ exists \emph{for all} $\omega \in \Omega.$ Therefore, by \cref{thm:dudley}, $\zetatilde$ is measurable.
\epf

\ble\label{lem:paeseriesmeas}
Let $V$ be a normed vector space. If there exist $\phi_j\in V,$ $j=0,1,\ldots$ and measurable functions $\Yj:\Omega \rightarrow \FF,$ $j \in \NN$ such that the series
\beqs
\phi_0 + \sum_{j=1}^\infty \Yj(\omega)\phi_j
\eeqs
exists in $V$ almost surely, then there exists a measurable function $\phi:\Omega\rightarrow V$ such that
\beqs
\phi(\omega) = \phi_0 + \sum_{j=1}^\infty \Yj(\omega)\phi_j
\eeqs
almost surely.
\ele

\bpf[Proof of \cref{lem:paeseriesmeas}]
By \cref{lem:summeas}, the partial sums $\phi_0 + \sum_{j=1}^m \Yj(\omega)\phi_j$, for $m \in \NN$ are measurable, and by assumption their limit as $m \rightarrow \infty$ exists almost surely. Therefore, applying \cref{lem:paelimitmeas} to the partial sums, we obtain the result.
\epf

\ble\label{lem:seriesexistsas}
The series expansions for $A$ and $n$ defined by \eqref{eq:nseries} exist almost surely in $\WoiDRRRdtd$ and $\WoiDRRR$ respectively.
\ele

\bpf[Proof of \cref{lem:seriesexistsas}]
The spaces $\WoiDRRRdtd$ and $\WoiDRRR$ are Banach spaces, by definition of their norms (see, e.g., \eqref{eq:normsdef}). Therefore it suffices to show that the partial sums of the series expansions for $A$ and $n$ in \eqref{eq:nseries} fare Cauchy sequences. As the proofs for $A$ and $n$ are completely analogous, we only give the proof for $A$ here.

First observe that since the random variables $\Yj$ in \eqref{eq:nseries} are uniformly distributed on $[-1/2,1/2],$ it follows that for all $j \in \NN$, $\esssup_{\omega \in \Omega} \abs{\Yj(\omega)} = \frac12.$
Therefore, we can conclude that the bound $\esssup_{\omega \in \Omega} \sup_{j \in \NN} \abs{\Yj(\omega)} \leq \half$ holds.
(For if not, then, there would exist $\Omegahat \subseteq \Omega$ with $\PP(\Omegahat) > 0$ such that for all $\omega \in \Omegahat$, $\sup_{j \in \NN} \abs{\Yj(\omega)} > \half.$
Then there would exist $\jhat \in \NN$ such that $\abs{\Yj(\omega)} > 1/2$ for all $\omega \in \Omegahat,$ which would give the contradiction $\esssup_{\omega \in \Omega} |\Yjhat(\omega)| > \half$.)

It now suffices to show that for $\PP$-almost every $\omega \in \Omega,$ the partial sums of the series expansion in \eqref{eq:nseries} form a Cauchy sequence. Recall that for $\PP$-almost every $\omega \in \Omega$
\beqs
\sup_{j \in \NN} \abs{\Yj(\omega)} \leq \half.
\eeqs
For such an $\omega$, and $m \in \NN,$ define the $m$th partial sum
\beqs
\Am(\omega) = \Az + \sum_{j=1}^m \Yj(\omega)\Psij.
\eeqs
It is straightforward to show that the sequence $\mleft(\Am(\omega)\mright)_{m \in \NN}$ forms a Cauchy sequence in $\WoiDRRRdtd$, using the assumption \eqref{eq:Apsimeas}; therefore, the series expansion for $A(\omega)$ in \eqref{eq:nseries} exists almost surely.
\epf

\ble\label{lem:seriesmeas}
The functions $A$ and $n$ defined by \eqref{eq:nseries} are measurable.
\ele

\bpf[Proof of \cref{lem:seriesmeas}]
The result immediately follows from \cref{lem:seriesexistsas,lem:paeseriesmeas}.
\epf


\bibliographystyle{plain}
\bibliography{biblio_thesis}

\end{document}
