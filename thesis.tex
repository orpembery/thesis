\documentclass{report}

\ifdefined\revisionversion
\def\baselinestretch{1.5} % Modified from Higham, Second Edition, p. 109
\fi


\usepackage{amsthm}
\usepackage[show]{ed}
\usepackage{mathtools}
%\usepackage{amsthm} % Needed?
\usepackage{mleftright}
\ifdefined\supervisorversion

\else
        \usepackage{showkeys}
    \fi
    % Based on https://tex.stackexchange.com/a/1495

\usepackage{amssymb}
\usepackage{enumitem}
\usepackage[margin=2cm]{geometry}

\usepackage{thmtools} 
\usepackage{booktabs}
\usepackage{color} 
\usepackage{tikz} 
\usepgflibrary{shapes.misc}
\usetikzlibrary{positioning}
\usetikzlibrary{matrix} 
\usetikzlibrary{patterns}
\usetikzlibrary{calc}
\usetikzlibrary{arrows.meta}
\usepackage{bbm} % Allows blackboard bold 1
\usepackage{cases} % Gives numcases environment
\usepackage{longtable}
\usepackage{tabu}
\usepackage{makecell}
\usepackage{cellspace}
%\usepackage{adjustbox}
\usepackage{threeparttable}
\usepackage{afterpage}
\usepackage{pdflscape}
\usepackage[algochapter,dotocloa,boxed]{algorithm2e}
% The following two packages should come last
\usepackage[hidelinks]{hyperref}
% Found out about hidelinks in https://tex.stackexchange.com/a/50754
\usepackage[capitalise]{cleveref}



%%%%%%%%%%%%%% All the below macros have been taken from Anton Geraschenko's files at stacky.net %%%%%%%%%%%%%%%

%	Bold Letters
\renewcommand{\AA}{\mathbb A}
\newcommand{\BB}{\mathbb B}
\newcommand{\CC}{\mathbb C}
\newcommand{\DD}{\mathbb D}
\newcommand{\EE}{\mathbb E}
\newcommand{\FF}{\mathbb F}
\newcommand{\GG}{\mathbb G}
\newcommand{\HH}{\mathbb H}
\newcommand{\II}{\mathbb I}
\newcommand{\JJ}{\mathbb J}
\newcommand{\KK}{\mathbb K}
\newcommand{\LL}{\mathbb L}
\newcommand{\MM}{\mathbb M}
\newcommand{\NN}{\mathbb N}
\newcommand{\OO}{\mathbb O}
\newcommand{\PP}{\mathbb P}
\newcommand{\QQ}{\mathbb Q}
\newcommand{\RR}{\mathbb R}
\renewcommand{\SS}{\mathbb S}
\newcommand{\TT}{\mathbb T}
\newcommand{\UU}{\mathbb U}
\newcommand{\VV}{\mathbb V}
\newcommand{\WW}{\mathbb W}
\newcommand{\XX}{\mathbb X}
\newcommand{\YY}{\mathbb Y}
\newcommand{\ZZ}{\mathbb Z}

%Cal Letters
\newcommand{\cA}{\mathcal A}
\newcommand{\cB}{\mathcal B}
\newcommand{\cC}{\mathcal C}
\newcommand{\cD}{\mathcal D}
\newcommand{\cE}{\mathcal E}
\newcommand{\cF}{\mathcal F}
\newcommand{\cG}{\mathcal G}
\newcommand{\cH}{\mathcal H}
\newcommand{\cI}{\mathcal I}
\newcommand{\cJ}{\mathcal J}
\newcommand{\cK}{\mathcal K}
\newcommand{\cL}{\mathcal L}
\newcommand{\cM}{\mathcal M}
\newcommand{\cN}{\mathcal N}
\newcommand{\cO}{\mathcal O}
\newcommand{\cP}{\mathcal P}
\newcommand{\cQ}{\mathcal Q}
\newcommand{\cR}{\mathcal R}
\newcommand{\cS}{\mathcal S}
\newcommand{\cT}{\mathcal T}
\newcommand{\cU}{\mathcal U}
\newcommand{\cV}{\mathcal V}
\newcommand{\cW}{\mathcal W}
\newcommand{\cX}{\mathcal X}
\newcommand{\cY}{\mathcal Y}
\newcommand{\cZ}{\mathcal Z}

% Bold
\newcommand{\ba}{\mathbf{a}}
\newcommand{\bb}{\mathbf{b}}
\newcommand{\bc}{\mathbf{c}}
\newcommand{\bd}{\mathbf{d}}
\newcommand{\be}{\mathbf{e}}
\newcommand{\bff}{\mathbf{f}}
\newcommand{\bg}{\mathbf{g}}
\newcommand{\bh}{\mathbf{h}}
\newcommand{\bi}{\mathbf{i}}
\newcommand{\bj}{\mathbf{j}}
\newcommand{\bk}{\mathbf{k}}
\newcommand{\bl}{\mathbf{l}}
\newcommand{\bm}{\mathbf{m}}
\newcommand{\bn}{\mathbf{n}}
\newcommand{\bp}{\mathbf{p}}
\newcommand{\bq}{\mathbf{q}}
\newcommand{\br}{\mathbf{r}}
\newcommand{\bs}{\mathbf{s}}
\newcommand{\bt}{\mathbf{t}}
\newcommand{\bu}{\mathbf{u}}
\newcommand{\bv}{\mathbf{v}}
\newcommand{\bw}{\mathbf{w}}
\newcommand{\bx}{\mathbf{x}}
\newcommand{\by}{\mathbf{y}}
\newcommand{\bz}{\mathbf{z}}

% Nicked from Euan Spence
\newcommand{\beq}{\begin{equation}}
\newcommand{\eeq}{\end{equation}}
\newcommand{\beqs}{\begin{equation*}}
\newcommand{\eeqs}{\end{equation*}}
\newcommand{\bit}{\begin{itemize}}
\newcommand{\eit}{\end{itemize}}
\newcommand{\ben}{\begin{enumerate}}
\newcommand{\een}{\end{enumerate}}
\newcommand{\bal}{\begin{align}}
\newcommand{\eal}{\end{align}}
\newcommand{\bals}{\begin{align*}}
\newcommand{\eals}{\end{align*}}
\newcommand{\bse}{\begin{subequations}}
\newcommand{\ese}{\end{subequations}}
\newcommand{\bpr}{\begin{proposition}}
\newcommand{\epr}{\end{proposition}}
\newcommand{\bre}{\begin{remark}}
\newcommand{\ere}{\end{remark}}
\newcommand{\bpf}{\begin{proof}}
\newcommand{\epf}{\end{proof}}
\newcommand{\ble}{\begin{lemma}}
\newcommand{\ele}{\end{lemma}}
\newcommand{\bco}{\begin{corollary}}
\newcommand{\eco}{\end{corollary}}
\newcommand{\bex}{\begin{example}}
\newcommand{\eex}{\end{example}}
\newcommand{\bth}{\begin{theorem}}
\newcommand{\enth}{\end{theorem}}
\newcommand{\bcon}{\begin{condition}}
\newcommand{\econ}{\end{condition}}
\newcommand{\bas}{\begin{assumption}}
\newcommand{\eas}{\end{assumption}}
\newcommand{\bde}{\begin{definition}}
\newcommand{\ede}{\end{definition}}



\newcommand{\ton}{\text{ on }}
\newcommand{\tin}{\text{ in }}
\newcommand{\tfa}{\text{ for all }}
\newcommand{\tfor}{\text{ for }}
\newcommand{\tas}{\text{ as }}
\newcommand{\tand}{\text{ and }}
\newcommand{\tst}{\text{ such that }}
\newcommand{\tif}{\text{ if }}
\newcommand{\tals}{\text{ almost surely}}
\ifdefined\supervisorversion
    \newcommand{\optodo}[1]{}
\else
    \newcommand{\optodo}[1]{\ednote{OPTODO: #1}}
    \fi
    % Based on https://tex.stackexchange.com/a/1495
\newcommand{\minispace}{\;\!}

\newcommand{\AmatoI}{\matI{\Amato}}
\newcommand{\Amato}{\Amat_{1}}
\newcommand{\Amat}{\mat{A}}
\newcommand{\mat}[1]{\mathbf{#1}}
\newcommand{\Amatt}{\Amat_{2}}
%\newcommand{\CMC}{C_{\mathrm{MC}}}
%\newcommand{\CMLMC}{C_{\mathrm{MLMC}}}
\newcommand{\eps}{\varepsilon}
\newcommand{\grad}{\nabla}
\newcommand{\Ad}{A_{\mathrm{det}}}
\newcommand{\nd}{n_{\mathrm{det}}}
\newcommand{\fd}{f_{\mathrm{det}}}
\newcommand{\ud}{u_{\mathrm{det}}}
\newcommand{\Pmat}{\mat{P}}
\newcommand{\matI}[1]{#1^{-1}}
\newcommand{\PmatI}{\matI{\Pmat}}
\newcommand{\butilde}{\widetilde{\bu}}
\newcommand{\AmatI}{\matI{\Amat}}
\newcommand{\Imat}{\mat{I}}
\newcommand{\Amin}{A_{\min}}
\newcommand{\Amax}{A_{\max}}
\newcommand{\EXP}[1]{\EE\mleft[#1\mright]}
\newcommand{\ddPPomega}{\dd\PP(\omega)}
\newcommand{\dd}{\mathrm{d}}
\newcommand{\bzero}{\mathbf{0}}
\newcommand{\Dm}{D_{-}}
\newcommand{\Dp}{D_{+}}
\newcommand{\de}{\coloneqq}
\newcommand{\RRd}{\RR^{d}}
\newcommand{\Dmclos}{\clos{\Dm}}
\newcommand{\clos}[1]{\overline{#1}}
\newcommand{\GD}{\Gamma_{D}}
\newcommand{\LtDp}{\Lt{\Dp}}
\newcommand{\Lt}[1]{L^{2}\mleft(#1\mright)}
\newcommand{\gD}{g_{D}}
\newcommand{\Hh}[1]{H^{1/2}\mleft(#1\mright)}
\newcommand{\HhGD}{\Hh{\GD}}
\newcommand{\LiDpRRdtd}{\Li{\Dp;\RRdtd}}
\newcommand{\LiDpRR}{\Li{\Dp;\RR}}
\newcommand{\Li}[1]{L^{\infty}\mleft(#1\mright)}
\newcommand{\nmin}{n_{\min}}
\newcommand{\nmax}{n_{\max}}
\newcommand{\RRdtd}{\RR^{d\times d}}
\newcommand{\abs}[1]{\mleft|#1\mright|}
\newcommand{\bxi}{\boldsymbol{\xi}}
\newcommand{\bxibar}{\overline{\bxi}}
\newcommand{\CCd}{\CC^{d}}
\newcommand{\HolocDp}{\Holoc{\Dp}}
\newcommand{\Holoc}[1]{H^{1}_{\mathrm{loc}}\mleft(#1\mright)}
\newcommand{\defn}[1]{\emph{#1}}
\newcommand{\trGD}{\trace_{D}}
\newcommand{\trace}{\gamma}
\newcommand{\dudr}{\frac{\partial u}{\partial r}}
\newcommand{\bxhat}{\hat{\bx}}
\newcommand{\dn}{\partial_{\nu}}
\newcommand{\gI}{g_{I}}
\newcommand{\Dtilde}{\widetilde{D}}
\newcommand{\compcont}{\subset\subset}
\newcommand{\GI}{\Gamma_{I}}
\newcommand{\LtD}{\Lt{D}}
\newcommand{\LtGI}{\Lt{\GI}}
\newcommand{\LiDRR}{\Li{D;\RR}}
\DeclareMathOperator{\supp}{supp}
\newcommand{\LiDRRdtd}{\Li{D;\RRdtd}}
\DeclareMathOperator{\dist}{dist}
\newcommand{\HoD}{\Ho{D}}
\newcommand{\Ho}[1]{H^{1}\mleft(#1\mright)}
\newcommand{\trGI}{\trace_{I}}
\newcommand{\HozD}[1]{H^{1}_{0,\mathrm{D}}\mleft(#1\mright)}
\newcommand{\aE}{a} % For EDP
\newcommand{\ui}{u_{i}}
\newcommand{\us}{u_{s}}
\newcommand{\BR}{B_{R}}
\newcommand{\DR}{D_{R}}
\newcommand{\HozDDR}{\HozD{\DR}}
\newcommand{\FE}{F}
\newcommand{\IPRRd}[2]{\IP{#1}{#2}_{\RRd}}
\newcommand{\IP}[2]{\mleft(#1,#2\mright)}
\newcommand{\vbar}{\overline{v}}
\newcommand{\DPGR}[2]{\IP{#1}{#2}_{\GR}}
\newcommand{\DtN}{T_{R}}
\newcommand{\DtNapprox}{\widetilde{T}}
\newcommand{\trGR}{\trace_{R}}
\newcommand{\GR}{\Gamma_{R}}
\newcommand{\aT}{a_{T}} % For truncated
\newcommand{\FT}{F_{T}}
\newcommand{\restrict}{\big|} % Advice given in https://tex.stackexchange.com/questions/22252/how-to-typeset-function-restrictions
\newcommand{\NTADp}[1]{\mathrm{NT}_{\mathrm{mat},\Dp}\mleft(#1\mright)}
\newcommand{\NTnDp}[1]{\mathrm{NT}_{\mathrm{scal},\Dp}\mleft(#1\mright)}
\newcommand{\NTAD}[1]{\mathrm{NT}_{\mathrm{mat},D}\mleft(#1\mright)}
\newcommand{\NTnD}[1]{\mathrm{NT}_{\mathrm{scal},D}\mleft(#1\mright)}
\newcommand{\CzoDpclosRRdtd}{\Czo{\Dpclos;\RRdtd}}
\newcommand{\Czo}[1]{C^{0,1}\mleft(#1\mright)}
\newcommand{\Dpclos}{\clos{\Dp}}
\newcommand{\CzoDpclosRR}{\Czo{\Dpclos;\RR}}
\newcommand{\muo}{\mu_{1}}
\newcommand{\mut}{\mu_{2}}
\newcommand{\NLtDR}[1]{\N{#1}_{\LtDR}}
\newcommand{\LtDR}{\Lt{\DR}}
\newcommand{\N}[1]{\mleft\|#1\mright\|}
\newcommand{\Co}{C_{1}}
\newcommand{\LI}{L_{I}}
\newcommand{\NLtD}[1]{\N{#1}_{\LtD}}
\newcommand{\NLtGI}[1]{\N{#1}_{\LtGI}}
\newcommand{\gradGI}{\grad_{\GI}}
\newcommand{\Ct}{C_{2}}
\newcommand{\Cttilde}{\widetilde{C}_{2}}
\newcommand{\NW}[1]{\N{#1}_{1,k}}
\newcommand{\NLiD}[1]{\N{#1}_{\LiD}}
\newcommand{\LiD}{\Li{D}}
\newcommand{\Ki}{K_{i}}
\newcommand{\Dclos}{\clos{D}}
\newcommand{\interior}[1]{\mathring{#1}}
\newcommand{\Kj}{K_{j}}
\newcommand{\set}[1]{\mleft\{#1\mright\}}
\newcommand{\vh}{v_{h}}
\newcommand{\st}{:}
\newcommand{\polyp}[1]{\cP_{p}\mleft(#1\mright)}
\newcommand{\Kiclos}{\clos{\Ki}}
\newcommand{\Cz}[1]{C^{0}\mleft(#1\mright)}
\newcommand{\CzD}{\Cz{D}}
\DeclareMathOperator{\diam}{diam}
\newcommand{\Vhp}{V_{h,p}}
\newcommand{\vhptilde}{\widetilde{v}_{h,p}}
\newcommand{\HmD}{\Hm{D}}
\newcommand{\Hm}[1]{H^{m}\mleft(#1\mright)}
%\newcommand{\NHsD}[1]{\N{#1}_{\HsD}}
%\newcommand{\HsD}{\Hs{D}}
%\newcommand{\Hs}[1]{H^{s}\mleft(#1\mright)}
\newcommand{\NHoD}[1]{\N{#1}_{\HoD}}
\newcommand{\uh}{u_{h}}
\newcommand{\uz}{u_{0}}
\newcommand{\Th}{\cT_{h}}
\newcommand{\tr}{\gamma}
%\newcommand{\dD}{\partial D}
\newcommand{\kz}{k_{0}}
%\newcommand{\HthtGD}{\Htht{\GD}}
\newcommand{\Htht}[1]{H^{3/2}\mleft(#1\mright)}
%\newcommand{\Woi
\newcommand{\aadj}{\widetilde{a}}
\newcommand{\vb}{\overline{v}}
\newcommand{\vo}{v_{1}}
\newcommand{\vt}{v_{2}}
\newcommand{\epsymb}{\star}
\newcommand{\api}{a_{\epsymb}}
\newcommand{\vtbar}{\overline{v}_{2}}
\newcommand{\CzoDRRRdtd}{\Czo{\DR;\RRdtd}}
\newcommand{\IPLtDR}[2]{\IP{#1}{#2}_{\LtDR}}
\newcommand{\HtDR}{\Ht{\DR}}
\newcommand{\Ht}[1]{H^{2}\mleft(#1\mright)}
\newcommand{\NHtDR}[1]{\N{#1}_{\HtDR}}
\newcommand{\CHtell}{C_{H^{2},\epsymb}}
\newcommand{\CHthh}{C_{H^2}}
\newcommand{\Nunsure}[1]{\N{#1}_{\HhGI}}
\newcommand{\HhGI}{\Hh{\GI}}
\newcommand{\NLtGD}[1]{\N{#1}_{\GD}}
\newcommand{\gradGD}{\grad_{\GD}}
\newcommand{\NLiDRRR}[1]{\N{#1}_{\LiDRRR}}
\newcommand{\LiDRRR}{\Li{\DR;\RR}}
\newcommand{\half}{\frac{1}{2}}
\newcommand{\Cfemo}{C_{\mathrm{FEM},1}}
\newcommand{\Cfemt}{C_{\mathrm{FEM},2}}
\newcommand{\Cfemth}{C_{\mathrm{FEM},3}}
\newcommand{\Npi}[1]{\N{#1}_{\epsymb}}
\newcommand{\quarter}{\frac{1}{4}}
\newcommand{\Ph}{\cP_{h}}
\newcommand{\Phn}{\cP_{h,n}}
\newcommand{\Ih}{\cI_{h}}
\newcommand{\wh}{w_{h}}
\newcommand{\whtilde}{\widetilde{w}_{h}}
\newcommand{\NHoDR}[1]{\N{#1}_{\HoDR}}
\newcommand{\HoDR}{\Ho{\DR}}
\newcommand{\Mfg}{M\mleft(f,\gI,\gD\mright)}
\newcommand{\LiDp}{\Li{\Dp}}
\newcommand{\LiDpCC}{\Li{\Dp;\CC}}
\newcommand{\bxz}{\mathbf{x}_{0}}
\newcommand{\Ball}[2]{B_{#1}\mleft(#2\mright)}
\newcommand{\bfzero}{\mathbf{0}}
\newcommand{\HhGR}{\Hh{\GR}}
\newcommand{\HmhGR}{\Hmh{\GR}}
\newcommand{\Hmh}[1]{H^{-1/2}\mleft(#1\mright)}
\newcommand{\Lap}{\Delta}
\newcommand{\Wmp}[1]{W^{m,p}\mleft(#1\mright)}
\newcommand{\WmpD}{\Wmp{D}}
\newcommand{\LtDpCC}{\Lt{\Dp;\CC}}
\newcommand{\trD}{\trGD}
\newcommand{\trI}{\trGI}
\newcommand{\CzoD}{\Czo{D}}
\newcommand{\WoiD}{\Woi{D}}
\newcommand{\Woi}[1]{W^{1,\infty}\mleft(#1\mright)}
\newcommand{\enlargement}[1]{#1_{+1}}
\newcommand{\beps}{\boldsymbol{\eps}}
\newcommand{\WoiDRRdtd}{\Woi{D;\RRdtd}}
\newcommand{\WoiDRR}{\Woi{D;\RR}}
\newcommand{\WoiDpclosRRdtd}{\Woi{\Dpclos;\RRdtd}}
\newcommand{\WoiDpclosRR}{\Woi{\Dpclos;\RR}}
\newcommand{\km}{k_{m}}
\newcommand{\um}{u_{m}}
\newcommand{\fm}{f_{m}}
\newcommand{\NLtDp}[1]{\N{#1}_{\LtDp}}
\newcommand{\NHtD}[1]{\N{#1}_{\HtD}}
\newcommand{\HtD}{\Ht{D}}
\newcommand{\uhat}{\hat{u}}
\newcommand{\VAR}[1]{\mathbb{V}\mleft[#1\mright]}
\newcommand{\uj}{u^{(j)}}
\newcommand{\Aj}{A^{(j)}}
\newcommand{\nj}{n^{(j)}}
\newcommand{\xhat}{\hat{x}}
\newcommand{\Amatj}{\Amat^{(j)}}
\newcommand{\At}{A_{2}}
\newcommand{\nt}{n_{2}}
\newcommand{\Dmat}{\mat{D}}
\newcommand{\HokDR}{\Hok{\DR}}
\newcommand{\Hok}[1]{H^{1}_{k}\mleft(#1\mright)}
\newcommand{\NLi}[1]{\N{#1}_{L^{\infty}}}
\newcommand{\NDmatk}[1]{\N{#1}_{\Dmatk}}
\newcommand{\Dmatk}{\Dmat_{k}}
\newcommand{\NDmatkI}[1]{\N{#1}_{\DmatkI}}
\newcommand{\DmatkI}{\Dmat_{k}^{-1}}
\newcommand{\vi}{v_{i}}
\newcommand{\phii}{\phi_{i}}
\newcommand{\NHokDR}[1]{\N{#1}_{\HokDR}}
\newcommand{\mpm}{m_{\pm}}
\newcommand{\spm}{s_{\pm}}
\newcommand{\Smat}{\mat{S}}
\newcommand{\Mmat}{\mat{M}}
\newcommand{\Nmat}{\mat{N}}
\newcommand{\Aomin}{A_{1,\min}}
\newcommand{\nomin}{n_{1,\min}}
\newcommand{\nomax}{n_{1,\max}}
\newcommand{\Mmatj}{\Mmat_{j}}
\newcommand{\tbu}{\butilde}
\newcommand{\tu}{\utilde}
\newcommand{\utilde}{\widetilde{u}}
% From here on I just defined Euan's macros to be mine
\newcommand{\matrixC}{\Cmat}
\newcommand{\Cmat}{\mat{C}}
\newcommand{\CCN}{\CC^{N}}
\newcommand{\bfr}{\br}
\newcommand{\re}{\mathrm{e}}
\newcommand{\dudnu}{\frac{\partial u}{\partial \nu}}
\newcommand{\balpha}{{\boldsymbol{\alpha}}}
\newcommand{\LtDCC}{\Lt{D;\CC}}
\newcommand{\HmhGI}{\Hmh{\GI}}
\newcommand{\HoGD}{\Ho{\GD}}




%%%% All the different constants for the new FEM bounds %%%%
\newcommand{\CAfn}[1]{C_{A,#1}}
\newcommand{\CAl}{\CAfn{l}}
\newcommand{\CintAl}{C_{\mathrm{int},A,l}}
\newcommand{\CscatAl}{C_{\mathrm{scat},A,l}}
\newcommand{\CtruncAl}{C_{\mathrm{trunc},A,l}}
\newcommand{\CAnk}{\Cfn{\mathrm{stab}}}%{C\mleft(A,n,k\mright)}
\newcommand{\CTrfn}[1]{C_{\mathrm{Tr},#1}}
\newcommand{\CTrm}{\CTrfn{m}}
\newcommand{\Cosc}{C_{\mathrm{osc}}}
\newcommand{\CTrlpo}{\CTrfn{l+1}}
\newcommand{\CAz}{\CAfn{0}}
\newcommand{\CAo}{\CAfn{1}}
\newcommand{\CTrt}{\CTrfn{2}}
\newcommand{\CTrj}{\CTrfn{j}}
\newcommand{\CAm}{\CAfn{m}}
\newcommand{\CTrmpo}{\CTrfn{m+1}}
\newcommand{\CTrth}{\CTrfn{3}}
\newcommand{\CAt}{\CAfn{2}}
\newcommand{\Czero}{C_{0}}
\newcommand{\CAj}{\CAfn{j}}
\newcommand{\CAjmo}{\CAfn{j-1}}
\newcommand{\Cfg}{C_{f,\gI}}
\newcommand{\CFEMo}{C_{\mathrm{FEM},1}}
\newcommand{\CFEMt}{C_{\mathrm{FEM},2}}
\newcommand{\CFEMotilde}{\CFEMo}
\newcommand{\CFEMttilde}{\CFEMt}
%\newcommand{\Cmszfn}[1]{C_{#1,0}}
\newcommand{\Cmmofn}[1]{\Cfn{\mathrm{proj},#1}}%{C_{#1,1}}
\newcommand{\Cprojfn}[1]{\Cmmofn{#1}}
%\newcommand{\Cnmszfn}[1]{C_{n,#1,0}}
%\newcommand{\Cnmsz}{\Cnmszfn{-s}}
%\newcommand{\Cmmo}{\Cmmofn{-m}}
\newcommand{\Cmmz}{\CSZm}
\newcommand{\Cinterpfn}[1]{\CSZfn{#1}}
\newcommand{\Cinterpm}{\Cinterpfn{m}}
\newcommand{\Cmafn}[1]{C_{#1,A}}
\newcommand{\Cma}{\Cmafn{m}}
\newcommand{\CinvVhp}{C_{\mathrm{inv},\Vhp}}
\newcommand{\Csumj}{\Csumfn{j}}
\newcommand{\Chinv}{C_{\mathrm{disc},\mathrm{inv}}}
\newcommand{\Cmmo}{\Cmmofn{-m}}
\newcommand{\Cinterpt}{\Cinterpfn{2}}
%\newcommand{\Cmfn}[1]{C_{#1}}
\newcommand{\Cm}{\Csumrecfn{m}}
\newcommand{\Cfn}[1]{C_{#1}}  
\newcommand{\CFEMLt}{\CFEMfn{L^{2}}}
\newcommand{\CFEMfn}[1]{C_{\mathrm{FEM},#1}}
\newcommand{\CFEMHo}{\CFEMfn{H^{1}}}
\newcommand{\Ccond}{\Chcond}
\newcommand{\Chcond}{C_{\mathrm{cond}}}
\newcommand{\Cboundaryfn}[1]{\Cfn{\labelcref{eq:boundarybound},#1}}%{\Cfn{\mathrm{boundary},#1}}
\newcommand{\Cboundaryo}{\Cboundaryfn{1}}
\newcommand{\Cboundaryt}{\Cboundaryfn{2}}
\newcommand{\Cmess}{\Cfn{\labelcref{eq:Qhnrhosum}}}
\newcommand{\CMT}{\Cfn{\mathrm{MT}}}
\newcommand{\Csumpmo}{\Csumfn{p-1}}
\newcommand{\Chighfn}[1]{\Cfn{\labelcref{eq:chigh},#1}}
\newcommand{\Chighmo}{\Chighfn{m,1}}
\newcommand{\Chighmt}{\Chighfn{m,2}}
\newcommand{\CBo}{\Cfn{\labelcref{eq:firstboundary},1}}
\newcommand{\CBt}{\Cfn{\labelcref{eq:firstboundary},2}}
\newcommand{\CBth}{\Cfn{\labelcref{eq:secondboundary}}}
\newcommand{\CRecfn}[1]{\Cfn{\labelcref{eq:readytorecurse},#1}}%{\Cfn{\mathrm{Rec},#1}}
\newcommand{\CRecofn}[1]{\CRecfn{1,#1}}
\newcommand{\CRectfn}[1]{\CRecfn{2,#1}}
\newcommand{\CRect}{\CRecfn{2}}
\newcommand{\CReco}{\CRecfn{1}}
\newcommand{\Cfirst}{\Cfn{\labelcref{eq:ltthetahbound},1}}%{\Cfn{\mathrm{first}}}
\newcommand{\Csecond}{\Cfn{\labelcref{eq:ltthetahbound},2}}%{\Cfn{\mathrm{second}}}
\newcommand{\Cc}{\Cfn{\mathrm{C}}}
\newcommand{\Csumfn}[1]{\Cfn{\labelcref{eq:negdiscsum},#1}}
\newcommand{\Cfourteen}{\Cfn{\labelcref{eq:innerprod1}}}
\newcommand{\Ctildemin}{\widetilde{C}_{\min}}
\newcommand{\Cboundfn}[2]{C_{#1,\mathrm{bound},#2}}
\newcommand{\CLtfn}[1]{\Cfn{\labelcref{eq:ltboundwithrho},#1}}
\newcommand{\CHofn}[1]{\Cfn{\labelcref{eq:hoboundwithrho},#1}}
\newcommand{\CLtboundo}{\CLtfn{1}}
\newcommand{\CLtboundt}{\CLtfn{2}}
\newcommand{\CHoboundo}{\CHofn{1}}
\newcommand{\CHoboundt}{\CHofn{2}}
\newcommand{\CFEMohat}{2\CFEMo} % Needs tidying
\newcommand{\CFEMthat}{2\CFEMt}
\newcommand{\Ccorfn}[1]{\Cfn{\mathrm{Cor},#1}}
\newcommand{\CcorLt}{\CFEMLt}%{\Ccorfn{L^{2}}}
\newcommand{\CcorHo}{\CFEMHo}%{\Ccorfn{H^{1}}}
\newcommand{\Coscfn}[1]{\Cfn{\mathrm{osc},#1}}
\newcommand{\CSZfn}[1]{\Cfn{\mathrm{BA},#1}}
\newcommand{\CSZm}{\CSZfn{m}}
\newcommand{\Cefn}[1]{\Cfn{\mathrm{expansion},#1}}
\newcommand{\Cej}{\Cefn{j}}
\newcommand{\CBanfn}[1]{\Cfn{\mathrm{mult},1,#1}}
\newcommand{\Cshiftfn}[1]{\Cfn{\mathrm{shift},-#1}}
\newcommand{\Cshiftz}{\Cfn{\mathrm{shift},0}}
\newcommand{\Cshiftmo}{\Cfn{\mathrm{shift},1}}
\newcommand{\Cwfn}[1]{\Cfn{\mathrm{weight},#1}}
\newcommand{\Cwmm}{\Cwfn{-m}}%,1}}
\newcommand{\Cwmmo}{\Cwfn{-m}}%,1}}
\newcommand{\Cwz}{\Cwfn{0}}
\newcommand{\Cwo}{\Cwfn{1}}
%\newcommand{\Ccomprfn}[1]{\Cfn{\mathrm{comp},#1}}
\newcommand{\Cprod}[1]{\Cfn{\mathrm{mult},2,#1}}
\newcommand{\Cmclean}[1]{\Cfn{H^{#1},\mathrm{prod}}}
\newcommand{\CP}{\Cfn{\mathrm{P}}}
% Why is this defined like this?
\newcommand{\errn}[1]{\cCfn{\mathrm{err},#1}\mleft(n\mright)}
\newcommand{\En}{\cCfn{\mathrm{err}}\mleft(n\mright)}
\newcommand{\cCfn}[1]{\cC_{#1}}
\newcommand{\shiftn}[1]{\cCfn{\mathrm{shift},#1}\mleft(n\mright)}
\newcommand{\Rn}{\cCfn{\mathrm{shift}}\mleft(n\mright)}
\newcommand{\nvar}{n_{\mathrm{var}}}
\newcommand{\Condn}{\cC}
\newcommand{\CLtn}{\Condn_{L^{2}}}
\newcommand{\CHon}{\Condn_{H^{1}}}
%\newcommand{\Cerrfn}[1]{\cCfn{\mathrm{err},#1}}
\newcommand{\Csumrecfn}[1]{\Cfn{\labelcref{eq:sumforrecursion},#1}}













%%%% End of the constants for the new FEM bounds %%%%
\newcommand{\Cpmoo}[1]{C^{p-1,1}\mleft(#1\mright)}
\newcommand{\CpmooDclos}{\Cpmoo{\Dclos}}
\newcommand{\Cpo}{C^{p,1}}
\newcommand{\Hpmo}[1]{H^{p-1}\mleft(#1\mright)}
\newcommand{\HpmoD}{\Hpmo{D}}
\newcommand{\Hpmh}[1]{H^{p-1/2}\mleft(#1\mright)}
\newcommand{\HpmhGI}{\Hpmh{\GI}}
\newcommand{\ftilde}{\widetilde{f}}
\newcommand{\Hl}[1]{H^{l}\mleft(#1\mright)}
\newcommand{\HlD}{\Hl{D}}
\newcommand{\gItilde}{\widetilde{\gI}}
\newcommand{\Hlph}[1]{H^{l+1/2}\mleft(#1\mright)}
\newcommand{\HlphGI}{\Hlph{\GI}}
\newcommand{\Hlpt}[1]{H^{l+2}\mleft(#1\mright)}
\newcommand{\HlptD}{\Hlpt{D}}
\newcommand{\NHlptD}[1]{\N{#1}_{\HlptD}}
\newcommand{\NHlD}[1]{\N{#1}_{\HlD}}
\newcommand{\NHlphGI}[1]{\N{#1}_{\HlphGI}}
\newcommand{\Dint}{D_{\mathrm{int}}}
\newcommand{\Dinttilde}{\widetilde{\Dint}}
\newcommand{\Dscat}{D_{\mathrm{scat}}}
\newcommand{\Dscatclos}{\clos{\Dscat}}
\newcommand{\Dtrunc}{D_{\mathrm{trunc}}}
\newcommand{\Dtruncclos}{\clos{\Dtrunc}}
\newcommand{\HlptDint}{\Hlpt{\Dint}}
\newcommand{\NHlptDint}[1]{\N{#1}_{\HlptDint}}
\newcommand{\HoDinttilde}{\Ho{\Dinttilde}}
\newcommand{\NHoDinttilde}[1]{\N{#1}_{\HoDinttilde}}
\newcommand{\HlDinttilde}{\Hl{\Dinttilde}}
\newcommand{\NHlDinttilde}[1]{\N{#1}_{\HlDinttilde}}
\newcommand{\HlptDscat}{\Hlpt{\Dscat}}
\newcommand{\NHlptDscat}[1]{\N{#1}_{\HlptDscat}}
\newcommand{\HlptDtrunc}{\Hlpt{\Dtrunc}}
\newcommand{\NHlptDtrunc}[1]{\N{#1}_{\HlptDtrunc}}
\newcommand{\SNHoD}[1]{\SN{#1}_{\HoD}}
\newcommand{\SNHoDn}[1]{\SN{#1}_{\HoD,n}}
\newcommand{\SNHtD}[1]{\SN{#1}_{\HtD}}
\newcommand{\NHhGI}[1]{\N{#1}_{\HhGI}}
\newcommand{\NHmmhGI}[1]{\N{#1}_{\HmmhGI}}
\newcommand{\HmmhGI}{\Hmmh{\GI}}
\newcommand{\Hmmh}[1]{H^{m-1/2}\mleft(#1\mright)}
\newcommand{\gtilde}{\widetilde{g}}
\newcommand{\HlpoD}{\Hlpo{D}}
\newcommand{\Hlpo}[1]{H^{l+1}\mleft(#1\mright)}
\newcommand{\NHlpoD}[1]{\N{#1}_{\HlpoD}}
\newcommand{\uosc}{u_{\mathrm{osc}}}
\newcommand{\Hppo}[1]{H^{p+1}\mleft(#1\mright)}
\newcommand{\HppoD}{\Hppo{D}}
\newcommand{\HjptD}{\Hjpt{D}}
\newcommand{\Hjpt}[1]{H^{j+2}\mleft(#1\mright)}
\newcommand{\NHjptD}[1]{\N{#1}_{\HjptD}}
\newcommand{\NHppoD}[1]{\N{#1}_{\HppoD}}
\newcommand{\Ffn}[1]{F_{#1}}
\newcommand{\Fjmo}{\Ffn{j-1}}
\newcommand{\Fj}{\Ffn{j}}
\newcommand{\vj}{v_{j}}
\newcommand{\HthD}{\Hth{D}}
\newcommand{\Hth}[1]{\Hfn{}{3}{#1}}
\newcommand{\Hfn}[3]{H_{#1}^{#2}\mleft(#3\mright)}
\newcommand{\NHthD}[1]{\N{#1}_{\HthD}}
\newcommand{\vjmt}{v_{j-2}}
\newcommand{\vjmo}{v_{j-1}}
\newcommand{\vz}{v_{0}}
\newcommand{\usj}{u_{j}}
\newcommand{\rl}{r_{l}}
\newcommand{\ro}{r_{1}}
\newcommand{\rt}{r_{2}}
\newcommand{\NHfD}[1]{\N{#1}_{\HfD}}
\newcommand{\HfD}{\Hf{D}}
\newcommand{\Hf}[1]{\Hfn{}{4}{#1}}
\newcommand{\Fo}{\Ffn{1}}
\newcommand{\Ft}{\Ffn{2}}
\newcommand{\Fpmt}{\Ffn{p-2}}
\newcommand{\Fp}{\Ffn{p}}
\newcommand{\CTrjpo}{\CTrfn{j+1}}
\newcommand{\rfn}[1]{r_{#1}}
\newcommand{\rj}{\rfn{j}}
\newcommand{\rjmt}{\rfn{j-2}}
\newcommand{\rjmo}{\rfn{j-1}}
\newcommand{\ujmo}{\ufn{j-1}}
\newcommand{\ujmt}{\ufn{j-2}}
\newcommand{\ufn}[1]{u_{#1}}
\newcommand{\rpmo}{\rfn{p-1}}
\newcommand{\Pj}{\Pfn{j}}
\newcommand{\Pfn}[1]{P_{#1}}
\newcommand{\pfn}[1]{q_{#1}}
\newcommand{\floor}[1]{\mleft\lfloor#1\mright\rfloor}
\newcommand{\uzhat}{\hat{u}_{0}}
\newcommand{\HozDD}{\HozD{D}}
%\newcommand{\NHmsD}[1]{\N{#1}_{\HmsD}}
%\newcommand{\HmsD}{\Hfn{}{-s}{D}}
\newcommand{\BAHoD}[2]{\inf_{#2 \in \Vhp} \NHoD{#1-#2}}
\newcommand{\BAHoDn}[2]{\inf_{#2 \in \Vhp} \NHnfn{1}{D}{#1-#2}}
\newcommand{\BALtD}[2]{\inf_{#2 \in \Vhp} \NLtD{#1-#2}}
\newcommand{\IPLtDn}[2]{\IP{#1}{#2}_{\LtD,n}}
\newcommand{\wbar}{\overline{w}}
\newcommand{\NLtDn}[1]{\N{#1}_{\LtD,n}}
\newcommand{\Qhn}{Q_{h,n}}
\newcommand{\NHmmnD}[1]{\N{#1}_{\HmmnD}}
\newcommand{\HmmnD}{\Hfn{n}{-m}{D}}
\newcommand{\BALtDn}[2]{\inf_{#2 \in \Vhp} \NLtDn{#1-#2}}
\newcommand{\vhhat}{\hat{v}_{h}}
%\newcommand{\SNHsD}[1]{\SN{#1}_{\HsD}}
\newcommand{\DeltaA}{\Delta_{A}}
\newcommand{\Deltah}{\Delta_{h}}
\newcommand{\dD}{\partial D}
\newcommand{\lambdafn}[1]{\lambda_{#1}}
\newcommand{\lambdao}{\lambdafn{1}}
\newcommand{\lambdat}{\lambdafn{2}}
\newcommand{\phifn}[1]{\phi_{#1}}
\newcommand{\phioh}{\phifn{1,h}}
\newcommand{\phidimVhph}{\phifn{\dimVhp,h}}
\newcommand{\dimVhp}{\dim\mleft(\Vhp\mright)}
\newcommand{\lambdaoh}{\lambdafn{1,h}}
\newcommand{\lambdadimVhph}{\lambdafn{\dimVhp,h}}
\newcommand{\phimh}{\phifn{m,h}}
\newcommand{\am}{a_{m}}
\newcommand{\lambdamh}{\lambdafn{m,h}}
\newcommand{\phim}{\phifn{m}}
\newcommand{\lambdam}{\lambdafn{m}}
\newcommand{\NmA}[1]{\N{#1}_{m,A}}
\newcommand{\HmmD}{\Hfn{}{-m}{D}}
\newcommand{\NHmmD}[1]{\N{#1}_{\HmmD}}
\newcommand{\NmmA}[1]{\N{#1}_{-m,A}}
\newcommand{\Loloc}[1]{\L^{1}_{\mathrm{loc}}\mleft(#1\mright)}
\newcommand{\LolocD}{\Loloc{D}}
\newcommand{\vhtilde}{\widetilde{v}_{h}}
\newcommand{\add}{a_{L^{2}}}
\newcommand{\Ldd}{L_{L^{2}}}
\newcommand{\NLiDop}[1]{\N{#1}_{\LiDop}}
\newcommand{\LiDop}{\Li{D;\mathrm{op}}}
\newcommand{\DeltaAmmt}{\sdres^-{m/2}}
%\newcommand{\DeltaAmt}{\DeltaA^{m/2}}
\newcommand{\Deltahmmt}{\Deltah^{-m/2}}
\newcommand{\Deltahmt}{\Deltah^{m/2}}
\newcommand{\Nmmh}[1]{\N{#1}_{-m,h}}
\newcommand{\Nmh}[1]{\N{#1}_{m,h}}
\newcommand{\aj}{a_{j}}
\newcommand{\phij}{\phifn{j}}
\newcommand{\phil}{\phifn{l}}
\newcommand{\lambdaj}{\lambdafn{j}}
\newcommand{\lambdal}{\lambdafn{l}}
\newcommand{\philbar}{\overline{\phil}}
\newcommand{\bsl}{b_{l}}
\newcommand{\bsj}{b_{j}}
\newcommand{\Njh}[1]{\N{#1}_{j,h}}
\newcommand{\Njhn}[1]{\N{#1}_{j,h,n}}
\newcommand{\Nmjh}[1]{\N{#1}_{-j,h}}
\newcommand{\Domain}[1]{\cD\mleft(#1\mright)}
\newcommand{\NHmD}[1]{\N{#1}_{\HmD}}
\newcommand{\NNz}{\NN_{0}}
\newcommand{\Njmohn}[1]{\N{#1}_{j-1,h,n}}
\newcommand{\Nmmohn}[1]{\N{#1}_{m-1,h,n}}
\newcommand{\Nohn}[1]{\N{#1}_{1,h,n}}
\newcommand{\Deltahh}{\Deltah^{1/2}}
\newcommand{\Nzhn}[1]{\N{#1}_{0,h,n}}
%\newcommand{\DomainDeltaA}{\Domain{\DeltaA}}
\newcommand{\wf}{w_{f}}
\newcommand{\zh}{z_{h}}
\newcommand{\DeltahI}{\Deltah^{-1}}
\newcommand{\sdres}{\sdsol}
\newcommand{\DeltaAI}{\sdres}%{\DeltaA^{-1}}
\newcommand{\sdsolfn}[1]{\cS_{#1}}
\newcommand{\sdsol}{\sdsolfn{n}}
\newcommand{\yh}{y_{h}}
\newcommand{\NHmmmtD}[1]{\N{#1}_{\HmmmtD}}
\newcommand{\HmmmtD}{\Hfn{}{-(m+2)}{D}}
\newcommand{\vtilde}{\widetilde{v}}
\newcommand{\NHmptD}[1]{\N{#1}_{\HmptD}}
\newcommand{\HmptD}{\Hfn{}{m+2}{D}}
\newcommand{\Deltahml}{\Deltah^{-l}}
\newcommand{\Deltahmlpo}{\Deltah^{-l+1}}
\newcommand{\NHDfn}[2]{\N{#2}_{\Hfn{}{#1}{D}}}
\newcommand{\Efn}[1]{C_{\labelcref{eq:evenrecursivesum},#1}}
\newcommand{\Nfn}[2]{\N{#2}_{#1}}
\newcommand{\NHmoD}[1]{\N{#1}_{\HmoD}}
\newcommand{\HmoD}{\Hfn{}{-1}{D}}
\newcommand{\Etildefn}[1]{\widetilde{C}_{\labelcref{eq:oddrecursive},#1}}
\newcommand{\thetah}{\theta_{h}}
\newcommand{\IPLtGI}[2]{\IP{#1}{#2}_{\LtGI}}
\newcommand{\Npmoh}[1]{\Nfn{p-1,h}{#1}}
\newcommand{\eh}{e_{h}}
\newcommand{\uhhat}{\hat{u}_{h}}
\newcommand{\ujh}{u_{j,h}}
\newcommand{\uosch}{u_{\mathrm{osc},h}}
\newcommand{\NHfn}[3]{\N{#3}_{\Hfn{}{#1}{#2}}}
\newcommand{\vfn}[1]{v_{#1}}
\newcommand{\wj}{w_{j}}
\newcommand{\NLtdD}[1]{\N{#1}_{\LtdD}}
\newcommand{\LtdD}{\Lt{\dD}}
\newcommand{\NHsDn}[1]{\Nfn{\HsD,n}{#1}}
\newcommand{\NHmmDn}[1]{\Nfn{\HmmD,n}{#1}}
\newcommand{\NHmDn}[1]{\Nfn{\HmD,n}{#1}}
\newcommand{\NHnfn}[3]{\N{#3}_{\Hfn{}{#1}{#2},n}}
\newcommand{\CrcompD}{\Crcompfn{r}{D}}
\newcommand{\Crcompfn}[2]{C^{#1}_{\mathrm{comp}}\mleft(#1\mright)}
\newcommand{\NWfn}[3]{\N{#3}_{\Wfn{#1}{#2}}}
\newcommand{\Wfn}[2]{W^{#1}\mleft(#2\mright)}
\newcommand{\NWoiD}[1]{\N{#1}_{\WoiD}}
\newcommand{\CmD}{C^{m}\mleft(D\mright)}
\newcommand{\CsD}{C^{s}\mleft(D\mright)}
\newcommand{\CmcompD}{C^{m}_{\mathrm{comp}}\mleft(D\mright)}
\newcommand{\WmiD}{\Wfn{m,\infty}{D}}
\newcommand{\WsiD}{\Wfn{s,\infty}{D}}
\newcommand{\NWmiD}[1]{\N{#1}_{\WmiD}}
\newcommand{\NWsiD}[1]{\N{#1}_{\WsiD}}
\newcommand{\NHoDn}[1]{\N{#1}_{\HoD,n}}
\newcommand{\Nmhn}[1]{\Nfn{m,h,n}{#1}}
\newcommand{\sdsolo}{\sdsolfn{1}}
\newcommand{\CfloordtD}{C^{\floor{d/2}}\mleft(D\mright)}
\newcommand{\Sn}{\sdsol}
\newcommand{\uHt}{u_{H^{2}}}
\newcommand{\uA}{u_{\cA}}
\newcommand{\Aij}{A_{i,j}}
\newcommand{\solfem}{\cS}
\newcommand{\solfems}{\solfem^{*}}
\newcommand{\wba}{\eta}
\newcommand{\wbaadj}{\wba^{*}}
\newcommand{\aep}{\api}
\newcommand{\Nep}[1]{\Npi{#1}}
\newcommand{\hk}[2]{\mleft(hk^{#1},hk^{#2}\mright)}
\newcommand{\hka}[1]{hk^{#1}}
\newcommand{\Cqo}{C_{\mathrm{qo}}}
\newcommand{\Cohat}{\hat{\Co}}
\newcommand{\SNHppoD}[1]{\SNfn{\HppoD}{#1}}
\newcommand{\SNfn}[2]{\SN{#2}_{#1}}
\newcommand{\ao}{a_{1}}
\newcommand{\at}{a_{2}}

\newcommand{\tat}{\text{ at }}
\newcommand{\tforall}{\text{ for all }}
\newcommand{\twith}{\text{ with }}

\crefname{assumption}{Assumption}{Assumptions}
\Crefname{assumption}{Assumption}{Assumptions}
\crefname{chapter}{Chapter}{Chapters}
\Crefname{chapter}{Chapter}{Chapters}
\crefname{listrequirement}{Requirement}{Requirements}
\Crefname{listrequirement}{Requirement}{Requirements}
\crefname{itemachievement}{Achievement}{Achievements}
\Crefname{itemachievement}{Achievement}{Achievements}
\crefname{itempart}{part}{parts}
\Crefname{itempart}{Part}{Parts}
\crefname{itemblank}{}{}
\Crefname{itemblank}{}{}
\crefname{problem}{Problem}{Problems}
\Crefname{problem}{Problem}{Problems}

% Making cleverref play nicely with newly defined environments
\crefname{convariable}{Condition}{Conditions}
\Crefname{convariable}{Condition}{Conditions}
\crefname{probvariable}{Problem}{Problems}
\Crefname{probvariable}{Problem}{Problems}
\crefname{myprobsec}{Problem}{Problems}
\Crefname{myprobsec}{Problem}{Problems}
\crefname{step}{Step}{Steps}
\Crefname{step}{Step}{Steps}
\crefname{point}{Point}{Points}
\Crefname{point}{Point}{Points}
\crefname{part}{Part}{Parts}
\Crefname{part}{Part}{Parts}
\crefname{condition}{Condition}{Conditions}
\Crefname{condition}{Condition}{Conditions}
\crefname{problem}{Problem}{Problems}
\Crefname{problem}{Problem}{Problems}

% Write out 'Figures' in full
\crefname{figure}{Figure}{Figures}
\Crefname{figure}{Figure}{Figures}

% Customising cleverref
\newcommand{\crefrangeconjunction}{--}

% Add a serial/Oxford comma by default.
\def\lastconj{, and~}
\newcommand{\creflastconjunction}{\lastconj}

% Making it do equations like eqref
\crefname{equation}{}{}
\Crefname{equation}{}{}

%%% Number equations within section
\numberwithin{equation}{chapter}


%%%%%%%%These conflict with SIAM style
\newtheorem{theorem}{Theorem}[chapter]

% This is only here if amsthm is disabled
%\newenvironment{proof}[1][Proof]{\noindent\emph{#1}\,}{\hfill$\square$}
\newtheorem{corollary}[theorem]{Corollary}
\newtheorem{definition}[theorem]{Definition}
\newtheorem{lemma}[theorem]{Lemma}
\newtheorem{proposition}[theorem]{Proposition}
\newtheorem{assumption}[theorem]{Assumption}
\newtheorem{example}[theorem]{Example}
\newtheorem{remark}[theorem]{Remark}
\newtheorem{summary}[theorem]{Summary}
\newtheorem{criterion}[theorem]{Criterion}
\newtheorem{notation}[theorem]{Notation}
\newtheorem{algorithm}[theorem]{Algorithm}
\newtheorem{construction}[theorem]{Construction}
\newtheorem{condition}[theorem]{Condition}
\newtheorem{problem}[theorem]{Problem}

\newcommand{\Ac}{\cA_{c}}
%\newcommand{\EXP}[1]{\mathbb{E}\mleft[#1\mright]}
%\newcommand{\HozD}{\Hoz{D}}
\newcommand{\Hoz}[1]{H^{1}_{0}\mleft(#1\mright)}
\newcommand{\Lc}{L_{c}}
%\newcommand{\LiD}{\Li{D}}
%\newcommand{\Li}[1]{L^{\infty}\mleft(#1\mright)}
\newcommand{\LpOXT}{L^{p}\mleft(\Omega;X_{2}\mright)}
\newcommand{\LpOY}{L^{p}\mleft(\Omega;Y\mright)}
\newcommand{\LqOXTS}{L^{q}\mleft(\Omega;X_{2}^{*}\mright)}
\newcommand{\LqOXT}{L^{q}\mleft(\Omega;X_{2}\mright)}
\newcommand{\LrOSXX}{L^{r}\mleft(\Omega;\SXX\mright)}
%\newcommand{\LtD}{\Lt{D}}
\newcommand{\LtOHozD}{\Lt{\Omega;\Hoz{D}}}
\newcommand{\LtOLtD}{\Lt{\Omega;\Lt{D}}}
%\newcommand{\Lt}[1]{L^{2}\mleft(#1\mright)}
\newcommand{\OFP}{\mleft(\Omega,\cF,\PP\mright)}
\newcommand{\Rmap}{J}
\newcommand{\SC}{\cS\cC}
\newcommand{\TC}{\cT_{\cC}}
\newcommand{\TR}{\cT_{\RR}}
\newcommand{\TX}{\cT_{X}}
%\newcommand{\abs}[1]{\mleft|#1\mright|}
\newcommand{\biggabs}[1]{\bigg|#1\bigg|}
\newcommand{\Bigabs}[1]{\Big|#1\Big|}
\newcommand{\ac}{a_{c}}
\newcommand{\amax}{a_{\max}}
\newcommand{\amin}{a_{\min}}
\newcommand{\ball}[3]{B^{#1}_{#2}\mleft(#3\mright)}
%\newcommand{\bas}{\begin{assumption}}
%\newcommand{\bde}{\begin{definition}}
%\newcommand{\bprob}{\begin{problem}}
%\newcommand{\bx}{\mathbf{x}}
%\newcommand{\dd}{\,\mathrm{d}}
%\newcommand{\eas}{\end{assumption}}
%\newcommand{\ede}{\end{definition}}
%\newcommand{\eprob}{\end{problem}}
%\newcommand{\grad}{\nabla}
\newcommand{\inftynormtop}{\cT_{\infty}}
\newcommand{\measrect}[2]{\cM_{#1\times#2}}
\newcommand{\openset}{O}
%\newcommand{\HmhGR}{\Hmh{\GR}}
%\newcommand{\Hmh}[1]{H^{-1/2}\mleft(#1\mright)}
\newcommand{\LiDR}{\Li{\DR}}
\newcommand{\LiDRCC}{\Li{\DR;\CC}}
\newcommand{\ptilde}{\tilde{p}}
\newcommand{\qtilde}{\tilde{q}}
%\newcommand{\set}[1]{\mleft\{#1\mright\}}
\newcommand{\tforalmostevery}{\text{ for almost every }}
\newcommand{\uW}{u_{W}}
%\newcommand{\tforall}{\text{ for all }}
\newcommand{\IPLtD}[2]{\IP{#1}{#2}_{\LtD}}
\newcommand{\form}[2]{a\mleft(#1,#2\mright)}
\newcommand{\NLtOHozD}[1]{\N{#1}_{\LtOHozD}}
\newcommand{\kmin}{\diffcoeff_{\min}}
\newcommand{\kmax}{\diffcoeff_{\max}}
\newcommand{\Acomega}{\cA_{c\mleft(\omega\mright)}}
\newcommand{\Lcomega}{\rhs_{c\mleft(\omega\mright)}}
\newcommand{\LtOX}{\Lt{\Omega;X}}
\newcommand{\LtOY}{\Lt{\Omega;Y}}
\newcommand{\LtOYas}{\LtOY^{*}}
\newcommand{\sol}{\cS}
\newcommand{\coeff}{c}
\newcommand{\comp}{\pi_{v}}
\newcommand{\compu}{\pi_{u}}
\newcommand{\pair}{P}
\newcommand{\eval}{\evalbase_{Z^{*}}}	
\newcommand{\rhs}{\cL}
\newcommand{\LqOYs}{\Lq{\Omega;Y^{*}}}
\newcommand{\toform}{\cA}
\newcommand{\toprod}{\pair_{v}}
\newcommand{\toprodu}{\pair_{u}}
\newcommand{\homspace}{\mathrm{B}\mleft(X,\Ys\mright)}
\newcommand{\LrOhomspace}{L^{r}\mleft(\Omega;\homspace\mright)}
\newcommand{\homspacetop}{\cT_{\homspace}}
\newcommand{\NLqOYs}[1]{\N{#1}_{\LqOYs}}
\newcommand{\bigNLqOYs}[1]{\bigN{#1}_{\LqOYs}}
\newcommand{\NYs}[1]{\N{#1}_{\Ys}}
\newcommand{\bigNYs}[1]{\bigN{#1}_{\Ys}}
\newcommand{\Ys}{Y^{*}}
\newcommand{\Nhomspace}[1]{\N{#1}_{\homspace}}
\newcommand{\NX}[1]{\N{#1}_{X}}
\newcommand{\bigNX}[1]{\bigN{#1}_{X}}
\newcommand{\NLrOhomspace}[1]{\N{#1}_{\LrOhomspace}}
\newcommand{\bigNLrOhomspace}[1]{\bigN{#1}_{\LrOhomspace}}
\newcommand{\NLtOX}[1]{\N{#1}_{\LtOX}}
\newcommand{\diffcoeff}{\kappa}
\newcommand{\dinfty}{d_{\infty}}
%\newcommand{\NLiD}[1]{\N{#1}_{\LiD}}
%\newcommand{\NLtD}[1]{\N{#1}_{\LtD}}
\newcommand{\acomega}{a_{c\mleft(\omega\mright)}}
\newcommand{\SAc}{\SA}
\newcommand{\AAA}{\mathfrak{A}}
\newcommand{\SLc}{\SL}
\newcommand{\LLL}{\mathfrak{L}}
%\newcommand{\ud}{u_{d}}
\newcommand{\vn}{v_{n}}
\newcommand{\LpOX}{\Lp{\Omega;X}}
\newcommand{\Lp}[1]{L^{p}\mleft(#1\mright)}
\newcommand{\NLpOX}[1]{\N{#1}_{\LpOX}}
\newcommand{\LoO}{\Lo{\Omega}}
\newcommand{\Lo}[1]{L^{1}\mleft(#1\mright)}
\newcommand{\TYs}{\cT_{Y^{*}}}
\newcommand{\boundconst}{B}
\newcommand{\NLoO}[1]{\N{#1}_{\LoO}}
\newcommand{\BigNLoO}[1]{\BigN{#1}_{\LoO}}
\newcommand{\LpOYas}{\Lp{\Omega;Y}^{*}}
\newcommand{\Acprime}{\cA_{c'}}
\newcommand{\Lcprime}{L_{c'}}
\newcommand{\Thomspace}{\cT_{\homspace}}
%\newcommand{\defn}[1]{\emph{#1}}
\newcommand{\fullrhs}{\rhs\circ\coeff}
\newcommand{\NY}[1]{\N{#1}_{Y}}
\newcommand{\Borel}[1]{\cB\mleft(#1\mright)}
\newcommand{\bigBorel}[1]{\cB\big(#1\big)}
\newcommand{\evalinv}{\eval^{-1}}
\newcommand{\Ballspace}[3]{B^{#1}_{#2}\mleft(#3\mright)}
\newcommand{\BZsrl}{\Ball{\Zs}{r}{l}}
\newcommand{\deltaH}{\delta_{\cH}}
\newcommand{\deltav}{\delta_{v}}
\newcommand{\BhomspacedvH}{\Ball{\homspace}{\deltav}{\cH}}
\newcommand{\BXdHv}{\Ball{X}{\deltaH}{v}}
\newcommand{\invball}{\evalinv\mleft(\BZsrl\mright)}
\newcommand{\LtOYs}{L^{2}\mleft(\Omega;Y^{*}\mright)}
\newcommand{\NLtO}[1]{\N{#1}_{\LtO}}
\newcommand{\BigNLtO}[1]{\BigN{#1}_{\LtO}}
\newcommand{\LtO}{\Lt{\Omega}}
\newcommand{\NLtOYs}[1]{\N{#1}_{\LtOYs}}
\newcommand{\NLtOY}[1]{\N{#1}_{\LtOY}}
\newcommand{\SA}{\AAA}
\newcommand{\SL}{\LLL}
\newcommand{\LiOhomspace}{\Li{\Omega;\homspace}}
%\newcommand{\vo}{v_{1}}
%\newcommand{\vt}{v_{2}}
%\newcommand{\half}{\frac{1}{2}}
\newcommand{\NLiO}[1]{\N{#1}_{\LiO}}
\newcommand{\LiO}{\Li{\Omega}}
\DeclareMathOperator{\esssup}{ess\,sup}
\DeclareMathOperator{\essinf}{ess\,inf}
\newcommand{\NLiOhomspace}[1]{\N{#1}_{\LiOhomspace}}
%\newcommand{\HoD}{\Ho{D}}
%\newcommand{\Ho}[1]{H^{1}\mleft(#1\mright)}
\newcommand{\IPX}[2]{\IP{#1}{#2}_{X}}
%\newcommand{\IP}[2]{\mleft\langle#1,#2\mright\rangle}
\newcommand{\IPBig}[2]{\Big\langle#1,#2\Big\rangle}
\newcommand{\IPbig}[2]{\big\langle#1,#2\big\rangle}
\newcommand{\toformcirccoeff}{\toform\circ\coeff}
%\newcommand{\ddPPomega}{\dd\PP\omega}
\newcommand{\Zs}{Z^{*}}
\newcommand{\homspaceZ}{\text{B}\mleft(X,\Zs\mright)}
\newcommand{\BhomspaceZdvH}{\Ball{\homspaceZ}{\deltav}{\cH}}
\newcommand{\NZs}[1]{\N{#1}_{\Zs}}
\newcommand{\bigNZs}[1]{\bigN{#1}_{\Zs}}
\newcommand{\NhomspaceZ}[1]{\N{#1}_{\homspaceZ}}
\newcommand{\bigNhomspaceZ}[1]{\bigN{#1}_{\homspaceZ}}
%\newcommand{\RRd}{\RR^{d}}
\newcommand{\posco}{\kappa}
\newcommand{\poscomin}{\posco_{\min}}
\newcommand{\poscomax}{\posco_{\max}}
\newcommand{\PPaeomegainOmega}{$\PP$-almost every $\omega \in \Omega$}
%\newcommand{\by}{\mathbf{y}}
\newcommand{\helmco}{n}
\newcommand{\helmcomin}{\helmco_{\min}}
\newcommand{\helmcomax}{\helmco_{\max}}
\newcommand{\LtOHoD}{\Lt{\Omega;\HoD}}
\newcommand{\IPY}[2]{\IP{#1}{#2}_{Y}}
\newcommand{\Lq}[1]{L^{q}\mleft(#1\mright)}
%\DeclareMathOperator{\supp}{ess\,supp}
\newcommand{\varmin}{\var_{\min}}
\newcommand{\varmax}{\var_{\max}}
\newcommand{\var}{n}
%\newcommand{\bxi}{\boldsymbol{\xi}}
%\newcommand{\vb}{\overline{v}}
\newcommand{\gradu}{\grad u}
\newcommand{\tendi}{\rightarrow \infty}
\newcommand{\gradv}{\grad v}
\newcommand{\cCR}{\cC_{R}}
\newcommand{\CSol}{\cC_{\mathrm{Sol}}}
\newcommand{\So}{S_{1}}
\newcommand{\Cot}{\widetilde{C_{1}}}
\newcommand{\Omegas}{\Omega_{\text{s}}}
\newcommand{\solDR}{\sol}
\newcommand{\graddD}{\grad_{\partial D}}
%\newcommand{\LtGI}{\Lt{\GI}}
%\newcommand{\GI}{\Gamma_{I}}
\newcommand{\NLtOLtD}[1]{\N{#1}_{\LtOLtD}}
\newcommand{\NLtOLtGI}[1]{\N{#1}_{\LtOLtGI}}
%\newcommand{\NLtGI}[1]{\N{#1}_{\LtGI}}
%\newcommand{\N}[1]{\mleft\|#1\mright\|}
\newcommand{\bigN}[1]{\big\|#1\big\|}
\newcommand{\BigN}[1]{\Big\|#1\Big\|}
\newcommand{\LtOLtGI}{\Lt{\Omega;\LtGI}}
\newcommand{\etamax}{\eta_{\max}}
\newcommand{\nsl}{n_{l}}
\newcommand{\nsu}{n_{u}}
\newcommand{\MpsiR}{M_{\psi,\RR}}
\newcommand{\mean}{n_{0}}
\newcommand{\RRLtOY}{\RR^{\LtOY}}
\newcommand{\Top}[1]{\cT_{#1}}
%\newcommand{\eps}{\varepsilon}
%\newcommand{\de}{\coloneqq}
%\newcommand{\st}{:}
\newcommand{\NS}[1]{\N{#1}_{S}}
\newcommand{\PMN}{P_{M,N}}
\newcommand{\pdiff}[2]{\frac{\partial #1}{\partial #2}}
%\newcommand{\DR}{D_{R}}
%\newcommand{\Dp}{D_{+}}
%\newcommand{\BR}{B_{R}}
%\newcommand{\HozDDR}{H^{1}_{0,D}\mleft(\DR\mright)}
%\newcommand{\LtDp}{\Lt{\Dp}}
%\newcommand{\compcont}{\subset\subset}
%\newcommand{\LiDpRR}{\Li{\Dp;\RR}}
%\newcommand{\nmin}{n_{\min}}
%\newcommand{\nmax}{n_{\max}}
%\newcommand{\tfae}{\text{ for almost every }}
\newcommand{\CzoDpRRdtd}{\Czo{\Dp;\spd}}
%\newcommand{\Amin}{A_{\min}}
\newcommand{\vtb}{\overline{\vt}}
\newcommand{\LtOHozDDR}{\Lt{\Omega;\HozDDR}}
\newcommand{\as}{\mathfrak{a}}
\newcommand{\Fs}{\mathfrak{F}}
\newcommand{\CzoDp}{\Czo{\Dp}}
%\newcommand{\Czo}[1]{C^{0,1}\mleft(#1\mright)}
\newcommand{\solSol}{\sol_{\textrm{Sol}}}
\newcommand{\WoiDRRRdtd}{\Woi{\DR;\spd}}
\newcommand{\WoiDRRRdtdsmall}{W^{1,\infty}(\DR;\spd)}
\newcommand{\WoiDRRRdtdsmallop}{W^{1,\infty}(\DR;\op)}
%\newcommand{\LiDRRR}{\Li{\DR;\RR}}
%\newcommand{\LtDR}{\Lt{\DR}}
\newcommand{\CzoDpRR}{\Czo{\Dp;\RR}}
\newcommand{\NLiDRRRdtd}[1]{\N{#1}_{\LiDRRRdtdop}}
\newcommand{\LiDRRRdtd}{\Li{\DR;\spd}}
\newcommand{\LiDRRRdtdop}{\Li{\DR;\op}}
%\newcommand{\NLiDRRR}[1]{\N{#1}_{\LiDRRR}}
%\newcommand{\NLtDR}[1]{\N{#1}_{\LtDR}}
\newcommand{\NHozDDR}[1]{\N{#1}_{\HozDDR}}
\newcommand{\LrOLiDRRRdtd}{\Lr{\Omega;\LiDRsym}}
\newcommand{\Lr}[1]{L^{r}\mleft(#1\mright)}
\newcommand{\LrOLiDRRR}{\Lr{\Omega;\LiDRRR}}
\newcommand{\LqOLtDR}{\Lq{\Omega;\LtDR}}
\newcommand{\IPGR}[2]{\IP{#1}{#2}_{\GR}}
\newcommand{\IPGRBig}[2]{\IPBig{#1}{#2}_{\GR}}
\newcommand{\IPGRbig}[2]{\IPbig{#1}{#2}_{\GR}}
\newcommand{\gvo}{\grad\vo}
\newcommand{\gvt}{\grad\vt}
%\newcommand{\GR}{\Gamma_{R}}
%\newcommand{\Nw}[1]{\NW{#1}}
\newcommand{\LtOLtDR}{\Lt{\Omega;\LtDR}}
\newcommand{\LiOLiDRRRdtd}{\Li{\Omega;\LiDRRRdtd}}
\newcommand{\LiOLiDRRR}{\Li{\Omega;\LiDRRR}}
\newcommand{\WoiRDpRRdtd}{W^{1,\infty}_{R}\mleft(\Dp;\spd\mright)}
\newcommand{\WoiRminDpRRdtd}{W^{1,\infty}_{R}\mleft(\Dp;\spd\mright)}

\newcommand{\WoiDpRRdtd}{\Woi{\Dp;\spd}}
%\newcommand{\Woi}[1]{W^{1,\infty}\mleft(#1\mright)}
%\newcommand{\RRdtd}{\RR^{d\times d}}
\newcommand{\Ls}{\mathfrak{L}}
\newcommand{\NLtOHozDDR}[1]{\N{#1}_{\LtOHozDDR}}
\newcommand{\BorelOX}{\cB\mleft(\Omega,X\mright)}
\newcommand{\Lizo}{\Li{\mleft(0,1\mright)}}
\newcommand{\Ltzo}{\Lt{\mleft(0,1\mright)}}
\newcommand{\Hozzo}{\Hoz{\mleft(0,1\mright)}}
\newcommand{\g}{\gamma}
\newcommand{\wo}{w_{1}}
\newcommand{\wt}{w_{2}}
\newcommand{\LtOHozzo}{\Lt{\Omega;\Hozzo}}
\newcommand{\LrOLizo}{\Lr{\Omega;\Lizo}}
\newcommand{\NLizo}[1]{\N{#1}_{\Lizo}}
\newcommand{\LtOHozo}{\Lt{\Omega;\Hozo}}
\newcommand{\Hozo}{\Ho{\mleft(0,1\mright)}}
\newcommand{\idzo}{\id_{\mleft(0,1\mright)}}
\DeclareMathOperator{\id}{id}
\newcommand{\NHozo}[1]{\N{#1}_{\Hozo}}
\newcommand{\NLtzo}[1]{\N{#1}_{\Ltzo}}
\newcommand{\Ao}{A_{1}}
\newcommand{\no}{n_{1}}
\newcommand{\fo}{f_{1}}
\newcommand{\Am}{A_{m}}
\newcommand{\nm}{n_{m}}
%\newcommand{\fm}{f_{m}}
\newcommand{\Az}{A_{0}}
\newcommand{\nz}{n_{0}}
\newcommand{\fz}{f_{0}}
%\newcommand{\DtN}{T_{R}}
\newcommand{\CDtN}{C_{\text{DtN}}}
\newcommand{\Clift}{C_{\lift}}
\newcommand{\NHhGR}[1]{\N{#1}_{\HhGR}}
%\newcommand{\HhGR}{\Hh{\GR}}
%\newcommand{\Hh}[1]{H^{1/2}\mleft(#1\mright)}
%\newcommand{\uz}{u_{0}}
\newcommand{\uo}{u_{1}}
\newcommand{\az}{a_{0}}
\newcommand{\HozDDRas}{\mleft(\HozDDR\mright)^{*}}
\newcommand{\NHozDDRas}[1]{\N{#1}_{\HozDDRas}}
\newcommand{\CzoDR}{\Czo{\DR}}
\newcommand{\WoiDR}{\Woi{\DR}}
%\newcommand{\Dm}{D_{-}}
%\newcommand{\GD}{\Gamma_{D}}
%\newcommand{\gD}{\gamma_{D}}
%\newcommand{\muo}{\mu_{1}}
%\newcommand{\mut}{\mu_{2}}
%\newcommand{\Co}{C_{1}}
\newcommand{\TS}{\Top{S}}
\newcommand{\evalY}{\evalbase_{Y^{*}}}
\newcommand{\evalbase}{\eta}
\newcommand{\Hone}{H_{1}}
%\newcommand{\Htwo}{H_{2}}
%\newcommand{\Hn}{H_{n}} % Test to see if this throws errors
%\newcommand{\No}{N_{1}}
%\newcommand{\Nt}{N_{2}}
%\newcommand{\HoDR}{\Ho{\DR}}
\newcommand{\ut}{u_{2}}
\newcommand{\owenshift}{0.5}
\newcommand{\owentextshift}{0.125}
\newcommand{\NLiOX}[1]{\N{#1}_{\LiOX}}
\newcommand{\LiOX}{\Li{\Omega;X}}
\newcommand{\M}{M}
\newcommand{\FM}{\cM}
\newcommand{\FN}{\cN}
\newcommand{\TopT}{\Top{T}}
\newcommand{\NV}[1]{\N{#1}_{V}}
\newcommand{\NLpOV}[1]{\N{#1}_{\LpOV}}
\newcommand{\LpOV}{\Lp{\Omega;V}}
\newcommand{\NLiOV}[1]{\N{#1}_{\LiOV}}
\newcommand{\LiOV}{\Li{\Omega;V}}
\newcommand{\D}{\cD}
\newcommand{\Ind}[1]{\mathbbm{1}_{#1}}
\newcommand{\evaly}{\zeta_{y}}
\newcommand{\diff}{\delta}
\newcommand{\Omegay}{\Omega_{y}}
\newcommand{\Omegayc}{\Omega^{c}_{y}}
\newcommand{\Omegat}{\widetilde{\Omega}}
\newcommand{\yn}{y_{n}}
\newcommand{\Omegayn}{\Omega_{\yn}}
\newcommand{\Omegaync}{\Omega^{c}_{\yn}}
\newcommand{\ynmseq}{\mleft(\ynm\mright)_{m \in \NN}}
\newcommand{\ynm}{y_{n_{m}}}
\newcommand{\Omegaynm}{\Omega_{\ynm}}
\newcommand{\Omegaynmc}{\Omega^{c}_{\ynm}}
\newcommand{\LtORR}{\Lt{\Omega;\RR}}
\newcommand{\TrR}{T_{R}}

\newcommand{\Xo}{X_{1}}
\newcommand{\Xm}{X_{m}}
\newcommand{\dm}{d_{m}}
\newcommand{\done}{d_{1}}
\newcommand{\xo}{x_{1}}
\newcommand{\xm}{x_{m}}
\newcommand{\yo}{y_{1}}
\newcommand{\ym}{y_{m}}
\newcommand{\dmetj}{d_{j}}
\newcommand{\xj}{x_{j}}
\newcommand{\yj}{y_{j}}
\newcommand{\aGm}{a_{\Az,\nz}}
\newcommand{\Lh}{L_{\fz}}
\newcommand{\Prodf}{P}
\newcommand{\measrectmany}[1]{\textrm{Rect}\mleft(#1\mright)}
\newcommand{\cCo}{\cC_{1}}
\newcommand{\cCm}{\cC_{m}}
\newcommand{\cCj}{\cC_{j}}
\newcommand{\Ctilde}{\widetilde{C}}
\newcommand{\nzmin}{n_{0,\min}}
\newcommand{\nzmax}{n_{0,\max}}
\newcommand{\Azmin}{A_{0,\min}}
\newcommand{\Azmax}{A_{0,\max}}
%\newcommand{\CCd}{\CC^{d}}
%\newcommand{\HolocDp}{H^{1}_{\textrm{loc}}\mleft(\Dp\mright)}
\newcommand{\consto}{\tau_{1}}
\newcommand{\constt}{\tau_{2}}
%\newcommand{\NLtDp}[1]{\N{#1}_{\LtDp}}
\newcommand{\Bo}{B_{1}}
\newcommand{\Bt}{B_{2}}
\newcommand{\ft}{f_{2}}
\newcommand{\LqOHozzos}{\Lq{\Omega;\Hozzos}}
\newcommand{\Hozzos}{\mleft(\Hozzo\mright)^{*}}
\newcommand{\NHozzos}[1]{\N{#1}_{\Hozzos}}
\newcommand{\NLqOHozzos}[1]{\N{#1}_{\LqOHozzos}}
\newcommand{\NLtOHozzos}[1]{\N{#1}_{\LtOHozzos}}
\newcommand{\LtOHozzos}{\Lt{\Omega;\Hozzos}}
\newcommand{\HhpDm}{\Hh{\pDm}}
\newcommand{\pDm}{\GD}
\newcommand{\uI}{u_{I}}
\newcommand{\LfO}{\Lf{\Omega}}
\newcommand{\Lf}[1]{L^{4}\mleft(#1\mright)}
\newcommand{\LfOLtD}{\Lf{\Omega;\LtD}}
\newcommand{\ri}{\mathrm{i}} % Nicked from Euan
\newcommand{\NLtOLtDR}[1]{\N{#1}_{\LtOLtDR}}
%\newcommand{\bxz}{\mathbf{\xz}}
\newcommand{\xz}{x_{0}}
\newcommand{\Eo}{E_{1}}
\newcommand{\Et}{E_{2}}
\newcommand{\Eth}{E_{3}}
\newcommand{\Ef}{E_{4}}
%\newcommand{\NHoDR}[1]{\N{#1}_{\HoDR}}
\newcommand{\lift}{\mathrm{lift}}
\newcommand{\Yj}{Y_{j}}
\newcommand{\psij}{\psi_{j}}
\newcommand{\Unif}{\mathrm{Unif}}
%\newcommand{\bzero}{\mathbf{0}}
\newcommand{\NCCd}[1]{\N{#1}_{\CCd}}
\newcommand{\WoiDRRR}{\Woi{\DR;\RR}}
\newcommand{\Zj}{Z_{j}}
\newcommand{\Psij}{\Psi_{j}}
\newcommand{\NopCCd}[1]{\N{#1}_{\op}}
\newcommand{\op}{\mathrm{op}}
%\newcommand{\LiDpRRdtd}{\Li{\Dp;\RRdtd}}
%\newcommand{\Amax}{A_{\max}}
\newcommand{\Cj}{C_{j}}
\newcommand{\fj}{f_{j}}
\newcommand{\CzoDb}{\Czo{\Db}}
\newcommand{\Db}{\clos{D}}
%\newcommand{\clos}[1]{\overline{#1}}
%\newcommand{\WoiD}{\Woi{D}}
\newcommand{\CzoDRRR}{\Czo{\DR;\RR}}
\newcommand{\RRN}{\RR^{N}}
\newcommand{\LrOCzoD}{\Lr{\CzoD}}
%\newcommand{\CzoD}{\Czo{D}}
%\newcommand{\tr}{\gamma_{R}}
\newcommand{\Ctr}{C_{\mathrm{tr}}}

%\newcommand{\HtDR}{\Ht{\DR}}
\newcommand{\ct}{c_{2}}
\newcommand{\SNHtDR}[1]{\SN{#1}_{\HtDR}}
\newcommand{\SN}[1]{\abs{#1}}
\newcommand{\co}{c_{1}}
\newcommand{\cz}{c_{0}}
\newcommand{\Apet}{B}
\newcommand{\NLiDRRRd}[1]{\N{#1}_{\LiDRRRd}}
\newcommand{\LiDRRRd}{\Li{\DR;\RRd}}
\newcommand{\CzoRIminDpRRdtd}{C^{0,1}_{R}\mleft(\Dp;\spd\mright)}
\newcommand{\LiRIDpRRdtd}{L^{\infty}_{R}\mleft(\Dp;\spd\mright)}
\newcommand{\CzoRIDpRRdtd}{C^{0,1}_{R}\mleft(\Dp;\spd\mright)}
\newcommand{\LiRoDpRR}{L^{\infty}_{R}\mleft(\Dp;\RR\mright)}
\newcommand{\LtRzDp}{L^{2}_{R}\mleft(\Dp\mright)}
%\newcommand{\CzoDRRRdtd}{\Czo{\DR;\RRdtd}}
\newcommand{\deltaz}{\delta_{0}}
\newcommand{\deltao}{\delta_{1}}
\newcommand{\deltat}{\delta_{2}}

%\newcommand{\NW}[1]{\N{#1}_{1,k}}
%\newcommand{\kz}{k_{0}}
\newcommand{\LpOHozD}{\Lp{\Omega;\HozD}}
\newcommand{\tiff}{\text{ iff }}
%\newcommand{\WoiDpclosRRdtd}{\Woi{\Dpclos;\RRdtd}}
%\newcommand{\Dpclos}{\clos{\Dp}}
%\newcommand{\WoiDpclosRR}{\Woi{\Dpclos;\RR}}
%\newcommand{\NHoD}[1]{\N{#1}_{\HoD}}
\newcommand{\LqOYas}{\LqOY^{*}}
\newcommand{\LqOY}{\Lq{\Omega;Y}}
%\newcommand{\CzoDpclosRR}{\Czo{\Dpclos;\RR}}
\newcommand{\CzoDRclosRRdtd}{\Czo{\DRclos;\spd}}
\newcommand{\DRclos}{\clos{\DR}}
\newcommand{\CzoDRclosRR}{\Czo{\DRclos;\RR}}
\newcommand{\scalarmult}{M_{v}}
\newcommand{\inc}{\iota}
\DeclareMathOperator{\spansetsub}{span}
\newcommand{\spanset}[1]{\spansetsub\set{#1}}
\newcommand{\QQFF}{\QQ_{\FF}}
\newcommand{\phio}{\phi_{1}}
\newcommand{\phit}{\phi_{2}}
\newcommand{\QQU}{\QQ_{U}}
\newcommand{\setsq}{S_{s,q}}
\newcommand{\NU}[1]{\N{#1}_{U}}
\newcommand{\dw}{r_{\omega}}
%\newcommand{\Dmclos}{\clos{\Dm}}
\newcommand{\RRp}{\RR^{+}}
\newcommand{\LrOCtzo}{\Lr{\Omega;\Ctzo}}
\newcommand{\Ctzo}{\Ctwo{\mleft(0,1\mright)}}
\newcommand{\Ctwo}[1]{C^{t}\mleft(#1\mright)}
\newcommand{\WoiDpRR}{\Woi{\Dp;\RR}}
\newcommand{\adiff}{\tilde{a}}
\newcommand{\sadiff}{\tilde{\as}}
\newcommand{\NWoiDRRRdtd}[1]{\N{#1}_{\WoiDRRRdtdsmallop}}
\newcommand{\WoiRDRRRdtd}{W^{1,\infty}_{R}\mleft(\DR;\spd\mright)}
\newcommand{\LiRDRRR}{L^{\infty}_{R}\mleft(\DR;\RR\mright)}
\newcommand{\LiRminDRRR}{L^{\infty}_{R,\min}\mleft(\DR;\RR\mright)}
\newcommand{\LtRDR}{L^{2}_{R}\mleft(\DR\mright)}
\newcommand{\Leb}{\lambda}
\newcommand{\Real}[1]{\mathfrak{R}\mleft(#1\mright)}
\newcommand{\kappaz}{\kappa_{0}}
\newcommand{\NLfOLtDR}[1]{\N{#1}_{\LfOLtDR}}
\newcommand{\LfOLtDR}{\Lf{\Omega;\LtDR}}
\newcommand{\IPLtO}[2]{\IP{#1}{#2}_{\LtO}}
\newcommand{\LiRDRRRdtd}{L^{\infty}_{R}\mleft(\DR;\spd\mright)}
\newcommand{\LiRminDRRRdtd}{L^{\infty}_{R}\mleft(\DR;\spd\mright)}
\newcommand{\WoiRminDRRRdtd}{W^{1,\infty}_{R}\mleft(\DR;\spd\mright)}
\newcommand{\LpHozDDR}{\Lp{\Omega;\HozDDR}}
\newcommand{\Do}{D_{1}}
\newcommand{\Dt}{D_{2}}
%\newcommand{\LiDp}{\Li{\Dp}}
%\newcommand{\LiDpCC}{\Li{\Dp;\CC}}
\newcommand{\LtOHoDR}{\Lt{\Omega;\HoDR}}
\newcommand{\bxib}{\overline{\bxi}}
\newcommand{\alphanz}{\alpha_{\nz}}
\newcommand{\alphaAz}{\alpha_{\Az}}
\newcommand{\NLtOHoD}[1]{\N{#1}_{\LtOHoD}}
\newcommand{\NWoiDRRR}[1]{\N{#1}_{\WoiDRRR}}
\newcommand{\metsp}{W}
\newcommand{\metric}{d}
\newcommand{\Omegahat}{\hat{\Omega}}
\newcommand{\jhat}{\hat{j}}
\newcommand{\Yjhat}{Y_{\jhat}}
\newcommand{\Omegatilde}{\widetilde{\Omega}}
\newcommand{\zetatilde}{\zeta}
\newcommand{\LtOHozDDRnormal}{L^2(\Omega,\HozDDR)}
\newcommand{\NLiDpRRdtd}[1]{\N{#1}_{\LiDpRRdtdop}}
\newcommand{\NLiDpRR}[1]{\N{#1}_{\LiDpRR}}
 % From stochastic paper
%%% Number equations within section
%\numberwithin{equation}{section}

%\newcounter{neverused}
%\newtheorem{theorem}{Theorem}[section]

%\newenvironment{proof}[1][Proof]{\noindent\emph{#1}\,}{\hfill$\square$}
%\newtheorem{corollary}[theorem]{Corollary}
%\newtheorem{definition}[theorem]{Definition}
%\newtheorem{lemma}[theorem]{Lemma}
%\newtheorem{proposition}[theorem]{Proposition}
%\newtheorem{assumption}{Assumption}
%\newtheorem{example}{Example}
%\newremark{remark}{Remark}
%\newsiamremark{summary}[theorem]{Summary}
%\newsiamremark{criterion}[theorem]{Criterion}
%\newtheorem{notation}[theorem]{Notation}
%\newtheorem{algorithm}[theorem]{Algorithm}
%\newtheorem{construction}[theorem]{Construction}
%\newtheorem{condition}{Condition}
%\newtheorem{problem}{Problem}
\newtheorem{neverused}{Labelled Problems and Theorems}

\newcounter{literallydoesntmatter}

% The following is inspired by, but, I feel, not copied from 
% https://tex.stackexchange.com/questions/53978/custom-theorem-numbering
%\newsiamthm{convariable}[neverused]{Condition}
\newtheorem{convariable}[literallydoesntmatter]{Condition}
\newcommand{\bconvar}[1]{\renewcommand{\theconvariable}{#1}\begin{convariable}}%\setcounter{theorem}{\value{theorem}-1}
%\newcommand{\bconvar}{\begin{convariable}}
\newcommand{\econvar}{\end{convariable}}
%\newsiamthm{probvariable}[neverused]{Problem}
\newtheorem{probvariable}[literallydoesntmatter]{Problem}
\newcommand{\bprobvar}[1]{\renewcommand{\theprobvariable}{#1}\begin{probvariable}}%\setcounter{theorem}{\value{theorem}-1}
\newcommand{\eprobvar}{\end{probvariable}}


\newcommand{\bconS}{\bconvar{S}}
\newcommand{\bconB}{\bconvar{B}}
\newcommand{\bconL}{\bconvar{L2}}
\newcommand{\bconAo}{\bconvar{A1}}
\newcommand{\bconLo}{\bconvar{L1}}
\newcommand{\bconA}{\bconvar{A2}}
\newcommand{\bconCo}{\bconvar{C1}}
\newcommand{\bconC}{\bconvar{C2}}
\newcommand{\bconAL}{\bconvar{AL}}
\newcommand{\bconK}{\bconvar{U}}

\newcommand{\bprobAE}{\bprobvar{AS}}
\newcommand{\bprobM}{\bprobvar{MAS}}
\newcommand{\bprobLT}{\bprobvar{SOAS}}
\newcommand{\bprobSVAR}{\bprobvar{SV}}


%\newtheorem{probo}[neverused]{Problem}
%\renewcommand{\theprobo}{{\ref{prob:sedp}}}
%
%\newtheorem{probt}[neverused]{Problem}
%\renewcommand{\theprobt}{{\ref{prob:msedp}}}
%
%\newtheorem{probth}[neverused]{Problem}
%\renewcommand{\theprobth}{{\ref{prob:somsedp}}}
%
%\newtheorem{probf}[neverused]{Problem}
%\renewcommand{\theprobf}{{\ref{prob:svsedp}}}
%










%%%%%%%%%%%Eike's new theorem version 1%%%%%%%%%%%%%%%%

%\newcounter{mycount}
%\newsiamthm{myprob}{Problem}
%\renewcommand{\themycount}{\Alph{mycount}}

%\addtoreset{mycount}{section}

%%%%%%%%%%%%Eike's new theorem version 2%%%%%%%%%%%5
%%%%%% give e.g Theorem 2.A 
%\newcounter{mycountsec}
%\newsiamthm{myprobsec}[mycountsec]{Problem}
%\newsiamthm{myprobsec}{Problem}
%\renewcommand{\themycountsec}{\thesection.\Alph{mycountsec}}

%\addtoreset{mycount}{section}
%%%%%% line above resets counter at beginning of section

%%%%%%%%%%%%%%%%%%%%%%%%%%%%%%%%%%


% Shortcuts for special theorems etc.
\newcommand{\bmps}{\begin{myprobsec}}
\newcommand{\emps}{\end{myprobsec}}
\newcommand{\bmp}{\begin{myprob}}
\newcommand{\emp}{\end{myprob}}

\newcommand{\Ml}{M_{l}}
\newcommand{\Mlmo}{M_{l-1}}
\newcommand{\Qhltilde}{\widetilde{\QoI}_{\hl}}
\newcommand{\Qhlmotilde}{\widetilde{\QoI}_{\hlmo}}
\newcommand{\QMl}{CHANGE\QoI_{\hl}}
\newcommand{\QMlmo}{CHANGE\QoI_{\hlmo}}
\newcommand{\QoI}{Q}
\newcommand{\uhl}{u_{l}}
\newcommand{\hl}{h_{l}}
\newcommand{\Yl}{Y_{l}}
\newcommand{\Ym}{Y_{m}}
\newcommand{\Yz}{Y_{0}}
\newcommand{\Qhztilde}{\widetilde{\QoI}_{\hz}}
\newcommand{\QhLtilde}{\widetilde{\QoI}_{\hL}}
\newcommand{\QMz}{CHANGE\QoI_{\hz}}
\newcommand{\Mz}{M_{0}}
\newcommand{\Ylhat}{\hat{\Yl}}
\newcommand{\Nl}{N_{l}}
\newcommand{\Ylj}{\Yl^{(j)}}
\newcommand{\QhatMLML}{CHANGE\hat{\QoI}^{ML}_{\hL}}
\newcommand{\QhatMLhL}{\hat{\QoI}^{ML}_{\hL}}
\newcommand{\QhatMC}{\hat{\QoI}^{MC}_{h}}
\newcommand{\ML}{M_{L}}
%\newcommand{\co}{c_{1}}
%\newcommand{\ct}{c_{2}}
\newcommand{\cth}{c_{3}}
\newcommand{\Cl}{\cC_{l}}
\newcommand{\Vl}{V_{l}}
\newcommand{\err}[1]{\mathrm{Err}\mleft(#1\mright)}
\newcommand{\CMLML}{CHANGE\cC^{ML}_{M_{L}}}
\newcommand{\CMLhL}{\cC^{ML}_{\hL}}
\renewcommand{\ln}[1]{\mathrm{ln}\mleft(#1\mright)}
\newcommand{\Cppw}{\Ccoarse}
\newcommand{\CFEM}{C_{FEM}}
\newcommand{\uinfty}{u_{\infty}}
\newcommand{\uS}{u^{S}}
%\newcommand{\uI}{u^{I}}
\newcommand{\CMh}{C_{Mh}}
\newcommand{\hlmo}{h_{l-1}}
\newcommand{\uhlmo}{u_{l-1}}
\newcommand{\hz}{h_{0}}
\newcommand{\NMC}{N_{MC}}
\newcommand{\QMLMC}{Q_{\MLMC}}
\newcommand{\MLMC}{M_{\LMC}}
%\newcommand{\LMC}{L_{MC}}
\newcommand{\LMC}{\tilde{L}}
\newcommand{\V}{\cV}
\newcommand{\CMC}{\Cost{\QhatMC}}
\newcommand{\hL}{h_{L}}
\newcommand{\hMC}{h_{MC}}
%\newcommand{\ceil}[1]{\lceil#1\rceil}
\newcommand{\Vj}{V_{j}}
%\newcommand{\Cj}{\cC_{j}}
\newcommand{\errQhatMLhL}{\err{\QhatMLhL}}
\newcommand{\QhL}{Q_{\hL}}
\newcommand{\hLMC}{h_{LMC}}
\newcommand{\Ccoarse}{C_{\mathrm{coarse}}}
\newcommand{\coarseexp}{\mu}
\newcommand{\splitting}{\chi}
\newcommand{\func}{\eta}
%\newcommand{\Do}{D_{1}}
%\newcommand{\Dt}{D_{2}}
%\newcommand{\muo}{\mu_{1}}
%\newcommand{\mut}{\mu_{2}}
%\newcommand{\Az}{A_{0}}
%\newcommand{\nz}{n_{0}}
%\newcommand{\bxz}{\bx_{0}}
\newcommand{\LD}{L_{D}}
%\newcommand{\LI}{L_{I}}
\newcommand{\aD}{a_{D}}
\newcommand{\aI}{a_{I}}
\newcommand{\NLtOLtGD}[1]{\N{#1}_{\LtOLtGD}}
\newcommand{\LtOLtGD}{\Lt{\Omega;\LtGD}}
\newcommand{\Cth}{C_{3}}
\newcommand{\Cthtilde}{\widetilde{C}_{3}}
%\newcommand{\NLtOLtGI}[1]{\N{#1}_{\LtOLtGI}}
%\newcommand{\LtOLtGI}{\Lt{\Omega;\LtGI}}
\newcommand{\Cf}{C_{4}}
\newcommand{\Cfi}{C_{5}}
\newcommand{\hetbeta}{\zeta}
\newcommand{\Cotilde}{\widetilde{C}_{1}}
\newcommand{\nmaxGI}{n_{\max,\GI}}
\newcommand{\Coo}{C^{1,1}}
\newcommand{\epsbx}{\eps_{\bx}}
%\newcommand{\Ball}[2]{B_{#1}\mleft(#2\mright)}
%\newcommand{\bzero}{\mathbf{0}}
%\newcommand{\IPGI}[2]{\IP{#1}{#2}_{\GI}}
%\newcommand{\HmhGI}{\Hmh{\GI}}
%\newcommand{\Hmh}[1]{H^{-1/2}\mleft(#1\mright)}
%\newcommand{\dd}{\mathrm{d}}
%\newcommand{\Leb}{\lambda}
%\newcommand{\LiDp}{\Li{\Dp}}
%\newcommand{\LiDpCC}{\Li{\Dp;\CC}}
%\newcommand{\compcont}{\subset\subset}
%\newcommand{\LtDp}{\Lt{\Dp}}
%\newcommand{\LtDpRR}{\Lt{\Dp;\RR}}
%\newcommand{\LiDpRR}{\Li{\Dp;\RR}}
%% \newcommand{\vtb}{\overline{v}_{2}}
%% \newcommand{\LtOHozDD}{\Lt{\Omega;\HozDD}}
%% \newcommand{\LtOLtD}{\Lt{\Omega;\LtD}}
%% \newcommand{\LiOLiDRR}{\Li{\Omega;\LiDRR}}
%% \newcommand{\WoiDRR}{\Woi{D;\RR}}
%% \newcommand{\Woi}[1]{W^{1,\infty}\mleft(#1\mright)}
%% \newcommand{\LtO}{\Lt{\Omega}}
%% \newcommand{\LfO}{\Lf{\Omega}}
%% \newcommand{\Lf}[1]{L^{4}\mleft(#1\mright)}
%% \newcommand{\NLtOLtD}[1]{\N{#1}_{\LtOLtD}}
%% \newcommand{\NLoO}[1]{\N{#1}_{\LoO}}
%% \newcommand{\LoO}{\Lo{\Omega}}
%% \newcommand{\Lo}[1]{L^{1}\mleft(#1\mright)}
%% \newcommand{\Yj}{Y_{j}}
%% \newcommand{\psij}{\psi_{j}}
%% \newcommand{\Unif}{\mathrm{unif}}
%% \newcommand{\NTn}[1]{\mathrm{NT}_{n}\mleft(#1\mright)}
\newcommand{\Lconst}{C_{L}}
%\newcommand{\ud}{u_{d}}
%% \DeclareMathOperator{\esssup}{ess\,sup}
%\newcommand{\us}{u_{s}}
%\newcommand{\Htht}[1]{H^{3/2}\mleft(#1\mright)}
\newcommand{\HthtGD}{\Htht{\GD}}
\newcommand{\HthtD}{\Htht{D}}
\newcommand{\HozDDprime}{\HozDD'}
%\newcommand{\HhGD}{\Hh{\GD}}
%\newcommand{\WoiDRRdtd}{\Woi{D;\RRdtd}}
\newcommand{\appspace}{U} % Space used in the appendix
\newcommand{\WspU}{\Wsp{\appspace}}
\newcommand{\Wsp}[1]{W^{s,p}\mleft(#1\mright)}
\newcommand{\WsmoppdU}{\Wsmopp{\dU}}
\newcommand{\dU}{\partial \appspace}
\newcommand{\Wsmopp}[1]{W^{s-\frac1p,p}\mleft(#1\mright)}
\newcommand{\udtilde}{\widetilde{u}_{d}}
\newcommand{\WspRRdsmDm}{\Wsp{\RRd\setminus\Dm}}
\newcommand{\Qhtilde}{\widetilde{Q}_{h}}
\newcommand{\Cmesh}{C_{\mathrm{mesh}}}
\newcommand{\mesh}{HERE DAWG}
%\newcommand{\RRp}{\RR^{+}}
\newcommand{\hmaxomega}{h^{\max}_{\omega}}
\newcommand{\homega}{h_{\omega}}
\newcommand{\hzomega}{{h_0}_{\omega}}
\newcommand{\Qtildehomega}{\widetilde{Q}_{\homega}}
\newcommand{\Qh}{Q_{h}}
\newcommand{\Ch}{\cC_{h}}
\newcommand{\cthtilde}{\widetilde{c}_{3}}
\newcommand{\cotilde}{\widetilde{c}_{1}}
\newcommand{\Cost}[1]{\cC\mleft(#1\mright)}
\newcommand{\LpO}{\Lp{\Omega}}
%\newcommand{\Lp}[1]{L^{p}\mleft(#1\mright)}
\newcommand{\Lqgamma}[1]{L^{q\gamma}\mleft(#1\mright)}
\newcommand{\LqgammaO}{\Lqgamma{\Omega}}
\newcommand{\NLpO}[1]{\N{#1}_{\LpO}}
\newcommand{\NLqgammaO}[1]{\N{#1}_{\LqgammaO}}
\newcommand{\Qhomegatilde}{\widetilde{Q}_{\homega}}
%\newcommand{\ddPPomega}{\dd\PP(\omega)}
\newcommand{\csum}[1]{C_{\mathrm{sum},#1}}
\newcommand{\csumdelta}{\csum{\delta}}
\newcommand{\csumgammambetat}{\csum{\frac{\gamma-\beta}{2}}}
\newcommand{\Cgammagtrbeta}{C_{\gamma > \beta}}
\newcommand{\Cgammalessbeta}{C_{\gamma < \beta}}
\newcommand{\Qhtildesj}{\Qhtilde^{(j)}}
\newcommand{\Omegabad}{\Omega_{\mathrm{bad}}}
\newcommand{\omegasj}{\omega^{(j)}}
\newcommand{\ho}{h_{1}}
\newcommand{\uhL}{u_{\hL}}
\newcommand{\uhz}{u_{\hz}}
\newcommand{\Yzhat}{\Yhat_{0}}
\newcommand{\Nz}{N_{0}}
\newcommand{\NL}{N_{L}}
\newcommand{\kh}{k_{h}}
\newcommand{\Qhl}{Q_{\hl}}
\newcommand{\Qhlmo}{Q_{\hlmo}}
\newcommand{\Qhat}{\hat{Q}}
\newcommand{\Yhatz}{\Yhat_{0}}
\newcommand{\Qhz}{Q_{\hz}}
\newcommand{\Yhatl}{\Yhat_{l}}
\newcommand{\NHo}[1]{\N{#1}_{H^{1}}}
\newcommand{\omegaso}{\omega^{(1)}}
\newcommand{\omegast}{\omega^{(2)}}
\newcommand{\Cdata}{C_{\mathrm{data}}}
\newcommand{\Qhomega}{Q_{\homega}}
\newcommand{\Qhmaxomega}{Q_{\hmaxomega}}




\title{The Helmholtz Equation in Heterogeneous and Random Media: Analysis and Numerics}
\author{Owen Rhys Pembery}
%\degree{Doctor of Philosophy}
%\department{Department of Mathematical Sciences}
%\degreemonthyear{September 2019}

\begin{document}

\maketitle

\setcounter{tocdepth}{2}
%\renewcommand{\familydefault}{\sfdefault}

\titleformat{\chapter}[display]{\normalfont\Huge}{\large\textsc{\chaptertitlename}\ \thechapter}{20pt}{\Huge}

\titleformat{\section}{\normalfont\Large}{\textsc{\thesection}}{1em}{\sc}

\titleformat{\subsection}{\normalfont\large\bfseries}{\thesubsection}{1em}{}
\titleformat{\subsubsection}{\normalfont\large\bfseries}{\thesubsubsection}{1em}{}
\titleformat{\paragraph}[runin]{\normalfont\normalsize}{\textit{\theparagraph}}{1em}{\textit}
\titleformat{\subparagraph}[runin]{\normalfont\normalsize\bfseries}{\thesubparagraph}{1em}{}

\titlespacing*{\chapter}{0pt}{0pt}{0pt}
\titlespacing*{\section}{0pt}{3.5ex plus 1ex minus .2ex}{0ex}%{2.3ex plus .2ex}
\titlespacing*{\subsection}{0pt}{3.25ex plus 1ex minus .2ex}{1.5ex plus .2ex}
\titlespacing*{\subsubsection}{0pt}{3.25ex plus 1ex minus .2ex}{0ex}%{1.5ex plus .2ex}
\titlespacing*{\paragraph}{0pt}{3.25ex plus 1ex minus .2ex}{1em}
\titlespacing*{\subparagraph} {\parindent}{3.25ex plus 1ex minus .2ex}{1em}


%% \pagestyle{fancy}

%% \fancyhead[LE,RO]{\thepage}
%% \fancyhead[RE]{\slshape \leftmark}
%% \fancyhead{LO]
%% \fancyfoot[C]{\thepage}
%% \headrulewidth 0pt
%% \footrulewidth 0 pt

\hyphenation{Helm-holtz}
\hyphenation{Prob-lem}


\tableofcontents



\chapter{Introduction}\label{chap:intro}
\section{The Subjects of the Thesis}
The subjects of this thesis are rigorous theory and fast methods for the stochastic Helmholtz equation
\beq\label{eq:introhh}
\grad \cdot \mleft(A\grad u\mright) + k^2 \, n \, u = -f,
\eeq
where $A,$ $n,$ and $f$ are random fields (i.e., they are spatially heterogeneous and random). We are particularly interested in theory and methods that are applicable for large values of the wavenumber $k$, as the case of large $k$ is of interest in applications, and theoretically and computationally demanding.

\subsection{Motivation from Applications for the High-Frequency Stochastic Helmholtz Equation}\label{sec:motivapp}

The Helmholtz equation is the simplest possible model of wave propagation. Indeed, if one seeks time-harmonic solutions of the scalar wave equation
\beq\label{eq:introwave}
n \, \frac{\partial^2 U}{\partial t^2} - \grad \cdot\mleft(A\grad U\mright) = F,
\eeq
that is, solutions of the form $U(t,\bx) = e^{-ikt}u(\bx)$, where $F(t,\bx) = e^{-ikt}f(\bx)$, then the spatial part $u$ satisfies \eqref{eq:introhh}; equivalently, \eqref{eq:introhh} is the Fourier transform in time  of \eqref{eq:introwave}. In certain scenarios, the time-harmonic Maxwell's equations reduce to \eqref{eq:introhh}, see, e.g., \cite[Remark 2.1]{MoSp:19} for this derivation.

The physical motivation for studying \eqref{eq:introhh} is, therefore, any physical scenario in which wave propagation can be modelled by either \eqref{eq:introwave} or Maxwell's equations. One prominent example of the usage of \eqref{eq:introwave} is in subsurface imaging, where the rock structures of the earth's crust are imaged by generating waves at the earth's surface, and recording the reflections of these waves from the rock structures. The waves within the earth's rock structures satisfy the elastic wave equation, but under the so-called acoustic approximation, this can be approximated by \eqref{eq:introwave}; see \cite[Section 1.2]{Ch:15} for a physical derivation of the elastic wave equation, and \cite[Section 1.2.6]{Ch:15} for a derivation and discussion of the acoustic approximation. Other physical scenarios involving waves modelled by either \eqref{eq:introwave} or Maxwell's equations are the propagation of sound in an inviscid fluid \cite[Section 2.1]{CoKr:13}, and Microwave imaging (see, e.g., \cite[Section 6.4]{BoDoGrSpTo:19}).

A mathematical and computational motivation for studying \eqref{eq:introhh} is that many of the difficulties one encounters when studying and numerically solving more complex wave-propagation models, such as the elastic wave equation, are also encountered with \eqref{eq:introhh}. Therefore \eqref{eq:introhh} is an appropriate starting point for mathematical study of, and numerical algorithms for, wave propagation models.

We now consider three characteristics of the above examples that will drive the theoretical and computational work in this thesis: high effective frequency, heterogeneity, and stochasticity.

The real-life examples above can have high effective frequency, that is, the wavenumber $k$ is large. The wavenumber may be large because either (i) the physical frequency is large, or (ii) the waves are low-frequency, but propagate over a large domain. Situation (i) arises in, e.g., non-destructive testing, where waves of frequency $1\times10^6$--$1\times10^7$ Hz (see, e.g., \cite{BS:14}) are passed through materials to image their interior, and situation (ii) arises in, e.g., seismic imaging, where the wave frequencies are in the range 1-100 Hz (see, e.g., \cite{Sc}) but the domain of interest is on the kilometre scale\footnote{E.g., the SEG Overthrust model \cite{AmBrKu:97}, a common benchmark for seismic imaging applications, has domain size 20km $\times$ 20km $\times$ 4.65 km.}. These low-frequency, large-domain problems have many wavelengths in the domain, and hence, when they are scaled to a domain of size $\approx 1,$ they give rise to problems with large (effective) frequency.

Also, many of the examples above are modelled by the Helmholtz equation with \emph{heterogeneous} coefficients. For example, in subsurface imaging of the Earth's crust, waves will pass through the sea, different types of rock, and materials contained within these rocks, such as water or oil. Each of these materials will have different properties, such as density and Lam\'e parameters, and therefore the coefficients $A$ and $n$ in \eqref{eq:introhh} will be heterogeneous (see, e.g. \cite[Section 1.2.4]{Ch:15} for an explanation of how the density and Lam\'e parameters manifest themselves in \eqref{eq:introwave}).

To understand the prescence of stochasticity in the above examples, we consider two possible problems associated with \eqref{eq:introhh}:
\ben
\item The \emph{forward problem}, where one knows the coefficients $A$ and $n$, and wishes to find solution $u$, and
\item The \emph{inverse problem}, where one knows the solution $u$ and wishes to find the coefficients $A$ and $n$.
  \een

The Helmholtz equation with \emph{random} coefficients arises when we model physical situations with uncertainty in the material parameters; this uncertainty can arise in both the forward and inverse problem. In the inverse problem, where one has sent an incident wave into an unknown medium, recorded the scattered wave, and reconstructs the medium, there will be uncertainties inherent in the process. For example, the scattered wave will only be recorded at discrete points in space, rather than everywhere, and these recordings will be subject to measurement error. These uncertanties in the measurement processes will result in uncertanties in the inferred properties of the medium. Alternatively, uncertainty arises in the forward problem, when we are already aware of uncertainty in our knowledge of the medium, and we wish to quantify the uncertainty in the wave passing through the uncertain medium. This occurs, for example, in radar imaging of ice sheets, where one wishes to know properties of the wave scattered by the ice, as in \cite{JiPi:18}.

This thesis will only focus on UQ for the forward problem. The forward problem and inverse problem share the common computational difficulty of needing to solve many (deterministic) realisations of \eqref{eq:introhh}. Whether the uncertainty in $A$ and $n$ has arisen as a result of the inverse or forward problem, most UQ algorithms will require many samples of the (random) solution of \eqref{eq:introhh}. As will be discussed below, obtaining one sample of the solution of \eqref{eq:introhh} is a considerable computational task, and so obtaining many (and `many' could easily mean thousands) of such samples is an even harder task. Reducing the computational cost of obtaining many samples of the solution of \eqref{eq:introhh} will be a main focus of the algorithms developed and studied in this thesis.%The computational techniques developed in this thesis are applicable to the inverse problem for \eqref{eq:introhh}. However, the inverse problem introduces other computational difficulties, unrelated to solving \eqref{eq:introhh},  Such difficulties are related to sampling the random coefficients $A$ and $n$, e.g., in wave propagation inverse problems the distributions of $A$ and $n$ may well be multi-modal.

\subsection{Solving the Helmholtz Equation Numerically}\label{sec:numsolve}

We have just stated that it is hard to solve the (deterministic) Helmholtz equation
\beq\label{eq:introdet}
\grad \cdot (\Ad \grad \ud) + k^2 \,\nd\,\ud = -\fd,
\eeq
i.e., a single realisation of \eqref{eq:introhh}, numerically; we now provide some background on why this is the case. When solving \eqref{eq:introdet} numerically we discretise it to obtain a linear system
\beq\label{eq:intromat}
\Amat \uvec = \fvec.
\eeq
\paragraph{Issues from finite elements} We are exclusively concerned with discretisation via finite elements, see \cref{chap:background} for the details of such a discretisation. The linear systems \eqref{eq:intromat} arising from standard finite-element discretisations of \eqref{eq:introdet} are hard to solve, as the matrices $\Amat$ are:
\ben
\item\label[itemblank]{it:nonh} non-Hermitian,
\item\label[itemblank]{it:indef} indefinite, and
  \item\label[itemblank]{it:large} large.
\een
    We will now briefly discuss each of these properties in turn, outlining why the matrices $\Amat$ have these properties, and how these properties affect the numerical solution of \eqref{eq:intromat}.

    For \cref{it:nonh}, the matrices $\Amat$ are non-Hermitian because the underlying boundary-value problems are not self-adjoint (see \cite[Section 4.2]{Sp:15} for a discussion of the non-selfadjointness of exterior-boundary-value problems for the Helmholtz equation). This lack of self-adjointness means the sesquilinear forms arising in standard variational formulations of \eqref{eq:introhh} are not conjugate symmetric, and this lack of conjugate symmetry is inherited by the discretisation matrices $\Amat$. If one uses an iterative solver for the linear system \cref{eq:intromat}, then the non-Hermitian nature of the matrices means that a solver that is suitable for such matrices, such as GMRES, must be used.

    For \cref{it:indef}, the matrices $\Amat$ are indefinite because the sesquilinear forms arising from standard variational formulations of \eqref{eq:introhh} are not coercive. This indefiniteness means that GMRES applied to \eqref{eq:intromat} may exhibit poor convergence properties, especially as the wavenumber $k$ is increased---see \cref{fig:GMRES}. In addition, the standard convergence theory for GMRES does not apply to indefinite systems, and so proving convergence results is also challenging.%\edsdnote{Both---similar to the previous paragraph, I'm finding it hard to find anything else to say here, have I missed anything?}

        \begin{figure}[h]
      \centering
      %% Creator: Matplotlib, PGF backend
%%
%% To include the figure in your LaTeX document, write
%%   \input{<filename>.pgf}
%%
%% Make sure the required packages are loaded in your preamble
%%   \usepackage{pgf}
%%
%% Figures using additional raster images can only be included by \input if
%% they are in the same directory as the main LaTeX file. For loading figures
%% from other directories you can use the `import` package
%%   \usepackage{import}
%% and then include the figures with
%%   \import{<path to file>}{<filename>.pgf}
%%
%% Matplotlib used the following preamble
%%   \usepackage{fontspec}
%%   \setmainfont{DejaVuSerif.ttf}[Path=/home/owen/progs/firedrake-complex/firedrake/lib/python3.5/site-packages/matplotlib/mpl-data/fonts/ttf/]
%%   \setsansfont{DejaVuSans.ttf}[Path=/home/owen/progs/firedrake-complex/firedrake/lib/python3.5/site-packages/matplotlib/mpl-data/fonts/ttf/]
%%   \setmonofont{DejaVuSansMono.ttf}[Path=/home/owen/progs/firedrake-complex/firedrake/lib/python3.5/site-packages/matplotlib/mpl-data/fonts/ttf/]
%%
\begingroup%
\makeatletter%
\begin{pgfpicture}%
\pgfpathrectangle{\pgfpointorigin}{\pgfqpoint{6.400000in}{4.800000in}}%
\pgfusepath{use as bounding box, clip}%
\begin{pgfscope}%
\pgfsetbuttcap%
\pgfsetmiterjoin%
\definecolor{currentfill}{rgb}{1.000000,1.000000,1.000000}%
\pgfsetfillcolor{currentfill}%
\pgfsetlinewidth{0.000000pt}%
\definecolor{currentstroke}{rgb}{1.000000,1.000000,1.000000}%
\pgfsetstrokecolor{currentstroke}%
\pgfsetdash{}{0pt}%
\pgfpathmoveto{\pgfqpoint{0.000000in}{0.000000in}}%
\pgfpathlineto{\pgfqpoint{6.400000in}{0.000000in}}%
\pgfpathlineto{\pgfqpoint{6.400000in}{4.800000in}}%
\pgfpathlineto{\pgfqpoint{0.000000in}{4.800000in}}%
\pgfpathclose%
\pgfusepath{fill}%
\end{pgfscope}%
\begin{pgfscope}%
\pgfsetbuttcap%
\pgfsetmiterjoin%
\definecolor{currentfill}{rgb}{1.000000,1.000000,1.000000}%
\pgfsetfillcolor{currentfill}%
\pgfsetlinewidth{0.000000pt}%
\definecolor{currentstroke}{rgb}{0.000000,0.000000,0.000000}%
\pgfsetstrokecolor{currentstroke}%
\pgfsetstrokeopacity{0.000000}%
\pgfsetdash{}{0pt}%
\pgfpathmoveto{\pgfqpoint{0.800000in}{0.528000in}}%
\pgfpathlineto{\pgfqpoint{5.760000in}{0.528000in}}%
\pgfpathlineto{\pgfqpoint{5.760000in}{4.224000in}}%
\pgfpathlineto{\pgfqpoint{0.800000in}{4.224000in}}%
\pgfpathclose%
\pgfusepath{fill}%
\end{pgfscope}%
\begin{pgfscope}%
\pgfsetbuttcap%
\pgfsetroundjoin%
\definecolor{currentfill}{rgb}{0.000000,0.000000,0.000000}%
\pgfsetfillcolor{currentfill}%
\pgfsetlinewidth{0.803000pt}%
\definecolor{currentstroke}{rgb}{0.000000,0.000000,0.000000}%
\pgfsetstrokecolor{currentstroke}%
\pgfsetdash{}{0pt}%
\pgfsys@defobject{currentmarker}{\pgfqpoint{0.000000in}{-0.048611in}}{\pgfqpoint{0.000000in}{0.000000in}}{%
\pgfpathmoveto{\pgfqpoint{0.000000in}{0.000000in}}%
\pgfpathlineto{\pgfqpoint{0.000000in}{-0.048611in}}%
\pgfusepath{stroke,fill}%
}%
\begin{pgfscope}%
\pgfsys@transformshift{1.526465in}{0.528000in}%
\pgfsys@useobject{currentmarker}{}%
\end{pgfscope}%
\end{pgfscope}%
\begin{pgfscope}%
\definecolor{textcolor}{rgb}{0.000000,0.000000,0.000000}%
\pgfsetstrokecolor{textcolor}%
\pgfsetfillcolor{textcolor}%
\pgftext[x=1.526465in,y=0.430778in,,top]{\color{textcolor}\sffamily\fontsize{10.000000}{12.000000}\selectfont \(\displaystyle 20\)}%
\end{pgfscope}%
\begin{pgfscope}%
\pgfsetbuttcap%
\pgfsetroundjoin%
\definecolor{currentfill}{rgb}{0.000000,0.000000,0.000000}%
\pgfsetfillcolor{currentfill}%
\pgfsetlinewidth{0.803000pt}%
\definecolor{currentstroke}{rgb}{0.000000,0.000000,0.000000}%
\pgfsetstrokecolor{currentstroke}%
\pgfsetdash{}{0pt}%
\pgfsys@defobject{currentmarker}{\pgfqpoint{0.000000in}{-0.048611in}}{\pgfqpoint{0.000000in}{0.000000in}}{%
\pgfpathmoveto{\pgfqpoint{0.000000in}{0.000000in}}%
\pgfpathlineto{\pgfqpoint{0.000000in}{-0.048611in}}%
\pgfusepath{stroke,fill}%
}%
\begin{pgfscope}%
\pgfsys@transformshift{2.528485in}{0.528000in}%
\pgfsys@useobject{currentmarker}{}%
\end{pgfscope}%
\end{pgfscope}%
\begin{pgfscope}%
\definecolor{textcolor}{rgb}{0.000000,0.000000,0.000000}%
\pgfsetstrokecolor{textcolor}%
\pgfsetfillcolor{textcolor}%
\pgftext[x=2.528485in,y=0.430778in,,top]{\color{textcolor}\sffamily\fontsize{10.000000}{12.000000}\selectfont \(\displaystyle 40\)}%
\end{pgfscope}%
\begin{pgfscope}%
\pgfsetbuttcap%
\pgfsetroundjoin%
\definecolor{currentfill}{rgb}{0.000000,0.000000,0.000000}%
\pgfsetfillcolor{currentfill}%
\pgfsetlinewidth{0.803000pt}%
\definecolor{currentstroke}{rgb}{0.000000,0.000000,0.000000}%
\pgfsetstrokecolor{currentstroke}%
\pgfsetdash{}{0pt}%
\pgfsys@defobject{currentmarker}{\pgfqpoint{0.000000in}{-0.048611in}}{\pgfqpoint{0.000000in}{0.000000in}}{%
\pgfpathmoveto{\pgfqpoint{0.000000in}{0.000000in}}%
\pgfpathlineto{\pgfqpoint{0.000000in}{-0.048611in}}%
\pgfusepath{stroke,fill}%
}%
\begin{pgfscope}%
\pgfsys@transformshift{3.530505in}{0.528000in}%
\pgfsys@useobject{currentmarker}{}%
\end{pgfscope}%
\end{pgfscope}%
\begin{pgfscope}%
\definecolor{textcolor}{rgb}{0.000000,0.000000,0.000000}%
\pgfsetstrokecolor{textcolor}%
\pgfsetfillcolor{textcolor}%
\pgftext[x=3.530505in,y=0.430778in,,top]{\color{textcolor}\sffamily\fontsize{10.000000}{12.000000}\selectfont \(\displaystyle 60\)}%
\end{pgfscope}%
\begin{pgfscope}%
\pgfsetbuttcap%
\pgfsetroundjoin%
\definecolor{currentfill}{rgb}{0.000000,0.000000,0.000000}%
\pgfsetfillcolor{currentfill}%
\pgfsetlinewidth{0.803000pt}%
\definecolor{currentstroke}{rgb}{0.000000,0.000000,0.000000}%
\pgfsetstrokecolor{currentstroke}%
\pgfsetdash{}{0pt}%
\pgfsys@defobject{currentmarker}{\pgfqpoint{0.000000in}{-0.048611in}}{\pgfqpoint{0.000000in}{0.000000in}}{%
\pgfpathmoveto{\pgfqpoint{0.000000in}{0.000000in}}%
\pgfpathlineto{\pgfqpoint{0.000000in}{-0.048611in}}%
\pgfusepath{stroke,fill}%
}%
\begin{pgfscope}%
\pgfsys@transformshift{4.532525in}{0.528000in}%
\pgfsys@useobject{currentmarker}{}%
\end{pgfscope}%
\end{pgfscope}%
\begin{pgfscope}%
\definecolor{textcolor}{rgb}{0.000000,0.000000,0.000000}%
\pgfsetstrokecolor{textcolor}%
\pgfsetfillcolor{textcolor}%
\pgftext[x=4.532525in,y=0.430778in,,top]{\color{textcolor}\sffamily\fontsize{10.000000}{12.000000}\selectfont \(\displaystyle 80\)}%
\end{pgfscope}%
\begin{pgfscope}%
\pgfsetbuttcap%
\pgfsetroundjoin%
\definecolor{currentfill}{rgb}{0.000000,0.000000,0.000000}%
\pgfsetfillcolor{currentfill}%
\pgfsetlinewidth{0.803000pt}%
\definecolor{currentstroke}{rgb}{0.000000,0.000000,0.000000}%
\pgfsetstrokecolor{currentstroke}%
\pgfsetdash{}{0pt}%
\pgfsys@defobject{currentmarker}{\pgfqpoint{0.000000in}{-0.048611in}}{\pgfqpoint{0.000000in}{0.000000in}}{%
\pgfpathmoveto{\pgfqpoint{0.000000in}{0.000000in}}%
\pgfpathlineto{\pgfqpoint{0.000000in}{-0.048611in}}%
\pgfusepath{stroke,fill}%
}%
\begin{pgfscope}%
\pgfsys@transformshift{5.534545in}{0.528000in}%
\pgfsys@useobject{currentmarker}{}%
\end{pgfscope}%
\end{pgfscope}%
\begin{pgfscope}%
\definecolor{textcolor}{rgb}{0.000000,0.000000,0.000000}%
\pgfsetstrokecolor{textcolor}%
\pgfsetfillcolor{textcolor}%
\pgftext[x=5.534545in,y=0.430778in,,top]{\color{textcolor}\sffamily\fontsize{10.000000}{12.000000}\selectfont \(\displaystyle 100\)}%
\end{pgfscope}%
\begin{pgfscope}%
\definecolor{textcolor}{rgb}{0.000000,0.000000,0.000000}%
\pgfsetstrokecolor{textcolor}%
\pgfsetfillcolor{textcolor}%
\pgftext[x=3.280000in,y=0.240809in,,top]{\color{textcolor}\sffamily\fontsize{10.000000}{12.000000}\selectfont \(\displaystyle k\)}%
\end{pgfscope}%
\begin{pgfscope}%
\pgfsetbuttcap%
\pgfsetroundjoin%
\definecolor{currentfill}{rgb}{0.000000,0.000000,0.000000}%
\pgfsetfillcolor{currentfill}%
\pgfsetlinewidth{0.803000pt}%
\definecolor{currentstroke}{rgb}{0.000000,0.000000,0.000000}%
\pgfsetstrokecolor{currentstroke}%
\pgfsetdash{}{0pt}%
\pgfsys@defobject{currentmarker}{\pgfqpoint{-0.048611in}{0.000000in}}{\pgfqpoint{0.000000in}{0.000000in}}{%
\pgfpathmoveto{\pgfqpoint{0.000000in}{0.000000in}}%
\pgfpathlineto{\pgfqpoint{-0.048611in}{0.000000in}}%
\pgfusepath{stroke,fill}%
}%
\begin{pgfscope}%
\pgfsys@transformshift{0.800000in}{0.581810in}%
\pgfsys@useobject{currentmarker}{}%
\end{pgfscope}%
\end{pgfscope}%
\begin{pgfscope}%
\definecolor{textcolor}{rgb}{0.000000,0.000000,0.000000}%
\pgfsetstrokecolor{textcolor}%
\pgfsetfillcolor{textcolor}%
\pgftext[x=0.633333in,y=0.529049in,left,base]{\color{textcolor}\sffamily\fontsize{10.000000}{12.000000}\selectfont \(\displaystyle 0\)}%
\end{pgfscope}%
\begin{pgfscope}%
\pgfsetbuttcap%
\pgfsetroundjoin%
\definecolor{currentfill}{rgb}{0.000000,0.000000,0.000000}%
\pgfsetfillcolor{currentfill}%
\pgfsetlinewidth{0.803000pt}%
\definecolor{currentstroke}{rgb}{0.000000,0.000000,0.000000}%
\pgfsetstrokecolor{currentstroke}%
\pgfsetdash{}{0pt}%
\pgfsys@defobject{currentmarker}{\pgfqpoint{-0.048611in}{0.000000in}}{\pgfqpoint{0.000000in}{0.000000in}}{%
\pgfpathmoveto{\pgfqpoint{0.000000in}{0.000000in}}%
\pgfpathlineto{\pgfqpoint{-0.048611in}{0.000000in}}%
\pgfusepath{stroke,fill}%
}%
\begin{pgfscope}%
\pgfsys@transformshift{0.800000in}{1.363933in}%
\pgfsys@useobject{currentmarker}{}%
\end{pgfscope}%
\end{pgfscope}%
\begin{pgfscope}%
\definecolor{textcolor}{rgb}{0.000000,0.000000,0.000000}%
\pgfsetstrokecolor{textcolor}%
\pgfsetfillcolor{textcolor}%
\pgftext[x=0.494444in,y=1.311171in,left,base]{\color{textcolor}\sffamily\fontsize{10.000000}{12.000000}\selectfont \(\displaystyle 500\)}%
\end{pgfscope}%
\begin{pgfscope}%
\pgfsetbuttcap%
\pgfsetroundjoin%
\definecolor{currentfill}{rgb}{0.000000,0.000000,0.000000}%
\pgfsetfillcolor{currentfill}%
\pgfsetlinewidth{0.803000pt}%
\definecolor{currentstroke}{rgb}{0.000000,0.000000,0.000000}%
\pgfsetstrokecolor{currentstroke}%
\pgfsetdash{}{0pt}%
\pgfsys@defobject{currentmarker}{\pgfqpoint{-0.048611in}{0.000000in}}{\pgfqpoint{0.000000in}{0.000000in}}{%
\pgfpathmoveto{\pgfqpoint{0.000000in}{0.000000in}}%
\pgfpathlineto{\pgfqpoint{-0.048611in}{0.000000in}}%
\pgfusepath{stroke,fill}%
}%
\begin{pgfscope}%
\pgfsys@transformshift{0.800000in}{2.146056in}%
\pgfsys@useobject{currentmarker}{}%
\end{pgfscope}%
\end{pgfscope}%
\begin{pgfscope}%
\definecolor{textcolor}{rgb}{0.000000,0.000000,0.000000}%
\pgfsetstrokecolor{textcolor}%
\pgfsetfillcolor{textcolor}%
\pgftext[x=0.424999in,y=2.093294in,left,base]{\color{textcolor}\sffamily\fontsize{10.000000}{12.000000}\selectfont \(\displaystyle 1000\)}%
\end{pgfscope}%
\begin{pgfscope}%
\pgfsetbuttcap%
\pgfsetroundjoin%
\definecolor{currentfill}{rgb}{0.000000,0.000000,0.000000}%
\pgfsetfillcolor{currentfill}%
\pgfsetlinewidth{0.803000pt}%
\definecolor{currentstroke}{rgb}{0.000000,0.000000,0.000000}%
\pgfsetstrokecolor{currentstroke}%
\pgfsetdash{}{0pt}%
\pgfsys@defobject{currentmarker}{\pgfqpoint{-0.048611in}{0.000000in}}{\pgfqpoint{0.000000in}{0.000000in}}{%
\pgfpathmoveto{\pgfqpoint{0.000000in}{0.000000in}}%
\pgfpathlineto{\pgfqpoint{-0.048611in}{0.000000in}}%
\pgfusepath{stroke,fill}%
}%
\begin{pgfscope}%
\pgfsys@transformshift{0.800000in}{2.928179in}%
\pgfsys@useobject{currentmarker}{}%
\end{pgfscope}%
\end{pgfscope}%
\begin{pgfscope}%
\definecolor{textcolor}{rgb}{0.000000,0.000000,0.000000}%
\pgfsetstrokecolor{textcolor}%
\pgfsetfillcolor{textcolor}%
\pgftext[x=0.424999in,y=2.875417in,left,base]{\color{textcolor}\sffamily\fontsize{10.000000}{12.000000}\selectfont \(\displaystyle 1500\)}%
\end{pgfscope}%
\begin{pgfscope}%
\pgfsetbuttcap%
\pgfsetroundjoin%
\definecolor{currentfill}{rgb}{0.000000,0.000000,0.000000}%
\pgfsetfillcolor{currentfill}%
\pgfsetlinewidth{0.803000pt}%
\definecolor{currentstroke}{rgb}{0.000000,0.000000,0.000000}%
\pgfsetstrokecolor{currentstroke}%
\pgfsetdash{}{0pt}%
\pgfsys@defobject{currentmarker}{\pgfqpoint{-0.048611in}{0.000000in}}{\pgfqpoint{0.000000in}{0.000000in}}{%
\pgfpathmoveto{\pgfqpoint{0.000000in}{0.000000in}}%
\pgfpathlineto{\pgfqpoint{-0.048611in}{0.000000in}}%
\pgfusepath{stroke,fill}%
}%
\begin{pgfscope}%
\pgfsys@transformshift{0.800000in}{3.710302in}%
\pgfsys@useobject{currentmarker}{}%
\end{pgfscope}%
\end{pgfscope}%
\begin{pgfscope}%
\definecolor{textcolor}{rgb}{0.000000,0.000000,0.000000}%
\pgfsetstrokecolor{textcolor}%
\pgfsetfillcolor{textcolor}%
\pgftext[x=0.424999in,y=3.657540in,left,base]{\color{textcolor}\sffamily\fontsize{10.000000}{12.000000}\selectfont \(\displaystyle 2000\)}%
\end{pgfscope}%
\begin{pgfscope}%
\definecolor{textcolor}{rgb}{0.000000,0.000000,0.000000}%
\pgfsetstrokecolor{textcolor}%
\pgfsetfillcolor{textcolor}%
\pgftext[x=0.369444in,y=2.376000in,,bottom,rotate=90.000000]{\color{textcolor}\sffamily\fontsize{10.000000}{12.000000}\selectfont Number of GMRES iterations}%
\end{pgfscope}%
\begin{pgfscope}%
\pgfpathrectangle{\pgfqpoint{0.800000in}{0.528000in}}{\pgfqpoint{4.960000in}{3.696000in}}%
\pgfusepath{clip}%
\pgfsetbuttcap%
\pgfsetroundjoin%
\pgfsetlinewidth{1.505625pt}%
\definecolor{currentstroke}{rgb}{0.000000,0.000000,0.000000}%
\pgfsetstrokecolor{currentstroke}%
\pgfsetdash{{5.550000pt}{2.400000pt}}{0.000000pt}%
\pgfpathmoveto{\pgfqpoint{1.025455in}{0.696000in}}%
\pgfpathlineto{\pgfqpoint{1.526465in}{0.986950in}}%
\pgfpathlineto{\pgfqpoint{2.027475in}{1.291978in}}%
\pgfpathlineto{\pgfqpoint{2.528485in}{1.601698in}}%
\pgfpathlineto{\pgfqpoint{3.029495in}{1.942704in}}%
\pgfpathlineto{\pgfqpoint{3.530505in}{2.307173in}}%
\pgfpathlineto{\pgfqpoint{4.031515in}{2.735777in}}%
\pgfpathlineto{\pgfqpoint{4.532525in}{3.092425in}}%
\pgfpathlineto{\pgfqpoint{5.033535in}{3.583598in}}%
\pgfpathlineto{\pgfqpoint{5.534545in}{4.056000in}}%
\pgfusepath{stroke}%
\end{pgfscope}%
\begin{pgfscope}%
\pgfpathrectangle{\pgfqpoint{0.800000in}{0.528000in}}{\pgfqpoint{4.960000in}{3.696000in}}%
\pgfusepath{clip}%
\pgfsetbuttcap%
\pgfsetmiterjoin%
\definecolor{currentfill}{rgb}{0.000000,0.000000,0.000000}%
\pgfsetfillcolor{currentfill}%
\pgfsetlinewidth{1.003750pt}%
\definecolor{currentstroke}{rgb}{0.000000,0.000000,0.000000}%
\pgfsetstrokecolor{currentstroke}%
\pgfsetdash{}{0pt}%
\pgfsys@defobject{currentmarker}{\pgfqpoint{-0.041667in}{-0.041667in}}{\pgfqpoint{0.041667in}{0.041667in}}{%
\pgfpathmoveto{\pgfqpoint{-0.041667in}{-0.041667in}}%
\pgfpathlineto{\pgfqpoint{0.041667in}{-0.041667in}}%
\pgfpathlineto{\pgfqpoint{0.041667in}{0.041667in}}%
\pgfpathlineto{\pgfqpoint{-0.041667in}{0.041667in}}%
\pgfpathclose%
\pgfusepath{stroke,fill}%
}%
\begin{pgfscope}%
\pgfsys@transformshift{1.025455in}{0.696000in}%
\pgfsys@useobject{currentmarker}{}%
\end{pgfscope}%
\begin{pgfscope}%
\pgfsys@transformshift{1.526465in}{0.986950in}%
\pgfsys@useobject{currentmarker}{}%
\end{pgfscope}%
\begin{pgfscope}%
\pgfsys@transformshift{2.027475in}{1.291978in}%
\pgfsys@useobject{currentmarker}{}%
\end{pgfscope}%
\begin{pgfscope}%
\pgfsys@transformshift{2.528485in}{1.601698in}%
\pgfsys@useobject{currentmarker}{}%
\end{pgfscope}%
\begin{pgfscope}%
\pgfsys@transformshift{3.029495in}{1.942704in}%
\pgfsys@useobject{currentmarker}{}%
\end{pgfscope}%
\begin{pgfscope}%
\pgfsys@transformshift{3.530505in}{2.307173in}%
\pgfsys@useobject{currentmarker}{}%
\end{pgfscope}%
\begin{pgfscope}%
\pgfsys@transformshift{4.031515in}{2.735777in}%
\pgfsys@useobject{currentmarker}{}%
\end{pgfscope}%
\begin{pgfscope}%
\pgfsys@transformshift{4.532525in}{3.092425in}%
\pgfsys@useobject{currentmarker}{}%
\end{pgfscope}%
\begin{pgfscope}%
\pgfsys@transformshift{5.033535in}{3.583598in}%
\pgfsys@useobject{currentmarker}{}%
\end{pgfscope}%
\begin{pgfscope}%
\pgfsys@transformshift{5.534545in}{4.056000in}%
\pgfsys@useobject{currentmarker}{}%
\end{pgfscope}%
\end{pgfscope}%
\begin{pgfscope}%
\pgfsetrectcap%
\pgfsetmiterjoin%
\pgfsetlinewidth{0.803000pt}%
\definecolor{currentstroke}{rgb}{0.000000,0.000000,0.000000}%
\pgfsetstrokecolor{currentstroke}%
\pgfsetdash{}{0pt}%
\pgfpathmoveto{\pgfqpoint{0.800000in}{0.528000in}}%
\pgfpathlineto{\pgfqpoint{0.800000in}{4.224000in}}%
\pgfusepath{stroke}%
\end{pgfscope}%
\begin{pgfscope}%
\pgfsetrectcap%
\pgfsetmiterjoin%
\pgfsetlinewidth{0.000000pt}%
\definecolor{currentstroke}{rgb}{0.000000,0.000000,0.000000}%
\pgfsetstrokecolor{currentstroke}%
\pgfsetstrokeopacity{0.000000}%
\pgfsetdash{}{0pt}%
\pgfpathmoveto{\pgfqpoint{5.760000in}{0.528000in}}%
\pgfpathlineto{\pgfqpoint{5.760000in}{4.224000in}}%
\pgfusepath{}%
\end{pgfscope}%
\begin{pgfscope}%
\pgfsetrectcap%
\pgfsetmiterjoin%
\pgfsetlinewidth{0.803000pt}%
\definecolor{currentstroke}{rgb}{0.000000,0.000000,0.000000}%
\pgfsetstrokecolor{currentstroke}%
\pgfsetdash{}{0pt}%
\pgfpathmoveto{\pgfqpoint{0.800000in}{0.528000in}}%
\pgfpathlineto{\pgfqpoint{5.760000in}{0.528000in}}%
\pgfusepath{stroke}%
\end{pgfscope}%
\begin{pgfscope}%
\pgfsetrectcap%
\pgfsetmiterjoin%
\pgfsetlinewidth{0.000000pt}%
\definecolor{currentstroke}{rgb}{0.000000,0.000000,0.000000}%
\pgfsetstrokecolor{currentstroke}%
\pgfsetstrokeopacity{0.000000}%
\pgfsetdash{}{0pt}%
\pgfpathmoveto{\pgfqpoint{0.800000in}{4.224000in}}%
\pgfpathlineto{\pgfqpoint{5.760000in}{4.224000in}}%
\pgfusepath{}%
\end{pgfscope}%
\end{pgfpicture}%
\makeatother%
\endgroup%

      \caption[Number of unpreconditioned GMRES iterations for the homogeneous Helmholtz equation]{Number of unpreconditioned GMRES iterations for the homogeneous Helmholtz equation on the unit square, with zero impedance boundary condition and $f=1$.\label{fig:GMRES}}
      \end{figure}
    
    For \cref{it:large}, the matrices $\Amat$ are large because the number of degrees of freedom must increase as $k$ increases. One can see this from interpolating the solution of \eqref{eq:introdet}; solutions $u$ of \eqref{eq:introdet} oscillate on a scale $1/k$, and therefore the number of degrees of freedom (interpolation points) must increase like $k^d$ (where $d$ is the spatial dimension) in order to keep the interpolation error for $u$ bounded. This need for increasing degrees of freedom with $k$ is illustrated in \cref{fig:introinterp}, where we see the interpolation error grows if the number of degrees of freedom is not increased with $k$. In practice one typically chooses to use 6--10 discretisation points in each dimension for each wavelength in the domain---this choice  empirically keeps the interpolation error at a reasonable size, but means the linear systems \eqref{eq:intromat} will have $k^d$ unknowns.

\begin{figure}[h]
    \centering
  \begin{subfigure}{0.45\textwidth}
%% Creator: Matplotlib, PGF backend
%%
%% To include the figure in your LaTeX document, write
%%   \input{<filename>.pgf}
%%
%% Make sure the required packages are loaded in your preamble
%%   \usepackage{pgf}
%%
%% Figures using additional raster images can only be included by \input if
%% they are in the same directory as the main LaTeX file. For loading figures
%% from other directories you can use the `import` package
%%   \usepackage{import}
%% and then include the figures with
%%   \import{<path to file>}{<filename>.pgf}
%%
%% Matplotlib used the following preamble
%%   \usepackage{fontspec}
%%   \setmainfont{DejaVuSerif.ttf}[Path=/home/owen/progs/firedrake-complex/firedrake/lib/python3.5/site-packages/matplotlib/mpl-data/fonts/ttf/]
%%   \setsansfont{DejaVuSans.ttf}[Path=/home/owen/progs/firedrake-complex/firedrake/lib/python3.5/site-packages/matplotlib/mpl-data/fonts/ttf/]
%%   \setmonofont{DejaVuSansMono.ttf}[Path=/home/owen/progs/firedrake-complex/firedrake/lib/python3.5/site-packages/matplotlib/mpl-data/fonts/ttf/]
%%
\begingroup%
\makeatletter%
\begin{pgfpicture}%
\pgfpathrectangle{\pgfpointorigin}{\pgfqpoint{3.000000in}{3.000000in}}%
\pgfusepath{use as bounding box, clip}%
\begin{pgfscope}%
\pgfsetbuttcap%
\pgfsetmiterjoin%
\definecolor{currentfill}{rgb}{1.000000,1.000000,1.000000}%
\pgfsetfillcolor{currentfill}%
\pgfsetlinewidth{0.000000pt}%
\definecolor{currentstroke}{rgb}{1.000000,1.000000,1.000000}%
\pgfsetstrokecolor{currentstroke}%
\pgfsetdash{}{0pt}%
\pgfpathmoveto{\pgfqpoint{0.000000in}{0.000000in}}%
\pgfpathlineto{\pgfqpoint{3.000000in}{0.000000in}}%
\pgfpathlineto{\pgfqpoint{3.000000in}{3.000000in}}%
\pgfpathlineto{\pgfqpoint{0.000000in}{3.000000in}}%
\pgfpathclose%
\pgfusepath{fill}%
\end{pgfscope}%
\begin{pgfscope}%
\pgfsetbuttcap%
\pgfsetmiterjoin%
\definecolor{currentfill}{rgb}{1.000000,1.000000,1.000000}%
\pgfsetfillcolor{currentfill}%
\pgfsetlinewidth{0.000000pt}%
\definecolor{currentstroke}{rgb}{0.000000,0.000000,0.000000}%
\pgfsetstrokecolor{currentstroke}%
\pgfsetstrokeopacity{0.000000}%
\pgfsetdash{}{0pt}%
\pgfpathmoveto{\pgfqpoint{0.375000in}{0.330000in}}%
\pgfpathlineto{\pgfqpoint{2.700000in}{0.330000in}}%
\pgfpathlineto{\pgfqpoint{2.700000in}{2.640000in}}%
\pgfpathlineto{\pgfqpoint{0.375000in}{2.640000in}}%
\pgfpathclose%
\pgfusepath{fill}%
\end{pgfscope}%
\begin{pgfscope}%
\pgfsetbuttcap%
\pgfsetroundjoin%
\definecolor{currentfill}{rgb}{0.000000,0.000000,0.000000}%
\pgfsetfillcolor{currentfill}%
\pgfsetlinewidth{0.803000pt}%
\definecolor{currentstroke}{rgb}{0.000000,0.000000,0.000000}%
\pgfsetstrokecolor{currentstroke}%
\pgfsetdash{}{0pt}%
\pgfsys@defobject{currentmarker}{\pgfqpoint{0.000000in}{-0.048611in}}{\pgfqpoint{0.000000in}{0.000000in}}{%
\pgfpathmoveto{\pgfqpoint{0.000000in}{0.000000in}}%
\pgfpathlineto{\pgfqpoint{0.000000in}{-0.048611in}}%
\pgfusepath{stroke,fill}%
}%
\begin{pgfscope}%
\pgfsys@transformshift{0.480682in}{0.330000in}%
\pgfsys@useobject{currentmarker}{}%
\end{pgfscope}%
\end{pgfscope}%
\begin{pgfscope}%
\definecolor{textcolor}{rgb}{0.000000,0.000000,0.000000}%
\pgfsetstrokecolor{textcolor}%
\pgfsetfillcolor{textcolor}%
\pgftext[x=0.480682in,y=0.232778in,,top]{\color{textcolor}\sffamily\fontsize{10.000000}{12.000000}\selectfont 0.0}%
\end{pgfscope}%
\begin{pgfscope}%
\pgfsetbuttcap%
\pgfsetroundjoin%
\definecolor{currentfill}{rgb}{0.000000,0.000000,0.000000}%
\pgfsetfillcolor{currentfill}%
\pgfsetlinewidth{0.803000pt}%
\definecolor{currentstroke}{rgb}{0.000000,0.000000,0.000000}%
\pgfsetstrokecolor{currentstroke}%
\pgfsetdash{}{0pt}%
\pgfsys@defobject{currentmarker}{\pgfqpoint{0.000000in}{-0.048611in}}{\pgfqpoint{0.000000in}{0.000000in}}{%
\pgfpathmoveto{\pgfqpoint{0.000000in}{0.000000in}}%
\pgfpathlineto{\pgfqpoint{0.000000in}{-0.048611in}}%
\pgfusepath{stroke,fill}%
}%
\begin{pgfscope}%
\pgfsys@transformshift{1.537500in}{0.330000in}%
\pgfsys@useobject{currentmarker}{}%
\end{pgfscope}%
\end{pgfscope}%
\begin{pgfscope}%
\definecolor{textcolor}{rgb}{0.000000,0.000000,0.000000}%
\pgfsetstrokecolor{textcolor}%
\pgfsetfillcolor{textcolor}%
\pgftext[x=1.537500in,y=0.232778in,,top]{\color{textcolor}\sffamily\fontsize{10.000000}{12.000000}\selectfont 0.5}%
\end{pgfscope}%
\begin{pgfscope}%
\pgfsetbuttcap%
\pgfsetroundjoin%
\definecolor{currentfill}{rgb}{0.000000,0.000000,0.000000}%
\pgfsetfillcolor{currentfill}%
\pgfsetlinewidth{0.803000pt}%
\definecolor{currentstroke}{rgb}{0.000000,0.000000,0.000000}%
\pgfsetstrokecolor{currentstroke}%
\pgfsetdash{}{0pt}%
\pgfsys@defobject{currentmarker}{\pgfqpoint{0.000000in}{-0.048611in}}{\pgfqpoint{0.000000in}{0.000000in}}{%
\pgfpathmoveto{\pgfqpoint{0.000000in}{0.000000in}}%
\pgfpathlineto{\pgfqpoint{0.000000in}{-0.048611in}}%
\pgfusepath{stroke,fill}%
}%
\begin{pgfscope}%
\pgfsys@transformshift{2.594318in}{0.330000in}%
\pgfsys@useobject{currentmarker}{}%
\end{pgfscope}%
\end{pgfscope}%
\begin{pgfscope}%
\definecolor{textcolor}{rgb}{0.000000,0.000000,0.000000}%
\pgfsetstrokecolor{textcolor}%
\pgfsetfillcolor{textcolor}%
\pgftext[x=2.594318in,y=0.232778in,,top]{\color{textcolor}\sffamily\fontsize{10.000000}{12.000000}\selectfont 1.0}%
\end{pgfscope}%
\begin{pgfscope}%
\pgfsetbuttcap%
\pgfsetroundjoin%
\definecolor{currentfill}{rgb}{0.000000,0.000000,0.000000}%
\pgfsetfillcolor{currentfill}%
\pgfsetlinewidth{0.803000pt}%
\definecolor{currentstroke}{rgb}{0.000000,0.000000,0.000000}%
\pgfsetstrokecolor{currentstroke}%
\pgfsetdash{}{0pt}%
\pgfsys@defobject{currentmarker}{\pgfqpoint{-0.048611in}{0.000000in}}{\pgfqpoint{0.000000in}{0.000000in}}{%
\pgfpathmoveto{\pgfqpoint{0.000000in}{0.000000in}}%
\pgfpathlineto{\pgfqpoint{-0.048611in}{0.000000in}}%
\pgfusepath{stroke,fill}%
}%
\begin{pgfscope}%
\pgfsys@transformshift{0.375000in}{0.434675in}%
\pgfsys@useobject{currentmarker}{}%
\end{pgfscope}%
\end{pgfscope}%
\begin{pgfscope}%
\definecolor{textcolor}{rgb}{0.000000,0.000000,0.000000}%
\pgfsetstrokecolor{textcolor}%
\pgfsetfillcolor{textcolor}%
\pgftext[x=0.073039in,y=0.381913in,left,base]{\color{textcolor}\sffamily\fontsize{10.000000}{12.000000}\selectfont −1}%
\end{pgfscope}%
\begin{pgfscope}%
\pgfsetbuttcap%
\pgfsetroundjoin%
\definecolor{currentfill}{rgb}{0.000000,0.000000,0.000000}%
\pgfsetfillcolor{currentfill}%
\pgfsetlinewidth{0.803000pt}%
\definecolor{currentstroke}{rgb}{0.000000,0.000000,0.000000}%
\pgfsetstrokecolor{currentstroke}%
\pgfsetdash{}{0pt}%
\pgfsys@defobject{currentmarker}{\pgfqpoint{-0.048611in}{0.000000in}}{\pgfqpoint{0.000000in}{0.000000in}}{%
\pgfpathmoveto{\pgfqpoint{0.000000in}{0.000000in}}%
\pgfpathlineto{\pgfqpoint{-0.048611in}{0.000000in}}%
\pgfusepath{stroke,fill}%
}%
\begin{pgfscope}%
\pgfsys@transformshift{0.375000in}{1.484855in}%
\pgfsys@useobject{currentmarker}{}%
\end{pgfscope}%
\end{pgfscope}%
\begin{pgfscope}%
\definecolor{textcolor}{rgb}{0.000000,0.000000,0.000000}%
\pgfsetstrokecolor{textcolor}%
\pgfsetfillcolor{textcolor}%
\pgftext[x=0.189413in,y=1.432094in,left,base]{\color{textcolor}\sffamily\fontsize{10.000000}{12.000000}\selectfont 0}%
\end{pgfscope}%
\begin{pgfscope}%
\pgfsetbuttcap%
\pgfsetroundjoin%
\definecolor{currentfill}{rgb}{0.000000,0.000000,0.000000}%
\pgfsetfillcolor{currentfill}%
\pgfsetlinewidth{0.803000pt}%
\definecolor{currentstroke}{rgb}{0.000000,0.000000,0.000000}%
\pgfsetstrokecolor{currentstroke}%
\pgfsetdash{}{0pt}%
\pgfsys@defobject{currentmarker}{\pgfqpoint{-0.048611in}{0.000000in}}{\pgfqpoint{0.000000in}{0.000000in}}{%
\pgfpathmoveto{\pgfqpoint{0.000000in}{0.000000in}}%
\pgfpathlineto{\pgfqpoint{-0.048611in}{0.000000in}}%
\pgfusepath{stroke,fill}%
}%
\begin{pgfscope}%
\pgfsys@transformshift{0.375000in}{2.535036in}%
\pgfsys@useobject{currentmarker}{}%
\end{pgfscope}%
\end{pgfscope}%
\begin{pgfscope}%
\definecolor{textcolor}{rgb}{0.000000,0.000000,0.000000}%
\pgfsetstrokecolor{textcolor}%
\pgfsetfillcolor{textcolor}%
\pgftext[x=0.189413in,y=2.482275in,left,base]{\color{textcolor}\sffamily\fontsize{10.000000}{12.000000}\selectfont 1}%
\end{pgfscope}%
\begin{pgfscope}%
\pgfpathrectangle{\pgfqpoint{0.375000in}{0.330000in}}{\pgfqpoint{2.325000in}{2.310000in}}%
\pgfusepath{clip}%
\pgfsetrectcap%
\pgfsetroundjoin%
\pgfsetlinewidth{1.505625pt}%
\definecolor{currentstroke}{rgb}{0.700000,0.700000,0.700000}%
\pgfsetstrokecolor{currentstroke}%
\pgfsetdash{}{0pt}%
\pgfpathmoveto{\pgfqpoint{0.480682in}{1.484855in}}%
\pgfpathlineto{\pgfqpoint{0.546733in}{1.807721in}}%
\pgfpathlineto{\pgfqpoint{0.586364in}{1.988339in}}%
\pgfpathlineto{\pgfqpoint{0.612784in}{2.099313in}}%
\pgfpathlineto{\pgfqpoint{0.639205in}{2.200699in}}%
\pgfpathlineto{\pgfqpoint{0.665625in}{2.290915in}}%
\pgfpathlineto{\pgfqpoint{0.692045in}{2.368552in}}%
\pgfpathlineto{\pgfqpoint{0.705256in}{2.402267in}}%
\pgfpathlineto{\pgfqpoint{0.718466in}{2.432399in}}%
\pgfpathlineto{\pgfqpoint{0.731676in}{2.458832in}}%
\pgfpathlineto{\pgfqpoint{0.744886in}{2.481461in}}%
\pgfpathlineto{\pgfqpoint{0.758097in}{2.500198in}}%
\pgfpathlineto{\pgfqpoint{0.771307in}{2.514970in}}%
\pgfpathlineto{\pgfqpoint{0.784517in}{2.525720in}}%
\pgfpathlineto{\pgfqpoint{0.797727in}{2.532405in}}%
\pgfpathlineto{\pgfqpoint{0.810937in}{2.535000in}}%
\pgfpathlineto{\pgfqpoint{0.824148in}{2.533494in}}%
\pgfpathlineto{\pgfqpoint{0.837358in}{2.527893in}}%
\pgfpathlineto{\pgfqpoint{0.850568in}{2.518218in}}%
\pgfpathlineto{\pgfqpoint{0.863778in}{2.504509in}}%
\pgfpathlineto{\pgfqpoint{0.876989in}{2.486818in}}%
\pgfpathlineto{\pgfqpoint{0.890199in}{2.465214in}}%
\pgfpathlineto{\pgfqpoint{0.903409in}{2.439782in}}%
\pgfpathlineto{\pgfqpoint{0.916619in}{2.410621in}}%
\pgfpathlineto{\pgfqpoint{0.929830in}{2.377845in}}%
\pgfpathlineto{\pgfqpoint{0.943040in}{2.341582in}}%
\pgfpathlineto{\pgfqpoint{0.969460in}{2.259173in}}%
\pgfpathlineto{\pgfqpoint{0.995881in}{2.164682in}}%
\pgfpathlineto{\pgfqpoint{1.022301in}{2.059582in}}%
\pgfpathlineto{\pgfqpoint{1.048722in}{1.945514in}}%
\pgfpathlineto{\pgfqpoint{1.088352in}{1.761521in}}%
\pgfpathlineto{\pgfqpoint{1.141193in}{1.502280in}}%
\pgfpathlineto{\pgfqpoint{1.207244in}{1.178614in}}%
\pgfpathlineto{\pgfqpoint{1.246875in}{0.996733in}}%
\pgfpathlineto{\pgfqpoint{1.273295in}{0.884613in}}%
\pgfpathlineto{\pgfqpoint{1.299716in}{0.781859in}}%
\pgfpathlineto{\pgfqpoint{1.326136in}{0.690076in}}%
\pgfpathlineto{\pgfqpoint{1.352557in}{0.610695in}}%
\pgfpathlineto{\pgfqpoint{1.365767in}{0.576051in}}%
\pgfpathlineto{\pgfqpoint{1.378977in}{0.544955in}}%
\pgfpathlineto{\pgfqpoint{1.392188in}{0.517530in}}%
\pgfpathlineto{\pgfqpoint{1.405398in}{0.493882in}}%
\pgfpathlineto{\pgfqpoint{1.418608in}{0.474103in}}%
\pgfpathlineto{\pgfqpoint{1.431818in}{0.458272in}}%
\pgfpathlineto{\pgfqpoint{1.445028in}{0.446450in}}%
\pgfpathlineto{\pgfqpoint{1.458239in}{0.438682in}}%
\pgfpathlineto{\pgfqpoint{1.471449in}{0.435000in}}%
\pgfpathlineto{\pgfqpoint{1.484659in}{0.435417in}}%
\pgfpathlineto{\pgfqpoint{1.497869in}{0.439933in}}%
\pgfpathlineto{\pgfqpoint{1.511080in}{0.448529in}}%
\pgfpathlineto{\pgfqpoint{1.524290in}{0.461172in}}%
\pgfpathlineto{\pgfqpoint{1.537500in}{0.477812in}}%
\pgfpathlineto{\pgfqpoint{1.550710in}{0.498384in}}%
\pgfpathlineto{\pgfqpoint{1.563920in}{0.522809in}}%
\pgfpathlineto{\pgfqpoint{1.577131in}{0.550991in}}%
\pgfpathlineto{\pgfqpoint{1.590341in}{0.582819in}}%
\pgfpathlineto{\pgfqpoint{1.603551in}{0.618170in}}%
\pgfpathlineto{\pgfqpoint{1.629972in}{0.698873in}}%
\pgfpathlineto{\pgfqpoint{1.656392in}{0.791841in}}%
\pgfpathlineto{\pgfqpoint{1.682813in}{0.895624in}}%
\pgfpathlineto{\pgfqpoint{1.709233in}{1.008601in}}%
\pgfpathlineto{\pgfqpoint{1.748864in}{1.191419in}}%
\pgfpathlineto{\pgfqpoint{1.801705in}{1.450011in}}%
\pgfpathlineto{\pgfqpoint{1.867756in}{1.774387in}}%
\pgfpathlineto{\pgfqpoint{1.907386in}{1.957483in}}%
\pgfpathlineto{\pgfqpoint{1.933807in}{2.070718in}}%
\pgfpathlineto{\pgfqpoint{1.960227in}{2.174810in}}%
\pgfpathlineto{\pgfqpoint{1.986648in}{2.268136in}}%
\pgfpathlineto{\pgfqpoint{2.013068in}{2.349239in}}%
\pgfpathlineto{\pgfqpoint{2.026278in}{2.384804in}}%
\pgfpathlineto{\pgfqpoint{2.039489in}{2.416854in}}%
\pgfpathlineto{\pgfqpoint{2.052699in}{2.445265in}}%
\pgfpathlineto{\pgfqpoint{2.065909in}{2.469925in}}%
\pgfpathlineto{\pgfqpoint{2.079119in}{2.490739in}}%
\pgfpathlineto{\pgfqpoint{2.092330in}{2.507624in}}%
\pgfpathlineto{\pgfqpoint{2.105540in}{2.520516in}}%
\pgfpathlineto{\pgfqpoint{2.118750in}{2.529364in}}%
\pgfpathlineto{\pgfqpoint{2.131960in}{2.534133in}}%
\pgfpathlineto{\pgfqpoint{2.145170in}{2.534804in}}%
\pgfpathlineto{\pgfqpoint{2.158381in}{2.531376in}}%
\pgfpathlineto{\pgfqpoint{2.171591in}{2.523860in}}%
\pgfpathlineto{\pgfqpoint{2.184801in}{2.512288in}}%
\pgfpathlineto{\pgfqpoint{2.198011in}{2.496703in}}%
\pgfpathlineto{\pgfqpoint{2.211222in}{2.477167in}}%
\pgfpathlineto{\pgfqpoint{2.224432in}{2.453757in}}%
\pgfpathlineto{\pgfqpoint{2.237642in}{2.426562in}}%
\pgfpathlineto{\pgfqpoint{2.250852in}{2.395691in}}%
\pgfpathlineto{\pgfqpoint{2.264062in}{2.361262in}}%
\pgfpathlineto{\pgfqpoint{2.290483in}{2.282286in}}%
\pgfpathlineto{\pgfqpoint{2.316903in}{2.190866in}}%
\pgfpathlineto{\pgfqpoint{2.343324in}{2.088429in}}%
\pgfpathlineto{\pgfqpoint{2.369744in}{1.976574in}}%
\pgfpathlineto{\pgfqpoint{2.409375in}{1.794982in}}%
\pgfpathlineto{\pgfqpoint{2.462216in}{1.537110in}}%
\pgfpathlineto{\pgfqpoint{2.528267in}{1.212113in}}%
\pgfpathlineto{\pgfqpoint{2.567898in}{1.027853in}}%
\pgfpathlineto{\pgfqpoint{2.594318in}{0.913535in}}%
\pgfpathlineto{\pgfqpoint{2.594318in}{0.913535in}}%
\pgfusepath{stroke}%
\end{pgfscope}%
\begin{pgfscope}%
\pgfpathrectangle{\pgfqpoint{0.375000in}{0.330000in}}{\pgfqpoint{2.325000in}{2.310000in}}%
\pgfusepath{clip}%
\pgfsetbuttcap%
\pgfsetroundjoin%
\pgfsetlinewidth{1.505625pt}%
\definecolor{currentstroke}{rgb}{0.000000,0.000000,0.000000}%
\pgfsetstrokecolor{currentstroke}%
\pgfsetdash{{5.550000pt}{2.400000pt}}{0.000000pt}%
\pgfpathmoveto{\pgfqpoint{0.480682in}{1.484855in}}%
\pgfpathlineto{\pgfqpoint{0.612784in}{2.099313in}}%
\pgfpathlineto{\pgfqpoint{0.744886in}{2.481461in}}%
\pgfpathlineto{\pgfqpoint{0.876989in}{2.486818in}}%
\pgfpathlineto{\pgfqpoint{1.009091in}{2.113359in}}%
\pgfpathlineto{\pgfqpoint{1.141193in}{1.502280in}}%
\pgfpathlineto{\pgfqpoint{1.273295in}{0.884613in}}%
\pgfpathlineto{\pgfqpoint{1.405398in}{0.493882in}}%
\pgfpathlineto{\pgfqpoint{1.537500in}{0.477812in}}%
\pgfpathlineto{\pgfqpoint{1.669602in}{0.842479in}}%
\pgfpathlineto{\pgfqpoint{1.801705in}{1.450011in}}%
\pgfpathlineto{\pgfqpoint{1.933807in}{2.070718in}}%
\pgfpathlineto{\pgfqpoint{2.065909in}{2.469925in}}%
\pgfpathlineto{\pgfqpoint{2.198011in}{2.496703in}}%
\pgfpathlineto{\pgfqpoint{2.330114in}{2.140928in}}%
\pgfpathlineto{\pgfqpoint{2.462216in}{1.537110in}}%
\pgfpathlineto{\pgfqpoint{2.594318in}{0.913535in}}%
\pgfusepath{stroke}%
\end{pgfscope}%
\begin{pgfscope}%
\pgfsetrectcap%
\pgfsetmiterjoin%
\pgfsetlinewidth{0.803000pt}%
\definecolor{currentstroke}{rgb}{0.000000,0.000000,0.000000}%
\pgfsetstrokecolor{currentstroke}%
\pgfsetdash{}{0pt}%
\pgfpathmoveto{\pgfqpoint{0.375000in}{0.330000in}}%
\pgfpathlineto{\pgfqpoint{0.375000in}{2.640000in}}%
\pgfusepath{stroke}%
\end{pgfscope}%
\begin{pgfscope}%
\pgfsetrectcap%
\pgfsetmiterjoin%
\pgfsetlinewidth{0.803000pt}%
\definecolor{currentstroke}{rgb}{0.000000,0.000000,0.000000}%
\pgfsetstrokecolor{currentstroke}%
\pgfsetdash{}{0pt}%
\pgfpathmoveto{\pgfqpoint{2.700000in}{0.330000in}}%
\pgfpathlineto{\pgfqpoint{2.700000in}{2.640000in}}%
\pgfusepath{stroke}%
\end{pgfscope}%
\begin{pgfscope}%
\pgfsetrectcap%
\pgfsetmiterjoin%
\pgfsetlinewidth{0.803000pt}%
\definecolor{currentstroke}{rgb}{0.000000,0.000000,0.000000}%
\pgfsetstrokecolor{currentstroke}%
\pgfsetdash{}{0pt}%
\pgfpathmoveto{\pgfqpoint{0.375000in}{0.330000in}}%
\pgfpathlineto{\pgfqpoint{2.700000in}{0.330000in}}%
\pgfusepath{stroke}%
\end{pgfscope}%
\begin{pgfscope}%
\pgfsetrectcap%
\pgfsetmiterjoin%
\pgfsetlinewidth{0.803000pt}%
\definecolor{currentstroke}{rgb}{0.000000,0.000000,0.000000}%
\pgfsetstrokecolor{currentstroke}%
\pgfsetdash{}{0pt}%
\pgfpathmoveto{\pgfqpoint{0.375000in}{2.640000in}}%
\pgfpathlineto{\pgfqpoint{2.700000in}{2.640000in}}%
\pgfusepath{stroke}%
\end{pgfscope}%
\end{pgfpicture}%
\makeatother%
\endgroup%

  \end{subfigure}
    \begin{subfigure}{0.45\textwidth}
%% Creator: Matplotlib, PGF backend
%%
%% To include the figure in your LaTeX document, write
%%   \input{<filename>.pgf}
%%
%% Make sure the required packages are loaded in your preamble
%%   \usepackage{pgf}
%%
%% Figures using additional raster images can only be included by \input if
%% they are in the same directory as the main LaTeX file. For loading figures
%% from other directories you can use the `import` package
%%   \usepackage{import}
%% and then include the figures with
%%   \import{<path to file>}{<filename>.pgf}
%%
%% Matplotlib used the following preamble
%%   \usepackage{fontspec}
%%   \setmainfont{DejaVuSerif.ttf}[Path=/home/owen/progs/firedrake-complex/firedrake/lib/python3.5/site-packages/matplotlib/mpl-data/fonts/ttf/]
%%   \setsansfont{DejaVuSans.ttf}[Path=/home/owen/progs/firedrake-complex/firedrake/lib/python3.5/site-packages/matplotlib/mpl-data/fonts/ttf/]
%%   \setmonofont{DejaVuSansMono.ttf}[Path=/home/owen/progs/firedrake-complex/firedrake/lib/python3.5/site-packages/matplotlib/mpl-data/fonts/ttf/]
%%
\begingroup%
\makeatletter%
\begin{pgfpicture}%
\pgfpathrectangle{\pgfpointorigin}{\pgfqpoint{3.000000in}{3.000000in}}%
\pgfusepath{use as bounding box, clip}%
\begin{pgfscope}%
\pgfsetbuttcap%
\pgfsetmiterjoin%
\definecolor{currentfill}{rgb}{1.000000,1.000000,1.000000}%
\pgfsetfillcolor{currentfill}%
\pgfsetlinewidth{0.000000pt}%
\definecolor{currentstroke}{rgb}{1.000000,1.000000,1.000000}%
\pgfsetstrokecolor{currentstroke}%
\pgfsetdash{}{0pt}%
\pgfpathmoveto{\pgfqpoint{0.000000in}{0.000000in}}%
\pgfpathlineto{\pgfqpoint{3.000000in}{0.000000in}}%
\pgfpathlineto{\pgfqpoint{3.000000in}{3.000000in}}%
\pgfpathlineto{\pgfqpoint{0.000000in}{3.000000in}}%
\pgfpathclose%
\pgfusepath{fill}%
\end{pgfscope}%
\begin{pgfscope}%
\pgfsetbuttcap%
\pgfsetmiterjoin%
\definecolor{currentfill}{rgb}{1.000000,1.000000,1.000000}%
\pgfsetfillcolor{currentfill}%
\pgfsetlinewidth{0.000000pt}%
\definecolor{currentstroke}{rgb}{0.000000,0.000000,0.000000}%
\pgfsetstrokecolor{currentstroke}%
\pgfsetstrokeopacity{0.000000}%
\pgfsetdash{}{0pt}%
\pgfpathmoveto{\pgfqpoint{0.375000in}{0.330000in}}%
\pgfpathlineto{\pgfqpoint{2.700000in}{0.330000in}}%
\pgfpathlineto{\pgfqpoint{2.700000in}{2.640000in}}%
\pgfpathlineto{\pgfqpoint{0.375000in}{2.640000in}}%
\pgfpathclose%
\pgfusepath{fill}%
\end{pgfscope}%
\begin{pgfscope}%
\pgfsetbuttcap%
\pgfsetroundjoin%
\definecolor{currentfill}{rgb}{0.000000,0.000000,0.000000}%
\pgfsetfillcolor{currentfill}%
\pgfsetlinewidth{0.803000pt}%
\definecolor{currentstroke}{rgb}{0.000000,0.000000,0.000000}%
\pgfsetstrokecolor{currentstroke}%
\pgfsetdash{}{0pt}%
\pgfsys@defobject{currentmarker}{\pgfqpoint{0.000000in}{-0.048611in}}{\pgfqpoint{0.000000in}{0.000000in}}{%
\pgfpathmoveto{\pgfqpoint{0.000000in}{0.000000in}}%
\pgfpathlineto{\pgfqpoint{0.000000in}{-0.048611in}}%
\pgfusepath{stroke,fill}%
}%
\begin{pgfscope}%
\pgfsys@transformshift{0.480682in}{0.330000in}%
\pgfsys@useobject{currentmarker}{}%
\end{pgfscope}%
\end{pgfscope}%
\begin{pgfscope}%
\definecolor{textcolor}{rgb}{0.000000,0.000000,0.000000}%
\pgfsetstrokecolor{textcolor}%
\pgfsetfillcolor{textcolor}%
\pgftext[x=0.480682in,y=0.232778in,,top]{\color{textcolor}\sffamily\fontsize{10.000000}{12.000000}\selectfont 0.0}%
\end{pgfscope}%
\begin{pgfscope}%
\pgfsetbuttcap%
\pgfsetroundjoin%
\definecolor{currentfill}{rgb}{0.000000,0.000000,0.000000}%
\pgfsetfillcolor{currentfill}%
\pgfsetlinewidth{0.803000pt}%
\definecolor{currentstroke}{rgb}{0.000000,0.000000,0.000000}%
\pgfsetstrokecolor{currentstroke}%
\pgfsetdash{}{0pt}%
\pgfsys@defobject{currentmarker}{\pgfqpoint{0.000000in}{-0.048611in}}{\pgfqpoint{0.000000in}{0.000000in}}{%
\pgfpathmoveto{\pgfqpoint{0.000000in}{0.000000in}}%
\pgfpathlineto{\pgfqpoint{0.000000in}{-0.048611in}}%
\pgfusepath{stroke,fill}%
}%
\begin{pgfscope}%
\pgfsys@transformshift{1.537500in}{0.330000in}%
\pgfsys@useobject{currentmarker}{}%
\end{pgfscope}%
\end{pgfscope}%
\begin{pgfscope}%
\definecolor{textcolor}{rgb}{0.000000,0.000000,0.000000}%
\pgfsetstrokecolor{textcolor}%
\pgfsetfillcolor{textcolor}%
\pgftext[x=1.537500in,y=0.232778in,,top]{\color{textcolor}\sffamily\fontsize{10.000000}{12.000000}\selectfont 0.5}%
\end{pgfscope}%
\begin{pgfscope}%
\pgfsetbuttcap%
\pgfsetroundjoin%
\definecolor{currentfill}{rgb}{0.000000,0.000000,0.000000}%
\pgfsetfillcolor{currentfill}%
\pgfsetlinewidth{0.803000pt}%
\definecolor{currentstroke}{rgb}{0.000000,0.000000,0.000000}%
\pgfsetstrokecolor{currentstroke}%
\pgfsetdash{}{0pt}%
\pgfsys@defobject{currentmarker}{\pgfqpoint{0.000000in}{-0.048611in}}{\pgfqpoint{0.000000in}{0.000000in}}{%
\pgfpathmoveto{\pgfqpoint{0.000000in}{0.000000in}}%
\pgfpathlineto{\pgfqpoint{0.000000in}{-0.048611in}}%
\pgfusepath{stroke,fill}%
}%
\begin{pgfscope}%
\pgfsys@transformshift{2.594318in}{0.330000in}%
\pgfsys@useobject{currentmarker}{}%
\end{pgfscope}%
\end{pgfscope}%
\begin{pgfscope}%
\definecolor{textcolor}{rgb}{0.000000,0.000000,0.000000}%
\pgfsetstrokecolor{textcolor}%
\pgfsetfillcolor{textcolor}%
\pgftext[x=2.594318in,y=0.232778in,,top]{\color{textcolor}\sffamily\fontsize{10.000000}{12.000000}\selectfont 1.0}%
\end{pgfscope}%
\begin{pgfscope}%
\pgfsetbuttcap%
\pgfsetroundjoin%
\definecolor{currentfill}{rgb}{0.000000,0.000000,0.000000}%
\pgfsetfillcolor{currentfill}%
\pgfsetlinewidth{0.803000pt}%
\definecolor{currentstroke}{rgb}{0.000000,0.000000,0.000000}%
\pgfsetstrokecolor{currentstroke}%
\pgfsetdash{}{0pt}%
\pgfsys@defobject{currentmarker}{\pgfqpoint{-0.048611in}{0.000000in}}{\pgfqpoint{0.000000in}{0.000000in}}{%
\pgfpathmoveto{\pgfqpoint{0.000000in}{0.000000in}}%
\pgfpathlineto{\pgfqpoint{-0.048611in}{0.000000in}}%
\pgfusepath{stroke,fill}%
}%
\begin{pgfscope}%
\pgfsys@transformshift{0.375000in}{0.435000in}%
\pgfsys@useobject{currentmarker}{}%
\end{pgfscope}%
\end{pgfscope}%
\begin{pgfscope}%
\definecolor{textcolor}{rgb}{0.000000,0.000000,0.000000}%
\pgfsetstrokecolor{textcolor}%
\pgfsetfillcolor{textcolor}%
\pgftext[x=0.073039in,y=0.382238in,left,base]{\color{textcolor}\sffamily\fontsize{10.000000}{12.000000}\selectfont −1}%
\end{pgfscope}%
\begin{pgfscope}%
\pgfsetbuttcap%
\pgfsetroundjoin%
\definecolor{currentfill}{rgb}{0.000000,0.000000,0.000000}%
\pgfsetfillcolor{currentfill}%
\pgfsetlinewidth{0.803000pt}%
\definecolor{currentstroke}{rgb}{0.000000,0.000000,0.000000}%
\pgfsetstrokecolor{currentstroke}%
\pgfsetdash{}{0pt}%
\pgfsys@defobject{currentmarker}{\pgfqpoint{-0.048611in}{0.000000in}}{\pgfqpoint{0.000000in}{0.000000in}}{%
\pgfpathmoveto{\pgfqpoint{0.000000in}{0.000000in}}%
\pgfpathlineto{\pgfqpoint{-0.048611in}{0.000000in}}%
\pgfusepath{stroke,fill}%
}%
\begin{pgfscope}%
\pgfsys@transformshift{0.375000in}{1.485018in}%
\pgfsys@useobject{currentmarker}{}%
\end{pgfscope}%
\end{pgfscope}%
\begin{pgfscope}%
\definecolor{textcolor}{rgb}{0.000000,0.000000,0.000000}%
\pgfsetstrokecolor{textcolor}%
\pgfsetfillcolor{textcolor}%
\pgftext[x=0.189413in,y=1.432256in,left,base]{\color{textcolor}\sffamily\fontsize{10.000000}{12.000000}\selectfont 0}%
\end{pgfscope}%
\begin{pgfscope}%
\pgfsetbuttcap%
\pgfsetroundjoin%
\definecolor{currentfill}{rgb}{0.000000,0.000000,0.000000}%
\pgfsetfillcolor{currentfill}%
\pgfsetlinewidth{0.803000pt}%
\definecolor{currentstroke}{rgb}{0.000000,0.000000,0.000000}%
\pgfsetstrokecolor{currentstroke}%
\pgfsetdash{}{0pt}%
\pgfsys@defobject{currentmarker}{\pgfqpoint{-0.048611in}{0.000000in}}{\pgfqpoint{0.000000in}{0.000000in}}{%
\pgfpathmoveto{\pgfqpoint{0.000000in}{0.000000in}}%
\pgfpathlineto{\pgfqpoint{-0.048611in}{0.000000in}}%
\pgfusepath{stroke,fill}%
}%
\begin{pgfscope}%
\pgfsys@transformshift{0.375000in}{2.535036in}%
\pgfsys@useobject{currentmarker}{}%
\end{pgfscope}%
\end{pgfscope}%
\begin{pgfscope}%
\definecolor{textcolor}{rgb}{0.000000,0.000000,0.000000}%
\pgfsetstrokecolor{textcolor}%
\pgfsetfillcolor{textcolor}%
\pgftext[x=0.189413in,y=2.482275in,left,base]{\color{textcolor}\sffamily\fontsize{10.000000}{12.000000}\selectfont 1}%
\end{pgfscope}%
\begin{pgfscope}%
\pgfpathrectangle{\pgfqpoint{0.375000in}{0.330000in}}{\pgfqpoint{2.325000in}{2.310000in}}%
\pgfusepath{clip}%
\pgfsetrectcap%
\pgfsetroundjoin%
\pgfsetlinewidth{1.505625pt}%
\definecolor{currentstroke}{rgb}{0.700000,0.700000,0.700000}%
\pgfsetstrokecolor{currentstroke}%
\pgfsetdash{}{0pt}%
\pgfpathmoveto{\pgfqpoint{0.480682in}{1.485018in}}%
\pgfpathlineto{\pgfqpoint{0.501818in}{1.988424in}}%
\pgfpathlineto{\pgfqpoint{0.512386in}{2.200751in}}%
\pgfpathlineto{\pgfqpoint{0.522955in}{2.368578in}}%
\pgfpathlineto{\pgfqpoint{0.530881in}{2.458844in}}%
\pgfpathlineto{\pgfqpoint{0.536165in}{2.500203in}}%
\pgfpathlineto{\pgfqpoint{0.541449in}{2.525722in}}%
\pgfpathlineto{\pgfqpoint{0.544091in}{2.532406in}}%
\pgfpathlineto{\pgfqpoint{0.546733in}{2.535000in}}%
\pgfpathlineto{\pgfqpoint{0.549375in}{2.533494in}}%
\pgfpathlineto{\pgfqpoint{0.552017in}{2.527894in}}%
\pgfpathlineto{\pgfqpoint{0.557301in}{2.504514in}}%
\pgfpathlineto{\pgfqpoint{0.562585in}{2.465225in}}%
\pgfpathlineto{\pgfqpoint{0.567869in}{2.410640in}}%
\pgfpathlineto{\pgfqpoint{0.575795in}{2.302009in}}%
\pgfpathlineto{\pgfqpoint{0.586364in}{2.113425in}}%
\pgfpathlineto{\pgfqpoint{0.599574in}{1.824368in}}%
\pgfpathlineto{\pgfqpoint{0.644489in}{0.782131in}}%
\pgfpathlineto{\pgfqpoint{0.655057in}{0.610993in}}%
\pgfpathlineto{\pgfqpoint{0.662983in}{0.517842in}}%
\pgfpathlineto{\pgfqpoint{0.668267in}{0.474422in}}%
\pgfpathlineto{\pgfqpoint{0.673551in}{0.446773in}}%
\pgfpathlineto{\pgfqpoint{0.676193in}{0.439007in}}%
\pgfpathlineto{\pgfqpoint{0.678835in}{0.435325in}}%
\pgfpathlineto{\pgfqpoint{0.681477in}{0.435742in}}%
\pgfpathlineto{\pgfqpoint{0.684119in}{0.440257in}}%
\pgfpathlineto{\pgfqpoint{0.686761in}{0.448852in}}%
\pgfpathlineto{\pgfqpoint{0.692045in}{0.478130in}}%
\pgfpathlineto{\pgfqpoint{0.697330in}{0.523121in}}%
\pgfpathlineto{\pgfqpoint{0.705256in}{0.618466in}}%
\pgfpathlineto{\pgfqpoint{0.713182in}{0.744188in}}%
\pgfpathlineto{\pgfqpoint{0.723750in}{0.951316in}}%
\pgfpathlineto{\pgfqpoint{0.739602in}{1.319612in}}%
\pgfpathlineto{\pgfqpoint{0.768665in}{2.015216in}}%
\pgfpathlineto{\pgfqpoint{0.779233in}{2.222962in}}%
\pgfpathlineto{\pgfqpoint{0.789801in}{2.384827in}}%
\pgfpathlineto{\pgfqpoint{0.797727in}{2.469935in}}%
\pgfpathlineto{\pgfqpoint{0.803011in}{2.507629in}}%
\pgfpathlineto{\pgfqpoint{0.808295in}{2.529365in}}%
\pgfpathlineto{\pgfqpoint{0.810937in}{2.534133in}}%
\pgfpathlineto{\pgfqpoint{0.813580in}{2.534804in}}%
\pgfpathlineto{\pgfqpoint{0.816222in}{2.531376in}}%
\pgfpathlineto{\pgfqpoint{0.818864in}{2.523862in}}%
\pgfpathlineto{\pgfqpoint{0.824148in}{2.496709in}}%
\pgfpathlineto{\pgfqpoint{0.829432in}{2.453769in}}%
\pgfpathlineto{\pgfqpoint{0.837358in}{2.361289in}}%
\pgfpathlineto{\pgfqpoint{0.845284in}{2.238092in}}%
\pgfpathlineto{\pgfqpoint{0.855852in}{2.033650in}}%
\pgfpathlineto{\pgfqpoint{0.869063in}{1.731833in}}%
\pgfpathlineto{\pgfqpoint{0.903409in}{0.913786in}}%
\pgfpathlineto{\pgfqpoint{0.913977in}{0.713571in}}%
\pgfpathlineto{\pgfqpoint{0.921903in}{0.594312in}}%
\pgfpathlineto{\pgfqpoint{0.929830in}{0.506276in}}%
\pgfpathlineto{\pgfqpoint{0.935114in}{0.466503in}}%
\pgfpathlineto{\pgfqpoint{0.940398in}{0.442623in}}%
\pgfpathlineto{\pgfqpoint{0.943040in}{0.436770in}}%
\pgfpathlineto{\pgfqpoint{0.945682in}{0.435010in}}%
\pgfpathlineto{\pgfqpoint{0.948324in}{0.437351in}}%
\pgfpathlineto{\pgfqpoint{0.950966in}{0.443782in}}%
\pgfpathlineto{\pgfqpoint{0.956250in}{0.468802in}}%
\pgfpathlineto{\pgfqpoint{0.961534in}{0.509680in}}%
\pgfpathlineto{\pgfqpoint{0.969460in}{0.599269in}}%
\pgfpathlineto{\pgfqpoint{0.977386in}{0.719907in}}%
\pgfpathlineto{\pgfqpoint{0.987955in}{0.921607in}}%
\pgfpathlineto{\pgfqpoint{1.001165in}{1.221303in}}%
\pgfpathlineto{\pgfqpoint{1.038153in}{2.096080in}}%
\pgfpathlineto{\pgfqpoint{1.048722in}{2.288341in}}%
\pgfpathlineto{\pgfqpoint{1.056648in}{2.400303in}}%
\pgfpathlineto{\pgfqpoint{1.064574in}{2.480180in}}%
\pgfpathlineto{\pgfqpoint{1.069858in}{2.514175in}}%
\pgfpathlineto{\pgfqpoint{1.075142in}{2.532110in}}%
\pgfpathlineto{\pgfqpoint{1.077784in}{2.534958in}}%
\pgfpathlineto{\pgfqpoint{1.080426in}{2.533706in}}%
\pgfpathlineto{\pgfqpoint{1.083068in}{2.528359in}}%
\pgfpathlineto{\pgfqpoint{1.088352in}{2.505479in}}%
\pgfpathlineto{\pgfqpoint{1.093636in}{2.466675in}}%
\pgfpathlineto{\pgfqpoint{1.098920in}{2.412552in}}%
\pgfpathlineto{\pgfqpoint{1.106847in}{2.304556in}}%
\pgfpathlineto{\pgfqpoint{1.117415in}{2.116676in}}%
\pgfpathlineto{\pgfqpoint{1.130625in}{1.828211in}}%
\pgfpathlineto{\pgfqpoint{1.175540in}{0.785155in}}%
\pgfpathlineto{\pgfqpoint{1.186108in}{0.613252in}}%
\pgfpathlineto{\pgfqpoint{1.194034in}{0.519431in}}%
\pgfpathlineto{\pgfqpoint{1.199318in}{0.475533in}}%
\pgfpathlineto{\pgfqpoint{1.204602in}{0.447388in}}%
\pgfpathlineto{\pgfqpoint{1.207244in}{0.439369in}}%
\pgfpathlineto{\pgfqpoint{1.209886in}{0.435434in}}%
\pgfpathlineto{\pgfqpoint{1.212528in}{0.435597in}}%
\pgfpathlineto{\pgfqpoint{1.215170in}{0.439859in}}%
\pgfpathlineto{\pgfqpoint{1.217812in}{0.448202in}}%
\pgfpathlineto{\pgfqpoint{1.223097in}{0.476985in}}%
\pgfpathlineto{\pgfqpoint{1.228381in}{0.521498in}}%
\pgfpathlineto{\pgfqpoint{1.236307in}{0.616178in}}%
\pgfpathlineto{\pgfqpoint{1.244233in}{0.741313in}}%
\pgfpathlineto{\pgfqpoint{1.254801in}{0.947820in}}%
\pgfpathlineto{\pgfqpoint{1.270653in}{1.315600in}}%
\pgfpathlineto{\pgfqpoint{1.299716in}{2.011704in}}%
\pgfpathlineto{\pgfqpoint{1.310284in}{2.220066in}}%
\pgfpathlineto{\pgfqpoint{1.320852in}{2.382725in}}%
\pgfpathlineto{\pgfqpoint{1.328778in}{2.468519in}}%
\pgfpathlineto{\pgfqpoint{1.334062in}{2.506698in}}%
\pgfpathlineto{\pgfqpoint{1.339347in}{2.528935in}}%
\pgfpathlineto{\pgfqpoint{1.341989in}{2.533956in}}%
\pgfpathlineto{\pgfqpoint{1.344631in}{2.534882in}}%
\pgfpathlineto{\pgfqpoint{1.347273in}{2.531707in}}%
\pgfpathlineto{\pgfqpoint{1.349915in}{2.524446in}}%
\pgfpathlineto{\pgfqpoint{1.355199in}{2.497790in}}%
\pgfpathlineto{\pgfqpoint{1.360483in}{2.455330in}}%
\pgfpathlineto{\pgfqpoint{1.368409in}{2.363522in}}%
\pgfpathlineto{\pgfqpoint{1.376335in}{2.240919in}}%
\pgfpathlineto{\pgfqpoint{1.386903in}{2.037111in}}%
\pgfpathlineto{\pgfqpoint{1.400114in}{1.735781in}}%
\pgfpathlineto{\pgfqpoint{1.437102in}{0.863142in}}%
\pgfpathlineto{\pgfqpoint{1.447670in}{0.673158in}}%
\pgfpathlineto{\pgfqpoint{1.455597in}{0.563261in}}%
\pgfpathlineto{\pgfqpoint{1.463523in}{0.485675in}}%
\pgfpathlineto{\pgfqpoint{1.468807in}{0.453295in}}%
\pgfpathlineto{\pgfqpoint{1.474091in}{0.437014in}}%
\pgfpathlineto{\pgfqpoint{1.476733in}{0.435000in}}%
\pgfpathlineto{\pgfqpoint{1.479375in}{0.437087in}}%
\pgfpathlineto{\pgfqpoint{1.482017in}{0.443265in}}%
\pgfpathlineto{\pgfqpoint{1.487301in}{0.467787in}}%
\pgfpathlineto{\pgfqpoint{1.492585in}{0.508182in}}%
\pgfpathlineto{\pgfqpoint{1.497869in}{0.563820in}}%
\pgfpathlineto{\pgfqpoint{1.505795in}{0.673897in}}%
\pgfpathlineto{\pgfqpoint{1.516364in}{0.864082in}}%
\pgfpathlineto{\pgfqpoint{1.529574in}{1.154477in}}%
\pgfpathlineto{\pgfqpoint{1.571847in}{2.145065in}}%
\pgfpathlineto{\pgfqpoint{1.582415in}{2.326582in}}%
\pgfpathlineto{\pgfqpoint{1.590341in}{2.428882in}}%
\pgfpathlineto{\pgfqpoint{1.595625in}{2.478876in}}%
\pgfpathlineto{\pgfqpoint{1.600909in}{2.513361in}}%
\pgfpathlineto{\pgfqpoint{1.606193in}{2.531799in}}%
\pgfpathlineto{\pgfqpoint{1.608835in}{2.534901in}}%
\pgfpathlineto{\pgfqpoint{1.611477in}{2.533903in}}%
\pgfpathlineto{\pgfqpoint{1.614119in}{2.528809in}}%
\pgfpathlineto{\pgfqpoint{1.616761in}{2.519639in}}%
\pgfpathlineto{\pgfqpoint{1.622045in}{2.489230in}}%
\pgfpathlineto{\pgfqpoint{1.627330in}{2.443151in}}%
\pgfpathlineto{\pgfqpoint{1.635256in}{2.346287in}}%
\pgfpathlineto{\pgfqpoint{1.643182in}{2.219233in}}%
\pgfpathlineto{\pgfqpoint{1.653750in}{2.010695in}}%
\pgfpathlineto{\pgfqpoint{1.669602in}{1.641239in}}%
\pgfpathlineto{\pgfqpoint{1.698665in}{0.946818in}}%
\pgfpathlineto{\pgfqpoint{1.709233in}{0.740491in}}%
\pgfpathlineto{\pgfqpoint{1.717159in}{0.615524in}}%
\pgfpathlineto{\pgfqpoint{1.725085in}{0.521035in}}%
\pgfpathlineto{\pgfqpoint{1.730369in}{0.476659in}}%
\pgfpathlineto{\pgfqpoint{1.735653in}{0.448018in}}%
\pgfpathlineto{\pgfqpoint{1.738295in}{0.439747in}}%
\pgfpathlineto{\pgfqpoint{1.740938in}{0.435559in}}%
\pgfpathlineto{\pgfqpoint{1.743580in}{0.435468in}}%
\pgfpathlineto{\pgfqpoint{1.746222in}{0.439476in}}%
\pgfpathlineto{\pgfqpoint{1.748864in}{0.447567in}}%
\pgfpathlineto{\pgfqpoint{1.754148in}{0.475855in}}%
\pgfpathlineto{\pgfqpoint{1.759432in}{0.519890in}}%
\pgfpathlineto{\pgfqpoint{1.767358in}{0.613902in}}%
\pgfpathlineto{\pgfqpoint{1.775284in}{0.738450in}}%
\pgfpathlineto{\pgfqpoint{1.785852in}{0.944332in}}%
\pgfpathlineto{\pgfqpoint{1.801705in}{1.311590in}}%
\pgfpathlineto{\pgfqpoint{1.830767in}{2.008184in}}%
\pgfpathlineto{\pgfqpoint{1.841335in}{2.217158in}}%
\pgfpathlineto{\pgfqpoint{1.851903in}{2.380610in}}%
\pgfpathlineto{\pgfqpoint{1.859830in}{2.467088in}}%
\pgfpathlineto{\pgfqpoint{1.865114in}{2.505753in}}%
\pgfpathlineto{\pgfqpoint{1.870398in}{2.528490in}}%
\pgfpathlineto{\pgfqpoint{1.873040in}{2.533764in}}%
\pgfpathlineto{\pgfqpoint{1.875682in}{2.534944in}}%
\pgfpathlineto{\pgfqpoint{1.878324in}{2.532023in}}%
\pgfpathlineto{\pgfqpoint{1.880966in}{2.525014in}}%
\pgfpathlineto{\pgfqpoint{1.886250in}{2.498855in}}%
\pgfpathlineto{\pgfqpoint{1.891534in}{2.456876in}}%
\pgfpathlineto{\pgfqpoint{1.899460in}{2.365741in}}%
\pgfpathlineto{\pgfqpoint{1.907386in}{2.243734in}}%
\pgfpathlineto{\pgfqpoint{1.917955in}{2.040564in}}%
\pgfpathlineto{\pgfqpoint{1.931165in}{1.739726in}}%
\pgfpathlineto{\pgfqpoint{1.968153in}{0.866422in}}%
\pgfpathlineto{\pgfqpoint{1.978722in}{0.675741in}}%
\pgfpathlineto{\pgfqpoint{1.986648in}{0.565214in}}%
\pgfpathlineto{\pgfqpoint{1.994574in}{0.486930in}}%
\pgfpathlineto{\pgfqpoint{1.999858in}{0.454058in}}%
\pgfpathlineto{\pgfqpoint{2.005142in}{0.437273in}}%
\pgfpathlineto{\pgfqpoint{2.007784in}{0.435006in}}%
\pgfpathlineto{\pgfqpoint{2.010426in}{0.436838in}}%
\pgfpathlineto{\pgfqpoint{2.013068in}{0.442764in}}%
\pgfpathlineto{\pgfqpoint{2.018352in}{0.466787in}}%
\pgfpathlineto{\pgfqpoint{2.023636in}{0.506698in}}%
\pgfpathlineto{\pgfqpoint{2.028920in}{0.561876in}}%
\pgfpathlineto{\pgfqpoint{2.036847in}{0.671322in}}%
\pgfpathlineto{\pgfqpoint{2.047415in}{0.860809in}}%
\pgfpathlineto{\pgfqpoint{2.060625in}{1.150622in}}%
\pgfpathlineto{\pgfqpoint{2.102898in}{2.141899in}}%
\pgfpathlineto{\pgfqpoint{2.113466in}{2.324145in}}%
\pgfpathlineto{\pgfqpoint{2.121392in}{2.427094in}}%
\pgfpathlineto{\pgfqpoint{2.129318in}{2.497021in}}%
\pgfpathlineto{\pgfqpoint{2.134602in}{2.524031in}}%
\pgfpathlineto{\pgfqpoint{2.137244in}{2.531473in}}%
\pgfpathlineto{\pgfqpoint{2.139886in}{2.534828in}}%
\pgfpathlineto{\pgfqpoint{2.142528in}{2.534084in}}%
\pgfpathlineto{\pgfqpoint{2.145170in}{2.529243in}}%
\pgfpathlineto{\pgfqpoint{2.147813in}{2.520325in}}%
\pgfpathlineto{\pgfqpoint{2.153097in}{2.490410in}}%
\pgfpathlineto{\pgfqpoint{2.158381in}{2.444806in}}%
\pgfpathlineto{\pgfqpoint{2.166307in}{2.348605in}}%
\pgfpathlineto{\pgfqpoint{2.174233in}{2.222133in}}%
\pgfpathlineto{\pgfqpoint{2.184801in}{2.014210in}}%
\pgfpathlineto{\pgfqpoint{2.200653in}{1.645257in}}%
\pgfpathlineto{\pgfqpoint{2.229716in}{0.950312in}}%
\pgfpathlineto{\pgfqpoint{2.240284in}{0.743362in}}%
\pgfpathlineto{\pgfqpoint{2.248210in}{0.617809in}}%
\pgfpathlineto{\pgfqpoint{2.256136in}{0.522654in}}%
\pgfpathlineto{\pgfqpoint{2.261420in}{0.477800in}}%
\pgfpathlineto{\pgfqpoint{2.266705in}{0.448664in}}%
\pgfpathlineto{\pgfqpoint{2.269347in}{0.440141in}}%
\pgfpathlineto{\pgfqpoint{2.271989in}{0.435699in}}%
\pgfpathlineto{\pgfqpoint{2.274631in}{0.435355in}}%
\pgfpathlineto{\pgfqpoint{2.277273in}{0.439109in}}%
\pgfpathlineto{\pgfqpoint{2.279915in}{0.446948in}}%
\pgfpathlineto{\pgfqpoint{2.285199in}{0.474739in}}%
\pgfpathlineto{\pgfqpoint{2.290483in}{0.518296in}}%
\pgfpathlineto{\pgfqpoint{2.298409in}{0.611639in}}%
\pgfpathlineto{\pgfqpoint{2.306335in}{0.735597in}}%
\pgfpathlineto{\pgfqpoint{2.316903in}{0.940852in}}%
\pgfpathlineto{\pgfqpoint{2.332756in}{1.307583in}}%
\pgfpathlineto{\pgfqpoint{2.364460in}{2.060631in}}%
\pgfpathlineto{\pgfqpoint{2.375028in}{2.260003in}}%
\pgfpathlineto{\pgfqpoint{2.382955in}{2.378482in}}%
\pgfpathlineto{\pgfqpoint{2.390881in}{2.465642in}}%
\pgfpathlineto{\pgfqpoint{2.396165in}{2.504792in}}%
\pgfpathlineto{\pgfqpoint{2.401449in}{2.528029in}}%
\pgfpathlineto{\pgfqpoint{2.404091in}{2.533557in}}%
\pgfpathlineto{\pgfqpoint{2.406733in}{2.534990in}}%
\pgfpathlineto{\pgfqpoint{2.409375in}{2.532323in}}%
\pgfpathlineto{\pgfqpoint{2.412017in}{2.525566in}}%
\pgfpathlineto{\pgfqpoint{2.417301in}{2.499905in}}%
\pgfpathlineto{\pgfqpoint{2.422585in}{2.458407in}}%
\pgfpathlineto{\pgfqpoint{2.430511in}{2.367947in}}%
\pgfpathlineto{\pgfqpoint{2.438437in}{2.246538in}}%
\pgfpathlineto{\pgfqpoint{2.449006in}{2.044009in}}%
\pgfpathlineto{\pgfqpoint{2.462216in}{1.743667in}}%
\pgfpathlineto{\pgfqpoint{2.499205in}{0.869710in}}%
\pgfpathlineto{\pgfqpoint{2.509773in}{0.678337in}}%
\pgfpathlineto{\pgfqpoint{2.517699in}{0.567182in}}%
\pgfpathlineto{\pgfqpoint{2.525625in}{0.488200in}}%
\pgfpathlineto{\pgfqpoint{2.530909in}{0.454836in}}%
\pgfpathlineto{\pgfqpoint{2.536193in}{0.437548in}}%
\pgfpathlineto{\pgfqpoint{2.538835in}{0.435027in}}%
\pgfpathlineto{\pgfqpoint{2.541477in}{0.436606in}}%
\pgfpathlineto{\pgfqpoint{2.544119in}{0.442279in}}%
\pgfpathlineto{\pgfqpoint{2.549403in}{0.465802in}}%
\pgfpathlineto{\pgfqpoint{2.554688in}{0.505230in}}%
\pgfpathlineto{\pgfqpoint{2.559972in}{0.559947in}}%
\pgfpathlineto{\pgfqpoint{2.567898in}{0.668760in}}%
\pgfpathlineto{\pgfqpoint{2.578466in}{0.857546in}}%
\pgfpathlineto{\pgfqpoint{2.591676in}{1.146772in}}%
\pgfpathlineto{\pgfqpoint{2.594318in}{1.209520in}}%
\pgfpathlineto{\pgfqpoint{2.594318in}{1.209520in}}%
\pgfusepath{stroke}%
\end{pgfscope}%
\begin{pgfscope}%
\pgfpathrectangle{\pgfqpoint{0.375000in}{0.330000in}}{\pgfqpoint{2.325000in}{2.310000in}}%
\pgfusepath{clip}%
\pgfsetbuttcap%
\pgfsetroundjoin%
\pgfsetlinewidth{1.505625pt}%
\definecolor{currentstroke}{rgb}{0.000000,0.000000,0.000000}%
\pgfsetstrokecolor{currentstroke}%
\pgfsetdash{{5.550000pt}{2.400000pt}}{0.000000pt}%
\pgfpathmoveto{\pgfqpoint{0.480682in}{1.485018in}}%
\pgfpathlineto{\pgfqpoint{0.612784in}{1.502440in}}%
\pgfpathlineto{\pgfqpoint{0.744886in}{1.450179in}}%
\pgfpathlineto{\pgfqpoint{0.876989in}{1.537264in}}%
\pgfpathlineto{\pgfqpoint{1.009091in}{1.415379in}}%
\pgfpathlineto{\pgfqpoint{1.141193in}{1.572031in}}%
\pgfpathlineto{\pgfqpoint{1.273295in}{1.380655in}}%
\pgfpathlineto{\pgfqpoint{1.405398in}{1.606702in}}%
\pgfpathlineto{\pgfqpoint{1.537500in}{1.346046in}}%
\pgfpathlineto{\pgfqpoint{1.669602in}{1.641239in}}%
\pgfpathlineto{\pgfqpoint{1.801705in}{1.311590in}}%
\pgfpathlineto{\pgfqpoint{1.933807in}{1.675604in}}%
\pgfpathlineto{\pgfqpoint{2.065909in}{1.277326in}}%
\pgfpathlineto{\pgfqpoint{2.198011in}{1.709759in}}%
\pgfpathlineto{\pgfqpoint{2.330114in}{1.243290in}}%
\pgfpathlineto{\pgfqpoint{2.462216in}{1.743667in}}%
\pgfpathlineto{\pgfqpoint{2.594318in}{1.209520in}}%
\pgfusepath{stroke}%
\end{pgfscope}%
\begin{pgfscope}%
\pgfsetrectcap%
\pgfsetmiterjoin%
\pgfsetlinewidth{0.803000pt}%
\definecolor{currentstroke}{rgb}{0.000000,0.000000,0.000000}%
\pgfsetstrokecolor{currentstroke}%
\pgfsetdash{}{0pt}%
\pgfpathmoveto{\pgfqpoint{0.375000in}{0.330000in}}%
\pgfpathlineto{\pgfqpoint{0.375000in}{2.640000in}}%
\pgfusepath{stroke}%
\end{pgfscope}%
\begin{pgfscope}%
\pgfsetrectcap%
\pgfsetmiterjoin%
\pgfsetlinewidth{0.803000pt}%
\definecolor{currentstroke}{rgb}{0.000000,0.000000,0.000000}%
\pgfsetstrokecolor{currentstroke}%
\pgfsetdash{}{0pt}%
\pgfpathmoveto{\pgfqpoint{2.700000in}{0.330000in}}%
\pgfpathlineto{\pgfqpoint{2.700000in}{2.640000in}}%
\pgfusepath{stroke}%
\end{pgfscope}%
\begin{pgfscope}%
\pgfsetrectcap%
\pgfsetmiterjoin%
\pgfsetlinewidth{0.803000pt}%
\definecolor{currentstroke}{rgb}{0.000000,0.000000,0.000000}%
\pgfsetstrokecolor{currentstroke}%
\pgfsetdash{}{0pt}%
\pgfpathmoveto{\pgfqpoint{0.375000in}{0.330000in}}%
\pgfpathlineto{\pgfqpoint{2.700000in}{0.330000in}}%
\pgfusepath{stroke}%
\end{pgfscope}%
\begin{pgfscope}%
\pgfsetrectcap%
\pgfsetmiterjoin%
\pgfsetlinewidth{0.803000pt}%
\definecolor{currentstroke}{rgb}{0.000000,0.000000,0.000000}%
\pgfsetstrokecolor{currentstroke}%
\pgfsetdash{}{0pt}%
\pgfpathmoveto{\pgfqpoint{0.375000in}{2.640000in}}%
\pgfpathlineto{\pgfqpoint{2.700000in}{2.640000in}}%
\pgfusepath{stroke}%
\end{pgfscope}%
\end{pgfpicture}%
\makeatother%
\endgroup%

    \end{subfigure}
\caption[The increasing interpolation error if the mesh is not refined with increasing frequency.]{\label{fig:introinterp} The increasing interpolation error if the mesh is not refined with increasing frequency. The left-hand plot show the interpolant of $\sin(10x)$ on a mesh with 10 points per wavelength. The right-hand plot shows the interpolant of $\sin(50x)$ on the same mesh.}
\end{figure}

However, discretising \eqref{eq:introdet} with a fixed number of points per wavelength is \emph{not} enough to keep the error in the finite-element solution of \eqref{eq:introdet} bounded as $k\rightarrow \infty$ when using fixed-order methods. This is because standard-finite-element methods applied to the Helmholtz equation suffer from pollution, where the numerically calculated wave has a different wavelength to the true solution $u$, and so `drifts' away from $u$; moreover, this error increases as $k$ increases. See \cref{fig:intropoll} for an illustration of this phenomenon, and \cref{sec:helmfedisc} for an extended discussion how this phenomenon is reflected in the accuracy of finite-element methods for the Helmholtz equation.

\begin{figure}[h]
  \centering
  \begin{subfigure}{0.45\textwidth}
%% Creator: Matplotlib, PGF backend
%%
%% To include the figure in your LaTeX document, write
%%   \input{<filename>.pgf}
%%
%% Make sure the required packages are loaded in your preamble
%%   \usepackage{pgf}
%%
%% Figures using additional raster images can only be included by \input if
%% they are in the same directory as the main LaTeX file. For loading figures
%% from other directories you can use the `import` package
%%   \usepackage{import}
%% and then include the figures with
%%   \import{<path to file>}{<filename>.pgf}
%%
%% Matplotlib used the following preamble
%%   \usepackage{fontspec}
%%   \setmainfont{DejaVuSerif.ttf}[Path=/home/owen/progs/firedrake-complex/firedrake/lib/python3.5/site-packages/matplotlib/mpl-data/fonts/ttf/]
%%   \setsansfont{DejaVuSans.ttf}[Path=/home/owen/progs/firedrake-complex/firedrake/lib/python3.5/site-packages/matplotlib/mpl-data/fonts/ttf/]
%%   \setmonofont{DejaVuSansMono.ttf}[Path=/home/owen/progs/firedrake-complex/firedrake/lib/python3.5/site-packages/matplotlib/mpl-data/fonts/ttf/]
%%
\begingroup%
\makeatletter%
\begin{pgfpicture}%
\pgfpathrectangle{\pgfpointorigin}{\pgfqpoint{3.000000in}{3.000000in}}%
\pgfusepath{use as bounding box, clip}%
\begin{pgfscope}%
\pgfsetbuttcap%
\pgfsetmiterjoin%
\definecolor{currentfill}{rgb}{1.000000,1.000000,1.000000}%
\pgfsetfillcolor{currentfill}%
\pgfsetlinewidth{0.000000pt}%
\definecolor{currentstroke}{rgb}{1.000000,1.000000,1.000000}%
\pgfsetstrokecolor{currentstroke}%
\pgfsetdash{}{0pt}%
\pgfpathmoveto{\pgfqpoint{0.000000in}{0.000000in}}%
\pgfpathlineto{\pgfqpoint{3.000000in}{0.000000in}}%
\pgfpathlineto{\pgfqpoint{3.000000in}{3.000000in}}%
\pgfpathlineto{\pgfqpoint{0.000000in}{3.000000in}}%
\pgfpathclose%
\pgfusepath{fill}%
\end{pgfscope}%
\begin{pgfscope}%
\pgfsetbuttcap%
\pgfsetmiterjoin%
\definecolor{currentfill}{rgb}{1.000000,1.000000,1.000000}%
\pgfsetfillcolor{currentfill}%
\pgfsetlinewidth{0.000000pt}%
\definecolor{currentstroke}{rgb}{0.000000,0.000000,0.000000}%
\pgfsetstrokecolor{currentstroke}%
\pgfsetstrokeopacity{0.000000}%
\pgfsetdash{}{0pt}%
\pgfpathmoveto{\pgfqpoint{0.375000in}{0.330000in}}%
\pgfpathlineto{\pgfqpoint{2.700000in}{0.330000in}}%
\pgfpathlineto{\pgfqpoint{2.700000in}{2.640000in}}%
\pgfpathlineto{\pgfqpoint{0.375000in}{2.640000in}}%
\pgfpathclose%
\pgfusepath{fill}%
\end{pgfscope}%
\begin{pgfscope}%
\pgfsetbuttcap%
\pgfsetroundjoin%
\definecolor{currentfill}{rgb}{0.000000,0.000000,0.000000}%
\pgfsetfillcolor{currentfill}%
\pgfsetlinewidth{0.803000pt}%
\definecolor{currentstroke}{rgb}{0.000000,0.000000,0.000000}%
\pgfsetstrokecolor{currentstroke}%
\pgfsetdash{}{0pt}%
\pgfsys@defobject{currentmarker}{\pgfqpoint{0.000000in}{-0.048611in}}{\pgfqpoint{0.000000in}{0.000000in}}{%
\pgfpathmoveto{\pgfqpoint{0.000000in}{0.000000in}}%
\pgfpathlineto{\pgfqpoint{0.000000in}{-0.048611in}}%
\pgfusepath{stroke,fill}%
}%
\begin{pgfscope}%
\pgfsys@transformshift{0.375000in}{0.330000in}%
\pgfsys@useobject{currentmarker}{}%
\end{pgfscope}%
\end{pgfscope}%
\begin{pgfscope}%
\definecolor{textcolor}{rgb}{0.000000,0.000000,0.000000}%
\pgfsetstrokecolor{textcolor}%
\pgfsetfillcolor{textcolor}%
\pgftext[x=0.375000in,y=0.232778in,,top]{\color{textcolor}\sffamily\fontsize{10.000000}{12.000000}\selectfont 0.00}%
\end{pgfscope}%
\begin{pgfscope}%
\pgfsetbuttcap%
\pgfsetroundjoin%
\definecolor{currentfill}{rgb}{0.000000,0.000000,0.000000}%
\pgfsetfillcolor{currentfill}%
\pgfsetlinewidth{0.803000pt}%
\definecolor{currentstroke}{rgb}{0.000000,0.000000,0.000000}%
\pgfsetstrokecolor{currentstroke}%
\pgfsetdash{}{0pt}%
\pgfsys@defobject{currentmarker}{\pgfqpoint{0.000000in}{-0.048611in}}{\pgfqpoint{0.000000in}{0.000000in}}{%
\pgfpathmoveto{\pgfqpoint{0.000000in}{0.000000in}}%
\pgfpathlineto{\pgfqpoint{0.000000in}{-0.048611in}}%
\pgfusepath{stroke,fill}%
}%
\begin{pgfscope}%
\pgfsys@transformshift{1.305000in}{0.330000in}%
\pgfsys@useobject{currentmarker}{}%
\end{pgfscope}%
\end{pgfscope}%
\begin{pgfscope}%
\definecolor{textcolor}{rgb}{0.000000,0.000000,0.000000}%
\pgfsetstrokecolor{textcolor}%
\pgfsetfillcolor{textcolor}%
\pgftext[x=1.305000in,y=0.232778in,,top]{\color{textcolor}\sffamily\fontsize{10.000000}{12.000000}\selectfont 0.01}%
\end{pgfscope}%
\begin{pgfscope}%
\pgfsetbuttcap%
\pgfsetroundjoin%
\definecolor{currentfill}{rgb}{0.000000,0.000000,0.000000}%
\pgfsetfillcolor{currentfill}%
\pgfsetlinewidth{0.803000pt}%
\definecolor{currentstroke}{rgb}{0.000000,0.000000,0.000000}%
\pgfsetstrokecolor{currentstroke}%
\pgfsetdash{}{0pt}%
\pgfsys@defobject{currentmarker}{\pgfqpoint{0.000000in}{-0.048611in}}{\pgfqpoint{0.000000in}{0.000000in}}{%
\pgfpathmoveto{\pgfqpoint{0.000000in}{0.000000in}}%
\pgfpathlineto{\pgfqpoint{0.000000in}{-0.048611in}}%
\pgfusepath{stroke,fill}%
}%
\begin{pgfscope}%
\pgfsys@transformshift{2.235000in}{0.330000in}%
\pgfsys@useobject{currentmarker}{}%
\end{pgfscope}%
\end{pgfscope}%
\begin{pgfscope}%
\definecolor{textcolor}{rgb}{0.000000,0.000000,0.000000}%
\pgfsetstrokecolor{textcolor}%
\pgfsetfillcolor{textcolor}%
\pgftext[x=2.235000in,y=0.232778in,,top]{\color{textcolor}\sffamily\fontsize{10.000000}{12.000000}\selectfont 0.02}%
\end{pgfscope}%
\begin{pgfscope}%
\pgfsetbuttcap%
\pgfsetroundjoin%
\definecolor{currentfill}{rgb}{0.000000,0.000000,0.000000}%
\pgfsetfillcolor{currentfill}%
\pgfsetlinewidth{0.803000pt}%
\definecolor{currentstroke}{rgb}{0.000000,0.000000,0.000000}%
\pgfsetstrokecolor{currentstroke}%
\pgfsetdash{}{0pt}%
\pgfsys@defobject{currentmarker}{\pgfqpoint{-0.048611in}{0.000000in}}{\pgfqpoint{0.000000in}{0.000000in}}{%
\pgfpathmoveto{\pgfqpoint{0.000000in}{0.000000in}}%
\pgfpathlineto{\pgfqpoint{-0.048611in}{0.000000in}}%
\pgfusepath{stroke,fill}%
}%
\begin{pgfscope}%
\pgfsys@transformshift{0.375000in}{0.435000in}%
\pgfsys@useobject{currentmarker}{}%
\end{pgfscope}%
\end{pgfscope}%
\begin{pgfscope}%
\definecolor{textcolor}{rgb}{0.000000,0.000000,0.000000}%
\pgfsetstrokecolor{textcolor}%
\pgfsetfillcolor{textcolor}%
\pgftext[x=0.073039in,y=0.382238in,left,base]{\color{textcolor}\sffamily\fontsize{10.000000}{12.000000}\selectfont −1}%
\end{pgfscope}%
\begin{pgfscope}%
\pgfsetbuttcap%
\pgfsetroundjoin%
\definecolor{currentfill}{rgb}{0.000000,0.000000,0.000000}%
\pgfsetfillcolor{currentfill}%
\pgfsetlinewidth{0.803000pt}%
\definecolor{currentstroke}{rgb}{0.000000,0.000000,0.000000}%
\pgfsetstrokecolor{currentstroke}%
\pgfsetdash{}{0pt}%
\pgfsys@defobject{currentmarker}{\pgfqpoint{-0.048611in}{0.000000in}}{\pgfqpoint{0.000000in}{0.000000in}}{%
\pgfpathmoveto{\pgfqpoint{0.000000in}{0.000000in}}%
\pgfpathlineto{\pgfqpoint{-0.048611in}{0.000000in}}%
\pgfusepath{stroke,fill}%
}%
\begin{pgfscope}%
\pgfsys@transformshift{0.375000in}{1.485000in}%
\pgfsys@useobject{currentmarker}{}%
\end{pgfscope}%
\end{pgfscope}%
\begin{pgfscope}%
\definecolor{textcolor}{rgb}{0.000000,0.000000,0.000000}%
\pgfsetstrokecolor{textcolor}%
\pgfsetfillcolor{textcolor}%
\pgftext[x=0.189413in,y=1.432238in,left,base]{\color{textcolor}\sffamily\fontsize{10.000000}{12.000000}\selectfont 0}%
\end{pgfscope}%
\begin{pgfscope}%
\pgfsetbuttcap%
\pgfsetroundjoin%
\definecolor{currentfill}{rgb}{0.000000,0.000000,0.000000}%
\pgfsetfillcolor{currentfill}%
\pgfsetlinewidth{0.803000pt}%
\definecolor{currentstroke}{rgb}{0.000000,0.000000,0.000000}%
\pgfsetstrokecolor{currentstroke}%
\pgfsetdash{}{0pt}%
\pgfsys@defobject{currentmarker}{\pgfqpoint{-0.048611in}{0.000000in}}{\pgfqpoint{0.000000in}{0.000000in}}{%
\pgfpathmoveto{\pgfqpoint{0.000000in}{0.000000in}}%
\pgfpathlineto{\pgfqpoint{-0.048611in}{0.000000in}}%
\pgfusepath{stroke,fill}%
}%
\begin{pgfscope}%
\pgfsys@transformshift{0.375000in}{2.535000in}%
\pgfsys@useobject{currentmarker}{}%
\end{pgfscope}%
\end{pgfscope}%
\begin{pgfscope}%
\definecolor{textcolor}{rgb}{0.000000,0.000000,0.000000}%
\pgfsetstrokecolor{textcolor}%
\pgfsetfillcolor{textcolor}%
\pgftext[x=0.189413in,y=2.482238in,left,base]{\color{textcolor}\sffamily\fontsize{10.000000}{12.000000}\selectfont 1}%
\end{pgfscope}%
\begin{pgfscope}%
\pgfpathrectangle{\pgfqpoint{0.375000in}{0.330000in}}{\pgfqpoint{2.325000in}{2.310000in}}%
\pgfusepath{clip}%
\pgfsetrectcap%
\pgfsetroundjoin%
\pgfsetlinewidth{1.505625pt}%
\definecolor{currentstroke}{rgb}{0.000000,0.000000,0.000000}%
\pgfsetstrokecolor{currentstroke}%
\pgfsetdash{}{0pt}%
\pgfpathmoveto{\pgfqpoint{0.375000in}{1.485000in}}%
\pgfpathlineto{\pgfqpoint{0.412971in}{1.901894in}}%
\pgfpathlineto{\pgfqpoint{0.433417in}{2.102023in}}%
\pgfpathlineto{\pgfqpoint{0.450942in}{2.250251in}}%
\pgfpathlineto{\pgfqpoint{0.465546in}{2.353272in}}%
\pgfpathlineto{\pgfqpoint{0.477230in}{2.420408in}}%
\pgfpathlineto{\pgfqpoint{0.485992in}{2.461135in}}%
\pgfpathlineto{\pgfqpoint{0.494755in}{2.493202in}}%
\pgfpathlineto{\pgfqpoint{0.503518in}{2.516325in}}%
\pgfpathlineto{\pgfqpoint{0.509359in}{2.526667in}}%
\pgfpathlineto{\pgfqpoint{0.515201in}{2.532900in}}%
\pgfpathlineto{\pgfqpoint{0.521043in}{2.535000in}}%
\pgfpathlineto{\pgfqpoint{0.526884in}{2.532958in}}%
\pgfpathlineto{\pgfqpoint{0.532726in}{2.526784in}}%
\pgfpathlineto{\pgfqpoint{0.538568in}{2.516499in}}%
\pgfpathlineto{\pgfqpoint{0.544410in}{2.502147in}}%
\pgfpathlineto{\pgfqpoint{0.553172in}{2.473118in}}%
\pgfpathlineto{\pgfqpoint{0.561935in}{2.435323in}}%
\pgfpathlineto{\pgfqpoint{0.573618in}{2.371884in}}%
\pgfpathlineto{\pgfqpoint{0.585302in}{2.294467in}}%
\pgfpathlineto{\pgfqpoint{0.599906in}{2.179917in}}%
\pgfpathlineto{\pgfqpoint{0.617431in}{2.020160in}}%
\pgfpathlineto{\pgfqpoint{0.637877in}{1.810267in}}%
\pgfpathlineto{\pgfqpoint{0.670006in}{1.452961in}}%
\pgfpathlineto{\pgfqpoint{0.705057in}{1.068963in}}%
\pgfpathlineto{\pgfqpoint{0.725503in}{0.868732in}}%
\pgfpathlineto{\pgfqpoint{0.743028in}{0.720389in}}%
\pgfpathlineto{\pgfqpoint{0.757632in}{0.617253in}}%
\pgfpathlineto{\pgfqpoint{0.769315in}{0.550016in}}%
\pgfpathlineto{\pgfqpoint{0.778078in}{0.509210in}}%
\pgfpathlineto{\pgfqpoint{0.786840in}{0.477060in}}%
\pgfpathlineto{\pgfqpoint{0.795603in}{0.453851in}}%
\pgfpathlineto{\pgfqpoint{0.801445in}{0.443451in}}%
\pgfpathlineto{\pgfqpoint{0.807286in}{0.437160in}}%
\pgfpathlineto{\pgfqpoint{0.813128in}{0.435001in}}%
\pgfpathlineto{\pgfqpoint{0.818970in}{0.436984in}}%
\pgfpathlineto{\pgfqpoint{0.824812in}{0.443100in}}%
\pgfpathlineto{\pgfqpoint{0.830653in}{0.453326in}}%
\pgfpathlineto{\pgfqpoint{0.836495in}{0.467622in}}%
\pgfpathlineto{\pgfqpoint{0.845258in}{0.496567in}}%
\pgfpathlineto{\pgfqpoint{0.854020in}{0.534281in}}%
\pgfpathlineto{\pgfqpoint{0.865704in}{0.597616in}}%
\pgfpathlineto{\pgfqpoint{0.877387in}{0.674938in}}%
\pgfpathlineto{\pgfqpoint{0.891991in}{0.789384in}}%
\pgfpathlineto{\pgfqpoint{0.909516in}{0.949037in}}%
\pgfpathlineto{\pgfqpoint{0.929962in}{1.158846in}}%
\pgfpathlineto{\pgfqpoint{0.962092in}{1.516106in}}%
\pgfpathlineto{\pgfqpoint{0.997142in}{1.900180in}}%
\pgfpathlineto{\pgfqpoint{1.017588in}{2.100512in}}%
\pgfpathlineto{\pgfqpoint{1.035113in}{2.248971in}}%
\pgfpathlineto{\pgfqpoint{1.049717in}{2.352221in}}%
\pgfpathlineto{\pgfqpoint{1.061401in}{2.419559in}}%
\pgfpathlineto{\pgfqpoint{1.070163in}{2.460445in}}%
\pgfpathlineto{\pgfqpoint{1.078926in}{2.492678in}}%
\pgfpathlineto{\pgfqpoint{1.087688in}{2.515972in}}%
\pgfpathlineto{\pgfqpoint{1.093530in}{2.526430in}}%
\pgfpathlineto{\pgfqpoint{1.099372in}{2.532780in}}%
\pgfpathlineto{\pgfqpoint{1.105214in}{2.534997in}}%
\pgfpathlineto{\pgfqpoint{1.111055in}{2.533073in}}%
\pgfpathlineto{\pgfqpoint{1.116897in}{2.527015in}}%
\pgfpathlineto{\pgfqpoint{1.122739in}{2.516847in}}%
\pgfpathlineto{\pgfqpoint{1.128580in}{2.502609in}}%
\pgfpathlineto{\pgfqpoint{1.137343in}{2.473748in}}%
\pgfpathlineto{\pgfqpoint{1.146106in}{2.436115in}}%
\pgfpathlineto{\pgfqpoint{1.157789in}{2.372882in}}%
\pgfpathlineto{\pgfqpoint{1.169472in}{2.295655in}}%
\pgfpathlineto{\pgfqpoint{1.184077in}{2.181315in}}%
\pgfpathlineto{\pgfqpoint{1.201602in}{2.021766in}}%
\pgfpathlineto{\pgfqpoint{1.222048in}{1.812042in}}%
\pgfpathlineto{\pgfqpoint{1.254177in}{1.454828in}}%
\pgfpathlineto{\pgfqpoint{1.289227in}{1.070678in}}%
\pgfpathlineto{\pgfqpoint{1.309673in}{0.870245in}}%
\pgfpathlineto{\pgfqpoint{1.327198in}{0.721670in}}%
\pgfpathlineto{\pgfqpoint{1.341803in}{0.618306in}}%
\pgfpathlineto{\pgfqpoint{1.353486in}{0.550867in}}%
\pgfpathlineto{\pgfqpoint{1.362249in}{0.509901in}}%
\pgfpathlineto{\pgfqpoint{1.371011in}{0.477584in}}%
\pgfpathlineto{\pgfqpoint{1.379774in}{0.454205in}}%
\pgfpathlineto{\pgfqpoint{1.385616in}{0.443689in}}%
\pgfpathlineto{\pgfqpoint{1.391457in}{0.437281in}}%
\pgfpathlineto{\pgfqpoint{1.397299in}{0.435005in}}%
\pgfpathlineto{\pgfqpoint{1.403141in}{0.436871in}}%
\pgfpathlineto{\pgfqpoint{1.408982in}{0.442870in}}%
\pgfpathlineto{\pgfqpoint{1.414824in}{0.452981in}}%
\pgfpathlineto{\pgfqpoint{1.420666in}{0.467162in}}%
\pgfpathlineto{\pgfqpoint{1.429428in}{0.495939in}}%
\pgfpathlineto{\pgfqpoint{1.438191in}{0.533490in}}%
\pgfpathlineto{\pgfqpoint{1.449874in}{0.596620in}}%
\pgfpathlineto{\pgfqpoint{1.461558in}{0.673752in}}%
\pgfpathlineto{\pgfqpoint{1.476162in}{0.787986in}}%
\pgfpathlineto{\pgfqpoint{1.493687in}{0.947432in}}%
\pgfpathlineto{\pgfqpoint{1.514133in}{1.157071in}}%
\pgfpathlineto{\pgfqpoint{1.546263in}{1.514239in}}%
\pgfpathlineto{\pgfqpoint{1.581313in}{1.898464in}}%
\pgfpathlineto{\pgfqpoint{1.601759in}{2.098998in}}%
\pgfpathlineto{\pgfqpoint{1.619284in}{2.247689in}}%
\pgfpathlineto{\pgfqpoint{1.633888in}{2.351167in}}%
\pgfpathlineto{\pgfqpoint{1.645572in}{2.418706in}}%
\pgfpathlineto{\pgfqpoint{1.657255in}{2.471529in}}%
\pgfpathlineto{\pgfqpoint{1.666018in}{2.500978in}}%
\pgfpathlineto{\pgfqpoint{1.674780in}{2.521415in}}%
\pgfpathlineto{\pgfqpoint{1.680622in}{2.529939in}}%
\pgfpathlineto{\pgfqpoint{1.686464in}{2.534342in}}%
\pgfpathlineto{\pgfqpoint{1.692305in}{2.534606in}}%
\pgfpathlineto{\pgfqpoint{1.698147in}{2.530730in}}%
\pgfpathlineto{\pgfqpoint{1.703989in}{2.522729in}}%
\pgfpathlineto{\pgfqpoint{1.709830in}{2.510635in}}%
\pgfpathlineto{\pgfqpoint{1.718593in}{2.484928in}}%
\pgfpathlineto{\pgfqpoint{1.727356in}{2.450351in}}%
\pgfpathlineto{\pgfqpoint{1.736118in}{2.407210in}}%
\pgfpathlineto{\pgfqpoint{1.747802in}{2.337040in}}%
\pgfpathlineto{\pgfqpoint{1.759485in}{2.253440in}}%
\pgfpathlineto{\pgfqpoint{1.774089in}{2.132080in}}%
\pgfpathlineto{\pgfqpoint{1.791614in}{1.965720in}}%
\pgfpathlineto{\pgfqpoint{1.814981in}{1.718540in}}%
\pgfpathlineto{\pgfqpoint{1.885082in}{0.954667in}}%
\pgfpathlineto{\pgfqpoint{1.902607in}{0.794293in}}%
\pgfpathlineto{\pgfqpoint{1.917211in}{0.679112in}}%
\pgfpathlineto{\pgfqpoint{1.928894in}{0.601127in}}%
\pgfpathlineto{\pgfqpoint{1.940578in}{0.537073in}}%
\pgfpathlineto{\pgfqpoint{1.949340in}{0.498791in}}%
\pgfpathlineto{\pgfqpoint{1.958103in}{0.469258in}}%
\pgfpathlineto{\pgfqpoint{1.966866in}{0.448735in}}%
\pgfpathlineto{\pgfqpoint{1.972707in}{0.440153in}}%
\pgfpathlineto{\pgfqpoint{1.978549in}{0.435692in}}%
\pgfpathlineto{\pgfqpoint{1.984391in}{0.435369in}}%
\pgfpathlineto{\pgfqpoint{1.990232in}{0.439187in}}%
\pgfpathlineto{\pgfqpoint{1.996074in}{0.447129in}}%
\pgfpathlineto{\pgfqpoint{2.001916in}{0.459166in}}%
\pgfpathlineto{\pgfqpoint{2.010678in}{0.484787in}}%
\pgfpathlineto{\pgfqpoint{2.019441in}{0.519282in}}%
\pgfpathlineto{\pgfqpoint{2.028204in}{0.562344in}}%
\pgfpathlineto{\pgfqpoint{2.039887in}{0.632415in}}%
\pgfpathlineto{\pgfqpoint{2.051570in}{0.715925in}}%
\pgfpathlineto{\pgfqpoint{2.066175in}{0.837185in}}%
\pgfpathlineto{\pgfqpoint{2.083700in}{1.003450in}}%
\pgfpathlineto{\pgfqpoint{2.107067in}{1.250550in}}%
\pgfpathlineto{\pgfqpoint{2.177167in}{2.014527in}}%
\pgfpathlineto{\pgfqpoint{2.194692in}{2.175004in}}%
\pgfpathlineto{\pgfqpoint{2.209296in}{2.290289in}}%
\pgfpathlineto{\pgfqpoint{2.220980in}{2.368369in}}%
\pgfpathlineto{\pgfqpoint{2.232663in}{2.432525in}}%
\pgfpathlineto{\pgfqpoint{2.241426in}{2.470888in}}%
\pgfpathlineto{\pgfqpoint{2.250188in}{2.500505in}}%
\pgfpathlineto{\pgfqpoint{2.258951in}{2.521114in}}%
\pgfpathlineto{\pgfqpoint{2.264793in}{2.529754in}}%
\pgfpathlineto{\pgfqpoint{2.270634in}{2.534274in}}%
\pgfpathlineto{\pgfqpoint{2.276476in}{2.534655in}}%
\pgfpathlineto{\pgfqpoint{2.282318in}{2.530896in}}%
\pgfpathlineto{\pgfqpoint{2.288160in}{2.523012in}}%
\pgfpathlineto{\pgfqpoint{2.294001in}{2.511033in}}%
\pgfpathlineto{\pgfqpoint{2.302764in}{2.485496in}}%
\pgfpathlineto{\pgfqpoint{2.311526in}{2.451084in}}%
\pgfpathlineto{\pgfqpoint{2.320289in}{2.408101in}}%
\pgfpathlineto{\pgfqpoint{2.331972in}{2.338129in}}%
\pgfpathlineto{\pgfqpoint{2.343656in}{2.254711in}}%
\pgfpathlineto{\pgfqpoint{2.358260in}{2.133550in}}%
\pgfpathlineto{\pgfqpoint{2.375785in}{1.967379in}}%
\pgfpathlineto{\pgfqpoint{2.399152in}{1.720360in}}%
\pgfpathlineto{\pgfqpoint{2.469253in}{0.956279in}}%
\pgfpathlineto{\pgfqpoint{2.486778in}{0.795700in}}%
\pgfpathlineto{\pgfqpoint{2.501382in}{0.680310in}}%
\pgfpathlineto{\pgfqpoint{2.513065in}{0.602136in}}%
\pgfpathlineto{\pgfqpoint{2.524749in}{0.537878in}}%
\pgfpathlineto{\pgfqpoint{2.533511in}{0.499434in}}%
\pgfpathlineto{\pgfqpoint{2.542274in}{0.469733in}}%
\pgfpathlineto{\pgfqpoint{2.551036in}{0.449038in}}%
\pgfpathlineto{\pgfqpoint{2.556878in}{0.440339in}}%
\pgfpathlineto{\pgfqpoint{2.562720in}{0.435761in}}%
\pgfpathlineto{\pgfqpoint{2.568562in}{0.435321in}}%
\pgfpathlineto{\pgfqpoint{2.574403in}{0.439022in}}%
\pgfpathlineto{\pgfqpoint{2.580245in}{0.446848in}}%
\pgfpathlineto{\pgfqpoint{2.586087in}{0.458769in}}%
\pgfpathlineto{\pgfqpoint{2.594849in}{0.484221in}}%
\pgfpathlineto{\pgfqpoint{2.603612in}{0.518551in}}%
\pgfpathlineto{\pgfqpoint{2.612374in}{0.561454in}}%
\pgfpathlineto{\pgfqpoint{2.624058in}{0.631327in}}%
\pgfpathlineto{\pgfqpoint{2.635741in}{0.714655in}}%
\pgfpathlineto{\pgfqpoint{2.650345in}{0.835717in}}%
\pgfpathlineto{\pgfqpoint{2.667871in}{1.001792in}}%
\pgfpathlineto{\pgfqpoint{2.691237in}{1.248730in}}%
\pgfpathlineto{\pgfqpoint{2.702921in}{1.378781in}}%
\pgfpathlineto{\pgfqpoint{2.702921in}{1.378781in}}%
\pgfusepath{stroke}%
\end{pgfscope}%
\begin{pgfscope}%
\pgfpathrectangle{\pgfqpoint{0.375000in}{0.330000in}}{\pgfqpoint{2.325000in}{2.310000in}}%
\pgfusepath{clip}%
\pgfsetbuttcap%
\pgfsetroundjoin%
\pgfsetlinewidth{1.505625pt}%
\definecolor{currentstroke}{rgb}{0.000000,0.000000,0.000000}%
\pgfsetstrokecolor{currentstroke}%
\pgfsetdash{{5.550000pt}{2.400000pt}}{0.000000pt}%
\pgfpathmoveto{\pgfqpoint{0.375000in}{1.485000in}}%
\pgfpathlineto{\pgfqpoint{0.433417in}{0.894794in}}%
\pgfpathlineto{\pgfqpoint{0.491834in}{0.523091in}}%
\pgfpathlineto{\pgfqpoint{0.550251in}{0.507501in}}%
\pgfpathlineto{\pgfqpoint{0.608668in}{0.853797in}}%
\pgfpathlineto{\pgfqpoint{0.667085in}{1.433774in}}%
\pgfpathlineto{\pgfqpoint{0.725503in}{2.032715in}}%
\pgfpathlineto{\pgfqpoint{0.783920in}{2.428884in}}%
\pgfpathlineto{\pgfqpoint{0.842337in}{2.475613in}}%
\pgfpathlineto{\pgfqpoint{0.900754in}{2.155601in}}%
\pgfpathlineto{\pgfqpoint{0.959171in}{1.587323in}}%
\pgfpathlineto{\pgfqpoint{1.017588in}{0.981163in}}%
\pgfpathlineto{\pgfqpoint{1.076005in}{0.561532in}}%
\pgfpathlineto{\pgfqpoint{1.134422in}{0.483782in}}%
\pgfpathlineto{\pgfqpoint{1.192839in}{0.776699in}}%
\pgfpathlineto{\pgfqpoint{1.251256in}{1.331840in}}%
\pgfpathlineto{\pgfqpoint{1.309673in}{1.943683in}}%
\pgfpathlineto{\pgfqpoint{1.368090in}{2.385714in}}%
\pgfpathlineto{\pgfqpoint{1.426508in}{2.494288in}}%
\pgfpathlineto{\pgfqpoint{1.484925in}{2.229207in}}%
\pgfpathlineto{\pgfqpoint{1.543342in}{1.688610in}}%
\pgfpathlineto{\pgfqpoint{1.601759in}{1.072633in}}%
\pgfpathlineto{\pgfqpoint{1.660176in}{0.609321in}}%
\pgfpathlineto{\pgfqpoint{1.718593in}{0.470198in}}%
\pgfpathlineto{\pgfqpoint{1.777010in}{0.706771in}}%
\pgfpathlineto{\pgfqpoint{1.835427in}{1.231456in}}%
\pgfpathlineto{\pgfqpoint{1.893844in}{1.850007in}}%
\pgfpathlineto{\pgfqpoint{1.952261in}{2.333427in}}%
\pgfpathlineto{\pgfqpoint{2.010678in}{2.502746in}}%
\pgfpathlineto{\pgfqpoint{2.069095in}{2.295280in}}%
\pgfpathlineto{\pgfqpoint{2.127513in}{1.787836in}}%
\pgfpathlineto{\pgfqpoint{2.185930in}{1.168277in}}%
\pgfpathlineto{\pgfqpoint{2.244347in}{0.665975in}}%
\pgfpathlineto{\pgfqpoint{2.302764in}{0.466887in}}%
\pgfpathlineto{\pgfqpoint{2.361181in}{0.644721in}}%
\pgfpathlineto{\pgfqpoint{2.419598in}{1.133639in}}%
\pgfpathlineto{\pgfqpoint{2.478015in}{1.752636in}}%
\pgfpathlineto{\pgfqpoint{2.536432in}{2.272550in}}%
\pgfpathlineto{\pgfqpoint{2.594849in}{2.500901in}}%
\pgfpathlineto{\pgfqpoint{2.653266in}{2.353150in}}%
\pgfpathlineto{\pgfqpoint{2.710000in}{1.897516in}}%
\pgfusepath{stroke}%
\end{pgfscope}%
\begin{pgfscope}%
\pgfsetrectcap%
\pgfsetmiterjoin%
\pgfsetlinewidth{0.803000pt}%
\definecolor{currentstroke}{rgb}{0.000000,0.000000,0.000000}%
\pgfsetstrokecolor{currentstroke}%
\pgfsetdash{}{0pt}%
\pgfpathmoveto{\pgfqpoint{0.375000in}{0.330000in}}%
\pgfpathlineto{\pgfqpoint{0.375000in}{2.640000in}}%
\pgfusepath{stroke}%
\end{pgfscope}%
\begin{pgfscope}%
\pgfsetrectcap%
\pgfsetmiterjoin%
\pgfsetlinewidth{0.803000pt}%
\definecolor{currentstroke}{rgb}{0.000000,0.000000,0.000000}%
\pgfsetstrokecolor{currentstroke}%
\pgfsetdash{}{0pt}%
\pgfpathmoveto{\pgfqpoint{2.700000in}{0.330000in}}%
\pgfpathlineto{\pgfqpoint{2.700000in}{2.640000in}}%
\pgfusepath{stroke}%
\end{pgfscope}%
\begin{pgfscope}%
\pgfsetrectcap%
\pgfsetmiterjoin%
\pgfsetlinewidth{0.803000pt}%
\definecolor{currentstroke}{rgb}{0.000000,0.000000,0.000000}%
\pgfsetstrokecolor{currentstroke}%
\pgfsetdash{}{0pt}%
\pgfpathmoveto{\pgfqpoint{0.375000in}{0.330000in}}%
\pgfpathlineto{\pgfqpoint{2.700000in}{0.330000in}}%
\pgfusepath{stroke}%
\end{pgfscope}%
\begin{pgfscope}%
\pgfsetrectcap%
\pgfsetmiterjoin%
\pgfsetlinewidth{0.803000pt}%
\definecolor{currentstroke}{rgb}{0.000000,0.000000,0.000000}%
\pgfsetstrokecolor{currentstroke}%
\pgfsetdash{}{0pt}%
\pgfpathmoveto{\pgfqpoint{0.375000in}{2.640000in}}%
\pgfpathlineto{\pgfqpoint{2.700000in}{2.640000in}}%
\pgfusepath{stroke}%
\end{pgfscope}%
\end{pgfpicture}%
\makeatother%
\endgroup%

  \end{subfigure}
    \begin{subfigure}{0.45\textwidth}
%% Creator: Matplotlib, PGF backend
%%
%% To include the figure in your LaTeX document, write
%%   \input{<filename>.pgf}
%%
%% Make sure the required packages are loaded in your preamble
%%   \usepackage{pgf}
%%
%% Figures using additional raster images can only be included by \input if
%% they are in the same directory as the main LaTeX file. For loading figures
%% from other directories you can use the `import` package
%%   \usepackage{import}
%% and then include the figures with
%%   \import{<path to file>}{<filename>.pgf}
%%
%% Matplotlib used the following preamble
%%   \usepackage{fontspec}
%%   \setmainfont{DejaVuSerif.ttf}[Path=/home/owen/progs/firedrake-complex/firedrake/lib/python3.5/site-packages/matplotlib/mpl-data/fonts/ttf/]
%%   \setsansfont{DejaVuSans.ttf}[Path=/home/owen/progs/firedrake-complex/firedrake/lib/python3.5/site-packages/matplotlib/mpl-data/fonts/ttf/]
%%   \setmonofont{DejaVuSansMono.ttf}[Path=/home/owen/progs/firedrake-complex/firedrake/lib/python3.5/site-packages/matplotlib/mpl-data/fonts/ttf/]
%%
\begingroup%
\makeatletter%
\begin{pgfpicture}%
\pgfpathrectangle{\pgfpointorigin}{\pgfqpoint{3.000000in}{3.000000in}}%
\pgfusepath{use as bounding box, clip}%
\begin{pgfscope}%
\pgfsetbuttcap%
\pgfsetmiterjoin%
\definecolor{currentfill}{rgb}{1.000000,1.000000,1.000000}%
\pgfsetfillcolor{currentfill}%
\pgfsetlinewidth{0.000000pt}%
\definecolor{currentstroke}{rgb}{1.000000,1.000000,1.000000}%
\pgfsetstrokecolor{currentstroke}%
\pgfsetdash{}{0pt}%
\pgfpathmoveto{\pgfqpoint{0.000000in}{0.000000in}}%
\pgfpathlineto{\pgfqpoint{3.000000in}{0.000000in}}%
\pgfpathlineto{\pgfqpoint{3.000000in}{3.000000in}}%
\pgfpathlineto{\pgfqpoint{0.000000in}{3.000000in}}%
\pgfpathclose%
\pgfusepath{fill}%
\end{pgfscope}%
\begin{pgfscope}%
\pgfsetbuttcap%
\pgfsetmiterjoin%
\definecolor{currentfill}{rgb}{1.000000,1.000000,1.000000}%
\pgfsetfillcolor{currentfill}%
\pgfsetlinewidth{0.000000pt}%
\definecolor{currentstroke}{rgb}{0.000000,0.000000,0.000000}%
\pgfsetstrokecolor{currentstroke}%
\pgfsetstrokeopacity{0.000000}%
\pgfsetdash{}{0pt}%
\pgfpathmoveto{\pgfqpoint{0.375000in}{0.330000in}}%
\pgfpathlineto{\pgfqpoint{2.700000in}{0.330000in}}%
\pgfpathlineto{\pgfqpoint{2.700000in}{2.640000in}}%
\pgfpathlineto{\pgfqpoint{0.375000in}{2.640000in}}%
\pgfpathclose%
\pgfusepath{fill}%
\end{pgfscope}%
\begin{pgfscope}%
\pgfsetbuttcap%
\pgfsetroundjoin%
\definecolor{currentfill}{rgb}{0.000000,0.000000,0.000000}%
\pgfsetfillcolor{currentfill}%
\pgfsetlinewidth{0.803000pt}%
\definecolor{currentstroke}{rgb}{0.000000,0.000000,0.000000}%
\pgfsetstrokecolor{currentstroke}%
\pgfsetdash{}{0pt}%
\pgfsys@defobject{currentmarker}{\pgfqpoint{0.000000in}{-0.048611in}}{\pgfqpoint{0.000000in}{0.000000in}}{%
\pgfpathmoveto{\pgfqpoint{0.000000in}{0.000000in}}%
\pgfpathlineto{\pgfqpoint{0.000000in}{-0.048611in}}%
\pgfusepath{stroke,fill}%
}%
\begin{pgfscope}%
\pgfsys@transformshift{0.840000in}{0.330000in}%
\pgfsys@useobject{currentmarker}{}%
\end{pgfscope}%
\end{pgfscope}%
\begin{pgfscope}%
\definecolor{textcolor}{rgb}{0.000000,0.000000,0.000000}%
\pgfsetstrokecolor{textcolor}%
\pgfsetfillcolor{textcolor}%
\pgftext[x=0.840000in,y=0.232778in,,top]{\color{textcolor}\sffamily\fontsize{10.000000}{12.000000}\selectfont \(\displaystyle 0.98\)}%
\end{pgfscope}%
\begin{pgfscope}%
\pgfsetbuttcap%
\pgfsetroundjoin%
\definecolor{currentfill}{rgb}{0.000000,0.000000,0.000000}%
\pgfsetfillcolor{currentfill}%
\pgfsetlinewidth{0.803000pt}%
\definecolor{currentstroke}{rgb}{0.000000,0.000000,0.000000}%
\pgfsetstrokecolor{currentstroke}%
\pgfsetdash{}{0pt}%
\pgfsys@defobject{currentmarker}{\pgfqpoint{0.000000in}{-0.048611in}}{\pgfqpoint{0.000000in}{0.000000in}}{%
\pgfpathmoveto{\pgfqpoint{0.000000in}{0.000000in}}%
\pgfpathlineto{\pgfqpoint{0.000000in}{-0.048611in}}%
\pgfusepath{stroke,fill}%
}%
\begin{pgfscope}%
\pgfsys@transformshift{1.770000in}{0.330000in}%
\pgfsys@useobject{currentmarker}{}%
\end{pgfscope}%
\end{pgfscope}%
\begin{pgfscope}%
\definecolor{textcolor}{rgb}{0.000000,0.000000,0.000000}%
\pgfsetstrokecolor{textcolor}%
\pgfsetfillcolor{textcolor}%
\pgftext[x=1.770000in,y=0.232778in,,top]{\color{textcolor}\sffamily\fontsize{10.000000}{12.000000}\selectfont \(\displaystyle 0.99\)}%
\end{pgfscope}%
\begin{pgfscope}%
\pgfsetbuttcap%
\pgfsetroundjoin%
\definecolor{currentfill}{rgb}{0.000000,0.000000,0.000000}%
\pgfsetfillcolor{currentfill}%
\pgfsetlinewidth{0.803000pt}%
\definecolor{currentstroke}{rgb}{0.000000,0.000000,0.000000}%
\pgfsetstrokecolor{currentstroke}%
\pgfsetdash{}{0pt}%
\pgfsys@defobject{currentmarker}{\pgfqpoint{0.000000in}{-0.048611in}}{\pgfqpoint{0.000000in}{0.000000in}}{%
\pgfpathmoveto{\pgfqpoint{0.000000in}{0.000000in}}%
\pgfpathlineto{\pgfqpoint{0.000000in}{-0.048611in}}%
\pgfusepath{stroke,fill}%
}%
\begin{pgfscope}%
\pgfsys@transformshift{2.700000in}{0.330000in}%
\pgfsys@useobject{currentmarker}{}%
\end{pgfscope}%
\end{pgfscope}%
\begin{pgfscope}%
\definecolor{textcolor}{rgb}{0.000000,0.000000,0.000000}%
\pgfsetstrokecolor{textcolor}%
\pgfsetfillcolor{textcolor}%
\pgftext[x=2.700000in,y=0.232778in,,top]{\color{textcolor}\sffamily\fontsize{10.000000}{12.000000}\selectfont \(\displaystyle 1.00\)}%
\end{pgfscope}%
\begin{pgfscope}%
\pgfsetbuttcap%
\pgfsetroundjoin%
\definecolor{currentfill}{rgb}{0.000000,0.000000,0.000000}%
\pgfsetfillcolor{currentfill}%
\pgfsetlinewidth{0.803000pt}%
\definecolor{currentstroke}{rgb}{0.000000,0.000000,0.000000}%
\pgfsetstrokecolor{currentstroke}%
\pgfsetdash{}{0pt}%
\pgfsys@defobject{currentmarker}{\pgfqpoint{-0.048611in}{0.000000in}}{\pgfqpoint{0.000000in}{0.000000in}}{%
\pgfpathmoveto{\pgfqpoint{0.000000in}{0.000000in}}%
\pgfpathlineto{\pgfqpoint{-0.048611in}{0.000000in}}%
\pgfusepath{stroke,fill}%
}%
\begin{pgfscope}%
\pgfsys@transformshift{0.375000in}{0.435000in}%
\pgfsys@useobject{currentmarker}{}%
\end{pgfscope}%
\end{pgfscope}%
\begin{pgfscope}%
\definecolor{textcolor}{rgb}{0.000000,0.000000,0.000000}%
\pgfsetstrokecolor{textcolor}%
\pgfsetfillcolor{textcolor}%
\pgftext[x=0.100308in,y=0.382238in,left,base]{\color{textcolor}\sffamily\fontsize{10.000000}{12.000000}\selectfont \(\displaystyle -1\)}%
\end{pgfscope}%
\begin{pgfscope}%
\pgfsetbuttcap%
\pgfsetroundjoin%
\definecolor{currentfill}{rgb}{0.000000,0.000000,0.000000}%
\pgfsetfillcolor{currentfill}%
\pgfsetlinewidth{0.803000pt}%
\definecolor{currentstroke}{rgb}{0.000000,0.000000,0.000000}%
\pgfsetstrokecolor{currentstroke}%
\pgfsetdash{}{0pt}%
\pgfsys@defobject{currentmarker}{\pgfqpoint{-0.048611in}{0.000000in}}{\pgfqpoint{0.000000in}{0.000000in}}{%
\pgfpathmoveto{\pgfqpoint{0.000000in}{0.000000in}}%
\pgfpathlineto{\pgfqpoint{-0.048611in}{0.000000in}}%
\pgfusepath{stroke,fill}%
}%
\begin{pgfscope}%
\pgfsys@transformshift{0.375000in}{1.485000in}%
\pgfsys@useobject{currentmarker}{}%
\end{pgfscope}%
\end{pgfscope}%
\begin{pgfscope}%
\definecolor{textcolor}{rgb}{0.000000,0.000000,0.000000}%
\pgfsetstrokecolor{textcolor}%
\pgfsetfillcolor{textcolor}%
\pgftext[x=0.208333in,y=1.432238in,left,base]{\color{textcolor}\sffamily\fontsize{10.000000}{12.000000}\selectfont \(\displaystyle 0\)}%
\end{pgfscope}%
\begin{pgfscope}%
\pgfsetbuttcap%
\pgfsetroundjoin%
\definecolor{currentfill}{rgb}{0.000000,0.000000,0.000000}%
\pgfsetfillcolor{currentfill}%
\pgfsetlinewidth{0.803000pt}%
\definecolor{currentstroke}{rgb}{0.000000,0.000000,0.000000}%
\pgfsetstrokecolor{currentstroke}%
\pgfsetdash{}{0pt}%
\pgfsys@defobject{currentmarker}{\pgfqpoint{-0.048611in}{0.000000in}}{\pgfqpoint{0.000000in}{0.000000in}}{%
\pgfpathmoveto{\pgfqpoint{0.000000in}{0.000000in}}%
\pgfpathlineto{\pgfqpoint{-0.048611in}{0.000000in}}%
\pgfusepath{stroke,fill}%
}%
\begin{pgfscope}%
\pgfsys@transformshift{0.375000in}{2.535000in}%
\pgfsys@useobject{currentmarker}{}%
\end{pgfscope}%
\end{pgfscope}%
\begin{pgfscope}%
\definecolor{textcolor}{rgb}{0.000000,0.000000,0.000000}%
\pgfsetstrokecolor{textcolor}%
\pgfsetfillcolor{textcolor}%
\pgftext[x=0.208333in,y=2.482238in,left,base]{\color{textcolor}\sffamily\fontsize{10.000000}{12.000000}\selectfont \(\displaystyle 1\)}%
\end{pgfscope}%
\begin{pgfscope}%
\pgfpathrectangle{\pgfqpoint{0.375000in}{0.330000in}}{\pgfqpoint{2.325000in}{2.310000in}}%
\pgfusepath{clip}%
\pgfsetrectcap%
\pgfsetroundjoin%
\pgfsetlinewidth{1.505625pt}%
\definecolor{currentstroke}{rgb}{0.000000,0.000000,0.000000}%
\pgfsetstrokecolor{currentstroke}%
\pgfsetdash{}{0pt}%
\pgfpathmoveto{\pgfqpoint{0.372079in}{2.408505in}}%
\pgfpathlineto{\pgfqpoint{0.383763in}{2.463831in}}%
\pgfpathlineto{\pgfqpoint{0.392525in}{2.495239in}}%
\pgfpathlineto{\pgfqpoint{0.401288in}{2.517685in}}%
\pgfpathlineto{\pgfqpoint{0.407129in}{2.527569in}}%
\pgfpathlineto{\pgfqpoint{0.412971in}{2.533341in}}%
\pgfpathlineto{\pgfqpoint{0.418813in}{2.534977in}}%
\pgfpathlineto{\pgfqpoint{0.424655in}{2.532472in}}%
\pgfpathlineto{\pgfqpoint{0.430496in}{2.525836in}}%
\pgfpathlineto{\pgfqpoint{0.436338in}{2.515094in}}%
\pgfpathlineto{\pgfqpoint{0.442180in}{2.500289in}}%
\pgfpathlineto{\pgfqpoint{0.450942in}{2.470596in}}%
\pgfpathlineto{\pgfqpoint{0.459705in}{2.432160in}}%
\pgfpathlineto{\pgfqpoint{0.471388in}{2.367910in}}%
\pgfpathlineto{\pgfqpoint{0.483072in}{2.289745in}}%
\pgfpathlineto{\pgfqpoint{0.497676in}{2.174365in}}%
\pgfpathlineto{\pgfqpoint{0.515201in}{2.013795in}}%
\pgfpathlineto{\pgfqpoint{0.535647in}{1.803239in}}%
\pgfpathlineto{\pgfqpoint{0.570697in}{1.412653in}}%
\pgfpathlineto{\pgfqpoint{0.602827in}{1.062195in}}%
\pgfpathlineto{\pgfqpoint{0.623273in}{0.862770in}}%
\pgfpathlineto{\pgfqpoint{0.637877in}{0.738156in}}%
\pgfpathlineto{\pgfqpoint{0.652481in}{0.631921in}}%
\pgfpathlineto{\pgfqpoint{0.664165in}{0.561940in}}%
\pgfpathlineto{\pgfqpoint{0.675848in}{0.506508in}}%
\pgfpathlineto{\pgfqpoint{0.684611in}{0.475016in}}%
\pgfpathlineto{\pgfqpoint{0.693373in}{0.452484in}}%
\pgfpathlineto{\pgfqpoint{0.699215in}{0.442542in}}%
\pgfpathlineto{\pgfqpoint{0.705057in}{0.436712in}}%
\pgfpathlineto{\pgfqpoint{0.710898in}{0.435017in}}%
\pgfpathlineto{\pgfqpoint{0.716740in}{0.437463in}}%
\pgfpathlineto{\pgfqpoint{0.722582in}{0.444041in}}%
\pgfpathlineto{\pgfqpoint{0.728423in}{0.454725in}}%
\pgfpathlineto{\pgfqpoint{0.734265in}{0.469473in}}%
\pgfpathlineto{\pgfqpoint{0.743028in}{0.499082in}}%
\pgfpathlineto{\pgfqpoint{0.751790in}{0.537438in}}%
\pgfpathlineto{\pgfqpoint{0.763474in}{0.601585in}}%
\pgfpathlineto{\pgfqpoint{0.775157in}{0.679656in}}%
\pgfpathlineto{\pgfqpoint{0.789761in}{0.794931in}}%
\pgfpathlineto{\pgfqpoint{0.807286in}{0.955398in}}%
\pgfpathlineto{\pgfqpoint{0.827732in}{1.165871in}}%
\pgfpathlineto{\pgfqpoint{0.862783in}{1.556415in}}%
\pgfpathlineto{\pgfqpoint{0.894912in}{1.906950in}}%
\pgfpathlineto{\pgfqpoint{0.915358in}{2.106478in}}%
\pgfpathlineto{\pgfqpoint{0.929962in}{2.231188in}}%
\pgfpathlineto{\pgfqpoint{0.944567in}{2.337535in}}%
\pgfpathlineto{\pgfqpoint{0.956250in}{2.407615in}}%
\pgfpathlineto{\pgfqpoint{0.967933in}{2.463153in}}%
\pgfpathlineto{\pgfqpoint{0.976696in}{2.494728in}}%
\pgfpathlineto{\pgfqpoint{0.985459in}{2.517346in}}%
\pgfpathlineto{\pgfqpoint{0.991300in}{2.527346in}}%
\pgfpathlineto{\pgfqpoint{0.997142in}{2.533234in}}%
\pgfpathlineto{\pgfqpoint{1.002984in}{2.534988in}}%
\pgfpathlineto{\pgfqpoint{1.008825in}{2.532600in}}%
\pgfpathlineto{\pgfqpoint{1.014667in}{2.526080in}}%
\pgfpathlineto{\pgfqpoint{1.020509in}{2.515454in}}%
\pgfpathlineto{\pgfqpoint{1.026351in}{2.500764in}}%
\pgfpathlineto{\pgfqpoint{1.035113in}{2.471238in}}%
\pgfpathlineto{\pgfqpoint{1.043876in}{2.432964in}}%
\pgfpathlineto{\pgfqpoint{1.055559in}{2.368920in}}%
\pgfpathlineto{\pgfqpoint{1.067242in}{2.290943in}}%
\pgfpathlineto{\pgfqpoint{1.081847in}{2.175772in}}%
\pgfpathlineto{\pgfqpoint{1.099372in}{2.015408in}}%
\pgfpathlineto{\pgfqpoint{1.119818in}{1.805018in}}%
\pgfpathlineto{\pgfqpoint{1.154868in}{1.414516in}}%
\pgfpathlineto{\pgfqpoint{1.186997in}{1.063905in}}%
\pgfpathlineto{\pgfqpoint{1.207443in}{0.864275in}}%
\pgfpathlineto{\pgfqpoint{1.222048in}{0.739469in}}%
\pgfpathlineto{\pgfqpoint{1.236652in}{0.633011in}}%
\pgfpathlineto{\pgfqpoint{1.248335in}{0.562831in}}%
\pgfpathlineto{\pgfqpoint{1.260019in}{0.507186in}}%
\pgfpathlineto{\pgfqpoint{1.268781in}{0.475528in}}%
\pgfpathlineto{\pgfqpoint{1.277544in}{0.452825in}}%
\pgfpathlineto{\pgfqpoint{1.283386in}{0.442767in}}%
\pgfpathlineto{\pgfqpoint{1.289227in}{0.436820in}}%
\pgfpathlineto{\pgfqpoint{1.295069in}{0.435008in}}%
\pgfpathlineto{\pgfqpoint{1.300911in}{0.437337in}}%
\pgfpathlineto{\pgfqpoint{1.306753in}{0.443798in}}%
\pgfpathlineto{\pgfqpoint{1.312594in}{0.454367in}}%
\pgfpathlineto{\pgfqpoint{1.318436in}{0.469000in}}%
\pgfpathlineto{\pgfqpoint{1.327198in}{0.498442in}}%
\pgfpathlineto{\pgfqpoint{1.335961in}{0.536635in}}%
\pgfpathlineto{\pgfqpoint{1.347644in}{0.600577in}}%
\pgfpathlineto{\pgfqpoint{1.359328in}{0.678459in}}%
\pgfpathlineto{\pgfqpoint{1.373932in}{0.793525in}}%
\pgfpathlineto{\pgfqpoint{1.391457in}{0.953787in}}%
\pgfpathlineto{\pgfqpoint{1.411903in}{1.164093in}}%
\pgfpathlineto{\pgfqpoint{1.446954in}{1.554552in}}%
\pgfpathlineto{\pgfqpoint{1.479083in}{1.905240in}}%
\pgfpathlineto{\pgfqpoint{1.499529in}{2.104972in}}%
\pgfpathlineto{\pgfqpoint{1.517054in}{2.252744in}}%
\pgfpathlineto{\pgfqpoint{1.531658in}{2.355318in}}%
\pgfpathlineto{\pgfqpoint{1.543342in}{2.422060in}}%
\pgfpathlineto{\pgfqpoint{1.552104in}{2.462473in}}%
\pgfpathlineto{\pgfqpoint{1.560867in}{2.494215in}}%
\pgfpathlineto{\pgfqpoint{1.569629in}{2.517003in}}%
\pgfpathlineto{\pgfqpoint{1.575471in}{2.527119in}}%
\pgfpathlineto{\pgfqpoint{1.581313in}{2.533124in}}%
\pgfpathlineto{\pgfqpoint{1.587155in}{2.534995in}}%
\pgfpathlineto{\pgfqpoint{1.592996in}{2.532725in}}%
\pgfpathlineto{\pgfqpoint{1.598838in}{2.526322in}}%
\pgfpathlineto{\pgfqpoint{1.604680in}{2.515811in}}%
\pgfpathlineto{\pgfqpoint{1.610521in}{2.501235in}}%
\pgfpathlineto{\pgfqpoint{1.619284in}{2.471878in}}%
\pgfpathlineto{\pgfqpoint{1.628046in}{2.433765in}}%
\pgfpathlineto{\pgfqpoint{1.639730in}{2.369926in}}%
\pgfpathlineto{\pgfqpoint{1.651413in}{2.292139in}}%
\pgfpathlineto{\pgfqpoint{1.666018in}{2.177177in}}%
\pgfpathlineto{\pgfqpoint{1.683543in}{2.017018in}}%
\pgfpathlineto{\pgfqpoint{1.703989in}{1.806796in}}%
\pgfpathlineto{\pgfqpoint{1.739039in}{1.416379in}}%
\pgfpathlineto{\pgfqpoint{1.771168in}{1.065616in}}%
\pgfpathlineto{\pgfqpoint{1.791614in}{0.865782in}}%
\pgfpathlineto{\pgfqpoint{1.809139in}{0.717893in}}%
\pgfpathlineto{\pgfqpoint{1.823744in}{0.615204in}}%
\pgfpathlineto{\pgfqpoint{1.835427in}{0.548361in}}%
\pgfpathlineto{\pgfqpoint{1.844190in}{0.507868in}}%
\pgfpathlineto{\pgfqpoint{1.852952in}{0.476044in}}%
\pgfpathlineto{\pgfqpoint{1.861715in}{0.453169in}}%
\pgfpathlineto{\pgfqpoint{1.867557in}{0.442996in}}%
\pgfpathlineto{\pgfqpoint{1.873398in}{0.436932in}}%
\pgfpathlineto{\pgfqpoint{1.879240in}{0.435002in}}%
\pgfpathlineto{\pgfqpoint{1.885082in}{0.437214in}}%
\pgfpathlineto{\pgfqpoint{1.890923in}{0.443559in}}%
\pgfpathlineto{\pgfqpoint{1.896765in}{0.454011in}}%
\pgfpathlineto{\pgfqpoint{1.902607in}{0.468530in}}%
\pgfpathlineto{\pgfqpoint{1.911369in}{0.497804in}}%
\pgfpathlineto{\pgfqpoint{1.920132in}{0.535835in}}%
\pgfpathlineto{\pgfqpoint{1.931815in}{0.599572in}}%
\pgfpathlineto{\pgfqpoint{1.943499in}{0.677265in}}%
\pgfpathlineto{\pgfqpoint{1.958103in}{0.792121in}}%
\pgfpathlineto{\pgfqpoint{1.975628in}{0.952177in}}%
\pgfpathlineto{\pgfqpoint{1.996074in}{1.162315in}}%
\pgfpathlineto{\pgfqpoint{2.031124in}{1.552689in}}%
\pgfpathlineto{\pgfqpoint{2.063254in}{1.903528in}}%
\pgfpathlineto{\pgfqpoint{2.083700in}{2.103464in}}%
\pgfpathlineto{\pgfqpoint{2.101225in}{2.251469in}}%
\pgfpathlineto{\pgfqpoint{2.115829in}{2.354272in}}%
\pgfpathlineto{\pgfqpoint{2.127513in}{2.421216in}}%
\pgfpathlineto{\pgfqpoint{2.136275in}{2.461790in}}%
\pgfpathlineto{\pgfqpoint{2.145038in}{2.493698in}}%
\pgfpathlineto{\pgfqpoint{2.153800in}{2.516658in}}%
\pgfpathlineto{\pgfqpoint{2.159642in}{2.526889in}}%
\pgfpathlineto{\pgfqpoint{2.165484in}{2.533011in}}%
\pgfpathlineto{\pgfqpoint{2.171325in}{2.534999in}}%
\pgfpathlineto{\pgfqpoint{2.177167in}{2.532846in}}%
\pgfpathlineto{\pgfqpoint{2.183009in}{2.526560in}}%
\pgfpathlineto{\pgfqpoint{2.188851in}{2.516165in}}%
\pgfpathlineto{\pgfqpoint{2.194692in}{2.501703in}}%
\pgfpathlineto{\pgfqpoint{2.203455in}{2.472514in}}%
\pgfpathlineto{\pgfqpoint{2.212217in}{2.434564in}}%
\pgfpathlineto{\pgfqpoint{2.223901in}{2.370930in}}%
\pgfpathlineto{\pgfqpoint{2.235584in}{2.293332in}}%
\pgfpathlineto{\pgfqpoint{2.250188in}{2.178580in}}%
\pgfpathlineto{\pgfqpoint{2.267714in}{2.018627in}}%
\pgfpathlineto{\pgfqpoint{2.288160in}{1.808573in}}%
\pgfpathlineto{\pgfqpoint{2.323210in}{1.418243in}}%
\pgfpathlineto{\pgfqpoint{2.355339in}{1.067328in}}%
\pgfpathlineto{\pgfqpoint{2.375785in}{0.867291in}}%
\pgfpathlineto{\pgfqpoint{2.393310in}{0.719169in}}%
\pgfpathlineto{\pgfqpoint{2.407915in}{0.616252in}}%
\pgfpathlineto{\pgfqpoint{2.419598in}{0.549207in}}%
\pgfpathlineto{\pgfqpoint{2.428361in}{0.508553in}}%
\pgfpathlineto{\pgfqpoint{2.437123in}{0.476562in}}%
\pgfpathlineto{\pgfqpoint{2.445886in}{0.453517in}}%
\pgfpathlineto{\pgfqpoint{2.451727in}{0.443227in}}%
\pgfpathlineto{\pgfqpoint{2.457569in}{0.437047in}}%
\pgfpathlineto{\pgfqpoint{2.463411in}{0.435000in}}%
\pgfpathlineto{\pgfqpoint{2.469253in}{0.437095in}}%
\pgfpathlineto{\pgfqpoint{2.475094in}{0.443323in}}%
\pgfpathlineto{\pgfqpoint{2.480936in}{0.453659in}}%
\pgfpathlineto{\pgfqpoint{2.486778in}{0.468064in}}%
\pgfpathlineto{\pgfqpoint{2.495540in}{0.497169in}}%
\pgfpathlineto{\pgfqpoint{2.504303in}{0.535038in}}%
\pgfpathlineto{\pgfqpoint{2.515986in}{0.598570in}}%
\pgfpathlineto{\pgfqpoint{2.527670in}{0.676073in}}%
\pgfpathlineto{\pgfqpoint{2.542274in}{0.790719in}}%
\pgfpathlineto{\pgfqpoint{2.559799in}{0.950569in}}%
\pgfpathlineto{\pgfqpoint{2.580245in}{1.160539in}}%
\pgfpathlineto{\pgfqpoint{2.612374in}{1.517886in}}%
\pgfpathlineto{\pgfqpoint{2.647425in}{1.901815in}}%
\pgfpathlineto{\pgfqpoint{2.667871in}{2.101954in}}%
\pgfpathlineto{\pgfqpoint{2.685396in}{2.250192in}}%
\pgfpathlineto{\pgfqpoint{2.700000in}{2.353224in}}%
\pgfpathlineto{\pgfqpoint{2.700000in}{2.353224in}}%
\pgfusepath{stroke}%
\end{pgfscope}%
\begin{pgfscope}%
\pgfpathrectangle{\pgfqpoint{0.375000in}{0.330000in}}{\pgfqpoint{2.325000in}{2.310000in}}%
\pgfusepath{clip}%
\pgfsetbuttcap%
\pgfsetroundjoin%
\pgfsetlinewidth{1.505625pt}%
\definecolor{currentstroke}{rgb}{0.000000,0.000000,0.000000}%
\pgfsetstrokecolor{currentstroke}%
\pgfsetdash{{5.550000pt}{2.400000pt}}{0.000000pt}%
\pgfpathmoveto{\pgfqpoint{0.365000in}{2.475472in}}%
\pgfpathlineto{\pgfqpoint{0.421734in}{2.426728in}}%
\pgfpathlineto{\pgfqpoint{0.480151in}{2.027896in}}%
\pgfpathlineto{\pgfqpoint{0.538568in}{1.428076in}}%
\pgfpathlineto{\pgfqpoint{0.596985in}{0.849329in}}%
\pgfpathlineto{\pgfqpoint{0.655402in}{0.505918in}}%
\pgfpathlineto{\pgfqpoint{0.713819in}{0.524978in}}%
\pgfpathlineto{\pgfqpoint{0.772236in}{0.899453in}}%
\pgfpathlineto{\pgfqpoint{0.830653in}{1.490706in}}%
\pgfpathlineto{\pgfqpoint{0.889070in}{2.079847in}}%
\pgfpathlineto{\pgfqpoint{0.947487in}{2.448766in}}%
\pgfpathlineto{\pgfqpoint{1.005905in}{2.460885in}}%
\pgfpathlineto{\pgfqpoint{1.064322in}{2.111715in}}%
\pgfpathlineto{\pgfqpoint{1.122739in}{1.530527in}}%
\pgfpathlineto{\pgfqpoint{1.181156in}{0.932483in}}%
\pgfpathlineto{\pgfqpoint{1.239573in}{0.538990in}}%
\pgfpathlineto{\pgfqpoint{1.297990in}{0.495724in}}%
\pgfpathlineto{\pgfqpoint{1.356407in}{0.818703in}}%
\pgfpathlineto{\pgfqpoint{1.414824in}{1.388356in}}%
\pgfpathlineto{\pgfqpoint{1.473241in}{1.993788in}}%
\pgfpathlineto{\pgfqpoint{1.531658in}{2.410858in}}%
\pgfpathlineto{\pgfqpoint{1.590075in}{2.485162in}}%
\pgfpathlineto{\pgfqpoint{1.648492in}{2.189191in}}%
\pgfpathlineto{\pgfqpoint{1.706910in}{1.632517in}}%
\pgfpathlineto{\pgfqpoint{1.765327in}{1.021230in}}%
\pgfpathlineto{\pgfqpoint{1.823744in}{0.581638in}}%
\pgfpathlineto{\pgfqpoint{1.882161in}{0.476485in}}%
\pgfpathlineto{\pgfqpoint{1.940578in}{0.744699in}}%
\pgfpathlineto{\pgfqpoint{1.998995in}{1.286984in}}%
\pgfpathlineto{\pgfqpoint{2.057412in}{1.902578in}}%
\pgfpathlineto{\pgfqpoint{2.115829in}{2.363578in}}%
\pgfpathlineto{\pgfqpoint{2.174246in}{2.499315in}}%
\pgfpathlineto{\pgfqpoint{2.232663in}{2.259537in}}%
\pgfpathlineto{\pgfqpoint{2.291080in}{1.733014in}}%
\pgfpathlineto{\pgfqpoint{2.349497in}{1.114672in}}%
\pgfpathlineto{\pgfqpoint{2.407915in}{0.633431in}}%
\pgfpathlineto{\pgfqpoint{2.466332in}{0.467455in}}%
\pgfpathlineto{\pgfqpoint{2.524749in}{0.678189in}}%
\pgfpathlineto{\pgfqpoint{2.583166in}{1.187617in}}%
\pgfpathlineto{\pgfqpoint{2.641583in}{1.807141in}}%
\pgfpathlineto{\pgfqpoint{2.700000in}{2.307403in}}%
\pgfusepath{stroke}%
\end{pgfscope}%
\begin{pgfscope}%
\pgfsetrectcap%
\pgfsetmiterjoin%
\pgfsetlinewidth{0.803000pt}%
\definecolor{currentstroke}{rgb}{0.000000,0.000000,0.000000}%
\pgfsetstrokecolor{currentstroke}%
\pgfsetdash{}{0pt}%
\pgfpathmoveto{\pgfqpoint{0.375000in}{0.330000in}}%
\pgfpathlineto{\pgfqpoint{0.375000in}{2.640000in}}%
\pgfusepath{stroke}%
\end{pgfscope}%
\begin{pgfscope}%
\pgfsetrectcap%
\pgfsetmiterjoin%
\pgfsetlinewidth{0.000000pt}%
\definecolor{currentstroke}{rgb}{0.000000,0.000000,0.000000}%
\pgfsetstrokecolor{currentstroke}%
\pgfsetstrokeopacity{0.000000}%
\pgfsetdash{}{0pt}%
\pgfpathmoveto{\pgfqpoint{2.700000in}{0.330000in}}%
\pgfpathlineto{\pgfqpoint{2.700000in}{2.640000in}}%
\pgfusepath{}%
\end{pgfscope}%
\begin{pgfscope}%
\pgfsetrectcap%
\pgfsetmiterjoin%
\pgfsetlinewidth{0.803000pt}%
\definecolor{currentstroke}{rgb}{0.000000,0.000000,0.000000}%
\pgfsetstrokecolor{currentstroke}%
\pgfsetdash{}{0pt}%
\pgfpathmoveto{\pgfqpoint{0.375000in}{0.330000in}}%
\pgfpathlineto{\pgfqpoint{2.700000in}{0.330000in}}%
\pgfusepath{stroke}%
\end{pgfscope}%
\begin{pgfscope}%
\pgfsetrectcap%
\pgfsetmiterjoin%
\pgfsetlinewidth{0.000000pt}%
\definecolor{currentstroke}{rgb}{0.000000,0.000000,0.000000}%
\pgfsetstrokecolor{currentstroke}%
\pgfsetstrokeopacity{0.000000}%
\pgfsetdash{}{0pt}%
\pgfpathmoveto{\pgfqpoint{0.375000in}{2.640000in}}%
\pgfpathlineto{\pgfqpoint{2.700000in}{2.640000in}}%
\pgfusepath{}%
\end{pgfscope}%
\end{pgfpicture}%
\makeatother%
\endgroup%

    \end{subfigure}
\caption[The pollution effect for the 1-d Helmholtz equation.]{\label{fig:intropoll} The pollution effect for the 1-d Helmholtz equation $u'' + k^2u = 0$ for $k=1000$, with a zero Dirichlet boundary condition on the left endpoint, and an impedance boundary condition on the right endpoint chosen so that the exact solution $u = \sin(kx)$. The solid line denotes the true solution, the dashed line denotes the finite-element approximation with 10 points per wavelength.}
\end{figure}


In summary, numerically solving the Helmholtz equation gives rise to large (size at least $ \sim k^d$) linear systems, and the size of these linear systems increases as $k$ increases. The table \cref{tab:introlinsys} presents the sizes of the linear systems one obtains for different values of $k$, depending on the spatial dimension, and the properties of the finite-element solution that one requires.
\begin{table}
  \centering
  \begin{tabular}{Sc Sc Sc Sc Sc Sc Sc}
    \toprule
 &\multicolumn{2}{c}{Interpolation error bounded}&\multicolumn{2}{c}{Finite-element error bounded}&\multicolumn{2}{c}{Quasi-optimality}\\
    &\multicolumn{2}{c}{($h = \pi/5 \times k^{-1}$)}&\multicolumn{2}{c}{($h = k^{-3/2}$)}&\multicolumn{2}{c}{($h = k^{-2}$)}\\
    \midrule
&2-D&3-D&2-D&3-D&2-D&3-D\\
\hline
$k$&&&&&&\\
$10$&$\approx 2.5\times 10^2$&$\approx 4 \times 10^3$&$10^3$&$\approx 3 \times 10^4$&$10^4$&$10^6$\\
$100$&$\approx 2.5\times 10^4$&$\approx 4 \times 10^6$&$10^6$&$10^9$&$10^8$&$10^{12}$\\
$1000$&$\approx 2.5\times 10^6$&$\approx 4 \times 10^9$&$10^9$&$\approx 3 \times 10^{13}$&$10^{12}$&$10^{18}$\\
\bottomrule
\end{tabular}
\caption[The number of degrees of freedom required to obtain various properties of finite-element approximations of the solution of the Helmholtz equation.]{\label{tab:introlinsys}The number of degrees of freedom that would be required to obtain various properties of finite-element approximations of the solution $u$ of \eqref{eq:introdet}, for various values of $k$, in 2-D and 3-D. All errors etc. would be measured in the weighted $H^1$ norm $\mleft(\SN{\cdot}_{H^1}^2 + k^2 \N{\cdot}_{L^2}^2\mright)^{1/2}.$ See \cref{sec:helmfe} for discussion of why one chooses the mesh conditions used here.}
\end{table}


\paragraph{Issues from linear solvers} We now turn our attention to how one might solve the large, indefinite, non-Hermitian linear systems \eqref{eq:intromat}. One option is to solve the linear systems \eqref{eq:intromat} using a direct solver (solvers that, up to machine precision, invert the linear system \eqref{eq:intromat} exactly, see, e.g., \cite{DuErRe:17}). Such solvers are incredibly competitive for solving \eqref{eq:introdet} in 2-D, if \eqref{eq:intromat} has up to $10^6$ unknowns; however, for larger linear systems \eqref{eq:intromat}, such as those obtained from 3-D discretisations, direct solvers are not as competitive as so-called iterative solvers, see, e.g., \cite[p. 70]{ElSiWa:14}. An iterative solver is one that does not solve \eqref{eq:intromat} exactly, but rather produces a sequence of approximations to the solution of \eqref{eq:intromat}. A standard iterative solver to use for non-Hermitian linear systems is GMRES; this is the solver we will use throughout this thesis. However, as seen in \cref{fig:GMRES}, GMRES applied to \eqref{eq:intromat} can perform very badly (the number of interations to acheive convergence can grow dramatically with $k$, and moreover, one cannot apply the standard convergence results for GMRES (originally presented in \cite{El:82} and given in a helpful form in \cite[Section 1]{BeGoTy:06}) to \eqref{eq:intromat} because the matrices $\Amat$ are typically indefinite. An explanation of how the wave-nature of the solution of the Helmholtz equation causes slow convergence of iterative methods for \eqref{eq:intromat} is explained in \cite[Section 2.1]{ErGa:12}, using a finite-difference approximation of the Helmholtz equation as an example.

As GMRES applied to \eqref{eq:intromat} performs badly, we consider preconditioning \eqref{eq:intromat}, that is, solving the equivalent linear system
\beq\label{eq:intropre}
\PmatI\Amat \uvec = \PmatI \fvec
\eeq
for some matrix $\Pmat$. The goal of preconditioning\footnote{We have only considered left-preconditioning, that is, multiplying $\Amat$ from the left by $\PmatI$. However, one can also-consider right-preconditioning, that is, solving the linear system $\Amat \PmatI \uvectilde = \fvec,$ the solution $\uvec$ is then given by $\uvec = \PmatI \uvectilde.$} is to choose the preconditioner $\Pmat$ such that:
\ben
\item\label[listrequirement]{it:intropreone} The equation \eqref{eq:intropre} is easy to solve iteratively, and
\item\label[listrequirement]{it:intropretwo} The action of $\PmatI$ is cheap to compute.
\een
The ideal preconditioner from the point of view of \cref{it:intropreone} is $\Pmat = \AmatI$, however, in view of \cref{it:intropretwo}, if we could cheaply compute the action of $\AmatI,$ we could cheaply solve \eqref{eq:intromat}, and there would be no need for preconditioning. Hence, one needs to balance the \cref{it:intropreone,it:intropretwo} so that one obtains a good preconditioner for $\Amat$ that is cheap to apply. There are several groups around the world working on the construction of good preconditioners for the Helmholtz equation, and this is an open research area. The design of such preconditioners is not the focus of this thesis, but we refer to, e.g., \cite{GaZh:19} for an overview of many recent preconditioners for the Helmholtz equation.

\paragraph{Issues from UQ} On top of all the above issues in solving the deterministic Helmholtz equation \eqref{eq:introdet}, when seeking to perform UQ calculations for the stochastic Helmholtz equation \eqref{eq:introhh} one often needs to solve many realisations of \eqref{eq:introhh}.  For example, if one wants to calculate $\EXP{Q(u)},$ where $Q$ is some quantity of interest, then one can use the sample average of many realisations of $u;$
\beqs
\EXP{Q(u)} = \int_{\Omega} Q(u(\omega)) \ddPPomega \approx \frac1N \sum_{j=1}^N Q(u(\omega^{(j)})),
\eeqs
where $\omega \in \Omega$ denotes a random sample.
Observe that to compute the sample average, one needs to solve many (which we emphasise again, could easily be thousands) different deterministic Helmholtz problems which, as has just been discussed, are each individually difficult to solve. This situation arises when using either sampling-based methods (such as Monte-Carlo methods) or interpolation-based methods (such as Stochastic-Collocation methods) to compute properties of the solution $u$ of \eqref{eq:introhh}, or approximations to $u$ itself. Rigorously studying \eqref{eq:introhh}, devising computational techniques to reduce the cost of such UQ calculations, and rigorously justifying this reduction, is the subject of this thesis.

%% The aims of this thesis are to conduct a detailed study of the Helmholtz equation in random media, with special regard to high-frequency behaviour, and to provide fast, rigorously justifiable UQ methods for the high-frequency Helmholtz equation. 

%% We then seek to design numerical methods for the high-frequency Helmholtz equation that provide speedup over n\"aive numerical methods, in order to make UQ calculations for the Helmholtz equation more feasible. We also wish to analyse these numerical methods and show how their behaviour (both speedup and computational cost) depend on the wavenumber $k$. We propose two complementary numerical methods to speed up UQ calculations for the Helmholtz equation.






\section{The Main Acheivements of the Thesis}

The main acheivements of the thesis are as follows:


\ben

  \item\label[itemachievement]{it:achievements-error} New erro bounds for the higher-order $h$-finite-element method applied to the (deterministic) Helmholtz equation in heterogeneous media; these bounds are explicit in their dependence on the wavenumber $k$ and on the coefficient $n$.
\een

These error bounds are the first for the Helmholtz equation in a heterogeneous medium; similar bounds have been proved for the Helmholtz equation in a homogeneous medium in, e.g., \cite{Wu:14,DuWu:15,ChNi:18} and for a homogeneous medium with a small, frequency-dependent nonlinear perturbation in \cite{WuZo:18}. These bounds (and their explicit dependence on $k$ and $n$) are crucial for the analysis of the numerical methods developed in this thesis.



\ben[resume]

\item\label[itemachievement]{it:achievements-bounds} Well-posedness results and a priori bounds on the solution of the Helmholtz equation in random media, where the results and bounds obtained are frequency-independent.  The arguments behind these results are written in a sufficiently general way that they can be used, in principle, to conclude similar results for a range of stochastic PDEs. 

  \een

This work is an advance on the only previous work in the literature, \cite{FeLiLo:15}, in which similar results and bounds were proved, but under restrictions that became more stringent as the frequency increased; in contrast the results in this thesis are frequency-independent. To prove such frequency-independent results, we will consider classes of random media that are almost-surely nontrapping. This nontrapping assumption will allow us to obtain our frequency-independent results. These results on well-posedness and a priori bounds are crucial for analysing the numerical methods that follow. Our results show that the problems we are solving are well-posed, and will enable us to rigorously prove properties of the numerical methods considered in this thesis. We expect that the arguments behind these results can be used in other cases where the bilinear form given by the PDE is indefinite, such as for the time-harmonic Maxwell's equations.


  \ben[resume]
\item\label[itemachievement]{it:achievements-nbpc} A computational strategy, which we call `nearby preconditioning', that reduces the computational cost of solving many realisations of the Helmholtz equation in random media. 
\item\label[itemachievement]{it:achievements-nbpcnum} Numerical experiments that indicate that nearby preconditioning is, in practice, more effective than can be rigorously proved at present. We demonstrate the effectiveness of nearby preconditioning when applied to a Quasi-Monte-Carlo method for the Helmholtz equation.
  \een

  %The reduction in computational cost is gained by reusing the preconditioner from one realisation of the Helmholtz equation for subsequent `nearby' realisations. This computational method is rigorously analysed, and its effectiveness is precisely characterised, although this effectiveness does degrade as the frequency of the problems is increased. However, the effectiveness does still degrade as frequency is increased.
  
  The nearby preconditioning strategy seeks to reduce the computational cost of assembling preconditioners for many deterministic Helmholtz problems. This reduction is achieved by re-using a preconditioner from one deterministic Helmholtz problem for other Helmholtz problems, provided the coefficients are close in some metric (hence the term `nearby'). As far as we are aware, this is the first time such a strategy has been rigorously studied for the Helmholtz equation.
  %We prove that if the coefficients of the underlying PDEs are sufficiently close (or `nearby'), then GMRES applied to $\AmatoI\Amatt$ (where $\Amato$ and $\Amatt$ arise from the discretisation of different Helmholtz problems) will converge in a number of iterations that is independent of $k.$ However, the conditions for 'sufficient closeness' that we prove depend on either $k$ or the mesh size $h$. The numerical experiments show the sharpness (in some cases) or the lack of sharpness (in other cases) of our proven results.

 % In \cref{chap:nbpc} we propose a computational technique, \emph{nearby preconditioning}, that speeds up the process of solving many realisations of the Helmholtz equation by reusing preconditioners from one realisation for (potentially) many subsequent `nearby' realisations. In this theoretical study, we assume that we have access to the action of an exact preconditioner for one Helmholtz problem, and study the convergence of GMRES for subsequent problems. That is, we investigate the convergence of GMRES applied to $\AmatoI\Amatt$, where $\Amato$ and $\Amatt$ are matrices arising from finite-element discretisations of the Helmholtz equation. We show that if the coefficients of the underlying PDEs are sufficiently close (or `nearby'), then GMRES applied to $\AmatoI\Amatt$ will converge in a number of iterations that is independent of $k.$ However, the conditions for 'sufficient closeness' that we prove depend on either $k$ or the mesh size $h$. We then provide numerical experiments showing the sharpness (in some cases) or the lack of sharpness (in other cases) of our proven results.

 
  \ben[resume]



\item\label[itemachievement]{it:achievements-mlmc} Theoretical analysis of the Monte-Carlo and Multi-Level Monte Carlo methods applied to the Helmholtz equation in random media.


%\item\label[itemachievement]{it:achievements-mlmcnum} Computational experiments for MLMC that show that, in many cases, the speedup one obtains using MLMC is greater than that predicting theoretically\ednote{I've no idea what the result of these computations will be at this point!}.

    \item\label[itemachievement]{it:achievements-qmc} Computational investigation into Quasi-Monte-Carlo methods applied to the Helmholtz equation in random media. 
  \een

 Multi-Level Monte-Carlo is a variance reduction technique that uses computations on a sequence of meshes to reduce the variance in UQ calculations, and therefore to reduce the number of realisations of the Helmholtz equation that need to be solved. We extend the existing abstract Multi-Level Monte-Carlo analysis in the literature to the case where the finite-element error is dependent on an additional parameter (here this parameter is the wavenumber $k$), and then apply this abstract analysis to the Helmholtz equation, showing that we obtain theoretical speedup over MC, with the relative speedup being independent of $k.$ We also perform preliminary computations giving insight into the behaviour of QMC methods for the Helmholtz equation; these results are the first available in the literature.%We then verify this speedup computationally, and also perform 

  
  %. We also show that the error in MC is independent of the wavenumber $k$, and we show that MLMC gives a cost reduction over MC, with the relative cost reduction being independent of $k.$
  
%The second strategy, somewhat orthogonal to the first, is to use a Multi-Level Monte Carlo (MLMC) method to reduce the number of samples needed when performing UQ calculations. By reducing the number of samples needed, we will decrease the computational cost for UQ calculations. The analysis of this method requires an extension of standard MLMC theory to the case where the numerical error is dependent on a parameter besides the mesh size (in our case, the error in finite-element calculations depends on the wavenumber $k$). We will see that a MLMC approach does reduce cost, both in theory and in practice, and that the relative cost reduction with respect to a standard Monte Carlo method does not degrade as $k$ grows.

%we study the multi-level Monte Carlo (MLMC) method for reducing the variance in UQ calculations for the Helmholtz equation in random media. We first extend the abstract theory for MLMC to the sitation when there is an additional parameter (in our case, the wavenumber $k$), alongside the mesh size $h$, governing the size of the error in numerical approximations. Having extended the abstract theory, we then apply it to the Helmholtz equation for a variety of different quantities of interest, and we prove that MLMC gives a cost reduction over the Monte-Carlo method, and that this reduction is independent of $k$. We then investigate MLMC numerically, and find that in many cases the speedup we observe in numerics is better than the speedup we can prove rigorously.

\section{The Structure of the thesis}

In \cref{chap:background} we give background material on the (deterministic) Helmholtz equation and its discretisation via finite elements; this material will be necessary to understand the rest of this thesis. We then present recent results in \cite{GrPeSp:19} concerning the well-posedness of the deterministic heterogeneous Helmholtz equation and a priori bounds on its solution of which the author of this thesis was a co-author. These results are used in some of the developments in this thesis, especially in \cref{chap:stochastic}. We also give an overview of the theory of finite-element discretiations of the Helmholtz equation, and prove the results in \cref{it:achievements-error}.

In \cref{chap:stochastic} we prove the results in \cref{it:achievements-bounds} for three formulations of the stochastic exterior Dirichlet problem (SEDP) for the Helmholtz equation in random media. These results are underpinned by the well-posedness results and a priori bounds obtained in \cite{GrPeSp:19} for the heterogeneous (but non-random) Helmholtz equation. \Cref{chap:stochastic} is a lightly-edited version of \cite{PeSp:19}, accpeted for publication in the SIAM/ASA Journal on Uncertainty Quantification.

In \cref{chap:nbpc} we develop the nearby preconditioning strategy described in \cref{it:achievements-nbpc}, prove the results on its effectiveness, and perform the numerical investigation of its effectiveness as in \cref{it:achievements-nbpcnum}. We then provide preliminary numerical investigation into QMC methods for the Helmholtz equation, as in \cref{it:achievements-qmc} before applying nearby preconditioning to a QMC method for the Helmholtz equation, obtaining speedup, as in \cref{it:achievements-nbpcnum}.

In \cref{chap:mlmc} we develop abstract Multi-Level Monte-Carlo theory as in \cref{it:achievements-mlmc}, and then apply this to the Helmholtz equation, observing theoretical speedup compared to the Monte-Carlo method.

%for MLMC later?:
%(that is the ratio $\CMC(\eps)/\CMLMC(\eps)$, where $\CMC(\eps)$ is the cost for the root-mean-sqaured


\chapter{PDE Theory of the Deterministic Helmholtz Equation and Theory of its Finite-Element Discretisation}\label{chap:background}
\section{Introduction}
This chapter has two main foci:
\ben
\item Theory for deterministic Helmholtz especially well-posedness and a priori bounds
\item Theory for FEM for deterministic Helmholtz especially error bounds
  \een

In this chapter:
\bit
\item Provie overview of bounds/well-posedness for Helmholtz
\bit
\item More complicated than stationary diffusion
\item Depends on trapping/nontrapping
\item review historical and recent research efforts in this area; esp. heterogeneous/stochastic cases
\eit
\item Defin EDP/TEDP that we'll consider in this thesis
\item Prove result on regularity of solutions
\item Recap in detail results from \cite{GrPeSp:19} on well-posedness and a priori bounds in particular case of nontrapping - crucial for our analysis of the stochastic problem.
\item Give background theory for FEM for Helmholtz
\item Prove new results on error bounds for Helmholtz in heterogeneous media
  \bit
\item Extend results in literature---only for homogeneous
\item Results crucial in analysis of numerical methods in subsequent chapters
\eit
\item Review history of FEM NA for Helmholtz
\bit
\item Discuss $k$-dependence of mesh conditions
\item  overview of historical development of results on quasi-optimality/bounded error
\item These mesh sizes are restricitive
\eit


Only new material that is the subject of this thesis is that in \cref{sec:errbound}
\eit

Will include:
\bit
\item Background theory/setup for Helmholtz
\item Bounds/well-posedness results from \cite{GrPeSp:19}
\item Discussion of well-posedness/bounds for Helmholtz incl. historical development - much more challenging than stationary diffusion
\item Background theory/setup for FEM
\item New contribution: Proofs of boundedness of error in heterogeneous case under mesh size $h \sim k^{-3/2}$
\item Discussion of FE error bounds/Quasi-optimality etc. for Helmholtz incl. historical development
  \eit

\section{PDE Theory of the Deterministic Helmholtz Equation}
  
  \subsection{Variational Formulations for the Helmholtz equation}
First recap/state background theory on Helmholtz.
\bit
\item Define Helmholtz EDP/TEDP/IIP\optodo{Note to self, at this stage, but the weakest assumptions on the problem you need to define it}
\item Define variational formulation (for both zero and non-zero Dirichlet data)
\item Proof of $H^m$ regularity when $f$ etc. is smooth enough\optodo{Why did you think you needed more than $H^2$?}
\item G\r{a}rding inequality
  \eit

  \subsection{Well-posedness and bounds}
\bit
\item Now recap bounds/well-posedness results proved in hetero
\item Will be crucial for stochastic bounds in \cref{chap:stochastic}
\item Will give bounds that are:
\bit
\item $k$-independent
\item explicit in $A$ and $n$
\eit
\item Explicitness in $A$ and $n$ needed to prove stochastic bounds
\item conditions we assume for proof are sufficient (but not necessary) for nontrapping.
\eit

\bit
\item Define $NT_A,$ $NT_n$ etc. ($n + \bx \cdot \grad n \geq \mu$ etc.)
\item Theorem: Bounds for EDP (with zero and non-zero Dirichlet data)
\item Theorem: Bounds for TEDP (with zero and non-zero Dirichlet data)
\item Result for non-zero Dirichlet data has $A=I$ to make things simpler, but could do for $A \neq I$
\item Why results are significant:
\bit
\item First $k$-explicit bounds where $A$ and $n$ both vary
\item First $k$-independent bounds explicit in $A$ and $n$
\item Enable analysis of stochastic problem - enables rigorous UQ
\eit
\item Can extend results to:
\bit
\item Discontinuous $A$ and $n$
\item $A$ and $n$ perturbations of $NT$
\eit
\eit

\subsection{Discussion of results on well-posedness and a priori bounds for the Helmholtz equation}
\bit
\item Will review history of well-posedness and a priori bounds for Helmholtz
\item Will contrast with stationary diffusion (becuase of UQ)
  \eit

  \bit
\item Well-posedness $=$ existence, uniqueness, a priori bound (continuous dependence)
\item Want bound explicit in $k$ and parameters
  \item Also $k$- and parameter-explicit bounds needed for $k$- and $parameter$-explicit analysis of numerical methods
  \item And to understand how medium affects wave propagation
    \item A lot more subtle than stationary diffusion - there only quantities that enter are $\Amin$ and $\Amax$
  \eit

Recall for stationary diffusion\optodo{PUT SOMEWHERE ELSE?}:
\bit
\item Variational formulation has bounded and coercive form
\item Lax--Milgram gives existence, uniqueness, a priori bound explicit in coeffs
\eit
  
Helmholtz:
\bit
\item Above approach doesn't work---not coercive for large $k$
\item But do have G\r{a}rding - indentity $+$ compact
\item So use Fredholm alternative to show uniqueness $\implies$ existence and bound (bound not explicit in $k$)
\item Show uniqueness via
\bit
\item UCP - but don't get a priori bound explicit in $k,$ parameters\optodo{Why does SRc give uniqueness for homogeneous problems?}
\item Proving a priori bound directly - get uniqueness
\eit
\eit
\bit
\item Will now talk about bounds
\item Focus on $k$-dependence (and later) parameter-dependence
\item Long history of proving bounds on Helmholtz from 'analysis' community \cite{Bl:73,Va:75,BlKa:77,Bu:98,PeVe:99,PoVo:99a,PoVo:99b,CaPoVo:99,Bu:02,Be:03,Ca:12,CaLePa:12,NgVo:12,Sh:18}\optodo{Check each and every reference}\optodo{Check about Morawetz} and recent interest from NA \cite{Me:95,CuFe:06,He:07,EsMe:12,Sp:14,FeLiLo:15,Ch:15,BaSpWu:16,BrGaPe:17,BaChGo:17,SaTo:18,OhVe:18,GrSa:18,GaSpWu:18,GrPeSp:19,MoSp:19}
\item Worst-case scenario - bounds grows exponentially in $k$
\bit
\item If everything $C^\infty$, shown for $n=1$ in \cite{Bu:02} and if $A=1$ and $n$ Lipschitz in \cite{Sh:18}. For both jumping across same $C^\infty$ interface in \cite{Bu:98}.
\item Graham-Sauter \cite{GrSa:18} show exponential blow up in (properties of) coefficients, not $k$ in 1-D\ednote{Ivan, can we chat about your work with Stefan to check I've understood it correctly?}
\item \cite{MoSp:19} - numerical evidence for super-algebraic growith with trapping jumps - proved in \cite{PoVo:99a,PoVo:99b,CaPoVo:99}.
\eit
\item But only get exponential blow up when problem is trapping
\item When everything $C^\infty,$ can define rays (for penetrable and impenetrable obstacle) and thus define nontrapping
\item Nontrapping gives $1/k$ cut-off resolvent bound
\item Will get cases later where things aren't so smooth, but still get cut-off resolvent bound
  \item Call these nontrapping
\item Trapping of rays (angle of incidence = angle of reflection) stay for arbitrarily long time
\item Can get trapping if there are jumps in one direction, but not in other direction - \cite{MoSp:19}
\item Actually, trapping only happens at very few frequencies-\cite{MoSp:19,LaSpWu:19}
\item Related to idea of `resonant frequency'
\item Can be caused by obstacles or variations in speed - c.f. total internal reflection
\item If problem non-trapping (i.e. rays escape) then get bound independent of $k$ - see hetero\optodo{Find a reference for this}\optodo{iff?}
\item Therefore results proving $k$-indep bounds under some conditions can be seen as providing sufficient conditions for problem to be nontrapping.
\item Such results use either:
\bit
\item Multiplier techniques going back to Morawetz (and Rellich) \optodo{List results, demarcating homo and hetero} which give $k$-explicit bound directly, or
\item more technical microlocal tools, which give nontrapping, from which you can conclude bound\optodo{Look at Vainberg/Melrose--Sj\"ostrand}
\item Recent work on bounds explicit in parameters---motivated in part by UQ - \cite{FeLiLo:15,GrPeSp:19,PeSp:18} prove bounds under conditions that ensure nontrapping, \cite{GaSpWu:18} assume problem is nontrapping and get `length of longest ray'.
\eit
\item In conclusion, subtle how things depend on $k,$ coefficients, and more difficult than stationary diffusion.
\eit

\bit
\item Now shift attention to NA of Helmholtz, specifically, FEM.
  \item Will prove new result on error bounds for Helmholtz in hetero media, generalises homo media result.
  \eit

\section{Theory of the Discretisation of the Helmholtz Equation}
  
\subsection{Finite-element theory}
Now give requisite background finite-element theory
\bit
\item Definition - Finite element/Finite element space
\item Definition - Triangulation
\item Definition - quasi-uniform
\item Definition - Nodal Finite element
\item Definition - Mesh size
\item Definition - Fe space (of order $p$) associated with a triangulation
\item Definition - Scott--Zhang quasi-interpolant
\item Remark - Nodal interpolant falls within Scott--Zhang framework
\item Lemma - approximation properties of Scott--Zhang quasi-interpolant
\item Definition - Finite-element problem
\item Remark - won't be considering non-conforming error\optodo{At least understand this}
\item Lemma - Norm equivalence (Lemma 3.7 in current draft of nearby preconditioning article)
\eit

\subsection{Error bound for the heterogeneous IIP}\label{sec:errbound}
\bit
\item This section contains new work
\item Prove heterogeneous analogue of results on bounded error if $h \lesssim k^{-3/2}$
\item Proof technique uses `elliptic projection' idea - see Feng \& Wu, Wu, Du \& Wu, Chaumont-Frelet \& Nicaise.
\item Prove for IIP only, as need to use results on a related PDE from Chaumont-Frelet and Nicaise; only proved with impedance boundary condition - expect can extend to, e.g., exact DtN.
\item These bounds crucial for MLMC analysis in \cref{chap:mlmc}
\item Will prove for non-zero Dirichlet boundary condition, as motivation for MLMC is scattering
\eit

\subsection{Discussion of FEM for Helmholtz}
\bit
\item Recall from intro $h\sim k^{-1}$ (fixed ppw as num. points $\sim 1/h$) means interpolation error bounded
\bit
\item Can see by interpolation bounds from, e.g., Scott-Zhang (see below)
\item In 1-D\optodo{Find out if it works in higher D} motivate also by Shannon--Nyquist theorem\optodo{Find Shannon Proc I. R. E., 31:10-21, '49 Communication in the prescence of noise}
\item Both make precise intuition that we have bounded errors under fixed points per wavelength
\eit
\item Also recall Helmholtz suffers from pollution - fixed ppw not enough to keep FEM error bounded
\item This section will summarise conditions under which error bounded, and related property of quasi-optimality holds
\item Plan:
\bit
\item Introduce concepts of bounded error and quasi-optimality
%\item Show (for contrast) that they are straightforward for stationary diffusion
\item Give overview of results on bounded error and Quasi-optimality for Helmholtz
\eit
\eit
\bit
\item Definition: error bounded
\item Would really like relative error bounded, however, not possible with current technology
\item Definition: Quasi-optimal
\item Natural question: what mesh size keeps error bounded?
\item Short history:
\bit
\item Bayliss, Goldstein, Turkel - computations showing $h\sim k^{-3/2}$ sufficient to keep error bounded
\item Ihlenburg, Babu\v{s}ka - prove error $\sim h^2k^3$ in 1-D if $h \sim k^{-1}$
\item Wu - error $\sim h^2k^3$ if $h \sim k^{-3/2}$ for 1st-order FEM
\item Du and Wu - error $\sim h^{2p}k^{2p+1}$ for higher-order FEM
\item Chaumont-Frelet \& Nicaise - similar result with corner singularities
\item Here in \cref{sec:errbound} first proof that $h\sim k^{-3/2}$ sufficient to keep relative error bounded in heterogeneous problems
\eit
\item For stationary diffusion, things are straightforward
\item Prove quasi optimality using C\'ea
\item Get FEM error bounded by using properties of interpolant, e.g., via Scott-Zhang.
\item No condition on $h$ needed, unlike Helmholtz
\item For Helmholtz Cea unavailable as not coercive
\item Proving Quasi-optimality more tricky
\item Instead use Aubin--Nitsche duality argument
\bit
\item Introduced by Nitsche\optodo{FOr what?}
\item Used by Aubin\optodo{For what}
\item Applied by Schatz to problems satisfying G\r{a}rding inequality
\item First applied to Helmholtz by Melenk
\item Obtain quasi-optimality under nontrapping assumptions and restrictive mesh condition $h \sim k^{-2}$---see \cite[Proposition 8.2.7]{Me:95} for homogeneous, \cite[Theorem 3]{GaSpWu:18} for heterogeneous, related bound in \cite[Section 1.4]{ChSpGiSm:17} sketched for parabolic trapping cases
\item Mesh condition is prohibitive for large $k$
\eit
\eit
Conclusion:
\bit
\item Rigorous NA of Helmholtz more demanding than, e.g., stationary diffusion
\item Mesh conditions require decreasing mesh size with $k$; esp. for quasi-optimality, very restrictive.
\eit


\optodo{Put size of linear systems in}
\optodo{Put in poor performance of GMRES?}
\optodo{Put in need for preconditioners?}


%%5 Can put the below somewhere %%

%% To keep the finite-element error bounded when solving \eqref{eq:introdet}, one must over-refine the numerical grid. That is, rather than using a fixed number of points per wavelength, one must increase the number of points per wavelength as $k$ increases. To achieve bounded finite-element error, one must refine the finite-element mesh size $h$ like $k^{-3/2}$. Whilst this result has been known numerically for some time, it was proven for \eqref{eq:introdet} only with constant coefficients (on various domains and for various finite-element spaces) in \cite{IhBa:95a,Wu:14,DuWu:15,ChNi:18}, and the first proof (to our knowledge) for \eqref{eq:introdet} with heterogeneous coefficients  is contained in \cref{chap:background}. Choosing $h \sim k^{-3/2}$ means \eqref{eq:intromat} is a linear system of size $k^{3d/2}$, larger than if one merely wants the interpolation error to be bounded. Hence, requiring a bounded finite-element error gives rise to very large linear systems.

%% More briefly, if one wants the finite-element solution to be quasi-optimal (that is, up to a constant, the finite-element solution is the best approximation in the finite-element space), then one must over-refine even more, and take $h \sim k^{-2}$. This mesh condition will give rise to linear systems with $k^{2d}$ degres of freedom. See \cref{chap:background} for further details on the necessity of this mesh condition, and further discussion of all the mesh conditions discussed above.\optodo{EDIT THE ABOVE TO REMOVE `OVER-REFINE'}



\section{Theory of the Finite-Element Discretisation of the Helmholtz Equation}\label{sec:helmfe}

We now shift our attention to the numerical analysis of the Helmholtz equation in heterogeneous media; in particular, we  study the finite-element method for the Helmholtz equation. We first state the variational formulations of the Helmholtz equation, define the finite element method, and recall results on the approximation properties of finite-element spaces. we then prove our main result, error bounds for the finite-element method for the heterogeneous Helmholtz equation, where the bounds hold for arbitrary (fixed) degree finite elements, and are explicit in $n$ and (in principle) $A$.

a new error bound for the finite-element method for the Helmholtz equation in heterogeneous media.

  \subsection{Variational Formulations for the Helmholtz equation}\label{sec:varform}
  The finite element method is based on the variational formulation of the Helmholtz equation; for simplicity of exposition, we state the variational formulation of \cref{prob:edp,prob:tedp} in the case $\gD = 0,$ although these can be generalised to the case $\gD \neq 0.$
  
\bprob[Variational formulation of EDP when $\gD = 0$]\label{prob:vedp}
Let $\Dp, A, n,$ and $f$ be as in \cref{prob:edp}. Choose $R>0$ such that $\supp f,$ $\supp(I-A),$ $\supp(1-n) \compcont \BR,$ and define $\DR \de \Dp \cap \BR.$

We say $u \in \HozDDR$ satisfies the \defn{variational formulation of the exterior Dirichlet problem} with $\gD = 0$ if
\beqs
\aE(u,v) = \FE(v)\quad \tfa v \in \HozDDR,
\eeqs
where
\beqs
\aE(w,v) \de \int_{\DR} \mleft(\mleft(A \grad w\mright)\cdot\grad \vbar - k^2 n\minispace w \vbar\mright) - \DPGR{\DtN \trGR w}{\trGR v}
\eeqs
and
\beqs
\FE(v) \de \int_{\DR} f\minispace\vbar,
\eeqs
where $\DtN:\HhGR\rightarrow \HmhGR$ is the Dirichlet-to-Neumann map for the homogeneous Helmholtz equation $\Lap u + k^2 u = 0$ combined with the Sommerfeld radiation condition in the exterior of $\BR$; and $\DPGR{\cdot}{\cdot}$ is the duality pairing on $\GR.$
\eprob

\ble[Equivalence of formulations for the EDP]\label{lem:edpform}
\Cref{prob:edp,prob:vedp} are equivalent. If $u \in \HolocDp$ solves \cref{prob:edp} then $u\restrict_{\DR} \in \HozDDR$ and $u\restrict_{\DR}$ solves \cref{prob:vedp}  (for $R$ as in \cref{prob:vedp}). Conversely, if $u \in \HozDDR$ solves \cref{prob:vedp}, then $u$ solves \cref{prob:edp}, if $u$ is extended to $\HolocDp$ by the solution of the exterior Dirichlet problem (in the exterior of $\DR$) for the homogeneous Helmholtz equation with the Sommerfeld radiation condition (with Dirichlet data $\tr u$ on $\partial \BR$).
\ele

For a proof of \cref{lem:edpform}, see \cite[Lemma 3.3]{GrPeSp:19}.

\bprob[Variational formulation of TEDP when $\gD = 0$]\label{prob:vtedp}
Let $D, A, n, f,$ and $\gI$ be as in \cref{prob:tedp}. We say $u \in \HozDDR$ satisfies the \defn{variational formulation of the truncated exterior Dirichlet problem} with $\gD = 0$ if
\beq\label{eq:vtedp}
\aT(u,v) = \FT(v) \tfa v \in \HozDDR,
\eeq
where
\beq\label{eq:aT}
\aT(w,v) \de \int_{\DR} \big(\mleft(A \grad w\mright)\cdot\grad \vbar - k^2 n\minispace w \vbar\big) - ik\int_{\GI} \trGI w\minispace\trGI \vbar
\eeq
and
\beqs
\FT(v) \de \int_{\DR} f\minispace\vbar + \int_{\GI} \gI \minispace\trGI \vbar
\eeqs
\eprob

\ble[Equivalence of formulations for the TEDP]\label{lem:tedpform}
\Cref{prob:tedp,prob:vtedp} are equivalent, i.e., $u \in \HozDDR$ solves \cref{prob:tedp} if, and only if, $u$ solves \cref{prob:vtedp}.
\ele

For a proof of \cref{lem:tedpform}, see \cite[Lemma A.7]{GrPeSp:19}.
  
\subsection{Background Concepts in Finite-Element Theory}\label{sec:fetheory}

We now give a brief summary of elementary concepts in finite-element theory. In the interest of brevity, we focus only on those concepts that we be needed to
prove the new error bound for finite-element discretisations of the Helmholtz equation in \cref{sec:fem} below.

Throughout this thesis, we use the terms `mesh` and `triangulation' to refer to a triangulation of simplices in the sense of Ciarlet \cite[Paragraphs (FEM1) p. 61 and ($\cT_{h}$5) p. 71]{Ci:91}. Similarly, we use $h$ to denote the mesh size of a mesh\footnote{Frequently in this thesis we will consider families of meshes indexed by their mesh size $h$.}, and we use $\Vhp$ to denote the set of continuous piecewise-polynomials of degree $p$ on a mesh with mesh size $h$.
  %%   triangulation of the domain. (Note that in this thesis we use the words `mesh' and `triangulation' interchangably; strictly speaking, the term `triangulation' only makes sense in 2-d, but we ignore this technicality.

  %%   \bde[Triangulation]
  %%   A triangulation of a polygonal domain $D$ is a finite collection of sets $\Ki \subseteq \Dclos$ such that
  %%   \ben
  %% \item $\interior{\Ki} \cap \interior{\Kj} = \emptyset$ for $ i \neq j,$
  %% \item $\bigcup_i \Ki = \Dclos,$ and
  %%   \item Something about triangles that works in 3-D too\opntodo{Find ref - Brezzi/Johnson? Make sure to use simplical meshes.}.
  %%   \een
  %%   \ede\opntodo{Change to definition in Ciarlet, Handbook of numerical analysis}

%% \bde[Mesh Size]
%% Given a mesh $\cT = \set{\Ki}$ on a domain $D$, the \defn{mesh size} of $\cT$ is defined to be
%% \beqs
%% h = \max_{\Ki} \diam{\Ki}.
%% \eeqs
%% \ede

%%     We can now define finite-element spaces associated with a triangulation; we first define the space of polynomials on a set.

%% \bde[Set of polynomials]
%% For $K \subseteq \RRd,$ $\polyp{K}$ is the set of polynomials defined on $K$ with total degree at most $p.$
%% \ede
%%     \bde[Finite element space of degree $p$]\label{def:fespace}
%%     Given a triangulation $\set{\Ki}$ with mesh size $h$ of a domain $D$, the \defn{(continuous) piecewise-polynomial finite-element space of degree $p$} associated with $\set{\Ki}$ is
%%     \beqs
%% \Vhp \de \set{\vh : D \rightarrow \CC \st \vh \in \CzD \tand \vh\restrict_{\Ki} \in \polyp{\Kiclos} \tforall \Ki }.
%%     \eeqs
%%     \ede

Throughout this thesis, we only consider the $h$-finite-element method, i.e., the degree $p$ of the polynomials associated with the space is assumed fixed, and we consider refining $h.$ This is in contrast to the $p$-finite-element method, where $h$ is fixed and $p$ is increased, and the $hp$-finite-element method, where both $h$ and $p$ are increased according to some rule. $hp$-finite-element methods for the homogeneous Helmholtz equation were analysed by Melenk and Sauter in \cite{MeSa:10,MeSa:11}; however, their analysis relies on a fully $p$-explicit analysis of the best approximation error of the Helmholtz equation in $hp$-finite-element spaces; such analysis is currently not possible for the heterogeneous problem. Throughout this \lcnamecref{sec:fetheory} we use finite-element method to refer to the $h$-finite-element method.
%\ednote{I couldn't immediately see in these results that one obtains exponential convergence with respect to the number of DOFs---have I missed something, or is this result contained somewhere else?}, where they showed one can obtain an exponential rate of convergence with respect to the number of degrees of freedom.
%\opntodo{Get Schwab's hp book and look this up}

%    We adopt the terminology that $\vhptilde$ is the \defn{quasi-interpolant} of $v.$ For a proof of \cref{lem:scottzhang} see, e.g., \cite[Corollary 4.4.24, Remark 4.4.27]{BrSc:08}.

%%     \bre[The function $\vhptilde$]
%% The function $\vhptilde$ in \cref{lem:scottzhang} can be constructed using `averaged Taylor polynomials', see, e.g., \cite{ScZh:90},\cite[Section 4.4]{BrSc:08} for details\footnote{Observe that in \cite{BrSc:08}, the authors use different notation to us; they use $m$ to denote the regularity of $v$ and $p$ to denote the integrability of $v,$ i.e., in \cite{BrSc:08}, $v \in \WmpD.$}.
%%     \ere

    With the concept of a finite-element space in place, we can now define the finite-element approximation to the variational problems \cref{prob:vedp,prob:vtedp}.

    \bprob[Finite-element approximation of \cref{prob:vtedp}]\label{prob:fevtedp}
    We say that $\uh \in \Vhp$ is the \defn{finite-element approximation of $u$} (the solution to \cref{prob:vtedp}) if
    \beqs
    a(\uh,\vh) = L(\vh) \tforall \vh \in \Vhp.
    \eeqs
    The finite-element approximation of \cref{prob:vedp} is defined analagously.
    \eprob

    The following \lcnamecref{lem:scottzhang} shows how the approximation properties of the space $\Vhp$ depend on $h$ and $p$:
    \ble[Existence of best approximation]\label{lem:scottzhang}
    Let $v \in \HmD$, $m \geq 1,$ and $p \in \set{1,\ldots,m}$. Then there exist a constant $\CSZm>0$ independent of $v$ and $\vhptilde \in \Vhp$ (but dependent on $p$) such that

    \beqs
\NLtD{v - \vhptilde} \leq \CSZm h^m \NHmD{v}
    \eeqs
     and
    \beqs
\NHoD{v - \vhptilde} \leq \CSZm h^{m-1} \NHmD{v} \tfor m \geq 1.
    \eeqs
    \ele

    
    \bre[Not considering variational crimes]
    Throughout this thesis we assume that \cref{prob:fevtedp} is a conforming discretisation if \cref{prob:vtedp}, i.e. $\Vhp \subset \HozDDR.$ This inclusion is true if $\DR$ can be triangulated; otherwise, we must modify the definition of \cref{prob:fevtedp} and commit a variational crime by approximating the boundary of $\DR$ by a polygon, or introducing mesh elements with curved boundaries using interpolated boundary conditions or isoparametric finite elements. See, e.g., \cite[Chapter 10]{BrSc:08} for an overview of the additional errors introduced by variational crimes (although not in the context of the Helmholtz equation). In this thesis we  ignore such variational crimes, and the additional errors they induce; such analysis is standard. Therefore for the remainder of this thesis we assume that \cref{lem:scottzhang} holds, even if we have not triangulated the domain $D$ exactly.
    \ere
    

%%     \subsection{Error bound for the heterogeneous IIP}\label{sec:errbound}

%%     We now move on to present new work---bounds on the error $\uh-u$ between the finite-element approximation and the true solution of \cref{prob:vtedp}. These bounds, proven here for the Helmholtz equation in \emph{heterogeneous} media are generalisations of results already in the literature that the finite-element error in the $L^2$- and $H^1$-norms is bounded (as $k\rightarrow \infty$) provided $h \lesssim k^{-(2p+1)/2p}$. These results are crucial for our analysis of the multi-level Monte Carlo method for the Helmholtz equation in \cref{chap:mlmc}.

%% The proof of these results uses a so-called `elliptic projection' technique, introduced by Feng and Wu in \cite{FeWu:09} for Discontinuous Galerkin methods for the Helmholtz equation and modified by Du and Wu \cite{DuWu:15}, where the variational formulation of a PDE related to \cref{prob:vtedp} is used as part of the proof. We  only prove results for \cref{prob:fevtedp}, and not for the finite-element approximation of \cref{prob:vedp}, as our proof uses properties of the related PDE and these properties have only been proven with impedance boundary conditions (in the recent preprint \cite{ChNiTo:18}) and not with an exact Dirichlet-to-Neumann map on the truncation boundary. However, we imagine the results proven for the related PDE with an impedance boundary condition also hold for an exact Dirichlet-to-Neumann boundary condition, and so we anticipate that the results we prove here also hold for finite-element approximations of \cref{prob:vedp}.\opntodo{This paragraph needs rewriting---acknowledge Du \& Wu}

\subsection{Discussion of the finite-element method for the Helmholtz equation}\label{sec:helmfedisc}

We now discuss three properties of the $h$-finite-element method for the Helmholtz equation:
\bit
\item The relative error $\displaystyle \frac{\NW{u-\uh}}{\NW{u}}$ is bounded,
\item The error $\NW{u-\uh}$ is bounded in terms of norms of the data $f$ and $\gI$, and
  \item The finite-element method is quasi-optimal, $\displaystyle \NW{u-\uh} \lesssim \inf_{\vh \in \Vhp} \NW{u-\vh}.$
    \eit

    The key question we ask is: What mesh conditions lead to each of the three properties above? (I.e., how must one refine $h$ as $k$ increases in order for the finite-element method to satisfy each of the three properties above?)

    We will now summarise the state of the literature regarding each of the three properties above. We will:
    \ben
  \item Define each of the three properties formally,
  \item\label{it:sum2} State the sharpest results known in the literature,
  \item\label{it:sum3} Give a complete overview of all results in the literature concerning the three properties,and
    \item\label{it:sum4} Give a detailed discussion of proof techniques for each of the three properties.
      \een

We do \cref{it:sum2,it:sum3} because to our knowledge there is not a complete overview of all of these three properties anywhere in the literature. Moreover, research interest in these three properties has recently increased (e.g., the recent papers \cite{ChNi:18,LiWu:18,ChGaNiTo:18,ChNi:19} are all concerned with one or more of these properties, and have all been completed within the last two years), and so we believe it will be helpful and timely to provide a complete overview of the state of the literature. We do \cref{it:sum4} because collection of the proof techniques used for each of these properties are different to those used for standard finite-element proofs for, e.g., the stationary diffusion equation. Moreover, the different proof techniques are different but related, and we do not know of anywhere in the literature where these techniques are expounded, compared, and constrasted. In view of the resurgence in interest in the properties above, we hope that such a review will be valuable for the research community.

      

\paragraph{Intuition for fixed number of points per wavelength} Before discussing the three properties listed above, we first give a brief discussion of the commonly-used heuristic `take a fixed number of points (or elements) per wavelength'.  Recall from \cref{sec:numsolve} that if one takes the mesh size in the finite element method $h \sim 1/k$, then one expects the interpolation (or best approximation) error is bounded uniformly in $k$. More rigorously, as solutions of the Helmholtz equation typically\footnote{If $f$ and $\gD$ are independent of $k$,  combining \cref{thm:edp} and standard elliptic regularity results, e.g., \cite[Theorems 4.16 and 4.18]{Mc:00} gives the fact that $\NHtD{u} \lesssim k$.} have $\NHtD{u} \sim k,$ one can bound the $H^1$-norm of the interpolation error if $h \sim 1/k$ using \cref{lem:scottzhang}.  As explained in \cref{sec:numsolve}, this restriction ensures there are a fixed number of discretisation points per wavelength of the solution, since the wavelength $\lambda = 2\pi/k.$

An alternative motivation for taking $h \sim 1/k$ is the Nyquist--Shannon sampling theorem (proved by Shannon in his seminal paper in information theory \cite[Theorem 1]{Sh:49}). The theorem states that any function $v$ (in 1-d) whose Fourier transform lies inside $[-k,k]$, for some $k>0$, is completely determined (via its Fourier series) by the point values $v(0)$, $v(\pm \mu),$ $v(\pm2\mu),  \ldots$, for any $\mu < 1/\mleft(2k\mright).$ Observe that $k$ is then the highest frequency present in $v$.  That is, if we interpolate $v$ at points spaced less than $ \lambda/(4\pi)=1/(2k)$ apart, where $\lambda = 2\pi/k$ is the wavelength associated with oscillations of frequency $k$, we can reconstruct $v$ from its Fourier transform. See, e.g., \cite[\S 5.21]{BaNaBe:00} for an explicit formula for reconstructioning $v$. Therefore, in the 1-d case we may reasonably expect that interpolating $v$ with points spaced less than $\lambda/(4\pi)$ apart will be a good approximation of $v$, as the Nyquist--Shannon theorem suggests such a spacing of points `captures' all of the oscillations in $v.$
%% The solution of the one-dimensional Helmholtz equation with constant coefficients is
%% \beq\label{eq:hh-1d}
%% u(x) = A \sin(kx) + B \cos(kx),
%% \eeq
%% (for some constants $A$ and $B$), and $\lambda = k/2\pi$ as $u$ has Fourier Transform
%% \beqs
%% \uhat(\xi) = \frac{A}{2i} \mleft(\delta\mleft(\xi-\frac{k}{2\pi}\mright) - \delta\mleft(\xi+\frac{k}{2\pi}\mright)\mright) + \frac{B}{2} \mleft(\delta\mleft(\xi-\frac{k}{2\pi}\mright) + \delta\mleft(\xi+\frac{k}{2\pi}\mright)\mright).
%% \eeqs
%% \opntodo{Find reference}
%% Therefore, if $u$ is sampled at regularly-spaced points that are strictly closer together than $\pi/k = 1/(2\lambda)$, then $u$ can be reconstructed perfectly from these samples.\opntodo{Consider writing something about mutliple dimensions} In conclusion, one expects the be able to interpolate the solution of the Helmholtz equation with uniform error in $k$ if $h \sim 1/k.$

%% However, as was stated in \cref{sec:numsolve}, the finite-element method for the Helmholtz equation suffers from pollution; and $h \sim 1/k$ is not sufficient to keep the finite-element error bounded as $k\rightarrow \infty.$ Therefore we  now provide an overview of previous work giving mesh conditions under which the finite-element error $\NW{u-\uh}$ is bounded as $k\rightarrow \infty$, as well as mesh conditions under which the finite-element method is quasi-optimal. Concerning errors, it would be more natural to consider the relative error $\NW{u-\uh}/\NW{u};$ and to investigate the conditions on $h$ which enable the relative error to be bounded as $k\rightarrow \infty.$ However, current technology typically only allows us to prove results for the error, not the relative error, and so we investigate the error instead.

%We will now briefly review the literature that seeks to answer the question `What mesh condition will keep the relative error bounded as $k \rightarrow \infty$?'


%% \bde[Bounded finite-element error]
%% Let $(\Th)_{h \in (0,1)}$ be a shape-regular family of triangulations of $\DR$, where we assume $h$ is a function of $k$ (e.g., $h(k) = 1/k$). We say that \defn{the error in the finite-element method is bounded uniformly in $k$} if for all $A,n$ in some class there exists $C(A,n)$ (but independent of $k$ and $h$) such that
%% \beqs
%% \NW{u-\uh} \leq C\mleft(\NLtD{f} + \NLtGD{\gD} + \NLtGI{\gI}\mright).
%% \eeqs
%% \ede



 %% In $d=2$ or $3$ with $D$ a convex polygon or polyhedron,  \cite[Theorem 6.1]{Wu:14} showed that the error (but not the relative error) in both the $H^1$ seminorm and the $L^2$ norm is bounded uniformly in $k$ if $h \lesssim k^{-3/2}$. The proof of \cite[Theorem 6.1]{Wu:14} showed an analagous result for a continuous interior-penalty method, and then took the limit as the penalty parameter tends to 0 to obtain the result for the continuous piecewise-linear finite-element method. In summary, the mesh condition $h \lesssim k^{-3/2}$ seems to be sufficient for the (relative) error to be bounded uniformly in $k.$

\subsubsection{Formal definition of properties of the finite-element method}\label{sec:femprops}

We now formally define the three properties mentioned above. These definitions may seem overly technical, however we believe their technical definition is necessary to capture the, at times, complex behaviour of finite-element methods for the Helmholtz equation. After stating the definitions, we give some examples of what they mean for finite-element discretisations of the Helmholtz equation. These definitions are based on \cite[Definition 2.3]{DiMoSp:19}.

In the following definitions, we consider the Helmholtz problems \cref{prob:vtedp}, and their discretisations \cref{prob:fevtedp} as parameterised by the wavenumber $k,$ that is, we write $u(k)$ for the solution of \cref{prob:vtedp} with wavenumber $k$, and $\uh(k)$ for its finite-element approximation.

\bde[$\hk{a}{b}$-accurate]\label{def:hkacc}
Given $a,b>0$, we that an $h$-finite-element method for the Helmholtz equation is \defn{$\hk{a}{b}$-accurate} if, given $0 < \eps < 1,$ and $\kz > 0,$ there exist $\Co(\kz), \Ct(\eps,\kz) > 0$ such that if
\beqs
hk^a \leq \Co
\eeqs
and
\beqs
hk^b \leq \Ct,
\eeqs
then the Galerkin solutions $\uh(k)$ satisfy
\beq\label{eq:hkacc}
\frac{\NW{u(k)-\uh(k)}}{\NW{u(k)}} \leq \eps
\eeq
for all $k \geq \kz.$

If $a=b$ we say the $h$-finite-element method is $\hka{a}$-accurate.
\ede

\bde[$\hk{a}{b}$-data-accurate]\label{def:hkdataacc}
Given $a,b>0$, we that an $h$-finite-element method for \cref{prob:vtedp} is \defn{$\hk{a}{b}$-data-accurate} if, given $0 < \eps < 1,$ and $\kz > 0,$ there exist $\Co(\kz), \Ct(\eps,\kz) > 0$ such that if
\beqs
hk^a \leq \Co
\eeqs
and
\beqs
hk^b \leq \Ct,
\eeqs
then the Galerkin solutions $\uh(k)$ satisfy
\beq\label{eq:hkdataacc}
\frac{\NW{u(k)-\uh(k)}}{\NLtD{f} + \NLtGI{\gI}} \leq \eps
\eeq
for all $k \geq \kz.$

If $a=b$ we say the $h$-finite-element method is $\hka{a}$-data-accurate.
\ede

Observe that the only difference between the definitions of $\hk{a}{b}$-accurate and $\hk{a}{b}$-data-accurate is that in \cref{eq:hkacc} the error is measured relative to the solution $u$, whereas in \cref{eq:hkdataacc} the error is measure relative to the `data' $f$ and $\gI$.

\bde[$\hka{a}$-quasi-optimal]\label{def:hkqo}
Given $a > 0,$ we say that an $h$-finite-element method for the Helmholtz equation is \defn{$\hka{a}$-quasi-optimal} if, given $\kz > 0$, there exist $\Co(\kz), \Cqo(\kz) > 0$ such that if
\beqs
hk^{a} \leq \Co,
\eeqs
then the Galerkin solutions $\uh(k)$ satisfy
\beqs
\NW{u(k) - \uh(k)} \leq \Cqo \inf_{\vh \in \Vhp} \NW{u(k) - \vh},
\eeqs
for all $k \geq \kz.$
\ede

\bre[Generalisations of $\hk{a}{b}$-data-accurate]
For discretisations of other Helmholtz problems (e.g., full-space problems with no truncation boundary $\GI$, or problems truncated with a perfectlly matched layer (PML)), then the denominator in \cref{eq:hkdataacc} should be adapted appropriately, e.g., the denominator should equal $\NLtD{f}$ for full-space problems and PML problems. In our discussion of the literature below (which includes such problems), we will make such an adaption without comment.
\ere

\bre[Relationship between $\hk{a}{b}$-accuracy and $\hk{a}{b}$-data-accuracy]
Observe that if an a priori bound such as \cref{eq:tedpbound} holds, then $\hk{a}{b}$-accuracy implies $\hk{a}b$-data-accuracy, as $\NW{u} \lesssim \NLtD{f} + \NLtGI{\gI}$.

However, in order to conclude $\hk{a}{b}$-accuracy from $\hk{a}b$-data-accuracy, one would need to show
\beq\label{eq:reversebound}
\NLtD{f} \lesssim \NW{u}.
\eeq
However, in general \cref{eq:reversebound} does not hold\ednote{Euan---can one write down an example of this \emph{easily}?}, and so one cannot, in general, conclude $\hk{a}{b}$-accuracy from $\hk{a}{b}$-data-accuracy.
\ere

\bre[The above conditions for heterogeneous problems]
Observe that for heterogeneous problems, the constants $\Co, \Ct$, and $\Cqo$ in \cref{def:hkacc,def:hkdataacc,def:hkqo} will all depend on the coefficients $A$ and $n.$ In general, this dependence is unknown. However, in our original work in \cref{sec:fem} below we prove that the $h$-finite-element method with polynomial degree $p$ is $\hka{(2p+1)/(2p)}$-data-accurate, where the dependence of $\Co$ and $\Ct$ on $n$ (and, in principle, $A$) is completely known.
\ere

To aid understanding of the above definitions, and to see how they connect with the usual presentation of finite-element results for the Helmholtz equation, we give the following illustrative example.

\bre[Example achieving $\hk{a}{b}$-data-accuracy]\label{rem:dataacc}
In \cite[Theorem 5.4]{ChGaNiTo:18}, Chaumont-Frelet, Gallistl, Nicaise, and Tomezyk proved\footnote{The result proved in \cite[Theorem 5.4]{ChGaNiTo:18} is slightly more complicated than that stated here; the right-hand side of \cref{eq:hkacceg} should be $\Cth \mleft(hk + h^{2p}k^{2p+1}\mright)\NLtD{f}.$ However, an analagous proof to the one presented here holds in the more complicated case.} that if $k \geq \kz$ and $h^{p+1}k^{p+2} \leq \Co$, then
\beq\label{eq:hkacceg}
\NW{u(k) - \uh(k)} \leq \Cth h^{2p}k^{2p+1}\NLtD{f}
\eeq
for some $\kz, \Co, \Cth > 0.$

For $0 < \eps < 1,$ if we take $\Ct = \eps/\Cth$ then it follows that the $h$-finite-element method is $\hk{(p+2)/(p+1)}{(2p+1)/(2p)}$-accurate.

We choose this example to illustrate:
\bit
\item That it is not always the case that $a=b$; here $a > b.$
\item That sometimes (as in this case) $a > b.$ I.e., whilst the mesh constraint `$hk^b$ sufficiently small' makes the finite-element error small, one can only apply this mesh constraint under the more restrictive mesh constraint `$hk^a$ sufficiently small'. Therefore, this result is somewhat limited in its practical applicability.
\eit
\ere

\subsubsection{Optimal mesh conditions for the finite-element method}

We now provide a brief overview of the optimal values of $a$ and $b$ for $\hk{a}{b}$-accuracy, $\hk{a}{b}$-data-accuracy, and $\hka{a}$-quasi-optimality. By `optimal', we mean the values of $a$ and $b$ that are smallest; these will correspond to the least restrictive conditions on the mesh size $h$. We will also, where possible, comment on whether these optimal conditions have been shown to be sharp. The literature reviews in this, and the next, section draw heavily on the literature reviews in \cite[pp. 182--183]{GrLoMeSp:14} and \cite[p. 112]{DiMoSp:19}. Unless otherwise stated, all the problems treated in this literature review are nontrapping (in the wider sense discussed under `\techtitle' in \cref{sec:wpdisc}), have constant coefficients, and and have a impedance boundary condition on at least part of the boundary.

\paragraph{$\hk{a}b$-accuracy} In 1-d the $h$-finite-element method has been proved to be $\hka{3/2}$-accurate for first-order finite elements by Ihlenburg and Babu\v{s}ka in \cite[Theorem 5 and Equation (3.25)]{IhBa:95a} and \cite[Equation (4.5.15)]{Ih:98}. For higher-order finite-elements (still in 1-d) they proved the $h$-finite-element method is $\hka{(2p+1)/(2p)}$-accurate in \cite[Corollary 3.2]{IhBa:97} and \cite[Theorem 4.27 and Equation 4.7.41]{Ih:98}. However, all the above results were only measured in the $H^1$ seminorm (and so we are slightly abusing the notation of $\hka{a}$-accuracy here), and the higher-order results were proved under the assumptions that $f \in \HpmoD,$ $u \in \HppoD,$ and $\SNHppoD{u} \sim k^p \SNHoD{u}.$ In addition, these results were proved using the properties of the `discrete Green's function' for the Helmholtz equation. This function was used to first prove an a priori bound on the finite-element approximation of the solution, and the error bounds were then concluded from this a priori bound. As the discrete Green's function is only known explicitly in 1-d, these results have not been generalised to higher dimensions.

These $\hk{a}b$-accuracy results were confirmed numerically (for $p=1$) by \cite[Figure 11]{IhBa:95a} and \cite[Figure 4.13]{Ih:98}, which showed that the relative error is bounded if $h \sim k^{-3/2}$ and by \cite[Figure 4.10]{Ih:98}, which showed that the relative error is not bounded if $h \sim k^{-1}.$ We note that the proofs in \cite{IhBa:95a,Ih:98} showed the finite-element error was bounded by $\SNHppoD{u}$, and then used the fact that
\beq\label{eq:ratio1d}
\frac{\SNHppoD{u}}{\SNHoD{u}} \sim k^p
\eeq
to bound the finite-element error by $\SNHoD{u},$ and hence conclude a bound on the relative error.

The relation \cref{eq:ratio1d} follows because the solution of the Helmholtz equation in 1-d is given by
\beq\label{eq:soln1d}
A \cos(kx) + B \sin(kx)
\eeq
(for some constants $A$ and $B$). Because \cref{eq:soln1d} only holds in 1-d, and therefore one can, in general, only prove \cref{eq:ratio1d} in 1-d, the proofs of $\hka{(2p+1)/(2p)}$-accuracy have not been extended to higher dimensions, although we conjecture they are true. The only computational results for higher dimensions are those by Bayliss, Goldstein, and Turkel, who showed in \cite[Section 3, Tables 1--3]{BaGoTu:85} that, for low wavenumbers $k \in (4.16,8.32)$ the relative error for first-order finite-elements is bounded if $h \sim k^{-3/2}$, but is not bounded if $h \sim k^{-1}.$

\paragraph{$\hk{a}b$-data-accuracy} In contrast to $\hk{a}b$-accuracy, there are many proofs (for different polynomial degrees and for different Helmholtz problems) of $\hk{a}b$-data-accuracy. The reason for far more proofs in this case is that one can usually show that the interpolation error $u -\Ih u$ is bounded by the data, and one can use this bound when bounding the finite-element error to conclude the finite-element error is bounded by the data.

The best results known to date are the same (in terms of $a$ and $b$) as those for $\hk{a}b$-accuracy, except results for $\hk{a}b$-data-accuracy hold in higher dimensions. In \cite[Theorem 5.1, Corollary 5.2]{DuWu:15} Du and Wu proved that the $h$-finite-element method for the IIP is $\hka{(2p+1)/(2p)}$-data-accurate for arbitrary (fixed) polynomial degree $p$ and $d = 2$ or $3$, provided $u \in \HppoD$ (although this result can be shown for lower-regularity solutions by combining \cite[Theorem 5.1]{DuWu:15} with \cite[Lemma 3.5]{DuWu:15}). This result was proved for the IIP in $d = 2$ or $3$ for $p=1$ by Wu in \cite{Wu:14}. Whilst there is no computational evidence to confirm the sharpness of this result, given it coincides with the (sharp) best case for $\hk{a}b$-accuracy, we expect the result for $\hk{a}b$-data-accuracy is also sharp.

\paragraph{$\hka{a}$-quasi-optimality} Finally, the best known result for $\hka{a}$-quasi-optimality for the Helmholtz equation is that the $h$-finite-element method is $\hka{(p+1)/p}$-quasi-optimal. This was first proved for $p=1$ in 1-d by Aziz, Kellog, and Stevens in \cite[Theorem 3.1]{AzKeSt:88} and Ihlenburg and Babu\v{s}ka in \cite[Theorem 3]{IhBa:95a} and \cite[Theorems 4.9 and 4.13]{Ih:98}, with \cite[Figures 7-9]{IhBa:95a} and \cite[Section 4.5.4 and Figures 4.11-4.12]{Ih:98} showing this result is sharp in 1-d (although Ihlenburg and Babu\v{s}ka only work in the $H^1$-semi norm). The result was proved for $p=1$ and $d=2$ for the IIP by Melenk \cite[Proposition 8.2.7]{Me:95}, and for higher-order finite elements (for the full-space problem, IIP, and EDP) by Melenk and Sauter in \cite[Corollary 5.6]{MeSa:10} and \cite[Theorem 5.8]{MeSa:11} respectively. This result was shown for a PML problem by Chaumont-Frelet, Gallistl, Nicaise, and Tomezyk in \cite[Theorem 5.1]{ChGaNiTo:18}, and extended to arbitrary time-harmonic wave propagation problems by Chaumont-Frelet and Nicaise in \cite[Theorem 2.15]{ChNi:19}, who showed that if the constant in the a priori bound \cref{eq:bgbound} grows like $k^\alpha,$ then the $h$-finite-element method is $\hka{(p+\alpha+1)/p}$-accurate. They give numerical experiments in \cite[Figure 8]{ChNi:19} showing the sharpness of $\hka{(p+1)/p}$-quasi-optimality for $p=1, 2$ for a heterogeneous IIP.

\paragraph{Link with dispersion error} When computing on regular grids, one can mathemtically analyse the `dispersion error' of the finite-element method for the Helmholtz equation; i.e., the difference between the wavenumber of the true solution $u$ (i.e., $k$) and the wavenumber of the approximation $\uh.$ We mention briefly that Ainsworth  analysed the dispersion error for the $hp$-finite-element method for the Helmholtz equation and proved the dispersion error is of the order $h^{2p}k^{2p+1}$ \cite[Equation (3.5)]{Ai:04}. Observe that this is the same order as the best-available results for $\hka{a}$-accuracy and $\hka{a}$-data-accuracy discussed above. Therefore, results from finite-element error analysis and dispersion analysis both suggest that the error (measured in a suitable sense in each case) is bounded if $h^{2p}k^{2p+1}$ is sufficiently small. See, e.g., \cite[Remark 5.3(a)]{DuWu:15} for more references on dispersion error analysis for the Helmholtz equation.

\paragraph{Observations} Observe that, in general, the mesh conditions required for quasi-optimality are more restrictive than those required for the error to be bounded. I.e., in general, if the finite-element method is $\hka{\ao}$-quasi-optimal and $\hka{\at}$-data-accurate, then $\ao > \at.$ Therefore, whilst one can show the finite-element method is $\hka{a}$-data-accurate (for some $a$) by first showing quasi-optimality, this approach will result in a more restrictive mesh condition than proving $\hka{a}$-data-accuracy directly. Also observe that the results for higher-order finite elements become less stringent as $p$ increases, i.e., the finite-element method is $\hka{(2p+1)/(2p)}$-data-accurate, and $(2p+1)/(2p) \downarrow 1$ as $p \rightarrow \infty.$

\paragraph{Complete summary of results in the literature} In \cref{tab:acc,tab:dataacc,tab:qo} we list all of the mathematical (as opposed to computational) results in the literature for $\hk{a}b$-accuracy, $\hk{a}b$-data-accuracy, and $\hka{a}$-quasi-optimality. We list these in chronological order, with any relevant comments in the `Notes' column. The `Proof technique' column details the method used in the proof; see \cref{sec:prooftechniques} below for an extended discussion of these techniques. However, we now make a few general comments on the history of these results.

Recall that proving quasi-optimality (or an error bound) for the finite-element method for the Helmholtz equation is more tricky than for the stationary diffusion equation \cref{eq:stdiff}. For the stationary diffusion equation, one immediately obtains quasi-optimality for \emph{any} mesh by C\'ea's Lemma (and one then obtains that the relative error is bounded by \cref{lem:scottzhang}). We emphasise again that this result holds for any mesh, with no restriction on $h.$

However, C\'ea's Lemma relies on the coercivity of the sesquilinear form, and the sesquilinear forms arising from standard discretisations of the Helmholtz equation are not coercive for large $k$. Therefore, to prove quasi-optimality, one instead uses the so-called Schatz argument, a modification of the standard Aubin--Nitsche duality argument\footnote{First introduced by Aubin \cite{Au:67} and Nitsche \cite{Ni:68} for coercive problems.} However, using the Schatz argument, one only obtains quasi-optimality under some $k$-dependent restriction on the mresh size $h.$

The Schatz argument was first used for problems satisfying a G\r{a}rding inequality by Schatz \cite{Sc:74} and first used for Helmholtz problems by Aziz, Kellogg, and Stephens \cite{AzKeSt:88}. In \cite{AzKeSt:88} they proved that in 1-d the finite-element method for the Helmholtz equation is $\hka{2}$-quasi-optimal, and this result was extended to $d=2$ by Melenk \cite{Me:95} in his PhD thesis. However, the Schatz argument was first presented in the framework we use below by Sauter \cite[Section 2]{Sa:06}. Observe that for large values of $k,$ $\hka{2}$-quasi-optimality and the related mesh restriction `$hk^2$ is sufficiently small' is computationally prohibitive---it would result in linear systems of size, e.g., $~10^{12}$, for the Helmholtz equation with $k=100$ in 3-d.

Because of the severe mesh restrictions required for quasi-optimality for Helmholtz problems, recent research efforts have been focused on directly proving error bounds for the finite-element method. The key proof technique has been so-called elliptic projection ideas; these ideas are at the heart of our results in \cref{sec:fem} below, and are discussed in more detail in \cref{sec:prooftechniques} below. To our knowledge, the first use of elliptic projections to prove error estimates for Galerkin approximations was by Wheeler\footnote{Mentioned in \cite{MaNo:03}.} \cite[Theorem 3.1 ff.]{Wh:73}, who proved bounds on the error for a nonlinear parabolic problem by splitting the error into the error from an elliptic projection and the remaining error between the elliptic projection and the Galerkin approximation. The idea of using elliptic projections for Helmholtz problems was first introduced to prove error bounds for discontinuous Galerkin methods for \cref{prob:tedp} by Feng and Wu \cite{FeWu:09,FeWu:11}, and then used for standard finite-element methods (or closely-related continuous-interior-penalty methods) beginning with the work of Wu and Zhu \cite{ZhWu:13,Wu:14}.

In this review we do not address $hp$-finite-element methods for the Helmholtz equation, that is, finite-element methods where one refines both $h$ and $p$. Such methods can be very effective; the error for such methods can converge exponentially with respect to the number of degrees of freedom (see, e.g., \cite[Theorem 4.51]{Sc:98}). However, rigorous error analysis for these methods requires finite-element-error bounds that are completely explicit in both $h$ and $p$.

The landmark results on $hp$-finite-element methods for the Helmholtz equation were achieved by Melenk and Sauter \cite{MeSa:10,MeSa:11} who used a novel splitting of the solution of the Helmholtz equation (see the comments underneath \cref{rem:osc} below) to show that the $hp$-finite-element method is quasi-optimal if $hk/p \leq \co$ and $p \geq \ct \ln k$ (for some constants $\co,\ct>0$). Melenk and Sauter proved this result for the full-space problem in \cite{MeSa:10} and for: (i) the exterior Dirichlet problem, and (ii) the interior impedance problem in an analytic domain or a 2-d convex polygon in \cite{MeSa:11}. These results were generalised to an arbitrary 2-d Lipschitz domain by Esterhazy and Melenk in \cite[Theorem 4.2]{EsMe:12}. Other results in the literature for $hp$-finite-element methods are those of Zhu and Wu \cite[Equation (1.7)]{ZhWu:13} who showed the $hp$-finite-element method for the Helmholtz equation is data-accurate\footnote{We do not use the $\hk{a}b$-accurate, etc., terminology for $hp$-methods, as it does not represent the interplay between the polynomial degree $p$ and the wavenumber $k$.} provided
\beqs
\frac{kh}p \leq C_0\mleft(\frac{p}k\mright)^{\frac1{p+1}},
\eeqs
where $C_0>0$ is some constant. Finally, in \cite[Corollary 3.2]{IhBa:97} Ihlenburg and Babu\v{s}ka proved a bound on the error for the finite-element method in 1-d that is explicit in $k,$ $h$, and $p$, although the bound is probably not sharp in $p,$ and therefore could not be used to prove results on $hp$-finite-element methods.

\afterpage{% Heard about from https://tex.stackexchange.com/questions/11471/how-to-wrap-text-around-landscape-page
\begin{landscape} % Heard about from https://tex.stackexchange.com/questions/19017/how-to-place-a-table-on-a-new-page-with-landscape-orientation-without-clearing-t/19021#19021 and https://tex.stackexchange.com/questions/25369/how-to-rotate-a-table 
\begin{table}[h]
  \centering
  \begin{threeparttable}[c]
    \begin{tabular}{Sc Sc Sc Sc}
  \toprule
  & $\hk{a}b$-accuracy & Notes & Proof technique\\
   \midrule
    \cite[Equation (3.25)]{IhBa:95a} & $\hk{1}{3/2}$ & \makecell{$d=1$, unit interval,\\$u(0)=0,$\\impedance boundary condition at $1$\\$H^1$ seminorm}&\makecell{Discrete Green's function\\(specific to $d=1$)}\\
    \cite[Theorem 4]{IhBa:95b} & $\hk{1}{3/2}$ & \makecell{$d=1$, unit interval,\\$u(0)=0,$\\impedance boundary condition at $1$\\$L^2$ norm\tnote{1}}&\makecell{Discrete Green's function\\error splitting using interpolant}\\
      \cite[Corollary 3.2]{IhBa:97}&$\hk{1}{(2p+1)/(2p)}$ & \makecell{$d=1$, unit interval,\\$u(0)=0,$\\impedance boundary condition at $1$\\$H^1$ seminorm}&\makecell{Discrete Green's function,\\error splitting using interpolant}\\
  \cite[Theorem 4.13 and equation (4.5.15)]{Ih:98}& $\hk{1}{3/2}$ & \makecell{$d=1$, unit interval,\\$u(0)=0,$\\impedance boundary condition at $1$\\$H^1$ seminorm}& Discrete Green's function\\
  \cite[Theorem 4.27 and equation (4.7.41)]{Ih:98}& $\hk{1}{(2p+1)/(2p)}$ & \makecell{$d=1$, unit interval,\\$u(0)=0,$\\impedance boundary condition at $1$\\$H^1$ seminorm}&\makecell{Discrete Green's function,\\error splitting using interpolant}\\ 
  \bottomrule
  \end{tabular}
  \begin{tablenotes}
\item [1] Actually, \cite[Theorem 4]{IhBa:95b} only proves a bound on the $L^2$-norm of the error in terms of the $H^2$-seminorm of the solution. However, when $d=1$, $\SNHtD{u} \sim k^2\NLtD{u}$ and so one can conclude $\hk{a}b$-accuracy.
  \end{tablenotes}
    \caption{$\hk{a}b$-accuracy for $h$-finite-element discretisations of the Helmholtz equation}\label{tab:acc}
\end{threeparttable}
\end{table}

\begin{table}[h]
  \centering
\begin{tabular}{Sc Sc Sc Sc}
  \toprule
 & $\hk{a}b$-data-accuracy  & Notes & Proof technique\\
  \midrule
  \cite[Theorem 5]{IhBa:95a} & $\hk{1}{3/2}$ & \makecell{$d=1$, unit interval,\\$u(0)=0,$\\impedance boundary condition at $1$}&\makecell{Discrete Green's function\\error splitting using interpolant}\\
      \cite[Corollary 4.2]{ZhWu:13}& $\hk{(p+2)/(p+1)}{(2p+1)/(2p)}$   &\makecell{$d=2,3$, IIP,\\$D$ smooth,\\bounds obtained for $hp$-finite-element method,\\so fully $p$-explicit}& Modified Schatz\\
      \cite[Theorem 5.1]{Wu:14} & $\hka{3/2}$  &\makecell{$d=2,3$, IIP,\\$D$ a star-shaped polygon/polyhedron}& Error splitting\\
      \cite[Corollary 5.2]{DuWu:15} & $\hka{(2p+1)/(2p)}$ &  $d=2,3$, IIP, $D$ smooth, star-shaped& Error splitting\\
        \cite[Theorem 5.2]{ChNi:18}& $\hk{3/(1+\alpha)}{3/2}$   &\makecell{$d=2$, TEDP with re-entrant corners,\\$\alpha \in (1/2,1)$ related to\\strength of corner singularities}& Modified Schatz\\
        \cite[Theorem 4.3 and Remark 4.2(iv)]{LiWu:18}& $\hka{3/2}$   &\makecell{$d = 1,2,3$,\\full-space problem truncated with PML,\\posed in a ball}& Modified Schatz\\
        \cite[Lemma 3.3]{WuZo:18}&$\hka{3/2}$&\makecell{IIP, $D$ convex,\\$n$ heterogeneous with $\NLiD{n-1} \lesssim 1/k$,\\part of an argument for\\a nonlinear heterogeneous Helmholtz problem}&Error splitting\\
  \cite[Theorem 5.4]{ChGaNiTo:18}&$\hk{(p+2)/(p+1)}{(2p+1)/(2p)}$  & $d=2$, EDP truncated with PML, posed in a ball& Modified Schatz\\
\bottomrule
\end{tabular}
\caption{$\hk{a}b$-data-accuracy for $h$-finite-element discretisations of the Helmholtz equation}\label{tab:dataacc}
\end{table}


  \begin{table}[h]
    \centering
\begin{tabular}{Sc Sc Sc Sc}
  \toprule
& $\hka{a}$-quasi-optimality & Notes & Proof technique\\
  \midrule
  \cite[Theorem 3.1]{AzKeSt:88} & $\hka{2}$ & $d=1$ & Schatz\\
  %?  \cite{DoSaShSc:93}\\
  \cite[Theorem 3]{IhBa:95a} & $\hka{2}$ & $d=1,$ $H^1$ seminorm & Schatz\\
  \cite[Corollary 2]{IhBa:95a} & $\hka{0}$ & $d=1,$ $H^1$ seminorm &\makecell{Discrete Green's function\\error splitting using interpolant}\\
  %? \cite{IhBa:95b}
  \cite[Theorems 4.9 and 4.13]{Ih:98} & $\hka{2}$ & \makecell{$d=1,$ $H^1$ seminorm,\\\cite[Theorem 4.9]{Ih:98} is \cite[Theorem 3]{IhBa:95a}} & \makecell{Schatz, and error splitting\\using interpolant,\\respectively}\\
  \cite[Proposition 8.2.7]{Me:95} & $\hka{2}$ & \makecell{$d=2,$\\IIP,\\$D$ smooth and star-shaped or convex} & Schatz\\
  \cite[Corollary 5.6]{MeSa:10} & $\hka{(p+1)/p}$ & Full-space problem & Schatz\\
%  \cite[Theorem 5.8]{MeSa:11} & $\hka{(p+1)/p}$ & \makecell{$d = 2,3$,\\polynomial growth of constant in a priori bound,\\IIP ($\Dm$ analytic or Lipschitz convex polygon)\\and EDP ($\Dm$ analytic)} & Schatz\\
  \cite[Theorem 5.1]{ChNi:18} & $\hka{2}$ & $d=2$, TEDP with re-entrant corners & Schatz\\
  \cite[Theorem 5.1]{ChGaNiTo:18} & $\hka{(p+1)/p}$&$d=2$, EDP truncated with PML, posed in a ball  &Schatz\\
    \cite{ChNi:19} & $\hka{(p+\alpha+1)/p}$ & \makecell{Arbitrary time-harmonic wave problem,\\a priori bound grows at rate $k^\alpha$} & Schatz\\
    \cite[Theorems 4.2 and 4.5, Remark 4.6(ii)]{GrSa:18} &$\hka{2}$ & \makecell{IIP, $D$ Lipschitz, star-shaped w.r.t. a ball,\\$n$ heterogenenous\\constants fully explicit in $n$}& Schatz\\
    \cite[Theorem 3]{GaSpWu:18} & $\hka{2}$ & \makecell{EDP,\\$A$ and $n$ heterogeneous,\\$\Dm,$ $A$, and $n$ $C^\infty$,\\constants fully explicit in $A$ and $n$} & Schatz\\
\bottomrule
\end{tabular}
\caption{$\hka{a}$-quasi-optimality for $h$-finite-element discretisations of the Helmholtz equation}\label{tab:qo}
\end{table}
\end{landscape}
}
\subsection{Extended discussion of proof techniques for finite-element errors for the Helmholtz equation}\label{sec:prooftechniques}
We now discuss in some detail the proof techniques used to obtain conditions under which the finite-element method applied to the Helmholtz equation either (i) is quasi-optimal in the weighted $H^1$ norm, or (ii) has bounded error in the weighted $H^1$ norm. We discuss these techniques because they are related, but different, and we hope that discussing the details, and comparing and contrasting these techniques will prove helpful to the research community, and will increase understanding of proofs that are, at times, a little technical. However, we also discuss these results to set the coefficient-explicit bounds in \cref{sec:fem} below in context.

We note that frequently quasi-optimality results are refereed to as \defn{asymptotic} error estimates, and bounded-error results (when they are proved under weaker mesh constraints than asymptotic result) are referred to as \defn{pre-asymptotic} error estimates. This terminology is used because the mesh conditions to ensure quasi-optimality are more restrictive than those for bounded error, and therefore they hold for smaller values of $h.$

For simplicity's sake, our exposition below will assume that we are treating the homogeneous Helmholtz TEDP (i.e., $A=I$ and $n=1$) with $\gI = 0,$ and that the problem is nontrapping\footnote{Recall that we say that the problem is nontrapping if we have a $k$-independent a priori bound, i.e., $C$ in \cref{eq:bgbound} is independent of $k.$}. Also, we suppress all of the constants involved, instead opting to use $\lesssim$ notation, where $a \lesssim b$ if $a \leq C b,$ with $C$ independent of $k$ and $h$. The new results we present in \cref{sec:fem} consider heterogeneous problems that may be trapping, and state all of the constants involved explicitly, at the price of being more technical to state.

\subsubsection{Merits and drawbacks of each argument}
We first briefly discuss the positive and negative points for each of the two classes of argument we will outline below: (modified) duality arguments and splitting arguments.

The merits of duality arguments are their simplicity---we will state the arguments almost in their entirety below, and they are closely related with familiar duality arguments used in finite-element-error analysis for more familiar problems, such as the stationary diffusion equation. In addition, using these arguments for higher-order finite elements (or $hp$-finite-elements) one can easily conclude error bounds under certain mesh conditions, where both the bounds and the conditions are completely $h$-, $p$-, and $k$-explicit.% Such an analysis was performed for quasi-optimality of $hp$-finite-element methods for the Helmholtz equation in homogeneous media by Melenk and Sauter \cite{MeSa:10,MeSa:11}.

However, the main drawback of duality arguments is their lack of sharpness in the conditions imposed on $h$. For example, as detailed in \cref{rem:dataacc}, Chaumont-Frelet, Gallistl, Nicaise and Tomezyk used a duality argument in \cite{ChGaNiTo:18} to prove that for a given Helmholtz problem the pollution term is of the order $h^2k + h^{2p}k^{2p+1}$ provided $h^{p+1}k^{p+2}$ is sufficiently small. I.e., the mesh restriction under which they prove a finite-element error bound (`$h^{p+1}k^{p+2}$ is sufficiently small') is \emph{more restrictive} than the mesh restriction needed to bound the pollution error uniformly in $k$ (namely `$h^{2p}k^{2p+1}$ is sufficiently small').

In contrast, the merit of splitting arguments is that they can give mesh conditions that are sharp in their $h$-dependence; in \cite{DuWu:15}, Du and Wu proved that the pollution term in the finite-element error is of the order $h^{2p}k^{2p+1},$ under the restriction that $h^{2p}k^{2p+1}$ is sufficiently small, i.e., the mesh restriction under which they prove a finite-element error bound is \emph{of the same order} as the mesh restriction needed to bound the pollution error uniformly in $k.$

However, the drawback of splitting arguments is their complexity. They involve proving bounds on the solution of discrete Helmholtz problems; such bounds are complicated to prove, especially in the higher-order cases, and the constants involved depend on the polynomial degree $p$ in highly complicated ways. These drawbacks limit the likelihood that such bounds can be used for $hp$-finite-element methods, where the dependence on the polynomial degree must be known explicitly.


\subsubsection{(Modified) Duality arguments}
The first class of arguments we consider are (modified) duality arguments. In these arguments, one uses a G\r{a}rding inequality satisfied by the Helmholtz equation to show that the error in the weighted $H^1$ norm is bounded by the best approximation error, plus error in the $L^2$ norm\footnote{Recall that for the stationary diffusion equation, one can show the finite-element error is bounded by the approximation error simply using coercivity and boundedness of the bilinear form---this is Cea's Lemma.}. One then uses a modification of the standard Aubin--Nitsche duality argument to bound the $L^2$ norm.

To show the error in the weighted $H^1$ norm is bounded by the error in the $L^2$ norm, observe
\begin{align}
\NW{u-\uh}^2 &\leq \Re{a(u-\uh,u-\uh)} + k^2 \NLtD{u-\uh}^2\nonumber\\
&= \Re{a(u-\uh,u-\vh)} + k^2 \NLtD{u-\uh}^2\text{ by Galerkin orthogonality}\nonumber\\
&\lesssim \NW{u-\uh}\NW{u-\vh} + k^2 \NLtD{u-\uh}^2,\label{eq:gardingerror}
\end{align}
for any $\vh \in \Vhp.$

The Schatz argument to prove quasi-optimality, outlined below, proceeds to show (for $p=1$)
\beq\label{eq:schatzbound}
k\NLtD{u-\uh} \lesssim hk^2 \NW{u-\uh},
\eeq
and combining \cref{eq:gardingerror,eq:schatzbound}, one has
\beqs
\NW{u-\uh}^2 \lesssim \NW{u-\uh}\NW{u-\vh} + \mleft(hk^2\mright)^2\NW{u-\uh}^2
\eeqs
cancelling a factor $\NW{u-\uh}$ from each side, and moving the final term onto the left-hand side. one obtains quasi-optimality if $hk^2$ is sufficiently small.

In contrast, so-called \defn{elliptic-projection-modified Schatz} (`modified Schatz' for short) arguments used to prove a finite-element error bound, also outlined below, show (for $p=1$)
\beq\label{eq:epbound}
k\NLtD{u-\uh} \lesssim h^2k^3 \NLtD{f},
\eeq
and then combine \cref{eq:gardingerror,eq:epbound} to show the finite-element error is bounded if $h^2k^3$ is sufficiently small---observe this is less restrictive than the requirement $hk^2$ is sufficiently small.
%% \begin{align}
%% \NW{u-\uh}^2 &\lesssim \NW{u-\uh}\NW{u-\vh} + h^2k^3 \NLtD{f}\nonumber\\
%% &\leq \eps \NW{u-\uh}^2 + \frac1\eps \NW{u-\vh}^2 +\mleft(h^2 k^3 \NLtD{f}\mright)^2\label{eq:epboundpart1}
%% \end{align}
%% Taking $\eps$ sufficiently small, and moving the first term in \cref{eq:epboundpart1} to the left-hand side, taking $\vh$ to be the best approximation of $u$ in $\Vhp,$ and using \cref{lem:scottzhang} and the fact that\footnote{For sufficiently smooth $\GD$ and $\GI$---see \cite[Remark 2.14]{GrPeSp:19}} we can show $\NHtD{u} \lesssim k\NLtD{f}$
%% \beqs
%% \NW{u-\uh} \lesssim \mleft(hk + h^2k^3\mright)\NLtD{f}.
%% \eeqs

We will now look over the Schatz argument in more detail, before considering modified Schatz arguments. We first establish some notation that will enable us to discuss best approximation errors for solutions of Helmholtz problems. We let $\solfem:\LtD\rightarrow \HoD$ denote the solution operator for \cref{prob:vtedp} with zero impedance boundary condition, and we let $\solfems$ denote the solution operator for the corresponding adjoint problem. That is, for any $\ftilde \in \LtD$ and for all $v \in \HozDD,$
\beqs
a(\solfem(\ftilde),v) = \IPLtD{\ftilde}{v}
\eeqs
and
\beqs
a(v,\solfems(\ftilde)) = \IPLtD{v}{\ftilde}.
\eeqs
One can, of course, define the corresponding solution operators in the case of non-zero impedance boundary conditions. We next define the approximability constants:
\beqs
\wba \de \sup_{\ftilde \in \LtD}\inf_{\vh \in \Vhp} \frac{\NW{\solfem(\ftilde) - \vh}}{\NLtD{\ftilde}}
\eeqs
and
\beqs
\wbaadj \de \sup_{\ftilde \in \LtD}\inf_{\vh \in \Vhp} \frac{\NW{\solfems(\ftilde) - \vh}}{\NLtD{\ftilde}}.
\eeqs
We are changing notation slightly from that normally prevelant in the literature; notation for $\wbaadj$ was first introduced by Sauter in \cite[Section 2.2]{Sa:06}, but $\wbaadj$ was instead denoted $\tilde{\eta}$ in \cite{Sa:06}. Later, $\wbaadj$ was denoted $\eta$ in \cite[Equation (4.5)]{MeSa:10}.  When dealing purely with the Schatz argument for quasi-optimality (as in \cite[Section 2.2]{Sa:06}) one only needs consider the approximation of adjoint problems in the duality-argument step, and hence one only needs notation for $\wbaadj$. However, in our exposition of elliptic-projection-based arguments below, we will also need to consider the approximation of standard Helmholtz problems, and hence we adopt the notation above.

\paragraph{The Schatz argument} We now recap the Schatz argument for proving quasi-optimality for an indefinite sesquilinear form. This recap will allow us to see where $\wbaadj$ enters the argument, and will also allow us to compare and contrast this argument with the arguments used to obtain a bounded finite-element error using an elliptic projection.

We let $\xi \in \HozDD$ solve the adjoint Helmholtz problem
\beqs
a(v,\xi) = \IPLtD{v}{u-\uh} \tforall v \in \HozDD.
\eeqs
Then, taking $v = u-\uh,$ we have
\begin{align}
  \NLtD{u-\uh}^2 &= a(u-\uh,\xi)\label{eq:schatz0}\\
  &=a(u-\uh,\xi - \vh) \text{ by Galerkin orthogonality for } u-\uh, \text{ for any } \vh \in \Vhp\nonumber\\
  &\lesssim \NW{u-\uh}\NW{\xi-\vh}\nonumber\\
  &\lesssim \NW{u-\uh} \wbaadj \NLtD{u-\uh}\nonumber.
\end{align}
Cancelling a factor of $\NLtD{u-\uh},$ we obtain
\beq\label{eq:schatz1}
\NLtD{u-\uh} \lesssim \wbaadj \NW{u-\uh}.
\eeq
Combining \cref{eq:gardingerror,eq:schatz1}, we have
\beqs
\NW{u-\uh}^2 \lesssim \NW{u-\uh}\NW{u-\vh} + \mleft(k\wbaadj\mright)^2 \NW{u-\uh}^2,
\eeqs
and hence by cancelling a factor $\NW{u-\uh}$ and taking the final term on to the left-hand side, we obtain
\beqs
\NW{u-\uh} \lesssim \inf_{\vh \in \Vhp} \NW{u-\vh} \tif k\wbaadj \text{ is sufficiently small}.
\eeqs
All results showing quasi-optimality for the Helmholtz equation (for different finite-element spaces and different domains) can then be seen as simply obtaining estimates on $\eta$ in these different scenarios; this literature is summarised in \cref{tab:qo,tab:acc,tab:dataacc} below.

\paragraph{Modified Schatz arguments} We now move on to compare and contrast the Schatz argument given above with elliptic-projection-modified-Schatz arguments. The main difference in the result of these arguments, compared to the Schatz argument given above, is that modified Schatz arguments only prove a \emph{bound} on the finite-element error, rather than proving quasi-optimality. However, these bounds are obtained under mesh restrictions that are \emph{less restrictive} than those required for quasi-optimality.

The elliptic projection of a function $ w \in \HozDD$ is the finite-element function $\Ph w$ that has the same action as $w$ on the finite-element space $\Vhp$ under some elliptic operator\footnote{The definition of the elliptic projection is dependent on the exact sesquilinear form used in the discretisation of the Helmholtz equation, and on the norm one is using to measure the error. For example, elliptic projection arguments originated in the study of discontinuous Galerkin methods for the Helmholtz equation (in \cite{FeWu:09,FeWu:11}); therefore the sesquilinear forms associated with the discretisation included penalty terms. These penalty terms were incorporated into the sesquilinear form $\aep$ and the norms used to measure the error also included these penalty terms. In the following exposition, we will work with standard finite-element discretisations of the Helmholtz equation and standard Sobolev norms, and so the elliptic projections we used will be based on this setting.}; i.e., $\Ph w$ is defined by
\beqs
\aep(\vh,\Ph w) = \aep(\vh,w) \tforall \vh \in \Vhp,
\eeqs
for some sesquilinear form $\aep$, such that $\Ph$ is well-defined. For Helmholtz problems\footnote{Recall we are considering discretisations of \cref{prob:vtedp}, and therefore the sesquilinear form we use is given by \cref{eq:aT}.}, choices for $\aep(\vo,\vt)$ used in the literature are either
\beq\label{eq:aepho}
\aep(\vo,\vt) = \IPLtD{\grad \vo}{\grad \vt},
\eeq
\beq\label{eq:aeplower}
\aep(\vo,\vt) = \IPLtD{\grad \vo}{\grad \vt} + \IPLtD{\vo}{\vt},
\eeq
or
\beq\label{eq:aepused}
\aep(\vo,\vt) = \IPLtD{\grad \vo}{\grad \vt} - ik\IPLtGI{\vo}{\vt}.
\eeq
These elliptic projections correspond to finding finite-element approximations of the solution of the PDEs
\begin{align}
  \Delta w &= F \tin D,\label{eq:sd1}\\
  w &= 0 \ton \GD,\tand\label{eq:sd2}\\
  \dn w &= 0 \ton \GI\label{eq:sd3};
\end{align}
\begin{align}
  \Delta w + w&= F \tin D,\label{eq:sd4}\\
  w &= 0 \ton \GD,\tand\label{eq:sd5}\\
  \dn w &= 0 \ton \GI;\label{eq:sd6}
\end{align}
or
\begin{align}
  \Delta w &= F \tin D,\label{eq:sd7}\\
  w &= 0 \ton \GD,\tand\label{eq:sd8}\\
  \dn w -ikw &= 0 \ton \GI\label{eq:sd9}
\end{align}
respectively, where $F$ is an appropriately chosen function. In the following exposition, we will assume $\aep$ is given by \cref{eq:aepused}.
Since $\Ph$ is a Galerkin projection, one can show that in its energy norm $\Nep{\cdot}$ (equivalent to the $k$-weighted $H^1$ norm $\NW{\cdot}$, with equivalence constants independent of $k$ via the multiplicative trace inequality and the Poincar\'e--Friedrich's inequality (\cref{lem:poincare} below) $\aep$ is coercive and continuous\footnote{For comment on showing coercivity and continuity for the other definitions of $\aep$, see \cref{rem:epdef} below.}, and hence is quasi-optimal:
\beq\label{eq:epho}
\Nep{w-\Ph w} \lesssim \inf_{\vh \in \Vhp} \Nep{w-\vh}.
\eeq
Also, by the Aubin--Nitsche duality argument (as $\aep$ is coercive),
\beq\label{eq:eplt}
\NLtD{w-\Ph w} \lesssim h \Nep{w-\Ph w}.
\eeq
\ednote{Euan---you said you wanted to have a discussion about what properties exactly are required for the elliptic projection---hopefully the exposition below is a starting point. I've gone with quasi-optimality for the time being, as it means one can see the analogues with the Schatz argument more clearly. But lets chat about this!}

Combining \cref{eq:epho,eq:eplt} is a modification of the Schatz argument above: We start from \cref{eq:schatz0}, but instead of introducing an arbitrary $\vh \in \Vhp$ into the second argument, we instead introduce the elliptic projection $\Ph \xi$ by Galerkin orthogonality for $u-\uh$:
\begin{align}
  \NLtD{u-\uh}^2 &= a(u-\uh,\xi-\Ph\xi)\nonumber\\
  &= \aep(u-\uh,\xi-\Ph\xi) - k^2\IPLtD{u-\uh}{\xi-\Ph\xi}\nonumber\\
  &= \aep(u-\Ih u,\xi-\Ph\xi) - k^2\IPLtD{u-\uh}{\xi-\Ph\xi} \text{ by Galerkin orthogonality for }\xi-\Ph\xi\nonumber\\
  &\lesssim \wbaadj \NLtD{f} \wba \NLtD{u-\uh} + k^2 \NLtD{u-\uh} \NLtD{\xi-\Ph\xi}\text{ by \cref{eq:epho}}\nonumber\\
  &\lesssim \wba\wbaadj \NLtD{f}\NLtD{u-\uh} + k^2 \NLtD{u-\uh} h\wba \NLtD{u-\uh}\text{ by \cref{eq:epho,eq:eplt}.}\label{eq:schatz2}
\end{align}
Therefore if $hk^2\wba$ is sufficiently small, the second term in \cref{eq:schatz2} can be absorbed into the left-hand side, and cancelling a factor of $\NLtD{u-\uh},$ we obtain
\beq\label{eq:ep1}
k \NLtD{u-\uh} \lesssim k \wba\wbaadj \NLtD{f} \tif hk^2\wba \text{ is sufficiently small,}
\eeq
this is \cref{eq:epbound}. (We consider $k\NLtD{u-\uh}$ so as to compare with the quasi-optimality results above, as the $L^2$ term in $\NW{\cdot}$ is also multiplied by a factor $k$.)

To obtain a bound on the error in the weighted $H^1$ norm\ednote{Both, Li and Wu \cite[Text after equation 4.15]{LiWu:18} use an error-spliiting-type technique to bound the $H^1$ error in terms of the $L^2$ error, but this only works because they're doing a PML problem, and so don't have a term defined on $\GI$. Would you mention this?}, we put \cref{eq:ep1} into \cref{eq:gardingerror} to get
\beqs
\NW{u-\uh}^2 \lesssim \NW{u-\uh}\NW{u-\vh} + \mleft(k\wba\wbaadj\mright)^2 \NLtD{f}^2,
\eeqs
and taking $\vh = \Ih v,$ by \cref{lem:scottzhang} we have
\beqs
\NW{u-\uh}^2 \lesssim \NW{u-\uh}\wba\NLtD{f} + \mleft(k\wba\wbaadj\mright)^2 \NLtD{f}^2,
\eeqs
We then obtain for any $\eps > 0$ by Cauchy's inequality
\beq\label{eq:ep2}
\NW{u-\uh}^2 \lesssim \eps \NW{u-\uh}^2 + \frac1\eps\wba^2 \NLtD{f}^2 + \mleft(k\wba\wbaadj\mright)^2 \NLtD{f}^2.
\eeq
Taking $\eps$ sufficiently small, moving the first term on the right-hand side of \cref{eq:ep2} to the left-hand side, and taking a square root, we obtain
\beq\label{eq:ep3}
\NW{u-\uh} \lesssim\mleft(\wba  + k\wba\wbaadj\mright) \NLtD{f} \tif hk^2\wba\text{ is sufficiently small}.
\eeq

The term $\wba\NLtD{f}$ on the right-hand side of \cref{eq:ep3} is, up to a constant, the best-approximation error for $u$ (i.e., the error when one interpolates/quasi-interpolates $u$---see \cref{sec:helmfedisc,lem:scottzhang} above), and the term $k\wba^2$ is the \defn{pollution} term arising from the numerical method. As for the quasi-optimality results above, we remark that different bounds on the finite-element error (for different finite-element space $\Vhp$ and different domains) can be thought of as proving bounds on $\eta$ in these different situations.
\ednote{Both---I don't yet have a clear explanation of \emph{why} this works. Currently, it seems just like a trick to me. I don't know if that's an issue!}
\subsubsection{Error-splitting arguments}\label{sec:errorsplit}
The second class of arguments used in the literature are error-splitting arguments, used to prove a bound on the finite-element error. In these arguments the finite-element error is split using the elliptic projection of the solution $u$. To begin, we make the trivial observation that
\beqs\label{eq:split1}
u-\uh = \mleft(u-\Ph u\mright) + \mleft(\Ph u - \uh\mright).
\eeqs
Proving an error bound for $u-\uh$ therefore reduces to proving an error bound for the elliptic projection error $u- \Ph u$ and proving a bound on the difference $\Ph u - \uh.$ The former can either be accomplished by showing quasi-optimality of the elliptic projection (as above) or by proving such an error bound directly. The former approach is taken in \cite{DuWu:15,LiWu:18,ChGaNiTo:18}, and the latter approach is taken in \cite[Lemma 5.2]{FeWu:09} (althought the proof is only contained in \cite[Lemma 5.2]{FeWu:08}), \cite[Lemma 4.3]{FeWu:11}, and \cite[Lemma 4.2]{Wu:14}. In \cite{FeWu:09,FeWu:11,Wu:14} the bound on the elliptic projection error is proved by observing that the sesquilinear form $\IPLtD{\grad \vo}{\grad \vt}$ is coercive on $\HozDD$, and then also controlling the additional term arising from the impedance boundary condition.

To bound the difference $\Ph u - \uh,$ one first shows that it solves a deterministic Helmholtz problem:
For $u-\uh,$ for any $\vh \in \Vhp$ we have $a(u-\uh,\vh) = 0$ and $\aep(u-\Ph u,\vh) = 0$ by Galerkin orthogonality for $u-\uh$ and $u-\Ph u$ respectively\footnote{For the arguments in this \namecref{sec:errorsplit}, we define the elliptic projection in the first argument, see \cref{eq:ellprojfirst} below.}. Therefore
\begin{align*}
  a(\Ph u - \uh,\vh) &= a(\Ph u - u,\vh) + a(u-\uh,\vh)\\
  &= a(\Ph u - u,\vh)\\
  &= \aep(\Ph u - u,\vh) - k^2\IPLtD{\Ph u - u}{\vh}\\
  &= - k^2\IPLtD{\Ph u - u}{\vh},
\end{align*}
that is, $\Ph u - \uh$ solves the \emph{discrete} Helmholtz problem
\beqs
a(\Ph u - \uh,\vh) = \IPLtD{\ftilde}{\vh} \tforall \vh \in \Vhp,
\eeqs
where $\ftilde = k^2\mleft(u-\Ph u\mright).$ One then uses the fact that $\Ph u - \uh$ satisfies a discrete Helmholtz problem to prove a bound on the difference directly.

In \cite{FeWu:09,FeWu:11,Wu:14} a discrete multiplier argument is used to prove a bound on $\Ph u - \uh$, reminiscent of the multiplier arguments used to prove a priori bounds on Helmholtz problems in \cref{sec:pdetheory}. In \cite{DuWu:15} an argument using higher-order (discrete) norms is used in an argument conceptually similar to the modified duality arguments above; this argument is the heart of the proof in \cref{sec:fem} below. In essence the argument in \cite{DuWu:15} reduces to showing the bounds
\beq\label{eq:duwu1}
\NLtD{\Ph u - \uh} \lesssim \mleft(h + \mleft(hk\mright)^p\mright)\NW{u-\Ph u} + \mleft(h^{p+1}k^2 + h^{2p}k^{p+2}\mright)\Nfn{p-1,h}{\Ph u - \uh}
\eeq
and
\beq\label{eq:duwu2}
\Nfn{p-1,h}{\Ph u - \uh} \lesssim h^{2-p} \NW{u- \Ph u} + k^{p-1} \NLtD{\Ph u - \uh},
\eeq
where $\Nfn{p-1,h}{\cdot}$ is a discrete norm analagous to the Sobolev norm of order $p-1$. \Cref{eq:duwu1,eq:duwu2} are then combined to show
\beqs
\NLtD{\Ph u - \uh} \lesssim \mleft(h + \mleft(hk\mright)^p\mright)\NW{u-\Ph u} + \mleft(\mleft(hk\mright)^{p+1} + h^{2p}k^{2p+1}\mright)\NLtD{\Ph u -\uh},
\eeqs
and the final term can be absorbed into the left-hand side if $h^{2p}k^{2p+1}$ is sufficiently small.

%\opntodo{For this section; for the first set you only need a bound on the elliptic projection error (with the right powers of $h$ and $k$). For the second one, We use quasi-optimality in $H^1$, the lower-order bounds in terms of the higher-order one, and a bound on the best approximation error. Maybe you could combine the first and third and get away with a bound on the elliptic projection error with the correct powers?}

Observe that when using the elliptic projection in such a splitting argument, there are two differences compared to the use of an elliptic projection in modified duality arguments:
\ben
\item One does not need the elliptic projection to be quasi-optimal \cref{eq:epho}, rather, one only need to bound the error $\NW{u-\Ph u}$, where $u$ solves a Helmholtz problem, in terms of $\NLtD{f}$.
\item The elliptic projection should be defined in the first argument, not the second, i.e.
  \beq\label{eq:ellprojfirst}
a(\Ph u,\vh) = a(u,\vh) \tforall \vh \in \Vhp.
  \eeq
\een


%% \subsubsection{Complete technical overview of the literature}\label{sec:litsum}
%% We now give a brief overview of the literature on rigorous quasi-optimality/error bounds for finite-element discretisations of the Helmholtz equation, where the proofs use one of the techniques outlined above. We record this information tersely, having given more insight into the various techniques above. In the columns marked `mesh condition' the quantities listed must be sufficiently small, e.g., if `mesh condition' reads `$hk^2$' then the condition is `$hk^2$ sufficiently small'. If a quantity depends on $p$, but `$p$ fixed' is not specified, then the result holds for varying $p,$ i.e., for $hp$-finite-element methods. IF $p$ does not appear, then the results are for first-order finite-element methods only.
%% Work
%% Mesh condition
%% Error bound
%% Notes
%\paragraph{Quasi-optimality}

%\opntodo{Would it be better to put bounds on $\eta$ here?}


%%5 Can put the below somewhere %%

%% To keep the finite-element error bounded when solving \eqref{eq:introdet}, one must over-refine the numerical grid. That is, rather than using a fixed number of points per wavelength, one must increase the number of points per wavelength as $k$ increases. To achieve bounded finite-element error, one must refine the finite-element mesh size $h$ like $k^{-3/2}$. Whilst this result has been known numerically for some time, it was proven for \eqref{eq:introdet} only with constant coefficients (on various domains and for various finite-element spaces) in \cite{IhBa:95a,Wu:14,DuWu:15,ChNi:18}, and the first proof (to our knowledge) for \eqref{eq:introdet} with heterogeneous coefficients  is contained in \cref{chap:background}. Choosing $h \sim k^{-3/2}$ means \eqref{eq:intromat} is a linear system of size $k^{3d/2}$, larger than if one merely wants the interpolation error to be bounded. Hence, requiring a bounded finite-element error gives rise to very large linear systems.

%% More briefly, if one wants the finite-element solution to be quasi-optimal (that is, up to a constant, the finite-element solution is the best approximation in the finite-element space), then one must over-refine even more, and take $h \sim k^{-2}$. This mesh condition will give rise to linear systems with $k^{2d}$ degres of freedom. See \cref{chap:background} for further details on the necessity of this mesh condition, and further discussion of all the mesh conditions discussed above.\opntodo{EDIT THE ABOVE TO REMOVE `OVER-REFINE'}


%% \subsection{New error bounds for the Helmholtz equation in heterogeneous media}\label{sec:heterr}
%% In this section, we prove that the finite-element approximation of the solution to the Helmholtz TEDP exists if $ h \lesssim k^{-3/2}.$ Moreover, we give an expression for the hidden constant that is completely explicit in $A$ and $n$, and we also prove a bound on the finite-element error, again completely explicit in $A$ and $n$. The argument in this section closely follows those in \cite{FeWu:11,ChNi:18} in its use of an elliptic projection argument to prove the required finite-element existence result and error bound. The paper \cite{FeWu:11} proved a similar result for the Helmholtz equation in homogeneous media, and \cite{ChNi:18} does so for the homogeneous Helmholtz equation with corner singularities.

%% Whilst we prove the results in this section for the TEDP, we expect that they can be extended to the Helmholtz Exterior Dirichlet Problem (EDP) where the infinite domain is truncated, and the Dirichlet-to-Neumann map is realised exactly on the truncated boundary. However, our proof below uses recently-proved bounds on the solution of a related problem to the TEDP from \cite{ChNiTo:18}; in order to extend our results to the EDP we would need analogues to the results in \cite{ChNiTo:18} for the EDP.

%% %% \paragraph{Problem Set-up} Let $\Dm$ be a bounded Lipschitz\ednote{We actually need this to be a $C^{k,\lambda}$ set, for $k+\lambda > 1.5,$ so that we can do the whole non-zero Dirichlet data thing. This is getting a bit complicated. I guess our options are (i) persevere, (ii) give up and just do the theory for zero Dirichlet data, or (iii) assume that we know $\ud,$ not just $\gD.$ Thoughts?} open set such that the open complement $\Dp\de \RRd\setminus \Dmclos$ is connected. Let $\Dtilde$ be a bounded connected Lipschitz open set such that $\Dmclos \subset\subset\Dtilde$. 
%% %% Let $D\de\Dtilde\setminus\Dm$, $\GD\de \partial \Dm$, and $\GI \de\partial \Dtilde$, so that $\partial D= \GD \cup \GI$ and $\GD\cap \GI = \emptyset$. Throughout $\tr$ will denote the trace onto the whole boundary $\dD,$ whereas $\trGI$ and $\trGD$ will denote the traces on $\GI$ and $\GD$ respectively. Throughout we assume there exists some $\kz > 0$ such that $k \geq \kz$. Let $\NW{v}$ denote the weighted $H^1$ norm on $\HoD$:
%% %% \beqs
%% %% \NW{v}^2 \de \NLtD{\grad v}^2 + k^2 \NLtD{v}^2.
%% %% \eeqs


%% %% Let
%% %% \bit
%% %% \item $f\in \LtD$ 
%% %% \item $\gD\in \HthtGD$,
%% %% \item $\gI\in \LtGI$
%% %% \item $n\in \LiDRR$ such that $\dist\mleft(\supp\mleft(1-n\mright),\GI\mright)>0$, satisfying
%% %% \beq
%% %% 0<\nmin \leq n\mleft(\bx\mright)\leq\nmax<\infty\,\, \text{ for almost every } \bx \in D,
%% %% \eeq
%% %% \item $A \in \WoiDRRdtd$ such that $\dist\mleft(\supp\mleft(I -A\mright),\GI\mright)>0$, $A$ is symmetric, and there exist $0<\Amin\leq \Amax<\infty$ such that
%% %% \beq\label{eq:AellEDP}
%% %%  \Amin |\bxi|^2\leq\mleft(A\mleft(\bx\mright) \bxi\mright) \cdot \overline{ \bxi}  \leq \Amax|\bxi|^2 \quad\text{ for almost every }\bx \in D \text{ and for all } \bxi\in \CCd.
%% %% \eeq
%% %% \eit
%% %we say $u\in \HoD$ satisfies the Helmholtz Truncated Exterior Dirichlet Problem (TEDP) if 
%% %\beqs
%% %\grad\cdot\mleft(A \grad u \mright) + k^2 n u = -f \quad \tin D,
%% %\eeqs
%% %\beqs
%% %\trGD u =\gD \quad\ton \GD,
%% %\eeqs
%% %and 
%% %\beq\label{eq:TEDP3}
%% %\dn u - \ii k  \trGI u = \gI \ton \GI.
%% %\eeq
%% In order to study the TEDP with $\gD\neq0,$ we must, in essence reformulate to the TEDP with $\gD=0$ but a different right-hand side for the domain term.

%% %% Define the space
%% %% \beqs
%% %% \HozDD \de \set{v \in \HoD \st \trGD u = 0}.
%% %% \eeqs
%% %and the sesquilinear form and antilinear functional
%% %The variational formulation of the TEDP with $\gD = 0$ is%\opntodo{Check exactly what's needed in hetero}
%% %
%% %\beq\label{eq:tedpz}
%% %\text{Find } u \in \HozDD\quad \tst\quad a(u,v) = F(v)\quad \tfa v \in \HozDD,
%% %\eeq
%% %
%% %where
%% %
%% %\beqs
%% %a(u,v) \de \int_D \mleft(A \grad u\mright)\cdot \grad \vb - k^2 n u\vb - ik \int_{\GI} \trGI u \trGI \vb\quad \tand\quad F(v) \de \int_D f\vb + \int_{\GI} \gI\trGI \vb.
%% %\eeqs
%% %
%% %% In order to deal with non-zero Dirichlet data $\gD,$ we let  $\ud \in \HtD$ be such that $\trGD \ud = \gD$, and $\esssup \ud \compcont D$. The proof that such a $\ud$ exists is in \cref{lem:ud}.
%% %% The variational formulation of the TEDP is then
%% %% \beq\label{eq:tedp}
%% %% \text{Find } u \in \HozDD\quad \tst\quad a(u,v) = F(v)\quad \tfa v \in \HozDD,
%% %% \eeq
%% %% where
%% %% \beqs
%% %% a(u,v) \de \int_D \mleft(A \grad u\mright)\cdot \grad \vb - k^2 n u\vb - ik \int_{\GI} \trGI u \trGI \vb
%% %% \eeqs
%% %% and
%% %% \beqs
%% %% F(v) \de  \int_D \mleft(f - \grad \cdot \mleft(A\grad \ud\mright) - k^2 n\ud\mright)\vb + \int_{\GI} \mleft(\gI-\dn\ud\mright)\trGI \vb.
%% %% \eeqs
%% %% The function $\us = u+ \ud$ is then the solution of the Helmholtz equation
%% %% \beqs
%% %% \grad\cdot\mleft(A \grad \us \mright) + k^2 n \us = -f \quad \tin D,
%% %% \eeqs
%% %% \beqs
%% %% \trGD \us =\gD \quad\ton \GD,
%% %% \eeqs
%% %% and 
%% %% \beq\label{eq:TEDP3}
%% %% \dn \us - \ii k  \trGI \us = \gI \ton \GI.
%% %% \eeq
%% %% \bre[Reducing the smoothness of $\gD$]
%% %% The assumption that $\gD \in \HthtGD$ is made so that the lifting $\ud$ of $\gD$ is in $\HtD$ (see \cref{app:ud}). As $\ud \in \HtD,$ the antilinear functional $F$ defined above is well-defined. We could reduce the smoothness of $\gD$ to $\HoGD$ (meaning $\ud \in \HthtD$) but this reduction in smoothness would then require us to reformulate the functional $F$ as
%% %% \beqs
%% %% F(v) = \int_D \mleft(A \grad \ud\mright)\cdot \grad \vb - k^2 n \ud \vb + f \vb + \int_{\GI} \mleft(\gI - \dn \ud\mright)\vb.
%% %% \eeqs\opntodo{Put a proof of this somewhere, in 28/2/19 notes}
%% %% With this reformulation, $F \in \HozDDprime,$ but does not have a representative function in $\LtD$. Our proofs below will use results from \cite{ChNiTo:18}, which are stated for the TEDP with zero Dirichlet boundary condition and $L^2$ right-hand side. To avoid the complications stated above, and to allow us to use the results in \cite{ChNiTo:18}, we therefore impose the additional smoothness on $\gD.$ Also, in the case with $F$ only in $\HozDDprime$, proving a priori bounds on the solution of the TEDP is more complicated (c.f., e.g., \cite[Theorem 2.5]{GrPeSp:19} and \cite[Corollary 2.16]{GrPeSp:19} which consider the analogous EDP). For the same reason, we assume $A \in \WoiDRRdtd;$ if we only had $A \in \LiDRRdtd,$ we could reformulate $F$ as outlined above, but we would have the same complications as just described.

%% We  restrict our meshes to the following class:
%% \bde[Shape-regular]
%% A family $(\Th)_{h \in (0,1)}$ of meshes of $\DR$ is said to be \defn{shape-regular} if there exists $\rho > 0$ such that for all $T \in \Th$ and for all $h \in (0,1]$
%%   \beqs
%% \diam B(T) \geq \rho \diam T,
%% \eeqs
%% where $B(T)$ is the largest ball contained in $T$ such that $T$ is star-shaped with respect to $B(T)$.
%% \ede

%% The fact that we cannot reduce the smoothness of $\gD$ further to $\HhGD$ is due to the Morawetz multiplier techniques used to obtain the a priori bounds in \cite{GrPeSp:19}, see \cite[(iii), p. 2874]{GrPeSp:19}.
%% %% \ere
%% Also, for later use we state the \defn{adjoint} problem.
%% \beq\label{eq:tedpadj}
%% \text{Find } u \in \HozDDR\quad \tst\quad \aadj(u,v) = F(v)\quad \tfa v \in \HozDDR,
%% \eeq
%% where
%% \beqs
%% \aadj(u,v) \de \int_D \mleft(A \grad u\mright)\cdot \grad \vb - k^2 n u\vb + ik \int_{\GI} \trGI u \trGI \vb.
%% \eeqs
%% %and
%% %\beqs
%% %\Fadj(v) \de \aadj(\uz,v) + \int_D f\vb + \int_{\GI} \gI\trGI \vb.
%% %\eeqs

%% The statement of the main result requires the following related sesquilinear form and \lcnamecref{lem:relatedwp}.

%% \bde[Related sesquilinear form]
%% For $\vo, \vt \in \HozDDR$ we define
%% \beqs
%% \api(\vo,\vt) = \int_D \IP{A \grad \vo}{\vt} - ik\int_{\GI} \vo\vtbar.
%% \eeqs
%% \ede

%% \ble[Related PDE is well-posed and solution is in $H^2$]\label{lem:relatedwp}
%% If $A \in \CzoDRRRdtd,$ then the solution $\psi \in \HozDDR$ of the related PDE
%% \beq\label{eq:relpde}
%% \api(u,v) = \IPLtDR{f}{v}\quad \tfa\quad v \in \HozDDR
%% \eeq
%% exists, is unique, is in $\HtDR,$ and satisfies the a priori bound
%% % \beqs\label{eq:relpdehobound}
%% % \NW{\psi} \lesssim \frac{\max\set{\Amin^{-1},1}}k,
%% % \eeqs
%% % and
%% % \beqs
%% % \NHoD{\psi} \lesssim \CHoell \NLtD{f}
%% % \eeqs
%% % and
%% \beqs
%% \NHtDR{\psi} \lesssim \CHtell \NLtDR{f}.
%% \eeqs
%% for some constant $\CHtell > 0$ depending on $A,$ but independent of $k.$
%% \ele

%% \bre[Proof of \cref{lem:relatedwp}]
%% \Cref{lem:relatedwp} is proved in \cite{ChNiTo:18}, although the dependence on $A$ is not made explicit.
%% \ere

%% %\paragraph{Finite-Element Set-up} Let $\Vh$ be the first-order linear finite-element space on some mesh on $D$ with mesh size $h.$

%% \bas[Existence, uniqueness, and an a priori bound]\label{ass:bound}
%% We assume that the coefficients $A$ and $n$ are such that for all $k \geq \kz$ the solutions of the \cref{prob:vtedp} and its adjoint \eqref{eq:tedpadj} exist, are unique, are in $\HtDR,$ and satisfy the bound
%% \beq\label{eq:hhbound}
%% \NHtDR{u} \lesssim \CHthh \,k \mleft(\NLtDR{f} + \Nunsure{g} + \NLtGD{\gradGD \gD} + k \NLtGD{\gD}\mright),
%% \eeq
%% where $\gradGD$ is the surface gradient on $\GD,$ $u$ is the solution of the TEDP or its adjoint, and $\CHthh >0$ is a constant dependent on $A$, $n,$ and possibly $k.$
%% % \footnote{Determining the dependence of $\CHthh$ on $A$ and $n$ could be tricky. It was done in \cite{ChScTe:13} for a $C^2$ domain with scalar $A$ and homogeneous Dirichlet boundary conditions.}
%%  \eas

%% %% \bde[Finite-element approximation] 
%% %%  The finite-element approximation to \eqref{eq:tedp} is the following:
%% %% \beq\label{eq:tedpfe}
%% %% \text{Find } \uh \in \Vh\quad \tst\quad a(\uh,\vh) = F(\vh)\quad \tfa \vh \in \Vh,
%% %% \eeq
%% %% \ede

%% The main theorem we prove is the following:

%% \bth[Finite-element-error bound]\label{thm:febound}
%% If $A \in \CzoDRRRdtd,$ $h \lesssim 1/k,$ \cref{ass:bound} holds, and
%% \beq\label{eq:hcond}
%% h \lesssim \mleft(\NLiDRRR{n} \mleft(\Amax + \half\mright)\CHtell \CHthh\mright)^{-1/2}k^{-3/2}, % There should be a factor of a half in front of the right-hand side of this, as it makes things clearer what's going on in the proof. However, since we're doing everything with \lesssim, a factor of a half doesn't matter. We could replace the half with any \eps in (0,1), but then the constant hidden in the \lesssim in \eqref{eq:hherrltbound} has a factor 1/\eps.
%% \eeq
%% the finite-element solution $\uh$ to the \cref{prob:fevtedp} exists, is unique, and satisfies the bounds
%% \beq\label{eq:hherrltbound}
%% \NLtD{u-\uh} \lesssim \Cfemo \mleft(hk\mright)^2 \mleft(\NLtD{f} + \Nunsure{\gI} + \NLtGD{\gradGD \gD} + k \NLtGD{\gD}\mright)
%% \eeq
%% and
%% \beq\label{eq:hherrwbound}
%% \NW{u-\uh} \lesssim \mleft(\Cfemt hk +  \Cfemth h^2k^3\mright)\mleft(\NLtD{f} + \Nunsure{\gI} + \NLtGD{\gradGD \gD} + k \NLtGD{\gD}\mright),
%% \eeq
%% where
%% \beqs
%% \Cfemo \de \mleft(\Amax + \half\mright)\CHthh^2,
%% \eeqs
%% \beqs
%% \Cfemt \de \frac{\Amax+\half}{\Amin} \CHthh,
%% \eeqs
%% \beqs
%% \Cfemth \de \frac{\mleft(\Amin+ \NLiDRRR{n}\mright)^{1/2}}{\Amin^{1/2}}\Cfemo,
%% \eeqs
%% and $u$ is the solution of \cref{prob:vtedp}.
%% \enth

%% \subsubsection{Properties of the Elliptic Projection, and a related PDE}

%% The proof technique we use below (adapted from \cite{FeWu:11,ChNi:18}) uses an `elliptic projection' of the solution of the TEDP using the related sesquilinear form $\api.$ We define the energy norm induced by the sesquilinear form $\api$:
%% \beqs
%% \Npi{\vo} = \sqrt{\abs{\api(\vo,\vo)}}.
%% \eeqs

%% \ble[Energy Norm is a norm]\label{lem:inducednorm}
%% The induced norm $\Npi{\cdot}$ is a norm on $\HoD.$
%% \ele

%% \bpf[Proof of \cref{lem:inducednorm}]
%% The main thing to check is that, for $v \in \HoD,$ $\Npi{v}=0 \implies v=0.$ By construction, if $\Npi{v}=0,$ then $\int_{D} \IP{A \grad v}{\grad v} =0$ and $\NLtGI{v}^2 = 0,$ as these are the real and imaginary parts of $\api(v,v).$ By \eqref{eq:AellEDP}, it follows that $\Amin \abs{\grad v}^2 \leq 0,$ and thus $v$ is constant. As $\NLtGI{v} = 0,$ it follows that $\trGI v =0,$ and hence by the trace theorem, as $v$ is constant, it follows that $v=0.$

%% Other properties of norms follow analagously as with any definition of an energy norm.
%% \epf
%% \ble[Energy norm is equivalent to weighted norm]\label{lem:normbound}
%% If $v \in \HoD,$ then
%% \beq\label{eq:boundew}
%% \Npi{v} \lesssim \sqrt{\Amax+\half}\NW{v}
%% \eeq
%% and
%% \beq\label{eq:boundwe}
%% \NW{v} \lesssim \max\set{\Amin^{-\half},1} \Npi{v}
%% \eeq
%% \ele

%% \bpf[Proof of Lemma \ref{lem:normbound}]
%% To show \eqref{eq:boundew}, for $ v \in \HoD$ we have
%% \begin{align*}
%%   \Npi{v}^2 &= \abs{\api(v,v)}\\
%%             &\lesssim \abs{\int_{D} \IP{A \grad v}{\grad v}} + k\NLtGI{v}^2 \\
%%             &\lesssim \abs{\int_{D} \IP{A \grad v}{\grad v}} + k\NLtD{v}\NHoD{v}, \text{ by the multiplicative trace inequality}\\
%%             &\lesssim \Amax \NLtD{\grad v}^2 + \half k^2 \NLtD{v}^2 + \half \NHoD{v}^2\\
%%   &\lesssim \mleft(\Amax+\half\mright)\NW{v}^2
%% \end{align*}
%% as required.

%% To show \eqref{eq:boundwe} we first show that, for $v \in \HoD,$ $\Npi{v} \gtrsim \min\set{\Amin^{\half},1} \mleft(\NLtD{\grad v} + k^{\half} \NLtGI{\trGI v}\mright)$:
%% \begin{align}
%%   \Npi{v} &= \mleft(\abs{\api(v,v)}\mright)^{\half}\nonumber\\
%%           &= \mleft(\mleft(\int_D \IP{A \grad v}{\grad v}\mright)^2 + k^2 \mleft(\int_{\GI}\abs{\trGI v}^2\mright)^2\mright)^{\quarter}\nonumber\\
%%   &\geq \mleft(\mleft(\int_D \Amin \abs{\grad v}\mright)^2 + k^2 \NLtGI{\trGI v}^4\mright)^{\quarter}\nonumber\\
%%           &= \mleft(\Amin^2 \NLtD{\grad v}^4 + k^2 \NLtGI{\trGI v}^4\mright)^{\quarter}\nonumber\\
%%           &\geq \min\set{\Amin^{\half},1}\mleft(\NLtD{\grad v}^4 + k^2 \NLtGI{\trGI v}^4\mright)^{\quarter}\nonumber\\
%%   &\gtrsim \min\set{\Amin^{\half},1} \mleft(\NLtD{\grad v} + k^{\half} \NLtGI{\trGI v}\mright), \text{ as } \mleft(x+y\mright)^4 \lesssim x^4 + y^4.\label{eq:Npifour}
%% \end{align}

%% We recall the fact that for $v \in \HoD,$
%% \beq\label{eq:poincarelike}
%% \NLtD{v} \lesssim \NLtD{\grad v} + \NLtGI{\trGI v},
%% \eeq
%% see, e.g., \cite[Equation (6.16)]{Sp:15}. We can then prove \eqref{eq:boundwe}:
%% \begin{align*}
%%    \NW{v} &\lesssim \NLtD{v}+ \NLtD{\grad v}\\
%%           &\lesssim \NLtGI{\trGI v}+ \NLtD{\grad v} + \NLtD{\grad v}\text{ by \eqref{eq:poincarelike}}\\
%%           &\lesssim k^{\half}\NLtGI{v} + \NLtD{\grad v}\\
%%   &\lesssim \max\set{\Amin^{-\half},1}\Npi{v}, \text{ by \eqref{eq:Npifour}.}
%% \end{align*}
%% \epf

%% % \ble[Bound on $L^2$ norm]\label{lem:ltbound}
%% % If $v \in \HoD$ then the bound
%% % \beqs
%% % \NLtD{v} \lesssim  \NLtGI{\trGI v} + \NLtD{\grad v}
%% % \eeqs
%% % holds.
%% % \ele

%% % \bpf[Proof of \cref{lem:ltbound}]
%% % \opntodo{Look at proof in IbyPs article}
%% % If $v \in \HozDD,$ then by the Poincar\'e inequality, we have that $\NLtD{v} \lesssim \NLtD{\grad v}.$ Alternatively, if $\GD = \emptyset$ and $\trGI v$ is constant, then $v - \trGI v \in \HozDD,$ and thus (abusing notation, and letting $\trGI v$ denote the value of the constant, and also a constant function defined on $D$ taking that value everywhere)
%% % \begin{align*}
%% %   \NLtD{v} &\leq \NLtD{\trGI v} + \NLtD{v-\trGI v}\\
%% %   &= \NLtGI{\trGI v} + \NLtD{v-\trGI v}\\
%% %            &\lesssim \NLtGI{\trGI v} +  \NLtD{\grad \mleft(v-\trGI v\mright)}\\
%% %              &= \NLtGI{\trGI v} + \NLtD{\grad v},
%% % \end{align*}
%% % as required.
%% % \epf

%% % \ble[Bound on weighted norm by energy norm]\label{lem:othernormbound}
%% % If $v \in \HozDD,$ or if $\GD = \emptyset$ and $\trGI v$ is constant, the bound
%% % \beq\label{eq:boundwe}
%% % \NW{v} \lesssim \max\set{\Amin^{-\half},1} \Npi{v}
%% % \eeq
%% % holds.
%% % \ele

%% % \bpf[Proof of \cref{lem:othernormbound}]
%% % \epf

%% We now define the elliptic projection of a function in $\HoD.$% and also define a related PDE that will be used in proving the approximation properties of the elliptic projection.

%% \bde[Elliptic Projection]
%% For $w \in \HoD$ we define the \defn{elliptic projection} $\Ph w \in \Vhp$ of $w$ by
%% \beq\label{eq:ellproj}
%% \api(\vh,\Ph w) = \api(\vh,w) \tfa \vh \in \Vhp.
%% \eeq
%% \ede

%% % \bde[Related PDE]\label{lem:relpde}
%% % Given $f \in \LtD$ we define the related (adjoint) PDE; find $\psi \in \HoD$ such that for all $v \in \HoD$
%% % \beq\label{eq:relpde}
%% % \api(\psi,v) = \IPLtD{f}{v}.
%% % \eeq\ede

%% % \bpf[Proof of \cref{lem:relatedwp}]
%% % By \eqref{eq:boundwe} we have, for $v \in \HozDD$
%% % \beqs
%% % \min\set{\Amin,1}\NW{v}^2 \lesssim \abs{\api(v,v)},
%% % \eeqs
%% % and we also have that
%% % \beqs
%% % \NWs{\IP{f}{\cdot}} \leq \frac1k \NLtD{f},
%% % \eeqs
%% % where $\NWs{\cdot}$ denotes the norm on $\HozDDs$ induced by $\NW{\cdot}.$ By the Lax--Milgram Theorem, we can therefore conclude that $\psi$ exists, is unique, and satisfies the bound
%% % \beqs
%% % \NW{\psi} \lesssim \frac{\max\set{\Amin^{-1},1}}{k}\NLtD{f}.
%% % \eeqs
%% % Use Grisvard Magic to get $H^2.$\opntodo{this}
%% % \epf

%% \ble[Properties of elliptic projection]\label{lem:ellprojbounds}
%% Let $A \in \CzoDRRRdtd.$ If $w \in \HtDR,$ then the elliptic projection $\Ph w$ exists, is unique, and the error satisfies the bounds
%% \beq\label{eq:ellprojenbound}
%% \Npi{w-\Ph w} \lesssim \sqrt{\Amax+\half}\,h\NHtDR{w},
%% \eeq
%% and
%% \beq\label{eq:ellprojltbound}
%% \NLtDR{w-\Ph w} \lesssim  \mleft(\Amax+\half\mright)\CHtell\,h^2\NHtDR{w}.
%% \eeq
%% \ele

%% \bpf[Proof of \cref{lem:ellprojbounds}]
%% We first assume $\Ph w$ exists. To show \eqref{eq:ellprojenbound} we apply C\'{e}a's Lemma in $\Vhp$ using the energy norm $\Npi{\cdot}$ to conclude
%% \beqs
%% \Npi{w-\Ph w} \leq \Npi{w-\Ih w}.
%% \eeqs
%% We then apply \cref{lem:normbound,lem:scottzhangbound} to conclude \eqref{eq:ellprojenbound}.

%% To prove \eqref{eq:ellprojltbound} we let $\psi$ solve the related PDE \eqref{eq:relpde} with $f = w-\Ph w.$ By \cref{lem:relatedwp} $\psi \in \HtDR$ and thus by  \cref{lem:normbound} and \cref{lem:scottzhangbound}
%% \beqs
%% \Npi{\psi - \Ih \psi} \lesssim \sqrt{\Amax + \half}\CHtell \,h\NLtDR{w-\Ph w}.
%% \eeqs

%% If we now set $v = w-\Ph w$ in \eqref{eq:relpde}, then we obtain
%% \begin{align}
%%   \NLtDR{w - \Ph w}^2 &= \api\mleft(\psi,w-\Ph w\mright)\nonumber\\
%%                      &= \api\mleft(\psi-\Ih \psi,w-\Ph w\mright) \text{ by Galerkin orthogonality for } w-\Ph w\nonumber\\
%%                      &\leq \Npi{\psi-\Ih \psi}\Npi{w-\Ph w}\nonumber\\
%%                        &\lesssim \sqrt{\Amax + \half}\CHtell \,h\NLtDR{w-\Ph w}\Npi{w-\Ph w}\label{eq:epltfinal}.
%% \end{align}
%% By cancelling $\NLtDR{w- \Ph w}$ from both sides of \eqref{eq:epltfinal} and using \eqref{eq:ellprojenbound} we obtain \eqref{eq:ellprojltbound}.

%% We have proved the bounds \eqref{eq:ellprojenbound} and \eqref{eq:ellprojltbound} under the assumption of existence. To show uniqueness, suppose $\wh, \whtilde$ both satisfy \eqref{eq:ellproj} (with $\Ph = \wh$ or $\whtilde$ respectively). Then by linearity, for all $\vh \in \Vhp,$
%% \beqs
%% \api\mleft(\vh,\Ph\mleft(\wh-\whtilde\mright)\mright) = \IP{\vh}{w-w} = 0.
%% \eeqs
%% That is, the function $\wh - \whtilde$ is an elliptic projection of the zero function.

%% Therefore, by \eqref{eq:ellprojltbound} $\NLtDR{0 - \mleft(\wh - \whtilde\mright)} \lesssim 0,$ i.e., $\wh = \whtilde.$ Therefore, if the elliptic projection $\Ph w$ exists, it is unique. As the space $\Vhp$ is finite-dimensional, by the Rank--Nullity Theorem, the uniqueness of $\Ph w$ implies its existence; hence $\Ph w$ exists, and is unique, as required.
%% \epf


%% \subsubsection{Proof of Main Result}

%% We let $\Ih$ denote the Scott--Zhang quasi-interpolant in $\Vhp$ (see \cite{ScZh:90}), and  use its following property.
%% \ble[Properties of Scott-Zhang interpolant]\label{lem:scottzhangbound}
%% let $h \lesssim 1/k.$ If $w \in \HtDR,$ then
%% \beq\label{eq:scottzhangbound}
%% \NW{w - \Ih w} \lesssim h \NHtDR{w}.
%% \eeq
%% \ele

%% \bpf[Proof of \cref{lem:scottzhangbound}]
%% The Scott-Zhang interpolant $\Ih w$ satisfies
%% \beq\label{eq:szlt}
%% \NLtDR{w-\Ih w} \lesssim h^2 \NHtDR{w}
%% \eeq
%% \and
%% \beq\label{eq:szho}
%% \NHoDR{w-\Ih w} \lesssim h \NHtDR{w}.
%% \eeq
%% Hence by the definition of $\NW{\cdot},$ by combining \eqref{eq:szlt} and \eqref{eq:szho} we have
%% \beqs
%% \NW{w-\Ih w} \lesssim h\mleft(1+hk\mright)\NHtDR{w}.
%% \eeqs
%% As $h\lesssim 1/k,$ \eqref{eq:scottzhangbound} follows.
%% \epf

%% The following \lcnamecref{cor:hhszbound} follows from \cref{ass:bound}, and is used to prove \cref{thm:febound}.

%% \bco\label{cor:hhszbound}
%% If $u$ is the solution of the Helmholtz Interior Impedance Problem (or its adjoint) then the error in the Scott--Zhang quasi-interpolant satisfies
%% \beq\label{eq:hhszbound}
%% \NW{u-\Ih u} \lesssim \CHthh hk \mleft(\NLtDR{f} + \Nunsure{g} + \NLtGD{\gradGD \gD} + k \NLtGD{\gD}\mright).
%% \eeq
%% \eco

%% The proof of the main theorem (\cref{thm:febound} below) also uses the fact that $a$ satisfies a G\r{a}rding inequality.
%% \ble[G\r{a}rding inequality]
%% If $v \in \HozDDR,$ then
%% \beq\label{eq:garding}
%% \Re\mleft(a(v,v)\mright) \geq \Amin \NW{v}^2 - k^2\mleft(\Amin + \NLiDRRR{n}\mright)\NLtDR{v}^2,
%% \eeq
%% where $\Re$ denotes the real part.
%% \ele

%% Finally, we recall \defn{Cauchy's inquality}: For all $a,b \in \RR$, and for all $\eps > 0,$
%% \beq\label{eq:cauchy}
%% ab \leq \frac{a^2}{2\eps} + \frac{\eps b^2}{2}.
%% \eeq

%% \bpf[Proof of \cref{cor:hhszbound}]
%% The proof follows from \cref{lem:scottzhangbound,ass:bound}.
%% \epf

%% We are now in a position to prove our main theorem.




%% \bpf[Proof of \cref{thm:febound}]
%% In this proof, for brevity we let
%% \beqs
%% \Mfg = \NLtDR{f} + \Nunsure{\gI} + \NLtGD{\gradGD \gD} + k \NLtGD{\gD}.
%% \eeqs
%% By \cref{ass:bound} the solution $u$ of the TEDP exists and is unique. Assume the finite-element solution $\uh$ exists. Let $\xi \in \HoD$ satisfy the adjoint TEDP \eqref{eq:tedpadj} with $f=u-\uh,$ $\gD=0,$ and $\gI=0.$ Taking complex conjugates, it follows that
%% \beq\label{eq:errordual}
%% a(v,\xi) = \IPLtDR{v}{u-\uh} \tfa v \in \HoDR.
%% \eeq
%% By \cref{ass:bound} $\xi$ exists, is unique and is in $\HtDR.$. Setting $v = u-\uh$ in \eqref{eq:errordual} we obtain
%% \begin{align*}
%%   \NLtDR{u-\uh}^2 &= a\mleft(u-\uh,\xi\mright)\\
%%                  &= a\mleft(u-\uh,\xi-\Ph \xi\mright) \quad\text{by Galerkin orthogonality for } u-\uh\\
%%                  &= \api\mleft(u-\uh,\xi-\Ph \xi\mright) - k^2 \IPLtDR{n\mleft(u-\uh\mright)}{\xi-\Ph \xi}\\
%%                  &= \api\mleft(u-\Ih u,\xi-\Ph \xi\mright) - k^2 \IPLtDR{n\mleft(u-\uh\mright)}{\xi-\Ph \xi}\\
%%   &\quad\quad\quad\text{by Galerkin orthogonality for }\xi  - \Ph \xi\\
%%                  &\leq \Npi{u-\Ih u}\Npi{\xi - \Ph \xi} + \NLiDRRR{n} k^2 \NLtDR{u-\uh}\NLtDR{\xi-\Ph \xi}\\
%%                  &\lesssim \sqrt{\Amax + \half}\, \CHthh\, hk \Mfg\Npi{\xi-\Ph \xi}\\
%%   &\quad\quad+  \NLiDRRR{n} k^2 \NLtDR{u-\uh}\NLtDR{\xi-\Ph \xi}\quad \text{by \eqref{eq:boundew} and \eqref{eq:hhszbound}}\\
%%                  &\lesssim \sqrt{\Amax + \half}\, \CHthh\, hk\Mfg\,\sqrt{\Amax + \half}\,h\NHtDR{\xi}\\
%%                  &\quad\quad  + \NLiDRRR{n} k^2 \NLtDR{u-\uh}\NLtDR{\xi-\Ph \xi}\quad \text{by \eqref{eq:ellprojenbound}}\\
%%                  &\lesssim \mleft(\Amax + \half\mright)\CHthh\, hk\Mfg \, \CHthh \,hk \NLtDR{u-\uh}\\
%% &\quad\quad  + \NLiDRRR{n} k^2 \NLtDR{u-\uh}\NLtDR{\xi-\Ph \xi}\quad \text{by \eqref{eq:hhbound}}\\
%% \end{align*}
%% Cancelling a factor of $\NLtDR{u-\uh}$ and rearranging terms we obtain
%% \beqs
%% \NLtDR{u-\uh} \lesssim \mleft(\Amax + \half\mright)\CHthh^2 \mleft(hk\mright)^2\Mfg + k^2 \NLiDRRR{n} \NLtDR{\xi - \Ph \xi}
%% \eeqs
%% and therefore
%% \begin{align*}
%% &  \NLtDR{u-\uh} \lesssim \mleft(\Amax + \half\mright)\CHthh^2 \mleft(hk\mright)^2 \Mfg\\
%% &\quad\quad  + h^2k^3 \NLiDRRR{n} \mleft(\Amax + \half\mright) \CHtell \CHthh \NLtDR{u-\uh}
%% \end{align*}
%%   using the definition of $\xi$, \eqref{eq:ellprojltbound}, and \eqref{eq:hhbound}Therefore if $h$ satisfies \eqref{eq:hcond} we obtain \eqref{eq:hherrltbound}.
%% % \beqs
%% % \half \NLtD{u-\uh} \lesssim \mleft(\Amax + \half\mright)\CHthh \mleft(hk\mright)^2 \mleft(\NLtD{f} + \Nunsure{g}\mright).
%% % \eeqs
%% % that is, if $\Cmess \de (\mleft(\NLiDRR{n} \mleft(\Amax + \half\mright) \CHthh \mright)^{-1/2},$ then
%% % \beqs
%% % \NLtD{u-\uh} \lesssim \mleft(\Amax + \half\mright)\CHthh \Cmess^2k^{-1} \mleft(\NLtD{f} + \Nunsure{g}\mright),
%% % \eeqs
%% To obtain the bound \eqref{eq:hherrwbound}, we use the G\r{a}rding inequality \eqref{eq:garding}:
%% \begin{align*}
%%   \Amin \NW{u-\uh}^2 &\leq \Re\mleft(a\mleft(u-\uh,u-\uh\mright)\mright) + k^2\mleft(\Amin+ \NLiDRRR{n}\mright) \NLtDR{u-\uh}^2\\
%%                      &= \Re\mleft(a\mleft(u-\uh,u-\Ih u\mright)\mright) + k^2\mleft(\Amin+ \NLiDRRR{n}\mright) \NLtDR{u-\uh}^2\\
%%   &\quad\quad\quad\text{by Galerkin orthogonality}\\
%%                      &\leq \mleft(\Amax+\half\mright) \NW{u-\uh}\NW{u-\Ih u} + k^2\mleft(\Amin+ \NLiDRRR{n}\mright) \NLtDR{u-\uh}^2\\
%%   &\quad\quad\quad\text{by \cref{lem:normbound}}\\
%%   &\leq \frac{\mleft(\Amax+\half\mright)^2}{2\Amin} \NW{u-\Ih u}^2 + \frac{\Amin}2 \NW{u-\uh}^2 + k^2\mleft(\Amin+ \NLiDRRR{n}\mright) \NLtDR{u-\uh}^2,\\
%% \end{align*}
%% by Cauchy's inequality \eqref{eq:cauchy} with $\eps = \Amin.$ Therefore,
%% \beqs
%% \NW{u-\uh}^2 \leq \frac2{\Amin} \mleft(\frac{\mleft(\Amax+\half\mright)^2}{2\Amin} \NW{u-\Ih u}^2+ k^2\mleft(\Amin+ \NLiDRRR{n}\mright) \NLtDR{u-\uh}^2\mright)
%% \eeqs
%% and hence
%% \beq\label{eq:hherrboundnearly}
%% \NW{u-\uh} \lesssim \frac1{\Amin^{1/2}} \mleft(\frac{\Amax+\half}{\Amin^{1/2}} \NW{u-\Ih u}+ k\mleft(\Amin+ \NLiDRRR{n}\mright)^{1/2} \NLtDR{u-\uh}\mright).
%% \eeq
%% By substituting \eqref{eq:hhszbound} and \eqref{eq:hherrltbound} into \eqref{eq:hherrboundnearly} we obtain \eqref{eq:hherrwbound}.

%% To show that $\uh$ exists, as in the proof of \cref{lem:ellprojbounds} we can use the error bound \eqref{eq:hherrltbound} to show that $\uh$ is unique, and we can then use the fact that $\Vhp$ is finite-dimensional to show that $\uh$ exists.
%% \epf

\section{New Finite-Element-Error bounds for the Heterogeneous Helmholtz Equation}\label{sec:fem}

We now prove our new error bounds for higher-order finite-element approximations of the solution of the Helmholtz equation in heterogeneous media. Our results are a generalisation of results proved by Du and Wu \cite{DuWu:15} for higher-order finite-element approximations of the Helmholtz equation in homogeneous media; our results and proofs broadly follow those in \cite{DuWu:15}, with the main differences that:
\ben
\item We modify the proofs to cater for the heterogeneity of the coefficients, and
\item The dependence of our results on all of the constants involved is explicit.
\een
  In particular our results are explicit in $n$ and $k$ and are (in principle) explicit in $A$.

The proofs of our results have many parts, and appear technical, largely due to the burden of explicitly keeping track of all of the constants involved. However, in essence, the proof consists of two ideas:
\ben
\item Decompose the error $u-\uh = (u - \Ph u) + (\Ph u - \uh)$, where $\Ph u$ is an elliptic projection of $u$, and
\item Bound the error $u - \Ph u$ using the fact that $\Ph u$ is a Galerkin projection.
\item Bound the error $\Ph u - \uh$ in higher-order `discrete' norms, using the fact that $\Ph u - \uh$ solves a discrete Helmholtz problem.
\een

The structure of the remainder of this \lcnamecref{sec:fem} is as follows. In \cref{sec:errbounds} we state our new finite-element-error bounds. In \cref{sec:decomp} we prove a decomposition of the solution that allows us to prove a higher-order best approximation bound. In \cref{sec:anbackground} we collect together some routine analysis results needed for our proofs. In \cref{sec:errgalerkin} we prove error bounds for a number of different Galerkin projections (including the elliptic projection) that we use in subsequent proofs. In \cref{sec:discsob} we develop a notion of discrete derivatives and discrete norms, and prove properties of these norms. These norms will allow us to define higher-order discrete norms of functions in our finite-element space. Finally in \cref{sec:fembound} we prove our main finite-element error bounds.



\subsection{Main result: new finite-element-error bounds}\label{sec:errbounds}
As we are using higher-order finite elements (which we assume are of degree $p$), we will require extra smoothness assumptions on $A$, $n,$ and the boundaries $\GD$ and $\GI$. We also make simple assumptions on $k,$ $\nmin$, and $\NLiD{n}$ in order to simplify our calculations.
\bas[Assumptions for higher-$p$ finite-element method bounds]\label{ass:highp}
Assume
\bit
\item $\GD \neq \emptyset,$
\item $\GD$ and $\GI$ are $\Cpo$,
\item $\Aij \in \CpmooDclos$ for all $i,j$, and
\item $n \in \LiD \cap \HpmoD \cap \CfloordtD.$
%\item $f \in \HpmoD$
%  \item $\gI \in \HpmhGI$
  \eit
  \eas
  The assumption $\GD \neq \emptyset$ allows us to show the elliptic projection operator defined in \cref{eq:epdef} below is well-defined. The regularity assumptions on $\GD,$ $\GI$, and $A$ ensure that we can apply a shift theorem for a related stationary diffusion equation up to order $p-1$. The assumption on $n$ ensures that for all $m \in [0,p-1]$ and for all $v \in \HmD,$ $nv \in \HmD$ (see \cref{thm:banachalg} below). However, observe that we make no additional assumptions on the data $f$ and $\gI,$ and so the solution $u$ may not be smoother than $H^2.$

We make the following \lcnamecref{ass:htwo} on the solution of \cref{prob:vtedp}. Recall that the adjoint of \cref{prob:vtedp} is the same as \cref{prob:vtedp}, except \cref{eq:vtedp} is replaced by
\beq\label{eq:adjoint}
\aT(v,u) = \overline{\FT(v)} \tforall v \in \HozDDR.
\eeq

\bas\label{ass:htwo}
\Cref{prob:vtedp} (and its adjoint) has a unique solution $u$ in $\HtD$, and there exists $\CAnk>0$ (possibly dependent on $A$, $n,$ and $k$) such that
\beq\label{eq:generalhtwo}
k \NLtD{u} + \SNHoD{u} + \frac1k \SNHtD{u} \leq \CAnk \Cfg,
\eeq
where $\Cfg \de \NLtD{f} + \NLtGI{\gI}$.
\eas

Finally, we make the following \lcnamecref{ass:convenient} to simplify the proofs in this section. \Cref{ass:convenient} is by no means necessary, but greatly simplifies the proofs.

\bas[Assumptions for convenience of proofs]\label{ass:convenient}
Assume: $k \geq 1$, $\NLiD{n} \geq 1,$ $\nmin \leq 1$, $hk \leq 1$, and there exists $\Ctildemin > 0$ independent of $k$ such that $\CAnk k \geq \Ctildemin.$
\eas
%Note that one could require $\NHhGI{\gI}$ on the right-hand side of \cref{eq:generalhtwo} (as $\gI \in \HhGI$); however, since the bound in \cref{thm:tedp} only has $\NLtGI{\gI}$, we use the form \cref{eq:generalhtwo} to include \cref{thm:tedp}.

Throughout this \lcnamecref{sec:fem}, we adopt the following piece of notation.
\beqs
\nvar = \frac{\NLiD{n}}{\nmin},
\eeqs
%% and
%% \beqs
%% \En = \max\set{\frac{\NHsD{n}}{\nmin^2} \st s \in (d/2,p-1];\frac{\NWsiD{n}}{\nmin^2}\st s \in [2,d/2];1}\nvar.
%% \eeqs
%The quantity $\En$ arises from bounds on the `elliptic projection' mentioned above in weighted norms, see \cref{lem:ellprojerrw} below.

Our main \lcnamecref{thm:fembound} is the following \lcnamecref{thm:fembound}.
\bth[Higher-order error bound for the heterogeneous Helmholtz equation]\label{thm:fembound}
Let $u$ be the solution of \cref{prob:vtedp}. Under \cref{ass:highp,ass:htwo,ass:convenient}, there exist constants $\CFEMLt, \CFEMHo, \Chcond > 0$, independent of $h$, $k$ and $n$, such that if
\beq\label{eq:hfemcond}
h \leq \Chcond \Condn(n) \CAnk^{-\frac1{2p}}k^{-1-\frac1{2p}},
\eeq
then the finite-element solution $\uh$ exists, is unique, and satisfies the error bounds
\beq
\NLtD{u-\uh} \leq \CcorLt \CLtn(n)\mleft(h^2 + \CAnk^2 (hk)^{2p}\mright)\Cfg\tand\label{eq:femltbound}
\eeq
\beq
\NW{u-\uh} \leq \CcorHo \CHon(n)\mleft(h + \CAnk (hk)^p + \CAnk^2k (hk)^{2p}\mright)\Cfg,\label{eq:femhobound}
\eeq
where
\beqs
\Condn(n) = \Pfn{p-2}\mleft(\NLiD{n}\mright)^{-\frac1{p+1}}\mleft(\mleft(\En\nvar\mright)^{\half(\floor{\frac{p-1}2}+1)}\NLiD{n}\nmin^{-\frac{p}2}\mright)^{-\frac{p-1}{2p}},
\eeqs
\beqs
\CLtn(n) = \nvar^6\nmin^{-(p+1)} \mleft(\En \nvar\mright)^{\floor{\frac{p-1}2}+1}\Pfn{p-2}\mleft(\NLiD{n}\mright)\tand
\eeqs
\beqs
\CHon(n) = \mleft(1+\mleft(\En\nvar\mright)^{\half(\floor{\frac{p-1}2}+1)}\NLiD{n}\nmin^{-\frac{p}2}\mright)\CLtn,
\eeqs
where the polynomial $\Pfn(x)$ is defined in \cref{eq:p} below, and the $n$-dependent constant $\En$ is defined in \cref{eq:en} below.
\enth
%\bco[Higher-order-finite-element bound]\label{cor:fembound}
%Under the assumptions of \cref{thm:fembound},
%\eco

The proof of \cref{thm:fembound} is on \cpageref{page:thmfembound} below.

Whilst the calculations in this \lcnamecref{sec:fem} are explicit in all the constants involved, these dependencies are complex and, to a large extent, unnecessary to understand the flow of the arguments. Therefore, for ease of reference, the definition of all the constants in this \lcnamecref{sec:fem} (which are many-layered and interdependent) are given in \cref{app:constants}; i.e., any constant introduced or defined in this \lcnamecref{sec:fem} will be listed in \cref{app:constants}.

\subsubsection{Discussion of new finite-element-error bounds}

\bre[Relationship of new bounds to the work of Du and Wu]
In \cite{DuWu:15} Du and Wu proved that the error for high-order finite-element methods for the \emph{homogeneous} Helmholtz interior impedance problem is bounded provided $h^{2p}k^{2p+1}$ is sufficiently small. Our proof follows theirs, and achieves analagous results---observe that the condition \cref{eq:hfemcond} requires $h^{2p}k^{2p+1}$ to be sufficiently small, where the `sufficient smallness' condition depends on $A$ and $n$. We note as above that our results and proof have the following modifications to those of Du and Wu (the modifications listed in order of their impact upon the proof).
\ben
\item We prove bounds for \emph{heterogeneous} $A$ and $n$; in particular, in several places in the proof we must work with $n$-weighted inner products and norms; see \cref{rem:why} for more details on why $n$-weighted inner products and norms are required.
\item We explicitly track all of the constants involved in the proof---our results are completely explicit in $n$, and are in theory explicit in $A$ (see \cref{rem:explicita} below for information on the explicitness in $A$).
\item We allow for the possibility that the Helmholtz problem is trapping---the constant $\CAnk$ appearing in \cref{eq:hfemcond} may depend on $k$ (as well as $A$ and $n$). If the Helmholtz problem was nontrapping, $\CAnk$ would be independent of $k$.
  \item We assume the existence of a Dirichlet scatterer $\Dm,$ as opposed to only considering the interior impedance problem, where $\Dm=\emptyset.$
\een
\ere

\bre[Why are the results not fully explicit in $A$?]\label{rem:explicita}
The condition \cref{eq:hfemcond} and the bounds \cref{eq:femltbound,eq:femhobound} are not fully explicit in $A$, i.e. the constants $\Ccond, \CcorLt,$ and $\CcorHo$ may depend on $A$. This dependence is because the constants in the shift theorem for the stationary diffusion equation (\cref{thm:shift} below) are not explicit in their $A$-dependence. In principle one can determine this dependence (it is determined for a right-hand side in $\LtD$ and a solution in $\HtD$ in \cite[Appendix A]{ChScTe:13}).
\ere
\bre[Why the appearance of $\nvar$?]\label{rem:nvar}
The quantity $\nvar = \NLiD{n}/\nmin$ appears in multiple places in the definition of the $n$-dependent constants $\Condn(n), \CLtn(n),$ and $\CHon(n).$ This appearance is mainly due to the fact that in multiple places in the proof of \cref{thm:fembound}, we must convert from working in an $n$-weighted norm to a standard norm, and then later convert back to an $n$-weighted norm again.

This conversion is usually necessary because certain results (such as \cref{thm:shift,lem:poincare,thm:trace,thm:multiplicativetrace}) are only available in the literature in standard (non-$n$-weighted) norms, and so to apply these results we must first transfer to non-$n$-weighted norms, apply the results, and then tranfer back. If one could prove these results for $n$-weighted norms (under sufficient smoothness conditions on $n$) with constants that were completely explicit in $n$, then many of the instances of $\nvar$ could be removed.
\ere

\bre[Bounds not sharp in $n$]\label{rem:nsharp}
The $n$-dependence of the constants $\Condn(n), \CLtn(n),$ and $\CHon(n)$ is almost certainly not sharp, because
\ben
\item The proof of \cref{thm:fembound} is complex, and involves recursively applying bounds on finite-element functions (as in, e.g., the proof of \cref{lem:negdiscsum}, see \cref{eq:evenrecursivesum,eq:oddrecursive}). These arguments may well result in non-sharp $n$-dependence.
  \item As described in \cref{rem:nvar} above, many of the appearances of $\nvar$ in the constants $\Condn(n), \CLtn(n),$ and $\CHon(n)$ is purely due to changing norms, then changing back, and so it is possible that the dependence of these constants on $\nvar$ (and therefore $n$) is not sharp.
\een
\ere

\bre[$p$-dependence]
To prove error bounds for $hp$-finite-element methods, one must know explicitly how the error bounds \cref{eq:femltbound,eq:femhobound} depend on $p$ as well as $h$ and $k$. Such an analysis was carried out for the Helmholtz equation in homogeneous media by Melenk and Sauter in \cite{MeSa:10,MeSa:11} and for the related time-harmonic Maxwell's equations in homogeneous media in \cite{MeSa:18}. In principle (as our results are explicit in all of the constants involved) one could calculate the $p$-dependence, however, such dependence is likely to be highly complicated, and may very well not be sharp, for similar reasons that the $n$-dependence of our results may not be sharp, as outlined in \cref{rem:nsharp} above.
\ere

\bre[Special cases of $n$]
In order to understand the constants $\Condn(n), \CLtn(n),$ and $\CHon(n)$ it may be instructive to note their behaviour in the following three cases:
\bit
\item If $n=1,$ then $\Condn(n)=1, \CLtn(n)=1,$ and $\CHon(n)=2$.
\item If $\nmin$ is fixed, and $\NLiD{n} \rightarrow \infty,$ then $\Condn(n)\rightarrow 0,$ (i.e., the condition \cref{eq:hfemcond} becomes more restrictive), $\CLtn(n)\rightarrow \infty,$ and $\CHon(n)\rightarrow \infty$ (i.e., the right-hand side of the error bounds \cref{eq:femltbound,eq:femhobound} become larger).
\item If $\NLiD{n}$ is fixed, and $\nmin \rightarrow 0,$ then $\Condn(n)\rightarrow 0,$ $\CLtn(n)\rightarrow \infty,$ and $\CHon(n)\rightarrow \infty$ (i.e., the right-hand side of the error bounds \cref{eq:femltbound,eq:femhobound} become larger).
  \eit
  One can easily check each of these behaviours using the definition of $\nvar$, the definition of the constants $\Condn(n),$ $\CLtn(n)$, and $\CHon(n)$, and the definition of the polynomial $\Pfn{j}(x)$ in \cref{eq:p} below, and the definition of $\En$ in \cref{eq:en} below.
  \ere

  \bre[Extension to different boundary conditions on $\GI$]
  As in \cite[Remark 5.4(e)]{DuWu:15}, we remark that it is not obvious how to extend the proof of \cref{thm:fembound} to a Helmholtz problem with an exact Dirichlet-to-Neumann boundary condition on $\GI.$ In \cref{lem:higherbound,lem:ltthetahbound} below one must bound terms involving $\thetah$ on the truncation boundary $\GI$; these bounds are achieved using \cref{lem:boundarybound}, where we bound $\NLtGI{\thetah}$ by higher-order discrete norms of $\thetah$ and by the weighted $H^1$-norm of $\rho.$

  However, a crucial part of the proof of \cref{lem:boundarybound} is the fact that one has the bound $k \NLtGI{\thetah}^2 \lesssim \IPLtGI{\DtNapprox \thetah}{\thetah}$, where $\DtNapprox$ is the impedance approximation to the DtN map $\DtN,$ i.e., $\DtNapprox = ik.$ To replicate the proof of \cref{lem:boundarybound} for an exact DtN boundary condition, one would need to show $k \NLtGI{\thetah}^2 \lesssim \IPLtGI{\DtN \thetah}{\theta}$ for the \emph{exact} DtN map $\DtN.$ As mentioned in \cite[Remark 5.4(e)]{DuWu:15}, whether such a bound holds in general is an open question; such as bound has only been shown in \cite[Equation (3.4b)]{MeSa:10} for the case $\GI$ is a circle (in 2-d) or a sphere (in 3-d).
  \ere

\subsection{Decomposition of solution and best approximation bound}\label{sec:decomp}

For the first part of the proof of \cref{thm:fembound}, we prove a best approximation bound in $\Vhp$ for the solution of the Helmholtz equation following the presentation in \cite{ChNi:19} (although we explicitly keep track of the constants involved at each point). In order to obtain bounds for high $p$, we require the following shift \lcnamecref{thm:shift}:

\bth[Shift theorem]\label{thm:shift}
Under \cref{ass:highp}, for all integers $l \in \mleft[0,p-1\mright]$ there exists a constant $\CAl>0$ (depending on $A$) such that if $\ftilde \in \HlD$ and $\gItilde \in \HlphGI$, then there exists a unique $\utilde \in \HlptD$ such that $\utilde$ solves
\beq\label{eq:shifteq}
\grad \cdot \mleft(A \grad \utilde\mright) = -\ftilde,
\eeq
\beq\label{eq:shiftdbc}
\dn \utilde = \gItilde, \tand
\eeq
\beq\label{eq:shiftnbc}
\trD \utilde = 0
\eeq
and $\utilde$ satisfies the bound
\beq\label{eq:shift}
\NHlptD{\utilde} \leq \CAl \mleft(\NHlD{\ftilde} + \NHlphGI{\gItilde}\mright).
\eeq
\enth

\bpf[Proof of \cref{thm:shift}]
The uniqueness and existence of $\utilde$ (in $\HozDD$) follows from the Lax--Milgram theorem, as the variational formulation of \cref{eq:shifteq,eq:shiftdbc,eq:shiftnbc} is bounded and coercive. The proof for the higher regularity bounds uses standard elliptic regularity estimates in the interior and near the boundaries $\GD$ and $\GI$, and the work of the proof is combining these estimates. As a reference for these estimates we use \cite[pp. 137-138]{Mc:00}; \cref{ass:highp} means we can apply these estimates, as we have the necessary higher regularity of the coefficients and the boundaries $\GD$ and $\GI$.

To deal with the interior regularity and regularity near the boundary separately, we define the following subsets of $D$: $\Dint,\Dinttilde,\Dscat,$ and $\Dtrunc$ (see \cref{fig:shift-int,fig:shift-scat,fig:shift-trunc} for a schematic) with the following properties:
\bit
\item $\Dint \compcont \Dinttilde \compcont D$,
\item $\GD \subset \Dscatclos$
\item $\dist(\Dscat,\GI) > 0 $
  \item $\GI \subset \Dtruncclos$
\item $\dist(\Dtrunc,\GD) > 0 $
  \eit
\def\shiftscale{1.4}
  \begin{figure}
\begin{centering}
\begin{tikzpicture}[even odd rule]
  \input{diagram-defs-smooth}
\draw[scale=\shiftscale] % Truncation
\truncationsmooth
\obstaclesmooth
;
    

\draw[scale=\shiftscale] % The obstacle
\obstaclesmooth
;




% The labels
\draw ($\shiftscale/\domainscale*(-0.1,0)$) node[fill=white] {$\Dm$};
%\draw (-3.5,-4.75) node[fill=white] {$D:=\Dtilde\setminus \clos{\Dm}$};

\draw ($\shiftscale/\domainscale*(2.0,-0.5)$) node[fill=white] {$\GD$};
%\draw[scale=\shiftscale] (2.418221983093764, 0.6211657082460498) node[fill=white] {$\BR$};
\def\GIdraw{\draw ($\shiftscale/\domainscale*(5.1,3.1)$) node[fill=white] {$\GI$};}
\GIdraw

% D_int
\draw[scale=\shiftscale] (0,0) circle [radius=2.2];
\draw[scale=\shiftscale] (0,0) circle [radius=1.5];

%D_int_tilde
\draw[scale=\shiftscale,dashed] (0,0) circle [radius=1.3];
\draw[scale=\shiftscale,dashed] (0,0) circle [radius=2.4];


\end{tikzpicture}

\caption{A schematic of the sets $\Dint$ (between solid circles only) and $\Dinttilde$ (between dotted circles) from the proof of \cref{thm:shift}.}\label{fig:shift-int}
\end{centering}
  \end{figure}
    \begin{figure}
\begin{centering}
\begin{tikzpicture}[even odd rule]
  \input{diagram-defs-smooth}
\draw[scale=\shiftscale] % Truncation
\truncationsmooth
\obstaclesmooth
;
    

\draw[scale=\shiftscale] % The obstacle
\obstaclesmooth
;




% The labels
\draw ($\shiftscale/\domainscale*(-0.1,0)$) node[fill=white] {$\Dm$};
%\draw (-3.5,-4.75) node[fill=white] {$D:=\Dtilde\setminus \clos{\Dm}$};

\draw ($\shiftscale/\domainscale*(2.0,-0.5)$) node[fill=white] {$\GD$};
%\draw[scale=\shiftscale] (2.418221983093764, 0.6211657082460498) node[fill=white] {$\BR$};
\def\GIdraw{\draw ($\shiftscale/\domainscale*(5.1,3.1)$) node[fill=white] {$\GI$};}
\GIdraw

% D_scat
  \filldraw[pattern=owennorthwest,scale=\shiftscale] (0,0) circle [radius=2.4]
  \obstacle;

  \draw ($\shiftscale*(1.27,1.27)$) node[fill=white] {$\Dscat$};
\end{tikzpicture}

\caption{A schematic of the set $\Dscat$ from the proof of \cref{thm:shift}.}\label{fig:shift-scat}
\end{centering}
    \end{figure}
      \begin{figure}
\begin{centering}
\begin{tikzpicture}[even odd rule]

    \input{diagram-defs-smooth}
\draw[scale=\shiftscale] % Truncation
\truncationsmooth
\obstaclesmooth
;
    

\draw[scale=\shiftscale] % The obstacle
\obstaclesmooth
;




% The labels
\draw ($\shiftscale/\domainscale*(-0.1,0)$) node[fill=white] {$\Dm$};
%\draw (-3.5,-4.75) node[fill=white] {$D:=\Dtilde\setminus \clos{\Dm}$};

\draw ($\shiftscale/\domainscale*(2.0,-0.5)$) node[fill=white] {$\GD$};
%\draw[scale=\shiftscale] (2.418221983093764, 0.6211657082460498) node[fill=white] {$\BR$};
\def\GIdraw{\draw ($\shiftscale/\domainscale*(5.1,3.1)$) node[fill=white] {$\GI$};}
\GIdraw

% D_trunc
  \filldraw[pattern=owennorthwest,scale=\shiftscale] (0,0) circle [radius=1.4]
  \truncation;

  \draw ($\shiftscale*(1.5,1.5)$) node[fill=white] {$\Dtrunc$};
\GIdraw

\end{tikzpicture}

\caption{A schematic of the set $\Dtrunc$ from the proof of \cref{thm:shift}.}\label{fig:shift-trunc}
\end{centering}
\end{figure}

    First applying interior regularity \cite[Theorem 4.16]{Mc:00} in $\Dinttilde,$ we obtain the bound
    \beq\label{eq:shiftint}
\NHlptDint{\utilde} \leq \CintAl \mleft(\NHoDinttilde{\utilde} + \NHlDinttilde{\ftilde}\mright).
\eeq
Applying regularity up to the boundary for Dirichlet data \cite[Theorem 4.18 (i)]{Mc:00} in $\Dscat,$ we obtain (as $\trD \utilde = 0$)
\beq\label{eq:shiftscat}
\NHlptDscat{\utilde} \leq \CscatAl \mleft(\NHoD{\utilde} + \NHlD{\ftilde}\mright)
\eeq
and similarly for Neumann data \cite[Theorem 4.18 (ii)]{Mc:00} in $\Dtrunc,$ we obtain
\beq\label{eq:shifttrunc}
\NHlptDtrunc{\utilde} \leq \CtruncAl \mleft(\NHoD{\utilde} + \NHlphGI{\dn \utilde} + \NHlD{\ftilde}\mright).
\eeq
Combining \cref{eq:shiftint,eq:shiftscat,eq:shifttrunc}, we obtain \cref{eq:shift}.
\epf




%% The following trace theorem is standard, see, e.g., \cite[Theorem 3.37]{Mc:00}.


The following \lcnamecref{lem:domainshift} follows from \cref{thm:shift}.

\bco\label{lem:domainshift}
Under \cref{ass:highp}, let $\ftilde \in \HlD$ and $\gtilde \in \HlpoD,$ for $0 \leq l \leq p-1$. If $\utilde \in \HlptD$ solves
\beqs
\grad \cdot \mleft(A\grad \utilde\mright) = -\ftilde,
\eeqs
\beqs \trGD u = 0,
\eeqs
and
\beqs
\dn u = \trGI \gtilde,
\eeqs
then
\beqs
\NHlptD{\utilde} \leq \CAl\mleft(1+\CTrlpo\mright)\mleft(\NHlD{\ftilde} + \NHlpoD{\trGI \gtilde}\mright).
\eeqs
\eco

The proof of \cref{lem:domainshift} requires the Trace theorem.

\bth[Trace Theorem]\label{thm:trace}
If $v \in \HmD,$ for $1/2 < m \leq p+1$, then there exists $\CTrm > 0$ independent of $v$ such that
\beqs
\NHmmhGI{\trI v} \leq \CTrm \NHmD{v}.
\eeqs
\enth

For a proof of \cref{thm:trace}, see \cite[Theorem 3.37]{Mc:00}

\bpf[Proof of \cref{lem:domainshift}]
By \cref{thm:shift,thm:trace}
\beqs
\NHlptD{\utilde} \leq \CAl \mleft(\NHlD{\ftilde} + \NHlphGI{\gtilde}\mright) \leq \CAl \mleft(\NHlD{\ftilde} + \CTrfn{l}\NHlpoD{\gtilde}\mright),
\eeqs
and the result follows.
\epf

We are now able to prove the following \lcnamecref{thm:expansion} that gives a decomposition of the solution $u$ of \cref{prob:vtedp} into lower-order, less oscillatory parts $\uj$ and a smoother, more-oscillatory part $\uosc$. The main advantage of this \lcnamecref{thm:expansion} is that it enables us to prove a higher-order best-approximation bound (\cref{lem:bestapprox} below) for solutions of the Helmholtz equation, \emph{even though} the solutions do not have high regularity.

\bth[Expansion of the solution of the Helmholtz equation]\label{thm:expansion}
Under \cref{ass:highp,ass:htwo} if $u$ is the solution of \cref{prob:tedp} or its adjoint then there exists $\uosc \in \HppoD$ and a sequence $\usj \in \HjptD,$ $j = 0,\ldots,p-2$ such that
\beq\label{eq:expansionid}
u = \uosc + \sum_{j=0}^{p-2} \usj.
\eeq
Furthermore,
\beq\label{eq:expansionuj}
\NHjptD{\usj} \leq \Cej \Pj(\NLiD{n}) k^j \Cfg,
\eeq
and
\beq\label{eq:expansionuosc}
\NHppoD{\uosc} \leq \Cosc \CAnk k^p \Cfg,
\eeq
where
\beq\label{eq:p}
\Pj(x) =
\begin{dcases}
1 & j = 0,1\\
x^{\floor{j/2}}& j \geq 2
\end{dcases}
\eeq.
%=\sum_{m=0}^{j-2}\pfn{j,m}x^m $
%% are polynomials of degree
%% \beq\label{eq:polydegree}
%% \begin{dcases}
%% \frac{j}2 & \tif  j \text{ is even}\\
%% \frac{j-1}2 \tif j  \text{ is odd}.
%% \end{dcases}
%% \eeq
%% (except for $j=0,1,$ where $\Pj$ is a polynomial of degree 0) given by the recurrence relation
%% \beq\label{eq:pjdef}
%% \Pj(x) = \CAj \mleft(1+\CTrjpo\mright)\mleft(x\Pfn{j-2}(x) + \Pfn{j-1}(x)\mright).
%% \eeq
% No idea if this coefficient stuf is right
%% \begin{align}
%% \label{eq:p1}\pfn{0,0}&= \CAz\\
%% \label{eq:p2}\pfn{1,0}&  = \CAz\CAo\mleft(1+\CTrt\mright)\\
%% \label{eq:p3}\pfn{j,m} &=
%% \begin{dcases}
%%  0, &\tif m \leq \floor{\frac{j-2}2}\\
%% \CAj\mleft(1+\CTrjpo\mright) \mleft(\pfn{j-2,m-1} + \pfn{j-1,m-1}\mright),&\tif \floor{\frac{j-2}2} < m \leq j-3
%% \end{dcases}\\
%% \label{eq:p4}\pfn{j,j-2}& = \CAj\mleft(1+\CTrjpo\mright) \pfn{j-1,j-3},
%% \end{align}
%% and $\Cosc$ is given by the recurrence relations
%% \begin{align}
%% \label{eq:osc1}\Coscfn{1} &= \CAo\mleft(1+\CTrt\mright),\\
%% \label{eq:osc2}\Coscfn{2} &= \CAt\mleft(1+\CTrth\mright)\mleft(1 + \CAo\mleft(1+\CTrt\mright)\mright)\tand\\
%% \label{eq:osc3}\Coscfn{j} &= \CAfn{j}\mleft(1+\CTrfn{j+1}\mright)\mleft(\Coscfn{j-1} + \Coscfn{j-2}\mright),
%% \end{align}
%% and
%% \beq\label{eq:cosc}
%% \Cosc = \frac{\Cefn{p-1}}{\max\set{1,\CAz}}.
%% \eeq
\enth

\bre[How oscillatory are the functions in \cref{thm:expansion}?]\label{rem:osc}
Recall that a higher power of $k$ appearing in an a priori bound indicates that a function is more oscillatory. In \cref{eq:expansionuj} below, the $(j+2)$th-order norm of $\usj$ is of order $k^j$ (i.e., the power of $k$ is \emph{two} orders of magnitude less than the order of the norm) whereas in \cref{eq:expansionuosc} the $(p+1)$st-order norm of $\uosc$ is of order $k^p$ (the power of $k$ is \emph{one} order of mangitude less that the order of the norm). Therefore, in this sense, $\uosc$ is `more oscillatory' than $\usj.$
\ere

\Cref{thm:expansion} is essentially just \cite[Theorem 1]{ChNi:19} in the particular case of a Helmholtz problem, but with the dependence on all the constants kept track of. The results in \cite{ChNi:19} are stated for a wider class of time-harmonic wave propagation problems, but the dependence on all of the constants is not made explicit.

We also point out (as in \cite[Section 7]{ChNi:19}) that \cref{thm:expansion} is reminiscent of results by Melenk and Sauter in \cite{MeSa:10,MeSa:11}who show, for the homogeneous Helmholtz equation that the solution $u$ can be decomposed as $u = \uHt + \uA,$ where $\uHt \in \HtD$ but is not oscillatory ($\NHtD{\uHt} \lesssim 1$), and $\uA$ is analytic but $\uA$ is oscillatory ($\NHmD{\uA} \lesssim k^{m-2}$ for all $m$)\footnote{These results are proved with no obstacle and $f$ given by a Dirac delta function in \cite[Lemma 3.5]{MeSa:10} and for: (i) the IIP with a bounded Lipschitz boundary that is either a 2-d polygon or analytic, or (ii) the EDP with an analytic scatterer, in \cite[Theorems 4.10, 4.20]{MeSa:11} respectively. In the former case the proof is under an assumption of a polynomial growth of the a priori bound \cite[Assumption 4.8]{MeSa:11}, i.e., $\CAnk$ is a polynomial in $k$.}. This decomposition is used in \cite{MeSa:10,MeSa:11} to prove convergence results for $hp$-finite element methods for the Helmholtz equation. %We note that the results in this \lcnamecref{sec:fem} could, in principle, be used in the analysis of $hp$-methods, as it is, in principle, possible to track the dependence of the constants on the polynomial degree $p.$

\bpf[Proof of \cref{thm:expansion}]
The idea of the proof is as follows. We write $u$ as a formal series expansion
\beq\label{eq:formalseries}
u = \sum_{j=0}^\infty \usj,
\eeq
and then substitute this series into the PDE \cref{eq:tedp} and the boundary condition \eqref{eq:ibc}. Equating powers of $k$, we derive a recursive sequence of stationary diffusion equations for the functions $\usj,$ with right-hand sides dependent on $\ujmo$ and $\ujmt$. We use this recursive sequence and \cref{lem:domainshift} to prove the a priori bounds \cref{eq:expansionuj}.

We then define the $l$th remainder $\rl = u - \sum_{j=0}^{l-1} \usj,$ and by applying the operator $\grad\cdot\mleft(A\grad \cdot\mright)$ with Neumann boundary conditions to $\rl$, we obtain a recursive sequence for the remainders $\rl,$ and can similarly prove a priori bounds for the functions $\rl$. The oscillatory function $\uosc$ is then just $\rpmo.$ The format of this proof is identical to that in \cite[Theorem 1]{ChNi:19}, except we keep track of all of the constants involved.

For the purposes of the proof, it is more convenient to define $\vj = \usj/k^j,$ so that the series expansion \cref{eq:formalseries} becomes\footnote{In \cite{ChNi:19} the notation is changed slighty, and the series expansion is defined as $u = \sum_{j=0}^\infty k^j \uj,$ i.e., the functions $\vj$ in our proof are denoted $\uj$ in \cite{ChNi:19}.}
\beq\label{eq:formalseriesv}
u = \sum_{j=0}^\infty k^j\vj
\eeq
as in \cite{ChNi:19}. Also, in this proof, all the boundary-value problems involve included a zero Dirichlet condition on the scatterer $\GD;$ we omit this boundary condition throughout the proof for brevity.

By applying the Helmholtz operator to the formal series \eqref{eq:formalseriesv} we obtain the following equations for $\vj \in \HjptD, j \geq 1$:
\beqs
\grad \cdot \mleft(A\grad \vz\mright) = -f \quad\tand\quad \dn \vz = \gI,
\eeqs
\beqs
\grad \cdot \mleft(A\grad \vo\mright) = 0\quad\tand\quad\dn \vo = i\vz,
\eeqs
and
\beq\label{eq:vj}
\grad \cdot \mleft(A\grad \vj\mright) = - n\vjmt\quad\tand\quad\dn \vz = i\vjmo \quad\tfor j \in \mleft[2,p-2\mright].
\eeq

By \cref{thm:shift} we immediately conclude the bound
\beq\label{eq:expuz}
\NHtD{\vz} \leq \CAz\Cfg \leq \Cefn{0}\Cfg,
\eeq
i.e., \cref{eq:expansionuj} for $j=0.$ By \cref{lem:domainshift,eq:expuz} we can conclude the bound
\beqs
\NHthD{\vo} \leq \CAo \mleft(1+\CTrt\mright)\NHtD{i\vz} \leq \max\set{1,\CAo}\mleft(1+\CTrfn{2}\mright)\Cefn{0} \Cfg = \Cefn{1}\Cfg,
\eeqs
i.e., \cref{eq:expansionuj} for $j=1$.

We prove the bound \cref{eq:expansionuj} for higher $j$ by induction. Let $j \in \mleft[2,p-2\mright]$ and suppose \cref{eq:expansionuj} holds for all $s \in [0,j-1].$ Using \cref{lem:domainshift}, we conclude that (using the observation that $\Cefn{j-1} = 2\max\set{1,\CAjmo}\mleft(1+\CTrj\mright)\Cefn{j-2}\geq \Cefn{j-2}$)
\begin{align*}
\NHfn{j+2}{D}{\vj} &\leq \CAj \mleft(1+\CTrjpo\mright)\mleft(\NLiD{n} \NHfn{j}{D}{\vfn{j-2}} + \NHfn{j+1}{D}{\vfn{j-1}}\mright)\\
&\leq \max\set{1,\CAj} \mleft(1+\CTrjpo\mright)\mleft(\NLiD{n} \Cefn{j-2} \Pfn{j-2}\mleft(\NLiD{n}\mright) + \Cefn{j-1} \Pfn{j-1}\mleft(\NLiD{n}\mright)\mright)\Cfg\\
&\leq 2\max\set{1,\CAj} \mleft(1+\CTrjpo\mright)\Cefn{j-1} \Pfn{j}(\NLiD{n})\Cfg\\
&= \Cefn{j} \Pfn{j}(\NLiD{n})\Cfg.
\end{align*}
%% Therefore the definition of the polynomials $\Pj$ \cref{eq:pjdef} holds (and it is straightforward to see that $\deg\mleft(\Pj\mright) = \deg\mleft(\Pfn{j-2}\mright)+1$, where $\deg$ denotes polynomial degree, and therefore since $\deg\mleft(\Pfn{0}\mright) = \deg\mleft(\Pfn{1}\mright)=0,$ \cref{eq:polydegree} holds. Hence, by the relationship between $\usj$ and $\vj$, we have the bound \cref{eq:expansionuj}.

We will now define the remainders $\rl$, and proceed similarly.
Let $\ro \in \HthD$ solve
\beqs
\grad \cdot \mleft(A\grad \ro\mright) = -k^2 u \tand \dn \ro = ik\trGI u.
\eeqs
Then by \cref{lem:domainshift}
\beqs%gin{align*}
\NHthD{\ro} \leq \CAo\mleft(1+\CTrt\mright)\mleft(k^2\NHoD{u} + k\NHtD{u}\mright)\leq \CAo\mleft(1+\CTrt\mright)k^2\CAnk\Cfg\leq \frac{\Cefn{1}}{\max\set{1,\CAz}}k^2\CAnk \Cfg
\eeqs%nd{align*}
by definition of $\Cefn{1}$.
Let $\rt \in \HfD$ solve
\beqs
\grad \cdot \mleft(A\grad \rt\mright) = -k^2 u \tand \dn \rt = ik\trGI \ro.
\eeqs
Then by \cref{lem:domainshift}
\begin{align*}
\NHfD{\rt} &\leq \CAt\mleft(1+\CTrth\mright)\mleft(k^2\NHtD{u} + k\NHthD{\ro}\mright)\\
&\leq \CAt\mleft(1+\CTrth\mright)\mleft(1 + \frac{\Cefn{1}}{\max\set{1,\CAz}}\mright)\CAnk k^3\Cfg\\
&\leq 2\CAt\mleft(1+\CTrth\mright)\frac{\Cefn{1}}{\max\set{1,\CAz}}\CAnk k^3 \Cfg\text{ as } \Cefn{1} \geq 1,\\
&\leq \frac{\Cefn{2}}{\max\set{1,\CAz}}\CAnk k^3\Cfg.
\end{align*}
Then for $j \geq 3,$ let $\rj \in \HjptD$ solve
\beqs
\grad \cdot \mleft(A\grad \rt\mright) = -k^2 \rjmt \tand \dn \rj = ik\trGI\rjmo.
\eeqs
And by induction and \cref{lem:domainshift} again, letting $\uosc = \rfn{p-1},$ we have \cref{eq:expansionuosc}. It is straightforward to see that $\rfn{p-1} + \sum_{j=1}^{p-2} \uj$ solves \cref{prob:tedp}, and therefore \cref{eq:expansionid} holds, as $u$ is unique.
\epf
%\{Just a note while I think about it - could you do DtN boundary conditions in this proof by testing with a different function when wanting to bound the $L^2$ norm on the boundary? I wonder if testing with the NtD of $u$ would mean you end up with a $\LtGI{u}^2$ term, and then you could use properties of the NtD map to bound other bits. Might foil $k$-dependency though.}
Using the expansion in \cref{thm:expansion}, we can prove the following error bound for the best approximation of $u$ in $\Vhp$:

%% We also have the following best-approximation error in higer-order Sobolev spaces:
%% \ble[Best approximation in $\HmD$]\label{lem:bestapproxhigh}
%% For integer $s$ in $[1,p+1]$, there exists $\Cinterps>0$ such that for every $v \in \HmD$ there exists $\vhhat \in \Vhp$ such that
%% \beqs
%% \NLtD{v - \vhhat} + h\NHoD{v-\vhhat} \leq \Cinterps h^s \SNHmD{v}.
%% \eeqs
%% \ele

\ble[Best approximation error bound]\label{lem:bestapprox}
Under \cref{ass:highp,ass:htwo}, there exist constants $\CFEMo, \CFEMt > 0$ independent of $k$ and $n$ (although dependent on $A$ and $p$) such that if $u$ solves \cref{prob:vtedp} or its adjoint, then there exists $\uhhat \in \Vhp$ such that
\beq\label{eq:bestapproxL2}
\NLtD{u-\uhhat} \leq \Pfn{p-2}\mleft(\NLiD{n}\mright)\mleft(\CFEMo  h^2 + \CFEMt \CAnk h\mleft(hk\mright)^p \mright)\Cfg,
\eeq
%% \beq\label{eq:bestapproxH1}
%% \NHoD{u-\uhhat} \leq \mleft(\CFEMo h + \CFEMt \CAnk \mleft(hk\mright)^p \mright)\Cfg \tand
%% \eeq
\beq\label{eq:bestapproxW}
\NW{u-\uhhat} \leq 2\Pfn{p-2}\mleft(\NLiD{n}\mright)\mleft(\CFEMo  h + \CFEMt \CAnk \mleft(hk\mright)^p \mright)\Cfg.
\eeq
\ele

%We write bounds for the standard and weighted $H^1$ norms separately, as we will use each individual bound in differents of proofs in \cref{sec:fembound}.

\bpf[Proof of \cref{lem:bestapprox}]
We apply \cref{lem:scottzhang} to all the $\usj$ and $\uosc$ in \cref{thm:expansion}, and obtain that there exist $\ujh \in \Vhp$ $j=0,\ldots,p-2$ and $\uosch \in\Vhp$ such that 
\beqs
\NLtD{\usj - \ujh} + h\NHoD{\usj - \ujh} \leq \Cinterpfn{j+2} \Cefn{j} \Pj\mleft(\NLiD{n}\mright) h^{j+2}k^j \Cfg
\eeqs
and
\beqs
\NLtD{\uosc - \uosch} + h\NHoD{\uosc - \uosch} \leq \Cinterpfn{p+1} \Cosc\CAnk h^{p+1}k^p \Cfg.
\eeqs
Therefore, by letting $\uhhat = \uosch + \sum_{j=0}^{p-2} \ujh,$ we have \cref{eq:bestapproxL2,eq:bestapproxW} (using the facts that $hk \leq 1$ and $\Pfn{p-2}(\NLiD{n}) \geq \Pfn{j}(\NLiD{n}) \geq 1$ for all $j \leq p-2$ since $\NLiD{n} \geq 1$).
\epf

\subsection{Routine analysis results}\label{sec:anbackground}
In this \lcnamecref{sec:anbackground} we collect together routine results that we use throughout the following proofs.

The following inequality is straightforward to prove: for $N \in \NN$ and $a_1,\ldots,a_N > 0$
\beq\label{eq:simple}
\sqrt{\sum_{j=1}^N a_j^2} \leq \sum_{j=1}^N a_j.
\eeq



\bth[Multiplicative Trace Inequality]\label{thm:multiplicativetrace}%BS THm 1.6.6
There exists $\CMT > 0$ such that for all $v \in \HoD$
\beqs
\NLtdD{v} \leq \CMT \NLtD{v}^\half \NHoD{v}^\half.
\eeqs
\enth

\ble[Poincar\'{e}--Friedrichs Inequality]\label{lem:poincare}
Let $\Gamma \subseteq D$ have nonvanishing $d-1$-dimensional measure. There exist constants $\CP,\Ctilde > 0$ depending only on $D$ and $\Gamma$ such that for all $v \in \HoD$

\beqs
\NLtD{v}^2 \leq \CP \SNHoD{v}^2 + \Ctilde \NLtGI{v}^2.
\eeqs

In particular, taking $\Gamma = \GD,$ for all $v \in \HozDD$
\beq\label{eq:poincare}
\NLtD{v} \leq \CP \SNHoD{v}.
\eeq
\ele
For a proof of \cref{lem:poincare} see \cite[Lemma A.14]{ToWi:05}.

\optodo{Think carefully about what this means for IIP---I think it means you need to include the impedance part in the elliptic projection, because otherwise the Poisson problem doesn't have a unique solution, so you're not going to be able to use Lax--Milgram. But then we don't have the higher-order shift Theorems. Write a remark to this effect, and say that we're going to assume the scatterer is non-empty---put this in an assumption. Scrap all of the above - Du and Wu add a lower--order term into their elliptic projection, which means it's always coercive (because the energy norm is the $H^1$ norm. Write a remark explaining this instead, and say that we're not going to do it, but you could. I think that's best.}
\bth[Multiplication in $\HmD$]\label{thm:banachalg}
\ben
\item\label[itempart]{it:ban1}If $m > d/2,$ then for all $\vo, \vt \in \HmD$, $\vo\vt \in \HmD$ and there exists a constant $\CBanfn{m} > 0$ independent of $\vo$ and $\vt$ such that
\beqs
\NHmD{\vo\vt} \leq \CBanfn{m} \NHmD{\vo}\NHmD{\vt}.
\eeqs
\item\label[itempart]{it:ban2}For any $m \in \NN,$ if $\vo \in \CmD$, $\supp\mleft(1-\vo\mright) \compcont D,$ and  $\vt \in \HmD,$ then
  \bit
\item$\vo \in \WmiD,$
\item$\vo\vt \in \HmD$, and
  \item there exists a constant $\Cprod{m} > 0$ independent of $\vo$ and $\vt$ such that%If $\vo \in \CrcompD$ for some $r \in \NN, r \geq m,$ and
\beq\label{eq:ban2}
\NHmD{\vo\vt} \leq \Cprod{m} \mleft(1+\NWmiD{\vo}\mright)\NHmD{\vt}.
\eeq
\eit
\een
\enth

\bpf[Proof of \cref{thm:banachalg}]
\Cref{it:ban1} is given in \cite[Section 1.8.1, Theorem]{Ma:11}. For \cref{it:ban2}, observe that as $1-\vo$ has compact support, $\vo \in \WmiD,$ and $1-\vo \in \CmcompD$. Therefore, by \cite[Theorem 3.20]{Mc:00}, there exists $\Cmclean{m} > 0$ such that\footnote{In \cite[Theorem 3.20]{Mc:00} $\Cmclean{m}$ is denoted $C_{m}$.}
\beqs
\NHmD{(1-\vo)\vt} \leq \Cmclean{m} \NWmiD{1-\vo}\NHmD{\vt}.
\eeqs
Therefore as $\NHmD{\vo\vt} \leq \NHmD{\mleft(\vo-1\mright)\vt} + \NHmD{\vt} = \NHmD{(1-\vo)\vt} + \NHmD{\vt}$, we have \cref{eq:ban2}.
\epf



\subsection{Error bounds for Galerkin projections}\label{sec:errgalerkin}
In this \lcnamecref{sec:errgalerkin} we state a sequence of error bounds in negative Sobolev norms for two different projection operators. The proofs of these error bounds are all simple modifications of the standard duality-argument proofs of finite-element errors in negative Sobolev norms, as in, e.g., \cite[Theorem 5.8.3]{BrSc:08}.
%\ednote{Both-should I write out at least one of these modified proofs?}. % Note to self, I have these proofs sketched out, but just use the argument from Brenner and Scott and use an L^2 inner product on the function, not their gradients, for the elliptic projection.
We first define the projections we use.

Define the elliptic projection $\Ph:\HozDD\rightarrow\Vhp$ by, for $w \in \HozDD$
\beq\label{eq:epdef}
\aep(\Ph w,\vh) = \aep(w,\vh) \tforall \vh \in \Vhp,
  \eeq
  where
  \beq\label{eq:aepdef}
  \aep(\vo,\vt) = \IPLtD{\grad \vo}{\grad \vt}.
  \eeq
%% and define the weighted elliptic projection $\Phn:\HozDD\rightarrow\Vhp$ by, for $w \in \HozDD$
%% \beqs
%% \IPLtD{A\grad\vh}{\grad\Phn w} = \IPLtDn{A\grad\vh}{\grad w} \tforall \vh \in \Vhp.
%% \eeqs
  Observe that $\Ph w$ is the finite-element approximation of the solution of the stationary diffusion problem with `diffusion coefficient' $A$ and right-hand side in $\HozDD'$ given by $\aep(w,\cdot).$
  
%% We define the $\LtD$-projection $\Qh:\HozDD\rightarrow \Vhp$ by, for $w \in \HozDD$
%% \beqs
%% \IPLtD{\Qh w}{\vh} = \IPLtD{w}{\vh} \tforall \vh \in \Vhp.
%% \eeqs
Define the $\LtD$ projection in the $n$-weighted norm $\Qhn:\HozDD\rightarrow \Vhp$ by, for $w \in \HozDD$
\beqs
\IPLtDn{\Qhn w}{\vh} = \IPLtDn{w}{\vh} \tforall \vh \in \Vhp,
\eeqs

where, for $v,w \in \LtD,$ $\IPLtDn{v}{w}$ is the $n$-weighted inner product
\beqs
\IPLtDn{v}{w} \de \int_{D} n v \wbar.
\eeqs

We also define the corresponding $n$-weighted $\LtD$ norm $\NLtDn{v} = \sqrt{\int_{D} n \abs{v}^2}$ and for $m \in \NN$, the $n$-weighted $\HmD$ norms
\beqs
\NHmDn{v}^2 \de \sum_{\alpha \st \abs{\alpha} \leq s}\NLtDn{D^\alpha v},
\eeqs
and the negative weighted Sobolev norms
\beq\label{eq:negweightnorm}
\NHnfn{-m}{D}{v} \de \sup_{w \in \HmD} \frac{\IPLtDn{v}{w}}{\NHmDn{w}}.
\eeq

Observe that, for $v \in \HmD,$
\beq\label{eq:nconv}
\nmin\NHmD{v} \leq \NHmDn{v} \leq \NLiD{n} \NHmD{v}.
\eeq
%We put the \emph{non-weighted} $H^s$-norm in the denominator of \cref{eq:negweightnorm}, but the \emph{weighted} inner product in the numerator for ease of manipulation in some of the following proofs---we could place the weighted $H^s$ norm in the denominator, and this would be equivalent to the current definition, up to a factor involving $n$.
%% so that we have, for all $v \in \LtD$
%% \beq\label{eq:neqweightid}
%% \NHmsD{n v} = \NHmsDn{v}.
%% \eeq
%% \beq\label{eq:negativeweightbounds}
%% \frac{\NHmsD{v}}{\NLiD{n}} \leq \NHmsDn{v} \leq \frac{NHmsD{v}}{\nmin}.
%% \eeq

The proofs of the error bounds for the Galerkin projections will use the following notation for the solution operator of a particular stationary diffusion problem. We let $\DeltaAI:\LtD\rightarrow\HtD$ denote the solution operator for the following stationary diffusion equation: given $\ftilde \in \LtD$ find $\utilde \in \HtD$ such that
\beq\label{eq:sdeq}
\grad \cdot \mleft(A\grad \utilde\mright) = -n\ftilde \text{ in } D
\eeq
\beq\label{eq:sddbc}
\trD \utilde = 0
\eeq
\beq\label{eq:sdnbc}
\dn \utilde = 0.
\eeq
%% $A$-weighted Laplacian; that is $\DeltaA w = \grad\cdot\mleft(\grad w\mright),$ and give it domain $\DomainDeltaA = \set{w \in \HtD \st \trD w = 0, \dn w = 0};$
%% hence $\DeltaA:\DomainDeltaA \rightarrow \LtD.$ Observe that, for any $f \in \LtD$ there exists $\wf \in \DomainDeltaA$ such that $\DeltaA \wf = -f.$
For the reason there is a factor $n$ on the right-hand side of \cref{eq:sdeq}, see \cref{rem:why} below. Observe that $\sdsol$ is well-defined by \cref{thm:shift} as $n \ftilde \in \LtD$. Also, observe that $\sdsol^{m}$ is well-defined for any $m \in \NN,$ as $\HtD \subseteq \LtD,$ and so one can place $\sdsol \ftilde$ on the right-hand side. 
For any $\ftilde \in \LtD$ and for any $v \in \HozDD,$ we have, by Green's identity,
\beqs
\int_D \mleft(A \grad \mleft(\DeltaAI\ftilde\mright)\mright)\cdot \grad \vb = \int_D n\ftilde \vb,
\eeqs
i.e.,
\beq\label{eq:deltaagreen}
\IPLtD{A\grad \mleft(\sdsol \ftilde\mright)}{\grad v} = \IPLtDn{\ftilde}{v}.
\eeq




We now state and prove error bounds for the two projections given above. As stated above, the proofs below are all modifications of the standard proof (in, e.g., \cite[Theorem 5.8.3]{BrSc:08}). We can, in essence, use the standard proof because all the projections defined above are Galerkin projections given by coercive and bounded sesquilinear forms on $\HoD$ (for $\Ph$) or $\LtD$ (for $\Qhn$).

\ble[Existence and uniqueness of Galerkin projections]\label{lem:eugp}
For any $w \in \HozDD$ the elliptic projection $\Ph w$ and the $n$-weighted $L^2$ projection $\Qhn w$ exist and are unique.
\ele

\bpf[Proof of \cref{lem:eugp}]
The existence and uniqueness of $\Ph w$ and $\Qhn w$ follows from the Lax--Milgram Theorem (see, e.g., \cite[Theorem 2.7.7]{BrSc:08}) applied in $\Vhp$ (as in, e.g., \cite[Corollary 2.7.13]{BrSc:08}), because $\Ph w$ and $\Qhn w$ are defined by sesquilinear forms that are continuous and coercive on $\HozDD$ (equipped with the weighted $H^1$ norm $\NW{\cdot}$) and $\LtD$ respectively.

Continuity and coercivity are immediate in the case of the $n$-weighted $L^2$ projection $\Qhn.$ For the elliptic projection $\Ph,$ continuity is immediate, and coercivity follows from the Poincar\'e inequality \cref{eq:poincare}, because we work in $\HozDD,$ and so all the functions we consider vanish on $\GD.$
\epf

\ble[Error bounds for elliptic projection]\label{lem:ellprojerr}
Under \cref{ass:highp}, for any integer $m \in [-1,p-1],$ there exists a constant $\Cmmo >0$ such that for all $w \in \HozDD$
\beq\label{eq:ellprojerr}
\NHmmD{w-\Ph w} \leq \Cmmo h^{s+1} \BAHoD{w}{\wh}.
\eeq
\ele
\optodo{Correct language to talk about $n$-weighted}


\ble[Error bounds for elliptic projection in $n$-weighted norms]\label{lem:ellprojerrw}
Under \cref{ass:highp}, for any integer $m \in [-1,p-1],$ there exists a constant $\Cwmm >0$ such that for all $w \in \HozDD$
\beqs
\NHnfn{-m}{D}{w-\Ph w} \leq \Cwmm \errn{m} h^{m+1} \BAHoDn{w}{\wh},
\eeqs
where
\beq\label{eq:errn}
\errn{m} =
\begin{dcases}
\frac{\NHmD{n}}{\nmin^2}\nvar &\tif m \in \mleft(\frac{d}2,p-1\mright]\\
\frac{\NWmiD{n}}{\nmin^2}\nvar &\tif m \in \mleft[1,\frac{d}2\mright]\\
\nvar &\tif m = -1,0.
\end{dcases}
\eeq
%% \ben
%% \item\label[itempart]{it:wep1} If $s \in (d/2,p-1],$ then
%% \beq\label{eq:wep1}%\label{eq:ellprojerr}
%% \NHnfn{-s}{D}{w-\Ph w} \leq \Cwmso \frac{\NHmD{n}}{\nmin^2}\nvar h^{s+1} \BAHoDn{w}{\wh}.
%% \eeq
%% \item\label[itempart]{it:wep2} If $s \in [2,d/2]$ then
%% \beq\label{eq:wep2}
%% \NHnfn{-s}{D}{w-\Ph w} \leq \Cwmso \frac{\NWsiD{n}}{\nmin^2}\nvar h^{s+1} \BAHoDn{w}{\wh}.
%% \eeq
%% \item\label[itempart]{it:wep4} For $s = 0$
%% \beq\label{eq:wep4}
%% \NLtDn{w-\Ph w} \leq \Cwz \nvar h \BAHoDn{w}{\wh}.
%% \eeq
%% \item\label[itempart]{it:wep3}Also, we have
%% \beq\label{eq:wep3}
%% \NHnfn{1}{D}{w-\Ph w} \leq \Cwo\nvar \BAHoDn{w}{\wh}.
%% \eeq
%% \een
\ele
We define
\beq\label{eq:en}
\En\de\max_{m = -1,\ldots,p-1}\set{\errn{m}}.
\eeq
\bpf[Proof of \cref{lem:ellprojerrw}]
For the case $m \geq 1$, let $\phi \in \HmD,$ and observe that by \cref{ass:highp,thm:banachalg} $n\phi\in \HmD$. Let $ \vtilde = \sdsol(\phi)$ and observe $\vtilde \in \Hfn{}{m+2}{D}$ by \cref{thm:shift}. Observe that for all $v \in \HozDD,$ by definition of $\vtilde$ and the $n$-weighted inner product $\IPLtDn{\cdot}{\cdot}$, we have
\beq\label{eq:vtilde}
\IPLtD{A \grad v}{\grad \vtilde} = \IPLtD{nv}{\phi} = \IPLtDn{v}{\phi}
\eeq
(where we multiply the complex conjugate of \cref{eq:sdeq} by $v$, and integrate by parts).
Taking $v = w-\Ph w$ in \cref{eq:vtilde}, we have
\begin{align}
\IPLtDn{v}{\phi} &= \IPLtD{A\grad\mleft(w-\Ph w\mright)}{\grad\mleft(v - \Ih v\mright)} \text{by Galerkin orthogonality for } \Ph\nonumber\\
&\leq \CSZfn{m+2}\NLiDop{A} \NHfn{m+2}{D}{\vtilde} h^{m+1} \SNHoD{w-\Ph w} \text{ by \cref{lem:scottzhang}}\nonumber\\
&\leq \CSZfn{m+2} \CAfn{m} \NLiDop{A} \NHfn{m}{D}{n\phi} h^{m+1}\SNHoD{w-\Ph w}\text{ by \cref{thm:shift}.}\label{eq:ellprojwpart}
\end{align}

For $m > d/2$ by \cref{thm:banachalg,eq:nconv} \cref{eq:ellprojwpart} is bounded above by
\beq\label{eq:wep1pt0}
\CSZfn{m+2} \CAfn{m} \CBanfn{m} \NLiDop{A} \frac{\NHfn{m}{D}{n}}{\nmin}\NHnfn{m}{D}{\phi} h^{m+1}\SNHoD{w-\Ph w}.
\eeq


Therefore we have, by definition of $\NHnfn{-m}{D}{\cdot}$ and $\NHoDn{\cdot}$,
\beq\label{eq:wep1pt1}
\NHnfn{-m}{D}{w-\Ph w} \leq \CSZfn{m+2}\CAfn{m} \CBanfn{m} \NLiDop{A} \frac{\NHmD{n}}{\nmin^2} h^{m+1} \NHoDn{w - \Ph w}.
\eeq
Applying C\'ea's Lemma to $\Ph$ in the $n$-weighted $H^1$ norm, we find
\beq\label{eq:wepcea}
\NHoDn{w-\Ph w} \leq \frac{2\NLiDop{A}}{\min\set{1,1/\CP^2}\Amin}\nvar\BAHoDn{w}{\wh},
\eeq
as $\Ph$ corresponds to a sesquilinear form with continuity constant $\NLiDop{A}/\nmin$ and coercivety constant
\beqs
\mleft(\Amin\min\set{1,1/\CP}\mright)/\mleft(2\NLiD{n}\mright).
\eeqs
Combining \cref{eq:wep1pt1,eq:wepcea}, we obtain \cref{eq:errn} in the case $m > d/2$.

For $s \in [1,d/2],$ \cref{thm:banachalg} yields, instead of \cref{eq:wep1pt0},
\beq\label{eq:wep2pt1}
2\CSZfn{m+2} \CAfn{m} \Cprod{m} \NLiDop{A} \frac{\NWmiD{n}}{\nmin}\NHnfn{m}{D}{\phi} h^{m+1}\SNHoD{w-\Ph w},
\eeq
(since $1 + \NWmiD{n} \leq 2\NWmiD{n},$ as $\NLiD{n} \geq 1$) and by a similar reasoning to that before, we obtain \cref{eq:errn}, in the case $m \in [1,d/2]$.

For the case $m=0$, using \cref{lem:ellprojerr} and then converting to the $n$-weighted $L^2$ norm yields \cref{eq:errn}. For $m=-1$, \cref{eq:wepcea} immediately gives \cref{eq:errn}. 
\epf

\bre[Subtleties regarding $\Ph$ for the IIP]\label{rem:epdef}
If we consider the IIP (i.e., we remove the assumption $\GD \neq \emptyset$ in \cref{ass:highp}), then we can no longer prove \nameCrefs{lem:eugp} \ref{lem:eugp}, \ref{lem:ellprojerr}, or \ref{lem:ellprojerrw}, because the sesquilinear form $\aep$ is no longer coercive on $\HozDD$. (Note $\HozDD = \HoD$ in this case.)

This lack of coercivity stems from the fact that we cannot apply the Poincar\'e inequality \cref{eq:poincare} to functions in $\HoD,$ because such functions are no longer zero on a portion of $\d D$ with non-zero $d-1$-dimensional measure. (Recall \cref{lem:poincare} requires such a property.) One can alternatively view this problem as arising from the fact that the stationary diffusion equation with Neumann boundary conditions does not have a unique solution (as one can simply add a constant to any solution). In the case $\GD=\emptyset,$ the PDE corresponding to the elliptic projection is precisely such a stationary diffusion equation: \cref{eq:sd1,eq:sd3} (\cref{eq:sd2} does not apply, as $\GD = \emptyset$).

The remedy for this lack of coercivity/lack of uniqueness is to change the sesquilinear form $\aep$ from \cref{eq:aepho} to either \cref{eq:aeplower} or \cref{eq:aepused}. This change corresponds to changing the PDE underlying the elliptic projection from \cref{eq:sd1,eq:sd3} to either \cref{eq:sd4,eq:sd6} or \cref{eq:sd7,eq:sd9} respectively. Either of these alternatives for $\aep$ is coercive in the weighted $H^1$ norm, and so the proof of \cref{lem:eugp} goes through as before.

However, if one defines $\ep$ using \cref{eq:aepused}, then one encounters the additional problem that one does not have a shift theorem (analagous to \cref{thm:shift}) for the underlying PDE \cref{eq:sd7,eq:sd9}. Such a shift theorem is used in the proof of \cref{lem:ellprojerr,lem:ellprojerrw} to prove bounds in negative-order norms. Whilst one could rewrite \cref{eq:sd7,eq:sd9} as
\beqs
\Delta w = F  \tin D \tand
\eeqs
\beqs
\dn w = ikw \ton \GI
\eeqs
and then use \cref{thm:shift} to obtain results analagous to \cref{thm:shift}, the constants in the resulting bounds \cref{eq:shift} would be \emph{$k$-dependent}. This $k$-dependence would mean the resulting proof of $\hka{(2p+1)/(wp)}$-data-accuracy would not work. Whilst a shift theorem for \cref{eq:sd7,eq:sd9} that \emph{is} $k$-independent has been proved by Chaumont-Frelet, Nicaise, and Tomezyk \cite[Theorems 3.1, 4.3, and 5.1]{ChNiTo:18}, this is a lowest-order shift theorem (i.e., the analogue of \cref{thm:shift} for $l=0$) and no mention is made in \cref{ChNiTo:18} of an extension to higher order.

Therefore, if one wishes to prove results for higher-order finite elements, the only reasonable course of action when considering the IIP is to use \cref{eq:aeplower} to define $\aep$. In this case one can then repeat the proof of \cref{thm:shift} almost verbatim (as the results from \cite{Mc:00} used in the proof of \cref{thm:shift} also hold for the PDE \cref{eq:sd4,eq:sd6}), and the proofs of \cref{lem:ellprojerr,lem:ellprojerrw} proceed as before. However, observe that this course of action works fr our proof because our proof technique (error splitting) allows us to bound terms arising on the truncation boundary $\GI$ (see \cref{lem:boundarybound} below). However, if one uses a modified Schatz technique, it is not clear to us how one would bound terms arising on the truncation boundary. Therefore, these terms on the tunrcation boundary must be incorporated into the elliptic projection, and so one must use \cref{eq:aepused} to define the elliptic projection. One must then accept that, with current technology, one will be limited to proving bounds for first-order methods, due to the lack of a shift theorem for the underlying PDE, as outlined above. (We note in passing that it was precisely this usage of an elliptic projection given by \cref{eq:aepused} by Chaumont-Frelet and Nicaise in \cite{ChNi:18} that led to the work \cref{ChNiTo:18}.)
\ere

%% We also give the following error bounds for the weighted elliptic projection, and give the proof, even though it is conceptually similar to the proof of \cref{lem:ellprojw}.\{Check the original one is still needed.}

%% \ble[Error bounds for weighted elliptic projection in $n$-weighted norms]\label{lem:wellprojerrw}
%% Let $s \in \ZZ.$ There exists a constant $\Cwmso >0$ such that for all $w \in \HozDD$
%% \ben
%% \item\label[itempart]{it:wwep1} If $s \in (d/2,p-1],$ and $n \in \HmD$ then
%% \beq\label{eq:wwep1}%\label{eq:ellprojerr}
%% \NHnfn{-s}{D}{w-\Phn w} \leq \Cwmso \frac{\NHmD{n}\NLiD{n}^2}{\nmin^3}h^{s+1} \BAHoDn{w}{\wh}.
%% \eeq
%% \item\label[itempart]{it:wwep2} If $s \in [1,d/2]$ and $n \in \CsD$, then
%% \beq\label{eq:wwep2}
%% \NHnfn{-s}{D}{w-\Phn w} \leq \Cwmso \frac{\NWsiD{n}\NLiD{n}^2}{\nmin^3} h^{s+1} \BAHoDn{w}{\wh}.
%% \eeq
%% \item\label[itempart]{it:wwep4} For $s = 0$
%% \beq\label{eq:wwep4}
%% \NLtDn{w-\Phn w} \leq \Cwfn{0,1} \frac{\NLiD{n}^3}{\nmin^3}h \BAHoDn{w}{\wh}
%% \eeq
%% \item\label[itempart]{it:wwep3} For $s = 1$
%% \beq\label{eq:it:wwep3}
%% \NHnfn{1}{D}{w-\Phn w} \leq \Cwfn{1,1}\frac{\NLiD{n}^2}{\nmin^2} \BAHoDn{w}{\wh}.
%% \eeq
%% \een
%% \ele

%% \bpf[Proof of \cref{lem:wellprojerrw}]
%% For \cref{it:wep1,it:wep2}, the proof follows the proof in \cite[Theorem 5.8.3]{BrSc:08}. Let $\phi \in \HmD,$ and observe that by \cref{thm:banachalg} $n\phi\in \HmD$ also. Let $ \vtilde = \sdsol(\phi),$ observe that by \cref{thm:shift} $\vtilde \in \Hfn{}{s+2}{D}$. For all $v \in \HozDD,$ we have
%% \beqs
%% \IPLtD{A \grad \vtilde}{\grad v} = \IPLtD{n\phi}{v} = \IPLtDn{\phi}{v}.
%% \eeqs
%% If we take $v = w-\Phn w,$ then we can compute
%% \begin{align}
%% \IPLtDn{\phi}{v} &= \IPLtD{A\grad\mleft(\vtilde - \vh\mright)}{\grad\mleft(w-\Phn w\mright)} \text{ for } \vh \text{ as in \cref{lem:scottzhang}, by Galerkin orthogonality for } \Phn\nonumber\\
%% &\leq \CSZfn{s+2}\NLiDop{A} \NHfn{s+2}{D}{\vtilde} h^{s+1} \SNHoD{w-\Phn w} \text{ by \cref{lem:scottzhang}}\nonumber\\
%% &\leq \CSZfn{s+2} \CAfn{s} \NLiDop{A} \NHfn{s}{D}{n\phi} h^{s+1}\SNHoD{w-\Phn w}\text{ by \cref{thm:shift}}\label{eq:ellprojwpart}
%% \end{align}

%% For \cref{it:wep1}, $s > d/2,$ and so by \cref{thm:banachalg}, \cref{eq:ellprojwpart} is bounded above by
%% \beq\label{eq:wep1pt0}
%% \CSZfn{s+2} \CAfn{s} \CBanfn{s} \NLiDop{A} \frac{\NHfn{s}{D}{n}}{\nmin}\NHnfn{s}{D}{\phi} h^{s+1}\SNHoD{w-\Phn w}
%% \eeq
%% by \cref{thm:banachalg,eq:nconv}.

%% Therefore we have, by definition of $\NHnfn{-s}{D}{\cdot}$ and $\NHoDn{\cdot}$,
%% \beq\label{eq:wep1pt1}
%% \NHnfn{-s}{D}{w-\Phn w} \leq \CSZfn{s+2}\CAfn{s} \CBanfn{s} \NLiDop{A} \frac{\NHmD{n}}{\nmin^2} h^{s+1} \NHoDn{w - \Phn w}.
%% \eeq
%% Applying C\'ea's Lemma to $\Phn$ in the weighted $H^1$ norm, we find
%% \beq\label{eq:wepcea}
%% \NHoDn{w-\Phn w} \leq \frac{2\NLiDop{A}\NLiD{n}^2}{\min\set{1,1/\CP^2}\Amin\nmin^2}\BAHoDn{w}{\wh},
%% \eeq
%% and then combining \cref{eq:wep1pt1,eq:wepcea}, we obtain \cref{eq:wep1}.

%% For \cref{it:wep2}, the application of \cref{thm:banachalg} yields, instead of \cref{eq:wep1pt0},
%% \beq\label{eq:wep2pt1}
%% 2\CSZfn{s+2} \CAfn{s} \Cprod{s} \NLiDop{A} \frac{\NWsiD{n}}{\nmin}\NHnfn{s}{D}{\phi} h^{s+1}\SNHoD{w-\Phn w},
%% \eeq
%% since $1 + \NWsiD{n} \leq 2\NWsiD{n},$ as $\NLiD{n} \geq 1,$ and by a similar reasoning to that before, we obtain \cref{eq:wep2}.

%% For \cref{it:wep3}, \cref{eq:wepcea} immediately gives the result. For \cref{it:wep4}, performing a standard duality argument in \emph{non-weighted} norms (and then using \cref{eq:it:wep3}, yields \cref{eq:wep4}.
%% \epf


%% The $\LtD$ projection satisfies the following error bound.
%% \ble[Error bounds for $\LtD$ projection]\label{lem:ltdprojerr}
%% Under \cref{ass:highp}, for any integer $m \in [0,p-1],$ for all $w \in \LtD$
%% \beqs
%% \NHmmD{w-\Qh w} \leq \Cmmz h^{m} \BALtD{w}{\wh}.
%% \eeqs
%% \ele

\ble[Error bounds for weighted $\LtD$ projection]\label{lem:wltdprojerr}
Under \cref{ass:highp}, for any integer $m \in [0,p-1],$ for all $w \in \HozDD$
\beq\label{eq:wltdprojerr}
\NHmmDn{w-\Qhn w} \leq \CSZfn{m} \frac{\NLiD{n}}{\nmin} h^{m} \BALtDn{w}{\wh}.
\eeq
\ele

\bpf[Proof of \cref{lem:wltdprojerr}]
Fix $\vtilde \in \HmD$. Trivially $\vtilde$ solves
\beq\label{eq:trivial}
\IPLtDn{v}{\vtilde} = \IPLtDn{v}{\vtilde} \tforall v \in \LtD.
\eeq
Then letting $v=w-\Qhn w$ in \cref{eq:trivial} and using Galerkin orthogonality we have
\beqs%gin{align*}
\IPLtDn{w-\Qhn w}{\vtilde} \leq \NLtDn{w-\Qhn w}\NLtDn{\vtilde-\Ih\vtilde}\leq \CSZfn{m} \frac{\NLiD{n}}{\nmin}\NLtDn{w-\Qhn w} h^m \NHmDn{\vtilde}.
\eeqs%nd{align*}
Taking the supremum over $\vtilde,$ we have
\beqs
\NHmmDn{w-\Qhn w} \leq \CSZfn{m} \frac{\NLiD{n}}{\nmin} h^m \NLtDn{w-\Qhn w},
\eeqs
and hence by C\'ea's Lemma (as the inner product $\IPLtDn{\cdot}{\cdot}$ is clearly bounded and coercive (with continuity and coercivity constants equal to 1) in the weighted $L^2$-norm $\NLtDn{\cdot}$) the result follows.
\epf

\subsection{Discrete Sobolev spaces}\label{sec:discsob}
When we analyse the high-order finite-element method, we need to measure higher-order norms of functions in the finite-element space $\Vhp.$ However, as these functions do not have higher-order weak derivatives, we must first define a notion of higher-order discrete derivatives, and then develop some theory of so-called discrete Sobolev spaces. We follow the presentation in \cite{DuWu:15}, albeit working in the heterogeneous case, and with some changes of notation. The main result of this \lcnamecref{sec:discsob} is \cref{lem:negdiscsum} below giving the relationship between negative-order discrete Sobolev norms and negative-order continuous Sobolev norms.

\bde[Discrete derivative operator]
Define the \defn{$A$-weighted discrete second derivative operator} $\Deltah:\Vhp\rightarrow\Vhp$ by, for $\wh \in \Vhp$,
\beq\label{eq:discderdef}
\IPLtDn{\Deltah \wh}{\vh} = \IPLtD{A \grad \wh}{\grad \vh} \tforall \vh \in \Vhp.
\eeq
\ede

\ble[Discrete derivative operator is well-defined]\label{lem:ddwd}
For any $\wh \in \Vhp,$ $\Deltah \wh$ exists and is unique.
\ele

\bpf[Proof of \cref{lem:ddwd}]
Choose an orthonormal (in the $n$-weighted inner product) basis  $(\phij)_j$ for $\Vhp.$ Write $\Delta\wh = \sum_j \wj \phij$, and take in turn $\vh = \phij$ for each $j;$ then \cref{eq:discderdef} is equivalent to the linear system $\Imat \bw = \bb,$ where $\bb_{j} = \IPLtDn{A \grad \wh}{\grad \phij}.$ The solution of this linear system clearly exists and is unique.
%% We equip $\Vhp$ with the $H^1$-norm. Observe that $\Deltah \wh$ satisfies the variational problem: Find $\vhtilde  \in \Vhp$ such that $\add(\uh,\vh) = \Ldd(\vh)$ for all $\vh \in \Vhp,$ where $\add(\uh,\vh) = \IPLtD{\uh}{\vh}$ and $\Ldd(\vh) = \IPLtD{A \grad \wh}{\grad \vh}.$ Observe that $\Ldd$ is bounded in $\Vhp$, as $\Ldd(\vh) \leq \NLiDop{A} \SNHoD{\wh}\SNHoD{\vh} \leq \NLiDop{A} \SNHoD{\wh}\NHoD{\vh},$ and $\add$ is coercive on $\Vhp,$ as, for $\vh \in \Vhp$, $\add(\vh,\vh) = \NLtD{\vh}^2 \geq \CinvVhp^2 \NHoD{\vh}^2$ by the standard inverse estimate\optodo{Add in?}. Therefore, by the Lax--Milgram Theorem applied in $\Vhp$ (as $\Vhp$ is a finite-dimensional inner product space over a complete field, it is a Hilbert space), $\Deltah \wh$ exists and is unique.
\epf

%% By\optodo{McLean Thm 4.12 - double check and maybe write out in more detail---exactly what operator are we talking about, especially if we want weak derivatives?}, there exists a sequence of eigenfunctions $\phio,\phit,\ldots \in \HoD$ of $\DeltaA$\optodo{What exactly does McLean mean here, if they don't have second-order derivatives?} and corresponding eigenvalues $0 < \lambdao<\lambdat < \cdots \rightarrow\infty$ such that the eigenfunctions form a complete orthonormal system in $\LtD.$\optodo{Maybe define this}

Since $A$ is real and symmetric, it is self-adjoint. Hence it follows that $\Deltah$ is self-adjoint in the $n$-weighted inner product, as
\beqs
\IPLtDn{\Deltah \wh}{\vh} = \IPLtD{A \grad \wh}{\grad \vh} = \IPLtD{\grad \wh}{A\grad \vh} = \overline{\IPLtDn{\Deltah \vh}{\wh}} = \IPLtDn{\wh}{\Deltah \vh}.
\eeqs
Therefore $\Deltah$ is diagonalisable, i.e., there exists a set of eigenfunctions $\phioh,\ldots,\phidimVhph$  with corresponding real eigenvalues\ednote{Both - is it common enough knowledge that a symmetric matrix has real eigenvalues, that I don't need to reference this?} $\lambdaoh, \ldots, \lambdadimVhph$ such that the $\phimh$ form an orthonormal (in the $n$-weighted inner product) basis of $\Vhp$.

\bde[Higher-order discrete derivative operators]\label{def:hodd}
For $\vh \in \Vhp$, if $\vh = \sum_{m=1}^{\dimVhp} \am \phimh,$ then for $j \in \RR$ define
\beqs
\Deltah^j \vh = \sum_{m=1}^{\dimVhp} \lambdamh^j \am \phimh.
\eeqs


\ede
Observe that for $\wh \in \Vhp,$ $\DeltahI\mleft(\Deltah\wh\mright) = \wh,$ i.e., one can think of $\DeltahI$ as being a `discrete solution operator' for the stationary diffusion equation \cref{eq:sdeq,eq:sddbc,eq:sdnbc}. I.e., $\DeltahI$ is a discrete counterpart to $\sdsol.$
%% Similarly, for $v \in \LtD,$ if $v = \sum_{m=1}^\infty \am \phim,$ then for $j \in \RR$ define
%% \beq\label{eq:deltaaseries}
%% \DeltaA^j v = \sum_{m=1}^\infty \lambdam^j \am \phim,
%% \eeq
%% if this series exists in $\LtD.$
%We let $\Domain{\DeltaA^j}$ denote the subset of $\LtD$ on which $\DeltaA^j$ is defined.
%% \bre[Negative powers of $\DeltaA$]
%% Observe that for \emph{every} $j \leq 0$, $\DeltaA^j v$ is defined for \emph{any} $v \in \LtD$ (i.e., $\Domain{\DeltaA^{j}} = \LtD$ for $j \leq 0$). For $\lambdam \geq 1,$ $\lambdam^j < \lambdam$, and only finitely many $\lambdam$ are in the interval $(0,1)$; therefore the series \cref{eq:deltaaseries} can be decomposed as a finite sum (for $\lambdam < 1$) and a convergent series (for $\lambdam \geq 1$).
%% \ere
%% \bre[Consistent Notation]
%% Observe that the notation $\DeltaA^j$ is consistent, i.e., $\DeltaA^0 v = v$, for $j \in \NN,$ $\DeltaA^j$ is equal to the $j$-fold application of $\DeltaA,$ and $\DeltaA^{-1}$ is the inverse of $\DeltaA.$\optodo{Maybe just double-check this is watertight.}\optodo{This needs work - need $A$ to be smooth enough to define proper higher-order derivatives.}
%% \ere
We can use the higher-order derivative operators to define discrete higher-order norms:

\bde[Discrete higher-order norm]
For $\vh \in \Vhp$ and $m \in \RR$, define
\beqs
\Nmhn{\vh} = \NLtDn{\Deltah^{m/2} v}.
\eeqs
\ede

%% \bde[$A$-weighted higher-order continuous norm]
%% For $v \in \LtD$ and $m \in \RR$, if $\DeltaA^{m/2} v$ exists, define
%% \beqs
%% \NmA{v} = \NLtD{\DeltaA^{m/2} v}.
%% \eeqs
%% \ede

%% In order to prove a relationship between the $A$-weighted higher order norm and the standard $H^m$ norms, we first must prove the following \lcnamecref{lem:normrelationshiptech}\optodo{THIS NEEDS PROVING AND I DON'T KNOW HOW.}
%% \ble[Relationship between $\DeltaA^m$ and standard higher-order derivatives]\label{lem:normrelationshiptech}
%% For $m \in \NN,$ $\Domain{\DeltaA^{m/2}} \subseteq \HmD,$ and there exists a constant $\Cma > 0$ such that for all $v \in \Domain{\DeltaA^{m}}$
%% \beqs
%% \NHmD{v} \leq \Cma \NLtD{\DeltaA^{m/2}v}.
%% \eeqs\optodo{The latter bit will probably use something shift-theorem-like, but you need to be careful because powers of the differential operator aren't defined standardly.}
%% \ele


%% \ble[Relationship between $A$-weighted and standard higher-order continuous norms]\label{lem:normrelationship}
%% For all $m \in \NNz,$ for all $v \in \HmmD,$
%% \beqs
%% \NHmmD{v} \geq \frac1{\Cma} \NmmA{v}.
%% \eeqs
%% \ele

%% \bpf[Proof of \cref{lem:normrelationship}]
%% For $v \in \HmmD,$ we have
%% \begin{align*}
%% \NHmmD{v} &= \sup_{w \in \HmD} \frac{\IPLtD{v}{w}}{\NHmD{w}}\\
%% &\geq \frac1{\Cma} \sup_{w \in \Domain{\DeltaA^{m/2}}} \frac{\IPLtD{v}{w}}{\NmA{w}} \text{ by \cref{lem:normrelationshiptech}}\\
%% & = \frac1{\Cma} \sup_{w \in \Domain{\DeltaA^{m/2}}} \frac{\IPLtD{\DeltaA^{-m/2}v}{\DeltaA^{m/2}w}}{\NmA{w}} \text{ by \cref{lem:intoip}}\\
%% &= \frac1{\Cma} \NmmA{v} \text{ the supremum is achieved when } \IPLtD{\DeltaA^{-m/2}v,\DeltaA^{m/2}w} = \NLtD{\DeltaA^{-m/2}v}\NLtD{\DeltaA^{m/2}w}
%% \end{align*}
%% \epf

\bre[Related literature for higher-order discrete norms]
To our knowledge, \cite{DuWu:15} is the only place in the literature where the above construction of higher-order discrete norms appears, although Thom\'{e}e defines these norms for negative integers $m$ in \cite[Equation above Lemma 1]{Th:80}. However, the idea of using a self-adjoint, coercive operator to define a norm can be found in, e.g., \cite[Section 2.1]{LiMa:72}, and the authors also observe \cite[Text at the bottom of page 9]{LiMa:72} that one can use a spectral decomposition to define arbitrary powers of such operators (analagous to \cref{def:hodd}). See \cite[Section 2.1]{BaKuMa:19} for a simpler exposition of defining a (fractional-order) Sobolev norm via an operator.

We observe in passing that \cref{def:hodd} is analagous to the spectral definition of the fractional Laplacian $\mleft(-\Delta\mright)^m$; see the recent review article \cite[Section 2.5.1]{LiPaGuSoGlZhMaCaMeAiKa:18} for an overview of this idea.
\ere

We will use the following \lcnamecref{lem:intoip} to bound the inner product of two discrete functions by their negative- and positive-higher-order discrete norms, or to transfer discrete derivatives from one argument of the inner product to the other.

\ble[Introduction of derivatives into inner product]\label{lem:intoip}
%% For $m \in \RR,$ $v \in \LtD$, and $w \in \LtD\cap\Domain{\DeltaA^{m/2}}$ we have
%% \beqs
%% \IPLtD{\DeltaAmmt v}{\DeltaAmt w} = \IPLtD{v}{w}.
%% \eeqs
%% Similarly, f
For $\vh, \wh \in \Vhp,$ and $m \in \RR$ we have
\beq\label{eq:feipsplit}
\IPLtDn{\vh}{\wh} = \IPLtDn{\Deltahmmt \vh}{\Deltahmt \wh} 
\eeq
and
\beq\label{eq:feiptrans}
\IPLtDn{\Deltah^m \vh}{\vh} = \IPLtDn{\Deltahmt \vh}{\Deltahmt \vh}.
\eeq
\ele
\bpf[Proof of \cref{lem:intoip}]
We only prove \cref{eq:feipsplit}, as the proof of \cref{eq:feiptrans} is analagous. Since $\vh,\wh \in \Vhp,$ there exist sequences $(\aj)_{j =1,\ldots,\dimVhp}$ and $(\bsl)_{l =1,\ldots,\dimVhp}$ such that $\vh = \sum_{j=1}^{\dimVhp} \aj\phij$ and $\wh = \sum_{l=1}^{\dimVhp} \bsl \phil.$ Then we have
\begin{align*}
\IPLtDn{\Deltahmmt \vh}{\Deltahmt \wh} &= \int_D n\mleft(\sum_{j=1}^{\dimVhp}\lambdaj^{-m/2} \aj\phij\mright)\overline{\mleft(\sum_{l=1}^{\dimVhp} \lambdal^{m/2}\bsl \phil\mright)}\\
&= \sum_{j,l=1}^{\dimVhp} \lambdaj^{-m/2} \lambdal^{m/2} \aj \bsl \int_D n \phij \philbar\quad \text{ as the } \lambdaj \text{ are real}\\
& =\sum_{j}^{\dimVhp} \aj \bsj \int_D n\abs{\phij}^2\quad \text{ as the } \phij \text{ are orthonormal}\\
&= \IPLtDn{\vh}{\wh},
\end{align*}
where the final line follows by repeating the above process in reverse without the factors $\lambdaj^{-m/2}$ and $\lambdal^{m/2}$.
\epf
The next \lcnamecref{cor:ipdiscbound} follows from \cref{lem:intoip} and the Cauchy--Schwarz inequality.

\bco[Inner product bounded by discrete norms]\label{cor:ipdiscbound}
If $\vh, \wh \in \Vhp,$ then for all $j \in \RR$
\beqs
\IPLtDn{\vh}{\wh} \leq \Njh{\vh}\Nmjh{\wh}.
\eeqs
\eco

We recall the standard inverse inequality for finite-element functions, so that we can prove an analagous inverse inequality for discrete norms.

\ble[Standard inverse inequality]\label{lem:inverseinequality}
There exists $\CinvVhp > 0$ such that for all $\vh \in \Vhp$
\beqs
\NHoD{\vh} \leq \CinvVhp h^{-1} \NLtD{\vh}.
\eeqs
\ele



\ble[Inverse inequality for discrete norms]\label{lem:inversediscrete}
For all $m \in \RR$, for all $\vh \in \Vhp$
\beqs
\Nmhn{\vh} \leq \Chinv \frac{1}{\nmin} h^{-1} \Nmmohn{\vh}.
\eeqs
\ele

\bpf[Proof of \cref{lem:inversediscrete}]
We only prove the case $m=1,$ as the other cases will follow immediately. We have
\begin{align*}
\Nfn{1,h,n}{\vh}^2 &= \IPLtDn{\Deltahh \vh}{\Deltahh \vh}\\
&= \IPLtDn{\Deltah \vh}{\vh} \text{ by \cref{eq:feiptrans}}\\
&= \IPLtD{A \grad \vh}{\grad \vh} \text{ by definition of } \Deltah\\
&\leq \NLiDop{A} \CinvVhp h^{-2}\frac{1}{\nmin^2} \NLtDn{\vh}^2
\end{align*}
by the standard inverse estimate, and the result follows as $\Nzhn{\cdot} = \NLtDn{\cdot}$.

For $m \neq 1$ we have
\begin{align*}
  \Nmhn{\vh} &= \NLtDn{\Deltah^{\frac{m}2}\vh}\\
  &= \NLtDn{\Deltah^{\half}\mleft(\Deltah^{\frac{m-1}2}\vh\mright)}\\
  &= \Nohn{\Deltah^{\frac{m-1}2}\vh}\\
  &\leq \Chinv \frac{1}{\nmin} h^{-1} \Nzhn{\Deltah^{\frac{m-1}2}\vh}\\
\end{align*}
by the result for $m=1,$ and the result for $m\neq 1$ follows.
\epf


\ble[Relationship between standard and discrete $H^1$ norms]\label{lem:h1contdisc}
Let $\vh \in \Vhp$. Then
\beqs
\SNHoD{\vh} \leq \Amin^{-\half} \Nohn{\vh}.
\eeqs
\ele

\bpf[Proof of \cref{lem:h1contdisc}]
We have, using \cref{eq:feiptrans},
\beqs
\Nohn{\vh}^2 = \IPLtDn{\Deltahh \vh}{\Deltahh \vh} = \IPLtDn{\Deltah \vh}{\vh}= \IPLtD{A \grad \vh}{\grad \vh} \geq \Amin \NLtD{\grad \vh}^2,
\eeqs
and the result follows.
\epf

To prove \cref{lem:negdiscsum} below we require the following \lcnamecref{lem:shiftnegativew} giving the shift theorem in negative weighted norms.

\ble[Shift theorem in negative weighted norms]\label{lem:shiftnegativew}
Let $\ftilde \in \LtD$ a d $m \in [0,p-1]$ be an integer. Under \cref{ass:highp} we have
\beq\label{eq:shiftnegativew}
\NHnfn{-m}{D}{\sdsol\ftilde} \leq \Cshiftfn{m} \shiftn{m} \NHnfn{-(m+2)}{D}{\ftilde},
\eeq
where
\beqs
\shiftn{m} \de 
\begin{dcases}
 \NHmD{n}\nvar &\tif m > \frac{d}2\\
\NWmiD{n}\nvar  &\tif m \in \mleft[1,\frac{d}2\mright]\\
\NLiD{n}\nvar &\tif m = 0\\
\NLiD{n}^2 &\tif m = -1.
\end{dcases}
\eeqs
\ele
We define
\beqs
\Rn = \max_{m = -1,\ldots,p-1}\set{\shiftn{m}}.
\eeqs

\bpf[Proof of \cref{lem:shiftnegativew}]
We first observe that the operator $\sdsolo$ is self-adjoint on $\LtD$. Let $\DeltaA:\HtD\rightarrow\LtD$ denote the stationary diffusion operator $\grad\cdot\mleft(A\grad\cdot\mright)$ with no boundary conditions applied. Then for any $\vo \in \LtD,$
\beq\label{eq:scomp}
\DeltaA \circ \sdsolo \vo = \vo.
\eeq
Moreover, by Green's Theorem, $\DeltaA$ is self-adjoint on the set $\set{v \in \HtD \st v \text{ satisfies \cref{eq:sddbc,eq:sdnbc}}},$ this set contains the image of $\sdsolo.$ Therefore, for any $\vo,\vt \in \LtD,$ we have
\beq\label{eq:solosa}
\IPLtD{\sdsolo \vo}{\vt} =\IPLtD{\sdsolo \vo}{\DeltaA \circ \sdsolo \vt} = \IPLtD{\DeltaA \circ \sdsolo \vo}{\sdsolo \vt} = \IPLtD{\vo}{\sdsolo \vt},
\eeq
using \cref{eq:scomp} and the fact that $\DeltaA$ is self adjoint. I.e., \cref{eq:solosa} shows $\sdsolo$ is self-adjoint on $\LtD.$

Observe that by \cref{thm:banachalg}, if $v \in \HmD$ then $nv \in \HmD,$ and therefore by \cref{thm:shift} $\sdsolo(nv) \in \HmptD.$ With these facts in place we can compute
\begin{align*}
\NHnfn{-m}{D}{\sdsol \ftilde} &= \sup_{v \in \HmD} \frac{\IPLtD{\sdsol \ftilde}{nv}}{\NHmDn{v}}\\
&= \sup_{v \in \HmD} \frac{\IPLtD{\sdsolo\mleft(n \ftilde\mright)}{nv}}{\NHmDn{v}}\\
&= \sup_{v \in \HmD} \frac{\IPLtDn{\ftilde}{\sdsolo\mleft(n v\mright)}}{\NHmDn{v}}\quad\tas \sdsolo \text{ is self-adjoint}\\
&\leq \sup_{v \in \HmD} \frac{\NHnfn{-(m+2)}{D}{\ftilde}\NHnfn{m+2}{D}{\sdsolo\mleft(nv\mright)} }{\NHmDn{v}}\\
&\leq \sup_{v \in \HmD}  \frac{\CAfn{m} \NLiD{n} \NHmD{nv} \NHnfn{-(m+2)}{D}{\ftilde}}{\NHmDn{v}}
%% & = \sup_{v \in \HmD} \frac{\IPLtD{ n\ftilde}{\sdsol(v)}}{\NHmD{v}}\text{ by definition of } \sdsol, \text{ and as } \sdsolo \text{ is self-adjoint}\\
%% &= \sup_{v \in \HmD} \frac{\IPLtDn{\ftilde}{\frac{\sdsol(nv)}n}}{\NHmD{v}}\\
%% &\leq \CBanfn{m+2}\sup_{v \in \HmD} \frac{\NHfn{-(m+2)}{D}{\ftilde}\NHfn{m+2}{D}{\sdsol(nv)}\NHfn{m+2}{D}{\frac1n}}{\NHmD{v}} \text{ by \cref{thm:banachalg}}\\
%% &\leq \CBanfn{m+2}\CAfn{m}\sup_{v \in \HmD} \frac{\NHfn{-(m+2)}{D}{\ftilde}\NHfn{m}{D}{nv}\NHfn{m+2}{D}{\frac1n}}{\NHmD{v}} \text{ by \cref{thm:shift}}\\
\end{align*}
and by applying \cref{thm:banachalg} to the term $\NHfn{m}{D}{nv}$ (or observing that $\NLtD{nv} \leq \NLiD{n}\NLtD{v}$ in the case $m=0$), the result follows, except for $m=-1.$

For $m=-1,$ we have, by the Lax--Milgram Theorem in non-weighted norms,
\beqs
\NHoD{\Sn \ftilde} \leq \NHmoD{n\ftilde}/\Amin.
\eeqs
One can straightforwardly show that
\beqs
\NHnfn{1}{D}{\Sn \ftilde} \leq \NLiD{n} \NHoD{\Sn \ftilde}
\eeqs
and
\beqs
\NHmoD{n\ftilde} \leq \NLiD{n} \NHnfn{-1}{D}{\ftilde},
\eeqs
and so the result follows.
\epf

%% \ble[Shift theorem in negative norms]\label{lem:shiftnegative}
%% For $m \in \NN,$ and $\ftilde \in \LtD,$ we have
%% \beqs
%% \NHmmD{\DeltaAI\ftilde} \leq \CAm \NHmmmtD{\ftilde}.
%% \eeqs
%% \ele

%% \bpf[Proof of \cref{lem:shiftnegative}]
%% Throughout the proof we let $\utilde$ denote $\DeltaAI\ftilde.$ As $\ftilde \in \LtD,$ it follows that $\utilde \in \HtD$ by \cref{thm:shift}, and so in particular $\utilde \in \HmmD.$ We will use the fact that $\sdsol$ is self-adjoint; this follows from the fact that the boundary-value problem \cref{eq:sdeq,eq:sddbc,eq:sdnbc} is self-adjoint\ednote{Both - do I need to show this? It's straightforward.}.% Note to self, have (Rv,w), where R is resolvent, for v in L^2, w in H^m. Have PR = Id (but not necessarily the other way round) so w = PRw. Then (Rv,w) = (Rv,PRw) = (PRv,Rw) (P self-adjoint) = (v,Rw). Denote differential operator (on the correct space) by P.
%% We can then compute
%% \begin{align*}
%% \NHmmD{\utilde} &=  \sup_{0 \neq \vtilde \in \HmD} \frac{\IPLtD{\ftilde}{\DeltaAI\vtilde}}{\NHmD{\vtilde}}\text{ as } \DeltaAI \text{ is self-adjoint}\\
%% &\leq \sup_{0 \neq \vtilde \in \HmD} \frac{\NHmmmtD{\ftilde}\CAm\NHmD{\vtilde}}{\NHmD{\vtilde}} \text{ by \cref{thm:shift}}\\
%% &= \CAm \NHmmmtD{\ftilde}
%% \end{align*}
%% as required. \epf

We can now prove the main result of this \lcnamecref{sec:discsob}.

\ble[Relationship between discrete and continuous negative-order norms]\label{lem:negdiscsum}
Under \cref{ass:highp}, for any integer $j \in [0,p+1],$ there exists a constant $\Csumj > 0$ such that for all $\vh \in \Vhp,$
\beq\label{eq:negdiscsum}
\Nfn{-j,h,n}{\vh} \leq \Csumj\mleft(\En\nvar\mright)^{\floor{\frac{j}2}}\NLiD{n} \sum_{m=0}^j h^{m} \NHnfn{-(j-m)}{D}{\vh}.
\eeq
\ele

\bpf[Proof of \cref{lem:negdiscsum}]
Let $\wh \in \Vhp,$ and define $\zh = \DeltahI \wh$ and $z = \DeltaAI \wh$ (observe $z$ is well-defined as $\Vhp \subseteq \LtD$). Then, for all $\vh \in \Vhp$, we have

\begin{align*}
\IPLtD{A \grad z}{\grad \vh} = \IPLtD{A \grad \mleft(\sdsol \wh\mright)}{\grad \vh} = \IPLtDn{\wh}{\vh},\quad \text{and}\\
\IPLtD{A \grad \zh}{\grad \vh} = \IPLtDn{\Deltah \zh}{\vh} = \IPLtDn{\wh}{\vh},
\end{align*}
where the equalities in the first line follow from the definition of $z$ and \cref{eq:deltaagreen}, and the equalities in the second line follows from \cref{eq:discderdef} and the definition of $\zh.$  Therefore (using the fact that $A$ is symmetric), for all $\vh \in \Vhp,$ $\IPLtD{A\grad z}{\grad \vh} = \IPLtD{A\grad \zh}{\grad \vh},$ i.e., $\zh = \Ph z.$

We now have, for $m \in [-1,p-1]$
\begin{align}
\NHnfn{-m}{D}{\DeltahI \wh} &\leq \NHnfn{-m}{D}{z} + \NHnfn{-m}{D}{z-\zh}\nonumber\\
&\leq \NHnfn{-m}{D}{z} + \Cwfn{-m}  \CAfn{0} \CSZfn{2} \errn{m} \NLiD{n} h^{m+2} \NLtD{\wh}\nonumber\\
&\quad\quad\text{ by \cref{lem:ellprojerrw,lem:scottzhang,thm:shift}, as }\zh = \Ph z\nonumber\\
&= \Cshiftfn{m} \shiftn{m} \NHnfn{-(m+2)}{D}{\wh} + \Cm \errn{m} \frac{\NLiD{n}}{\nmin} h^{m+2} \NLtDn{\wh}\label{eq:sumforrecursion}
\end{align}
by \cref{lem:shiftnegativew}.
From \cref{eq:sumforrecursion}, we can conclude that, for $l \in \NN$ and $\vh \in \Vhp$, writing $\wh = \Deltahmlpo \vh$,
\beq\label{eq:lrecursion}
\NHnfn{-m}{D}{\Deltahml \vh} \leq \Cshiftfn{m}\shiftn{m} \NHnfn{-(m+2)}{D}{\Deltahmlpo \vh} + \Cm \errn{m} \frac{\NLiD{n}}{\nmin} h^{m+2} \NLtDn{\Deltahmlpo \vh}
\eeq
as $\Deltahml = \DeltahI \Deltahmlpo$. We now use \cref{eq:lrecursion} recursively to bound $\Nfn{j,h,n}{\vh}$.

If $j = 2l,$ then one can show inductively using \cref{eq:lrecursion} that for any integer $t \in [0,l]$
\beq\label{eq:evenrecursivesum}
\Nfn{-2l,h,n}{\vh} \leq \mleft(\En\nvar\mright)^t\sum_{m=0}^t \Efn{m,t} h^{2m} \NHDfn{-2(t-m)}{\vh} ,
\eeq
\optodo{Fix creflabelformat for equations in subscripts}
where we define the constants $\Efn{m,t}$ inductively by
\begin{align}
\label{eq:Emt1}\Efn{0,0} &=1,\\
\label{eq:Emt2}\Efn{m,t} &= \Cshiftfn{2(t-1-m)} \Efn{m,t-1} \tfor 0 \leq m \leq t-1, \quad\tand\\
\label{eq:Emt3}\Efn{t,t} &= \sum_{m=0}^{t-1} \Csumrecfn{2(t-1-m)} \Efn{m,t-1}.
\end{align}
To see this recurrence, we prove the inductive step: suppose
\beqs
\Nfn{-2l,h,n}{\vh} \leq \mleft(\En\nvar\mright)^{t-1}\sum_{m=0}^{t-1} \Efn{m,t-1} h^{2m} \NHDfn{-2(t-1-m)}{\vh}.
\eeqs
Then using \cref{eq:lrecursion}, we have
\begin{align*}
\Nfn{-2l,h,n}{\vh} &\leq \mleft(\En\nvar\mright)^{t-1}\sum_{m=0}^{t-1} \Efn{m,t-1} h^{2m} \mleft(\Cshiftfn{2(t-1-m)}\shiftn{2(t-1-m)} \NHnfn{-(2(t-1-m)+2}{D}{\Deltah^{-l+t} \vh}\mright.\\
&\quad\quad\mleft.+ \Csumrecfn{2(t-1-m)} \errn{2(t-1-m)} \nvar h^{2(t-1-m)+2} \NLtDn{\Deltah^{-l+t} \vh}\mright),
\end{align*}
which upon rearranging, and using the fact that $\shiftn{2(t-1-m)} \leq \errn{2(t-1-m)}$ and $\nvar \geq 1,$ yields \cref{eq:evenrecursivesum}, with the recurrence \cref{eq:Emt1,eq:Emt2,eq:Emt3}.

If $j=2l+1,$ then we first reduce $\Nfn{-j,h}{\vh}$ to a point analagous to the even case, and then proceed as before. Let $\wh$ and $\zh$ be as at the beginning of the proof, and let $z$ solve the variational formulation\footnote{We use the variational formulation here, as we will need to bound the $H^1$-norm of $z$ by the $H^{-1}$-norm of $\wh$, which is immediate from the Lax--Milgram theorem.}  of \cref{eq:sdeq,eq:sddbc,eq:sdnbc} with $\ftilde = \wh$. Observe that we still have $\zh = \Ph z,$ and

\beq\label{eq:LMHmo}
\NHfn{1}{D}{z} \leq \frac1{\Amin} \NHfn{-1}{D}{n\wh}
\eeq
by the Lax--Milgram Theorem. Then
\begin{align}
\NLtDn{\Deltah^{-1/2}\wh} &= \NLtDn{\Deltah^{1/2} \zh}\nonumber\\
&= \IPLtDn{\Deltah \zh}{\zh} \text{ by \cref{eq:feiptrans}}\\
&= \IPLtD{A \grad \zh}{\grad \zh}\nonumber\\
&= \NLiDop{A} \NHoD{\Ph z}\nonumber\text{ by the Cauchy--Schwartz inequality and the definition of } \zh\\
&\leq \NLiDop{A}\mleft(\NHoD{z} + \Cprojfn{1}\NHoD{0 - z}\mright) \text{ by \cref{lem:ellprojerr}}\nonumber\\
&\leq \frac{\mleft(1+\Cprojfn{1}\mright)\NLiDop{A}}{\Amin}  \NHmoD{n\wh}\text{ by \cref{eq:LMHmo}}\label{eq:deltahhalf}\\
&\leq \frac{\mleft(1+\Cprojfn{1}\mright)\NLiDop{A}}{\Amin}\NLiD{n}  \NHnfn{-1}{D}{\wh}\text{ as in the proof of \cref{lem:shiftnegativew}}\nonumber
\end{align}
We now return to $\Njhn{\vh}$:
\begin{align*}
  \Nfn{j,h,n}{\vh} &= \NLtDn{\Deltah^{-l-1/2} \vh}\\
  &= \NLtDn{\Deltah^{-1/2} \Deltah^{-l} \vh}\\
&\leq \frac{\mleft(1+\Cprojfn{1}\mright)\NLiDop{A}}{\Amin}\NLiD{n}\NHnfn{-1}{D}{\Deltah^{-l}\vh}
\end{align*}
by \cref{eq:deltahhalf}.

Similarly to \cref{eq:evenrecursivesum}, one can use \cref{eq:lrecursion} recursively to show that, for any integer $t \in [0,l]$
\beq\label{eq:oddrecursive}
\NHnfn{-1}{D}{\Deltah^{-l}\vh} \leq \mleft(\En \nvar\mright)^t\mleft(\Etildefn{0,t} \NHnfn{-(2t+1)}{D}{\Deltah^{-l+t}\vh} + \sum_{m=0}^t \Etildefn{m,t} h^{2m+1}  \NHnfn{-2(t-m)}{D}{\Deltah^{-l+t}\vh}\mright),
\eeq
where  we define the $\Etildefn{m,t}$ inductively for $t \in [0,l]$ by
\begin{align}
\label{eq:Etilde1}\Etildefn{0,0} &= 1,\\
%\label{eq:Etilde2}\Etildefn{1,1} &= \Csumrecfn{1}\\
\label{eq:Etilde3}\Etildefn{0,t} &= \Etildefn{0,t-1}\Cshiftfn{2(t-1)+1)},\\
\label{eq:Etilde4}\Etildefn{m,t} &= \Etildefn{m,t-1}\Cshiftfn{2(t-1-m)}\tfor m = 1,\ldots,t-1,\quad\text{and}\\
\label{eq:Etilde5}\Etildefn{t,t} &= \Csumrecfn{2(t-1)+1} + \sum_{m=0}^{t-1}\Etildefn{m,t-1}\Csumrecfn{2(t-1-m)}.
\end{align}
To show \cref{eq:oddrecursive,eq:Etilde1,eq:Etilde3,eq:Etilde4,eq:Etilde5}, observe that \cref{eq:Etilde1} is immediate. We show \cref{eq:oddrecursive,eq:Etilde3,eq:Etilde4,eq:Etilde5} by induction. Suppose %\cref{eq:oddrecursive} holds for $t-1,$ then
\begin{align*}
  \NHnfn{-1}{D}{\Deltah^{-l}\vh} \leq \mleft(\En \nvar\mright)^{t-1}&\Bigg(\Etildefn{0,t-1} \NHnfn{-(2(t-1)+1)}{D}{\Deltah^{-l+t-1}\vh}\\
& \quad + \sum_{m=0}^{t-1} \Etildefn{m,t-1} h^{2m+1}  \NHnfn{-2(t-1-m)}{D}{\Deltah^{-l+t-1}\vh}\Bigg).
\end{align*}
Then
\begin{align}
  &\NHnfn{m}{D}{\Deltah^{-l}\vh} \leq\nonumber\\
  &\quad\mleft(\En \nvar\mright)^{t-1} \Etildefn{0,t-1}\bigg(\Cshiftfn{2(t-1)+1} \shiftn{2(t-1)+1}\NHnfn{-(2(t-1)+1)}{D}{\Deltah^{-l + t}\vh}\nonumber\\
    &\quad\quad\quad\quad\quad\quad\quad\quad\quad\quad\quad\quad\quad\quad+\Csumrecfn{2(t-1)t+1} \errn{2(t-1)+1}\nvar h^{2t+1} \NLtDn{\Deltah^{-l + t}\vh}\bigg)\nonumber\\
  &\quad+ \mleft(\En \nvar\mright)^{t-1}\sum_{m=1}^{t-1} \Etildefn{m,t-1} h^{2m+1}\Big(\Cshiftfn{2(t-1-m)} \shiftn{2(t-1-m)}\NHnfn{-2(t-1-m)}{D}{\Deltah^{-l + t}\vh}\nonumber\\
  &\quad\quad\quad\quad\quad\quad\quad\quad\quad\quad\quad\quad\quad\quad\quad\quad\quad\quad\quad\quad+ h^{2(t-m)} \Csumrecfn{2(t-1-m)}\errn{2(t-1-m)}\nvar \NLtDn{\Deltah^{-l + t}\vh}\Big)\label{eq:verylongeq}
%% &= \Etildefn{0,t-1} \CAfn{2t-1} \NHfn{-(2(t-1)+1)}{D}{\Deltah^{-l + t}\vh}\\
%% &\quad\quad+ \mleft(\Csumrecfn{2t-1}\sum_{m=1}^{t-1} \Etildefn{m,t-1}\CAfn{2(t-1-m)}\mright) h^{2t+1}\NLtD{\Deltah^{-l + t}\vh}\\
%% &\quad\quad+ \sum_{m=1}^{t-1} \Etildefn{m,t-1} h^{2m+1} \CAfn{2(t-1-m)}\NHfn{-2(t-m)}{D}{\Deltah^{-l + t}\vh},
\end{align}
and rearranging \cref{eq:verylongeq}, and using the facts that $\shiftn{m} \leq \errn{m}$ for all $m$ and $\nvar \geq 1,$ we obtain \cref{eq:oddrecursive} with the constants $\Etildefn{m,t}$ given by \cref{eq:Etilde1,eq:Etilde3,eq:Etilde4,eq:Etilde5}. Therefore, we conclude that if $j = 2l+1$
\begin{align}
\Nfn{-j,h,n}{\vh} \leq& \frac{\mleft(1+\Cprojfn{1}\mright)\NLiDop{A}}{\Amin}\NLiD{n}\mleft(\En \nvar\mright)^{l}\mleft(\Etildefn{0,l} \NHnfn{-(2l+1)}{D}{\vh}\mright.\nonumber\\
&\mleft.\hphantom{\frac{\mleft(1+\Cprojfn{1}\mright)\NLiDop{A}}{\Amin}\NLiD{n}\mleft(\En \nvar\mright)^{l}}
\quad\quad+ \sum_{m=0}^{l} \Etildefn{m,l} h^{2m+1}  \NHnfn{-2(l-m)}{D}{\vh}\mright)\label{eq:oddfinal}
\end{align}
and combining \cref{eq:evenrecursivesum} (with $t=l$) and \cref{eq:oddfinal}, the result follows.%\optodo{Maybe put recursion in, but it's all in notes.}
\epf

\bre[Why is the factor $n$ only on the right-hand side of \cref{eq:sdeq}?]\label{rem:why}
The reason for defining $\Sn$ with a factor $n$ on the right-hand side of \cref{eq:sdeq} but not on the left-hand side is somewhat buried in the proof of \cref{thm:fembound} and its associated lemmas. However, we give an overview of the reason here.

All the bounds in the proofs of \cref{lem:boundarybound,lem:higherbound,lem:ltthetahbound} are in weighted discrete norms, because to prove bounds on the weighted $L^2$ projection $\Qhn$ we work in weighted higher-order discrete norms (as in \cref{lem:wltdprojerr}), rather than in non-weighted norms. Because we only work in weighted norms, we need \cref{lem:intoip} below to hold in the \emph{weighted} inner product. To prove \cref{lem:intoip} in the $n$-weighted inner product, we therefore use the $n$-weighted inner product on the left-hand side of \cref{eq:discderdef}, where the $\Deltah$ operator appears. Because we use the $n$-weighted inner product on the left-hand side of \cref{eq:discderdef}, we must then use the $n$-weighted inner product on the right-hand side of \cref{eq:deltaagreen}. Using this inner product in \cref{eq:deltaagreen} ensures the beginning of the proof of \cref{lem:negdiscsum} works---see the proof for more details.

We do not, however, put a factor $n$ on the left-hand side of \cref{eq:sdeq,eq:deltaagreen}. If we did, when we apply \cref{thm:shift} to $\Sn \ftilde$ (as we do in \cref{lem:shiftnegativew}), the resulting bounds would not be fully explicit in $n$ (because in \cref{thm:shift} we do not know explicitly how the constants depend on the `diffusion coefficient'). Not putting a factor $n$ on the left-hand side of \cref{eq:sdeq} is the reason why there is a \emph{non-weighted} inner product on the right-hand side of \cref{eq:discderdef} (so that the beginning of the proof of \cref{lem:negdiscsum}, as mentioned above, works).
\ere

\subsection{Proof of \cref{thm:fembound}}\label{sec:fembound}

Having established the necessary preliminary results about discrete Sobolev spaces, we are now in a position to prove our main theorem, \cref{thm:fembound}, which we do via a series of lemmas. The proof proceeds via a splitting argument, as discussed in \cref{sec:errorsplit}.

For ease of notation in the following lemmas, we follow the notation of \cite{DuWu:15} and let 
\beqs
\rho \de u - \Ph u, \tand
\eeqs
\beqs
\thetah \de \Ph u - \uh.
\eeqs


The following \lcnamecref{lem:simpleform} shows that $\thetah$ solves a discrete Helmholtz problem with data $k^2 \rho$ (in $D$) and $ik \rho$ (on $\GI$).
\ble[$\thetah$ solves a discrete Helmholtz problem]\label{lem:simpleform}
For any $\vh \in \Vhp,$
\beq\label{eq:thetaform}
a(\thetah,\vh) = k^2\IPLtDn{\Qhn\rho}{\vh} + ik \IPLtGI{\rho}{\vh}.
\eeq
\ele

\bpf[Proof of \cref{lem:simpleform}]
Let $\vh \in \Vhp.$ Then $a(\thetah,\vh) = a(u-\uh,\vh) - a(\rho,\vh) = -a(\rho,\vh)$ by Galerkin orthogonality. By definition of $a,$ we have $-a(\rho,\vh) = -\IPLtD{A\grad\rho}{\vh} + k^2 \IPLtD{n\rho}{\vh} + ik\IPLtGI{\rho}{\vh}.$ By Galerkin orthogonality for $\rho = u - \Ph u$, $\IPLtD{A\grad\rho}{\vh} = 0$, and so by the definition of the $n$-weighted $L^2$ inner product, and the $n$-weighted $L^2$-projection $\Qhn,$ the result follows.
\epf

As the definition of the elliptic projection $\Ph$ (\cref{eq:epdef,eq:aepdef} above) uses a Neumann boundary condition on $\GI$, rather than an impedance boundary condition, (see\optodo{Insert ref or write it} above for this rationale) the right-hand side of \cref{eq:thetaform} includes a term defined on the truncation boundary $\GI.$\optodo{search for impedance boundary and replace with truncation boundary} If the definition of the elliptic projection used an imepdance boundary condition, this term would disappear. The presence of this term means that when we bound $\NLtDn{\thetah}$ and $\Npmohn{\thetah}$ in the proofs of \cref{lem:higherbound,lem:ltthetahbound} below, we will encounter terms involving $\NLtGI{\thetah}$. Therefore, we first prove a bound on $\NLtGI{\thetah}.$

\ble[Bound on $\NLtGI{\thetah}$ by $\Npmoh{\thetah}$]\label{lem:boundarybound}
Under the assumptions of \cref{thm:fembound}, we have
\beq\label{eq:boundarybound}
\NLtGI{\thetah}^2 \leq \Cboundaryo \mleft(\En\nvar\mright)^{2\mleft(\floor{\frac{p-1}2}+1\mright)}\NLiD{n}^4 k^2 h^{2p-1} \Nfn{p-1,h,n}{\thetah}^2 + \Cboundaryt h \NW{\rho}^2,
\eeq
\ele

\bpf[Proof of \cref{lem:boundarybound}]
In \cref{eq:thetaform}, let $\vh = \thetah,$ and take the imaginary part to obtain
\beq\label{eq:boundboundarybelow}
-k \NLtGI{\thetah}^2 \leq \Im k^2 \IPLtDn{\Qhn \rho}{\thetah} + \Re k \IPLtGI{\rho}{\thetah}.
\eeq
Therefore by \cref{cor:ipdiscbound}
\beq
\NLtGI{\thetah}^2 \leq  k \Nfn{1-p,h,n}{\Qhn \rho}\Nfn{p-1,h,n}{\thetah} + \NLtGI{\rho}\NLtGI{\thetah}.\label{eq:thetaboundarypart}
%% \nonumber\\
%% &\leq  k \NLiD{n}^2 \Nfn{1-p,h}{\Qhn \rho}\Nfn{p-1,h}{\thetah} + \NLtGI{\rho}\NLtGI{\thetah}
\eeq
We first bound the negative norm $\Nfn{1-p,h}{\Qhn\rho}$, to do this we use \cref{lem:negdiscsum}. However, as we will apply \cref{lem:negdiscsum} we need to estimate negative (standard) Sobolev norms of $\Qhn\rho$; for integers $m \in [0,p-1]$ we have (observing that $\Qhn \Ph u = \Ph u$ as $\Ph u \in \Vhp$).
\begin{align}
\NHnfn{-(p-1-m)}{D}{\Qhn\rho} &\leq \NHnfn{-(p-1-m)}{D}{\Qhn u - u} + \NHnfn{-(p-1-m)}{D}{u - \Ph u}\nonumber\\
&\leq \CSZfn{(p-1-m)} \nvar h^{(p-1-m)} \NLtDn{u-\Ph u}\nonumber\\
&\quad+ \Cwfn{-(p-1-m)}\errn{(p-1-m)} h^{p-m} \NHoDn{u - \Ph u}\nonumber\\
&\quad\quad\text{ by \cref{lem:ellprojerrw,lem:wltdprojerr}, taking } \wh = \Ph u \text{ in \cref{lem:ellprojerrw,eq:wltdprojerr},}\nonumber\\
&\leq \mleft(\CSZfn{(p-1-m)} \Cwz + \Cwfn{-(p-1-m)}\mright)\En\nvar\NLiD{n} h^{p-m} \NW{\rho}\label{eq:Qhnrhoneg}
\end{align}
by \cref{lem:ellprojerrw}. By \cref{lem:negdiscsum,eq:Qhnrhoneg} we have
\begin{align}
\Nfn{1-p,h,n}{\Qhn \rho}&\leq \mleft(\En\nvar\mright)^{\floor{\frac{p-1}2}}\NLiD{n} \Csumfn{p-1} \sum_{m=0}^{p-1} h^{m} \NHfn{-(p-1-m)}{D}{\Qhn \rho} \nonumber\\
&\leq \Cmess \mleft(\En\nvar\mright)^{\floor{\frac{p-1}2}+1}\NLiD{n}^2 h^p \NW{\rho}.\label{eq:Qhnrhosum}
\end{align}
To bound $\NLtGI{\rho}$appearing in  the second term on the right-hand side of \cref{eq:thetaboundarypart} we use \cref{thm:multiplicativetrace,lem:ellprojerr}. We take $\wh = \Ph u$ in \cref{eq:ellprojerr} and use the fact that $\NHoD{\cdot} \leq \NW{\cdot}$ to obtain
\beq\label{eq:rhomtbound}
\NLtGI{\rho} \leq \CMT\NHoD{\rho}^{1/2}\NLtD{\rho}^{1/2} \leq \CMT \Cprojfn{0}^{\half} h^\half \NHoD{\rho}\leq \CMT \Cprojfn{0}^{\half} h^{\half} \NW{\rho}.
\eeq
Therefore by \cref{eq:rhomtbound} and Young's inequality, we obtain
\beq\label{eq:rhothetamt}
\NLtGI{\rho}\NLtGI{\thetah} \leq \half \CMT^2 \Cprojfn{0} h \NW{\rho}^2 + \half \NLtGI{\thetah}^2.
\eeq
By combining \cref{eq:thetaboundarypart,eq:Qhnrhosum,eq:rhothetamt} we have
\begin{align}
\NLtGI{\thetah}^2 &\leq k \Cmess \mleft(\En\nvar\mright)^{\floor{\frac{p-1}2}+1}\NLiD{n}^2 h^p \NW{\rho}\Nfn{p-1,h,n}{\thetah}\nonumber\\
&\quad+ \half \CMT^2 \Cprojfn{0} h\NW{\rho}^2 + \half \NLtGI{\thetah}^2.\label{eq:thetahboundnear}
\end{align}
By using Young's inequality on the first term in \cref{eq:thetahboundnear}, and moving the $\NLtGI{\thetah}^2/2$ term onto the left-hand side, we obtain \cref{eq:boundarybound}.
\epf

We can now prove the main two \lcnamecrefs{lem:ltthetahbound} in the proof of \cref{thm:fembound}.

\ble[Bound on higher-order discrete norms of $\thetah$ by $\NLtD{\thetah}$]\label{lem:higherbound}
Under the assumptions of \cref{thm:fembound}, for integer $m \in [1,p-1]$ there exist constants $\Chighmo,$ $\Chighmt > 0$ such that
\begin{align}
\Nfn{m,h,n}{\thetah} &\leq \Chighmo \mleft(\mleft(\En\nvar\mright)^{\half(\floor{\frac{p-1}2}+1)}\NLiD{n}\nmin^{-\frac{p}2}\mright)^mk^m \NLtD{\thetah}\nonumber\\
&\quad\quad+ \Chighmt \nvar^3\NLiD{n}\nmin^{1-m} h^{1-m} \NW{\rho}.\label{eq:chigh}
\end{align}
\ele
\bpf[Proof of \cref{lem:higherbound}]
By inserting the definitions of $a$ and $\Deltah$ into \cref{eq:thetaform} and rearranging, we have for any $\vh \in \Vhp$
\beqs
\IPLtDn{\Deltah \thetah}{\vh} = k^2 \IPLtDn{\thetah}{\vh} + k^2\IPLtDn{\Qhn \rho}{\vh} + ik \IPLtGI{\thetah}{\vh} + ik \IPLtGI{\rho}{\vh}.
\eeqs
Therefore, if we take $\vh = \Deltah^{m-1}\thetah$, by \cref{lem:intoip} we have
\beq\label{eq:deltahm}
\Nfn{m,h,n}{\thetah}^2 = k^2 \Nfn{m-1,h,n}{\thetah}^2 + k^2 \IPLtDn{\Deltah^{\frac{m-1}2} \Qhn \rho}{\Deltah^{\frac{m-1}2}\thetah} + ik\IPLtGI{\thetah}{\Deltah^{m-1} \thetah} + ik \IPLtGI{\rho}{\Deltah^{m-1} \thetah}.
\eeq
We now proceed to bound the two terms in \cref{eq:deltahm} defined on the truncation boundary $\GI.$ For the first term, we have
\begin{align}
\IPLtGI{\thetah}{\Deltah^{m-1}\thetah} &\leq \NLtGI{\thetah}\NLtGI{\Deltah^{m-1}\thetah}\nonumber\\
&\leq \CMT \CinvVhp^{1/2} \NLtGI{\thetah} h^{-\half} \NLtD{\Deltah^{m-1} \thetah}\nonumber\\
&\quad\quad\text{ by \cref{thm:multiplicativetrace,lem:inverseinequality}}\nonumber\\
&= \CMT \CinvVhp^{1/2} h^{-\half}\NLtGI{\thetah}\nmin^{-1}\Nfn{2m-2,h,n}{\thetah}\text{ by the definition of }\Nfn{2m-2,h,n}{\cdot},\nonumber\\
&\leq \CMT \CinvVhp^{1/2} \Chinv^{m-1} h^{-m+\half} \nmin^{-m} \NLtGI{\thetah}\Nfn{m-1,h,n}{\thetah}\nonumber\\
&\quad\text{ by \cref{lem:inversediscrete} applied } m-1 \text{ times,}\label{eq:useitagain}\\
&\leq \CMT \CinvVhp^{1/2} \Chinv^{m-1} h^{-m+\half} \nmin^{-m}\nonumber\\
&\quad\quad\mleft(\Cboundaryo^{\half} \mleft(\En\nvar\mright)^{\floor{\frac{p-1}2}+1}\NLiD{n}^2 k h^{p-\half} \Nfn{p-1,h,n}{\thetah} + \Cboundaryt^{\half} h^{\half} \NW{\rho}\mright)\Nfn{m-1,h,n}{\thetah}\nonumber\\
&\quad\quad\text{ by \cref{lem:boundarybound} and \cref{eq:simple}}\nonumber\\
&\leq \bigg(\CBo \mleft(\En\nvar\mright)^{\floor{\frac{p-1}2}+1}\NLiD{n}^2\nmin^{-p}k \Nfn{m-1,h,n}{\thetah}\nonumber\\
&\quad\quad\quad+ \CBt \nmin^{-m}h^{1-m} \NW{\rho}\bigg)\Nfn{m-1,h,n}{\thetah}\label{eq:firstboundary}
\end{align}
by \cref{lem:inversediscrete} applied $p-m$ times.

To bound the second boundary term in \cref{eq:deltahm}, we have
\beq\label{eq:secondboundarytemp}
\IPLtGI{\rho}{\Deltah^{m-1} \thetah} \leq \CMT \CinvVhp^{1/2} \Chinv^{m-1}\nmin^{-m}h^{\half-m}\NLtGI{\rho}\Nfn{m-1,h}{\thetah}
\eeq
using the same reasoning as we used to obtain \cref{eq:useitagain} above. By \cref{thm:multiplicativetrace,lem:ellprojerr} (with $\wh = \Ph u$) we have
\beq\label{eq:secondboundarytemp2}
\NLtGI{\rho} \leq \CMT \Cprojfn{0}^{\half} h^{\half} \NW{\rho}.
\eeq

Inserting \cref{eq:secondboundarytemp2} into \cref{eq:secondboundarytemp} we obtain
\beq\label{eq:secondboundary}
\IPLtD{\rho}{\Deltah^{m-1}\thetah} \leq \CBth \nmin^{-m}h^{1-m} \NW{\rho} \Nfn{m-1,h}{\thetah}.
\eeq

Therefore, from\renewcommand{\creflastconjunction}{,~} \cref{eq:deltahm,eq:firstboundary,eq:secondboundary},\renewcommand{\creflastconjunction}{\lastconj} and the Cauchy--Schwartz inequality, we have\optodo{Write Young's and Cauchy's ineqlauities somewhere}
\begin{align*}
\Nfn{m,h}{\thetah}^2 &\leq k^2 \Nfn{m-1,h}{\thetah}^2 + k^2 \Nfn{m-1,h,n}{\Qhn \rho}\Nfn{m-1,h,n}{\thetah}\nonumber\\
&\quad\quad+ k\mleft(\CBo \mleft(\En\nvar\mright)^{\floor{\frac{p-1}2}+1}\NLiD{n}^2\nmin^{-p}k \Nfn{m-1,h,n}{\thetah} + \CBt \nmin^{-m}h^{1-m} \NW{\rho}\mright)\Nfn{m-1,h,n}{\thetah}\nonumber\\
&\quad\quad+ k\CBth \nmin^{-m}h^{1-m} \NW{\rho} \Nfn{m-1,h}{\thetah}%\label{eq:thetahighnearlynearly}.
\end{align*}
Therefore using Young's inequality we have
\begin{align*}
\Nfn{m,h}{\thetah}^2 &\leq k^2\mleft(\frac32 + \CBo \mleft(\En\nvar\mright)^{\floor{\frac{p-1}2}+1}\NLiD{n}^2\nmin^{-p} + \half \CBt \nmin^{-m} + \half\CBth \nmin^{-m}\mright)\Nfn{m-1,h,n}{\thetah}^2\\
&\quad\quad+ \frac{k^2}2 \Nfn{m-1,h,n}{\Qhn \rho}^2 +  h^{2(1-m)} \NW{\rho}^2,
\end{align*}
and by \cref{eq:simple}
\begin{align}
\Nfn{m,h}{\thetah} &\leq k\mleft(\sqrt{\frac32} + \CBo^{\half} \mleft(\En\nvar\mright)^{\half(\floor{\frac{p-1}2}+1)}\NLiD{n}\nmin^{-\frac{p}2} + \frac1{\sqrt{2}} \CBt^{\half} \nmin^{-\frac{m}2} + \frac1{\sqrt{2}}\CBth^{\half} \nmin^{-\frac{m}2}\mright)\Nfn{m-1,h,n}{\thetah}\nonumber\\
&\quad\quad+ \frac{k}{\sqrt{2}} \Nfn{m-1,h,n}{\Qhn \rho} +  h^{1-m} \NW{\rho},\label{eq:thetahighnearly}
\end{align}

We now proceed to bound $\Nfn{m-1,h,n}{\Qhn \rho}$: By \cref{lem:inversediscrete} we have
\begin{align}
  \Nfn{m-1,h,n}{\Qhn \rho} &\leq\Chinv^{m-1} \nmin^{1-m}h^{1-m} \NLtDn{\Qhn\rho}\nonumber\\
  &\leq \Chinv^{m-1} \nmin^{1-m}h^{1-m} \mleft(\NLtDn{\Qhn u - u} + \NLtDn{u-\Ph u}\mright)\label{eq:qhnmmo}
\end{align}
as in the proof of \cref{lem:boundarybound} using the fact that $\Qhn \Ph u = \Ph u.$

Using \cref{lem:wltdprojerr} we can bound the first of these terms by the second:
\beq\label{eq:firsttermbound}
\NLtDn{\Qhn u - u} \leq \CSZfn{0} \nvar \NLtDn{u-\Ph u}.
\eeq
We can also bound
\beq\label{eq:secondtermbound}
\NLtDn{u - \Ph u} \leq \Cwfn{0} \nvar^3h \NHnfn{1}{D}{\rho} \leq \Cwfn{0} \nvar^3\NLiD{n} h \NW{\rho}
\eeq
by \cref{lem:ellprojerrw}. Therefore by \cref{eq:qhnmmo,eq:firsttermbound,eq:secondtermbound} we have (as $kh \leq 1$ and $\nvar \geq 1$)
\beq\label{eq:qhnnearly}
k \Nfn{m-1,h}{\Qhn \rho} \leq \Chinv^{m-1}\Cwz \mleft(1+\CSZfn{0}\mright)\nvar^3\NLiD{n}\nmin^{1-m} h^{1-m}\NW{\rho}.
\eeq
Therefore using \cref{eq:thetahighnearly,eq:qhnnearly} (and the fact that $\NLiD{n}, \nmin^{-1} \geq 1,$ and so $\En, \nvar \geq1$)  we obtain

\begin{align}
  \Nfn{m,h}{\thetah} &\leq \CRecofn{m} \mleft(1+\mleft(\En\nvar\mright)^{\half(\floor{\frac{p-1}2}+1)}\NLiD{n}\nmin^{-\frac{p}2}\mright)k \Nfn{m-1,h}{\thetah}\nonumber\\
  &\quad+ \CRect \nvar^3\NLiD{n}\nmin^{1-m} h^{1-m} \NW{\rho}.\label{eq:readytorecurse}
\end{align}
Using \cref{eq:readytorecurse} recursively, and the facts that $\nmin^{-1} \geq 1$ and $hk \leq 1,$ we obtain \cref{eq:chigh}.
\epf

The following \lcnamecref{lem:continuity} is straightforward to prove, and used in the proof of \cref{lem:ltthetahbound} below
\ble[Continuity of $a$]\label{lem:continuity}
For any $\vo, \vt \in \HozDD,$
\beqs
\abs{a(\vo,\vt)} \leq \Cc \NLiD{n} \NW{\vo}\NW{\vt}.
\eeqs
\ele

\ble[Bound on $\NLtDn{\thetah}$ by $\Nfn{p-1,h,n}{\thetah}$]\label{lem:ltthetahbound}
Under the assumptions of \cref{thm:fembound},%there exist constants $\Cfirst, \Csecond > 0$ such that
\begin{align}
\NLtD{\thetah}&\leq \mleft(\Cfirst\NW{\rho} +\Csecond  k^2h^p\Nfn{p-1,h,n}{\thetah}\mright)\mleft(\En \nvar\mright)^{\floor{\frac{p-1}2}+1}\NLiD{n}^2\nonumber\\
&\quad\quad\Pfn{p-2}\mleft(\NLiD{n}\mright)\mleft(\CFEMotilde h + \CFEMttilde \CAnk (hk)^p\mright).\label{eq:ltthetahbound}
\end{align}
\ele

\bpf[Proof of \cref{lem:ltthetahbound}]
The proof initially uses the standard duality technique, but then becomes more complex than standard proofs. This complexity is due to the fact that (i) we are bounding $\thetah$, not the finite-element error $u-\uh$, and (ii) we are bounding $\thetah$ by its higher-order-discrete norms, rather than by its $H^1$-norm as in the Aubin--Nitsche argument.
\optodo{$a$ should be $\aT$ throughout.}
Consider the adjoint variational problem: Find $w \in \HozDD$ such that for all $v \in \HozDD$
\beq\label{eq:adjointtheta}
a(v,w) = \IPLtD{v}{\thetah}.
\eeq
(I.e., $w$ solves the adjoint problem \cref{eq:adjoint} with right-hand side given by $n\thetah$.) Let $\eh \de u -\uh$ be the finite-element error, put $v = \eh$ in \cref{eq:adjointtheta} and take the complex conjugate\footnote{The reason we take the complex conjugate is that to apply Galerkin orthogonality for $w-\Ph w$, the term $w-\Ph w$ must be the first argument of $\aT$. Alternatively, one could define $\widetilde{\Ph}$ to be the analagoue of the elliptic projection but defined in the second argument, show analogies of the error bounds \cref{lem:ellprojerr} and proceed with the proof of this \lcnamecref{lem:ltthetahbound}. However, for simplicity, we instead take the complex ocnjufate of \cref{eq:adjointtheta}.} to obtain
\beqs
\IPLtD{\thetah}{\eh} = \overline{\aT(\eh,w-\Ph w)} = \aep(w-\Ph w,\eh) - k^2 \IPLtDn{w-\Ph w}{\eh} + ik \NLtGI{w - \Ph w}{\eh}.
\eeqs
By Galerkin orthogonality for $w-\Ph w,$ we have (recalling $\eh = \rho + \thetah$)
\begin{align}
  \IPLtD{\thetah}{\eh} &= \aep(w-\Ph w,\rho) - k^2 \IPLtDn{w-\Ph w}{\eh} + ik \NLtGI{w - \Ph w}{\eh}\nonumber\\
&= \overline{a(\rho,w-\Ph w)}  -k^2 \IPLtDn{w-\Ph w}{\thetah} -ik \IPLtGI{w-\Ph w}{\thetah}.\label{eq:doublego}
%&=a(\rho,w-\Ph w) -k^2 \IPLtDn{\thetah}{w-\Ph w} -ik \IPLtGI{\thetah}{w-\Ph w}
\end{align}

Therefore as $\eh = \rho + \thetah$ we can rearrange \cref{eq:doublego} and use Cauchy-Schwartz to obtain
\begin{align}
\NLtD{\thetah}^2 &\leq \Cc\NLiD{n} \NW{\rho} \NW{w- \Ph w} + k^2 \abs{\IPLtDn{\thetah}{w- \Ph w}}\nonumber\\
&+ k \abs{\IPLtGI{\thetah}{w-\Ph w}} + \NLtDn{\rho}\NLtDn{\thetah}\label{eq:boundingLtwo}
\end{align}
By combining \cref{lem:bestapprox,lem:ellprojerr}, we can show (as $w$ satisfies an adjoint Helmholtz problem with right-hand side $\thetah$)
\beq\label{eq:wlt}
\NLtD{w - \Ph w} \leq \Cprojfn{0} \Pfn{p-2}\mleft(\NLiD{n}\mright)\mleft(\CFEMotilde h^2 + \CFEMttilde \CAnk h(hk)^p\mright)\NLtD{\thetah}
\eeq
and
\beq\label{eq:who}
\NHoD{w - \Ph w} \leq 2\Cprojfn{-1} \Pfn{p-2}\mleft(\NLiD{n}\mright)\mleft(\CFEMotilde h + \CFEMttilde \CAnk (hk)^p\mright)\NLtD{\thetah}.
\eeq
We will be able to use \cref{eq:wlt,eq:who} to bound terms involving $w - \Ph w$ in \cref{eq:boundingLtwo}. We first estimate the inner product terms in \cref{eq:boundingLtwo}:
\begin{align}
\abs{\IPLtDn{\thetah}{w- \Ph w}} &= \abs{\IPLtDn{\thetah}{\Qhn w - \Ph w}}\nonumber\\
&\leq \Nfn{p-1,h,n}{\thetah}\Nfn{1-p,h,n}{\Qhn w - \Ph w}\text{ by \cref{lem:intoip}}\nonumber\\
&\leq \Nfn{p-1,h,n}{\thetah} \Csumfn{p-1} \mleft(\En \nvar\mright)^{\floor{\frac{p-1}2}}\NLiD{n}\nonumber\\
&\quad\quad\sum_{m=0}^{p-1} h^{m}\mleft(\NHnfn{-(p-1-m)}{D}{\Qhn w - w}+ \NHnfn{-(p-1-m)}{D}{w - \Ph w}\mright) \text{ by \cref{lem:negdiscsum}}\nonumber\\
&\leq \Nfn{p-1,h,n}{\thetah} \Csumfn{p-1}\mleft(\En \nvar\mright)^{\floor{\frac{p-1}2}}\NLiD{n} h^{p-1}\nonumber\\
&\quad\quad\sum_{m=0}^{p-1} \mleft(\CSZfn{p-1-m}\nvar \NLtDn{w - \Ph w} + \Cwfn{p-1-m} \errn{p-1-m} h \NHoDn{w - \Ph w}\mright)\nonumber\\
&\quad\quad\text{ by \cref{lem:wltdprojerr,lem:ellprojerrw}, taking $\wh = \Ph u$ in \cref{lem:ellprojerrw} and \cref{eq:wltdprojerr},}\nonumber\\
&\leq \Nfn{p-1,h,n}{\thetah} \Csumfn{p-1}\sum_{m=0}^{p-1} \mleft(\CSZfn{p-1-m}\Cprojfn{0} + \Cwfn{p-1-m} \Cprojfn{-1}\mright)                         \mleft(\En \nvar\mright)^{\floor{\frac{p-1}2}+1}\NLiD{n}^2 \nonumber\\
&\quad\quad h^p \Pfn{p-2}\mleft(\NLiD{n}\mright)\mleft(\CFEMotilde h + \CFEMttilde \CAnk (hk)^p\mright) \NLtD{\thetah} \text{ by \cref{eq:wlt,eq:who},}\nonumber\\
&=\Nfn{p-1,h,n}{\thetah}  \Cfourteen \mleft(\En \nvar\mright)^{\floor{\frac{p-1}2}+1}\NLiD{n}^2 h^p \Pfn{p-2}\mleft(\NLiD{n}\mright)\nonumber\\
&\quad\quad\mleft(\CFEMotilde h + \CFEMttilde \CAnk (hk)^p\mright) \NLtD{\thetah} \label{eq:innerprod1}.
\end{align}

We now estimate the other inner product term
\begin{align}
\abs{\IPLtGI{\thetah}{w - \Ph w}} &\leq \CMT\CinvVhp\mleft(\Cboundaryo^{\half} \mleft(\En\nvar\mright)^{\floor{\frac{p-1}2}+1}\NLiD{n} k h^{p-\half} \Nfn{p-1,h,n}{\thetah}\mright.\nonumber\\
&\mleft.\quad\quad+ \Cboundaryt^{\half} h^{\half} \NW{\rho}\mright) h^{-\half}\NLtD{w - \Ph w}\nonumber\\
&\quad\quad\text{ by \cref{eq:simple}, \cref{lem:inverseinequality,lem:boundarybound,thm:multiplicativetrace},}
\nonumber\\
&\leq \CMT \CinvVhp \Cprojfn{0}\mleft(\Cboundaryo^{\half} \mleft(\En\nvar\mright)^{\floor{\frac{p-1}2}+1}\NLiD{n} k h^{p} \Nfn{p-1,h,n}{\thetah} + \Cboundaryt^{\half} h \NW{\rho}\mright)\nonumber\\
&\quad\quad \Pfn{p-2}\mleft(\NLiD{n}\mright)\mleft(\CFEMotilde h + \CFEMttilde \CAnk (hk)^p\mright)\NLtD{\thetah}\label{eq:innerprod2}
\end{align}
 by \cref{eq:wlt}.

We now insert \cref{eq:wlt,eq:who,eq:innerprod1,eq:innerprod2} into \cref{eq:boundingLtwo}:
\begin{align}
\NLtD{\thetah}^2 &\leq \Bigg[\Cc\NLiD{n} \NW{\rho} \mleft(\Cprojfn{0} + 2\Cprojfn{1}\mright)\\
&+k^2\Nfn{p-1,h,n}{\thetah}  \Cfourteen \mleft(\En \nvar\mright)^{\floor{\frac{p-1}2}+1}\NLiD{n}^2 h^p \\
&+k\CMT \CinvVhp \Cprojfn{0} \mleft(\Cboundaryo^{\half} \mleft(\En\nvar\mright)^{\floor{\frac{p-1}2}+1}\NLiD{n} k h^{p} \Nfn{p-1,h,n}{\thetah} + \Cboundaryt^{\half} h \NW{\rho}\mright)\Bigg]\nonumber\\
&\quad\quad \Pfn{p-2}\mleft(\NLiD{n}\mright)\mleft(\CFEMotilde h + \CFEMttilde \CAnk (hk)^p\mright)\NLtD{\thetah}\\
&+\NLtDn{\rho}\NLtDn{\thetah}\nonumber\\
%newline
&\leq \Bigg[\mleft(\Cc\mleft(\Cprojfn{0} + 2\Cprojfn{1}\mright)\NLiD{n} + \CMT \CinvVhp \Cprojfn{0} \Cboundaryt^{\half} + \Cprojfn{0}\nvar  \mright)\NW{\rho} \\
&+\mleft(  \Cfourteen \mleft(\En \nvar\mright)^{\floor{\frac{p-1}2}+1}\NLiD{n}^2 + \CMT \CinvVhp \Cprojfn{0}\Cboundaryo^{\half} \mleft(\En\nvar\mright)^{\floor{\frac{p-1}2}+1}\NLiD{n}\mright)k^2h^p\Nfn{p-1,h,n}{\thetah}\Bigg]\nonumber\\
&\quad\quad \Pfn{p-2}\mleft(\NLiD{n}\mright)\mleft(\CFEMotilde h + \CFEMttilde \CAnk (hk)^p\mright)\NLtD{\thetah}.\label{eq:nearlythere}
\end{align}
Rearranging \cref{eq:nearlythere} and using \cref{lem:ellprojerr} and the fact that $hk \leq 1$ we have
  \begin{align*}
\NLtD{\thetah}^2&\leq \Bigg[\mleft(\Cc\mleft(\Cprojfn{0} + 2\Cprojfn{1}\mright) + \CMT \CinvVhp \Cprojfn{0} \Cboundaryt^{\half} + \Cprojfn{0}  \mright)\NW{\rho} \nonumber\\
&+\mleft(  \Cfourteen  + \CMT \CinvVhp \Cprojfn{0}\Cboundaryo^{\half} \mright)k^2h^p\Nfn{p-1,h,n}{\thetah}\Bigg]\nonumber\\
&\quad\quad \mleft(\En \nvar\mright)^{\floor{\frac{p-1}2}+1}\NLiD{n}^2\Pfn{p-2}\mleft(\NLiD{n}\mright)\mleft(\CFEMotilde h + \CFEMttilde \CAnk (hk)^p\mright)\NLtD{\thetah}.
\end{align*}
  Using Young's inequality to separate out the $\NLtD{\thetah}$ term on the right-hand side and then moving the resulting $\NLtD{\thetah}^2$ term to the left hand side, followed by \cref{eq:simple}, we obtain
\begin{align}
\NLtD{\thetah}&\leq \Bigg[\mleft(\Cc\mleft(\Cprojfn{0} + 2\Cprojfn{1}\mright) + \CMT \CinvVhp \Cprojfn{0} \Cboundaryt^{\half} + \Cprojfn{0}  \mright)\NW{\rho}\nonumber \\
&+\mleft(  \Cfourteen  + \CMT \CinvVhp \Cprojfn{0}\Cboundaryo^{\half} \mright)k^2h^p\Nfn{p-1,h,n}{\thetah}\Bigg]\nonumber\\
&\quad\quad \mleft(\En \nvar\mright)^{\floor{\frac{p-1}2}+1}\NLiD{n}^2\Pfn{p-2}\mleft(\NLiD{n}\mright)\mleft(\CFEMotilde h + \CFEMttilde \CAnk (hk)^p\mright)\label{eq:part3penultimate}.
\end{align}
Upon rearranging \cref{eq:part3penultimate} we obtain \cref{eq:ltthetahbound}.

\epf
With all our lemmas proved, we can now prove our main \lcnamecref{thm:fembound}.

\bpf[Proof of \cref{thm:fembound}]
\label{page:thmfembound}
By \cref{lem:higherbound} (with $m=p-1$) in \cref{eq:ltthetahbound} and using the fact that $\NLiD{n}, \En, \nvar \geq 1$, we have
\begin{align}
\NLtD{\thetah}&\leq \mleft(\Cfirst + \Csecond\Chighfn{p-1,2} \nvar^3\NLiD{n}\nmin^{2-p} \mright)\mleft(\En \nvar\mright)^{\floor{\frac{p-1}2}+1}\NLiD{n}^2\nonumber\\
&\quad\quad\Pfn{p-2}\mleft(\NLiD{n}\mright)\mleft(\CFEMotilde h + \CFEMttilde \CAnk (hk)^p\mright)\NW{\rho}\nonumber\\
&+\Csecond  \Chighfn{p-1,1} \mleft(\mleft(\En\nvar\mright)^{\half(\floor{\frac{p-1}2}+1)}\NLiD{n}\nmin^{-\frac{p}2}\mright)^{p-1}\mleft(\En \nvar\mright)^{\floor{\frac{p-1}2}+1}\NLiD{n}^2\nonumber\\
&\quad\quad\Pfn{p-2}\mleft(\NLiD{n}\mright)\mleft(\CFEMotilde (hk )^{p+1}+ \CFEMttilde \CAnk h^{2p}k^{2p+1}\mright)\NLtD{\thetah}\label{eq:doublehkdep}
\end{align}
Choosing $h$ according to \cref{eq:hfemcond}, \cref{eq:doublehkdep} simplifies to
\begin{align*}
\NLtD{\thetah} &\leq \mleft(\Cfirst + \Csecond\Chighfn{p-1,2} \nvar^4\nmin^{1-p} \mright)\mleft(\En \nvar\mright)^{\floor{\frac{p-1}2}+1}\NLiD{n}^2\\
&\quad\quad\Pfn{p-2}\mleft(\NLiD{n}\mright)\mleft(\CFEMotilde h + \CFEMttilde \CAnk (hk)^p\mright)\NW{\rho} + \half \NLtD{\thetah},
\end{align*}

and therefore it follows that
\beq\label{eq:ltboundwithrho}
\NLtD{\thetah} \leq \nvar^6\nmin^{-(p+1)} \mleft(\En \nvar\mright)^{\floor{\frac{p-1}2}+1} \Pfn{p-2}\mleft(\NLiD{n}\mright)\mleft(\CLtboundo h + \CLtboundt  \CAnk (hk)^p\mright)\NW{\rho}.
\eeq

It only now remains to bound the weighted $H^1$ norm of $\thetah.$ By \cref{lem:h1contdisc,lem:higherbound} (with $m=1$) we have
\begin{align}
\SNHoD{\thetah} &\leq \Amin^{\half} \mleft[\Chighfn{1,1} \mleft(\mleft(\En\nvar\mright)^{\half(\floor{\frac{p-1}2}+1)}\NLiD{n}\nmin^{-\frac{p}2}\mright)k \NLtD{\thetah}\mright.\nonumber\\
&\mleft.\quad\quad+ \Chighmt \nvar^3\NLiD{n} \NW{\rho}\mright],\label{eq:sntheta}
\end{align}

and by combining \cref{eq:sntheta,eq:ltboundwithrho} we obtain (as $hk \leq 1$ and $\nvar,\NLiD{n} \geq 1$)
\begin{align}
\SNHoD{\thetah} &\leq
\mleft(\mleft(\En\nvar\mright)^{\half(\floor{\frac{p-1}2}+1)}\NLiD{n}\nmin^{-\frac{p}2}\mright)\nvar^6\nmin^{-(p+1)} \mleft(\En \nvar\mright)^{\floor{\frac{p-1}2}+1}\Pfn{p-2}\mleft(\NLiD{n}\mright)\nonumber\\
&\quad\quad\mleft(\CHoboundo  +\CHoboundt \CAnk k(hk)^p\mright)\NW{\rho}.\label{eq:hoboundwithrho}
\end{align}
Therefore by combining \cref{eq:ltboundwithrho,eq:hoboundwithrho}, using \cref{lem:ellprojerr} with $m=1$ and \cref{lem:scottzhang} to bound $\NW{\rho}$, and using the fact that $hk \leq 1,$ we obtain \cref{eq:femltbound,eq:femhobound}.
\epf
\optodo{sort table width}



\subsection{Constants from \cref{sec:fem}}\label{app:constants}
To summarise the constants used in \cref{sec:fem}, we use the following table, where the constants are given in logical order (i.e., constants lower in the table will only depend on constants higher in the table). As well as giving the definitions of the constants, we also state the place (Theorem, Lemma, etc.) where they are defined. If a constant is not defined in terms of other constants, but rather is given in the statement of a Theorem or Lemma, then the `Definition' column is left blank.

\begin{longtabu}{ccc}
  \toprule
  Constant & Definition & Defined/Introduced\\
  \midrule
  \endhead
  $\CintAl$ & ---& \cite[Theorem 4.16]{Mc:00}\\
  $\CscatAl$ & ---& \cite[Theorem 4.18(i)]{Mc:00}\\
    $\CtruncAl$ & ---& \cite[Theorem 4.18(ii)]{Mc:00}\\
  $\CAl$ & $\CintAl + \CscatAl + \CtruncAl.$ & \cref{thm:shift}\\
  $\Cmclean{m}$&---&\cite[Theorem 3.20]{Mc:00}\\ 
  $\Cprod{m}$ &---& $1+\Cmclean{m}$\\
  $\CAnk$ &---& \cref{ass:htwo}\\
  $\Cfg$ &---& \cref{ass:htwo}\\
  $\CTrs$ &---& \cref{thm:trace}\\
  $\CBanfn{s}$ & --- & \cref{thm:banachalg}\\
  $\Cej$ & $\mleft(\max\set{1,\CAz} \prod_{l=1}^j \max\set{1,\CAfn{l}} \mleft(1+\CTrfn{l+1}\mright)\mright)2^{j-1}$ ($j \geq 1$); $\max\set{1,\CAz}$ ($j=0$)& \cref{thm:expansion}\\
  $\Pj$ & See \cref{eq:p}& \cref{thm:expansion}\\
  $\Coscfn{j}$ & See \cref{eq:osc1,eq:osc2,eq:osc3} & \cref{thm:expansion}\\
  $\Cosc$ & $\Cefn{p-1}/\max\set{1,\CAz}$ & \cref{thm:expansion}\\
  $\CSZs$ &---& \cref{lem:scottzhang}\\
  $\Cshiftfn{m}$ &$\CAfn{m}\CBanfn{m}$ if $ m \in (d/2,p-1]$& \cref{lem:shiftnegativew}\\
    $\Cshiftfn{m}$ &$2\CAfn{m}\Cprod{m}$ if $m \in [1,d/2]$& \cref{lem:shiftnegativew}\\
      $\Cshiftfn{0}$ &$\CAfn{m}$& \cref{lem:shiftnegativew}\\
            $\Cshiftfn{-1}$ &$1/\Amin$& \cref{lem:shiftnegativew}\\
  $\CFEMo$ & $\sum_{j=0}^{p-2}\Cinterpfn{j+2}\Cefn{j} $ & \cref{lem:bestapprox}\\
  $\CFEMt$ & $\CFEMt = \Cinterpfn{p+1}\Cosc$ & \cref{lem:bestapprox}\\
  $\Cmso$ &$\frac{2\CSZfn{s+2}\CAfn{s}\CBanfn{s}\NLiDop{A}^2}{\min\set{1,\frac1{\CP^2}}\Amin}$ if $s \in (\frac{d}2,p-1]$& \cref{lem:ellprojerr} \\
  $\Cmso$ &$\frac{4\CSZfn{s+2}\CAfn{s}\Cprod{s}\NLiDop{A}^2}{\min\set{1,\frac1{\CP^2}}\Amin}$ if $s \in (1,\frac{d}2]$& \cref{lem:ellprojerr} \\
  $\Cfn{-1,1}$ &$\frac{2\NLiDop{A}}{\min\set{1,1/\CP^2}\Amin}$& \cref{lem:ellprojerr} \\
  $\Cfn{0,1}$ &$\frac{2\NLiDop{A}^2\CSZfn{2}\CAfn{0}}{\min\set{1,1/\CP^2}\Amin}$& \cref{lem:ellprojerr} \\
  $\Cwmso$ &$\CSZfn{m+2}\CAfn{m}\CBanfn{m}\frac{\NLiDop{A}^2}{\Amin}$& \cref{lem:ellprojerr} \\
  $\CinvVhp$ &---& \cref{lem:inverseinequality}\\
  $\Chinv$ & $\CinvVhp \NLiDop{A}^{1/2}$ & \cref{lem:inversediscrete}\\
  $\Cm$ & $\Cwfn{-m,1}  \CAfn{0} \CSZfn{2}$ & Proof of \cref{lem:negdiscsum}\\
  $\Efn{m,t}$ & See \cref{eq:Emt1,eq:Emt2,eq:Emt3} & Proof of \cref{lem:negdiscsum}\\
  $\Etildefn{m,t}$ & See \cref{eq:Etilde1,eq:Etilde3,eq:Etilde4} & Proof of \cref{lem:negdiscsum}\\
  $\Csumj$ &$\max_{m \in \set{0,\ldots,l}}\set{\Efn{m,l},\frac{\mleft(1+\Cfn{1,1}\mright)\NLiDop{A}}{\Amin}\Etildefn{m,l}}$ & Proof of \cref{lem:negdiscsum}\\
  $\Csecond$ &&\\
  $\CRecofn{????}$&&\\
  $\Ctildemin$&&\\
  $\Chcond$ & $\frac1{4^{2p}} \mleft(\Csecond\Chighfn{p-1,1}\mright)^{-\frac1{2p}} \min\set{\CFEMo^{-\frac1{2p}},\CFEMt^{-\frac1{p+1}}}$ & \cref{thm:fembound}\\
  $\CFEMLt$ &$\Cfn{-1,1} \max\set{\CLtboundo,\CLtboundt}$&\\
  $\CFEMHo$ &$\Cfn{-1,1}\max\set{\CHoboundo,\CHoboundt}$&\\
  $\Cmess$ &$\Csumpmo \mleft(\sum_{m=0}^{p-1} \CSZfn{(p-1-m)} \Cwfn{0} + \Cwfn{-(p-1-m)}\mright)$&Proof of \cref{lem:boundarybound}\\
  $\CMT$ &&\\
  $\Cboundaryo$ & $\Cmess^2$ & \cref{lem:boundarybound} \\
  $\Cboundaryt$ & $1 + \CMT^2 \Cfn{0,1}$ & \cref{lem:boundarybound}\\
  $\CBo$&$\CMT \CinvVhp^{1/2} \Chinv^{m-1} \Cboundaryo^{\half} \Chinv^{p-m}$&Proof of \cref{lem:higherbound}\\
  $\CBt$&$\CMT \CinvVhp^{1/2} \Chinv^{m-1} \Cboundaryt^{\half}$&Proof of \cref{lem:higherbound}\\
  $\CBth$&$\CMT^2 \CinvVhp^{\half} \Chinv^{m-1}\Cfn{0,1}^{\half}$&Proof of \cref{lem:higherbound}\\
  $\CRecofn{????}fn{m}$&$\sqrt{\frac32} + \CBo^{\half} + \frac{\CBt^{\half}}{\sqrt{2}}+ \frac{\CBth^{\half}}{\sqrt{2}}$&Proof of \cref{lem:higherbound}\\
  $\CRect$&$\frac1{\sqrt{2}}\Chinv^{m-1}\Cwfn{0}\mleft(1+\CSZfn{0}\mright)+1$&Proof of \cref{lem:higherbound}\\
  $\Chighmo$ & $\prod_{j=1}^m\CRecofn{j}$&Proof of \cref{lem:higherbound}\\
  $\Chighmt$ &$\sum_{j=1}^{m} \mleft(\prod_{l=j+1}^m \CRecofn{l}\mright) \CRect$&Proof of \cref{lem:higherbound}\\
  $\Cfourteen$ &$\Csumfn{p-1}\sum_{m=0}^{p-1} \mleft(\CSZfn{p-1-m}\Cfn{0,1} + \Cwfn{p-1-m} \Cfn{-1,1}\mright)$&Proof of \cref{lem:ltthetabound}\\
  $\Cfirst$ &$\sqrt{2}\max\set{\NLiDop{A},1,\CMT}\mleft(\Cfn{0,1} + 2\Cfn{1,1}\mright) + \CMT \CinvVhp \Cfn{0,1} \Cboundaryt^{\half} + \Cfn{0,1}$&Proof of \cref{lem:ltthetahbound}\\
    $\Csecond$ &$\Cfourteen  + \CMT \CinvVhp \Cfn{0,1}\Cboundaryo^{\half}$&Proof of \cref{lem:ltthetahbound}\\
  $\CLtboundo$&$ 2\mleft(\Cfirst + \Csecond\Chighfn{p-1,1}\mright)\CFEMo$&Proof of \cref{thm:fembound}\\
  $\CLtboundt$&$2\mleft(\Cfirst + \Csecond\Chighfn{p-1,1}\mright)\CFEMt$&Proof of \cref{thm:fembound}\\
  $\CHoboundo$&$\Amin^{\half}\Chighfn{1,1}\CLtboundo+\Chighmt$&Proof of \cref{thm:fembound}\\
  $\CHoboundt$&$\Amin^{\half}\Chighfn{1,1}\CLtboundt$&Proof of \cref{thm:fembound}\\
  $\CFEMLt$&$\mleft(\Cfn{0,1} + \Cfn{-1,1}\mright) \max\set{\CLtboundo,\CLtboundt}$&\Cref{thm:fembound}\\
  $\CFEMHo$&$\mleft(\Cfn{0,1} + \Cfn{-1,1}\mright)\max\set{\CHoboundo+\CLtboundo,\CHoboundt+\CLtboundt}$&\Cref{thm:fembound}\\
  $\CcorLt$&$4\CFEMLt\max\set{\CFEMo.\CFEMt}$&\Cref{cor:fembound}\\
  $\CcorHo$&$8\CFEMHo\max\set{\CFEMo.\CFEMt}$&\Cref{cor:fembound}\\
\bottomrule
\end{longtabu}


%\input{discrete-sobolev-spaces}

%\input{new-fem-bounds}

\chapter{Well-posedness of Formulations of the Stochastic Helmholtz Equation}\label{chap:stochastic}
\section{Introduction}\label{sec:intro}
The goals of this paper are to prove results on the well-posedness of variational formulations of the stochastic Helmholtz equation
\beq\label{eq:hh-intro}
\grad\cdot\mleft(A(\omega)\grad u(\omega)\mright) + k^2n(\omega)u(\omega) = -f(\omega),
\eeq
as well as a priori bounds on its solution that are explicit in the wavenumber $k$ and the material coefficients $A$ and $n.$


We consider \eqref{eq:hh-intro} with physical domain either $\RRd, \, d=2,3,$ or $\RR^d\setminus\Dmclos,$ where $\Dm$ (referred to as the \defn{obstacle}) is a bounded, Lipschitz, open set such that $\RRd\setminus\Dmclos$ is connected, and

\bit
\item $\omega$ is an element of the underlying probability space,
\item $A$ is a symmetric-positive-definite matrix-valued random field such that $\supp(I-A)$ is compact,
\item $n$ is a positive real-valued random field such that $\supp(1-n)$ is compact,
\item $f$ is a real-valued random field such that $\supp f$ is compact, and
  \item $k>0$ is the wavenumber,
  \eit
and we are particularly interested in the case where the wavenumber $k$ is large.

\paragraph{Motivation} The motivation for establishing well-posedness and proving a priori bounds on the solution of \eqref{eq:hh-intro} is the growing interest in Uncertainty Quantification (UQ) for the Helmholtz equation; see e.g.~\cite{XiSh:07,TsXiYi:11,BuGh:14,GaHa:15,FeLiLo:15,FeLiNi:18,LiWaZh:18,HiScScSc:15,BaCaHaZh:18}. (In this PDE context, by `UQ' we mean theory and algorithms for computing statistics of quantities of interest involving PDEs \emph{either} posed on a random domain \emph{or} having random coefficients.) There is a large literature on UQ for the stationary diffusion equation
\beq\label{eq:diffusion}
-\grad\cdot (\kappa(\omega) \grad u(\omega))=f(\omega),
\eeq
due in part to its large number of applications (e.g.~in modelling groundwater flow), and a priori bounds on the solution are vital for the rigorous analysis of UQ algorithms; see e.g.~\cite{BaTeZo:04,BaNoTe:07,Gi:10,MuSt:11,ChScTe:13}. In contrast, whilst \eqref{eq:hh-intro} has many applications (e.g.~in geophysics and electromagnetics), there is much less rigorous theory of UQ for the Helmholtz equation. The main reason for this is that the (deterministic) PDE theory of \eqref{eq:hh-intro} when $k$ is large is much more complicated that the analogous theory for \eqref{eq:diffusion}.
 
 \paragraph{Related previous work} To our knowledge, the only work that considers \eqref{eq:hh-intro} with large $k$ and attempts to establish either (i) well-posedness of variational formulations or (ii) a priori bounds is \cite{FeLiLo:15}, which considers both (i) and (ii) for \eqref{eq:hh-intro} posed in a bounded domain with an impedance boundary condition. We discuss the results of \cite{FeLiLo:15} further in \cref{sec:otherwork}, but we highlight here that (a) \cite{FeLiLo:15} considers $A=I$ and $n=1+\eta,$ with $\eta$ random and the magnitude of $\eta$ decreasing with $k,$  whereas we consider classes of $A$ and $n$ that allow $k$-independent random perturbations, and (b) in its well-posedness result, \cite{FeLiLo:15}  invokes Fredholm theory to conclude existence of a solution, but this relies on an incorrect assumption about compact inclusion of Bochner spaces---see \cref{sec:federico} below. In \cref{sec:otherwork} we also discuss the papers \cite{BuGh:14,JeSc:16,JeScZe:17,HiScScSc:15} on the theory of UQ for either \eqref{eq:hh-intro} or the related time-harmonic Maxwell's equations; in these papers either the $k$-explicit well-posedness is not a primary concern or $k$ is assumed to be small. Our hope is that the results in the present paper can be used in the rigorous theory of UQ for Helmholtz problems with large $k.$
 
\paragraph{The contributions of this paper} The main results in this paper, \cref{thm:hh-gen,thm:hh-hetero} below, concern well-posedness and a priori bounds for the solutions of various formulations of the stochastic Helmholtz equation; these formulations include those used in sampling-based UQ algorithms (\cref{prob:msedp,prob:somsedp} below) and in the stochastic Galerkin method (\cref{prob:svsedp} below). These are the first such results for arbitrarily large $k$ and for $A$ and $n$ varying independently of $k$. These results are proved by combining:
\ben
\item bounds for the Helmholtz equation in \cite{GrPeSp:19} with $A$ and $n$ deterministic but spatially-varying, with
\item general arguments (i.e.~not specific to Helmholtz) presented here for proving a priori bounds and well-posedness of variational formulations of linear elliptic SPDEs.
\een
Regarding 1: the $k$-dependence of the bounds on $u$ in terms of $f$ depends crucially on whether or not $A$, $n$, and $\Dm$ are such that there exist trapped rays. In the trapping case, the solution operator can grow exponentially in $k$ (see \cite{Ra:71,Bu:98,PoVo:99a,CaPo:02,Be:03} and \cite[Section 2.5]{BeChGrLaLi:11}, and the reviews in \cite[Section 6]{MoSp:19}, \cite[Section 1.1]{ChSpGiSm:17}, and \cite[Section 1]{GrPeSp:19}); in contrast, in the nontrapping case, the solution operator is bounded uniformly in $k$ (see \cite{Va:75,MeSj:78,Bu:02}). The bounds in \cite{GrPeSp:19} are under conditions on $A,n,$ and $\Dm$ that ensure nontrapping of rays; the significance of these bounds is that they are the first (deterministic) bounds for the Helmholtz scattering problem in which both $A$ and $n$ vary and the bounds are explicit in $A$ and $n$ (as well as in $k$). This feature of being explicit in $A$ and $n$ is crucial in allowing us to prove the results in this paper when $A$ and $n$ are random fields.

Regarding 2: the main reason these general arguments are needed is the fact that the variational formulations of both the deterministic and the stochastic Helmholtz equation are not coercive, and so one cannot use the Lax--Milgram theorem to conclude well-posedness and an a priori bound.  In the deterministic case, the remedy for the lack of coercivity of the Helmholtz equation is to use Fredholm theory, but this is \emph{not} applicable to the stochastic variational formulation of the Helmholtz equation because the necessary compactness results do not hold in Bochner spaces (see \cref{sec:federico} below). Our solution to this lack of coercivity and failure of Fredholm theory is to use well-posedness results and bounds from the deterministic case to prove results for the stochastic case. We work `pathwise' by integrating the deterministic results over probability space, identifying conditions under which the necessary quantities are indeed integrable. Our approach is given in a general framework that, given (i) deterministic well-posedness results and a priori bounds that are explicit in all the coefficients, and (ii) measurability and integrability conditions on the stochastic quantities, returns corresponding well-posedness results, a priori bounds, and equivalence results for different formulations of the stochastic problem. One reason we state our well-posedness results in general (i.e.~not only in the specific case of the Helmholtz equation) is that we expect that they can be used in the future to prove well-posedness results for the time-harmonic Maxwell's equations in random media. A nontechnical summary of the ideas behind our general well-posedness results is given in \cref{rem:nontechnical} below. Some of these results are similar in spirit to the results about the PDE \eqref{eq:diffusion} in \cite{Gi:10,MuSt:11} (which deal with the failure of Lax--Milgram for the stochastic variational problem for \eqref{eq:diffusion} in the case when the coefficient $\kappa$ is not uniformly bounded above and below), and our general arguments use some of the ideas and technical tools from these two papers. 
 

\subsection{Statement of main results}\label{sec:hh-results}

\paragraph{Notation and basic definitions} We now give the setting for our stochastic Helmholtz problems. Whilst this setting is similar to the deterministic setting in \cref{sec:detprobs} above, one key difference is that we specify the ball $\BR$ which contains the inhomogeneities in $A$, $n$, and $f$. In contrast, in \cref{sec:detprobs} we simply assume $I-A,$ $1-n$, and $f$ have compact support. Let either (i) $\Dm \subset \RRd,$ $d=2,3,$ be a bounded Lipschitz open set such that $\bzero \in \Dm$ and the open complement $\Dp\de \RR^d\setminus \overline{\Dm}$ is connected, or (ii) $\Dm = \emptyset.$ Let $\GD = \partial \Dm.$ 
Fix $R>0$ and let $\BR$ be the ball of radius $R$ centred at the origin. Define $\GR := \partial \BR$ and $\DR \de \Dp \cap \BR$ (see \cref{fig:domain}). Let $\gamma$ denote the trace operator from $\DR$ to $\partial \DR = \GD \cup \GR$ and define $\HozDDR \de \set{v \in \HoDR \st \gamma v = 0 \ton \GD}.$ 
 
Let $\TrR: H^{1/2}(\Gamma_R) \rightarrow H^{-1/2}(\Gamma_R)$ be the Dirichlet-to-Neumann map for the deterministic equation $\Delta u+k^2 u=0$ posed in the exterior of $\BR$ with the Sommerfeld radiation condition 
\beq\label{eq:src}
\frac{\partial u}{\partial r}(\bx) - \ri ku(\bx) = o\mleft(\frac1{r^{(d-1)/2}}\mright) \text{ as } r\de\abs{\bx}\rightarrow \infty, \text{ uniformly in } \frac{\bx}{\abs{\bx}};
\eeq
see \cite[Section 2.6.3]{Ne:01} and \cite[Equations 3.5 and 3.6]{ChMo:08} for an explicit expression for $\TrR$  in terms of Hankel functions and Fourier series ($d=2$)/spherical harmonics ($d=3$). Let $\IPGR{\cdot}{\cdot}$ be the duality pairing on $\GR$ between $\HmhGR$ and $\HhGR$ and write $\dd\Leb$ for Lebesgue measure. Throughout this \lcnamecref{chap:stochastic} we explicitly include the measure when writing integrals, to help distinguish between integrals over the spatial domain (using $\dd\lambda$) and integrals over the probability space (using $\dd\PP).$

%Let $\LiDpRRdtd$ be the set of all matrix-valued functions $A:\Dp\rightarrow\RRdtd$ such that $A_{i,j} \in \LiDpRR$ for all $i,j = 1,\ldots,d.$ Where the range of functions is $\CC$ we suppress the second argument in a function space, e.g.~we write $\LiDp$ for $\LiDpCC.$
For $\Az \in \RRdtd,$ we write $\NopCCd{\Az}$ for the operator norm induced by the Euclidean vector norm on $\CCd$, and for $A:\DR \rightarrow \RRdtd,$ we write $\NLiDRop{A}$ for the norm $\NLiDRRR{\Nop{A(\bx)}}.$ Observe that, by the equivalence of all norms on the finite-dimensional space $\RRdtd,$ $\NLiDRop{\cdot}$ is equivalent to the more standard norm
\beqs
\NLiDRRRdtd{A} \de \sup_{i,j = 1,\ldots,d} \NLiDRRR{{A}_{i,j}}.
\eeqs
We define $\NWoiDRRRdtd{\cdot}$ by
\beqs
\NWoiDRRRdtd{A} \de \sup_{i,j = 1,\ldots,d} \NWoiDRRR{{A}_{i,j}}.
\eeqs

We write $\Do \compcont \Dt$ if $\Do$ is a compact subset of the open set $\Dt.$ Let $\OFP$ be a complete probability space. Throughout this \lcnamecref{chap:stochastic}, unless stated otherwise we equip a topological space with its Borel $\sigma$-algebra. See \cref{app:mtBs} for a summary of the measure-theoretic concepts used in this \lcnamecref{chap:stochastic}. Let
\bit
\item $f:\Omega\rightarrow\LtDp$ be such that $\supp f(\omega) \compcont \BR$ almost surely,
\item $n:\Omega\rightarrow \LiDpRR$ be such that $\supp(1-n) \compcont \BR$ almost surely and there exist $\nmin,$ $\nmax:\Omega\rightarrow \RR$ such that
%\beqs
$0 < \nmin(\omega) \leq n(\omega)(\bx) \leq \nmax(\omega)$
%\eeqs
for almost every $\bx \in \Dp$ almost surely, and
\item $A:\Omega\rightarrow\LiDpspd$ be such that $\supp(I-A) \compcont \BR$ and there exist $\Amin,\Amax:\Omega \rightarrow \RR$ such that $0 < \Amin(\omega) < \Amax(\omega)$ almost surely and
%\beqs
$\Amin(\omega)\abs{\bxi}^2 \leq \big(A(\omega)(\bx)\bxi\big)\cdot\bxi \leq \Amax(\omega)\abs{\bxi}^2$
%\eeqs
for almost every $\bx \in \Dp$ and for all $\bxi \in \CCd$ almost surely.
\eit
If $v:\Omega \rightarrow Z$ for some function space $Z$ of functions on $\RRd,$ we abuse notation slightly and write $v(\omega,\bx)$ instead of $v(\omega)(\bx).$


%\def\domainscale{2}

\begin{tikzpicture}[even odd rule]
  % \draw (-2,2) .. controls (-1,3) and (-2,3) .. (-1,2.5);
  % \draw (-2,2) .. controls (-1,4) and (-2,3.5) .. (-1,2.5);
  % Sketch
  %\draw[color=gray] (-2,2) -- (-1,2.5) -- (0,2.5) -- (2,3) -- (3,2.5) -- (4,1.5) -- (4,-1.5) -- (3,-2.5) -- (2,-3) -- (0.5,-3) -- (-3,-1) -- (-4,0.5) -- cycle;

  % \draw[color=gray]
    % (4.5,0) --
    % (4.25,-2) --
    % (3,-3) --
    % (2.25,-3.5) --
    % (1.5,-3) --
    % (0.5,-2.5) --
    % (-0.75,-3) --
    % (-2,-3.5) --
    % (-3,-3) --
    % (-3.75,-1.5) --
    % (-5.25,-1) --
    % (-6,-0.75) --
    % (-6,0.25) --
    % (-4.5,1.5) --
    % (-3.5,2.25) --
    % (-3,3.5) --
    % (0,3) --
    % (0.75,3) --
    % (1.5,3.5) --
    % (3,3.5) --
    % (3.75,3) --
    % (4.25,1.75) --
    % cycle;
    
    % This will hopefully be some lines
    \pgfdeclarepatternformonly{owennortheast}
    {\pgfpointorigin}{\pgfpoint{1cm}{1cm}}
    {\pgfpoint{1cm}{1cm}}
    {
    \pgfpathmoveto{\pgfpointorigin}
    \pgfpathlineto{\pgfpoint{1cm}{1cm}}
    \pgfsetlinewidth{0.5\pgflinewidth}
    \pgfusepath{stroke}
    }
    

     \pgfdeclarepatternformonly{owennorthwest}
    {\pgfpointorigin}{\pgfpoint{1cm}{1cm}}
    {\pgfpoint{1cm}{1cm}}
    {
    \pgfpathmoveto{\pgfpoint{1cm}{0cm}}
    \pgfpathlineto{\pgfpoint{0cm}{1cm}}
   \pgfsetlinewidth{0.5\pgflinewidth}
    \pgfusepath{stroke}
    }

    \filldraw[pattern=owennorthwest,scale=\domainscale] % Truncation
    (3.5, 0.0) --
    (1.081559480312316, 3.3286978070330373) .. controls (-0.581559480312316, 2.3) ..
    (-2.8315594803123156, 2.0572483830236563) .. controls (-3.8315594803123156, 0.0572483830236563) and (-3.8315594803123165, -0.0572483830236554) ..
    (-2.8315594803123165, -2.0572483830236554) .. controls (-2.3315594803123165, -3.0572483830236554) and (-0.5815594803123152, -3.4286978070330377) ..
    (1.0815594803123152, -3.3286978070330377) --
    %% (3.0, 0.0) --
    %% (0.9270509831248424, 2.8531695488854605) --
    %% (-2.427050983124842, 1.7633557568774196) --
    %% (-2.427050983124843, -1.7633557568774192) --
    %% (0.9270509831248417, -2.853169548885461) --
    %% (4.0, 0.0) --
    %% (1.2360679774997898, 3.804226065180614) --
    %% (-3.2360679774997894, 2.351141009169893) --
    %% (-3.2360679774997902, -2.351141009169892) --
    %% (1.236067977499789, -3.8042260651806146) --
    cycle;

    %% \filldraw[pattern=owennorthwest] % Truncation

    %% cycle;
    
   %% [scale=2/3] (4.5,0) .. controls (4.5,-0.5) and (4.25,-2) .. %A
   %%  (3,-3) .. controls (2.875,-3.1) and (2.25,-3.5) ..%B
   %%  (1.5,-3) .. controls (0.75,-2.5) and (0.5,-2.5) ..%C
   %%  (-0.75,-3) .. controls (-2,-3.5) and (-2.8,-3.4)..%D
   %%  (-3,-3) .. controls (-3.75,-1.5) and (-4.95,-1.1) ..%E
   %%  (-5.25,-1) .. controls (-6,-0.75) and (-6,0.25) ..%F
   %%  (-4.5,1.5) .. controls (-4.35,1.625) and (-3.5,2.25) ..%G
   %%  (-3,3.5) .. controls (0,3) and (0.6,2.9) ..%H
   %%  (1.5,3.5) .. controls (2.25,4) and (3.5,3.5) ..%I
   %%  (3.75,3) .. controls (4,2.5) and (4.5,0.5)..%J
    %%  cycle;
    

    \filldraw[pattern=owennortheast,scale=\domainscale] % The obstacle
    (1.0, 0.0) .. controls (1.0,0.3) and (1.0071067811865476, 0.7071067811865475) ..
    (0.7071067811865476, 0.7071067811865475) .. controls (0.4071067811865476, 0.7071067811865475) and (0.2, 0.8) ..
    (0.0, 1.0) .. controls (-0.2,1.2) and (-0.5071067811865475, 0.9071067811865476) ..
    (-0.7071067811865475, 0.7071067811865476) .. controls (-0.9071067811865475, 0.5071067811865476) and (-1.2,0.2) ..
    (-1.0, 0.0) .. controls (-0.8,-0.2) and (-0.7071067811865477, -0.5071067811865475) ..
    (-0.7071067811865477, -0.7071067811865475) .. controls (-0.7071067811865477, -0.9071067811865475) and (-0.2,-0.8) ..
    (0.0, -1.0) .. controls (0.2,-0.6) and (0.4071067811865474, -0.6071067811865477) .. % The corner at the start of this line is the one where there's a 'proper' corner
    (0.7071067811865474, -0.7071067811865477) .. controls (1.0071067811865474, -0.8071067811865477) and (1.0,-0.2) ..
    cycle
    ;

    \draw[scale=\domainscale] (0,0) circle [radius=2.25]; % The ball
    
    %% (4.5,0) -- %controls (4.5,-0.5) and (4.25,-2) .. %A
    %% (3,-3) -- %controls (2.875,-3.1) and (2.25,-3.5) ..%B
    %% (1.5,-3) -- %controls (0.75,-2.5) and (0.5,-2.5) ..%C
    %% (-0.75,-3) -- %controls (-2,-3.5) and (-2.8,-3.4)..%D
    %% (-3,-3) -- %controls (-3.75,-1.5) and (-4.95,-1.1) ..%E
    %% (-5.25,-1) -- %controls (-6,-0.75) and (-6,0.25) ..%F
    %% (-4.5,1.5) -- %controls (-4.35,1.625) and (-3.5,2.25) ..%G
    %% (-3,3.5) -- %controls (0,3) and (0.6,2.9) ..%H
    %% (1.5,3.5) -- %controls (2.25,4) and (3.5,3.5) ..%I
    %% (3.75,3) -- %controls (4,2.5) and (4.5,0.5)..%J
    %% cycle;

% The labels

\draw (0,0) node[fill=white] {$\Dm$};
\draw (-3.5,-3.5) node[fill=white] {$D:=\Dtilde\setminus \clos{\Dm}$};

\draw (3.2,-2) node[fill=white] {$\GD$};

\draw (3.7,7) node {$\GI$};
    
  % \draw (-2,2) .. controls (-1,3) and (-2,1.5) .. (-1,2.5)
  % .. controls (0,1.5) and (-1,1) .. (0,2.5)
  % .. controls (1,3) and (1,4.5) .. (2,3)
  % .. controls (3,1.5) and (2,2) .. (3,2.5)
  % .. controls (4,3) and (4,2.5) .. (4,1.5)
  % .. controls (4,0.5) and (4.5,-0.5) .. (4,-1.5)
  % .. controls (3.5,-2.5) and (4,-3.5) .. (3,-2.5)
  % .. controls \hout and \iin .. (2,-3)
  % .. controls (1,-4) and (1.5,-4) .. (0.5,-3)
  % .. controls (-0.5,-2) and (-2,-1) .. (-3,-1)
  % .. controls (-4,-1) and (-5,-0.25) .. (-4,0.5)
  % .. controls (-3,1.25) and (-3,1) .. cycle;
\end{tikzpicture}


\paragraph{Variational Formulations} We consider three different formulations of the  \emph{Helmholtz stochastic exterior Dirichlet problem} (stochastic EDP); \cref{prob:msedp,prob:somsedp,prob:svsedp} below.

Define the sesquilinear form $a(\omega)$ on $\HozDDR \times \HozDDR$ by
\beq\label{eq:SEDPa}
\mleft[a(\omega)\mright]\mleft(\vo,\vt\mright)\de\int_{D_R}\Big( \mleft( A(\omega) \grad \vo\mright)\cdot \grad \vtb 
 - k^2 n(\omega)\, \vo\,\vtb \Big)\dd\Leb- \big\langle T_R \gamma \vo,\gamma \vt\big\rangle_{\Gamma_R},
 \eeq
 and the antilinear functional $L(\omega)$ on $\HozDDR$ by
\beq\label{eq:SEDPL}
\mleft[L(\omega)\mright](\vt)\de \int_{D_R} f(\omega)\, \vtb\,\dd\Leb.
\eeq
Define the sesquilinear form $\as$ on $L^2\big(\Omega;H_{0,D}^1(D_R)\big)\times L^2\big(\Omega;H_{0,D}^1(D_R)\big)$ and the antilinear functional $\Ls$ on $L^2\big(\Omega;H_{0,D}^1(D_R)\big)$ by 
\beq\label{eq:SEDPas}
\as\mleft(\vo,\vt\mright)\de \int_\Omega \mleft[a(\omega)\mright]\mleft(\vo(\omega),\vt(\omega)\mright)\dd\PP(\omega)
\quad\text{ and } \quad
%\eeq
%
%\beq
%\label{eq:SEDPLs}
\Ls(\vt)\de \int_\Omega \mleft[L(\omega)\mright]\mleft(\vt(\omega)\mright)\dd\PP(\omega).
\eeq
We consider the following three problems:

\bprob[Measurable EDP almost surely]\label{prob:msedp}
Find a measurable $u:\Omega\rightarrow\HozDDR$ such that
\vspace{-2ex}
\beqs
\mleft[a(\omega)\mright]\mleft(u(\omega),v\mright) = \mleft[L(\omega)\mright](v) \tforall v \in \HozDDR \text{ almost surely.}
\eeqs
\eprob

\bprob[Second-order EDP almost surely]\label{prob:somsedp}
Find $u\in L^2\big(\Omega;H_{0,D}^1(D_R)\big)$ such that
\beqs
\mleft[a(\omega)\mright]\mleft(u(\omega),v\mright) = \mleft[L(\omega)\mright](v) \tforall v \in \HozDDR \text{ almost surely.}
\eeqs
\eprob

\bprob[Stochastic variational EDP]\label{prob:svsedp}
Find $u\in L^2\big(\Omega;H_{0,D}^1(D_R)\big)$ such that
\beqs
\as(u,v) = \Ls(v) \tforall v \in \LtOHozDDR.
\eeqs
\eprob

%\bre[The relationship between \cref{prob:msedp,prob:somsedp,prob:svsedp}]\label{rem:lit}

\Cref{prob:somsedp} is the foundation of sampling-based UQ methods, such as Monte-Carlo and Stochastic-Collocation methods; its analogue for the stationary diffusion equation is well-studied in, e.g., \cite{XiHe:05,BaNoTe:07,NoTeWe:08a,Ch:12,ChScTe:13,TeJaWeGu:15,KuNu:16,HeLaSc:18}. Similarly \cref{prob:svsedp} is the foundation of the Stochastic Galerkin method (a finite-element method in $\Omega \times D,$ where $D$ is the spatial domain), and is studied for the Helmholtz Interior Impedance Problem in \cite{FeLiLo:15}, and its analogue for the stationary diffusion equation is considered in, e.g., \cite{BaTeZo:04,KhSc:11,BaScZo:11,GuWeZh:14}.

%The prevalance of \cref{prob:somsedp,prob:svsedp} in the literature therefore raises two questions:
%\ben
%\item\label{q:whyone} Why consider \cref{prob:msedp}?
%\item\label{q:wpe} When are the well-posedness and equivalence of \cref{prob:msedp,prob:somsedp,prob:svsedp} non-trivial to establish?
%\een
%
%We address the first question in the next remark. Regarding the second question, in the stationary diffusion case when the coefficient $\kappa$ in \eqref{eq:diffusion} is uniformly bounded above and below on $\Omega \times D,$ the bilinear form for the analogue of \cref{prob:svsedp} is coercive on $\Omega \times D,$ and so we can obtain well-posedness and an a priori bound on the solution using the Lax--Milgram theorem. As we show in the general framework below, one can then obtain similar results for \cref{prob:somsedp,prob:msedp} under mild regularity and integrability assumptions. However, when the coefficient $\kappa$ is not bounded above and below on $\Omega \times D$ (as in the well-studied case of a lognormal random field), one can only show well-posedness and an a priori bound for the analogue of \cref{prob:somsedp} using a pathwise approach (as in \cite{ChScTe:13}) and for the analogue of \cref{prob:svsedp} by modifiying the functions spaces used, as in \cite{Gi:10,MuSt:11}.
%
%In the Helmholtz case, even when the coefficients $A$ and $n$ are bounded uniformly above and below, the sesquilinear form $\as$ in \cref{prob:svsedp} is not coercive on $\Omega\times \DR$, so one cannot use the Lax--Milgram theorem, and Fredholm theory (which can be used in the deterministic case to conclude existence and uniqueness) is not applicable, because the Bochner space $\LtOHoDR$ is not compactly contained in $\LtOLtDR$---see \cref{sec:federico}. Therefore, currently the only way to obtain well-posedness results and a priori bounds on the solution of \cref{prob:svsedp} is to use the results in this paper. We use a pathwise approach, imposing conditions on $A$ and $n$ such that we can construct a solution to \cref{prob:msedp}, and then showing this solution is also a solution of \cref{prob:somsedp,prob:svsedp} with the necessary properties. 
%\ere

\bre[Why consider \cref{prob:msedp}?]\label{rem:whyone}

The difference between \cref{prob:msedp,prob:somsedp} is that \cref{prob:msedp} requires no integrability of $u$ over $\Omega$, but \cref{prob:somsedp} requires $u \in \LtOHozDDRnormal$. Since all the theory for sampling-based UQ methods assume some integrability of the solution, the natural question is: why consider \cref{prob:msedp} at all?

The main reason we consider \cref{prob:msedp} is that, given the existing PDE theory for the Helmholtz equation, we can prove existence of a solution to \cref{prob:msedp} under general conditions on $A$ and $n$, but there is no current prospect of proving existence of a solution to \cref{prob:somsedp} under general conditions on $A$ and $n$. The explanation for this consists of the following three points:
\ben
\item The only two known ways to obtain a solution to \cref{prob:somsedp} are: (i) obtain a deterministic a priori bound, explicit in all parameters, and integrate (followed, e.g., in \cite{ChScTe:13} for \eqref{eq:diffusion} with lognormal coefficients) and (ii) obtain a solution to \cref{prob:svsedp} and show this is a solution to \cref{prob:somsedp}. In the Helmholtz case, doing (ii) is difficult as neither the Lax--Milgram theorem nor Fredholm theory is applicable (as explained in the introduction), and so we follow the approach in (i).
\item The only known bounds on the solution of the Helmholtz equation explicit in all parameters are those recently obtained for nontrapping scenarios in \cite{GrPeSp:19,GaSpWu:18}.
\item Obtaining a bound explicit in all the parameters for any general class of $A$ and $n$, e.g., $A \in \WoiDRspd$ and $n \in \LiDRRR$ is well beyond current techniques. Indeed, a general class of $A$ and $n$ will include both trapping and nontrapping scenarios, and such a bound would need to capture the exponential blow-up in $k$ for trapping $A$ and $n$, the uniform boundedness in $k$ for nontrapping $A$ and $n$, and be explicit in $A$ and $n$.
\een
Given this fact that there is no current prospect of proving existence of a solution to \cref{prob:somsedp} under general conditions on $A$ and $n$ we keep \cref{prob:msedp} so that we prove an (albeit weaker) existence result for the Helmholtz equation with general coefficients.


%or Fredholm theory are not
%
% We can show that $u \in \LtOHozDDRnormal$ by obtaining an a priori bound in $\HozDDR$ on the solution of the corresponding deterministic problem, where the bound is explicit in all of the stochastic parameters, checking conditions under which this bound can be integrated over $\Omega,$ and then integrating this deterministic bound over $\Omega$ to obtain a bound in $\LtOHozDDRnormal.$ This is precisely the approach followed in \cite{ChScTe:13} to obtain well-posedness results and an a priori bound for the stationary diffusion equation with a lognormal coefficient.
%
%However, in the Helmholtz case, as mentioned in \cref{sec:intro}, establishing an a priori bound on the corresponding deterministic problem that is explicit in all of the parameters of interest (such as $A$ and $n$) is difficult---the only such bounds where $A$ and $n$ both vary are those recently obtained in \cite{GrPeSp:19}, where these bounds are obtained under certain conditions on $A$ and $n$. Therefore, one can only prove that a solution of \cref{prob:somsedp,prob:svsedp} exists for small classes of $A$ and $n$---see \cref{thm:hh-hetero} below for such a result. Hence, given all the existing PDE theory for the Helmholtz equation, one \emph{cannot} prove \emph{existence} results for \cref{prob:somsedp,prob:svsedp} for general coefficients, including so-called trapping coefficients; \cref{prob:msedp} is the \emph{only} problem for which one can prove existence and uniqueness---see \cref{thm:hh-gen} below for this result.
%
%Moreover, \cref{prob:msedp} (or rather, its more general analogue \cref{prob:meas} in \cref{sec:notdef} below) can be used as a stepping stone to proving well-posedness results and a priori bounds for the more general analogues of \cref{prob:somsedp,prob:svsedp} (\cref{prob:lt,prob:svar} below). If one can show that a solution of \cref{prob:meas} exists (e.g., by constructing a solution pathwise), and that \cref{con:coeffstoform,con:A,con:coeffstofunc,con:L,con:cborel,con:C,con:B,con:K} in \cref{sec:cons} below hold, then one can show well-posedness and equivalence of all of \cref{prob:meas,prob:lt,prob:svar} and prove an a priori bound on the solution. See \cref{fig:ladder} for a summary of these results.
\ere

\bre[Measurability of $u$ in \cref{prob:msedp}]
It is natural to construct  the solution of \cref{prob:msedp} pathwise; that is, one defines $u(\omega)$ to be the solution of the deterministic problem with coefficients $A(\omega)$ and $n(\omega).$ However, is it then not obvious that $u$ is measurable.
In the proof of \cref{thm:hh-gen} below, we show that the measurability of $u$ follows from
\ben
\item a natural condition on the measurability of the coefficients and data (\cref{con:cborel} below), and 
\item the continuity of the map taking the coefficients of the deterministic PDE to the solution of the deterministic PDE (see \cref{lem:solcont} below).
  \een
\ere


In \cref{thm:hh-gen,thm:hh-hetero} we prove results on the well-posedness of \cref{prob:msedp,prob:somsedp,prob:svsedp} under conditions on $A,$ $n,$ $f,$ and $\Dm.$ Although $A,n,$ and $f$ are defined on $\Dp,$ since $\supp(I-A),$ $\supp(1-n),$ and $\supp f$ are compactly contained in  $\DR$ we can consider $A,n,$ and $f$ as functions on $\DR.$

\bcon[Regularity and stochastic regularity of $f,$ $A,$ and $n$]\label{con:hh-fAn}
The random fields $f, A,$ and $n$ satisfy $f \in \LtOLtDR,$  $A:\Omega \rightarrow \WoiDRspd$ with $A \in \LiOLiDRRRdtd,$ and $n \in \LiOLiDRRR.$
\econ

\bth[Equivalence of variational problems]\label{thm:hh-gen}
Under \cref{con:hh-fAn}:
\bit
\item The maps $\as$ and $\Ls$ (defined by \eqref{eq:SEDPas}) are well-defined.
\item $u \in %\LtOHozDDR$ 
L^2\big(\Omega;H_{0,D}^1(D_R)\big)$
solves \cref{prob:somsedp} if and only if $u$ solves \cref{prob:svsedp}.
\item If $u \in
L^2\big(\Omega;H_{0,D}^1(D_R)\big)$
%\in \LtOHozDDR$ 
solves \cref{prob:somsedp}, then any member of the equivalence class of $u$ solves \cref{prob:msedp}.
\item The solution of \cref{prob:msedp} exists and is unique up to modification on a set of measure zero in $\Omega.$
\item The solution of \cref{prob:somsedp,prob:svsedp} is unique in $L^2\big(\Omega;H_{0,D}^1(D_R)\big)$. %$\LtOHozDDR.$
\eit
\enth

The proof of \cref{thm:hh-gen} is on \cpageref{page:thmhhgenproof} below.

Observe that the only relationship between formulations not proved in \cref{thm:hh-gen} is: if $u:\Omega\rightarrow \HozDDR$  solves \cref{prob:msedp} then $u \in 
L^2\big(\Omega;H_{0,D}^1(D_R)\big)$
%\LtOHozDDR$ 
and $u$ solves \cref{prob:somsedp}. \cref{thm:hh-hetero} below includes this relationship, but we need additional assumptions on $A,n,$ and $\Dm.$ We use the notation established in \cref{def:NT}.

%% \bde[A particular class of (deterministic) nontrapping coefficients]\label{def:hh-nontrapping}
%% Let $\muo,\mut >0,$ $\Az \in \WoiDRRRdtd$ with $\supp(I-\Az) \compcont \BR$, and $\nz \in \WoiDRRR$ with $\supp(1-\nz) \compcont \BR.$ We write $\Az \in \NTA{\muo}$ and $\nz \in \NTn{\mut}$ if
%% \beq\label{eq:hh-Acond}
%% \Az(\bx) - \mleft(\bx\cdot\grad\mright)\Az(\bx) \geq \muo \quad \tand \quad \nz(\bx) + \bx\cdot\grad \nz(\bx) \geq \mut
%% \eeq
%% for almost every $\bx \in \DR,$ where the first inequality holds in the sense of quadratic forms.
%% \ede

%The significance of the class of coefficients in \cref{def:hh-nontrapping} is that \cref{thm:edp} proves bounds on the solution of \eqref{eq:hh-intro} for such $A$ and $n,$ where the constant in the bound only depends on $\muo, \mut, k, R, $ and $d.$

\bcon[{$k$-independent nontrapping conditions on (random) $A$ and $n$}]\label{con:hh-hetero}

\

\noindent The random fields $A$ and $n$ satisfy $A:\Omega\rightarrow\WoiDRspd$ and $n:\Omega\rightarrow \WoiDRRR$ with $\supp(I-A(\omega)) \compcont \BR$ and $\supp(1-n(\omega)) \compcont \BR$ almost surely.  Furthermore, there exist $\muo, \mut:\Omega\rightarrow \RR,$ independent of $f,$ with $\muo(\omega),\mut(\omega) > 0$ almost surely and $1/\muo,1/\mut \in \LtORR$  such that $A(\omega) \in \NTA{\muo(\omega)}$ almost surely and $n(\omega) \in \NTn{\mut(\omega)}$ almost surely.
\econ


\bde[Star-shaped]
The set $D \subseteq \RRd$ is \defn{star-shaped with respect to} the point $\bxz$ if for any $\bx \in D$ the line segment $\mleft[\bxz,\bx\mright] \subseteq D.$
\ede

\bth[{Equivalence of variational problems in a nontrapping case}]\label{thm:hh-hetero}

\

\noindent Let $\Dm$ be star-shaped with respect to the origin. Under \cref{con:hh-fAn,con:hh-hetero}:
\bit
\item The maps $\as$ and $\Ls$ (defined by \eqref{eq:SEDPas}) are well-defined.
\item \cref{prob:msedp,prob:somsedp,prob:svsedp} are all equivalent.
\item The solution $u \in %\LtOHozDDR$ 
L^2\big(\Omega;H_{0,D}^1(D_R)\big)$
of these problems exists, is unique, and, given $\kz > 0,$ satisfies the bound
\beq\label{eq:Sbound1}
\NLtOLtDR{\grad u}^2 + k^2\NLtOLtDR{u}^2\leq \NLoO{\Co} \NLtOLtDR{f}^2
\eeq
for all $k\geq\kz$, where $\Co:\Omega\rightarrow\RR$ is given by
\beq\label{eq:C1}
\Co = \max\mleft\{\frac1{\mu_1},\frac1{\mu_2}\mright\}\mleft(\frac{R^2}{\mu_1} + \frac{2}{\mu_2}\mleft(R+ \frac{d-1}{2\kz}\mright)^2\mright).
\eeq
\eit
\enth

The proof of \cref{thm:hh-hetero} is on \cpageref{page:thmhhheteroproof} below.

As highlighted above, \cref{thm:hh-hetero} is obtained by combining
deterministic a priori bounds from \cref{thm:edp} with the general
arguments in \cref{sec:general} about well-posedness of variational
formulations of stochastic PDEs. \cref{thm:hh-hetero} uses the most basic a priori bound proved
in \cite{GrPeSp:19} (from \cite[Theorem 2.5]{GrPeSp:19}), but \cite{GrPeSp:19}
contains several extensions of this bound. \cref{rem:planewave,rem:ones,rem:tedp,rem:jumps,rem:kdep}
outline the implications of these (deterministic) extensions for
the stochastic Helmholtz equation.

\bre[Dirichlet boundary conditions on $\GD$ and plane-wave incidence]\label{rem:planewave}

\

The formulations of the stochastic EDP above assume that $u=0$ on the boundary $\GD.$ An important scattering problem for which $u \neq 0$ on $\GD$ is when $u$ is the field scattered by an incident plane wave; in this case $\gamma u = -\gamma \uI,$ where $\uI$ is the incident plane wave \cite[p. 107]{ChGrLaSp:12}.

The results in this \lcnamecref{chap:stochastic} can be easily extended to the case when $u\neq0$ on $\pDm$ using \cite[Theorem 2.19(ii)]{GrPeSp:19}  which proves a priori (deterministic) bounds in this case. One subtlety, however, is that $f$ is then not necessarily independent of $\muo$ and $\mut.$ Indeed in this case
%\beqs%\label{eq:fplanewave}
$f = -\grad\cdot\mleft(A\grad \uI\mright) - k^2n\uI$.
%\eeqs
If $\muo$ depends on $A$ and $\mut$ depends on $n$ then %\eqref{eq:fplanewave} shows that 
$f$ may be not be independent of $\muo$ and $\mut.$ One can produce an analogue of \cref{thm:hh-hetero} in the case where $f,\muo,$ and $\mut$ are dependent, but one requires $1/\muo, 1/\mut \in \LfO$ and $f \in \LfOLtD;$ see \cref{rem:notindep} below.
\ere

\bre[The case when either $n=1$ or $A=I$]\label{rem:ones}
When either $n=1$ or $A=I,$ \cite[Theorem 2.19]{GrPeSp:19} gives deterministic bounds under weaker conditions on $A$ and $n$ respectively; the corresponding results for the stochastic case are that:

\bit
\item 

When $n=1$  almost surely, the condition $A(\omega) \in \NTA{\muo(\omega)}$ in \cref{con:hh-hetero} can be improved to
%\beqs
$2A(\omega) - \mleft(\bx \cdot \nabla \mright)A(\omega) \geq \muo(\omega)$
%\eeqs
for almost every $\bx\in \Dp,$ almost surely.

\item When $A=I$ almost surely, the condition $n(\omega) \in \NTn{\mut(\omega)}$ in \cref{con:hh-hetero} can be improved to:
\beq\label{eq:nimproved}
2n(\omega) + \bx \cdot \nabla n(\omega) \geq \mut(\omega) \,\text{ for almost every $\bx \in \Dp$, almost surely}.
\eeq
%for almost every $\bx\in \Dp,$ almost surely.
\eit
\ere

%% \bre[Geometric interpretation of the conditions on $A$ and $n$ in \cref{def:hh-nontrapping}]
%% Recall that the $k\tendi$ asymptotics of solutions of the Helmholtz
%% equation are governed by the behaviour of rays (see, e.g.,
%% \cite{BaBu:91}). Given (deterministic) $\Az$ and $\nz,$ the Helmholtz EDP is \defn{nontrapping} if all rays starting
%% in $\DR$ and evolving according to the
%% Hamiltonian flow defined by the symbol of $\grad\cdot\mleft(\Az\grad u\mright) + k^2\nz u = -\fz$  escape from $\DR$ after some uniform time (see, e.g., \cite[Definition 1.1]{Bu:02}); the EDP is \defn{trapping} otherwise.
%% The $k$-dependence of the solution operator depends strongly on
%% whether the problem is trapping, and the type of trapping present;
%% see, e.g., the overview discussion in \cref{sec:wpdisc} above.

%% The conditions on $A$ and $n$ in \cref{con:hh-hetero} and the star-shapedness restriction
%% on $\Dm$ are sufficient for the Helmholtz stochastic EDP to be nontrapping almost surely. As noted in \cref{rem:ones}, when $A=I$ almost surely the condition on $n$  can be improved  from that in \eqref{eq:hh-Acond} to \eqref{eq:nimproved} using \cite[Theorem 2.19(ii)]{GrPeSp:19}.
%% The condition \eqref{eq:nimproved} is equivalent  to nontrapping when $n$ is radial, i.e. $n(\omega,\bx)=
%% n(\omega,\abs{\bx})$. Indeed, if $n$ is radial and $2n(\omega,\bx)+ \bx \cdot\nabla n(\omega,\bx)<0$ at a point $\bx \in \RRd$,
%% then the deterministic Helmholtz EDP given by $n(\omega,\bx)$ is trapping; see \cite{Ra:71} and \cite[Theorem 7.7]{GrPeSp:19}.
%% \ere

\bre[The Helmholtz stochastic truncated exterior Dirichlet problem]\label{rem:tedp}

\

When applying the Galerkin method to \cref{prob:msedp,prob:somsedp,prob:svsedp}, the Dirichlet-to-Neumann map $\TrR$ is expensive to compute. Therefore, it is
common to approximate the DtN map on $\GR$ by an `absorbing
boundary condition' (see, e.g., \cite[Section 3.3]{Ih:98} and the references
therein), the  simplest of which is the impedance boundary condition
$\partial u/\partial \nu - \ri k u=0$. We call the Helmholtz stochastic EDP posed in $\DR$ with
an impedance boundary condition on $\GR$ the stochastic \emph{truncated
exterior Dirichlet problem} (stochastic TEDP). In fact, since we no longer need to know
the DtN map explicitly on the truncation boundary, the truncation
boundary can be arbitrary (i.e. it does not have to be just a circle/sphere). Note that in the case when the obstacle is the empty set, the TEDP is just the Interior Impedance Problem.

The results in this \lcnamecref{chap:stochastic} also hold for the stochastic TEDP (with arbitrary Lipschitz truncation boundary) under an analogue of \cref{con:hh-hetero} based on the deterministic bounds in \cite[Theorem A.6(i)]{GrPeSp:19} instead of \cite[Theorem 2.5]{GrPeSp:19}.
\ere

\bre[Discontinuous $A$ and $n$]\label{rem:jumps}
The requirements on $A$ and $n$ in \cref{con:hh-hetero} require them to be continuous (since  $\WoiDR = \CzoDR$ as $\DR$ is Lipschitz; see, e.g., \cite[Section 4.2.3, Theorem 5]{EvGa:92}). In addition to proving deterministic a priori bounds for the class of $A$ and $n$ in \cref{con:hh-hetero}, the paper \cite{GrPeSp:19} proves deterministic bounds for discontinuous $A$ and $n$ satisfying \cref{eq:ANT,eq:nNT} in a distributional sense; see \cite[Theorem 2.7]{GrPeSp:19}. In this case, when moving outward from the obstacle to infinity,  $A$ can jump downwards and $n$ can
jump upwards on interfaces that are star-shaped. (When the jumps are in the opposite direction, the problem is trapping; see \cite{PoVo:99a} and \cite[Section 6]{MoSp:19}). The
well-posedness results and a priori bounds in this \lcnamecref{chap:stochastic} can therefore be adapted to prove results about the stochastic Helmholtz equation
for a class of random $A$ and $n$ that allows nontrapping jumps on randomly-placed star-shaped interfaces.
\ere

\bre[$k$-dependent $A$ and $n$]\label{rem:kdep}
In this \lcnamecref{chap:stochastic} we focus on random fields $A$ and $n$ varying independently of $k;$ this corresponds to a fixed physical
medium, characterised by $A$ and $n$, with waves of frequency $k$ passing through.
In \cref{sec:generating} below we construct $A$ and $n$ as ($k$-independent) $W^{1,\infty}$ perturbations of random fields $\Az$ and $\nz$ satisfying \cref{con:hh-hetero}.
We note, however, that results for $A$ and $n$ being
\emph{$k$-dependent} $L^\infty$ perturbations (i.e. rougher, but $k$-dependent perturbations) of $\Az$
and $\nz$ satisfying \cref{con:hh-hetero} can easily be obtained.

The basis for these bounds is observing that \emph{deterministic} a priori bounds hold when
(a) $A\in \NTA{\mu_1}$, $n = \nz + \eta,$ where $\nz \in \NTn{\mu_2}$
and $k\NLiDRRR{\eta}$ is sufficiently small, and
(b) $A=\Az+B$, $n=\nz+\eta$, where $\Az\in \NTA{\mu_1}$, $\nz \in
\NTn{\mu_2}$, $k\NLiDRRR{\eta}$ and $k \NWoiDRRRdtd{B}$ are
both sufficiently small, and $A, n,$ and $D_-$ are such that $u\in
H^2(D_R)$ (see, e.g., \cite[Theorem 4.18(i)]{Mc:00} or \cite[Theorems
2.3.3.2 and 2.4.2.5]{Gr:85} for these latter requirements). Given these deterministic bounds, the general arguments in this \lcnamecref{chap:stochastic}
can then be used to prove well-posedness of the analogous stochastic
problems.

To understand why bounds hold in the case (a), observe that one can
write the PDE as
\beq\label{eq:pert}
\nabla \cdot(A\grad u) +k^2 n_0 u = -f - k^2 \eta u;
\eeq
if $k\NLiDRRR{\eta}$ is sufficiently small then the contribution from
the $k^2 \eta u$ term on the right-hand side of \eqref{eq:pert} can be
absorbed into the $k^2\|u\|^2_{L^2(D_R)}$ term appearing on the left-hand
side of the bound (the deterministic analogue of \eqref{eq:Sbound1}).
In the case $\nz=1,$ this is essentially the argument used to prove the
a priori bound in \cite[Theorem 2.4]{FeLiLo:15} (see \cite[Remark
2.15]{GrPeSp:19}).
The reason bounds hold in the case (b) is similar, except now we need
the $H^2$ norm of $u$ on the left-hand side of the bound (as well as
the $H^1$ norm) to absorb the contribution from the $\nabla \cdot
(B\grad u)$ term on the right-hand side.
\ere


\subsection{Random fields satisfying \cref{con:hh-hetero}}\label{sec:generating}
The main focus of this paper is proving well-posedness of the variational formulations of the stochastic Helmholtz equation, and a priori bounds on the solution, for the most-general class of $A$ and $n$ allowed by the deterministic bounds in \cite{GrPeSp:19}. However, in this section, motivated by the Karhunen-Lo\`eve expansion (see e.g.~\cite[p.~201ff.]{LoPoSh:14}) and similar expansions of material coefficients for the stationary diffusion equation \cite[Section 2.1]{KuNu:16}, we consider $A$ and $n$ as series expansions around known non-random fields $\Az$ and $\nz$ satisfying \cref{con:hh-hetero} (i.e., \cref{con:hh-hetero} is satisfied for $\nz, \Az$ independent of $\omega \in \Omega$, and therefore $\muo,\mut$ independent of $\omega$).
Define
\beq\label{eq:nseries}
A(\omega,\bx) = \Az(\bx) + \sum_{j=1}^\infty \Yj(\omega) \Psij(\bx)\quad\text{and}\quad n(\omega,\bx) = \nz(\bx) + \sum_{j=1}^\infty\Zj(\omega) \psij(\bx) ,
\eeq
where:
\bit
\item $\supp\mleft(1-\Az\mright),\,\supp \mleft(I-\nz\mright) \compcont \BR,$
\item $\Az$ and $\nz$ satisfy \cref{con:hh-hetero} with $\muo$ and $\mut$ independent of $\omega \in \Omega$
\item $\Yj,\Zj \sim \Unif(-1/2,1/2)$ i.i.d.,
\item $\Psij \in \WoiDRRRdtd$ with $\supp \Psij \compcont \BR$ for all $j =1,\ldots,m$,
\beq\label{eq:Apsimeas}
\sum_{j=1}^\infty \NWoiDRRRdtd{\Psij} < \infty, \tand
\eeq
\beq\label{eq:Apsipos}
\sum_{j=1}^\infty \esssup_{\bx \in \DR} \NopCCd{\Psij} < 2\Azmin,
\eeq
where $\Azmin > 0$ is such that $\Azmin\abs{\bxi}^2 \leq \big(A(\bx)\bxi\big)\cdot \bxi$ for almost every $\bx \in \Dp$ and for all $\bxi \in \CCd$, and where $\NopCCd{\cdot}$ is the operator norm induced by the Euclidean vector norm on $\CCd$ (i.e., $\NopCCd{\cdot}$ is the spectral norm).
\item $\psij \in \WoiDRRR$ with $\supp \psij \compcont \BR$ for all $j = 1,\ldots,m$,
\beq\label{eq:npsimeas}
\sum_{j=1}^\infty \NWoiDRRR{\psij} < \infty, \tand
\eeq
\beq\label{eq:npsipos}
\sum_{j=1}^\infty \NLiDRRR{\psij} < 2 \nzmin,
\eeq
where $\nzmin \de \essinf_{\bx \in \DR} \nz(\bx),$ and
\eit
%Regarding the measurability of $n$ and $A$ defined by \eqref{eq:nseries}: the proof for this particular setting of mappings into a separable subspace of a general normed vector space is analagous to the proof in the case of a sum of real-valued measurable mappings, and so we do not give it here.

The assumptions \eqref{eq:Apsipos} and \eqref{eq:npsipos} ensure that $A > 0$ (in the sense of quadratic forms) and $n > 0$  almost surely, and the assumptions \eqref{eq:Apsimeas} and \eqref{eq:npsimeas} are used to prove $A$ and $n$ are measurable.

Regarding the measurability of $A$ and $n$ defined by \eqref{eq:nseries}: the proof that $A$ and $n$ given by \eqref{eq:nseries} are measurable is given in \cref{lem:seriesmeas}, and relies on the proof that the sum of measurable functions
is measurable. This latter result is standard, but we have not been able to find this result for this particular setting of mappings
into a separable subspace of a general normed vector space, and so we briefly give it in \cref{lem:summeas}. 

The following lemmas give sufficient conditions for the series in \eqref{eq:nseries} to satisfy \cref{con:hh-hetero}.

\ble[Series expansion of $A$ satisfies \cref{con:hh-hetero}]\label{lem:Agen}
Let $\mu > 0$, $\delta \in \mleft(0,1\mright).$ If $\Az \in \NTA{\mu},$ and
\beq\label{eq:Aseriescond}
\sum_{j=1}^\infty \esssup_{\bx \in \DR} \NopCCd{\Psij(\bx) - \mleft(\bx\cdot\grad\mright)\Psij(\bx)} \leq 2\delta\mu,
\eeq
then $A \in \NTA{(1-\delta)\mu}$ almost surely.
\ele

\bpf[Proof of \cref{lem:Agen}]
Since $\Az \in \NTA{\mu},$ we have 
\beq\label{eq:Aseries1}
\Big(\mleft(A(\omega,\bx) - \mleft(\bx\cdot\grad\mright)A(\omega,\bx)\mright)\bxi\Big)\cdot\bxib \geq\mu \abs{\bxi}^2+\sum_{j=1}^\infty  \Big(\Yj(\omega)\mleft(\Psij(\bx)- \mleft(\bx\cdot\grad\mright)\Psij(\bx)\mright)\bxi\Big)\cdot\bxib
\eeq
for all $\bxi \in \CCd,$ for almost every $\bx \in \DR,$ almost surely. As $\Yj \sim \Unif(-1/2,1/2)$ for all $j$ and the bound \eqref{eq:Aseriescond} holds, the right-hand side of \eqref{eq:Aseries1} is bounded below by
\beqs
\mu \abs{\bxi}^2 -\half 2\delta \mu \abs{\bxi}^2 = (1-\delta)\mu\abs{\bxi}^2\text{ almost surely.}
\eeqs
Since $\bxi \in \CCd$ was arbitrary, it follows that $A(\omega) \in \NTA{(1-\delta)\mu)}$ almost surely, as required.
\epf

\ble[Series expansion of $n$ satisfies \cref{con:hh-hetero}]\label{lem:ngen}
Let $\mu > 0$ and $\delta \in \mleft(0,1\mright).$ If $\nz \in \NTn{\mu}$ and
\beq\label{eq:nseriescond}
\sum_{j=1}^m\NLiDRRR{\psij(\bx) + \bx\cdot\grad\psij(\bx)} \leq 2\delta\mu,
\eeq
then $n \in \NTn{(1-\delta)\mu}.$
\ele

The proof of \cref{lem:ngen} is omitted, since it is similar to the proof of \cref{lem:Agen}; in fact it is simpler, because it involves scalars rather than matrices.


\subsection{Outline of the chapter} In \cref{sec:otherwork} we discuss our results in the context of related literature. In \cref{sec:general} we  state general results on a priori bounds and well-posedness for stochastic variational formulations. In \cref{sec:genproof} we prove the results in \cref{sec:general}. In \cref{sec:hhproof} we prove \cref{thm:hh-gen,thm:hh-hetero}.% In \cref{sec:federico} we discuss the failure of Fredholm theory for the stochastic variational formulation of Helmholtz problems. In \cref{app:mtBs} we recap results from measure theory and the theory of Bochner spaces.


\subsection{Discussion of the main results in the context of other work on UQ for time-harmonic wave equations}\label{sec:otherwork}

In this section we discuss existing results on well-posedness of \eqref{eq:hh-intro}, as well as analogous results for the elastic wave equation and the time-harmonic Maxwell's equations. The most closely-related work to this \lcnamecref{chap:stochastic} is \cite{FeLiLo:15} (and its analogue for elastic waves \cite{FeLo:17}), in that a large component of \cite{FeLiLo:15} consists of attempting to prove well-posedness and a priori bounds for the stochastic variational formulation (i.e.~\cref{prob:svsedp}) of the Helmholtz Interior Impedance Problem; i.e., \eqref{eq:hh-intro} with $A=I$ and stochastic $n$ posed in a bounded domain with an impedance boundary condition $\partial u/\partial \nu - ik u = g$. (Recall that this boundary condition is a simple approximation to the Dirichlet-to-Neumann map $\TrR$ defined above \eqref{eq:src}.) Under the assumption of existence, \cite{FeLiLo:15} shows that for any $k>0$ the solution is unique and satisfies an a priori bound of the form \eqref{eq:Sbound1} (with different constant $\Co$), provided $n=1+\eta$ where the random field $\eta$ satisfies (almost surely) $\N{\eta}_{L^\infty} \leq C/k$ for some $C>0$ independent of $k$. \cite{FeLiLo:15} then invokes Fredholm theory to conclude existence, but this relies on an incorrect assumption about compact inclusion of Bochner spaces---see \cref{sec:federico} below. However, combining \cref{thm:hh-gen,rem:tedp,rem:kdep} with $A=I$ and $\nz=1+\eta$ (with $\eta$ as above) produces an analogous result to \cref{thm:hh-hetero}, and gives a correct proof of \cite[Theorem 2.5]{FeLiLo:15}. Therefore the analysis of the Monte Carlo interior penalty discontinuous Galerkin method in \cite{FeLiLo:15} can proceed under the assumptions of \cref{thm:hh-gen,rem:tedp,rem:kdep}.

The papers \cite{HiScScSc:15} and \cite{Sc:17} consider the Helmholtz transmission problem with a stochastic interface, i.e.~\eqref{eq:hh-intro} posed in $\RRd$ with both $A$ and $n$ piecewise constant and jumping on a common, randomly-located interface. A component of this work is establishing well-posedness of \cref{prob:msedp} for this setup. To do this, the authors make the assumption that $k$ is small (to avoid problems with trapping mentioned above---see the comments after \cite[Theorem 4.3]{HiScScSc:15}); the sesquilinear form $a$ is then coercive and a priori bounds (in principle explicit in $A$ and $n$) follow in Sobolev norms \cite[Lemma 4.5]{HiScScSc:15} and H\"older norms \cite[Theorem 5.1 and Corollary 5.2]{Sc:17}. By \cref{rem:jumps}, the results of this \lcnamecref{chap:stochastic} can be used to obtain the analogous well-posedness result for large $k$ in the case of nontrapping jumps.

The paper \cite{BuGh:14} studies the \defn{Bayesian inverse problem} associated to \eqref{eq:hh-intro} with $A=I$ and $n=1$  posed in the  exterior of a Dirichlet obstacle. That is, \cite{BuGh:14} analyses computing the posterior distribution of the shape of the obstacle given noisy observations of the acoustic field in the exterior of the obstacle. A component of the analysis in \cite{BuGh:14} is the well-posedness of the forward problem for an obstacle with a variable boundary \cite[Proposition 3.5]{BuGh:14}. Instead of mapping the problem to one with  a fixed domain and variable $A$ and $n,$ \cite{BuGh:14} instead works with the variability of the obstacle directly, using boundary-integral equations. The $k$-dependence of the solution operator is not considered, but would enter in \cite[Lemma 3.1]{BuGh:14}.

%The paper \cite{KaPo:18} considers the time-harmonic Maxwell's equations posed on a fixed domain, with the material coefficient $\sigma$ defined by a random field on a subset of the domain. Part of the analysis of this paper is proving the well-posedness of the analogue of \cref{prob:somsedp} for the time-harmonic Maxwell's equations. The authors achieve this well-posedness result by working in a low-frequency regime, so that the analogue of the sesquilinear form $a$ is bounded and coercive (analagous to the approach in \cite{HiScScSc:15}), and therefore the Lax--Milgram Theorem can be applied to the deterministic problem to obtain existence, uniqueness, and a deterministic a priori bound. This bound is then converted into a bound for the stochastic problem using similar techniques to those in this paper. 

The papers \cite{JeScZe:17} and \cite{JeSc:16} consider the time-harmonic Maxwell's equations with (i) the material coefficients $\eps,\mu$ constant in the exterior of a perfectly-conducting random obstacle and (ii) $\eps,\mu$ piecewise constant and jumping on a common randomly located interface; in both cases these problems are mapped to problems where the domain/interface is fixed and $\eps$ and $\mu$ are random and heterogeneous. The papers \cite{JeScZe:17} and \cite{JeSc:16} essentially consider the analogue of \cref{prob:msedp} for the time-harmonic Maxwell's equations, obtaining well-posednes from the corresponding results for the related deterministic problems. Similarly the paper \cite{EsJe:19} considers analogues of \cref{prob:msedp} for the Helmholtz equation with constant coefficients and a random (penetrable or inpenetrable) obstacle. The paper \cite{EsJe:19} obtains \emph{deterministic} tensor-product boundary-integral equations for the statistical moments of $u.$


\section[General stochastic a priori bounds and well-posedness]{General results on proving a priori bounds and well-posedness of stochastic variational formulations}\label{sec:gen-framework}\label{sec:general}
In this section we state general results for proving a priori bounds and well-posedness results for variational formulations of linear elliptic SPDEs.

\subsection{Notation and definitions of the variational formulations}\label{sec:notdef}
Let $\OFP$ be a complete probability space. Let $X$ and $Y$ be separable Banach spaces over a field $\FF,$ (where $\FF = \RR$ or $\CC$).
Let $\homspace$ denote the space of bounded linear maps $X\rightarrow\Ys.$ Let $\cC$ be a topological space with topology $\TC.$ Given maps
\beqs
\coeff:\Omega\rightarrow\cC,\quad\toform:\cC \rightarrow \homspace,\quad\text{and } \rhs:\cC \rightarrow \Ys,
\eeqs let $\SAc:\LtOX \rightarrow \LtOYas$ and  $\SLc \in \LtOYas$ be defined by
\beq\label{eq:SA}
\big[\SAc(u)\big](v) \de \int_\Omega \big[\Acomega u(\omega)\big]\big(v(\omega)\big) \dd\PP(\omega) \quad\text{and} \quad 
\SLc(v) \de \int_\Omega \Lcomega\big(v(\omega)\big) \dd \PP(\omega)
\eeq
%for $v \in \LtOY,$ and let be defined by
%\beq\label{eq:SL}
%%\SLc(v) \de \int_\Omega \Lcomega\big(v(\omega)\big) \dd \PP(\omega)
%\eeq
for $v \in \LtOY.$ Recall that a bounded linear map $X\rightarrow\Ys$ is equivalent to a sesquilinear (or bilinear) form on $X \times Y;$ see e.g. \cite[Lemma 2.1.38]{SaSc:11}. To keep notation compact, we write $\Acomega=(\toform \circ c)(\omega)$ and $\Lcomega=(\rhs\circ c)(\omega).$



\bre[Interpretation of the space $\cC$]
The space $\cC$ is the `space of inputs'. For the stochastic Helmholtz EDP in \cref{sec:hh-results} the space $\cC$ is defined in \cref{def:cCHh} below, but the upshot of this definition is that for any $\omega \in \Omega$ the triple $(A(\omega),n(\omega),f(\omega))$ is an element of $\cC.$
The maps $\coeff,$ $\toform,$ and $\rhs$ are given by $\coeff = \mleft(A,n,f\mright),$ $\toform = a,$ and $\rhs = L,$ where $a$ and $L$ are given by \eqref{eq:SEDPa} and \eqref{eq:SEDPL} respectively and the equality $\toform=a$ is meant in the sense of the one-to-one correspondence between $\homspace$ and sesquilinear forms on $X\times Y.$
\ere

The following three problems are the analogues in this general setting of \cref{prob:msedp,prob:somsedp,prob:svsedp} in \cref{sec:intro}.

\bprobM[Measurable variational formulation almost surely]\label{prob:meas}
Find a measurable function $u:\Omega \rightarrow X$ such that 
\beq\label{eq:aeeq}
\Acomega u(\omega) = \Lcomega \tin \Ys
\eeq
almost surely.
\eprobvar

\bprobLT[Second-order moment variational formulation almost surely]\label{prob:lt}
Find $u \in \LtOX$ such that \eqref{eq:aeeq} holds almost surely.
\eprobvar

\bprobSVAR[Stochastic variational formulation]\label{prob:svar}
Find $u \in \LtOX$ such that
\beq\label{eq:stoeq}
\SAc u = \SLc \tin \LtOYas.
\eeq
\eprobvar

\bre[Immediate relationships between formulations]\label{rem:imm}
Since $\LtOX \subseteq \BorelOX$ (the space of all measurable functions $\Omega\rightarrow X$)
it is immediate that if $u$ solves \cref{prob:lt} then every member of the equivalence class of $u$ solves \cref{prob:meas}.
\ere

\subsection{Conditions on $\toform,$ $\rhs,$ and $\coeff$}\label{sec:cons}
We now state the conditions under which we prove results about the equivalence of \cref{prob:meas,prob:lt,prob:svar}.


\bconAo[$\toform$ is continuous] \label{con:coeffstoform}
The function $\toform:\cC \rightarrow \homspace$ is continuous, where we place the norm topology on $X,$ the dual norm topology on $\Ys$, and the operator norm topology on $\homspace.$
\econvar

\bconA[Regularity of {$\toform\circ\coeff$}]\label{con:A}
The map $\toform\circ\coeff \in \LiOhomspace.$
\econvar


We note that \cref{con:A} is violated in the well-studied case of a log-normal coefficient $\kappa$ for the stationary diffusion equation \eqref{eq:diffusion}; in order to ensure the stochastic variational formulation is well-defined in this case, one must change the space of test functions as in \cite{Gi:10,MuSt:11}

\bconLo[$\rhs$ is continuous] \label{con:coeffstofunc}
The function $\rhs:\cC \rightarrow \Ys$  is continuous, where we place the dual norm topology on $\Ys.$
\econvar

\bconL[Regularity of {$\rhs\circ\coeff$}]\label{con:L}
The map $\rhs\circ \coeff \in \LtOYs.$
\econvar

\bconCo[$\coeff$ is measurable]\label{con:cborel}
 The function $c:\Omega \rightarrow \cC$ is measurable.
\econvar

To state the next condition, we need to recall the following definition.

\bde[{$\PP$}-essentially separably valued {\cite[p26]{Ry:02}}]\label{def:sepval}
Let $\mleft(S,\Top{S}\mright)$ be a topological space. A function $h:\Omega\rightarrow S$ is \defn{$\PP$-essentially separably valued} if there exists $E \in \cF$ such that $\PP(E) = 1$ and $h(E)$ is contained in a separable subset of $S$.
\ede


\bconC[{$\coeff$} is {$\PP$}-essentially separably valued]\label{con:C}
The map $\coeff:\Omega\rightarrow\cC$  is $\PP$-essentially separably valued.
\econvar

\bre[Why do we need \cref{con:C}?]
The theory of Bochner spaces requires strong measurability of functions (see \cref{def:strongmeas,def:bochnerspace} below). However, the proof techniques used in this paper rely heavily on the measurability of functions (see \cref{def:meas} below). In separable spaces these two notions are equivalent (see \cref{cor:pettis}). However, some of the spaces we encounter (such as $\LiDRRR$) are not separable. Therefore, in our arguments we use \cref{con:C} along with the Pettis Measurability Theorem (\cref{thm:pettis} below) to conclude that measurable functions are strongly measurable.
\ere

\bconB[A priori bound almost surely]\label{con:B}
There exist $\Cj,\fj:\Omega\rightarrow \RR, \,j=1,\ldots,m$ such that $\Cj\fj \in \LoO$ for all $j=1,\ldots,m$ and the bound
\beq \label{eq:sbe1}
\NX{u(\omega)}^2 \leq \sum_{j=1}^m\Cj(\omega)\fj(\omega)
\eeq
holds almost surely.
\econvar

\bre[Notation in the a priori bound]
We use the notation $f_j$ in the right-hand side of \eqref{eq:sbe1} to emphasise the fact that typically these terms relate to the right-hand sides of the PDE in question. For the stochastic Helmholtz EDP, $m=1,$ $f_1 = \NLtD{f}^2,$ and $C_1$ is given by \eqref{eq:C1}.
\ere


\bconK[Uniqueness almost surely]\label{con:K}
$\ker\mleft(\Acomega\mright) = \set{0}$ $\PP$-almost surely.
\econvar

The condition  $\ker\mleft(\Acomega\mright) = \set{0}$ $\PP$-almost surely can be stated as: given $\cG \in \LtOYas,$ for $\PP$-almost every $\omega \in \Omega$ the deterministic problem $\Acomega \uz = \cG$ has a unique solution,

\subsection{Results on the equivalence of \cref{prob:meas,,prob:lt,,prob:svar}}


\bth[Measurable solution implies second-order solution]\label{thm:3}
Under \cref{con:B}, if $u$ solves \cref{prob:meas} then $u$ solves \cref{prob:lt}  and satisfies the stochastic a priori bound
\beq\label{eq:sbresult}
\NLtOX{u}^2 \leq\sum_{j=1}^m \NLoO{C_jf_j}.
\eeq
\enth

Note that the right-hand side of the stochastic a priori bound \eqref{eq:sbresult} is the expectation of the right-hand side of the bound \eqref{eq:sbe1}.

\ble[Stochastic variational formulation well-defined]\label{lem:svarwelldefined}
Under \cref{con:coeffstoform,,con:A,,con:coeffstofunc,,con:L,,con:cborel,,con:C},  the maps $\SA$ and $\SL$ defined by \eqref{eq:SA} are well-defined in the sense that
\beq\label{eq:finite}
\mleft[\SA(\vo)\mright](\vt),\, \SL(\vt) < \infty \quad\text{for all } \vo \in \LtOX, \text{ for all }\vt \in \LtOY.
\eeq
\ele

\bth[Second-order solution implies stochastic variational solution]\label{thm:11}
Under \cref{con:coeffstofunc,,con:L,,con:cborel,,con:C}, if $u$ solves \cref{prob:lt} then $u$ solves \cref{prob:svar}.
\enth

\bth[Stochastic variational solution implies second-order solution]\label{thm:12}
If \cref{prob:svar} is well-defined and $u$ solves \cref{prob:svar}, then $u$ solves \cref{prob:lt}.
\enth

\Cref{thm:3,thm:11,thm:12,lem:svarwelldefined} are summarised in \cref{fig:ladder}.

\begin{figure}[h]
  \centering
  \scalebox{0.90}{
\begin{tikzpicture}

% Inspired by https://tex.stackexchange.com/questions/109102/anchor-arrow-start-position

\draw (0,1) node [rounded rectangle, fill=gray!45!white] (meas) {\Cref{prob:meas}};



\draw (0,-1) node [rounded rectangle, fill=gray!45!white] (lt) {\Cref{prob:lt}};


\draw (0,-3) node [rounded rectangle, fill=gray!45!white] (svar) {\Cref{prob:svar}};


\path node [left = \owenshift of meas.north] (meas top left) {};
\path node [right = \owenshift of meas.north] (meas top right) {};
\path node [left = \owenshift of meas.south] (meas bottom left) {};
\path node [right = \owenshift of meas.south] (meas bottom right) {};

\path node [left = \owenshift of lt.north] (lt top left) {};
\path node [right = \owenshift of lt.north] (lt top right) {};
\path node [left = \owenshift of lt.south] (lt bottom left) {};
\path node [right = \owenshift of lt.south] (lt bottom right) {};

\path node [left = \owenshift of svar.north] (svar top left) {};
\path node [right = \owenshift of svar.north] (svar top right) {};




% Arrows
\begin{scope}[->]


\draw
(meas bottom right)
--
node[right,align=center,text width=5cm] {Under \cref{con:B}, get stochastic a priori bound \eqref{eq:sbresult} (\Cref{thm:3})}
 (lt top right);

\draw
(lt top left)
--
node[left] {Immediate}
 (meas bottom left);

\draw
(lt bottom right)
--
node[right,align=center,text width=5cm] {Under \cref{con:coeffstofunc,,con:L,,con:cborel,,con:C},  (\Cref{thm:11})}
 (svar top right);

\draw
(svar top left)
--
node[left,align=center,text width=5cm] {If \cref{prob:svar} is well-defined (\cref{thm:12})}
(lt bottom left);
\end{scope}

\path node [align=center,below = \owentextshift of svar,text width=10cm] (svar wd) {Well-defined under  \cref{con:coeffstoform,,con:A,,con:coeffstofunc,,con:L,,con:cborel,,con:C} (\Cref{lem:svarwelldefined})};


\end{tikzpicture}
}
\caption[The relationship between the different variational formulations of stochastic PDEs]{The relationship between the variational formulations. An arrow from Problem P to Problem Q with Conditions R indicates `under Conditions R, the solution of Problem P is a solution of Problem Q'}\label{fig:ladder}
\end{figure}


\bre[\Cref{con:L} in \cref{thm:11}]
In \Cref{thm:11} we could replace \cref{con:L} with \cref{con:A}, and the result would still hold---see the proof for further details. However, \cref{con:L} is less restrictive than \cref{con:A}, as it only requires $L^2$ integrability of $\rhs\circ\coeff$ as opposed to essential boundedness of $\toform\circ\coeff.$ 
\ere

\ble[Showing uniqueness of the solution to \cref{prob:meas,prob:lt,prob:svar}]\label{lem:uniq}
If \cref{con:K} holds, then
\ben
\item\label[part]{part:uniq1} the solution to \cref{prob:meas} (if it exists) is unique up to modification on a set of $\PP$-measure 0 in $\Omega$, 
\item\label[part]{part:uniq2} the solution to \cref{prob:lt} (if it exists) is unique in $\LtOX,$ and 
\item\label[part]{part:uniq3} if \cref{prob:svar} is well-defined, the solution to \cref{prob:svar} (if it exists) is unique in $\LtOX.$
\een
\ele

\bre[Informal discussion on the ideas behind the equivalence results]\label{rem:nontechnical}
The diagram in \cref{fig:ladder} summarises the relationships between the variational formulations, and the conditions under which they hold. Moving `up' the left-hand side of the diagram, we prove a solution of \cref{prob:svar} is a solution of \cref{prob:lt} in \cref{thm:12}; the key idea in this theorem is to use a particular set of test functions and the general measure-theory result of \cref{lem:gotoae} below; this approach was used for the stationary diffusion equation \eqref{eq:diffusion} with log-normal coefficients in \cite{Gi:10}, and for a wider class of coefficients in \cite{MuSt:11}.

Moving `down' the right-hand side, we prove a solution of \cref{prob:meas} is a solution of \cref{prob:lt} in \cref{thm:3}; the key part of this proof is that the bound in \cref{con:B} gives information on the integrability of the solution $u.$ (In the case of \eqref{eq:diffusion} with uniformly coercive and bounded coefficient $\kappa,$ the analogous integrability result follows from the Lax--Milgram theorem; \cite[Proposition 2.4]{Ch:12} proves an equivalent result for \eqref{eq:diffusion} with lognormal coefficient $\kappa$ with an isotropic Lipschitz covariance function.) Proving a solution of \cref{prob:lt} is a solution of \cref{prob:svar} in \cref{thm:11} essentially amounts to posing conditions such that the quantities $\mleft[\Acomega\mleft(u(\omega)\mright)\mright] \mleft(v(\omega)\mright)$ and $\Lcomega \mleft(v(\omega)\mright)$ are Bochner integrable for any $v \in \LtOY,$ so that \eqref{eq:stoeq} makes sense. \Cref{lem:svarwelldefined} shows that the stronger property \eqref{eq:finite} holds, and requires stronger assumptions than \cref{thm:11}, since the proof of \cref{thm:11} uses the additional information that u solves \cref{prob:lt}.
\ere

\bre[Changing the condition $u \in \LtOX$]
Here we seek the solution $u \in \LtOX$ but we could instead require $u \in \LpOX,$ for some $p>0$ and require $\SA u = \SL$ in $\LqOYas,$ for some $q>0$ (i.e. use test functions in $\LqOY$). In this case, the proof of \cref{thm:12} would be nearly identical, as the space $\D$ of test functions used there %(see \eqref{eq:Dspace} below) 
is a subset of $\LqOY$ for all $q>0.$ One could also develop analogues of \cref{thm:3,thm:11,lem:svarwelldefined} in this setting---see e.g.~\cite[Theorem 3.20]{Gi:10} for an example of this approach for the stationary diffusion equation with lognormal diffusion coefficient.
\ere

\bre[Non-reliance on the Lax-Milgram theorem]
The above results hold for an arbitrary sesquilinear form and hence are applicable to a wide variety of PDEs; their main advantage is that they apply to PDEs whose stochastic variational formulations are not coercive. For example, as noted in \cref{sec:intro}, for the stationary diffusion equation \eqref{eq:diffusion} with coefficient $\kappa$ bounded uniformly below in $\omega,$ the bilinear form of \cref{prob:svar} is coercive; existence and uniqueness follow from the Lax-Milgram theorem, and hence the chain of results above leading to the well-posedness of \cref{prob:svar} is not necessary.
\ere

\bre[Overview of how these results are applied to the Helmholtz equation in \cref{sec:hhproof}]

We obtain the results for the Helmholtz equation via the following steps (which could also be applied to other SPDEs fitting into this framework):
\ben
\item\label[step]{it:step1pw} Define the map $\coeff$ (via $A,n,$ and $f$) such that for almost every $\omega \in \Omega$ there exists a solution of the deterministic Helmholtz EDP corresponding to $\coeff(\omega).$
\item\label[step]{it:step2pw} Define $u:\Omega\rightarrow X$ to map $\omega$ to the solution of the deterministic problem corresponding to $\coeff(\omega).$
\item Prove that \cref{con:coeffstoform,,con:A,,con:coeffstofunc,,con:L,,con:cborel,,con:C,,con:B,,con:K} hold, so that one can apply \cref{thm:3,,thm:11,,thm:12} along with \cref{lem:svarwelldefined,lem:uniq} to show \cref{prob:svsedp} is well-defined and $u$ is unique and satisfies \cref{prob:msedp,prob:somsedp,prob:svsedp}.
\een
\Cref{it:step1pw,it:step2pw} can be thought of as constructing a solution pathwise.
\ere


\section{Proof of the results in \cref{sec:general}}\label{sec:genproof}

\subsection{Preliminary lemmas}\label{sec:prelemmanary}
To simplify notation, we introduce the following definition.
\bde[Pairing map]
For fixed $\coeff:\Omega\rightarrow\cC,\,\toform:\Omega\rightarrow\homspace,$ given $v:\Omega\rightarrow X$ we define the map $\comp:\Omega\rightarrow \Ys$ by
\beq\label{eq:comp}
\comp(\omega) \de \mleft[\mleft(\toform\circ\coeff\mright)(\omega)\mright]\mleft(v(\omega)\mright).
\eeq
\ede

A key ingredient in proving that the stochastic variational formulation is well-defined (\cref{lem:svarwelldefined}) is showing that the maps $\compu$ and $\fullrhs$ are measurable. Showing that $\fullrhs $ is measurable is straightforward (see \cref{lem:thetaborel} below), but showing that $\compu$ is measurable is not. This is because $\fullrhs$ depends on $\omega$ only through its dependence on $\coeff,$ but $\compu$ depends on $\omega$ through both the dependence of $\toform\circ\coeff$ on $\omega$ and the dependence of $u$ on $\omega;$ it is this dual dependence that causes the extra complication.

\ble[$\fullrhs$ is measurable] \label{lem:thetaborel}
Under \cref{con:coeffstofunc,con:cborel} the function $\fullrhs$ is measurable.
\ele

\begin{proof}[Proof of \cref{lem:thetaborel}]
The map $\coeff$ is measurable (by \cref{con:cborel})
  and $\rhs$ is continuous  (by \cref{con:coeffstofunc}), therefore \cref{lem:contplusmeas} implies that $\fullrhs $ is measurable.
\end{proof}

We now move on to the more-involved process of showing $\comp$ is measurable.

\bde[Product map]
For $v:\Omega\rightarrow X,$ let $\toprod:\Omega\rightarrow \homspace \times X$ be defined by $\toprod(\omega) = \big(\mleft(\toform\circ\coeff\mright)(\omega),v(\omega)\big).$
\ede

\ble[Product map is measurable]\label{lem:Pmeas}
When $\homspace \times X$ is equipped with the product topology,  if \cref{con:coeffstoform,con:cborel} hold, and if $v:\Omega\rightarrow X$ is measurable, then $\toprod:\Omega\rightarrow \homspace\times X$ is measurable.
\ele

\bpf[Proof of \cref{lem:Pmeas}]
By the result on the measurability of the Cartesian product of measureable functions (\cref{lem:measprod}), $\toprod$ is measurable with respect to $\big(\cF,\bigBorel{\homspace}\otimes \Borel{X}\big)$ (where $\cB$ denotes the Borel $\sigma$-algebra---see \cref{def:borelsigma}), as both of the coordinate functions $\toform\circ c$ and $v$ are measurable. Since $\homspace$ and $X$ are both metric spaces, they are both Hausdorff. As $X$ is separable, \cref{lem:bogachev} on the product of Borel $\sigma$-algebras imples $\bigBorel{\homspace} \otimes \Borel{X} = \bigBorel{\homspace \times X}.$ Hence $\toprod$ is measurable with respect to $\big(\cF,\bigBorel{\homspace\times X}\big).$
\epf

\bde[Evaluation map]
Let $Z$ be a separable Banach space. The function $\eval:\homspaceZ \times X \rightarrow \Zs$ is defined by
\beq\label{eq:evaldef}
\eval\big(\mleft(\cH,v\mright)\big) \de \cH(v) \quad\tfor \cH \in \homspaceZ \tand v \in X.
\eeq
\ede

Observe that the pairing, product, and evaluation maps ($\comp, \toprod,$ and, $\evalY$ respectively) are related by $\comp = \evalY \circ \toprod.$

\ble[Evaluation map is continuous]\label{lem:vstarcont}
Let $Z$ be a separable Banach space. The map $\eval$ is continuous with respect to the product topology on $\homspaceZ \times X$ and the dual norm topology on $\Zs.$
\ele

The proof of \cref{lem:vstarcont} is straightforward and omitted.

\ble[$\comp$ is measurable] \label{lem:gammaborel}
If  \cref{con:coeffstoform,con:cborel} hold and $v$ is measurable, then the function $\comp$ as defined by \eqref{eq:comp} is measurable.
\ele

\begin{proof}[Proof of \cref{lem:gammaborel}]
By \cref{lem:Pmeas} $\toprod$ is measurable
 and by \cref{lem:vstarcont} $\evalY$ is continuous. Therefore \cref{lem:contplusmeas} implies that $\comp = \evalY \circ \toprod$ is measurable.
 \end{proof}



\subsection{Proofs of \cref{thm:3,thm:11,thm:12,lem:svarwelldefined,lem:uniq}}
\bpf[Proof of \cref{thm:3}]
We need to show $u:\Omega \rightarrow X$ is strongly measurable, satisfies the bound \eqref{eq:sbresult}, and therefore is Bochner integrable and is in the space $\LtOX.$ Our plan is to use \cref{cor:bochnersimple} to show $u$ is Bochner integrable, and establish \eqref{eq:sbresult} as a by-product. Since $u$ solves \cref{prob:meas}, $u$ is measurable. As $X$ is separable, it follows from \cref{cor:pettis} that $u$ is strongly measurable.
Define $N:X \rightarrow \RR$ by
$ N(v) \de \N{v}_X^2.$
Since $N$ is continuous, \cref{lem:contplusmeas} implies $N \circ u:\Omega \rightarrow \RR$ is measurable. 
Therefore, since both the left- and right-hand sides of \eqref{eq:sbe1} are measurable and \eqref{eq:sbe1} holds for almost every $\omega \in \Omega$ we can integrate \eqref{eq:sbe1} over $\Omega$ with respect to $\PP$ and obtain
\beq\label{eq:sbmid}
\int_\Omega \NX{u(\omega)}^2 \dd\PP(\omega) \leq \sum_{j=1}^m \NLoO{\Cj\fj},
\eeq
the right-hand side of which is finite since \cref{con:B} includes that $\Cj\fj \in \LoO$ for all $j = 1,\ldots,m.$ Since $u$ is strongly measurable, the bound \eqref{eq:sbmid} and \cref{cor:bochnersimple} with $p=2$ imply that $u$ is Bochner integrable. The norm $\NLtOX{u}$ is thus well-defined by \cref{def:bochnernorm} and \eqref{eq:sbmid} shows that \eqref{eq:sbresult} holds, and so in particular $\NLtOX{u} < \infty.$
\epf

\bpf[Proof of \cref{lem:svarwelldefined}]
We must show that for any $\vo \in \LtOX$ and any $\vt \in \LtOY$:
\bit
\item The quantities $\big[\Acomega \vo(\omega)\big]\big(\vt(\omega)\big)$ and $\Lcomega\big(\vt(\omega)\big)$ are Bochner integrable, so that the definitions of $\SA$ and $\SL$ as integrals over $\Omega$ make sense.
\item The maps $\SA(\vo)$ and $\SL$ are linear and bounded on $\LtOY,$ that is, $\SA:\LtOX\rightarrow\LtOYas$ and $\SL \in \LtOYas.$
\eit
It follows from these two points that $\SA$ and $\SL$ are well-defined.
Thanks to the groundwork laid in \cref{sec:prelemmanary}, the measurability of $\big[\Acomega \vo(\omega)\big]\big(\vt(\omega)\big)$ and $\Lcomega\big(\vt(\omega)\big)$  follows from \cref{lem:gammaborel,lem:thetaborel} (which need \cref{con:coeffstoform,con:coeffstofunc,con:C}).
Their $\PP$-essential separability follows from \cref{con:coeffstoform,con:coeffstofunc,con:C,lem:esssep} and thus their strong measurability follows from \cref{cor:pettis} on the equivalence of measurability and strong measurability when the image is separable. Their Bochner integrability then follows from the Bochner integrability condition in \cref{thm:bochnercond} (with $V=\FF$) and the Cauchy--Schwartz inequality since
\begin{align}
\int_\Omega \abs{\Lcomega\big(\vt(\omega)\big)}\dd\PP(\omega)&\leq \int_\Omega \NYs{\mleft(\rhs\circ\coeff\mright)(\omega)}\NY{\vt(\omega)}\dd\PP(\omega)\nonumber\\
&\leq \NLtOYs{\rhs\circ\coeff}\NLtOY{\vt},\label{eq:Lfirst}
\end{align}
which is finite by \cref{con:L}, and 
\begin{align}
\int_\Omega \Big|\big[\Acomega \vo(\omega)\big]\big(\vt(\omega)\big)\Big|\dd\PP(\omega) %&\leq \int_\Omega \NYs{\Acomega \vo(\omega)}\NY{\vt(\omega)}\dd\PP(\omega)\nonumber\\
&\leq \esssup_{\omega \in \Omega} \Nhomspace{\Acomega} \int_\Omega \NX{\vo(\omega)}\NY{\vt(\omega)}\dd\PP(\omega)\nonumber\\
&\leq \NLiOhomspace{\toform\circ\coeff}\NLtOX{\vo}\NLtOY{\vt},\label{eq:Afirst}
\end{align}
which is finite by \cref{con:A}.

We now show $\SL\in  \LtOYas$ and $\SA:\LtOX \rightarrow \LtOYas.$ Observe that 

\noindent $\abs{\SL(\vt)} \leq \int_\Omega \abs{\Lcomega\mleft(\vt(\omega)\mright)}\dd\PP(\omega)$ and $\abs{\mleft[\SA \mleft(\vo\mright)\mright](\vt)} \leq \int_\Omega \abs{\mleft[\Acomega \vo(\omega)\mright]\mleft(\vt(\omega)\mright)}\dd\PP(\omega)$ and thus by \eqref{eq:Lfirst} and \eqref{eq:Afirst} $\SL$ and $\SA(\vo)$ are bounded. They are clearly linear, and so it follows that $\SL \in \LtOYas$ and $\SA(\vo)\in \LtOYas,$ i.e., $\SA:\LtOX \rightarrow \LtOYas.$
\epf

\bpf[Proof of \cref{thm:11}]
In order to show that $u$ solves \cref{prob:svar}, we must show:
\ben
\item\label[point]{it:111} either the functional $\SLc \in \LtOYas$ or the functional $\SA(u) \in \LtOYas$, and
\item\label[point]{it:112} the equality \eqref{eq:stoeq} holds.
\een

For \cref{it:111} we show that $\SL \in \LtOYas,$ (since this is easier than showing $\SA(u) \in \LtOYas$); in fact the proof of this is contained in the proof of \cref{lem:svarwelldefined}.

For \cref{it:112}, since $u$ solves \cref{prob:lt}, for $\PP$-almost every $\omega \in \Omega$ we have
%\beqs
$\Acomega u(\omega) = \Lcomega$
%\eeqs
in $\Ys.$ Hence, for any $v \in \LtOY$ we have
\beq\label{eq:midwaytoeu2}
\big[\Acomega u(\omega)\big]\big(v(\omega)\big) = \Lcomega\big(v(\omega)\big)
\eeq
for $\PP$-almost every $\omega \in \Omega.$ Since $\SL \in \LtOYas$, the right-hand side of \eqref{eq:midwaytoeu2} is a strongly measurable function with finite integral. Hence the left-hand side of \eqref{eq:midwaytoeu2} is as well, and we can integrate over $\Omega$ to conclude
%\beqs
$\big[\SAc u\big](v) = \SLc(v) \tforall v \in \LtOY,$
%\eeqs
that is, $\SAc u = \SLc$ in $\LtOYas.$
\epf
The following lemma is needed for the proof of \cref{thm:12}.
\ble\label{lem:settheory}
Let $\diff:\Omega\times Y \rightarrow \FF.$  For $y \in Y,$ define $\Omegay \de \set{\omega \in \Omega \st \diff(\omega,y)=0}$ and define $\Omegat \de \set{\omega \in \Omega \st \diff(\omega,y)=0 \tforall y \in Y}.$ If
\bit
\item for all $\omega \in \Omega,$ $\diff(\omega,\cdot)$ is a continuous functional on $Y$ and
\item for all $y \in Y,$ the map $\diff(\cdot,y):\Omega\rightarrow \FF$ is measurable and $\PP(\Omegay)=1,$
  \eit
  then $\PP(\Omegat)=1.$
\ele

\bpf[Proof of \cref{lem:settheory}]
We must show that the set $\Omegat \in \cF,$ and $\PP(\Omegat)=1.$ Observe that, for any $y \in Y$, the set $\Omegay \in \cF,$ since  $\Omegay = \delta(\cdot,y)^{-1}\mleft(\set{0}\mright),$ which  is the preimage under a measurable map of a measurable set. 

Since $Y$ is a Hilbert space, it is separable, and therefore it has a countable dense subset $\mleft(\yn\mright)_{n \in \NN}.$ We will show that $\PP\mleft(\cap_{n \in \NN} \Omegayn\mright)=1$ and $\Omegat = \cap_{n \in \NN} \Omegayn.$ The set $\cap_{n \in \NN} \Omegayn \in \cF,$ as $\cF$ is a $\sigma$-algebra and $\PP\mleft(\cup_{n \in \NN} \Omegaync\mright) \leq \sum_{n \in \NN} \PP\mleft(\Omegaync\mright) = 0,$ and hence $\PP\mleft(\cap_{n\in\NN} \Omegayn\mright)=1.$ To next show $\Omegat =  \cap_{n \in \NN} \Omegayn$ we observe that $\Omegat = \cap_{y \in Y} \Omegay$ and $\cap_{y \in Y} \Omegay \subseteq \cap_{n \in \NN} \Omegayn.$ It therefore suffices to show $\cap_{n \in \NN} \Omegayn \subseteq \cap_{y \in Y} \Omegay$ to conclude $\Omegat =  \cap_{n \in \NN} \Omegayn.$


Fix $y \in Y.$ By density of $\mleft(\yn\mright)_{n \in \NN}$, there exists a subsequence $\ynmseq$ such that $\ynm \rightarrow y$ as $m \rightarrow \infty.$ Fix $\omega \in \cap_{n \in \NN} \Omegayn.$ Note that $\omega \in \cap_{m \in \NN} \Omegaynm;$ that is, for all $m \in \NN,$ $\diff(\omega,\ynm) =0.$ As $\diff(\omega,\cdot)$ is a continuous function on $Y$, $\diff(\omega,\ynm) \rightarrow \diff(\omega,y)$ as $m \rightarrow \infty.$ But as previously noted, $\diff(\omega,\ynm)=0$ for all $m \in \NN.$ Hence we must have $\diff(\omega,y)=0,$ and thus $\omega \in \Omegay.$ Since $\omega \in \cap_{n \in \NN} \Omegayn$ was arbitrary, it follows that $\cap_{n \in \NN} \Omegayn \subseteq \Omegay,$ and since $y \in Y$ was arbitrary, it follows that $\cap_{n \in \NN} \Omegayn \subseteq \cap_{y \in Y} \Omegay$ as required.
\epf

\bpf[Proof of \cref{thm:12}]
Let $u \in \LtOX$ solve \cref{prob:svar}. We need to show that $u$ solves \cref{prob:lt}. Observe that $u$ solving \cref{prob:lt} means $\Acomega(u(\omega)) = \mleft(\Lcomega\mright)(\omega)$ in $\Ys$ for almost every $\omega \in \Omega.$ We now use an idea from \cite[Theorem 3.3]{Gi:10}. Our plan is to use test functions of the form $y\Ind{E},$ where $y \in Y$ and $E \in \cF$ to reduce \cref{prob:svar} to the statement
\beqs
\int_E \mleft[\Acomega\big(u(\omega)\big)\mright]\big(y(\omega)\big) \dd\PP(\omega) = \int_E \mleft[\mleft(\Lcomega\mright)(\omega)\mright]\big(y(\omega)\big) \dd\PP(\omega)\quad \tforall E \in \cF
\eeqs
and then show this implies $u$ satisfies \cref{prob:lt} via \cref{lem:gotoae}.

First define the space
%\beqs%\label{eq:Dspace}
$\D := \set{y\Ind{E} \st y \in Y, E \in \cF}.$
%\eeqs
It is straightforward to see that the elements of $\D$ are maps from $\Omega$ to $Y.$ The fact that $\D \subseteq \LtOY$ follows via the following three steps:

\ben
\item The elements of $\D$ are measurable, indeed the indicator function of a measurable set is a measurable function $\Omega\rightarrow\RR,$ and multiplication by $y \in Y$ is a continuous function $\RR\rightarrow Y.$ Hence elements of $\D$ are measurable by \cref{lem:contplusmeas}.
\item As $Y$ is a separable Hilbert space, it follows from \cref{cor:pettis} that the elements of $\D$ are strongly measurable.
\item $\NLtOY{y\Ind{E}} = \sqrt{\PP\mleft(E\mright)}\NY{y} < \infty$ for all $y \in Y, E \in \cF.$
  \een

  

Since \cref{prob:svar} is well-defined, and $u$ solves \cref{prob:svar}, and $\D \subseteq \LtOY,$ we have that $\mleft[\SA u\mright](v) = \SL(v) \tforall v \in \D.$ Therefore, we have

\beq\label{eq:initialint}
\int_\Omega \mleft[\Acomega\mleft(u(\omega)\mright)\mright]\mleft(y\Ind{E}(\omega)\mright) \dd\PP(\omega) = \int_\Omega \mleft[\Lcomega\mright]\mleft(y\Ind{E}(\omega)\mright) \dd\PP(\omega)
\eeq
for all $y \in Y$ and $E \in \cF.$ If we define $\diff:\Omega\times Y \rightarrow \FF$ by $\diff(\omega,y) \de \mleft[\Acomega\mleft(u(\omega)\mright) - \Lcomega\mright]\mleft(y\mright)$ then, by the definition of $\Ind{E},$ \eqref{eq:initialint} becomes
\beq\label{eq:intoverE}
\int_E \diff(\omega,y) \dd\PP(\omega)=0\quad \tforall E \in \cF.
\eeq
 To conclude $u$ solves \cref{prob:lt} we must show $\diff(\omega,y)=0$ for all $y \in Y,$ almost surely. We will use \cref{lem:gotoae}, so the first step is to show that for all $y \in Y$ $\diff(\cdot,y)$ is Bochner integrable. This follows from the fact that \cref{prob:svar} is well-defined, and thus the quantities $\big[\Acomega \vo(\omega)\big]\big(\vt(\omega)\big)$ and $\Lcomega\big(\vt(\omega)\big)$ are Bochner integrable for any $\vo\in \LtOX,\vt \in \LtOY.$ In particular, they are Bochner integrable when $\vo=u,$ and $\vt=y\Ind{E}$ and thus their difference $\diff$ is Bochner integrable. Secondly, $\diff(\omega,\cdot)$ is a continuous function on $Y$ since $\Acomega\mleft(u(\omega)\mright)$ and $\mleft(\Lcomega\mright)(\omega) \in \Ys,$ for all $\omega \in \Omega.$

We now show $\diff(\omega,y)=0$ for all $y \in Y,$ almost surely. For $y \in Y$ define the set $\Omegay \de \set{\omega \in \Omega \st \diff(\omega,y)=0};$ by \eqref{eq:intoverE} and \cref{lem:gotoae} we have that $\PP(\Omegay)=1$ for all $y \in Y.$ By \cref{lem:settheory}, $  \diff(\omega,y)=0$ for all $y \in Y$, almost surely, that is, $\Acomega u(\omega) = \Lcomega$ almost surely; it follows that $u$ solves \cref{prob:lt}.
\epf

\bre[Connection with the argument in {\cite[Remark 2.2]{MuSt:11}}]
The argument in 

\noindent \cref{lem:settheory} and the final part of \cref{thm:12} closely mirrors the result in \cite[Remark 2.2]{MuSt:11}. Indeed, we prove in general that
\beqs
\PP\big(\diff(\omega,y)=0\big)=1\text{ for all } y \in Y \quad\text{implies} \quad\PP\big(\diff(\omega,y)=1\text{ for all } y \in Y\big)=1,
\eeqs
and \cite[Remark 2.2]{MuSt:11} shows an analogous result for the stationary diffusion equation \eqref{eq:diffusion} with non-uniformly coercive and unbounded coefficient $\kappa.$
\ere

\bpf[Proof of \cref{lem:uniq}]
\emph{Proof of \cref{part:uniq1}.} Suppose $\uo,\ut:\Omega\rightarrow X$ solve \cref{prob:meas}. Let $E = \set{\omega \in \Omega \st \uo(\omega) \neq \ut(\omega)}.$ Denote by $\Eo$ and $\Et$ the sets (of measure zero) where the variational problems for $\uo$ and $\ut$ fail to hold, i.e. $\Eo,\Et \in \cF$ with $\PP(\Eo)=\PP(\Et)=0$ and %such that  and
\beqs
\Acomega\mleft(\uo(\omega)\mright) \neq \Lcomega \tiff \omega \in \Eo,\quad\text{and}\quad \Acomega\mleft(\ut(\omega)\mright) \neq \Lcomega \tiff \omega \in \Et.
\eeqs As  $\ker\mleft(\Acomega\mright) = \set{0}$ $\PP$-almost surely, there exists $\Eth \in \cF$ such that $\PP(\Eth) = 0$ and
%\beqs

\noindent $\ker\mleft(\Acomega\mright) \neq \set{0} \tiff \omega \in \Eth.$
%\eeqs
We claim $E \subseteq \Eo\cup\Et\cup\Eth.$ Indeed, if $\uo(\omega) \neq \ut(\omega)$ then either: (i) at least one of $\uo$ and $\ut$ does not solve \cref{prob:meas} at $\omega$ or (ii) $\uo$ and $\ut$ both solve \cref{prob:meas} at $\omega,$ but $\ker\mleft(\Acomega\mright) \neq \set{0}.$
Since $\PP(E_j)=0, j = 1,2,3,$ we have $\PP(\Eo\cup\Et\cup\Eth) = 0.$  Therefore $E \in \cF$ and $\PP(E)=0$ since $\OFP$ is a complete probability space; hence $\uo = \ut$ almost surely, as required.

\emph{Proof of \cref{part:uniq2}.} By \cref{rem:imm}, if $\uo,\ut \in \LtOX$ solve \cref{prob:lt}, then all the representatives of the equivalence classes of $\uo$ and $\ut$ solve \cref{prob:meas}. Hence, by \cref{part:uniq1}, any representative of $\uo$ and any representative of $\ut$ differ only on some set (depending on the representatives) of $\PP$-measure zero in $\Omega.$ Therefore $\uo=\ut$ in $\LtOX,$ by definition of $\LtOX.$

\emph{Proof of \cref{part:uniq3}.} As \cref{prob:svar} is well-defined, by \cref{rem:imm,thm:12}, if $\uo$ and $\ut$ solve \cref{prob:svar}, then $\uo$ and $\ut$ also solve \cref{prob:meas}. We then repeat the reasoning in the proof of \cref{part:uniq2} to show $\uo=\ut$ in $\LtOX.$
\epf


\section{Proofs of \cref{thm:hh-gen,thm:hh-hetero}}\label{sec:hhproof}
In \cref{sec:placing} we place the Helmholtz stochastic EDP into the framework developed in \cref{sec:gen-framework}. In \cref{sec:hh-cond} we give sufficient conditions for the Helmholtz stochastic EDP to satisfy \cref{con:coeffstoform,,con:coeffstofunc,,con:cborel}, etc.. In \cref{sec:applying} we apply the general theory developed in \cref{sec:gen-framework} to prove \cref{thm:hh-gen,,thm:hh-hetero}.

\subsection{Placing the Helmholtz stochastic EDP into the framework of \cref{sec:gen-framework}}\label{sec:placing}
Recall $R>0$ is fixed. We let $X=Y=\HozDDR$ and define the norm $\NW{v}^2 \de \NLtDR{\grad v}^2 + k^2 \NLtDR{v}^2$ on $\HozDDR.$ Throughout this section, $\Az,\nz,$ and $\fz$ will be deterministic functions. Recall that since the supports of $1-n,$ $I-A,$ and $f$ are compactly contained in $\BR$, we can consider $A, n,$ and $f$ as functions on $\DR$ rather than on $\Dp$. In order to define the space $\cC$ and the maps $\coeff,\toform,$ and $\rhs$ we define the following function spaces on $\DR$.

\bde[Compact-support spaces]\label{def:compsuppspace}
Let
\beqs
\LtRDR \de \set{\fz \in \LtDR \st \supp\mleft(\fz\mright) \compcont \BR}.
\eeqs
\begin{align*}
\LiRminDRRR \de \big\{&\nz \in \LiDRRR \st\supp\mleft(1-\nz\mright) \compcont \BR,\\
&\text{there exists } \alphanz > 0 \text{ such that } \nz(\bx) \geq \alphanz \text{ almost everywhere }\big\},\\
\hspace{-3cm}\LiRminDRRRdtd \de \Big\{&\Az \in \LiDRRRdtd \st\Az(\bx) \text{ is symmetric almost everywhere,}\\
&\supp\mleft(I-\Az\mright) \compcont \BR, \text{there exists } \alphaAz>0 \text{ s.~t.~}\alphaAz \leq \Az(\bx)\\
&\text{almost everywhere, in the sense of quadratic forms}\Big\},\text{ and}\\
%\end{align*}
%\begin{align*}
&\hspace{-3cm}\WoiRminDRRRdtd \de \set{\Az \in \LiRminDRRRdtd \st\Az \in \WoiDRRRdtd}.
\end{align*}
\ede

Observe that the norm on $\LiDRRR$ induces a metric on $\LiRminDRRR,$ and similarly for $\LiRDRRRdtd,$ $\WoiRminDRRRdtd,$ and $\LtRDR.$ These spaces are not vector spaces, and are not complete, but completeness and being a vector space is not required in what follows---we only need them to be metric spaces.


\bde[Deterministic form and functional]

\noindent For $\mleft(\Az,\nz,\fz\mright) \in \LiRDRRRdtd \times \LiRminDRRR \times \LtRDR$ let the sesquilinear form $\aGm$ on $\HozDDR \times \HozDDR$  and the antilinear functional $\Lh$ on $\HozDDR$ be given by
\begin{align*}
\aGm\mleft(\vo,\vt\mright) &\de \int_{D_R} \Big(\mleft(\Az \grad \vo\mright)\cdot \grad \vtb \rangle 
 - k^2 \nz\, \vo\,\vtb \Big)\dd\Leb- \big\langle T_R \gamma \vo,\gamma \vt\big\rangle_{\Gamma_R}, \quad\text{and}\\
 \Lh(\vt) &\de \int_{D_R} \fz\, \vtb\,\dd\Leb, \quad\text{ for } \vo, \vt \in \HozDDR.
\end{align*}
%for $\vo,\vt \in \HozDDR.$
\ede

\bprob[Helmholtz EDP]\label{prob:edp}
For $\mleft(\Az,\nz,\fz\mright) \in \LiRDRRRdtd \times \LiRDRRR \times \LtRDR$ find 
$\uz \in \HozDDR$ such that $\aGm(\uz,v) = \Lh(v) \tforall v \in \HozDDR.$
\eprob

\bde[$\dinfty$ metric]\label{def:dinfty}
Let $\mleft(\Xo,\done\mright),\ldots,\mleft(\Xm,\dm\mright)$ be metric spaces. The \defn{$\dinfty$ metric} on the Cartesian product $\Xo\times\cdots\times\Xm$ is defined by
\beqs
\dinfty\mleft(\mleft(\xo,\ldots,\xm\mright),\mleft(\yo,\ldots,\ym\mright)\mright) \de \max_{j = 1,\ldots,m} \dmetj\mleft(\xj,\yj\mright).
\eeqs
\ede


\bde[The input space $\cC$]\label{def:cCHh}
We let
%\beqs
$\cC \de \WoiRminDRRRdtd \times \LiRminDRRR \times \LtRDR$
%\eeqs
with topology given by the $\dinfty$ metric.
\ede


\bde[The input map $\coeff$]\label{def:inputmap}
Define $\coeff:\Omega\rightarrow\cC$ by
%\beq\label{eq:hh-c}
$\coeff(\omega) = \mleft(A(\omega),n(\omega),f(\omega)\mright).$
%\eeq
\ede

\bde[The maps $\toform$ and $\rhs$ for the Helmholtz stochastic EDP]
Let
\beq\label{eq:hhform}
\toform\mleft(\mleft(\Az,\nz,\fz\mright)\mright) \de \aGm \quad \text{and} \quad \rhs\mleft(\mleft(\Az,\nz,\fz\mright)\mright) \de \Lh,
\eeq
where the definition of $\toform$ is understood in terms of the equivalence between $\homspace$ and sesquilinear forms on $X \times Y.$
\ede

\subsection{Verifying the Helmholtz stochastic EDP satisfies the general conditions in \cref{sec:gen-framework}}\label{sec:hh-cond}

\ble[{\Cref{con:cborel,con:C}} for Helmholtz stochastic EDP]\label{lem:hh-borelC}
If $A,n,$ and $f$ are strongly measurable, then $\coeff$ defined by  \cref{def:inputmap} satisfies \cref{con:cborel,con:C}.
\ele

\bpf
Since $A,n,$ and $f$ are strongly measurable, by \cref{thm:pettis} they are measurable and $\PP$-essentially separably valued. By \cref{lem:measprod}, it follows that $\coeff$ is measurable, so $\coeff$ satisfies \cref{con:cborel}. By \cref{lem:prodsep}, it follows that $\coeff$ is $\PP$-essentially separably valued, so $\coeff$ satisfies \cref{con:C}.
\epf

\ble[\Cref{con:coeffstoform,con:coeffstofunc} for Helmholtz stochastic EDP]\label{lem:hh-AL}
The maps $\toform$ and $\rhs$ given by \eqref{eq:hhform} satisfy \cref{con:coeffstoform,con:coeffstofunc}.
\ele

\bpf[Proof of \cref{lem:hh-AL}]
We need to show that if $(\Am,\nm,\fm) \rightarrow (\Az,\nz,\fz)$ in $\cC$ then $\toform((\Am,\nm,\fm)) \rightarrow \toform((\Az,\nz,\fz))$ in $\homspace,$ and similarly for $\rhs.$ We have, for $\vo \in X,\vt \in Y,$
\begin{align*}
&\biggabs{\Big[\big[\toform\mleft(\Am,\nm,\fm\mright) - \toform\mleft(\Az,\nz,\fz\mright)\big]\mleft(\vo\mright)\Big]\mleft(\vt\mright)}\\
&\quad\quad\quad\,\,= \abs{\int_{\DR} \Big(\big(\mleft(\Am-\Az\mright)\grad \vo\big)\cdot\grad\vtb- k^2 \mleft(\nm-\nz\mright) \vo\vtb\Big)\dd\Leb}\\
&\quad\quad\quad\,\,\leq\NLiDRRRdtd{\Am-\Az}\NLtDR{\grad \vo}\NLtDR{\grad \vt} 
\\
&\quad\quad\quad\,\,\quad\quad
+k^2\NLiDRRR{\nm-\nz}\NLtDR{\vo}\NLtDR{\vt}\\
&\quad\quad\quad\,\,\leq 2\dinfty((\Am,\nm,\fm),(\Az,\nz,\fz))\Nw{\vo}\Nw{\vt},
\end{align*}
Hence if $(\Am,\nm,\fm) \rightarrow (\Az,\nz,\fz)$ in $\cC,$ then $\toform((\Am,\nm,\fm)) \rightarrow \toform((\Az,\nz,\fz))$ in 

\noindent $\homspace.$ We also have
\beqs
\Bigabs{\big[\rhs\mleft(\mleft(\Am,\nm,\fm\mright),\mright) - \rhs\mleft(\mleft(\Az,\nz,\fz\mright)\mright)\big]\mleft(\vt\mright)} 
 = \abs{\int_{\DR} \mleft(\fm- \fz\mright)\vtb\,\dd\Leb}
 %&\hspace{-2cm}\leq\NLtDR{\fm-\fz}\NLtDR{\vt}
 \leq \NLtDR{\fm-\fz} \frac{\Nw{\vt}}{k}.
\eeqs
Hence if $(\Am,\nm,\fm) \rightarrow (\Az,\nz,\fz)$ in $\cC,$ then $\rhs((\Am,\nm,\fm)) \rightarrow \rhs((\Az,\nz,\fz))$ in $\Ys.$
\epf

\bde[The solution operator $\sol$]
Define $\sol:\cC\rightarrow \HozDDR$ by letting 

\noindent $\sol\mleft(\Az,\nz,\fz\mright) \in \HozDDR$ be the solution of the Helmholtz EDP (\cref{prob:edp}).
\ede

\bth[$\sol$ is well defined]\label{thm:deteu}
For $\mleft(\Az,\nz,\fz\mright) \in \cC$ the solution $\sol\mleft(\mleft(\Az,\nz,\fz\mright)\mright)$ of the Helmholtz EDP (\cref{prob:edp}) exists, is unique, and depends continuously on $\fz.$
\enth

\bpf[Proof of \cref{thm:deteu}] Since $\Real{-\langle \TrR \gamma v,\gamma v\rangle_{\GR}} \geq 0$ for all $v \in \HozDDR$ (see, e.g.~\cite[Theorem 2.6.4]{Ne:01}), $\aGm$ satisfies a G\r{a}rding inequality. Since the inclusion $\HozDDR \hookrightarrow \LtDR$ is compact, Fredholm theory shows that uniqueness implies well-posedness (see, e.g.~\cite[Theorem 2.34]{Mc:00}). Since $A$ is Lipschitz and $n$ is $L^\infty$, uniqueness follows from the unique continuation results in  \cite{JeKe:85,GaLi:87};
%\cite{GaLi:87};
see \cite[Section 2]{GrSa:18} for these results specifically applied to Helmholtz problems.
\epf


\ble[Continuity of solution operator for Helmholtz stochastic EDP]\label{lem:solcont}
For the 

\noindent  Helmholtz stochastic EDP, the solution operator $\sol:\cC\rightarrow\HozDDR$ is continuous.
\ele

\bpf[Sketch Proof of \cref{lem:solcont}]
Let $(\Az,\nz,\fz), (\Ao,\no,\fo) \in \cC,$ with $\sol((\Az,\nz,\fz)) = \uz$ and $\sol((\Ao,\no,\fo)) = \uo.$
Then for any $v \in \HozDDR$ we have, for $j=0,1,$

\beqs\label{eq:S1}
%\noindent$\label{eq:S1}
\mleft[\mleft[\toform((A_j, n_j, f_j))\mright](u_j)\mright](v) = \mleft[\rhs((A_j,n_j,f_j))\mright](v).% \tand
%\mleft[\mleft[\toform((\Ao,\no,\fo))\mright](\uo)\mright](v) = \mleft[\rhs((\Ao,\no,\fo))\mright](v).
\eeqs%$
Continuity of $\sol$ then follows from:

\ben
\item Deriving the Helmholtz equation with coefficients $\Az$ and $\nz$ satisfied by $\ud\de\uz-\uo.$
\item\label[point]{it:contrecall} Recalling that the well-posedness result of \cref{thm:deteu} holds when $\fz \in \LtRDR$ is replaced by a right-hand side in $(\HozDDR)^*$; see, e.g., \cite[Theorem 2.34]{Mc:00}.
\item Applying the result in \cref{it:contrecall} to obtain a bound
%  \beqs
  $\NW{\ud} \leq C(\Az,\nz) \NHozDDRas{F}.$
%  \eeqs

\item Showing $\NHozDDRas{F}$ depends on $\NLtDR{\grad \uo},$ $\NLtDR{\uo},$ $\NLiDRRRdtd{\Ao-\Az},$ $\NLiDRRR{\no-\nz},$ and $\NLtD{\fz-\fo}.$
\item Eliminating the dependence on $\uo$ by writing $\uo = \uz-\ud$ and moving terms in $\ud$ to the left-hand side, to obtain a bound on $\ud$ of the form
  \begin{align*}
&\NLtDR{\grad \ud} + k\NLtDR{\ud} \\
&\qquad\leq \Ctilde\Big(\uz,\Az,\nz,\NLiDRRRdtd{\Ao-\Az},\NLiDRRR{\no-\nz},\NLtDR{\fz-\fo}\Big).
\end{align*}
\item Concluding that $\ud\rightarrow0$ in $\HozDDR$ as $(\Ao,\no,\fo) \rightarrow (\Az,\nz,\fz)$ in $\cC.$
\een\epf

\ble[\Cref{con:K} for the Helmholtz stochastic EDP]\label{lem:hh-K}
The Helmholtz stochastic EDP satisfies \cref{con:K}.
\ele

\bpf[Proof of \cref{lem:hh-K}]
This condition holds immediately from \cref{thm:deteu}.
%The fact that this %uniqueness 
%condition holds is immediate from \cref{thm:deteu}.
\epf

To prove that \cref{con:B} holds for the Helmholtz stochastic EDP, we first state the deterministic analogues of \cref{con:hh-hetero,thm:hh-hetero}.

\bcon[Nontrapping condition for Helmholtz EDP {\cite[Condition 2.4]{GrPeSp:18}}]\label{cond:1}
$d=2,3$, $\Dm$ is star-shaped with respect to the origin, $\Az\in \WoiDRRRdtd$, $\nz\in \WoiDRRR$, and there exist $\consto, \constt>0$ such that,
for almost every $\bx\in \Dp$,
\begin{align}\label{eq:A1}
&\Az(\bx) - (\bx\cdot\nabla)\Az(\bx) \geq \consto, \text{ in the sense of quadratic forms, and }\\
\label{eq:n1}
& \hspace{2cm}\nz(\bx)+ \bx\cdot\nabla \nz(\bx) \geq \constt. %\text{ for almost every }\bx\in \Dp.
\end{align}
\econ

\bth[Well-posedness of the Helmholtz EDP under \cref{cond:1} {\cite[Theorem 2.5]{GrPeSp:18}}]\label{thm:eubedp}
Let $(\Az,\nz,\fz) \in \cC$ and suppose $\Az$ and $\nz$ satisfy \cref{cond:1}. Then the solution of the Helmholtz EDP (\cref{prob:edp}) exists and is unique. Furthermore, given $\kz > 0$ for all $k \geq \kz,$ the solution $\uz$ of the Helmholtz EDP satisfies the bound 
\beq\label{eq:heterobound1}
\consto \NLtDR{\grad \uz}^2 + \constt k^2 \NLtDR{\uz}^2\leq \Co \NLtDR{\fz}^2,\,\, \text{where} \,\, \Co := 4\mleft[\frac{R^2}{\consto} + \frac{1}{\constt}\mleft(R+ \frac{d-1}{2\kz}\mright)^2\mright].
\eeq
\enth

We can now prove \cref{con:B} holds for the Helmholtz stochastic EDP.

\ble[\Cref{con:B} for Helmholtz stochastic EDP]\label{lem:hh-B}
If \cref{con:hh-fAn,con:hh-hetero} hold, then \cref{con:B} holds for the Helmholtz stochastic EDP.
\ele

\bpf[Proof of \cref{lem:hh-B}]
As \cref{con:hh-hetero} holds,  \cref{cond:1} holds for $\PP$-almost every $\omega \in \Omega$ (with $\Az = A(\omega),$ $\nz = n(\omega),$ $\consto = \muo(\omega),$ and $\constt = \mut(\omega)$). Hence, by \cref{thm:eubedp} the bound \eqref{eq:sbe1} holds for all $k \geq \kz$, with $X = \HozDDR, m=1,$
\beqs
\Co(\omega) = \frac4{\min\set{\muo(\omega),\mut(\omega)}}\mleft[\frac{R^2}{\muo(\omega)} + \frac{1}{\mut(\omega)}\mleft(R+ \frac{d-1}{2\kz}\mright)^2\mright],
\eeqs
and $\fo = \NLtDR{f(\omega)}^2.$ It now remains to show that $\Co\,\NLtDR{f}^2 \in \LoO.$ We first show $\Co\,\NLtDR{f}^2$ is measurable and then show that it lies in $\LoO.$ To show measurability, we rewrite $\Co(\omega)$ as 
\beqs
\Co(\omega) =  \max\left\{\frac{2 R^2}{\mu_1^2(\omega)} + \frac{2}{\mu_1(\omega)\mu_2(\omega)}\left(R+ \frac{d-1}{2\kz}\right)^2,\frac{2 R^2}{\mu_1(\omega)\mu_2(\omega)} + \frac{2}{\mu_2^2(\omega)}\left(R+ \frac{d-1}{2\kz}\right)^2\right\}.
\eeqs
The functions $\muo^{-1}$ and $\mut^{-1}$ are measurable by assumption; to conclude $\Co$ is measurable we use the facts (see e.g. \cite[Theorems 19.C, 20.A]{Ha:74}): (i) the square of a measurable function is measurable, and (ii) the product, sum, and maximum of two measurable functions are measurable. Under 
\cref{con:hh-fAn}, the function $f$ lies in the Bochner space $\LtOLtDR.$ Therefore, $f$ is strongly measurable and hence $f$ is measurable  by \cref{thm:pettis}. The map $f\mapsto\NLtDR{f}^2$ is clearly continuous, and therefore $\fo$ is measurable by \cref{lem:contplusmeas}. As the product of two measurable functions is measurable, it follows that $\Co\,\NLtDR{f}^2$ is measurable.

We now show that $\Co\NLtDR{f}^2 \in \LoO.$ The assumptions $1/\muo,1/\mut \in \LtO$ and the Cauchy--Schwarz inequality imply $1/(\muo\mut) \in \LoO.$ Therefore the maps,
\beqs\omega \mapsto \frac{2 R^2}{\mu_1^2(\omega)} + \frac{2}{\mu_1(\omega)\mu_2(\omega)}\left(R+ \frac{d-1}{2\kz}\right)^2 \text{ and } \omega \mapsto \frac{2 R^2}{\mu_1(\omega)\mu_2(\omega)} + \frac{2}{\mu_2^2(\omega)}\left(R+ \frac{d-1}{2\kz}\right)^2
\eeqs
are in $\LoO.$ Since the maximum of two functions in $\LoO$ is also in $\LoO,$ it follows that $\Co \in \LoO.$ \cref{con:hh-fAn} implies that $\NLtDR{f}^2 \in \LoO.$

To conclude $\Co\NLtDR{f}^2 \in \LoO,$ observe that the only dependence of $\Co$ on $\omega$ is through $\muo$ and $\mut.$ As $\muo$ and $\mut$ are assumed independent of $f,$ and measurable functions of independent random variables are independent \cite[p.236]{Lo:77} it follows that $\Co$ and $\NLtDR{f}^2$ are independent, and therefore
\begin{align}\nonumber
&\hspace{-2ex}\NLoO{\Co\NLtDR{f}^2} = \int_\Omega \Co(\omega) \NLtDR{f(\omega)}^2 \dd\PP(\omega) 
\\ &= \mleft(\int_\Omega \Co(\omega) \dd\PP(\omega)\mright)\mleft(\int_\Omega \NLtDR{f(\omega)}^2 \dd\PP(\omega)\mright)= \NLoO{\Co}\NLoO{\NLtDR{f}^2} < \infty.\label{eq:normsplit}
\end{align}
Therefore $\Co\NLtD{f}^2 \in \LoO$ as required. We take the expectation (equivalently, the $L^1$ norm) of \eqref{eq:heterobound1} (with $\Az = A(\omega)$ etc.) and use \eqref{eq:normsplit} to obtain \eqref{eq:Sbound1}.
\epf

\bre[The case when $f,$ $\muo,$ and $\mut$ are not independent]\label{rem:notindep}
\Cref{rem:planewave} shows that for the physically relevant example of scattering by a plane wave, $f,$ $\muo,$ and $\mut$ may not be independent. In this case, if we replace the requirements in \cref{con:hh-hetero} that $f \in \LtOLtD$ and $1/\muo,\,1/\mut \in \LtO$ with the stronger requirements $f \in \LfOLtD$ and $1/\muo,\,1/\mut \in \LfO$, then one can obtain the bound
\beqs
\NLtOHozDDR{\grad u}^2 + k^2\NLtOHozDDR{ u}^2\leq \NLtO{\Co} \NLfOLtDR{f}^2.
\eeqs
Indeed, instead of independence, we use the Cauchy--Schwartz inequality in \eqref{eq:normsplit} to conclude
\beqs
\NLoO{\Co \NLtDR{f}^2}  \leq \NLtO{\Co}\NLtO{\NLtDR{f}^2} = \NLtO{\Co} \NLfOLtDR{f}^2.
\eeqs
\ere

\ble[\Cref{con:L} for Helmholtz stochastic EDP]\label{lem:hh-L}
If $f \in \LtOLtDR$ and $A$ and $n$ are strongly measurable, then \cref{con:L} holds for the Helmholtz stochastic EDP.
\ele

\bpf[Proof of \cref{lem:hh-L}]
Since $A,n,$ and $f$ are strongly measurable, \cref{con:cborel,con:C} hold by \cref{lem:hh-borelC}; i.e., $\coeff$ is both measurable and $\PP$-essentially separably valued. Furthermore, by \cref{thm:pettis} $\coeff$ is strongly measurable. By \cref{lem:hh-AL}, \cref{con:coeffstofunc} holds, so the map $\rhs$ is continuous. Hence, by \cref{lem:contplusstrong}, $\rhs\circ\coeff$ is strongly measurable. We also have that
%\beqs
$\NYs{\mleft(\rhs\circ\coeff\mright)(\omega)} = \NLtDR{f(\omega)}/k,$
%\eeqs
and thus $\rhs\circ\coeff \in \LtOYs$ since $f \in \LtOLtDR,$ i.e.~\cref{con:L} holds.
\epf

\ble[\Cref{con:A} for the Helmholtz stochastic EDP]\label{lem:hh-A}

\noindent If $A \in \LiOLiDRRRdtd,$ $n \in \LiOLiDRRR,$ and $f$ is strongly measurable, then \cref{con:A} holds for the Helmholtz stochastic EDP.
\ele

\bpf[Proof of \cref{lem:hh-A}]
A near-identical argument to that at the beginning of the proof of \cref{lem:hh-L} shows $\toform\circ\coeff$ is strongly measurable. Recall that the Dirichlet-to-Neumann operator $\TrR$ is continuous from $\HhGR$ to $\HmhGR,$ see e.g.~\cite[Theorem 2.6.4]{Ne:01}. Let $\vo \in X, \vt \in Y,$ and observe that the Cauchy--Schwartz inequality and these properties of $\TrR$
%, and the fact that the norm on $\LiDRRRdtd$ is defined as 
%\beqs
%\NLiDRRRdtd{\Az} = \esssup_{\bx \in \DR}\NopCCd{\Az(\bx)}
%\eeqs
imply that there exists $C(k) > 0$ such that 
\begin{align*}
\bigg|\Big[\mleft[\Acomega\mright](\vo)\Big](\vt)\bigg| &= \abs{\int_{\DR} \Big(\mleft(A(\omega) \grad \vo\mright)\cdot\grad \vtb - k^2n(\omega) \vo\vtb \Big)\dd\Leb- \IPGRbig{\DtN \vo}{\vt}}\\
&\hspace{-1cm}\leq \NLiDRRRdtd{A(\omega)}  \NLtDR{\grad \vo}\NLtDR{\grad \vt}\\
&\hspace{-1cm}\quad+ k^2 \NLiDRRR{n(\omega)}\NLtDR{\vo}\NLtDR{\vt}+ C(k) \NHhGR{\gamma\vo}\NHhGR{\gamma\vt},
\end{align*}
where we have used the fact that the two norms
\beq\label{eq:normsdef}
 \esssup_{\bx \in \DR} \NopCCd{A(\omega,\bx)} \quad\tand\quad\NLiDRRRdtd{A(\omega)} \de \max_{i,j \in \set{1,\ldots,d}} \NLiDRRR{A_{i,j}(\omega)}
\eeq
are equivalent.
Since the trace operator $\gamma$ is continuous from $\HoDR$ to $\HhGR$ (see, e.g.~\cite[Theorem 3.38]{Mc:00}), there exists $\Ctilde > 0$ such that
\beqs
\Nhomspace{\mleft(\toform\circ\coeff\mright)(\omega)}\leq\Ctilde\max\set{\NLiDRRRdtd{A(\omega)},\NLiDRRR{n(\omega)},C(k)}\Nw{\vo}\Nw{\vt}.
\eeqs
and hence $\toform\circ\coeff \in \LiOhomspace.$
\epf

\subsection{Proofs of \cref{thm:hh-gen,thm:hh-hetero}}\label{sec:applying}

\bpf[Proof of \cref{thm:hh-gen}]
We construct a solution of \cref{prob:msedp} by letting $u = \sol\circ\coeff$ (which is well-defined by \cref{thm:deteu}), and observe that, by construction, $\mleft[a(\omega)\mright]\mleft(u(\omega),v\mright) = \mleft[L(\omega)\mright](v)$ for all $v \in \HozDDR$ almost surely. It follows that $u$ is measurable by \cref{con:hh-fAn,lem:solcont,lem:solcont,lem:contplusmeas}, and so $u$ solves \cref{prob:msedp}. We therefore proceed to apply the general theory.

\Cref{con:coeffstoform,con:coeffstofunc} hold by \cref{lem:hh-AL};
\cref{con:A} holds by \cref{lem:hh-A};
\cref{con:L} holds by \cref{lem:hh-L};
 \cref{con:cborel,con:C} hold by \cref{lem:hh-borelC,con:hh-fAn};
and \cref{con:K} holds by \cref{lem:hh-K}. Therefore we can apply \cref{thm:11,thm:12,lem:svarwelldefined,lem:uniq} to conclude the results.
\epf

\bpf[Proof of \cref{thm:hh-hetero}]
All the conclusions of \cref{thm:hh-gen} hold, and we only need to show that if $u$ solves \cref{prob:msedp} then it also solves \cref{prob:somsedp}. \Cref{con:B} holds by \cref{con:hh-fAn,con:hh-hetero,lem:hh-B}. The result then follows from \cref{thm:3}.
\epf




\chapter{Nearby Preconditioning for the Helmholtz Equation}\label{chap:nbpc}
\section{Introduction and Motivation from UQ}\label{sec:intronbpc}

\subsection{Statement of the problem}\label{sec:problem}
Let $\Aj, \nj$, $j=1,2$ satisfy the properties of $A$ and $n$ in \cref{prob:edp}, and let $\uj$ be the corresponding solution of \cref{prob:edp}, with $\Dm,$ $f$, etc. as in \cref{prob:edp}.

Let $\Amatj$, $j=1,2,$ be the Galerkin matrices corresponding to $h$-finite-element discretisations of \cref{eq:hhedp} truncated to $\DR:= \Dp \cap \BR$, with the radiation condition realised exactly on $\GR:=\partial \BR$ via the exact Dirichlet-to-Neumann map (see \cref{eq:matrixAjdef} below for a precise definition of $\Amatj$). Our results also hold in the case when the radiation condition is approximated by an impedance boundary condition, and the truncation boundary is not necessarily $\GR$; see \cref{sec:TEDP}.
 
This \lcnamecref{chap:nbpc} answers the following question:

\bit
\item[Q1.]\label[itemblank]{it:nbpcq1} How small must $\NLi{\Aso - \Ast}$ and 
$\NLi{\nso - \nst}$ be (in some norm, in terms of $k$-dependence) for GMRES 
applied to $(\Amat^{(1)})^{-1}\Amat^{(2)}$ to converge in a $k$-independent number of iterations
%$ to be a good preconditioner for $\Amat^{(2)}$
 for arbitrarily large $k$? 
\eit

\subsection{Motivation from uncertainty quantification of the Helmholtz equation} 
Consider the Helmholtz equation 
\beqs%\label{eq:pde}
\nabla\cdot\big(A(\omega,\bx) \nabla u(\omega,\bx) \big) + k^2 n(\omega,\bx) u(\omega,\bx) =-f(\bx), \quad \bx\in\Dp,
\eeqs
where $A(,\bx\omega)$ and $n(\omega,\bx)$ are \emph{random fields}, and $\omega$ is an element of the underlying probability space.
To calculate quantities of interest of the solution $u(\omega,\cdot)$, one must solve many deterministic Helmholtz problems, with each one corresponding to different realisations of the coefficients $A(\omega,\cdot)$ and $n(\omega,\cdot)$.
Solving all these deterministic problems is a very computationally-intensive task because linear systems arising from discretisations of the Helmholtz equation are notoriously difficult to solve; see the discussion in \cref{sec:numsolve} above.
\optodo{Insert comments about preconditioners for Helmholtz}
\optodo{Insert comments about complexity of preconditioners}
\optodo{Insert comments about why we restrict to LU}
%The search for good preconditioners for discretisations of the H
%Therefore, solving 

%Our motivation for tackling this question comes from calculating quantities of interest for the Helmholtz equation with \emph{random} coefficients $A$ and $n$. Such a calculation requires the solution of many deterministic Helmholtz problems, each with different $A$ and $n$, and

Suppose one calculates the LU factorisation of the Galerkin matrix for one realisation of the coefficients $A$ and $n$. The answer to \cref{it:nbpcq1} above makes $k$-explicit the extent to which this factorisation can be used as an effective preconditioner for Galerkin matrices arising from different realisations of $A$ and $n$.

\subsection{Statement of the main results}\label{sec:main}
The main results about Q1, \cref{thm:1} and \cref{cor:1,cor:1a} below, are proved under two conditions. These are defined precisely in \cref{cond:1nbpc,cond:2} below, but they can be informally stated as 
\bit
\item the obstacle $\Dm$ and the coefficients $\Aso$ and $\nso$ are such that $u^{(1)}$ exists and the problem is \emph{nontrapping} (i.e.~all the rays of geometric optics starting in $\DR$ leave $\DR$ after some uniform time), and
\item the meshsize $h$ and polynomial degree $p$ in the finite-element method are chosen to depend on $k$ to ensure that the 
finite-element solution to the problem with coefficients $\Aso$ and $\nso$ exists, is unique, and 
%Galerkin method (with coefficients $\Aso$ and $\no$) 
is uniformly accurate as $k\tendi$. 
\eit 

We prove results \emph{both} in the Euclidean norm on matrices, denoted by $\|\cdot\|_2$ (induced by the Euclidean norm on vectors), \emph{and} in the weighted norms $\NDmatk{\cdot}$ and $\NDmatkI{\cdot}$ defined by 
\beq\label{eq:Dk}
\NDmatk{\bv}^2:= \big( \Dmatk \bv, \bv\big)_2 = %\big( (\Smat_I + k^2 \Mmat_1)\bv,\bv\big)_2 
\N{v_h}^2_{\HokDR}
\quad \tand
\quad \NDmatkI{\bv}^2:= \big( \Dmatk^{-1} \bv, \bv\big)_2 %= %\big( (\Smat_I + k^2 \Mmat_1)\bv,\bv\big)_2 
%\N{v_h}^2_{\HokDR}
\eeq
where $\Dmatk$ is given in terms of familiar finite-element stiffness- and mass-matrices by \cref{eq:Dk2} below and $\vh =\sum_i \vi \phii$.

The PDE analysis of the Helmholtz equation naturally takes place in the norm $\NHokDR{\cdot}$, and \cref{eq:Dk} shows that the norm $\NDmatk{\cdot}$ is
the discrete analogue of the norm $\NHokDR{\cdot}$. %See \cref{eq:Dk2} and \cref{eq:Dk3} below for this norm expressed in terms of the Euc
The norms $\NDmatk{\cdot}$ and $\NDmatkI{\cdot}$ recently appeared in results about the convergence of domain-decomposition methods %in this norm are proved 
for the Helmholtz equation \cite{GrSpVa:17}, \cite{GrSpZo:18}, and for the time-harmonic Maxwell equations \cite{BoDoGrSpTo:19}. 

Our main results about Q1 are the following.

\begin{theorem}[Main ingredient to answer to Q1]\label{thm:1}
Assume that $\Dm$, $\Aso$, and $\nso$ satisfy \cref{cond:1nbpc}, and assume that $h$ and $p$ satisfy \cref{cond:2}. 
Let the $k$- and $h$-independent constants $\mpm$ and $\spm$ be given as in \cref{lem:normequiv} below.
Then, given $\kz>0$, there exist $\Co, \Ct>0$, independent of $h$ and $k$ (but dependent on $\Dm, \Aso, \nso$, $p$, and $\kz$) such that
\begin{align}\nonumber
&\max\Big\{
\NDmatk{\Imat - (\Amat^{(1)})^{-1}\Amat^{(2)}}, 
\N{\Imat -\Amat^{(2)} (\Amat^{(1)})^{-1}}_{(\Dmat_k)^{-1}}
\Big\}\\
&\hspace{3cm} 
\leq C_1 \,k \,
\big\|\Aso-\Ast\big\|_{L^\infty(\DR)} + C_2 \, k \, \big\|\nso-\nst\big\|_{L^\infty(\DR)}
\label{eq:main1}
\end{align}
and 
\begin{align}\nonumber
&\max\Big\{
\N{\Imat - (\Amat^{(1)})^{-1}\Amat^{(2)}}_2, 
\N{\Imat -\Amat^{(2)} (\Amat^{(1)})^{-1}}_2
\Big\}\\
&\hspace{0cm} 
\leq C_1 \,\left(\frac{s_+}{m_-}\right) \,\frac{1}{h} \,
\big\|\Aso-\Ast\big\|_{L^\infty(\DR)} + C_2 \, \left(\frac{m_+}{m_-} \right)k \, \big\|\nso-\nst\big\|_{L^\infty(\DR)}
\label{eq:main1a}
\end{align}
for all $k\geq k_0$. 
\end{theorem}

Two notes regarding \cref{thm:1}: (i) the constants $C_1$ and $C_2$ are expressed explicitly in \cref{eq:C1nbpc} and \cref{eq:C2} below in terms of constants appearing in \cref{cond:1nbpc,cond:2}, and (ii) the $L^\infty(\DR)$ norm on a matrix-valued functions appearing on the right-hand sides of \cref{eq:main1} and \cref{eq:main1a} is defined by
\beqs
\N{A}_{L^\infty(\DR)}:= \esssup_{\bx\in\DR}\N{A(\bx)}_2.
\eeqs

By using the field-of-values-based theory for the convergence of GMRES introduced in \cite{El:82}, \cite{EiElSc:83} (the so-called `Elman estimate'), we have the following corollaries.

\begin{corollary}[Answer to Q1: $k$-independent weighted GMRES iterations]\label{cor:1}
Assume that $\Dm$, $\Aso$, and $\nso$ satisfy \cref{cond:1nbpc}, and $h$ and $p$ satisfy \cref{cond:2}. Given $k_0>0$,
let $C_1$ and $C_2$ be as in \cref{thm:1}. Then if 
% there exists $C_2>0$, independent of $h$ and $k$ (but dependent on $\Dm, \Aso, \nso$, $p$, and $k_0$) and given explicitly in \cref{eq:C2} below,
% such that if 
\beq\label{eq:cond}
C_1 \,k \,\big\|\Aso-\Ast\big\|_{L^\infty(\DR)} +C_2 \, k\, \big\|\nso-\nst\big\|_{L^\infty(\DR)}
\leq \frac{1}{2}
\eeq
for all $k\geq k_0$, then \emph{both} weighted GMRES working in $\|\cdot\|_{\Dmat_k}$ (and the associated inner product) applied to 
\beq\label{eq:pcsystem1}
(\Amat^{(1)})^{-1}\Amat^{(2)}\bu = \bff
\eeq
\emph{and} weighted GMRES working in $\|\cdot\|_{(\Dmat_k)^{-1}}$ (and the associated inner product) applied to 
\beq\label{eq:pcsystem2}
\Amat^{(2)}(\Amat^{(1)})^{-1}\bv = \bff
\eeq
 converge in a $k$-independent number of iterations for all $k\geq k_0$.
\end{corollary}

\begin{corollary}[Answer to Q1: $k$-independent (unweighted) GMRES iterations]\label{cor:1a}
Assume that $\Dm$, $\Aso$, and $\nso$ satisfy \cref{cond:1nbpc}, and $h$ and $p$ satisfy \cref{cond:2}. Given $k_0>0$,
let $C_1$ and $C_2$ be as in \cref{thm:1}, and let $s_{\pm}$ and $m_{\pm}$ be as in \cref{lem:normequiv} (note that all these constants are independent of $k$, $h$, and $p$). Then if 
% there exists $C_2>0$, independent of $h$ and $k$ (but dependent on $\Dm, \Aso, \nso$, $p$, and $k_0$) and given explicitly in \cref{eq:C2} below,
% such that if 
\beq\label{eq:conda}
 C_1 \,\left(\frac{s_+}{m_-}\right) \,\frac{1}{h} \,
\big\|\Aso-\Ast\big\|_{L^\infty(\DR)} + C_2 \, \left(\frac{m_+}{m_-} \right)k \, \big\|\nso-\nst\big\|_{L^\infty(\DR)}
\leq \frac{1}{2}
\eeq
for all $k\geq k_0$, then standard GMRES (working in the Euclidean norm and inner product) applied to either of the equations \cref{eq:pcsystem1} or \cref{eq:pcsystem2}
%\beqs
%(\Amat^{(1)})^{-1}\Amat^{(2)}\bu = \bff\quad\text{ or } \quad\Amat^{(2)}(\Amat^{(1)})^{-1}\bv = \bff
%\eeqs
 converges in a $k$-independent number of iterations for all $k\geq k_0$.
\end{corollary}



The factor $1/2$ on the right-hand sides of \cref{eq:cond} and \cref{eq:conda} can be replaced by any number $<1$ and the result still holds, although the number of GMRES iterations is then larger -- but still independent of $k$.
%In \cref{sec:proofFEM}, the constant $C_2$ is expressed explicitly in terms of $C_1$.

\bre
When $h\sim  k^{-1}$, the bounds \cref{eq:main1} and \cref{eq:main1a} (and hence also \cref{eq:cond} and \cref{eq:conda}) are identical in their $k$-dependence; however, when $h\ll k^{-1}$ (as one needs to take to overcome the pollution effect, as discussed in \cref{sec:helmfedisc}) the bound \cref{eq:main1a} for standard GMRES is more pessimistic than the bound \cref{eq:main1} for weighted GMRES.
\ere


\paragraph{How sharp are \cref{thm:1} and \cref{cor:1,cor:1a} in their $k$-dependence?}
Numerical experiments in \cref{sec:num} indicate that the condition \cref{eq:cond} is sharp, in that the $k$ in \cref{eq:cond} cannot be replaced by $k^\alpha$ for $\alpha<1$ \ednote{put some plots in here}. This suspected sharpness of \cref{eq:cond} is also supported by the PDE-result \cref{thm:2} below. Indeed, \cref{thm:2} % and \cref{lem:1} 
 shows that the condition
\beqs
k\,
\big\|\Aso-\Ast\big\|_{L^\infty(\DR)} \quad\text{ and } \quad k\,\big\|\nso-\nst\big\|_{L^\infty(\DR)}
%\Big) 
\quad\text{ both sufficiently small}
\eeqs
is not only an answer to Q1 (about finite-element discretisations), but is also the natural answer to the analogue of Q1 at the level of PDEs, namely 
\bit
\item[Q2.]
How small must $\|\Aso - \Ast\|_{L^\infty}$ and 
$\|\nso - \nst\|_{L^\infty}$ be (in terms of $k$-dependence) for the relative error in approximating 
%$u^{(1)}$ to be a good approximation to 
$u^{(2)}$ by $u^{(1)}$ to be bounded independently of $k$ for arbitrarily-large $k$? 
\eit
\cref{lem:sharp} then shows that the condition ``$k\|\nso - \nst\|_{L^\infty}$ sufficiently small" is the \emph{provably-sharp} answer to Q2 when $\Aso= \Ast= I$.

%Before stating these PDE results, we define the weighted $H^1$ norm
%\beq\label{eq:1knorm}
%\N{v}^2_{\HokDR} := \N{\grad v}^2_{L^2(\DR)} + k^2 \N{v}^2_{L^2(\DR)} \quad \tfor v \in H^1_{0,D}(\DR),
%\eeq
%where the space $H^1_{0,D}(\DR)$, defined by \cref{eq:spaceEDP} below, is the natural space containing the solution of the exterior Dirichlet problem. 
To state these PDE results, we use the notation for $a,b>0$ that $a\lesssim b$ when $a\leq C b$ for some $C>0$, independent of $k$, and $a\sim b$ if $a\lesssim b$ and $b\lesssim a$.


%The sharpness of \cref{eq:cond} and \cref{eq:main1} is also supported by the answer to the analogue of Q1 at the level of PDEs. Indeed, the following \cref{thm:2} is the analogue of \cref{thm:1} 

\begin{theorem}[Answer to Q2 (the PDE analogue of Q1)]\label{thm:2}
%Given $f\in L^2(\DR)$ such that $\supp \, f \subset \BR$, 
Let $\Dm$, $\Aso$, and $\nso$ satisfy \cref{cond:1nbpc}, and let $\Dm$, $\Ast$, and $\nst$ be such that $u^{(2)}$ exists
for any $f\in L^2(\DR)$ such that $\supp \, f \subset \BR$. 
Then, given $k_0>0$, there exists $C_3>0$, independent of $k$ and given explicitly in terms of $\Dm$, $\Aso$, and $\nso$ in \cref{eq:C3} below, such that
\beq\label{eq:PDEbound}
\frac{\big\|u^{(1)}-u^{(2)}\big\|_{\HokDR}
}{
\N{u^{(2)}}_{\HokDR}
}\leq C_3 \,k\, \max\Big\{\big\|\Aso-\Ast\big\|_{L^\infty(\DR)}\,,\, \big\|\nso-\nst\big\|_{L^\infty(\DR)}\Big\}%\right)
\eeq
for all $k\geq k_0$. 
\end{theorem}

\ble[Sharpness of the bound \cref{eq:PDEbound} when $\Aso = \Ast= I$]\label{lem:sharp}

\

\noi 
There exist $f, \,\nso$, and $\nst$ (with $\nso\neq \nst$) such that 
the corresponding solutions $u^{(1)}$ and $u^{(2)}$ of the exterior Dirichlet problem with $\Aso = \Ast= I$ exist, are unique, and satisfy
\beq\label{eq:sharp1}
\frac{\N{u^{(1)}-u^{(2)}}_{\HokDR}
}{
\N{u^{(2)}}_{\HokDR}
}
\sim 
\frac{\N{u^{(1)}-u^{(2)}}_{L^2(\DR)}
}{
\N{u^{(2)}}_{L^2(\DR)}
}\sim k \big\|\nso-\nst\big\|_{L^\infty(\DR)}.
\eeq
%\noi (ii) There exist $f, \Aso, \Ast$, (with $\Aso\not\equiv \Ast$), such that 
%the corresponding solutions $u^{(1)}$ and $u^{(2)}$ of the exterior Dirichlet problem with $\nso \equiv \nst\equiv 1$ exist, are unique, and satisfy
%%There exist $f\in L^2(\DR), \Aj \in C^{0,1}(\DR)$, $j=1,2$ (with $\Aso\not\equiv \Ast$), such that the corresponding solutions $u^{(1)}$ and $u^{(2)}$ of the exterior Dirichlet problem with $\nso \equiv \nst\equiv 1$ satisfy
%\beq\label{eq:sharp2}
%\frac{\N{u^{(1)}-u^{(2)}}_{\HokDR}
%}{
%\N{u^{(2)}}_{\HokDR}
%}
%\sim 
%\frac{\N{u^{(1)}-u^{(2)}}_{L^2(\DR)}
%}{
%\N{u^{(2)}}_{L^2(\DR)}
%}\sim k \big\|\Aso-\Ast\big\|_{L^\infty(\DR)}.
%\eeq
\ele


\section{Numerical experiments}\label{sec:num}
\optodo{Put $L^\infty$ stuff in here}
\subsection{Investigating the sharpness of \cref{thm:1} and \cref{cor:1}}

\Cref{thm:1,cor:1} were stated for the exterior Dirichlet problem, but, as highlighted in \cref{sec:problem} and shown in \cref{sec:TEDP}, they hold also for the truncated exterior Dirichlet problem (TEDP). Recall that in this problem the Sommerfeld radiation condition \cref{eq:src} is approximated by an impedance boundary condition (or, equivalently, when the DtN operator $T_R$ is approximated by $\ri k$); see \cref{prob:tedp}. The numerical experiments in this section seek to verify the analogues of \cref{thm:1} and \cref{cor:1} for the TEDP, and investigate their sharpness. More specifically, the experiments seek to verify whether the condition \cref{eq:cond} is:
\ben
\item sufficient, and
\item necessary
  \een
  for \emph{standard} GMRES applied to \cref{eq:pcsystem} to converge in a number of iterations that is independent of $k.$

  Recall from the discussion around \cref{thm:2,lem:sharp} above that we expect, based on the PDE results \cref{thm:2,lem:sharp} that the condition `$k\NLiD{\Aso-\Ast} + k\NLiD{\nso-\nst}$ is sufficiently small' is a necessary and sufficient condition for standard GMRES applied to \cref{eq:pcsystem} to converge in a $k$-independent number of iterations, even thought we can only prove this is a sufficient condition for \emph{weighted} GMRES.

  To verify these expectations, we perform numerical experiments for the Interior Impedance Problem (i.e. \cref{prob:tedp} in 2-d with $\Dm = \emptyset$) with $D$ being the unit square, $\Aso = I$, $\nso = 1$, and $f$ and $\gI$ corresponding to a plane wave incident from the bottom left passing through a homogeneous medium given by coefficients $\Aso$ and $\nso$ as above. We discretise with first-order finite-elements on a uniform grid with $h \sim k^{-3/2}.$

  We perform experiments for $A$ and $n$ separately, i.e., we perform experiments with $\Ast=I$ and $\nst$ varying, and then experiments with $\Ast$ varying and $\nst=1.$ We define $\Ast$ and $\nst$ to be piecewise constant on a $10\times10$ square grid, with their values on each square chosen independently at random, with the restriction that $\nst > 0$ almost surely and $\Ast$ is (piecewise) positive-definite almost surely. We control $\NLiD{\Aso-\Ast}$ and $\NLiD{\nso-\nst}$ as detailed below, and solve the linear systems \cref{eq:pcsystem} for $k = 20,40,60,80,100$ using standard GMRES. We record the number of GMRES iterations taken to achieve a tolerance of\optodo{Put in later}. We perform experiments for three different values of $\alpha\approx\NLiD{\Aso-\Ast}$ (and $\NLiD{\nso-\nst}$), taking $\alpha = 0.1, 0.1/k^{1/2}, 0.1/k.$ We expect that when $\alpha = 0.1$ or $\alpha = 0.1/k^{1/2},$ the number of GMRES iterations required for convergence will increase as $k$ increases, whereas we expect that when $\alpha = 0.1/k$ the number of GMRES iterations required for convergence will remain bounded as $k$ increases, even though this behaviour has only been proven for $\LiD{\Aso-\Ast}$ for weighted GMRES.\optodo{Need to mention exact $LU$ somewhere}

  The computations were carried out on a single node of the Balena High Performance Computing (HPC) Service at the University of Bath. The computations were carried out using the Firedrake software\optodo{Citations}, which in uses the PETSc\optodo{cite} and Chaco\optodo{cite} software. PETSc utilised the MUMPS\optodo{cite} solver to perform the $LU$ factorisations.

  For both $\NLiD{\Aso-\Ast}$ and $\NLiD{\nso-\nst}$ we see the expected behaviour in all three cases. When $\alpha = 0.1$ we see (possibly exponential) growth in the number of GMRES iterations needed to achieve convergence, and when $\alpha = 0.1/k,$ we see that the number of GMRES iterations is bounded as $k$ increases. When $\alpha = 0.1/k^{1/2}$ we see slower growth as $k$ increases, but growth nonetheless.

  The behaviour for $\alpha = 0.1$ indicates that the effectiveness of this `nearby preconditioning' strategy deteriorates rapdily as $k$ increases when the difference in the Helmholtz problems is independent of $k$. The behaviour for $\alpha = 0.1/k$ verifies that the condition `$k\NLiD{\Aso-\Ast} + k\NLiD{\nso-\nst}$ is sufficiently small' is sufficient to achieve bounded GMRES iterations in this case, and the behaviour for $\alpha = 0.1/k^{1/2}$ indicates that this condition is sharp; reducing the magnitude of the difference betwen the Helmholtz problems as $k$ increases, but not at the rate $1/k$ is insufficient for the nearby preconditioning strategy to achieve $k$-independent GMRES iterations.
  \optodo{Insert both tables and figures}


%Say that these are for the TEDP defined in \cref{def:TEDP}.

\section{Definitions and conditions}\label{sec:3}

\subsection{The variational problem and the Galerkin method}\label{sec:vpGm}
As this \lcnamecref{chap:nbpc} concerns finite-element discretisations of the Helmholtz equation, we will work with the variation formulation of \cref{prob:edp}, \cref{prob:vedp} above.

\bre[The EDP with data in $(\HokDR)'$]
In \cref{prob:edp} we defined the EDP with the antilinear functional $\LE$ arising from an $f\in \LtDR$. In the rest of the \lcnamecref{chap:nbpc}, 
%\item In the rest of the paper, we usually consider the EDP with data given by $F$ defined in \cref{eq:EDPvar}, but sometimes 
we sometimes consider the EDP with general $\LE\in \HozDDRp$ and we indicate when this is the case.
In this latter situation, we define the dual norm by
\beq\label{eq:dualnorm}
\NHokDRp{L}= \sup_{v\in \HozDDR} \frac{\abs{\LE(v)}}{\NHokDR{v}}.
\eeq
%where $\|\cdot\|_{\HokDR}$ is defined by \cref{eq:1knorm}.
\ere


For the remainder of this \lcnamecref{chap:nbpc}, we let $(\Vhp)_{h>0}$ be a family of finite-dimensional, nested subspaces of $\HozDDR$, whose union is dense in $\HozDDR$. Furthermore let $\Vhp$ consist of piecewise-polynomials on a simplicial mesh with mesh-size $h$
%\ednote{Euan says: have problem that want to allow $C^{1,1}$ $\Dm$, so that statements later about $H^2$ regularity are covered, but easiest to define triangulation and hence subspaces on Lipschitz domains -- Euan to discuss with Ivan}
and fixed polynomial degree $p$. (Note that the dimension $N$ of $\Vhp$ then satisfies $N\sim h^{-d}$.) We define the analogue of the finite-element approximation \cref{prob:fevtedp} for \cref{prob:vedp} with general data $L \in \HokDRp.$

\bprob[Finite-element approximation of \cref{prob:vedp} with general data]\label{prob:fevedpgen}
We say that $\uh \in \Vhp$ is the \defn{finite-element approximation of $u$} (the solution to \cref{prob:vedp} with general right-hand side $L \in \HokDRp$) if
\beq\label{eq:galerkin}
\aE(\uh,\vh) = \LE(\vh) \tforall \vh \in \Vhp.
\eeq
\eprob

%Definition of Galerkin method

We now define the matrices associated with our finite-element discretisation. Let $\{\phi_i, i= 1, \ldots, N\}$ be a basis for $\Vhp$ with each $\phi_i$ \emph{real-valued}.
Let 
\beq\label{eq:matrixSjdef}
\big(\Smat_{A}\big)_{ij}:= \int_\Omega \big(A \nabla \phi_j)\cdot\nabla \phi_i, \quad
\big(\Mmat_{n}\big)_{ij}:= \int_\Omega n\,\phi_i\, \phi_j,
\quad\tand\quad
\big(\Nmat\big)_{ij}:= \int_{\GR} T_R (\gamma\phi_j) \,\gamma \phi_i.
\eeq
Note that both $\Smat_A$ and $\Mmat_n$ are \emph{real-valued}, but $\Nmat$ is \emph{complex-valued} (because the DtN operator $T_R$ is complex-valued).
Let
\beq\label{eq:matrixAdef}
\Amat := \Smat_{A} - k^2 \Mmat_{n} - \Nmat,
\eeq
and let $u_h:= \sum_j u_j \phi_j$. Then \cref{eq:galerkin} implies that
\beqs
\Amat \bu = \bff,
\eeqs
where $(\bff)_i := \FE(\phi_i)$.
Similar to above we let 
\beq\label{eq:matrixAjdef}
\Amatj := \Smat_{A^{(j)}} - k^2 \Mmat_{n^{(j)}} - \Nmat.
\eeq
Finally, let 
\beq\label{eq:Dk2}
\Dmat_k:= \Smat_I + k^2 \Mmat_1;
\eeq
so that \cref{eq:Dk} holds.
%then the weighted norm $\|\cdot\|_{\Dmat_k}$ is given by 
%\beq\label{eq:Dk3}
%\N{\bv}_{\Dmat_k}^2:=   \N{v_h}^2_{\HokDR}=\big( \Dmat_k \bv,\bv\big)_2,
%\eeq
%for
%$v_h =\sum_i v_i \phi_i$.

We recall the definition of quasi-uniformity (c.f. \cite[Definition 4.4.13]{BrSc:08}):

\bde[Quasi-uniform]\label{def:quasiuniform}
Let $\set{\cTh}_h$ be a set of triangulations of $\DR$ indexed by their mesh size $h.$ For each $T \in \cTh,$ let $\BT$ denote the largest ball contained in $T$.
If there exists $\rho > 0$ such that 
\beqs
\min\set{\diam \BT \st T \in \cT} \geq \rho h,
\eeqs
then $\set{\cTh}_h$ is said to be \defn{quasi-uniform}.
\ede


\ble[Norm equivalences of FE functions]\label{lem:normequiv}
\ednote{Euan says: need to say using a nodal basis here} If $\Vhp$ is quasi-uniform, then
there exist $m_\pm$ and $s_\pm$, independent of $h$ and $p$, such that
\beq\label{eq:normequiv1}
m_- h^{d/2} \N{\bv}_2 \leq \N{v_h}_{\LtDR} \leq m_+ h^{d/2} \N{\bv}_2,
\eeq
and
\beq\label{eq:normequiv2}
s_- h^{d/2} \N{\bv}_2 \leq \N{\nabla v_h}_{\LtDR} \leq s_+ h^{d/2-1} \N{\bv}_2,
\eeq
for all finite-element functions $v_h =\sum_i v_i \phi_i \in \Vhp$.
\ele

Written in terms of the matrices $\Mmat_1$ and $\Smat_I$ defined in \cref{eq:matrixSjdef}, the bounds \cref{eq:normequiv1} and \cref{eq:normequiv2} are, respectively, the familiar bounds
\beqs
(\Mmat_1 \bv,\bv)_2 \sim h^d \N{\bv}^2_2 \quad\tand\quad h^{d}\N{\bv}^2_2 \lesssim (\Smat_I \bv,\bv)_2 \lesssim h^{d-2} \N{\bv}^2_2.
\eeqs

\bpf[Sketch proof of \cref{lem:normequiv}]\ednote{can omit this if can find a good reference. One possibility \cite[Chapter V, Lemma 2.6]{Br:97}.}
The inequalities in \cref{eq:normequiv1} follow from writing $\|v_h\|_{\LtDR}$ as a sum of integrals over elements of the mesh, and then mapping to the reference element \ednote{Euan to discuss with Ivan}.
%\beqs
%\N{v_h}^2_{L^2(\O
%\eeqs
Then, \cref{eq:normequiv2} follows from \cref{eq:normequiv1} and the inequalities
\beqs
\N{v_h}_{L^2(\DR)}\lesssim \N{\nabla v_h}_{L^2(\DR)}\lesssim \frac{1}{h} \N{v_h}_{L^2(\DR)},
\eeqs
the first of which follows from the Poincar\'e inequality, since $v_h \in \HozDDR$
(see, e.g., \cite[Proposition 5.3.4]{BrSc:00}), the second of which follows from a standard inverse estimate (see, e.g., \cite[Theorem 4.5.11]{BrSc:00}).
\epf


Finally, we need the concept of the \emph{adjoint} sesquilinear form to $a(\cdot,\cdot)$.
\begin{definition}[The adjoint sesquilinear form $a^*(\cdot,\cdot)$]\label{def:adjoint}
Let $\Dm$, $n$, and $A$, be as in the definition of the EDP (\cref{prob:edp}). The adjoint sesquilinear form, $a^*(u,v)$, to $a(\cdot,\cdot)$ defined in \cref{eq:aedp} is given by
\beq\label{eq:EDPadjoint}
a^*(u,v) := \int_{\DR} 
\Big((A \grad u)\cdot\grad \vb
 - k^2 n u\vb\Big) - \big\langle \gamma u,T_R(\gamma v)\big\rangle_{\GR}.
\eeq
\end{definition}

\noi It is then straightforward to check that
\beq\label{eq:A*}
\Amat^* := \Smat_A -k^2 \Mmat_n - \Nmat^*
\eeq
(where $^*$ denotes conjugate transpose) is the Galerkin matrix for the sesquilinear form $a^*(\cdot,\cdot)$; i.e.~$(\Amat^*)_{ij} = a^*(\phi_j, \phi_i)$.

\ble[Link between variational problems involving $a(\cdot,\cdot)$ and $a^*(\cdot,\cdot)$]\label{lem:adjoint}
Given $F\in (\HozDDR)'$, if $u$ is the solution to the variational problem
\beq\label{eq:adjoint1}
a^*(u,v)= F(v) \quad\tfa v\in \HozDDR,
\eeq
then $\overline{u}$ satisfies
\beq\label{eq:adjoint2}
a(\overline{u},v)= \overline{F(\overline{v})} \quad\tfa v\in \HozDDR.
\eeq
\ele

For the proof of \cref{lem:adjoint} we need the following property of the DtN map $T_R$:
\beq\label{eq:DtN}
\big\langle T_R\psi, \overline{\phi} \big\rangle_\Gamma = \big\langle T_R \phi, \overline{\psi}\big\rangle_\Gamma \quad\tfa \phi,\psi \in H^{1/2}(\GR).
\eeq
This property follows from the fact that, if $u$ and $v$ are solutions of the homogeneous Helmholtz equation $\Delta u +k^2 u=0$ in $\RRd\setminus \overline{\BR}$, both satisfying the Sommerfeld radiation condition \cref{eq:src}, then
\beqs
\int_{\GR} (\gamma u)\, \pdiff{v}{n} = \int_{\GR} (\gamma v)\,\pdiff{u}{n};
\eeqs
see, e.g., \cite[Lemma 6.13]{Sp:15}.

\bpf[Proof of \cref{lem:adjoint}]
From \cref{eq:adjoint1} we have that 
\beqs
\overline{a^*(u,\overline{v})}= \overline{F(\overline{v})} \quad\tfa v\in \HozDDR.
\eeqs
Using the definition of $a^*(\cdot,\cdot)$ and the property \cref{eq:DtN} in the left-hand side of this last equation, we find \cref{eq:adjoint2}.
\epf

\subsection{Nontrapping condition on $\Aso, \nso,$ and $\Dm$ (\cref{cond:1nbpc})}\label{sec:cond1}
In our proofs of the effectiveness we will assume that the \cref{prob:edp} with coefficients $\Aso$ and $\nso$ is nontrapping. Recall our earlier discussion of this concept in \cref{sec:wpbounds,sec:wpdisc} above. Observe that we \emph{do not} impose any behaviour on \cref{prob:edp} with coefficients $\Ast$ and $\nst$.

\begin{condition}[Nontrapping bound on $u^{(1)}$]\label{cond:1nbpc}
$\Aso, \nso,$ and $\Dm$ are such that, given $f\in L^2(\DR)$ with $\supp \, f \subset \BR$, 
the solution of the EDP %(\cref{prob:edp}) 
$u^{(1)}$ exists, is unique, and, given $k_0>0$, $u^{(1)}$ satisfies the bound 
\beq\label{eq:bound1}
\big\|u^{(1)}\big\|_{\HokDR} \leq C^{(1)}_{\rm bound} \N{f}_{L^2(\Dp)} \quad \tfa k\geq k_0,
\eeq
where $C^{(1)}_{\rm bound}$ is independent of $k$, but dependent on $\Aso, \nso, \Dm, R$, and $k_0$.
\end{condition}

Using the G\aa rding inequality \cref{eq:gardingbrief}, one can prove the following result; see \cite[Lemma 5.1]{GrPeSp:19}.

\ble[Bound for data in $(\HozDDR)'$]\label{lem:H1}
%With the sesquilinear form $a(\cdot,\cdot)$ defined by \cref{eq:EDPa} with $A=\Aso$ and $n=\nso$, 
Given $\widetilde{F}\in (H^1_{0,D}(\DR))^*$, let $\widetilde{u}$ be the solution of the variational problem
\beqs
\text{ find } \,\,\widetilde{u} \in H^1_{0,D}(\DR) \,\,\tst \,\,
a^{(1)}(\widetilde{u},v)=\widetilde{F}(v) \,\, \tfa v\in H^1_{0,D}(\DR).
\eeqs
If \cref{cond:1nbpc} holds, then $\widetilde{u}$ exists, is unique, and satisfies the bound
\beq\label{eq:bound2}
\N{\widetilde{u}}_{\HokDR} \leq \frac{1}{\min\{\Asomin,\nsomin\}}\left( 1 + 2 C^{(1)}_{\rm bound}\nsomax  k\right) \big\|\widetilde{F}\big\|_{(\HokDR)'}
\eeq
for all $k\geq k_0$.
\ele




\subsection{$k$-independent accuracy of the FE solution (\cref{cond:2})}\label{sec:cond2}
We also make assumptions on the accuracy of the finite-element method applied to \cref{prob:edp} with coefficients $\Aso$ and $\nso$. In the language of our extended discussions in \cref{sec:helmfedisc} above, we assume that the finite-element method applied to \cref{prob:vedp} with right-hand sides in $\LtDR$ or $\HozDDRp$ is $\hk{a}b$-data-accurate, for some $a$ and $b$.
\begin{condition}[$k$-independent accuracy of the FE solution for $a^{(1)}(\cdot,\cdot)$]
\label{cond:2}

\

(i) Given $k_0>0$, $h$ and $p$ are such that, given $f\in \LtDR$ with $\supp\, f \subset \BR$, the solution $u_h$ of the Galerkin method \cref{eq:galerkin} with $a(\cdot,\cdot)=a^{(1)}(\cdot,\cdot)$ (and with $(v)$ defined in \cref{eq:Ledp}) exists and is unique for all $k\geq k_0$. % (so that the matrix $\Amato$ is invertible). 
Furthermore, if $f= n\sum_j \alpha_j\phi_j$ for some  $n\in \LiDRRR$ and $\alpha_j \in \CC$ (i.e.~$f$ is an arbitrary element of $\Vhp$ multiplied by $n$), then
\beq\label{eq:bound3}
\N{u-u_h}_{\HokDR} \leq C^{(1)}_{\rm FEM1} \N{f}_{\LtDR} \quad\tfa k\geq k_0, 
\eeq
where $C^{(1)}_{\rm FEM1}$  is independent of $k$ and $h$, but dependent on $\Aso, \nso, \Dm, R, k_0$, and $p$.

(ii) Given $k_0>0$, $h$ and $p$ are such that, given $F\in (\HozDDR)'$, the solution $u_h$ of the Galerkin method \cref{eq:galerkin} 
with $a(\cdot,\cdot)=a^{(1)}(\cdot,\cdot)$
exists, and is unique for all $k\geq k_0$.
%(so that the matrix $\Amato$ is invertible). 
Furthermore, if $F(v)= (A\nabla \widetilde{f},\nabla v)_{\LtDR}$, where $A\in L^\infty(\DR, \RR^{d\times d})$, $A$ is symmetric, and $\widetilde{f} := \sum_j \alpha_j \phi_j$ with $\alpha_j\in \CC$
 (i.e.~$\widetilde{f}$ is an arbitrary element of $\Vhp$), then
\beq\label{eq:bound4}
\N{u-u_h}_{\HokDR} \leq C^{(1)}_{\rm FEM2}\,k\, \N{F}_{(\HokDR)'} \quad\tfa k\geq k_0, 
\eeq
where $C^{(1)}_{\rm FEM2}$  is independent of $k$ and $h$, but dependent on $\Aso, \nso, \Dm, R, k_0$, and $p$.
\end{condition}



\section{Proofs of the main results}\label{sec:proofs}

\subsection{Proof of \cref{thm:1}} 

The following two lemmas are the heart of the proof of \cref{thm:1}.

\ble[Bounds on $(\Amato)^{-1} \Mmat_{n}$]\label{lem:keylemma1}
Assume that \cref{cond:1nbpc} holds, and assume that Part (i) of \cref{cond:2} holds. Then, for $n\in \LiDRRR$,
\beq\label{eq:keybound1}
\max\Big\{\big\| (\Amato)^{-1} \Mmat_{n} \big\|_{\Dmat_k}, \,\,
\big\|  \Mmat_{n}(\Amato)^{-1} \big\|_{(\Dmat_k)^{-1}}
\Big\}\leq 
C_2
%\frac{m_+}{m_-} \left[ C_{\rm FEM1}^{(1)} + C_{\rm bound}^{(1)}\right] 
\frac{\N{n}_{L^\infty(\DR)}}{k}
\eeq
and 
\beq\label{eq:keybound1a}
\max\Big\{\big\| (\Amato)^{-1} \Mmat_{n} \big\|_2, \,\,
\big\|  \Mmat_{n}(\Amato)^{-1} \big\|_2 
\Big\}\leq 
C_2 
%\frac{m_+}{m_-} \left[ C_{\rm FEM1}^{(1)} + C_{\rm bound}^{(1)}\right] 
\left(\frac{m_+}{m_-}\right) \frac{\N{n}_{L^\infty(\DR)}}{k}
\eeq
for all $k\geq k_0$,
where
\beq\label{eq:C2}
C_2:=%\frac{m_+}{m_-} 
%\left[ 
C_{\rm FEM1}^{(1)} + C_{\rm bound}^{(1)}.%\right].
\eeq
\ele

\ble[Bounds on $(\Amato)^{-1} \Smat_A$]\label{lem:keylemma2}
Assume that \cref{cond:1nbpc} holds, and assume that Part (ii) of \cref{cond:2} holds. Then, for $A\in L^\infty(\DR,\RR^{d\times d})$,
\beq\label{eq:keybound2}
\max\Big\{\big\| (\Amato)^{-1} \Smat_A \big\|_{(\Dmat_k)^{-1}}, \,\,
\big\| \Smat_A (\Amato)^{-1} \big\|_{\Dmat_k}\Big\} \leq C_1\, k\N{A}_{L^\infty(\DR)}
\eeq
and
\beq\label{eq:keybound2a}
\max\Big\{\big\| (\Amato)^{-1} \Smat_A \big\|_2, \,\,
\big\| \Smat_A (\Amato)^{-1} \big\|_2\Big\} \leq C_1\,\left(\frac{s_+}{m_-}\right) \frac{1}{h}\N{A}_{L^\infty(\DR)}
\eeq
%\begin{align}\nonumber
%&\max\Big\{\big\| (\Amato)^{-1} \Smat_A \big\|_2, \,\,
%\big\| \Smat_A (\Amato)^{-1} \big\|_2\Big\}\nonumber \\
%&\hspace{2cm}
% \leq \frac{s_+}{s_-} \left[ C_{\rm FEM2}^{(1)} + 
% \frac{1}{\min\big\{\Asomin,\nsomin\big\}}\left( \frac{1}{k_0} + 2 C^{(1)}_{\rm bound}\nsomax  \right) \right]k\N{A}_{L^\infty(\DR)}\label{eq:keybound2}
%% + C_{\rm bound}^{(1)}\right) \frac{\N{n}_{L^\infty(\DR)}}{k}.
%\end{align}
for all $k\geq k_0$, where
\beq\label{eq:C1nbpc}
C_1:=%\frac{s_+}{s_-} 
\left[ C_{\rm FEM2}^{(1)} + 
 \frac{1}{\min\big\{\Asomin,\nsomin\big\}}\left( \frac{1}{k_0} + 2 C^{(1)}_{\rm bound}\nsomax  \right) \right]
\eeq
\ele

\bpf[Proof of \cref{thm:1} using \cref{lem:keylemma1,lem:keylemma2}]
Using the definition of the matrices $\Amatj, \Smat_A$, and $\Mmatj$ in \cref{eq:matrixAjdef} and \cref{eq:matrixSjdef}, we have
\begin{align}\nonumber
\Imat - (\Amato)^{-1}\Amatt = (\Amato)^{-1}\big(\Amato-\Amatt\big) &=  (\Amato)^{-1}\left( \Smat_{A^{(1)}} - \Smat_{A^{(2)}} - k^2 \big(\Mmat_{n^{(1)}}-\Mmat_{n^{(2)}}\big)\right)\\
&= (\Amato)^{-1}\left( \Smat_{A^{(1)}-A^{(2)}} - k^2 \Mmat_{n^{(1)}-n^{(2)}}\right),\label{eq:idea1}
\end{align}
and similarly 
\beq\label{eq:idea2}
\Imat -\Amatt  (\Amato)^{-1}= \left( \Smat_{A^{(1)}-A^{(2)}} - k^2 \Mmat_{n^{(1)}-n^{(2)}}\right)(\Amato)^{-1}.
\eeq
The bound  \cref{eq:main1} on $\|\Imat - (\Amato)^{-1}\Amatt\|_2$ and  $\|\Imat - \Amatt(\Amato)^{-1}\|_2$ then follows from using the bounds \cref{eq:keybound1} and \cref{eq:keybound2} in \cref{eq:idea1} and \cref{eq:idea2}.
%
%, and $C_1$, $C_2$ in \cref{eq:main1} are given explicitly by
%\beq\label{eq:C1nbpc}
%C_1:=%\frac{s_+}{s_-} 
%\left[ C_{\rm FEM2}^{(1)} + 
% \frac{1}{\min\big\{\Asomin,\nsomin\big\}}\left( \frac{1}{k_0} + 2 C^{(1)}_{\rm bound}\nsomax  \right) \right] \,\,\tand\,\,
%\quad C_2:=  %+ \frac{m_+}{m_-} 
% \left[ C_{\rm FEM1}^{(1)} + C_{\rm bound}^{(1)}\right].
%\eeq
\epf

\

\bpf[Proof of \cref{lem:keylemma1}]
We first concentrate on proving \cref{eq:keybound1}.
Given $\bff \in \CC^N$ and $n\in \LiDRRR$, we create a variational problem whose Galerkin discretisation leads to the equation $\Amato \tbu = \Mmat_n\,\bff$.
Indeed, let $\widetilde{f} := \sum_j f_j \phi_j\in \HozDDR$. Define $\widetilde{u}$ to be the solution of the variational problem 
\beq\label{eq:411}
a^{(1)}(\widetilde{u},v)= (n\widetilde{f},v)_{L^2(\Omega)} \quad\text{ for all } v\in H^1(\Omega),
\eeq
let $\tu_h$ be the solution of the finite-element approximation of \cref{eq:411}, i.e.,
\beq\label{eq:41}
a^{(1)}(\tu_h,v_h)= (n\widetilde{f},v_h)_{L^2(\Omega)} \quad\text{ for all } v_h\in \Vhp,
\eeq
and let $\tbu$ be the vector of nodal values of $\tu_h$ \ednote{rephrase this as haven't (yet) assumed that $\{\phi_i\}$ is a nodal basis}. The definition of $\widetilde{f}$ then implies that \cref{eq:41} is equivalent to the linear system $\Amato \tbu = \Mmat_{n}\,\bff$, and so to obtain a bound on $\|(\Amato)^{-1}\Mmat_n\|_{\Dmat_k}$ we need to bound $\|\tbu\|_{\Dmat_k}$ in terms of $\|\bff\|_{\Dmat_k}$. Because of the definition 
of $\|\cdot\|_{\Dmat_k}$ in \cref{eq:Dk}, this is equivalent to bounding $\|\tu_h\|_{\HokDR}$ in terms of $\|\widetilde{f}\|_{\HokDR}$.

%First observe that the bound \cref{eq:bound3} from Part (i) of \cref{cond:2} holds for the solution of the variational problem
%\beqs%\label{eq:411}
%a^{(1)}(u,v)= (n\phi_j,v)_{L^2(\Omega)} \quad\text{ for all } v\in H^1(\Omega),
%\eeqs
%and hence, by linearity, it also holds for the solution $\widetilde{u}$ of the variational problem \cref{eq:411}.

Using %the bounds in \cref{eq:normequiv1}, 
the triangle inequality and the bounds \cref{eq:bound3} and \cref{eq:bound1} from \cref{cond:2,cond:1nbpc} respectively, we find
%Note that the hypotheses imply that the bound on the solution operator 
%\cref{eq:bound_unif} holds (by \cref{cor:uniform}), and also that if $h k\sqrt{|k^2-\eps|} \leq C_1$ then quasi-optimality \cref{eq:qoeps_lemma} holds (by \cref{lem:qo}).
%Starting with \cref{eq:equiv} we then have 
\begin{align}
%m_- h^{d/2}k \N{\tbu}_2 \leq k\N{\tu_h}_{\LtDR}\leq  
\N{\tu_h}_{\HokDR} \leq
\N{\tu-\tu_h}_{\HokDR} + \N{\tu}_{\HokDR} \label{eq:mainevent1}
& \leq C^{(1)}_{\rm FEM1}\NLtDR{n\ftilde} + C^{(1)}_{\rm bound}\NLtDR{n\ftilde} \\ 
& \leq \mleft(C^{(1)}_{\rm FEM1} + C^{(1)}_{\rm bound}\mright)\NLtDR{n}\NLtDR{\ftilde} \label{eq:mainevent1a} \\
& \leq\big(C^{(1)}_{\rm FEM1}+  C^{(1)}_{\rm bound}\big)\N{n}_{L^\infty(\DR)}\frac{\big\|\widetilde{f}\big\|_{\HokDR}}{k};\nonumber
%& \leq\big(C^{(1)}_{\rm FEM1}+  C^{(1)}_{\rm bound}\big)\N{n}_{L^\infty(\DR)} m_+ h^{d/2} \N{\bff}_2,
\end{align}
the bound on $\|(\Amato)^{-1}\Mmat_n\|_{\Dmat_k}$ in \cref{eq:keybound1} then follows from the definition of $\|\cdot\|_{\Dmat_k}$ in \cref{eq:Dk} and the definition of $C_2$ \cref{eq:C2}.

To prove the bound on $\|\Mmat_n(\Amato)^{-1}\|_{(\Dmat_k)^{-1}}$ in \cref{eq:keybound1}, first observe that the definitions of $\|\cdot\|_{\Dmat_k}$ and $\|\cdot\|_{(\Dmat_k)^{-1}}$ imply that, for any matrix $\matrixC$ and for any $\bv\in \CC^N$,
\beq\label{eq:A380-0}
\frac{
\big\|\matrixC \bv \big\|_{(\Dmat_k)^{-1}}
}{
\big\|\bv\|_{(\Dmat_k)^{-1}}
} = 
\frac{
\big\|\matrixC^* \bw \big\|_{\Dmat_k}
}{
\big\|\bw\|_{\Dmat_k}
}
\eeq
where $\bw := (\Dmat_k)^{1/2}\bv$, and where $\matrixC^*$ is the conjugate transpose of $\matrixC$ (i.e.~the adjoint with respect to $(\cdot,\cdot)_2$).
Therefore, since $\Mmat_n$ is a real, symmetric matrix,
\beqs%
\frac{
\big\|\Mmat_n (\Amato)^{-1}\bv\big\|_{(\Dmat_k)^{-1}}
}{
\N{\bv}_{(\Dmat_k)^{-1}}
}
=
\frac{
\big\|((\Amato)^*)^{-1}\Mmat_n\bw\big\|_{\Dmat_k}
}{
\N{\bw}_{\Dmat_k}
},
 \eeqs
 so that 
\beq\label{eq:A380} 
 \big\|\Mmat_n (\Amato)^{-1}\big\|_{(\Dmat_k)^{-1}}=\big\|((\Amato)^*)^{-1}\Mmat_n\big\|_{\Dmat_k}.
 \eeq 
Recall from the text below \cref{eq:A*} that $(\Amato)^*$ is the Galerkin matrix corresponding to the variational problem \cref{eq:adjoint1} -- the adjoint problem. \cref{lem:adjoint} implies that if the EDP %with coefficients $A^{(1)}$ and $n^{(1)}$ 
satisfies \cref{cond:1nbpc,cond:2}, then so does the adjoint problem. Therefore, the argument above leading to the bound on $\|(\Amato)^{-1}\Mmat_n\|_{\Dmat_k}$ under \cref{cond:1nbpc} and Part (i) of \cref{cond:2} proves the same bound on $\|((\Amato)^*)^{-1}\Mmat_n\|_{\Dmat_k}$, and then, using \cref{eq:A380}, also on $\big\|\Mmat_n(\Amato)^{-1}\big\|_{(\Dmat_k)^{-1}}$.

To prove the bound on  $\|(\Amato)^{-1}\Mmat_n\|_{2}$ in \cref{eq:keybound1a}, we use the bounds 
\beqs
m_- h^{d/2} k \N{\tbu}_2 \leq k \N{\widetilde{u}_h}_{\LtDR} \leq \N{\widetilde{u}_h}_{\HokDR}
\,\tand\,
\big\|\widetilde{f}\big\|_{\LtDR} \leq m_+ h^{d/2}\N{\bff}_2,
\eeqs
on either side of the inequality \cref{eq:mainevent1}, with these bounds coming from \cref{eq:normequiv1}. The proof of the bound on 
$\|\Mmat_n((\Amato)^*)^{-1}\|_{2}$ in \cref{eq:keybound1a} follows in a similar way to above, using the fact that 
$\|\Mmat_n (\Amato)^{-1}\|_2=\|((\Amato)^*)^{-1}\Mmat_n\|_2$ (compare to \cref{eq:A380}).
%, namely the variational problem \cref{eq:EDPvar} with the operator $T_R$ in $a^{(1)}(\cdot,\cdot)$ replaced by $\overline{T_R}$ (corresponding to the $-\ri k$ in the radiation condition \cref{eq:src} being changed to $+\ri k$).
%
%Now, if $u$ is the solution of the adjoint problem with data $F(v)$, then $\overline{u}$ is the solution of the original problem with data $\overline{F(\overline{v})}$; 
%
%in particular if $F(v)$ is as in \cref{eq:EDPa}, then the $L^2$ data of the adjoint problem is just $\overline{f}$. Therefore, if the EDP satisfies \cref{cond:1nbpc,cond:2}, then so does its adjoint, and
% the bound in \cref{eq:keybound1} on $\|(\Amato)^{-1}\Mmat_n\|_{2}$ also holds for $\|((\Amato)^*)^{-1}\Mmat_n\|_{2}$.
\epf

\

\bpf[Proof of \cref{lem:keylemma2}]
In a similar way to the proof of \cref{lem:keylemma1}, given $\bff \in \CC^N$ and a symmetric $A\in L^\infty(\DR, \RR^{d\times d})$, let $\widetilde{f} := \sum_j f_j \phi_j$ and observe that $\widetilde{f} \in \HozDDR$. Define $\widetilde{u}$ to be the solution of the variational problem 
\beq\label{eq:411a}
a^{(1)}(\widetilde{u},v)= \widetilde{F}(v) \quad\text{ for all } v\in H^1(\Omega),
\quad\text{ where } \quad
 \widetilde{F}(v) :=(A\nabla\widetilde{f},\nabla v)_{L^2(\Omega)}.
\eeq
Observe that the definition of the norms $\|\cdot\|_{(\HokDR)'}$ \cref{eq:dualnorm} and $\|\cdot\|_{\HokDR}$ \cref{eq:weightednorm} and the Cauchy-Schwarz inequality imply that
\begin{align}
\big\| \widetilde{F}\big\|_{(\HokDR)'}&\leq \big\|A\nabla \widetilde{f}\big\|_{\LtDR}\nonumber\\
&\leq \big\|A\big\|_{L^\infty(\DR)} \big\|\nabla \widetilde{f}\big\|_{\LtDR}\label{eq:Fbounda}\\
&\leq \big\|A\big\|_{L^\infty(\DR)} \big\| \widetilde{f}\big\|_{\HokDR}.\label{eq:Fbound}
\end{align}
Let $\tu_h$ be the solution of the finite element approximation of \cref{eq:411a}, i.e.,
\beq\label{eq:41a}
a^{(1)}(\tu_h,v_h)= \widetilde{F}(v_h) \quad\text{ for all } v_h\in \Vhp,
\eeq
and let $\tbu$ be the vector of nodal values of $\tu_h$. The definition of $\widetilde{f}$ then implies that \cref{eq:41a} is equivalent to $\Amato \tbu = \Smat_A\,\bff$. 

Similar to in the proof of \cref{lem:keylemma1},
using the triangle inequality, the bound \cref{eq:bound4} from \cref{cond:2}, the bound \cref{eq:bound2} from \cref{lem:H1}, the bound \cref{eq:Fbound}, and the definition of $C_1$ \cref{eq:C1nbpc},
we find
%Note that the hypotheses imply that the bound on the solution operator 
%\cref{eq:bound_unif} holds (by \cref{cor:uniform}), and also that if $h k\sqrt{|k^2-\eps|} \leq C_1$ then quasi-optimality \cref{eq:qoeps_lemma} holds (by \cref{lem:qo}).
%Starting with \cref{eq:equiv} we then have 
\begin{align}\nonumber 
%s_- h^{(d-2)/2} \N{\tbu}_2 &\leq \N{\nabla \tu_h}_{\LtDR}\leq  
\N{\tu_h}_{\HokDR} &\leq
\N{\tu-\tu_h}_{\HokDR} + \N{\tu}_{\HokDR},\nonumber \\ \nonumber
& \leq \left[ C^{(1)}_{\rm FEM2} k + 
\frac{1}{\min\{\Asomin,\nsomin\}}\left( 1 + 2 C^{(1)}_{\rm bound}\nsomax k  \right) 
\right]\big\|\widetilde{F}\big\|_{(\HokDR)'},\\
&\leq C_1 \, k\, 
%\left[C^{(1)}_{\rm FEM2} k + \frac{1}{\min\{\Asomin,\nsomin\}}\left( 1 + 2 C^{(1)}_{\rm bound}\nsomaxk  \right) \right]
\big\|A\big\|_{L^\infty(\DR)} \big\|\nabla\widetilde{f}\big\|_{\LtDR},\label{eq:mainevent2}\\
&\leq C_1 \, k\, 
%\left[C^{(1)}_{\rm FEM2} k + \frac{1}{\min\{\Asomin,\nsomin\}}\left( 1 + 2 C^{(1)}_{\rm bound}\nsomaxk  \right) \right]
\big\|A\big\|_{L^\infty(\DR)} \big\|\widetilde{f}\big\|_{\HokDR},\nonumber
%&\leq \left[ C^{(1)}_{\rm FEM2} k + 
%\frac{1}{\min\{\Asomin,\nsomin\}}\left( 1 + 2 C^{(1)}_{\rm bound}\nsomaxk  \right) 
%\right]\big\|A\big\|_{L^\infty(\DR)}s_+ h^{(d-2)/2} \N{\bff}_2,
\end{align}
and the bound on $\|(\Amato)^{-1}\Smat_A\|_{\Dmat_k}$ in \cref{eq:keybound2} follows.

The bound on $\|\Smat_A(\Amato)^{-1}\|_{(\Dmat_k)^{-1}}$ follows in a similar way to how we obtained the 
bound on  $\|(\Amato)^{-1}\Mmat_n\|_{(\Dmat_k)^{-1}}$ from the bound on $\|\Mmat_n(\Amato)^{-1}\|_{\Dmat_k}$ in Part (i). Indeed, 
\cref{eq:A380-0} and the fact that $\Smat_A$ is a real, symmetric matrix imply that 
\beq\label{eq:A380-2} 
 \big\|\Smat_A (\Amato)^{-1}\big\|_{(\Dmat_k)^{-1}}=\big\|\big((\Amato)^*\big)^{-1}\Smat_A\big\|_{\Dmat_k}
 \eeq 
%since 
%\beqs
%\big\|\Smat_A(\Amato)^{-1}\big\|_{2}=\big\|(\Smat_A(\Amato)^{-1})^*\big\|_{2}=\big\|((\Amato)^*)^{-1}\Smat_A\big\|_{2},
%\eeqs
(compare to \cref{eq:A380}),
and then the arguments in the proof of Part (i) imply that 
the bound in \cref{eq:keybound2} on $\|(\Amato)^{-1}\Smat_A\|_{\Dmat_k}$ also holds for $\|((\Amato)^*)^{-1}\Smat_A\|_{\Dmat_k}$.

To prove the bound on  $\|(\Amato)^{-1}\Smat_A\|_{2}$ in \cref{eq:keybound2a}, we use the bounds 
\beqs
m_- h^{d/2} k \N{\tbu}_2 \leq k \N{\widetilde{u}_h}_{\LtDR} \leq \N{\widetilde{u}_h}_{\HokDR}
\,\tand\,
\big\|\nabla \widetilde{f}\big\|_{\LtDR} \leq s_+ h^{d/2-1}\N{\bff}_2,
\eeqs
on either side of the inequality \cref{eq:mainevent2}, with these bounds coming from \cref{eq:normequiv1}.and \cref{eq:normequiv2} respectively. The proof of the bound on 
$\|\Smat_A((\Amato)^*)^{-1}\|_{2}$ in \cref{eq:keybound2a} follows in a similar way to above, using the fact that 
$\|\Smat_A (\Amato)^{-1}\|_2=\|((\Amato)^*)^{-1}\Smat_A\|_2$ (compare to \cref{eq:A380}).
\epf

\bre[Link to the results of \cite{GaGrSp:15}]
A result analogous to the Euclidean-norm bounds in \cref{thm:1} was proved in \cite{GaGrSp:15} for the case that $\Aso= \Ast= I$, $\nst= 1$, and $\nso = 1 + \ri\eps/k^2$, with the ``absorption parameter"/``shift" $\eps$ satisfying $0<\eps\lesssim k^2$. The motivation for proving this result was that the so-called ``shifted Laplacian preconditioning" of the Helmholtz equation is based on preconditioning (with these choices of parameters) $\Amatt$ with an approximation of $\Amato$. Similar to in \cref{cor:1}, bounds on $\|\Imat -  (\Amato)^{-1}\Amatt \|_2$ and 
$\|\Imat - \Amatt  (\Amato)^{-1}\|_2$
 then give upper bounds on large the ``shift" $\eps$ can be for GMRES to converge in a $k$-independent number of iterations in the case when the action of $(\Amato)^{-1}$ is computed exactly.

%\cite[Lemma 4.1]{GaGrSp:15}
The main differences between \cite{GaGrSp:15} and the present paper are that (i)  \cite{GaGrSp:15} focused on the TEDP, not the EDP,
(ii) \cite{GaGrSp:15} focused on the particular case that $\Dm$ is star-shaped with respect to a ball, finding a $k$- and $\eps$-explicit expression for $C^{(1)}_{\rm bound}$ in this case using Morawetz identities,
(iii) \cite{GaGrSp:15} required a bound analogous to that on 
$(\Amato)^{-1}\Mmat_{n}$, along with one on $(\Amato)^{-1}\Nmat$ (in the case that $T_R$ is approximated by $\ri k$), but \emph{not} on 
$(\Amato)^{-1}\Smat_{A}$, and (iv) \cite{GaGrSp:15} only proved bounds in the $\|\cdot\|_2$ norm.
%The result of \cref{thm:1}
\ere

%\bre[Analogue of \cref{thm:1} in a weighted norm]\label{rem:weight1}
%The PDE analysis of the Helmholtz equation naturally takes place in the weighted $H^1$ norm $\|\cdot\|_{\HokDR}$ defined by \cref{eq:1knorm}. The discrete analogue of this norm is the norm $\|\cdot\|_{\Dmat_k}$ defined by 
%\beq\label{eq:Dk}
%\N{\bv}_{\Dmat_k}^2:= \big( (\Smat_I + k^2 \Mmat_1)\bv,\bv\big)_2 = \N{v_h}^2_{\HokDR}
%\eeq
%for
%$v_h =\sum_i v_i \phi_i$. 
%This norm is used, e.g., in recent results about the convergence of domain-decomposition methods %in this norm are proved 
%for the Helmholtz equation \cite{GrSpVa:17}, \cite{GrSpZo:18}, and for the time-harmonic Maxwell equations \cite{BoDoGrSpTo:19}. 
%
%Inspecting the proof of \cref{lem:keylemma}, we see that the bounds \cref{eq:keybound1} and \cref{eq:keybound2} hold with the $\|\cdot\|_2$ norm replaced by the $\|\cdot\|_{\Dmat_k}$ norm and without the terms involving $m_\pm$ and $s_\pm$ on the right-hand side. \cref{thm:1} 
%%(and also \cref{cor:1}) 
%therefore also holds with the $\|\cdot\|_2$ norm replaced by the $\|\cdot\|_{\Dmat_k}$ norm and the constant $C_1$ modified appropriately.
%\ere

\subsection{Proof of \cref{cor:1,cor:1a}}

We first give the set-up for weighted GMRES.
Consider the abstract  linear system 
% \begin{equation*}
$\matrixC \bx = \bd$
%\end{equation*}
in $\mathbb{C}^N$, where $\matrixC \in \CC^{N\times N}$ is invertible.   
Given an initial guess $\bx^0$, we introduce the residual $\br^0 := \bd- C \bx^0$ and 
the usual Krylov spaces:  
\beqs  
\cK^m(C, \br^0) := \mathrm{span}\big\{\matrixC^j \br^0 : j = 0, \ldots, m-1\big\}.
\eeqs
Let $(\cdot , \cdot )_{\Dmat}$ denote the inner product on $\CC^n$ 
induced by some Hermitian positive-definite matrix $\Dmat$, i.e.~
%\begin{equation*}
$(\bv,\bw)_{\Dmat} := (\Dmat \bv, \bw)_2,$
%\end{equation*} 
with induced norm $\Vert \cdot \Vert_\Dmat$. For $m \geq 1$, define   $\bx^m$  to be  the unique element of $\cK^m$ satisfying  the  
 minimal residual  property: 
$$ \ \Vert \br^m \Vert_\Dmat := \Vert \bd - \matrixC \bx^m \Vert_\Dmat \ = \ \min_{\bx \in \cK^m(C, \br^0)} \Vert {\bd} - {\matrixC} {\bx} \Vert_\Dmat. $$
When $\Dmat = \Imat$ this is just the usual GMRES algorithm, but for  more general  $\Dmat$ it 
is the weighted GMRES method \cite{Es:98} in which case  
its implementation requires the application of the weighted Arnoldi process \cite{GuPe:14}.
Let 
\beq\label{eq:fov}
W_\Dmat(\matrixC):= \Big\{ (\matrixC \bx, \bx)_{\Dmat} : \bx \in \CCN, \|\bx\|_\Dmat=1\Big\};
\eeq
$W_\Dmat(\matrixC)$ is called the \emph{numerical range} or \emph{field of values} of $\matrixC$ (in the $(\cdot,\cdot)_\Dmat$ inner product).

%Recall the so-called ``Elman estimate" for GMRES

\begin{theorem}[Elman estimate for weighted GMRES]\label{thm:GMRES1_intro} 
Let $\matrixC$ be a matrix with $0\notin W(\matrixC)$. Let $\beta\in[0,\pi/2)$ be defined such that
\beq\label{eq:cosbeta}
\cos \beta := \frac{\mathrm{dist}\big(0, W(\matrixC)\big)}{\N{\matrixC}_{2}}.
\eeq
If the matrix equation $\matrixC \bx = \by$ is solved using weighted GMRES then, 
for $m\in \mathbb{N}$, the GMRES residual $\br_m$ %:= \matrixC \bx_m - \by$ 
satisfies
\beq\label{eq:Elman}
\frac{\N{\bfr_m}_{\Dmat}}{\N{\bfr_0}_{\Dmat}} \leq \sin^m \beta. %, \quad \text{ where}\quad 
\eeq
\end{theorem}
The bound \cref{eq:Elman} with $\Dmat=\Imat$ was originally proved in \cite{El:82} (see also \cite[Theorem 3.3]{EiElSc:83}) and appears in the form above in \cite[Equation 1.2]{BeGoTy:06}. The bound \cref{eq:Elman} (for arbitrary Hermitian positive-definite $\Dmat$) was stated (without proof) in \cite{CaWi:92} and proved in \cite[Theorem 5.1]{GrSpVa:17}. % (see also \cite[Remark 5.2]{GrSpVa:17}). 



\cref{thm:GMRES1_intro} has the following corollary, and the proofs of \cref{cor:1,cor:1a} follow from combining this with \cref{thm:1}.

\begin{corollary}
\label{cor:GMRES_intro} 
If $\|\Imat - \matrixC \|_\Dmat \leq \sigma < 1$, then, with $\beta$ defined as in \cref{eq:cosbeta},
\beqs
\cos \beta \geq \frac{1-\sigma}{1+\sigma}\eeqs
and
\beq\label{eq:gmressin}
\sin \beta \leq \frac{2 \sqrt{\sigma}}{(1+\sigma)^2}.
\eeq
\end{corollary}

\bpf[Proof of \cref{cor:1}]
This follows from \cref{thm:1} by applying \cref{cor:GMRES_intro} first with $\matrixC= (\Amato)^{-1} \Amatt$, $\Dmat=\Dmat_k$, and $\sigma=1/2$, and then with $\matrixC= \Amatt(\Amato)^{-1} $, $\Dmat=(\Dmat_k)^{-1}$, and $\sigma=1/2$.
\epf

\

\bpf[Proof of \cref{cor:1a}]
This follows from \cref{thm:1} by applying \cref{cor:GMRES_intro} first with $\matrixC= (\Amato)^{-1} \Amatt$, $\Dmat=\Imat$, and $\sigma=1/2$, and then with $\matrixC= \Amatt(\Amato)^{-1} $, $\Dmat=\Imat$, and $\sigma=1/2$.
\epf


\bre[The improvement of the Elman estimate \cref{eq:Elman} in \cite{BeGoTy:06}]
A stronger result than \cref{eq:Elman} is given for standard (unweighted) GMRES in \cite[Theorem 2.1]{BeGoTy:06}, and then converted to a result about weighted GMRES in \cite[Theorem 5.3]{BoDoGrSpTo:19}; indeed, the convergence factor $\sin \beta$ is replaced by a function of $\beta$ strictly less than $\sin\beta$ for $\beta\in (0,\pi/2)$. Using this stronger result, however, does not improve the $k$-dependence of \cref{cor:1}.
\ere


%\section{Proof of }\label{sec:proofPDE}

\subsection{Proofs of \cref{thm:2}, and \cref{lem:sharp}}

\bpf[Proof of \cref{thm:2}]
%We first prove the upper bound \cref{eq:PDEbound}.
By \cref{prob:edp}, $u^{(1)}$ and $u^{(2)}$ satisfy $a^{(1)}(u^{(1)}, v) = F(v)$ and 
$a^{(2)}(u^{(2)}, v) = F(v)$ respectively. Subtracting these equations, we find that $u^{(1)}- u^{(2)}$ satisfies the variational problem
\beq\label{eq:vp1}
a^{(1)}(u^{(1)}-u^{(2)},v) = \widetilde{F}(v) \quad\tfa v\in H^1_{0,D}(\DR)
\eeq
where
\beqs
 \widetilde{F}(v):= \int_{\DR} \left((\Ast-\Aso) \nabla u^{(2)}\right) \cdot\overline{\nabla v} + k^2 (\nso-\nst) u^{(2)}\overline{v}.
\eeqs
Now, by the Cauchy-Schwarz inequality and the definition of the norm $\|\cdot\|_{\HokDR}$ \cref{eq:weightednorm}, we have that
\begin{align*}
| \widetilde{F}(v)| &\leq \big\|\Aso-\Ast\big\|_{L^\infty(\DR)} \big\|\nabla u^{(2)}\big\|_{L^2(\DR)}
\N{\nabla v}_{L^2(\DR)} 
\\& \hspace{5cm}+ k^2 
\big\|\nso-\nst\big\|_{L^\infty(\DR)} \big\| u^{(2)}\big\|_{L^2(\DR)}
\N{v}_{L^2(\DR)}\\
&\leq\max\Big\{\big\|\Aso-\Ast\big\|_{L^\infty(\DR)}\,,\, \big\|\nso-\nst\big\|_{L^\infty(\DR)}\Big\}
\big\| u^{(2)}\big\|_{\HokDR} \N{v}_{\HokDR}.
\end{align*}
and thus, by the definition of the norm $\|\cdot\|_{(\HokDR)'}$ \cref{eq:dualnorm},
\beqs
\big\|\widetilde{F}\big\|_{(\HokDR)'}\leq \max\Big\{\big\|\Aso-\Ast\big\|_{L^\infty(\DR)}\,,\, \big\|\nso-\nst\big\|_{L^\infty(\DR)}\Big\}
\big\| u^{(2)}\big\|_{\HokDR}.
\eeqs
Since \cref{cond:1nbpc} holds, we can then apply the result of \cref{lem:H1}, i.e.~the bound \cref{eq:bound2}, to the solution of the variational problem \cref{eq:vp1}  to find that 
\begin{align*}
\frac{\big\| u^{(1)} - u^{(1)}\big\|_{\HokDR}}
{\big\| u^{(2)}\big\|_{\HokDR}, 
}
 \leq 
\,&\frac{1}{\min\big\{\Asomin,\nsomin\big\}}\left( 1 + 2 C^{(1)}_{\rm bound}\nsomax  k\right)
\\
&\quad\times \left(\max\Big\{\big\|\Aso-\Ast\big\|_{L^\infty(\DR)}\,,\, \big\|\nso-\nst\big\|_{L^\infty(\DR)}\Big\}\right),
\end{align*}
and then the result \cref{eq:PDEbound} follows with 
\beq\label{eq:C3}
C_3:= \frac{1}{\min\big\{\Asomin,\nsomin\big\}}\left( \frac{1}{k_0} + 2 C^{(1)}_{\rm bound}\nsomax  \right).
\eeq
\epf

\bpf[Proof of \cref{lem:sharp}]
We actually prove the stronger result that given any function $c(k)$ such that $c(k)>0$ for all $k>0$, there exist 
$f, \nso, \nst$ (with $\nso\not= \nst$), such that, firstly,
\beqs
\big\|\nso-\nst\big\|_{L^\infty(\DR)} \sim c(k)
\eeqs
and, secondly,
the corresponding solutions $u^{(1)}$ and $u^{(2)}$ of the exterior Dirichlet problem with $\Aso = \Ast= I$ exist, are unique, and satisfy \cref{eq:sharp1}. 

The heart of the proof is the equation
\beq\label{eq:obs1}
(\Delta + k^2) \big(\re^{\ri k r}\chi(r)\big) =  \re^{\ri k r}\left(\ri k \frac{d-1}{r} \chi(r) + 2 \ri k \diff{\chi}{r}(r) + \Delta \chi(r)\right)=: -\widetilde{f}(r),
\eeq
where $\chi(r)$ is chosen to have $\supp \chi \subset \DR$.
This equation proves the sharpness of the nontrapping resolvent estimate \cref{eq:bound1}, since both the $L^2(\DR)$ norm of $\widetilde{f}$ and the $\HokDR$ norm of $\re^{\ri kr}\chi(r)$ are proportional to $k$, and hence each to other (see, e.g., \cite[Lemma 3.10]{ChMo:08},  \cite[Lemma 4.12]{Sp:14}).

The overall idea of the proof is to set things up so that $(u^{(1)}-u^{(2)})(\bx) = \re^{\ri k r}\chi(r)$, the rationale being that \cref{eq:obs1} proves the sharpness of \cref{eq:bound1}, and \cref{eq:bound1} and its corollary \cref{eq:bound2} (applied to $u^{(1)}-u^{(2)}$) are the main ingredients in the proof of \cref{thm:2}.
% to prove the sharpness of \cref{thm:2} (at least when $\Aj:=I$, $j=1,2,$), 

Observe that, when $\Aj:=I$, $j=1,2,$ and $\nso:=1$, the variational problem \cref{eq:vp1} implies that 
\beq\label{eq:obs2}
\Delta \big( u^{(1)} - u^{(2)}\big) + k^2 \big( u^{(1)} - u^{(2)}\big) = -k^2 \big(1-\nst\big)u^{(2)}.
\eeq
Let $\nst:= 1 + c(k)\widetilde{\chi}$ with $\widetilde{\chi}= \widetilde{\chi}(r)$, $\widetilde{\chi}\in C^{\infty}(\DR)$, $\widetilde{\chi}\not = 1$ (so that $\nst\not = \nso$), and 
 $\supp \, \widetilde{\chi} \subset\DR$. 
%observe then that $\nst(\bx) >1$ for all $\bx \in \DR$ and thus, in particular, %for some function $c(k)>0$ for all $k>0$ 
%$\nst\not = \nso$. 
As above, let $\chi=\chi(r)$ with $\chi \in C^{\infty}(\DR)$ and
$\supp \,\chi$ a compact subset of $\DR$. Then with $\widetilde{f}$ defined in \cref{eq:obs1}, our goal is to let 
\beq\label{eq:obs3}
u^{(2)}(\bx):= -\frac{1}{k^2 c(k)}\frac{\widetilde{f}(r)}{\widetilde{\chi}(r)} \quad\tand\quad  f(\bx):= -\big(\Delta +k^2 \nst(\bx)\big) u^{(2)}(\bx);
\eeq
however, since $\widetilde{\chi}(r)$ has compact support, we need to tie both the support of $\widetilde{\chi}$ and how fast $\widetilde{\chi}$ vanishes in a neighbourhood of its support to the definition of $\chi$ for both $u^{(2)}$ and $f$ to be well defined.

Setting aside for the moment this need to synchronise the definitions of $\chi$ and $\widetilde{\chi}$, since $\supp \,\widetilde{f}$ is a compact subset of $\DR$, so is 
$\supp \,u^{(2)}$, and so $u^{(2)}$ is then the solution of the exterior Dirichlet problem (in the sense of \cref{prob:edp}) with data $f$ defined in \cref{eq:obs3} and coefficient $\nst:=1 + c(k)\widetilde{\chi}$.
Finally, define $u^{(1)}$ to be the solution of the exterior Dirichlet problem with $f$ defined in \cref{eq:obs3}. The whole point of these definitions is that, combined with \cref{eq:obs1} and \cref{eq:obs2} and the uniqueness of the solution of the exterior Dirichlet problem, they imply that 
\beq\label{eq:obs4}
u^{(1)}(\bx)- u^{(2)}(\bx) = \re^{\ri k x_1}\chi(r),
\eeq
and from this we therefore have that
\beqs
\big\|u^{(1)}-u^{(2)}\big\|_{L^2(\DR)} \sim 1
\quad \tand \quad
\big\|u^{(1)}-u^{(2)}\big\|_{\HokDR} \sim k.
\eeqs
Furthermore, the definitions of $u^{(2)}$ \cref{eq:obs3} and $\widetilde{f}$ \cref{eq:obs1} imply that
\beqs
\big\| u^{(2)}\big\|_{L^2(\DR)} \sim \frac{1}{k\, c(k)} \quad\tand \quad 
\big\| u^{(2)}\big\|_{\HokDR} \sim \frac{1}{c(k)},
\eeqs 
and, since $\|\nso- \nst\|_{L^\infty(\DR)} = c(k)$, \cref{eq:sharp1} holds.

Therefore, to complete the proof, we only need to show that there exists a choice of $\chi$ and $\widetilde{\chi}$ for which $u^{(2)}$ and $f$ defined by \cref{eq:obs3} are 
in $H^{1}(\DR)$ and $\LtDR$ respectively (in fact, we prove that they are in $W^{1,\infty}(\DR)$ and $L^\infty(\DR)$ respectively).
%well-defined. 
Since $\chi$ and $\widetilde{\chi} \in C^\infty(\DR)$, the only issue is what happens at the boundary of the support of $\widetilde{\chi}$, where $u^{(2)}$ has the potential to be singular.
Since $\overline{\Omega_-} \subset \BR$, there exist $0<R_1<R_2<R$ such that $\overline{\Dm} \subset B_{R_2}\setminus B_{R_1} \subset \BR$. Let $\supp \chi = B_{R_2}\setminus B_{R_1}$ and let $\chi$ vanish to order $m$ at $r= R_1$ and $r=R_2$; i.e.~$\chi(r) \sim (r-R_1)^m$ as $r \rightarrow (R_1)^+$ and 
$\chi(r) \sim (R_2-r)^m$ as $r \rightarrow (R_2)^-$. The definition of $\widetilde{f}$ \cref{eq:obs3} then implies that $\widetilde{f}$ vanishes to order $m-2$. Let $\widetilde{\chi}(r)$ vanish to order $n$ at $r= R_1$ and $r=R_2$. 
We now claim that if $m >n+4$, then $u^{(2)}\in W^{1,\infty}(\DR)$ and $f$ $\in L^\infty(\DR)$. Indeed,  
straightforward calculation from \cref{eq:obs3} shows that  $u^{(2)}(r) \sim (r-R_1)^{m-n-2}$, $\nabla u^{(2)}(r) \sim (r-R_1)^{m-n-3}$, and $\Delta u^{(2)}(r) \sim (r-R_1)^{m-n-4}$ as $r \rightarrow (R_1)^+$, with analogous behaviour at $r=R_2$.
%vanishes to order $m-n-2$ and $\Delta u^{(2)}$ vanishes to order $m-n-4$ at $r= R_1$ and $r=R_2$. 
The assumption 
$m >n+4$ therefore implies that $u^{(2)}$, $\nabla u ^{(2)}$, and $\Delta u^{(2)}$ vanish (and hence are finite) at $r=R_1$ and $r=R_2$.
\epf

\bre[Why doesn't \cref{lem:sharp} cover the case $\Aso\neq  \Ast$?]
When $\nj:=1$, $j=1,2,$, $\Aso:=I$, and $\Ast:= I + c(k)\widetilde{\chi}$, the variational problem \cref{eq:vp1} implies that 
\beqs%\label{eq:obs2}
\Delta \big( u^{(1)} - u^{(2)}\big) + k^2 \big( u^{(1)} - u^{(2)}\big) = c(k)\nabla\cdot \big(\widetilde{\chi}\nabla u^{(2)}\big).
\eeqs
It is now much harder than in \cref{eq:obs2} to set things up so that $ u^{(1)}(\bx) - u^{(2)}(\bx)=\re^{\ri kr}\chi(r)$ (so that one can then use \cref{eq:obs1}).
\ere

%\section{Proof of \cref{lem:2}}

\section{Extension of the results to weaker norms and probabilistic convergence results}
\optodo{Insert chat}
\subsection{Analogues of \cref{thm:1,cor:1,cor:1a} in weaker norms}\label{sec:weaknorm}
\optodo{Chat about weaker norms, in light of $L^1$ numerical results}

Recall from \cref{sec:num,sec:main} that GMRES applied to $\AmatoI \Amatt$ converges in a $k$-independent number of iterations if $\NLiDRRR{\nso-\nst} \sim 1/k$ (with an analagous result for $\Aso-\Ast$).\optodo{Insert somewhere, and write, some chat about why $1/k$ is the right thing to consider} This result show that $1/k$ is a sharp threshold when we consider the \emph{magnitude}  of the difference between $\nso$ and $\nst$. However, this result does not say anything about the \emph{spatial} variability\optodo{Somehow thanks Stefan, Nilima comments from MAFELAP?} between $\nso$ and $\nst$. For example if $\nso$ and $\nst$ (defined on the unit square) are given by
\beq\label{eq:noweak}
\nso(\bx) =
\begin{dcases}
  1 &\tif \bx_1 \leq \half\\
  2  &\tif \bx_1 > \half
  \end{dcases}
\eeq
and
\beq\label{eq:ntweak}
\nst(\bx) =
\begin{dcases}
  1 &\tif \bx_1 \leq \half+\alpha\\
  2  &\tif \bx_1 > \half+\alpha
  \end{dcases}
\eeq
for some $0 < \alpha < 1/2,$ then $\NLiDRRR{\nso-\nst} = 1$ for all $\alpha$, but one would expect that for small $\alpha$ the corresponding solutions of \cref{prob:edp} satisfy $\uso \approx \ust,$ and one might expect that GMRES applied to $\AmatoI\Amatt$ would converge in a $k$-independent number of iterations. Therefore, in this \lcnamecref{sec:weaknorm} we seek to obtain analogues of \cref{thm:1,cor:1,cor:1a} with the difference in $\nso-\nst$ and $\Aso-\Ast$ measured in weaker norms than the $L^\infty$ norm.
\optodo{Probably put something in somewhere about PDE result being sharp}
The (realistic) best-case result we could obtain would be that GMRES applied to $\AmatoI\Amatt$ converges in a $k$-independent number of iterations if $\NLoDRRR{\nso-\nst} \lesssim 1/k$. This result is `best' in the sense that it depends optimally on $k$; recall the disucssion in\optodo{Insert ref} that the length scale $1/k$ is the length scale governing the behaviour of Helmholtz problems. It is also `best' in the sense that the $L^1$ norm provides the maximal amount of `weight' to the spatial variance of $\nso-\nst$\optodo{Write this gooder} whilst also controlling the magnitude of $\nso-\nst$.

We will give numerical results indicating that this best-case result is sharp (our results actually indicate that we obtain $k$-independent convergence is $\NLqDRRR{\nso-\nst}\sim 1/k$ for any $q>0$). We will also provide theory results that are, to our knowledge, the best one can prove, although they are sub-optimal in both $q$ and the dependence on $k.$



\subsubsection{Theory in weaker norms}\label{sec:weakertheory}
The reason that the terms $\NLiDR{\Aso-\Ast}$ and $\NLiDR{\nso-\nst}$ appear in \cref{thm:1} is that the terms $\NLiDR{n}$ and $\NLiDR{A}$ appear in \cref{lem:keylemma1,lem:keylemma2}, respectively. These terms appear because in \cref{eq:mainevent1a,eq:Fbounda} we use the bounds
\beq\label{eq:keynbound}
\NLtDR{n\ftilde} \leq \NLiDR{n}\NLtDR{\ftilde}
\eeq
and
\beq\label{eq:keyAbound}
\NLtDR{A \grad \ftilde} \leq \NLiDR{A}\NLtDR{\grad \ftilde}
\eeq
respectively, for some function $\ftilde,$ and these bounds are carried through the rest of the proof.

However, if we instead use the variant of H\"older's inequality\optodo{Find ref} to bound \cref{eq:keynbound,eq:keyAbound} we instead obtain
\beq\label{eq:keynbound2}
\NLtDR{n\ftilde} \leq \NLqtildeDR{n}\NLptildeDR{\ftilde}
\eeq
and
\beq\label{eq:keyAbound2}
\NLtDR{A\grad\ftilde} \leq \NLqtildeDR{A}\NLptildeDR{\ftilde}.
\eeq
If we then apply\optodo{Write somewhere the inverse inequality from Brenner and Scott} to \cref{eq:keynbound2,eq:keyAbound2} we obtain
\beq\label{eq:keynboundfinal}
\NLtDR{n\ftilde} \leq \Cinvptilde \NLqtildeDR{n} h^{d\mleft(\frac1{\ptilde} - \half\mright)} \NLtDR{\ftilde}
\eeq
and
\beq\label{eq:keyAboundfinal}
\NLtDR{A\grad\ftilde} \leq \Cinvptilde \NLqtildeDR{A} h^{d\mleft(\frac1{\ptilde} - \half\mright)} \NLtDR{\grad\ftilde}.
\eeq

Replacing \cref{eq:mainevent1a,eq:Fbounda} with \cref{eq:keynboundfinal,eq:keyAboundfinal}, and proceeding as in the proofs of \cref{lem:keylemma1,lem:keylemma2}, we obtain \cref{lem:keylemma1a,lem:keylemma2a} below, the analogues of \cref{lem:keylemma1,lem:keylemma2}.

\ble[Alternative bounds on $(\Amato)^{-1} \Mmat_{n}$]\label{lem:keylemma1a}
Under the assumptions of \cref{lem:keylemma1}, for $n\in \LiDRRR$ and for any $\ptilde,\qtilde > 2$ such that $1/\ptilde + 1/\qtilde = \half$,
\beq\label{eq:keybound12}
\max\set{\NDk{\AmatoI \Mmatn},\NDkI{\Mmatn\AmatoI}} \leq \Cttilde h^{d\mleft(\frac1{\ptilde}-\half\mright)} \frac{\NLqtildeDR{n}}k
\eeq
and 
\beq\label{eq:keybound1a2}
\max\set{\Nt{\AmatoI \Mmatn},\Nt{\Mmatn\AmatoI}} \leq \Cttilde\mleft(\frac{\mplus}{\mminus}\mright) h^{d\mleft(\frac1{\ptilde}-\half\mright)} \frac{\NLqtildeDR{n}}k
\eeq
for all $k\geq \kz$,
where
\beq\label{eq:C2tilde}
\Cttilde\de%\frac{m_+}{m_-} 
%\left[ 
\Cinvptilde\Ct,
\eeq
where $\Ct$ is defined by \cref{eq:C2}.
\ele

\ble[Alternative bounds on $(\Amato)^{-1} \Smat_A$]\label{lem:keylemma2a}
Under the assumptions of \cref{lem:keylemma2}, for $A\in L^\infty(\DR,\RR^{d\times d})$ and for any $\ptilde,\qtilde > 2$ such that $1/\ptilde + 1/\qtilde = \half$,
\beq\label{eq:keybound22}
\max\set{\NDk{\AmatoI \SmatA},\NDkI{\SmatA\AmatoI}} \leq \Cotilde h^{d\mleft(\frac1{\ptilde}-\half\mright)}k \NLqtildeDR{A}
\eeq
and
\beq\label{eq:keybound2a2}
\max\set{\Nt{\AmatoI \SmatA},\Nt{\SmatA\AmatoI}} \leq \Cotilde\mleft(\frac{\splus}{\mminus}\mright) h^{d\mleft(\frac1{\ptilde}-\half\mright)-1} \NLqtildeDR{A}
\eeq
%\begin{align}\nonumber
%&\max\Big\{\big\| (\Amato)^{-1} \Smat_A \big\|_2, \,\,
%\big\| \Smat_A (\Amato)^{-1} \big\|_2\Big\}\nonumber \\
%&\hspace{2cm}
% \leq \frac{s_+}{s_-} \left[ C_{\rm FEM2}^{(1)} + 
% \frac{1}{\min\big\{\Asomin,\nsomin\big\}}\left( \frac{1}{k_0} + 2 C^{(1)}_{\rm bound}\nsomax  \right) \right]k\N{A}_{L^\infty(\DR)}\label{eq:keybound2}
%% + C_{\rm bound}^{(1)}\right) \frac{\N{n}_{L^\infty(\DR)}}{k}.
%\end{align}
for all $k\geq k_0$, where
\beq\label{eq:C1tildenbpc}
\Cotilde \de \Cinvptilde\Co,
\eeq
where $\Co$ is given by \cref{eq:C1nbpc}.
\ele


\begin{theorem}[Alternative main ingredient to answer to Q1]\label{thm:1alt}
Let the assumptions of \cref{thm:1} hold.   Then, given $\kz>0$ and $\ptilde,\qtilde >2$ such that $1/\ptilde + 1/\qtilde = 1/2$, there exist $\Cotilde, \Cttilde>0$, independent of $h$ and $k$ (but dependent on $\Dm, \Aso, \nso$, $p$, $\ptilde$, and $\kz$) such that
\begin{align}\nonumber
&\max\set{\NDk{\Imat - \AmatoI\Amatt},\NDkI{\Imat -\Amatt\AmatoI}}\\
&\hspace{3cm} 
\leq \Cotilde kh^{d\mleft(\frac1{\ptilde}-\half\mright)} \NLqtildeDR{\Aso-\Ast} + \Cttilde  kh^{d\mleft(\frac1{\ptilde}-\half\mright)}  \NLqtildeDR{\nso-\nst}
\label{eq:main1alt}
\end{align}
and 
\begin{align}\nonumber
&\max\set{\Nt{\Imat - \AmatoI\Amatt}, \Nt{\Imat -\Amatt\AmatoI}}\\
&\hspace{0cm}
\leq \Cotilde \mleft(\frac{\splus}{\mminus}\mright) h^{d\mleft(\frac1{\ptilde}-\half\mright)-1}\NLqtildeDR{\Aso-\Ast} + \Cttilde \mleft(\frac{\mplus}{\mminus}\mright) kh^{d\mleft(\frac1{\ptilde}-\half\mright)}\NLqtildeDR{\nso-\nst}
\label{eq:main1aalt}
\end{align}
for all $k\geq k_0$. 
\end{theorem}

\begin{corollary}[Alternative answer to Q1: $k$-independent weighted GMRES iterations]\label{cor:1alt}
Let the assumptions of \cref{cor:1a} hold.  Given $k_0>0$ and $\ptilde,\qtilde >2$ such that $1/\ptilde + 1/\qtilde = 1/2$, let $\Cotilde$ and $\Cttilde$ be as in \cref{thm:1alt}. Then if 
% there exists $C_2>0$, independent of $h$ and $k$ (but dependent on $\Dm, \Aso, \nso$, $p$, and $k_0$) and given explicitly in \cref{eq:C2} below,
% such that if 
\beq\label{eq:condalt}
\Cotilde kh^{d\mleft(\frac1{\ptilde}-\half\mright)} \NLqtildeDR{\Aso-\Ast} +\Cttilde  kh^{d\mleft(\frac1{\ptilde}-\half\mright)} \NLqtildeDR{\nso-\nst}
\leq \frac{1}{2}
\eeq
for all $k\geq k_0$, then \emph{both} weighted GMRES working in $\|\cdot\|_{\Dmat_k}$ (and the associated inner product) applied to \cref{eq:pcsystem1} \emph{and} weighted GMRES working in $\|\cdot\|_{(\Dmat_k)^{-1}}$ (and the associated inner product) applied to \cref{eq:pcsystem2}  converge in a $k$-independent number of iterations for all $k\geq k_0$.
\end{corollary}

\begin{corollary}[Alternative answer to Q1: $k$-independent (unweighted) GMRES iterations]\label{cor:1aalt}
Let the assumptions of \cref{cor:1a} hold.  Given $k_0>0$, and $\ptilde,\qtilde >2$ such that $1/\ptilde + 1/\qtilde = 1/2$. let $\Cotilde$ and $\Cttilde$ be as in \cref{thm:1alt}. Then if 
% there exists $C_2>0$, independent of $h$ and $k$ (but dependent on $\Dm, \Aso, \nso$, $p$, and $k_0$) and given explicitly in \cref{eq:C2} below,
% such that if 
\beq\label{eq:condaalt}
\Cotilde \mleft(\frac{\splus}{\mminus}\mright) h^{d\mleft(\frac1{\ptilde}-\half\mright)-1} \NLqtildeDR{\Aso-\Ast} + \Cttilde \mleft(\frac{\mplus}{\mminus}\mright) kh^{d\mleft(\frac1{\ptilde}-\half\mright)} \NLqtildeDR{\nso-\nst} \leq \half
\eeq
for all $k\geq k_0$, then standard GMRES (working in the Euclidean norm and inner product) applied to either of the equations \cref{eq:pcsystem1} or \cref{eq:pcsystem2}
%\beqs
%(\Amat^{(1)})^{-1}\Amat^{(2)}\bu = \bff\quad\text{ or } \quad\Amat^{(2)}(\Amat^{(1)})^{-1}\bv = \bff
%\eeqs
 converges in a $k$-independent number of iterations for all $k\geq k_0$.
\end{corollary}
\optodo{Need to have some chat about the trade-off with these, and the `theoretical' best $1/k$ for $L^1$}


\subsubsection{Numerics in weaker norms}\label{sec:weakernumerics}
For our computations, we solve the Interior Impedance Problem in 2-d on the unit square with $f$ and $\gI$ corresponding to a plane wave passing through homogeneous media. We let $\Aso=\Ast=I,$ and we define $\nso$ and $\nst$ by \cref{eq:noweak,eq:ntweak}. For $\alpha = 0.2,0.2/k^{0.1},0.2/k^{0.2},\ldots,0.2/k$ and for $k=10,20,\ldots,100$ we use GMRES to solve $\AmatoI\Amatt = \AmatoI \bff$ (for $\bff$ given by the Helmholtz problem), and we record the number of GMRES iterations taken to achieve convergence. Our convergence criteria are a realative residual tolerance of $10^{-5}$ and an absolute residual tolerance of $10^{-50}$ (the PETSc defaults).\optodo{Can lots of this numerical stuff be put in, e.g., and appendix to save repeating myself?}.

Our results in\optodo{Insert fig refs} indicate the following conclusions for $\NLqDRR{\nso-\nst} \sim 0.1/k^{\beta}$ :
\bit
\item For $\beta \in (0,0.6)$ there is clear growth of the number of GMRES iterations with $k$,
\item For $\beta = 1$ there is clear boundedness of the number of GMRES iterations with $k$, and
  \item for $\beta \in (0.7,0.9)$ it is unclear if the number of GMRES iterations grows with $k.$
\eit

If we compare our numerical results with the theory results in \cref{cor:aalt}, we see that the theory (with $h \sim k^{-3/2}$ and $d=2$) predicts that the number of iterations will remain bounded if $\NLqDRR{\nso-\nst} k^{1+3/q}$ is sufficiently small, for any $q \in (2,\infty).$ Our computed results indicate that this result is not sharp. The computed results indicate that if $\NLqDRR{\nso-\nst} \sim k^{-1}$ for any $q \geq 1,$ then the number of GMRES iterations is bounded as $k$ increases. Observe that the `best case' $1/k$ condition (see \optodo{Ref} above for why this is the best case) is only predicted by the theory in the $q\rightarrow \infty$ limit.\optodo{Probably need to write much more here}


\begin{figure}
%% Creator: Matplotlib, PGF backend
%%
%% To include the figure in your LaTeX document, write
%%   \input{<filename>.pgf}
%%
%% Make sure the required packages are loaded in your preamble
%%   \usepackage{pgf}
%%
%% Figures using additional raster images can only be included by \input if
%% they are in the same directory as the main LaTeX file. For loading figures
%% from other directories you can use the `import` package
%%   \usepackage{import}
%% and then include the figures with
%%   \import{<path to file>}{<filename>.pgf}
%%
%% Matplotlib used the following preamble
%%   \usepackage{fontspec}
%%   \setmainfont{DejaVuSerif.ttf}[Path=/home/owen/progs/firedrake-complex/firedrake/lib/python3.5/site-packages/matplotlib/mpl-data/fonts/ttf/]
%%   \setsansfont{DejaVuSans.ttf}[Path=/home/owen/progs/firedrake-complex/firedrake/lib/python3.5/site-packages/matplotlib/mpl-data/fonts/ttf/]
%%   \setmonofont{DejaVuSansMono.ttf}[Path=/home/owen/progs/firedrake-complex/firedrake/lib/python3.5/site-packages/matplotlib/mpl-data/fonts/ttf/]
%%
\begingroup%
\makeatletter%
\begin{pgfpicture}%
\pgfpathrectangle{\pgfpointorigin}{\pgfqpoint{6.400000in}{4.800000in}}%
\pgfusepath{use as bounding box, clip}%
\begin{pgfscope}%
\pgfsetbuttcap%
\pgfsetmiterjoin%
\definecolor{currentfill}{rgb}{1.000000,1.000000,1.000000}%
\pgfsetfillcolor{currentfill}%
\pgfsetlinewidth{0.000000pt}%
\definecolor{currentstroke}{rgb}{1.000000,1.000000,1.000000}%
\pgfsetstrokecolor{currentstroke}%
\pgfsetdash{}{0pt}%
\pgfpathmoveto{\pgfqpoint{0.000000in}{0.000000in}}%
\pgfpathlineto{\pgfqpoint{6.400000in}{0.000000in}}%
\pgfpathlineto{\pgfqpoint{6.400000in}{4.800000in}}%
\pgfpathlineto{\pgfqpoint{0.000000in}{4.800000in}}%
\pgfpathclose%
\pgfusepath{fill}%
\end{pgfscope}%
\begin{pgfscope}%
\pgfsetbuttcap%
\pgfsetmiterjoin%
\definecolor{currentfill}{rgb}{1.000000,1.000000,1.000000}%
\pgfsetfillcolor{currentfill}%
\pgfsetlinewidth{0.000000pt}%
\definecolor{currentstroke}{rgb}{0.000000,0.000000,0.000000}%
\pgfsetstrokecolor{currentstroke}%
\pgfsetstrokeopacity{0.000000}%
\pgfsetdash{}{0pt}%
\pgfpathmoveto{\pgfqpoint{0.800000in}{0.528000in}}%
\pgfpathlineto{\pgfqpoint{5.760000in}{0.528000in}}%
\pgfpathlineto{\pgfqpoint{5.760000in}{4.224000in}}%
\pgfpathlineto{\pgfqpoint{0.800000in}{4.224000in}}%
\pgfpathclose%
\pgfusepath{fill}%
\end{pgfscope}%
\begin{pgfscope}%
\pgfsetbuttcap%
\pgfsetroundjoin%
\definecolor{currentfill}{rgb}{0.000000,0.000000,0.000000}%
\pgfsetfillcolor{currentfill}%
\pgfsetlinewidth{0.803000pt}%
\definecolor{currentstroke}{rgb}{0.000000,0.000000,0.000000}%
\pgfsetstrokecolor{currentstroke}%
\pgfsetdash{}{0pt}%
\pgfsys@defobject{currentmarker}{\pgfqpoint{0.000000in}{-0.048611in}}{\pgfqpoint{0.000000in}{0.000000in}}{%
\pgfpathmoveto{\pgfqpoint{0.000000in}{0.000000in}}%
\pgfpathlineto{\pgfqpoint{0.000000in}{-0.048611in}}%
\pgfusepath{stroke,fill}%
}%
\begin{pgfscope}%
\pgfsys@transformshift{1.250909in}{0.528000in}%
\pgfsys@useobject{currentmarker}{}%
\end{pgfscope}%
\end{pgfscope}%
\begin{pgfscope}%
\definecolor{textcolor}{rgb}{0.000000,0.000000,0.000000}%
\pgfsetstrokecolor{textcolor}%
\pgfsetfillcolor{textcolor}%
\pgftext[x=1.250909in,y=0.430778in,,top]{\color{textcolor}\sffamily\fontsize{10.000000}{12.000000}\selectfont \(\displaystyle 10\)}%
\end{pgfscope}%
\begin{pgfscope}%
\pgfsetbuttcap%
\pgfsetroundjoin%
\definecolor{currentfill}{rgb}{0.000000,0.000000,0.000000}%
\pgfsetfillcolor{currentfill}%
\pgfsetlinewidth{0.803000pt}%
\definecolor{currentstroke}{rgb}{0.000000,0.000000,0.000000}%
\pgfsetstrokecolor{currentstroke}%
\pgfsetdash{}{0pt}%
\pgfsys@defobject{currentmarker}{\pgfqpoint{0.000000in}{-0.048611in}}{\pgfqpoint{0.000000in}{0.000000in}}{%
\pgfpathmoveto{\pgfqpoint{0.000000in}{0.000000in}}%
\pgfpathlineto{\pgfqpoint{0.000000in}{-0.048611in}}%
\pgfusepath{stroke,fill}%
}%
\begin{pgfscope}%
\pgfsys@transformshift{1.701818in}{0.528000in}%
\pgfsys@useobject{currentmarker}{}%
\end{pgfscope}%
\end{pgfscope}%
\begin{pgfscope}%
\definecolor{textcolor}{rgb}{0.000000,0.000000,0.000000}%
\pgfsetstrokecolor{textcolor}%
\pgfsetfillcolor{textcolor}%
\pgftext[x=1.701818in,y=0.430778in,,top]{\color{textcolor}\sffamily\fontsize{10.000000}{12.000000}\selectfont \(\displaystyle 20\)}%
\end{pgfscope}%
\begin{pgfscope}%
\pgfsetbuttcap%
\pgfsetroundjoin%
\definecolor{currentfill}{rgb}{0.000000,0.000000,0.000000}%
\pgfsetfillcolor{currentfill}%
\pgfsetlinewidth{0.803000pt}%
\definecolor{currentstroke}{rgb}{0.000000,0.000000,0.000000}%
\pgfsetstrokecolor{currentstroke}%
\pgfsetdash{}{0pt}%
\pgfsys@defobject{currentmarker}{\pgfqpoint{0.000000in}{-0.048611in}}{\pgfqpoint{0.000000in}{0.000000in}}{%
\pgfpathmoveto{\pgfqpoint{0.000000in}{0.000000in}}%
\pgfpathlineto{\pgfqpoint{0.000000in}{-0.048611in}}%
\pgfusepath{stroke,fill}%
}%
\begin{pgfscope}%
\pgfsys@transformshift{2.152727in}{0.528000in}%
\pgfsys@useobject{currentmarker}{}%
\end{pgfscope}%
\end{pgfscope}%
\begin{pgfscope}%
\definecolor{textcolor}{rgb}{0.000000,0.000000,0.000000}%
\pgfsetstrokecolor{textcolor}%
\pgfsetfillcolor{textcolor}%
\pgftext[x=2.152727in,y=0.430778in,,top]{\color{textcolor}\sffamily\fontsize{10.000000}{12.000000}\selectfont \(\displaystyle 30\)}%
\end{pgfscope}%
\begin{pgfscope}%
\pgfsetbuttcap%
\pgfsetroundjoin%
\definecolor{currentfill}{rgb}{0.000000,0.000000,0.000000}%
\pgfsetfillcolor{currentfill}%
\pgfsetlinewidth{0.803000pt}%
\definecolor{currentstroke}{rgb}{0.000000,0.000000,0.000000}%
\pgfsetstrokecolor{currentstroke}%
\pgfsetdash{}{0pt}%
\pgfsys@defobject{currentmarker}{\pgfqpoint{0.000000in}{-0.048611in}}{\pgfqpoint{0.000000in}{0.000000in}}{%
\pgfpathmoveto{\pgfqpoint{0.000000in}{0.000000in}}%
\pgfpathlineto{\pgfqpoint{0.000000in}{-0.048611in}}%
\pgfusepath{stroke,fill}%
}%
\begin{pgfscope}%
\pgfsys@transformshift{2.603636in}{0.528000in}%
\pgfsys@useobject{currentmarker}{}%
\end{pgfscope}%
\end{pgfscope}%
\begin{pgfscope}%
\definecolor{textcolor}{rgb}{0.000000,0.000000,0.000000}%
\pgfsetstrokecolor{textcolor}%
\pgfsetfillcolor{textcolor}%
\pgftext[x=2.603636in,y=0.430778in,,top]{\color{textcolor}\sffamily\fontsize{10.000000}{12.000000}\selectfont \(\displaystyle 40\)}%
\end{pgfscope}%
\begin{pgfscope}%
\pgfsetbuttcap%
\pgfsetroundjoin%
\definecolor{currentfill}{rgb}{0.000000,0.000000,0.000000}%
\pgfsetfillcolor{currentfill}%
\pgfsetlinewidth{0.803000pt}%
\definecolor{currentstroke}{rgb}{0.000000,0.000000,0.000000}%
\pgfsetstrokecolor{currentstroke}%
\pgfsetdash{}{0pt}%
\pgfsys@defobject{currentmarker}{\pgfqpoint{0.000000in}{-0.048611in}}{\pgfqpoint{0.000000in}{0.000000in}}{%
\pgfpathmoveto{\pgfqpoint{0.000000in}{0.000000in}}%
\pgfpathlineto{\pgfqpoint{0.000000in}{-0.048611in}}%
\pgfusepath{stroke,fill}%
}%
\begin{pgfscope}%
\pgfsys@transformshift{3.054545in}{0.528000in}%
\pgfsys@useobject{currentmarker}{}%
\end{pgfscope}%
\end{pgfscope}%
\begin{pgfscope}%
\definecolor{textcolor}{rgb}{0.000000,0.000000,0.000000}%
\pgfsetstrokecolor{textcolor}%
\pgfsetfillcolor{textcolor}%
\pgftext[x=3.054545in,y=0.430778in,,top]{\color{textcolor}\sffamily\fontsize{10.000000}{12.000000}\selectfont \(\displaystyle 50\)}%
\end{pgfscope}%
\begin{pgfscope}%
\pgfsetbuttcap%
\pgfsetroundjoin%
\definecolor{currentfill}{rgb}{0.000000,0.000000,0.000000}%
\pgfsetfillcolor{currentfill}%
\pgfsetlinewidth{0.803000pt}%
\definecolor{currentstroke}{rgb}{0.000000,0.000000,0.000000}%
\pgfsetstrokecolor{currentstroke}%
\pgfsetdash{}{0pt}%
\pgfsys@defobject{currentmarker}{\pgfqpoint{0.000000in}{-0.048611in}}{\pgfqpoint{0.000000in}{0.000000in}}{%
\pgfpathmoveto{\pgfqpoint{0.000000in}{0.000000in}}%
\pgfpathlineto{\pgfqpoint{0.000000in}{-0.048611in}}%
\pgfusepath{stroke,fill}%
}%
\begin{pgfscope}%
\pgfsys@transformshift{3.505455in}{0.528000in}%
\pgfsys@useobject{currentmarker}{}%
\end{pgfscope}%
\end{pgfscope}%
\begin{pgfscope}%
\definecolor{textcolor}{rgb}{0.000000,0.000000,0.000000}%
\pgfsetstrokecolor{textcolor}%
\pgfsetfillcolor{textcolor}%
\pgftext[x=3.505455in,y=0.430778in,,top]{\color{textcolor}\sffamily\fontsize{10.000000}{12.000000}\selectfont \(\displaystyle 60\)}%
\end{pgfscope}%
\begin{pgfscope}%
\pgfsetbuttcap%
\pgfsetroundjoin%
\definecolor{currentfill}{rgb}{0.000000,0.000000,0.000000}%
\pgfsetfillcolor{currentfill}%
\pgfsetlinewidth{0.803000pt}%
\definecolor{currentstroke}{rgb}{0.000000,0.000000,0.000000}%
\pgfsetstrokecolor{currentstroke}%
\pgfsetdash{}{0pt}%
\pgfsys@defobject{currentmarker}{\pgfqpoint{0.000000in}{-0.048611in}}{\pgfqpoint{0.000000in}{0.000000in}}{%
\pgfpathmoveto{\pgfqpoint{0.000000in}{0.000000in}}%
\pgfpathlineto{\pgfqpoint{0.000000in}{-0.048611in}}%
\pgfusepath{stroke,fill}%
}%
\begin{pgfscope}%
\pgfsys@transformshift{3.956364in}{0.528000in}%
\pgfsys@useobject{currentmarker}{}%
\end{pgfscope}%
\end{pgfscope}%
\begin{pgfscope}%
\definecolor{textcolor}{rgb}{0.000000,0.000000,0.000000}%
\pgfsetstrokecolor{textcolor}%
\pgfsetfillcolor{textcolor}%
\pgftext[x=3.956364in,y=0.430778in,,top]{\color{textcolor}\sffamily\fontsize{10.000000}{12.000000}\selectfont \(\displaystyle 70\)}%
\end{pgfscope}%
\begin{pgfscope}%
\pgfsetbuttcap%
\pgfsetroundjoin%
\definecolor{currentfill}{rgb}{0.000000,0.000000,0.000000}%
\pgfsetfillcolor{currentfill}%
\pgfsetlinewidth{0.803000pt}%
\definecolor{currentstroke}{rgb}{0.000000,0.000000,0.000000}%
\pgfsetstrokecolor{currentstroke}%
\pgfsetdash{}{0pt}%
\pgfsys@defobject{currentmarker}{\pgfqpoint{0.000000in}{-0.048611in}}{\pgfqpoint{0.000000in}{0.000000in}}{%
\pgfpathmoveto{\pgfqpoint{0.000000in}{0.000000in}}%
\pgfpathlineto{\pgfqpoint{0.000000in}{-0.048611in}}%
\pgfusepath{stroke,fill}%
}%
\begin{pgfscope}%
\pgfsys@transformshift{4.407273in}{0.528000in}%
\pgfsys@useobject{currentmarker}{}%
\end{pgfscope}%
\end{pgfscope}%
\begin{pgfscope}%
\definecolor{textcolor}{rgb}{0.000000,0.000000,0.000000}%
\pgfsetstrokecolor{textcolor}%
\pgfsetfillcolor{textcolor}%
\pgftext[x=4.407273in,y=0.430778in,,top]{\color{textcolor}\sffamily\fontsize{10.000000}{12.000000}\selectfont \(\displaystyle 80\)}%
\end{pgfscope}%
\begin{pgfscope}%
\pgfsetbuttcap%
\pgfsetroundjoin%
\definecolor{currentfill}{rgb}{0.000000,0.000000,0.000000}%
\pgfsetfillcolor{currentfill}%
\pgfsetlinewidth{0.803000pt}%
\definecolor{currentstroke}{rgb}{0.000000,0.000000,0.000000}%
\pgfsetstrokecolor{currentstroke}%
\pgfsetdash{}{0pt}%
\pgfsys@defobject{currentmarker}{\pgfqpoint{0.000000in}{-0.048611in}}{\pgfqpoint{0.000000in}{0.000000in}}{%
\pgfpathmoveto{\pgfqpoint{0.000000in}{0.000000in}}%
\pgfpathlineto{\pgfqpoint{0.000000in}{-0.048611in}}%
\pgfusepath{stroke,fill}%
}%
\begin{pgfscope}%
\pgfsys@transformshift{4.858182in}{0.528000in}%
\pgfsys@useobject{currentmarker}{}%
\end{pgfscope}%
\end{pgfscope}%
\begin{pgfscope}%
\definecolor{textcolor}{rgb}{0.000000,0.000000,0.000000}%
\pgfsetstrokecolor{textcolor}%
\pgfsetfillcolor{textcolor}%
\pgftext[x=4.858182in,y=0.430778in,,top]{\color{textcolor}\sffamily\fontsize{10.000000}{12.000000}\selectfont \(\displaystyle 90\)}%
\end{pgfscope}%
\begin{pgfscope}%
\pgfsetbuttcap%
\pgfsetroundjoin%
\definecolor{currentfill}{rgb}{0.000000,0.000000,0.000000}%
\pgfsetfillcolor{currentfill}%
\pgfsetlinewidth{0.803000pt}%
\definecolor{currentstroke}{rgb}{0.000000,0.000000,0.000000}%
\pgfsetstrokecolor{currentstroke}%
\pgfsetdash{}{0pt}%
\pgfsys@defobject{currentmarker}{\pgfqpoint{0.000000in}{-0.048611in}}{\pgfqpoint{0.000000in}{0.000000in}}{%
\pgfpathmoveto{\pgfqpoint{0.000000in}{0.000000in}}%
\pgfpathlineto{\pgfqpoint{0.000000in}{-0.048611in}}%
\pgfusepath{stroke,fill}%
}%
\begin{pgfscope}%
\pgfsys@transformshift{5.309091in}{0.528000in}%
\pgfsys@useobject{currentmarker}{}%
\end{pgfscope}%
\end{pgfscope}%
\begin{pgfscope}%
\definecolor{textcolor}{rgb}{0.000000,0.000000,0.000000}%
\pgfsetstrokecolor{textcolor}%
\pgfsetfillcolor{textcolor}%
\pgftext[x=5.309091in,y=0.430778in,,top]{\color{textcolor}\sffamily\fontsize{10.000000}{12.000000}\selectfont \(\displaystyle 100\)}%
\end{pgfscope}%
\begin{pgfscope}%
\definecolor{textcolor}{rgb}{0.000000,0.000000,0.000000}%
\pgfsetstrokecolor{textcolor}%
\pgfsetfillcolor{textcolor}%
\pgftext[x=3.280000in,y=0.240809in,,top]{\color{textcolor}\sffamily\fontsize{10.000000}{12.000000}\selectfont \(\displaystyle k\)}%
\end{pgfscope}%
\begin{pgfscope}%
\pgfsetbuttcap%
\pgfsetroundjoin%
\definecolor{currentfill}{rgb}{0.000000,0.000000,0.000000}%
\pgfsetfillcolor{currentfill}%
\pgfsetlinewidth{0.803000pt}%
\definecolor{currentstroke}{rgb}{0.000000,0.000000,0.000000}%
\pgfsetstrokecolor{currentstroke}%
\pgfsetdash{}{0pt}%
\pgfsys@defobject{currentmarker}{\pgfqpoint{-0.048611in}{0.000000in}}{\pgfqpoint{0.000000in}{0.000000in}}{%
\pgfpathmoveto{\pgfqpoint{0.000000in}{0.000000in}}%
\pgfpathlineto{\pgfqpoint{-0.048611in}{0.000000in}}%
\pgfusepath{stroke,fill}%
}%
\begin{pgfscope}%
\pgfsys@transformshift{0.800000in}{0.678442in}%
\pgfsys@useobject{currentmarker}{}%
\end{pgfscope}%
\end{pgfscope}%
\begin{pgfscope}%
\definecolor{textcolor}{rgb}{0.000000,0.000000,0.000000}%
\pgfsetstrokecolor{textcolor}%
\pgfsetfillcolor{textcolor}%
\pgftext[x=0.633333in,y=0.625680in,left,base]{\color{textcolor}\sffamily\fontsize{10.000000}{12.000000}\selectfont \(\displaystyle 0\)}%
\end{pgfscope}%
\begin{pgfscope}%
\pgfsetbuttcap%
\pgfsetroundjoin%
\definecolor{currentfill}{rgb}{0.000000,0.000000,0.000000}%
\pgfsetfillcolor{currentfill}%
\pgfsetlinewidth{0.803000pt}%
\definecolor{currentstroke}{rgb}{0.000000,0.000000,0.000000}%
\pgfsetstrokecolor{currentstroke}%
\pgfsetdash{}{0pt}%
\pgfsys@defobject{currentmarker}{\pgfqpoint{-0.048611in}{0.000000in}}{\pgfqpoint{0.000000in}{0.000000in}}{%
\pgfpathmoveto{\pgfqpoint{0.000000in}{0.000000in}}%
\pgfpathlineto{\pgfqpoint{-0.048611in}{0.000000in}}%
\pgfusepath{stroke,fill}%
}%
\begin{pgfscope}%
\pgfsys@transformshift{0.800000in}{1.077492in}%
\pgfsys@useobject{currentmarker}{}%
\end{pgfscope}%
\end{pgfscope}%
\begin{pgfscope}%
\definecolor{textcolor}{rgb}{0.000000,0.000000,0.000000}%
\pgfsetstrokecolor{textcolor}%
\pgfsetfillcolor{textcolor}%
\pgftext[x=0.494444in,y=1.024730in,left,base]{\color{textcolor}\sffamily\fontsize{10.000000}{12.000000}\selectfont \(\displaystyle 250\)}%
\end{pgfscope}%
\begin{pgfscope}%
\pgfsetbuttcap%
\pgfsetroundjoin%
\definecolor{currentfill}{rgb}{0.000000,0.000000,0.000000}%
\pgfsetfillcolor{currentfill}%
\pgfsetlinewidth{0.803000pt}%
\definecolor{currentstroke}{rgb}{0.000000,0.000000,0.000000}%
\pgfsetstrokecolor{currentstroke}%
\pgfsetdash{}{0pt}%
\pgfsys@defobject{currentmarker}{\pgfqpoint{-0.048611in}{0.000000in}}{\pgfqpoint{0.000000in}{0.000000in}}{%
\pgfpathmoveto{\pgfqpoint{0.000000in}{0.000000in}}%
\pgfpathlineto{\pgfqpoint{-0.048611in}{0.000000in}}%
\pgfusepath{stroke,fill}%
}%
\begin{pgfscope}%
\pgfsys@transformshift{0.800000in}{1.476542in}%
\pgfsys@useobject{currentmarker}{}%
\end{pgfscope}%
\end{pgfscope}%
\begin{pgfscope}%
\definecolor{textcolor}{rgb}{0.000000,0.000000,0.000000}%
\pgfsetstrokecolor{textcolor}%
\pgfsetfillcolor{textcolor}%
\pgftext[x=0.494444in,y=1.423780in,left,base]{\color{textcolor}\sffamily\fontsize{10.000000}{12.000000}\selectfont \(\displaystyle 500\)}%
\end{pgfscope}%
\begin{pgfscope}%
\pgfsetbuttcap%
\pgfsetroundjoin%
\definecolor{currentfill}{rgb}{0.000000,0.000000,0.000000}%
\pgfsetfillcolor{currentfill}%
\pgfsetlinewidth{0.803000pt}%
\definecolor{currentstroke}{rgb}{0.000000,0.000000,0.000000}%
\pgfsetstrokecolor{currentstroke}%
\pgfsetdash{}{0pt}%
\pgfsys@defobject{currentmarker}{\pgfqpoint{-0.048611in}{0.000000in}}{\pgfqpoint{0.000000in}{0.000000in}}{%
\pgfpathmoveto{\pgfqpoint{0.000000in}{0.000000in}}%
\pgfpathlineto{\pgfqpoint{-0.048611in}{0.000000in}}%
\pgfusepath{stroke,fill}%
}%
\begin{pgfscope}%
\pgfsys@transformshift{0.800000in}{1.875591in}%
\pgfsys@useobject{currentmarker}{}%
\end{pgfscope}%
\end{pgfscope}%
\begin{pgfscope}%
\definecolor{textcolor}{rgb}{0.000000,0.000000,0.000000}%
\pgfsetstrokecolor{textcolor}%
\pgfsetfillcolor{textcolor}%
\pgftext[x=0.494444in,y=1.822830in,left,base]{\color{textcolor}\sffamily\fontsize{10.000000}{12.000000}\selectfont \(\displaystyle 750\)}%
\end{pgfscope}%
\begin{pgfscope}%
\pgfsetbuttcap%
\pgfsetroundjoin%
\definecolor{currentfill}{rgb}{0.000000,0.000000,0.000000}%
\pgfsetfillcolor{currentfill}%
\pgfsetlinewidth{0.803000pt}%
\definecolor{currentstroke}{rgb}{0.000000,0.000000,0.000000}%
\pgfsetstrokecolor{currentstroke}%
\pgfsetdash{}{0pt}%
\pgfsys@defobject{currentmarker}{\pgfqpoint{-0.048611in}{0.000000in}}{\pgfqpoint{0.000000in}{0.000000in}}{%
\pgfpathmoveto{\pgfqpoint{0.000000in}{0.000000in}}%
\pgfpathlineto{\pgfqpoint{-0.048611in}{0.000000in}}%
\pgfusepath{stroke,fill}%
}%
\begin{pgfscope}%
\pgfsys@transformshift{0.800000in}{2.274641in}%
\pgfsys@useobject{currentmarker}{}%
\end{pgfscope}%
\end{pgfscope}%
\begin{pgfscope}%
\definecolor{textcolor}{rgb}{0.000000,0.000000,0.000000}%
\pgfsetstrokecolor{textcolor}%
\pgfsetfillcolor{textcolor}%
\pgftext[x=0.424999in,y=2.221880in,left,base]{\color{textcolor}\sffamily\fontsize{10.000000}{12.000000}\selectfont \(\displaystyle 1000\)}%
\end{pgfscope}%
\begin{pgfscope}%
\pgfsetbuttcap%
\pgfsetroundjoin%
\definecolor{currentfill}{rgb}{0.000000,0.000000,0.000000}%
\pgfsetfillcolor{currentfill}%
\pgfsetlinewidth{0.803000pt}%
\definecolor{currentstroke}{rgb}{0.000000,0.000000,0.000000}%
\pgfsetstrokecolor{currentstroke}%
\pgfsetdash{}{0pt}%
\pgfsys@defobject{currentmarker}{\pgfqpoint{-0.048611in}{0.000000in}}{\pgfqpoint{0.000000in}{0.000000in}}{%
\pgfpathmoveto{\pgfqpoint{0.000000in}{0.000000in}}%
\pgfpathlineto{\pgfqpoint{-0.048611in}{0.000000in}}%
\pgfusepath{stroke,fill}%
}%
\begin{pgfscope}%
\pgfsys@transformshift{0.800000in}{2.673691in}%
\pgfsys@useobject{currentmarker}{}%
\end{pgfscope}%
\end{pgfscope}%
\begin{pgfscope}%
\definecolor{textcolor}{rgb}{0.000000,0.000000,0.000000}%
\pgfsetstrokecolor{textcolor}%
\pgfsetfillcolor{textcolor}%
\pgftext[x=0.424999in,y=2.620930in,left,base]{\color{textcolor}\sffamily\fontsize{10.000000}{12.000000}\selectfont \(\displaystyle 1250\)}%
\end{pgfscope}%
\begin{pgfscope}%
\pgfsetbuttcap%
\pgfsetroundjoin%
\definecolor{currentfill}{rgb}{0.000000,0.000000,0.000000}%
\pgfsetfillcolor{currentfill}%
\pgfsetlinewidth{0.803000pt}%
\definecolor{currentstroke}{rgb}{0.000000,0.000000,0.000000}%
\pgfsetstrokecolor{currentstroke}%
\pgfsetdash{}{0pt}%
\pgfsys@defobject{currentmarker}{\pgfqpoint{-0.048611in}{0.000000in}}{\pgfqpoint{0.000000in}{0.000000in}}{%
\pgfpathmoveto{\pgfqpoint{0.000000in}{0.000000in}}%
\pgfpathlineto{\pgfqpoint{-0.048611in}{0.000000in}}%
\pgfusepath{stroke,fill}%
}%
\begin{pgfscope}%
\pgfsys@transformshift{0.800000in}{3.072741in}%
\pgfsys@useobject{currentmarker}{}%
\end{pgfscope}%
\end{pgfscope}%
\begin{pgfscope}%
\definecolor{textcolor}{rgb}{0.000000,0.000000,0.000000}%
\pgfsetstrokecolor{textcolor}%
\pgfsetfillcolor{textcolor}%
\pgftext[x=0.424999in,y=3.019980in,left,base]{\color{textcolor}\sffamily\fontsize{10.000000}{12.000000}\selectfont \(\displaystyle 1500\)}%
\end{pgfscope}%
\begin{pgfscope}%
\pgfsetbuttcap%
\pgfsetroundjoin%
\definecolor{currentfill}{rgb}{0.000000,0.000000,0.000000}%
\pgfsetfillcolor{currentfill}%
\pgfsetlinewidth{0.803000pt}%
\definecolor{currentstroke}{rgb}{0.000000,0.000000,0.000000}%
\pgfsetstrokecolor{currentstroke}%
\pgfsetdash{}{0pt}%
\pgfsys@defobject{currentmarker}{\pgfqpoint{-0.048611in}{0.000000in}}{\pgfqpoint{0.000000in}{0.000000in}}{%
\pgfpathmoveto{\pgfqpoint{0.000000in}{0.000000in}}%
\pgfpathlineto{\pgfqpoint{-0.048611in}{0.000000in}}%
\pgfusepath{stroke,fill}%
}%
\begin{pgfscope}%
\pgfsys@transformshift{0.800000in}{3.471791in}%
\pgfsys@useobject{currentmarker}{}%
\end{pgfscope}%
\end{pgfscope}%
\begin{pgfscope}%
\definecolor{textcolor}{rgb}{0.000000,0.000000,0.000000}%
\pgfsetstrokecolor{textcolor}%
\pgfsetfillcolor{textcolor}%
\pgftext[x=0.424999in,y=3.419029in,left,base]{\color{textcolor}\sffamily\fontsize{10.000000}{12.000000}\selectfont \(\displaystyle 1750\)}%
\end{pgfscope}%
\begin{pgfscope}%
\pgfsetbuttcap%
\pgfsetroundjoin%
\definecolor{currentfill}{rgb}{0.000000,0.000000,0.000000}%
\pgfsetfillcolor{currentfill}%
\pgfsetlinewidth{0.803000pt}%
\definecolor{currentstroke}{rgb}{0.000000,0.000000,0.000000}%
\pgfsetstrokecolor{currentstroke}%
\pgfsetdash{}{0pt}%
\pgfsys@defobject{currentmarker}{\pgfqpoint{-0.048611in}{0.000000in}}{\pgfqpoint{0.000000in}{0.000000in}}{%
\pgfpathmoveto{\pgfqpoint{0.000000in}{0.000000in}}%
\pgfpathlineto{\pgfqpoint{-0.048611in}{0.000000in}}%
\pgfusepath{stroke,fill}%
}%
\begin{pgfscope}%
\pgfsys@transformshift{0.800000in}{3.870841in}%
\pgfsys@useobject{currentmarker}{}%
\end{pgfscope}%
\end{pgfscope}%
\begin{pgfscope}%
\definecolor{textcolor}{rgb}{0.000000,0.000000,0.000000}%
\pgfsetstrokecolor{textcolor}%
\pgfsetfillcolor{textcolor}%
\pgftext[x=0.424999in,y=3.818079in,left,base]{\color{textcolor}\sffamily\fontsize{10.000000}{12.000000}\selectfont \(\displaystyle 2000\)}%
\end{pgfscope}%
\begin{pgfscope}%
\definecolor{textcolor}{rgb}{0.000000,0.000000,0.000000}%
\pgfsetstrokecolor{textcolor}%
\pgfsetfillcolor{textcolor}%
\pgftext[x=0.369444in,y=2.376000in,,bottom,rotate=90.000000]{\color{textcolor}\sffamily\fontsize{10.000000}{12.000000}\selectfont Number of GMRES iterations}%
\end{pgfscope}%
\begin{pgfscope}%
\pgfpathrectangle{\pgfqpoint{0.800000in}{0.528000in}}{\pgfqpoint{4.960000in}{3.696000in}}%
\pgfusepath{clip}%
\pgfsetbuttcap%
\pgfsetroundjoin%
\pgfsetlinewidth{1.505625pt}%
\definecolor{currentstroke}{rgb}{0.843137,0.000000,0.000000}%
\pgfsetstrokecolor{currentstroke}%
\pgfsetdash{{5.550000pt}{2.400000pt}}{0.000000pt}%
\pgfpathmoveto{\pgfqpoint{1.250909in}{0.700789in}}%
\pgfpathlineto{\pgfqpoint{1.701818in}{0.742290in}}%
\pgfpathlineto{\pgfqpoint{2.152727in}{0.868390in}}%
\pgfpathlineto{\pgfqpoint{2.603636in}{1.090261in}}%
\pgfpathlineto{\pgfqpoint{3.054545in}{1.360019in}}%
\pgfpathlineto{\pgfqpoint{3.505455in}{1.679259in}}%
\pgfpathlineto{\pgfqpoint{3.956364in}{2.178869in}}%
\pgfpathlineto{\pgfqpoint{4.407273in}{2.712000in}}%
\pgfpathlineto{\pgfqpoint{4.858182in}{3.384000in}}%
\pgfpathlineto{\pgfqpoint{5.309091in}{4.056000in}}%
\pgfusepath{stroke}%
\end{pgfscope}%
\begin{pgfscope}%
\pgfpathrectangle{\pgfqpoint{0.800000in}{0.528000in}}{\pgfqpoint{4.960000in}{3.696000in}}%
\pgfusepath{clip}%
\pgfsetbuttcap%
\pgfsetroundjoin%
\definecolor{currentfill}{rgb}{0.843137,0.000000,0.000000}%
\pgfsetfillcolor{currentfill}%
\pgfsetlinewidth{1.003750pt}%
\definecolor{currentstroke}{rgb}{0.843137,0.000000,0.000000}%
\pgfsetstrokecolor{currentstroke}%
\pgfsetdash{}{0pt}%
\pgfsys@defobject{currentmarker}{\pgfqpoint{-0.041667in}{-0.041667in}}{\pgfqpoint{0.041667in}{0.041667in}}{%
\pgfpathmoveto{\pgfqpoint{0.000000in}{-0.041667in}}%
\pgfpathcurveto{\pgfqpoint{0.011050in}{-0.041667in}}{\pgfqpoint{0.021649in}{-0.037276in}}{\pgfqpoint{0.029463in}{-0.029463in}}%
\pgfpathcurveto{\pgfqpoint{0.037276in}{-0.021649in}}{\pgfqpoint{0.041667in}{-0.011050in}}{\pgfqpoint{0.041667in}{0.000000in}}%
\pgfpathcurveto{\pgfqpoint{0.041667in}{0.011050in}}{\pgfqpoint{0.037276in}{0.021649in}}{\pgfqpoint{0.029463in}{0.029463in}}%
\pgfpathcurveto{\pgfqpoint{0.021649in}{0.037276in}}{\pgfqpoint{0.011050in}{0.041667in}}{\pgfqpoint{0.000000in}{0.041667in}}%
\pgfpathcurveto{\pgfqpoint{-0.011050in}{0.041667in}}{\pgfqpoint{-0.021649in}{0.037276in}}{\pgfqpoint{-0.029463in}{0.029463in}}%
\pgfpathcurveto{\pgfqpoint{-0.037276in}{0.021649in}}{\pgfqpoint{-0.041667in}{0.011050in}}{\pgfqpoint{-0.041667in}{0.000000in}}%
\pgfpathcurveto{\pgfqpoint{-0.041667in}{-0.011050in}}{\pgfqpoint{-0.037276in}{-0.021649in}}{\pgfqpoint{-0.029463in}{-0.029463in}}%
\pgfpathcurveto{\pgfqpoint{-0.021649in}{-0.037276in}}{\pgfqpoint{-0.011050in}{-0.041667in}}{\pgfqpoint{0.000000in}{-0.041667in}}%
\pgfpathclose%
\pgfusepath{stroke,fill}%
}%
\begin{pgfscope}%
\pgfsys@transformshift{1.250909in}{0.700789in}%
\pgfsys@useobject{currentmarker}{}%
\end{pgfscope}%
\begin{pgfscope}%
\pgfsys@transformshift{1.701818in}{0.742290in}%
\pgfsys@useobject{currentmarker}{}%
\end{pgfscope}%
\begin{pgfscope}%
\pgfsys@transformshift{2.152727in}{0.868390in}%
\pgfsys@useobject{currentmarker}{}%
\end{pgfscope}%
\begin{pgfscope}%
\pgfsys@transformshift{2.603636in}{1.090261in}%
\pgfsys@useobject{currentmarker}{}%
\end{pgfscope}%
\begin{pgfscope}%
\pgfsys@transformshift{3.054545in}{1.360019in}%
\pgfsys@useobject{currentmarker}{}%
\end{pgfscope}%
\begin{pgfscope}%
\pgfsys@transformshift{3.505455in}{1.679259in}%
\pgfsys@useobject{currentmarker}{}%
\end{pgfscope}%
\begin{pgfscope}%
\pgfsys@transformshift{3.956364in}{2.178869in}%
\pgfsys@useobject{currentmarker}{}%
\end{pgfscope}%
\begin{pgfscope}%
\pgfsys@transformshift{4.407273in}{2.712000in}%
\pgfsys@useobject{currentmarker}{}%
\end{pgfscope}%
\begin{pgfscope}%
\pgfsys@transformshift{4.858182in}{3.384000in}%
\pgfsys@useobject{currentmarker}{}%
\end{pgfscope}%
\begin{pgfscope}%
\pgfsys@transformshift{5.309091in}{4.056000in}%
\pgfsys@useobject{currentmarker}{}%
\end{pgfscope}%
\end{pgfscope}%
\begin{pgfscope}%
\pgfpathrectangle{\pgfqpoint{0.800000in}{0.528000in}}{\pgfqpoint{4.960000in}{3.696000in}}%
\pgfusepath{clip}%
\pgfsetbuttcap%
\pgfsetroundjoin%
\pgfsetlinewidth{1.505625pt}%
\definecolor{currentstroke}{rgb}{0.549020,0.235294,1.000000}%
\pgfsetstrokecolor{currentstroke}%
\pgfsetdash{{5.550000pt}{2.400000pt}}{0.000000pt}%
\pgfpathmoveto{\pgfqpoint{1.250909in}{0.699192in}}%
\pgfpathlineto{\pgfqpoint{1.701818in}{0.721539in}}%
\pgfpathlineto{\pgfqpoint{2.152727in}{0.790176in}}%
\pgfpathlineto{\pgfqpoint{2.603636in}{0.913083in}}%
\pgfpathlineto{\pgfqpoint{3.054545in}{1.096646in}}%
\pgfpathlineto{\pgfqpoint{3.505455in}{1.307344in}}%
\pgfpathlineto{\pgfqpoint{3.956364in}{1.620200in}}%
\pgfpathlineto{\pgfqpoint{4.407273in}{1.995306in}}%
\pgfpathlineto{\pgfqpoint{4.858182in}{2.478955in}}%
\pgfpathlineto{\pgfqpoint{5.309091in}{2.901948in}}%
\pgfusepath{stroke}%
\end{pgfscope}%
\begin{pgfscope}%
\pgfpathrectangle{\pgfqpoint{0.800000in}{0.528000in}}{\pgfqpoint{4.960000in}{3.696000in}}%
\pgfusepath{clip}%
\pgfsetbuttcap%
\pgfsetmiterjoin%
\definecolor{currentfill}{rgb}{0.549020,0.235294,1.000000}%
\pgfsetfillcolor{currentfill}%
\pgfsetlinewidth{1.003750pt}%
\definecolor{currentstroke}{rgb}{0.549020,0.235294,1.000000}%
\pgfsetstrokecolor{currentstroke}%
\pgfsetdash{}{0pt}%
\pgfsys@defobject{currentmarker}{\pgfqpoint{-0.041667in}{-0.041667in}}{\pgfqpoint{0.041667in}{0.041667in}}{%
\pgfpathmoveto{\pgfqpoint{0.000000in}{0.041667in}}%
\pgfpathlineto{\pgfqpoint{-0.041667in}{-0.041667in}}%
\pgfpathlineto{\pgfqpoint{0.041667in}{-0.041667in}}%
\pgfpathclose%
\pgfusepath{stroke,fill}%
}%
\begin{pgfscope}%
\pgfsys@transformshift{1.250909in}{0.699192in}%
\pgfsys@useobject{currentmarker}{}%
\end{pgfscope}%
\begin{pgfscope}%
\pgfsys@transformshift{1.701818in}{0.721539in}%
\pgfsys@useobject{currentmarker}{}%
\end{pgfscope}%
\begin{pgfscope}%
\pgfsys@transformshift{2.152727in}{0.790176in}%
\pgfsys@useobject{currentmarker}{}%
\end{pgfscope}%
\begin{pgfscope}%
\pgfsys@transformshift{2.603636in}{0.913083in}%
\pgfsys@useobject{currentmarker}{}%
\end{pgfscope}%
\begin{pgfscope}%
\pgfsys@transformshift{3.054545in}{1.096646in}%
\pgfsys@useobject{currentmarker}{}%
\end{pgfscope}%
\begin{pgfscope}%
\pgfsys@transformshift{3.505455in}{1.307344in}%
\pgfsys@useobject{currentmarker}{}%
\end{pgfscope}%
\begin{pgfscope}%
\pgfsys@transformshift{3.956364in}{1.620200in}%
\pgfsys@useobject{currentmarker}{}%
\end{pgfscope}%
\begin{pgfscope}%
\pgfsys@transformshift{4.407273in}{1.995306in}%
\pgfsys@useobject{currentmarker}{}%
\end{pgfscope}%
\begin{pgfscope}%
\pgfsys@transformshift{4.858182in}{2.478955in}%
\pgfsys@useobject{currentmarker}{}%
\end{pgfscope}%
\begin{pgfscope}%
\pgfsys@transformshift{5.309091in}{2.901948in}%
\pgfsys@useobject{currentmarker}{}%
\end{pgfscope}%
\end{pgfscope}%
\begin{pgfscope}%
\pgfpathrectangle{\pgfqpoint{0.800000in}{0.528000in}}{\pgfqpoint{4.960000in}{3.696000in}}%
\pgfusepath{clip}%
\pgfsetbuttcap%
\pgfsetroundjoin%
\pgfsetlinewidth{1.505625pt}%
\definecolor{currentstroke}{rgb}{0.007843,0.533333,0.000000}%
\pgfsetstrokecolor{currentstroke}%
\pgfsetdash{{5.550000pt}{2.400000pt}}{0.000000pt}%
\pgfpathmoveto{\pgfqpoint{1.250909in}{0.697596in}}%
\pgfpathlineto{\pgfqpoint{1.701818in}{0.713558in}}%
\pgfpathlineto{\pgfqpoint{2.152727in}{0.742290in}}%
\pgfpathlineto{\pgfqpoint{2.603636in}{0.801349in}}%
\pgfpathlineto{\pgfqpoint{3.054545in}{0.892333in}}%
\pgfpathlineto{\pgfqpoint{3.505455in}{0.996086in}}%
\pgfpathlineto{\pgfqpoint{3.956364in}{1.144532in}}%
\pgfpathlineto{\pgfqpoint{4.407273in}{1.347249in}}%
\pgfpathlineto{\pgfqpoint{4.858182in}{1.557948in}}%
\pgfpathlineto{\pgfqpoint{5.309091in}{1.837283in}}%
\pgfusepath{stroke}%
\end{pgfscope}%
\begin{pgfscope}%
\pgfpathrectangle{\pgfqpoint{0.800000in}{0.528000in}}{\pgfqpoint{4.960000in}{3.696000in}}%
\pgfusepath{clip}%
\pgfsetbuttcap%
\pgfsetmiterjoin%
\definecolor{currentfill}{rgb}{0.007843,0.533333,0.000000}%
\pgfsetfillcolor{currentfill}%
\pgfsetlinewidth{1.003750pt}%
\definecolor{currentstroke}{rgb}{0.007843,0.533333,0.000000}%
\pgfsetstrokecolor{currentstroke}%
\pgfsetdash{}{0pt}%
\pgfsys@defobject{currentmarker}{\pgfqpoint{-0.041667in}{-0.041667in}}{\pgfqpoint{0.041667in}{0.041667in}}{%
\pgfpathmoveto{\pgfqpoint{-0.000000in}{-0.041667in}}%
\pgfpathlineto{\pgfqpoint{0.041667in}{0.041667in}}%
\pgfpathlineto{\pgfqpoint{-0.041667in}{0.041667in}}%
\pgfpathclose%
\pgfusepath{stroke,fill}%
}%
\begin{pgfscope}%
\pgfsys@transformshift{1.250909in}{0.697596in}%
\pgfsys@useobject{currentmarker}{}%
\end{pgfscope}%
\begin{pgfscope}%
\pgfsys@transformshift{1.701818in}{0.713558in}%
\pgfsys@useobject{currentmarker}{}%
\end{pgfscope}%
\begin{pgfscope}%
\pgfsys@transformshift{2.152727in}{0.742290in}%
\pgfsys@useobject{currentmarker}{}%
\end{pgfscope}%
\begin{pgfscope}%
\pgfsys@transformshift{2.603636in}{0.801349in}%
\pgfsys@useobject{currentmarker}{}%
\end{pgfscope}%
\begin{pgfscope}%
\pgfsys@transformshift{3.054545in}{0.892333in}%
\pgfsys@useobject{currentmarker}{}%
\end{pgfscope}%
\begin{pgfscope}%
\pgfsys@transformshift{3.505455in}{0.996086in}%
\pgfsys@useobject{currentmarker}{}%
\end{pgfscope}%
\begin{pgfscope}%
\pgfsys@transformshift{3.956364in}{1.144532in}%
\pgfsys@useobject{currentmarker}{}%
\end{pgfscope}%
\begin{pgfscope}%
\pgfsys@transformshift{4.407273in}{1.347249in}%
\pgfsys@useobject{currentmarker}{}%
\end{pgfscope}%
\begin{pgfscope}%
\pgfsys@transformshift{4.858182in}{1.557948in}%
\pgfsys@useobject{currentmarker}{}%
\end{pgfscope}%
\begin{pgfscope}%
\pgfsys@transformshift{5.309091in}{1.837283in}%
\pgfsys@useobject{currentmarker}{}%
\end{pgfscope}%
\end{pgfscope}%
\begin{pgfscope}%
\pgfpathrectangle{\pgfqpoint{0.800000in}{0.528000in}}{\pgfqpoint{4.960000in}{3.696000in}}%
\pgfusepath{clip}%
\pgfsetbuttcap%
\pgfsetroundjoin%
\pgfsetlinewidth{1.505625pt}%
\definecolor{currentstroke}{rgb}{0.000000,0.674510,0.780392}%
\pgfsetstrokecolor{currentstroke}%
\pgfsetdash{{5.550000pt}{2.400000pt}}{0.000000pt}%
\pgfpathmoveto{\pgfqpoint{1.250909in}{0.696000in}}%
\pgfpathlineto{\pgfqpoint{1.701818in}{0.707173in}}%
\pgfpathlineto{\pgfqpoint{2.152727in}{0.718347in}}%
\pgfpathlineto{\pgfqpoint{2.603636in}{0.742290in}}%
\pgfpathlineto{\pgfqpoint{3.054545in}{0.771021in}}%
\pgfpathlineto{\pgfqpoint{3.505455in}{0.815715in}}%
\pgfpathlineto{\pgfqpoint{3.956364in}{0.868390in}}%
\pgfpathlineto{\pgfqpoint{4.407273in}{0.938622in}}%
\pgfpathlineto{\pgfqpoint{4.858182in}{1.012048in}}%
\pgfpathlineto{\pgfqpoint{5.309091in}{1.109416in}}%
\pgfusepath{stroke}%
\end{pgfscope}%
\begin{pgfscope}%
\pgfpathrectangle{\pgfqpoint{0.800000in}{0.528000in}}{\pgfqpoint{4.960000in}{3.696000in}}%
\pgfusepath{clip}%
\pgfsetbuttcap%
\pgfsetmiterjoin%
\definecolor{currentfill}{rgb}{0.000000,0.674510,0.780392}%
\pgfsetfillcolor{currentfill}%
\pgfsetlinewidth{1.003750pt}%
\definecolor{currentstroke}{rgb}{0.000000,0.674510,0.780392}%
\pgfsetstrokecolor{currentstroke}%
\pgfsetdash{}{0pt}%
\pgfsys@defobject{currentmarker}{\pgfqpoint{-0.041667in}{-0.041667in}}{\pgfqpoint{0.041667in}{0.041667in}}{%
\pgfpathmoveto{\pgfqpoint{0.041667in}{-0.000000in}}%
\pgfpathlineto{\pgfqpoint{-0.041667in}{0.041667in}}%
\pgfpathlineto{\pgfqpoint{-0.041667in}{-0.041667in}}%
\pgfpathclose%
\pgfusepath{stroke,fill}%
}%
\begin{pgfscope}%
\pgfsys@transformshift{1.250909in}{0.696000in}%
\pgfsys@useobject{currentmarker}{}%
\end{pgfscope}%
\begin{pgfscope}%
\pgfsys@transformshift{1.701818in}{0.707173in}%
\pgfsys@useobject{currentmarker}{}%
\end{pgfscope}%
\begin{pgfscope}%
\pgfsys@transformshift{2.152727in}{0.718347in}%
\pgfsys@useobject{currentmarker}{}%
\end{pgfscope}%
\begin{pgfscope}%
\pgfsys@transformshift{2.603636in}{0.742290in}%
\pgfsys@useobject{currentmarker}{}%
\end{pgfscope}%
\begin{pgfscope}%
\pgfsys@transformshift{3.054545in}{0.771021in}%
\pgfsys@useobject{currentmarker}{}%
\end{pgfscope}%
\begin{pgfscope}%
\pgfsys@transformshift{3.505455in}{0.815715in}%
\pgfsys@useobject{currentmarker}{}%
\end{pgfscope}%
\begin{pgfscope}%
\pgfsys@transformshift{3.956364in}{0.868390in}%
\pgfsys@useobject{currentmarker}{}%
\end{pgfscope}%
\begin{pgfscope}%
\pgfsys@transformshift{4.407273in}{0.938622in}%
\pgfsys@useobject{currentmarker}{}%
\end{pgfscope}%
\begin{pgfscope}%
\pgfsys@transformshift{4.858182in}{1.012048in}%
\pgfsys@useobject{currentmarker}{}%
\end{pgfscope}%
\begin{pgfscope}%
\pgfsys@transformshift{5.309091in}{1.109416in}%
\pgfsys@useobject{currentmarker}{}%
\end{pgfscope}%
\end{pgfscope}%
\begin{pgfscope}%
\pgfsetrectcap%
\pgfsetmiterjoin%
\pgfsetlinewidth{0.803000pt}%
\definecolor{currentstroke}{rgb}{0.000000,0.000000,0.000000}%
\pgfsetstrokecolor{currentstroke}%
\pgfsetdash{}{0pt}%
\pgfpathmoveto{\pgfqpoint{0.800000in}{0.528000in}}%
\pgfpathlineto{\pgfqpoint{0.800000in}{4.224000in}}%
\pgfusepath{stroke}%
\end{pgfscope}%
\begin{pgfscope}%
\pgfsetrectcap%
\pgfsetmiterjoin%
\pgfsetlinewidth{0.000000pt}%
\definecolor{currentstroke}{rgb}{0.000000,0.000000,0.000000}%
\pgfsetstrokecolor{currentstroke}%
\pgfsetstrokeopacity{0.000000}%
\pgfsetdash{}{0pt}%
\pgfpathmoveto{\pgfqpoint{5.760000in}{0.528000in}}%
\pgfpathlineto{\pgfqpoint{5.760000in}{4.224000in}}%
\pgfusepath{}%
\end{pgfscope}%
\begin{pgfscope}%
\pgfsetrectcap%
\pgfsetmiterjoin%
\pgfsetlinewidth{0.803000pt}%
\definecolor{currentstroke}{rgb}{0.000000,0.000000,0.000000}%
\pgfsetstrokecolor{currentstroke}%
\pgfsetdash{}{0pt}%
\pgfpathmoveto{\pgfqpoint{0.800000in}{0.528000in}}%
\pgfpathlineto{\pgfqpoint{5.760000in}{0.528000in}}%
\pgfusepath{stroke}%
\end{pgfscope}%
\begin{pgfscope}%
\pgfsetrectcap%
\pgfsetmiterjoin%
\pgfsetlinewidth{0.000000pt}%
\definecolor{currentstroke}{rgb}{0.000000,0.000000,0.000000}%
\pgfsetstrokecolor{currentstroke}%
\pgfsetstrokeopacity{0.000000}%
\pgfsetdash{}{0pt}%
\pgfpathmoveto{\pgfqpoint{0.800000in}{4.224000in}}%
\pgfpathlineto{\pgfqpoint{5.760000in}{4.224000in}}%
\pgfusepath{}%
\end{pgfscope}%
\begin{pgfscope}%
\pgfsetbuttcap%
\pgfsetmiterjoin%
\definecolor{currentfill}{rgb}{1.000000,1.000000,1.000000}%
\pgfsetfillcolor{currentfill}%
\pgfsetfillopacity{0.800000}%
\pgfsetlinewidth{1.003750pt}%
\definecolor{currentstroke}{rgb}{0.800000,0.800000,0.800000}%
\pgfsetstrokecolor{currentstroke}%
\pgfsetstrokeopacity{0.800000}%
\pgfsetdash{}{0pt}%
\pgfpathmoveto{\pgfqpoint{0.897222in}{3.297460in}}%
\pgfpathlineto{\pgfqpoint{1.790209in}{3.297460in}}%
\pgfpathquadraticcurveto{\pgfqpoint{1.817987in}{3.297460in}}{\pgfqpoint{1.817987in}{3.325238in}}%
\pgfpathlineto{\pgfqpoint{1.817987in}{4.126778in}}%
\pgfpathquadraticcurveto{\pgfqpoint{1.817987in}{4.154556in}}{\pgfqpoint{1.790209in}{4.154556in}}%
\pgfpathlineto{\pgfqpoint{0.897222in}{4.154556in}}%
\pgfpathquadraticcurveto{\pgfqpoint{0.869444in}{4.154556in}}{\pgfqpoint{0.869444in}{4.126778in}}%
\pgfpathlineto{\pgfqpoint{0.869444in}{3.325238in}}%
\pgfpathquadraticcurveto{\pgfqpoint{0.869444in}{3.297460in}}{\pgfqpoint{0.897222in}{3.297460in}}%
\pgfpathclose%
\pgfusepath{stroke,fill}%
\end{pgfscope}%
\begin{pgfscope}%
\pgfsetbuttcap%
\pgfsetroundjoin%
\pgfsetlinewidth{1.505625pt}%
\definecolor{currentstroke}{rgb}{0.843137,0.000000,0.000000}%
\pgfsetstrokecolor{currentstroke}%
\pgfsetdash{{5.550000pt}{2.400000pt}}{0.000000pt}%
\pgfpathmoveto{\pgfqpoint{0.925000in}{4.042088in}}%
\pgfpathlineto{\pgfqpoint{1.202778in}{4.042088in}}%
\pgfusepath{stroke}%
\end{pgfscope}%
\begin{pgfscope}%
\pgfsetbuttcap%
\pgfsetroundjoin%
\definecolor{currentfill}{rgb}{0.843137,0.000000,0.000000}%
\pgfsetfillcolor{currentfill}%
\pgfsetlinewidth{1.003750pt}%
\definecolor{currentstroke}{rgb}{0.843137,0.000000,0.000000}%
\pgfsetstrokecolor{currentstroke}%
\pgfsetdash{}{0pt}%
\pgfsys@defobject{currentmarker}{\pgfqpoint{-0.041667in}{-0.041667in}}{\pgfqpoint{0.041667in}{0.041667in}}{%
\pgfpathmoveto{\pgfqpoint{0.000000in}{-0.041667in}}%
\pgfpathcurveto{\pgfqpoint{0.011050in}{-0.041667in}}{\pgfqpoint{0.021649in}{-0.037276in}}{\pgfqpoint{0.029463in}{-0.029463in}}%
\pgfpathcurveto{\pgfqpoint{0.037276in}{-0.021649in}}{\pgfqpoint{0.041667in}{-0.011050in}}{\pgfqpoint{0.041667in}{0.000000in}}%
\pgfpathcurveto{\pgfqpoint{0.041667in}{0.011050in}}{\pgfqpoint{0.037276in}{0.021649in}}{\pgfqpoint{0.029463in}{0.029463in}}%
\pgfpathcurveto{\pgfqpoint{0.021649in}{0.037276in}}{\pgfqpoint{0.011050in}{0.041667in}}{\pgfqpoint{0.000000in}{0.041667in}}%
\pgfpathcurveto{\pgfqpoint{-0.011050in}{0.041667in}}{\pgfqpoint{-0.021649in}{0.037276in}}{\pgfqpoint{-0.029463in}{0.029463in}}%
\pgfpathcurveto{\pgfqpoint{-0.037276in}{0.021649in}}{\pgfqpoint{-0.041667in}{0.011050in}}{\pgfqpoint{-0.041667in}{0.000000in}}%
\pgfpathcurveto{\pgfqpoint{-0.041667in}{-0.011050in}}{\pgfqpoint{-0.037276in}{-0.021649in}}{\pgfqpoint{-0.029463in}{-0.029463in}}%
\pgfpathcurveto{\pgfqpoint{-0.021649in}{-0.037276in}}{\pgfqpoint{-0.011050in}{-0.041667in}}{\pgfqpoint{0.000000in}{-0.041667in}}%
\pgfpathclose%
\pgfusepath{stroke,fill}%
}%
\begin{pgfscope}%
\pgfsys@transformshift{1.063889in}{4.042088in}%
\pgfsys@useobject{currentmarker}{}%
\end{pgfscope}%
\end{pgfscope}%
\begin{pgfscope}%
\definecolor{textcolor}{rgb}{0.000000,0.000000,0.000000}%
\pgfsetstrokecolor{textcolor}%
\pgfsetfillcolor{textcolor}%
\pgftext[x=1.313889in,y=3.993477in,left,base]{\color{textcolor}\sffamily\fontsize{10.000000}{12.000000}\selectfont \(\displaystyle \beta = 0\)}%
\end{pgfscope}%
\begin{pgfscope}%
\pgfsetbuttcap%
\pgfsetroundjoin%
\pgfsetlinewidth{1.505625pt}%
\definecolor{currentstroke}{rgb}{0.549020,0.235294,1.000000}%
\pgfsetstrokecolor{currentstroke}%
\pgfsetdash{{5.550000pt}{2.400000pt}}{0.000000pt}%
\pgfpathmoveto{\pgfqpoint{0.925000in}{3.838231in}}%
\pgfpathlineto{\pgfqpoint{1.202778in}{3.838231in}}%
\pgfusepath{stroke}%
\end{pgfscope}%
\begin{pgfscope}%
\pgfsetbuttcap%
\pgfsetmiterjoin%
\definecolor{currentfill}{rgb}{0.549020,0.235294,1.000000}%
\pgfsetfillcolor{currentfill}%
\pgfsetlinewidth{1.003750pt}%
\definecolor{currentstroke}{rgb}{0.549020,0.235294,1.000000}%
\pgfsetstrokecolor{currentstroke}%
\pgfsetdash{}{0pt}%
\pgfsys@defobject{currentmarker}{\pgfqpoint{-0.041667in}{-0.041667in}}{\pgfqpoint{0.041667in}{0.041667in}}{%
\pgfpathmoveto{\pgfqpoint{0.000000in}{0.041667in}}%
\pgfpathlineto{\pgfqpoint{-0.041667in}{-0.041667in}}%
\pgfpathlineto{\pgfqpoint{0.041667in}{-0.041667in}}%
\pgfpathclose%
\pgfusepath{stroke,fill}%
}%
\begin{pgfscope}%
\pgfsys@transformshift{1.063889in}{3.838231in}%
\pgfsys@useobject{currentmarker}{}%
\end{pgfscope}%
\end{pgfscope}%
\begin{pgfscope}%
\definecolor{textcolor}{rgb}{0.000000,0.000000,0.000000}%
\pgfsetstrokecolor{textcolor}%
\pgfsetfillcolor{textcolor}%
\pgftext[x=1.313889in,y=3.789620in,left,base]{\color{textcolor}\sffamily\fontsize{10.000000}{12.000000}\selectfont \(\displaystyle \beta = 0.1\)}%
\end{pgfscope}%
\begin{pgfscope}%
\pgfsetbuttcap%
\pgfsetroundjoin%
\pgfsetlinewidth{1.505625pt}%
\definecolor{currentstroke}{rgb}{0.007843,0.533333,0.000000}%
\pgfsetstrokecolor{currentstroke}%
\pgfsetdash{{5.550000pt}{2.400000pt}}{0.000000pt}%
\pgfpathmoveto{\pgfqpoint{0.925000in}{3.634374in}}%
\pgfpathlineto{\pgfqpoint{1.202778in}{3.634374in}}%
\pgfusepath{stroke}%
\end{pgfscope}%
\begin{pgfscope}%
\pgfsetbuttcap%
\pgfsetmiterjoin%
\definecolor{currentfill}{rgb}{0.007843,0.533333,0.000000}%
\pgfsetfillcolor{currentfill}%
\pgfsetlinewidth{1.003750pt}%
\definecolor{currentstroke}{rgb}{0.007843,0.533333,0.000000}%
\pgfsetstrokecolor{currentstroke}%
\pgfsetdash{}{0pt}%
\pgfsys@defobject{currentmarker}{\pgfqpoint{-0.041667in}{-0.041667in}}{\pgfqpoint{0.041667in}{0.041667in}}{%
\pgfpathmoveto{\pgfqpoint{-0.000000in}{-0.041667in}}%
\pgfpathlineto{\pgfqpoint{0.041667in}{0.041667in}}%
\pgfpathlineto{\pgfqpoint{-0.041667in}{0.041667in}}%
\pgfpathclose%
\pgfusepath{stroke,fill}%
}%
\begin{pgfscope}%
\pgfsys@transformshift{1.063889in}{3.634374in}%
\pgfsys@useobject{currentmarker}{}%
\end{pgfscope}%
\end{pgfscope}%
\begin{pgfscope}%
\definecolor{textcolor}{rgb}{0.000000,0.000000,0.000000}%
\pgfsetstrokecolor{textcolor}%
\pgfsetfillcolor{textcolor}%
\pgftext[x=1.313889in,y=3.585762in,left,base]{\color{textcolor}\sffamily\fontsize{10.000000}{12.000000}\selectfont \(\displaystyle \beta = 0.2\)}%
\end{pgfscope}%
\begin{pgfscope}%
\pgfsetbuttcap%
\pgfsetroundjoin%
\pgfsetlinewidth{1.505625pt}%
\definecolor{currentstroke}{rgb}{0.000000,0.674510,0.780392}%
\pgfsetstrokecolor{currentstroke}%
\pgfsetdash{{5.550000pt}{2.400000pt}}{0.000000pt}%
\pgfpathmoveto{\pgfqpoint{0.925000in}{3.430516in}}%
\pgfpathlineto{\pgfqpoint{1.202778in}{3.430516in}}%
\pgfusepath{stroke}%
\end{pgfscope}%
\begin{pgfscope}%
\pgfsetbuttcap%
\pgfsetmiterjoin%
\definecolor{currentfill}{rgb}{0.000000,0.674510,0.780392}%
\pgfsetfillcolor{currentfill}%
\pgfsetlinewidth{1.003750pt}%
\definecolor{currentstroke}{rgb}{0.000000,0.674510,0.780392}%
\pgfsetstrokecolor{currentstroke}%
\pgfsetdash{}{0pt}%
\pgfsys@defobject{currentmarker}{\pgfqpoint{-0.041667in}{-0.041667in}}{\pgfqpoint{0.041667in}{0.041667in}}{%
\pgfpathmoveto{\pgfqpoint{0.041667in}{-0.000000in}}%
\pgfpathlineto{\pgfqpoint{-0.041667in}{0.041667in}}%
\pgfpathlineto{\pgfqpoint{-0.041667in}{-0.041667in}}%
\pgfpathclose%
\pgfusepath{stroke,fill}%
}%
\begin{pgfscope}%
\pgfsys@transformshift{1.063889in}{3.430516in}%
\pgfsys@useobject{currentmarker}{}%
\end{pgfscope}%
\end{pgfscope}%
\begin{pgfscope}%
\definecolor{textcolor}{rgb}{0.000000,0.000000,0.000000}%
\pgfsetstrokecolor{textcolor}%
\pgfsetfillcolor{textcolor}%
\pgftext[x=1.313889in,y=3.381905in,left,base]{\color{textcolor}\sffamily\fontsize{10.000000}{12.000000}\selectfont \(\displaystyle \beta = 0.3\)}%
\end{pgfscope}%
\end{pgfpicture}%
\makeatother%
\endgroup%

  \caption{GMRES iteration counts for $\AmatoI\Amatt$ given by \cref{eq:noweak,eq:ntweak}, where $\alpha = 0.2/k^\beta,$ for $\beta = 0,0.1,0.2,0.3.$}\label{fig:l1low}
\end{figure}

\begin{figure}
  %% Creator: Matplotlib, PGF backend
%%
%% To include the figure in your LaTeX document, write
%%   \input{<filename>.pgf}
%%
%% Make sure the required packages are loaded in your preamble
%%   \usepackage{pgf}
%%
%% Figures using additional raster images can only be included by \input if
%% they are in the same directory as the main LaTeX file. For loading figures
%% from other directories you can use the `import` package
%%   \usepackage{import}
%% and then include the figures with
%%   \import{<path to file>}{<filename>.pgf}
%%
%% Matplotlib used the following preamble
%%   \usepackage{fontspec}
%%   \setmainfont{DejaVuSerif.ttf}[Path=/home/owen/progs/firedrake-complex/firedrake/lib/python3.5/site-packages/matplotlib/mpl-data/fonts/ttf/]
%%   \setsansfont{DejaVuSans.ttf}[Path=/home/owen/progs/firedrake-complex/firedrake/lib/python3.5/site-packages/matplotlib/mpl-data/fonts/ttf/]
%%   \setmonofont{DejaVuSansMono.ttf}[Path=/home/owen/progs/firedrake-complex/firedrake/lib/python3.5/site-packages/matplotlib/mpl-data/fonts/ttf/]
%%
\begingroup%
\makeatletter%
\begin{pgfpicture}%
\pgfpathrectangle{\pgfpointorigin}{\pgfqpoint{6.400000in}{4.800000in}}%
\pgfusepath{use as bounding box, clip}%
\begin{pgfscope}%
\pgfsetbuttcap%
\pgfsetmiterjoin%
\definecolor{currentfill}{rgb}{1.000000,1.000000,1.000000}%
\pgfsetfillcolor{currentfill}%
\pgfsetlinewidth{0.000000pt}%
\definecolor{currentstroke}{rgb}{1.000000,1.000000,1.000000}%
\pgfsetstrokecolor{currentstroke}%
\pgfsetdash{}{0pt}%
\pgfpathmoveto{\pgfqpoint{0.000000in}{0.000000in}}%
\pgfpathlineto{\pgfqpoint{6.400000in}{0.000000in}}%
\pgfpathlineto{\pgfqpoint{6.400000in}{4.800000in}}%
\pgfpathlineto{\pgfqpoint{0.000000in}{4.800000in}}%
\pgfpathclose%
\pgfusepath{fill}%
\end{pgfscope}%
\begin{pgfscope}%
\pgfsetbuttcap%
\pgfsetmiterjoin%
\definecolor{currentfill}{rgb}{1.000000,1.000000,1.000000}%
\pgfsetfillcolor{currentfill}%
\pgfsetlinewidth{0.000000pt}%
\definecolor{currentstroke}{rgb}{0.000000,0.000000,0.000000}%
\pgfsetstrokecolor{currentstroke}%
\pgfsetstrokeopacity{0.000000}%
\pgfsetdash{}{0pt}%
\pgfpathmoveto{\pgfqpoint{0.800000in}{0.528000in}}%
\pgfpathlineto{\pgfqpoint{5.760000in}{0.528000in}}%
\pgfpathlineto{\pgfqpoint{5.760000in}{4.224000in}}%
\pgfpathlineto{\pgfqpoint{0.800000in}{4.224000in}}%
\pgfpathclose%
\pgfusepath{fill}%
\end{pgfscope}%
\begin{pgfscope}%
\pgfsetbuttcap%
\pgfsetroundjoin%
\definecolor{currentfill}{rgb}{0.000000,0.000000,0.000000}%
\pgfsetfillcolor{currentfill}%
\pgfsetlinewidth{0.803000pt}%
\definecolor{currentstroke}{rgb}{0.000000,0.000000,0.000000}%
\pgfsetstrokecolor{currentstroke}%
\pgfsetdash{}{0pt}%
\pgfsys@defobject{currentmarker}{\pgfqpoint{0.000000in}{-0.048611in}}{\pgfqpoint{0.000000in}{0.000000in}}{%
\pgfpathmoveto{\pgfqpoint{0.000000in}{0.000000in}}%
\pgfpathlineto{\pgfqpoint{0.000000in}{-0.048611in}}%
\pgfusepath{stroke,fill}%
}%
\begin{pgfscope}%
\pgfsys@transformshift{1.250909in}{0.528000in}%
\pgfsys@useobject{currentmarker}{}%
\end{pgfscope}%
\end{pgfscope}%
\begin{pgfscope}%
\definecolor{textcolor}{rgb}{0.000000,0.000000,0.000000}%
\pgfsetstrokecolor{textcolor}%
\pgfsetfillcolor{textcolor}%
\pgftext[x=1.250909in,y=0.430778in,,top]{\color{textcolor}\sffamily\fontsize{10.000000}{12.000000}\selectfont 10}%
\end{pgfscope}%
\begin{pgfscope}%
\pgfsetbuttcap%
\pgfsetroundjoin%
\definecolor{currentfill}{rgb}{0.000000,0.000000,0.000000}%
\pgfsetfillcolor{currentfill}%
\pgfsetlinewidth{0.803000pt}%
\definecolor{currentstroke}{rgb}{0.000000,0.000000,0.000000}%
\pgfsetstrokecolor{currentstroke}%
\pgfsetdash{}{0pt}%
\pgfsys@defobject{currentmarker}{\pgfqpoint{0.000000in}{-0.048611in}}{\pgfqpoint{0.000000in}{0.000000in}}{%
\pgfpathmoveto{\pgfqpoint{0.000000in}{0.000000in}}%
\pgfpathlineto{\pgfqpoint{0.000000in}{-0.048611in}}%
\pgfusepath{stroke,fill}%
}%
\begin{pgfscope}%
\pgfsys@transformshift{1.701818in}{0.528000in}%
\pgfsys@useobject{currentmarker}{}%
\end{pgfscope}%
\end{pgfscope}%
\begin{pgfscope}%
\definecolor{textcolor}{rgb}{0.000000,0.000000,0.000000}%
\pgfsetstrokecolor{textcolor}%
\pgfsetfillcolor{textcolor}%
\pgftext[x=1.701818in,y=0.430778in,,top]{\color{textcolor}\sffamily\fontsize{10.000000}{12.000000}\selectfont 20}%
\end{pgfscope}%
\begin{pgfscope}%
\pgfsetbuttcap%
\pgfsetroundjoin%
\definecolor{currentfill}{rgb}{0.000000,0.000000,0.000000}%
\pgfsetfillcolor{currentfill}%
\pgfsetlinewidth{0.803000pt}%
\definecolor{currentstroke}{rgb}{0.000000,0.000000,0.000000}%
\pgfsetstrokecolor{currentstroke}%
\pgfsetdash{}{0pt}%
\pgfsys@defobject{currentmarker}{\pgfqpoint{0.000000in}{-0.048611in}}{\pgfqpoint{0.000000in}{0.000000in}}{%
\pgfpathmoveto{\pgfqpoint{0.000000in}{0.000000in}}%
\pgfpathlineto{\pgfqpoint{0.000000in}{-0.048611in}}%
\pgfusepath{stroke,fill}%
}%
\begin{pgfscope}%
\pgfsys@transformshift{2.152727in}{0.528000in}%
\pgfsys@useobject{currentmarker}{}%
\end{pgfscope}%
\end{pgfscope}%
\begin{pgfscope}%
\definecolor{textcolor}{rgb}{0.000000,0.000000,0.000000}%
\pgfsetstrokecolor{textcolor}%
\pgfsetfillcolor{textcolor}%
\pgftext[x=2.152727in,y=0.430778in,,top]{\color{textcolor}\sffamily\fontsize{10.000000}{12.000000}\selectfont 30}%
\end{pgfscope}%
\begin{pgfscope}%
\pgfsetbuttcap%
\pgfsetroundjoin%
\definecolor{currentfill}{rgb}{0.000000,0.000000,0.000000}%
\pgfsetfillcolor{currentfill}%
\pgfsetlinewidth{0.803000pt}%
\definecolor{currentstroke}{rgb}{0.000000,0.000000,0.000000}%
\pgfsetstrokecolor{currentstroke}%
\pgfsetdash{}{0pt}%
\pgfsys@defobject{currentmarker}{\pgfqpoint{0.000000in}{-0.048611in}}{\pgfqpoint{0.000000in}{0.000000in}}{%
\pgfpathmoveto{\pgfqpoint{0.000000in}{0.000000in}}%
\pgfpathlineto{\pgfqpoint{0.000000in}{-0.048611in}}%
\pgfusepath{stroke,fill}%
}%
\begin{pgfscope}%
\pgfsys@transformshift{2.603636in}{0.528000in}%
\pgfsys@useobject{currentmarker}{}%
\end{pgfscope}%
\end{pgfscope}%
\begin{pgfscope}%
\definecolor{textcolor}{rgb}{0.000000,0.000000,0.000000}%
\pgfsetstrokecolor{textcolor}%
\pgfsetfillcolor{textcolor}%
\pgftext[x=2.603636in,y=0.430778in,,top]{\color{textcolor}\sffamily\fontsize{10.000000}{12.000000}\selectfont 40}%
\end{pgfscope}%
\begin{pgfscope}%
\pgfsetbuttcap%
\pgfsetroundjoin%
\definecolor{currentfill}{rgb}{0.000000,0.000000,0.000000}%
\pgfsetfillcolor{currentfill}%
\pgfsetlinewidth{0.803000pt}%
\definecolor{currentstroke}{rgb}{0.000000,0.000000,0.000000}%
\pgfsetstrokecolor{currentstroke}%
\pgfsetdash{}{0pt}%
\pgfsys@defobject{currentmarker}{\pgfqpoint{0.000000in}{-0.048611in}}{\pgfqpoint{0.000000in}{0.000000in}}{%
\pgfpathmoveto{\pgfqpoint{0.000000in}{0.000000in}}%
\pgfpathlineto{\pgfqpoint{0.000000in}{-0.048611in}}%
\pgfusepath{stroke,fill}%
}%
\begin{pgfscope}%
\pgfsys@transformshift{3.054545in}{0.528000in}%
\pgfsys@useobject{currentmarker}{}%
\end{pgfscope}%
\end{pgfscope}%
\begin{pgfscope}%
\definecolor{textcolor}{rgb}{0.000000,0.000000,0.000000}%
\pgfsetstrokecolor{textcolor}%
\pgfsetfillcolor{textcolor}%
\pgftext[x=3.054545in,y=0.430778in,,top]{\color{textcolor}\sffamily\fontsize{10.000000}{12.000000}\selectfont 50}%
\end{pgfscope}%
\begin{pgfscope}%
\pgfsetbuttcap%
\pgfsetroundjoin%
\definecolor{currentfill}{rgb}{0.000000,0.000000,0.000000}%
\pgfsetfillcolor{currentfill}%
\pgfsetlinewidth{0.803000pt}%
\definecolor{currentstroke}{rgb}{0.000000,0.000000,0.000000}%
\pgfsetstrokecolor{currentstroke}%
\pgfsetdash{}{0pt}%
\pgfsys@defobject{currentmarker}{\pgfqpoint{0.000000in}{-0.048611in}}{\pgfqpoint{0.000000in}{0.000000in}}{%
\pgfpathmoveto{\pgfqpoint{0.000000in}{0.000000in}}%
\pgfpathlineto{\pgfqpoint{0.000000in}{-0.048611in}}%
\pgfusepath{stroke,fill}%
}%
\begin{pgfscope}%
\pgfsys@transformshift{3.505455in}{0.528000in}%
\pgfsys@useobject{currentmarker}{}%
\end{pgfscope}%
\end{pgfscope}%
\begin{pgfscope}%
\definecolor{textcolor}{rgb}{0.000000,0.000000,0.000000}%
\pgfsetstrokecolor{textcolor}%
\pgfsetfillcolor{textcolor}%
\pgftext[x=3.505455in,y=0.430778in,,top]{\color{textcolor}\sffamily\fontsize{10.000000}{12.000000}\selectfont 60}%
\end{pgfscope}%
\begin{pgfscope}%
\pgfsetbuttcap%
\pgfsetroundjoin%
\definecolor{currentfill}{rgb}{0.000000,0.000000,0.000000}%
\pgfsetfillcolor{currentfill}%
\pgfsetlinewidth{0.803000pt}%
\definecolor{currentstroke}{rgb}{0.000000,0.000000,0.000000}%
\pgfsetstrokecolor{currentstroke}%
\pgfsetdash{}{0pt}%
\pgfsys@defobject{currentmarker}{\pgfqpoint{0.000000in}{-0.048611in}}{\pgfqpoint{0.000000in}{0.000000in}}{%
\pgfpathmoveto{\pgfqpoint{0.000000in}{0.000000in}}%
\pgfpathlineto{\pgfqpoint{0.000000in}{-0.048611in}}%
\pgfusepath{stroke,fill}%
}%
\begin{pgfscope}%
\pgfsys@transformshift{3.956364in}{0.528000in}%
\pgfsys@useobject{currentmarker}{}%
\end{pgfscope}%
\end{pgfscope}%
\begin{pgfscope}%
\definecolor{textcolor}{rgb}{0.000000,0.000000,0.000000}%
\pgfsetstrokecolor{textcolor}%
\pgfsetfillcolor{textcolor}%
\pgftext[x=3.956364in,y=0.430778in,,top]{\color{textcolor}\sffamily\fontsize{10.000000}{12.000000}\selectfont 70}%
\end{pgfscope}%
\begin{pgfscope}%
\pgfsetbuttcap%
\pgfsetroundjoin%
\definecolor{currentfill}{rgb}{0.000000,0.000000,0.000000}%
\pgfsetfillcolor{currentfill}%
\pgfsetlinewidth{0.803000pt}%
\definecolor{currentstroke}{rgb}{0.000000,0.000000,0.000000}%
\pgfsetstrokecolor{currentstroke}%
\pgfsetdash{}{0pt}%
\pgfsys@defobject{currentmarker}{\pgfqpoint{0.000000in}{-0.048611in}}{\pgfqpoint{0.000000in}{0.000000in}}{%
\pgfpathmoveto{\pgfqpoint{0.000000in}{0.000000in}}%
\pgfpathlineto{\pgfqpoint{0.000000in}{-0.048611in}}%
\pgfusepath{stroke,fill}%
}%
\begin{pgfscope}%
\pgfsys@transformshift{4.407273in}{0.528000in}%
\pgfsys@useobject{currentmarker}{}%
\end{pgfscope}%
\end{pgfscope}%
\begin{pgfscope}%
\definecolor{textcolor}{rgb}{0.000000,0.000000,0.000000}%
\pgfsetstrokecolor{textcolor}%
\pgfsetfillcolor{textcolor}%
\pgftext[x=4.407273in,y=0.430778in,,top]{\color{textcolor}\sffamily\fontsize{10.000000}{12.000000}\selectfont 80}%
\end{pgfscope}%
\begin{pgfscope}%
\pgfsetbuttcap%
\pgfsetroundjoin%
\definecolor{currentfill}{rgb}{0.000000,0.000000,0.000000}%
\pgfsetfillcolor{currentfill}%
\pgfsetlinewidth{0.803000pt}%
\definecolor{currentstroke}{rgb}{0.000000,0.000000,0.000000}%
\pgfsetstrokecolor{currentstroke}%
\pgfsetdash{}{0pt}%
\pgfsys@defobject{currentmarker}{\pgfqpoint{0.000000in}{-0.048611in}}{\pgfqpoint{0.000000in}{0.000000in}}{%
\pgfpathmoveto{\pgfqpoint{0.000000in}{0.000000in}}%
\pgfpathlineto{\pgfqpoint{0.000000in}{-0.048611in}}%
\pgfusepath{stroke,fill}%
}%
\begin{pgfscope}%
\pgfsys@transformshift{4.858182in}{0.528000in}%
\pgfsys@useobject{currentmarker}{}%
\end{pgfscope}%
\end{pgfscope}%
\begin{pgfscope}%
\definecolor{textcolor}{rgb}{0.000000,0.000000,0.000000}%
\pgfsetstrokecolor{textcolor}%
\pgfsetfillcolor{textcolor}%
\pgftext[x=4.858182in,y=0.430778in,,top]{\color{textcolor}\sffamily\fontsize{10.000000}{12.000000}\selectfont 90}%
\end{pgfscope}%
\begin{pgfscope}%
\pgfsetbuttcap%
\pgfsetroundjoin%
\definecolor{currentfill}{rgb}{0.000000,0.000000,0.000000}%
\pgfsetfillcolor{currentfill}%
\pgfsetlinewidth{0.803000pt}%
\definecolor{currentstroke}{rgb}{0.000000,0.000000,0.000000}%
\pgfsetstrokecolor{currentstroke}%
\pgfsetdash{}{0pt}%
\pgfsys@defobject{currentmarker}{\pgfqpoint{0.000000in}{-0.048611in}}{\pgfqpoint{0.000000in}{0.000000in}}{%
\pgfpathmoveto{\pgfqpoint{0.000000in}{0.000000in}}%
\pgfpathlineto{\pgfqpoint{0.000000in}{-0.048611in}}%
\pgfusepath{stroke,fill}%
}%
\begin{pgfscope}%
\pgfsys@transformshift{5.309091in}{0.528000in}%
\pgfsys@useobject{currentmarker}{}%
\end{pgfscope}%
\end{pgfscope}%
\begin{pgfscope}%
\definecolor{textcolor}{rgb}{0.000000,0.000000,0.000000}%
\pgfsetstrokecolor{textcolor}%
\pgfsetfillcolor{textcolor}%
\pgftext[x=5.309091in,y=0.430778in,,top]{\color{textcolor}\sffamily\fontsize{10.000000}{12.000000}\selectfont 100}%
\end{pgfscope}%
\begin{pgfscope}%
\definecolor{textcolor}{rgb}{0.000000,0.000000,0.000000}%
\pgfsetstrokecolor{textcolor}%
\pgfsetfillcolor{textcolor}%
\pgftext[x=3.280000in,y=0.240809in,,top]{\color{textcolor}\sffamily\fontsize{10.000000}{12.000000}\selectfont \(\displaystyle k\)}%
\end{pgfscope}%
\begin{pgfscope}%
\pgfsetbuttcap%
\pgfsetroundjoin%
\definecolor{currentfill}{rgb}{0.000000,0.000000,0.000000}%
\pgfsetfillcolor{currentfill}%
\pgfsetlinewidth{0.803000pt}%
\definecolor{currentstroke}{rgb}{0.000000,0.000000,0.000000}%
\pgfsetstrokecolor{currentstroke}%
\pgfsetdash{}{0pt}%
\pgfsys@defobject{currentmarker}{\pgfqpoint{-0.048611in}{0.000000in}}{\pgfqpoint{0.000000in}{0.000000in}}{%
\pgfpathmoveto{\pgfqpoint{0.000000in}{0.000000in}}%
\pgfpathlineto{\pgfqpoint{-0.048611in}{0.000000in}}%
\pgfusepath{stroke,fill}%
}%
\begin{pgfscope}%
\pgfsys@transformshift{0.800000in}{0.770667in}%
\pgfsys@useobject{currentmarker}{}%
\end{pgfscope}%
\end{pgfscope}%
\begin{pgfscope}%
\definecolor{textcolor}{rgb}{0.000000,0.000000,0.000000}%
\pgfsetstrokecolor{textcolor}%
\pgfsetfillcolor{textcolor}%
\pgftext[x=0.526047in,y=0.717905in,left,base]{\color{textcolor}\sffamily\fontsize{10.000000}{12.000000}\selectfont 10}%
\end{pgfscope}%
\begin{pgfscope}%
\pgfsetbuttcap%
\pgfsetroundjoin%
\definecolor{currentfill}{rgb}{0.000000,0.000000,0.000000}%
\pgfsetfillcolor{currentfill}%
\pgfsetlinewidth{0.803000pt}%
\definecolor{currentstroke}{rgb}{0.000000,0.000000,0.000000}%
\pgfsetstrokecolor{currentstroke}%
\pgfsetdash{}{0pt}%
\pgfsys@defobject{currentmarker}{\pgfqpoint{-0.048611in}{0.000000in}}{\pgfqpoint{0.000000in}{0.000000in}}{%
\pgfpathmoveto{\pgfqpoint{0.000000in}{0.000000in}}%
\pgfpathlineto{\pgfqpoint{-0.048611in}{0.000000in}}%
\pgfusepath{stroke,fill}%
}%
\begin{pgfscope}%
\pgfsys@transformshift{0.800000in}{1.144000in}%
\pgfsys@useobject{currentmarker}{}%
\end{pgfscope}%
\end{pgfscope}%
\begin{pgfscope}%
\definecolor{textcolor}{rgb}{0.000000,0.000000,0.000000}%
\pgfsetstrokecolor{textcolor}%
\pgfsetfillcolor{textcolor}%
\pgftext[x=0.526047in,y=1.091238in,left,base]{\color{textcolor}\sffamily\fontsize{10.000000}{12.000000}\selectfont 20}%
\end{pgfscope}%
\begin{pgfscope}%
\pgfsetbuttcap%
\pgfsetroundjoin%
\definecolor{currentfill}{rgb}{0.000000,0.000000,0.000000}%
\pgfsetfillcolor{currentfill}%
\pgfsetlinewidth{0.803000pt}%
\definecolor{currentstroke}{rgb}{0.000000,0.000000,0.000000}%
\pgfsetstrokecolor{currentstroke}%
\pgfsetdash{}{0pt}%
\pgfsys@defobject{currentmarker}{\pgfqpoint{-0.048611in}{0.000000in}}{\pgfqpoint{0.000000in}{0.000000in}}{%
\pgfpathmoveto{\pgfqpoint{0.000000in}{0.000000in}}%
\pgfpathlineto{\pgfqpoint{-0.048611in}{0.000000in}}%
\pgfusepath{stroke,fill}%
}%
\begin{pgfscope}%
\pgfsys@transformshift{0.800000in}{1.517333in}%
\pgfsys@useobject{currentmarker}{}%
\end{pgfscope}%
\end{pgfscope}%
\begin{pgfscope}%
\definecolor{textcolor}{rgb}{0.000000,0.000000,0.000000}%
\pgfsetstrokecolor{textcolor}%
\pgfsetfillcolor{textcolor}%
\pgftext[x=0.526047in,y=1.464572in,left,base]{\color{textcolor}\sffamily\fontsize{10.000000}{12.000000}\selectfont 30}%
\end{pgfscope}%
\begin{pgfscope}%
\pgfsetbuttcap%
\pgfsetroundjoin%
\definecolor{currentfill}{rgb}{0.000000,0.000000,0.000000}%
\pgfsetfillcolor{currentfill}%
\pgfsetlinewidth{0.803000pt}%
\definecolor{currentstroke}{rgb}{0.000000,0.000000,0.000000}%
\pgfsetstrokecolor{currentstroke}%
\pgfsetdash{}{0pt}%
\pgfsys@defobject{currentmarker}{\pgfqpoint{-0.048611in}{0.000000in}}{\pgfqpoint{0.000000in}{0.000000in}}{%
\pgfpathmoveto{\pgfqpoint{0.000000in}{0.000000in}}%
\pgfpathlineto{\pgfqpoint{-0.048611in}{0.000000in}}%
\pgfusepath{stroke,fill}%
}%
\begin{pgfscope}%
\pgfsys@transformshift{0.800000in}{1.890667in}%
\pgfsys@useobject{currentmarker}{}%
\end{pgfscope}%
\end{pgfscope}%
\begin{pgfscope}%
\definecolor{textcolor}{rgb}{0.000000,0.000000,0.000000}%
\pgfsetstrokecolor{textcolor}%
\pgfsetfillcolor{textcolor}%
\pgftext[x=0.526047in,y=1.837905in,left,base]{\color{textcolor}\sffamily\fontsize{10.000000}{12.000000}\selectfont 40}%
\end{pgfscope}%
\begin{pgfscope}%
\pgfsetbuttcap%
\pgfsetroundjoin%
\definecolor{currentfill}{rgb}{0.000000,0.000000,0.000000}%
\pgfsetfillcolor{currentfill}%
\pgfsetlinewidth{0.803000pt}%
\definecolor{currentstroke}{rgb}{0.000000,0.000000,0.000000}%
\pgfsetstrokecolor{currentstroke}%
\pgfsetdash{}{0pt}%
\pgfsys@defobject{currentmarker}{\pgfqpoint{-0.048611in}{0.000000in}}{\pgfqpoint{0.000000in}{0.000000in}}{%
\pgfpathmoveto{\pgfqpoint{0.000000in}{0.000000in}}%
\pgfpathlineto{\pgfqpoint{-0.048611in}{0.000000in}}%
\pgfusepath{stroke,fill}%
}%
\begin{pgfscope}%
\pgfsys@transformshift{0.800000in}{2.264000in}%
\pgfsys@useobject{currentmarker}{}%
\end{pgfscope}%
\end{pgfscope}%
\begin{pgfscope}%
\definecolor{textcolor}{rgb}{0.000000,0.000000,0.000000}%
\pgfsetstrokecolor{textcolor}%
\pgfsetfillcolor{textcolor}%
\pgftext[x=0.526047in,y=2.211238in,left,base]{\color{textcolor}\sffamily\fontsize{10.000000}{12.000000}\selectfont 50}%
\end{pgfscope}%
\begin{pgfscope}%
\pgfsetbuttcap%
\pgfsetroundjoin%
\definecolor{currentfill}{rgb}{0.000000,0.000000,0.000000}%
\pgfsetfillcolor{currentfill}%
\pgfsetlinewidth{0.803000pt}%
\definecolor{currentstroke}{rgb}{0.000000,0.000000,0.000000}%
\pgfsetstrokecolor{currentstroke}%
\pgfsetdash{}{0pt}%
\pgfsys@defobject{currentmarker}{\pgfqpoint{-0.048611in}{0.000000in}}{\pgfqpoint{0.000000in}{0.000000in}}{%
\pgfpathmoveto{\pgfqpoint{0.000000in}{0.000000in}}%
\pgfpathlineto{\pgfqpoint{-0.048611in}{0.000000in}}%
\pgfusepath{stroke,fill}%
}%
\begin{pgfscope}%
\pgfsys@transformshift{0.800000in}{2.637333in}%
\pgfsys@useobject{currentmarker}{}%
\end{pgfscope}%
\end{pgfscope}%
\begin{pgfscope}%
\definecolor{textcolor}{rgb}{0.000000,0.000000,0.000000}%
\pgfsetstrokecolor{textcolor}%
\pgfsetfillcolor{textcolor}%
\pgftext[x=0.526047in,y=2.584572in,left,base]{\color{textcolor}\sffamily\fontsize{10.000000}{12.000000}\selectfont 60}%
\end{pgfscope}%
\begin{pgfscope}%
\pgfsetbuttcap%
\pgfsetroundjoin%
\definecolor{currentfill}{rgb}{0.000000,0.000000,0.000000}%
\pgfsetfillcolor{currentfill}%
\pgfsetlinewidth{0.803000pt}%
\definecolor{currentstroke}{rgb}{0.000000,0.000000,0.000000}%
\pgfsetstrokecolor{currentstroke}%
\pgfsetdash{}{0pt}%
\pgfsys@defobject{currentmarker}{\pgfqpoint{-0.048611in}{0.000000in}}{\pgfqpoint{0.000000in}{0.000000in}}{%
\pgfpathmoveto{\pgfqpoint{0.000000in}{0.000000in}}%
\pgfpathlineto{\pgfqpoint{-0.048611in}{0.000000in}}%
\pgfusepath{stroke,fill}%
}%
\begin{pgfscope}%
\pgfsys@transformshift{0.800000in}{3.010667in}%
\pgfsys@useobject{currentmarker}{}%
\end{pgfscope}%
\end{pgfscope}%
\begin{pgfscope}%
\definecolor{textcolor}{rgb}{0.000000,0.000000,0.000000}%
\pgfsetstrokecolor{textcolor}%
\pgfsetfillcolor{textcolor}%
\pgftext[x=0.526047in,y=2.957905in,left,base]{\color{textcolor}\sffamily\fontsize{10.000000}{12.000000}\selectfont 70}%
\end{pgfscope}%
\begin{pgfscope}%
\pgfsetbuttcap%
\pgfsetroundjoin%
\definecolor{currentfill}{rgb}{0.000000,0.000000,0.000000}%
\pgfsetfillcolor{currentfill}%
\pgfsetlinewidth{0.803000pt}%
\definecolor{currentstroke}{rgb}{0.000000,0.000000,0.000000}%
\pgfsetstrokecolor{currentstroke}%
\pgfsetdash{}{0pt}%
\pgfsys@defobject{currentmarker}{\pgfqpoint{-0.048611in}{0.000000in}}{\pgfqpoint{0.000000in}{0.000000in}}{%
\pgfpathmoveto{\pgfqpoint{0.000000in}{0.000000in}}%
\pgfpathlineto{\pgfqpoint{-0.048611in}{0.000000in}}%
\pgfusepath{stroke,fill}%
}%
\begin{pgfscope}%
\pgfsys@transformshift{0.800000in}{3.384000in}%
\pgfsys@useobject{currentmarker}{}%
\end{pgfscope}%
\end{pgfscope}%
\begin{pgfscope}%
\definecolor{textcolor}{rgb}{0.000000,0.000000,0.000000}%
\pgfsetstrokecolor{textcolor}%
\pgfsetfillcolor{textcolor}%
\pgftext[x=0.526047in,y=3.331238in,left,base]{\color{textcolor}\sffamily\fontsize{10.000000}{12.000000}\selectfont 80}%
\end{pgfscope}%
\begin{pgfscope}%
\pgfsetbuttcap%
\pgfsetroundjoin%
\definecolor{currentfill}{rgb}{0.000000,0.000000,0.000000}%
\pgfsetfillcolor{currentfill}%
\pgfsetlinewidth{0.803000pt}%
\definecolor{currentstroke}{rgb}{0.000000,0.000000,0.000000}%
\pgfsetstrokecolor{currentstroke}%
\pgfsetdash{}{0pt}%
\pgfsys@defobject{currentmarker}{\pgfqpoint{-0.048611in}{0.000000in}}{\pgfqpoint{0.000000in}{0.000000in}}{%
\pgfpathmoveto{\pgfqpoint{0.000000in}{0.000000in}}%
\pgfpathlineto{\pgfqpoint{-0.048611in}{0.000000in}}%
\pgfusepath{stroke,fill}%
}%
\begin{pgfscope}%
\pgfsys@transformshift{0.800000in}{3.757333in}%
\pgfsys@useobject{currentmarker}{}%
\end{pgfscope}%
\end{pgfscope}%
\begin{pgfscope}%
\definecolor{textcolor}{rgb}{0.000000,0.000000,0.000000}%
\pgfsetstrokecolor{textcolor}%
\pgfsetfillcolor{textcolor}%
\pgftext[x=0.526047in,y=3.704572in,left,base]{\color{textcolor}\sffamily\fontsize{10.000000}{12.000000}\selectfont 90}%
\end{pgfscope}%
\begin{pgfscope}%
\pgfsetbuttcap%
\pgfsetroundjoin%
\definecolor{currentfill}{rgb}{0.000000,0.000000,0.000000}%
\pgfsetfillcolor{currentfill}%
\pgfsetlinewidth{0.803000pt}%
\definecolor{currentstroke}{rgb}{0.000000,0.000000,0.000000}%
\pgfsetstrokecolor{currentstroke}%
\pgfsetdash{}{0pt}%
\pgfsys@defobject{currentmarker}{\pgfqpoint{-0.048611in}{0.000000in}}{\pgfqpoint{0.000000in}{0.000000in}}{%
\pgfpathmoveto{\pgfqpoint{0.000000in}{0.000000in}}%
\pgfpathlineto{\pgfqpoint{-0.048611in}{0.000000in}}%
\pgfusepath{stroke,fill}%
}%
\begin{pgfscope}%
\pgfsys@transformshift{0.800000in}{4.130667in}%
\pgfsys@useobject{currentmarker}{}%
\end{pgfscope}%
\end{pgfscope}%
\begin{pgfscope}%
\definecolor{textcolor}{rgb}{0.000000,0.000000,0.000000}%
\pgfsetstrokecolor{textcolor}%
\pgfsetfillcolor{textcolor}%
\pgftext[x=0.437682in,y=4.077905in,left,base]{\color{textcolor}\sffamily\fontsize{10.000000}{12.000000}\selectfont 100}%
\end{pgfscope}%
\begin{pgfscope}%
\definecolor{textcolor}{rgb}{0.000000,0.000000,0.000000}%
\pgfsetstrokecolor{textcolor}%
\pgfsetfillcolor{textcolor}%
\pgftext[x=0.382126in,y=2.376000in,,bottom,rotate=90.000000]{\color{textcolor}\sffamily\fontsize{10.000000}{12.000000}\selectfont Number of GMRES iterations}%
\end{pgfscope}%
\begin{pgfscope}%
\pgfpathrectangle{\pgfqpoint{0.800000in}{0.528000in}}{\pgfqpoint{4.960000in}{3.696000in}}%
\pgfusepath{clip}%
\pgfsetbuttcap%
\pgfsetroundjoin%
\pgfsetlinewidth{1.505625pt}%
\definecolor{currentstroke}{rgb}{0.000000,0.000000,0.000000}%
\pgfsetstrokecolor{currentstroke}%
\pgfsetdash{{5.550000pt}{2.400000pt}}{0.000000pt}%
\pgfpathmoveto{\pgfqpoint{1.250909in}{0.770667in}}%
\pgfpathlineto{\pgfqpoint{1.701818in}{0.957333in}}%
\pgfpathlineto{\pgfqpoint{2.152727in}{1.144000in}}%
\pgfpathlineto{\pgfqpoint{2.603636in}{1.330667in}}%
\pgfpathlineto{\pgfqpoint{3.054545in}{1.517333in}}%
\pgfpathlineto{\pgfqpoint{3.505455in}{1.965333in}}%
\pgfpathlineto{\pgfqpoint{3.956364in}{2.376000in}}%
\pgfpathlineto{\pgfqpoint{4.407273in}{2.786667in}}%
\pgfpathlineto{\pgfqpoint{4.858182in}{3.421333in}}%
\pgfpathlineto{\pgfqpoint{5.309091in}{4.056000in}}%
\pgfusepath{stroke}%
\end{pgfscope}%
\begin{pgfscope}%
\pgfpathrectangle{\pgfqpoint{0.800000in}{0.528000in}}{\pgfqpoint{4.960000in}{3.696000in}}%
\pgfusepath{clip}%
\pgfsetbuttcap%
\pgfsetroundjoin%
\definecolor{currentfill}{rgb}{0.000000,0.000000,0.000000}%
\pgfsetfillcolor{currentfill}%
\pgfsetlinewidth{1.003750pt}%
\definecolor{currentstroke}{rgb}{0.000000,0.000000,0.000000}%
\pgfsetstrokecolor{currentstroke}%
\pgfsetdash{}{0pt}%
\pgfsys@defobject{currentmarker}{\pgfqpoint{-0.041667in}{-0.041667in}}{\pgfqpoint{0.041667in}{0.041667in}}{%
\pgfpathmoveto{\pgfqpoint{0.000000in}{-0.041667in}}%
\pgfpathcurveto{\pgfqpoint{0.011050in}{-0.041667in}}{\pgfqpoint{0.021649in}{-0.037276in}}{\pgfqpoint{0.029463in}{-0.029463in}}%
\pgfpathcurveto{\pgfqpoint{0.037276in}{-0.021649in}}{\pgfqpoint{0.041667in}{-0.011050in}}{\pgfqpoint{0.041667in}{0.000000in}}%
\pgfpathcurveto{\pgfqpoint{0.041667in}{0.011050in}}{\pgfqpoint{0.037276in}{0.021649in}}{\pgfqpoint{0.029463in}{0.029463in}}%
\pgfpathcurveto{\pgfqpoint{0.021649in}{0.037276in}}{\pgfqpoint{0.011050in}{0.041667in}}{\pgfqpoint{0.000000in}{0.041667in}}%
\pgfpathcurveto{\pgfqpoint{-0.011050in}{0.041667in}}{\pgfqpoint{-0.021649in}{0.037276in}}{\pgfqpoint{-0.029463in}{0.029463in}}%
\pgfpathcurveto{\pgfqpoint{-0.037276in}{0.021649in}}{\pgfqpoint{-0.041667in}{0.011050in}}{\pgfqpoint{-0.041667in}{0.000000in}}%
\pgfpathcurveto{\pgfqpoint{-0.041667in}{-0.011050in}}{\pgfqpoint{-0.037276in}{-0.021649in}}{\pgfqpoint{-0.029463in}{-0.029463in}}%
\pgfpathcurveto{\pgfqpoint{-0.021649in}{-0.037276in}}{\pgfqpoint{-0.011050in}{-0.041667in}}{\pgfqpoint{0.000000in}{-0.041667in}}%
\pgfpathclose%
\pgfusepath{stroke,fill}%
}%
\begin{pgfscope}%
\pgfsys@transformshift{1.250909in}{0.770667in}%
\pgfsys@useobject{currentmarker}{}%
\end{pgfscope}%
\begin{pgfscope}%
\pgfsys@transformshift{1.701818in}{0.957333in}%
\pgfsys@useobject{currentmarker}{}%
\end{pgfscope}%
\begin{pgfscope}%
\pgfsys@transformshift{2.152727in}{1.144000in}%
\pgfsys@useobject{currentmarker}{}%
\end{pgfscope}%
\begin{pgfscope}%
\pgfsys@transformshift{2.603636in}{1.330667in}%
\pgfsys@useobject{currentmarker}{}%
\end{pgfscope}%
\begin{pgfscope}%
\pgfsys@transformshift{3.054545in}{1.517333in}%
\pgfsys@useobject{currentmarker}{}%
\end{pgfscope}%
\begin{pgfscope}%
\pgfsys@transformshift{3.505455in}{1.965333in}%
\pgfsys@useobject{currentmarker}{}%
\end{pgfscope}%
\begin{pgfscope}%
\pgfsys@transformshift{3.956364in}{2.376000in}%
\pgfsys@useobject{currentmarker}{}%
\end{pgfscope}%
\begin{pgfscope}%
\pgfsys@transformshift{4.407273in}{2.786667in}%
\pgfsys@useobject{currentmarker}{}%
\end{pgfscope}%
\begin{pgfscope}%
\pgfsys@transformshift{4.858182in}{3.421333in}%
\pgfsys@useobject{currentmarker}{}%
\end{pgfscope}%
\begin{pgfscope}%
\pgfsys@transformshift{5.309091in}{4.056000in}%
\pgfsys@useobject{currentmarker}{}%
\end{pgfscope}%
\end{pgfscope}%
\begin{pgfscope}%
\pgfpathrectangle{\pgfqpoint{0.800000in}{0.528000in}}{\pgfqpoint{4.960000in}{3.696000in}}%
\pgfusepath{clip}%
\pgfsetbuttcap%
\pgfsetroundjoin%
\pgfsetlinewidth{1.505625pt}%
\definecolor{currentstroke}{rgb}{0.000000,0.000000,0.000000}%
\pgfsetstrokecolor{currentstroke}%
\pgfsetdash{{5.550000pt}{2.400000pt}}{0.000000pt}%
\pgfpathmoveto{\pgfqpoint{1.250909in}{0.770667in}}%
\pgfpathlineto{\pgfqpoint{1.701818in}{0.882667in}}%
\pgfpathlineto{\pgfqpoint{2.152727in}{0.994667in}}%
\pgfpathlineto{\pgfqpoint{2.603636in}{1.106667in}}%
\pgfpathlineto{\pgfqpoint{3.054545in}{1.218667in}}%
\pgfpathlineto{\pgfqpoint{3.505455in}{1.330667in}}%
\pgfpathlineto{\pgfqpoint{3.956364in}{1.442667in}}%
\pgfpathlineto{\pgfqpoint{4.407273in}{1.554667in}}%
\pgfpathlineto{\pgfqpoint{4.858182in}{1.778667in}}%
\pgfpathlineto{\pgfqpoint{5.309091in}{1.928000in}}%
\pgfusepath{stroke}%
\end{pgfscope}%
\begin{pgfscope}%
\pgfpathrectangle{\pgfqpoint{0.800000in}{0.528000in}}{\pgfqpoint{4.960000in}{3.696000in}}%
\pgfusepath{clip}%
\pgfsetbuttcap%
\pgfsetmiterjoin%
\definecolor{currentfill}{rgb}{0.000000,0.000000,0.000000}%
\pgfsetfillcolor{currentfill}%
\pgfsetlinewidth{1.003750pt}%
\definecolor{currentstroke}{rgb}{0.000000,0.000000,0.000000}%
\pgfsetstrokecolor{currentstroke}%
\pgfsetdash{}{0pt}%
\pgfsys@defobject{currentmarker}{\pgfqpoint{-0.041667in}{-0.041667in}}{\pgfqpoint{0.041667in}{0.041667in}}{%
\pgfpathmoveto{\pgfqpoint{-0.000000in}{-0.041667in}}%
\pgfpathlineto{\pgfqpoint{0.041667in}{0.041667in}}%
\pgfpathlineto{\pgfqpoint{-0.041667in}{0.041667in}}%
\pgfpathclose%
\pgfusepath{stroke,fill}%
}%
\begin{pgfscope}%
\pgfsys@transformshift{1.250909in}{0.770667in}%
\pgfsys@useobject{currentmarker}{}%
\end{pgfscope}%
\begin{pgfscope}%
\pgfsys@transformshift{1.701818in}{0.882667in}%
\pgfsys@useobject{currentmarker}{}%
\end{pgfscope}%
\begin{pgfscope}%
\pgfsys@transformshift{2.152727in}{0.994667in}%
\pgfsys@useobject{currentmarker}{}%
\end{pgfscope}%
\begin{pgfscope}%
\pgfsys@transformshift{2.603636in}{1.106667in}%
\pgfsys@useobject{currentmarker}{}%
\end{pgfscope}%
\begin{pgfscope}%
\pgfsys@transformshift{3.054545in}{1.218667in}%
\pgfsys@useobject{currentmarker}{}%
\end{pgfscope}%
\begin{pgfscope}%
\pgfsys@transformshift{3.505455in}{1.330667in}%
\pgfsys@useobject{currentmarker}{}%
\end{pgfscope}%
\begin{pgfscope}%
\pgfsys@transformshift{3.956364in}{1.442667in}%
\pgfsys@useobject{currentmarker}{}%
\end{pgfscope}%
\begin{pgfscope}%
\pgfsys@transformshift{4.407273in}{1.554667in}%
\pgfsys@useobject{currentmarker}{}%
\end{pgfscope}%
\begin{pgfscope}%
\pgfsys@transformshift{4.858182in}{1.778667in}%
\pgfsys@useobject{currentmarker}{}%
\end{pgfscope}%
\begin{pgfscope}%
\pgfsys@transformshift{5.309091in}{1.928000in}%
\pgfsys@useobject{currentmarker}{}%
\end{pgfscope}%
\end{pgfscope}%
\begin{pgfscope}%
\pgfpathrectangle{\pgfqpoint{0.800000in}{0.528000in}}{\pgfqpoint{4.960000in}{3.696000in}}%
\pgfusepath{clip}%
\pgfsetbuttcap%
\pgfsetroundjoin%
\pgfsetlinewidth{1.505625pt}%
\definecolor{currentstroke}{rgb}{0.000000,0.000000,0.000000}%
\pgfsetstrokecolor{currentstroke}%
\pgfsetdash{{5.550000pt}{2.400000pt}}{0.000000pt}%
\pgfpathmoveto{\pgfqpoint{1.250909in}{0.733333in}}%
\pgfpathlineto{\pgfqpoint{1.701818in}{0.808000in}}%
\pgfpathlineto{\pgfqpoint{2.152727in}{0.882667in}}%
\pgfpathlineto{\pgfqpoint{2.603636in}{0.920000in}}%
\pgfpathlineto{\pgfqpoint{3.054545in}{0.994667in}}%
\pgfpathlineto{\pgfqpoint{3.505455in}{1.032000in}}%
\pgfpathlineto{\pgfqpoint{3.956364in}{1.106667in}}%
\pgfpathlineto{\pgfqpoint{4.407273in}{1.106667in}}%
\pgfpathlineto{\pgfqpoint{4.858182in}{1.181333in}}%
\pgfpathlineto{\pgfqpoint{5.309091in}{1.218667in}}%
\pgfusepath{stroke}%
\end{pgfscope}%
\begin{pgfscope}%
\pgfpathrectangle{\pgfqpoint{0.800000in}{0.528000in}}{\pgfqpoint{4.960000in}{3.696000in}}%
\pgfusepath{clip}%
\pgfsetbuttcap%
\pgfsetmiterjoin%
\definecolor{currentfill}{rgb}{0.000000,0.000000,0.000000}%
\pgfsetfillcolor{currentfill}%
\pgfsetlinewidth{1.003750pt}%
\definecolor{currentstroke}{rgb}{0.000000,0.000000,0.000000}%
\pgfsetstrokecolor{currentstroke}%
\pgfsetdash{}{0pt}%
\pgfsys@defobject{currentmarker}{\pgfqpoint{-0.041667in}{-0.041667in}}{\pgfqpoint{0.041667in}{0.041667in}}{%
\pgfpathmoveto{\pgfqpoint{-0.020833in}{-0.041667in}}%
\pgfpathlineto{\pgfqpoint{0.000000in}{-0.020833in}}%
\pgfpathlineto{\pgfqpoint{0.020833in}{-0.041667in}}%
\pgfpathlineto{\pgfqpoint{0.041667in}{-0.020833in}}%
\pgfpathlineto{\pgfqpoint{0.020833in}{0.000000in}}%
\pgfpathlineto{\pgfqpoint{0.041667in}{0.020833in}}%
\pgfpathlineto{\pgfqpoint{0.020833in}{0.041667in}}%
\pgfpathlineto{\pgfqpoint{0.000000in}{0.020833in}}%
\pgfpathlineto{\pgfqpoint{-0.020833in}{0.041667in}}%
\pgfpathlineto{\pgfqpoint{-0.041667in}{0.020833in}}%
\pgfpathlineto{\pgfqpoint{-0.020833in}{0.000000in}}%
\pgfpathlineto{\pgfqpoint{-0.041667in}{-0.020833in}}%
\pgfpathclose%
\pgfusepath{stroke,fill}%
}%
\begin{pgfscope}%
\pgfsys@transformshift{1.250909in}{0.733333in}%
\pgfsys@useobject{currentmarker}{}%
\end{pgfscope}%
\begin{pgfscope}%
\pgfsys@transformshift{1.701818in}{0.808000in}%
\pgfsys@useobject{currentmarker}{}%
\end{pgfscope}%
\begin{pgfscope}%
\pgfsys@transformshift{2.152727in}{0.882667in}%
\pgfsys@useobject{currentmarker}{}%
\end{pgfscope}%
\begin{pgfscope}%
\pgfsys@transformshift{2.603636in}{0.920000in}%
\pgfsys@useobject{currentmarker}{}%
\end{pgfscope}%
\begin{pgfscope}%
\pgfsys@transformshift{3.054545in}{0.994667in}%
\pgfsys@useobject{currentmarker}{}%
\end{pgfscope}%
\begin{pgfscope}%
\pgfsys@transformshift{3.505455in}{1.032000in}%
\pgfsys@useobject{currentmarker}{}%
\end{pgfscope}%
\begin{pgfscope}%
\pgfsys@transformshift{3.956364in}{1.106667in}%
\pgfsys@useobject{currentmarker}{}%
\end{pgfscope}%
\begin{pgfscope}%
\pgfsys@transformshift{4.407273in}{1.106667in}%
\pgfsys@useobject{currentmarker}{}%
\end{pgfscope}%
\begin{pgfscope}%
\pgfsys@transformshift{4.858182in}{1.181333in}%
\pgfsys@useobject{currentmarker}{}%
\end{pgfscope}%
\begin{pgfscope}%
\pgfsys@transformshift{5.309091in}{1.218667in}%
\pgfsys@useobject{currentmarker}{}%
\end{pgfscope}%
\end{pgfscope}%
\begin{pgfscope}%
\pgfpathrectangle{\pgfqpoint{0.800000in}{0.528000in}}{\pgfqpoint{4.960000in}{3.696000in}}%
\pgfusepath{clip}%
\pgfsetbuttcap%
\pgfsetroundjoin%
\pgfsetlinewidth{1.505625pt}%
\definecolor{currentstroke}{rgb}{0.000000,0.000000,0.000000}%
\pgfsetstrokecolor{currentstroke}%
\pgfsetdash{{5.550000pt}{2.400000pt}}{0.000000pt}%
\pgfpathmoveto{\pgfqpoint{1.250909in}{0.696000in}}%
\pgfpathlineto{\pgfqpoint{1.701818in}{0.733333in}}%
\pgfpathlineto{\pgfqpoint{2.152727in}{0.770667in}}%
\pgfpathlineto{\pgfqpoint{2.603636in}{0.808000in}}%
\pgfpathlineto{\pgfqpoint{3.054545in}{0.845333in}}%
\pgfpathlineto{\pgfqpoint{3.505455in}{0.882667in}}%
\pgfpathlineto{\pgfqpoint{3.956364in}{0.882667in}}%
\pgfpathlineto{\pgfqpoint{4.407273in}{0.920000in}}%
\pgfpathlineto{\pgfqpoint{4.858182in}{0.920000in}}%
\pgfpathlineto{\pgfqpoint{5.309091in}{0.920000in}}%
\pgfusepath{stroke}%
\end{pgfscope}%
\begin{pgfscope}%
\pgfpathrectangle{\pgfqpoint{0.800000in}{0.528000in}}{\pgfqpoint{4.960000in}{3.696000in}}%
\pgfusepath{clip}%
\pgfsetbuttcap%
\pgfsetmiterjoin%
\definecolor{currentfill}{rgb}{0.000000,0.000000,0.000000}%
\pgfsetfillcolor{currentfill}%
\pgfsetlinewidth{1.003750pt}%
\definecolor{currentstroke}{rgb}{0.000000,0.000000,0.000000}%
\pgfsetstrokecolor{currentstroke}%
\pgfsetdash{}{0pt}%
\pgfsys@defobject{currentmarker}{\pgfqpoint{-0.035355in}{-0.058926in}}{\pgfqpoint{0.035355in}{0.058926in}}{%
\pgfpathmoveto{\pgfqpoint{-0.000000in}{-0.058926in}}%
\pgfpathlineto{\pgfqpoint{0.035355in}{0.000000in}}%
\pgfpathlineto{\pgfqpoint{0.000000in}{0.058926in}}%
\pgfpathlineto{\pgfqpoint{-0.035355in}{0.000000in}}%
\pgfpathclose%
\pgfusepath{stroke,fill}%
}%
\begin{pgfscope}%
\pgfsys@transformshift{1.250909in}{0.696000in}%
\pgfsys@useobject{currentmarker}{}%
\end{pgfscope}%
\begin{pgfscope}%
\pgfsys@transformshift{1.701818in}{0.733333in}%
\pgfsys@useobject{currentmarker}{}%
\end{pgfscope}%
\begin{pgfscope}%
\pgfsys@transformshift{2.152727in}{0.770667in}%
\pgfsys@useobject{currentmarker}{}%
\end{pgfscope}%
\begin{pgfscope}%
\pgfsys@transformshift{2.603636in}{0.808000in}%
\pgfsys@useobject{currentmarker}{}%
\end{pgfscope}%
\begin{pgfscope}%
\pgfsys@transformshift{3.054545in}{0.845333in}%
\pgfsys@useobject{currentmarker}{}%
\end{pgfscope}%
\begin{pgfscope}%
\pgfsys@transformshift{3.505455in}{0.882667in}%
\pgfsys@useobject{currentmarker}{}%
\end{pgfscope}%
\begin{pgfscope}%
\pgfsys@transformshift{3.956364in}{0.882667in}%
\pgfsys@useobject{currentmarker}{}%
\end{pgfscope}%
\begin{pgfscope}%
\pgfsys@transformshift{4.407273in}{0.920000in}%
\pgfsys@useobject{currentmarker}{}%
\end{pgfscope}%
\begin{pgfscope}%
\pgfsys@transformshift{4.858182in}{0.920000in}%
\pgfsys@useobject{currentmarker}{}%
\end{pgfscope}%
\begin{pgfscope}%
\pgfsys@transformshift{5.309091in}{0.920000in}%
\pgfsys@useobject{currentmarker}{}%
\end{pgfscope}%
\end{pgfscope}%
\begin{pgfscope}%
\pgfsetrectcap%
\pgfsetmiterjoin%
\pgfsetlinewidth{0.803000pt}%
\definecolor{currentstroke}{rgb}{0.000000,0.000000,0.000000}%
\pgfsetstrokecolor{currentstroke}%
\pgfsetdash{}{0pt}%
\pgfpathmoveto{\pgfqpoint{0.800000in}{0.528000in}}%
\pgfpathlineto{\pgfqpoint{0.800000in}{4.224000in}}%
\pgfusepath{stroke}%
\end{pgfscope}%
\begin{pgfscope}%
\pgfsetrectcap%
\pgfsetmiterjoin%
\pgfsetlinewidth{0.803000pt}%
\definecolor{currentstroke}{rgb}{0.000000,0.000000,0.000000}%
\pgfsetstrokecolor{currentstroke}%
\pgfsetdash{}{0pt}%
\pgfpathmoveto{\pgfqpoint{5.760000in}{0.528000in}}%
\pgfpathlineto{\pgfqpoint{5.760000in}{4.224000in}}%
\pgfusepath{stroke}%
\end{pgfscope}%
\begin{pgfscope}%
\pgfsetrectcap%
\pgfsetmiterjoin%
\pgfsetlinewidth{0.803000pt}%
\definecolor{currentstroke}{rgb}{0.000000,0.000000,0.000000}%
\pgfsetstrokecolor{currentstroke}%
\pgfsetdash{}{0pt}%
\pgfpathmoveto{\pgfqpoint{0.800000in}{0.528000in}}%
\pgfpathlineto{\pgfqpoint{5.760000in}{0.528000in}}%
\pgfusepath{stroke}%
\end{pgfscope}%
\begin{pgfscope}%
\pgfsetrectcap%
\pgfsetmiterjoin%
\pgfsetlinewidth{0.803000pt}%
\definecolor{currentstroke}{rgb}{0.000000,0.000000,0.000000}%
\pgfsetstrokecolor{currentstroke}%
\pgfsetdash{}{0pt}%
\pgfpathmoveto{\pgfqpoint{0.800000in}{4.224000in}}%
\pgfpathlineto{\pgfqpoint{5.760000in}{4.224000in}}%
\pgfusepath{stroke}%
\end{pgfscope}%
\begin{pgfscope}%
\pgfsetbuttcap%
\pgfsetmiterjoin%
\definecolor{currentfill}{rgb}{1.000000,1.000000,1.000000}%
\pgfsetfillcolor{currentfill}%
\pgfsetfillopacity{0.800000}%
\pgfsetlinewidth{1.003750pt}%
\definecolor{currentstroke}{rgb}{0.800000,0.800000,0.800000}%
\pgfsetstrokecolor{currentstroke}%
\pgfsetstrokeopacity{0.800000}%
\pgfsetdash{}{0pt}%
\pgfpathmoveto{\pgfqpoint{0.897222in}{3.297460in}}%
\pgfpathlineto{\pgfqpoint{1.790209in}{3.297460in}}%
\pgfpathquadraticcurveto{\pgfqpoint{1.817987in}{3.297460in}}{\pgfqpoint{1.817987in}{3.325238in}}%
\pgfpathlineto{\pgfqpoint{1.817987in}{4.126778in}}%
\pgfpathquadraticcurveto{\pgfqpoint{1.817987in}{4.154556in}}{\pgfqpoint{1.790209in}{4.154556in}}%
\pgfpathlineto{\pgfqpoint{0.897222in}{4.154556in}}%
\pgfpathquadraticcurveto{\pgfqpoint{0.869444in}{4.154556in}}{\pgfqpoint{0.869444in}{4.126778in}}%
\pgfpathlineto{\pgfqpoint{0.869444in}{3.325238in}}%
\pgfpathquadraticcurveto{\pgfqpoint{0.869444in}{3.297460in}}{\pgfqpoint{0.897222in}{3.297460in}}%
\pgfpathclose%
\pgfusepath{stroke,fill}%
\end{pgfscope}%
\begin{pgfscope}%
\pgfsetbuttcap%
\pgfsetroundjoin%
\pgfsetlinewidth{1.505625pt}%
\definecolor{currentstroke}{rgb}{0.000000,0.000000,0.000000}%
\pgfsetstrokecolor{currentstroke}%
\pgfsetdash{{5.550000pt}{2.400000pt}}{0.000000pt}%
\pgfpathmoveto{\pgfqpoint{0.925000in}{4.042088in}}%
\pgfpathlineto{\pgfqpoint{1.202778in}{4.042088in}}%
\pgfusepath{stroke}%
\end{pgfscope}%
\begin{pgfscope}%
\pgfsetbuttcap%
\pgfsetroundjoin%
\definecolor{currentfill}{rgb}{0.000000,0.000000,0.000000}%
\pgfsetfillcolor{currentfill}%
\pgfsetlinewidth{1.003750pt}%
\definecolor{currentstroke}{rgb}{0.000000,0.000000,0.000000}%
\pgfsetstrokecolor{currentstroke}%
\pgfsetdash{}{0pt}%
\pgfsys@defobject{currentmarker}{\pgfqpoint{-0.041667in}{-0.041667in}}{\pgfqpoint{0.041667in}{0.041667in}}{%
\pgfpathmoveto{\pgfqpoint{0.000000in}{-0.041667in}}%
\pgfpathcurveto{\pgfqpoint{0.011050in}{-0.041667in}}{\pgfqpoint{0.021649in}{-0.037276in}}{\pgfqpoint{0.029463in}{-0.029463in}}%
\pgfpathcurveto{\pgfqpoint{0.037276in}{-0.021649in}}{\pgfqpoint{0.041667in}{-0.011050in}}{\pgfqpoint{0.041667in}{0.000000in}}%
\pgfpathcurveto{\pgfqpoint{0.041667in}{0.011050in}}{\pgfqpoint{0.037276in}{0.021649in}}{\pgfqpoint{0.029463in}{0.029463in}}%
\pgfpathcurveto{\pgfqpoint{0.021649in}{0.037276in}}{\pgfqpoint{0.011050in}{0.041667in}}{\pgfqpoint{0.000000in}{0.041667in}}%
\pgfpathcurveto{\pgfqpoint{-0.011050in}{0.041667in}}{\pgfqpoint{-0.021649in}{0.037276in}}{\pgfqpoint{-0.029463in}{0.029463in}}%
\pgfpathcurveto{\pgfqpoint{-0.037276in}{0.021649in}}{\pgfqpoint{-0.041667in}{0.011050in}}{\pgfqpoint{-0.041667in}{0.000000in}}%
\pgfpathcurveto{\pgfqpoint{-0.041667in}{-0.011050in}}{\pgfqpoint{-0.037276in}{-0.021649in}}{\pgfqpoint{-0.029463in}{-0.029463in}}%
\pgfpathcurveto{\pgfqpoint{-0.021649in}{-0.037276in}}{\pgfqpoint{-0.011050in}{-0.041667in}}{\pgfqpoint{0.000000in}{-0.041667in}}%
\pgfpathclose%
\pgfusepath{stroke,fill}%
}%
\begin{pgfscope}%
\pgfsys@transformshift{1.063889in}{4.042088in}%
\pgfsys@useobject{currentmarker}{}%
\end{pgfscope}%
\end{pgfscope}%
\begin{pgfscope}%
\definecolor{textcolor}{rgb}{0.000000,0.000000,0.000000}%
\pgfsetstrokecolor{textcolor}%
\pgfsetfillcolor{textcolor}%
\pgftext[x=1.313889in,y=3.993477in,left,base]{\color{textcolor}\sffamily\fontsize{10.000000}{12.000000}\selectfont \(\displaystyle \beta = 0.4\)}%
\end{pgfscope}%
\begin{pgfscope}%
\pgfsetbuttcap%
\pgfsetroundjoin%
\pgfsetlinewidth{1.505625pt}%
\definecolor{currentstroke}{rgb}{0.000000,0.000000,0.000000}%
\pgfsetstrokecolor{currentstroke}%
\pgfsetdash{{5.550000pt}{2.400000pt}}{0.000000pt}%
\pgfpathmoveto{\pgfqpoint{0.925000in}{3.838231in}}%
\pgfpathlineto{\pgfqpoint{1.202778in}{3.838231in}}%
\pgfusepath{stroke}%
\end{pgfscope}%
\begin{pgfscope}%
\pgfsetbuttcap%
\pgfsetmiterjoin%
\definecolor{currentfill}{rgb}{0.000000,0.000000,0.000000}%
\pgfsetfillcolor{currentfill}%
\pgfsetlinewidth{1.003750pt}%
\definecolor{currentstroke}{rgb}{0.000000,0.000000,0.000000}%
\pgfsetstrokecolor{currentstroke}%
\pgfsetdash{}{0pt}%
\pgfsys@defobject{currentmarker}{\pgfqpoint{-0.041667in}{-0.041667in}}{\pgfqpoint{0.041667in}{0.041667in}}{%
\pgfpathmoveto{\pgfqpoint{-0.000000in}{-0.041667in}}%
\pgfpathlineto{\pgfqpoint{0.041667in}{0.041667in}}%
\pgfpathlineto{\pgfqpoint{-0.041667in}{0.041667in}}%
\pgfpathclose%
\pgfusepath{stroke,fill}%
}%
\begin{pgfscope}%
\pgfsys@transformshift{1.063889in}{3.838231in}%
\pgfsys@useobject{currentmarker}{}%
\end{pgfscope}%
\end{pgfscope}%
\begin{pgfscope}%
\definecolor{textcolor}{rgb}{0.000000,0.000000,0.000000}%
\pgfsetstrokecolor{textcolor}%
\pgfsetfillcolor{textcolor}%
\pgftext[x=1.313889in,y=3.789620in,left,base]{\color{textcolor}\sffamily\fontsize{10.000000}{12.000000}\selectfont \(\displaystyle \beta = 0.5\)}%
\end{pgfscope}%
\begin{pgfscope}%
\pgfsetbuttcap%
\pgfsetroundjoin%
\pgfsetlinewidth{1.505625pt}%
\definecolor{currentstroke}{rgb}{0.000000,0.000000,0.000000}%
\pgfsetstrokecolor{currentstroke}%
\pgfsetdash{{5.550000pt}{2.400000pt}}{0.000000pt}%
\pgfpathmoveto{\pgfqpoint{0.925000in}{3.634374in}}%
\pgfpathlineto{\pgfqpoint{1.202778in}{3.634374in}}%
\pgfusepath{stroke}%
\end{pgfscope}%
\begin{pgfscope}%
\pgfsetbuttcap%
\pgfsetmiterjoin%
\definecolor{currentfill}{rgb}{0.000000,0.000000,0.000000}%
\pgfsetfillcolor{currentfill}%
\pgfsetlinewidth{1.003750pt}%
\definecolor{currentstroke}{rgb}{0.000000,0.000000,0.000000}%
\pgfsetstrokecolor{currentstroke}%
\pgfsetdash{}{0pt}%
\pgfsys@defobject{currentmarker}{\pgfqpoint{-0.041667in}{-0.041667in}}{\pgfqpoint{0.041667in}{0.041667in}}{%
\pgfpathmoveto{\pgfqpoint{-0.020833in}{-0.041667in}}%
\pgfpathlineto{\pgfqpoint{0.000000in}{-0.020833in}}%
\pgfpathlineto{\pgfqpoint{0.020833in}{-0.041667in}}%
\pgfpathlineto{\pgfqpoint{0.041667in}{-0.020833in}}%
\pgfpathlineto{\pgfqpoint{0.020833in}{0.000000in}}%
\pgfpathlineto{\pgfqpoint{0.041667in}{0.020833in}}%
\pgfpathlineto{\pgfqpoint{0.020833in}{0.041667in}}%
\pgfpathlineto{\pgfqpoint{0.000000in}{0.020833in}}%
\pgfpathlineto{\pgfqpoint{-0.020833in}{0.041667in}}%
\pgfpathlineto{\pgfqpoint{-0.041667in}{0.020833in}}%
\pgfpathlineto{\pgfqpoint{-0.020833in}{0.000000in}}%
\pgfpathlineto{\pgfqpoint{-0.041667in}{-0.020833in}}%
\pgfpathclose%
\pgfusepath{stroke,fill}%
}%
\begin{pgfscope}%
\pgfsys@transformshift{1.063889in}{3.634374in}%
\pgfsys@useobject{currentmarker}{}%
\end{pgfscope}%
\end{pgfscope}%
\begin{pgfscope}%
\definecolor{textcolor}{rgb}{0.000000,0.000000,0.000000}%
\pgfsetstrokecolor{textcolor}%
\pgfsetfillcolor{textcolor}%
\pgftext[x=1.313889in,y=3.585762in,left,base]{\color{textcolor}\sffamily\fontsize{10.000000}{12.000000}\selectfont \(\displaystyle \beta = 0.6\)}%
\end{pgfscope}%
\begin{pgfscope}%
\pgfsetbuttcap%
\pgfsetroundjoin%
\pgfsetlinewidth{1.505625pt}%
\definecolor{currentstroke}{rgb}{0.000000,0.000000,0.000000}%
\pgfsetstrokecolor{currentstroke}%
\pgfsetdash{{5.550000pt}{2.400000pt}}{0.000000pt}%
\pgfpathmoveto{\pgfqpoint{0.925000in}{3.430516in}}%
\pgfpathlineto{\pgfqpoint{1.202778in}{3.430516in}}%
\pgfusepath{stroke}%
\end{pgfscope}%
\begin{pgfscope}%
\pgfsetbuttcap%
\pgfsetmiterjoin%
\definecolor{currentfill}{rgb}{0.000000,0.000000,0.000000}%
\pgfsetfillcolor{currentfill}%
\pgfsetlinewidth{1.003750pt}%
\definecolor{currentstroke}{rgb}{0.000000,0.000000,0.000000}%
\pgfsetstrokecolor{currentstroke}%
\pgfsetdash{}{0pt}%
\pgfsys@defobject{currentmarker}{\pgfqpoint{-0.035355in}{-0.058926in}}{\pgfqpoint{0.035355in}{0.058926in}}{%
\pgfpathmoveto{\pgfqpoint{-0.000000in}{-0.058926in}}%
\pgfpathlineto{\pgfqpoint{0.035355in}{0.000000in}}%
\pgfpathlineto{\pgfqpoint{0.000000in}{0.058926in}}%
\pgfpathlineto{\pgfqpoint{-0.035355in}{0.000000in}}%
\pgfpathclose%
\pgfusepath{stroke,fill}%
}%
\begin{pgfscope}%
\pgfsys@transformshift{1.063889in}{3.430516in}%
\pgfsys@useobject{currentmarker}{}%
\end{pgfscope}%
\end{pgfscope}%
\begin{pgfscope}%
\definecolor{textcolor}{rgb}{0.000000,0.000000,0.000000}%
\pgfsetstrokecolor{textcolor}%
\pgfsetfillcolor{textcolor}%
\pgftext[x=1.313889in,y=3.381905in,left,base]{\color{textcolor}\sffamily\fontsize{10.000000}{12.000000}\selectfont \(\displaystyle \beta = 0.7\)}%
\end{pgfscope}%
\end{pgfpicture}%
\makeatother%
\endgroup%

    \caption{GMRES iteration counts for $\AmatoI\Amatt$ given by \cref{eq:noweak,eq:ntweak}, where $\alpha = 0.2/k^\beta,$ for $\beta = 0.4,0.5,0.6,0.7.$}\label{fig:l1med}
\end{figure}
    
    \begin{figure}
    %% Creator: Matplotlib, PGF backend
%%
%% To include the figure in your LaTeX document, write
%%   \input{<filename>.pgf}
%%
%% Make sure the required packages are loaded in your preamble
%%   \usepackage{pgf}
%%
%% Figures using additional raster images can only be included by \input if
%% they are in the same directory as the main LaTeX file. For loading figures
%% from other directories you can use the `import` package
%%   \usepackage{import}
%% and then include the figures with
%%   \import{<path to file>}{<filename>.pgf}
%%
%% Matplotlib used the following preamble
%%   \usepackage{fontspec}
%%   \setmainfont{DejaVuSerif.ttf}[Path=/home/owen/progs/firedrake-complex/firedrake/lib/python3.5/site-packages/matplotlib/mpl-data/fonts/ttf/]
%%   \setsansfont{DejaVuSans.ttf}[Path=/home/owen/progs/firedrake-complex/firedrake/lib/python3.5/site-packages/matplotlib/mpl-data/fonts/ttf/]
%%   \setmonofont{DejaVuSansMono.ttf}[Path=/home/owen/progs/firedrake-complex/firedrake/lib/python3.5/site-packages/matplotlib/mpl-data/fonts/ttf/]
%%
\begingroup%
\makeatletter%
\begin{pgfpicture}%
\pgfpathrectangle{\pgfpointorigin}{\pgfqpoint{6.400000in}{4.800000in}}%
\pgfusepath{use as bounding box, clip}%
\begin{pgfscope}%
\pgfsetbuttcap%
\pgfsetmiterjoin%
\definecolor{currentfill}{rgb}{1.000000,1.000000,1.000000}%
\pgfsetfillcolor{currentfill}%
\pgfsetlinewidth{0.000000pt}%
\definecolor{currentstroke}{rgb}{1.000000,1.000000,1.000000}%
\pgfsetstrokecolor{currentstroke}%
\pgfsetdash{}{0pt}%
\pgfpathmoveto{\pgfqpoint{0.000000in}{0.000000in}}%
\pgfpathlineto{\pgfqpoint{6.400000in}{0.000000in}}%
\pgfpathlineto{\pgfqpoint{6.400000in}{4.800000in}}%
\pgfpathlineto{\pgfqpoint{0.000000in}{4.800000in}}%
\pgfpathclose%
\pgfusepath{fill}%
\end{pgfscope}%
\begin{pgfscope}%
\pgfsetbuttcap%
\pgfsetmiterjoin%
\definecolor{currentfill}{rgb}{1.000000,1.000000,1.000000}%
\pgfsetfillcolor{currentfill}%
\pgfsetlinewidth{0.000000pt}%
\definecolor{currentstroke}{rgb}{0.000000,0.000000,0.000000}%
\pgfsetstrokecolor{currentstroke}%
\pgfsetstrokeopacity{0.000000}%
\pgfsetdash{}{0pt}%
\pgfpathmoveto{\pgfqpoint{0.800000in}{0.528000in}}%
\pgfpathlineto{\pgfqpoint{5.760000in}{0.528000in}}%
\pgfpathlineto{\pgfqpoint{5.760000in}{4.224000in}}%
\pgfpathlineto{\pgfqpoint{0.800000in}{4.224000in}}%
\pgfpathclose%
\pgfusepath{fill}%
\end{pgfscope}%
\begin{pgfscope}%
\pgfsetbuttcap%
\pgfsetroundjoin%
\definecolor{currentfill}{rgb}{0.000000,0.000000,0.000000}%
\pgfsetfillcolor{currentfill}%
\pgfsetlinewidth{0.803000pt}%
\definecolor{currentstroke}{rgb}{0.000000,0.000000,0.000000}%
\pgfsetstrokecolor{currentstroke}%
\pgfsetdash{}{0pt}%
\pgfsys@defobject{currentmarker}{\pgfqpoint{0.000000in}{-0.048611in}}{\pgfqpoint{0.000000in}{0.000000in}}{%
\pgfpathmoveto{\pgfqpoint{0.000000in}{0.000000in}}%
\pgfpathlineto{\pgfqpoint{0.000000in}{-0.048611in}}%
\pgfusepath{stroke,fill}%
}%
\begin{pgfscope}%
\pgfsys@transformshift{1.250909in}{0.528000in}%
\pgfsys@useobject{currentmarker}{}%
\end{pgfscope}%
\end{pgfscope}%
\begin{pgfscope}%
\definecolor{textcolor}{rgb}{0.000000,0.000000,0.000000}%
\pgfsetstrokecolor{textcolor}%
\pgfsetfillcolor{textcolor}%
\pgftext[x=1.250909in,y=0.430778in,,top]{\color{textcolor}\sffamily\fontsize{10.000000}{12.000000}\selectfont 10}%
\end{pgfscope}%
\begin{pgfscope}%
\pgfsetbuttcap%
\pgfsetroundjoin%
\definecolor{currentfill}{rgb}{0.000000,0.000000,0.000000}%
\pgfsetfillcolor{currentfill}%
\pgfsetlinewidth{0.803000pt}%
\definecolor{currentstroke}{rgb}{0.000000,0.000000,0.000000}%
\pgfsetstrokecolor{currentstroke}%
\pgfsetdash{}{0pt}%
\pgfsys@defobject{currentmarker}{\pgfqpoint{0.000000in}{-0.048611in}}{\pgfqpoint{0.000000in}{0.000000in}}{%
\pgfpathmoveto{\pgfqpoint{0.000000in}{0.000000in}}%
\pgfpathlineto{\pgfqpoint{0.000000in}{-0.048611in}}%
\pgfusepath{stroke,fill}%
}%
\begin{pgfscope}%
\pgfsys@transformshift{1.701818in}{0.528000in}%
\pgfsys@useobject{currentmarker}{}%
\end{pgfscope}%
\end{pgfscope}%
\begin{pgfscope}%
\definecolor{textcolor}{rgb}{0.000000,0.000000,0.000000}%
\pgfsetstrokecolor{textcolor}%
\pgfsetfillcolor{textcolor}%
\pgftext[x=1.701818in,y=0.430778in,,top]{\color{textcolor}\sffamily\fontsize{10.000000}{12.000000}\selectfont 20}%
\end{pgfscope}%
\begin{pgfscope}%
\pgfsetbuttcap%
\pgfsetroundjoin%
\definecolor{currentfill}{rgb}{0.000000,0.000000,0.000000}%
\pgfsetfillcolor{currentfill}%
\pgfsetlinewidth{0.803000pt}%
\definecolor{currentstroke}{rgb}{0.000000,0.000000,0.000000}%
\pgfsetstrokecolor{currentstroke}%
\pgfsetdash{}{0pt}%
\pgfsys@defobject{currentmarker}{\pgfqpoint{0.000000in}{-0.048611in}}{\pgfqpoint{0.000000in}{0.000000in}}{%
\pgfpathmoveto{\pgfqpoint{0.000000in}{0.000000in}}%
\pgfpathlineto{\pgfqpoint{0.000000in}{-0.048611in}}%
\pgfusepath{stroke,fill}%
}%
\begin{pgfscope}%
\pgfsys@transformshift{2.152727in}{0.528000in}%
\pgfsys@useobject{currentmarker}{}%
\end{pgfscope}%
\end{pgfscope}%
\begin{pgfscope}%
\definecolor{textcolor}{rgb}{0.000000,0.000000,0.000000}%
\pgfsetstrokecolor{textcolor}%
\pgfsetfillcolor{textcolor}%
\pgftext[x=2.152727in,y=0.430778in,,top]{\color{textcolor}\sffamily\fontsize{10.000000}{12.000000}\selectfont 30}%
\end{pgfscope}%
\begin{pgfscope}%
\pgfsetbuttcap%
\pgfsetroundjoin%
\definecolor{currentfill}{rgb}{0.000000,0.000000,0.000000}%
\pgfsetfillcolor{currentfill}%
\pgfsetlinewidth{0.803000pt}%
\definecolor{currentstroke}{rgb}{0.000000,0.000000,0.000000}%
\pgfsetstrokecolor{currentstroke}%
\pgfsetdash{}{0pt}%
\pgfsys@defobject{currentmarker}{\pgfqpoint{0.000000in}{-0.048611in}}{\pgfqpoint{0.000000in}{0.000000in}}{%
\pgfpathmoveto{\pgfqpoint{0.000000in}{0.000000in}}%
\pgfpathlineto{\pgfqpoint{0.000000in}{-0.048611in}}%
\pgfusepath{stroke,fill}%
}%
\begin{pgfscope}%
\pgfsys@transformshift{2.603636in}{0.528000in}%
\pgfsys@useobject{currentmarker}{}%
\end{pgfscope}%
\end{pgfscope}%
\begin{pgfscope}%
\definecolor{textcolor}{rgb}{0.000000,0.000000,0.000000}%
\pgfsetstrokecolor{textcolor}%
\pgfsetfillcolor{textcolor}%
\pgftext[x=2.603636in,y=0.430778in,,top]{\color{textcolor}\sffamily\fontsize{10.000000}{12.000000}\selectfont 40}%
\end{pgfscope}%
\begin{pgfscope}%
\pgfsetbuttcap%
\pgfsetroundjoin%
\definecolor{currentfill}{rgb}{0.000000,0.000000,0.000000}%
\pgfsetfillcolor{currentfill}%
\pgfsetlinewidth{0.803000pt}%
\definecolor{currentstroke}{rgb}{0.000000,0.000000,0.000000}%
\pgfsetstrokecolor{currentstroke}%
\pgfsetdash{}{0pt}%
\pgfsys@defobject{currentmarker}{\pgfqpoint{0.000000in}{-0.048611in}}{\pgfqpoint{0.000000in}{0.000000in}}{%
\pgfpathmoveto{\pgfqpoint{0.000000in}{0.000000in}}%
\pgfpathlineto{\pgfqpoint{0.000000in}{-0.048611in}}%
\pgfusepath{stroke,fill}%
}%
\begin{pgfscope}%
\pgfsys@transformshift{3.054545in}{0.528000in}%
\pgfsys@useobject{currentmarker}{}%
\end{pgfscope}%
\end{pgfscope}%
\begin{pgfscope}%
\definecolor{textcolor}{rgb}{0.000000,0.000000,0.000000}%
\pgfsetstrokecolor{textcolor}%
\pgfsetfillcolor{textcolor}%
\pgftext[x=3.054545in,y=0.430778in,,top]{\color{textcolor}\sffamily\fontsize{10.000000}{12.000000}\selectfont 50}%
\end{pgfscope}%
\begin{pgfscope}%
\pgfsetbuttcap%
\pgfsetroundjoin%
\definecolor{currentfill}{rgb}{0.000000,0.000000,0.000000}%
\pgfsetfillcolor{currentfill}%
\pgfsetlinewidth{0.803000pt}%
\definecolor{currentstroke}{rgb}{0.000000,0.000000,0.000000}%
\pgfsetstrokecolor{currentstroke}%
\pgfsetdash{}{0pt}%
\pgfsys@defobject{currentmarker}{\pgfqpoint{0.000000in}{-0.048611in}}{\pgfqpoint{0.000000in}{0.000000in}}{%
\pgfpathmoveto{\pgfqpoint{0.000000in}{0.000000in}}%
\pgfpathlineto{\pgfqpoint{0.000000in}{-0.048611in}}%
\pgfusepath{stroke,fill}%
}%
\begin{pgfscope}%
\pgfsys@transformshift{3.505455in}{0.528000in}%
\pgfsys@useobject{currentmarker}{}%
\end{pgfscope}%
\end{pgfscope}%
\begin{pgfscope}%
\definecolor{textcolor}{rgb}{0.000000,0.000000,0.000000}%
\pgfsetstrokecolor{textcolor}%
\pgfsetfillcolor{textcolor}%
\pgftext[x=3.505455in,y=0.430778in,,top]{\color{textcolor}\sffamily\fontsize{10.000000}{12.000000}\selectfont 60}%
\end{pgfscope}%
\begin{pgfscope}%
\pgfsetbuttcap%
\pgfsetroundjoin%
\definecolor{currentfill}{rgb}{0.000000,0.000000,0.000000}%
\pgfsetfillcolor{currentfill}%
\pgfsetlinewidth{0.803000pt}%
\definecolor{currentstroke}{rgb}{0.000000,0.000000,0.000000}%
\pgfsetstrokecolor{currentstroke}%
\pgfsetdash{}{0pt}%
\pgfsys@defobject{currentmarker}{\pgfqpoint{0.000000in}{-0.048611in}}{\pgfqpoint{0.000000in}{0.000000in}}{%
\pgfpathmoveto{\pgfqpoint{0.000000in}{0.000000in}}%
\pgfpathlineto{\pgfqpoint{0.000000in}{-0.048611in}}%
\pgfusepath{stroke,fill}%
}%
\begin{pgfscope}%
\pgfsys@transformshift{3.956364in}{0.528000in}%
\pgfsys@useobject{currentmarker}{}%
\end{pgfscope}%
\end{pgfscope}%
\begin{pgfscope}%
\definecolor{textcolor}{rgb}{0.000000,0.000000,0.000000}%
\pgfsetstrokecolor{textcolor}%
\pgfsetfillcolor{textcolor}%
\pgftext[x=3.956364in,y=0.430778in,,top]{\color{textcolor}\sffamily\fontsize{10.000000}{12.000000}\selectfont 70}%
\end{pgfscope}%
\begin{pgfscope}%
\pgfsetbuttcap%
\pgfsetroundjoin%
\definecolor{currentfill}{rgb}{0.000000,0.000000,0.000000}%
\pgfsetfillcolor{currentfill}%
\pgfsetlinewidth{0.803000pt}%
\definecolor{currentstroke}{rgb}{0.000000,0.000000,0.000000}%
\pgfsetstrokecolor{currentstroke}%
\pgfsetdash{}{0pt}%
\pgfsys@defobject{currentmarker}{\pgfqpoint{0.000000in}{-0.048611in}}{\pgfqpoint{0.000000in}{0.000000in}}{%
\pgfpathmoveto{\pgfqpoint{0.000000in}{0.000000in}}%
\pgfpathlineto{\pgfqpoint{0.000000in}{-0.048611in}}%
\pgfusepath{stroke,fill}%
}%
\begin{pgfscope}%
\pgfsys@transformshift{4.407273in}{0.528000in}%
\pgfsys@useobject{currentmarker}{}%
\end{pgfscope}%
\end{pgfscope}%
\begin{pgfscope}%
\definecolor{textcolor}{rgb}{0.000000,0.000000,0.000000}%
\pgfsetstrokecolor{textcolor}%
\pgfsetfillcolor{textcolor}%
\pgftext[x=4.407273in,y=0.430778in,,top]{\color{textcolor}\sffamily\fontsize{10.000000}{12.000000}\selectfont 80}%
\end{pgfscope}%
\begin{pgfscope}%
\pgfsetbuttcap%
\pgfsetroundjoin%
\definecolor{currentfill}{rgb}{0.000000,0.000000,0.000000}%
\pgfsetfillcolor{currentfill}%
\pgfsetlinewidth{0.803000pt}%
\definecolor{currentstroke}{rgb}{0.000000,0.000000,0.000000}%
\pgfsetstrokecolor{currentstroke}%
\pgfsetdash{}{0pt}%
\pgfsys@defobject{currentmarker}{\pgfqpoint{0.000000in}{-0.048611in}}{\pgfqpoint{0.000000in}{0.000000in}}{%
\pgfpathmoveto{\pgfqpoint{0.000000in}{0.000000in}}%
\pgfpathlineto{\pgfqpoint{0.000000in}{-0.048611in}}%
\pgfusepath{stroke,fill}%
}%
\begin{pgfscope}%
\pgfsys@transformshift{4.858182in}{0.528000in}%
\pgfsys@useobject{currentmarker}{}%
\end{pgfscope}%
\end{pgfscope}%
\begin{pgfscope}%
\definecolor{textcolor}{rgb}{0.000000,0.000000,0.000000}%
\pgfsetstrokecolor{textcolor}%
\pgfsetfillcolor{textcolor}%
\pgftext[x=4.858182in,y=0.430778in,,top]{\color{textcolor}\sffamily\fontsize{10.000000}{12.000000}\selectfont 90}%
\end{pgfscope}%
\begin{pgfscope}%
\pgfsetbuttcap%
\pgfsetroundjoin%
\definecolor{currentfill}{rgb}{0.000000,0.000000,0.000000}%
\pgfsetfillcolor{currentfill}%
\pgfsetlinewidth{0.803000pt}%
\definecolor{currentstroke}{rgb}{0.000000,0.000000,0.000000}%
\pgfsetstrokecolor{currentstroke}%
\pgfsetdash{}{0pt}%
\pgfsys@defobject{currentmarker}{\pgfqpoint{0.000000in}{-0.048611in}}{\pgfqpoint{0.000000in}{0.000000in}}{%
\pgfpathmoveto{\pgfqpoint{0.000000in}{0.000000in}}%
\pgfpathlineto{\pgfqpoint{0.000000in}{-0.048611in}}%
\pgfusepath{stroke,fill}%
}%
\begin{pgfscope}%
\pgfsys@transformshift{5.309091in}{0.528000in}%
\pgfsys@useobject{currentmarker}{}%
\end{pgfscope}%
\end{pgfscope}%
\begin{pgfscope}%
\definecolor{textcolor}{rgb}{0.000000,0.000000,0.000000}%
\pgfsetstrokecolor{textcolor}%
\pgfsetfillcolor{textcolor}%
\pgftext[x=5.309091in,y=0.430778in,,top]{\color{textcolor}\sffamily\fontsize{10.000000}{12.000000}\selectfont 100}%
\end{pgfscope}%
\begin{pgfscope}%
\definecolor{textcolor}{rgb}{0.000000,0.000000,0.000000}%
\pgfsetstrokecolor{textcolor}%
\pgfsetfillcolor{textcolor}%
\pgftext[x=3.280000in,y=0.240809in,,top]{\color{textcolor}\sffamily\fontsize{10.000000}{12.000000}\selectfont \(\displaystyle k\)}%
\end{pgfscope}%
\begin{pgfscope}%
\pgfsetbuttcap%
\pgfsetroundjoin%
\definecolor{currentfill}{rgb}{0.000000,0.000000,0.000000}%
\pgfsetfillcolor{currentfill}%
\pgfsetlinewidth{0.803000pt}%
\definecolor{currentstroke}{rgb}{0.000000,0.000000,0.000000}%
\pgfsetstrokecolor{currentstroke}%
\pgfsetdash{}{0pt}%
\pgfsys@defobject{currentmarker}{\pgfqpoint{-0.048611in}{0.000000in}}{\pgfqpoint{0.000000in}{0.000000in}}{%
\pgfpathmoveto{\pgfqpoint{0.000000in}{0.000000in}}%
\pgfpathlineto{\pgfqpoint{-0.048611in}{0.000000in}}%
\pgfusepath{stroke,fill}%
}%
\begin{pgfscope}%
\pgfsys@transformshift{0.800000in}{0.696000in}%
\pgfsys@useobject{currentmarker}{}%
\end{pgfscope}%
\end{pgfscope}%
\begin{pgfscope}%
\definecolor{textcolor}{rgb}{0.000000,0.000000,0.000000}%
\pgfsetstrokecolor{textcolor}%
\pgfsetfillcolor{textcolor}%
\pgftext[x=0.614413in,y=0.643238in,left,base]{\color{textcolor}\sffamily\fontsize{10.000000}{12.000000}\selectfont 6}%
\end{pgfscope}%
\begin{pgfscope}%
\pgfsetbuttcap%
\pgfsetroundjoin%
\definecolor{currentfill}{rgb}{0.000000,0.000000,0.000000}%
\pgfsetfillcolor{currentfill}%
\pgfsetlinewidth{0.803000pt}%
\definecolor{currentstroke}{rgb}{0.000000,0.000000,0.000000}%
\pgfsetstrokecolor{currentstroke}%
\pgfsetdash{}{0pt}%
\pgfsys@defobject{currentmarker}{\pgfqpoint{-0.048611in}{0.000000in}}{\pgfqpoint{0.000000in}{0.000000in}}{%
\pgfpathmoveto{\pgfqpoint{0.000000in}{0.000000in}}%
\pgfpathlineto{\pgfqpoint{-0.048611in}{0.000000in}}%
\pgfusepath{stroke,fill}%
}%
\begin{pgfscope}%
\pgfsys@transformshift{0.800000in}{1.536000in}%
\pgfsys@useobject{currentmarker}{}%
\end{pgfscope}%
\end{pgfscope}%
\begin{pgfscope}%
\definecolor{textcolor}{rgb}{0.000000,0.000000,0.000000}%
\pgfsetstrokecolor{textcolor}%
\pgfsetfillcolor{textcolor}%
\pgftext[x=0.614413in,y=1.483238in,left,base]{\color{textcolor}\sffamily\fontsize{10.000000}{12.000000}\selectfont 7}%
\end{pgfscope}%
\begin{pgfscope}%
\pgfsetbuttcap%
\pgfsetroundjoin%
\definecolor{currentfill}{rgb}{0.000000,0.000000,0.000000}%
\pgfsetfillcolor{currentfill}%
\pgfsetlinewidth{0.803000pt}%
\definecolor{currentstroke}{rgb}{0.000000,0.000000,0.000000}%
\pgfsetstrokecolor{currentstroke}%
\pgfsetdash{}{0pt}%
\pgfsys@defobject{currentmarker}{\pgfqpoint{-0.048611in}{0.000000in}}{\pgfqpoint{0.000000in}{0.000000in}}{%
\pgfpathmoveto{\pgfqpoint{0.000000in}{0.000000in}}%
\pgfpathlineto{\pgfqpoint{-0.048611in}{0.000000in}}%
\pgfusepath{stroke,fill}%
}%
\begin{pgfscope}%
\pgfsys@transformshift{0.800000in}{2.376000in}%
\pgfsys@useobject{currentmarker}{}%
\end{pgfscope}%
\end{pgfscope}%
\begin{pgfscope}%
\definecolor{textcolor}{rgb}{0.000000,0.000000,0.000000}%
\pgfsetstrokecolor{textcolor}%
\pgfsetfillcolor{textcolor}%
\pgftext[x=0.614413in,y=2.323238in,left,base]{\color{textcolor}\sffamily\fontsize{10.000000}{12.000000}\selectfont 8}%
\end{pgfscope}%
\begin{pgfscope}%
\pgfsetbuttcap%
\pgfsetroundjoin%
\definecolor{currentfill}{rgb}{0.000000,0.000000,0.000000}%
\pgfsetfillcolor{currentfill}%
\pgfsetlinewidth{0.803000pt}%
\definecolor{currentstroke}{rgb}{0.000000,0.000000,0.000000}%
\pgfsetstrokecolor{currentstroke}%
\pgfsetdash{}{0pt}%
\pgfsys@defobject{currentmarker}{\pgfqpoint{-0.048611in}{0.000000in}}{\pgfqpoint{0.000000in}{0.000000in}}{%
\pgfpathmoveto{\pgfqpoint{0.000000in}{0.000000in}}%
\pgfpathlineto{\pgfqpoint{-0.048611in}{0.000000in}}%
\pgfusepath{stroke,fill}%
}%
\begin{pgfscope}%
\pgfsys@transformshift{0.800000in}{3.216000in}%
\pgfsys@useobject{currentmarker}{}%
\end{pgfscope}%
\end{pgfscope}%
\begin{pgfscope}%
\definecolor{textcolor}{rgb}{0.000000,0.000000,0.000000}%
\pgfsetstrokecolor{textcolor}%
\pgfsetfillcolor{textcolor}%
\pgftext[x=0.614413in,y=3.163238in,left,base]{\color{textcolor}\sffamily\fontsize{10.000000}{12.000000}\selectfont 9}%
\end{pgfscope}%
\begin{pgfscope}%
\pgfsetbuttcap%
\pgfsetroundjoin%
\definecolor{currentfill}{rgb}{0.000000,0.000000,0.000000}%
\pgfsetfillcolor{currentfill}%
\pgfsetlinewidth{0.803000pt}%
\definecolor{currentstroke}{rgb}{0.000000,0.000000,0.000000}%
\pgfsetstrokecolor{currentstroke}%
\pgfsetdash{}{0pt}%
\pgfsys@defobject{currentmarker}{\pgfqpoint{-0.048611in}{0.000000in}}{\pgfqpoint{0.000000in}{0.000000in}}{%
\pgfpathmoveto{\pgfqpoint{0.000000in}{0.000000in}}%
\pgfpathlineto{\pgfqpoint{-0.048611in}{0.000000in}}%
\pgfusepath{stroke,fill}%
}%
\begin{pgfscope}%
\pgfsys@transformshift{0.800000in}{4.056000in}%
\pgfsys@useobject{currentmarker}{}%
\end{pgfscope}%
\end{pgfscope}%
\begin{pgfscope}%
\definecolor{textcolor}{rgb}{0.000000,0.000000,0.000000}%
\pgfsetstrokecolor{textcolor}%
\pgfsetfillcolor{textcolor}%
\pgftext[x=0.526047in,y=4.003238in,left,base]{\color{textcolor}\sffamily\fontsize{10.000000}{12.000000}\selectfont 10}%
\end{pgfscope}%
\begin{pgfscope}%
\definecolor{textcolor}{rgb}{0.000000,0.000000,0.000000}%
\pgfsetstrokecolor{textcolor}%
\pgfsetfillcolor{textcolor}%
\pgftext[x=0.470492in,y=2.376000in,,bottom,rotate=90.000000]{\color{textcolor}\sffamily\fontsize{10.000000}{12.000000}\selectfont Number of GMRES iterations}%
\end{pgfscope}%
\begin{pgfscope}%
\pgfpathrectangle{\pgfqpoint{0.800000in}{0.528000in}}{\pgfqpoint{4.960000in}{3.696000in}}%
\pgfusepath{clip}%
\pgfsetbuttcap%
\pgfsetroundjoin%
\pgfsetlinewidth{1.505625pt}%
\definecolor{currentstroke}{rgb}{0.843137,0.000000,0.000000}%
\pgfsetstrokecolor{currentstroke}%
\pgfsetdash{{5.550000pt}{2.400000pt}}{0.000000pt}%
\pgfpathmoveto{\pgfqpoint{1.250909in}{2.376000in}}%
\pgfpathlineto{\pgfqpoint{1.701818in}{2.376000in}}%
\pgfpathlineto{\pgfqpoint{2.152727in}{3.216000in}}%
\pgfpathlineto{\pgfqpoint{2.603636in}{3.216000in}}%
\pgfpathlineto{\pgfqpoint{3.054545in}{4.056000in}}%
\pgfpathlineto{\pgfqpoint{3.505455in}{4.056000in}}%
\pgfpathlineto{\pgfqpoint{3.956364in}{4.056000in}}%
\pgfpathlineto{\pgfqpoint{4.407273in}{4.056000in}}%
\pgfpathlineto{\pgfqpoint{4.858182in}{4.056000in}}%
\pgfpathlineto{\pgfqpoint{5.309091in}{4.056000in}}%
\pgfusepath{stroke}%
\end{pgfscope}%
\begin{pgfscope}%
\pgfpathrectangle{\pgfqpoint{0.800000in}{0.528000in}}{\pgfqpoint{4.960000in}{3.696000in}}%
\pgfusepath{clip}%
\pgfsetbuttcap%
\pgfsetroundjoin%
\definecolor{currentfill}{rgb}{0.843137,0.000000,0.000000}%
\pgfsetfillcolor{currentfill}%
\pgfsetlinewidth{1.003750pt}%
\definecolor{currentstroke}{rgb}{0.843137,0.000000,0.000000}%
\pgfsetstrokecolor{currentstroke}%
\pgfsetdash{}{0pt}%
\pgfsys@defobject{currentmarker}{\pgfqpoint{-0.041667in}{-0.041667in}}{\pgfqpoint{0.041667in}{0.041667in}}{%
\pgfpathmoveto{\pgfqpoint{0.000000in}{-0.041667in}}%
\pgfpathcurveto{\pgfqpoint{0.011050in}{-0.041667in}}{\pgfqpoint{0.021649in}{-0.037276in}}{\pgfqpoint{0.029463in}{-0.029463in}}%
\pgfpathcurveto{\pgfqpoint{0.037276in}{-0.021649in}}{\pgfqpoint{0.041667in}{-0.011050in}}{\pgfqpoint{0.041667in}{0.000000in}}%
\pgfpathcurveto{\pgfqpoint{0.041667in}{0.011050in}}{\pgfqpoint{0.037276in}{0.021649in}}{\pgfqpoint{0.029463in}{0.029463in}}%
\pgfpathcurveto{\pgfqpoint{0.021649in}{0.037276in}}{\pgfqpoint{0.011050in}{0.041667in}}{\pgfqpoint{0.000000in}{0.041667in}}%
\pgfpathcurveto{\pgfqpoint{-0.011050in}{0.041667in}}{\pgfqpoint{-0.021649in}{0.037276in}}{\pgfqpoint{-0.029463in}{0.029463in}}%
\pgfpathcurveto{\pgfqpoint{-0.037276in}{0.021649in}}{\pgfqpoint{-0.041667in}{0.011050in}}{\pgfqpoint{-0.041667in}{0.000000in}}%
\pgfpathcurveto{\pgfqpoint{-0.041667in}{-0.011050in}}{\pgfqpoint{-0.037276in}{-0.021649in}}{\pgfqpoint{-0.029463in}{-0.029463in}}%
\pgfpathcurveto{\pgfqpoint{-0.021649in}{-0.037276in}}{\pgfqpoint{-0.011050in}{-0.041667in}}{\pgfqpoint{0.000000in}{-0.041667in}}%
\pgfpathclose%
\pgfusepath{stroke,fill}%
}%
\begin{pgfscope}%
\pgfsys@transformshift{1.250909in}{2.376000in}%
\pgfsys@useobject{currentmarker}{}%
\end{pgfscope}%
\begin{pgfscope}%
\pgfsys@transformshift{1.701818in}{2.376000in}%
\pgfsys@useobject{currentmarker}{}%
\end{pgfscope}%
\begin{pgfscope}%
\pgfsys@transformshift{2.152727in}{3.216000in}%
\pgfsys@useobject{currentmarker}{}%
\end{pgfscope}%
\begin{pgfscope}%
\pgfsys@transformshift{2.603636in}{3.216000in}%
\pgfsys@useobject{currentmarker}{}%
\end{pgfscope}%
\begin{pgfscope}%
\pgfsys@transformshift{3.054545in}{4.056000in}%
\pgfsys@useobject{currentmarker}{}%
\end{pgfscope}%
\begin{pgfscope}%
\pgfsys@transformshift{3.505455in}{4.056000in}%
\pgfsys@useobject{currentmarker}{}%
\end{pgfscope}%
\begin{pgfscope}%
\pgfsys@transformshift{3.956364in}{4.056000in}%
\pgfsys@useobject{currentmarker}{}%
\end{pgfscope}%
\begin{pgfscope}%
\pgfsys@transformshift{4.407273in}{4.056000in}%
\pgfsys@useobject{currentmarker}{}%
\end{pgfscope}%
\begin{pgfscope}%
\pgfsys@transformshift{4.858182in}{4.056000in}%
\pgfsys@useobject{currentmarker}{}%
\end{pgfscope}%
\begin{pgfscope}%
\pgfsys@transformshift{5.309091in}{4.056000in}%
\pgfsys@useobject{currentmarker}{}%
\end{pgfscope}%
\end{pgfscope}%
\begin{pgfscope}%
\pgfpathrectangle{\pgfqpoint{0.800000in}{0.528000in}}{\pgfqpoint{4.960000in}{3.696000in}}%
\pgfusepath{clip}%
\pgfsetbuttcap%
\pgfsetroundjoin%
\pgfsetlinewidth{1.505625pt}%
\definecolor{currentstroke}{rgb}{0.549020,0.235294,1.000000}%
\pgfsetstrokecolor{currentstroke}%
\pgfsetdash{{5.550000pt}{2.400000pt}}{0.000000pt}%
\pgfpathmoveto{\pgfqpoint{1.250909in}{1.536000in}}%
\pgfpathlineto{\pgfqpoint{1.701818in}{1.536000in}}%
\pgfpathlineto{\pgfqpoint{2.152727in}{2.376000in}}%
\pgfpathlineto{\pgfqpoint{2.603636in}{2.376000in}}%
\pgfpathlineto{\pgfqpoint{3.054545in}{2.376000in}}%
\pgfpathlineto{\pgfqpoint{3.505455in}{2.376000in}}%
\pgfpathlineto{\pgfqpoint{3.956364in}{2.376000in}}%
\pgfpathlineto{\pgfqpoint{4.407273in}{2.376000in}}%
\pgfpathlineto{\pgfqpoint{4.858182in}{2.376000in}}%
\pgfpathlineto{\pgfqpoint{5.309091in}{2.376000in}}%
\pgfusepath{stroke}%
\end{pgfscope}%
\begin{pgfscope}%
\pgfpathrectangle{\pgfqpoint{0.800000in}{0.528000in}}{\pgfqpoint{4.960000in}{3.696000in}}%
\pgfusepath{clip}%
\pgfsetbuttcap%
\pgfsetmiterjoin%
\definecolor{currentfill}{rgb}{0.549020,0.235294,1.000000}%
\pgfsetfillcolor{currentfill}%
\pgfsetlinewidth{1.003750pt}%
\definecolor{currentstroke}{rgb}{0.549020,0.235294,1.000000}%
\pgfsetstrokecolor{currentstroke}%
\pgfsetdash{}{0pt}%
\pgfsys@defobject{currentmarker}{\pgfqpoint{-0.041667in}{-0.041667in}}{\pgfqpoint{0.041667in}{0.041667in}}{%
\pgfpathmoveto{\pgfqpoint{0.000000in}{0.041667in}}%
\pgfpathlineto{\pgfqpoint{-0.041667in}{-0.041667in}}%
\pgfpathlineto{\pgfqpoint{0.041667in}{-0.041667in}}%
\pgfpathclose%
\pgfusepath{stroke,fill}%
}%
\begin{pgfscope}%
\pgfsys@transformshift{1.250909in}{1.536000in}%
\pgfsys@useobject{currentmarker}{}%
\end{pgfscope}%
\begin{pgfscope}%
\pgfsys@transformshift{1.701818in}{1.536000in}%
\pgfsys@useobject{currentmarker}{}%
\end{pgfscope}%
\begin{pgfscope}%
\pgfsys@transformshift{2.152727in}{2.376000in}%
\pgfsys@useobject{currentmarker}{}%
\end{pgfscope}%
\begin{pgfscope}%
\pgfsys@transformshift{2.603636in}{2.376000in}%
\pgfsys@useobject{currentmarker}{}%
\end{pgfscope}%
\begin{pgfscope}%
\pgfsys@transformshift{3.054545in}{2.376000in}%
\pgfsys@useobject{currentmarker}{}%
\end{pgfscope}%
\begin{pgfscope}%
\pgfsys@transformshift{3.505455in}{2.376000in}%
\pgfsys@useobject{currentmarker}{}%
\end{pgfscope}%
\begin{pgfscope}%
\pgfsys@transformshift{3.956364in}{2.376000in}%
\pgfsys@useobject{currentmarker}{}%
\end{pgfscope}%
\begin{pgfscope}%
\pgfsys@transformshift{4.407273in}{2.376000in}%
\pgfsys@useobject{currentmarker}{}%
\end{pgfscope}%
\begin{pgfscope}%
\pgfsys@transformshift{4.858182in}{2.376000in}%
\pgfsys@useobject{currentmarker}{}%
\end{pgfscope}%
\begin{pgfscope}%
\pgfsys@transformshift{5.309091in}{2.376000in}%
\pgfsys@useobject{currentmarker}{}%
\end{pgfscope}%
\end{pgfscope}%
\begin{pgfscope}%
\pgfpathrectangle{\pgfqpoint{0.800000in}{0.528000in}}{\pgfqpoint{4.960000in}{3.696000in}}%
\pgfusepath{clip}%
\pgfsetbuttcap%
\pgfsetroundjoin%
\pgfsetlinewidth{1.505625pt}%
\definecolor{currentstroke}{rgb}{0.007843,0.533333,0.000000}%
\pgfsetstrokecolor{currentstroke}%
\pgfsetdash{{5.550000pt}{2.400000pt}}{0.000000pt}%
\pgfpathmoveto{\pgfqpoint{1.250909in}{1.536000in}}%
\pgfpathlineto{\pgfqpoint{1.701818in}{0.696000in}}%
\pgfpathlineto{\pgfqpoint{2.152727in}{1.536000in}}%
\pgfpathlineto{\pgfqpoint{2.603636in}{1.536000in}}%
\pgfpathlineto{\pgfqpoint{3.054545in}{1.536000in}}%
\pgfpathlineto{\pgfqpoint{3.505455in}{1.536000in}}%
\pgfpathlineto{\pgfqpoint{3.956364in}{1.536000in}}%
\pgfpathlineto{\pgfqpoint{4.407273in}{1.536000in}}%
\pgfpathlineto{\pgfqpoint{4.858182in}{1.536000in}}%
\pgfpathlineto{\pgfqpoint{5.309091in}{1.536000in}}%
\pgfusepath{stroke}%
\end{pgfscope}%
\begin{pgfscope}%
\pgfpathrectangle{\pgfqpoint{0.800000in}{0.528000in}}{\pgfqpoint{4.960000in}{3.696000in}}%
\pgfusepath{clip}%
\pgfsetbuttcap%
\pgfsetmiterjoin%
\definecolor{currentfill}{rgb}{0.007843,0.533333,0.000000}%
\pgfsetfillcolor{currentfill}%
\pgfsetlinewidth{1.003750pt}%
\definecolor{currentstroke}{rgb}{0.007843,0.533333,0.000000}%
\pgfsetstrokecolor{currentstroke}%
\pgfsetdash{}{0pt}%
\pgfsys@defobject{currentmarker}{\pgfqpoint{-0.041667in}{-0.041667in}}{\pgfqpoint{0.041667in}{0.041667in}}{%
\pgfpathmoveto{\pgfqpoint{-0.000000in}{-0.041667in}}%
\pgfpathlineto{\pgfqpoint{0.041667in}{0.041667in}}%
\pgfpathlineto{\pgfqpoint{-0.041667in}{0.041667in}}%
\pgfpathclose%
\pgfusepath{stroke,fill}%
}%
\begin{pgfscope}%
\pgfsys@transformshift{1.250909in}{1.536000in}%
\pgfsys@useobject{currentmarker}{}%
\end{pgfscope}%
\begin{pgfscope}%
\pgfsys@transformshift{1.701818in}{0.696000in}%
\pgfsys@useobject{currentmarker}{}%
\end{pgfscope}%
\begin{pgfscope}%
\pgfsys@transformshift{2.152727in}{1.536000in}%
\pgfsys@useobject{currentmarker}{}%
\end{pgfscope}%
\begin{pgfscope}%
\pgfsys@transformshift{2.603636in}{1.536000in}%
\pgfsys@useobject{currentmarker}{}%
\end{pgfscope}%
\begin{pgfscope}%
\pgfsys@transformshift{3.054545in}{1.536000in}%
\pgfsys@useobject{currentmarker}{}%
\end{pgfscope}%
\begin{pgfscope}%
\pgfsys@transformshift{3.505455in}{1.536000in}%
\pgfsys@useobject{currentmarker}{}%
\end{pgfscope}%
\begin{pgfscope}%
\pgfsys@transformshift{3.956364in}{1.536000in}%
\pgfsys@useobject{currentmarker}{}%
\end{pgfscope}%
\begin{pgfscope}%
\pgfsys@transformshift{4.407273in}{1.536000in}%
\pgfsys@useobject{currentmarker}{}%
\end{pgfscope}%
\begin{pgfscope}%
\pgfsys@transformshift{4.858182in}{1.536000in}%
\pgfsys@useobject{currentmarker}{}%
\end{pgfscope}%
\begin{pgfscope}%
\pgfsys@transformshift{5.309091in}{1.536000in}%
\pgfsys@useobject{currentmarker}{}%
\end{pgfscope}%
\end{pgfscope}%
\begin{pgfscope}%
\pgfsetrectcap%
\pgfsetmiterjoin%
\pgfsetlinewidth{0.803000pt}%
\definecolor{currentstroke}{rgb}{0.000000,0.000000,0.000000}%
\pgfsetstrokecolor{currentstroke}%
\pgfsetdash{}{0pt}%
\pgfpathmoveto{\pgfqpoint{0.800000in}{0.528000in}}%
\pgfpathlineto{\pgfqpoint{0.800000in}{4.224000in}}%
\pgfusepath{stroke}%
\end{pgfscope}%
\begin{pgfscope}%
\pgfsetrectcap%
\pgfsetmiterjoin%
\pgfsetlinewidth{0.803000pt}%
\definecolor{currentstroke}{rgb}{0.000000,0.000000,0.000000}%
\pgfsetstrokecolor{currentstroke}%
\pgfsetdash{}{0pt}%
\pgfpathmoveto{\pgfqpoint{5.760000in}{0.528000in}}%
\pgfpathlineto{\pgfqpoint{5.760000in}{4.224000in}}%
\pgfusepath{stroke}%
\end{pgfscope}%
\begin{pgfscope}%
\pgfsetrectcap%
\pgfsetmiterjoin%
\pgfsetlinewidth{0.803000pt}%
\definecolor{currentstroke}{rgb}{0.000000,0.000000,0.000000}%
\pgfsetstrokecolor{currentstroke}%
\pgfsetdash{}{0pt}%
\pgfpathmoveto{\pgfqpoint{0.800000in}{0.528000in}}%
\pgfpathlineto{\pgfqpoint{5.760000in}{0.528000in}}%
\pgfusepath{stroke}%
\end{pgfscope}%
\begin{pgfscope}%
\pgfsetrectcap%
\pgfsetmiterjoin%
\pgfsetlinewidth{0.803000pt}%
\definecolor{currentstroke}{rgb}{0.000000,0.000000,0.000000}%
\pgfsetstrokecolor{currentstroke}%
\pgfsetdash{}{0pt}%
\pgfpathmoveto{\pgfqpoint{0.800000in}{4.224000in}}%
\pgfpathlineto{\pgfqpoint{5.760000in}{4.224000in}}%
\pgfusepath{stroke}%
\end{pgfscope}%
\begin{pgfscope}%
\pgfsetbuttcap%
\pgfsetmiterjoin%
\definecolor{currentfill}{rgb}{1.000000,1.000000,1.000000}%
\pgfsetfillcolor{currentfill}%
\pgfsetfillopacity{0.800000}%
\pgfsetlinewidth{1.003750pt}%
\definecolor{currentstroke}{rgb}{0.800000,0.800000,0.800000}%
\pgfsetstrokecolor{currentstroke}%
\pgfsetstrokeopacity{0.800000}%
\pgfsetdash{}{0pt}%
\pgfpathmoveto{\pgfqpoint{0.897222in}{3.501317in}}%
\pgfpathlineto{\pgfqpoint{1.790209in}{3.501317in}}%
\pgfpathquadraticcurveto{\pgfqpoint{1.817987in}{3.501317in}}{\pgfqpoint{1.817987in}{3.529095in}}%
\pgfpathlineto{\pgfqpoint{1.817987in}{4.126778in}}%
\pgfpathquadraticcurveto{\pgfqpoint{1.817987in}{4.154556in}}{\pgfqpoint{1.790209in}{4.154556in}}%
\pgfpathlineto{\pgfqpoint{0.897222in}{4.154556in}}%
\pgfpathquadraticcurveto{\pgfqpoint{0.869444in}{4.154556in}}{\pgfqpoint{0.869444in}{4.126778in}}%
\pgfpathlineto{\pgfqpoint{0.869444in}{3.529095in}}%
\pgfpathquadraticcurveto{\pgfqpoint{0.869444in}{3.501317in}}{\pgfqpoint{0.897222in}{3.501317in}}%
\pgfpathclose%
\pgfusepath{stroke,fill}%
\end{pgfscope}%
\begin{pgfscope}%
\pgfsetbuttcap%
\pgfsetroundjoin%
\pgfsetlinewidth{1.505625pt}%
\definecolor{currentstroke}{rgb}{0.843137,0.000000,0.000000}%
\pgfsetstrokecolor{currentstroke}%
\pgfsetdash{{5.550000pt}{2.400000pt}}{0.000000pt}%
\pgfpathmoveto{\pgfqpoint{0.925000in}{4.042088in}}%
\pgfpathlineto{\pgfqpoint{1.202778in}{4.042088in}}%
\pgfusepath{stroke}%
\end{pgfscope}%
\begin{pgfscope}%
\pgfsetbuttcap%
\pgfsetroundjoin%
\definecolor{currentfill}{rgb}{0.843137,0.000000,0.000000}%
\pgfsetfillcolor{currentfill}%
\pgfsetlinewidth{1.003750pt}%
\definecolor{currentstroke}{rgb}{0.843137,0.000000,0.000000}%
\pgfsetstrokecolor{currentstroke}%
\pgfsetdash{}{0pt}%
\pgfsys@defobject{currentmarker}{\pgfqpoint{-0.041667in}{-0.041667in}}{\pgfqpoint{0.041667in}{0.041667in}}{%
\pgfpathmoveto{\pgfqpoint{0.000000in}{-0.041667in}}%
\pgfpathcurveto{\pgfqpoint{0.011050in}{-0.041667in}}{\pgfqpoint{0.021649in}{-0.037276in}}{\pgfqpoint{0.029463in}{-0.029463in}}%
\pgfpathcurveto{\pgfqpoint{0.037276in}{-0.021649in}}{\pgfqpoint{0.041667in}{-0.011050in}}{\pgfqpoint{0.041667in}{0.000000in}}%
\pgfpathcurveto{\pgfqpoint{0.041667in}{0.011050in}}{\pgfqpoint{0.037276in}{0.021649in}}{\pgfqpoint{0.029463in}{0.029463in}}%
\pgfpathcurveto{\pgfqpoint{0.021649in}{0.037276in}}{\pgfqpoint{0.011050in}{0.041667in}}{\pgfqpoint{0.000000in}{0.041667in}}%
\pgfpathcurveto{\pgfqpoint{-0.011050in}{0.041667in}}{\pgfqpoint{-0.021649in}{0.037276in}}{\pgfqpoint{-0.029463in}{0.029463in}}%
\pgfpathcurveto{\pgfqpoint{-0.037276in}{0.021649in}}{\pgfqpoint{-0.041667in}{0.011050in}}{\pgfqpoint{-0.041667in}{0.000000in}}%
\pgfpathcurveto{\pgfqpoint{-0.041667in}{-0.011050in}}{\pgfqpoint{-0.037276in}{-0.021649in}}{\pgfqpoint{-0.029463in}{-0.029463in}}%
\pgfpathcurveto{\pgfqpoint{-0.021649in}{-0.037276in}}{\pgfqpoint{-0.011050in}{-0.041667in}}{\pgfqpoint{0.000000in}{-0.041667in}}%
\pgfpathclose%
\pgfusepath{stroke,fill}%
}%
\begin{pgfscope}%
\pgfsys@transformshift{1.063889in}{4.042088in}%
\pgfsys@useobject{currentmarker}{}%
\end{pgfscope}%
\end{pgfscope}%
\begin{pgfscope}%
\definecolor{textcolor}{rgb}{0.000000,0.000000,0.000000}%
\pgfsetstrokecolor{textcolor}%
\pgfsetfillcolor{textcolor}%
\pgftext[x=1.313889in,y=3.993477in,left,base]{\color{textcolor}\sffamily\fontsize{10.000000}{12.000000}\selectfont \(\displaystyle \beta = 0.8\)}%
\end{pgfscope}%
\begin{pgfscope}%
\pgfsetbuttcap%
\pgfsetroundjoin%
\pgfsetlinewidth{1.505625pt}%
\definecolor{currentstroke}{rgb}{0.549020,0.235294,1.000000}%
\pgfsetstrokecolor{currentstroke}%
\pgfsetdash{{5.550000pt}{2.400000pt}}{0.000000pt}%
\pgfpathmoveto{\pgfqpoint{0.925000in}{3.838231in}}%
\pgfpathlineto{\pgfqpoint{1.202778in}{3.838231in}}%
\pgfusepath{stroke}%
\end{pgfscope}%
\begin{pgfscope}%
\pgfsetbuttcap%
\pgfsetmiterjoin%
\definecolor{currentfill}{rgb}{0.549020,0.235294,1.000000}%
\pgfsetfillcolor{currentfill}%
\pgfsetlinewidth{1.003750pt}%
\definecolor{currentstroke}{rgb}{0.549020,0.235294,1.000000}%
\pgfsetstrokecolor{currentstroke}%
\pgfsetdash{}{0pt}%
\pgfsys@defobject{currentmarker}{\pgfqpoint{-0.041667in}{-0.041667in}}{\pgfqpoint{0.041667in}{0.041667in}}{%
\pgfpathmoveto{\pgfqpoint{0.000000in}{0.041667in}}%
\pgfpathlineto{\pgfqpoint{-0.041667in}{-0.041667in}}%
\pgfpathlineto{\pgfqpoint{0.041667in}{-0.041667in}}%
\pgfpathclose%
\pgfusepath{stroke,fill}%
}%
\begin{pgfscope}%
\pgfsys@transformshift{1.063889in}{3.838231in}%
\pgfsys@useobject{currentmarker}{}%
\end{pgfscope}%
\end{pgfscope}%
\begin{pgfscope}%
\definecolor{textcolor}{rgb}{0.000000,0.000000,0.000000}%
\pgfsetstrokecolor{textcolor}%
\pgfsetfillcolor{textcolor}%
\pgftext[x=1.313889in,y=3.789620in,left,base]{\color{textcolor}\sffamily\fontsize{10.000000}{12.000000}\selectfont \(\displaystyle \beta = 0.9\)}%
\end{pgfscope}%
\begin{pgfscope}%
\pgfsetbuttcap%
\pgfsetroundjoin%
\pgfsetlinewidth{1.505625pt}%
\definecolor{currentstroke}{rgb}{0.007843,0.533333,0.000000}%
\pgfsetstrokecolor{currentstroke}%
\pgfsetdash{{5.550000pt}{2.400000pt}}{0.000000pt}%
\pgfpathmoveto{\pgfqpoint{0.925000in}{3.634374in}}%
\pgfpathlineto{\pgfqpoint{1.202778in}{3.634374in}}%
\pgfusepath{stroke}%
\end{pgfscope}%
\begin{pgfscope}%
\pgfsetbuttcap%
\pgfsetmiterjoin%
\definecolor{currentfill}{rgb}{0.007843,0.533333,0.000000}%
\pgfsetfillcolor{currentfill}%
\pgfsetlinewidth{1.003750pt}%
\definecolor{currentstroke}{rgb}{0.007843,0.533333,0.000000}%
\pgfsetstrokecolor{currentstroke}%
\pgfsetdash{}{0pt}%
\pgfsys@defobject{currentmarker}{\pgfqpoint{-0.041667in}{-0.041667in}}{\pgfqpoint{0.041667in}{0.041667in}}{%
\pgfpathmoveto{\pgfqpoint{-0.000000in}{-0.041667in}}%
\pgfpathlineto{\pgfqpoint{0.041667in}{0.041667in}}%
\pgfpathlineto{\pgfqpoint{-0.041667in}{0.041667in}}%
\pgfpathclose%
\pgfusepath{stroke,fill}%
}%
\begin{pgfscope}%
\pgfsys@transformshift{1.063889in}{3.634374in}%
\pgfsys@useobject{currentmarker}{}%
\end{pgfscope}%
\end{pgfscope}%
\begin{pgfscope}%
\definecolor{textcolor}{rgb}{0.000000,0.000000,0.000000}%
\pgfsetstrokecolor{textcolor}%
\pgfsetfillcolor{textcolor}%
\pgftext[x=1.313889in,y=3.585763in,left,base]{\color{textcolor}\sffamily\fontsize{10.000000}{12.000000}\selectfont \(\displaystyle \beta = 1\)}%
\end{pgfscope}%
\end{pgfpicture}%
\makeatother%
\endgroup%

      \caption{GMRES iteration counts for $\AmatoI\Amatt$ given by \cref{eq:noweak,eq:ntweak}, where $\alpha = 0.2/k^\beta,$ for $\beta = 0.8,0.9,1.$}\label{fig:l1high}
\end{figure}
      
\begin{table}
  \centering
  \begin{tabular}{Sc Sc Sc Sc Sc Sc Sc Sc Sc ScSc }
\toprule

$\eps$\textbackslash$k$ &  10.0  &  20.0  &  30.0  &  40.0  &  50.0  &  60.0  &  70.0  &  80.0  &  90.0  &  100.0 \\

\midrule

0.0 &     14 &     40 &    119 &    258 &    427 &    627 &    940 &   1274 &   1695 &   2116 \\

0.1 &     13 &     27 &     70 &    147 &    262 &    394 &    590 &    825 &   1128 &   1393 \\

0.2 &     12 &     22 &     40 &     77 &    134 &    199 &    292 &    419 &    551 &    726 \\

0.3 &     11 &     18 &     25 &     40 &     58 &     86 &    119 &    163 &    209 &    270 \\

0.4 &     10 &     15 &     20 &     25 &     30 &     42 &     53 &     64 &     81 &     98 \\

0.5 &     10 &     13 &     16 &     19 &     22 &     25 &     28 &     31 &     37 &     41 \\

0.6 &      9 &     11 &     13 &     14 &     16 &     17 &     19 &     19 &     21 &     22 \\

0.7 &      8 &      9 &     10 &     11 &     12 &     13 &     13 &     14 &     14 &     14 \\

0.8 &      8 &      8 &      9 &      9 &     10 &     10 &     10 &     10 &     10 &     10 \\

0.9 &      7 &      7 &      8 &      8 &      8 &      8 &      8 &      8 &      8 &      8 \\

1.0 &      7 &      6 &      7 &      7 &      7 &      7 &      7 &      7 &      7 &      7 \\

\bottomrule

\end{tabular}


  \caption{GMRES iteration counts for $\AmatoI\Amatt$ given by \cref{eq:noweak,eq:ntweak}, where $\alpha = 0.2/k^\beta.$}\label{tab:l1}
  \end{table}

\section{Applying nearby preconditioning to stochastic problems}\label{sec:nbpcstochastic}

We now apply nearby preconditioning to \emph{stochastic} Helmholtz problems, i.e., the Helmholtz problems defined and described in \cref{chap:stochastic}. We have two applications. In \cref{sec:probnbpc} we develop a probabalistic analogue of \cref{cor:1} above, and in \cref{sec:qmcnbpc} we apply nearby preconditioning to a Quasi-Monte-Carlo (QMC) method, showing that we have significant computational gains by using nearby preconditioning.

Throughout this \cref{sec:nbpcstochastic} we consider \cref{prob:msedp} from \cref{chap:stochastic} but with $A=I$. I.e., for simplicity's sake we only consider the case of random $n$, although everything we sya could be easily extended to random $A$. To maintain consistent notation with the rest of this \cref{chap:nbpc} we will use a superscript ${}^{(2)}$ to refer to the stochastic problem (e.g., the random coefficient will be $\nst(\omega)$, the solution will be $\ust(\omega)$, the matrices arising from the finite-element discretiation will be $\Amatt(\omega),$ etc.. We also let $\nso \in \LiDRRR$ define a \emph{deterministic} Helmholtz problem, as at the beginning of this \cref{sec:nbpc}; we will use this deterministic Helmholtz problem to precondition the (discretisations of) the realisations of the stochastic Helmholtz problem. I.e., we will consider the performance of GMRES applied to
\beq\label{eq:stopc}
\AmatoI\Amatt(\omega)\bu = \AmatoI \bff.
\eeq
\optodo{Comment just doing Linfty}
\subsection{Probabilistic analogue of \cref{cor:1a}}
If we apply \cref{cor:1a} to the above set-up we can straightforwardly conclude the following \cref{cor:stonbpcas}

\bco[Almost-sure nearby preconditioning]\label{cor:stonbpcas}
Let the assumptions of \cref{cor:1a} hold, and let $\nt$ satisfy the assumptions at the start of \cref{sec:hh-results}. Then the number of GMRES iterations needed to solve \cref{eq:stopc} is bounded independently of $k$ almost surely if
\beq\label{eq:nbpcas}
\NLiDRRR{\nso-\nst(\omega)} \leq \frac1{2\Ct k}
\eeq
almost surely.
\eco
The numerical results in \cref{sec:num} above can be seen as confirming this result.

\bre[\Cref{cor:stonbpcas} is not ideal]\label{rem:notideal}
\Cref{cor:stonbpcas} is not ideal for two main reasons:
\ben
\item\label{it:notideal1} The condition \cref{eq:nbpcas} must hold almost surely, and
  \item\label{it:notideal2} \Cref{cor:stonbpcas} does not give any explicit information on how the number of GMRES iterations depends on $\NLiDRRR{\nso-\nst(\omega)}.$
    \een
    \Cref{it:notideal1} is not ideal because in many physically realistic problems $\NLiDRRR{\no-\nt(\omega)}$ may be unbounded (e.g., if $\nt$ is a lognormal random field) or even if bounded may not satisfy the condition \cref{eq:nbpcas} almost surely. \Cref{it:notideal2} is not ideal because it means one cannot infer information about the distribution of the number of GMRES iterations from the distribution of$\NLiDRRR{\nso-\nst(\omega)}.$\optodo{Fix Items}
    \ere
    In order to correct the deficiencies described in \cref{rem:notideal} we will first prove a bound on the number of GMRES iterations depending explicitly on $\NLiDRRR{\nso-\nst(\omega)}$, and then use this bound to prove a probabalistic estimate on the number of GMRES iterations.

\optodo{CHange $\sigma$ to $\alpha$ in Elman stuff}
The following \lcnamecref{lem:probgmres1} is an \emph{deterministic} result, i.e., it does not require $\nt$ to be a random field. Therefore for this \lcnamecref{lem:probgmres1} only, we assume $\nst$ is as given at the beginning of this \lcnamecref{chap:nbpc}.
\ble[Maximum number of GMRES iterations]\label{lem:probgmres1}
Let $0 < \eps < 1,$ and let $\dofs$ denote the number of degrees of freedom.\optodo{Mentione $\dofs$ equals size of matrices}

Let $\GMRES{\eps}{\nso}{\nst}$ denote the number of iterations required for GMRES in the unweighted norm $\Nt{\cdot}$ with $\Nt{\brz} = 1,$ applied to
\beqs
\AmatoI\Amatt  \bu = \AmatoI \bff
\eeqs
to converge to within a tolerance $\eps.$

Then there exists a function $\Gfnname:\RRp\rightarrow [0,\dofs]$ such that
\beqs
\GMRES{\eps}{\nso}{\nst} \leq \Gfn{\nso-\nst}.
\eeqs

Moreover, $\Gfnname$ is given by
\beq\label{eq:gdef}
\Gfn{\NLiDR{\nso-\nst}} =
\begin{dcases}
\min\set{N,\frac{\log{\eps}}{\log\mleft(\frac{2\alpha^{1/2}}{\mleft(1+\alpha\mright)^2}\mright)}+1} & \tif \alpha < 1\\
N & \tif \alpha \geq 1,
\end{dcases}
\eeq

where $\alpha = \Ct k \NLiDR{\nso-\nst},$ where $\Ct$ is given by \cref{eq:C2}.
\ele
\optodo{Put in remark somewhere abour why its $+1$ not ceiling}
\bpf[Proof of \cref{lem:probgmres1}]
For $\alpha \geq 1$, the result is immediate from\optodo{Find the fact that GMRES converges in at most $N$, put it in and ref it here}. For $\alpha < 1,$ if we insert \cref{eq:gmressin} into \cref{eq:Elman} (with $\Dmat=\Imat,$ so $\NDmat{\cdot} = \Nt{\cdot}$), we obtain, for $m \in \NN$
\beq\label{eq:gmressub}
\frac{\Nt{\brm}}{\Nt{\brz}} \leq \mleft(\frac{2\sqrt{\alpha}}{\mleft(1+\alpha\mright)^2}\mright)^m.
\eeq
To obtain an lower on the number of iterations needed to obtain the solution to within a tolerance $\eps,$ we set the right-hand side of \cref{eq:gmressub} to be less than $\eps$ and solve for $m$ to obtain that the GMRES residual is less than $\eps$ (recall we assume $\Nt{\brz} = 1$) if
\beq\label{eq:mlower}
m \geq \frac{\log{\eps}}{\log\mleft(\frac{2\alpha^{1/2}}{\mleft(1+\alpha\mright)^2}\mright)}.
\eeq
Hence, if $m$ is the lowest integer satisfying \cref{eq:mlower}, then
\beq\label{eq:gmressub2}
m =\ceil{\frac{\log{\eps}}{\log\mleft(\frac{2\alpha^{1/2}}{\mleft(1+\alpha\mright)^2}\mright)}}.
\eeq
The result for $\alpha < 1$ therefore follows from \cref{eq:gmressub2} and\optodo{at most $N$ result}.
\epf

\bre[Why the dependence on $\alpha$ in \cref{lem:probgmres1}?]
The reason that \cref{eq:gdef} has two cases depending on $\alpha = \Ct k \NLiDRRR{\nso-\nst}$ is because the \lcnamecref{cor:GMRES_intro} only holds if $\alpha < 1$. Therefore if $\alpha \geq 1$ the only result available for us to apply is\optodo{worst case result}, stating that GMRES converges in at most $N$ iterations.
\ere
\optodo{Insert pictures of $G$}

\Cref{lem:probgmres1} gives us the relationship between the (bound on the) number of GMRES iterations and $\NLiDRRR{\nso-\nst(\omega)}.$ We can therefore infer probabalistic properties of the number of GMRES iterations from the probability distribution of $\NLiDRRR{\nso-\nst(\omega)}.$



\optodo{For probabalistic notation ,see that chapter}
\optodo{$\LiDR$ should be $\LiDRRR$}
\bth[Probabilistic GMRES convergence]\label{thm:probgmres}
Let $\nso \in \LiDRRR$ be fixed, and let $\nst:\Omega\rightarrow\LiDRRR$ be a random field. Let $\eps$ and $\dofs$ be as in \cref{lem:probgmres1}, and let $\Aso = \Ast = I.$ Let $\Amatt(\omega)$ denote the (random) Galerkin matrix corresponding to the discretisation of \cref{prob:edp}\optodo{Change?} with coefficients $\Aso=I$ and $\nst(\omega)$.

Fix $ R \in \NN.$ Then
\beq\label{eq:GMRESprob}
\PP\mleft(\Gfn{\NLiDRRR{\nso-\nst}} \leq R\mright)\leq\PP\mleft(\GMRES{\eps}{\nso}{\nst} \leq R\mright) .
\eeq
\enth

\bpf[Proof of \cref{thm:probgmres}]
By \cref{lem:probgmres1} we have the implication: if $\Gfn{\nso-\nst(\omega)} \leq R$, then $\GMRES{\eps}{\nso}{\nst(\omega)} \leq R.$ Therefore we have the set inclusion
\beqs
\set{\omega \in \Omega \st \Gfn{\nso-\nst(\omega)} \leq R} \subseteq \set{\omega \in \Omega \st \GMRES{\eps}{\nso}{\nst(\omega)} \leq R}.
\eeqs
The result immediately follows.
\epf
\optodo{Put in about $\# GMRES$ being a RV}
\bre[The expression \cref{eq:GMRESprob} is computable]
Because the function $\Gfnname$ is not invertible (as is clear from \optodo{pics} above), one cannot write the left-hand side of \cref{eq:GMRESprob} as
\beqs
\PP\mleft(\NLiDRRR{\nso-\nst} \leq \GfnnameI\mleft(\mleft[0,R\mright]\mright)\mright).
\eeqs
However, one can still compute the set
\beqs
\GfnnameI\mleft(\mleft[0,R\mright]\mright) = \set{\alpha \st \Gfn{\alpha} \in \mleft[0,R\mright]}
\eeqs
(where $\GfnnameI$ here denotes the pullback), and therefore one can compute the probabilities in \cref{eq:GMRESprob}. The main effort in computing $\GfnI{\mleft[0,R\mright]}$ is finding if there are any values of $\alpha < 1$ such that $\Gfn{alpha} = R$. However, these values can be computed numerically using standard root-finding software.
\ere

\bre[\Cref{thm:probgmres} is pessimistic]\label{rem:pessimistic}
Observe that we expect the bound in \cref{thm:probgmres} to be pessimistic, i.e., we expect that \cref{eq:GMRESprob} is not sharp. We especially expect \cref{eq:GMRESprob} is not sharp for large values of $R:$

One can show via elementary calculus that (assuming that for $\alpha < 1$ the expression involving $\alpha$ in \cref{eq:gdef} is always $\leq N$) for $\alpha < 1$ $\Gfnname$ achieves its maximum when $\alpha = 1/3$. Also observe that $\Gfnname$ over the range $\alpha < 1$ only depends on $k$ through the dependence of $\alpha$ on $k.$ Therefore, the maximum of $\Gfnname$ over $\alpha \in (0,1)$ is independent of $k.$ Let $\Gfnmaxlo$ denote the value of this maximum. Then, for any $R \in \mleft(\Gfnmaxlo,N\mright)$, the estimate $\PP\mleft(\Gfn{\nso-\nst} \leq R\mright)$ is equal to $\PP\mleft(\NLiDRRR{\nso-\nst} \leq 1/\mleft(\Ct k\mright)\mright),$ i.e., the lower-bound estimate is \emph{independent} of $R$. This is almost certainly not sharp - we would expect $\PP\mleft(\GMRES{\eps}{\nso}{\nst} \leq R\mright)$ to increase with $R$. However, because the only rigorous result we have available if $\alpha \geq 1$ is\optodo{N result}, we cannot prove this expectation rigorously.
\ere

\subsubsection{Qualitative predictions from \cref{thm:probgmres}}\label{sec:qualgmres}

Notwithstanding the comments in \cref{rem:pessimistic}, we wil show in this \lcnamecref{sec:qualgmres} that \cref{thm:probgmres} gives \emph{qualitatively} correct predictions 
\optodo{Add in computations vs theory for 1000 samples once done}
\optodo{Add picture of predicted prob leq 10}



\optodo{Add some examples of what happens as $k\rightarrow \infty$}
\optodo{Add something about s.d. going like $1/k$ and some plots to back it up (Maybe add Gamma/(change to exp plus 1?) experiments if they can happen very quickly)}







\subsection{Application to Quasi-Monte-Carlo methods for the Helmholtz equation}\label{sec:nbpcqmc}

We now apply nearby preconditioning to a Quasi-Monte-Carlo (QMC) method for the Helmholtz equation. We begin with a brief description of QMC methods, before detailing two ways we apply nearby preconditioning to these methods. Finally, we give computational results for applying nearby preconditioning to QMC methods for the Helmholtz equation.

\subsubsection{Brief description of QMC}

QMC methods (or rules) are high-dimensional integration rules, designed to yield superior rates of convergence (with respect to the number of integration points) compared to Monte-Carlo methods. Suppose one wants to approximate $\EXP{Q},$ where $Q$ is some random variable (later in this \lcnamecref{sec:nbpcqmc}, $Q$ will be a function of a stochastic PDE). By definition, the expectation is
\beq\label{eq:qmcexpdef}
\EXP{Q} = \int_\Omega Q(\omega) \ddPPomega.
\eeq

If we now suppose $Q$ depends on the sample space via a finite set of random variables $\Uo,\ldots,\UJ$, then we can rewrite \cref{eq:qmcexpdef} as
\beq\label{eq:qmcexp2}
\EXP{Q} = \int_\Omega Q\mleft((\Uo(\omega),\ldots,\UJ(\omega)\mright) \ddPPomega.
\eeq
If, for example, the $\Uj$ are all Uniform random variables on $\mleft[-1/2,1/2\mright]$, then \cref{eq:qmcexp2} can be rewritten as
\beq\label{eq:qmcexp3}
\EXP{Q} = \int_{\cube{J}} Q\mleft(\by\mright) \dd\lambda,
\eeq
where $\by \in \cube{J}$ and $\lambda$ denotes Lebesgue measure.

Any integration rule, or method for approximating $\EXP{Q}$, can then be seen as a method for approximating the $J$-dimensional integral on the right-hand side of \cref{eq:qmcexp3}. Sampling-based rules will choose points $\byo,\ldots,\byNpoints \in \cube{J}$ and use the approximation
\beqs
\EXP{Q} \approx \frac1{\Npoints}\sum_{l=1}^{\Npoints} Q\mleft(\byl\mright).
\eeqs
Monte-Carlo and Quasi-Monte-Carlo rules are methods for choosing the points $\byl$. In a Monte-Carlo rule the points are chosen at random in accordance with the associated probability distribution. For example, in this case, the points are chosen according to the Uniform distribution on $\cube{J}$. Observe that Monte-Carlo rules do not need the dependence of $Q$ on $\omega$ to take the form prescribed above, they apply to any random variable.

Quasi-Monte-Carlo rules, in contrast to Monte-Carlo rules, do require the dependence on $\omega$ to take the form prescribed above, as QMC rules are high-dimensional integration rules (where we are integrating across the high-dimensional cube $\cube{J}).$ In QMC rules the points $\byl$ are not chosen completely at random, unlike Monte-Carlo rules.

The advantages of QMC rules is that they can exhibit higher rates of convergence compared to Monte-Carlo rules; Monte Carlo rules typically converge with rate $1/\Npoints^{1/2}$ (see\optodo{Ref when done}), whereas QMC rules can converge with rates up to $1/\Npoints$\optodo{Insert ref} or with even higher rates for higher-order QMC rules\optodo{Refs}.

In applying QMC rules to stochastic PDEs, we assume that the random coefficient is dependent on fintely many (or countably many) random variables, as in \cref{eq:qmcexp2} above, and we then use QMC rules to estimate expectations of quantities of interest for the stochastic PDE. Seee\optodo{ref} below for this setup worked out in more detail for the Helmholtz equation. Applying QMC rules to stochastic PDEs is a vibrant and active research area. For recent overviews of this field, see \cite{KuNu:16} (and the associated tutorial \cite{KuNU:18a}) and \cite{KuNu:18b}. We observe that there is currently no rigorous study of how QMC methods behave for the Helmholtz equation, although we undertstand such work is underway in \cite{GaKuSl}.

The key idea in applying nearby preconditioning to QMC methods for the Helmholtz equation is that we do not just choose one realisation $\no$ of the coefficient for which to calculate the preconditioner, rather, we choose as many as a are necessary to ensure the number of GMRES iterations across all samples remains bounded. The key questions the algorithms in this \lcnamecref{sec:nbpcqmc} answer are therefore: How many preconditioners should one calculate, and for which realisations of the coefficient should they be calculated?






\optodo{Need to first explain the iterative procedure}
\subsection{Sequential algorithm}

\begin{algorithm}[h]
\DontPrintSemicolon
\SetKwInOut{Input}{input}\SetKwInOut{Output}{output}
\SetKwFunction{Nearest}{nearest}

\Input{$\maxGMRES$,$\SQMC$}
\BlankLine
Choose starting point $\bystart$\;
$\bypre \defined \bystart$\;
$\Sremaining \defined \SQMC\setminus\set{\bypre}$\;
Calculate preconditioner $\Pmat$ at $\bypre$\;
$\bycurrent \defined$ \Nearest{$\bypre,\Sremaining$}\;
\While{$\Sremaining \neq \emptyset$}{
\lIf{GMRES applied to $\Pmat \Amat(\bycurrent) = \Pmat \bb$ converges in fewer than $\maxGMRES$ iterations}{
$\Sremaining \defined \Sremaining\setminus\set{\bycurrent}$\;
$\bycurrent \defined$ \Nearest{$\bypre,\Sremaining$}\;
}
\Else{
$\bypre \defined \bycurrent$\;
Calculate preconditioner $\Pmat$ at $\bypre$\;
}
}
\caption{Algorithm to perform all solves in a QMC method using nearby preconditioning}
\end{algorithm}
\subsection{Parallelisable tensor-product algorithm}
\optodo{Need to put exposition in here to explain why this is the right thing for a tensor product grid, etc.}
Define
\beqs
\Npreidealj(r) = \max\set{\frac{J \sqrt{\lambdaj}}{2r},1}.
\eeqs
The `ideal' total number of QMC points is
\beqs
\Npreideal(r)=\prod_{j=1}^J  \Npreidealj(r)
\eeqs

Want to calculate the number of preconditioners $\Npre$, the set
\beqs
\Spre=\set{\ypreo,\ldots,\ypreNpre}
\eeqs
of QMC points at which to calculate the preconditioner and the map
\beqs
\nearestpre:\SQMC\rightarrow\Spre
\eeqs
taking each QMC point to its nearest (in the induced spatial $L^\infty$ norm) preconditioner, where $\SQMC$ is the set of QMC points.

\begin{algorithm}[h]
\DontPrintSemicolon
\SetKwInOut{Input}{input}\SetKwInOut{Output}{output}
\SetKwFunction{Round}{round}

\Input{$\Npretarget \in \NN$}
\Output{$\Spre$, $\nearestpre$}
\BlankLine
Solve (numerically) $\Npreideal(\rideal) = \Npretarget$ for $\rideal$\;
\For{j $= 1$ \KwTo $J$}{
Calculate $\Npreactualj =$ \Round{$\Npreidealj(\rideal)$}\;
Define $\Sprej$ to be set of $\Npreactualj$ equally spaced points in $\mleft[-1/2,1/2\mright]$\;
}
Define $\displaystyle\Npre = \prod_{j=1}^J \Npreactualj$\;
Define $\Spre$ by taking all possible tensor products of points in $\Sprej$\;
\For{l $=1$ \KwTo $\NQMC$}{
Calculate $\nearestpre\mleft(\by^{(l)}\mright)$ by brute force\;
}
\caption{Algorithm to determine $\Spre$ and $\nearestpre$}
\end{algorithm}

\section{Extension of the results to the truncated exterior Dirichlet problem}\label{sec:TEDP}

%\subsection{Definition of the TEDP and analogues of the results in \cref{sec:3}}
\optodo{Ref to above}
\paragraph{The impedance boundary $\Gamma_I$.} By comparing \cref{eq:src,eq:ibc}, we see that, in the case $g_I=0$, the TEDP approximates the DtN operator $T_R$ by $\ri k$. Indeed, by using Green's first identity and the definition of the normal derivative (see, e.g., \cite[Lemma 4.3]{Mc:00}), show that the boundary condition on $\Gamma_I$ imposed in the variational problem \cref{prob:vtedp} is 
%In this BVP, the DtN operator $T_R$ Sommerfeld radiation condition 
\beq\label{eq:imp}
\dudnu - \ri k\gamma u = g_I \ton \Gamma_I.
\eeq
where $\nu$ is the unit outward-pointing normal vector to $\Omega$ on $\Gamma_I$.\optodo{Keep or not?}

\paragraph{Existence and uniqueness of a solution to the TEDP.} The sesquilinear form $\aT(\cdot,\cdot)$ defined in \cref{eq:aT} satisfies the G\aa rding inequality \cref{eq:gardingbrief}, and existence and uniqueness of a solution to the TEDP follow under the same condition on $A$ (piecewise-Lipschitz) as for the EDP, as discussed in \cref{sec:vpGm}; in the case of Lipschitz scalar $A$, these unique-continuation arguments are summarised in \cite[\S2]{GrSa:18}.\optodo{Check this stuff is above}

\paragraph{Finite-element/Galerkin solution.}
The Galerkin matrix is defined exactly as in \cref{eq:matrixAdef}, except that 
\beq\label{eq:NTEDP}
\big(\Nmat\big)_{ij}:= \ri k\int_{\Gamma_I}  (\gamma\phi_i) \,\gamma \phi_j.
\eeq

\paragraph{The adjoint sesquilinear form.} For the TEDP, the adjoint sesquilinear form is given by 
\beq\label{eq:TEDPadjoint}
a^*(u,v) := \int_{\DR} 
\Big((A \grad u)\cdot\grad \vb
 - k^2 n u\vb\Big) +\ri k\int_{\Gamma_I} \gamma u\, \overline{\gamma v};
\eeq
then \cref{eq:A*} holds (with $\Nmat$ now given by \cref{eq:NTEDP}, and the analogue of \cref{lem:adjoint} follows in a straightforward way.


\paragraph{The analogues of \cref{cond:1nbpc,cond:2}.}
The statement of the TEDP analogues of \cref{cond:1nbpc,cond:2} are the same as for the EDP, apart from the following.
\ben
\item
$\supp \,f$ need not be a subset of $\widetilde{\Omega}$ (i.e.~the support of $f$ can go up to the impedance boundary $\Gamma_I$), and
\item the assumption $g_I= 0$ needs to be added to \cref{cond:1nbpc} and Part (i) of \cref{cond:2}.
\een
 Note that, since $a(\cdot,\cdot)$ for the TEDP satisfies the same G\aa rding inequality \cref{eq:gardingbrief} as the $a(\cdot,\cdot)$ for the EDP, \cref{lem:H1} holds for the TEDP under the TEDP-analogue of \cref{cond:1nbpc}.

\paragraph{The main results \cref{thm:1,cor:1}.}
Since \cref{cond:1nbpc,cond:2} are essentially unchanged from the EDP case, \cref{lem:keylemma1,lem:keylemma2} hold for the TEDP, and thus so do \cref{thm:1,cor:1,cor:1a}.

\paragraph{The PDE results \cref{thm:2} and \cref{lem:sharp}.}

The PDE bound \cref{thm:2} relies only on \cref{lem:H1}, which, as stated above, also holds for the TEDP. Therefore \cref{thm:2} holds for the TEDP under the TEDP-analogue of \cref{cond:1nbpc} described above. The construction in \cref{lem:sharp} to show sharpness of the bound in \cref{thm:1} (at least when $\Aso= \Ast= I$) also holds for the TEDP; this is because one can choose the supports of $\chi$ and $\widetilde{\chi}$ to be contained inside $\widetilde{\Omega}$, and then $u^{(1)}$ and $u^{(2)}$ defined in \cref{lem:sharp} satisfy the impedance boundary condition \cref{eq:imp} on $\Gamma_I$.

%% \paragraph{When the TEDP-analogue of \cref{cond:1nbpc} holds.}

%% In \cref{sec:cond1hold} we discussed 4 situations (Cases 1-4) where \cref{cond:1nbpc} is proved to hold for the EDP. We now discuss the TEDP-analogues of these.
%% %Cases 1, 3, and 4 (there is no proof yet for the TEDP-analogue of Case 2).

%% \emph{Cases 1 and 2: $\Aso$, $\nso$, and $\Gamma_I$  are $C^\infty$.} 
%% With the rays defined as in the EDP case (by the Melrose--Sj{\"o}strand generalized bicharacteristic flow 
%% \cite[\S24.3]{Ho:85}), the TEDP-analogue of nontrapping for the EDP is the assumption that 
%% every ray eventually hits the boundary at a \emph{non-diffractive point} (defined in \cite[Page 1037]{BaLeRa:92}). Note that, in the case $\Dm=\emptyset$ $\Aso= I$, and $\nso=1$, every ray eventually hits the boundary at a non-diffractive point by \cite[Lemma 5.3]{BaSpWu:16}.
%% Under the additional assumption that $\nso= 1$, \cref{cond:1nbpc} follows from the results of \cite{BaLeRa:92} by combining \cite[Theorem 1.8]{BaSpWu:16} and \cite[Remark 5.6]{BaSpWu:16}, but $C^{(1)}_{\rm bound}$ is not given explicitly.

%% \emph{Case 3: $\Dm$ is starshaped with respect to the origin, $\Aso$ and $\nso$ are Lipschitz and satisfy radial monotonicity-like conditions.}
%% When $\Gamma_I$ is also starshaped with respect to the origin and $A$ and $n$ satisfy \cref{eq:A1nbpc} and \cref{eq:n1nbpc} respectively (with $\Dp$ replaced by $\Omega$), 
%% \cite[Theorem A.6(i)]{GrPeSp:19} proves that
%% \cref{cond:1nbpc} holds, with an explicit expression for $C^{(1)}_{\rm bound}$. Analogous results when (a) $2\Aso - (\bx\cdot\nabla)\Aso \geq \mu_1$ and $\nso= 1$,
%% and  (b) $\Aso= I$ and  $2\nso + \bx \cdot \nabla \nso \geq \mu_2$, 
%% are contained in \cite[Theorem A.6(ii)]{GrPeSp:19} and \cite[Theorem A.6(iii)]{GrPeSp:19} respectively.
%% When $A$ is scalar, these results were also proved in \cite[Theorem 1]{BrGaPe:17} and, when $\Aso= I$ and $\Dm=\emptyset$, also in \cite[Theorem 3.2]{GrSa:18}.

%% \emph{Case 4: %\item[Case 4:]
%%  $\Aso$ and $\nso$ are allowed to be discontinuous.}
%% %\een
%% \cref{cond:1nbpc} is proved in \cite{CaVo:10} (without an explicit expression for $C^{(1)}_{\rm bound}$) when $\Dm$ is $C^\infty$ and nontrapping, $\Gamma_I$ is $C^\infty$, $\Aso= I $, and $\nso$ is a piecewise-constant, monotonically non-decreasing function, jumping on interfaces that are $C^\infty$ with strictly positive curvature.
%% Recall from \cref{cond:1nbpc} that \cite[Theorem 2.7]{GrPeSp:19} proves that \cref{cond:1nbpc} holds for the EDP (with an explicit expression for $C^{(1)}_{\rm bound}$) when $\Dm$ is starshaped with respect to the origin, $A$ and $n$ are $L^\infty$, with $A$ monotonically \emph{non-increasing} in the radial direction, and $n$ monotonically \emph{non-decreasing}. This proof can be extended to the TEDP, with the additional assumption that $\Gamma_I$ is star-shaped with respect to the origin; see the discussion in \cite[Section A.2]{GrPeSp:19}.

%\cref{cond:1nbpc} is proved, with an explicit expression for $C^{(1)}_{\rm bound}$, when 

%\newpage
%
%\section*{Questions for Th\'eo}
%
%\ben
%\item At the place marked A on the scanned pages, you seem to use the inequality 
%\beq\label{eq:Theo1}
%\vert\vert\vert \xi - \cP_h \xi\vert\vert\vert \lesssim h^\alpha \N{u_\phi- \cP_h u_\phi}_{0,\Omega}.
%\eeq
%\een
%
%\newpag

\section*{Owen to do list}
\ben
\item Varying  $\|\Aso-\Ast\|_{L^\infty}$ and $\|\nso-\nst\|_{L^\infty}$ in standard GMRES.
\item Computations where $\|\Aso-\Ast\|_{L^\infty}$ and $\|\nso-\nst\|_{L^\infty}$ are sometimes large; is having the standard deviations of these $\sim 1/k$ good enough for $k$-independent GMRES iterations?
\item ***on backburner*** Checking under what conditions (if any) Part (ii) \cref{cond:2} holds by running the following experiment:
%\item Exciting experiments for random $n$ that you told us about last week.
%\item In the weighted norm, the condition on $A$ is ``$k \|\Aso-\Ast\|_{L^\infty}$ sufficiently small" but in the Euclidean norm the best we have so far is ``$h^{-1} \|\Aso-\Ast\|_{L^\infty}$ sufficiently small". You indicated before that experiments seemed to indicate that ``$k \|\Aso-\Ast\|_{L^\infty}$ sufficiently small" seemed correct for the Euclidean norm too. The next time we meet, can you show me these results please?
%\item Please run the following numerical experiment.
\bit
\item TEDP with $\Omega$ a square/rectangle.
\item $\Aso$ being at least Lipschitz (but smooth is fine). To keep things simple, just take scalar- (as opposed to matrix-) valued $\Aso$ and don't worry about making it nontrapping.
\item Smoothness of $\nso$ doesn't really matter, just take smooth in the first instance for simplicity (and also don't worry about nontrapping).
\item $\Vhp$ piecewise linear.
\item Linear system $\Amato \bu = \Smat_{A} \balpha$ for some arbitrary complex-valued vector $\balpha$ and some arbitrary $A\in L^\infty$. (I claim this corresponds to the problem described in Part (ii) of \cref{cond:2},  but please check this!)
\item For each $\Aso, \nso, \balpha$, solve linear system for increasing values of $k$, first with $h\sim k^{-2}$, and then with $h\sim k^{-3/2}$.
\item Goal: see if the bound \cref{eq:bound4} holds, using 
\beqs
\N{\sum_j \alpha_j (A\nabla \phi_j)}_{\LtDR} \quad \text{ as a proxy for } \quad \N{F}_{(\HokDR)'}.
\eeqs
\eit
%\item Varying  $\|\Aso-\Ast\|_{L^\infty}$ and $\|\nso-\nst\|_{L^\infty}$ in \emph{weighted} GMRES.
\een



\chapter{Multi-Level Monte Carlo for the Helmholtz Equation}\label{chap:mlmc}

\section{The Multi-Level Monte Carlo Complexity Theorem for the Helmholtz equation}\label{sec:comp}

%% In this \lcnamecref{sec:comp} we state and prove an abstract result on the convergence of multi-level Monte Carlo methods, laregly following the proof of \cite[Theorem 1]{ClGiScTe:11}. Our result is a generalisation of \cite[Theorem 1]{ClGiScTe:11} in the following three ways:
%% \ben
%% \item In \cite{ClGiScTe:11} it is assumed that the convergence of the approximate QoIs $\Qhl$, and the cost of producing samples of these QoIs, only depends on the parameter $\hl$ (where, in stochastic PDE applications, $\hl$ is the mesh size for the finite-element discretisation). However, in this work, we assume that the convergence and cost also depend on another parameter $k,$ and we make the dependence of the final computational cost of the MLMC method explicit in $k.$ In our application to the Helmholtz equation, $k$ will be the wavenumber of the problem.
%% \item In \cite{ClGiScTe:11} it is assumed that the approximating QoIS $\Qhl$ exist for all levels $l$. This corresponds to the finite-element solution of the PDE under investigation existing for all mesh sizes $h.$ Whilst this assumption is true for the stationary diffusion equation studied in \cite{ClGiScTe:11}, it is \emph{not} true for the Helmholtz equation that we study here. Therefore we make the additional assumption (\cref{ass:qoie} below) that $\Qhl$ only exists for sufficiently small $\hl.$
%% \item In \cite{ClGiScTe:11} the error $\eps$ incurred in the MLMC method is equally divided between the bias and the variance of the MLMC method (see the Proof of \cref{thm:mlmccomp3}). However, in this work we assume that there is a quantity $\splitting \in (0,1)$ (see \cref{ass:splittingbounds}), possibly dependent on $k$ that allows a vairable `split' of the error between the bias and the variance. Our main use of this is in\optodo{Insert refs once it's done}, where we use this variable splitting to compensate for the fact that to bound the (squared) bias error by $\eps^2/2$ would mean we take $\hL \lesssim k^{-1},$ but to ensure the finite-element solution $\uh$ exists, we must take $\hL \lesssim k^{-3/2}.$
%% \een
%% We now proceed to prove our abstract MLMC convergence result, comtaining the generalisations metioned above.

Let $\OFP$ be a probability space, and let $Q$ be a random variable\footnote{One can think of $Q$ as being $Q(u),$ where $u$ is the solution of some stochastic PDE.} on $\OFP$ such that $\EXP{Q} < \infty.$ We will refer to $Q$ as the \defn{quantity of interest} or QoI. In order to define the multi-level Monte Carlo (MLMC) method for estimating $\EXP{Q},$ we must also define the following quantities, following  \cite[Theorem 1]{ClGiScTe:11}. We assume there exist
\bit
\item A set of levels\footnote{One can think of $\hl$ as the mesh size associated with level $l$.} $\set{\hl}_{l=0}^L$ ($L$ to be chosen) such that $\hl =\frac{\hlmo}s$ for $l \geq 1.$
\item A set of random variables (that may or may not exist)\footnote{One can think of $\Qhtilde$ as $Q(\uh),$ where $\uhl$ is the finite-element solution of the PDE with mesh size $h  $.} $\set{\Qhtilde}_{h \in (0,1)}.$
  \eit

We denote the random variables $\Qhtilde,$ in order to simplify the notation for mesh dependence in what follows.

In order to do things for the Helmholtz equation, we use the following assumption:

\bas[Mesh conditions for existence and uniqueness]
There exists a measurable function $\Cmesh:\Omega\rightarrow \RRp$ and a function $\mesh:\RRp\rightarrow \RRp$ such that $\Qhtilde(\omega)$ exists (and satisfies the error bounds etc. below) if
\beqs
h \leq \Cmesh(\omega)\mesh(k).
\eeqs
Note that $\mesh(k)\rightarrow 0$ as $k\rightarrow \infty.$
\eas

Observe that for a given $k, \omega$ there is no guarantee that $\Qhtilde(\omega)$ exists. Therefore, we follow [Graham, Parkinson, Scheichl] and define
\beq\label{eq:hmaxomega}
\hmaxomega \de \Cmesh(\omega)\mesh(k).
\eeq
We then define
\beqs
\homega \de \max\set{h,\hmaxomega}
\eeqs
and subsequently define
\beqs
\Qh(\omega) \de \Qtildehomega.
\eeqs
That is, the random variable $\Qh$ is (thinking about things in terms of PDEs etc.) the QoI evaluated at the numerical solution, where that solution is taken on a mesh that is the finer of $h$ and $\hmaxomega$. This guarantees the QoI exists, and the error bounds below hold.

\bre[What is $\mesh(k)$?]
If nontrapping, $\mesh(k)=k^{-3/2}$. If trapping, more stringent, nothing proved in literature, but would expect to be similar to results for contant wavespeed.
\ere

With this setup in place, we define the following quantities.

We define the correction operators\optodo{You may be able to save some time computing these - if both $\hl$ and $\hlmo$ are larger that $\hmaxomega$, then the difference between them is zero.} between the levels by $\Yl \de \Qhl - \Qhlmo, l \geq 1,$ $\Yz = \Qhz.$ We let $\Ylhat$ be an unbiased estimator of $\Yl$, i.e., $\EXP{\Ylhat} = \EXP{\Yl}.$ In what follows $\Ylhat$ will be the Monte Carlo estimator
 \beqs
\Ylhat \de \frac1{\Nl}\sum_{i=1}^{\Nl} \Yli,
 \eeqs
 with $\Nl$ to be chosen, where $\Yli$ denotes independent samples of $\Yl$. Finally we are able to define the \defn{multi-level Monte Carlo estimator}
 \beqs
 \QhatMLhL \de \sum_{l=1}^L \Ylhat,
 \eeqs
 where the $\Ylhat$ are independent.

  The following assumptions
  % \lcnamecrefs{ass:coarse}
   will form the backbone of our analysis. They are a generalisation of the assumptions contained in \cite{ClGiScTe:11,ChScTe:13} for the MLMC method, the generalisation being that we assume that the quantities below depend not only on the levels $\hl$ but also on some additional parameter $k>1.$ When this theory is applied to the Helmholtz equation, $k$ will be the wavenumber of the Helmholtz equation.

%% The following assumption (which will be realised in a more concrete setting for the Helmholtz equation) concerns the existence of the approximating QoIs $\Qhl.$

%% \bas[Existence of $\Qhl$]\label{ass:qoie}
%% There exist $\Ccoarse,\coarseexp > 0$ with $\Ccoarse$ independent of $k$ such that if
%% \beqs
%% \hl \leq \Ccoarse k^{-\coarseexp},
%% \eeqs
%% then the QoI $\Qhl$ exists.
%% \eas

\bas[Convergence of numerical method]\label{ass:a}
There exist $\co, \alpha, \sigma> 0$, such that $\co$ is independent of $h$ and $k$, and
\beqs
\abs{\EXP{\Qh-Q}} \leq \co k^\sigma h^{\alpha}.
\eeqs
\eas

\bas[Variance of correction operators]\label{ass:b}
There exist $\ct, \beta, \tau > 0$, such that $\ct$ is independent of $h$ and $k,$ and
\beqs
\Vl \de \VAR{\Yl} \leq \ct k^\tau\hl^{\beta},
\eeqs  where $\VAR{\cdot}$ denotes variance.
\eas

\bas[Cost of one sample]\label{ass:costone}
There exist $\cthtilde, \gamma > 0$ such that $\cthtilde$ is independent of $h$ and $k$, and if $\Qhtilde(\omega)$ exists, then
\beqs
\Cost{\Qhtilde(\omega)} \leq \cthtilde(\omega) h^{-\gamma},
\eeqs
\eas

In order to obtain a nice expression for the cost of computing one sample of $\Qh,$ we require the following assumption on the coarse space:

\bas[Dependence of coarse space on $k$]\label{ass:coarse}
We let
\beqs
\hz = \Ccoarse \mesh(k).
\eeqs
for some chosen constant $\Ccoarse > 0.$
\eas

\ble[Expected cost of one sample]\label{lem:c}
If
\beq\label{eq:cass}
\cthtilde \in \LpO\text{ for some }p \geq 1 \tand 1/\Cmesh \in \LqgammaO \tfor q \text{ the conjugate exponent of } p,
\eeq
then
\beq\label{eq:singlecost}
\EXP{\Cost{\Qh}} \leq \cth h^{-\gamma},
\eeq
where $\cth = \NLoO{\cthtilde} + \Ccoarse^\gamma \NLpO{\cthtilde}\NLqgammaO{\Cmesh^{-1}}^\gamma.$
\ele

\bpf[Proof of \cref{lem:c}]
The proof follows closely that in \cite[Lemma 5.8]{GrScPa:19}.

We have
\begin{align}
\EXP{\Cost{\Qh}} &= \int_{\Omega} \Cost{\Qhomegatilde(\omega)} \ddPPomega\nonumber\\
&\leq \int_\Omega \cthtilde(\omega) \homega^{-\gamma} \ddPPomega \text{ by \eqref{eq:singlecost}}\nonumber\\
&\leq \int_{\Omega} \cthtilde(\omega) \mleft(h^{-\gamma} +  \mleft(\hmaxomega\mright)^{-\gamma} \mright) \ddPPomega \text{ as } \homega = \max\set{h,\hmaxomega} \leq h + \hmaxomega\nonumber\\
&=\int_{\Omega} \cthtilde(\omega) \mleft(h^{-\gamma} + \Cmesh^{-\gamma} \mesh(k)^{-\gamma}\mright)\ddPPomega \text{ by \eqref{eq:hmaxomega}}\nonumber\\
= h^{-\gamma} \NLoO{\cthtilde} + \mesh(k)^{-\gamma}\EXP{\cthtilde\Cmesh^{-\gamma}}\label{eq:cfinal}
\end{align}
As the assumptions \eqref{eq:cass} hold, the result follows.
\epf
 
% We write $\Vl$ for $\VAR{\Yl}.$
 
 We want to determine the choices of $L$ and $\Nl, l = 0,\ldots,L,$ such that the root-mean-squared eror (RMSE)
 \beqs
 \err{\QhatMLhL} \de \mleft(\EXP{\mleft(\QhatMLhL - \EXP{Q}\mright)^2}\mright)^{\half}
 \eeqs
 satisfies $\err{\QhatMLhL} \leq \eps,$ for some pre-defined $\eps > 0.$

The proof of the main \lcnamecref{thm:mlmccomp} will require the following \lcnamecref{lem:sumbound}.

\ble\label{lem:sumbound}
If $L$ is given by
\beq\label{eq:Ldef}
L = \ceil{\frac1\alpha\log_{s}\mleft(\sqrt{2}\co  \Ccoarse^\alpha k^{\sigma}\mesh(k)^\alpha \eps^{-1}\mright)},
\eeq
then, for $s>1$ and $\delta \in \RR,$ we have the bound
\beq\label{eq:sumbound}
\sum_{l=0}^{L} s^{\delta l} \leq
\begin{cases}
L+1 & \tif \delta = 0,\\
\frac{\mleft(\sqrt{2}\co\mright)^{\frac\delta\alpha}\Ccoarse^{\delta}s^{\delta}}{1-s^{-\delta}}k^{\frac{\delta\sigma}{\alpha}}\mesh(k)^\delta\eps^{-\frac\delta\alpha} &\tif \delta >0\\
\frac{\mleft(\sqrt{2}\co\mright)^{\frac\delta\alpha}\Ccoarse^{\delta}}{1-s^{-\delta}}k^{\frac{\delta\sigma}{\alpha}}\mesh(k)^\delta\eps^{-\frac\delta\alpha}&\tif \delta < 0
\end{cases}
\eeq
\ele

\bpf[Proof of \cref{lem:sumbound}]
The proof follows that in \cite{ClGiScTe:11}. We first observe that, since $L$ is given by \eqref{eq:Ldef}, it follows that
\beq\label{eq:Lbounds}
\frac1\alpha\log_s\mleft(\sqrt{2}\co\Ccoarse^\alpha k^{\sigma}\mesh(k)^\alpha \eps^{-1}\mright) \leq L < \frac1\alpha\log_s\mleft(\sqrt{2}\co\Ccoarse^\alpha k^{\sigma}\mesh(k)^\alpha \eps^{-1}\mright) + 1.
\eeq
Rearranging \eqref{eq:Lbounds}, we obtain the bounds
\beq\label{eq:saLbounds}
\sqrt{2}\co \Ccoarse^\alpha k^{\sigma}\mesh(k)^\alpha\eps^{-1} \leq s^{\alpha L} < \sqrt{2}\co \Ccoarse^\alpha k^{\sigma}\mesh(k)^\alpha\eps^{-1}s^\alpha.
\eeq
If $\delta > 0,$ then we use the right-hand bound in \eqref{eq:saLbounds} to obtain
\beq\label{eq:sdLpos}
s^{\delta L} < \mleft(\sqrt{2}\co\mright)^{\frac\delta\alpha}\Cppw^{\delta}k^{\frac{\delta\sigma}{\alpha}}\mesh(k)^\delta\eps^{-\frac\delta\alpha}s^{\delta},
\eeq
and if $\delta < 0,$ we use the left-hand bound in \eqref{eq:saLbounds} to obtain
\beq\label{eq:sdLneg}
s^{\delta L} \leq \mleft(\sqrt{2}\co\mright)^{\frac\delta\alpha}\Cppw^{\delta}k^{\frac{\delta\sigma}{\alpha}}\mesh(k)^\delta\eps^{-\frac\delta\alpha}.
\eeq
We now observe that, for $\delta \neq 0,$
\begin{align}
\sum_{l=0}^L s^{\delta l} &= \frac{s^{\delta\mleft(L+1\mright)} -1}{s^{\delta}-1}\nonumber\\
&= \frac{s^{\delta L} - s^{-\delta}}{1-s^{-\delta}}\nonumber\\
&\leq \frac{s^{\delta L}}{1-s^{-\delta}},\label{eq:ssumbound}
\end{align}
since $s^{-\delta} > 0,$ as $s >0.$ Combining \eqref{eq:ssumbound} with \eqref{eq:sdLpos} and \eqref{eq:sdLneg}, we obtain \eqref{eq:sumbound} in the cases $\delta \neq 0.$ The case $\delta=0$ is straightforward.
\epf


%
 \paragraph{The nice case, where $k^{-\sigma/\alpha} \lesssim k^{-\coarseexp}.$}
\optodo{Might need todo something with the constants, as we need $\hL$ (as calculated) $< \hz.$ ensuring the constants are monotone is probably sufficient, as it'll just mean `for $\eps$ sufficiently small'.}
 The following theorem describes the computational effort needed to obtain RMSE $\leq \eps$. It is exactly the same as \cite[Theorem 1]{ClGiScTe:11}, but with the dependence on all the parameters explicit.%, and with some additional cases enumerated. %\Cref{thm:mlmccomp} contains more cases than in \cite[Theorem 1]{ClGiScTe:11} because \cite[Theorem 1]{ClGiScTe:11} makes the assumption throughout that $\alpha \geq 1/2\min\set{\beta,\gamma}.$ This assumption does not always hold for the Helmholtz equation (see the cases of a direct solver in 3-D below), however, examining the proof of \cite[Theorem 1]{ClGiScTe:11}  shows that in any given case, one only needs the assumption $\alpha \geq \beta/2$ or the assumption $\alpha \geq \gamma2$, never both at the same time. Therefore, for convenience, we explicitly state when these conditions are needed, and for completeness, we give the results when these conditions are violated. 

  The next two \lcnamecref{ass:powersnice}\optodo{plural} means that the restriction on the coarse space in \cref{ass:coarse} do not come into play,

 \bas[Epsilon sufficiently small]\label{ass:constants}
 Assume
 \beqs
\eps \leq \sqrt{2} \co \Ccoarse^{\alpha}.
 \eeqs
 \eas

 \bas\label{ass:powersnice}
 Suppose
 \beqs
\frac{\sigma}{\alpha} \geq \coarseexp.
 \eeqs
 \eas
 
 \bth[MLMC Complexity Theorem]\label{thm:mlmccomp}
If \cref{ass:constants,ass:powersnice} hold, $L$ is given by
\beq\label{eq:Lcond}
L = \ceil{\frac1\alpha\log_{s}\mleft(\sqrt{2}\co  \Ccoarse^\alpha k^{\sigma-\coarseexp\alpha} \eps^{-1}\mright)},
\eeq
that is,
\beqs
\hL \leq \mleft(\frac{\eps}{\sqrt{2}\co k^\sigma}\mright)^{\frac1\alpha},
\eeqs
and the number of samples on each computational level is given by
\beqs
\Nl = \ceil{\frac2{\eps^{2}} \mleft(\frac{\Vl}{\Cl}\mright)^{\half}\sum_{j=0}^{L} \mleft(\Vj\Cj\mright)^{\half}},
\eeqs
then computational effort $\CMLhL(\eps)$ required to obtain $\err{\QhatMLhL} \leq \eps$ satisfies the bounds
 
 \begin{numcases}{ \CMLhL(\eps) \lesssim}
 k^{\tau + \rho+\coarseexp\mleft(\gamma - \beta\mright)}\eps^{-2}\mleft(\log_s\mleft(\sqrt{2}\co\Cppw^\alpha k^{\sigma-\coarseexp\alpha} \eps^{-1}\mright)+2\alpha\mright)^2 +  k^{\rho +  \frac{\gamma\sigma}\alpha}\eps^{-\frac\gamma\alpha}
 & if $\beta = \gamma$,\label{eq:mlmchheq}\\ 
k^{\tau + \rho+\mleft(\gamma-\beta\mright)\frac\sigma\alpha}\eps^{-2+\mleft(\frac{\beta-\gamma}{\alpha}\mright)}
 +  k^{\rho +  \frac{\gamma\sigma}\alpha}\eps^{-\frac\gamma\alpha} & otherwise.\label{eq:mlmchhoth}
\end{numcases}
 \enth
 \optodo{Need to say why the latter two cases are the same - in one case $\gamma/\alpha$ dominates, and in the other case the other term dominates? (At least in the Cliffe et. al. set up)}
 \bpf[Proof of \cref{thm:mlmccomp}]
 \ednote{This isn't all the details of the proof, but the bits I've skipped over are exactly the same as those in {\cite{ClGiScTe:11}}.}
 
We first decompose the (squared) mean-squared error into the bias error and the sampling error:

\beqs
\errQhatMLhL^2 = \mleft(\EXP{\QhatMLhL} - \EXP{Q}\mright)^2 + \underbrace{\EXP{\mleft(\QhatMLhL - \EXP{\QhatMLhL}\mright)^2}}_{V\de},
\eeqs
the first term is the \emph{bias}, and the second term is the \emph{variance} of the estimator $\QhatMLhL.$ We now proceed to choose the parameters $L$ and $\Nl, l = 0,\ldots,L$ such that we can bound both the bias and the variance by $\eps^2/2.$

We first bound the bias, to do this, we only need to choose $L.$ One can show\ednote{As in {\cite{ClGiScTe:11}}} that the bias is equal to $\abs{\EXP{\QhL - Q}}^2.$ By \cref{ass:constants,ass:powersnice}, we don't need to worry about the coarse mesh restriction\optodo{Make this proper speak}. Therefore a sufficient condition for the bias to be $\leq \eps^2/2$ is (by \cref{ass:a})
\beqs
\co k^\sigma \hL^\alpha \leq \frac{\eps}{\sqrt{2}},
\eeqs
that is
\beq\label{eq:hLcond}
\hL \leq \mleft(\frac{\eps}{\sqrt{2}\co k^\sigma}\mright)^{\frac1\alpha}.
\eeq
\ednote{Observe that if $Q$ is the weighted $H^1$ norm, then we assume (see below for details) $\alpha=2$ and $\sigma=3,$ so we require $\hL \lesssim k^{-\frac32}.$ If we take $Q$ to be the $L^2$ norm, and assume $\alpha=2$ and $\sigma=2,$ then we only require $\hL \lesssim k^{-1}.$}

As $\hL = \hz s^{-L},$ it follows from \eqref{eq:hLcond} that a sufficient condition for the bias to be $\leq \eps^2/2$ is
\beq\label{eq:Lcondpart}
L = \ceil{\frac1\alpha\log_s\mleft(\sqrt{2}\co k^\sigma \hz^\alpha \eps^{-1}\mright)}.
\eeq
As $\hz = \Ccoarse k^{-\coarseexp},$ we can simplify \eqref{eq:Lcondpart} to obtain \eqref{eq:Lcond}.
% \beqs
% L = \ceil{\frac1\alpha\log_s\mleft(\sqrt{2}\co\Ccoarse^\alpha k^{\sigma-\coarseexp\alpha} \eps^{-1}\mright)}.
% \eeqs

We now seek to bound the variance. One can show\ednote{Again, as in \cite{ClGiScTe:11}} the variance $V = \sum_{l=0}^L \Nl^{-1} \Vl,$ and the cost is $\cC = \sum_{l=0}^L \Nl \Cl.$

To find the optimal number of samples per level (the values of $\Nl, l=0,\ldots,L$) we formulate this as an optimisation problem to find $\Nl$ that minimise $\cC$, subject to $V=\eps/2.$ This can be solved using a Lagrange multiplier as in \cite{Gi:15}, and we obtain
\beq\label{eq:Nl}
\Nl = \ceil{\frac2{\eps^{2}} \mleft(\frac{\Vl}{\Cl}\mright)^{\half}\sum_{j=0}^L \mleft(\Vj\Cj\mright)^{\half}}.
\eeq
\optodo{Check this is correct, should it be divided by the sum?}
We now just need to infer the computational complexity for MLMC with $L$ given by \eqref{eq:Lcond} and the $\Nl$ given by \eqref{eq:Nl}.

The computational complexity $\cC$ is given by
\begin{align}
\cC &= \sum_{l=0}^{L} \Cl \Nl\nonumber\\
&\leq \sum_{l=0}^L \Cl \mleft(\frac2{\eps^{2}} \mleft(\frac{\Vl}{\Cl}\mright)^{\half}\sum_{j=0}^L \mleft(\Vj\Cj\mright)^{\half} + 1\mright) \text{ (by \eqref{eq:Nl})}\nonumber\\
&= 2\eps^{-2}\mleft(\sum_{l=0}^L\mleft(\Vl\Cl\mright)^{\half}\mright)^2 + \sum_{l=0}^L \Cl\nonumber\\
&\leq 2 \ct \cth k^{\tau + \rho}\eps^{-2} \mleft(\sum_{l=0}^L \hl^{\frac{\beta-\gamma}2}\mright)^2 + \cth k^\rho \sum_{l=0}^L \hl^{-\gamma} \text{ (by \cref{ass:b,ass:c})}\nonumber\\
&= 2 \ct\cth k^{\tau + \rho}\eps^{-2}\hz^{\beta-\gamma}\mleft(\sum_{l=0}^L s^{-l\mleft(\frac{\beta-\gamma}2\mright)}\mright)^2 + \cth k^\rho \hz^{-\gamma} \sum_{l=0}^L s^{\gamma l} \text{ (by definition of } \hl\text{ )}\nonumber\\
&=2 \ct\cth \Cppw^{\beta-\gamma}k^{\tau + \rho+\coarseexp\mleft(\gamma - \beta\mright)}\eps^{-2}\mleft(\sum_{l=0}^L s^{l\mleft(\frac{\gamma-\beta}2\mright)}\mright)^2 + \cth\Cppw^{-\gamma} k^{\rho + \gamma\coarseexp}  \sum_{l=0}^L s^{\gamma l} \text{ (by definition of } \hz\text{ )}\nonumber\\
&\leq2\ct\cth \Cppw^{\beta-\gamma}k^{\tau + \rho+\coarseexp\mleft(\gamma - \beta\mright)}\eps^{-2}\mleft(\sum_{l=0}^L s^{l\mleft(\frac{\gamma-\beta}2\mright)}\mright)^2 +  \frac{\mleft(\sqrt{2}\co\mright)^{\frac\gamma\alpha}\cth s^{\gamma}}{1-s^{-\gamma}}k^{\rho +  \frac{\gamma\sigma}\alpha}\eps^{-\frac\gamma\alpha} \text{ (since }\gamma>0,\text{ by \cref{lem:sumbound})}.\label{eq:complexitymidway}
\end{align}
To bound the sum in the first part of \eqref{eq:complexitymidway}, we must distinguish three cases based on $\gamma - \beta.$

If $\gamma=\beta,$ then \eqref{eq:complexitymidway} becomes (using \cref{lem:sumbound})
\begin{multline}
2 \ct\cth \Cppw^{\beta-\gamma}k^{\tau + \rho+\coarseexp\mleft(\gamma - \beta\mright)}\eps^{-2}\mleft(L+1\mright)^2 +  \frac{\mleft(\sqrt{2}\co\mright)^{\frac\gamma\alpha}\cth s^{\gamma}}{1-s^{-\gamma}}k^{\rho +  \frac{\gamma\sigma}\alpha}\eps^{-\frac\gamma\alpha}\\
\leq
2\ct\cth \Cppw^{\beta-\gamma}k^{\tau + \rho+\coarseexp\mleft(\gamma - \beta\mright)}\eps^{-2}\mleft(\frac1\alpha\log_s\mleft(\sqrt{2}\co\Cppw^\alpha k^{\sigma-\coarseexp\alpha} \eps^{-1}\mright)+2\mright)^2 +  \frac{\mleft(\sqrt{2}\co\mright)^{\frac\gamma\alpha}\cth s^{\gamma}}{1-s^{-\gamma}}k^{\rho +  \frac{\gamma\sigma}\alpha}\eps^{-\frac\gamma\alpha}\label{eq:gammaequal}
\end{multline}
by \eqref{eq:Lcond}.

If $\gamma > \beta$ then by \cref{lem:sumbound} \eqref{eq:complexitymidway} becomes
\beq
2 \ct\cth \Cppw^{\beta-\gamma}
\frac{\mleft(\sqrt{2}\co\mright)^{\mleft(\frac{\gamma-\beta}{\alpha}\mright)}\Cppw^{\mleft(\gamma-\beta\mright)}s^{\gamma-\beta}}{\mleft(1-s^{\mleft(\frac{\beta-\gamma}2\mright)}\mright)^{2}}k^{\tau + \rho+\mleft(\gamma-\beta\mright)\frac\sigma\alpha}\eps^{-2+\mleft(\frac{\beta-\gamma}{\alpha}\mright)}
 +  \frac{\mleft(\sqrt{2}\co\mright)^{\frac\gamma\alpha}\cth s^{\gamma}}{1-s^{-\gamma}}k^{\rho +  \frac{\gamma\sigma}\alpha}\eps^{-\frac\gamma\alpha}.\label{eq:gammagtr}
\eeq
If $\gamma < \beta,$ then by \cref{lem:sumbound} \eqref{eq:complexitymidway} becomes
\beq
2 \ct\cth \Cppw^{\beta-\gamma}
\frac{\mleft(\sqrt{2}\co\mright)^{\mleft(\frac{\gamma-\beta}{\alpha}\mright)}\Cppw^{\mleft(\gamma-\beta\mright)}}{\mleft(1-s^{\mleft(\frac{\beta-\gamma}2\mright)}\mright)^{2}}k^{\tau + \rho+\mleft(\gamma-\beta\mright)\frac\sigma\alpha}\eps^{-2+\mleft(\frac{\beta-\gamma}{\alpha}\mright)}
 +  \frac{\mleft(\sqrt{2}\co\mright)^{\frac\gamma\alpha}\cth s^{\gamma}}{1-s^{-\gamma}}k^{\rho +  \frac{\gamma\sigma}\alpha}\eps^{-\frac\gamma\alpha},\label{eq:gammaless}
\eeq
the only difference from \eqref{eq:gammagtr} being the loss of the $s^{\gamma-\beta}$ term.

Removing all the terms that are not of interest from \eqref{eq:gammaequal}, \eqref{eq:gammagtr}, and \eqref{eq:gammaless}, we obtain \eqref{eq:mlmchheq} and \eqref{eq:mlmchhoth}.
\epf


The following theorem describes the computational effort needed to obtain RMSE $\leq \eps$. It is exactly the same as \cite[Theorem 1]{ClGiScTe:11}, but with the dependence on all the parameters explicit.%, and with some additional cases enumerated. %\Cref{thm:mlmccomp} contains more cases than in \cite[Theorem 1]{ClGiScTe:11} because \cite[Theorem 1]{ClGiScTe:11} makes the assumption throughout that $\alpha \geq 1/2\min\set{\beta,\gamma}.$ This assumption does not always hold for the Helmholtz equation (see the cases of a direct solver in 3-D below), however, examining the proof of \cite[Theorem 1]{ClGiScTe:11}  shows that in any given case, one only needs the assumption $\alpha \geq \beta/2$ or the assumption $\alpha \geq \gamma2$, never both at the same time. Therefore, for convenience, we explicitly state when these conditions are needed, and for completeness, we give the results when these conditions are violated. 

%% The following \lcnamecref{ass:constants3} will ensure that \cref{ass:qoie} is satisfied.

%%  \bas[$\eps$ sufficiently small]\label{ass:constants3}
%%  Assume
%%  \beqs
%% \eps \leq \sqrt{2} \co \Ccoarse^{\alpha} k^{\sigma-\coarseexp\alpha}.
%%  \eeqs
%%  \eas

\paragraph{Notation}

$\Cl \de \cth\hl^{-\gamma}$

$\cC \de \EXP{\Cost{\QhatMLhL}}$

\bas[Assumptions on $\eps$ and $k$ to make things nicer]\label{ass:epsk}
\beqs
\eps \leq \min\set{\frac{\sqrt{2}\co\Ccoarse^\alpha}{s^{2\alpha}},\frac1{\sqrt{2}\co\Ccoarse^\alpha}},
\eeqs
and
\beqs
k^\sigma \mesh(k)^\alpha \geq 1.
\eeqs
\eas

\bth[MLMC Complexity Theorem]\label{thm:mlmccomp}
If \cref{ass:epsk} holds,  $L$ is given by \eqref{eq:Ldef}
that is,
\beq\label{eq:hLcond}
\hL \leq \mleft(\frac\eps{\sqrt{2}\co k^{\sigma}}\mright)^{\frac1\alpha},
\eeq
and the number of samples on each computational level is given by
\beqs
\Nl = \ceil{\frac2{\eps^{2}} \mleft(\frac{\Vl}{\Cl}\mright)^{\half}\sum_{j=0}^{L} \mleft(\Vj\Cj\mright)^{\half}},
\eeqs
then computational effort $\CMLhL(\eps)$ required to obtain $\err{\QhatMLhL} \leq \eps$ satisfies the bounds
 
 \begin{numcases}{ \CMLhL(\eps) \lesssim}
k^\tau \mleft(\log\mleft(\frac{k^\sigma \mesh(k)^\alpha}\eps\mright)\mright)^2 \eps^{-2} + k^{\frac{\gamma\sigma}\alpha} \eps^{-\frac\gamma\alpha}  & if $\beta = \gamma$,\label{eq:mlmchheq}\\ 
k^{\tau + \mleft(\gamma-\beta\mright)\frac\sigma\alpha} \eps^{-2-\frac{\gamma-\beta}{\alpha}} + k^{\frac{\gamma\sigma}{\alpha}}\eps^{-\frac\gamma\alpha} & otherwise.\label{eq:mlmchhoth}
\end{numcases}
 \enth
 \bpf[Proof of \cref{thm:mlmccomp}]
We first decompose the (squared) mean-squared error into the bias error and the sampling error:

\beqs
\errQhatMLhL^2 = \mleft(\EXP{\QhatMLhL} - \EXP{Q}\mright)^2 + \underbrace{\EXP{\mleft(\QhatMLhL - \EXP{\QhatMLhL}\mright)^2}}_{V\de},
\eeqs
the first term is the \emph{bias}, and the second term is the \emph{variance} of the estimator $\QhatMLhL.$ We now proceed to choose the parameters $L$ and $\Nl, l = 0,\ldots,L$ such that we can bound both the bias and the variance by $\eps^2/2.$

We first bound the bias, to do this, we only need to choose $L.$ One can show that the bias is equal to $\abs{\EXP{\QhL - Q}}^2.$ Therefore a sufficient condition for the bias to be $\leq \eps^2/2$ is (by \cref{ass:a})
\beqs
\co k^\sigma \hL^\alpha \leq \frac{\eps}{\sqrt{2}},
\eeqs
that is, \eqref{eq:hLcond}. As $\hL = \hz s^{-L},$ it follows from \eqref{eq:hLcond} that a sufficient condition for the bias to be $\leq \eps^2/2$ is
\beq\label{eq:Lcondpart}
L = \ceil{\frac1\alpha\log_s\mleft(\sqrt{2}\co k^\sigma \hz^\alpha \eps^{-1}\mright)}.
\eeq
As $\hz = \Ccoarse \mesh(k),$ we can simplify \eqref{eq:Lcondpart} to obtain \eqref{eq:Ldef}.
% \beqs
% L = \ceil{\frac1\alpha\log_s\mleft(\sqrt{2}\co\Ccoarse^\alpha k^{\sigma-\coarseexp\alpha} \eps^{-1}\mright)}.
% \eeqs

We now seek to bound the variance. One can show the variance $V = \sum_{l=0}^L \Nl^{-1} \Vl,$ and the cost is: (following \cite{GrPaSc:19})
\begin{align}
\cC &= \EXP{\Cost{\QhatMLhL}}\nonumber\\
&\leq \sum_{l=0}^L \EXP{\Cost{\Ylhat}}\text{less equal as could save if you've had to over-refine on both levels in $\Ylhat$ - difference will be zero}\nonumber\\
&= \sum_{l=0}^L \sum_{i=1}^{\Nl} \EXP{\Cost{\Yli}}\nonumber\\
&\leq \sum_{l=0}^L \sum_{i=0}^{\Nl} \mleft(\EXP{\Cost{\Qhl}} + \EXP{\Cost{\Qhlmo}}\mright)\nonumber\\
&\leq \sum_{l=0}^L \Nl \mleft(\cth \hl^{-\gamma} + \cth \hlmo^{-\gamma}\mright)\nonumber\\
&=\sum_{l=0}^L \Nl\mleft(1+s^{-\gamma}\mright) \cth \hl^{-\gamma}\nonumber\\
&= \mleft(1+s^{-\gamma}\mright) \sum_{l=0}^L \Nl\Cl\label{eq:Cboundformin}
\end{align}

To find the optimal number of samples per level (the values of $\Nl, l=0,\ldots,L$) we formulate this as an optimisation problem to find $\Nl$ that minimise \eqref{eq:Cboundformin}, subject to $V=\eps/2.$ This can be solved using a Lagrange multiplier as in \cite{Gi:15}, and we obtain
\beq\label{eq:Nl}
\Nl = \ceil{\frac2{\eps^{2}} \mleft(\frac{\Vl}{\Cl}\mright)^{\half}\sum_{j=0}^L \mleft(\Vj\Cj\mright)^{\half}}.
\eeq
We now just need to infer the computational complexity for MLMC with $L$ given by \eqref{eq:Ldef} and the $\Nl$ given by \eqref{eq:Nl}.

The computational complexity $\cC$ is given by
\begin{align}
\cC &\leq \mleft(1+s^{-\gamma}\mright)\sum_{l=0}^{L} \Cl \Nl\nonumber\\
&\leq \mleft(1+s^{-\gamma}\mright)\sum_{l=0}^L \Cl \mleft(\frac2{\eps^{2}} \mleft(\frac{\Vl}{\Cl}\mright)^{\half}\sum_{j=0}^L \mleft(\Vj\Cj\mright)^{\half} + 1\mright) \text{ (by \eqref{eq:Nl})}\nonumber\\
&= 2\eps^{-2}\mleft(1+s^{-\gamma}\mright)\mleft(\sum_{l=0}^L\mleft(\Vl\Cl\mright)^{\half}\mright)^2 + \mleft(1+s^{-\gamma}\mright)\sum_{l=0}^L \Cl\nonumber\\
&= 2 \ct \cth \mleft(1+s^{-\gamma}\mright)k^{\tau}\eps^{-2} \mleft(\sum_{l=0}^L \hl^{\frac{\beta-\gamma}2}\mright)^2 + \cth \mleft(1+s^{-\gamma}\mright) \sum_{l=0}^L \hl^{-\gamma} \text{ (by \cref{ass:b,ass:c})}\nonumber\\
&= 2 \ct\cth \mleft(1+s^{-\gamma}\mright)k^{\tau}\eps^{-2}\hz^{\beta-\gamma}\mleft(\sum_{l=0}^L s^{l\mleft(\frac{\gamma-\beta}2\mright)}\mright)^2 + \cth\mleft(1+s^{-\gamma}\mright) \hz^{-\gamma} \sum_{l=0}^L s^{\gamma l} \text{ (by definition of } \hl\text{ )}\label{eq:complexitymidway}\\
%% &=2 \ct\cth \Cppw^{\beta-\gamma}k^{\tau + \rho+\coarseexp\mleft(\gamma - \beta\mright)}\eps^{-2}\mleft(\sum_{l=0}^L s^{l\mleft(\frac{\gamma-\beta}2\mright)}\mright)^2 + \cth\Cppw^{-\gamma} k^{\rho + \gamma\coarseexp}  \sum_{l=0}^L s^{\gamma l} \text{ (by definition of } \hz\text{ )}\nonumber\\
%% &\leq2\ct\cth \Cppw^{\beta-\gamma}k^{\tau + \rho+\coarseexp\mleft(\gamma - \beta\mright)}\eps^{-2}\mleft(\sum_{l=0}^L s^{l\mleft(\frac{\gamma-\beta}2\mright)}\mright)^2 +  \frac{\mleft(\sqrt{2}\co\mright)^{\frac\gamma\alpha}\cth s^{\gamma}}{1-s^{-\gamma}}k^{\rho +  \frac{\gamma\sigma}\alpha}\eps^{-\frac\gamma\alpha} \text{ (since }\gamma>0,\text{ by \cref{lem:sumbound})}.\label{eq:complexitymidway}
\end{align}

Using \cref{lem:sumbound}, the second term in \eqref{eq:complexitymidway} can be bounded (as $\gamma > 0$) by %(letting \csumdelta \de \mleft(\sqrt{2}\co\mright)^{\frac\delta\alpha} \Ccoarse^\delta / \mleft(1-s^{-\delta}\mright)$)
\beq\label{eq:firstterm}
\frac{\mleft(1+s^{-\gamma}\mright) \cth \hz^{-\gamma} \mleft(\sqrt{2}\co\mright)^{\frac\gamma\alpha} s^\gamma \Ccoarse^\gamma}{1-s^{-\gamma}} k^{\frac{\gamma\sigma}\alpha} \mesh(k)^\gamma \eps^{-\frac\gamma\alpha}
= \frac{\mleft(1+s^{-\gamma}\mright)\cth \mleft(\sqrt{2}\co\mright)^{\frac\gamma\alpha} s^\gamma}{1-s^{-\gamma}} k^{\frac{\gamma\sigma}\alpha} \eps^{-\frac\gamma\alpha}
\eeq

To bound the sum in the first part of \eqref{eq:complexitymidway}, we must distinguish three cases based on $\gamma - \beta.$


If $\gamma=\beta,$ then the first part of \eqref{eq:complexitymidway} becomes (using \cref{lem:sumbound})
\beq
2 \ct\cth \mleft(1+s^{-\gamma}\mright)k^{\tau}\eps^{-2}\mleft(L+1\mright)^2 \leq 2 \ct\cth \mleft(1+s^{-\gamma}\mright)k^{\tau}\eps^{-2}\mleft(\frac1\alpha \log_s \mleft(\frac{\sqrt{2} \co \Ccoarse^\alpha k^\sigma \mesh(k)^\alpha}\eps\mright)+2\mright)^2,
\label{eq:gammaequal}
\eeq
by \eqref{eq:Ldef}. We wish to simplify \eqref{eq:gammaequal}, so that it is of the form $\mathrm{Constant} \times \text{Terms involving } \eps \text{ and } k.$ To achieve this simplification, we use \cref{ass:epsk}. As $k^\sigma \mesh(k)^\alpha \geq 1$ and $\eps \leq \mleft(\sqrt{2} \co \Ccoarse^{\alpha}\mright)/s^{2\alpha},$ it follows that
\beqs
2 \leq \frac1\alpha \log_s \mleft(\frac{\sqrt{2} \co \Ccoarse^\alpha k^\sigma \mesh(k)^\alpha}\eps\mright),
\eeqs
and thus \eqref{eq:gammaequal} can be bounded by
\beq\label{eq:gammaequalpart1}
8 \ct\cth \mleft(1+s^{-\gamma}\mright)k^{\tau}\eps^{-2}\mleft(\frac1\alpha \log_s \mleft(\frac{\sqrt{2} \co \Ccoarse^\alpha k^\sigma \mesh(k)^\alpha}\eps\mright)\mright)^2.
\eeq
As $k^\sigma \mesh(k)^\alpha \geq 1$ and $\eps \leq 1/\mleft(\sqrt{2}\co\Ccoarse^\alpha\mright),$ we can bound \eqref{eq:gammaequalpart1} by (including a change of base in the logarithm)
\beq\label{eq:gammaequalfinal}
\frac{32 \ct\cth \mleft(1+s^{-\gamma}\mright)}{\alpha^2 \mleft(\log(s)\mright)^2} k^\tau \mleft(\log\mleft(\frac{k^\sigma \mesh(k)^\alpha}\eps\mright)\mright)^2.
\eeq

For simplicity in what follows, we define
\beqs
\csumdelta \de \frac{\mleft(\sqrt{2}\co\mright)^{\frac\delta\alpha}\Ccoarse^{\delta}}{1-s^{-\delta}}.
\eeqs

If $\gamma > \beta$ then by \cref{lem:sumbound} the first term in \eqref{eq:complexitymidway} becomes\optodo{Check if the below is right - it seems to be saying the complexity is independent of the coarse mesh}
\beq
\eps^{-2}2\ct\cth \mleft(1+s^{-\gamma}\mright) k^\tau \hz^{\beta-\gamma}\mleft(\csumgammambetat s^{\frac{\gamma-\beta}2} k^{\frac{\gamma-\beta}2\frac\sigma\alpha} \mesh(k)^{\frac{\gamma-\beta}2} \eps^{-\frac{\gamma-\beta}{2\alpha}}\mright)^2 = \Cgammagtrbeta k^{\tau + \mleft(\gamma-\beta\mright)\frac\sigma\alpha} \eps^{-2-\frac{\gamma-\beta}{\alpha}},\label{eq:gammagtr}
\eeq
where
\beqs
\Cgammagtrbeta \de 2\ct\cth\mleft(1+s^{-\gamma}\mright)\csumgammambetat^2 s^{\gamma-\beta} \Ccoarse^{\beta-\gamma}.
\eeqs
If $\gamma < \beta,$ then analagously the first term in \eqref{eq:complexitymidway} is
\beqs
\Cgammalessbeta k^{\tau + \mleft(\gamma-\beta\mright)\frac\sigma\alpha} \eps^{-2-\frac{\gamma-\beta}{\alpha}},
\eeqs
where
\beqs
\Cgammalessbeta \de \frac{\Cgammagtrbeta}{s^{\gamma-\beta}}.
\eeqs

We now combine \eqref{eq:firstterm}, \eqref{eq:gammaequalfinal}, \eqref{eq:gammagtr}, and \eqref{eq:gammaless} and supress all the constants to obtain the result.
%Removing all the terms that are not of interest from \eqref{eq:gammaequal}, \eqref{eq:gammagtr}, and \eqref{eq:gammaless}, we obtain \eqref{eq:mlmchheq} and \eqref{eq:mlmchhoth}.
\epf


%
\paragraph{The nasty case, where $k^{-\sigma/\alpha} \gtrsim k^{-\coarseexp}.$}

 \bas\label{ass:powersnasty}
 Suppose
 \beqs
\frac{\sigma}{\alpha} \leq \coarseexp.
 \eeqs
 \eas

  \bas[Epsilon sufficiently small]\label{ass:constantsnasty}
 Assume
 \beqs
\eps \leq \co \Ccoarse^{\alpha}.
 \eeqs
 \eas

\bth[MLMC Complexity Theorem]\label{thm:mlmccomp2}
Assume \cref{ass:powersnasty,ass:constantsnasty} hold. Assume $k \geq 1.$ If $L$ is given by
\beq\label{eq:Lcond2}
L = \ceil{\frac1\alpha\log_{s}\mleft(\sqrt{2}\co  \Ccoarse^\alpha k^{\sigma-\coarseexp\alpha} \eps^{-1}\mright)},
\eeq
that is,
\beqs
\hL \leq \mleft(\frac{\eps}{\sqrt{2}\co k^\sigma}\mright)^{\frac1\alpha},
\eeqs
and the number of samples on each computational level is given by
\beqs
\Nl = \ceil{\frac2{\eps^{2}} \mleft(\frac{\Vl}{\Cl}\mright)^{\half}\sum_{j=0}^{L} \mleft(\Vj\Cj\mright)^{\half}},
\eeqs
then computational effort $\CMLhL(\eps)$ required to obtain $\err{\QhatMLhL} \leq \eps$ satisfies the bounds
 
 \begin{numcases}{ \CMLhL(\eps) \lesssim}
 k^{\tau + \rho+\coarseexp\mleft(\gamma - \beta\mright)}\eps^{-2}\mleft(\log_s\mleft(\sqrt{2}\co\Cppw^\alpha k^{\sigma-\coarseexp\alpha} \eps^{-1}\mright)+2\alpha\mright)^2 +  k^{\rho +  \frac{\gamma\sigma}\alpha}\eps^{-\frac\gamma\alpha}
 & if $\beta = \gamma$,\label{eq:mlmchheq2}\\ 
k^{\tau + \rho+\mleft(\gamma-\beta\mright)\frac\sigma\alpha}\eps^{-2+\mleft(\frac{\beta-\gamma}{\alpha}\mright)}
 +  k^{\rho +  \frac{\gamma\sigma}\alpha}\eps^{-\frac\gamma\alpha} & otherwise.\label{eq:mlmchhoth2}
\end{numcases}
 \enth
 \optodo{Need to say why the latter two cases are the same - in one case $\gamma/\alpha$ dominates, and in the other case the other term dominates? (At least in the Cliffe et. al. set up)}

 \bpf[Proof of \cref{thm:mlmccomp}]
 \ednote{This isn't all the details of the proof, but the bits I've skipped over are exactly the same as those in {\cite{ClGiScTe:11}}.}
Do the bias-variance decomposition.  
 We now proceed to choose the parameters $L$ and $\Nl, l = 0,\ldots,L$, however, because the dominant term in this case\optodo{show this} is the \emph{Coarse restriction} we tentatively\ednote{I have no idea at this stage whether this will work} We bound the bias by $\eps^2k^{\sigma-\alpha\coarseexp}$ and the variance by $\eps^2\mleft(1-k^{\sigma-\alpha\coarseexp}\mright).$\optodo{This is all fine by the assumptions}

We first bound the bias, to do this, we only need to choose $L.$ One can show\ednote{As in {\cite{ClGiScTe:11}}} that the bias is equal to $\abs{\EXP{\QhL - Q}}^2.$ Therefore a sufficient condition for the bias to be $\leq \eps^2k^{\sigma-\alpha\coarseexp}$ is (by \cref{ass:a})
\beqs
\co k^\sigma \hL^\alpha \leq \eps k^{\frac{\sigma-\alpha\coarseexp}2},
\eeqs
that is
\beq\label{eq:hLcond2}
\hL \leq \mleft(\frac{\eps}{\sqrt{2}\co k^\sigma}\mright)^{\frac1\alpha}.
\eeq
\ednote{Observe that if $Q$ is the weighted $H^1$ norm, then we assume (see below for details) $\alpha=2$ and $\sigma=3,$ so we require $\hL \lesssim k^{-\frac32}.$ If we take $Q$ to be the $L^2$ norm, and assume $\alpha=2$ and $\sigma=2,$ then we only require $\hL \lesssim k^{-1}.$}

As $\hL = \hz s^{-L},$ it follows from \eqref{eq:hLcond2} that a sufficient condition for the bias to be $\leq \eps^2/2$ is
\beq\label{eq:Lcondpart2}
L = \ceil{\frac1\alpha\log_s\mleft(\sqrt{2}\co k^\sigma \hz^\alpha \eps^{-1}\mright)}.
\eeq
As $\hz = \Ccoarse k^{-\coarseexp},$ we can simplify \eqref{eq:Lcondpart2} to obtain \eqref{eq:Lcond2}.
% \beqs
% L = \ceil{\frac1\alpha\log_s\mleft(\sqrt{2}\co\Ccoarse^\alpha k^{\sigma-\coarseexp\alpha} \eps^{-1}\mright)}.
% \eeqs

We now seek to bound the variance. One can show\ednote{Again, as in \cite{ClGiScTe:11}} the variance $V = \sum_{l=0}^L \Nl^{-1} \Vl,$ and the cost is $\cC = \sum_{l=0}^L \Nl \Cl.$

To find the optimal number of samples per level (the values of $\Nl, l=0,\ldots,L$) we formulate this as an optimisation problem to find $\Nl$ that minimise $\cC$, subject to $V=\eps/2.$ This can be solved using a Lagrange multiplier as in \cite{Gi:15}, and we obtain
\beq\label{eq:Nl2}
\Nl = \ceil{\frac2{\eps^{2}} \mleft(\frac{\Vl}{\Cl}\mright)^{\half}\sum_{j=0}^L \mleft(\Vj\Cj\mright)^{\half}}.
\eeq
\optodo{Check this is correct, should it be divided by the sum?}
We now just need to infer the computational complexity for MLMC with $L$ given by \eqref{eq:Lcond2} and the $\Nl$ given by \eqref{eq:Nl2}.

The computational complexity $\cC$ is given by
\begin{align}
\cC &= \sum_{l=0}^{L} \Cl \Nl\nonumber\\
&\leq \sum_{l=0}^L \Cl \mleft(\frac2{\eps^{2}} \mleft(\frac{\Vl}{\Cl}\mright)^{\half}\sum_{j=0}^L \mleft(\Vj\Cj\mright)^{\half} + 1\mright) \text{ (by \eqref{eq:Nl2})}\nonumber\\
&= 2\eps^{-2}\mleft(\sum_{l=0}^L\mleft(\Vl\Cl\mright)^{\half}\mright)^2 + \sum_{l=0}^L \Cl\nonumber\\
&\leq 2 \ct \cth k^{\tau + \rho}\eps^{-2} \mleft(\sum_{l=0}^L \hl^{\frac{\beta-\gamma}2}\mright)^2 + \cth k^\rho \sum_{l=0}^L \hl^{-\gamma} \text{ (by \cref{ass:b,ass:c})}\nonumber\\
&= 2 \ct\cth k^{\tau + \rho}\eps^{-2}\hz^{\beta-\gamma}\mleft(\sum_{l=0}^L s^{-l\mleft(\frac{\beta-\gamma}2\mright)}\mright)^2 + \cth k^\rho \hz^{-\gamma} \sum_{l=0}^L s^{\gamma l} \text{ (by definition of } \hl\text{ )}\nonumber\\
&=2 \ct\cth \Cppw^{\beta-\gamma}k^{\tau + \rho+\coarseexp\mleft(\gamma - \beta\mright)}\eps^{-2}\mleft(\sum_{l=0}^L s^{l\mleft(\frac{\gamma-\beta}2\mright)}\mright)^2 + \cth\Cppw^{-\gamma} k^{\rho + \gamma\coarseexp}  \sum_{l=0}^L s^{\gamma l} \text{ (by definition of } \hz\text{ )}\nonumber\\
&\leq2\ct\cth \Cppw^{\beta-\gamma}k^{\tau + \rho+\coarseexp\mleft(\gamma - \beta\mright)}\eps^{-2}\mleft(\sum_{l=0}^L s^{l\mleft(\frac{\gamma-\beta}2\mright)}\mright)^2 +  \frac{\mleft(\sqrt{2}\co\mright)^{\frac\gamma\alpha}\cth s^{\gamma}}{1-s^{-\gamma}}k^{\rho +  \frac{\gamma\sigma}\alpha}\eps^{-\frac\gamma\alpha} \text{ (since }\gamma>0,\text{ by \cref{lem:sumbound})}.\label{eq:complexitymidway2}
\end{align}
To bound the sum in the first part of \eqref{eq:complexitymidway2}, we must distinguish three cases based on $\gamma - \beta.$

If $\gamma=\beta,$ then \eqref{eq:complexitymidway2} becomes (using \cref{lem:sumbound})
\begin{multline}
2 \ct\cth \Cppw^{\beta-\gamma}k^{\tau + \rho+\coarseexp\mleft(\gamma - \beta\mright)}\eps^{-2}\mleft(L+1\mright)^2 +  \frac{\mleft(\sqrt{2}\co\mright)^{\frac\gamma\alpha}\cth s^{\gamma}}{1-s^{-\gamma}}k^{\rho +  \frac{\gamma\sigma}\alpha}\eps^{-\frac\gamma\alpha}\\
\leq
2\ct\cth \Cppw^{\beta-\gamma}k^{\tau + \rho+\coarseexp\mleft(\gamma - \beta\mright)}\eps^{-2}\mleft(\frac1\alpha\log_s\mleft(\sqrt{2}\co\Cppw^\alpha k^{\sigma-\coarseexp\alpha} \eps^{-1}\mright)+2\mright)^2 +  \frac{\mleft(\sqrt{2}\co\mright)^{\frac\gamma\alpha}\cth s^{\gamma}}{1-s^{-\gamma}}k^{\rho +  \frac{\gamma\sigma}\alpha}\eps^{-\frac\gamma\alpha}\label{eq:gammaequal2}
\end{multline}
by \eqref{eq:Lcond2}.

If $\gamma > \beta$ then by \cref{lem:sumbound} \eqref{eq:complexitymidway2} becomes
\beq
2 \ct\cth \Cppw^{\beta-\gamma}
\frac{\mleft(\sqrt{2}\co\mright)^{\mleft(\frac{\gamma-\beta}{\alpha}\mright)}\Cppw^{\mleft(\gamma-\beta\mright)}s^{\gamma-\beta}}{\mleft(1-s^{\mleft(\frac{\beta-\gamma}2\mright)}\mright)^{2}}k^{\tau + \rho+\mleft(\gamma-\beta\mright)\frac\sigma\alpha}\eps^{-2+\mleft(\frac{\beta-\gamma}{\alpha}\mright)}
 +  \frac{\mleft(\sqrt{2}\co\mright)^{\frac\gamma\alpha}\cth s^{\gamma}}{1-s^{-\gamma}}k^{\rho +  \frac{\gamma\sigma}\alpha}\eps^{-\frac\gamma\alpha}.\label{eq:gammagtr2}
\eeq
If $\gamma < \beta,$ then by \cref{lem:sumbound} \eqref{eq:complexitymidway2} becomes
\beq
2 \ct\cth \Cppw^{\beta-\gamma}
\frac{\mleft(\sqrt{2}\co\mright)^{\mleft(\frac{\gamma-\beta}{\alpha}\mright)}\Cppw^{\mleft(\gamma-\beta\mright)}}{\mleft(1-s^{\mleft(\frac{\beta-\gamma}2\mright)}\mright)^{2}}k^{\tau + \rho+\mleft(\gamma-\beta\mright)\frac\sigma\alpha}\eps^{-2+\mleft(\frac{\beta-\gamma}{\alpha}\mright)}
 +  \frac{\mleft(\sqrt{2}\co\mright)^{\frac\gamma\alpha}\cth s^{\gamma}}{1-s^{-\gamma}}k^{\rho +  \frac{\gamma\sigma}\alpha}\eps^{-\frac\gamma\alpha},\label{eq:gammaless2}
\eeq
the only difference from \eqref{eq:gammagtr2} being the loss of the $s^{\gamma-\beta}$ term.

Removing all the terms that are not of interest from \eqref{eq:gammaequal2}, \eqref{eq:gammagtr2}, and \eqref{eq:gammaless2}, we obtain \eqref{eq:mlmchheq2} and \eqref{eq:mlmchhoth2}.
\epf



\subsection{The lemma in generality}

\ble\label{lem:sumboundnew}
If $L$ is given by
\beq\label{eq:Ldefgen}
L = \ceil{\Lconst\log_{s}\mleft( \func \eps^{-1}\mright)},
\eeq
for some $\func > 0,$ then, for $s>1$ and $\delta \in \RR,$ we have the bound
\beq\label{eq:sumboundgen}
\sum_{l=0}^{L} s^{\delta l} \leq
\begin{cases}
L+1 & \tif \delta = 0,\\
\frac{s^{\delta}}{1-s^{-\delta}}\func^{\delta\Lconst}\eps^{-\delta\Lconst} &\tif \delta >0\\
\frac{1}{1-s^{-\delta}}\func^{\delta\Lconst}\eps^{-\delta\Lconst}&\tif \delta < 0
\end{cases}
\eeq
\beq\label{eq:sumboundLmo}
\sum_{l=0}^{L} s^{\delta l} \leq
\begin{cases}
L & \tif \delta = 0,\\
\frac{s^{\delta}}{1-s^{-\delta}}\func^{\delta\Lconst}\eps^{-\delta\Lconst} &\tif \delta >0\\
\frac{1}{1-s^{-\delta}}\func^{\delta\Lconst}\eps^{-\delta\Lconst}&\tif \delta < 0
\end{cases}
\eeq
\optodo{Tidy}
\ele

\bpf[Proof of \cref{lem:sumboundnew}]
The proof follows that in \cite{ClGiScTe:11}. We first observe that, since $L$ is given by \eqref{eq:Ldefgen}, it follows that
\beq\label{eq:Lboundsgen}
\Lconst\log_s\mleft(\func \eps^{-1}\mright) \leq L < \Lconst\log_s\mleft(\func \eps^{-1}\mright) + 1.
\eeq
Rearranging \eqref{eq:Lboundsgen}, we obtain the bounds
\beq\label{eq:saLbounds}
\mleft( \func\eps^{-1}\mright)^{\alpha \Lconst} \leq s^{\alpha L} < \mleft( \func\eps^{-1}\mright)^{\alpha \Lconst}s^\alpha.
\eeq
If $\delta > 0,$ then we use the right-hand bound in \eqref{eq:saLbounds} to obtain
\beq\label{eq:sdLpos}
s^{\delta L} < \func^{\delta\Lconst}\eps^{-\delta\Lconst}s^{\delta},
\eeq
and if $\delta < 0,$ we use the left-hand bound in \eqref{eq:saLbounds} to obtain
\beq\label{eq:sdLneg}
s^{\delta L} \leq \func^{\delta\Lconst}\eps^{-\delta\Lconst}.
\eeq
We now observe that, for $\delta \neq 0,$
\begin{align}
\sum_{l=0}^L s^{\delta l} &= \frac{s^{\delta\mleft(L+1\mright)} -1}{s^{\delta}-1}\nonumber\\
&= \frac{s^{\delta L} - s^{-\delta}}{1-s^{-\delta}}\nonumber\\
&\leq \frac{s^{\delta L}}{1-s^{-\delta}},\label{eq:ssumbound}
\end{align}
since $s^{-\delta} > 0,$ as $s >0.$ Combining \eqref{eq:ssumbound} with \eqref{eq:sdLpos} and \eqref{eq:sdLneg}, we obtain \eqref{eq:sumboundgen} in the cases $\delta \neq 0.$ The case $\delta=0$ is straightforward.



\paragraph{For sum up to $L-1$}
The proof follows that in \cite{ClGiScTe:11}. We first observe that, since $L$ is given by \eqref{eq:Ldefgen}, it follows that
\beq\label{eq:Lmobounds}
\Lconst\log_s\mleft(\func \eps^{-1}\mright)-1 \leq L-1 < \Lconst\log_s\mleft(\func \eps^{-1}\mright).
\eeq
Rearranging \eqref{eq:Lmobounds}, we obtain the bounds
\beq\label{eq:saLmobounds}
\mleft( \func\eps^{-1}\mright)^{\alpha \Lconst}s^{-\alpha} \leq s^{\alpha (L-1)} < \mleft( \func\eps^{-1}\mright)^{\alpha \Lconst}.
\eeq
If $\delta > 0,$ then we use the right-hand bound in \eqref{eq:saLmobounds} to obtain
\beq\label{eq:sdLmopos}
s^{\delta (L-1)} < \func^{\delta\Lconst}\eps^{-\delta\Lconst}
\eeq
and if $\delta < 0,$ we use the left-hand bound in \eqref{eq:saLmobounds} to obtain
\beq\label{eq:sdLmoneg}
s^{\delta (L-1)} \leq \func^{\delta\Lconst}\eps^{-\delta\Lconst}s^{-\delta}.
\eeq
We now observe that, for $\delta \neq 0,$
\begin{align}
\sum_{l=0}^{L-1} s^{\delta l} &= \frac{s^{\delta\mleft(L\mright)} -1}{s^{\delta}-1}\nonumber\\
&= \frac{s^{\delta (L-1)} - s^{-\delta}}{1-s^{-\delta}}\nonumber\\
&\leq \frac{s^{\delta (L-1)}}{1-s^{-\delta}},\label{eq:ssumboundLmo}
\end{align}
since $s^{-\delta} > 0,$ as $s >0.$ Combining \eqref{eq:ssumboundLmo} with \eqref{eq:sdLmopos} and \eqref{eq:sdLmoneg}, we obtain \eqref{eq:sumboundLmo} in the cases $\delta \neq 0.$ The case $\delta=0$ is straightforward.
\epf


\appendix

\chapter{A Simple Example of an Oscillatory Bayesian Posterior}\label{app:osc}
We construct a simple 1-dimensional example to show that the posteriors arising in Bayesian inverse problems, and the objective functions arising in inverse problems, for the Helmholtz equation are typically oscillatory/multi-modal. Whilst such a statement is commonly made in the literature (see, e.g.,\optodo{Find some references}) and observed in practive (see, e.g.,\optodo{Find more references}), we do not know of an explicit construction of such a posterior, and so we give one for a simple, 1-dimensional Helmholtz problem here.

We consider the 1-dimensional homogeneous Helmholtz equation, posed on on the interval $(-L,0)$, where ultimately $L$ will be an unknown to be determined. We impose a Dirichlet boundary condition given by an incoming plane wave and a sound-soft scatterer at $-L,$ and an outgoing impedance boundary condition at 0. That is, we consider the PDE
\beq\label{eq:app1dhh}
u'' + k^2 u = 0 \tin (-L,0),
\eeq
\beq\label{eq:app1dibc}
u(L) = -\exp(ikL) \tat -L
\eeq
\beq\label{eq:app1ddbc}
u'(0) - iku(0) = 0 \tat 0.
\eeq

We now suppose that $L$ is unknown, that we have measured $u(0)$ (with experimental error), and we wish to know the posterior distribution of $L$. A similar argument to the one presented below would also give analagous results for deterministic inverse problems for $L$. We will show that the posterior distribution of $L$ is multi-modal (that is, it has multiple local maxima/minima) and oscillates on a length scale $2\pi/k.$

To proceed, we first derive an expression for $u(0)$ as a function of $L$. We will then use this expression to formulate the Bayesian posterior for $L$, given knowledge of $u(0).$

Solutions of \eqref{eq:app1dhh} are of the form
\beqs
A\exp(ikx) + B\exp(-ikx),
\eeqs
for some (complex) constants $A$ and $B$. The impedance boundary condition \eqref{eq:app1dibc} immediately implies $B=0$. Then applying the Dirichlet boundary condition \eqref{eq:app1ddbc}\optodo{Finish this later, and talk to someone about it.}

Setup:
\bit
\item Want to determine length of interval (proxy for distance to scatterer) in simple 1-D model
\item Incoming wave, gives DBC \@ $L$ - $\exp(ikx)$ for scattered wave
\item IBC \@ 0\optodo{Make sure you get the sign of the derivative correct!}
\item Helmholtz
\item think of varying $L$/consider BIP for $L$, and show that solution/posterior is an oscialltory/multi-modal function of $L$ on wavelength $2\pi/k$  
\eit

Problem:
\bit
\item For simplicity, Uniform prior on $L$\optodo{Make sure big enough to capture truth}
\item Derive expression for $u(0)$
\item Formulate likelihood (up to proportionality)
\item Show $u(0)$ and likelihood are oscillatory functions of $L$ on length $2\pi/k$
\eit


\chapter{Failure of Fredholm theory for the stochastic variational formulation of Helmholtz problems}\label{sec:federico}
The standard approach to proving existence and uniqueness of a (deterministic) Helmholtz BVP is to  show that the associated sesquilinear form satisfies a G\r{a}rding inequality, and then apply Fredholm theory to deduce that existence and uniqueness are equivalent; see, e.g., \cite[Theorem 4.10]{Mc:00}. This procedure relies on the fact that the inclusion $\HozDDR \hookrightarrow \LtDR$ is compact; see, e.g., \cite[Theorem 3.27]{Mc:00}.

As noted in \cref{sec:otherwork}, the analysis in \cite{FeLiLo:15} of \cref{prob:svsedp} for the Helmholtz Interior Impedance Problem mimics this approach and assums that $\LtOHoD$ is compactly contained  in $\LtOLtD,$ where $D$ is the spatial domain. Here we briefly show $\LtOHoD$ is \emph{not} compactly contained  in $\LtOLtD$ by giving an explicit example of a bounded sequence in  $\LtOHoD$ that has no convergent subsequence in $\LtOLtD.$ Necessary and sufficient conditions for a subset of $L^p\mleft(\mleft[0,T\mright];B\mright),$ for $B$ a Banach space, to be compact, can be found in \cite{Si:86}. In particular, \cite{Si:86} shows that a space $C$ being compactly contained in a space $B$ does not by itself imply $L^2\mleft(\mleft[0,T\mright];C\mright)$ is compactly contained in $L^2\mleft(\mleft[0,T\mright];B\mright).$

\begin{example} \label{ex:federico}
Let $(\Omega,\cF,\PP) = (\mleft[0,1\mright],\cB(\mleft[0,1\mright]),\lambda).$ Let $D$ be a compact subset of $\mathbb{R}^d.$ Since $\LtO$ is separable, it has an orthonormal basis, which we denote by $(f_m)_{m \in \NN}.$ Let $u_m \in  \LtOHoD$ be defined by $ u_m(\omega)(x) \de f_m(\omega), \tfa x \in D,$
i.e., for each value of $\omega,$ $u_m(\omega)$ is a constant function on $D$ and so $\NHoD{u_m(\omega)} = \NLtD{u_m(\omega)}.$ Then
\beqs
\NLtOHoD{u_m}^2 = \int_\Omega \NHoD{u_m(\omega)}^2 \dd\PP(\omega) = \lambda(D)^2\int_\Omega \abs{f_m(\omega)}^2 \dd\PP(\omega)= \NLtO{f_m}^2 \lambda(D)^2,
\eeqs
and so $u_m$ is a bounded sequence in $\LtOHoD.$ However, for $n \neq m,$ we have
\begin{align*}
\NLtOLtD{u_m-u_n}^2 &= \int_\Omega \NLtD{u_m(\omega)-u_n(\omega)}^2 \dd\PP(\omega)\\
&= \lambda(D)^2 \int_\Omega \abs{u_m(\omega) - u_n(\omega)}^2 \dd\PP(\omega) = \lambda(D)^2\NLtO{f_m-f_n}^2 = 2\lambda(D)^2
\end{align*}
%\beqs
%\NLtOLtD{u_m-u_n}^2 = \int_\Omega \NLtD{u_m(\omega)-u_n(\omega)}^2 \dd\PP(\omega) = \lambda(D)^2\NLtO{f_m-f_n}^2 = 2\lambda(D)^2
%\eeqs
if $n \neq m,$ since the $f_m$ form an orthonormal basis for $\LtD.$ Therefore $(u_m)_{m \in \NN}$ is bounded in $\LtOHoD$ but does not have a convergent subsequence in $\LtOLtD,$ and thus the inclusion of $\LtOHoD$ into $\LtOLtD$ cannot be compact.
\end{example}
 	

\chapter{Recap of basic material on measure theory and Bochner spaces}\label{app:mtBs}
Recall that here, and in the rest of this thesis, $\OFP$ is a complete probability space.

\section{Recap of measure theory results}

We first recall some results from measure theory, with our main reference \cite{Bo:07}. Even though \cite{Bo:07} mainly considers maps with image $\RR,$ the results we quote for more general images are straightforward generalisations of the results in \cite{Bo:07}.

\begin{definition}[Measurable map]\label{def:meas}
If $(\M,\FM)$ and $(N,\FN)$ are measurable spaces, we say that $f:\M\rightarrow N$ is measurable (with respect to $(\FM,\FN)$) if $f^{-1}(E) \in \FM$ for all $E \in \FN.$
\end{definition}



\bde[Borel {$\sigma$}-algebra]\label{def:borelsigma}
If $(S,\TS)$ is a topological space, the \defn{Borel $\sigma$-algebra} $\Borel{S}$ on $S$ is the $\sigma$-algebra generated by $\TS.$
\ede


If $V$ is any topological space (including a Hilbert, Banach, metric, or normed vector space) then we will always take the Borel $\sigma$-algebra on $V$ unless stated otherwise.


\ble[Continuous maps are measurable { \cite[Theorem 2.1.2]{Bo:07}}] \label{lem:contmeas}
Any continuous function between two topological spaces is measurable.
\ele





\ble[The composition of a measurable map with a continuous map is measurable {\cite[Text at the top of p. 146]{Bo:07}}]%{\cite[Text at top of p. 146]{Bo:07}}]
Let $(\M,\FM)$ be a measurable space and let $\mleft(S,\TS\mright)$ and $\mleft(T,\TopT\mright)$ be topological spaces. Let $f:M \rightarrow S$ be measurable and let $h : S \rightarrow T$ be continuous. Then $h \circ f$ is measurable.\label{lem:contplusmeas}
\ele

\bde[Product $\sigma$-algebra {\cite[Section IV.11]{Do:94}}]\label{def:prodsigma}
Let $\mleft(\M_1,\FM_1\mright),\ldots,\mleft(\M_m,\FM_m\mright)$ be measurable spaces. The \emph{product $\sigma$-algebra} $\M_1\otimes\cdots\otimes\M_m$ is defined as the $\sigma$-algebra generated by the set of measurable rectangles
$%\label{eq:measrectdef}
\set{R_{1} \times \cdots \times R_m  \st R_{1} \in \FM_1, \ldots, R_m \in \FM_m}.
$
\ede

\ble[Measurability of the Cartesian product of measurable functions]\label{lem:measprod}

\

\noindent Let $\mleft(\M_1,\FM_1\mright),\ldots,\mleft(\M_m,\FM_m\mright)$ be measurable spaces and $h_j:\Omega \rightarrow \M_j,\, j = 1,\ldots,m$ be measurable functions. Then the product map $\Prodf:\Omega \rightarrow \M_1 \times \cdots \times \M_m$ given by
%\beqs
$\Prodf(\omega) \de \mleft(h_1(\omega),\ldots,h_m(\omega)\mright)$
%\eeqs
is measurable with respect to $\mleft(\cF,\FM_1 \otimes \cdots\otimes\FM_m\mright).$
\ele

\bpf[Sketch proof of \cref{lem:measprod}]

Let $\measrectmany{\FM_1,\ldots,\FM_m}$ denote the set of measurable rectangles, as in \cref{def:prodsigma}. Let %Define the set $\cP$ by
%\beqs
$\cP \de \set{C \subseteq \M_1 \times \cdots \times \M_m \st \Prodf^{-1}\mleft(C\mright) \in \cF}.$
%\eeqs
The proof of the lemma consists of the following straightforward steps, whose proofs are omitted:
%\ben
%\item 
(i) Show $\measrectmany{\FM_1,\ldots,\FM_m} \subseteq \cP.$
%\item 
(ii) Show $\cP$ is a $\sigma$-algebra.
%\item 
(iii) Deduce $\FM_1 \otimes \cdots \otimes \FM_m \subseteq \cP$ (since $\FM_1 \otimes \cdots \otimes \FM_m$ is generated by measurable rectangles).
%\item 
(iv) Conclude $\Prodf$ is measurable with respect to $\mleft(\cF,\FM_1\otimes\cdots\otimes\FM_m\mright).$
%\een
\epf

\ble[The product of Borel $\sigma$-algebras is the Borel $\sigma$-algebra of the product {\cite[Lemma 6.2.1 (i)]{Bo:07}}]
Let $H_1,H_2$ be Hausdorff spaces and let $H_2$ have a countable base (e.g.~$H_2$ could be a separable metric space). Then $\Borel{H_1\times H_2} = \Borel{H_1} \otimes \Borel{H_2},$ where $\Borel{H_1\times H_2}$ is the Borel $\sigma$-algebra of the product topology on $H_1\times H_2.$\label{lem:bogachev}
\ele

\section{Recap of results on Bochner spaces}

We now recap the theory of Bochner spaces, using \cite{DiUh:77} as our main reference. In what follows the space $V$ is always a Banach space.

\bde[Simple function]
A function $v:\Omega \rightarrow V$ is \defn{simple} if there exist $v_1,\ldots,v_m \in V$ and $E_1,\ldots,E_m \in \cF$ such that
%\beqs
$v = \sum_{i=1}^m v_i \chi_{E_{i}},$
%\eeqs
where $\chi_{E_{i}}$ is the indicator function on $E_{i}.$
\ede

\bde[Strongly measurable]\label{def:strongmeas}
A function $v:\Omega \rightarrow V$ is \defn{strongly measurable}\footnote{In \cite{DiUh:77} the authors use the term \defn{$\mu$-measurable} instead of \defn{strongly measurable} (where $\mu$ is the measure on the domain of the functions under consideration).} 
if there exists a sequence of simple functions $(\vn)_{n \in \NN}$ such that
%\beqs
$\lim_{n \rightarrow \infty} \NV{\vn - v} = 0$, %\quad
$\PP$-almost everywhere.
%\eeqs
\ede

%Note that \cite{DiUh:77} uses the term \defn{$\mu$-measurable} instead of \defn{strongly measurable} (where $\mu$ is the measure on the domain of the functions under consideration).


\bde[Bochner integrable {\cite[p. 49]{DiUh:77}}]
A strongly measurable function $v:\Omega \rightarrow V$ is called \defn{Bochner integrable} if there exists a sequence of simple functions $(\vn)_{n \in \NN}$ such that
\beqs
\lim_{n \rightarrow \infty} \int_{\Omega} \NV{\vn(\omega) - v(\omega)} \dd\PP(\omega) = 0.
\eeqs
\ede

\bth[Condition for Bochner integrability {\cite[Theorem II.2.2]{DiUh:77}}]\label{thm:bochnercond}

\

\noindent A strongly meas\-ur\-able function $v:\Omega \rightarrow V$ is Bochner integrable if and only if $\int_\Omega \NV{v} \dd\PP < \infty.$
\enth

\bco[Sufficient condition for Bochner integrability]\label{cor:bochnersimple}
Let $p \geq 1.$ If a strongly measurable function $v:\Omega \rightarrow V$ has $\int_\Omega \NV{v}^p \dd\PP < \infty,$ then $v$ is Bochner integrable.
\eco



\bde[Bochner norm]\label{def:bochnernorm}
For a Bochner integrable function $v:\Omega\rightarrow V,$ let
\beqs
\NLpOV{v} \de \mleft(\int_\Omega \NV{v(\omega)}^p \dd\PP(\omega)\mright)^{1/p}, \,1 \leq p < \infty, \,\,\text{and}\,\,\NLiOV{v} \de \esssup_{\omega \in \Omega} \NV{v(\omega)}.
\eeqs
\ede

\bde[Bochner space]\label{def:bochnerspace}
Let $1\leq p \leq \infty.$ Then
\beqs
\LpOV \de \set{v:\Omega\rightarrow V \st v \text{ is Bochner integrable,}\,\NLpOV{v} < \infty}.
\eeqs
\ede

\bde[Complete probability space]
A probability space $\OFP$ is complete if for every $\Eo \in \cF$ with $\PP(\Eo)=0,$ the inclusion $\Et \subseteq \Eo$ implies that $\Et \in \cF.$
\ede

\bde[Separable space]
A topological space is \defn{separable} if it contains a countable, dense subset.
\ede

\bde[$\sigma$-finite]
A probability space $\OFP$ is \defn{$\sigma$-finite} if there exist $E_1,E_2,\ldots \in \cF$ such that $\Omega = \cup_{m=1}^\infty E_m.$
\ede

\bth[Pettis measurability theorem {\cite[Proposition 2.15]{Ry:02}}]\label{thm:pettis}
Let $\OFP$ be a complete $\sigma$-finite measure space. The following are equivalent for a function $v:\Omega \rightarrow V$:
%\ben
%\item 
(i) $v$ is  strongly measurable,
%\item 
(ii) $v$ is measurable and $\PP$-essentially separably valued.
%\een
\enth

\bco[Equivalence of measurable and strongly measurable when the image is separable]
Let $\OFP$ be a $\sigma$-finite measure space. If $V$ is a separable Banach space, then a function $v:\Omega\rightarrow V$ is strongly measurable if, and only if, it is measurable.\label{cor:pettis}
\eco

\ble[The composition of a continuous map and a {$\PP$}-essentially separably valued map]
Let $\mleft(S,\TS\mright)$ and $\mleft(T,\TopT\mright)$ be topological spaces. If $f_1:\Omega \rightarrow S$ and $f_2:S\rightarrow T$ are such that $f_1$ is $\PP$-essentially separably valued and $f_2$ is continuous, then $f_2\circ f_1$ is $\PP$-essentially separably valued.\label{lem:esssep}
\ele

\bpf[Proof of \cref{lem:esssep}]
As $f_1$ is $\PP$-essentially separably valued, there exists $E \in \cF$ such that $\PP(E) = 1$ and $f_1(E) \subseteq G \subseteq S,$ where $G$ is separable. As $\ft$ is continuous, $\ft(G)$ is separable \cite[Theorem 16.4(a)]{Wi:70}. Therefore, since $\mleft(\ft \circ\fo\mright)(E) \subseteq \ft(G),$ it follows that $\ft\circ\fo$ is $\PP$-essentially separably valued.
\epf



\ble[The composition of a continuous map and a strongly measurable map]\label{lem:contplusstrong}

\

\noindent If $\Bo$ and $\Bt$ are Banach spaces and there exist $f_1:\Omega \rightarrow \Bo$ and $f_2:\Bo\rightarrow \Bt$ such that $f_1$ is strongly measurable and $f_2$ is continuous, then $f_2\circ f_1$ is strongly measurable.
\ele

\bpf[Proof of \cref{lem:contplusstrong}]
By \cref{thm:pettis}, $\fo$ is both measurable and $\PP$-essentially separably valued. Therefore we can apply \cref{lem:contplusmeas,lem:esssep} to conclude $\ft \circ \fo$ is both measurable and $\PP$-essentially separably valued. Hence by \cref{thm:pettis} $\ft\circ\fo$ is strongly measurable.
\epf

\ble[Zero in all integrals implies zero almost everywhere {\cite[Corollary II.2.5]{DiUh:77}}]\label{lem:gotoae}

\

\noindent If $\alpha$ is  Bochner integrable and  $\int_E \alpha(\omega) \dd\PP(\omega) = 0 $ for each $E \in \cF$ then $\alpha=0$ $\PP$-almost everywhere.
\ele

\ble[Cartesian product of $\PP$-essentially separably valued maps]\label{lem:prodsep}
Let 

\noindent $\mleft(\cCo,\Top{\cCo}\mright),\ldots,\mleft(\cCm,\Top{\cCm}\mright)$ be topological spaces, and let $s_j:\Omega\rightarrow\cCj,\,j=1,\ldots,m$ be $\PP$-essentially separably valued. Define $\cC \de \cCo \times \cdots \times \cCm$ and equip $\cC$ with the product topology. Then the map $f:\Omega\rightarrow \cC$ given by $s(\omega) \de \mleft(s_1(\omega),\ldots,s_m(\omega)\mright) $ is $\PP$-essentially separably valued.
\ele

The proof of \cref{lem:prodsep} is straightforward and omitted.


\chapter{Measurability of series expansions (used in \cref{sec:generating})}\label{app:meas}
Here we collect together results from measure theory that allow us to conclude in \cref{lem:seriesmeas} that the series expansions for $A$ and $n$ in \cref{sec:generating} are measurable. As mentioned in \cref{sec:generating}, the proof that the sum of measurable functions is measurable is standard, but we have not been able to find this result stated in the literature for this particular setting of mappings into a separable subspace of a general normed vector space.
\ble\label{lem:sepsum}
If $U$ is a separable normed vector space, $m \in \NN,$ and $\phi_j:\Omega\rightarrow U,$ $j=1,\ldots,m$  are measurable functions, then $\phi_1+\cdots+\phi_m : \Omega\rightarrow U$ is measurable.
\ele

\bpf[Sketch proof of \cref{lem:sepsum}]
By induction, it is sufficient to show the result for $m=2.$ We let $\Ballspace{U}{r}{v}$ denote the ball of radius $r>0$ about $v \in U$. To show $\phio+\phit$ is measurable, we let $v \in U, r>0$ and we show $\mleft(\phio+\phit\mright)^{-1}\mleft(\Ballspace{U}{r}{v}\mright) \in \cF.$ Let $\QQU$ denote a countable dense subset of $U,$ which exists as $U$ is separable. Let $\QQFF$ denote a countable dense subset of the field $\FF$, which exists as $\FF = \RR$ or $\CC.$

For $s \in \QQFF, q \in \QQU$ let
\beqs
\setsq = \set{\omega \in \Omega \st \NU{\phio(\omega)-\half v - q} < s} \cap \set{\omega \in \Omega \st \NU{\phit(\omega)-\half v + q} < r-s}.
\eeqs
We claim
\beq\label{eq:sumseteq}
\mleft(\phio+\phit\mright)^{-1}\mleft(\Ballspace{U}{r}{v}\mright) = \bigcup_{s \in \QQFF} \bigcup_{q \in \QQU} \setsq,
\eeq
and the result then follows as the right-hand side is an element of the $\sigma$-algebra $\cF.$ To show \eqref{eq:sumseteq}, let $\omega \in \cup_{s \in \QQFF}\cup_{q \in \QQU} \setsq,$ and let $s \in \QQFF, q \in \QQU$ be such that $\omega \in \setsq.$ Then it follows from the triangle inequality that $\omega \in \mleft(\phio+\phit\mright)^{-1}\mleft(\Ballspace{U}{r}{v}\mright).$
Now let $\omega \in \mleft(\phio+\phit\mright)^{-1}\mleft(\Ballspace{U}{r}{v}\mright),$ define $\dw \de r - \NU{\phio(\omega)+\phit(\omega)-v} > 0,$ fix $s \in \QQFF \cap (0,\dw/2),$ and choose $q \in \QQU$ such that $\NU{\phio(\omega) - v/2 -q} < s.$ Then again it follows from the triangle inequality that $\omega \in \setsq,$ and thus \eqref{eq:sumseteq} holds, as required.
\epf

% Corollary: finite sums lying in a separable subspace of a normed vector space are measurable
\bco\label{cor:sepsubsum}
If $V$ is a normed vector space, $U \subseteq V$ is a separable subspace, and $\phi_j:\Omega\rightarrow U,$  $j=1,\ldots,m$  are measurable functions, then $\phi_1+\cdots+\phi_m : \Omega\rightarrow U$ is measurable.
\eco

% Lemma: scalar measurable function multiplied by a vector is measurable (as it's continuous)
\ble\label{lem:scalarmultmeas}
Let $V$ be a normed vector space. If $v \in V$ and $Y:\Omega\rightarrow \FF$ is a measurable function, then $Yv:\Omega\rightarrow V$ is a measurable function.
\ele

\bpf[Proof of \cref{lem:scalarmultmeas}]
The map $\scalarmult:\FF\rightarrow V$ given by $\scalarmult(x) = xv$ is continuous. As $Yv = \scalarmult \circ Y,$ it follows from \cref{lem:contplusmeas} that $Yv$ is measurable.
\epf

% Lemma: Inclusion map is Borel
\ble\label{lem:incborel}
If $V$ is a normed vector space and $U \subseteq V,$ then the inclusion map $\inc:U\rightarrow V$ is measurable.
\ele

\bpf[Proof of \cref{lem:incborel}]
As $\inc$ is continuous, it immediately follows that it is measurable.
\epf

% Corollary: measurable into subspace means measurable into space
\bco\label{cor:meassubmeansmeas}
If $V$ is a normed vector space, $U\subseteq V$ and $\phi:\Omega\rightarrow U$ is measurable, then $\phi:\Omega\rightarrow V$ is measurable.
\eco

\bpf[Proof of \cref{cor:meassubmeansmeas}]
This is immediate from \cref{lem:incborel} and \cref{lem:contplusmeas}.
\epf

% Lemma: the $\FF$-span  of a finite set of functions is a separable subspace if $\FF$ is separable
\ble\label{lem:spansep}
If $V$ is a normed vector space, $m \in \NN,$ and $\phi_1,\ldots,\phi_m \in V$ for $j = 1,\ldots,m$ then $\spanset{\phi_1,\ldots,\phi_m}$ is a separable subspace of $V.$
\ele

\bpf[Sketch Proof of \cref{lem:spansep}]
As $\FF = \RR$ or $\CC,$ it has a separable subset $\QQFF.$ Since a finite product of countable sets is countable, the set
\beqs
\set{\Ballspace{V}{1/n}{q_1\phi_1 + \cdots + q_m \phi_m} \st n \in \NN, q_1,\ldots,q_m \in \QQFF}
\eeqs
is a countable base for the topology on $\spanset{\phi_1,\ldots,\phi_m}$ induced by the norm $\NV{\cdot}.$ 
\epf

\ble\label{lem:summeas}
If $V$ is a normed vector space, $m \in \NN,$ and for $j = 1,\ldots,m$, $\phi_j\in V$ and $Y_j : \Omega \rightarrow \FF$ are measurable, then the function $\phi:\Omega\rightarrow V$ given by
\beqs
\phi(\omega) = \phi_0 + \sum_{j=1}^m Y_j(\omega)\phi_j
\eeqs
is measurable.
\ele

\bpf[Proof of \cref{lem:summeas}]
The subspace $U=\spanset{\phi_0,\phi_1,\ldots,\phi_m}$ is separable by \cref{lem:spansep}, and it is clear that the image of $\phi$ lies in $U.$ By \cref{lem:scalarmultmeas} and \cref{cor:sepsubsum}, $\phi:\Omega\rightarrow U$ is measurable, and therefore $\phi:\Omega\rightarrow V$ is measurable by \cref{cor:meassubmeansmeas}.
\epf

We now prove that almost-surely convergent sequences of measurable functions are measurable, and we then apply this result to the partial sums in the definitions of $A$ and $n$ in \eqref{eq:nseries}. %In order to prove this, from this point onwards we make the assumption that the space $V$ is complete. This assumption is satisfied by the spaces involved in the definitions of $A$ and $n$ ($\WoiDRRRdtd$ and $\WoiDRRR$ respectively).

We will use the following theorem to establish that the almost-sure limit of a sequence of measurable functions is measurable.

\bth[{\cite[Theorem 4.2.2]{Du:02}}]\label{thm:dudley}
Let $(\metsp,\metric)$ be a metric space. Suppose the functions $\zeta_j:\Omega \rightarrow \metsp$ are measurable, for all $j \in \NN.$ If the limit
\beqs
\zeta(\omega) = \lim_{j \rightarrow \infty} \zeta_j(\omega)
\eeqs
exists for every $\omega \in \Omega,$ then the function $\zeta:\Omega \rightarrow \metsp$ is measurable.
\enth

\bco\label{lem:paelimitmeas}
Let $(\metsp,\metric)$ be a metric space. Suppose the functions $\zeta_m:\Omega \rightarrow \metsp$ are measurable, for all $m \in \NN.$ If the limit
\beq\label{eq:zetalimit}
\lim_{m \rightarrow \infty} \zeta_m(\omega)
\eeq
exists almost surely, then there exists a measurable function $\zeta:\Omega \rightarrow \metsp$ such that
\beqs
\zeta(\omega) = \lim_{m \rightarrow \infty} \zeta_m(\omega)
\eeqs
whenever the limit exists.
\eco

\bpf[Proof of \cref{lem:paelimitmeas}]
Following \cite{El:11}, we define $\Omegatilde = \set{\omega \in \Omega \st \text{ \eqref{eq:zetalimit} exists}}.$ Then, for $m \in \NN$ define $\zetatilde_m:\Omega\rightarrow \metsp$ by
\beqs
\zetatilde_m(\omega) =
\begin{cases}
\zeta_m(\omega) & \tif \omega \in \Omegatilde\\
0 & \tif \omega \not\in\Omegatilde
\end{cases}
\eeqs
Observe that, by construction, the limit $\zetatilde(\omega) = \lim_{m\rightarrow\infty} \zetatilde_m(\omega)$ exists \emph{for all} $\omega \in \Omega$ and the functions $\zetatilde_m$ are measurable. Therefore, by \cref{thm:dudley}, $\zetatilde$ is measurable.
\epf

\ble\label{lem:paeseriesmeas}
Let $V$ be a normed vector space. If there exist $\phi_j\in V,$ $j=0,1,\ldots$ and measurable functions $\Yj:\Omega \rightarrow \FF,$ $j \in \NN$ such that the series
\beqs
\phi_0 + \sum_{j=1}^\infty \Yj(\omega)\phi_j
\eeqs
exists in $V$ almost surely, then there exists a measurable function $\phi:\Omega\rightarrow V$ such that
\beqs
\phi(\omega) = \phi_0 + \sum_{j=1}^\infty \Yj(\omega)\phi_j
\eeqs
almost surely.
\ele

\bpf[Proof of \cref{lem:paeseriesmeas}]
By \cref{lem:summeas}, the partial sums $\phi_0 + \sum_{j=1}^m \Yj(\omega)\phi_j$, for $m \in \NN$ are measurable, and by assumption their limit as $m \rightarrow \infty$ exists almost surely. Therefore, applying \cref{lem:paelimitmeas} to the partial sums, we obtain the result.
\epf

\ble\label{lem:seriesexistsas}
The series expansions for both $A$ and $n$ defined by \eqref{eq:nseries} exist in $\WoiDRRRdtd$ and $\WoiDRRR$ almost surely, respectively.
\ele

\bpf[Proof of \cref{lem:seriesexistsas}]
The spaces $\WoiDRRRdtd$ and $\WoiDRRR$ are Banach spaces, by definition of their norms (see \cref{eq:ineditsnorm1,eq:ineditsnorm2}). Therefore it suffices to show that the partial sums of the series expansions for $A$ and $n$ in \eqref{eq:nseries} are Cauchy sequences. As the proofs for $A$ and $n$ are completely analogous, we only give the proof for $A$ here.

First observe that since each of the random variables $\Yj$ in \eqref{eq:nseries} is uniformly distributed on $[-1/2,1/2],$ it follows that for all $j \in \NN$, $\esssup_{\omega \in \Omega} \abs{\Yj(\omega)} = \frac12.$
Therefore, we can conclude that the bound $\esssup_{\omega \in \Omega} \sup_{j \in \NN} \abs{\Yj(\omega)} \leq \half$ holds.
(For if not, then, there would exist $\Omegahat \subseteq \Omega$ with $\PP(\Omegahat) > 0$ such that for all $\omega \in \Omegahat$, $\sup_{j \in \NN} \abs{\Yj(\omega)} > \half.$
Then there would exist $\jhat \in \NN$ such that $\abs{\Yjhat(\omega)} > 1/2$ for all $\omega \in \Omegahat,$ which would give the contradiction $\esssup_{\omega \in \Omega} |\Yjhat(\omega)| > \half$.)

It now suffices to show that for $\PP$-almost every $\omega \in \Omega,$ the partial sums of the series expansion in \eqref{eq:nseries} form a Cauchy sequence. Recall that for $\PP$-almost every $\omega \in \Omega$
\beqs
\sup_{j \in \NN} \abs{\Yj(\omega)} \leq \half.
\eeqs
For such an $\omega$, and $m \in \NN,$ define the $m$th partial sum
\beqs
\Am(\omega) = \Az + \sum_{j=1}^m \Yj(\omega)\Psij.
\eeqs
It is straightforward to show that $\mleft(\Am(\omega)\mright)_{m \in \NN}$ is a Cauchy sequence in $\WoiDRRRdtd$, using the assumption \eqref{eq:Apsimeas}; therefore, the series expansion for $A(\omega)$ in \eqref{eq:nseries} exists almost surely.
\epf

\ble\label{lem:seriesmeas}
The functions $A$ and $n$ defined by \eqref{eq:nseries} are measurable.
\ele

\bpf[Proof of \cref{lem:seriesmeas}]
The result immediately follows from \cref{lem:seriesexistsas,lem:paeseriesmeas}.
\epf


\chapter{Error estimators for complex random variables}\label{app:complexerror}
In this \lcnamecref{app:complexerror}, we recall the definition of some elementary properties of \emph{complex-valued} random variables, properties that are slightly different to their analogues for real-valued random variables. We then prove that the standard unbiased estimator of the variance (with Bessel's correction) of a complex-valued random variable  is indeed an unbiased estimator of the variance.

\bde[Complex-valued random variable {\cite[Equation (4.2)]{Pa:18}}]
If $\OFP$ is a probability space, then a \defn{complex-valued} random variable is a map $Y:\Omega\rightarrow \CC,$ such that $\Re Y$ and $\Im Y$ are real-valued random variables.
\ede

\bde[Mean and variance of a complex-valued random variable {\cite[Equations (5.8) and (5.27)]{Pa:18}}]
Let $Y$ be a complex-valued random variable. The \defn{expectation of $Y$} is
\beqs
\EXP{Y} \de \EXP{\Re{Y}} + i \EXP{\Im{Y}},
\eeqs
if it exists. The \defn{variance of $Y$} is
\beqs
\VAR{Y} \de \EXP{\abs{Y}^2} - \abs{\EXP{Y}}^2
\eeqs
if it exists.
\ede

\bde[$\sigma$-algebra generated by a random variable]
Let $Y$ be a complex-valued random variable. The \defn{$\sigma$-algebra generated by $Y$} is
\beqs
\sigalggen{Y} \de \set{Y^{-1}\mleft(E\mright) \st E \in \sigma\mleft(\RR\mright)},
\eeqs
where $Y^{-1}$ denotes the pullback.
\ede

\bde[Independent $\sigma$-algebras]
Two $\sigma$-algebras $\cFo$ and $\cFt$ on $\Omega$ are \defn{independent} if all their sets are independent, i.e.
\beqs
\PP\mleft(\Eo\mright) \cap \PP\mleft(\Et\mright) = \PP\mleft(\Eo\mright)\PP\mleft(\Et\mright)
\eeqs
for all $\Eo \in \cFo$ and $\Et \in \cFt.$
\ede

\bde[Independent random variables]
Two complex-valued random variables $\Yo$ and $\Yt$ are \defn{independent} if their respective generated $\sigma$-algebras are independent.
\ede

\ble[Independent implies uncorrelated]\label{lem:complexindep}
If $\Yo$ and $\Yt$ are independent complex-valued random variables, then
\beqs
\EXP{\Yo\Ytbar} = \EXP{\Yo}\EXP{\Ytbar}.
\eeqs
\ele

The proof of \cref{lem:complexindep} is identical to the real case.

\bde[Monte-Carlo estimator for $\EXP{Y}.$]
Let $Y$ be a complex-valued random variable, and $\Yo,\ldots,\YN$ be independent and identically distributed to $Y$. The \defn{Monte-Carlo estimator of $\EXP{Y}$} is
\beqs
\Yhat \de\frac1N \sum_{l=1}^N \Yl.
\eeqs
\ede

\bde[Unbiased estimator of the variance of the Monte-Carlo estimator]
Let $Y$ be a complex-valued random variable, and $\Yhat$ the Monte-Carlo estimator of $\EXP{Y}$. The estimator $\unbiased{\Yhat}{N}$ of $\VAR{\Yhat}$ is
\beq\label{eq:unbiased}
\unbiased{\Yhat}{N} \de \frac1{N(N-1)} \sum_{j=1}^N \abs{\Yj - \Yhat}^2.
\eeq
\ede

Observe that \cref{eq:unbiased} defines an estimator for the variance of \emph{the Monte-Carlo estimator $\Yhat$}; this is in contrast to more standard statistical settings, where one constructs an estimator of the variance of $Y$. The factor $1/(N-1)$ in \cref{eq:unbiased} is known as \defn{Bessel's correction}, and ensures the estimator is unbiased, as we now prove.

\ble[The unbiased estimator is unbiased]\label{lem:unbiased}
Let $Y$ be a complex-valued random variable and  $\Yhat$ the Monte-Carlo estimator of $\EXP{Y}$. Then $\unbiased{\Yhat}{N}$ is unbiased, i.e.,
\beqs
\EXP{\unbiased{\Yhat}{N}} = \VAR{\Yhat}.
\eeqs
\ele

The proof of \cref{lem:unbiased} is nearly identical to the proof for an unbiased estimator for $\VAR{Y}$ in the real-valued case. Nevertheless, we write the proof out in full, as we have not been able to find this exact result anywhere in the literature.

\bpf[Proof of \cref{lem:unbiased}]
%% We introduce the notation
%% \beqs
%% \mu \de \abs{\EXP{Y}}\quad\tand\quad\sigma \de \VAR{Y}
%% \eeqs
%% and we observe that
%% \beq\label{eq:varrelation}
%% \EXP{\abs{Y}^2} = \sigma^2 + \mu^2.
%% \eeq
Firstly, note that
\beqs
\VAR{\Yhat} = \VAR{\frac1N \sum_{l=1}^N \Yl} = \frac1{N^2} \VAR{\sum_{l=1}^N \Yl} = \frac1N \VAR{Y}.
\eeqs
Therefore, it is sufficient to show that $\unbiased{\Yhat}{N}$ is an unbiased estimator for $\VAR{Y}/N,$ or equivalently, $N\unbiased{\Yhat}{N}$ is an unbiased estimator for $\VAR{Y}$ (i.e. $\EXP{N\unbiased{\Yhat}{N}} = \VAR{Y}$). We show the latter by direct computation. Observe that
\beqs
\EXP{N\unbiased{\Yhat}{N}} = \frac1{N-1} \EXP{\sum_{j=1}^N \abs{\Yj - \Yhat}^2}= \frac1{N-1} \sum_{j=1}^N\EXP{ \abs{\Yj - \Yhat}^2},
\eeqs
therefore, it is sufficient for us to show
\beq\label{eq:comperror1}
\EXP{\abs{\Yj-\Yhat}^2} = \frac{N-1}N\VAR{Y},
\eeq
as the $\Yj$s are all independently and identically distributed. We show \cref{eq:comperror1} by direct computation.

\begin{align}
\EXP{\abs{\Yj-\Yhat}^2} &= \EXP{\abs{\Yj}^2 - \Yj \Yhatbar - \Yjbar \Yhat + \abs{\Yhat}^2}\nonumber\\
&= \EXP{\abs{\Yj}^2} - \frac1N\sum_{l=1}^N \mleft(\EXP{\Yj\Ylbar + \Yjbar \Yl}\mright) + \EXP{\abs{\Yhat}^2}\nonumber\\
&= \EXP{\abs{\Yj}^2} - \frac2N \EXP{\abs{\Yj}^2} - \frac1N \sum_{l \neq j} \mleft(\EXP{\Yj}\EXP{\Ylbar} + \EXP{\Yjbar}\EXP{\Yl}\mright) + \EXP{\abs{\Yhat}^2}\nonumber\\
&= \frac{N-2}N \EXP{\abs{Y}^2} - \frac{2(N-1)}N \abs{\EXP{Y}}^2 + \EXP{\abs{\Yhat}^2},\label{eq:comperror2}
  \end{align}
since the $\Yl$s have has the same distribution as $Y$. We now turn our attention to simplifying $\EXP{\abs{\Yhat}^2}.$ We have
\begin{align}
\EXP{\abs{\Yhat}^2} &= \EXP{\mleft(\frac1N \sum_{l=1}^N \Yl\mright)\overline{\mleft(\frac1N \sum_{m=1}^N \Ym\mright)}}\nonumber\\
&= \frac1{N^2}\EXP{ \sum_{l=1}^N \abs{\Yl}^2 + \sum_{l=1}^N\sum_{l \neq m} \Yl\Ymbar}\nonumber\\
&= \frac1N \EXP{\abs{Y}^2} + \frac{N(N-1)}{N^2}\abs{\EXP{Y}}^2,\label{eq:comperror3}
\end{align}
since the  $\Yl$ are i.i.d. Therefore combining \cref{eq:comperror2,eq:comperror3}, we obtain

\begin{align*}
\EXP{\abs{\Yj-\Yhat}^2} &= \frac{N-2}N \EXP{\abs{Y}^2} - \frac{2(N-1)}N \abs{\EXP{Y}}^2 + \frac1N \EXP{\abs{Y}^2} + \frac{N-1}{N}\abs{\EXP{Y}}^2\\
&= \frac{N-1}N \EXP{\abs{Y}^2} - \frac{N-1}N \abs{\EXP{Y}}^2\\
&= \frac{N-1}N \mleft(\VAR{Y}+\abs{\EXP{Y}}^2\mright) - \frac{N-1}N \abs{\EXP{Y}}^2,
\end{align*}
which gives us \cref{eq:comperror1}, as required.
\epf



\chapter{Numerical investigation of QMC convergence for the Helmholtz equation}\label{app:hhqmcconv}
In this \lcnamecref{app:hhqmcconv} we give plots showing the dependence of the QMC error $\QMCerror{\NQMC}{\Nshifts}$ on $\NQMC$ for \emph{fixed} values of $k$ and increasing $\NQMC$. Aside from the fact that we vary $\NQMC$, our computational setup is as in \cref{sec:nbpcqmcnumerics}.

\begin{figure}[h]
    \centering
    %% Creator: Matplotlib, PGF backend
%%
%% To include the figure in your LaTeX document, write
%%   \input{<filename>.pgf}
%%
%% Make sure the required packages are loaded in your preamble
%%   \usepackage{pgf}
%%
%% Figures using additional raster images can only be included by \input if
%% they are in the same directory as the main LaTeX file. For loading figures
%% from other directories you can use the `import` package
%%   \usepackage{import}
%% and then include the figures with
%%   \import{<path to file>}{<filename>.pgf}
%%
%% Matplotlib used the following preamble
%%   \usepackage{fontspec}
%%   \setmainfont{DejaVuSerif.ttf}[Path=/home/owen/progs/firedrake-complex/firedrake/lib/python3.5/site-packages/matplotlib/mpl-data/fonts/ttf/]
%%   \setsansfont{DejaVuSans.ttf}[Path=/home/owen/progs/firedrake-complex/firedrake/lib/python3.5/site-packages/matplotlib/mpl-data/fonts/ttf/]
%%   \setmonofont{DejaVuSansMono.ttf}[Path=/home/owen/progs/firedrake-complex/firedrake/lib/python3.5/site-packages/matplotlib/mpl-data/fonts/ttf/]
%%
\begingroup%
\makeatletter%
\begin{pgfpicture}%
\pgfpathrectangle{\pgfpointorigin}{\pgfqpoint{5.000000in}{4.000000in}}%
\pgfusepath{use as bounding box, clip}%
\begin{pgfscope}%
\pgfsetbuttcap%
\pgfsetmiterjoin%
\definecolor{currentfill}{rgb}{1.000000,1.000000,1.000000}%
\pgfsetfillcolor{currentfill}%
\pgfsetlinewidth{0.000000pt}%
\definecolor{currentstroke}{rgb}{1.000000,1.000000,1.000000}%
\pgfsetstrokecolor{currentstroke}%
\pgfsetdash{}{0pt}%
\pgfpathmoveto{\pgfqpoint{0.000000in}{0.000000in}}%
\pgfpathlineto{\pgfqpoint{5.000000in}{0.000000in}}%
\pgfpathlineto{\pgfqpoint{5.000000in}{4.000000in}}%
\pgfpathlineto{\pgfqpoint{0.000000in}{4.000000in}}%
\pgfpathclose%
\pgfusepath{fill}%
\end{pgfscope}%
\begin{pgfscope}%
\pgfsetbuttcap%
\pgfsetmiterjoin%
\definecolor{currentfill}{rgb}{1.000000,1.000000,1.000000}%
\pgfsetfillcolor{currentfill}%
\pgfsetlinewidth{0.000000pt}%
\definecolor{currentstroke}{rgb}{0.000000,0.000000,0.000000}%
\pgfsetstrokecolor{currentstroke}%
\pgfsetstrokeopacity{0.000000}%
\pgfsetdash{}{0pt}%
\pgfpathmoveto{\pgfqpoint{0.625000in}{0.440000in}}%
\pgfpathlineto{\pgfqpoint{4.500000in}{0.440000in}}%
\pgfpathlineto{\pgfqpoint{4.500000in}{3.520000in}}%
\pgfpathlineto{\pgfqpoint{0.625000in}{3.520000in}}%
\pgfpathclose%
\pgfusepath{fill}%
\end{pgfscope}%
\begin{pgfscope}%
\pgfsetbuttcap%
\pgfsetroundjoin%
\definecolor{currentfill}{rgb}{0.000000,0.000000,0.000000}%
\pgfsetfillcolor{currentfill}%
\pgfsetlinewidth{0.803000pt}%
\definecolor{currentstroke}{rgb}{0.000000,0.000000,0.000000}%
\pgfsetstrokecolor{currentstroke}%
\pgfsetdash{}{0pt}%
\pgfsys@defobject{currentmarker}{\pgfqpoint{0.000000in}{-0.048611in}}{\pgfqpoint{0.000000in}{0.000000in}}{%
\pgfpathmoveto{\pgfqpoint{0.000000in}{0.000000in}}%
\pgfpathlineto{\pgfqpoint{0.000000in}{-0.048611in}}%
\pgfusepath{stroke,fill}%
}%
\begin{pgfscope}%
\pgfsys@transformshift{0.801136in}{0.440000in}%
\pgfsys@useobject{currentmarker}{}%
\end{pgfscope}%
\end{pgfscope}%
\begin{pgfscope}%
\definecolor{textcolor}{rgb}{0.000000,0.000000,0.000000}%
\pgfsetstrokecolor{textcolor}%
\pgfsetfillcolor{textcolor}%
\pgftext[x=0.801136in,y=0.342778in,,top]{\color{textcolor}\sffamily\fontsize{10.000000}{12.000000}\selectfont \(\displaystyle 10^{0}\)}%
\end{pgfscope}%
\begin{pgfscope}%
\pgfsetbuttcap%
\pgfsetroundjoin%
\definecolor{currentfill}{rgb}{0.000000,0.000000,0.000000}%
\pgfsetfillcolor{currentfill}%
\pgfsetlinewidth{0.803000pt}%
\definecolor{currentstroke}{rgb}{0.000000,0.000000,0.000000}%
\pgfsetstrokecolor{currentstroke}%
\pgfsetdash{}{0pt}%
\pgfsys@defobject{currentmarker}{\pgfqpoint{0.000000in}{-0.048611in}}{\pgfqpoint{0.000000in}{0.000000in}}{%
\pgfpathmoveto{\pgfqpoint{0.000000in}{0.000000in}}%
\pgfpathlineto{\pgfqpoint{0.000000in}{-0.048611in}}%
\pgfusepath{stroke,fill}%
}%
\begin{pgfscope}%
\pgfsys@transformshift{1.864977in}{0.440000in}%
\pgfsys@useobject{currentmarker}{}%
\end{pgfscope}%
\end{pgfscope}%
\begin{pgfscope}%
\definecolor{textcolor}{rgb}{0.000000,0.000000,0.000000}%
\pgfsetstrokecolor{textcolor}%
\pgfsetfillcolor{textcolor}%
\pgftext[x=1.864977in,y=0.342778in,,top]{\color{textcolor}\sffamily\fontsize{10.000000}{12.000000}\selectfont \(\displaystyle 10^{1}\)}%
\end{pgfscope}%
\begin{pgfscope}%
\pgfsetbuttcap%
\pgfsetroundjoin%
\definecolor{currentfill}{rgb}{0.000000,0.000000,0.000000}%
\pgfsetfillcolor{currentfill}%
\pgfsetlinewidth{0.803000pt}%
\definecolor{currentstroke}{rgb}{0.000000,0.000000,0.000000}%
\pgfsetstrokecolor{currentstroke}%
\pgfsetdash{}{0pt}%
\pgfsys@defobject{currentmarker}{\pgfqpoint{0.000000in}{-0.048611in}}{\pgfqpoint{0.000000in}{0.000000in}}{%
\pgfpathmoveto{\pgfqpoint{0.000000in}{0.000000in}}%
\pgfpathlineto{\pgfqpoint{0.000000in}{-0.048611in}}%
\pgfusepath{stroke,fill}%
}%
\begin{pgfscope}%
\pgfsys@transformshift{2.928818in}{0.440000in}%
\pgfsys@useobject{currentmarker}{}%
\end{pgfscope}%
\end{pgfscope}%
\begin{pgfscope}%
\definecolor{textcolor}{rgb}{0.000000,0.000000,0.000000}%
\pgfsetstrokecolor{textcolor}%
\pgfsetfillcolor{textcolor}%
\pgftext[x=2.928818in,y=0.342778in,,top]{\color{textcolor}\sffamily\fontsize{10.000000}{12.000000}\selectfont \(\displaystyle 10^{2}\)}%
\end{pgfscope}%
\begin{pgfscope}%
\pgfsetbuttcap%
\pgfsetroundjoin%
\definecolor{currentfill}{rgb}{0.000000,0.000000,0.000000}%
\pgfsetfillcolor{currentfill}%
\pgfsetlinewidth{0.803000pt}%
\definecolor{currentstroke}{rgb}{0.000000,0.000000,0.000000}%
\pgfsetstrokecolor{currentstroke}%
\pgfsetdash{}{0pt}%
\pgfsys@defobject{currentmarker}{\pgfqpoint{0.000000in}{-0.048611in}}{\pgfqpoint{0.000000in}{0.000000in}}{%
\pgfpathmoveto{\pgfqpoint{0.000000in}{0.000000in}}%
\pgfpathlineto{\pgfqpoint{0.000000in}{-0.048611in}}%
\pgfusepath{stroke,fill}%
}%
\begin{pgfscope}%
\pgfsys@transformshift{3.992658in}{0.440000in}%
\pgfsys@useobject{currentmarker}{}%
\end{pgfscope}%
\end{pgfscope}%
\begin{pgfscope}%
\definecolor{textcolor}{rgb}{0.000000,0.000000,0.000000}%
\pgfsetstrokecolor{textcolor}%
\pgfsetfillcolor{textcolor}%
\pgftext[x=3.992658in,y=0.342778in,,top]{\color{textcolor}\sffamily\fontsize{10.000000}{12.000000}\selectfont \(\displaystyle 10^{3}\)}%
\end{pgfscope}%
\begin{pgfscope}%
\pgfsetbuttcap%
\pgfsetroundjoin%
\definecolor{currentfill}{rgb}{0.000000,0.000000,0.000000}%
\pgfsetfillcolor{currentfill}%
\pgfsetlinewidth{0.602250pt}%
\definecolor{currentstroke}{rgb}{0.000000,0.000000,0.000000}%
\pgfsetstrokecolor{currentstroke}%
\pgfsetdash{}{0pt}%
\pgfsys@defobject{currentmarker}{\pgfqpoint{0.000000in}{-0.027778in}}{\pgfqpoint{0.000000in}{0.000000in}}{%
\pgfpathmoveto{\pgfqpoint{0.000000in}{0.000000in}}%
\pgfpathlineto{\pgfqpoint{0.000000in}{-0.027778in}}%
\pgfusepath{stroke,fill}%
}%
\begin{pgfscope}%
\pgfsys@transformshift{0.636345in}{0.440000in}%
\pgfsys@useobject{currentmarker}{}%
\end{pgfscope}%
\end{pgfscope}%
\begin{pgfscope}%
\pgfsetbuttcap%
\pgfsetroundjoin%
\definecolor{currentfill}{rgb}{0.000000,0.000000,0.000000}%
\pgfsetfillcolor{currentfill}%
\pgfsetlinewidth{0.602250pt}%
\definecolor{currentstroke}{rgb}{0.000000,0.000000,0.000000}%
\pgfsetstrokecolor{currentstroke}%
\pgfsetdash{}{0pt}%
\pgfsys@defobject{currentmarker}{\pgfqpoint{0.000000in}{-0.027778in}}{\pgfqpoint{0.000000in}{0.000000in}}{%
\pgfpathmoveto{\pgfqpoint{0.000000in}{0.000000in}}%
\pgfpathlineto{\pgfqpoint{0.000000in}{-0.027778in}}%
\pgfusepath{stroke,fill}%
}%
\begin{pgfscope}%
\pgfsys@transformshift{0.698040in}{0.440000in}%
\pgfsys@useobject{currentmarker}{}%
\end{pgfscope}%
\end{pgfscope}%
\begin{pgfscope}%
\pgfsetbuttcap%
\pgfsetroundjoin%
\definecolor{currentfill}{rgb}{0.000000,0.000000,0.000000}%
\pgfsetfillcolor{currentfill}%
\pgfsetlinewidth{0.602250pt}%
\definecolor{currentstroke}{rgb}{0.000000,0.000000,0.000000}%
\pgfsetstrokecolor{currentstroke}%
\pgfsetdash{}{0pt}%
\pgfsys@defobject{currentmarker}{\pgfqpoint{0.000000in}{-0.027778in}}{\pgfqpoint{0.000000in}{0.000000in}}{%
\pgfpathmoveto{\pgfqpoint{0.000000in}{0.000000in}}%
\pgfpathlineto{\pgfqpoint{0.000000in}{-0.027778in}}%
\pgfusepath{stroke,fill}%
}%
\begin{pgfscope}%
\pgfsys@transformshift{0.752458in}{0.440000in}%
\pgfsys@useobject{currentmarker}{}%
\end{pgfscope}%
\end{pgfscope}%
\begin{pgfscope}%
\pgfsetbuttcap%
\pgfsetroundjoin%
\definecolor{currentfill}{rgb}{0.000000,0.000000,0.000000}%
\pgfsetfillcolor{currentfill}%
\pgfsetlinewidth{0.602250pt}%
\definecolor{currentstroke}{rgb}{0.000000,0.000000,0.000000}%
\pgfsetstrokecolor{currentstroke}%
\pgfsetdash{}{0pt}%
\pgfsys@defobject{currentmarker}{\pgfqpoint{0.000000in}{-0.027778in}}{\pgfqpoint{0.000000in}{0.000000in}}{%
\pgfpathmoveto{\pgfqpoint{0.000000in}{0.000000in}}%
\pgfpathlineto{\pgfqpoint{0.000000in}{-0.027778in}}%
\pgfusepath{stroke,fill}%
}%
\begin{pgfscope}%
\pgfsys@transformshift{1.121384in}{0.440000in}%
\pgfsys@useobject{currentmarker}{}%
\end{pgfscope}%
\end{pgfscope}%
\begin{pgfscope}%
\pgfsetbuttcap%
\pgfsetroundjoin%
\definecolor{currentfill}{rgb}{0.000000,0.000000,0.000000}%
\pgfsetfillcolor{currentfill}%
\pgfsetlinewidth{0.602250pt}%
\definecolor{currentstroke}{rgb}{0.000000,0.000000,0.000000}%
\pgfsetstrokecolor{currentstroke}%
\pgfsetdash{}{0pt}%
\pgfsys@defobject{currentmarker}{\pgfqpoint{0.000000in}{-0.027778in}}{\pgfqpoint{0.000000in}{0.000000in}}{%
\pgfpathmoveto{\pgfqpoint{0.000000in}{0.000000in}}%
\pgfpathlineto{\pgfqpoint{0.000000in}{-0.027778in}}%
\pgfusepath{stroke,fill}%
}%
\begin{pgfscope}%
\pgfsys@transformshift{1.308717in}{0.440000in}%
\pgfsys@useobject{currentmarker}{}%
\end{pgfscope}%
\end{pgfscope}%
\begin{pgfscope}%
\pgfsetbuttcap%
\pgfsetroundjoin%
\definecolor{currentfill}{rgb}{0.000000,0.000000,0.000000}%
\pgfsetfillcolor{currentfill}%
\pgfsetlinewidth{0.602250pt}%
\definecolor{currentstroke}{rgb}{0.000000,0.000000,0.000000}%
\pgfsetstrokecolor{currentstroke}%
\pgfsetdash{}{0pt}%
\pgfsys@defobject{currentmarker}{\pgfqpoint{0.000000in}{-0.027778in}}{\pgfqpoint{0.000000in}{0.000000in}}{%
\pgfpathmoveto{\pgfqpoint{0.000000in}{0.000000in}}%
\pgfpathlineto{\pgfqpoint{0.000000in}{-0.027778in}}%
\pgfusepath{stroke,fill}%
}%
\begin{pgfscope}%
\pgfsys@transformshift{1.441632in}{0.440000in}%
\pgfsys@useobject{currentmarker}{}%
\end{pgfscope}%
\end{pgfscope}%
\begin{pgfscope}%
\pgfsetbuttcap%
\pgfsetroundjoin%
\definecolor{currentfill}{rgb}{0.000000,0.000000,0.000000}%
\pgfsetfillcolor{currentfill}%
\pgfsetlinewidth{0.602250pt}%
\definecolor{currentstroke}{rgb}{0.000000,0.000000,0.000000}%
\pgfsetstrokecolor{currentstroke}%
\pgfsetdash{}{0pt}%
\pgfsys@defobject{currentmarker}{\pgfqpoint{0.000000in}{-0.027778in}}{\pgfqpoint{0.000000in}{0.000000in}}{%
\pgfpathmoveto{\pgfqpoint{0.000000in}{0.000000in}}%
\pgfpathlineto{\pgfqpoint{0.000000in}{-0.027778in}}%
\pgfusepath{stroke,fill}%
}%
\begin{pgfscope}%
\pgfsys@transformshift{1.544729in}{0.440000in}%
\pgfsys@useobject{currentmarker}{}%
\end{pgfscope}%
\end{pgfscope}%
\begin{pgfscope}%
\pgfsetbuttcap%
\pgfsetroundjoin%
\definecolor{currentfill}{rgb}{0.000000,0.000000,0.000000}%
\pgfsetfillcolor{currentfill}%
\pgfsetlinewidth{0.602250pt}%
\definecolor{currentstroke}{rgb}{0.000000,0.000000,0.000000}%
\pgfsetstrokecolor{currentstroke}%
\pgfsetdash{}{0pt}%
\pgfsys@defobject{currentmarker}{\pgfqpoint{0.000000in}{-0.027778in}}{\pgfqpoint{0.000000in}{0.000000in}}{%
\pgfpathmoveto{\pgfqpoint{0.000000in}{0.000000in}}%
\pgfpathlineto{\pgfqpoint{0.000000in}{-0.027778in}}%
\pgfusepath{stroke,fill}%
}%
\begin{pgfscope}%
\pgfsys@transformshift{1.628965in}{0.440000in}%
\pgfsys@useobject{currentmarker}{}%
\end{pgfscope}%
\end{pgfscope}%
\begin{pgfscope}%
\pgfsetbuttcap%
\pgfsetroundjoin%
\definecolor{currentfill}{rgb}{0.000000,0.000000,0.000000}%
\pgfsetfillcolor{currentfill}%
\pgfsetlinewidth{0.602250pt}%
\definecolor{currentstroke}{rgb}{0.000000,0.000000,0.000000}%
\pgfsetstrokecolor{currentstroke}%
\pgfsetdash{}{0pt}%
\pgfsys@defobject{currentmarker}{\pgfqpoint{0.000000in}{-0.027778in}}{\pgfqpoint{0.000000in}{0.000000in}}{%
\pgfpathmoveto{\pgfqpoint{0.000000in}{0.000000in}}%
\pgfpathlineto{\pgfqpoint{0.000000in}{-0.027778in}}%
\pgfusepath{stroke,fill}%
}%
\begin{pgfscope}%
\pgfsys@transformshift{1.700186in}{0.440000in}%
\pgfsys@useobject{currentmarker}{}%
\end{pgfscope}%
\end{pgfscope}%
\begin{pgfscope}%
\pgfsetbuttcap%
\pgfsetroundjoin%
\definecolor{currentfill}{rgb}{0.000000,0.000000,0.000000}%
\pgfsetfillcolor{currentfill}%
\pgfsetlinewidth{0.602250pt}%
\definecolor{currentstroke}{rgb}{0.000000,0.000000,0.000000}%
\pgfsetstrokecolor{currentstroke}%
\pgfsetdash{}{0pt}%
\pgfsys@defobject{currentmarker}{\pgfqpoint{0.000000in}{-0.027778in}}{\pgfqpoint{0.000000in}{0.000000in}}{%
\pgfpathmoveto{\pgfqpoint{0.000000in}{0.000000in}}%
\pgfpathlineto{\pgfqpoint{0.000000in}{-0.027778in}}%
\pgfusepath{stroke,fill}%
}%
\begin{pgfscope}%
\pgfsys@transformshift{1.761880in}{0.440000in}%
\pgfsys@useobject{currentmarker}{}%
\end{pgfscope}%
\end{pgfscope}%
\begin{pgfscope}%
\pgfsetbuttcap%
\pgfsetroundjoin%
\definecolor{currentfill}{rgb}{0.000000,0.000000,0.000000}%
\pgfsetfillcolor{currentfill}%
\pgfsetlinewidth{0.602250pt}%
\definecolor{currentstroke}{rgb}{0.000000,0.000000,0.000000}%
\pgfsetstrokecolor{currentstroke}%
\pgfsetdash{}{0pt}%
\pgfsys@defobject{currentmarker}{\pgfqpoint{0.000000in}{-0.027778in}}{\pgfqpoint{0.000000in}{0.000000in}}{%
\pgfpathmoveto{\pgfqpoint{0.000000in}{0.000000in}}%
\pgfpathlineto{\pgfqpoint{0.000000in}{-0.027778in}}%
\pgfusepath{stroke,fill}%
}%
\begin{pgfscope}%
\pgfsys@transformshift{1.816298in}{0.440000in}%
\pgfsys@useobject{currentmarker}{}%
\end{pgfscope}%
\end{pgfscope}%
\begin{pgfscope}%
\pgfsetbuttcap%
\pgfsetroundjoin%
\definecolor{currentfill}{rgb}{0.000000,0.000000,0.000000}%
\pgfsetfillcolor{currentfill}%
\pgfsetlinewidth{0.602250pt}%
\definecolor{currentstroke}{rgb}{0.000000,0.000000,0.000000}%
\pgfsetstrokecolor{currentstroke}%
\pgfsetdash{}{0pt}%
\pgfsys@defobject{currentmarker}{\pgfqpoint{0.000000in}{-0.027778in}}{\pgfqpoint{0.000000in}{0.000000in}}{%
\pgfpathmoveto{\pgfqpoint{0.000000in}{0.000000in}}%
\pgfpathlineto{\pgfqpoint{0.000000in}{-0.027778in}}%
\pgfusepath{stroke,fill}%
}%
\begin{pgfscope}%
\pgfsys@transformshift{2.185225in}{0.440000in}%
\pgfsys@useobject{currentmarker}{}%
\end{pgfscope}%
\end{pgfscope}%
\begin{pgfscope}%
\pgfsetbuttcap%
\pgfsetroundjoin%
\definecolor{currentfill}{rgb}{0.000000,0.000000,0.000000}%
\pgfsetfillcolor{currentfill}%
\pgfsetlinewidth{0.602250pt}%
\definecolor{currentstroke}{rgb}{0.000000,0.000000,0.000000}%
\pgfsetstrokecolor{currentstroke}%
\pgfsetdash{}{0pt}%
\pgfsys@defobject{currentmarker}{\pgfqpoint{0.000000in}{-0.027778in}}{\pgfqpoint{0.000000in}{0.000000in}}{%
\pgfpathmoveto{\pgfqpoint{0.000000in}{0.000000in}}%
\pgfpathlineto{\pgfqpoint{0.000000in}{-0.027778in}}%
\pgfusepath{stroke,fill}%
}%
\begin{pgfscope}%
\pgfsys@transformshift{2.372558in}{0.440000in}%
\pgfsys@useobject{currentmarker}{}%
\end{pgfscope}%
\end{pgfscope}%
\begin{pgfscope}%
\pgfsetbuttcap%
\pgfsetroundjoin%
\definecolor{currentfill}{rgb}{0.000000,0.000000,0.000000}%
\pgfsetfillcolor{currentfill}%
\pgfsetlinewidth{0.602250pt}%
\definecolor{currentstroke}{rgb}{0.000000,0.000000,0.000000}%
\pgfsetstrokecolor{currentstroke}%
\pgfsetdash{}{0pt}%
\pgfsys@defobject{currentmarker}{\pgfqpoint{0.000000in}{-0.027778in}}{\pgfqpoint{0.000000in}{0.000000in}}{%
\pgfpathmoveto{\pgfqpoint{0.000000in}{0.000000in}}%
\pgfpathlineto{\pgfqpoint{0.000000in}{-0.027778in}}%
\pgfusepath{stroke,fill}%
}%
\begin{pgfscope}%
\pgfsys@transformshift{2.505473in}{0.440000in}%
\pgfsys@useobject{currentmarker}{}%
\end{pgfscope}%
\end{pgfscope}%
\begin{pgfscope}%
\pgfsetbuttcap%
\pgfsetroundjoin%
\definecolor{currentfill}{rgb}{0.000000,0.000000,0.000000}%
\pgfsetfillcolor{currentfill}%
\pgfsetlinewidth{0.602250pt}%
\definecolor{currentstroke}{rgb}{0.000000,0.000000,0.000000}%
\pgfsetstrokecolor{currentstroke}%
\pgfsetdash{}{0pt}%
\pgfsys@defobject{currentmarker}{\pgfqpoint{0.000000in}{-0.027778in}}{\pgfqpoint{0.000000in}{0.000000in}}{%
\pgfpathmoveto{\pgfqpoint{0.000000in}{0.000000in}}%
\pgfpathlineto{\pgfqpoint{0.000000in}{-0.027778in}}%
\pgfusepath{stroke,fill}%
}%
\begin{pgfscope}%
\pgfsys@transformshift{2.608570in}{0.440000in}%
\pgfsys@useobject{currentmarker}{}%
\end{pgfscope}%
\end{pgfscope}%
\begin{pgfscope}%
\pgfsetbuttcap%
\pgfsetroundjoin%
\definecolor{currentfill}{rgb}{0.000000,0.000000,0.000000}%
\pgfsetfillcolor{currentfill}%
\pgfsetlinewidth{0.602250pt}%
\definecolor{currentstroke}{rgb}{0.000000,0.000000,0.000000}%
\pgfsetstrokecolor{currentstroke}%
\pgfsetdash{}{0pt}%
\pgfsys@defobject{currentmarker}{\pgfqpoint{0.000000in}{-0.027778in}}{\pgfqpoint{0.000000in}{0.000000in}}{%
\pgfpathmoveto{\pgfqpoint{0.000000in}{0.000000in}}%
\pgfpathlineto{\pgfqpoint{0.000000in}{-0.027778in}}%
\pgfusepath{stroke,fill}%
}%
\begin{pgfscope}%
\pgfsys@transformshift{2.692806in}{0.440000in}%
\pgfsys@useobject{currentmarker}{}%
\end{pgfscope}%
\end{pgfscope}%
\begin{pgfscope}%
\pgfsetbuttcap%
\pgfsetroundjoin%
\definecolor{currentfill}{rgb}{0.000000,0.000000,0.000000}%
\pgfsetfillcolor{currentfill}%
\pgfsetlinewidth{0.602250pt}%
\definecolor{currentstroke}{rgb}{0.000000,0.000000,0.000000}%
\pgfsetstrokecolor{currentstroke}%
\pgfsetdash{}{0pt}%
\pgfsys@defobject{currentmarker}{\pgfqpoint{0.000000in}{-0.027778in}}{\pgfqpoint{0.000000in}{0.000000in}}{%
\pgfpathmoveto{\pgfqpoint{0.000000in}{0.000000in}}%
\pgfpathlineto{\pgfqpoint{0.000000in}{-0.027778in}}%
\pgfusepath{stroke,fill}%
}%
\begin{pgfscope}%
\pgfsys@transformshift{2.764027in}{0.440000in}%
\pgfsys@useobject{currentmarker}{}%
\end{pgfscope}%
\end{pgfscope}%
\begin{pgfscope}%
\pgfsetbuttcap%
\pgfsetroundjoin%
\definecolor{currentfill}{rgb}{0.000000,0.000000,0.000000}%
\pgfsetfillcolor{currentfill}%
\pgfsetlinewidth{0.602250pt}%
\definecolor{currentstroke}{rgb}{0.000000,0.000000,0.000000}%
\pgfsetstrokecolor{currentstroke}%
\pgfsetdash{}{0pt}%
\pgfsys@defobject{currentmarker}{\pgfqpoint{0.000000in}{-0.027778in}}{\pgfqpoint{0.000000in}{0.000000in}}{%
\pgfpathmoveto{\pgfqpoint{0.000000in}{0.000000in}}%
\pgfpathlineto{\pgfqpoint{0.000000in}{-0.027778in}}%
\pgfusepath{stroke,fill}%
}%
\begin{pgfscope}%
\pgfsys@transformshift{2.825721in}{0.440000in}%
\pgfsys@useobject{currentmarker}{}%
\end{pgfscope}%
\end{pgfscope}%
\begin{pgfscope}%
\pgfsetbuttcap%
\pgfsetroundjoin%
\definecolor{currentfill}{rgb}{0.000000,0.000000,0.000000}%
\pgfsetfillcolor{currentfill}%
\pgfsetlinewidth{0.602250pt}%
\definecolor{currentstroke}{rgb}{0.000000,0.000000,0.000000}%
\pgfsetstrokecolor{currentstroke}%
\pgfsetdash{}{0pt}%
\pgfsys@defobject{currentmarker}{\pgfqpoint{0.000000in}{-0.027778in}}{\pgfqpoint{0.000000in}{0.000000in}}{%
\pgfpathmoveto{\pgfqpoint{0.000000in}{0.000000in}}%
\pgfpathlineto{\pgfqpoint{0.000000in}{-0.027778in}}%
\pgfusepath{stroke,fill}%
}%
\begin{pgfscope}%
\pgfsys@transformshift{2.880139in}{0.440000in}%
\pgfsys@useobject{currentmarker}{}%
\end{pgfscope}%
\end{pgfscope}%
\begin{pgfscope}%
\pgfsetbuttcap%
\pgfsetroundjoin%
\definecolor{currentfill}{rgb}{0.000000,0.000000,0.000000}%
\pgfsetfillcolor{currentfill}%
\pgfsetlinewidth{0.602250pt}%
\definecolor{currentstroke}{rgb}{0.000000,0.000000,0.000000}%
\pgfsetstrokecolor{currentstroke}%
\pgfsetdash{}{0pt}%
\pgfsys@defobject{currentmarker}{\pgfqpoint{0.000000in}{-0.027778in}}{\pgfqpoint{0.000000in}{0.000000in}}{%
\pgfpathmoveto{\pgfqpoint{0.000000in}{0.000000in}}%
\pgfpathlineto{\pgfqpoint{0.000000in}{-0.027778in}}%
\pgfusepath{stroke,fill}%
}%
\begin{pgfscope}%
\pgfsys@transformshift{3.249066in}{0.440000in}%
\pgfsys@useobject{currentmarker}{}%
\end{pgfscope}%
\end{pgfscope}%
\begin{pgfscope}%
\pgfsetbuttcap%
\pgfsetroundjoin%
\definecolor{currentfill}{rgb}{0.000000,0.000000,0.000000}%
\pgfsetfillcolor{currentfill}%
\pgfsetlinewidth{0.602250pt}%
\definecolor{currentstroke}{rgb}{0.000000,0.000000,0.000000}%
\pgfsetstrokecolor{currentstroke}%
\pgfsetdash{}{0pt}%
\pgfsys@defobject{currentmarker}{\pgfqpoint{0.000000in}{-0.027778in}}{\pgfqpoint{0.000000in}{0.000000in}}{%
\pgfpathmoveto{\pgfqpoint{0.000000in}{0.000000in}}%
\pgfpathlineto{\pgfqpoint{0.000000in}{-0.027778in}}%
\pgfusepath{stroke,fill}%
}%
\begin{pgfscope}%
\pgfsys@transformshift{3.436399in}{0.440000in}%
\pgfsys@useobject{currentmarker}{}%
\end{pgfscope}%
\end{pgfscope}%
\begin{pgfscope}%
\pgfsetbuttcap%
\pgfsetroundjoin%
\definecolor{currentfill}{rgb}{0.000000,0.000000,0.000000}%
\pgfsetfillcolor{currentfill}%
\pgfsetlinewidth{0.602250pt}%
\definecolor{currentstroke}{rgb}{0.000000,0.000000,0.000000}%
\pgfsetstrokecolor{currentstroke}%
\pgfsetdash{}{0pt}%
\pgfsys@defobject{currentmarker}{\pgfqpoint{0.000000in}{-0.027778in}}{\pgfqpoint{0.000000in}{0.000000in}}{%
\pgfpathmoveto{\pgfqpoint{0.000000in}{0.000000in}}%
\pgfpathlineto{\pgfqpoint{0.000000in}{-0.027778in}}%
\pgfusepath{stroke,fill}%
}%
\begin{pgfscope}%
\pgfsys@transformshift{3.569313in}{0.440000in}%
\pgfsys@useobject{currentmarker}{}%
\end{pgfscope}%
\end{pgfscope}%
\begin{pgfscope}%
\pgfsetbuttcap%
\pgfsetroundjoin%
\definecolor{currentfill}{rgb}{0.000000,0.000000,0.000000}%
\pgfsetfillcolor{currentfill}%
\pgfsetlinewidth{0.602250pt}%
\definecolor{currentstroke}{rgb}{0.000000,0.000000,0.000000}%
\pgfsetstrokecolor{currentstroke}%
\pgfsetdash{}{0pt}%
\pgfsys@defobject{currentmarker}{\pgfqpoint{0.000000in}{-0.027778in}}{\pgfqpoint{0.000000in}{0.000000in}}{%
\pgfpathmoveto{\pgfqpoint{0.000000in}{0.000000in}}%
\pgfpathlineto{\pgfqpoint{0.000000in}{-0.027778in}}%
\pgfusepath{stroke,fill}%
}%
\begin{pgfscope}%
\pgfsys@transformshift{3.672410in}{0.440000in}%
\pgfsys@useobject{currentmarker}{}%
\end{pgfscope}%
\end{pgfscope}%
\begin{pgfscope}%
\pgfsetbuttcap%
\pgfsetroundjoin%
\definecolor{currentfill}{rgb}{0.000000,0.000000,0.000000}%
\pgfsetfillcolor{currentfill}%
\pgfsetlinewidth{0.602250pt}%
\definecolor{currentstroke}{rgb}{0.000000,0.000000,0.000000}%
\pgfsetstrokecolor{currentstroke}%
\pgfsetdash{}{0pt}%
\pgfsys@defobject{currentmarker}{\pgfqpoint{0.000000in}{-0.027778in}}{\pgfqpoint{0.000000in}{0.000000in}}{%
\pgfpathmoveto{\pgfqpoint{0.000000in}{0.000000in}}%
\pgfpathlineto{\pgfqpoint{0.000000in}{-0.027778in}}%
\pgfusepath{stroke,fill}%
}%
\begin{pgfscope}%
\pgfsys@transformshift{3.756646in}{0.440000in}%
\pgfsys@useobject{currentmarker}{}%
\end{pgfscope}%
\end{pgfscope}%
\begin{pgfscope}%
\pgfsetbuttcap%
\pgfsetroundjoin%
\definecolor{currentfill}{rgb}{0.000000,0.000000,0.000000}%
\pgfsetfillcolor{currentfill}%
\pgfsetlinewidth{0.602250pt}%
\definecolor{currentstroke}{rgb}{0.000000,0.000000,0.000000}%
\pgfsetstrokecolor{currentstroke}%
\pgfsetdash{}{0pt}%
\pgfsys@defobject{currentmarker}{\pgfqpoint{0.000000in}{-0.027778in}}{\pgfqpoint{0.000000in}{0.000000in}}{%
\pgfpathmoveto{\pgfqpoint{0.000000in}{0.000000in}}%
\pgfpathlineto{\pgfqpoint{0.000000in}{-0.027778in}}%
\pgfusepath{stroke,fill}%
}%
\begin{pgfscope}%
\pgfsys@transformshift{3.827867in}{0.440000in}%
\pgfsys@useobject{currentmarker}{}%
\end{pgfscope}%
\end{pgfscope}%
\begin{pgfscope}%
\pgfsetbuttcap%
\pgfsetroundjoin%
\definecolor{currentfill}{rgb}{0.000000,0.000000,0.000000}%
\pgfsetfillcolor{currentfill}%
\pgfsetlinewidth{0.602250pt}%
\definecolor{currentstroke}{rgb}{0.000000,0.000000,0.000000}%
\pgfsetstrokecolor{currentstroke}%
\pgfsetdash{}{0pt}%
\pgfsys@defobject{currentmarker}{\pgfqpoint{0.000000in}{-0.027778in}}{\pgfqpoint{0.000000in}{0.000000in}}{%
\pgfpathmoveto{\pgfqpoint{0.000000in}{0.000000in}}%
\pgfpathlineto{\pgfqpoint{0.000000in}{-0.027778in}}%
\pgfusepath{stroke,fill}%
}%
\begin{pgfscope}%
\pgfsys@transformshift{3.889561in}{0.440000in}%
\pgfsys@useobject{currentmarker}{}%
\end{pgfscope}%
\end{pgfscope}%
\begin{pgfscope}%
\pgfsetbuttcap%
\pgfsetroundjoin%
\definecolor{currentfill}{rgb}{0.000000,0.000000,0.000000}%
\pgfsetfillcolor{currentfill}%
\pgfsetlinewidth{0.602250pt}%
\definecolor{currentstroke}{rgb}{0.000000,0.000000,0.000000}%
\pgfsetstrokecolor{currentstroke}%
\pgfsetdash{}{0pt}%
\pgfsys@defobject{currentmarker}{\pgfqpoint{0.000000in}{-0.027778in}}{\pgfqpoint{0.000000in}{0.000000in}}{%
\pgfpathmoveto{\pgfqpoint{0.000000in}{0.000000in}}%
\pgfpathlineto{\pgfqpoint{0.000000in}{-0.027778in}}%
\pgfusepath{stroke,fill}%
}%
\begin{pgfscope}%
\pgfsys@transformshift{3.943980in}{0.440000in}%
\pgfsys@useobject{currentmarker}{}%
\end{pgfscope}%
\end{pgfscope}%
\begin{pgfscope}%
\pgfsetbuttcap%
\pgfsetroundjoin%
\definecolor{currentfill}{rgb}{0.000000,0.000000,0.000000}%
\pgfsetfillcolor{currentfill}%
\pgfsetlinewidth{0.602250pt}%
\definecolor{currentstroke}{rgb}{0.000000,0.000000,0.000000}%
\pgfsetstrokecolor{currentstroke}%
\pgfsetdash{}{0pt}%
\pgfsys@defobject{currentmarker}{\pgfqpoint{0.000000in}{-0.027778in}}{\pgfqpoint{0.000000in}{0.000000in}}{%
\pgfpathmoveto{\pgfqpoint{0.000000in}{0.000000in}}%
\pgfpathlineto{\pgfqpoint{0.000000in}{-0.027778in}}%
\pgfusepath{stroke,fill}%
}%
\begin{pgfscope}%
\pgfsys@transformshift{4.312906in}{0.440000in}%
\pgfsys@useobject{currentmarker}{}%
\end{pgfscope}%
\end{pgfscope}%
\begin{pgfscope}%
\pgfsetbuttcap%
\pgfsetroundjoin%
\definecolor{currentfill}{rgb}{0.000000,0.000000,0.000000}%
\pgfsetfillcolor{currentfill}%
\pgfsetlinewidth{0.602250pt}%
\definecolor{currentstroke}{rgb}{0.000000,0.000000,0.000000}%
\pgfsetstrokecolor{currentstroke}%
\pgfsetdash{}{0pt}%
\pgfsys@defobject{currentmarker}{\pgfqpoint{0.000000in}{-0.027778in}}{\pgfqpoint{0.000000in}{0.000000in}}{%
\pgfpathmoveto{\pgfqpoint{0.000000in}{0.000000in}}%
\pgfpathlineto{\pgfqpoint{0.000000in}{-0.027778in}}%
\pgfusepath{stroke,fill}%
}%
\begin{pgfscope}%
\pgfsys@transformshift{4.500239in}{0.440000in}%
\pgfsys@useobject{currentmarker}{}%
\end{pgfscope}%
\end{pgfscope}%
\begin{pgfscope}%
\definecolor{textcolor}{rgb}{0.000000,0.000000,0.000000}%
\pgfsetstrokecolor{textcolor}%
\pgfsetfillcolor{textcolor}%
\pgftext[x=2.562500in,y=0.152809in,,top]{\color{textcolor}\sffamily\fontsize{10.000000}{12.000000}\selectfont \(\displaystyle N_\mathrm{QMC}\)}%
\end{pgfscope}%
\begin{pgfscope}%
\pgfsetbuttcap%
\pgfsetroundjoin%
\definecolor{currentfill}{rgb}{0.000000,0.000000,0.000000}%
\pgfsetfillcolor{currentfill}%
\pgfsetlinewidth{0.803000pt}%
\definecolor{currentstroke}{rgb}{0.000000,0.000000,0.000000}%
\pgfsetstrokecolor{currentstroke}%
\pgfsetdash{}{0pt}%
\pgfsys@defobject{currentmarker}{\pgfqpoint{-0.048611in}{0.000000in}}{\pgfqpoint{0.000000in}{0.000000in}}{%
\pgfpathmoveto{\pgfqpoint{0.000000in}{0.000000in}}%
\pgfpathlineto{\pgfqpoint{-0.048611in}{0.000000in}}%
\pgfusepath{stroke,fill}%
}%
\begin{pgfscope}%
\pgfsys@transformshift{0.625000in}{0.968375in}%
\pgfsys@useobject{currentmarker}{}%
\end{pgfscope}%
\end{pgfscope}%
\begin{pgfscope}%
\definecolor{textcolor}{rgb}{0.000000,0.000000,0.000000}%
\pgfsetstrokecolor{textcolor}%
\pgfsetfillcolor{textcolor}%
\pgftext[x=0.239775in,y=0.915613in,left,base]{\color{textcolor}\sffamily\fontsize{10.000000}{12.000000}\selectfont \(\displaystyle 10^{-5}\)}%
\end{pgfscope}%
\begin{pgfscope}%
\pgfsetbuttcap%
\pgfsetroundjoin%
\definecolor{currentfill}{rgb}{0.000000,0.000000,0.000000}%
\pgfsetfillcolor{currentfill}%
\pgfsetlinewidth{0.803000pt}%
\definecolor{currentstroke}{rgb}{0.000000,0.000000,0.000000}%
\pgfsetstrokecolor{currentstroke}%
\pgfsetdash{}{0pt}%
\pgfsys@defobject{currentmarker}{\pgfqpoint{-0.048611in}{0.000000in}}{\pgfqpoint{0.000000in}{0.000000in}}{%
\pgfpathmoveto{\pgfqpoint{0.000000in}{0.000000in}}%
\pgfpathlineto{\pgfqpoint{-0.048611in}{0.000000in}}%
\pgfusepath{stroke,fill}%
}%
\begin{pgfscope}%
\pgfsys@transformshift{0.625000in}{1.830007in}%
\pgfsys@useobject{currentmarker}{}%
\end{pgfscope}%
\end{pgfscope}%
\begin{pgfscope}%
\definecolor{textcolor}{rgb}{0.000000,0.000000,0.000000}%
\pgfsetstrokecolor{textcolor}%
\pgfsetfillcolor{textcolor}%
\pgftext[x=0.239775in,y=1.777245in,left,base]{\color{textcolor}\sffamily\fontsize{10.000000}{12.000000}\selectfont \(\displaystyle 10^{-4}\)}%
\end{pgfscope}%
\begin{pgfscope}%
\pgfsetbuttcap%
\pgfsetroundjoin%
\definecolor{currentfill}{rgb}{0.000000,0.000000,0.000000}%
\pgfsetfillcolor{currentfill}%
\pgfsetlinewidth{0.803000pt}%
\definecolor{currentstroke}{rgb}{0.000000,0.000000,0.000000}%
\pgfsetstrokecolor{currentstroke}%
\pgfsetdash{}{0pt}%
\pgfsys@defobject{currentmarker}{\pgfqpoint{-0.048611in}{0.000000in}}{\pgfqpoint{0.000000in}{0.000000in}}{%
\pgfpathmoveto{\pgfqpoint{0.000000in}{0.000000in}}%
\pgfpathlineto{\pgfqpoint{-0.048611in}{0.000000in}}%
\pgfusepath{stroke,fill}%
}%
\begin{pgfscope}%
\pgfsys@transformshift{0.625000in}{2.691638in}%
\pgfsys@useobject{currentmarker}{}%
\end{pgfscope}%
\end{pgfscope}%
\begin{pgfscope}%
\definecolor{textcolor}{rgb}{0.000000,0.000000,0.000000}%
\pgfsetstrokecolor{textcolor}%
\pgfsetfillcolor{textcolor}%
\pgftext[x=0.239775in,y=2.638877in,left,base]{\color{textcolor}\sffamily\fontsize{10.000000}{12.000000}\selectfont \(\displaystyle 10^{-3}\)}%
\end{pgfscope}%
\begin{pgfscope}%
\pgfsetbuttcap%
\pgfsetroundjoin%
\definecolor{currentfill}{rgb}{0.000000,0.000000,0.000000}%
\pgfsetfillcolor{currentfill}%
\pgfsetlinewidth{0.602250pt}%
\definecolor{currentstroke}{rgb}{0.000000,0.000000,0.000000}%
\pgfsetstrokecolor{currentstroke}%
\pgfsetdash{}{0pt}%
\pgfsys@defobject{currentmarker}{\pgfqpoint{-0.027778in}{0.000000in}}{\pgfqpoint{0.000000in}{0.000000in}}{%
\pgfpathmoveto{\pgfqpoint{0.000000in}{0.000000in}}%
\pgfpathlineto{\pgfqpoint{-0.027778in}{0.000000in}}%
\pgfusepath{stroke,fill}%
}%
\begin{pgfscope}%
\pgfsys@transformshift{0.625000in}{0.517846in}%
\pgfsys@useobject{currentmarker}{}%
\end{pgfscope}%
\end{pgfscope}%
\begin{pgfscope}%
\pgfsetbuttcap%
\pgfsetroundjoin%
\definecolor{currentfill}{rgb}{0.000000,0.000000,0.000000}%
\pgfsetfillcolor{currentfill}%
\pgfsetlinewidth{0.602250pt}%
\definecolor{currentstroke}{rgb}{0.000000,0.000000,0.000000}%
\pgfsetstrokecolor{currentstroke}%
\pgfsetdash{}{0pt}%
\pgfsys@defobject{currentmarker}{\pgfqpoint{-0.027778in}{0.000000in}}{\pgfqpoint{0.000000in}{0.000000in}}{%
\pgfpathmoveto{\pgfqpoint{0.000000in}{0.000000in}}%
\pgfpathlineto{\pgfqpoint{-0.027778in}{0.000000in}}%
\pgfusepath{stroke,fill}%
}%
\begin{pgfscope}%
\pgfsys@transformshift{0.625000in}{0.625497in}%
\pgfsys@useobject{currentmarker}{}%
\end{pgfscope}%
\end{pgfscope}%
\begin{pgfscope}%
\pgfsetbuttcap%
\pgfsetroundjoin%
\definecolor{currentfill}{rgb}{0.000000,0.000000,0.000000}%
\pgfsetfillcolor{currentfill}%
\pgfsetlinewidth{0.602250pt}%
\definecolor{currentstroke}{rgb}{0.000000,0.000000,0.000000}%
\pgfsetstrokecolor{currentstroke}%
\pgfsetdash{}{0pt}%
\pgfsys@defobject{currentmarker}{\pgfqpoint{-0.027778in}{0.000000in}}{\pgfqpoint{0.000000in}{0.000000in}}{%
\pgfpathmoveto{\pgfqpoint{0.000000in}{0.000000in}}%
\pgfpathlineto{\pgfqpoint{-0.027778in}{0.000000in}}%
\pgfusepath{stroke,fill}%
}%
\begin{pgfscope}%
\pgfsys@transformshift{0.625000in}{0.708998in}%
\pgfsys@useobject{currentmarker}{}%
\end{pgfscope}%
\end{pgfscope}%
\begin{pgfscope}%
\pgfsetbuttcap%
\pgfsetroundjoin%
\definecolor{currentfill}{rgb}{0.000000,0.000000,0.000000}%
\pgfsetfillcolor{currentfill}%
\pgfsetlinewidth{0.602250pt}%
\definecolor{currentstroke}{rgb}{0.000000,0.000000,0.000000}%
\pgfsetstrokecolor{currentstroke}%
\pgfsetdash{}{0pt}%
\pgfsys@defobject{currentmarker}{\pgfqpoint{-0.027778in}{0.000000in}}{\pgfqpoint{0.000000in}{0.000000in}}{%
\pgfpathmoveto{\pgfqpoint{0.000000in}{0.000000in}}%
\pgfpathlineto{\pgfqpoint{-0.027778in}{0.000000in}}%
\pgfusepath{stroke,fill}%
}%
\begin{pgfscope}%
\pgfsys@transformshift{0.625000in}{0.777223in}%
\pgfsys@useobject{currentmarker}{}%
\end{pgfscope}%
\end{pgfscope}%
\begin{pgfscope}%
\pgfsetbuttcap%
\pgfsetroundjoin%
\definecolor{currentfill}{rgb}{0.000000,0.000000,0.000000}%
\pgfsetfillcolor{currentfill}%
\pgfsetlinewidth{0.602250pt}%
\definecolor{currentstroke}{rgb}{0.000000,0.000000,0.000000}%
\pgfsetstrokecolor{currentstroke}%
\pgfsetdash{}{0pt}%
\pgfsys@defobject{currentmarker}{\pgfqpoint{-0.027778in}{0.000000in}}{\pgfqpoint{0.000000in}{0.000000in}}{%
\pgfpathmoveto{\pgfqpoint{0.000000in}{0.000000in}}%
\pgfpathlineto{\pgfqpoint{-0.027778in}{0.000000in}}%
\pgfusepath{stroke,fill}%
}%
\begin{pgfscope}%
\pgfsys@transformshift{0.625000in}{0.834907in}%
\pgfsys@useobject{currentmarker}{}%
\end{pgfscope}%
\end{pgfscope}%
\begin{pgfscope}%
\pgfsetbuttcap%
\pgfsetroundjoin%
\definecolor{currentfill}{rgb}{0.000000,0.000000,0.000000}%
\pgfsetfillcolor{currentfill}%
\pgfsetlinewidth{0.602250pt}%
\definecolor{currentstroke}{rgb}{0.000000,0.000000,0.000000}%
\pgfsetstrokecolor{currentstroke}%
\pgfsetdash{}{0pt}%
\pgfsys@defobject{currentmarker}{\pgfqpoint{-0.027778in}{0.000000in}}{\pgfqpoint{0.000000in}{0.000000in}}{%
\pgfpathmoveto{\pgfqpoint{0.000000in}{0.000000in}}%
\pgfpathlineto{\pgfqpoint{-0.027778in}{0.000000in}}%
\pgfusepath{stroke,fill}%
}%
\begin{pgfscope}%
\pgfsys@transformshift{0.625000in}{0.884874in}%
\pgfsys@useobject{currentmarker}{}%
\end{pgfscope}%
\end{pgfscope}%
\begin{pgfscope}%
\pgfsetbuttcap%
\pgfsetroundjoin%
\definecolor{currentfill}{rgb}{0.000000,0.000000,0.000000}%
\pgfsetfillcolor{currentfill}%
\pgfsetlinewidth{0.602250pt}%
\definecolor{currentstroke}{rgb}{0.000000,0.000000,0.000000}%
\pgfsetstrokecolor{currentstroke}%
\pgfsetdash{}{0pt}%
\pgfsys@defobject{currentmarker}{\pgfqpoint{-0.027778in}{0.000000in}}{\pgfqpoint{0.000000in}{0.000000in}}{%
\pgfpathmoveto{\pgfqpoint{0.000000in}{0.000000in}}%
\pgfpathlineto{\pgfqpoint{-0.027778in}{0.000000in}}%
\pgfusepath{stroke,fill}%
}%
\begin{pgfscope}%
\pgfsys@transformshift{0.625000in}{0.928949in}%
\pgfsys@useobject{currentmarker}{}%
\end{pgfscope}%
\end{pgfscope}%
\begin{pgfscope}%
\pgfsetbuttcap%
\pgfsetroundjoin%
\definecolor{currentfill}{rgb}{0.000000,0.000000,0.000000}%
\pgfsetfillcolor{currentfill}%
\pgfsetlinewidth{0.602250pt}%
\definecolor{currentstroke}{rgb}{0.000000,0.000000,0.000000}%
\pgfsetstrokecolor{currentstroke}%
\pgfsetdash{}{0pt}%
\pgfsys@defobject{currentmarker}{\pgfqpoint{-0.027778in}{0.000000in}}{\pgfqpoint{0.000000in}{0.000000in}}{%
\pgfpathmoveto{\pgfqpoint{0.000000in}{0.000000in}}%
\pgfpathlineto{\pgfqpoint{-0.027778in}{0.000000in}}%
\pgfusepath{stroke,fill}%
}%
\begin{pgfscope}%
\pgfsys@transformshift{0.625000in}{1.227752in}%
\pgfsys@useobject{currentmarker}{}%
\end{pgfscope}%
\end{pgfscope}%
\begin{pgfscope}%
\pgfsetbuttcap%
\pgfsetroundjoin%
\definecolor{currentfill}{rgb}{0.000000,0.000000,0.000000}%
\pgfsetfillcolor{currentfill}%
\pgfsetlinewidth{0.602250pt}%
\definecolor{currentstroke}{rgb}{0.000000,0.000000,0.000000}%
\pgfsetstrokecolor{currentstroke}%
\pgfsetdash{}{0pt}%
\pgfsys@defobject{currentmarker}{\pgfqpoint{-0.027778in}{0.000000in}}{\pgfqpoint{0.000000in}{0.000000in}}{%
\pgfpathmoveto{\pgfqpoint{0.000000in}{0.000000in}}%
\pgfpathlineto{\pgfqpoint{-0.027778in}{0.000000in}}%
\pgfusepath{stroke,fill}%
}%
\begin{pgfscope}%
\pgfsys@transformshift{0.625000in}{1.379478in}%
\pgfsys@useobject{currentmarker}{}%
\end{pgfscope}%
\end{pgfscope}%
\begin{pgfscope}%
\pgfsetbuttcap%
\pgfsetroundjoin%
\definecolor{currentfill}{rgb}{0.000000,0.000000,0.000000}%
\pgfsetfillcolor{currentfill}%
\pgfsetlinewidth{0.602250pt}%
\definecolor{currentstroke}{rgb}{0.000000,0.000000,0.000000}%
\pgfsetstrokecolor{currentstroke}%
\pgfsetdash{}{0pt}%
\pgfsys@defobject{currentmarker}{\pgfqpoint{-0.027778in}{0.000000in}}{\pgfqpoint{0.000000in}{0.000000in}}{%
\pgfpathmoveto{\pgfqpoint{0.000000in}{0.000000in}}%
\pgfpathlineto{\pgfqpoint{-0.027778in}{0.000000in}}%
\pgfusepath{stroke,fill}%
}%
\begin{pgfscope}%
\pgfsys@transformshift{0.625000in}{1.487129in}%
\pgfsys@useobject{currentmarker}{}%
\end{pgfscope}%
\end{pgfscope}%
\begin{pgfscope}%
\pgfsetbuttcap%
\pgfsetroundjoin%
\definecolor{currentfill}{rgb}{0.000000,0.000000,0.000000}%
\pgfsetfillcolor{currentfill}%
\pgfsetlinewidth{0.602250pt}%
\definecolor{currentstroke}{rgb}{0.000000,0.000000,0.000000}%
\pgfsetstrokecolor{currentstroke}%
\pgfsetdash{}{0pt}%
\pgfsys@defobject{currentmarker}{\pgfqpoint{-0.027778in}{0.000000in}}{\pgfqpoint{0.000000in}{0.000000in}}{%
\pgfpathmoveto{\pgfqpoint{0.000000in}{0.000000in}}%
\pgfpathlineto{\pgfqpoint{-0.027778in}{0.000000in}}%
\pgfusepath{stroke,fill}%
}%
\begin{pgfscope}%
\pgfsys@transformshift{0.625000in}{1.570630in}%
\pgfsys@useobject{currentmarker}{}%
\end{pgfscope}%
\end{pgfscope}%
\begin{pgfscope}%
\pgfsetbuttcap%
\pgfsetroundjoin%
\definecolor{currentfill}{rgb}{0.000000,0.000000,0.000000}%
\pgfsetfillcolor{currentfill}%
\pgfsetlinewidth{0.602250pt}%
\definecolor{currentstroke}{rgb}{0.000000,0.000000,0.000000}%
\pgfsetstrokecolor{currentstroke}%
\pgfsetdash{}{0pt}%
\pgfsys@defobject{currentmarker}{\pgfqpoint{-0.027778in}{0.000000in}}{\pgfqpoint{0.000000in}{0.000000in}}{%
\pgfpathmoveto{\pgfqpoint{0.000000in}{0.000000in}}%
\pgfpathlineto{\pgfqpoint{-0.027778in}{0.000000in}}%
\pgfusepath{stroke,fill}%
}%
\begin{pgfscope}%
\pgfsys@transformshift{0.625000in}{1.638855in}%
\pgfsys@useobject{currentmarker}{}%
\end{pgfscope}%
\end{pgfscope}%
\begin{pgfscope}%
\pgfsetbuttcap%
\pgfsetroundjoin%
\definecolor{currentfill}{rgb}{0.000000,0.000000,0.000000}%
\pgfsetfillcolor{currentfill}%
\pgfsetlinewidth{0.602250pt}%
\definecolor{currentstroke}{rgb}{0.000000,0.000000,0.000000}%
\pgfsetstrokecolor{currentstroke}%
\pgfsetdash{}{0pt}%
\pgfsys@defobject{currentmarker}{\pgfqpoint{-0.027778in}{0.000000in}}{\pgfqpoint{0.000000in}{0.000000in}}{%
\pgfpathmoveto{\pgfqpoint{0.000000in}{0.000000in}}%
\pgfpathlineto{\pgfqpoint{-0.027778in}{0.000000in}}%
\pgfusepath{stroke,fill}%
}%
\begin{pgfscope}%
\pgfsys@transformshift{0.625000in}{1.696538in}%
\pgfsys@useobject{currentmarker}{}%
\end{pgfscope}%
\end{pgfscope}%
\begin{pgfscope}%
\pgfsetbuttcap%
\pgfsetroundjoin%
\definecolor{currentfill}{rgb}{0.000000,0.000000,0.000000}%
\pgfsetfillcolor{currentfill}%
\pgfsetlinewidth{0.602250pt}%
\definecolor{currentstroke}{rgb}{0.000000,0.000000,0.000000}%
\pgfsetstrokecolor{currentstroke}%
\pgfsetdash{}{0pt}%
\pgfsys@defobject{currentmarker}{\pgfqpoint{-0.027778in}{0.000000in}}{\pgfqpoint{0.000000in}{0.000000in}}{%
\pgfpathmoveto{\pgfqpoint{0.000000in}{0.000000in}}%
\pgfpathlineto{\pgfqpoint{-0.027778in}{0.000000in}}%
\pgfusepath{stroke,fill}%
}%
\begin{pgfscope}%
\pgfsys@transformshift{0.625000in}{1.746506in}%
\pgfsys@useobject{currentmarker}{}%
\end{pgfscope}%
\end{pgfscope}%
\begin{pgfscope}%
\pgfsetbuttcap%
\pgfsetroundjoin%
\definecolor{currentfill}{rgb}{0.000000,0.000000,0.000000}%
\pgfsetfillcolor{currentfill}%
\pgfsetlinewidth{0.602250pt}%
\definecolor{currentstroke}{rgb}{0.000000,0.000000,0.000000}%
\pgfsetstrokecolor{currentstroke}%
\pgfsetdash{}{0pt}%
\pgfsys@defobject{currentmarker}{\pgfqpoint{-0.027778in}{0.000000in}}{\pgfqpoint{0.000000in}{0.000000in}}{%
\pgfpathmoveto{\pgfqpoint{0.000000in}{0.000000in}}%
\pgfpathlineto{\pgfqpoint{-0.027778in}{0.000000in}}%
\pgfusepath{stroke,fill}%
}%
\begin{pgfscope}%
\pgfsys@transformshift{0.625000in}{1.790581in}%
\pgfsys@useobject{currentmarker}{}%
\end{pgfscope}%
\end{pgfscope}%
\begin{pgfscope}%
\pgfsetbuttcap%
\pgfsetroundjoin%
\definecolor{currentfill}{rgb}{0.000000,0.000000,0.000000}%
\pgfsetfillcolor{currentfill}%
\pgfsetlinewidth{0.602250pt}%
\definecolor{currentstroke}{rgb}{0.000000,0.000000,0.000000}%
\pgfsetstrokecolor{currentstroke}%
\pgfsetdash{}{0pt}%
\pgfsys@defobject{currentmarker}{\pgfqpoint{-0.027778in}{0.000000in}}{\pgfqpoint{0.000000in}{0.000000in}}{%
\pgfpathmoveto{\pgfqpoint{0.000000in}{0.000000in}}%
\pgfpathlineto{\pgfqpoint{-0.027778in}{0.000000in}}%
\pgfusepath{stroke,fill}%
}%
\begin{pgfscope}%
\pgfsys@transformshift{0.625000in}{2.089384in}%
\pgfsys@useobject{currentmarker}{}%
\end{pgfscope}%
\end{pgfscope}%
\begin{pgfscope}%
\pgfsetbuttcap%
\pgfsetroundjoin%
\definecolor{currentfill}{rgb}{0.000000,0.000000,0.000000}%
\pgfsetfillcolor{currentfill}%
\pgfsetlinewidth{0.602250pt}%
\definecolor{currentstroke}{rgb}{0.000000,0.000000,0.000000}%
\pgfsetstrokecolor{currentstroke}%
\pgfsetdash{}{0pt}%
\pgfsys@defobject{currentmarker}{\pgfqpoint{-0.027778in}{0.000000in}}{\pgfqpoint{0.000000in}{0.000000in}}{%
\pgfpathmoveto{\pgfqpoint{0.000000in}{0.000000in}}%
\pgfpathlineto{\pgfqpoint{-0.027778in}{0.000000in}}%
\pgfusepath{stroke,fill}%
}%
\begin{pgfscope}%
\pgfsys@transformshift{0.625000in}{2.241110in}%
\pgfsys@useobject{currentmarker}{}%
\end{pgfscope}%
\end{pgfscope}%
\begin{pgfscope}%
\pgfsetbuttcap%
\pgfsetroundjoin%
\definecolor{currentfill}{rgb}{0.000000,0.000000,0.000000}%
\pgfsetfillcolor{currentfill}%
\pgfsetlinewidth{0.602250pt}%
\definecolor{currentstroke}{rgb}{0.000000,0.000000,0.000000}%
\pgfsetstrokecolor{currentstroke}%
\pgfsetdash{}{0pt}%
\pgfsys@defobject{currentmarker}{\pgfqpoint{-0.027778in}{0.000000in}}{\pgfqpoint{0.000000in}{0.000000in}}{%
\pgfpathmoveto{\pgfqpoint{0.000000in}{0.000000in}}%
\pgfpathlineto{\pgfqpoint{-0.027778in}{0.000000in}}%
\pgfusepath{stroke,fill}%
}%
\begin{pgfscope}%
\pgfsys@transformshift{0.625000in}{2.348761in}%
\pgfsys@useobject{currentmarker}{}%
\end{pgfscope}%
\end{pgfscope}%
\begin{pgfscope}%
\pgfsetbuttcap%
\pgfsetroundjoin%
\definecolor{currentfill}{rgb}{0.000000,0.000000,0.000000}%
\pgfsetfillcolor{currentfill}%
\pgfsetlinewidth{0.602250pt}%
\definecolor{currentstroke}{rgb}{0.000000,0.000000,0.000000}%
\pgfsetstrokecolor{currentstroke}%
\pgfsetdash{}{0pt}%
\pgfsys@defobject{currentmarker}{\pgfqpoint{-0.027778in}{0.000000in}}{\pgfqpoint{0.000000in}{0.000000in}}{%
\pgfpathmoveto{\pgfqpoint{0.000000in}{0.000000in}}%
\pgfpathlineto{\pgfqpoint{-0.027778in}{0.000000in}}%
\pgfusepath{stroke,fill}%
}%
\begin{pgfscope}%
\pgfsys@transformshift{0.625000in}{2.432261in}%
\pgfsys@useobject{currentmarker}{}%
\end{pgfscope}%
\end{pgfscope}%
\begin{pgfscope}%
\pgfsetbuttcap%
\pgfsetroundjoin%
\definecolor{currentfill}{rgb}{0.000000,0.000000,0.000000}%
\pgfsetfillcolor{currentfill}%
\pgfsetlinewidth{0.602250pt}%
\definecolor{currentstroke}{rgb}{0.000000,0.000000,0.000000}%
\pgfsetstrokecolor{currentstroke}%
\pgfsetdash{}{0pt}%
\pgfsys@defobject{currentmarker}{\pgfqpoint{-0.027778in}{0.000000in}}{\pgfqpoint{0.000000in}{0.000000in}}{%
\pgfpathmoveto{\pgfqpoint{0.000000in}{0.000000in}}%
\pgfpathlineto{\pgfqpoint{-0.027778in}{0.000000in}}%
\pgfusepath{stroke,fill}%
}%
\begin{pgfscope}%
\pgfsys@transformshift{0.625000in}{2.500486in}%
\pgfsys@useobject{currentmarker}{}%
\end{pgfscope}%
\end{pgfscope}%
\begin{pgfscope}%
\pgfsetbuttcap%
\pgfsetroundjoin%
\definecolor{currentfill}{rgb}{0.000000,0.000000,0.000000}%
\pgfsetfillcolor{currentfill}%
\pgfsetlinewidth{0.602250pt}%
\definecolor{currentstroke}{rgb}{0.000000,0.000000,0.000000}%
\pgfsetstrokecolor{currentstroke}%
\pgfsetdash{}{0pt}%
\pgfsys@defobject{currentmarker}{\pgfqpoint{-0.027778in}{0.000000in}}{\pgfqpoint{0.000000in}{0.000000in}}{%
\pgfpathmoveto{\pgfqpoint{0.000000in}{0.000000in}}%
\pgfpathlineto{\pgfqpoint{-0.027778in}{0.000000in}}%
\pgfusepath{stroke,fill}%
}%
\begin{pgfscope}%
\pgfsys@transformshift{0.625000in}{2.558170in}%
\pgfsys@useobject{currentmarker}{}%
\end{pgfscope}%
\end{pgfscope}%
\begin{pgfscope}%
\pgfsetbuttcap%
\pgfsetroundjoin%
\definecolor{currentfill}{rgb}{0.000000,0.000000,0.000000}%
\pgfsetfillcolor{currentfill}%
\pgfsetlinewidth{0.602250pt}%
\definecolor{currentstroke}{rgb}{0.000000,0.000000,0.000000}%
\pgfsetstrokecolor{currentstroke}%
\pgfsetdash{}{0pt}%
\pgfsys@defobject{currentmarker}{\pgfqpoint{-0.027778in}{0.000000in}}{\pgfqpoint{0.000000in}{0.000000in}}{%
\pgfpathmoveto{\pgfqpoint{0.000000in}{0.000000in}}%
\pgfpathlineto{\pgfqpoint{-0.027778in}{0.000000in}}%
\pgfusepath{stroke,fill}%
}%
\begin{pgfscope}%
\pgfsys@transformshift{0.625000in}{2.608138in}%
\pgfsys@useobject{currentmarker}{}%
\end{pgfscope}%
\end{pgfscope}%
\begin{pgfscope}%
\pgfsetbuttcap%
\pgfsetroundjoin%
\definecolor{currentfill}{rgb}{0.000000,0.000000,0.000000}%
\pgfsetfillcolor{currentfill}%
\pgfsetlinewidth{0.602250pt}%
\definecolor{currentstroke}{rgb}{0.000000,0.000000,0.000000}%
\pgfsetstrokecolor{currentstroke}%
\pgfsetdash{}{0pt}%
\pgfsys@defobject{currentmarker}{\pgfqpoint{-0.027778in}{0.000000in}}{\pgfqpoint{0.000000in}{0.000000in}}{%
\pgfpathmoveto{\pgfqpoint{0.000000in}{0.000000in}}%
\pgfpathlineto{\pgfqpoint{-0.027778in}{0.000000in}}%
\pgfusepath{stroke,fill}%
}%
\begin{pgfscope}%
\pgfsys@transformshift{0.625000in}{2.652212in}%
\pgfsys@useobject{currentmarker}{}%
\end{pgfscope}%
\end{pgfscope}%
\begin{pgfscope}%
\pgfsetbuttcap%
\pgfsetroundjoin%
\definecolor{currentfill}{rgb}{0.000000,0.000000,0.000000}%
\pgfsetfillcolor{currentfill}%
\pgfsetlinewidth{0.602250pt}%
\definecolor{currentstroke}{rgb}{0.000000,0.000000,0.000000}%
\pgfsetstrokecolor{currentstroke}%
\pgfsetdash{}{0pt}%
\pgfsys@defobject{currentmarker}{\pgfqpoint{-0.027778in}{0.000000in}}{\pgfqpoint{0.000000in}{0.000000in}}{%
\pgfpathmoveto{\pgfqpoint{0.000000in}{0.000000in}}%
\pgfpathlineto{\pgfqpoint{-0.027778in}{0.000000in}}%
\pgfusepath{stroke,fill}%
}%
\begin{pgfscope}%
\pgfsys@transformshift{0.625000in}{2.951015in}%
\pgfsys@useobject{currentmarker}{}%
\end{pgfscope}%
\end{pgfscope}%
\begin{pgfscope}%
\pgfsetbuttcap%
\pgfsetroundjoin%
\definecolor{currentfill}{rgb}{0.000000,0.000000,0.000000}%
\pgfsetfillcolor{currentfill}%
\pgfsetlinewidth{0.602250pt}%
\definecolor{currentstroke}{rgb}{0.000000,0.000000,0.000000}%
\pgfsetstrokecolor{currentstroke}%
\pgfsetdash{}{0pt}%
\pgfsys@defobject{currentmarker}{\pgfqpoint{-0.027778in}{0.000000in}}{\pgfqpoint{0.000000in}{0.000000in}}{%
\pgfpathmoveto{\pgfqpoint{0.000000in}{0.000000in}}%
\pgfpathlineto{\pgfqpoint{-0.027778in}{0.000000in}}%
\pgfusepath{stroke,fill}%
}%
\begin{pgfscope}%
\pgfsys@transformshift{0.625000in}{3.102741in}%
\pgfsys@useobject{currentmarker}{}%
\end{pgfscope}%
\end{pgfscope}%
\begin{pgfscope}%
\pgfsetbuttcap%
\pgfsetroundjoin%
\definecolor{currentfill}{rgb}{0.000000,0.000000,0.000000}%
\pgfsetfillcolor{currentfill}%
\pgfsetlinewidth{0.602250pt}%
\definecolor{currentstroke}{rgb}{0.000000,0.000000,0.000000}%
\pgfsetstrokecolor{currentstroke}%
\pgfsetdash{}{0pt}%
\pgfsys@defobject{currentmarker}{\pgfqpoint{-0.027778in}{0.000000in}}{\pgfqpoint{0.000000in}{0.000000in}}{%
\pgfpathmoveto{\pgfqpoint{0.000000in}{0.000000in}}%
\pgfpathlineto{\pgfqpoint{-0.027778in}{0.000000in}}%
\pgfusepath{stroke,fill}%
}%
\begin{pgfscope}%
\pgfsys@transformshift{0.625000in}{3.210392in}%
\pgfsys@useobject{currentmarker}{}%
\end{pgfscope}%
\end{pgfscope}%
\begin{pgfscope}%
\pgfsetbuttcap%
\pgfsetroundjoin%
\definecolor{currentfill}{rgb}{0.000000,0.000000,0.000000}%
\pgfsetfillcolor{currentfill}%
\pgfsetlinewidth{0.602250pt}%
\definecolor{currentstroke}{rgb}{0.000000,0.000000,0.000000}%
\pgfsetstrokecolor{currentstroke}%
\pgfsetdash{}{0pt}%
\pgfsys@defobject{currentmarker}{\pgfqpoint{-0.027778in}{0.000000in}}{\pgfqpoint{0.000000in}{0.000000in}}{%
\pgfpathmoveto{\pgfqpoint{0.000000in}{0.000000in}}%
\pgfpathlineto{\pgfqpoint{-0.027778in}{0.000000in}}%
\pgfusepath{stroke,fill}%
}%
\begin{pgfscope}%
\pgfsys@transformshift{0.625000in}{3.293893in}%
\pgfsys@useobject{currentmarker}{}%
\end{pgfscope}%
\end{pgfscope}%
\begin{pgfscope}%
\pgfsetbuttcap%
\pgfsetroundjoin%
\definecolor{currentfill}{rgb}{0.000000,0.000000,0.000000}%
\pgfsetfillcolor{currentfill}%
\pgfsetlinewidth{0.602250pt}%
\definecolor{currentstroke}{rgb}{0.000000,0.000000,0.000000}%
\pgfsetstrokecolor{currentstroke}%
\pgfsetdash{}{0pt}%
\pgfsys@defobject{currentmarker}{\pgfqpoint{-0.027778in}{0.000000in}}{\pgfqpoint{0.000000in}{0.000000in}}{%
\pgfpathmoveto{\pgfqpoint{0.000000in}{0.000000in}}%
\pgfpathlineto{\pgfqpoint{-0.027778in}{0.000000in}}%
\pgfusepath{stroke,fill}%
}%
\begin{pgfscope}%
\pgfsys@transformshift{0.625000in}{3.362118in}%
\pgfsys@useobject{currentmarker}{}%
\end{pgfscope}%
\end{pgfscope}%
\begin{pgfscope}%
\pgfsetbuttcap%
\pgfsetroundjoin%
\definecolor{currentfill}{rgb}{0.000000,0.000000,0.000000}%
\pgfsetfillcolor{currentfill}%
\pgfsetlinewidth{0.602250pt}%
\definecolor{currentstroke}{rgb}{0.000000,0.000000,0.000000}%
\pgfsetstrokecolor{currentstroke}%
\pgfsetdash{}{0pt}%
\pgfsys@defobject{currentmarker}{\pgfqpoint{-0.027778in}{0.000000in}}{\pgfqpoint{0.000000in}{0.000000in}}{%
\pgfpathmoveto{\pgfqpoint{0.000000in}{0.000000in}}%
\pgfpathlineto{\pgfqpoint{-0.027778in}{0.000000in}}%
\pgfusepath{stroke,fill}%
}%
\begin{pgfscope}%
\pgfsys@transformshift{0.625000in}{3.419802in}%
\pgfsys@useobject{currentmarker}{}%
\end{pgfscope}%
\end{pgfscope}%
\begin{pgfscope}%
\pgfsetbuttcap%
\pgfsetroundjoin%
\definecolor{currentfill}{rgb}{0.000000,0.000000,0.000000}%
\pgfsetfillcolor{currentfill}%
\pgfsetlinewidth{0.602250pt}%
\definecolor{currentstroke}{rgb}{0.000000,0.000000,0.000000}%
\pgfsetstrokecolor{currentstroke}%
\pgfsetdash{}{0pt}%
\pgfsys@defobject{currentmarker}{\pgfqpoint{-0.027778in}{0.000000in}}{\pgfqpoint{0.000000in}{0.000000in}}{%
\pgfpathmoveto{\pgfqpoint{0.000000in}{0.000000in}}%
\pgfpathlineto{\pgfqpoint{-0.027778in}{0.000000in}}%
\pgfusepath{stroke,fill}%
}%
\begin{pgfscope}%
\pgfsys@transformshift{0.625000in}{3.469769in}%
\pgfsys@useobject{currentmarker}{}%
\end{pgfscope}%
\end{pgfscope}%
\begin{pgfscope}%
\pgfsetbuttcap%
\pgfsetroundjoin%
\definecolor{currentfill}{rgb}{0.000000,0.000000,0.000000}%
\pgfsetfillcolor{currentfill}%
\pgfsetlinewidth{0.602250pt}%
\definecolor{currentstroke}{rgb}{0.000000,0.000000,0.000000}%
\pgfsetstrokecolor{currentstroke}%
\pgfsetdash{}{0pt}%
\pgfsys@defobject{currentmarker}{\pgfqpoint{-0.027778in}{0.000000in}}{\pgfqpoint{0.000000in}{0.000000in}}{%
\pgfpathmoveto{\pgfqpoint{0.000000in}{0.000000in}}%
\pgfpathlineto{\pgfqpoint{-0.027778in}{0.000000in}}%
\pgfusepath{stroke,fill}%
}%
\begin{pgfscope}%
\pgfsys@transformshift{0.625000in}{3.513844in}%
\pgfsys@useobject{currentmarker}{}%
\end{pgfscope}%
\end{pgfscope}%
\begin{pgfscope}%
\definecolor{textcolor}{rgb}{0.000000,0.000000,0.000000}%
\pgfsetstrokecolor{textcolor}%
\pgfsetfillcolor{textcolor}%
\pgftext[x=0.184220in,y=1.980000in,,bottom,rotate=90.000000]{\color{textcolor}\sffamily\fontsize{10.000000}{12.000000}\selectfont QMC Error}%
\end{pgfscope}%
\begin{pgfscope}%
\pgfpathrectangle{\pgfqpoint{0.625000in}{0.440000in}}{\pgfqpoint{3.875000in}{3.080000in}}%
\pgfusepath{clip}%
\pgfsetbuttcap%
\pgfsetroundjoin%
\definecolor{currentfill}{rgb}{0.000000,0.000000,0.000000}%
\pgfsetfillcolor{currentfill}%
\pgfsetlinewidth{1.003750pt}%
\definecolor{currentstroke}{rgb}{0.000000,0.000000,0.000000}%
\pgfsetstrokecolor{currentstroke}%
\pgfsetdash{}{0pt}%
\pgfsys@defobject{currentmarker}{\pgfqpoint{-0.041667in}{-0.041667in}}{\pgfqpoint{0.041667in}{0.041667in}}{%
\pgfpathmoveto{\pgfqpoint{0.000000in}{-0.041667in}}%
\pgfpathcurveto{\pgfqpoint{0.011050in}{-0.041667in}}{\pgfqpoint{0.021649in}{-0.037276in}}{\pgfqpoint{0.029463in}{-0.029463in}}%
\pgfpathcurveto{\pgfqpoint{0.037276in}{-0.021649in}}{\pgfqpoint{0.041667in}{-0.011050in}}{\pgfqpoint{0.041667in}{0.000000in}}%
\pgfpathcurveto{\pgfqpoint{0.041667in}{0.011050in}}{\pgfqpoint{0.037276in}{0.021649in}}{\pgfqpoint{0.029463in}{0.029463in}}%
\pgfpathcurveto{\pgfqpoint{0.021649in}{0.037276in}}{\pgfqpoint{0.011050in}{0.041667in}}{\pgfqpoint{0.000000in}{0.041667in}}%
\pgfpathcurveto{\pgfqpoint{-0.011050in}{0.041667in}}{\pgfqpoint{-0.021649in}{0.037276in}}{\pgfqpoint{-0.029463in}{0.029463in}}%
\pgfpathcurveto{\pgfqpoint{-0.037276in}{0.021649in}}{\pgfqpoint{-0.041667in}{0.011050in}}{\pgfqpoint{-0.041667in}{0.000000in}}%
\pgfpathcurveto{\pgfqpoint{-0.041667in}{-0.011050in}}{\pgfqpoint{-0.037276in}{-0.021649in}}{\pgfqpoint{-0.029463in}{-0.029463in}}%
\pgfpathcurveto{\pgfqpoint{-0.021649in}{-0.037276in}}{\pgfqpoint{-0.011050in}{-0.041667in}}{\pgfqpoint{0.000000in}{-0.041667in}}%
\pgfpathclose%
\pgfusepath{stroke,fill}%
}%
\begin{pgfscope}%
\pgfsys@transformshift{0.801136in}{3.380000in}%
\pgfsys@useobject{currentmarker}{}%
\end{pgfscope}%
\begin{pgfscope}%
\pgfsys@transformshift{1.121384in}{2.881372in}%
\pgfsys@useobject{currentmarker}{}%
\end{pgfscope}%
\begin{pgfscope}%
\pgfsys@transformshift{1.441632in}{2.705294in}%
\pgfsys@useobject{currentmarker}{}%
\end{pgfscope}%
\begin{pgfscope}%
\pgfsys@transformshift{1.761880in}{2.568877in}%
\pgfsys@useobject{currentmarker}{}%
\end{pgfscope}%
\begin{pgfscope}%
\pgfsys@transformshift{2.082128in}{2.378385in}%
\pgfsys@useobject{currentmarker}{}%
\end{pgfscope}%
\begin{pgfscope}%
\pgfsys@transformshift{2.402376in}{2.328634in}%
\pgfsys@useobject{currentmarker}{}%
\end{pgfscope}%
\begin{pgfscope}%
\pgfsys@transformshift{2.722624in}{2.284289in}%
\pgfsys@useobject{currentmarker}{}%
\end{pgfscope}%
\begin{pgfscope}%
\pgfsys@transformshift{3.042872in}{1.822551in}%
\pgfsys@useobject{currentmarker}{}%
\end{pgfscope}%
\begin{pgfscope}%
\pgfsys@transformshift{3.363120in}{1.455532in}%
\pgfsys@useobject{currentmarker}{}%
\end{pgfscope}%
\begin{pgfscope}%
\pgfsys@transformshift{3.683368in}{1.224165in}%
\pgfsys@useobject{currentmarker}{}%
\end{pgfscope}%
\begin{pgfscope}%
\pgfsys@transformshift{4.003616in}{0.879579in}%
\pgfsys@useobject{currentmarker}{}%
\end{pgfscope}%
\begin{pgfscope}%
\pgfsys@transformshift{4.323864in}{0.580000in}%
\pgfsys@useobject{currentmarker}{}%
\end{pgfscope}%
\end{pgfscope}%
\begin{pgfscope}%
\pgfpathrectangle{\pgfqpoint{0.625000in}{0.440000in}}{\pgfqpoint{3.875000in}{3.080000in}}%
\pgfusepath{clip}%
\pgfsetbuttcap%
\pgfsetroundjoin%
\pgfsetlinewidth{1.505625pt}%
\definecolor{currentstroke}{rgb}{0.000000,0.000000,0.000000}%
\pgfsetstrokecolor{currentstroke}%
\pgfsetdash{{5.550000pt}{2.400000pt}}{0.000000pt}%
\pgfpathmoveto{\pgfqpoint{0.801136in}{3.318850in}}%
\pgfpathlineto{\pgfqpoint{4.323864in}{0.762596in}}%
\pgfusepath{stroke}%
\end{pgfscope}%
\begin{pgfscope}%
\pgfsetrectcap%
\pgfsetmiterjoin%
\pgfsetlinewidth{0.803000pt}%
\definecolor{currentstroke}{rgb}{0.000000,0.000000,0.000000}%
\pgfsetstrokecolor{currentstroke}%
\pgfsetdash{}{0pt}%
\pgfpathmoveto{\pgfqpoint{0.625000in}{0.440000in}}%
\pgfpathlineto{\pgfqpoint{0.625000in}{3.520000in}}%
\pgfusepath{stroke}%
\end{pgfscope}%
\begin{pgfscope}%
\pgfsetrectcap%
\pgfsetmiterjoin%
\pgfsetlinewidth{0.803000pt}%
\definecolor{currentstroke}{rgb}{0.000000,0.000000,0.000000}%
\pgfsetstrokecolor{currentstroke}%
\pgfsetdash{}{0pt}%
\pgfpathmoveto{\pgfqpoint{4.500000in}{0.440000in}}%
\pgfpathlineto{\pgfqpoint{4.500000in}{3.520000in}}%
\pgfusepath{stroke}%
\end{pgfscope}%
\begin{pgfscope}%
\pgfsetrectcap%
\pgfsetmiterjoin%
\pgfsetlinewidth{0.803000pt}%
\definecolor{currentstroke}{rgb}{0.000000,0.000000,0.000000}%
\pgfsetstrokecolor{currentstroke}%
\pgfsetdash{}{0pt}%
\pgfpathmoveto{\pgfqpoint{0.625000in}{0.440000in}}%
\pgfpathlineto{\pgfqpoint{4.500000in}{0.440000in}}%
\pgfusepath{stroke}%
\end{pgfscope}%
\begin{pgfscope}%
\pgfsetrectcap%
\pgfsetmiterjoin%
\pgfsetlinewidth{0.803000pt}%
\definecolor{currentstroke}{rgb}{0.000000,0.000000,0.000000}%
\pgfsetstrokecolor{currentstroke}%
\pgfsetdash{}{0pt}%
\pgfpathmoveto{\pgfqpoint{0.625000in}{3.520000in}}%
\pgfpathlineto{\pgfqpoint{4.500000in}{3.520000in}}%
\pgfusepath{stroke}%
\end{pgfscope}%
\begin{pgfscope}%
\pgfsetbuttcap%
\pgfsetmiterjoin%
\definecolor{currentfill}{rgb}{1.000000,1.000000,1.000000}%
\pgfsetfillcolor{currentfill}%
\pgfsetfillopacity{0.800000}%
\pgfsetlinewidth{1.003750pt}%
\definecolor{currentstroke}{rgb}{0.800000,0.800000,0.800000}%
\pgfsetstrokecolor{currentstroke}%
\pgfsetstrokeopacity{0.800000}%
\pgfsetdash{}{0pt}%
\pgfpathmoveto{\pgfqpoint{3.042243in}{3.155267in}}%
\pgfpathlineto{\pgfqpoint{4.402778in}{3.155267in}}%
\pgfpathquadraticcurveto{\pgfqpoint{4.430556in}{3.155267in}}{\pgfqpoint{4.430556in}{3.183045in}}%
\pgfpathlineto{\pgfqpoint{4.430556in}{3.422778in}}%
\pgfpathquadraticcurveto{\pgfqpoint{4.430556in}{3.450556in}}{\pgfqpoint{4.402778in}{3.450556in}}%
\pgfpathlineto{\pgfqpoint{3.042243in}{3.450556in}}%
\pgfpathquadraticcurveto{\pgfqpoint{3.014465in}{3.450556in}}{\pgfqpoint{3.014465in}{3.422778in}}%
\pgfpathlineto{\pgfqpoint{3.014465in}{3.183045in}}%
\pgfpathquadraticcurveto{\pgfqpoint{3.014465in}{3.155267in}}{\pgfqpoint{3.042243in}{3.155267in}}%
\pgfpathclose%
\pgfusepath{stroke,fill}%
\end{pgfscope}%
\begin{pgfscope}%
\pgfsetbuttcap%
\pgfsetroundjoin%
\pgfsetlinewidth{1.505625pt}%
\definecolor{currentstroke}{rgb}{0.000000,0.000000,0.000000}%
\pgfsetstrokecolor{currentstroke}%
\pgfsetdash{{5.550000pt}{2.400000pt}}{0.000000pt}%
\pgfpathmoveto{\pgfqpoint{3.070021in}{3.321543in}}%
\pgfpathlineto{\pgfqpoint{3.347799in}{3.321543in}}%
\pgfusepath{stroke}%
\end{pgfscope}%
\begin{pgfscope}%
\definecolor{textcolor}{rgb}{0.000000,0.000000,0.000000}%
\pgfsetstrokecolor{textcolor}%
\pgfsetfillcolor{textcolor}%
\pgftext[x=3.458910in,y=3.272932in,left,base]{\color{textcolor}\sffamily\fontsize{10.000000}{12.000000}\selectfont \(\displaystyle 0.0053N_{\mathrm{QMC}}^{-0.8959}\)}%
\end{pgfscope}%
\end{pgfpicture}%
\makeatother%
\endgroup%

    \caption[Quasi-Monte-Carlo error, for $Q=\int_D u$ and $k=10$.]{Quasi-Monte-Carlo error with increasing number of Quasi-Monte-Carlo points, for $Q=\int_D u$ and $k=10$. \label{fig:qmcintegral10}}
  \end{figure}
\begin{figure}[h]
  \centering
%% Creator: Matplotlib, PGF backend
%%
%% To include the figure in your LaTeX document, write
%%   \input{<filename>.pgf}
%%
%% Make sure the required packages are loaded in your preamble
%%   \usepackage{pgf}
%%
%% Figures using additional raster images can only be included by \input if
%% they are in the same directory as the main LaTeX file. For loading figures
%% from other directories you can use the `import` package
%%   \usepackage{import}
%% and then include the figures with
%%   \import{<path to file>}{<filename>.pgf}
%%
%% Matplotlib used the following preamble
%%   \usepackage{fontspec}
%%   \setmainfont{DejaVuSerif.ttf}[Path=/home/owen/progs/firedrake-complex/firedrake/lib/python3.5/site-packages/matplotlib/mpl-data/fonts/ttf/]
%%   \setsansfont{DejaVuSans.ttf}[Path=/home/owen/progs/firedrake-complex/firedrake/lib/python3.5/site-packages/matplotlib/mpl-data/fonts/ttf/]
%%   \setmonofont{DejaVuSansMono.ttf}[Path=/home/owen/progs/firedrake-complex/firedrake/lib/python3.5/site-packages/matplotlib/mpl-data/fonts/ttf/]
%%
\begingroup%
\makeatletter%
\begin{pgfpicture}%
\pgfpathrectangle{\pgfpointorigin}{\pgfqpoint{5.000000in}{4.000000in}}%
\pgfusepath{use as bounding box, clip}%
\begin{pgfscope}%
\pgfsetbuttcap%
\pgfsetmiterjoin%
\definecolor{currentfill}{rgb}{1.000000,1.000000,1.000000}%
\pgfsetfillcolor{currentfill}%
\pgfsetlinewidth{0.000000pt}%
\definecolor{currentstroke}{rgb}{1.000000,1.000000,1.000000}%
\pgfsetstrokecolor{currentstroke}%
\pgfsetdash{}{0pt}%
\pgfpathmoveto{\pgfqpoint{0.000000in}{0.000000in}}%
\pgfpathlineto{\pgfqpoint{5.000000in}{0.000000in}}%
\pgfpathlineto{\pgfqpoint{5.000000in}{4.000000in}}%
\pgfpathlineto{\pgfqpoint{0.000000in}{4.000000in}}%
\pgfpathclose%
\pgfusepath{fill}%
\end{pgfscope}%
\begin{pgfscope}%
\pgfsetbuttcap%
\pgfsetmiterjoin%
\definecolor{currentfill}{rgb}{1.000000,1.000000,1.000000}%
\pgfsetfillcolor{currentfill}%
\pgfsetlinewidth{0.000000pt}%
\definecolor{currentstroke}{rgb}{0.000000,0.000000,0.000000}%
\pgfsetstrokecolor{currentstroke}%
\pgfsetstrokeopacity{0.000000}%
\pgfsetdash{}{0pt}%
\pgfpathmoveto{\pgfqpoint{0.625000in}{0.440000in}}%
\pgfpathlineto{\pgfqpoint{4.500000in}{0.440000in}}%
\pgfpathlineto{\pgfqpoint{4.500000in}{3.520000in}}%
\pgfpathlineto{\pgfqpoint{0.625000in}{3.520000in}}%
\pgfpathclose%
\pgfusepath{fill}%
\end{pgfscope}%
\begin{pgfscope}%
\pgfsetbuttcap%
\pgfsetroundjoin%
\definecolor{currentfill}{rgb}{0.000000,0.000000,0.000000}%
\pgfsetfillcolor{currentfill}%
\pgfsetlinewidth{0.803000pt}%
\definecolor{currentstroke}{rgb}{0.000000,0.000000,0.000000}%
\pgfsetstrokecolor{currentstroke}%
\pgfsetdash{}{0pt}%
\pgfsys@defobject{currentmarker}{\pgfqpoint{0.000000in}{-0.048611in}}{\pgfqpoint{0.000000in}{0.000000in}}{%
\pgfpathmoveto{\pgfqpoint{0.000000in}{0.000000in}}%
\pgfpathlineto{\pgfqpoint{0.000000in}{-0.048611in}}%
\pgfusepath{stroke,fill}%
}%
\begin{pgfscope}%
\pgfsys@transformshift{0.801136in}{0.440000in}%
\pgfsys@useobject{currentmarker}{}%
\end{pgfscope}%
\end{pgfscope}%
\begin{pgfscope}%
\definecolor{textcolor}{rgb}{0.000000,0.000000,0.000000}%
\pgfsetstrokecolor{textcolor}%
\pgfsetfillcolor{textcolor}%
\pgftext[x=0.801136in,y=0.342778in,,top]{\color{textcolor}\sffamily\fontsize{10.000000}{12.000000}\selectfont \(\displaystyle 10^{0}\)}%
\end{pgfscope}%
\begin{pgfscope}%
\pgfsetbuttcap%
\pgfsetroundjoin%
\definecolor{currentfill}{rgb}{0.000000,0.000000,0.000000}%
\pgfsetfillcolor{currentfill}%
\pgfsetlinewidth{0.803000pt}%
\definecolor{currentstroke}{rgb}{0.000000,0.000000,0.000000}%
\pgfsetstrokecolor{currentstroke}%
\pgfsetdash{}{0pt}%
\pgfsys@defobject{currentmarker}{\pgfqpoint{0.000000in}{-0.048611in}}{\pgfqpoint{0.000000in}{0.000000in}}{%
\pgfpathmoveto{\pgfqpoint{0.000000in}{0.000000in}}%
\pgfpathlineto{\pgfqpoint{0.000000in}{-0.048611in}}%
\pgfusepath{stroke,fill}%
}%
\begin{pgfscope}%
\pgfsys@transformshift{1.864977in}{0.440000in}%
\pgfsys@useobject{currentmarker}{}%
\end{pgfscope}%
\end{pgfscope}%
\begin{pgfscope}%
\definecolor{textcolor}{rgb}{0.000000,0.000000,0.000000}%
\pgfsetstrokecolor{textcolor}%
\pgfsetfillcolor{textcolor}%
\pgftext[x=1.864977in,y=0.342778in,,top]{\color{textcolor}\sffamily\fontsize{10.000000}{12.000000}\selectfont \(\displaystyle 10^{1}\)}%
\end{pgfscope}%
\begin{pgfscope}%
\pgfsetbuttcap%
\pgfsetroundjoin%
\definecolor{currentfill}{rgb}{0.000000,0.000000,0.000000}%
\pgfsetfillcolor{currentfill}%
\pgfsetlinewidth{0.803000pt}%
\definecolor{currentstroke}{rgb}{0.000000,0.000000,0.000000}%
\pgfsetstrokecolor{currentstroke}%
\pgfsetdash{}{0pt}%
\pgfsys@defobject{currentmarker}{\pgfqpoint{0.000000in}{-0.048611in}}{\pgfqpoint{0.000000in}{0.000000in}}{%
\pgfpathmoveto{\pgfqpoint{0.000000in}{0.000000in}}%
\pgfpathlineto{\pgfqpoint{0.000000in}{-0.048611in}}%
\pgfusepath{stroke,fill}%
}%
\begin{pgfscope}%
\pgfsys@transformshift{2.928818in}{0.440000in}%
\pgfsys@useobject{currentmarker}{}%
\end{pgfscope}%
\end{pgfscope}%
\begin{pgfscope}%
\definecolor{textcolor}{rgb}{0.000000,0.000000,0.000000}%
\pgfsetstrokecolor{textcolor}%
\pgfsetfillcolor{textcolor}%
\pgftext[x=2.928818in,y=0.342778in,,top]{\color{textcolor}\sffamily\fontsize{10.000000}{12.000000}\selectfont \(\displaystyle 10^{2}\)}%
\end{pgfscope}%
\begin{pgfscope}%
\pgfsetbuttcap%
\pgfsetroundjoin%
\definecolor{currentfill}{rgb}{0.000000,0.000000,0.000000}%
\pgfsetfillcolor{currentfill}%
\pgfsetlinewidth{0.803000pt}%
\definecolor{currentstroke}{rgb}{0.000000,0.000000,0.000000}%
\pgfsetstrokecolor{currentstroke}%
\pgfsetdash{}{0pt}%
\pgfsys@defobject{currentmarker}{\pgfqpoint{0.000000in}{-0.048611in}}{\pgfqpoint{0.000000in}{0.000000in}}{%
\pgfpathmoveto{\pgfqpoint{0.000000in}{0.000000in}}%
\pgfpathlineto{\pgfqpoint{0.000000in}{-0.048611in}}%
\pgfusepath{stroke,fill}%
}%
\begin{pgfscope}%
\pgfsys@transformshift{3.992658in}{0.440000in}%
\pgfsys@useobject{currentmarker}{}%
\end{pgfscope}%
\end{pgfscope}%
\begin{pgfscope}%
\definecolor{textcolor}{rgb}{0.000000,0.000000,0.000000}%
\pgfsetstrokecolor{textcolor}%
\pgfsetfillcolor{textcolor}%
\pgftext[x=3.992658in,y=0.342778in,,top]{\color{textcolor}\sffamily\fontsize{10.000000}{12.000000}\selectfont \(\displaystyle 10^{3}\)}%
\end{pgfscope}%
\begin{pgfscope}%
\pgfsetbuttcap%
\pgfsetroundjoin%
\definecolor{currentfill}{rgb}{0.000000,0.000000,0.000000}%
\pgfsetfillcolor{currentfill}%
\pgfsetlinewidth{0.602250pt}%
\definecolor{currentstroke}{rgb}{0.000000,0.000000,0.000000}%
\pgfsetstrokecolor{currentstroke}%
\pgfsetdash{}{0pt}%
\pgfsys@defobject{currentmarker}{\pgfqpoint{0.000000in}{-0.027778in}}{\pgfqpoint{0.000000in}{0.000000in}}{%
\pgfpathmoveto{\pgfqpoint{0.000000in}{0.000000in}}%
\pgfpathlineto{\pgfqpoint{0.000000in}{-0.027778in}}%
\pgfusepath{stroke,fill}%
}%
\begin{pgfscope}%
\pgfsys@transformshift{0.636345in}{0.440000in}%
\pgfsys@useobject{currentmarker}{}%
\end{pgfscope}%
\end{pgfscope}%
\begin{pgfscope}%
\pgfsetbuttcap%
\pgfsetroundjoin%
\definecolor{currentfill}{rgb}{0.000000,0.000000,0.000000}%
\pgfsetfillcolor{currentfill}%
\pgfsetlinewidth{0.602250pt}%
\definecolor{currentstroke}{rgb}{0.000000,0.000000,0.000000}%
\pgfsetstrokecolor{currentstroke}%
\pgfsetdash{}{0pt}%
\pgfsys@defobject{currentmarker}{\pgfqpoint{0.000000in}{-0.027778in}}{\pgfqpoint{0.000000in}{0.000000in}}{%
\pgfpathmoveto{\pgfqpoint{0.000000in}{0.000000in}}%
\pgfpathlineto{\pgfqpoint{0.000000in}{-0.027778in}}%
\pgfusepath{stroke,fill}%
}%
\begin{pgfscope}%
\pgfsys@transformshift{0.698040in}{0.440000in}%
\pgfsys@useobject{currentmarker}{}%
\end{pgfscope}%
\end{pgfscope}%
\begin{pgfscope}%
\pgfsetbuttcap%
\pgfsetroundjoin%
\definecolor{currentfill}{rgb}{0.000000,0.000000,0.000000}%
\pgfsetfillcolor{currentfill}%
\pgfsetlinewidth{0.602250pt}%
\definecolor{currentstroke}{rgb}{0.000000,0.000000,0.000000}%
\pgfsetstrokecolor{currentstroke}%
\pgfsetdash{}{0pt}%
\pgfsys@defobject{currentmarker}{\pgfqpoint{0.000000in}{-0.027778in}}{\pgfqpoint{0.000000in}{0.000000in}}{%
\pgfpathmoveto{\pgfqpoint{0.000000in}{0.000000in}}%
\pgfpathlineto{\pgfqpoint{0.000000in}{-0.027778in}}%
\pgfusepath{stroke,fill}%
}%
\begin{pgfscope}%
\pgfsys@transformshift{0.752458in}{0.440000in}%
\pgfsys@useobject{currentmarker}{}%
\end{pgfscope}%
\end{pgfscope}%
\begin{pgfscope}%
\pgfsetbuttcap%
\pgfsetroundjoin%
\definecolor{currentfill}{rgb}{0.000000,0.000000,0.000000}%
\pgfsetfillcolor{currentfill}%
\pgfsetlinewidth{0.602250pt}%
\definecolor{currentstroke}{rgb}{0.000000,0.000000,0.000000}%
\pgfsetstrokecolor{currentstroke}%
\pgfsetdash{}{0pt}%
\pgfsys@defobject{currentmarker}{\pgfqpoint{0.000000in}{-0.027778in}}{\pgfqpoint{0.000000in}{0.000000in}}{%
\pgfpathmoveto{\pgfqpoint{0.000000in}{0.000000in}}%
\pgfpathlineto{\pgfqpoint{0.000000in}{-0.027778in}}%
\pgfusepath{stroke,fill}%
}%
\begin{pgfscope}%
\pgfsys@transformshift{1.121384in}{0.440000in}%
\pgfsys@useobject{currentmarker}{}%
\end{pgfscope}%
\end{pgfscope}%
\begin{pgfscope}%
\pgfsetbuttcap%
\pgfsetroundjoin%
\definecolor{currentfill}{rgb}{0.000000,0.000000,0.000000}%
\pgfsetfillcolor{currentfill}%
\pgfsetlinewidth{0.602250pt}%
\definecolor{currentstroke}{rgb}{0.000000,0.000000,0.000000}%
\pgfsetstrokecolor{currentstroke}%
\pgfsetdash{}{0pt}%
\pgfsys@defobject{currentmarker}{\pgfqpoint{0.000000in}{-0.027778in}}{\pgfqpoint{0.000000in}{0.000000in}}{%
\pgfpathmoveto{\pgfqpoint{0.000000in}{0.000000in}}%
\pgfpathlineto{\pgfqpoint{0.000000in}{-0.027778in}}%
\pgfusepath{stroke,fill}%
}%
\begin{pgfscope}%
\pgfsys@transformshift{1.308717in}{0.440000in}%
\pgfsys@useobject{currentmarker}{}%
\end{pgfscope}%
\end{pgfscope}%
\begin{pgfscope}%
\pgfsetbuttcap%
\pgfsetroundjoin%
\definecolor{currentfill}{rgb}{0.000000,0.000000,0.000000}%
\pgfsetfillcolor{currentfill}%
\pgfsetlinewidth{0.602250pt}%
\definecolor{currentstroke}{rgb}{0.000000,0.000000,0.000000}%
\pgfsetstrokecolor{currentstroke}%
\pgfsetdash{}{0pt}%
\pgfsys@defobject{currentmarker}{\pgfqpoint{0.000000in}{-0.027778in}}{\pgfqpoint{0.000000in}{0.000000in}}{%
\pgfpathmoveto{\pgfqpoint{0.000000in}{0.000000in}}%
\pgfpathlineto{\pgfqpoint{0.000000in}{-0.027778in}}%
\pgfusepath{stroke,fill}%
}%
\begin{pgfscope}%
\pgfsys@transformshift{1.441632in}{0.440000in}%
\pgfsys@useobject{currentmarker}{}%
\end{pgfscope}%
\end{pgfscope}%
\begin{pgfscope}%
\pgfsetbuttcap%
\pgfsetroundjoin%
\definecolor{currentfill}{rgb}{0.000000,0.000000,0.000000}%
\pgfsetfillcolor{currentfill}%
\pgfsetlinewidth{0.602250pt}%
\definecolor{currentstroke}{rgb}{0.000000,0.000000,0.000000}%
\pgfsetstrokecolor{currentstroke}%
\pgfsetdash{}{0pt}%
\pgfsys@defobject{currentmarker}{\pgfqpoint{0.000000in}{-0.027778in}}{\pgfqpoint{0.000000in}{0.000000in}}{%
\pgfpathmoveto{\pgfqpoint{0.000000in}{0.000000in}}%
\pgfpathlineto{\pgfqpoint{0.000000in}{-0.027778in}}%
\pgfusepath{stroke,fill}%
}%
\begin{pgfscope}%
\pgfsys@transformshift{1.544729in}{0.440000in}%
\pgfsys@useobject{currentmarker}{}%
\end{pgfscope}%
\end{pgfscope}%
\begin{pgfscope}%
\pgfsetbuttcap%
\pgfsetroundjoin%
\definecolor{currentfill}{rgb}{0.000000,0.000000,0.000000}%
\pgfsetfillcolor{currentfill}%
\pgfsetlinewidth{0.602250pt}%
\definecolor{currentstroke}{rgb}{0.000000,0.000000,0.000000}%
\pgfsetstrokecolor{currentstroke}%
\pgfsetdash{}{0pt}%
\pgfsys@defobject{currentmarker}{\pgfqpoint{0.000000in}{-0.027778in}}{\pgfqpoint{0.000000in}{0.000000in}}{%
\pgfpathmoveto{\pgfqpoint{0.000000in}{0.000000in}}%
\pgfpathlineto{\pgfqpoint{0.000000in}{-0.027778in}}%
\pgfusepath{stroke,fill}%
}%
\begin{pgfscope}%
\pgfsys@transformshift{1.628965in}{0.440000in}%
\pgfsys@useobject{currentmarker}{}%
\end{pgfscope}%
\end{pgfscope}%
\begin{pgfscope}%
\pgfsetbuttcap%
\pgfsetroundjoin%
\definecolor{currentfill}{rgb}{0.000000,0.000000,0.000000}%
\pgfsetfillcolor{currentfill}%
\pgfsetlinewidth{0.602250pt}%
\definecolor{currentstroke}{rgb}{0.000000,0.000000,0.000000}%
\pgfsetstrokecolor{currentstroke}%
\pgfsetdash{}{0pt}%
\pgfsys@defobject{currentmarker}{\pgfqpoint{0.000000in}{-0.027778in}}{\pgfqpoint{0.000000in}{0.000000in}}{%
\pgfpathmoveto{\pgfqpoint{0.000000in}{0.000000in}}%
\pgfpathlineto{\pgfqpoint{0.000000in}{-0.027778in}}%
\pgfusepath{stroke,fill}%
}%
\begin{pgfscope}%
\pgfsys@transformshift{1.700186in}{0.440000in}%
\pgfsys@useobject{currentmarker}{}%
\end{pgfscope}%
\end{pgfscope}%
\begin{pgfscope}%
\pgfsetbuttcap%
\pgfsetroundjoin%
\definecolor{currentfill}{rgb}{0.000000,0.000000,0.000000}%
\pgfsetfillcolor{currentfill}%
\pgfsetlinewidth{0.602250pt}%
\definecolor{currentstroke}{rgb}{0.000000,0.000000,0.000000}%
\pgfsetstrokecolor{currentstroke}%
\pgfsetdash{}{0pt}%
\pgfsys@defobject{currentmarker}{\pgfqpoint{0.000000in}{-0.027778in}}{\pgfqpoint{0.000000in}{0.000000in}}{%
\pgfpathmoveto{\pgfqpoint{0.000000in}{0.000000in}}%
\pgfpathlineto{\pgfqpoint{0.000000in}{-0.027778in}}%
\pgfusepath{stroke,fill}%
}%
\begin{pgfscope}%
\pgfsys@transformshift{1.761880in}{0.440000in}%
\pgfsys@useobject{currentmarker}{}%
\end{pgfscope}%
\end{pgfscope}%
\begin{pgfscope}%
\pgfsetbuttcap%
\pgfsetroundjoin%
\definecolor{currentfill}{rgb}{0.000000,0.000000,0.000000}%
\pgfsetfillcolor{currentfill}%
\pgfsetlinewidth{0.602250pt}%
\definecolor{currentstroke}{rgb}{0.000000,0.000000,0.000000}%
\pgfsetstrokecolor{currentstroke}%
\pgfsetdash{}{0pt}%
\pgfsys@defobject{currentmarker}{\pgfqpoint{0.000000in}{-0.027778in}}{\pgfqpoint{0.000000in}{0.000000in}}{%
\pgfpathmoveto{\pgfqpoint{0.000000in}{0.000000in}}%
\pgfpathlineto{\pgfqpoint{0.000000in}{-0.027778in}}%
\pgfusepath{stroke,fill}%
}%
\begin{pgfscope}%
\pgfsys@transformshift{1.816298in}{0.440000in}%
\pgfsys@useobject{currentmarker}{}%
\end{pgfscope}%
\end{pgfscope}%
\begin{pgfscope}%
\pgfsetbuttcap%
\pgfsetroundjoin%
\definecolor{currentfill}{rgb}{0.000000,0.000000,0.000000}%
\pgfsetfillcolor{currentfill}%
\pgfsetlinewidth{0.602250pt}%
\definecolor{currentstroke}{rgb}{0.000000,0.000000,0.000000}%
\pgfsetstrokecolor{currentstroke}%
\pgfsetdash{}{0pt}%
\pgfsys@defobject{currentmarker}{\pgfqpoint{0.000000in}{-0.027778in}}{\pgfqpoint{0.000000in}{0.000000in}}{%
\pgfpathmoveto{\pgfqpoint{0.000000in}{0.000000in}}%
\pgfpathlineto{\pgfqpoint{0.000000in}{-0.027778in}}%
\pgfusepath{stroke,fill}%
}%
\begin{pgfscope}%
\pgfsys@transformshift{2.185225in}{0.440000in}%
\pgfsys@useobject{currentmarker}{}%
\end{pgfscope}%
\end{pgfscope}%
\begin{pgfscope}%
\pgfsetbuttcap%
\pgfsetroundjoin%
\definecolor{currentfill}{rgb}{0.000000,0.000000,0.000000}%
\pgfsetfillcolor{currentfill}%
\pgfsetlinewidth{0.602250pt}%
\definecolor{currentstroke}{rgb}{0.000000,0.000000,0.000000}%
\pgfsetstrokecolor{currentstroke}%
\pgfsetdash{}{0pt}%
\pgfsys@defobject{currentmarker}{\pgfqpoint{0.000000in}{-0.027778in}}{\pgfqpoint{0.000000in}{0.000000in}}{%
\pgfpathmoveto{\pgfqpoint{0.000000in}{0.000000in}}%
\pgfpathlineto{\pgfqpoint{0.000000in}{-0.027778in}}%
\pgfusepath{stroke,fill}%
}%
\begin{pgfscope}%
\pgfsys@transformshift{2.372558in}{0.440000in}%
\pgfsys@useobject{currentmarker}{}%
\end{pgfscope}%
\end{pgfscope}%
\begin{pgfscope}%
\pgfsetbuttcap%
\pgfsetroundjoin%
\definecolor{currentfill}{rgb}{0.000000,0.000000,0.000000}%
\pgfsetfillcolor{currentfill}%
\pgfsetlinewidth{0.602250pt}%
\definecolor{currentstroke}{rgb}{0.000000,0.000000,0.000000}%
\pgfsetstrokecolor{currentstroke}%
\pgfsetdash{}{0pt}%
\pgfsys@defobject{currentmarker}{\pgfqpoint{0.000000in}{-0.027778in}}{\pgfqpoint{0.000000in}{0.000000in}}{%
\pgfpathmoveto{\pgfqpoint{0.000000in}{0.000000in}}%
\pgfpathlineto{\pgfqpoint{0.000000in}{-0.027778in}}%
\pgfusepath{stroke,fill}%
}%
\begin{pgfscope}%
\pgfsys@transformshift{2.505473in}{0.440000in}%
\pgfsys@useobject{currentmarker}{}%
\end{pgfscope}%
\end{pgfscope}%
\begin{pgfscope}%
\pgfsetbuttcap%
\pgfsetroundjoin%
\definecolor{currentfill}{rgb}{0.000000,0.000000,0.000000}%
\pgfsetfillcolor{currentfill}%
\pgfsetlinewidth{0.602250pt}%
\definecolor{currentstroke}{rgb}{0.000000,0.000000,0.000000}%
\pgfsetstrokecolor{currentstroke}%
\pgfsetdash{}{0pt}%
\pgfsys@defobject{currentmarker}{\pgfqpoint{0.000000in}{-0.027778in}}{\pgfqpoint{0.000000in}{0.000000in}}{%
\pgfpathmoveto{\pgfqpoint{0.000000in}{0.000000in}}%
\pgfpathlineto{\pgfqpoint{0.000000in}{-0.027778in}}%
\pgfusepath{stroke,fill}%
}%
\begin{pgfscope}%
\pgfsys@transformshift{2.608570in}{0.440000in}%
\pgfsys@useobject{currentmarker}{}%
\end{pgfscope}%
\end{pgfscope}%
\begin{pgfscope}%
\pgfsetbuttcap%
\pgfsetroundjoin%
\definecolor{currentfill}{rgb}{0.000000,0.000000,0.000000}%
\pgfsetfillcolor{currentfill}%
\pgfsetlinewidth{0.602250pt}%
\definecolor{currentstroke}{rgb}{0.000000,0.000000,0.000000}%
\pgfsetstrokecolor{currentstroke}%
\pgfsetdash{}{0pt}%
\pgfsys@defobject{currentmarker}{\pgfqpoint{0.000000in}{-0.027778in}}{\pgfqpoint{0.000000in}{0.000000in}}{%
\pgfpathmoveto{\pgfqpoint{0.000000in}{0.000000in}}%
\pgfpathlineto{\pgfqpoint{0.000000in}{-0.027778in}}%
\pgfusepath{stroke,fill}%
}%
\begin{pgfscope}%
\pgfsys@transformshift{2.692806in}{0.440000in}%
\pgfsys@useobject{currentmarker}{}%
\end{pgfscope}%
\end{pgfscope}%
\begin{pgfscope}%
\pgfsetbuttcap%
\pgfsetroundjoin%
\definecolor{currentfill}{rgb}{0.000000,0.000000,0.000000}%
\pgfsetfillcolor{currentfill}%
\pgfsetlinewidth{0.602250pt}%
\definecolor{currentstroke}{rgb}{0.000000,0.000000,0.000000}%
\pgfsetstrokecolor{currentstroke}%
\pgfsetdash{}{0pt}%
\pgfsys@defobject{currentmarker}{\pgfqpoint{0.000000in}{-0.027778in}}{\pgfqpoint{0.000000in}{0.000000in}}{%
\pgfpathmoveto{\pgfqpoint{0.000000in}{0.000000in}}%
\pgfpathlineto{\pgfqpoint{0.000000in}{-0.027778in}}%
\pgfusepath{stroke,fill}%
}%
\begin{pgfscope}%
\pgfsys@transformshift{2.764027in}{0.440000in}%
\pgfsys@useobject{currentmarker}{}%
\end{pgfscope}%
\end{pgfscope}%
\begin{pgfscope}%
\pgfsetbuttcap%
\pgfsetroundjoin%
\definecolor{currentfill}{rgb}{0.000000,0.000000,0.000000}%
\pgfsetfillcolor{currentfill}%
\pgfsetlinewidth{0.602250pt}%
\definecolor{currentstroke}{rgb}{0.000000,0.000000,0.000000}%
\pgfsetstrokecolor{currentstroke}%
\pgfsetdash{}{0pt}%
\pgfsys@defobject{currentmarker}{\pgfqpoint{0.000000in}{-0.027778in}}{\pgfqpoint{0.000000in}{0.000000in}}{%
\pgfpathmoveto{\pgfqpoint{0.000000in}{0.000000in}}%
\pgfpathlineto{\pgfqpoint{0.000000in}{-0.027778in}}%
\pgfusepath{stroke,fill}%
}%
\begin{pgfscope}%
\pgfsys@transformshift{2.825721in}{0.440000in}%
\pgfsys@useobject{currentmarker}{}%
\end{pgfscope}%
\end{pgfscope}%
\begin{pgfscope}%
\pgfsetbuttcap%
\pgfsetroundjoin%
\definecolor{currentfill}{rgb}{0.000000,0.000000,0.000000}%
\pgfsetfillcolor{currentfill}%
\pgfsetlinewidth{0.602250pt}%
\definecolor{currentstroke}{rgb}{0.000000,0.000000,0.000000}%
\pgfsetstrokecolor{currentstroke}%
\pgfsetdash{}{0pt}%
\pgfsys@defobject{currentmarker}{\pgfqpoint{0.000000in}{-0.027778in}}{\pgfqpoint{0.000000in}{0.000000in}}{%
\pgfpathmoveto{\pgfqpoint{0.000000in}{0.000000in}}%
\pgfpathlineto{\pgfqpoint{0.000000in}{-0.027778in}}%
\pgfusepath{stroke,fill}%
}%
\begin{pgfscope}%
\pgfsys@transformshift{2.880139in}{0.440000in}%
\pgfsys@useobject{currentmarker}{}%
\end{pgfscope}%
\end{pgfscope}%
\begin{pgfscope}%
\pgfsetbuttcap%
\pgfsetroundjoin%
\definecolor{currentfill}{rgb}{0.000000,0.000000,0.000000}%
\pgfsetfillcolor{currentfill}%
\pgfsetlinewidth{0.602250pt}%
\definecolor{currentstroke}{rgb}{0.000000,0.000000,0.000000}%
\pgfsetstrokecolor{currentstroke}%
\pgfsetdash{}{0pt}%
\pgfsys@defobject{currentmarker}{\pgfqpoint{0.000000in}{-0.027778in}}{\pgfqpoint{0.000000in}{0.000000in}}{%
\pgfpathmoveto{\pgfqpoint{0.000000in}{0.000000in}}%
\pgfpathlineto{\pgfqpoint{0.000000in}{-0.027778in}}%
\pgfusepath{stroke,fill}%
}%
\begin{pgfscope}%
\pgfsys@transformshift{3.249066in}{0.440000in}%
\pgfsys@useobject{currentmarker}{}%
\end{pgfscope}%
\end{pgfscope}%
\begin{pgfscope}%
\pgfsetbuttcap%
\pgfsetroundjoin%
\definecolor{currentfill}{rgb}{0.000000,0.000000,0.000000}%
\pgfsetfillcolor{currentfill}%
\pgfsetlinewidth{0.602250pt}%
\definecolor{currentstroke}{rgb}{0.000000,0.000000,0.000000}%
\pgfsetstrokecolor{currentstroke}%
\pgfsetdash{}{0pt}%
\pgfsys@defobject{currentmarker}{\pgfqpoint{0.000000in}{-0.027778in}}{\pgfqpoint{0.000000in}{0.000000in}}{%
\pgfpathmoveto{\pgfqpoint{0.000000in}{0.000000in}}%
\pgfpathlineto{\pgfqpoint{0.000000in}{-0.027778in}}%
\pgfusepath{stroke,fill}%
}%
\begin{pgfscope}%
\pgfsys@transformshift{3.436399in}{0.440000in}%
\pgfsys@useobject{currentmarker}{}%
\end{pgfscope}%
\end{pgfscope}%
\begin{pgfscope}%
\pgfsetbuttcap%
\pgfsetroundjoin%
\definecolor{currentfill}{rgb}{0.000000,0.000000,0.000000}%
\pgfsetfillcolor{currentfill}%
\pgfsetlinewidth{0.602250pt}%
\definecolor{currentstroke}{rgb}{0.000000,0.000000,0.000000}%
\pgfsetstrokecolor{currentstroke}%
\pgfsetdash{}{0pt}%
\pgfsys@defobject{currentmarker}{\pgfqpoint{0.000000in}{-0.027778in}}{\pgfqpoint{0.000000in}{0.000000in}}{%
\pgfpathmoveto{\pgfqpoint{0.000000in}{0.000000in}}%
\pgfpathlineto{\pgfqpoint{0.000000in}{-0.027778in}}%
\pgfusepath{stroke,fill}%
}%
\begin{pgfscope}%
\pgfsys@transformshift{3.569313in}{0.440000in}%
\pgfsys@useobject{currentmarker}{}%
\end{pgfscope}%
\end{pgfscope}%
\begin{pgfscope}%
\pgfsetbuttcap%
\pgfsetroundjoin%
\definecolor{currentfill}{rgb}{0.000000,0.000000,0.000000}%
\pgfsetfillcolor{currentfill}%
\pgfsetlinewidth{0.602250pt}%
\definecolor{currentstroke}{rgb}{0.000000,0.000000,0.000000}%
\pgfsetstrokecolor{currentstroke}%
\pgfsetdash{}{0pt}%
\pgfsys@defobject{currentmarker}{\pgfqpoint{0.000000in}{-0.027778in}}{\pgfqpoint{0.000000in}{0.000000in}}{%
\pgfpathmoveto{\pgfqpoint{0.000000in}{0.000000in}}%
\pgfpathlineto{\pgfqpoint{0.000000in}{-0.027778in}}%
\pgfusepath{stroke,fill}%
}%
\begin{pgfscope}%
\pgfsys@transformshift{3.672410in}{0.440000in}%
\pgfsys@useobject{currentmarker}{}%
\end{pgfscope}%
\end{pgfscope}%
\begin{pgfscope}%
\pgfsetbuttcap%
\pgfsetroundjoin%
\definecolor{currentfill}{rgb}{0.000000,0.000000,0.000000}%
\pgfsetfillcolor{currentfill}%
\pgfsetlinewidth{0.602250pt}%
\definecolor{currentstroke}{rgb}{0.000000,0.000000,0.000000}%
\pgfsetstrokecolor{currentstroke}%
\pgfsetdash{}{0pt}%
\pgfsys@defobject{currentmarker}{\pgfqpoint{0.000000in}{-0.027778in}}{\pgfqpoint{0.000000in}{0.000000in}}{%
\pgfpathmoveto{\pgfqpoint{0.000000in}{0.000000in}}%
\pgfpathlineto{\pgfqpoint{0.000000in}{-0.027778in}}%
\pgfusepath{stroke,fill}%
}%
\begin{pgfscope}%
\pgfsys@transformshift{3.756646in}{0.440000in}%
\pgfsys@useobject{currentmarker}{}%
\end{pgfscope}%
\end{pgfscope}%
\begin{pgfscope}%
\pgfsetbuttcap%
\pgfsetroundjoin%
\definecolor{currentfill}{rgb}{0.000000,0.000000,0.000000}%
\pgfsetfillcolor{currentfill}%
\pgfsetlinewidth{0.602250pt}%
\definecolor{currentstroke}{rgb}{0.000000,0.000000,0.000000}%
\pgfsetstrokecolor{currentstroke}%
\pgfsetdash{}{0pt}%
\pgfsys@defobject{currentmarker}{\pgfqpoint{0.000000in}{-0.027778in}}{\pgfqpoint{0.000000in}{0.000000in}}{%
\pgfpathmoveto{\pgfqpoint{0.000000in}{0.000000in}}%
\pgfpathlineto{\pgfqpoint{0.000000in}{-0.027778in}}%
\pgfusepath{stroke,fill}%
}%
\begin{pgfscope}%
\pgfsys@transformshift{3.827867in}{0.440000in}%
\pgfsys@useobject{currentmarker}{}%
\end{pgfscope}%
\end{pgfscope}%
\begin{pgfscope}%
\pgfsetbuttcap%
\pgfsetroundjoin%
\definecolor{currentfill}{rgb}{0.000000,0.000000,0.000000}%
\pgfsetfillcolor{currentfill}%
\pgfsetlinewidth{0.602250pt}%
\definecolor{currentstroke}{rgb}{0.000000,0.000000,0.000000}%
\pgfsetstrokecolor{currentstroke}%
\pgfsetdash{}{0pt}%
\pgfsys@defobject{currentmarker}{\pgfqpoint{0.000000in}{-0.027778in}}{\pgfqpoint{0.000000in}{0.000000in}}{%
\pgfpathmoveto{\pgfqpoint{0.000000in}{0.000000in}}%
\pgfpathlineto{\pgfqpoint{0.000000in}{-0.027778in}}%
\pgfusepath{stroke,fill}%
}%
\begin{pgfscope}%
\pgfsys@transformshift{3.889561in}{0.440000in}%
\pgfsys@useobject{currentmarker}{}%
\end{pgfscope}%
\end{pgfscope}%
\begin{pgfscope}%
\pgfsetbuttcap%
\pgfsetroundjoin%
\definecolor{currentfill}{rgb}{0.000000,0.000000,0.000000}%
\pgfsetfillcolor{currentfill}%
\pgfsetlinewidth{0.602250pt}%
\definecolor{currentstroke}{rgb}{0.000000,0.000000,0.000000}%
\pgfsetstrokecolor{currentstroke}%
\pgfsetdash{}{0pt}%
\pgfsys@defobject{currentmarker}{\pgfqpoint{0.000000in}{-0.027778in}}{\pgfqpoint{0.000000in}{0.000000in}}{%
\pgfpathmoveto{\pgfqpoint{0.000000in}{0.000000in}}%
\pgfpathlineto{\pgfqpoint{0.000000in}{-0.027778in}}%
\pgfusepath{stroke,fill}%
}%
\begin{pgfscope}%
\pgfsys@transformshift{3.943980in}{0.440000in}%
\pgfsys@useobject{currentmarker}{}%
\end{pgfscope}%
\end{pgfscope}%
\begin{pgfscope}%
\pgfsetbuttcap%
\pgfsetroundjoin%
\definecolor{currentfill}{rgb}{0.000000,0.000000,0.000000}%
\pgfsetfillcolor{currentfill}%
\pgfsetlinewidth{0.602250pt}%
\definecolor{currentstroke}{rgb}{0.000000,0.000000,0.000000}%
\pgfsetstrokecolor{currentstroke}%
\pgfsetdash{}{0pt}%
\pgfsys@defobject{currentmarker}{\pgfqpoint{0.000000in}{-0.027778in}}{\pgfqpoint{0.000000in}{0.000000in}}{%
\pgfpathmoveto{\pgfqpoint{0.000000in}{0.000000in}}%
\pgfpathlineto{\pgfqpoint{0.000000in}{-0.027778in}}%
\pgfusepath{stroke,fill}%
}%
\begin{pgfscope}%
\pgfsys@transformshift{4.312906in}{0.440000in}%
\pgfsys@useobject{currentmarker}{}%
\end{pgfscope}%
\end{pgfscope}%
\begin{pgfscope}%
\pgfsetbuttcap%
\pgfsetroundjoin%
\definecolor{currentfill}{rgb}{0.000000,0.000000,0.000000}%
\pgfsetfillcolor{currentfill}%
\pgfsetlinewidth{0.602250pt}%
\definecolor{currentstroke}{rgb}{0.000000,0.000000,0.000000}%
\pgfsetstrokecolor{currentstroke}%
\pgfsetdash{}{0pt}%
\pgfsys@defobject{currentmarker}{\pgfqpoint{0.000000in}{-0.027778in}}{\pgfqpoint{0.000000in}{0.000000in}}{%
\pgfpathmoveto{\pgfqpoint{0.000000in}{0.000000in}}%
\pgfpathlineto{\pgfqpoint{0.000000in}{-0.027778in}}%
\pgfusepath{stroke,fill}%
}%
\begin{pgfscope}%
\pgfsys@transformshift{4.500239in}{0.440000in}%
\pgfsys@useobject{currentmarker}{}%
\end{pgfscope}%
\end{pgfscope}%
\begin{pgfscope}%
\definecolor{textcolor}{rgb}{0.000000,0.000000,0.000000}%
\pgfsetstrokecolor{textcolor}%
\pgfsetfillcolor{textcolor}%
\pgftext[x=2.562500in,y=0.152809in,,top]{\color{textcolor}\sffamily\fontsize{10.000000}{12.000000}\selectfont \(\displaystyle N_\mathrm{QMC}\)}%
\end{pgfscope}%
\begin{pgfscope}%
\pgfsetbuttcap%
\pgfsetroundjoin%
\definecolor{currentfill}{rgb}{0.000000,0.000000,0.000000}%
\pgfsetfillcolor{currentfill}%
\pgfsetlinewidth{0.803000pt}%
\definecolor{currentstroke}{rgb}{0.000000,0.000000,0.000000}%
\pgfsetstrokecolor{currentstroke}%
\pgfsetdash{}{0pt}%
\pgfsys@defobject{currentmarker}{\pgfqpoint{-0.048611in}{0.000000in}}{\pgfqpoint{0.000000in}{0.000000in}}{%
\pgfpathmoveto{\pgfqpoint{0.000000in}{0.000000in}}%
\pgfpathlineto{\pgfqpoint{-0.048611in}{0.000000in}}%
\pgfusepath{stroke,fill}%
}%
\begin{pgfscope}%
\pgfsys@transformshift{0.625000in}{1.242709in}%
\pgfsys@useobject{currentmarker}{}%
\end{pgfscope}%
\end{pgfscope}%
\begin{pgfscope}%
\definecolor{textcolor}{rgb}{0.000000,0.000000,0.000000}%
\pgfsetstrokecolor{textcolor}%
\pgfsetfillcolor{textcolor}%
\pgftext[x=0.239775in,y=1.189948in,left,base]{\color{textcolor}\sffamily\fontsize{10.000000}{12.000000}\selectfont \(\displaystyle 10^{-5}\)}%
\end{pgfscope}%
\begin{pgfscope}%
\pgfsetbuttcap%
\pgfsetroundjoin%
\definecolor{currentfill}{rgb}{0.000000,0.000000,0.000000}%
\pgfsetfillcolor{currentfill}%
\pgfsetlinewidth{0.803000pt}%
\definecolor{currentstroke}{rgb}{0.000000,0.000000,0.000000}%
\pgfsetstrokecolor{currentstroke}%
\pgfsetdash{}{0pt}%
\pgfsys@defobject{currentmarker}{\pgfqpoint{-0.048611in}{0.000000in}}{\pgfqpoint{0.000000in}{0.000000in}}{%
\pgfpathmoveto{\pgfqpoint{0.000000in}{0.000000in}}%
\pgfpathlineto{\pgfqpoint{-0.048611in}{0.000000in}}%
\pgfusepath{stroke,fill}%
}%
\begin{pgfscope}%
\pgfsys@transformshift{0.625000in}{2.150285in}%
\pgfsys@useobject{currentmarker}{}%
\end{pgfscope}%
\end{pgfscope}%
\begin{pgfscope}%
\definecolor{textcolor}{rgb}{0.000000,0.000000,0.000000}%
\pgfsetstrokecolor{textcolor}%
\pgfsetfillcolor{textcolor}%
\pgftext[x=0.239775in,y=2.097523in,left,base]{\color{textcolor}\sffamily\fontsize{10.000000}{12.000000}\selectfont \(\displaystyle 10^{-4}\)}%
\end{pgfscope}%
\begin{pgfscope}%
\pgfsetbuttcap%
\pgfsetroundjoin%
\definecolor{currentfill}{rgb}{0.000000,0.000000,0.000000}%
\pgfsetfillcolor{currentfill}%
\pgfsetlinewidth{0.803000pt}%
\definecolor{currentstroke}{rgb}{0.000000,0.000000,0.000000}%
\pgfsetstrokecolor{currentstroke}%
\pgfsetdash{}{0pt}%
\pgfsys@defobject{currentmarker}{\pgfqpoint{-0.048611in}{0.000000in}}{\pgfqpoint{0.000000in}{0.000000in}}{%
\pgfpathmoveto{\pgfqpoint{0.000000in}{0.000000in}}%
\pgfpathlineto{\pgfqpoint{-0.048611in}{0.000000in}}%
\pgfusepath{stroke,fill}%
}%
\begin{pgfscope}%
\pgfsys@transformshift{0.625000in}{3.057860in}%
\pgfsys@useobject{currentmarker}{}%
\end{pgfscope}%
\end{pgfscope}%
\begin{pgfscope}%
\definecolor{textcolor}{rgb}{0.000000,0.000000,0.000000}%
\pgfsetstrokecolor{textcolor}%
\pgfsetfillcolor{textcolor}%
\pgftext[x=0.239775in,y=3.005099in,left,base]{\color{textcolor}\sffamily\fontsize{10.000000}{12.000000}\selectfont \(\displaystyle 10^{-3}\)}%
\end{pgfscope}%
\begin{pgfscope}%
\pgfsetbuttcap%
\pgfsetroundjoin%
\definecolor{currentfill}{rgb}{0.000000,0.000000,0.000000}%
\pgfsetfillcolor{currentfill}%
\pgfsetlinewidth{0.602250pt}%
\definecolor{currentstroke}{rgb}{0.000000,0.000000,0.000000}%
\pgfsetstrokecolor{currentstroke}%
\pgfsetdash{}{0pt}%
\pgfsys@defobject{currentmarker}{\pgfqpoint{-0.027778in}{0.000000in}}{\pgfqpoint{0.000000in}{0.000000in}}{%
\pgfpathmoveto{\pgfqpoint{0.000000in}{0.000000in}}%
\pgfpathlineto{\pgfqpoint{-0.027778in}{0.000000in}}%
\pgfusepath{stroke,fill}%
}%
\begin{pgfscope}%
\pgfsys@transformshift{0.625000in}{0.608341in}%
\pgfsys@useobject{currentmarker}{}%
\end{pgfscope}%
\end{pgfscope}%
\begin{pgfscope}%
\pgfsetbuttcap%
\pgfsetroundjoin%
\definecolor{currentfill}{rgb}{0.000000,0.000000,0.000000}%
\pgfsetfillcolor{currentfill}%
\pgfsetlinewidth{0.602250pt}%
\definecolor{currentstroke}{rgb}{0.000000,0.000000,0.000000}%
\pgfsetstrokecolor{currentstroke}%
\pgfsetdash{}{0pt}%
\pgfsys@defobject{currentmarker}{\pgfqpoint{-0.027778in}{0.000000in}}{\pgfqpoint{0.000000in}{0.000000in}}{%
\pgfpathmoveto{\pgfqpoint{0.000000in}{0.000000in}}%
\pgfpathlineto{\pgfqpoint{-0.027778in}{0.000000in}}%
\pgfusepath{stroke,fill}%
}%
\begin{pgfscope}%
\pgfsys@transformshift{0.625000in}{0.768157in}%
\pgfsys@useobject{currentmarker}{}%
\end{pgfscope}%
\end{pgfscope}%
\begin{pgfscope}%
\pgfsetbuttcap%
\pgfsetroundjoin%
\definecolor{currentfill}{rgb}{0.000000,0.000000,0.000000}%
\pgfsetfillcolor{currentfill}%
\pgfsetlinewidth{0.602250pt}%
\definecolor{currentstroke}{rgb}{0.000000,0.000000,0.000000}%
\pgfsetstrokecolor{currentstroke}%
\pgfsetdash{}{0pt}%
\pgfsys@defobject{currentmarker}{\pgfqpoint{-0.027778in}{0.000000in}}{\pgfqpoint{0.000000in}{0.000000in}}{%
\pgfpathmoveto{\pgfqpoint{0.000000in}{0.000000in}}%
\pgfpathlineto{\pgfqpoint{-0.027778in}{0.000000in}}%
\pgfusepath{stroke,fill}%
}%
\begin{pgfscope}%
\pgfsys@transformshift{0.625000in}{0.881549in}%
\pgfsys@useobject{currentmarker}{}%
\end{pgfscope}%
\end{pgfscope}%
\begin{pgfscope}%
\pgfsetbuttcap%
\pgfsetroundjoin%
\definecolor{currentfill}{rgb}{0.000000,0.000000,0.000000}%
\pgfsetfillcolor{currentfill}%
\pgfsetlinewidth{0.602250pt}%
\definecolor{currentstroke}{rgb}{0.000000,0.000000,0.000000}%
\pgfsetstrokecolor{currentstroke}%
\pgfsetdash{}{0pt}%
\pgfsys@defobject{currentmarker}{\pgfqpoint{-0.027778in}{0.000000in}}{\pgfqpoint{0.000000in}{0.000000in}}{%
\pgfpathmoveto{\pgfqpoint{0.000000in}{0.000000in}}%
\pgfpathlineto{\pgfqpoint{-0.027778in}{0.000000in}}%
\pgfusepath{stroke,fill}%
}%
\begin{pgfscope}%
\pgfsys@transformshift{0.625000in}{0.969502in}%
\pgfsys@useobject{currentmarker}{}%
\end{pgfscope}%
\end{pgfscope}%
\begin{pgfscope}%
\pgfsetbuttcap%
\pgfsetroundjoin%
\definecolor{currentfill}{rgb}{0.000000,0.000000,0.000000}%
\pgfsetfillcolor{currentfill}%
\pgfsetlinewidth{0.602250pt}%
\definecolor{currentstroke}{rgb}{0.000000,0.000000,0.000000}%
\pgfsetstrokecolor{currentstroke}%
\pgfsetdash{}{0pt}%
\pgfsys@defobject{currentmarker}{\pgfqpoint{-0.027778in}{0.000000in}}{\pgfqpoint{0.000000in}{0.000000in}}{%
\pgfpathmoveto{\pgfqpoint{0.000000in}{0.000000in}}%
\pgfpathlineto{\pgfqpoint{-0.027778in}{0.000000in}}%
\pgfusepath{stroke,fill}%
}%
\begin{pgfscope}%
\pgfsys@transformshift{0.625000in}{1.041365in}%
\pgfsys@useobject{currentmarker}{}%
\end{pgfscope}%
\end{pgfscope}%
\begin{pgfscope}%
\pgfsetbuttcap%
\pgfsetroundjoin%
\definecolor{currentfill}{rgb}{0.000000,0.000000,0.000000}%
\pgfsetfillcolor{currentfill}%
\pgfsetlinewidth{0.602250pt}%
\definecolor{currentstroke}{rgb}{0.000000,0.000000,0.000000}%
\pgfsetstrokecolor{currentstroke}%
\pgfsetdash{}{0pt}%
\pgfsys@defobject{currentmarker}{\pgfqpoint{-0.027778in}{0.000000in}}{\pgfqpoint{0.000000in}{0.000000in}}{%
\pgfpathmoveto{\pgfqpoint{0.000000in}{0.000000in}}%
\pgfpathlineto{\pgfqpoint{-0.027778in}{0.000000in}}%
\pgfusepath{stroke,fill}%
}%
\begin{pgfscope}%
\pgfsys@transformshift{0.625000in}{1.102124in}%
\pgfsys@useobject{currentmarker}{}%
\end{pgfscope}%
\end{pgfscope}%
\begin{pgfscope}%
\pgfsetbuttcap%
\pgfsetroundjoin%
\definecolor{currentfill}{rgb}{0.000000,0.000000,0.000000}%
\pgfsetfillcolor{currentfill}%
\pgfsetlinewidth{0.602250pt}%
\definecolor{currentstroke}{rgb}{0.000000,0.000000,0.000000}%
\pgfsetstrokecolor{currentstroke}%
\pgfsetdash{}{0pt}%
\pgfsys@defobject{currentmarker}{\pgfqpoint{-0.027778in}{0.000000in}}{\pgfqpoint{0.000000in}{0.000000in}}{%
\pgfpathmoveto{\pgfqpoint{0.000000in}{0.000000in}}%
\pgfpathlineto{\pgfqpoint{-0.027778in}{0.000000in}}%
\pgfusepath{stroke,fill}%
}%
\begin{pgfscope}%
\pgfsys@transformshift{0.625000in}{1.154756in}%
\pgfsys@useobject{currentmarker}{}%
\end{pgfscope}%
\end{pgfscope}%
\begin{pgfscope}%
\pgfsetbuttcap%
\pgfsetroundjoin%
\definecolor{currentfill}{rgb}{0.000000,0.000000,0.000000}%
\pgfsetfillcolor{currentfill}%
\pgfsetlinewidth{0.602250pt}%
\definecolor{currentstroke}{rgb}{0.000000,0.000000,0.000000}%
\pgfsetstrokecolor{currentstroke}%
\pgfsetdash{}{0pt}%
\pgfsys@defobject{currentmarker}{\pgfqpoint{-0.027778in}{0.000000in}}{\pgfqpoint{0.000000in}{0.000000in}}{%
\pgfpathmoveto{\pgfqpoint{0.000000in}{0.000000in}}%
\pgfpathlineto{\pgfqpoint{-0.027778in}{0.000000in}}%
\pgfusepath{stroke,fill}%
}%
\begin{pgfscope}%
\pgfsys@transformshift{0.625000in}{1.201181in}%
\pgfsys@useobject{currentmarker}{}%
\end{pgfscope}%
\end{pgfscope}%
\begin{pgfscope}%
\pgfsetbuttcap%
\pgfsetroundjoin%
\definecolor{currentfill}{rgb}{0.000000,0.000000,0.000000}%
\pgfsetfillcolor{currentfill}%
\pgfsetlinewidth{0.602250pt}%
\definecolor{currentstroke}{rgb}{0.000000,0.000000,0.000000}%
\pgfsetstrokecolor{currentstroke}%
\pgfsetdash{}{0pt}%
\pgfsys@defobject{currentmarker}{\pgfqpoint{-0.027778in}{0.000000in}}{\pgfqpoint{0.000000in}{0.000000in}}{%
\pgfpathmoveto{\pgfqpoint{0.000000in}{0.000000in}}%
\pgfpathlineto{\pgfqpoint{-0.027778in}{0.000000in}}%
\pgfusepath{stroke,fill}%
}%
\begin{pgfscope}%
\pgfsys@transformshift{0.625000in}{1.515917in}%
\pgfsys@useobject{currentmarker}{}%
\end{pgfscope}%
\end{pgfscope}%
\begin{pgfscope}%
\pgfsetbuttcap%
\pgfsetroundjoin%
\definecolor{currentfill}{rgb}{0.000000,0.000000,0.000000}%
\pgfsetfillcolor{currentfill}%
\pgfsetlinewidth{0.602250pt}%
\definecolor{currentstroke}{rgb}{0.000000,0.000000,0.000000}%
\pgfsetstrokecolor{currentstroke}%
\pgfsetdash{}{0pt}%
\pgfsys@defobject{currentmarker}{\pgfqpoint{-0.027778in}{0.000000in}}{\pgfqpoint{0.000000in}{0.000000in}}{%
\pgfpathmoveto{\pgfqpoint{0.000000in}{0.000000in}}%
\pgfpathlineto{\pgfqpoint{-0.027778in}{0.000000in}}%
\pgfusepath{stroke,fill}%
}%
\begin{pgfscope}%
\pgfsys@transformshift{0.625000in}{1.675733in}%
\pgfsys@useobject{currentmarker}{}%
\end{pgfscope}%
\end{pgfscope}%
\begin{pgfscope}%
\pgfsetbuttcap%
\pgfsetroundjoin%
\definecolor{currentfill}{rgb}{0.000000,0.000000,0.000000}%
\pgfsetfillcolor{currentfill}%
\pgfsetlinewidth{0.602250pt}%
\definecolor{currentstroke}{rgb}{0.000000,0.000000,0.000000}%
\pgfsetstrokecolor{currentstroke}%
\pgfsetdash{}{0pt}%
\pgfsys@defobject{currentmarker}{\pgfqpoint{-0.027778in}{0.000000in}}{\pgfqpoint{0.000000in}{0.000000in}}{%
\pgfpathmoveto{\pgfqpoint{0.000000in}{0.000000in}}%
\pgfpathlineto{\pgfqpoint{-0.027778in}{0.000000in}}%
\pgfusepath{stroke,fill}%
}%
\begin{pgfscope}%
\pgfsys@transformshift{0.625000in}{1.789124in}%
\pgfsys@useobject{currentmarker}{}%
\end{pgfscope}%
\end{pgfscope}%
\begin{pgfscope}%
\pgfsetbuttcap%
\pgfsetroundjoin%
\definecolor{currentfill}{rgb}{0.000000,0.000000,0.000000}%
\pgfsetfillcolor{currentfill}%
\pgfsetlinewidth{0.602250pt}%
\definecolor{currentstroke}{rgb}{0.000000,0.000000,0.000000}%
\pgfsetstrokecolor{currentstroke}%
\pgfsetdash{}{0pt}%
\pgfsys@defobject{currentmarker}{\pgfqpoint{-0.027778in}{0.000000in}}{\pgfqpoint{0.000000in}{0.000000in}}{%
\pgfpathmoveto{\pgfqpoint{0.000000in}{0.000000in}}%
\pgfpathlineto{\pgfqpoint{-0.027778in}{0.000000in}}%
\pgfusepath{stroke,fill}%
}%
\begin{pgfscope}%
\pgfsys@transformshift{0.625000in}{1.877077in}%
\pgfsys@useobject{currentmarker}{}%
\end{pgfscope}%
\end{pgfscope}%
\begin{pgfscope}%
\pgfsetbuttcap%
\pgfsetroundjoin%
\definecolor{currentfill}{rgb}{0.000000,0.000000,0.000000}%
\pgfsetfillcolor{currentfill}%
\pgfsetlinewidth{0.602250pt}%
\definecolor{currentstroke}{rgb}{0.000000,0.000000,0.000000}%
\pgfsetstrokecolor{currentstroke}%
\pgfsetdash{}{0pt}%
\pgfsys@defobject{currentmarker}{\pgfqpoint{-0.027778in}{0.000000in}}{\pgfqpoint{0.000000in}{0.000000in}}{%
\pgfpathmoveto{\pgfqpoint{0.000000in}{0.000000in}}%
\pgfpathlineto{\pgfqpoint{-0.027778in}{0.000000in}}%
\pgfusepath{stroke,fill}%
}%
\begin{pgfscope}%
\pgfsys@transformshift{0.625000in}{1.948940in}%
\pgfsys@useobject{currentmarker}{}%
\end{pgfscope}%
\end{pgfscope}%
\begin{pgfscope}%
\pgfsetbuttcap%
\pgfsetroundjoin%
\definecolor{currentfill}{rgb}{0.000000,0.000000,0.000000}%
\pgfsetfillcolor{currentfill}%
\pgfsetlinewidth{0.602250pt}%
\definecolor{currentstroke}{rgb}{0.000000,0.000000,0.000000}%
\pgfsetstrokecolor{currentstroke}%
\pgfsetdash{}{0pt}%
\pgfsys@defobject{currentmarker}{\pgfqpoint{-0.027778in}{0.000000in}}{\pgfqpoint{0.000000in}{0.000000in}}{%
\pgfpathmoveto{\pgfqpoint{0.000000in}{0.000000in}}%
\pgfpathlineto{\pgfqpoint{-0.027778in}{0.000000in}}%
\pgfusepath{stroke,fill}%
}%
\begin{pgfscope}%
\pgfsys@transformshift{0.625000in}{2.009700in}%
\pgfsys@useobject{currentmarker}{}%
\end{pgfscope}%
\end{pgfscope}%
\begin{pgfscope}%
\pgfsetbuttcap%
\pgfsetroundjoin%
\definecolor{currentfill}{rgb}{0.000000,0.000000,0.000000}%
\pgfsetfillcolor{currentfill}%
\pgfsetlinewidth{0.602250pt}%
\definecolor{currentstroke}{rgb}{0.000000,0.000000,0.000000}%
\pgfsetstrokecolor{currentstroke}%
\pgfsetdash{}{0pt}%
\pgfsys@defobject{currentmarker}{\pgfqpoint{-0.027778in}{0.000000in}}{\pgfqpoint{0.000000in}{0.000000in}}{%
\pgfpathmoveto{\pgfqpoint{0.000000in}{0.000000in}}%
\pgfpathlineto{\pgfqpoint{-0.027778in}{0.000000in}}%
\pgfusepath{stroke,fill}%
}%
\begin{pgfscope}%
\pgfsys@transformshift{0.625000in}{2.062332in}%
\pgfsys@useobject{currentmarker}{}%
\end{pgfscope}%
\end{pgfscope}%
\begin{pgfscope}%
\pgfsetbuttcap%
\pgfsetroundjoin%
\definecolor{currentfill}{rgb}{0.000000,0.000000,0.000000}%
\pgfsetfillcolor{currentfill}%
\pgfsetlinewidth{0.602250pt}%
\definecolor{currentstroke}{rgb}{0.000000,0.000000,0.000000}%
\pgfsetstrokecolor{currentstroke}%
\pgfsetdash{}{0pt}%
\pgfsys@defobject{currentmarker}{\pgfqpoint{-0.027778in}{0.000000in}}{\pgfqpoint{0.000000in}{0.000000in}}{%
\pgfpathmoveto{\pgfqpoint{0.000000in}{0.000000in}}%
\pgfpathlineto{\pgfqpoint{-0.027778in}{0.000000in}}%
\pgfusepath{stroke,fill}%
}%
\begin{pgfscope}%
\pgfsys@transformshift{0.625000in}{2.108756in}%
\pgfsys@useobject{currentmarker}{}%
\end{pgfscope}%
\end{pgfscope}%
\begin{pgfscope}%
\pgfsetbuttcap%
\pgfsetroundjoin%
\definecolor{currentfill}{rgb}{0.000000,0.000000,0.000000}%
\pgfsetfillcolor{currentfill}%
\pgfsetlinewidth{0.602250pt}%
\definecolor{currentstroke}{rgb}{0.000000,0.000000,0.000000}%
\pgfsetstrokecolor{currentstroke}%
\pgfsetdash{}{0pt}%
\pgfsys@defobject{currentmarker}{\pgfqpoint{-0.027778in}{0.000000in}}{\pgfqpoint{0.000000in}{0.000000in}}{%
\pgfpathmoveto{\pgfqpoint{0.000000in}{0.000000in}}%
\pgfpathlineto{\pgfqpoint{-0.027778in}{0.000000in}}%
\pgfusepath{stroke,fill}%
}%
\begin{pgfscope}%
\pgfsys@transformshift{0.625000in}{2.423492in}%
\pgfsys@useobject{currentmarker}{}%
\end{pgfscope}%
\end{pgfscope}%
\begin{pgfscope}%
\pgfsetbuttcap%
\pgfsetroundjoin%
\definecolor{currentfill}{rgb}{0.000000,0.000000,0.000000}%
\pgfsetfillcolor{currentfill}%
\pgfsetlinewidth{0.602250pt}%
\definecolor{currentstroke}{rgb}{0.000000,0.000000,0.000000}%
\pgfsetstrokecolor{currentstroke}%
\pgfsetdash{}{0pt}%
\pgfsys@defobject{currentmarker}{\pgfqpoint{-0.027778in}{0.000000in}}{\pgfqpoint{0.000000in}{0.000000in}}{%
\pgfpathmoveto{\pgfqpoint{0.000000in}{0.000000in}}%
\pgfpathlineto{\pgfqpoint{-0.027778in}{0.000000in}}%
\pgfusepath{stroke,fill}%
}%
\begin{pgfscope}%
\pgfsys@transformshift{0.625000in}{2.583308in}%
\pgfsys@useobject{currentmarker}{}%
\end{pgfscope}%
\end{pgfscope}%
\begin{pgfscope}%
\pgfsetbuttcap%
\pgfsetroundjoin%
\definecolor{currentfill}{rgb}{0.000000,0.000000,0.000000}%
\pgfsetfillcolor{currentfill}%
\pgfsetlinewidth{0.602250pt}%
\definecolor{currentstroke}{rgb}{0.000000,0.000000,0.000000}%
\pgfsetstrokecolor{currentstroke}%
\pgfsetdash{}{0pt}%
\pgfsys@defobject{currentmarker}{\pgfqpoint{-0.027778in}{0.000000in}}{\pgfqpoint{0.000000in}{0.000000in}}{%
\pgfpathmoveto{\pgfqpoint{0.000000in}{0.000000in}}%
\pgfpathlineto{\pgfqpoint{-0.027778in}{0.000000in}}%
\pgfusepath{stroke,fill}%
}%
\begin{pgfscope}%
\pgfsys@transformshift{0.625000in}{2.696700in}%
\pgfsys@useobject{currentmarker}{}%
\end{pgfscope}%
\end{pgfscope}%
\begin{pgfscope}%
\pgfsetbuttcap%
\pgfsetroundjoin%
\definecolor{currentfill}{rgb}{0.000000,0.000000,0.000000}%
\pgfsetfillcolor{currentfill}%
\pgfsetlinewidth{0.602250pt}%
\definecolor{currentstroke}{rgb}{0.000000,0.000000,0.000000}%
\pgfsetstrokecolor{currentstroke}%
\pgfsetdash{}{0pt}%
\pgfsys@defobject{currentmarker}{\pgfqpoint{-0.027778in}{0.000000in}}{\pgfqpoint{0.000000in}{0.000000in}}{%
\pgfpathmoveto{\pgfqpoint{0.000000in}{0.000000in}}%
\pgfpathlineto{\pgfqpoint{-0.027778in}{0.000000in}}%
\pgfusepath{stroke,fill}%
}%
\begin{pgfscope}%
\pgfsys@transformshift{0.625000in}{2.784653in}%
\pgfsys@useobject{currentmarker}{}%
\end{pgfscope}%
\end{pgfscope}%
\begin{pgfscope}%
\pgfsetbuttcap%
\pgfsetroundjoin%
\definecolor{currentfill}{rgb}{0.000000,0.000000,0.000000}%
\pgfsetfillcolor{currentfill}%
\pgfsetlinewidth{0.602250pt}%
\definecolor{currentstroke}{rgb}{0.000000,0.000000,0.000000}%
\pgfsetstrokecolor{currentstroke}%
\pgfsetdash{}{0pt}%
\pgfsys@defobject{currentmarker}{\pgfqpoint{-0.027778in}{0.000000in}}{\pgfqpoint{0.000000in}{0.000000in}}{%
\pgfpathmoveto{\pgfqpoint{0.000000in}{0.000000in}}%
\pgfpathlineto{\pgfqpoint{-0.027778in}{0.000000in}}%
\pgfusepath{stroke,fill}%
}%
\begin{pgfscope}%
\pgfsys@transformshift{0.625000in}{2.856516in}%
\pgfsys@useobject{currentmarker}{}%
\end{pgfscope}%
\end{pgfscope}%
\begin{pgfscope}%
\pgfsetbuttcap%
\pgfsetroundjoin%
\definecolor{currentfill}{rgb}{0.000000,0.000000,0.000000}%
\pgfsetfillcolor{currentfill}%
\pgfsetlinewidth{0.602250pt}%
\definecolor{currentstroke}{rgb}{0.000000,0.000000,0.000000}%
\pgfsetstrokecolor{currentstroke}%
\pgfsetdash{}{0pt}%
\pgfsys@defobject{currentmarker}{\pgfqpoint{-0.027778in}{0.000000in}}{\pgfqpoint{0.000000in}{0.000000in}}{%
\pgfpathmoveto{\pgfqpoint{0.000000in}{0.000000in}}%
\pgfpathlineto{\pgfqpoint{-0.027778in}{0.000000in}}%
\pgfusepath{stroke,fill}%
}%
\begin{pgfscope}%
\pgfsys@transformshift{0.625000in}{2.917275in}%
\pgfsys@useobject{currentmarker}{}%
\end{pgfscope}%
\end{pgfscope}%
\begin{pgfscope}%
\pgfsetbuttcap%
\pgfsetroundjoin%
\definecolor{currentfill}{rgb}{0.000000,0.000000,0.000000}%
\pgfsetfillcolor{currentfill}%
\pgfsetlinewidth{0.602250pt}%
\definecolor{currentstroke}{rgb}{0.000000,0.000000,0.000000}%
\pgfsetstrokecolor{currentstroke}%
\pgfsetdash{}{0pt}%
\pgfsys@defobject{currentmarker}{\pgfqpoint{-0.027778in}{0.000000in}}{\pgfqpoint{0.000000in}{0.000000in}}{%
\pgfpathmoveto{\pgfqpoint{0.000000in}{0.000000in}}%
\pgfpathlineto{\pgfqpoint{-0.027778in}{0.000000in}}%
\pgfusepath{stroke,fill}%
}%
\begin{pgfscope}%
\pgfsys@transformshift{0.625000in}{2.969907in}%
\pgfsys@useobject{currentmarker}{}%
\end{pgfscope}%
\end{pgfscope}%
\begin{pgfscope}%
\pgfsetbuttcap%
\pgfsetroundjoin%
\definecolor{currentfill}{rgb}{0.000000,0.000000,0.000000}%
\pgfsetfillcolor{currentfill}%
\pgfsetlinewidth{0.602250pt}%
\definecolor{currentstroke}{rgb}{0.000000,0.000000,0.000000}%
\pgfsetstrokecolor{currentstroke}%
\pgfsetdash{}{0pt}%
\pgfsys@defobject{currentmarker}{\pgfqpoint{-0.027778in}{0.000000in}}{\pgfqpoint{0.000000in}{0.000000in}}{%
\pgfpathmoveto{\pgfqpoint{0.000000in}{0.000000in}}%
\pgfpathlineto{\pgfqpoint{-0.027778in}{0.000000in}}%
\pgfusepath{stroke,fill}%
}%
\begin{pgfscope}%
\pgfsys@transformshift{0.625000in}{3.016332in}%
\pgfsys@useobject{currentmarker}{}%
\end{pgfscope}%
\end{pgfscope}%
\begin{pgfscope}%
\pgfsetbuttcap%
\pgfsetroundjoin%
\definecolor{currentfill}{rgb}{0.000000,0.000000,0.000000}%
\pgfsetfillcolor{currentfill}%
\pgfsetlinewidth{0.602250pt}%
\definecolor{currentstroke}{rgb}{0.000000,0.000000,0.000000}%
\pgfsetstrokecolor{currentstroke}%
\pgfsetdash{}{0pt}%
\pgfsys@defobject{currentmarker}{\pgfqpoint{-0.027778in}{0.000000in}}{\pgfqpoint{0.000000in}{0.000000in}}{%
\pgfpathmoveto{\pgfqpoint{0.000000in}{0.000000in}}%
\pgfpathlineto{\pgfqpoint{-0.027778in}{0.000000in}}%
\pgfusepath{stroke,fill}%
}%
\begin{pgfscope}%
\pgfsys@transformshift{0.625000in}{3.331068in}%
\pgfsys@useobject{currentmarker}{}%
\end{pgfscope}%
\end{pgfscope}%
\begin{pgfscope}%
\pgfsetbuttcap%
\pgfsetroundjoin%
\definecolor{currentfill}{rgb}{0.000000,0.000000,0.000000}%
\pgfsetfillcolor{currentfill}%
\pgfsetlinewidth{0.602250pt}%
\definecolor{currentstroke}{rgb}{0.000000,0.000000,0.000000}%
\pgfsetstrokecolor{currentstroke}%
\pgfsetdash{}{0pt}%
\pgfsys@defobject{currentmarker}{\pgfqpoint{-0.027778in}{0.000000in}}{\pgfqpoint{0.000000in}{0.000000in}}{%
\pgfpathmoveto{\pgfqpoint{0.000000in}{0.000000in}}%
\pgfpathlineto{\pgfqpoint{-0.027778in}{0.000000in}}%
\pgfusepath{stroke,fill}%
}%
\begin{pgfscope}%
\pgfsys@transformshift{0.625000in}{3.490884in}%
\pgfsys@useobject{currentmarker}{}%
\end{pgfscope}%
\end{pgfscope}%
\begin{pgfscope}%
\definecolor{textcolor}{rgb}{0.000000,0.000000,0.000000}%
\pgfsetstrokecolor{textcolor}%
\pgfsetfillcolor{textcolor}%
\pgftext[x=0.184220in,y=1.980000in,,bottom,rotate=90.000000]{\color{textcolor}\sffamily\fontsize{10.000000}{12.000000}\selectfont QMC Error}%
\end{pgfscope}%
\begin{pgfscope}%
\pgfpathrectangle{\pgfqpoint{0.625000in}{0.440000in}}{\pgfqpoint{3.875000in}{3.080000in}}%
\pgfusepath{clip}%
\pgfsetbuttcap%
\pgfsetroundjoin%
\definecolor{currentfill}{rgb}{0.000000,0.000000,0.000000}%
\pgfsetfillcolor{currentfill}%
\pgfsetlinewidth{1.003750pt}%
\definecolor{currentstroke}{rgb}{0.000000,0.000000,0.000000}%
\pgfsetstrokecolor{currentstroke}%
\pgfsetdash{}{0pt}%
\pgfsys@defobject{currentmarker}{\pgfqpoint{-0.041667in}{-0.041667in}}{\pgfqpoint{0.041667in}{0.041667in}}{%
\pgfpathmoveto{\pgfqpoint{0.000000in}{-0.041667in}}%
\pgfpathcurveto{\pgfqpoint{0.011050in}{-0.041667in}}{\pgfqpoint{0.021649in}{-0.037276in}}{\pgfqpoint{0.029463in}{-0.029463in}}%
\pgfpathcurveto{\pgfqpoint{0.037276in}{-0.021649in}}{\pgfqpoint{0.041667in}{-0.011050in}}{\pgfqpoint{0.041667in}{0.000000in}}%
\pgfpathcurveto{\pgfqpoint{0.041667in}{0.011050in}}{\pgfqpoint{0.037276in}{0.021649in}}{\pgfqpoint{0.029463in}{0.029463in}}%
\pgfpathcurveto{\pgfqpoint{0.021649in}{0.037276in}}{\pgfqpoint{0.011050in}{0.041667in}}{\pgfqpoint{0.000000in}{0.041667in}}%
\pgfpathcurveto{\pgfqpoint{-0.011050in}{0.041667in}}{\pgfqpoint{-0.021649in}{0.037276in}}{\pgfqpoint{-0.029463in}{0.029463in}}%
\pgfpathcurveto{\pgfqpoint{-0.037276in}{0.021649in}}{\pgfqpoint{-0.041667in}{0.011050in}}{\pgfqpoint{-0.041667in}{0.000000in}}%
\pgfpathcurveto{\pgfqpoint{-0.041667in}{-0.011050in}}{\pgfqpoint{-0.037276in}{-0.021649in}}{\pgfqpoint{-0.029463in}{-0.029463in}}%
\pgfpathcurveto{\pgfqpoint{-0.021649in}{-0.037276in}}{\pgfqpoint{-0.011050in}{-0.041667in}}{\pgfqpoint{0.000000in}{-0.041667in}}%
\pgfpathclose%
\pgfusepath{stroke,fill}%
}%
\begin{pgfscope}%
\pgfsys@transformshift{0.801136in}{3.306470in}%
\pgfsys@useobject{currentmarker}{}%
\end{pgfscope}%
\begin{pgfscope}%
\pgfsys@transformshift{1.121384in}{3.121810in}%
\pgfsys@useobject{currentmarker}{}%
\end{pgfscope}%
\begin{pgfscope}%
\pgfsys@transformshift{1.441632in}{2.837426in}%
\pgfsys@useobject{currentmarker}{}%
\end{pgfscope}%
\begin{pgfscope}%
\pgfsys@transformshift{1.761880in}{2.526660in}%
\pgfsys@useobject{currentmarker}{}%
\end{pgfscope}%
\begin{pgfscope}%
\pgfsys@transformshift{2.082128in}{2.403147in}%
\pgfsys@useobject{currentmarker}{}%
\end{pgfscope}%
\begin{pgfscope}%
\pgfsys@transformshift{2.402376in}{2.273604in}%
\pgfsys@useobject{currentmarker}{}%
\end{pgfscope}%
\begin{pgfscope}%
\pgfsys@transformshift{2.722624in}{2.160225in}%
\pgfsys@useobject{currentmarker}{}%
\end{pgfscope}%
\begin{pgfscope}%
\pgfsys@transformshift{3.042872in}{1.488191in}%
\pgfsys@useobject{currentmarker}{}%
\end{pgfscope}%
\begin{pgfscope}%
\pgfsys@transformshift{3.363120in}{1.335769in}%
\pgfsys@useobject{currentmarker}{}%
\end{pgfscope}%
\begin{pgfscope}%
\pgfsys@transformshift{3.683368in}{1.215978in}%
\pgfsys@useobject{currentmarker}{}%
\end{pgfscope}%
\begin{pgfscope}%
\pgfsys@transformshift{4.003616in}{0.693682in}%
\pgfsys@useobject{currentmarker}{}%
\end{pgfscope}%
\begin{pgfscope}%
\pgfsys@transformshift{4.323864in}{0.580000in}%
\pgfsys@useobject{currentmarker}{}%
\end{pgfscope}%
\end{pgfscope}%
\begin{pgfscope}%
\pgfpathrectangle{\pgfqpoint{0.625000in}{0.440000in}}{\pgfqpoint{3.875000in}{3.080000in}}%
\pgfusepath{clip}%
\pgfsetbuttcap%
\pgfsetroundjoin%
\pgfsetlinewidth{1.505625pt}%
\definecolor{currentstroke}{rgb}{0.000000,0.000000,0.000000}%
\pgfsetstrokecolor{currentstroke}%
\pgfsetdash{{5.550000pt}{2.400000pt}}{0.000000pt}%
\pgfpathmoveto{\pgfqpoint{0.801136in}{3.380000in}}%
\pgfpathlineto{\pgfqpoint{4.323864in}{0.610494in}}%
\pgfusepath{stroke}%
\end{pgfscope}%
\begin{pgfscope}%
\pgfsetrectcap%
\pgfsetmiterjoin%
\pgfsetlinewidth{0.803000pt}%
\definecolor{currentstroke}{rgb}{0.000000,0.000000,0.000000}%
\pgfsetstrokecolor{currentstroke}%
\pgfsetdash{}{0pt}%
\pgfpathmoveto{\pgfqpoint{0.625000in}{0.440000in}}%
\pgfpathlineto{\pgfqpoint{0.625000in}{3.520000in}}%
\pgfusepath{stroke}%
\end{pgfscope}%
\begin{pgfscope}%
\pgfsetrectcap%
\pgfsetmiterjoin%
\pgfsetlinewidth{0.803000pt}%
\definecolor{currentstroke}{rgb}{0.000000,0.000000,0.000000}%
\pgfsetstrokecolor{currentstroke}%
\pgfsetdash{}{0pt}%
\pgfpathmoveto{\pgfqpoint{4.500000in}{0.440000in}}%
\pgfpathlineto{\pgfqpoint{4.500000in}{3.520000in}}%
\pgfusepath{stroke}%
\end{pgfscope}%
\begin{pgfscope}%
\pgfsetrectcap%
\pgfsetmiterjoin%
\pgfsetlinewidth{0.803000pt}%
\definecolor{currentstroke}{rgb}{0.000000,0.000000,0.000000}%
\pgfsetstrokecolor{currentstroke}%
\pgfsetdash{}{0pt}%
\pgfpathmoveto{\pgfqpoint{0.625000in}{0.440000in}}%
\pgfpathlineto{\pgfqpoint{4.500000in}{0.440000in}}%
\pgfusepath{stroke}%
\end{pgfscope}%
\begin{pgfscope}%
\pgfsetrectcap%
\pgfsetmiterjoin%
\pgfsetlinewidth{0.803000pt}%
\definecolor{currentstroke}{rgb}{0.000000,0.000000,0.000000}%
\pgfsetstrokecolor{currentstroke}%
\pgfsetdash{}{0pt}%
\pgfpathmoveto{\pgfqpoint{0.625000in}{3.520000in}}%
\pgfpathlineto{\pgfqpoint{4.500000in}{3.520000in}}%
\pgfusepath{stroke}%
\end{pgfscope}%
\begin{pgfscope}%
\pgfsetbuttcap%
\pgfsetmiterjoin%
\definecolor{currentfill}{rgb}{1.000000,1.000000,1.000000}%
\pgfsetfillcolor{currentfill}%
\pgfsetfillopacity{0.800000}%
\pgfsetlinewidth{1.003750pt}%
\definecolor{currentstroke}{rgb}{0.800000,0.800000,0.800000}%
\pgfsetstrokecolor{currentstroke}%
\pgfsetstrokeopacity{0.800000}%
\pgfsetdash{}{0pt}%
\pgfpathmoveto{\pgfqpoint{3.042243in}{3.155267in}}%
\pgfpathlineto{\pgfqpoint{4.402778in}{3.155267in}}%
\pgfpathquadraticcurveto{\pgfqpoint{4.430556in}{3.155267in}}{\pgfqpoint{4.430556in}{3.183045in}}%
\pgfpathlineto{\pgfqpoint{4.430556in}{3.422778in}}%
\pgfpathquadraticcurveto{\pgfqpoint{4.430556in}{3.450556in}}{\pgfqpoint{4.402778in}{3.450556in}}%
\pgfpathlineto{\pgfqpoint{3.042243in}{3.450556in}}%
\pgfpathquadraticcurveto{\pgfqpoint{3.014465in}{3.450556in}}{\pgfqpoint{3.014465in}{3.422778in}}%
\pgfpathlineto{\pgfqpoint{3.014465in}{3.183045in}}%
\pgfpathquadraticcurveto{\pgfqpoint{3.014465in}{3.155267in}}{\pgfqpoint{3.042243in}{3.155267in}}%
\pgfpathclose%
\pgfusepath{stroke,fill}%
\end{pgfscope}%
\begin{pgfscope}%
\pgfsetbuttcap%
\pgfsetroundjoin%
\pgfsetlinewidth{1.505625pt}%
\definecolor{currentstroke}{rgb}{0.000000,0.000000,0.000000}%
\pgfsetstrokecolor{currentstroke}%
\pgfsetdash{{5.550000pt}{2.400000pt}}{0.000000pt}%
\pgfpathmoveto{\pgfqpoint{3.070021in}{3.321543in}}%
\pgfpathlineto{\pgfqpoint{3.347799in}{3.321543in}}%
\pgfusepath{stroke}%
\end{pgfscope}%
\begin{pgfscope}%
\definecolor{textcolor}{rgb}{0.000000,0.000000,0.000000}%
\pgfsetstrokecolor{textcolor}%
\pgfsetfillcolor{textcolor}%
\pgftext[x=3.458910in,y=3.272932in,left,base]{\color{textcolor}\sffamily\fontsize{10.000000}{12.000000}\selectfont \(\displaystyle 0.0022N_{\mathrm{QMC}}^{-0.9215}\)}%
\end{pgfscope}%
\end{pgfpicture}%
\makeatother%
\endgroup%

\caption[Quasi-Monte-Carlo error, for $Q=\int_D u$ and $k=20$.]{Quasi-Monte-Carlo error with increasing number of Quasi-Monte-Carlo points, for $Q=\int_D u$ and $k=20$. \label{fig:qmcintegral20}}
\end{figure}
\begin{figure}[h]
    \centering
    %% Creator: Matplotlib, PGF backend
%%
%% To include the figure in your LaTeX document, write
%%   \input{<filename>.pgf}
%%
%% Make sure the required packages are loaded in your preamble
%%   \usepackage{pgf}
%%
%% Figures using additional raster images can only be included by \input if
%% they are in the same directory as the main LaTeX file. For loading figures
%% from other directories you can use the `import` package
%%   \usepackage{import}
%% and then include the figures with
%%   \import{<path to file>}{<filename>.pgf}
%%
%% Matplotlib used the following preamble
%%   \usepackage{fontspec}
%%   \setmainfont{DejaVuSerif.ttf}[Path=/home/owen/progs/firedrake-complex/firedrake/lib/python3.5/site-packages/matplotlib/mpl-data/fonts/ttf/]
%%   \setsansfont{DejaVuSans.ttf}[Path=/home/owen/progs/firedrake-complex/firedrake/lib/python3.5/site-packages/matplotlib/mpl-data/fonts/ttf/]
%%   \setmonofont{DejaVuSansMono.ttf}[Path=/home/owen/progs/firedrake-complex/firedrake/lib/python3.5/site-packages/matplotlib/mpl-data/fonts/ttf/]
%%
\begingroup%
\makeatletter%
\begin{pgfpicture}%
\pgfpathrectangle{\pgfpointorigin}{\pgfqpoint{5.000000in}{4.000000in}}%
\pgfusepath{use as bounding box, clip}%
\begin{pgfscope}%
\pgfsetbuttcap%
\pgfsetmiterjoin%
\definecolor{currentfill}{rgb}{1.000000,1.000000,1.000000}%
\pgfsetfillcolor{currentfill}%
\pgfsetlinewidth{0.000000pt}%
\definecolor{currentstroke}{rgb}{1.000000,1.000000,1.000000}%
\pgfsetstrokecolor{currentstroke}%
\pgfsetdash{}{0pt}%
\pgfpathmoveto{\pgfqpoint{0.000000in}{0.000000in}}%
\pgfpathlineto{\pgfqpoint{5.000000in}{0.000000in}}%
\pgfpathlineto{\pgfqpoint{5.000000in}{4.000000in}}%
\pgfpathlineto{\pgfqpoint{0.000000in}{4.000000in}}%
\pgfpathclose%
\pgfusepath{fill}%
\end{pgfscope}%
\begin{pgfscope}%
\pgfsetbuttcap%
\pgfsetmiterjoin%
\definecolor{currentfill}{rgb}{1.000000,1.000000,1.000000}%
\pgfsetfillcolor{currentfill}%
\pgfsetlinewidth{0.000000pt}%
\definecolor{currentstroke}{rgb}{0.000000,0.000000,0.000000}%
\pgfsetstrokecolor{currentstroke}%
\pgfsetstrokeopacity{0.000000}%
\pgfsetdash{}{0pt}%
\pgfpathmoveto{\pgfqpoint{0.625000in}{0.440000in}}%
\pgfpathlineto{\pgfqpoint{4.500000in}{0.440000in}}%
\pgfpathlineto{\pgfqpoint{4.500000in}{3.520000in}}%
\pgfpathlineto{\pgfqpoint{0.625000in}{3.520000in}}%
\pgfpathclose%
\pgfusepath{fill}%
\end{pgfscope}%
\begin{pgfscope}%
\pgfsetbuttcap%
\pgfsetroundjoin%
\definecolor{currentfill}{rgb}{0.000000,0.000000,0.000000}%
\pgfsetfillcolor{currentfill}%
\pgfsetlinewidth{0.803000pt}%
\definecolor{currentstroke}{rgb}{0.000000,0.000000,0.000000}%
\pgfsetstrokecolor{currentstroke}%
\pgfsetdash{}{0pt}%
\pgfsys@defobject{currentmarker}{\pgfqpoint{0.000000in}{-0.048611in}}{\pgfqpoint{0.000000in}{0.000000in}}{%
\pgfpathmoveto{\pgfqpoint{0.000000in}{0.000000in}}%
\pgfpathlineto{\pgfqpoint{0.000000in}{-0.048611in}}%
\pgfusepath{stroke,fill}%
}%
\begin{pgfscope}%
\pgfsys@transformshift{0.801136in}{0.440000in}%
\pgfsys@useobject{currentmarker}{}%
\end{pgfscope}%
\end{pgfscope}%
\begin{pgfscope}%
\definecolor{textcolor}{rgb}{0.000000,0.000000,0.000000}%
\pgfsetstrokecolor{textcolor}%
\pgfsetfillcolor{textcolor}%
\pgftext[x=0.801136in,y=0.342778in,,top]{\color{textcolor}\sffamily\fontsize{11.000000}{13.200000}\selectfont \(\displaystyle 10^{0}\)}%
\end{pgfscope}%
\begin{pgfscope}%
\pgfsetbuttcap%
\pgfsetroundjoin%
\definecolor{currentfill}{rgb}{0.000000,0.000000,0.000000}%
\pgfsetfillcolor{currentfill}%
\pgfsetlinewidth{0.803000pt}%
\definecolor{currentstroke}{rgb}{0.000000,0.000000,0.000000}%
\pgfsetstrokecolor{currentstroke}%
\pgfsetdash{}{0pt}%
\pgfsys@defobject{currentmarker}{\pgfqpoint{0.000000in}{-0.048611in}}{\pgfqpoint{0.000000in}{0.000000in}}{%
\pgfpathmoveto{\pgfqpoint{0.000000in}{0.000000in}}%
\pgfpathlineto{\pgfqpoint{0.000000in}{-0.048611in}}%
\pgfusepath{stroke,fill}%
}%
\begin{pgfscope}%
\pgfsys@transformshift{1.864977in}{0.440000in}%
\pgfsys@useobject{currentmarker}{}%
\end{pgfscope}%
\end{pgfscope}%
\begin{pgfscope}%
\definecolor{textcolor}{rgb}{0.000000,0.000000,0.000000}%
\pgfsetstrokecolor{textcolor}%
\pgfsetfillcolor{textcolor}%
\pgftext[x=1.864977in,y=0.342778in,,top]{\color{textcolor}\sffamily\fontsize{11.000000}{13.200000}\selectfont \(\displaystyle 10^{1}\)}%
\end{pgfscope}%
\begin{pgfscope}%
\pgfsetbuttcap%
\pgfsetroundjoin%
\definecolor{currentfill}{rgb}{0.000000,0.000000,0.000000}%
\pgfsetfillcolor{currentfill}%
\pgfsetlinewidth{0.803000pt}%
\definecolor{currentstroke}{rgb}{0.000000,0.000000,0.000000}%
\pgfsetstrokecolor{currentstroke}%
\pgfsetdash{}{0pt}%
\pgfsys@defobject{currentmarker}{\pgfqpoint{0.000000in}{-0.048611in}}{\pgfqpoint{0.000000in}{0.000000in}}{%
\pgfpathmoveto{\pgfqpoint{0.000000in}{0.000000in}}%
\pgfpathlineto{\pgfqpoint{0.000000in}{-0.048611in}}%
\pgfusepath{stroke,fill}%
}%
\begin{pgfscope}%
\pgfsys@transformshift{2.928818in}{0.440000in}%
\pgfsys@useobject{currentmarker}{}%
\end{pgfscope}%
\end{pgfscope}%
\begin{pgfscope}%
\definecolor{textcolor}{rgb}{0.000000,0.000000,0.000000}%
\pgfsetstrokecolor{textcolor}%
\pgfsetfillcolor{textcolor}%
\pgftext[x=2.928818in,y=0.342778in,,top]{\color{textcolor}\sffamily\fontsize{11.000000}{13.200000}\selectfont \(\displaystyle 10^{2}\)}%
\end{pgfscope}%
\begin{pgfscope}%
\pgfsetbuttcap%
\pgfsetroundjoin%
\definecolor{currentfill}{rgb}{0.000000,0.000000,0.000000}%
\pgfsetfillcolor{currentfill}%
\pgfsetlinewidth{0.803000pt}%
\definecolor{currentstroke}{rgb}{0.000000,0.000000,0.000000}%
\pgfsetstrokecolor{currentstroke}%
\pgfsetdash{}{0pt}%
\pgfsys@defobject{currentmarker}{\pgfqpoint{0.000000in}{-0.048611in}}{\pgfqpoint{0.000000in}{0.000000in}}{%
\pgfpathmoveto{\pgfqpoint{0.000000in}{0.000000in}}%
\pgfpathlineto{\pgfqpoint{0.000000in}{-0.048611in}}%
\pgfusepath{stroke,fill}%
}%
\begin{pgfscope}%
\pgfsys@transformshift{3.992658in}{0.440000in}%
\pgfsys@useobject{currentmarker}{}%
\end{pgfscope}%
\end{pgfscope}%
\begin{pgfscope}%
\definecolor{textcolor}{rgb}{0.000000,0.000000,0.000000}%
\pgfsetstrokecolor{textcolor}%
\pgfsetfillcolor{textcolor}%
\pgftext[x=3.992658in,y=0.342778in,,top]{\color{textcolor}\sffamily\fontsize{11.000000}{13.200000}\selectfont \(\displaystyle 10^{3}\)}%
\end{pgfscope}%
\begin{pgfscope}%
\pgfsetbuttcap%
\pgfsetroundjoin%
\definecolor{currentfill}{rgb}{0.000000,0.000000,0.000000}%
\pgfsetfillcolor{currentfill}%
\pgfsetlinewidth{0.602250pt}%
\definecolor{currentstroke}{rgb}{0.000000,0.000000,0.000000}%
\pgfsetstrokecolor{currentstroke}%
\pgfsetdash{}{0pt}%
\pgfsys@defobject{currentmarker}{\pgfqpoint{0.000000in}{-0.027778in}}{\pgfqpoint{0.000000in}{0.000000in}}{%
\pgfpathmoveto{\pgfqpoint{0.000000in}{0.000000in}}%
\pgfpathlineto{\pgfqpoint{0.000000in}{-0.027778in}}%
\pgfusepath{stroke,fill}%
}%
\begin{pgfscope}%
\pgfsys@transformshift{0.636345in}{0.440000in}%
\pgfsys@useobject{currentmarker}{}%
\end{pgfscope}%
\end{pgfscope}%
\begin{pgfscope}%
\pgfsetbuttcap%
\pgfsetroundjoin%
\definecolor{currentfill}{rgb}{0.000000,0.000000,0.000000}%
\pgfsetfillcolor{currentfill}%
\pgfsetlinewidth{0.602250pt}%
\definecolor{currentstroke}{rgb}{0.000000,0.000000,0.000000}%
\pgfsetstrokecolor{currentstroke}%
\pgfsetdash{}{0pt}%
\pgfsys@defobject{currentmarker}{\pgfqpoint{0.000000in}{-0.027778in}}{\pgfqpoint{0.000000in}{0.000000in}}{%
\pgfpathmoveto{\pgfqpoint{0.000000in}{0.000000in}}%
\pgfpathlineto{\pgfqpoint{0.000000in}{-0.027778in}}%
\pgfusepath{stroke,fill}%
}%
\begin{pgfscope}%
\pgfsys@transformshift{0.698040in}{0.440000in}%
\pgfsys@useobject{currentmarker}{}%
\end{pgfscope}%
\end{pgfscope}%
\begin{pgfscope}%
\pgfsetbuttcap%
\pgfsetroundjoin%
\definecolor{currentfill}{rgb}{0.000000,0.000000,0.000000}%
\pgfsetfillcolor{currentfill}%
\pgfsetlinewidth{0.602250pt}%
\definecolor{currentstroke}{rgb}{0.000000,0.000000,0.000000}%
\pgfsetstrokecolor{currentstroke}%
\pgfsetdash{}{0pt}%
\pgfsys@defobject{currentmarker}{\pgfqpoint{0.000000in}{-0.027778in}}{\pgfqpoint{0.000000in}{0.000000in}}{%
\pgfpathmoveto{\pgfqpoint{0.000000in}{0.000000in}}%
\pgfpathlineto{\pgfqpoint{0.000000in}{-0.027778in}}%
\pgfusepath{stroke,fill}%
}%
\begin{pgfscope}%
\pgfsys@transformshift{0.752458in}{0.440000in}%
\pgfsys@useobject{currentmarker}{}%
\end{pgfscope}%
\end{pgfscope}%
\begin{pgfscope}%
\pgfsetbuttcap%
\pgfsetroundjoin%
\definecolor{currentfill}{rgb}{0.000000,0.000000,0.000000}%
\pgfsetfillcolor{currentfill}%
\pgfsetlinewidth{0.602250pt}%
\definecolor{currentstroke}{rgb}{0.000000,0.000000,0.000000}%
\pgfsetstrokecolor{currentstroke}%
\pgfsetdash{}{0pt}%
\pgfsys@defobject{currentmarker}{\pgfqpoint{0.000000in}{-0.027778in}}{\pgfqpoint{0.000000in}{0.000000in}}{%
\pgfpathmoveto{\pgfqpoint{0.000000in}{0.000000in}}%
\pgfpathlineto{\pgfqpoint{0.000000in}{-0.027778in}}%
\pgfusepath{stroke,fill}%
}%
\begin{pgfscope}%
\pgfsys@transformshift{1.121384in}{0.440000in}%
\pgfsys@useobject{currentmarker}{}%
\end{pgfscope}%
\end{pgfscope}%
\begin{pgfscope}%
\pgfsetbuttcap%
\pgfsetroundjoin%
\definecolor{currentfill}{rgb}{0.000000,0.000000,0.000000}%
\pgfsetfillcolor{currentfill}%
\pgfsetlinewidth{0.602250pt}%
\definecolor{currentstroke}{rgb}{0.000000,0.000000,0.000000}%
\pgfsetstrokecolor{currentstroke}%
\pgfsetdash{}{0pt}%
\pgfsys@defobject{currentmarker}{\pgfqpoint{0.000000in}{-0.027778in}}{\pgfqpoint{0.000000in}{0.000000in}}{%
\pgfpathmoveto{\pgfqpoint{0.000000in}{0.000000in}}%
\pgfpathlineto{\pgfqpoint{0.000000in}{-0.027778in}}%
\pgfusepath{stroke,fill}%
}%
\begin{pgfscope}%
\pgfsys@transformshift{1.308717in}{0.440000in}%
\pgfsys@useobject{currentmarker}{}%
\end{pgfscope}%
\end{pgfscope}%
\begin{pgfscope}%
\pgfsetbuttcap%
\pgfsetroundjoin%
\definecolor{currentfill}{rgb}{0.000000,0.000000,0.000000}%
\pgfsetfillcolor{currentfill}%
\pgfsetlinewidth{0.602250pt}%
\definecolor{currentstroke}{rgb}{0.000000,0.000000,0.000000}%
\pgfsetstrokecolor{currentstroke}%
\pgfsetdash{}{0pt}%
\pgfsys@defobject{currentmarker}{\pgfqpoint{0.000000in}{-0.027778in}}{\pgfqpoint{0.000000in}{0.000000in}}{%
\pgfpathmoveto{\pgfqpoint{0.000000in}{0.000000in}}%
\pgfpathlineto{\pgfqpoint{0.000000in}{-0.027778in}}%
\pgfusepath{stroke,fill}%
}%
\begin{pgfscope}%
\pgfsys@transformshift{1.441632in}{0.440000in}%
\pgfsys@useobject{currentmarker}{}%
\end{pgfscope}%
\end{pgfscope}%
\begin{pgfscope}%
\pgfsetbuttcap%
\pgfsetroundjoin%
\definecolor{currentfill}{rgb}{0.000000,0.000000,0.000000}%
\pgfsetfillcolor{currentfill}%
\pgfsetlinewidth{0.602250pt}%
\definecolor{currentstroke}{rgb}{0.000000,0.000000,0.000000}%
\pgfsetstrokecolor{currentstroke}%
\pgfsetdash{}{0pt}%
\pgfsys@defobject{currentmarker}{\pgfqpoint{0.000000in}{-0.027778in}}{\pgfqpoint{0.000000in}{0.000000in}}{%
\pgfpathmoveto{\pgfqpoint{0.000000in}{0.000000in}}%
\pgfpathlineto{\pgfqpoint{0.000000in}{-0.027778in}}%
\pgfusepath{stroke,fill}%
}%
\begin{pgfscope}%
\pgfsys@transformshift{1.544729in}{0.440000in}%
\pgfsys@useobject{currentmarker}{}%
\end{pgfscope}%
\end{pgfscope}%
\begin{pgfscope}%
\pgfsetbuttcap%
\pgfsetroundjoin%
\definecolor{currentfill}{rgb}{0.000000,0.000000,0.000000}%
\pgfsetfillcolor{currentfill}%
\pgfsetlinewidth{0.602250pt}%
\definecolor{currentstroke}{rgb}{0.000000,0.000000,0.000000}%
\pgfsetstrokecolor{currentstroke}%
\pgfsetdash{}{0pt}%
\pgfsys@defobject{currentmarker}{\pgfqpoint{0.000000in}{-0.027778in}}{\pgfqpoint{0.000000in}{0.000000in}}{%
\pgfpathmoveto{\pgfqpoint{0.000000in}{0.000000in}}%
\pgfpathlineto{\pgfqpoint{0.000000in}{-0.027778in}}%
\pgfusepath{stroke,fill}%
}%
\begin{pgfscope}%
\pgfsys@transformshift{1.628965in}{0.440000in}%
\pgfsys@useobject{currentmarker}{}%
\end{pgfscope}%
\end{pgfscope}%
\begin{pgfscope}%
\pgfsetbuttcap%
\pgfsetroundjoin%
\definecolor{currentfill}{rgb}{0.000000,0.000000,0.000000}%
\pgfsetfillcolor{currentfill}%
\pgfsetlinewidth{0.602250pt}%
\definecolor{currentstroke}{rgb}{0.000000,0.000000,0.000000}%
\pgfsetstrokecolor{currentstroke}%
\pgfsetdash{}{0pt}%
\pgfsys@defobject{currentmarker}{\pgfqpoint{0.000000in}{-0.027778in}}{\pgfqpoint{0.000000in}{0.000000in}}{%
\pgfpathmoveto{\pgfqpoint{0.000000in}{0.000000in}}%
\pgfpathlineto{\pgfqpoint{0.000000in}{-0.027778in}}%
\pgfusepath{stroke,fill}%
}%
\begin{pgfscope}%
\pgfsys@transformshift{1.700186in}{0.440000in}%
\pgfsys@useobject{currentmarker}{}%
\end{pgfscope}%
\end{pgfscope}%
\begin{pgfscope}%
\pgfsetbuttcap%
\pgfsetroundjoin%
\definecolor{currentfill}{rgb}{0.000000,0.000000,0.000000}%
\pgfsetfillcolor{currentfill}%
\pgfsetlinewidth{0.602250pt}%
\definecolor{currentstroke}{rgb}{0.000000,0.000000,0.000000}%
\pgfsetstrokecolor{currentstroke}%
\pgfsetdash{}{0pt}%
\pgfsys@defobject{currentmarker}{\pgfqpoint{0.000000in}{-0.027778in}}{\pgfqpoint{0.000000in}{0.000000in}}{%
\pgfpathmoveto{\pgfqpoint{0.000000in}{0.000000in}}%
\pgfpathlineto{\pgfqpoint{0.000000in}{-0.027778in}}%
\pgfusepath{stroke,fill}%
}%
\begin{pgfscope}%
\pgfsys@transformshift{1.761880in}{0.440000in}%
\pgfsys@useobject{currentmarker}{}%
\end{pgfscope}%
\end{pgfscope}%
\begin{pgfscope}%
\pgfsetbuttcap%
\pgfsetroundjoin%
\definecolor{currentfill}{rgb}{0.000000,0.000000,0.000000}%
\pgfsetfillcolor{currentfill}%
\pgfsetlinewidth{0.602250pt}%
\definecolor{currentstroke}{rgb}{0.000000,0.000000,0.000000}%
\pgfsetstrokecolor{currentstroke}%
\pgfsetdash{}{0pt}%
\pgfsys@defobject{currentmarker}{\pgfqpoint{0.000000in}{-0.027778in}}{\pgfqpoint{0.000000in}{0.000000in}}{%
\pgfpathmoveto{\pgfqpoint{0.000000in}{0.000000in}}%
\pgfpathlineto{\pgfqpoint{0.000000in}{-0.027778in}}%
\pgfusepath{stroke,fill}%
}%
\begin{pgfscope}%
\pgfsys@transformshift{1.816298in}{0.440000in}%
\pgfsys@useobject{currentmarker}{}%
\end{pgfscope}%
\end{pgfscope}%
\begin{pgfscope}%
\pgfsetbuttcap%
\pgfsetroundjoin%
\definecolor{currentfill}{rgb}{0.000000,0.000000,0.000000}%
\pgfsetfillcolor{currentfill}%
\pgfsetlinewidth{0.602250pt}%
\definecolor{currentstroke}{rgb}{0.000000,0.000000,0.000000}%
\pgfsetstrokecolor{currentstroke}%
\pgfsetdash{}{0pt}%
\pgfsys@defobject{currentmarker}{\pgfqpoint{0.000000in}{-0.027778in}}{\pgfqpoint{0.000000in}{0.000000in}}{%
\pgfpathmoveto{\pgfqpoint{0.000000in}{0.000000in}}%
\pgfpathlineto{\pgfqpoint{0.000000in}{-0.027778in}}%
\pgfusepath{stroke,fill}%
}%
\begin{pgfscope}%
\pgfsys@transformshift{2.185225in}{0.440000in}%
\pgfsys@useobject{currentmarker}{}%
\end{pgfscope}%
\end{pgfscope}%
\begin{pgfscope}%
\pgfsetbuttcap%
\pgfsetroundjoin%
\definecolor{currentfill}{rgb}{0.000000,0.000000,0.000000}%
\pgfsetfillcolor{currentfill}%
\pgfsetlinewidth{0.602250pt}%
\definecolor{currentstroke}{rgb}{0.000000,0.000000,0.000000}%
\pgfsetstrokecolor{currentstroke}%
\pgfsetdash{}{0pt}%
\pgfsys@defobject{currentmarker}{\pgfqpoint{0.000000in}{-0.027778in}}{\pgfqpoint{0.000000in}{0.000000in}}{%
\pgfpathmoveto{\pgfqpoint{0.000000in}{0.000000in}}%
\pgfpathlineto{\pgfqpoint{0.000000in}{-0.027778in}}%
\pgfusepath{stroke,fill}%
}%
\begin{pgfscope}%
\pgfsys@transformshift{2.372558in}{0.440000in}%
\pgfsys@useobject{currentmarker}{}%
\end{pgfscope}%
\end{pgfscope}%
\begin{pgfscope}%
\pgfsetbuttcap%
\pgfsetroundjoin%
\definecolor{currentfill}{rgb}{0.000000,0.000000,0.000000}%
\pgfsetfillcolor{currentfill}%
\pgfsetlinewidth{0.602250pt}%
\definecolor{currentstroke}{rgb}{0.000000,0.000000,0.000000}%
\pgfsetstrokecolor{currentstroke}%
\pgfsetdash{}{0pt}%
\pgfsys@defobject{currentmarker}{\pgfqpoint{0.000000in}{-0.027778in}}{\pgfqpoint{0.000000in}{0.000000in}}{%
\pgfpathmoveto{\pgfqpoint{0.000000in}{0.000000in}}%
\pgfpathlineto{\pgfqpoint{0.000000in}{-0.027778in}}%
\pgfusepath{stroke,fill}%
}%
\begin{pgfscope}%
\pgfsys@transformshift{2.505473in}{0.440000in}%
\pgfsys@useobject{currentmarker}{}%
\end{pgfscope}%
\end{pgfscope}%
\begin{pgfscope}%
\pgfsetbuttcap%
\pgfsetroundjoin%
\definecolor{currentfill}{rgb}{0.000000,0.000000,0.000000}%
\pgfsetfillcolor{currentfill}%
\pgfsetlinewidth{0.602250pt}%
\definecolor{currentstroke}{rgb}{0.000000,0.000000,0.000000}%
\pgfsetstrokecolor{currentstroke}%
\pgfsetdash{}{0pt}%
\pgfsys@defobject{currentmarker}{\pgfqpoint{0.000000in}{-0.027778in}}{\pgfqpoint{0.000000in}{0.000000in}}{%
\pgfpathmoveto{\pgfqpoint{0.000000in}{0.000000in}}%
\pgfpathlineto{\pgfqpoint{0.000000in}{-0.027778in}}%
\pgfusepath{stroke,fill}%
}%
\begin{pgfscope}%
\pgfsys@transformshift{2.608570in}{0.440000in}%
\pgfsys@useobject{currentmarker}{}%
\end{pgfscope}%
\end{pgfscope}%
\begin{pgfscope}%
\pgfsetbuttcap%
\pgfsetroundjoin%
\definecolor{currentfill}{rgb}{0.000000,0.000000,0.000000}%
\pgfsetfillcolor{currentfill}%
\pgfsetlinewidth{0.602250pt}%
\definecolor{currentstroke}{rgb}{0.000000,0.000000,0.000000}%
\pgfsetstrokecolor{currentstroke}%
\pgfsetdash{}{0pt}%
\pgfsys@defobject{currentmarker}{\pgfqpoint{0.000000in}{-0.027778in}}{\pgfqpoint{0.000000in}{0.000000in}}{%
\pgfpathmoveto{\pgfqpoint{0.000000in}{0.000000in}}%
\pgfpathlineto{\pgfqpoint{0.000000in}{-0.027778in}}%
\pgfusepath{stroke,fill}%
}%
\begin{pgfscope}%
\pgfsys@transformshift{2.692806in}{0.440000in}%
\pgfsys@useobject{currentmarker}{}%
\end{pgfscope}%
\end{pgfscope}%
\begin{pgfscope}%
\pgfsetbuttcap%
\pgfsetroundjoin%
\definecolor{currentfill}{rgb}{0.000000,0.000000,0.000000}%
\pgfsetfillcolor{currentfill}%
\pgfsetlinewidth{0.602250pt}%
\definecolor{currentstroke}{rgb}{0.000000,0.000000,0.000000}%
\pgfsetstrokecolor{currentstroke}%
\pgfsetdash{}{0pt}%
\pgfsys@defobject{currentmarker}{\pgfqpoint{0.000000in}{-0.027778in}}{\pgfqpoint{0.000000in}{0.000000in}}{%
\pgfpathmoveto{\pgfqpoint{0.000000in}{0.000000in}}%
\pgfpathlineto{\pgfqpoint{0.000000in}{-0.027778in}}%
\pgfusepath{stroke,fill}%
}%
\begin{pgfscope}%
\pgfsys@transformshift{2.764027in}{0.440000in}%
\pgfsys@useobject{currentmarker}{}%
\end{pgfscope}%
\end{pgfscope}%
\begin{pgfscope}%
\pgfsetbuttcap%
\pgfsetroundjoin%
\definecolor{currentfill}{rgb}{0.000000,0.000000,0.000000}%
\pgfsetfillcolor{currentfill}%
\pgfsetlinewidth{0.602250pt}%
\definecolor{currentstroke}{rgb}{0.000000,0.000000,0.000000}%
\pgfsetstrokecolor{currentstroke}%
\pgfsetdash{}{0pt}%
\pgfsys@defobject{currentmarker}{\pgfqpoint{0.000000in}{-0.027778in}}{\pgfqpoint{0.000000in}{0.000000in}}{%
\pgfpathmoveto{\pgfqpoint{0.000000in}{0.000000in}}%
\pgfpathlineto{\pgfqpoint{0.000000in}{-0.027778in}}%
\pgfusepath{stroke,fill}%
}%
\begin{pgfscope}%
\pgfsys@transformshift{2.825721in}{0.440000in}%
\pgfsys@useobject{currentmarker}{}%
\end{pgfscope}%
\end{pgfscope}%
\begin{pgfscope}%
\pgfsetbuttcap%
\pgfsetroundjoin%
\definecolor{currentfill}{rgb}{0.000000,0.000000,0.000000}%
\pgfsetfillcolor{currentfill}%
\pgfsetlinewidth{0.602250pt}%
\definecolor{currentstroke}{rgb}{0.000000,0.000000,0.000000}%
\pgfsetstrokecolor{currentstroke}%
\pgfsetdash{}{0pt}%
\pgfsys@defobject{currentmarker}{\pgfqpoint{0.000000in}{-0.027778in}}{\pgfqpoint{0.000000in}{0.000000in}}{%
\pgfpathmoveto{\pgfqpoint{0.000000in}{0.000000in}}%
\pgfpathlineto{\pgfqpoint{0.000000in}{-0.027778in}}%
\pgfusepath{stroke,fill}%
}%
\begin{pgfscope}%
\pgfsys@transformshift{2.880139in}{0.440000in}%
\pgfsys@useobject{currentmarker}{}%
\end{pgfscope}%
\end{pgfscope}%
\begin{pgfscope}%
\pgfsetbuttcap%
\pgfsetroundjoin%
\definecolor{currentfill}{rgb}{0.000000,0.000000,0.000000}%
\pgfsetfillcolor{currentfill}%
\pgfsetlinewidth{0.602250pt}%
\definecolor{currentstroke}{rgb}{0.000000,0.000000,0.000000}%
\pgfsetstrokecolor{currentstroke}%
\pgfsetdash{}{0pt}%
\pgfsys@defobject{currentmarker}{\pgfqpoint{0.000000in}{-0.027778in}}{\pgfqpoint{0.000000in}{0.000000in}}{%
\pgfpathmoveto{\pgfqpoint{0.000000in}{0.000000in}}%
\pgfpathlineto{\pgfqpoint{0.000000in}{-0.027778in}}%
\pgfusepath{stroke,fill}%
}%
\begin{pgfscope}%
\pgfsys@transformshift{3.249066in}{0.440000in}%
\pgfsys@useobject{currentmarker}{}%
\end{pgfscope}%
\end{pgfscope}%
\begin{pgfscope}%
\pgfsetbuttcap%
\pgfsetroundjoin%
\definecolor{currentfill}{rgb}{0.000000,0.000000,0.000000}%
\pgfsetfillcolor{currentfill}%
\pgfsetlinewidth{0.602250pt}%
\definecolor{currentstroke}{rgb}{0.000000,0.000000,0.000000}%
\pgfsetstrokecolor{currentstroke}%
\pgfsetdash{}{0pt}%
\pgfsys@defobject{currentmarker}{\pgfqpoint{0.000000in}{-0.027778in}}{\pgfqpoint{0.000000in}{0.000000in}}{%
\pgfpathmoveto{\pgfqpoint{0.000000in}{0.000000in}}%
\pgfpathlineto{\pgfqpoint{0.000000in}{-0.027778in}}%
\pgfusepath{stroke,fill}%
}%
\begin{pgfscope}%
\pgfsys@transformshift{3.436399in}{0.440000in}%
\pgfsys@useobject{currentmarker}{}%
\end{pgfscope}%
\end{pgfscope}%
\begin{pgfscope}%
\pgfsetbuttcap%
\pgfsetroundjoin%
\definecolor{currentfill}{rgb}{0.000000,0.000000,0.000000}%
\pgfsetfillcolor{currentfill}%
\pgfsetlinewidth{0.602250pt}%
\definecolor{currentstroke}{rgb}{0.000000,0.000000,0.000000}%
\pgfsetstrokecolor{currentstroke}%
\pgfsetdash{}{0pt}%
\pgfsys@defobject{currentmarker}{\pgfqpoint{0.000000in}{-0.027778in}}{\pgfqpoint{0.000000in}{0.000000in}}{%
\pgfpathmoveto{\pgfqpoint{0.000000in}{0.000000in}}%
\pgfpathlineto{\pgfqpoint{0.000000in}{-0.027778in}}%
\pgfusepath{stroke,fill}%
}%
\begin{pgfscope}%
\pgfsys@transformshift{3.569313in}{0.440000in}%
\pgfsys@useobject{currentmarker}{}%
\end{pgfscope}%
\end{pgfscope}%
\begin{pgfscope}%
\pgfsetbuttcap%
\pgfsetroundjoin%
\definecolor{currentfill}{rgb}{0.000000,0.000000,0.000000}%
\pgfsetfillcolor{currentfill}%
\pgfsetlinewidth{0.602250pt}%
\definecolor{currentstroke}{rgb}{0.000000,0.000000,0.000000}%
\pgfsetstrokecolor{currentstroke}%
\pgfsetdash{}{0pt}%
\pgfsys@defobject{currentmarker}{\pgfqpoint{0.000000in}{-0.027778in}}{\pgfqpoint{0.000000in}{0.000000in}}{%
\pgfpathmoveto{\pgfqpoint{0.000000in}{0.000000in}}%
\pgfpathlineto{\pgfqpoint{0.000000in}{-0.027778in}}%
\pgfusepath{stroke,fill}%
}%
\begin{pgfscope}%
\pgfsys@transformshift{3.672410in}{0.440000in}%
\pgfsys@useobject{currentmarker}{}%
\end{pgfscope}%
\end{pgfscope}%
\begin{pgfscope}%
\pgfsetbuttcap%
\pgfsetroundjoin%
\definecolor{currentfill}{rgb}{0.000000,0.000000,0.000000}%
\pgfsetfillcolor{currentfill}%
\pgfsetlinewidth{0.602250pt}%
\definecolor{currentstroke}{rgb}{0.000000,0.000000,0.000000}%
\pgfsetstrokecolor{currentstroke}%
\pgfsetdash{}{0pt}%
\pgfsys@defobject{currentmarker}{\pgfqpoint{0.000000in}{-0.027778in}}{\pgfqpoint{0.000000in}{0.000000in}}{%
\pgfpathmoveto{\pgfqpoint{0.000000in}{0.000000in}}%
\pgfpathlineto{\pgfqpoint{0.000000in}{-0.027778in}}%
\pgfusepath{stroke,fill}%
}%
\begin{pgfscope}%
\pgfsys@transformshift{3.756646in}{0.440000in}%
\pgfsys@useobject{currentmarker}{}%
\end{pgfscope}%
\end{pgfscope}%
\begin{pgfscope}%
\pgfsetbuttcap%
\pgfsetroundjoin%
\definecolor{currentfill}{rgb}{0.000000,0.000000,0.000000}%
\pgfsetfillcolor{currentfill}%
\pgfsetlinewidth{0.602250pt}%
\definecolor{currentstroke}{rgb}{0.000000,0.000000,0.000000}%
\pgfsetstrokecolor{currentstroke}%
\pgfsetdash{}{0pt}%
\pgfsys@defobject{currentmarker}{\pgfqpoint{0.000000in}{-0.027778in}}{\pgfqpoint{0.000000in}{0.000000in}}{%
\pgfpathmoveto{\pgfqpoint{0.000000in}{0.000000in}}%
\pgfpathlineto{\pgfqpoint{0.000000in}{-0.027778in}}%
\pgfusepath{stroke,fill}%
}%
\begin{pgfscope}%
\pgfsys@transformshift{3.827867in}{0.440000in}%
\pgfsys@useobject{currentmarker}{}%
\end{pgfscope}%
\end{pgfscope}%
\begin{pgfscope}%
\pgfsetbuttcap%
\pgfsetroundjoin%
\definecolor{currentfill}{rgb}{0.000000,0.000000,0.000000}%
\pgfsetfillcolor{currentfill}%
\pgfsetlinewidth{0.602250pt}%
\definecolor{currentstroke}{rgb}{0.000000,0.000000,0.000000}%
\pgfsetstrokecolor{currentstroke}%
\pgfsetdash{}{0pt}%
\pgfsys@defobject{currentmarker}{\pgfqpoint{0.000000in}{-0.027778in}}{\pgfqpoint{0.000000in}{0.000000in}}{%
\pgfpathmoveto{\pgfqpoint{0.000000in}{0.000000in}}%
\pgfpathlineto{\pgfqpoint{0.000000in}{-0.027778in}}%
\pgfusepath{stroke,fill}%
}%
\begin{pgfscope}%
\pgfsys@transformshift{3.889561in}{0.440000in}%
\pgfsys@useobject{currentmarker}{}%
\end{pgfscope}%
\end{pgfscope}%
\begin{pgfscope}%
\pgfsetbuttcap%
\pgfsetroundjoin%
\definecolor{currentfill}{rgb}{0.000000,0.000000,0.000000}%
\pgfsetfillcolor{currentfill}%
\pgfsetlinewidth{0.602250pt}%
\definecolor{currentstroke}{rgb}{0.000000,0.000000,0.000000}%
\pgfsetstrokecolor{currentstroke}%
\pgfsetdash{}{0pt}%
\pgfsys@defobject{currentmarker}{\pgfqpoint{0.000000in}{-0.027778in}}{\pgfqpoint{0.000000in}{0.000000in}}{%
\pgfpathmoveto{\pgfqpoint{0.000000in}{0.000000in}}%
\pgfpathlineto{\pgfqpoint{0.000000in}{-0.027778in}}%
\pgfusepath{stroke,fill}%
}%
\begin{pgfscope}%
\pgfsys@transformshift{3.943980in}{0.440000in}%
\pgfsys@useobject{currentmarker}{}%
\end{pgfscope}%
\end{pgfscope}%
\begin{pgfscope}%
\pgfsetbuttcap%
\pgfsetroundjoin%
\definecolor{currentfill}{rgb}{0.000000,0.000000,0.000000}%
\pgfsetfillcolor{currentfill}%
\pgfsetlinewidth{0.602250pt}%
\definecolor{currentstroke}{rgb}{0.000000,0.000000,0.000000}%
\pgfsetstrokecolor{currentstroke}%
\pgfsetdash{}{0pt}%
\pgfsys@defobject{currentmarker}{\pgfqpoint{0.000000in}{-0.027778in}}{\pgfqpoint{0.000000in}{0.000000in}}{%
\pgfpathmoveto{\pgfqpoint{0.000000in}{0.000000in}}%
\pgfpathlineto{\pgfqpoint{0.000000in}{-0.027778in}}%
\pgfusepath{stroke,fill}%
}%
\begin{pgfscope}%
\pgfsys@transformshift{4.312906in}{0.440000in}%
\pgfsys@useobject{currentmarker}{}%
\end{pgfscope}%
\end{pgfscope}%
\begin{pgfscope}%
\pgfsetbuttcap%
\pgfsetroundjoin%
\definecolor{currentfill}{rgb}{0.000000,0.000000,0.000000}%
\pgfsetfillcolor{currentfill}%
\pgfsetlinewidth{0.602250pt}%
\definecolor{currentstroke}{rgb}{0.000000,0.000000,0.000000}%
\pgfsetstrokecolor{currentstroke}%
\pgfsetdash{}{0pt}%
\pgfsys@defobject{currentmarker}{\pgfqpoint{0.000000in}{-0.027778in}}{\pgfqpoint{0.000000in}{0.000000in}}{%
\pgfpathmoveto{\pgfqpoint{0.000000in}{0.000000in}}%
\pgfpathlineto{\pgfqpoint{0.000000in}{-0.027778in}}%
\pgfusepath{stroke,fill}%
}%
\begin{pgfscope}%
\pgfsys@transformshift{4.500239in}{0.440000in}%
\pgfsys@useobject{currentmarker}{}%
\end{pgfscope}%
\end{pgfscope}%
\begin{pgfscope}%
\definecolor{textcolor}{rgb}{0.000000,0.000000,0.000000}%
\pgfsetstrokecolor{textcolor}%
\pgfsetfillcolor{textcolor}%
\pgftext[x=2.562500in,y=0.139368in,,top]{\color{textcolor}\sffamily\fontsize{11.000000}{13.200000}\selectfont \(\displaystyle N_\mathrm{QMC}\)}%
\end{pgfscope}%
\begin{pgfscope}%
\pgfsetbuttcap%
\pgfsetroundjoin%
\definecolor{currentfill}{rgb}{0.000000,0.000000,0.000000}%
\pgfsetfillcolor{currentfill}%
\pgfsetlinewidth{0.803000pt}%
\definecolor{currentstroke}{rgb}{0.000000,0.000000,0.000000}%
\pgfsetstrokecolor{currentstroke}%
\pgfsetdash{}{0pt}%
\pgfsys@defobject{currentmarker}{\pgfqpoint{-0.048611in}{0.000000in}}{\pgfqpoint{0.000000in}{0.000000in}}{%
\pgfpathmoveto{\pgfqpoint{0.000000in}{0.000000in}}%
\pgfpathlineto{\pgfqpoint{-0.048611in}{0.000000in}}%
\pgfusepath{stroke,fill}%
}%
\begin{pgfscope}%
\pgfsys@transformshift{0.625000in}{0.568430in}%
\pgfsys@useobject{currentmarker}{}%
\end{pgfscope}%
\end{pgfscope}%
\begin{pgfscope}%
\definecolor{textcolor}{rgb}{0.000000,0.000000,0.000000}%
\pgfsetstrokecolor{textcolor}%
\pgfsetfillcolor{textcolor}%
\pgftext[x=0.217899in,y=0.510393in,left,base]{\color{textcolor}\sffamily\fontsize{11.000000}{13.200000}\selectfont \(\displaystyle 10^{-6}\)}%
\end{pgfscope}%
\begin{pgfscope}%
\pgfsetbuttcap%
\pgfsetroundjoin%
\definecolor{currentfill}{rgb}{0.000000,0.000000,0.000000}%
\pgfsetfillcolor{currentfill}%
\pgfsetlinewidth{0.803000pt}%
\definecolor{currentstroke}{rgb}{0.000000,0.000000,0.000000}%
\pgfsetstrokecolor{currentstroke}%
\pgfsetdash{}{0pt}%
\pgfsys@defobject{currentmarker}{\pgfqpoint{-0.048611in}{0.000000in}}{\pgfqpoint{0.000000in}{0.000000in}}{%
\pgfpathmoveto{\pgfqpoint{0.000000in}{0.000000in}}%
\pgfpathlineto{\pgfqpoint{-0.048611in}{0.000000in}}%
\pgfusepath{stroke,fill}%
}%
\begin{pgfscope}%
\pgfsys@transformshift{0.625000in}{1.480263in}%
\pgfsys@useobject{currentmarker}{}%
\end{pgfscope}%
\end{pgfscope}%
\begin{pgfscope}%
\definecolor{textcolor}{rgb}{0.000000,0.000000,0.000000}%
\pgfsetstrokecolor{textcolor}%
\pgfsetfillcolor{textcolor}%
\pgftext[x=0.217899in,y=1.422225in,left,base]{\color{textcolor}\sffamily\fontsize{11.000000}{13.200000}\selectfont \(\displaystyle 10^{-5}\)}%
\end{pgfscope}%
\begin{pgfscope}%
\pgfsetbuttcap%
\pgfsetroundjoin%
\definecolor{currentfill}{rgb}{0.000000,0.000000,0.000000}%
\pgfsetfillcolor{currentfill}%
\pgfsetlinewidth{0.803000pt}%
\definecolor{currentstroke}{rgb}{0.000000,0.000000,0.000000}%
\pgfsetstrokecolor{currentstroke}%
\pgfsetdash{}{0pt}%
\pgfsys@defobject{currentmarker}{\pgfqpoint{-0.048611in}{0.000000in}}{\pgfqpoint{0.000000in}{0.000000in}}{%
\pgfpathmoveto{\pgfqpoint{0.000000in}{0.000000in}}%
\pgfpathlineto{\pgfqpoint{-0.048611in}{0.000000in}}%
\pgfusepath{stroke,fill}%
}%
\begin{pgfscope}%
\pgfsys@transformshift{0.625000in}{2.392096in}%
\pgfsys@useobject{currentmarker}{}%
\end{pgfscope}%
\end{pgfscope}%
\begin{pgfscope}%
\definecolor{textcolor}{rgb}{0.000000,0.000000,0.000000}%
\pgfsetstrokecolor{textcolor}%
\pgfsetfillcolor{textcolor}%
\pgftext[x=0.217899in,y=2.334058in,left,base]{\color{textcolor}\sffamily\fontsize{11.000000}{13.200000}\selectfont \(\displaystyle 10^{-4}\)}%
\end{pgfscope}%
\begin{pgfscope}%
\pgfsetbuttcap%
\pgfsetroundjoin%
\definecolor{currentfill}{rgb}{0.000000,0.000000,0.000000}%
\pgfsetfillcolor{currentfill}%
\pgfsetlinewidth{0.803000pt}%
\definecolor{currentstroke}{rgb}{0.000000,0.000000,0.000000}%
\pgfsetstrokecolor{currentstroke}%
\pgfsetdash{}{0pt}%
\pgfsys@defobject{currentmarker}{\pgfqpoint{-0.048611in}{0.000000in}}{\pgfqpoint{0.000000in}{0.000000in}}{%
\pgfpathmoveto{\pgfqpoint{0.000000in}{0.000000in}}%
\pgfpathlineto{\pgfqpoint{-0.048611in}{0.000000in}}%
\pgfusepath{stroke,fill}%
}%
\begin{pgfscope}%
\pgfsys@transformshift{0.625000in}{3.303929in}%
\pgfsys@useobject{currentmarker}{}%
\end{pgfscope}%
\end{pgfscope}%
\begin{pgfscope}%
\definecolor{textcolor}{rgb}{0.000000,0.000000,0.000000}%
\pgfsetstrokecolor{textcolor}%
\pgfsetfillcolor{textcolor}%
\pgftext[x=0.217899in,y=3.245891in,left,base]{\color{textcolor}\sffamily\fontsize{11.000000}{13.200000}\selectfont \(\displaystyle 10^{-3}\)}%
\end{pgfscope}%
\begin{pgfscope}%
\pgfsetbuttcap%
\pgfsetroundjoin%
\definecolor{currentfill}{rgb}{0.000000,0.000000,0.000000}%
\pgfsetfillcolor{currentfill}%
\pgfsetlinewidth{0.602250pt}%
\definecolor{currentstroke}{rgb}{0.000000,0.000000,0.000000}%
\pgfsetstrokecolor{currentstroke}%
\pgfsetdash{}{0pt}%
\pgfsys@defobject{currentmarker}{\pgfqpoint{-0.027778in}{0.000000in}}{\pgfqpoint{0.000000in}{0.000000in}}{%
\pgfpathmoveto{\pgfqpoint{0.000000in}{0.000000in}}%
\pgfpathlineto{\pgfqpoint{-0.027778in}{0.000000in}}%
\pgfusepath{stroke,fill}%
}%
\begin{pgfscope}%
\pgfsys@transformshift{0.625000in}{0.480064in}%
\pgfsys@useobject{currentmarker}{}%
\end{pgfscope}%
\end{pgfscope}%
\begin{pgfscope}%
\pgfsetbuttcap%
\pgfsetroundjoin%
\definecolor{currentfill}{rgb}{0.000000,0.000000,0.000000}%
\pgfsetfillcolor{currentfill}%
\pgfsetlinewidth{0.602250pt}%
\definecolor{currentstroke}{rgb}{0.000000,0.000000,0.000000}%
\pgfsetstrokecolor{currentstroke}%
\pgfsetdash{}{0pt}%
\pgfsys@defobject{currentmarker}{\pgfqpoint{-0.027778in}{0.000000in}}{\pgfqpoint{0.000000in}{0.000000in}}{%
\pgfpathmoveto{\pgfqpoint{0.000000in}{0.000000in}}%
\pgfpathlineto{\pgfqpoint{-0.027778in}{0.000000in}}%
\pgfusepath{stroke,fill}%
}%
\begin{pgfscope}%
\pgfsys@transformshift{0.625000in}{0.526707in}%
\pgfsys@useobject{currentmarker}{}%
\end{pgfscope}%
\end{pgfscope}%
\begin{pgfscope}%
\pgfsetbuttcap%
\pgfsetroundjoin%
\definecolor{currentfill}{rgb}{0.000000,0.000000,0.000000}%
\pgfsetfillcolor{currentfill}%
\pgfsetlinewidth{0.602250pt}%
\definecolor{currentstroke}{rgb}{0.000000,0.000000,0.000000}%
\pgfsetstrokecolor{currentstroke}%
\pgfsetdash{}{0pt}%
\pgfsys@defobject{currentmarker}{\pgfqpoint{-0.027778in}{0.000000in}}{\pgfqpoint{0.000000in}{0.000000in}}{%
\pgfpathmoveto{\pgfqpoint{0.000000in}{0.000000in}}%
\pgfpathlineto{\pgfqpoint{-0.027778in}{0.000000in}}%
\pgfusepath{stroke,fill}%
}%
\begin{pgfscope}%
\pgfsys@transformshift{0.625000in}{0.842919in}%
\pgfsys@useobject{currentmarker}{}%
\end{pgfscope}%
\end{pgfscope}%
\begin{pgfscope}%
\pgfsetbuttcap%
\pgfsetroundjoin%
\definecolor{currentfill}{rgb}{0.000000,0.000000,0.000000}%
\pgfsetfillcolor{currentfill}%
\pgfsetlinewidth{0.602250pt}%
\definecolor{currentstroke}{rgb}{0.000000,0.000000,0.000000}%
\pgfsetstrokecolor{currentstroke}%
\pgfsetdash{}{0pt}%
\pgfsys@defobject{currentmarker}{\pgfqpoint{-0.027778in}{0.000000in}}{\pgfqpoint{0.000000in}{0.000000in}}{%
\pgfpathmoveto{\pgfqpoint{0.000000in}{0.000000in}}%
\pgfpathlineto{\pgfqpoint{-0.027778in}{0.000000in}}%
\pgfusepath{stroke,fill}%
}%
\begin{pgfscope}%
\pgfsys@transformshift{0.625000in}{1.003485in}%
\pgfsys@useobject{currentmarker}{}%
\end{pgfscope}%
\end{pgfscope}%
\begin{pgfscope}%
\pgfsetbuttcap%
\pgfsetroundjoin%
\definecolor{currentfill}{rgb}{0.000000,0.000000,0.000000}%
\pgfsetfillcolor{currentfill}%
\pgfsetlinewidth{0.602250pt}%
\definecolor{currentstroke}{rgb}{0.000000,0.000000,0.000000}%
\pgfsetstrokecolor{currentstroke}%
\pgfsetdash{}{0pt}%
\pgfsys@defobject{currentmarker}{\pgfqpoint{-0.027778in}{0.000000in}}{\pgfqpoint{0.000000in}{0.000000in}}{%
\pgfpathmoveto{\pgfqpoint{0.000000in}{0.000000in}}%
\pgfpathlineto{\pgfqpoint{-0.027778in}{0.000000in}}%
\pgfusepath{stroke,fill}%
}%
\begin{pgfscope}%
\pgfsys@transformshift{0.625000in}{1.117408in}%
\pgfsys@useobject{currentmarker}{}%
\end{pgfscope}%
\end{pgfscope}%
\begin{pgfscope}%
\pgfsetbuttcap%
\pgfsetroundjoin%
\definecolor{currentfill}{rgb}{0.000000,0.000000,0.000000}%
\pgfsetfillcolor{currentfill}%
\pgfsetlinewidth{0.602250pt}%
\definecolor{currentstroke}{rgb}{0.000000,0.000000,0.000000}%
\pgfsetstrokecolor{currentstroke}%
\pgfsetdash{}{0pt}%
\pgfsys@defobject{currentmarker}{\pgfqpoint{-0.027778in}{0.000000in}}{\pgfqpoint{0.000000in}{0.000000in}}{%
\pgfpathmoveto{\pgfqpoint{0.000000in}{0.000000in}}%
\pgfpathlineto{\pgfqpoint{-0.027778in}{0.000000in}}%
\pgfusepath{stroke,fill}%
}%
\begin{pgfscope}%
\pgfsys@transformshift{0.625000in}{1.205774in}%
\pgfsys@useobject{currentmarker}{}%
\end{pgfscope}%
\end{pgfscope}%
\begin{pgfscope}%
\pgfsetbuttcap%
\pgfsetroundjoin%
\definecolor{currentfill}{rgb}{0.000000,0.000000,0.000000}%
\pgfsetfillcolor{currentfill}%
\pgfsetlinewidth{0.602250pt}%
\definecolor{currentstroke}{rgb}{0.000000,0.000000,0.000000}%
\pgfsetstrokecolor{currentstroke}%
\pgfsetdash{}{0pt}%
\pgfsys@defobject{currentmarker}{\pgfqpoint{-0.027778in}{0.000000in}}{\pgfqpoint{0.000000in}{0.000000in}}{%
\pgfpathmoveto{\pgfqpoint{0.000000in}{0.000000in}}%
\pgfpathlineto{\pgfqpoint{-0.027778in}{0.000000in}}%
\pgfusepath{stroke,fill}%
}%
\begin{pgfscope}%
\pgfsys@transformshift{0.625000in}{1.277974in}%
\pgfsys@useobject{currentmarker}{}%
\end{pgfscope}%
\end{pgfscope}%
\begin{pgfscope}%
\pgfsetbuttcap%
\pgfsetroundjoin%
\definecolor{currentfill}{rgb}{0.000000,0.000000,0.000000}%
\pgfsetfillcolor{currentfill}%
\pgfsetlinewidth{0.602250pt}%
\definecolor{currentstroke}{rgb}{0.000000,0.000000,0.000000}%
\pgfsetstrokecolor{currentstroke}%
\pgfsetdash{}{0pt}%
\pgfsys@defobject{currentmarker}{\pgfqpoint{-0.027778in}{0.000000in}}{\pgfqpoint{0.000000in}{0.000000in}}{%
\pgfpathmoveto{\pgfqpoint{0.000000in}{0.000000in}}%
\pgfpathlineto{\pgfqpoint{-0.027778in}{0.000000in}}%
\pgfusepath{stroke,fill}%
}%
\begin{pgfscope}%
\pgfsys@transformshift{0.625000in}{1.339018in}%
\pgfsys@useobject{currentmarker}{}%
\end{pgfscope}%
\end{pgfscope}%
\begin{pgfscope}%
\pgfsetbuttcap%
\pgfsetroundjoin%
\definecolor{currentfill}{rgb}{0.000000,0.000000,0.000000}%
\pgfsetfillcolor{currentfill}%
\pgfsetlinewidth{0.602250pt}%
\definecolor{currentstroke}{rgb}{0.000000,0.000000,0.000000}%
\pgfsetstrokecolor{currentstroke}%
\pgfsetdash{}{0pt}%
\pgfsys@defobject{currentmarker}{\pgfqpoint{-0.027778in}{0.000000in}}{\pgfqpoint{0.000000in}{0.000000in}}{%
\pgfpathmoveto{\pgfqpoint{0.000000in}{0.000000in}}%
\pgfpathlineto{\pgfqpoint{-0.027778in}{0.000000in}}%
\pgfusepath{stroke,fill}%
}%
\begin{pgfscope}%
\pgfsys@transformshift{0.625000in}{1.391897in}%
\pgfsys@useobject{currentmarker}{}%
\end{pgfscope}%
\end{pgfscope}%
\begin{pgfscope}%
\pgfsetbuttcap%
\pgfsetroundjoin%
\definecolor{currentfill}{rgb}{0.000000,0.000000,0.000000}%
\pgfsetfillcolor{currentfill}%
\pgfsetlinewidth{0.602250pt}%
\definecolor{currentstroke}{rgb}{0.000000,0.000000,0.000000}%
\pgfsetstrokecolor{currentstroke}%
\pgfsetdash{}{0pt}%
\pgfsys@defobject{currentmarker}{\pgfqpoint{-0.027778in}{0.000000in}}{\pgfqpoint{0.000000in}{0.000000in}}{%
\pgfpathmoveto{\pgfqpoint{0.000000in}{0.000000in}}%
\pgfpathlineto{\pgfqpoint{-0.027778in}{0.000000in}}%
\pgfusepath{stroke,fill}%
}%
\begin{pgfscope}%
\pgfsys@transformshift{0.625000in}{1.438540in}%
\pgfsys@useobject{currentmarker}{}%
\end{pgfscope}%
\end{pgfscope}%
\begin{pgfscope}%
\pgfsetbuttcap%
\pgfsetroundjoin%
\definecolor{currentfill}{rgb}{0.000000,0.000000,0.000000}%
\pgfsetfillcolor{currentfill}%
\pgfsetlinewidth{0.602250pt}%
\definecolor{currentstroke}{rgb}{0.000000,0.000000,0.000000}%
\pgfsetstrokecolor{currentstroke}%
\pgfsetdash{}{0pt}%
\pgfsys@defobject{currentmarker}{\pgfqpoint{-0.027778in}{0.000000in}}{\pgfqpoint{0.000000in}{0.000000in}}{%
\pgfpathmoveto{\pgfqpoint{0.000000in}{0.000000in}}%
\pgfpathlineto{\pgfqpoint{-0.027778in}{0.000000in}}%
\pgfusepath{stroke,fill}%
}%
\begin{pgfscope}%
\pgfsys@transformshift{0.625000in}{1.754752in}%
\pgfsys@useobject{currentmarker}{}%
\end{pgfscope}%
\end{pgfscope}%
\begin{pgfscope}%
\pgfsetbuttcap%
\pgfsetroundjoin%
\definecolor{currentfill}{rgb}{0.000000,0.000000,0.000000}%
\pgfsetfillcolor{currentfill}%
\pgfsetlinewidth{0.602250pt}%
\definecolor{currentstroke}{rgb}{0.000000,0.000000,0.000000}%
\pgfsetstrokecolor{currentstroke}%
\pgfsetdash{}{0pt}%
\pgfsys@defobject{currentmarker}{\pgfqpoint{-0.027778in}{0.000000in}}{\pgfqpoint{0.000000in}{0.000000in}}{%
\pgfpathmoveto{\pgfqpoint{0.000000in}{0.000000in}}%
\pgfpathlineto{\pgfqpoint{-0.027778in}{0.000000in}}%
\pgfusepath{stroke,fill}%
}%
\begin{pgfscope}%
\pgfsys@transformshift{0.625000in}{1.915318in}%
\pgfsys@useobject{currentmarker}{}%
\end{pgfscope}%
\end{pgfscope}%
\begin{pgfscope}%
\pgfsetbuttcap%
\pgfsetroundjoin%
\definecolor{currentfill}{rgb}{0.000000,0.000000,0.000000}%
\pgfsetfillcolor{currentfill}%
\pgfsetlinewidth{0.602250pt}%
\definecolor{currentstroke}{rgb}{0.000000,0.000000,0.000000}%
\pgfsetstrokecolor{currentstroke}%
\pgfsetdash{}{0pt}%
\pgfsys@defobject{currentmarker}{\pgfqpoint{-0.027778in}{0.000000in}}{\pgfqpoint{0.000000in}{0.000000in}}{%
\pgfpathmoveto{\pgfqpoint{0.000000in}{0.000000in}}%
\pgfpathlineto{\pgfqpoint{-0.027778in}{0.000000in}}%
\pgfusepath{stroke,fill}%
}%
\begin{pgfscope}%
\pgfsys@transformshift{0.625000in}{2.029241in}%
\pgfsys@useobject{currentmarker}{}%
\end{pgfscope}%
\end{pgfscope}%
\begin{pgfscope}%
\pgfsetbuttcap%
\pgfsetroundjoin%
\definecolor{currentfill}{rgb}{0.000000,0.000000,0.000000}%
\pgfsetfillcolor{currentfill}%
\pgfsetlinewidth{0.602250pt}%
\definecolor{currentstroke}{rgb}{0.000000,0.000000,0.000000}%
\pgfsetstrokecolor{currentstroke}%
\pgfsetdash{}{0pt}%
\pgfsys@defobject{currentmarker}{\pgfqpoint{-0.027778in}{0.000000in}}{\pgfqpoint{0.000000in}{0.000000in}}{%
\pgfpathmoveto{\pgfqpoint{0.000000in}{0.000000in}}%
\pgfpathlineto{\pgfqpoint{-0.027778in}{0.000000in}}%
\pgfusepath{stroke,fill}%
}%
\begin{pgfscope}%
\pgfsys@transformshift{0.625000in}{2.117607in}%
\pgfsys@useobject{currentmarker}{}%
\end{pgfscope}%
\end{pgfscope}%
\begin{pgfscope}%
\pgfsetbuttcap%
\pgfsetroundjoin%
\definecolor{currentfill}{rgb}{0.000000,0.000000,0.000000}%
\pgfsetfillcolor{currentfill}%
\pgfsetlinewidth{0.602250pt}%
\definecolor{currentstroke}{rgb}{0.000000,0.000000,0.000000}%
\pgfsetstrokecolor{currentstroke}%
\pgfsetdash{}{0pt}%
\pgfsys@defobject{currentmarker}{\pgfqpoint{-0.027778in}{0.000000in}}{\pgfqpoint{0.000000in}{0.000000in}}{%
\pgfpathmoveto{\pgfqpoint{0.000000in}{0.000000in}}%
\pgfpathlineto{\pgfqpoint{-0.027778in}{0.000000in}}%
\pgfusepath{stroke,fill}%
}%
\begin{pgfscope}%
\pgfsys@transformshift{0.625000in}{2.189807in}%
\pgfsys@useobject{currentmarker}{}%
\end{pgfscope}%
\end{pgfscope}%
\begin{pgfscope}%
\pgfsetbuttcap%
\pgfsetroundjoin%
\definecolor{currentfill}{rgb}{0.000000,0.000000,0.000000}%
\pgfsetfillcolor{currentfill}%
\pgfsetlinewidth{0.602250pt}%
\definecolor{currentstroke}{rgb}{0.000000,0.000000,0.000000}%
\pgfsetstrokecolor{currentstroke}%
\pgfsetdash{}{0pt}%
\pgfsys@defobject{currentmarker}{\pgfqpoint{-0.027778in}{0.000000in}}{\pgfqpoint{0.000000in}{0.000000in}}{%
\pgfpathmoveto{\pgfqpoint{0.000000in}{0.000000in}}%
\pgfpathlineto{\pgfqpoint{-0.027778in}{0.000000in}}%
\pgfusepath{stroke,fill}%
}%
\begin{pgfscope}%
\pgfsys@transformshift{0.625000in}{2.250851in}%
\pgfsys@useobject{currentmarker}{}%
\end{pgfscope}%
\end{pgfscope}%
\begin{pgfscope}%
\pgfsetbuttcap%
\pgfsetroundjoin%
\definecolor{currentfill}{rgb}{0.000000,0.000000,0.000000}%
\pgfsetfillcolor{currentfill}%
\pgfsetlinewidth{0.602250pt}%
\definecolor{currentstroke}{rgb}{0.000000,0.000000,0.000000}%
\pgfsetstrokecolor{currentstroke}%
\pgfsetdash{}{0pt}%
\pgfsys@defobject{currentmarker}{\pgfqpoint{-0.027778in}{0.000000in}}{\pgfqpoint{0.000000in}{0.000000in}}{%
\pgfpathmoveto{\pgfqpoint{0.000000in}{0.000000in}}%
\pgfpathlineto{\pgfqpoint{-0.027778in}{0.000000in}}%
\pgfusepath{stroke,fill}%
}%
\begin{pgfscope}%
\pgfsys@transformshift{0.625000in}{2.303730in}%
\pgfsys@useobject{currentmarker}{}%
\end{pgfscope}%
\end{pgfscope}%
\begin{pgfscope}%
\pgfsetbuttcap%
\pgfsetroundjoin%
\definecolor{currentfill}{rgb}{0.000000,0.000000,0.000000}%
\pgfsetfillcolor{currentfill}%
\pgfsetlinewidth{0.602250pt}%
\definecolor{currentstroke}{rgb}{0.000000,0.000000,0.000000}%
\pgfsetstrokecolor{currentstroke}%
\pgfsetdash{}{0pt}%
\pgfsys@defobject{currentmarker}{\pgfqpoint{-0.027778in}{0.000000in}}{\pgfqpoint{0.000000in}{0.000000in}}{%
\pgfpathmoveto{\pgfqpoint{0.000000in}{0.000000in}}%
\pgfpathlineto{\pgfqpoint{-0.027778in}{0.000000in}}%
\pgfusepath{stroke,fill}%
}%
\begin{pgfscope}%
\pgfsys@transformshift{0.625000in}{2.350373in}%
\pgfsys@useobject{currentmarker}{}%
\end{pgfscope}%
\end{pgfscope}%
\begin{pgfscope}%
\pgfsetbuttcap%
\pgfsetroundjoin%
\definecolor{currentfill}{rgb}{0.000000,0.000000,0.000000}%
\pgfsetfillcolor{currentfill}%
\pgfsetlinewidth{0.602250pt}%
\definecolor{currentstroke}{rgb}{0.000000,0.000000,0.000000}%
\pgfsetstrokecolor{currentstroke}%
\pgfsetdash{}{0pt}%
\pgfsys@defobject{currentmarker}{\pgfqpoint{-0.027778in}{0.000000in}}{\pgfqpoint{0.000000in}{0.000000in}}{%
\pgfpathmoveto{\pgfqpoint{0.000000in}{0.000000in}}%
\pgfpathlineto{\pgfqpoint{-0.027778in}{0.000000in}}%
\pgfusepath{stroke,fill}%
}%
\begin{pgfscope}%
\pgfsys@transformshift{0.625000in}{2.666585in}%
\pgfsys@useobject{currentmarker}{}%
\end{pgfscope}%
\end{pgfscope}%
\begin{pgfscope}%
\pgfsetbuttcap%
\pgfsetroundjoin%
\definecolor{currentfill}{rgb}{0.000000,0.000000,0.000000}%
\pgfsetfillcolor{currentfill}%
\pgfsetlinewidth{0.602250pt}%
\definecolor{currentstroke}{rgb}{0.000000,0.000000,0.000000}%
\pgfsetstrokecolor{currentstroke}%
\pgfsetdash{}{0pt}%
\pgfsys@defobject{currentmarker}{\pgfqpoint{-0.027778in}{0.000000in}}{\pgfqpoint{0.000000in}{0.000000in}}{%
\pgfpathmoveto{\pgfqpoint{0.000000in}{0.000000in}}%
\pgfpathlineto{\pgfqpoint{-0.027778in}{0.000000in}}%
\pgfusepath{stroke,fill}%
}%
\begin{pgfscope}%
\pgfsys@transformshift{0.625000in}{2.827151in}%
\pgfsys@useobject{currentmarker}{}%
\end{pgfscope}%
\end{pgfscope}%
\begin{pgfscope}%
\pgfsetbuttcap%
\pgfsetroundjoin%
\definecolor{currentfill}{rgb}{0.000000,0.000000,0.000000}%
\pgfsetfillcolor{currentfill}%
\pgfsetlinewidth{0.602250pt}%
\definecolor{currentstroke}{rgb}{0.000000,0.000000,0.000000}%
\pgfsetstrokecolor{currentstroke}%
\pgfsetdash{}{0pt}%
\pgfsys@defobject{currentmarker}{\pgfqpoint{-0.027778in}{0.000000in}}{\pgfqpoint{0.000000in}{0.000000in}}{%
\pgfpathmoveto{\pgfqpoint{0.000000in}{0.000000in}}%
\pgfpathlineto{\pgfqpoint{-0.027778in}{0.000000in}}%
\pgfusepath{stroke,fill}%
}%
\begin{pgfscope}%
\pgfsys@transformshift{0.625000in}{2.941074in}%
\pgfsys@useobject{currentmarker}{}%
\end{pgfscope}%
\end{pgfscope}%
\begin{pgfscope}%
\pgfsetbuttcap%
\pgfsetroundjoin%
\definecolor{currentfill}{rgb}{0.000000,0.000000,0.000000}%
\pgfsetfillcolor{currentfill}%
\pgfsetlinewidth{0.602250pt}%
\definecolor{currentstroke}{rgb}{0.000000,0.000000,0.000000}%
\pgfsetstrokecolor{currentstroke}%
\pgfsetdash{}{0pt}%
\pgfsys@defobject{currentmarker}{\pgfqpoint{-0.027778in}{0.000000in}}{\pgfqpoint{0.000000in}{0.000000in}}{%
\pgfpathmoveto{\pgfqpoint{0.000000in}{0.000000in}}%
\pgfpathlineto{\pgfqpoint{-0.027778in}{0.000000in}}%
\pgfusepath{stroke,fill}%
}%
\begin{pgfscope}%
\pgfsys@transformshift{0.625000in}{3.029440in}%
\pgfsys@useobject{currentmarker}{}%
\end{pgfscope}%
\end{pgfscope}%
\begin{pgfscope}%
\pgfsetbuttcap%
\pgfsetroundjoin%
\definecolor{currentfill}{rgb}{0.000000,0.000000,0.000000}%
\pgfsetfillcolor{currentfill}%
\pgfsetlinewidth{0.602250pt}%
\definecolor{currentstroke}{rgb}{0.000000,0.000000,0.000000}%
\pgfsetstrokecolor{currentstroke}%
\pgfsetdash{}{0pt}%
\pgfsys@defobject{currentmarker}{\pgfqpoint{-0.027778in}{0.000000in}}{\pgfqpoint{0.000000in}{0.000000in}}{%
\pgfpathmoveto{\pgfqpoint{0.000000in}{0.000000in}}%
\pgfpathlineto{\pgfqpoint{-0.027778in}{0.000000in}}%
\pgfusepath{stroke,fill}%
}%
\begin{pgfscope}%
\pgfsys@transformshift{0.625000in}{3.101640in}%
\pgfsys@useobject{currentmarker}{}%
\end{pgfscope}%
\end{pgfscope}%
\begin{pgfscope}%
\pgfsetbuttcap%
\pgfsetroundjoin%
\definecolor{currentfill}{rgb}{0.000000,0.000000,0.000000}%
\pgfsetfillcolor{currentfill}%
\pgfsetlinewidth{0.602250pt}%
\definecolor{currentstroke}{rgb}{0.000000,0.000000,0.000000}%
\pgfsetstrokecolor{currentstroke}%
\pgfsetdash{}{0pt}%
\pgfsys@defobject{currentmarker}{\pgfqpoint{-0.027778in}{0.000000in}}{\pgfqpoint{0.000000in}{0.000000in}}{%
\pgfpathmoveto{\pgfqpoint{0.000000in}{0.000000in}}%
\pgfpathlineto{\pgfqpoint{-0.027778in}{0.000000in}}%
\pgfusepath{stroke,fill}%
}%
\begin{pgfscope}%
\pgfsys@transformshift{0.625000in}{3.162684in}%
\pgfsys@useobject{currentmarker}{}%
\end{pgfscope}%
\end{pgfscope}%
\begin{pgfscope}%
\pgfsetbuttcap%
\pgfsetroundjoin%
\definecolor{currentfill}{rgb}{0.000000,0.000000,0.000000}%
\pgfsetfillcolor{currentfill}%
\pgfsetlinewidth{0.602250pt}%
\definecolor{currentstroke}{rgb}{0.000000,0.000000,0.000000}%
\pgfsetstrokecolor{currentstroke}%
\pgfsetdash{}{0pt}%
\pgfsys@defobject{currentmarker}{\pgfqpoint{-0.027778in}{0.000000in}}{\pgfqpoint{0.000000in}{0.000000in}}{%
\pgfpathmoveto{\pgfqpoint{0.000000in}{0.000000in}}%
\pgfpathlineto{\pgfqpoint{-0.027778in}{0.000000in}}%
\pgfusepath{stroke,fill}%
}%
\begin{pgfscope}%
\pgfsys@transformshift{0.625000in}{3.215563in}%
\pgfsys@useobject{currentmarker}{}%
\end{pgfscope}%
\end{pgfscope}%
\begin{pgfscope}%
\pgfsetbuttcap%
\pgfsetroundjoin%
\definecolor{currentfill}{rgb}{0.000000,0.000000,0.000000}%
\pgfsetfillcolor{currentfill}%
\pgfsetlinewidth{0.602250pt}%
\definecolor{currentstroke}{rgb}{0.000000,0.000000,0.000000}%
\pgfsetstrokecolor{currentstroke}%
\pgfsetdash{}{0pt}%
\pgfsys@defobject{currentmarker}{\pgfqpoint{-0.027778in}{0.000000in}}{\pgfqpoint{0.000000in}{0.000000in}}{%
\pgfpathmoveto{\pgfqpoint{0.000000in}{0.000000in}}%
\pgfpathlineto{\pgfqpoint{-0.027778in}{0.000000in}}%
\pgfusepath{stroke,fill}%
}%
\begin{pgfscope}%
\pgfsys@transformshift{0.625000in}{3.262206in}%
\pgfsys@useobject{currentmarker}{}%
\end{pgfscope}%
\end{pgfscope}%
\begin{pgfscope}%
\definecolor{textcolor}{rgb}{0.000000,0.000000,0.000000}%
\pgfsetstrokecolor{textcolor}%
\pgfsetfillcolor{textcolor}%
\pgftext[x=0.162343in,y=1.980000in,,bottom,rotate=90.000000]{\color{textcolor}\sffamily\fontsize{11.000000}{13.200000}\selectfont \textrm{QMC Error}}%
\end{pgfscope}%
\begin{pgfscope}%
\pgfpathrectangle{\pgfqpoint{0.625000in}{0.440000in}}{\pgfqpoint{3.875000in}{3.080000in}}%
\pgfusepath{clip}%
\pgfsetbuttcap%
\pgfsetroundjoin%
\definecolor{currentfill}{rgb}{0.000000,0.000000,0.000000}%
\pgfsetfillcolor{currentfill}%
\pgfsetlinewidth{1.003750pt}%
\definecolor{currentstroke}{rgb}{0.000000,0.000000,0.000000}%
\pgfsetstrokecolor{currentstroke}%
\pgfsetdash{}{0pt}%
\pgfsys@defobject{currentmarker}{\pgfqpoint{-0.041667in}{-0.041667in}}{\pgfqpoint{0.041667in}{0.041667in}}{%
\pgfpathmoveto{\pgfqpoint{0.000000in}{-0.041667in}}%
\pgfpathcurveto{\pgfqpoint{0.011050in}{-0.041667in}}{\pgfqpoint{0.021649in}{-0.037276in}}{\pgfqpoint{0.029463in}{-0.029463in}}%
\pgfpathcurveto{\pgfqpoint{0.037276in}{-0.021649in}}{\pgfqpoint{0.041667in}{-0.011050in}}{\pgfqpoint{0.041667in}{0.000000in}}%
\pgfpathcurveto{\pgfqpoint{0.041667in}{0.011050in}}{\pgfqpoint{0.037276in}{0.021649in}}{\pgfqpoint{0.029463in}{0.029463in}}%
\pgfpathcurveto{\pgfqpoint{0.021649in}{0.037276in}}{\pgfqpoint{0.011050in}{0.041667in}}{\pgfqpoint{0.000000in}{0.041667in}}%
\pgfpathcurveto{\pgfqpoint{-0.011050in}{0.041667in}}{\pgfqpoint{-0.021649in}{0.037276in}}{\pgfqpoint{-0.029463in}{0.029463in}}%
\pgfpathcurveto{\pgfqpoint{-0.037276in}{0.021649in}}{\pgfqpoint{-0.041667in}{0.011050in}}{\pgfqpoint{-0.041667in}{0.000000in}}%
\pgfpathcurveto{\pgfqpoint{-0.041667in}{-0.011050in}}{\pgfqpoint{-0.037276in}{-0.021649in}}{\pgfqpoint{-0.029463in}{-0.029463in}}%
\pgfpathcurveto{\pgfqpoint{-0.021649in}{-0.037276in}}{\pgfqpoint{-0.011050in}{-0.041667in}}{\pgfqpoint{0.000000in}{-0.041667in}}%
\pgfpathclose%
\pgfusepath{stroke,fill}%
}%
\begin{pgfscope}%
\pgfsys@transformshift{0.801136in}{3.291741in}%
\pgfsys@useobject{currentmarker}{}%
\end{pgfscope}%
\begin{pgfscope}%
\pgfsys@transformshift{1.121384in}{3.237733in}%
\pgfsys@useobject{currentmarker}{}%
\end{pgfscope}%
\begin{pgfscope}%
\pgfsys@transformshift{1.441632in}{2.806540in}%
\pgfsys@useobject{currentmarker}{}%
\end{pgfscope}%
\begin{pgfscope}%
\pgfsys@transformshift{1.761880in}{2.538022in}%
\pgfsys@useobject{currentmarker}{}%
\end{pgfscope}%
\begin{pgfscope}%
\pgfsys@transformshift{2.082128in}{2.359554in}%
\pgfsys@useobject{currentmarker}{}%
\end{pgfscope}%
\begin{pgfscope}%
\pgfsys@transformshift{2.402376in}{2.169793in}%
\pgfsys@useobject{currentmarker}{}%
\end{pgfscope}%
\begin{pgfscope}%
\pgfsys@transformshift{2.722624in}{2.036107in}%
\pgfsys@useobject{currentmarker}{}%
\end{pgfscope}%
\begin{pgfscope}%
\pgfsys@transformshift{3.042872in}{1.803463in}%
\pgfsys@useobject{currentmarker}{}%
\end{pgfscope}%
\begin{pgfscope}%
\pgfsys@transformshift{3.363120in}{1.371129in}%
\pgfsys@useobject{currentmarker}{}%
\end{pgfscope}%
\begin{pgfscope}%
\pgfsys@transformshift{3.683368in}{1.117751in}%
\pgfsys@useobject{currentmarker}{}%
\end{pgfscope}%
\begin{pgfscope}%
\pgfsys@transformshift{4.003616in}{0.789563in}%
\pgfsys@useobject{currentmarker}{}%
\end{pgfscope}%
\begin{pgfscope}%
\pgfsys@transformshift{4.323864in}{0.580000in}%
\pgfsys@useobject{currentmarker}{}%
\end{pgfscope}%
\end{pgfscope}%
\begin{pgfscope}%
\pgfpathrectangle{\pgfqpoint{0.625000in}{0.440000in}}{\pgfqpoint{3.875000in}{3.080000in}}%
\pgfusepath{clip}%
\pgfsetbuttcap%
\pgfsetroundjoin%
\pgfsetlinewidth{1.505625pt}%
\definecolor{currentstroke}{rgb}{0.000000,0.000000,0.000000}%
\pgfsetstrokecolor{currentstroke}%
\pgfsetdash{{5.550000pt}{2.400000pt}}{0.000000pt}%
\pgfpathmoveto{\pgfqpoint{0.801136in}{3.380000in}}%
\pgfpathlineto{\pgfqpoint{4.323864in}{0.636899in}}%
\pgfusepath{stroke}%
\end{pgfscope}%
\begin{pgfscope}%
\pgfsetrectcap%
\pgfsetmiterjoin%
\pgfsetlinewidth{0.803000pt}%
\definecolor{currentstroke}{rgb}{0.000000,0.000000,0.000000}%
\pgfsetstrokecolor{currentstroke}%
\pgfsetdash{}{0pt}%
\pgfpathmoveto{\pgfqpoint{0.625000in}{0.440000in}}%
\pgfpathlineto{\pgfqpoint{0.625000in}{3.520000in}}%
\pgfusepath{stroke}%
\end{pgfscope}%
\begin{pgfscope}%
\pgfsetrectcap%
\pgfsetmiterjoin%
\pgfsetlinewidth{0.000000pt}%
\definecolor{currentstroke}{rgb}{0.000000,0.000000,0.000000}%
\pgfsetstrokecolor{currentstroke}%
\pgfsetstrokeopacity{0.000000}%
\pgfsetdash{}{0pt}%
\pgfpathmoveto{\pgfqpoint{4.500000in}{0.440000in}}%
\pgfpathlineto{\pgfqpoint{4.500000in}{3.520000in}}%
\pgfusepath{}%
\end{pgfscope}%
\begin{pgfscope}%
\pgfsetrectcap%
\pgfsetmiterjoin%
\pgfsetlinewidth{0.803000pt}%
\definecolor{currentstroke}{rgb}{0.000000,0.000000,0.000000}%
\pgfsetstrokecolor{currentstroke}%
\pgfsetdash{}{0pt}%
\pgfpathmoveto{\pgfqpoint{0.625000in}{0.440000in}}%
\pgfpathlineto{\pgfqpoint{4.500000in}{0.440000in}}%
\pgfusepath{stroke}%
\end{pgfscope}%
\begin{pgfscope}%
\pgfsetrectcap%
\pgfsetmiterjoin%
\pgfsetlinewidth{0.000000pt}%
\definecolor{currentstroke}{rgb}{0.000000,0.000000,0.000000}%
\pgfsetstrokecolor{currentstroke}%
\pgfsetstrokeopacity{0.000000}%
\pgfsetdash{}{0pt}%
\pgfpathmoveto{\pgfqpoint{0.625000in}{3.520000in}}%
\pgfpathlineto{\pgfqpoint{4.500000in}{3.520000in}}%
\pgfusepath{}%
\end{pgfscope}%
\begin{pgfscope}%
\pgfsetbuttcap%
\pgfsetmiterjoin%
\definecolor{currentfill}{rgb}{1.000000,1.000000,1.000000}%
\pgfsetfillcolor{currentfill}%
\pgfsetfillopacity{0.800000}%
\pgfsetlinewidth{1.003750pt}%
\definecolor{currentstroke}{rgb}{0.800000,0.800000,0.800000}%
\pgfsetstrokecolor{currentstroke}%
\pgfsetstrokeopacity{0.800000}%
\pgfsetdash{}{0pt}%
\pgfpathmoveto{\pgfqpoint{2.916233in}{3.119488in}}%
\pgfpathlineto{\pgfqpoint{4.393056in}{3.119488in}}%
\pgfpathquadraticcurveto{\pgfqpoint{4.423611in}{3.119488in}}{\pgfqpoint{4.423611in}{3.150043in}}%
\pgfpathlineto{\pgfqpoint{4.423611in}{3.413056in}}%
\pgfpathquadraticcurveto{\pgfqpoint{4.423611in}{3.443611in}}{\pgfqpoint{4.393056in}{3.443611in}}%
\pgfpathlineto{\pgfqpoint{2.916233in}{3.443611in}}%
\pgfpathquadraticcurveto{\pgfqpoint{2.885678in}{3.443611in}}{\pgfqpoint{2.885678in}{3.413056in}}%
\pgfpathlineto{\pgfqpoint{2.885678in}{3.150043in}}%
\pgfpathquadraticcurveto{\pgfqpoint{2.885678in}{3.119488in}}{\pgfqpoint{2.916233in}{3.119488in}}%
\pgfpathclose%
\pgfusepath{stroke,fill}%
\end{pgfscope}%
\begin{pgfscope}%
\pgfsetbuttcap%
\pgfsetroundjoin%
\pgfsetlinewidth{1.505625pt}%
\definecolor{currentstroke}{rgb}{0.000000,0.000000,0.000000}%
\pgfsetstrokecolor{currentstroke}%
\pgfsetdash{{5.550000pt}{2.400000pt}}{0.000000pt}%
\pgfpathmoveto{\pgfqpoint{2.946789in}{3.300408in}}%
\pgfpathlineto{\pgfqpoint{3.252344in}{3.300408in}}%
\pgfusepath{stroke}%
\end{pgfscope}%
\begin{pgfscope}%
\definecolor{textcolor}{rgb}{0.000000,0.000000,0.000000}%
\pgfsetstrokecolor{textcolor}%
\pgfsetfillcolor{textcolor}%
\pgftext[x=3.374567in,y=3.246936in,left,base]{\color{textcolor}\sffamily\fontsize{11.000000}{13.200000}\selectfont \(\displaystyle 0.0012N_{\mathrm{QMC}}^{-0.9084}\)}%
\end{pgfscope}%
\end{pgfpicture}%
\makeatother%
\endgroup%

    \caption[Quasi-Monte-Carlo error, for $Q=\int_D u$ and $k=30$.]{Quasi-Monte-Carlo error with increasing number of Quasi-Monte-Carlo points, for $Q=\int_D u$ and $k=30$. \label{fig:qmcintegral30}}
\end{figure}
\begin{figure}[h]
  \centering
%% Creator: Matplotlib, PGF backend
%%
%% To include the figure in your LaTeX document, write
%%   \input{<filename>.pgf}
%%
%% Make sure the required packages are loaded in your preamble
%%   \usepackage{pgf}
%%
%% Figures using additional raster images can only be included by \input if
%% they are in the same directory as the main LaTeX file. For loading figures
%% from other directories you can use the `import` package
%%   \usepackage{import}
%% and then include the figures with
%%   \import{<path to file>}{<filename>.pgf}
%%
%% Matplotlib used the following preamble
%%   \usepackage{fontspec}
%%   \setmainfont{DejaVuSerif.ttf}[Path=/home/owen/progs/firedrake-complex/firedrake/lib/python3.5/site-packages/matplotlib/mpl-data/fonts/ttf/]
%%   \setsansfont{DejaVuSans.ttf}[Path=/home/owen/progs/firedrake-complex/firedrake/lib/python3.5/site-packages/matplotlib/mpl-data/fonts/ttf/]
%%   \setmonofont{DejaVuSansMono.ttf}[Path=/home/owen/progs/firedrake-complex/firedrake/lib/python3.5/site-packages/matplotlib/mpl-data/fonts/ttf/]
%%
\begingroup%
\makeatletter%
\begin{pgfpicture}%
\pgfpathrectangle{\pgfpointorigin}{\pgfqpoint{5.000000in}{4.000000in}}%
\pgfusepath{use as bounding box, clip}%
\begin{pgfscope}%
\pgfsetbuttcap%
\pgfsetmiterjoin%
\definecolor{currentfill}{rgb}{1.000000,1.000000,1.000000}%
\pgfsetfillcolor{currentfill}%
\pgfsetlinewidth{0.000000pt}%
\definecolor{currentstroke}{rgb}{1.000000,1.000000,1.000000}%
\pgfsetstrokecolor{currentstroke}%
\pgfsetdash{}{0pt}%
\pgfpathmoveto{\pgfqpoint{0.000000in}{0.000000in}}%
\pgfpathlineto{\pgfqpoint{5.000000in}{0.000000in}}%
\pgfpathlineto{\pgfqpoint{5.000000in}{4.000000in}}%
\pgfpathlineto{\pgfqpoint{0.000000in}{4.000000in}}%
\pgfpathclose%
\pgfusepath{fill}%
\end{pgfscope}%
\begin{pgfscope}%
\pgfsetbuttcap%
\pgfsetmiterjoin%
\definecolor{currentfill}{rgb}{1.000000,1.000000,1.000000}%
\pgfsetfillcolor{currentfill}%
\pgfsetlinewidth{0.000000pt}%
\definecolor{currentstroke}{rgb}{0.000000,0.000000,0.000000}%
\pgfsetstrokecolor{currentstroke}%
\pgfsetstrokeopacity{0.000000}%
\pgfsetdash{}{0pt}%
\pgfpathmoveto{\pgfqpoint{0.625000in}{0.440000in}}%
\pgfpathlineto{\pgfqpoint{4.500000in}{0.440000in}}%
\pgfpathlineto{\pgfqpoint{4.500000in}{3.520000in}}%
\pgfpathlineto{\pgfqpoint{0.625000in}{3.520000in}}%
\pgfpathclose%
\pgfusepath{fill}%
\end{pgfscope}%
\begin{pgfscope}%
\pgfsetbuttcap%
\pgfsetroundjoin%
\definecolor{currentfill}{rgb}{0.000000,0.000000,0.000000}%
\pgfsetfillcolor{currentfill}%
\pgfsetlinewidth{0.803000pt}%
\definecolor{currentstroke}{rgb}{0.000000,0.000000,0.000000}%
\pgfsetstrokecolor{currentstroke}%
\pgfsetdash{}{0pt}%
\pgfsys@defobject{currentmarker}{\pgfqpoint{0.000000in}{-0.048611in}}{\pgfqpoint{0.000000in}{0.000000in}}{%
\pgfpathmoveto{\pgfqpoint{0.000000in}{0.000000in}}%
\pgfpathlineto{\pgfqpoint{0.000000in}{-0.048611in}}%
\pgfusepath{stroke,fill}%
}%
\begin{pgfscope}%
\pgfsys@transformshift{0.801136in}{0.440000in}%
\pgfsys@useobject{currentmarker}{}%
\end{pgfscope}%
\end{pgfscope}%
\begin{pgfscope}%
\definecolor{textcolor}{rgb}{0.000000,0.000000,0.000000}%
\pgfsetstrokecolor{textcolor}%
\pgfsetfillcolor{textcolor}%
\pgftext[x=0.801136in,y=0.342778in,,top]{\color{textcolor}\sffamily\fontsize{10.000000}{12.000000}\selectfont \(\displaystyle {10^{0}}\)}%
\end{pgfscope}%
\begin{pgfscope}%
\pgfsetbuttcap%
\pgfsetroundjoin%
\definecolor{currentfill}{rgb}{0.000000,0.000000,0.000000}%
\pgfsetfillcolor{currentfill}%
\pgfsetlinewidth{0.803000pt}%
\definecolor{currentstroke}{rgb}{0.000000,0.000000,0.000000}%
\pgfsetstrokecolor{currentstroke}%
\pgfsetdash{}{0pt}%
\pgfsys@defobject{currentmarker}{\pgfqpoint{0.000000in}{-0.048611in}}{\pgfqpoint{0.000000in}{0.000000in}}{%
\pgfpathmoveto{\pgfqpoint{0.000000in}{0.000000in}}%
\pgfpathlineto{\pgfqpoint{0.000000in}{-0.048611in}}%
\pgfusepath{stroke,fill}%
}%
\begin{pgfscope}%
\pgfsys@transformshift{1.864977in}{0.440000in}%
\pgfsys@useobject{currentmarker}{}%
\end{pgfscope}%
\end{pgfscope}%
\begin{pgfscope}%
\definecolor{textcolor}{rgb}{0.000000,0.000000,0.000000}%
\pgfsetstrokecolor{textcolor}%
\pgfsetfillcolor{textcolor}%
\pgftext[x=1.864977in,y=0.342778in,,top]{\color{textcolor}\sffamily\fontsize{10.000000}{12.000000}\selectfont \(\displaystyle {10^{1}}\)}%
\end{pgfscope}%
\begin{pgfscope}%
\pgfsetbuttcap%
\pgfsetroundjoin%
\definecolor{currentfill}{rgb}{0.000000,0.000000,0.000000}%
\pgfsetfillcolor{currentfill}%
\pgfsetlinewidth{0.803000pt}%
\definecolor{currentstroke}{rgb}{0.000000,0.000000,0.000000}%
\pgfsetstrokecolor{currentstroke}%
\pgfsetdash{}{0pt}%
\pgfsys@defobject{currentmarker}{\pgfqpoint{0.000000in}{-0.048611in}}{\pgfqpoint{0.000000in}{0.000000in}}{%
\pgfpathmoveto{\pgfqpoint{0.000000in}{0.000000in}}%
\pgfpathlineto{\pgfqpoint{0.000000in}{-0.048611in}}%
\pgfusepath{stroke,fill}%
}%
\begin{pgfscope}%
\pgfsys@transformshift{2.928818in}{0.440000in}%
\pgfsys@useobject{currentmarker}{}%
\end{pgfscope}%
\end{pgfscope}%
\begin{pgfscope}%
\definecolor{textcolor}{rgb}{0.000000,0.000000,0.000000}%
\pgfsetstrokecolor{textcolor}%
\pgfsetfillcolor{textcolor}%
\pgftext[x=2.928818in,y=0.342778in,,top]{\color{textcolor}\sffamily\fontsize{10.000000}{12.000000}\selectfont \(\displaystyle {10^{2}}\)}%
\end{pgfscope}%
\begin{pgfscope}%
\pgfsetbuttcap%
\pgfsetroundjoin%
\definecolor{currentfill}{rgb}{0.000000,0.000000,0.000000}%
\pgfsetfillcolor{currentfill}%
\pgfsetlinewidth{0.803000pt}%
\definecolor{currentstroke}{rgb}{0.000000,0.000000,0.000000}%
\pgfsetstrokecolor{currentstroke}%
\pgfsetdash{}{0pt}%
\pgfsys@defobject{currentmarker}{\pgfqpoint{0.000000in}{-0.048611in}}{\pgfqpoint{0.000000in}{0.000000in}}{%
\pgfpathmoveto{\pgfqpoint{0.000000in}{0.000000in}}%
\pgfpathlineto{\pgfqpoint{0.000000in}{-0.048611in}}%
\pgfusepath{stroke,fill}%
}%
\begin{pgfscope}%
\pgfsys@transformshift{3.992658in}{0.440000in}%
\pgfsys@useobject{currentmarker}{}%
\end{pgfscope}%
\end{pgfscope}%
\begin{pgfscope}%
\definecolor{textcolor}{rgb}{0.000000,0.000000,0.000000}%
\pgfsetstrokecolor{textcolor}%
\pgfsetfillcolor{textcolor}%
\pgftext[x=3.992658in,y=0.342778in,,top]{\color{textcolor}\sffamily\fontsize{10.000000}{12.000000}\selectfont \(\displaystyle {10^{3}}\)}%
\end{pgfscope}%
\begin{pgfscope}%
\pgfsetbuttcap%
\pgfsetroundjoin%
\definecolor{currentfill}{rgb}{0.000000,0.000000,0.000000}%
\pgfsetfillcolor{currentfill}%
\pgfsetlinewidth{0.602250pt}%
\definecolor{currentstroke}{rgb}{0.000000,0.000000,0.000000}%
\pgfsetstrokecolor{currentstroke}%
\pgfsetdash{}{0pt}%
\pgfsys@defobject{currentmarker}{\pgfqpoint{0.000000in}{-0.027778in}}{\pgfqpoint{0.000000in}{0.000000in}}{%
\pgfpathmoveto{\pgfqpoint{0.000000in}{0.000000in}}%
\pgfpathlineto{\pgfqpoint{0.000000in}{-0.027778in}}%
\pgfusepath{stroke,fill}%
}%
\begin{pgfscope}%
\pgfsys@transformshift{0.636345in}{0.440000in}%
\pgfsys@useobject{currentmarker}{}%
\end{pgfscope}%
\end{pgfscope}%
\begin{pgfscope}%
\pgfsetbuttcap%
\pgfsetroundjoin%
\definecolor{currentfill}{rgb}{0.000000,0.000000,0.000000}%
\pgfsetfillcolor{currentfill}%
\pgfsetlinewidth{0.602250pt}%
\definecolor{currentstroke}{rgb}{0.000000,0.000000,0.000000}%
\pgfsetstrokecolor{currentstroke}%
\pgfsetdash{}{0pt}%
\pgfsys@defobject{currentmarker}{\pgfqpoint{0.000000in}{-0.027778in}}{\pgfqpoint{0.000000in}{0.000000in}}{%
\pgfpathmoveto{\pgfqpoint{0.000000in}{0.000000in}}%
\pgfpathlineto{\pgfqpoint{0.000000in}{-0.027778in}}%
\pgfusepath{stroke,fill}%
}%
\begin{pgfscope}%
\pgfsys@transformshift{0.698040in}{0.440000in}%
\pgfsys@useobject{currentmarker}{}%
\end{pgfscope}%
\end{pgfscope}%
\begin{pgfscope}%
\pgfsetbuttcap%
\pgfsetroundjoin%
\definecolor{currentfill}{rgb}{0.000000,0.000000,0.000000}%
\pgfsetfillcolor{currentfill}%
\pgfsetlinewidth{0.602250pt}%
\definecolor{currentstroke}{rgb}{0.000000,0.000000,0.000000}%
\pgfsetstrokecolor{currentstroke}%
\pgfsetdash{}{0pt}%
\pgfsys@defobject{currentmarker}{\pgfqpoint{0.000000in}{-0.027778in}}{\pgfqpoint{0.000000in}{0.000000in}}{%
\pgfpathmoveto{\pgfqpoint{0.000000in}{0.000000in}}%
\pgfpathlineto{\pgfqpoint{0.000000in}{-0.027778in}}%
\pgfusepath{stroke,fill}%
}%
\begin{pgfscope}%
\pgfsys@transformshift{0.752458in}{0.440000in}%
\pgfsys@useobject{currentmarker}{}%
\end{pgfscope}%
\end{pgfscope}%
\begin{pgfscope}%
\pgfsetbuttcap%
\pgfsetroundjoin%
\definecolor{currentfill}{rgb}{0.000000,0.000000,0.000000}%
\pgfsetfillcolor{currentfill}%
\pgfsetlinewidth{0.602250pt}%
\definecolor{currentstroke}{rgb}{0.000000,0.000000,0.000000}%
\pgfsetstrokecolor{currentstroke}%
\pgfsetdash{}{0pt}%
\pgfsys@defobject{currentmarker}{\pgfqpoint{0.000000in}{-0.027778in}}{\pgfqpoint{0.000000in}{0.000000in}}{%
\pgfpathmoveto{\pgfqpoint{0.000000in}{0.000000in}}%
\pgfpathlineto{\pgfqpoint{0.000000in}{-0.027778in}}%
\pgfusepath{stroke,fill}%
}%
\begin{pgfscope}%
\pgfsys@transformshift{1.121384in}{0.440000in}%
\pgfsys@useobject{currentmarker}{}%
\end{pgfscope}%
\end{pgfscope}%
\begin{pgfscope}%
\pgfsetbuttcap%
\pgfsetroundjoin%
\definecolor{currentfill}{rgb}{0.000000,0.000000,0.000000}%
\pgfsetfillcolor{currentfill}%
\pgfsetlinewidth{0.602250pt}%
\definecolor{currentstroke}{rgb}{0.000000,0.000000,0.000000}%
\pgfsetstrokecolor{currentstroke}%
\pgfsetdash{}{0pt}%
\pgfsys@defobject{currentmarker}{\pgfqpoint{0.000000in}{-0.027778in}}{\pgfqpoint{0.000000in}{0.000000in}}{%
\pgfpathmoveto{\pgfqpoint{0.000000in}{0.000000in}}%
\pgfpathlineto{\pgfqpoint{0.000000in}{-0.027778in}}%
\pgfusepath{stroke,fill}%
}%
\begin{pgfscope}%
\pgfsys@transformshift{1.308717in}{0.440000in}%
\pgfsys@useobject{currentmarker}{}%
\end{pgfscope}%
\end{pgfscope}%
\begin{pgfscope}%
\pgfsetbuttcap%
\pgfsetroundjoin%
\definecolor{currentfill}{rgb}{0.000000,0.000000,0.000000}%
\pgfsetfillcolor{currentfill}%
\pgfsetlinewidth{0.602250pt}%
\definecolor{currentstroke}{rgb}{0.000000,0.000000,0.000000}%
\pgfsetstrokecolor{currentstroke}%
\pgfsetdash{}{0pt}%
\pgfsys@defobject{currentmarker}{\pgfqpoint{0.000000in}{-0.027778in}}{\pgfqpoint{0.000000in}{0.000000in}}{%
\pgfpathmoveto{\pgfqpoint{0.000000in}{0.000000in}}%
\pgfpathlineto{\pgfqpoint{0.000000in}{-0.027778in}}%
\pgfusepath{stroke,fill}%
}%
\begin{pgfscope}%
\pgfsys@transformshift{1.441632in}{0.440000in}%
\pgfsys@useobject{currentmarker}{}%
\end{pgfscope}%
\end{pgfscope}%
\begin{pgfscope}%
\pgfsetbuttcap%
\pgfsetroundjoin%
\definecolor{currentfill}{rgb}{0.000000,0.000000,0.000000}%
\pgfsetfillcolor{currentfill}%
\pgfsetlinewidth{0.602250pt}%
\definecolor{currentstroke}{rgb}{0.000000,0.000000,0.000000}%
\pgfsetstrokecolor{currentstroke}%
\pgfsetdash{}{0pt}%
\pgfsys@defobject{currentmarker}{\pgfqpoint{0.000000in}{-0.027778in}}{\pgfqpoint{0.000000in}{0.000000in}}{%
\pgfpathmoveto{\pgfqpoint{0.000000in}{0.000000in}}%
\pgfpathlineto{\pgfqpoint{0.000000in}{-0.027778in}}%
\pgfusepath{stroke,fill}%
}%
\begin{pgfscope}%
\pgfsys@transformshift{1.544729in}{0.440000in}%
\pgfsys@useobject{currentmarker}{}%
\end{pgfscope}%
\end{pgfscope}%
\begin{pgfscope}%
\pgfsetbuttcap%
\pgfsetroundjoin%
\definecolor{currentfill}{rgb}{0.000000,0.000000,0.000000}%
\pgfsetfillcolor{currentfill}%
\pgfsetlinewidth{0.602250pt}%
\definecolor{currentstroke}{rgb}{0.000000,0.000000,0.000000}%
\pgfsetstrokecolor{currentstroke}%
\pgfsetdash{}{0pt}%
\pgfsys@defobject{currentmarker}{\pgfqpoint{0.000000in}{-0.027778in}}{\pgfqpoint{0.000000in}{0.000000in}}{%
\pgfpathmoveto{\pgfqpoint{0.000000in}{0.000000in}}%
\pgfpathlineto{\pgfqpoint{0.000000in}{-0.027778in}}%
\pgfusepath{stroke,fill}%
}%
\begin{pgfscope}%
\pgfsys@transformshift{1.628965in}{0.440000in}%
\pgfsys@useobject{currentmarker}{}%
\end{pgfscope}%
\end{pgfscope}%
\begin{pgfscope}%
\pgfsetbuttcap%
\pgfsetroundjoin%
\definecolor{currentfill}{rgb}{0.000000,0.000000,0.000000}%
\pgfsetfillcolor{currentfill}%
\pgfsetlinewidth{0.602250pt}%
\definecolor{currentstroke}{rgb}{0.000000,0.000000,0.000000}%
\pgfsetstrokecolor{currentstroke}%
\pgfsetdash{}{0pt}%
\pgfsys@defobject{currentmarker}{\pgfqpoint{0.000000in}{-0.027778in}}{\pgfqpoint{0.000000in}{0.000000in}}{%
\pgfpathmoveto{\pgfqpoint{0.000000in}{0.000000in}}%
\pgfpathlineto{\pgfqpoint{0.000000in}{-0.027778in}}%
\pgfusepath{stroke,fill}%
}%
\begin{pgfscope}%
\pgfsys@transformshift{1.700186in}{0.440000in}%
\pgfsys@useobject{currentmarker}{}%
\end{pgfscope}%
\end{pgfscope}%
\begin{pgfscope}%
\pgfsetbuttcap%
\pgfsetroundjoin%
\definecolor{currentfill}{rgb}{0.000000,0.000000,0.000000}%
\pgfsetfillcolor{currentfill}%
\pgfsetlinewidth{0.602250pt}%
\definecolor{currentstroke}{rgb}{0.000000,0.000000,0.000000}%
\pgfsetstrokecolor{currentstroke}%
\pgfsetdash{}{0pt}%
\pgfsys@defobject{currentmarker}{\pgfqpoint{0.000000in}{-0.027778in}}{\pgfqpoint{0.000000in}{0.000000in}}{%
\pgfpathmoveto{\pgfqpoint{0.000000in}{0.000000in}}%
\pgfpathlineto{\pgfqpoint{0.000000in}{-0.027778in}}%
\pgfusepath{stroke,fill}%
}%
\begin{pgfscope}%
\pgfsys@transformshift{1.761880in}{0.440000in}%
\pgfsys@useobject{currentmarker}{}%
\end{pgfscope}%
\end{pgfscope}%
\begin{pgfscope}%
\pgfsetbuttcap%
\pgfsetroundjoin%
\definecolor{currentfill}{rgb}{0.000000,0.000000,0.000000}%
\pgfsetfillcolor{currentfill}%
\pgfsetlinewidth{0.602250pt}%
\definecolor{currentstroke}{rgb}{0.000000,0.000000,0.000000}%
\pgfsetstrokecolor{currentstroke}%
\pgfsetdash{}{0pt}%
\pgfsys@defobject{currentmarker}{\pgfqpoint{0.000000in}{-0.027778in}}{\pgfqpoint{0.000000in}{0.000000in}}{%
\pgfpathmoveto{\pgfqpoint{0.000000in}{0.000000in}}%
\pgfpathlineto{\pgfqpoint{0.000000in}{-0.027778in}}%
\pgfusepath{stroke,fill}%
}%
\begin{pgfscope}%
\pgfsys@transformshift{1.816298in}{0.440000in}%
\pgfsys@useobject{currentmarker}{}%
\end{pgfscope}%
\end{pgfscope}%
\begin{pgfscope}%
\pgfsetbuttcap%
\pgfsetroundjoin%
\definecolor{currentfill}{rgb}{0.000000,0.000000,0.000000}%
\pgfsetfillcolor{currentfill}%
\pgfsetlinewidth{0.602250pt}%
\definecolor{currentstroke}{rgb}{0.000000,0.000000,0.000000}%
\pgfsetstrokecolor{currentstroke}%
\pgfsetdash{}{0pt}%
\pgfsys@defobject{currentmarker}{\pgfqpoint{0.000000in}{-0.027778in}}{\pgfqpoint{0.000000in}{0.000000in}}{%
\pgfpathmoveto{\pgfqpoint{0.000000in}{0.000000in}}%
\pgfpathlineto{\pgfqpoint{0.000000in}{-0.027778in}}%
\pgfusepath{stroke,fill}%
}%
\begin{pgfscope}%
\pgfsys@transformshift{2.185225in}{0.440000in}%
\pgfsys@useobject{currentmarker}{}%
\end{pgfscope}%
\end{pgfscope}%
\begin{pgfscope}%
\pgfsetbuttcap%
\pgfsetroundjoin%
\definecolor{currentfill}{rgb}{0.000000,0.000000,0.000000}%
\pgfsetfillcolor{currentfill}%
\pgfsetlinewidth{0.602250pt}%
\definecolor{currentstroke}{rgb}{0.000000,0.000000,0.000000}%
\pgfsetstrokecolor{currentstroke}%
\pgfsetdash{}{0pt}%
\pgfsys@defobject{currentmarker}{\pgfqpoint{0.000000in}{-0.027778in}}{\pgfqpoint{0.000000in}{0.000000in}}{%
\pgfpathmoveto{\pgfqpoint{0.000000in}{0.000000in}}%
\pgfpathlineto{\pgfqpoint{0.000000in}{-0.027778in}}%
\pgfusepath{stroke,fill}%
}%
\begin{pgfscope}%
\pgfsys@transformshift{2.372558in}{0.440000in}%
\pgfsys@useobject{currentmarker}{}%
\end{pgfscope}%
\end{pgfscope}%
\begin{pgfscope}%
\pgfsetbuttcap%
\pgfsetroundjoin%
\definecolor{currentfill}{rgb}{0.000000,0.000000,0.000000}%
\pgfsetfillcolor{currentfill}%
\pgfsetlinewidth{0.602250pt}%
\definecolor{currentstroke}{rgb}{0.000000,0.000000,0.000000}%
\pgfsetstrokecolor{currentstroke}%
\pgfsetdash{}{0pt}%
\pgfsys@defobject{currentmarker}{\pgfqpoint{0.000000in}{-0.027778in}}{\pgfqpoint{0.000000in}{0.000000in}}{%
\pgfpathmoveto{\pgfqpoint{0.000000in}{0.000000in}}%
\pgfpathlineto{\pgfqpoint{0.000000in}{-0.027778in}}%
\pgfusepath{stroke,fill}%
}%
\begin{pgfscope}%
\pgfsys@transformshift{2.505473in}{0.440000in}%
\pgfsys@useobject{currentmarker}{}%
\end{pgfscope}%
\end{pgfscope}%
\begin{pgfscope}%
\pgfsetbuttcap%
\pgfsetroundjoin%
\definecolor{currentfill}{rgb}{0.000000,0.000000,0.000000}%
\pgfsetfillcolor{currentfill}%
\pgfsetlinewidth{0.602250pt}%
\definecolor{currentstroke}{rgb}{0.000000,0.000000,0.000000}%
\pgfsetstrokecolor{currentstroke}%
\pgfsetdash{}{0pt}%
\pgfsys@defobject{currentmarker}{\pgfqpoint{0.000000in}{-0.027778in}}{\pgfqpoint{0.000000in}{0.000000in}}{%
\pgfpathmoveto{\pgfqpoint{0.000000in}{0.000000in}}%
\pgfpathlineto{\pgfqpoint{0.000000in}{-0.027778in}}%
\pgfusepath{stroke,fill}%
}%
\begin{pgfscope}%
\pgfsys@transformshift{2.608570in}{0.440000in}%
\pgfsys@useobject{currentmarker}{}%
\end{pgfscope}%
\end{pgfscope}%
\begin{pgfscope}%
\pgfsetbuttcap%
\pgfsetroundjoin%
\definecolor{currentfill}{rgb}{0.000000,0.000000,0.000000}%
\pgfsetfillcolor{currentfill}%
\pgfsetlinewidth{0.602250pt}%
\definecolor{currentstroke}{rgb}{0.000000,0.000000,0.000000}%
\pgfsetstrokecolor{currentstroke}%
\pgfsetdash{}{0pt}%
\pgfsys@defobject{currentmarker}{\pgfqpoint{0.000000in}{-0.027778in}}{\pgfqpoint{0.000000in}{0.000000in}}{%
\pgfpathmoveto{\pgfqpoint{0.000000in}{0.000000in}}%
\pgfpathlineto{\pgfqpoint{0.000000in}{-0.027778in}}%
\pgfusepath{stroke,fill}%
}%
\begin{pgfscope}%
\pgfsys@transformshift{2.692806in}{0.440000in}%
\pgfsys@useobject{currentmarker}{}%
\end{pgfscope}%
\end{pgfscope}%
\begin{pgfscope}%
\pgfsetbuttcap%
\pgfsetroundjoin%
\definecolor{currentfill}{rgb}{0.000000,0.000000,0.000000}%
\pgfsetfillcolor{currentfill}%
\pgfsetlinewidth{0.602250pt}%
\definecolor{currentstroke}{rgb}{0.000000,0.000000,0.000000}%
\pgfsetstrokecolor{currentstroke}%
\pgfsetdash{}{0pt}%
\pgfsys@defobject{currentmarker}{\pgfqpoint{0.000000in}{-0.027778in}}{\pgfqpoint{0.000000in}{0.000000in}}{%
\pgfpathmoveto{\pgfqpoint{0.000000in}{0.000000in}}%
\pgfpathlineto{\pgfqpoint{0.000000in}{-0.027778in}}%
\pgfusepath{stroke,fill}%
}%
\begin{pgfscope}%
\pgfsys@transformshift{2.764027in}{0.440000in}%
\pgfsys@useobject{currentmarker}{}%
\end{pgfscope}%
\end{pgfscope}%
\begin{pgfscope}%
\pgfsetbuttcap%
\pgfsetroundjoin%
\definecolor{currentfill}{rgb}{0.000000,0.000000,0.000000}%
\pgfsetfillcolor{currentfill}%
\pgfsetlinewidth{0.602250pt}%
\definecolor{currentstroke}{rgb}{0.000000,0.000000,0.000000}%
\pgfsetstrokecolor{currentstroke}%
\pgfsetdash{}{0pt}%
\pgfsys@defobject{currentmarker}{\pgfqpoint{0.000000in}{-0.027778in}}{\pgfqpoint{0.000000in}{0.000000in}}{%
\pgfpathmoveto{\pgfqpoint{0.000000in}{0.000000in}}%
\pgfpathlineto{\pgfqpoint{0.000000in}{-0.027778in}}%
\pgfusepath{stroke,fill}%
}%
\begin{pgfscope}%
\pgfsys@transformshift{2.825721in}{0.440000in}%
\pgfsys@useobject{currentmarker}{}%
\end{pgfscope}%
\end{pgfscope}%
\begin{pgfscope}%
\pgfsetbuttcap%
\pgfsetroundjoin%
\definecolor{currentfill}{rgb}{0.000000,0.000000,0.000000}%
\pgfsetfillcolor{currentfill}%
\pgfsetlinewidth{0.602250pt}%
\definecolor{currentstroke}{rgb}{0.000000,0.000000,0.000000}%
\pgfsetstrokecolor{currentstroke}%
\pgfsetdash{}{0pt}%
\pgfsys@defobject{currentmarker}{\pgfqpoint{0.000000in}{-0.027778in}}{\pgfqpoint{0.000000in}{0.000000in}}{%
\pgfpathmoveto{\pgfqpoint{0.000000in}{0.000000in}}%
\pgfpathlineto{\pgfqpoint{0.000000in}{-0.027778in}}%
\pgfusepath{stroke,fill}%
}%
\begin{pgfscope}%
\pgfsys@transformshift{2.880139in}{0.440000in}%
\pgfsys@useobject{currentmarker}{}%
\end{pgfscope}%
\end{pgfscope}%
\begin{pgfscope}%
\pgfsetbuttcap%
\pgfsetroundjoin%
\definecolor{currentfill}{rgb}{0.000000,0.000000,0.000000}%
\pgfsetfillcolor{currentfill}%
\pgfsetlinewidth{0.602250pt}%
\definecolor{currentstroke}{rgb}{0.000000,0.000000,0.000000}%
\pgfsetstrokecolor{currentstroke}%
\pgfsetdash{}{0pt}%
\pgfsys@defobject{currentmarker}{\pgfqpoint{0.000000in}{-0.027778in}}{\pgfqpoint{0.000000in}{0.000000in}}{%
\pgfpathmoveto{\pgfqpoint{0.000000in}{0.000000in}}%
\pgfpathlineto{\pgfqpoint{0.000000in}{-0.027778in}}%
\pgfusepath{stroke,fill}%
}%
\begin{pgfscope}%
\pgfsys@transformshift{3.249066in}{0.440000in}%
\pgfsys@useobject{currentmarker}{}%
\end{pgfscope}%
\end{pgfscope}%
\begin{pgfscope}%
\pgfsetbuttcap%
\pgfsetroundjoin%
\definecolor{currentfill}{rgb}{0.000000,0.000000,0.000000}%
\pgfsetfillcolor{currentfill}%
\pgfsetlinewidth{0.602250pt}%
\definecolor{currentstroke}{rgb}{0.000000,0.000000,0.000000}%
\pgfsetstrokecolor{currentstroke}%
\pgfsetdash{}{0pt}%
\pgfsys@defobject{currentmarker}{\pgfqpoint{0.000000in}{-0.027778in}}{\pgfqpoint{0.000000in}{0.000000in}}{%
\pgfpathmoveto{\pgfqpoint{0.000000in}{0.000000in}}%
\pgfpathlineto{\pgfqpoint{0.000000in}{-0.027778in}}%
\pgfusepath{stroke,fill}%
}%
\begin{pgfscope}%
\pgfsys@transformshift{3.436399in}{0.440000in}%
\pgfsys@useobject{currentmarker}{}%
\end{pgfscope}%
\end{pgfscope}%
\begin{pgfscope}%
\pgfsetbuttcap%
\pgfsetroundjoin%
\definecolor{currentfill}{rgb}{0.000000,0.000000,0.000000}%
\pgfsetfillcolor{currentfill}%
\pgfsetlinewidth{0.602250pt}%
\definecolor{currentstroke}{rgb}{0.000000,0.000000,0.000000}%
\pgfsetstrokecolor{currentstroke}%
\pgfsetdash{}{0pt}%
\pgfsys@defobject{currentmarker}{\pgfqpoint{0.000000in}{-0.027778in}}{\pgfqpoint{0.000000in}{0.000000in}}{%
\pgfpathmoveto{\pgfqpoint{0.000000in}{0.000000in}}%
\pgfpathlineto{\pgfqpoint{0.000000in}{-0.027778in}}%
\pgfusepath{stroke,fill}%
}%
\begin{pgfscope}%
\pgfsys@transformshift{3.569313in}{0.440000in}%
\pgfsys@useobject{currentmarker}{}%
\end{pgfscope}%
\end{pgfscope}%
\begin{pgfscope}%
\pgfsetbuttcap%
\pgfsetroundjoin%
\definecolor{currentfill}{rgb}{0.000000,0.000000,0.000000}%
\pgfsetfillcolor{currentfill}%
\pgfsetlinewidth{0.602250pt}%
\definecolor{currentstroke}{rgb}{0.000000,0.000000,0.000000}%
\pgfsetstrokecolor{currentstroke}%
\pgfsetdash{}{0pt}%
\pgfsys@defobject{currentmarker}{\pgfqpoint{0.000000in}{-0.027778in}}{\pgfqpoint{0.000000in}{0.000000in}}{%
\pgfpathmoveto{\pgfqpoint{0.000000in}{0.000000in}}%
\pgfpathlineto{\pgfqpoint{0.000000in}{-0.027778in}}%
\pgfusepath{stroke,fill}%
}%
\begin{pgfscope}%
\pgfsys@transformshift{3.672410in}{0.440000in}%
\pgfsys@useobject{currentmarker}{}%
\end{pgfscope}%
\end{pgfscope}%
\begin{pgfscope}%
\pgfsetbuttcap%
\pgfsetroundjoin%
\definecolor{currentfill}{rgb}{0.000000,0.000000,0.000000}%
\pgfsetfillcolor{currentfill}%
\pgfsetlinewidth{0.602250pt}%
\definecolor{currentstroke}{rgb}{0.000000,0.000000,0.000000}%
\pgfsetstrokecolor{currentstroke}%
\pgfsetdash{}{0pt}%
\pgfsys@defobject{currentmarker}{\pgfqpoint{0.000000in}{-0.027778in}}{\pgfqpoint{0.000000in}{0.000000in}}{%
\pgfpathmoveto{\pgfqpoint{0.000000in}{0.000000in}}%
\pgfpathlineto{\pgfqpoint{0.000000in}{-0.027778in}}%
\pgfusepath{stroke,fill}%
}%
\begin{pgfscope}%
\pgfsys@transformshift{3.756646in}{0.440000in}%
\pgfsys@useobject{currentmarker}{}%
\end{pgfscope}%
\end{pgfscope}%
\begin{pgfscope}%
\pgfsetbuttcap%
\pgfsetroundjoin%
\definecolor{currentfill}{rgb}{0.000000,0.000000,0.000000}%
\pgfsetfillcolor{currentfill}%
\pgfsetlinewidth{0.602250pt}%
\definecolor{currentstroke}{rgb}{0.000000,0.000000,0.000000}%
\pgfsetstrokecolor{currentstroke}%
\pgfsetdash{}{0pt}%
\pgfsys@defobject{currentmarker}{\pgfqpoint{0.000000in}{-0.027778in}}{\pgfqpoint{0.000000in}{0.000000in}}{%
\pgfpathmoveto{\pgfqpoint{0.000000in}{0.000000in}}%
\pgfpathlineto{\pgfqpoint{0.000000in}{-0.027778in}}%
\pgfusepath{stroke,fill}%
}%
\begin{pgfscope}%
\pgfsys@transformshift{3.827867in}{0.440000in}%
\pgfsys@useobject{currentmarker}{}%
\end{pgfscope}%
\end{pgfscope}%
\begin{pgfscope}%
\pgfsetbuttcap%
\pgfsetroundjoin%
\definecolor{currentfill}{rgb}{0.000000,0.000000,0.000000}%
\pgfsetfillcolor{currentfill}%
\pgfsetlinewidth{0.602250pt}%
\definecolor{currentstroke}{rgb}{0.000000,0.000000,0.000000}%
\pgfsetstrokecolor{currentstroke}%
\pgfsetdash{}{0pt}%
\pgfsys@defobject{currentmarker}{\pgfqpoint{0.000000in}{-0.027778in}}{\pgfqpoint{0.000000in}{0.000000in}}{%
\pgfpathmoveto{\pgfqpoint{0.000000in}{0.000000in}}%
\pgfpathlineto{\pgfqpoint{0.000000in}{-0.027778in}}%
\pgfusepath{stroke,fill}%
}%
\begin{pgfscope}%
\pgfsys@transformshift{3.889561in}{0.440000in}%
\pgfsys@useobject{currentmarker}{}%
\end{pgfscope}%
\end{pgfscope}%
\begin{pgfscope}%
\pgfsetbuttcap%
\pgfsetroundjoin%
\definecolor{currentfill}{rgb}{0.000000,0.000000,0.000000}%
\pgfsetfillcolor{currentfill}%
\pgfsetlinewidth{0.602250pt}%
\definecolor{currentstroke}{rgb}{0.000000,0.000000,0.000000}%
\pgfsetstrokecolor{currentstroke}%
\pgfsetdash{}{0pt}%
\pgfsys@defobject{currentmarker}{\pgfqpoint{0.000000in}{-0.027778in}}{\pgfqpoint{0.000000in}{0.000000in}}{%
\pgfpathmoveto{\pgfqpoint{0.000000in}{0.000000in}}%
\pgfpathlineto{\pgfqpoint{0.000000in}{-0.027778in}}%
\pgfusepath{stroke,fill}%
}%
\begin{pgfscope}%
\pgfsys@transformshift{3.943980in}{0.440000in}%
\pgfsys@useobject{currentmarker}{}%
\end{pgfscope}%
\end{pgfscope}%
\begin{pgfscope}%
\pgfsetbuttcap%
\pgfsetroundjoin%
\definecolor{currentfill}{rgb}{0.000000,0.000000,0.000000}%
\pgfsetfillcolor{currentfill}%
\pgfsetlinewidth{0.602250pt}%
\definecolor{currentstroke}{rgb}{0.000000,0.000000,0.000000}%
\pgfsetstrokecolor{currentstroke}%
\pgfsetdash{}{0pt}%
\pgfsys@defobject{currentmarker}{\pgfqpoint{0.000000in}{-0.027778in}}{\pgfqpoint{0.000000in}{0.000000in}}{%
\pgfpathmoveto{\pgfqpoint{0.000000in}{0.000000in}}%
\pgfpathlineto{\pgfqpoint{0.000000in}{-0.027778in}}%
\pgfusepath{stroke,fill}%
}%
\begin{pgfscope}%
\pgfsys@transformshift{4.312906in}{0.440000in}%
\pgfsys@useobject{currentmarker}{}%
\end{pgfscope}%
\end{pgfscope}%
\begin{pgfscope}%
\pgfsetbuttcap%
\pgfsetroundjoin%
\definecolor{currentfill}{rgb}{0.000000,0.000000,0.000000}%
\pgfsetfillcolor{currentfill}%
\pgfsetlinewidth{0.602250pt}%
\definecolor{currentstroke}{rgb}{0.000000,0.000000,0.000000}%
\pgfsetstrokecolor{currentstroke}%
\pgfsetdash{}{0pt}%
\pgfsys@defobject{currentmarker}{\pgfqpoint{0.000000in}{-0.027778in}}{\pgfqpoint{0.000000in}{0.000000in}}{%
\pgfpathmoveto{\pgfqpoint{0.000000in}{0.000000in}}%
\pgfpathlineto{\pgfqpoint{0.000000in}{-0.027778in}}%
\pgfusepath{stroke,fill}%
}%
\begin{pgfscope}%
\pgfsys@transformshift{4.500239in}{0.440000in}%
\pgfsys@useobject{currentmarker}{}%
\end{pgfscope}%
\end{pgfscope}%
\begin{pgfscope}%
\definecolor{textcolor}{rgb}{0.000000,0.000000,0.000000}%
\pgfsetstrokecolor{textcolor}%
\pgfsetfillcolor{textcolor}%
\pgftext[x=2.562500in,y=0.152809in,,top]{\color{textcolor}\sffamily\fontsize{10.000000}{12.000000}\selectfont \(\displaystyle N_\mathrm{QMC}\)}%
\end{pgfscope}%
\begin{pgfscope}%
\pgfsetbuttcap%
\pgfsetroundjoin%
\definecolor{currentfill}{rgb}{0.000000,0.000000,0.000000}%
\pgfsetfillcolor{currentfill}%
\pgfsetlinewidth{0.803000pt}%
\definecolor{currentstroke}{rgb}{0.000000,0.000000,0.000000}%
\pgfsetstrokecolor{currentstroke}%
\pgfsetdash{}{0pt}%
\pgfsys@defobject{currentmarker}{\pgfqpoint{-0.048611in}{0.000000in}}{\pgfqpoint{0.000000in}{0.000000in}}{%
\pgfpathmoveto{\pgfqpoint{0.000000in}{0.000000in}}%
\pgfpathlineto{\pgfqpoint{-0.048611in}{0.000000in}}%
\pgfusepath{stroke,fill}%
}%
\begin{pgfscope}%
\pgfsys@transformshift{0.625000in}{0.633396in}%
\pgfsys@useobject{currentmarker}{}%
\end{pgfscope}%
\end{pgfscope}%
\begin{pgfscope}%
\definecolor{textcolor}{rgb}{0.000000,0.000000,0.000000}%
\pgfsetstrokecolor{textcolor}%
\pgfsetfillcolor{textcolor}%
\pgftext[x=0.239775in,y=0.580635in,left,base]{\color{textcolor}\sffamily\fontsize{10.000000}{12.000000}\selectfont \(\displaystyle {10^{-6}}\)}%
\end{pgfscope}%
\begin{pgfscope}%
\pgfsetbuttcap%
\pgfsetroundjoin%
\definecolor{currentfill}{rgb}{0.000000,0.000000,0.000000}%
\pgfsetfillcolor{currentfill}%
\pgfsetlinewidth{0.803000pt}%
\definecolor{currentstroke}{rgb}{0.000000,0.000000,0.000000}%
\pgfsetstrokecolor{currentstroke}%
\pgfsetdash{}{0pt}%
\pgfsys@defobject{currentmarker}{\pgfqpoint{-0.048611in}{0.000000in}}{\pgfqpoint{0.000000in}{0.000000in}}{%
\pgfpathmoveto{\pgfqpoint{0.000000in}{0.000000in}}%
\pgfpathlineto{\pgfqpoint{-0.048611in}{0.000000in}}%
\pgfusepath{stroke,fill}%
}%
\begin{pgfscope}%
\pgfsys@transformshift{0.625000in}{1.648123in}%
\pgfsys@useobject{currentmarker}{}%
\end{pgfscope}%
\end{pgfscope}%
\begin{pgfscope}%
\definecolor{textcolor}{rgb}{0.000000,0.000000,0.000000}%
\pgfsetstrokecolor{textcolor}%
\pgfsetfillcolor{textcolor}%
\pgftext[x=0.239775in,y=1.595361in,left,base]{\color{textcolor}\sffamily\fontsize{10.000000}{12.000000}\selectfont \(\displaystyle {10^{-5}}\)}%
\end{pgfscope}%
\begin{pgfscope}%
\pgfsetbuttcap%
\pgfsetroundjoin%
\definecolor{currentfill}{rgb}{0.000000,0.000000,0.000000}%
\pgfsetfillcolor{currentfill}%
\pgfsetlinewidth{0.803000pt}%
\definecolor{currentstroke}{rgb}{0.000000,0.000000,0.000000}%
\pgfsetstrokecolor{currentstroke}%
\pgfsetdash{}{0pt}%
\pgfsys@defobject{currentmarker}{\pgfqpoint{-0.048611in}{0.000000in}}{\pgfqpoint{0.000000in}{0.000000in}}{%
\pgfpathmoveto{\pgfqpoint{0.000000in}{0.000000in}}%
\pgfpathlineto{\pgfqpoint{-0.048611in}{0.000000in}}%
\pgfusepath{stroke,fill}%
}%
\begin{pgfscope}%
\pgfsys@transformshift{0.625000in}{2.662850in}%
\pgfsys@useobject{currentmarker}{}%
\end{pgfscope}%
\end{pgfscope}%
\begin{pgfscope}%
\definecolor{textcolor}{rgb}{0.000000,0.000000,0.000000}%
\pgfsetstrokecolor{textcolor}%
\pgfsetfillcolor{textcolor}%
\pgftext[x=0.239775in,y=2.610088in,left,base]{\color{textcolor}\sffamily\fontsize{10.000000}{12.000000}\selectfont \(\displaystyle {10^{-4}}\)}%
\end{pgfscope}%
\begin{pgfscope}%
\pgfsetbuttcap%
\pgfsetroundjoin%
\definecolor{currentfill}{rgb}{0.000000,0.000000,0.000000}%
\pgfsetfillcolor{currentfill}%
\pgfsetlinewidth{0.602250pt}%
\definecolor{currentstroke}{rgb}{0.000000,0.000000,0.000000}%
\pgfsetstrokecolor{currentstroke}%
\pgfsetdash{}{0pt}%
\pgfsys@defobject{currentmarker}{\pgfqpoint{-0.027778in}{0.000000in}}{\pgfqpoint{0.000000in}{0.000000in}}{%
\pgfpathmoveto{\pgfqpoint{0.000000in}{0.000000in}}%
\pgfpathlineto{\pgfqpoint{-0.027778in}{0.000000in}}%
\pgfusepath{stroke,fill}%
}%
\begin{pgfscope}%
\pgfsys@transformshift{0.625000in}{0.476213in}%
\pgfsys@useobject{currentmarker}{}%
\end{pgfscope}%
\end{pgfscope}%
\begin{pgfscope}%
\pgfsetbuttcap%
\pgfsetroundjoin%
\definecolor{currentfill}{rgb}{0.000000,0.000000,0.000000}%
\pgfsetfillcolor{currentfill}%
\pgfsetlinewidth{0.602250pt}%
\definecolor{currentstroke}{rgb}{0.000000,0.000000,0.000000}%
\pgfsetstrokecolor{currentstroke}%
\pgfsetdash{}{0pt}%
\pgfsys@defobject{currentmarker}{\pgfqpoint{-0.027778in}{0.000000in}}{\pgfqpoint{0.000000in}{0.000000in}}{%
\pgfpathmoveto{\pgfqpoint{0.000000in}{0.000000in}}%
\pgfpathlineto{\pgfqpoint{-0.027778in}{0.000000in}}%
\pgfusepath{stroke,fill}%
}%
\begin{pgfscope}%
\pgfsys@transformshift{0.625000in}{0.535059in}%
\pgfsys@useobject{currentmarker}{}%
\end{pgfscope}%
\end{pgfscope}%
\begin{pgfscope}%
\pgfsetbuttcap%
\pgfsetroundjoin%
\definecolor{currentfill}{rgb}{0.000000,0.000000,0.000000}%
\pgfsetfillcolor{currentfill}%
\pgfsetlinewidth{0.602250pt}%
\definecolor{currentstroke}{rgb}{0.000000,0.000000,0.000000}%
\pgfsetstrokecolor{currentstroke}%
\pgfsetdash{}{0pt}%
\pgfsys@defobject{currentmarker}{\pgfqpoint{-0.027778in}{0.000000in}}{\pgfqpoint{0.000000in}{0.000000in}}{%
\pgfpathmoveto{\pgfqpoint{0.000000in}{0.000000in}}%
\pgfpathlineto{\pgfqpoint{-0.027778in}{0.000000in}}%
\pgfusepath{stroke,fill}%
}%
\begin{pgfscope}%
\pgfsys@transformshift{0.625000in}{0.586965in}%
\pgfsys@useobject{currentmarker}{}%
\end{pgfscope}%
\end{pgfscope}%
\begin{pgfscope}%
\pgfsetbuttcap%
\pgfsetroundjoin%
\definecolor{currentfill}{rgb}{0.000000,0.000000,0.000000}%
\pgfsetfillcolor{currentfill}%
\pgfsetlinewidth{0.602250pt}%
\definecolor{currentstroke}{rgb}{0.000000,0.000000,0.000000}%
\pgfsetstrokecolor{currentstroke}%
\pgfsetdash{}{0pt}%
\pgfsys@defobject{currentmarker}{\pgfqpoint{-0.027778in}{0.000000in}}{\pgfqpoint{0.000000in}{0.000000in}}{%
\pgfpathmoveto{\pgfqpoint{0.000000in}{0.000000in}}%
\pgfpathlineto{\pgfqpoint{-0.027778in}{0.000000in}}%
\pgfusepath{stroke,fill}%
}%
\begin{pgfscope}%
\pgfsys@transformshift{0.625000in}{0.938859in}%
\pgfsys@useobject{currentmarker}{}%
\end{pgfscope}%
\end{pgfscope}%
\begin{pgfscope}%
\pgfsetbuttcap%
\pgfsetroundjoin%
\definecolor{currentfill}{rgb}{0.000000,0.000000,0.000000}%
\pgfsetfillcolor{currentfill}%
\pgfsetlinewidth{0.602250pt}%
\definecolor{currentstroke}{rgb}{0.000000,0.000000,0.000000}%
\pgfsetstrokecolor{currentstroke}%
\pgfsetdash{}{0pt}%
\pgfsys@defobject{currentmarker}{\pgfqpoint{-0.027778in}{0.000000in}}{\pgfqpoint{0.000000in}{0.000000in}}{%
\pgfpathmoveto{\pgfqpoint{0.000000in}{0.000000in}}%
\pgfpathlineto{\pgfqpoint{-0.027778in}{0.000000in}}%
\pgfusepath{stroke,fill}%
}%
\begin{pgfscope}%
\pgfsys@transformshift{0.625000in}{1.117544in}%
\pgfsys@useobject{currentmarker}{}%
\end{pgfscope}%
\end{pgfscope}%
\begin{pgfscope}%
\pgfsetbuttcap%
\pgfsetroundjoin%
\definecolor{currentfill}{rgb}{0.000000,0.000000,0.000000}%
\pgfsetfillcolor{currentfill}%
\pgfsetlinewidth{0.602250pt}%
\definecolor{currentstroke}{rgb}{0.000000,0.000000,0.000000}%
\pgfsetstrokecolor{currentstroke}%
\pgfsetdash{}{0pt}%
\pgfsys@defobject{currentmarker}{\pgfqpoint{-0.027778in}{0.000000in}}{\pgfqpoint{0.000000in}{0.000000in}}{%
\pgfpathmoveto{\pgfqpoint{0.000000in}{0.000000in}}%
\pgfpathlineto{\pgfqpoint{-0.027778in}{0.000000in}}%
\pgfusepath{stroke,fill}%
}%
\begin{pgfscope}%
\pgfsys@transformshift{0.625000in}{1.244323in}%
\pgfsys@useobject{currentmarker}{}%
\end{pgfscope}%
\end{pgfscope}%
\begin{pgfscope}%
\pgfsetbuttcap%
\pgfsetroundjoin%
\definecolor{currentfill}{rgb}{0.000000,0.000000,0.000000}%
\pgfsetfillcolor{currentfill}%
\pgfsetlinewidth{0.602250pt}%
\definecolor{currentstroke}{rgb}{0.000000,0.000000,0.000000}%
\pgfsetstrokecolor{currentstroke}%
\pgfsetdash{}{0pt}%
\pgfsys@defobject{currentmarker}{\pgfqpoint{-0.027778in}{0.000000in}}{\pgfqpoint{0.000000in}{0.000000in}}{%
\pgfpathmoveto{\pgfqpoint{0.000000in}{0.000000in}}%
\pgfpathlineto{\pgfqpoint{-0.027778in}{0.000000in}}%
\pgfusepath{stroke,fill}%
}%
\begin{pgfscope}%
\pgfsys@transformshift{0.625000in}{1.342660in}%
\pgfsys@useobject{currentmarker}{}%
\end{pgfscope}%
\end{pgfscope}%
\begin{pgfscope}%
\pgfsetbuttcap%
\pgfsetroundjoin%
\definecolor{currentfill}{rgb}{0.000000,0.000000,0.000000}%
\pgfsetfillcolor{currentfill}%
\pgfsetlinewidth{0.602250pt}%
\definecolor{currentstroke}{rgb}{0.000000,0.000000,0.000000}%
\pgfsetstrokecolor{currentstroke}%
\pgfsetdash{}{0pt}%
\pgfsys@defobject{currentmarker}{\pgfqpoint{-0.027778in}{0.000000in}}{\pgfqpoint{0.000000in}{0.000000in}}{%
\pgfpathmoveto{\pgfqpoint{0.000000in}{0.000000in}}%
\pgfpathlineto{\pgfqpoint{-0.027778in}{0.000000in}}%
\pgfusepath{stroke,fill}%
}%
\begin{pgfscope}%
\pgfsys@transformshift{0.625000in}{1.423007in}%
\pgfsys@useobject{currentmarker}{}%
\end{pgfscope}%
\end{pgfscope}%
\begin{pgfscope}%
\pgfsetbuttcap%
\pgfsetroundjoin%
\definecolor{currentfill}{rgb}{0.000000,0.000000,0.000000}%
\pgfsetfillcolor{currentfill}%
\pgfsetlinewidth{0.602250pt}%
\definecolor{currentstroke}{rgb}{0.000000,0.000000,0.000000}%
\pgfsetstrokecolor{currentstroke}%
\pgfsetdash{}{0pt}%
\pgfsys@defobject{currentmarker}{\pgfqpoint{-0.027778in}{0.000000in}}{\pgfqpoint{0.000000in}{0.000000in}}{%
\pgfpathmoveto{\pgfqpoint{0.000000in}{0.000000in}}%
\pgfpathlineto{\pgfqpoint{-0.027778in}{0.000000in}}%
\pgfusepath{stroke,fill}%
}%
\begin{pgfscope}%
\pgfsys@transformshift{0.625000in}{1.490940in}%
\pgfsys@useobject{currentmarker}{}%
\end{pgfscope}%
\end{pgfscope}%
\begin{pgfscope}%
\pgfsetbuttcap%
\pgfsetroundjoin%
\definecolor{currentfill}{rgb}{0.000000,0.000000,0.000000}%
\pgfsetfillcolor{currentfill}%
\pgfsetlinewidth{0.602250pt}%
\definecolor{currentstroke}{rgb}{0.000000,0.000000,0.000000}%
\pgfsetstrokecolor{currentstroke}%
\pgfsetdash{}{0pt}%
\pgfsys@defobject{currentmarker}{\pgfqpoint{-0.027778in}{0.000000in}}{\pgfqpoint{0.000000in}{0.000000in}}{%
\pgfpathmoveto{\pgfqpoint{0.000000in}{0.000000in}}%
\pgfpathlineto{\pgfqpoint{-0.027778in}{0.000000in}}%
\pgfusepath{stroke,fill}%
}%
\begin{pgfscope}%
\pgfsys@transformshift{0.625000in}{1.549786in}%
\pgfsys@useobject{currentmarker}{}%
\end{pgfscope}%
\end{pgfscope}%
\begin{pgfscope}%
\pgfsetbuttcap%
\pgfsetroundjoin%
\definecolor{currentfill}{rgb}{0.000000,0.000000,0.000000}%
\pgfsetfillcolor{currentfill}%
\pgfsetlinewidth{0.602250pt}%
\definecolor{currentstroke}{rgb}{0.000000,0.000000,0.000000}%
\pgfsetstrokecolor{currentstroke}%
\pgfsetdash{}{0pt}%
\pgfsys@defobject{currentmarker}{\pgfqpoint{-0.027778in}{0.000000in}}{\pgfqpoint{0.000000in}{0.000000in}}{%
\pgfpathmoveto{\pgfqpoint{0.000000in}{0.000000in}}%
\pgfpathlineto{\pgfqpoint{-0.027778in}{0.000000in}}%
\pgfusepath{stroke,fill}%
}%
\begin{pgfscope}%
\pgfsys@transformshift{0.625000in}{1.601692in}%
\pgfsys@useobject{currentmarker}{}%
\end{pgfscope}%
\end{pgfscope}%
\begin{pgfscope}%
\pgfsetbuttcap%
\pgfsetroundjoin%
\definecolor{currentfill}{rgb}{0.000000,0.000000,0.000000}%
\pgfsetfillcolor{currentfill}%
\pgfsetlinewidth{0.602250pt}%
\definecolor{currentstroke}{rgb}{0.000000,0.000000,0.000000}%
\pgfsetstrokecolor{currentstroke}%
\pgfsetdash{}{0pt}%
\pgfsys@defobject{currentmarker}{\pgfqpoint{-0.027778in}{0.000000in}}{\pgfqpoint{0.000000in}{0.000000in}}{%
\pgfpathmoveto{\pgfqpoint{0.000000in}{0.000000in}}%
\pgfpathlineto{\pgfqpoint{-0.027778in}{0.000000in}}%
\pgfusepath{stroke,fill}%
}%
\begin{pgfscope}%
\pgfsys@transformshift{0.625000in}{1.953586in}%
\pgfsys@useobject{currentmarker}{}%
\end{pgfscope}%
\end{pgfscope}%
\begin{pgfscope}%
\pgfsetbuttcap%
\pgfsetroundjoin%
\definecolor{currentfill}{rgb}{0.000000,0.000000,0.000000}%
\pgfsetfillcolor{currentfill}%
\pgfsetlinewidth{0.602250pt}%
\definecolor{currentstroke}{rgb}{0.000000,0.000000,0.000000}%
\pgfsetstrokecolor{currentstroke}%
\pgfsetdash{}{0pt}%
\pgfsys@defobject{currentmarker}{\pgfqpoint{-0.027778in}{0.000000in}}{\pgfqpoint{0.000000in}{0.000000in}}{%
\pgfpathmoveto{\pgfqpoint{0.000000in}{0.000000in}}%
\pgfpathlineto{\pgfqpoint{-0.027778in}{0.000000in}}%
\pgfusepath{stroke,fill}%
}%
\begin{pgfscope}%
\pgfsys@transformshift{0.625000in}{2.132271in}%
\pgfsys@useobject{currentmarker}{}%
\end{pgfscope}%
\end{pgfscope}%
\begin{pgfscope}%
\pgfsetbuttcap%
\pgfsetroundjoin%
\definecolor{currentfill}{rgb}{0.000000,0.000000,0.000000}%
\pgfsetfillcolor{currentfill}%
\pgfsetlinewidth{0.602250pt}%
\definecolor{currentstroke}{rgb}{0.000000,0.000000,0.000000}%
\pgfsetstrokecolor{currentstroke}%
\pgfsetdash{}{0pt}%
\pgfsys@defobject{currentmarker}{\pgfqpoint{-0.027778in}{0.000000in}}{\pgfqpoint{0.000000in}{0.000000in}}{%
\pgfpathmoveto{\pgfqpoint{0.000000in}{0.000000in}}%
\pgfpathlineto{\pgfqpoint{-0.027778in}{0.000000in}}%
\pgfusepath{stroke,fill}%
}%
\begin{pgfscope}%
\pgfsys@transformshift{0.625000in}{2.259049in}%
\pgfsys@useobject{currentmarker}{}%
\end{pgfscope}%
\end{pgfscope}%
\begin{pgfscope}%
\pgfsetbuttcap%
\pgfsetroundjoin%
\definecolor{currentfill}{rgb}{0.000000,0.000000,0.000000}%
\pgfsetfillcolor{currentfill}%
\pgfsetlinewidth{0.602250pt}%
\definecolor{currentstroke}{rgb}{0.000000,0.000000,0.000000}%
\pgfsetstrokecolor{currentstroke}%
\pgfsetdash{}{0pt}%
\pgfsys@defobject{currentmarker}{\pgfqpoint{-0.027778in}{0.000000in}}{\pgfqpoint{0.000000in}{0.000000in}}{%
\pgfpathmoveto{\pgfqpoint{0.000000in}{0.000000in}}%
\pgfpathlineto{\pgfqpoint{-0.027778in}{0.000000in}}%
\pgfusepath{stroke,fill}%
}%
\begin{pgfscope}%
\pgfsys@transformshift{0.625000in}{2.357387in}%
\pgfsys@useobject{currentmarker}{}%
\end{pgfscope}%
\end{pgfscope}%
\begin{pgfscope}%
\pgfsetbuttcap%
\pgfsetroundjoin%
\definecolor{currentfill}{rgb}{0.000000,0.000000,0.000000}%
\pgfsetfillcolor{currentfill}%
\pgfsetlinewidth{0.602250pt}%
\definecolor{currentstroke}{rgb}{0.000000,0.000000,0.000000}%
\pgfsetstrokecolor{currentstroke}%
\pgfsetdash{}{0pt}%
\pgfsys@defobject{currentmarker}{\pgfqpoint{-0.027778in}{0.000000in}}{\pgfqpoint{0.000000in}{0.000000in}}{%
\pgfpathmoveto{\pgfqpoint{0.000000in}{0.000000in}}%
\pgfpathlineto{\pgfqpoint{-0.027778in}{0.000000in}}%
\pgfusepath{stroke,fill}%
}%
\begin{pgfscope}%
\pgfsys@transformshift{0.625000in}{2.437734in}%
\pgfsys@useobject{currentmarker}{}%
\end{pgfscope}%
\end{pgfscope}%
\begin{pgfscope}%
\pgfsetbuttcap%
\pgfsetroundjoin%
\definecolor{currentfill}{rgb}{0.000000,0.000000,0.000000}%
\pgfsetfillcolor{currentfill}%
\pgfsetlinewidth{0.602250pt}%
\definecolor{currentstroke}{rgb}{0.000000,0.000000,0.000000}%
\pgfsetstrokecolor{currentstroke}%
\pgfsetdash{}{0pt}%
\pgfsys@defobject{currentmarker}{\pgfqpoint{-0.027778in}{0.000000in}}{\pgfqpoint{0.000000in}{0.000000in}}{%
\pgfpathmoveto{\pgfqpoint{0.000000in}{0.000000in}}%
\pgfpathlineto{\pgfqpoint{-0.027778in}{0.000000in}}%
\pgfusepath{stroke,fill}%
}%
\begin{pgfscope}%
\pgfsys@transformshift{0.625000in}{2.505667in}%
\pgfsys@useobject{currentmarker}{}%
\end{pgfscope}%
\end{pgfscope}%
\begin{pgfscope}%
\pgfsetbuttcap%
\pgfsetroundjoin%
\definecolor{currentfill}{rgb}{0.000000,0.000000,0.000000}%
\pgfsetfillcolor{currentfill}%
\pgfsetlinewidth{0.602250pt}%
\definecolor{currentstroke}{rgb}{0.000000,0.000000,0.000000}%
\pgfsetstrokecolor{currentstroke}%
\pgfsetdash{}{0pt}%
\pgfsys@defobject{currentmarker}{\pgfqpoint{-0.027778in}{0.000000in}}{\pgfqpoint{0.000000in}{0.000000in}}{%
\pgfpathmoveto{\pgfqpoint{0.000000in}{0.000000in}}%
\pgfpathlineto{\pgfqpoint{-0.027778in}{0.000000in}}%
\pgfusepath{stroke,fill}%
}%
\begin{pgfscope}%
\pgfsys@transformshift{0.625000in}{2.564513in}%
\pgfsys@useobject{currentmarker}{}%
\end{pgfscope}%
\end{pgfscope}%
\begin{pgfscope}%
\pgfsetbuttcap%
\pgfsetroundjoin%
\definecolor{currentfill}{rgb}{0.000000,0.000000,0.000000}%
\pgfsetfillcolor{currentfill}%
\pgfsetlinewidth{0.602250pt}%
\definecolor{currentstroke}{rgb}{0.000000,0.000000,0.000000}%
\pgfsetstrokecolor{currentstroke}%
\pgfsetdash{}{0pt}%
\pgfsys@defobject{currentmarker}{\pgfqpoint{-0.027778in}{0.000000in}}{\pgfqpoint{0.000000in}{0.000000in}}{%
\pgfpathmoveto{\pgfqpoint{0.000000in}{0.000000in}}%
\pgfpathlineto{\pgfqpoint{-0.027778in}{0.000000in}}%
\pgfusepath{stroke,fill}%
}%
\begin{pgfscope}%
\pgfsys@transformshift{0.625000in}{2.616418in}%
\pgfsys@useobject{currentmarker}{}%
\end{pgfscope}%
\end{pgfscope}%
\begin{pgfscope}%
\pgfsetbuttcap%
\pgfsetroundjoin%
\definecolor{currentfill}{rgb}{0.000000,0.000000,0.000000}%
\pgfsetfillcolor{currentfill}%
\pgfsetlinewidth{0.602250pt}%
\definecolor{currentstroke}{rgb}{0.000000,0.000000,0.000000}%
\pgfsetstrokecolor{currentstroke}%
\pgfsetdash{}{0pt}%
\pgfsys@defobject{currentmarker}{\pgfqpoint{-0.027778in}{0.000000in}}{\pgfqpoint{0.000000in}{0.000000in}}{%
\pgfpathmoveto{\pgfqpoint{0.000000in}{0.000000in}}%
\pgfpathlineto{\pgfqpoint{-0.027778in}{0.000000in}}%
\pgfusepath{stroke,fill}%
}%
\begin{pgfscope}%
\pgfsys@transformshift{0.625000in}{2.968313in}%
\pgfsys@useobject{currentmarker}{}%
\end{pgfscope}%
\end{pgfscope}%
\begin{pgfscope}%
\pgfsetbuttcap%
\pgfsetroundjoin%
\definecolor{currentfill}{rgb}{0.000000,0.000000,0.000000}%
\pgfsetfillcolor{currentfill}%
\pgfsetlinewidth{0.602250pt}%
\definecolor{currentstroke}{rgb}{0.000000,0.000000,0.000000}%
\pgfsetstrokecolor{currentstroke}%
\pgfsetdash{}{0pt}%
\pgfsys@defobject{currentmarker}{\pgfqpoint{-0.027778in}{0.000000in}}{\pgfqpoint{0.000000in}{0.000000in}}{%
\pgfpathmoveto{\pgfqpoint{0.000000in}{0.000000in}}%
\pgfpathlineto{\pgfqpoint{-0.027778in}{0.000000in}}%
\pgfusepath{stroke,fill}%
}%
\begin{pgfscope}%
\pgfsys@transformshift{0.625000in}{3.146998in}%
\pgfsys@useobject{currentmarker}{}%
\end{pgfscope}%
\end{pgfscope}%
\begin{pgfscope}%
\pgfsetbuttcap%
\pgfsetroundjoin%
\definecolor{currentfill}{rgb}{0.000000,0.000000,0.000000}%
\pgfsetfillcolor{currentfill}%
\pgfsetlinewidth{0.602250pt}%
\definecolor{currentstroke}{rgb}{0.000000,0.000000,0.000000}%
\pgfsetstrokecolor{currentstroke}%
\pgfsetdash{}{0pt}%
\pgfsys@defobject{currentmarker}{\pgfqpoint{-0.027778in}{0.000000in}}{\pgfqpoint{0.000000in}{0.000000in}}{%
\pgfpathmoveto{\pgfqpoint{0.000000in}{0.000000in}}%
\pgfpathlineto{\pgfqpoint{-0.027778in}{0.000000in}}%
\pgfusepath{stroke,fill}%
}%
\begin{pgfscope}%
\pgfsys@transformshift{0.625000in}{3.273776in}%
\pgfsys@useobject{currentmarker}{}%
\end{pgfscope}%
\end{pgfscope}%
\begin{pgfscope}%
\pgfsetbuttcap%
\pgfsetroundjoin%
\definecolor{currentfill}{rgb}{0.000000,0.000000,0.000000}%
\pgfsetfillcolor{currentfill}%
\pgfsetlinewidth{0.602250pt}%
\definecolor{currentstroke}{rgb}{0.000000,0.000000,0.000000}%
\pgfsetstrokecolor{currentstroke}%
\pgfsetdash{}{0pt}%
\pgfsys@defobject{currentmarker}{\pgfqpoint{-0.027778in}{0.000000in}}{\pgfqpoint{0.000000in}{0.000000in}}{%
\pgfpathmoveto{\pgfqpoint{0.000000in}{0.000000in}}%
\pgfpathlineto{\pgfqpoint{-0.027778in}{0.000000in}}%
\pgfusepath{stroke,fill}%
}%
\begin{pgfscope}%
\pgfsys@transformshift{0.625000in}{3.372113in}%
\pgfsys@useobject{currentmarker}{}%
\end{pgfscope}%
\end{pgfscope}%
\begin{pgfscope}%
\pgfsetbuttcap%
\pgfsetroundjoin%
\definecolor{currentfill}{rgb}{0.000000,0.000000,0.000000}%
\pgfsetfillcolor{currentfill}%
\pgfsetlinewidth{0.602250pt}%
\definecolor{currentstroke}{rgb}{0.000000,0.000000,0.000000}%
\pgfsetstrokecolor{currentstroke}%
\pgfsetdash{}{0pt}%
\pgfsys@defobject{currentmarker}{\pgfqpoint{-0.027778in}{0.000000in}}{\pgfqpoint{0.000000in}{0.000000in}}{%
\pgfpathmoveto{\pgfqpoint{0.000000in}{0.000000in}}%
\pgfpathlineto{\pgfqpoint{-0.027778in}{0.000000in}}%
\pgfusepath{stroke,fill}%
}%
\begin{pgfscope}%
\pgfsys@transformshift{0.625000in}{3.452461in}%
\pgfsys@useobject{currentmarker}{}%
\end{pgfscope}%
\end{pgfscope}%
\begin{pgfscope}%
\pgfsetbuttcap%
\pgfsetroundjoin%
\definecolor{currentfill}{rgb}{0.000000,0.000000,0.000000}%
\pgfsetfillcolor{currentfill}%
\pgfsetlinewidth{0.602250pt}%
\definecolor{currentstroke}{rgb}{0.000000,0.000000,0.000000}%
\pgfsetstrokecolor{currentstroke}%
\pgfsetdash{}{0pt}%
\pgfsys@defobject{currentmarker}{\pgfqpoint{-0.027778in}{0.000000in}}{\pgfqpoint{0.000000in}{0.000000in}}{%
\pgfpathmoveto{\pgfqpoint{0.000000in}{0.000000in}}%
\pgfpathlineto{\pgfqpoint{-0.027778in}{0.000000in}}%
\pgfusepath{stroke,fill}%
}%
\begin{pgfscope}%
\pgfsys@transformshift{0.625000in}{3.520393in}%
\pgfsys@useobject{currentmarker}{}%
\end{pgfscope}%
\end{pgfscope}%
\begin{pgfscope}%
\definecolor{textcolor}{rgb}{0.000000,0.000000,0.000000}%
\pgfsetstrokecolor{textcolor}%
\pgfsetfillcolor{textcolor}%
\pgftext[x=0.184220in,y=1.980000in,,bottom,rotate=90.000000]{\color{textcolor}\sffamily\fontsize{10.000000}{12.000000}\selectfont QMC Error}%
\end{pgfscope}%
\begin{pgfscope}%
\pgfpathrectangle{\pgfqpoint{0.625000in}{0.440000in}}{\pgfqpoint{3.875000in}{3.080000in}}%
\pgfusepath{clip}%
\pgfsetbuttcap%
\pgfsetroundjoin%
\definecolor{currentfill}{rgb}{0.000000,0.000000,0.000000}%
\pgfsetfillcolor{currentfill}%
\pgfsetlinewidth{1.003750pt}%
\definecolor{currentstroke}{rgb}{0.000000,0.000000,0.000000}%
\pgfsetstrokecolor{currentstroke}%
\pgfsetdash{}{0pt}%
\pgfsys@defobject{currentmarker}{\pgfqpoint{-0.041667in}{-0.041667in}}{\pgfqpoint{0.041667in}{0.041667in}}{%
\pgfpathmoveto{\pgfqpoint{0.000000in}{-0.041667in}}%
\pgfpathcurveto{\pgfqpoint{0.011050in}{-0.041667in}}{\pgfqpoint{0.021649in}{-0.037276in}}{\pgfqpoint{0.029463in}{-0.029463in}}%
\pgfpathcurveto{\pgfqpoint{0.037276in}{-0.021649in}}{\pgfqpoint{0.041667in}{-0.011050in}}{\pgfqpoint{0.041667in}{0.000000in}}%
\pgfpathcurveto{\pgfqpoint{0.041667in}{0.011050in}}{\pgfqpoint{0.037276in}{0.021649in}}{\pgfqpoint{0.029463in}{0.029463in}}%
\pgfpathcurveto{\pgfqpoint{0.021649in}{0.037276in}}{\pgfqpoint{0.011050in}{0.041667in}}{\pgfqpoint{0.000000in}{0.041667in}}%
\pgfpathcurveto{\pgfqpoint{-0.011050in}{0.041667in}}{\pgfqpoint{-0.021649in}{0.037276in}}{\pgfqpoint{-0.029463in}{0.029463in}}%
\pgfpathcurveto{\pgfqpoint{-0.037276in}{0.021649in}}{\pgfqpoint{-0.041667in}{0.011050in}}{\pgfqpoint{-0.041667in}{0.000000in}}%
\pgfpathcurveto{\pgfqpoint{-0.041667in}{-0.011050in}}{\pgfqpoint{-0.037276in}{-0.021649in}}{\pgfqpoint{-0.029463in}{-0.029463in}}%
\pgfpathcurveto{\pgfqpoint{-0.021649in}{-0.037276in}}{\pgfqpoint{-0.011050in}{-0.041667in}}{\pgfqpoint{0.000000in}{-0.041667in}}%
\pgfpathclose%
\pgfusepath{stroke,fill}%
}%
\begin{pgfscope}%
\pgfsys@transformshift{0.801136in}{3.380000in}%
\pgfsys@useobject{currentmarker}{}%
\end{pgfscope}%
\begin{pgfscope}%
\pgfsys@transformshift{1.121384in}{3.239211in}%
\pgfsys@useobject{currentmarker}{}%
\end{pgfscope}%
\begin{pgfscope}%
\pgfsys@transformshift{1.441632in}{2.626112in}%
\pgfsys@useobject{currentmarker}{}%
\end{pgfscope}%
\begin{pgfscope}%
\pgfsys@transformshift{1.761880in}{2.537193in}%
\pgfsys@useobject{currentmarker}{}%
\end{pgfscope}%
\begin{pgfscope}%
\pgfsys@transformshift{2.082128in}{2.236922in}%
\pgfsys@useobject{currentmarker}{}%
\end{pgfscope}%
\begin{pgfscope}%
\pgfsys@transformshift{2.402376in}{1.995219in}%
\pgfsys@useobject{currentmarker}{}%
\end{pgfscope}%
\begin{pgfscope}%
\pgfsys@transformshift{2.722624in}{1.982213in}%
\pgfsys@useobject{currentmarker}{}%
\end{pgfscope}%
\begin{pgfscope}%
\pgfsys@transformshift{3.042872in}{1.923228in}%
\pgfsys@useobject{currentmarker}{}%
\end{pgfscope}%
\begin{pgfscope}%
\pgfsys@transformshift{3.363120in}{1.407454in}%
\pgfsys@useobject{currentmarker}{}%
\end{pgfscope}%
\begin{pgfscope}%
\pgfsys@transformshift{3.683368in}{1.107192in}%
\pgfsys@useobject{currentmarker}{}%
\end{pgfscope}%
\begin{pgfscope}%
\pgfsys@transformshift{4.003616in}{0.901027in}%
\pgfsys@useobject{currentmarker}{}%
\end{pgfscope}%
\begin{pgfscope}%
\pgfsys@transformshift{4.323864in}{0.580000in}%
\pgfsys@useobject{currentmarker}{}%
\end{pgfscope}%
\end{pgfscope}%
\begin{pgfscope}%
\pgfpathrectangle{\pgfqpoint{0.625000in}{0.440000in}}{\pgfqpoint{3.875000in}{3.080000in}}%
\pgfusepath{clip}%
\pgfsetbuttcap%
\pgfsetroundjoin%
\pgfsetlinewidth{1.505625pt}%
\definecolor{currentstroke}{rgb}{0.000000,0.000000,0.000000}%
\pgfsetstrokecolor{currentstroke}%
\pgfsetdash{{5.550000pt}{2.400000pt}}{0.000000pt}%
\pgfpathmoveto{\pgfqpoint{0.801136in}{3.321421in}}%
\pgfpathlineto{\pgfqpoint{4.323864in}{0.664541in}}%
\pgfusepath{stroke}%
\end{pgfscope}%
\begin{pgfscope}%
\pgfsetrectcap%
\pgfsetmiterjoin%
\pgfsetlinewidth{0.803000pt}%
\definecolor{currentstroke}{rgb}{0.000000,0.000000,0.000000}%
\pgfsetstrokecolor{currentstroke}%
\pgfsetdash{}{0pt}%
\pgfpathmoveto{\pgfqpoint{0.625000in}{0.440000in}}%
\pgfpathlineto{\pgfqpoint{0.625000in}{3.520000in}}%
\pgfusepath{stroke}%
\end{pgfscope}%
\begin{pgfscope}%
\pgfsetrectcap%
\pgfsetmiterjoin%
\pgfsetlinewidth{0.803000pt}%
\definecolor{currentstroke}{rgb}{0.000000,0.000000,0.000000}%
\pgfsetstrokecolor{currentstroke}%
\pgfsetdash{}{0pt}%
\pgfpathmoveto{\pgfqpoint{4.500000in}{0.440000in}}%
\pgfpathlineto{\pgfqpoint{4.500000in}{3.520000in}}%
\pgfusepath{stroke}%
\end{pgfscope}%
\begin{pgfscope}%
\pgfsetrectcap%
\pgfsetmiterjoin%
\pgfsetlinewidth{0.803000pt}%
\definecolor{currentstroke}{rgb}{0.000000,0.000000,0.000000}%
\pgfsetstrokecolor{currentstroke}%
\pgfsetdash{}{0pt}%
\pgfpathmoveto{\pgfqpoint{0.625000in}{0.440000in}}%
\pgfpathlineto{\pgfqpoint{4.500000in}{0.440000in}}%
\pgfusepath{stroke}%
\end{pgfscope}%
\begin{pgfscope}%
\pgfsetrectcap%
\pgfsetmiterjoin%
\pgfsetlinewidth{0.803000pt}%
\definecolor{currentstroke}{rgb}{0.000000,0.000000,0.000000}%
\pgfsetstrokecolor{currentstroke}%
\pgfsetdash{}{0pt}%
\pgfpathmoveto{\pgfqpoint{0.625000in}{3.520000in}}%
\pgfpathlineto{\pgfqpoint{4.500000in}{3.520000in}}%
\pgfusepath{stroke}%
\end{pgfscope}%
\begin{pgfscope}%
\pgfsetbuttcap%
\pgfsetmiterjoin%
\definecolor{currentfill}{rgb}{1.000000,1.000000,1.000000}%
\pgfsetfillcolor{currentfill}%
\pgfsetfillopacity{0.800000}%
\pgfsetlinewidth{1.003750pt}%
\definecolor{currentstroke}{rgb}{0.800000,0.800000,0.800000}%
\pgfsetstrokecolor{currentstroke}%
\pgfsetstrokeopacity{0.800000}%
\pgfsetdash{}{0pt}%
\pgfpathmoveto{\pgfqpoint{3.042243in}{3.155267in}}%
\pgfpathlineto{\pgfqpoint{4.402778in}{3.155267in}}%
\pgfpathquadraticcurveto{\pgfqpoint{4.430556in}{3.155267in}}{\pgfqpoint{4.430556in}{3.183045in}}%
\pgfpathlineto{\pgfqpoint{4.430556in}{3.422778in}}%
\pgfpathquadraticcurveto{\pgfqpoint{4.430556in}{3.450556in}}{\pgfqpoint{4.402778in}{3.450556in}}%
\pgfpathlineto{\pgfqpoint{3.042243in}{3.450556in}}%
\pgfpathquadraticcurveto{\pgfqpoint{3.014465in}{3.450556in}}{\pgfqpoint{3.014465in}{3.422778in}}%
\pgfpathlineto{\pgfqpoint{3.014465in}{3.183045in}}%
\pgfpathquadraticcurveto{\pgfqpoint{3.014465in}{3.155267in}}{\pgfqpoint{3.042243in}{3.155267in}}%
\pgfpathclose%
\pgfusepath{stroke,fill}%
\end{pgfscope}%
\begin{pgfscope}%
\pgfsetbuttcap%
\pgfsetroundjoin%
\pgfsetlinewidth{1.505625pt}%
\definecolor{currentstroke}{rgb}{0.000000,0.000000,0.000000}%
\pgfsetstrokecolor{currentstroke}%
\pgfsetdash{{5.550000pt}{2.400000pt}}{0.000000pt}%
\pgfpathmoveto{\pgfqpoint{3.070021in}{3.321543in}}%
\pgfpathlineto{\pgfqpoint{3.347799in}{3.321543in}}%
\pgfusepath{stroke}%
\end{pgfscope}%
\begin{pgfscope}%
\definecolor{textcolor}{rgb}{0.000000,0.000000,0.000000}%
\pgfsetstrokecolor{textcolor}%
\pgfsetfillcolor{textcolor}%
\pgftext[x=3.458910in,y=3.272932in,left,base]{\color{textcolor}\sffamily\fontsize{10.000000}{12.000000}\selectfont \(\displaystyle 0.0004N_{\mathrm{QMC}}^{-0.7907}\)}%
\end{pgfscope}%
\end{pgfpicture}%
\makeatother%
\endgroup%

\caption[Quasi-Monte-Carlo error, for $Q=\int_D u$ and $k=40$.]{Quasi-Monte-Carlo error with increasing number of Quasi-Monte-Carlo points, for $Q=\int_D u$ and $k=40$. \label{fig:qmcintegral40}}
\end{figure}
\begin{figure}[h]
    \centering
    %% Creator: Matplotlib, PGF backend
%%
%% To include the figure in your LaTeX document, write
%%   \input{<filename>.pgf}
%%
%% Make sure the required packages are loaded in your preamble
%%   \usepackage{pgf}
%%
%% Figures using additional raster images can only be included by \input if
%% they are in the same directory as the main LaTeX file. For loading figures
%% from other directories you can use the `import` package
%%   \usepackage{import}
%% and then include the figures with
%%   \import{<path to file>}{<filename>.pgf}
%%
%% Matplotlib used the following preamble
%%   \usepackage{fontspec}
%%   \setmainfont{DejaVuSerif.ttf}[Path=/home/owen/progs/firedrake-complex/firedrake/lib/python3.5/site-packages/matplotlib/mpl-data/fonts/ttf/]
%%   \setsansfont{DejaVuSans.ttf}[Path=/home/owen/progs/firedrake-complex/firedrake/lib/python3.5/site-packages/matplotlib/mpl-data/fonts/ttf/]
%%   \setmonofont{DejaVuSansMono.ttf}[Path=/home/owen/progs/firedrake-complex/firedrake/lib/python3.5/site-packages/matplotlib/mpl-data/fonts/ttf/]
%%
\begingroup%
\makeatletter%
\begin{pgfpicture}%
\pgfpathrectangle{\pgfpointorigin}{\pgfqpoint{5.000000in}{4.000000in}}%
\pgfusepath{use as bounding box, clip}%
\begin{pgfscope}%
\pgfsetbuttcap%
\pgfsetmiterjoin%
\definecolor{currentfill}{rgb}{1.000000,1.000000,1.000000}%
\pgfsetfillcolor{currentfill}%
\pgfsetlinewidth{0.000000pt}%
\definecolor{currentstroke}{rgb}{1.000000,1.000000,1.000000}%
\pgfsetstrokecolor{currentstroke}%
\pgfsetdash{}{0pt}%
\pgfpathmoveto{\pgfqpoint{0.000000in}{0.000000in}}%
\pgfpathlineto{\pgfqpoint{5.000000in}{0.000000in}}%
\pgfpathlineto{\pgfqpoint{5.000000in}{4.000000in}}%
\pgfpathlineto{\pgfqpoint{0.000000in}{4.000000in}}%
\pgfpathclose%
\pgfusepath{fill}%
\end{pgfscope}%
\begin{pgfscope}%
\pgfsetbuttcap%
\pgfsetmiterjoin%
\definecolor{currentfill}{rgb}{1.000000,1.000000,1.000000}%
\pgfsetfillcolor{currentfill}%
\pgfsetlinewidth{0.000000pt}%
\definecolor{currentstroke}{rgb}{0.000000,0.000000,0.000000}%
\pgfsetstrokecolor{currentstroke}%
\pgfsetstrokeopacity{0.000000}%
\pgfsetdash{}{0pt}%
\pgfpathmoveto{\pgfqpoint{0.625000in}{0.440000in}}%
\pgfpathlineto{\pgfqpoint{4.500000in}{0.440000in}}%
\pgfpathlineto{\pgfqpoint{4.500000in}{3.520000in}}%
\pgfpathlineto{\pgfqpoint{0.625000in}{3.520000in}}%
\pgfpathclose%
\pgfusepath{fill}%
\end{pgfscope}%
\begin{pgfscope}%
\pgfsetbuttcap%
\pgfsetroundjoin%
\definecolor{currentfill}{rgb}{0.000000,0.000000,0.000000}%
\pgfsetfillcolor{currentfill}%
\pgfsetlinewidth{0.803000pt}%
\definecolor{currentstroke}{rgb}{0.000000,0.000000,0.000000}%
\pgfsetstrokecolor{currentstroke}%
\pgfsetdash{}{0pt}%
\pgfsys@defobject{currentmarker}{\pgfqpoint{0.000000in}{-0.048611in}}{\pgfqpoint{0.000000in}{0.000000in}}{%
\pgfpathmoveto{\pgfqpoint{0.000000in}{0.000000in}}%
\pgfpathlineto{\pgfqpoint{0.000000in}{-0.048611in}}%
\pgfusepath{stroke,fill}%
}%
\begin{pgfscope}%
\pgfsys@transformshift{0.801136in}{0.440000in}%
\pgfsys@useobject{currentmarker}{}%
\end{pgfscope}%
\end{pgfscope}%
\begin{pgfscope}%
\definecolor{textcolor}{rgb}{0.000000,0.000000,0.000000}%
\pgfsetstrokecolor{textcolor}%
\pgfsetfillcolor{textcolor}%
\pgftext[x=0.801136in,y=0.342778in,,top]{\color{textcolor}\sffamily\fontsize{10.000000}{12.000000}\selectfont \(\displaystyle 10^{0}\)}%
\end{pgfscope}%
\begin{pgfscope}%
\pgfsetbuttcap%
\pgfsetroundjoin%
\definecolor{currentfill}{rgb}{0.000000,0.000000,0.000000}%
\pgfsetfillcolor{currentfill}%
\pgfsetlinewidth{0.803000pt}%
\definecolor{currentstroke}{rgb}{0.000000,0.000000,0.000000}%
\pgfsetstrokecolor{currentstroke}%
\pgfsetdash{}{0pt}%
\pgfsys@defobject{currentmarker}{\pgfqpoint{0.000000in}{-0.048611in}}{\pgfqpoint{0.000000in}{0.000000in}}{%
\pgfpathmoveto{\pgfqpoint{0.000000in}{0.000000in}}%
\pgfpathlineto{\pgfqpoint{0.000000in}{-0.048611in}}%
\pgfusepath{stroke,fill}%
}%
\begin{pgfscope}%
\pgfsys@transformshift{1.864977in}{0.440000in}%
\pgfsys@useobject{currentmarker}{}%
\end{pgfscope}%
\end{pgfscope}%
\begin{pgfscope}%
\definecolor{textcolor}{rgb}{0.000000,0.000000,0.000000}%
\pgfsetstrokecolor{textcolor}%
\pgfsetfillcolor{textcolor}%
\pgftext[x=1.864977in,y=0.342778in,,top]{\color{textcolor}\sffamily\fontsize{10.000000}{12.000000}\selectfont \(\displaystyle 10^{1}\)}%
\end{pgfscope}%
\begin{pgfscope}%
\pgfsetbuttcap%
\pgfsetroundjoin%
\definecolor{currentfill}{rgb}{0.000000,0.000000,0.000000}%
\pgfsetfillcolor{currentfill}%
\pgfsetlinewidth{0.803000pt}%
\definecolor{currentstroke}{rgb}{0.000000,0.000000,0.000000}%
\pgfsetstrokecolor{currentstroke}%
\pgfsetdash{}{0pt}%
\pgfsys@defobject{currentmarker}{\pgfqpoint{0.000000in}{-0.048611in}}{\pgfqpoint{0.000000in}{0.000000in}}{%
\pgfpathmoveto{\pgfqpoint{0.000000in}{0.000000in}}%
\pgfpathlineto{\pgfqpoint{0.000000in}{-0.048611in}}%
\pgfusepath{stroke,fill}%
}%
\begin{pgfscope}%
\pgfsys@transformshift{2.928818in}{0.440000in}%
\pgfsys@useobject{currentmarker}{}%
\end{pgfscope}%
\end{pgfscope}%
\begin{pgfscope}%
\definecolor{textcolor}{rgb}{0.000000,0.000000,0.000000}%
\pgfsetstrokecolor{textcolor}%
\pgfsetfillcolor{textcolor}%
\pgftext[x=2.928818in,y=0.342778in,,top]{\color{textcolor}\sffamily\fontsize{10.000000}{12.000000}\selectfont \(\displaystyle 10^{2}\)}%
\end{pgfscope}%
\begin{pgfscope}%
\pgfsetbuttcap%
\pgfsetroundjoin%
\definecolor{currentfill}{rgb}{0.000000,0.000000,0.000000}%
\pgfsetfillcolor{currentfill}%
\pgfsetlinewidth{0.803000pt}%
\definecolor{currentstroke}{rgb}{0.000000,0.000000,0.000000}%
\pgfsetstrokecolor{currentstroke}%
\pgfsetdash{}{0pt}%
\pgfsys@defobject{currentmarker}{\pgfqpoint{0.000000in}{-0.048611in}}{\pgfqpoint{0.000000in}{0.000000in}}{%
\pgfpathmoveto{\pgfqpoint{0.000000in}{0.000000in}}%
\pgfpathlineto{\pgfqpoint{0.000000in}{-0.048611in}}%
\pgfusepath{stroke,fill}%
}%
\begin{pgfscope}%
\pgfsys@transformshift{3.992658in}{0.440000in}%
\pgfsys@useobject{currentmarker}{}%
\end{pgfscope}%
\end{pgfscope}%
\begin{pgfscope}%
\definecolor{textcolor}{rgb}{0.000000,0.000000,0.000000}%
\pgfsetstrokecolor{textcolor}%
\pgfsetfillcolor{textcolor}%
\pgftext[x=3.992658in,y=0.342778in,,top]{\color{textcolor}\sffamily\fontsize{10.000000}{12.000000}\selectfont \(\displaystyle 10^{3}\)}%
\end{pgfscope}%
\begin{pgfscope}%
\pgfsetbuttcap%
\pgfsetroundjoin%
\definecolor{currentfill}{rgb}{0.000000,0.000000,0.000000}%
\pgfsetfillcolor{currentfill}%
\pgfsetlinewidth{0.602250pt}%
\definecolor{currentstroke}{rgb}{0.000000,0.000000,0.000000}%
\pgfsetstrokecolor{currentstroke}%
\pgfsetdash{}{0pt}%
\pgfsys@defobject{currentmarker}{\pgfqpoint{0.000000in}{-0.027778in}}{\pgfqpoint{0.000000in}{0.000000in}}{%
\pgfpathmoveto{\pgfqpoint{0.000000in}{0.000000in}}%
\pgfpathlineto{\pgfqpoint{0.000000in}{-0.027778in}}%
\pgfusepath{stroke,fill}%
}%
\begin{pgfscope}%
\pgfsys@transformshift{0.636345in}{0.440000in}%
\pgfsys@useobject{currentmarker}{}%
\end{pgfscope}%
\end{pgfscope}%
\begin{pgfscope}%
\pgfsetbuttcap%
\pgfsetroundjoin%
\definecolor{currentfill}{rgb}{0.000000,0.000000,0.000000}%
\pgfsetfillcolor{currentfill}%
\pgfsetlinewidth{0.602250pt}%
\definecolor{currentstroke}{rgb}{0.000000,0.000000,0.000000}%
\pgfsetstrokecolor{currentstroke}%
\pgfsetdash{}{0pt}%
\pgfsys@defobject{currentmarker}{\pgfqpoint{0.000000in}{-0.027778in}}{\pgfqpoint{0.000000in}{0.000000in}}{%
\pgfpathmoveto{\pgfqpoint{0.000000in}{0.000000in}}%
\pgfpathlineto{\pgfqpoint{0.000000in}{-0.027778in}}%
\pgfusepath{stroke,fill}%
}%
\begin{pgfscope}%
\pgfsys@transformshift{0.698040in}{0.440000in}%
\pgfsys@useobject{currentmarker}{}%
\end{pgfscope}%
\end{pgfscope}%
\begin{pgfscope}%
\pgfsetbuttcap%
\pgfsetroundjoin%
\definecolor{currentfill}{rgb}{0.000000,0.000000,0.000000}%
\pgfsetfillcolor{currentfill}%
\pgfsetlinewidth{0.602250pt}%
\definecolor{currentstroke}{rgb}{0.000000,0.000000,0.000000}%
\pgfsetstrokecolor{currentstroke}%
\pgfsetdash{}{0pt}%
\pgfsys@defobject{currentmarker}{\pgfqpoint{0.000000in}{-0.027778in}}{\pgfqpoint{0.000000in}{0.000000in}}{%
\pgfpathmoveto{\pgfqpoint{0.000000in}{0.000000in}}%
\pgfpathlineto{\pgfqpoint{0.000000in}{-0.027778in}}%
\pgfusepath{stroke,fill}%
}%
\begin{pgfscope}%
\pgfsys@transformshift{0.752458in}{0.440000in}%
\pgfsys@useobject{currentmarker}{}%
\end{pgfscope}%
\end{pgfscope}%
\begin{pgfscope}%
\pgfsetbuttcap%
\pgfsetroundjoin%
\definecolor{currentfill}{rgb}{0.000000,0.000000,0.000000}%
\pgfsetfillcolor{currentfill}%
\pgfsetlinewidth{0.602250pt}%
\definecolor{currentstroke}{rgb}{0.000000,0.000000,0.000000}%
\pgfsetstrokecolor{currentstroke}%
\pgfsetdash{}{0pt}%
\pgfsys@defobject{currentmarker}{\pgfqpoint{0.000000in}{-0.027778in}}{\pgfqpoint{0.000000in}{0.000000in}}{%
\pgfpathmoveto{\pgfqpoint{0.000000in}{0.000000in}}%
\pgfpathlineto{\pgfqpoint{0.000000in}{-0.027778in}}%
\pgfusepath{stroke,fill}%
}%
\begin{pgfscope}%
\pgfsys@transformshift{1.121384in}{0.440000in}%
\pgfsys@useobject{currentmarker}{}%
\end{pgfscope}%
\end{pgfscope}%
\begin{pgfscope}%
\pgfsetbuttcap%
\pgfsetroundjoin%
\definecolor{currentfill}{rgb}{0.000000,0.000000,0.000000}%
\pgfsetfillcolor{currentfill}%
\pgfsetlinewidth{0.602250pt}%
\definecolor{currentstroke}{rgb}{0.000000,0.000000,0.000000}%
\pgfsetstrokecolor{currentstroke}%
\pgfsetdash{}{0pt}%
\pgfsys@defobject{currentmarker}{\pgfqpoint{0.000000in}{-0.027778in}}{\pgfqpoint{0.000000in}{0.000000in}}{%
\pgfpathmoveto{\pgfqpoint{0.000000in}{0.000000in}}%
\pgfpathlineto{\pgfqpoint{0.000000in}{-0.027778in}}%
\pgfusepath{stroke,fill}%
}%
\begin{pgfscope}%
\pgfsys@transformshift{1.308717in}{0.440000in}%
\pgfsys@useobject{currentmarker}{}%
\end{pgfscope}%
\end{pgfscope}%
\begin{pgfscope}%
\pgfsetbuttcap%
\pgfsetroundjoin%
\definecolor{currentfill}{rgb}{0.000000,0.000000,0.000000}%
\pgfsetfillcolor{currentfill}%
\pgfsetlinewidth{0.602250pt}%
\definecolor{currentstroke}{rgb}{0.000000,0.000000,0.000000}%
\pgfsetstrokecolor{currentstroke}%
\pgfsetdash{}{0pt}%
\pgfsys@defobject{currentmarker}{\pgfqpoint{0.000000in}{-0.027778in}}{\pgfqpoint{0.000000in}{0.000000in}}{%
\pgfpathmoveto{\pgfqpoint{0.000000in}{0.000000in}}%
\pgfpathlineto{\pgfqpoint{0.000000in}{-0.027778in}}%
\pgfusepath{stroke,fill}%
}%
\begin{pgfscope}%
\pgfsys@transformshift{1.441632in}{0.440000in}%
\pgfsys@useobject{currentmarker}{}%
\end{pgfscope}%
\end{pgfscope}%
\begin{pgfscope}%
\pgfsetbuttcap%
\pgfsetroundjoin%
\definecolor{currentfill}{rgb}{0.000000,0.000000,0.000000}%
\pgfsetfillcolor{currentfill}%
\pgfsetlinewidth{0.602250pt}%
\definecolor{currentstroke}{rgb}{0.000000,0.000000,0.000000}%
\pgfsetstrokecolor{currentstroke}%
\pgfsetdash{}{0pt}%
\pgfsys@defobject{currentmarker}{\pgfqpoint{0.000000in}{-0.027778in}}{\pgfqpoint{0.000000in}{0.000000in}}{%
\pgfpathmoveto{\pgfqpoint{0.000000in}{0.000000in}}%
\pgfpathlineto{\pgfqpoint{0.000000in}{-0.027778in}}%
\pgfusepath{stroke,fill}%
}%
\begin{pgfscope}%
\pgfsys@transformshift{1.544729in}{0.440000in}%
\pgfsys@useobject{currentmarker}{}%
\end{pgfscope}%
\end{pgfscope}%
\begin{pgfscope}%
\pgfsetbuttcap%
\pgfsetroundjoin%
\definecolor{currentfill}{rgb}{0.000000,0.000000,0.000000}%
\pgfsetfillcolor{currentfill}%
\pgfsetlinewidth{0.602250pt}%
\definecolor{currentstroke}{rgb}{0.000000,0.000000,0.000000}%
\pgfsetstrokecolor{currentstroke}%
\pgfsetdash{}{0pt}%
\pgfsys@defobject{currentmarker}{\pgfqpoint{0.000000in}{-0.027778in}}{\pgfqpoint{0.000000in}{0.000000in}}{%
\pgfpathmoveto{\pgfqpoint{0.000000in}{0.000000in}}%
\pgfpathlineto{\pgfqpoint{0.000000in}{-0.027778in}}%
\pgfusepath{stroke,fill}%
}%
\begin{pgfscope}%
\pgfsys@transformshift{1.628965in}{0.440000in}%
\pgfsys@useobject{currentmarker}{}%
\end{pgfscope}%
\end{pgfscope}%
\begin{pgfscope}%
\pgfsetbuttcap%
\pgfsetroundjoin%
\definecolor{currentfill}{rgb}{0.000000,0.000000,0.000000}%
\pgfsetfillcolor{currentfill}%
\pgfsetlinewidth{0.602250pt}%
\definecolor{currentstroke}{rgb}{0.000000,0.000000,0.000000}%
\pgfsetstrokecolor{currentstroke}%
\pgfsetdash{}{0pt}%
\pgfsys@defobject{currentmarker}{\pgfqpoint{0.000000in}{-0.027778in}}{\pgfqpoint{0.000000in}{0.000000in}}{%
\pgfpathmoveto{\pgfqpoint{0.000000in}{0.000000in}}%
\pgfpathlineto{\pgfqpoint{0.000000in}{-0.027778in}}%
\pgfusepath{stroke,fill}%
}%
\begin{pgfscope}%
\pgfsys@transformshift{1.700186in}{0.440000in}%
\pgfsys@useobject{currentmarker}{}%
\end{pgfscope}%
\end{pgfscope}%
\begin{pgfscope}%
\pgfsetbuttcap%
\pgfsetroundjoin%
\definecolor{currentfill}{rgb}{0.000000,0.000000,0.000000}%
\pgfsetfillcolor{currentfill}%
\pgfsetlinewidth{0.602250pt}%
\definecolor{currentstroke}{rgb}{0.000000,0.000000,0.000000}%
\pgfsetstrokecolor{currentstroke}%
\pgfsetdash{}{0pt}%
\pgfsys@defobject{currentmarker}{\pgfqpoint{0.000000in}{-0.027778in}}{\pgfqpoint{0.000000in}{0.000000in}}{%
\pgfpathmoveto{\pgfqpoint{0.000000in}{0.000000in}}%
\pgfpathlineto{\pgfqpoint{0.000000in}{-0.027778in}}%
\pgfusepath{stroke,fill}%
}%
\begin{pgfscope}%
\pgfsys@transformshift{1.761880in}{0.440000in}%
\pgfsys@useobject{currentmarker}{}%
\end{pgfscope}%
\end{pgfscope}%
\begin{pgfscope}%
\pgfsetbuttcap%
\pgfsetroundjoin%
\definecolor{currentfill}{rgb}{0.000000,0.000000,0.000000}%
\pgfsetfillcolor{currentfill}%
\pgfsetlinewidth{0.602250pt}%
\definecolor{currentstroke}{rgb}{0.000000,0.000000,0.000000}%
\pgfsetstrokecolor{currentstroke}%
\pgfsetdash{}{0pt}%
\pgfsys@defobject{currentmarker}{\pgfqpoint{0.000000in}{-0.027778in}}{\pgfqpoint{0.000000in}{0.000000in}}{%
\pgfpathmoveto{\pgfqpoint{0.000000in}{0.000000in}}%
\pgfpathlineto{\pgfqpoint{0.000000in}{-0.027778in}}%
\pgfusepath{stroke,fill}%
}%
\begin{pgfscope}%
\pgfsys@transformshift{1.816298in}{0.440000in}%
\pgfsys@useobject{currentmarker}{}%
\end{pgfscope}%
\end{pgfscope}%
\begin{pgfscope}%
\pgfsetbuttcap%
\pgfsetroundjoin%
\definecolor{currentfill}{rgb}{0.000000,0.000000,0.000000}%
\pgfsetfillcolor{currentfill}%
\pgfsetlinewidth{0.602250pt}%
\definecolor{currentstroke}{rgb}{0.000000,0.000000,0.000000}%
\pgfsetstrokecolor{currentstroke}%
\pgfsetdash{}{0pt}%
\pgfsys@defobject{currentmarker}{\pgfqpoint{0.000000in}{-0.027778in}}{\pgfqpoint{0.000000in}{0.000000in}}{%
\pgfpathmoveto{\pgfqpoint{0.000000in}{0.000000in}}%
\pgfpathlineto{\pgfqpoint{0.000000in}{-0.027778in}}%
\pgfusepath{stroke,fill}%
}%
\begin{pgfscope}%
\pgfsys@transformshift{2.185225in}{0.440000in}%
\pgfsys@useobject{currentmarker}{}%
\end{pgfscope}%
\end{pgfscope}%
\begin{pgfscope}%
\pgfsetbuttcap%
\pgfsetroundjoin%
\definecolor{currentfill}{rgb}{0.000000,0.000000,0.000000}%
\pgfsetfillcolor{currentfill}%
\pgfsetlinewidth{0.602250pt}%
\definecolor{currentstroke}{rgb}{0.000000,0.000000,0.000000}%
\pgfsetstrokecolor{currentstroke}%
\pgfsetdash{}{0pt}%
\pgfsys@defobject{currentmarker}{\pgfqpoint{0.000000in}{-0.027778in}}{\pgfqpoint{0.000000in}{0.000000in}}{%
\pgfpathmoveto{\pgfqpoint{0.000000in}{0.000000in}}%
\pgfpathlineto{\pgfqpoint{0.000000in}{-0.027778in}}%
\pgfusepath{stroke,fill}%
}%
\begin{pgfscope}%
\pgfsys@transformshift{2.372558in}{0.440000in}%
\pgfsys@useobject{currentmarker}{}%
\end{pgfscope}%
\end{pgfscope}%
\begin{pgfscope}%
\pgfsetbuttcap%
\pgfsetroundjoin%
\definecolor{currentfill}{rgb}{0.000000,0.000000,0.000000}%
\pgfsetfillcolor{currentfill}%
\pgfsetlinewidth{0.602250pt}%
\definecolor{currentstroke}{rgb}{0.000000,0.000000,0.000000}%
\pgfsetstrokecolor{currentstroke}%
\pgfsetdash{}{0pt}%
\pgfsys@defobject{currentmarker}{\pgfqpoint{0.000000in}{-0.027778in}}{\pgfqpoint{0.000000in}{0.000000in}}{%
\pgfpathmoveto{\pgfqpoint{0.000000in}{0.000000in}}%
\pgfpathlineto{\pgfqpoint{0.000000in}{-0.027778in}}%
\pgfusepath{stroke,fill}%
}%
\begin{pgfscope}%
\pgfsys@transformshift{2.505473in}{0.440000in}%
\pgfsys@useobject{currentmarker}{}%
\end{pgfscope}%
\end{pgfscope}%
\begin{pgfscope}%
\pgfsetbuttcap%
\pgfsetroundjoin%
\definecolor{currentfill}{rgb}{0.000000,0.000000,0.000000}%
\pgfsetfillcolor{currentfill}%
\pgfsetlinewidth{0.602250pt}%
\definecolor{currentstroke}{rgb}{0.000000,0.000000,0.000000}%
\pgfsetstrokecolor{currentstroke}%
\pgfsetdash{}{0pt}%
\pgfsys@defobject{currentmarker}{\pgfqpoint{0.000000in}{-0.027778in}}{\pgfqpoint{0.000000in}{0.000000in}}{%
\pgfpathmoveto{\pgfqpoint{0.000000in}{0.000000in}}%
\pgfpathlineto{\pgfqpoint{0.000000in}{-0.027778in}}%
\pgfusepath{stroke,fill}%
}%
\begin{pgfscope}%
\pgfsys@transformshift{2.608570in}{0.440000in}%
\pgfsys@useobject{currentmarker}{}%
\end{pgfscope}%
\end{pgfscope}%
\begin{pgfscope}%
\pgfsetbuttcap%
\pgfsetroundjoin%
\definecolor{currentfill}{rgb}{0.000000,0.000000,0.000000}%
\pgfsetfillcolor{currentfill}%
\pgfsetlinewidth{0.602250pt}%
\definecolor{currentstroke}{rgb}{0.000000,0.000000,0.000000}%
\pgfsetstrokecolor{currentstroke}%
\pgfsetdash{}{0pt}%
\pgfsys@defobject{currentmarker}{\pgfqpoint{0.000000in}{-0.027778in}}{\pgfqpoint{0.000000in}{0.000000in}}{%
\pgfpathmoveto{\pgfqpoint{0.000000in}{0.000000in}}%
\pgfpathlineto{\pgfqpoint{0.000000in}{-0.027778in}}%
\pgfusepath{stroke,fill}%
}%
\begin{pgfscope}%
\pgfsys@transformshift{2.692806in}{0.440000in}%
\pgfsys@useobject{currentmarker}{}%
\end{pgfscope}%
\end{pgfscope}%
\begin{pgfscope}%
\pgfsetbuttcap%
\pgfsetroundjoin%
\definecolor{currentfill}{rgb}{0.000000,0.000000,0.000000}%
\pgfsetfillcolor{currentfill}%
\pgfsetlinewidth{0.602250pt}%
\definecolor{currentstroke}{rgb}{0.000000,0.000000,0.000000}%
\pgfsetstrokecolor{currentstroke}%
\pgfsetdash{}{0pt}%
\pgfsys@defobject{currentmarker}{\pgfqpoint{0.000000in}{-0.027778in}}{\pgfqpoint{0.000000in}{0.000000in}}{%
\pgfpathmoveto{\pgfqpoint{0.000000in}{0.000000in}}%
\pgfpathlineto{\pgfqpoint{0.000000in}{-0.027778in}}%
\pgfusepath{stroke,fill}%
}%
\begin{pgfscope}%
\pgfsys@transformshift{2.764027in}{0.440000in}%
\pgfsys@useobject{currentmarker}{}%
\end{pgfscope}%
\end{pgfscope}%
\begin{pgfscope}%
\pgfsetbuttcap%
\pgfsetroundjoin%
\definecolor{currentfill}{rgb}{0.000000,0.000000,0.000000}%
\pgfsetfillcolor{currentfill}%
\pgfsetlinewidth{0.602250pt}%
\definecolor{currentstroke}{rgb}{0.000000,0.000000,0.000000}%
\pgfsetstrokecolor{currentstroke}%
\pgfsetdash{}{0pt}%
\pgfsys@defobject{currentmarker}{\pgfqpoint{0.000000in}{-0.027778in}}{\pgfqpoint{0.000000in}{0.000000in}}{%
\pgfpathmoveto{\pgfqpoint{0.000000in}{0.000000in}}%
\pgfpathlineto{\pgfqpoint{0.000000in}{-0.027778in}}%
\pgfusepath{stroke,fill}%
}%
\begin{pgfscope}%
\pgfsys@transformshift{2.825721in}{0.440000in}%
\pgfsys@useobject{currentmarker}{}%
\end{pgfscope}%
\end{pgfscope}%
\begin{pgfscope}%
\pgfsetbuttcap%
\pgfsetroundjoin%
\definecolor{currentfill}{rgb}{0.000000,0.000000,0.000000}%
\pgfsetfillcolor{currentfill}%
\pgfsetlinewidth{0.602250pt}%
\definecolor{currentstroke}{rgb}{0.000000,0.000000,0.000000}%
\pgfsetstrokecolor{currentstroke}%
\pgfsetdash{}{0pt}%
\pgfsys@defobject{currentmarker}{\pgfqpoint{0.000000in}{-0.027778in}}{\pgfqpoint{0.000000in}{0.000000in}}{%
\pgfpathmoveto{\pgfqpoint{0.000000in}{0.000000in}}%
\pgfpathlineto{\pgfqpoint{0.000000in}{-0.027778in}}%
\pgfusepath{stroke,fill}%
}%
\begin{pgfscope}%
\pgfsys@transformshift{2.880139in}{0.440000in}%
\pgfsys@useobject{currentmarker}{}%
\end{pgfscope}%
\end{pgfscope}%
\begin{pgfscope}%
\pgfsetbuttcap%
\pgfsetroundjoin%
\definecolor{currentfill}{rgb}{0.000000,0.000000,0.000000}%
\pgfsetfillcolor{currentfill}%
\pgfsetlinewidth{0.602250pt}%
\definecolor{currentstroke}{rgb}{0.000000,0.000000,0.000000}%
\pgfsetstrokecolor{currentstroke}%
\pgfsetdash{}{0pt}%
\pgfsys@defobject{currentmarker}{\pgfqpoint{0.000000in}{-0.027778in}}{\pgfqpoint{0.000000in}{0.000000in}}{%
\pgfpathmoveto{\pgfqpoint{0.000000in}{0.000000in}}%
\pgfpathlineto{\pgfqpoint{0.000000in}{-0.027778in}}%
\pgfusepath{stroke,fill}%
}%
\begin{pgfscope}%
\pgfsys@transformshift{3.249066in}{0.440000in}%
\pgfsys@useobject{currentmarker}{}%
\end{pgfscope}%
\end{pgfscope}%
\begin{pgfscope}%
\pgfsetbuttcap%
\pgfsetroundjoin%
\definecolor{currentfill}{rgb}{0.000000,0.000000,0.000000}%
\pgfsetfillcolor{currentfill}%
\pgfsetlinewidth{0.602250pt}%
\definecolor{currentstroke}{rgb}{0.000000,0.000000,0.000000}%
\pgfsetstrokecolor{currentstroke}%
\pgfsetdash{}{0pt}%
\pgfsys@defobject{currentmarker}{\pgfqpoint{0.000000in}{-0.027778in}}{\pgfqpoint{0.000000in}{0.000000in}}{%
\pgfpathmoveto{\pgfqpoint{0.000000in}{0.000000in}}%
\pgfpathlineto{\pgfqpoint{0.000000in}{-0.027778in}}%
\pgfusepath{stroke,fill}%
}%
\begin{pgfscope}%
\pgfsys@transformshift{3.436399in}{0.440000in}%
\pgfsys@useobject{currentmarker}{}%
\end{pgfscope}%
\end{pgfscope}%
\begin{pgfscope}%
\pgfsetbuttcap%
\pgfsetroundjoin%
\definecolor{currentfill}{rgb}{0.000000,0.000000,0.000000}%
\pgfsetfillcolor{currentfill}%
\pgfsetlinewidth{0.602250pt}%
\definecolor{currentstroke}{rgb}{0.000000,0.000000,0.000000}%
\pgfsetstrokecolor{currentstroke}%
\pgfsetdash{}{0pt}%
\pgfsys@defobject{currentmarker}{\pgfqpoint{0.000000in}{-0.027778in}}{\pgfqpoint{0.000000in}{0.000000in}}{%
\pgfpathmoveto{\pgfqpoint{0.000000in}{0.000000in}}%
\pgfpathlineto{\pgfqpoint{0.000000in}{-0.027778in}}%
\pgfusepath{stroke,fill}%
}%
\begin{pgfscope}%
\pgfsys@transformshift{3.569313in}{0.440000in}%
\pgfsys@useobject{currentmarker}{}%
\end{pgfscope}%
\end{pgfscope}%
\begin{pgfscope}%
\pgfsetbuttcap%
\pgfsetroundjoin%
\definecolor{currentfill}{rgb}{0.000000,0.000000,0.000000}%
\pgfsetfillcolor{currentfill}%
\pgfsetlinewidth{0.602250pt}%
\definecolor{currentstroke}{rgb}{0.000000,0.000000,0.000000}%
\pgfsetstrokecolor{currentstroke}%
\pgfsetdash{}{0pt}%
\pgfsys@defobject{currentmarker}{\pgfqpoint{0.000000in}{-0.027778in}}{\pgfqpoint{0.000000in}{0.000000in}}{%
\pgfpathmoveto{\pgfqpoint{0.000000in}{0.000000in}}%
\pgfpathlineto{\pgfqpoint{0.000000in}{-0.027778in}}%
\pgfusepath{stroke,fill}%
}%
\begin{pgfscope}%
\pgfsys@transformshift{3.672410in}{0.440000in}%
\pgfsys@useobject{currentmarker}{}%
\end{pgfscope}%
\end{pgfscope}%
\begin{pgfscope}%
\pgfsetbuttcap%
\pgfsetroundjoin%
\definecolor{currentfill}{rgb}{0.000000,0.000000,0.000000}%
\pgfsetfillcolor{currentfill}%
\pgfsetlinewidth{0.602250pt}%
\definecolor{currentstroke}{rgb}{0.000000,0.000000,0.000000}%
\pgfsetstrokecolor{currentstroke}%
\pgfsetdash{}{0pt}%
\pgfsys@defobject{currentmarker}{\pgfqpoint{0.000000in}{-0.027778in}}{\pgfqpoint{0.000000in}{0.000000in}}{%
\pgfpathmoveto{\pgfqpoint{0.000000in}{0.000000in}}%
\pgfpathlineto{\pgfqpoint{0.000000in}{-0.027778in}}%
\pgfusepath{stroke,fill}%
}%
\begin{pgfscope}%
\pgfsys@transformshift{3.756646in}{0.440000in}%
\pgfsys@useobject{currentmarker}{}%
\end{pgfscope}%
\end{pgfscope}%
\begin{pgfscope}%
\pgfsetbuttcap%
\pgfsetroundjoin%
\definecolor{currentfill}{rgb}{0.000000,0.000000,0.000000}%
\pgfsetfillcolor{currentfill}%
\pgfsetlinewidth{0.602250pt}%
\definecolor{currentstroke}{rgb}{0.000000,0.000000,0.000000}%
\pgfsetstrokecolor{currentstroke}%
\pgfsetdash{}{0pt}%
\pgfsys@defobject{currentmarker}{\pgfqpoint{0.000000in}{-0.027778in}}{\pgfqpoint{0.000000in}{0.000000in}}{%
\pgfpathmoveto{\pgfqpoint{0.000000in}{0.000000in}}%
\pgfpathlineto{\pgfqpoint{0.000000in}{-0.027778in}}%
\pgfusepath{stroke,fill}%
}%
\begin{pgfscope}%
\pgfsys@transformshift{3.827867in}{0.440000in}%
\pgfsys@useobject{currentmarker}{}%
\end{pgfscope}%
\end{pgfscope}%
\begin{pgfscope}%
\pgfsetbuttcap%
\pgfsetroundjoin%
\definecolor{currentfill}{rgb}{0.000000,0.000000,0.000000}%
\pgfsetfillcolor{currentfill}%
\pgfsetlinewidth{0.602250pt}%
\definecolor{currentstroke}{rgb}{0.000000,0.000000,0.000000}%
\pgfsetstrokecolor{currentstroke}%
\pgfsetdash{}{0pt}%
\pgfsys@defobject{currentmarker}{\pgfqpoint{0.000000in}{-0.027778in}}{\pgfqpoint{0.000000in}{0.000000in}}{%
\pgfpathmoveto{\pgfqpoint{0.000000in}{0.000000in}}%
\pgfpathlineto{\pgfqpoint{0.000000in}{-0.027778in}}%
\pgfusepath{stroke,fill}%
}%
\begin{pgfscope}%
\pgfsys@transformshift{3.889561in}{0.440000in}%
\pgfsys@useobject{currentmarker}{}%
\end{pgfscope}%
\end{pgfscope}%
\begin{pgfscope}%
\pgfsetbuttcap%
\pgfsetroundjoin%
\definecolor{currentfill}{rgb}{0.000000,0.000000,0.000000}%
\pgfsetfillcolor{currentfill}%
\pgfsetlinewidth{0.602250pt}%
\definecolor{currentstroke}{rgb}{0.000000,0.000000,0.000000}%
\pgfsetstrokecolor{currentstroke}%
\pgfsetdash{}{0pt}%
\pgfsys@defobject{currentmarker}{\pgfqpoint{0.000000in}{-0.027778in}}{\pgfqpoint{0.000000in}{0.000000in}}{%
\pgfpathmoveto{\pgfqpoint{0.000000in}{0.000000in}}%
\pgfpathlineto{\pgfqpoint{0.000000in}{-0.027778in}}%
\pgfusepath{stroke,fill}%
}%
\begin{pgfscope}%
\pgfsys@transformshift{3.943980in}{0.440000in}%
\pgfsys@useobject{currentmarker}{}%
\end{pgfscope}%
\end{pgfscope}%
\begin{pgfscope}%
\pgfsetbuttcap%
\pgfsetroundjoin%
\definecolor{currentfill}{rgb}{0.000000,0.000000,0.000000}%
\pgfsetfillcolor{currentfill}%
\pgfsetlinewidth{0.602250pt}%
\definecolor{currentstroke}{rgb}{0.000000,0.000000,0.000000}%
\pgfsetstrokecolor{currentstroke}%
\pgfsetdash{}{0pt}%
\pgfsys@defobject{currentmarker}{\pgfqpoint{0.000000in}{-0.027778in}}{\pgfqpoint{0.000000in}{0.000000in}}{%
\pgfpathmoveto{\pgfqpoint{0.000000in}{0.000000in}}%
\pgfpathlineto{\pgfqpoint{0.000000in}{-0.027778in}}%
\pgfusepath{stroke,fill}%
}%
\begin{pgfscope}%
\pgfsys@transformshift{4.312906in}{0.440000in}%
\pgfsys@useobject{currentmarker}{}%
\end{pgfscope}%
\end{pgfscope}%
\begin{pgfscope}%
\pgfsetbuttcap%
\pgfsetroundjoin%
\definecolor{currentfill}{rgb}{0.000000,0.000000,0.000000}%
\pgfsetfillcolor{currentfill}%
\pgfsetlinewidth{0.602250pt}%
\definecolor{currentstroke}{rgb}{0.000000,0.000000,0.000000}%
\pgfsetstrokecolor{currentstroke}%
\pgfsetdash{}{0pt}%
\pgfsys@defobject{currentmarker}{\pgfqpoint{0.000000in}{-0.027778in}}{\pgfqpoint{0.000000in}{0.000000in}}{%
\pgfpathmoveto{\pgfqpoint{0.000000in}{0.000000in}}%
\pgfpathlineto{\pgfqpoint{0.000000in}{-0.027778in}}%
\pgfusepath{stroke,fill}%
}%
\begin{pgfscope}%
\pgfsys@transformshift{4.500239in}{0.440000in}%
\pgfsys@useobject{currentmarker}{}%
\end{pgfscope}%
\end{pgfscope}%
\begin{pgfscope}%
\definecolor{textcolor}{rgb}{0.000000,0.000000,0.000000}%
\pgfsetstrokecolor{textcolor}%
\pgfsetfillcolor{textcolor}%
\pgftext[x=2.562500in,y=0.152809in,,top]{\color{textcolor}\sffamily\fontsize{10.000000}{12.000000}\selectfont \(\displaystyle N_\mathrm{QMC}\)}%
\end{pgfscope}%
\begin{pgfscope}%
\pgfsetbuttcap%
\pgfsetroundjoin%
\definecolor{currentfill}{rgb}{0.000000,0.000000,0.000000}%
\pgfsetfillcolor{currentfill}%
\pgfsetlinewidth{0.803000pt}%
\definecolor{currentstroke}{rgb}{0.000000,0.000000,0.000000}%
\pgfsetstrokecolor{currentstroke}%
\pgfsetdash{}{0pt}%
\pgfsys@defobject{currentmarker}{\pgfqpoint{-0.048611in}{0.000000in}}{\pgfqpoint{0.000000in}{0.000000in}}{%
\pgfpathmoveto{\pgfqpoint{0.000000in}{0.000000in}}%
\pgfpathlineto{\pgfqpoint{-0.048611in}{0.000000in}}%
\pgfusepath{stroke,fill}%
}%
\begin{pgfscope}%
\pgfsys@transformshift{0.625000in}{1.556338in}%
\pgfsys@useobject{currentmarker}{}%
\end{pgfscope}%
\end{pgfscope}%
\begin{pgfscope}%
\definecolor{textcolor}{rgb}{0.000000,0.000000,0.000000}%
\pgfsetstrokecolor{textcolor}%
\pgfsetfillcolor{textcolor}%
\pgftext[x=0.239775in,y=1.503577in,left,base]{\color{textcolor}\sffamily\fontsize{10.000000}{12.000000}\selectfont \(\displaystyle 10^{-5}\)}%
\end{pgfscope}%
\begin{pgfscope}%
\pgfsetbuttcap%
\pgfsetroundjoin%
\definecolor{currentfill}{rgb}{0.000000,0.000000,0.000000}%
\pgfsetfillcolor{currentfill}%
\pgfsetlinewidth{0.803000pt}%
\definecolor{currentstroke}{rgb}{0.000000,0.000000,0.000000}%
\pgfsetstrokecolor{currentstroke}%
\pgfsetdash{}{0pt}%
\pgfsys@defobject{currentmarker}{\pgfqpoint{-0.048611in}{0.000000in}}{\pgfqpoint{0.000000in}{0.000000in}}{%
\pgfpathmoveto{\pgfqpoint{0.000000in}{0.000000in}}%
\pgfpathlineto{\pgfqpoint{-0.048611in}{0.000000in}}%
\pgfusepath{stroke,fill}%
}%
\begin{pgfscope}%
\pgfsys@transformshift{0.625000in}{2.742769in}%
\pgfsys@useobject{currentmarker}{}%
\end{pgfscope}%
\end{pgfscope}%
\begin{pgfscope}%
\definecolor{textcolor}{rgb}{0.000000,0.000000,0.000000}%
\pgfsetstrokecolor{textcolor}%
\pgfsetfillcolor{textcolor}%
\pgftext[x=0.239775in,y=2.690007in,left,base]{\color{textcolor}\sffamily\fontsize{10.000000}{12.000000}\selectfont \(\displaystyle 10^{-4}\)}%
\end{pgfscope}%
\begin{pgfscope}%
\pgfsetbuttcap%
\pgfsetroundjoin%
\definecolor{currentfill}{rgb}{0.000000,0.000000,0.000000}%
\pgfsetfillcolor{currentfill}%
\pgfsetlinewidth{0.602250pt}%
\definecolor{currentstroke}{rgb}{0.000000,0.000000,0.000000}%
\pgfsetstrokecolor{currentstroke}%
\pgfsetdash{}{0pt}%
\pgfsys@defobject{currentmarker}{\pgfqpoint{-0.027778in}{0.000000in}}{\pgfqpoint{0.000000in}{0.000000in}}{%
\pgfpathmoveto{\pgfqpoint{0.000000in}{0.000000in}}%
\pgfpathlineto{\pgfqpoint{-0.027778in}{0.000000in}}%
\pgfusepath{stroke,fill}%
}%
\begin{pgfscope}%
\pgfsys@transformshift{0.625000in}{0.727059in}%
\pgfsys@useobject{currentmarker}{}%
\end{pgfscope}%
\end{pgfscope}%
\begin{pgfscope}%
\pgfsetbuttcap%
\pgfsetroundjoin%
\definecolor{currentfill}{rgb}{0.000000,0.000000,0.000000}%
\pgfsetfillcolor{currentfill}%
\pgfsetlinewidth{0.602250pt}%
\definecolor{currentstroke}{rgb}{0.000000,0.000000,0.000000}%
\pgfsetstrokecolor{currentstroke}%
\pgfsetdash{}{0pt}%
\pgfsys@defobject{currentmarker}{\pgfqpoint{-0.027778in}{0.000000in}}{\pgfqpoint{0.000000in}{0.000000in}}{%
\pgfpathmoveto{\pgfqpoint{0.000000in}{0.000000in}}%
\pgfpathlineto{\pgfqpoint{-0.027778in}{0.000000in}}%
\pgfusepath{stroke,fill}%
}%
\begin{pgfscope}%
\pgfsys@transformshift{0.625000in}{0.935979in}%
\pgfsys@useobject{currentmarker}{}%
\end{pgfscope}%
\end{pgfscope}%
\begin{pgfscope}%
\pgfsetbuttcap%
\pgfsetroundjoin%
\definecolor{currentfill}{rgb}{0.000000,0.000000,0.000000}%
\pgfsetfillcolor{currentfill}%
\pgfsetlinewidth{0.602250pt}%
\definecolor{currentstroke}{rgb}{0.000000,0.000000,0.000000}%
\pgfsetstrokecolor{currentstroke}%
\pgfsetdash{}{0pt}%
\pgfsys@defobject{currentmarker}{\pgfqpoint{-0.027778in}{0.000000in}}{\pgfqpoint{0.000000in}{0.000000in}}{%
\pgfpathmoveto{\pgfqpoint{0.000000in}{0.000000in}}%
\pgfpathlineto{\pgfqpoint{-0.027778in}{0.000000in}}%
\pgfusepath{stroke,fill}%
}%
\begin{pgfscope}%
\pgfsys@transformshift{0.625000in}{1.084210in}%
\pgfsys@useobject{currentmarker}{}%
\end{pgfscope}%
\end{pgfscope}%
\begin{pgfscope}%
\pgfsetbuttcap%
\pgfsetroundjoin%
\definecolor{currentfill}{rgb}{0.000000,0.000000,0.000000}%
\pgfsetfillcolor{currentfill}%
\pgfsetlinewidth{0.602250pt}%
\definecolor{currentstroke}{rgb}{0.000000,0.000000,0.000000}%
\pgfsetstrokecolor{currentstroke}%
\pgfsetdash{}{0pt}%
\pgfsys@defobject{currentmarker}{\pgfqpoint{-0.027778in}{0.000000in}}{\pgfqpoint{0.000000in}{0.000000in}}{%
\pgfpathmoveto{\pgfqpoint{0.000000in}{0.000000in}}%
\pgfpathlineto{\pgfqpoint{-0.027778in}{0.000000in}}%
\pgfusepath{stroke,fill}%
}%
\begin{pgfscope}%
\pgfsys@transformshift{0.625000in}{1.199187in}%
\pgfsys@useobject{currentmarker}{}%
\end{pgfscope}%
\end{pgfscope}%
\begin{pgfscope}%
\pgfsetbuttcap%
\pgfsetroundjoin%
\definecolor{currentfill}{rgb}{0.000000,0.000000,0.000000}%
\pgfsetfillcolor{currentfill}%
\pgfsetlinewidth{0.602250pt}%
\definecolor{currentstroke}{rgb}{0.000000,0.000000,0.000000}%
\pgfsetstrokecolor{currentstroke}%
\pgfsetdash{}{0pt}%
\pgfsys@defobject{currentmarker}{\pgfqpoint{-0.027778in}{0.000000in}}{\pgfqpoint{0.000000in}{0.000000in}}{%
\pgfpathmoveto{\pgfqpoint{0.000000in}{0.000000in}}%
\pgfpathlineto{\pgfqpoint{-0.027778in}{0.000000in}}%
\pgfusepath{stroke,fill}%
}%
\begin{pgfscope}%
\pgfsys@transformshift{0.625000in}{1.293130in}%
\pgfsys@useobject{currentmarker}{}%
\end{pgfscope}%
\end{pgfscope}%
\begin{pgfscope}%
\pgfsetbuttcap%
\pgfsetroundjoin%
\definecolor{currentfill}{rgb}{0.000000,0.000000,0.000000}%
\pgfsetfillcolor{currentfill}%
\pgfsetlinewidth{0.602250pt}%
\definecolor{currentstroke}{rgb}{0.000000,0.000000,0.000000}%
\pgfsetstrokecolor{currentstroke}%
\pgfsetdash{}{0pt}%
\pgfsys@defobject{currentmarker}{\pgfqpoint{-0.027778in}{0.000000in}}{\pgfqpoint{0.000000in}{0.000000in}}{%
\pgfpathmoveto{\pgfqpoint{0.000000in}{0.000000in}}%
\pgfpathlineto{\pgfqpoint{-0.027778in}{0.000000in}}%
\pgfusepath{stroke,fill}%
}%
\begin{pgfscope}%
\pgfsys@transformshift{0.625000in}{1.372558in}%
\pgfsys@useobject{currentmarker}{}%
\end{pgfscope}%
\end{pgfscope}%
\begin{pgfscope}%
\pgfsetbuttcap%
\pgfsetroundjoin%
\definecolor{currentfill}{rgb}{0.000000,0.000000,0.000000}%
\pgfsetfillcolor{currentfill}%
\pgfsetlinewidth{0.602250pt}%
\definecolor{currentstroke}{rgb}{0.000000,0.000000,0.000000}%
\pgfsetstrokecolor{currentstroke}%
\pgfsetdash{}{0pt}%
\pgfsys@defobject{currentmarker}{\pgfqpoint{-0.027778in}{0.000000in}}{\pgfqpoint{0.000000in}{0.000000in}}{%
\pgfpathmoveto{\pgfqpoint{0.000000in}{0.000000in}}%
\pgfpathlineto{\pgfqpoint{-0.027778in}{0.000000in}}%
\pgfusepath{stroke,fill}%
}%
\begin{pgfscope}%
\pgfsys@transformshift{0.625000in}{1.441361in}%
\pgfsys@useobject{currentmarker}{}%
\end{pgfscope}%
\end{pgfscope}%
\begin{pgfscope}%
\pgfsetbuttcap%
\pgfsetroundjoin%
\definecolor{currentfill}{rgb}{0.000000,0.000000,0.000000}%
\pgfsetfillcolor{currentfill}%
\pgfsetlinewidth{0.602250pt}%
\definecolor{currentstroke}{rgb}{0.000000,0.000000,0.000000}%
\pgfsetstrokecolor{currentstroke}%
\pgfsetdash{}{0pt}%
\pgfsys@defobject{currentmarker}{\pgfqpoint{-0.027778in}{0.000000in}}{\pgfqpoint{0.000000in}{0.000000in}}{%
\pgfpathmoveto{\pgfqpoint{0.000000in}{0.000000in}}%
\pgfpathlineto{\pgfqpoint{-0.027778in}{0.000000in}}%
\pgfusepath{stroke,fill}%
}%
\begin{pgfscope}%
\pgfsys@transformshift{0.625000in}{1.502050in}%
\pgfsys@useobject{currentmarker}{}%
\end{pgfscope}%
\end{pgfscope}%
\begin{pgfscope}%
\pgfsetbuttcap%
\pgfsetroundjoin%
\definecolor{currentfill}{rgb}{0.000000,0.000000,0.000000}%
\pgfsetfillcolor{currentfill}%
\pgfsetlinewidth{0.602250pt}%
\definecolor{currentstroke}{rgb}{0.000000,0.000000,0.000000}%
\pgfsetstrokecolor{currentstroke}%
\pgfsetdash{}{0pt}%
\pgfsys@defobject{currentmarker}{\pgfqpoint{-0.027778in}{0.000000in}}{\pgfqpoint{0.000000in}{0.000000in}}{%
\pgfpathmoveto{\pgfqpoint{0.000000in}{0.000000in}}%
\pgfpathlineto{\pgfqpoint{-0.027778in}{0.000000in}}%
\pgfusepath{stroke,fill}%
}%
\begin{pgfscope}%
\pgfsys@transformshift{0.625000in}{1.913489in}%
\pgfsys@useobject{currentmarker}{}%
\end{pgfscope}%
\end{pgfscope}%
\begin{pgfscope}%
\pgfsetbuttcap%
\pgfsetroundjoin%
\definecolor{currentfill}{rgb}{0.000000,0.000000,0.000000}%
\pgfsetfillcolor{currentfill}%
\pgfsetlinewidth{0.602250pt}%
\definecolor{currentstroke}{rgb}{0.000000,0.000000,0.000000}%
\pgfsetstrokecolor{currentstroke}%
\pgfsetdash{}{0pt}%
\pgfsys@defobject{currentmarker}{\pgfqpoint{-0.027778in}{0.000000in}}{\pgfqpoint{0.000000in}{0.000000in}}{%
\pgfpathmoveto{\pgfqpoint{0.000000in}{0.000000in}}%
\pgfpathlineto{\pgfqpoint{-0.027778in}{0.000000in}}%
\pgfusepath{stroke,fill}%
}%
\begin{pgfscope}%
\pgfsys@transformshift{0.625000in}{2.122409in}%
\pgfsys@useobject{currentmarker}{}%
\end{pgfscope}%
\end{pgfscope}%
\begin{pgfscope}%
\pgfsetbuttcap%
\pgfsetroundjoin%
\definecolor{currentfill}{rgb}{0.000000,0.000000,0.000000}%
\pgfsetfillcolor{currentfill}%
\pgfsetlinewidth{0.602250pt}%
\definecolor{currentstroke}{rgb}{0.000000,0.000000,0.000000}%
\pgfsetstrokecolor{currentstroke}%
\pgfsetdash{}{0pt}%
\pgfsys@defobject{currentmarker}{\pgfqpoint{-0.027778in}{0.000000in}}{\pgfqpoint{0.000000in}{0.000000in}}{%
\pgfpathmoveto{\pgfqpoint{0.000000in}{0.000000in}}%
\pgfpathlineto{\pgfqpoint{-0.027778in}{0.000000in}}%
\pgfusepath{stroke,fill}%
}%
\begin{pgfscope}%
\pgfsys@transformshift{0.625000in}{2.270641in}%
\pgfsys@useobject{currentmarker}{}%
\end{pgfscope}%
\end{pgfscope}%
\begin{pgfscope}%
\pgfsetbuttcap%
\pgfsetroundjoin%
\definecolor{currentfill}{rgb}{0.000000,0.000000,0.000000}%
\pgfsetfillcolor{currentfill}%
\pgfsetlinewidth{0.602250pt}%
\definecolor{currentstroke}{rgb}{0.000000,0.000000,0.000000}%
\pgfsetstrokecolor{currentstroke}%
\pgfsetdash{}{0pt}%
\pgfsys@defobject{currentmarker}{\pgfqpoint{-0.027778in}{0.000000in}}{\pgfqpoint{0.000000in}{0.000000in}}{%
\pgfpathmoveto{\pgfqpoint{0.000000in}{0.000000in}}%
\pgfpathlineto{\pgfqpoint{-0.027778in}{0.000000in}}%
\pgfusepath{stroke,fill}%
}%
\begin{pgfscope}%
\pgfsys@transformshift{0.625000in}{2.385618in}%
\pgfsys@useobject{currentmarker}{}%
\end{pgfscope}%
\end{pgfscope}%
\begin{pgfscope}%
\pgfsetbuttcap%
\pgfsetroundjoin%
\definecolor{currentfill}{rgb}{0.000000,0.000000,0.000000}%
\pgfsetfillcolor{currentfill}%
\pgfsetlinewidth{0.602250pt}%
\definecolor{currentstroke}{rgb}{0.000000,0.000000,0.000000}%
\pgfsetstrokecolor{currentstroke}%
\pgfsetdash{}{0pt}%
\pgfsys@defobject{currentmarker}{\pgfqpoint{-0.027778in}{0.000000in}}{\pgfqpoint{0.000000in}{0.000000in}}{%
\pgfpathmoveto{\pgfqpoint{0.000000in}{0.000000in}}%
\pgfpathlineto{\pgfqpoint{-0.027778in}{0.000000in}}%
\pgfusepath{stroke,fill}%
}%
\begin{pgfscope}%
\pgfsys@transformshift{0.625000in}{2.479561in}%
\pgfsys@useobject{currentmarker}{}%
\end{pgfscope}%
\end{pgfscope}%
\begin{pgfscope}%
\pgfsetbuttcap%
\pgfsetroundjoin%
\definecolor{currentfill}{rgb}{0.000000,0.000000,0.000000}%
\pgfsetfillcolor{currentfill}%
\pgfsetlinewidth{0.602250pt}%
\definecolor{currentstroke}{rgb}{0.000000,0.000000,0.000000}%
\pgfsetstrokecolor{currentstroke}%
\pgfsetdash{}{0pt}%
\pgfsys@defobject{currentmarker}{\pgfqpoint{-0.027778in}{0.000000in}}{\pgfqpoint{0.000000in}{0.000000in}}{%
\pgfpathmoveto{\pgfqpoint{0.000000in}{0.000000in}}%
\pgfpathlineto{\pgfqpoint{-0.027778in}{0.000000in}}%
\pgfusepath{stroke,fill}%
}%
\begin{pgfscope}%
\pgfsys@transformshift{0.625000in}{2.558988in}%
\pgfsys@useobject{currentmarker}{}%
\end{pgfscope}%
\end{pgfscope}%
\begin{pgfscope}%
\pgfsetbuttcap%
\pgfsetroundjoin%
\definecolor{currentfill}{rgb}{0.000000,0.000000,0.000000}%
\pgfsetfillcolor{currentfill}%
\pgfsetlinewidth{0.602250pt}%
\definecolor{currentstroke}{rgb}{0.000000,0.000000,0.000000}%
\pgfsetstrokecolor{currentstroke}%
\pgfsetdash{}{0pt}%
\pgfsys@defobject{currentmarker}{\pgfqpoint{-0.027778in}{0.000000in}}{\pgfqpoint{0.000000in}{0.000000in}}{%
\pgfpathmoveto{\pgfqpoint{0.000000in}{0.000000in}}%
\pgfpathlineto{\pgfqpoint{-0.027778in}{0.000000in}}%
\pgfusepath{stroke,fill}%
}%
\begin{pgfscope}%
\pgfsys@transformshift{0.625000in}{2.627792in}%
\pgfsys@useobject{currentmarker}{}%
\end{pgfscope}%
\end{pgfscope}%
\begin{pgfscope}%
\pgfsetbuttcap%
\pgfsetroundjoin%
\definecolor{currentfill}{rgb}{0.000000,0.000000,0.000000}%
\pgfsetfillcolor{currentfill}%
\pgfsetlinewidth{0.602250pt}%
\definecolor{currentstroke}{rgb}{0.000000,0.000000,0.000000}%
\pgfsetstrokecolor{currentstroke}%
\pgfsetdash{}{0pt}%
\pgfsys@defobject{currentmarker}{\pgfqpoint{-0.027778in}{0.000000in}}{\pgfqpoint{0.000000in}{0.000000in}}{%
\pgfpathmoveto{\pgfqpoint{0.000000in}{0.000000in}}%
\pgfpathlineto{\pgfqpoint{-0.027778in}{0.000000in}}%
\pgfusepath{stroke,fill}%
}%
\begin{pgfscope}%
\pgfsys@transformshift{0.625000in}{2.688481in}%
\pgfsys@useobject{currentmarker}{}%
\end{pgfscope}%
\end{pgfscope}%
\begin{pgfscope}%
\pgfsetbuttcap%
\pgfsetroundjoin%
\definecolor{currentfill}{rgb}{0.000000,0.000000,0.000000}%
\pgfsetfillcolor{currentfill}%
\pgfsetlinewidth{0.602250pt}%
\definecolor{currentstroke}{rgb}{0.000000,0.000000,0.000000}%
\pgfsetstrokecolor{currentstroke}%
\pgfsetdash{}{0pt}%
\pgfsys@defobject{currentmarker}{\pgfqpoint{-0.027778in}{0.000000in}}{\pgfqpoint{0.000000in}{0.000000in}}{%
\pgfpathmoveto{\pgfqpoint{0.000000in}{0.000000in}}%
\pgfpathlineto{\pgfqpoint{-0.027778in}{0.000000in}}%
\pgfusepath{stroke,fill}%
}%
\begin{pgfscope}%
\pgfsys@transformshift{0.625000in}{3.099920in}%
\pgfsys@useobject{currentmarker}{}%
\end{pgfscope}%
\end{pgfscope}%
\begin{pgfscope}%
\pgfsetbuttcap%
\pgfsetroundjoin%
\definecolor{currentfill}{rgb}{0.000000,0.000000,0.000000}%
\pgfsetfillcolor{currentfill}%
\pgfsetlinewidth{0.602250pt}%
\definecolor{currentstroke}{rgb}{0.000000,0.000000,0.000000}%
\pgfsetstrokecolor{currentstroke}%
\pgfsetdash{}{0pt}%
\pgfsys@defobject{currentmarker}{\pgfqpoint{-0.027778in}{0.000000in}}{\pgfqpoint{0.000000in}{0.000000in}}{%
\pgfpathmoveto{\pgfqpoint{0.000000in}{0.000000in}}%
\pgfpathlineto{\pgfqpoint{-0.027778in}{0.000000in}}%
\pgfusepath{stroke,fill}%
}%
\begin{pgfscope}%
\pgfsys@transformshift{0.625000in}{3.308840in}%
\pgfsys@useobject{currentmarker}{}%
\end{pgfscope}%
\end{pgfscope}%
\begin{pgfscope}%
\pgfsetbuttcap%
\pgfsetroundjoin%
\definecolor{currentfill}{rgb}{0.000000,0.000000,0.000000}%
\pgfsetfillcolor{currentfill}%
\pgfsetlinewidth{0.602250pt}%
\definecolor{currentstroke}{rgb}{0.000000,0.000000,0.000000}%
\pgfsetstrokecolor{currentstroke}%
\pgfsetdash{}{0pt}%
\pgfsys@defobject{currentmarker}{\pgfqpoint{-0.027778in}{0.000000in}}{\pgfqpoint{0.000000in}{0.000000in}}{%
\pgfpathmoveto{\pgfqpoint{0.000000in}{0.000000in}}%
\pgfpathlineto{\pgfqpoint{-0.027778in}{0.000000in}}%
\pgfusepath{stroke,fill}%
}%
\begin{pgfscope}%
\pgfsys@transformshift{0.625000in}{3.457071in}%
\pgfsys@useobject{currentmarker}{}%
\end{pgfscope}%
\end{pgfscope}%
\begin{pgfscope}%
\definecolor{textcolor}{rgb}{0.000000,0.000000,0.000000}%
\pgfsetstrokecolor{textcolor}%
\pgfsetfillcolor{textcolor}%
\pgftext[x=0.184220in,y=1.980000in,,bottom,rotate=90.000000]{\color{textcolor}\sffamily\fontsize{10.000000}{12.000000}\selectfont QMC Error}%
\end{pgfscope}%
\begin{pgfscope}%
\pgfpathrectangle{\pgfqpoint{0.625000in}{0.440000in}}{\pgfqpoint{3.875000in}{3.080000in}}%
\pgfusepath{clip}%
\pgfsetbuttcap%
\pgfsetroundjoin%
\definecolor{currentfill}{rgb}{0.000000,0.000000,0.000000}%
\pgfsetfillcolor{currentfill}%
\pgfsetlinewidth{1.003750pt}%
\definecolor{currentstroke}{rgb}{0.000000,0.000000,0.000000}%
\pgfsetstrokecolor{currentstroke}%
\pgfsetdash{}{0pt}%
\pgfsys@defobject{currentmarker}{\pgfqpoint{-0.041667in}{-0.041667in}}{\pgfqpoint{0.041667in}{0.041667in}}{%
\pgfpathmoveto{\pgfqpoint{0.000000in}{-0.041667in}}%
\pgfpathcurveto{\pgfqpoint{0.011050in}{-0.041667in}}{\pgfqpoint{0.021649in}{-0.037276in}}{\pgfqpoint{0.029463in}{-0.029463in}}%
\pgfpathcurveto{\pgfqpoint{0.037276in}{-0.021649in}}{\pgfqpoint{0.041667in}{-0.011050in}}{\pgfqpoint{0.041667in}{0.000000in}}%
\pgfpathcurveto{\pgfqpoint{0.041667in}{0.011050in}}{\pgfqpoint{0.037276in}{0.021649in}}{\pgfqpoint{0.029463in}{0.029463in}}%
\pgfpathcurveto{\pgfqpoint{0.021649in}{0.037276in}}{\pgfqpoint{0.011050in}{0.041667in}}{\pgfqpoint{0.000000in}{0.041667in}}%
\pgfpathcurveto{\pgfqpoint{-0.011050in}{0.041667in}}{\pgfqpoint{-0.021649in}{0.037276in}}{\pgfqpoint{-0.029463in}{0.029463in}}%
\pgfpathcurveto{\pgfqpoint{-0.037276in}{0.021649in}}{\pgfqpoint{-0.041667in}{0.011050in}}{\pgfqpoint{-0.041667in}{0.000000in}}%
\pgfpathcurveto{\pgfqpoint{-0.041667in}{-0.011050in}}{\pgfqpoint{-0.037276in}{-0.021649in}}{\pgfqpoint{-0.029463in}{-0.029463in}}%
\pgfpathcurveto{\pgfqpoint{-0.021649in}{-0.037276in}}{\pgfqpoint{-0.011050in}{-0.041667in}}{\pgfqpoint{0.000000in}{-0.041667in}}%
\pgfpathclose%
\pgfusepath{stroke,fill}%
}%
\begin{pgfscope}%
\pgfsys@transformshift{0.801136in}{3.344809in}%
\pgfsys@useobject{currentmarker}{}%
\end{pgfscope}%
\begin{pgfscope}%
\pgfsys@transformshift{1.121384in}{3.066954in}%
\pgfsys@useobject{currentmarker}{}%
\end{pgfscope}%
\begin{pgfscope}%
\pgfsys@transformshift{1.441632in}{2.815121in}%
\pgfsys@useobject{currentmarker}{}%
\end{pgfscope}%
\begin{pgfscope}%
\pgfsys@transformshift{1.761880in}{2.554644in}%
\pgfsys@useobject{currentmarker}{}%
\end{pgfscope}%
\begin{pgfscope}%
\pgfsys@transformshift{2.082128in}{2.319456in}%
\pgfsys@useobject{currentmarker}{}%
\end{pgfscope}%
\begin{pgfscope}%
\pgfsys@transformshift{2.402376in}{2.151365in}%
\pgfsys@useobject{currentmarker}{}%
\end{pgfscope}%
\begin{pgfscope}%
\pgfsys@transformshift{2.722624in}{2.066804in}%
\pgfsys@useobject{currentmarker}{}%
\end{pgfscope}%
\begin{pgfscope}%
\pgfsys@transformshift{3.042872in}{2.048000in}%
\pgfsys@useobject{currentmarker}{}%
\end{pgfscope}%
\begin{pgfscope}%
\pgfsys@transformshift{3.363120in}{1.244977in}%
\pgfsys@useobject{currentmarker}{}%
\end{pgfscope}%
\begin{pgfscope}%
\pgfsys@transformshift{3.683368in}{0.876190in}%
\pgfsys@useobject{currentmarker}{}%
\end{pgfscope}%
\begin{pgfscope}%
\pgfsys@transformshift{4.003616in}{0.686605in}%
\pgfsys@useobject{currentmarker}{}%
\end{pgfscope}%
\begin{pgfscope}%
\pgfsys@transformshift{4.323864in}{0.585074in}%
\pgfsys@useobject{currentmarker}{}%
\end{pgfscope}%
\end{pgfscope}%
\begin{pgfscope}%
\pgfpathrectangle{\pgfqpoint{0.625000in}{0.440000in}}{\pgfqpoint{3.875000in}{3.080000in}}%
\pgfusepath{clip}%
\pgfsetbuttcap%
\pgfsetroundjoin%
\pgfsetlinewidth{1.505625pt}%
\definecolor{currentstroke}{rgb}{0.000000,0.000000,0.000000}%
\pgfsetstrokecolor{currentstroke}%
\pgfsetdash{{5.550000pt}{2.400000pt}}{0.000000pt}%
\pgfpathmoveto{\pgfqpoint{0.801136in}{3.380000in}}%
\pgfpathlineto{\pgfqpoint{4.323864in}{0.580000in}}%
\pgfusepath{stroke}%
\end{pgfscope}%
\begin{pgfscope}%
\pgfsetrectcap%
\pgfsetmiterjoin%
\pgfsetlinewidth{0.803000pt}%
\definecolor{currentstroke}{rgb}{0.000000,0.000000,0.000000}%
\pgfsetstrokecolor{currentstroke}%
\pgfsetdash{}{0pt}%
\pgfpathmoveto{\pgfqpoint{0.625000in}{0.440000in}}%
\pgfpathlineto{\pgfqpoint{0.625000in}{3.520000in}}%
\pgfusepath{stroke}%
\end{pgfscope}%
\begin{pgfscope}%
\pgfsetrectcap%
\pgfsetmiterjoin%
\pgfsetlinewidth{0.000000pt}%
\definecolor{currentstroke}{rgb}{0.000000,0.000000,0.000000}%
\pgfsetstrokecolor{currentstroke}%
\pgfsetstrokeopacity{0.000000}%
\pgfsetdash{}{0pt}%
\pgfpathmoveto{\pgfqpoint{4.500000in}{0.440000in}}%
\pgfpathlineto{\pgfqpoint{4.500000in}{3.520000in}}%
\pgfusepath{}%
\end{pgfscope}%
\begin{pgfscope}%
\pgfsetrectcap%
\pgfsetmiterjoin%
\pgfsetlinewidth{0.803000pt}%
\definecolor{currentstroke}{rgb}{0.000000,0.000000,0.000000}%
\pgfsetstrokecolor{currentstroke}%
\pgfsetdash{}{0pt}%
\pgfpathmoveto{\pgfqpoint{0.625000in}{0.440000in}}%
\pgfpathlineto{\pgfqpoint{4.500000in}{0.440000in}}%
\pgfusepath{stroke}%
\end{pgfscope}%
\begin{pgfscope}%
\pgfsetrectcap%
\pgfsetmiterjoin%
\pgfsetlinewidth{0.000000pt}%
\definecolor{currentstroke}{rgb}{0.000000,0.000000,0.000000}%
\pgfsetstrokecolor{currentstroke}%
\pgfsetstrokeopacity{0.000000}%
\pgfsetdash{}{0pt}%
\pgfpathmoveto{\pgfqpoint{0.625000in}{3.520000in}}%
\pgfpathlineto{\pgfqpoint{4.500000in}{3.520000in}}%
\pgfusepath{}%
\end{pgfscope}%
\begin{pgfscope}%
\pgfsetbuttcap%
\pgfsetmiterjoin%
\definecolor{currentfill}{rgb}{1.000000,1.000000,1.000000}%
\pgfsetfillcolor{currentfill}%
\pgfsetfillopacity{0.800000}%
\pgfsetlinewidth{1.003750pt}%
\definecolor{currentstroke}{rgb}{0.800000,0.800000,0.800000}%
\pgfsetstrokecolor{currentstroke}%
\pgfsetstrokeopacity{0.800000}%
\pgfsetdash{}{0pt}%
\pgfpathmoveto{\pgfqpoint{3.042243in}{3.155267in}}%
\pgfpathlineto{\pgfqpoint{4.402778in}{3.155267in}}%
\pgfpathquadraticcurveto{\pgfqpoint{4.430556in}{3.155267in}}{\pgfqpoint{4.430556in}{3.183045in}}%
\pgfpathlineto{\pgfqpoint{4.430556in}{3.422778in}}%
\pgfpathquadraticcurveto{\pgfqpoint{4.430556in}{3.450556in}}{\pgfqpoint{4.402778in}{3.450556in}}%
\pgfpathlineto{\pgfqpoint{3.042243in}{3.450556in}}%
\pgfpathquadraticcurveto{\pgfqpoint{3.014465in}{3.450556in}}{\pgfqpoint{3.014465in}{3.422778in}}%
\pgfpathlineto{\pgfqpoint{3.014465in}{3.183045in}}%
\pgfpathquadraticcurveto{\pgfqpoint{3.014465in}{3.155267in}}{\pgfqpoint{3.042243in}{3.155267in}}%
\pgfpathclose%
\pgfusepath{stroke,fill}%
\end{pgfscope}%
\begin{pgfscope}%
\pgfsetbuttcap%
\pgfsetroundjoin%
\pgfsetlinewidth{1.505625pt}%
\definecolor{currentstroke}{rgb}{0.000000,0.000000,0.000000}%
\pgfsetstrokecolor{currentstroke}%
\pgfsetdash{{5.550000pt}{2.400000pt}}{0.000000pt}%
\pgfpathmoveto{\pgfqpoint{3.070021in}{3.321543in}}%
\pgfpathlineto{\pgfqpoint{3.347799in}{3.321543in}}%
\pgfusepath{stroke}%
\end{pgfscope}%
\begin{pgfscope}%
\definecolor{textcolor}{rgb}{0.000000,0.000000,0.000000}%
\pgfsetstrokecolor{textcolor}%
\pgfsetfillcolor{textcolor}%
\pgftext[x=3.458910in,y=3.272932in,left,base]{\color{textcolor}\sffamily\fontsize{10.000000}{12.000000}\selectfont \(\displaystyle 0.0003N_{\mathrm{QMC}}^{-0.7127}\)}%
\end{pgfscope}%
\end{pgfpicture}%
\makeatother%
\endgroup%

    \caption[Quasi-Monte-Carlo error, for $Q=\int_D u$ and $k=50$.]{Quasi-Monte-Carlo error with increasing number of Quasi-Monte-Carlo points, for $Q=\int_D u$ and $k=50$. \label{fig:qmcintegral50}}
  \end{figure}
\begin{figure}[h]
  \centering
%% Creator: Matplotlib, PGF backend
%%
%% To include the figure in your LaTeX document, write
%%   \input{<filename>.pgf}
%%
%% Make sure the required packages are loaded in your preamble
%%   \usepackage{pgf}
%%
%% Figures using additional raster images can only be included by \input if
%% they are in the same directory as the main LaTeX file. For loading figures
%% from other directories you can use the `import` package
%%   \usepackage{import}
%% and then include the figures with
%%   \import{<path to file>}{<filename>.pgf}
%%
%% Matplotlib used the following preamble
%%   \usepackage{fontspec}
%%   \setmainfont{DejaVuSerif.ttf}[Path=/home/owen/progs/firedrake-complex/firedrake/lib/python3.5/site-packages/matplotlib/mpl-data/fonts/ttf/]
%%   \setsansfont{DejaVuSans.ttf}[Path=/home/owen/progs/firedrake-complex/firedrake/lib/python3.5/site-packages/matplotlib/mpl-data/fonts/ttf/]
%%   \setmonofont{DejaVuSansMono.ttf}[Path=/home/owen/progs/firedrake-complex/firedrake/lib/python3.5/site-packages/matplotlib/mpl-data/fonts/ttf/]
%%
\begingroup%
\makeatletter%
\begin{pgfpicture}%
\pgfpathrectangle{\pgfpointorigin}{\pgfqpoint{5.000000in}{4.000000in}}%
\pgfusepath{use as bounding box, clip}%
\begin{pgfscope}%
\pgfsetbuttcap%
\pgfsetmiterjoin%
\definecolor{currentfill}{rgb}{1.000000,1.000000,1.000000}%
\pgfsetfillcolor{currentfill}%
\pgfsetlinewidth{0.000000pt}%
\definecolor{currentstroke}{rgb}{1.000000,1.000000,1.000000}%
\pgfsetstrokecolor{currentstroke}%
\pgfsetdash{}{0pt}%
\pgfpathmoveto{\pgfqpoint{0.000000in}{0.000000in}}%
\pgfpathlineto{\pgfqpoint{5.000000in}{0.000000in}}%
\pgfpathlineto{\pgfqpoint{5.000000in}{4.000000in}}%
\pgfpathlineto{\pgfqpoint{0.000000in}{4.000000in}}%
\pgfpathclose%
\pgfusepath{fill}%
\end{pgfscope}%
\begin{pgfscope}%
\pgfsetbuttcap%
\pgfsetmiterjoin%
\definecolor{currentfill}{rgb}{1.000000,1.000000,1.000000}%
\pgfsetfillcolor{currentfill}%
\pgfsetlinewidth{0.000000pt}%
\definecolor{currentstroke}{rgb}{0.000000,0.000000,0.000000}%
\pgfsetstrokecolor{currentstroke}%
\pgfsetstrokeopacity{0.000000}%
\pgfsetdash{}{0pt}%
\pgfpathmoveto{\pgfqpoint{0.625000in}{0.440000in}}%
\pgfpathlineto{\pgfqpoint{4.500000in}{0.440000in}}%
\pgfpathlineto{\pgfqpoint{4.500000in}{3.520000in}}%
\pgfpathlineto{\pgfqpoint{0.625000in}{3.520000in}}%
\pgfpathclose%
\pgfusepath{fill}%
\end{pgfscope}%
\begin{pgfscope}%
\pgfsetbuttcap%
\pgfsetroundjoin%
\definecolor{currentfill}{rgb}{0.000000,0.000000,0.000000}%
\pgfsetfillcolor{currentfill}%
\pgfsetlinewidth{0.803000pt}%
\definecolor{currentstroke}{rgb}{0.000000,0.000000,0.000000}%
\pgfsetstrokecolor{currentstroke}%
\pgfsetdash{}{0pt}%
\pgfsys@defobject{currentmarker}{\pgfqpoint{0.000000in}{-0.048611in}}{\pgfqpoint{0.000000in}{0.000000in}}{%
\pgfpathmoveto{\pgfqpoint{0.000000in}{0.000000in}}%
\pgfpathlineto{\pgfqpoint{0.000000in}{-0.048611in}}%
\pgfusepath{stroke,fill}%
}%
\begin{pgfscope}%
\pgfsys@transformshift{0.801136in}{0.440000in}%
\pgfsys@useobject{currentmarker}{}%
\end{pgfscope}%
\end{pgfscope}%
\begin{pgfscope}%
\definecolor{textcolor}{rgb}{0.000000,0.000000,0.000000}%
\pgfsetstrokecolor{textcolor}%
\pgfsetfillcolor{textcolor}%
\pgftext[x=0.801136in,y=0.342778in,,top]{\color{textcolor}\sffamily\fontsize{10.000000}{12.000000}\selectfont \(\displaystyle 10^{0}\)}%
\end{pgfscope}%
\begin{pgfscope}%
\pgfsetbuttcap%
\pgfsetroundjoin%
\definecolor{currentfill}{rgb}{0.000000,0.000000,0.000000}%
\pgfsetfillcolor{currentfill}%
\pgfsetlinewidth{0.803000pt}%
\definecolor{currentstroke}{rgb}{0.000000,0.000000,0.000000}%
\pgfsetstrokecolor{currentstroke}%
\pgfsetdash{}{0pt}%
\pgfsys@defobject{currentmarker}{\pgfqpoint{0.000000in}{-0.048611in}}{\pgfqpoint{0.000000in}{0.000000in}}{%
\pgfpathmoveto{\pgfqpoint{0.000000in}{0.000000in}}%
\pgfpathlineto{\pgfqpoint{0.000000in}{-0.048611in}}%
\pgfusepath{stroke,fill}%
}%
\begin{pgfscope}%
\pgfsys@transformshift{1.864977in}{0.440000in}%
\pgfsys@useobject{currentmarker}{}%
\end{pgfscope}%
\end{pgfscope}%
\begin{pgfscope}%
\definecolor{textcolor}{rgb}{0.000000,0.000000,0.000000}%
\pgfsetstrokecolor{textcolor}%
\pgfsetfillcolor{textcolor}%
\pgftext[x=1.864977in,y=0.342778in,,top]{\color{textcolor}\sffamily\fontsize{10.000000}{12.000000}\selectfont \(\displaystyle 10^{1}\)}%
\end{pgfscope}%
\begin{pgfscope}%
\pgfsetbuttcap%
\pgfsetroundjoin%
\definecolor{currentfill}{rgb}{0.000000,0.000000,0.000000}%
\pgfsetfillcolor{currentfill}%
\pgfsetlinewidth{0.803000pt}%
\definecolor{currentstroke}{rgb}{0.000000,0.000000,0.000000}%
\pgfsetstrokecolor{currentstroke}%
\pgfsetdash{}{0pt}%
\pgfsys@defobject{currentmarker}{\pgfqpoint{0.000000in}{-0.048611in}}{\pgfqpoint{0.000000in}{0.000000in}}{%
\pgfpathmoveto{\pgfqpoint{0.000000in}{0.000000in}}%
\pgfpathlineto{\pgfqpoint{0.000000in}{-0.048611in}}%
\pgfusepath{stroke,fill}%
}%
\begin{pgfscope}%
\pgfsys@transformshift{2.928818in}{0.440000in}%
\pgfsys@useobject{currentmarker}{}%
\end{pgfscope}%
\end{pgfscope}%
\begin{pgfscope}%
\definecolor{textcolor}{rgb}{0.000000,0.000000,0.000000}%
\pgfsetstrokecolor{textcolor}%
\pgfsetfillcolor{textcolor}%
\pgftext[x=2.928818in,y=0.342778in,,top]{\color{textcolor}\sffamily\fontsize{10.000000}{12.000000}\selectfont \(\displaystyle 10^{2}\)}%
\end{pgfscope}%
\begin{pgfscope}%
\pgfsetbuttcap%
\pgfsetroundjoin%
\definecolor{currentfill}{rgb}{0.000000,0.000000,0.000000}%
\pgfsetfillcolor{currentfill}%
\pgfsetlinewidth{0.803000pt}%
\definecolor{currentstroke}{rgb}{0.000000,0.000000,0.000000}%
\pgfsetstrokecolor{currentstroke}%
\pgfsetdash{}{0pt}%
\pgfsys@defobject{currentmarker}{\pgfqpoint{0.000000in}{-0.048611in}}{\pgfqpoint{0.000000in}{0.000000in}}{%
\pgfpathmoveto{\pgfqpoint{0.000000in}{0.000000in}}%
\pgfpathlineto{\pgfqpoint{0.000000in}{-0.048611in}}%
\pgfusepath{stroke,fill}%
}%
\begin{pgfscope}%
\pgfsys@transformshift{3.992658in}{0.440000in}%
\pgfsys@useobject{currentmarker}{}%
\end{pgfscope}%
\end{pgfscope}%
\begin{pgfscope}%
\definecolor{textcolor}{rgb}{0.000000,0.000000,0.000000}%
\pgfsetstrokecolor{textcolor}%
\pgfsetfillcolor{textcolor}%
\pgftext[x=3.992658in,y=0.342778in,,top]{\color{textcolor}\sffamily\fontsize{10.000000}{12.000000}\selectfont \(\displaystyle 10^{3}\)}%
\end{pgfscope}%
\begin{pgfscope}%
\pgfsetbuttcap%
\pgfsetroundjoin%
\definecolor{currentfill}{rgb}{0.000000,0.000000,0.000000}%
\pgfsetfillcolor{currentfill}%
\pgfsetlinewidth{0.602250pt}%
\definecolor{currentstroke}{rgb}{0.000000,0.000000,0.000000}%
\pgfsetstrokecolor{currentstroke}%
\pgfsetdash{}{0pt}%
\pgfsys@defobject{currentmarker}{\pgfqpoint{0.000000in}{-0.027778in}}{\pgfqpoint{0.000000in}{0.000000in}}{%
\pgfpathmoveto{\pgfqpoint{0.000000in}{0.000000in}}%
\pgfpathlineto{\pgfqpoint{0.000000in}{-0.027778in}}%
\pgfusepath{stroke,fill}%
}%
\begin{pgfscope}%
\pgfsys@transformshift{0.636345in}{0.440000in}%
\pgfsys@useobject{currentmarker}{}%
\end{pgfscope}%
\end{pgfscope}%
\begin{pgfscope}%
\pgfsetbuttcap%
\pgfsetroundjoin%
\definecolor{currentfill}{rgb}{0.000000,0.000000,0.000000}%
\pgfsetfillcolor{currentfill}%
\pgfsetlinewidth{0.602250pt}%
\definecolor{currentstroke}{rgb}{0.000000,0.000000,0.000000}%
\pgfsetstrokecolor{currentstroke}%
\pgfsetdash{}{0pt}%
\pgfsys@defobject{currentmarker}{\pgfqpoint{0.000000in}{-0.027778in}}{\pgfqpoint{0.000000in}{0.000000in}}{%
\pgfpathmoveto{\pgfqpoint{0.000000in}{0.000000in}}%
\pgfpathlineto{\pgfqpoint{0.000000in}{-0.027778in}}%
\pgfusepath{stroke,fill}%
}%
\begin{pgfscope}%
\pgfsys@transformshift{0.698040in}{0.440000in}%
\pgfsys@useobject{currentmarker}{}%
\end{pgfscope}%
\end{pgfscope}%
\begin{pgfscope}%
\pgfsetbuttcap%
\pgfsetroundjoin%
\definecolor{currentfill}{rgb}{0.000000,0.000000,0.000000}%
\pgfsetfillcolor{currentfill}%
\pgfsetlinewidth{0.602250pt}%
\definecolor{currentstroke}{rgb}{0.000000,0.000000,0.000000}%
\pgfsetstrokecolor{currentstroke}%
\pgfsetdash{}{0pt}%
\pgfsys@defobject{currentmarker}{\pgfqpoint{0.000000in}{-0.027778in}}{\pgfqpoint{0.000000in}{0.000000in}}{%
\pgfpathmoveto{\pgfqpoint{0.000000in}{0.000000in}}%
\pgfpathlineto{\pgfqpoint{0.000000in}{-0.027778in}}%
\pgfusepath{stroke,fill}%
}%
\begin{pgfscope}%
\pgfsys@transformshift{0.752458in}{0.440000in}%
\pgfsys@useobject{currentmarker}{}%
\end{pgfscope}%
\end{pgfscope}%
\begin{pgfscope}%
\pgfsetbuttcap%
\pgfsetroundjoin%
\definecolor{currentfill}{rgb}{0.000000,0.000000,0.000000}%
\pgfsetfillcolor{currentfill}%
\pgfsetlinewidth{0.602250pt}%
\definecolor{currentstroke}{rgb}{0.000000,0.000000,0.000000}%
\pgfsetstrokecolor{currentstroke}%
\pgfsetdash{}{0pt}%
\pgfsys@defobject{currentmarker}{\pgfqpoint{0.000000in}{-0.027778in}}{\pgfqpoint{0.000000in}{0.000000in}}{%
\pgfpathmoveto{\pgfqpoint{0.000000in}{0.000000in}}%
\pgfpathlineto{\pgfqpoint{0.000000in}{-0.027778in}}%
\pgfusepath{stroke,fill}%
}%
\begin{pgfscope}%
\pgfsys@transformshift{1.121384in}{0.440000in}%
\pgfsys@useobject{currentmarker}{}%
\end{pgfscope}%
\end{pgfscope}%
\begin{pgfscope}%
\pgfsetbuttcap%
\pgfsetroundjoin%
\definecolor{currentfill}{rgb}{0.000000,0.000000,0.000000}%
\pgfsetfillcolor{currentfill}%
\pgfsetlinewidth{0.602250pt}%
\definecolor{currentstroke}{rgb}{0.000000,0.000000,0.000000}%
\pgfsetstrokecolor{currentstroke}%
\pgfsetdash{}{0pt}%
\pgfsys@defobject{currentmarker}{\pgfqpoint{0.000000in}{-0.027778in}}{\pgfqpoint{0.000000in}{0.000000in}}{%
\pgfpathmoveto{\pgfqpoint{0.000000in}{0.000000in}}%
\pgfpathlineto{\pgfqpoint{0.000000in}{-0.027778in}}%
\pgfusepath{stroke,fill}%
}%
\begin{pgfscope}%
\pgfsys@transformshift{1.308717in}{0.440000in}%
\pgfsys@useobject{currentmarker}{}%
\end{pgfscope}%
\end{pgfscope}%
\begin{pgfscope}%
\pgfsetbuttcap%
\pgfsetroundjoin%
\definecolor{currentfill}{rgb}{0.000000,0.000000,0.000000}%
\pgfsetfillcolor{currentfill}%
\pgfsetlinewidth{0.602250pt}%
\definecolor{currentstroke}{rgb}{0.000000,0.000000,0.000000}%
\pgfsetstrokecolor{currentstroke}%
\pgfsetdash{}{0pt}%
\pgfsys@defobject{currentmarker}{\pgfqpoint{0.000000in}{-0.027778in}}{\pgfqpoint{0.000000in}{0.000000in}}{%
\pgfpathmoveto{\pgfqpoint{0.000000in}{0.000000in}}%
\pgfpathlineto{\pgfqpoint{0.000000in}{-0.027778in}}%
\pgfusepath{stroke,fill}%
}%
\begin{pgfscope}%
\pgfsys@transformshift{1.441632in}{0.440000in}%
\pgfsys@useobject{currentmarker}{}%
\end{pgfscope}%
\end{pgfscope}%
\begin{pgfscope}%
\pgfsetbuttcap%
\pgfsetroundjoin%
\definecolor{currentfill}{rgb}{0.000000,0.000000,0.000000}%
\pgfsetfillcolor{currentfill}%
\pgfsetlinewidth{0.602250pt}%
\definecolor{currentstroke}{rgb}{0.000000,0.000000,0.000000}%
\pgfsetstrokecolor{currentstroke}%
\pgfsetdash{}{0pt}%
\pgfsys@defobject{currentmarker}{\pgfqpoint{0.000000in}{-0.027778in}}{\pgfqpoint{0.000000in}{0.000000in}}{%
\pgfpathmoveto{\pgfqpoint{0.000000in}{0.000000in}}%
\pgfpathlineto{\pgfqpoint{0.000000in}{-0.027778in}}%
\pgfusepath{stroke,fill}%
}%
\begin{pgfscope}%
\pgfsys@transformshift{1.544729in}{0.440000in}%
\pgfsys@useobject{currentmarker}{}%
\end{pgfscope}%
\end{pgfscope}%
\begin{pgfscope}%
\pgfsetbuttcap%
\pgfsetroundjoin%
\definecolor{currentfill}{rgb}{0.000000,0.000000,0.000000}%
\pgfsetfillcolor{currentfill}%
\pgfsetlinewidth{0.602250pt}%
\definecolor{currentstroke}{rgb}{0.000000,0.000000,0.000000}%
\pgfsetstrokecolor{currentstroke}%
\pgfsetdash{}{0pt}%
\pgfsys@defobject{currentmarker}{\pgfqpoint{0.000000in}{-0.027778in}}{\pgfqpoint{0.000000in}{0.000000in}}{%
\pgfpathmoveto{\pgfqpoint{0.000000in}{0.000000in}}%
\pgfpathlineto{\pgfqpoint{0.000000in}{-0.027778in}}%
\pgfusepath{stroke,fill}%
}%
\begin{pgfscope}%
\pgfsys@transformshift{1.628965in}{0.440000in}%
\pgfsys@useobject{currentmarker}{}%
\end{pgfscope}%
\end{pgfscope}%
\begin{pgfscope}%
\pgfsetbuttcap%
\pgfsetroundjoin%
\definecolor{currentfill}{rgb}{0.000000,0.000000,0.000000}%
\pgfsetfillcolor{currentfill}%
\pgfsetlinewidth{0.602250pt}%
\definecolor{currentstroke}{rgb}{0.000000,0.000000,0.000000}%
\pgfsetstrokecolor{currentstroke}%
\pgfsetdash{}{0pt}%
\pgfsys@defobject{currentmarker}{\pgfqpoint{0.000000in}{-0.027778in}}{\pgfqpoint{0.000000in}{0.000000in}}{%
\pgfpathmoveto{\pgfqpoint{0.000000in}{0.000000in}}%
\pgfpathlineto{\pgfqpoint{0.000000in}{-0.027778in}}%
\pgfusepath{stroke,fill}%
}%
\begin{pgfscope}%
\pgfsys@transformshift{1.700186in}{0.440000in}%
\pgfsys@useobject{currentmarker}{}%
\end{pgfscope}%
\end{pgfscope}%
\begin{pgfscope}%
\pgfsetbuttcap%
\pgfsetroundjoin%
\definecolor{currentfill}{rgb}{0.000000,0.000000,0.000000}%
\pgfsetfillcolor{currentfill}%
\pgfsetlinewidth{0.602250pt}%
\definecolor{currentstroke}{rgb}{0.000000,0.000000,0.000000}%
\pgfsetstrokecolor{currentstroke}%
\pgfsetdash{}{0pt}%
\pgfsys@defobject{currentmarker}{\pgfqpoint{0.000000in}{-0.027778in}}{\pgfqpoint{0.000000in}{0.000000in}}{%
\pgfpathmoveto{\pgfqpoint{0.000000in}{0.000000in}}%
\pgfpathlineto{\pgfqpoint{0.000000in}{-0.027778in}}%
\pgfusepath{stroke,fill}%
}%
\begin{pgfscope}%
\pgfsys@transformshift{1.761880in}{0.440000in}%
\pgfsys@useobject{currentmarker}{}%
\end{pgfscope}%
\end{pgfscope}%
\begin{pgfscope}%
\pgfsetbuttcap%
\pgfsetroundjoin%
\definecolor{currentfill}{rgb}{0.000000,0.000000,0.000000}%
\pgfsetfillcolor{currentfill}%
\pgfsetlinewidth{0.602250pt}%
\definecolor{currentstroke}{rgb}{0.000000,0.000000,0.000000}%
\pgfsetstrokecolor{currentstroke}%
\pgfsetdash{}{0pt}%
\pgfsys@defobject{currentmarker}{\pgfqpoint{0.000000in}{-0.027778in}}{\pgfqpoint{0.000000in}{0.000000in}}{%
\pgfpathmoveto{\pgfqpoint{0.000000in}{0.000000in}}%
\pgfpathlineto{\pgfqpoint{0.000000in}{-0.027778in}}%
\pgfusepath{stroke,fill}%
}%
\begin{pgfscope}%
\pgfsys@transformshift{1.816298in}{0.440000in}%
\pgfsys@useobject{currentmarker}{}%
\end{pgfscope}%
\end{pgfscope}%
\begin{pgfscope}%
\pgfsetbuttcap%
\pgfsetroundjoin%
\definecolor{currentfill}{rgb}{0.000000,0.000000,0.000000}%
\pgfsetfillcolor{currentfill}%
\pgfsetlinewidth{0.602250pt}%
\definecolor{currentstroke}{rgb}{0.000000,0.000000,0.000000}%
\pgfsetstrokecolor{currentstroke}%
\pgfsetdash{}{0pt}%
\pgfsys@defobject{currentmarker}{\pgfqpoint{0.000000in}{-0.027778in}}{\pgfqpoint{0.000000in}{0.000000in}}{%
\pgfpathmoveto{\pgfqpoint{0.000000in}{0.000000in}}%
\pgfpathlineto{\pgfqpoint{0.000000in}{-0.027778in}}%
\pgfusepath{stroke,fill}%
}%
\begin{pgfscope}%
\pgfsys@transformshift{2.185225in}{0.440000in}%
\pgfsys@useobject{currentmarker}{}%
\end{pgfscope}%
\end{pgfscope}%
\begin{pgfscope}%
\pgfsetbuttcap%
\pgfsetroundjoin%
\definecolor{currentfill}{rgb}{0.000000,0.000000,0.000000}%
\pgfsetfillcolor{currentfill}%
\pgfsetlinewidth{0.602250pt}%
\definecolor{currentstroke}{rgb}{0.000000,0.000000,0.000000}%
\pgfsetstrokecolor{currentstroke}%
\pgfsetdash{}{0pt}%
\pgfsys@defobject{currentmarker}{\pgfqpoint{0.000000in}{-0.027778in}}{\pgfqpoint{0.000000in}{0.000000in}}{%
\pgfpathmoveto{\pgfqpoint{0.000000in}{0.000000in}}%
\pgfpathlineto{\pgfqpoint{0.000000in}{-0.027778in}}%
\pgfusepath{stroke,fill}%
}%
\begin{pgfscope}%
\pgfsys@transformshift{2.372558in}{0.440000in}%
\pgfsys@useobject{currentmarker}{}%
\end{pgfscope}%
\end{pgfscope}%
\begin{pgfscope}%
\pgfsetbuttcap%
\pgfsetroundjoin%
\definecolor{currentfill}{rgb}{0.000000,0.000000,0.000000}%
\pgfsetfillcolor{currentfill}%
\pgfsetlinewidth{0.602250pt}%
\definecolor{currentstroke}{rgb}{0.000000,0.000000,0.000000}%
\pgfsetstrokecolor{currentstroke}%
\pgfsetdash{}{0pt}%
\pgfsys@defobject{currentmarker}{\pgfqpoint{0.000000in}{-0.027778in}}{\pgfqpoint{0.000000in}{0.000000in}}{%
\pgfpathmoveto{\pgfqpoint{0.000000in}{0.000000in}}%
\pgfpathlineto{\pgfqpoint{0.000000in}{-0.027778in}}%
\pgfusepath{stroke,fill}%
}%
\begin{pgfscope}%
\pgfsys@transformshift{2.505473in}{0.440000in}%
\pgfsys@useobject{currentmarker}{}%
\end{pgfscope}%
\end{pgfscope}%
\begin{pgfscope}%
\pgfsetbuttcap%
\pgfsetroundjoin%
\definecolor{currentfill}{rgb}{0.000000,0.000000,0.000000}%
\pgfsetfillcolor{currentfill}%
\pgfsetlinewidth{0.602250pt}%
\definecolor{currentstroke}{rgb}{0.000000,0.000000,0.000000}%
\pgfsetstrokecolor{currentstroke}%
\pgfsetdash{}{0pt}%
\pgfsys@defobject{currentmarker}{\pgfqpoint{0.000000in}{-0.027778in}}{\pgfqpoint{0.000000in}{0.000000in}}{%
\pgfpathmoveto{\pgfqpoint{0.000000in}{0.000000in}}%
\pgfpathlineto{\pgfqpoint{0.000000in}{-0.027778in}}%
\pgfusepath{stroke,fill}%
}%
\begin{pgfscope}%
\pgfsys@transformshift{2.608570in}{0.440000in}%
\pgfsys@useobject{currentmarker}{}%
\end{pgfscope}%
\end{pgfscope}%
\begin{pgfscope}%
\pgfsetbuttcap%
\pgfsetroundjoin%
\definecolor{currentfill}{rgb}{0.000000,0.000000,0.000000}%
\pgfsetfillcolor{currentfill}%
\pgfsetlinewidth{0.602250pt}%
\definecolor{currentstroke}{rgb}{0.000000,0.000000,0.000000}%
\pgfsetstrokecolor{currentstroke}%
\pgfsetdash{}{0pt}%
\pgfsys@defobject{currentmarker}{\pgfqpoint{0.000000in}{-0.027778in}}{\pgfqpoint{0.000000in}{0.000000in}}{%
\pgfpathmoveto{\pgfqpoint{0.000000in}{0.000000in}}%
\pgfpathlineto{\pgfqpoint{0.000000in}{-0.027778in}}%
\pgfusepath{stroke,fill}%
}%
\begin{pgfscope}%
\pgfsys@transformshift{2.692806in}{0.440000in}%
\pgfsys@useobject{currentmarker}{}%
\end{pgfscope}%
\end{pgfscope}%
\begin{pgfscope}%
\pgfsetbuttcap%
\pgfsetroundjoin%
\definecolor{currentfill}{rgb}{0.000000,0.000000,0.000000}%
\pgfsetfillcolor{currentfill}%
\pgfsetlinewidth{0.602250pt}%
\definecolor{currentstroke}{rgb}{0.000000,0.000000,0.000000}%
\pgfsetstrokecolor{currentstroke}%
\pgfsetdash{}{0pt}%
\pgfsys@defobject{currentmarker}{\pgfqpoint{0.000000in}{-0.027778in}}{\pgfqpoint{0.000000in}{0.000000in}}{%
\pgfpathmoveto{\pgfqpoint{0.000000in}{0.000000in}}%
\pgfpathlineto{\pgfqpoint{0.000000in}{-0.027778in}}%
\pgfusepath{stroke,fill}%
}%
\begin{pgfscope}%
\pgfsys@transformshift{2.764027in}{0.440000in}%
\pgfsys@useobject{currentmarker}{}%
\end{pgfscope}%
\end{pgfscope}%
\begin{pgfscope}%
\pgfsetbuttcap%
\pgfsetroundjoin%
\definecolor{currentfill}{rgb}{0.000000,0.000000,0.000000}%
\pgfsetfillcolor{currentfill}%
\pgfsetlinewidth{0.602250pt}%
\definecolor{currentstroke}{rgb}{0.000000,0.000000,0.000000}%
\pgfsetstrokecolor{currentstroke}%
\pgfsetdash{}{0pt}%
\pgfsys@defobject{currentmarker}{\pgfqpoint{0.000000in}{-0.027778in}}{\pgfqpoint{0.000000in}{0.000000in}}{%
\pgfpathmoveto{\pgfqpoint{0.000000in}{0.000000in}}%
\pgfpathlineto{\pgfqpoint{0.000000in}{-0.027778in}}%
\pgfusepath{stroke,fill}%
}%
\begin{pgfscope}%
\pgfsys@transformshift{2.825721in}{0.440000in}%
\pgfsys@useobject{currentmarker}{}%
\end{pgfscope}%
\end{pgfscope}%
\begin{pgfscope}%
\pgfsetbuttcap%
\pgfsetroundjoin%
\definecolor{currentfill}{rgb}{0.000000,0.000000,0.000000}%
\pgfsetfillcolor{currentfill}%
\pgfsetlinewidth{0.602250pt}%
\definecolor{currentstroke}{rgb}{0.000000,0.000000,0.000000}%
\pgfsetstrokecolor{currentstroke}%
\pgfsetdash{}{0pt}%
\pgfsys@defobject{currentmarker}{\pgfqpoint{0.000000in}{-0.027778in}}{\pgfqpoint{0.000000in}{0.000000in}}{%
\pgfpathmoveto{\pgfqpoint{0.000000in}{0.000000in}}%
\pgfpathlineto{\pgfqpoint{0.000000in}{-0.027778in}}%
\pgfusepath{stroke,fill}%
}%
\begin{pgfscope}%
\pgfsys@transformshift{2.880139in}{0.440000in}%
\pgfsys@useobject{currentmarker}{}%
\end{pgfscope}%
\end{pgfscope}%
\begin{pgfscope}%
\pgfsetbuttcap%
\pgfsetroundjoin%
\definecolor{currentfill}{rgb}{0.000000,0.000000,0.000000}%
\pgfsetfillcolor{currentfill}%
\pgfsetlinewidth{0.602250pt}%
\definecolor{currentstroke}{rgb}{0.000000,0.000000,0.000000}%
\pgfsetstrokecolor{currentstroke}%
\pgfsetdash{}{0pt}%
\pgfsys@defobject{currentmarker}{\pgfqpoint{0.000000in}{-0.027778in}}{\pgfqpoint{0.000000in}{0.000000in}}{%
\pgfpathmoveto{\pgfqpoint{0.000000in}{0.000000in}}%
\pgfpathlineto{\pgfqpoint{0.000000in}{-0.027778in}}%
\pgfusepath{stroke,fill}%
}%
\begin{pgfscope}%
\pgfsys@transformshift{3.249066in}{0.440000in}%
\pgfsys@useobject{currentmarker}{}%
\end{pgfscope}%
\end{pgfscope}%
\begin{pgfscope}%
\pgfsetbuttcap%
\pgfsetroundjoin%
\definecolor{currentfill}{rgb}{0.000000,0.000000,0.000000}%
\pgfsetfillcolor{currentfill}%
\pgfsetlinewidth{0.602250pt}%
\definecolor{currentstroke}{rgb}{0.000000,0.000000,0.000000}%
\pgfsetstrokecolor{currentstroke}%
\pgfsetdash{}{0pt}%
\pgfsys@defobject{currentmarker}{\pgfqpoint{0.000000in}{-0.027778in}}{\pgfqpoint{0.000000in}{0.000000in}}{%
\pgfpathmoveto{\pgfqpoint{0.000000in}{0.000000in}}%
\pgfpathlineto{\pgfqpoint{0.000000in}{-0.027778in}}%
\pgfusepath{stroke,fill}%
}%
\begin{pgfscope}%
\pgfsys@transformshift{3.436399in}{0.440000in}%
\pgfsys@useobject{currentmarker}{}%
\end{pgfscope}%
\end{pgfscope}%
\begin{pgfscope}%
\pgfsetbuttcap%
\pgfsetroundjoin%
\definecolor{currentfill}{rgb}{0.000000,0.000000,0.000000}%
\pgfsetfillcolor{currentfill}%
\pgfsetlinewidth{0.602250pt}%
\definecolor{currentstroke}{rgb}{0.000000,0.000000,0.000000}%
\pgfsetstrokecolor{currentstroke}%
\pgfsetdash{}{0pt}%
\pgfsys@defobject{currentmarker}{\pgfqpoint{0.000000in}{-0.027778in}}{\pgfqpoint{0.000000in}{0.000000in}}{%
\pgfpathmoveto{\pgfqpoint{0.000000in}{0.000000in}}%
\pgfpathlineto{\pgfqpoint{0.000000in}{-0.027778in}}%
\pgfusepath{stroke,fill}%
}%
\begin{pgfscope}%
\pgfsys@transformshift{3.569313in}{0.440000in}%
\pgfsys@useobject{currentmarker}{}%
\end{pgfscope}%
\end{pgfscope}%
\begin{pgfscope}%
\pgfsetbuttcap%
\pgfsetroundjoin%
\definecolor{currentfill}{rgb}{0.000000,0.000000,0.000000}%
\pgfsetfillcolor{currentfill}%
\pgfsetlinewidth{0.602250pt}%
\definecolor{currentstroke}{rgb}{0.000000,0.000000,0.000000}%
\pgfsetstrokecolor{currentstroke}%
\pgfsetdash{}{0pt}%
\pgfsys@defobject{currentmarker}{\pgfqpoint{0.000000in}{-0.027778in}}{\pgfqpoint{0.000000in}{0.000000in}}{%
\pgfpathmoveto{\pgfqpoint{0.000000in}{0.000000in}}%
\pgfpathlineto{\pgfqpoint{0.000000in}{-0.027778in}}%
\pgfusepath{stroke,fill}%
}%
\begin{pgfscope}%
\pgfsys@transformshift{3.672410in}{0.440000in}%
\pgfsys@useobject{currentmarker}{}%
\end{pgfscope}%
\end{pgfscope}%
\begin{pgfscope}%
\pgfsetbuttcap%
\pgfsetroundjoin%
\definecolor{currentfill}{rgb}{0.000000,0.000000,0.000000}%
\pgfsetfillcolor{currentfill}%
\pgfsetlinewidth{0.602250pt}%
\definecolor{currentstroke}{rgb}{0.000000,0.000000,0.000000}%
\pgfsetstrokecolor{currentstroke}%
\pgfsetdash{}{0pt}%
\pgfsys@defobject{currentmarker}{\pgfqpoint{0.000000in}{-0.027778in}}{\pgfqpoint{0.000000in}{0.000000in}}{%
\pgfpathmoveto{\pgfqpoint{0.000000in}{0.000000in}}%
\pgfpathlineto{\pgfqpoint{0.000000in}{-0.027778in}}%
\pgfusepath{stroke,fill}%
}%
\begin{pgfscope}%
\pgfsys@transformshift{3.756646in}{0.440000in}%
\pgfsys@useobject{currentmarker}{}%
\end{pgfscope}%
\end{pgfscope}%
\begin{pgfscope}%
\pgfsetbuttcap%
\pgfsetroundjoin%
\definecolor{currentfill}{rgb}{0.000000,0.000000,0.000000}%
\pgfsetfillcolor{currentfill}%
\pgfsetlinewidth{0.602250pt}%
\definecolor{currentstroke}{rgb}{0.000000,0.000000,0.000000}%
\pgfsetstrokecolor{currentstroke}%
\pgfsetdash{}{0pt}%
\pgfsys@defobject{currentmarker}{\pgfqpoint{0.000000in}{-0.027778in}}{\pgfqpoint{0.000000in}{0.000000in}}{%
\pgfpathmoveto{\pgfqpoint{0.000000in}{0.000000in}}%
\pgfpathlineto{\pgfqpoint{0.000000in}{-0.027778in}}%
\pgfusepath{stroke,fill}%
}%
\begin{pgfscope}%
\pgfsys@transformshift{3.827867in}{0.440000in}%
\pgfsys@useobject{currentmarker}{}%
\end{pgfscope}%
\end{pgfscope}%
\begin{pgfscope}%
\pgfsetbuttcap%
\pgfsetroundjoin%
\definecolor{currentfill}{rgb}{0.000000,0.000000,0.000000}%
\pgfsetfillcolor{currentfill}%
\pgfsetlinewidth{0.602250pt}%
\definecolor{currentstroke}{rgb}{0.000000,0.000000,0.000000}%
\pgfsetstrokecolor{currentstroke}%
\pgfsetdash{}{0pt}%
\pgfsys@defobject{currentmarker}{\pgfqpoint{0.000000in}{-0.027778in}}{\pgfqpoint{0.000000in}{0.000000in}}{%
\pgfpathmoveto{\pgfqpoint{0.000000in}{0.000000in}}%
\pgfpathlineto{\pgfqpoint{0.000000in}{-0.027778in}}%
\pgfusepath{stroke,fill}%
}%
\begin{pgfscope}%
\pgfsys@transformshift{3.889561in}{0.440000in}%
\pgfsys@useobject{currentmarker}{}%
\end{pgfscope}%
\end{pgfscope}%
\begin{pgfscope}%
\pgfsetbuttcap%
\pgfsetroundjoin%
\definecolor{currentfill}{rgb}{0.000000,0.000000,0.000000}%
\pgfsetfillcolor{currentfill}%
\pgfsetlinewidth{0.602250pt}%
\definecolor{currentstroke}{rgb}{0.000000,0.000000,0.000000}%
\pgfsetstrokecolor{currentstroke}%
\pgfsetdash{}{0pt}%
\pgfsys@defobject{currentmarker}{\pgfqpoint{0.000000in}{-0.027778in}}{\pgfqpoint{0.000000in}{0.000000in}}{%
\pgfpathmoveto{\pgfqpoint{0.000000in}{0.000000in}}%
\pgfpathlineto{\pgfqpoint{0.000000in}{-0.027778in}}%
\pgfusepath{stroke,fill}%
}%
\begin{pgfscope}%
\pgfsys@transformshift{3.943980in}{0.440000in}%
\pgfsys@useobject{currentmarker}{}%
\end{pgfscope}%
\end{pgfscope}%
\begin{pgfscope}%
\pgfsetbuttcap%
\pgfsetroundjoin%
\definecolor{currentfill}{rgb}{0.000000,0.000000,0.000000}%
\pgfsetfillcolor{currentfill}%
\pgfsetlinewidth{0.602250pt}%
\definecolor{currentstroke}{rgb}{0.000000,0.000000,0.000000}%
\pgfsetstrokecolor{currentstroke}%
\pgfsetdash{}{0pt}%
\pgfsys@defobject{currentmarker}{\pgfqpoint{0.000000in}{-0.027778in}}{\pgfqpoint{0.000000in}{0.000000in}}{%
\pgfpathmoveto{\pgfqpoint{0.000000in}{0.000000in}}%
\pgfpathlineto{\pgfqpoint{0.000000in}{-0.027778in}}%
\pgfusepath{stroke,fill}%
}%
\begin{pgfscope}%
\pgfsys@transformshift{4.312906in}{0.440000in}%
\pgfsys@useobject{currentmarker}{}%
\end{pgfscope}%
\end{pgfscope}%
\begin{pgfscope}%
\pgfsetbuttcap%
\pgfsetroundjoin%
\definecolor{currentfill}{rgb}{0.000000,0.000000,0.000000}%
\pgfsetfillcolor{currentfill}%
\pgfsetlinewidth{0.602250pt}%
\definecolor{currentstroke}{rgb}{0.000000,0.000000,0.000000}%
\pgfsetstrokecolor{currentstroke}%
\pgfsetdash{}{0pt}%
\pgfsys@defobject{currentmarker}{\pgfqpoint{0.000000in}{-0.027778in}}{\pgfqpoint{0.000000in}{0.000000in}}{%
\pgfpathmoveto{\pgfqpoint{0.000000in}{0.000000in}}%
\pgfpathlineto{\pgfqpoint{0.000000in}{-0.027778in}}%
\pgfusepath{stroke,fill}%
}%
\begin{pgfscope}%
\pgfsys@transformshift{4.500239in}{0.440000in}%
\pgfsys@useobject{currentmarker}{}%
\end{pgfscope}%
\end{pgfscope}%
\begin{pgfscope}%
\definecolor{textcolor}{rgb}{0.000000,0.000000,0.000000}%
\pgfsetstrokecolor{textcolor}%
\pgfsetfillcolor{textcolor}%
\pgftext[x=2.562500in,y=0.152809in,,top]{\color{textcolor}\sffamily\fontsize{10.000000}{12.000000}\selectfont \(\displaystyle N_\mathrm{QMC}\)}%
\end{pgfscope}%
\begin{pgfscope}%
\pgfsetbuttcap%
\pgfsetroundjoin%
\definecolor{currentfill}{rgb}{0.000000,0.000000,0.000000}%
\pgfsetfillcolor{currentfill}%
\pgfsetlinewidth{0.803000pt}%
\definecolor{currentstroke}{rgb}{0.000000,0.000000,0.000000}%
\pgfsetstrokecolor{currentstroke}%
\pgfsetdash{}{0pt}%
\pgfsys@defobject{currentmarker}{\pgfqpoint{-0.048611in}{0.000000in}}{\pgfqpoint{0.000000in}{0.000000in}}{%
\pgfpathmoveto{\pgfqpoint{0.000000in}{0.000000in}}%
\pgfpathlineto{\pgfqpoint{-0.048611in}{0.000000in}}%
\pgfusepath{stroke,fill}%
}%
\begin{pgfscope}%
\pgfsys@transformshift{0.625000in}{1.468214in}%
\pgfsys@useobject{currentmarker}{}%
\end{pgfscope}%
\end{pgfscope}%
\begin{pgfscope}%
\definecolor{textcolor}{rgb}{0.000000,0.000000,0.000000}%
\pgfsetstrokecolor{textcolor}%
\pgfsetfillcolor{textcolor}%
\pgftext[x=0.239775in,y=1.415452in,left,base]{\color{textcolor}\sffamily\fontsize{10.000000}{12.000000}\selectfont \(\displaystyle 10^{-5}\)}%
\end{pgfscope}%
\begin{pgfscope}%
\pgfsetbuttcap%
\pgfsetroundjoin%
\definecolor{currentfill}{rgb}{0.000000,0.000000,0.000000}%
\pgfsetfillcolor{currentfill}%
\pgfsetlinewidth{0.803000pt}%
\definecolor{currentstroke}{rgb}{0.000000,0.000000,0.000000}%
\pgfsetstrokecolor{currentstroke}%
\pgfsetdash{}{0pt}%
\pgfsys@defobject{currentmarker}{\pgfqpoint{-0.048611in}{0.000000in}}{\pgfqpoint{0.000000in}{0.000000in}}{%
\pgfpathmoveto{\pgfqpoint{0.000000in}{0.000000in}}%
\pgfpathlineto{\pgfqpoint{-0.048611in}{0.000000in}}%
\pgfusepath{stroke,fill}%
}%
\begin{pgfscope}%
\pgfsys@transformshift{0.625000in}{2.891641in}%
\pgfsys@useobject{currentmarker}{}%
\end{pgfscope}%
\end{pgfscope}%
\begin{pgfscope}%
\definecolor{textcolor}{rgb}{0.000000,0.000000,0.000000}%
\pgfsetstrokecolor{textcolor}%
\pgfsetfillcolor{textcolor}%
\pgftext[x=0.239775in,y=2.838879in,left,base]{\color{textcolor}\sffamily\fontsize{10.000000}{12.000000}\selectfont \(\displaystyle 10^{-4}\)}%
\end{pgfscope}%
\begin{pgfscope}%
\pgfsetbuttcap%
\pgfsetroundjoin%
\definecolor{currentfill}{rgb}{0.000000,0.000000,0.000000}%
\pgfsetfillcolor{currentfill}%
\pgfsetlinewidth{0.602250pt}%
\definecolor{currentstroke}{rgb}{0.000000,0.000000,0.000000}%
\pgfsetstrokecolor{currentstroke}%
\pgfsetdash{}{0pt}%
\pgfsys@defobject{currentmarker}{\pgfqpoint{-0.027778in}{0.000000in}}{\pgfqpoint{0.000000in}{0.000000in}}{%
\pgfpathmoveto{\pgfqpoint{0.000000in}{0.000000in}}%
\pgfpathlineto{\pgfqpoint{-0.027778in}{0.000000in}}%
\pgfusepath{stroke,fill}%
}%
\begin{pgfscope}%
\pgfsys@transformshift{0.625000in}{0.473281in}%
\pgfsys@useobject{currentmarker}{}%
\end{pgfscope}%
\end{pgfscope}%
\begin{pgfscope}%
\pgfsetbuttcap%
\pgfsetroundjoin%
\definecolor{currentfill}{rgb}{0.000000,0.000000,0.000000}%
\pgfsetfillcolor{currentfill}%
\pgfsetlinewidth{0.602250pt}%
\definecolor{currentstroke}{rgb}{0.000000,0.000000,0.000000}%
\pgfsetstrokecolor{currentstroke}%
\pgfsetdash{}{0pt}%
\pgfsys@defobject{currentmarker}{\pgfqpoint{-0.027778in}{0.000000in}}{\pgfqpoint{0.000000in}{0.000000in}}{%
\pgfpathmoveto{\pgfqpoint{0.000000in}{0.000000in}}%
\pgfpathlineto{\pgfqpoint{-0.027778in}{0.000000in}}%
\pgfusepath{stroke,fill}%
}%
\begin{pgfscope}%
\pgfsys@transformshift{0.625000in}{0.723934in}%
\pgfsys@useobject{currentmarker}{}%
\end{pgfscope}%
\end{pgfscope}%
\begin{pgfscope}%
\pgfsetbuttcap%
\pgfsetroundjoin%
\definecolor{currentfill}{rgb}{0.000000,0.000000,0.000000}%
\pgfsetfillcolor{currentfill}%
\pgfsetlinewidth{0.602250pt}%
\definecolor{currentstroke}{rgb}{0.000000,0.000000,0.000000}%
\pgfsetstrokecolor{currentstroke}%
\pgfsetdash{}{0pt}%
\pgfsys@defobject{currentmarker}{\pgfqpoint{-0.027778in}{0.000000in}}{\pgfqpoint{0.000000in}{0.000000in}}{%
\pgfpathmoveto{\pgfqpoint{0.000000in}{0.000000in}}%
\pgfpathlineto{\pgfqpoint{-0.027778in}{0.000000in}}%
\pgfusepath{stroke,fill}%
}%
\begin{pgfscope}%
\pgfsys@transformshift{0.625000in}{0.901775in}%
\pgfsys@useobject{currentmarker}{}%
\end{pgfscope}%
\end{pgfscope}%
\begin{pgfscope}%
\pgfsetbuttcap%
\pgfsetroundjoin%
\definecolor{currentfill}{rgb}{0.000000,0.000000,0.000000}%
\pgfsetfillcolor{currentfill}%
\pgfsetlinewidth{0.602250pt}%
\definecolor{currentstroke}{rgb}{0.000000,0.000000,0.000000}%
\pgfsetstrokecolor{currentstroke}%
\pgfsetdash{}{0pt}%
\pgfsys@defobject{currentmarker}{\pgfqpoint{-0.027778in}{0.000000in}}{\pgfqpoint{0.000000in}{0.000000in}}{%
\pgfpathmoveto{\pgfqpoint{0.000000in}{0.000000in}}%
\pgfpathlineto{\pgfqpoint{-0.027778in}{0.000000in}}%
\pgfusepath{stroke,fill}%
}%
\begin{pgfscope}%
\pgfsys@transformshift{0.625000in}{1.039720in}%
\pgfsys@useobject{currentmarker}{}%
\end{pgfscope}%
\end{pgfscope}%
\begin{pgfscope}%
\pgfsetbuttcap%
\pgfsetroundjoin%
\definecolor{currentfill}{rgb}{0.000000,0.000000,0.000000}%
\pgfsetfillcolor{currentfill}%
\pgfsetlinewidth{0.602250pt}%
\definecolor{currentstroke}{rgb}{0.000000,0.000000,0.000000}%
\pgfsetstrokecolor{currentstroke}%
\pgfsetdash{}{0pt}%
\pgfsys@defobject{currentmarker}{\pgfqpoint{-0.027778in}{0.000000in}}{\pgfqpoint{0.000000in}{0.000000in}}{%
\pgfpathmoveto{\pgfqpoint{0.000000in}{0.000000in}}%
\pgfpathlineto{\pgfqpoint{-0.027778in}{0.000000in}}%
\pgfusepath{stroke,fill}%
}%
\begin{pgfscope}%
\pgfsys@transformshift{0.625000in}{1.152428in}%
\pgfsys@useobject{currentmarker}{}%
\end{pgfscope}%
\end{pgfscope}%
\begin{pgfscope}%
\pgfsetbuttcap%
\pgfsetroundjoin%
\definecolor{currentfill}{rgb}{0.000000,0.000000,0.000000}%
\pgfsetfillcolor{currentfill}%
\pgfsetlinewidth{0.602250pt}%
\definecolor{currentstroke}{rgb}{0.000000,0.000000,0.000000}%
\pgfsetstrokecolor{currentstroke}%
\pgfsetdash{}{0pt}%
\pgfsys@defobject{currentmarker}{\pgfqpoint{-0.027778in}{0.000000in}}{\pgfqpoint{0.000000in}{0.000000in}}{%
\pgfpathmoveto{\pgfqpoint{0.000000in}{0.000000in}}%
\pgfpathlineto{\pgfqpoint{-0.027778in}{0.000000in}}%
\pgfusepath{stroke,fill}%
}%
\begin{pgfscope}%
\pgfsys@transformshift{0.625000in}{1.247722in}%
\pgfsys@useobject{currentmarker}{}%
\end{pgfscope}%
\end{pgfscope}%
\begin{pgfscope}%
\pgfsetbuttcap%
\pgfsetroundjoin%
\definecolor{currentfill}{rgb}{0.000000,0.000000,0.000000}%
\pgfsetfillcolor{currentfill}%
\pgfsetlinewidth{0.602250pt}%
\definecolor{currentstroke}{rgb}{0.000000,0.000000,0.000000}%
\pgfsetstrokecolor{currentstroke}%
\pgfsetdash{}{0pt}%
\pgfsys@defobject{currentmarker}{\pgfqpoint{-0.027778in}{0.000000in}}{\pgfqpoint{0.000000in}{0.000000in}}{%
\pgfpathmoveto{\pgfqpoint{0.000000in}{0.000000in}}%
\pgfpathlineto{\pgfqpoint{-0.027778in}{0.000000in}}%
\pgfusepath{stroke,fill}%
}%
\begin{pgfscope}%
\pgfsys@transformshift{0.625000in}{1.330269in}%
\pgfsys@useobject{currentmarker}{}%
\end{pgfscope}%
\end{pgfscope}%
\begin{pgfscope}%
\pgfsetbuttcap%
\pgfsetroundjoin%
\definecolor{currentfill}{rgb}{0.000000,0.000000,0.000000}%
\pgfsetfillcolor{currentfill}%
\pgfsetlinewidth{0.602250pt}%
\definecolor{currentstroke}{rgb}{0.000000,0.000000,0.000000}%
\pgfsetstrokecolor{currentstroke}%
\pgfsetdash{}{0pt}%
\pgfsys@defobject{currentmarker}{\pgfqpoint{-0.027778in}{0.000000in}}{\pgfqpoint{0.000000in}{0.000000in}}{%
\pgfpathmoveto{\pgfqpoint{0.000000in}{0.000000in}}%
\pgfpathlineto{\pgfqpoint{-0.027778in}{0.000000in}}%
\pgfusepath{stroke,fill}%
}%
\begin{pgfscope}%
\pgfsys@transformshift{0.625000in}{1.403081in}%
\pgfsys@useobject{currentmarker}{}%
\end{pgfscope}%
\end{pgfscope}%
\begin{pgfscope}%
\pgfsetbuttcap%
\pgfsetroundjoin%
\definecolor{currentfill}{rgb}{0.000000,0.000000,0.000000}%
\pgfsetfillcolor{currentfill}%
\pgfsetlinewidth{0.602250pt}%
\definecolor{currentstroke}{rgb}{0.000000,0.000000,0.000000}%
\pgfsetstrokecolor{currentstroke}%
\pgfsetdash{}{0pt}%
\pgfsys@defobject{currentmarker}{\pgfqpoint{-0.027778in}{0.000000in}}{\pgfqpoint{0.000000in}{0.000000in}}{%
\pgfpathmoveto{\pgfqpoint{0.000000in}{0.000000in}}%
\pgfpathlineto{\pgfqpoint{-0.027778in}{0.000000in}}%
\pgfusepath{stroke,fill}%
}%
\begin{pgfscope}%
\pgfsys@transformshift{0.625000in}{1.896708in}%
\pgfsys@useobject{currentmarker}{}%
\end{pgfscope}%
\end{pgfscope}%
\begin{pgfscope}%
\pgfsetbuttcap%
\pgfsetroundjoin%
\definecolor{currentfill}{rgb}{0.000000,0.000000,0.000000}%
\pgfsetfillcolor{currentfill}%
\pgfsetlinewidth{0.602250pt}%
\definecolor{currentstroke}{rgb}{0.000000,0.000000,0.000000}%
\pgfsetstrokecolor{currentstroke}%
\pgfsetdash{}{0pt}%
\pgfsys@defobject{currentmarker}{\pgfqpoint{-0.027778in}{0.000000in}}{\pgfqpoint{0.000000in}{0.000000in}}{%
\pgfpathmoveto{\pgfqpoint{0.000000in}{0.000000in}}%
\pgfpathlineto{\pgfqpoint{-0.027778in}{0.000000in}}%
\pgfusepath{stroke,fill}%
}%
\begin{pgfscope}%
\pgfsys@transformshift{0.625000in}{2.147361in}%
\pgfsys@useobject{currentmarker}{}%
\end{pgfscope}%
\end{pgfscope}%
\begin{pgfscope}%
\pgfsetbuttcap%
\pgfsetroundjoin%
\definecolor{currentfill}{rgb}{0.000000,0.000000,0.000000}%
\pgfsetfillcolor{currentfill}%
\pgfsetlinewidth{0.602250pt}%
\definecolor{currentstroke}{rgb}{0.000000,0.000000,0.000000}%
\pgfsetstrokecolor{currentstroke}%
\pgfsetdash{}{0pt}%
\pgfsys@defobject{currentmarker}{\pgfqpoint{-0.027778in}{0.000000in}}{\pgfqpoint{0.000000in}{0.000000in}}{%
\pgfpathmoveto{\pgfqpoint{0.000000in}{0.000000in}}%
\pgfpathlineto{\pgfqpoint{-0.027778in}{0.000000in}}%
\pgfusepath{stroke,fill}%
}%
\begin{pgfscope}%
\pgfsys@transformshift{0.625000in}{2.325202in}%
\pgfsys@useobject{currentmarker}{}%
\end{pgfscope}%
\end{pgfscope}%
\begin{pgfscope}%
\pgfsetbuttcap%
\pgfsetroundjoin%
\definecolor{currentfill}{rgb}{0.000000,0.000000,0.000000}%
\pgfsetfillcolor{currentfill}%
\pgfsetlinewidth{0.602250pt}%
\definecolor{currentstroke}{rgb}{0.000000,0.000000,0.000000}%
\pgfsetstrokecolor{currentstroke}%
\pgfsetdash{}{0pt}%
\pgfsys@defobject{currentmarker}{\pgfqpoint{-0.027778in}{0.000000in}}{\pgfqpoint{0.000000in}{0.000000in}}{%
\pgfpathmoveto{\pgfqpoint{0.000000in}{0.000000in}}%
\pgfpathlineto{\pgfqpoint{-0.027778in}{0.000000in}}%
\pgfusepath{stroke,fill}%
}%
\begin{pgfscope}%
\pgfsys@transformshift{0.625000in}{2.463147in}%
\pgfsys@useobject{currentmarker}{}%
\end{pgfscope}%
\end{pgfscope}%
\begin{pgfscope}%
\pgfsetbuttcap%
\pgfsetroundjoin%
\definecolor{currentfill}{rgb}{0.000000,0.000000,0.000000}%
\pgfsetfillcolor{currentfill}%
\pgfsetlinewidth{0.602250pt}%
\definecolor{currentstroke}{rgb}{0.000000,0.000000,0.000000}%
\pgfsetstrokecolor{currentstroke}%
\pgfsetdash{}{0pt}%
\pgfsys@defobject{currentmarker}{\pgfqpoint{-0.027778in}{0.000000in}}{\pgfqpoint{0.000000in}{0.000000in}}{%
\pgfpathmoveto{\pgfqpoint{0.000000in}{0.000000in}}%
\pgfpathlineto{\pgfqpoint{-0.027778in}{0.000000in}}%
\pgfusepath{stroke,fill}%
}%
\begin{pgfscope}%
\pgfsys@transformshift{0.625000in}{2.575855in}%
\pgfsys@useobject{currentmarker}{}%
\end{pgfscope}%
\end{pgfscope}%
\begin{pgfscope}%
\pgfsetbuttcap%
\pgfsetroundjoin%
\definecolor{currentfill}{rgb}{0.000000,0.000000,0.000000}%
\pgfsetfillcolor{currentfill}%
\pgfsetlinewidth{0.602250pt}%
\definecolor{currentstroke}{rgb}{0.000000,0.000000,0.000000}%
\pgfsetstrokecolor{currentstroke}%
\pgfsetdash{}{0pt}%
\pgfsys@defobject{currentmarker}{\pgfqpoint{-0.027778in}{0.000000in}}{\pgfqpoint{0.000000in}{0.000000in}}{%
\pgfpathmoveto{\pgfqpoint{0.000000in}{0.000000in}}%
\pgfpathlineto{\pgfqpoint{-0.027778in}{0.000000in}}%
\pgfusepath{stroke,fill}%
}%
\begin{pgfscope}%
\pgfsys@transformshift{0.625000in}{2.671149in}%
\pgfsys@useobject{currentmarker}{}%
\end{pgfscope}%
\end{pgfscope}%
\begin{pgfscope}%
\pgfsetbuttcap%
\pgfsetroundjoin%
\definecolor{currentfill}{rgb}{0.000000,0.000000,0.000000}%
\pgfsetfillcolor{currentfill}%
\pgfsetlinewidth{0.602250pt}%
\definecolor{currentstroke}{rgb}{0.000000,0.000000,0.000000}%
\pgfsetstrokecolor{currentstroke}%
\pgfsetdash{}{0pt}%
\pgfsys@defobject{currentmarker}{\pgfqpoint{-0.027778in}{0.000000in}}{\pgfqpoint{0.000000in}{0.000000in}}{%
\pgfpathmoveto{\pgfqpoint{0.000000in}{0.000000in}}%
\pgfpathlineto{\pgfqpoint{-0.027778in}{0.000000in}}%
\pgfusepath{stroke,fill}%
}%
\begin{pgfscope}%
\pgfsys@transformshift{0.625000in}{2.753696in}%
\pgfsys@useobject{currentmarker}{}%
\end{pgfscope}%
\end{pgfscope}%
\begin{pgfscope}%
\pgfsetbuttcap%
\pgfsetroundjoin%
\definecolor{currentfill}{rgb}{0.000000,0.000000,0.000000}%
\pgfsetfillcolor{currentfill}%
\pgfsetlinewidth{0.602250pt}%
\definecolor{currentstroke}{rgb}{0.000000,0.000000,0.000000}%
\pgfsetstrokecolor{currentstroke}%
\pgfsetdash{}{0pt}%
\pgfsys@defobject{currentmarker}{\pgfqpoint{-0.027778in}{0.000000in}}{\pgfqpoint{0.000000in}{0.000000in}}{%
\pgfpathmoveto{\pgfqpoint{0.000000in}{0.000000in}}%
\pgfpathlineto{\pgfqpoint{-0.027778in}{0.000000in}}%
\pgfusepath{stroke,fill}%
}%
\begin{pgfscope}%
\pgfsys@transformshift{0.625000in}{2.826508in}%
\pgfsys@useobject{currentmarker}{}%
\end{pgfscope}%
\end{pgfscope}%
\begin{pgfscope}%
\pgfsetbuttcap%
\pgfsetroundjoin%
\definecolor{currentfill}{rgb}{0.000000,0.000000,0.000000}%
\pgfsetfillcolor{currentfill}%
\pgfsetlinewidth{0.602250pt}%
\definecolor{currentstroke}{rgb}{0.000000,0.000000,0.000000}%
\pgfsetstrokecolor{currentstroke}%
\pgfsetdash{}{0pt}%
\pgfsys@defobject{currentmarker}{\pgfqpoint{-0.027778in}{0.000000in}}{\pgfqpoint{0.000000in}{0.000000in}}{%
\pgfpathmoveto{\pgfqpoint{0.000000in}{0.000000in}}%
\pgfpathlineto{\pgfqpoint{-0.027778in}{0.000000in}}%
\pgfusepath{stroke,fill}%
}%
\begin{pgfscope}%
\pgfsys@transformshift{0.625000in}{3.320135in}%
\pgfsys@useobject{currentmarker}{}%
\end{pgfscope}%
\end{pgfscope}%
\begin{pgfscope}%
\definecolor{textcolor}{rgb}{0.000000,0.000000,0.000000}%
\pgfsetstrokecolor{textcolor}%
\pgfsetfillcolor{textcolor}%
\pgftext[x=0.184220in,y=1.980000in,,bottom,rotate=90.000000]{\color{textcolor}\sffamily\fontsize{10.000000}{12.000000}\selectfont QMC Error}%
\end{pgfscope}%
\begin{pgfscope}%
\pgfpathrectangle{\pgfqpoint{0.625000in}{0.440000in}}{\pgfqpoint{3.875000in}{3.080000in}}%
\pgfusepath{clip}%
\pgfsetbuttcap%
\pgfsetroundjoin%
\definecolor{currentfill}{rgb}{0.000000,0.000000,0.000000}%
\pgfsetfillcolor{currentfill}%
\pgfsetlinewidth{1.003750pt}%
\definecolor{currentstroke}{rgb}{0.000000,0.000000,0.000000}%
\pgfsetstrokecolor{currentstroke}%
\pgfsetdash{}{0pt}%
\pgfsys@defobject{currentmarker}{\pgfqpoint{-0.041667in}{-0.041667in}}{\pgfqpoint{0.041667in}{0.041667in}}{%
\pgfpathmoveto{\pgfqpoint{0.000000in}{-0.041667in}}%
\pgfpathcurveto{\pgfqpoint{0.011050in}{-0.041667in}}{\pgfqpoint{0.021649in}{-0.037276in}}{\pgfqpoint{0.029463in}{-0.029463in}}%
\pgfpathcurveto{\pgfqpoint{0.037276in}{-0.021649in}}{\pgfqpoint{0.041667in}{-0.011050in}}{\pgfqpoint{0.041667in}{0.000000in}}%
\pgfpathcurveto{\pgfqpoint{0.041667in}{0.011050in}}{\pgfqpoint{0.037276in}{0.021649in}}{\pgfqpoint{0.029463in}{0.029463in}}%
\pgfpathcurveto{\pgfqpoint{0.021649in}{0.037276in}}{\pgfqpoint{0.011050in}{0.041667in}}{\pgfqpoint{0.000000in}{0.041667in}}%
\pgfpathcurveto{\pgfqpoint{-0.011050in}{0.041667in}}{\pgfqpoint{-0.021649in}{0.037276in}}{\pgfqpoint{-0.029463in}{0.029463in}}%
\pgfpathcurveto{\pgfqpoint{-0.037276in}{0.021649in}}{\pgfqpoint{-0.041667in}{0.011050in}}{\pgfqpoint{-0.041667in}{0.000000in}}%
\pgfpathcurveto{\pgfqpoint{-0.041667in}{-0.011050in}}{\pgfqpoint{-0.037276in}{-0.021649in}}{\pgfqpoint{-0.029463in}{-0.029463in}}%
\pgfpathcurveto{\pgfqpoint{-0.021649in}{-0.037276in}}{\pgfqpoint{-0.011050in}{-0.041667in}}{\pgfqpoint{0.000000in}{-0.041667in}}%
\pgfpathclose%
\pgfusepath{stroke,fill}%
}%
\begin{pgfscope}%
\pgfsys@transformshift{0.801136in}{3.362454in}%
\pgfsys@useobject{currentmarker}{}%
\end{pgfscope}%
\begin{pgfscope}%
\pgfsys@transformshift{1.121384in}{3.128092in}%
\pgfsys@useobject{currentmarker}{}%
\end{pgfscope}%
\begin{pgfscope}%
\pgfsys@transformshift{1.441632in}{2.856268in}%
\pgfsys@useobject{currentmarker}{}%
\end{pgfscope}%
\begin{pgfscope}%
\pgfsys@transformshift{1.761880in}{2.590998in}%
\pgfsys@useobject{currentmarker}{}%
\end{pgfscope}%
\begin{pgfscope}%
\pgfsys@transformshift{2.082128in}{2.479748in}%
\pgfsys@useobject{currentmarker}{}%
\end{pgfscope}%
\begin{pgfscope}%
\pgfsys@transformshift{2.402376in}{2.028177in}%
\pgfsys@useobject{currentmarker}{}%
\end{pgfscope}%
\begin{pgfscope}%
\pgfsys@transformshift{2.722624in}{1.936282in}%
\pgfsys@useobject{currentmarker}{}%
\end{pgfscope}%
\begin{pgfscope}%
\pgfsys@transformshift{3.042872in}{1.843050in}%
\pgfsys@useobject{currentmarker}{}%
\end{pgfscope}%
\begin{pgfscope}%
\pgfsys@transformshift{3.363120in}{1.119653in}%
\pgfsys@useobject{currentmarker}{}%
\end{pgfscope}%
\begin{pgfscope}%
\pgfsys@transformshift{3.683368in}{0.808905in}%
\pgfsys@useobject{currentmarker}{}%
\end{pgfscope}%
\begin{pgfscope}%
\pgfsys@transformshift{4.003616in}{0.820012in}%
\pgfsys@useobject{currentmarker}{}%
\end{pgfscope}%
\begin{pgfscope}%
\pgfsys@transformshift{4.323864in}{0.786361in}%
\pgfsys@useobject{currentmarker}{}%
\end{pgfscope}%
\end{pgfscope}%
\begin{pgfscope}%
\pgfpathrectangle{\pgfqpoint{0.625000in}{0.440000in}}{\pgfqpoint{3.875000in}{3.080000in}}%
\pgfusepath{clip}%
\pgfsetbuttcap%
\pgfsetroundjoin%
\pgfsetlinewidth{1.505625pt}%
\definecolor{currentstroke}{rgb}{0.000000,0.000000,0.000000}%
\pgfsetstrokecolor{currentstroke}%
\pgfsetdash{{5.550000pt}{2.400000pt}}{0.000000pt}%
\pgfpathmoveto{\pgfqpoint{0.801136in}{3.380000in}}%
\pgfpathlineto{\pgfqpoint{4.323864in}{0.580000in}}%
\pgfusepath{stroke}%
\end{pgfscope}%
\begin{pgfscope}%
\pgfsetrectcap%
\pgfsetmiterjoin%
\pgfsetlinewidth{0.803000pt}%
\definecolor{currentstroke}{rgb}{0.000000,0.000000,0.000000}%
\pgfsetstrokecolor{currentstroke}%
\pgfsetdash{}{0pt}%
\pgfpathmoveto{\pgfqpoint{0.625000in}{0.440000in}}%
\pgfpathlineto{\pgfqpoint{0.625000in}{3.520000in}}%
\pgfusepath{stroke}%
\end{pgfscope}%
\begin{pgfscope}%
\pgfsetrectcap%
\pgfsetmiterjoin%
\pgfsetlinewidth{0.803000pt}%
\definecolor{currentstroke}{rgb}{0.000000,0.000000,0.000000}%
\pgfsetstrokecolor{currentstroke}%
\pgfsetdash{}{0pt}%
\pgfpathmoveto{\pgfqpoint{4.500000in}{0.440000in}}%
\pgfpathlineto{\pgfqpoint{4.500000in}{3.520000in}}%
\pgfusepath{stroke}%
\end{pgfscope}%
\begin{pgfscope}%
\pgfsetrectcap%
\pgfsetmiterjoin%
\pgfsetlinewidth{0.803000pt}%
\definecolor{currentstroke}{rgb}{0.000000,0.000000,0.000000}%
\pgfsetstrokecolor{currentstroke}%
\pgfsetdash{}{0pt}%
\pgfpathmoveto{\pgfqpoint{0.625000in}{0.440000in}}%
\pgfpathlineto{\pgfqpoint{4.500000in}{0.440000in}}%
\pgfusepath{stroke}%
\end{pgfscope}%
\begin{pgfscope}%
\pgfsetrectcap%
\pgfsetmiterjoin%
\pgfsetlinewidth{0.803000pt}%
\definecolor{currentstroke}{rgb}{0.000000,0.000000,0.000000}%
\pgfsetstrokecolor{currentstroke}%
\pgfsetdash{}{0pt}%
\pgfpathmoveto{\pgfqpoint{0.625000in}{3.520000in}}%
\pgfpathlineto{\pgfqpoint{4.500000in}{3.520000in}}%
\pgfusepath{stroke}%
\end{pgfscope}%
\begin{pgfscope}%
\pgfsetbuttcap%
\pgfsetmiterjoin%
\definecolor{currentfill}{rgb}{1.000000,1.000000,1.000000}%
\pgfsetfillcolor{currentfill}%
\pgfsetfillopacity{0.800000}%
\pgfsetlinewidth{1.003750pt}%
\definecolor{currentstroke}{rgb}{0.800000,0.800000,0.800000}%
\pgfsetstrokecolor{currentstroke}%
\pgfsetstrokeopacity{0.800000}%
\pgfsetdash{}{0pt}%
\pgfpathmoveto{\pgfqpoint{3.042243in}{3.155267in}}%
\pgfpathlineto{\pgfqpoint{4.402778in}{3.155267in}}%
\pgfpathquadraticcurveto{\pgfqpoint{4.430556in}{3.155267in}}{\pgfqpoint{4.430556in}{3.183045in}}%
\pgfpathlineto{\pgfqpoint{4.430556in}{3.422778in}}%
\pgfpathquadraticcurveto{\pgfqpoint{4.430556in}{3.450556in}}{\pgfqpoint{4.402778in}{3.450556in}}%
\pgfpathlineto{\pgfqpoint{3.042243in}{3.450556in}}%
\pgfpathquadraticcurveto{\pgfqpoint{3.014465in}{3.450556in}}{\pgfqpoint{3.014465in}{3.422778in}}%
\pgfpathlineto{\pgfqpoint{3.014465in}{3.183045in}}%
\pgfpathquadraticcurveto{\pgfqpoint{3.014465in}{3.155267in}}{\pgfqpoint{3.042243in}{3.155267in}}%
\pgfpathclose%
\pgfusepath{stroke,fill}%
\end{pgfscope}%
\begin{pgfscope}%
\pgfsetbuttcap%
\pgfsetroundjoin%
\pgfsetlinewidth{1.505625pt}%
\definecolor{currentstroke}{rgb}{0.000000,0.000000,0.000000}%
\pgfsetstrokecolor{currentstroke}%
\pgfsetdash{{5.550000pt}{2.400000pt}}{0.000000pt}%
\pgfpathmoveto{\pgfqpoint{3.070021in}{3.321543in}}%
\pgfpathlineto{\pgfqpoint{3.347799in}{3.321543in}}%
\pgfusepath{stroke}%
\end{pgfscope}%
\begin{pgfscope}%
\definecolor{textcolor}{rgb}{0.000000,0.000000,0.000000}%
\pgfsetstrokecolor{textcolor}%
\pgfsetfillcolor{textcolor}%
\pgftext[x=3.458910in,y=3.272932in,left,base]{\color{textcolor}\sffamily\fontsize{10.000000}{12.000000}\selectfont \(\displaystyle 0.0002N_{\mathrm{QMC}}^{-0.5940}\)}%
\end{pgfscope}%
\end{pgfpicture}%
\makeatother%
\endgroup%

\caption[Quasi-Monte-Carlo error, for $Q=\int_D u$ and $k=60$.]{Quasi-Monte-Carlo error with increasing number of Quasi-Monte-Carlo points, for $Q=\int_D u$ and $k=60$. \label{fig:qmcintegral60}}
\end{figure}


\begin{figure}[h]
    \centering
    %% Creator: Matplotlib, PGF backend
%%
%% To include the figure in your LaTeX document, write
%%   \input{<filename>.pgf}
%%
%% Make sure the required packages are loaded in your preamble
%%   \usepackage{pgf}
%%
%% Figures using additional raster images can only be included by \input if
%% they are in the same directory as the main LaTeX file. For loading figures
%% from other directories you can use the `import` package
%%   \usepackage{import}
%% and then include the figures with
%%   \import{<path to file>}{<filename>.pgf}
%%
%% Matplotlib used the following preamble
%%   \usepackage{fontspec}
%%   \setmainfont{DejaVuSerif.ttf}[Path=/home/owen/progs/firedrake-complex/firedrake/lib/python3.5/site-packages/matplotlib/mpl-data/fonts/ttf/]
%%   \setsansfont{DejaVuSans.ttf}[Path=/home/owen/progs/firedrake-complex/firedrake/lib/python3.5/site-packages/matplotlib/mpl-data/fonts/ttf/]
%%   \setmonofont{DejaVuSansMono.ttf}[Path=/home/owen/progs/firedrake-complex/firedrake/lib/python3.5/site-packages/matplotlib/mpl-data/fonts/ttf/]
%%
\begingroup%
\makeatletter%
\begin{pgfpicture}%
\pgfpathrectangle{\pgfpointorigin}{\pgfqpoint{5.000000in}{4.000000in}}%
\pgfusepath{use as bounding box, clip}%
\begin{pgfscope}%
\pgfsetbuttcap%
\pgfsetmiterjoin%
\definecolor{currentfill}{rgb}{1.000000,1.000000,1.000000}%
\pgfsetfillcolor{currentfill}%
\pgfsetlinewidth{0.000000pt}%
\definecolor{currentstroke}{rgb}{1.000000,1.000000,1.000000}%
\pgfsetstrokecolor{currentstroke}%
\pgfsetdash{}{0pt}%
\pgfpathmoveto{\pgfqpoint{0.000000in}{0.000000in}}%
\pgfpathlineto{\pgfqpoint{5.000000in}{0.000000in}}%
\pgfpathlineto{\pgfqpoint{5.000000in}{4.000000in}}%
\pgfpathlineto{\pgfqpoint{0.000000in}{4.000000in}}%
\pgfpathclose%
\pgfusepath{fill}%
\end{pgfscope}%
\begin{pgfscope}%
\pgfsetbuttcap%
\pgfsetmiterjoin%
\definecolor{currentfill}{rgb}{1.000000,1.000000,1.000000}%
\pgfsetfillcolor{currentfill}%
\pgfsetlinewidth{0.000000pt}%
\definecolor{currentstroke}{rgb}{0.000000,0.000000,0.000000}%
\pgfsetstrokecolor{currentstroke}%
\pgfsetstrokeopacity{0.000000}%
\pgfsetdash{}{0pt}%
\pgfpathmoveto{\pgfqpoint{0.625000in}{0.440000in}}%
\pgfpathlineto{\pgfqpoint{4.500000in}{0.440000in}}%
\pgfpathlineto{\pgfqpoint{4.500000in}{3.520000in}}%
\pgfpathlineto{\pgfqpoint{0.625000in}{3.520000in}}%
\pgfpathclose%
\pgfusepath{fill}%
\end{pgfscope}%
\begin{pgfscope}%
\pgfsetbuttcap%
\pgfsetroundjoin%
\definecolor{currentfill}{rgb}{0.000000,0.000000,0.000000}%
\pgfsetfillcolor{currentfill}%
\pgfsetlinewidth{0.803000pt}%
\definecolor{currentstroke}{rgb}{0.000000,0.000000,0.000000}%
\pgfsetstrokecolor{currentstroke}%
\pgfsetdash{}{0pt}%
\pgfsys@defobject{currentmarker}{\pgfqpoint{0.000000in}{-0.048611in}}{\pgfqpoint{0.000000in}{0.000000in}}{%
\pgfpathmoveto{\pgfqpoint{0.000000in}{0.000000in}}%
\pgfpathlineto{\pgfqpoint{0.000000in}{-0.048611in}}%
\pgfusepath{stroke,fill}%
}%
\begin{pgfscope}%
\pgfsys@transformshift{0.801136in}{0.440000in}%
\pgfsys@useobject{currentmarker}{}%
\end{pgfscope}%
\end{pgfscope}%
\begin{pgfscope}%
\definecolor{textcolor}{rgb}{0.000000,0.000000,0.000000}%
\pgfsetstrokecolor{textcolor}%
\pgfsetfillcolor{textcolor}%
\pgftext[x=0.801136in,y=0.342778in,,top]{\color{textcolor}\sffamily\fontsize{11.000000}{13.200000}\selectfont \(\displaystyle 10^{0}\)}%
\end{pgfscope}%
\begin{pgfscope}%
\pgfsetbuttcap%
\pgfsetroundjoin%
\definecolor{currentfill}{rgb}{0.000000,0.000000,0.000000}%
\pgfsetfillcolor{currentfill}%
\pgfsetlinewidth{0.803000pt}%
\definecolor{currentstroke}{rgb}{0.000000,0.000000,0.000000}%
\pgfsetstrokecolor{currentstroke}%
\pgfsetdash{}{0pt}%
\pgfsys@defobject{currentmarker}{\pgfqpoint{0.000000in}{-0.048611in}}{\pgfqpoint{0.000000in}{0.000000in}}{%
\pgfpathmoveto{\pgfqpoint{0.000000in}{0.000000in}}%
\pgfpathlineto{\pgfqpoint{0.000000in}{-0.048611in}}%
\pgfusepath{stroke,fill}%
}%
\begin{pgfscope}%
\pgfsys@transformshift{1.864977in}{0.440000in}%
\pgfsys@useobject{currentmarker}{}%
\end{pgfscope}%
\end{pgfscope}%
\begin{pgfscope}%
\definecolor{textcolor}{rgb}{0.000000,0.000000,0.000000}%
\pgfsetstrokecolor{textcolor}%
\pgfsetfillcolor{textcolor}%
\pgftext[x=1.864977in,y=0.342778in,,top]{\color{textcolor}\sffamily\fontsize{11.000000}{13.200000}\selectfont \(\displaystyle 10^{1}\)}%
\end{pgfscope}%
\begin{pgfscope}%
\pgfsetbuttcap%
\pgfsetroundjoin%
\definecolor{currentfill}{rgb}{0.000000,0.000000,0.000000}%
\pgfsetfillcolor{currentfill}%
\pgfsetlinewidth{0.803000pt}%
\definecolor{currentstroke}{rgb}{0.000000,0.000000,0.000000}%
\pgfsetstrokecolor{currentstroke}%
\pgfsetdash{}{0pt}%
\pgfsys@defobject{currentmarker}{\pgfqpoint{0.000000in}{-0.048611in}}{\pgfqpoint{0.000000in}{0.000000in}}{%
\pgfpathmoveto{\pgfqpoint{0.000000in}{0.000000in}}%
\pgfpathlineto{\pgfqpoint{0.000000in}{-0.048611in}}%
\pgfusepath{stroke,fill}%
}%
\begin{pgfscope}%
\pgfsys@transformshift{2.928818in}{0.440000in}%
\pgfsys@useobject{currentmarker}{}%
\end{pgfscope}%
\end{pgfscope}%
\begin{pgfscope}%
\definecolor{textcolor}{rgb}{0.000000,0.000000,0.000000}%
\pgfsetstrokecolor{textcolor}%
\pgfsetfillcolor{textcolor}%
\pgftext[x=2.928818in,y=0.342778in,,top]{\color{textcolor}\sffamily\fontsize{11.000000}{13.200000}\selectfont \(\displaystyle 10^{2}\)}%
\end{pgfscope}%
\begin{pgfscope}%
\pgfsetbuttcap%
\pgfsetroundjoin%
\definecolor{currentfill}{rgb}{0.000000,0.000000,0.000000}%
\pgfsetfillcolor{currentfill}%
\pgfsetlinewidth{0.803000pt}%
\definecolor{currentstroke}{rgb}{0.000000,0.000000,0.000000}%
\pgfsetstrokecolor{currentstroke}%
\pgfsetdash{}{0pt}%
\pgfsys@defobject{currentmarker}{\pgfqpoint{0.000000in}{-0.048611in}}{\pgfqpoint{0.000000in}{0.000000in}}{%
\pgfpathmoveto{\pgfqpoint{0.000000in}{0.000000in}}%
\pgfpathlineto{\pgfqpoint{0.000000in}{-0.048611in}}%
\pgfusepath{stroke,fill}%
}%
\begin{pgfscope}%
\pgfsys@transformshift{3.992658in}{0.440000in}%
\pgfsys@useobject{currentmarker}{}%
\end{pgfscope}%
\end{pgfscope}%
\begin{pgfscope}%
\definecolor{textcolor}{rgb}{0.000000,0.000000,0.000000}%
\pgfsetstrokecolor{textcolor}%
\pgfsetfillcolor{textcolor}%
\pgftext[x=3.992658in,y=0.342778in,,top]{\color{textcolor}\sffamily\fontsize{11.000000}{13.200000}\selectfont \(\displaystyle 10^{3}\)}%
\end{pgfscope}%
\begin{pgfscope}%
\pgfsetbuttcap%
\pgfsetroundjoin%
\definecolor{currentfill}{rgb}{0.000000,0.000000,0.000000}%
\pgfsetfillcolor{currentfill}%
\pgfsetlinewidth{0.602250pt}%
\definecolor{currentstroke}{rgb}{0.000000,0.000000,0.000000}%
\pgfsetstrokecolor{currentstroke}%
\pgfsetdash{}{0pt}%
\pgfsys@defobject{currentmarker}{\pgfqpoint{0.000000in}{-0.027778in}}{\pgfqpoint{0.000000in}{0.000000in}}{%
\pgfpathmoveto{\pgfqpoint{0.000000in}{0.000000in}}%
\pgfpathlineto{\pgfqpoint{0.000000in}{-0.027778in}}%
\pgfusepath{stroke,fill}%
}%
\begin{pgfscope}%
\pgfsys@transformshift{0.636345in}{0.440000in}%
\pgfsys@useobject{currentmarker}{}%
\end{pgfscope}%
\end{pgfscope}%
\begin{pgfscope}%
\pgfsetbuttcap%
\pgfsetroundjoin%
\definecolor{currentfill}{rgb}{0.000000,0.000000,0.000000}%
\pgfsetfillcolor{currentfill}%
\pgfsetlinewidth{0.602250pt}%
\definecolor{currentstroke}{rgb}{0.000000,0.000000,0.000000}%
\pgfsetstrokecolor{currentstroke}%
\pgfsetdash{}{0pt}%
\pgfsys@defobject{currentmarker}{\pgfqpoint{0.000000in}{-0.027778in}}{\pgfqpoint{0.000000in}{0.000000in}}{%
\pgfpathmoveto{\pgfqpoint{0.000000in}{0.000000in}}%
\pgfpathlineto{\pgfqpoint{0.000000in}{-0.027778in}}%
\pgfusepath{stroke,fill}%
}%
\begin{pgfscope}%
\pgfsys@transformshift{0.698040in}{0.440000in}%
\pgfsys@useobject{currentmarker}{}%
\end{pgfscope}%
\end{pgfscope}%
\begin{pgfscope}%
\pgfsetbuttcap%
\pgfsetroundjoin%
\definecolor{currentfill}{rgb}{0.000000,0.000000,0.000000}%
\pgfsetfillcolor{currentfill}%
\pgfsetlinewidth{0.602250pt}%
\definecolor{currentstroke}{rgb}{0.000000,0.000000,0.000000}%
\pgfsetstrokecolor{currentstroke}%
\pgfsetdash{}{0pt}%
\pgfsys@defobject{currentmarker}{\pgfqpoint{0.000000in}{-0.027778in}}{\pgfqpoint{0.000000in}{0.000000in}}{%
\pgfpathmoveto{\pgfqpoint{0.000000in}{0.000000in}}%
\pgfpathlineto{\pgfqpoint{0.000000in}{-0.027778in}}%
\pgfusepath{stroke,fill}%
}%
\begin{pgfscope}%
\pgfsys@transformshift{0.752458in}{0.440000in}%
\pgfsys@useobject{currentmarker}{}%
\end{pgfscope}%
\end{pgfscope}%
\begin{pgfscope}%
\pgfsetbuttcap%
\pgfsetroundjoin%
\definecolor{currentfill}{rgb}{0.000000,0.000000,0.000000}%
\pgfsetfillcolor{currentfill}%
\pgfsetlinewidth{0.602250pt}%
\definecolor{currentstroke}{rgb}{0.000000,0.000000,0.000000}%
\pgfsetstrokecolor{currentstroke}%
\pgfsetdash{}{0pt}%
\pgfsys@defobject{currentmarker}{\pgfqpoint{0.000000in}{-0.027778in}}{\pgfqpoint{0.000000in}{0.000000in}}{%
\pgfpathmoveto{\pgfqpoint{0.000000in}{0.000000in}}%
\pgfpathlineto{\pgfqpoint{0.000000in}{-0.027778in}}%
\pgfusepath{stroke,fill}%
}%
\begin{pgfscope}%
\pgfsys@transformshift{1.121384in}{0.440000in}%
\pgfsys@useobject{currentmarker}{}%
\end{pgfscope}%
\end{pgfscope}%
\begin{pgfscope}%
\pgfsetbuttcap%
\pgfsetroundjoin%
\definecolor{currentfill}{rgb}{0.000000,0.000000,0.000000}%
\pgfsetfillcolor{currentfill}%
\pgfsetlinewidth{0.602250pt}%
\definecolor{currentstroke}{rgb}{0.000000,0.000000,0.000000}%
\pgfsetstrokecolor{currentstroke}%
\pgfsetdash{}{0pt}%
\pgfsys@defobject{currentmarker}{\pgfqpoint{0.000000in}{-0.027778in}}{\pgfqpoint{0.000000in}{0.000000in}}{%
\pgfpathmoveto{\pgfqpoint{0.000000in}{0.000000in}}%
\pgfpathlineto{\pgfqpoint{0.000000in}{-0.027778in}}%
\pgfusepath{stroke,fill}%
}%
\begin{pgfscope}%
\pgfsys@transformshift{1.308717in}{0.440000in}%
\pgfsys@useobject{currentmarker}{}%
\end{pgfscope}%
\end{pgfscope}%
\begin{pgfscope}%
\pgfsetbuttcap%
\pgfsetroundjoin%
\definecolor{currentfill}{rgb}{0.000000,0.000000,0.000000}%
\pgfsetfillcolor{currentfill}%
\pgfsetlinewidth{0.602250pt}%
\definecolor{currentstroke}{rgb}{0.000000,0.000000,0.000000}%
\pgfsetstrokecolor{currentstroke}%
\pgfsetdash{}{0pt}%
\pgfsys@defobject{currentmarker}{\pgfqpoint{0.000000in}{-0.027778in}}{\pgfqpoint{0.000000in}{0.000000in}}{%
\pgfpathmoveto{\pgfqpoint{0.000000in}{0.000000in}}%
\pgfpathlineto{\pgfqpoint{0.000000in}{-0.027778in}}%
\pgfusepath{stroke,fill}%
}%
\begin{pgfscope}%
\pgfsys@transformshift{1.441632in}{0.440000in}%
\pgfsys@useobject{currentmarker}{}%
\end{pgfscope}%
\end{pgfscope}%
\begin{pgfscope}%
\pgfsetbuttcap%
\pgfsetroundjoin%
\definecolor{currentfill}{rgb}{0.000000,0.000000,0.000000}%
\pgfsetfillcolor{currentfill}%
\pgfsetlinewidth{0.602250pt}%
\definecolor{currentstroke}{rgb}{0.000000,0.000000,0.000000}%
\pgfsetstrokecolor{currentstroke}%
\pgfsetdash{}{0pt}%
\pgfsys@defobject{currentmarker}{\pgfqpoint{0.000000in}{-0.027778in}}{\pgfqpoint{0.000000in}{0.000000in}}{%
\pgfpathmoveto{\pgfqpoint{0.000000in}{0.000000in}}%
\pgfpathlineto{\pgfqpoint{0.000000in}{-0.027778in}}%
\pgfusepath{stroke,fill}%
}%
\begin{pgfscope}%
\pgfsys@transformshift{1.544729in}{0.440000in}%
\pgfsys@useobject{currentmarker}{}%
\end{pgfscope}%
\end{pgfscope}%
\begin{pgfscope}%
\pgfsetbuttcap%
\pgfsetroundjoin%
\definecolor{currentfill}{rgb}{0.000000,0.000000,0.000000}%
\pgfsetfillcolor{currentfill}%
\pgfsetlinewidth{0.602250pt}%
\definecolor{currentstroke}{rgb}{0.000000,0.000000,0.000000}%
\pgfsetstrokecolor{currentstroke}%
\pgfsetdash{}{0pt}%
\pgfsys@defobject{currentmarker}{\pgfqpoint{0.000000in}{-0.027778in}}{\pgfqpoint{0.000000in}{0.000000in}}{%
\pgfpathmoveto{\pgfqpoint{0.000000in}{0.000000in}}%
\pgfpathlineto{\pgfqpoint{0.000000in}{-0.027778in}}%
\pgfusepath{stroke,fill}%
}%
\begin{pgfscope}%
\pgfsys@transformshift{1.628965in}{0.440000in}%
\pgfsys@useobject{currentmarker}{}%
\end{pgfscope}%
\end{pgfscope}%
\begin{pgfscope}%
\pgfsetbuttcap%
\pgfsetroundjoin%
\definecolor{currentfill}{rgb}{0.000000,0.000000,0.000000}%
\pgfsetfillcolor{currentfill}%
\pgfsetlinewidth{0.602250pt}%
\definecolor{currentstroke}{rgb}{0.000000,0.000000,0.000000}%
\pgfsetstrokecolor{currentstroke}%
\pgfsetdash{}{0pt}%
\pgfsys@defobject{currentmarker}{\pgfqpoint{0.000000in}{-0.027778in}}{\pgfqpoint{0.000000in}{0.000000in}}{%
\pgfpathmoveto{\pgfqpoint{0.000000in}{0.000000in}}%
\pgfpathlineto{\pgfqpoint{0.000000in}{-0.027778in}}%
\pgfusepath{stroke,fill}%
}%
\begin{pgfscope}%
\pgfsys@transformshift{1.700186in}{0.440000in}%
\pgfsys@useobject{currentmarker}{}%
\end{pgfscope}%
\end{pgfscope}%
\begin{pgfscope}%
\pgfsetbuttcap%
\pgfsetroundjoin%
\definecolor{currentfill}{rgb}{0.000000,0.000000,0.000000}%
\pgfsetfillcolor{currentfill}%
\pgfsetlinewidth{0.602250pt}%
\definecolor{currentstroke}{rgb}{0.000000,0.000000,0.000000}%
\pgfsetstrokecolor{currentstroke}%
\pgfsetdash{}{0pt}%
\pgfsys@defobject{currentmarker}{\pgfqpoint{0.000000in}{-0.027778in}}{\pgfqpoint{0.000000in}{0.000000in}}{%
\pgfpathmoveto{\pgfqpoint{0.000000in}{0.000000in}}%
\pgfpathlineto{\pgfqpoint{0.000000in}{-0.027778in}}%
\pgfusepath{stroke,fill}%
}%
\begin{pgfscope}%
\pgfsys@transformshift{1.761880in}{0.440000in}%
\pgfsys@useobject{currentmarker}{}%
\end{pgfscope}%
\end{pgfscope}%
\begin{pgfscope}%
\pgfsetbuttcap%
\pgfsetroundjoin%
\definecolor{currentfill}{rgb}{0.000000,0.000000,0.000000}%
\pgfsetfillcolor{currentfill}%
\pgfsetlinewidth{0.602250pt}%
\definecolor{currentstroke}{rgb}{0.000000,0.000000,0.000000}%
\pgfsetstrokecolor{currentstroke}%
\pgfsetdash{}{0pt}%
\pgfsys@defobject{currentmarker}{\pgfqpoint{0.000000in}{-0.027778in}}{\pgfqpoint{0.000000in}{0.000000in}}{%
\pgfpathmoveto{\pgfqpoint{0.000000in}{0.000000in}}%
\pgfpathlineto{\pgfqpoint{0.000000in}{-0.027778in}}%
\pgfusepath{stroke,fill}%
}%
\begin{pgfscope}%
\pgfsys@transformshift{1.816298in}{0.440000in}%
\pgfsys@useobject{currentmarker}{}%
\end{pgfscope}%
\end{pgfscope}%
\begin{pgfscope}%
\pgfsetbuttcap%
\pgfsetroundjoin%
\definecolor{currentfill}{rgb}{0.000000,0.000000,0.000000}%
\pgfsetfillcolor{currentfill}%
\pgfsetlinewidth{0.602250pt}%
\definecolor{currentstroke}{rgb}{0.000000,0.000000,0.000000}%
\pgfsetstrokecolor{currentstroke}%
\pgfsetdash{}{0pt}%
\pgfsys@defobject{currentmarker}{\pgfqpoint{0.000000in}{-0.027778in}}{\pgfqpoint{0.000000in}{0.000000in}}{%
\pgfpathmoveto{\pgfqpoint{0.000000in}{0.000000in}}%
\pgfpathlineto{\pgfqpoint{0.000000in}{-0.027778in}}%
\pgfusepath{stroke,fill}%
}%
\begin{pgfscope}%
\pgfsys@transformshift{2.185225in}{0.440000in}%
\pgfsys@useobject{currentmarker}{}%
\end{pgfscope}%
\end{pgfscope}%
\begin{pgfscope}%
\pgfsetbuttcap%
\pgfsetroundjoin%
\definecolor{currentfill}{rgb}{0.000000,0.000000,0.000000}%
\pgfsetfillcolor{currentfill}%
\pgfsetlinewidth{0.602250pt}%
\definecolor{currentstroke}{rgb}{0.000000,0.000000,0.000000}%
\pgfsetstrokecolor{currentstroke}%
\pgfsetdash{}{0pt}%
\pgfsys@defobject{currentmarker}{\pgfqpoint{0.000000in}{-0.027778in}}{\pgfqpoint{0.000000in}{0.000000in}}{%
\pgfpathmoveto{\pgfqpoint{0.000000in}{0.000000in}}%
\pgfpathlineto{\pgfqpoint{0.000000in}{-0.027778in}}%
\pgfusepath{stroke,fill}%
}%
\begin{pgfscope}%
\pgfsys@transformshift{2.372558in}{0.440000in}%
\pgfsys@useobject{currentmarker}{}%
\end{pgfscope}%
\end{pgfscope}%
\begin{pgfscope}%
\pgfsetbuttcap%
\pgfsetroundjoin%
\definecolor{currentfill}{rgb}{0.000000,0.000000,0.000000}%
\pgfsetfillcolor{currentfill}%
\pgfsetlinewidth{0.602250pt}%
\definecolor{currentstroke}{rgb}{0.000000,0.000000,0.000000}%
\pgfsetstrokecolor{currentstroke}%
\pgfsetdash{}{0pt}%
\pgfsys@defobject{currentmarker}{\pgfqpoint{0.000000in}{-0.027778in}}{\pgfqpoint{0.000000in}{0.000000in}}{%
\pgfpathmoveto{\pgfqpoint{0.000000in}{0.000000in}}%
\pgfpathlineto{\pgfqpoint{0.000000in}{-0.027778in}}%
\pgfusepath{stroke,fill}%
}%
\begin{pgfscope}%
\pgfsys@transformshift{2.505473in}{0.440000in}%
\pgfsys@useobject{currentmarker}{}%
\end{pgfscope}%
\end{pgfscope}%
\begin{pgfscope}%
\pgfsetbuttcap%
\pgfsetroundjoin%
\definecolor{currentfill}{rgb}{0.000000,0.000000,0.000000}%
\pgfsetfillcolor{currentfill}%
\pgfsetlinewidth{0.602250pt}%
\definecolor{currentstroke}{rgb}{0.000000,0.000000,0.000000}%
\pgfsetstrokecolor{currentstroke}%
\pgfsetdash{}{0pt}%
\pgfsys@defobject{currentmarker}{\pgfqpoint{0.000000in}{-0.027778in}}{\pgfqpoint{0.000000in}{0.000000in}}{%
\pgfpathmoveto{\pgfqpoint{0.000000in}{0.000000in}}%
\pgfpathlineto{\pgfqpoint{0.000000in}{-0.027778in}}%
\pgfusepath{stroke,fill}%
}%
\begin{pgfscope}%
\pgfsys@transformshift{2.608570in}{0.440000in}%
\pgfsys@useobject{currentmarker}{}%
\end{pgfscope}%
\end{pgfscope}%
\begin{pgfscope}%
\pgfsetbuttcap%
\pgfsetroundjoin%
\definecolor{currentfill}{rgb}{0.000000,0.000000,0.000000}%
\pgfsetfillcolor{currentfill}%
\pgfsetlinewidth{0.602250pt}%
\definecolor{currentstroke}{rgb}{0.000000,0.000000,0.000000}%
\pgfsetstrokecolor{currentstroke}%
\pgfsetdash{}{0pt}%
\pgfsys@defobject{currentmarker}{\pgfqpoint{0.000000in}{-0.027778in}}{\pgfqpoint{0.000000in}{0.000000in}}{%
\pgfpathmoveto{\pgfqpoint{0.000000in}{0.000000in}}%
\pgfpathlineto{\pgfqpoint{0.000000in}{-0.027778in}}%
\pgfusepath{stroke,fill}%
}%
\begin{pgfscope}%
\pgfsys@transformshift{2.692806in}{0.440000in}%
\pgfsys@useobject{currentmarker}{}%
\end{pgfscope}%
\end{pgfscope}%
\begin{pgfscope}%
\pgfsetbuttcap%
\pgfsetroundjoin%
\definecolor{currentfill}{rgb}{0.000000,0.000000,0.000000}%
\pgfsetfillcolor{currentfill}%
\pgfsetlinewidth{0.602250pt}%
\definecolor{currentstroke}{rgb}{0.000000,0.000000,0.000000}%
\pgfsetstrokecolor{currentstroke}%
\pgfsetdash{}{0pt}%
\pgfsys@defobject{currentmarker}{\pgfqpoint{0.000000in}{-0.027778in}}{\pgfqpoint{0.000000in}{0.000000in}}{%
\pgfpathmoveto{\pgfqpoint{0.000000in}{0.000000in}}%
\pgfpathlineto{\pgfqpoint{0.000000in}{-0.027778in}}%
\pgfusepath{stroke,fill}%
}%
\begin{pgfscope}%
\pgfsys@transformshift{2.764027in}{0.440000in}%
\pgfsys@useobject{currentmarker}{}%
\end{pgfscope}%
\end{pgfscope}%
\begin{pgfscope}%
\pgfsetbuttcap%
\pgfsetroundjoin%
\definecolor{currentfill}{rgb}{0.000000,0.000000,0.000000}%
\pgfsetfillcolor{currentfill}%
\pgfsetlinewidth{0.602250pt}%
\definecolor{currentstroke}{rgb}{0.000000,0.000000,0.000000}%
\pgfsetstrokecolor{currentstroke}%
\pgfsetdash{}{0pt}%
\pgfsys@defobject{currentmarker}{\pgfqpoint{0.000000in}{-0.027778in}}{\pgfqpoint{0.000000in}{0.000000in}}{%
\pgfpathmoveto{\pgfqpoint{0.000000in}{0.000000in}}%
\pgfpathlineto{\pgfqpoint{0.000000in}{-0.027778in}}%
\pgfusepath{stroke,fill}%
}%
\begin{pgfscope}%
\pgfsys@transformshift{2.825721in}{0.440000in}%
\pgfsys@useobject{currentmarker}{}%
\end{pgfscope}%
\end{pgfscope}%
\begin{pgfscope}%
\pgfsetbuttcap%
\pgfsetroundjoin%
\definecolor{currentfill}{rgb}{0.000000,0.000000,0.000000}%
\pgfsetfillcolor{currentfill}%
\pgfsetlinewidth{0.602250pt}%
\definecolor{currentstroke}{rgb}{0.000000,0.000000,0.000000}%
\pgfsetstrokecolor{currentstroke}%
\pgfsetdash{}{0pt}%
\pgfsys@defobject{currentmarker}{\pgfqpoint{0.000000in}{-0.027778in}}{\pgfqpoint{0.000000in}{0.000000in}}{%
\pgfpathmoveto{\pgfqpoint{0.000000in}{0.000000in}}%
\pgfpathlineto{\pgfqpoint{0.000000in}{-0.027778in}}%
\pgfusepath{stroke,fill}%
}%
\begin{pgfscope}%
\pgfsys@transformshift{2.880139in}{0.440000in}%
\pgfsys@useobject{currentmarker}{}%
\end{pgfscope}%
\end{pgfscope}%
\begin{pgfscope}%
\pgfsetbuttcap%
\pgfsetroundjoin%
\definecolor{currentfill}{rgb}{0.000000,0.000000,0.000000}%
\pgfsetfillcolor{currentfill}%
\pgfsetlinewidth{0.602250pt}%
\definecolor{currentstroke}{rgb}{0.000000,0.000000,0.000000}%
\pgfsetstrokecolor{currentstroke}%
\pgfsetdash{}{0pt}%
\pgfsys@defobject{currentmarker}{\pgfqpoint{0.000000in}{-0.027778in}}{\pgfqpoint{0.000000in}{0.000000in}}{%
\pgfpathmoveto{\pgfqpoint{0.000000in}{0.000000in}}%
\pgfpathlineto{\pgfqpoint{0.000000in}{-0.027778in}}%
\pgfusepath{stroke,fill}%
}%
\begin{pgfscope}%
\pgfsys@transformshift{3.249066in}{0.440000in}%
\pgfsys@useobject{currentmarker}{}%
\end{pgfscope}%
\end{pgfscope}%
\begin{pgfscope}%
\pgfsetbuttcap%
\pgfsetroundjoin%
\definecolor{currentfill}{rgb}{0.000000,0.000000,0.000000}%
\pgfsetfillcolor{currentfill}%
\pgfsetlinewidth{0.602250pt}%
\definecolor{currentstroke}{rgb}{0.000000,0.000000,0.000000}%
\pgfsetstrokecolor{currentstroke}%
\pgfsetdash{}{0pt}%
\pgfsys@defobject{currentmarker}{\pgfqpoint{0.000000in}{-0.027778in}}{\pgfqpoint{0.000000in}{0.000000in}}{%
\pgfpathmoveto{\pgfqpoint{0.000000in}{0.000000in}}%
\pgfpathlineto{\pgfqpoint{0.000000in}{-0.027778in}}%
\pgfusepath{stroke,fill}%
}%
\begin{pgfscope}%
\pgfsys@transformshift{3.436399in}{0.440000in}%
\pgfsys@useobject{currentmarker}{}%
\end{pgfscope}%
\end{pgfscope}%
\begin{pgfscope}%
\pgfsetbuttcap%
\pgfsetroundjoin%
\definecolor{currentfill}{rgb}{0.000000,0.000000,0.000000}%
\pgfsetfillcolor{currentfill}%
\pgfsetlinewidth{0.602250pt}%
\definecolor{currentstroke}{rgb}{0.000000,0.000000,0.000000}%
\pgfsetstrokecolor{currentstroke}%
\pgfsetdash{}{0pt}%
\pgfsys@defobject{currentmarker}{\pgfqpoint{0.000000in}{-0.027778in}}{\pgfqpoint{0.000000in}{0.000000in}}{%
\pgfpathmoveto{\pgfqpoint{0.000000in}{0.000000in}}%
\pgfpathlineto{\pgfqpoint{0.000000in}{-0.027778in}}%
\pgfusepath{stroke,fill}%
}%
\begin{pgfscope}%
\pgfsys@transformshift{3.569313in}{0.440000in}%
\pgfsys@useobject{currentmarker}{}%
\end{pgfscope}%
\end{pgfscope}%
\begin{pgfscope}%
\pgfsetbuttcap%
\pgfsetroundjoin%
\definecolor{currentfill}{rgb}{0.000000,0.000000,0.000000}%
\pgfsetfillcolor{currentfill}%
\pgfsetlinewidth{0.602250pt}%
\definecolor{currentstroke}{rgb}{0.000000,0.000000,0.000000}%
\pgfsetstrokecolor{currentstroke}%
\pgfsetdash{}{0pt}%
\pgfsys@defobject{currentmarker}{\pgfqpoint{0.000000in}{-0.027778in}}{\pgfqpoint{0.000000in}{0.000000in}}{%
\pgfpathmoveto{\pgfqpoint{0.000000in}{0.000000in}}%
\pgfpathlineto{\pgfqpoint{0.000000in}{-0.027778in}}%
\pgfusepath{stroke,fill}%
}%
\begin{pgfscope}%
\pgfsys@transformshift{3.672410in}{0.440000in}%
\pgfsys@useobject{currentmarker}{}%
\end{pgfscope}%
\end{pgfscope}%
\begin{pgfscope}%
\pgfsetbuttcap%
\pgfsetroundjoin%
\definecolor{currentfill}{rgb}{0.000000,0.000000,0.000000}%
\pgfsetfillcolor{currentfill}%
\pgfsetlinewidth{0.602250pt}%
\definecolor{currentstroke}{rgb}{0.000000,0.000000,0.000000}%
\pgfsetstrokecolor{currentstroke}%
\pgfsetdash{}{0pt}%
\pgfsys@defobject{currentmarker}{\pgfqpoint{0.000000in}{-0.027778in}}{\pgfqpoint{0.000000in}{0.000000in}}{%
\pgfpathmoveto{\pgfqpoint{0.000000in}{0.000000in}}%
\pgfpathlineto{\pgfqpoint{0.000000in}{-0.027778in}}%
\pgfusepath{stroke,fill}%
}%
\begin{pgfscope}%
\pgfsys@transformshift{3.756646in}{0.440000in}%
\pgfsys@useobject{currentmarker}{}%
\end{pgfscope}%
\end{pgfscope}%
\begin{pgfscope}%
\pgfsetbuttcap%
\pgfsetroundjoin%
\definecolor{currentfill}{rgb}{0.000000,0.000000,0.000000}%
\pgfsetfillcolor{currentfill}%
\pgfsetlinewidth{0.602250pt}%
\definecolor{currentstroke}{rgb}{0.000000,0.000000,0.000000}%
\pgfsetstrokecolor{currentstroke}%
\pgfsetdash{}{0pt}%
\pgfsys@defobject{currentmarker}{\pgfqpoint{0.000000in}{-0.027778in}}{\pgfqpoint{0.000000in}{0.000000in}}{%
\pgfpathmoveto{\pgfqpoint{0.000000in}{0.000000in}}%
\pgfpathlineto{\pgfqpoint{0.000000in}{-0.027778in}}%
\pgfusepath{stroke,fill}%
}%
\begin{pgfscope}%
\pgfsys@transformshift{3.827867in}{0.440000in}%
\pgfsys@useobject{currentmarker}{}%
\end{pgfscope}%
\end{pgfscope}%
\begin{pgfscope}%
\pgfsetbuttcap%
\pgfsetroundjoin%
\definecolor{currentfill}{rgb}{0.000000,0.000000,0.000000}%
\pgfsetfillcolor{currentfill}%
\pgfsetlinewidth{0.602250pt}%
\definecolor{currentstroke}{rgb}{0.000000,0.000000,0.000000}%
\pgfsetstrokecolor{currentstroke}%
\pgfsetdash{}{0pt}%
\pgfsys@defobject{currentmarker}{\pgfqpoint{0.000000in}{-0.027778in}}{\pgfqpoint{0.000000in}{0.000000in}}{%
\pgfpathmoveto{\pgfqpoint{0.000000in}{0.000000in}}%
\pgfpathlineto{\pgfqpoint{0.000000in}{-0.027778in}}%
\pgfusepath{stroke,fill}%
}%
\begin{pgfscope}%
\pgfsys@transformshift{3.889561in}{0.440000in}%
\pgfsys@useobject{currentmarker}{}%
\end{pgfscope}%
\end{pgfscope}%
\begin{pgfscope}%
\pgfsetbuttcap%
\pgfsetroundjoin%
\definecolor{currentfill}{rgb}{0.000000,0.000000,0.000000}%
\pgfsetfillcolor{currentfill}%
\pgfsetlinewidth{0.602250pt}%
\definecolor{currentstroke}{rgb}{0.000000,0.000000,0.000000}%
\pgfsetstrokecolor{currentstroke}%
\pgfsetdash{}{0pt}%
\pgfsys@defobject{currentmarker}{\pgfqpoint{0.000000in}{-0.027778in}}{\pgfqpoint{0.000000in}{0.000000in}}{%
\pgfpathmoveto{\pgfqpoint{0.000000in}{0.000000in}}%
\pgfpathlineto{\pgfqpoint{0.000000in}{-0.027778in}}%
\pgfusepath{stroke,fill}%
}%
\begin{pgfscope}%
\pgfsys@transformshift{3.943980in}{0.440000in}%
\pgfsys@useobject{currentmarker}{}%
\end{pgfscope}%
\end{pgfscope}%
\begin{pgfscope}%
\pgfsetbuttcap%
\pgfsetroundjoin%
\definecolor{currentfill}{rgb}{0.000000,0.000000,0.000000}%
\pgfsetfillcolor{currentfill}%
\pgfsetlinewidth{0.602250pt}%
\definecolor{currentstroke}{rgb}{0.000000,0.000000,0.000000}%
\pgfsetstrokecolor{currentstroke}%
\pgfsetdash{}{0pt}%
\pgfsys@defobject{currentmarker}{\pgfqpoint{0.000000in}{-0.027778in}}{\pgfqpoint{0.000000in}{0.000000in}}{%
\pgfpathmoveto{\pgfqpoint{0.000000in}{0.000000in}}%
\pgfpathlineto{\pgfqpoint{0.000000in}{-0.027778in}}%
\pgfusepath{stroke,fill}%
}%
\begin{pgfscope}%
\pgfsys@transformshift{4.312906in}{0.440000in}%
\pgfsys@useobject{currentmarker}{}%
\end{pgfscope}%
\end{pgfscope}%
\begin{pgfscope}%
\pgfsetbuttcap%
\pgfsetroundjoin%
\definecolor{currentfill}{rgb}{0.000000,0.000000,0.000000}%
\pgfsetfillcolor{currentfill}%
\pgfsetlinewidth{0.602250pt}%
\definecolor{currentstroke}{rgb}{0.000000,0.000000,0.000000}%
\pgfsetstrokecolor{currentstroke}%
\pgfsetdash{}{0pt}%
\pgfsys@defobject{currentmarker}{\pgfqpoint{0.000000in}{-0.027778in}}{\pgfqpoint{0.000000in}{0.000000in}}{%
\pgfpathmoveto{\pgfqpoint{0.000000in}{0.000000in}}%
\pgfpathlineto{\pgfqpoint{0.000000in}{-0.027778in}}%
\pgfusepath{stroke,fill}%
}%
\begin{pgfscope}%
\pgfsys@transformshift{4.500239in}{0.440000in}%
\pgfsys@useobject{currentmarker}{}%
\end{pgfscope}%
\end{pgfscope}%
\begin{pgfscope}%
\definecolor{textcolor}{rgb}{0.000000,0.000000,0.000000}%
\pgfsetstrokecolor{textcolor}%
\pgfsetfillcolor{textcolor}%
\pgftext[x=2.562500in,y=0.139368in,,top]{\color{textcolor}\sffamily\fontsize{11.000000}{13.200000}\selectfont \(\displaystyle N_\mathrm{QMC}\)}%
\end{pgfscope}%
\begin{pgfscope}%
\pgfsetbuttcap%
\pgfsetroundjoin%
\definecolor{currentfill}{rgb}{0.000000,0.000000,0.000000}%
\pgfsetfillcolor{currentfill}%
\pgfsetlinewidth{0.803000pt}%
\definecolor{currentstroke}{rgb}{0.000000,0.000000,0.000000}%
\pgfsetstrokecolor{currentstroke}%
\pgfsetdash{}{0pt}%
\pgfsys@defobject{currentmarker}{\pgfqpoint{-0.048611in}{0.000000in}}{\pgfqpoint{0.000000in}{0.000000in}}{%
\pgfpathmoveto{\pgfqpoint{0.000000in}{0.000000in}}%
\pgfpathlineto{\pgfqpoint{-0.048611in}{0.000000in}}%
\pgfusepath{stroke,fill}%
}%
\begin{pgfscope}%
\pgfsys@transformshift{0.625000in}{0.845103in}%
\pgfsys@useobject{currentmarker}{}%
\end{pgfscope}%
\end{pgfscope}%
\begin{pgfscope}%
\definecolor{textcolor}{rgb}{0.000000,0.000000,0.000000}%
\pgfsetstrokecolor{textcolor}%
\pgfsetfillcolor{textcolor}%
\pgftext[x=0.217899in,y=0.787065in,left,base]{\color{textcolor}\sffamily\fontsize{11.000000}{13.200000}\selectfont \(\displaystyle 10^{-5}\)}%
\end{pgfscope}%
\begin{pgfscope}%
\pgfsetbuttcap%
\pgfsetroundjoin%
\definecolor{currentfill}{rgb}{0.000000,0.000000,0.000000}%
\pgfsetfillcolor{currentfill}%
\pgfsetlinewidth{0.803000pt}%
\definecolor{currentstroke}{rgb}{0.000000,0.000000,0.000000}%
\pgfsetstrokecolor{currentstroke}%
\pgfsetdash{}{0pt}%
\pgfsys@defobject{currentmarker}{\pgfqpoint{-0.048611in}{0.000000in}}{\pgfqpoint{0.000000in}{0.000000in}}{%
\pgfpathmoveto{\pgfqpoint{0.000000in}{0.000000in}}%
\pgfpathlineto{\pgfqpoint{-0.048611in}{0.000000in}}%
\pgfusepath{stroke,fill}%
}%
\begin{pgfscope}%
\pgfsys@transformshift{0.625000in}{1.707028in}%
\pgfsys@useobject{currentmarker}{}%
\end{pgfscope}%
\end{pgfscope}%
\begin{pgfscope}%
\definecolor{textcolor}{rgb}{0.000000,0.000000,0.000000}%
\pgfsetstrokecolor{textcolor}%
\pgfsetfillcolor{textcolor}%
\pgftext[x=0.217899in,y=1.648990in,left,base]{\color{textcolor}\sffamily\fontsize{11.000000}{13.200000}\selectfont \(\displaystyle 10^{-4}\)}%
\end{pgfscope}%
\begin{pgfscope}%
\pgfsetbuttcap%
\pgfsetroundjoin%
\definecolor{currentfill}{rgb}{0.000000,0.000000,0.000000}%
\pgfsetfillcolor{currentfill}%
\pgfsetlinewidth{0.803000pt}%
\definecolor{currentstroke}{rgb}{0.000000,0.000000,0.000000}%
\pgfsetstrokecolor{currentstroke}%
\pgfsetdash{}{0pt}%
\pgfsys@defobject{currentmarker}{\pgfqpoint{-0.048611in}{0.000000in}}{\pgfqpoint{0.000000in}{0.000000in}}{%
\pgfpathmoveto{\pgfqpoint{0.000000in}{0.000000in}}%
\pgfpathlineto{\pgfqpoint{-0.048611in}{0.000000in}}%
\pgfusepath{stroke,fill}%
}%
\begin{pgfscope}%
\pgfsys@transformshift{0.625000in}{2.568953in}%
\pgfsys@useobject{currentmarker}{}%
\end{pgfscope}%
\end{pgfscope}%
\begin{pgfscope}%
\definecolor{textcolor}{rgb}{0.000000,0.000000,0.000000}%
\pgfsetstrokecolor{textcolor}%
\pgfsetfillcolor{textcolor}%
\pgftext[x=0.217899in,y=2.510916in,left,base]{\color{textcolor}\sffamily\fontsize{11.000000}{13.200000}\selectfont \(\displaystyle 10^{-3}\)}%
\end{pgfscope}%
\begin{pgfscope}%
\pgfsetbuttcap%
\pgfsetroundjoin%
\definecolor{currentfill}{rgb}{0.000000,0.000000,0.000000}%
\pgfsetfillcolor{currentfill}%
\pgfsetlinewidth{0.803000pt}%
\definecolor{currentstroke}{rgb}{0.000000,0.000000,0.000000}%
\pgfsetstrokecolor{currentstroke}%
\pgfsetdash{}{0pt}%
\pgfsys@defobject{currentmarker}{\pgfqpoint{-0.048611in}{0.000000in}}{\pgfqpoint{0.000000in}{0.000000in}}{%
\pgfpathmoveto{\pgfqpoint{0.000000in}{0.000000in}}%
\pgfpathlineto{\pgfqpoint{-0.048611in}{0.000000in}}%
\pgfusepath{stroke,fill}%
}%
\begin{pgfscope}%
\pgfsys@transformshift{0.625000in}{3.430879in}%
\pgfsys@useobject{currentmarker}{}%
\end{pgfscope}%
\end{pgfscope}%
\begin{pgfscope}%
\definecolor{textcolor}{rgb}{0.000000,0.000000,0.000000}%
\pgfsetstrokecolor{textcolor}%
\pgfsetfillcolor{textcolor}%
\pgftext[x=0.217899in,y=3.372841in,left,base]{\color{textcolor}\sffamily\fontsize{11.000000}{13.200000}\selectfont \(\displaystyle 10^{-2}\)}%
\end{pgfscope}%
\begin{pgfscope}%
\pgfsetbuttcap%
\pgfsetroundjoin%
\definecolor{currentfill}{rgb}{0.000000,0.000000,0.000000}%
\pgfsetfillcolor{currentfill}%
\pgfsetlinewidth{0.602250pt}%
\definecolor{currentstroke}{rgb}{0.000000,0.000000,0.000000}%
\pgfsetstrokecolor{currentstroke}%
\pgfsetdash{}{0pt}%
\pgfsys@defobject{currentmarker}{\pgfqpoint{-0.027778in}{0.000000in}}{\pgfqpoint{0.000000in}{0.000000in}}{%
\pgfpathmoveto{\pgfqpoint{0.000000in}{0.000000in}}%
\pgfpathlineto{\pgfqpoint{-0.027778in}{0.000000in}}%
\pgfusepath{stroke,fill}%
}%
\begin{pgfscope}%
\pgfsys@transformshift{0.625000in}{0.502108in}%
\pgfsys@useobject{currentmarker}{}%
\end{pgfscope}%
\end{pgfscope}%
\begin{pgfscope}%
\pgfsetbuttcap%
\pgfsetroundjoin%
\definecolor{currentfill}{rgb}{0.000000,0.000000,0.000000}%
\pgfsetfillcolor{currentfill}%
\pgfsetlinewidth{0.602250pt}%
\definecolor{currentstroke}{rgb}{0.000000,0.000000,0.000000}%
\pgfsetstrokecolor{currentstroke}%
\pgfsetdash{}{0pt}%
\pgfsys@defobject{currentmarker}{\pgfqpoint{-0.027778in}{0.000000in}}{\pgfqpoint{0.000000in}{0.000000in}}{%
\pgfpathmoveto{\pgfqpoint{0.000000in}{0.000000in}}%
\pgfpathlineto{\pgfqpoint{-0.027778in}{0.000000in}}%
\pgfusepath{stroke,fill}%
}%
\begin{pgfscope}%
\pgfsys@transformshift{0.625000in}{0.585637in}%
\pgfsys@useobject{currentmarker}{}%
\end{pgfscope}%
\end{pgfscope}%
\begin{pgfscope}%
\pgfsetbuttcap%
\pgfsetroundjoin%
\definecolor{currentfill}{rgb}{0.000000,0.000000,0.000000}%
\pgfsetfillcolor{currentfill}%
\pgfsetlinewidth{0.602250pt}%
\definecolor{currentstroke}{rgb}{0.000000,0.000000,0.000000}%
\pgfsetstrokecolor{currentstroke}%
\pgfsetdash{}{0pt}%
\pgfsys@defobject{currentmarker}{\pgfqpoint{-0.027778in}{0.000000in}}{\pgfqpoint{0.000000in}{0.000000in}}{%
\pgfpathmoveto{\pgfqpoint{0.000000in}{0.000000in}}%
\pgfpathlineto{\pgfqpoint{-0.027778in}{0.000000in}}%
\pgfusepath{stroke,fill}%
}%
\begin{pgfscope}%
\pgfsys@transformshift{0.625000in}{0.653886in}%
\pgfsys@useobject{currentmarker}{}%
\end{pgfscope}%
\end{pgfscope}%
\begin{pgfscope}%
\pgfsetbuttcap%
\pgfsetroundjoin%
\definecolor{currentfill}{rgb}{0.000000,0.000000,0.000000}%
\pgfsetfillcolor{currentfill}%
\pgfsetlinewidth{0.602250pt}%
\definecolor{currentstroke}{rgb}{0.000000,0.000000,0.000000}%
\pgfsetstrokecolor{currentstroke}%
\pgfsetdash{}{0pt}%
\pgfsys@defobject{currentmarker}{\pgfqpoint{-0.027778in}{0.000000in}}{\pgfqpoint{0.000000in}{0.000000in}}{%
\pgfpathmoveto{\pgfqpoint{0.000000in}{0.000000in}}%
\pgfpathlineto{\pgfqpoint{-0.027778in}{0.000000in}}%
\pgfusepath{stroke,fill}%
}%
\begin{pgfscope}%
\pgfsys@transformshift{0.625000in}{0.711589in}%
\pgfsys@useobject{currentmarker}{}%
\end{pgfscope}%
\end{pgfscope}%
\begin{pgfscope}%
\pgfsetbuttcap%
\pgfsetroundjoin%
\definecolor{currentfill}{rgb}{0.000000,0.000000,0.000000}%
\pgfsetfillcolor{currentfill}%
\pgfsetlinewidth{0.602250pt}%
\definecolor{currentstroke}{rgb}{0.000000,0.000000,0.000000}%
\pgfsetstrokecolor{currentstroke}%
\pgfsetdash{}{0pt}%
\pgfsys@defobject{currentmarker}{\pgfqpoint{-0.027778in}{0.000000in}}{\pgfqpoint{0.000000in}{0.000000in}}{%
\pgfpathmoveto{\pgfqpoint{0.000000in}{0.000000in}}%
\pgfpathlineto{\pgfqpoint{-0.027778in}{0.000000in}}%
\pgfusepath{stroke,fill}%
}%
\begin{pgfscope}%
\pgfsys@transformshift{0.625000in}{0.761574in}%
\pgfsys@useobject{currentmarker}{}%
\end{pgfscope}%
\end{pgfscope}%
\begin{pgfscope}%
\pgfsetbuttcap%
\pgfsetroundjoin%
\definecolor{currentfill}{rgb}{0.000000,0.000000,0.000000}%
\pgfsetfillcolor{currentfill}%
\pgfsetlinewidth{0.602250pt}%
\definecolor{currentstroke}{rgb}{0.000000,0.000000,0.000000}%
\pgfsetstrokecolor{currentstroke}%
\pgfsetdash{}{0pt}%
\pgfsys@defobject{currentmarker}{\pgfqpoint{-0.027778in}{0.000000in}}{\pgfqpoint{0.000000in}{0.000000in}}{%
\pgfpathmoveto{\pgfqpoint{0.000000in}{0.000000in}}%
\pgfpathlineto{\pgfqpoint{-0.027778in}{0.000000in}}%
\pgfusepath{stroke,fill}%
}%
\begin{pgfscope}%
\pgfsys@transformshift{0.625000in}{0.805663in}%
\pgfsys@useobject{currentmarker}{}%
\end{pgfscope}%
\end{pgfscope}%
\begin{pgfscope}%
\pgfsetbuttcap%
\pgfsetroundjoin%
\definecolor{currentfill}{rgb}{0.000000,0.000000,0.000000}%
\pgfsetfillcolor{currentfill}%
\pgfsetlinewidth{0.602250pt}%
\definecolor{currentstroke}{rgb}{0.000000,0.000000,0.000000}%
\pgfsetstrokecolor{currentstroke}%
\pgfsetdash{}{0pt}%
\pgfsys@defobject{currentmarker}{\pgfqpoint{-0.027778in}{0.000000in}}{\pgfqpoint{0.000000in}{0.000000in}}{%
\pgfpathmoveto{\pgfqpoint{0.000000in}{0.000000in}}%
\pgfpathlineto{\pgfqpoint{-0.027778in}{0.000000in}}%
\pgfusepath{stroke,fill}%
}%
\begin{pgfscope}%
\pgfsys@transformshift{0.625000in}{1.104568in}%
\pgfsys@useobject{currentmarker}{}%
\end{pgfscope}%
\end{pgfscope}%
\begin{pgfscope}%
\pgfsetbuttcap%
\pgfsetroundjoin%
\definecolor{currentfill}{rgb}{0.000000,0.000000,0.000000}%
\pgfsetfillcolor{currentfill}%
\pgfsetlinewidth{0.602250pt}%
\definecolor{currentstroke}{rgb}{0.000000,0.000000,0.000000}%
\pgfsetstrokecolor{currentstroke}%
\pgfsetdash{}{0pt}%
\pgfsys@defobject{currentmarker}{\pgfqpoint{-0.027778in}{0.000000in}}{\pgfqpoint{0.000000in}{0.000000in}}{%
\pgfpathmoveto{\pgfqpoint{0.000000in}{0.000000in}}%
\pgfpathlineto{\pgfqpoint{-0.027778in}{0.000000in}}%
\pgfusepath{stroke,fill}%
}%
\begin{pgfscope}%
\pgfsys@transformshift{0.625000in}{1.256346in}%
\pgfsys@useobject{currentmarker}{}%
\end{pgfscope}%
\end{pgfscope}%
\begin{pgfscope}%
\pgfsetbuttcap%
\pgfsetroundjoin%
\definecolor{currentfill}{rgb}{0.000000,0.000000,0.000000}%
\pgfsetfillcolor{currentfill}%
\pgfsetlinewidth{0.602250pt}%
\definecolor{currentstroke}{rgb}{0.000000,0.000000,0.000000}%
\pgfsetstrokecolor{currentstroke}%
\pgfsetdash{}{0pt}%
\pgfsys@defobject{currentmarker}{\pgfqpoint{-0.027778in}{0.000000in}}{\pgfqpoint{0.000000in}{0.000000in}}{%
\pgfpathmoveto{\pgfqpoint{0.000000in}{0.000000in}}%
\pgfpathlineto{\pgfqpoint{-0.027778in}{0.000000in}}%
\pgfusepath{stroke,fill}%
}%
\begin{pgfscope}%
\pgfsys@transformshift{0.625000in}{1.364033in}%
\pgfsys@useobject{currentmarker}{}%
\end{pgfscope}%
\end{pgfscope}%
\begin{pgfscope}%
\pgfsetbuttcap%
\pgfsetroundjoin%
\definecolor{currentfill}{rgb}{0.000000,0.000000,0.000000}%
\pgfsetfillcolor{currentfill}%
\pgfsetlinewidth{0.602250pt}%
\definecolor{currentstroke}{rgb}{0.000000,0.000000,0.000000}%
\pgfsetstrokecolor{currentstroke}%
\pgfsetdash{}{0pt}%
\pgfsys@defobject{currentmarker}{\pgfqpoint{-0.027778in}{0.000000in}}{\pgfqpoint{0.000000in}{0.000000in}}{%
\pgfpathmoveto{\pgfqpoint{0.000000in}{0.000000in}}%
\pgfpathlineto{\pgfqpoint{-0.027778in}{0.000000in}}%
\pgfusepath{stroke,fill}%
}%
\begin{pgfscope}%
\pgfsys@transformshift{0.625000in}{1.447563in}%
\pgfsys@useobject{currentmarker}{}%
\end{pgfscope}%
\end{pgfscope}%
\begin{pgfscope}%
\pgfsetbuttcap%
\pgfsetroundjoin%
\definecolor{currentfill}{rgb}{0.000000,0.000000,0.000000}%
\pgfsetfillcolor{currentfill}%
\pgfsetlinewidth{0.602250pt}%
\definecolor{currentstroke}{rgb}{0.000000,0.000000,0.000000}%
\pgfsetstrokecolor{currentstroke}%
\pgfsetdash{}{0pt}%
\pgfsys@defobject{currentmarker}{\pgfqpoint{-0.027778in}{0.000000in}}{\pgfqpoint{0.000000in}{0.000000in}}{%
\pgfpathmoveto{\pgfqpoint{0.000000in}{0.000000in}}%
\pgfpathlineto{\pgfqpoint{-0.027778in}{0.000000in}}%
\pgfusepath{stroke,fill}%
}%
\begin{pgfscope}%
\pgfsys@transformshift{0.625000in}{1.515811in}%
\pgfsys@useobject{currentmarker}{}%
\end{pgfscope}%
\end{pgfscope}%
\begin{pgfscope}%
\pgfsetbuttcap%
\pgfsetroundjoin%
\definecolor{currentfill}{rgb}{0.000000,0.000000,0.000000}%
\pgfsetfillcolor{currentfill}%
\pgfsetlinewidth{0.602250pt}%
\definecolor{currentstroke}{rgb}{0.000000,0.000000,0.000000}%
\pgfsetstrokecolor{currentstroke}%
\pgfsetdash{}{0pt}%
\pgfsys@defobject{currentmarker}{\pgfqpoint{-0.027778in}{0.000000in}}{\pgfqpoint{0.000000in}{0.000000in}}{%
\pgfpathmoveto{\pgfqpoint{0.000000in}{0.000000in}}%
\pgfpathlineto{\pgfqpoint{-0.027778in}{0.000000in}}%
\pgfusepath{stroke,fill}%
}%
\begin{pgfscope}%
\pgfsys@transformshift{0.625000in}{1.573514in}%
\pgfsys@useobject{currentmarker}{}%
\end{pgfscope}%
\end{pgfscope}%
\begin{pgfscope}%
\pgfsetbuttcap%
\pgfsetroundjoin%
\definecolor{currentfill}{rgb}{0.000000,0.000000,0.000000}%
\pgfsetfillcolor{currentfill}%
\pgfsetlinewidth{0.602250pt}%
\definecolor{currentstroke}{rgb}{0.000000,0.000000,0.000000}%
\pgfsetstrokecolor{currentstroke}%
\pgfsetdash{}{0pt}%
\pgfsys@defobject{currentmarker}{\pgfqpoint{-0.027778in}{0.000000in}}{\pgfqpoint{0.000000in}{0.000000in}}{%
\pgfpathmoveto{\pgfqpoint{0.000000in}{0.000000in}}%
\pgfpathlineto{\pgfqpoint{-0.027778in}{0.000000in}}%
\pgfusepath{stroke,fill}%
}%
\begin{pgfscope}%
\pgfsys@transformshift{0.625000in}{1.623499in}%
\pgfsys@useobject{currentmarker}{}%
\end{pgfscope}%
\end{pgfscope}%
\begin{pgfscope}%
\pgfsetbuttcap%
\pgfsetroundjoin%
\definecolor{currentfill}{rgb}{0.000000,0.000000,0.000000}%
\pgfsetfillcolor{currentfill}%
\pgfsetlinewidth{0.602250pt}%
\definecolor{currentstroke}{rgb}{0.000000,0.000000,0.000000}%
\pgfsetstrokecolor{currentstroke}%
\pgfsetdash{}{0pt}%
\pgfsys@defobject{currentmarker}{\pgfqpoint{-0.027778in}{0.000000in}}{\pgfqpoint{0.000000in}{0.000000in}}{%
\pgfpathmoveto{\pgfqpoint{0.000000in}{0.000000in}}%
\pgfpathlineto{\pgfqpoint{-0.027778in}{0.000000in}}%
\pgfusepath{stroke,fill}%
}%
\begin{pgfscope}%
\pgfsys@transformshift{0.625000in}{1.667589in}%
\pgfsys@useobject{currentmarker}{}%
\end{pgfscope}%
\end{pgfscope}%
\begin{pgfscope}%
\pgfsetbuttcap%
\pgfsetroundjoin%
\definecolor{currentfill}{rgb}{0.000000,0.000000,0.000000}%
\pgfsetfillcolor{currentfill}%
\pgfsetlinewidth{0.602250pt}%
\definecolor{currentstroke}{rgb}{0.000000,0.000000,0.000000}%
\pgfsetstrokecolor{currentstroke}%
\pgfsetdash{}{0pt}%
\pgfsys@defobject{currentmarker}{\pgfqpoint{-0.027778in}{0.000000in}}{\pgfqpoint{0.000000in}{0.000000in}}{%
\pgfpathmoveto{\pgfqpoint{0.000000in}{0.000000in}}%
\pgfpathlineto{\pgfqpoint{-0.027778in}{0.000000in}}%
\pgfusepath{stroke,fill}%
}%
\begin{pgfscope}%
\pgfsys@transformshift{0.625000in}{1.966493in}%
\pgfsys@useobject{currentmarker}{}%
\end{pgfscope}%
\end{pgfscope}%
\begin{pgfscope}%
\pgfsetbuttcap%
\pgfsetroundjoin%
\definecolor{currentfill}{rgb}{0.000000,0.000000,0.000000}%
\pgfsetfillcolor{currentfill}%
\pgfsetlinewidth{0.602250pt}%
\definecolor{currentstroke}{rgb}{0.000000,0.000000,0.000000}%
\pgfsetstrokecolor{currentstroke}%
\pgfsetdash{}{0pt}%
\pgfsys@defobject{currentmarker}{\pgfqpoint{-0.027778in}{0.000000in}}{\pgfqpoint{0.000000in}{0.000000in}}{%
\pgfpathmoveto{\pgfqpoint{0.000000in}{0.000000in}}%
\pgfpathlineto{\pgfqpoint{-0.027778in}{0.000000in}}%
\pgfusepath{stroke,fill}%
}%
\begin{pgfscope}%
\pgfsys@transformshift{0.625000in}{2.118271in}%
\pgfsys@useobject{currentmarker}{}%
\end{pgfscope}%
\end{pgfscope}%
\begin{pgfscope}%
\pgfsetbuttcap%
\pgfsetroundjoin%
\definecolor{currentfill}{rgb}{0.000000,0.000000,0.000000}%
\pgfsetfillcolor{currentfill}%
\pgfsetlinewidth{0.602250pt}%
\definecolor{currentstroke}{rgb}{0.000000,0.000000,0.000000}%
\pgfsetstrokecolor{currentstroke}%
\pgfsetdash{}{0pt}%
\pgfsys@defobject{currentmarker}{\pgfqpoint{-0.027778in}{0.000000in}}{\pgfqpoint{0.000000in}{0.000000in}}{%
\pgfpathmoveto{\pgfqpoint{0.000000in}{0.000000in}}%
\pgfpathlineto{\pgfqpoint{-0.027778in}{0.000000in}}%
\pgfusepath{stroke,fill}%
}%
\begin{pgfscope}%
\pgfsys@transformshift{0.625000in}{2.225959in}%
\pgfsys@useobject{currentmarker}{}%
\end{pgfscope}%
\end{pgfscope}%
\begin{pgfscope}%
\pgfsetbuttcap%
\pgfsetroundjoin%
\definecolor{currentfill}{rgb}{0.000000,0.000000,0.000000}%
\pgfsetfillcolor{currentfill}%
\pgfsetlinewidth{0.602250pt}%
\definecolor{currentstroke}{rgb}{0.000000,0.000000,0.000000}%
\pgfsetstrokecolor{currentstroke}%
\pgfsetdash{}{0pt}%
\pgfsys@defobject{currentmarker}{\pgfqpoint{-0.027778in}{0.000000in}}{\pgfqpoint{0.000000in}{0.000000in}}{%
\pgfpathmoveto{\pgfqpoint{0.000000in}{0.000000in}}%
\pgfpathlineto{\pgfqpoint{-0.027778in}{0.000000in}}%
\pgfusepath{stroke,fill}%
}%
\begin{pgfscope}%
\pgfsys@transformshift{0.625000in}{2.309488in}%
\pgfsys@useobject{currentmarker}{}%
\end{pgfscope}%
\end{pgfscope}%
\begin{pgfscope}%
\pgfsetbuttcap%
\pgfsetroundjoin%
\definecolor{currentfill}{rgb}{0.000000,0.000000,0.000000}%
\pgfsetfillcolor{currentfill}%
\pgfsetlinewidth{0.602250pt}%
\definecolor{currentstroke}{rgb}{0.000000,0.000000,0.000000}%
\pgfsetstrokecolor{currentstroke}%
\pgfsetdash{}{0pt}%
\pgfsys@defobject{currentmarker}{\pgfqpoint{-0.027778in}{0.000000in}}{\pgfqpoint{0.000000in}{0.000000in}}{%
\pgfpathmoveto{\pgfqpoint{0.000000in}{0.000000in}}%
\pgfpathlineto{\pgfqpoint{-0.027778in}{0.000000in}}%
\pgfusepath{stroke,fill}%
}%
\begin{pgfscope}%
\pgfsys@transformshift{0.625000in}{2.377736in}%
\pgfsys@useobject{currentmarker}{}%
\end{pgfscope}%
\end{pgfscope}%
\begin{pgfscope}%
\pgfsetbuttcap%
\pgfsetroundjoin%
\definecolor{currentfill}{rgb}{0.000000,0.000000,0.000000}%
\pgfsetfillcolor{currentfill}%
\pgfsetlinewidth{0.602250pt}%
\definecolor{currentstroke}{rgb}{0.000000,0.000000,0.000000}%
\pgfsetstrokecolor{currentstroke}%
\pgfsetdash{}{0pt}%
\pgfsys@defobject{currentmarker}{\pgfqpoint{-0.027778in}{0.000000in}}{\pgfqpoint{0.000000in}{0.000000in}}{%
\pgfpathmoveto{\pgfqpoint{0.000000in}{0.000000in}}%
\pgfpathlineto{\pgfqpoint{-0.027778in}{0.000000in}}%
\pgfusepath{stroke,fill}%
}%
\begin{pgfscope}%
\pgfsys@transformshift{0.625000in}{2.435439in}%
\pgfsys@useobject{currentmarker}{}%
\end{pgfscope}%
\end{pgfscope}%
\begin{pgfscope}%
\pgfsetbuttcap%
\pgfsetroundjoin%
\definecolor{currentfill}{rgb}{0.000000,0.000000,0.000000}%
\pgfsetfillcolor{currentfill}%
\pgfsetlinewidth{0.602250pt}%
\definecolor{currentstroke}{rgb}{0.000000,0.000000,0.000000}%
\pgfsetstrokecolor{currentstroke}%
\pgfsetdash{}{0pt}%
\pgfsys@defobject{currentmarker}{\pgfqpoint{-0.027778in}{0.000000in}}{\pgfqpoint{0.000000in}{0.000000in}}{%
\pgfpathmoveto{\pgfqpoint{0.000000in}{0.000000in}}%
\pgfpathlineto{\pgfqpoint{-0.027778in}{0.000000in}}%
\pgfusepath{stroke,fill}%
}%
\begin{pgfscope}%
\pgfsys@transformshift{0.625000in}{2.485424in}%
\pgfsys@useobject{currentmarker}{}%
\end{pgfscope}%
\end{pgfscope}%
\begin{pgfscope}%
\pgfsetbuttcap%
\pgfsetroundjoin%
\definecolor{currentfill}{rgb}{0.000000,0.000000,0.000000}%
\pgfsetfillcolor{currentfill}%
\pgfsetlinewidth{0.602250pt}%
\definecolor{currentstroke}{rgb}{0.000000,0.000000,0.000000}%
\pgfsetstrokecolor{currentstroke}%
\pgfsetdash{}{0pt}%
\pgfsys@defobject{currentmarker}{\pgfqpoint{-0.027778in}{0.000000in}}{\pgfqpoint{0.000000in}{0.000000in}}{%
\pgfpathmoveto{\pgfqpoint{0.000000in}{0.000000in}}%
\pgfpathlineto{\pgfqpoint{-0.027778in}{0.000000in}}%
\pgfusepath{stroke,fill}%
}%
\begin{pgfscope}%
\pgfsys@transformshift{0.625000in}{2.529514in}%
\pgfsys@useobject{currentmarker}{}%
\end{pgfscope}%
\end{pgfscope}%
\begin{pgfscope}%
\pgfsetbuttcap%
\pgfsetroundjoin%
\definecolor{currentfill}{rgb}{0.000000,0.000000,0.000000}%
\pgfsetfillcolor{currentfill}%
\pgfsetlinewidth{0.602250pt}%
\definecolor{currentstroke}{rgb}{0.000000,0.000000,0.000000}%
\pgfsetstrokecolor{currentstroke}%
\pgfsetdash{}{0pt}%
\pgfsys@defobject{currentmarker}{\pgfqpoint{-0.027778in}{0.000000in}}{\pgfqpoint{0.000000in}{0.000000in}}{%
\pgfpathmoveto{\pgfqpoint{0.000000in}{0.000000in}}%
\pgfpathlineto{\pgfqpoint{-0.027778in}{0.000000in}}%
\pgfusepath{stroke,fill}%
}%
\begin{pgfscope}%
\pgfsys@transformshift{0.625000in}{2.828419in}%
\pgfsys@useobject{currentmarker}{}%
\end{pgfscope}%
\end{pgfscope}%
\begin{pgfscope}%
\pgfsetbuttcap%
\pgfsetroundjoin%
\definecolor{currentfill}{rgb}{0.000000,0.000000,0.000000}%
\pgfsetfillcolor{currentfill}%
\pgfsetlinewidth{0.602250pt}%
\definecolor{currentstroke}{rgb}{0.000000,0.000000,0.000000}%
\pgfsetstrokecolor{currentstroke}%
\pgfsetdash{}{0pt}%
\pgfsys@defobject{currentmarker}{\pgfqpoint{-0.027778in}{0.000000in}}{\pgfqpoint{0.000000in}{0.000000in}}{%
\pgfpathmoveto{\pgfqpoint{0.000000in}{0.000000in}}%
\pgfpathlineto{\pgfqpoint{-0.027778in}{0.000000in}}%
\pgfusepath{stroke,fill}%
}%
\begin{pgfscope}%
\pgfsys@transformshift{0.625000in}{2.980196in}%
\pgfsys@useobject{currentmarker}{}%
\end{pgfscope}%
\end{pgfscope}%
\begin{pgfscope}%
\pgfsetbuttcap%
\pgfsetroundjoin%
\definecolor{currentfill}{rgb}{0.000000,0.000000,0.000000}%
\pgfsetfillcolor{currentfill}%
\pgfsetlinewidth{0.602250pt}%
\definecolor{currentstroke}{rgb}{0.000000,0.000000,0.000000}%
\pgfsetstrokecolor{currentstroke}%
\pgfsetdash{}{0pt}%
\pgfsys@defobject{currentmarker}{\pgfqpoint{-0.027778in}{0.000000in}}{\pgfqpoint{0.000000in}{0.000000in}}{%
\pgfpathmoveto{\pgfqpoint{0.000000in}{0.000000in}}%
\pgfpathlineto{\pgfqpoint{-0.027778in}{0.000000in}}%
\pgfusepath{stroke,fill}%
}%
\begin{pgfscope}%
\pgfsys@transformshift{0.625000in}{3.087884in}%
\pgfsys@useobject{currentmarker}{}%
\end{pgfscope}%
\end{pgfscope}%
\begin{pgfscope}%
\pgfsetbuttcap%
\pgfsetroundjoin%
\definecolor{currentfill}{rgb}{0.000000,0.000000,0.000000}%
\pgfsetfillcolor{currentfill}%
\pgfsetlinewidth{0.602250pt}%
\definecolor{currentstroke}{rgb}{0.000000,0.000000,0.000000}%
\pgfsetstrokecolor{currentstroke}%
\pgfsetdash{}{0pt}%
\pgfsys@defobject{currentmarker}{\pgfqpoint{-0.027778in}{0.000000in}}{\pgfqpoint{0.000000in}{0.000000in}}{%
\pgfpathmoveto{\pgfqpoint{0.000000in}{0.000000in}}%
\pgfpathlineto{\pgfqpoint{-0.027778in}{0.000000in}}%
\pgfusepath{stroke,fill}%
}%
\begin{pgfscope}%
\pgfsys@transformshift{0.625000in}{3.171413in}%
\pgfsys@useobject{currentmarker}{}%
\end{pgfscope}%
\end{pgfscope}%
\begin{pgfscope}%
\pgfsetbuttcap%
\pgfsetroundjoin%
\definecolor{currentfill}{rgb}{0.000000,0.000000,0.000000}%
\pgfsetfillcolor{currentfill}%
\pgfsetlinewidth{0.602250pt}%
\definecolor{currentstroke}{rgb}{0.000000,0.000000,0.000000}%
\pgfsetstrokecolor{currentstroke}%
\pgfsetdash{}{0pt}%
\pgfsys@defobject{currentmarker}{\pgfqpoint{-0.027778in}{0.000000in}}{\pgfqpoint{0.000000in}{0.000000in}}{%
\pgfpathmoveto{\pgfqpoint{0.000000in}{0.000000in}}%
\pgfpathlineto{\pgfqpoint{-0.027778in}{0.000000in}}%
\pgfusepath{stroke,fill}%
}%
\begin{pgfscope}%
\pgfsys@transformshift{0.625000in}{3.239662in}%
\pgfsys@useobject{currentmarker}{}%
\end{pgfscope}%
\end{pgfscope}%
\begin{pgfscope}%
\pgfsetbuttcap%
\pgfsetroundjoin%
\definecolor{currentfill}{rgb}{0.000000,0.000000,0.000000}%
\pgfsetfillcolor{currentfill}%
\pgfsetlinewidth{0.602250pt}%
\definecolor{currentstroke}{rgb}{0.000000,0.000000,0.000000}%
\pgfsetstrokecolor{currentstroke}%
\pgfsetdash{}{0pt}%
\pgfsys@defobject{currentmarker}{\pgfqpoint{-0.027778in}{0.000000in}}{\pgfqpoint{0.000000in}{0.000000in}}{%
\pgfpathmoveto{\pgfqpoint{0.000000in}{0.000000in}}%
\pgfpathlineto{\pgfqpoint{-0.027778in}{0.000000in}}%
\pgfusepath{stroke,fill}%
}%
\begin{pgfscope}%
\pgfsys@transformshift{0.625000in}{3.297365in}%
\pgfsys@useobject{currentmarker}{}%
\end{pgfscope}%
\end{pgfscope}%
\begin{pgfscope}%
\pgfsetbuttcap%
\pgfsetroundjoin%
\definecolor{currentfill}{rgb}{0.000000,0.000000,0.000000}%
\pgfsetfillcolor{currentfill}%
\pgfsetlinewidth{0.602250pt}%
\definecolor{currentstroke}{rgb}{0.000000,0.000000,0.000000}%
\pgfsetstrokecolor{currentstroke}%
\pgfsetdash{}{0pt}%
\pgfsys@defobject{currentmarker}{\pgfqpoint{-0.027778in}{0.000000in}}{\pgfqpoint{0.000000in}{0.000000in}}{%
\pgfpathmoveto{\pgfqpoint{0.000000in}{0.000000in}}%
\pgfpathlineto{\pgfqpoint{-0.027778in}{0.000000in}}%
\pgfusepath{stroke,fill}%
}%
\begin{pgfscope}%
\pgfsys@transformshift{0.625000in}{3.347349in}%
\pgfsys@useobject{currentmarker}{}%
\end{pgfscope}%
\end{pgfscope}%
\begin{pgfscope}%
\pgfsetbuttcap%
\pgfsetroundjoin%
\definecolor{currentfill}{rgb}{0.000000,0.000000,0.000000}%
\pgfsetfillcolor{currentfill}%
\pgfsetlinewidth{0.602250pt}%
\definecolor{currentstroke}{rgb}{0.000000,0.000000,0.000000}%
\pgfsetstrokecolor{currentstroke}%
\pgfsetdash{}{0pt}%
\pgfsys@defobject{currentmarker}{\pgfqpoint{-0.027778in}{0.000000in}}{\pgfqpoint{0.000000in}{0.000000in}}{%
\pgfpathmoveto{\pgfqpoint{0.000000in}{0.000000in}}%
\pgfpathlineto{\pgfqpoint{-0.027778in}{0.000000in}}%
\pgfusepath{stroke,fill}%
}%
\begin{pgfscope}%
\pgfsys@transformshift{0.625000in}{3.391439in}%
\pgfsys@useobject{currentmarker}{}%
\end{pgfscope}%
\end{pgfscope}%
\begin{pgfscope}%
\definecolor{textcolor}{rgb}{0.000000,0.000000,0.000000}%
\pgfsetstrokecolor{textcolor}%
\pgfsetfillcolor{textcolor}%
\pgftext[x=0.162343in,y=1.980000in,,bottom,rotate=90.000000]{\color{textcolor}\sffamily\fontsize{11.000000}{13.200000}\selectfont \textrm{QMC Error}}%
\end{pgfscope}%
\begin{pgfscope}%
\pgfpathrectangle{\pgfqpoint{0.625000in}{0.440000in}}{\pgfqpoint{3.875000in}{3.080000in}}%
\pgfusepath{clip}%
\pgfsetbuttcap%
\pgfsetroundjoin%
\definecolor{currentfill}{rgb}{0.000000,0.000000,0.000000}%
\pgfsetfillcolor{currentfill}%
\pgfsetlinewidth{1.003750pt}%
\definecolor{currentstroke}{rgb}{0.000000,0.000000,0.000000}%
\pgfsetstrokecolor{currentstroke}%
\pgfsetdash{}{0pt}%
\pgfsys@defobject{currentmarker}{\pgfqpoint{-0.041667in}{-0.041667in}}{\pgfqpoint{0.041667in}{0.041667in}}{%
\pgfpathmoveto{\pgfqpoint{0.000000in}{-0.041667in}}%
\pgfpathcurveto{\pgfqpoint{0.011050in}{-0.041667in}}{\pgfqpoint{0.021649in}{-0.037276in}}{\pgfqpoint{0.029463in}{-0.029463in}}%
\pgfpathcurveto{\pgfqpoint{0.037276in}{-0.021649in}}{\pgfqpoint{0.041667in}{-0.011050in}}{\pgfqpoint{0.041667in}{0.000000in}}%
\pgfpathcurveto{\pgfqpoint{0.041667in}{0.011050in}}{\pgfqpoint{0.037276in}{0.021649in}}{\pgfqpoint{0.029463in}{0.029463in}}%
\pgfpathcurveto{\pgfqpoint{0.021649in}{0.037276in}}{\pgfqpoint{0.011050in}{0.041667in}}{\pgfqpoint{0.000000in}{0.041667in}}%
\pgfpathcurveto{\pgfqpoint{-0.011050in}{0.041667in}}{\pgfqpoint{-0.021649in}{0.037276in}}{\pgfqpoint{-0.029463in}{0.029463in}}%
\pgfpathcurveto{\pgfqpoint{-0.037276in}{0.021649in}}{\pgfqpoint{-0.041667in}{0.011050in}}{\pgfqpoint{-0.041667in}{0.000000in}}%
\pgfpathcurveto{\pgfqpoint{-0.041667in}{-0.011050in}}{\pgfqpoint{-0.037276in}{-0.021649in}}{\pgfqpoint{-0.029463in}{-0.029463in}}%
\pgfpathcurveto{\pgfqpoint{-0.021649in}{-0.037276in}}{\pgfqpoint{-0.011050in}{-0.041667in}}{\pgfqpoint{0.000000in}{-0.041667in}}%
\pgfpathclose%
\pgfusepath{stroke,fill}%
}%
\begin{pgfscope}%
\pgfsys@transformshift{0.801136in}{3.380000in}%
\pgfsys@useobject{currentmarker}{}%
\end{pgfscope}%
\begin{pgfscope}%
\pgfsys@transformshift{1.121384in}{3.073373in}%
\pgfsys@useobject{currentmarker}{}%
\end{pgfscope}%
\begin{pgfscope}%
\pgfsys@transformshift{1.441632in}{2.826451in}%
\pgfsys@useobject{currentmarker}{}%
\end{pgfscope}%
\begin{pgfscope}%
\pgfsys@transformshift{1.761880in}{2.597298in}%
\pgfsys@useobject{currentmarker}{}%
\end{pgfscope}%
\begin{pgfscope}%
\pgfsys@transformshift{2.082128in}{2.327382in}%
\pgfsys@useobject{currentmarker}{}%
\end{pgfscope}%
\begin{pgfscope}%
\pgfsys@transformshift{2.402376in}{2.103568in}%
\pgfsys@useobject{currentmarker}{}%
\end{pgfscope}%
\begin{pgfscope}%
\pgfsys@transformshift{2.722624in}{1.814804in}%
\pgfsys@useobject{currentmarker}{}%
\end{pgfscope}%
\begin{pgfscope}%
\pgfsys@transformshift{3.042872in}{1.589506in}%
\pgfsys@useobject{currentmarker}{}%
\end{pgfscope}%
\begin{pgfscope}%
\pgfsys@transformshift{3.363120in}{1.370488in}%
\pgfsys@useobject{currentmarker}{}%
\end{pgfscope}%
\begin{pgfscope}%
\pgfsys@transformshift{3.683368in}{1.172616in}%
\pgfsys@useobject{currentmarker}{}%
\end{pgfscope}%
\begin{pgfscope}%
\pgfsys@transformshift{4.003616in}{0.855069in}%
\pgfsys@useobject{currentmarker}{}%
\end{pgfscope}%
\begin{pgfscope}%
\pgfsys@transformshift{4.323864in}{0.580000in}%
\pgfsys@useobject{currentmarker}{}%
\end{pgfscope}%
\end{pgfscope}%
\begin{pgfscope}%
\pgfpathrectangle{\pgfqpoint{0.625000in}{0.440000in}}{\pgfqpoint{3.875000in}{3.080000in}}%
\pgfusepath{clip}%
\pgfsetbuttcap%
\pgfsetroundjoin%
\pgfsetlinewidth{1.505625pt}%
\definecolor{currentstroke}{rgb}{0.000000,0.000000,0.000000}%
\pgfsetstrokecolor{currentstroke}%
\pgfsetdash{{5.550000pt}{2.400000pt}}{0.000000pt}%
\pgfpathmoveto{\pgfqpoint{0.801136in}{3.339175in}}%
\pgfpathlineto{\pgfqpoint{4.323864in}{0.609251in}}%
\pgfusepath{stroke}%
\end{pgfscope}%
\begin{pgfscope}%
\pgfsetrectcap%
\pgfsetmiterjoin%
\pgfsetlinewidth{0.803000pt}%
\definecolor{currentstroke}{rgb}{0.000000,0.000000,0.000000}%
\pgfsetstrokecolor{currentstroke}%
\pgfsetdash{}{0pt}%
\pgfpathmoveto{\pgfqpoint{0.625000in}{0.440000in}}%
\pgfpathlineto{\pgfqpoint{0.625000in}{3.520000in}}%
\pgfusepath{stroke}%
\end{pgfscope}%
\begin{pgfscope}%
\pgfsetrectcap%
\pgfsetmiterjoin%
\pgfsetlinewidth{0.000000pt}%
\definecolor{currentstroke}{rgb}{0.000000,0.000000,0.000000}%
\pgfsetstrokecolor{currentstroke}%
\pgfsetstrokeopacity{0.000000}%
\pgfsetdash{}{0pt}%
\pgfpathmoveto{\pgfqpoint{4.500000in}{0.440000in}}%
\pgfpathlineto{\pgfqpoint{4.500000in}{3.520000in}}%
\pgfusepath{}%
\end{pgfscope}%
\begin{pgfscope}%
\pgfsetrectcap%
\pgfsetmiterjoin%
\pgfsetlinewidth{0.803000pt}%
\definecolor{currentstroke}{rgb}{0.000000,0.000000,0.000000}%
\pgfsetstrokecolor{currentstroke}%
\pgfsetdash{}{0pt}%
\pgfpathmoveto{\pgfqpoint{0.625000in}{0.440000in}}%
\pgfpathlineto{\pgfqpoint{4.500000in}{0.440000in}}%
\pgfusepath{stroke}%
\end{pgfscope}%
\begin{pgfscope}%
\pgfsetrectcap%
\pgfsetmiterjoin%
\pgfsetlinewidth{0.000000pt}%
\definecolor{currentstroke}{rgb}{0.000000,0.000000,0.000000}%
\pgfsetstrokecolor{currentstroke}%
\pgfsetstrokeopacity{0.000000}%
\pgfsetdash{}{0pt}%
\pgfpathmoveto{\pgfqpoint{0.625000in}{3.520000in}}%
\pgfpathlineto{\pgfqpoint{4.500000in}{3.520000in}}%
\pgfusepath{}%
\end{pgfscope}%
\begin{pgfscope}%
\pgfsetbuttcap%
\pgfsetmiterjoin%
\definecolor{currentfill}{rgb}{1.000000,1.000000,1.000000}%
\pgfsetfillcolor{currentfill}%
\pgfsetfillopacity{0.800000}%
\pgfsetlinewidth{1.003750pt}%
\definecolor{currentstroke}{rgb}{0.800000,0.800000,0.800000}%
\pgfsetstrokecolor{currentstroke}%
\pgfsetstrokeopacity{0.800000}%
\pgfsetdash{}{0pt}%
\pgfpathmoveto{\pgfqpoint{2.916233in}{3.119488in}}%
\pgfpathlineto{\pgfqpoint{4.393056in}{3.119488in}}%
\pgfpathquadraticcurveto{\pgfqpoint{4.423611in}{3.119488in}}{\pgfqpoint{4.423611in}{3.150043in}}%
\pgfpathlineto{\pgfqpoint{4.423611in}{3.413056in}}%
\pgfpathquadraticcurveto{\pgfqpoint{4.423611in}{3.443611in}}{\pgfqpoint{4.393056in}{3.443611in}}%
\pgfpathlineto{\pgfqpoint{2.916233in}{3.443611in}}%
\pgfpathquadraticcurveto{\pgfqpoint{2.885678in}{3.443611in}}{\pgfqpoint{2.885678in}{3.413056in}}%
\pgfpathlineto{\pgfqpoint{2.885678in}{3.150043in}}%
\pgfpathquadraticcurveto{\pgfqpoint{2.885678in}{3.119488in}}{\pgfqpoint{2.916233in}{3.119488in}}%
\pgfpathclose%
\pgfusepath{stroke,fill}%
\end{pgfscope}%
\begin{pgfscope}%
\pgfsetbuttcap%
\pgfsetroundjoin%
\pgfsetlinewidth{1.505625pt}%
\definecolor{currentstroke}{rgb}{0.000000,0.000000,0.000000}%
\pgfsetstrokecolor{currentstroke}%
\pgfsetdash{{5.550000pt}{2.400000pt}}{0.000000pt}%
\pgfpathmoveto{\pgfqpoint{2.946789in}{3.300408in}}%
\pgfpathlineto{\pgfqpoint{3.252344in}{3.300408in}}%
\pgfusepath{stroke}%
\end{pgfscope}%
\begin{pgfscope}%
\definecolor{textcolor}{rgb}{0.000000,0.000000,0.000000}%
\pgfsetstrokecolor{textcolor}%
\pgfsetfillcolor{textcolor}%
\pgftext[x=3.374567in,y=3.246936in,left,base]{\color{textcolor}\sffamily\fontsize{11.000000}{13.200000}\selectfont \(\displaystyle 0.0078N_{\mathrm{QMC}}^{-0.9564}\)}%
\end{pgfscope}%
\end{pgfpicture}%
\makeatother%
\endgroup%

    \caption[Quasi-Monte-Carlo error, for $Q=u(\bzero)$ and $k=10$.]{Quasi-Monte-Carlo error with increasing number of Quasi-Monte-Carlo points, for $Q=u(\bzero)$ and $k=10$. \label{fig:qmcorigin10}}
  \end{figure}
\begin{figure}[h]
  \centering
%% Creator: Matplotlib, PGF backend
%%
%% To include the figure in your LaTeX document, write
%%   \input{<filename>.pgf}
%%
%% Make sure the required packages are loaded in your preamble
%%   \usepackage{pgf}
%%
%% Figures using additional raster images can only be included by \input if
%% they are in the same directory as the main LaTeX file. For loading figures
%% from other directories you can use the `import` package
%%   \usepackage{import}
%% and then include the figures with
%%   \import{<path to file>}{<filename>.pgf}
%%
%% Matplotlib used the following preamble
%%   \usepackage{fontspec}
%%   \setmainfont{DejaVuSerif.ttf}[Path=/home/owen/progs/firedrake-complex/firedrake/lib/python3.5/site-packages/matplotlib/mpl-data/fonts/ttf/]
%%   \setsansfont{DejaVuSans.ttf}[Path=/home/owen/progs/firedrake-complex/firedrake/lib/python3.5/site-packages/matplotlib/mpl-data/fonts/ttf/]
%%   \setmonofont{DejaVuSansMono.ttf}[Path=/home/owen/progs/firedrake-complex/firedrake/lib/python3.5/site-packages/matplotlib/mpl-data/fonts/ttf/]
%%
\begingroup%
\makeatletter%
\begin{pgfpicture}%
\pgfpathrectangle{\pgfpointorigin}{\pgfqpoint{5.000000in}{4.000000in}}%
\pgfusepath{use as bounding box, clip}%
\begin{pgfscope}%
\pgfsetbuttcap%
\pgfsetmiterjoin%
\definecolor{currentfill}{rgb}{1.000000,1.000000,1.000000}%
\pgfsetfillcolor{currentfill}%
\pgfsetlinewidth{0.000000pt}%
\definecolor{currentstroke}{rgb}{1.000000,1.000000,1.000000}%
\pgfsetstrokecolor{currentstroke}%
\pgfsetdash{}{0pt}%
\pgfpathmoveto{\pgfqpoint{0.000000in}{0.000000in}}%
\pgfpathlineto{\pgfqpoint{5.000000in}{0.000000in}}%
\pgfpathlineto{\pgfqpoint{5.000000in}{4.000000in}}%
\pgfpathlineto{\pgfqpoint{0.000000in}{4.000000in}}%
\pgfpathclose%
\pgfusepath{fill}%
\end{pgfscope}%
\begin{pgfscope}%
\pgfsetbuttcap%
\pgfsetmiterjoin%
\definecolor{currentfill}{rgb}{1.000000,1.000000,1.000000}%
\pgfsetfillcolor{currentfill}%
\pgfsetlinewidth{0.000000pt}%
\definecolor{currentstroke}{rgb}{0.000000,0.000000,0.000000}%
\pgfsetstrokecolor{currentstroke}%
\pgfsetstrokeopacity{0.000000}%
\pgfsetdash{}{0pt}%
\pgfpathmoveto{\pgfqpoint{0.625000in}{0.440000in}}%
\pgfpathlineto{\pgfqpoint{4.500000in}{0.440000in}}%
\pgfpathlineto{\pgfqpoint{4.500000in}{3.520000in}}%
\pgfpathlineto{\pgfqpoint{0.625000in}{3.520000in}}%
\pgfpathclose%
\pgfusepath{fill}%
\end{pgfscope}%
\begin{pgfscope}%
\pgfsetbuttcap%
\pgfsetroundjoin%
\definecolor{currentfill}{rgb}{0.000000,0.000000,0.000000}%
\pgfsetfillcolor{currentfill}%
\pgfsetlinewidth{0.803000pt}%
\definecolor{currentstroke}{rgb}{0.000000,0.000000,0.000000}%
\pgfsetstrokecolor{currentstroke}%
\pgfsetdash{}{0pt}%
\pgfsys@defobject{currentmarker}{\pgfqpoint{0.000000in}{-0.048611in}}{\pgfqpoint{0.000000in}{0.000000in}}{%
\pgfpathmoveto{\pgfqpoint{0.000000in}{0.000000in}}%
\pgfpathlineto{\pgfqpoint{0.000000in}{-0.048611in}}%
\pgfusepath{stroke,fill}%
}%
\begin{pgfscope}%
\pgfsys@transformshift{0.801136in}{0.440000in}%
\pgfsys@useobject{currentmarker}{}%
\end{pgfscope}%
\end{pgfscope}%
\begin{pgfscope}%
\definecolor{textcolor}{rgb}{0.000000,0.000000,0.000000}%
\pgfsetstrokecolor{textcolor}%
\pgfsetfillcolor{textcolor}%
\pgftext[x=0.801136in,y=0.342778in,,top]{\color{textcolor}\sffamily\fontsize{11.000000}{13.200000}\selectfont \(\displaystyle 10^{0}\)}%
\end{pgfscope}%
\begin{pgfscope}%
\pgfsetbuttcap%
\pgfsetroundjoin%
\definecolor{currentfill}{rgb}{0.000000,0.000000,0.000000}%
\pgfsetfillcolor{currentfill}%
\pgfsetlinewidth{0.803000pt}%
\definecolor{currentstroke}{rgb}{0.000000,0.000000,0.000000}%
\pgfsetstrokecolor{currentstroke}%
\pgfsetdash{}{0pt}%
\pgfsys@defobject{currentmarker}{\pgfqpoint{0.000000in}{-0.048611in}}{\pgfqpoint{0.000000in}{0.000000in}}{%
\pgfpathmoveto{\pgfqpoint{0.000000in}{0.000000in}}%
\pgfpathlineto{\pgfqpoint{0.000000in}{-0.048611in}}%
\pgfusepath{stroke,fill}%
}%
\begin{pgfscope}%
\pgfsys@transformshift{1.864977in}{0.440000in}%
\pgfsys@useobject{currentmarker}{}%
\end{pgfscope}%
\end{pgfscope}%
\begin{pgfscope}%
\definecolor{textcolor}{rgb}{0.000000,0.000000,0.000000}%
\pgfsetstrokecolor{textcolor}%
\pgfsetfillcolor{textcolor}%
\pgftext[x=1.864977in,y=0.342778in,,top]{\color{textcolor}\sffamily\fontsize{11.000000}{13.200000}\selectfont \(\displaystyle 10^{1}\)}%
\end{pgfscope}%
\begin{pgfscope}%
\pgfsetbuttcap%
\pgfsetroundjoin%
\definecolor{currentfill}{rgb}{0.000000,0.000000,0.000000}%
\pgfsetfillcolor{currentfill}%
\pgfsetlinewidth{0.803000pt}%
\definecolor{currentstroke}{rgb}{0.000000,0.000000,0.000000}%
\pgfsetstrokecolor{currentstroke}%
\pgfsetdash{}{0pt}%
\pgfsys@defobject{currentmarker}{\pgfqpoint{0.000000in}{-0.048611in}}{\pgfqpoint{0.000000in}{0.000000in}}{%
\pgfpathmoveto{\pgfqpoint{0.000000in}{0.000000in}}%
\pgfpathlineto{\pgfqpoint{0.000000in}{-0.048611in}}%
\pgfusepath{stroke,fill}%
}%
\begin{pgfscope}%
\pgfsys@transformshift{2.928818in}{0.440000in}%
\pgfsys@useobject{currentmarker}{}%
\end{pgfscope}%
\end{pgfscope}%
\begin{pgfscope}%
\definecolor{textcolor}{rgb}{0.000000,0.000000,0.000000}%
\pgfsetstrokecolor{textcolor}%
\pgfsetfillcolor{textcolor}%
\pgftext[x=2.928818in,y=0.342778in,,top]{\color{textcolor}\sffamily\fontsize{11.000000}{13.200000}\selectfont \(\displaystyle 10^{2}\)}%
\end{pgfscope}%
\begin{pgfscope}%
\pgfsetbuttcap%
\pgfsetroundjoin%
\definecolor{currentfill}{rgb}{0.000000,0.000000,0.000000}%
\pgfsetfillcolor{currentfill}%
\pgfsetlinewidth{0.803000pt}%
\definecolor{currentstroke}{rgb}{0.000000,0.000000,0.000000}%
\pgfsetstrokecolor{currentstroke}%
\pgfsetdash{}{0pt}%
\pgfsys@defobject{currentmarker}{\pgfqpoint{0.000000in}{-0.048611in}}{\pgfqpoint{0.000000in}{0.000000in}}{%
\pgfpathmoveto{\pgfqpoint{0.000000in}{0.000000in}}%
\pgfpathlineto{\pgfqpoint{0.000000in}{-0.048611in}}%
\pgfusepath{stroke,fill}%
}%
\begin{pgfscope}%
\pgfsys@transformshift{3.992658in}{0.440000in}%
\pgfsys@useobject{currentmarker}{}%
\end{pgfscope}%
\end{pgfscope}%
\begin{pgfscope}%
\definecolor{textcolor}{rgb}{0.000000,0.000000,0.000000}%
\pgfsetstrokecolor{textcolor}%
\pgfsetfillcolor{textcolor}%
\pgftext[x=3.992658in,y=0.342778in,,top]{\color{textcolor}\sffamily\fontsize{11.000000}{13.200000}\selectfont \(\displaystyle 10^{3}\)}%
\end{pgfscope}%
\begin{pgfscope}%
\pgfsetbuttcap%
\pgfsetroundjoin%
\definecolor{currentfill}{rgb}{0.000000,0.000000,0.000000}%
\pgfsetfillcolor{currentfill}%
\pgfsetlinewidth{0.602250pt}%
\definecolor{currentstroke}{rgb}{0.000000,0.000000,0.000000}%
\pgfsetstrokecolor{currentstroke}%
\pgfsetdash{}{0pt}%
\pgfsys@defobject{currentmarker}{\pgfqpoint{0.000000in}{-0.027778in}}{\pgfqpoint{0.000000in}{0.000000in}}{%
\pgfpathmoveto{\pgfqpoint{0.000000in}{0.000000in}}%
\pgfpathlineto{\pgfqpoint{0.000000in}{-0.027778in}}%
\pgfusepath{stroke,fill}%
}%
\begin{pgfscope}%
\pgfsys@transformshift{0.636345in}{0.440000in}%
\pgfsys@useobject{currentmarker}{}%
\end{pgfscope}%
\end{pgfscope}%
\begin{pgfscope}%
\pgfsetbuttcap%
\pgfsetroundjoin%
\definecolor{currentfill}{rgb}{0.000000,0.000000,0.000000}%
\pgfsetfillcolor{currentfill}%
\pgfsetlinewidth{0.602250pt}%
\definecolor{currentstroke}{rgb}{0.000000,0.000000,0.000000}%
\pgfsetstrokecolor{currentstroke}%
\pgfsetdash{}{0pt}%
\pgfsys@defobject{currentmarker}{\pgfqpoint{0.000000in}{-0.027778in}}{\pgfqpoint{0.000000in}{0.000000in}}{%
\pgfpathmoveto{\pgfqpoint{0.000000in}{0.000000in}}%
\pgfpathlineto{\pgfqpoint{0.000000in}{-0.027778in}}%
\pgfusepath{stroke,fill}%
}%
\begin{pgfscope}%
\pgfsys@transformshift{0.698040in}{0.440000in}%
\pgfsys@useobject{currentmarker}{}%
\end{pgfscope}%
\end{pgfscope}%
\begin{pgfscope}%
\pgfsetbuttcap%
\pgfsetroundjoin%
\definecolor{currentfill}{rgb}{0.000000,0.000000,0.000000}%
\pgfsetfillcolor{currentfill}%
\pgfsetlinewidth{0.602250pt}%
\definecolor{currentstroke}{rgb}{0.000000,0.000000,0.000000}%
\pgfsetstrokecolor{currentstroke}%
\pgfsetdash{}{0pt}%
\pgfsys@defobject{currentmarker}{\pgfqpoint{0.000000in}{-0.027778in}}{\pgfqpoint{0.000000in}{0.000000in}}{%
\pgfpathmoveto{\pgfqpoint{0.000000in}{0.000000in}}%
\pgfpathlineto{\pgfqpoint{0.000000in}{-0.027778in}}%
\pgfusepath{stroke,fill}%
}%
\begin{pgfscope}%
\pgfsys@transformshift{0.752458in}{0.440000in}%
\pgfsys@useobject{currentmarker}{}%
\end{pgfscope}%
\end{pgfscope}%
\begin{pgfscope}%
\pgfsetbuttcap%
\pgfsetroundjoin%
\definecolor{currentfill}{rgb}{0.000000,0.000000,0.000000}%
\pgfsetfillcolor{currentfill}%
\pgfsetlinewidth{0.602250pt}%
\definecolor{currentstroke}{rgb}{0.000000,0.000000,0.000000}%
\pgfsetstrokecolor{currentstroke}%
\pgfsetdash{}{0pt}%
\pgfsys@defobject{currentmarker}{\pgfqpoint{0.000000in}{-0.027778in}}{\pgfqpoint{0.000000in}{0.000000in}}{%
\pgfpathmoveto{\pgfqpoint{0.000000in}{0.000000in}}%
\pgfpathlineto{\pgfqpoint{0.000000in}{-0.027778in}}%
\pgfusepath{stroke,fill}%
}%
\begin{pgfscope}%
\pgfsys@transformshift{1.121384in}{0.440000in}%
\pgfsys@useobject{currentmarker}{}%
\end{pgfscope}%
\end{pgfscope}%
\begin{pgfscope}%
\pgfsetbuttcap%
\pgfsetroundjoin%
\definecolor{currentfill}{rgb}{0.000000,0.000000,0.000000}%
\pgfsetfillcolor{currentfill}%
\pgfsetlinewidth{0.602250pt}%
\definecolor{currentstroke}{rgb}{0.000000,0.000000,0.000000}%
\pgfsetstrokecolor{currentstroke}%
\pgfsetdash{}{0pt}%
\pgfsys@defobject{currentmarker}{\pgfqpoint{0.000000in}{-0.027778in}}{\pgfqpoint{0.000000in}{0.000000in}}{%
\pgfpathmoveto{\pgfqpoint{0.000000in}{0.000000in}}%
\pgfpathlineto{\pgfqpoint{0.000000in}{-0.027778in}}%
\pgfusepath{stroke,fill}%
}%
\begin{pgfscope}%
\pgfsys@transformshift{1.308717in}{0.440000in}%
\pgfsys@useobject{currentmarker}{}%
\end{pgfscope}%
\end{pgfscope}%
\begin{pgfscope}%
\pgfsetbuttcap%
\pgfsetroundjoin%
\definecolor{currentfill}{rgb}{0.000000,0.000000,0.000000}%
\pgfsetfillcolor{currentfill}%
\pgfsetlinewidth{0.602250pt}%
\definecolor{currentstroke}{rgb}{0.000000,0.000000,0.000000}%
\pgfsetstrokecolor{currentstroke}%
\pgfsetdash{}{0pt}%
\pgfsys@defobject{currentmarker}{\pgfqpoint{0.000000in}{-0.027778in}}{\pgfqpoint{0.000000in}{0.000000in}}{%
\pgfpathmoveto{\pgfqpoint{0.000000in}{0.000000in}}%
\pgfpathlineto{\pgfqpoint{0.000000in}{-0.027778in}}%
\pgfusepath{stroke,fill}%
}%
\begin{pgfscope}%
\pgfsys@transformshift{1.441632in}{0.440000in}%
\pgfsys@useobject{currentmarker}{}%
\end{pgfscope}%
\end{pgfscope}%
\begin{pgfscope}%
\pgfsetbuttcap%
\pgfsetroundjoin%
\definecolor{currentfill}{rgb}{0.000000,0.000000,0.000000}%
\pgfsetfillcolor{currentfill}%
\pgfsetlinewidth{0.602250pt}%
\definecolor{currentstroke}{rgb}{0.000000,0.000000,0.000000}%
\pgfsetstrokecolor{currentstroke}%
\pgfsetdash{}{0pt}%
\pgfsys@defobject{currentmarker}{\pgfqpoint{0.000000in}{-0.027778in}}{\pgfqpoint{0.000000in}{0.000000in}}{%
\pgfpathmoveto{\pgfqpoint{0.000000in}{0.000000in}}%
\pgfpathlineto{\pgfqpoint{0.000000in}{-0.027778in}}%
\pgfusepath{stroke,fill}%
}%
\begin{pgfscope}%
\pgfsys@transformshift{1.544729in}{0.440000in}%
\pgfsys@useobject{currentmarker}{}%
\end{pgfscope}%
\end{pgfscope}%
\begin{pgfscope}%
\pgfsetbuttcap%
\pgfsetroundjoin%
\definecolor{currentfill}{rgb}{0.000000,0.000000,0.000000}%
\pgfsetfillcolor{currentfill}%
\pgfsetlinewidth{0.602250pt}%
\definecolor{currentstroke}{rgb}{0.000000,0.000000,0.000000}%
\pgfsetstrokecolor{currentstroke}%
\pgfsetdash{}{0pt}%
\pgfsys@defobject{currentmarker}{\pgfqpoint{0.000000in}{-0.027778in}}{\pgfqpoint{0.000000in}{0.000000in}}{%
\pgfpathmoveto{\pgfqpoint{0.000000in}{0.000000in}}%
\pgfpathlineto{\pgfqpoint{0.000000in}{-0.027778in}}%
\pgfusepath{stroke,fill}%
}%
\begin{pgfscope}%
\pgfsys@transformshift{1.628965in}{0.440000in}%
\pgfsys@useobject{currentmarker}{}%
\end{pgfscope}%
\end{pgfscope}%
\begin{pgfscope}%
\pgfsetbuttcap%
\pgfsetroundjoin%
\definecolor{currentfill}{rgb}{0.000000,0.000000,0.000000}%
\pgfsetfillcolor{currentfill}%
\pgfsetlinewidth{0.602250pt}%
\definecolor{currentstroke}{rgb}{0.000000,0.000000,0.000000}%
\pgfsetstrokecolor{currentstroke}%
\pgfsetdash{}{0pt}%
\pgfsys@defobject{currentmarker}{\pgfqpoint{0.000000in}{-0.027778in}}{\pgfqpoint{0.000000in}{0.000000in}}{%
\pgfpathmoveto{\pgfqpoint{0.000000in}{0.000000in}}%
\pgfpathlineto{\pgfqpoint{0.000000in}{-0.027778in}}%
\pgfusepath{stroke,fill}%
}%
\begin{pgfscope}%
\pgfsys@transformshift{1.700186in}{0.440000in}%
\pgfsys@useobject{currentmarker}{}%
\end{pgfscope}%
\end{pgfscope}%
\begin{pgfscope}%
\pgfsetbuttcap%
\pgfsetroundjoin%
\definecolor{currentfill}{rgb}{0.000000,0.000000,0.000000}%
\pgfsetfillcolor{currentfill}%
\pgfsetlinewidth{0.602250pt}%
\definecolor{currentstroke}{rgb}{0.000000,0.000000,0.000000}%
\pgfsetstrokecolor{currentstroke}%
\pgfsetdash{}{0pt}%
\pgfsys@defobject{currentmarker}{\pgfqpoint{0.000000in}{-0.027778in}}{\pgfqpoint{0.000000in}{0.000000in}}{%
\pgfpathmoveto{\pgfqpoint{0.000000in}{0.000000in}}%
\pgfpathlineto{\pgfqpoint{0.000000in}{-0.027778in}}%
\pgfusepath{stroke,fill}%
}%
\begin{pgfscope}%
\pgfsys@transformshift{1.761880in}{0.440000in}%
\pgfsys@useobject{currentmarker}{}%
\end{pgfscope}%
\end{pgfscope}%
\begin{pgfscope}%
\pgfsetbuttcap%
\pgfsetroundjoin%
\definecolor{currentfill}{rgb}{0.000000,0.000000,0.000000}%
\pgfsetfillcolor{currentfill}%
\pgfsetlinewidth{0.602250pt}%
\definecolor{currentstroke}{rgb}{0.000000,0.000000,0.000000}%
\pgfsetstrokecolor{currentstroke}%
\pgfsetdash{}{0pt}%
\pgfsys@defobject{currentmarker}{\pgfqpoint{0.000000in}{-0.027778in}}{\pgfqpoint{0.000000in}{0.000000in}}{%
\pgfpathmoveto{\pgfqpoint{0.000000in}{0.000000in}}%
\pgfpathlineto{\pgfqpoint{0.000000in}{-0.027778in}}%
\pgfusepath{stroke,fill}%
}%
\begin{pgfscope}%
\pgfsys@transformshift{1.816298in}{0.440000in}%
\pgfsys@useobject{currentmarker}{}%
\end{pgfscope}%
\end{pgfscope}%
\begin{pgfscope}%
\pgfsetbuttcap%
\pgfsetroundjoin%
\definecolor{currentfill}{rgb}{0.000000,0.000000,0.000000}%
\pgfsetfillcolor{currentfill}%
\pgfsetlinewidth{0.602250pt}%
\definecolor{currentstroke}{rgb}{0.000000,0.000000,0.000000}%
\pgfsetstrokecolor{currentstroke}%
\pgfsetdash{}{0pt}%
\pgfsys@defobject{currentmarker}{\pgfqpoint{0.000000in}{-0.027778in}}{\pgfqpoint{0.000000in}{0.000000in}}{%
\pgfpathmoveto{\pgfqpoint{0.000000in}{0.000000in}}%
\pgfpathlineto{\pgfqpoint{0.000000in}{-0.027778in}}%
\pgfusepath{stroke,fill}%
}%
\begin{pgfscope}%
\pgfsys@transformshift{2.185225in}{0.440000in}%
\pgfsys@useobject{currentmarker}{}%
\end{pgfscope}%
\end{pgfscope}%
\begin{pgfscope}%
\pgfsetbuttcap%
\pgfsetroundjoin%
\definecolor{currentfill}{rgb}{0.000000,0.000000,0.000000}%
\pgfsetfillcolor{currentfill}%
\pgfsetlinewidth{0.602250pt}%
\definecolor{currentstroke}{rgb}{0.000000,0.000000,0.000000}%
\pgfsetstrokecolor{currentstroke}%
\pgfsetdash{}{0pt}%
\pgfsys@defobject{currentmarker}{\pgfqpoint{0.000000in}{-0.027778in}}{\pgfqpoint{0.000000in}{0.000000in}}{%
\pgfpathmoveto{\pgfqpoint{0.000000in}{0.000000in}}%
\pgfpathlineto{\pgfqpoint{0.000000in}{-0.027778in}}%
\pgfusepath{stroke,fill}%
}%
\begin{pgfscope}%
\pgfsys@transformshift{2.372558in}{0.440000in}%
\pgfsys@useobject{currentmarker}{}%
\end{pgfscope}%
\end{pgfscope}%
\begin{pgfscope}%
\pgfsetbuttcap%
\pgfsetroundjoin%
\definecolor{currentfill}{rgb}{0.000000,0.000000,0.000000}%
\pgfsetfillcolor{currentfill}%
\pgfsetlinewidth{0.602250pt}%
\definecolor{currentstroke}{rgb}{0.000000,0.000000,0.000000}%
\pgfsetstrokecolor{currentstroke}%
\pgfsetdash{}{0pt}%
\pgfsys@defobject{currentmarker}{\pgfqpoint{0.000000in}{-0.027778in}}{\pgfqpoint{0.000000in}{0.000000in}}{%
\pgfpathmoveto{\pgfqpoint{0.000000in}{0.000000in}}%
\pgfpathlineto{\pgfqpoint{0.000000in}{-0.027778in}}%
\pgfusepath{stroke,fill}%
}%
\begin{pgfscope}%
\pgfsys@transformshift{2.505473in}{0.440000in}%
\pgfsys@useobject{currentmarker}{}%
\end{pgfscope}%
\end{pgfscope}%
\begin{pgfscope}%
\pgfsetbuttcap%
\pgfsetroundjoin%
\definecolor{currentfill}{rgb}{0.000000,0.000000,0.000000}%
\pgfsetfillcolor{currentfill}%
\pgfsetlinewidth{0.602250pt}%
\definecolor{currentstroke}{rgb}{0.000000,0.000000,0.000000}%
\pgfsetstrokecolor{currentstroke}%
\pgfsetdash{}{0pt}%
\pgfsys@defobject{currentmarker}{\pgfqpoint{0.000000in}{-0.027778in}}{\pgfqpoint{0.000000in}{0.000000in}}{%
\pgfpathmoveto{\pgfqpoint{0.000000in}{0.000000in}}%
\pgfpathlineto{\pgfqpoint{0.000000in}{-0.027778in}}%
\pgfusepath{stroke,fill}%
}%
\begin{pgfscope}%
\pgfsys@transformshift{2.608570in}{0.440000in}%
\pgfsys@useobject{currentmarker}{}%
\end{pgfscope}%
\end{pgfscope}%
\begin{pgfscope}%
\pgfsetbuttcap%
\pgfsetroundjoin%
\definecolor{currentfill}{rgb}{0.000000,0.000000,0.000000}%
\pgfsetfillcolor{currentfill}%
\pgfsetlinewidth{0.602250pt}%
\definecolor{currentstroke}{rgb}{0.000000,0.000000,0.000000}%
\pgfsetstrokecolor{currentstroke}%
\pgfsetdash{}{0pt}%
\pgfsys@defobject{currentmarker}{\pgfqpoint{0.000000in}{-0.027778in}}{\pgfqpoint{0.000000in}{0.000000in}}{%
\pgfpathmoveto{\pgfqpoint{0.000000in}{0.000000in}}%
\pgfpathlineto{\pgfqpoint{0.000000in}{-0.027778in}}%
\pgfusepath{stroke,fill}%
}%
\begin{pgfscope}%
\pgfsys@transformshift{2.692806in}{0.440000in}%
\pgfsys@useobject{currentmarker}{}%
\end{pgfscope}%
\end{pgfscope}%
\begin{pgfscope}%
\pgfsetbuttcap%
\pgfsetroundjoin%
\definecolor{currentfill}{rgb}{0.000000,0.000000,0.000000}%
\pgfsetfillcolor{currentfill}%
\pgfsetlinewidth{0.602250pt}%
\definecolor{currentstroke}{rgb}{0.000000,0.000000,0.000000}%
\pgfsetstrokecolor{currentstroke}%
\pgfsetdash{}{0pt}%
\pgfsys@defobject{currentmarker}{\pgfqpoint{0.000000in}{-0.027778in}}{\pgfqpoint{0.000000in}{0.000000in}}{%
\pgfpathmoveto{\pgfqpoint{0.000000in}{0.000000in}}%
\pgfpathlineto{\pgfqpoint{0.000000in}{-0.027778in}}%
\pgfusepath{stroke,fill}%
}%
\begin{pgfscope}%
\pgfsys@transformshift{2.764027in}{0.440000in}%
\pgfsys@useobject{currentmarker}{}%
\end{pgfscope}%
\end{pgfscope}%
\begin{pgfscope}%
\pgfsetbuttcap%
\pgfsetroundjoin%
\definecolor{currentfill}{rgb}{0.000000,0.000000,0.000000}%
\pgfsetfillcolor{currentfill}%
\pgfsetlinewidth{0.602250pt}%
\definecolor{currentstroke}{rgb}{0.000000,0.000000,0.000000}%
\pgfsetstrokecolor{currentstroke}%
\pgfsetdash{}{0pt}%
\pgfsys@defobject{currentmarker}{\pgfqpoint{0.000000in}{-0.027778in}}{\pgfqpoint{0.000000in}{0.000000in}}{%
\pgfpathmoveto{\pgfqpoint{0.000000in}{0.000000in}}%
\pgfpathlineto{\pgfqpoint{0.000000in}{-0.027778in}}%
\pgfusepath{stroke,fill}%
}%
\begin{pgfscope}%
\pgfsys@transformshift{2.825721in}{0.440000in}%
\pgfsys@useobject{currentmarker}{}%
\end{pgfscope}%
\end{pgfscope}%
\begin{pgfscope}%
\pgfsetbuttcap%
\pgfsetroundjoin%
\definecolor{currentfill}{rgb}{0.000000,0.000000,0.000000}%
\pgfsetfillcolor{currentfill}%
\pgfsetlinewidth{0.602250pt}%
\definecolor{currentstroke}{rgb}{0.000000,0.000000,0.000000}%
\pgfsetstrokecolor{currentstroke}%
\pgfsetdash{}{0pt}%
\pgfsys@defobject{currentmarker}{\pgfqpoint{0.000000in}{-0.027778in}}{\pgfqpoint{0.000000in}{0.000000in}}{%
\pgfpathmoveto{\pgfqpoint{0.000000in}{0.000000in}}%
\pgfpathlineto{\pgfqpoint{0.000000in}{-0.027778in}}%
\pgfusepath{stroke,fill}%
}%
\begin{pgfscope}%
\pgfsys@transformshift{2.880139in}{0.440000in}%
\pgfsys@useobject{currentmarker}{}%
\end{pgfscope}%
\end{pgfscope}%
\begin{pgfscope}%
\pgfsetbuttcap%
\pgfsetroundjoin%
\definecolor{currentfill}{rgb}{0.000000,0.000000,0.000000}%
\pgfsetfillcolor{currentfill}%
\pgfsetlinewidth{0.602250pt}%
\definecolor{currentstroke}{rgb}{0.000000,0.000000,0.000000}%
\pgfsetstrokecolor{currentstroke}%
\pgfsetdash{}{0pt}%
\pgfsys@defobject{currentmarker}{\pgfqpoint{0.000000in}{-0.027778in}}{\pgfqpoint{0.000000in}{0.000000in}}{%
\pgfpathmoveto{\pgfqpoint{0.000000in}{0.000000in}}%
\pgfpathlineto{\pgfqpoint{0.000000in}{-0.027778in}}%
\pgfusepath{stroke,fill}%
}%
\begin{pgfscope}%
\pgfsys@transformshift{3.249066in}{0.440000in}%
\pgfsys@useobject{currentmarker}{}%
\end{pgfscope}%
\end{pgfscope}%
\begin{pgfscope}%
\pgfsetbuttcap%
\pgfsetroundjoin%
\definecolor{currentfill}{rgb}{0.000000,0.000000,0.000000}%
\pgfsetfillcolor{currentfill}%
\pgfsetlinewidth{0.602250pt}%
\definecolor{currentstroke}{rgb}{0.000000,0.000000,0.000000}%
\pgfsetstrokecolor{currentstroke}%
\pgfsetdash{}{0pt}%
\pgfsys@defobject{currentmarker}{\pgfqpoint{0.000000in}{-0.027778in}}{\pgfqpoint{0.000000in}{0.000000in}}{%
\pgfpathmoveto{\pgfqpoint{0.000000in}{0.000000in}}%
\pgfpathlineto{\pgfqpoint{0.000000in}{-0.027778in}}%
\pgfusepath{stroke,fill}%
}%
\begin{pgfscope}%
\pgfsys@transformshift{3.436399in}{0.440000in}%
\pgfsys@useobject{currentmarker}{}%
\end{pgfscope}%
\end{pgfscope}%
\begin{pgfscope}%
\pgfsetbuttcap%
\pgfsetroundjoin%
\definecolor{currentfill}{rgb}{0.000000,0.000000,0.000000}%
\pgfsetfillcolor{currentfill}%
\pgfsetlinewidth{0.602250pt}%
\definecolor{currentstroke}{rgb}{0.000000,0.000000,0.000000}%
\pgfsetstrokecolor{currentstroke}%
\pgfsetdash{}{0pt}%
\pgfsys@defobject{currentmarker}{\pgfqpoint{0.000000in}{-0.027778in}}{\pgfqpoint{0.000000in}{0.000000in}}{%
\pgfpathmoveto{\pgfqpoint{0.000000in}{0.000000in}}%
\pgfpathlineto{\pgfqpoint{0.000000in}{-0.027778in}}%
\pgfusepath{stroke,fill}%
}%
\begin{pgfscope}%
\pgfsys@transformshift{3.569313in}{0.440000in}%
\pgfsys@useobject{currentmarker}{}%
\end{pgfscope}%
\end{pgfscope}%
\begin{pgfscope}%
\pgfsetbuttcap%
\pgfsetroundjoin%
\definecolor{currentfill}{rgb}{0.000000,0.000000,0.000000}%
\pgfsetfillcolor{currentfill}%
\pgfsetlinewidth{0.602250pt}%
\definecolor{currentstroke}{rgb}{0.000000,0.000000,0.000000}%
\pgfsetstrokecolor{currentstroke}%
\pgfsetdash{}{0pt}%
\pgfsys@defobject{currentmarker}{\pgfqpoint{0.000000in}{-0.027778in}}{\pgfqpoint{0.000000in}{0.000000in}}{%
\pgfpathmoveto{\pgfqpoint{0.000000in}{0.000000in}}%
\pgfpathlineto{\pgfqpoint{0.000000in}{-0.027778in}}%
\pgfusepath{stroke,fill}%
}%
\begin{pgfscope}%
\pgfsys@transformshift{3.672410in}{0.440000in}%
\pgfsys@useobject{currentmarker}{}%
\end{pgfscope}%
\end{pgfscope}%
\begin{pgfscope}%
\pgfsetbuttcap%
\pgfsetroundjoin%
\definecolor{currentfill}{rgb}{0.000000,0.000000,0.000000}%
\pgfsetfillcolor{currentfill}%
\pgfsetlinewidth{0.602250pt}%
\definecolor{currentstroke}{rgb}{0.000000,0.000000,0.000000}%
\pgfsetstrokecolor{currentstroke}%
\pgfsetdash{}{0pt}%
\pgfsys@defobject{currentmarker}{\pgfqpoint{0.000000in}{-0.027778in}}{\pgfqpoint{0.000000in}{0.000000in}}{%
\pgfpathmoveto{\pgfqpoint{0.000000in}{0.000000in}}%
\pgfpathlineto{\pgfqpoint{0.000000in}{-0.027778in}}%
\pgfusepath{stroke,fill}%
}%
\begin{pgfscope}%
\pgfsys@transformshift{3.756646in}{0.440000in}%
\pgfsys@useobject{currentmarker}{}%
\end{pgfscope}%
\end{pgfscope}%
\begin{pgfscope}%
\pgfsetbuttcap%
\pgfsetroundjoin%
\definecolor{currentfill}{rgb}{0.000000,0.000000,0.000000}%
\pgfsetfillcolor{currentfill}%
\pgfsetlinewidth{0.602250pt}%
\definecolor{currentstroke}{rgb}{0.000000,0.000000,0.000000}%
\pgfsetstrokecolor{currentstroke}%
\pgfsetdash{}{0pt}%
\pgfsys@defobject{currentmarker}{\pgfqpoint{0.000000in}{-0.027778in}}{\pgfqpoint{0.000000in}{0.000000in}}{%
\pgfpathmoveto{\pgfqpoint{0.000000in}{0.000000in}}%
\pgfpathlineto{\pgfqpoint{0.000000in}{-0.027778in}}%
\pgfusepath{stroke,fill}%
}%
\begin{pgfscope}%
\pgfsys@transformshift{3.827867in}{0.440000in}%
\pgfsys@useobject{currentmarker}{}%
\end{pgfscope}%
\end{pgfscope}%
\begin{pgfscope}%
\pgfsetbuttcap%
\pgfsetroundjoin%
\definecolor{currentfill}{rgb}{0.000000,0.000000,0.000000}%
\pgfsetfillcolor{currentfill}%
\pgfsetlinewidth{0.602250pt}%
\definecolor{currentstroke}{rgb}{0.000000,0.000000,0.000000}%
\pgfsetstrokecolor{currentstroke}%
\pgfsetdash{}{0pt}%
\pgfsys@defobject{currentmarker}{\pgfqpoint{0.000000in}{-0.027778in}}{\pgfqpoint{0.000000in}{0.000000in}}{%
\pgfpathmoveto{\pgfqpoint{0.000000in}{0.000000in}}%
\pgfpathlineto{\pgfqpoint{0.000000in}{-0.027778in}}%
\pgfusepath{stroke,fill}%
}%
\begin{pgfscope}%
\pgfsys@transformshift{3.889561in}{0.440000in}%
\pgfsys@useobject{currentmarker}{}%
\end{pgfscope}%
\end{pgfscope}%
\begin{pgfscope}%
\pgfsetbuttcap%
\pgfsetroundjoin%
\definecolor{currentfill}{rgb}{0.000000,0.000000,0.000000}%
\pgfsetfillcolor{currentfill}%
\pgfsetlinewidth{0.602250pt}%
\definecolor{currentstroke}{rgb}{0.000000,0.000000,0.000000}%
\pgfsetstrokecolor{currentstroke}%
\pgfsetdash{}{0pt}%
\pgfsys@defobject{currentmarker}{\pgfqpoint{0.000000in}{-0.027778in}}{\pgfqpoint{0.000000in}{0.000000in}}{%
\pgfpathmoveto{\pgfqpoint{0.000000in}{0.000000in}}%
\pgfpathlineto{\pgfqpoint{0.000000in}{-0.027778in}}%
\pgfusepath{stroke,fill}%
}%
\begin{pgfscope}%
\pgfsys@transformshift{3.943980in}{0.440000in}%
\pgfsys@useobject{currentmarker}{}%
\end{pgfscope}%
\end{pgfscope}%
\begin{pgfscope}%
\pgfsetbuttcap%
\pgfsetroundjoin%
\definecolor{currentfill}{rgb}{0.000000,0.000000,0.000000}%
\pgfsetfillcolor{currentfill}%
\pgfsetlinewidth{0.602250pt}%
\definecolor{currentstroke}{rgb}{0.000000,0.000000,0.000000}%
\pgfsetstrokecolor{currentstroke}%
\pgfsetdash{}{0pt}%
\pgfsys@defobject{currentmarker}{\pgfqpoint{0.000000in}{-0.027778in}}{\pgfqpoint{0.000000in}{0.000000in}}{%
\pgfpathmoveto{\pgfqpoint{0.000000in}{0.000000in}}%
\pgfpathlineto{\pgfqpoint{0.000000in}{-0.027778in}}%
\pgfusepath{stroke,fill}%
}%
\begin{pgfscope}%
\pgfsys@transformshift{4.312906in}{0.440000in}%
\pgfsys@useobject{currentmarker}{}%
\end{pgfscope}%
\end{pgfscope}%
\begin{pgfscope}%
\pgfsetbuttcap%
\pgfsetroundjoin%
\definecolor{currentfill}{rgb}{0.000000,0.000000,0.000000}%
\pgfsetfillcolor{currentfill}%
\pgfsetlinewidth{0.602250pt}%
\definecolor{currentstroke}{rgb}{0.000000,0.000000,0.000000}%
\pgfsetstrokecolor{currentstroke}%
\pgfsetdash{}{0pt}%
\pgfsys@defobject{currentmarker}{\pgfqpoint{0.000000in}{-0.027778in}}{\pgfqpoint{0.000000in}{0.000000in}}{%
\pgfpathmoveto{\pgfqpoint{0.000000in}{0.000000in}}%
\pgfpathlineto{\pgfqpoint{0.000000in}{-0.027778in}}%
\pgfusepath{stroke,fill}%
}%
\begin{pgfscope}%
\pgfsys@transformshift{4.500239in}{0.440000in}%
\pgfsys@useobject{currentmarker}{}%
\end{pgfscope}%
\end{pgfscope}%
\begin{pgfscope}%
\definecolor{textcolor}{rgb}{0.000000,0.000000,0.000000}%
\pgfsetstrokecolor{textcolor}%
\pgfsetfillcolor{textcolor}%
\pgftext[x=2.562500in,y=0.139368in,,top]{\color{textcolor}\sffamily\fontsize{11.000000}{13.200000}\selectfont \(\displaystyle N_\mathrm{QMC}\)}%
\end{pgfscope}%
\begin{pgfscope}%
\pgfsetbuttcap%
\pgfsetroundjoin%
\definecolor{currentfill}{rgb}{0.000000,0.000000,0.000000}%
\pgfsetfillcolor{currentfill}%
\pgfsetlinewidth{0.803000pt}%
\definecolor{currentstroke}{rgb}{0.000000,0.000000,0.000000}%
\pgfsetstrokecolor{currentstroke}%
\pgfsetdash{}{0pt}%
\pgfsys@defobject{currentmarker}{\pgfqpoint{-0.048611in}{0.000000in}}{\pgfqpoint{0.000000in}{0.000000in}}{%
\pgfpathmoveto{\pgfqpoint{0.000000in}{0.000000in}}%
\pgfpathlineto{\pgfqpoint{-0.048611in}{0.000000in}}%
\pgfusepath{stroke,fill}%
}%
\begin{pgfscope}%
\pgfsys@transformshift{0.625000in}{1.416371in}%
\pgfsys@useobject{currentmarker}{}%
\end{pgfscope}%
\end{pgfscope}%
\begin{pgfscope}%
\definecolor{textcolor}{rgb}{0.000000,0.000000,0.000000}%
\pgfsetstrokecolor{textcolor}%
\pgfsetfillcolor{textcolor}%
\pgftext[x=0.217899in,y=1.358334in,left,base]{\color{textcolor}\sffamily\fontsize{11.000000}{13.200000}\selectfont \(\displaystyle 10^{-4}\)}%
\end{pgfscope}%
\begin{pgfscope}%
\pgfsetbuttcap%
\pgfsetroundjoin%
\definecolor{currentfill}{rgb}{0.000000,0.000000,0.000000}%
\pgfsetfillcolor{currentfill}%
\pgfsetlinewidth{0.803000pt}%
\definecolor{currentstroke}{rgb}{0.000000,0.000000,0.000000}%
\pgfsetstrokecolor{currentstroke}%
\pgfsetdash{}{0pt}%
\pgfsys@defobject{currentmarker}{\pgfqpoint{-0.048611in}{0.000000in}}{\pgfqpoint{0.000000in}{0.000000in}}{%
\pgfpathmoveto{\pgfqpoint{0.000000in}{0.000000in}}%
\pgfpathlineto{\pgfqpoint{-0.048611in}{0.000000in}}%
\pgfusepath{stroke,fill}%
}%
\begin{pgfscope}%
\pgfsys@transformshift{0.625000in}{2.406710in}%
\pgfsys@useobject{currentmarker}{}%
\end{pgfscope}%
\end{pgfscope}%
\begin{pgfscope}%
\definecolor{textcolor}{rgb}{0.000000,0.000000,0.000000}%
\pgfsetstrokecolor{textcolor}%
\pgfsetfillcolor{textcolor}%
\pgftext[x=0.217899in,y=2.348672in,left,base]{\color{textcolor}\sffamily\fontsize{11.000000}{13.200000}\selectfont \(\displaystyle 10^{-3}\)}%
\end{pgfscope}%
\begin{pgfscope}%
\pgfsetbuttcap%
\pgfsetroundjoin%
\definecolor{currentfill}{rgb}{0.000000,0.000000,0.000000}%
\pgfsetfillcolor{currentfill}%
\pgfsetlinewidth{0.803000pt}%
\definecolor{currentstroke}{rgb}{0.000000,0.000000,0.000000}%
\pgfsetstrokecolor{currentstroke}%
\pgfsetdash{}{0pt}%
\pgfsys@defobject{currentmarker}{\pgfqpoint{-0.048611in}{0.000000in}}{\pgfqpoint{0.000000in}{0.000000in}}{%
\pgfpathmoveto{\pgfqpoint{0.000000in}{0.000000in}}%
\pgfpathlineto{\pgfqpoint{-0.048611in}{0.000000in}}%
\pgfusepath{stroke,fill}%
}%
\begin{pgfscope}%
\pgfsys@transformshift{0.625000in}{3.397049in}%
\pgfsys@useobject{currentmarker}{}%
\end{pgfscope}%
\end{pgfscope}%
\begin{pgfscope}%
\definecolor{textcolor}{rgb}{0.000000,0.000000,0.000000}%
\pgfsetstrokecolor{textcolor}%
\pgfsetfillcolor{textcolor}%
\pgftext[x=0.217899in,y=3.339011in,left,base]{\color{textcolor}\sffamily\fontsize{11.000000}{13.200000}\selectfont \(\displaystyle 10^{-2}\)}%
\end{pgfscope}%
\begin{pgfscope}%
\pgfsetbuttcap%
\pgfsetroundjoin%
\definecolor{currentfill}{rgb}{0.000000,0.000000,0.000000}%
\pgfsetfillcolor{currentfill}%
\pgfsetlinewidth{0.602250pt}%
\definecolor{currentstroke}{rgb}{0.000000,0.000000,0.000000}%
\pgfsetstrokecolor{currentstroke}%
\pgfsetdash{}{0pt}%
\pgfsys@defobject{currentmarker}{\pgfqpoint{-0.027778in}{0.000000in}}{\pgfqpoint{0.000000in}{0.000000in}}{%
\pgfpathmoveto{\pgfqpoint{0.000000in}{0.000000in}}%
\pgfpathlineto{\pgfqpoint{-0.027778in}{0.000000in}}%
\pgfusepath{stroke,fill}%
}%
\begin{pgfscope}%
\pgfsys@transformshift{0.625000in}{0.724154in}%
\pgfsys@useobject{currentmarker}{}%
\end{pgfscope}%
\end{pgfscope}%
\begin{pgfscope}%
\pgfsetbuttcap%
\pgfsetroundjoin%
\definecolor{currentfill}{rgb}{0.000000,0.000000,0.000000}%
\pgfsetfillcolor{currentfill}%
\pgfsetlinewidth{0.602250pt}%
\definecolor{currentstroke}{rgb}{0.000000,0.000000,0.000000}%
\pgfsetstrokecolor{currentstroke}%
\pgfsetdash{}{0pt}%
\pgfsys@defobject{currentmarker}{\pgfqpoint{-0.027778in}{0.000000in}}{\pgfqpoint{0.000000in}{0.000000in}}{%
\pgfpathmoveto{\pgfqpoint{0.000000in}{0.000000in}}%
\pgfpathlineto{\pgfqpoint{-0.027778in}{0.000000in}}%
\pgfusepath{stroke,fill}%
}%
\begin{pgfscope}%
\pgfsys@transformshift{0.625000in}{0.898544in}%
\pgfsys@useobject{currentmarker}{}%
\end{pgfscope}%
\end{pgfscope}%
\begin{pgfscope}%
\pgfsetbuttcap%
\pgfsetroundjoin%
\definecolor{currentfill}{rgb}{0.000000,0.000000,0.000000}%
\pgfsetfillcolor{currentfill}%
\pgfsetlinewidth{0.602250pt}%
\definecolor{currentstroke}{rgb}{0.000000,0.000000,0.000000}%
\pgfsetstrokecolor{currentstroke}%
\pgfsetdash{}{0pt}%
\pgfsys@defobject{currentmarker}{\pgfqpoint{-0.027778in}{0.000000in}}{\pgfqpoint{0.000000in}{0.000000in}}{%
\pgfpathmoveto{\pgfqpoint{0.000000in}{0.000000in}}%
\pgfpathlineto{\pgfqpoint{-0.027778in}{0.000000in}}%
\pgfusepath{stroke,fill}%
}%
\begin{pgfscope}%
\pgfsys@transformshift{0.625000in}{1.022276in}%
\pgfsys@useobject{currentmarker}{}%
\end{pgfscope}%
\end{pgfscope}%
\begin{pgfscope}%
\pgfsetbuttcap%
\pgfsetroundjoin%
\definecolor{currentfill}{rgb}{0.000000,0.000000,0.000000}%
\pgfsetfillcolor{currentfill}%
\pgfsetlinewidth{0.602250pt}%
\definecolor{currentstroke}{rgb}{0.000000,0.000000,0.000000}%
\pgfsetstrokecolor{currentstroke}%
\pgfsetdash{}{0pt}%
\pgfsys@defobject{currentmarker}{\pgfqpoint{-0.027778in}{0.000000in}}{\pgfqpoint{0.000000in}{0.000000in}}{%
\pgfpathmoveto{\pgfqpoint{0.000000in}{0.000000in}}%
\pgfpathlineto{\pgfqpoint{-0.027778in}{0.000000in}}%
\pgfusepath{stroke,fill}%
}%
\begin{pgfscope}%
\pgfsys@transformshift{0.625000in}{1.118250in}%
\pgfsys@useobject{currentmarker}{}%
\end{pgfscope}%
\end{pgfscope}%
\begin{pgfscope}%
\pgfsetbuttcap%
\pgfsetroundjoin%
\definecolor{currentfill}{rgb}{0.000000,0.000000,0.000000}%
\pgfsetfillcolor{currentfill}%
\pgfsetlinewidth{0.602250pt}%
\definecolor{currentstroke}{rgb}{0.000000,0.000000,0.000000}%
\pgfsetstrokecolor{currentstroke}%
\pgfsetdash{}{0pt}%
\pgfsys@defobject{currentmarker}{\pgfqpoint{-0.027778in}{0.000000in}}{\pgfqpoint{0.000000in}{0.000000in}}{%
\pgfpathmoveto{\pgfqpoint{0.000000in}{0.000000in}}%
\pgfpathlineto{\pgfqpoint{-0.027778in}{0.000000in}}%
\pgfusepath{stroke,fill}%
}%
\begin{pgfscope}%
\pgfsys@transformshift{0.625000in}{1.196666in}%
\pgfsys@useobject{currentmarker}{}%
\end{pgfscope}%
\end{pgfscope}%
\begin{pgfscope}%
\pgfsetbuttcap%
\pgfsetroundjoin%
\definecolor{currentfill}{rgb}{0.000000,0.000000,0.000000}%
\pgfsetfillcolor{currentfill}%
\pgfsetlinewidth{0.602250pt}%
\definecolor{currentstroke}{rgb}{0.000000,0.000000,0.000000}%
\pgfsetstrokecolor{currentstroke}%
\pgfsetdash{}{0pt}%
\pgfsys@defobject{currentmarker}{\pgfqpoint{-0.027778in}{0.000000in}}{\pgfqpoint{0.000000in}{0.000000in}}{%
\pgfpathmoveto{\pgfqpoint{0.000000in}{0.000000in}}%
\pgfpathlineto{\pgfqpoint{-0.027778in}{0.000000in}}%
\pgfusepath{stroke,fill}%
}%
\begin{pgfscope}%
\pgfsys@transformshift{0.625000in}{1.262966in}%
\pgfsys@useobject{currentmarker}{}%
\end{pgfscope}%
\end{pgfscope}%
\begin{pgfscope}%
\pgfsetbuttcap%
\pgfsetroundjoin%
\definecolor{currentfill}{rgb}{0.000000,0.000000,0.000000}%
\pgfsetfillcolor{currentfill}%
\pgfsetlinewidth{0.602250pt}%
\definecolor{currentstroke}{rgb}{0.000000,0.000000,0.000000}%
\pgfsetstrokecolor{currentstroke}%
\pgfsetdash{}{0pt}%
\pgfsys@defobject{currentmarker}{\pgfqpoint{-0.027778in}{0.000000in}}{\pgfqpoint{0.000000in}{0.000000in}}{%
\pgfpathmoveto{\pgfqpoint{0.000000in}{0.000000in}}%
\pgfpathlineto{\pgfqpoint{-0.027778in}{0.000000in}}%
\pgfusepath{stroke,fill}%
}%
\begin{pgfscope}%
\pgfsys@transformshift{0.625000in}{1.320398in}%
\pgfsys@useobject{currentmarker}{}%
\end{pgfscope}%
\end{pgfscope}%
\begin{pgfscope}%
\pgfsetbuttcap%
\pgfsetroundjoin%
\definecolor{currentfill}{rgb}{0.000000,0.000000,0.000000}%
\pgfsetfillcolor{currentfill}%
\pgfsetlinewidth{0.602250pt}%
\definecolor{currentstroke}{rgb}{0.000000,0.000000,0.000000}%
\pgfsetstrokecolor{currentstroke}%
\pgfsetdash{}{0pt}%
\pgfsys@defobject{currentmarker}{\pgfqpoint{-0.027778in}{0.000000in}}{\pgfqpoint{0.000000in}{0.000000in}}{%
\pgfpathmoveto{\pgfqpoint{0.000000in}{0.000000in}}%
\pgfpathlineto{\pgfqpoint{-0.027778in}{0.000000in}}%
\pgfusepath{stroke,fill}%
}%
\begin{pgfscope}%
\pgfsys@transformshift{0.625000in}{1.371056in}%
\pgfsys@useobject{currentmarker}{}%
\end{pgfscope}%
\end{pgfscope}%
\begin{pgfscope}%
\pgfsetbuttcap%
\pgfsetroundjoin%
\definecolor{currentfill}{rgb}{0.000000,0.000000,0.000000}%
\pgfsetfillcolor{currentfill}%
\pgfsetlinewidth{0.602250pt}%
\definecolor{currentstroke}{rgb}{0.000000,0.000000,0.000000}%
\pgfsetstrokecolor{currentstroke}%
\pgfsetdash{}{0pt}%
\pgfsys@defobject{currentmarker}{\pgfqpoint{-0.027778in}{0.000000in}}{\pgfqpoint{0.000000in}{0.000000in}}{%
\pgfpathmoveto{\pgfqpoint{0.000000in}{0.000000in}}%
\pgfpathlineto{\pgfqpoint{-0.027778in}{0.000000in}}%
\pgfusepath{stroke,fill}%
}%
\begin{pgfscope}%
\pgfsys@transformshift{0.625000in}{1.714493in}%
\pgfsys@useobject{currentmarker}{}%
\end{pgfscope}%
\end{pgfscope}%
\begin{pgfscope}%
\pgfsetbuttcap%
\pgfsetroundjoin%
\definecolor{currentfill}{rgb}{0.000000,0.000000,0.000000}%
\pgfsetfillcolor{currentfill}%
\pgfsetlinewidth{0.602250pt}%
\definecolor{currentstroke}{rgb}{0.000000,0.000000,0.000000}%
\pgfsetstrokecolor{currentstroke}%
\pgfsetdash{}{0pt}%
\pgfsys@defobject{currentmarker}{\pgfqpoint{-0.027778in}{0.000000in}}{\pgfqpoint{0.000000in}{0.000000in}}{%
\pgfpathmoveto{\pgfqpoint{0.000000in}{0.000000in}}%
\pgfpathlineto{\pgfqpoint{-0.027778in}{0.000000in}}%
\pgfusepath{stroke,fill}%
}%
\begin{pgfscope}%
\pgfsys@transformshift{0.625000in}{1.888883in}%
\pgfsys@useobject{currentmarker}{}%
\end{pgfscope}%
\end{pgfscope}%
\begin{pgfscope}%
\pgfsetbuttcap%
\pgfsetroundjoin%
\definecolor{currentfill}{rgb}{0.000000,0.000000,0.000000}%
\pgfsetfillcolor{currentfill}%
\pgfsetlinewidth{0.602250pt}%
\definecolor{currentstroke}{rgb}{0.000000,0.000000,0.000000}%
\pgfsetstrokecolor{currentstroke}%
\pgfsetdash{}{0pt}%
\pgfsys@defobject{currentmarker}{\pgfqpoint{-0.027778in}{0.000000in}}{\pgfqpoint{0.000000in}{0.000000in}}{%
\pgfpathmoveto{\pgfqpoint{0.000000in}{0.000000in}}%
\pgfpathlineto{\pgfqpoint{-0.027778in}{0.000000in}}%
\pgfusepath{stroke,fill}%
}%
\begin{pgfscope}%
\pgfsys@transformshift{0.625000in}{2.012615in}%
\pgfsys@useobject{currentmarker}{}%
\end{pgfscope}%
\end{pgfscope}%
\begin{pgfscope}%
\pgfsetbuttcap%
\pgfsetroundjoin%
\definecolor{currentfill}{rgb}{0.000000,0.000000,0.000000}%
\pgfsetfillcolor{currentfill}%
\pgfsetlinewidth{0.602250pt}%
\definecolor{currentstroke}{rgb}{0.000000,0.000000,0.000000}%
\pgfsetstrokecolor{currentstroke}%
\pgfsetdash{}{0pt}%
\pgfsys@defobject{currentmarker}{\pgfqpoint{-0.027778in}{0.000000in}}{\pgfqpoint{0.000000in}{0.000000in}}{%
\pgfpathmoveto{\pgfqpoint{0.000000in}{0.000000in}}%
\pgfpathlineto{\pgfqpoint{-0.027778in}{0.000000in}}%
\pgfusepath{stroke,fill}%
}%
\begin{pgfscope}%
\pgfsys@transformshift{0.625000in}{2.108588in}%
\pgfsys@useobject{currentmarker}{}%
\end{pgfscope}%
\end{pgfscope}%
\begin{pgfscope}%
\pgfsetbuttcap%
\pgfsetroundjoin%
\definecolor{currentfill}{rgb}{0.000000,0.000000,0.000000}%
\pgfsetfillcolor{currentfill}%
\pgfsetlinewidth{0.602250pt}%
\definecolor{currentstroke}{rgb}{0.000000,0.000000,0.000000}%
\pgfsetstrokecolor{currentstroke}%
\pgfsetdash{}{0pt}%
\pgfsys@defobject{currentmarker}{\pgfqpoint{-0.027778in}{0.000000in}}{\pgfqpoint{0.000000in}{0.000000in}}{%
\pgfpathmoveto{\pgfqpoint{0.000000in}{0.000000in}}%
\pgfpathlineto{\pgfqpoint{-0.027778in}{0.000000in}}%
\pgfusepath{stroke,fill}%
}%
\begin{pgfscope}%
\pgfsys@transformshift{0.625000in}{2.187005in}%
\pgfsys@useobject{currentmarker}{}%
\end{pgfscope}%
\end{pgfscope}%
\begin{pgfscope}%
\pgfsetbuttcap%
\pgfsetroundjoin%
\definecolor{currentfill}{rgb}{0.000000,0.000000,0.000000}%
\pgfsetfillcolor{currentfill}%
\pgfsetlinewidth{0.602250pt}%
\definecolor{currentstroke}{rgb}{0.000000,0.000000,0.000000}%
\pgfsetstrokecolor{currentstroke}%
\pgfsetdash{}{0pt}%
\pgfsys@defobject{currentmarker}{\pgfqpoint{-0.027778in}{0.000000in}}{\pgfqpoint{0.000000in}{0.000000in}}{%
\pgfpathmoveto{\pgfqpoint{0.000000in}{0.000000in}}%
\pgfpathlineto{\pgfqpoint{-0.027778in}{0.000000in}}%
\pgfusepath{stroke,fill}%
}%
\begin{pgfscope}%
\pgfsys@transformshift{0.625000in}{2.253305in}%
\pgfsys@useobject{currentmarker}{}%
\end{pgfscope}%
\end{pgfscope}%
\begin{pgfscope}%
\pgfsetbuttcap%
\pgfsetroundjoin%
\definecolor{currentfill}{rgb}{0.000000,0.000000,0.000000}%
\pgfsetfillcolor{currentfill}%
\pgfsetlinewidth{0.602250pt}%
\definecolor{currentstroke}{rgb}{0.000000,0.000000,0.000000}%
\pgfsetstrokecolor{currentstroke}%
\pgfsetdash{}{0pt}%
\pgfsys@defobject{currentmarker}{\pgfqpoint{-0.027778in}{0.000000in}}{\pgfqpoint{0.000000in}{0.000000in}}{%
\pgfpathmoveto{\pgfqpoint{0.000000in}{0.000000in}}%
\pgfpathlineto{\pgfqpoint{-0.027778in}{0.000000in}}%
\pgfusepath{stroke,fill}%
}%
\begin{pgfscope}%
\pgfsys@transformshift{0.625000in}{2.310736in}%
\pgfsys@useobject{currentmarker}{}%
\end{pgfscope}%
\end{pgfscope}%
\begin{pgfscope}%
\pgfsetbuttcap%
\pgfsetroundjoin%
\definecolor{currentfill}{rgb}{0.000000,0.000000,0.000000}%
\pgfsetfillcolor{currentfill}%
\pgfsetlinewidth{0.602250pt}%
\definecolor{currentstroke}{rgb}{0.000000,0.000000,0.000000}%
\pgfsetstrokecolor{currentstroke}%
\pgfsetdash{}{0pt}%
\pgfsys@defobject{currentmarker}{\pgfqpoint{-0.027778in}{0.000000in}}{\pgfqpoint{0.000000in}{0.000000in}}{%
\pgfpathmoveto{\pgfqpoint{0.000000in}{0.000000in}}%
\pgfpathlineto{\pgfqpoint{-0.027778in}{0.000000in}}%
\pgfusepath{stroke,fill}%
}%
\begin{pgfscope}%
\pgfsys@transformshift{0.625000in}{2.361395in}%
\pgfsys@useobject{currentmarker}{}%
\end{pgfscope}%
\end{pgfscope}%
\begin{pgfscope}%
\pgfsetbuttcap%
\pgfsetroundjoin%
\definecolor{currentfill}{rgb}{0.000000,0.000000,0.000000}%
\pgfsetfillcolor{currentfill}%
\pgfsetlinewidth{0.602250pt}%
\definecolor{currentstroke}{rgb}{0.000000,0.000000,0.000000}%
\pgfsetstrokecolor{currentstroke}%
\pgfsetdash{}{0pt}%
\pgfsys@defobject{currentmarker}{\pgfqpoint{-0.027778in}{0.000000in}}{\pgfqpoint{0.000000in}{0.000000in}}{%
\pgfpathmoveto{\pgfqpoint{0.000000in}{0.000000in}}%
\pgfpathlineto{\pgfqpoint{-0.027778in}{0.000000in}}%
\pgfusepath{stroke,fill}%
}%
\begin{pgfscope}%
\pgfsys@transformshift{0.625000in}{2.704832in}%
\pgfsys@useobject{currentmarker}{}%
\end{pgfscope}%
\end{pgfscope}%
\begin{pgfscope}%
\pgfsetbuttcap%
\pgfsetroundjoin%
\definecolor{currentfill}{rgb}{0.000000,0.000000,0.000000}%
\pgfsetfillcolor{currentfill}%
\pgfsetlinewidth{0.602250pt}%
\definecolor{currentstroke}{rgb}{0.000000,0.000000,0.000000}%
\pgfsetstrokecolor{currentstroke}%
\pgfsetdash{}{0pt}%
\pgfsys@defobject{currentmarker}{\pgfqpoint{-0.027778in}{0.000000in}}{\pgfqpoint{0.000000in}{0.000000in}}{%
\pgfpathmoveto{\pgfqpoint{0.000000in}{0.000000in}}%
\pgfpathlineto{\pgfqpoint{-0.027778in}{0.000000in}}%
\pgfusepath{stroke,fill}%
}%
\begin{pgfscope}%
\pgfsys@transformshift{0.625000in}{2.879222in}%
\pgfsys@useobject{currentmarker}{}%
\end{pgfscope}%
\end{pgfscope}%
\begin{pgfscope}%
\pgfsetbuttcap%
\pgfsetroundjoin%
\definecolor{currentfill}{rgb}{0.000000,0.000000,0.000000}%
\pgfsetfillcolor{currentfill}%
\pgfsetlinewidth{0.602250pt}%
\definecolor{currentstroke}{rgb}{0.000000,0.000000,0.000000}%
\pgfsetstrokecolor{currentstroke}%
\pgfsetdash{}{0pt}%
\pgfsys@defobject{currentmarker}{\pgfqpoint{-0.027778in}{0.000000in}}{\pgfqpoint{0.000000in}{0.000000in}}{%
\pgfpathmoveto{\pgfqpoint{0.000000in}{0.000000in}}%
\pgfpathlineto{\pgfqpoint{-0.027778in}{0.000000in}}%
\pgfusepath{stroke,fill}%
}%
\begin{pgfscope}%
\pgfsys@transformshift{0.625000in}{3.002953in}%
\pgfsys@useobject{currentmarker}{}%
\end{pgfscope}%
\end{pgfscope}%
\begin{pgfscope}%
\pgfsetbuttcap%
\pgfsetroundjoin%
\definecolor{currentfill}{rgb}{0.000000,0.000000,0.000000}%
\pgfsetfillcolor{currentfill}%
\pgfsetlinewidth{0.602250pt}%
\definecolor{currentstroke}{rgb}{0.000000,0.000000,0.000000}%
\pgfsetstrokecolor{currentstroke}%
\pgfsetdash{}{0pt}%
\pgfsys@defobject{currentmarker}{\pgfqpoint{-0.027778in}{0.000000in}}{\pgfqpoint{0.000000in}{0.000000in}}{%
\pgfpathmoveto{\pgfqpoint{0.000000in}{0.000000in}}%
\pgfpathlineto{\pgfqpoint{-0.027778in}{0.000000in}}%
\pgfusepath{stroke,fill}%
}%
\begin{pgfscope}%
\pgfsys@transformshift{0.625000in}{3.098927in}%
\pgfsys@useobject{currentmarker}{}%
\end{pgfscope}%
\end{pgfscope}%
\begin{pgfscope}%
\pgfsetbuttcap%
\pgfsetroundjoin%
\definecolor{currentfill}{rgb}{0.000000,0.000000,0.000000}%
\pgfsetfillcolor{currentfill}%
\pgfsetlinewidth{0.602250pt}%
\definecolor{currentstroke}{rgb}{0.000000,0.000000,0.000000}%
\pgfsetstrokecolor{currentstroke}%
\pgfsetdash{}{0pt}%
\pgfsys@defobject{currentmarker}{\pgfqpoint{-0.027778in}{0.000000in}}{\pgfqpoint{0.000000in}{0.000000in}}{%
\pgfpathmoveto{\pgfqpoint{0.000000in}{0.000000in}}%
\pgfpathlineto{\pgfqpoint{-0.027778in}{0.000000in}}%
\pgfusepath{stroke,fill}%
}%
\begin{pgfscope}%
\pgfsys@transformshift{0.625000in}{3.177343in}%
\pgfsys@useobject{currentmarker}{}%
\end{pgfscope}%
\end{pgfscope}%
\begin{pgfscope}%
\pgfsetbuttcap%
\pgfsetroundjoin%
\definecolor{currentfill}{rgb}{0.000000,0.000000,0.000000}%
\pgfsetfillcolor{currentfill}%
\pgfsetlinewidth{0.602250pt}%
\definecolor{currentstroke}{rgb}{0.000000,0.000000,0.000000}%
\pgfsetstrokecolor{currentstroke}%
\pgfsetdash{}{0pt}%
\pgfsys@defobject{currentmarker}{\pgfqpoint{-0.027778in}{0.000000in}}{\pgfqpoint{0.000000in}{0.000000in}}{%
\pgfpathmoveto{\pgfqpoint{0.000000in}{0.000000in}}%
\pgfpathlineto{\pgfqpoint{-0.027778in}{0.000000in}}%
\pgfusepath{stroke,fill}%
}%
\begin{pgfscope}%
\pgfsys@transformshift{0.625000in}{3.243643in}%
\pgfsys@useobject{currentmarker}{}%
\end{pgfscope}%
\end{pgfscope}%
\begin{pgfscope}%
\pgfsetbuttcap%
\pgfsetroundjoin%
\definecolor{currentfill}{rgb}{0.000000,0.000000,0.000000}%
\pgfsetfillcolor{currentfill}%
\pgfsetlinewidth{0.602250pt}%
\definecolor{currentstroke}{rgb}{0.000000,0.000000,0.000000}%
\pgfsetstrokecolor{currentstroke}%
\pgfsetdash{}{0pt}%
\pgfsys@defobject{currentmarker}{\pgfqpoint{-0.027778in}{0.000000in}}{\pgfqpoint{0.000000in}{0.000000in}}{%
\pgfpathmoveto{\pgfqpoint{0.000000in}{0.000000in}}%
\pgfpathlineto{\pgfqpoint{-0.027778in}{0.000000in}}%
\pgfusepath{stroke,fill}%
}%
\begin{pgfscope}%
\pgfsys@transformshift{0.625000in}{3.301075in}%
\pgfsys@useobject{currentmarker}{}%
\end{pgfscope}%
\end{pgfscope}%
\begin{pgfscope}%
\pgfsetbuttcap%
\pgfsetroundjoin%
\definecolor{currentfill}{rgb}{0.000000,0.000000,0.000000}%
\pgfsetfillcolor{currentfill}%
\pgfsetlinewidth{0.602250pt}%
\definecolor{currentstroke}{rgb}{0.000000,0.000000,0.000000}%
\pgfsetstrokecolor{currentstroke}%
\pgfsetdash{}{0pt}%
\pgfsys@defobject{currentmarker}{\pgfqpoint{-0.027778in}{0.000000in}}{\pgfqpoint{0.000000in}{0.000000in}}{%
\pgfpathmoveto{\pgfqpoint{0.000000in}{0.000000in}}%
\pgfpathlineto{\pgfqpoint{-0.027778in}{0.000000in}}%
\pgfusepath{stroke,fill}%
}%
\begin{pgfscope}%
\pgfsys@transformshift{0.625000in}{3.351733in}%
\pgfsys@useobject{currentmarker}{}%
\end{pgfscope}%
\end{pgfscope}%
\begin{pgfscope}%
\definecolor{textcolor}{rgb}{0.000000,0.000000,0.000000}%
\pgfsetstrokecolor{textcolor}%
\pgfsetfillcolor{textcolor}%
\pgftext[x=0.162343in,y=1.980000in,,bottom,rotate=90.000000]{\color{textcolor}\sffamily\fontsize{11.000000}{13.200000}\selectfont \textrm{QMC Error}}%
\end{pgfscope}%
\begin{pgfscope}%
\pgfpathrectangle{\pgfqpoint{0.625000in}{0.440000in}}{\pgfqpoint{3.875000in}{3.080000in}}%
\pgfusepath{clip}%
\pgfsetbuttcap%
\pgfsetroundjoin%
\definecolor{currentfill}{rgb}{0.000000,0.000000,0.000000}%
\pgfsetfillcolor{currentfill}%
\pgfsetlinewidth{1.003750pt}%
\definecolor{currentstroke}{rgb}{0.000000,0.000000,0.000000}%
\pgfsetstrokecolor{currentstroke}%
\pgfsetdash{}{0pt}%
\pgfsys@defobject{currentmarker}{\pgfqpoint{-0.041667in}{-0.041667in}}{\pgfqpoint{0.041667in}{0.041667in}}{%
\pgfpathmoveto{\pgfqpoint{0.000000in}{-0.041667in}}%
\pgfpathcurveto{\pgfqpoint{0.011050in}{-0.041667in}}{\pgfqpoint{0.021649in}{-0.037276in}}{\pgfqpoint{0.029463in}{-0.029463in}}%
\pgfpathcurveto{\pgfqpoint{0.037276in}{-0.021649in}}{\pgfqpoint{0.041667in}{-0.011050in}}{\pgfqpoint{0.041667in}{0.000000in}}%
\pgfpathcurveto{\pgfqpoint{0.041667in}{0.011050in}}{\pgfqpoint{0.037276in}{0.021649in}}{\pgfqpoint{0.029463in}{0.029463in}}%
\pgfpathcurveto{\pgfqpoint{0.021649in}{0.037276in}}{\pgfqpoint{0.011050in}{0.041667in}}{\pgfqpoint{0.000000in}{0.041667in}}%
\pgfpathcurveto{\pgfqpoint{-0.011050in}{0.041667in}}{\pgfqpoint{-0.021649in}{0.037276in}}{\pgfqpoint{-0.029463in}{0.029463in}}%
\pgfpathcurveto{\pgfqpoint{-0.037276in}{0.021649in}}{\pgfqpoint{-0.041667in}{0.011050in}}{\pgfqpoint{-0.041667in}{0.000000in}}%
\pgfpathcurveto{\pgfqpoint{-0.041667in}{-0.011050in}}{\pgfqpoint{-0.037276in}{-0.021649in}}{\pgfqpoint{-0.029463in}{-0.029463in}}%
\pgfpathcurveto{\pgfqpoint{-0.021649in}{-0.037276in}}{\pgfqpoint{-0.011050in}{-0.041667in}}{\pgfqpoint{0.000000in}{-0.041667in}}%
\pgfpathclose%
\pgfusepath{stroke,fill}%
}%
\begin{pgfscope}%
\pgfsys@transformshift{0.801136in}{3.380000in}%
\pgfsys@useobject{currentmarker}{}%
\end{pgfscope}%
\begin{pgfscope}%
\pgfsys@transformshift{1.121384in}{2.966975in}%
\pgfsys@useobject{currentmarker}{}%
\end{pgfscope}%
\begin{pgfscope}%
\pgfsys@transformshift{1.441632in}{2.726819in}%
\pgfsys@useobject{currentmarker}{}%
\end{pgfscope}%
\begin{pgfscope}%
\pgfsys@transformshift{1.761880in}{2.510081in}%
\pgfsys@useobject{currentmarker}{}%
\end{pgfscope}%
\begin{pgfscope}%
\pgfsys@transformshift{2.082128in}{2.271898in}%
\pgfsys@useobject{currentmarker}{}%
\end{pgfscope}%
\begin{pgfscope}%
\pgfsys@transformshift{2.402376in}{2.099190in}%
\pgfsys@useobject{currentmarker}{}%
\end{pgfscope}%
\begin{pgfscope}%
\pgfsys@transformshift{2.722624in}{2.031790in}%
\pgfsys@useobject{currentmarker}{}%
\end{pgfscope}%
\begin{pgfscope}%
\pgfsys@transformshift{3.042872in}{1.872007in}%
\pgfsys@useobject{currentmarker}{}%
\end{pgfscope}%
\begin{pgfscope}%
\pgfsys@transformshift{3.363120in}{1.281635in}%
\pgfsys@useobject{currentmarker}{}%
\end{pgfscope}%
\begin{pgfscope}%
\pgfsys@transformshift{3.683368in}{0.952042in}%
\pgfsys@useobject{currentmarker}{}%
\end{pgfscope}%
\begin{pgfscope}%
\pgfsys@transformshift{4.003616in}{0.753924in}%
\pgfsys@useobject{currentmarker}{}%
\end{pgfscope}%
\begin{pgfscope}%
\pgfsys@transformshift{4.323864in}{0.580000in}%
\pgfsys@useobject{currentmarker}{}%
\end{pgfscope}%
\end{pgfscope}%
\begin{pgfscope}%
\pgfpathrectangle{\pgfqpoint{0.625000in}{0.440000in}}{\pgfqpoint{3.875000in}{3.080000in}}%
\pgfusepath{clip}%
\pgfsetbuttcap%
\pgfsetroundjoin%
\pgfsetlinewidth{1.505625pt}%
\definecolor{currentstroke}{rgb}{0.000000,0.000000,0.000000}%
\pgfsetstrokecolor{currentstroke}%
\pgfsetdash{{5.550000pt}{2.400000pt}}{0.000000pt}%
\pgfpathmoveto{\pgfqpoint{0.801136in}{3.308931in}}%
\pgfpathlineto{\pgfqpoint{4.323864in}{0.595462in}}%
\pgfusepath{stroke}%
\end{pgfscope}%
\begin{pgfscope}%
\pgfsetrectcap%
\pgfsetmiterjoin%
\pgfsetlinewidth{0.803000pt}%
\definecolor{currentstroke}{rgb}{0.000000,0.000000,0.000000}%
\pgfsetstrokecolor{currentstroke}%
\pgfsetdash{}{0pt}%
\pgfpathmoveto{\pgfqpoint{0.625000in}{0.440000in}}%
\pgfpathlineto{\pgfqpoint{0.625000in}{3.520000in}}%
\pgfusepath{stroke}%
\end{pgfscope}%
\begin{pgfscope}%
\pgfsetrectcap%
\pgfsetmiterjoin%
\pgfsetlinewidth{0.000000pt}%
\definecolor{currentstroke}{rgb}{0.000000,0.000000,0.000000}%
\pgfsetstrokecolor{currentstroke}%
\pgfsetstrokeopacity{0.000000}%
\pgfsetdash{}{0pt}%
\pgfpathmoveto{\pgfqpoint{4.500000in}{0.440000in}}%
\pgfpathlineto{\pgfqpoint{4.500000in}{3.520000in}}%
\pgfusepath{}%
\end{pgfscope}%
\begin{pgfscope}%
\pgfsetrectcap%
\pgfsetmiterjoin%
\pgfsetlinewidth{0.803000pt}%
\definecolor{currentstroke}{rgb}{0.000000,0.000000,0.000000}%
\pgfsetstrokecolor{currentstroke}%
\pgfsetdash{}{0pt}%
\pgfpathmoveto{\pgfqpoint{0.625000in}{0.440000in}}%
\pgfpathlineto{\pgfqpoint{4.500000in}{0.440000in}}%
\pgfusepath{stroke}%
\end{pgfscope}%
\begin{pgfscope}%
\pgfsetrectcap%
\pgfsetmiterjoin%
\pgfsetlinewidth{0.000000pt}%
\definecolor{currentstroke}{rgb}{0.000000,0.000000,0.000000}%
\pgfsetstrokecolor{currentstroke}%
\pgfsetstrokeopacity{0.000000}%
\pgfsetdash{}{0pt}%
\pgfpathmoveto{\pgfqpoint{0.625000in}{3.520000in}}%
\pgfpathlineto{\pgfqpoint{4.500000in}{3.520000in}}%
\pgfusepath{}%
\end{pgfscope}%
\begin{pgfscope}%
\pgfsetbuttcap%
\pgfsetmiterjoin%
\definecolor{currentfill}{rgb}{1.000000,1.000000,1.000000}%
\pgfsetfillcolor{currentfill}%
\pgfsetfillopacity{0.800000}%
\pgfsetlinewidth{1.003750pt}%
\definecolor{currentstroke}{rgb}{0.800000,0.800000,0.800000}%
\pgfsetstrokecolor{currentstroke}%
\pgfsetstrokeopacity{0.800000}%
\pgfsetdash{}{0pt}%
\pgfpathmoveto{\pgfqpoint{2.916233in}{3.119488in}}%
\pgfpathlineto{\pgfqpoint{4.393056in}{3.119488in}}%
\pgfpathquadraticcurveto{\pgfqpoint{4.423611in}{3.119488in}}{\pgfqpoint{4.423611in}{3.150043in}}%
\pgfpathlineto{\pgfqpoint{4.423611in}{3.413056in}}%
\pgfpathquadraticcurveto{\pgfqpoint{4.423611in}{3.443611in}}{\pgfqpoint{4.393056in}{3.443611in}}%
\pgfpathlineto{\pgfqpoint{2.916233in}{3.443611in}}%
\pgfpathquadraticcurveto{\pgfqpoint{2.885678in}{3.443611in}}{\pgfqpoint{2.885678in}{3.413056in}}%
\pgfpathlineto{\pgfqpoint{2.885678in}{3.150043in}}%
\pgfpathquadraticcurveto{\pgfqpoint{2.885678in}{3.119488in}}{\pgfqpoint{2.916233in}{3.119488in}}%
\pgfpathclose%
\pgfusepath{stroke,fill}%
\end{pgfscope}%
\begin{pgfscope}%
\pgfsetbuttcap%
\pgfsetroundjoin%
\pgfsetlinewidth{1.505625pt}%
\definecolor{currentstroke}{rgb}{0.000000,0.000000,0.000000}%
\pgfsetstrokecolor{currentstroke}%
\pgfsetdash{{5.550000pt}{2.400000pt}}{0.000000pt}%
\pgfpathmoveto{\pgfqpoint{2.946789in}{3.300408in}}%
\pgfpathlineto{\pgfqpoint{3.252344in}{3.300408in}}%
\pgfusepath{stroke}%
\end{pgfscope}%
\begin{pgfscope}%
\definecolor{textcolor}{rgb}{0.000000,0.000000,0.000000}%
\pgfsetstrokecolor{textcolor}%
\pgfsetfillcolor{textcolor}%
\pgftext[x=3.374567in,y=3.246936in,left,base]{\color{textcolor}\sffamily\fontsize{11.000000}{13.200000}\selectfont \(\displaystyle 0.0081N_{\mathrm{QMC}}^{-0.8274}\)}%
\end{pgfscope}%
\end{pgfpicture}%
\makeatother%
\endgroup%

\caption[Quasi-Monte-Carlo error, for $Q=u(\bzero)$ and $k=20$.]{Quasi-Monte-Carlo error with increasing number of Quasi-Monte-Carlo points, for $Q=u(\bzero)$ and $k=20$. \label{fig:qmcorigin20}}
\end{figure}
\begin{figure}[h]
    \centering

    %% Creator: Matplotlib, PGF backend
%%
%% To include the figure in your LaTeX document, write
%%   \input{<filename>.pgf}
%%
%% Make sure the required packages are loaded in your preamble
%%   \usepackage{pgf}
%%
%% Figures using additional raster images can only be included by \input if
%% they are in the same directory as the main LaTeX file. For loading figures
%% from other directories you can use the `import` package
%%   \usepackage{import}
%% and then include the figures with
%%   \import{<path to file>}{<filename>.pgf}
%%
%% Matplotlib used the following preamble
%%   \usepackage{fontspec}
%%   \setmainfont{DejaVuSerif.ttf}[Path=/home/owen/progs/firedrake-complex/firedrake/lib/python3.5/site-packages/matplotlib/mpl-data/fonts/ttf/]
%%   \setsansfont{DejaVuSans.ttf}[Path=/home/owen/progs/firedrake-complex/firedrake/lib/python3.5/site-packages/matplotlib/mpl-data/fonts/ttf/]
%%   \setmonofont{DejaVuSansMono.ttf}[Path=/home/owen/progs/firedrake-complex/firedrake/lib/python3.5/site-packages/matplotlib/mpl-data/fonts/ttf/]
%%
\begingroup%
\makeatletter%
\begin{pgfpicture}%
\pgfpathrectangle{\pgfpointorigin}{\pgfqpoint{5.000000in}{4.000000in}}%
\pgfusepath{use as bounding box, clip}%
\begin{pgfscope}%
\pgfsetbuttcap%
\pgfsetmiterjoin%
\definecolor{currentfill}{rgb}{1.000000,1.000000,1.000000}%
\pgfsetfillcolor{currentfill}%
\pgfsetlinewidth{0.000000pt}%
\definecolor{currentstroke}{rgb}{1.000000,1.000000,1.000000}%
\pgfsetstrokecolor{currentstroke}%
\pgfsetdash{}{0pt}%
\pgfpathmoveto{\pgfqpoint{0.000000in}{0.000000in}}%
\pgfpathlineto{\pgfqpoint{5.000000in}{0.000000in}}%
\pgfpathlineto{\pgfqpoint{5.000000in}{4.000000in}}%
\pgfpathlineto{\pgfqpoint{0.000000in}{4.000000in}}%
\pgfpathclose%
\pgfusepath{fill}%
\end{pgfscope}%
\begin{pgfscope}%
\pgfsetbuttcap%
\pgfsetmiterjoin%
\definecolor{currentfill}{rgb}{1.000000,1.000000,1.000000}%
\pgfsetfillcolor{currentfill}%
\pgfsetlinewidth{0.000000pt}%
\definecolor{currentstroke}{rgb}{0.000000,0.000000,0.000000}%
\pgfsetstrokecolor{currentstroke}%
\pgfsetstrokeopacity{0.000000}%
\pgfsetdash{}{0pt}%
\pgfpathmoveto{\pgfqpoint{0.625000in}{0.440000in}}%
\pgfpathlineto{\pgfqpoint{4.500000in}{0.440000in}}%
\pgfpathlineto{\pgfqpoint{4.500000in}{3.520000in}}%
\pgfpathlineto{\pgfqpoint{0.625000in}{3.520000in}}%
\pgfpathclose%
\pgfusepath{fill}%
\end{pgfscope}%
\begin{pgfscope}%
\pgfsetbuttcap%
\pgfsetroundjoin%
\definecolor{currentfill}{rgb}{0.000000,0.000000,0.000000}%
\pgfsetfillcolor{currentfill}%
\pgfsetlinewidth{0.803000pt}%
\definecolor{currentstroke}{rgb}{0.000000,0.000000,0.000000}%
\pgfsetstrokecolor{currentstroke}%
\pgfsetdash{}{0pt}%
\pgfsys@defobject{currentmarker}{\pgfqpoint{0.000000in}{-0.048611in}}{\pgfqpoint{0.000000in}{0.000000in}}{%
\pgfpathmoveto{\pgfqpoint{0.000000in}{0.000000in}}%
\pgfpathlineto{\pgfqpoint{0.000000in}{-0.048611in}}%
\pgfusepath{stroke,fill}%
}%
\begin{pgfscope}%
\pgfsys@transformshift{0.801136in}{0.440000in}%
\pgfsys@useobject{currentmarker}{}%
\end{pgfscope}%
\end{pgfscope}%
\begin{pgfscope}%
\definecolor{textcolor}{rgb}{0.000000,0.000000,0.000000}%
\pgfsetstrokecolor{textcolor}%
\pgfsetfillcolor{textcolor}%
\pgftext[x=0.801136in,y=0.342778in,,top]{\color{textcolor}\sffamily\fontsize{10.000000}{12.000000}\selectfont \(\displaystyle {10^{0}}\)}%
\end{pgfscope}%
\begin{pgfscope}%
\pgfsetbuttcap%
\pgfsetroundjoin%
\definecolor{currentfill}{rgb}{0.000000,0.000000,0.000000}%
\pgfsetfillcolor{currentfill}%
\pgfsetlinewidth{0.803000pt}%
\definecolor{currentstroke}{rgb}{0.000000,0.000000,0.000000}%
\pgfsetstrokecolor{currentstroke}%
\pgfsetdash{}{0pt}%
\pgfsys@defobject{currentmarker}{\pgfqpoint{0.000000in}{-0.048611in}}{\pgfqpoint{0.000000in}{0.000000in}}{%
\pgfpathmoveto{\pgfqpoint{0.000000in}{0.000000in}}%
\pgfpathlineto{\pgfqpoint{0.000000in}{-0.048611in}}%
\pgfusepath{stroke,fill}%
}%
\begin{pgfscope}%
\pgfsys@transformshift{1.864977in}{0.440000in}%
\pgfsys@useobject{currentmarker}{}%
\end{pgfscope}%
\end{pgfscope}%
\begin{pgfscope}%
\definecolor{textcolor}{rgb}{0.000000,0.000000,0.000000}%
\pgfsetstrokecolor{textcolor}%
\pgfsetfillcolor{textcolor}%
\pgftext[x=1.864977in,y=0.342778in,,top]{\color{textcolor}\sffamily\fontsize{10.000000}{12.000000}\selectfont \(\displaystyle {10^{1}}\)}%
\end{pgfscope}%
\begin{pgfscope}%
\pgfsetbuttcap%
\pgfsetroundjoin%
\definecolor{currentfill}{rgb}{0.000000,0.000000,0.000000}%
\pgfsetfillcolor{currentfill}%
\pgfsetlinewidth{0.803000pt}%
\definecolor{currentstroke}{rgb}{0.000000,0.000000,0.000000}%
\pgfsetstrokecolor{currentstroke}%
\pgfsetdash{}{0pt}%
\pgfsys@defobject{currentmarker}{\pgfqpoint{0.000000in}{-0.048611in}}{\pgfqpoint{0.000000in}{0.000000in}}{%
\pgfpathmoveto{\pgfqpoint{0.000000in}{0.000000in}}%
\pgfpathlineto{\pgfqpoint{0.000000in}{-0.048611in}}%
\pgfusepath{stroke,fill}%
}%
\begin{pgfscope}%
\pgfsys@transformshift{2.928818in}{0.440000in}%
\pgfsys@useobject{currentmarker}{}%
\end{pgfscope}%
\end{pgfscope}%
\begin{pgfscope}%
\definecolor{textcolor}{rgb}{0.000000,0.000000,0.000000}%
\pgfsetstrokecolor{textcolor}%
\pgfsetfillcolor{textcolor}%
\pgftext[x=2.928818in,y=0.342778in,,top]{\color{textcolor}\sffamily\fontsize{10.000000}{12.000000}\selectfont \(\displaystyle {10^{2}}\)}%
\end{pgfscope}%
\begin{pgfscope}%
\pgfsetbuttcap%
\pgfsetroundjoin%
\definecolor{currentfill}{rgb}{0.000000,0.000000,0.000000}%
\pgfsetfillcolor{currentfill}%
\pgfsetlinewidth{0.803000pt}%
\definecolor{currentstroke}{rgb}{0.000000,0.000000,0.000000}%
\pgfsetstrokecolor{currentstroke}%
\pgfsetdash{}{0pt}%
\pgfsys@defobject{currentmarker}{\pgfqpoint{0.000000in}{-0.048611in}}{\pgfqpoint{0.000000in}{0.000000in}}{%
\pgfpathmoveto{\pgfqpoint{0.000000in}{0.000000in}}%
\pgfpathlineto{\pgfqpoint{0.000000in}{-0.048611in}}%
\pgfusepath{stroke,fill}%
}%
\begin{pgfscope}%
\pgfsys@transformshift{3.992658in}{0.440000in}%
\pgfsys@useobject{currentmarker}{}%
\end{pgfscope}%
\end{pgfscope}%
\begin{pgfscope}%
\definecolor{textcolor}{rgb}{0.000000,0.000000,0.000000}%
\pgfsetstrokecolor{textcolor}%
\pgfsetfillcolor{textcolor}%
\pgftext[x=3.992658in,y=0.342778in,,top]{\color{textcolor}\sffamily\fontsize{10.000000}{12.000000}\selectfont \(\displaystyle {10^{3}}\)}%
\end{pgfscope}%
\begin{pgfscope}%
\pgfsetbuttcap%
\pgfsetroundjoin%
\definecolor{currentfill}{rgb}{0.000000,0.000000,0.000000}%
\pgfsetfillcolor{currentfill}%
\pgfsetlinewidth{0.602250pt}%
\definecolor{currentstroke}{rgb}{0.000000,0.000000,0.000000}%
\pgfsetstrokecolor{currentstroke}%
\pgfsetdash{}{0pt}%
\pgfsys@defobject{currentmarker}{\pgfqpoint{0.000000in}{-0.027778in}}{\pgfqpoint{0.000000in}{0.000000in}}{%
\pgfpathmoveto{\pgfqpoint{0.000000in}{0.000000in}}%
\pgfpathlineto{\pgfqpoint{0.000000in}{-0.027778in}}%
\pgfusepath{stroke,fill}%
}%
\begin{pgfscope}%
\pgfsys@transformshift{0.636345in}{0.440000in}%
\pgfsys@useobject{currentmarker}{}%
\end{pgfscope}%
\end{pgfscope}%
\begin{pgfscope}%
\pgfsetbuttcap%
\pgfsetroundjoin%
\definecolor{currentfill}{rgb}{0.000000,0.000000,0.000000}%
\pgfsetfillcolor{currentfill}%
\pgfsetlinewidth{0.602250pt}%
\definecolor{currentstroke}{rgb}{0.000000,0.000000,0.000000}%
\pgfsetstrokecolor{currentstroke}%
\pgfsetdash{}{0pt}%
\pgfsys@defobject{currentmarker}{\pgfqpoint{0.000000in}{-0.027778in}}{\pgfqpoint{0.000000in}{0.000000in}}{%
\pgfpathmoveto{\pgfqpoint{0.000000in}{0.000000in}}%
\pgfpathlineto{\pgfqpoint{0.000000in}{-0.027778in}}%
\pgfusepath{stroke,fill}%
}%
\begin{pgfscope}%
\pgfsys@transformshift{0.698040in}{0.440000in}%
\pgfsys@useobject{currentmarker}{}%
\end{pgfscope}%
\end{pgfscope}%
\begin{pgfscope}%
\pgfsetbuttcap%
\pgfsetroundjoin%
\definecolor{currentfill}{rgb}{0.000000,0.000000,0.000000}%
\pgfsetfillcolor{currentfill}%
\pgfsetlinewidth{0.602250pt}%
\definecolor{currentstroke}{rgb}{0.000000,0.000000,0.000000}%
\pgfsetstrokecolor{currentstroke}%
\pgfsetdash{}{0pt}%
\pgfsys@defobject{currentmarker}{\pgfqpoint{0.000000in}{-0.027778in}}{\pgfqpoint{0.000000in}{0.000000in}}{%
\pgfpathmoveto{\pgfqpoint{0.000000in}{0.000000in}}%
\pgfpathlineto{\pgfqpoint{0.000000in}{-0.027778in}}%
\pgfusepath{stroke,fill}%
}%
\begin{pgfscope}%
\pgfsys@transformshift{0.752458in}{0.440000in}%
\pgfsys@useobject{currentmarker}{}%
\end{pgfscope}%
\end{pgfscope}%
\begin{pgfscope}%
\pgfsetbuttcap%
\pgfsetroundjoin%
\definecolor{currentfill}{rgb}{0.000000,0.000000,0.000000}%
\pgfsetfillcolor{currentfill}%
\pgfsetlinewidth{0.602250pt}%
\definecolor{currentstroke}{rgb}{0.000000,0.000000,0.000000}%
\pgfsetstrokecolor{currentstroke}%
\pgfsetdash{}{0pt}%
\pgfsys@defobject{currentmarker}{\pgfqpoint{0.000000in}{-0.027778in}}{\pgfqpoint{0.000000in}{0.000000in}}{%
\pgfpathmoveto{\pgfqpoint{0.000000in}{0.000000in}}%
\pgfpathlineto{\pgfqpoint{0.000000in}{-0.027778in}}%
\pgfusepath{stroke,fill}%
}%
\begin{pgfscope}%
\pgfsys@transformshift{1.121384in}{0.440000in}%
\pgfsys@useobject{currentmarker}{}%
\end{pgfscope}%
\end{pgfscope}%
\begin{pgfscope}%
\pgfsetbuttcap%
\pgfsetroundjoin%
\definecolor{currentfill}{rgb}{0.000000,0.000000,0.000000}%
\pgfsetfillcolor{currentfill}%
\pgfsetlinewidth{0.602250pt}%
\definecolor{currentstroke}{rgb}{0.000000,0.000000,0.000000}%
\pgfsetstrokecolor{currentstroke}%
\pgfsetdash{}{0pt}%
\pgfsys@defobject{currentmarker}{\pgfqpoint{0.000000in}{-0.027778in}}{\pgfqpoint{0.000000in}{0.000000in}}{%
\pgfpathmoveto{\pgfqpoint{0.000000in}{0.000000in}}%
\pgfpathlineto{\pgfqpoint{0.000000in}{-0.027778in}}%
\pgfusepath{stroke,fill}%
}%
\begin{pgfscope}%
\pgfsys@transformshift{1.308717in}{0.440000in}%
\pgfsys@useobject{currentmarker}{}%
\end{pgfscope}%
\end{pgfscope}%
\begin{pgfscope}%
\pgfsetbuttcap%
\pgfsetroundjoin%
\definecolor{currentfill}{rgb}{0.000000,0.000000,0.000000}%
\pgfsetfillcolor{currentfill}%
\pgfsetlinewidth{0.602250pt}%
\definecolor{currentstroke}{rgb}{0.000000,0.000000,0.000000}%
\pgfsetstrokecolor{currentstroke}%
\pgfsetdash{}{0pt}%
\pgfsys@defobject{currentmarker}{\pgfqpoint{0.000000in}{-0.027778in}}{\pgfqpoint{0.000000in}{0.000000in}}{%
\pgfpathmoveto{\pgfqpoint{0.000000in}{0.000000in}}%
\pgfpathlineto{\pgfqpoint{0.000000in}{-0.027778in}}%
\pgfusepath{stroke,fill}%
}%
\begin{pgfscope}%
\pgfsys@transformshift{1.441632in}{0.440000in}%
\pgfsys@useobject{currentmarker}{}%
\end{pgfscope}%
\end{pgfscope}%
\begin{pgfscope}%
\pgfsetbuttcap%
\pgfsetroundjoin%
\definecolor{currentfill}{rgb}{0.000000,0.000000,0.000000}%
\pgfsetfillcolor{currentfill}%
\pgfsetlinewidth{0.602250pt}%
\definecolor{currentstroke}{rgb}{0.000000,0.000000,0.000000}%
\pgfsetstrokecolor{currentstroke}%
\pgfsetdash{}{0pt}%
\pgfsys@defobject{currentmarker}{\pgfqpoint{0.000000in}{-0.027778in}}{\pgfqpoint{0.000000in}{0.000000in}}{%
\pgfpathmoveto{\pgfqpoint{0.000000in}{0.000000in}}%
\pgfpathlineto{\pgfqpoint{0.000000in}{-0.027778in}}%
\pgfusepath{stroke,fill}%
}%
\begin{pgfscope}%
\pgfsys@transformshift{1.544729in}{0.440000in}%
\pgfsys@useobject{currentmarker}{}%
\end{pgfscope}%
\end{pgfscope}%
\begin{pgfscope}%
\pgfsetbuttcap%
\pgfsetroundjoin%
\definecolor{currentfill}{rgb}{0.000000,0.000000,0.000000}%
\pgfsetfillcolor{currentfill}%
\pgfsetlinewidth{0.602250pt}%
\definecolor{currentstroke}{rgb}{0.000000,0.000000,0.000000}%
\pgfsetstrokecolor{currentstroke}%
\pgfsetdash{}{0pt}%
\pgfsys@defobject{currentmarker}{\pgfqpoint{0.000000in}{-0.027778in}}{\pgfqpoint{0.000000in}{0.000000in}}{%
\pgfpathmoveto{\pgfqpoint{0.000000in}{0.000000in}}%
\pgfpathlineto{\pgfqpoint{0.000000in}{-0.027778in}}%
\pgfusepath{stroke,fill}%
}%
\begin{pgfscope}%
\pgfsys@transformshift{1.628965in}{0.440000in}%
\pgfsys@useobject{currentmarker}{}%
\end{pgfscope}%
\end{pgfscope}%
\begin{pgfscope}%
\pgfsetbuttcap%
\pgfsetroundjoin%
\definecolor{currentfill}{rgb}{0.000000,0.000000,0.000000}%
\pgfsetfillcolor{currentfill}%
\pgfsetlinewidth{0.602250pt}%
\definecolor{currentstroke}{rgb}{0.000000,0.000000,0.000000}%
\pgfsetstrokecolor{currentstroke}%
\pgfsetdash{}{0pt}%
\pgfsys@defobject{currentmarker}{\pgfqpoint{0.000000in}{-0.027778in}}{\pgfqpoint{0.000000in}{0.000000in}}{%
\pgfpathmoveto{\pgfqpoint{0.000000in}{0.000000in}}%
\pgfpathlineto{\pgfqpoint{0.000000in}{-0.027778in}}%
\pgfusepath{stroke,fill}%
}%
\begin{pgfscope}%
\pgfsys@transformshift{1.700186in}{0.440000in}%
\pgfsys@useobject{currentmarker}{}%
\end{pgfscope}%
\end{pgfscope}%
\begin{pgfscope}%
\pgfsetbuttcap%
\pgfsetroundjoin%
\definecolor{currentfill}{rgb}{0.000000,0.000000,0.000000}%
\pgfsetfillcolor{currentfill}%
\pgfsetlinewidth{0.602250pt}%
\definecolor{currentstroke}{rgb}{0.000000,0.000000,0.000000}%
\pgfsetstrokecolor{currentstroke}%
\pgfsetdash{}{0pt}%
\pgfsys@defobject{currentmarker}{\pgfqpoint{0.000000in}{-0.027778in}}{\pgfqpoint{0.000000in}{0.000000in}}{%
\pgfpathmoveto{\pgfqpoint{0.000000in}{0.000000in}}%
\pgfpathlineto{\pgfqpoint{0.000000in}{-0.027778in}}%
\pgfusepath{stroke,fill}%
}%
\begin{pgfscope}%
\pgfsys@transformshift{1.761880in}{0.440000in}%
\pgfsys@useobject{currentmarker}{}%
\end{pgfscope}%
\end{pgfscope}%
\begin{pgfscope}%
\pgfsetbuttcap%
\pgfsetroundjoin%
\definecolor{currentfill}{rgb}{0.000000,0.000000,0.000000}%
\pgfsetfillcolor{currentfill}%
\pgfsetlinewidth{0.602250pt}%
\definecolor{currentstroke}{rgb}{0.000000,0.000000,0.000000}%
\pgfsetstrokecolor{currentstroke}%
\pgfsetdash{}{0pt}%
\pgfsys@defobject{currentmarker}{\pgfqpoint{0.000000in}{-0.027778in}}{\pgfqpoint{0.000000in}{0.000000in}}{%
\pgfpathmoveto{\pgfqpoint{0.000000in}{0.000000in}}%
\pgfpathlineto{\pgfqpoint{0.000000in}{-0.027778in}}%
\pgfusepath{stroke,fill}%
}%
\begin{pgfscope}%
\pgfsys@transformshift{1.816298in}{0.440000in}%
\pgfsys@useobject{currentmarker}{}%
\end{pgfscope}%
\end{pgfscope}%
\begin{pgfscope}%
\pgfsetbuttcap%
\pgfsetroundjoin%
\definecolor{currentfill}{rgb}{0.000000,0.000000,0.000000}%
\pgfsetfillcolor{currentfill}%
\pgfsetlinewidth{0.602250pt}%
\definecolor{currentstroke}{rgb}{0.000000,0.000000,0.000000}%
\pgfsetstrokecolor{currentstroke}%
\pgfsetdash{}{0pt}%
\pgfsys@defobject{currentmarker}{\pgfqpoint{0.000000in}{-0.027778in}}{\pgfqpoint{0.000000in}{0.000000in}}{%
\pgfpathmoveto{\pgfqpoint{0.000000in}{0.000000in}}%
\pgfpathlineto{\pgfqpoint{0.000000in}{-0.027778in}}%
\pgfusepath{stroke,fill}%
}%
\begin{pgfscope}%
\pgfsys@transformshift{2.185225in}{0.440000in}%
\pgfsys@useobject{currentmarker}{}%
\end{pgfscope}%
\end{pgfscope}%
\begin{pgfscope}%
\pgfsetbuttcap%
\pgfsetroundjoin%
\definecolor{currentfill}{rgb}{0.000000,0.000000,0.000000}%
\pgfsetfillcolor{currentfill}%
\pgfsetlinewidth{0.602250pt}%
\definecolor{currentstroke}{rgb}{0.000000,0.000000,0.000000}%
\pgfsetstrokecolor{currentstroke}%
\pgfsetdash{}{0pt}%
\pgfsys@defobject{currentmarker}{\pgfqpoint{0.000000in}{-0.027778in}}{\pgfqpoint{0.000000in}{0.000000in}}{%
\pgfpathmoveto{\pgfqpoint{0.000000in}{0.000000in}}%
\pgfpathlineto{\pgfqpoint{0.000000in}{-0.027778in}}%
\pgfusepath{stroke,fill}%
}%
\begin{pgfscope}%
\pgfsys@transformshift{2.372558in}{0.440000in}%
\pgfsys@useobject{currentmarker}{}%
\end{pgfscope}%
\end{pgfscope}%
\begin{pgfscope}%
\pgfsetbuttcap%
\pgfsetroundjoin%
\definecolor{currentfill}{rgb}{0.000000,0.000000,0.000000}%
\pgfsetfillcolor{currentfill}%
\pgfsetlinewidth{0.602250pt}%
\definecolor{currentstroke}{rgb}{0.000000,0.000000,0.000000}%
\pgfsetstrokecolor{currentstroke}%
\pgfsetdash{}{0pt}%
\pgfsys@defobject{currentmarker}{\pgfqpoint{0.000000in}{-0.027778in}}{\pgfqpoint{0.000000in}{0.000000in}}{%
\pgfpathmoveto{\pgfqpoint{0.000000in}{0.000000in}}%
\pgfpathlineto{\pgfqpoint{0.000000in}{-0.027778in}}%
\pgfusepath{stroke,fill}%
}%
\begin{pgfscope}%
\pgfsys@transformshift{2.505473in}{0.440000in}%
\pgfsys@useobject{currentmarker}{}%
\end{pgfscope}%
\end{pgfscope}%
\begin{pgfscope}%
\pgfsetbuttcap%
\pgfsetroundjoin%
\definecolor{currentfill}{rgb}{0.000000,0.000000,0.000000}%
\pgfsetfillcolor{currentfill}%
\pgfsetlinewidth{0.602250pt}%
\definecolor{currentstroke}{rgb}{0.000000,0.000000,0.000000}%
\pgfsetstrokecolor{currentstroke}%
\pgfsetdash{}{0pt}%
\pgfsys@defobject{currentmarker}{\pgfqpoint{0.000000in}{-0.027778in}}{\pgfqpoint{0.000000in}{0.000000in}}{%
\pgfpathmoveto{\pgfqpoint{0.000000in}{0.000000in}}%
\pgfpathlineto{\pgfqpoint{0.000000in}{-0.027778in}}%
\pgfusepath{stroke,fill}%
}%
\begin{pgfscope}%
\pgfsys@transformshift{2.608570in}{0.440000in}%
\pgfsys@useobject{currentmarker}{}%
\end{pgfscope}%
\end{pgfscope}%
\begin{pgfscope}%
\pgfsetbuttcap%
\pgfsetroundjoin%
\definecolor{currentfill}{rgb}{0.000000,0.000000,0.000000}%
\pgfsetfillcolor{currentfill}%
\pgfsetlinewidth{0.602250pt}%
\definecolor{currentstroke}{rgb}{0.000000,0.000000,0.000000}%
\pgfsetstrokecolor{currentstroke}%
\pgfsetdash{}{0pt}%
\pgfsys@defobject{currentmarker}{\pgfqpoint{0.000000in}{-0.027778in}}{\pgfqpoint{0.000000in}{0.000000in}}{%
\pgfpathmoveto{\pgfqpoint{0.000000in}{0.000000in}}%
\pgfpathlineto{\pgfqpoint{0.000000in}{-0.027778in}}%
\pgfusepath{stroke,fill}%
}%
\begin{pgfscope}%
\pgfsys@transformshift{2.692806in}{0.440000in}%
\pgfsys@useobject{currentmarker}{}%
\end{pgfscope}%
\end{pgfscope}%
\begin{pgfscope}%
\pgfsetbuttcap%
\pgfsetroundjoin%
\definecolor{currentfill}{rgb}{0.000000,0.000000,0.000000}%
\pgfsetfillcolor{currentfill}%
\pgfsetlinewidth{0.602250pt}%
\definecolor{currentstroke}{rgb}{0.000000,0.000000,0.000000}%
\pgfsetstrokecolor{currentstroke}%
\pgfsetdash{}{0pt}%
\pgfsys@defobject{currentmarker}{\pgfqpoint{0.000000in}{-0.027778in}}{\pgfqpoint{0.000000in}{0.000000in}}{%
\pgfpathmoveto{\pgfqpoint{0.000000in}{0.000000in}}%
\pgfpathlineto{\pgfqpoint{0.000000in}{-0.027778in}}%
\pgfusepath{stroke,fill}%
}%
\begin{pgfscope}%
\pgfsys@transformshift{2.764027in}{0.440000in}%
\pgfsys@useobject{currentmarker}{}%
\end{pgfscope}%
\end{pgfscope}%
\begin{pgfscope}%
\pgfsetbuttcap%
\pgfsetroundjoin%
\definecolor{currentfill}{rgb}{0.000000,0.000000,0.000000}%
\pgfsetfillcolor{currentfill}%
\pgfsetlinewidth{0.602250pt}%
\definecolor{currentstroke}{rgb}{0.000000,0.000000,0.000000}%
\pgfsetstrokecolor{currentstroke}%
\pgfsetdash{}{0pt}%
\pgfsys@defobject{currentmarker}{\pgfqpoint{0.000000in}{-0.027778in}}{\pgfqpoint{0.000000in}{0.000000in}}{%
\pgfpathmoveto{\pgfqpoint{0.000000in}{0.000000in}}%
\pgfpathlineto{\pgfqpoint{0.000000in}{-0.027778in}}%
\pgfusepath{stroke,fill}%
}%
\begin{pgfscope}%
\pgfsys@transformshift{2.825721in}{0.440000in}%
\pgfsys@useobject{currentmarker}{}%
\end{pgfscope}%
\end{pgfscope}%
\begin{pgfscope}%
\pgfsetbuttcap%
\pgfsetroundjoin%
\definecolor{currentfill}{rgb}{0.000000,0.000000,0.000000}%
\pgfsetfillcolor{currentfill}%
\pgfsetlinewidth{0.602250pt}%
\definecolor{currentstroke}{rgb}{0.000000,0.000000,0.000000}%
\pgfsetstrokecolor{currentstroke}%
\pgfsetdash{}{0pt}%
\pgfsys@defobject{currentmarker}{\pgfqpoint{0.000000in}{-0.027778in}}{\pgfqpoint{0.000000in}{0.000000in}}{%
\pgfpathmoveto{\pgfqpoint{0.000000in}{0.000000in}}%
\pgfpathlineto{\pgfqpoint{0.000000in}{-0.027778in}}%
\pgfusepath{stroke,fill}%
}%
\begin{pgfscope}%
\pgfsys@transformshift{2.880139in}{0.440000in}%
\pgfsys@useobject{currentmarker}{}%
\end{pgfscope}%
\end{pgfscope}%
\begin{pgfscope}%
\pgfsetbuttcap%
\pgfsetroundjoin%
\definecolor{currentfill}{rgb}{0.000000,0.000000,0.000000}%
\pgfsetfillcolor{currentfill}%
\pgfsetlinewidth{0.602250pt}%
\definecolor{currentstroke}{rgb}{0.000000,0.000000,0.000000}%
\pgfsetstrokecolor{currentstroke}%
\pgfsetdash{}{0pt}%
\pgfsys@defobject{currentmarker}{\pgfqpoint{0.000000in}{-0.027778in}}{\pgfqpoint{0.000000in}{0.000000in}}{%
\pgfpathmoveto{\pgfqpoint{0.000000in}{0.000000in}}%
\pgfpathlineto{\pgfqpoint{0.000000in}{-0.027778in}}%
\pgfusepath{stroke,fill}%
}%
\begin{pgfscope}%
\pgfsys@transformshift{3.249066in}{0.440000in}%
\pgfsys@useobject{currentmarker}{}%
\end{pgfscope}%
\end{pgfscope}%
\begin{pgfscope}%
\pgfsetbuttcap%
\pgfsetroundjoin%
\definecolor{currentfill}{rgb}{0.000000,0.000000,0.000000}%
\pgfsetfillcolor{currentfill}%
\pgfsetlinewidth{0.602250pt}%
\definecolor{currentstroke}{rgb}{0.000000,0.000000,0.000000}%
\pgfsetstrokecolor{currentstroke}%
\pgfsetdash{}{0pt}%
\pgfsys@defobject{currentmarker}{\pgfqpoint{0.000000in}{-0.027778in}}{\pgfqpoint{0.000000in}{0.000000in}}{%
\pgfpathmoveto{\pgfqpoint{0.000000in}{0.000000in}}%
\pgfpathlineto{\pgfqpoint{0.000000in}{-0.027778in}}%
\pgfusepath{stroke,fill}%
}%
\begin{pgfscope}%
\pgfsys@transformshift{3.436399in}{0.440000in}%
\pgfsys@useobject{currentmarker}{}%
\end{pgfscope}%
\end{pgfscope}%
\begin{pgfscope}%
\pgfsetbuttcap%
\pgfsetroundjoin%
\definecolor{currentfill}{rgb}{0.000000,0.000000,0.000000}%
\pgfsetfillcolor{currentfill}%
\pgfsetlinewidth{0.602250pt}%
\definecolor{currentstroke}{rgb}{0.000000,0.000000,0.000000}%
\pgfsetstrokecolor{currentstroke}%
\pgfsetdash{}{0pt}%
\pgfsys@defobject{currentmarker}{\pgfqpoint{0.000000in}{-0.027778in}}{\pgfqpoint{0.000000in}{0.000000in}}{%
\pgfpathmoveto{\pgfqpoint{0.000000in}{0.000000in}}%
\pgfpathlineto{\pgfqpoint{0.000000in}{-0.027778in}}%
\pgfusepath{stroke,fill}%
}%
\begin{pgfscope}%
\pgfsys@transformshift{3.569313in}{0.440000in}%
\pgfsys@useobject{currentmarker}{}%
\end{pgfscope}%
\end{pgfscope}%
\begin{pgfscope}%
\pgfsetbuttcap%
\pgfsetroundjoin%
\definecolor{currentfill}{rgb}{0.000000,0.000000,0.000000}%
\pgfsetfillcolor{currentfill}%
\pgfsetlinewidth{0.602250pt}%
\definecolor{currentstroke}{rgb}{0.000000,0.000000,0.000000}%
\pgfsetstrokecolor{currentstroke}%
\pgfsetdash{}{0pt}%
\pgfsys@defobject{currentmarker}{\pgfqpoint{0.000000in}{-0.027778in}}{\pgfqpoint{0.000000in}{0.000000in}}{%
\pgfpathmoveto{\pgfqpoint{0.000000in}{0.000000in}}%
\pgfpathlineto{\pgfqpoint{0.000000in}{-0.027778in}}%
\pgfusepath{stroke,fill}%
}%
\begin{pgfscope}%
\pgfsys@transformshift{3.672410in}{0.440000in}%
\pgfsys@useobject{currentmarker}{}%
\end{pgfscope}%
\end{pgfscope}%
\begin{pgfscope}%
\pgfsetbuttcap%
\pgfsetroundjoin%
\definecolor{currentfill}{rgb}{0.000000,0.000000,0.000000}%
\pgfsetfillcolor{currentfill}%
\pgfsetlinewidth{0.602250pt}%
\definecolor{currentstroke}{rgb}{0.000000,0.000000,0.000000}%
\pgfsetstrokecolor{currentstroke}%
\pgfsetdash{}{0pt}%
\pgfsys@defobject{currentmarker}{\pgfqpoint{0.000000in}{-0.027778in}}{\pgfqpoint{0.000000in}{0.000000in}}{%
\pgfpathmoveto{\pgfqpoint{0.000000in}{0.000000in}}%
\pgfpathlineto{\pgfqpoint{0.000000in}{-0.027778in}}%
\pgfusepath{stroke,fill}%
}%
\begin{pgfscope}%
\pgfsys@transformshift{3.756646in}{0.440000in}%
\pgfsys@useobject{currentmarker}{}%
\end{pgfscope}%
\end{pgfscope}%
\begin{pgfscope}%
\pgfsetbuttcap%
\pgfsetroundjoin%
\definecolor{currentfill}{rgb}{0.000000,0.000000,0.000000}%
\pgfsetfillcolor{currentfill}%
\pgfsetlinewidth{0.602250pt}%
\definecolor{currentstroke}{rgb}{0.000000,0.000000,0.000000}%
\pgfsetstrokecolor{currentstroke}%
\pgfsetdash{}{0pt}%
\pgfsys@defobject{currentmarker}{\pgfqpoint{0.000000in}{-0.027778in}}{\pgfqpoint{0.000000in}{0.000000in}}{%
\pgfpathmoveto{\pgfqpoint{0.000000in}{0.000000in}}%
\pgfpathlineto{\pgfqpoint{0.000000in}{-0.027778in}}%
\pgfusepath{stroke,fill}%
}%
\begin{pgfscope}%
\pgfsys@transformshift{3.827867in}{0.440000in}%
\pgfsys@useobject{currentmarker}{}%
\end{pgfscope}%
\end{pgfscope}%
\begin{pgfscope}%
\pgfsetbuttcap%
\pgfsetroundjoin%
\definecolor{currentfill}{rgb}{0.000000,0.000000,0.000000}%
\pgfsetfillcolor{currentfill}%
\pgfsetlinewidth{0.602250pt}%
\definecolor{currentstroke}{rgb}{0.000000,0.000000,0.000000}%
\pgfsetstrokecolor{currentstroke}%
\pgfsetdash{}{0pt}%
\pgfsys@defobject{currentmarker}{\pgfqpoint{0.000000in}{-0.027778in}}{\pgfqpoint{0.000000in}{0.000000in}}{%
\pgfpathmoveto{\pgfqpoint{0.000000in}{0.000000in}}%
\pgfpathlineto{\pgfqpoint{0.000000in}{-0.027778in}}%
\pgfusepath{stroke,fill}%
}%
\begin{pgfscope}%
\pgfsys@transformshift{3.889561in}{0.440000in}%
\pgfsys@useobject{currentmarker}{}%
\end{pgfscope}%
\end{pgfscope}%
\begin{pgfscope}%
\pgfsetbuttcap%
\pgfsetroundjoin%
\definecolor{currentfill}{rgb}{0.000000,0.000000,0.000000}%
\pgfsetfillcolor{currentfill}%
\pgfsetlinewidth{0.602250pt}%
\definecolor{currentstroke}{rgb}{0.000000,0.000000,0.000000}%
\pgfsetstrokecolor{currentstroke}%
\pgfsetdash{}{0pt}%
\pgfsys@defobject{currentmarker}{\pgfqpoint{0.000000in}{-0.027778in}}{\pgfqpoint{0.000000in}{0.000000in}}{%
\pgfpathmoveto{\pgfqpoint{0.000000in}{0.000000in}}%
\pgfpathlineto{\pgfqpoint{0.000000in}{-0.027778in}}%
\pgfusepath{stroke,fill}%
}%
\begin{pgfscope}%
\pgfsys@transformshift{3.943980in}{0.440000in}%
\pgfsys@useobject{currentmarker}{}%
\end{pgfscope}%
\end{pgfscope}%
\begin{pgfscope}%
\pgfsetbuttcap%
\pgfsetroundjoin%
\definecolor{currentfill}{rgb}{0.000000,0.000000,0.000000}%
\pgfsetfillcolor{currentfill}%
\pgfsetlinewidth{0.602250pt}%
\definecolor{currentstroke}{rgb}{0.000000,0.000000,0.000000}%
\pgfsetstrokecolor{currentstroke}%
\pgfsetdash{}{0pt}%
\pgfsys@defobject{currentmarker}{\pgfqpoint{0.000000in}{-0.027778in}}{\pgfqpoint{0.000000in}{0.000000in}}{%
\pgfpathmoveto{\pgfqpoint{0.000000in}{0.000000in}}%
\pgfpathlineto{\pgfqpoint{0.000000in}{-0.027778in}}%
\pgfusepath{stroke,fill}%
}%
\begin{pgfscope}%
\pgfsys@transformshift{4.312906in}{0.440000in}%
\pgfsys@useobject{currentmarker}{}%
\end{pgfscope}%
\end{pgfscope}%
\begin{pgfscope}%
\pgfsetbuttcap%
\pgfsetroundjoin%
\definecolor{currentfill}{rgb}{0.000000,0.000000,0.000000}%
\pgfsetfillcolor{currentfill}%
\pgfsetlinewidth{0.602250pt}%
\definecolor{currentstroke}{rgb}{0.000000,0.000000,0.000000}%
\pgfsetstrokecolor{currentstroke}%
\pgfsetdash{}{0pt}%
\pgfsys@defobject{currentmarker}{\pgfqpoint{0.000000in}{-0.027778in}}{\pgfqpoint{0.000000in}{0.000000in}}{%
\pgfpathmoveto{\pgfqpoint{0.000000in}{0.000000in}}%
\pgfpathlineto{\pgfqpoint{0.000000in}{-0.027778in}}%
\pgfusepath{stroke,fill}%
}%
\begin{pgfscope}%
\pgfsys@transformshift{4.500239in}{0.440000in}%
\pgfsys@useobject{currentmarker}{}%
\end{pgfscope}%
\end{pgfscope}%
\begin{pgfscope}%
\definecolor{textcolor}{rgb}{0.000000,0.000000,0.000000}%
\pgfsetstrokecolor{textcolor}%
\pgfsetfillcolor{textcolor}%
\pgftext[x=2.562500in,y=0.152809in,,top]{\color{textcolor}\sffamily\fontsize{10.000000}{12.000000}\selectfont \(\displaystyle N_\mathrm{QMC}\)}%
\end{pgfscope}%
\begin{pgfscope}%
\pgfsetbuttcap%
\pgfsetroundjoin%
\definecolor{currentfill}{rgb}{0.000000,0.000000,0.000000}%
\pgfsetfillcolor{currentfill}%
\pgfsetlinewidth{0.803000pt}%
\definecolor{currentstroke}{rgb}{0.000000,0.000000,0.000000}%
\pgfsetstrokecolor{currentstroke}%
\pgfsetdash{}{0pt}%
\pgfsys@defobject{currentmarker}{\pgfqpoint{-0.048611in}{0.000000in}}{\pgfqpoint{0.000000in}{0.000000in}}{%
\pgfpathmoveto{\pgfqpoint{0.000000in}{0.000000in}}%
\pgfpathlineto{\pgfqpoint{-0.048611in}{0.000000in}}%
\pgfusepath{stroke,fill}%
}%
\begin{pgfscope}%
\pgfsys@transformshift{0.625000in}{1.410969in}%
\pgfsys@useobject{currentmarker}{}%
\end{pgfscope}%
\end{pgfscope}%
\begin{pgfscope}%
\definecolor{textcolor}{rgb}{0.000000,0.000000,0.000000}%
\pgfsetstrokecolor{textcolor}%
\pgfsetfillcolor{textcolor}%
\pgftext[x=0.239775in,y=1.358207in,left,base]{\color{textcolor}\sffamily\fontsize{10.000000}{12.000000}\selectfont \(\displaystyle {10^{-4}}\)}%
\end{pgfscope}%
\begin{pgfscope}%
\pgfsetbuttcap%
\pgfsetroundjoin%
\definecolor{currentfill}{rgb}{0.000000,0.000000,0.000000}%
\pgfsetfillcolor{currentfill}%
\pgfsetlinewidth{0.803000pt}%
\definecolor{currentstroke}{rgb}{0.000000,0.000000,0.000000}%
\pgfsetstrokecolor{currentstroke}%
\pgfsetdash{}{0pt}%
\pgfsys@defobject{currentmarker}{\pgfqpoint{-0.048611in}{0.000000in}}{\pgfqpoint{0.000000in}{0.000000in}}{%
\pgfpathmoveto{\pgfqpoint{0.000000in}{0.000000in}}%
\pgfpathlineto{\pgfqpoint{-0.048611in}{0.000000in}}%
\pgfusepath{stroke,fill}%
}%
\begin{pgfscope}%
\pgfsys@transformshift{0.625000in}{2.408726in}%
\pgfsys@useobject{currentmarker}{}%
\end{pgfscope}%
\end{pgfscope}%
\begin{pgfscope}%
\definecolor{textcolor}{rgb}{0.000000,0.000000,0.000000}%
\pgfsetstrokecolor{textcolor}%
\pgfsetfillcolor{textcolor}%
\pgftext[x=0.239775in,y=2.355964in,left,base]{\color{textcolor}\sffamily\fontsize{10.000000}{12.000000}\selectfont \(\displaystyle {10^{-3}}\)}%
\end{pgfscope}%
\begin{pgfscope}%
\pgfsetbuttcap%
\pgfsetroundjoin%
\definecolor{currentfill}{rgb}{0.000000,0.000000,0.000000}%
\pgfsetfillcolor{currentfill}%
\pgfsetlinewidth{0.803000pt}%
\definecolor{currentstroke}{rgb}{0.000000,0.000000,0.000000}%
\pgfsetstrokecolor{currentstroke}%
\pgfsetdash{}{0pt}%
\pgfsys@defobject{currentmarker}{\pgfqpoint{-0.048611in}{0.000000in}}{\pgfqpoint{0.000000in}{0.000000in}}{%
\pgfpathmoveto{\pgfqpoint{0.000000in}{0.000000in}}%
\pgfpathlineto{\pgfqpoint{-0.048611in}{0.000000in}}%
\pgfusepath{stroke,fill}%
}%
\begin{pgfscope}%
\pgfsys@transformshift{0.625000in}{3.406483in}%
\pgfsys@useobject{currentmarker}{}%
\end{pgfscope}%
\end{pgfscope}%
\begin{pgfscope}%
\definecolor{textcolor}{rgb}{0.000000,0.000000,0.000000}%
\pgfsetstrokecolor{textcolor}%
\pgfsetfillcolor{textcolor}%
\pgftext[x=0.239775in,y=3.353721in,left,base]{\color{textcolor}\sffamily\fontsize{10.000000}{12.000000}\selectfont \(\displaystyle {10^{-2}}\)}%
\end{pgfscope}%
\begin{pgfscope}%
\pgfsetbuttcap%
\pgfsetroundjoin%
\definecolor{currentfill}{rgb}{0.000000,0.000000,0.000000}%
\pgfsetfillcolor{currentfill}%
\pgfsetlinewidth{0.602250pt}%
\definecolor{currentstroke}{rgb}{0.000000,0.000000,0.000000}%
\pgfsetstrokecolor{currentstroke}%
\pgfsetdash{}{0pt}%
\pgfsys@defobject{currentmarker}{\pgfqpoint{-0.027778in}{0.000000in}}{\pgfqpoint{0.000000in}{0.000000in}}{%
\pgfpathmoveto{\pgfqpoint{0.000000in}{0.000000in}}%
\pgfpathlineto{\pgfqpoint{-0.027778in}{0.000000in}}%
\pgfusepath{stroke,fill}%
}%
\begin{pgfscope}%
\pgfsys@transformshift{0.625000in}{0.713567in}%
\pgfsys@useobject{currentmarker}{}%
\end{pgfscope}%
\end{pgfscope}%
\begin{pgfscope}%
\pgfsetbuttcap%
\pgfsetroundjoin%
\definecolor{currentfill}{rgb}{0.000000,0.000000,0.000000}%
\pgfsetfillcolor{currentfill}%
\pgfsetlinewidth{0.602250pt}%
\definecolor{currentstroke}{rgb}{0.000000,0.000000,0.000000}%
\pgfsetstrokecolor{currentstroke}%
\pgfsetdash{}{0pt}%
\pgfsys@defobject{currentmarker}{\pgfqpoint{-0.027778in}{0.000000in}}{\pgfqpoint{0.000000in}{0.000000in}}{%
\pgfpathmoveto{\pgfqpoint{0.000000in}{0.000000in}}%
\pgfpathlineto{\pgfqpoint{-0.027778in}{0.000000in}}%
\pgfusepath{stroke,fill}%
}%
\begin{pgfscope}%
\pgfsys@transformshift{0.625000in}{0.889263in}%
\pgfsys@useobject{currentmarker}{}%
\end{pgfscope}%
\end{pgfscope}%
\begin{pgfscope}%
\pgfsetbuttcap%
\pgfsetroundjoin%
\definecolor{currentfill}{rgb}{0.000000,0.000000,0.000000}%
\pgfsetfillcolor{currentfill}%
\pgfsetlinewidth{0.602250pt}%
\definecolor{currentstroke}{rgb}{0.000000,0.000000,0.000000}%
\pgfsetstrokecolor{currentstroke}%
\pgfsetdash{}{0pt}%
\pgfsys@defobject{currentmarker}{\pgfqpoint{-0.027778in}{0.000000in}}{\pgfqpoint{0.000000in}{0.000000in}}{%
\pgfpathmoveto{\pgfqpoint{0.000000in}{0.000000in}}%
\pgfpathlineto{\pgfqpoint{-0.027778in}{0.000000in}}%
\pgfusepath{stroke,fill}%
}%
\begin{pgfscope}%
\pgfsys@transformshift{0.625000in}{1.013922in}%
\pgfsys@useobject{currentmarker}{}%
\end{pgfscope}%
\end{pgfscope}%
\begin{pgfscope}%
\pgfsetbuttcap%
\pgfsetroundjoin%
\definecolor{currentfill}{rgb}{0.000000,0.000000,0.000000}%
\pgfsetfillcolor{currentfill}%
\pgfsetlinewidth{0.602250pt}%
\definecolor{currentstroke}{rgb}{0.000000,0.000000,0.000000}%
\pgfsetstrokecolor{currentstroke}%
\pgfsetdash{}{0pt}%
\pgfsys@defobject{currentmarker}{\pgfqpoint{-0.027778in}{0.000000in}}{\pgfqpoint{0.000000in}{0.000000in}}{%
\pgfpathmoveto{\pgfqpoint{0.000000in}{0.000000in}}%
\pgfpathlineto{\pgfqpoint{-0.027778in}{0.000000in}}%
\pgfusepath{stroke,fill}%
}%
\begin{pgfscope}%
\pgfsys@transformshift{0.625000in}{1.110614in}%
\pgfsys@useobject{currentmarker}{}%
\end{pgfscope}%
\end{pgfscope}%
\begin{pgfscope}%
\pgfsetbuttcap%
\pgfsetroundjoin%
\definecolor{currentfill}{rgb}{0.000000,0.000000,0.000000}%
\pgfsetfillcolor{currentfill}%
\pgfsetlinewidth{0.602250pt}%
\definecolor{currentstroke}{rgb}{0.000000,0.000000,0.000000}%
\pgfsetstrokecolor{currentstroke}%
\pgfsetdash{}{0pt}%
\pgfsys@defobject{currentmarker}{\pgfqpoint{-0.027778in}{0.000000in}}{\pgfqpoint{0.000000in}{0.000000in}}{%
\pgfpathmoveto{\pgfqpoint{0.000000in}{0.000000in}}%
\pgfpathlineto{\pgfqpoint{-0.027778in}{0.000000in}}%
\pgfusepath{stroke,fill}%
}%
\begin{pgfscope}%
\pgfsys@transformshift{0.625000in}{1.189618in}%
\pgfsys@useobject{currentmarker}{}%
\end{pgfscope}%
\end{pgfscope}%
\begin{pgfscope}%
\pgfsetbuttcap%
\pgfsetroundjoin%
\definecolor{currentfill}{rgb}{0.000000,0.000000,0.000000}%
\pgfsetfillcolor{currentfill}%
\pgfsetlinewidth{0.602250pt}%
\definecolor{currentstroke}{rgb}{0.000000,0.000000,0.000000}%
\pgfsetstrokecolor{currentstroke}%
\pgfsetdash{}{0pt}%
\pgfsys@defobject{currentmarker}{\pgfqpoint{-0.027778in}{0.000000in}}{\pgfqpoint{0.000000in}{0.000000in}}{%
\pgfpathmoveto{\pgfqpoint{0.000000in}{0.000000in}}%
\pgfpathlineto{\pgfqpoint{-0.027778in}{0.000000in}}%
\pgfusepath{stroke,fill}%
}%
\begin{pgfscope}%
\pgfsys@transformshift{0.625000in}{1.256415in}%
\pgfsys@useobject{currentmarker}{}%
\end{pgfscope}%
\end{pgfscope}%
\begin{pgfscope}%
\pgfsetbuttcap%
\pgfsetroundjoin%
\definecolor{currentfill}{rgb}{0.000000,0.000000,0.000000}%
\pgfsetfillcolor{currentfill}%
\pgfsetlinewidth{0.602250pt}%
\definecolor{currentstroke}{rgb}{0.000000,0.000000,0.000000}%
\pgfsetstrokecolor{currentstroke}%
\pgfsetdash{}{0pt}%
\pgfsys@defobject{currentmarker}{\pgfqpoint{-0.027778in}{0.000000in}}{\pgfqpoint{0.000000in}{0.000000in}}{%
\pgfpathmoveto{\pgfqpoint{0.000000in}{0.000000in}}%
\pgfpathlineto{\pgfqpoint{-0.027778in}{0.000000in}}%
\pgfusepath{stroke,fill}%
}%
\begin{pgfscope}%
\pgfsys@transformshift{0.625000in}{1.314276in}%
\pgfsys@useobject{currentmarker}{}%
\end{pgfscope}%
\end{pgfscope}%
\begin{pgfscope}%
\pgfsetbuttcap%
\pgfsetroundjoin%
\definecolor{currentfill}{rgb}{0.000000,0.000000,0.000000}%
\pgfsetfillcolor{currentfill}%
\pgfsetlinewidth{0.602250pt}%
\definecolor{currentstroke}{rgb}{0.000000,0.000000,0.000000}%
\pgfsetstrokecolor{currentstroke}%
\pgfsetdash{}{0pt}%
\pgfsys@defobject{currentmarker}{\pgfqpoint{-0.027778in}{0.000000in}}{\pgfqpoint{0.000000in}{0.000000in}}{%
\pgfpathmoveto{\pgfqpoint{0.000000in}{0.000000in}}%
\pgfpathlineto{\pgfqpoint{-0.027778in}{0.000000in}}%
\pgfusepath{stroke,fill}%
}%
\begin{pgfscope}%
\pgfsys@transformshift{0.625000in}{1.365314in}%
\pgfsys@useobject{currentmarker}{}%
\end{pgfscope}%
\end{pgfscope}%
\begin{pgfscope}%
\pgfsetbuttcap%
\pgfsetroundjoin%
\definecolor{currentfill}{rgb}{0.000000,0.000000,0.000000}%
\pgfsetfillcolor{currentfill}%
\pgfsetlinewidth{0.602250pt}%
\definecolor{currentstroke}{rgb}{0.000000,0.000000,0.000000}%
\pgfsetstrokecolor{currentstroke}%
\pgfsetdash{}{0pt}%
\pgfsys@defobject{currentmarker}{\pgfqpoint{-0.027778in}{0.000000in}}{\pgfqpoint{0.000000in}{0.000000in}}{%
\pgfpathmoveto{\pgfqpoint{0.000000in}{0.000000in}}%
\pgfpathlineto{\pgfqpoint{-0.027778in}{0.000000in}}%
\pgfusepath{stroke,fill}%
}%
\begin{pgfscope}%
\pgfsys@transformshift{0.625000in}{1.711324in}%
\pgfsys@useobject{currentmarker}{}%
\end{pgfscope}%
\end{pgfscope}%
\begin{pgfscope}%
\pgfsetbuttcap%
\pgfsetroundjoin%
\definecolor{currentfill}{rgb}{0.000000,0.000000,0.000000}%
\pgfsetfillcolor{currentfill}%
\pgfsetlinewidth{0.602250pt}%
\definecolor{currentstroke}{rgb}{0.000000,0.000000,0.000000}%
\pgfsetstrokecolor{currentstroke}%
\pgfsetdash{}{0pt}%
\pgfsys@defobject{currentmarker}{\pgfqpoint{-0.027778in}{0.000000in}}{\pgfqpoint{0.000000in}{0.000000in}}{%
\pgfpathmoveto{\pgfqpoint{0.000000in}{0.000000in}}%
\pgfpathlineto{\pgfqpoint{-0.027778in}{0.000000in}}%
\pgfusepath{stroke,fill}%
}%
\begin{pgfscope}%
\pgfsys@transformshift{0.625000in}{1.887020in}%
\pgfsys@useobject{currentmarker}{}%
\end{pgfscope}%
\end{pgfscope}%
\begin{pgfscope}%
\pgfsetbuttcap%
\pgfsetroundjoin%
\definecolor{currentfill}{rgb}{0.000000,0.000000,0.000000}%
\pgfsetfillcolor{currentfill}%
\pgfsetlinewidth{0.602250pt}%
\definecolor{currentstroke}{rgb}{0.000000,0.000000,0.000000}%
\pgfsetstrokecolor{currentstroke}%
\pgfsetdash{}{0pt}%
\pgfsys@defobject{currentmarker}{\pgfqpoint{-0.027778in}{0.000000in}}{\pgfqpoint{0.000000in}{0.000000in}}{%
\pgfpathmoveto{\pgfqpoint{0.000000in}{0.000000in}}%
\pgfpathlineto{\pgfqpoint{-0.027778in}{0.000000in}}%
\pgfusepath{stroke,fill}%
}%
\begin{pgfscope}%
\pgfsys@transformshift{0.625000in}{2.011678in}%
\pgfsys@useobject{currentmarker}{}%
\end{pgfscope}%
\end{pgfscope}%
\begin{pgfscope}%
\pgfsetbuttcap%
\pgfsetroundjoin%
\definecolor{currentfill}{rgb}{0.000000,0.000000,0.000000}%
\pgfsetfillcolor{currentfill}%
\pgfsetlinewidth{0.602250pt}%
\definecolor{currentstroke}{rgb}{0.000000,0.000000,0.000000}%
\pgfsetstrokecolor{currentstroke}%
\pgfsetdash{}{0pt}%
\pgfsys@defobject{currentmarker}{\pgfqpoint{-0.027778in}{0.000000in}}{\pgfqpoint{0.000000in}{0.000000in}}{%
\pgfpathmoveto{\pgfqpoint{0.000000in}{0.000000in}}%
\pgfpathlineto{\pgfqpoint{-0.027778in}{0.000000in}}%
\pgfusepath{stroke,fill}%
}%
\begin{pgfscope}%
\pgfsys@transformshift{0.625000in}{2.108371in}%
\pgfsys@useobject{currentmarker}{}%
\end{pgfscope}%
\end{pgfscope}%
\begin{pgfscope}%
\pgfsetbuttcap%
\pgfsetroundjoin%
\definecolor{currentfill}{rgb}{0.000000,0.000000,0.000000}%
\pgfsetfillcolor{currentfill}%
\pgfsetlinewidth{0.602250pt}%
\definecolor{currentstroke}{rgb}{0.000000,0.000000,0.000000}%
\pgfsetstrokecolor{currentstroke}%
\pgfsetdash{}{0pt}%
\pgfsys@defobject{currentmarker}{\pgfqpoint{-0.027778in}{0.000000in}}{\pgfqpoint{0.000000in}{0.000000in}}{%
\pgfpathmoveto{\pgfqpoint{0.000000in}{0.000000in}}%
\pgfpathlineto{\pgfqpoint{-0.027778in}{0.000000in}}%
\pgfusepath{stroke,fill}%
}%
\begin{pgfscope}%
\pgfsys@transformshift{0.625000in}{2.187375in}%
\pgfsys@useobject{currentmarker}{}%
\end{pgfscope}%
\end{pgfscope}%
\begin{pgfscope}%
\pgfsetbuttcap%
\pgfsetroundjoin%
\definecolor{currentfill}{rgb}{0.000000,0.000000,0.000000}%
\pgfsetfillcolor{currentfill}%
\pgfsetlinewidth{0.602250pt}%
\definecolor{currentstroke}{rgb}{0.000000,0.000000,0.000000}%
\pgfsetstrokecolor{currentstroke}%
\pgfsetdash{}{0pt}%
\pgfsys@defobject{currentmarker}{\pgfqpoint{-0.027778in}{0.000000in}}{\pgfqpoint{0.000000in}{0.000000in}}{%
\pgfpathmoveto{\pgfqpoint{0.000000in}{0.000000in}}%
\pgfpathlineto{\pgfqpoint{-0.027778in}{0.000000in}}%
\pgfusepath{stroke,fill}%
}%
\begin{pgfscope}%
\pgfsys@transformshift{0.625000in}{2.254171in}%
\pgfsys@useobject{currentmarker}{}%
\end{pgfscope}%
\end{pgfscope}%
\begin{pgfscope}%
\pgfsetbuttcap%
\pgfsetroundjoin%
\definecolor{currentfill}{rgb}{0.000000,0.000000,0.000000}%
\pgfsetfillcolor{currentfill}%
\pgfsetlinewidth{0.602250pt}%
\definecolor{currentstroke}{rgb}{0.000000,0.000000,0.000000}%
\pgfsetstrokecolor{currentstroke}%
\pgfsetdash{}{0pt}%
\pgfsys@defobject{currentmarker}{\pgfqpoint{-0.027778in}{0.000000in}}{\pgfqpoint{0.000000in}{0.000000in}}{%
\pgfpathmoveto{\pgfqpoint{0.000000in}{0.000000in}}%
\pgfpathlineto{\pgfqpoint{-0.027778in}{0.000000in}}%
\pgfusepath{stroke,fill}%
}%
\begin{pgfscope}%
\pgfsys@transformshift{0.625000in}{2.312033in}%
\pgfsys@useobject{currentmarker}{}%
\end{pgfscope}%
\end{pgfscope}%
\begin{pgfscope}%
\pgfsetbuttcap%
\pgfsetroundjoin%
\definecolor{currentfill}{rgb}{0.000000,0.000000,0.000000}%
\pgfsetfillcolor{currentfill}%
\pgfsetlinewidth{0.602250pt}%
\definecolor{currentstroke}{rgb}{0.000000,0.000000,0.000000}%
\pgfsetstrokecolor{currentstroke}%
\pgfsetdash{}{0pt}%
\pgfsys@defobject{currentmarker}{\pgfqpoint{-0.027778in}{0.000000in}}{\pgfqpoint{0.000000in}{0.000000in}}{%
\pgfpathmoveto{\pgfqpoint{0.000000in}{0.000000in}}%
\pgfpathlineto{\pgfqpoint{-0.027778in}{0.000000in}}%
\pgfusepath{stroke,fill}%
}%
\begin{pgfscope}%
\pgfsys@transformshift{0.625000in}{2.363071in}%
\pgfsys@useobject{currentmarker}{}%
\end{pgfscope}%
\end{pgfscope}%
\begin{pgfscope}%
\pgfsetbuttcap%
\pgfsetroundjoin%
\definecolor{currentfill}{rgb}{0.000000,0.000000,0.000000}%
\pgfsetfillcolor{currentfill}%
\pgfsetlinewidth{0.602250pt}%
\definecolor{currentstroke}{rgb}{0.000000,0.000000,0.000000}%
\pgfsetstrokecolor{currentstroke}%
\pgfsetdash{}{0pt}%
\pgfsys@defobject{currentmarker}{\pgfqpoint{-0.027778in}{0.000000in}}{\pgfqpoint{0.000000in}{0.000000in}}{%
\pgfpathmoveto{\pgfqpoint{0.000000in}{0.000000in}}%
\pgfpathlineto{\pgfqpoint{-0.027778in}{0.000000in}}%
\pgfusepath{stroke,fill}%
}%
\begin{pgfscope}%
\pgfsys@transformshift{0.625000in}{2.709081in}%
\pgfsys@useobject{currentmarker}{}%
\end{pgfscope}%
\end{pgfscope}%
\begin{pgfscope}%
\pgfsetbuttcap%
\pgfsetroundjoin%
\definecolor{currentfill}{rgb}{0.000000,0.000000,0.000000}%
\pgfsetfillcolor{currentfill}%
\pgfsetlinewidth{0.602250pt}%
\definecolor{currentstroke}{rgb}{0.000000,0.000000,0.000000}%
\pgfsetstrokecolor{currentstroke}%
\pgfsetdash{}{0pt}%
\pgfsys@defobject{currentmarker}{\pgfqpoint{-0.027778in}{0.000000in}}{\pgfqpoint{0.000000in}{0.000000in}}{%
\pgfpathmoveto{\pgfqpoint{0.000000in}{0.000000in}}%
\pgfpathlineto{\pgfqpoint{-0.027778in}{0.000000in}}%
\pgfusepath{stroke,fill}%
}%
\begin{pgfscope}%
\pgfsys@transformshift{0.625000in}{2.884777in}%
\pgfsys@useobject{currentmarker}{}%
\end{pgfscope}%
\end{pgfscope}%
\begin{pgfscope}%
\pgfsetbuttcap%
\pgfsetroundjoin%
\definecolor{currentfill}{rgb}{0.000000,0.000000,0.000000}%
\pgfsetfillcolor{currentfill}%
\pgfsetlinewidth{0.602250pt}%
\definecolor{currentstroke}{rgb}{0.000000,0.000000,0.000000}%
\pgfsetstrokecolor{currentstroke}%
\pgfsetdash{}{0pt}%
\pgfsys@defobject{currentmarker}{\pgfqpoint{-0.027778in}{0.000000in}}{\pgfqpoint{0.000000in}{0.000000in}}{%
\pgfpathmoveto{\pgfqpoint{0.000000in}{0.000000in}}%
\pgfpathlineto{\pgfqpoint{-0.027778in}{0.000000in}}%
\pgfusepath{stroke,fill}%
}%
\begin{pgfscope}%
\pgfsys@transformshift{0.625000in}{3.009435in}%
\pgfsys@useobject{currentmarker}{}%
\end{pgfscope}%
\end{pgfscope}%
\begin{pgfscope}%
\pgfsetbuttcap%
\pgfsetroundjoin%
\definecolor{currentfill}{rgb}{0.000000,0.000000,0.000000}%
\pgfsetfillcolor{currentfill}%
\pgfsetlinewidth{0.602250pt}%
\definecolor{currentstroke}{rgb}{0.000000,0.000000,0.000000}%
\pgfsetstrokecolor{currentstroke}%
\pgfsetdash{}{0pt}%
\pgfsys@defobject{currentmarker}{\pgfqpoint{-0.027778in}{0.000000in}}{\pgfqpoint{0.000000in}{0.000000in}}{%
\pgfpathmoveto{\pgfqpoint{0.000000in}{0.000000in}}%
\pgfpathlineto{\pgfqpoint{-0.027778in}{0.000000in}}%
\pgfusepath{stroke,fill}%
}%
\begin{pgfscope}%
\pgfsys@transformshift{0.625000in}{3.106128in}%
\pgfsys@useobject{currentmarker}{}%
\end{pgfscope}%
\end{pgfscope}%
\begin{pgfscope}%
\pgfsetbuttcap%
\pgfsetroundjoin%
\definecolor{currentfill}{rgb}{0.000000,0.000000,0.000000}%
\pgfsetfillcolor{currentfill}%
\pgfsetlinewidth{0.602250pt}%
\definecolor{currentstroke}{rgb}{0.000000,0.000000,0.000000}%
\pgfsetstrokecolor{currentstroke}%
\pgfsetdash{}{0pt}%
\pgfsys@defobject{currentmarker}{\pgfqpoint{-0.027778in}{0.000000in}}{\pgfqpoint{0.000000in}{0.000000in}}{%
\pgfpathmoveto{\pgfqpoint{0.000000in}{0.000000in}}%
\pgfpathlineto{\pgfqpoint{-0.027778in}{0.000000in}}%
\pgfusepath{stroke,fill}%
}%
\begin{pgfscope}%
\pgfsys@transformshift{0.625000in}{3.185132in}%
\pgfsys@useobject{currentmarker}{}%
\end{pgfscope}%
\end{pgfscope}%
\begin{pgfscope}%
\pgfsetbuttcap%
\pgfsetroundjoin%
\definecolor{currentfill}{rgb}{0.000000,0.000000,0.000000}%
\pgfsetfillcolor{currentfill}%
\pgfsetlinewidth{0.602250pt}%
\definecolor{currentstroke}{rgb}{0.000000,0.000000,0.000000}%
\pgfsetstrokecolor{currentstroke}%
\pgfsetdash{}{0pt}%
\pgfsys@defobject{currentmarker}{\pgfqpoint{-0.027778in}{0.000000in}}{\pgfqpoint{0.000000in}{0.000000in}}{%
\pgfpathmoveto{\pgfqpoint{0.000000in}{0.000000in}}%
\pgfpathlineto{\pgfqpoint{-0.027778in}{0.000000in}}%
\pgfusepath{stroke,fill}%
}%
\begin{pgfscope}%
\pgfsys@transformshift{0.625000in}{3.251928in}%
\pgfsys@useobject{currentmarker}{}%
\end{pgfscope}%
\end{pgfscope}%
\begin{pgfscope}%
\pgfsetbuttcap%
\pgfsetroundjoin%
\definecolor{currentfill}{rgb}{0.000000,0.000000,0.000000}%
\pgfsetfillcolor{currentfill}%
\pgfsetlinewidth{0.602250pt}%
\definecolor{currentstroke}{rgb}{0.000000,0.000000,0.000000}%
\pgfsetstrokecolor{currentstroke}%
\pgfsetdash{}{0pt}%
\pgfsys@defobject{currentmarker}{\pgfqpoint{-0.027778in}{0.000000in}}{\pgfqpoint{0.000000in}{0.000000in}}{%
\pgfpathmoveto{\pgfqpoint{0.000000in}{0.000000in}}%
\pgfpathlineto{\pgfqpoint{-0.027778in}{0.000000in}}%
\pgfusepath{stroke,fill}%
}%
\begin{pgfscope}%
\pgfsys@transformshift{0.625000in}{3.309790in}%
\pgfsys@useobject{currentmarker}{}%
\end{pgfscope}%
\end{pgfscope}%
\begin{pgfscope}%
\pgfsetbuttcap%
\pgfsetroundjoin%
\definecolor{currentfill}{rgb}{0.000000,0.000000,0.000000}%
\pgfsetfillcolor{currentfill}%
\pgfsetlinewidth{0.602250pt}%
\definecolor{currentstroke}{rgb}{0.000000,0.000000,0.000000}%
\pgfsetstrokecolor{currentstroke}%
\pgfsetdash{}{0pt}%
\pgfsys@defobject{currentmarker}{\pgfqpoint{-0.027778in}{0.000000in}}{\pgfqpoint{0.000000in}{0.000000in}}{%
\pgfpathmoveto{\pgfqpoint{0.000000in}{0.000000in}}%
\pgfpathlineto{\pgfqpoint{-0.027778in}{0.000000in}}%
\pgfusepath{stroke,fill}%
}%
\begin{pgfscope}%
\pgfsys@transformshift{0.625000in}{3.360828in}%
\pgfsys@useobject{currentmarker}{}%
\end{pgfscope}%
\end{pgfscope}%
\begin{pgfscope}%
\definecolor{textcolor}{rgb}{0.000000,0.000000,0.000000}%
\pgfsetstrokecolor{textcolor}%
\pgfsetfillcolor{textcolor}%
\pgftext[x=0.184220in,y=1.980000in,,bottom,rotate=90.000000]{\color{textcolor}\sffamily\fontsize{10.000000}{12.000000}\selectfont QMC Error}%
\end{pgfscope}%
\begin{pgfscope}%
\pgfpathrectangle{\pgfqpoint{0.625000in}{0.440000in}}{\pgfqpoint{3.875000in}{3.080000in}}%
\pgfusepath{clip}%
\pgfsetbuttcap%
\pgfsetroundjoin%
\definecolor{currentfill}{rgb}{0.000000,0.000000,0.000000}%
\pgfsetfillcolor{currentfill}%
\pgfsetlinewidth{1.003750pt}%
\definecolor{currentstroke}{rgb}{0.000000,0.000000,0.000000}%
\pgfsetstrokecolor{currentstroke}%
\pgfsetdash{}{0pt}%
\pgfsys@defobject{currentmarker}{\pgfqpoint{-0.041667in}{-0.041667in}}{\pgfqpoint{0.041667in}{0.041667in}}{%
\pgfpathmoveto{\pgfqpoint{0.000000in}{-0.041667in}}%
\pgfpathcurveto{\pgfqpoint{0.011050in}{-0.041667in}}{\pgfqpoint{0.021649in}{-0.037276in}}{\pgfqpoint{0.029463in}{-0.029463in}}%
\pgfpathcurveto{\pgfqpoint{0.037276in}{-0.021649in}}{\pgfqpoint{0.041667in}{-0.011050in}}{\pgfqpoint{0.041667in}{0.000000in}}%
\pgfpathcurveto{\pgfqpoint{0.041667in}{0.011050in}}{\pgfqpoint{0.037276in}{0.021649in}}{\pgfqpoint{0.029463in}{0.029463in}}%
\pgfpathcurveto{\pgfqpoint{0.021649in}{0.037276in}}{\pgfqpoint{0.011050in}{0.041667in}}{\pgfqpoint{0.000000in}{0.041667in}}%
\pgfpathcurveto{\pgfqpoint{-0.011050in}{0.041667in}}{\pgfqpoint{-0.021649in}{0.037276in}}{\pgfqpoint{-0.029463in}{0.029463in}}%
\pgfpathcurveto{\pgfqpoint{-0.037276in}{0.021649in}}{\pgfqpoint{-0.041667in}{0.011050in}}{\pgfqpoint{-0.041667in}{0.000000in}}%
\pgfpathcurveto{\pgfqpoint{-0.041667in}{-0.011050in}}{\pgfqpoint{-0.037276in}{-0.021649in}}{\pgfqpoint{-0.029463in}{-0.029463in}}%
\pgfpathcurveto{\pgfqpoint{-0.021649in}{-0.037276in}}{\pgfqpoint{-0.011050in}{-0.041667in}}{\pgfqpoint{0.000000in}{-0.041667in}}%
\pgfpathclose%
\pgfusepath{stroke,fill}%
}%
\begin{pgfscope}%
\pgfsys@transformshift{0.801136in}{3.380000in}%
\pgfsys@useobject{currentmarker}{}%
\end{pgfscope}%
\begin{pgfscope}%
\pgfsys@transformshift{1.121384in}{3.061292in}%
\pgfsys@useobject{currentmarker}{}%
\end{pgfscope}%
\begin{pgfscope}%
\pgfsys@transformshift{1.441632in}{2.359261in}%
\pgfsys@useobject{currentmarker}{}%
\end{pgfscope}%
\begin{pgfscope}%
\pgfsys@transformshift{1.761880in}{2.252766in}%
\pgfsys@useobject{currentmarker}{}%
\end{pgfscope}%
\begin{pgfscope}%
\pgfsys@transformshift{2.082128in}{2.216701in}%
\pgfsys@useobject{currentmarker}{}%
\end{pgfscope}%
\begin{pgfscope}%
\pgfsys@transformshift{2.402376in}{1.952936in}%
\pgfsys@useobject{currentmarker}{}%
\end{pgfscope}%
\begin{pgfscope}%
\pgfsys@transformshift{2.722624in}{1.742545in}%
\pgfsys@useobject{currentmarker}{}%
\end{pgfscope}%
\begin{pgfscope}%
\pgfsys@transformshift{3.042872in}{1.760565in}%
\pgfsys@useobject{currentmarker}{}%
\end{pgfscope}%
\begin{pgfscope}%
\pgfsys@transformshift{3.363120in}{1.179217in}%
\pgfsys@useobject{currentmarker}{}%
\end{pgfscope}%
\begin{pgfscope}%
\pgfsys@transformshift{3.683368in}{0.847468in}%
\pgfsys@useobject{currentmarker}{}%
\end{pgfscope}%
\begin{pgfscope}%
\pgfsys@transformshift{4.003616in}{0.700742in}%
\pgfsys@useobject{currentmarker}{}%
\end{pgfscope}%
\begin{pgfscope}%
\pgfsys@transformshift{4.323864in}{0.857538in}%
\pgfsys@useobject{currentmarker}{}%
\end{pgfscope}%
\end{pgfscope}%
\begin{pgfscope}%
\pgfpathrectangle{\pgfqpoint{0.625000in}{0.440000in}}{\pgfqpoint{3.875000in}{3.080000in}}%
\pgfusepath{clip}%
\pgfsetbuttcap%
\pgfsetroundjoin%
\pgfsetlinewidth{1.505625pt}%
\definecolor{currentstroke}{rgb}{0.000000,0.000000,0.000000}%
\pgfsetstrokecolor{currentstroke}%
\pgfsetdash{{5.550000pt}{2.400000pt}}{0.000000pt}%
\pgfpathmoveto{\pgfqpoint{0.801136in}{3.138505in}}%
\pgfpathlineto{\pgfqpoint{4.323864in}{0.580000in}}%
\pgfusepath{stroke}%
\end{pgfscope}%
\begin{pgfscope}%
\pgfsetrectcap%
\pgfsetmiterjoin%
\pgfsetlinewidth{0.803000pt}%
\definecolor{currentstroke}{rgb}{0.000000,0.000000,0.000000}%
\pgfsetstrokecolor{currentstroke}%
\pgfsetdash{}{0pt}%
\pgfpathmoveto{\pgfqpoint{0.625000in}{0.440000in}}%
\pgfpathlineto{\pgfqpoint{0.625000in}{3.520000in}}%
\pgfusepath{stroke}%
\end{pgfscope}%
\begin{pgfscope}%
\pgfsetrectcap%
\pgfsetmiterjoin%
\pgfsetlinewidth{0.803000pt}%
\definecolor{currentstroke}{rgb}{0.000000,0.000000,0.000000}%
\pgfsetstrokecolor{currentstroke}%
\pgfsetdash{}{0pt}%
\pgfpathmoveto{\pgfqpoint{4.500000in}{0.440000in}}%
\pgfpathlineto{\pgfqpoint{4.500000in}{3.520000in}}%
\pgfusepath{stroke}%
\end{pgfscope}%
\begin{pgfscope}%
\pgfsetrectcap%
\pgfsetmiterjoin%
\pgfsetlinewidth{0.803000pt}%
\definecolor{currentstroke}{rgb}{0.000000,0.000000,0.000000}%
\pgfsetstrokecolor{currentstroke}%
\pgfsetdash{}{0pt}%
\pgfpathmoveto{\pgfqpoint{0.625000in}{0.440000in}}%
\pgfpathlineto{\pgfqpoint{4.500000in}{0.440000in}}%
\pgfusepath{stroke}%
\end{pgfscope}%
\begin{pgfscope}%
\pgfsetrectcap%
\pgfsetmiterjoin%
\pgfsetlinewidth{0.803000pt}%
\definecolor{currentstroke}{rgb}{0.000000,0.000000,0.000000}%
\pgfsetstrokecolor{currentstroke}%
\pgfsetdash{}{0pt}%
\pgfpathmoveto{\pgfqpoint{0.625000in}{3.520000in}}%
\pgfpathlineto{\pgfqpoint{4.500000in}{3.520000in}}%
\pgfusepath{stroke}%
\end{pgfscope}%
\begin{pgfscope}%
\pgfsetbuttcap%
\pgfsetmiterjoin%
\definecolor{currentfill}{rgb}{1.000000,1.000000,1.000000}%
\pgfsetfillcolor{currentfill}%
\pgfsetfillopacity{0.800000}%
\pgfsetlinewidth{1.003750pt}%
\definecolor{currentstroke}{rgb}{0.800000,0.800000,0.800000}%
\pgfsetstrokecolor{currentstroke}%
\pgfsetstrokeopacity{0.800000}%
\pgfsetdash{}{0pt}%
\pgfpathmoveto{\pgfqpoint{3.042243in}{3.155267in}}%
\pgfpathlineto{\pgfqpoint{4.402778in}{3.155267in}}%
\pgfpathquadraticcurveto{\pgfqpoint{4.430556in}{3.155267in}}{\pgfqpoint{4.430556in}{3.183045in}}%
\pgfpathlineto{\pgfqpoint{4.430556in}{3.422778in}}%
\pgfpathquadraticcurveto{\pgfqpoint{4.430556in}{3.450556in}}{\pgfqpoint{4.402778in}{3.450556in}}%
\pgfpathlineto{\pgfqpoint{3.042243in}{3.450556in}}%
\pgfpathquadraticcurveto{\pgfqpoint{3.014465in}{3.450556in}}{\pgfqpoint{3.014465in}{3.422778in}}%
\pgfpathlineto{\pgfqpoint{3.014465in}{3.183045in}}%
\pgfpathquadraticcurveto{\pgfqpoint{3.014465in}{3.155267in}}{\pgfqpoint{3.042243in}{3.155267in}}%
\pgfpathclose%
\pgfusepath{stroke,fill}%
\end{pgfscope}%
\begin{pgfscope}%
\pgfsetbuttcap%
\pgfsetroundjoin%
\pgfsetlinewidth{1.505625pt}%
\definecolor{currentstroke}{rgb}{0.000000,0.000000,0.000000}%
\pgfsetstrokecolor{currentstroke}%
\pgfsetdash{{5.550000pt}{2.400000pt}}{0.000000pt}%
\pgfpathmoveto{\pgfqpoint{3.070021in}{3.321543in}}%
\pgfpathlineto{\pgfqpoint{3.347799in}{3.321543in}}%
\pgfusepath{stroke}%
\end{pgfscope}%
\begin{pgfscope}%
\definecolor{textcolor}{rgb}{0.000000,0.000000,0.000000}%
\pgfsetstrokecolor{textcolor}%
\pgfsetfillcolor{textcolor}%
\pgftext[x=3.458910in,y=3.272932in,left,base]{\color{textcolor}\sffamily\fontsize{10.000000}{12.000000}\selectfont \(\displaystyle 0.0053N_{\mathrm{QMC}}^{-0.7743}\)}%
\end{pgfscope}%
\end{pgfpicture}%
\makeatother%
\endgroup%

    \caption[Quasi-Monte-Carlo error, for $Q=u(\bzero)$ and $k=30$.]{Quasi-Monte-Carlo error with increasing number of Quasi-Monte-Carlo points, for $Q=u(\bzero)$ and $k=30$. \label{fig:qmcorigin30}}
  \end{figure}
\begin{figure}[h]
  \centering
%% Creator: Matplotlib, PGF backend
%%
%% To include the figure in your LaTeX document, write
%%   \input{<filename>.pgf}
%%
%% Make sure the required packages are loaded in your preamble
%%   \usepackage{pgf}
%%
%% Figures using additional raster images can only be included by \input if
%% they are in the same directory as the main LaTeX file. For loading figures
%% from other directories you can use the `import` package
%%   \usepackage{import}
%% and then include the figures with
%%   \import{<path to file>}{<filename>.pgf}
%%
%% Matplotlib used the following preamble
%%   \usepackage{fontspec}
%%   \setmainfont{DejaVuSerif.ttf}[Path=/home/owen/progs/firedrake-complex/firedrake/lib/python3.5/site-packages/matplotlib/mpl-data/fonts/ttf/]
%%   \setsansfont{DejaVuSans.ttf}[Path=/home/owen/progs/firedrake-complex/firedrake/lib/python3.5/site-packages/matplotlib/mpl-data/fonts/ttf/]
%%   \setmonofont{DejaVuSansMono.ttf}[Path=/home/owen/progs/firedrake-complex/firedrake/lib/python3.5/site-packages/matplotlib/mpl-data/fonts/ttf/]
%%
\begingroup%
\makeatletter%
\begin{pgfpicture}%
\pgfpathrectangle{\pgfpointorigin}{\pgfqpoint{3.000000in}{3.000000in}}%
\pgfusepath{use as bounding box, clip}%
\begin{pgfscope}%
\pgfsetbuttcap%
\pgfsetmiterjoin%
\definecolor{currentfill}{rgb}{1.000000,1.000000,1.000000}%
\pgfsetfillcolor{currentfill}%
\pgfsetlinewidth{0.000000pt}%
\definecolor{currentstroke}{rgb}{1.000000,1.000000,1.000000}%
\pgfsetstrokecolor{currentstroke}%
\pgfsetdash{}{0pt}%
\pgfpathmoveto{\pgfqpoint{0.000000in}{0.000000in}}%
\pgfpathlineto{\pgfqpoint{3.000000in}{0.000000in}}%
\pgfpathlineto{\pgfqpoint{3.000000in}{3.000000in}}%
\pgfpathlineto{\pgfqpoint{0.000000in}{3.000000in}}%
\pgfpathclose%
\pgfusepath{fill}%
\end{pgfscope}%
\begin{pgfscope}%
\pgfsetbuttcap%
\pgfsetmiterjoin%
\definecolor{currentfill}{rgb}{1.000000,1.000000,1.000000}%
\pgfsetfillcolor{currentfill}%
\pgfsetlinewidth{0.000000pt}%
\definecolor{currentstroke}{rgb}{0.000000,0.000000,0.000000}%
\pgfsetstrokecolor{currentstroke}%
\pgfsetstrokeopacity{0.000000}%
\pgfsetdash{}{0pt}%
\pgfpathmoveto{\pgfqpoint{0.375000in}{0.330000in}}%
\pgfpathlineto{\pgfqpoint{2.700000in}{0.330000in}}%
\pgfpathlineto{\pgfqpoint{2.700000in}{2.640000in}}%
\pgfpathlineto{\pgfqpoint{0.375000in}{2.640000in}}%
\pgfpathclose%
\pgfusepath{fill}%
\end{pgfscope}%
\begin{pgfscope}%
\pgfsetbuttcap%
\pgfsetroundjoin%
\definecolor{currentfill}{rgb}{0.000000,0.000000,0.000000}%
\pgfsetfillcolor{currentfill}%
\pgfsetlinewidth{0.803000pt}%
\definecolor{currentstroke}{rgb}{0.000000,0.000000,0.000000}%
\pgfsetstrokecolor{currentstroke}%
\pgfsetdash{}{0pt}%
\pgfsys@defobject{currentmarker}{\pgfqpoint{0.000000in}{-0.048611in}}{\pgfqpoint{0.000000in}{0.000000in}}{%
\pgfpathmoveto{\pgfqpoint{0.000000in}{0.000000in}}%
\pgfpathlineto{\pgfqpoint{0.000000in}{-0.048611in}}%
\pgfusepath{stroke,fill}%
}%
\begin{pgfscope}%
\pgfsys@transformshift{0.480682in}{0.330000in}%
\pgfsys@useobject{currentmarker}{}%
\end{pgfscope}%
\end{pgfscope}%
\begin{pgfscope}%
\definecolor{textcolor}{rgb}{0.000000,0.000000,0.000000}%
\pgfsetstrokecolor{textcolor}%
\pgfsetfillcolor{textcolor}%
\pgftext[x=0.480682in,y=0.232778in,,top]{\color{textcolor}\sffamily\fontsize{10.000000}{12.000000}\selectfont \(\displaystyle {10^{0}}\)}%
\end{pgfscope}%
\begin{pgfscope}%
\pgfsetbuttcap%
\pgfsetroundjoin%
\definecolor{currentfill}{rgb}{0.000000,0.000000,0.000000}%
\pgfsetfillcolor{currentfill}%
\pgfsetlinewidth{0.803000pt}%
\definecolor{currentstroke}{rgb}{0.000000,0.000000,0.000000}%
\pgfsetstrokecolor{currentstroke}%
\pgfsetdash{}{0pt}%
\pgfsys@defobject{currentmarker}{\pgfqpoint{0.000000in}{-0.048611in}}{\pgfqpoint{0.000000in}{0.000000in}}{%
\pgfpathmoveto{\pgfqpoint{0.000000in}{0.000000in}}%
\pgfpathlineto{\pgfqpoint{0.000000in}{-0.048611in}}%
\pgfusepath{stroke,fill}%
}%
\begin{pgfscope}%
\pgfsys@transformshift{1.118986in}{0.330000in}%
\pgfsys@useobject{currentmarker}{}%
\end{pgfscope}%
\end{pgfscope}%
\begin{pgfscope}%
\definecolor{textcolor}{rgb}{0.000000,0.000000,0.000000}%
\pgfsetstrokecolor{textcolor}%
\pgfsetfillcolor{textcolor}%
\pgftext[x=1.118986in,y=0.232778in,,top]{\color{textcolor}\sffamily\fontsize{10.000000}{12.000000}\selectfont \(\displaystyle {10^{1}}\)}%
\end{pgfscope}%
\begin{pgfscope}%
\pgfsetbuttcap%
\pgfsetroundjoin%
\definecolor{currentfill}{rgb}{0.000000,0.000000,0.000000}%
\pgfsetfillcolor{currentfill}%
\pgfsetlinewidth{0.803000pt}%
\definecolor{currentstroke}{rgb}{0.000000,0.000000,0.000000}%
\pgfsetstrokecolor{currentstroke}%
\pgfsetdash{}{0pt}%
\pgfsys@defobject{currentmarker}{\pgfqpoint{0.000000in}{-0.048611in}}{\pgfqpoint{0.000000in}{0.000000in}}{%
\pgfpathmoveto{\pgfqpoint{0.000000in}{0.000000in}}%
\pgfpathlineto{\pgfqpoint{0.000000in}{-0.048611in}}%
\pgfusepath{stroke,fill}%
}%
\begin{pgfscope}%
\pgfsys@transformshift{1.757291in}{0.330000in}%
\pgfsys@useobject{currentmarker}{}%
\end{pgfscope}%
\end{pgfscope}%
\begin{pgfscope}%
\definecolor{textcolor}{rgb}{0.000000,0.000000,0.000000}%
\pgfsetstrokecolor{textcolor}%
\pgfsetfillcolor{textcolor}%
\pgftext[x=1.757291in,y=0.232778in,,top]{\color{textcolor}\sffamily\fontsize{10.000000}{12.000000}\selectfont \(\displaystyle {10^{2}}\)}%
\end{pgfscope}%
\begin{pgfscope}%
\pgfsetbuttcap%
\pgfsetroundjoin%
\definecolor{currentfill}{rgb}{0.000000,0.000000,0.000000}%
\pgfsetfillcolor{currentfill}%
\pgfsetlinewidth{0.803000pt}%
\definecolor{currentstroke}{rgb}{0.000000,0.000000,0.000000}%
\pgfsetstrokecolor{currentstroke}%
\pgfsetdash{}{0pt}%
\pgfsys@defobject{currentmarker}{\pgfqpoint{0.000000in}{-0.048611in}}{\pgfqpoint{0.000000in}{0.000000in}}{%
\pgfpathmoveto{\pgfqpoint{0.000000in}{0.000000in}}%
\pgfpathlineto{\pgfqpoint{0.000000in}{-0.048611in}}%
\pgfusepath{stroke,fill}%
}%
\begin{pgfscope}%
\pgfsys@transformshift{2.395595in}{0.330000in}%
\pgfsys@useobject{currentmarker}{}%
\end{pgfscope}%
\end{pgfscope}%
\begin{pgfscope}%
\definecolor{textcolor}{rgb}{0.000000,0.000000,0.000000}%
\pgfsetstrokecolor{textcolor}%
\pgfsetfillcolor{textcolor}%
\pgftext[x=2.395595in,y=0.232778in,,top]{\color{textcolor}\sffamily\fontsize{10.000000}{12.000000}\selectfont \(\displaystyle {10^{3}}\)}%
\end{pgfscope}%
\begin{pgfscope}%
\pgfsetbuttcap%
\pgfsetroundjoin%
\definecolor{currentfill}{rgb}{0.000000,0.000000,0.000000}%
\pgfsetfillcolor{currentfill}%
\pgfsetlinewidth{0.602250pt}%
\definecolor{currentstroke}{rgb}{0.000000,0.000000,0.000000}%
\pgfsetstrokecolor{currentstroke}%
\pgfsetdash{}{0pt}%
\pgfsys@defobject{currentmarker}{\pgfqpoint{0.000000in}{-0.027778in}}{\pgfqpoint{0.000000in}{0.000000in}}{%
\pgfpathmoveto{\pgfqpoint{0.000000in}{0.000000in}}%
\pgfpathlineto{\pgfqpoint{0.000000in}{-0.027778in}}%
\pgfusepath{stroke,fill}%
}%
\begin{pgfscope}%
\pgfsys@transformshift{0.381807in}{0.330000in}%
\pgfsys@useobject{currentmarker}{}%
\end{pgfscope}%
\end{pgfscope}%
\begin{pgfscope}%
\pgfsetbuttcap%
\pgfsetroundjoin%
\definecolor{currentfill}{rgb}{0.000000,0.000000,0.000000}%
\pgfsetfillcolor{currentfill}%
\pgfsetlinewidth{0.602250pt}%
\definecolor{currentstroke}{rgb}{0.000000,0.000000,0.000000}%
\pgfsetstrokecolor{currentstroke}%
\pgfsetdash{}{0pt}%
\pgfsys@defobject{currentmarker}{\pgfqpoint{0.000000in}{-0.027778in}}{\pgfqpoint{0.000000in}{0.000000in}}{%
\pgfpathmoveto{\pgfqpoint{0.000000in}{0.000000in}}%
\pgfpathlineto{\pgfqpoint{0.000000in}{-0.027778in}}%
\pgfusepath{stroke,fill}%
}%
\begin{pgfscope}%
\pgfsys@transformshift{0.418824in}{0.330000in}%
\pgfsys@useobject{currentmarker}{}%
\end{pgfscope}%
\end{pgfscope}%
\begin{pgfscope}%
\pgfsetbuttcap%
\pgfsetroundjoin%
\definecolor{currentfill}{rgb}{0.000000,0.000000,0.000000}%
\pgfsetfillcolor{currentfill}%
\pgfsetlinewidth{0.602250pt}%
\definecolor{currentstroke}{rgb}{0.000000,0.000000,0.000000}%
\pgfsetstrokecolor{currentstroke}%
\pgfsetdash{}{0pt}%
\pgfsys@defobject{currentmarker}{\pgfqpoint{0.000000in}{-0.027778in}}{\pgfqpoint{0.000000in}{0.000000in}}{%
\pgfpathmoveto{\pgfqpoint{0.000000in}{0.000000in}}%
\pgfpathlineto{\pgfqpoint{0.000000in}{-0.027778in}}%
\pgfusepath{stroke,fill}%
}%
\begin{pgfscope}%
\pgfsys@transformshift{0.451475in}{0.330000in}%
\pgfsys@useobject{currentmarker}{}%
\end{pgfscope}%
\end{pgfscope}%
\begin{pgfscope}%
\pgfsetbuttcap%
\pgfsetroundjoin%
\definecolor{currentfill}{rgb}{0.000000,0.000000,0.000000}%
\pgfsetfillcolor{currentfill}%
\pgfsetlinewidth{0.602250pt}%
\definecolor{currentstroke}{rgb}{0.000000,0.000000,0.000000}%
\pgfsetstrokecolor{currentstroke}%
\pgfsetdash{}{0pt}%
\pgfsys@defobject{currentmarker}{\pgfqpoint{0.000000in}{-0.027778in}}{\pgfqpoint{0.000000in}{0.000000in}}{%
\pgfpathmoveto{\pgfqpoint{0.000000in}{0.000000in}}%
\pgfpathlineto{\pgfqpoint{0.000000in}{-0.027778in}}%
\pgfusepath{stroke,fill}%
}%
\begin{pgfscope}%
\pgfsys@transformshift{0.672831in}{0.330000in}%
\pgfsys@useobject{currentmarker}{}%
\end{pgfscope}%
\end{pgfscope}%
\begin{pgfscope}%
\pgfsetbuttcap%
\pgfsetroundjoin%
\definecolor{currentfill}{rgb}{0.000000,0.000000,0.000000}%
\pgfsetfillcolor{currentfill}%
\pgfsetlinewidth{0.602250pt}%
\definecolor{currentstroke}{rgb}{0.000000,0.000000,0.000000}%
\pgfsetstrokecolor{currentstroke}%
\pgfsetdash{}{0pt}%
\pgfsys@defobject{currentmarker}{\pgfqpoint{0.000000in}{-0.027778in}}{\pgfqpoint{0.000000in}{0.000000in}}{%
\pgfpathmoveto{\pgfqpoint{0.000000in}{0.000000in}}%
\pgfpathlineto{\pgfqpoint{0.000000in}{-0.027778in}}%
\pgfusepath{stroke,fill}%
}%
\begin{pgfscope}%
\pgfsys@transformshift{0.785230in}{0.330000in}%
\pgfsys@useobject{currentmarker}{}%
\end{pgfscope}%
\end{pgfscope}%
\begin{pgfscope}%
\pgfsetbuttcap%
\pgfsetroundjoin%
\definecolor{currentfill}{rgb}{0.000000,0.000000,0.000000}%
\pgfsetfillcolor{currentfill}%
\pgfsetlinewidth{0.602250pt}%
\definecolor{currentstroke}{rgb}{0.000000,0.000000,0.000000}%
\pgfsetstrokecolor{currentstroke}%
\pgfsetdash{}{0pt}%
\pgfsys@defobject{currentmarker}{\pgfqpoint{0.000000in}{-0.027778in}}{\pgfqpoint{0.000000in}{0.000000in}}{%
\pgfpathmoveto{\pgfqpoint{0.000000in}{0.000000in}}%
\pgfpathlineto{\pgfqpoint{0.000000in}{-0.027778in}}%
\pgfusepath{stroke,fill}%
}%
\begin{pgfscope}%
\pgfsys@transformshift{0.864979in}{0.330000in}%
\pgfsys@useobject{currentmarker}{}%
\end{pgfscope}%
\end{pgfscope}%
\begin{pgfscope}%
\pgfsetbuttcap%
\pgfsetroundjoin%
\definecolor{currentfill}{rgb}{0.000000,0.000000,0.000000}%
\pgfsetfillcolor{currentfill}%
\pgfsetlinewidth{0.602250pt}%
\definecolor{currentstroke}{rgb}{0.000000,0.000000,0.000000}%
\pgfsetstrokecolor{currentstroke}%
\pgfsetdash{}{0pt}%
\pgfsys@defobject{currentmarker}{\pgfqpoint{0.000000in}{-0.027778in}}{\pgfqpoint{0.000000in}{0.000000in}}{%
\pgfpathmoveto{\pgfqpoint{0.000000in}{0.000000in}}%
\pgfpathlineto{\pgfqpoint{0.000000in}{-0.027778in}}%
\pgfusepath{stroke,fill}%
}%
\begin{pgfscope}%
\pgfsys@transformshift{0.926837in}{0.330000in}%
\pgfsys@useobject{currentmarker}{}%
\end{pgfscope}%
\end{pgfscope}%
\begin{pgfscope}%
\pgfsetbuttcap%
\pgfsetroundjoin%
\definecolor{currentfill}{rgb}{0.000000,0.000000,0.000000}%
\pgfsetfillcolor{currentfill}%
\pgfsetlinewidth{0.602250pt}%
\definecolor{currentstroke}{rgb}{0.000000,0.000000,0.000000}%
\pgfsetstrokecolor{currentstroke}%
\pgfsetdash{}{0pt}%
\pgfsys@defobject{currentmarker}{\pgfqpoint{0.000000in}{-0.027778in}}{\pgfqpoint{0.000000in}{0.000000in}}{%
\pgfpathmoveto{\pgfqpoint{0.000000in}{0.000000in}}%
\pgfpathlineto{\pgfqpoint{0.000000in}{-0.027778in}}%
\pgfusepath{stroke,fill}%
}%
\begin{pgfscope}%
\pgfsys@transformshift{0.977379in}{0.330000in}%
\pgfsys@useobject{currentmarker}{}%
\end{pgfscope}%
\end{pgfscope}%
\begin{pgfscope}%
\pgfsetbuttcap%
\pgfsetroundjoin%
\definecolor{currentfill}{rgb}{0.000000,0.000000,0.000000}%
\pgfsetfillcolor{currentfill}%
\pgfsetlinewidth{0.602250pt}%
\definecolor{currentstroke}{rgb}{0.000000,0.000000,0.000000}%
\pgfsetstrokecolor{currentstroke}%
\pgfsetdash{}{0pt}%
\pgfsys@defobject{currentmarker}{\pgfqpoint{0.000000in}{-0.027778in}}{\pgfqpoint{0.000000in}{0.000000in}}{%
\pgfpathmoveto{\pgfqpoint{0.000000in}{0.000000in}}%
\pgfpathlineto{\pgfqpoint{0.000000in}{-0.027778in}}%
\pgfusepath{stroke,fill}%
}%
\begin{pgfscope}%
\pgfsys@transformshift{1.020112in}{0.330000in}%
\pgfsys@useobject{currentmarker}{}%
\end{pgfscope}%
\end{pgfscope}%
\begin{pgfscope}%
\pgfsetbuttcap%
\pgfsetroundjoin%
\definecolor{currentfill}{rgb}{0.000000,0.000000,0.000000}%
\pgfsetfillcolor{currentfill}%
\pgfsetlinewidth{0.602250pt}%
\definecolor{currentstroke}{rgb}{0.000000,0.000000,0.000000}%
\pgfsetstrokecolor{currentstroke}%
\pgfsetdash{}{0pt}%
\pgfsys@defobject{currentmarker}{\pgfqpoint{0.000000in}{-0.027778in}}{\pgfqpoint{0.000000in}{0.000000in}}{%
\pgfpathmoveto{\pgfqpoint{0.000000in}{0.000000in}}%
\pgfpathlineto{\pgfqpoint{0.000000in}{-0.027778in}}%
\pgfusepath{stroke,fill}%
}%
\begin{pgfscope}%
\pgfsys@transformshift{1.057128in}{0.330000in}%
\pgfsys@useobject{currentmarker}{}%
\end{pgfscope}%
\end{pgfscope}%
\begin{pgfscope}%
\pgfsetbuttcap%
\pgfsetroundjoin%
\definecolor{currentfill}{rgb}{0.000000,0.000000,0.000000}%
\pgfsetfillcolor{currentfill}%
\pgfsetlinewidth{0.602250pt}%
\definecolor{currentstroke}{rgb}{0.000000,0.000000,0.000000}%
\pgfsetstrokecolor{currentstroke}%
\pgfsetdash{}{0pt}%
\pgfsys@defobject{currentmarker}{\pgfqpoint{0.000000in}{-0.027778in}}{\pgfqpoint{0.000000in}{0.000000in}}{%
\pgfpathmoveto{\pgfqpoint{0.000000in}{0.000000in}}%
\pgfpathlineto{\pgfqpoint{0.000000in}{-0.027778in}}%
\pgfusepath{stroke,fill}%
}%
\begin{pgfscope}%
\pgfsys@transformshift{1.089779in}{0.330000in}%
\pgfsys@useobject{currentmarker}{}%
\end{pgfscope}%
\end{pgfscope}%
\begin{pgfscope}%
\pgfsetbuttcap%
\pgfsetroundjoin%
\definecolor{currentfill}{rgb}{0.000000,0.000000,0.000000}%
\pgfsetfillcolor{currentfill}%
\pgfsetlinewidth{0.602250pt}%
\definecolor{currentstroke}{rgb}{0.000000,0.000000,0.000000}%
\pgfsetstrokecolor{currentstroke}%
\pgfsetdash{}{0pt}%
\pgfsys@defobject{currentmarker}{\pgfqpoint{0.000000in}{-0.027778in}}{\pgfqpoint{0.000000in}{0.000000in}}{%
\pgfpathmoveto{\pgfqpoint{0.000000in}{0.000000in}}%
\pgfpathlineto{\pgfqpoint{0.000000in}{-0.027778in}}%
\pgfusepath{stroke,fill}%
}%
\begin{pgfscope}%
\pgfsys@transformshift{1.311135in}{0.330000in}%
\pgfsys@useobject{currentmarker}{}%
\end{pgfscope}%
\end{pgfscope}%
\begin{pgfscope}%
\pgfsetbuttcap%
\pgfsetroundjoin%
\definecolor{currentfill}{rgb}{0.000000,0.000000,0.000000}%
\pgfsetfillcolor{currentfill}%
\pgfsetlinewidth{0.602250pt}%
\definecolor{currentstroke}{rgb}{0.000000,0.000000,0.000000}%
\pgfsetstrokecolor{currentstroke}%
\pgfsetdash{}{0pt}%
\pgfsys@defobject{currentmarker}{\pgfqpoint{0.000000in}{-0.027778in}}{\pgfqpoint{0.000000in}{0.000000in}}{%
\pgfpathmoveto{\pgfqpoint{0.000000in}{0.000000in}}%
\pgfpathlineto{\pgfqpoint{0.000000in}{-0.027778in}}%
\pgfusepath{stroke,fill}%
}%
\begin{pgfscope}%
\pgfsys@transformshift{1.423535in}{0.330000in}%
\pgfsys@useobject{currentmarker}{}%
\end{pgfscope}%
\end{pgfscope}%
\begin{pgfscope}%
\pgfsetbuttcap%
\pgfsetroundjoin%
\definecolor{currentfill}{rgb}{0.000000,0.000000,0.000000}%
\pgfsetfillcolor{currentfill}%
\pgfsetlinewidth{0.602250pt}%
\definecolor{currentstroke}{rgb}{0.000000,0.000000,0.000000}%
\pgfsetstrokecolor{currentstroke}%
\pgfsetdash{}{0pt}%
\pgfsys@defobject{currentmarker}{\pgfqpoint{0.000000in}{-0.027778in}}{\pgfqpoint{0.000000in}{0.000000in}}{%
\pgfpathmoveto{\pgfqpoint{0.000000in}{0.000000in}}%
\pgfpathlineto{\pgfqpoint{0.000000in}{-0.027778in}}%
\pgfusepath{stroke,fill}%
}%
\begin{pgfscope}%
\pgfsys@transformshift{1.503284in}{0.330000in}%
\pgfsys@useobject{currentmarker}{}%
\end{pgfscope}%
\end{pgfscope}%
\begin{pgfscope}%
\pgfsetbuttcap%
\pgfsetroundjoin%
\definecolor{currentfill}{rgb}{0.000000,0.000000,0.000000}%
\pgfsetfillcolor{currentfill}%
\pgfsetlinewidth{0.602250pt}%
\definecolor{currentstroke}{rgb}{0.000000,0.000000,0.000000}%
\pgfsetstrokecolor{currentstroke}%
\pgfsetdash{}{0pt}%
\pgfsys@defobject{currentmarker}{\pgfqpoint{0.000000in}{-0.027778in}}{\pgfqpoint{0.000000in}{0.000000in}}{%
\pgfpathmoveto{\pgfqpoint{0.000000in}{0.000000in}}%
\pgfpathlineto{\pgfqpoint{0.000000in}{-0.027778in}}%
\pgfusepath{stroke,fill}%
}%
\begin{pgfscope}%
\pgfsys@transformshift{1.565142in}{0.330000in}%
\pgfsys@useobject{currentmarker}{}%
\end{pgfscope}%
\end{pgfscope}%
\begin{pgfscope}%
\pgfsetbuttcap%
\pgfsetroundjoin%
\definecolor{currentfill}{rgb}{0.000000,0.000000,0.000000}%
\pgfsetfillcolor{currentfill}%
\pgfsetlinewidth{0.602250pt}%
\definecolor{currentstroke}{rgb}{0.000000,0.000000,0.000000}%
\pgfsetstrokecolor{currentstroke}%
\pgfsetdash{}{0pt}%
\pgfsys@defobject{currentmarker}{\pgfqpoint{0.000000in}{-0.027778in}}{\pgfqpoint{0.000000in}{0.000000in}}{%
\pgfpathmoveto{\pgfqpoint{0.000000in}{0.000000in}}%
\pgfpathlineto{\pgfqpoint{0.000000in}{-0.027778in}}%
\pgfusepath{stroke,fill}%
}%
\begin{pgfscope}%
\pgfsys@transformshift{1.615684in}{0.330000in}%
\pgfsys@useobject{currentmarker}{}%
\end{pgfscope}%
\end{pgfscope}%
\begin{pgfscope}%
\pgfsetbuttcap%
\pgfsetroundjoin%
\definecolor{currentfill}{rgb}{0.000000,0.000000,0.000000}%
\pgfsetfillcolor{currentfill}%
\pgfsetlinewidth{0.602250pt}%
\definecolor{currentstroke}{rgb}{0.000000,0.000000,0.000000}%
\pgfsetstrokecolor{currentstroke}%
\pgfsetdash{}{0pt}%
\pgfsys@defobject{currentmarker}{\pgfqpoint{0.000000in}{-0.027778in}}{\pgfqpoint{0.000000in}{0.000000in}}{%
\pgfpathmoveto{\pgfqpoint{0.000000in}{0.000000in}}%
\pgfpathlineto{\pgfqpoint{0.000000in}{-0.027778in}}%
\pgfusepath{stroke,fill}%
}%
\begin{pgfscope}%
\pgfsys@transformshift{1.658416in}{0.330000in}%
\pgfsys@useobject{currentmarker}{}%
\end{pgfscope}%
\end{pgfscope}%
\begin{pgfscope}%
\pgfsetbuttcap%
\pgfsetroundjoin%
\definecolor{currentfill}{rgb}{0.000000,0.000000,0.000000}%
\pgfsetfillcolor{currentfill}%
\pgfsetlinewidth{0.602250pt}%
\definecolor{currentstroke}{rgb}{0.000000,0.000000,0.000000}%
\pgfsetstrokecolor{currentstroke}%
\pgfsetdash{}{0pt}%
\pgfsys@defobject{currentmarker}{\pgfqpoint{0.000000in}{-0.027778in}}{\pgfqpoint{0.000000in}{0.000000in}}{%
\pgfpathmoveto{\pgfqpoint{0.000000in}{0.000000in}}%
\pgfpathlineto{\pgfqpoint{0.000000in}{-0.027778in}}%
\pgfusepath{stroke,fill}%
}%
\begin{pgfscope}%
\pgfsys@transformshift{1.695432in}{0.330000in}%
\pgfsys@useobject{currentmarker}{}%
\end{pgfscope}%
\end{pgfscope}%
\begin{pgfscope}%
\pgfsetbuttcap%
\pgfsetroundjoin%
\definecolor{currentfill}{rgb}{0.000000,0.000000,0.000000}%
\pgfsetfillcolor{currentfill}%
\pgfsetlinewidth{0.602250pt}%
\definecolor{currentstroke}{rgb}{0.000000,0.000000,0.000000}%
\pgfsetstrokecolor{currentstroke}%
\pgfsetdash{}{0pt}%
\pgfsys@defobject{currentmarker}{\pgfqpoint{0.000000in}{-0.027778in}}{\pgfqpoint{0.000000in}{0.000000in}}{%
\pgfpathmoveto{\pgfqpoint{0.000000in}{0.000000in}}%
\pgfpathlineto{\pgfqpoint{0.000000in}{-0.027778in}}%
\pgfusepath{stroke,fill}%
}%
\begin{pgfscope}%
\pgfsys@transformshift{1.728083in}{0.330000in}%
\pgfsys@useobject{currentmarker}{}%
\end{pgfscope}%
\end{pgfscope}%
\begin{pgfscope}%
\pgfsetbuttcap%
\pgfsetroundjoin%
\definecolor{currentfill}{rgb}{0.000000,0.000000,0.000000}%
\pgfsetfillcolor{currentfill}%
\pgfsetlinewidth{0.602250pt}%
\definecolor{currentstroke}{rgb}{0.000000,0.000000,0.000000}%
\pgfsetstrokecolor{currentstroke}%
\pgfsetdash{}{0pt}%
\pgfsys@defobject{currentmarker}{\pgfqpoint{0.000000in}{-0.027778in}}{\pgfqpoint{0.000000in}{0.000000in}}{%
\pgfpathmoveto{\pgfqpoint{0.000000in}{0.000000in}}%
\pgfpathlineto{\pgfqpoint{0.000000in}{-0.027778in}}%
\pgfusepath{stroke,fill}%
}%
\begin{pgfscope}%
\pgfsys@transformshift{1.949439in}{0.330000in}%
\pgfsys@useobject{currentmarker}{}%
\end{pgfscope}%
\end{pgfscope}%
\begin{pgfscope}%
\pgfsetbuttcap%
\pgfsetroundjoin%
\definecolor{currentfill}{rgb}{0.000000,0.000000,0.000000}%
\pgfsetfillcolor{currentfill}%
\pgfsetlinewidth{0.602250pt}%
\definecolor{currentstroke}{rgb}{0.000000,0.000000,0.000000}%
\pgfsetstrokecolor{currentstroke}%
\pgfsetdash{}{0pt}%
\pgfsys@defobject{currentmarker}{\pgfqpoint{0.000000in}{-0.027778in}}{\pgfqpoint{0.000000in}{0.000000in}}{%
\pgfpathmoveto{\pgfqpoint{0.000000in}{0.000000in}}%
\pgfpathlineto{\pgfqpoint{0.000000in}{-0.027778in}}%
\pgfusepath{stroke,fill}%
}%
\begin{pgfscope}%
\pgfsys@transformshift{2.061839in}{0.330000in}%
\pgfsys@useobject{currentmarker}{}%
\end{pgfscope}%
\end{pgfscope}%
\begin{pgfscope}%
\pgfsetbuttcap%
\pgfsetroundjoin%
\definecolor{currentfill}{rgb}{0.000000,0.000000,0.000000}%
\pgfsetfillcolor{currentfill}%
\pgfsetlinewidth{0.602250pt}%
\definecolor{currentstroke}{rgb}{0.000000,0.000000,0.000000}%
\pgfsetstrokecolor{currentstroke}%
\pgfsetdash{}{0pt}%
\pgfsys@defobject{currentmarker}{\pgfqpoint{0.000000in}{-0.027778in}}{\pgfqpoint{0.000000in}{0.000000in}}{%
\pgfpathmoveto{\pgfqpoint{0.000000in}{0.000000in}}%
\pgfpathlineto{\pgfqpoint{0.000000in}{-0.027778in}}%
\pgfusepath{stroke,fill}%
}%
\begin{pgfscope}%
\pgfsys@transformshift{2.141588in}{0.330000in}%
\pgfsys@useobject{currentmarker}{}%
\end{pgfscope}%
\end{pgfscope}%
\begin{pgfscope}%
\pgfsetbuttcap%
\pgfsetroundjoin%
\definecolor{currentfill}{rgb}{0.000000,0.000000,0.000000}%
\pgfsetfillcolor{currentfill}%
\pgfsetlinewidth{0.602250pt}%
\definecolor{currentstroke}{rgb}{0.000000,0.000000,0.000000}%
\pgfsetstrokecolor{currentstroke}%
\pgfsetdash{}{0pt}%
\pgfsys@defobject{currentmarker}{\pgfqpoint{0.000000in}{-0.027778in}}{\pgfqpoint{0.000000in}{0.000000in}}{%
\pgfpathmoveto{\pgfqpoint{0.000000in}{0.000000in}}%
\pgfpathlineto{\pgfqpoint{0.000000in}{-0.027778in}}%
\pgfusepath{stroke,fill}%
}%
\begin{pgfscope}%
\pgfsys@transformshift{2.203446in}{0.330000in}%
\pgfsys@useobject{currentmarker}{}%
\end{pgfscope}%
\end{pgfscope}%
\begin{pgfscope}%
\pgfsetbuttcap%
\pgfsetroundjoin%
\definecolor{currentfill}{rgb}{0.000000,0.000000,0.000000}%
\pgfsetfillcolor{currentfill}%
\pgfsetlinewidth{0.602250pt}%
\definecolor{currentstroke}{rgb}{0.000000,0.000000,0.000000}%
\pgfsetstrokecolor{currentstroke}%
\pgfsetdash{}{0pt}%
\pgfsys@defobject{currentmarker}{\pgfqpoint{0.000000in}{-0.027778in}}{\pgfqpoint{0.000000in}{0.000000in}}{%
\pgfpathmoveto{\pgfqpoint{0.000000in}{0.000000in}}%
\pgfpathlineto{\pgfqpoint{0.000000in}{-0.027778in}}%
\pgfusepath{stroke,fill}%
}%
\begin{pgfscope}%
\pgfsys@transformshift{2.253988in}{0.330000in}%
\pgfsys@useobject{currentmarker}{}%
\end{pgfscope}%
\end{pgfscope}%
\begin{pgfscope}%
\pgfsetbuttcap%
\pgfsetroundjoin%
\definecolor{currentfill}{rgb}{0.000000,0.000000,0.000000}%
\pgfsetfillcolor{currentfill}%
\pgfsetlinewidth{0.602250pt}%
\definecolor{currentstroke}{rgb}{0.000000,0.000000,0.000000}%
\pgfsetstrokecolor{currentstroke}%
\pgfsetdash{}{0pt}%
\pgfsys@defobject{currentmarker}{\pgfqpoint{0.000000in}{-0.027778in}}{\pgfqpoint{0.000000in}{0.000000in}}{%
\pgfpathmoveto{\pgfqpoint{0.000000in}{0.000000in}}%
\pgfpathlineto{\pgfqpoint{0.000000in}{-0.027778in}}%
\pgfusepath{stroke,fill}%
}%
\begin{pgfscope}%
\pgfsys@transformshift{2.296720in}{0.330000in}%
\pgfsys@useobject{currentmarker}{}%
\end{pgfscope}%
\end{pgfscope}%
\begin{pgfscope}%
\pgfsetbuttcap%
\pgfsetroundjoin%
\definecolor{currentfill}{rgb}{0.000000,0.000000,0.000000}%
\pgfsetfillcolor{currentfill}%
\pgfsetlinewidth{0.602250pt}%
\definecolor{currentstroke}{rgb}{0.000000,0.000000,0.000000}%
\pgfsetstrokecolor{currentstroke}%
\pgfsetdash{}{0pt}%
\pgfsys@defobject{currentmarker}{\pgfqpoint{0.000000in}{-0.027778in}}{\pgfqpoint{0.000000in}{0.000000in}}{%
\pgfpathmoveto{\pgfqpoint{0.000000in}{0.000000in}}%
\pgfpathlineto{\pgfqpoint{0.000000in}{-0.027778in}}%
\pgfusepath{stroke,fill}%
}%
\begin{pgfscope}%
\pgfsys@transformshift{2.333737in}{0.330000in}%
\pgfsys@useobject{currentmarker}{}%
\end{pgfscope}%
\end{pgfscope}%
\begin{pgfscope}%
\pgfsetbuttcap%
\pgfsetroundjoin%
\definecolor{currentfill}{rgb}{0.000000,0.000000,0.000000}%
\pgfsetfillcolor{currentfill}%
\pgfsetlinewidth{0.602250pt}%
\definecolor{currentstroke}{rgb}{0.000000,0.000000,0.000000}%
\pgfsetstrokecolor{currentstroke}%
\pgfsetdash{}{0pt}%
\pgfsys@defobject{currentmarker}{\pgfqpoint{0.000000in}{-0.027778in}}{\pgfqpoint{0.000000in}{0.000000in}}{%
\pgfpathmoveto{\pgfqpoint{0.000000in}{0.000000in}}%
\pgfpathlineto{\pgfqpoint{0.000000in}{-0.027778in}}%
\pgfusepath{stroke,fill}%
}%
\begin{pgfscope}%
\pgfsys@transformshift{2.366388in}{0.330000in}%
\pgfsys@useobject{currentmarker}{}%
\end{pgfscope}%
\end{pgfscope}%
\begin{pgfscope}%
\pgfsetbuttcap%
\pgfsetroundjoin%
\definecolor{currentfill}{rgb}{0.000000,0.000000,0.000000}%
\pgfsetfillcolor{currentfill}%
\pgfsetlinewidth{0.602250pt}%
\definecolor{currentstroke}{rgb}{0.000000,0.000000,0.000000}%
\pgfsetstrokecolor{currentstroke}%
\pgfsetdash{}{0pt}%
\pgfsys@defobject{currentmarker}{\pgfqpoint{0.000000in}{-0.027778in}}{\pgfqpoint{0.000000in}{0.000000in}}{%
\pgfpathmoveto{\pgfqpoint{0.000000in}{0.000000in}}%
\pgfpathlineto{\pgfqpoint{0.000000in}{-0.027778in}}%
\pgfusepath{stroke,fill}%
}%
\begin{pgfscope}%
\pgfsys@transformshift{2.587744in}{0.330000in}%
\pgfsys@useobject{currentmarker}{}%
\end{pgfscope}%
\end{pgfscope}%
\begin{pgfscope}%
\pgfsetbuttcap%
\pgfsetroundjoin%
\definecolor{currentfill}{rgb}{0.000000,0.000000,0.000000}%
\pgfsetfillcolor{currentfill}%
\pgfsetlinewidth{0.602250pt}%
\definecolor{currentstroke}{rgb}{0.000000,0.000000,0.000000}%
\pgfsetstrokecolor{currentstroke}%
\pgfsetdash{}{0pt}%
\pgfsys@defobject{currentmarker}{\pgfqpoint{0.000000in}{-0.027778in}}{\pgfqpoint{0.000000in}{0.000000in}}{%
\pgfpathmoveto{\pgfqpoint{0.000000in}{0.000000in}}%
\pgfpathlineto{\pgfqpoint{0.000000in}{-0.027778in}}%
\pgfusepath{stroke,fill}%
}%
\begin{pgfscope}%
\pgfsys@transformshift{2.700143in}{0.330000in}%
\pgfsys@useobject{currentmarker}{}%
\end{pgfscope}%
\end{pgfscope}%
\begin{pgfscope}%
\definecolor{textcolor}{rgb}{0.000000,0.000000,0.000000}%
\pgfsetstrokecolor{textcolor}%
\pgfsetfillcolor{textcolor}%
\pgftext[x=1.537500in,y=0.042809in,,top]{\color{textcolor}\sffamily\fontsize{10.000000}{12.000000}\selectfont \(\displaystyle N_\mathrm{QMC}\)}%
\end{pgfscope}%
\begin{pgfscope}%
\pgfsetbuttcap%
\pgfsetroundjoin%
\definecolor{currentfill}{rgb}{0.000000,0.000000,0.000000}%
\pgfsetfillcolor{currentfill}%
\pgfsetlinewidth{0.803000pt}%
\definecolor{currentstroke}{rgb}{0.000000,0.000000,0.000000}%
\pgfsetstrokecolor{currentstroke}%
\pgfsetdash{}{0pt}%
\pgfsys@defobject{currentmarker}{\pgfqpoint{-0.048611in}{0.000000in}}{\pgfqpoint{0.000000in}{0.000000in}}{%
\pgfpathmoveto{\pgfqpoint{0.000000in}{0.000000in}}%
\pgfpathlineto{\pgfqpoint{-0.048611in}{0.000000in}}%
\pgfusepath{stroke,fill}%
}%
\begin{pgfscope}%
\pgfsys@transformshift{0.375000in}{0.984888in}%
\pgfsys@useobject{currentmarker}{}%
\end{pgfscope}%
\end{pgfscope}%
\begin{pgfscope}%
\definecolor{textcolor}{rgb}{0.000000,0.000000,0.000000}%
\pgfsetstrokecolor{textcolor}%
\pgfsetfillcolor{textcolor}%
\pgftext[x=-0.010225in,y=0.932126in,left,base]{\color{textcolor}\sffamily\fontsize{10.000000}{12.000000}\selectfont \(\displaystyle {10^{-4}}\)}%
\end{pgfscope}%
\begin{pgfscope}%
\pgfsetbuttcap%
\pgfsetroundjoin%
\definecolor{currentfill}{rgb}{0.000000,0.000000,0.000000}%
\pgfsetfillcolor{currentfill}%
\pgfsetlinewidth{0.803000pt}%
\definecolor{currentstroke}{rgb}{0.000000,0.000000,0.000000}%
\pgfsetstrokecolor{currentstroke}%
\pgfsetdash{}{0pt}%
\pgfsys@defobject{currentmarker}{\pgfqpoint{-0.048611in}{0.000000in}}{\pgfqpoint{0.000000in}{0.000000in}}{%
\pgfpathmoveto{\pgfqpoint{0.000000in}{0.000000in}}%
\pgfpathlineto{\pgfqpoint{-0.048611in}{0.000000in}}%
\pgfusepath{stroke,fill}%
}%
\begin{pgfscope}%
\pgfsys@transformshift{0.375000in}{1.788611in}%
\pgfsys@useobject{currentmarker}{}%
\end{pgfscope}%
\end{pgfscope}%
\begin{pgfscope}%
\definecolor{textcolor}{rgb}{0.000000,0.000000,0.000000}%
\pgfsetstrokecolor{textcolor}%
\pgfsetfillcolor{textcolor}%
\pgftext[x=-0.010225in,y=1.735849in,left,base]{\color{textcolor}\sffamily\fontsize{10.000000}{12.000000}\selectfont \(\displaystyle {10^{-3}}\)}%
\end{pgfscope}%
\begin{pgfscope}%
\pgfsetbuttcap%
\pgfsetroundjoin%
\definecolor{currentfill}{rgb}{0.000000,0.000000,0.000000}%
\pgfsetfillcolor{currentfill}%
\pgfsetlinewidth{0.803000pt}%
\definecolor{currentstroke}{rgb}{0.000000,0.000000,0.000000}%
\pgfsetstrokecolor{currentstroke}%
\pgfsetdash{}{0pt}%
\pgfsys@defobject{currentmarker}{\pgfqpoint{-0.048611in}{0.000000in}}{\pgfqpoint{0.000000in}{0.000000in}}{%
\pgfpathmoveto{\pgfqpoint{0.000000in}{0.000000in}}%
\pgfpathlineto{\pgfqpoint{-0.048611in}{0.000000in}}%
\pgfusepath{stroke,fill}%
}%
\begin{pgfscope}%
\pgfsys@transformshift{0.375000in}{2.592334in}%
\pgfsys@useobject{currentmarker}{}%
\end{pgfscope}%
\end{pgfscope}%
\begin{pgfscope}%
\definecolor{textcolor}{rgb}{0.000000,0.000000,0.000000}%
\pgfsetstrokecolor{textcolor}%
\pgfsetfillcolor{textcolor}%
\pgftext[x=-0.010225in,y=2.539573in,left,base]{\color{textcolor}\sffamily\fontsize{10.000000}{12.000000}\selectfont \(\displaystyle {10^{-2}}\)}%
\end{pgfscope}%
\begin{pgfscope}%
\pgfsetbuttcap%
\pgfsetroundjoin%
\definecolor{currentfill}{rgb}{0.000000,0.000000,0.000000}%
\pgfsetfillcolor{currentfill}%
\pgfsetlinewidth{0.602250pt}%
\definecolor{currentstroke}{rgb}{0.000000,0.000000,0.000000}%
\pgfsetstrokecolor{currentstroke}%
\pgfsetdash{}{0pt}%
\pgfsys@defobject{currentmarker}{\pgfqpoint{-0.027778in}{0.000000in}}{\pgfqpoint{0.000000in}{0.000000in}}{%
\pgfpathmoveto{\pgfqpoint{0.000000in}{0.000000in}}%
\pgfpathlineto{\pgfqpoint{-0.027778in}{0.000000in}}%
\pgfusepath{stroke,fill}%
}%
\begin{pgfscope}%
\pgfsys@transformshift{0.375000in}{0.423109in}%
\pgfsys@useobject{currentmarker}{}%
\end{pgfscope}%
\end{pgfscope}%
\begin{pgfscope}%
\pgfsetbuttcap%
\pgfsetroundjoin%
\definecolor{currentfill}{rgb}{0.000000,0.000000,0.000000}%
\pgfsetfillcolor{currentfill}%
\pgfsetlinewidth{0.602250pt}%
\definecolor{currentstroke}{rgb}{0.000000,0.000000,0.000000}%
\pgfsetstrokecolor{currentstroke}%
\pgfsetdash{}{0pt}%
\pgfsys@defobject{currentmarker}{\pgfqpoint{-0.027778in}{0.000000in}}{\pgfqpoint{0.000000in}{0.000000in}}{%
\pgfpathmoveto{\pgfqpoint{0.000000in}{0.000000in}}%
\pgfpathlineto{\pgfqpoint{-0.027778in}{0.000000in}}%
\pgfusepath{stroke,fill}%
}%
\begin{pgfscope}%
\pgfsys@transformshift{0.375000in}{0.564638in}%
\pgfsys@useobject{currentmarker}{}%
\end{pgfscope}%
\end{pgfscope}%
\begin{pgfscope}%
\pgfsetbuttcap%
\pgfsetroundjoin%
\definecolor{currentfill}{rgb}{0.000000,0.000000,0.000000}%
\pgfsetfillcolor{currentfill}%
\pgfsetlinewidth{0.602250pt}%
\definecolor{currentstroke}{rgb}{0.000000,0.000000,0.000000}%
\pgfsetstrokecolor{currentstroke}%
\pgfsetdash{}{0pt}%
\pgfsys@defobject{currentmarker}{\pgfqpoint{-0.027778in}{0.000000in}}{\pgfqpoint{0.000000in}{0.000000in}}{%
\pgfpathmoveto{\pgfqpoint{0.000000in}{0.000000in}}%
\pgfpathlineto{\pgfqpoint{-0.027778in}{0.000000in}}%
\pgfusepath{stroke,fill}%
}%
\begin{pgfscope}%
\pgfsys@transformshift{0.375000in}{0.665054in}%
\pgfsys@useobject{currentmarker}{}%
\end{pgfscope}%
\end{pgfscope}%
\begin{pgfscope}%
\pgfsetbuttcap%
\pgfsetroundjoin%
\definecolor{currentfill}{rgb}{0.000000,0.000000,0.000000}%
\pgfsetfillcolor{currentfill}%
\pgfsetlinewidth{0.602250pt}%
\definecolor{currentstroke}{rgb}{0.000000,0.000000,0.000000}%
\pgfsetstrokecolor{currentstroke}%
\pgfsetdash{}{0pt}%
\pgfsys@defobject{currentmarker}{\pgfqpoint{-0.027778in}{0.000000in}}{\pgfqpoint{0.000000in}{0.000000in}}{%
\pgfpathmoveto{\pgfqpoint{0.000000in}{0.000000in}}%
\pgfpathlineto{\pgfqpoint{-0.027778in}{0.000000in}}%
\pgfusepath{stroke,fill}%
}%
\begin{pgfscope}%
\pgfsys@transformshift{0.375000in}{0.742943in}%
\pgfsys@useobject{currentmarker}{}%
\end{pgfscope}%
\end{pgfscope}%
\begin{pgfscope}%
\pgfsetbuttcap%
\pgfsetroundjoin%
\definecolor{currentfill}{rgb}{0.000000,0.000000,0.000000}%
\pgfsetfillcolor{currentfill}%
\pgfsetlinewidth{0.602250pt}%
\definecolor{currentstroke}{rgb}{0.000000,0.000000,0.000000}%
\pgfsetstrokecolor{currentstroke}%
\pgfsetdash{}{0pt}%
\pgfsys@defobject{currentmarker}{\pgfqpoint{-0.027778in}{0.000000in}}{\pgfqpoint{0.000000in}{0.000000in}}{%
\pgfpathmoveto{\pgfqpoint{0.000000in}{0.000000in}}%
\pgfpathlineto{\pgfqpoint{-0.027778in}{0.000000in}}%
\pgfusepath{stroke,fill}%
}%
\begin{pgfscope}%
\pgfsys@transformshift{0.375000in}{0.806583in}%
\pgfsys@useobject{currentmarker}{}%
\end{pgfscope}%
\end{pgfscope}%
\begin{pgfscope}%
\pgfsetbuttcap%
\pgfsetroundjoin%
\definecolor{currentfill}{rgb}{0.000000,0.000000,0.000000}%
\pgfsetfillcolor{currentfill}%
\pgfsetlinewidth{0.602250pt}%
\definecolor{currentstroke}{rgb}{0.000000,0.000000,0.000000}%
\pgfsetstrokecolor{currentstroke}%
\pgfsetdash{}{0pt}%
\pgfsys@defobject{currentmarker}{\pgfqpoint{-0.027778in}{0.000000in}}{\pgfqpoint{0.000000in}{0.000000in}}{%
\pgfpathmoveto{\pgfqpoint{0.000000in}{0.000000in}}%
\pgfpathlineto{\pgfqpoint{-0.027778in}{0.000000in}}%
\pgfusepath{stroke,fill}%
}%
\begin{pgfscope}%
\pgfsys@transformshift{0.375000in}{0.860389in}%
\pgfsys@useobject{currentmarker}{}%
\end{pgfscope}%
\end{pgfscope}%
\begin{pgfscope}%
\pgfsetbuttcap%
\pgfsetroundjoin%
\definecolor{currentfill}{rgb}{0.000000,0.000000,0.000000}%
\pgfsetfillcolor{currentfill}%
\pgfsetlinewidth{0.602250pt}%
\definecolor{currentstroke}{rgb}{0.000000,0.000000,0.000000}%
\pgfsetstrokecolor{currentstroke}%
\pgfsetdash{}{0pt}%
\pgfsys@defobject{currentmarker}{\pgfqpoint{-0.027778in}{0.000000in}}{\pgfqpoint{0.000000in}{0.000000in}}{%
\pgfpathmoveto{\pgfqpoint{0.000000in}{0.000000in}}%
\pgfpathlineto{\pgfqpoint{-0.027778in}{0.000000in}}%
\pgfusepath{stroke,fill}%
}%
\begin{pgfscope}%
\pgfsys@transformshift{0.375000in}{0.906999in}%
\pgfsys@useobject{currentmarker}{}%
\end{pgfscope}%
\end{pgfscope}%
\begin{pgfscope}%
\pgfsetbuttcap%
\pgfsetroundjoin%
\definecolor{currentfill}{rgb}{0.000000,0.000000,0.000000}%
\pgfsetfillcolor{currentfill}%
\pgfsetlinewidth{0.602250pt}%
\definecolor{currentstroke}{rgb}{0.000000,0.000000,0.000000}%
\pgfsetstrokecolor{currentstroke}%
\pgfsetdash{}{0pt}%
\pgfsys@defobject{currentmarker}{\pgfqpoint{-0.027778in}{0.000000in}}{\pgfqpoint{0.000000in}{0.000000in}}{%
\pgfpathmoveto{\pgfqpoint{0.000000in}{0.000000in}}%
\pgfpathlineto{\pgfqpoint{-0.027778in}{0.000000in}}%
\pgfusepath{stroke,fill}%
}%
\begin{pgfscope}%
\pgfsys@transformshift{0.375000in}{0.948111in}%
\pgfsys@useobject{currentmarker}{}%
\end{pgfscope}%
\end{pgfscope}%
\begin{pgfscope}%
\pgfsetbuttcap%
\pgfsetroundjoin%
\definecolor{currentfill}{rgb}{0.000000,0.000000,0.000000}%
\pgfsetfillcolor{currentfill}%
\pgfsetlinewidth{0.602250pt}%
\definecolor{currentstroke}{rgb}{0.000000,0.000000,0.000000}%
\pgfsetstrokecolor{currentstroke}%
\pgfsetdash{}{0pt}%
\pgfsys@defobject{currentmarker}{\pgfqpoint{-0.027778in}{0.000000in}}{\pgfqpoint{0.000000in}{0.000000in}}{%
\pgfpathmoveto{\pgfqpoint{0.000000in}{0.000000in}}%
\pgfpathlineto{\pgfqpoint{-0.027778in}{0.000000in}}%
\pgfusepath{stroke,fill}%
}%
\begin{pgfscope}%
\pgfsys@transformshift{0.375000in}{1.226832in}%
\pgfsys@useobject{currentmarker}{}%
\end{pgfscope}%
\end{pgfscope}%
\begin{pgfscope}%
\pgfsetbuttcap%
\pgfsetroundjoin%
\definecolor{currentfill}{rgb}{0.000000,0.000000,0.000000}%
\pgfsetfillcolor{currentfill}%
\pgfsetlinewidth{0.602250pt}%
\definecolor{currentstroke}{rgb}{0.000000,0.000000,0.000000}%
\pgfsetstrokecolor{currentstroke}%
\pgfsetdash{}{0pt}%
\pgfsys@defobject{currentmarker}{\pgfqpoint{-0.027778in}{0.000000in}}{\pgfqpoint{0.000000in}{0.000000in}}{%
\pgfpathmoveto{\pgfqpoint{0.000000in}{0.000000in}}%
\pgfpathlineto{\pgfqpoint{-0.027778in}{0.000000in}}%
\pgfusepath{stroke,fill}%
}%
\begin{pgfscope}%
\pgfsys@transformshift{0.375000in}{1.368361in}%
\pgfsys@useobject{currentmarker}{}%
\end{pgfscope}%
\end{pgfscope}%
\begin{pgfscope}%
\pgfsetbuttcap%
\pgfsetroundjoin%
\definecolor{currentfill}{rgb}{0.000000,0.000000,0.000000}%
\pgfsetfillcolor{currentfill}%
\pgfsetlinewidth{0.602250pt}%
\definecolor{currentstroke}{rgb}{0.000000,0.000000,0.000000}%
\pgfsetstrokecolor{currentstroke}%
\pgfsetdash{}{0pt}%
\pgfsys@defobject{currentmarker}{\pgfqpoint{-0.027778in}{0.000000in}}{\pgfqpoint{0.000000in}{0.000000in}}{%
\pgfpathmoveto{\pgfqpoint{0.000000in}{0.000000in}}%
\pgfpathlineto{\pgfqpoint{-0.027778in}{0.000000in}}%
\pgfusepath{stroke,fill}%
}%
\begin{pgfscope}%
\pgfsys@transformshift{0.375000in}{1.468777in}%
\pgfsys@useobject{currentmarker}{}%
\end{pgfscope}%
\end{pgfscope}%
\begin{pgfscope}%
\pgfsetbuttcap%
\pgfsetroundjoin%
\definecolor{currentfill}{rgb}{0.000000,0.000000,0.000000}%
\pgfsetfillcolor{currentfill}%
\pgfsetlinewidth{0.602250pt}%
\definecolor{currentstroke}{rgb}{0.000000,0.000000,0.000000}%
\pgfsetstrokecolor{currentstroke}%
\pgfsetdash{}{0pt}%
\pgfsys@defobject{currentmarker}{\pgfqpoint{-0.027778in}{0.000000in}}{\pgfqpoint{0.000000in}{0.000000in}}{%
\pgfpathmoveto{\pgfqpoint{0.000000in}{0.000000in}}%
\pgfpathlineto{\pgfqpoint{-0.027778in}{0.000000in}}%
\pgfusepath{stroke,fill}%
}%
\begin{pgfscope}%
\pgfsys@transformshift{0.375000in}{1.546666in}%
\pgfsys@useobject{currentmarker}{}%
\end{pgfscope}%
\end{pgfscope}%
\begin{pgfscope}%
\pgfsetbuttcap%
\pgfsetroundjoin%
\definecolor{currentfill}{rgb}{0.000000,0.000000,0.000000}%
\pgfsetfillcolor{currentfill}%
\pgfsetlinewidth{0.602250pt}%
\definecolor{currentstroke}{rgb}{0.000000,0.000000,0.000000}%
\pgfsetstrokecolor{currentstroke}%
\pgfsetdash{}{0pt}%
\pgfsys@defobject{currentmarker}{\pgfqpoint{-0.027778in}{0.000000in}}{\pgfqpoint{0.000000in}{0.000000in}}{%
\pgfpathmoveto{\pgfqpoint{0.000000in}{0.000000in}}%
\pgfpathlineto{\pgfqpoint{-0.027778in}{0.000000in}}%
\pgfusepath{stroke,fill}%
}%
\begin{pgfscope}%
\pgfsys@transformshift{0.375000in}{1.610306in}%
\pgfsys@useobject{currentmarker}{}%
\end{pgfscope}%
\end{pgfscope}%
\begin{pgfscope}%
\pgfsetbuttcap%
\pgfsetroundjoin%
\definecolor{currentfill}{rgb}{0.000000,0.000000,0.000000}%
\pgfsetfillcolor{currentfill}%
\pgfsetlinewidth{0.602250pt}%
\definecolor{currentstroke}{rgb}{0.000000,0.000000,0.000000}%
\pgfsetstrokecolor{currentstroke}%
\pgfsetdash{}{0pt}%
\pgfsys@defobject{currentmarker}{\pgfqpoint{-0.027778in}{0.000000in}}{\pgfqpoint{0.000000in}{0.000000in}}{%
\pgfpathmoveto{\pgfqpoint{0.000000in}{0.000000in}}%
\pgfpathlineto{\pgfqpoint{-0.027778in}{0.000000in}}%
\pgfusepath{stroke,fill}%
}%
\begin{pgfscope}%
\pgfsys@transformshift{0.375000in}{1.664113in}%
\pgfsys@useobject{currentmarker}{}%
\end{pgfscope}%
\end{pgfscope}%
\begin{pgfscope}%
\pgfsetbuttcap%
\pgfsetroundjoin%
\definecolor{currentfill}{rgb}{0.000000,0.000000,0.000000}%
\pgfsetfillcolor{currentfill}%
\pgfsetlinewidth{0.602250pt}%
\definecolor{currentstroke}{rgb}{0.000000,0.000000,0.000000}%
\pgfsetstrokecolor{currentstroke}%
\pgfsetdash{}{0pt}%
\pgfsys@defobject{currentmarker}{\pgfqpoint{-0.027778in}{0.000000in}}{\pgfqpoint{0.000000in}{0.000000in}}{%
\pgfpathmoveto{\pgfqpoint{0.000000in}{0.000000in}}%
\pgfpathlineto{\pgfqpoint{-0.027778in}{0.000000in}}%
\pgfusepath{stroke,fill}%
}%
\begin{pgfscope}%
\pgfsys@transformshift{0.375000in}{1.710722in}%
\pgfsys@useobject{currentmarker}{}%
\end{pgfscope}%
\end{pgfscope}%
\begin{pgfscope}%
\pgfsetbuttcap%
\pgfsetroundjoin%
\definecolor{currentfill}{rgb}{0.000000,0.000000,0.000000}%
\pgfsetfillcolor{currentfill}%
\pgfsetlinewidth{0.602250pt}%
\definecolor{currentstroke}{rgb}{0.000000,0.000000,0.000000}%
\pgfsetstrokecolor{currentstroke}%
\pgfsetdash{}{0pt}%
\pgfsys@defobject{currentmarker}{\pgfqpoint{-0.027778in}{0.000000in}}{\pgfqpoint{0.000000in}{0.000000in}}{%
\pgfpathmoveto{\pgfqpoint{0.000000in}{0.000000in}}%
\pgfpathlineto{\pgfqpoint{-0.027778in}{0.000000in}}%
\pgfusepath{stroke,fill}%
}%
\begin{pgfscope}%
\pgfsys@transformshift{0.375000in}{1.751835in}%
\pgfsys@useobject{currentmarker}{}%
\end{pgfscope}%
\end{pgfscope}%
\begin{pgfscope}%
\pgfsetbuttcap%
\pgfsetroundjoin%
\definecolor{currentfill}{rgb}{0.000000,0.000000,0.000000}%
\pgfsetfillcolor{currentfill}%
\pgfsetlinewidth{0.602250pt}%
\definecolor{currentstroke}{rgb}{0.000000,0.000000,0.000000}%
\pgfsetstrokecolor{currentstroke}%
\pgfsetdash{}{0pt}%
\pgfsys@defobject{currentmarker}{\pgfqpoint{-0.027778in}{0.000000in}}{\pgfqpoint{0.000000in}{0.000000in}}{%
\pgfpathmoveto{\pgfqpoint{0.000000in}{0.000000in}}%
\pgfpathlineto{\pgfqpoint{-0.027778in}{0.000000in}}%
\pgfusepath{stroke,fill}%
}%
\begin{pgfscope}%
\pgfsys@transformshift{0.375000in}{2.030556in}%
\pgfsys@useobject{currentmarker}{}%
\end{pgfscope}%
\end{pgfscope}%
\begin{pgfscope}%
\pgfsetbuttcap%
\pgfsetroundjoin%
\definecolor{currentfill}{rgb}{0.000000,0.000000,0.000000}%
\pgfsetfillcolor{currentfill}%
\pgfsetlinewidth{0.602250pt}%
\definecolor{currentstroke}{rgb}{0.000000,0.000000,0.000000}%
\pgfsetstrokecolor{currentstroke}%
\pgfsetdash{}{0pt}%
\pgfsys@defobject{currentmarker}{\pgfqpoint{-0.027778in}{0.000000in}}{\pgfqpoint{0.000000in}{0.000000in}}{%
\pgfpathmoveto{\pgfqpoint{0.000000in}{0.000000in}}%
\pgfpathlineto{\pgfqpoint{-0.027778in}{0.000000in}}%
\pgfusepath{stroke,fill}%
}%
\begin{pgfscope}%
\pgfsys@transformshift{0.375000in}{2.172084in}%
\pgfsys@useobject{currentmarker}{}%
\end{pgfscope}%
\end{pgfscope}%
\begin{pgfscope}%
\pgfsetbuttcap%
\pgfsetroundjoin%
\definecolor{currentfill}{rgb}{0.000000,0.000000,0.000000}%
\pgfsetfillcolor{currentfill}%
\pgfsetlinewidth{0.602250pt}%
\definecolor{currentstroke}{rgb}{0.000000,0.000000,0.000000}%
\pgfsetstrokecolor{currentstroke}%
\pgfsetdash{}{0pt}%
\pgfsys@defobject{currentmarker}{\pgfqpoint{-0.027778in}{0.000000in}}{\pgfqpoint{0.000000in}{0.000000in}}{%
\pgfpathmoveto{\pgfqpoint{0.000000in}{0.000000in}}%
\pgfpathlineto{\pgfqpoint{-0.027778in}{0.000000in}}%
\pgfusepath{stroke,fill}%
}%
\begin{pgfscope}%
\pgfsys@transformshift{0.375000in}{2.272501in}%
\pgfsys@useobject{currentmarker}{}%
\end{pgfscope}%
\end{pgfscope}%
\begin{pgfscope}%
\pgfsetbuttcap%
\pgfsetroundjoin%
\definecolor{currentfill}{rgb}{0.000000,0.000000,0.000000}%
\pgfsetfillcolor{currentfill}%
\pgfsetlinewidth{0.602250pt}%
\definecolor{currentstroke}{rgb}{0.000000,0.000000,0.000000}%
\pgfsetstrokecolor{currentstroke}%
\pgfsetdash{}{0pt}%
\pgfsys@defobject{currentmarker}{\pgfqpoint{-0.027778in}{0.000000in}}{\pgfqpoint{0.000000in}{0.000000in}}{%
\pgfpathmoveto{\pgfqpoint{0.000000in}{0.000000in}}%
\pgfpathlineto{\pgfqpoint{-0.027778in}{0.000000in}}%
\pgfusepath{stroke,fill}%
}%
\begin{pgfscope}%
\pgfsys@transformshift{0.375000in}{2.350389in}%
\pgfsys@useobject{currentmarker}{}%
\end{pgfscope}%
\end{pgfscope}%
\begin{pgfscope}%
\pgfsetbuttcap%
\pgfsetroundjoin%
\definecolor{currentfill}{rgb}{0.000000,0.000000,0.000000}%
\pgfsetfillcolor{currentfill}%
\pgfsetlinewidth{0.602250pt}%
\definecolor{currentstroke}{rgb}{0.000000,0.000000,0.000000}%
\pgfsetstrokecolor{currentstroke}%
\pgfsetdash{}{0pt}%
\pgfsys@defobject{currentmarker}{\pgfqpoint{-0.027778in}{0.000000in}}{\pgfqpoint{0.000000in}{0.000000in}}{%
\pgfpathmoveto{\pgfqpoint{0.000000in}{0.000000in}}%
\pgfpathlineto{\pgfqpoint{-0.027778in}{0.000000in}}%
\pgfusepath{stroke,fill}%
}%
\begin{pgfscope}%
\pgfsys@transformshift{0.375000in}{2.414029in}%
\pgfsys@useobject{currentmarker}{}%
\end{pgfscope}%
\end{pgfscope}%
\begin{pgfscope}%
\pgfsetbuttcap%
\pgfsetroundjoin%
\definecolor{currentfill}{rgb}{0.000000,0.000000,0.000000}%
\pgfsetfillcolor{currentfill}%
\pgfsetlinewidth{0.602250pt}%
\definecolor{currentstroke}{rgb}{0.000000,0.000000,0.000000}%
\pgfsetstrokecolor{currentstroke}%
\pgfsetdash{}{0pt}%
\pgfsys@defobject{currentmarker}{\pgfqpoint{-0.027778in}{0.000000in}}{\pgfqpoint{0.000000in}{0.000000in}}{%
\pgfpathmoveto{\pgfqpoint{0.000000in}{0.000000in}}%
\pgfpathlineto{\pgfqpoint{-0.027778in}{0.000000in}}%
\pgfusepath{stroke,fill}%
}%
\begin{pgfscope}%
\pgfsys@transformshift{0.375000in}{2.467836in}%
\pgfsys@useobject{currentmarker}{}%
\end{pgfscope}%
\end{pgfscope}%
\begin{pgfscope}%
\pgfsetbuttcap%
\pgfsetroundjoin%
\definecolor{currentfill}{rgb}{0.000000,0.000000,0.000000}%
\pgfsetfillcolor{currentfill}%
\pgfsetlinewidth{0.602250pt}%
\definecolor{currentstroke}{rgb}{0.000000,0.000000,0.000000}%
\pgfsetstrokecolor{currentstroke}%
\pgfsetdash{}{0pt}%
\pgfsys@defobject{currentmarker}{\pgfqpoint{-0.027778in}{0.000000in}}{\pgfqpoint{0.000000in}{0.000000in}}{%
\pgfpathmoveto{\pgfqpoint{0.000000in}{0.000000in}}%
\pgfpathlineto{\pgfqpoint{-0.027778in}{0.000000in}}%
\pgfusepath{stroke,fill}%
}%
\begin{pgfscope}%
\pgfsys@transformshift{0.375000in}{2.514445in}%
\pgfsys@useobject{currentmarker}{}%
\end{pgfscope}%
\end{pgfscope}%
\begin{pgfscope}%
\pgfsetbuttcap%
\pgfsetroundjoin%
\definecolor{currentfill}{rgb}{0.000000,0.000000,0.000000}%
\pgfsetfillcolor{currentfill}%
\pgfsetlinewidth{0.602250pt}%
\definecolor{currentstroke}{rgb}{0.000000,0.000000,0.000000}%
\pgfsetstrokecolor{currentstroke}%
\pgfsetdash{}{0pt}%
\pgfsys@defobject{currentmarker}{\pgfqpoint{-0.027778in}{0.000000in}}{\pgfqpoint{0.000000in}{0.000000in}}{%
\pgfpathmoveto{\pgfqpoint{0.000000in}{0.000000in}}%
\pgfpathlineto{\pgfqpoint{-0.027778in}{0.000000in}}%
\pgfusepath{stroke,fill}%
}%
\begin{pgfscope}%
\pgfsys@transformshift{0.375000in}{2.555558in}%
\pgfsys@useobject{currentmarker}{}%
\end{pgfscope}%
\end{pgfscope}%
\begin{pgfscope}%
\definecolor{textcolor}{rgb}{0.000000,0.000000,0.000000}%
\pgfsetstrokecolor{textcolor}%
\pgfsetfillcolor{textcolor}%
\pgftext[x=-0.065780in,y=1.485000in,,bottom,rotate=90.000000]{\color{textcolor}\sffamily\fontsize{10.000000}{12.000000}\selectfont QMC Error}%
\end{pgfscope}%
\begin{pgfscope}%
\pgfpathrectangle{\pgfqpoint{0.375000in}{0.330000in}}{\pgfqpoint{2.325000in}{2.310000in}}%
\pgfusepath{clip}%
\pgfsetbuttcap%
\pgfsetroundjoin%
\definecolor{currentfill}{rgb}{0.000000,0.000000,0.000000}%
\pgfsetfillcolor{currentfill}%
\pgfsetlinewidth{1.003750pt}%
\definecolor{currentstroke}{rgb}{0.000000,0.000000,0.000000}%
\pgfsetstrokecolor{currentstroke}%
\pgfsetdash{}{0pt}%
\pgfsys@defobject{currentmarker}{\pgfqpoint{-0.041667in}{-0.041667in}}{\pgfqpoint{0.041667in}{0.041667in}}{%
\pgfpathmoveto{\pgfqpoint{0.000000in}{-0.041667in}}%
\pgfpathcurveto{\pgfqpoint{0.011050in}{-0.041667in}}{\pgfqpoint{0.021649in}{-0.037276in}}{\pgfqpoint{0.029463in}{-0.029463in}}%
\pgfpathcurveto{\pgfqpoint{0.037276in}{-0.021649in}}{\pgfqpoint{0.041667in}{-0.011050in}}{\pgfqpoint{0.041667in}{0.000000in}}%
\pgfpathcurveto{\pgfqpoint{0.041667in}{0.011050in}}{\pgfqpoint{0.037276in}{0.021649in}}{\pgfqpoint{0.029463in}{0.029463in}}%
\pgfpathcurveto{\pgfqpoint{0.021649in}{0.037276in}}{\pgfqpoint{0.011050in}{0.041667in}}{\pgfqpoint{0.000000in}{0.041667in}}%
\pgfpathcurveto{\pgfqpoint{-0.011050in}{0.041667in}}{\pgfqpoint{-0.021649in}{0.037276in}}{\pgfqpoint{-0.029463in}{0.029463in}}%
\pgfpathcurveto{\pgfqpoint{-0.037276in}{0.021649in}}{\pgfqpoint{-0.041667in}{0.011050in}}{\pgfqpoint{-0.041667in}{0.000000in}}%
\pgfpathcurveto{\pgfqpoint{-0.041667in}{-0.011050in}}{\pgfqpoint{-0.037276in}{-0.021649in}}{\pgfqpoint{-0.029463in}{-0.029463in}}%
\pgfpathcurveto{\pgfqpoint{-0.021649in}{-0.037276in}}{\pgfqpoint{-0.011050in}{-0.041667in}}{\pgfqpoint{0.000000in}{-0.041667in}}%
\pgfpathclose%
\pgfusepath{stroke,fill}%
}%
\begin{pgfscope}%
\pgfsys@transformshift{0.480682in}{2.535000in}%
\pgfsys@useobject{currentmarker}{}%
\end{pgfscope}%
\begin{pgfscope}%
\pgfsys@transformshift{0.672831in}{2.318587in}%
\pgfsys@useobject{currentmarker}{}%
\end{pgfscope}%
\begin{pgfscope}%
\pgfsys@transformshift{0.864979in}{2.000669in}%
\pgfsys@useobject{currentmarker}{}%
\end{pgfscope}%
\begin{pgfscope}%
\pgfsys@transformshift{1.057128in}{1.734669in}%
\pgfsys@useobject{currentmarker}{}%
\end{pgfscope}%
\begin{pgfscope}%
\pgfsys@transformshift{1.249277in}{1.854668in}%
\pgfsys@useobject{currentmarker}{}%
\end{pgfscope}%
\begin{pgfscope}%
\pgfsys@transformshift{1.441426in}{1.176551in}%
\pgfsys@useobject{currentmarker}{}%
\end{pgfscope}%
\begin{pgfscope}%
\pgfsys@transformshift{1.633574in}{1.349518in}%
\pgfsys@useobject{currentmarker}{}%
\end{pgfscope}%
\begin{pgfscope}%
\pgfsys@transformshift{1.825723in}{1.329626in}%
\pgfsys@useobject{currentmarker}{}%
\end{pgfscope}%
\begin{pgfscope}%
\pgfsys@transformshift{2.017872in}{0.816388in}%
\pgfsys@useobject{currentmarker}{}%
\end{pgfscope}%
\begin{pgfscope}%
\pgfsys@transformshift{2.210021in}{0.808596in}%
\pgfsys@useobject{currentmarker}{}%
\end{pgfscope}%
\begin{pgfscope}%
\pgfsys@transformshift{2.402169in}{0.677816in}%
\pgfsys@useobject{currentmarker}{}%
\end{pgfscope}%
\begin{pgfscope}%
\pgfsys@transformshift{2.594318in}{0.500063in}%
\pgfsys@useobject{currentmarker}{}%
\end{pgfscope}%
\end{pgfscope}%
\begin{pgfscope}%
\pgfpathrectangle{\pgfqpoint{0.375000in}{0.330000in}}{\pgfqpoint{2.325000in}{2.310000in}}%
\pgfusepath{clip}%
\pgfsetbuttcap%
\pgfsetroundjoin%
\pgfsetlinewidth{1.505625pt}%
\definecolor{currentstroke}{rgb}{0.000000,0.000000,0.000000}%
\pgfsetstrokecolor{currentstroke}%
\pgfsetdash{{5.550000pt}{2.400000pt}}{0.000000pt}%
\pgfpathmoveto{\pgfqpoint{0.480682in}{2.415358in}}%
\pgfpathlineto{\pgfqpoint{2.594318in}{0.435000in}}%
\pgfusepath{stroke}%
\end{pgfscope}%
\begin{pgfscope}%
\pgfsetrectcap%
\pgfsetmiterjoin%
\pgfsetlinewidth{0.803000pt}%
\definecolor{currentstroke}{rgb}{0.000000,0.000000,0.000000}%
\pgfsetstrokecolor{currentstroke}%
\pgfsetdash{}{0pt}%
\pgfpathmoveto{\pgfqpoint{0.375000in}{0.330000in}}%
\pgfpathlineto{\pgfqpoint{0.375000in}{2.640000in}}%
\pgfusepath{stroke}%
\end{pgfscope}%
\begin{pgfscope}%
\pgfsetrectcap%
\pgfsetmiterjoin%
\pgfsetlinewidth{0.803000pt}%
\definecolor{currentstroke}{rgb}{0.000000,0.000000,0.000000}%
\pgfsetstrokecolor{currentstroke}%
\pgfsetdash{}{0pt}%
\pgfpathmoveto{\pgfqpoint{2.700000in}{0.330000in}}%
\pgfpathlineto{\pgfqpoint{2.700000in}{2.640000in}}%
\pgfusepath{stroke}%
\end{pgfscope}%
\begin{pgfscope}%
\pgfsetrectcap%
\pgfsetmiterjoin%
\pgfsetlinewidth{0.803000pt}%
\definecolor{currentstroke}{rgb}{0.000000,0.000000,0.000000}%
\pgfsetstrokecolor{currentstroke}%
\pgfsetdash{}{0pt}%
\pgfpathmoveto{\pgfqpoint{0.375000in}{0.330000in}}%
\pgfpathlineto{\pgfqpoint{2.700000in}{0.330000in}}%
\pgfusepath{stroke}%
\end{pgfscope}%
\begin{pgfscope}%
\pgfsetrectcap%
\pgfsetmiterjoin%
\pgfsetlinewidth{0.803000pt}%
\definecolor{currentstroke}{rgb}{0.000000,0.000000,0.000000}%
\pgfsetstrokecolor{currentstroke}%
\pgfsetdash{}{0pt}%
\pgfpathmoveto{\pgfqpoint{0.375000in}{2.640000in}}%
\pgfpathlineto{\pgfqpoint{2.700000in}{2.640000in}}%
\pgfusepath{stroke}%
\end{pgfscope}%
\begin{pgfscope}%
\pgfsetbuttcap%
\pgfsetmiterjoin%
\definecolor{currentfill}{rgb}{1.000000,1.000000,1.000000}%
\pgfsetfillcolor{currentfill}%
\pgfsetfillopacity{0.800000}%
\pgfsetlinewidth{1.003750pt}%
\definecolor{currentstroke}{rgb}{0.800000,0.800000,0.800000}%
\pgfsetstrokecolor{currentstroke}%
\pgfsetstrokeopacity{0.800000}%
\pgfsetdash{}{0pt}%
\pgfpathmoveto{\pgfqpoint{1.242243in}{2.275267in}}%
\pgfpathlineto{\pgfqpoint{2.602778in}{2.275267in}}%
\pgfpathquadraticcurveto{\pgfqpoint{2.630556in}{2.275267in}}{\pgfqpoint{2.630556in}{2.303045in}}%
\pgfpathlineto{\pgfqpoint{2.630556in}{2.542778in}}%
\pgfpathquadraticcurveto{\pgfqpoint{2.630556in}{2.570556in}}{\pgfqpoint{2.602778in}{2.570556in}}%
\pgfpathlineto{\pgfqpoint{1.242243in}{2.570556in}}%
\pgfpathquadraticcurveto{\pgfqpoint{1.214465in}{2.570556in}}{\pgfqpoint{1.214465in}{2.542778in}}%
\pgfpathlineto{\pgfqpoint{1.214465in}{2.303045in}}%
\pgfpathquadraticcurveto{\pgfqpoint{1.214465in}{2.275267in}}{\pgfqpoint{1.242243in}{2.275267in}}%
\pgfpathclose%
\pgfusepath{stroke,fill}%
\end{pgfscope}%
\begin{pgfscope}%
\pgfsetbuttcap%
\pgfsetroundjoin%
\pgfsetlinewidth{1.505625pt}%
\definecolor{currentstroke}{rgb}{0.000000,0.000000,0.000000}%
\pgfsetstrokecolor{currentstroke}%
\pgfsetdash{{5.550000pt}{2.400000pt}}{0.000000pt}%
\pgfpathmoveto{\pgfqpoint{1.270021in}{2.441543in}}%
\pgfpathlineto{\pgfqpoint{1.547799in}{2.441543in}}%
\pgfusepath{stroke}%
\end{pgfscope}%
\begin{pgfscope}%
\definecolor{textcolor}{rgb}{0.000000,0.000000,0.000000}%
\pgfsetstrokecolor{textcolor}%
\pgfsetfillcolor{textcolor}%
\pgftext[x=1.658910in,y=2.392932in,left,base]{\color{textcolor}\sffamily\fontsize{10.000000}{12.000000}\selectfont \(\displaystyle 0.0060N_{\mathrm{QMC}}^{-0.7441}\)}%
\end{pgfscope}%
\end{pgfpicture}%
\makeatother%
\endgroup%

\caption[Quasi-Monte-Carlo error, for $Q=u(\bzero)$ and $k=40$.]{Quasi-Monte-Carlo error with increasing number of Quasi-Monte-Carlo points, for $Q=u(\bzero)$ and $k=40$. \label{fig:qmcorigin40}}
\end{figure}
\begin{figure}[h]
    \centering
    %% Creator: Matplotlib, PGF backend
%%
%% To include the figure in your LaTeX document, write
%%   \input{<filename>.pgf}
%%
%% Make sure the required packages are loaded in your preamble
%%   \usepackage{pgf}
%%
%% Figures using additional raster images can only be included by \input if
%% they are in the same directory as the main LaTeX file. For loading figures
%% from other directories you can use the `import` package
%%   \usepackage{import}
%% and then include the figures with
%%   \import{<path to file>}{<filename>.pgf}
%%
%% Matplotlib used the following preamble
%%   \usepackage{fontspec}
%%   \setmainfont{DejaVuSerif.ttf}[Path=/home/owen/progs/firedrake-complex/firedrake/lib/python3.5/site-packages/matplotlib/mpl-data/fonts/ttf/]
%%   \setsansfont{DejaVuSans.ttf}[Path=/home/owen/progs/firedrake-complex/firedrake/lib/python3.5/site-packages/matplotlib/mpl-data/fonts/ttf/]
%%   \setmonofont{DejaVuSansMono.ttf}[Path=/home/owen/progs/firedrake-complex/firedrake/lib/python3.5/site-packages/matplotlib/mpl-data/fonts/ttf/]
%%
\begingroup%
\makeatletter%
\begin{pgfpicture}%
\pgfpathrectangle{\pgfpointorigin}{\pgfqpoint{5.000000in}{4.000000in}}%
\pgfusepath{use as bounding box, clip}%
\begin{pgfscope}%
\pgfsetbuttcap%
\pgfsetmiterjoin%
\definecolor{currentfill}{rgb}{1.000000,1.000000,1.000000}%
\pgfsetfillcolor{currentfill}%
\pgfsetlinewidth{0.000000pt}%
\definecolor{currentstroke}{rgb}{1.000000,1.000000,1.000000}%
\pgfsetstrokecolor{currentstroke}%
\pgfsetdash{}{0pt}%
\pgfpathmoveto{\pgfqpoint{0.000000in}{0.000000in}}%
\pgfpathlineto{\pgfqpoint{5.000000in}{0.000000in}}%
\pgfpathlineto{\pgfqpoint{5.000000in}{4.000000in}}%
\pgfpathlineto{\pgfqpoint{0.000000in}{4.000000in}}%
\pgfpathclose%
\pgfusepath{fill}%
\end{pgfscope}%
\begin{pgfscope}%
\pgfsetbuttcap%
\pgfsetmiterjoin%
\definecolor{currentfill}{rgb}{1.000000,1.000000,1.000000}%
\pgfsetfillcolor{currentfill}%
\pgfsetlinewidth{0.000000pt}%
\definecolor{currentstroke}{rgb}{0.000000,0.000000,0.000000}%
\pgfsetstrokecolor{currentstroke}%
\pgfsetstrokeopacity{0.000000}%
\pgfsetdash{}{0pt}%
\pgfpathmoveto{\pgfqpoint{0.625000in}{0.440000in}}%
\pgfpathlineto{\pgfqpoint{4.500000in}{0.440000in}}%
\pgfpathlineto{\pgfqpoint{4.500000in}{3.520000in}}%
\pgfpathlineto{\pgfqpoint{0.625000in}{3.520000in}}%
\pgfpathclose%
\pgfusepath{fill}%
\end{pgfscope}%
\begin{pgfscope}%
\pgfsetbuttcap%
\pgfsetroundjoin%
\definecolor{currentfill}{rgb}{0.000000,0.000000,0.000000}%
\pgfsetfillcolor{currentfill}%
\pgfsetlinewidth{0.803000pt}%
\definecolor{currentstroke}{rgb}{0.000000,0.000000,0.000000}%
\pgfsetstrokecolor{currentstroke}%
\pgfsetdash{}{0pt}%
\pgfsys@defobject{currentmarker}{\pgfqpoint{0.000000in}{-0.048611in}}{\pgfqpoint{0.000000in}{0.000000in}}{%
\pgfpathmoveto{\pgfqpoint{0.000000in}{0.000000in}}%
\pgfpathlineto{\pgfqpoint{0.000000in}{-0.048611in}}%
\pgfusepath{stroke,fill}%
}%
\begin{pgfscope}%
\pgfsys@transformshift{0.801136in}{0.440000in}%
\pgfsys@useobject{currentmarker}{}%
\end{pgfscope}%
\end{pgfscope}%
\begin{pgfscope}%
\definecolor{textcolor}{rgb}{0.000000,0.000000,0.000000}%
\pgfsetstrokecolor{textcolor}%
\pgfsetfillcolor{textcolor}%
\pgftext[x=0.801136in,y=0.342778in,,top]{\color{textcolor}\sffamily\fontsize{10.000000}{12.000000}\selectfont \(\displaystyle 10^{0}\)}%
\end{pgfscope}%
\begin{pgfscope}%
\pgfsetbuttcap%
\pgfsetroundjoin%
\definecolor{currentfill}{rgb}{0.000000,0.000000,0.000000}%
\pgfsetfillcolor{currentfill}%
\pgfsetlinewidth{0.803000pt}%
\definecolor{currentstroke}{rgb}{0.000000,0.000000,0.000000}%
\pgfsetstrokecolor{currentstroke}%
\pgfsetdash{}{0pt}%
\pgfsys@defobject{currentmarker}{\pgfqpoint{0.000000in}{-0.048611in}}{\pgfqpoint{0.000000in}{0.000000in}}{%
\pgfpathmoveto{\pgfqpoint{0.000000in}{0.000000in}}%
\pgfpathlineto{\pgfqpoint{0.000000in}{-0.048611in}}%
\pgfusepath{stroke,fill}%
}%
\begin{pgfscope}%
\pgfsys@transformshift{1.864977in}{0.440000in}%
\pgfsys@useobject{currentmarker}{}%
\end{pgfscope}%
\end{pgfscope}%
\begin{pgfscope}%
\definecolor{textcolor}{rgb}{0.000000,0.000000,0.000000}%
\pgfsetstrokecolor{textcolor}%
\pgfsetfillcolor{textcolor}%
\pgftext[x=1.864977in,y=0.342778in,,top]{\color{textcolor}\sffamily\fontsize{10.000000}{12.000000}\selectfont \(\displaystyle 10^{1}\)}%
\end{pgfscope}%
\begin{pgfscope}%
\pgfsetbuttcap%
\pgfsetroundjoin%
\definecolor{currentfill}{rgb}{0.000000,0.000000,0.000000}%
\pgfsetfillcolor{currentfill}%
\pgfsetlinewidth{0.803000pt}%
\definecolor{currentstroke}{rgb}{0.000000,0.000000,0.000000}%
\pgfsetstrokecolor{currentstroke}%
\pgfsetdash{}{0pt}%
\pgfsys@defobject{currentmarker}{\pgfqpoint{0.000000in}{-0.048611in}}{\pgfqpoint{0.000000in}{0.000000in}}{%
\pgfpathmoveto{\pgfqpoint{0.000000in}{0.000000in}}%
\pgfpathlineto{\pgfqpoint{0.000000in}{-0.048611in}}%
\pgfusepath{stroke,fill}%
}%
\begin{pgfscope}%
\pgfsys@transformshift{2.928818in}{0.440000in}%
\pgfsys@useobject{currentmarker}{}%
\end{pgfscope}%
\end{pgfscope}%
\begin{pgfscope}%
\definecolor{textcolor}{rgb}{0.000000,0.000000,0.000000}%
\pgfsetstrokecolor{textcolor}%
\pgfsetfillcolor{textcolor}%
\pgftext[x=2.928818in,y=0.342778in,,top]{\color{textcolor}\sffamily\fontsize{10.000000}{12.000000}\selectfont \(\displaystyle 10^{2}\)}%
\end{pgfscope}%
\begin{pgfscope}%
\pgfsetbuttcap%
\pgfsetroundjoin%
\definecolor{currentfill}{rgb}{0.000000,0.000000,0.000000}%
\pgfsetfillcolor{currentfill}%
\pgfsetlinewidth{0.803000pt}%
\definecolor{currentstroke}{rgb}{0.000000,0.000000,0.000000}%
\pgfsetstrokecolor{currentstroke}%
\pgfsetdash{}{0pt}%
\pgfsys@defobject{currentmarker}{\pgfqpoint{0.000000in}{-0.048611in}}{\pgfqpoint{0.000000in}{0.000000in}}{%
\pgfpathmoveto{\pgfqpoint{0.000000in}{0.000000in}}%
\pgfpathlineto{\pgfqpoint{0.000000in}{-0.048611in}}%
\pgfusepath{stroke,fill}%
}%
\begin{pgfscope}%
\pgfsys@transformshift{3.992658in}{0.440000in}%
\pgfsys@useobject{currentmarker}{}%
\end{pgfscope}%
\end{pgfscope}%
\begin{pgfscope}%
\definecolor{textcolor}{rgb}{0.000000,0.000000,0.000000}%
\pgfsetstrokecolor{textcolor}%
\pgfsetfillcolor{textcolor}%
\pgftext[x=3.992658in,y=0.342778in,,top]{\color{textcolor}\sffamily\fontsize{10.000000}{12.000000}\selectfont \(\displaystyle 10^{3}\)}%
\end{pgfscope}%
\begin{pgfscope}%
\pgfsetbuttcap%
\pgfsetroundjoin%
\definecolor{currentfill}{rgb}{0.000000,0.000000,0.000000}%
\pgfsetfillcolor{currentfill}%
\pgfsetlinewidth{0.602250pt}%
\definecolor{currentstroke}{rgb}{0.000000,0.000000,0.000000}%
\pgfsetstrokecolor{currentstroke}%
\pgfsetdash{}{0pt}%
\pgfsys@defobject{currentmarker}{\pgfqpoint{0.000000in}{-0.027778in}}{\pgfqpoint{0.000000in}{0.000000in}}{%
\pgfpathmoveto{\pgfqpoint{0.000000in}{0.000000in}}%
\pgfpathlineto{\pgfqpoint{0.000000in}{-0.027778in}}%
\pgfusepath{stroke,fill}%
}%
\begin{pgfscope}%
\pgfsys@transformshift{0.636345in}{0.440000in}%
\pgfsys@useobject{currentmarker}{}%
\end{pgfscope}%
\end{pgfscope}%
\begin{pgfscope}%
\pgfsetbuttcap%
\pgfsetroundjoin%
\definecolor{currentfill}{rgb}{0.000000,0.000000,0.000000}%
\pgfsetfillcolor{currentfill}%
\pgfsetlinewidth{0.602250pt}%
\definecolor{currentstroke}{rgb}{0.000000,0.000000,0.000000}%
\pgfsetstrokecolor{currentstroke}%
\pgfsetdash{}{0pt}%
\pgfsys@defobject{currentmarker}{\pgfqpoint{0.000000in}{-0.027778in}}{\pgfqpoint{0.000000in}{0.000000in}}{%
\pgfpathmoveto{\pgfqpoint{0.000000in}{0.000000in}}%
\pgfpathlineto{\pgfqpoint{0.000000in}{-0.027778in}}%
\pgfusepath{stroke,fill}%
}%
\begin{pgfscope}%
\pgfsys@transformshift{0.698040in}{0.440000in}%
\pgfsys@useobject{currentmarker}{}%
\end{pgfscope}%
\end{pgfscope}%
\begin{pgfscope}%
\pgfsetbuttcap%
\pgfsetroundjoin%
\definecolor{currentfill}{rgb}{0.000000,0.000000,0.000000}%
\pgfsetfillcolor{currentfill}%
\pgfsetlinewidth{0.602250pt}%
\definecolor{currentstroke}{rgb}{0.000000,0.000000,0.000000}%
\pgfsetstrokecolor{currentstroke}%
\pgfsetdash{}{0pt}%
\pgfsys@defobject{currentmarker}{\pgfqpoint{0.000000in}{-0.027778in}}{\pgfqpoint{0.000000in}{0.000000in}}{%
\pgfpathmoveto{\pgfqpoint{0.000000in}{0.000000in}}%
\pgfpathlineto{\pgfqpoint{0.000000in}{-0.027778in}}%
\pgfusepath{stroke,fill}%
}%
\begin{pgfscope}%
\pgfsys@transformshift{0.752458in}{0.440000in}%
\pgfsys@useobject{currentmarker}{}%
\end{pgfscope}%
\end{pgfscope}%
\begin{pgfscope}%
\pgfsetbuttcap%
\pgfsetroundjoin%
\definecolor{currentfill}{rgb}{0.000000,0.000000,0.000000}%
\pgfsetfillcolor{currentfill}%
\pgfsetlinewidth{0.602250pt}%
\definecolor{currentstroke}{rgb}{0.000000,0.000000,0.000000}%
\pgfsetstrokecolor{currentstroke}%
\pgfsetdash{}{0pt}%
\pgfsys@defobject{currentmarker}{\pgfqpoint{0.000000in}{-0.027778in}}{\pgfqpoint{0.000000in}{0.000000in}}{%
\pgfpathmoveto{\pgfqpoint{0.000000in}{0.000000in}}%
\pgfpathlineto{\pgfqpoint{0.000000in}{-0.027778in}}%
\pgfusepath{stroke,fill}%
}%
\begin{pgfscope}%
\pgfsys@transformshift{1.121384in}{0.440000in}%
\pgfsys@useobject{currentmarker}{}%
\end{pgfscope}%
\end{pgfscope}%
\begin{pgfscope}%
\pgfsetbuttcap%
\pgfsetroundjoin%
\definecolor{currentfill}{rgb}{0.000000,0.000000,0.000000}%
\pgfsetfillcolor{currentfill}%
\pgfsetlinewidth{0.602250pt}%
\definecolor{currentstroke}{rgb}{0.000000,0.000000,0.000000}%
\pgfsetstrokecolor{currentstroke}%
\pgfsetdash{}{0pt}%
\pgfsys@defobject{currentmarker}{\pgfqpoint{0.000000in}{-0.027778in}}{\pgfqpoint{0.000000in}{0.000000in}}{%
\pgfpathmoveto{\pgfqpoint{0.000000in}{0.000000in}}%
\pgfpathlineto{\pgfqpoint{0.000000in}{-0.027778in}}%
\pgfusepath{stroke,fill}%
}%
\begin{pgfscope}%
\pgfsys@transformshift{1.308717in}{0.440000in}%
\pgfsys@useobject{currentmarker}{}%
\end{pgfscope}%
\end{pgfscope}%
\begin{pgfscope}%
\pgfsetbuttcap%
\pgfsetroundjoin%
\definecolor{currentfill}{rgb}{0.000000,0.000000,0.000000}%
\pgfsetfillcolor{currentfill}%
\pgfsetlinewidth{0.602250pt}%
\definecolor{currentstroke}{rgb}{0.000000,0.000000,0.000000}%
\pgfsetstrokecolor{currentstroke}%
\pgfsetdash{}{0pt}%
\pgfsys@defobject{currentmarker}{\pgfqpoint{0.000000in}{-0.027778in}}{\pgfqpoint{0.000000in}{0.000000in}}{%
\pgfpathmoveto{\pgfqpoint{0.000000in}{0.000000in}}%
\pgfpathlineto{\pgfqpoint{0.000000in}{-0.027778in}}%
\pgfusepath{stroke,fill}%
}%
\begin{pgfscope}%
\pgfsys@transformshift{1.441632in}{0.440000in}%
\pgfsys@useobject{currentmarker}{}%
\end{pgfscope}%
\end{pgfscope}%
\begin{pgfscope}%
\pgfsetbuttcap%
\pgfsetroundjoin%
\definecolor{currentfill}{rgb}{0.000000,0.000000,0.000000}%
\pgfsetfillcolor{currentfill}%
\pgfsetlinewidth{0.602250pt}%
\definecolor{currentstroke}{rgb}{0.000000,0.000000,0.000000}%
\pgfsetstrokecolor{currentstroke}%
\pgfsetdash{}{0pt}%
\pgfsys@defobject{currentmarker}{\pgfqpoint{0.000000in}{-0.027778in}}{\pgfqpoint{0.000000in}{0.000000in}}{%
\pgfpathmoveto{\pgfqpoint{0.000000in}{0.000000in}}%
\pgfpathlineto{\pgfqpoint{0.000000in}{-0.027778in}}%
\pgfusepath{stroke,fill}%
}%
\begin{pgfscope}%
\pgfsys@transformshift{1.544729in}{0.440000in}%
\pgfsys@useobject{currentmarker}{}%
\end{pgfscope}%
\end{pgfscope}%
\begin{pgfscope}%
\pgfsetbuttcap%
\pgfsetroundjoin%
\definecolor{currentfill}{rgb}{0.000000,0.000000,0.000000}%
\pgfsetfillcolor{currentfill}%
\pgfsetlinewidth{0.602250pt}%
\definecolor{currentstroke}{rgb}{0.000000,0.000000,0.000000}%
\pgfsetstrokecolor{currentstroke}%
\pgfsetdash{}{0pt}%
\pgfsys@defobject{currentmarker}{\pgfqpoint{0.000000in}{-0.027778in}}{\pgfqpoint{0.000000in}{0.000000in}}{%
\pgfpathmoveto{\pgfqpoint{0.000000in}{0.000000in}}%
\pgfpathlineto{\pgfqpoint{0.000000in}{-0.027778in}}%
\pgfusepath{stroke,fill}%
}%
\begin{pgfscope}%
\pgfsys@transformshift{1.628965in}{0.440000in}%
\pgfsys@useobject{currentmarker}{}%
\end{pgfscope}%
\end{pgfscope}%
\begin{pgfscope}%
\pgfsetbuttcap%
\pgfsetroundjoin%
\definecolor{currentfill}{rgb}{0.000000,0.000000,0.000000}%
\pgfsetfillcolor{currentfill}%
\pgfsetlinewidth{0.602250pt}%
\definecolor{currentstroke}{rgb}{0.000000,0.000000,0.000000}%
\pgfsetstrokecolor{currentstroke}%
\pgfsetdash{}{0pt}%
\pgfsys@defobject{currentmarker}{\pgfqpoint{0.000000in}{-0.027778in}}{\pgfqpoint{0.000000in}{0.000000in}}{%
\pgfpathmoveto{\pgfqpoint{0.000000in}{0.000000in}}%
\pgfpathlineto{\pgfqpoint{0.000000in}{-0.027778in}}%
\pgfusepath{stroke,fill}%
}%
\begin{pgfscope}%
\pgfsys@transformshift{1.700186in}{0.440000in}%
\pgfsys@useobject{currentmarker}{}%
\end{pgfscope}%
\end{pgfscope}%
\begin{pgfscope}%
\pgfsetbuttcap%
\pgfsetroundjoin%
\definecolor{currentfill}{rgb}{0.000000,0.000000,0.000000}%
\pgfsetfillcolor{currentfill}%
\pgfsetlinewidth{0.602250pt}%
\definecolor{currentstroke}{rgb}{0.000000,0.000000,0.000000}%
\pgfsetstrokecolor{currentstroke}%
\pgfsetdash{}{0pt}%
\pgfsys@defobject{currentmarker}{\pgfqpoint{0.000000in}{-0.027778in}}{\pgfqpoint{0.000000in}{0.000000in}}{%
\pgfpathmoveto{\pgfqpoint{0.000000in}{0.000000in}}%
\pgfpathlineto{\pgfqpoint{0.000000in}{-0.027778in}}%
\pgfusepath{stroke,fill}%
}%
\begin{pgfscope}%
\pgfsys@transformshift{1.761880in}{0.440000in}%
\pgfsys@useobject{currentmarker}{}%
\end{pgfscope}%
\end{pgfscope}%
\begin{pgfscope}%
\pgfsetbuttcap%
\pgfsetroundjoin%
\definecolor{currentfill}{rgb}{0.000000,0.000000,0.000000}%
\pgfsetfillcolor{currentfill}%
\pgfsetlinewidth{0.602250pt}%
\definecolor{currentstroke}{rgb}{0.000000,0.000000,0.000000}%
\pgfsetstrokecolor{currentstroke}%
\pgfsetdash{}{0pt}%
\pgfsys@defobject{currentmarker}{\pgfqpoint{0.000000in}{-0.027778in}}{\pgfqpoint{0.000000in}{0.000000in}}{%
\pgfpathmoveto{\pgfqpoint{0.000000in}{0.000000in}}%
\pgfpathlineto{\pgfqpoint{0.000000in}{-0.027778in}}%
\pgfusepath{stroke,fill}%
}%
\begin{pgfscope}%
\pgfsys@transformshift{1.816298in}{0.440000in}%
\pgfsys@useobject{currentmarker}{}%
\end{pgfscope}%
\end{pgfscope}%
\begin{pgfscope}%
\pgfsetbuttcap%
\pgfsetroundjoin%
\definecolor{currentfill}{rgb}{0.000000,0.000000,0.000000}%
\pgfsetfillcolor{currentfill}%
\pgfsetlinewidth{0.602250pt}%
\definecolor{currentstroke}{rgb}{0.000000,0.000000,0.000000}%
\pgfsetstrokecolor{currentstroke}%
\pgfsetdash{}{0pt}%
\pgfsys@defobject{currentmarker}{\pgfqpoint{0.000000in}{-0.027778in}}{\pgfqpoint{0.000000in}{0.000000in}}{%
\pgfpathmoveto{\pgfqpoint{0.000000in}{0.000000in}}%
\pgfpathlineto{\pgfqpoint{0.000000in}{-0.027778in}}%
\pgfusepath{stroke,fill}%
}%
\begin{pgfscope}%
\pgfsys@transformshift{2.185225in}{0.440000in}%
\pgfsys@useobject{currentmarker}{}%
\end{pgfscope}%
\end{pgfscope}%
\begin{pgfscope}%
\pgfsetbuttcap%
\pgfsetroundjoin%
\definecolor{currentfill}{rgb}{0.000000,0.000000,0.000000}%
\pgfsetfillcolor{currentfill}%
\pgfsetlinewidth{0.602250pt}%
\definecolor{currentstroke}{rgb}{0.000000,0.000000,0.000000}%
\pgfsetstrokecolor{currentstroke}%
\pgfsetdash{}{0pt}%
\pgfsys@defobject{currentmarker}{\pgfqpoint{0.000000in}{-0.027778in}}{\pgfqpoint{0.000000in}{0.000000in}}{%
\pgfpathmoveto{\pgfqpoint{0.000000in}{0.000000in}}%
\pgfpathlineto{\pgfqpoint{0.000000in}{-0.027778in}}%
\pgfusepath{stroke,fill}%
}%
\begin{pgfscope}%
\pgfsys@transformshift{2.372558in}{0.440000in}%
\pgfsys@useobject{currentmarker}{}%
\end{pgfscope}%
\end{pgfscope}%
\begin{pgfscope}%
\pgfsetbuttcap%
\pgfsetroundjoin%
\definecolor{currentfill}{rgb}{0.000000,0.000000,0.000000}%
\pgfsetfillcolor{currentfill}%
\pgfsetlinewidth{0.602250pt}%
\definecolor{currentstroke}{rgb}{0.000000,0.000000,0.000000}%
\pgfsetstrokecolor{currentstroke}%
\pgfsetdash{}{0pt}%
\pgfsys@defobject{currentmarker}{\pgfqpoint{0.000000in}{-0.027778in}}{\pgfqpoint{0.000000in}{0.000000in}}{%
\pgfpathmoveto{\pgfqpoint{0.000000in}{0.000000in}}%
\pgfpathlineto{\pgfqpoint{0.000000in}{-0.027778in}}%
\pgfusepath{stroke,fill}%
}%
\begin{pgfscope}%
\pgfsys@transformshift{2.505473in}{0.440000in}%
\pgfsys@useobject{currentmarker}{}%
\end{pgfscope}%
\end{pgfscope}%
\begin{pgfscope}%
\pgfsetbuttcap%
\pgfsetroundjoin%
\definecolor{currentfill}{rgb}{0.000000,0.000000,0.000000}%
\pgfsetfillcolor{currentfill}%
\pgfsetlinewidth{0.602250pt}%
\definecolor{currentstroke}{rgb}{0.000000,0.000000,0.000000}%
\pgfsetstrokecolor{currentstroke}%
\pgfsetdash{}{0pt}%
\pgfsys@defobject{currentmarker}{\pgfqpoint{0.000000in}{-0.027778in}}{\pgfqpoint{0.000000in}{0.000000in}}{%
\pgfpathmoveto{\pgfqpoint{0.000000in}{0.000000in}}%
\pgfpathlineto{\pgfqpoint{0.000000in}{-0.027778in}}%
\pgfusepath{stroke,fill}%
}%
\begin{pgfscope}%
\pgfsys@transformshift{2.608570in}{0.440000in}%
\pgfsys@useobject{currentmarker}{}%
\end{pgfscope}%
\end{pgfscope}%
\begin{pgfscope}%
\pgfsetbuttcap%
\pgfsetroundjoin%
\definecolor{currentfill}{rgb}{0.000000,0.000000,0.000000}%
\pgfsetfillcolor{currentfill}%
\pgfsetlinewidth{0.602250pt}%
\definecolor{currentstroke}{rgb}{0.000000,0.000000,0.000000}%
\pgfsetstrokecolor{currentstroke}%
\pgfsetdash{}{0pt}%
\pgfsys@defobject{currentmarker}{\pgfqpoint{0.000000in}{-0.027778in}}{\pgfqpoint{0.000000in}{0.000000in}}{%
\pgfpathmoveto{\pgfqpoint{0.000000in}{0.000000in}}%
\pgfpathlineto{\pgfqpoint{0.000000in}{-0.027778in}}%
\pgfusepath{stroke,fill}%
}%
\begin{pgfscope}%
\pgfsys@transformshift{2.692806in}{0.440000in}%
\pgfsys@useobject{currentmarker}{}%
\end{pgfscope}%
\end{pgfscope}%
\begin{pgfscope}%
\pgfsetbuttcap%
\pgfsetroundjoin%
\definecolor{currentfill}{rgb}{0.000000,0.000000,0.000000}%
\pgfsetfillcolor{currentfill}%
\pgfsetlinewidth{0.602250pt}%
\definecolor{currentstroke}{rgb}{0.000000,0.000000,0.000000}%
\pgfsetstrokecolor{currentstroke}%
\pgfsetdash{}{0pt}%
\pgfsys@defobject{currentmarker}{\pgfqpoint{0.000000in}{-0.027778in}}{\pgfqpoint{0.000000in}{0.000000in}}{%
\pgfpathmoveto{\pgfqpoint{0.000000in}{0.000000in}}%
\pgfpathlineto{\pgfqpoint{0.000000in}{-0.027778in}}%
\pgfusepath{stroke,fill}%
}%
\begin{pgfscope}%
\pgfsys@transformshift{2.764027in}{0.440000in}%
\pgfsys@useobject{currentmarker}{}%
\end{pgfscope}%
\end{pgfscope}%
\begin{pgfscope}%
\pgfsetbuttcap%
\pgfsetroundjoin%
\definecolor{currentfill}{rgb}{0.000000,0.000000,0.000000}%
\pgfsetfillcolor{currentfill}%
\pgfsetlinewidth{0.602250pt}%
\definecolor{currentstroke}{rgb}{0.000000,0.000000,0.000000}%
\pgfsetstrokecolor{currentstroke}%
\pgfsetdash{}{0pt}%
\pgfsys@defobject{currentmarker}{\pgfqpoint{0.000000in}{-0.027778in}}{\pgfqpoint{0.000000in}{0.000000in}}{%
\pgfpathmoveto{\pgfqpoint{0.000000in}{0.000000in}}%
\pgfpathlineto{\pgfqpoint{0.000000in}{-0.027778in}}%
\pgfusepath{stroke,fill}%
}%
\begin{pgfscope}%
\pgfsys@transformshift{2.825721in}{0.440000in}%
\pgfsys@useobject{currentmarker}{}%
\end{pgfscope}%
\end{pgfscope}%
\begin{pgfscope}%
\pgfsetbuttcap%
\pgfsetroundjoin%
\definecolor{currentfill}{rgb}{0.000000,0.000000,0.000000}%
\pgfsetfillcolor{currentfill}%
\pgfsetlinewidth{0.602250pt}%
\definecolor{currentstroke}{rgb}{0.000000,0.000000,0.000000}%
\pgfsetstrokecolor{currentstroke}%
\pgfsetdash{}{0pt}%
\pgfsys@defobject{currentmarker}{\pgfqpoint{0.000000in}{-0.027778in}}{\pgfqpoint{0.000000in}{0.000000in}}{%
\pgfpathmoveto{\pgfqpoint{0.000000in}{0.000000in}}%
\pgfpathlineto{\pgfqpoint{0.000000in}{-0.027778in}}%
\pgfusepath{stroke,fill}%
}%
\begin{pgfscope}%
\pgfsys@transformshift{2.880139in}{0.440000in}%
\pgfsys@useobject{currentmarker}{}%
\end{pgfscope}%
\end{pgfscope}%
\begin{pgfscope}%
\pgfsetbuttcap%
\pgfsetroundjoin%
\definecolor{currentfill}{rgb}{0.000000,0.000000,0.000000}%
\pgfsetfillcolor{currentfill}%
\pgfsetlinewidth{0.602250pt}%
\definecolor{currentstroke}{rgb}{0.000000,0.000000,0.000000}%
\pgfsetstrokecolor{currentstroke}%
\pgfsetdash{}{0pt}%
\pgfsys@defobject{currentmarker}{\pgfqpoint{0.000000in}{-0.027778in}}{\pgfqpoint{0.000000in}{0.000000in}}{%
\pgfpathmoveto{\pgfqpoint{0.000000in}{0.000000in}}%
\pgfpathlineto{\pgfqpoint{0.000000in}{-0.027778in}}%
\pgfusepath{stroke,fill}%
}%
\begin{pgfscope}%
\pgfsys@transformshift{3.249066in}{0.440000in}%
\pgfsys@useobject{currentmarker}{}%
\end{pgfscope}%
\end{pgfscope}%
\begin{pgfscope}%
\pgfsetbuttcap%
\pgfsetroundjoin%
\definecolor{currentfill}{rgb}{0.000000,0.000000,0.000000}%
\pgfsetfillcolor{currentfill}%
\pgfsetlinewidth{0.602250pt}%
\definecolor{currentstroke}{rgb}{0.000000,0.000000,0.000000}%
\pgfsetstrokecolor{currentstroke}%
\pgfsetdash{}{0pt}%
\pgfsys@defobject{currentmarker}{\pgfqpoint{0.000000in}{-0.027778in}}{\pgfqpoint{0.000000in}{0.000000in}}{%
\pgfpathmoveto{\pgfqpoint{0.000000in}{0.000000in}}%
\pgfpathlineto{\pgfqpoint{0.000000in}{-0.027778in}}%
\pgfusepath{stroke,fill}%
}%
\begin{pgfscope}%
\pgfsys@transformshift{3.436399in}{0.440000in}%
\pgfsys@useobject{currentmarker}{}%
\end{pgfscope}%
\end{pgfscope}%
\begin{pgfscope}%
\pgfsetbuttcap%
\pgfsetroundjoin%
\definecolor{currentfill}{rgb}{0.000000,0.000000,0.000000}%
\pgfsetfillcolor{currentfill}%
\pgfsetlinewidth{0.602250pt}%
\definecolor{currentstroke}{rgb}{0.000000,0.000000,0.000000}%
\pgfsetstrokecolor{currentstroke}%
\pgfsetdash{}{0pt}%
\pgfsys@defobject{currentmarker}{\pgfqpoint{0.000000in}{-0.027778in}}{\pgfqpoint{0.000000in}{0.000000in}}{%
\pgfpathmoveto{\pgfqpoint{0.000000in}{0.000000in}}%
\pgfpathlineto{\pgfqpoint{0.000000in}{-0.027778in}}%
\pgfusepath{stroke,fill}%
}%
\begin{pgfscope}%
\pgfsys@transformshift{3.569313in}{0.440000in}%
\pgfsys@useobject{currentmarker}{}%
\end{pgfscope}%
\end{pgfscope}%
\begin{pgfscope}%
\pgfsetbuttcap%
\pgfsetroundjoin%
\definecolor{currentfill}{rgb}{0.000000,0.000000,0.000000}%
\pgfsetfillcolor{currentfill}%
\pgfsetlinewidth{0.602250pt}%
\definecolor{currentstroke}{rgb}{0.000000,0.000000,0.000000}%
\pgfsetstrokecolor{currentstroke}%
\pgfsetdash{}{0pt}%
\pgfsys@defobject{currentmarker}{\pgfqpoint{0.000000in}{-0.027778in}}{\pgfqpoint{0.000000in}{0.000000in}}{%
\pgfpathmoveto{\pgfqpoint{0.000000in}{0.000000in}}%
\pgfpathlineto{\pgfqpoint{0.000000in}{-0.027778in}}%
\pgfusepath{stroke,fill}%
}%
\begin{pgfscope}%
\pgfsys@transformshift{3.672410in}{0.440000in}%
\pgfsys@useobject{currentmarker}{}%
\end{pgfscope}%
\end{pgfscope}%
\begin{pgfscope}%
\pgfsetbuttcap%
\pgfsetroundjoin%
\definecolor{currentfill}{rgb}{0.000000,0.000000,0.000000}%
\pgfsetfillcolor{currentfill}%
\pgfsetlinewidth{0.602250pt}%
\definecolor{currentstroke}{rgb}{0.000000,0.000000,0.000000}%
\pgfsetstrokecolor{currentstroke}%
\pgfsetdash{}{0pt}%
\pgfsys@defobject{currentmarker}{\pgfqpoint{0.000000in}{-0.027778in}}{\pgfqpoint{0.000000in}{0.000000in}}{%
\pgfpathmoveto{\pgfqpoint{0.000000in}{0.000000in}}%
\pgfpathlineto{\pgfqpoint{0.000000in}{-0.027778in}}%
\pgfusepath{stroke,fill}%
}%
\begin{pgfscope}%
\pgfsys@transformshift{3.756646in}{0.440000in}%
\pgfsys@useobject{currentmarker}{}%
\end{pgfscope}%
\end{pgfscope}%
\begin{pgfscope}%
\pgfsetbuttcap%
\pgfsetroundjoin%
\definecolor{currentfill}{rgb}{0.000000,0.000000,0.000000}%
\pgfsetfillcolor{currentfill}%
\pgfsetlinewidth{0.602250pt}%
\definecolor{currentstroke}{rgb}{0.000000,0.000000,0.000000}%
\pgfsetstrokecolor{currentstroke}%
\pgfsetdash{}{0pt}%
\pgfsys@defobject{currentmarker}{\pgfqpoint{0.000000in}{-0.027778in}}{\pgfqpoint{0.000000in}{0.000000in}}{%
\pgfpathmoveto{\pgfqpoint{0.000000in}{0.000000in}}%
\pgfpathlineto{\pgfqpoint{0.000000in}{-0.027778in}}%
\pgfusepath{stroke,fill}%
}%
\begin{pgfscope}%
\pgfsys@transformshift{3.827867in}{0.440000in}%
\pgfsys@useobject{currentmarker}{}%
\end{pgfscope}%
\end{pgfscope}%
\begin{pgfscope}%
\pgfsetbuttcap%
\pgfsetroundjoin%
\definecolor{currentfill}{rgb}{0.000000,0.000000,0.000000}%
\pgfsetfillcolor{currentfill}%
\pgfsetlinewidth{0.602250pt}%
\definecolor{currentstroke}{rgb}{0.000000,0.000000,0.000000}%
\pgfsetstrokecolor{currentstroke}%
\pgfsetdash{}{0pt}%
\pgfsys@defobject{currentmarker}{\pgfqpoint{0.000000in}{-0.027778in}}{\pgfqpoint{0.000000in}{0.000000in}}{%
\pgfpathmoveto{\pgfqpoint{0.000000in}{0.000000in}}%
\pgfpathlineto{\pgfqpoint{0.000000in}{-0.027778in}}%
\pgfusepath{stroke,fill}%
}%
\begin{pgfscope}%
\pgfsys@transformshift{3.889561in}{0.440000in}%
\pgfsys@useobject{currentmarker}{}%
\end{pgfscope}%
\end{pgfscope}%
\begin{pgfscope}%
\pgfsetbuttcap%
\pgfsetroundjoin%
\definecolor{currentfill}{rgb}{0.000000,0.000000,0.000000}%
\pgfsetfillcolor{currentfill}%
\pgfsetlinewidth{0.602250pt}%
\definecolor{currentstroke}{rgb}{0.000000,0.000000,0.000000}%
\pgfsetstrokecolor{currentstroke}%
\pgfsetdash{}{0pt}%
\pgfsys@defobject{currentmarker}{\pgfqpoint{0.000000in}{-0.027778in}}{\pgfqpoint{0.000000in}{0.000000in}}{%
\pgfpathmoveto{\pgfqpoint{0.000000in}{0.000000in}}%
\pgfpathlineto{\pgfqpoint{0.000000in}{-0.027778in}}%
\pgfusepath{stroke,fill}%
}%
\begin{pgfscope}%
\pgfsys@transformshift{3.943980in}{0.440000in}%
\pgfsys@useobject{currentmarker}{}%
\end{pgfscope}%
\end{pgfscope}%
\begin{pgfscope}%
\pgfsetbuttcap%
\pgfsetroundjoin%
\definecolor{currentfill}{rgb}{0.000000,0.000000,0.000000}%
\pgfsetfillcolor{currentfill}%
\pgfsetlinewidth{0.602250pt}%
\definecolor{currentstroke}{rgb}{0.000000,0.000000,0.000000}%
\pgfsetstrokecolor{currentstroke}%
\pgfsetdash{}{0pt}%
\pgfsys@defobject{currentmarker}{\pgfqpoint{0.000000in}{-0.027778in}}{\pgfqpoint{0.000000in}{0.000000in}}{%
\pgfpathmoveto{\pgfqpoint{0.000000in}{0.000000in}}%
\pgfpathlineto{\pgfqpoint{0.000000in}{-0.027778in}}%
\pgfusepath{stroke,fill}%
}%
\begin{pgfscope}%
\pgfsys@transformshift{4.312906in}{0.440000in}%
\pgfsys@useobject{currentmarker}{}%
\end{pgfscope}%
\end{pgfscope}%
\begin{pgfscope}%
\pgfsetbuttcap%
\pgfsetroundjoin%
\definecolor{currentfill}{rgb}{0.000000,0.000000,0.000000}%
\pgfsetfillcolor{currentfill}%
\pgfsetlinewidth{0.602250pt}%
\definecolor{currentstroke}{rgb}{0.000000,0.000000,0.000000}%
\pgfsetstrokecolor{currentstroke}%
\pgfsetdash{}{0pt}%
\pgfsys@defobject{currentmarker}{\pgfqpoint{0.000000in}{-0.027778in}}{\pgfqpoint{0.000000in}{0.000000in}}{%
\pgfpathmoveto{\pgfqpoint{0.000000in}{0.000000in}}%
\pgfpathlineto{\pgfqpoint{0.000000in}{-0.027778in}}%
\pgfusepath{stroke,fill}%
}%
\begin{pgfscope}%
\pgfsys@transformshift{4.500239in}{0.440000in}%
\pgfsys@useobject{currentmarker}{}%
\end{pgfscope}%
\end{pgfscope}%
\begin{pgfscope}%
\definecolor{textcolor}{rgb}{0.000000,0.000000,0.000000}%
\pgfsetstrokecolor{textcolor}%
\pgfsetfillcolor{textcolor}%
\pgftext[x=2.562500in,y=0.152809in,,top]{\color{textcolor}\sffamily\fontsize{10.000000}{12.000000}\selectfont \(\displaystyle N_\mathrm{QMC}\)}%
\end{pgfscope}%
\begin{pgfscope}%
\pgfsetbuttcap%
\pgfsetroundjoin%
\definecolor{currentfill}{rgb}{0.000000,0.000000,0.000000}%
\pgfsetfillcolor{currentfill}%
\pgfsetlinewidth{0.803000pt}%
\definecolor{currentstroke}{rgb}{0.000000,0.000000,0.000000}%
\pgfsetstrokecolor{currentstroke}%
\pgfsetdash{}{0pt}%
\pgfsys@defobject{currentmarker}{\pgfqpoint{-0.048611in}{0.000000in}}{\pgfqpoint{0.000000in}{0.000000in}}{%
\pgfpathmoveto{\pgfqpoint{0.000000in}{0.000000in}}%
\pgfpathlineto{\pgfqpoint{-0.048611in}{0.000000in}}%
\pgfusepath{stroke,fill}%
}%
\begin{pgfscope}%
\pgfsys@transformshift{0.625000in}{0.786246in}%
\pgfsys@useobject{currentmarker}{}%
\end{pgfscope}%
\end{pgfscope}%
\begin{pgfscope}%
\definecolor{textcolor}{rgb}{0.000000,0.000000,0.000000}%
\pgfsetstrokecolor{textcolor}%
\pgfsetfillcolor{textcolor}%
\pgftext[x=0.239775in,y=0.733484in,left,base]{\color{textcolor}\sffamily\fontsize{10.000000}{12.000000}\selectfont \(\displaystyle 10^{-4}\)}%
\end{pgfscope}%
\begin{pgfscope}%
\pgfsetbuttcap%
\pgfsetroundjoin%
\definecolor{currentfill}{rgb}{0.000000,0.000000,0.000000}%
\pgfsetfillcolor{currentfill}%
\pgfsetlinewidth{0.803000pt}%
\definecolor{currentstroke}{rgb}{0.000000,0.000000,0.000000}%
\pgfsetstrokecolor{currentstroke}%
\pgfsetdash{}{0pt}%
\pgfsys@defobject{currentmarker}{\pgfqpoint{-0.048611in}{0.000000in}}{\pgfqpoint{0.000000in}{0.000000in}}{%
\pgfpathmoveto{\pgfqpoint{0.000000in}{0.000000in}}%
\pgfpathlineto{\pgfqpoint{-0.048611in}{0.000000in}}%
\pgfusepath{stroke,fill}%
}%
\begin{pgfscope}%
\pgfsys@transformshift{0.625000in}{2.073804in}%
\pgfsys@useobject{currentmarker}{}%
\end{pgfscope}%
\end{pgfscope}%
\begin{pgfscope}%
\definecolor{textcolor}{rgb}{0.000000,0.000000,0.000000}%
\pgfsetstrokecolor{textcolor}%
\pgfsetfillcolor{textcolor}%
\pgftext[x=0.239775in,y=2.021042in,left,base]{\color{textcolor}\sffamily\fontsize{10.000000}{12.000000}\selectfont \(\displaystyle 10^{-3}\)}%
\end{pgfscope}%
\begin{pgfscope}%
\pgfsetbuttcap%
\pgfsetroundjoin%
\definecolor{currentfill}{rgb}{0.000000,0.000000,0.000000}%
\pgfsetfillcolor{currentfill}%
\pgfsetlinewidth{0.803000pt}%
\definecolor{currentstroke}{rgb}{0.000000,0.000000,0.000000}%
\pgfsetstrokecolor{currentstroke}%
\pgfsetdash{}{0pt}%
\pgfsys@defobject{currentmarker}{\pgfqpoint{-0.048611in}{0.000000in}}{\pgfqpoint{0.000000in}{0.000000in}}{%
\pgfpathmoveto{\pgfqpoint{0.000000in}{0.000000in}}%
\pgfpathlineto{\pgfqpoint{-0.048611in}{0.000000in}}%
\pgfusepath{stroke,fill}%
}%
\begin{pgfscope}%
\pgfsys@transformshift{0.625000in}{3.361362in}%
\pgfsys@useobject{currentmarker}{}%
\end{pgfscope}%
\end{pgfscope}%
\begin{pgfscope}%
\definecolor{textcolor}{rgb}{0.000000,0.000000,0.000000}%
\pgfsetstrokecolor{textcolor}%
\pgfsetfillcolor{textcolor}%
\pgftext[x=0.239775in,y=3.308600in,left,base]{\color{textcolor}\sffamily\fontsize{10.000000}{12.000000}\selectfont \(\displaystyle 10^{-2}\)}%
\end{pgfscope}%
\begin{pgfscope}%
\pgfsetbuttcap%
\pgfsetroundjoin%
\definecolor{currentfill}{rgb}{0.000000,0.000000,0.000000}%
\pgfsetfillcolor{currentfill}%
\pgfsetlinewidth{0.602250pt}%
\definecolor{currentstroke}{rgb}{0.000000,0.000000,0.000000}%
\pgfsetstrokecolor{currentstroke}%
\pgfsetdash{}{0pt}%
\pgfsys@defobject{currentmarker}{\pgfqpoint{-0.027778in}{0.000000in}}{\pgfqpoint{0.000000in}{0.000000in}}{%
\pgfpathmoveto{\pgfqpoint{0.000000in}{0.000000in}}%
\pgfpathlineto{\pgfqpoint{-0.027778in}{0.000000in}}%
\pgfusepath{stroke,fill}%
}%
\begin{pgfscope}%
\pgfsys@transformshift{0.625000in}{0.500603in}%
\pgfsys@useobject{currentmarker}{}%
\end{pgfscope}%
\end{pgfscope}%
\begin{pgfscope}%
\pgfsetbuttcap%
\pgfsetroundjoin%
\definecolor{currentfill}{rgb}{0.000000,0.000000,0.000000}%
\pgfsetfillcolor{currentfill}%
\pgfsetlinewidth{0.602250pt}%
\definecolor{currentstroke}{rgb}{0.000000,0.000000,0.000000}%
\pgfsetstrokecolor{currentstroke}%
\pgfsetdash{}{0pt}%
\pgfsys@defobject{currentmarker}{\pgfqpoint{-0.027778in}{0.000000in}}{\pgfqpoint{0.000000in}{0.000000in}}{%
\pgfpathmoveto{\pgfqpoint{0.000000in}{0.000000in}}%
\pgfpathlineto{\pgfqpoint{-0.027778in}{0.000000in}}%
\pgfusepath{stroke,fill}%
}%
\begin{pgfscope}%
\pgfsys@transformshift{0.625000in}{0.586801in}%
\pgfsys@useobject{currentmarker}{}%
\end{pgfscope}%
\end{pgfscope}%
\begin{pgfscope}%
\pgfsetbuttcap%
\pgfsetroundjoin%
\definecolor{currentfill}{rgb}{0.000000,0.000000,0.000000}%
\pgfsetfillcolor{currentfill}%
\pgfsetlinewidth{0.602250pt}%
\definecolor{currentstroke}{rgb}{0.000000,0.000000,0.000000}%
\pgfsetstrokecolor{currentstroke}%
\pgfsetdash{}{0pt}%
\pgfsys@defobject{currentmarker}{\pgfqpoint{-0.027778in}{0.000000in}}{\pgfqpoint{0.000000in}{0.000000in}}{%
\pgfpathmoveto{\pgfqpoint{0.000000in}{0.000000in}}%
\pgfpathlineto{\pgfqpoint{-0.027778in}{0.000000in}}%
\pgfusepath{stroke,fill}%
}%
\begin{pgfscope}%
\pgfsys@transformshift{0.625000in}{0.661469in}%
\pgfsys@useobject{currentmarker}{}%
\end{pgfscope}%
\end{pgfscope}%
\begin{pgfscope}%
\pgfsetbuttcap%
\pgfsetroundjoin%
\definecolor{currentfill}{rgb}{0.000000,0.000000,0.000000}%
\pgfsetfillcolor{currentfill}%
\pgfsetlinewidth{0.602250pt}%
\definecolor{currentstroke}{rgb}{0.000000,0.000000,0.000000}%
\pgfsetstrokecolor{currentstroke}%
\pgfsetdash{}{0pt}%
\pgfsys@defobject{currentmarker}{\pgfqpoint{-0.027778in}{0.000000in}}{\pgfqpoint{0.000000in}{0.000000in}}{%
\pgfpathmoveto{\pgfqpoint{0.000000in}{0.000000in}}%
\pgfpathlineto{\pgfqpoint{-0.027778in}{0.000000in}}%
\pgfusepath{stroke,fill}%
}%
\begin{pgfscope}%
\pgfsys@transformshift{0.625000in}{0.727330in}%
\pgfsys@useobject{currentmarker}{}%
\end{pgfscope}%
\end{pgfscope}%
\begin{pgfscope}%
\pgfsetbuttcap%
\pgfsetroundjoin%
\definecolor{currentfill}{rgb}{0.000000,0.000000,0.000000}%
\pgfsetfillcolor{currentfill}%
\pgfsetlinewidth{0.602250pt}%
\definecolor{currentstroke}{rgb}{0.000000,0.000000,0.000000}%
\pgfsetstrokecolor{currentstroke}%
\pgfsetdash{}{0pt}%
\pgfsys@defobject{currentmarker}{\pgfqpoint{-0.027778in}{0.000000in}}{\pgfqpoint{0.000000in}{0.000000in}}{%
\pgfpathmoveto{\pgfqpoint{0.000000in}{0.000000in}}%
\pgfpathlineto{\pgfqpoint{-0.027778in}{0.000000in}}%
\pgfusepath{stroke,fill}%
}%
\begin{pgfscope}%
\pgfsys@transformshift{0.625000in}{1.173839in}%
\pgfsys@useobject{currentmarker}{}%
\end{pgfscope}%
\end{pgfscope}%
\begin{pgfscope}%
\pgfsetbuttcap%
\pgfsetroundjoin%
\definecolor{currentfill}{rgb}{0.000000,0.000000,0.000000}%
\pgfsetfillcolor{currentfill}%
\pgfsetlinewidth{0.602250pt}%
\definecolor{currentstroke}{rgb}{0.000000,0.000000,0.000000}%
\pgfsetstrokecolor{currentstroke}%
\pgfsetdash{}{0pt}%
\pgfsys@defobject{currentmarker}{\pgfqpoint{-0.027778in}{0.000000in}}{\pgfqpoint{0.000000in}{0.000000in}}{%
\pgfpathmoveto{\pgfqpoint{0.000000in}{0.000000in}}%
\pgfpathlineto{\pgfqpoint{-0.027778in}{0.000000in}}%
\pgfusepath{stroke,fill}%
}%
\begin{pgfscope}%
\pgfsys@transformshift{0.625000in}{1.400567in}%
\pgfsys@useobject{currentmarker}{}%
\end{pgfscope}%
\end{pgfscope}%
\begin{pgfscope}%
\pgfsetbuttcap%
\pgfsetroundjoin%
\definecolor{currentfill}{rgb}{0.000000,0.000000,0.000000}%
\pgfsetfillcolor{currentfill}%
\pgfsetlinewidth{0.602250pt}%
\definecolor{currentstroke}{rgb}{0.000000,0.000000,0.000000}%
\pgfsetstrokecolor{currentstroke}%
\pgfsetdash{}{0pt}%
\pgfsys@defobject{currentmarker}{\pgfqpoint{-0.027778in}{0.000000in}}{\pgfqpoint{0.000000in}{0.000000in}}{%
\pgfpathmoveto{\pgfqpoint{0.000000in}{0.000000in}}%
\pgfpathlineto{\pgfqpoint{-0.027778in}{0.000000in}}%
\pgfusepath{stroke,fill}%
}%
\begin{pgfscope}%
\pgfsys@transformshift{0.625000in}{1.561433in}%
\pgfsys@useobject{currentmarker}{}%
\end{pgfscope}%
\end{pgfscope}%
\begin{pgfscope}%
\pgfsetbuttcap%
\pgfsetroundjoin%
\definecolor{currentfill}{rgb}{0.000000,0.000000,0.000000}%
\pgfsetfillcolor{currentfill}%
\pgfsetlinewidth{0.602250pt}%
\definecolor{currentstroke}{rgb}{0.000000,0.000000,0.000000}%
\pgfsetstrokecolor{currentstroke}%
\pgfsetdash{}{0pt}%
\pgfsys@defobject{currentmarker}{\pgfqpoint{-0.027778in}{0.000000in}}{\pgfqpoint{0.000000in}{0.000000in}}{%
\pgfpathmoveto{\pgfqpoint{0.000000in}{0.000000in}}%
\pgfpathlineto{\pgfqpoint{-0.027778in}{0.000000in}}%
\pgfusepath{stroke,fill}%
}%
\begin{pgfscope}%
\pgfsys@transformshift{0.625000in}{1.686210in}%
\pgfsys@useobject{currentmarker}{}%
\end{pgfscope}%
\end{pgfscope}%
\begin{pgfscope}%
\pgfsetbuttcap%
\pgfsetroundjoin%
\definecolor{currentfill}{rgb}{0.000000,0.000000,0.000000}%
\pgfsetfillcolor{currentfill}%
\pgfsetlinewidth{0.602250pt}%
\definecolor{currentstroke}{rgb}{0.000000,0.000000,0.000000}%
\pgfsetstrokecolor{currentstroke}%
\pgfsetdash{}{0pt}%
\pgfsys@defobject{currentmarker}{\pgfqpoint{-0.027778in}{0.000000in}}{\pgfqpoint{0.000000in}{0.000000in}}{%
\pgfpathmoveto{\pgfqpoint{0.000000in}{0.000000in}}%
\pgfpathlineto{\pgfqpoint{-0.027778in}{0.000000in}}%
\pgfusepath{stroke,fill}%
}%
\begin{pgfscope}%
\pgfsys@transformshift{0.625000in}{1.788161in}%
\pgfsys@useobject{currentmarker}{}%
\end{pgfscope}%
\end{pgfscope}%
\begin{pgfscope}%
\pgfsetbuttcap%
\pgfsetroundjoin%
\definecolor{currentfill}{rgb}{0.000000,0.000000,0.000000}%
\pgfsetfillcolor{currentfill}%
\pgfsetlinewidth{0.602250pt}%
\definecolor{currentstroke}{rgb}{0.000000,0.000000,0.000000}%
\pgfsetstrokecolor{currentstroke}%
\pgfsetdash{}{0pt}%
\pgfsys@defobject{currentmarker}{\pgfqpoint{-0.027778in}{0.000000in}}{\pgfqpoint{0.000000in}{0.000000in}}{%
\pgfpathmoveto{\pgfqpoint{0.000000in}{0.000000in}}%
\pgfpathlineto{\pgfqpoint{-0.027778in}{0.000000in}}%
\pgfusepath{stroke,fill}%
}%
\begin{pgfscope}%
\pgfsys@transformshift{0.625000in}{1.874359in}%
\pgfsys@useobject{currentmarker}{}%
\end{pgfscope}%
\end{pgfscope}%
\begin{pgfscope}%
\pgfsetbuttcap%
\pgfsetroundjoin%
\definecolor{currentfill}{rgb}{0.000000,0.000000,0.000000}%
\pgfsetfillcolor{currentfill}%
\pgfsetlinewidth{0.602250pt}%
\definecolor{currentstroke}{rgb}{0.000000,0.000000,0.000000}%
\pgfsetstrokecolor{currentstroke}%
\pgfsetdash{}{0pt}%
\pgfsys@defobject{currentmarker}{\pgfqpoint{-0.027778in}{0.000000in}}{\pgfqpoint{0.000000in}{0.000000in}}{%
\pgfpathmoveto{\pgfqpoint{0.000000in}{0.000000in}}%
\pgfpathlineto{\pgfqpoint{-0.027778in}{0.000000in}}%
\pgfusepath{stroke,fill}%
}%
\begin{pgfscope}%
\pgfsys@transformshift{0.625000in}{1.949027in}%
\pgfsys@useobject{currentmarker}{}%
\end{pgfscope}%
\end{pgfscope}%
\begin{pgfscope}%
\pgfsetbuttcap%
\pgfsetroundjoin%
\definecolor{currentfill}{rgb}{0.000000,0.000000,0.000000}%
\pgfsetfillcolor{currentfill}%
\pgfsetlinewidth{0.602250pt}%
\definecolor{currentstroke}{rgb}{0.000000,0.000000,0.000000}%
\pgfsetstrokecolor{currentstroke}%
\pgfsetdash{}{0pt}%
\pgfsys@defobject{currentmarker}{\pgfqpoint{-0.027778in}{0.000000in}}{\pgfqpoint{0.000000in}{0.000000in}}{%
\pgfpathmoveto{\pgfqpoint{0.000000in}{0.000000in}}%
\pgfpathlineto{\pgfqpoint{-0.027778in}{0.000000in}}%
\pgfusepath{stroke,fill}%
}%
\begin{pgfscope}%
\pgfsys@transformshift{0.625000in}{2.014888in}%
\pgfsys@useobject{currentmarker}{}%
\end{pgfscope}%
\end{pgfscope}%
\begin{pgfscope}%
\pgfsetbuttcap%
\pgfsetroundjoin%
\definecolor{currentfill}{rgb}{0.000000,0.000000,0.000000}%
\pgfsetfillcolor{currentfill}%
\pgfsetlinewidth{0.602250pt}%
\definecolor{currentstroke}{rgb}{0.000000,0.000000,0.000000}%
\pgfsetstrokecolor{currentstroke}%
\pgfsetdash{}{0pt}%
\pgfsys@defobject{currentmarker}{\pgfqpoint{-0.027778in}{0.000000in}}{\pgfqpoint{0.000000in}{0.000000in}}{%
\pgfpathmoveto{\pgfqpoint{0.000000in}{0.000000in}}%
\pgfpathlineto{\pgfqpoint{-0.027778in}{0.000000in}}%
\pgfusepath{stroke,fill}%
}%
\begin{pgfscope}%
\pgfsys@transformshift{0.625000in}{2.461398in}%
\pgfsys@useobject{currentmarker}{}%
\end{pgfscope}%
\end{pgfscope}%
\begin{pgfscope}%
\pgfsetbuttcap%
\pgfsetroundjoin%
\definecolor{currentfill}{rgb}{0.000000,0.000000,0.000000}%
\pgfsetfillcolor{currentfill}%
\pgfsetlinewidth{0.602250pt}%
\definecolor{currentstroke}{rgb}{0.000000,0.000000,0.000000}%
\pgfsetstrokecolor{currentstroke}%
\pgfsetdash{}{0pt}%
\pgfsys@defobject{currentmarker}{\pgfqpoint{-0.027778in}{0.000000in}}{\pgfqpoint{0.000000in}{0.000000in}}{%
\pgfpathmoveto{\pgfqpoint{0.000000in}{0.000000in}}%
\pgfpathlineto{\pgfqpoint{-0.027778in}{0.000000in}}%
\pgfusepath{stroke,fill}%
}%
\begin{pgfscope}%
\pgfsys@transformshift{0.625000in}{2.688125in}%
\pgfsys@useobject{currentmarker}{}%
\end{pgfscope}%
\end{pgfscope}%
\begin{pgfscope}%
\pgfsetbuttcap%
\pgfsetroundjoin%
\definecolor{currentfill}{rgb}{0.000000,0.000000,0.000000}%
\pgfsetfillcolor{currentfill}%
\pgfsetlinewidth{0.602250pt}%
\definecolor{currentstroke}{rgb}{0.000000,0.000000,0.000000}%
\pgfsetstrokecolor{currentstroke}%
\pgfsetdash{}{0pt}%
\pgfsys@defobject{currentmarker}{\pgfqpoint{-0.027778in}{0.000000in}}{\pgfqpoint{0.000000in}{0.000000in}}{%
\pgfpathmoveto{\pgfqpoint{0.000000in}{0.000000in}}%
\pgfpathlineto{\pgfqpoint{-0.027778in}{0.000000in}}%
\pgfusepath{stroke,fill}%
}%
\begin{pgfscope}%
\pgfsys@transformshift{0.625000in}{2.848991in}%
\pgfsys@useobject{currentmarker}{}%
\end{pgfscope}%
\end{pgfscope}%
\begin{pgfscope}%
\pgfsetbuttcap%
\pgfsetroundjoin%
\definecolor{currentfill}{rgb}{0.000000,0.000000,0.000000}%
\pgfsetfillcolor{currentfill}%
\pgfsetlinewidth{0.602250pt}%
\definecolor{currentstroke}{rgb}{0.000000,0.000000,0.000000}%
\pgfsetstrokecolor{currentstroke}%
\pgfsetdash{}{0pt}%
\pgfsys@defobject{currentmarker}{\pgfqpoint{-0.027778in}{0.000000in}}{\pgfqpoint{0.000000in}{0.000000in}}{%
\pgfpathmoveto{\pgfqpoint{0.000000in}{0.000000in}}%
\pgfpathlineto{\pgfqpoint{-0.027778in}{0.000000in}}%
\pgfusepath{stroke,fill}%
}%
\begin{pgfscope}%
\pgfsys@transformshift{0.625000in}{2.973768in}%
\pgfsys@useobject{currentmarker}{}%
\end{pgfscope}%
\end{pgfscope}%
\begin{pgfscope}%
\pgfsetbuttcap%
\pgfsetroundjoin%
\definecolor{currentfill}{rgb}{0.000000,0.000000,0.000000}%
\pgfsetfillcolor{currentfill}%
\pgfsetlinewidth{0.602250pt}%
\definecolor{currentstroke}{rgb}{0.000000,0.000000,0.000000}%
\pgfsetstrokecolor{currentstroke}%
\pgfsetdash{}{0pt}%
\pgfsys@defobject{currentmarker}{\pgfqpoint{-0.027778in}{0.000000in}}{\pgfqpoint{0.000000in}{0.000000in}}{%
\pgfpathmoveto{\pgfqpoint{0.000000in}{0.000000in}}%
\pgfpathlineto{\pgfqpoint{-0.027778in}{0.000000in}}%
\pgfusepath{stroke,fill}%
}%
\begin{pgfscope}%
\pgfsys@transformshift{0.625000in}{3.075719in}%
\pgfsys@useobject{currentmarker}{}%
\end{pgfscope}%
\end{pgfscope}%
\begin{pgfscope}%
\pgfsetbuttcap%
\pgfsetroundjoin%
\definecolor{currentfill}{rgb}{0.000000,0.000000,0.000000}%
\pgfsetfillcolor{currentfill}%
\pgfsetlinewidth{0.602250pt}%
\definecolor{currentstroke}{rgb}{0.000000,0.000000,0.000000}%
\pgfsetstrokecolor{currentstroke}%
\pgfsetdash{}{0pt}%
\pgfsys@defobject{currentmarker}{\pgfqpoint{-0.027778in}{0.000000in}}{\pgfqpoint{0.000000in}{0.000000in}}{%
\pgfpathmoveto{\pgfqpoint{0.000000in}{0.000000in}}%
\pgfpathlineto{\pgfqpoint{-0.027778in}{0.000000in}}%
\pgfusepath{stroke,fill}%
}%
\begin{pgfscope}%
\pgfsys@transformshift{0.625000in}{3.161917in}%
\pgfsys@useobject{currentmarker}{}%
\end{pgfscope}%
\end{pgfscope}%
\begin{pgfscope}%
\pgfsetbuttcap%
\pgfsetroundjoin%
\definecolor{currentfill}{rgb}{0.000000,0.000000,0.000000}%
\pgfsetfillcolor{currentfill}%
\pgfsetlinewidth{0.602250pt}%
\definecolor{currentstroke}{rgb}{0.000000,0.000000,0.000000}%
\pgfsetstrokecolor{currentstroke}%
\pgfsetdash{}{0pt}%
\pgfsys@defobject{currentmarker}{\pgfqpoint{-0.027778in}{0.000000in}}{\pgfqpoint{0.000000in}{0.000000in}}{%
\pgfpathmoveto{\pgfqpoint{0.000000in}{0.000000in}}%
\pgfpathlineto{\pgfqpoint{-0.027778in}{0.000000in}}%
\pgfusepath{stroke,fill}%
}%
\begin{pgfscope}%
\pgfsys@transformshift{0.625000in}{3.236585in}%
\pgfsys@useobject{currentmarker}{}%
\end{pgfscope}%
\end{pgfscope}%
\begin{pgfscope}%
\pgfsetbuttcap%
\pgfsetroundjoin%
\definecolor{currentfill}{rgb}{0.000000,0.000000,0.000000}%
\pgfsetfillcolor{currentfill}%
\pgfsetlinewidth{0.602250pt}%
\definecolor{currentstroke}{rgb}{0.000000,0.000000,0.000000}%
\pgfsetstrokecolor{currentstroke}%
\pgfsetdash{}{0pt}%
\pgfsys@defobject{currentmarker}{\pgfqpoint{-0.027778in}{0.000000in}}{\pgfqpoint{0.000000in}{0.000000in}}{%
\pgfpathmoveto{\pgfqpoint{0.000000in}{0.000000in}}%
\pgfpathlineto{\pgfqpoint{-0.027778in}{0.000000in}}%
\pgfusepath{stroke,fill}%
}%
\begin{pgfscope}%
\pgfsys@transformshift{0.625000in}{3.302447in}%
\pgfsys@useobject{currentmarker}{}%
\end{pgfscope}%
\end{pgfscope}%
\begin{pgfscope}%
\definecolor{textcolor}{rgb}{0.000000,0.000000,0.000000}%
\pgfsetstrokecolor{textcolor}%
\pgfsetfillcolor{textcolor}%
\pgftext[x=0.184220in,y=1.980000in,,bottom,rotate=90.000000]{\color{textcolor}\sffamily\fontsize{10.000000}{12.000000}\selectfont QMC Error}%
\end{pgfscope}%
\begin{pgfscope}%
\pgfpathrectangle{\pgfqpoint{0.625000in}{0.440000in}}{\pgfqpoint{3.875000in}{3.080000in}}%
\pgfusepath{clip}%
\pgfsetbuttcap%
\pgfsetroundjoin%
\definecolor{currentfill}{rgb}{0.000000,0.000000,0.000000}%
\pgfsetfillcolor{currentfill}%
\pgfsetlinewidth{1.003750pt}%
\definecolor{currentstroke}{rgb}{0.000000,0.000000,0.000000}%
\pgfsetstrokecolor{currentstroke}%
\pgfsetdash{}{0pt}%
\pgfsys@defobject{currentmarker}{\pgfqpoint{-0.041667in}{-0.041667in}}{\pgfqpoint{0.041667in}{0.041667in}}{%
\pgfpathmoveto{\pgfqpoint{0.000000in}{-0.041667in}}%
\pgfpathcurveto{\pgfqpoint{0.011050in}{-0.041667in}}{\pgfqpoint{0.021649in}{-0.037276in}}{\pgfqpoint{0.029463in}{-0.029463in}}%
\pgfpathcurveto{\pgfqpoint{0.037276in}{-0.021649in}}{\pgfqpoint{0.041667in}{-0.011050in}}{\pgfqpoint{0.041667in}{0.000000in}}%
\pgfpathcurveto{\pgfqpoint{0.041667in}{0.011050in}}{\pgfqpoint{0.037276in}{0.021649in}}{\pgfqpoint{0.029463in}{0.029463in}}%
\pgfpathcurveto{\pgfqpoint{0.021649in}{0.037276in}}{\pgfqpoint{0.011050in}{0.041667in}}{\pgfqpoint{0.000000in}{0.041667in}}%
\pgfpathcurveto{\pgfqpoint{-0.011050in}{0.041667in}}{\pgfqpoint{-0.021649in}{0.037276in}}{\pgfqpoint{-0.029463in}{0.029463in}}%
\pgfpathcurveto{\pgfqpoint{-0.037276in}{0.021649in}}{\pgfqpoint{-0.041667in}{0.011050in}}{\pgfqpoint{-0.041667in}{0.000000in}}%
\pgfpathcurveto{\pgfqpoint{-0.041667in}{-0.011050in}}{\pgfqpoint{-0.037276in}{-0.021649in}}{\pgfqpoint{-0.029463in}{-0.029463in}}%
\pgfpathcurveto{\pgfqpoint{-0.021649in}{-0.037276in}}{\pgfqpoint{-0.011050in}{-0.041667in}}{\pgfqpoint{0.000000in}{-0.041667in}}%
\pgfpathclose%
\pgfusepath{stroke,fill}%
}%
\begin{pgfscope}%
\pgfsys@transformshift{0.801136in}{3.380000in}%
\pgfsys@useobject{currentmarker}{}%
\end{pgfscope}%
\begin{pgfscope}%
\pgfsys@transformshift{1.121384in}{3.120657in}%
\pgfsys@useobject{currentmarker}{}%
\end{pgfscope}%
\begin{pgfscope}%
\pgfsys@transformshift{1.441632in}{2.857017in}%
\pgfsys@useobject{currentmarker}{}%
\end{pgfscope}%
\begin{pgfscope}%
\pgfsys@transformshift{1.761880in}{2.495016in}%
\pgfsys@useobject{currentmarker}{}%
\end{pgfscope}%
\begin{pgfscope}%
\pgfsys@transformshift{2.082128in}{2.438211in}%
\pgfsys@useobject{currentmarker}{}%
\end{pgfscope}%
\begin{pgfscope}%
\pgfsys@transformshift{2.402376in}{1.852087in}%
\pgfsys@useobject{currentmarker}{}%
\end{pgfscope}%
\begin{pgfscope}%
\pgfsys@transformshift{2.722624in}{1.592113in}%
\pgfsys@useobject{currentmarker}{}%
\end{pgfscope}%
\begin{pgfscope}%
\pgfsys@transformshift{3.042872in}{1.567134in}%
\pgfsys@useobject{currentmarker}{}%
\end{pgfscope}%
\begin{pgfscope}%
\pgfsys@transformshift{3.363120in}{1.174048in}%
\pgfsys@useobject{currentmarker}{}%
\end{pgfscope}%
\begin{pgfscope}%
\pgfsys@transformshift{3.683368in}{0.955248in}%
\pgfsys@useobject{currentmarker}{}%
\end{pgfscope}%
\begin{pgfscope}%
\pgfsys@transformshift{4.003616in}{0.916398in}%
\pgfsys@useobject{currentmarker}{}%
\end{pgfscope}%
\begin{pgfscope}%
\pgfsys@transformshift{4.323864in}{0.856156in}%
\pgfsys@useobject{currentmarker}{}%
\end{pgfscope}%
\end{pgfscope}%
\begin{pgfscope}%
\pgfpathrectangle{\pgfqpoint{0.625000in}{0.440000in}}{\pgfqpoint{3.875000in}{3.080000in}}%
\pgfusepath{clip}%
\pgfsetbuttcap%
\pgfsetroundjoin%
\pgfsetlinewidth{1.505625pt}%
\definecolor{currentstroke}{rgb}{0.000000,0.000000,0.000000}%
\pgfsetstrokecolor{currentstroke}%
\pgfsetdash{{5.550000pt}{2.400000pt}}{0.000000pt}%
\pgfpathmoveto{\pgfqpoint{0.801136in}{3.287348in}}%
\pgfpathlineto{\pgfqpoint{4.323864in}{0.580000in}}%
\pgfusepath{stroke}%
\end{pgfscope}%
\begin{pgfscope}%
\pgfsetrectcap%
\pgfsetmiterjoin%
\pgfsetlinewidth{0.803000pt}%
\definecolor{currentstroke}{rgb}{0.000000,0.000000,0.000000}%
\pgfsetstrokecolor{currentstroke}%
\pgfsetdash{}{0pt}%
\pgfpathmoveto{\pgfqpoint{0.625000in}{0.440000in}}%
\pgfpathlineto{\pgfqpoint{0.625000in}{3.520000in}}%
\pgfusepath{stroke}%
\end{pgfscope}%
\begin{pgfscope}%
\pgfsetrectcap%
\pgfsetmiterjoin%
\pgfsetlinewidth{0.803000pt}%
\definecolor{currentstroke}{rgb}{0.000000,0.000000,0.000000}%
\pgfsetstrokecolor{currentstroke}%
\pgfsetdash{}{0pt}%
\pgfpathmoveto{\pgfqpoint{4.500000in}{0.440000in}}%
\pgfpathlineto{\pgfqpoint{4.500000in}{3.520000in}}%
\pgfusepath{stroke}%
\end{pgfscope}%
\begin{pgfscope}%
\pgfsetrectcap%
\pgfsetmiterjoin%
\pgfsetlinewidth{0.803000pt}%
\definecolor{currentstroke}{rgb}{0.000000,0.000000,0.000000}%
\pgfsetstrokecolor{currentstroke}%
\pgfsetdash{}{0pt}%
\pgfpathmoveto{\pgfqpoint{0.625000in}{0.440000in}}%
\pgfpathlineto{\pgfqpoint{4.500000in}{0.440000in}}%
\pgfusepath{stroke}%
\end{pgfscope}%
\begin{pgfscope}%
\pgfsetrectcap%
\pgfsetmiterjoin%
\pgfsetlinewidth{0.803000pt}%
\definecolor{currentstroke}{rgb}{0.000000,0.000000,0.000000}%
\pgfsetstrokecolor{currentstroke}%
\pgfsetdash{}{0pt}%
\pgfpathmoveto{\pgfqpoint{0.625000in}{3.520000in}}%
\pgfpathlineto{\pgfqpoint{4.500000in}{3.520000in}}%
\pgfusepath{stroke}%
\end{pgfscope}%
\begin{pgfscope}%
\pgfsetbuttcap%
\pgfsetmiterjoin%
\definecolor{currentfill}{rgb}{1.000000,1.000000,1.000000}%
\pgfsetfillcolor{currentfill}%
\pgfsetfillopacity{0.800000}%
\pgfsetlinewidth{1.003750pt}%
\definecolor{currentstroke}{rgb}{0.800000,0.800000,0.800000}%
\pgfsetstrokecolor{currentstroke}%
\pgfsetstrokeopacity{0.800000}%
\pgfsetdash{}{0pt}%
\pgfpathmoveto{\pgfqpoint{3.042243in}{3.155267in}}%
\pgfpathlineto{\pgfqpoint{4.402778in}{3.155267in}}%
\pgfpathquadraticcurveto{\pgfqpoint{4.430556in}{3.155267in}}{\pgfqpoint{4.430556in}{3.183045in}}%
\pgfpathlineto{\pgfqpoint{4.430556in}{3.422778in}}%
\pgfpathquadraticcurveto{\pgfqpoint{4.430556in}{3.450556in}}{\pgfqpoint{4.402778in}{3.450556in}}%
\pgfpathlineto{\pgfqpoint{3.042243in}{3.450556in}}%
\pgfpathquadraticcurveto{\pgfqpoint{3.014465in}{3.450556in}}{\pgfqpoint{3.014465in}{3.422778in}}%
\pgfpathlineto{\pgfqpoint{3.014465in}{3.183045in}}%
\pgfpathquadraticcurveto{\pgfqpoint{3.014465in}{3.155267in}}{\pgfqpoint{3.042243in}{3.155267in}}%
\pgfpathclose%
\pgfusepath{stroke,fill}%
\end{pgfscope}%
\begin{pgfscope}%
\pgfsetbuttcap%
\pgfsetroundjoin%
\pgfsetlinewidth{1.505625pt}%
\definecolor{currentstroke}{rgb}{0.000000,0.000000,0.000000}%
\pgfsetstrokecolor{currentstroke}%
\pgfsetdash{{5.550000pt}{2.400000pt}}{0.000000pt}%
\pgfpathmoveto{\pgfqpoint{3.070021in}{3.321543in}}%
\pgfpathlineto{\pgfqpoint{3.347799in}{3.321543in}}%
\pgfusepath{stroke}%
\end{pgfscope}%
\begin{pgfscope}%
\definecolor{textcolor}{rgb}{0.000000,0.000000,0.000000}%
\pgfsetstrokecolor{textcolor}%
\pgfsetfillcolor{textcolor}%
\pgftext[x=3.458910in,y=3.272932in,left,base]{\color{textcolor}\sffamily\fontsize{10.000000}{12.000000}\selectfont \(\displaystyle 0.0087N_{\mathrm{QMC}}^{-0.6350}\)}%
\end{pgfscope}%
\end{pgfpicture}%
\makeatother%
\endgroup%

    \caption[Quasi-Monte-Carlo error, for $Q=u(\bzero)$ and $k=50$.]{Quasi-Monte-Carlo error with increasing number of Quasi-Monte-Carlo points, for $Q=u(\bzero)$ and $k=50$. \label{fig:qmcorigin50}}
  \end{figure}
\begin{figure}[h]
  \centering
%% Creator: Matplotlib, PGF backend
%%
%% To include the figure in your LaTeX document, write
%%   \input{<filename>.pgf}
%%
%% Make sure the required packages are loaded in your preamble
%%   \usepackage{pgf}
%%
%% Figures using additional raster images can only be included by \input if
%% they are in the same directory as the main LaTeX file. For loading figures
%% from other directories you can use the `import` package
%%   \usepackage{import}
%% and then include the figures with
%%   \import{<path to file>}{<filename>.pgf}
%%
%% Matplotlib used the following preamble
%%   \usepackage{fontspec}
%%   \setmainfont{DejaVuSerif.ttf}[Path=/home/owen/progs/firedrake-complex/firedrake/lib/python3.5/site-packages/matplotlib/mpl-data/fonts/ttf/]
%%   \setsansfont{DejaVuSans.ttf}[Path=/home/owen/progs/firedrake-complex/firedrake/lib/python3.5/site-packages/matplotlib/mpl-data/fonts/ttf/]
%%   \setmonofont{DejaVuSansMono.ttf}[Path=/home/owen/progs/firedrake-complex/firedrake/lib/python3.5/site-packages/matplotlib/mpl-data/fonts/ttf/]
%%
\begingroup%
\makeatletter%
\begin{pgfpicture}%
\pgfpathrectangle{\pgfpointorigin}{\pgfqpoint{5.000000in}{4.000000in}}%
\pgfusepath{use as bounding box, clip}%
\begin{pgfscope}%
\pgfsetbuttcap%
\pgfsetmiterjoin%
\definecolor{currentfill}{rgb}{1.000000,1.000000,1.000000}%
\pgfsetfillcolor{currentfill}%
\pgfsetlinewidth{0.000000pt}%
\definecolor{currentstroke}{rgb}{1.000000,1.000000,1.000000}%
\pgfsetstrokecolor{currentstroke}%
\pgfsetdash{}{0pt}%
\pgfpathmoveto{\pgfqpoint{0.000000in}{0.000000in}}%
\pgfpathlineto{\pgfqpoint{5.000000in}{0.000000in}}%
\pgfpathlineto{\pgfqpoint{5.000000in}{4.000000in}}%
\pgfpathlineto{\pgfqpoint{0.000000in}{4.000000in}}%
\pgfpathclose%
\pgfusepath{fill}%
\end{pgfscope}%
\begin{pgfscope}%
\pgfsetbuttcap%
\pgfsetmiterjoin%
\definecolor{currentfill}{rgb}{1.000000,1.000000,1.000000}%
\pgfsetfillcolor{currentfill}%
\pgfsetlinewidth{0.000000pt}%
\definecolor{currentstroke}{rgb}{0.000000,0.000000,0.000000}%
\pgfsetstrokecolor{currentstroke}%
\pgfsetstrokeopacity{0.000000}%
\pgfsetdash{}{0pt}%
\pgfpathmoveto{\pgfqpoint{0.625000in}{0.440000in}}%
\pgfpathlineto{\pgfqpoint{4.500000in}{0.440000in}}%
\pgfpathlineto{\pgfqpoint{4.500000in}{3.520000in}}%
\pgfpathlineto{\pgfqpoint{0.625000in}{3.520000in}}%
\pgfpathclose%
\pgfusepath{fill}%
\end{pgfscope}%
\begin{pgfscope}%
\pgfsetbuttcap%
\pgfsetroundjoin%
\definecolor{currentfill}{rgb}{0.000000,0.000000,0.000000}%
\pgfsetfillcolor{currentfill}%
\pgfsetlinewidth{0.803000pt}%
\definecolor{currentstroke}{rgb}{0.000000,0.000000,0.000000}%
\pgfsetstrokecolor{currentstroke}%
\pgfsetdash{}{0pt}%
\pgfsys@defobject{currentmarker}{\pgfqpoint{0.000000in}{-0.048611in}}{\pgfqpoint{0.000000in}{0.000000in}}{%
\pgfpathmoveto{\pgfqpoint{0.000000in}{0.000000in}}%
\pgfpathlineto{\pgfqpoint{0.000000in}{-0.048611in}}%
\pgfusepath{stroke,fill}%
}%
\begin{pgfscope}%
\pgfsys@transformshift{0.801136in}{0.440000in}%
\pgfsys@useobject{currentmarker}{}%
\end{pgfscope}%
\end{pgfscope}%
\begin{pgfscope}%
\definecolor{textcolor}{rgb}{0.000000,0.000000,0.000000}%
\pgfsetstrokecolor{textcolor}%
\pgfsetfillcolor{textcolor}%
\pgftext[x=0.801136in,y=0.342778in,,top]{\color{textcolor}\sffamily\fontsize{10.000000}{12.000000}\selectfont \(\displaystyle {10^{0}}\)}%
\end{pgfscope}%
\begin{pgfscope}%
\pgfsetbuttcap%
\pgfsetroundjoin%
\definecolor{currentfill}{rgb}{0.000000,0.000000,0.000000}%
\pgfsetfillcolor{currentfill}%
\pgfsetlinewidth{0.803000pt}%
\definecolor{currentstroke}{rgb}{0.000000,0.000000,0.000000}%
\pgfsetstrokecolor{currentstroke}%
\pgfsetdash{}{0pt}%
\pgfsys@defobject{currentmarker}{\pgfqpoint{0.000000in}{-0.048611in}}{\pgfqpoint{0.000000in}{0.000000in}}{%
\pgfpathmoveto{\pgfqpoint{0.000000in}{0.000000in}}%
\pgfpathlineto{\pgfqpoint{0.000000in}{-0.048611in}}%
\pgfusepath{stroke,fill}%
}%
\begin{pgfscope}%
\pgfsys@transformshift{1.864977in}{0.440000in}%
\pgfsys@useobject{currentmarker}{}%
\end{pgfscope}%
\end{pgfscope}%
\begin{pgfscope}%
\definecolor{textcolor}{rgb}{0.000000,0.000000,0.000000}%
\pgfsetstrokecolor{textcolor}%
\pgfsetfillcolor{textcolor}%
\pgftext[x=1.864977in,y=0.342778in,,top]{\color{textcolor}\sffamily\fontsize{10.000000}{12.000000}\selectfont \(\displaystyle {10^{1}}\)}%
\end{pgfscope}%
\begin{pgfscope}%
\pgfsetbuttcap%
\pgfsetroundjoin%
\definecolor{currentfill}{rgb}{0.000000,0.000000,0.000000}%
\pgfsetfillcolor{currentfill}%
\pgfsetlinewidth{0.803000pt}%
\definecolor{currentstroke}{rgb}{0.000000,0.000000,0.000000}%
\pgfsetstrokecolor{currentstroke}%
\pgfsetdash{}{0pt}%
\pgfsys@defobject{currentmarker}{\pgfqpoint{0.000000in}{-0.048611in}}{\pgfqpoint{0.000000in}{0.000000in}}{%
\pgfpathmoveto{\pgfqpoint{0.000000in}{0.000000in}}%
\pgfpathlineto{\pgfqpoint{0.000000in}{-0.048611in}}%
\pgfusepath{stroke,fill}%
}%
\begin{pgfscope}%
\pgfsys@transformshift{2.928818in}{0.440000in}%
\pgfsys@useobject{currentmarker}{}%
\end{pgfscope}%
\end{pgfscope}%
\begin{pgfscope}%
\definecolor{textcolor}{rgb}{0.000000,0.000000,0.000000}%
\pgfsetstrokecolor{textcolor}%
\pgfsetfillcolor{textcolor}%
\pgftext[x=2.928818in,y=0.342778in,,top]{\color{textcolor}\sffamily\fontsize{10.000000}{12.000000}\selectfont \(\displaystyle {10^{2}}\)}%
\end{pgfscope}%
\begin{pgfscope}%
\pgfsetbuttcap%
\pgfsetroundjoin%
\definecolor{currentfill}{rgb}{0.000000,0.000000,0.000000}%
\pgfsetfillcolor{currentfill}%
\pgfsetlinewidth{0.803000pt}%
\definecolor{currentstroke}{rgb}{0.000000,0.000000,0.000000}%
\pgfsetstrokecolor{currentstroke}%
\pgfsetdash{}{0pt}%
\pgfsys@defobject{currentmarker}{\pgfqpoint{0.000000in}{-0.048611in}}{\pgfqpoint{0.000000in}{0.000000in}}{%
\pgfpathmoveto{\pgfqpoint{0.000000in}{0.000000in}}%
\pgfpathlineto{\pgfqpoint{0.000000in}{-0.048611in}}%
\pgfusepath{stroke,fill}%
}%
\begin{pgfscope}%
\pgfsys@transformshift{3.992658in}{0.440000in}%
\pgfsys@useobject{currentmarker}{}%
\end{pgfscope}%
\end{pgfscope}%
\begin{pgfscope}%
\definecolor{textcolor}{rgb}{0.000000,0.000000,0.000000}%
\pgfsetstrokecolor{textcolor}%
\pgfsetfillcolor{textcolor}%
\pgftext[x=3.992658in,y=0.342778in,,top]{\color{textcolor}\sffamily\fontsize{10.000000}{12.000000}\selectfont \(\displaystyle {10^{3}}\)}%
\end{pgfscope}%
\begin{pgfscope}%
\pgfsetbuttcap%
\pgfsetroundjoin%
\definecolor{currentfill}{rgb}{0.000000,0.000000,0.000000}%
\pgfsetfillcolor{currentfill}%
\pgfsetlinewidth{0.602250pt}%
\definecolor{currentstroke}{rgb}{0.000000,0.000000,0.000000}%
\pgfsetstrokecolor{currentstroke}%
\pgfsetdash{}{0pt}%
\pgfsys@defobject{currentmarker}{\pgfqpoint{0.000000in}{-0.027778in}}{\pgfqpoint{0.000000in}{0.000000in}}{%
\pgfpathmoveto{\pgfqpoint{0.000000in}{0.000000in}}%
\pgfpathlineto{\pgfqpoint{0.000000in}{-0.027778in}}%
\pgfusepath{stroke,fill}%
}%
\begin{pgfscope}%
\pgfsys@transformshift{0.636345in}{0.440000in}%
\pgfsys@useobject{currentmarker}{}%
\end{pgfscope}%
\end{pgfscope}%
\begin{pgfscope}%
\pgfsetbuttcap%
\pgfsetroundjoin%
\definecolor{currentfill}{rgb}{0.000000,0.000000,0.000000}%
\pgfsetfillcolor{currentfill}%
\pgfsetlinewidth{0.602250pt}%
\definecolor{currentstroke}{rgb}{0.000000,0.000000,0.000000}%
\pgfsetstrokecolor{currentstroke}%
\pgfsetdash{}{0pt}%
\pgfsys@defobject{currentmarker}{\pgfqpoint{0.000000in}{-0.027778in}}{\pgfqpoint{0.000000in}{0.000000in}}{%
\pgfpathmoveto{\pgfqpoint{0.000000in}{0.000000in}}%
\pgfpathlineto{\pgfqpoint{0.000000in}{-0.027778in}}%
\pgfusepath{stroke,fill}%
}%
\begin{pgfscope}%
\pgfsys@transformshift{0.698040in}{0.440000in}%
\pgfsys@useobject{currentmarker}{}%
\end{pgfscope}%
\end{pgfscope}%
\begin{pgfscope}%
\pgfsetbuttcap%
\pgfsetroundjoin%
\definecolor{currentfill}{rgb}{0.000000,0.000000,0.000000}%
\pgfsetfillcolor{currentfill}%
\pgfsetlinewidth{0.602250pt}%
\definecolor{currentstroke}{rgb}{0.000000,0.000000,0.000000}%
\pgfsetstrokecolor{currentstroke}%
\pgfsetdash{}{0pt}%
\pgfsys@defobject{currentmarker}{\pgfqpoint{0.000000in}{-0.027778in}}{\pgfqpoint{0.000000in}{0.000000in}}{%
\pgfpathmoveto{\pgfqpoint{0.000000in}{0.000000in}}%
\pgfpathlineto{\pgfqpoint{0.000000in}{-0.027778in}}%
\pgfusepath{stroke,fill}%
}%
\begin{pgfscope}%
\pgfsys@transformshift{0.752458in}{0.440000in}%
\pgfsys@useobject{currentmarker}{}%
\end{pgfscope}%
\end{pgfscope}%
\begin{pgfscope}%
\pgfsetbuttcap%
\pgfsetroundjoin%
\definecolor{currentfill}{rgb}{0.000000,0.000000,0.000000}%
\pgfsetfillcolor{currentfill}%
\pgfsetlinewidth{0.602250pt}%
\definecolor{currentstroke}{rgb}{0.000000,0.000000,0.000000}%
\pgfsetstrokecolor{currentstroke}%
\pgfsetdash{}{0pt}%
\pgfsys@defobject{currentmarker}{\pgfqpoint{0.000000in}{-0.027778in}}{\pgfqpoint{0.000000in}{0.000000in}}{%
\pgfpathmoveto{\pgfqpoint{0.000000in}{0.000000in}}%
\pgfpathlineto{\pgfqpoint{0.000000in}{-0.027778in}}%
\pgfusepath{stroke,fill}%
}%
\begin{pgfscope}%
\pgfsys@transformshift{1.121384in}{0.440000in}%
\pgfsys@useobject{currentmarker}{}%
\end{pgfscope}%
\end{pgfscope}%
\begin{pgfscope}%
\pgfsetbuttcap%
\pgfsetroundjoin%
\definecolor{currentfill}{rgb}{0.000000,0.000000,0.000000}%
\pgfsetfillcolor{currentfill}%
\pgfsetlinewidth{0.602250pt}%
\definecolor{currentstroke}{rgb}{0.000000,0.000000,0.000000}%
\pgfsetstrokecolor{currentstroke}%
\pgfsetdash{}{0pt}%
\pgfsys@defobject{currentmarker}{\pgfqpoint{0.000000in}{-0.027778in}}{\pgfqpoint{0.000000in}{0.000000in}}{%
\pgfpathmoveto{\pgfqpoint{0.000000in}{0.000000in}}%
\pgfpathlineto{\pgfqpoint{0.000000in}{-0.027778in}}%
\pgfusepath{stroke,fill}%
}%
\begin{pgfscope}%
\pgfsys@transformshift{1.308717in}{0.440000in}%
\pgfsys@useobject{currentmarker}{}%
\end{pgfscope}%
\end{pgfscope}%
\begin{pgfscope}%
\pgfsetbuttcap%
\pgfsetroundjoin%
\definecolor{currentfill}{rgb}{0.000000,0.000000,0.000000}%
\pgfsetfillcolor{currentfill}%
\pgfsetlinewidth{0.602250pt}%
\definecolor{currentstroke}{rgb}{0.000000,0.000000,0.000000}%
\pgfsetstrokecolor{currentstroke}%
\pgfsetdash{}{0pt}%
\pgfsys@defobject{currentmarker}{\pgfqpoint{0.000000in}{-0.027778in}}{\pgfqpoint{0.000000in}{0.000000in}}{%
\pgfpathmoveto{\pgfqpoint{0.000000in}{0.000000in}}%
\pgfpathlineto{\pgfqpoint{0.000000in}{-0.027778in}}%
\pgfusepath{stroke,fill}%
}%
\begin{pgfscope}%
\pgfsys@transformshift{1.441632in}{0.440000in}%
\pgfsys@useobject{currentmarker}{}%
\end{pgfscope}%
\end{pgfscope}%
\begin{pgfscope}%
\pgfsetbuttcap%
\pgfsetroundjoin%
\definecolor{currentfill}{rgb}{0.000000,0.000000,0.000000}%
\pgfsetfillcolor{currentfill}%
\pgfsetlinewidth{0.602250pt}%
\definecolor{currentstroke}{rgb}{0.000000,0.000000,0.000000}%
\pgfsetstrokecolor{currentstroke}%
\pgfsetdash{}{0pt}%
\pgfsys@defobject{currentmarker}{\pgfqpoint{0.000000in}{-0.027778in}}{\pgfqpoint{0.000000in}{0.000000in}}{%
\pgfpathmoveto{\pgfqpoint{0.000000in}{0.000000in}}%
\pgfpathlineto{\pgfqpoint{0.000000in}{-0.027778in}}%
\pgfusepath{stroke,fill}%
}%
\begin{pgfscope}%
\pgfsys@transformshift{1.544729in}{0.440000in}%
\pgfsys@useobject{currentmarker}{}%
\end{pgfscope}%
\end{pgfscope}%
\begin{pgfscope}%
\pgfsetbuttcap%
\pgfsetroundjoin%
\definecolor{currentfill}{rgb}{0.000000,0.000000,0.000000}%
\pgfsetfillcolor{currentfill}%
\pgfsetlinewidth{0.602250pt}%
\definecolor{currentstroke}{rgb}{0.000000,0.000000,0.000000}%
\pgfsetstrokecolor{currentstroke}%
\pgfsetdash{}{0pt}%
\pgfsys@defobject{currentmarker}{\pgfqpoint{0.000000in}{-0.027778in}}{\pgfqpoint{0.000000in}{0.000000in}}{%
\pgfpathmoveto{\pgfqpoint{0.000000in}{0.000000in}}%
\pgfpathlineto{\pgfqpoint{0.000000in}{-0.027778in}}%
\pgfusepath{stroke,fill}%
}%
\begin{pgfscope}%
\pgfsys@transformshift{1.628965in}{0.440000in}%
\pgfsys@useobject{currentmarker}{}%
\end{pgfscope}%
\end{pgfscope}%
\begin{pgfscope}%
\pgfsetbuttcap%
\pgfsetroundjoin%
\definecolor{currentfill}{rgb}{0.000000,0.000000,0.000000}%
\pgfsetfillcolor{currentfill}%
\pgfsetlinewidth{0.602250pt}%
\definecolor{currentstroke}{rgb}{0.000000,0.000000,0.000000}%
\pgfsetstrokecolor{currentstroke}%
\pgfsetdash{}{0pt}%
\pgfsys@defobject{currentmarker}{\pgfqpoint{0.000000in}{-0.027778in}}{\pgfqpoint{0.000000in}{0.000000in}}{%
\pgfpathmoveto{\pgfqpoint{0.000000in}{0.000000in}}%
\pgfpathlineto{\pgfqpoint{0.000000in}{-0.027778in}}%
\pgfusepath{stroke,fill}%
}%
\begin{pgfscope}%
\pgfsys@transformshift{1.700186in}{0.440000in}%
\pgfsys@useobject{currentmarker}{}%
\end{pgfscope}%
\end{pgfscope}%
\begin{pgfscope}%
\pgfsetbuttcap%
\pgfsetroundjoin%
\definecolor{currentfill}{rgb}{0.000000,0.000000,0.000000}%
\pgfsetfillcolor{currentfill}%
\pgfsetlinewidth{0.602250pt}%
\definecolor{currentstroke}{rgb}{0.000000,0.000000,0.000000}%
\pgfsetstrokecolor{currentstroke}%
\pgfsetdash{}{0pt}%
\pgfsys@defobject{currentmarker}{\pgfqpoint{0.000000in}{-0.027778in}}{\pgfqpoint{0.000000in}{0.000000in}}{%
\pgfpathmoveto{\pgfqpoint{0.000000in}{0.000000in}}%
\pgfpathlineto{\pgfqpoint{0.000000in}{-0.027778in}}%
\pgfusepath{stroke,fill}%
}%
\begin{pgfscope}%
\pgfsys@transformshift{1.761880in}{0.440000in}%
\pgfsys@useobject{currentmarker}{}%
\end{pgfscope}%
\end{pgfscope}%
\begin{pgfscope}%
\pgfsetbuttcap%
\pgfsetroundjoin%
\definecolor{currentfill}{rgb}{0.000000,0.000000,0.000000}%
\pgfsetfillcolor{currentfill}%
\pgfsetlinewidth{0.602250pt}%
\definecolor{currentstroke}{rgb}{0.000000,0.000000,0.000000}%
\pgfsetstrokecolor{currentstroke}%
\pgfsetdash{}{0pt}%
\pgfsys@defobject{currentmarker}{\pgfqpoint{0.000000in}{-0.027778in}}{\pgfqpoint{0.000000in}{0.000000in}}{%
\pgfpathmoveto{\pgfqpoint{0.000000in}{0.000000in}}%
\pgfpathlineto{\pgfqpoint{0.000000in}{-0.027778in}}%
\pgfusepath{stroke,fill}%
}%
\begin{pgfscope}%
\pgfsys@transformshift{1.816298in}{0.440000in}%
\pgfsys@useobject{currentmarker}{}%
\end{pgfscope}%
\end{pgfscope}%
\begin{pgfscope}%
\pgfsetbuttcap%
\pgfsetroundjoin%
\definecolor{currentfill}{rgb}{0.000000,0.000000,0.000000}%
\pgfsetfillcolor{currentfill}%
\pgfsetlinewidth{0.602250pt}%
\definecolor{currentstroke}{rgb}{0.000000,0.000000,0.000000}%
\pgfsetstrokecolor{currentstroke}%
\pgfsetdash{}{0pt}%
\pgfsys@defobject{currentmarker}{\pgfqpoint{0.000000in}{-0.027778in}}{\pgfqpoint{0.000000in}{0.000000in}}{%
\pgfpathmoveto{\pgfqpoint{0.000000in}{0.000000in}}%
\pgfpathlineto{\pgfqpoint{0.000000in}{-0.027778in}}%
\pgfusepath{stroke,fill}%
}%
\begin{pgfscope}%
\pgfsys@transformshift{2.185225in}{0.440000in}%
\pgfsys@useobject{currentmarker}{}%
\end{pgfscope}%
\end{pgfscope}%
\begin{pgfscope}%
\pgfsetbuttcap%
\pgfsetroundjoin%
\definecolor{currentfill}{rgb}{0.000000,0.000000,0.000000}%
\pgfsetfillcolor{currentfill}%
\pgfsetlinewidth{0.602250pt}%
\definecolor{currentstroke}{rgb}{0.000000,0.000000,0.000000}%
\pgfsetstrokecolor{currentstroke}%
\pgfsetdash{}{0pt}%
\pgfsys@defobject{currentmarker}{\pgfqpoint{0.000000in}{-0.027778in}}{\pgfqpoint{0.000000in}{0.000000in}}{%
\pgfpathmoveto{\pgfqpoint{0.000000in}{0.000000in}}%
\pgfpathlineto{\pgfqpoint{0.000000in}{-0.027778in}}%
\pgfusepath{stroke,fill}%
}%
\begin{pgfscope}%
\pgfsys@transformshift{2.372558in}{0.440000in}%
\pgfsys@useobject{currentmarker}{}%
\end{pgfscope}%
\end{pgfscope}%
\begin{pgfscope}%
\pgfsetbuttcap%
\pgfsetroundjoin%
\definecolor{currentfill}{rgb}{0.000000,0.000000,0.000000}%
\pgfsetfillcolor{currentfill}%
\pgfsetlinewidth{0.602250pt}%
\definecolor{currentstroke}{rgb}{0.000000,0.000000,0.000000}%
\pgfsetstrokecolor{currentstroke}%
\pgfsetdash{}{0pt}%
\pgfsys@defobject{currentmarker}{\pgfqpoint{0.000000in}{-0.027778in}}{\pgfqpoint{0.000000in}{0.000000in}}{%
\pgfpathmoveto{\pgfqpoint{0.000000in}{0.000000in}}%
\pgfpathlineto{\pgfqpoint{0.000000in}{-0.027778in}}%
\pgfusepath{stroke,fill}%
}%
\begin{pgfscope}%
\pgfsys@transformshift{2.505473in}{0.440000in}%
\pgfsys@useobject{currentmarker}{}%
\end{pgfscope}%
\end{pgfscope}%
\begin{pgfscope}%
\pgfsetbuttcap%
\pgfsetroundjoin%
\definecolor{currentfill}{rgb}{0.000000,0.000000,0.000000}%
\pgfsetfillcolor{currentfill}%
\pgfsetlinewidth{0.602250pt}%
\definecolor{currentstroke}{rgb}{0.000000,0.000000,0.000000}%
\pgfsetstrokecolor{currentstroke}%
\pgfsetdash{}{0pt}%
\pgfsys@defobject{currentmarker}{\pgfqpoint{0.000000in}{-0.027778in}}{\pgfqpoint{0.000000in}{0.000000in}}{%
\pgfpathmoveto{\pgfqpoint{0.000000in}{0.000000in}}%
\pgfpathlineto{\pgfqpoint{0.000000in}{-0.027778in}}%
\pgfusepath{stroke,fill}%
}%
\begin{pgfscope}%
\pgfsys@transformshift{2.608570in}{0.440000in}%
\pgfsys@useobject{currentmarker}{}%
\end{pgfscope}%
\end{pgfscope}%
\begin{pgfscope}%
\pgfsetbuttcap%
\pgfsetroundjoin%
\definecolor{currentfill}{rgb}{0.000000,0.000000,0.000000}%
\pgfsetfillcolor{currentfill}%
\pgfsetlinewidth{0.602250pt}%
\definecolor{currentstroke}{rgb}{0.000000,0.000000,0.000000}%
\pgfsetstrokecolor{currentstroke}%
\pgfsetdash{}{0pt}%
\pgfsys@defobject{currentmarker}{\pgfqpoint{0.000000in}{-0.027778in}}{\pgfqpoint{0.000000in}{0.000000in}}{%
\pgfpathmoveto{\pgfqpoint{0.000000in}{0.000000in}}%
\pgfpathlineto{\pgfqpoint{0.000000in}{-0.027778in}}%
\pgfusepath{stroke,fill}%
}%
\begin{pgfscope}%
\pgfsys@transformshift{2.692806in}{0.440000in}%
\pgfsys@useobject{currentmarker}{}%
\end{pgfscope}%
\end{pgfscope}%
\begin{pgfscope}%
\pgfsetbuttcap%
\pgfsetroundjoin%
\definecolor{currentfill}{rgb}{0.000000,0.000000,0.000000}%
\pgfsetfillcolor{currentfill}%
\pgfsetlinewidth{0.602250pt}%
\definecolor{currentstroke}{rgb}{0.000000,0.000000,0.000000}%
\pgfsetstrokecolor{currentstroke}%
\pgfsetdash{}{0pt}%
\pgfsys@defobject{currentmarker}{\pgfqpoint{0.000000in}{-0.027778in}}{\pgfqpoint{0.000000in}{0.000000in}}{%
\pgfpathmoveto{\pgfqpoint{0.000000in}{0.000000in}}%
\pgfpathlineto{\pgfqpoint{0.000000in}{-0.027778in}}%
\pgfusepath{stroke,fill}%
}%
\begin{pgfscope}%
\pgfsys@transformshift{2.764027in}{0.440000in}%
\pgfsys@useobject{currentmarker}{}%
\end{pgfscope}%
\end{pgfscope}%
\begin{pgfscope}%
\pgfsetbuttcap%
\pgfsetroundjoin%
\definecolor{currentfill}{rgb}{0.000000,0.000000,0.000000}%
\pgfsetfillcolor{currentfill}%
\pgfsetlinewidth{0.602250pt}%
\definecolor{currentstroke}{rgb}{0.000000,0.000000,0.000000}%
\pgfsetstrokecolor{currentstroke}%
\pgfsetdash{}{0pt}%
\pgfsys@defobject{currentmarker}{\pgfqpoint{0.000000in}{-0.027778in}}{\pgfqpoint{0.000000in}{0.000000in}}{%
\pgfpathmoveto{\pgfqpoint{0.000000in}{0.000000in}}%
\pgfpathlineto{\pgfqpoint{0.000000in}{-0.027778in}}%
\pgfusepath{stroke,fill}%
}%
\begin{pgfscope}%
\pgfsys@transformshift{2.825721in}{0.440000in}%
\pgfsys@useobject{currentmarker}{}%
\end{pgfscope}%
\end{pgfscope}%
\begin{pgfscope}%
\pgfsetbuttcap%
\pgfsetroundjoin%
\definecolor{currentfill}{rgb}{0.000000,0.000000,0.000000}%
\pgfsetfillcolor{currentfill}%
\pgfsetlinewidth{0.602250pt}%
\definecolor{currentstroke}{rgb}{0.000000,0.000000,0.000000}%
\pgfsetstrokecolor{currentstroke}%
\pgfsetdash{}{0pt}%
\pgfsys@defobject{currentmarker}{\pgfqpoint{0.000000in}{-0.027778in}}{\pgfqpoint{0.000000in}{0.000000in}}{%
\pgfpathmoveto{\pgfqpoint{0.000000in}{0.000000in}}%
\pgfpathlineto{\pgfqpoint{0.000000in}{-0.027778in}}%
\pgfusepath{stroke,fill}%
}%
\begin{pgfscope}%
\pgfsys@transformshift{2.880139in}{0.440000in}%
\pgfsys@useobject{currentmarker}{}%
\end{pgfscope}%
\end{pgfscope}%
\begin{pgfscope}%
\pgfsetbuttcap%
\pgfsetroundjoin%
\definecolor{currentfill}{rgb}{0.000000,0.000000,0.000000}%
\pgfsetfillcolor{currentfill}%
\pgfsetlinewidth{0.602250pt}%
\definecolor{currentstroke}{rgb}{0.000000,0.000000,0.000000}%
\pgfsetstrokecolor{currentstroke}%
\pgfsetdash{}{0pt}%
\pgfsys@defobject{currentmarker}{\pgfqpoint{0.000000in}{-0.027778in}}{\pgfqpoint{0.000000in}{0.000000in}}{%
\pgfpathmoveto{\pgfqpoint{0.000000in}{0.000000in}}%
\pgfpathlineto{\pgfqpoint{0.000000in}{-0.027778in}}%
\pgfusepath{stroke,fill}%
}%
\begin{pgfscope}%
\pgfsys@transformshift{3.249066in}{0.440000in}%
\pgfsys@useobject{currentmarker}{}%
\end{pgfscope}%
\end{pgfscope}%
\begin{pgfscope}%
\pgfsetbuttcap%
\pgfsetroundjoin%
\definecolor{currentfill}{rgb}{0.000000,0.000000,0.000000}%
\pgfsetfillcolor{currentfill}%
\pgfsetlinewidth{0.602250pt}%
\definecolor{currentstroke}{rgb}{0.000000,0.000000,0.000000}%
\pgfsetstrokecolor{currentstroke}%
\pgfsetdash{}{0pt}%
\pgfsys@defobject{currentmarker}{\pgfqpoint{0.000000in}{-0.027778in}}{\pgfqpoint{0.000000in}{0.000000in}}{%
\pgfpathmoveto{\pgfqpoint{0.000000in}{0.000000in}}%
\pgfpathlineto{\pgfqpoint{0.000000in}{-0.027778in}}%
\pgfusepath{stroke,fill}%
}%
\begin{pgfscope}%
\pgfsys@transformshift{3.436399in}{0.440000in}%
\pgfsys@useobject{currentmarker}{}%
\end{pgfscope}%
\end{pgfscope}%
\begin{pgfscope}%
\pgfsetbuttcap%
\pgfsetroundjoin%
\definecolor{currentfill}{rgb}{0.000000,0.000000,0.000000}%
\pgfsetfillcolor{currentfill}%
\pgfsetlinewidth{0.602250pt}%
\definecolor{currentstroke}{rgb}{0.000000,0.000000,0.000000}%
\pgfsetstrokecolor{currentstroke}%
\pgfsetdash{}{0pt}%
\pgfsys@defobject{currentmarker}{\pgfqpoint{0.000000in}{-0.027778in}}{\pgfqpoint{0.000000in}{0.000000in}}{%
\pgfpathmoveto{\pgfqpoint{0.000000in}{0.000000in}}%
\pgfpathlineto{\pgfqpoint{0.000000in}{-0.027778in}}%
\pgfusepath{stroke,fill}%
}%
\begin{pgfscope}%
\pgfsys@transformshift{3.569313in}{0.440000in}%
\pgfsys@useobject{currentmarker}{}%
\end{pgfscope}%
\end{pgfscope}%
\begin{pgfscope}%
\pgfsetbuttcap%
\pgfsetroundjoin%
\definecolor{currentfill}{rgb}{0.000000,0.000000,0.000000}%
\pgfsetfillcolor{currentfill}%
\pgfsetlinewidth{0.602250pt}%
\definecolor{currentstroke}{rgb}{0.000000,0.000000,0.000000}%
\pgfsetstrokecolor{currentstroke}%
\pgfsetdash{}{0pt}%
\pgfsys@defobject{currentmarker}{\pgfqpoint{0.000000in}{-0.027778in}}{\pgfqpoint{0.000000in}{0.000000in}}{%
\pgfpathmoveto{\pgfqpoint{0.000000in}{0.000000in}}%
\pgfpathlineto{\pgfqpoint{0.000000in}{-0.027778in}}%
\pgfusepath{stroke,fill}%
}%
\begin{pgfscope}%
\pgfsys@transformshift{3.672410in}{0.440000in}%
\pgfsys@useobject{currentmarker}{}%
\end{pgfscope}%
\end{pgfscope}%
\begin{pgfscope}%
\pgfsetbuttcap%
\pgfsetroundjoin%
\definecolor{currentfill}{rgb}{0.000000,0.000000,0.000000}%
\pgfsetfillcolor{currentfill}%
\pgfsetlinewidth{0.602250pt}%
\definecolor{currentstroke}{rgb}{0.000000,0.000000,0.000000}%
\pgfsetstrokecolor{currentstroke}%
\pgfsetdash{}{0pt}%
\pgfsys@defobject{currentmarker}{\pgfqpoint{0.000000in}{-0.027778in}}{\pgfqpoint{0.000000in}{0.000000in}}{%
\pgfpathmoveto{\pgfqpoint{0.000000in}{0.000000in}}%
\pgfpathlineto{\pgfqpoint{0.000000in}{-0.027778in}}%
\pgfusepath{stroke,fill}%
}%
\begin{pgfscope}%
\pgfsys@transformshift{3.756646in}{0.440000in}%
\pgfsys@useobject{currentmarker}{}%
\end{pgfscope}%
\end{pgfscope}%
\begin{pgfscope}%
\pgfsetbuttcap%
\pgfsetroundjoin%
\definecolor{currentfill}{rgb}{0.000000,0.000000,0.000000}%
\pgfsetfillcolor{currentfill}%
\pgfsetlinewidth{0.602250pt}%
\definecolor{currentstroke}{rgb}{0.000000,0.000000,0.000000}%
\pgfsetstrokecolor{currentstroke}%
\pgfsetdash{}{0pt}%
\pgfsys@defobject{currentmarker}{\pgfqpoint{0.000000in}{-0.027778in}}{\pgfqpoint{0.000000in}{0.000000in}}{%
\pgfpathmoveto{\pgfqpoint{0.000000in}{0.000000in}}%
\pgfpathlineto{\pgfqpoint{0.000000in}{-0.027778in}}%
\pgfusepath{stroke,fill}%
}%
\begin{pgfscope}%
\pgfsys@transformshift{3.827867in}{0.440000in}%
\pgfsys@useobject{currentmarker}{}%
\end{pgfscope}%
\end{pgfscope}%
\begin{pgfscope}%
\pgfsetbuttcap%
\pgfsetroundjoin%
\definecolor{currentfill}{rgb}{0.000000,0.000000,0.000000}%
\pgfsetfillcolor{currentfill}%
\pgfsetlinewidth{0.602250pt}%
\definecolor{currentstroke}{rgb}{0.000000,0.000000,0.000000}%
\pgfsetstrokecolor{currentstroke}%
\pgfsetdash{}{0pt}%
\pgfsys@defobject{currentmarker}{\pgfqpoint{0.000000in}{-0.027778in}}{\pgfqpoint{0.000000in}{0.000000in}}{%
\pgfpathmoveto{\pgfqpoint{0.000000in}{0.000000in}}%
\pgfpathlineto{\pgfqpoint{0.000000in}{-0.027778in}}%
\pgfusepath{stroke,fill}%
}%
\begin{pgfscope}%
\pgfsys@transformshift{3.889561in}{0.440000in}%
\pgfsys@useobject{currentmarker}{}%
\end{pgfscope}%
\end{pgfscope}%
\begin{pgfscope}%
\pgfsetbuttcap%
\pgfsetroundjoin%
\definecolor{currentfill}{rgb}{0.000000,0.000000,0.000000}%
\pgfsetfillcolor{currentfill}%
\pgfsetlinewidth{0.602250pt}%
\definecolor{currentstroke}{rgb}{0.000000,0.000000,0.000000}%
\pgfsetstrokecolor{currentstroke}%
\pgfsetdash{}{0pt}%
\pgfsys@defobject{currentmarker}{\pgfqpoint{0.000000in}{-0.027778in}}{\pgfqpoint{0.000000in}{0.000000in}}{%
\pgfpathmoveto{\pgfqpoint{0.000000in}{0.000000in}}%
\pgfpathlineto{\pgfqpoint{0.000000in}{-0.027778in}}%
\pgfusepath{stroke,fill}%
}%
\begin{pgfscope}%
\pgfsys@transformshift{3.943980in}{0.440000in}%
\pgfsys@useobject{currentmarker}{}%
\end{pgfscope}%
\end{pgfscope}%
\begin{pgfscope}%
\pgfsetbuttcap%
\pgfsetroundjoin%
\definecolor{currentfill}{rgb}{0.000000,0.000000,0.000000}%
\pgfsetfillcolor{currentfill}%
\pgfsetlinewidth{0.602250pt}%
\definecolor{currentstroke}{rgb}{0.000000,0.000000,0.000000}%
\pgfsetstrokecolor{currentstroke}%
\pgfsetdash{}{0pt}%
\pgfsys@defobject{currentmarker}{\pgfqpoint{0.000000in}{-0.027778in}}{\pgfqpoint{0.000000in}{0.000000in}}{%
\pgfpathmoveto{\pgfqpoint{0.000000in}{0.000000in}}%
\pgfpathlineto{\pgfqpoint{0.000000in}{-0.027778in}}%
\pgfusepath{stroke,fill}%
}%
\begin{pgfscope}%
\pgfsys@transformshift{4.312906in}{0.440000in}%
\pgfsys@useobject{currentmarker}{}%
\end{pgfscope}%
\end{pgfscope}%
\begin{pgfscope}%
\pgfsetbuttcap%
\pgfsetroundjoin%
\definecolor{currentfill}{rgb}{0.000000,0.000000,0.000000}%
\pgfsetfillcolor{currentfill}%
\pgfsetlinewidth{0.602250pt}%
\definecolor{currentstroke}{rgb}{0.000000,0.000000,0.000000}%
\pgfsetstrokecolor{currentstroke}%
\pgfsetdash{}{0pt}%
\pgfsys@defobject{currentmarker}{\pgfqpoint{0.000000in}{-0.027778in}}{\pgfqpoint{0.000000in}{0.000000in}}{%
\pgfpathmoveto{\pgfqpoint{0.000000in}{0.000000in}}%
\pgfpathlineto{\pgfqpoint{0.000000in}{-0.027778in}}%
\pgfusepath{stroke,fill}%
}%
\begin{pgfscope}%
\pgfsys@transformshift{4.500239in}{0.440000in}%
\pgfsys@useobject{currentmarker}{}%
\end{pgfscope}%
\end{pgfscope}%
\begin{pgfscope}%
\definecolor{textcolor}{rgb}{0.000000,0.000000,0.000000}%
\pgfsetstrokecolor{textcolor}%
\pgfsetfillcolor{textcolor}%
\pgftext[x=2.562500in,y=0.152809in,,top]{\color{textcolor}\sffamily\fontsize{10.000000}{12.000000}\selectfont \(\displaystyle N_\mathrm{QMC}\)}%
\end{pgfscope}%
\begin{pgfscope}%
\pgfsetbuttcap%
\pgfsetroundjoin%
\definecolor{currentfill}{rgb}{0.000000,0.000000,0.000000}%
\pgfsetfillcolor{currentfill}%
\pgfsetlinewidth{0.803000pt}%
\definecolor{currentstroke}{rgb}{0.000000,0.000000,0.000000}%
\pgfsetstrokecolor{currentstroke}%
\pgfsetdash{}{0pt}%
\pgfsys@defobject{currentmarker}{\pgfqpoint{-0.048611in}{0.000000in}}{\pgfqpoint{0.000000in}{0.000000in}}{%
\pgfpathmoveto{\pgfqpoint{0.000000in}{0.000000in}}%
\pgfpathlineto{\pgfqpoint{-0.048611in}{0.000000in}}%
\pgfusepath{stroke,fill}%
}%
\begin{pgfscope}%
\pgfsys@transformshift{0.625000in}{1.119339in}%
\pgfsys@useobject{currentmarker}{}%
\end{pgfscope}%
\end{pgfscope}%
\begin{pgfscope}%
\definecolor{textcolor}{rgb}{0.000000,0.000000,0.000000}%
\pgfsetstrokecolor{textcolor}%
\pgfsetfillcolor{textcolor}%
\pgftext[x=0.239775in,y=1.066578in,left,base]{\color{textcolor}\sffamily\fontsize{10.000000}{12.000000}\selectfont \(\displaystyle {10^{-4}}\)}%
\end{pgfscope}%
\begin{pgfscope}%
\pgfsetbuttcap%
\pgfsetroundjoin%
\definecolor{currentfill}{rgb}{0.000000,0.000000,0.000000}%
\pgfsetfillcolor{currentfill}%
\pgfsetlinewidth{0.803000pt}%
\definecolor{currentstroke}{rgb}{0.000000,0.000000,0.000000}%
\pgfsetstrokecolor{currentstroke}%
\pgfsetdash{}{0pt}%
\pgfsys@defobject{currentmarker}{\pgfqpoint{-0.048611in}{0.000000in}}{\pgfqpoint{0.000000in}{0.000000in}}{%
\pgfpathmoveto{\pgfqpoint{0.000000in}{0.000000in}}%
\pgfpathlineto{\pgfqpoint{-0.048611in}{0.000000in}}%
\pgfusepath{stroke,fill}%
}%
\begin{pgfscope}%
\pgfsys@transformshift{0.625000in}{2.265024in}%
\pgfsys@useobject{currentmarker}{}%
\end{pgfscope}%
\end{pgfscope}%
\begin{pgfscope}%
\definecolor{textcolor}{rgb}{0.000000,0.000000,0.000000}%
\pgfsetstrokecolor{textcolor}%
\pgfsetfillcolor{textcolor}%
\pgftext[x=0.239775in,y=2.212263in,left,base]{\color{textcolor}\sffamily\fontsize{10.000000}{12.000000}\selectfont \(\displaystyle {10^{-3}}\)}%
\end{pgfscope}%
\begin{pgfscope}%
\pgfsetbuttcap%
\pgfsetroundjoin%
\definecolor{currentfill}{rgb}{0.000000,0.000000,0.000000}%
\pgfsetfillcolor{currentfill}%
\pgfsetlinewidth{0.803000pt}%
\definecolor{currentstroke}{rgb}{0.000000,0.000000,0.000000}%
\pgfsetstrokecolor{currentstroke}%
\pgfsetdash{}{0pt}%
\pgfsys@defobject{currentmarker}{\pgfqpoint{-0.048611in}{0.000000in}}{\pgfqpoint{0.000000in}{0.000000in}}{%
\pgfpathmoveto{\pgfqpoint{0.000000in}{0.000000in}}%
\pgfpathlineto{\pgfqpoint{-0.048611in}{0.000000in}}%
\pgfusepath{stroke,fill}%
}%
\begin{pgfscope}%
\pgfsys@transformshift{0.625000in}{3.410709in}%
\pgfsys@useobject{currentmarker}{}%
\end{pgfscope}%
\end{pgfscope}%
\begin{pgfscope}%
\definecolor{textcolor}{rgb}{0.000000,0.000000,0.000000}%
\pgfsetstrokecolor{textcolor}%
\pgfsetfillcolor{textcolor}%
\pgftext[x=0.239775in,y=3.357948in,left,base]{\color{textcolor}\sffamily\fontsize{10.000000}{12.000000}\selectfont \(\displaystyle {10^{-2}}\)}%
\end{pgfscope}%
\begin{pgfscope}%
\pgfsetbuttcap%
\pgfsetroundjoin%
\definecolor{currentfill}{rgb}{0.000000,0.000000,0.000000}%
\pgfsetfillcolor{currentfill}%
\pgfsetlinewidth{0.602250pt}%
\definecolor{currentstroke}{rgb}{0.000000,0.000000,0.000000}%
\pgfsetstrokecolor{currentstroke}%
\pgfsetdash{}{0pt}%
\pgfsys@defobject{currentmarker}{\pgfqpoint{-0.027778in}{0.000000in}}{\pgfqpoint{0.000000in}{0.000000in}}{%
\pgfpathmoveto{\pgfqpoint{0.000000in}{0.000000in}}%
\pgfpathlineto{\pgfqpoint{-0.027778in}{0.000000in}}%
\pgfusepath{stroke,fill}%
}%
\begin{pgfscope}%
\pgfsys@transformshift{0.625000in}{0.520285in}%
\pgfsys@useobject{currentmarker}{}%
\end{pgfscope}%
\end{pgfscope}%
\begin{pgfscope}%
\pgfsetbuttcap%
\pgfsetroundjoin%
\definecolor{currentfill}{rgb}{0.000000,0.000000,0.000000}%
\pgfsetfillcolor{currentfill}%
\pgfsetlinewidth{0.602250pt}%
\definecolor{currentstroke}{rgb}{0.000000,0.000000,0.000000}%
\pgfsetstrokecolor{currentstroke}%
\pgfsetdash{}{0pt}%
\pgfsys@defobject{currentmarker}{\pgfqpoint{-0.027778in}{0.000000in}}{\pgfqpoint{0.000000in}{0.000000in}}{%
\pgfpathmoveto{\pgfqpoint{0.000000in}{0.000000in}}%
\pgfpathlineto{\pgfqpoint{-0.027778in}{0.000000in}}%
\pgfusepath{stroke,fill}%
}%
\begin{pgfscope}%
\pgfsys@transformshift{0.625000in}{0.663425in}%
\pgfsys@useobject{currentmarker}{}%
\end{pgfscope}%
\end{pgfscope}%
\begin{pgfscope}%
\pgfsetbuttcap%
\pgfsetroundjoin%
\definecolor{currentfill}{rgb}{0.000000,0.000000,0.000000}%
\pgfsetfillcolor{currentfill}%
\pgfsetlinewidth{0.602250pt}%
\definecolor{currentstroke}{rgb}{0.000000,0.000000,0.000000}%
\pgfsetstrokecolor{currentstroke}%
\pgfsetdash{}{0pt}%
\pgfsys@defobject{currentmarker}{\pgfqpoint{-0.027778in}{0.000000in}}{\pgfqpoint{0.000000in}{0.000000in}}{%
\pgfpathmoveto{\pgfqpoint{0.000000in}{0.000000in}}%
\pgfpathlineto{\pgfqpoint{-0.027778in}{0.000000in}}%
\pgfusepath{stroke,fill}%
}%
\begin{pgfscope}%
\pgfsys@transformshift{0.625000in}{0.774454in}%
\pgfsys@useobject{currentmarker}{}%
\end{pgfscope}%
\end{pgfscope}%
\begin{pgfscope}%
\pgfsetbuttcap%
\pgfsetroundjoin%
\definecolor{currentfill}{rgb}{0.000000,0.000000,0.000000}%
\pgfsetfillcolor{currentfill}%
\pgfsetlinewidth{0.602250pt}%
\definecolor{currentstroke}{rgb}{0.000000,0.000000,0.000000}%
\pgfsetstrokecolor{currentstroke}%
\pgfsetdash{}{0pt}%
\pgfsys@defobject{currentmarker}{\pgfqpoint{-0.027778in}{0.000000in}}{\pgfqpoint{0.000000in}{0.000000in}}{%
\pgfpathmoveto{\pgfqpoint{0.000000in}{0.000000in}}%
\pgfpathlineto{\pgfqpoint{-0.027778in}{0.000000in}}%
\pgfusepath{stroke,fill}%
}%
\begin{pgfscope}%
\pgfsys@transformshift{0.625000in}{0.865170in}%
\pgfsys@useobject{currentmarker}{}%
\end{pgfscope}%
\end{pgfscope}%
\begin{pgfscope}%
\pgfsetbuttcap%
\pgfsetroundjoin%
\definecolor{currentfill}{rgb}{0.000000,0.000000,0.000000}%
\pgfsetfillcolor{currentfill}%
\pgfsetlinewidth{0.602250pt}%
\definecolor{currentstroke}{rgb}{0.000000,0.000000,0.000000}%
\pgfsetstrokecolor{currentstroke}%
\pgfsetdash{}{0pt}%
\pgfsys@defobject{currentmarker}{\pgfqpoint{-0.027778in}{0.000000in}}{\pgfqpoint{0.000000in}{0.000000in}}{%
\pgfpathmoveto{\pgfqpoint{0.000000in}{0.000000in}}%
\pgfpathlineto{\pgfqpoint{-0.027778in}{0.000000in}}%
\pgfusepath{stroke,fill}%
}%
\begin{pgfscope}%
\pgfsys@transformshift{0.625000in}{0.941870in}%
\pgfsys@useobject{currentmarker}{}%
\end{pgfscope}%
\end{pgfscope}%
\begin{pgfscope}%
\pgfsetbuttcap%
\pgfsetroundjoin%
\definecolor{currentfill}{rgb}{0.000000,0.000000,0.000000}%
\pgfsetfillcolor{currentfill}%
\pgfsetlinewidth{0.602250pt}%
\definecolor{currentstroke}{rgb}{0.000000,0.000000,0.000000}%
\pgfsetstrokecolor{currentstroke}%
\pgfsetdash{}{0pt}%
\pgfsys@defobject{currentmarker}{\pgfqpoint{-0.027778in}{0.000000in}}{\pgfqpoint{0.000000in}{0.000000in}}{%
\pgfpathmoveto{\pgfqpoint{0.000000in}{0.000000in}}%
\pgfpathlineto{\pgfqpoint{-0.027778in}{0.000000in}}%
\pgfusepath{stroke,fill}%
}%
\begin{pgfscope}%
\pgfsys@transformshift{0.625000in}{1.008311in}%
\pgfsys@useobject{currentmarker}{}%
\end{pgfscope}%
\end{pgfscope}%
\begin{pgfscope}%
\pgfsetbuttcap%
\pgfsetroundjoin%
\definecolor{currentfill}{rgb}{0.000000,0.000000,0.000000}%
\pgfsetfillcolor{currentfill}%
\pgfsetlinewidth{0.602250pt}%
\definecolor{currentstroke}{rgb}{0.000000,0.000000,0.000000}%
\pgfsetstrokecolor{currentstroke}%
\pgfsetdash{}{0pt}%
\pgfsys@defobject{currentmarker}{\pgfqpoint{-0.027778in}{0.000000in}}{\pgfqpoint{0.000000in}{0.000000in}}{%
\pgfpathmoveto{\pgfqpoint{0.000000in}{0.000000in}}%
\pgfpathlineto{\pgfqpoint{-0.027778in}{0.000000in}}%
\pgfusepath{stroke,fill}%
}%
\begin{pgfscope}%
\pgfsys@transformshift{0.625000in}{1.066916in}%
\pgfsys@useobject{currentmarker}{}%
\end{pgfscope}%
\end{pgfscope}%
\begin{pgfscope}%
\pgfsetbuttcap%
\pgfsetroundjoin%
\definecolor{currentfill}{rgb}{0.000000,0.000000,0.000000}%
\pgfsetfillcolor{currentfill}%
\pgfsetlinewidth{0.602250pt}%
\definecolor{currentstroke}{rgb}{0.000000,0.000000,0.000000}%
\pgfsetstrokecolor{currentstroke}%
\pgfsetdash{}{0pt}%
\pgfsys@defobject{currentmarker}{\pgfqpoint{-0.027778in}{0.000000in}}{\pgfqpoint{0.000000in}{0.000000in}}{%
\pgfpathmoveto{\pgfqpoint{0.000000in}{0.000000in}}%
\pgfpathlineto{\pgfqpoint{-0.027778in}{0.000000in}}%
\pgfusepath{stroke,fill}%
}%
\begin{pgfscope}%
\pgfsys@transformshift{0.625000in}{1.464225in}%
\pgfsys@useobject{currentmarker}{}%
\end{pgfscope}%
\end{pgfscope}%
\begin{pgfscope}%
\pgfsetbuttcap%
\pgfsetroundjoin%
\definecolor{currentfill}{rgb}{0.000000,0.000000,0.000000}%
\pgfsetfillcolor{currentfill}%
\pgfsetlinewidth{0.602250pt}%
\definecolor{currentstroke}{rgb}{0.000000,0.000000,0.000000}%
\pgfsetstrokecolor{currentstroke}%
\pgfsetdash{}{0pt}%
\pgfsys@defobject{currentmarker}{\pgfqpoint{-0.027778in}{0.000000in}}{\pgfqpoint{0.000000in}{0.000000in}}{%
\pgfpathmoveto{\pgfqpoint{0.000000in}{0.000000in}}%
\pgfpathlineto{\pgfqpoint{-0.027778in}{0.000000in}}%
\pgfusepath{stroke,fill}%
}%
\begin{pgfscope}%
\pgfsys@transformshift{0.625000in}{1.665970in}%
\pgfsys@useobject{currentmarker}{}%
\end{pgfscope}%
\end{pgfscope}%
\begin{pgfscope}%
\pgfsetbuttcap%
\pgfsetroundjoin%
\definecolor{currentfill}{rgb}{0.000000,0.000000,0.000000}%
\pgfsetfillcolor{currentfill}%
\pgfsetlinewidth{0.602250pt}%
\definecolor{currentstroke}{rgb}{0.000000,0.000000,0.000000}%
\pgfsetstrokecolor{currentstroke}%
\pgfsetdash{}{0pt}%
\pgfsys@defobject{currentmarker}{\pgfqpoint{-0.027778in}{0.000000in}}{\pgfqpoint{0.000000in}{0.000000in}}{%
\pgfpathmoveto{\pgfqpoint{0.000000in}{0.000000in}}%
\pgfpathlineto{\pgfqpoint{-0.027778in}{0.000000in}}%
\pgfusepath{stroke,fill}%
}%
\begin{pgfscope}%
\pgfsys@transformshift{0.625000in}{1.809110in}%
\pgfsys@useobject{currentmarker}{}%
\end{pgfscope}%
\end{pgfscope}%
\begin{pgfscope}%
\pgfsetbuttcap%
\pgfsetroundjoin%
\definecolor{currentfill}{rgb}{0.000000,0.000000,0.000000}%
\pgfsetfillcolor{currentfill}%
\pgfsetlinewidth{0.602250pt}%
\definecolor{currentstroke}{rgb}{0.000000,0.000000,0.000000}%
\pgfsetstrokecolor{currentstroke}%
\pgfsetdash{}{0pt}%
\pgfsys@defobject{currentmarker}{\pgfqpoint{-0.027778in}{0.000000in}}{\pgfqpoint{0.000000in}{0.000000in}}{%
\pgfpathmoveto{\pgfqpoint{0.000000in}{0.000000in}}%
\pgfpathlineto{\pgfqpoint{-0.027778in}{0.000000in}}%
\pgfusepath{stroke,fill}%
}%
\begin{pgfscope}%
\pgfsys@transformshift{0.625000in}{1.920139in}%
\pgfsys@useobject{currentmarker}{}%
\end{pgfscope}%
\end{pgfscope}%
\begin{pgfscope}%
\pgfsetbuttcap%
\pgfsetroundjoin%
\definecolor{currentfill}{rgb}{0.000000,0.000000,0.000000}%
\pgfsetfillcolor{currentfill}%
\pgfsetlinewidth{0.602250pt}%
\definecolor{currentstroke}{rgb}{0.000000,0.000000,0.000000}%
\pgfsetstrokecolor{currentstroke}%
\pgfsetdash{}{0pt}%
\pgfsys@defobject{currentmarker}{\pgfqpoint{-0.027778in}{0.000000in}}{\pgfqpoint{0.000000in}{0.000000in}}{%
\pgfpathmoveto{\pgfqpoint{0.000000in}{0.000000in}}%
\pgfpathlineto{\pgfqpoint{-0.027778in}{0.000000in}}%
\pgfusepath{stroke,fill}%
}%
\begin{pgfscope}%
\pgfsys@transformshift{0.625000in}{2.010855in}%
\pgfsys@useobject{currentmarker}{}%
\end{pgfscope}%
\end{pgfscope}%
\begin{pgfscope}%
\pgfsetbuttcap%
\pgfsetroundjoin%
\definecolor{currentfill}{rgb}{0.000000,0.000000,0.000000}%
\pgfsetfillcolor{currentfill}%
\pgfsetlinewidth{0.602250pt}%
\definecolor{currentstroke}{rgb}{0.000000,0.000000,0.000000}%
\pgfsetstrokecolor{currentstroke}%
\pgfsetdash{}{0pt}%
\pgfsys@defobject{currentmarker}{\pgfqpoint{-0.027778in}{0.000000in}}{\pgfqpoint{0.000000in}{0.000000in}}{%
\pgfpathmoveto{\pgfqpoint{0.000000in}{0.000000in}}%
\pgfpathlineto{\pgfqpoint{-0.027778in}{0.000000in}}%
\pgfusepath{stroke,fill}%
}%
\begin{pgfscope}%
\pgfsys@transformshift{0.625000in}{2.087555in}%
\pgfsys@useobject{currentmarker}{}%
\end{pgfscope}%
\end{pgfscope}%
\begin{pgfscope}%
\pgfsetbuttcap%
\pgfsetroundjoin%
\definecolor{currentfill}{rgb}{0.000000,0.000000,0.000000}%
\pgfsetfillcolor{currentfill}%
\pgfsetlinewidth{0.602250pt}%
\definecolor{currentstroke}{rgb}{0.000000,0.000000,0.000000}%
\pgfsetstrokecolor{currentstroke}%
\pgfsetdash{}{0pt}%
\pgfsys@defobject{currentmarker}{\pgfqpoint{-0.027778in}{0.000000in}}{\pgfqpoint{0.000000in}{0.000000in}}{%
\pgfpathmoveto{\pgfqpoint{0.000000in}{0.000000in}}%
\pgfpathlineto{\pgfqpoint{-0.027778in}{0.000000in}}%
\pgfusepath{stroke,fill}%
}%
\begin{pgfscope}%
\pgfsys@transformshift{0.625000in}{2.153996in}%
\pgfsys@useobject{currentmarker}{}%
\end{pgfscope}%
\end{pgfscope}%
\begin{pgfscope}%
\pgfsetbuttcap%
\pgfsetroundjoin%
\definecolor{currentfill}{rgb}{0.000000,0.000000,0.000000}%
\pgfsetfillcolor{currentfill}%
\pgfsetlinewidth{0.602250pt}%
\definecolor{currentstroke}{rgb}{0.000000,0.000000,0.000000}%
\pgfsetstrokecolor{currentstroke}%
\pgfsetdash{}{0pt}%
\pgfsys@defobject{currentmarker}{\pgfqpoint{-0.027778in}{0.000000in}}{\pgfqpoint{0.000000in}{0.000000in}}{%
\pgfpathmoveto{\pgfqpoint{0.000000in}{0.000000in}}%
\pgfpathlineto{\pgfqpoint{-0.027778in}{0.000000in}}%
\pgfusepath{stroke,fill}%
}%
\begin{pgfscope}%
\pgfsys@transformshift{0.625000in}{2.212601in}%
\pgfsys@useobject{currentmarker}{}%
\end{pgfscope}%
\end{pgfscope}%
\begin{pgfscope}%
\pgfsetbuttcap%
\pgfsetroundjoin%
\definecolor{currentfill}{rgb}{0.000000,0.000000,0.000000}%
\pgfsetfillcolor{currentfill}%
\pgfsetlinewidth{0.602250pt}%
\definecolor{currentstroke}{rgb}{0.000000,0.000000,0.000000}%
\pgfsetstrokecolor{currentstroke}%
\pgfsetdash{}{0pt}%
\pgfsys@defobject{currentmarker}{\pgfqpoint{-0.027778in}{0.000000in}}{\pgfqpoint{0.000000in}{0.000000in}}{%
\pgfpathmoveto{\pgfqpoint{0.000000in}{0.000000in}}%
\pgfpathlineto{\pgfqpoint{-0.027778in}{0.000000in}}%
\pgfusepath{stroke,fill}%
}%
\begin{pgfscope}%
\pgfsys@transformshift{0.625000in}{2.609910in}%
\pgfsys@useobject{currentmarker}{}%
\end{pgfscope}%
\end{pgfscope}%
\begin{pgfscope}%
\pgfsetbuttcap%
\pgfsetroundjoin%
\definecolor{currentfill}{rgb}{0.000000,0.000000,0.000000}%
\pgfsetfillcolor{currentfill}%
\pgfsetlinewidth{0.602250pt}%
\definecolor{currentstroke}{rgb}{0.000000,0.000000,0.000000}%
\pgfsetstrokecolor{currentstroke}%
\pgfsetdash{}{0pt}%
\pgfsys@defobject{currentmarker}{\pgfqpoint{-0.027778in}{0.000000in}}{\pgfqpoint{0.000000in}{0.000000in}}{%
\pgfpathmoveto{\pgfqpoint{0.000000in}{0.000000in}}%
\pgfpathlineto{\pgfqpoint{-0.027778in}{0.000000in}}%
\pgfusepath{stroke,fill}%
}%
\begin{pgfscope}%
\pgfsys@transformshift{0.625000in}{2.811655in}%
\pgfsys@useobject{currentmarker}{}%
\end{pgfscope}%
\end{pgfscope}%
\begin{pgfscope}%
\pgfsetbuttcap%
\pgfsetroundjoin%
\definecolor{currentfill}{rgb}{0.000000,0.000000,0.000000}%
\pgfsetfillcolor{currentfill}%
\pgfsetlinewidth{0.602250pt}%
\definecolor{currentstroke}{rgb}{0.000000,0.000000,0.000000}%
\pgfsetstrokecolor{currentstroke}%
\pgfsetdash{}{0pt}%
\pgfsys@defobject{currentmarker}{\pgfqpoint{-0.027778in}{0.000000in}}{\pgfqpoint{0.000000in}{0.000000in}}{%
\pgfpathmoveto{\pgfqpoint{0.000000in}{0.000000in}}%
\pgfpathlineto{\pgfqpoint{-0.027778in}{0.000000in}}%
\pgfusepath{stroke,fill}%
}%
\begin{pgfscope}%
\pgfsys@transformshift{0.625000in}{2.954795in}%
\pgfsys@useobject{currentmarker}{}%
\end{pgfscope}%
\end{pgfscope}%
\begin{pgfscope}%
\pgfsetbuttcap%
\pgfsetroundjoin%
\definecolor{currentfill}{rgb}{0.000000,0.000000,0.000000}%
\pgfsetfillcolor{currentfill}%
\pgfsetlinewidth{0.602250pt}%
\definecolor{currentstroke}{rgb}{0.000000,0.000000,0.000000}%
\pgfsetstrokecolor{currentstroke}%
\pgfsetdash{}{0pt}%
\pgfsys@defobject{currentmarker}{\pgfqpoint{-0.027778in}{0.000000in}}{\pgfqpoint{0.000000in}{0.000000in}}{%
\pgfpathmoveto{\pgfqpoint{0.000000in}{0.000000in}}%
\pgfpathlineto{\pgfqpoint{-0.027778in}{0.000000in}}%
\pgfusepath{stroke,fill}%
}%
\begin{pgfscope}%
\pgfsys@transformshift{0.625000in}{3.065824in}%
\pgfsys@useobject{currentmarker}{}%
\end{pgfscope}%
\end{pgfscope}%
\begin{pgfscope}%
\pgfsetbuttcap%
\pgfsetroundjoin%
\definecolor{currentfill}{rgb}{0.000000,0.000000,0.000000}%
\pgfsetfillcolor{currentfill}%
\pgfsetlinewidth{0.602250pt}%
\definecolor{currentstroke}{rgb}{0.000000,0.000000,0.000000}%
\pgfsetstrokecolor{currentstroke}%
\pgfsetdash{}{0pt}%
\pgfsys@defobject{currentmarker}{\pgfqpoint{-0.027778in}{0.000000in}}{\pgfqpoint{0.000000in}{0.000000in}}{%
\pgfpathmoveto{\pgfqpoint{0.000000in}{0.000000in}}%
\pgfpathlineto{\pgfqpoint{-0.027778in}{0.000000in}}%
\pgfusepath{stroke,fill}%
}%
\begin{pgfscope}%
\pgfsys@transformshift{0.625000in}{3.156540in}%
\pgfsys@useobject{currentmarker}{}%
\end{pgfscope}%
\end{pgfscope}%
\begin{pgfscope}%
\pgfsetbuttcap%
\pgfsetroundjoin%
\definecolor{currentfill}{rgb}{0.000000,0.000000,0.000000}%
\pgfsetfillcolor{currentfill}%
\pgfsetlinewidth{0.602250pt}%
\definecolor{currentstroke}{rgb}{0.000000,0.000000,0.000000}%
\pgfsetstrokecolor{currentstroke}%
\pgfsetdash{}{0pt}%
\pgfsys@defobject{currentmarker}{\pgfqpoint{-0.027778in}{0.000000in}}{\pgfqpoint{0.000000in}{0.000000in}}{%
\pgfpathmoveto{\pgfqpoint{0.000000in}{0.000000in}}%
\pgfpathlineto{\pgfqpoint{-0.027778in}{0.000000in}}%
\pgfusepath{stroke,fill}%
}%
\begin{pgfscope}%
\pgfsys@transformshift{0.625000in}{3.233240in}%
\pgfsys@useobject{currentmarker}{}%
\end{pgfscope}%
\end{pgfscope}%
\begin{pgfscope}%
\pgfsetbuttcap%
\pgfsetroundjoin%
\definecolor{currentfill}{rgb}{0.000000,0.000000,0.000000}%
\pgfsetfillcolor{currentfill}%
\pgfsetlinewidth{0.602250pt}%
\definecolor{currentstroke}{rgb}{0.000000,0.000000,0.000000}%
\pgfsetstrokecolor{currentstroke}%
\pgfsetdash{}{0pt}%
\pgfsys@defobject{currentmarker}{\pgfqpoint{-0.027778in}{0.000000in}}{\pgfqpoint{0.000000in}{0.000000in}}{%
\pgfpathmoveto{\pgfqpoint{0.000000in}{0.000000in}}%
\pgfpathlineto{\pgfqpoint{-0.027778in}{0.000000in}}%
\pgfusepath{stroke,fill}%
}%
\begin{pgfscope}%
\pgfsys@transformshift{0.625000in}{3.299681in}%
\pgfsys@useobject{currentmarker}{}%
\end{pgfscope}%
\end{pgfscope}%
\begin{pgfscope}%
\pgfsetbuttcap%
\pgfsetroundjoin%
\definecolor{currentfill}{rgb}{0.000000,0.000000,0.000000}%
\pgfsetfillcolor{currentfill}%
\pgfsetlinewidth{0.602250pt}%
\definecolor{currentstroke}{rgb}{0.000000,0.000000,0.000000}%
\pgfsetstrokecolor{currentstroke}%
\pgfsetdash{}{0pt}%
\pgfsys@defobject{currentmarker}{\pgfqpoint{-0.027778in}{0.000000in}}{\pgfqpoint{0.000000in}{0.000000in}}{%
\pgfpathmoveto{\pgfqpoint{0.000000in}{0.000000in}}%
\pgfpathlineto{\pgfqpoint{-0.027778in}{0.000000in}}%
\pgfusepath{stroke,fill}%
}%
\begin{pgfscope}%
\pgfsys@transformshift{0.625000in}{3.358285in}%
\pgfsys@useobject{currentmarker}{}%
\end{pgfscope}%
\end{pgfscope}%
\begin{pgfscope}%
\definecolor{textcolor}{rgb}{0.000000,0.000000,0.000000}%
\pgfsetstrokecolor{textcolor}%
\pgfsetfillcolor{textcolor}%
\pgftext[x=0.184220in,y=1.980000in,,bottom,rotate=90.000000]{\color{textcolor}\sffamily\fontsize{10.000000}{12.000000}\selectfont QMC Error}%
\end{pgfscope}%
\begin{pgfscope}%
\pgfpathrectangle{\pgfqpoint{0.625000in}{0.440000in}}{\pgfqpoint{3.875000in}{3.080000in}}%
\pgfusepath{clip}%
\pgfsetbuttcap%
\pgfsetroundjoin%
\definecolor{currentfill}{rgb}{0.000000,0.000000,0.000000}%
\pgfsetfillcolor{currentfill}%
\pgfsetlinewidth{1.003750pt}%
\definecolor{currentstroke}{rgb}{0.000000,0.000000,0.000000}%
\pgfsetstrokecolor{currentstroke}%
\pgfsetdash{}{0pt}%
\pgfsys@defobject{currentmarker}{\pgfqpoint{-0.041667in}{-0.041667in}}{\pgfqpoint{0.041667in}{0.041667in}}{%
\pgfpathmoveto{\pgfqpoint{0.000000in}{-0.041667in}}%
\pgfpathcurveto{\pgfqpoint{0.011050in}{-0.041667in}}{\pgfqpoint{0.021649in}{-0.037276in}}{\pgfqpoint{0.029463in}{-0.029463in}}%
\pgfpathcurveto{\pgfqpoint{0.037276in}{-0.021649in}}{\pgfqpoint{0.041667in}{-0.011050in}}{\pgfqpoint{0.041667in}{0.000000in}}%
\pgfpathcurveto{\pgfqpoint{0.041667in}{0.011050in}}{\pgfqpoint{0.037276in}{0.021649in}}{\pgfqpoint{0.029463in}{0.029463in}}%
\pgfpathcurveto{\pgfqpoint{0.021649in}{0.037276in}}{\pgfqpoint{0.011050in}{0.041667in}}{\pgfqpoint{0.000000in}{0.041667in}}%
\pgfpathcurveto{\pgfqpoint{-0.011050in}{0.041667in}}{\pgfqpoint{-0.021649in}{0.037276in}}{\pgfqpoint{-0.029463in}{0.029463in}}%
\pgfpathcurveto{\pgfqpoint{-0.037276in}{0.021649in}}{\pgfqpoint{-0.041667in}{0.011050in}}{\pgfqpoint{-0.041667in}{0.000000in}}%
\pgfpathcurveto{\pgfqpoint{-0.041667in}{-0.011050in}}{\pgfqpoint{-0.037276in}{-0.021649in}}{\pgfqpoint{-0.029463in}{-0.029463in}}%
\pgfpathcurveto{\pgfqpoint{-0.021649in}{-0.037276in}}{\pgfqpoint{-0.011050in}{-0.041667in}}{\pgfqpoint{0.000000in}{-0.041667in}}%
\pgfpathclose%
\pgfusepath{stroke,fill}%
}%
\begin{pgfscope}%
\pgfsys@transformshift{0.801136in}{3.380000in}%
\pgfsys@useobject{currentmarker}{}%
\end{pgfscope}%
\begin{pgfscope}%
\pgfsys@transformshift{1.121384in}{3.023848in}%
\pgfsys@useobject{currentmarker}{}%
\end{pgfscope}%
\begin{pgfscope}%
\pgfsys@transformshift{1.441632in}{2.778700in}%
\pgfsys@useobject{currentmarker}{}%
\end{pgfscope}%
\begin{pgfscope}%
\pgfsys@transformshift{1.761880in}{2.429782in}%
\pgfsys@useobject{currentmarker}{}%
\end{pgfscope}%
\begin{pgfscope}%
\pgfsys@transformshift{2.082128in}{2.174668in}%
\pgfsys@useobject{currentmarker}{}%
\end{pgfscope}%
\begin{pgfscope}%
\pgfsys@transformshift{2.402376in}{1.808912in}%
\pgfsys@useobject{currentmarker}{}%
\end{pgfscope}%
\begin{pgfscope}%
\pgfsys@transformshift{2.722624in}{1.244509in}%
\pgfsys@useobject{currentmarker}{}%
\end{pgfscope}%
\begin{pgfscope}%
\pgfsys@transformshift{3.042872in}{1.418469in}%
\pgfsys@useobject{currentmarker}{}%
\end{pgfscope}%
\begin{pgfscope}%
\pgfsys@transformshift{3.363120in}{1.249850in}%
\pgfsys@useobject{currentmarker}{}%
\end{pgfscope}%
\begin{pgfscope}%
\pgfsys@transformshift{3.683368in}{0.951714in}%
\pgfsys@useobject{currentmarker}{}%
\end{pgfscope}%
\begin{pgfscope}%
\pgfsys@transformshift{4.003616in}{0.963875in}%
\pgfsys@useobject{currentmarker}{}%
\end{pgfscope}%
\begin{pgfscope}%
\pgfsys@transformshift{4.323864in}{0.949733in}%
\pgfsys@useobject{currentmarker}{}%
\end{pgfscope}%
\end{pgfscope}%
\begin{pgfscope}%
\pgfpathrectangle{\pgfqpoint{0.625000in}{0.440000in}}{\pgfqpoint{3.875000in}{3.080000in}}%
\pgfusepath{clip}%
\pgfsetbuttcap%
\pgfsetroundjoin%
\pgfsetlinewidth{1.505625pt}%
\definecolor{currentstroke}{rgb}{0.000000,0.000000,0.000000}%
\pgfsetstrokecolor{currentstroke}%
\pgfsetdash{{5.550000pt}{2.400000pt}}{0.000000pt}%
\pgfpathmoveto{\pgfqpoint{0.801136in}{3.149010in}}%
\pgfpathlineto{\pgfqpoint{4.323864in}{0.580000in}}%
\pgfusepath{stroke}%
\end{pgfscope}%
\begin{pgfscope}%
\pgfsetrectcap%
\pgfsetmiterjoin%
\pgfsetlinewidth{0.803000pt}%
\definecolor{currentstroke}{rgb}{0.000000,0.000000,0.000000}%
\pgfsetstrokecolor{currentstroke}%
\pgfsetdash{}{0pt}%
\pgfpathmoveto{\pgfqpoint{0.625000in}{0.440000in}}%
\pgfpathlineto{\pgfqpoint{0.625000in}{3.520000in}}%
\pgfusepath{stroke}%
\end{pgfscope}%
\begin{pgfscope}%
\pgfsetrectcap%
\pgfsetmiterjoin%
\pgfsetlinewidth{0.803000pt}%
\definecolor{currentstroke}{rgb}{0.000000,0.000000,0.000000}%
\pgfsetstrokecolor{currentstroke}%
\pgfsetdash{}{0pt}%
\pgfpathmoveto{\pgfqpoint{4.500000in}{0.440000in}}%
\pgfpathlineto{\pgfqpoint{4.500000in}{3.520000in}}%
\pgfusepath{stroke}%
\end{pgfscope}%
\begin{pgfscope}%
\pgfsetrectcap%
\pgfsetmiterjoin%
\pgfsetlinewidth{0.803000pt}%
\definecolor{currentstroke}{rgb}{0.000000,0.000000,0.000000}%
\pgfsetstrokecolor{currentstroke}%
\pgfsetdash{}{0pt}%
\pgfpathmoveto{\pgfqpoint{0.625000in}{0.440000in}}%
\pgfpathlineto{\pgfqpoint{4.500000in}{0.440000in}}%
\pgfusepath{stroke}%
\end{pgfscope}%
\begin{pgfscope}%
\pgfsetrectcap%
\pgfsetmiterjoin%
\pgfsetlinewidth{0.803000pt}%
\definecolor{currentstroke}{rgb}{0.000000,0.000000,0.000000}%
\pgfsetstrokecolor{currentstroke}%
\pgfsetdash{}{0pt}%
\pgfpathmoveto{\pgfqpoint{0.625000in}{3.520000in}}%
\pgfpathlineto{\pgfqpoint{4.500000in}{3.520000in}}%
\pgfusepath{stroke}%
\end{pgfscope}%
\begin{pgfscope}%
\pgfsetbuttcap%
\pgfsetmiterjoin%
\definecolor{currentfill}{rgb}{1.000000,1.000000,1.000000}%
\pgfsetfillcolor{currentfill}%
\pgfsetfillopacity{0.800000}%
\pgfsetlinewidth{1.003750pt}%
\definecolor{currentstroke}{rgb}{0.800000,0.800000,0.800000}%
\pgfsetstrokecolor{currentstroke}%
\pgfsetstrokeopacity{0.800000}%
\pgfsetdash{}{0pt}%
\pgfpathmoveto{\pgfqpoint{3.042243in}{3.155267in}}%
\pgfpathlineto{\pgfqpoint{4.402778in}{3.155267in}}%
\pgfpathquadraticcurveto{\pgfqpoint{4.430556in}{3.155267in}}{\pgfqpoint{4.430556in}{3.183045in}}%
\pgfpathlineto{\pgfqpoint{4.430556in}{3.422778in}}%
\pgfpathquadraticcurveto{\pgfqpoint{4.430556in}{3.450556in}}{\pgfqpoint{4.402778in}{3.450556in}}%
\pgfpathlineto{\pgfqpoint{3.042243in}{3.450556in}}%
\pgfpathquadraticcurveto{\pgfqpoint{3.014465in}{3.450556in}}{\pgfqpoint{3.014465in}{3.422778in}}%
\pgfpathlineto{\pgfqpoint{3.014465in}{3.183045in}}%
\pgfpathquadraticcurveto{\pgfqpoint{3.014465in}{3.155267in}}{\pgfqpoint{3.042243in}{3.155267in}}%
\pgfpathclose%
\pgfusepath{stroke,fill}%
\end{pgfscope}%
\begin{pgfscope}%
\pgfsetbuttcap%
\pgfsetroundjoin%
\pgfsetlinewidth{1.505625pt}%
\definecolor{currentstroke}{rgb}{0.000000,0.000000,0.000000}%
\pgfsetstrokecolor{currentstroke}%
\pgfsetdash{{5.550000pt}{2.400000pt}}{0.000000pt}%
\pgfpathmoveto{\pgfqpoint{3.070021in}{3.321543in}}%
\pgfpathlineto{\pgfqpoint{3.347799in}{3.321543in}}%
\pgfusepath{stroke}%
\end{pgfscope}%
\begin{pgfscope}%
\definecolor{textcolor}{rgb}{0.000000,0.000000,0.000000}%
\pgfsetstrokecolor{textcolor}%
\pgfsetfillcolor{textcolor}%
\pgftext[x=3.458910in,y=3.272932in,left,base]{\color{textcolor}\sffamily\fontsize{10.000000}{12.000000}\selectfont \(\displaystyle 0.0059N_{\mathrm{QMC}}^{-0.6771}\)}%
\end{pgfscope}%
\end{pgfpicture}%
\makeatother%
\endgroup%

\caption[Quasi-Monte-Carlo error, for $Q=u(\bzero)$ and $k=60$.]{Quasi-Monte-Carlo error with increasing number of Quasi-Monte-Carlo points, for $Q=u(\bzero)$ and $k=60$. \label{fig:qmcorigin60}}
\end{figure}

\begin{figure}[h]
    \centering
    %% Creator: Matplotlib, PGF backend
%%
%% To include the figure in your LaTeX document, write
%%   \input{<filename>.pgf}
%%
%% Make sure the required packages are loaded in your preamble
%%   \usepackage{pgf}
%%
%% Figures using additional raster images can only be included by \input if
%% they are in the same directory as the main LaTeX file. For loading figures
%% from other directories you can use the `import` package
%%   \usepackage{import}
%% and then include the figures with
%%   \import{<path to file>}{<filename>.pgf}
%%
%% Matplotlib used the following preamble
%%   \usepackage{fontspec}
%%   \setmainfont{DejaVuSerif.ttf}[Path=/home/owen/progs/firedrake-complex/firedrake/lib/python3.5/site-packages/matplotlib/mpl-data/fonts/ttf/]
%%   \setsansfont{DejaVuSans.ttf}[Path=/home/owen/progs/firedrake-complex/firedrake/lib/python3.5/site-packages/matplotlib/mpl-data/fonts/ttf/]
%%   \setmonofont{DejaVuSansMono.ttf}[Path=/home/owen/progs/firedrake-complex/firedrake/lib/python3.5/site-packages/matplotlib/mpl-data/fonts/ttf/]
%%
\begingroup%
\makeatletter%
\begin{pgfpicture}%
\pgfpathrectangle{\pgfpointorigin}{\pgfqpoint{5.000000in}{4.000000in}}%
\pgfusepath{use as bounding box, clip}%
\begin{pgfscope}%
\pgfsetbuttcap%
\pgfsetmiterjoin%
\definecolor{currentfill}{rgb}{1.000000,1.000000,1.000000}%
\pgfsetfillcolor{currentfill}%
\pgfsetlinewidth{0.000000pt}%
\definecolor{currentstroke}{rgb}{1.000000,1.000000,1.000000}%
\pgfsetstrokecolor{currentstroke}%
\pgfsetdash{}{0pt}%
\pgfpathmoveto{\pgfqpoint{0.000000in}{0.000000in}}%
\pgfpathlineto{\pgfqpoint{5.000000in}{0.000000in}}%
\pgfpathlineto{\pgfqpoint{5.000000in}{4.000000in}}%
\pgfpathlineto{\pgfqpoint{0.000000in}{4.000000in}}%
\pgfpathclose%
\pgfusepath{fill}%
\end{pgfscope}%
\begin{pgfscope}%
\pgfsetbuttcap%
\pgfsetmiterjoin%
\definecolor{currentfill}{rgb}{1.000000,1.000000,1.000000}%
\pgfsetfillcolor{currentfill}%
\pgfsetlinewidth{0.000000pt}%
\definecolor{currentstroke}{rgb}{0.000000,0.000000,0.000000}%
\pgfsetstrokecolor{currentstroke}%
\pgfsetstrokeopacity{0.000000}%
\pgfsetdash{}{0pt}%
\pgfpathmoveto{\pgfqpoint{0.625000in}{0.440000in}}%
\pgfpathlineto{\pgfqpoint{4.500000in}{0.440000in}}%
\pgfpathlineto{\pgfqpoint{4.500000in}{3.520000in}}%
\pgfpathlineto{\pgfqpoint{0.625000in}{3.520000in}}%
\pgfpathclose%
\pgfusepath{fill}%
\end{pgfscope}%
\begin{pgfscope}%
\pgfsetbuttcap%
\pgfsetroundjoin%
\definecolor{currentfill}{rgb}{0.000000,0.000000,0.000000}%
\pgfsetfillcolor{currentfill}%
\pgfsetlinewidth{0.803000pt}%
\definecolor{currentstroke}{rgb}{0.000000,0.000000,0.000000}%
\pgfsetstrokecolor{currentstroke}%
\pgfsetdash{}{0pt}%
\pgfsys@defobject{currentmarker}{\pgfqpoint{0.000000in}{-0.048611in}}{\pgfqpoint{0.000000in}{0.000000in}}{%
\pgfpathmoveto{\pgfqpoint{0.000000in}{0.000000in}}%
\pgfpathlineto{\pgfqpoint{0.000000in}{-0.048611in}}%
\pgfusepath{stroke,fill}%
}%
\begin{pgfscope}%
\pgfsys@transformshift{0.801136in}{0.440000in}%
\pgfsys@useobject{currentmarker}{}%
\end{pgfscope}%
\end{pgfscope}%
\begin{pgfscope}%
\definecolor{textcolor}{rgb}{0.000000,0.000000,0.000000}%
\pgfsetstrokecolor{textcolor}%
\pgfsetfillcolor{textcolor}%
\pgftext[x=0.801136in,y=0.342778in,,top]{\color{textcolor}\sffamily\fontsize{10.000000}{12.000000}\selectfont \(\displaystyle 10^{0}\)}%
\end{pgfscope}%
\begin{pgfscope}%
\pgfsetbuttcap%
\pgfsetroundjoin%
\definecolor{currentfill}{rgb}{0.000000,0.000000,0.000000}%
\pgfsetfillcolor{currentfill}%
\pgfsetlinewidth{0.803000pt}%
\definecolor{currentstroke}{rgb}{0.000000,0.000000,0.000000}%
\pgfsetstrokecolor{currentstroke}%
\pgfsetdash{}{0pt}%
\pgfsys@defobject{currentmarker}{\pgfqpoint{0.000000in}{-0.048611in}}{\pgfqpoint{0.000000in}{0.000000in}}{%
\pgfpathmoveto{\pgfqpoint{0.000000in}{0.000000in}}%
\pgfpathlineto{\pgfqpoint{0.000000in}{-0.048611in}}%
\pgfusepath{stroke,fill}%
}%
\begin{pgfscope}%
\pgfsys@transformshift{1.864977in}{0.440000in}%
\pgfsys@useobject{currentmarker}{}%
\end{pgfscope}%
\end{pgfscope}%
\begin{pgfscope}%
\definecolor{textcolor}{rgb}{0.000000,0.000000,0.000000}%
\pgfsetstrokecolor{textcolor}%
\pgfsetfillcolor{textcolor}%
\pgftext[x=1.864977in,y=0.342778in,,top]{\color{textcolor}\sffamily\fontsize{10.000000}{12.000000}\selectfont \(\displaystyle 10^{1}\)}%
\end{pgfscope}%
\begin{pgfscope}%
\pgfsetbuttcap%
\pgfsetroundjoin%
\definecolor{currentfill}{rgb}{0.000000,0.000000,0.000000}%
\pgfsetfillcolor{currentfill}%
\pgfsetlinewidth{0.803000pt}%
\definecolor{currentstroke}{rgb}{0.000000,0.000000,0.000000}%
\pgfsetstrokecolor{currentstroke}%
\pgfsetdash{}{0pt}%
\pgfsys@defobject{currentmarker}{\pgfqpoint{0.000000in}{-0.048611in}}{\pgfqpoint{0.000000in}{0.000000in}}{%
\pgfpathmoveto{\pgfqpoint{0.000000in}{0.000000in}}%
\pgfpathlineto{\pgfqpoint{0.000000in}{-0.048611in}}%
\pgfusepath{stroke,fill}%
}%
\begin{pgfscope}%
\pgfsys@transformshift{2.928818in}{0.440000in}%
\pgfsys@useobject{currentmarker}{}%
\end{pgfscope}%
\end{pgfscope}%
\begin{pgfscope}%
\definecolor{textcolor}{rgb}{0.000000,0.000000,0.000000}%
\pgfsetstrokecolor{textcolor}%
\pgfsetfillcolor{textcolor}%
\pgftext[x=2.928818in,y=0.342778in,,top]{\color{textcolor}\sffamily\fontsize{10.000000}{12.000000}\selectfont \(\displaystyle 10^{2}\)}%
\end{pgfscope}%
\begin{pgfscope}%
\pgfsetbuttcap%
\pgfsetroundjoin%
\definecolor{currentfill}{rgb}{0.000000,0.000000,0.000000}%
\pgfsetfillcolor{currentfill}%
\pgfsetlinewidth{0.803000pt}%
\definecolor{currentstroke}{rgb}{0.000000,0.000000,0.000000}%
\pgfsetstrokecolor{currentstroke}%
\pgfsetdash{}{0pt}%
\pgfsys@defobject{currentmarker}{\pgfqpoint{0.000000in}{-0.048611in}}{\pgfqpoint{0.000000in}{0.000000in}}{%
\pgfpathmoveto{\pgfqpoint{0.000000in}{0.000000in}}%
\pgfpathlineto{\pgfqpoint{0.000000in}{-0.048611in}}%
\pgfusepath{stroke,fill}%
}%
\begin{pgfscope}%
\pgfsys@transformshift{3.992658in}{0.440000in}%
\pgfsys@useobject{currentmarker}{}%
\end{pgfscope}%
\end{pgfscope}%
\begin{pgfscope}%
\definecolor{textcolor}{rgb}{0.000000,0.000000,0.000000}%
\pgfsetstrokecolor{textcolor}%
\pgfsetfillcolor{textcolor}%
\pgftext[x=3.992658in,y=0.342778in,,top]{\color{textcolor}\sffamily\fontsize{10.000000}{12.000000}\selectfont \(\displaystyle 10^{3}\)}%
\end{pgfscope}%
\begin{pgfscope}%
\pgfsetbuttcap%
\pgfsetroundjoin%
\definecolor{currentfill}{rgb}{0.000000,0.000000,0.000000}%
\pgfsetfillcolor{currentfill}%
\pgfsetlinewidth{0.602250pt}%
\definecolor{currentstroke}{rgb}{0.000000,0.000000,0.000000}%
\pgfsetstrokecolor{currentstroke}%
\pgfsetdash{}{0pt}%
\pgfsys@defobject{currentmarker}{\pgfqpoint{0.000000in}{-0.027778in}}{\pgfqpoint{0.000000in}{0.000000in}}{%
\pgfpathmoveto{\pgfqpoint{0.000000in}{0.000000in}}%
\pgfpathlineto{\pgfqpoint{0.000000in}{-0.027778in}}%
\pgfusepath{stroke,fill}%
}%
\begin{pgfscope}%
\pgfsys@transformshift{0.636345in}{0.440000in}%
\pgfsys@useobject{currentmarker}{}%
\end{pgfscope}%
\end{pgfscope}%
\begin{pgfscope}%
\pgfsetbuttcap%
\pgfsetroundjoin%
\definecolor{currentfill}{rgb}{0.000000,0.000000,0.000000}%
\pgfsetfillcolor{currentfill}%
\pgfsetlinewidth{0.602250pt}%
\definecolor{currentstroke}{rgb}{0.000000,0.000000,0.000000}%
\pgfsetstrokecolor{currentstroke}%
\pgfsetdash{}{0pt}%
\pgfsys@defobject{currentmarker}{\pgfqpoint{0.000000in}{-0.027778in}}{\pgfqpoint{0.000000in}{0.000000in}}{%
\pgfpathmoveto{\pgfqpoint{0.000000in}{0.000000in}}%
\pgfpathlineto{\pgfqpoint{0.000000in}{-0.027778in}}%
\pgfusepath{stroke,fill}%
}%
\begin{pgfscope}%
\pgfsys@transformshift{0.698040in}{0.440000in}%
\pgfsys@useobject{currentmarker}{}%
\end{pgfscope}%
\end{pgfscope}%
\begin{pgfscope}%
\pgfsetbuttcap%
\pgfsetroundjoin%
\definecolor{currentfill}{rgb}{0.000000,0.000000,0.000000}%
\pgfsetfillcolor{currentfill}%
\pgfsetlinewidth{0.602250pt}%
\definecolor{currentstroke}{rgb}{0.000000,0.000000,0.000000}%
\pgfsetstrokecolor{currentstroke}%
\pgfsetdash{}{0pt}%
\pgfsys@defobject{currentmarker}{\pgfqpoint{0.000000in}{-0.027778in}}{\pgfqpoint{0.000000in}{0.000000in}}{%
\pgfpathmoveto{\pgfqpoint{0.000000in}{0.000000in}}%
\pgfpathlineto{\pgfqpoint{0.000000in}{-0.027778in}}%
\pgfusepath{stroke,fill}%
}%
\begin{pgfscope}%
\pgfsys@transformshift{0.752458in}{0.440000in}%
\pgfsys@useobject{currentmarker}{}%
\end{pgfscope}%
\end{pgfscope}%
\begin{pgfscope}%
\pgfsetbuttcap%
\pgfsetroundjoin%
\definecolor{currentfill}{rgb}{0.000000,0.000000,0.000000}%
\pgfsetfillcolor{currentfill}%
\pgfsetlinewidth{0.602250pt}%
\definecolor{currentstroke}{rgb}{0.000000,0.000000,0.000000}%
\pgfsetstrokecolor{currentstroke}%
\pgfsetdash{}{0pt}%
\pgfsys@defobject{currentmarker}{\pgfqpoint{0.000000in}{-0.027778in}}{\pgfqpoint{0.000000in}{0.000000in}}{%
\pgfpathmoveto{\pgfqpoint{0.000000in}{0.000000in}}%
\pgfpathlineto{\pgfqpoint{0.000000in}{-0.027778in}}%
\pgfusepath{stroke,fill}%
}%
\begin{pgfscope}%
\pgfsys@transformshift{1.121384in}{0.440000in}%
\pgfsys@useobject{currentmarker}{}%
\end{pgfscope}%
\end{pgfscope}%
\begin{pgfscope}%
\pgfsetbuttcap%
\pgfsetroundjoin%
\definecolor{currentfill}{rgb}{0.000000,0.000000,0.000000}%
\pgfsetfillcolor{currentfill}%
\pgfsetlinewidth{0.602250pt}%
\definecolor{currentstroke}{rgb}{0.000000,0.000000,0.000000}%
\pgfsetstrokecolor{currentstroke}%
\pgfsetdash{}{0pt}%
\pgfsys@defobject{currentmarker}{\pgfqpoint{0.000000in}{-0.027778in}}{\pgfqpoint{0.000000in}{0.000000in}}{%
\pgfpathmoveto{\pgfqpoint{0.000000in}{0.000000in}}%
\pgfpathlineto{\pgfqpoint{0.000000in}{-0.027778in}}%
\pgfusepath{stroke,fill}%
}%
\begin{pgfscope}%
\pgfsys@transformshift{1.308717in}{0.440000in}%
\pgfsys@useobject{currentmarker}{}%
\end{pgfscope}%
\end{pgfscope}%
\begin{pgfscope}%
\pgfsetbuttcap%
\pgfsetroundjoin%
\definecolor{currentfill}{rgb}{0.000000,0.000000,0.000000}%
\pgfsetfillcolor{currentfill}%
\pgfsetlinewidth{0.602250pt}%
\definecolor{currentstroke}{rgb}{0.000000,0.000000,0.000000}%
\pgfsetstrokecolor{currentstroke}%
\pgfsetdash{}{0pt}%
\pgfsys@defobject{currentmarker}{\pgfqpoint{0.000000in}{-0.027778in}}{\pgfqpoint{0.000000in}{0.000000in}}{%
\pgfpathmoveto{\pgfqpoint{0.000000in}{0.000000in}}%
\pgfpathlineto{\pgfqpoint{0.000000in}{-0.027778in}}%
\pgfusepath{stroke,fill}%
}%
\begin{pgfscope}%
\pgfsys@transformshift{1.441632in}{0.440000in}%
\pgfsys@useobject{currentmarker}{}%
\end{pgfscope}%
\end{pgfscope}%
\begin{pgfscope}%
\pgfsetbuttcap%
\pgfsetroundjoin%
\definecolor{currentfill}{rgb}{0.000000,0.000000,0.000000}%
\pgfsetfillcolor{currentfill}%
\pgfsetlinewidth{0.602250pt}%
\definecolor{currentstroke}{rgb}{0.000000,0.000000,0.000000}%
\pgfsetstrokecolor{currentstroke}%
\pgfsetdash{}{0pt}%
\pgfsys@defobject{currentmarker}{\pgfqpoint{0.000000in}{-0.027778in}}{\pgfqpoint{0.000000in}{0.000000in}}{%
\pgfpathmoveto{\pgfqpoint{0.000000in}{0.000000in}}%
\pgfpathlineto{\pgfqpoint{0.000000in}{-0.027778in}}%
\pgfusepath{stroke,fill}%
}%
\begin{pgfscope}%
\pgfsys@transformshift{1.544729in}{0.440000in}%
\pgfsys@useobject{currentmarker}{}%
\end{pgfscope}%
\end{pgfscope}%
\begin{pgfscope}%
\pgfsetbuttcap%
\pgfsetroundjoin%
\definecolor{currentfill}{rgb}{0.000000,0.000000,0.000000}%
\pgfsetfillcolor{currentfill}%
\pgfsetlinewidth{0.602250pt}%
\definecolor{currentstroke}{rgb}{0.000000,0.000000,0.000000}%
\pgfsetstrokecolor{currentstroke}%
\pgfsetdash{}{0pt}%
\pgfsys@defobject{currentmarker}{\pgfqpoint{0.000000in}{-0.027778in}}{\pgfqpoint{0.000000in}{0.000000in}}{%
\pgfpathmoveto{\pgfqpoint{0.000000in}{0.000000in}}%
\pgfpathlineto{\pgfqpoint{0.000000in}{-0.027778in}}%
\pgfusepath{stroke,fill}%
}%
\begin{pgfscope}%
\pgfsys@transformshift{1.628965in}{0.440000in}%
\pgfsys@useobject{currentmarker}{}%
\end{pgfscope}%
\end{pgfscope}%
\begin{pgfscope}%
\pgfsetbuttcap%
\pgfsetroundjoin%
\definecolor{currentfill}{rgb}{0.000000,0.000000,0.000000}%
\pgfsetfillcolor{currentfill}%
\pgfsetlinewidth{0.602250pt}%
\definecolor{currentstroke}{rgb}{0.000000,0.000000,0.000000}%
\pgfsetstrokecolor{currentstroke}%
\pgfsetdash{}{0pt}%
\pgfsys@defobject{currentmarker}{\pgfqpoint{0.000000in}{-0.027778in}}{\pgfqpoint{0.000000in}{0.000000in}}{%
\pgfpathmoveto{\pgfqpoint{0.000000in}{0.000000in}}%
\pgfpathlineto{\pgfqpoint{0.000000in}{-0.027778in}}%
\pgfusepath{stroke,fill}%
}%
\begin{pgfscope}%
\pgfsys@transformshift{1.700186in}{0.440000in}%
\pgfsys@useobject{currentmarker}{}%
\end{pgfscope}%
\end{pgfscope}%
\begin{pgfscope}%
\pgfsetbuttcap%
\pgfsetroundjoin%
\definecolor{currentfill}{rgb}{0.000000,0.000000,0.000000}%
\pgfsetfillcolor{currentfill}%
\pgfsetlinewidth{0.602250pt}%
\definecolor{currentstroke}{rgb}{0.000000,0.000000,0.000000}%
\pgfsetstrokecolor{currentstroke}%
\pgfsetdash{}{0pt}%
\pgfsys@defobject{currentmarker}{\pgfqpoint{0.000000in}{-0.027778in}}{\pgfqpoint{0.000000in}{0.000000in}}{%
\pgfpathmoveto{\pgfqpoint{0.000000in}{0.000000in}}%
\pgfpathlineto{\pgfqpoint{0.000000in}{-0.027778in}}%
\pgfusepath{stroke,fill}%
}%
\begin{pgfscope}%
\pgfsys@transformshift{1.761880in}{0.440000in}%
\pgfsys@useobject{currentmarker}{}%
\end{pgfscope}%
\end{pgfscope}%
\begin{pgfscope}%
\pgfsetbuttcap%
\pgfsetroundjoin%
\definecolor{currentfill}{rgb}{0.000000,0.000000,0.000000}%
\pgfsetfillcolor{currentfill}%
\pgfsetlinewidth{0.602250pt}%
\definecolor{currentstroke}{rgb}{0.000000,0.000000,0.000000}%
\pgfsetstrokecolor{currentstroke}%
\pgfsetdash{}{0pt}%
\pgfsys@defobject{currentmarker}{\pgfqpoint{0.000000in}{-0.027778in}}{\pgfqpoint{0.000000in}{0.000000in}}{%
\pgfpathmoveto{\pgfqpoint{0.000000in}{0.000000in}}%
\pgfpathlineto{\pgfqpoint{0.000000in}{-0.027778in}}%
\pgfusepath{stroke,fill}%
}%
\begin{pgfscope}%
\pgfsys@transformshift{1.816298in}{0.440000in}%
\pgfsys@useobject{currentmarker}{}%
\end{pgfscope}%
\end{pgfscope}%
\begin{pgfscope}%
\pgfsetbuttcap%
\pgfsetroundjoin%
\definecolor{currentfill}{rgb}{0.000000,0.000000,0.000000}%
\pgfsetfillcolor{currentfill}%
\pgfsetlinewidth{0.602250pt}%
\definecolor{currentstroke}{rgb}{0.000000,0.000000,0.000000}%
\pgfsetstrokecolor{currentstroke}%
\pgfsetdash{}{0pt}%
\pgfsys@defobject{currentmarker}{\pgfqpoint{0.000000in}{-0.027778in}}{\pgfqpoint{0.000000in}{0.000000in}}{%
\pgfpathmoveto{\pgfqpoint{0.000000in}{0.000000in}}%
\pgfpathlineto{\pgfqpoint{0.000000in}{-0.027778in}}%
\pgfusepath{stroke,fill}%
}%
\begin{pgfscope}%
\pgfsys@transformshift{2.185225in}{0.440000in}%
\pgfsys@useobject{currentmarker}{}%
\end{pgfscope}%
\end{pgfscope}%
\begin{pgfscope}%
\pgfsetbuttcap%
\pgfsetroundjoin%
\definecolor{currentfill}{rgb}{0.000000,0.000000,0.000000}%
\pgfsetfillcolor{currentfill}%
\pgfsetlinewidth{0.602250pt}%
\definecolor{currentstroke}{rgb}{0.000000,0.000000,0.000000}%
\pgfsetstrokecolor{currentstroke}%
\pgfsetdash{}{0pt}%
\pgfsys@defobject{currentmarker}{\pgfqpoint{0.000000in}{-0.027778in}}{\pgfqpoint{0.000000in}{0.000000in}}{%
\pgfpathmoveto{\pgfqpoint{0.000000in}{0.000000in}}%
\pgfpathlineto{\pgfqpoint{0.000000in}{-0.027778in}}%
\pgfusepath{stroke,fill}%
}%
\begin{pgfscope}%
\pgfsys@transformshift{2.372558in}{0.440000in}%
\pgfsys@useobject{currentmarker}{}%
\end{pgfscope}%
\end{pgfscope}%
\begin{pgfscope}%
\pgfsetbuttcap%
\pgfsetroundjoin%
\definecolor{currentfill}{rgb}{0.000000,0.000000,0.000000}%
\pgfsetfillcolor{currentfill}%
\pgfsetlinewidth{0.602250pt}%
\definecolor{currentstroke}{rgb}{0.000000,0.000000,0.000000}%
\pgfsetstrokecolor{currentstroke}%
\pgfsetdash{}{0pt}%
\pgfsys@defobject{currentmarker}{\pgfqpoint{0.000000in}{-0.027778in}}{\pgfqpoint{0.000000in}{0.000000in}}{%
\pgfpathmoveto{\pgfqpoint{0.000000in}{0.000000in}}%
\pgfpathlineto{\pgfqpoint{0.000000in}{-0.027778in}}%
\pgfusepath{stroke,fill}%
}%
\begin{pgfscope}%
\pgfsys@transformshift{2.505473in}{0.440000in}%
\pgfsys@useobject{currentmarker}{}%
\end{pgfscope}%
\end{pgfscope}%
\begin{pgfscope}%
\pgfsetbuttcap%
\pgfsetroundjoin%
\definecolor{currentfill}{rgb}{0.000000,0.000000,0.000000}%
\pgfsetfillcolor{currentfill}%
\pgfsetlinewidth{0.602250pt}%
\definecolor{currentstroke}{rgb}{0.000000,0.000000,0.000000}%
\pgfsetstrokecolor{currentstroke}%
\pgfsetdash{}{0pt}%
\pgfsys@defobject{currentmarker}{\pgfqpoint{0.000000in}{-0.027778in}}{\pgfqpoint{0.000000in}{0.000000in}}{%
\pgfpathmoveto{\pgfqpoint{0.000000in}{0.000000in}}%
\pgfpathlineto{\pgfqpoint{0.000000in}{-0.027778in}}%
\pgfusepath{stroke,fill}%
}%
\begin{pgfscope}%
\pgfsys@transformshift{2.608570in}{0.440000in}%
\pgfsys@useobject{currentmarker}{}%
\end{pgfscope}%
\end{pgfscope}%
\begin{pgfscope}%
\pgfsetbuttcap%
\pgfsetroundjoin%
\definecolor{currentfill}{rgb}{0.000000,0.000000,0.000000}%
\pgfsetfillcolor{currentfill}%
\pgfsetlinewidth{0.602250pt}%
\definecolor{currentstroke}{rgb}{0.000000,0.000000,0.000000}%
\pgfsetstrokecolor{currentstroke}%
\pgfsetdash{}{0pt}%
\pgfsys@defobject{currentmarker}{\pgfqpoint{0.000000in}{-0.027778in}}{\pgfqpoint{0.000000in}{0.000000in}}{%
\pgfpathmoveto{\pgfqpoint{0.000000in}{0.000000in}}%
\pgfpathlineto{\pgfqpoint{0.000000in}{-0.027778in}}%
\pgfusepath{stroke,fill}%
}%
\begin{pgfscope}%
\pgfsys@transformshift{2.692806in}{0.440000in}%
\pgfsys@useobject{currentmarker}{}%
\end{pgfscope}%
\end{pgfscope}%
\begin{pgfscope}%
\pgfsetbuttcap%
\pgfsetroundjoin%
\definecolor{currentfill}{rgb}{0.000000,0.000000,0.000000}%
\pgfsetfillcolor{currentfill}%
\pgfsetlinewidth{0.602250pt}%
\definecolor{currentstroke}{rgb}{0.000000,0.000000,0.000000}%
\pgfsetstrokecolor{currentstroke}%
\pgfsetdash{}{0pt}%
\pgfsys@defobject{currentmarker}{\pgfqpoint{0.000000in}{-0.027778in}}{\pgfqpoint{0.000000in}{0.000000in}}{%
\pgfpathmoveto{\pgfqpoint{0.000000in}{0.000000in}}%
\pgfpathlineto{\pgfqpoint{0.000000in}{-0.027778in}}%
\pgfusepath{stroke,fill}%
}%
\begin{pgfscope}%
\pgfsys@transformshift{2.764027in}{0.440000in}%
\pgfsys@useobject{currentmarker}{}%
\end{pgfscope}%
\end{pgfscope}%
\begin{pgfscope}%
\pgfsetbuttcap%
\pgfsetroundjoin%
\definecolor{currentfill}{rgb}{0.000000,0.000000,0.000000}%
\pgfsetfillcolor{currentfill}%
\pgfsetlinewidth{0.602250pt}%
\definecolor{currentstroke}{rgb}{0.000000,0.000000,0.000000}%
\pgfsetstrokecolor{currentstroke}%
\pgfsetdash{}{0pt}%
\pgfsys@defobject{currentmarker}{\pgfqpoint{0.000000in}{-0.027778in}}{\pgfqpoint{0.000000in}{0.000000in}}{%
\pgfpathmoveto{\pgfqpoint{0.000000in}{0.000000in}}%
\pgfpathlineto{\pgfqpoint{0.000000in}{-0.027778in}}%
\pgfusepath{stroke,fill}%
}%
\begin{pgfscope}%
\pgfsys@transformshift{2.825721in}{0.440000in}%
\pgfsys@useobject{currentmarker}{}%
\end{pgfscope}%
\end{pgfscope}%
\begin{pgfscope}%
\pgfsetbuttcap%
\pgfsetroundjoin%
\definecolor{currentfill}{rgb}{0.000000,0.000000,0.000000}%
\pgfsetfillcolor{currentfill}%
\pgfsetlinewidth{0.602250pt}%
\definecolor{currentstroke}{rgb}{0.000000,0.000000,0.000000}%
\pgfsetstrokecolor{currentstroke}%
\pgfsetdash{}{0pt}%
\pgfsys@defobject{currentmarker}{\pgfqpoint{0.000000in}{-0.027778in}}{\pgfqpoint{0.000000in}{0.000000in}}{%
\pgfpathmoveto{\pgfqpoint{0.000000in}{0.000000in}}%
\pgfpathlineto{\pgfqpoint{0.000000in}{-0.027778in}}%
\pgfusepath{stroke,fill}%
}%
\begin{pgfscope}%
\pgfsys@transformshift{2.880139in}{0.440000in}%
\pgfsys@useobject{currentmarker}{}%
\end{pgfscope}%
\end{pgfscope}%
\begin{pgfscope}%
\pgfsetbuttcap%
\pgfsetroundjoin%
\definecolor{currentfill}{rgb}{0.000000,0.000000,0.000000}%
\pgfsetfillcolor{currentfill}%
\pgfsetlinewidth{0.602250pt}%
\definecolor{currentstroke}{rgb}{0.000000,0.000000,0.000000}%
\pgfsetstrokecolor{currentstroke}%
\pgfsetdash{}{0pt}%
\pgfsys@defobject{currentmarker}{\pgfqpoint{0.000000in}{-0.027778in}}{\pgfqpoint{0.000000in}{0.000000in}}{%
\pgfpathmoveto{\pgfqpoint{0.000000in}{0.000000in}}%
\pgfpathlineto{\pgfqpoint{0.000000in}{-0.027778in}}%
\pgfusepath{stroke,fill}%
}%
\begin{pgfscope}%
\pgfsys@transformshift{3.249066in}{0.440000in}%
\pgfsys@useobject{currentmarker}{}%
\end{pgfscope}%
\end{pgfscope}%
\begin{pgfscope}%
\pgfsetbuttcap%
\pgfsetroundjoin%
\definecolor{currentfill}{rgb}{0.000000,0.000000,0.000000}%
\pgfsetfillcolor{currentfill}%
\pgfsetlinewidth{0.602250pt}%
\definecolor{currentstroke}{rgb}{0.000000,0.000000,0.000000}%
\pgfsetstrokecolor{currentstroke}%
\pgfsetdash{}{0pt}%
\pgfsys@defobject{currentmarker}{\pgfqpoint{0.000000in}{-0.027778in}}{\pgfqpoint{0.000000in}{0.000000in}}{%
\pgfpathmoveto{\pgfqpoint{0.000000in}{0.000000in}}%
\pgfpathlineto{\pgfqpoint{0.000000in}{-0.027778in}}%
\pgfusepath{stroke,fill}%
}%
\begin{pgfscope}%
\pgfsys@transformshift{3.436399in}{0.440000in}%
\pgfsys@useobject{currentmarker}{}%
\end{pgfscope}%
\end{pgfscope}%
\begin{pgfscope}%
\pgfsetbuttcap%
\pgfsetroundjoin%
\definecolor{currentfill}{rgb}{0.000000,0.000000,0.000000}%
\pgfsetfillcolor{currentfill}%
\pgfsetlinewidth{0.602250pt}%
\definecolor{currentstroke}{rgb}{0.000000,0.000000,0.000000}%
\pgfsetstrokecolor{currentstroke}%
\pgfsetdash{}{0pt}%
\pgfsys@defobject{currentmarker}{\pgfqpoint{0.000000in}{-0.027778in}}{\pgfqpoint{0.000000in}{0.000000in}}{%
\pgfpathmoveto{\pgfqpoint{0.000000in}{0.000000in}}%
\pgfpathlineto{\pgfqpoint{0.000000in}{-0.027778in}}%
\pgfusepath{stroke,fill}%
}%
\begin{pgfscope}%
\pgfsys@transformshift{3.569313in}{0.440000in}%
\pgfsys@useobject{currentmarker}{}%
\end{pgfscope}%
\end{pgfscope}%
\begin{pgfscope}%
\pgfsetbuttcap%
\pgfsetroundjoin%
\definecolor{currentfill}{rgb}{0.000000,0.000000,0.000000}%
\pgfsetfillcolor{currentfill}%
\pgfsetlinewidth{0.602250pt}%
\definecolor{currentstroke}{rgb}{0.000000,0.000000,0.000000}%
\pgfsetstrokecolor{currentstroke}%
\pgfsetdash{}{0pt}%
\pgfsys@defobject{currentmarker}{\pgfqpoint{0.000000in}{-0.027778in}}{\pgfqpoint{0.000000in}{0.000000in}}{%
\pgfpathmoveto{\pgfqpoint{0.000000in}{0.000000in}}%
\pgfpathlineto{\pgfqpoint{0.000000in}{-0.027778in}}%
\pgfusepath{stroke,fill}%
}%
\begin{pgfscope}%
\pgfsys@transformshift{3.672410in}{0.440000in}%
\pgfsys@useobject{currentmarker}{}%
\end{pgfscope}%
\end{pgfscope}%
\begin{pgfscope}%
\pgfsetbuttcap%
\pgfsetroundjoin%
\definecolor{currentfill}{rgb}{0.000000,0.000000,0.000000}%
\pgfsetfillcolor{currentfill}%
\pgfsetlinewidth{0.602250pt}%
\definecolor{currentstroke}{rgb}{0.000000,0.000000,0.000000}%
\pgfsetstrokecolor{currentstroke}%
\pgfsetdash{}{0pt}%
\pgfsys@defobject{currentmarker}{\pgfqpoint{0.000000in}{-0.027778in}}{\pgfqpoint{0.000000in}{0.000000in}}{%
\pgfpathmoveto{\pgfqpoint{0.000000in}{0.000000in}}%
\pgfpathlineto{\pgfqpoint{0.000000in}{-0.027778in}}%
\pgfusepath{stroke,fill}%
}%
\begin{pgfscope}%
\pgfsys@transformshift{3.756646in}{0.440000in}%
\pgfsys@useobject{currentmarker}{}%
\end{pgfscope}%
\end{pgfscope}%
\begin{pgfscope}%
\pgfsetbuttcap%
\pgfsetroundjoin%
\definecolor{currentfill}{rgb}{0.000000,0.000000,0.000000}%
\pgfsetfillcolor{currentfill}%
\pgfsetlinewidth{0.602250pt}%
\definecolor{currentstroke}{rgb}{0.000000,0.000000,0.000000}%
\pgfsetstrokecolor{currentstroke}%
\pgfsetdash{}{0pt}%
\pgfsys@defobject{currentmarker}{\pgfqpoint{0.000000in}{-0.027778in}}{\pgfqpoint{0.000000in}{0.000000in}}{%
\pgfpathmoveto{\pgfqpoint{0.000000in}{0.000000in}}%
\pgfpathlineto{\pgfqpoint{0.000000in}{-0.027778in}}%
\pgfusepath{stroke,fill}%
}%
\begin{pgfscope}%
\pgfsys@transformshift{3.827867in}{0.440000in}%
\pgfsys@useobject{currentmarker}{}%
\end{pgfscope}%
\end{pgfscope}%
\begin{pgfscope}%
\pgfsetbuttcap%
\pgfsetroundjoin%
\definecolor{currentfill}{rgb}{0.000000,0.000000,0.000000}%
\pgfsetfillcolor{currentfill}%
\pgfsetlinewidth{0.602250pt}%
\definecolor{currentstroke}{rgb}{0.000000,0.000000,0.000000}%
\pgfsetstrokecolor{currentstroke}%
\pgfsetdash{}{0pt}%
\pgfsys@defobject{currentmarker}{\pgfqpoint{0.000000in}{-0.027778in}}{\pgfqpoint{0.000000in}{0.000000in}}{%
\pgfpathmoveto{\pgfqpoint{0.000000in}{0.000000in}}%
\pgfpathlineto{\pgfqpoint{0.000000in}{-0.027778in}}%
\pgfusepath{stroke,fill}%
}%
\begin{pgfscope}%
\pgfsys@transformshift{3.889561in}{0.440000in}%
\pgfsys@useobject{currentmarker}{}%
\end{pgfscope}%
\end{pgfscope}%
\begin{pgfscope}%
\pgfsetbuttcap%
\pgfsetroundjoin%
\definecolor{currentfill}{rgb}{0.000000,0.000000,0.000000}%
\pgfsetfillcolor{currentfill}%
\pgfsetlinewidth{0.602250pt}%
\definecolor{currentstroke}{rgb}{0.000000,0.000000,0.000000}%
\pgfsetstrokecolor{currentstroke}%
\pgfsetdash{}{0pt}%
\pgfsys@defobject{currentmarker}{\pgfqpoint{0.000000in}{-0.027778in}}{\pgfqpoint{0.000000in}{0.000000in}}{%
\pgfpathmoveto{\pgfqpoint{0.000000in}{0.000000in}}%
\pgfpathlineto{\pgfqpoint{0.000000in}{-0.027778in}}%
\pgfusepath{stroke,fill}%
}%
\begin{pgfscope}%
\pgfsys@transformshift{3.943980in}{0.440000in}%
\pgfsys@useobject{currentmarker}{}%
\end{pgfscope}%
\end{pgfscope}%
\begin{pgfscope}%
\pgfsetbuttcap%
\pgfsetroundjoin%
\definecolor{currentfill}{rgb}{0.000000,0.000000,0.000000}%
\pgfsetfillcolor{currentfill}%
\pgfsetlinewidth{0.602250pt}%
\definecolor{currentstroke}{rgb}{0.000000,0.000000,0.000000}%
\pgfsetstrokecolor{currentstroke}%
\pgfsetdash{}{0pt}%
\pgfsys@defobject{currentmarker}{\pgfqpoint{0.000000in}{-0.027778in}}{\pgfqpoint{0.000000in}{0.000000in}}{%
\pgfpathmoveto{\pgfqpoint{0.000000in}{0.000000in}}%
\pgfpathlineto{\pgfqpoint{0.000000in}{-0.027778in}}%
\pgfusepath{stroke,fill}%
}%
\begin{pgfscope}%
\pgfsys@transformshift{4.312906in}{0.440000in}%
\pgfsys@useobject{currentmarker}{}%
\end{pgfscope}%
\end{pgfscope}%
\begin{pgfscope}%
\pgfsetbuttcap%
\pgfsetroundjoin%
\definecolor{currentfill}{rgb}{0.000000,0.000000,0.000000}%
\pgfsetfillcolor{currentfill}%
\pgfsetlinewidth{0.602250pt}%
\definecolor{currentstroke}{rgb}{0.000000,0.000000,0.000000}%
\pgfsetstrokecolor{currentstroke}%
\pgfsetdash{}{0pt}%
\pgfsys@defobject{currentmarker}{\pgfqpoint{0.000000in}{-0.027778in}}{\pgfqpoint{0.000000in}{0.000000in}}{%
\pgfpathmoveto{\pgfqpoint{0.000000in}{0.000000in}}%
\pgfpathlineto{\pgfqpoint{0.000000in}{-0.027778in}}%
\pgfusepath{stroke,fill}%
}%
\begin{pgfscope}%
\pgfsys@transformshift{4.500239in}{0.440000in}%
\pgfsys@useobject{currentmarker}{}%
\end{pgfscope}%
\end{pgfscope}%
\begin{pgfscope}%
\definecolor{textcolor}{rgb}{0.000000,0.000000,0.000000}%
\pgfsetstrokecolor{textcolor}%
\pgfsetfillcolor{textcolor}%
\pgftext[x=2.562500in,y=0.152809in,,top]{\color{textcolor}\sffamily\fontsize{10.000000}{12.000000}\selectfont \(\displaystyle N_\mathrm{QMC}\)}%
\end{pgfscope}%
\begin{pgfscope}%
\pgfsetbuttcap%
\pgfsetroundjoin%
\definecolor{currentfill}{rgb}{0.000000,0.000000,0.000000}%
\pgfsetfillcolor{currentfill}%
\pgfsetlinewidth{0.803000pt}%
\definecolor{currentstroke}{rgb}{0.000000,0.000000,0.000000}%
\pgfsetstrokecolor{currentstroke}%
\pgfsetdash{}{0pt}%
\pgfsys@defobject{currentmarker}{\pgfqpoint{-0.048611in}{0.000000in}}{\pgfqpoint{0.000000in}{0.000000in}}{%
\pgfpathmoveto{\pgfqpoint{0.000000in}{0.000000in}}%
\pgfpathlineto{\pgfqpoint{-0.048611in}{0.000000in}}%
\pgfusepath{stroke,fill}%
}%
\begin{pgfscope}%
\pgfsys@transformshift{0.625000in}{0.875402in}%
\pgfsys@useobject{currentmarker}{}%
\end{pgfscope}%
\end{pgfscope}%
\begin{pgfscope}%
\definecolor{textcolor}{rgb}{0.000000,0.000000,0.000000}%
\pgfsetstrokecolor{textcolor}%
\pgfsetfillcolor{textcolor}%
\pgftext[x=0.239775in,y=0.822640in,left,base]{\color{textcolor}\sffamily\fontsize{10.000000}{12.000000}\selectfont \(\displaystyle 10^{-4}\)}%
\end{pgfscope}%
\begin{pgfscope}%
\pgfsetbuttcap%
\pgfsetroundjoin%
\definecolor{currentfill}{rgb}{0.000000,0.000000,0.000000}%
\pgfsetfillcolor{currentfill}%
\pgfsetlinewidth{0.803000pt}%
\definecolor{currentstroke}{rgb}{0.000000,0.000000,0.000000}%
\pgfsetstrokecolor{currentstroke}%
\pgfsetdash{}{0pt}%
\pgfsys@defobject{currentmarker}{\pgfqpoint{-0.048611in}{0.000000in}}{\pgfqpoint{0.000000in}{0.000000in}}{%
\pgfpathmoveto{\pgfqpoint{0.000000in}{0.000000in}}%
\pgfpathlineto{\pgfqpoint{-0.048611in}{0.000000in}}%
\pgfusepath{stroke,fill}%
}%
\begin{pgfscope}%
\pgfsys@transformshift{0.625000in}{1.693068in}%
\pgfsys@useobject{currentmarker}{}%
\end{pgfscope}%
\end{pgfscope}%
\begin{pgfscope}%
\definecolor{textcolor}{rgb}{0.000000,0.000000,0.000000}%
\pgfsetstrokecolor{textcolor}%
\pgfsetfillcolor{textcolor}%
\pgftext[x=0.239775in,y=1.640306in,left,base]{\color{textcolor}\sffamily\fontsize{10.000000}{12.000000}\selectfont \(\displaystyle 10^{-3}\)}%
\end{pgfscope}%
\begin{pgfscope}%
\pgfsetbuttcap%
\pgfsetroundjoin%
\definecolor{currentfill}{rgb}{0.000000,0.000000,0.000000}%
\pgfsetfillcolor{currentfill}%
\pgfsetlinewidth{0.803000pt}%
\definecolor{currentstroke}{rgb}{0.000000,0.000000,0.000000}%
\pgfsetstrokecolor{currentstroke}%
\pgfsetdash{}{0pt}%
\pgfsys@defobject{currentmarker}{\pgfqpoint{-0.048611in}{0.000000in}}{\pgfqpoint{0.000000in}{0.000000in}}{%
\pgfpathmoveto{\pgfqpoint{0.000000in}{0.000000in}}%
\pgfpathlineto{\pgfqpoint{-0.048611in}{0.000000in}}%
\pgfusepath{stroke,fill}%
}%
\begin{pgfscope}%
\pgfsys@transformshift{0.625000in}{2.510734in}%
\pgfsys@useobject{currentmarker}{}%
\end{pgfscope}%
\end{pgfscope}%
\begin{pgfscope}%
\definecolor{textcolor}{rgb}{0.000000,0.000000,0.000000}%
\pgfsetstrokecolor{textcolor}%
\pgfsetfillcolor{textcolor}%
\pgftext[x=0.239775in,y=2.457973in,left,base]{\color{textcolor}\sffamily\fontsize{10.000000}{12.000000}\selectfont \(\displaystyle 10^{-2}\)}%
\end{pgfscope}%
\begin{pgfscope}%
\pgfsetbuttcap%
\pgfsetroundjoin%
\definecolor{currentfill}{rgb}{0.000000,0.000000,0.000000}%
\pgfsetfillcolor{currentfill}%
\pgfsetlinewidth{0.803000pt}%
\definecolor{currentstroke}{rgb}{0.000000,0.000000,0.000000}%
\pgfsetstrokecolor{currentstroke}%
\pgfsetdash{}{0pt}%
\pgfsys@defobject{currentmarker}{\pgfqpoint{-0.048611in}{0.000000in}}{\pgfqpoint{0.000000in}{0.000000in}}{%
\pgfpathmoveto{\pgfqpoint{0.000000in}{0.000000in}}%
\pgfpathlineto{\pgfqpoint{-0.048611in}{0.000000in}}%
\pgfusepath{stroke,fill}%
}%
\begin{pgfscope}%
\pgfsys@transformshift{0.625000in}{3.328400in}%
\pgfsys@useobject{currentmarker}{}%
\end{pgfscope}%
\end{pgfscope}%
\begin{pgfscope}%
\definecolor{textcolor}{rgb}{0.000000,0.000000,0.000000}%
\pgfsetstrokecolor{textcolor}%
\pgfsetfillcolor{textcolor}%
\pgftext[x=0.239775in,y=3.275639in,left,base]{\color{textcolor}\sffamily\fontsize{10.000000}{12.000000}\selectfont \(\displaystyle 10^{-1}\)}%
\end{pgfscope}%
\begin{pgfscope}%
\pgfsetbuttcap%
\pgfsetroundjoin%
\definecolor{currentfill}{rgb}{0.000000,0.000000,0.000000}%
\pgfsetfillcolor{currentfill}%
\pgfsetlinewidth{0.602250pt}%
\definecolor{currentstroke}{rgb}{0.000000,0.000000,0.000000}%
\pgfsetstrokecolor{currentstroke}%
\pgfsetdash{}{0pt}%
\pgfsys@defobject{currentmarker}{\pgfqpoint{-0.027778in}{0.000000in}}{\pgfqpoint{0.000000in}{0.000000in}}{%
\pgfpathmoveto{\pgfqpoint{0.000000in}{0.000000in}}%
\pgfpathlineto{\pgfqpoint{-0.027778in}{0.000000in}}%
\pgfusepath{stroke,fill}%
}%
\begin{pgfscope}%
\pgfsys@transformshift{0.625000in}{0.447861in}%
\pgfsys@useobject{currentmarker}{}%
\end{pgfscope}%
\end{pgfscope}%
\begin{pgfscope}%
\pgfsetbuttcap%
\pgfsetroundjoin%
\definecolor{currentfill}{rgb}{0.000000,0.000000,0.000000}%
\pgfsetfillcolor{currentfill}%
\pgfsetlinewidth{0.602250pt}%
\definecolor{currentstroke}{rgb}{0.000000,0.000000,0.000000}%
\pgfsetstrokecolor{currentstroke}%
\pgfsetdash{}{0pt}%
\pgfsys@defobject{currentmarker}{\pgfqpoint{-0.027778in}{0.000000in}}{\pgfqpoint{0.000000in}{0.000000in}}{%
\pgfpathmoveto{\pgfqpoint{0.000000in}{0.000000in}}%
\pgfpathlineto{\pgfqpoint{-0.027778in}{0.000000in}}%
\pgfusepath{stroke,fill}%
}%
\begin{pgfscope}%
\pgfsys@transformshift{0.625000in}{0.550020in}%
\pgfsys@useobject{currentmarker}{}%
\end{pgfscope}%
\end{pgfscope}%
\begin{pgfscope}%
\pgfsetbuttcap%
\pgfsetroundjoin%
\definecolor{currentfill}{rgb}{0.000000,0.000000,0.000000}%
\pgfsetfillcolor{currentfill}%
\pgfsetlinewidth{0.602250pt}%
\definecolor{currentstroke}{rgb}{0.000000,0.000000,0.000000}%
\pgfsetstrokecolor{currentstroke}%
\pgfsetdash{}{0pt}%
\pgfsys@defobject{currentmarker}{\pgfqpoint{-0.027778in}{0.000000in}}{\pgfqpoint{0.000000in}{0.000000in}}{%
\pgfpathmoveto{\pgfqpoint{0.000000in}{0.000000in}}%
\pgfpathlineto{\pgfqpoint{-0.027778in}{0.000000in}}%
\pgfusepath{stroke,fill}%
}%
\begin{pgfscope}%
\pgfsys@transformshift{0.625000in}{0.629260in}%
\pgfsys@useobject{currentmarker}{}%
\end{pgfscope}%
\end{pgfscope}%
\begin{pgfscope}%
\pgfsetbuttcap%
\pgfsetroundjoin%
\definecolor{currentfill}{rgb}{0.000000,0.000000,0.000000}%
\pgfsetfillcolor{currentfill}%
\pgfsetlinewidth{0.602250pt}%
\definecolor{currentstroke}{rgb}{0.000000,0.000000,0.000000}%
\pgfsetstrokecolor{currentstroke}%
\pgfsetdash{}{0pt}%
\pgfsys@defobject{currentmarker}{\pgfqpoint{-0.027778in}{0.000000in}}{\pgfqpoint{0.000000in}{0.000000in}}{%
\pgfpathmoveto{\pgfqpoint{0.000000in}{0.000000in}}%
\pgfpathlineto{\pgfqpoint{-0.027778in}{0.000000in}}%
\pgfusepath{stroke,fill}%
}%
\begin{pgfscope}%
\pgfsys@transformshift{0.625000in}{0.694003in}%
\pgfsys@useobject{currentmarker}{}%
\end{pgfscope}%
\end{pgfscope}%
\begin{pgfscope}%
\pgfsetbuttcap%
\pgfsetroundjoin%
\definecolor{currentfill}{rgb}{0.000000,0.000000,0.000000}%
\pgfsetfillcolor{currentfill}%
\pgfsetlinewidth{0.602250pt}%
\definecolor{currentstroke}{rgb}{0.000000,0.000000,0.000000}%
\pgfsetstrokecolor{currentstroke}%
\pgfsetdash{}{0pt}%
\pgfsys@defobject{currentmarker}{\pgfqpoint{-0.027778in}{0.000000in}}{\pgfqpoint{0.000000in}{0.000000in}}{%
\pgfpathmoveto{\pgfqpoint{0.000000in}{0.000000in}}%
\pgfpathlineto{\pgfqpoint{-0.027778in}{0.000000in}}%
\pgfusepath{stroke,fill}%
}%
\begin{pgfscope}%
\pgfsys@transformshift{0.625000in}{0.748744in}%
\pgfsys@useobject{currentmarker}{}%
\end{pgfscope}%
\end{pgfscope}%
\begin{pgfscope}%
\pgfsetbuttcap%
\pgfsetroundjoin%
\definecolor{currentfill}{rgb}{0.000000,0.000000,0.000000}%
\pgfsetfillcolor{currentfill}%
\pgfsetlinewidth{0.602250pt}%
\definecolor{currentstroke}{rgb}{0.000000,0.000000,0.000000}%
\pgfsetstrokecolor{currentstroke}%
\pgfsetdash{}{0pt}%
\pgfsys@defobject{currentmarker}{\pgfqpoint{-0.027778in}{0.000000in}}{\pgfqpoint{0.000000in}{0.000000in}}{%
\pgfpathmoveto{\pgfqpoint{0.000000in}{0.000000in}}%
\pgfpathlineto{\pgfqpoint{-0.027778in}{0.000000in}}%
\pgfusepath{stroke,fill}%
}%
\begin{pgfscope}%
\pgfsys@transformshift{0.625000in}{0.796162in}%
\pgfsys@useobject{currentmarker}{}%
\end{pgfscope}%
\end{pgfscope}%
\begin{pgfscope}%
\pgfsetbuttcap%
\pgfsetroundjoin%
\definecolor{currentfill}{rgb}{0.000000,0.000000,0.000000}%
\pgfsetfillcolor{currentfill}%
\pgfsetlinewidth{0.602250pt}%
\definecolor{currentstroke}{rgb}{0.000000,0.000000,0.000000}%
\pgfsetstrokecolor{currentstroke}%
\pgfsetdash{}{0pt}%
\pgfsys@defobject{currentmarker}{\pgfqpoint{-0.027778in}{0.000000in}}{\pgfqpoint{0.000000in}{0.000000in}}{%
\pgfpathmoveto{\pgfqpoint{0.000000in}{0.000000in}}%
\pgfpathlineto{\pgfqpoint{-0.027778in}{0.000000in}}%
\pgfusepath{stroke,fill}%
}%
\begin{pgfscope}%
\pgfsys@transformshift{0.625000in}{0.837987in}%
\pgfsys@useobject{currentmarker}{}%
\end{pgfscope}%
\end{pgfscope}%
\begin{pgfscope}%
\pgfsetbuttcap%
\pgfsetroundjoin%
\definecolor{currentfill}{rgb}{0.000000,0.000000,0.000000}%
\pgfsetfillcolor{currentfill}%
\pgfsetlinewidth{0.602250pt}%
\definecolor{currentstroke}{rgb}{0.000000,0.000000,0.000000}%
\pgfsetstrokecolor{currentstroke}%
\pgfsetdash{}{0pt}%
\pgfsys@defobject{currentmarker}{\pgfqpoint{-0.027778in}{0.000000in}}{\pgfqpoint{0.000000in}{0.000000in}}{%
\pgfpathmoveto{\pgfqpoint{0.000000in}{0.000000in}}%
\pgfpathlineto{\pgfqpoint{-0.027778in}{0.000000in}}%
\pgfusepath{stroke,fill}%
}%
\begin{pgfscope}%
\pgfsys@transformshift{0.625000in}{1.121544in}%
\pgfsys@useobject{currentmarker}{}%
\end{pgfscope}%
\end{pgfscope}%
\begin{pgfscope}%
\pgfsetbuttcap%
\pgfsetroundjoin%
\definecolor{currentfill}{rgb}{0.000000,0.000000,0.000000}%
\pgfsetfillcolor{currentfill}%
\pgfsetlinewidth{0.602250pt}%
\definecolor{currentstroke}{rgb}{0.000000,0.000000,0.000000}%
\pgfsetstrokecolor{currentstroke}%
\pgfsetdash{}{0pt}%
\pgfsys@defobject{currentmarker}{\pgfqpoint{-0.027778in}{0.000000in}}{\pgfqpoint{0.000000in}{0.000000in}}{%
\pgfpathmoveto{\pgfqpoint{0.000000in}{0.000000in}}%
\pgfpathlineto{\pgfqpoint{-0.027778in}{0.000000in}}%
\pgfusepath{stroke,fill}%
}%
\begin{pgfscope}%
\pgfsys@transformshift{0.625000in}{1.265528in}%
\pgfsys@useobject{currentmarker}{}%
\end{pgfscope}%
\end{pgfscope}%
\begin{pgfscope}%
\pgfsetbuttcap%
\pgfsetroundjoin%
\definecolor{currentfill}{rgb}{0.000000,0.000000,0.000000}%
\pgfsetfillcolor{currentfill}%
\pgfsetlinewidth{0.602250pt}%
\definecolor{currentstroke}{rgb}{0.000000,0.000000,0.000000}%
\pgfsetstrokecolor{currentstroke}%
\pgfsetdash{}{0pt}%
\pgfsys@defobject{currentmarker}{\pgfqpoint{-0.027778in}{0.000000in}}{\pgfqpoint{0.000000in}{0.000000in}}{%
\pgfpathmoveto{\pgfqpoint{0.000000in}{0.000000in}}%
\pgfpathlineto{\pgfqpoint{-0.027778in}{0.000000in}}%
\pgfusepath{stroke,fill}%
}%
\begin{pgfscope}%
\pgfsys@transformshift{0.625000in}{1.367686in}%
\pgfsys@useobject{currentmarker}{}%
\end{pgfscope}%
\end{pgfscope}%
\begin{pgfscope}%
\pgfsetbuttcap%
\pgfsetroundjoin%
\definecolor{currentfill}{rgb}{0.000000,0.000000,0.000000}%
\pgfsetfillcolor{currentfill}%
\pgfsetlinewidth{0.602250pt}%
\definecolor{currentstroke}{rgb}{0.000000,0.000000,0.000000}%
\pgfsetstrokecolor{currentstroke}%
\pgfsetdash{}{0pt}%
\pgfsys@defobject{currentmarker}{\pgfqpoint{-0.027778in}{0.000000in}}{\pgfqpoint{0.000000in}{0.000000in}}{%
\pgfpathmoveto{\pgfqpoint{0.000000in}{0.000000in}}%
\pgfpathlineto{\pgfqpoint{-0.027778in}{0.000000in}}%
\pgfusepath{stroke,fill}%
}%
\begin{pgfscope}%
\pgfsys@transformshift{0.625000in}{1.446926in}%
\pgfsys@useobject{currentmarker}{}%
\end{pgfscope}%
\end{pgfscope}%
\begin{pgfscope}%
\pgfsetbuttcap%
\pgfsetroundjoin%
\definecolor{currentfill}{rgb}{0.000000,0.000000,0.000000}%
\pgfsetfillcolor{currentfill}%
\pgfsetlinewidth{0.602250pt}%
\definecolor{currentstroke}{rgb}{0.000000,0.000000,0.000000}%
\pgfsetstrokecolor{currentstroke}%
\pgfsetdash{}{0pt}%
\pgfsys@defobject{currentmarker}{\pgfqpoint{-0.027778in}{0.000000in}}{\pgfqpoint{0.000000in}{0.000000in}}{%
\pgfpathmoveto{\pgfqpoint{0.000000in}{0.000000in}}%
\pgfpathlineto{\pgfqpoint{-0.027778in}{0.000000in}}%
\pgfusepath{stroke,fill}%
}%
\begin{pgfscope}%
\pgfsys@transformshift{0.625000in}{1.511670in}%
\pgfsys@useobject{currentmarker}{}%
\end{pgfscope}%
\end{pgfscope}%
\begin{pgfscope}%
\pgfsetbuttcap%
\pgfsetroundjoin%
\definecolor{currentfill}{rgb}{0.000000,0.000000,0.000000}%
\pgfsetfillcolor{currentfill}%
\pgfsetlinewidth{0.602250pt}%
\definecolor{currentstroke}{rgb}{0.000000,0.000000,0.000000}%
\pgfsetstrokecolor{currentstroke}%
\pgfsetdash{}{0pt}%
\pgfsys@defobject{currentmarker}{\pgfqpoint{-0.027778in}{0.000000in}}{\pgfqpoint{0.000000in}{0.000000in}}{%
\pgfpathmoveto{\pgfqpoint{0.000000in}{0.000000in}}%
\pgfpathlineto{\pgfqpoint{-0.027778in}{0.000000in}}%
\pgfusepath{stroke,fill}%
}%
\begin{pgfscope}%
\pgfsys@transformshift{0.625000in}{1.566410in}%
\pgfsys@useobject{currentmarker}{}%
\end{pgfscope}%
\end{pgfscope}%
\begin{pgfscope}%
\pgfsetbuttcap%
\pgfsetroundjoin%
\definecolor{currentfill}{rgb}{0.000000,0.000000,0.000000}%
\pgfsetfillcolor{currentfill}%
\pgfsetlinewidth{0.602250pt}%
\definecolor{currentstroke}{rgb}{0.000000,0.000000,0.000000}%
\pgfsetstrokecolor{currentstroke}%
\pgfsetdash{}{0pt}%
\pgfsys@defobject{currentmarker}{\pgfqpoint{-0.027778in}{0.000000in}}{\pgfqpoint{0.000000in}{0.000000in}}{%
\pgfpathmoveto{\pgfqpoint{0.000000in}{0.000000in}}%
\pgfpathlineto{\pgfqpoint{-0.027778in}{0.000000in}}%
\pgfusepath{stroke,fill}%
}%
\begin{pgfscope}%
\pgfsys@transformshift{0.625000in}{1.613828in}%
\pgfsys@useobject{currentmarker}{}%
\end{pgfscope}%
\end{pgfscope}%
\begin{pgfscope}%
\pgfsetbuttcap%
\pgfsetroundjoin%
\definecolor{currentfill}{rgb}{0.000000,0.000000,0.000000}%
\pgfsetfillcolor{currentfill}%
\pgfsetlinewidth{0.602250pt}%
\definecolor{currentstroke}{rgb}{0.000000,0.000000,0.000000}%
\pgfsetstrokecolor{currentstroke}%
\pgfsetdash{}{0pt}%
\pgfsys@defobject{currentmarker}{\pgfqpoint{-0.027778in}{0.000000in}}{\pgfqpoint{0.000000in}{0.000000in}}{%
\pgfpathmoveto{\pgfqpoint{0.000000in}{0.000000in}}%
\pgfpathlineto{\pgfqpoint{-0.027778in}{0.000000in}}%
\pgfusepath{stroke,fill}%
}%
\begin{pgfscope}%
\pgfsys@transformshift{0.625000in}{1.655654in}%
\pgfsys@useobject{currentmarker}{}%
\end{pgfscope}%
\end{pgfscope}%
\begin{pgfscope}%
\pgfsetbuttcap%
\pgfsetroundjoin%
\definecolor{currentfill}{rgb}{0.000000,0.000000,0.000000}%
\pgfsetfillcolor{currentfill}%
\pgfsetlinewidth{0.602250pt}%
\definecolor{currentstroke}{rgb}{0.000000,0.000000,0.000000}%
\pgfsetstrokecolor{currentstroke}%
\pgfsetdash{}{0pt}%
\pgfsys@defobject{currentmarker}{\pgfqpoint{-0.027778in}{0.000000in}}{\pgfqpoint{0.000000in}{0.000000in}}{%
\pgfpathmoveto{\pgfqpoint{0.000000in}{0.000000in}}%
\pgfpathlineto{\pgfqpoint{-0.027778in}{0.000000in}}%
\pgfusepath{stroke,fill}%
}%
\begin{pgfscope}%
\pgfsys@transformshift{0.625000in}{1.939210in}%
\pgfsys@useobject{currentmarker}{}%
\end{pgfscope}%
\end{pgfscope}%
\begin{pgfscope}%
\pgfsetbuttcap%
\pgfsetroundjoin%
\definecolor{currentfill}{rgb}{0.000000,0.000000,0.000000}%
\pgfsetfillcolor{currentfill}%
\pgfsetlinewidth{0.602250pt}%
\definecolor{currentstroke}{rgb}{0.000000,0.000000,0.000000}%
\pgfsetstrokecolor{currentstroke}%
\pgfsetdash{}{0pt}%
\pgfsys@defobject{currentmarker}{\pgfqpoint{-0.027778in}{0.000000in}}{\pgfqpoint{0.000000in}{0.000000in}}{%
\pgfpathmoveto{\pgfqpoint{0.000000in}{0.000000in}}%
\pgfpathlineto{\pgfqpoint{-0.027778in}{0.000000in}}%
\pgfusepath{stroke,fill}%
}%
\begin{pgfscope}%
\pgfsys@transformshift{0.625000in}{2.083194in}%
\pgfsys@useobject{currentmarker}{}%
\end{pgfscope}%
\end{pgfscope}%
\begin{pgfscope}%
\pgfsetbuttcap%
\pgfsetroundjoin%
\definecolor{currentfill}{rgb}{0.000000,0.000000,0.000000}%
\pgfsetfillcolor{currentfill}%
\pgfsetlinewidth{0.602250pt}%
\definecolor{currentstroke}{rgb}{0.000000,0.000000,0.000000}%
\pgfsetstrokecolor{currentstroke}%
\pgfsetdash{}{0pt}%
\pgfsys@defobject{currentmarker}{\pgfqpoint{-0.027778in}{0.000000in}}{\pgfqpoint{0.000000in}{0.000000in}}{%
\pgfpathmoveto{\pgfqpoint{0.000000in}{0.000000in}}%
\pgfpathlineto{\pgfqpoint{-0.027778in}{0.000000in}}%
\pgfusepath{stroke,fill}%
}%
\begin{pgfscope}%
\pgfsys@transformshift{0.625000in}{2.185352in}%
\pgfsys@useobject{currentmarker}{}%
\end{pgfscope}%
\end{pgfscope}%
\begin{pgfscope}%
\pgfsetbuttcap%
\pgfsetroundjoin%
\definecolor{currentfill}{rgb}{0.000000,0.000000,0.000000}%
\pgfsetfillcolor{currentfill}%
\pgfsetlinewidth{0.602250pt}%
\definecolor{currentstroke}{rgb}{0.000000,0.000000,0.000000}%
\pgfsetstrokecolor{currentstroke}%
\pgfsetdash{}{0pt}%
\pgfsys@defobject{currentmarker}{\pgfqpoint{-0.027778in}{0.000000in}}{\pgfqpoint{0.000000in}{0.000000in}}{%
\pgfpathmoveto{\pgfqpoint{0.000000in}{0.000000in}}%
\pgfpathlineto{\pgfqpoint{-0.027778in}{0.000000in}}%
\pgfusepath{stroke,fill}%
}%
\begin{pgfscope}%
\pgfsys@transformshift{0.625000in}{2.264592in}%
\pgfsys@useobject{currentmarker}{}%
\end{pgfscope}%
\end{pgfscope}%
\begin{pgfscope}%
\pgfsetbuttcap%
\pgfsetroundjoin%
\definecolor{currentfill}{rgb}{0.000000,0.000000,0.000000}%
\pgfsetfillcolor{currentfill}%
\pgfsetlinewidth{0.602250pt}%
\definecolor{currentstroke}{rgb}{0.000000,0.000000,0.000000}%
\pgfsetstrokecolor{currentstroke}%
\pgfsetdash{}{0pt}%
\pgfsys@defobject{currentmarker}{\pgfqpoint{-0.027778in}{0.000000in}}{\pgfqpoint{0.000000in}{0.000000in}}{%
\pgfpathmoveto{\pgfqpoint{0.000000in}{0.000000in}}%
\pgfpathlineto{\pgfqpoint{-0.027778in}{0.000000in}}%
\pgfusepath{stroke,fill}%
}%
\begin{pgfscope}%
\pgfsys@transformshift{0.625000in}{2.329336in}%
\pgfsys@useobject{currentmarker}{}%
\end{pgfscope}%
\end{pgfscope}%
\begin{pgfscope}%
\pgfsetbuttcap%
\pgfsetroundjoin%
\definecolor{currentfill}{rgb}{0.000000,0.000000,0.000000}%
\pgfsetfillcolor{currentfill}%
\pgfsetlinewidth{0.602250pt}%
\definecolor{currentstroke}{rgb}{0.000000,0.000000,0.000000}%
\pgfsetstrokecolor{currentstroke}%
\pgfsetdash{}{0pt}%
\pgfsys@defobject{currentmarker}{\pgfqpoint{-0.027778in}{0.000000in}}{\pgfqpoint{0.000000in}{0.000000in}}{%
\pgfpathmoveto{\pgfqpoint{0.000000in}{0.000000in}}%
\pgfpathlineto{\pgfqpoint{-0.027778in}{0.000000in}}%
\pgfusepath{stroke,fill}%
}%
\begin{pgfscope}%
\pgfsys@transformshift{0.625000in}{2.384076in}%
\pgfsys@useobject{currentmarker}{}%
\end{pgfscope}%
\end{pgfscope}%
\begin{pgfscope}%
\pgfsetbuttcap%
\pgfsetroundjoin%
\definecolor{currentfill}{rgb}{0.000000,0.000000,0.000000}%
\pgfsetfillcolor{currentfill}%
\pgfsetlinewidth{0.602250pt}%
\definecolor{currentstroke}{rgb}{0.000000,0.000000,0.000000}%
\pgfsetstrokecolor{currentstroke}%
\pgfsetdash{}{0pt}%
\pgfsys@defobject{currentmarker}{\pgfqpoint{-0.027778in}{0.000000in}}{\pgfqpoint{0.000000in}{0.000000in}}{%
\pgfpathmoveto{\pgfqpoint{0.000000in}{0.000000in}}%
\pgfpathlineto{\pgfqpoint{-0.027778in}{0.000000in}}%
\pgfusepath{stroke,fill}%
}%
\begin{pgfscope}%
\pgfsys@transformshift{0.625000in}{2.431494in}%
\pgfsys@useobject{currentmarker}{}%
\end{pgfscope}%
\end{pgfscope}%
\begin{pgfscope}%
\pgfsetbuttcap%
\pgfsetroundjoin%
\definecolor{currentfill}{rgb}{0.000000,0.000000,0.000000}%
\pgfsetfillcolor{currentfill}%
\pgfsetlinewidth{0.602250pt}%
\definecolor{currentstroke}{rgb}{0.000000,0.000000,0.000000}%
\pgfsetstrokecolor{currentstroke}%
\pgfsetdash{}{0pt}%
\pgfsys@defobject{currentmarker}{\pgfqpoint{-0.027778in}{0.000000in}}{\pgfqpoint{0.000000in}{0.000000in}}{%
\pgfpathmoveto{\pgfqpoint{0.000000in}{0.000000in}}%
\pgfpathlineto{\pgfqpoint{-0.027778in}{0.000000in}}%
\pgfusepath{stroke,fill}%
}%
\begin{pgfscope}%
\pgfsys@transformshift{0.625000in}{2.473320in}%
\pgfsys@useobject{currentmarker}{}%
\end{pgfscope}%
\end{pgfscope}%
\begin{pgfscope}%
\pgfsetbuttcap%
\pgfsetroundjoin%
\definecolor{currentfill}{rgb}{0.000000,0.000000,0.000000}%
\pgfsetfillcolor{currentfill}%
\pgfsetlinewidth{0.602250pt}%
\definecolor{currentstroke}{rgb}{0.000000,0.000000,0.000000}%
\pgfsetstrokecolor{currentstroke}%
\pgfsetdash{}{0pt}%
\pgfsys@defobject{currentmarker}{\pgfqpoint{-0.027778in}{0.000000in}}{\pgfqpoint{0.000000in}{0.000000in}}{%
\pgfpathmoveto{\pgfqpoint{0.000000in}{0.000000in}}%
\pgfpathlineto{\pgfqpoint{-0.027778in}{0.000000in}}%
\pgfusepath{stroke,fill}%
}%
\begin{pgfscope}%
\pgfsys@transformshift{0.625000in}{2.756876in}%
\pgfsys@useobject{currentmarker}{}%
\end{pgfscope}%
\end{pgfscope}%
\begin{pgfscope}%
\pgfsetbuttcap%
\pgfsetroundjoin%
\definecolor{currentfill}{rgb}{0.000000,0.000000,0.000000}%
\pgfsetfillcolor{currentfill}%
\pgfsetlinewidth{0.602250pt}%
\definecolor{currentstroke}{rgb}{0.000000,0.000000,0.000000}%
\pgfsetstrokecolor{currentstroke}%
\pgfsetdash{}{0pt}%
\pgfsys@defobject{currentmarker}{\pgfqpoint{-0.027778in}{0.000000in}}{\pgfqpoint{0.000000in}{0.000000in}}{%
\pgfpathmoveto{\pgfqpoint{0.000000in}{0.000000in}}%
\pgfpathlineto{\pgfqpoint{-0.027778in}{0.000000in}}%
\pgfusepath{stroke,fill}%
}%
\begin{pgfscope}%
\pgfsys@transformshift{0.625000in}{2.900860in}%
\pgfsys@useobject{currentmarker}{}%
\end{pgfscope}%
\end{pgfscope}%
\begin{pgfscope}%
\pgfsetbuttcap%
\pgfsetroundjoin%
\definecolor{currentfill}{rgb}{0.000000,0.000000,0.000000}%
\pgfsetfillcolor{currentfill}%
\pgfsetlinewidth{0.602250pt}%
\definecolor{currentstroke}{rgb}{0.000000,0.000000,0.000000}%
\pgfsetstrokecolor{currentstroke}%
\pgfsetdash{}{0pt}%
\pgfsys@defobject{currentmarker}{\pgfqpoint{-0.027778in}{0.000000in}}{\pgfqpoint{0.000000in}{0.000000in}}{%
\pgfpathmoveto{\pgfqpoint{0.000000in}{0.000000in}}%
\pgfpathlineto{\pgfqpoint{-0.027778in}{0.000000in}}%
\pgfusepath{stroke,fill}%
}%
\begin{pgfscope}%
\pgfsys@transformshift{0.625000in}{3.003018in}%
\pgfsys@useobject{currentmarker}{}%
\end{pgfscope}%
\end{pgfscope}%
\begin{pgfscope}%
\pgfsetbuttcap%
\pgfsetroundjoin%
\definecolor{currentfill}{rgb}{0.000000,0.000000,0.000000}%
\pgfsetfillcolor{currentfill}%
\pgfsetlinewidth{0.602250pt}%
\definecolor{currentstroke}{rgb}{0.000000,0.000000,0.000000}%
\pgfsetstrokecolor{currentstroke}%
\pgfsetdash{}{0pt}%
\pgfsys@defobject{currentmarker}{\pgfqpoint{-0.027778in}{0.000000in}}{\pgfqpoint{0.000000in}{0.000000in}}{%
\pgfpathmoveto{\pgfqpoint{0.000000in}{0.000000in}}%
\pgfpathlineto{\pgfqpoint{-0.027778in}{0.000000in}}%
\pgfusepath{stroke,fill}%
}%
\begin{pgfscope}%
\pgfsys@transformshift{0.625000in}{3.082258in}%
\pgfsys@useobject{currentmarker}{}%
\end{pgfscope}%
\end{pgfscope}%
\begin{pgfscope}%
\pgfsetbuttcap%
\pgfsetroundjoin%
\definecolor{currentfill}{rgb}{0.000000,0.000000,0.000000}%
\pgfsetfillcolor{currentfill}%
\pgfsetlinewidth{0.602250pt}%
\definecolor{currentstroke}{rgb}{0.000000,0.000000,0.000000}%
\pgfsetstrokecolor{currentstroke}%
\pgfsetdash{}{0pt}%
\pgfsys@defobject{currentmarker}{\pgfqpoint{-0.027778in}{0.000000in}}{\pgfqpoint{0.000000in}{0.000000in}}{%
\pgfpathmoveto{\pgfqpoint{0.000000in}{0.000000in}}%
\pgfpathlineto{\pgfqpoint{-0.027778in}{0.000000in}}%
\pgfusepath{stroke,fill}%
}%
\begin{pgfscope}%
\pgfsys@transformshift{0.625000in}{3.147002in}%
\pgfsys@useobject{currentmarker}{}%
\end{pgfscope}%
\end{pgfscope}%
\begin{pgfscope}%
\pgfsetbuttcap%
\pgfsetroundjoin%
\definecolor{currentfill}{rgb}{0.000000,0.000000,0.000000}%
\pgfsetfillcolor{currentfill}%
\pgfsetlinewidth{0.602250pt}%
\definecolor{currentstroke}{rgb}{0.000000,0.000000,0.000000}%
\pgfsetstrokecolor{currentstroke}%
\pgfsetdash{}{0pt}%
\pgfsys@defobject{currentmarker}{\pgfqpoint{-0.027778in}{0.000000in}}{\pgfqpoint{0.000000in}{0.000000in}}{%
\pgfpathmoveto{\pgfqpoint{0.000000in}{0.000000in}}%
\pgfpathlineto{\pgfqpoint{-0.027778in}{0.000000in}}%
\pgfusepath{stroke,fill}%
}%
\begin{pgfscope}%
\pgfsys@transformshift{0.625000in}{3.201742in}%
\pgfsys@useobject{currentmarker}{}%
\end{pgfscope}%
\end{pgfscope}%
\begin{pgfscope}%
\pgfsetbuttcap%
\pgfsetroundjoin%
\definecolor{currentfill}{rgb}{0.000000,0.000000,0.000000}%
\pgfsetfillcolor{currentfill}%
\pgfsetlinewidth{0.602250pt}%
\definecolor{currentstroke}{rgb}{0.000000,0.000000,0.000000}%
\pgfsetstrokecolor{currentstroke}%
\pgfsetdash{}{0pt}%
\pgfsys@defobject{currentmarker}{\pgfqpoint{-0.027778in}{0.000000in}}{\pgfqpoint{0.000000in}{0.000000in}}{%
\pgfpathmoveto{\pgfqpoint{0.000000in}{0.000000in}}%
\pgfpathlineto{\pgfqpoint{-0.027778in}{0.000000in}}%
\pgfusepath{stroke,fill}%
}%
\begin{pgfscope}%
\pgfsys@transformshift{0.625000in}{3.249160in}%
\pgfsys@useobject{currentmarker}{}%
\end{pgfscope}%
\end{pgfscope}%
\begin{pgfscope}%
\pgfsetbuttcap%
\pgfsetroundjoin%
\definecolor{currentfill}{rgb}{0.000000,0.000000,0.000000}%
\pgfsetfillcolor{currentfill}%
\pgfsetlinewidth{0.602250pt}%
\definecolor{currentstroke}{rgb}{0.000000,0.000000,0.000000}%
\pgfsetstrokecolor{currentstroke}%
\pgfsetdash{}{0pt}%
\pgfsys@defobject{currentmarker}{\pgfqpoint{-0.027778in}{0.000000in}}{\pgfqpoint{0.000000in}{0.000000in}}{%
\pgfpathmoveto{\pgfqpoint{0.000000in}{0.000000in}}%
\pgfpathlineto{\pgfqpoint{-0.027778in}{0.000000in}}%
\pgfusepath{stroke,fill}%
}%
\begin{pgfscope}%
\pgfsys@transformshift{0.625000in}{3.290986in}%
\pgfsys@useobject{currentmarker}{}%
\end{pgfscope}%
\end{pgfscope}%
\begin{pgfscope}%
\definecolor{textcolor}{rgb}{0.000000,0.000000,0.000000}%
\pgfsetstrokecolor{textcolor}%
\pgfsetfillcolor{textcolor}%
\pgftext[x=0.184220in,y=1.980000in,,bottom,rotate=90.000000]{\color{textcolor}\sffamily\fontsize{10.000000}{12.000000}\selectfont QMC Error}%
\end{pgfscope}%
\begin{pgfscope}%
\pgfpathrectangle{\pgfqpoint{0.625000in}{0.440000in}}{\pgfqpoint{3.875000in}{3.080000in}}%
\pgfusepath{clip}%
\pgfsetbuttcap%
\pgfsetroundjoin%
\definecolor{currentfill}{rgb}{0.000000,0.000000,0.000000}%
\pgfsetfillcolor{currentfill}%
\pgfsetlinewidth{1.003750pt}%
\definecolor{currentstroke}{rgb}{0.000000,0.000000,0.000000}%
\pgfsetstrokecolor{currentstroke}%
\pgfsetdash{}{0pt}%
\pgfsys@defobject{currentmarker}{\pgfqpoint{-0.041667in}{-0.041667in}}{\pgfqpoint{0.041667in}{0.041667in}}{%
\pgfpathmoveto{\pgfqpoint{0.000000in}{-0.041667in}}%
\pgfpathcurveto{\pgfqpoint{0.011050in}{-0.041667in}}{\pgfqpoint{0.021649in}{-0.037276in}}{\pgfqpoint{0.029463in}{-0.029463in}}%
\pgfpathcurveto{\pgfqpoint{0.037276in}{-0.021649in}}{\pgfqpoint{0.041667in}{-0.011050in}}{\pgfqpoint{0.041667in}{0.000000in}}%
\pgfpathcurveto{\pgfqpoint{0.041667in}{0.011050in}}{\pgfqpoint{0.037276in}{0.021649in}}{\pgfqpoint{0.029463in}{0.029463in}}%
\pgfpathcurveto{\pgfqpoint{0.021649in}{0.037276in}}{\pgfqpoint{0.011050in}{0.041667in}}{\pgfqpoint{0.000000in}{0.041667in}}%
\pgfpathcurveto{\pgfqpoint{-0.011050in}{0.041667in}}{\pgfqpoint{-0.021649in}{0.037276in}}{\pgfqpoint{-0.029463in}{0.029463in}}%
\pgfpathcurveto{\pgfqpoint{-0.037276in}{0.021649in}}{\pgfqpoint{-0.041667in}{0.011050in}}{\pgfqpoint{-0.041667in}{0.000000in}}%
\pgfpathcurveto{\pgfqpoint{-0.041667in}{-0.011050in}}{\pgfqpoint{-0.037276in}{-0.021649in}}{\pgfqpoint{-0.029463in}{-0.029463in}}%
\pgfpathcurveto{\pgfqpoint{-0.021649in}{-0.037276in}}{\pgfqpoint{-0.011050in}{-0.041667in}}{\pgfqpoint{0.000000in}{-0.041667in}}%
\pgfpathclose%
\pgfusepath{stroke,fill}%
}%
\begin{pgfscope}%
\pgfsys@transformshift{0.801136in}{3.380000in}%
\pgfsys@useobject{currentmarker}{}%
\end{pgfscope}%
\begin{pgfscope}%
\pgfsys@transformshift{1.121384in}{3.007078in}%
\pgfsys@useobject{currentmarker}{}%
\end{pgfscope}%
\begin{pgfscope}%
\pgfsys@transformshift{1.441632in}{2.783973in}%
\pgfsys@useobject{currentmarker}{}%
\end{pgfscope}%
\begin{pgfscope}%
\pgfsys@transformshift{1.761880in}{2.499284in}%
\pgfsys@useobject{currentmarker}{}%
\end{pgfscope}%
\begin{pgfscope}%
\pgfsys@transformshift{2.082128in}{2.259271in}%
\pgfsys@useobject{currentmarker}{}%
\end{pgfscope}%
\begin{pgfscope}%
\pgfsys@transformshift{2.402376in}{2.153945in}%
\pgfsys@useobject{currentmarker}{}%
\end{pgfscope}%
\begin{pgfscope}%
\pgfsys@transformshift{2.722624in}{1.989929in}%
\pgfsys@useobject{currentmarker}{}%
\end{pgfscope}%
\begin{pgfscope}%
\pgfsys@transformshift{3.042872in}{1.674793in}%
\pgfsys@useobject{currentmarker}{}%
\end{pgfscope}%
\begin{pgfscope}%
\pgfsys@transformshift{3.363120in}{1.288609in}%
\pgfsys@useobject{currentmarker}{}%
\end{pgfscope}%
\begin{pgfscope}%
\pgfsys@transformshift{3.683368in}{1.213072in}%
\pgfsys@useobject{currentmarker}{}%
\end{pgfscope}%
\begin{pgfscope}%
\pgfsys@transformshift{4.003616in}{0.846293in}%
\pgfsys@useobject{currentmarker}{}%
\end{pgfscope}%
\begin{pgfscope}%
\pgfsys@transformshift{4.323864in}{0.580000in}%
\pgfsys@useobject{currentmarker}{}%
\end{pgfscope}%
\end{pgfscope}%
\begin{pgfscope}%
\pgfpathrectangle{\pgfqpoint{0.625000in}{0.440000in}}{\pgfqpoint{3.875000in}{3.080000in}}%
\pgfusepath{clip}%
\pgfsetbuttcap%
\pgfsetroundjoin%
\pgfsetlinewidth{1.505625pt}%
\definecolor{currentstroke}{rgb}{0.000000,0.000000,0.000000}%
\pgfsetstrokecolor{currentstroke}%
\pgfsetdash{{5.550000pt}{2.400000pt}}{0.000000pt}%
\pgfpathmoveto{\pgfqpoint{0.801136in}{3.304063in}}%
\pgfpathlineto{\pgfqpoint{4.323864in}{0.641978in}}%
\pgfusepath{stroke}%
\end{pgfscope}%
\begin{pgfscope}%
\pgfsetrectcap%
\pgfsetmiterjoin%
\pgfsetlinewidth{0.803000pt}%
\definecolor{currentstroke}{rgb}{0.000000,0.000000,0.000000}%
\pgfsetstrokecolor{currentstroke}%
\pgfsetdash{}{0pt}%
\pgfpathmoveto{\pgfqpoint{0.625000in}{0.440000in}}%
\pgfpathlineto{\pgfqpoint{0.625000in}{3.520000in}}%
\pgfusepath{stroke}%
\end{pgfscope}%
\begin{pgfscope}%
\pgfsetrectcap%
\pgfsetmiterjoin%
\pgfsetlinewidth{0.000000pt}%
\definecolor{currentstroke}{rgb}{0.000000,0.000000,0.000000}%
\pgfsetstrokecolor{currentstroke}%
\pgfsetstrokeopacity{0.000000}%
\pgfsetdash{}{0pt}%
\pgfpathmoveto{\pgfqpoint{4.500000in}{0.440000in}}%
\pgfpathlineto{\pgfqpoint{4.500000in}{3.520000in}}%
\pgfusepath{}%
\end{pgfscope}%
\begin{pgfscope}%
\pgfsetrectcap%
\pgfsetmiterjoin%
\pgfsetlinewidth{0.803000pt}%
\definecolor{currentstroke}{rgb}{0.000000,0.000000,0.000000}%
\pgfsetstrokecolor{currentstroke}%
\pgfsetdash{}{0pt}%
\pgfpathmoveto{\pgfqpoint{0.625000in}{0.440000in}}%
\pgfpathlineto{\pgfqpoint{4.500000in}{0.440000in}}%
\pgfusepath{stroke}%
\end{pgfscope}%
\begin{pgfscope}%
\pgfsetrectcap%
\pgfsetmiterjoin%
\pgfsetlinewidth{0.000000pt}%
\definecolor{currentstroke}{rgb}{0.000000,0.000000,0.000000}%
\pgfsetstrokecolor{currentstroke}%
\pgfsetstrokeopacity{0.000000}%
\pgfsetdash{}{0pt}%
\pgfpathmoveto{\pgfqpoint{0.625000in}{3.520000in}}%
\pgfpathlineto{\pgfqpoint{4.500000in}{3.520000in}}%
\pgfusepath{}%
\end{pgfscope}%
\begin{pgfscope}%
\pgfsetbuttcap%
\pgfsetmiterjoin%
\definecolor{currentfill}{rgb}{1.000000,1.000000,1.000000}%
\pgfsetfillcolor{currentfill}%
\pgfsetfillopacity{0.800000}%
\pgfsetlinewidth{1.003750pt}%
\definecolor{currentstroke}{rgb}{0.800000,0.800000,0.800000}%
\pgfsetstrokecolor{currentstroke}%
\pgfsetstrokeopacity{0.800000}%
\pgfsetdash{}{0pt}%
\pgfpathmoveto{\pgfqpoint{3.042243in}{3.155267in}}%
\pgfpathlineto{\pgfqpoint{4.402778in}{3.155267in}}%
\pgfpathquadraticcurveto{\pgfqpoint{4.430556in}{3.155267in}}{\pgfqpoint{4.430556in}{3.183045in}}%
\pgfpathlineto{\pgfqpoint{4.430556in}{3.422778in}}%
\pgfpathquadraticcurveto{\pgfqpoint{4.430556in}{3.450556in}}{\pgfqpoint{4.402778in}{3.450556in}}%
\pgfpathlineto{\pgfqpoint{3.042243in}{3.450556in}}%
\pgfpathquadraticcurveto{\pgfqpoint{3.014465in}{3.450556in}}{\pgfqpoint{3.014465in}{3.422778in}}%
\pgfpathlineto{\pgfqpoint{3.014465in}{3.183045in}}%
\pgfpathquadraticcurveto{\pgfqpoint{3.014465in}{3.155267in}}{\pgfqpoint{3.042243in}{3.155267in}}%
\pgfpathclose%
\pgfusepath{stroke,fill}%
\end{pgfscope}%
\begin{pgfscope}%
\pgfsetbuttcap%
\pgfsetroundjoin%
\pgfsetlinewidth{1.505625pt}%
\definecolor{currentstroke}{rgb}{0.000000,0.000000,0.000000}%
\pgfsetstrokecolor{currentstroke}%
\pgfsetdash{{5.550000pt}{2.400000pt}}{0.000000pt}%
\pgfpathmoveto{\pgfqpoint{3.070021in}{3.321543in}}%
\pgfpathlineto{\pgfqpoint{3.347799in}{3.321543in}}%
\pgfusepath{stroke}%
\end{pgfscope}%
\begin{pgfscope}%
\definecolor{textcolor}{rgb}{0.000000,0.000000,0.000000}%
\pgfsetstrokecolor{textcolor}%
\pgfsetfillcolor{textcolor}%
\pgftext[x=3.458910in,y=3.272932in,left,base]{\color{textcolor}\sffamily\fontsize{10.000000}{12.000000}\selectfont \(\displaystyle 0.0933N_{\mathrm{QMC}}^{-0.9832}\)}%
\end{pgfscope}%
\end{pgfpicture}%
\makeatother%
\endgroup%

    \caption[Quasi-Monte-Carlo error, for $Q=u((1,1))$ and $k=10$.]{Quasi-Monte-Carlo error with increasing number of Quasi-Monte-Carlo points, for $Q=u((1,1))$ and $k=10$. \label{fig:qmctop_right10}}
  \end{figure}
\begin{figure}[h]
  \centering
%% Creator: Matplotlib, PGF backend
%%
%% To include the figure in your LaTeX document, write
%%   \input{<filename>.pgf}
%%
%% Make sure the required packages are loaded in your preamble
%%   \usepackage{pgf}
%%
%% Figures using additional raster images can only be included by \input if
%% they are in the same directory as the main LaTeX file. For loading figures
%% from other directories you can use the `import` package
%%   \usepackage{import}
%% and then include the figures with
%%   \import{<path to file>}{<filename>.pgf}
%%
%% Matplotlib used the following preamble
%%   \usepackage{fontspec}
%%   \setmainfont{DejaVuSerif.ttf}[Path=/home/owen/progs/firedrake-complex/firedrake/lib/python3.5/site-packages/matplotlib/mpl-data/fonts/ttf/]
%%   \setsansfont{DejaVuSans.ttf}[Path=/home/owen/progs/firedrake-complex/firedrake/lib/python3.5/site-packages/matplotlib/mpl-data/fonts/ttf/]
%%   \setmonofont{DejaVuSansMono.ttf}[Path=/home/owen/progs/firedrake-complex/firedrake/lib/python3.5/site-packages/matplotlib/mpl-data/fonts/ttf/]
%%
\begingroup%
\makeatletter%
\begin{pgfpicture}%
\pgfpathrectangle{\pgfpointorigin}{\pgfqpoint{5.000000in}{4.000000in}}%
\pgfusepath{use as bounding box, clip}%
\begin{pgfscope}%
\pgfsetbuttcap%
\pgfsetmiterjoin%
\definecolor{currentfill}{rgb}{1.000000,1.000000,1.000000}%
\pgfsetfillcolor{currentfill}%
\pgfsetlinewidth{0.000000pt}%
\definecolor{currentstroke}{rgb}{1.000000,1.000000,1.000000}%
\pgfsetstrokecolor{currentstroke}%
\pgfsetdash{}{0pt}%
\pgfpathmoveto{\pgfqpoint{0.000000in}{0.000000in}}%
\pgfpathlineto{\pgfqpoint{5.000000in}{0.000000in}}%
\pgfpathlineto{\pgfqpoint{5.000000in}{4.000000in}}%
\pgfpathlineto{\pgfqpoint{0.000000in}{4.000000in}}%
\pgfpathclose%
\pgfusepath{fill}%
\end{pgfscope}%
\begin{pgfscope}%
\pgfsetbuttcap%
\pgfsetmiterjoin%
\definecolor{currentfill}{rgb}{1.000000,1.000000,1.000000}%
\pgfsetfillcolor{currentfill}%
\pgfsetlinewidth{0.000000pt}%
\definecolor{currentstroke}{rgb}{0.000000,0.000000,0.000000}%
\pgfsetstrokecolor{currentstroke}%
\pgfsetstrokeopacity{0.000000}%
\pgfsetdash{}{0pt}%
\pgfpathmoveto{\pgfqpoint{0.625000in}{0.440000in}}%
\pgfpathlineto{\pgfqpoint{4.500000in}{0.440000in}}%
\pgfpathlineto{\pgfqpoint{4.500000in}{3.520000in}}%
\pgfpathlineto{\pgfqpoint{0.625000in}{3.520000in}}%
\pgfpathclose%
\pgfusepath{fill}%
\end{pgfscope}%
\begin{pgfscope}%
\pgfsetbuttcap%
\pgfsetroundjoin%
\definecolor{currentfill}{rgb}{0.000000,0.000000,0.000000}%
\pgfsetfillcolor{currentfill}%
\pgfsetlinewidth{0.803000pt}%
\definecolor{currentstroke}{rgb}{0.000000,0.000000,0.000000}%
\pgfsetstrokecolor{currentstroke}%
\pgfsetdash{}{0pt}%
\pgfsys@defobject{currentmarker}{\pgfqpoint{0.000000in}{-0.048611in}}{\pgfqpoint{0.000000in}{0.000000in}}{%
\pgfpathmoveto{\pgfqpoint{0.000000in}{0.000000in}}%
\pgfpathlineto{\pgfqpoint{0.000000in}{-0.048611in}}%
\pgfusepath{stroke,fill}%
}%
\begin{pgfscope}%
\pgfsys@transformshift{0.801136in}{0.440000in}%
\pgfsys@useobject{currentmarker}{}%
\end{pgfscope}%
\end{pgfscope}%
\begin{pgfscope}%
\definecolor{textcolor}{rgb}{0.000000,0.000000,0.000000}%
\pgfsetstrokecolor{textcolor}%
\pgfsetfillcolor{textcolor}%
\pgftext[x=0.801136in,y=0.342778in,,top]{\color{textcolor}\sffamily\fontsize{10.000000}{12.000000}\selectfont \(\displaystyle {10^{0}}\)}%
\end{pgfscope}%
\begin{pgfscope}%
\pgfsetbuttcap%
\pgfsetroundjoin%
\definecolor{currentfill}{rgb}{0.000000,0.000000,0.000000}%
\pgfsetfillcolor{currentfill}%
\pgfsetlinewidth{0.803000pt}%
\definecolor{currentstroke}{rgb}{0.000000,0.000000,0.000000}%
\pgfsetstrokecolor{currentstroke}%
\pgfsetdash{}{0pt}%
\pgfsys@defobject{currentmarker}{\pgfqpoint{0.000000in}{-0.048611in}}{\pgfqpoint{0.000000in}{0.000000in}}{%
\pgfpathmoveto{\pgfqpoint{0.000000in}{0.000000in}}%
\pgfpathlineto{\pgfqpoint{0.000000in}{-0.048611in}}%
\pgfusepath{stroke,fill}%
}%
\begin{pgfscope}%
\pgfsys@transformshift{1.864977in}{0.440000in}%
\pgfsys@useobject{currentmarker}{}%
\end{pgfscope}%
\end{pgfscope}%
\begin{pgfscope}%
\definecolor{textcolor}{rgb}{0.000000,0.000000,0.000000}%
\pgfsetstrokecolor{textcolor}%
\pgfsetfillcolor{textcolor}%
\pgftext[x=1.864977in,y=0.342778in,,top]{\color{textcolor}\sffamily\fontsize{10.000000}{12.000000}\selectfont \(\displaystyle {10^{1}}\)}%
\end{pgfscope}%
\begin{pgfscope}%
\pgfsetbuttcap%
\pgfsetroundjoin%
\definecolor{currentfill}{rgb}{0.000000,0.000000,0.000000}%
\pgfsetfillcolor{currentfill}%
\pgfsetlinewidth{0.803000pt}%
\definecolor{currentstroke}{rgb}{0.000000,0.000000,0.000000}%
\pgfsetstrokecolor{currentstroke}%
\pgfsetdash{}{0pt}%
\pgfsys@defobject{currentmarker}{\pgfqpoint{0.000000in}{-0.048611in}}{\pgfqpoint{0.000000in}{0.000000in}}{%
\pgfpathmoveto{\pgfqpoint{0.000000in}{0.000000in}}%
\pgfpathlineto{\pgfqpoint{0.000000in}{-0.048611in}}%
\pgfusepath{stroke,fill}%
}%
\begin{pgfscope}%
\pgfsys@transformshift{2.928818in}{0.440000in}%
\pgfsys@useobject{currentmarker}{}%
\end{pgfscope}%
\end{pgfscope}%
\begin{pgfscope}%
\definecolor{textcolor}{rgb}{0.000000,0.000000,0.000000}%
\pgfsetstrokecolor{textcolor}%
\pgfsetfillcolor{textcolor}%
\pgftext[x=2.928818in,y=0.342778in,,top]{\color{textcolor}\sffamily\fontsize{10.000000}{12.000000}\selectfont \(\displaystyle {10^{2}}\)}%
\end{pgfscope}%
\begin{pgfscope}%
\pgfsetbuttcap%
\pgfsetroundjoin%
\definecolor{currentfill}{rgb}{0.000000,0.000000,0.000000}%
\pgfsetfillcolor{currentfill}%
\pgfsetlinewidth{0.803000pt}%
\definecolor{currentstroke}{rgb}{0.000000,0.000000,0.000000}%
\pgfsetstrokecolor{currentstroke}%
\pgfsetdash{}{0pt}%
\pgfsys@defobject{currentmarker}{\pgfqpoint{0.000000in}{-0.048611in}}{\pgfqpoint{0.000000in}{0.000000in}}{%
\pgfpathmoveto{\pgfqpoint{0.000000in}{0.000000in}}%
\pgfpathlineto{\pgfqpoint{0.000000in}{-0.048611in}}%
\pgfusepath{stroke,fill}%
}%
\begin{pgfscope}%
\pgfsys@transformshift{3.992658in}{0.440000in}%
\pgfsys@useobject{currentmarker}{}%
\end{pgfscope}%
\end{pgfscope}%
\begin{pgfscope}%
\definecolor{textcolor}{rgb}{0.000000,0.000000,0.000000}%
\pgfsetstrokecolor{textcolor}%
\pgfsetfillcolor{textcolor}%
\pgftext[x=3.992658in,y=0.342778in,,top]{\color{textcolor}\sffamily\fontsize{10.000000}{12.000000}\selectfont \(\displaystyle {10^{3}}\)}%
\end{pgfscope}%
\begin{pgfscope}%
\pgfsetbuttcap%
\pgfsetroundjoin%
\definecolor{currentfill}{rgb}{0.000000,0.000000,0.000000}%
\pgfsetfillcolor{currentfill}%
\pgfsetlinewidth{0.602250pt}%
\definecolor{currentstroke}{rgb}{0.000000,0.000000,0.000000}%
\pgfsetstrokecolor{currentstroke}%
\pgfsetdash{}{0pt}%
\pgfsys@defobject{currentmarker}{\pgfqpoint{0.000000in}{-0.027778in}}{\pgfqpoint{0.000000in}{0.000000in}}{%
\pgfpathmoveto{\pgfqpoint{0.000000in}{0.000000in}}%
\pgfpathlineto{\pgfqpoint{0.000000in}{-0.027778in}}%
\pgfusepath{stroke,fill}%
}%
\begin{pgfscope}%
\pgfsys@transformshift{0.636345in}{0.440000in}%
\pgfsys@useobject{currentmarker}{}%
\end{pgfscope}%
\end{pgfscope}%
\begin{pgfscope}%
\pgfsetbuttcap%
\pgfsetroundjoin%
\definecolor{currentfill}{rgb}{0.000000,0.000000,0.000000}%
\pgfsetfillcolor{currentfill}%
\pgfsetlinewidth{0.602250pt}%
\definecolor{currentstroke}{rgb}{0.000000,0.000000,0.000000}%
\pgfsetstrokecolor{currentstroke}%
\pgfsetdash{}{0pt}%
\pgfsys@defobject{currentmarker}{\pgfqpoint{0.000000in}{-0.027778in}}{\pgfqpoint{0.000000in}{0.000000in}}{%
\pgfpathmoveto{\pgfqpoint{0.000000in}{0.000000in}}%
\pgfpathlineto{\pgfqpoint{0.000000in}{-0.027778in}}%
\pgfusepath{stroke,fill}%
}%
\begin{pgfscope}%
\pgfsys@transformshift{0.698040in}{0.440000in}%
\pgfsys@useobject{currentmarker}{}%
\end{pgfscope}%
\end{pgfscope}%
\begin{pgfscope}%
\pgfsetbuttcap%
\pgfsetroundjoin%
\definecolor{currentfill}{rgb}{0.000000,0.000000,0.000000}%
\pgfsetfillcolor{currentfill}%
\pgfsetlinewidth{0.602250pt}%
\definecolor{currentstroke}{rgb}{0.000000,0.000000,0.000000}%
\pgfsetstrokecolor{currentstroke}%
\pgfsetdash{}{0pt}%
\pgfsys@defobject{currentmarker}{\pgfqpoint{0.000000in}{-0.027778in}}{\pgfqpoint{0.000000in}{0.000000in}}{%
\pgfpathmoveto{\pgfqpoint{0.000000in}{0.000000in}}%
\pgfpathlineto{\pgfqpoint{0.000000in}{-0.027778in}}%
\pgfusepath{stroke,fill}%
}%
\begin{pgfscope}%
\pgfsys@transformshift{0.752458in}{0.440000in}%
\pgfsys@useobject{currentmarker}{}%
\end{pgfscope}%
\end{pgfscope}%
\begin{pgfscope}%
\pgfsetbuttcap%
\pgfsetroundjoin%
\definecolor{currentfill}{rgb}{0.000000,0.000000,0.000000}%
\pgfsetfillcolor{currentfill}%
\pgfsetlinewidth{0.602250pt}%
\definecolor{currentstroke}{rgb}{0.000000,0.000000,0.000000}%
\pgfsetstrokecolor{currentstroke}%
\pgfsetdash{}{0pt}%
\pgfsys@defobject{currentmarker}{\pgfqpoint{0.000000in}{-0.027778in}}{\pgfqpoint{0.000000in}{0.000000in}}{%
\pgfpathmoveto{\pgfqpoint{0.000000in}{0.000000in}}%
\pgfpathlineto{\pgfqpoint{0.000000in}{-0.027778in}}%
\pgfusepath{stroke,fill}%
}%
\begin{pgfscope}%
\pgfsys@transformshift{1.121384in}{0.440000in}%
\pgfsys@useobject{currentmarker}{}%
\end{pgfscope}%
\end{pgfscope}%
\begin{pgfscope}%
\pgfsetbuttcap%
\pgfsetroundjoin%
\definecolor{currentfill}{rgb}{0.000000,0.000000,0.000000}%
\pgfsetfillcolor{currentfill}%
\pgfsetlinewidth{0.602250pt}%
\definecolor{currentstroke}{rgb}{0.000000,0.000000,0.000000}%
\pgfsetstrokecolor{currentstroke}%
\pgfsetdash{}{0pt}%
\pgfsys@defobject{currentmarker}{\pgfqpoint{0.000000in}{-0.027778in}}{\pgfqpoint{0.000000in}{0.000000in}}{%
\pgfpathmoveto{\pgfqpoint{0.000000in}{0.000000in}}%
\pgfpathlineto{\pgfqpoint{0.000000in}{-0.027778in}}%
\pgfusepath{stroke,fill}%
}%
\begin{pgfscope}%
\pgfsys@transformshift{1.308717in}{0.440000in}%
\pgfsys@useobject{currentmarker}{}%
\end{pgfscope}%
\end{pgfscope}%
\begin{pgfscope}%
\pgfsetbuttcap%
\pgfsetroundjoin%
\definecolor{currentfill}{rgb}{0.000000,0.000000,0.000000}%
\pgfsetfillcolor{currentfill}%
\pgfsetlinewidth{0.602250pt}%
\definecolor{currentstroke}{rgb}{0.000000,0.000000,0.000000}%
\pgfsetstrokecolor{currentstroke}%
\pgfsetdash{}{0pt}%
\pgfsys@defobject{currentmarker}{\pgfqpoint{0.000000in}{-0.027778in}}{\pgfqpoint{0.000000in}{0.000000in}}{%
\pgfpathmoveto{\pgfqpoint{0.000000in}{0.000000in}}%
\pgfpathlineto{\pgfqpoint{0.000000in}{-0.027778in}}%
\pgfusepath{stroke,fill}%
}%
\begin{pgfscope}%
\pgfsys@transformshift{1.441632in}{0.440000in}%
\pgfsys@useobject{currentmarker}{}%
\end{pgfscope}%
\end{pgfscope}%
\begin{pgfscope}%
\pgfsetbuttcap%
\pgfsetroundjoin%
\definecolor{currentfill}{rgb}{0.000000,0.000000,0.000000}%
\pgfsetfillcolor{currentfill}%
\pgfsetlinewidth{0.602250pt}%
\definecolor{currentstroke}{rgb}{0.000000,0.000000,0.000000}%
\pgfsetstrokecolor{currentstroke}%
\pgfsetdash{}{0pt}%
\pgfsys@defobject{currentmarker}{\pgfqpoint{0.000000in}{-0.027778in}}{\pgfqpoint{0.000000in}{0.000000in}}{%
\pgfpathmoveto{\pgfqpoint{0.000000in}{0.000000in}}%
\pgfpathlineto{\pgfqpoint{0.000000in}{-0.027778in}}%
\pgfusepath{stroke,fill}%
}%
\begin{pgfscope}%
\pgfsys@transformshift{1.544729in}{0.440000in}%
\pgfsys@useobject{currentmarker}{}%
\end{pgfscope}%
\end{pgfscope}%
\begin{pgfscope}%
\pgfsetbuttcap%
\pgfsetroundjoin%
\definecolor{currentfill}{rgb}{0.000000,0.000000,0.000000}%
\pgfsetfillcolor{currentfill}%
\pgfsetlinewidth{0.602250pt}%
\definecolor{currentstroke}{rgb}{0.000000,0.000000,0.000000}%
\pgfsetstrokecolor{currentstroke}%
\pgfsetdash{}{0pt}%
\pgfsys@defobject{currentmarker}{\pgfqpoint{0.000000in}{-0.027778in}}{\pgfqpoint{0.000000in}{0.000000in}}{%
\pgfpathmoveto{\pgfqpoint{0.000000in}{0.000000in}}%
\pgfpathlineto{\pgfqpoint{0.000000in}{-0.027778in}}%
\pgfusepath{stroke,fill}%
}%
\begin{pgfscope}%
\pgfsys@transformshift{1.628965in}{0.440000in}%
\pgfsys@useobject{currentmarker}{}%
\end{pgfscope}%
\end{pgfscope}%
\begin{pgfscope}%
\pgfsetbuttcap%
\pgfsetroundjoin%
\definecolor{currentfill}{rgb}{0.000000,0.000000,0.000000}%
\pgfsetfillcolor{currentfill}%
\pgfsetlinewidth{0.602250pt}%
\definecolor{currentstroke}{rgb}{0.000000,0.000000,0.000000}%
\pgfsetstrokecolor{currentstroke}%
\pgfsetdash{}{0pt}%
\pgfsys@defobject{currentmarker}{\pgfqpoint{0.000000in}{-0.027778in}}{\pgfqpoint{0.000000in}{0.000000in}}{%
\pgfpathmoveto{\pgfqpoint{0.000000in}{0.000000in}}%
\pgfpathlineto{\pgfqpoint{0.000000in}{-0.027778in}}%
\pgfusepath{stroke,fill}%
}%
\begin{pgfscope}%
\pgfsys@transformshift{1.700186in}{0.440000in}%
\pgfsys@useobject{currentmarker}{}%
\end{pgfscope}%
\end{pgfscope}%
\begin{pgfscope}%
\pgfsetbuttcap%
\pgfsetroundjoin%
\definecolor{currentfill}{rgb}{0.000000,0.000000,0.000000}%
\pgfsetfillcolor{currentfill}%
\pgfsetlinewidth{0.602250pt}%
\definecolor{currentstroke}{rgb}{0.000000,0.000000,0.000000}%
\pgfsetstrokecolor{currentstroke}%
\pgfsetdash{}{0pt}%
\pgfsys@defobject{currentmarker}{\pgfqpoint{0.000000in}{-0.027778in}}{\pgfqpoint{0.000000in}{0.000000in}}{%
\pgfpathmoveto{\pgfqpoint{0.000000in}{0.000000in}}%
\pgfpathlineto{\pgfqpoint{0.000000in}{-0.027778in}}%
\pgfusepath{stroke,fill}%
}%
\begin{pgfscope}%
\pgfsys@transformshift{1.761880in}{0.440000in}%
\pgfsys@useobject{currentmarker}{}%
\end{pgfscope}%
\end{pgfscope}%
\begin{pgfscope}%
\pgfsetbuttcap%
\pgfsetroundjoin%
\definecolor{currentfill}{rgb}{0.000000,0.000000,0.000000}%
\pgfsetfillcolor{currentfill}%
\pgfsetlinewidth{0.602250pt}%
\definecolor{currentstroke}{rgb}{0.000000,0.000000,0.000000}%
\pgfsetstrokecolor{currentstroke}%
\pgfsetdash{}{0pt}%
\pgfsys@defobject{currentmarker}{\pgfqpoint{0.000000in}{-0.027778in}}{\pgfqpoint{0.000000in}{0.000000in}}{%
\pgfpathmoveto{\pgfqpoint{0.000000in}{0.000000in}}%
\pgfpathlineto{\pgfqpoint{0.000000in}{-0.027778in}}%
\pgfusepath{stroke,fill}%
}%
\begin{pgfscope}%
\pgfsys@transformshift{1.816298in}{0.440000in}%
\pgfsys@useobject{currentmarker}{}%
\end{pgfscope}%
\end{pgfscope}%
\begin{pgfscope}%
\pgfsetbuttcap%
\pgfsetroundjoin%
\definecolor{currentfill}{rgb}{0.000000,0.000000,0.000000}%
\pgfsetfillcolor{currentfill}%
\pgfsetlinewidth{0.602250pt}%
\definecolor{currentstroke}{rgb}{0.000000,0.000000,0.000000}%
\pgfsetstrokecolor{currentstroke}%
\pgfsetdash{}{0pt}%
\pgfsys@defobject{currentmarker}{\pgfqpoint{0.000000in}{-0.027778in}}{\pgfqpoint{0.000000in}{0.000000in}}{%
\pgfpathmoveto{\pgfqpoint{0.000000in}{0.000000in}}%
\pgfpathlineto{\pgfqpoint{0.000000in}{-0.027778in}}%
\pgfusepath{stroke,fill}%
}%
\begin{pgfscope}%
\pgfsys@transformshift{2.185225in}{0.440000in}%
\pgfsys@useobject{currentmarker}{}%
\end{pgfscope}%
\end{pgfscope}%
\begin{pgfscope}%
\pgfsetbuttcap%
\pgfsetroundjoin%
\definecolor{currentfill}{rgb}{0.000000,0.000000,0.000000}%
\pgfsetfillcolor{currentfill}%
\pgfsetlinewidth{0.602250pt}%
\definecolor{currentstroke}{rgb}{0.000000,0.000000,0.000000}%
\pgfsetstrokecolor{currentstroke}%
\pgfsetdash{}{0pt}%
\pgfsys@defobject{currentmarker}{\pgfqpoint{0.000000in}{-0.027778in}}{\pgfqpoint{0.000000in}{0.000000in}}{%
\pgfpathmoveto{\pgfqpoint{0.000000in}{0.000000in}}%
\pgfpathlineto{\pgfqpoint{0.000000in}{-0.027778in}}%
\pgfusepath{stroke,fill}%
}%
\begin{pgfscope}%
\pgfsys@transformshift{2.372558in}{0.440000in}%
\pgfsys@useobject{currentmarker}{}%
\end{pgfscope}%
\end{pgfscope}%
\begin{pgfscope}%
\pgfsetbuttcap%
\pgfsetroundjoin%
\definecolor{currentfill}{rgb}{0.000000,0.000000,0.000000}%
\pgfsetfillcolor{currentfill}%
\pgfsetlinewidth{0.602250pt}%
\definecolor{currentstroke}{rgb}{0.000000,0.000000,0.000000}%
\pgfsetstrokecolor{currentstroke}%
\pgfsetdash{}{0pt}%
\pgfsys@defobject{currentmarker}{\pgfqpoint{0.000000in}{-0.027778in}}{\pgfqpoint{0.000000in}{0.000000in}}{%
\pgfpathmoveto{\pgfqpoint{0.000000in}{0.000000in}}%
\pgfpathlineto{\pgfqpoint{0.000000in}{-0.027778in}}%
\pgfusepath{stroke,fill}%
}%
\begin{pgfscope}%
\pgfsys@transformshift{2.505473in}{0.440000in}%
\pgfsys@useobject{currentmarker}{}%
\end{pgfscope}%
\end{pgfscope}%
\begin{pgfscope}%
\pgfsetbuttcap%
\pgfsetroundjoin%
\definecolor{currentfill}{rgb}{0.000000,0.000000,0.000000}%
\pgfsetfillcolor{currentfill}%
\pgfsetlinewidth{0.602250pt}%
\definecolor{currentstroke}{rgb}{0.000000,0.000000,0.000000}%
\pgfsetstrokecolor{currentstroke}%
\pgfsetdash{}{0pt}%
\pgfsys@defobject{currentmarker}{\pgfqpoint{0.000000in}{-0.027778in}}{\pgfqpoint{0.000000in}{0.000000in}}{%
\pgfpathmoveto{\pgfqpoint{0.000000in}{0.000000in}}%
\pgfpathlineto{\pgfqpoint{0.000000in}{-0.027778in}}%
\pgfusepath{stroke,fill}%
}%
\begin{pgfscope}%
\pgfsys@transformshift{2.608570in}{0.440000in}%
\pgfsys@useobject{currentmarker}{}%
\end{pgfscope}%
\end{pgfscope}%
\begin{pgfscope}%
\pgfsetbuttcap%
\pgfsetroundjoin%
\definecolor{currentfill}{rgb}{0.000000,0.000000,0.000000}%
\pgfsetfillcolor{currentfill}%
\pgfsetlinewidth{0.602250pt}%
\definecolor{currentstroke}{rgb}{0.000000,0.000000,0.000000}%
\pgfsetstrokecolor{currentstroke}%
\pgfsetdash{}{0pt}%
\pgfsys@defobject{currentmarker}{\pgfqpoint{0.000000in}{-0.027778in}}{\pgfqpoint{0.000000in}{0.000000in}}{%
\pgfpathmoveto{\pgfqpoint{0.000000in}{0.000000in}}%
\pgfpathlineto{\pgfqpoint{0.000000in}{-0.027778in}}%
\pgfusepath{stroke,fill}%
}%
\begin{pgfscope}%
\pgfsys@transformshift{2.692806in}{0.440000in}%
\pgfsys@useobject{currentmarker}{}%
\end{pgfscope}%
\end{pgfscope}%
\begin{pgfscope}%
\pgfsetbuttcap%
\pgfsetroundjoin%
\definecolor{currentfill}{rgb}{0.000000,0.000000,0.000000}%
\pgfsetfillcolor{currentfill}%
\pgfsetlinewidth{0.602250pt}%
\definecolor{currentstroke}{rgb}{0.000000,0.000000,0.000000}%
\pgfsetstrokecolor{currentstroke}%
\pgfsetdash{}{0pt}%
\pgfsys@defobject{currentmarker}{\pgfqpoint{0.000000in}{-0.027778in}}{\pgfqpoint{0.000000in}{0.000000in}}{%
\pgfpathmoveto{\pgfqpoint{0.000000in}{0.000000in}}%
\pgfpathlineto{\pgfqpoint{0.000000in}{-0.027778in}}%
\pgfusepath{stroke,fill}%
}%
\begin{pgfscope}%
\pgfsys@transformshift{2.764027in}{0.440000in}%
\pgfsys@useobject{currentmarker}{}%
\end{pgfscope}%
\end{pgfscope}%
\begin{pgfscope}%
\pgfsetbuttcap%
\pgfsetroundjoin%
\definecolor{currentfill}{rgb}{0.000000,0.000000,0.000000}%
\pgfsetfillcolor{currentfill}%
\pgfsetlinewidth{0.602250pt}%
\definecolor{currentstroke}{rgb}{0.000000,0.000000,0.000000}%
\pgfsetstrokecolor{currentstroke}%
\pgfsetdash{}{0pt}%
\pgfsys@defobject{currentmarker}{\pgfqpoint{0.000000in}{-0.027778in}}{\pgfqpoint{0.000000in}{0.000000in}}{%
\pgfpathmoveto{\pgfqpoint{0.000000in}{0.000000in}}%
\pgfpathlineto{\pgfqpoint{0.000000in}{-0.027778in}}%
\pgfusepath{stroke,fill}%
}%
\begin{pgfscope}%
\pgfsys@transformshift{2.825721in}{0.440000in}%
\pgfsys@useobject{currentmarker}{}%
\end{pgfscope}%
\end{pgfscope}%
\begin{pgfscope}%
\pgfsetbuttcap%
\pgfsetroundjoin%
\definecolor{currentfill}{rgb}{0.000000,0.000000,0.000000}%
\pgfsetfillcolor{currentfill}%
\pgfsetlinewidth{0.602250pt}%
\definecolor{currentstroke}{rgb}{0.000000,0.000000,0.000000}%
\pgfsetstrokecolor{currentstroke}%
\pgfsetdash{}{0pt}%
\pgfsys@defobject{currentmarker}{\pgfqpoint{0.000000in}{-0.027778in}}{\pgfqpoint{0.000000in}{0.000000in}}{%
\pgfpathmoveto{\pgfqpoint{0.000000in}{0.000000in}}%
\pgfpathlineto{\pgfqpoint{0.000000in}{-0.027778in}}%
\pgfusepath{stroke,fill}%
}%
\begin{pgfscope}%
\pgfsys@transformshift{2.880139in}{0.440000in}%
\pgfsys@useobject{currentmarker}{}%
\end{pgfscope}%
\end{pgfscope}%
\begin{pgfscope}%
\pgfsetbuttcap%
\pgfsetroundjoin%
\definecolor{currentfill}{rgb}{0.000000,0.000000,0.000000}%
\pgfsetfillcolor{currentfill}%
\pgfsetlinewidth{0.602250pt}%
\definecolor{currentstroke}{rgb}{0.000000,0.000000,0.000000}%
\pgfsetstrokecolor{currentstroke}%
\pgfsetdash{}{0pt}%
\pgfsys@defobject{currentmarker}{\pgfqpoint{0.000000in}{-0.027778in}}{\pgfqpoint{0.000000in}{0.000000in}}{%
\pgfpathmoveto{\pgfqpoint{0.000000in}{0.000000in}}%
\pgfpathlineto{\pgfqpoint{0.000000in}{-0.027778in}}%
\pgfusepath{stroke,fill}%
}%
\begin{pgfscope}%
\pgfsys@transformshift{3.249066in}{0.440000in}%
\pgfsys@useobject{currentmarker}{}%
\end{pgfscope}%
\end{pgfscope}%
\begin{pgfscope}%
\pgfsetbuttcap%
\pgfsetroundjoin%
\definecolor{currentfill}{rgb}{0.000000,0.000000,0.000000}%
\pgfsetfillcolor{currentfill}%
\pgfsetlinewidth{0.602250pt}%
\definecolor{currentstroke}{rgb}{0.000000,0.000000,0.000000}%
\pgfsetstrokecolor{currentstroke}%
\pgfsetdash{}{0pt}%
\pgfsys@defobject{currentmarker}{\pgfqpoint{0.000000in}{-0.027778in}}{\pgfqpoint{0.000000in}{0.000000in}}{%
\pgfpathmoveto{\pgfqpoint{0.000000in}{0.000000in}}%
\pgfpathlineto{\pgfqpoint{0.000000in}{-0.027778in}}%
\pgfusepath{stroke,fill}%
}%
\begin{pgfscope}%
\pgfsys@transformshift{3.436399in}{0.440000in}%
\pgfsys@useobject{currentmarker}{}%
\end{pgfscope}%
\end{pgfscope}%
\begin{pgfscope}%
\pgfsetbuttcap%
\pgfsetroundjoin%
\definecolor{currentfill}{rgb}{0.000000,0.000000,0.000000}%
\pgfsetfillcolor{currentfill}%
\pgfsetlinewidth{0.602250pt}%
\definecolor{currentstroke}{rgb}{0.000000,0.000000,0.000000}%
\pgfsetstrokecolor{currentstroke}%
\pgfsetdash{}{0pt}%
\pgfsys@defobject{currentmarker}{\pgfqpoint{0.000000in}{-0.027778in}}{\pgfqpoint{0.000000in}{0.000000in}}{%
\pgfpathmoveto{\pgfqpoint{0.000000in}{0.000000in}}%
\pgfpathlineto{\pgfqpoint{0.000000in}{-0.027778in}}%
\pgfusepath{stroke,fill}%
}%
\begin{pgfscope}%
\pgfsys@transformshift{3.569313in}{0.440000in}%
\pgfsys@useobject{currentmarker}{}%
\end{pgfscope}%
\end{pgfscope}%
\begin{pgfscope}%
\pgfsetbuttcap%
\pgfsetroundjoin%
\definecolor{currentfill}{rgb}{0.000000,0.000000,0.000000}%
\pgfsetfillcolor{currentfill}%
\pgfsetlinewidth{0.602250pt}%
\definecolor{currentstroke}{rgb}{0.000000,0.000000,0.000000}%
\pgfsetstrokecolor{currentstroke}%
\pgfsetdash{}{0pt}%
\pgfsys@defobject{currentmarker}{\pgfqpoint{0.000000in}{-0.027778in}}{\pgfqpoint{0.000000in}{0.000000in}}{%
\pgfpathmoveto{\pgfqpoint{0.000000in}{0.000000in}}%
\pgfpathlineto{\pgfqpoint{0.000000in}{-0.027778in}}%
\pgfusepath{stroke,fill}%
}%
\begin{pgfscope}%
\pgfsys@transformshift{3.672410in}{0.440000in}%
\pgfsys@useobject{currentmarker}{}%
\end{pgfscope}%
\end{pgfscope}%
\begin{pgfscope}%
\pgfsetbuttcap%
\pgfsetroundjoin%
\definecolor{currentfill}{rgb}{0.000000,0.000000,0.000000}%
\pgfsetfillcolor{currentfill}%
\pgfsetlinewidth{0.602250pt}%
\definecolor{currentstroke}{rgb}{0.000000,0.000000,0.000000}%
\pgfsetstrokecolor{currentstroke}%
\pgfsetdash{}{0pt}%
\pgfsys@defobject{currentmarker}{\pgfqpoint{0.000000in}{-0.027778in}}{\pgfqpoint{0.000000in}{0.000000in}}{%
\pgfpathmoveto{\pgfqpoint{0.000000in}{0.000000in}}%
\pgfpathlineto{\pgfqpoint{0.000000in}{-0.027778in}}%
\pgfusepath{stroke,fill}%
}%
\begin{pgfscope}%
\pgfsys@transformshift{3.756646in}{0.440000in}%
\pgfsys@useobject{currentmarker}{}%
\end{pgfscope}%
\end{pgfscope}%
\begin{pgfscope}%
\pgfsetbuttcap%
\pgfsetroundjoin%
\definecolor{currentfill}{rgb}{0.000000,0.000000,0.000000}%
\pgfsetfillcolor{currentfill}%
\pgfsetlinewidth{0.602250pt}%
\definecolor{currentstroke}{rgb}{0.000000,0.000000,0.000000}%
\pgfsetstrokecolor{currentstroke}%
\pgfsetdash{}{0pt}%
\pgfsys@defobject{currentmarker}{\pgfqpoint{0.000000in}{-0.027778in}}{\pgfqpoint{0.000000in}{0.000000in}}{%
\pgfpathmoveto{\pgfqpoint{0.000000in}{0.000000in}}%
\pgfpathlineto{\pgfqpoint{0.000000in}{-0.027778in}}%
\pgfusepath{stroke,fill}%
}%
\begin{pgfscope}%
\pgfsys@transformshift{3.827867in}{0.440000in}%
\pgfsys@useobject{currentmarker}{}%
\end{pgfscope}%
\end{pgfscope}%
\begin{pgfscope}%
\pgfsetbuttcap%
\pgfsetroundjoin%
\definecolor{currentfill}{rgb}{0.000000,0.000000,0.000000}%
\pgfsetfillcolor{currentfill}%
\pgfsetlinewidth{0.602250pt}%
\definecolor{currentstroke}{rgb}{0.000000,0.000000,0.000000}%
\pgfsetstrokecolor{currentstroke}%
\pgfsetdash{}{0pt}%
\pgfsys@defobject{currentmarker}{\pgfqpoint{0.000000in}{-0.027778in}}{\pgfqpoint{0.000000in}{0.000000in}}{%
\pgfpathmoveto{\pgfqpoint{0.000000in}{0.000000in}}%
\pgfpathlineto{\pgfqpoint{0.000000in}{-0.027778in}}%
\pgfusepath{stroke,fill}%
}%
\begin{pgfscope}%
\pgfsys@transformshift{3.889561in}{0.440000in}%
\pgfsys@useobject{currentmarker}{}%
\end{pgfscope}%
\end{pgfscope}%
\begin{pgfscope}%
\pgfsetbuttcap%
\pgfsetroundjoin%
\definecolor{currentfill}{rgb}{0.000000,0.000000,0.000000}%
\pgfsetfillcolor{currentfill}%
\pgfsetlinewidth{0.602250pt}%
\definecolor{currentstroke}{rgb}{0.000000,0.000000,0.000000}%
\pgfsetstrokecolor{currentstroke}%
\pgfsetdash{}{0pt}%
\pgfsys@defobject{currentmarker}{\pgfqpoint{0.000000in}{-0.027778in}}{\pgfqpoint{0.000000in}{0.000000in}}{%
\pgfpathmoveto{\pgfqpoint{0.000000in}{0.000000in}}%
\pgfpathlineto{\pgfqpoint{0.000000in}{-0.027778in}}%
\pgfusepath{stroke,fill}%
}%
\begin{pgfscope}%
\pgfsys@transformshift{3.943980in}{0.440000in}%
\pgfsys@useobject{currentmarker}{}%
\end{pgfscope}%
\end{pgfscope}%
\begin{pgfscope}%
\pgfsetbuttcap%
\pgfsetroundjoin%
\definecolor{currentfill}{rgb}{0.000000,0.000000,0.000000}%
\pgfsetfillcolor{currentfill}%
\pgfsetlinewidth{0.602250pt}%
\definecolor{currentstroke}{rgb}{0.000000,0.000000,0.000000}%
\pgfsetstrokecolor{currentstroke}%
\pgfsetdash{}{0pt}%
\pgfsys@defobject{currentmarker}{\pgfqpoint{0.000000in}{-0.027778in}}{\pgfqpoint{0.000000in}{0.000000in}}{%
\pgfpathmoveto{\pgfqpoint{0.000000in}{0.000000in}}%
\pgfpathlineto{\pgfqpoint{0.000000in}{-0.027778in}}%
\pgfusepath{stroke,fill}%
}%
\begin{pgfscope}%
\pgfsys@transformshift{4.312906in}{0.440000in}%
\pgfsys@useobject{currentmarker}{}%
\end{pgfscope}%
\end{pgfscope}%
\begin{pgfscope}%
\pgfsetbuttcap%
\pgfsetroundjoin%
\definecolor{currentfill}{rgb}{0.000000,0.000000,0.000000}%
\pgfsetfillcolor{currentfill}%
\pgfsetlinewidth{0.602250pt}%
\definecolor{currentstroke}{rgb}{0.000000,0.000000,0.000000}%
\pgfsetstrokecolor{currentstroke}%
\pgfsetdash{}{0pt}%
\pgfsys@defobject{currentmarker}{\pgfqpoint{0.000000in}{-0.027778in}}{\pgfqpoint{0.000000in}{0.000000in}}{%
\pgfpathmoveto{\pgfqpoint{0.000000in}{0.000000in}}%
\pgfpathlineto{\pgfqpoint{0.000000in}{-0.027778in}}%
\pgfusepath{stroke,fill}%
}%
\begin{pgfscope}%
\pgfsys@transformshift{4.500239in}{0.440000in}%
\pgfsys@useobject{currentmarker}{}%
\end{pgfscope}%
\end{pgfscope}%
\begin{pgfscope}%
\definecolor{textcolor}{rgb}{0.000000,0.000000,0.000000}%
\pgfsetstrokecolor{textcolor}%
\pgfsetfillcolor{textcolor}%
\pgftext[x=2.562500in,y=0.152809in,,top]{\color{textcolor}\sffamily\fontsize{10.000000}{12.000000}\selectfont \(\displaystyle N_\mathrm{QMC}\)}%
\end{pgfscope}%
\begin{pgfscope}%
\pgfsetbuttcap%
\pgfsetroundjoin%
\definecolor{currentfill}{rgb}{0.000000,0.000000,0.000000}%
\pgfsetfillcolor{currentfill}%
\pgfsetlinewidth{0.803000pt}%
\definecolor{currentstroke}{rgb}{0.000000,0.000000,0.000000}%
\pgfsetstrokecolor{currentstroke}%
\pgfsetdash{}{0pt}%
\pgfsys@defobject{currentmarker}{\pgfqpoint{-0.048611in}{0.000000in}}{\pgfqpoint{0.000000in}{0.000000in}}{%
\pgfpathmoveto{\pgfqpoint{0.000000in}{0.000000in}}%
\pgfpathlineto{\pgfqpoint{-0.048611in}{0.000000in}}%
\pgfusepath{stroke,fill}%
}%
\begin{pgfscope}%
\pgfsys@transformshift{0.625000in}{0.802462in}%
\pgfsys@useobject{currentmarker}{}%
\end{pgfscope}%
\end{pgfscope}%
\begin{pgfscope}%
\definecolor{textcolor}{rgb}{0.000000,0.000000,0.000000}%
\pgfsetstrokecolor{textcolor}%
\pgfsetfillcolor{textcolor}%
\pgftext[x=0.239775in,y=0.749700in,left,base]{\color{textcolor}\sffamily\fontsize{10.000000}{12.000000}\selectfont \(\displaystyle {10^{-4}}\)}%
\end{pgfscope}%
\begin{pgfscope}%
\pgfsetbuttcap%
\pgfsetroundjoin%
\definecolor{currentfill}{rgb}{0.000000,0.000000,0.000000}%
\pgfsetfillcolor{currentfill}%
\pgfsetlinewidth{0.803000pt}%
\definecolor{currentstroke}{rgb}{0.000000,0.000000,0.000000}%
\pgfsetstrokecolor{currentstroke}%
\pgfsetdash{}{0pt}%
\pgfsys@defobject{currentmarker}{\pgfqpoint{-0.048611in}{0.000000in}}{\pgfqpoint{0.000000in}{0.000000in}}{%
\pgfpathmoveto{\pgfqpoint{0.000000in}{0.000000in}}%
\pgfpathlineto{\pgfqpoint{-0.048611in}{0.000000in}}%
\pgfusepath{stroke,fill}%
}%
\begin{pgfscope}%
\pgfsys@transformshift{0.625000in}{1.726819in}%
\pgfsys@useobject{currentmarker}{}%
\end{pgfscope}%
\end{pgfscope}%
\begin{pgfscope}%
\definecolor{textcolor}{rgb}{0.000000,0.000000,0.000000}%
\pgfsetstrokecolor{textcolor}%
\pgfsetfillcolor{textcolor}%
\pgftext[x=0.239775in,y=1.674057in,left,base]{\color{textcolor}\sffamily\fontsize{10.000000}{12.000000}\selectfont \(\displaystyle {10^{-3}}\)}%
\end{pgfscope}%
\begin{pgfscope}%
\pgfsetbuttcap%
\pgfsetroundjoin%
\definecolor{currentfill}{rgb}{0.000000,0.000000,0.000000}%
\pgfsetfillcolor{currentfill}%
\pgfsetlinewidth{0.803000pt}%
\definecolor{currentstroke}{rgb}{0.000000,0.000000,0.000000}%
\pgfsetstrokecolor{currentstroke}%
\pgfsetdash{}{0pt}%
\pgfsys@defobject{currentmarker}{\pgfqpoint{-0.048611in}{0.000000in}}{\pgfqpoint{0.000000in}{0.000000in}}{%
\pgfpathmoveto{\pgfqpoint{0.000000in}{0.000000in}}%
\pgfpathlineto{\pgfqpoint{-0.048611in}{0.000000in}}%
\pgfusepath{stroke,fill}%
}%
\begin{pgfscope}%
\pgfsys@transformshift{0.625000in}{2.651175in}%
\pgfsys@useobject{currentmarker}{}%
\end{pgfscope}%
\end{pgfscope}%
\begin{pgfscope}%
\definecolor{textcolor}{rgb}{0.000000,0.000000,0.000000}%
\pgfsetstrokecolor{textcolor}%
\pgfsetfillcolor{textcolor}%
\pgftext[x=0.239775in,y=2.598414in,left,base]{\color{textcolor}\sffamily\fontsize{10.000000}{12.000000}\selectfont \(\displaystyle {10^{-2}}\)}%
\end{pgfscope}%
\begin{pgfscope}%
\pgfsetbuttcap%
\pgfsetroundjoin%
\definecolor{currentfill}{rgb}{0.000000,0.000000,0.000000}%
\pgfsetfillcolor{currentfill}%
\pgfsetlinewidth{0.602250pt}%
\definecolor{currentstroke}{rgb}{0.000000,0.000000,0.000000}%
\pgfsetstrokecolor{currentstroke}%
\pgfsetdash{}{0pt}%
\pgfsys@defobject{currentmarker}{\pgfqpoint{-0.027778in}{0.000000in}}{\pgfqpoint{0.000000in}{0.000000in}}{%
\pgfpathmoveto{\pgfqpoint{0.000000in}{0.000000in}}%
\pgfpathlineto{\pgfqpoint{-0.027778in}{0.000000in}}%
\pgfusepath{stroke,fill}%
}%
\begin{pgfscope}%
\pgfsys@transformshift{0.625000in}{0.524203in}%
\pgfsys@useobject{currentmarker}{}%
\end{pgfscope}%
\end{pgfscope}%
\begin{pgfscope}%
\pgfsetbuttcap%
\pgfsetroundjoin%
\definecolor{currentfill}{rgb}{0.000000,0.000000,0.000000}%
\pgfsetfillcolor{currentfill}%
\pgfsetlinewidth{0.602250pt}%
\definecolor{currentstroke}{rgb}{0.000000,0.000000,0.000000}%
\pgfsetstrokecolor{currentstroke}%
\pgfsetdash{}{0pt}%
\pgfsys@defobject{currentmarker}{\pgfqpoint{-0.027778in}{0.000000in}}{\pgfqpoint{0.000000in}{0.000000in}}{%
\pgfpathmoveto{\pgfqpoint{0.000000in}{0.000000in}}%
\pgfpathlineto{\pgfqpoint{-0.027778in}{0.000000in}}%
\pgfusepath{stroke,fill}%
}%
\begin{pgfscope}%
\pgfsys@transformshift{0.625000in}{0.597395in}%
\pgfsys@useobject{currentmarker}{}%
\end{pgfscope}%
\end{pgfscope}%
\begin{pgfscope}%
\pgfsetbuttcap%
\pgfsetroundjoin%
\definecolor{currentfill}{rgb}{0.000000,0.000000,0.000000}%
\pgfsetfillcolor{currentfill}%
\pgfsetlinewidth{0.602250pt}%
\definecolor{currentstroke}{rgb}{0.000000,0.000000,0.000000}%
\pgfsetstrokecolor{currentstroke}%
\pgfsetdash{}{0pt}%
\pgfsys@defobject{currentmarker}{\pgfqpoint{-0.027778in}{0.000000in}}{\pgfqpoint{0.000000in}{0.000000in}}{%
\pgfpathmoveto{\pgfqpoint{0.000000in}{0.000000in}}%
\pgfpathlineto{\pgfqpoint{-0.027778in}{0.000000in}}%
\pgfusepath{stroke,fill}%
}%
\begin{pgfscope}%
\pgfsys@transformshift{0.625000in}{0.659277in}%
\pgfsys@useobject{currentmarker}{}%
\end{pgfscope}%
\end{pgfscope}%
\begin{pgfscope}%
\pgfsetbuttcap%
\pgfsetroundjoin%
\definecolor{currentfill}{rgb}{0.000000,0.000000,0.000000}%
\pgfsetfillcolor{currentfill}%
\pgfsetlinewidth{0.602250pt}%
\definecolor{currentstroke}{rgb}{0.000000,0.000000,0.000000}%
\pgfsetstrokecolor{currentstroke}%
\pgfsetdash{}{0pt}%
\pgfsys@defobject{currentmarker}{\pgfqpoint{-0.027778in}{0.000000in}}{\pgfqpoint{0.000000in}{0.000000in}}{%
\pgfpathmoveto{\pgfqpoint{0.000000in}{0.000000in}}%
\pgfpathlineto{\pgfqpoint{-0.027778in}{0.000000in}}%
\pgfusepath{stroke,fill}%
}%
\begin{pgfscope}%
\pgfsys@transformshift{0.625000in}{0.712883in}%
\pgfsys@useobject{currentmarker}{}%
\end{pgfscope}%
\end{pgfscope}%
\begin{pgfscope}%
\pgfsetbuttcap%
\pgfsetroundjoin%
\definecolor{currentfill}{rgb}{0.000000,0.000000,0.000000}%
\pgfsetfillcolor{currentfill}%
\pgfsetlinewidth{0.602250pt}%
\definecolor{currentstroke}{rgb}{0.000000,0.000000,0.000000}%
\pgfsetstrokecolor{currentstroke}%
\pgfsetdash{}{0pt}%
\pgfsys@defobject{currentmarker}{\pgfqpoint{-0.027778in}{0.000000in}}{\pgfqpoint{0.000000in}{0.000000in}}{%
\pgfpathmoveto{\pgfqpoint{0.000000in}{0.000000in}}%
\pgfpathlineto{\pgfqpoint{-0.027778in}{0.000000in}}%
\pgfusepath{stroke,fill}%
}%
\begin{pgfscope}%
\pgfsys@transformshift{0.625000in}{0.760166in}%
\pgfsys@useobject{currentmarker}{}%
\end{pgfscope}%
\end{pgfscope}%
\begin{pgfscope}%
\pgfsetbuttcap%
\pgfsetroundjoin%
\definecolor{currentfill}{rgb}{0.000000,0.000000,0.000000}%
\pgfsetfillcolor{currentfill}%
\pgfsetlinewidth{0.602250pt}%
\definecolor{currentstroke}{rgb}{0.000000,0.000000,0.000000}%
\pgfsetstrokecolor{currentstroke}%
\pgfsetdash{}{0pt}%
\pgfsys@defobject{currentmarker}{\pgfqpoint{-0.027778in}{0.000000in}}{\pgfqpoint{0.000000in}{0.000000in}}{%
\pgfpathmoveto{\pgfqpoint{0.000000in}{0.000000in}}%
\pgfpathlineto{\pgfqpoint{-0.027778in}{0.000000in}}%
\pgfusepath{stroke,fill}%
}%
\begin{pgfscope}%
\pgfsys@transformshift{0.625000in}{1.080721in}%
\pgfsys@useobject{currentmarker}{}%
\end{pgfscope}%
\end{pgfscope}%
\begin{pgfscope}%
\pgfsetbuttcap%
\pgfsetroundjoin%
\definecolor{currentfill}{rgb}{0.000000,0.000000,0.000000}%
\pgfsetfillcolor{currentfill}%
\pgfsetlinewidth{0.602250pt}%
\definecolor{currentstroke}{rgb}{0.000000,0.000000,0.000000}%
\pgfsetstrokecolor{currentstroke}%
\pgfsetdash{}{0pt}%
\pgfsys@defobject{currentmarker}{\pgfqpoint{-0.027778in}{0.000000in}}{\pgfqpoint{0.000000in}{0.000000in}}{%
\pgfpathmoveto{\pgfqpoint{0.000000in}{0.000000in}}%
\pgfpathlineto{\pgfqpoint{-0.027778in}{0.000000in}}%
\pgfusepath{stroke,fill}%
}%
\begin{pgfscope}%
\pgfsys@transformshift{0.625000in}{1.243492in}%
\pgfsys@useobject{currentmarker}{}%
\end{pgfscope}%
\end{pgfscope}%
\begin{pgfscope}%
\pgfsetbuttcap%
\pgfsetroundjoin%
\definecolor{currentfill}{rgb}{0.000000,0.000000,0.000000}%
\pgfsetfillcolor{currentfill}%
\pgfsetlinewidth{0.602250pt}%
\definecolor{currentstroke}{rgb}{0.000000,0.000000,0.000000}%
\pgfsetstrokecolor{currentstroke}%
\pgfsetdash{}{0pt}%
\pgfsys@defobject{currentmarker}{\pgfqpoint{-0.027778in}{0.000000in}}{\pgfqpoint{0.000000in}{0.000000in}}{%
\pgfpathmoveto{\pgfqpoint{0.000000in}{0.000000in}}%
\pgfpathlineto{\pgfqpoint{-0.027778in}{0.000000in}}%
\pgfusepath{stroke,fill}%
}%
\begin{pgfscope}%
\pgfsys@transformshift{0.625000in}{1.358980in}%
\pgfsys@useobject{currentmarker}{}%
\end{pgfscope}%
\end{pgfscope}%
\begin{pgfscope}%
\pgfsetbuttcap%
\pgfsetroundjoin%
\definecolor{currentfill}{rgb}{0.000000,0.000000,0.000000}%
\pgfsetfillcolor{currentfill}%
\pgfsetlinewidth{0.602250pt}%
\definecolor{currentstroke}{rgb}{0.000000,0.000000,0.000000}%
\pgfsetstrokecolor{currentstroke}%
\pgfsetdash{}{0pt}%
\pgfsys@defobject{currentmarker}{\pgfqpoint{-0.027778in}{0.000000in}}{\pgfqpoint{0.000000in}{0.000000in}}{%
\pgfpathmoveto{\pgfqpoint{0.000000in}{0.000000in}}%
\pgfpathlineto{\pgfqpoint{-0.027778in}{0.000000in}}%
\pgfusepath{stroke,fill}%
}%
\begin{pgfscope}%
\pgfsys@transformshift{0.625000in}{1.448560in}%
\pgfsys@useobject{currentmarker}{}%
\end{pgfscope}%
\end{pgfscope}%
\begin{pgfscope}%
\pgfsetbuttcap%
\pgfsetroundjoin%
\definecolor{currentfill}{rgb}{0.000000,0.000000,0.000000}%
\pgfsetfillcolor{currentfill}%
\pgfsetlinewidth{0.602250pt}%
\definecolor{currentstroke}{rgb}{0.000000,0.000000,0.000000}%
\pgfsetstrokecolor{currentstroke}%
\pgfsetdash{}{0pt}%
\pgfsys@defobject{currentmarker}{\pgfqpoint{-0.027778in}{0.000000in}}{\pgfqpoint{0.000000in}{0.000000in}}{%
\pgfpathmoveto{\pgfqpoint{0.000000in}{0.000000in}}%
\pgfpathlineto{\pgfqpoint{-0.027778in}{0.000000in}}%
\pgfusepath{stroke,fill}%
}%
\begin{pgfscope}%
\pgfsys@transformshift{0.625000in}{1.521751in}%
\pgfsys@useobject{currentmarker}{}%
\end{pgfscope}%
\end{pgfscope}%
\begin{pgfscope}%
\pgfsetbuttcap%
\pgfsetroundjoin%
\definecolor{currentfill}{rgb}{0.000000,0.000000,0.000000}%
\pgfsetfillcolor{currentfill}%
\pgfsetlinewidth{0.602250pt}%
\definecolor{currentstroke}{rgb}{0.000000,0.000000,0.000000}%
\pgfsetstrokecolor{currentstroke}%
\pgfsetdash{}{0pt}%
\pgfsys@defobject{currentmarker}{\pgfqpoint{-0.027778in}{0.000000in}}{\pgfqpoint{0.000000in}{0.000000in}}{%
\pgfpathmoveto{\pgfqpoint{0.000000in}{0.000000in}}%
\pgfpathlineto{\pgfqpoint{-0.027778in}{0.000000in}}%
\pgfusepath{stroke,fill}%
}%
\begin{pgfscope}%
\pgfsys@transformshift{0.625000in}{1.583634in}%
\pgfsys@useobject{currentmarker}{}%
\end{pgfscope}%
\end{pgfscope}%
\begin{pgfscope}%
\pgfsetbuttcap%
\pgfsetroundjoin%
\definecolor{currentfill}{rgb}{0.000000,0.000000,0.000000}%
\pgfsetfillcolor{currentfill}%
\pgfsetlinewidth{0.602250pt}%
\definecolor{currentstroke}{rgb}{0.000000,0.000000,0.000000}%
\pgfsetstrokecolor{currentstroke}%
\pgfsetdash{}{0pt}%
\pgfsys@defobject{currentmarker}{\pgfqpoint{-0.027778in}{0.000000in}}{\pgfqpoint{0.000000in}{0.000000in}}{%
\pgfpathmoveto{\pgfqpoint{0.000000in}{0.000000in}}%
\pgfpathlineto{\pgfqpoint{-0.027778in}{0.000000in}}%
\pgfusepath{stroke,fill}%
}%
\begin{pgfscope}%
\pgfsys@transformshift{0.625000in}{1.637239in}%
\pgfsys@useobject{currentmarker}{}%
\end{pgfscope}%
\end{pgfscope}%
\begin{pgfscope}%
\pgfsetbuttcap%
\pgfsetroundjoin%
\definecolor{currentfill}{rgb}{0.000000,0.000000,0.000000}%
\pgfsetfillcolor{currentfill}%
\pgfsetlinewidth{0.602250pt}%
\definecolor{currentstroke}{rgb}{0.000000,0.000000,0.000000}%
\pgfsetstrokecolor{currentstroke}%
\pgfsetdash{}{0pt}%
\pgfsys@defobject{currentmarker}{\pgfqpoint{-0.027778in}{0.000000in}}{\pgfqpoint{0.000000in}{0.000000in}}{%
\pgfpathmoveto{\pgfqpoint{0.000000in}{0.000000in}}%
\pgfpathlineto{\pgfqpoint{-0.027778in}{0.000000in}}%
\pgfusepath{stroke,fill}%
}%
\begin{pgfscope}%
\pgfsys@transformshift{0.625000in}{1.684522in}%
\pgfsys@useobject{currentmarker}{}%
\end{pgfscope}%
\end{pgfscope}%
\begin{pgfscope}%
\pgfsetbuttcap%
\pgfsetroundjoin%
\definecolor{currentfill}{rgb}{0.000000,0.000000,0.000000}%
\pgfsetfillcolor{currentfill}%
\pgfsetlinewidth{0.602250pt}%
\definecolor{currentstroke}{rgb}{0.000000,0.000000,0.000000}%
\pgfsetstrokecolor{currentstroke}%
\pgfsetdash{}{0pt}%
\pgfsys@defobject{currentmarker}{\pgfqpoint{-0.027778in}{0.000000in}}{\pgfqpoint{0.000000in}{0.000000in}}{%
\pgfpathmoveto{\pgfqpoint{0.000000in}{0.000000in}}%
\pgfpathlineto{\pgfqpoint{-0.027778in}{0.000000in}}%
\pgfusepath{stroke,fill}%
}%
\begin{pgfscope}%
\pgfsys@transformshift{0.625000in}{2.005078in}%
\pgfsys@useobject{currentmarker}{}%
\end{pgfscope}%
\end{pgfscope}%
\begin{pgfscope}%
\pgfsetbuttcap%
\pgfsetroundjoin%
\definecolor{currentfill}{rgb}{0.000000,0.000000,0.000000}%
\pgfsetfillcolor{currentfill}%
\pgfsetlinewidth{0.602250pt}%
\definecolor{currentstroke}{rgb}{0.000000,0.000000,0.000000}%
\pgfsetstrokecolor{currentstroke}%
\pgfsetdash{}{0pt}%
\pgfsys@defobject{currentmarker}{\pgfqpoint{-0.027778in}{0.000000in}}{\pgfqpoint{0.000000in}{0.000000in}}{%
\pgfpathmoveto{\pgfqpoint{0.000000in}{0.000000in}}%
\pgfpathlineto{\pgfqpoint{-0.027778in}{0.000000in}}%
\pgfusepath{stroke,fill}%
}%
\begin{pgfscope}%
\pgfsys@transformshift{0.625000in}{2.167849in}%
\pgfsys@useobject{currentmarker}{}%
\end{pgfscope}%
\end{pgfscope}%
\begin{pgfscope}%
\pgfsetbuttcap%
\pgfsetroundjoin%
\definecolor{currentfill}{rgb}{0.000000,0.000000,0.000000}%
\pgfsetfillcolor{currentfill}%
\pgfsetlinewidth{0.602250pt}%
\definecolor{currentstroke}{rgb}{0.000000,0.000000,0.000000}%
\pgfsetstrokecolor{currentstroke}%
\pgfsetdash{}{0pt}%
\pgfsys@defobject{currentmarker}{\pgfqpoint{-0.027778in}{0.000000in}}{\pgfqpoint{0.000000in}{0.000000in}}{%
\pgfpathmoveto{\pgfqpoint{0.000000in}{0.000000in}}%
\pgfpathlineto{\pgfqpoint{-0.027778in}{0.000000in}}%
\pgfusepath{stroke,fill}%
}%
\begin{pgfscope}%
\pgfsys@transformshift{0.625000in}{2.283337in}%
\pgfsys@useobject{currentmarker}{}%
\end{pgfscope}%
\end{pgfscope}%
\begin{pgfscope}%
\pgfsetbuttcap%
\pgfsetroundjoin%
\definecolor{currentfill}{rgb}{0.000000,0.000000,0.000000}%
\pgfsetfillcolor{currentfill}%
\pgfsetlinewidth{0.602250pt}%
\definecolor{currentstroke}{rgb}{0.000000,0.000000,0.000000}%
\pgfsetstrokecolor{currentstroke}%
\pgfsetdash{}{0pt}%
\pgfsys@defobject{currentmarker}{\pgfqpoint{-0.027778in}{0.000000in}}{\pgfqpoint{0.000000in}{0.000000in}}{%
\pgfpathmoveto{\pgfqpoint{0.000000in}{0.000000in}}%
\pgfpathlineto{\pgfqpoint{-0.027778in}{0.000000in}}%
\pgfusepath{stroke,fill}%
}%
\begin{pgfscope}%
\pgfsys@transformshift{0.625000in}{2.372916in}%
\pgfsys@useobject{currentmarker}{}%
\end{pgfscope}%
\end{pgfscope}%
\begin{pgfscope}%
\pgfsetbuttcap%
\pgfsetroundjoin%
\definecolor{currentfill}{rgb}{0.000000,0.000000,0.000000}%
\pgfsetfillcolor{currentfill}%
\pgfsetlinewidth{0.602250pt}%
\definecolor{currentstroke}{rgb}{0.000000,0.000000,0.000000}%
\pgfsetstrokecolor{currentstroke}%
\pgfsetdash{}{0pt}%
\pgfsys@defobject{currentmarker}{\pgfqpoint{-0.027778in}{0.000000in}}{\pgfqpoint{0.000000in}{0.000000in}}{%
\pgfpathmoveto{\pgfqpoint{0.000000in}{0.000000in}}%
\pgfpathlineto{\pgfqpoint{-0.027778in}{0.000000in}}%
\pgfusepath{stroke,fill}%
}%
\begin{pgfscope}%
\pgfsys@transformshift{0.625000in}{2.446108in}%
\pgfsys@useobject{currentmarker}{}%
\end{pgfscope}%
\end{pgfscope}%
\begin{pgfscope}%
\pgfsetbuttcap%
\pgfsetroundjoin%
\definecolor{currentfill}{rgb}{0.000000,0.000000,0.000000}%
\pgfsetfillcolor{currentfill}%
\pgfsetlinewidth{0.602250pt}%
\definecolor{currentstroke}{rgb}{0.000000,0.000000,0.000000}%
\pgfsetstrokecolor{currentstroke}%
\pgfsetdash{}{0pt}%
\pgfsys@defobject{currentmarker}{\pgfqpoint{-0.027778in}{0.000000in}}{\pgfqpoint{0.000000in}{0.000000in}}{%
\pgfpathmoveto{\pgfqpoint{0.000000in}{0.000000in}}%
\pgfpathlineto{\pgfqpoint{-0.027778in}{0.000000in}}%
\pgfusepath{stroke,fill}%
}%
\begin{pgfscope}%
\pgfsys@transformshift{0.625000in}{2.507991in}%
\pgfsys@useobject{currentmarker}{}%
\end{pgfscope}%
\end{pgfscope}%
\begin{pgfscope}%
\pgfsetbuttcap%
\pgfsetroundjoin%
\definecolor{currentfill}{rgb}{0.000000,0.000000,0.000000}%
\pgfsetfillcolor{currentfill}%
\pgfsetlinewidth{0.602250pt}%
\definecolor{currentstroke}{rgb}{0.000000,0.000000,0.000000}%
\pgfsetstrokecolor{currentstroke}%
\pgfsetdash{}{0pt}%
\pgfsys@defobject{currentmarker}{\pgfqpoint{-0.027778in}{0.000000in}}{\pgfqpoint{0.000000in}{0.000000in}}{%
\pgfpathmoveto{\pgfqpoint{0.000000in}{0.000000in}}%
\pgfpathlineto{\pgfqpoint{-0.027778in}{0.000000in}}%
\pgfusepath{stroke,fill}%
}%
\begin{pgfscope}%
\pgfsys@transformshift{0.625000in}{2.561596in}%
\pgfsys@useobject{currentmarker}{}%
\end{pgfscope}%
\end{pgfscope}%
\begin{pgfscope}%
\pgfsetbuttcap%
\pgfsetroundjoin%
\definecolor{currentfill}{rgb}{0.000000,0.000000,0.000000}%
\pgfsetfillcolor{currentfill}%
\pgfsetlinewidth{0.602250pt}%
\definecolor{currentstroke}{rgb}{0.000000,0.000000,0.000000}%
\pgfsetstrokecolor{currentstroke}%
\pgfsetdash{}{0pt}%
\pgfsys@defobject{currentmarker}{\pgfqpoint{-0.027778in}{0.000000in}}{\pgfqpoint{0.000000in}{0.000000in}}{%
\pgfpathmoveto{\pgfqpoint{0.000000in}{0.000000in}}%
\pgfpathlineto{\pgfqpoint{-0.027778in}{0.000000in}}%
\pgfusepath{stroke,fill}%
}%
\begin{pgfscope}%
\pgfsys@transformshift{0.625000in}{2.608879in}%
\pgfsys@useobject{currentmarker}{}%
\end{pgfscope}%
\end{pgfscope}%
\begin{pgfscope}%
\pgfsetbuttcap%
\pgfsetroundjoin%
\definecolor{currentfill}{rgb}{0.000000,0.000000,0.000000}%
\pgfsetfillcolor{currentfill}%
\pgfsetlinewidth{0.602250pt}%
\definecolor{currentstroke}{rgb}{0.000000,0.000000,0.000000}%
\pgfsetstrokecolor{currentstroke}%
\pgfsetdash{}{0pt}%
\pgfsys@defobject{currentmarker}{\pgfqpoint{-0.027778in}{0.000000in}}{\pgfqpoint{0.000000in}{0.000000in}}{%
\pgfpathmoveto{\pgfqpoint{0.000000in}{0.000000in}}%
\pgfpathlineto{\pgfqpoint{-0.027778in}{0.000000in}}%
\pgfusepath{stroke,fill}%
}%
\begin{pgfscope}%
\pgfsys@transformshift{0.625000in}{2.929434in}%
\pgfsys@useobject{currentmarker}{}%
\end{pgfscope}%
\end{pgfscope}%
\begin{pgfscope}%
\pgfsetbuttcap%
\pgfsetroundjoin%
\definecolor{currentfill}{rgb}{0.000000,0.000000,0.000000}%
\pgfsetfillcolor{currentfill}%
\pgfsetlinewidth{0.602250pt}%
\definecolor{currentstroke}{rgb}{0.000000,0.000000,0.000000}%
\pgfsetstrokecolor{currentstroke}%
\pgfsetdash{}{0pt}%
\pgfsys@defobject{currentmarker}{\pgfqpoint{-0.027778in}{0.000000in}}{\pgfqpoint{0.000000in}{0.000000in}}{%
\pgfpathmoveto{\pgfqpoint{0.000000in}{0.000000in}}%
\pgfpathlineto{\pgfqpoint{-0.027778in}{0.000000in}}%
\pgfusepath{stroke,fill}%
}%
\begin{pgfscope}%
\pgfsys@transformshift{0.625000in}{3.092206in}%
\pgfsys@useobject{currentmarker}{}%
\end{pgfscope}%
\end{pgfscope}%
\begin{pgfscope}%
\pgfsetbuttcap%
\pgfsetroundjoin%
\definecolor{currentfill}{rgb}{0.000000,0.000000,0.000000}%
\pgfsetfillcolor{currentfill}%
\pgfsetlinewidth{0.602250pt}%
\definecolor{currentstroke}{rgb}{0.000000,0.000000,0.000000}%
\pgfsetstrokecolor{currentstroke}%
\pgfsetdash{}{0pt}%
\pgfsys@defobject{currentmarker}{\pgfqpoint{-0.027778in}{0.000000in}}{\pgfqpoint{0.000000in}{0.000000in}}{%
\pgfpathmoveto{\pgfqpoint{0.000000in}{0.000000in}}%
\pgfpathlineto{\pgfqpoint{-0.027778in}{0.000000in}}%
\pgfusepath{stroke,fill}%
}%
\begin{pgfscope}%
\pgfsys@transformshift{0.625000in}{3.207694in}%
\pgfsys@useobject{currentmarker}{}%
\end{pgfscope}%
\end{pgfscope}%
\begin{pgfscope}%
\pgfsetbuttcap%
\pgfsetroundjoin%
\definecolor{currentfill}{rgb}{0.000000,0.000000,0.000000}%
\pgfsetfillcolor{currentfill}%
\pgfsetlinewidth{0.602250pt}%
\definecolor{currentstroke}{rgb}{0.000000,0.000000,0.000000}%
\pgfsetstrokecolor{currentstroke}%
\pgfsetdash{}{0pt}%
\pgfsys@defobject{currentmarker}{\pgfqpoint{-0.027778in}{0.000000in}}{\pgfqpoint{0.000000in}{0.000000in}}{%
\pgfpathmoveto{\pgfqpoint{0.000000in}{0.000000in}}%
\pgfpathlineto{\pgfqpoint{-0.027778in}{0.000000in}}%
\pgfusepath{stroke,fill}%
}%
\begin{pgfscope}%
\pgfsys@transformshift{0.625000in}{3.297273in}%
\pgfsys@useobject{currentmarker}{}%
\end{pgfscope}%
\end{pgfscope}%
\begin{pgfscope}%
\pgfsetbuttcap%
\pgfsetroundjoin%
\definecolor{currentfill}{rgb}{0.000000,0.000000,0.000000}%
\pgfsetfillcolor{currentfill}%
\pgfsetlinewidth{0.602250pt}%
\definecolor{currentstroke}{rgb}{0.000000,0.000000,0.000000}%
\pgfsetstrokecolor{currentstroke}%
\pgfsetdash{}{0pt}%
\pgfsys@defobject{currentmarker}{\pgfqpoint{-0.027778in}{0.000000in}}{\pgfqpoint{0.000000in}{0.000000in}}{%
\pgfpathmoveto{\pgfqpoint{0.000000in}{0.000000in}}%
\pgfpathlineto{\pgfqpoint{-0.027778in}{0.000000in}}%
\pgfusepath{stroke,fill}%
}%
\begin{pgfscope}%
\pgfsys@transformshift{0.625000in}{3.370465in}%
\pgfsys@useobject{currentmarker}{}%
\end{pgfscope}%
\end{pgfscope}%
\begin{pgfscope}%
\pgfsetbuttcap%
\pgfsetroundjoin%
\definecolor{currentfill}{rgb}{0.000000,0.000000,0.000000}%
\pgfsetfillcolor{currentfill}%
\pgfsetlinewidth{0.602250pt}%
\definecolor{currentstroke}{rgb}{0.000000,0.000000,0.000000}%
\pgfsetstrokecolor{currentstroke}%
\pgfsetdash{}{0pt}%
\pgfsys@defobject{currentmarker}{\pgfqpoint{-0.027778in}{0.000000in}}{\pgfqpoint{0.000000in}{0.000000in}}{%
\pgfpathmoveto{\pgfqpoint{0.000000in}{0.000000in}}%
\pgfpathlineto{\pgfqpoint{-0.027778in}{0.000000in}}%
\pgfusepath{stroke,fill}%
}%
\begin{pgfscope}%
\pgfsys@transformshift{0.625000in}{3.432347in}%
\pgfsys@useobject{currentmarker}{}%
\end{pgfscope}%
\end{pgfscope}%
\begin{pgfscope}%
\pgfsetbuttcap%
\pgfsetroundjoin%
\definecolor{currentfill}{rgb}{0.000000,0.000000,0.000000}%
\pgfsetfillcolor{currentfill}%
\pgfsetlinewidth{0.602250pt}%
\definecolor{currentstroke}{rgb}{0.000000,0.000000,0.000000}%
\pgfsetstrokecolor{currentstroke}%
\pgfsetdash{}{0pt}%
\pgfsys@defobject{currentmarker}{\pgfqpoint{-0.027778in}{0.000000in}}{\pgfqpoint{0.000000in}{0.000000in}}{%
\pgfpathmoveto{\pgfqpoint{0.000000in}{0.000000in}}%
\pgfpathlineto{\pgfqpoint{-0.027778in}{0.000000in}}%
\pgfusepath{stroke,fill}%
}%
\begin{pgfscope}%
\pgfsys@transformshift{0.625000in}{3.485953in}%
\pgfsys@useobject{currentmarker}{}%
\end{pgfscope}%
\end{pgfscope}%
\begin{pgfscope}%
\definecolor{textcolor}{rgb}{0.000000,0.000000,0.000000}%
\pgfsetstrokecolor{textcolor}%
\pgfsetfillcolor{textcolor}%
\pgftext[x=0.184220in,y=1.980000in,,bottom,rotate=90.000000]{\color{textcolor}\sffamily\fontsize{10.000000}{12.000000}\selectfont QMC Error}%
\end{pgfscope}%
\begin{pgfscope}%
\pgfpathrectangle{\pgfqpoint{0.625000in}{0.440000in}}{\pgfqpoint{3.875000in}{3.080000in}}%
\pgfusepath{clip}%
\pgfsetbuttcap%
\pgfsetroundjoin%
\definecolor{currentfill}{rgb}{0.000000,0.000000,0.000000}%
\pgfsetfillcolor{currentfill}%
\pgfsetlinewidth{1.003750pt}%
\definecolor{currentstroke}{rgb}{0.000000,0.000000,0.000000}%
\pgfsetstrokecolor{currentstroke}%
\pgfsetdash{}{0pt}%
\pgfsys@defobject{currentmarker}{\pgfqpoint{-0.041667in}{-0.041667in}}{\pgfqpoint{0.041667in}{0.041667in}}{%
\pgfpathmoveto{\pgfqpoint{0.000000in}{-0.041667in}}%
\pgfpathcurveto{\pgfqpoint{0.011050in}{-0.041667in}}{\pgfqpoint{0.021649in}{-0.037276in}}{\pgfqpoint{0.029463in}{-0.029463in}}%
\pgfpathcurveto{\pgfqpoint{0.037276in}{-0.021649in}}{\pgfqpoint{0.041667in}{-0.011050in}}{\pgfqpoint{0.041667in}{0.000000in}}%
\pgfpathcurveto{\pgfqpoint{0.041667in}{0.011050in}}{\pgfqpoint{0.037276in}{0.021649in}}{\pgfqpoint{0.029463in}{0.029463in}}%
\pgfpathcurveto{\pgfqpoint{0.021649in}{0.037276in}}{\pgfqpoint{0.011050in}{0.041667in}}{\pgfqpoint{0.000000in}{0.041667in}}%
\pgfpathcurveto{\pgfqpoint{-0.011050in}{0.041667in}}{\pgfqpoint{-0.021649in}{0.037276in}}{\pgfqpoint{-0.029463in}{0.029463in}}%
\pgfpathcurveto{\pgfqpoint{-0.037276in}{0.021649in}}{\pgfqpoint{-0.041667in}{0.011050in}}{\pgfqpoint{-0.041667in}{0.000000in}}%
\pgfpathcurveto{\pgfqpoint{-0.041667in}{-0.011050in}}{\pgfqpoint{-0.037276in}{-0.021649in}}{\pgfqpoint{-0.029463in}{-0.029463in}}%
\pgfpathcurveto{\pgfqpoint{-0.021649in}{-0.037276in}}{\pgfqpoint{-0.011050in}{-0.041667in}}{\pgfqpoint{0.000000in}{-0.041667in}}%
\pgfpathclose%
\pgfusepath{stroke,fill}%
}%
\begin{pgfscope}%
\pgfsys@transformshift{0.801136in}{3.142507in}%
\pgfsys@useobject{currentmarker}{}%
\end{pgfscope}%
\begin{pgfscope}%
\pgfsys@transformshift{1.121384in}{3.226869in}%
\pgfsys@useobject{currentmarker}{}%
\end{pgfscope}%
\begin{pgfscope}%
\pgfsys@transformshift{1.441632in}{2.904620in}%
\pgfsys@useobject{currentmarker}{}%
\end{pgfscope}%
\begin{pgfscope}%
\pgfsys@transformshift{1.761880in}{2.609192in}%
\pgfsys@useobject{currentmarker}{}%
\end{pgfscope}%
\begin{pgfscope}%
\pgfsys@transformshift{2.082128in}{2.225769in}%
\pgfsys@useobject{currentmarker}{}%
\end{pgfscope}%
\begin{pgfscope}%
\pgfsys@transformshift{2.402376in}{2.378906in}%
\pgfsys@useobject{currentmarker}{}%
\end{pgfscope}%
\begin{pgfscope}%
\pgfsys@transformshift{2.722624in}{2.150907in}%
\pgfsys@useobject{currentmarker}{}%
\end{pgfscope}%
\begin{pgfscope}%
\pgfsys@transformshift{3.042872in}{1.835433in}%
\pgfsys@useobject{currentmarker}{}%
\end{pgfscope}%
\begin{pgfscope}%
\pgfsys@transformshift{3.363120in}{1.296576in}%
\pgfsys@useobject{currentmarker}{}%
\end{pgfscope}%
\begin{pgfscope}%
\pgfsys@transformshift{3.683368in}{1.125153in}%
\pgfsys@useobject{currentmarker}{}%
\end{pgfscope}%
\begin{pgfscope}%
\pgfsys@transformshift{4.003616in}{0.775031in}%
\pgfsys@useobject{currentmarker}{}%
\end{pgfscope}%
\begin{pgfscope}%
\pgfsys@transformshift{4.323864in}{0.580000in}%
\pgfsys@useobject{currentmarker}{}%
\end{pgfscope}%
\end{pgfscope}%
\begin{pgfscope}%
\pgfpathrectangle{\pgfqpoint{0.625000in}{0.440000in}}{\pgfqpoint{3.875000in}{3.080000in}}%
\pgfusepath{clip}%
\pgfsetbuttcap%
\pgfsetroundjoin%
\pgfsetlinewidth{1.505625pt}%
\definecolor{currentstroke}{rgb}{0.000000,0.000000,0.000000}%
\pgfsetstrokecolor{currentstroke}%
\pgfsetdash{{5.550000pt}{2.400000pt}}{0.000000pt}%
\pgfpathmoveto{\pgfqpoint{0.801136in}{3.380000in}}%
\pgfpathlineto{\pgfqpoint{4.323864in}{0.661827in}}%
\pgfusepath{stroke}%
\end{pgfscope}%
\begin{pgfscope}%
\pgfsetrectcap%
\pgfsetmiterjoin%
\pgfsetlinewidth{0.803000pt}%
\definecolor{currentstroke}{rgb}{0.000000,0.000000,0.000000}%
\pgfsetstrokecolor{currentstroke}%
\pgfsetdash{}{0pt}%
\pgfpathmoveto{\pgfqpoint{0.625000in}{0.440000in}}%
\pgfpathlineto{\pgfqpoint{0.625000in}{3.520000in}}%
\pgfusepath{stroke}%
\end{pgfscope}%
\begin{pgfscope}%
\pgfsetrectcap%
\pgfsetmiterjoin%
\pgfsetlinewidth{0.803000pt}%
\definecolor{currentstroke}{rgb}{0.000000,0.000000,0.000000}%
\pgfsetstrokecolor{currentstroke}%
\pgfsetdash{}{0pt}%
\pgfpathmoveto{\pgfqpoint{4.500000in}{0.440000in}}%
\pgfpathlineto{\pgfqpoint{4.500000in}{3.520000in}}%
\pgfusepath{stroke}%
\end{pgfscope}%
\begin{pgfscope}%
\pgfsetrectcap%
\pgfsetmiterjoin%
\pgfsetlinewidth{0.803000pt}%
\definecolor{currentstroke}{rgb}{0.000000,0.000000,0.000000}%
\pgfsetstrokecolor{currentstroke}%
\pgfsetdash{}{0pt}%
\pgfpathmoveto{\pgfqpoint{0.625000in}{0.440000in}}%
\pgfpathlineto{\pgfqpoint{4.500000in}{0.440000in}}%
\pgfusepath{stroke}%
\end{pgfscope}%
\begin{pgfscope}%
\pgfsetrectcap%
\pgfsetmiterjoin%
\pgfsetlinewidth{0.803000pt}%
\definecolor{currentstroke}{rgb}{0.000000,0.000000,0.000000}%
\pgfsetstrokecolor{currentstroke}%
\pgfsetdash{}{0pt}%
\pgfpathmoveto{\pgfqpoint{0.625000in}{3.520000in}}%
\pgfpathlineto{\pgfqpoint{4.500000in}{3.520000in}}%
\pgfusepath{stroke}%
\end{pgfscope}%
\begin{pgfscope}%
\pgfsetbuttcap%
\pgfsetmiterjoin%
\definecolor{currentfill}{rgb}{1.000000,1.000000,1.000000}%
\pgfsetfillcolor{currentfill}%
\pgfsetfillopacity{0.800000}%
\pgfsetlinewidth{1.003750pt}%
\definecolor{currentstroke}{rgb}{0.800000,0.800000,0.800000}%
\pgfsetstrokecolor{currentstroke}%
\pgfsetstrokeopacity{0.800000}%
\pgfsetdash{}{0pt}%
\pgfpathmoveto{\pgfqpoint{3.042243in}{3.155267in}}%
\pgfpathlineto{\pgfqpoint{4.402778in}{3.155267in}}%
\pgfpathquadraticcurveto{\pgfqpoint{4.430556in}{3.155267in}}{\pgfqpoint{4.430556in}{3.183045in}}%
\pgfpathlineto{\pgfqpoint{4.430556in}{3.422778in}}%
\pgfpathquadraticcurveto{\pgfqpoint{4.430556in}{3.450556in}}{\pgfqpoint{4.402778in}{3.450556in}}%
\pgfpathlineto{\pgfqpoint{3.042243in}{3.450556in}}%
\pgfpathquadraticcurveto{\pgfqpoint{3.014465in}{3.450556in}}{\pgfqpoint{3.014465in}{3.422778in}}%
\pgfpathlineto{\pgfqpoint{3.014465in}{3.183045in}}%
\pgfpathquadraticcurveto{\pgfqpoint{3.014465in}{3.155267in}}{\pgfqpoint{3.042243in}{3.155267in}}%
\pgfpathclose%
\pgfusepath{stroke,fill}%
\end{pgfscope}%
\begin{pgfscope}%
\pgfsetbuttcap%
\pgfsetroundjoin%
\pgfsetlinewidth{1.505625pt}%
\definecolor{currentstroke}{rgb}{0.000000,0.000000,0.000000}%
\pgfsetstrokecolor{currentstroke}%
\pgfsetdash{{5.550000pt}{2.400000pt}}{0.000000pt}%
\pgfpathmoveto{\pgfqpoint{3.070021in}{3.321543in}}%
\pgfpathlineto{\pgfqpoint{3.347799in}{3.321543in}}%
\pgfusepath{stroke}%
\end{pgfscope}%
\begin{pgfscope}%
\definecolor{textcolor}{rgb}{0.000000,0.000000,0.000000}%
\pgfsetstrokecolor{textcolor}%
\pgfsetfillcolor{textcolor}%
\pgftext[x=3.458910in,y=3.272932in,left,base]{\color{textcolor}\sffamily\fontsize{10.000000}{12.000000}\selectfont \(\displaystyle 0.0614N_{\mathrm{QMC}}^{-0.8880}\)}%
\end{pgfscope}%
\end{pgfpicture}%
\makeatother%
\endgroup%

\caption[Quasi-Monte-Carlo error, for $Q=u((1,1))$ and $k=20$.]{Quasi-Monte-Carlo error with increasing number of Quasi-Monte-Carlo points, for $Q=u((1,1))$ and $k=20$. \label{fig:qmctop_right20}}
\end{figure}
\begin{figure}[h]
    \centering
    %% Creator: Matplotlib, PGF backend
%%
%% To include the figure in your LaTeX document, write
%%   \input{<filename>.pgf}
%%
%% Make sure the required packages are loaded in your preamble
%%   \usepackage{pgf}
%%
%% Figures using additional raster images can only be included by \input if
%% they are in the same directory as the main LaTeX file. For loading figures
%% from other directories you can use the `import` package
%%   \usepackage{import}
%% and then include the figures with
%%   \import{<path to file>}{<filename>.pgf}
%%
%% Matplotlib used the following preamble
%%   \usepackage{fontspec}
%%   \setmainfont{DejaVuSerif.ttf}[Path=/home/owen/progs/firedrake-complex/firedrake/lib/python3.5/site-packages/matplotlib/mpl-data/fonts/ttf/]
%%   \setsansfont{DejaVuSans.ttf}[Path=/home/owen/progs/firedrake-complex/firedrake/lib/python3.5/site-packages/matplotlib/mpl-data/fonts/ttf/]
%%   \setmonofont{DejaVuSansMono.ttf}[Path=/home/owen/progs/firedrake-complex/firedrake/lib/python3.5/site-packages/matplotlib/mpl-data/fonts/ttf/]
%%
\begingroup%
\makeatletter%
\begin{pgfpicture}%
\pgfpathrectangle{\pgfpointorigin}{\pgfqpoint{5.000000in}{4.000000in}}%
\pgfusepath{use as bounding box, clip}%
\begin{pgfscope}%
\pgfsetbuttcap%
\pgfsetmiterjoin%
\definecolor{currentfill}{rgb}{1.000000,1.000000,1.000000}%
\pgfsetfillcolor{currentfill}%
\pgfsetlinewidth{0.000000pt}%
\definecolor{currentstroke}{rgb}{1.000000,1.000000,1.000000}%
\pgfsetstrokecolor{currentstroke}%
\pgfsetdash{}{0pt}%
\pgfpathmoveto{\pgfqpoint{0.000000in}{0.000000in}}%
\pgfpathlineto{\pgfqpoint{5.000000in}{0.000000in}}%
\pgfpathlineto{\pgfqpoint{5.000000in}{4.000000in}}%
\pgfpathlineto{\pgfqpoint{0.000000in}{4.000000in}}%
\pgfpathclose%
\pgfusepath{fill}%
\end{pgfscope}%
\begin{pgfscope}%
\pgfsetbuttcap%
\pgfsetmiterjoin%
\definecolor{currentfill}{rgb}{1.000000,1.000000,1.000000}%
\pgfsetfillcolor{currentfill}%
\pgfsetlinewidth{0.000000pt}%
\definecolor{currentstroke}{rgb}{0.000000,0.000000,0.000000}%
\pgfsetstrokecolor{currentstroke}%
\pgfsetstrokeopacity{0.000000}%
\pgfsetdash{}{0pt}%
\pgfpathmoveto{\pgfqpoint{0.625000in}{0.440000in}}%
\pgfpathlineto{\pgfqpoint{4.500000in}{0.440000in}}%
\pgfpathlineto{\pgfqpoint{4.500000in}{3.520000in}}%
\pgfpathlineto{\pgfqpoint{0.625000in}{3.520000in}}%
\pgfpathclose%
\pgfusepath{fill}%
\end{pgfscope}%
\begin{pgfscope}%
\pgfsetbuttcap%
\pgfsetroundjoin%
\definecolor{currentfill}{rgb}{0.000000,0.000000,0.000000}%
\pgfsetfillcolor{currentfill}%
\pgfsetlinewidth{0.803000pt}%
\definecolor{currentstroke}{rgb}{0.000000,0.000000,0.000000}%
\pgfsetstrokecolor{currentstroke}%
\pgfsetdash{}{0pt}%
\pgfsys@defobject{currentmarker}{\pgfqpoint{0.000000in}{-0.048611in}}{\pgfqpoint{0.000000in}{0.000000in}}{%
\pgfpathmoveto{\pgfqpoint{0.000000in}{0.000000in}}%
\pgfpathlineto{\pgfqpoint{0.000000in}{-0.048611in}}%
\pgfusepath{stroke,fill}%
}%
\begin{pgfscope}%
\pgfsys@transformshift{0.801136in}{0.440000in}%
\pgfsys@useobject{currentmarker}{}%
\end{pgfscope}%
\end{pgfscope}%
\begin{pgfscope}%
\definecolor{textcolor}{rgb}{0.000000,0.000000,0.000000}%
\pgfsetstrokecolor{textcolor}%
\pgfsetfillcolor{textcolor}%
\pgftext[x=0.801136in,y=0.342778in,,top]{\color{textcolor}\sffamily\fontsize{10.000000}{12.000000}\selectfont \(\displaystyle {10^{0}}\)}%
\end{pgfscope}%
\begin{pgfscope}%
\pgfsetbuttcap%
\pgfsetroundjoin%
\definecolor{currentfill}{rgb}{0.000000,0.000000,0.000000}%
\pgfsetfillcolor{currentfill}%
\pgfsetlinewidth{0.803000pt}%
\definecolor{currentstroke}{rgb}{0.000000,0.000000,0.000000}%
\pgfsetstrokecolor{currentstroke}%
\pgfsetdash{}{0pt}%
\pgfsys@defobject{currentmarker}{\pgfqpoint{0.000000in}{-0.048611in}}{\pgfqpoint{0.000000in}{0.000000in}}{%
\pgfpathmoveto{\pgfqpoint{0.000000in}{0.000000in}}%
\pgfpathlineto{\pgfqpoint{0.000000in}{-0.048611in}}%
\pgfusepath{stroke,fill}%
}%
\begin{pgfscope}%
\pgfsys@transformshift{1.864977in}{0.440000in}%
\pgfsys@useobject{currentmarker}{}%
\end{pgfscope}%
\end{pgfscope}%
\begin{pgfscope}%
\definecolor{textcolor}{rgb}{0.000000,0.000000,0.000000}%
\pgfsetstrokecolor{textcolor}%
\pgfsetfillcolor{textcolor}%
\pgftext[x=1.864977in,y=0.342778in,,top]{\color{textcolor}\sffamily\fontsize{10.000000}{12.000000}\selectfont \(\displaystyle {10^{1}}\)}%
\end{pgfscope}%
\begin{pgfscope}%
\pgfsetbuttcap%
\pgfsetroundjoin%
\definecolor{currentfill}{rgb}{0.000000,0.000000,0.000000}%
\pgfsetfillcolor{currentfill}%
\pgfsetlinewidth{0.803000pt}%
\definecolor{currentstroke}{rgb}{0.000000,0.000000,0.000000}%
\pgfsetstrokecolor{currentstroke}%
\pgfsetdash{}{0pt}%
\pgfsys@defobject{currentmarker}{\pgfqpoint{0.000000in}{-0.048611in}}{\pgfqpoint{0.000000in}{0.000000in}}{%
\pgfpathmoveto{\pgfqpoint{0.000000in}{0.000000in}}%
\pgfpathlineto{\pgfqpoint{0.000000in}{-0.048611in}}%
\pgfusepath{stroke,fill}%
}%
\begin{pgfscope}%
\pgfsys@transformshift{2.928818in}{0.440000in}%
\pgfsys@useobject{currentmarker}{}%
\end{pgfscope}%
\end{pgfscope}%
\begin{pgfscope}%
\definecolor{textcolor}{rgb}{0.000000,0.000000,0.000000}%
\pgfsetstrokecolor{textcolor}%
\pgfsetfillcolor{textcolor}%
\pgftext[x=2.928818in,y=0.342778in,,top]{\color{textcolor}\sffamily\fontsize{10.000000}{12.000000}\selectfont \(\displaystyle {10^{2}}\)}%
\end{pgfscope}%
\begin{pgfscope}%
\pgfsetbuttcap%
\pgfsetroundjoin%
\definecolor{currentfill}{rgb}{0.000000,0.000000,0.000000}%
\pgfsetfillcolor{currentfill}%
\pgfsetlinewidth{0.803000pt}%
\definecolor{currentstroke}{rgb}{0.000000,0.000000,0.000000}%
\pgfsetstrokecolor{currentstroke}%
\pgfsetdash{}{0pt}%
\pgfsys@defobject{currentmarker}{\pgfqpoint{0.000000in}{-0.048611in}}{\pgfqpoint{0.000000in}{0.000000in}}{%
\pgfpathmoveto{\pgfqpoint{0.000000in}{0.000000in}}%
\pgfpathlineto{\pgfqpoint{0.000000in}{-0.048611in}}%
\pgfusepath{stroke,fill}%
}%
\begin{pgfscope}%
\pgfsys@transformshift{3.992658in}{0.440000in}%
\pgfsys@useobject{currentmarker}{}%
\end{pgfscope}%
\end{pgfscope}%
\begin{pgfscope}%
\definecolor{textcolor}{rgb}{0.000000,0.000000,0.000000}%
\pgfsetstrokecolor{textcolor}%
\pgfsetfillcolor{textcolor}%
\pgftext[x=3.992658in,y=0.342778in,,top]{\color{textcolor}\sffamily\fontsize{10.000000}{12.000000}\selectfont \(\displaystyle {10^{3}}\)}%
\end{pgfscope}%
\begin{pgfscope}%
\pgfsetbuttcap%
\pgfsetroundjoin%
\definecolor{currentfill}{rgb}{0.000000,0.000000,0.000000}%
\pgfsetfillcolor{currentfill}%
\pgfsetlinewidth{0.602250pt}%
\definecolor{currentstroke}{rgb}{0.000000,0.000000,0.000000}%
\pgfsetstrokecolor{currentstroke}%
\pgfsetdash{}{0pt}%
\pgfsys@defobject{currentmarker}{\pgfqpoint{0.000000in}{-0.027778in}}{\pgfqpoint{0.000000in}{0.000000in}}{%
\pgfpathmoveto{\pgfqpoint{0.000000in}{0.000000in}}%
\pgfpathlineto{\pgfqpoint{0.000000in}{-0.027778in}}%
\pgfusepath{stroke,fill}%
}%
\begin{pgfscope}%
\pgfsys@transformshift{0.636345in}{0.440000in}%
\pgfsys@useobject{currentmarker}{}%
\end{pgfscope}%
\end{pgfscope}%
\begin{pgfscope}%
\pgfsetbuttcap%
\pgfsetroundjoin%
\definecolor{currentfill}{rgb}{0.000000,0.000000,0.000000}%
\pgfsetfillcolor{currentfill}%
\pgfsetlinewidth{0.602250pt}%
\definecolor{currentstroke}{rgb}{0.000000,0.000000,0.000000}%
\pgfsetstrokecolor{currentstroke}%
\pgfsetdash{}{0pt}%
\pgfsys@defobject{currentmarker}{\pgfqpoint{0.000000in}{-0.027778in}}{\pgfqpoint{0.000000in}{0.000000in}}{%
\pgfpathmoveto{\pgfqpoint{0.000000in}{0.000000in}}%
\pgfpathlineto{\pgfqpoint{0.000000in}{-0.027778in}}%
\pgfusepath{stroke,fill}%
}%
\begin{pgfscope}%
\pgfsys@transformshift{0.698040in}{0.440000in}%
\pgfsys@useobject{currentmarker}{}%
\end{pgfscope}%
\end{pgfscope}%
\begin{pgfscope}%
\pgfsetbuttcap%
\pgfsetroundjoin%
\definecolor{currentfill}{rgb}{0.000000,0.000000,0.000000}%
\pgfsetfillcolor{currentfill}%
\pgfsetlinewidth{0.602250pt}%
\definecolor{currentstroke}{rgb}{0.000000,0.000000,0.000000}%
\pgfsetstrokecolor{currentstroke}%
\pgfsetdash{}{0pt}%
\pgfsys@defobject{currentmarker}{\pgfqpoint{0.000000in}{-0.027778in}}{\pgfqpoint{0.000000in}{0.000000in}}{%
\pgfpathmoveto{\pgfqpoint{0.000000in}{0.000000in}}%
\pgfpathlineto{\pgfqpoint{0.000000in}{-0.027778in}}%
\pgfusepath{stroke,fill}%
}%
\begin{pgfscope}%
\pgfsys@transformshift{0.752458in}{0.440000in}%
\pgfsys@useobject{currentmarker}{}%
\end{pgfscope}%
\end{pgfscope}%
\begin{pgfscope}%
\pgfsetbuttcap%
\pgfsetroundjoin%
\definecolor{currentfill}{rgb}{0.000000,0.000000,0.000000}%
\pgfsetfillcolor{currentfill}%
\pgfsetlinewidth{0.602250pt}%
\definecolor{currentstroke}{rgb}{0.000000,0.000000,0.000000}%
\pgfsetstrokecolor{currentstroke}%
\pgfsetdash{}{0pt}%
\pgfsys@defobject{currentmarker}{\pgfqpoint{0.000000in}{-0.027778in}}{\pgfqpoint{0.000000in}{0.000000in}}{%
\pgfpathmoveto{\pgfqpoint{0.000000in}{0.000000in}}%
\pgfpathlineto{\pgfqpoint{0.000000in}{-0.027778in}}%
\pgfusepath{stroke,fill}%
}%
\begin{pgfscope}%
\pgfsys@transformshift{1.121384in}{0.440000in}%
\pgfsys@useobject{currentmarker}{}%
\end{pgfscope}%
\end{pgfscope}%
\begin{pgfscope}%
\pgfsetbuttcap%
\pgfsetroundjoin%
\definecolor{currentfill}{rgb}{0.000000,0.000000,0.000000}%
\pgfsetfillcolor{currentfill}%
\pgfsetlinewidth{0.602250pt}%
\definecolor{currentstroke}{rgb}{0.000000,0.000000,0.000000}%
\pgfsetstrokecolor{currentstroke}%
\pgfsetdash{}{0pt}%
\pgfsys@defobject{currentmarker}{\pgfqpoint{0.000000in}{-0.027778in}}{\pgfqpoint{0.000000in}{0.000000in}}{%
\pgfpathmoveto{\pgfqpoint{0.000000in}{0.000000in}}%
\pgfpathlineto{\pgfqpoint{0.000000in}{-0.027778in}}%
\pgfusepath{stroke,fill}%
}%
\begin{pgfscope}%
\pgfsys@transformshift{1.308717in}{0.440000in}%
\pgfsys@useobject{currentmarker}{}%
\end{pgfscope}%
\end{pgfscope}%
\begin{pgfscope}%
\pgfsetbuttcap%
\pgfsetroundjoin%
\definecolor{currentfill}{rgb}{0.000000,0.000000,0.000000}%
\pgfsetfillcolor{currentfill}%
\pgfsetlinewidth{0.602250pt}%
\definecolor{currentstroke}{rgb}{0.000000,0.000000,0.000000}%
\pgfsetstrokecolor{currentstroke}%
\pgfsetdash{}{0pt}%
\pgfsys@defobject{currentmarker}{\pgfqpoint{0.000000in}{-0.027778in}}{\pgfqpoint{0.000000in}{0.000000in}}{%
\pgfpathmoveto{\pgfqpoint{0.000000in}{0.000000in}}%
\pgfpathlineto{\pgfqpoint{0.000000in}{-0.027778in}}%
\pgfusepath{stroke,fill}%
}%
\begin{pgfscope}%
\pgfsys@transformshift{1.441632in}{0.440000in}%
\pgfsys@useobject{currentmarker}{}%
\end{pgfscope}%
\end{pgfscope}%
\begin{pgfscope}%
\pgfsetbuttcap%
\pgfsetroundjoin%
\definecolor{currentfill}{rgb}{0.000000,0.000000,0.000000}%
\pgfsetfillcolor{currentfill}%
\pgfsetlinewidth{0.602250pt}%
\definecolor{currentstroke}{rgb}{0.000000,0.000000,0.000000}%
\pgfsetstrokecolor{currentstroke}%
\pgfsetdash{}{0pt}%
\pgfsys@defobject{currentmarker}{\pgfqpoint{0.000000in}{-0.027778in}}{\pgfqpoint{0.000000in}{0.000000in}}{%
\pgfpathmoveto{\pgfqpoint{0.000000in}{0.000000in}}%
\pgfpathlineto{\pgfqpoint{0.000000in}{-0.027778in}}%
\pgfusepath{stroke,fill}%
}%
\begin{pgfscope}%
\pgfsys@transformshift{1.544729in}{0.440000in}%
\pgfsys@useobject{currentmarker}{}%
\end{pgfscope}%
\end{pgfscope}%
\begin{pgfscope}%
\pgfsetbuttcap%
\pgfsetroundjoin%
\definecolor{currentfill}{rgb}{0.000000,0.000000,0.000000}%
\pgfsetfillcolor{currentfill}%
\pgfsetlinewidth{0.602250pt}%
\definecolor{currentstroke}{rgb}{0.000000,0.000000,0.000000}%
\pgfsetstrokecolor{currentstroke}%
\pgfsetdash{}{0pt}%
\pgfsys@defobject{currentmarker}{\pgfqpoint{0.000000in}{-0.027778in}}{\pgfqpoint{0.000000in}{0.000000in}}{%
\pgfpathmoveto{\pgfqpoint{0.000000in}{0.000000in}}%
\pgfpathlineto{\pgfqpoint{0.000000in}{-0.027778in}}%
\pgfusepath{stroke,fill}%
}%
\begin{pgfscope}%
\pgfsys@transformshift{1.628965in}{0.440000in}%
\pgfsys@useobject{currentmarker}{}%
\end{pgfscope}%
\end{pgfscope}%
\begin{pgfscope}%
\pgfsetbuttcap%
\pgfsetroundjoin%
\definecolor{currentfill}{rgb}{0.000000,0.000000,0.000000}%
\pgfsetfillcolor{currentfill}%
\pgfsetlinewidth{0.602250pt}%
\definecolor{currentstroke}{rgb}{0.000000,0.000000,0.000000}%
\pgfsetstrokecolor{currentstroke}%
\pgfsetdash{}{0pt}%
\pgfsys@defobject{currentmarker}{\pgfqpoint{0.000000in}{-0.027778in}}{\pgfqpoint{0.000000in}{0.000000in}}{%
\pgfpathmoveto{\pgfqpoint{0.000000in}{0.000000in}}%
\pgfpathlineto{\pgfqpoint{0.000000in}{-0.027778in}}%
\pgfusepath{stroke,fill}%
}%
\begin{pgfscope}%
\pgfsys@transformshift{1.700186in}{0.440000in}%
\pgfsys@useobject{currentmarker}{}%
\end{pgfscope}%
\end{pgfscope}%
\begin{pgfscope}%
\pgfsetbuttcap%
\pgfsetroundjoin%
\definecolor{currentfill}{rgb}{0.000000,0.000000,0.000000}%
\pgfsetfillcolor{currentfill}%
\pgfsetlinewidth{0.602250pt}%
\definecolor{currentstroke}{rgb}{0.000000,0.000000,0.000000}%
\pgfsetstrokecolor{currentstroke}%
\pgfsetdash{}{0pt}%
\pgfsys@defobject{currentmarker}{\pgfqpoint{0.000000in}{-0.027778in}}{\pgfqpoint{0.000000in}{0.000000in}}{%
\pgfpathmoveto{\pgfqpoint{0.000000in}{0.000000in}}%
\pgfpathlineto{\pgfqpoint{0.000000in}{-0.027778in}}%
\pgfusepath{stroke,fill}%
}%
\begin{pgfscope}%
\pgfsys@transformshift{1.761880in}{0.440000in}%
\pgfsys@useobject{currentmarker}{}%
\end{pgfscope}%
\end{pgfscope}%
\begin{pgfscope}%
\pgfsetbuttcap%
\pgfsetroundjoin%
\definecolor{currentfill}{rgb}{0.000000,0.000000,0.000000}%
\pgfsetfillcolor{currentfill}%
\pgfsetlinewidth{0.602250pt}%
\definecolor{currentstroke}{rgb}{0.000000,0.000000,0.000000}%
\pgfsetstrokecolor{currentstroke}%
\pgfsetdash{}{0pt}%
\pgfsys@defobject{currentmarker}{\pgfqpoint{0.000000in}{-0.027778in}}{\pgfqpoint{0.000000in}{0.000000in}}{%
\pgfpathmoveto{\pgfqpoint{0.000000in}{0.000000in}}%
\pgfpathlineto{\pgfqpoint{0.000000in}{-0.027778in}}%
\pgfusepath{stroke,fill}%
}%
\begin{pgfscope}%
\pgfsys@transformshift{1.816298in}{0.440000in}%
\pgfsys@useobject{currentmarker}{}%
\end{pgfscope}%
\end{pgfscope}%
\begin{pgfscope}%
\pgfsetbuttcap%
\pgfsetroundjoin%
\definecolor{currentfill}{rgb}{0.000000,0.000000,0.000000}%
\pgfsetfillcolor{currentfill}%
\pgfsetlinewidth{0.602250pt}%
\definecolor{currentstroke}{rgb}{0.000000,0.000000,0.000000}%
\pgfsetstrokecolor{currentstroke}%
\pgfsetdash{}{0pt}%
\pgfsys@defobject{currentmarker}{\pgfqpoint{0.000000in}{-0.027778in}}{\pgfqpoint{0.000000in}{0.000000in}}{%
\pgfpathmoveto{\pgfqpoint{0.000000in}{0.000000in}}%
\pgfpathlineto{\pgfqpoint{0.000000in}{-0.027778in}}%
\pgfusepath{stroke,fill}%
}%
\begin{pgfscope}%
\pgfsys@transformshift{2.185225in}{0.440000in}%
\pgfsys@useobject{currentmarker}{}%
\end{pgfscope}%
\end{pgfscope}%
\begin{pgfscope}%
\pgfsetbuttcap%
\pgfsetroundjoin%
\definecolor{currentfill}{rgb}{0.000000,0.000000,0.000000}%
\pgfsetfillcolor{currentfill}%
\pgfsetlinewidth{0.602250pt}%
\definecolor{currentstroke}{rgb}{0.000000,0.000000,0.000000}%
\pgfsetstrokecolor{currentstroke}%
\pgfsetdash{}{0pt}%
\pgfsys@defobject{currentmarker}{\pgfqpoint{0.000000in}{-0.027778in}}{\pgfqpoint{0.000000in}{0.000000in}}{%
\pgfpathmoveto{\pgfqpoint{0.000000in}{0.000000in}}%
\pgfpathlineto{\pgfqpoint{0.000000in}{-0.027778in}}%
\pgfusepath{stroke,fill}%
}%
\begin{pgfscope}%
\pgfsys@transformshift{2.372558in}{0.440000in}%
\pgfsys@useobject{currentmarker}{}%
\end{pgfscope}%
\end{pgfscope}%
\begin{pgfscope}%
\pgfsetbuttcap%
\pgfsetroundjoin%
\definecolor{currentfill}{rgb}{0.000000,0.000000,0.000000}%
\pgfsetfillcolor{currentfill}%
\pgfsetlinewidth{0.602250pt}%
\definecolor{currentstroke}{rgb}{0.000000,0.000000,0.000000}%
\pgfsetstrokecolor{currentstroke}%
\pgfsetdash{}{0pt}%
\pgfsys@defobject{currentmarker}{\pgfqpoint{0.000000in}{-0.027778in}}{\pgfqpoint{0.000000in}{0.000000in}}{%
\pgfpathmoveto{\pgfqpoint{0.000000in}{0.000000in}}%
\pgfpathlineto{\pgfqpoint{0.000000in}{-0.027778in}}%
\pgfusepath{stroke,fill}%
}%
\begin{pgfscope}%
\pgfsys@transformshift{2.505473in}{0.440000in}%
\pgfsys@useobject{currentmarker}{}%
\end{pgfscope}%
\end{pgfscope}%
\begin{pgfscope}%
\pgfsetbuttcap%
\pgfsetroundjoin%
\definecolor{currentfill}{rgb}{0.000000,0.000000,0.000000}%
\pgfsetfillcolor{currentfill}%
\pgfsetlinewidth{0.602250pt}%
\definecolor{currentstroke}{rgb}{0.000000,0.000000,0.000000}%
\pgfsetstrokecolor{currentstroke}%
\pgfsetdash{}{0pt}%
\pgfsys@defobject{currentmarker}{\pgfqpoint{0.000000in}{-0.027778in}}{\pgfqpoint{0.000000in}{0.000000in}}{%
\pgfpathmoveto{\pgfqpoint{0.000000in}{0.000000in}}%
\pgfpathlineto{\pgfqpoint{0.000000in}{-0.027778in}}%
\pgfusepath{stroke,fill}%
}%
\begin{pgfscope}%
\pgfsys@transformshift{2.608570in}{0.440000in}%
\pgfsys@useobject{currentmarker}{}%
\end{pgfscope}%
\end{pgfscope}%
\begin{pgfscope}%
\pgfsetbuttcap%
\pgfsetroundjoin%
\definecolor{currentfill}{rgb}{0.000000,0.000000,0.000000}%
\pgfsetfillcolor{currentfill}%
\pgfsetlinewidth{0.602250pt}%
\definecolor{currentstroke}{rgb}{0.000000,0.000000,0.000000}%
\pgfsetstrokecolor{currentstroke}%
\pgfsetdash{}{0pt}%
\pgfsys@defobject{currentmarker}{\pgfqpoint{0.000000in}{-0.027778in}}{\pgfqpoint{0.000000in}{0.000000in}}{%
\pgfpathmoveto{\pgfqpoint{0.000000in}{0.000000in}}%
\pgfpathlineto{\pgfqpoint{0.000000in}{-0.027778in}}%
\pgfusepath{stroke,fill}%
}%
\begin{pgfscope}%
\pgfsys@transformshift{2.692806in}{0.440000in}%
\pgfsys@useobject{currentmarker}{}%
\end{pgfscope}%
\end{pgfscope}%
\begin{pgfscope}%
\pgfsetbuttcap%
\pgfsetroundjoin%
\definecolor{currentfill}{rgb}{0.000000,0.000000,0.000000}%
\pgfsetfillcolor{currentfill}%
\pgfsetlinewidth{0.602250pt}%
\definecolor{currentstroke}{rgb}{0.000000,0.000000,0.000000}%
\pgfsetstrokecolor{currentstroke}%
\pgfsetdash{}{0pt}%
\pgfsys@defobject{currentmarker}{\pgfqpoint{0.000000in}{-0.027778in}}{\pgfqpoint{0.000000in}{0.000000in}}{%
\pgfpathmoveto{\pgfqpoint{0.000000in}{0.000000in}}%
\pgfpathlineto{\pgfqpoint{0.000000in}{-0.027778in}}%
\pgfusepath{stroke,fill}%
}%
\begin{pgfscope}%
\pgfsys@transformshift{2.764027in}{0.440000in}%
\pgfsys@useobject{currentmarker}{}%
\end{pgfscope}%
\end{pgfscope}%
\begin{pgfscope}%
\pgfsetbuttcap%
\pgfsetroundjoin%
\definecolor{currentfill}{rgb}{0.000000,0.000000,0.000000}%
\pgfsetfillcolor{currentfill}%
\pgfsetlinewidth{0.602250pt}%
\definecolor{currentstroke}{rgb}{0.000000,0.000000,0.000000}%
\pgfsetstrokecolor{currentstroke}%
\pgfsetdash{}{0pt}%
\pgfsys@defobject{currentmarker}{\pgfqpoint{0.000000in}{-0.027778in}}{\pgfqpoint{0.000000in}{0.000000in}}{%
\pgfpathmoveto{\pgfqpoint{0.000000in}{0.000000in}}%
\pgfpathlineto{\pgfqpoint{0.000000in}{-0.027778in}}%
\pgfusepath{stroke,fill}%
}%
\begin{pgfscope}%
\pgfsys@transformshift{2.825721in}{0.440000in}%
\pgfsys@useobject{currentmarker}{}%
\end{pgfscope}%
\end{pgfscope}%
\begin{pgfscope}%
\pgfsetbuttcap%
\pgfsetroundjoin%
\definecolor{currentfill}{rgb}{0.000000,0.000000,0.000000}%
\pgfsetfillcolor{currentfill}%
\pgfsetlinewidth{0.602250pt}%
\definecolor{currentstroke}{rgb}{0.000000,0.000000,0.000000}%
\pgfsetstrokecolor{currentstroke}%
\pgfsetdash{}{0pt}%
\pgfsys@defobject{currentmarker}{\pgfqpoint{0.000000in}{-0.027778in}}{\pgfqpoint{0.000000in}{0.000000in}}{%
\pgfpathmoveto{\pgfqpoint{0.000000in}{0.000000in}}%
\pgfpathlineto{\pgfqpoint{0.000000in}{-0.027778in}}%
\pgfusepath{stroke,fill}%
}%
\begin{pgfscope}%
\pgfsys@transformshift{2.880139in}{0.440000in}%
\pgfsys@useobject{currentmarker}{}%
\end{pgfscope}%
\end{pgfscope}%
\begin{pgfscope}%
\pgfsetbuttcap%
\pgfsetroundjoin%
\definecolor{currentfill}{rgb}{0.000000,0.000000,0.000000}%
\pgfsetfillcolor{currentfill}%
\pgfsetlinewidth{0.602250pt}%
\definecolor{currentstroke}{rgb}{0.000000,0.000000,0.000000}%
\pgfsetstrokecolor{currentstroke}%
\pgfsetdash{}{0pt}%
\pgfsys@defobject{currentmarker}{\pgfqpoint{0.000000in}{-0.027778in}}{\pgfqpoint{0.000000in}{0.000000in}}{%
\pgfpathmoveto{\pgfqpoint{0.000000in}{0.000000in}}%
\pgfpathlineto{\pgfqpoint{0.000000in}{-0.027778in}}%
\pgfusepath{stroke,fill}%
}%
\begin{pgfscope}%
\pgfsys@transformshift{3.249066in}{0.440000in}%
\pgfsys@useobject{currentmarker}{}%
\end{pgfscope}%
\end{pgfscope}%
\begin{pgfscope}%
\pgfsetbuttcap%
\pgfsetroundjoin%
\definecolor{currentfill}{rgb}{0.000000,0.000000,0.000000}%
\pgfsetfillcolor{currentfill}%
\pgfsetlinewidth{0.602250pt}%
\definecolor{currentstroke}{rgb}{0.000000,0.000000,0.000000}%
\pgfsetstrokecolor{currentstroke}%
\pgfsetdash{}{0pt}%
\pgfsys@defobject{currentmarker}{\pgfqpoint{0.000000in}{-0.027778in}}{\pgfqpoint{0.000000in}{0.000000in}}{%
\pgfpathmoveto{\pgfqpoint{0.000000in}{0.000000in}}%
\pgfpathlineto{\pgfqpoint{0.000000in}{-0.027778in}}%
\pgfusepath{stroke,fill}%
}%
\begin{pgfscope}%
\pgfsys@transformshift{3.436399in}{0.440000in}%
\pgfsys@useobject{currentmarker}{}%
\end{pgfscope}%
\end{pgfscope}%
\begin{pgfscope}%
\pgfsetbuttcap%
\pgfsetroundjoin%
\definecolor{currentfill}{rgb}{0.000000,0.000000,0.000000}%
\pgfsetfillcolor{currentfill}%
\pgfsetlinewidth{0.602250pt}%
\definecolor{currentstroke}{rgb}{0.000000,0.000000,0.000000}%
\pgfsetstrokecolor{currentstroke}%
\pgfsetdash{}{0pt}%
\pgfsys@defobject{currentmarker}{\pgfqpoint{0.000000in}{-0.027778in}}{\pgfqpoint{0.000000in}{0.000000in}}{%
\pgfpathmoveto{\pgfqpoint{0.000000in}{0.000000in}}%
\pgfpathlineto{\pgfqpoint{0.000000in}{-0.027778in}}%
\pgfusepath{stroke,fill}%
}%
\begin{pgfscope}%
\pgfsys@transformshift{3.569313in}{0.440000in}%
\pgfsys@useobject{currentmarker}{}%
\end{pgfscope}%
\end{pgfscope}%
\begin{pgfscope}%
\pgfsetbuttcap%
\pgfsetroundjoin%
\definecolor{currentfill}{rgb}{0.000000,0.000000,0.000000}%
\pgfsetfillcolor{currentfill}%
\pgfsetlinewidth{0.602250pt}%
\definecolor{currentstroke}{rgb}{0.000000,0.000000,0.000000}%
\pgfsetstrokecolor{currentstroke}%
\pgfsetdash{}{0pt}%
\pgfsys@defobject{currentmarker}{\pgfqpoint{0.000000in}{-0.027778in}}{\pgfqpoint{0.000000in}{0.000000in}}{%
\pgfpathmoveto{\pgfqpoint{0.000000in}{0.000000in}}%
\pgfpathlineto{\pgfqpoint{0.000000in}{-0.027778in}}%
\pgfusepath{stroke,fill}%
}%
\begin{pgfscope}%
\pgfsys@transformshift{3.672410in}{0.440000in}%
\pgfsys@useobject{currentmarker}{}%
\end{pgfscope}%
\end{pgfscope}%
\begin{pgfscope}%
\pgfsetbuttcap%
\pgfsetroundjoin%
\definecolor{currentfill}{rgb}{0.000000,0.000000,0.000000}%
\pgfsetfillcolor{currentfill}%
\pgfsetlinewidth{0.602250pt}%
\definecolor{currentstroke}{rgb}{0.000000,0.000000,0.000000}%
\pgfsetstrokecolor{currentstroke}%
\pgfsetdash{}{0pt}%
\pgfsys@defobject{currentmarker}{\pgfqpoint{0.000000in}{-0.027778in}}{\pgfqpoint{0.000000in}{0.000000in}}{%
\pgfpathmoveto{\pgfqpoint{0.000000in}{0.000000in}}%
\pgfpathlineto{\pgfqpoint{0.000000in}{-0.027778in}}%
\pgfusepath{stroke,fill}%
}%
\begin{pgfscope}%
\pgfsys@transformshift{3.756646in}{0.440000in}%
\pgfsys@useobject{currentmarker}{}%
\end{pgfscope}%
\end{pgfscope}%
\begin{pgfscope}%
\pgfsetbuttcap%
\pgfsetroundjoin%
\definecolor{currentfill}{rgb}{0.000000,0.000000,0.000000}%
\pgfsetfillcolor{currentfill}%
\pgfsetlinewidth{0.602250pt}%
\definecolor{currentstroke}{rgb}{0.000000,0.000000,0.000000}%
\pgfsetstrokecolor{currentstroke}%
\pgfsetdash{}{0pt}%
\pgfsys@defobject{currentmarker}{\pgfqpoint{0.000000in}{-0.027778in}}{\pgfqpoint{0.000000in}{0.000000in}}{%
\pgfpathmoveto{\pgfqpoint{0.000000in}{0.000000in}}%
\pgfpathlineto{\pgfqpoint{0.000000in}{-0.027778in}}%
\pgfusepath{stroke,fill}%
}%
\begin{pgfscope}%
\pgfsys@transformshift{3.827867in}{0.440000in}%
\pgfsys@useobject{currentmarker}{}%
\end{pgfscope}%
\end{pgfscope}%
\begin{pgfscope}%
\pgfsetbuttcap%
\pgfsetroundjoin%
\definecolor{currentfill}{rgb}{0.000000,0.000000,0.000000}%
\pgfsetfillcolor{currentfill}%
\pgfsetlinewidth{0.602250pt}%
\definecolor{currentstroke}{rgb}{0.000000,0.000000,0.000000}%
\pgfsetstrokecolor{currentstroke}%
\pgfsetdash{}{0pt}%
\pgfsys@defobject{currentmarker}{\pgfqpoint{0.000000in}{-0.027778in}}{\pgfqpoint{0.000000in}{0.000000in}}{%
\pgfpathmoveto{\pgfqpoint{0.000000in}{0.000000in}}%
\pgfpathlineto{\pgfqpoint{0.000000in}{-0.027778in}}%
\pgfusepath{stroke,fill}%
}%
\begin{pgfscope}%
\pgfsys@transformshift{3.889561in}{0.440000in}%
\pgfsys@useobject{currentmarker}{}%
\end{pgfscope}%
\end{pgfscope}%
\begin{pgfscope}%
\pgfsetbuttcap%
\pgfsetroundjoin%
\definecolor{currentfill}{rgb}{0.000000,0.000000,0.000000}%
\pgfsetfillcolor{currentfill}%
\pgfsetlinewidth{0.602250pt}%
\definecolor{currentstroke}{rgb}{0.000000,0.000000,0.000000}%
\pgfsetstrokecolor{currentstroke}%
\pgfsetdash{}{0pt}%
\pgfsys@defobject{currentmarker}{\pgfqpoint{0.000000in}{-0.027778in}}{\pgfqpoint{0.000000in}{0.000000in}}{%
\pgfpathmoveto{\pgfqpoint{0.000000in}{0.000000in}}%
\pgfpathlineto{\pgfqpoint{0.000000in}{-0.027778in}}%
\pgfusepath{stroke,fill}%
}%
\begin{pgfscope}%
\pgfsys@transformshift{3.943980in}{0.440000in}%
\pgfsys@useobject{currentmarker}{}%
\end{pgfscope}%
\end{pgfscope}%
\begin{pgfscope}%
\pgfsetbuttcap%
\pgfsetroundjoin%
\definecolor{currentfill}{rgb}{0.000000,0.000000,0.000000}%
\pgfsetfillcolor{currentfill}%
\pgfsetlinewidth{0.602250pt}%
\definecolor{currentstroke}{rgb}{0.000000,0.000000,0.000000}%
\pgfsetstrokecolor{currentstroke}%
\pgfsetdash{}{0pt}%
\pgfsys@defobject{currentmarker}{\pgfqpoint{0.000000in}{-0.027778in}}{\pgfqpoint{0.000000in}{0.000000in}}{%
\pgfpathmoveto{\pgfqpoint{0.000000in}{0.000000in}}%
\pgfpathlineto{\pgfqpoint{0.000000in}{-0.027778in}}%
\pgfusepath{stroke,fill}%
}%
\begin{pgfscope}%
\pgfsys@transformshift{4.312906in}{0.440000in}%
\pgfsys@useobject{currentmarker}{}%
\end{pgfscope}%
\end{pgfscope}%
\begin{pgfscope}%
\pgfsetbuttcap%
\pgfsetroundjoin%
\definecolor{currentfill}{rgb}{0.000000,0.000000,0.000000}%
\pgfsetfillcolor{currentfill}%
\pgfsetlinewidth{0.602250pt}%
\definecolor{currentstroke}{rgb}{0.000000,0.000000,0.000000}%
\pgfsetstrokecolor{currentstroke}%
\pgfsetdash{}{0pt}%
\pgfsys@defobject{currentmarker}{\pgfqpoint{0.000000in}{-0.027778in}}{\pgfqpoint{0.000000in}{0.000000in}}{%
\pgfpathmoveto{\pgfqpoint{0.000000in}{0.000000in}}%
\pgfpathlineto{\pgfqpoint{0.000000in}{-0.027778in}}%
\pgfusepath{stroke,fill}%
}%
\begin{pgfscope}%
\pgfsys@transformshift{4.500239in}{0.440000in}%
\pgfsys@useobject{currentmarker}{}%
\end{pgfscope}%
\end{pgfscope}%
\begin{pgfscope}%
\definecolor{textcolor}{rgb}{0.000000,0.000000,0.000000}%
\pgfsetstrokecolor{textcolor}%
\pgfsetfillcolor{textcolor}%
\pgftext[x=2.562500in,y=0.152809in,,top]{\color{textcolor}\sffamily\fontsize{10.000000}{12.000000}\selectfont \(\displaystyle N_\mathrm{QMC}\)}%
\end{pgfscope}%
\begin{pgfscope}%
\pgfsetbuttcap%
\pgfsetroundjoin%
\definecolor{currentfill}{rgb}{0.000000,0.000000,0.000000}%
\pgfsetfillcolor{currentfill}%
\pgfsetlinewidth{0.803000pt}%
\definecolor{currentstroke}{rgb}{0.000000,0.000000,0.000000}%
\pgfsetstrokecolor{currentstroke}%
\pgfsetdash{}{0pt}%
\pgfsys@defobject{currentmarker}{\pgfqpoint{-0.048611in}{0.000000in}}{\pgfqpoint{0.000000in}{0.000000in}}{%
\pgfpathmoveto{\pgfqpoint{0.000000in}{0.000000in}}%
\pgfpathlineto{\pgfqpoint{-0.048611in}{0.000000in}}%
\pgfusepath{stroke,fill}%
}%
\begin{pgfscope}%
\pgfsys@transformshift{0.625000in}{0.492329in}%
\pgfsys@useobject{currentmarker}{}%
\end{pgfscope}%
\end{pgfscope}%
\begin{pgfscope}%
\definecolor{textcolor}{rgb}{0.000000,0.000000,0.000000}%
\pgfsetstrokecolor{textcolor}%
\pgfsetfillcolor{textcolor}%
\pgftext[x=0.239775in,y=0.439568in,left,base]{\color{textcolor}\sffamily\fontsize{10.000000}{12.000000}\selectfont \(\displaystyle {10^{-4}}\)}%
\end{pgfscope}%
\begin{pgfscope}%
\pgfsetbuttcap%
\pgfsetroundjoin%
\definecolor{currentfill}{rgb}{0.000000,0.000000,0.000000}%
\pgfsetfillcolor{currentfill}%
\pgfsetlinewidth{0.803000pt}%
\definecolor{currentstroke}{rgb}{0.000000,0.000000,0.000000}%
\pgfsetstrokecolor{currentstroke}%
\pgfsetdash{}{0pt}%
\pgfsys@defobject{currentmarker}{\pgfqpoint{-0.048611in}{0.000000in}}{\pgfqpoint{0.000000in}{0.000000in}}{%
\pgfpathmoveto{\pgfqpoint{0.000000in}{0.000000in}}%
\pgfpathlineto{\pgfqpoint{-0.048611in}{0.000000in}}%
\pgfusepath{stroke,fill}%
}%
\begin{pgfscope}%
\pgfsys@transformshift{0.625000in}{1.378741in}%
\pgfsys@useobject{currentmarker}{}%
\end{pgfscope}%
\end{pgfscope}%
\begin{pgfscope}%
\definecolor{textcolor}{rgb}{0.000000,0.000000,0.000000}%
\pgfsetstrokecolor{textcolor}%
\pgfsetfillcolor{textcolor}%
\pgftext[x=0.239775in,y=1.325979in,left,base]{\color{textcolor}\sffamily\fontsize{10.000000}{12.000000}\selectfont \(\displaystyle {10^{-3}}\)}%
\end{pgfscope}%
\begin{pgfscope}%
\pgfsetbuttcap%
\pgfsetroundjoin%
\definecolor{currentfill}{rgb}{0.000000,0.000000,0.000000}%
\pgfsetfillcolor{currentfill}%
\pgfsetlinewidth{0.803000pt}%
\definecolor{currentstroke}{rgb}{0.000000,0.000000,0.000000}%
\pgfsetstrokecolor{currentstroke}%
\pgfsetdash{}{0pt}%
\pgfsys@defobject{currentmarker}{\pgfqpoint{-0.048611in}{0.000000in}}{\pgfqpoint{0.000000in}{0.000000in}}{%
\pgfpathmoveto{\pgfqpoint{0.000000in}{0.000000in}}%
\pgfpathlineto{\pgfqpoint{-0.048611in}{0.000000in}}%
\pgfusepath{stroke,fill}%
}%
\begin{pgfscope}%
\pgfsys@transformshift{0.625000in}{2.265152in}%
\pgfsys@useobject{currentmarker}{}%
\end{pgfscope}%
\end{pgfscope}%
\begin{pgfscope}%
\definecolor{textcolor}{rgb}{0.000000,0.000000,0.000000}%
\pgfsetstrokecolor{textcolor}%
\pgfsetfillcolor{textcolor}%
\pgftext[x=0.239775in,y=2.212391in,left,base]{\color{textcolor}\sffamily\fontsize{10.000000}{12.000000}\selectfont \(\displaystyle {10^{-2}}\)}%
\end{pgfscope}%
\begin{pgfscope}%
\pgfsetbuttcap%
\pgfsetroundjoin%
\definecolor{currentfill}{rgb}{0.000000,0.000000,0.000000}%
\pgfsetfillcolor{currentfill}%
\pgfsetlinewidth{0.803000pt}%
\definecolor{currentstroke}{rgb}{0.000000,0.000000,0.000000}%
\pgfsetstrokecolor{currentstroke}%
\pgfsetdash{}{0pt}%
\pgfsys@defobject{currentmarker}{\pgfqpoint{-0.048611in}{0.000000in}}{\pgfqpoint{0.000000in}{0.000000in}}{%
\pgfpathmoveto{\pgfqpoint{0.000000in}{0.000000in}}%
\pgfpathlineto{\pgfqpoint{-0.048611in}{0.000000in}}%
\pgfusepath{stroke,fill}%
}%
\begin{pgfscope}%
\pgfsys@transformshift{0.625000in}{3.151564in}%
\pgfsys@useobject{currentmarker}{}%
\end{pgfscope}%
\end{pgfscope}%
\begin{pgfscope}%
\definecolor{textcolor}{rgb}{0.000000,0.000000,0.000000}%
\pgfsetstrokecolor{textcolor}%
\pgfsetfillcolor{textcolor}%
\pgftext[x=0.239775in,y=3.098802in,left,base]{\color{textcolor}\sffamily\fontsize{10.000000}{12.000000}\selectfont \(\displaystyle {10^{-1}}\)}%
\end{pgfscope}%
\begin{pgfscope}%
\pgfsetbuttcap%
\pgfsetroundjoin%
\definecolor{currentfill}{rgb}{0.000000,0.000000,0.000000}%
\pgfsetfillcolor{currentfill}%
\pgfsetlinewidth{0.602250pt}%
\definecolor{currentstroke}{rgb}{0.000000,0.000000,0.000000}%
\pgfsetstrokecolor{currentstroke}%
\pgfsetdash{}{0pt}%
\pgfsys@defobject{currentmarker}{\pgfqpoint{-0.027778in}{0.000000in}}{\pgfqpoint{0.000000in}{0.000000in}}{%
\pgfpathmoveto{\pgfqpoint{0.000000in}{0.000000in}}%
\pgfpathlineto{\pgfqpoint{-0.027778in}{0.000000in}}%
\pgfusepath{stroke,fill}%
}%
\begin{pgfscope}%
\pgfsys@transformshift{0.625000in}{0.451769in}%
\pgfsys@useobject{currentmarker}{}%
\end{pgfscope}%
\end{pgfscope}%
\begin{pgfscope}%
\pgfsetbuttcap%
\pgfsetroundjoin%
\definecolor{currentfill}{rgb}{0.000000,0.000000,0.000000}%
\pgfsetfillcolor{currentfill}%
\pgfsetlinewidth{0.602250pt}%
\definecolor{currentstroke}{rgb}{0.000000,0.000000,0.000000}%
\pgfsetstrokecolor{currentstroke}%
\pgfsetdash{}{0pt}%
\pgfsys@defobject{currentmarker}{\pgfqpoint{-0.027778in}{0.000000in}}{\pgfqpoint{0.000000in}{0.000000in}}{%
\pgfpathmoveto{\pgfqpoint{0.000000in}{0.000000in}}%
\pgfpathlineto{\pgfqpoint{-0.027778in}{0.000000in}}%
\pgfusepath{stroke,fill}%
}%
\begin{pgfscope}%
\pgfsys@transformshift{0.625000in}{0.759166in}%
\pgfsys@useobject{currentmarker}{}%
\end{pgfscope}%
\end{pgfscope}%
\begin{pgfscope}%
\pgfsetbuttcap%
\pgfsetroundjoin%
\definecolor{currentfill}{rgb}{0.000000,0.000000,0.000000}%
\pgfsetfillcolor{currentfill}%
\pgfsetlinewidth{0.602250pt}%
\definecolor{currentstroke}{rgb}{0.000000,0.000000,0.000000}%
\pgfsetstrokecolor{currentstroke}%
\pgfsetdash{}{0pt}%
\pgfsys@defobject{currentmarker}{\pgfqpoint{-0.027778in}{0.000000in}}{\pgfqpoint{0.000000in}{0.000000in}}{%
\pgfpathmoveto{\pgfqpoint{0.000000in}{0.000000in}}%
\pgfpathlineto{\pgfqpoint{-0.027778in}{0.000000in}}%
\pgfusepath{stroke,fill}%
}%
\begin{pgfscope}%
\pgfsys@transformshift{0.625000in}{0.915255in}%
\pgfsys@useobject{currentmarker}{}%
\end{pgfscope}%
\end{pgfscope}%
\begin{pgfscope}%
\pgfsetbuttcap%
\pgfsetroundjoin%
\definecolor{currentfill}{rgb}{0.000000,0.000000,0.000000}%
\pgfsetfillcolor{currentfill}%
\pgfsetlinewidth{0.602250pt}%
\definecolor{currentstroke}{rgb}{0.000000,0.000000,0.000000}%
\pgfsetstrokecolor{currentstroke}%
\pgfsetdash{}{0pt}%
\pgfsys@defobject{currentmarker}{\pgfqpoint{-0.027778in}{0.000000in}}{\pgfqpoint{0.000000in}{0.000000in}}{%
\pgfpathmoveto{\pgfqpoint{0.000000in}{0.000000in}}%
\pgfpathlineto{\pgfqpoint{-0.027778in}{0.000000in}}%
\pgfusepath{stroke,fill}%
}%
\begin{pgfscope}%
\pgfsys@transformshift{0.625000in}{1.026002in}%
\pgfsys@useobject{currentmarker}{}%
\end{pgfscope}%
\end{pgfscope}%
\begin{pgfscope}%
\pgfsetbuttcap%
\pgfsetroundjoin%
\definecolor{currentfill}{rgb}{0.000000,0.000000,0.000000}%
\pgfsetfillcolor{currentfill}%
\pgfsetlinewidth{0.602250pt}%
\definecolor{currentstroke}{rgb}{0.000000,0.000000,0.000000}%
\pgfsetstrokecolor{currentstroke}%
\pgfsetdash{}{0pt}%
\pgfsys@defobject{currentmarker}{\pgfqpoint{-0.027778in}{0.000000in}}{\pgfqpoint{0.000000in}{0.000000in}}{%
\pgfpathmoveto{\pgfqpoint{0.000000in}{0.000000in}}%
\pgfpathlineto{\pgfqpoint{-0.027778in}{0.000000in}}%
\pgfusepath{stroke,fill}%
}%
\begin{pgfscope}%
\pgfsys@transformshift{0.625000in}{1.111904in}%
\pgfsys@useobject{currentmarker}{}%
\end{pgfscope}%
\end{pgfscope}%
\begin{pgfscope}%
\pgfsetbuttcap%
\pgfsetroundjoin%
\definecolor{currentfill}{rgb}{0.000000,0.000000,0.000000}%
\pgfsetfillcolor{currentfill}%
\pgfsetlinewidth{0.602250pt}%
\definecolor{currentstroke}{rgb}{0.000000,0.000000,0.000000}%
\pgfsetstrokecolor{currentstroke}%
\pgfsetdash{}{0pt}%
\pgfsys@defobject{currentmarker}{\pgfqpoint{-0.027778in}{0.000000in}}{\pgfqpoint{0.000000in}{0.000000in}}{%
\pgfpathmoveto{\pgfqpoint{0.000000in}{0.000000in}}%
\pgfpathlineto{\pgfqpoint{-0.027778in}{0.000000in}}%
\pgfusepath{stroke,fill}%
}%
\begin{pgfscope}%
\pgfsys@transformshift{0.625000in}{1.182091in}%
\pgfsys@useobject{currentmarker}{}%
\end{pgfscope}%
\end{pgfscope}%
\begin{pgfscope}%
\pgfsetbuttcap%
\pgfsetroundjoin%
\definecolor{currentfill}{rgb}{0.000000,0.000000,0.000000}%
\pgfsetfillcolor{currentfill}%
\pgfsetlinewidth{0.602250pt}%
\definecolor{currentstroke}{rgb}{0.000000,0.000000,0.000000}%
\pgfsetstrokecolor{currentstroke}%
\pgfsetdash{}{0pt}%
\pgfsys@defobject{currentmarker}{\pgfqpoint{-0.027778in}{0.000000in}}{\pgfqpoint{0.000000in}{0.000000in}}{%
\pgfpathmoveto{\pgfqpoint{0.000000in}{0.000000in}}%
\pgfpathlineto{\pgfqpoint{-0.027778in}{0.000000in}}%
\pgfusepath{stroke,fill}%
}%
\begin{pgfscope}%
\pgfsys@transformshift{0.625000in}{1.241434in}%
\pgfsys@useobject{currentmarker}{}%
\end{pgfscope}%
\end{pgfscope}%
\begin{pgfscope}%
\pgfsetbuttcap%
\pgfsetroundjoin%
\definecolor{currentfill}{rgb}{0.000000,0.000000,0.000000}%
\pgfsetfillcolor{currentfill}%
\pgfsetlinewidth{0.602250pt}%
\definecolor{currentstroke}{rgb}{0.000000,0.000000,0.000000}%
\pgfsetstrokecolor{currentstroke}%
\pgfsetdash{}{0pt}%
\pgfsys@defobject{currentmarker}{\pgfqpoint{-0.027778in}{0.000000in}}{\pgfqpoint{0.000000in}{0.000000in}}{%
\pgfpathmoveto{\pgfqpoint{0.000000in}{0.000000in}}%
\pgfpathlineto{\pgfqpoint{-0.027778in}{0.000000in}}%
\pgfusepath{stroke,fill}%
}%
\begin{pgfscope}%
\pgfsys@transformshift{0.625000in}{1.292839in}%
\pgfsys@useobject{currentmarker}{}%
\end{pgfscope}%
\end{pgfscope}%
\begin{pgfscope}%
\pgfsetbuttcap%
\pgfsetroundjoin%
\definecolor{currentfill}{rgb}{0.000000,0.000000,0.000000}%
\pgfsetfillcolor{currentfill}%
\pgfsetlinewidth{0.602250pt}%
\definecolor{currentstroke}{rgb}{0.000000,0.000000,0.000000}%
\pgfsetstrokecolor{currentstroke}%
\pgfsetdash{}{0pt}%
\pgfsys@defobject{currentmarker}{\pgfqpoint{-0.027778in}{0.000000in}}{\pgfqpoint{0.000000in}{0.000000in}}{%
\pgfpathmoveto{\pgfqpoint{0.000000in}{0.000000in}}%
\pgfpathlineto{\pgfqpoint{-0.027778in}{0.000000in}}%
\pgfusepath{stroke,fill}%
}%
\begin{pgfscope}%
\pgfsys@transformshift{0.625000in}{1.338181in}%
\pgfsys@useobject{currentmarker}{}%
\end{pgfscope}%
\end{pgfscope}%
\begin{pgfscope}%
\pgfsetbuttcap%
\pgfsetroundjoin%
\definecolor{currentfill}{rgb}{0.000000,0.000000,0.000000}%
\pgfsetfillcolor{currentfill}%
\pgfsetlinewidth{0.602250pt}%
\definecolor{currentstroke}{rgb}{0.000000,0.000000,0.000000}%
\pgfsetstrokecolor{currentstroke}%
\pgfsetdash{}{0pt}%
\pgfsys@defobject{currentmarker}{\pgfqpoint{-0.027778in}{0.000000in}}{\pgfqpoint{0.000000in}{0.000000in}}{%
\pgfpathmoveto{\pgfqpoint{0.000000in}{0.000000in}}%
\pgfpathlineto{\pgfqpoint{-0.027778in}{0.000000in}}%
\pgfusepath{stroke,fill}%
}%
\begin{pgfscope}%
\pgfsys@transformshift{0.625000in}{1.645577in}%
\pgfsys@useobject{currentmarker}{}%
\end{pgfscope}%
\end{pgfscope}%
\begin{pgfscope}%
\pgfsetbuttcap%
\pgfsetroundjoin%
\definecolor{currentfill}{rgb}{0.000000,0.000000,0.000000}%
\pgfsetfillcolor{currentfill}%
\pgfsetlinewidth{0.602250pt}%
\definecolor{currentstroke}{rgb}{0.000000,0.000000,0.000000}%
\pgfsetstrokecolor{currentstroke}%
\pgfsetdash{}{0pt}%
\pgfsys@defobject{currentmarker}{\pgfqpoint{-0.027778in}{0.000000in}}{\pgfqpoint{0.000000in}{0.000000in}}{%
\pgfpathmoveto{\pgfqpoint{0.000000in}{0.000000in}}%
\pgfpathlineto{\pgfqpoint{-0.027778in}{0.000000in}}%
\pgfusepath{stroke,fill}%
}%
\begin{pgfscope}%
\pgfsys@transformshift{0.625000in}{1.801666in}%
\pgfsys@useobject{currentmarker}{}%
\end{pgfscope}%
\end{pgfscope}%
\begin{pgfscope}%
\pgfsetbuttcap%
\pgfsetroundjoin%
\definecolor{currentfill}{rgb}{0.000000,0.000000,0.000000}%
\pgfsetfillcolor{currentfill}%
\pgfsetlinewidth{0.602250pt}%
\definecolor{currentstroke}{rgb}{0.000000,0.000000,0.000000}%
\pgfsetstrokecolor{currentstroke}%
\pgfsetdash{}{0pt}%
\pgfsys@defobject{currentmarker}{\pgfqpoint{-0.027778in}{0.000000in}}{\pgfqpoint{0.000000in}{0.000000in}}{%
\pgfpathmoveto{\pgfqpoint{0.000000in}{0.000000in}}%
\pgfpathlineto{\pgfqpoint{-0.027778in}{0.000000in}}%
\pgfusepath{stroke,fill}%
}%
\begin{pgfscope}%
\pgfsys@transformshift{0.625000in}{1.912414in}%
\pgfsys@useobject{currentmarker}{}%
\end{pgfscope}%
\end{pgfscope}%
\begin{pgfscope}%
\pgfsetbuttcap%
\pgfsetroundjoin%
\definecolor{currentfill}{rgb}{0.000000,0.000000,0.000000}%
\pgfsetfillcolor{currentfill}%
\pgfsetlinewidth{0.602250pt}%
\definecolor{currentstroke}{rgb}{0.000000,0.000000,0.000000}%
\pgfsetstrokecolor{currentstroke}%
\pgfsetdash{}{0pt}%
\pgfsys@defobject{currentmarker}{\pgfqpoint{-0.027778in}{0.000000in}}{\pgfqpoint{0.000000in}{0.000000in}}{%
\pgfpathmoveto{\pgfqpoint{0.000000in}{0.000000in}}%
\pgfpathlineto{\pgfqpoint{-0.027778in}{0.000000in}}%
\pgfusepath{stroke,fill}%
}%
\begin{pgfscope}%
\pgfsys@transformshift{0.625000in}{1.998316in}%
\pgfsys@useobject{currentmarker}{}%
\end{pgfscope}%
\end{pgfscope}%
\begin{pgfscope}%
\pgfsetbuttcap%
\pgfsetroundjoin%
\definecolor{currentfill}{rgb}{0.000000,0.000000,0.000000}%
\pgfsetfillcolor{currentfill}%
\pgfsetlinewidth{0.602250pt}%
\definecolor{currentstroke}{rgb}{0.000000,0.000000,0.000000}%
\pgfsetstrokecolor{currentstroke}%
\pgfsetdash{}{0pt}%
\pgfsys@defobject{currentmarker}{\pgfqpoint{-0.027778in}{0.000000in}}{\pgfqpoint{0.000000in}{0.000000in}}{%
\pgfpathmoveto{\pgfqpoint{0.000000in}{0.000000in}}%
\pgfpathlineto{\pgfqpoint{-0.027778in}{0.000000in}}%
\pgfusepath{stroke,fill}%
}%
\begin{pgfscope}%
\pgfsys@transformshift{0.625000in}{2.068503in}%
\pgfsys@useobject{currentmarker}{}%
\end{pgfscope}%
\end{pgfscope}%
\begin{pgfscope}%
\pgfsetbuttcap%
\pgfsetroundjoin%
\definecolor{currentfill}{rgb}{0.000000,0.000000,0.000000}%
\pgfsetfillcolor{currentfill}%
\pgfsetlinewidth{0.602250pt}%
\definecolor{currentstroke}{rgb}{0.000000,0.000000,0.000000}%
\pgfsetstrokecolor{currentstroke}%
\pgfsetdash{}{0pt}%
\pgfsys@defobject{currentmarker}{\pgfqpoint{-0.027778in}{0.000000in}}{\pgfqpoint{0.000000in}{0.000000in}}{%
\pgfpathmoveto{\pgfqpoint{0.000000in}{0.000000in}}%
\pgfpathlineto{\pgfqpoint{-0.027778in}{0.000000in}}%
\pgfusepath{stroke,fill}%
}%
\begin{pgfscope}%
\pgfsys@transformshift{0.625000in}{2.127845in}%
\pgfsys@useobject{currentmarker}{}%
\end{pgfscope}%
\end{pgfscope}%
\begin{pgfscope}%
\pgfsetbuttcap%
\pgfsetroundjoin%
\definecolor{currentfill}{rgb}{0.000000,0.000000,0.000000}%
\pgfsetfillcolor{currentfill}%
\pgfsetlinewidth{0.602250pt}%
\definecolor{currentstroke}{rgb}{0.000000,0.000000,0.000000}%
\pgfsetstrokecolor{currentstroke}%
\pgfsetdash{}{0pt}%
\pgfsys@defobject{currentmarker}{\pgfqpoint{-0.027778in}{0.000000in}}{\pgfqpoint{0.000000in}{0.000000in}}{%
\pgfpathmoveto{\pgfqpoint{0.000000in}{0.000000in}}%
\pgfpathlineto{\pgfqpoint{-0.027778in}{0.000000in}}%
\pgfusepath{stroke,fill}%
}%
\begin{pgfscope}%
\pgfsys@transformshift{0.625000in}{2.179250in}%
\pgfsys@useobject{currentmarker}{}%
\end{pgfscope}%
\end{pgfscope}%
\begin{pgfscope}%
\pgfsetbuttcap%
\pgfsetroundjoin%
\definecolor{currentfill}{rgb}{0.000000,0.000000,0.000000}%
\pgfsetfillcolor{currentfill}%
\pgfsetlinewidth{0.602250pt}%
\definecolor{currentstroke}{rgb}{0.000000,0.000000,0.000000}%
\pgfsetstrokecolor{currentstroke}%
\pgfsetdash{}{0pt}%
\pgfsys@defobject{currentmarker}{\pgfqpoint{-0.027778in}{0.000000in}}{\pgfqpoint{0.000000in}{0.000000in}}{%
\pgfpathmoveto{\pgfqpoint{0.000000in}{0.000000in}}%
\pgfpathlineto{\pgfqpoint{-0.027778in}{0.000000in}}%
\pgfusepath{stroke,fill}%
}%
\begin{pgfscope}%
\pgfsys@transformshift{0.625000in}{2.224592in}%
\pgfsys@useobject{currentmarker}{}%
\end{pgfscope}%
\end{pgfscope}%
\begin{pgfscope}%
\pgfsetbuttcap%
\pgfsetroundjoin%
\definecolor{currentfill}{rgb}{0.000000,0.000000,0.000000}%
\pgfsetfillcolor{currentfill}%
\pgfsetlinewidth{0.602250pt}%
\definecolor{currentstroke}{rgb}{0.000000,0.000000,0.000000}%
\pgfsetstrokecolor{currentstroke}%
\pgfsetdash{}{0pt}%
\pgfsys@defobject{currentmarker}{\pgfqpoint{-0.027778in}{0.000000in}}{\pgfqpoint{0.000000in}{0.000000in}}{%
\pgfpathmoveto{\pgfqpoint{0.000000in}{0.000000in}}%
\pgfpathlineto{\pgfqpoint{-0.027778in}{0.000000in}}%
\pgfusepath{stroke,fill}%
}%
\begin{pgfscope}%
\pgfsys@transformshift{0.625000in}{2.531989in}%
\pgfsys@useobject{currentmarker}{}%
\end{pgfscope}%
\end{pgfscope}%
\begin{pgfscope}%
\pgfsetbuttcap%
\pgfsetroundjoin%
\definecolor{currentfill}{rgb}{0.000000,0.000000,0.000000}%
\pgfsetfillcolor{currentfill}%
\pgfsetlinewidth{0.602250pt}%
\definecolor{currentstroke}{rgb}{0.000000,0.000000,0.000000}%
\pgfsetstrokecolor{currentstroke}%
\pgfsetdash{}{0pt}%
\pgfsys@defobject{currentmarker}{\pgfqpoint{-0.027778in}{0.000000in}}{\pgfqpoint{0.000000in}{0.000000in}}{%
\pgfpathmoveto{\pgfqpoint{0.000000in}{0.000000in}}%
\pgfpathlineto{\pgfqpoint{-0.027778in}{0.000000in}}%
\pgfusepath{stroke,fill}%
}%
\begin{pgfscope}%
\pgfsys@transformshift{0.625000in}{2.688078in}%
\pgfsys@useobject{currentmarker}{}%
\end{pgfscope}%
\end{pgfscope}%
\begin{pgfscope}%
\pgfsetbuttcap%
\pgfsetroundjoin%
\definecolor{currentfill}{rgb}{0.000000,0.000000,0.000000}%
\pgfsetfillcolor{currentfill}%
\pgfsetlinewidth{0.602250pt}%
\definecolor{currentstroke}{rgb}{0.000000,0.000000,0.000000}%
\pgfsetstrokecolor{currentstroke}%
\pgfsetdash{}{0pt}%
\pgfsys@defobject{currentmarker}{\pgfqpoint{-0.027778in}{0.000000in}}{\pgfqpoint{0.000000in}{0.000000in}}{%
\pgfpathmoveto{\pgfqpoint{0.000000in}{0.000000in}}%
\pgfpathlineto{\pgfqpoint{-0.027778in}{0.000000in}}%
\pgfusepath{stroke,fill}%
}%
\begin{pgfscope}%
\pgfsys@transformshift{0.625000in}{2.798825in}%
\pgfsys@useobject{currentmarker}{}%
\end{pgfscope}%
\end{pgfscope}%
\begin{pgfscope}%
\pgfsetbuttcap%
\pgfsetroundjoin%
\definecolor{currentfill}{rgb}{0.000000,0.000000,0.000000}%
\pgfsetfillcolor{currentfill}%
\pgfsetlinewidth{0.602250pt}%
\definecolor{currentstroke}{rgb}{0.000000,0.000000,0.000000}%
\pgfsetstrokecolor{currentstroke}%
\pgfsetdash{}{0pt}%
\pgfsys@defobject{currentmarker}{\pgfqpoint{-0.027778in}{0.000000in}}{\pgfqpoint{0.000000in}{0.000000in}}{%
\pgfpathmoveto{\pgfqpoint{0.000000in}{0.000000in}}%
\pgfpathlineto{\pgfqpoint{-0.027778in}{0.000000in}}%
\pgfusepath{stroke,fill}%
}%
\begin{pgfscope}%
\pgfsys@transformshift{0.625000in}{2.884727in}%
\pgfsys@useobject{currentmarker}{}%
\end{pgfscope}%
\end{pgfscope}%
\begin{pgfscope}%
\pgfsetbuttcap%
\pgfsetroundjoin%
\definecolor{currentfill}{rgb}{0.000000,0.000000,0.000000}%
\pgfsetfillcolor{currentfill}%
\pgfsetlinewidth{0.602250pt}%
\definecolor{currentstroke}{rgb}{0.000000,0.000000,0.000000}%
\pgfsetstrokecolor{currentstroke}%
\pgfsetdash{}{0pt}%
\pgfsys@defobject{currentmarker}{\pgfqpoint{-0.027778in}{0.000000in}}{\pgfqpoint{0.000000in}{0.000000in}}{%
\pgfpathmoveto{\pgfqpoint{0.000000in}{0.000000in}}%
\pgfpathlineto{\pgfqpoint{-0.027778in}{0.000000in}}%
\pgfusepath{stroke,fill}%
}%
\begin{pgfscope}%
\pgfsys@transformshift{0.625000in}{2.954914in}%
\pgfsys@useobject{currentmarker}{}%
\end{pgfscope}%
\end{pgfscope}%
\begin{pgfscope}%
\pgfsetbuttcap%
\pgfsetroundjoin%
\definecolor{currentfill}{rgb}{0.000000,0.000000,0.000000}%
\pgfsetfillcolor{currentfill}%
\pgfsetlinewidth{0.602250pt}%
\definecolor{currentstroke}{rgb}{0.000000,0.000000,0.000000}%
\pgfsetstrokecolor{currentstroke}%
\pgfsetdash{}{0pt}%
\pgfsys@defobject{currentmarker}{\pgfqpoint{-0.027778in}{0.000000in}}{\pgfqpoint{0.000000in}{0.000000in}}{%
\pgfpathmoveto{\pgfqpoint{0.000000in}{0.000000in}}%
\pgfpathlineto{\pgfqpoint{-0.027778in}{0.000000in}}%
\pgfusepath{stroke,fill}%
}%
\begin{pgfscope}%
\pgfsys@transformshift{0.625000in}{3.014257in}%
\pgfsys@useobject{currentmarker}{}%
\end{pgfscope}%
\end{pgfscope}%
\begin{pgfscope}%
\pgfsetbuttcap%
\pgfsetroundjoin%
\definecolor{currentfill}{rgb}{0.000000,0.000000,0.000000}%
\pgfsetfillcolor{currentfill}%
\pgfsetlinewidth{0.602250pt}%
\definecolor{currentstroke}{rgb}{0.000000,0.000000,0.000000}%
\pgfsetstrokecolor{currentstroke}%
\pgfsetdash{}{0pt}%
\pgfsys@defobject{currentmarker}{\pgfqpoint{-0.027778in}{0.000000in}}{\pgfqpoint{0.000000in}{0.000000in}}{%
\pgfpathmoveto{\pgfqpoint{0.000000in}{0.000000in}}%
\pgfpathlineto{\pgfqpoint{-0.027778in}{0.000000in}}%
\pgfusepath{stroke,fill}%
}%
\begin{pgfscope}%
\pgfsys@transformshift{0.625000in}{3.065662in}%
\pgfsys@useobject{currentmarker}{}%
\end{pgfscope}%
\end{pgfscope}%
\begin{pgfscope}%
\pgfsetbuttcap%
\pgfsetroundjoin%
\definecolor{currentfill}{rgb}{0.000000,0.000000,0.000000}%
\pgfsetfillcolor{currentfill}%
\pgfsetlinewidth{0.602250pt}%
\definecolor{currentstroke}{rgb}{0.000000,0.000000,0.000000}%
\pgfsetstrokecolor{currentstroke}%
\pgfsetdash{}{0pt}%
\pgfsys@defobject{currentmarker}{\pgfqpoint{-0.027778in}{0.000000in}}{\pgfqpoint{0.000000in}{0.000000in}}{%
\pgfpathmoveto{\pgfqpoint{0.000000in}{0.000000in}}%
\pgfpathlineto{\pgfqpoint{-0.027778in}{0.000000in}}%
\pgfusepath{stroke,fill}%
}%
\begin{pgfscope}%
\pgfsys@transformshift{0.625000in}{3.111004in}%
\pgfsys@useobject{currentmarker}{}%
\end{pgfscope}%
\end{pgfscope}%
\begin{pgfscope}%
\pgfsetbuttcap%
\pgfsetroundjoin%
\definecolor{currentfill}{rgb}{0.000000,0.000000,0.000000}%
\pgfsetfillcolor{currentfill}%
\pgfsetlinewidth{0.602250pt}%
\definecolor{currentstroke}{rgb}{0.000000,0.000000,0.000000}%
\pgfsetstrokecolor{currentstroke}%
\pgfsetdash{}{0pt}%
\pgfsys@defobject{currentmarker}{\pgfqpoint{-0.027778in}{0.000000in}}{\pgfqpoint{0.000000in}{0.000000in}}{%
\pgfpathmoveto{\pgfqpoint{0.000000in}{0.000000in}}%
\pgfpathlineto{\pgfqpoint{-0.027778in}{0.000000in}}%
\pgfusepath{stroke,fill}%
}%
\begin{pgfscope}%
\pgfsys@transformshift{0.625000in}{3.418400in}%
\pgfsys@useobject{currentmarker}{}%
\end{pgfscope}%
\end{pgfscope}%
\begin{pgfscope}%
\definecolor{textcolor}{rgb}{0.000000,0.000000,0.000000}%
\pgfsetstrokecolor{textcolor}%
\pgfsetfillcolor{textcolor}%
\pgftext[x=0.184220in,y=1.980000in,,bottom,rotate=90.000000]{\color{textcolor}\sffamily\fontsize{10.000000}{12.000000}\selectfont QMC Error}%
\end{pgfscope}%
\begin{pgfscope}%
\pgfpathrectangle{\pgfqpoint{0.625000in}{0.440000in}}{\pgfqpoint{3.875000in}{3.080000in}}%
\pgfusepath{clip}%
\pgfsetbuttcap%
\pgfsetroundjoin%
\definecolor{currentfill}{rgb}{0.000000,0.000000,0.000000}%
\pgfsetfillcolor{currentfill}%
\pgfsetlinewidth{1.003750pt}%
\definecolor{currentstroke}{rgb}{0.000000,0.000000,0.000000}%
\pgfsetstrokecolor{currentstroke}%
\pgfsetdash{}{0pt}%
\pgfsys@defobject{currentmarker}{\pgfqpoint{-0.041667in}{-0.041667in}}{\pgfqpoint{0.041667in}{0.041667in}}{%
\pgfpathmoveto{\pgfqpoint{0.000000in}{-0.041667in}}%
\pgfpathcurveto{\pgfqpoint{0.011050in}{-0.041667in}}{\pgfqpoint{0.021649in}{-0.037276in}}{\pgfqpoint{0.029463in}{-0.029463in}}%
\pgfpathcurveto{\pgfqpoint{0.037276in}{-0.021649in}}{\pgfqpoint{0.041667in}{-0.011050in}}{\pgfqpoint{0.041667in}{0.000000in}}%
\pgfpathcurveto{\pgfqpoint{0.041667in}{0.011050in}}{\pgfqpoint{0.037276in}{0.021649in}}{\pgfqpoint{0.029463in}{0.029463in}}%
\pgfpathcurveto{\pgfqpoint{0.021649in}{0.037276in}}{\pgfqpoint{0.011050in}{0.041667in}}{\pgfqpoint{0.000000in}{0.041667in}}%
\pgfpathcurveto{\pgfqpoint{-0.011050in}{0.041667in}}{\pgfqpoint{-0.021649in}{0.037276in}}{\pgfqpoint{-0.029463in}{0.029463in}}%
\pgfpathcurveto{\pgfqpoint{-0.037276in}{0.021649in}}{\pgfqpoint{-0.041667in}{0.011050in}}{\pgfqpoint{-0.041667in}{0.000000in}}%
\pgfpathcurveto{\pgfqpoint{-0.041667in}{-0.011050in}}{\pgfqpoint{-0.037276in}{-0.021649in}}{\pgfqpoint{-0.029463in}{-0.029463in}}%
\pgfpathcurveto{\pgfqpoint{-0.021649in}{-0.037276in}}{\pgfqpoint{-0.011050in}{-0.041667in}}{\pgfqpoint{0.000000in}{-0.041667in}}%
\pgfpathclose%
\pgfusepath{stroke,fill}%
}%
\begin{pgfscope}%
\pgfsys@transformshift{0.801136in}{3.323170in}%
\pgfsys@useobject{currentmarker}{}%
\end{pgfscope}%
\begin{pgfscope}%
\pgfsys@transformshift{1.121384in}{3.233277in}%
\pgfsys@useobject{currentmarker}{}%
\end{pgfscope}%
\begin{pgfscope}%
\pgfsys@transformshift{1.441632in}{2.750228in}%
\pgfsys@useobject{currentmarker}{}%
\end{pgfscope}%
\begin{pgfscope}%
\pgfsys@transformshift{1.761880in}{2.594917in}%
\pgfsys@useobject{currentmarker}{}%
\end{pgfscope}%
\begin{pgfscope}%
\pgfsys@transformshift{2.082128in}{2.381958in}%
\pgfsys@useobject{currentmarker}{}%
\end{pgfscope}%
\begin{pgfscope}%
\pgfsys@transformshift{2.402376in}{2.219457in}%
\pgfsys@useobject{currentmarker}{}%
\end{pgfscope}%
\begin{pgfscope}%
\pgfsys@transformshift{2.722624in}{2.132912in}%
\pgfsys@useobject{currentmarker}{}%
\end{pgfscope}%
\begin{pgfscope}%
\pgfsys@transformshift{3.042872in}{2.028213in}%
\pgfsys@useobject{currentmarker}{}%
\end{pgfscope}%
\begin{pgfscope}%
\pgfsys@transformshift{3.363120in}{1.481509in}%
\pgfsys@useobject{currentmarker}{}%
\end{pgfscope}%
\begin{pgfscope}%
\pgfsys@transformshift{3.683368in}{1.230701in}%
\pgfsys@useobject{currentmarker}{}%
\end{pgfscope}%
\begin{pgfscope}%
\pgfsys@transformshift{4.003616in}{1.002147in}%
\pgfsys@useobject{currentmarker}{}%
\end{pgfscope}%
\begin{pgfscope}%
\pgfsys@transformshift{4.323864in}{0.580000in}%
\pgfsys@useobject{currentmarker}{}%
\end{pgfscope}%
\end{pgfscope}%
\begin{pgfscope}%
\pgfpathrectangle{\pgfqpoint{0.625000in}{0.440000in}}{\pgfqpoint{3.875000in}{3.080000in}}%
\pgfusepath{clip}%
\pgfsetbuttcap%
\pgfsetroundjoin%
\pgfsetlinewidth{1.505625pt}%
\definecolor{currentstroke}{rgb}{0.000000,0.000000,0.000000}%
\pgfsetstrokecolor{currentstroke}%
\pgfsetdash{{5.550000pt}{2.400000pt}}{0.000000pt}%
\pgfpathmoveto{\pgfqpoint{0.801136in}{3.380000in}}%
\pgfpathlineto{\pgfqpoint{4.323864in}{0.779748in}}%
\pgfusepath{stroke}%
\end{pgfscope}%
\begin{pgfscope}%
\pgfsetrectcap%
\pgfsetmiterjoin%
\pgfsetlinewidth{0.803000pt}%
\definecolor{currentstroke}{rgb}{0.000000,0.000000,0.000000}%
\pgfsetstrokecolor{currentstroke}%
\pgfsetdash{}{0pt}%
\pgfpathmoveto{\pgfqpoint{0.625000in}{0.440000in}}%
\pgfpathlineto{\pgfqpoint{0.625000in}{3.520000in}}%
\pgfusepath{stroke}%
\end{pgfscope}%
\begin{pgfscope}%
\pgfsetrectcap%
\pgfsetmiterjoin%
\pgfsetlinewidth{0.803000pt}%
\definecolor{currentstroke}{rgb}{0.000000,0.000000,0.000000}%
\pgfsetstrokecolor{currentstroke}%
\pgfsetdash{}{0pt}%
\pgfpathmoveto{\pgfqpoint{4.500000in}{0.440000in}}%
\pgfpathlineto{\pgfqpoint{4.500000in}{3.520000in}}%
\pgfusepath{stroke}%
\end{pgfscope}%
\begin{pgfscope}%
\pgfsetrectcap%
\pgfsetmiterjoin%
\pgfsetlinewidth{0.803000pt}%
\definecolor{currentstroke}{rgb}{0.000000,0.000000,0.000000}%
\pgfsetstrokecolor{currentstroke}%
\pgfsetdash{}{0pt}%
\pgfpathmoveto{\pgfqpoint{0.625000in}{0.440000in}}%
\pgfpathlineto{\pgfqpoint{4.500000in}{0.440000in}}%
\pgfusepath{stroke}%
\end{pgfscope}%
\begin{pgfscope}%
\pgfsetrectcap%
\pgfsetmiterjoin%
\pgfsetlinewidth{0.803000pt}%
\definecolor{currentstroke}{rgb}{0.000000,0.000000,0.000000}%
\pgfsetstrokecolor{currentstroke}%
\pgfsetdash{}{0pt}%
\pgfpathmoveto{\pgfqpoint{0.625000in}{3.520000in}}%
\pgfpathlineto{\pgfqpoint{4.500000in}{3.520000in}}%
\pgfusepath{stroke}%
\end{pgfscope}%
\begin{pgfscope}%
\pgfsetbuttcap%
\pgfsetmiterjoin%
\definecolor{currentfill}{rgb}{1.000000,1.000000,1.000000}%
\pgfsetfillcolor{currentfill}%
\pgfsetfillopacity{0.800000}%
\pgfsetlinewidth{1.003750pt}%
\definecolor{currentstroke}{rgb}{0.800000,0.800000,0.800000}%
\pgfsetstrokecolor{currentstroke}%
\pgfsetstrokeopacity{0.800000}%
\pgfsetdash{}{0pt}%
\pgfpathmoveto{\pgfqpoint{3.042243in}{3.155267in}}%
\pgfpathlineto{\pgfqpoint{4.402778in}{3.155267in}}%
\pgfpathquadraticcurveto{\pgfqpoint{4.430556in}{3.155267in}}{\pgfqpoint{4.430556in}{3.183045in}}%
\pgfpathlineto{\pgfqpoint{4.430556in}{3.422778in}}%
\pgfpathquadraticcurveto{\pgfqpoint{4.430556in}{3.450556in}}{\pgfqpoint{4.402778in}{3.450556in}}%
\pgfpathlineto{\pgfqpoint{3.042243in}{3.450556in}}%
\pgfpathquadraticcurveto{\pgfqpoint{3.014465in}{3.450556in}}{\pgfqpoint{3.014465in}{3.422778in}}%
\pgfpathlineto{\pgfqpoint{3.014465in}{3.183045in}}%
\pgfpathquadraticcurveto{\pgfqpoint{3.014465in}{3.155267in}}{\pgfqpoint{3.042243in}{3.155267in}}%
\pgfpathclose%
\pgfusepath{stroke,fill}%
\end{pgfscope}%
\begin{pgfscope}%
\pgfsetbuttcap%
\pgfsetroundjoin%
\pgfsetlinewidth{1.505625pt}%
\definecolor{currentstroke}{rgb}{0.000000,0.000000,0.000000}%
\pgfsetstrokecolor{currentstroke}%
\pgfsetdash{{5.550000pt}{2.400000pt}}{0.000000pt}%
\pgfpathmoveto{\pgfqpoint{3.070021in}{3.321543in}}%
\pgfpathlineto{\pgfqpoint{3.347799in}{3.321543in}}%
\pgfusepath{stroke}%
\end{pgfscope}%
\begin{pgfscope}%
\definecolor{textcolor}{rgb}{0.000000,0.000000,0.000000}%
\pgfsetstrokecolor{textcolor}%
\pgfsetfillcolor{textcolor}%
\pgftext[x=3.458910in,y=3.272932in,left,base]{\color{textcolor}\sffamily\fontsize{10.000000}{12.000000}\selectfont \(\displaystyle 0.1810N_{\mathrm{QMC}}^{-0.8858}\)}%
\end{pgfscope}%
\end{pgfpicture}%
\makeatother%
\endgroup%

    \caption[Quasi-Monte-Carlo error, for $Q=u((1,1))$ and $k=30$.]{Quasi-Monte-Carlo error with increasing number of Quasi-Monte-Carlo points, for $Q=u((1,1))$ and $k=30$. \label{fig:qmctop_right30}}
  \end{figure}
\begin{figure}[h]
  \centering
%% Creator: Matplotlib, PGF backend
%%
%% To include the figure in your LaTeX document, write
%%   \input{<filename>.pgf}
%%
%% Make sure the required packages are loaded in your preamble
%%   \usepackage{pgf}
%%
%% Figures using additional raster images can only be included by \input if
%% they are in the same directory as the main LaTeX file. For loading figures
%% from other directories you can use the `import` package
%%   \usepackage{import}
%% and then include the figures with
%%   \import{<path to file>}{<filename>.pgf}
%%
%% Matplotlib used the following preamble
%%   \usepackage{fontspec}
%%   \setmainfont{DejaVuSerif.ttf}[Path=/home/owen/progs/firedrake-complex/firedrake/lib/python3.5/site-packages/matplotlib/mpl-data/fonts/ttf/]
%%   \setsansfont{DejaVuSans.ttf}[Path=/home/owen/progs/firedrake-complex/firedrake/lib/python3.5/site-packages/matplotlib/mpl-data/fonts/ttf/]
%%   \setmonofont{DejaVuSansMono.ttf}[Path=/home/owen/progs/firedrake-complex/firedrake/lib/python3.5/site-packages/matplotlib/mpl-data/fonts/ttf/]
%%
\begingroup%
\makeatletter%
\begin{pgfpicture}%
\pgfpathrectangle{\pgfpointorigin}{\pgfqpoint{5.000000in}{5.000000in}}%
\pgfusepath{use as bounding box, clip}%
\begin{pgfscope}%
\pgfsetbuttcap%
\pgfsetmiterjoin%
\definecolor{currentfill}{rgb}{1.000000,1.000000,1.000000}%
\pgfsetfillcolor{currentfill}%
\pgfsetlinewidth{0.000000pt}%
\definecolor{currentstroke}{rgb}{1.000000,1.000000,1.000000}%
\pgfsetstrokecolor{currentstroke}%
\pgfsetdash{}{0pt}%
\pgfpathmoveto{\pgfqpoint{0.000000in}{0.000000in}}%
\pgfpathlineto{\pgfqpoint{5.000000in}{0.000000in}}%
\pgfpathlineto{\pgfqpoint{5.000000in}{5.000000in}}%
\pgfpathlineto{\pgfqpoint{0.000000in}{5.000000in}}%
\pgfpathclose%
\pgfusepath{fill}%
\end{pgfscope}%
\begin{pgfscope}%
\pgfsetbuttcap%
\pgfsetmiterjoin%
\definecolor{currentfill}{rgb}{1.000000,1.000000,1.000000}%
\pgfsetfillcolor{currentfill}%
\pgfsetlinewidth{0.000000pt}%
\definecolor{currentstroke}{rgb}{0.000000,0.000000,0.000000}%
\pgfsetstrokecolor{currentstroke}%
\pgfsetstrokeopacity{0.000000}%
\pgfsetdash{}{0pt}%
\pgfpathmoveto{\pgfqpoint{0.625000in}{0.550000in}}%
\pgfpathlineto{\pgfqpoint{4.500000in}{0.550000in}}%
\pgfpathlineto{\pgfqpoint{4.500000in}{4.400000in}}%
\pgfpathlineto{\pgfqpoint{0.625000in}{4.400000in}}%
\pgfpathclose%
\pgfusepath{fill}%
\end{pgfscope}%
\begin{pgfscope}%
\pgfsetbuttcap%
\pgfsetroundjoin%
\definecolor{currentfill}{rgb}{0.000000,0.000000,0.000000}%
\pgfsetfillcolor{currentfill}%
\pgfsetlinewidth{0.803000pt}%
\definecolor{currentstroke}{rgb}{0.000000,0.000000,0.000000}%
\pgfsetstrokecolor{currentstroke}%
\pgfsetdash{}{0pt}%
\pgfsys@defobject{currentmarker}{\pgfqpoint{0.000000in}{-0.048611in}}{\pgfqpoint{0.000000in}{0.000000in}}{%
\pgfpathmoveto{\pgfqpoint{0.000000in}{0.000000in}}%
\pgfpathlineto{\pgfqpoint{0.000000in}{-0.048611in}}%
\pgfusepath{stroke,fill}%
}%
\begin{pgfscope}%
\pgfsys@transformshift{0.801136in}{0.550000in}%
\pgfsys@useobject{currentmarker}{}%
\end{pgfscope}%
\end{pgfscope}%
\begin{pgfscope}%
\definecolor{textcolor}{rgb}{0.000000,0.000000,0.000000}%
\pgfsetstrokecolor{textcolor}%
\pgfsetfillcolor{textcolor}%
\pgftext[x=0.801136in,y=0.452778in,,top]{\color{textcolor}\sffamily\fontsize{11.000000}{13.200000}\selectfont \(\displaystyle 10^{0}\)}%
\end{pgfscope}%
\begin{pgfscope}%
\pgfsetbuttcap%
\pgfsetroundjoin%
\definecolor{currentfill}{rgb}{0.000000,0.000000,0.000000}%
\pgfsetfillcolor{currentfill}%
\pgfsetlinewidth{0.803000pt}%
\definecolor{currentstroke}{rgb}{0.000000,0.000000,0.000000}%
\pgfsetstrokecolor{currentstroke}%
\pgfsetdash{}{0pt}%
\pgfsys@defobject{currentmarker}{\pgfqpoint{0.000000in}{-0.048611in}}{\pgfqpoint{0.000000in}{0.000000in}}{%
\pgfpathmoveto{\pgfqpoint{0.000000in}{0.000000in}}%
\pgfpathlineto{\pgfqpoint{0.000000in}{-0.048611in}}%
\pgfusepath{stroke,fill}%
}%
\begin{pgfscope}%
\pgfsys@transformshift{1.864977in}{0.550000in}%
\pgfsys@useobject{currentmarker}{}%
\end{pgfscope}%
\end{pgfscope}%
\begin{pgfscope}%
\definecolor{textcolor}{rgb}{0.000000,0.000000,0.000000}%
\pgfsetstrokecolor{textcolor}%
\pgfsetfillcolor{textcolor}%
\pgftext[x=1.864977in,y=0.452778in,,top]{\color{textcolor}\sffamily\fontsize{11.000000}{13.200000}\selectfont \(\displaystyle 10^{1}\)}%
\end{pgfscope}%
\begin{pgfscope}%
\pgfsetbuttcap%
\pgfsetroundjoin%
\definecolor{currentfill}{rgb}{0.000000,0.000000,0.000000}%
\pgfsetfillcolor{currentfill}%
\pgfsetlinewidth{0.803000pt}%
\definecolor{currentstroke}{rgb}{0.000000,0.000000,0.000000}%
\pgfsetstrokecolor{currentstroke}%
\pgfsetdash{}{0pt}%
\pgfsys@defobject{currentmarker}{\pgfqpoint{0.000000in}{-0.048611in}}{\pgfqpoint{0.000000in}{0.000000in}}{%
\pgfpathmoveto{\pgfqpoint{0.000000in}{0.000000in}}%
\pgfpathlineto{\pgfqpoint{0.000000in}{-0.048611in}}%
\pgfusepath{stroke,fill}%
}%
\begin{pgfscope}%
\pgfsys@transformshift{2.928818in}{0.550000in}%
\pgfsys@useobject{currentmarker}{}%
\end{pgfscope}%
\end{pgfscope}%
\begin{pgfscope}%
\definecolor{textcolor}{rgb}{0.000000,0.000000,0.000000}%
\pgfsetstrokecolor{textcolor}%
\pgfsetfillcolor{textcolor}%
\pgftext[x=2.928818in,y=0.452778in,,top]{\color{textcolor}\sffamily\fontsize{11.000000}{13.200000}\selectfont \(\displaystyle 10^{2}\)}%
\end{pgfscope}%
\begin{pgfscope}%
\pgfsetbuttcap%
\pgfsetroundjoin%
\definecolor{currentfill}{rgb}{0.000000,0.000000,0.000000}%
\pgfsetfillcolor{currentfill}%
\pgfsetlinewidth{0.803000pt}%
\definecolor{currentstroke}{rgb}{0.000000,0.000000,0.000000}%
\pgfsetstrokecolor{currentstroke}%
\pgfsetdash{}{0pt}%
\pgfsys@defobject{currentmarker}{\pgfqpoint{0.000000in}{-0.048611in}}{\pgfqpoint{0.000000in}{0.000000in}}{%
\pgfpathmoveto{\pgfqpoint{0.000000in}{0.000000in}}%
\pgfpathlineto{\pgfqpoint{0.000000in}{-0.048611in}}%
\pgfusepath{stroke,fill}%
}%
\begin{pgfscope}%
\pgfsys@transformshift{3.992658in}{0.550000in}%
\pgfsys@useobject{currentmarker}{}%
\end{pgfscope}%
\end{pgfscope}%
\begin{pgfscope}%
\definecolor{textcolor}{rgb}{0.000000,0.000000,0.000000}%
\pgfsetstrokecolor{textcolor}%
\pgfsetfillcolor{textcolor}%
\pgftext[x=3.992658in,y=0.452778in,,top]{\color{textcolor}\sffamily\fontsize{11.000000}{13.200000}\selectfont \(\displaystyle 10^{3}\)}%
\end{pgfscope}%
\begin{pgfscope}%
\pgfsetbuttcap%
\pgfsetroundjoin%
\definecolor{currentfill}{rgb}{0.000000,0.000000,0.000000}%
\pgfsetfillcolor{currentfill}%
\pgfsetlinewidth{0.602250pt}%
\definecolor{currentstroke}{rgb}{0.000000,0.000000,0.000000}%
\pgfsetstrokecolor{currentstroke}%
\pgfsetdash{}{0pt}%
\pgfsys@defobject{currentmarker}{\pgfqpoint{0.000000in}{-0.027778in}}{\pgfqpoint{0.000000in}{0.000000in}}{%
\pgfpathmoveto{\pgfqpoint{0.000000in}{0.000000in}}%
\pgfpathlineto{\pgfqpoint{0.000000in}{-0.027778in}}%
\pgfusepath{stroke,fill}%
}%
\begin{pgfscope}%
\pgfsys@transformshift{0.636345in}{0.550000in}%
\pgfsys@useobject{currentmarker}{}%
\end{pgfscope}%
\end{pgfscope}%
\begin{pgfscope}%
\pgfsetbuttcap%
\pgfsetroundjoin%
\definecolor{currentfill}{rgb}{0.000000,0.000000,0.000000}%
\pgfsetfillcolor{currentfill}%
\pgfsetlinewidth{0.602250pt}%
\definecolor{currentstroke}{rgb}{0.000000,0.000000,0.000000}%
\pgfsetstrokecolor{currentstroke}%
\pgfsetdash{}{0pt}%
\pgfsys@defobject{currentmarker}{\pgfqpoint{0.000000in}{-0.027778in}}{\pgfqpoint{0.000000in}{0.000000in}}{%
\pgfpathmoveto{\pgfqpoint{0.000000in}{0.000000in}}%
\pgfpathlineto{\pgfqpoint{0.000000in}{-0.027778in}}%
\pgfusepath{stroke,fill}%
}%
\begin{pgfscope}%
\pgfsys@transformshift{0.698040in}{0.550000in}%
\pgfsys@useobject{currentmarker}{}%
\end{pgfscope}%
\end{pgfscope}%
\begin{pgfscope}%
\pgfsetbuttcap%
\pgfsetroundjoin%
\definecolor{currentfill}{rgb}{0.000000,0.000000,0.000000}%
\pgfsetfillcolor{currentfill}%
\pgfsetlinewidth{0.602250pt}%
\definecolor{currentstroke}{rgb}{0.000000,0.000000,0.000000}%
\pgfsetstrokecolor{currentstroke}%
\pgfsetdash{}{0pt}%
\pgfsys@defobject{currentmarker}{\pgfqpoint{0.000000in}{-0.027778in}}{\pgfqpoint{0.000000in}{0.000000in}}{%
\pgfpathmoveto{\pgfqpoint{0.000000in}{0.000000in}}%
\pgfpathlineto{\pgfqpoint{0.000000in}{-0.027778in}}%
\pgfusepath{stroke,fill}%
}%
\begin{pgfscope}%
\pgfsys@transformshift{0.752458in}{0.550000in}%
\pgfsys@useobject{currentmarker}{}%
\end{pgfscope}%
\end{pgfscope}%
\begin{pgfscope}%
\pgfsetbuttcap%
\pgfsetroundjoin%
\definecolor{currentfill}{rgb}{0.000000,0.000000,0.000000}%
\pgfsetfillcolor{currentfill}%
\pgfsetlinewidth{0.602250pt}%
\definecolor{currentstroke}{rgb}{0.000000,0.000000,0.000000}%
\pgfsetstrokecolor{currentstroke}%
\pgfsetdash{}{0pt}%
\pgfsys@defobject{currentmarker}{\pgfqpoint{0.000000in}{-0.027778in}}{\pgfqpoint{0.000000in}{0.000000in}}{%
\pgfpathmoveto{\pgfqpoint{0.000000in}{0.000000in}}%
\pgfpathlineto{\pgfqpoint{0.000000in}{-0.027778in}}%
\pgfusepath{stroke,fill}%
}%
\begin{pgfscope}%
\pgfsys@transformshift{1.121384in}{0.550000in}%
\pgfsys@useobject{currentmarker}{}%
\end{pgfscope}%
\end{pgfscope}%
\begin{pgfscope}%
\pgfsetbuttcap%
\pgfsetroundjoin%
\definecolor{currentfill}{rgb}{0.000000,0.000000,0.000000}%
\pgfsetfillcolor{currentfill}%
\pgfsetlinewidth{0.602250pt}%
\definecolor{currentstroke}{rgb}{0.000000,0.000000,0.000000}%
\pgfsetstrokecolor{currentstroke}%
\pgfsetdash{}{0pt}%
\pgfsys@defobject{currentmarker}{\pgfqpoint{0.000000in}{-0.027778in}}{\pgfqpoint{0.000000in}{0.000000in}}{%
\pgfpathmoveto{\pgfqpoint{0.000000in}{0.000000in}}%
\pgfpathlineto{\pgfqpoint{0.000000in}{-0.027778in}}%
\pgfusepath{stroke,fill}%
}%
\begin{pgfscope}%
\pgfsys@transformshift{1.308717in}{0.550000in}%
\pgfsys@useobject{currentmarker}{}%
\end{pgfscope}%
\end{pgfscope}%
\begin{pgfscope}%
\pgfsetbuttcap%
\pgfsetroundjoin%
\definecolor{currentfill}{rgb}{0.000000,0.000000,0.000000}%
\pgfsetfillcolor{currentfill}%
\pgfsetlinewidth{0.602250pt}%
\definecolor{currentstroke}{rgb}{0.000000,0.000000,0.000000}%
\pgfsetstrokecolor{currentstroke}%
\pgfsetdash{}{0pt}%
\pgfsys@defobject{currentmarker}{\pgfqpoint{0.000000in}{-0.027778in}}{\pgfqpoint{0.000000in}{0.000000in}}{%
\pgfpathmoveto{\pgfqpoint{0.000000in}{0.000000in}}%
\pgfpathlineto{\pgfqpoint{0.000000in}{-0.027778in}}%
\pgfusepath{stroke,fill}%
}%
\begin{pgfscope}%
\pgfsys@transformshift{1.441632in}{0.550000in}%
\pgfsys@useobject{currentmarker}{}%
\end{pgfscope}%
\end{pgfscope}%
\begin{pgfscope}%
\pgfsetbuttcap%
\pgfsetroundjoin%
\definecolor{currentfill}{rgb}{0.000000,0.000000,0.000000}%
\pgfsetfillcolor{currentfill}%
\pgfsetlinewidth{0.602250pt}%
\definecolor{currentstroke}{rgb}{0.000000,0.000000,0.000000}%
\pgfsetstrokecolor{currentstroke}%
\pgfsetdash{}{0pt}%
\pgfsys@defobject{currentmarker}{\pgfqpoint{0.000000in}{-0.027778in}}{\pgfqpoint{0.000000in}{0.000000in}}{%
\pgfpathmoveto{\pgfqpoint{0.000000in}{0.000000in}}%
\pgfpathlineto{\pgfqpoint{0.000000in}{-0.027778in}}%
\pgfusepath{stroke,fill}%
}%
\begin{pgfscope}%
\pgfsys@transformshift{1.544729in}{0.550000in}%
\pgfsys@useobject{currentmarker}{}%
\end{pgfscope}%
\end{pgfscope}%
\begin{pgfscope}%
\pgfsetbuttcap%
\pgfsetroundjoin%
\definecolor{currentfill}{rgb}{0.000000,0.000000,0.000000}%
\pgfsetfillcolor{currentfill}%
\pgfsetlinewidth{0.602250pt}%
\definecolor{currentstroke}{rgb}{0.000000,0.000000,0.000000}%
\pgfsetstrokecolor{currentstroke}%
\pgfsetdash{}{0pt}%
\pgfsys@defobject{currentmarker}{\pgfqpoint{0.000000in}{-0.027778in}}{\pgfqpoint{0.000000in}{0.000000in}}{%
\pgfpathmoveto{\pgfqpoint{0.000000in}{0.000000in}}%
\pgfpathlineto{\pgfqpoint{0.000000in}{-0.027778in}}%
\pgfusepath{stroke,fill}%
}%
\begin{pgfscope}%
\pgfsys@transformshift{1.628965in}{0.550000in}%
\pgfsys@useobject{currentmarker}{}%
\end{pgfscope}%
\end{pgfscope}%
\begin{pgfscope}%
\pgfsetbuttcap%
\pgfsetroundjoin%
\definecolor{currentfill}{rgb}{0.000000,0.000000,0.000000}%
\pgfsetfillcolor{currentfill}%
\pgfsetlinewidth{0.602250pt}%
\definecolor{currentstroke}{rgb}{0.000000,0.000000,0.000000}%
\pgfsetstrokecolor{currentstroke}%
\pgfsetdash{}{0pt}%
\pgfsys@defobject{currentmarker}{\pgfqpoint{0.000000in}{-0.027778in}}{\pgfqpoint{0.000000in}{0.000000in}}{%
\pgfpathmoveto{\pgfqpoint{0.000000in}{0.000000in}}%
\pgfpathlineto{\pgfqpoint{0.000000in}{-0.027778in}}%
\pgfusepath{stroke,fill}%
}%
\begin{pgfscope}%
\pgfsys@transformshift{1.700186in}{0.550000in}%
\pgfsys@useobject{currentmarker}{}%
\end{pgfscope}%
\end{pgfscope}%
\begin{pgfscope}%
\pgfsetbuttcap%
\pgfsetroundjoin%
\definecolor{currentfill}{rgb}{0.000000,0.000000,0.000000}%
\pgfsetfillcolor{currentfill}%
\pgfsetlinewidth{0.602250pt}%
\definecolor{currentstroke}{rgb}{0.000000,0.000000,0.000000}%
\pgfsetstrokecolor{currentstroke}%
\pgfsetdash{}{0pt}%
\pgfsys@defobject{currentmarker}{\pgfqpoint{0.000000in}{-0.027778in}}{\pgfqpoint{0.000000in}{0.000000in}}{%
\pgfpathmoveto{\pgfqpoint{0.000000in}{0.000000in}}%
\pgfpathlineto{\pgfqpoint{0.000000in}{-0.027778in}}%
\pgfusepath{stroke,fill}%
}%
\begin{pgfscope}%
\pgfsys@transformshift{1.761880in}{0.550000in}%
\pgfsys@useobject{currentmarker}{}%
\end{pgfscope}%
\end{pgfscope}%
\begin{pgfscope}%
\pgfsetbuttcap%
\pgfsetroundjoin%
\definecolor{currentfill}{rgb}{0.000000,0.000000,0.000000}%
\pgfsetfillcolor{currentfill}%
\pgfsetlinewidth{0.602250pt}%
\definecolor{currentstroke}{rgb}{0.000000,0.000000,0.000000}%
\pgfsetstrokecolor{currentstroke}%
\pgfsetdash{}{0pt}%
\pgfsys@defobject{currentmarker}{\pgfqpoint{0.000000in}{-0.027778in}}{\pgfqpoint{0.000000in}{0.000000in}}{%
\pgfpathmoveto{\pgfqpoint{0.000000in}{0.000000in}}%
\pgfpathlineto{\pgfqpoint{0.000000in}{-0.027778in}}%
\pgfusepath{stroke,fill}%
}%
\begin{pgfscope}%
\pgfsys@transformshift{1.816298in}{0.550000in}%
\pgfsys@useobject{currentmarker}{}%
\end{pgfscope}%
\end{pgfscope}%
\begin{pgfscope}%
\pgfsetbuttcap%
\pgfsetroundjoin%
\definecolor{currentfill}{rgb}{0.000000,0.000000,0.000000}%
\pgfsetfillcolor{currentfill}%
\pgfsetlinewidth{0.602250pt}%
\definecolor{currentstroke}{rgb}{0.000000,0.000000,0.000000}%
\pgfsetstrokecolor{currentstroke}%
\pgfsetdash{}{0pt}%
\pgfsys@defobject{currentmarker}{\pgfqpoint{0.000000in}{-0.027778in}}{\pgfqpoint{0.000000in}{0.000000in}}{%
\pgfpathmoveto{\pgfqpoint{0.000000in}{0.000000in}}%
\pgfpathlineto{\pgfqpoint{0.000000in}{-0.027778in}}%
\pgfusepath{stroke,fill}%
}%
\begin{pgfscope}%
\pgfsys@transformshift{2.185225in}{0.550000in}%
\pgfsys@useobject{currentmarker}{}%
\end{pgfscope}%
\end{pgfscope}%
\begin{pgfscope}%
\pgfsetbuttcap%
\pgfsetroundjoin%
\definecolor{currentfill}{rgb}{0.000000,0.000000,0.000000}%
\pgfsetfillcolor{currentfill}%
\pgfsetlinewidth{0.602250pt}%
\definecolor{currentstroke}{rgb}{0.000000,0.000000,0.000000}%
\pgfsetstrokecolor{currentstroke}%
\pgfsetdash{}{0pt}%
\pgfsys@defobject{currentmarker}{\pgfqpoint{0.000000in}{-0.027778in}}{\pgfqpoint{0.000000in}{0.000000in}}{%
\pgfpathmoveto{\pgfqpoint{0.000000in}{0.000000in}}%
\pgfpathlineto{\pgfqpoint{0.000000in}{-0.027778in}}%
\pgfusepath{stroke,fill}%
}%
\begin{pgfscope}%
\pgfsys@transformshift{2.372558in}{0.550000in}%
\pgfsys@useobject{currentmarker}{}%
\end{pgfscope}%
\end{pgfscope}%
\begin{pgfscope}%
\pgfsetbuttcap%
\pgfsetroundjoin%
\definecolor{currentfill}{rgb}{0.000000,0.000000,0.000000}%
\pgfsetfillcolor{currentfill}%
\pgfsetlinewidth{0.602250pt}%
\definecolor{currentstroke}{rgb}{0.000000,0.000000,0.000000}%
\pgfsetstrokecolor{currentstroke}%
\pgfsetdash{}{0pt}%
\pgfsys@defobject{currentmarker}{\pgfqpoint{0.000000in}{-0.027778in}}{\pgfqpoint{0.000000in}{0.000000in}}{%
\pgfpathmoveto{\pgfqpoint{0.000000in}{0.000000in}}%
\pgfpathlineto{\pgfqpoint{0.000000in}{-0.027778in}}%
\pgfusepath{stroke,fill}%
}%
\begin{pgfscope}%
\pgfsys@transformshift{2.505473in}{0.550000in}%
\pgfsys@useobject{currentmarker}{}%
\end{pgfscope}%
\end{pgfscope}%
\begin{pgfscope}%
\pgfsetbuttcap%
\pgfsetroundjoin%
\definecolor{currentfill}{rgb}{0.000000,0.000000,0.000000}%
\pgfsetfillcolor{currentfill}%
\pgfsetlinewidth{0.602250pt}%
\definecolor{currentstroke}{rgb}{0.000000,0.000000,0.000000}%
\pgfsetstrokecolor{currentstroke}%
\pgfsetdash{}{0pt}%
\pgfsys@defobject{currentmarker}{\pgfqpoint{0.000000in}{-0.027778in}}{\pgfqpoint{0.000000in}{0.000000in}}{%
\pgfpathmoveto{\pgfqpoint{0.000000in}{0.000000in}}%
\pgfpathlineto{\pgfqpoint{0.000000in}{-0.027778in}}%
\pgfusepath{stroke,fill}%
}%
\begin{pgfscope}%
\pgfsys@transformshift{2.608570in}{0.550000in}%
\pgfsys@useobject{currentmarker}{}%
\end{pgfscope}%
\end{pgfscope}%
\begin{pgfscope}%
\pgfsetbuttcap%
\pgfsetroundjoin%
\definecolor{currentfill}{rgb}{0.000000,0.000000,0.000000}%
\pgfsetfillcolor{currentfill}%
\pgfsetlinewidth{0.602250pt}%
\definecolor{currentstroke}{rgb}{0.000000,0.000000,0.000000}%
\pgfsetstrokecolor{currentstroke}%
\pgfsetdash{}{0pt}%
\pgfsys@defobject{currentmarker}{\pgfqpoint{0.000000in}{-0.027778in}}{\pgfqpoint{0.000000in}{0.000000in}}{%
\pgfpathmoveto{\pgfqpoint{0.000000in}{0.000000in}}%
\pgfpathlineto{\pgfqpoint{0.000000in}{-0.027778in}}%
\pgfusepath{stroke,fill}%
}%
\begin{pgfscope}%
\pgfsys@transformshift{2.692806in}{0.550000in}%
\pgfsys@useobject{currentmarker}{}%
\end{pgfscope}%
\end{pgfscope}%
\begin{pgfscope}%
\pgfsetbuttcap%
\pgfsetroundjoin%
\definecolor{currentfill}{rgb}{0.000000,0.000000,0.000000}%
\pgfsetfillcolor{currentfill}%
\pgfsetlinewidth{0.602250pt}%
\definecolor{currentstroke}{rgb}{0.000000,0.000000,0.000000}%
\pgfsetstrokecolor{currentstroke}%
\pgfsetdash{}{0pt}%
\pgfsys@defobject{currentmarker}{\pgfqpoint{0.000000in}{-0.027778in}}{\pgfqpoint{0.000000in}{0.000000in}}{%
\pgfpathmoveto{\pgfqpoint{0.000000in}{0.000000in}}%
\pgfpathlineto{\pgfqpoint{0.000000in}{-0.027778in}}%
\pgfusepath{stroke,fill}%
}%
\begin{pgfscope}%
\pgfsys@transformshift{2.764027in}{0.550000in}%
\pgfsys@useobject{currentmarker}{}%
\end{pgfscope}%
\end{pgfscope}%
\begin{pgfscope}%
\pgfsetbuttcap%
\pgfsetroundjoin%
\definecolor{currentfill}{rgb}{0.000000,0.000000,0.000000}%
\pgfsetfillcolor{currentfill}%
\pgfsetlinewidth{0.602250pt}%
\definecolor{currentstroke}{rgb}{0.000000,0.000000,0.000000}%
\pgfsetstrokecolor{currentstroke}%
\pgfsetdash{}{0pt}%
\pgfsys@defobject{currentmarker}{\pgfqpoint{0.000000in}{-0.027778in}}{\pgfqpoint{0.000000in}{0.000000in}}{%
\pgfpathmoveto{\pgfqpoint{0.000000in}{0.000000in}}%
\pgfpathlineto{\pgfqpoint{0.000000in}{-0.027778in}}%
\pgfusepath{stroke,fill}%
}%
\begin{pgfscope}%
\pgfsys@transformshift{2.825721in}{0.550000in}%
\pgfsys@useobject{currentmarker}{}%
\end{pgfscope}%
\end{pgfscope}%
\begin{pgfscope}%
\pgfsetbuttcap%
\pgfsetroundjoin%
\definecolor{currentfill}{rgb}{0.000000,0.000000,0.000000}%
\pgfsetfillcolor{currentfill}%
\pgfsetlinewidth{0.602250pt}%
\definecolor{currentstroke}{rgb}{0.000000,0.000000,0.000000}%
\pgfsetstrokecolor{currentstroke}%
\pgfsetdash{}{0pt}%
\pgfsys@defobject{currentmarker}{\pgfqpoint{0.000000in}{-0.027778in}}{\pgfqpoint{0.000000in}{0.000000in}}{%
\pgfpathmoveto{\pgfqpoint{0.000000in}{0.000000in}}%
\pgfpathlineto{\pgfqpoint{0.000000in}{-0.027778in}}%
\pgfusepath{stroke,fill}%
}%
\begin{pgfscope}%
\pgfsys@transformshift{2.880139in}{0.550000in}%
\pgfsys@useobject{currentmarker}{}%
\end{pgfscope}%
\end{pgfscope}%
\begin{pgfscope}%
\pgfsetbuttcap%
\pgfsetroundjoin%
\definecolor{currentfill}{rgb}{0.000000,0.000000,0.000000}%
\pgfsetfillcolor{currentfill}%
\pgfsetlinewidth{0.602250pt}%
\definecolor{currentstroke}{rgb}{0.000000,0.000000,0.000000}%
\pgfsetstrokecolor{currentstroke}%
\pgfsetdash{}{0pt}%
\pgfsys@defobject{currentmarker}{\pgfqpoint{0.000000in}{-0.027778in}}{\pgfqpoint{0.000000in}{0.000000in}}{%
\pgfpathmoveto{\pgfqpoint{0.000000in}{0.000000in}}%
\pgfpathlineto{\pgfqpoint{0.000000in}{-0.027778in}}%
\pgfusepath{stroke,fill}%
}%
\begin{pgfscope}%
\pgfsys@transformshift{3.249066in}{0.550000in}%
\pgfsys@useobject{currentmarker}{}%
\end{pgfscope}%
\end{pgfscope}%
\begin{pgfscope}%
\pgfsetbuttcap%
\pgfsetroundjoin%
\definecolor{currentfill}{rgb}{0.000000,0.000000,0.000000}%
\pgfsetfillcolor{currentfill}%
\pgfsetlinewidth{0.602250pt}%
\definecolor{currentstroke}{rgb}{0.000000,0.000000,0.000000}%
\pgfsetstrokecolor{currentstroke}%
\pgfsetdash{}{0pt}%
\pgfsys@defobject{currentmarker}{\pgfqpoint{0.000000in}{-0.027778in}}{\pgfqpoint{0.000000in}{0.000000in}}{%
\pgfpathmoveto{\pgfqpoint{0.000000in}{0.000000in}}%
\pgfpathlineto{\pgfqpoint{0.000000in}{-0.027778in}}%
\pgfusepath{stroke,fill}%
}%
\begin{pgfscope}%
\pgfsys@transformshift{3.436399in}{0.550000in}%
\pgfsys@useobject{currentmarker}{}%
\end{pgfscope}%
\end{pgfscope}%
\begin{pgfscope}%
\pgfsetbuttcap%
\pgfsetroundjoin%
\definecolor{currentfill}{rgb}{0.000000,0.000000,0.000000}%
\pgfsetfillcolor{currentfill}%
\pgfsetlinewidth{0.602250pt}%
\definecolor{currentstroke}{rgb}{0.000000,0.000000,0.000000}%
\pgfsetstrokecolor{currentstroke}%
\pgfsetdash{}{0pt}%
\pgfsys@defobject{currentmarker}{\pgfqpoint{0.000000in}{-0.027778in}}{\pgfqpoint{0.000000in}{0.000000in}}{%
\pgfpathmoveto{\pgfqpoint{0.000000in}{0.000000in}}%
\pgfpathlineto{\pgfqpoint{0.000000in}{-0.027778in}}%
\pgfusepath{stroke,fill}%
}%
\begin{pgfscope}%
\pgfsys@transformshift{3.569313in}{0.550000in}%
\pgfsys@useobject{currentmarker}{}%
\end{pgfscope}%
\end{pgfscope}%
\begin{pgfscope}%
\pgfsetbuttcap%
\pgfsetroundjoin%
\definecolor{currentfill}{rgb}{0.000000,0.000000,0.000000}%
\pgfsetfillcolor{currentfill}%
\pgfsetlinewidth{0.602250pt}%
\definecolor{currentstroke}{rgb}{0.000000,0.000000,0.000000}%
\pgfsetstrokecolor{currentstroke}%
\pgfsetdash{}{0pt}%
\pgfsys@defobject{currentmarker}{\pgfqpoint{0.000000in}{-0.027778in}}{\pgfqpoint{0.000000in}{0.000000in}}{%
\pgfpathmoveto{\pgfqpoint{0.000000in}{0.000000in}}%
\pgfpathlineto{\pgfqpoint{0.000000in}{-0.027778in}}%
\pgfusepath{stroke,fill}%
}%
\begin{pgfscope}%
\pgfsys@transformshift{3.672410in}{0.550000in}%
\pgfsys@useobject{currentmarker}{}%
\end{pgfscope}%
\end{pgfscope}%
\begin{pgfscope}%
\pgfsetbuttcap%
\pgfsetroundjoin%
\definecolor{currentfill}{rgb}{0.000000,0.000000,0.000000}%
\pgfsetfillcolor{currentfill}%
\pgfsetlinewidth{0.602250pt}%
\definecolor{currentstroke}{rgb}{0.000000,0.000000,0.000000}%
\pgfsetstrokecolor{currentstroke}%
\pgfsetdash{}{0pt}%
\pgfsys@defobject{currentmarker}{\pgfqpoint{0.000000in}{-0.027778in}}{\pgfqpoint{0.000000in}{0.000000in}}{%
\pgfpathmoveto{\pgfqpoint{0.000000in}{0.000000in}}%
\pgfpathlineto{\pgfqpoint{0.000000in}{-0.027778in}}%
\pgfusepath{stroke,fill}%
}%
\begin{pgfscope}%
\pgfsys@transformshift{3.756646in}{0.550000in}%
\pgfsys@useobject{currentmarker}{}%
\end{pgfscope}%
\end{pgfscope}%
\begin{pgfscope}%
\pgfsetbuttcap%
\pgfsetroundjoin%
\definecolor{currentfill}{rgb}{0.000000,0.000000,0.000000}%
\pgfsetfillcolor{currentfill}%
\pgfsetlinewidth{0.602250pt}%
\definecolor{currentstroke}{rgb}{0.000000,0.000000,0.000000}%
\pgfsetstrokecolor{currentstroke}%
\pgfsetdash{}{0pt}%
\pgfsys@defobject{currentmarker}{\pgfqpoint{0.000000in}{-0.027778in}}{\pgfqpoint{0.000000in}{0.000000in}}{%
\pgfpathmoveto{\pgfqpoint{0.000000in}{0.000000in}}%
\pgfpathlineto{\pgfqpoint{0.000000in}{-0.027778in}}%
\pgfusepath{stroke,fill}%
}%
\begin{pgfscope}%
\pgfsys@transformshift{3.827867in}{0.550000in}%
\pgfsys@useobject{currentmarker}{}%
\end{pgfscope}%
\end{pgfscope}%
\begin{pgfscope}%
\pgfsetbuttcap%
\pgfsetroundjoin%
\definecolor{currentfill}{rgb}{0.000000,0.000000,0.000000}%
\pgfsetfillcolor{currentfill}%
\pgfsetlinewidth{0.602250pt}%
\definecolor{currentstroke}{rgb}{0.000000,0.000000,0.000000}%
\pgfsetstrokecolor{currentstroke}%
\pgfsetdash{}{0pt}%
\pgfsys@defobject{currentmarker}{\pgfqpoint{0.000000in}{-0.027778in}}{\pgfqpoint{0.000000in}{0.000000in}}{%
\pgfpathmoveto{\pgfqpoint{0.000000in}{0.000000in}}%
\pgfpathlineto{\pgfqpoint{0.000000in}{-0.027778in}}%
\pgfusepath{stroke,fill}%
}%
\begin{pgfscope}%
\pgfsys@transformshift{3.889561in}{0.550000in}%
\pgfsys@useobject{currentmarker}{}%
\end{pgfscope}%
\end{pgfscope}%
\begin{pgfscope}%
\pgfsetbuttcap%
\pgfsetroundjoin%
\definecolor{currentfill}{rgb}{0.000000,0.000000,0.000000}%
\pgfsetfillcolor{currentfill}%
\pgfsetlinewidth{0.602250pt}%
\definecolor{currentstroke}{rgb}{0.000000,0.000000,0.000000}%
\pgfsetstrokecolor{currentstroke}%
\pgfsetdash{}{0pt}%
\pgfsys@defobject{currentmarker}{\pgfqpoint{0.000000in}{-0.027778in}}{\pgfqpoint{0.000000in}{0.000000in}}{%
\pgfpathmoveto{\pgfqpoint{0.000000in}{0.000000in}}%
\pgfpathlineto{\pgfqpoint{0.000000in}{-0.027778in}}%
\pgfusepath{stroke,fill}%
}%
\begin{pgfscope}%
\pgfsys@transformshift{3.943980in}{0.550000in}%
\pgfsys@useobject{currentmarker}{}%
\end{pgfscope}%
\end{pgfscope}%
\begin{pgfscope}%
\pgfsetbuttcap%
\pgfsetroundjoin%
\definecolor{currentfill}{rgb}{0.000000,0.000000,0.000000}%
\pgfsetfillcolor{currentfill}%
\pgfsetlinewidth{0.602250pt}%
\definecolor{currentstroke}{rgb}{0.000000,0.000000,0.000000}%
\pgfsetstrokecolor{currentstroke}%
\pgfsetdash{}{0pt}%
\pgfsys@defobject{currentmarker}{\pgfqpoint{0.000000in}{-0.027778in}}{\pgfqpoint{0.000000in}{0.000000in}}{%
\pgfpathmoveto{\pgfqpoint{0.000000in}{0.000000in}}%
\pgfpathlineto{\pgfqpoint{0.000000in}{-0.027778in}}%
\pgfusepath{stroke,fill}%
}%
\begin{pgfscope}%
\pgfsys@transformshift{4.312906in}{0.550000in}%
\pgfsys@useobject{currentmarker}{}%
\end{pgfscope}%
\end{pgfscope}%
\begin{pgfscope}%
\pgfsetbuttcap%
\pgfsetroundjoin%
\definecolor{currentfill}{rgb}{0.000000,0.000000,0.000000}%
\pgfsetfillcolor{currentfill}%
\pgfsetlinewidth{0.602250pt}%
\definecolor{currentstroke}{rgb}{0.000000,0.000000,0.000000}%
\pgfsetstrokecolor{currentstroke}%
\pgfsetdash{}{0pt}%
\pgfsys@defobject{currentmarker}{\pgfqpoint{0.000000in}{-0.027778in}}{\pgfqpoint{0.000000in}{0.000000in}}{%
\pgfpathmoveto{\pgfqpoint{0.000000in}{0.000000in}}%
\pgfpathlineto{\pgfqpoint{0.000000in}{-0.027778in}}%
\pgfusepath{stroke,fill}%
}%
\begin{pgfscope}%
\pgfsys@transformshift{4.500239in}{0.550000in}%
\pgfsys@useobject{currentmarker}{}%
\end{pgfscope}%
\end{pgfscope}%
\begin{pgfscope}%
\definecolor{textcolor}{rgb}{0.000000,0.000000,0.000000}%
\pgfsetstrokecolor{textcolor}%
\pgfsetfillcolor{textcolor}%
\pgftext[x=2.562500in,y=0.249368in,,top]{\color{textcolor}\sffamily\fontsize{11.000000}{13.200000}\selectfont \(\displaystyle N_\mathrm{QMC}\)}%
\end{pgfscope}%
\begin{pgfscope}%
\pgfsetbuttcap%
\pgfsetroundjoin%
\definecolor{currentfill}{rgb}{0.000000,0.000000,0.000000}%
\pgfsetfillcolor{currentfill}%
\pgfsetlinewidth{0.803000pt}%
\definecolor{currentstroke}{rgb}{0.000000,0.000000,0.000000}%
\pgfsetstrokecolor{currentstroke}%
\pgfsetdash{}{0pt}%
\pgfsys@defobject{currentmarker}{\pgfqpoint{-0.048611in}{0.000000in}}{\pgfqpoint{0.000000in}{0.000000in}}{%
\pgfpathmoveto{\pgfqpoint{0.000000in}{0.000000in}}%
\pgfpathlineto{\pgfqpoint{-0.048611in}{0.000000in}}%
\pgfusepath{stroke,fill}%
}%
\begin{pgfscope}%
\pgfsys@transformshift{0.625000in}{1.162334in}%
\pgfsys@useobject{currentmarker}{}%
\end{pgfscope}%
\end{pgfscope}%
\begin{pgfscope}%
\definecolor{textcolor}{rgb}{0.000000,0.000000,0.000000}%
\pgfsetstrokecolor{textcolor}%
\pgfsetfillcolor{textcolor}%
\pgftext[x=0.217899in,y=1.104296in,left,base]{\color{textcolor}\sffamily\fontsize{11.000000}{13.200000}\selectfont \(\displaystyle 10^{-3}\)}%
\end{pgfscope}%
\begin{pgfscope}%
\pgfsetbuttcap%
\pgfsetroundjoin%
\definecolor{currentfill}{rgb}{0.000000,0.000000,0.000000}%
\pgfsetfillcolor{currentfill}%
\pgfsetlinewidth{0.803000pt}%
\definecolor{currentstroke}{rgb}{0.000000,0.000000,0.000000}%
\pgfsetstrokecolor{currentstroke}%
\pgfsetdash{}{0pt}%
\pgfsys@defobject{currentmarker}{\pgfqpoint{-0.048611in}{0.000000in}}{\pgfqpoint{0.000000in}{0.000000in}}{%
\pgfpathmoveto{\pgfqpoint{0.000000in}{0.000000in}}%
\pgfpathlineto{\pgfqpoint{-0.048611in}{0.000000in}}%
\pgfusepath{stroke,fill}%
}%
\begin{pgfscope}%
\pgfsys@transformshift{0.625000in}{2.547348in}%
\pgfsys@useobject{currentmarker}{}%
\end{pgfscope}%
\end{pgfscope}%
\begin{pgfscope}%
\definecolor{textcolor}{rgb}{0.000000,0.000000,0.000000}%
\pgfsetstrokecolor{textcolor}%
\pgfsetfillcolor{textcolor}%
\pgftext[x=0.217899in,y=2.489310in,left,base]{\color{textcolor}\sffamily\fontsize{11.000000}{13.200000}\selectfont \(\displaystyle 10^{-2}\)}%
\end{pgfscope}%
\begin{pgfscope}%
\pgfsetbuttcap%
\pgfsetroundjoin%
\definecolor{currentfill}{rgb}{0.000000,0.000000,0.000000}%
\pgfsetfillcolor{currentfill}%
\pgfsetlinewidth{0.803000pt}%
\definecolor{currentstroke}{rgb}{0.000000,0.000000,0.000000}%
\pgfsetstrokecolor{currentstroke}%
\pgfsetdash{}{0pt}%
\pgfsys@defobject{currentmarker}{\pgfqpoint{-0.048611in}{0.000000in}}{\pgfqpoint{0.000000in}{0.000000in}}{%
\pgfpathmoveto{\pgfqpoint{0.000000in}{0.000000in}}%
\pgfpathlineto{\pgfqpoint{-0.048611in}{0.000000in}}%
\pgfusepath{stroke,fill}%
}%
\begin{pgfscope}%
\pgfsys@transformshift{0.625000in}{3.932361in}%
\pgfsys@useobject{currentmarker}{}%
\end{pgfscope}%
\end{pgfscope}%
\begin{pgfscope}%
\definecolor{textcolor}{rgb}{0.000000,0.000000,0.000000}%
\pgfsetstrokecolor{textcolor}%
\pgfsetfillcolor{textcolor}%
\pgftext[x=0.217899in,y=3.874324in,left,base]{\color{textcolor}\sffamily\fontsize{11.000000}{13.200000}\selectfont \(\displaystyle 10^{-1}\)}%
\end{pgfscope}%
\begin{pgfscope}%
\pgfsetbuttcap%
\pgfsetroundjoin%
\definecolor{currentfill}{rgb}{0.000000,0.000000,0.000000}%
\pgfsetfillcolor{currentfill}%
\pgfsetlinewidth{0.602250pt}%
\definecolor{currentstroke}{rgb}{0.000000,0.000000,0.000000}%
\pgfsetstrokecolor{currentstroke}%
\pgfsetdash{}{0pt}%
\pgfsys@defobject{currentmarker}{\pgfqpoint{-0.027778in}{0.000000in}}{\pgfqpoint{0.000000in}{0.000000in}}{%
\pgfpathmoveto{\pgfqpoint{0.000000in}{0.000000in}}%
\pgfpathlineto{\pgfqpoint{-0.027778in}{0.000000in}}%
\pgfusepath{stroke,fill}%
}%
\begin{pgfscope}%
\pgfsys@transformshift{0.625000in}{0.611182in}%
\pgfsys@useobject{currentmarker}{}%
\end{pgfscope}%
\end{pgfscope}%
\begin{pgfscope}%
\pgfsetbuttcap%
\pgfsetroundjoin%
\definecolor{currentfill}{rgb}{0.000000,0.000000,0.000000}%
\pgfsetfillcolor{currentfill}%
\pgfsetlinewidth{0.602250pt}%
\definecolor{currentstroke}{rgb}{0.000000,0.000000,0.000000}%
\pgfsetstrokecolor{currentstroke}%
\pgfsetdash{}{0pt}%
\pgfsys@defobject{currentmarker}{\pgfqpoint{-0.027778in}{0.000000in}}{\pgfqpoint{0.000000in}{0.000000in}}{%
\pgfpathmoveto{\pgfqpoint{0.000000in}{0.000000in}}%
\pgfpathlineto{\pgfqpoint{-0.027778in}{0.000000in}}%
\pgfusepath{stroke,fill}%
}%
\begin{pgfscope}%
\pgfsys@transformshift{0.625000in}{0.745403in}%
\pgfsys@useobject{currentmarker}{}%
\end{pgfscope}%
\end{pgfscope}%
\begin{pgfscope}%
\pgfsetbuttcap%
\pgfsetroundjoin%
\definecolor{currentfill}{rgb}{0.000000,0.000000,0.000000}%
\pgfsetfillcolor{currentfill}%
\pgfsetlinewidth{0.602250pt}%
\definecolor{currentstroke}{rgb}{0.000000,0.000000,0.000000}%
\pgfsetstrokecolor{currentstroke}%
\pgfsetdash{}{0pt}%
\pgfsys@defobject{currentmarker}{\pgfqpoint{-0.027778in}{0.000000in}}{\pgfqpoint{0.000000in}{0.000000in}}{%
\pgfpathmoveto{\pgfqpoint{0.000000in}{0.000000in}}%
\pgfpathlineto{\pgfqpoint{-0.027778in}{0.000000in}}%
\pgfusepath{stroke,fill}%
}%
\begin{pgfscope}%
\pgfsys@transformshift{0.625000in}{0.855070in}%
\pgfsys@useobject{currentmarker}{}%
\end{pgfscope}%
\end{pgfscope}%
\begin{pgfscope}%
\pgfsetbuttcap%
\pgfsetroundjoin%
\definecolor{currentfill}{rgb}{0.000000,0.000000,0.000000}%
\pgfsetfillcolor{currentfill}%
\pgfsetlinewidth{0.602250pt}%
\definecolor{currentstroke}{rgb}{0.000000,0.000000,0.000000}%
\pgfsetstrokecolor{currentstroke}%
\pgfsetdash{}{0pt}%
\pgfsys@defobject{currentmarker}{\pgfqpoint{-0.027778in}{0.000000in}}{\pgfqpoint{0.000000in}{0.000000in}}{%
\pgfpathmoveto{\pgfqpoint{0.000000in}{0.000000in}}%
\pgfpathlineto{\pgfqpoint{-0.027778in}{0.000000in}}%
\pgfusepath{stroke,fill}%
}%
\begin{pgfscope}%
\pgfsys@transformshift{0.625000in}{0.947793in}%
\pgfsys@useobject{currentmarker}{}%
\end{pgfscope}%
\end{pgfscope}%
\begin{pgfscope}%
\pgfsetbuttcap%
\pgfsetroundjoin%
\definecolor{currentfill}{rgb}{0.000000,0.000000,0.000000}%
\pgfsetfillcolor{currentfill}%
\pgfsetlinewidth{0.602250pt}%
\definecolor{currentstroke}{rgb}{0.000000,0.000000,0.000000}%
\pgfsetstrokecolor{currentstroke}%
\pgfsetdash{}{0pt}%
\pgfsys@defobject{currentmarker}{\pgfqpoint{-0.027778in}{0.000000in}}{\pgfqpoint{0.000000in}{0.000000in}}{%
\pgfpathmoveto{\pgfqpoint{0.000000in}{0.000000in}}%
\pgfpathlineto{\pgfqpoint{-0.027778in}{0.000000in}}%
\pgfusepath{stroke,fill}%
}%
\begin{pgfscope}%
\pgfsys@transformshift{0.625000in}{1.028112in}%
\pgfsys@useobject{currentmarker}{}%
\end{pgfscope}%
\end{pgfscope}%
\begin{pgfscope}%
\pgfsetbuttcap%
\pgfsetroundjoin%
\definecolor{currentfill}{rgb}{0.000000,0.000000,0.000000}%
\pgfsetfillcolor{currentfill}%
\pgfsetlinewidth{0.602250pt}%
\definecolor{currentstroke}{rgb}{0.000000,0.000000,0.000000}%
\pgfsetstrokecolor{currentstroke}%
\pgfsetdash{}{0pt}%
\pgfsys@defobject{currentmarker}{\pgfqpoint{-0.027778in}{0.000000in}}{\pgfqpoint{0.000000in}{0.000000in}}{%
\pgfpathmoveto{\pgfqpoint{0.000000in}{0.000000in}}%
\pgfpathlineto{\pgfqpoint{-0.027778in}{0.000000in}}%
\pgfusepath{stroke,fill}%
}%
\begin{pgfscope}%
\pgfsys@transformshift{0.625000in}{1.098959in}%
\pgfsys@useobject{currentmarker}{}%
\end{pgfscope}%
\end{pgfscope}%
\begin{pgfscope}%
\pgfsetbuttcap%
\pgfsetroundjoin%
\definecolor{currentfill}{rgb}{0.000000,0.000000,0.000000}%
\pgfsetfillcolor{currentfill}%
\pgfsetlinewidth{0.602250pt}%
\definecolor{currentstroke}{rgb}{0.000000,0.000000,0.000000}%
\pgfsetstrokecolor{currentstroke}%
\pgfsetdash{}{0pt}%
\pgfsys@defobject{currentmarker}{\pgfqpoint{-0.027778in}{0.000000in}}{\pgfqpoint{0.000000in}{0.000000in}}{%
\pgfpathmoveto{\pgfqpoint{0.000000in}{0.000000in}}%
\pgfpathlineto{\pgfqpoint{-0.027778in}{0.000000in}}%
\pgfusepath{stroke,fill}%
}%
\begin{pgfscope}%
\pgfsys@transformshift{0.625000in}{1.579265in}%
\pgfsys@useobject{currentmarker}{}%
\end{pgfscope}%
\end{pgfscope}%
\begin{pgfscope}%
\pgfsetbuttcap%
\pgfsetroundjoin%
\definecolor{currentfill}{rgb}{0.000000,0.000000,0.000000}%
\pgfsetfillcolor{currentfill}%
\pgfsetlinewidth{0.602250pt}%
\definecolor{currentstroke}{rgb}{0.000000,0.000000,0.000000}%
\pgfsetstrokecolor{currentstroke}%
\pgfsetdash{}{0pt}%
\pgfsys@defobject{currentmarker}{\pgfqpoint{-0.027778in}{0.000000in}}{\pgfqpoint{0.000000in}{0.000000in}}{%
\pgfpathmoveto{\pgfqpoint{0.000000in}{0.000000in}}%
\pgfpathlineto{\pgfqpoint{-0.027778in}{0.000000in}}%
\pgfusepath{stroke,fill}%
}%
\begin{pgfscope}%
\pgfsys@transformshift{0.625000in}{1.823154in}%
\pgfsys@useobject{currentmarker}{}%
\end{pgfscope}%
\end{pgfscope}%
\begin{pgfscope}%
\pgfsetbuttcap%
\pgfsetroundjoin%
\definecolor{currentfill}{rgb}{0.000000,0.000000,0.000000}%
\pgfsetfillcolor{currentfill}%
\pgfsetlinewidth{0.602250pt}%
\definecolor{currentstroke}{rgb}{0.000000,0.000000,0.000000}%
\pgfsetstrokecolor{currentstroke}%
\pgfsetdash{}{0pt}%
\pgfsys@defobject{currentmarker}{\pgfqpoint{-0.027778in}{0.000000in}}{\pgfqpoint{0.000000in}{0.000000in}}{%
\pgfpathmoveto{\pgfqpoint{0.000000in}{0.000000in}}%
\pgfpathlineto{\pgfqpoint{-0.027778in}{0.000000in}}%
\pgfusepath{stroke,fill}%
}%
\begin{pgfscope}%
\pgfsys@transformshift{0.625000in}{1.996195in}%
\pgfsys@useobject{currentmarker}{}%
\end{pgfscope}%
\end{pgfscope}%
\begin{pgfscope}%
\pgfsetbuttcap%
\pgfsetroundjoin%
\definecolor{currentfill}{rgb}{0.000000,0.000000,0.000000}%
\pgfsetfillcolor{currentfill}%
\pgfsetlinewidth{0.602250pt}%
\definecolor{currentstroke}{rgb}{0.000000,0.000000,0.000000}%
\pgfsetstrokecolor{currentstroke}%
\pgfsetdash{}{0pt}%
\pgfsys@defobject{currentmarker}{\pgfqpoint{-0.027778in}{0.000000in}}{\pgfqpoint{0.000000in}{0.000000in}}{%
\pgfpathmoveto{\pgfqpoint{0.000000in}{0.000000in}}%
\pgfpathlineto{\pgfqpoint{-0.027778in}{0.000000in}}%
\pgfusepath{stroke,fill}%
}%
\begin{pgfscope}%
\pgfsys@transformshift{0.625000in}{2.130417in}%
\pgfsys@useobject{currentmarker}{}%
\end{pgfscope}%
\end{pgfscope}%
\begin{pgfscope}%
\pgfsetbuttcap%
\pgfsetroundjoin%
\definecolor{currentfill}{rgb}{0.000000,0.000000,0.000000}%
\pgfsetfillcolor{currentfill}%
\pgfsetlinewidth{0.602250pt}%
\definecolor{currentstroke}{rgb}{0.000000,0.000000,0.000000}%
\pgfsetstrokecolor{currentstroke}%
\pgfsetdash{}{0pt}%
\pgfsys@defobject{currentmarker}{\pgfqpoint{-0.027778in}{0.000000in}}{\pgfqpoint{0.000000in}{0.000000in}}{%
\pgfpathmoveto{\pgfqpoint{0.000000in}{0.000000in}}%
\pgfpathlineto{\pgfqpoint{-0.027778in}{0.000000in}}%
\pgfusepath{stroke,fill}%
}%
\begin{pgfscope}%
\pgfsys@transformshift{0.625000in}{2.240084in}%
\pgfsys@useobject{currentmarker}{}%
\end{pgfscope}%
\end{pgfscope}%
\begin{pgfscope}%
\pgfsetbuttcap%
\pgfsetroundjoin%
\definecolor{currentfill}{rgb}{0.000000,0.000000,0.000000}%
\pgfsetfillcolor{currentfill}%
\pgfsetlinewidth{0.602250pt}%
\definecolor{currentstroke}{rgb}{0.000000,0.000000,0.000000}%
\pgfsetstrokecolor{currentstroke}%
\pgfsetdash{}{0pt}%
\pgfsys@defobject{currentmarker}{\pgfqpoint{-0.027778in}{0.000000in}}{\pgfqpoint{0.000000in}{0.000000in}}{%
\pgfpathmoveto{\pgfqpoint{0.000000in}{0.000000in}}%
\pgfpathlineto{\pgfqpoint{-0.027778in}{0.000000in}}%
\pgfusepath{stroke,fill}%
}%
\begin{pgfscope}%
\pgfsys@transformshift{0.625000in}{2.332806in}%
\pgfsys@useobject{currentmarker}{}%
\end{pgfscope}%
\end{pgfscope}%
\begin{pgfscope}%
\pgfsetbuttcap%
\pgfsetroundjoin%
\definecolor{currentfill}{rgb}{0.000000,0.000000,0.000000}%
\pgfsetfillcolor{currentfill}%
\pgfsetlinewidth{0.602250pt}%
\definecolor{currentstroke}{rgb}{0.000000,0.000000,0.000000}%
\pgfsetstrokecolor{currentstroke}%
\pgfsetdash{}{0pt}%
\pgfsys@defobject{currentmarker}{\pgfqpoint{-0.027778in}{0.000000in}}{\pgfqpoint{0.000000in}{0.000000in}}{%
\pgfpathmoveto{\pgfqpoint{0.000000in}{0.000000in}}%
\pgfpathlineto{\pgfqpoint{-0.027778in}{0.000000in}}%
\pgfusepath{stroke,fill}%
}%
\begin{pgfscope}%
\pgfsys@transformshift{0.625000in}{2.413126in}%
\pgfsys@useobject{currentmarker}{}%
\end{pgfscope}%
\end{pgfscope}%
\begin{pgfscope}%
\pgfsetbuttcap%
\pgfsetroundjoin%
\definecolor{currentfill}{rgb}{0.000000,0.000000,0.000000}%
\pgfsetfillcolor{currentfill}%
\pgfsetlinewidth{0.602250pt}%
\definecolor{currentstroke}{rgb}{0.000000,0.000000,0.000000}%
\pgfsetstrokecolor{currentstroke}%
\pgfsetdash{}{0pt}%
\pgfsys@defobject{currentmarker}{\pgfqpoint{-0.027778in}{0.000000in}}{\pgfqpoint{0.000000in}{0.000000in}}{%
\pgfpathmoveto{\pgfqpoint{0.000000in}{0.000000in}}%
\pgfpathlineto{\pgfqpoint{-0.027778in}{0.000000in}}%
\pgfusepath{stroke,fill}%
}%
\begin{pgfscope}%
\pgfsys@transformshift{0.625000in}{2.483973in}%
\pgfsys@useobject{currentmarker}{}%
\end{pgfscope}%
\end{pgfscope}%
\begin{pgfscope}%
\pgfsetbuttcap%
\pgfsetroundjoin%
\definecolor{currentfill}{rgb}{0.000000,0.000000,0.000000}%
\pgfsetfillcolor{currentfill}%
\pgfsetlinewidth{0.602250pt}%
\definecolor{currentstroke}{rgb}{0.000000,0.000000,0.000000}%
\pgfsetstrokecolor{currentstroke}%
\pgfsetdash{}{0pt}%
\pgfsys@defobject{currentmarker}{\pgfqpoint{-0.027778in}{0.000000in}}{\pgfqpoint{0.000000in}{0.000000in}}{%
\pgfpathmoveto{\pgfqpoint{0.000000in}{0.000000in}}%
\pgfpathlineto{\pgfqpoint{-0.027778in}{0.000000in}}%
\pgfusepath{stroke,fill}%
}%
\begin{pgfscope}%
\pgfsys@transformshift{0.625000in}{2.964278in}%
\pgfsys@useobject{currentmarker}{}%
\end{pgfscope}%
\end{pgfscope}%
\begin{pgfscope}%
\pgfsetbuttcap%
\pgfsetroundjoin%
\definecolor{currentfill}{rgb}{0.000000,0.000000,0.000000}%
\pgfsetfillcolor{currentfill}%
\pgfsetlinewidth{0.602250pt}%
\definecolor{currentstroke}{rgb}{0.000000,0.000000,0.000000}%
\pgfsetstrokecolor{currentstroke}%
\pgfsetdash{}{0pt}%
\pgfsys@defobject{currentmarker}{\pgfqpoint{-0.027778in}{0.000000in}}{\pgfqpoint{0.000000in}{0.000000in}}{%
\pgfpathmoveto{\pgfqpoint{0.000000in}{0.000000in}}%
\pgfpathlineto{\pgfqpoint{-0.027778in}{0.000000in}}%
\pgfusepath{stroke,fill}%
}%
\begin{pgfscope}%
\pgfsys@transformshift{0.625000in}{3.208167in}%
\pgfsys@useobject{currentmarker}{}%
\end{pgfscope}%
\end{pgfscope}%
\begin{pgfscope}%
\pgfsetbuttcap%
\pgfsetroundjoin%
\definecolor{currentfill}{rgb}{0.000000,0.000000,0.000000}%
\pgfsetfillcolor{currentfill}%
\pgfsetlinewidth{0.602250pt}%
\definecolor{currentstroke}{rgb}{0.000000,0.000000,0.000000}%
\pgfsetstrokecolor{currentstroke}%
\pgfsetdash{}{0pt}%
\pgfsys@defobject{currentmarker}{\pgfqpoint{-0.027778in}{0.000000in}}{\pgfqpoint{0.000000in}{0.000000in}}{%
\pgfpathmoveto{\pgfqpoint{0.000000in}{0.000000in}}%
\pgfpathlineto{\pgfqpoint{-0.027778in}{0.000000in}}%
\pgfusepath{stroke,fill}%
}%
\begin{pgfscope}%
\pgfsys@transformshift{0.625000in}{3.381209in}%
\pgfsys@useobject{currentmarker}{}%
\end{pgfscope}%
\end{pgfscope}%
\begin{pgfscope}%
\pgfsetbuttcap%
\pgfsetroundjoin%
\definecolor{currentfill}{rgb}{0.000000,0.000000,0.000000}%
\pgfsetfillcolor{currentfill}%
\pgfsetlinewidth{0.602250pt}%
\definecolor{currentstroke}{rgb}{0.000000,0.000000,0.000000}%
\pgfsetstrokecolor{currentstroke}%
\pgfsetdash{}{0pt}%
\pgfsys@defobject{currentmarker}{\pgfqpoint{-0.027778in}{0.000000in}}{\pgfqpoint{0.000000in}{0.000000in}}{%
\pgfpathmoveto{\pgfqpoint{0.000000in}{0.000000in}}%
\pgfpathlineto{\pgfqpoint{-0.027778in}{0.000000in}}%
\pgfusepath{stroke,fill}%
}%
\begin{pgfscope}%
\pgfsys@transformshift{0.625000in}{3.515431in}%
\pgfsys@useobject{currentmarker}{}%
\end{pgfscope}%
\end{pgfscope}%
\begin{pgfscope}%
\pgfsetbuttcap%
\pgfsetroundjoin%
\definecolor{currentfill}{rgb}{0.000000,0.000000,0.000000}%
\pgfsetfillcolor{currentfill}%
\pgfsetlinewidth{0.602250pt}%
\definecolor{currentstroke}{rgb}{0.000000,0.000000,0.000000}%
\pgfsetstrokecolor{currentstroke}%
\pgfsetdash{}{0pt}%
\pgfsys@defobject{currentmarker}{\pgfqpoint{-0.027778in}{0.000000in}}{\pgfqpoint{0.000000in}{0.000000in}}{%
\pgfpathmoveto{\pgfqpoint{0.000000in}{0.000000in}}%
\pgfpathlineto{\pgfqpoint{-0.027778in}{0.000000in}}%
\pgfusepath{stroke,fill}%
}%
\begin{pgfscope}%
\pgfsys@transformshift{0.625000in}{3.625098in}%
\pgfsys@useobject{currentmarker}{}%
\end{pgfscope}%
\end{pgfscope}%
\begin{pgfscope}%
\pgfsetbuttcap%
\pgfsetroundjoin%
\definecolor{currentfill}{rgb}{0.000000,0.000000,0.000000}%
\pgfsetfillcolor{currentfill}%
\pgfsetlinewidth{0.602250pt}%
\definecolor{currentstroke}{rgb}{0.000000,0.000000,0.000000}%
\pgfsetstrokecolor{currentstroke}%
\pgfsetdash{}{0pt}%
\pgfsys@defobject{currentmarker}{\pgfqpoint{-0.027778in}{0.000000in}}{\pgfqpoint{0.000000in}{0.000000in}}{%
\pgfpathmoveto{\pgfqpoint{0.000000in}{0.000000in}}%
\pgfpathlineto{\pgfqpoint{-0.027778in}{0.000000in}}%
\pgfusepath{stroke,fill}%
}%
\begin{pgfscope}%
\pgfsys@transformshift{0.625000in}{3.717820in}%
\pgfsys@useobject{currentmarker}{}%
\end{pgfscope}%
\end{pgfscope}%
\begin{pgfscope}%
\pgfsetbuttcap%
\pgfsetroundjoin%
\definecolor{currentfill}{rgb}{0.000000,0.000000,0.000000}%
\pgfsetfillcolor{currentfill}%
\pgfsetlinewidth{0.602250pt}%
\definecolor{currentstroke}{rgb}{0.000000,0.000000,0.000000}%
\pgfsetstrokecolor{currentstroke}%
\pgfsetdash{}{0pt}%
\pgfsys@defobject{currentmarker}{\pgfqpoint{-0.027778in}{0.000000in}}{\pgfqpoint{0.000000in}{0.000000in}}{%
\pgfpathmoveto{\pgfqpoint{0.000000in}{0.000000in}}%
\pgfpathlineto{\pgfqpoint{-0.027778in}{0.000000in}}%
\pgfusepath{stroke,fill}%
}%
\begin{pgfscope}%
\pgfsys@transformshift{0.625000in}{3.798140in}%
\pgfsys@useobject{currentmarker}{}%
\end{pgfscope}%
\end{pgfscope}%
\begin{pgfscope}%
\pgfsetbuttcap%
\pgfsetroundjoin%
\definecolor{currentfill}{rgb}{0.000000,0.000000,0.000000}%
\pgfsetfillcolor{currentfill}%
\pgfsetlinewidth{0.602250pt}%
\definecolor{currentstroke}{rgb}{0.000000,0.000000,0.000000}%
\pgfsetstrokecolor{currentstroke}%
\pgfsetdash{}{0pt}%
\pgfsys@defobject{currentmarker}{\pgfqpoint{-0.027778in}{0.000000in}}{\pgfqpoint{0.000000in}{0.000000in}}{%
\pgfpathmoveto{\pgfqpoint{0.000000in}{0.000000in}}%
\pgfpathlineto{\pgfqpoint{-0.027778in}{0.000000in}}%
\pgfusepath{stroke,fill}%
}%
\begin{pgfscope}%
\pgfsys@transformshift{0.625000in}{3.868987in}%
\pgfsys@useobject{currentmarker}{}%
\end{pgfscope}%
\end{pgfscope}%
\begin{pgfscope}%
\pgfsetbuttcap%
\pgfsetroundjoin%
\definecolor{currentfill}{rgb}{0.000000,0.000000,0.000000}%
\pgfsetfillcolor{currentfill}%
\pgfsetlinewidth{0.602250pt}%
\definecolor{currentstroke}{rgb}{0.000000,0.000000,0.000000}%
\pgfsetstrokecolor{currentstroke}%
\pgfsetdash{}{0pt}%
\pgfsys@defobject{currentmarker}{\pgfqpoint{-0.027778in}{0.000000in}}{\pgfqpoint{0.000000in}{0.000000in}}{%
\pgfpathmoveto{\pgfqpoint{0.000000in}{0.000000in}}%
\pgfpathlineto{\pgfqpoint{-0.027778in}{0.000000in}}%
\pgfusepath{stroke,fill}%
}%
\begin{pgfscope}%
\pgfsys@transformshift{0.625000in}{4.349292in}%
\pgfsys@useobject{currentmarker}{}%
\end{pgfscope}%
\end{pgfscope}%
\begin{pgfscope}%
\definecolor{textcolor}{rgb}{0.000000,0.000000,0.000000}%
\pgfsetstrokecolor{textcolor}%
\pgfsetfillcolor{textcolor}%
\pgftext[x=0.162343in,y=2.475000in,,bottom,rotate=90.000000]{\color{textcolor}\sffamily\fontsize{11.000000}{13.200000}\selectfont \textrm{QMC Error}}%
\end{pgfscope}%
\begin{pgfscope}%
\pgfpathrectangle{\pgfqpoint{0.625000in}{0.550000in}}{\pgfqpoint{3.875000in}{3.850000in}}%
\pgfusepath{clip}%
\pgfsetbuttcap%
\pgfsetroundjoin%
\definecolor{currentfill}{rgb}{0.000000,0.000000,0.000000}%
\pgfsetfillcolor{currentfill}%
\pgfsetlinewidth{1.003750pt}%
\definecolor{currentstroke}{rgb}{0.000000,0.000000,0.000000}%
\pgfsetstrokecolor{currentstroke}%
\pgfsetdash{}{0pt}%
\pgfsys@defobject{currentmarker}{\pgfqpoint{-0.041667in}{-0.041667in}}{\pgfqpoint{0.041667in}{0.041667in}}{%
\pgfpathmoveto{\pgfqpoint{0.000000in}{-0.041667in}}%
\pgfpathcurveto{\pgfqpoint{0.011050in}{-0.041667in}}{\pgfqpoint{0.021649in}{-0.037276in}}{\pgfqpoint{0.029463in}{-0.029463in}}%
\pgfpathcurveto{\pgfqpoint{0.037276in}{-0.021649in}}{\pgfqpoint{0.041667in}{-0.011050in}}{\pgfqpoint{0.041667in}{0.000000in}}%
\pgfpathcurveto{\pgfqpoint{0.041667in}{0.011050in}}{\pgfqpoint{0.037276in}{0.021649in}}{\pgfqpoint{0.029463in}{0.029463in}}%
\pgfpathcurveto{\pgfqpoint{0.021649in}{0.037276in}}{\pgfqpoint{0.011050in}{0.041667in}}{\pgfqpoint{0.000000in}{0.041667in}}%
\pgfpathcurveto{\pgfqpoint{-0.011050in}{0.041667in}}{\pgfqpoint{-0.021649in}{0.037276in}}{\pgfqpoint{-0.029463in}{0.029463in}}%
\pgfpathcurveto{\pgfqpoint{-0.037276in}{0.021649in}}{\pgfqpoint{-0.041667in}{0.011050in}}{\pgfqpoint{-0.041667in}{0.000000in}}%
\pgfpathcurveto{\pgfqpoint{-0.041667in}{-0.011050in}}{\pgfqpoint{-0.037276in}{-0.021649in}}{\pgfqpoint{-0.029463in}{-0.029463in}}%
\pgfpathcurveto{\pgfqpoint{-0.021649in}{-0.037276in}}{\pgfqpoint{-0.011050in}{-0.041667in}}{\pgfqpoint{0.000000in}{-0.041667in}}%
\pgfpathclose%
\pgfusepath{stroke,fill}%
}%
\begin{pgfscope}%
\pgfsys@transformshift{0.801136in}{4.225000in}%
\pgfsys@useobject{currentmarker}{}%
\end{pgfscope}%
\begin{pgfscope}%
\pgfsys@transformshift{1.121384in}{4.063200in}%
\pgfsys@useobject{currentmarker}{}%
\end{pgfscope}%
\begin{pgfscope}%
\pgfsys@transformshift{1.441632in}{3.367777in}%
\pgfsys@useobject{currentmarker}{}%
\end{pgfscope}%
\begin{pgfscope}%
\pgfsys@transformshift{1.761880in}{3.207302in}%
\pgfsys@useobject{currentmarker}{}%
\end{pgfscope}%
\begin{pgfscope}%
\pgfsys@transformshift{2.082128in}{2.866856in}%
\pgfsys@useobject{currentmarker}{}%
\end{pgfscope}%
\begin{pgfscope}%
\pgfsys@transformshift{2.402376in}{2.570900in}%
\pgfsys@useobject{currentmarker}{}%
\end{pgfscope}%
\begin{pgfscope}%
\pgfsys@transformshift{2.722624in}{2.518603in}%
\pgfsys@useobject{currentmarker}{}%
\end{pgfscope}%
\begin{pgfscope}%
\pgfsys@transformshift{3.042872in}{2.454612in}%
\pgfsys@useobject{currentmarker}{}%
\end{pgfscope}%
\begin{pgfscope}%
\pgfsys@transformshift{3.363120in}{1.645336in}%
\pgfsys@useobject{currentmarker}{}%
\end{pgfscope}%
\begin{pgfscope}%
\pgfsys@transformshift{3.683368in}{1.261927in}%
\pgfsys@useobject{currentmarker}{}%
\end{pgfscope}%
\begin{pgfscope}%
\pgfsys@transformshift{4.003616in}{1.015727in}%
\pgfsys@useobject{currentmarker}{}%
\end{pgfscope}%
\begin{pgfscope}%
\pgfsys@transformshift{4.323864in}{0.725000in}%
\pgfsys@useobject{currentmarker}{}%
\end{pgfscope}%
\end{pgfscope}%
\begin{pgfscope}%
\pgfpathrectangle{\pgfqpoint{0.625000in}{0.550000in}}{\pgfqpoint{3.875000in}{3.850000in}}%
\pgfusepath{clip}%
\pgfsetbuttcap%
\pgfsetroundjoin%
\pgfsetlinewidth{1.505625pt}%
\definecolor{currentstroke}{rgb}{0.000000,0.000000,0.000000}%
\pgfsetstrokecolor{currentstroke}%
\pgfsetdash{{5.550000pt}{2.400000pt}}{0.000000pt}%
\pgfpathmoveto{\pgfqpoint{0.801136in}{4.219810in}}%
\pgfpathlineto{\pgfqpoint{4.323864in}{0.767230in}}%
\pgfusepath{stroke}%
\end{pgfscope}%
\begin{pgfscope}%
\pgfsetrectcap%
\pgfsetmiterjoin%
\pgfsetlinewidth{0.803000pt}%
\definecolor{currentstroke}{rgb}{0.000000,0.000000,0.000000}%
\pgfsetstrokecolor{currentstroke}%
\pgfsetdash{}{0pt}%
\pgfpathmoveto{\pgfqpoint{0.625000in}{0.550000in}}%
\pgfpathlineto{\pgfqpoint{0.625000in}{4.400000in}}%
\pgfusepath{stroke}%
\end{pgfscope}%
\begin{pgfscope}%
\pgfsetrectcap%
\pgfsetmiterjoin%
\pgfsetlinewidth{0.000000pt}%
\definecolor{currentstroke}{rgb}{0.000000,0.000000,0.000000}%
\pgfsetstrokecolor{currentstroke}%
\pgfsetstrokeopacity{0.000000}%
\pgfsetdash{}{0pt}%
\pgfpathmoveto{\pgfqpoint{4.500000in}{0.550000in}}%
\pgfpathlineto{\pgfqpoint{4.500000in}{4.400000in}}%
\pgfusepath{}%
\end{pgfscope}%
\begin{pgfscope}%
\pgfsetrectcap%
\pgfsetmiterjoin%
\pgfsetlinewidth{0.803000pt}%
\definecolor{currentstroke}{rgb}{0.000000,0.000000,0.000000}%
\pgfsetstrokecolor{currentstroke}%
\pgfsetdash{}{0pt}%
\pgfpathmoveto{\pgfqpoint{0.625000in}{0.550000in}}%
\pgfpathlineto{\pgfqpoint{4.500000in}{0.550000in}}%
\pgfusepath{stroke}%
\end{pgfscope}%
\begin{pgfscope}%
\pgfsetrectcap%
\pgfsetmiterjoin%
\pgfsetlinewidth{0.000000pt}%
\definecolor{currentstroke}{rgb}{0.000000,0.000000,0.000000}%
\pgfsetstrokecolor{currentstroke}%
\pgfsetstrokeopacity{0.000000}%
\pgfsetdash{}{0pt}%
\pgfpathmoveto{\pgfqpoint{0.625000in}{4.400000in}}%
\pgfpathlineto{\pgfqpoint{4.500000in}{4.400000in}}%
\pgfusepath{}%
\end{pgfscope}%
\begin{pgfscope}%
\pgfsetbuttcap%
\pgfsetmiterjoin%
\definecolor{currentfill}{rgb}{1.000000,1.000000,1.000000}%
\pgfsetfillcolor{currentfill}%
\pgfsetfillopacity{0.800000}%
\pgfsetlinewidth{1.003750pt}%
\definecolor{currentstroke}{rgb}{0.800000,0.800000,0.800000}%
\pgfsetstrokecolor{currentstroke}%
\pgfsetstrokeopacity{0.800000}%
\pgfsetdash{}{0pt}%
\pgfpathmoveto{\pgfqpoint{2.916233in}{3.999488in}}%
\pgfpathlineto{\pgfqpoint{4.393056in}{3.999488in}}%
\pgfpathquadraticcurveto{\pgfqpoint{4.423611in}{3.999488in}}{\pgfqpoint{4.423611in}{4.030043in}}%
\pgfpathlineto{\pgfqpoint{4.423611in}{4.293056in}}%
\pgfpathquadraticcurveto{\pgfqpoint{4.423611in}{4.323611in}}{\pgfqpoint{4.393056in}{4.323611in}}%
\pgfpathlineto{\pgfqpoint{2.916233in}{4.323611in}}%
\pgfpathquadraticcurveto{\pgfqpoint{2.885678in}{4.323611in}}{\pgfqpoint{2.885678in}{4.293056in}}%
\pgfpathlineto{\pgfqpoint{2.885678in}{4.030043in}}%
\pgfpathquadraticcurveto{\pgfqpoint{2.885678in}{3.999488in}}{\pgfqpoint{2.916233in}{3.999488in}}%
\pgfpathclose%
\pgfusepath{stroke,fill}%
\end{pgfscope}%
\begin{pgfscope}%
\pgfsetbuttcap%
\pgfsetroundjoin%
\pgfsetlinewidth{1.505625pt}%
\definecolor{currentstroke}{rgb}{0.000000,0.000000,0.000000}%
\pgfsetstrokecolor{currentstroke}%
\pgfsetdash{{5.550000pt}{2.400000pt}}{0.000000pt}%
\pgfpathmoveto{\pgfqpoint{2.946789in}{4.180408in}}%
\pgfpathlineto{\pgfqpoint{3.252344in}{4.180408in}}%
\pgfusepath{stroke}%
\end{pgfscope}%
\begin{pgfscope}%
\definecolor{textcolor}{rgb}{0.000000,0.000000,0.000000}%
\pgfsetstrokecolor{textcolor}%
\pgfsetfillcolor{textcolor}%
\pgftext[x=3.374567in,y=4.126936in,left,base]{\color{textcolor}\sffamily\fontsize{11.000000}{13.200000}\selectfont \(\displaystyle 0.1612N_{\mathrm{QMC}}^{-0.7528}\)}%
\end{pgfscope}%
\end{pgfpicture}%
\makeatother%
\endgroup%

\caption[Quasi-Monte-Carlo error, for $Q=u((1,1))$ and $k=40$.]{Quasi-Monte-Carlo error with increasing number of Quasi-Monte-Carlo points, for $Q=u((1,1))$ and $k=40$. \label{fig:qmctop_right40}}
\end{figure}
\begin{figure}[h]
    \centering
    %% Creator: Matplotlib, PGF backend
%%
%% To include the figure in your LaTeX document, write
%%   \input{<filename>.pgf}
%%
%% Make sure the required packages are loaded in your preamble
%%   \usepackage{pgf}
%%
%% Figures using additional raster images can only be included by \input if
%% they are in the same directory as the main LaTeX file. For loading figures
%% from other directories you can use the `import` package
%%   \usepackage{import}
%% and then include the figures with
%%   \import{<path to file>}{<filename>.pgf}
%%
%% Matplotlib used the following preamble
%%   \usepackage{fontspec}
%%   \setmainfont{DejaVuSerif.ttf}[Path=/home/owen/progs/firedrake-complex/firedrake/lib/python3.5/site-packages/matplotlib/mpl-data/fonts/ttf/]
%%   \setsansfont{DejaVuSans.ttf}[Path=/home/owen/progs/firedrake-complex/firedrake/lib/python3.5/site-packages/matplotlib/mpl-data/fonts/ttf/]
%%   \setmonofont{DejaVuSansMono.ttf}[Path=/home/owen/progs/firedrake-complex/firedrake/lib/python3.5/site-packages/matplotlib/mpl-data/fonts/ttf/]
%%
\begingroup%
\makeatletter%
\begin{pgfpicture}%
\pgfpathrectangle{\pgfpointorigin}{\pgfqpoint{5.000000in}{5.000000in}}%
\pgfusepath{use as bounding box, clip}%
\begin{pgfscope}%
\pgfsetbuttcap%
\pgfsetmiterjoin%
\definecolor{currentfill}{rgb}{1.000000,1.000000,1.000000}%
\pgfsetfillcolor{currentfill}%
\pgfsetlinewidth{0.000000pt}%
\definecolor{currentstroke}{rgb}{1.000000,1.000000,1.000000}%
\pgfsetstrokecolor{currentstroke}%
\pgfsetdash{}{0pt}%
\pgfpathmoveto{\pgfqpoint{0.000000in}{0.000000in}}%
\pgfpathlineto{\pgfqpoint{5.000000in}{0.000000in}}%
\pgfpathlineto{\pgfqpoint{5.000000in}{5.000000in}}%
\pgfpathlineto{\pgfqpoint{0.000000in}{5.000000in}}%
\pgfpathclose%
\pgfusepath{fill}%
\end{pgfscope}%
\begin{pgfscope}%
\pgfsetbuttcap%
\pgfsetmiterjoin%
\definecolor{currentfill}{rgb}{1.000000,1.000000,1.000000}%
\pgfsetfillcolor{currentfill}%
\pgfsetlinewidth{0.000000pt}%
\definecolor{currentstroke}{rgb}{0.000000,0.000000,0.000000}%
\pgfsetstrokecolor{currentstroke}%
\pgfsetstrokeopacity{0.000000}%
\pgfsetdash{}{0pt}%
\pgfpathmoveto{\pgfqpoint{0.625000in}{0.550000in}}%
\pgfpathlineto{\pgfqpoint{4.500000in}{0.550000in}}%
\pgfpathlineto{\pgfqpoint{4.500000in}{4.400000in}}%
\pgfpathlineto{\pgfqpoint{0.625000in}{4.400000in}}%
\pgfpathclose%
\pgfusepath{fill}%
\end{pgfscope}%
\begin{pgfscope}%
\pgfsetbuttcap%
\pgfsetroundjoin%
\definecolor{currentfill}{rgb}{0.000000,0.000000,0.000000}%
\pgfsetfillcolor{currentfill}%
\pgfsetlinewidth{0.803000pt}%
\definecolor{currentstroke}{rgb}{0.000000,0.000000,0.000000}%
\pgfsetstrokecolor{currentstroke}%
\pgfsetdash{}{0pt}%
\pgfsys@defobject{currentmarker}{\pgfqpoint{0.000000in}{-0.048611in}}{\pgfqpoint{0.000000in}{0.000000in}}{%
\pgfpathmoveto{\pgfqpoint{0.000000in}{0.000000in}}%
\pgfpathlineto{\pgfqpoint{0.000000in}{-0.048611in}}%
\pgfusepath{stroke,fill}%
}%
\begin{pgfscope}%
\pgfsys@transformshift{0.801136in}{0.550000in}%
\pgfsys@useobject{currentmarker}{}%
\end{pgfscope}%
\end{pgfscope}%
\begin{pgfscope}%
\definecolor{textcolor}{rgb}{0.000000,0.000000,0.000000}%
\pgfsetstrokecolor{textcolor}%
\pgfsetfillcolor{textcolor}%
\pgftext[x=0.801136in,y=0.452778in,,top]{\color{textcolor}\sffamily\fontsize{11.000000}{13.200000}\selectfont \(\displaystyle 10^{0}\)}%
\end{pgfscope}%
\begin{pgfscope}%
\pgfsetbuttcap%
\pgfsetroundjoin%
\definecolor{currentfill}{rgb}{0.000000,0.000000,0.000000}%
\pgfsetfillcolor{currentfill}%
\pgfsetlinewidth{0.803000pt}%
\definecolor{currentstroke}{rgb}{0.000000,0.000000,0.000000}%
\pgfsetstrokecolor{currentstroke}%
\pgfsetdash{}{0pt}%
\pgfsys@defobject{currentmarker}{\pgfqpoint{0.000000in}{-0.048611in}}{\pgfqpoint{0.000000in}{0.000000in}}{%
\pgfpathmoveto{\pgfqpoint{0.000000in}{0.000000in}}%
\pgfpathlineto{\pgfqpoint{0.000000in}{-0.048611in}}%
\pgfusepath{stroke,fill}%
}%
\begin{pgfscope}%
\pgfsys@transformshift{1.864977in}{0.550000in}%
\pgfsys@useobject{currentmarker}{}%
\end{pgfscope}%
\end{pgfscope}%
\begin{pgfscope}%
\definecolor{textcolor}{rgb}{0.000000,0.000000,0.000000}%
\pgfsetstrokecolor{textcolor}%
\pgfsetfillcolor{textcolor}%
\pgftext[x=1.864977in,y=0.452778in,,top]{\color{textcolor}\sffamily\fontsize{11.000000}{13.200000}\selectfont \(\displaystyle 10^{1}\)}%
\end{pgfscope}%
\begin{pgfscope}%
\pgfsetbuttcap%
\pgfsetroundjoin%
\definecolor{currentfill}{rgb}{0.000000,0.000000,0.000000}%
\pgfsetfillcolor{currentfill}%
\pgfsetlinewidth{0.803000pt}%
\definecolor{currentstroke}{rgb}{0.000000,0.000000,0.000000}%
\pgfsetstrokecolor{currentstroke}%
\pgfsetdash{}{0pt}%
\pgfsys@defobject{currentmarker}{\pgfqpoint{0.000000in}{-0.048611in}}{\pgfqpoint{0.000000in}{0.000000in}}{%
\pgfpathmoveto{\pgfqpoint{0.000000in}{0.000000in}}%
\pgfpathlineto{\pgfqpoint{0.000000in}{-0.048611in}}%
\pgfusepath{stroke,fill}%
}%
\begin{pgfscope}%
\pgfsys@transformshift{2.928818in}{0.550000in}%
\pgfsys@useobject{currentmarker}{}%
\end{pgfscope}%
\end{pgfscope}%
\begin{pgfscope}%
\definecolor{textcolor}{rgb}{0.000000,0.000000,0.000000}%
\pgfsetstrokecolor{textcolor}%
\pgfsetfillcolor{textcolor}%
\pgftext[x=2.928818in,y=0.452778in,,top]{\color{textcolor}\sffamily\fontsize{11.000000}{13.200000}\selectfont \(\displaystyle 10^{2}\)}%
\end{pgfscope}%
\begin{pgfscope}%
\pgfsetbuttcap%
\pgfsetroundjoin%
\definecolor{currentfill}{rgb}{0.000000,0.000000,0.000000}%
\pgfsetfillcolor{currentfill}%
\pgfsetlinewidth{0.803000pt}%
\definecolor{currentstroke}{rgb}{0.000000,0.000000,0.000000}%
\pgfsetstrokecolor{currentstroke}%
\pgfsetdash{}{0pt}%
\pgfsys@defobject{currentmarker}{\pgfqpoint{0.000000in}{-0.048611in}}{\pgfqpoint{0.000000in}{0.000000in}}{%
\pgfpathmoveto{\pgfqpoint{0.000000in}{0.000000in}}%
\pgfpathlineto{\pgfqpoint{0.000000in}{-0.048611in}}%
\pgfusepath{stroke,fill}%
}%
\begin{pgfscope}%
\pgfsys@transformshift{3.992658in}{0.550000in}%
\pgfsys@useobject{currentmarker}{}%
\end{pgfscope}%
\end{pgfscope}%
\begin{pgfscope}%
\definecolor{textcolor}{rgb}{0.000000,0.000000,0.000000}%
\pgfsetstrokecolor{textcolor}%
\pgfsetfillcolor{textcolor}%
\pgftext[x=3.992658in,y=0.452778in,,top]{\color{textcolor}\sffamily\fontsize{11.000000}{13.200000}\selectfont \(\displaystyle 10^{3}\)}%
\end{pgfscope}%
\begin{pgfscope}%
\pgfsetbuttcap%
\pgfsetroundjoin%
\definecolor{currentfill}{rgb}{0.000000,0.000000,0.000000}%
\pgfsetfillcolor{currentfill}%
\pgfsetlinewidth{0.602250pt}%
\definecolor{currentstroke}{rgb}{0.000000,0.000000,0.000000}%
\pgfsetstrokecolor{currentstroke}%
\pgfsetdash{}{0pt}%
\pgfsys@defobject{currentmarker}{\pgfqpoint{0.000000in}{-0.027778in}}{\pgfqpoint{0.000000in}{0.000000in}}{%
\pgfpathmoveto{\pgfqpoint{0.000000in}{0.000000in}}%
\pgfpathlineto{\pgfqpoint{0.000000in}{-0.027778in}}%
\pgfusepath{stroke,fill}%
}%
\begin{pgfscope}%
\pgfsys@transformshift{0.636345in}{0.550000in}%
\pgfsys@useobject{currentmarker}{}%
\end{pgfscope}%
\end{pgfscope}%
\begin{pgfscope}%
\pgfsetbuttcap%
\pgfsetroundjoin%
\definecolor{currentfill}{rgb}{0.000000,0.000000,0.000000}%
\pgfsetfillcolor{currentfill}%
\pgfsetlinewidth{0.602250pt}%
\definecolor{currentstroke}{rgb}{0.000000,0.000000,0.000000}%
\pgfsetstrokecolor{currentstroke}%
\pgfsetdash{}{0pt}%
\pgfsys@defobject{currentmarker}{\pgfqpoint{0.000000in}{-0.027778in}}{\pgfqpoint{0.000000in}{0.000000in}}{%
\pgfpathmoveto{\pgfqpoint{0.000000in}{0.000000in}}%
\pgfpathlineto{\pgfqpoint{0.000000in}{-0.027778in}}%
\pgfusepath{stroke,fill}%
}%
\begin{pgfscope}%
\pgfsys@transformshift{0.698040in}{0.550000in}%
\pgfsys@useobject{currentmarker}{}%
\end{pgfscope}%
\end{pgfscope}%
\begin{pgfscope}%
\pgfsetbuttcap%
\pgfsetroundjoin%
\definecolor{currentfill}{rgb}{0.000000,0.000000,0.000000}%
\pgfsetfillcolor{currentfill}%
\pgfsetlinewidth{0.602250pt}%
\definecolor{currentstroke}{rgb}{0.000000,0.000000,0.000000}%
\pgfsetstrokecolor{currentstroke}%
\pgfsetdash{}{0pt}%
\pgfsys@defobject{currentmarker}{\pgfqpoint{0.000000in}{-0.027778in}}{\pgfqpoint{0.000000in}{0.000000in}}{%
\pgfpathmoveto{\pgfqpoint{0.000000in}{0.000000in}}%
\pgfpathlineto{\pgfqpoint{0.000000in}{-0.027778in}}%
\pgfusepath{stroke,fill}%
}%
\begin{pgfscope}%
\pgfsys@transformshift{0.752458in}{0.550000in}%
\pgfsys@useobject{currentmarker}{}%
\end{pgfscope}%
\end{pgfscope}%
\begin{pgfscope}%
\pgfsetbuttcap%
\pgfsetroundjoin%
\definecolor{currentfill}{rgb}{0.000000,0.000000,0.000000}%
\pgfsetfillcolor{currentfill}%
\pgfsetlinewidth{0.602250pt}%
\definecolor{currentstroke}{rgb}{0.000000,0.000000,0.000000}%
\pgfsetstrokecolor{currentstroke}%
\pgfsetdash{}{0pt}%
\pgfsys@defobject{currentmarker}{\pgfqpoint{0.000000in}{-0.027778in}}{\pgfqpoint{0.000000in}{0.000000in}}{%
\pgfpathmoveto{\pgfqpoint{0.000000in}{0.000000in}}%
\pgfpathlineto{\pgfqpoint{0.000000in}{-0.027778in}}%
\pgfusepath{stroke,fill}%
}%
\begin{pgfscope}%
\pgfsys@transformshift{1.121384in}{0.550000in}%
\pgfsys@useobject{currentmarker}{}%
\end{pgfscope}%
\end{pgfscope}%
\begin{pgfscope}%
\pgfsetbuttcap%
\pgfsetroundjoin%
\definecolor{currentfill}{rgb}{0.000000,0.000000,0.000000}%
\pgfsetfillcolor{currentfill}%
\pgfsetlinewidth{0.602250pt}%
\definecolor{currentstroke}{rgb}{0.000000,0.000000,0.000000}%
\pgfsetstrokecolor{currentstroke}%
\pgfsetdash{}{0pt}%
\pgfsys@defobject{currentmarker}{\pgfqpoint{0.000000in}{-0.027778in}}{\pgfqpoint{0.000000in}{0.000000in}}{%
\pgfpathmoveto{\pgfqpoint{0.000000in}{0.000000in}}%
\pgfpathlineto{\pgfqpoint{0.000000in}{-0.027778in}}%
\pgfusepath{stroke,fill}%
}%
\begin{pgfscope}%
\pgfsys@transformshift{1.308717in}{0.550000in}%
\pgfsys@useobject{currentmarker}{}%
\end{pgfscope}%
\end{pgfscope}%
\begin{pgfscope}%
\pgfsetbuttcap%
\pgfsetroundjoin%
\definecolor{currentfill}{rgb}{0.000000,0.000000,0.000000}%
\pgfsetfillcolor{currentfill}%
\pgfsetlinewidth{0.602250pt}%
\definecolor{currentstroke}{rgb}{0.000000,0.000000,0.000000}%
\pgfsetstrokecolor{currentstroke}%
\pgfsetdash{}{0pt}%
\pgfsys@defobject{currentmarker}{\pgfqpoint{0.000000in}{-0.027778in}}{\pgfqpoint{0.000000in}{0.000000in}}{%
\pgfpathmoveto{\pgfqpoint{0.000000in}{0.000000in}}%
\pgfpathlineto{\pgfqpoint{0.000000in}{-0.027778in}}%
\pgfusepath{stroke,fill}%
}%
\begin{pgfscope}%
\pgfsys@transformshift{1.441632in}{0.550000in}%
\pgfsys@useobject{currentmarker}{}%
\end{pgfscope}%
\end{pgfscope}%
\begin{pgfscope}%
\pgfsetbuttcap%
\pgfsetroundjoin%
\definecolor{currentfill}{rgb}{0.000000,0.000000,0.000000}%
\pgfsetfillcolor{currentfill}%
\pgfsetlinewidth{0.602250pt}%
\definecolor{currentstroke}{rgb}{0.000000,0.000000,0.000000}%
\pgfsetstrokecolor{currentstroke}%
\pgfsetdash{}{0pt}%
\pgfsys@defobject{currentmarker}{\pgfqpoint{0.000000in}{-0.027778in}}{\pgfqpoint{0.000000in}{0.000000in}}{%
\pgfpathmoveto{\pgfqpoint{0.000000in}{0.000000in}}%
\pgfpathlineto{\pgfqpoint{0.000000in}{-0.027778in}}%
\pgfusepath{stroke,fill}%
}%
\begin{pgfscope}%
\pgfsys@transformshift{1.544729in}{0.550000in}%
\pgfsys@useobject{currentmarker}{}%
\end{pgfscope}%
\end{pgfscope}%
\begin{pgfscope}%
\pgfsetbuttcap%
\pgfsetroundjoin%
\definecolor{currentfill}{rgb}{0.000000,0.000000,0.000000}%
\pgfsetfillcolor{currentfill}%
\pgfsetlinewidth{0.602250pt}%
\definecolor{currentstroke}{rgb}{0.000000,0.000000,0.000000}%
\pgfsetstrokecolor{currentstroke}%
\pgfsetdash{}{0pt}%
\pgfsys@defobject{currentmarker}{\pgfqpoint{0.000000in}{-0.027778in}}{\pgfqpoint{0.000000in}{0.000000in}}{%
\pgfpathmoveto{\pgfqpoint{0.000000in}{0.000000in}}%
\pgfpathlineto{\pgfqpoint{0.000000in}{-0.027778in}}%
\pgfusepath{stroke,fill}%
}%
\begin{pgfscope}%
\pgfsys@transformshift{1.628965in}{0.550000in}%
\pgfsys@useobject{currentmarker}{}%
\end{pgfscope}%
\end{pgfscope}%
\begin{pgfscope}%
\pgfsetbuttcap%
\pgfsetroundjoin%
\definecolor{currentfill}{rgb}{0.000000,0.000000,0.000000}%
\pgfsetfillcolor{currentfill}%
\pgfsetlinewidth{0.602250pt}%
\definecolor{currentstroke}{rgb}{0.000000,0.000000,0.000000}%
\pgfsetstrokecolor{currentstroke}%
\pgfsetdash{}{0pt}%
\pgfsys@defobject{currentmarker}{\pgfqpoint{0.000000in}{-0.027778in}}{\pgfqpoint{0.000000in}{0.000000in}}{%
\pgfpathmoveto{\pgfqpoint{0.000000in}{0.000000in}}%
\pgfpathlineto{\pgfqpoint{0.000000in}{-0.027778in}}%
\pgfusepath{stroke,fill}%
}%
\begin{pgfscope}%
\pgfsys@transformshift{1.700186in}{0.550000in}%
\pgfsys@useobject{currentmarker}{}%
\end{pgfscope}%
\end{pgfscope}%
\begin{pgfscope}%
\pgfsetbuttcap%
\pgfsetroundjoin%
\definecolor{currentfill}{rgb}{0.000000,0.000000,0.000000}%
\pgfsetfillcolor{currentfill}%
\pgfsetlinewidth{0.602250pt}%
\definecolor{currentstroke}{rgb}{0.000000,0.000000,0.000000}%
\pgfsetstrokecolor{currentstroke}%
\pgfsetdash{}{0pt}%
\pgfsys@defobject{currentmarker}{\pgfqpoint{0.000000in}{-0.027778in}}{\pgfqpoint{0.000000in}{0.000000in}}{%
\pgfpathmoveto{\pgfqpoint{0.000000in}{0.000000in}}%
\pgfpathlineto{\pgfqpoint{0.000000in}{-0.027778in}}%
\pgfusepath{stroke,fill}%
}%
\begin{pgfscope}%
\pgfsys@transformshift{1.761880in}{0.550000in}%
\pgfsys@useobject{currentmarker}{}%
\end{pgfscope}%
\end{pgfscope}%
\begin{pgfscope}%
\pgfsetbuttcap%
\pgfsetroundjoin%
\definecolor{currentfill}{rgb}{0.000000,0.000000,0.000000}%
\pgfsetfillcolor{currentfill}%
\pgfsetlinewidth{0.602250pt}%
\definecolor{currentstroke}{rgb}{0.000000,0.000000,0.000000}%
\pgfsetstrokecolor{currentstroke}%
\pgfsetdash{}{0pt}%
\pgfsys@defobject{currentmarker}{\pgfqpoint{0.000000in}{-0.027778in}}{\pgfqpoint{0.000000in}{0.000000in}}{%
\pgfpathmoveto{\pgfqpoint{0.000000in}{0.000000in}}%
\pgfpathlineto{\pgfqpoint{0.000000in}{-0.027778in}}%
\pgfusepath{stroke,fill}%
}%
\begin{pgfscope}%
\pgfsys@transformshift{1.816298in}{0.550000in}%
\pgfsys@useobject{currentmarker}{}%
\end{pgfscope}%
\end{pgfscope}%
\begin{pgfscope}%
\pgfsetbuttcap%
\pgfsetroundjoin%
\definecolor{currentfill}{rgb}{0.000000,0.000000,0.000000}%
\pgfsetfillcolor{currentfill}%
\pgfsetlinewidth{0.602250pt}%
\definecolor{currentstroke}{rgb}{0.000000,0.000000,0.000000}%
\pgfsetstrokecolor{currentstroke}%
\pgfsetdash{}{0pt}%
\pgfsys@defobject{currentmarker}{\pgfqpoint{0.000000in}{-0.027778in}}{\pgfqpoint{0.000000in}{0.000000in}}{%
\pgfpathmoveto{\pgfqpoint{0.000000in}{0.000000in}}%
\pgfpathlineto{\pgfqpoint{0.000000in}{-0.027778in}}%
\pgfusepath{stroke,fill}%
}%
\begin{pgfscope}%
\pgfsys@transformshift{2.185225in}{0.550000in}%
\pgfsys@useobject{currentmarker}{}%
\end{pgfscope}%
\end{pgfscope}%
\begin{pgfscope}%
\pgfsetbuttcap%
\pgfsetroundjoin%
\definecolor{currentfill}{rgb}{0.000000,0.000000,0.000000}%
\pgfsetfillcolor{currentfill}%
\pgfsetlinewidth{0.602250pt}%
\definecolor{currentstroke}{rgb}{0.000000,0.000000,0.000000}%
\pgfsetstrokecolor{currentstroke}%
\pgfsetdash{}{0pt}%
\pgfsys@defobject{currentmarker}{\pgfqpoint{0.000000in}{-0.027778in}}{\pgfqpoint{0.000000in}{0.000000in}}{%
\pgfpathmoveto{\pgfqpoint{0.000000in}{0.000000in}}%
\pgfpathlineto{\pgfqpoint{0.000000in}{-0.027778in}}%
\pgfusepath{stroke,fill}%
}%
\begin{pgfscope}%
\pgfsys@transformshift{2.372558in}{0.550000in}%
\pgfsys@useobject{currentmarker}{}%
\end{pgfscope}%
\end{pgfscope}%
\begin{pgfscope}%
\pgfsetbuttcap%
\pgfsetroundjoin%
\definecolor{currentfill}{rgb}{0.000000,0.000000,0.000000}%
\pgfsetfillcolor{currentfill}%
\pgfsetlinewidth{0.602250pt}%
\definecolor{currentstroke}{rgb}{0.000000,0.000000,0.000000}%
\pgfsetstrokecolor{currentstroke}%
\pgfsetdash{}{0pt}%
\pgfsys@defobject{currentmarker}{\pgfqpoint{0.000000in}{-0.027778in}}{\pgfqpoint{0.000000in}{0.000000in}}{%
\pgfpathmoveto{\pgfqpoint{0.000000in}{0.000000in}}%
\pgfpathlineto{\pgfqpoint{0.000000in}{-0.027778in}}%
\pgfusepath{stroke,fill}%
}%
\begin{pgfscope}%
\pgfsys@transformshift{2.505473in}{0.550000in}%
\pgfsys@useobject{currentmarker}{}%
\end{pgfscope}%
\end{pgfscope}%
\begin{pgfscope}%
\pgfsetbuttcap%
\pgfsetroundjoin%
\definecolor{currentfill}{rgb}{0.000000,0.000000,0.000000}%
\pgfsetfillcolor{currentfill}%
\pgfsetlinewidth{0.602250pt}%
\definecolor{currentstroke}{rgb}{0.000000,0.000000,0.000000}%
\pgfsetstrokecolor{currentstroke}%
\pgfsetdash{}{0pt}%
\pgfsys@defobject{currentmarker}{\pgfqpoint{0.000000in}{-0.027778in}}{\pgfqpoint{0.000000in}{0.000000in}}{%
\pgfpathmoveto{\pgfqpoint{0.000000in}{0.000000in}}%
\pgfpathlineto{\pgfqpoint{0.000000in}{-0.027778in}}%
\pgfusepath{stroke,fill}%
}%
\begin{pgfscope}%
\pgfsys@transformshift{2.608570in}{0.550000in}%
\pgfsys@useobject{currentmarker}{}%
\end{pgfscope}%
\end{pgfscope}%
\begin{pgfscope}%
\pgfsetbuttcap%
\pgfsetroundjoin%
\definecolor{currentfill}{rgb}{0.000000,0.000000,0.000000}%
\pgfsetfillcolor{currentfill}%
\pgfsetlinewidth{0.602250pt}%
\definecolor{currentstroke}{rgb}{0.000000,0.000000,0.000000}%
\pgfsetstrokecolor{currentstroke}%
\pgfsetdash{}{0pt}%
\pgfsys@defobject{currentmarker}{\pgfqpoint{0.000000in}{-0.027778in}}{\pgfqpoint{0.000000in}{0.000000in}}{%
\pgfpathmoveto{\pgfqpoint{0.000000in}{0.000000in}}%
\pgfpathlineto{\pgfqpoint{0.000000in}{-0.027778in}}%
\pgfusepath{stroke,fill}%
}%
\begin{pgfscope}%
\pgfsys@transformshift{2.692806in}{0.550000in}%
\pgfsys@useobject{currentmarker}{}%
\end{pgfscope}%
\end{pgfscope}%
\begin{pgfscope}%
\pgfsetbuttcap%
\pgfsetroundjoin%
\definecolor{currentfill}{rgb}{0.000000,0.000000,0.000000}%
\pgfsetfillcolor{currentfill}%
\pgfsetlinewidth{0.602250pt}%
\definecolor{currentstroke}{rgb}{0.000000,0.000000,0.000000}%
\pgfsetstrokecolor{currentstroke}%
\pgfsetdash{}{0pt}%
\pgfsys@defobject{currentmarker}{\pgfqpoint{0.000000in}{-0.027778in}}{\pgfqpoint{0.000000in}{0.000000in}}{%
\pgfpathmoveto{\pgfqpoint{0.000000in}{0.000000in}}%
\pgfpathlineto{\pgfqpoint{0.000000in}{-0.027778in}}%
\pgfusepath{stroke,fill}%
}%
\begin{pgfscope}%
\pgfsys@transformshift{2.764027in}{0.550000in}%
\pgfsys@useobject{currentmarker}{}%
\end{pgfscope}%
\end{pgfscope}%
\begin{pgfscope}%
\pgfsetbuttcap%
\pgfsetroundjoin%
\definecolor{currentfill}{rgb}{0.000000,0.000000,0.000000}%
\pgfsetfillcolor{currentfill}%
\pgfsetlinewidth{0.602250pt}%
\definecolor{currentstroke}{rgb}{0.000000,0.000000,0.000000}%
\pgfsetstrokecolor{currentstroke}%
\pgfsetdash{}{0pt}%
\pgfsys@defobject{currentmarker}{\pgfqpoint{0.000000in}{-0.027778in}}{\pgfqpoint{0.000000in}{0.000000in}}{%
\pgfpathmoveto{\pgfqpoint{0.000000in}{0.000000in}}%
\pgfpathlineto{\pgfqpoint{0.000000in}{-0.027778in}}%
\pgfusepath{stroke,fill}%
}%
\begin{pgfscope}%
\pgfsys@transformshift{2.825721in}{0.550000in}%
\pgfsys@useobject{currentmarker}{}%
\end{pgfscope}%
\end{pgfscope}%
\begin{pgfscope}%
\pgfsetbuttcap%
\pgfsetroundjoin%
\definecolor{currentfill}{rgb}{0.000000,0.000000,0.000000}%
\pgfsetfillcolor{currentfill}%
\pgfsetlinewidth{0.602250pt}%
\definecolor{currentstroke}{rgb}{0.000000,0.000000,0.000000}%
\pgfsetstrokecolor{currentstroke}%
\pgfsetdash{}{0pt}%
\pgfsys@defobject{currentmarker}{\pgfqpoint{0.000000in}{-0.027778in}}{\pgfqpoint{0.000000in}{0.000000in}}{%
\pgfpathmoveto{\pgfqpoint{0.000000in}{0.000000in}}%
\pgfpathlineto{\pgfqpoint{0.000000in}{-0.027778in}}%
\pgfusepath{stroke,fill}%
}%
\begin{pgfscope}%
\pgfsys@transformshift{2.880139in}{0.550000in}%
\pgfsys@useobject{currentmarker}{}%
\end{pgfscope}%
\end{pgfscope}%
\begin{pgfscope}%
\pgfsetbuttcap%
\pgfsetroundjoin%
\definecolor{currentfill}{rgb}{0.000000,0.000000,0.000000}%
\pgfsetfillcolor{currentfill}%
\pgfsetlinewidth{0.602250pt}%
\definecolor{currentstroke}{rgb}{0.000000,0.000000,0.000000}%
\pgfsetstrokecolor{currentstroke}%
\pgfsetdash{}{0pt}%
\pgfsys@defobject{currentmarker}{\pgfqpoint{0.000000in}{-0.027778in}}{\pgfqpoint{0.000000in}{0.000000in}}{%
\pgfpathmoveto{\pgfqpoint{0.000000in}{0.000000in}}%
\pgfpathlineto{\pgfqpoint{0.000000in}{-0.027778in}}%
\pgfusepath{stroke,fill}%
}%
\begin{pgfscope}%
\pgfsys@transformshift{3.249066in}{0.550000in}%
\pgfsys@useobject{currentmarker}{}%
\end{pgfscope}%
\end{pgfscope}%
\begin{pgfscope}%
\pgfsetbuttcap%
\pgfsetroundjoin%
\definecolor{currentfill}{rgb}{0.000000,0.000000,0.000000}%
\pgfsetfillcolor{currentfill}%
\pgfsetlinewidth{0.602250pt}%
\definecolor{currentstroke}{rgb}{0.000000,0.000000,0.000000}%
\pgfsetstrokecolor{currentstroke}%
\pgfsetdash{}{0pt}%
\pgfsys@defobject{currentmarker}{\pgfqpoint{0.000000in}{-0.027778in}}{\pgfqpoint{0.000000in}{0.000000in}}{%
\pgfpathmoveto{\pgfqpoint{0.000000in}{0.000000in}}%
\pgfpathlineto{\pgfqpoint{0.000000in}{-0.027778in}}%
\pgfusepath{stroke,fill}%
}%
\begin{pgfscope}%
\pgfsys@transformshift{3.436399in}{0.550000in}%
\pgfsys@useobject{currentmarker}{}%
\end{pgfscope}%
\end{pgfscope}%
\begin{pgfscope}%
\pgfsetbuttcap%
\pgfsetroundjoin%
\definecolor{currentfill}{rgb}{0.000000,0.000000,0.000000}%
\pgfsetfillcolor{currentfill}%
\pgfsetlinewidth{0.602250pt}%
\definecolor{currentstroke}{rgb}{0.000000,0.000000,0.000000}%
\pgfsetstrokecolor{currentstroke}%
\pgfsetdash{}{0pt}%
\pgfsys@defobject{currentmarker}{\pgfqpoint{0.000000in}{-0.027778in}}{\pgfqpoint{0.000000in}{0.000000in}}{%
\pgfpathmoveto{\pgfqpoint{0.000000in}{0.000000in}}%
\pgfpathlineto{\pgfqpoint{0.000000in}{-0.027778in}}%
\pgfusepath{stroke,fill}%
}%
\begin{pgfscope}%
\pgfsys@transformshift{3.569313in}{0.550000in}%
\pgfsys@useobject{currentmarker}{}%
\end{pgfscope}%
\end{pgfscope}%
\begin{pgfscope}%
\pgfsetbuttcap%
\pgfsetroundjoin%
\definecolor{currentfill}{rgb}{0.000000,0.000000,0.000000}%
\pgfsetfillcolor{currentfill}%
\pgfsetlinewidth{0.602250pt}%
\definecolor{currentstroke}{rgb}{0.000000,0.000000,0.000000}%
\pgfsetstrokecolor{currentstroke}%
\pgfsetdash{}{0pt}%
\pgfsys@defobject{currentmarker}{\pgfqpoint{0.000000in}{-0.027778in}}{\pgfqpoint{0.000000in}{0.000000in}}{%
\pgfpathmoveto{\pgfqpoint{0.000000in}{0.000000in}}%
\pgfpathlineto{\pgfqpoint{0.000000in}{-0.027778in}}%
\pgfusepath{stroke,fill}%
}%
\begin{pgfscope}%
\pgfsys@transformshift{3.672410in}{0.550000in}%
\pgfsys@useobject{currentmarker}{}%
\end{pgfscope}%
\end{pgfscope}%
\begin{pgfscope}%
\pgfsetbuttcap%
\pgfsetroundjoin%
\definecolor{currentfill}{rgb}{0.000000,0.000000,0.000000}%
\pgfsetfillcolor{currentfill}%
\pgfsetlinewidth{0.602250pt}%
\definecolor{currentstroke}{rgb}{0.000000,0.000000,0.000000}%
\pgfsetstrokecolor{currentstroke}%
\pgfsetdash{}{0pt}%
\pgfsys@defobject{currentmarker}{\pgfqpoint{0.000000in}{-0.027778in}}{\pgfqpoint{0.000000in}{0.000000in}}{%
\pgfpathmoveto{\pgfqpoint{0.000000in}{0.000000in}}%
\pgfpathlineto{\pgfqpoint{0.000000in}{-0.027778in}}%
\pgfusepath{stroke,fill}%
}%
\begin{pgfscope}%
\pgfsys@transformshift{3.756646in}{0.550000in}%
\pgfsys@useobject{currentmarker}{}%
\end{pgfscope}%
\end{pgfscope}%
\begin{pgfscope}%
\pgfsetbuttcap%
\pgfsetroundjoin%
\definecolor{currentfill}{rgb}{0.000000,0.000000,0.000000}%
\pgfsetfillcolor{currentfill}%
\pgfsetlinewidth{0.602250pt}%
\definecolor{currentstroke}{rgb}{0.000000,0.000000,0.000000}%
\pgfsetstrokecolor{currentstroke}%
\pgfsetdash{}{0pt}%
\pgfsys@defobject{currentmarker}{\pgfqpoint{0.000000in}{-0.027778in}}{\pgfqpoint{0.000000in}{0.000000in}}{%
\pgfpathmoveto{\pgfqpoint{0.000000in}{0.000000in}}%
\pgfpathlineto{\pgfqpoint{0.000000in}{-0.027778in}}%
\pgfusepath{stroke,fill}%
}%
\begin{pgfscope}%
\pgfsys@transformshift{3.827867in}{0.550000in}%
\pgfsys@useobject{currentmarker}{}%
\end{pgfscope}%
\end{pgfscope}%
\begin{pgfscope}%
\pgfsetbuttcap%
\pgfsetroundjoin%
\definecolor{currentfill}{rgb}{0.000000,0.000000,0.000000}%
\pgfsetfillcolor{currentfill}%
\pgfsetlinewidth{0.602250pt}%
\definecolor{currentstroke}{rgb}{0.000000,0.000000,0.000000}%
\pgfsetstrokecolor{currentstroke}%
\pgfsetdash{}{0pt}%
\pgfsys@defobject{currentmarker}{\pgfqpoint{0.000000in}{-0.027778in}}{\pgfqpoint{0.000000in}{0.000000in}}{%
\pgfpathmoveto{\pgfqpoint{0.000000in}{0.000000in}}%
\pgfpathlineto{\pgfqpoint{0.000000in}{-0.027778in}}%
\pgfusepath{stroke,fill}%
}%
\begin{pgfscope}%
\pgfsys@transformshift{3.889561in}{0.550000in}%
\pgfsys@useobject{currentmarker}{}%
\end{pgfscope}%
\end{pgfscope}%
\begin{pgfscope}%
\pgfsetbuttcap%
\pgfsetroundjoin%
\definecolor{currentfill}{rgb}{0.000000,0.000000,0.000000}%
\pgfsetfillcolor{currentfill}%
\pgfsetlinewidth{0.602250pt}%
\definecolor{currentstroke}{rgb}{0.000000,0.000000,0.000000}%
\pgfsetstrokecolor{currentstroke}%
\pgfsetdash{}{0pt}%
\pgfsys@defobject{currentmarker}{\pgfqpoint{0.000000in}{-0.027778in}}{\pgfqpoint{0.000000in}{0.000000in}}{%
\pgfpathmoveto{\pgfqpoint{0.000000in}{0.000000in}}%
\pgfpathlineto{\pgfqpoint{0.000000in}{-0.027778in}}%
\pgfusepath{stroke,fill}%
}%
\begin{pgfscope}%
\pgfsys@transformshift{3.943980in}{0.550000in}%
\pgfsys@useobject{currentmarker}{}%
\end{pgfscope}%
\end{pgfscope}%
\begin{pgfscope}%
\pgfsetbuttcap%
\pgfsetroundjoin%
\definecolor{currentfill}{rgb}{0.000000,0.000000,0.000000}%
\pgfsetfillcolor{currentfill}%
\pgfsetlinewidth{0.602250pt}%
\definecolor{currentstroke}{rgb}{0.000000,0.000000,0.000000}%
\pgfsetstrokecolor{currentstroke}%
\pgfsetdash{}{0pt}%
\pgfsys@defobject{currentmarker}{\pgfqpoint{0.000000in}{-0.027778in}}{\pgfqpoint{0.000000in}{0.000000in}}{%
\pgfpathmoveto{\pgfqpoint{0.000000in}{0.000000in}}%
\pgfpathlineto{\pgfqpoint{0.000000in}{-0.027778in}}%
\pgfusepath{stroke,fill}%
}%
\begin{pgfscope}%
\pgfsys@transformshift{4.312906in}{0.550000in}%
\pgfsys@useobject{currentmarker}{}%
\end{pgfscope}%
\end{pgfscope}%
\begin{pgfscope}%
\pgfsetbuttcap%
\pgfsetroundjoin%
\definecolor{currentfill}{rgb}{0.000000,0.000000,0.000000}%
\pgfsetfillcolor{currentfill}%
\pgfsetlinewidth{0.602250pt}%
\definecolor{currentstroke}{rgb}{0.000000,0.000000,0.000000}%
\pgfsetstrokecolor{currentstroke}%
\pgfsetdash{}{0pt}%
\pgfsys@defobject{currentmarker}{\pgfqpoint{0.000000in}{-0.027778in}}{\pgfqpoint{0.000000in}{0.000000in}}{%
\pgfpathmoveto{\pgfqpoint{0.000000in}{0.000000in}}%
\pgfpathlineto{\pgfqpoint{0.000000in}{-0.027778in}}%
\pgfusepath{stroke,fill}%
}%
\begin{pgfscope}%
\pgfsys@transformshift{4.500239in}{0.550000in}%
\pgfsys@useobject{currentmarker}{}%
\end{pgfscope}%
\end{pgfscope}%
\begin{pgfscope}%
\definecolor{textcolor}{rgb}{0.000000,0.000000,0.000000}%
\pgfsetstrokecolor{textcolor}%
\pgfsetfillcolor{textcolor}%
\pgftext[x=2.562500in,y=0.249368in,,top]{\color{textcolor}\sffamily\fontsize{11.000000}{13.200000}\selectfont \(\displaystyle N_\mathrm{QMC}\)}%
\end{pgfscope}%
\begin{pgfscope}%
\pgfsetbuttcap%
\pgfsetroundjoin%
\definecolor{currentfill}{rgb}{0.000000,0.000000,0.000000}%
\pgfsetfillcolor{currentfill}%
\pgfsetlinewidth{0.803000pt}%
\definecolor{currentstroke}{rgb}{0.000000,0.000000,0.000000}%
\pgfsetstrokecolor{currentstroke}%
\pgfsetdash{}{0pt}%
\pgfsys@defobject{currentmarker}{\pgfqpoint{-0.048611in}{0.000000in}}{\pgfqpoint{0.000000in}{0.000000in}}{%
\pgfpathmoveto{\pgfqpoint{0.000000in}{0.000000in}}%
\pgfpathlineto{\pgfqpoint{-0.048611in}{0.000000in}}%
\pgfusepath{stroke,fill}%
}%
\begin{pgfscope}%
\pgfsys@transformshift{0.625000in}{1.062177in}%
\pgfsys@useobject{currentmarker}{}%
\end{pgfscope}%
\end{pgfscope}%
\begin{pgfscope}%
\definecolor{textcolor}{rgb}{0.000000,0.000000,0.000000}%
\pgfsetstrokecolor{textcolor}%
\pgfsetfillcolor{textcolor}%
\pgftext[x=0.217899in,y=1.004140in,left,base]{\color{textcolor}\sffamily\fontsize{11.000000}{13.200000}\selectfont \(\displaystyle 10^{-3}\)}%
\end{pgfscope}%
\begin{pgfscope}%
\pgfsetbuttcap%
\pgfsetroundjoin%
\definecolor{currentfill}{rgb}{0.000000,0.000000,0.000000}%
\pgfsetfillcolor{currentfill}%
\pgfsetlinewidth{0.803000pt}%
\definecolor{currentstroke}{rgb}{0.000000,0.000000,0.000000}%
\pgfsetstrokecolor{currentstroke}%
\pgfsetdash{}{0pt}%
\pgfsys@defobject{currentmarker}{\pgfqpoint{-0.048611in}{0.000000in}}{\pgfqpoint{0.000000in}{0.000000in}}{%
\pgfpathmoveto{\pgfqpoint{0.000000in}{0.000000in}}%
\pgfpathlineto{\pgfqpoint{-0.048611in}{0.000000in}}%
\pgfusepath{stroke,fill}%
}%
\begin{pgfscope}%
\pgfsys@transformshift{0.625000in}{2.452458in}%
\pgfsys@useobject{currentmarker}{}%
\end{pgfscope}%
\end{pgfscope}%
\begin{pgfscope}%
\definecolor{textcolor}{rgb}{0.000000,0.000000,0.000000}%
\pgfsetstrokecolor{textcolor}%
\pgfsetfillcolor{textcolor}%
\pgftext[x=0.217899in,y=2.394420in,left,base]{\color{textcolor}\sffamily\fontsize{11.000000}{13.200000}\selectfont \(\displaystyle 10^{-2}\)}%
\end{pgfscope}%
\begin{pgfscope}%
\pgfsetbuttcap%
\pgfsetroundjoin%
\definecolor{currentfill}{rgb}{0.000000,0.000000,0.000000}%
\pgfsetfillcolor{currentfill}%
\pgfsetlinewidth{0.803000pt}%
\definecolor{currentstroke}{rgb}{0.000000,0.000000,0.000000}%
\pgfsetstrokecolor{currentstroke}%
\pgfsetdash{}{0pt}%
\pgfsys@defobject{currentmarker}{\pgfqpoint{-0.048611in}{0.000000in}}{\pgfqpoint{0.000000in}{0.000000in}}{%
\pgfpathmoveto{\pgfqpoint{0.000000in}{0.000000in}}%
\pgfpathlineto{\pgfqpoint{-0.048611in}{0.000000in}}%
\pgfusepath{stroke,fill}%
}%
\begin{pgfscope}%
\pgfsys@transformshift{0.625000in}{3.842738in}%
\pgfsys@useobject{currentmarker}{}%
\end{pgfscope}%
\end{pgfscope}%
\begin{pgfscope}%
\definecolor{textcolor}{rgb}{0.000000,0.000000,0.000000}%
\pgfsetstrokecolor{textcolor}%
\pgfsetfillcolor{textcolor}%
\pgftext[x=0.217899in,y=3.784701in,left,base]{\color{textcolor}\sffamily\fontsize{11.000000}{13.200000}\selectfont \(\displaystyle 10^{-1}\)}%
\end{pgfscope}%
\begin{pgfscope}%
\pgfsetbuttcap%
\pgfsetroundjoin%
\definecolor{currentfill}{rgb}{0.000000,0.000000,0.000000}%
\pgfsetfillcolor{currentfill}%
\pgfsetlinewidth{0.602250pt}%
\definecolor{currentstroke}{rgb}{0.000000,0.000000,0.000000}%
\pgfsetstrokecolor{currentstroke}%
\pgfsetdash{}{0pt}%
\pgfsys@defobject{currentmarker}{\pgfqpoint{-0.027778in}{0.000000in}}{\pgfqpoint{0.000000in}{0.000000in}}{%
\pgfpathmoveto{\pgfqpoint{0.000000in}{0.000000in}}%
\pgfpathlineto{\pgfqpoint{-0.027778in}{0.000000in}}%
\pgfusepath{stroke,fill}%
}%
\begin{pgfscope}%
\pgfsys@transformshift{0.625000in}{0.643661in}%
\pgfsys@useobject{currentmarker}{}%
\end{pgfscope}%
\end{pgfscope}%
\begin{pgfscope}%
\pgfsetbuttcap%
\pgfsetroundjoin%
\definecolor{currentfill}{rgb}{0.000000,0.000000,0.000000}%
\pgfsetfillcolor{currentfill}%
\pgfsetlinewidth{0.602250pt}%
\definecolor{currentstroke}{rgb}{0.000000,0.000000,0.000000}%
\pgfsetstrokecolor{currentstroke}%
\pgfsetdash{}{0pt}%
\pgfsys@defobject{currentmarker}{\pgfqpoint{-0.027778in}{0.000000in}}{\pgfqpoint{0.000000in}{0.000000in}}{%
\pgfpathmoveto{\pgfqpoint{0.000000in}{0.000000in}}%
\pgfpathlineto{\pgfqpoint{-0.027778in}{0.000000in}}%
\pgfusepath{stroke,fill}%
}%
\begin{pgfscope}%
\pgfsys@transformshift{0.625000in}{0.753745in}%
\pgfsys@useobject{currentmarker}{}%
\end{pgfscope}%
\end{pgfscope}%
\begin{pgfscope}%
\pgfsetbuttcap%
\pgfsetroundjoin%
\definecolor{currentfill}{rgb}{0.000000,0.000000,0.000000}%
\pgfsetfillcolor{currentfill}%
\pgfsetlinewidth{0.602250pt}%
\definecolor{currentstroke}{rgb}{0.000000,0.000000,0.000000}%
\pgfsetstrokecolor{currentstroke}%
\pgfsetdash{}{0pt}%
\pgfsys@defobject{currentmarker}{\pgfqpoint{-0.027778in}{0.000000in}}{\pgfqpoint{0.000000in}{0.000000in}}{%
\pgfpathmoveto{\pgfqpoint{0.000000in}{0.000000in}}%
\pgfpathlineto{\pgfqpoint{-0.027778in}{0.000000in}}%
\pgfusepath{stroke,fill}%
}%
\begin{pgfscope}%
\pgfsys@transformshift{0.625000in}{0.846820in}%
\pgfsys@useobject{currentmarker}{}%
\end{pgfscope}%
\end{pgfscope}%
\begin{pgfscope}%
\pgfsetbuttcap%
\pgfsetroundjoin%
\definecolor{currentfill}{rgb}{0.000000,0.000000,0.000000}%
\pgfsetfillcolor{currentfill}%
\pgfsetlinewidth{0.602250pt}%
\definecolor{currentstroke}{rgb}{0.000000,0.000000,0.000000}%
\pgfsetstrokecolor{currentstroke}%
\pgfsetdash{}{0pt}%
\pgfsys@defobject{currentmarker}{\pgfqpoint{-0.027778in}{0.000000in}}{\pgfqpoint{0.000000in}{0.000000in}}{%
\pgfpathmoveto{\pgfqpoint{0.000000in}{0.000000in}}%
\pgfpathlineto{\pgfqpoint{-0.027778in}{0.000000in}}%
\pgfusepath{stroke,fill}%
}%
\begin{pgfscope}%
\pgfsys@transformshift{0.625000in}{0.927445in}%
\pgfsys@useobject{currentmarker}{}%
\end{pgfscope}%
\end{pgfscope}%
\begin{pgfscope}%
\pgfsetbuttcap%
\pgfsetroundjoin%
\definecolor{currentfill}{rgb}{0.000000,0.000000,0.000000}%
\pgfsetfillcolor{currentfill}%
\pgfsetlinewidth{0.602250pt}%
\definecolor{currentstroke}{rgb}{0.000000,0.000000,0.000000}%
\pgfsetstrokecolor{currentstroke}%
\pgfsetdash{}{0pt}%
\pgfsys@defobject{currentmarker}{\pgfqpoint{-0.027778in}{0.000000in}}{\pgfqpoint{0.000000in}{0.000000in}}{%
\pgfpathmoveto{\pgfqpoint{0.000000in}{0.000000in}}%
\pgfpathlineto{\pgfqpoint{-0.027778in}{0.000000in}}%
\pgfusepath{stroke,fill}%
}%
\begin{pgfscope}%
\pgfsys@transformshift{0.625000in}{0.998562in}%
\pgfsys@useobject{currentmarker}{}%
\end{pgfscope}%
\end{pgfscope}%
\begin{pgfscope}%
\pgfsetbuttcap%
\pgfsetroundjoin%
\definecolor{currentfill}{rgb}{0.000000,0.000000,0.000000}%
\pgfsetfillcolor{currentfill}%
\pgfsetlinewidth{0.602250pt}%
\definecolor{currentstroke}{rgb}{0.000000,0.000000,0.000000}%
\pgfsetstrokecolor{currentstroke}%
\pgfsetdash{}{0pt}%
\pgfsys@defobject{currentmarker}{\pgfqpoint{-0.027778in}{0.000000in}}{\pgfqpoint{0.000000in}{0.000000in}}{%
\pgfpathmoveto{\pgfqpoint{0.000000in}{0.000000in}}%
\pgfpathlineto{\pgfqpoint{-0.027778in}{0.000000in}}%
\pgfusepath{stroke,fill}%
}%
\begin{pgfscope}%
\pgfsys@transformshift{0.625000in}{1.480693in}%
\pgfsys@useobject{currentmarker}{}%
\end{pgfscope}%
\end{pgfscope}%
\begin{pgfscope}%
\pgfsetbuttcap%
\pgfsetroundjoin%
\definecolor{currentfill}{rgb}{0.000000,0.000000,0.000000}%
\pgfsetfillcolor{currentfill}%
\pgfsetlinewidth{0.602250pt}%
\definecolor{currentstroke}{rgb}{0.000000,0.000000,0.000000}%
\pgfsetstrokecolor{currentstroke}%
\pgfsetdash{}{0pt}%
\pgfsys@defobject{currentmarker}{\pgfqpoint{-0.027778in}{0.000000in}}{\pgfqpoint{0.000000in}{0.000000in}}{%
\pgfpathmoveto{\pgfqpoint{0.000000in}{0.000000in}}%
\pgfpathlineto{\pgfqpoint{-0.027778in}{0.000000in}}%
\pgfusepath{stroke,fill}%
}%
\begin{pgfscope}%
\pgfsys@transformshift{0.625000in}{1.725510in}%
\pgfsys@useobject{currentmarker}{}%
\end{pgfscope}%
\end{pgfscope}%
\begin{pgfscope}%
\pgfsetbuttcap%
\pgfsetroundjoin%
\definecolor{currentfill}{rgb}{0.000000,0.000000,0.000000}%
\pgfsetfillcolor{currentfill}%
\pgfsetlinewidth{0.602250pt}%
\definecolor{currentstroke}{rgb}{0.000000,0.000000,0.000000}%
\pgfsetstrokecolor{currentstroke}%
\pgfsetdash{}{0pt}%
\pgfsys@defobject{currentmarker}{\pgfqpoint{-0.027778in}{0.000000in}}{\pgfqpoint{0.000000in}{0.000000in}}{%
\pgfpathmoveto{\pgfqpoint{0.000000in}{0.000000in}}%
\pgfpathlineto{\pgfqpoint{-0.027778in}{0.000000in}}%
\pgfusepath{stroke,fill}%
}%
\begin{pgfscope}%
\pgfsys@transformshift{0.625000in}{1.899210in}%
\pgfsys@useobject{currentmarker}{}%
\end{pgfscope}%
\end{pgfscope}%
\begin{pgfscope}%
\pgfsetbuttcap%
\pgfsetroundjoin%
\definecolor{currentfill}{rgb}{0.000000,0.000000,0.000000}%
\pgfsetfillcolor{currentfill}%
\pgfsetlinewidth{0.602250pt}%
\definecolor{currentstroke}{rgb}{0.000000,0.000000,0.000000}%
\pgfsetstrokecolor{currentstroke}%
\pgfsetdash{}{0pt}%
\pgfsys@defobject{currentmarker}{\pgfqpoint{-0.027778in}{0.000000in}}{\pgfqpoint{0.000000in}{0.000000in}}{%
\pgfpathmoveto{\pgfqpoint{0.000000in}{0.000000in}}%
\pgfpathlineto{\pgfqpoint{-0.027778in}{0.000000in}}%
\pgfusepath{stroke,fill}%
}%
\begin{pgfscope}%
\pgfsys@transformshift{0.625000in}{2.033942in}%
\pgfsys@useobject{currentmarker}{}%
\end{pgfscope}%
\end{pgfscope}%
\begin{pgfscope}%
\pgfsetbuttcap%
\pgfsetroundjoin%
\definecolor{currentfill}{rgb}{0.000000,0.000000,0.000000}%
\pgfsetfillcolor{currentfill}%
\pgfsetlinewidth{0.602250pt}%
\definecolor{currentstroke}{rgb}{0.000000,0.000000,0.000000}%
\pgfsetstrokecolor{currentstroke}%
\pgfsetdash{}{0pt}%
\pgfsys@defobject{currentmarker}{\pgfqpoint{-0.027778in}{0.000000in}}{\pgfqpoint{0.000000in}{0.000000in}}{%
\pgfpathmoveto{\pgfqpoint{0.000000in}{0.000000in}}%
\pgfpathlineto{\pgfqpoint{-0.027778in}{0.000000in}}%
\pgfusepath{stroke,fill}%
}%
\begin{pgfscope}%
\pgfsys@transformshift{0.625000in}{2.144026in}%
\pgfsys@useobject{currentmarker}{}%
\end{pgfscope}%
\end{pgfscope}%
\begin{pgfscope}%
\pgfsetbuttcap%
\pgfsetroundjoin%
\definecolor{currentfill}{rgb}{0.000000,0.000000,0.000000}%
\pgfsetfillcolor{currentfill}%
\pgfsetlinewidth{0.602250pt}%
\definecolor{currentstroke}{rgb}{0.000000,0.000000,0.000000}%
\pgfsetstrokecolor{currentstroke}%
\pgfsetdash{}{0pt}%
\pgfsys@defobject{currentmarker}{\pgfqpoint{-0.027778in}{0.000000in}}{\pgfqpoint{0.000000in}{0.000000in}}{%
\pgfpathmoveto{\pgfqpoint{0.000000in}{0.000000in}}%
\pgfpathlineto{\pgfqpoint{-0.027778in}{0.000000in}}%
\pgfusepath{stroke,fill}%
}%
\begin{pgfscope}%
\pgfsys@transformshift{0.625000in}{2.237101in}%
\pgfsys@useobject{currentmarker}{}%
\end{pgfscope}%
\end{pgfscope}%
\begin{pgfscope}%
\pgfsetbuttcap%
\pgfsetroundjoin%
\definecolor{currentfill}{rgb}{0.000000,0.000000,0.000000}%
\pgfsetfillcolor{currentfill}%
\pgfsetlinewidth{0.602250pt}%
\definecolor{currentstroke}{rgb}{0.000000,0.000000,0.000000}%
\pgfsetstrokecolor{currentstroke}%
\pgfsetdash{}{0pt}%
\pgfsys@defobject{currentmarker}{\pgfqpoint{-0.027778in}{0.000000in}}{\pgfqpoint{0.000000in}{0.000000in}}{%
\pgfpathmoveto{\pgfqpoint{0.000000in}{0.000000in}}%
\pgfpathlineto{\pgfqpoint{-0.027778in}{0.000000in}}%
\pgfusepath{stroke,fill}%
}%
\begin{pgfscope}%
\pgfsys@transformshift{0.625000in}{2.317726in}%
\pgfsys@useobject{currentmarker}{}%
\end{pgfscope}%
\end{pgfscope}%
\begin{pgfscope}%
\pgfsetbuttcap%
\pgfsetroundjoin%
\definecolor{currentfill}{rgb}{0.000000,0.000000,0.000000}%
\pgfsetfillcolor{currentfill}%
\pgfsetlinewidth{0.602250pt}%
\definecolor{currentstroke}{rgb}{0.000000,0.000000,0.000000}%
\pgfsetstrokecolor{currentstroke}%
\pgfsetdash{}{0pt}%
\pgfsys@defobject{currentmarker}{\pgfqpoint{-0.027778in}{0.000000in}}{\pgfqpoint{0.000000in}{0.000000in}}{%
\pgfpathmoveto{\pgfqpoint{0.000000in}{0.000000in}}%
\pgfpathlineto{\pgfqpoint{-0.027778in}{0.000000in}}%
\pgfusepath{stroke,fill}%
}%
\begin{pgfscope}%
\pgfsys@transformshift{0.625000in}{2.388842in}%
\pgfsys@useobject{currentmarker}{}%
\end{pgfscope}%
\end{pgfscope}%
\begin{pgfscope}%
\pgfsetbuttcap%
\pgfsetroundjoin%
\definecolor{currentfill}{rgb}{0.000000,0.000000,0.000000}%
\pgfsetfillcolor{currentfill}%
\pgfsetlinewidth{0.602250pt}%
\definecolor{currentstroke}{rgb}{0.000000,0.000000,0.000000}%
\pgfsetstrokecolor{currentstroke}%
\pgfsetdash{}{0pt}%
\pgfsys@defobject{currentmarker}{\pgfqpoint{-0.027778in}{0.000000in}}{\pgfqpoint{0.000000in}{0.000000in}}{%
\pgfpathmoveto{\pgfqpoint{0.000000in}{0.000000in}}%
\pgfpathlineto{\pgfqpoint{-0.027778in}{0.000000in}}%
\pgfusepath{stroke,fill}%
}%
\begin{pgfscope}%
\pgfsys@transformshift{0.625000in}{2.870974in}%
\pgfsys@useobject{currentmarker}{}%
\end{pgfscope}%
\end{pgfscope}%
\begin{pgfscope}%
\pgfsetbuttcap%
\pgfsetroundjoin%
\definecolor{currentfill}{rgb}{0.000000,0.000000,0.000000}%
\pgfsetfillcolor{currentfill}%
\pgfsetlinewidth{0.602250pt}%
\definecolor{currentstroke}{rgb}{0.000000,0.000000,0.000000}%
\pgfsetstrokecolor{currentstroke}%
\pgfsetdash{}{0pt}%
\pgfsys@defobject{currentmarker}{\pgfqpoint{-0.027778in}{0.000000in}}{\pgfqpoint{0.000000in}{0.000000in}}{%
\pgfpathmoveto{\pgfqpoint{0.000000in}{0.000000in}}%
\pgfpathlineto{\pgfqpoint{-0.027778in}{0.000000in}}%
\pgfusepath{stroke,fill}%
}%
\begin{pgfscope}%
\pgfsys@transformshift{0.625000in}{3.115790in}%
\pgfsys@useobject{currentmarker}{}%
\end{pgfscope}%
\end{pgfscope}%
\begin{pgfscope}%
\pgfsetbuttcap%
\pgfsetroundjoin%
\definecolor{currentfill}{rgb}{0.000000,0.000000,0.000000}%
\pgfsetfillcolor{currentfill}%
\pgfsetlinewidth{0.602250pt}%
\definecolor{currentstroke}{rgb}{0.000000,0.000000,0.000000}%
\pgfsetstrokecolor{currentstroke}%
\pgfsetdash{}{0pt}%
\pgfsys@defobject{currentmarker}{\pgfqpoint{-0.027778in}{0.000000in}}{\pgfqpoint{0.000000in}{0.000000in}}{%
\pgfpathmoveto{\pgfqpoint{0.000000in}{0.000000in}}%
\pgfpathlineto{\pgfqpoint{-0.027778in}{0.000000in}}%
\pgfusepath{stroke,fill}%
}%
\begin{pgfscope}%
\pgfsys@transformshift{0.625000in}{3.289490in}%
\pgfsys@useobject{currentmarker}{}%
\end{pgfscope}%
\end{pgfscope}%
\begin{pgfscope}%
\pgfsetbuttcap%
\pgfsetroundjoin%
\definecolor{currentfill}{rgb}{0.000000,0.000000,0.000000}%
\pgfsetfillcolor{currentfill}%
\pgfsetlinewidth{0.602250pt}%
\definecolor{currentstroke}{rgb}{0.000000,0.000000,0.000000}%
\pgfsetstrokecolor{currentstroke}%
\pgfsetdash{}{0pt}%
\pgfsys@defobject{currentmarker}{\pgfqpoint{-0.027778in}{0.000000in}}{\pgfqpoint{0.000000in}{0.000000in}}{%
\pgfpathmoveto{\pgfqpoint{0.000000in}{0.000000in}}%
\pgfpathlineto{\pgfqpoint{-0.027778in}{0.000000in}}%
\pgfusepath{stroke,fill}%
}%
\begin{pgfscope}%
\pgfsys@transformshift{0.625000in}{3.424222in}%
\pgfsys@useobject{currentmarker}{}%
\end{pgfscope}%
\end{pgfscope}%
\begin{pgfscope}%
\pgfsetbuttcap%
\pgfsetroundjoin%
\definecolor{currentfill}{rgb}{0.000000,0.000000,0.000000}%
\pgfsetfillcolor{currentfill}%
\pgfsetlinewidth{0.602250pt}%
\definecolor{currentstroke}{rgb}{0.000000,0.000000,0.000000}%
\pgfsetstrokecolor{currentstroke}%
\pgfsetdash{}{0pt}%
\pgfsys@defobject{currentmarker}{\pgfqpoint{-0.027778in}{0.000000in}}{\pgfqpoint{0.000000in}{0.000000in}}{%
\pgfpathmoveto{\pgfqpoint{0.000000in}{0.000000in}}%
\pgfpathlineto{\pgfqpoint{-0.027778in}{0.000000in}}%
\pgfusepath{stroke,fill}%
}%
\begin{pgfscope}%
\pgfsys@transformshift{0.625000in}{3.534306in}%
\pgfsys@useobject{currentmarker}{}%
\end{pgfscope}%
\end{pgfscope}%
\begin{pgfscope}%
\pgfsetbuttcap%
\pgfsetroundjoin%
\definecolor{currentfill}{rgb}{0.000000,0.000000,0.000000}%
\pgfsetfillcolor{currentfill}%
\pgfsetlinewidth{0.602250pt}%
\definecolor{currentstroke}{rgb}{0.000000,0.000000,0.000000}%
\pgfsetstrokecolor{currentstroke}%
\pgfsetdash{}{0pt}%
\pgfsys@defobject{currentmarker}{\pgfqpoint{-0.027778in}{0.000000in}}{\pgfqpoint{0.000000in}{0.000000in}}{%
\pgfpathmoveto{\pgfqpoint{0.000000in}{0.000000in}}%
\pgfpathlineto{\pgfqpoint{-0.027778in}{0.000000in}}%
\pgfusepath{stroke,fill}%
}%
\begin{pgfscope}%
\pgfsys@transformshift{0.625000in}{3.627381in}%
\pgfsys@useobject{currentmarker}{}%
\end{pgfscope}%
\end{pgfscope}%
\begin{pgfscope}%
\pgfsetbuttcap%
\pgfsetroundjoin%
\definecolor{currentfill}{rgb}{0.000000,0.000000,0.000000}%
\pgfsetfillcolor{currentfill}%
\pgfsetlinewidth{0.602250pt}%
\definecolor{currentstroke}{rgb}{0.000000,0.000000,0.000000}%
\pgfsetstrokecolor{currentstroke}%
\pgfsetdash{}{0pt}%
\pgfsys@defobject{currentmarker}{\pgfqpoint{-0.027778in}{0.000000in}}{\pgfqpoint{0.000000in}{0.000000in}}{%
\pgfpathmoveto{\pgfqpoint{0.000000in}{0.000000in}}%
\pgfpathlineto{\pgfqpoint{-0.027778in}{0.000000in}}%
\pgfusepath{stroke,fill}%
}%
\begin{pgfscope}%
\pgfsys@transformshift{0.625000in}{3.708006in}%
\pgfsys@useobject{currentmarker}{}%
\end{pgfscope}%
\end{pgfscope}%
\begin{pgfscope}%
\pgfsetbuttcap%
\pgfsetroundjoin%
\definecolor{currentfill}{rgb}{0.000000,0.000000,0.000000}%
\pgfsetfillcolor{currentfill}%
\pgfsetlinewidth{0.602250pt}%
\definecolor{currentstroke}{rgb}{0.000000,0.000000,0.000000}%
\pgfsetstrokecolor{currentstroke}%
\pgfsetdash{}{0pt}%
\pgfsys@defobject{currentmarker}{\pgfqpoint{-0.027778in}{0.000000in}}{\pgfqpoint{0.000000in}{0.000000in}}{%
\pgfpathmoveto{\pgfqpoint{0.000000in}{0.000000in}}%
\pgfpathlineto{\pgfqpoint{-0.027778in}{0.000000in}}%
\pgfusepath{stroke,fill}%
}%
\begin{pgfscope}%
\pgfsys@transformshift{0.625000in}{3.779122in}%
\pgfsys@useobject{currentmarker}{}%
\end{pgfscope}%
\end{pgfscope}%
\begin{pgfscope}%
\pgfsetbuttcap%
\pgfsetroundjoin%
\definecolor{currentfill}{rgb}{0.000000,0.000000,0.000000}%
\pgfsetfillcolor{currentfill}%
\pgfsetlinewidth{0.602250pt}%
\definecolor{currentstroke}{rgb}{0.000000,0.000000,0.000000}%
\pgfsetstrokecolor{currentstroke}%
\pgfsetdash{}{0pt}%
\pgfsys@defobject{currentmarker}{\pgfqpoint{-0.027778in}{0.000000in}}{\pgfqpoint{0.000000in}{0.000000in}}{%
\pgfpathmoveto{\pgfqpoint{0.000000in}{0.000000in}}%
\pgfpathlineto{\pgfqpoint{-0.027778in}{0.000000in}}%
\pgfusepath{stroke,fill}%
}%
\begin{pgfscope}%
\pgfsys@transformshift{0.625000in}{4.261254in}%
\pgfsys@useobject{currentmarker}{}%
\end{pgfscope}%
\end{pgfscope}%
\begin{pgfscope}%
\definecolor{textcolor}{rgb}{0.000000,0.000000,0.000000}%
\pgfsetstrokecolor{textcolor}%
\pgfsetfillcolor{textcolor}%
\pgftext[x=0.162343in,y=2.475000in,,bottom,rotate=90.000000]{\color{textcolor}\sffamily\fontsize{11.000000}{13.200000}\selectfont \textrm{QMC Error}}%
\end{pgfscope}%
\begin{pgfscope}%
\pgfpathrectangle{\pgfqpoint{0.625000in}{0.550000in}}{\pgfqpoint{3.875000in}{3.850000in}}%
\pgfusepath{clip}%
\pgfsetbuttcap%
\pgfsetroundjoin%
\definecolor{currentfill}{rgb}{0.000000,0.000000,0.000000}%
\pgfsetfillcolor{currentfill}%
\pgfsetlinewidth{1.003750pt}%
\definecolor{currentstroke}{rgb}{0.000000,0.000000,0.000000}%
\pgfsetstrokecolor{currentstroke}%
\pgfsetdash{}{0pt}%
\pgfsys@defobject{currentmarker}{\pgfqpoint{-0.041667in}{-0.041667in}}{\pgfqpoint{0.041667in}{0.041667in}}{%
\pgfpathmoveto{\pgfqpoint{0.000000in}{-0.041667in}}%
\pgfpathcurveto{\pgfqpoint{0.011050in}{-0.041667in}}{\pgfqpoint{0.021649in}{-0.037276in}}{\pgfqpoint{0.029463in}{-0.029463in}}%
\pgfpathcurveto{\pgfqpoint{0.037276in}{-0.021649in}}{\pgfqpoint{0.041667in}{-0.011050in}}{\pgfqpoint{0.041667in}{0.000000in}}%
\pgfpathcurveto{\pgfqpoint{0.041667in}{0.011050in}}{\pgfqpoint{0.037276in}{0.021649in}}{\pgfqpoint{0.029463in}{0.029463in}}%
\pgfpathcurveto{\pgfqpoint{0.021649in}{0.037276in}}{\pgfqpoint{0.011050in}{0.041667in}}{\pgfqpoint{0.000000in}{0.041667in}}%
\pgfpathcurveto{\pgfqpoint{-0.011050in}{0.041667in}}{\pgfqpoint{-0.021649in}{0.037276in}}{\pgfqpoint{-0.029463in}{0.029463in}}%
\pgfpathcurveto{\pgfqpoint{-0.037276in}{0.021649in}}{\pgfqpoint{-0.041667in}{0.011050in}}{\pgfqpoint{-0.041667in}{0.000000in}}%
\pgfpathcurveto{\pgfqpoint{-0.041667in}{-0.011050in}}{\pgfqpoint{-0.037276in}{-0.021649in}}{\pgfqpoint{-0.029463in}{-0.029463in}}%
\pgfpathcurveto{\pgfqpoint{-0.021649in}{-0.037276in}}{\pgfqpoint{-0.011050in}{-0.041667in}}{\pgfqpoint{0.000000in}{-0.041667in}}%
\pgfpathclose%
\pgfusepath{stroke,fill}%
}%
\begin{pgfscope}%
\pgfsys@transformshift{0.801136in}{4.132403in}%
\pgfsys@useobject{currentmarker}{}%
\end{pgfscope}%
\begin{pgfscope}%
\pgfsys@transformshift{1.121384in}{3.870260in}%
\pgfsys@useobject{currentmarker}{}%
\end{pgfscope}%
\begin{pgfscope}%
\pgfsys@transformshift{1.441632in}{3.394780in}%
\pgfsys@useobject{currentmarker}{}%
\end{pgfscope}%
\begin{pgfscope}%
\pgfsys@transformshift{1.761880in}{3.232976in}%
\pgfsys@useobject{currentmarker}{}%
\end{pgfscope}%
\begin{pgfscope}%
\pgfsys@transformshift{2.082128in}{2.940571in}%
\pgfsys@useobject{currentmarker}{}%
\end{pgfscope}%
\begin{pgfscope}%
\pgfsys@transformshift{2.402376in}{2.785753in}%
\pgfsys@useobject{currentmarker}{}%
\end{pgfscope}%
\begin{pgfscope}%
\pgfsys@transformshift{2.722624in}{2.750488in}%
\pgfsys@useobject{currentmarker}{}%
\end{pgfscope}%
\begin{pgfscope}%
\pgfsys@transformshift{3.042872in}{2.726799in}%
\pgfsys@useobject{currentmarker}{}%
\end{pgfscope}%
\begin{pgfscope}%
\pgfsys@transformshift{3.363120in}{1.665585in}%
\pgfsys@useobject{currentmarker}{}%
\end{pgfscope}%
\begin{pgfscope}%
\pgfsys@transformshift{3.683368in}{1.173059in}%
\pgfsys@useobject{currentmarker}{}%
\end{pgfscope}%
\begin{pgfscope}%
\pgfsys@transformshift{4.003616in}{0.871988in}%
\pgfsys@useobject{currentmarker}{}%
\end{pgfscope}%
\begin{pgfscope}%
\pgfsys@transformshift{4.323864in}{0.725000in}%
\pgfsys@useobject{currentmarker}{}%
\end{pgfscope}%
\end{pgfscope}%
\begin{pgfscope}%
\pgfpathrectangle{\pgfqpoint{0.625000in}{0.550000in}}{\pgfqpoint{3.875000in}{3.850000in}}%
\pgfusepath{clip}%
\pgfsetbuttcap%
\pgfsetroundjoin%
\pgfsetlinewidth{1.505625pt}%
\definecolor{currentstroke}{rgb}{0.000000,0.000000,0.000000}%
\pgfsetstrokecolor{currentstroke}%
\pgfsetdash{{5.550000pt}{2.400000pt}}{0.000000pt}%
\pgfpathmoveto{\pgfqpoint{0.801136in}{4.225000in}}%
\pgfpathlineto{\pgfqpoint{4.323864in}{0.819944in}}%
\pgfusepath{stroke}%
\end{pgfscope}%
\begin{pgfscope}%
\pgfsetrectcap%
\pgfsetmiterjoin%
\pgfsetlinewidth{0.803000pt}%
\definecolor{currentstroke}{rgb}{0.000000,0.000000,0.000000}%
\pgfsetstrokecolor{currentstroke}%
\pgfsetdash{}{0pt}%
\pgfpathmoveto{\pgfqpoint{0.625000in}{0.550000in}}%
\pgfpathlineto{\pgfqpoint{0.625000in}{4.400000in}}%
\pgfusepath{stroke}%
\end{pgfscope}%
\begin{pgfscope}%
\pgfsetrectcap%
\pgfsetmiterjoin%
\pgfsetlinewidth{0.000000pt}%
\definecolor{currentstroke}{rgb}{0.000000,0.000000,0.000000}%
\pgfsetstrokecolor{currentstroke}%
\pgfsetstrokeopacity{0.000000}%
\pgfsetdash{}{0pt}%
\pgfpathmoveto{\pgfqpoint{4.500000in}{0.550000in}}%
\pgfpathlineto{\pgfqpoint{4.500000in}{4.400000in}}%
\pgfusepath{}%
\end{pgfscope}%
\begin{pgfscope}%
\pgfsetrectcap%
\pgfsetmiterjoin%
\pgfsetlinewidth{0.803000pt}%
\definecolor{currentstroke}{rgb}{0.000000,0.000000,0.000000}%
\pgfsetstrokecolor{currentstroke}%
\pgfsetdash{}{0pt}%
\pgfpathmoveto{\pgfqpoint{0.625000in}{0.550000in}}%
\pgfpathlineto{\pgfqpoint{4.500000in}{0.550000in}}%
\pgfusepath{stroke}%
\end{pgfscope}%
\begin{pgfscope}%
\pgfsetrectcap%
\pgfsetmiterjoin%
\pgfsetlinewidth{0.000000pt}%
\definecolor{currentstroke}{rgb}{0.000000,0.000000,0.000000}%
\pgfsetstrokecolor{currentstroke}%
\pgfsetstrokeopacity{0.000000}%
\pgfsetdash{}{0pt}%
\pgfpathmoveto{\pgfqpoint{0.625000in}{4.400000in}}%
\pgfpathlineto{\pgfqpoint{4.500000in}{4.400000in}}%
\pgfusepath{}%
\end{pgfscope}%
\begin{pgfscope}%
\pgfsetbuttcap%
\pgfsetmiterjoin%
\definecolor{currentfill}{rgb}{1.000000,1.000000,1.000000}%
\pgfsetfillcolor{currentfill}%
\pgfsetfillopacity{0.800000}%
\pgfsetlinewidth{1.003750pt}%
\definecolor{currentstroke}{rgb}{0.800000,0.800000,0.800000}%
\pgfsetstrokecolor{currentstroke}%
\pgfsetstrokeopacity{0.800000}%
\pgfsetdash{}{0pt}%
\pgfpathmoveto{\pgfqpoint{2.916233in}{3.999488in}}%
\pgfpathlineto{\pgfqpoint{4.393056in}{3.999488in}}%
\pgfpathquadraticcurveto{\pgfqpoint{4.423611in}{3.999488in}}{\pgfqpoint{4.423611in}{4.030043in}}%
\pgfpathlineto{\pgfqpoint{4.423611in}{4.293056in}}%
\pgfpathquadraticcurveto{\pgfqpoint{4.423611in}{4.323611in}}{\pgfqpoint{4.393056in}{4.323611in}}%
\pgfpathlineto{\pgfqpoint{2.916233in}{4.323611in}}%
\pgfpathquadraticcurveto{\pgfqpoint{2.885678in}{4.323611in}}{\pgfqpoint{2.885678in}{4.293056in}}%
\pgfpathlineto{\pgfqpoint{2.885678in}{4.030043in}}%
\pgfpathquadraticcurveto{\pgfqpoint{2.885678in}{3.999488in}}{\pgfqpoint{2.916233in}{3.999488in}}%
\pgfpathclose%
\pgfusepath{stroke,fill}%
\end{pgfscope}%
\begin{pgfscope}%
\pgfsetbuttcap%
\pgfsetroundjoin%
\pgfsetlinewidth{1.505625pt}%
\definecolor{currentstroke}{rgb}{0.000000,0.000000,0.000000}%
\pgfsetstrokecolor{currentstroke}%
\pgfsetdash{{5.550000pt}{2.400000pt}}{0.000000pt}%
\pgfpathmoveto{\pgfqpoint{2.946789in}{4.180408in}}%
\pgfpathlineto{\pgfqpoint{3.252344in}{4.180408in}}%
\pgfusepath{stroke}%
\end{pgfscope}%
\begin{pgfscope}%
\definecolor{textcolor}{rgb}{0.000000,0.000000,0.000000}%
\pgfsetstrokecolor{textcolor}%
\pgfsetfillcolor{textcolor}%
\pgftext[x=3.374567in,y=4.126936in,left,base]{\color{textcolor}\sffamily\fontsize{11.000000}{13.200000}\selectfont \(\displaystyle 0.1883N_{\mathrm{QMC}}^{-0.7396}\)}%
\end{pgfscope}%
\end{pgfpicture}%
\makeatother%
\endgroup%

    \caption[Quasi-Monte-Carlo error, for $Q=u((1,1))$ and $k=50$.]{Quasi-Monte-Carlo error with increasing number of Quasi-Monte-Carlo points, for $Q=u((1,1))$ and $k=50$. \label{fig:qmctop_right50}}
  \end{figure}
\begin{figure}[h]
  \centering
%% Creator: Matplotlib, PGF backend
%%
%% To include the figure in your LaTeX document, write
%%   \input{<filename>.pgf}
%%
%% Make sure the required packages are loaded in your preamble
%%   \usepackage{pgf}
%%
%% Figures using additional raster images can only be included by \input if
%% they are in the same directory as the main LaTeX file. For loading figures
%% from other directories you can use the `import` package
%%   \usepackage{import}
%% and then include the figures with
%%   \import{<path to file>}{<filename>.pgf}
%%
%% Matplotlib used the following preamble
%%   \usepackage{fontspec}
%%   \setmainfont{DejaVuSerif.ttf}[Path=/home/owen/progs/firedrake-complex/firedrake/lib/python3.5/site-packages/matplotlib/mpl-data/fonts/ttf/]
%%   \setsansfont{DejaVuSans.ttf}[Path=/home/owen/progs/firedrake-complex/firedrake/lib/python3.5/site-packages/matplotlib/mpl-data/fonts/ttf/]
%%   \setmonofont{DejaVuSansMono.ttf}[Path=/home/owen/progs/firedrake-complex/firedrake/lib/python3.5/site-packages/matplotlib/mpl-data/fonts/ttf/]
%%
\begingroup%
\makeatletter%
\begin{pgfpicture}%
\pgfpathrectangle{\pgfpointorigin}{\pgfqpoint{3.000000in}{3.000000in}}%
\pgfusepath{use as bounding box, clip}%
\begin{pgfscope}%
\pgfsetbuttcap%
\pgfsetmiterjoin%
\definecolor{currentfill}{rgb}{1.000000,1.000000,1.000000}%
\pgfsetfillcolor{currentfill}%
\pgfsetlinewidth{0.000000pt}%
\definecolor{currentstroke}{rgb}{1.000000,1.000000,1.000000}%
\pgfsetstrokecolor{currentstroke}%
\pgfsetdash{}{0pt}%
\pgfpathmoveto{\pgfqpoint{0.000000in}{0.000000in}}%
\pgfpathlineto{\pgfqpoint{3.000000in}{0.000000in}}%
\pgfpathlineto{\pgfqpoint{3.000000in}{3.000000in}}%
\pgfpathlineto{\pgfqpoint{0.000000in}{3.000000in}}%
\pgfpathclose%
\pgfusepath{fill}%
\end{pgfscope}%
\begin{pgfscope}%
\pgfsetbuttcap%
\pgfsetmiterjoin%
\definecolor{currentfill}{rgb}{1.000000,1.000000,1.000000}%
\pgfsetfillcolor{currentfill}%
\pgfsetlinewidth{0.000000pt}%
\definecolor{currentstroke}{rgb}{0.000000,0.000000,0.000000}%
\pgfsetstrokecolor{currentstroke}%
\pgfsetstrokeopacity{0.000000}%
\pgfsetdash{}{0pt}%
\pgfpathmoveto{\pgfqpoint{0.375000in}{0.330000in}}%
\pgfpathlineto{\pgfqpoint{2.700000in}{0.330000in}}%
\pgfpathlineto{\pgfqpoint{2.700000in}{2.640000in}}%
\pgfpathlineto{\pgfqpoint{0.375000in}{2.640000in}}%
\pgfpathclose%
\pgfusepath{fill}%
\end{pgfscope}%
\begin{pgfscope}%
\pgfsetbuttcap%
\pgfsetroundjoin%
\definecolor{currentfill}{rgb}{0.000000,0.000000,0.000000}%
\pgfsetfillcolor{currentfill}%
\pgfsetlinewidth{0.803000pt}%
\definecolor{currentstroke}{rgb}{0.000000,0.000000,0.000000}%
\pgfsetstrokecolor{currentstroke}%
\pgfsetdash{}{0pt}%
\pgfsys@defobject{currentmarker}{\pgfqpoint{0.000000in}{-0.048611in}}{\pgfqpoint{0.000000in}{0.000000in}}{%
\pgfpathmoveto{\pgfqpoint{0.000000in}{0.000000in}}%
\pgfpathlineto{\pgfqpoint{0.000000in}{-0.048611in}}%
\pgfusepath{stroke,fill}%
}%
\begin{pgfscope}%
\pgfsys@transformshift{0.480682in}{0.330000in}%
\pgfsys@useobject{currentmarker}{}%
\end{pgfscope}%
\end{pgfscope}%
\begin{pgfscope}%
\definecolor{textcolor}{rgb}{0.000000,0.000000,0.000000}%
\pgfsetstrokecolor{textcolor}%
\pgfsetfillcolor{textcolor}%
\pgftext[x=0.480682in,y=0.232778in,,top]{\color{textcolor}\sffamily\fontsize{10.000000}{12.000000}\selectfont \(\displaystyle {10^{0}}\)}%
\end{pgfscope}%
\begin{pgfscope}%
\pgfsetbuttcap%
\pgfsetroundjoin%
\definecolor{currentfill}{rgb}{0.000000,0.000000,0.000000}%
\pgfsetfillcolor{currentfill}%
\pgfsetlinewidth{0.803000pt}%
\definecolor{currentstroke}{rgb}{0.000000,0.000000,0.000000}%
\pgfsetstrokecolor{currentstroke}%
\pgfsetdash{}{0pt}%
\pgfsys@defobject{currentmarker}{\pgfqpoint{0.000000in}{-0.048611in}}{\pgfqpoint{0.000000in}{0.000000in}}{%
\pgfpathmoveto{\pgfqpoint{0.000000in}{0.000000in}}%
\pgfpathlineto{\pgfqpoint{0.000000in}{-0.048611in}}%
\pgfusepath{stroke,fill}%
}%
\begin{pgfscope}%
\pgfsys@transformshift{1.118986in}{0.330000in}%
\pgfsys@useobject{currentmarker}{}%
\end{pgfscope}%
\end{pgfscope}%
\begin{pgfscope}%
\definecolor{textcolor}{rgb}{0.000000,0.000000,0.000000}%
\pgfsetstrokecolor{textcolor}%
\pgfsetfillcolor{textcolor}%
\pgftext[x=1.118986in,y=0.232778in,,top]{\color{textcolor}\sffamily\fontsize{10.000000}{12.000000}\selectfont \(\displaystyle {10^{1}}\)}%
\end{pgfscope}%
\begin{pgfscope}%
\pgfsetbuttcap%
\pgfsetroundjoin%
\definecolor{currentfill}{rgb}{0.000000,0.000000,0.000000}%
\pgfsetfillcolor{currentfill}%
\pgfsetlinewidth{0.803000pt}%
\definecolor{currentstroke}{rgb}{0.000000,0.000000,0.000000}%
\pgfsetstrokecolor{currentstroke}%
\pgfsetdash{}{0pt}%
\pgfsys@defobject{currentmarker}{\pgfqpoint{0.000000in}{-0.048611in}}{\pgfqpoint{0.000000in}{0.000000in}}{%
\pgfpathmoveto{\pgfqpoint{0.000000in}{0.000000in}}%
\pgfpathlineto{\pgfqpoint{0.000000in}{-0.048611in}}%
\pgfusepath{stroke,fill}%
}%
\begin{pgfscope}%
\pgfsys@transformshift{1.757291in}{0.330000in}%
\pgfsys@useobject{currentmarker}{}%
\end{pgfscope}%
\end{pgfscope}%
\begin{pgfscope}%
\definecolor{textcolor}{rgb}{0.000000,0.000000,0.000000}%
\pgfsetstrokecolor{textcolor}%
\pgfsetfillcolor{textcolor}%
\pgftext[x=1.757291in,y=0.232778in,,top]{\color{textcolor}\sffamily\fontsize{10.000000}{12.000000}\selectfont \(\displaystyle {10^{2}}\)}%
\end{pgfscope}%
\begin{pgfscope}%
\pgfsetbuttcap%
\pgfsetroundjoin%
\definecolor{currentfill}{rgb}{0.000000,0.000000,0.000000}%
\pgfsetfillcolor{currentfill}%
\pgfsetlinewidth{0.803000pt}%
\definecolor{currentstroke}{rgb}{0.000000,0.000000,0.000000}%
\pgfsetstrokecolor{currentstroke}%
\pgfsetdash{}{0pt}%
\pgfsys@defobject{currentmarker}{\pgfqpoint{0.000000in}{-0.048611in}}{\pgfqpoint{0.000000in}{0.000000in}}{%
\pgfpathmoveto{\pgfqpoint{0.000000in}{0.000000in}}%
\pgfpathlineto{\pgfqpoint{0.000000in}{-0.048611in}}%
\pgfusepath{stroke,fill}%
}%
\begin{pgfscope}%
\pgfsys@transformshift{2.395595in}{0.330000in}%
\pgfsys@useobject{currentmarker}{}%
\end{pgfscope}%
\end{pgfscope}%
\begin{pgfscope}%
\definecolor{textcolor}{rgb}{0.000000,0.000000,0.000000}%
\pgfsetstrokecolor{textcolor}%
\pgfsetfillcolor{textcolor}%
\pgftext[x=2.395595in,y=0.232778in,,top]{\color{textcolor}\sffamily\fontsize{10.000000}{12.000000}\selectfont \(\displaystyle {10^{3}}\)}%
\end{pgfscope}%
\begin{pgfscope}%
\pgfsetbuttcap%
\pgfsetroundjoin%
\definecolor{currentfill}{rgb}{0.000000,0.000000,0.000000}%
\pgfsetfillcolor{currentfill}%
\pgfsetlinewidth{0.602250pt}%
\definecolor{currentstroke}{rgb}{0.000000,0.000000,0.000000}%
\pgfsetstrokecolor{currentstroke}%
\pgfsetdash{}{0pt}%
\pgfsys@defobject{currentmarker}{\pgfqpoint{0.000000in}{-0.027778in}}{\pgfqpoint{0.000000in}{0.000000in}}{%
\pgfpathmoveto{\pgfqpoint{0.000000in}{0.000000in}}%
\pgfpathlineto{\pgfqpoint{0.000000in}{-0.027778in}}%
\pgfusepath{stroke,fill}%
}%
\begin{pgfscope}%
\pgfsys@transformshift{0.381807in}{0.330000in}%
\pgfsys@useobject{currentmarker}{}%
\end{pgfscope}%
\end{pgfscope}%
\begin{pgfscope}%
\pgfsetbuttcap%
\pgfsetroundjoin%
\definecolor{currentfill}{rgb}{0.000000,0.000000,0.000000}%
\pgfsetfillcolor{currentfill}%
\pgfsetlinewidth{0.602250pt}%
\definecolor{currentstroke}{rgb}{0.000000,0.000000,0.000000}%
\pgfsetstrokecolor{currentstroke}%
\pgfsetdash{}{0pt}%
\pgfsys@defobject{currentmarker}{\pgfqpoint{0.000000in}{-0.027778in}}{\pgfqpoint{0.000000in}{0.000000in}}{%
\pgfpathmoveto{\pgfqpoint{0.000000in}{0.000000in}}%
\pgfpathlineto{\pgfqpoint{0.000000in}{-0.027778in}}%
\pgfusepath{stroke,fill}%
}%
\begin{pgfscope}%
\pgfsys@transformshift{0.418824in}{0.330000in}%
\pgfsys@useobject{currentmarker}{}%
\end{pgfscope}%
\end{pgfscope}%
\begin{pgfscope}%
\pgfsetbuttcap%
\pgfsetroundjoin%
\definecolor{currentfill}{rgb}{0.000000,0.000000,0.000000}%
\pgfsetfillcolor{currentfill}%
\pgfsetlinewidth{0.602250pt}%
\definecolor{currentstroke}{rgb}{0.000000,0.000000,0.000000}%
\pgfsetstrokecolor{currentstroke}%
\pgfsetdash{}{0pt}%
\pgfsys@defobject{currentmarker}{\pgfqpoint{0.000000in}{-0.027778in}}{\pgfqpoint{0.000000in}{0.000000in}}{%
\pgfpathmoveto{\pgfqpoint{0.000000in}{0.000000in}}%
\pgfpathlineto{\pgfqpoint{0.000000in}{-0.027778in}}%
\pgfusepath{stroke,fill}%
}%
\begin{pgfscope}%
\pgfsys@transformshift{0.451475in}{0.330000in}%
\pgfsys@useobject{currentmarker}{}%
\end{pgfscope}%
\end{pgfscope}%
\begin{pgfscope}%
\pgfsetbuttcap%
\pgfsetroundjoin%
\definecolor{currentfill}{rgb}{0.000000,0.000000,0.000000}%
\pgfsetfillcolor{currentfill}%
\pgfsetlinewidth{0.602250pt}%
\definecolor{currentstroke}{rgb}{0.000000,0.000000,0.000000}%
\pgfsetstrokecolor{currentstroke}%
\pgfsetdash{}{0pt}%
\pgfsys@defobject{currentmarker}{\pgfqpoint{0.000000in}{-0.027778in}}{\pgfqpoint{0.000000in}{0.000000in}}{%
\pgfpathmoveto{\pgfqpoint{0.000000in}{0.000000in}}%
\pgfpathlineto{\pgfqpoint{0.000000in}{-0.027778in}}%
\pgfusepath{stroke,fill}%
}%
\begin{pgfscope}%
\pgfsys@transformshift{0.672831in}{0.330000in}%
\pgfsys@useobject{currentmarker}{}%
\end{pgfscope}%
\end{pgfscope}%
\begin{pgfscope}%
\pgfsetbuttcap%
\pgfsetroundjoin%
\definecolor{currentfill}{rgb}{0.000000,0.000000,0.000000}%
\pgfsetfillcolor{currentfill}%
\pgfsetlinewidth{0.602250pt}%
\definecolor{currentstroke}{rgb}{0.000000,0.000000,0.000000}%
\pgfsetstrokecolor{currentstroke}%
\pgfsetdash{}{0pt}%
\pgfsys@defobject{currentmarker}{\pgfqpoint{0.000000in}{-0.027778in}}{\pgfqpoint{0.000000in}{0.000000in}}{%
\pgfpathmoveto{\pgfqpoint{0.000000in}{0.000000in}}%
\pgfpathlineto{\pgfqpoint{0.000000in}{-0.027778in}}%
\pgfusepath{stroke,fill}%
}%
\begin{pgfscope}%
\pgfsys@transformshift{0.785230in}{0.330000in}%
\pgfsys@useobject{currentmarker}{}%
\end{pgfscope}%
\end{pgfscope}%
\begin{pgfscope}%
\pgfsetbuttcap%
\pgfsetroundjoin%
\definecolor{currentfill}{rgb}{0.000000,0.000000,0.000000}%
\pgfsetfillcolor{currentfill}%
\pgfsetlinewidth{0.602250pt}%
\definecolor{currentstroke}{rgb}{0.000000,0.000000,0.000000}%
\pgfsetstrokecolor{currentstroke}%
\pgfsetdash{}{0pt}%
\pgfsys@defobject{currentmarker}{\pgfqpoint{0.000000in}{-0.027778in}}{\pgfqpoint{0.000000in}{0.000000in}}{%
\pgfpathmoveto{\pgfqpoint{0.000000in}{0.000000in}}%
\pgfpathlineto{\pgfqpoint{0.000000in}{-0.027778in}}%
\pgfusepath{stroke,fill}%
}%
\begin{pgfscope}%
\pgfsys@transformshift{0.864979in}{0.330000in}%
\pgfsys@useobject{currentmarker}{}%
\end{pgfscope}%
\end{pgfscope}%
\begin{pgfscope}%
\pgfsetbuttcap%
\pgfsetroundjoin%
\definecolor{currentfill}{rgb}{0.000000,0.000000,0.000000}%
\pgfsetfillcolor{currentfill}%
\pgfsetlinewidth{0.602250pt}%
\definecolor{currentstroke}{rgb}{0.000000,0.000000,0.000000}%
\pgfsetstrokecolor{currentstroke}%
\pgfsetdash{}{0pt}%
\pgfsys@defobject{currentmarker}{\pgfqpoint{0.000000in}{-0.027778in}}{\pgfqpoint{0.000000in}{0.000000in}}{%
\pgfpathmoveto{\pgfqpoint{0.000000in}{0.000000in}}%
\pgfpathlineto{\pgfqpoint{0.000000in}{-0.027778in}}%
\pgfusepath{stroke,fill}%
}%
\begin{pgfscope}%
\pgfsys@transformshift{0.926837in}{0.330000in}%
\pgfsys@useobject{currentmarker}{}%
\end{pgfscope}%
\end{pgfscope}%
\begin{pgfscope}%
\pgfsetbuttcap%
\pgfsetroundjoin%
\definecolor{currentfill}{rgb}{0.000000,0.000000,0.000000}%
\pgfsetfillcolor{currentfill}%
\pgfsetlinewidth{0.602250pt}%
\definecolor{currentstroke}{rgb}{0.000000,0.000000,0.000000}%
\pgfsetstrokecolor{currentstroke}%
\pgfsetdash{}{0pt}%
\pgfsys@defobject{currentmarker}{\pgfqpoint{0.000000in}{-0.027778in}}{\pgfqpoint{0.000000in}{0.000000in}}{%
\pgfpathmoveto{\pgfqpoint{0.000000in}{0.000000in}}%
\pgfpathlineto{\pgfqpoint{0.000000in}{-0.027778in}}%
\pgfusepath{stroke,fill}%
}%
\begin{pgfscope}%
\pgfsys@transformshift{0.977379in}{0.330000in}%
\pgfsys@useobject{currentmarker}{}%
\end{pgfscope}%
\end{pgfscope}%
\begin{pgfscope}%
\pgfsetbuttcap%
\pgfsetroundjoin%
\definecolor{currentfill}{rgb}{0.000000,0.000000,0.000000}%
\pgfsetfillcolor{currentfill}%
\pgfsetlinewidth{0.602250pt}%
\definecolor{currentstroke}{rgb}{0.000000,0.000000,0.000000}%
\pgfsetstrokecolor{currentstroke}%
\pgfsetdash{}{0pt}%
\pgfsys@defobject{currentmarker}{\pgfqpoint{0.000000in}{-0.027778in}}{\pgfqpoint{0.000000in}{0.000000in}}{%
\pgfpathmoveto{\pgfqpoint{0.000000in}{0.000000in}}%
\pgfpathlineto{\pgfqpoint{0.000000in}{-0.027778in}}%
\pgfusepath{stroke,fill}%
}%
\begin{pgfscope}%
\pgfsys@transformshift{1.020112in}{0.330000in}%
\pgfsys@useobject{currentmarker}{}%
\end{pgfscope}%
\end{pgfscope}%
\begin{pgfscope}%
\pgfsetbuttcap%
\pgfsetroundjoin%
\definecolor{currentfill}{rgb}{0.000000,0.000000,0.000000}%
\pgfsetfillcolor{currentfill}%
\pgfsetlinewidth{0.602250pt}%
\definecolor{currentstroke}{rgb}{0.000000,0.000000,0.000000}%
\pgfsetstrokecolor{currentstroke}%
\pgfsetdash{}{0pt}%
\pgfsys@defobject{currentmarker}{\pgfqpoint{0.000000in}{-0.027778in}}{\pgfqpoint{0.000000in}{0.000000in}}{%
\pgfpathmoveto{\pgfqpoint{0.000000in}{0.000000in}}%
\pgfpathlineto{\pgfqpoint{0.000000in}{-0.027778in}}%
\pgfusepath{stroke,fill}%
}%
\begin{pgfscope}%
\pgfsys@transformshift{1.057128in}{0.330000in}%
\pgfsys@useobject{currentmarker}{}%
\end{pgfscope}%
\end{pgfscope}%
\begin{pgfscope}%
\pgfsetbuttcap%
\pgfsetroundjoin%
\definecolor{currentfill}{rgb}{0.000000,0.000000,0.000000}%
\pgfsetfillcolor{currentfill}%
\pgfsetlinewidth{0.602250pt}%
\definecolor{currentstroke}{rgb}{0.000000,0.000000,0.000000}%
\pgfsetstrokecolor{currentstroke}%
\pgfsetdash{}{0pt}%
\pgfsys@defobject{currentmarker}{\pgfqpoint{0.000000in}{-0.027778in}}{\pgfqpoint{0.000000in}{0.000000in}}{%
\pgfpathmoveto{\pgfqpoint{0.000000in}{0.000000in}}%
\pgfpathlineto{\pgfqpoint{0.000000in}{-0.027778in}}%
\pgfusepath{stroke,fill}%
}%
\begin{pgfscope}%
\pgfsys@transformshift{1.089779in}{0.330000in}%
\pgfsys@useobject{currentmarker}{}%
\end{pgfscope}%
\end{pgfscope}%
\begin{pgfscope}%
\pgfsetbuttcap%
\pgfsetroundjoin%
\definecolor{currentfill}{rgb}{0.000000,0.000000,0.000000}%
\pgfsetfillcolor{currentfill}%
\pgfsetlinewidth{0.602250pt}%
\definecolor{currentstroke}{rgb}{0.000000,0.000000,0.000000}%
\pgfsetstrokecolor{currentstroke}%
\pgfsetdash{}{0pt}%
\pgfsys@defobject{currentmarker}{\pgfqpoint{0.000000in}{-0.027778in}}{\pgfqpoint{0.000000in}{0.000000in}}{%
\pgfpathmoveto{\pgfqpoint{0.000000in}{0.000000in}}%
\pgfpathlineto{\pgfqpoint{0.000000in}{-0.027778in}}%
\pgfusepath{stroke,fill}%
}%
\begin{pgfscope}%
\pgfsys@transformshift{1.311135in}{0.330000in}%
\pgfsys@useobject{currentmarker}{}%
\end{pgfscope}%
\end{pgfscope}%
\begin{pgfscope}%
\pgfsetbuttcap%
\pgfsetroundjoin%
\definecolor{currentfill}{rgb}{0.000000,0.000000,0.000000}%
\pgfsetfillcolor{currentfill}%
\pgfsetlinewidth{0.602250pt}%
\definecolor{currentstroke}{rgb}{0.000000,0.000000,0.000000}%
\pgfsetstrokecolor{currentstroke}%
\pgfsetdash{}{0pt}%
\pgfsys@defobject{currentmarker}{\pgfqpoint{0.000000in}{-0.027778in}}{\pgfqpoint{0.000000in}{0.000000in}}{%
\pgfpathmoveto{\pgfqpoint{0.000000in}{0.000000in}}%
\pgfpathlineto{\pgfqpoint{0.000000in}{-0.027778in}}%
\pgfusepath{stroke,fill}%
}%
\begin{pgfscope}%
\pgfsys@transformshift{1.423535in}{0.330000in}%
\pgfsys@useobject{currentmarker}{}%
\end{pgfscope}%
\end{pgfscope}%
\begin{pgfscope}%
\pgfsetbuttcap%
\pgfsetroundjoin%
\definecolor{currentfill}{rgb}{0.000000,0.000000,0.000000}%
\pgfsetfillcolor{currentfill}%
\pgfsetlinewidth{0.602250pt}%
\definecolor{currentstroke}{rgb}{0.000000,0.000000,0.000000}%
\pgfsetstrokecolor{currentstroke}%
\pgfsetdash{}{0pt}%
\pgfsys@defobject{currentmarker}{\pgfqpoint{0.000000in}{-0.027778in}}{\pgfqpoint{0.000000in}{0.000000in}}{%
\pgfpathmoveto{\pgfqpoint{0.000000in}{0.000000in}}%
\pgfpathlineto{\pgfqpoint{0.000000in}{-0.027778in}}%
\pgfusepath{stroke,fill}%
}%
\begin{pgfscope}%
\pgfsys@transformshift{1.503284in}{0.330000in}%
\pgfsys@useobject{currentmarker}{}%
\end{pgfscope}%
\end{pgfscope}%
\begin{pgfscope}%
\pgfsetbuttcap%
\pgfsetroundjoin%
\definecolor{currentfill}{rgb}{0.000000,0.000000,0.000000}%
\pgfsetfillcolor{currentfill}%
\pgfsetlinewidth{0.602250pt}%
\definecolor{currentstroke}{rgb}{0.000000,0.000000,0.000000}%
\pgfsetstrokecolor{currentstroke}%
\pgfsetdash{}{0pt}%
\pgfsys@defobject{currentmarker}{\pgfqpoint{0.000000in}{-0.027778in}}{\pgfqpoint{0.000000in}{0.000000in}}{%
\pgfpathmoveto{\pgfqpoint{0.000000in}{0.000000in}}%
\pgfpathlineto{\pgfqpoint{0.000000in}{-0.027778in}}%
\pgfusepath{stroke,fill}%
}%
\begin{pgfscope}%
\pgfsys@transformshift{1.565142in}{0.330000in}%
\pgfsys@useobject{currentmarker}{}%
\end{pgfscope}%
\end{pgfscope}%
\begin{pgfscope}%
\pgfsetbuttcap%
\pgfsetroundjoin%
\definecolor{currentfill}{rgb}{0.000000,0.000000,0.000000}%
\pgfsetfillcolor{currentfill}%
\pgfsetlinewidth{0.602250pt}%
\definecolor{currentstroke}{rgb}{0.000000,0.000000,0.000000}%
\pgfsetstrokecolor{currentstroke}%
\pgfsetdash{}{0pt}%
\pgfsys@defobject{currentmarker}{\pgfqpoint{0.000000in}{-0.027778in}}{\pgfqpoint{0.000000in}{0.000000in}}{%
\pgfpathmoveto{\pgfqpoint{0.000000in}{0.000000in}}%
\pgfpathlineto{\pgfqpoint{0.000000in}{-0.027778in}}%
\pgfusepath{stroke,fill}%
}%
\begin{pgfscope}%
\pgfsys@transformshift{1.615684in}{0.330000in}%
\pgfsys@useobject{currentmarker}{}%
\end{pgfscope}%
\end{pgfscope}%
\begin{pgfscope}%
\pgfsetbuttcap%
\pgfsetroundjoin%
\definecolor{currentfill}{rgb}{0.000000,0.000000,0.000000}%
\pgfsetfillcolor{currentfill}%
\pgfsetlinewidth{0.602250pt}%
\definecolor{currentstroke}{rgb}{0.000000,0.000000,0.000000}%
\pgfsetstrokecolor{currentstroke}%
\pgfsetdash{}{0pt}%
\pgfsys@defobject{currentmarker}{\pgfqpoint{0.000000in}{-0.027778in}}{\pgfqpoint{0.000000in}{0.000000in}}{%
\pgfpathmoveto{\pgfqpoint{0.000000in}{0.000000in}}%
\pgfpathlineto{\pgfqpoint{0.000000in}{-0.027778in}}%
\pgfusepath{stroke,fill}%
}%
\begin{pgfscope}%
\pgfsys@transformshift{1.658416in}{0.330000in}%
\pgfsys@useobject{currentmarker}{}%
\end{pgfscope}%
\end{pgfscope}%
\begin{pgfscope}%
\pgfsetbuttcap%
\pgfsetroundjoin%
\definecolor{currentfill}{rgb}{0.000000,0.000000,0.000000}%
\pgfsetfillcolor{currentfill}%
\pgfsetlinewidth{0.602250pt}%
\definecolor{currentstroke}{rgb}{0.000000,0.000000,0.000000}%
\pgfsetstrokecolor{currentstroke}%
\pgfsetdash{}{0pt}%
\pgfsys@defobject{currentmarker}{\pgfqpoint{0.000000in}{-0.027778in}}{\pgfqpoint{0.000000in}{0.000000in}}{%
\pgfpathmoveto{\pgfqpoint{0.000000in}{0.000000in}}%
\pgfpathlineto{\pgfqpoint{0.000000in}{-0.027778in}}%
\pgfusepath{stroke,fill}%
}%
\begin{pgfscope}%
\pgfsys@transformshift{1.695432in}{0.330000in}%
\pgfsys@useobject{currentmarker}{}%
\end{pgfscope}%
\end{pgfscope}%
\begin{pgfscope}%
\pgfsetbuttcap%
\pgfsetroundjoin%
\definecolor{currentfill}{rgb}{0.000000,0.000000,0.000000}%
\pgfsetfillcolor{currentfill}%
\pgfsetlinewidth{0.602250pt}%
\definecolor{currentstroke}{rgb}{0.000000,0.000000,0.000000}%
\pgfsetstrokecolor{currentstroke}%
\pgfsetdash{}{0pt}%
\pgfsys@defobject{currentmarker}{\pgfqpoint{0.000000in}{-0.027778in}}{\pgfqpoint{0.000000in}{0.000000in}}{%
\pgfpathmoveto{\pgfqpoint{0.000000in}{0.000000in}}%
\pgfpathlineto{\pgfqpoint{0.000000in}{-0.027778in}}%
\pgfusepath{stroke,fill}%
}%
\begin{pgfscope}%
\pgfsys@transformshift{1.728083in}{0.330000in}%
\pgfsys@useobject{currentmarker}{}%
\end{pgfscope}%
\end{pgfscope}%
\begin{pgfscope}%
\pgfsetbuttcap%
\pgfsetroundjoin%
\definecolor{currentfill}{rgb}{0.000000,0.000000,0.000000}%
\pgfsetfillcolor{currentfill}%
\pgfsetlinewidth{0.602250pt}%
\definecolor{currentstroke}{rgb}{0.000000,0.000000,0.000000}%
\pgfsetstrokecolor{currentstroke}%
\pgfsetdash{}{0pt}%
\pgfsys@defobject{currentmarker}{\pgfqpoint{0.000000in}{-0.027778in}}{\pgfqpoint{0.000000in}{0.000000in}}{%
\pgfpathmoveto{\pgfqpoint{0.000000in}{0.000000in}}%
\pgfpathlineto{\pgfqpoint{0.000000in}{-0.027778in}}%
\pgfusepath{stroke,fill}%
}%
\begin{pgfscope}%
\pgfsys@transformshift{1.949439in}{0.330000in}%
\pgfsys@useobject{currentmarker}{}%
\end{pgfscope}%
\end{pgfscope}%
\begin{pgfscope}%
\pgfsetbuttcap%
\pgfsetroundjoin%
\definecolor{currentfill}{rgb}{0.000000,0.000000,0.000000}%
\pgfsetfillcolor{currentfill}%
\pgfsetlinewidth{0.602250pt}%
\definecolor{currentstroke}{rgb}{0.000000,0.000000,0.000000}%
\pgfsetstrokecolor{currentstroke}%
\pgfsetdash{}{0pt}%
\pgfsys@defobject{currentmarker}{\pgfqpoint{0.000000in}{-0.027778in}}{\pgfqpoint{0.000000in}{0.000000in}}{%
\pgfpathmoveto{\pgfqpoint{0.000000in}{0.000000in}}%
\pgfpathlineto{\pgfqpoint{0.000000in}{-0.027778in}}%
\pgfusepath{stroke,fill}%
}%
\begin{pgfscope}%
\pgfsys@transformshift{2.061839in}{0.330000in}%
\pgfsys@useobject{currentmarker}{}%
\end{pgfscope}%
\end{pgfscope}%
\begin{pgfscope}%
\pgfsetbuttcap%
\pgfsetroundjoin%
\definecolor{currentfill}{rgb}{0.000000,0.000000,0.000000}%
\pgfsetfillcolor{currentfill}%
\pgfsetlinewidth{0.602250pt}%
\definecolor{currentstroke}{rgb}{0.000000,0.000000,0.000000}%
\pgfsetstrokecolor{currentstroke}%
\pgfsetdash{}{0pt}%
\pgfsys@defobject{currentmarker}{\pgfqpoint{0.000000in}{-0.027778in}}{\pgfqpoint{0.000000in}{0.000000in}}{%
\pgfpathmoveto{\pgfqpoint{0.000000in}{0.000000in}}%
\pgfpathlineto{\pgfqpoint{0.000000in}{-0.027778in}}%
\pgfusepath{stroke,fill}%
}%
\begin{pgfscope}%
\pgfsys@transformshift{2.141588in}{0.330000in}%
\pgfsys@useobject{currentmarker}{}%
\end{pgfscope}%
\end{pgfscope}%
\begin{pgfscope}%
\pgfsetbuttcap%
\pgfsetroundjoin%
\definecolor{currentfill}{rgb}{0.000000,0.000000,0.000000}%
\pgfsetfillcolor{currentfill}%
\pgfsetlinewidth{0.602250pt}%
\definecolor{currentstroke}{rgb}{0.000000,0.000000,0.000000}%
\pgfsetstrokecolor{currentstroke}%
\pgfsetdash{}{0pt}%
\pgfsys@defobject{currentmarker}{\pgfqpoint{0.000000in}{-0.027778in}}{\pgfqpoint{0.000000in}{0.000000in}}{%
\pgfpathmoveto{\pgfqpoint{0.000000in}{0.000000in}}%
\pgfpathlineto{\pgfqpoint{0.000000in}{-0.027778in}}%
\pgfusepath{stroke,fill}%
}%
\begin{pgfscope}%
\pgfsys@transformshift{2.203446in}{0.330000in}%
\pgfsys@useobject{currentmarker}{}%
\end{pgfscope}%
\end{pgfscope}%
\begin{pgfscope}%
\pgfsetbuttcap%
\pgfsetroundjoin%
\definecolor{currentfill}{rgb}{0.000000,0.000000,0.000000}%
\pgfsetfillcolor{currentfill}%
\pgfsetlinewidth{0.602250pt}%
\definecolor{currentstroke}{rgb}{0.000000,0.000000,0.000000}%
\pgfsetstrokecolor{currentstroke}%
\pgfsetdash{}{0pt}%
\pgfsys@defobject{currentmarker}{\pgfqpoint{0.000000in}{-0.027778in}}{\pgfqpoint{0.000000in}{0.000000in}}{%
\pgfpathmoveto{\pgfqpoint{0.000000in}{0.000000in}}%
\pgfpathlineto{\pgfqpoint{0.000000in}{-0.027778in}}%
\pgfusepath{stroke,fill}%
}%
\begin{pgfscope}%
\pgfsys@transformshift{2.253988in}{0.330000in}%
\pgfsys@useobject{currentmarker}{}%
\end{pgfscope}%
\end{pgfscope}%
\begin{pgfscope}%
\pgfsetbuttcap%
\pgfsetroundjoin%
\definecolor{currentfill}{rgb}{0.000000,0.000000,0.000000}%
\pgfsetfillcolor{currentfill}%
\pgfsetlinewidth{0.602250pt}%
\definecolor{currentstroke}{rgb}{0.000000,0.000000,0.000000}%
\pgfsetstrokecolor{currentstroke}%
\pgfsetdash{}{0pt}%
\pgfsys@defobject{currentmarker}{\pgfqpoint{0.000000in}{-0.027778in}}{\pgfqpoint{0.000000in}{0.000000in}}{%
\pgfpathmoveto{\pgfqpoint{0.000000in}{0.000000in}}%
\pgfpathlineto{\pgfqpoint{0.000000in}{-0.027778in}}%
\pgfusepath{stroke,fill}%
}%
\begin{pgfscope}%
\pgfsys@transformshift{2.296720in}{0.330000in}%
\pgfsys@useobject{currentmarker}{}%
\end{pgfscope}%
\end{pgfscope}%
\begin{pgfscope}%
\pgfsetbuttcap%
\pgfsetroundjoin%
\definecolor{currentfill}{rgb}{0.000000,0.000000,0.000000}%
\pgfsetfillcolor{currentfill}%
\pgfsetlinewidth{0.602250pt}%
\definecolor{currentstroke}{rgb}{0.000000,0.000000,0.000000}%
\pgfsetstrokecolor{currentstroke}%
\pgfsetdash{}{0pt}%
\pgfsys@defobject{currentmarker}{\pgfqpoint{0.000000in}{-0.027778in}}{\pgfqpoint{0.000000in}{0.000000in}}{%
\pgfpathmoveto{\pgfqpoint{0.000000in}{0.000000in}}%
\pgfpathlineto{\pgfqpoint{0.000000in}{-0.027778in}}%
\pgfusepath{stroke,fill}%
}%
\begin{pgfscope}%
\pgfsys@transformshift{2.333737in}{0.330000in}%
\pgfsys@useobject{currentmarker}{}%
\end{pgfscope}%
\end{pgfscope}%
\begin{pgfscope}%
\pgfsetbuttcap%
\pgfsetroundjoin%
\definecolor{currentfill}{rgb}{0.000000,0.000000,0.000000}%
\pgfsetfillcolor{currentfill}%
\pgfsetlinewidth{0.602250pt}%
\definecolor{currentstroke}{rgb}{0.000000,0.000000,0.000000}%
\pgfsetstrokecolor{currentstroke}%
\pgfsetdash{}{0pt}%
\pgfsys@defobject{currentmarker}{\pgfqpoint{0.000000in}{-0.027778in}}{\pgfqpoint{0.000000in}{0.000000in}}{%
\pgfpathmoveto{\pgfqpoint{0.000000in}{0.000000in}}%
\pgfpathlineto{\pgfqpoint{0.000000in}{-0.027778in}}%
\pgfusepath{stroke,fill}%
}%
\begin{pgfscope}%
\pgfsys@transformshift{2.366388in}{0.330000in}%
\pgfsys@useobject{currentmarker}{}%
\end{pgfscope}%
\end{pgfscope}%
\begin{pgfscope}%
\pgfsetbuttcap%
\pgfsetroundjoin%
\definecolor{currentfill}{rgb}{0.000000,0.000000,0.000000}%
\pgfsetfillcolor{currentfill}%
\pgfsetlinewidth{0.602250pt}%
\definecolor{currentstroke}{rgb}{0.000000,0.000000,0.000000}%
\pgfsetstrokecolor{currentstroke}%
\pgfsetdash{}{0pt}%
\pgfsys@defobject{currentmarker}{\pgfqpoint{0.000000in}{-0.027778in}}{\pgfqpoint{0.000000in}{0.000000in}}{%
\pgfpathmoveto{\pgfqpoint{0.000000in}{0.000000in}}%
\pgfpathlineto{\pgfqpoint{0.000000in}{-0.027778in}}%
\pgfusepath{stroke,fill}%
}%
\begin{pgfscope}%
\pgfsys@transformshift{2.587744in}{0.330000in}%
\pgfsys@useobject{currentmarker}{}%
\end{pgfscope}%
\end{pgfscope}%
\begin{pgfscope}%
\pgfsetbuttcap%
\pgfsetroundjoin%
\definecolor{currentfill}{rgb}{0.000000,0.000000,0.000000}%
\pgfsetfillcolor{currentfill}%
\pgfsetlinewidth{0.602250pt}%
\definecolor{currentstroke}{rgb}{0.000000,0.000000,0.000000}%
\pgfsetstrokecolor{currentstroke}%
\pgfsetdash{}{0pt}%
\pgfsys@defobject{currentmarker}{\pgfqpoint{0.000000in}{-0.027778in}}{\pgfqpoint{0.000000in}{0.000000in}}{%
\pgfpathmoveto{\pgfqpoint{0.000000in}{0.000000in}}%
\pgfpathlineto{\pgfqpoint{0.000000in}{-0.027778in}}%
\pgfusepath{stroke,fill}%
}%
\begin{pgfscope}%
\pgfsys@transformshift{2.700143in}{0.330000in}%
\pgfsys@useobject{currentmarker}{}%
\end{pgfscope}%
\end{pgfscope}%
\begin{pgfscope}%
\definecolor{textcolor}{rgb}{0.000000,0.000000,0.000000}%
\pgfsetstrokecolor{textcolor}%
\pgfsetfillcolor{textcolor}%
\pgftext[x=1.537500in,y=0.042809in,,top]{\color{textcolor}\sffamily\fontsize{10.000000}{12.000000}\selectfont \(\displaystyle N_\mathrm{QMC}\)}%
\end{pgfscope}%
\begin{pgfscope}%
\pgfsetbuttcap%
\pgfsetroundjoin%
\definecolor{currentfill}{rgb}{0.000000,0.000000,0.000000}%
\pgfsetfillcolor{currentfill}%
\pgfsetlinewidth{0.803000pt}%
\definecolor{currentstroke}{rgb}{0.000000,0.000000,0.000000}%
\pgfsetstrokecolor{currentstroke}%
\pgfsetdash{}{0pt}%
\pgfsys@defobject{currentmarker}{\pgfqpoint{-0.048611in}{0.000000in}}{\pgfqpoint{0.000000in}{0.000000in}}{%
\pgfpathmoveto{\pgfqpoint{0.000000in}{0.000000in}}%
\pgfpathlineto{\pgfqpoint{-0.048611in}{0.000000in}}%
\pgfusepath{stroke,fill}%
}%
\begin{pgfscope}%
\pgfsys@transformshift{0.375000in}{0.863537in}%
\pgfsys@useobject{currentmarker}{}%
\end{pgfscope}%
\end{pgfscope}%
\begin{pgfscope}%
\definecolor{textcolor}{rgb}{0.000000,0.000000,0.000000}%
\pgfsetstrokecolor{textcolor}%
\pgfsetfillcolor{textcolor}%
\pgftext[x=-0.010225in,y=0.810775in,left,base]{\color{textcolor}\sffamily\fontsize{10.000000}{12.000000}\selectfont \(\displaystyle {10^{-3}}\)}%
\end{pgfscope}%
\begin{pgfscope}%
\pgfsetbuttcap%
\pgfsetroundjoin%
\definecolor{currentfill}{rgb}{0.000000,0.000000,0.000000}%
\pgfsetfillcolor{currentfill}%
\pgfsetlinewidth{0.803000pt}%
\definecolor{currentstroke}{rgb}{0.000000,0.000000,0.000000}%
\pgfsetstrokecolor{currentstroke}%
\pgfsetdash{}{0pt}%
\pgfsys@defobject{currentmarker}{\pgfqpoint{-0.048611in}{0.000000in}}{\pgfqpoint{0.000000in}{0.000000in}}{%
\pgfpathmoveto{\pgfqpoint{0.000000in}{0.000000in}}%
\pgfpathlineto{\pgfqpoint{-0.048611in}{0.000000in}}%
\pgfusepath{stroke,fill}%
}%
\begin{pgfscope}%
\pgfsys@transformshift{0.375000in}{1.763350in}%
\pgfsys@useobject{currentmarker}{}%
\end{pgfscope}%
\end{pgfscope}%
\begin{pgfscope}%
\definecolor{textcolor}{rgb}{0.000000,0.000000,0.000000}%
\pgfsetstrokecolor{textcolor}%
\pgfsetfillcolor{textcolor}%
\pgftext[x=-0.010225in,y=1.710588in,left,base]{\color{textcolor}\sffamily\fontsize{10.000000}{12.000000}\selectfont \(\displaystyle {10^{-2}}\)}%
\end{pgfscope}%
\begin{pgfscope}%
\pgfsetbuttcap%
\pgfsetroundjoin%
\definecolor{currentfill}{rgb}{0.000000,0.000000,0.000000}%
\pgfsetfillcolor{currentfill}%
\pgfsetlinewidth{0.602250pt}%
\definecolor{currentstroke}{rgb}{0.000000,0.000000,0.000000}%
\pgfsetstrokecolor{currentstroke}%
\pgfsetdash{}{0pt}%
\pgfsys@defobject{currentmarker}{\pgfqpoint{-0.027778in}{0.000000in}}{\pgfqpoint{0.000000in}{0.000000in}}{%
\pgfpathmoveto{\pgfqpoint{0.000000in}{0.000000in}}%
\pgfpathlineto{\pgfqpoint{-0.027778in}{0.000000in}}%
\pgfusepath{stroke,fill}%
}%
\begin{pgfscope}%
\pgfsys@transformshift{0.375000in}{0.393044in}%
\pgfsys@useobject{currentmarker}{}%
\end{pgfscope}%
\end{pgfscope}%
\begin{pgfscope}%
\pgfsetbuttcap%
\pgfsetroundjoin%
\definecolor{currentfill}{rgb}{0.000000,0.000000,0.000000}%
\pgfsetfillcolor{currentfill}%
\pgfsetlinewidth{0.602250pt}%
\definecolor{currentstroke}{rgb}{0.000000,0.000000,0.000000}%
\pgfsetstrokecolor{currentstroke}%
\pgfsetdash{}{0pt}%
\pgfsys@defobject{currentmarker}{\pgfqpoint{-0.027778in}{0.000000in}}{\pgfqpoint{0.000000in}{0.000000in}}{%
\pgfpathmoveto{\pgfqpoint{0.000000in}{0.000000in}}%
\pgfpathlineto{\pgfqpoint{-0.027778in}{0.000000in}}%
\pgfusepath{stroke,fill}%
}%
\begin{pgfscope}%
\pgfsys@transformshift{0.375000in}{0.505465in}%
\pgfsys@useobject{currentmarker}{}%
\end{pgfscope}%
\end{pgfscope}%
\begin{pgfscope}%
\pgfsetbuttcap%
\pgfsetroundjoin%
\definecolor{currentfill}{rgb}{0.000000,0.000000,0.000000}%
\pgfsetfillcolor{currentfill}%
\pgfsetlinewidth{0.602250pt}%
\definecolor{currentstroke}{rgb}{0.000000,0.000000,0.000000}%
\pgfsetstrokecolor{currentstroke}%
\pgfsetdash{}{0pt}%
\pgfsys@defobject{currentmarker}{\pgfqpoint{-0.027778in}{0.000000in}}{\pgfqpoint{0.000000in}{0.000000in}}{%
\pgfpathmoveto{\pgfqpoint{0.000000in}{0.000000in}}%
\pgfpathlineto{\pgfqpoint{-0.027778in}{0.000000in}}%
\pgfusepath{stroke,fill}%
}%
\begin{pgfscope}%
\pgfsys@transformshift{0.375000in}{0.592666in}%
\pgfsys@useobject{currentmarker}{}%
\end{pgfscope}%
\end{pgfscope}%
\begin{pgfscope}%
\pgfsetbuttcap%
\pgfsetroundjoin%
\definecolor{currentfill}{rgb}{0.000000,0.000000,0.000000}%
\pgfsetfillcolor{currentfill}%
\pgfsetlinewidth{0.602250pt}%
\definecolor{currentstroke}{rgb}{0.000000,0.000000,0.000000}%
\pgfsetstrokecolor{currentstroke}%
\pgfsetdash{}{0pt}%
\pgfsys@defobject{currentmarker}{\pgfqpoint{-0.027778in}{0.000000in}}{\pgfqpoint{0.000000in}{0.000000in}}{%
\pgfpathmoveto{\pgfqpoint{0.000000in}{0.000000in}}%
\pgfpathlineto{\pgfqpoint{-0.027778in}{0.000000in}}%
\pgfusepath{stroke,fill}%
}%
\begin{pgfscope}%
\pgfsys@transformshift{0.375000in}{0.663914in}%
\pgfsys@useobject{currentmarker}{}%
\end{pgfscope}%
\end{pgfscope}%
\begin{pgfscope}%
\pgfsetbuttcap%
\pgfsetroundjoin%
\definecolor{currentfill}{rgb}{0.000000,0.000000,0.000000}%
\pgfsetfillcolor{currentfill}%
\pgfsetlinewidth{0.602250pt}%
\definecolor{currentstroke}{rgb}{0.000000,0.000000,0.000000}%
\pgfsetstrokecolor{currentstroke}%
\pgfsetdash{}{0pt}%
\pgfsys@defobject{currentmarker}{\pgfqpoint{-0.027778in}{0.000000in}}{\pgfqpoint{0.000000in}{0.000000in}}{%
\pgfpathmoveto{\pgfqpoint{0.000000in}{0.000000in}}%
\pgfpathlineto{\pgfqpoint{-0.027778in}{0.000000in}}%
\pgfusepath{stroke,fill}%
}%
\begin{pgfscope}%
\pgfsys@transformshift{0.375000in}{0.724154in}%
\pgfsys@useobject{currentmarker}{}%
\end{pgfscope}%
\end{pgfscope}%
\begin{pgfscope}%
\pgfsetbuttcap%
\pgfsetroundjoin%
\definecolor{currentfill}{rgb}{0.000000,0.000000,0.000000}%
\pgfsetfillcolor{currentfill}%
\pgfsetlinewidth{0.602250pt}%
\definecolor{currentstroke}{rgb}{0.000000,0.000000,0.000000}%
\pgfsetstrokecolor{currentstroke}%
\pgfsetdash{}{0pt}%
\pgfsys@defobject{currentmarker}{\pgfqpoint{-0.027778in}{0.000000in}}{\pgfqpoint{0.000000in}{0.000000in}}{%
\pgfpathmoveto{\pgfqpoint{0.000000in}{0.000000in}}%
\pgfpathlineto{\pgfqpoint{-0.027778in}{0.000000in}}%
\pgfusepath{stroke,fill}%
}%
\begin{pgfscope}%
\pgfsys@transformshift{0.375000in}{0.776336in}%
\pgfsys@useobject{currentmarker}{}%
\end{pgfscope}%
\end{pgfscope}%
\begin{pgfscope}%
\pgfsetbuttcap%
\pgfsetroundjoin%
\definecolor{currentfill}{rgb}{0.000000,0.000000,0.000000}%
\pgfsetfillcolor{currentfill}%
\pgfsetlinewidth{0.602250pt}%
\definecolor{currentstroke}{rgb}{0.000000,0.000000,0.000000}%
\pgfsetstrokecolor{currentstroke}%
\pgfsetdash{}{0pt}%
\pgfsys@defobject{currentmarker}{\pgfqpoint{-0.027778in}{0.000000in}}{\pgfqpoint{0.000000in}{0.000000in}}{%
\pgfpathmoveto{\pgfqpoint{0.000000in}{0.000000in}}%
\pgfpathlineto{\pgfqpoint{-0.027778in}{0.000000in}}%
\pgfusepath{stroke,fill}%
}%
\begin{pgfscope}%
\pgfsys@transformshift{0.375000in}{0.822364in}%
\pgfsys@useobject{currentmarker}{}%
\end{pgfscope}%
\end{pgfscope}%
\begin{pgfscope}%
\pgfsetbuttcap%
\pgfsetroundjoin%
\definecolor{currentfill}{rgb}{0.000000,0.000000,0.000000}%
\pgfsetfillcolor{currentfill}%
\pgfsetlinewidth{0.602250pt}%
\definecolor{currentstroke}{rgb}{0.000000,0.000000,0.000000}%
\pgfsetstrokecolor{currentstroke}%
\pgfsetdash{}{0pt}%
\pgfsys@defobject{currentmarker}{\pgfqpoint{-0.027778in}{0.000000in}}{\pgfqpoint{0.000000in}{0.000000in}}{%
\pgfpathmoveto{\pgfqpoint{0.000000in}{0.000000in}}%
\pgfpathlineto{\pgfqpoint{-0.027778in}{0.000000in}}%
\pgfusepath{stroke,fill}%
}%
\begin{pgfscope}%
\pgfsys@transformshift{0.375000in}{1.134408in}%
\pgfsys@useobject{currentmarker}{}%
\end{pgfscope}%
\end{pgfscope}%
\begin{pgfscope}%
\pgfsetbuttcap%
\pgfsetroundjoin%
\definecolor{currentfill}{rgb}{0.000000,0.000000,0.000000}%
\pgfsetfillcolor{currentfill}%
\pgfsetlinewidth{0.602250pt}%
\definecolor{currentstroke}{rgb}{0.000000,0.000000,0.000000}%
\pgfsetstrokecolor{currentstroke}%
\pgfsetdash{}{0pt}%
\pgfsys@defobject{currentmarker}{\pgfqpoint{-0.027778in}{0.000000in}}{\pgfqpoint{0.000000in}{0.000000in}}{%
\pgfpathmoveto{\pgfqpoint{0.000000in}{0.000000in}}%
\pgfpathlineto{\pgfqpoint{-0.027778in}{0.000000in}}%
\pgfusepath{stroke,fill}%
}%
\begin{pgfscope}%
\pgfsys@transformshift{0.375000in}{1.292857in}%
\pgfsys@useobject{currentmarker}{}%
\end{pgfscope}%
\end{pgfscope}%
\begin{pgfscope}%
\pgfsetbuttcap%
\pgfsetroundjoin%
\definecolor{currentfill}{rgb}{0.000000,0.000000,0.000000}%
\pgfsetfillcolor{currentfill}%
\pgfsetlinewidth{0.602250pt}%
\definecolor{currentstroke}{rgb}{0.000000,0.000000,0.000000}%
\pgfsetstrokecolor{currentstroke}%
\pgfsetdash{}{0pt}%
\pgfsys@defobject{currentmarker}{\pgfqpoint{-0.027778in}{0.000000in}}{\pgfqpoint{0.000000in}{0.000000in}}{%
\pgfpathmoveto{\pgfqpoint{0.000000in}{0.000000in}}%
\pgfpathlineto{\pgfqpoint{-0.027778in}{0.000000in}}%
\pgfusepath{stroke,fill}%
}%
\begin{pgfscope}%
\pgfsys@transformshift{0.375000in}{1.405278in}%
\pgfsys@useobject{currentmarker}{}%
\end{pgfscope}%
\end{pgfscope}%
\begin{pgfscope}%
\pgfsetbuttcap%
\pgfsetroundjoin%
\definecolor{currentfill}{rgb}{0.000000,0.000000,0.000000}%
\pgfsetfillcolor{currentfill}%
\pgfsetlinewidth{0.602250pt}%
\definecolor{currentstroke}{rgb}{0.000000,0.000000,0.000000}%
\pgfsetstrokecolor{currentstroke}%
\pgfsetdash{}{0pt}%
\pgfsys@defobject{currentmarker}{\pgfqpoint{-0.027778in}{0.000000in}}{\pgfqpoint{0.000000in}{0.000000in}}{%
\pgfpathmoveto{\pgfqpoint{0.000000in}{0.000000in}}%
\pgfpathlineto{\pgfqpoint{-0.027778in}{0.000000in}}%
\pgfusepath{stroke,fill}%
}%
\begin{pgfscope}%
\pgfsys@transformshift{0.375000in}{1.492479in}%
\pgfsys@useobject{currentmarker}{}%
\end{pgfscope}%
\end{pgfscope}%
\begin{pgfscope}%
\pgfsetbuttcap%
\pgfsetroundjoin%
\definecolor{currentfill}{rgb}{0.000000,0.000000,0.000000}%
\pgfsetfillcolor{currentfill}%
\pgfsetlinewidth{0.602250pt}%
\definecolor{currentstroke}{rgb}{0.000000,0.000000,0.000000}%
\pgfsetstrokecolor{currentstroke}%
\pgfsetdash{}{0pt}%
\pgfsys@defobject{currentmarker}{\pgfqpoint{-0.027778in}{0.000000in}}{\pgfqpoint{0.000000in}{0.000000in}}{%
\pgfpathmoveto{\pgfqpoint{0.000000in}{0.000000in}}%
\pgfpathlineto{\pgfqpoint{-0.027778in}{0.000000in}}%
\pgfusepath{stroke,fill}%
}%
\begin{pgfscope}%
\pgfsys@transformshift{0.375000in}{1.563728in}%
\pgfsys@useobject{currentmarker}{}%
\end{pgfscope}%
\end{pgfscope}%
\begin{pgfscope}%
\pgfsetbuttcap%
\pgfsetroundjoin%
\definecolor{currentfill}{rgb}{0.000000,0.000000,0.000000}%
\pgfsetfillcolor{currentfill}%
\pgfsetlinewidth{0.602250pt}%
\definecolor{currentstroke}{rgb}{0.000000,0.000000,0.000000}%
\pgfsetstrokecolor{currentstroke}%
\pgfsetdash{}{0pt}%
\pgfsys@defobject{currentmarker}{\pgfqpoint{-0.027778in}{0.000000in}}{\pgfqpoint{0.000000in}{0.000000in}}{%
\pgfpathmoveto{\pgfqpoint{0.000000in}{0.000000in}}%
\pgfpathlineto{\pgfqpoint{-0.027778in}{0.000000in}}%
\pgfusepath{stroke,fill}%
}%
\begin{pgfscope}%
\pgfsys@transformshift{0.375000in}{1.623967in}%
\pgfsys@useobject{currentmarker}{}%
\end{pgfscope}%
\end{pgfscope}%
\begin{pgfscope}%
\pgfsetbuttcap%
\pgfsetroundjoin%
\definecolor{currentfill}{rgb}{0.000000,0.000000,0.000000}%
\pgfsetfillcolor{currentfill}%
\pgfsetlinewidth{0.602250pt}%
\definecolor{currentstroke}{rgb}{0.000000,0.000000,0.000000}%
\pgfsetstrokecolor{currentstroke}%
\pgfsetdash{}{0pt}%
\pgfsys@defobject{currentmarker}{\pgfqpoint{-0.027778in}{0.000000in}}{\pgfqpoint{0.000000in}{0.000000in}}{%
\pgfpathmoveto{\pgfqpoint{0.000000in}{0.000000in}}%
\pgfpathlineto{\pgfqpoint{-0.027778in}{0.000000in}}%
\pgfusepath{stroke,fill}%
}%
\begin{pgfscope}%
\pgfsys@transformshift{0.375000in}{1.676149in}%
\pgfsys@useobject{currentmarker}{}%
\end{pgfscope}%
\end{pgfscope}%
\begin{pgfscope}%
\pgfsetbuttcap%
\pgfsetroundjoin%
\definecolor{currentfill}{rgb}{0.000000,0.000000,0.000000}%
\pgfsetfillcolor{currentfill}%
\pgfsetlinewidth{0.602250pt}%
\definecolor{currentstroke}{rgb}{0.000000,0.000000,0.000000}%
\pgfsetstrokecolor{currentstroke}%
\pgfsetdash{}{0pt}%
\pgfsys@defobject{currentmarker}{\pgfqpoint{-0.027778in}{0.000000in}}{\pgfqpoint{0.000000in}{0.000000in}}{%
\pgfpathmoveto{\pgfqpoint{0.000000in}{0.000000in}}%
\pgfpathlineto{\pgfqpoint{-0.027778in}{0.000000in}}%
\pgfusepath{stroke,fill}%
}%
\begin{pgfscope}%
\pgfsys@transformshift{0.375000in}{1.722177in}%
\pgfsys@useobject{currentmarker}{}%
\end{pgfscope}%
\end{pgfscope}%
\begin{pgfscope}%
\pgfsetbuttcap%
\pgfsetroundjoin%
\definecolor{currentfill}{rgb}{0.000000,0.000000,0.000000}%
\pgfsetfillcolor{currentfill}%
\pgfsetlinewidth{0.602250pt}%
\definecolor{currentstroke}{rgb}{0.000000,0.000000,0.000000}%
\pgfsetstrokecolor{currentstroke}%
\pgfsetdash{}{0pt}%
\pgfsys@defobject{currentmarker}{\pgfqpoint{-0.027778in}{0.000000in}}{\pgfqpoint{0.000000in}{0.000000in}}{%
\pgfpathmoveto{\pgfqpoint{0.000000in}{0.000000in}}%
\pgfpathlineto{\pgfqpoint{-0.027778in}{0.000000in}}%
\pgfusepath{stroke,fill}%
}%
\begin{pgfscope}%
\pgfsys@transformshift{0.375000in}{2.034221in}%
\pgfsys@useobject{currentmarker}{}%
\end{pgfscope}%
\end{pgfscope}%
\begin{pgfscope}%
\pgfsetbuttcap%
\pgfsetroundjoin%
\definecolor{currentfill}{rgb}{0.000000,0.000000,0.000000}%
\pgfsetfillcolor{currentfill}%
\pgfsetlinewidth{0.602250pt}%
\definecolor{currentstroke}{rgb}{0.000000,0.000000,0.000000}%
\pgfsetstrokecolor{currentstroke}%
\pgfsetdash{}{0pt}%
\pgfsys@defobject{currentmarker}{\pgfqpoint{-0.027778in}{0.000000in}}{\pgfqpoint{0.000000in}{0.000000in}}{%
\pgfpathmoveto{\pgfqpoint{0.000000in}{0.000000in}}%
\pgfpathlineto{\pgfqpoint{-0.027778in}{0.000000in}}%
\pgfusepath{stroke,fill}%
}%
\begin{pgfscope}%
\pgfsys@transformshift{0.375000in}{2.192670in}%
\pgfsys@useobject{currentmarker}{}%
\end{pgfscope}%
\end{pgfscope}%
\begin{pgfscope}%
\pgfsetbuttcap%
\pgfsetroundjoin%
\definecolor{currentfill}{rgb}{0.000000,0.000000,0.000000}%
\pgfsetfillcolor{currentfill}%
\pgfsetlinewidth{0.602250pt}%
\definecolor{currentstroke}{rgb}{0.000000,0.000000,0.000000}%
\pgfsetstrokecolor{currentstroke}%
\pgfsetdash{}{0pt}%
\pgfsys@defobject{currentmarker}{\pgfqpoint{-0.027778in}{0.000000in}}{\pgfqpoint{0.000000in}{0.000000in}}{%
\pgfpathmoveto{\pgfqpoint{0.000000in}{0.000000in}}%
\pgfpathlineto{\pgfqpoint{-0.027778in}{0.000000in}}%
\pgfusepath{stroke,fill}%
}%
\begin{pgfscope}%
\pgfsys@transformshift{0.375000in}{2.305092in}%
\pgfsys@useobject{currentmarker}{}%
\end{pgfscope}%
\end{pgfscope}%
\begin{pgfscope}%
\pgfsetbuttcap%
\pgfsetroundjoin%
\definecolor{currentfill}{rgb}{0.000000,0.000000,0.000000}%
\pgfsetfillcolor{currentfill}%
\pgfsetlinewidth{0.602250pt}%
\definecolor{currentstroke}{rgb}{0.000000,0.000000,0.000000}%
\pgfsetstrokecolor{currentstroke}%
\pgfsetdash{}{0pt}%
\pgfsys@defobject{currentmarker}{\pgfqpoint{-0.027778in}{0.000000in}}{\pgfqpoint{0.000000in}{0.000000in}}{%
\pgfpathmoveto{\pgfqpoint{0.000000in}{0.000000in}}%
\pgfpathlineto{\pgfqpoint{-0.027778in}{0.000000in}}%
\pgfusepath{stroke,fill}%
}%
\begin{pgfscope}%
\pgfsys@transformshift{0.375000in}{2.392292in}%
\pgfsys@useobject{currentmarker}{}%
\end{pgfscope}%
\end{pgfscope}%
\begin{pgfscope}%
\pgfsetbuttcap%
\pgfsetroundjoin%
\definecolor{currentfill}{rgb}{0.000000,0.000000,0.000000}%
\pgfsetfillcolor{currentfill}%
\pgfsetlinewidth{0.602250pt}%
\definecolor{currentstroke}{rgb}{0.000000,0.000000,0.000000}%
\pgfsetstrokecolor{currentstroke}%
\pgfsetdash{}{0pt}%
\pgfsys@defobject{currentmarker}{\pgfqpoint{-0.027778in}{0.000000in}}{\pgfqpoint{0.000000in}{0.000000in}}{%
\pgfpathmoveto{\pgfqpoint{0.000000in}{0.000000in}}%
\pgfpathlineto{\pgfqpoint{-0.027778in}{0.000000in}}%
\pgfusepath{stroke,fill}%
}%
\begin{pgfscope}%
\pgfsys@transformshift{0.375000in}{2.463541in}%
\pgfsys@useobject{currentmarker}{}%
\end{pgfscope}%
\end{pgfscope}%
\begin{pgfscope}%
\pgfsetbuttcap%
\pgfsetroundjoin%
\definecolor{currentfill}{rgb}{0.000000,0.000000,0.000000}%
\pgfsetfillcolor{currentfill}%
\pgfsetlinewidth{0.602250pt}%
\definecolor{currentstroke}{rgb}{0.000000,0.000000,0.000000}%
\pgfsetstrokecolor{currentstroke}%
\pgfsetdash{}{0pt}%
\pgfsys@defobject{currentmarker}{\pgfqpoint{-0.027778in}{0.000000in}}{\pgfqpoint{0.000000in}{0.000000in}}{%
\pgfpathmoveto{\pgfqpoint{0.000000in}{0.000000in}}%
\pgfpathlineto{\pgfqpoint{-0.027778in}{0.000000in}}%
\pgfusepath{stroke,fill}%
}%
\begin{pgfscope}%
\pgfsys@transformshift{0.375000in}{2.523780in}%
\pgfsys@useobject{currentmarker}{}%
\end{pgfscope}%
\end{pgfscope}%
\begin{pgfscope}%
\pgfsetbuttcap%
\pgfsetroundjoin%
\definecolor{currentfill}{rgb}{0.000000,0.000000,0.000000}%
\pgfsetfillcolor{currentfill}%
\pgfsetlinewidth{0.602250pt}%
\definecolor{currentstroke}{rgb}{0.000000,0.000000,0.000000}%
\pgfsetstrokecolor{currentstroke}%
\pgfsetdash{}{0pt}%
\pgfsys@defobject{currentmarker}{\pgfqpoint{-0.027778in}{0.000000in}}{\pgfqpoint{0.000000in}{0.000000in}}{%
\pgfpathmoveto{\pgfqpoint{0.000000in}{0.000000in}}%
\pgfpathlineto{\pgfqpoint{-0.027778in}{0.000000in}}%
\pgfusepath{stroke,fill}%
}%
\begin{pgfscope}%
\pgfsys@transformshift{0.375000in}{2.575962in}%
\pgfsys@useobject{currentmarker}{}%
\end{pgfscope}%
\end{pgfscope}%
\begin{pgfscope}%
\pgfsetbuttcap%
\pgfsetroundjoin%
\definecolor{currentfill}{rgb}{0.000000,0.000000,0.000000}%
\pgfsetfillcolor{currentfill}%
\pgfsetlinewidth{0.602250pt}%
\definecolor{currentstroke}{rgb}{0.000000,0.000000,0.000000}%
\pgfsetstrokecolor{currentstroke}%
\pgfsetdash{}{0pt}%
\pgfsys@defobject{currentmarker}{\pgfqpoint{-0.027778in}{0.000000in}}{\pgfqpoint{0.000000in}{0.000000in}}{%
\pgfpathmoveto{\pgfqpoint{0.000000in}{0.000000in}}%
\pgfpathlineto{\pgfqpoint{-0.027778in}{0.000000in}}%
\pgfusepath{stroke,fill}%
}%
\begin{pgfscope}%
\pgfsys@transformshift{0.375000in}{2.621990in}%
\pgfsys@useobject{currentmarker}{}%
\end{pgfscope}%
\end{pgfscope}%
\begin{pgfscope}%
\definecolor{textcolor}{rgb}{0.000000,0.000000,0.000000}%
\pgfsetstrokecolor{textcolor}%
\pgfsetfillcolor{textcolor}%
\pgftext[x=-0.065780in,y=1.485000in,,bottom,rotate=90.000000]{\color{textcolor}\sffamily\fontsize{10.000000}{12.000000}\selectfont QMC Error}%
\end{pgfscope}%
\begin{pgfscope}%
\pgfpathrectangle{\pgfqpoint{0.375000in}{0.330000in}}{\pgfqpoint{2.325000in}{2.310000in}}%
\pgfusepath{clip}%
\pgfsetbuttcap%
\pgfsetroundjoin%
\definecolor{currentfill}{rgb}{0.000000,0.000000,0.000000}%
\pgfsetfillcolor{currentfill}%
\pgfsetlinewidth{1.003750pt}%
\definecolor{currentstroke}{rgb}{0.000000,0.000000,0.000000}%
\pgfsetstrokecolor{currentstroke}%
\pgfsetdash{}{0pt}%
\pgfsys@defobject{currentmarker}{\pgfqpoint{-0.041667in}{-0.041667in}}{\pgfqpoint{0.041667in}{0.041667in}}{%
\pgfpathmoveto{\pgfqpoint{0.000000in}{-0.041667in}}%
\pgfpathcurveto{\pgfqpoint{0.011050in}{-0.041667in}}{\pgfqpoint{0.021649in}{-0.037276in}}{\pgfqpoint{0.029463in}{-0.029463in}}%
\pgfpathcurveto{\pgfqpoint{0.037276in}{-0.021649in}}{\pgfqpoint{0.041667in}{-0.011050in}}{\pgfqpoint{0.041667in}{0.000000in}}%
\pgfpathcurveto{\pgfqpoint{0.041667in}{0.011050in}}{\pgfqpoint{0.037276in}{0.021649in}}{\pgfqpoint{0.029463in}{0.029463in}}%
\pgfpathcurveto{\pgfqpoint{0.021649in}{0.037276in}}{\pgfqpoint{0.011050in}{0.041667in}}{\pgfqpoint{0.000000in}{0.041667in}}%
\pgfpathcurveto{\pgfqpoint{-0.011050in}{0.041667in}}{\pgfqpoint{-0.021649in}{0.037276in}}{\pgfqpoint{-0.029463in}{0.029463in}}%
\pgfpathcurveto{\pgfqpoint{-0.037276in}{0.021649in}}{\pgfqpoint{-0.041667in}{0.011050in}}{\pgfqpoint{-0.041667in}{0.000000in}}%
\pgfpathcurveto{\pgfqpoint{-0.041667in}{-0.011050in}}{\pgfqpoint{-0.037276in}{-0.021649in}}{\pgfqpoint{-0.029463in}{-0.029463in}}%
\pgfpathcurveto{\pgfqpoint{-0.021649in}{-0.037276in}}{\pgfqpoint{-0.011050in}{-0.041667in}}{\pgfqpoint{0.000000in}{-0.041667in}}%
\pgfpathclose%
\pgfusepath{stroke,fill}%
}%
\begin{pgfscope}%
\pgfsys@transformshift{0.480682in}{2.487311in}%
\pgfsys@useobject{currentmarker}{}%
\end{pgfscope}%
\begin{pgfscope}%
\pgfsys@transformshift{0.672831in}{2.402518in}%
\pgfsys@useobject{currentmarker}{}%
\end{pgfscope}%
\begin{pgfscope}%
\pgfsys@transformshift{0.864979in}{2.337984in}%
\pgfsys@useobject{currentmarker}{}%
\end{pgfscope}%
\begin{pgfscope}%
\pgfsys@transformshift{1.057128in}{1.999518in}%
\pgfsys@useobject{currentmarker}{}%
\end{pgfscope}%
\begin{pgfscope}%
\pgfsys@transformshift{1.249277in}{1.827197in}%
\pgfsys@useobject{currentmarker}{}%
\end{pgfscope}%
\begin{pgfscope}%
\pgfsys@transformshift{1.441426in}{1.254926in}%
\pgfsys@useobject{currentmarker}{}%
\end{pgfscope}%
\begin{pgfscope}%
\pgfsys@transformshift{1.633574in}{1.584169in}%
\pgfsys@useobject{currentmarker}{}%
\end{pgfscope}%
\begin{pgfscope}%
\pgfsys@transformshift{1.825723in}{1.682185in}%
\pgfsys@useobject{currentmarker}{}%
\end{pgfscope}%
\begin{pgfscope}%
\pgfsys@transformshift{2.017872in}{1.008966in}%
\pgfsys@useobject{currentmarker}{}%
\end{pgfscope}%
\begin{pgfscope}%
\pgfsys@transformshift{2.210021in}{0.435000in}%
\pgfsys@useobject{currentmarker}{}%
\end{pgfscope}%
\begin{pgfscope}%
\pgfsys@transformshift{2.402169in}{0.780615in}%
\pgfsys@useobject{currentmarker}{}%
\end{pgfscope}%
\begin{pgfscope}%
\pgfsys@transformshift{2.594318in}{0.847802in}%
\pgfsys@useobject{currentmarker}{}%
\end{pgfscope}%
\end{pgfscope}%
\begin{pgfscope}%
\pgfpathrectangle{\pgfqpoint{0.375000in}{0.330000in}}{\pgfqpoint{2.325000in}{2.310000in}}%
\pgfusepath{clip}%
\pgfsetbuttcap%
\pgfsetroundjoin%
\pgfsetlinewidth{1.505625pt}%
\definecolor{currentstroke}{rgb}{0.000000,0.000000,0.000000}%
\pgfsetstrokecolor{currentstroke}%
\pgfsetdash{{5.550000pt}{2.400000pt}}{0.000000pt}%
\pgfpathmoveto{\pgfqpoint{0.480682in}{2.535000in}}%
\pgfpathlineto{\pgfqpoint{2.594318in}{0.573032in}}%
\pgfusepath{stroke}%
\end{pgfscope}%
\begin{pgfscope}%
\pgfsetrectcap%
\pgfsetmiterjoin%
\pgfsetlinewidth{0.803000pt}%
\definecolor{currentstroke}{rgb}{0.000000,0.000000,0.000000}%
\pgfsetstrokecolor{currentstroke}%
\pgfsetdash{}{0pt}%
\pgfpathmoveto{\pgfqpoint{0.375000in}{0.330000in}}%
\pgfpathlineto{\pgfqpoint{0.375000in}{2.640000in}}%
\pgfusepath{stroke}%
\end{pgfscope}%
\begin{pgfscope}%
\pgfsetrectcap%
\pgfsetmiterjoin%
\pgfsetlinewidth{0.803000pt}%
\definecolor{currentstroke}{rgb}{0.000000,0.000000,0.000000}%
\pgfsetstrokecolor{currentstroke}%
\pgfsetdash{}{0pt}%
\pgfpathmoveto{\pgfqpoint{2.700000in}{0.330000in}}%
\pgfpathlineto{\pgfqpoint{2.700000in}{2.640000in}}%
\pgfusepath{stroke}%
\end{pgfscope}%
\begin{pgfscope}%
\pgfsetrectcap%
\pgfsetmiterjoin%
\pgfsetlinewidth{0.803000pt}%
\definecolor{currentstroke}{rgb}{0.000000,0.000000,0.000000}%
\pgfsetstrokecolor{currentstroke}%
\pgfsetdash{}{0pt}%
\pgfpathmoveto{\pgfqpoint{0.375000in}{0.330000in}}%
\pgfpathlineto{\pgfqpoint{2.700000in}{0.330000in}}%
\pgfusepath{stroke}%
\end{pgfscope}%
\begin{pgfscope}%
\pgfsetrectcap%
\pgfsetmiterjoin%
\pgfsetlinewidth{0.803000pt}%
\definecolor{currentstroke}{rgb}{0.000000,0.000000,0.000000}%
\pgfsetstrokecolor{currentstroke}%
\pgfsetdash{}{0pt}%
\pgfpathmoveto{\pgfqpoint{0.375000in}{2.640000in}}%
\pgfpathlineto{\pgfqpoint{2.700000in}{2.640000in}}%
\pgfusepath{stroke}%
\end{pgfscope}%
\begin{pgfscope}%
\pgfsetbuttcap%
\pgfsetmiterjoin%
\definecolor{currentfill}{rgb}{1.000000,1.000000,1.000000}%
\pgfsetfillcolor{currentfill}%
\pgfsetfillopacity{0.800000}%
\pgfsetlinewidth{1.003750pt}%
\definecolor{currentstroke}{rgb}{0.800000,0.800000,0.800000}%
\pgfsetstrokecolor{currentstroke}%
\pgfsetstrokeopacity{0.800000}%
\pgfsetdash{}{0pt}%
\pgfpathmoveto{\pgfqpoint{1.242243in}{2.275267in}}%
\pgfpathlineto{\pgfqpoint{2.602778in}{2.275267in}}%
\pgfpathquadraticcurveto{\pgfqpoint{2.630556in}{2.275267in}}{\pgfqpoint{2.630556in}{2.303045in}}%
\pgfpathlineto{\pgfqpoint{2.630556in}{2.542778in}}%
\pgfpathquadraticcurveto{\pgfqpoint{2.630556in}{2.570556in}}{\pgfqpoint{2.602778in}{2.570556in}}%
\pgfpathlineto{\pgfqpoint{1.242243in}{2.570556in}}%
\pgfpathquadraticcurveto{\pgfqpoint{1.214465in}{2.570556in}}{\pgfqpoint{1.214465in}{2.542778in}}%
\pgfpathlineto{\pgfqpoint{1.214465in}{2.303045in}}%
\pgfpathquadraticcurveto{\pgfqpoint{1.214465in}{2.275267in}}{\pgfqpoint{1.242243in}{2.275267in}}%
\pgfpathclose%
\pgfusepath{stroke,fill}%
\end{pgfscope}%
\begin{pgfscope}%
\pgfsetbuttcap%
\pgfsetroundjoin%
\pgfsetlinewidth{1.505625pt}%
\definecolor{currentstroke}{rgb}{0.000000,0.000000,0.000000}%
\pgfsetstrokecolor{currentstroke}%
\pgfsetdash{{5.550000pt}{2.400000pt}}{0.000000pt}%
\pgfpathmoveto{\pgfqpoint{1.270021in}{2.441543in}}%
\pgfpathlineto{\pgfqpoint{1.547799in}{2.441543in}}%
\pgfusepath{stroke}%
\end{pgfscope}%
\begin{pgfscope}%
\definecolor{textcolor}{rgb}{0.000000,0.000000,0.000000}%
\pgfsetstrokecolor{textcolor}%
\pgfsetfillcolor{textcolor}%
\pgftext[x=1.658910in,y=2.392932in,left,base]{\color{textcolor}\sffamily\fontsize{10.000000}{12.000000}\selectfont \(\displaystyle 0.0720N_{\mathrm{QMC}}^{-0.6584}\)}%
\end{pgfscope}%
\end{pgfpicture}%
\makeatother%
\endgroup%

\caption[Quasi-Monte-Carlo error, for $Q=u((1,1))$ and $k=60$.]{Quasi-Monte-Carlo error with increasing number of Quasi-Monte-Carlo points, for $Q=u((1,1))$ and $k=60$. \label{fig:qmctop_right60}}
\end{figure}

\begin{figure}[h]
    \centering
    %% Creator: Matplotlib, PGF backend
%%
%% To include the figure in your LaTeX document, write
%%   \input{<filename>.pgf}
%%
%% Make sure the required packages are loaded in your preamble
%%   \usepackage{pgf}
%%
%% Figures using additional raster images can only be included by \input if
%% they are in the same directory as the main LaTeX file. For loading figures
%% from other directories you can use the `import` package
%%   \usepackage{import}
%% and then include the figures with
%%   \import{<path to file>}{<filename>.pgf}
%%
%% Matplotlib used the following preamble
%%   \usepackage{fontspec}
%%   \setmainfont{DejaVuSerif.ttf}[Path=/home/owen/progs/firedrake-complex/firedrake/lib/python3.5/site-packages/matplotlib/mpl-data/fonts/ttf/]
%%   \setsansfont{DejaVuSans.ttf}[Path=/home/owen/progs/firedrake-complex/firedrake/lib/python3.5/site-packages/matplotlib/mpl-data/fonts/ttf/]
%%   \setmonofont{DejaVuSansMono.ttf}[Path=/home/owen/progs/firedrake-complex/firedrake/lib/python3.5/site-packages/matplotlib/mpl-data/fonts/ttf/]
%%
\begingroup%
\makeatletter%
\begin{pgfpicture}%
\pgfpathrectangle{\pgfpointorigin}{\pgfqpoint{5.000000in}{4.000000in}}%
\pgfusepath{use as bounding box, clip}%
\begin{pgfscope}%
\pgfsetbuttcap%
\pgfsetmiterjoin%
\definecolor{currentfill}{rgb}{1.000000,1.000000,1.000000}%
\pgfsetfillcolor{currentfill}%
\pgfsetlinewidth{0.000000pt}%
\definecolor{currentstroke}{rgb}{1.000000,1.000000,1.000000}%
\pgfsetstrokecolor{currentstroke}%
\pgfsetdash{}{0pt}%
\pgfpathmoveto{\pgfqpoint{0.000000in}{0.000000in}}%
\pgfpathlineto{\pgfqpoint{5.000000in}{0.000000in}}%
\pgfpathlineto{\pgfqpoint{5.000000in}{4.000000in}}%
\pgfpathlineto{\pgfqpoint{0.000000in}{4.000000in}}%
\pgfpathclose%
\pgfusepath{fill}%
\end{pgfscope}%
\begin{pgfscope}%
\pgfsetbuttcap%
\pgfsetmiterjoin%
\definecolor{currentfill}{rgb}{1.000000,1.000000,1.000000}%
\pgfsetfillcolor{currentfill}%
\pgfsetlinewidth{0.000000pt}%
\definecolor{currentstroke}{rgb}{0.000000,0.000000,0.000000}%
\pgfsetstrokecolor{currentstroke}%
\pgfsetstrokeopacity{0.000000}%
\pgfsetdash{}{0pt}%
\pgfpathmoveto{\pgfqpoint{0.625000in}{0.440000in}}%
\pgfpathlineto{\pgfqpoint{4.500000in}{0.440000in}}%
\pgfpathlineto{\pgfqpoint{4.500000in}{3.520000in}}%
\pgfpathlineto{\pgfqpoint{0.625000in}{3.520000in}}%
\pgfpathclose%
\pgfusepath{fill}%
\end{pgfscope}%
\begin{pgfscope}%
\pgfsetbuttcap%
\pgfsetroundjoin%
\definecolor{currentfill}{rgb}{0.000000,0.000000,0.000000}%
\pgfsetfillcolor{currentfill}%
\pgfsetlinewidth{0.803000pt}%
\definecolor{currentstroke}{rgb}{0.000000,0.000000,0.000000}%
\pgfsetstrokecolor{currentstroke}%
\pgfsetdash{}{0pt}%
\pgfsys@defobject{currentmarker}{\pgfqpoint{0.000000in}{-0.048611in}}{\pgfqpoint{0.000000in}{0.000000in}}{%
\pgfpathmoveto{\pgfqpoint{0.000000in}{0.000000in}}%
\pgfpathlineto{\pgfqpoint{0.000000in}{-0.048611in}}%
\pgfusepath{stroke,fill}%
}%
\begin{pgfscope}%
\pgfsys@transformshift{0.801136in}{0.440000in}%
\pgfsys@useobject{currentmarker}{}%
\end{pgfscope}%
\end{pgfscope}%
\begin{pgfscope}%
\definecolor{textcolor}{rgb}{0.000000,0.000000,0.000000}%
\pgfsetstrokecolor{textcolor}%
\pgfsetfillcolor{textcolor}%
\pgftext[x=0.801136in,y=0.342778in,,top]{\color{textcolor}\sffamily\fontsize{10.000000}{12.000000}\selectfont \(\displaystyle {10^{0}}\)}%
\end{pgfscope}%
\begin{pgfscope}%
\pgfsetbuttcap%
\pgfsetroundjoin%
\definecolor{currentfill}{rgb}{0.000000,0.000000,0.000000}%
\pgfsetfillcolor{currentfill}%
\pgfsetlinewidth{0.803000pt}%
\definecolor{currentstroke}{rgb}{0.000000,0.000000,0.000000}%
\pgfsetstrokecolor{currentstroke}%
\pgfsetdash{}{0pt}%
\pgfsys@defobject{currentmarker}{\pgfqpoint{0.000000in}{-0.048611in}}{\pgfqpoint{0.000000in}{0.000000in}}{%
\pgfpathmoveto{\pgfqpoint{0.000000in}{0.000000in}}%
\pgfpathlineto{\pgfqpoint{0.000000in}{-0.048611in}}%
\pgfusepath{stroke,fill}%
}%
\begin{pgfscope}%
\pgfsys@transformshift{1.864977in}{0.440000in}%
\pgfsys@useobject{currentmarker}{}%
\end{pgfscope}%
\end{pgfscope}%
\begin{pgfscope}%
\definecolor{textcolor}{rgb}{0.000000,0.000000,0.000000}%
\pgfsetstrokecolor{textcolor}%
\pgfsetfillcolor{textcolor}%
\pgftext[x=1.864977in,y=0.342778in,,top]{\color{textcolor}\sffamily\fontsize{10.000000}{12.000000}\selectfont \(\displaystyle {10^{1}}\)}%
\end{pgfscope}%
\begin{pgfscope}%
\pgfsetbuttcap%
\pgfsetroundjoin%
\definecolor{currentfill}{rgb}{0.000000,0.000000,0.000000}%
\pgfsetfillcolor{currentfill}%
\pgfsetlinewidth{0.803000pt}%
\definecolor{currentstroke}{rgb}{0.000000,0.000000,0.000000}%
\pgfsetstrokecolor{currentstroke}%
\pgfsetdash{}{0pt}%
\pgfsys@defobject{currentmarker}{\pgfqpoint{0.000000in}{-0.048611in}}{\pgfqpoint{0.000000in}{0.000000in}}{%
\pgfpathmoveto{\pgfqpoint{0.000000in}{0.000000in}}%
\pgfpathlineto{\pgfqpoint{0.000000in}{-0.048611in}}%
\pgfusepath{stroke,fill}%
}%
\begin{pgfscope}%
\pgfsys@transformshift{2.928818in}{0.440000in}%
\pgfsys@useobject{currentmarker}{}%
\end{pgfscope}%
\end{pgfscope}%
\begin{pgfscope}%
\definecolor{textcolor}{rgb}{0.000000,0.000000,0.000000}%
\pgfsetstrokecolor{textcolor}%
\pgfsetfillcolor{textcolor}%
\pgftext[x=2.928818in,y=0.342778in,,top]{\color{textcolor}\sffamily\fontsize{10.000000}{12.000000}\selectfont \(\displaystyle {10^{2}}\)}%
\end{pgfscope}%
\begin{pgfscope}%
\pgfsetbuttcap%
\pgfsetroundjoin%
\definecolor{currentfill}{rgb}{0.000000,0.000000,0.000000}%
\pgfsetfillcolor{currentfill}%
\pgfsetlinewidth{0.803000pt}%
\definecolor{currentstroke}{rgb}{0.000000,0.000000,0.000000}%
\pgfsetstrokecolor{currentstroke}%
\pgfsetdash{}{0pt}%
\pgfsys@defobject{currentmarker}{\pgfqpoint{0.000000in}{-0.048611in}}{\pgfqpoint{0.000000in}{0.000000in}}{%
\pgfpathmoveto{\pgfqpoint{0.000000in}{0.000000in}}%
\pgfpathlineto{\pgfqpoint{0.000000in}{-0.048611in}}%
\pgfusepath{stroke,fill}%
}%
\begin{pgfscope}%
\pgfsys@transformshift{3.992658in}{0.440000in}%
\pgfsys@useobject{currentmarker}{}%
\end{pgfscope}%
\end{pgfscope}%
\begin{pgfscope}%
\definecolor{textcolor}{rgb}{0.000000,0.000000,0.000000}%
\pgfsetstrokecolor{textcolor}%
\pgfsetfillcolor{textcolor}%
\pgftext[x=3.992658in,y=0.342778in,,top]{\color{textcolor}\sffamily\fontsize{10.000000}{12.000000}\selectfont \(\displaystyle {10^{3}}\)}%
\end{pgfscope}%
\begin{pgfscope}%
\pgfsetbuttcap%
\pgfsetroundjoin%
\definecolor{currentfill}{rgb}{0.000000,0.000000,0.000000}%
\pgfsetfillcolor{currentfill}%
\pgfsetlinewidth{0.602250pt}%
\definecolor{currentstroke}{rgb}{0.000000,0.000000,0.000000}%
\pgfsetstrokecolor{currentstroke}%
\pgfsetdash{}{0pt}%
\pgfsys@defobject{currentmarker}{\pgfqpoint{0.000000in}{-0.027778in}}{\pgfqpoint{0.000000in}{0.000000in}}{%
\pgfpathmoveto{\pgfqpoint{0.000000in}{0.000000in}}%
\pgfpathlineto{\pgfqpoint{0.000000in}{-0.027778in}}%
\pgfusepath{stroke,fill}%
}%
\begin{pgfscope}%
\pgfsys@transformshift{0.636345in}{0.440000in}%
\pgfsys@useobject{currentmarker}{}%
\end{pgfscope}%
\end{pgfscope}%
\begin{pgfscope}%
\pgfsetbuttcap%
\pgfsetroundjoin%
\definecolor{currentfill}{rgb}{0.000000,0.000000,0.000000}%
\pgfsetfillcolor{currentfill}%
\pgfsetlinewidth{0.602250pt}%
\definecolor{currentstroke}{rgb}{0.000000,0.000000,0.000000}%
\pgfsetstrokecolor{currentstroke}%
\pgfsetdash{}{0pt}%
\pgfsys@defobject{currentmarker}{\pgfqpoint{0.000000in}{-0.027778in}}{\pgfqpoint{0.000000in}{0.000000in}}{%
\pgfpathmoveto{\pgfqpoint{0.000000in}{0.000000in}}%
\pgfpathlineto{\pgfqpoint{0.000000in}{-0.027778in}}%
\pgfusepath{stroke,fill}%
}%
\begin{pgfscope}%
\pgfsys@transformshift{0.698040in}{0.440000in}%
\pgfsys@useobject{currentmarker}{}%
\end{pgfscope}%
\end{pgfscope}%
\begin{pgfscope}%
\pgfsetbuttcap%
\pgfsetroundjoin%
\definecolor{currentfill}{rgb}{0.000000,0.000000,0.000000}%
\pgfsetfillcolor{currentfill}%
\pgfsetlinewidth{0.602250pt}%
\definecolor{currentstroke}{rgb}{0.000000,0.000000,0.000000}%
\pgfsetstrokecolor{currentstroke}%
\pgfsetdash{}{0pt}%
\pgfsys@defobject{currentmarker}{\pgfqpoint{0.000000in}{-0.027778in}}{\pgfqpoint{0.000000in}{0.000000in}}{%
\pgfpathmoveto{\pgfqpoint{0.000000in}{0.000000in}}%
\pgfpathlineto{\pgfqpoint{0.000000in}{-0.027778in}}%
\pgfusepath{stroke,fill}%
}%
\begin{pgfscope}%
\pgfsys@transformshift{0.752458in}{0.440000in}%
\pgfsys@useobject{currentmarker}{}%
\end{pgfscope}%
\end{pgfscope}%
\begin{pgfscope}%
\pgfsetbuttcap%
\pgfsetroundjoin%
\definecolor{currentfill}{rgb}{0.000000,0.000000,0.000000}%
\pgfsetfillcolor{currentfill}%
\pgfsetlinewidth{0.602250pt}%
\definecolor{currentstroke}{rgb}{0.000000,0.000000,0.000000}%
\pgfsetstrokecolor{currentstroke}%
\pgfsetdash{}{0pt}%
\pgfsys@defobject{currentmarker}{\pgfqpoint{0.000000in}{-0.027778in}}{\pgfqpoint{0.000000in}{0.000000in}}{%
\pgfpathmoveto{\pgfqpoint{0.000000in}{0.000000in}}%
\pgfpathlineto{\pgfqpoint{0.000000in}{-0.027778in}}%
\pgfusepath{stroke,fill}%
}%
\begin{pgfscope}%
\pgfsys@transformshift{1.121384in}{0.440000in}%
\pgfsys@useobject{currentmarker}{}%
\end{pgfscope}%
\end{pgfscope}%
\begin{pgfscope}%
\pgfsetbuttcap%
\pgfsetroundjoin%
\definecolor{currentfill}{rgb}{0.000000,0.000000,0.000000}%
\pgfsetfillcolor{currentfill}%
\pgfsetlinewidth{0.602250pt}%
\definecolor{currentstroke}{rgb}{0.000000,0.000000,0.000000}%
\pgfsetstrokecolor{currentstroke}%
\pgfsetdash{}{0pt}%
\pgfsys@defobject{currentmarker}{\pgfqpoint{0.000000in}{-0.027778in}}{\pgfqpoint{0.000000in}{0.000000in}}{%
\pgfpathmoveto{\pgfqpoint{0.000000in}{0.000000in}}%
\pgfpathlineto{\pgfqpoint{0.000000in}{-0.027778in}}%
\pgfusepath{stroke,fill}%
}%
\begin{pgfscope}%
\pgfsys@transformshift{1.308717in}{0.440000in}%
\pgfsys@useobject{currentmarker}{}%
\end{pgfscope}%
\end{pgfscope}%
\begin{pgfscope}%
\pgfsetbuttcap%
\pgfsetroundjoin%
\definecolor{currentfill}{rgb}{0.000000,0.000000,0.000000}%
\pgfsetfillcolor{currentfill}%
\pgfsetlinewidth{0.602250pt}%
\definecolor{currentstroke}{rgb}{0.000000,0.000000,0.000000}%
\pgfsetstrokecolor{currentstroke}%
\pgfsetdash{}{0pt}%
\pgfsys@defobject{currentmarker}{\pgfqpoint{0.000000in}{-0.027778in}}{\pgfqpoint{0.000000in}{0.000000in}}{%
\pgfpathmoveto{\pgfqpoint{0.000000in}{0.000000in}}%
\pgfpathlineto{\pgfqpoint{0.000000in}{-0.027778in}}%
\pgfusepath{stroke,fill}%
}%
\begin{pgfscope}%
\pgfsys@transformshift{1.441632in}{0.440000in}%
\pgfsys@useobject{currentmarker}{}%
\end{pgfscope}%
\end{pgfscope}%
\begin{pgfscope}%
\pgfsetbuttcap%
\pgfsetroundjoin%
\definecolor{currentfill}{rgb}{0.000000,0.000000,0.000000}%
\pgfsetfillcolor{currentfill}%
\pgfsetlinewidth{0.602250pt}%
\definecolor{currentstroke}{rgb}{0.000000,0.000000,0.000000}%
\pgfsetstrokecolor{currentstroke}%
\pgfsetdash{}{0pt}%
\pgfsys@defobject{currentmarker}{\pgfqpoint{0.000000in}{-0.027778in}}{\pgfqpoint{0.000000in}{0.000000in}}{%
\pgfpathmoveto{\pgfqpoint{0.000000in}{0.000000in}}%
\pgfpathlineto{\pgfqpoint{0.000000in}{-0.027778in}}%
\pgfusepath{stroke,fill}%
}%
\begin{pgfscope}%
\pgfsys@transformshift{1.544729in}{0.440000in}%
\pgfsys@useobject{currentmarker}{}%
\end{pgfscope}%
\end{pgfscope}%
\begin{pgfscope}%
\pgfsetbuttcap%
\pgfsetroundjoin%
\definecolor{currentfill}{rgb}{0.000000,0.000000,0.000000}%
\pgfsetfillcolor{currentfill}%
\pgfsetlinewidth{0.602250pt}%
\definecolor{currentstroke}{rgb}{0.000000,0.000000,0.000000}%
\pgfsetstrokecolor{currentstroke}%
\pgfsetdash{}{0pt}%
\pgfsys@defobject{currentmarker}{\pgfqpoint{0.000000in}{-0.027778in}}{\pgfqpoint{0.000000in}{0.000000in}}{%
\pgfpathmoveto{\pgfqpoint{0.000000in}{0.000000in}}%
\pgfpathlineto{\pgfqpoint{0.000000in}{-0.027778in}}%
\pgfusepath{stroke,fill}%
}%
\begin{pgfscope}%
\pgfsys@transformshift{1.628965in}{0.440000in}%
\pgfsys@useobject{currentmarker}{}%
\end{pgfscope}%
\end{pgfscope}%
\begin{pgfscope}%
\pgfsetbuttcap%
\pgfsetroundjoin%
\definecolor{currentfill}{rgb}{0.000000,0.000000,0.000000}%
\pgfsetfillcolor{currentfill}%
\pgfsetlinewidth{0.602250pt}%
\definecolor{currentstroke}{rgb}{0.000000,0.000000,0.000000}%
\pgfsetstrokecolor{currentstroke}%
\pgfsetdash{}{0pt}%
\pgfsys@defobject{currentmarker}{\pgfqpoint{0.000000in}{-0.027778in}}{\pgfqpoint{0.000000in}{0.000000in}}{%
\pgfpathmoveto{\pgfqpoint{0.000000in}{0.000000in}}%
\pgfpathlineto{\pgfqpoint{0.000000in}{-0.027778in}}%
\pgfusepath{stroke,fill}%
}%
\begin{pgfscope}%
\pgfsys@transformshift{1.700186in}{0.440000in}%
\pgfsys@useobject{currentmarker}{}%
\end{pgfscope}%
\end{pgfscope}%
\begin{pgfscope}%
\pgfsetbuttcap%
\pgfsetroundjoin%
\definecolor{currentfill}{rgb}{0.000000,0.000000,0.000000}%
\pgfsetfillcolor{currentfill}%
\pgfsetlinewidth{0.602250pt}%
\definecolor{currentstroke}{rgb}{0.000000,0.000000,0.000000}%
\pgfsetstrokecolor{currentstroke}%
\pgfsetdash{}{0pt}%
\pgfsys@defobject{currentmarker}{\pgfqpoint{0.000000in}{-0.027778in}}{\pgfqpoint{0.000000in}{0.000000in}}{%
\pgfpathmoveto{\pgfqpoint{0.000000in}{0.000000in}}%
\pgfpathlineto{\pgfqpoint{0.000000in}{-0.027778in}}%
\pgfusepath{stroke,fill}%
}%
\begin{pgfscope}%
\pgfsys@transformshift{1.761880in}{0.440000in}%
\pgfsys@useobject{currentmarker}{}%
\end{pgfscope}%
\end{pgfscope}%
\begin{pgfscope}%
\pgfsetbuttcap%
\pgfsetroundjoin%
\definecolor{currentfill}{rgb}{0.000000,0.000000,0.000000}%
\pgfsetfillcolor{currentfill}%
\pgfsetlinewidth{0.602250pt}%
\definecolor{currentstroke}{rgb}{0.000000,0.000000,0.000000}%
\pgfsetstrokecolor{currentstroke}%
\pgfsetdash{}{0pt}%
\pgfsys@defobject{currentmarker}{\pgfqpoint{0.000000in}{-0.027778in}}{\pgfqpoint{0.000000in}{0.000000in}}{%
\pgfpathmoveto{\pgfqpoint{0.000000in}{0.000000in}}%
\pgfpathlineto{\pgfqpoint{0.000000in}{-0.027778in}}%
\pgfusepath{stroke,fill}%
}%
\begin{pgfscope}%
\pgfsys@transformshift{1.816298in}{0.440000in}%
\pgfsys@useobject{currentmarker}{}%
\end{pgfscope}%
\end{pgfscope}%
\begin{pgfscope}%
\pgfsetbuttcap%
\pgfsetroundjoin%
\definecolor{currentfill}{rgb}{0.000000,0.000000,0.000000}%
\pgfsetfillcolor{currentfill}%
\pgfsetlinewidth{0.602250pt}%
\definecolor{currentstroke}{rgb}{0.000000,0.000000,0.000000}%
\pgfsetstrokecolor{currentstroke}%
\pgfsetdash{}{0pt}%
\pgfsys@defobject{currentmarker}{\pgfqpoint{0.000000in}{-0.027778in}}{\pgfqpoint{0.000000in}{0.000000in}}{%
\pgfpathmoveto{\pgfqpoint{0.000000in}{0.000000in}}%
\pgfpathlineto{\pgfqpoint{0.000000in}{-0.027778in}}%
\pgfusepath{stroke,fill}%
}%
\begin{pgfscope}%
\pgfsys@transformshift{2.185225in}{0.440000in}%
\pgfsys@useobject{currentmarker}{}%
\end{pgfscope}%
\end{pgfscope}%
\begin{pgfscope}%
\pgfsetbuttcap%
\pgfsetroundjoin%
\definecolor{currentfill}{rgb}{0.000000,0.000000,0.000000}%
\pgfsetfillcolor{currentfill}%
\pgfsetlinewidth{0.602250pt}%
\definecolor{currentstroke}{rgb}{0.000000,0.000000,0.000000}%
\pgfsetstrokecolor{currentstroke}%
\pgfsetdash{}{0pt}%
\pgfsys@defobject{currentmarker}{\pgfqpoint{0.000000in}{-0.027778in}}{\pgfqpoint{0.000000in}{0.000000in}}{%
\pgfpathmoveto{\pgfqpoint{0.000000in}{0.000000in}}%
\pgfpathlineto{\pgfqpoint{0.000000in}{-0.027778in}}%
\pgfusepath{stroke,fill}%
}%
\begin{pgfscope}%
\pgfsys@transformshift{2.372558in}{0.440000in}%
\pgfsys@useobject{currentmarker}{}%
\end{pgfscope}%
\end{pgfscope}%
\begin{pgfscope}%
\pgfsetbuttcap%
\pgfsetroundjoin%
\definecolor{currentfill}{rgb}{0.000000,0.000000,0.000000}%
\pgfsetfillcolor{currentfill}%
\pgfsetlinewidth{0.602250pt}%
\definecolor{currentstroke}{rgb}{0.000000,0.000000,0.000000}%
\pgfsetstrokecolor{currentstroke}%
\pgfsetdash{}{0pt}%
\pgfsys@defobject{currentmarker}{\pgfqpoint{0.000000in}{-0.027778in}}{\pgfqpoint{0.000000in}{0.000000in}}{%
\pgfpathmoveto{\pgfqpoint{0.000000in}{0.000000in}}%
\pgfpathlineto{\pgfqpoint{0.000000in}{-0.027778in}}%
\pgfusepath{stroke,fill}%
}%
\begin{pgfscope}%
\pgfsys@transformshift{2.505473in}{0.440000in}%
\pgfsys@useobject{currentmarker}{}%
\end{pgfscope}%
\end{pgfscope}%
\begin{pgfscope}%
\pgfsetbuttcap%
\pgfsetroundjoin%
\definecolor{currentfill}{rgb}{0.000000,0.000000,0.000000}%
\pgfsetfillcolor{currentfill}%
\pgfsetlinewidth{0.602250pt}%
\definecolor{currentstroke}{rgb}{0.000000,0.000000,0.000000}%
\pgfsetstrokecolor{currentstroke}%
\pgfsetdash{}{0pt}%
\pgfsys@defobject{currentmarker}{\pgfqpoint{0.000000in}{-0.027778in}}{\pgfqpoint{0.000000in}{0.000000in}}{%
\pgfpathmoveto{\pgfqpoint{0.000000in}{0.000000in}}%
\pgfpathlineto{\pgfqpoint{0.000000in}{-0.027778in}}%
\pgfusepath{stroke,fill}%
}%
\begin{pgfscope}%
\pgfsys@transformshift{2.608570in}{0.440000in}%
\pgfsys@useobject{currentmarker}{}%
\end{pgfscope}%
\end{pgfscope}%
\begin{pgfscope}%
\pgfsetbuttcap%
\pgfsetroundjoin%
\definecolor{currentfill}{rgb}{0.000000,0.000000,0.000000}%
\pgfsetfillcolor{currentfill}%
\pgfsetlinewidth{0.602250pt}%
\definecolor{currentstroke}{rgb}{0.000000,0.000000,0.000000}%
\pgfsetstrokecolor{currentstroke}%
\pgfsetdash{}{0pt}%
\pgfsys@defobject{currentmarker}{\pgfqpoint{0.000000in}{-0.027778in}}{\pgfqpoint{0.000000in}{0.000000in}}{%
\pgfpathmoveto{\pgfqpoint{0.000000in}{0.000000in}}%
\pgfpathlineto{\pgfqpoint{0.000000in}{-0.027778in}}%
\pgfusepath{stroke,fill}%
}%
\begin{pgfscope}%
\pgfsys@transformshift{2.692806in}{0.440000in}%
\pgfsys@useobject{currentmarker}{}%
\end{pgfscope}%
\end{pgfscope}%
\begin{pgfscope}%
\pgfsetbuttcap%
\pgfsetroundjoin%
\definecolor{currentfill}{rgb}{0.000000,0.000000,0.000000}%
\pgfsetfillcolor{currentfill}%
\pgfsetlinewidth{0.602250pt}%
\definecolor{currentstroke}{rgb}{0.000000,0.000000,0.000000}%
\pgfsetstrokecolor{currentstroke}%
\pgfsetdash{}{0pt}%
\pgfsys@defobject{currentmarker}{\pgfqpoint{0.000000in}{-0.027778in}}{\pgfqpoint{0.000000in}{0.000000in}}{%
\pgfpathmoveto{\pgfqpoint{0.000000in}{0.000000in}}%
\pgfpathlineto{\pgfqpoint{0.000000in}{-0.027778in}}%
\pgfusepath{stroke,fill}%
}%
\begin{pgfscope}%
\pgfsys@transformshift{2.764027in}{0.440000in}%
\pgfsys@useobject{currentmarker}{}%
\end{pgfscope}%
\end{pgfscope}%
\begin{pgfscope}%
\pgfsetbuttcap%
\pgfsetroundjoin%
\definecolor{currentfill}{rgb}{0.000000,0.000000,0.000000}%
\pgfsetfillcolor{currentfill}%
\pgfsetlinewidth{0.602250pt}%
\definecolor{currentstroke}{rgb}{0.000000,0.000000,0.000000}%
\pgfsetstrokecolor{currentstroke}%
\pgfsetdash{}{0pt}%
\pgfsys@defobject{currentmarker}{\pgfqpoint{0.000000in}{-0.027778in}}{\pgfqpoint{0.000000in}{0.000000in}}{%
\pgfpathmoveto{\pgfqpoint{0.000000in}{0.000000in}}%
\pgfpathlineto{\pgfqpoint{0.000000in}{-0.027778in}}%
\pgfusepath{stroke,fill}%
}%
\begin{pgfscope}%
\pgfsys@transformshift{2.825721in}{0.440000in}%
\pgfsys@useobject{currentmarker}{}%
\end{pgfscope}%
\end{pgfscope}%
\begin{pgfscope}%
\pgfsetbuttcap%
\pgfsetroundjoin%
\definecolor{currentfill}{rgb}{0.000000,0.000000,0.000000}%
\pgfsetfillcolor{currentfill}%
\pgfsetlinewidth{0.602250pt}%
\definecolor{currentstroke}{rgb}{0.000000,0.000000,0.000000}%
\pgfsetstrokecolor{currentstroke}%
\pgfsetdash{}{0pt}%
\pgfsys@defobject{currentmarker}{\pgfqpoint{0.000000in}{-0.027778in}}{\pgfqpoint{0.000000in}{0.000000in}}{%
\pgfpathmoveto{\pgfqpoint{0.000000in}{0.000000in}}%
\pgfpathlineto{\pgfqpoint{0.000000in}{-0.027778in}}%
\pgfusepath{stroke,fill}%
}%
\begin{pgfscope}%
\pgfsys@transformshift{2.880139in}{0.440000in}%
\pgfsys@useobject{currentmarker}{}%
\end{pgfscope}%
\end{pgfscope}%
\begin{pgfscope}%
\pgfsetbuttcap%
\pgfsetroundjoin%
\definecolor{currentfill}{rgb}{0.000000,0.000000,0.000000}%
\pgfsetfillcolor{currentfill}%
\pgfsetlinewidth{0.602250pt}%
\definecolor{currentstroke}{rgb}{0.000000,0.000000,0.000000}%
\pgfsetstrokecolor{currentstroke}%
\pgfsetdash{}{0pt}%
\pgfsys@defobject{currentmarker}{\pgfqpoint{0.000000in}{-0.027778in}}{\pgfqpoint{0.000000in}{0.000000in}}{%
\pgfpathmoveto{\pgfqpoint{0.000000in}{0.000000in}}%
\pgfpathlineto{\pgfqpoint{0.000000in}{-0.027778in}}%
\pgfusepath{stroke,fill}%
}%
\begin{pgfscope}%
\pgfsys@transformshift{3.249066in}{0.440000in}%
\pgfsys@useobject{currentmarker}{}%
\end{pgfscope}%
\end{pgfscope}%
\begin{pgfscope}%
\pgfsetbuttcap%
\pgfsetroundjoin%
\definecolor{currentfill}{rgb}{0.000000,0.000000,0.000000}%
\pgfsetfillcolor{currentfill}%
\pgfsetlinewidth{0.602250pt}%
\definecolor{currentstroke}{rgb}{0.000000,0.000000,0.000000}%
\pgfsetstrokecolor{currentstroke}%
\pgfsetdash{}{0pt}%
\pgfsys@defobject{currentmarker}{\pgfqpoint{0.000000in}{-0.027778in}}{\pgfqpoint{0.000000in}{0.000000in}}{%
\pgfpathmoveto{\pgfqpoint{0.000000in}{0.000000in}}%
\pgfpathlineto{\pgfqpoint{0.000000in}{-0.027778in}}%
\pgfusepath{stroke,fill}%
}%
\begin{pgfscope}%
\pgfsys@transformshift{3.436399in}{0.440000in}%
\pgfsys@useobject{currentmarker}{}%
\end{pgfscope}%
\end{pgfscope}%
\begin{pgfscope}%
\pgfsetbuttcap%
\pgfsetroundjoin%
\definecolor{currentfill}{rgb}{0.000000,0.000000,0.000000}%
\pgfsetfillcolor{currentfill}%
\pgfsetlinewidth{0.602250pt}%
\definecolor{currentstroke}{rgb}{0.000000,0.000000,0.000000}%
\pgfsetstrokecolor{currentstroke}%
\pgfsetdash{}{0pt}%
\pgfsys@defobject{currentmarker}{\pgfqpoint{0.000000in}{-0.027778in}}{\pgfqpoint{0.000000in}{0.000000in}}{%
\pgfpathmoveto{\pgfqpoint{0.000000in}{0.000000in}}%
\pgfpathlineto{\pgfqpoint{0.000000in}{-0.027778in}}%
\pgfusepath{stroke,fill}%
}%
\begin{pgfscope}%
\pgfsys@transformshift{3.569313in}{0.440000in}%
\pgfsys@useobject{currentmarker}{}%
\end{pgfscope}%
\end{pgfscope}%
\begin{pgfscope}%
\pgfsetbuttcap%
\pgfsetroundjoin%
\definecolor{currentfill}{rgb}{0.000000,0.000000,0.000000}%
\pgfsetfillcolor{currentfill}%
\pgfsetlinewidth{0.602250pt}%
\definecolor{currentstroke}{rgb}{0.000000,0.000000,0.000000}%
\pgfsetstrokecolor{currentstroke}%
\pgfsetdash{}{0pt}%
\pgfsys@defobject{currentmarker}{\pgfqpoint{0.000000in}{-0.027778in}}{\pgfqpoint{0.000000in}{0.000000in}}{%
\pgfpathmoveto{\pgfqpoint{0.000000in}{0.000000in}}%
\pgfpathlineto{\pgfqpoint{0.000000in}{-0.027778in}}%
\pgfusepath{stroke,fill}%
}%
\begin{pgfscope}%
\pgfsys@transformshift{3.672410in}{0.440000in}%
\pgfsys@useobject{currentmarker}{}%
\end{pgfscope}%
\end{pgfscope}%
\begin{pgfscope}%
\pgfsetbuttcap%
\pgfsetroundjoin%
\definecolor{currentfill}{rgb}{0.000000,0.000000,0.000000}%
\pgfsetfillcolor{currentfill}%
\pgfsetlinewidth{0.602250pt}%
\definecolor{currentstroke}{rgb}{0.000000,0.000000,0.000000}%
\pgfsetstrokecolor{currentstroke}%
\pgfsetdash{}{0pt}%
\pgfsys@defobject{currentmarker}{\pgfqpoint{0.000000in}{-0.027778in}}{\pgfqpoint{0.000000in}{0.000000in}}{%
\pgfpathmoveto{\pgfqpoint{0.000000in}{0.000000in}}%
\pgfpathlineto{\pgfqpoint{0.000000in}{-0.027778in}}%
\pgfusepath{stroke,fill}%
}%
\begin{pgfscope}%
\pgfsys@transformshift{3.756646in}{0.440000in}%
\pgfsys@useobject{currentmarker}{}%
\end{pgfscope}%
\end{pgfscope}%
\begin{pgfscope}%
\pgfsetbuttcap%
\pgfsetroundjoin%
\definecolor{currentfill}{rgb}{0.000000,0.000000,0.000000}%
\pgfsetfillcolor{currentfill}%
\pgfsetlinewidth{0.602250pt}%
\definecolor{currentstroke}{rgb}{0.000000,0.000000,0.000000}%
\pgfsetstrokecolor{currentstroke}%
\pgfsetdash{}{0pt}%
\pgfsys@defobject{currentmarker}{\pgfqpoint{0.000000in}{-0.027778in}}{\pgfqpoint{0.000000in}{0.000000in}}{%
\pgfpathmoveto{\pgfqpoint{0.000000in}{0.000000in}}%
\pgfpathlineto{\pgfqpoint{0.000000in}{-0.027778in}}%
\pgfusepath{stroke,fill}%
}%
\begin{pgfscope}%
\pgfsys@transformshift{3.827867in}{0.440000in}%
\pgfsys@useobject{currentmarker}{}%
\end{pgfscope}%
\end{pgfscope}%
\begin{pgfscope}%
\pgfsetbuttcap%
\pgfsetroundjoin%
\definecolor{currentfill}{rgb}{0.000000,0.000000,0.000000}%
\pgfsetfillcolor{currentfill}%
\pgfsetlinewidth{0.602250pt}%
\definecolor{currentstroke}{rgb}{0.000000,0.000000,0.000000}%
\pgfsetstrokecolor{currentstroke}%
\pgfsetdash{}{0pt}%
\pgfsys@defobject{currentmarker}{\pgfqpoint{0.000000in}{-0.027778in}}{\pgfqpoint{0.000000in}{0.000000in}}{%
\pgfpathmoveto{\pgfqpoint{0.000000in}{0.000000in}}%
\pgfpathlineto{\pgfqpoint{0.000000in}{-0.027778in}}%
\pgfusepath{stroke,fill}%
}%
\begin{pgfscope}%
\pgfsys@transformshift{3.889561in}{0.440000in}%
\pgfsys@useobject{currentmarker}{}%
\end{pgfscope}%
\end{pgfscope}%
\begin{pgfscope}%
\pgfsetbuttcap%
\pgfsetroundjoin%
\definecolor{currentfill}{rgb}{0.000000,0.000000,0.000000}%
\pgfsetfillcolor{currentfill}%
\pgfsetlinewidth{0.602250pt}%
\definecolor{currentstroke}{rgb}{0.000000,0.000000,0.000000}%
\pgfsetstrokecolor{currentstroke}%
\pgfsetdash{}{0pt}%
\pgfsys@defobject{currentmarker}{\pgfqpoint{0.000000in}{-0.027778in}}{\pgfqpoint{0.000000in}{0.000000in}}{%
\pgfpathmoveto{\pgfqpoint{0.000000in}{0.000000in}}%
\pgfpathlineto{\pgfqpoint{0.000000in}{-0.027778in}}%
\pgfusepath{stroke,fill}%
}%
\begin{pgfscope}%
\pgfsys@transformshift{3.943980in}{0.440000in}%
\pgfsys@useobject{currentmarker}{}%
\end{pgfscope}%
\end{pgfscope}%
\begin{pgfscope}%
\pgfsetbuttcap%
\pgfsetroundjoin%
\definecolor{currentfill}{rgb}{0.000000,0.000000,0.000000}%
\pgfsetfillcolor{currentfill}%
\pgfsetlinewidth{0.602250pt}%
\definecolor{currentstroke}{rgb}{0.000000,0.000000,0.000000}%
\pgfsetstrokecolor{currentstroke}%
\pgfsetdash{}{0pt}%
\pgfsys@defobject{currentmarker}{\pgfqpoint{0.000000in}{-0.027778in}}{\pgfqpoint{0.000000in}{0.000000in}}{%
\pgfpathmoveto{\pgfqpoint{0.000000in}{0.000000in}}%
\pgfpathlineto{\pgfqpoint{0.000000in}{-0.027778in}}%
\pgfusepath{stroke,fill}%
}%
\begin{pgfscope}%
\pgfsys@transformshift{4.312906in}{0.440000in}%
\pgfsys@useobject{currentmarker}{}%
\end{pgfscope}%
\end{pgfscope}%
\begin{pgfscope}%
\pgfsetbuttcap%
\pgfsetroundjoin%
\definecolor{currentfill}{rgb}{0.000000,0.000000,0.000000}%
\pgfsetfillcolor{currentfill}%
\pgfsetlinewidth{0.602250pt}%
\definecolor{currentstroke}{rgb}{0.000000,0.000000,0.000000}%
\pgfsetstrokecolor{currentstroke}%
\pgfsetdash{}{0pt}%
\pgfsys@defobject{currentmarker}{\pgfqpoint{0.000000in}{-0.027778in}}{\pgfqpoint{0.000000in}{0.000000in}}{%
\pgfpathmoveto{\pgfqpoint{0.000000in}{0.000000in}}%
\pgfpathlineto{\pgfqpoint{0.000000in}{-0.027778in}}%
\pgfusepath{stroke,fill}%
}%
\begin{pgfscope}%
\pgfsys@transformshift{4.500239in}{0.440000in}%
\pgfsys@useobject{currentmarker}{}%
\end{pgfscope}%
\end{pgfscope}%
\begin{pgfscope}%
\definecolor{textcolor}{rgb}{0.000000,0.000000,0.000000}%
\pgfsetstrokecolor{textcolor}%
\pgfsetfillcolor{textcolor}%
\pgftext[x=2.562500in,y=0.152809in,,top]{\color{textcolor}\sffamily\fontsize{10.000000}{12.000000}\selectfont \(\displaystyle N_\mathrm{QMC}\)}%
\end{pgfscope}%
\begin{pgfscope}%
\pgfsetbuttcap%
\pgfsetroundjoin%
\definecolor{currentfill}{rgb}{0.000000,0.000000,0.000000}%
\pgfsetfillcolor{currentfill}%
\pgfsetlinewidth{0.803000pt}%
\definecolor{currentstroke}{rgb}{0.000000,0.000000,0.000000}%
\pgfsetstrokecolor{currentstroke}%
\pgfsetdash{}{0pt}%
\pgfsys@defobject{currentmarker}{\pgfqpoint{-0.048611in}{0.000000in}}{\pgfqpoint{0.000000in}{0.000000in}}{%
\pgfpathmoveto{\pgfqpoint{0.000000in}{0.000000in}}%
\pgfpathlineto{\pgfqpoint{-0.048611in}{0.000000in}}%
\pgfusepath{stroke,fill}%
}%
\begin{pgfscope}%
\pgfsys@transformshift{0.625000in}{0.980476in}%
\pgfsys@useobject{currentmarker}{}%
\end{pgfscope}%
\end{pgfscope}%
\begin{pgfscope}%
\definecolor{textcolor}{rgb}{0.000000,0.000000,0.000000}%
\pgfsetstrokecolor{textcolor}%
\pgfsetfillcolor{textcolor}%
\pgftext[x=0.239775in,y=0.927715in,left,base]{\color{textcolor}\sffamily\fontsize{10.000000}{12.000000}\selectfont \(\displaystyle {10^{-3}}\)}%
\end{pgfscope}%
\begin{pgfscope}%
\pgfsetbuttcap%
\pgfsetroundjoin%
\definecolor{currentfill}{rgb}{0.000000,0.000000,0.000000}%
\pgfsetfillcolor{currentfill}%
\pgfsetlinewidth{0.803000pt}%
\definecolor{currentstroke}{rgb}{0.000000,0.000000,0.000000}%
\pgfsetstrokecolor{currentstroke}%
\pgfsetdash{}{0pt}%
\pgfsys@defobject{currentmarker}{\pgfqpoint{-0.048611in}{0.000000in}}{\pgfqpoint{0.000000in}{0.000000in}}{%
\pgfpathmoveto{\pgfqpoint{0.000000in}{0.000000in}}%
\pgfpathlineto{\pgfqpoint{-0.048611in}{0.000000in}}%
\pgfusepath{stroke,fill}%
}%
\begin{pgfscope}%
\pgfsys@transformshift{0.625000in}{1.786727in}%
\pgfsys@useobject{currentmarker}{}%
\end{pgfscope}%
\end{pgfscope}%
\begin{pgfscope}%
\definecolor{textcolor}{rgb}{0.000000,0.000000,0.000000}%
\pgfsetstrokecolor{textcolor}%
\pgfsetfillcolor{textcolor}%
\pgftext[x=0.239775in,y=1.733965in,left,base]{\color{textcolor}\sffamily\fontsize{10.000000}{12.000000}\selectfont \(\displaystyle {10^{-2}}\)}%
\end{pgfscope}%
\begin{pgfscope}%
\pgfsetbuttcap%
\pgfsetroundjoin%
\definecolor{currentfill}{rgb}{0.000000,0.000000,0.000000}%
\pgfsetfillcolor{currentfill}%
\pgfsetlinewidth{0.803000pt}%
\definecolor{currentstroke}{rgb}{0.000000,0.000000,0.000000}%
\pgfsetstrokecolor{currentstroke}%
\pgfsetdash{}{0pt}%
\pgfsys@defobject{currentmarker}{\pgfqpoint{-0.048611in}{0.000000in}}{\pgfqpoint{0.000000in}{0.000000in}}{%
\pgfpathmoveto{\pgfqpoint{0.000000in}{0.000000in}}%
\pgfpathlineto{\pgfqpoint{-0.048611in}{0.000000in}}%
\pgfusepath{stroke,fill}%
}%
\begin{pgfscope}%
\pgfsys@transformshift{0.625000in}{2.592977in}%
\pgfsys@useobject{currentmarker}{}%
\end{pgfscope}%
\end{pgfscope}%
\begin{pgfscope}%
\definecolor{textcolor}{rgb}{0.000000,0.000000,0.000000}%
\pgfsetstrokecolor{textcolor}%
\pgfsetfillcolor{textcolor}%
\pgftext[x=0.239775in,y=2.540215in,left,base]{\color{textcolor}\sffamily\fontsize{10.000000}{12.000000}\selectfont \(\displaystyle {10^{-1}}\)}%
\end{pgfscope}%
\begin{pgfscope}%
\pgfsetbuttcap%
\pgfsetroundjoin%
\definecolor{currentfill}{rgb}{0.000000,0.000000,0.000000}%
\pgfsetfillcolor{currentfill}%
\pgfsetlinewidth{0.803000pt}%
\definecolor{currentstroke}{rgb}{0.000000,0.000000,0.000000}%
\pgfsetstrokecolor{currentstroke}%
\pgfsetdash{}{0pt}%
\pgfsys@defobject{currentmarker}{\pgfqpoint{-0.048611in}{0.000000in}}{\pgfqpoint{0.000000in}{0.000000in}}{%
\pgfpathmoveto{\pgfqpoint{0.000000in}{0.000000in}}%
\pgfpathlineto{\pgfqpoint{-0.048611in}{0.000000in}}%
\pgfusepath{stroke,fill}%
}%
\begin{pgfscope}%
\pgfsys@transformshift{0.625000in}{3.399227in}%
\pgfsys@useobject{currentmarker}{}%
\end{pgfscope}%
\end{pgfscope}%
\begin{pgfscope}%
\definecolor{textcolor}{rgb}{0.000000,0.000000,0.000000}%
\pgfsetstrokecolor{textcolor}%
\pgfsetfillcolor{textcolor}%
\pgftext[x=0.326581in,y=3.346466in,left,base]{\color{textcolor}\sffamily\fontsize{10.000000}{12.000000}\selectfont \(\displaystyle {10^{0}}\)}%
\end{pgfscope}%
\begin{pgfscope}%
\pgfsetbuttcap%
\pgfsetroundjoin%
\definecolor{currentfill}{rgb}{0.000000,0.000000,0.000000}%
\pgfsetfillcolor{currentfill}%
\pgfsetlinewidth{0.602250pt}%
\definecolor{currentstroke}{rgb}{0.000000,0.000000,0.000000}%
\pgfsetstrokecolor{currentstroke}%
\pgfsetdash{}{0pt}%
\pgfsys@defobject{currentmarker}{\pgfqpoint{-0.027778in}{0.000000in}}{\pgfqpoint{0.000000in}{0.000000in}}{%
\pgfpathmoveto{\pgfqpoint{0.000000in}{0.000000in}}%
\pgfpathlineto{\pgfqpoint{-0.027778in}{0.000000in}}%
\pgfusepath{stroke,fill}%
}%
\begin{pgfscope}%
\pgfsys@transformshift{0.625000in}{0.558905in}%
\pgfsys@useobject{currentmarker}{}%
\end{pgfscope}%
\end{pgfscope}%
\begin{pgfscope}%
\pgfsetbuttcap%
\pgfsetroundjoin%
\definecolor{currentfill}{rgb}{0.000000,0.000000,0.000000}%
\pgfsetfillcolor{currentfill}%
\pgfsetlinewidth{0.602250pt}%
\definecolor{currentstroke}{rgb}{0.000000,0.000000,0.000000}%
\pgfsetstrokecolor{currentstroke}%
\pgfsetdash{}{0pt}%
\pgfsys@defobject{currentmarker}{\pgfqpoint{-0.027778in}{0.000000in}}{\pgfqpoint{0.000000in}{0.000000in}}{%
\pgfpathmoveto{\pgfqpoint{0.000000in}{0.000000in}}%
\pgfpathlineto{\pgfqpoint{-0.027778in}{0.000000in}}%
\pgfusepath{stroke,fill}%
}%
\begin{pgfscope}%
\pgfsys@transformshift{0.625000in}{0.659637in}%
\pgfsys@useobject{currentmarker}{}%
\end{pgfscope}%
\end{pgfscope}%
\begin{pgfscope}%
\pgfsetbuttcap%
\pgfsetroundjoin%
\definecolor{currentfill}{rgb}{0.000000,0.000000,0.000000}%
\pgfsetfillcolor{currentfill}%
\pgfsetlinewidth{0.602250pt}%
\definecolor{currentstroke}{rgb}{0.000000,0.000000,0.000000}%
\pgfsetstrokecolor{currentstroke}%
\pgfsetdash{}{0pt}%
\pgfsys@defobject{currentmarker}{\pgfqpoint{-0.027778in}{0.000000in}}{\pgfqpoint{0.000000in}{0.000000in}}{%
\pgfpathmoveto{\pgfqpoint{0.000000in}{0.000000in}}%
\pgfpathlineto{\pgfqpoint{-0.027778in}{0.000000in}}%
\pgfusepath{stroke,fill}%
}%
\begin{pgfscope}%
\pgfsys@transformshift{0.625000in}{0.737771in}%
\pgfsys@useobject{currentmarker}{}%
\end{pgfscope}%
\end{pgfscope}%
\begin{pgfscope}%
\pgfsetbuttcap%
\pgfsetroundjoin%
\definecolor{currentfill}{rgb}{0.000000,0.000000,0.000000}%
\pgfsetfillcolor{currentfill}%
\pgfsetlinewidth{0.602250pt}%
\definecolor{currentstroke}{rgb}{0.000000,0.000000,0.000000}%
\pgfsetstrokecolor{currentstroke}%
\pgfsetdash{}{0pt}%
\pgfsys@defobject{currentmarker}{\pgfqpoint{-0.027778in}{0.000000in}}{\pgfqpoint{0.000000in}{0.000000in}}{%
\pgfpathmoveto{\pgfqpoint{0.000000in}{0.000000in}}%
\pgfpathlineto{\pgfqpoint{-0.027778in}{0.000000in}}%
\pgfusepath{stroke,fill}%
}%
\begin{pgfscope}%
\pgfsys@transformshift{0.625000in}{0.801611in}%
\pgfsys@useobject{currentmarker}{}%
\end{pgfscope}%
\end{pgfscope}%
\begin{pgfscope}%
\pgfsetbuttcap%
\pgfsetroundjoin%
\definecolor{currentfill}{rgb}{0.000000,0.000000,0.000000}%
\pgfsetfillcolor{currentfill}%
\pgfsetlinewidth{0.602250pt}%
\definecolor{currentstroke}{rgb}{0.000000,0.000000,0.000000}%
\pgfsetstrokecolor{currentstroke}%
\pgfsetdash{}{0pt}%
\pgfsys@defobject{currentmarker}{\pgfqpoint{-0.027778in}{0.000000in}}{\pgfqpoint{0.000000in}{0.000000in}}{%
\pgfpathmoveto{\pgfqpoint{0.000000in}{0.000000in}}%
\pgfpathlineto{\pgfqpoint{-0.027778in}{0.000000in}}%
\pgfusepath{stroke,fill}%
}%
\begin{pgfscope}%
\pgfsys@transformshift{0.625000in}{0.855586in}%
\pgfsys@useobject{currentmarker}{}%
\end{pgfscope}%
\end{pgfscope}%
\begin{pgfscope}%
\pgfsetbuttcap%
\pgfsetroundjoin%
\definecolor{currentfill}{rgb}{0.000000,0.000000,0.000000}%
\pgfsetfillcolor{currentfill}%
\pgfsetlinewidth{0.602250pt}%
\definecolor{currentstroke}{rgb}{0.000000,0.000000,0.000000}%
\pgfsetstrokecolor{currentstroke}%
\pgfsetdash{}{0pt}%
\pgfsys@defobject{currentmarker}{\pgfqpoint{-0.027778in}{0.000000in}}{\pgfqpoint{0.000000in}{0.000000in}}{%
\pgfpathmoveto{\pgfqpoint{0.000000in}{0.000000in}}%
\pgfpathlineto{\pgfqpoint{-0.027778in}{0.000000in}}%
\pgfusepath{stroke,fill}%
}%
\begin{pgfscope}%
\pgfsys@transformshift{0.625000in}{0.902342in}%
\pgfsys@useobject{currentmarker}{}%
\end{pgfscope}%
\end{pgfscope}%
\begin{pgfscope}%
\pgfsetbuttcap%
\pgfsetroundjoin%
\definecolor{currentfill}{rgb}{0.000000,0.000000,0.000000}%
\pgfsetfillcolor{currentfill}%
\pgfsetlinewidth{0.602250pt}%
\definecolor{currentstroke}{rgb}{0.000000,0.000000,0.000000}%
\pgfsetstrokecolor{currentstroke}%
\pgfsetdash{}{0pt}%
\pgfsys@defobject{currentmarker}{\pgfqpoint{-0.027778in}{0.000000in}}{\pgfqpoint{0.000000in}{0.000000in}}{%
\pgfpathmoveto{\pgfqpoint{0.000000in}{0.000000in}}%
\pgfpathlineto{\pgfqpoint{-0.027778in}{0.000000in}}%
\pgfusepath{stroke,fill}%
}%
\begin{pgfscope}%
\pgfsys@transformshift{0.625000in}{0.943584in}%
\pgfsys@useobject{currentmarker}{}%
\end{pgfscope}%
\end{pgfscope}%
\begin{pgfscope}%
\pgfsetbuttcap%
\pgfsetroundjoin%
\definecolor{currentfill}{rgb}{0.000000,0.000000,0.000000}%
\pgfsetfillcolor{currentfill}%
\pgfsetlinewidth{0.602250pt}%
\definecolor{currentstroke}{rgb}{0.000000,0.000000,0.000000}%
\pgfsetstrokecolor{currentstroke}%
\pgfsetdash{}{0pt}%
\pgfsys@defobject{currentmarker}{\pgfqpoint{-0.027778in}{0.000000in}}{\pgfqpoint{0.000000in}{0.000000in}}{%
\pgfpathmoveto{\pgfqpoint{0.000000in}{0.000000in}}%
\pgfpathlineto{\pgfqpoint{-0.027778in}{0.000000in}}%
\pgfusepath{stroke,fill}%
}%
\begin{pgfscope}%
\pgfsys@transformshift{0.625000in}{1.223182in}%
\pgfsys@useobject{currentmarker}{}%
\end{pgfscope}%
\end{pgfscope}%
\begin{pgfscope}%
\pgfsetbuttcap%
\pgfsetroundjoin%
\definecolor{currentfill}{rgb}{0.000000,0.000000,0.000000}%
\pgfsetfillcolor{currentfill}%
\pgfsetlinewidth{0.602250pt}%
\definecolor{currentstroke}{rgb}{0.000000,0.000000,0.000000}%
\pgfsetstrokecolor{currentstroke}%
\pgfsetdash{}{0pt}%
\pgfsys@defobject{currentmarker}{\pgfqpoint{-0.027778in}{0.000000in}}{\pgfqpoint{0.000000in}{0.000000in}}{%
\pgfpathmoveto{\pgfqpoint{0.000000in}{0.000000in}}%
\pgfpathlineto{\pgfqpoint{-0.027778in}{0.000000in}}%
\pgfusepath{stroke,fill}%
}%
\begin{pgfscope}%
\pgfsys@transformshift{0.625000in}{1.365155in}%
\pgfsys@useobject{currentmarker}{}%
\end{pgfscope}%
\end{pgfscope}%
\begin{pgfscope}%
\pgfsetbuttcap%
\pgfsetroundjoin%
\definecolor{currentfill}{rgb}{0.000000,0.000000,0.000000}%
\pgfsetfillcolor{currentfill}%
\pgfsetlinewidth{0.602250pt}%
\definecolor{currentstroke}{rgb}{0.000000,0.000000,0.000000}%
\pgfsetstrokecolor{currentstroke}%
\pgfsetdash{}{0pt}%
\pgfsys@defobject{currentmarker}{\pgfqpoint{-0.027778in}{0.000000in}}{\pgfqpoint{0.000000in}{0.000000in}}{%
\pgfpathmoveto{\pgfqpoint{0.000000in}{0.000000in}}%
\pgfpathlineto{\pgfqpoint{-0.027778in}{0.000000in}}%
\pgfusepath{stroke,fill}%
}%
\begin{pgfscope}%
\pgfsys@transformshift{0.625000in}{1.465887in}%
\pgfsys@useobject{currentmarker}{}%
\end{pgfscope}%
\end{pgfscope}%
\begin{pgfscope}%
\pgfsetbuttcap%
\pgfsetroundjoin%
\definecolor{currentfill}{rgb}{0.000000,0.000000,0.000000}%
\pgfsetfillcolor{currentfill}%
\pgfsetlinewidth{0.602250pt}%
\definecolor{currentstroke}{rgb}{0.000000,0.000000,0.000000}%
\pgfsetstrokecolor{currentstroke}%
\pgfsetdash{}{0pt}%
\pgfsys@defobject{currentmarker}{\pgfqpoint{-0.027778in}{0.000000in}}{\pgfqpoint{0.000000in}{0.000000in}}{%
\pgfpathmoveto{\pgfqpoint{0.000000in}{0.000000in}}%
\pgfpathlineto{\pgfqpoint{-0.027778in}{0.000000in}}%
\pgfusepath{stroke,fill}%
}%
\begin{pgfscope}%
\pgfsys@transformshift{0.625000in}{1.544021in}%
\pgfsys@useobject{currentmarker}{}%
\end{pgfscope}%
\end{pgfscope}%
\begin{pgfscope}%
\pgfsetbuttcap%
\pgfsetroundjoin%
\definecolor{currentfill}{rgb}{0.000000,0.000000,0.000000}%
\pgfsetfillcolor{currentfill}%
\pgfsetlinewidth{0.602250pt}%
\definecolor{currentstroke}{rgb}{0.000000,0.000000,0.000000}%
\pgfsetstrokecolor{currentstroke}%
\pgfsetdash{}{0pt}%
\pgfsys@defobject{currentmarker}{\pgfqpoint{-0.027778in}{0.000000in}}{\pgfqpoint{0.000000in}{0.000000in}}{%
\pgfpathmoveto{\pgfqpoint{0.000000in}{0.000000in}}%
\pgfpathlineto{\pgfqpoint{-0.027778in}{0.000000in}}%
\pgfusepath{stroke,fill}%
}%
\begin{pgfscope}%
\pgfsys@transformshift{0.625000in}{1.607861in}%
\pgfsys@useobject{currentmarker}{}%
\end{pgfscope}%
\end{pgfscope}%
\begin{pgfscope}%
\pgfsetbuttcap%
\pgfsetroundjoin%
\definecolor{currentfill}{rgb}{0.000000,0.000000,0.000000}%
\pgfsetfillcolor{currentfill}%
\pgfsetlinewidth{0.602250pt}%
\definecolor{currentstroke}{rgb}{0.000000,0.000000,0.000000}%
\pgfsetstrokecolor{currentstroke}%
\pgfsetdash{}{0pt}%
\pgfsys@defobject{currentmarker}{\pgfqpoint{-0.027778in}{0.000000in}}{\pgfqpoint{0.000000in}{0.000000in}}{%
\pgfpathmoveto{\pgfqpoint{0.000000in}{0.000000in}}%
\pgfpathlineto{\pgfqpoint{-0.027778in}{0.000000in}}%
\pgfusepath{stroke,fill}%
}%
\begin{pgfscope}%
\pgfsys@transformshift{0.625000in}{1.661837in}%
\pgfsys@useobject{currentmarker}{}%
\end{pgfscope}%
\end{pgfscope}%
\begin{pgfscope}%
\pgfsetbuttcap%
\pgfsetroundjoin%
\definecolor{currentfill}{rgb}{0.000000,0.000000,0.000000}%
\pgfsetfillcolor{currentfill}%
\pgfsetlinewidth{0.602250pt}%
\definecolor{currentstroke}{rgb}{0.000000,0.000000,0.000000}%
\pgfsetstrokecolor{currentstroke}%
\pgfsetdash{}{0pt}%
\pgfsys@defobject{currentmarker}{\pgfqpoint{-0.027778in}{0.000000in}}{\pgfqpoint{0.000000in}{0.000000in}}{%
\pgfpathmoveto{\pgfqpoint{0.000000in}{0.000000in}}%
\pgfpathlineto{\pgfqpoint{-0.027778in}{0.000000in}}%
\pgfusepath{stroke,fill}%
}%
\begin{pgfscope}%
\pgfsys@transformshift{0.625000in}{1.708593in}%
\pgfsys@useobject{currentmarker}{}%
\end{pgfscope}%
\end{pgfscope}%
\begin{pgfscope}%
\pgfsetbuttcap%
\pgfsetroundjoin%
\definecolor{currentfill}{rgb}{0.000000,0.000000,0.000000}%
\pgfsetfillcolor{currentfill}%
\pgfsetlinewidth{0.602250pt}%
\definecolor{currentstroke}{rgb}{0.000000,0.000000,0.000000}%
\pgfsetstrokecolor{currentstroke}%
\pgfsetdash{}{0pt}%
\pgfsys@defobject{currentmarker}{\pgfqpoint{-0.027778in}{0.000000in}}{\pgfqpoint{0.000000in}{0.000000in}}{%
\pgfpathmoveto{\pgfqpoint{0.000000in}{0.000000in}}%
\pgfpathlineto{\pgfqpoint{-0.027778in}{0.000000in}}%
\pgfusepath{stroke,fill}%
}%
\begin{pgfscope}%
\pgfsys@transformshift{0.625000in}{1.749835in}%
\pgfsys@useobject{currentmarker}{}%
\end{pgfscope}%
\end{pgfscope}%
\begin{pgfscope}%
\pgfsetbuttcap%
\pgfsetroundjoin%
\definecolor{currentfill}{rgb}{0.000000,0.000000,0.000000}%
\pgfsetfillcolor{currentfill}%
\pgfsetlinewidth{0.602250pt}%
\definecolor{currentstroke}{rgb}{0.000000,0.000000,0.000000}%
\pgfsetstrokecolor{currentstroke}%
\pgfsetdash{}{0pt}%
\pgfsys@defobject{currentmarker}{\pgfqpoint{-0.027778in}{0.000000in}}{\pgfqpoint{0.000000in}{0.000000in}}{%
\pgfpathmoveto{\pgfqpoint{0.000000in}{0.000000in}}%
\pgfpathlineto{\pgfqpoint{-0.027778in}{0.000000in}}%
\pgfusepath{stroke,fill}%
}%
\begin{pgfscope}%
\pgfsys@transformshift{0.625000in}{2.029432in}%
\pgfsys@useobject{currentmarker}{}%
\end{pgfscope}%
\end{pgfscope}%
\begin{pgfscope}%
\pgfsetbuttcap%
\pgfsetroundjoin%
\definecolor{currentfill}{rgb}{0.000000,0.000000,0.000000}%
\pgfsetfillcolor{currentfill}%
\pgfsetlinewidth{0.602250pt}%
\definecolor{currentstroke}{rgb}{0.000000,0.000000,0.000000}%
\pgfsetstrokecolor{currentstroke}%
\pgfsetdash{}{0pt}%
\pgfsys@defobject{currentmarker}{\pgfqpoint{-0.027778in}{0.000000in}}{\pgfqpoint{0.000000in}{0.000000in}}{%
\pgfpathmoveto{\pgfqpoint{0.000000in}{0.000000in}}%
\pgfpathlineto{\pgfqpoint{-0.027778in}{0.000000in}}%
\pgfusepath{stroke,fill}%
}%
\begin{pgfscope}%
\pgfsys@transformshift{0.625000in}{2.171406in}%
\pgfsys@useobject{currentmarker}{}%
\end{pgfscope}%
\end{pgfscope}%
\begin{pgfscope}%
\pgfsetbuttcap%
\pgfsetroundjoin%
\definecolor{currentfill}{rgb}{0.000000,0.000000,0.000000}%
\pgfsetfillcolor{currentfill}%
\pgfsetlinewidth{0.602250pt}%
\definecolor{currentstroke}{rgb}{0.000000,0.000000,0.000000}%
\pgfsetstrokecolor{currentstroke}%
\pgfsetdash{}{0pt}%
\pgfsys@defobject{currentmarker}{\pgfqpoint{-0.027778in}{0.000000in}}{\pgfqpoint{0.000000in}{0.000000in}}{%
\pgfpathmoveto{\pgfqpoint{0.000000in}{0.000000in}}%
\pgfpathlineto{\pgfqpoint{-0.027778in}{0.000000in}}%
\pgfusepath{stroke,fill}%
}%
\begin{pgfscope}%
\pgfsys@transformshift{0.625000in}{2.272138in}%
\pgfsys@useobject{currentmarker}{}%
\end{pgfscope}%
\end{pgfscope}%
\begin{pgfscope}%
\pgfsetbuttcap%
\pgfsetroundjoin%
\definecolor{currentfill}{rgb}{0.000000,0.000000,0.000000}%
\pgfsetfillcolor{currentfill}%
\pgfsetlinewidth{0.602250pt}%
\definecolor{currentstroke}{rgb}{0.000000,0.000000,0.000000}%
\pgfsetstrokecolor{currentstroke}%
\pgfsetdash{}{0pt}%
\pgfsys@defobject{currentmarker}{\pgfqpoint{-0.027778in}{0.000000in}}{\pgfqpoint{0.000000in}{0.000000in}}{%
\pgfpathmoveto{\pgfqpoint{0.000000in}{0.000000in}}%
\pgfpathlineto{\pgfqpoint{-0.027778in}{0.000000in}}%
\pgfusepath{stroke,fill}%
}%
\begin{pgfscope}%
\pgfsys@transformshift{0.625000in}{2.350271in}%
\pgfsys@useobject{currentmarker}{}%
\end{pgfscope}%
\end{pgfscope}%
\begin{pgfscope}%
\pgfsetbuttcap%
\pgfsetroundjoin%
\definecolor{currentfill}{rgb}{0.000000,0.000000,0.000000}%
\pgfsetfillcolor{currentfill}%
\pgfsetlinewidth{0.602250pt}%
\definecolor{currentstroke}{rgb}{0.000000,0.000000,0.000000}%
\pgfsetstrokecolor{currentstroke}%
\pgfsetdash{}{0pt}%
\pgfsys@defobject{currentmarker}{\pgfqpoint{-0.027778in}{0.000000in}}{\pgfqpoint{0.000000in}{0.000000in}}{%
\pgfpathmoveto{\pgfqpoint{0.000000in}{0.000000in}}%
\pgfpathlineto{\pgfqpoint{-0.027778in}{0.000000in}}%
\pgfusepath{stroke,fill}%
}%
\begin{pgfscope}%
\pgfsys@transformshift{0.625000in}{2.414111in}%
\pgfsys@useobject{currentmarker}{}%
\end{pgfscope}%
\end{pgfscope}%
\begin{pgfscope}%
\pgfsetbuttcap%
\pgfsetroundjoin%
\definecolor{currentfill}{rgb}{0.000000,0.000000,0.000000}%
\pgfsetfillcolor{currentfill}%
\pgfsetlinewidth{0.602250pt}%
\definecolor{currentstroke}{rgb}{0.000000,0.000000,0.000000}%
\pgfsetstrokecolor{currentstroke}%
\pgfsetdash{}{0pt}%
\pgfsys@defobject{currentmarker}{\pgfqpoint{-0.027778in}{0.000000in}}{\pgfqpoint{0.000000in}{0.000000in}}{%
\pgfpathmoveto{\pgfqpoint{0.000000in}{0.000000in}}%
\pgfpathlineto{\pgfqpoint{-0.027778in}{0.000000in}}%
\pgfusepath{stroke,fill}%
}%
\begin{pgfscope}%
\pgfsys@transformshift{0.625000in}{2.468087in}%
\pgfsys@useobject{currentmarker}{}%
\end{pgfscope}%
\end{pgfscope}%
\begin{pgfscope}%
\pgfsetbuttcap%
\pgfsetroundjoin%
\definecolor{currentfill}{rgb}{0.000000,0.000000,0.000000}%
\pgfsetfillcolor{currentfill}%
\pgfsetlinewidth{0.602250pt}%
\definecolor{currentstroke}{rgb}{0.000000,0.000000,0.000000}%
\pgfsetstrokecolor{currentstroke}%
\pgfsetdash{}{0pt}%
\pgfsys@defobject{currentmarker}{\pgfqpoint{-0.027778in}{0.000000in}}{\pgfqpoint{0.000000in}{0.000000in}}{%
\pgfpathmoveto{\pgfqpoint{0.000000in}{0.000000in}}%
\pgfpathlineto{\pgfqpoint{-0.027778in}{0.000000in}}%
\pgfusepath{stroke,fill}%
}%
\begin{pgfscope}%
\pgfsys@transformshift{0.625000in}{2.514843in}%
\pgfsys@useobject{currentmarker}{}%
\end{pgfscope}%
\end{pgfscope}%
\begin{pgfscope}%
\pgfsetbuttcap%
\pgfsetroundjoin%
\definecolor{currentfill}{rgb}{0.000000,0.000000,0.000000}%
\pgfsetfillcolor{currentfill}%
\pgfsetlinewidth{0.602250pt}%
\definecolor{currentstroke}{rgb}{0.000000,0.000000,0.000000}%
\pgfsetstrokecolor{currentstroke}%
\pgfsetdash{}{0pt}%
\pgfsys@defobject{currentmarker}{\pgfqpoint{-0.027778in}{0.000000in}}{\pgfqpoint{0.000000in}{0.000000in}}{%
\pgfpathmoveto{\pgfqpoint{0.000000in}{0.000000in}}%
\pgfpathlineto{\pgfqpoint{-0.027778in}{0.000000in}}%
\pgfusepath{stroke,fill}%
}%
\begin{pgfscope}%
\pgfsys@transformshift{0.625000in}{2.556085in}%
\pgfsys@useobject{currentmarker}{}%
\end{pgfscope}%
\end{pgfscope}%
\begin{pgfscope}%
\pgfsetbuttcap%
\pgfsetroundjoin%
\definecolor{currentfill}{rgb}{0.000000,0.000000,0.000000}%
\pgfsetfillcolor{currentfill}%
\pgfsetlinewidth{0.602250pt}%
\definecolor{currentstroke}{rgb}{0.000000,0.000000,0.000000}%
\pgfsetstrokecolor{currentstroke}%
\pgfsetdash{}{0pt}%
\pgfsys@defobject{currentmarker}{\pgfqpoint{-0.027778in}{0.000000in}}{\pgfqpoint{0.000000in}{0.000000in}}{%
\pgfpathmoveto{\pgfqpoint{0.000000in}{0.000000in}}%
\pgfpathlineto{\pgfqpoint{-0.027778in}{0.000000in}}%
\pgfusepath{stroke,fill}%
}%
\begin{pgfscope}%
\pgfsys@transformshift{0.625000in}{2.835682in}%
\pgfsys@useobject{currentmarker}{}%
\end{pgfscope}%
\end{pgfscope}%
\begin{pgfscope}%
\pgfsetbuttcap%
\pgfsetroundjoin%
\definecolor{currentfill}{rgb}{0.000000,0.000000,0.000000}%
\pgfsetfillcolor{currentfill}%
\pgfsetlinewidth{0.602250pt}%
\definecolor{currentstroke}{rgb}{0.000000,0.000000,0.000000}%
\pgfsetstrokecolor{currentstroke}%
\pgfsetdash{}{0pt}%
\pgfsys@defobject{currentmarker}{\pgfqpoint{-0.027778in}{0.000000in}}{\pgfqpoint{0.000000in}{0.000000in}}{%
\pgfpathmoveto{\pgfqpoint{0.000000in}{0.000000in}}%
\pgfpathlineto{\pgfqpoint{-0.027778in}{0.000000in}}%
\pgfusepath{stroke,fill}%
}%
\begin{pgfscope}%
\pgfsys@transformshift{0.625000in}{2.977656in}%
\pgfsys@useobject{currentmarker}{}%
\end{pgfscope}%
\end{pgfscope}%
\begin{pgfscope}%
\pgfsetbuttcap%
\pgfsetroundjoin%
\definecolor{currentfill}{rgb}{0.000000,0.000000,0.000000}%
\pgfsetfillcolor{currentfill}%
\pgfsetlinewidth{0.602250pt}%
\definecolor{currentstroke}{rgb}{0.000000,0.000000,0.000000}%
\pgfsetstrokecolor{currentstroke}%
\pgfsetdash{}{0pt}%
\pgfsys@defobject{currentmarker}{\pgfqpoint{-0.027778in}{0.000000in}}{\pgfqpoint{0.000000in}{0.000000in}}{%
\pgfpathmoveto{\pgfqpoint{0.000000in}{0.000000in}}%
\pgfpathlineto{\pgfqpoint{-0.027778in}{0.000000in}}%
\pgfusepath{stroke,fill}%
}%
\begin{pgfscope}%
\pgfsys@transformshift{0.625000in}{3.078388in}%
\pgfsys@useobject{currentmarker}{}%
\end{pgfscope}%
\end{pgfscope}%
\begin{pgfscope}%
\pgfsetbuttcap%
\pgfsetroundjoin%
\definecolor{currentfill}{rgb}{0.000000,0.000000,0.000000}%
\pgfsetfillcolor{currentfill}%
\pgfsetlinewidth{0.602250pt}%
\definecolor{currentstroke}{rgb}{0.000000,0.000000,0.000000}%
\pgfsetstrokecolor{currentstroke}%
\pgfsetdash{}{0pt}%
\pgfsys@defobject{currentmarker}{\pgfqpoint{-0.027778in}{0.000000in}}{\pgfqpoint{0.000000in}{0.000000in}}{%
\pgfpathmoveto{\pgfqpoint{0.000000in}{0.000000in}}%
\pgfpathlineto{\pgfqpoint{-0.027778in}{0.000000in}}%
\pgfusepath{stroke,fill}%
}%
\begin{pgfscope}%
\pgfsys@transformshift{0.625000in}{3.156522in}%
\pgfsys@useobject{currentmarker}{}%
\end{pgfscope}%
\end{pgfscope}%
\begin{pgfscope}%
\pgfsetbuttcap%
\pgfsetroundjoin%
\definecolor{currentfill}{rgb}{0.000000,0.000000,0.000000}%
\pgfsetfillcolor{currentfill}%
\pgfsetlinewidth{0.602250pt}%
\definecolor{currentstroke}{rgb}{0.000000,0.000000,0.000000}%
\pgfsetstrokecolor{currentstroke}%
\pgfsetdash{}{0pt}%
\pgfsys@defobject{currentmarker}{\pgfqpoint{-0.027778in}{0.000000in}}{\pgfqpoint{0.000000in}{0.000000in}}{%
\pgfpathmoveto{\pgfqpoint{0.000000in}{0.000000in}}%
\pgfpathlineto{\pgfqpoint{-0.027778in}{0.000000in}}%
\pgfusepath{stroke,fill}%
}%
\begin{pgfscope}%
\pgfsys@transformshift{0.625000in}{3.220362in}%
\pgfsys@useobject{currentmarker}{}%
\end{pgfscope}%
\end{pgfscope}%
\begin{pgfscope}%
\pgfsetbuttcap%
\pgfsetroundjoin%
\definecolor{currentfill}{rgb}{0.000000,0.000000,0.000000}%
\pgfsetfillcolor{currentfill}%
\pgfsetlinewidth{0.602250pt}%
\definecolor{currentstroke}{rgb}{0.000000,0.000000,0.000000}%
\pgfsetstrokecolor{currentstroke}%
\pgfsetdash{}{0pt}%
\pgfsys@defobject{currentmarker}{\pgfqpoint{-0.027778in}{0.000000in}}{\pgfqpoint{0.000000in}{0.000000in}}{%
\pgfpathmoveto{\pgfqpoint{0.000000in}{0.000000in}}%
\pgfpathlineto{\pgfqpoint{-0.027778in}{0.000000in}}%
\pgfusepath{stroke,fill}%
}%
\begin{pgfscope}%
\pgfsys@transformshift{0.625000in}{3.274337in}%
\pgfsys@useobject{currentmarker}{}%
\end{pgfscope}%
\end{pgfscope}%
\begin{pgfscope}%
\pgfsetbuttcap%
\pgfsetroundjoin%
\definecolor{currentfill}{rgb}{0.000000,0.000000,0.000000}%
\pgfsetfillcolor{currentfill}%
\pgfsetlinewidth{0.602250pt}%
\definecolor{currentstroke}{rgb}{0.000000,0.000000,0.000000}%
\pgfsetstrokecolor{currentstroke}%
\pgfsetdash{}{0pt}%
\pgfsys@defobject{currentmarker}{\pgfqpoint{-0.027778in}{0.000000in}}{\pgfqpoint{0.000000in}{0.000000in}}{%
\pgfpathmoveto{\pgfqpoint{0.000000in}{0.000000in}}%
\pgfpathlineto{\pgfqpoint{-0.027778in}{0.000000in}}%
\pgfusepath{stroke,fill}%
}%
\begin{pgfscope}%
\pgfsys@transformshift{0.625000in}{3.321093in}%
\pgfsys@useobject{currentmarker}{}%
\end{pgfscope}%
\end{pgfscope}%
\begin{pgfscope}%
\pgfsetbuttcap%
\pgfsetroundjoin%
\definecolor{currentfill}{rgb}{0.000000,0.000000,0.000000}%
\pgfsetfillcolor{currentfill}%
\pgfsetlinewidth{0.602250pt}%
\definecolor{currentstroke}{rgb}{0.000000,0.000000,0.000000}%
\pgfsetstrokecolor{currentstroke}%
\pgfsetdash{}{0pt}%
\pgfsys@defobject{currentmarker}{\pgfqpoint{-0.027778in}{0.000000in}}{\pgfqpoint{0.000000in}{0.000000in}}{%
\pgfpathmoveto{\pgfqpoint{0.000000in}{0.000000in}}%
\pgfpathlineto{\pgfqpoint{-0.027778in}{0.000000in}}%
\pgfusepath{stroke,fill}%
}%
\begin{pgfscope}%
\pgfsys@transformshift{0.625000in}{3.362335in}%
\pgfsys@useobject{currentmarker}{}%
\end{pgfscope}%
\end{pgfscope}%
\begin{pgfscope}%
\definecolor{textcolor}{rgb}{0.000000,0.000000,0.000000}%
\pgfsetstrokecolor{textcolor}%
\pgfsetfillcolor{textcolor}%
\pgftext[x=0.184220in,y=1.980000in,,bottom,rotate=90.000000]{\color{textcolor}\sffamily\fontsize{10.000000}{12.000000}\selectfont QMC Error}%
\end{pgfscope}%
\begin{pgfscope}%
\pgfpathrectangle{\pgfqpoint{0.625000in}{0.440000in}}{\pgfqpoint{3.875000in}{3.080000in}}%
\pgfusepath{clip}%
\pgfsetbuttcap%
\pgfsetroundjoin%
\definecolor{currentfill}{rgb}{0.000000,0.000000,0.000000}%
\pgfsetfillcolor{currentfill}%
\pgfsetlinewidth{1.003750pt}%
\definecolor{currentstroke}{rgb}{0.000000,0.000000,0.000000}%
\pgfsetstrokecolor{currentstroke}%
\pgfsetdash{}{0pt}%
\pgfsys@defobject{currentmarker}{\pgfqpoint{-0.041667in}{-0.041667in}}{\pgfqpoint{0.041667in}{0.041667in}}{%
\pgfpathmoveto{\pgfqpoint{0.000000in}{-0.041667in}}%
\pgfpathcurveto{\pgfqpoint{0.011050in}{-0.041667in}}{\pgfqpoint{0.021649in}{-0.037276in}}{\pgfqpoint{0.029463in}{-0.029463in}}%
\pgfpathcurveto{\pgfqpoint{0.037276in}{-0.021649in}}{\pgfqpoint{0.041667in}{-0.011050in}}{\pgfqpoint{0.041667in}{0.000000in}}%
\pgfpathcurveto{\pgfqpoint{0.041667in}{0.011050in}}{\pgfqpoint{0.037276in}{0.021649in}}{\pgfqpoint{0.029463in}{0.029463in}}%
\pgfpathcurveto{\pgfqpoint{0.021649in}{0.037276in}}{\pgfqpoint{0.011050in}{0.041667in}}{\pgfqpoint{0.000000in}{0.041667in}}%
\pgfpathcurveto{\pgfqpoint{-0.011050in}{0.041667in}}{\pgfqpoint{-0.021649in}{0.037276in}}{\pgfqpoint{-0.029463in}{0.029463in}}%
\pgfpathcurveto{\pgfqpoint{-0.037276in}{0.021649in}}{\pgfqpoint{-0.041667in}{0.011050in}}{\pgfqpoint{-0.041667in}{0.000000in}}%
\pgfpathcurveto{\pgfqpoint{-0.041667in}{-0.011050in}}{\pgfqpoint{-0.037276in}{-0.021649in}}{\pgfqpoint{-0.029463in}{-0.029463in}}%
\pgfpathcurveto{\pgfqpoint{-0.021649in}{-0.037276in}}{\pgfqpoint{-0.011050in}{-0.041667in}}{\pgfqpoint{0.000000in}{-0.041667in}}%
\pgfpathclose%
\pgfusepath{stroke,fill}%
}%
\begin{pgfscope}%
\pgfsys@transformshift{0.801136in}{3.380000in}%
\pgfsys@useobject{currentmarker}{}%
\end{pgfscope}%
\begin{pgfscope}%
\pgfsys@transformshift{1.121384in}{3.034255in}%
\pgfsys@useobject{currentmarker}{}%
\end{pgfscope}%
\begin{pgfscope}%
\pgfsys@transformshift{1.441632in}{2.815030in}%
\pgfsys@useobject{currentmarker}{}%
\end{pgfscope}%
\begin{pgfscope}%
\pgfsys@transformshift{1.761880in}{2.537215in}%
\pgfsys@useobject{currentmarker}{}%
\end{pgfscope}%
\begin{pgfscope}%
\pgfsys@transformshift{2.082128in}{2.294807in}%
\pgfsys@useobject{currentmarker}{}%
\end{pgfscope}%
\begin{pgfscope}%
\pgfsys@transformshift{2.402376in}{2.188383in}%
\pgfsys@useobject{currentmarker}{}%
\end{pgfscope}%
\begin{pgfscope}%
\pgfsys@transformshift{2.722624in}{1.916103in}%
\pgfsys@useobject{currentmarker}{}%
\end{pgfscope}%
\begin{pgfscope}%
\pgfsys@transformshift{3.042872in}{1.648617in}%
\pgfsys@useobject{currentmarker}{}%
\end{pgfscope}%
\begin{pgfscope}%
\pgfsys@transformshift{3.363120in}{1.230239in}%
\pgfsys@useobject{currentmarker}{}%
\end{pgfscope}%
\begin{pgfscope}%
\pgfsys@transformshift{3.683368in}{1.200314in}%
\pgfsys@useobject{currentmarker}{}%
\end{pgfscope}%
\begin{pgfscope}%
\pgfsys@transformshift{4.003616in}{0.861794in}%
\pgfsys@useobject{currentmarker}{}%
\end{pgfscope}%
\begin{pgfscope}%
\pgfsys@transformshift{4.323864in}{0.580000in}%
\pgfsys@useobject{currentmarker}{}%
\end{pgfscope}%
\end{pgfscope}%
\begin{pgfscope}%
\pgfpathrectangle{\pgfqpoint{0.625000in}{0.440000in}}{\pgfqpoint{3.875000in}{3.080000in}}%
\pgfusepath{clip}%
\pgfsetbuttcap%
\pgfsetroundjoin%
\pgfsetlinewidth{1.505625pt}%
\definecolor{currentstroke}{rgb}{0.000000,0.000000,0.000000}%
\pgfsetstrokecolor{currentstroke}%
\pgfsetdash{{5.550000pt}{2.400000pt}}{0.000000pt}%
\pgfpathmoveto{\pgfqpoint{0.801136in}{3.327760in}}%
\pgfpathlineto{\pgfqpoint{4.323864in}{0.620033in}}%
\pgfusepath{stroke}%
\end{pgfscope}%
\begin{pgfscope}%
\pgfsetrectcap%
\pgfsetmiterjoin%
\pgfsetlinewidth{0.803000pt}%
\definecolor{currentstroke}{rgb}{0.000000,0.000000,0.000000}%
\pgfsetstrokecolor{currentstroke}%
\pgfsetdash{}{0pt}%
\pgfpathmoveto{\pgfqpoint{0.625000in}{0.440000in}}%
\pgfpathlineto{\pgfqpoint{0.625000in}{3.520000in}}%
\pgfusepath{stroke}%
\end{pgfscope}%
\begin{pgfscope}%
\pgfsetrectcap%
\pgfsetmiterjoin%
\pgfsetlinewidth{0.803000pt}%
\definecolor{currentstroke}{rgb}{0.000000,0.000000,0.000000}%
\pgfsetstrokecolor{currentstroke}%
\pgfsetdash{}{0pt}%
\pgfpathmoveto{\pgfqpoint{4.500000in}{0.440000in}}%
\pgfpathlineto{\pgfqpoint{4.500000in}{3.520000in}}%
\pgfusepath{stroke}%
\end{pgfscope}%
\begin{pgfscope}%
\pgfsetrectcap%
\pgfsetmiterjoin%
\pgfsetlinewidth{0.803000pt}%
\definecolor{currentstroke}{rgb}{0.000000,0.000000,0.000000}%
\pgfsetstrokecolor{currentstroke}%
\pgfsetdash{}{0pt}%
\pgfpathmoveto{\pgfqpoint{0.625000in}{0.440000in}}%
\pgfpathlineto{\pgfqpoint{4.500000in}{0.440000in}}%
\pgfusepath{stroke}%
\end{pgfscope}%
\begin{pgfscope}%
\pgfsetrectcap%
\pgfsetmiterjoin%
\pgfsetlinewidth{0.803000pt}%
\definecolor{currentstroke}{rgb}{0.000000,0.000000,0.000000}%
\pgfsetstrokecolor{currentstroke}%
\pgfsetdash{}{0pt}%
\pgfpathmoveto{\pgfqpoint{0.625000in}{3.520000in}}%
\pgfpathlineto{\pgfqpoint{4.500000in}{3.520000in}}%
\pgfusepath{stroke}%
\end{pgfscope}%
\begin{pgfscope}%
\pgfsetbuttcap%
\pgfsetmiterjoin%
\definecolor{currentfill}{rgb}{1.000000,1.000000,1.000000}%
\pgfsetfillcolor{currentfill}%
\pgfsetfillopacity{0.800000}%
\pgfsetlinewidth{1.003750pt}%
\definecolor{currentstroke}{rgb}{0.800000,0.800000,0.800000}%
\pgfsetstrokecolor{currentstroke}%
\pgfsetstrokeopacity{0.800000}%
\pgfsetdash{}{0pt}%
\pgfpathmoveto{\pgfqpoint{3.042243in}{3.155267in}}%
\pgfpathlineto{\pgfqpoint{4.402778in}{3.155267in}}%
\pgfpathquadraticcurveto{\pgfqpoint{4.430556in}{3.155267in}}{\pgfqpoint{4.430556in}{3.183045in}}%
\pgfpathlineto{\pgfqpoint{4.430556in}{3.422778in}}%
\pgfpathquadraticcurveto{\pgfqpoint{4.430556in}{3.450556in}}{\pgfqpoint{4.402778in}{3.450556in}}%
\pgfpathlineto{\pgfqpoint{3.042243in}{3.450556in}}%
\pgfpathquadraticcurveto{\pgfqpoint{3.014465in}{3.450556in}}{\pgfqpoint{3.014465in}{3.422778in}}%
\pgfpathlineto{\pgfqpoint{3.014465in}{3.183045in}}%
\pgfpathquadraticcurveto{\pgfqpoint{3.014465in}{3.155267in}}{\pgfqpoint{3.042243in}{3.155267in}}%
\pgfpathclose%
\pgfusepath{stroke,fill}%
\end{pgfscope}%
\begin{pgfscope}%
\pgfsetbuttcap%
\pgfsetroundjoin%
\pgfsetlinewidth{1.505625pt}%
\definecolor{currentstroke}{rgb}{0.000000,0.000000,0.000000}%
\pgfsetstrokecolor{currentstroke}%
\pgfsetdash{{5.550000pt}{2.400000pt}}{0.000000pt}%
\pgfpathmoveto{\pgfqpoint{3.070021in}{3.321543in}}%
\pgfpathlineto{\pgfqpoint{3.347799in}{3.321543in}}%
\pgfusepath{stroke}%
\end{pgfscope}%
\begin{pgfscope}%
\definecolor{textcolor}{rgb}{0.000000,0.000000,0.000000}%
\pgfsetstrokecolor{textcolor}%
\pgfsetfillcolor{textcolor}%
\pgftext[x=3.458910in,y=3.272932in,left,base]{\color{textcolor}\sffamily\fontsize{10.000000}{12.000000}\selectfont \(\displaystyle 0.8153N_{\mathrm{QMC}}^{-1.0142}\)}%
\end{pgfscope}%
\end{pgfpicture}%
\makeatother%
\endgroup%

    \caption[Quasi-Monte-Carlo error, for $Q=\grad u((1,1))$ and $k=10$.]{Quasi-Monte-Carlo error with increasing number of Quasi-Monte-Carlo points, for $Q=\grad u((1,1))$ and $k=10$. \label{fig:qmcgradient_top_right10}}
  \end{figure}
\begin{figure}[h]
  \centering
%% Creator: Matplotlib, PGF backend
%%
%% To include the figure in your LaTeX document, write
%%   \input{<filename>.pgf}
%%
%% Make sure the required packages are loaded in your preamble
%%   \usepackage{pgf}
%%
%% Figures using additional raster images can only be included by \input if
%% they are in the same directory as the main LaTeX file. For loading figures
%% from other directories you can use the `import` package
%%   \usepackage{import}
%% and then include the figures with
%%   \import{<path to file>}{<filename>.pgf}
%%
%% Matplotlib used the following preamble
%%   \usepackage{fontspec}
%%   \setmainfont{DejaVuSerif.ttf}[Path=/home/owen/progs/firedrake-complex/firedrake/lib/python3.5/site-packages/matplotlib/mpl-data/fonts/ttf/]
%%   \setsansfont{DejaVuSans.ttf}[Path=/home/owen/progs/firedrake-complex/firedrake/lib/python3.5/site-packages/matplotlib/mpl-data/fonts/ttf/]
%%   \setmonofont{DejaVuSansMono.ttf}[Path=/home/owen/progs/firedrake-complex/firedrake/lib/python3.5/site-packages/matplotlib/mpl-data/fonts/ttf/]
%%
\begingroup%
\makeatletter%
\begin{pgfpicture}%
\pgfpathrectangle{\pgfpointorigin}{\pgfqpoint{3.000000in}{3.000000in}}%
\pgfusepath{use as bounding box, clip}%
\begin{pgfscope}%
\pgfsetbuttcap%
\pgfsetmiterjoin%
\definecolor{currentfill}{rgb}{1.000000,1.000000,1.000000}%
\pgfsetfillcolor{currentfill}%
\pgfsetlinewidth{0.000000pt}%
\definecolor{currentstroke}{rgb}{1.000000,1.000000,1.000000}%
\pgfsetstrokecolor{currentstroke}%
\pgfsetdash{}{0pt}%
\pgfpathmoveto{\pgfqpoint{0.000000in}{0.000000in}}%
\pgfpathlineto{\pgfqpoint{3.000000in}{0.000000in}}%
\pgfpathlineto{\pgfqpoint{3.000000in}{3.000000in}}%
\pgfpathlineto{\pgfqpoint{0.000000in}{3.000000in}}%
\pgfpathclose%
\pgfusepath{fill}%
\end{pgfscope}%
\begin{pgfscope}%
\pgfsetbuttcap%
\pgfsetmiterjoin%
\definecolor{currentfill}{rgb}{1.000000,1.000000,1.000000}%
\pgfsetfillcolor{currentfill}%
\pgfsetlinewidth{0.000000pt}%
\definecolor{currentstroke}{rgb}{0.000000,0.000000,0.000000}%
\pgfsetstrokecolor{currentstroke}%
\pgfsetstrokeopacity{0.000000}%
\pgfsetdash{}{0pt}%
\pgfpathmoveto{\pgfqpoint{0.375000in}{0.330000in}}%
\pgfpathlineto{\pgfqpoint{2.700000in}{0.330000in}}%
\pgfpathlineto{\pgfqpoint{2.700000in}{2.640000in}}%
\pgfpathlineto{\pgfqpoint{0.375000in}{2.640000in}}%
\pgfpathclose%
\pgfusepath{fill}%
\end{pgfscope}%
\begin{pgfscope}%
\pgfsetbuttcap%
\pgfsetroundjoin%
\definecolor{currentfill}{rgb}{0.000000,0.000000,0.000000}%
\pgfsetfillcolor{currentfill}%
\pgfsetlinewidth{0.803000pt}%
\definecolor{currentstroke}{rgb}{0.000000,0.000000,0.000000}%
\pgfsetstrokecolor{currentstroke}%
\pgfsetdash{}{0pt}%
\pgfsys@defobject{currentmarker}{\pgfqpoint{0.000000in}{-0.048611in}}{\pgfqpoint{0.000000in}{0.000000in}}{%
\pgfpathmoveto{\pgfqpoint{0.000000in}{0.000000in}}%
\pgfpathlineto{\pgfqpoint{0.000000in}{-0.048611in}}%
\pgfusepath{stroke,fill}%
}%
\begin{pgfscope}%
\pgfsys@transformshift{0.480682in}{0.330000in}%
\pgfsys@useobject{currentmarker}{}%
\end{pgfscope}%
\end{pgfscope}%
\begin{pgfscope}%
\definecolor{textcolor}{rgb}{0.000000,0.000000,0.000000}%
\pgfsetstrokecolor{textcolor}%
\pgfsetfillcolor{textcolor}%
\pgftext[x=0.480682in,y=0.232778in,,top]{\color{textcolor}\sffamily\fontsize{10.000000}{12.000000}\selectfont \(\displaystyle {10^{0}}\)}%
\end{pgfscope}%
\begin{pgfscope}%
\pgfsetbuttcap%
\pgfsetroundjoin%
\definecolor{currentfill}{rgb}{0.000000,0.000000,0.000000}%
\pgfsetfillcolor{currentfill}%
\pgfsetlinewidth{0.803000pt}%
\definecolor{currentstroke}{rgb}{0.000000,0.000000,0.000000}%
\pgfsetstrokecolor{currentstroke}%
\pgfsetdash{}{0pt}%
\pgfsys@defobject{currentmarker}{\pgfqpoint{0.000000in}{-0.048611in}}{\pgfqpoint{0.000000in}{0.000000in}}{%
\pgfpathmoveto{\pgfqpoint{0.000000in}{0.000000in}}%
\pgfpathlineto{\pgfqpoint{0.000000in}{-0.048611in}}%
\pgfusepath{stroke,fill}%
}%
\begin{pgfscope}%
\pgfsys@transformshift{1.118986in}{0.330000in}%
\pgfsys@useobject{currentmarker}{}%
\end{pgfscope}%
\end{pgfscope}%
\begin{pgfscope}%
\definecolor{textcolor}{rgb}{0.000000,0.000000,0.000000}%
\pgfsetstrokecolor{textcolor}%
\pgfsetfillcolor{textcolor}%
\pgftext[x=1.118986in,y=0.232778in,,top]{\color{textcolor}\sffamily\fontsize{10.000000}{12.000000}\selectfont \(\displaystyle {10^{1}}\)}%
\end{pgfscope}%
\begin{pgfscope}%
\pgfsetbuttcap%
\pgfsetroundjoin%
\definecolor{currentfill}{rgb}{0.000000,0.000000,0.000000}%
\pgfsetfillcolor{currentfill}%
\pgfsetlinewidth{0.803000pt}%
\definecolor{currentstroke}{rgb}{0.000000,0.000000,0.000000}%
\pgfsetstrokecolor{currentstroke}%
\pgfsetdash{}{0pt}%
\pgfsys@defobject{currentmarker}{\pgfqpoint{0.000000in}{-0.048611in}}{\pgfqpoint{0.000000in}{0.000000in}}{%
\pgfpathmoveto{\pgfqpoint{0.000000in}{0.000000in}}%
\pgfpathlineto{\pgfqpoint{0.000000in}{-0.048611in}}%
\pgfusepath{stroke,fill}%
}%
\begin{pgfscope}%
\pgfsys@transformshift{1.757291in}{0.330000in}%
\pgfsys@useobject{currentmarker}{}%
\end{pgfscope}%
\end{pgfscope}%
\begin{pgfscope}%
\definecolor{textcolor}{rgb}{0.000000,0.000000,0.000000}%
\pgfsetstrokecolor{textcolor}%
\pgfsetfillcolor{textcolor}%
\pgftext[x=1.757291in,y=0.232778in,,top]{\color{textcolor}\sffamily\fontsize{10.000000}{12.000000}\selectfont \(\displaystyle {10^{2}}\)}%
\end{pgfscope}%
\begin{pgfscope}%
\pgfsetbuttcap%
\pgfsetroundjoin%
\definecolor{currentfill}{rgb}{0.000000,0.000000,0.000000}%
\pgfsetfillcolor{currentfill}%
\pgfsetlinewidth{0.803000pt}%
\definecolor{currentstroke}{rgb}{0.000000,0.000000,0.000000}%
\pgfsetstrokecolor{currentstroke}%
\pgfsetdash{}{0pt}%
\pgfsys@defobject{currentmarker}{\pgfqpoint{0.000000in}{-0.048611in}}{\pgfqpoint{0.000000in}{0.000000in}}{%
\pgfpathmoveto{\pgfqpoint{0.000000in}{0.000000in}}%
\pgfpathlineto{\pgfqpoint{0.000000in}{-0.048611in}}%
\pgfusepath{stroke,fill}%
}%
\begin{pgfscope}%
\pgfsys@transformshift{2.395595in}{0.330000in}%
\pgfsys@useobject{currentmarker}{}%
\end{pgfscope}%
\end{pgfscope}%
\begin{pgfscope}%
\definecolor{textcolor}{rgb}{0.000000,0.000000,0.000000}%
\pgfsetstrokecolor{textcolor}%
\pgfsetfillcolor{textcolor}%
\pgftext[x=2.395595in,y=0.232778in,,top]{\color{textcolor}\sffamily\fontsize{10.000000}{12.000000}\selectfont \(\displaystyle {10^{3}}\)}%
\end{pgfscope}%
\begin{pgfscope}%
\pgfsetbuttcap%
\pgfsetroundjoin%
\definecolor{currentfill}{rgb}{0.000000,0.000000,0.000000}%
\pgfsetfillcolor{currentfill}%
\pgfsetlinewidth{0.602250pt}%
\definecolor{currentstroke}{rgb}{0.000000,0.000000,0.000000}%
\pgfsetstrokecolor{currentstroke}%
\pgfsetdash{}{0pt}%
\pgfsys@defobject{currentmarker}{\pgfqpoint{0.000000in}{-0.027778in}}{\pgfqpoint{0.000000in}{0.000000in}}{%
\pgfpathmoveto{\pgfqpoint{0.000000in}{0.000000in}}%
\pgfpathlineto{\pgfqpoint{0.000000in}{-0.027778in}}%
\pgfusepath{stroke,fill}%
}%
\begin{pgfscope}%
\pgfsys@transformshift{0.381807in}{0.330000in}%
\pgfsys@useobject{currentmarker}{}%
\end{pgfscope}%
\end{pgfscope}%
\begin{pgfscope}%
\pgfsetbuttcap%
\pgfsetroundjoin%
\definecolor{currentfill}{rgb}{0.000000,0.000000,0.000000}%
\pgfsetfillcolor{currentfill}%
\pgfsetlinewidth{0.602250pt}%
\definecolor{currentstroke}{rgb}{0.000000,0.000000,0.000000}%
\pgfsetstrokecolor{currentstroke}%
\pgfsetdash{}{0pt}%
\pgfsys@defobject{currentmarker}{\pgfqpoint{0.000000in}{-0.027778in}}{\pgfqpoint{0.000000in}{0.000000in}}{%
\pgfpathmoveto{\pgfqpoint{0.000000in}{0.000000in}}%
\pgfpathlineto{\pgfqpoint{0.000000in}{-0.027778in}}%
\pgfusepath{stroke,fill}%
}%
\begin{pgfscope}%
\pgfsys@transformshift{0.418824in}{0.330000in}%
\pgfsys@useobject{currentmarker}{}%
\end{pgfscope}%
\end{pgfscope}%
\begin{pgfscope}%
\pgfsetbuttcap%
\pgfsetroundjoin%
\definecolor{currentfill}{rgb}{0.000000,0.000000,0.000000}%
\pgfsetfillcolor{currentfill}%
\pgfsetlinewidth{0.602250pt}%
\definecolor{currentstroke}{rgb}{0.000000,0.000000,0.000000}%
\pgfsetstrokecolor{currentstroke}%
\pgfsetdash{}{0pt}%
\pgfsys@defobject{currentmarker}{\pgfqpoint{0.000000in}{-0.027778in}}{\pgfqpoint{0.000000in}{0.000000in}}{%
\pgfpathmoveto{\pgfqpoint{0.000000in}{0.000000in}}%
\pgfpathlineto{\pgfqpoint{0.000000in}{-0.027778in}}%
\pgfusepath{stroke,fill}%
}%
\begin{pgfscope}%
\pgfsys@transformshift{0.451475in}{0.330000in}%
\pgfsys@useobject{currentmarker}{}%
\end{pgfscope}%
\end{pgfscope}%
\begin{pgfscope}%
\pgfsetbuttcap%
\pgfsetroundjoin%
\definecolor{currentfill}{rgb}{0.000000,0.000000,0.000000}%
\pgfsetfillcolor{currentfill}%
\pgfsetlinewidth{0.602250pt}%
\definecolor{currentstroke}{rgb}{0.000000,0.000000,0.000000}%
\pgfsetstrokecolor{currentstroke}%
\pgfsetdash{}{0pt}%
\pgfsys@defobject{currentmarker}{\pgfqpoint{0.000000in}{-0.027778in}}{\pgfqpoint{0.000000in}{0.000000in}}{%
\pgfpathmoveto{\pgfqpoint{0.000000in}{0.000000in}}%
\pgfpathlineto{\pgfqpoint{0.000000in}{-0.027778in}}%
\pgfusepath{stroke,fill}%
}%
\begin{pgfscope}%
\pgfsys@transformshift{0.672831in}{0.330000in}%
\pgfsys@useobject{currentmarker}{}%
\end{pgfscope}%
\end{pgfscope}%
\begin{pgfscope}%
\pgfsetbuttcap%
\pgfsetroundjoin%
\definecolor{currentfill}{rgb}{0.000000,0.000000,0.000000}%
\pgfsetfillcolor{currentfill}%
\pgfsetlinewidth{0.602250pt}%
\definecolor{currentstroke}{rgb}{0.000000,0.000000,0.000000}%
\pgfsetstrokecolor{currentstroke}%
\pgfsetdash{}{0pt}%
\pgfsys@defobject{currentmarker}{\pgfqpoint{0.000000in}{-0.027778in}}{\pgfqpoint{0.000000in}{0.000000in}}{%
\pgfpathmoveto{\pgfqpoint{0.000000in}{0.000000in}}%
\pgfpathlineto{\pgfqpoint{0.000000in}{-0.027778in}}%
\pgfusepath{stroke,fill}%
}%
\begin{pgfscope}%
\pgfsys@transformshift{0.785230in}{0.330000in}%
\pgfsys@useobject{currentmarker}{}%
\end{pgfscope}%
\end{pgfscope}%
\begin{pgfscope}%
\pgfsetbuttcap%
\pgfsetroundjoin%
\definecolor{currentfill}{rgb}{0.000000,0.000000,0.000000}%
\pgfsetfillcolor{currentfill}%
\pgfsetlinewidth{0.602250pt}%
\definecolor{currentstroke}{rgb}{0.000000,0.000000,0.000000}%
\pgfsetstrokecolor{currentstroke}%
\pgfsetdash{}{0pt}%
\pgfsys@defobject{currentmarker}{\pgfqpoint{0.000000in}{-0.027778in}}{\pgfqpoint{0.000000in}{0.000000in}}{%
\pgfpathmoveto{\pgfqpoint{0.000000in}{0.000000in}}%
\pgfpathlineto{\pgfqpoint{0.000000in}{-0.027778in}}%
\pgfusepath{stroke,fill}%
}%
\begin{pgfscope}%
\pgfsys@transformshift{0.864979in}{0.330000in}%
\pgfsys@useobject{currentmarker}{}%
\end{pgfscope}%
\end{pgfscope}%
\begin{pgfscope}%
\pgfsetbuttcap%
\pgfsetroundjoin%
\definecolor{currentfill}{rgb}{0.000000,0.000000,0.000000}%
\pgfsetfillcolor{currentfill}%
\pgfsetlinewidth{0.602250pt}%
\definecolor{currentstroke}{rgb}{0.000000,0.000000,0.000000}%
\pgfsetstrokecolor{currentstroke}%
\pgfsetdash{}{0pt}%
\pgfsys@defobject{currentmarker}{\pgfqpoint{0.000000in}{-0.027778in}}{\pgfqpoint{0.000000in}{0.000000in}}{%
\pgfpathmoveto{\pgfqpoint{0.000000in}{0.000000in}}%
\pgfpathlineto{\pgfqpoint{0.000000in}{-0.027778in}}%
\pgfusepath{stroke,fill}%
}%
\begin{pgfscope}%
\pgfsys@transformshift{0.926837in}{0.330000in}%
\pgfsys@useobject{currentmarker}{}%
\end{pgfscope}%
\end{pgfscope}%
\begin{pgfscope}%
\pgfsetbuttcap%
\pgfsetroundjoin%
\definecolor{currentfill}{rgb}{0.000000,0.000000,0.000000}%
\pgfsetfillcolor{currentfill}%
\pgfsetlinewidth{0.602250pt}%
\definecolor{currentstroke}{rgb}{0.000000,0.000000,0.000000}%
\pgfsetstrokecolor{currentstroke}%
\pgfsetdash{}{0pt}%
\pgfsys@defobject{currentmarker}{\pgfqpoint{0.000000in}{-0.027778in}}{\pgfqpoint{0.000000in}{0.000000in}}{%
\pgfpathmoveto{\pgfqpoint{0.000000in}{0.000000in}}%
\pgfpathlineto{\pgfqpoint{0.000000in}{-0.027778in}}%
\pgfusepath{stroke,fill}%
}%
\begin{pgfscope}%
\pgfsys@transformshift{0.977379in}{0.330000in}%
\pgfsys@useobject{currentmarker}{}%
\end{pgfscope}%
\end{pgfscope}%
\begin{pgfscope}%
\pgfsetbuttcap%
\pgfsetroundjoin%
\definecolor{currentfill}{rgb}{0.000000,0.000000,0.000000}%
\pgfsetfillcolor{currentfill}%
\pgfsetlinewidth{0.602250pt}%
\definecolor{currentstroke}{rgb}{0.000000,0.000000,0.000000}%
\pgfsetstrokecolor{currentstroke}%
\pgfsetdash{}{0pt}%
\pgfsys@defobject{currentmarker}{\pgfqpoint{0.000000in}{-0.027778in}}{\pgfqpoint{0.000000in}{0.000000in}}{%
\pgfpathmoveto{\pgfqpoint{0.000000in}{0.000000in}}%
\pgfpathlineto{\pgfqpoint{0.000000in}{-0.027778in}}%
\pgfusepath{stroke,fill}%
}%
\begin{pgfscope}%
\pgfsys@transformshift{1.020112in}{0.330000in}%
\pgfsys@useobject{currentmarker}{}%
\end{pgfscope}%
\end{pgfscope}%
\begin{pgfscope}%
\pgfsetbuttcap%
\pgfsetroundjoin%
\definecolor{currentfill}{rgb}{0.000000,0.000000,0.000000}%
\pgfsetfillcolor{currentfill}%
\pgfsetlinewidth{0.602250pt}%
\definecolor{currentstroke}{rgb}{0.000000,0.000000,0.000000}%
\pgfsetstrokecolor{currentstroke}%
\pgfsetdash{}{0pt}%
\pgfsys@defobject{currentmarker}{\pgfqpoint{0.000000in}{-0.027778in}}{\pgfqpoint{0.000000in}{0.000000in}}{%
\pgfpathmoveto{\pgfqpoint{0.000000in}{0.000000in}}%
\pgfpathlineto{\pgfqpoint{0.000000in}{-0.027778in}}%
\pgfusepath{stroke,fill}%
}%
\begin{pgfscope}%
\pgfsys@transformshift{1.057128in}{0.330000in}%
\pgfsys@useobject{currentmarker}{}%
\end{pgfscope}%
\end{pgfscope}%
\begin{pgfscope}%
\pgfsetbuttcap%
\pgfsetroundjoin%
\definecolor{currentfill}{rgb}{0.000000,0.000000,0.000000}%
\pgfsetfillcolor{currentfill}%
\pgfsetlinewidth{0.602250pt}%
\definecolor{currentstroke}{rgb}{0.000000,0.000000,0.000000}%
\pgfsetstrokecolor{currentstroke}%
\pgfsetdash{}{0pt}%
\pgfsys@defobject{currentmarker}{\pgfqpoint{0.000000in}{-0.027778in}}{\pgfqpoint{0.000000in}{0.000000in}}{%
\pgfpathmoveto{\pgfqpoint{0.000000in}{0.000000in}}%
\pgfpathlineto{\pgfqpoint{0.000000in}{-0.027778in}}%
\pgfusepath{stroke,fill}%
}%
\begin{pgfscope}%
\pgfsys@transformshift{1.089779in}{0.330000in}%
\pgfsys@useobject{currentmarker}{}%
\end{pgfscope}%
\end{pgfscope}%
\begin{pgfscope}%
\pgfsetbuttcap%
\pgfsetroundjoin%
\definecolor{currentfill}{rgb}{0.000000,0.000000,0.000000}%
\pgfsetfillcolor{currentfill}%
\pgfsetlinewidth{0.602250pt}%
\definecolor{currentstroke}{rgb}{0.000000,0.000000,0.000000}%
\pgfsetstrokecolor{currentstroke}%
\pgfsetdash{}{0pt}%
\pgfsys@defobject{currentmarker}{\pgfqpoint{0.000000in}{-0.027778in}}{\pgfqpoint{0.000000in}{0.000000in}}{%
\pgfpathmoveto{\pgfqpoint{0.000000in}{0.000000in}}%
\pgfpathlineto{\pgfqpoint{0.000000in}{-0.027778in}}%
\pgfusepath{stroke,fill}%
}%
\begin{pgfscope}%
\pgfsys@transformshift{1.311135in}{0.330000in}%
\pgfsys@useobject{currentmarker}{}%
\end{pgfscope}%
\end{pgfscope}%
\begin{pgfscope}%
\pgfsetbuttcap%
\pgfsetroundjoin%
\definecolor{currentfill}{rgb}{0.000000,0.000000,0.000000}%
\pgfsetfillcolor{currentfill}%
\pgfsetlinewidth{0.602250pt}%
\definecolor{currentstroke}{rgb}{0.000000,0.000000,0.000000}%
\pgfsetstrokecolor{currentstroke}%
\pgfsetdash{}{0pt}%
\pgfsys@defobject{currentmarker}{\pgfqpoint{0.000000in}{-0.027778in}}{\pgfqpoint{0.000000in}{0.000000in}}{%
\pgfpathmoveto{\pgfqpoint{0.000000in}{0.000000in}}%
\pgfpathlineto{\pgfqpoint{0.000000in}{-0.027778in}}%
\pgfusepath{stroke,fill}%
}%
\begin{pgfscope}%
\pgfsys@transformshift{1.423535in}{0.330000in}%
\pgfsys@useobject{currentmarker}{}%
\end{pgfscope}%
\end{pgfscope}%
\begin{pgfscope}%
\pgfsetbuttcap%
\pgfsetroundjoin%
\definecolor{currentfill}{rgb}{0.000000,0.000000,0.000000}%
\pgfsetfillcolor{currentfill}%
\pgfsetlinewidth{0.602250pt}%
\definecolor{currentstroke}{rgb}{0.000000,0.000000,0.000000}%
\pgfsetstrokecolor{currentstroke}%
\pgfsetdash{}{0pt}%
\pgfsys@defobject{currentmarker}{\pgfqpoint{0.000000in}{-0.027778in}}{\pgfqpoint{0.000000in}{0.000000in}}{%
\pgfpathmoveto{\pgfqpoint{0.000000in}{0.000000in}}%
\pgfpathlineto{\pgfqpoint{0.000000in}{-0.027778in}}%
\pgfusepath{stroke,fill}%
}%
\begin{pgfscope}%
\pgfsys@transformshift{1.503284in}{0.330000in}%
\pgfsys@useobject{currentmarker}{}%
\end{pgfscope}%
\end{pgfscope}%
\begin{pgfscope}%
\pgfsetbuttcap%
\pgfsetroundjoin%
\definecolor{currentfill}{rgb}{0.000000,0.000000,0.000000}%
\pgfsetfillcolor{currentfill}%
\pgfsetlinewidth{0.602250pt}%
\definecolor{currentstroke}{rgb}{0.000000,0.000000,0.000000}%
\pgfsetstrokecolor{currentstroke}%
\pgfsetdash{}{0pt}%
\pgfsys@defobject{currentmarker}{\pgfqpoint{0.000000in}{-0.027778in}}{\pgfqpoint{0.000000in}{0.000000in}}{%
\pgfpathmoveto{\pgfqpoint{0.000000in}{0.000000in}}%
\pgfpathlineto{\pgfqpoint{0.000000in}{-0.027778in}}%
\pgfusepath{stroke,fill}%
}%
\begin{pgfscope}%
\pgfsys@transformshift{1.565142in}{0.330000in}%
\pgfsys@useobject{currentmarker}{}%
\end{pgfscope}%
\end{pgfscope}%
\begin{pgfscope}%
\pgfsetbuttcap%
\pgfsetroundjoin%
\definecolor{currentfill}{rgb}{0.000000,0.000000,0.000000}%
\pgfsetfillcolor{currentfill}%
\pgfsetlinewidth{0.602250pt}%
\definecolor{currentstroke}{rgb}{0.000000,0.000000,0.000000}%
\pgfsetstrokecolor{currentstroke}%
\pgfsetdash{}{0pt}%
\pgfsys@defobject{currentmarker}{\pgfqpoint{0.000000in}{-0.027778in}}{\pgfqpoint{0.000000in}{0.000000in}}{%
\pgfpathmoveto{\pgfqpoint{0.000000in}{0.000000in}}%
\pgfpathlineto{\pgfqpoint{0.000000in}{-0.027778in}}%
\pgfusepath{stroke,fill}%
}%
\begin{pgfscope}%
\pgfsys@transformshift{1.615684in}{0.330000in}%
\pgfsys@useobject{currentmarker}{}%
\end{pgfscope}%
\end{pgfscope}%
\begin{pgfscope}%
\pgfsetbuttcap%
\pgfsetroundjoin%
\definecolor{currentfill}{rgb}{0.000000,0.000000,0.000000}%
\pgfsetfillcolor{currentfill}%
\pgfsetlinewidth{0.602250pt}%
\definecolor{currentstroke}{rgb}{0.000000,0.000000,0.000000}%
\pgfsetstrokecolor{currentstroke}%
\pgfsetdash{}{0pt}%
\pgfsys@defobject{currentmarker}{\pgfqpoint{0.000000in}{-0.027778in}}{\pgfqpoint{0.000000in}{0.000000in}}{%
\pgfpathmoveto{\pgfqpoint{0.000000in}{0.000000in}}%
\pgfpathlineto{\pgfqpoint{0.000000in}{-0.027778in}}%
\pgfusepath{stroke,fill}%
}%
\begin{pgfscope}%
\pgfsys@transformshift{1.658416in}{0.330000in}%
\pgfsys@useobject{currentmarker}{}%
\end{pgfscope}%
\end{pgfscope}%
\begin{pgfscope}%
\pgfsetbuttcap%
\pgfsetroundjoin%
\definecolor{currentfill}{rgb}{0.000000,0.000000,0.000000}%
\pgfsetfillcolor{currentfill}%
\pgfsetlinewidth{0.602250pt}%
\definecolor{currentstroke}{rgb}{0.000000,0.000000,0.000000}%
\pgfsetstrokecolor{currentstroke}%
\pgfsetdash{}{0pt}%
\pgfsys@defobject{currentmarker}{\pgfqpoint{0.000000in}{-0.027778in}}{\pgfqpoint{0.000000in}{0.000000in}}{%
\pgfpathmoveto{\pgfqpoint{0.000000in}{0.000000in}}%
\pgfpathlineto{\pgfqpoint{0.000000in}{-0.027778in}}%
\pgfusepath{stroke,fill}%
}%
\begin{pgfscope}%
\pgfsys@transformshift{1.695432in}{0.330000in}%
\pgfsys@useobject{currentmarker}{}%
\end{pgfscope}%
\end{pgfscope}%
\begin{pgfscope}%
\pgfsetbuttcap%
\pgfsetroundjoin%
\definecolor{currentfill}{rgb}{0.000000,0.000000,0.000000}%
\pgfsetfillcolor{currentfill}%
\pgfsetlinewidth{0.602250pt}%
\definecolor{currentstroke}{rgb}{0.000000,0.000000,0.000000}%
\pgfsetstrokecolor{currentstroke}%
\pgfsetdash{}{0pt}%
\pgfsys@defobject{currentmarker}{\pgfqpoint{0.000000in}{-0.027778in}}{\pgfqpoint{0.000000in}{0.000000in}}{%
\pgfpathmoveto{\pgfqpoint{0.000000in}{0.000000in}}%
\pgfpathlineto{\pgfqpoint{0.000000in}{-0.027778in}}%
\pgfusepath{stroke,fill}%
}%
\begin{pgfscope}%
\pgfsys@transformshift{1.728083in}{0.330000in}%
\pgfsys@useobject{currentmarker}{}%
\end{pgfscope}%
\end{pgfscope}%
\begin{pgfscope}%
\pgfsetbuttcap%
\pgfsetroundjoin%
\definecolor{currentfill}{rgb}{0.000000,0.000000,0.000000}%
\pgfsetfillcolor{currentfill}%
\pgfsetlinewidth{0.602250pt}%
\definecolor{currentstroke}{rgb}{0.000000,0.000000,0.000000}%
\pgfsetstrokecolor{currentstroke}%
\pgfsetdash{}{0pt}%
\pgfsys@defobject{currentmarker}{\pgfqpoint{0.000000in}{-0.027778in}}{\pgfqpoint{0.000000in}{0.000000in}}{%
\pgfpathmoveto{\pgfqpoint{0.000000in}{0.000000in}}%
\pgfpathlineto{\pgfqpoint{0.000000in}{-0.027778in}}%
\pgfusepath{stroke,fill}%
}%
\begin{pgfscope}%
\pgfsys@transformshift{1.949439in}{0.330000in}%
\pgfsys@useobject{currentmarker}{}%
\end{pgfscope}%
\end{pgfscope}%
\begin{pgfscope}%
\pgfsetbuttcap%
\pgfsetroundjoin%
\definecolor{currentfill}{rgb}{0.000000,0.000000,0.000000}%
\pgfsetfillcolor{currentfill}%
\pgfsetlinewidth{0.602250pt}%
\definecolor{currentstroke}{rgb}{0.000000,0.000000,0.000000}%
\pgfsetstrokecolor{currentstroke}%
\pgfsetdash{}{0pt}%
\pgfsys@defobject{currentmarker}{\pgfqpoint{0.000000in}{-0.027778in}}{\pgfqpoint{0.000000in}{0.000000in}}{%
\pgfpathmoveto{\pgfqpoint{0.000000in}{0.000000in}}%
\pgfpathlineto{\pgfqpoint{0.000000in}{-0.027778in}}%
\pgfusepath{stroke,fill}%
}%
\begin{pgfscope}%
\pgfsys@transformshift{2.061839in}{0.330000in}%
\pgfsys@useobject{currentmarker}{}%
\end{pgfscope}%
\end{pgfscope}%
\begin{pgfscope}%
\pgfsetbuttcap%
\pgfsetroundjoin%
\definecolor{currentfill}{rgb}{0.000000,0.000000,0.000000}%
\pgfsetfillcolor{currentfill}%
\pgfsetlinewidth{0.602250pt}%
\definecolor{currentstroke}{rgb}{0.000000,0.000000,0.000000}%
\pgfsetstrokecolor{currentstroke}%
\pgfsetdash{}{0pt}%
\pgfsys@defobject{currentmarker}{\pgfqpoint{0.000000in}{-0.027778in}}{\pgfqpoint{0.000000in}{0.000000in}}{%
\pgfpathmoveto{\pgfqpoint{0.000000in}{0.000000in}}%
\pgfpathlineto{\pgfqpoint{0.000000in}{-0.027778in}}%
\pgfusepath{stroke,fill}%
}%
\begin{pgfscope}%
\pgfsys@transformshift{2.141588in}{0.330000in}%
\pgfsys@useobject{currentmarker}{}%
\end{pgfscope}%
\end{pgfscope}%
\begin{pgfscope}%
\pgfsetbuttcap%
\pgfsetroundjoin%
\definecolor{currentfill}{rgb}{0.000000,0.000000,0.000000}%
\pgfsetfillcolor{currentfill}%
\pgfsetlinewidth{0.602250pt}%
\definecolor{currentstroke}{rgb}{0.000000,0.000000,0.000000}%
\pgfsetstrokecolor{currentstroke}%
\pgfsetdash{}{0pt}%
\pgfsys@defobject{currentmarker}{\pgfqpoint{0.000000in}{-0.027778in}}{\pgfqpoint{0.000000in}{0.000000in}}{%
\pgfpathmoveto{\pgfqpoint{0.000000in}{0.000000in}}%
\pgfpathlineto{\pgfqpoint{0.000000in}{-0.027778in}}%
\pgfusepath{stroke,fill}%
}%
\begin{pgfscope}%
\pgfsys@transformshift{2.203446in}{0.330000in}%
\pgfsys@useobject{currentmarker}{}%
\end{pgfscope}%
\end{pgfscope}%
\begin{pgfscope}%
\pgfsetbuttcap%
\pgfsetroundjoin%
\definecolor{currentfill}{rgb}{0.000000,0.000000,0.000000}%
\pgfsetfillcolor{currentfill}%
\pgfsetlinewidth{0.602250pt}%
\definecolor{currentstroke}{rgb}{0.000000,0.000000,0.000000}%
\pgfsetstrokecolor{currentstroke}%
\pgfsetdash{}{0pt}%
\pgfsys@defobject{currentmarker}{\pgfqpoint{0.000000in}{-0.027778in}}{\pgfqpoint{0.000000in}{0.000000in}}{%
\pgfpathmoveto{\pgfqpoint{0.000000in}{0.000000in}}%
\pgfpathlineto{\pgfqpoint{0.000000in}{-0.027778in}}%
\pgfusepath{stroke,fill}%
}%
\begin{pgfscope}%
\pgfsys@transformshift{2.253988in}{0.330000in}%
\pgfsys@useobject{currentmarker}{}%
\end{pgfscope}%
\end{pgfscope}%
\begin{pgfscope}%
\pgfsetbuttcap%
\pgfsetroundjoin%
\definecolor{currentfill}{rgb}{0.000000,0.000000,0.000000}%
\pgfsetfillcolor{currentfill}%
\pgfsetlinewidth{0.602250pt}%
\definecolor{currentstroke}{rgb}{0.000000,0.000000,0.000000}%
\pgfsetstrokecolor{currentstroke}%
\pgfsetdash{}{0pt}%
\pgfsys@defobject{currentmarker}{\pgfqpoint{0.000000in}{-0.027778in}}{\pgfqpoint{0.000000in}{0.000000in}}{%
\pgfpathmoveto{\pgfqpoint{0.000000in}{0.000000in}}%
\pgfpathlineto{\pgfqpoint{0.000000in}{-0.027778in}}%
\pgfusepath{stroke,fill}%
}%
\begin{pgfscope}%
\pgfsys@transformshift{2.296720in}{0.330000in}%
\pgfsys@useobject{currentmarker}{}%
\end{pgfscope}%
\end{pgfscope}%
\begin{pgfscope}%
\pgfsetbuttcap%
\pgfsetroundjoin%
\definecolor{currentfill}{rgb}{0.000000,0.000000,0.000000}%
\pgfsetfillcolor{currentfill}%
\pgfsetlinewidth{0.602250pt}%
\definecolor{currentstroke}{rgb}{0.000000,0.000000,0.000000}%
\pgfsetstrokecolor{currentstroke}%
\pgfsetdash{}{0pt}%
\pgfsys@defobject{currentmarker}{\pgfqpoint{0.000000in}{-0.027778in}}{\pgfqpoint{0.000000in}{0.000000in}}{%
\pgfpathmoveto{\pgfqpoint{0.000000in}{0.000000in}}%
\pgfpathlineto{\pgfqpoint{0.000000in}{-0.027778in}}%
\pgfusepath{stroke,fill}%
}%
\begin{pgfscope}%
\pgfsys@transformshift{2.333737in}{0.330000in}%
\pgfsys@useobject{currentmarker}{}%
\end{pgfscope}%
\end{pgfscope}%
\begin{pgfscope}%
\pgfsetbuttcap%
\pgfsetroundjoin%
\definecolor{currentfill}{rgb}{0.000000,0.000000,0.000000}%
\pgfsetfillcolor{currentfill}%
\pgfsetlinewidth{0.602250pt}%
\definecolor{currentstroke}{rgb}{0.000000,0.000000,0.000000}%
\pgfsetstrokecolor{currentstroke}%
\pgfsetdash{}{0pt}%
\pgfsys@defobject{currentmarker}{\pgfqpoint{0.000000in}{-0.027778in}}{\pgfqpoint{0.000000in}{0.000000in}}{%
\pgfpathmoveto{\pgfqpoint{0.000000in}{0.000000in}}%
\pgfpathlineto{\pgfqpoint{0.000000in}{-0.027778in}}%
\pgfusepath{stroke,fill}%
}%
\begin{pgfscope}%
\pgfsys@transformshift{2.366388in}{0.330000in}%
\pgfsys@useobject{currentmarker}{}%
\end{pgfscope}%
\end{pgfscope}%
\begin{pgfscope}%
\pgfsetbuttcap%
\pgfsetroundjoin%
\definecolor{currentfill}{rgb}{0.000000,0.000000,0.000000}%
\pgfsetfillcolor{currentfill}%
\pgfsetlinewidth{0.602250pt}%
\definecolor{currentstroke}{rgb}{0.000000,0.000000,0.000000}%
\pgfsetstrokecolor{currentstroke}%
\pgfsetdash{}{0pt}%
\pgfsys@defobject{currentmarker}{\pgfqpoint{0.000000in}{-0.027778in}}{\pgfqpoint{0.000000in}{0.000000in}}{%
\pgfpathmoveto{\pgfqpoint{0.000000in}{0.000000in}}%
\pgfpathlineto{\pgfqpoint{0.000000in}{-0.027778in}}%
\pgfusepath{stroke,fill}%
}%
\begin{pgfscope}%
\pgfsys@transformshift{2.587744in}{0.330000in}%
\pgfsys@useobject{currentmarker}{}%
\end{pgfscope}%
\end{pgfscope}%
\begin{pgfscope}%
\pgfsetbuttcap%
\pgfsetroundjoin%
\definecolor{currentfill}{rgb}{0.000000,0.000000,0.000000}%
\pgfsetfillcolor{currentfill}%
\pgfsetlinewidth{0.602250pt}%
\definecolor{currentstroke}{rgb}{0.000000,0.000000,0.000000}%
\pgfsetstrokecolor{currentstroke}%
\pgfsetdash{}{0pt}%
\pgfsys@defobject{currentmarker}{\pgfqpoint{0.000000in}{-0.027778in}}{\pgfqpoint{0.000000in}{0.000000in}}{%
\pgfpathmoveto{\pgfqpoint{0.000000in}{0.000000in}}%
\pgfpathlineto{\pgfqpoint{0.000000in}{-0.027778in}}%
\pgfusepath{stroke,fill}%
}%
\begin{pgfscope}%
\pgfsys@transformshift{2.700143in}{0.330000in}%
\pgfsys@useobject{currentmarker}{}%
\end{pgfscope}%
\end{pgfscope}%
\begin{pgfscope}%
\definecolor{textcolor}{rgb}{0.000000,0.000000,0.000000}%
\pgfsetstrokecolor{textcolor}%
\pgfsetfillcolor{textcolor}%
\pgftext[x=1.537500in,y=0.042809in,,top]{\color{textcolor}\sffamily\fontsize{10.000000}{12.000000}\selectfont \(\displaystyle N_\mathrm{QMC}\)}%
\end{pgfscope}%
\begin{pgfscope}%
\pgfsetbuttcap%
\pgfsetroundjoin%
\definecolor{currentfill}{rgb}{0.000000,0.000000,0.000000}%
\pgfsetfillcolor{currentfill}%
\pgfsetlinewidth{0.803000pt}%
\definecolor{currentstroke}{rgb}{0.000000,0.000000,0.000000}%
\pgfsetstrokecolor{currentstroke}%
\pgfsetdash{}{0pt}%
\pgfsys@defobject{currentmarker}{\pgfqpoint{-0.048611in}{0.000000in}}{\pgfqpoint{0.000000in}{0.000000in}}{%
\pgfpathmoveto{\pgfqpoint{0.000000in}{0.000000in}}%
\pgfpathlineto{\pgfqpoint{-0.048611in}{0.000000in}}%
\pgfusepath{stroke,fill}%
}%
\begin{pgfscope}%
\pgfsys@transformshift{0.375000in}{0.393986in}%
\pgfsys@useobject{currentmarker}{}%
\end{pgfscope}%
\end{pgfscope}%
\begin{pgfscope}%
\definecolor{textcolor}{rgb}{0.000000,0.000000,0.000000}%
\pgfsetstrokecolor{textcolor}%
\pgfsetfillcolor{textcolor}%
\pgftext[x=-0.010225in,y=0.341224in,left,base]{\color{textcolor}\sffamily\fontsize{10.000000}{12.000000}\selectfont \(\displaystyle {10^{-3}}\)}%
\end{pgfscope}%
\begin{pgfscope}%
\pgfsetbuttcap%
\pgfsetroundjoin%
\definecolor{currentfill}{rgb}{0.000000,0.000000,0.000000}%
\pgfsetfillcolor{currentfill}%
\pgfsetlinewidth{0.803000pt}%
\definecolor{currentstroke}{rgb}{0.000000,0.000000,0.000000}%
\pgfsetstrokecolor{currentstroke}%
\pgfsetdash{}{0pt}%
\pgfsys@defobject{currentmarker}{\pgfqpoint{-0.048611in}{0.000000in}}{\pgfqpoint{0.000000in}{0.000000in}}{%
\pgfpathmoveto{\pgfqpoint{0.000000in}{0.000000in}}%
\pgfpathlineto{\pgfqpoint{-0.048611in}{0.000000in}}%
\pgfusepath{stroke,fill}%
}%
\begin{pgfscope}%
\pgfsys@transformshift{0.375000in}{1.087408in}%
\pgfsys@useobject{currentmarker}{}%
\end{pgfscope}%
\end{pgfscope}%
\begin{pgfscope}%
\definecolor{textcolor}{rgb}{0.000000,0.000000,0.000000}%
\pgfsetstrokecolor{textcolor}%
\pgfsetfillcolor{textcolor}%
\pgftext[x=-0.010225in,y=1.034647in,left,base]{\color{textcolor}\sffamily\fontsize{10.000000}{12.000000}\selectfont \(\displaystyle {10^{-2}}\)}%
\end{pgfscope}%
\begin{pgfscope}%
\pgfsetbuttcap%
\pgfsetroundjoin%
\definecolor{currentfill}{rgb}{0.000000,0.000000,0.000000}%
\pgfsetfillcolor{currentfill}%
\pgfsetlinewidth{0.803000pt}%
\definecolor{currentstroke}{rgb}{0.000000,0.000000,0.000000}%
\pgfsetstrokecolor{currentstroke}%
\pgfsetdash{}{0pt}%
\pgfsys@defobject{currentmarker}{\pgfqpoint{-0.048611in}{0.000000in}}{\pgfqpoint{0.000000in}{0.000000in}}{%
\pgfpathmoveto{\pgfqpoint{0.000000in}{0.000000in}}%
\pgfpathlineto{\pgfqpoint{-0.048611in}{0.000000in}}%
\pgfusepath{stroke,fill}%
}%
\begin{pgfscope}%
\pgfsys@transformshift{0.375000in}{1.780830in}%
\pgfsys@useobject{currentmarker}{}%
\end{pgfscope}%
\end{pgfscope}%
\begin{pgfscope}%
\definecolor{textcolor}{rgb}{0.000000,0.000000,0.000000}%
\pgfsetstrokecolor{textcolor}%
\pgfsetfillcolor{textcolor}%
\pgftext[x=-0.010225in,y=1.728069in,left,base]{\color{textcolor}\sffamily\fontsize{10.000000}{12.000000}\selectfont \(\displaystyle {10^{-1}}\)}%
\end{pgfscope}%
\begin{pgfscope}%
\pgfsetbuttcap%
\pgfsetroundjoin%
\definecolor{currentfill}{rgb}{0.000000,0.000000,0.000000}%
\pgfsetfillcolor{currentfill}%
\pgfsetlinewidth{0.803000pt}%
\definecolor{currentstroke}{rgb}{0.000000,0.000000,0.000000}%
\pgfsetstrokecolor{currentstroke}%
\pgfsetdash{}{0pt}%
\pgfsys@defobject{currentmarker}{\pgfqpoint{-0.048611in}{0.000000in}}{\pgfqpoint{0.000000in}{0.000000in}}{%
\pgfpathmoveto{\pgfqpoint{0.000000in}{0.000000in}}%
\pgfpathlineto{\pgfqpoint{-0.048611in}{0.000000in}}%
\pgfusepath{stroke,fill}%
}%
\begin{pgfscope}%
\pgfsys@transformshift{0.375000in}{2.474253in}%
\pgfsys@useobject{currentmarker}{}%
\end{pgfscope}%
\end{pgfscope}%
\begin{pgfscope}%
\definecolor{textcolor}{rgb}{0.000000,0.000000,0.000000}%
\pgfsetstrokecolor{textcolor}%
\pgfsetfillcolor{textcolor}%
\pgftext[x=0.076581in,y=2.421491in,left,base]{\color{textcolor}\sffamily\fontsize{10.000000}{12.000000}\selectfont \(\displaystyle {10^{0}}\)}%
\end{pgfscope}%
\begin{pgfscope}%
\pgfsetbuttcap%
\pgfsetroundjoin%
\definecolor{currentfill}{rgb}{0.000000,0.000000,0.000000}%
\pgfsetfillcolor{currentfill}%
\pgfsetlinewidth{0.602250pt}%
\definecolor{currentstroke}{rgb}{0.000000,0.000000,0.000000}%
\pgfsetstrokecolor{currentstroke}%
\pgfsetdash{}{0pt}%
\pgfsys@defobject{currentmarker}{\pgfqpoint{-0.027778in}{0.000000in}}{\pgfqpoint{0.000000in}{0.000000in}}{%
\pgfpathmoveto{\pgfqpoint{0.000000in}{0.000000in}}%
\pgfpathlineto{\pgfqpoint{-0.027778in}{0.000000in}}%
\pgfusepath{stroke,fill}%
}%
\begin{pgfscope}%
\pgfsys@transformshift{0.375000in}{0.326786in}%
\pgfsys@useobject{currentmarker}{}%
\end{pgfscope}%
\end{pgfscope}%
\begin{pgfscope}%
\pgfsetbuttcap%
\pgfsetroundjoin%
\definecolor{currentfill}{rgb}{0.000000,0.000000,0.000000}%
\pgfsetfillcolor{currentfill}%
\pgfsetlinewidth{0.602250pt}%
\definecolor{currentstroke}{rgb}{0.000000,0.000000,0.000000}%
\pgfsetstrokecolor{currentstroke}%
\pgfsetdash{}{0pt}%
\pgfsys@defobject{currentmarker}{\pgfqpoint{-0.027778in}{0.000000in}}{\pgfqpoint{0.000000in}{0.000000in}}{%
\pgfpathmoveto{\pgfqpoint{0.000000in}{0.000000in}}%
\pgfpathlineto{\pgfqpoint{-0.027778in}{0.000000in}}%
\pgfusepath{stroke,fill}%
}%
\begin{pgfscope}%
\pgfsys@transformshift{0.375000in}{0.362257in}%
\pgfsys@useobject{currentmarker}{}%
\end{pgfscope}%
\end{pgfscope}%
\begin{pgfscope}%
\pgfsetbuttcap%
\pgfsetroundjoin%
\definecolor{currentfill}{rgb}{0.000000,0.000000,0.000000}%
\pgfsetfillcolor{currentfill}%
\pgfsetlinewidth{0.602250pt}%
\definecolor{currentstroke}{rgb}{0.000000,0.000000,0.000000}%
\pgfsetstrokecolor{currentstroke}%
\pgfsetdash{}{0pt}%
\pgfsys@defobject{currentmarker}{\pgfqpoint{-0.027778in}{0.000000in}}{\pgfqpoint{0.000000in}{0.000000in}}{%
\pgfpathmoveto{\pgfqpoint{0.000000in}{0.000000in}}%
\pgfpathlineto{\pgfqpoint{-0.027778in}{0.000000in}}%
\pgfusepath{stroke,fill}%
}%
\begin{pgfscope}%
\pgfsys@transformshift{0.375000in}{0.602727in}%
\pgfsys@useobject{currentmarker}{}%
\end{pgfscope}%
\end{pgfscope}%
\begin{pgfscope}%
\pgfsetbuttcap%
\pgfsetroundjoin%
\definecolor{currentfill}{rgb}{0.000000,0.000000,0.000000}%
\pgfsetfillcolor{currentfill}%
\pgfsetlinewidth{0.602250pt}%
\definecolor{currentstroke}{rgb}{0.000000,0.000000,0.000000}%
\pgfsetstrokecolor{currentstroke}%
\pgfsetdash{}{0pt}%
\pgfsys@defobject{currentmarker}{\pgfqpoint{-0.027778in}{0.000000in}}{\pgfqpoint{0.000000in}{0.000000in}}{%
\pgfpathmoveto{\pgfqpoint{0.000000in}{0.000000in}}%
\pgfpathlineto{\pgfqpoint{-0.027778in}{0.000000in}}%
\pgfusepath{stroke,fill}%
}%
\begin{pgfscope}%
\pgfsys@transformshift{0.375000in}{0.724832in}%
\pgfsys@useobject{currentmarker}{}%
\end{pgfscope}%
\end{pgfscope}%
\begin{pgfscope}%
\pgfsetbuttcap%
\pgfsetroundjoin%
\definecolor{currentfill}{rgb}{0.000000,0.000000,0.000000}%
\pgfsetfillcolor{currentfill}%
\pgfsetlinewidth{0.602250pt}%
\definecolor{currentstroke}{rgb}{0.000000,0.000000,0.000000}%
\pgfsetstrokecolor{currentstroke}%
\pgfsetdash{}{0pt}%
\pgfsys@defobject{currentmarker}{\pgfqpoint{-0.027778in}{0.000000in}}{\pgfqpoint{0.000000in}{0.000000in}}{%
\pgfpathmoveto{\pgfqpoint{0.000000in}{0.000000in}}%
\pgfpathlineto{\pgfqpoint{-0.027778in}{0.000000in}}%
\pgfusepath{stroke,fill}%
}%
\begin{pgfscope}%
\pgfsys@transformshift{0.375000in}{0.811468in}%
\pgfsys@useobject{currentmarker}{}%
\end{pgfscope}%
\end{pgfscope}%
\begin{pgfscope}%
\pgfsetbuttcap%
\pgfsetroundjoin%
\definecolor{currentfill}{rgb}{0.000000,0.000000,0.000000}%
\pgfsetfillcolor{currentfill}%
\pgfsetlinewidth{0.602250pt}%
\definecolor{currentstroke}{rgb}{0.000000,0.000000,0.000000}%
\pgfsetstrokecolor{currentstroke}%
\pgfsetdash{}{0pt}%
\pgfsys@defobject{currentmarker}{\pgfqpoint{-0.027778in}{0.000000in}}{\pgfqpoint{0.000000in}{0.000000in}}{%
\pgfpathmoveto{\pgfqpoint{0.000000in}{0.000000in}}%
\pgfpathlineto{\pgfqpoint{-0.027778in}{0.000000in}}%
\pgfusepath{stroke,fill}%
}%
\begin{pgfscope}%
\pgfsys@transformshift{0.375000in}{0.878667in}%
\pgfsys@useobject{currentmarker}{}%
\end{pgfscope}%
\end{pgfscope}%
\begin{pgfscope}%
\pgfsetbuttcap%
\pgfsetroundjoin%
\definecolor{currentfill}{rgb}{0.000000,0.000000,0.000000}%
\pgfsetfillcolor{currentfill}%
\pgfsetlinewidth{0.602250pt}%
\definecolor{currentstroke}{rgb}{0.000000,0.000000,0.000000}%
\pgfsetstrokecolor{currentstroke}%
\pgfsetdash{}{0pt}%
\pgfsys@defobject{currentmarker}{\pgfqpoint{-0.027778in}{0.000000in}}{\pgfqpoint{0.000000in}{0.000000in}}{%
\pgfpathmoveto{\pgfqpoint{0.000000in}{0.000000in}}%
\pgfpathlineto{\pgfqpoint{-0.027778in}{0.000000in}}%
\pgfusepath{stroke,fill}%
}%
\begin{pgfscope}%
\pgfsys@transformshift{0.375000in}{0.933573in}%
\pgfsys@useobject{currentmarker}{}%
\end{pgfscope}%
\end{pgfscope}%
\begin{pgfscope}%
\pgfsetbuttcap%
\pgfsetroundjoin%
\definecolor{currentfill}{rgb}{0.000000,0.000000,0.000000}%
\pgfsetfillcolor{currentfill}%
\pgfsetlinewidth{0.602250pt}%
\definecolor{currentstroke}{rgb}{0.000000,0.000000,0.000000}%
\pgfsetstrokecolor{currentstroke}%
\pgfsetdash{}{0pt}%
\pgfsys@defobject{currentmarker}{\pgfqpoint{-0.027778in}{0.000000in}}{\pgfqpoint{0.000000in}{0.000000in}}{%
\pgfpathmoveto{\pgfqpoint{0.000000in}{0.000000in}}%
\pgfpathlineto{\pgfqpoint{-0.027778in}{0.000000in}}%
\pgfusepath{stroke,fill}%
}%
\begin{pgfscope}%
\pgfsys@transformshift{0.375000in}{0.979996in}%
\pgfsys@useobject{currentmarker}{}%
\end{pgfscope}%
\end{pgfscope}%
\begin{pgfscope}%
\pgfsetbuttcap%
\pgfsetroundjoin%
\definecolor{currentfill}{rgb}{0.000000,0.000000,0.000000}%
\pgfsetfillcolor{currentfill}%
\pgfsetlinewidth{0.602250pt}%
\definecolor{currentstroke}{rgb}{0.000000,0.000000,0.000000}%
\pgfsetstrokecolor{currentstroke}%
\pgfsetdash{}{0pt}%
\pgfsys@defobject{currentmarker}{\pgfqpoint{-0.027778in}{0.000000in}}{\pgfqpoint{0.000000in}{0.000000in}}{%
\pgfpathmoveto{\pgfqpoint{0.000000in}{0.000000in}}%
\pgfpathlineto{\pgfqpoint{-0.027778in}{0.000000in}}%
\pgfusepath{stroke,fill}%
}%
\begin{pgfscope}%
\pgfsys@transformshift{0.375000in}{1.020209in}%
\pgfsys@useobject{currentmarker}{}%
\end{pgfscope}%
\end{pgfscope}%
\begin{pgfscope}%
\pgfsetbuttcap%
\pgfsetroundjoin%
\definecolor{currentfill}{rgb}{0.000000,0.000000,0.000000}%
\pgfsetfillcolor{currentfill}%
\pgfsetlinewidth{0.602250pt}%
\definecolor{currentstroke}{rgb}{0.000000,0.000000,0.000000}%
\pgfsetstrokecolor{currentstroke}%
\pgfsetdash{}{0pt}%
\pgfsys@defobject{currentmarker}{\pgfqpoint{-0.027778in}{0.000000in}}{\pgfqpoint{0.000000in}{0.000000in}}{%
\pgfpathmoveto{\pgfqpoint{0.000000in}{0.000000in}}%
\pgfpathlineto{\pgfqpoint{-0.027778in}{0.000000in}}%
\pgfusepath{stroke,fill}%
}%
\begin{pgfscope}%
\pgfsys@transformshift{0.375000in}{1.055679in}%
\pgfsys@useobject{currentmarker}{}%
\end{pgfscope}%
\end{pgfscope}%
\begin{pgfscope}%
\pgfsetbuttcap%
\pgfsetroundjoin%
\definecolor{currentfill}{rgb}{0.000000,0.000000,0.000000}%
\pgfsetfillcolor{currentfill}%
\pgfsetlinewidth{0.602250pt}%
\definecolor{currentstroke}{rgb}{0.000000,0.000000,0.000000}%
\pgfsetstrokecolor{currentstroke}%
\pgfsetdash{}{0pt}%
\pgfsys@defobject{currentmarker}{\pgfqpoint{-0.027778in}{0.000000in}}{\pgfqpoint{0.000000in}{0.000000in}}{%
\pgfpathmoveto{\pgfqpoint{0.000000in}{0.000000in}}%
\pgfpathlineto{\pgfqpoint{-0.027778in}{0.000000in}}%
\pgfusepath{stroke,fill}%
}%
\begin{pgfscope}%
\pgfsys@transformshift{0.375000in}{1.296149in}%
\pgfsys@useobject{currentmarker}{}%
\end{pgfscope}%
\end{pgfscope}%
\begin{pgfscope}%
\pgfsetbuttcap%
\pgfsetroundjoin%
\definecolor{currentfill}{rgb}{0.000000,0.000000,0.000000}%
\pgfsetfillcolor{currentfill}%
\pgfsetlinewidth{0.602250pt}%
\definecolor{currentstroke}{rgb}{0.000000,0.000000,0.000000}%
\pgfsetstrokecolor{currentstroke}%
\pgfsetdash{}{0pt}%
\pgfsys@defobject{currentmarker}{\pgfqpoint{-0.027778in}{0.000000in}}{\pgfqpoint{0.000000in}{0.000000in}}{%
\pgfpathmoveto{\pgfqpoint{0.000000in}{0.000000in}}%
\pgfpathlineto{\pgfqpoint{-0.027778in}{0.000000in}}%
\pgfusepath{stroke,fill}%
}%
\begin{pgfscope}%
\pgfsys@transformshift{0.375000in}{1.418255in}%
\pgfsys@useobject{currentmarker}{}%
\end{pgfscope}%
\end{pgfscope}%
\begin{pgfscope}%
\pgfsetbuttcap%
\pgfsetroundjoin%
\definecolor{currentfill}{rgb}{0.000000,0.000000,0.000000}%
\pgfsetfillcolor{currentfill}%
\pgfsetlinewidth{0.602250pt}%
\definecolor{currentstroke}{rgb}{0.000000,0.000000,0.000000}%
\pgfsetstrokecolor{currentstroke}%
\pgfsetdash{}{0pt}%
\pgfsys@defobject{currentmarker}{\pgfqpoint{-0.027778in}{0.000000in}}{\pgfqpoint{0.000000in}{0.000000in}}{%
\pgfpathmoveto{\pgfqpoint{0.000000in}{0.000000in}}%
\pgfpathlineto{\pgfqpoint{-0.027778in}{0.000000in}}%
\pgfusepath{stroke,fill}%
}%
\begin{pgfscope}%
\pgfsys@transformshift{0.375000in}{1.504890in}%
\pgfsys@useobject{currentmarker}{}%
\end{pgfscope}%
\end{pgfscope}%
\begin{pgfscope}%
\pgfsetbuttcap%
\pgfsetroundjoin%
\definecolor{currentfill}{rgb}{0.000000,0.000000,0.000000}%
\pgfsetfillcolor{currentfill}%
\pgfsetlinewidth{0.602250pt}%
\definecolor{currentstroke}{rgb}{0.000000,0.000000,0.000000}%
\pgfsetstrokecolor{currentstroke}%
\pgfsetdash{}{0pt}%
\pgfsys@defobject{currentmarker}{\pgfqpoint{-0.027778in}{0.000000in}}{\pgfqpoint{0.000000in}{0.000000in}}{%
\pgfpathmoveto{\pgfqpoint{0.000000in}{0.000000in}}%
\pgfpathlineto{\pgfqpoint{-0.027778in}{0.000000in}}%
\pgfusepath{stroke,fill}%
}%
\begin{pgfscope}%
\pgfsys@transformshift{0.375000in}{1.572090in}%
\pgfsys@useobject{currentmarker}{}%
\end{pgfscope}%
\end{pgfscope}%
\begin{pgfscope}%
\pgfsetbuttcap%
\pgfsetroundjoin%
\definecolor{currentfill}{rgb}{0.000000,0.000000,0.000000}%
\pgfsetfillcolor{currentfill}%
\pgfsetlinewidth{0.602250pt}%
\definecolor{currentstroke}{rgb}{0.000000,0.000000,0.000000}%
\pgfsetstrokecolor{currentstroke}%
\pgfsetdash{}{0pt}%
\pgfsys@defobject{currentmarker}{\pgfqpoint{-0.027778in}{0.000000in}}{\pgfqpoint{0.000000in}{0.000000in}}{%
\pgfpathmoveto{\pgfqpoint{0.000000in}{0.000000in}}%
\pgfpathlineto{\pgfqpoint{-0.027778in}{0.000000in}}%
\pgfusepath{stroke,fill}%
}%
\begin{pgfscope}%
\pgfsys@transformshift{0.375000in}{1.626996in}%
\pgfsys@useobject{currentmarker}{}%
\end{pgfscope}%
\end{pgfscope}%
\begin{pgfscope}%
\pgfsetbuttcap%
\pgfsetroundjoin%
\definecolor{currentfill}{rgb}{0.000000,0.000000,0.000000}%
\pgfsetfillcolor{currentfill}%
\pgfsetlinewidth{0.602250pt}%
\definecolor{currentstroke}{rgb}{0.000000,0.000000,0.000000}%
\pgfsetstrokecolor{currentstroke}%
\pgfsetdash{}{0pt}%
\pgfsys@defobject{currentmarker}{\pgfqpoint{-0.027778in}{0.000000in}}{\pgfqpoint{0.000000in}{0.000000in}}{%
\pgfpathmoveto{\pgfqpoint{0.000000in}{0.000000in}}%
\pgfpathlineto{\pgfqpoint{-0.027778in}{0.000000in}}%
\pgfusepath{stroke,fill}%
}%
\begin{pgfscope}%
\pgfsys@transformshift{0.375000in}{1.673418in}%
\pgfsys@useobject{currentmarker}{}%
\end{pgfscope}%
\end{pgfscope}%
\begin{pgfscope}%
\pgfsetbuttcap%
\pgfsetroundjoin%
\definecolor{currentfill}{rgb}{0.000000,0.000000,0.000000}%
\pgfsetfillcolor{currentfill}%
\pgfsetlinewidth{0.602250pt}%
\definecolor{currentstroke}{rgb}{0.000000,0.000000,0.000000}%
\pgfsetstrokecolor{currentstroke}%
\pgfsetdash{}{0pt}%
\pgfsys@defobject{currentmarker}{\pgfqpoint{-0.027778in}{0.000000in}}{\pgfqpoint{0.000000in}{0.000000in}}{%
\pgfpathmoveto{\pgfqpoint{0.000000in}{0.000000in}}%
\pgfpathlineto{\pgfqpoint{-0.027778in}{0.000000in}}%
\pgfusepath{stroke,fill}%
}%
\begin{pgfscope}%
\pgfsys@transformshift{0.375000in}{1.713631in}%
\pgfsys@useobject{currentmarker}{}%
\end{pgfscope}%
\end{pgfscope}%
\begin{pgfscope}%
\pgfsetbuttcap%
\pgfsetroundjoin%
\definecolor{currentfill}{rgb}{0.000000,0.000000,0.000000}%
\pgfsetfillcolor{currentfill}%
\pgfsetlinewidth{0.602250pt}%
\definecolor{currentstroke}{rgb}{0.000000,0.000000,0.000000}%
\pgfsetstrokecolor{currentstroke}%
\pgfsetdash{}{0pt}%
\pgfsys@defobject{currentmarker}{\pgfqpoint{-0.027778in}{0.000000in}}{\pgfqpoint{0.000000in}{0.000000in}}{%
\pgfpathmoveto{\pgfqpoint{0.000000in}{0.000000in}}%
\pgfpathlineto{\pgfqpoint{-0.027778in}{0.000000in}}%
\pgfusepath{stroke,fill}%
}%
\begin{pgfscope}%
\pgfsys@transformshift{0.375000in}{1.749101in}%
\pgfsys@useobject{currentmarker}{}%
\end{pgfscope}%
\end{pgfscope}%
\begin{pgfscope}%
\pgfsetbuttcap%
\pgfsetroundjoin%
\definecolor{currentfill}{rgb}{0.000000,0.000000,0.000000}%
\pgfsetfillcolor{currentfill}%
\pgfsetlinewidth{0.602250pt}%
\definecolor{currentstroke}{rgb}{0.000000,0.000000,0.000000}%
\pgfsetstrokecolor{currentstroke}%
\pgfsetdash{}{0pt}%
\pgfsys@defobject{currentmarker}{\pgfqpoint{-0.027778in}{0.000000in}}{\pgfqpoint{0.000000in}{0.000000in}}{%
\pgfpathmoveto{\pgfqpoint{0.000000in}{0.000000in}}%
\pgfpathlineto{\pgfqpoint{-0.027778in}{0.000000in}}%
\pgfusepath{stroke,fill}%
}%
\begin{pgfscope}%
\pgfsys@transformshift{0.375000in}{1.989571in}%
\pgfsys@useobject{currentmarker}{}%
\end{pgfscope}%
\end{pgfscope}%
\begin{pgfscope}%
\pgfsetbuttcap%
\pgfsetroundjoin%
\definecolor{currentfill}{rgb}{0.000000,0.000000,0.000000}%
\pgfsetfillcolor{currentfill}%
\pgfsetlinewidth{0.602250pt}%
\definecolor{currentstroke}{rgb}{0.000000,0.000000,0.000000}%
\pgfsetstrokecolor{currentstroke}%
\pgfsetdash{}{0pt}%
\pgfsys@defobject{currentmarker}{\pgfqpoint{-0.027778in}{0.000000in}}{\pgfqpoint{0.000000in}{0.000000in}}{%
\pgfpathmoveto{\pgfqpoint{0.000000in}{0.000000in}}%
\pgfpathlineto{\pgfqpoint{-0.027778in}{0.000000in}}%
\pgfusepath{stroke,fill}%
}%
\begin{pgfscope}%
\pgfsys@transformshift{0.375000in}{2.111677in}%
\pgfsys@useobject{currentmarker}{}%
\end{pgfscope}%
\end{pgfscope}%
\begin{pgfscope}%
\pgfsetbuttcap%
\pgfsetroundjoin%
\definecolor{currentfill}{rgb}{0.000000,0.000000,0.000000}%
\pgfsetfillcolor{currentfill}%
\pgfsetlinewidth{0.602250pt}%
\definecolor{currentstroke}{rgb}{0.000000,0.000000,0.000000}%
\pgfsetstrokecolor{currentstroke}%
\pgfsetdash{}{0pt}%
\pgfsys@defobject{currentmarker}{\pgfqpoint{-0.027778in}{0.000000in}}{\pgfqpoint{0.000000in}{0.000000in}}{%
\pgfpathmoveto{\pgfqpoint{0.000000in}{0.000000in}}%
\pgfpathlineto{\pgfqpoint{-0.027778in}{0.000000in}}%
\pgfusepath{stroke,fill}%
}%
\begin{pgfscope}%
\pgfsys@transformshift{0.375000in}{2.198312in}%
\pgfsys@useobject{currentmarker}{}%
\end{pgfscope}%
\end{pgfscope}%
\begin{pgfscope}%
\pgfsetbuttcap%
\pgfsetroundjoin%
\definecolor{currentfill}{rgb}{0.000000,0.000000,0.000000}%
\pgfsetfillcolor{currentfill}%
\pgfsetlinewidth{0.602250pt}%
\definecolor{currentstroke}{rgb}{0.000000,0.000000,0.000000}%
\pgfsetstrokecolor{currentstroke}%
\pgfsetdash{}{0pt}%
\pgfsys@defobject{currentmarker}{\pgfqpoint{-0.027778in}{0.000000in}}{\pgfqpoint{0.000000in}{0.000000in}}{%
\pgfpathmoveto{\pgfqpoint{0.000000in}{0.000000in}}%
\pgfpathlineto{\pgfqpoint{-0.027778in}{0.000000in}}%
\pgfusepath{stroke,fill}%
}%
\begin{pgfscope}%
\pgfsys@transformshift{0.375000in}{2.265512in}%
\pgfsys@useobject{currentmarker}{}%
\end{pgfscope}%
\end{pgfscope}%
\begin{pgfscope}%
\pgfsetbuttcap%
\pgfsetroundjoin%
\definecolor{currentfill}{rgb}{0.000000,0.000000,0.000000}%
\pgfsetfillcolor{currentfill}%
\pgfsetlinewidth{0.602250pt}%
\definecolor{currentstroke}{rgb}{0.000000,0.000000,0.000000}%
\pgfsetstrokecolor{currentstroke}%
\pgfsetdash{}{0pt}%
\pgfsys@defobject{currentmarker}{\pgfqpoint{-0.027778in}{0.000000in}}{\pgfqpoint{0.000000in}{0.000000in}}{%
\pgfpathmoveto{\pgfqpoint{0.000000in}{0.000000in}}%
\pgfpathlineto{\pgfqpoint{-0.027778in}{0.000000in}}%
\pgfusepath{stroke,fill}%
}%
\begin{pgfscope}%
\pgfsys@transformshift{0.375000in}{2.320418in}%
\pgfsys@useobject{currentmarker}{}%
\end{pgfscope}%
\end{pgfscope}%
\begin{pgfscope}%
\pgfsetbuttcap%
\pgfsetroundjoin%
\definecolor{currentfill}{rgb}{0.000000,0.000000,0.000000}%
\pgfsetfillcolor{currentfill}%
\pgfsetlinewidth{0.602250pt}%
\definecolor{currentstroke}{rgb}{0.000000,0.000000,0.000000}%
\pgfsetstrokecolor{currentstroke}%
\pgfsetdash{}{0pt}%
\pgfsys@defobject{currentmarker}{\pgfqpoint{-0.027778in}{0.000000in}}{\pgfqpoint{0.000000in}{0.000000in}}{%
\pgfpathmoveto{\pgfqpoint{0.000000in}{0.000000in}}%
\pgfpathlineto{\pgfqpoint{-0.027778in}{0.000000in}}%
\pgfusepath{stroke,fill}%
}%
\begin{pgfscope}%
\pgfsys@transformshift{0.375000in}{2.366840in}%
\pgfsys@useobject{currentmarker}{}%
\end{pgfscope}%
\end{pgfscope}%
\begin{pgfscope}%
\pgfsetbuttcap%
\pgfsetroundjoin%
\definecolor{currentfill}{rgb}{0.000000,0.000000,0.000000}%
\pgfsetfillcolor{currentfill}%
\pgfsetlinewidth{0.602250pt}%
\definecolor{currentstroke}{rgb}{0.000000,0.000000,0.000000}%
\pgfsetstrokecolor{currentstroke}%
\pgfsetdash{}{0pt}%
\pgfsys@defobject{currentmarker}{\pgfqpoint{-0.027778in}{0.000000in}}{\pgfqpoint{0.000000in}{0.000000in}}{%
\pgfpathmoveto{\pgfqpoint{0.000000in}{0.000000in}}%
\pgfpathlineto{\pgfqpoint{-0.027778in}{0.000000in}}%
\pgfusepath{stroke,fill}%
}%
\begin{pgfscope}%
\pgfsys@transformshift{0.375000in}{2.407053in}%
\pgfsys@useobject{currentmarker}{}%
\end{pgfscope}%
\end{pgfscope}%
\begin{pgfscope}%
\pgfsetbuttcap%
\pgfsetroundjoin%
\definecolor{currentfill}{rgb}{0.000000,0.000000,0.000000}%
\pgfsetfillcolor{currentfill}%
\pgfsetlinewidth{0.602250pt}%
\definecolor{currentstroke}{rgb}{0.000000,0.000000,0.000000}%
\pgfsetstrokecolor{currentstroke}%
\pgfsetdash{}{0pt}%
\pgfsys@defobject{currentmarker}{\pgfqpoint{-0.027778in}{0.000000in}}{\pgfqpoint{0.000000in}{0.000000in}}{%
\pgfpathmoveto{\pgfqpoint{0.000000in}{0.000000in}}%
\pgfpathlineto{\pgfqpoint{-0.027778in}{0.000000in}}%
\pgfusepath{stroke,fill}%
}%
\begin{pgfscope}%
\pgfsys@transformshift{0.375000in}{2.442524in}%
\pgfsys@useobject{currentmarker}{}%
\end{pgfscope}%
\end{pgfscope}%
\begin{pgfscope}%
\definecolor{textcolor}{rgb}{0.000000,0.000000,0.000000}%
\pgfsetstrokecolor{textcolor}%
\pgfsetfillcolor{textcolor}%
\pgftext[x=-0.065780in,y=1.485000in,,bottom,rotate=90.000000]{\color{textcolor}\sffamily\fontsize{10.000000}{12.000000}\selectfont QMC Error}%
\end{pgfscope}%
\begin{pgfscope}%
\pgfpathrectangle{\pgfqpoint{0.375000in}{0.330000in}}{\pgfqpoint{2.325000in}{2.310000in}}%
\pgfusepath{clip}%
\pgfsetbuttcap%
\pgfsetroundjoin%
\definecolor{currentfill}{rgb}{0.000000,0.000000,0.000000}%
\pgfsetfillcolor{currentfill}%
\pgfsetlinewidth{1.003750pt}%
\definecolor{currentstroke}{rgb}{0.000000,0.000000,0.000000}%
\pgfsetstrokecolor{currentstroke}%
\pgfsetdash{}{0pt}%
\pgfsys@defobject{currentmarker}{\pgfqpoint{-0.041667in}{-0.041667in}}{\pgfqpoint{0.041667in}{0.041667in}}{%
\pgfpathmoveto{\pgfqpoint{0.000000in}{-0.041667in}}%
\pgfpathcurveto{\pgfqpoint{0.011050in}{-0.041667in}}{\pgfqpoint{0.021649in}{-0.037276in}}{\pgfqpoint{0.029463in}{-0.029463in}}%
\pgfpathcurveto{\pgfqpoint{0.037276in}{-0.021649in}}{\pgfqpoint{0.041667in}{-0.011050in}}{\pgfqpoint{0.041667in}{0.000000in}}%
\pgfpathcurveto{\pgfqpoint{0.041667in}{0.011050in}}{\pgfqpoint{0.037276in}{0.021649in}}{\pgfqpoint{0.029463in}{0.029463in}}%
\pgfpathcurveto{\pgfqpoint{0.021649in}{0.037276in}}{\pgfqpoint{0.011050in}{0.041667in}}{\pgfqpoint{0.000000in}{0.041667in}}%
\pgfpathcurveto{\pgfqpoint{-0.011050in}{0.041667in}}{\pgfqpoint{-0.021649in}{0.037276in}}{\pgfqpoint{-0.029463in}{0.029463in}}%
\pgfpathcurveto{\pgfqpoint{-0.037276in}{0.021649in}}{\pgfqpoint{-0.041667in}{0.011050in}}{\pgfqpoint{-0.041667in}{0.000000in}}%
\pgfpathcurveto{\pgfqpoint{-0.041667in}{-0.011050in}}{\pgfqpoint{-0.037276in}{-0.021649in}}{\pgfqpoint{-0.029463in}{-0.029463in}}%
\pgfpathcurveto{\pgfqpoint{-0.021649in}{-0.037276in}}{\pgfqpoint{-0.011050in}{-0.041667in}}{\pgfqpoint{0.000000in}{-0.041667in}}%
\pgfpathclose%
\pgfusepath{stroke,fill}%
}%
\begin{pgfscope}%
\pgfsys@transformshift{0.480682in}{2.356258in}%
\pgfsys@useobject{currentmarker}{}%
\end{pgfscope}%
\begin{pgfscope}%
\pgfsys@transformshift{0.672831in}{2.420509in}%
\pgfsys@useobject{currentmarker}{}%
\end{pgfscope}%
\begin{pgfscope}%
\pgfsys@transformshift{0.864979in}{2.178632in}%
\pgfsys@useobject{currentmarker}{}%
\end{pgfscope}%
\begin{pgfscope}%
\pgfsys@transformshift{1.057128in}{1.957257in}%
\pgfsys@useobject{currentmarker}{}%
\end{pgfscope}%
\begin{pgfscope}%
\pgfsys@transformshift{1.249277in}{1.669093in}%
\pgfsys@useobject{currentmarker}{}%
\end{pgfscope}%
\begin{pgfscope}%
\pgfsys@transformshift{1.441426in}{1.784749in}%
\pgfsys@useobject{currentmarker}{}%
\end{pgfscope}%
\begin{pgfscope}%
\pgfsys@transformshift{1.633574in}{1.613793in}%
\pgfsys@useobject{currentmarker}{}%
\end{pgfscope}%
\begin{pgfscope}%
\pgfsys@transformshift{1.825723in}{1.376658in}%
\pgfsys@useobject{currentmarker}{}%
\end{pgfscope}%
\begin{pgfscope}%
\pgfsys@transformshift{2.017872in}{0.973068in}%
\pgfsys@useobject{currentmarker}{}%
\end{pgfscope}%
\begin{pgfscope}%
\pgfsys@transformshift{2.210021in}{0.844365in}%
\pgfsys@useobject{currentmarker}{}%
\end{pgfscope}%
\begin{pgfscope}%
\pgfsys@transformshift{2.402169in}{0.581746in}%
\pgfsys@useobject{currentmarker}{}%
\end{pgfscope}%
\begin{pgfscope}%
\pgfsys@transformshift{2.594318in}{0.435000in}%
\pgfsys@useobject{currentmarker}{}%
\end{pgfscope}%
\end{pgfscope}%
\begin{pgfscope}%
\pgfpathrectangle{\pgfqpoint{0.375000in}{0.330000in}}{\pgfqpoint{2.325000in}{2.310000in}}%
\pgfusepath{clip}%
\pgfsetbuttcap%
\pgfsetroundjoin%
\pgfsetlinewidth{1.505625pt}%
\definecolor{currentstroke}{rgb}{0.000000,0.000000,0.000000}%
\pgfsetstrokecolor{currentstroke}%
\pgfsetdash{{5.550000pt}{2.400000pt}}{0.000000pt}%
\pgfpathmoveto{\pgfqpoint{0.480682in}{2.535000in}}%
\pgfpathlineto{\pgfqpoint{2.594318in}{0.496855in}}%
\pgfusepath{stroke}%
\end{pgfscope}%
\begin{pgfscope}%
\pgfsetrectcap%
\pgfsetmiterjoin%
\pgfsetlinewidth{0.803000pt}%
\definecolor{currentstroke}{rgb}{0.000000,0.000000,0.000000}%
\pgfsetstrokecolor{currentstroke}%
\pgfsetdash{}{0pt}%
\pgfpathmoveto{\pgfqpoint{0.375000in}{0.330000in}}%
\pgfpathlineto{\pgfqpoint{0.375000in}{2.640000in}}%
\pgfusepath{stroke}%
\end{pgfscope}%
\begin{pgfscope}%
\pgfsetrectcap%
\pgfsetmiterjoin%
\pgfsetlinewidth{0.803000pt}%
\definecolor{currentstroke}{rgb}{0.000000,0.000000,0.000000}%
\pgfsetstrokecolor{currentstroke}%
\pgfsetdash{}{0pt}%
\pgfpathmoveto{\pgfqpoint{2.700000in}{0.330000in}}%
\pgfpathlineto{\pgfqpoint{2.700000in}{2.640000in}}%
\pgfusepath{stroke}%
\end{pgfscope}%
\begin{pgfscope}%
\pgfsetrectcap%
\pgfsetmiterjoin%
\pgfsetlinewidth{0.803000pt}%
\definecolor{currentstroke}{rgb}{0.000000,0.000000,0.000000}%
\pgfsetstrokecolor{currentstroke}%
\pgfsetdash{}{0pt}%
\pgfpathmoveto{\pgfqpoint{0.375000in}{0.330000in}}%
\pgfpathlineto{\pgfqpoint{2.700000in}{0.330000in}}%
\pgfusepath{stroke}%
\end{pgfscope}%
\begin{pgfscope}%
\pgfsetrectcap%
\pgfsetmiterjoin%
\pgfsetlinewidth{0.803000pt}%
\definecolor{currentstroke}{rgb}{0.000000,0.000000,0.000000}%
\pgfsetstrokecolor{currentstroke}%
\pgfsetdash{}{0pt}%
\pgfpathmoveto{\pgfqpoint{0.375000in}{2.640000in}}%
\pgfpathlineto{\pgfqpoint{2.700000in}{2.640000in}}%
\pgfusepath{stroke}%
\end{pgfscope}%
\begin{pgfscope}%
\pgfsetbuttcap%
\pgfsetmiterjoin%
\definecolor{currentfill}{rgb}{1.000000,1.000000,1.000000}%
\pgfsetfillcolor{currentfill}%
\pgfsetfillopacity{0.800000}%
\pgfsetlinewidth{1.003750pt}%
\definecolor{currentstroke}{rgb}{0.800000,0.800000,0.800000}%
\pgfsetstrokecolor{currentstroke}%
\pgfsetstrokeopacity{0.800000}%
\pgfsetdash{}{0pt}%
\pgfpathmoveto{\pgfqpoint{1.242243in}{2.275267in}}%
\pgfpathlineto{\pgfqpoint{2.602778in}{2.275267in}}%
\pgfpathquadraticcurveto{\pgfqpoint{2.630556in}{2.275267in}}{\pgfqpoint{2.630556in}{2.303045in}}%
\pgfpathlineto{\pgfqpoint{2.630556in}{2.542778in}}%
\pgfpathquadraticcurveto{\pgfqpoint{2.630556in}{2.570556in}}{\pgfqpoint{2.602778in}{2.570556in}}%
\pgfpathlineto{\pgfqpoint{1.242243in}{2.570556in}}%
\pgfpathquadraticcurveto{\pgfqpoint{1.214465in}{2.570556in}}{\pgfqpoint{1.214465in}{2.542778in}}%
\pgfpathlineto{\pgfqpoint{1.214465in}{2.303045in}}%
\pgfpathquadraticcurveto{\pgfqpoint{1.214465in}{2.275267in}}{\pgfqpoint{1.242243in}{2.275267in}}%
\pgfpathclose%
\pgfusepath{stroke,fill}%
\end{pgfscope}%
\begin{pgfscope}%
\pgfsetbuttcap%
\pgfsetroundjoin%
\pgfsetlinewidth{1.505625pt}%
\definecolor{currentstroke}{rgb}{0.000000,0.000000,0.000000}%
\pgfsetstrokecolor{currentstroke}%
\pgfsetdash{{5.550000pt}{2.400000pt}}{0.000000pt}%
\pgfpathmoveto{\pgfqpoint{1.270021in}{2.441543in}}%
\pgfpathlineto{\pgfqpoint{1.547799in}{2.441543in}}%
\pgfusepath{stroke}%
\end{pgfscope}%
\begin{pgfscope}%
\definecolor{textcolor}{rgb}{0.000000,0.000000,0.000000}%
\pgfsetstrokecolor{textcolor}%
\pgfsetfillcolor{textcolor}%
\pgftext[x=1.658910in,y=2.392932in,left,base]{\color{textcolor}\sffamily\fontsize{10.000000}{12.000000}\selectfont \(\displaystyle 1.2235N_{\mathrm{QMC}}^{-0.8876}\)}%
\end{pgfscope}%
\end{pgfpicture}%
\makeatother%
\endgroup%

\caption[Quasi-Monte-Carlo error, for $Q=\grad u((1,1))$ and $k=20$.]{Quasi-Monte-Carlo error with increasing number of Quasi-Monte-Carlo points, for $Q=\grad u((1,1))$ and $k=20$. \label{fig:qmcgradient_top_right20}}
\end{figure}
\begin{figure}[h]
    \centering
    %% Creator: Matplotlib, PGF backend
%%
%% To include the figure in your LaTeX document, write
%%   \input{<filename>.pgf}
%%
%% Make sure the required packages are loaded in your preamble
%%   \usepackage{pgf}
%%
%% Figures using additional raster images can only be included by \input if
%% they are in the same directory as the main LaTeX file. For loading figures
%% from other directories you can use the `import` package
%%   \usepackage{import}
%% and then include the figures with
%%   \import{<path to file>}{<filename>.pgf}
%%
%% Matplotlib used the following preamble
%%   \usepackage{fontspec}
%%   \setmainfont{DejaVuSerif.ttf}[Path=/home/owen/progs/firedrake-complex/firedrake/lib/python3.5/site-packages/matplotlib/mpl-data/fonts/ttf/]
%%   \setsansfont{DejaVuSans.ttf}[Path=/home/owen/progs/firedrake-complex/firedrake/lib/python3.5/site-packages/matplotlib/mpl-data/fonts/ttf/]
%%   \setmonofont{DejaVuSansMono.ttf}[Path=/home/owen/progs/firedrake-complex/firedrake/lib/python3.5/site-packages/matplotlib/mpl-data/fonts/ttf/]
%%
\begingroup%
\makeatletter%
\begin{pgfpicture}%
\pgfpathrectangle{\pgfpointorigin}{\pgfqpoint{5.000000in}{4.000000in}}%
\pgfusepath{use as bounding box, clip}%
\begin{pgfscope}%
\pgfsetbuttcap%
\pgfsetmiterjoin%
\definecolor{currentfill}{rgb}{1.000000,1.000000,1.000000}%
\pgfsetfillcolor{currentfill}%
\pgfsetlinewidth{0.000000pt}%
\definecolor{currentstroke}{rgb}{1.000000,1.000000,1.000000}%
\pgfsetstrokecolor{currentstroke}%
\pgfsetdash{}{0pt}%
\pgfpathmoveto{\pgfqpoint{0.000000in}{0.000000in}}%
\pgfpathlineto{\pgfqpoint{5.000000in}{0.000000in}}%
\pgfpathlineto{\pgfqpoint{5.000000in}{4.000000in}}%
\pgfpathlineto{\pgfqpoint{0.000000in}{4.000000in}}%
\pgfpathclose%
\pgfusepath{fill}%
\end{pgfscope}%
\begin{pgfscope}%
\pgfsetbuttcap%
\pgfsetmiterjoin%
\definecolor{currentfill}{rgb}{1.000000,1.000000,1.000000}%
\pgfsetfillcolor{currentfill}%
\pgfsetlinewidth{0.000000pt}%
\definecolor{currentstroke}{rgb}{0.000000,0.000000,0.000000}%
\pgfsetstrokecolor{currentstroke}%
\pgfsetstrokeopacity{0.000000}%
\pgfsetdash{}{0pt}%
\pgfpathmoveto{\pgfqpoint{0.625000in}{0.440000in}}%
\pgfpathlineto{\pgfqpoint{4.500000in}{0.440000in}}%
\pgfpathlineto{\pgfqpoint{4.500000in}{3.520000in}}%
\pgfpathlineto{\pgfqpoint{0.625000in}{3.520000in}}%
\pgfpathclose%
\pgfusepath{fill}%
\end{pgfscope}%
\begin{pgfscope}%
\pgfsetbuttcap%
\pgfsetroundjoin%
\definecolor{currentfill}{rgb}{0.000000,0.000000,0.000000}%
\pgfsetfillcolor{currentfill}%
\pgfsetlinewidth{0.803000pt}%
\definecolor{currentstroke}{rgb}{0.000000,0.000000,0.000000}%
\pgfsetstrokecolor{currentstroke}%
\pgfsetdash{}{0pt}%
\pgfsys@defobject{currentmarker}{\pgfqpoint{0.000000in}{-0.048611in}}{\pgfqpoint{0.000000in}{0.000000in}}{%
\pgfpathmoveto{\pgfqpoint{0.000000in}{0.000000in}}%
\pgfpathlineto{\pgfqpoint{0.000000in}{-0.048611in}}%
\pgfusepath{stroke,fill}%
}%
\begin{pgfscope}%
\pgfsys@transformshift{0.801136in}{0.440000in}%
\pgfsys@useobject{currentmarker}{}%
\end{pgfscope}%
\end{pgfscope}%
\begin{pgfscope}%
\definecolor{textcolor}{rgb}{0.000000,0.000000,0.000000}%
\pgfsetstrokecolor{textcolor}%
\pgfsetfillcolor{textcolor}%
\pgftext[x=0.801136in,y=0.342778in,,top]{\color{textcolor}\sffamily\fontsize{10.000000}{12.000000}\selectfont \(\displaystyle 10^{0}\)}%
\end{pgfscope}%
\begin{pgfscope}%
\pgfsetbuttcap%
\pgfsetroundjoin%
\definecolor{currentfill}{rgb}{0.000000,0.000000,0.000000}%
\pgfsetfillcolor{currentfill}%
\pgfsetlinewidth{0.803000pt}%
\definecolor{currentstroke}{rgb}{0.000000,0.000000,0.000000}%
\pgfsetstrokecolor{currentstroke}%
\pgfsetdash{}{0pt}%
\pgfsys@defobject{currentmarker}{\pgfqpoint{0.000000in}{-0.048611in}}{\pgfqpoint{0.000000in}{0.000000in}}{%
\pgfpathmoveto{\pgfqpoint{0.000000in}{0.000000in}}%
\pgfpathlineto{\pgfqpoint{0.000000in}{-0.048611in}}%
\pgfusepath{stroke,fill}%
}%
\begin{pgfscope}%
\pgfsys@transformshift{1.864977in}{0.440000in}%
\pgfsys@useobject{currentmarker}{}%
\end{pgfscope}%
\end{pgfscope}%
\begin{pgfscope}%
\definecolor{textcolor}{rgb}{0.000000,0.000000,0.000000}%
\pgfsetstrokecolor{textcolor}%
\pgfsetfillcolor{textcolor}%
\pgftext[x=1.864977in,y=0.342778in,,top]{\color{textcolor}\sffamily\fontsize{10.000000}{12.000000}\selectfont \(\displaystyle 10^{1}\)}%
\end{pgfscope}%
\begin{pgfscope}%
\pgfsetbuttcap%
\pgfsetroundjoin%
\definecolor{currentfill}{rgb}{0.000000,0.000000,0.000000}%
\pgfsetfillcolor{currentfill}%
\pgfsetlinewidth{0.803000pt}%
\definecolor{currentstroke}{rgb}{0.000000,0.000000,0.000000}%
\pgfsetstrokecolor{currentstroke}%
\pgfsetdash{}{0pt}%
\pgfsys@defobject{currentmarker}{\pgfqpoint{0.000000in}{-0.048611in}}{\pgfqpoint{0.000000in}{0.000000in}}{%
\pgfpathmoveto{\pgfqpoint{0.000000in}{0.000000in}}%
\pgfpathlineto{\pgfqpoint{0.000000in}{-0.048611in}}%
\pgfusepath{stroke,fill}%
}%
\begin{pgfscope}%
\pgfsys@transformshift{2.928818in}{0.440000in}%
\pgfsys@useobject{currentmarker}{}%
\end{pgfscope}%
\end{pgfscope}%
\begin{pgfscope}%
\definecolor{textcolor}{rgb}{0.000000,0.000000,0.000000}%
\pgfsetstrokecolor{textcolor}%
\pgfsetfillcolor{textcolor}%
\pgftext[x=2.928818in,y=0.342778in,,top]{\color{textcolor}\sffamily\fontsize{10.000000}{12.000000}\selectfont \(\displaystyle 10^{2}\)}%
\end{pgfscope}%
\begin{pgfscope}%
\pgfsetbuttcap%
\pgfsetroundjoin%
\definecolor{currentfill}{rgb}{0.000000,0.000000,0.000000}%
\pgfsetfillcolor{currentfill}%
\pgfsetlinewidth{0.803000pt}%
\definecolor{currentstroke}{rgb}{0.000000,0.000000,0.000000}%
\pgfsetstrokecolor{currentstroke}%
\pgfsetdash{}{0pt}%
\pgfsys@defobject{currentmarker}{\pgfqpoint{0.000000in}{-0.048611in}}{\pgfqpoint{0.000000in}{0.000000in}}{%
\pgfpathmoveto{\pgfqpoint{0.000000in}{0.000000in}}%
\pgfpathlineto{\pgfqpoint{0.000000in}{-0.048611in}}%
\pgfusepath{stroke,fill}%
}%
\begin{pgfscope}%
\pgfsys@transformshift{3.992658in}{0.440000in}%
\pgfsys@useobject{currentmarker}{}%
\end{pgfscope}%
\end{pgfscope}%
\begin{pgfscope}%
\definecolor{textcolor}{rgb}{0.000000,0.000000,0.000000}%
\pgfsetstrokecolor{textcolor}%
\pgfsetfillcolor{textcolor}%
\pgftext[x=3.992658in,y=0.342778in,,top]{\color{textcolor}\sffamily\fontsize{10.000000}{12.000000}\selectfont \(\displaystyle 10^{3}\)}%
\end{pgfscope}%
\begin{pgfscope}%
\pgfsetbuttcap%
\pgfsetroundjoin%
\definecolor{currentfill}{rgb}{0.000000,0.000000,0.000000}%
\pgfsetfillcolor{currentfill}%
\pgfsetlinewidth{0.602250pt}%
\definecolor{currentstroke}{rgb}{0.000000,0.000000,0.000000}%
\pgfsetstrokecolor{currentstroke}%
\pgfsetdash{}{0pt}%
\pgfsys@defobject{currentmarker}{\pgfqpoint{0.000000in}{-0.027778in}}{\pgfqpoint{0.000000in}{0.000000in}}{%
\pgfpathmoveto{\pgfqpoint{0.000000in}{0.000000in}}%
\pgfpathlineto{\pgfqpoint{0.000000in}{-0.027778in}}%
\pgfusepath{stroke,fill}%
}%
\begin{pgfscope}%
\pgfsys@transformshift{0.636345in}{0.440000in}%
\pgfsys@useobject{currentmarker}{}%
\end{pgfscope}%
\end{pgfscope}%
\begin{pgfscope}%
\pgfsetbuttcap%
\pgfsetroundjoin%
\definecolor{currentfill}{rgb}{0.000000,0.000000,0.000000}%
\pgfsetfillcolor{currentfill}%
\pgfsetlinewidth{0.602250pt}%
\definecolor{currentstroke}{rgb}{0.000000,0.000000,0.000000}%
\pgfsetstrokecolor{currentstroke}%
\pgfsetdash{}{0pt}%
\pgfsys@defobject{currentmarker}{\pgfqpoint{0.000000in}{-0.027778in}}{\pgfqpoint{0.000000in}{0.000000in}}{%
\pgfpathmoveto{\pgfqpoint{0.000000in}{0.000000in}}%
\pgfpathlineto{\pgfqpoint{0.000000in}{-0.027778in}}%
\pgfusepath{stroke,fill}%
}%
\begin{pgfscope}%
\pgfsys@transformshift{0.698040in}{0.440000in}%
\pgfsys@useobject{currentmarker}{}%
\end{pgfscope}%
\end{pgfscope}%
\begin{pgfscope}%
\pgfsetbuttcap%
\pgfsetroundjoin%
\definecolor{currentfill}{rgb}{0.000000,0.000000,0.000000}%
\pgfsetfillcolor{currentfill}%
\pgfsetlinewidth{0.602250pt}%
\definecolor{currentstroke}{rgb}{0.000000,0.000000,0.000000}%
\pgfsetstrokecolor{currentstroke}%
\pgfsetdash{}{0pt}%
\pgfsys@defobject{currentmarker}{\pgfqpoint{0.000000in}{-0.027778in}}{\pgfqpoint{0.000000in}{0.000000in}}{%
\pgfpathmoveto{\pgfqpoint{0.000000in}{0.000000in}}%
\pgfpathlineto{\pgfqpoint{0.000000in}{-0.027778in}}%
\pgfusepath{stroke,fill}%
}%
\begin{pgfscope}%
\pgfsys@transformshift{0.752458in}{0.440000in}%
\pgfsys@useobject{currentmarker}{}%
\end{pgfscope}%
\end{pgfscope}%
\begin{pgfscope}%
\pgfsetbuttcap%
\pgfsetroundjoin%
\definecolor{currentfill}{rgb}{0.000000,0.000000,0.000000}%
\pgfsetfillcolor{currentfill}%
\pgfsetlinewidth{0.602250pt}%
\definecolor{currentstroke}{rgb}{0.000000,0.000000,0.000000}%
\pgfsetstrokecolor{currentstroke}%
\pgfsetdash{}{0pt}%
\pgfsys@defobject{currentmarker}{\pgfqpoint{0.000000in}{-0.027778in}}{\pgfqpoint{0.000000in}{0.000000in}}{%
\pgfpathmoveto{\pgfqpoint{0.000000in}{0.000000in}}%
\pgfpathlineto{\pgfqpoint{0.000000in}{-0.027778in}}%
\pgfusepath{stroke,fill}%
}%
\begin{pgfscope}%
\pgfsys@transformshift{1.121384in}{0.440000in}%
\pgfsys@useobject{currentmarker}{}%
\end{pgfscope}%
\end{pgfscope}%
\begin{pgfscope}%
\pgfsetbuttcap%
\pgfsetroundjoin%
\definecolor{currentfill}{rgb}{0.000000,0.000000,0.000000}%
\pgfsetfillcolor{currentfill}%
\pgfsetlinewidth{0.602250pt}%
\definecolor{currentstroke}{rgb}{0.000000,0.000000,0.000000}%
\pgfsetstrokecolor{currentstroke}%
\pgfsetdash{}{0pt}%
\pgfsys@defobject{currentmarker}{\pgfqpoint{0.000000in}{-0.027778in}}{\pgfqpoint{0.000000in}{0.000000in}}{%
\pgfpathmoveto{\pgfqpoint{0.000000in}{0.000000in}}%
\pgfpathlineto{\pgfqpoint{0.000000in}{-0.027778in}}%
\pgfusepath{stroke,fill}%
}%
\begin{pgfscope}%
\pgfsys@transformshift{1.308717in}{0.440000in}%
\pgfsys@useobject{currentmarker}{}%
\end{pgfscope}%
\end{pgfscope}%
\begin{pgfscope}%
\pgfsetbuttcap%
\pgfsetroundjoin%
\definecolor{currentfill}{rgb}{0.000000,0.000000,0.000000}%
\pgfsetfillcolor{currentfill}%
\pgfsetlinewidth{0.602250pt}%
\definecolor{currentstroke}{rgb}{0.000000,0.000000,0.000000}%
\pgfsetstrokecolor{currentstroke}%
\pgfsetdash{}{0pt}%
\pgfsys@defobject{currentmarker}{\pgfqpoint{0.000000in}{-0.027778in}}{\pgfqpoint{0.000000in}{0.000000in}}{%
\pgfpathmoveto{\pgfqpoint{0.000000in}{0.000000in}}%
\pgfpathlineto{\pgfqpoint{0.000000in}{-0.027778in}}%
\pgfusepath{stroke,fill}%
}%
\begin{pgfscope}%
\pgfsys@transformshift{1.441632in}{0.440000in}%
\pgfsys@useobject{currentmarker}{}%
\end{pgfscope}%
\end{pgfscope}%
\begin{pgfscope}%
\pgfsetbuttcap%
\pgfsetroundjoin%
\definecolor{currentfill}{rgb}{0.000000,0.000000,0.000000}%
\pgfsetfillcolor{currentfill}%
\pgfsetlinewidth{0.602250pt}%
\definecolor{currentstroke}{rgb}{0.000000,0.000000,0.000000}%
\pgfsetstrokecolor{currentstroke}%
\pgfsetdash{}{0pt}%
\pgfsys@defobject{currentmarker}{\pgfqpoint{0.000000in}{-0.027778in}}{\pgfqpoint{0.000000in}{0.000000in}}{%
\pgfpathmoveto{\pgfqpoint{0.000000in}{0.000000in}}%
\pgfpathlineto{\pgfqpoint{0.000000in}{-0.027778in}}%
\pgfusepath{stroke,fill}%
}%
\begin{pgfscope}%
\pgfsys@transformshift{1.544729in}{0.440000in}%
\pgfsys@useobject{currentmarker}{}%
\end{pgfscope}%
\end{pgfscope}%
\begin{pgfscope}%
\pgfsetbuttcap%
\pgfsetroundjoin%
\definecolor{currentfill}{rgb}{0.000000,0.000000,0.000000}%
\pgfsetfillcolor{currentfill}%
\pgfsetlinewidth{0.602250pt}%
\definecolor{currentstroke}{rgb}{0.000000,0.000000,0.000000}%
\pgfsetstrokecolor{currentstroke}%
\pgfsetdash{}{0pt}%
\pgfsys@defobject{currentmarker}{\pgfqpoint{0.000000in}{-0.027778in}}{\pgfqpoint{0.000000in}{0.000000in}}{%
\pgfpathmoveto{\pgfqpoint{0.000000in}{0.000000in}}%
\pgfpathlineto{\pgfqpoint{0.000000in}{-0.027778in}}%
\pgfusepath{stroke,fill}%
}%
\begin{pgfscope}%
\pgfsys@transformshift{1.628965in}{0.440000in}%
\pgfsys@useobject{currentmarker}{}%
\end{pgfscope}%
\end{pgfscope}%
\begin{pgfscope}%
\pgfsetbuttcap%
\pgfsetroundjoin%
\definecolor{currentfill}{rgb}{0.000000,0.000000,0.000000}%
\pgfsetfillcolor{currentfill}%
\pgfsetlinewidth{0.602250pt}%
\definecolor{currentstroke}{rgb}{0.000000,0.000000,0.000000}%
\pgfsetstrokecolor{currentstroke}%
\pgfsetdash{}{0pt}%
\pgfsys@defobject{currentmarker}{\pgfqpoint{0.000000in}{-0.027778in}}{\pgfqpoint{0.000000in}{0.000000in}}{%
\pgfpathmoveto{\pgfqpoint{0.000000in}{0.000000in}}%
\pgfpathlineto{\pgfqpoint{0.000000in}{-0.027778in}}%
\pgfusepath{stroke,fill}%
}%
\begin{pgfscope}%
\pgfsys@transformshift{1.700186in}{0.440000in}%
\pgfsys@useobject{currentmarker}{}%
\end{pgfscope}%
\end{pgfscope}%
\begin{pgfscope}%
\pgfsetbuttcap%
\pgfsetroundjoin%
\definecolor{currentfill}{rgb}{0.000000,0.000000,0.000000}%
\pgfsetfillcolor{currentfill}%
\pgfsetlinewidth{0.602250pt}%
\definecolor{currentstroke}{rgb}{0.000000,0.000000,0.000000}%
\pgfsetstrokecolor{currentstroke}%
\pgfsetdash{}{0pt}%
\pgfsys@defobject{currentmarker}{\pgfqpoint{0.000000in}{-0.027778in}}{\pgfqpoint{0.000000in}{0.000000in}}{%
\pgfpathmoveto{\pgfqpoint{0.000000in}{0.000000in}}%
\pgfpathlineto{\pgfqpoint{0.000000in}{-0.027778in}}%
\pgfusepath{stroke,fill}%
}%
\begin{pgfscope}%
\pgfsys@transformshift{1.761880in}{0.440000in}%
\pgfsys@useobject{currentmarker}{}%
\end{pgfscope}%
\end{pgfscope}%
\begin{pgfscope}%
\pgfsetbuttcap%
\pgfsetroundjoin%
\definecolor{currentfill}{rgb}{0.000000,0.000000,0.000000}%
\pgfsetfillcolor{currentfill}%
\pgfsetlinewidth{0.602250pt}%
\definecolor{currentstroke}{rgb}{0.000000,0.000000,0.000000}%
\pgfsetstrokecolor{currentstroke}%
\pgfsetdash{}{0pt}%
\pgfsys@defobject{currentmarker}{\pgfqpoint{0.000000in}{-0.027778in}}{\pgfqpoint{0.000000in}{0.000000in}}{%
\pgfpathmoveto{\pgfqpoint{0.000000in}{0.000000in}}%
\pgfpathlineto{\pgfqpoint{0.000000in}{-0.027778in}}%
\pgfusepath{stroke,fill}%
}%
\begin{pgfscope}%
\pgfsys@transformshift{1.816298in}{0.440000in}%
\pgfsys@useobject{currentmarker}{}%
\end{pgfscope}%
\end{pgfscope}%
\begin{pgfscope}%
\pgfsetbuttcap%
\pgfsetroundjoin%
\definecolor{currentfill}{rgb}{0.000000,0.000000,0.000000}%
\pgfsetfillcolor{currentfill}%
\pgfsetlinewidth{0.602250pt}%
\definecolor{currentstroke}{rgb}{0.000000,0.000000,0.000000}%
\pgfsetstrokecolor{currentstroke}%
\pgfsetdash{}{0pt}%
\pgfsys@defobject{currentmarker}{\pgfqpoint{0.000000in}{-0.027778in}}{\pgfqpoint{0.000000in}{0.000000in}}{%
\pgfpathmoveto{\pgfqpoint{0.000000in}{0.000000in}}%
\pgfpathlineto{\pgfqpoint{0.000000in}{-0.027778in}}%
\pgfusepath{stroke,fill}%
}%
\begin{pgfscope}%
\pgfsys@transformshift{2.185225in}{0.440000in}%
\pgfsys@useobject{currentmarker}{}%
\end{pgfscope}%
\end{pgfscope}%
\begin{pgfscope}%
\pgfsetbuttcap%
\pgfsetroundjoin%
\definecolor{currentfill}{rgb}{0.000000,0.000000,0.000000}%
\pgfsetfillcolor{currentfill}%
\pgfsetlinewidth{0.602250pt}%
\definecolor{currentstroke}{rgb}{0.000000,0.000000,0.000000}%
\pgfsetstrokecolor{currentstroke}%
\pgfsetdash{}{0pt}%
\pgfsys@defobject{currentmarker}{\pgfqpoint{0.000000in}{-0.027778in}}{\pgfqpoint{0.000000in}{0.000000in}}{%
\pgfpathmoveto{\pgfqpoint{0.000000in}{0.000000in}}%
\pgfpathlineto{\pgfqpoint{0.000000in}{-0.027778in}}%
\pgfusepath{stroke,fill}%
}%
\begin{pgfscope}%
\pgfsys@transformshift{2.372558in}{0.440000in}%
\pgfsys@useobject{currentmarker}{}%
\end{pgfscope}%
\end{pgfscope}%
\begin{pgfscope}%
\pgfsetbuttcap%
\pgfsetroundjoin%
\definecolor{currentfill}{rgb}{0.000000,0.000000,0.000000}%
\pgfsetfillcolor{currentfill}%
\pgfsetlinewidth{0.602250pt}%
\definecolor{currentstroke}{rgb}{0.000000,0.000000,0.000000}%
\pgfsetstrokecolor{currentstroke}%
\pgfsetdash{}{0pt}%
\pgfsys@defobject{currentmarker}{\pgfqpoint{0.000000in}{-0.027778in}}{\pgfqpoint{0.000000in}{0.000000in}}{%
\pgfpathmoveto{\pgfqpoint{0.000000in}{0.000000in}}%
\pgfpathlineto{\pgfqpoint{0.000000in}{-0.027778in}}%
\pgfusepath{stroke,fill}%
}%
\begin{pgfscope}%
\pgfsys@transformshift{2.505473in}{0.440000in}%
\pgfsys@useobject{currentmarker}{}%
\end{pgfscope}%
\end{pgfscope}%
\begin{pgfscope}%
\pgfsetbuttcap%
\pgfsetroundjoin%
\definecolor{currentfill}{rgb}{0.000000,0.000000,0.000000}%
\pgfsetfillcolor{currentfill}%
\pgfsetlinewidth{0.602250pt}%
\definecolor{currentstroke}{rgb}{0.000000,0.000000,0.000000}%
\pgfsetstrokecolor{currentstroke}%
\pgfsetdash{}{0pt}%
\pgfsys@defobject{currentmarker}{\pgfqpoint{0.000000in}{-0.027778in}}{\pgfqpoint{0.000000in}{0.000000in}}{%
\pgfpathmoveto{\pgfqpoint{0.000000in}{0.000000in}}%
\pgfpathlineto{\pgfqpoint{0.000000in}{-0.027778in}}%
\pgfusepath{stroke,fill}%
}%
\begin{pgfscope}%
\pgfsys@transformshift{2.608570in}{0.440000in}%
\pgfsys@useobject{currentmarker}{}%
\end{pgfscope}%
\end{pgfscope}%
\begin{pgfscope}%
\pgfsetbuttcap%
\pgfsetroundjoin%
\definecolor{currentfill}{rgb}{0.000000,0.000000,0.000000}%
\pgfsetfillcolor{currentfill}%
\pgfsetlinewidth{0.602250pt}%
\definecolor{currentstroke}{rgb}{0.000000,0.000000,0.000000}%
\pgfsetstrokecolor{currentstroke}%
\pgfsetdash{}{0pt}%
\pgfsys@defobject{currentmarker}{\pgfqpoint{0.000000in}{-0.027778in}}{\pgfqpoint{0.000000in}{0.000000in}}{%
\pgfpathmoveto{\pgfqpoint{0.000000in}{0.000000in}}%
\pgfpathlineto{\pgfqpoint{0.000000in}{-0.027778in}}%
\pgfusepath{stroke,fill}%
}%
\begin{pgfscope}%
\pgfsys@transformshift{2.692806in}{0.440000in}%
\pgfsys@useobject{currentmarker}{}%
\end{pgfscope}%
\end{pgfscope}%
\begin{pgfscope}%
\pgfsetbuttcap%
\pgfsetroundjoin%
\definecolor{currentfill}{rgb}{0.000000,0.000000,0.000000}%
\pgfsetfillcolor{currentfill}%
\pgfsetlinewidth{0.602250pt}%
\definecolor{currentstroke}{rgb}{0.000000,0.000000,0.000000}%
\pgfsetstrokecolor{currentstroke}%
\pgfsetdash{}{0pt}%
\pgfsys@defobject{currentmarker}{\pgfqpoint{0.000000in}{-0.027778in}}{\pgfqpoint{0.000000in}{0.000000in}}{%
\pgfpathmoveto{\pgfqpoint{0.000000in}{0.000000in}}%
\pgfpathlineto{\pgfqpoint{0.000000in}{-0.027778in}}%
\pgfusepath{stroke,fill}%
}%
\begin{pgfscope}%
\pgfsys@transformshift{2.764027in}{0.440000in}%
\pgfsys@useobject{currentmarker}{}%
\end{pgfscope}%
\end{pgfscope}%
\begin{pgfscope}%
\pgfsetbuttcap%
\pgfsetroundjoin%
\definecolor{currentfill}{rgb}{0.000000,0.000000,0.000000}%
\pgfsetfillcolor{currentfill}%
\pgfsetlinewidth{0.602250pt}%
\definecolor{currentstroke}{rgb}{0.000000,0.000000,0.000000}%
\pgfsetstrokecolor{currentstroke}%
\pgfsetdash{}{0pt}%
\pgfsys@defobject{currentmarker}{\pgfqpoint{0.000000in}{-0.027778in}}{\pgfqpoint{0.000000in}{0.000000in}}{%
\pgfpathmoveto{\pgfqpoint{0.000000in}{0.000000in}}%
\pgfpathlineto{\pgfqpoint{0.000000in}{-0.027778in}}%
\pgfusepath{stroke,fill}%
}%
\begin{pgfscope}%
\pgfsys@transformshift{2.825721in}{0.440000in}%
\pgfsys@useobject{currentmarker}{}%
\end{pgfscope}%
\end{pgfscope}%
\begin{pgfscope}%
\pgfsetbuttcap%
\pgfsetroundjoin%
\definecolor{currentfill}{rgb}{0.000000,0.000000,0.000000}%
\pgfsetfillcolor{currentfill}%
\pgfsetlinewidth{0.602250pt}%
\definecolor{currentstroke}{rgb}{0.000000,0.000000,0.000000}%
\pgfsetstrokecolor{currentstroke}%
\pgfsetdash{}{0pt}%
\pgfsys@defobject{currentmarker}{\pgfqpoint{0.000000in}{-0.027778in}}{\pgfqpoint{0.000000in}{0.000000in}}{%
\pgfpathmoveto{\pgfqpoint{0.000000in}{0.000000in}}%
\pgfpathlineto{\pgfqpoint{0.000000in}{-0.027778in}}%
\pgfusepath{stroke,fill}%
}%
\begin{pgfscope}%
\pgfsys@transformshift{2.880139in}{0.440000in}%
\pgfsys@useobject{currentmarker}{}%
\end{pgfscope}%
\end{pgfscope}%
\begin{pgfscope}%
\pgfsetbuttcap%
\pgfsetroundjoin%
\definecolor{currentfill}{rgb}{0.000000,0.000000,0.000000}%
\pgfsetfillcolor{currentfill}%
\pgfsetlinewidth{0.602250pt}%
\definecolor{currentstroke}{rgb}{0.000000,0.000000,0.000000}%
\pgfsetstrokecolor{currentstroke}%
\pgfsetdash{}{0pt}%
\pgfsys@defobject{currentmarker}{\pgfqpoint{0.000000in}{-0.027778in}}{\pgfqpoint{0.000000in}{0.000000in}}{%
\pgfpathmoveto{\pgfqpoint{0.000000in}{0.000000in}}%
\pgfpathlineto{\pgfqpoint{0.000000in}{-0.027778in}}%
\pgfusepath{stroke,fill}%
}%
\begin{pgfscope}%
\pgfsys@transformshift{3.249066in}{0.440000in}%
\pgfsys@useobject{currentmarker}{}%
\end{pgfscope}%
\end{pgfscope}%
\begin{pgfscope}%
\pgfsetbuttcap%
\pgfsetroundjoin%
\definecolor{currentfill}{rgb}{0.000000,0.000000,0.000000}%
\pgfsetfillcolor{currentfill}%
\pgfsetlinewidth{0.602250pt}%
\definecolor{currentstroke}{rgb}{0.000000,0.000000,0.000000}%
\pgfsetstrokecolor{currentstroke}%
\pgfsetdash{}{0pt}%
\pgfsys@defobject{currentmarker}{\pgfqpoint{0.000000in}{-0.027778in}}{\pgfqpoint{0.000000in}{0.000000in}}{%
\pgfpathmoveto{\pgfqpoint{0.000000in}{0.000000in}}%
\pgfpathlineto{\pgfqpoint{0.000000in}{-0.027778in}}%
\pgfusepath{stroke,fill}%
}%
\begin{pgfscope}%
\pgfsys@transformshift{3.436399in}{0.440000in}%
\pgfsys@useobject{currentmarker}{}%
\end{pgfscope}%
\end{pgfscope}%
\begin{pgfscope}%
\pgfsetbuttcap%
\pgfsetroundjoin%
\definecolor{currentfill}{rgb}{0.000000,0.000000,0.000000}%
\pgfsetfillcolor{currentfill}%
\pgfsetlinewidth{0.602250pt}%
\definecolor{currentstroke}{rgb}{0.000000,0.000000,0.000000}%
\pgfsetstrokecolor{currentstroke}%
\pgfsetdash{}{0pt}%
\pgfsys@defobject{currentmarker}{\pgfqpoint{0.000000in}{-0.027778in}}{\pgfqpoint{0.000000in}{0.000000in}}{%
\pgfpathmoveto{\pgfqpoint{0.000000in}{0.000000in}}%
\pgfpathlineto{\pgfqpoint{0.000000in}{-0.027778in}}%
\pgfusepath{stroke,fill}%
}%
\begin{pgfscope}%
\pgfsys@transformshift{3.569313in}{0.440000in}%
\pgfsys@useobject{currentmarker}{}%
\end{pgfscope}%
\end{pgfscope}%
\begin{pgfscope}%
\pgfsetbuttcap%
\pgfsetroundjoin%
\definecolor{currentfill}{rgb}{0.000000,0.000000,0.000000}%
\pgfsetfillcolor{currentfill}%
\pgfsetlinewidth{0.602250pt}%
\definecolor{currentstroke}{rgb}{0.000000,0.000000,0.000000}%
\pgfsetstrokecolor{currentstroke}%
\pgfsetdash{}{0pt}%
\pgfsys@defobject{currentmarker}{\pgfqpoint{0.000000in}{-0.027778in}}{\pgfqpoint{0.000000in}{0.000000in}}{%
\pgfpathmoveto{\pgfqpoint{0.000000in}{0.000000in}}%
\pgfpathlineto{\pgfqpoint{0.000000in}{-0.027778in}}%
\pgfusepath{stroke,fill}%
}%
\begin{pgfscope}%
\pgfsys@transformshift{3.672410in}{0.440000in}%
\pgfsys@useobject{currentmarker}{}%
\end{pgfscope}%
\end{pgfscope}%
\begin{pgfscope}%
\pgfsetbuttcap%
\pgfsetroundjoin%
\definecolor{currentfill}{rgb}{0.000000,0.000000,0.000000}%
\pgfsetfillcolor{currentfill}%
\pgfsetlinewidth{0.602250pt}%
\definecolor{currentstroke}{rgb}{0.000000,0.000000,0.000000}%
\pgfsetstrokecolor{currentstroke}%
\pgfsetdash{}{0pt}%
\pgfsys@defobject{currentmarker}{\pgfqpoint{0.000000in}{-0.027778in}}{\pgfqpoint{0.000000in}{0.000000in}}{%
\pgfpathmoveto{\pgfqpoint{0.000000in}{0.000000in}}%
\pgfpathlineto{\pgfqpoint{0.000000in}{-0.027778in}}%
\pgfusepath{stroke,fill}%
}%
\begin{pgfscope}%
\pgfsys@transformshift{3.756646in}{0.440000in}%
\pgfsys@useobject{currentmarker}{}%
\end{pgfscope}%
\end{pgfscope}%
\begin{pgfscope}%
\pgfsetbuttcap%
\pgfsetroundjoin%
\definecolor{currentfill}{rgb}{0.000000,0.000000,0.000000}%
\pgfsetfillcolor{currentfill}%
\pgfsetlinewidth{0.602250pt}%
\definecolor{currentstroke}{rgb}{0.000000,0.000000,0.000000}%
\pgfsetstrokecolor{currentstroke}%
\pgfsetdash{}{0pt}%
\pgfsys@defobject{currentmarker}{\pgfqpoint{0.000000in}{-0.027778in}}{\pgfqpoint{0.000000in}{0.000000in}}{%
\pgfpathmoveto{\pgfqpoint{0.000000in}{0.000000in}}%
\pgfpathlineto{\pgfqpoint{0.000000in}{-0.027778in}}%
\pgfusepath{stroke,fill}%
}%
\begin{pgfscope}%
\pgfsys@transformshift{3.827867in}{0.440000in}%
\pgfsys@useobject{currentmarker}{}%
\end{pgfscope}%
\end{pgfscope}%
\begin{pgfscope}%
\pgfsetbuttcap%
\pgfsetroundjoin%
\definecolor{currentfill}{rgb}{0.000000,0.000000,0.000000}%
\pgfsetfillcolor{currentfill}%
\pgfsetlinewidth{0.602250pt}%
\definecolor{currentstroke}{rgb}{0.000000,0.000000,0.000000}%
\pgfsetstrokecolor{currentstroke}%
\pgfsetdash{}{0pt}%
\pgfsys@defobject{currentmarker}{\pgfqpoint{0.000000in}{-0.027778in}}{\pgfqpoint{0.000000in}{0.000000in}}{%
\pgfpathmoveto{\pgfqpoint{0.000000in}{0.000000in}}%
\pgfpathlineto{\pgfqpoint{0.000000in}{-0.027778in}}%
\pgfusepath{stroke,fill}%
}%
\begin{pgfscope}%
\pgfsys@transformshift{3.889561in}{0.440000in}%
\pgfsys@useobject{currentmarker}{}%
\end{pgfscope}%
\end{pgfscope}%
\begin{pgfscope}%
\pgfsetbuttcap%
\pgfsetroundjoin%
\definecolor{currentfill}{rgb}{0.000000,0.000000,0.000000}%
\pgfsetfillcolor{currentfill}%
\pgfsetlinewidth{0.602250pt}%
\definecolor{currentstroke}{rgb}{0.000000,0.000000,0.000000}%
\pgfsetstrokecolor{currentstroke}%
\pgfsetdash{}{0pt}%
\pgfsys@defobject{currentmarker}{\pgfqpoint{0.000000in}{-0.027778in}}{\pgfqpoint{0.000000in}{0.000000in}}{%
\pgfpathmoveto{\pgfqpoint{0.000000in}{0.000000in}}%
\pgfpathlineto{\pgfqpoint{0.000000in}{-0.027778in}}%
\pgfusepath{stroke,fill}%
}%
\begin{pgfscope}%
\pgfsys@transformshift{3.943980in}{0.440000in}%
\pgfsys@useobject{currentmarker}{}%
\end{pgfscope}%
\end{pgfscope}%
\begin{pgfscope}%
\pgfsetbuttcap%
\pgfsetroundjoin%
\definecolor{currentfill}{rgb}{0.000000,0.000000,0.000000}%
\pgfsetfillcolor{currentfill}%
\pgfsetlinewidth{0.602250pt}%
\definecolor{currentstroke}{rgb}{0.000000,0.000000,0.000000}%
\pgfsetstrokecolor{currentstroke}%
\pgfsetdash{}{0pt}%
\pgfsys@defobject{currentmarker}{\pgfqpoint{0.000000in}{-0.027778in}}{\pgfqpoint{0.000000in}{0.000000in}}{%
\pgfpathmoveto{\pgfqpoint{0.000000in}{0.000000in}}%
\pgfpathlineto{\pgfqpoint{0.000000in}{-0.027778in}}%
\pgfusepath{stroke,fill}%
}%
\begin{pgfscope}%
\pgfsys@transformshift{4.312906in}{0.440000in}%
\pgfsys@useobject{currentmarker}{}%
\end{pgfscope}%
\end{pgfscope}%
\begin{pgfscope}%
\pgfsetbuttcap%
\pgfsetroundjoin%
\definecolor{currentfill}{rgb}{0.000000,0.000000,0.000000}%
\pgfsetfillcolor{currentfill}%
\pgfsetlinewidth{0.602250pt}%
\definecolor{currentstroke}{rgb}{0.000000,0.000000,0.000000}%
\pgfsetstrokecolor{currentstroke}%
\pgfsetdash{}{0pt}%
\pgfsys@defobject{currentmarker}{\pgfqpoint{0.000000in}{-0.027778in}}{\pgfqpoint{0.000000in}{0.000000in}}{%
\pgfpathmoveto{\pgfqpoint{0.000000in}{0.000000in}}%
\pgfpathlineto{\pgfqpoint{0.000000in}{-0.027778in}}%
\pgfusepath{stroke,fill}%
}%
\begin{pgfscope}%
\pgfsys@transformshift{4.500239in}{0.440000in}%
\pgfsys@useobject{currentmarker}{}%
\end{pgfscope}%
\end{pgfscope}%
\begin{pgfscope}%
\definecolor{textcolor}{rgb}{0.000000,0.000000,0.000000}%
\pgfsetstrokecolor{textcolor}%
\pgfsetfillcolor{textcolor}%
\pgftext[x=2.562500in,y=0.152809in,,top]{\color{textcolor}\sffamily\fontsize{10.000000}{12.000000}\selectfont \(\displaystyle N_\mathrm{QMC}\)}%
\end{pgfscope}%
\begin{pgfscope}%
\pgfsetbuttcap%
\pgfsetroundjoin%
\definecolor{currentfill}{rgb}{0.000000,0.000000,0.000000}%
\pgfsetfillcolor{currentfill}%
\pgfsetlinewidth{0.803000pt}%
\definecolor{currentstroke}{rgb}{0.000000,0.000000,0.000000}%
\pgfsetstrokecolor{currentstroke}%
\pgfsetdash{}{0pt}%
\pgfsys@defobject{currentmarker}{\pgfqpoint{-0.048611in}{0.000000in}}{\pgfqpoint{0.000000in}{0.000000in}}{%
\pgfpathmoveto{\pgfqpoint{0.000000in}{0.000000in}}%
\pgfpathlineto{\pgfqpoint{-0.048611in}{0.000000in}}%
\pgfusepath{stroke,fill}%
}%
\begin{pgfscope}%
\pgfsys@transformshift{0.625000in}{0.956915in}%
\pgfsys@useobject{currentmarker}{}%
\end{pgfscope}%
\end{pgfscope}%
\begin{pgfscope}%
\definecolor{textcolor}{rgb}{0.000000,0.000000,0.000000}%
\pgfsetstrokecolor{textcolor}%
\pgfsetfillcolor{textcolor}%
\pgftext[x=0.239775in,y=0.904154in,left,base]{\color{textcolor}\sffamily\fontsize{10.000000}{12.000000}\selectfont \(\displaystyle 10^{-2}\)}%
\end{pgfscope}%
\begin{pgfscope}%
\pgfsetbuttcap%
\pgfsetroundjoin%
\definecolor{currentfill}{rgb}{0.000000,0.000000,0.000000}%
\pgfsetfillcolor{currentfill}%
\pgfsetlinewidth{0.803000pt}%
\definecolor{currentstroke}{rgb}{0.000000,0.000000,0.000000}%
\pgfsetstrokecolor{currentstroke}%
\pgfsetdash{}{0pt}%
\pgfsys@defobject{currentmarker}{\pgfqpoint{-0.048611in}{0.000000in}}{\pgfqpoint{0.000000in}{0.000000in}}{%
\pgfpathmoveto{\pgfqpoint{0.000000in}{0.000000in}}%
\pgfpathlineto{\pgfqpoint{-0.048611in}{0.000000in}}%
\pgfusepath{stroke,fill}%
}%
\begin{pgfscope}%
\pgfsys@transformshift{0.625000in}{1.843146in}%
\pgfsys@useobject{currentmarker}{}%
\end{pgfscope}%
\end{pgfscope}%
\begin{pgfscope}%
\definecolor{textcolor}{rgb}{0.000000,0.000000,0.000000}%
\pgfsetstrokecolor{textcolor}%
\pgfsetfillcolor{textcolor}%
\pgftext[x=0.239775in,y=1.790385in,left,base]{\color{textcolor}\sffamily\fontsize{10.000000}{12.000000}\selectfont \(\displaystyle 10^{-1}\)}%
\end{pgfscope}%
\begin{pgfscope}%
\pgfsetbuttcap%
\pgfsetroundjoin%
\definecolor{currentfill}{rgb}{0.000000,0.000000,0.000000}%
\pgfsetfillcolor{currentfill}%
\pgfsetlinewidth{0.803000pt}%
\definecolor{currentstroke}{rgb}{0.000000,0.000000,0.000000}%
\pgfsetstrokecolor{currentstroke}%
\pgfsetdash{}{0pt}%
\pgfsys@defobject{currentmarker}{\pgfqpoint{-0.048611in}{0.000000in}}{\pgfqpoint{0.000000in}{0.000000in}}{%
\pgfpathmoveto{\pgfqpoint{0.000000in}{0.000000in}}%
\pgfpathlineto{\pgfqpoint{-0.048611in}{0.000000in}}%
\pgfusepath{stroke,fill}%
}%
\begin{pgfscope}%
\pgfsys@transformshift{0.625000in}{2.729378in}%
\pgfsys@useobject{currentmarker}{}%
\end{pgfscope}%
\end{pgfscope}%
\begin{pgfscope}%
\definecolor{textcolor}{rgb}{0.000000,0.000000,0.000000}%
\pgfsetstrokecolor{textcolor}%
\pgfsetfillcolor{textcolor}%
\pgftext[x=0.326581in,y=2.676616in,left,base]{\color{textcolor}\sffamily\fontsize{10.000000}{12.000000}\selectfont \(\displaystyle 10^{0}\)}%
\end{pgfscope}%
\begin{pgfscope}%
\pgfsetbuttcap%
\pgfsetroundjoin%
\definecolor{currentfill}{rgb}{0.000000,0.000000,0.000000}%
\pgfsetfillcolor{currentfill}%
\pgfsetlinewidth{0.602250pt}%
\definecolor{currentstroke}{rgb}{0.000000,0.000000,0.000000}%
\pgfsetstrokecolor{currentstroke}%
\pgfsetdash{}{0pt}%
\pgfsys@defobject{currentmarker}{\pgfqpoint{-0.027778in}{0.000000in}}{\pgfqpoint{0.000000in}{0.000000in}}{%
\pgfpathmoveto{\pgfqpoint{0.000000in}{0.000000in}}%
\pgfpathlineto{\pgfqpoint{-0.027778in}{0.000000in}}%
\pgfusepath{stroke,fill}%
}%
\begin{pgfscope}%
\pgfsys@transformshift{0.625000in}{0.493524in}%
\pgfsys@useobject{currentmarker}{}%
\end{pgfscope}%
\end{pgfscope}%
\begin{pgfscope}%
\pgfsetbuttcap%
\pgfsetroundjoin%
\definecolor{currentfill}{rgb}{0.000000,0.000000,0.000000}%
\pgfsetfillcolor{currentfill}%
\pgfsetlinewidth{0.602250pt}%
\definecolor{currentstroke}{rgb}{0.000000,0.000000,0.000000}%
\pgfsetstrokecolor{currentstroke}%
\pgfsetdash{}{0pt}%
\pgfsys@defobject{currentmarker}{\pgfqpoint{-0.027778in}{0.000000in}}{\pgfqpoint{0.000000in}{0.000000in}}{%
\pgfpathmoveto{\pgfqpoint{0.000000in}{0.000000in}}%
\pgfpathlineto{\pgfqpoint{-0.027778in}{0.000000in}}%
\pgfusepath{stroke,fill}%
}%
\begin{pgfscope}%
\pgfsys@transformshift{0.625000in}{0.604248in}%
\pgfsys@useobject{currentmarker}{}%
\end{pgfscope}%
\end{pgfscope}%
\begin{pgfscope}%
\pgfsetbuttcap%
\pgfsetroundjoin%
\definecolor{currentfill}{rgb}{0.000000,0.000000,0.000000}%
\pgfsetfillcolor{currentfill}%
\pgfsetlinewidth{0.602250pt}%
\definecolor{currentstroke}{rgb}{0.000000,0.000000,0.000000}%
\pgfsetstrokecolor{currentstroke}%
\pgfsetdash{}{0pt}%
\pgfsys@defobject{currentmarker}{\pgfqpoint{-0.027778in}{0.000000in}}{\pgfqpoint{0.000000in}{0.000000in}}{%
\pgfpathmoveto{\pgfqpoint{0.000000in}{0.000000in}}%
\pgfpathlineto{\pgfqpoint{-0.027778in}{0.000000in}}%
\pgfusepath{stroke,fill}%
}%
\begin{pgfscope}%
\pgfsys@transformshift{0.625000in}{0.690133in}%
\pgfsys@useobject{currentmarker}{}%
\end{pgfscope}%
\end{pgfscope}%
\begin{pgfscope}%
\pgfsetbuttcap%
\pgfsetroundjoin%
\definecolor{currentfill}{rgb}{0.000000,0.000000,0.000000}%
\pgfsetfillcolor{currentfill}%
\pgfsetlinewidth{0.602250pt}%
\definecolor{currentstroke}{rgb}{0.000000,0.000000,0.000000}%
\pgfsetstrokecolor{currentstroke}%
\pgfsetdash{}{0pt}%
\pgfsys@defobject{currentmarker}{\pgfqpoint{-0.027778in}{0.000000in}}{\pgfqpoint{0.000000in}{0.000000in}}{%
\pgfpathmoveto{\pgfqpoint{0.000000in}{0.000000in}}%
\pgfpathlineto{\pgfqpoint{-0.027778in}{0.000000in}}%
\pgfusepath{stroke,fill}%
}%
\begin{pgfscope}%
\pgfsys@transformshift{0.625000in}{0.760306in}%
\pgfsys@useobject{currentmarker}{}%
\end{pgfscope}%
\end{pgfscope}%
\begin{pgfscope}%
\pgfsetbuttcap%
\pgfsetroundjoin%
\definecolor{currentfill}{rgb}{0.000000,0.000000,0.000000}%
\pgfsetfillcolor{currentfill}%
\pgfsetlinewidth{0.602250pt}%
\definecolor{currentstroke}{rgb}{0.000000,0.000000,0.000000}%
\pgfsetstrokecolor{currentstroke}%
\pgfsetdash{}{0pt}%
\pgfsys@defobject{currentmarker}{\pgfqpoint{-0.027778in}{0.000000in}}{\pgfqpoint{0.000000in}{0.000000in}}{%
\pgfpathmoveto{\pgfqpoint{0.000000in}{0.000000in}}%
\pgfpathlineto{\pgfqpoint{-0.027778in}{0.000000in}}%
\pgfusepath{stroke,fill}%
}%
\begin{pgfscope}%
\pgfsys@transformshift{0.625000in}{0.819636in}%
\pgfsys@useobject{currentmarker}{}%
\end{pgfscope}%
\end{pgfscope}%
\begin{pgfscope}%
\pgfsetbuttcap%
\pgfsetroundjoin%
\definecolor{currentfill}{rgb}{0.000000,0.000000,0.000000}%
\pgfsetfillcolor{currentfill}%
\pgfsetlinewidth{0.602250pt}%
\definecolor{currentstroke}{rgb}{0.000000,0.000000,0.000000}%
\pgfsetstrokecolor{currentstroke}%
\pgfsetdash{}{0pt}%
\pgfsys@defobject{currentmarker}{\pgfqpoint{-0.027778in}{0.000000in}}{\pgfqpoint{0.000000in}{0.000000in}}{%
\pgfpathmoveto{\pgfqpoint{0.000000in}{0.000000in}}%
\pgfpathlineto{\pgfqpoint{-0.027778in}{0.000000in}}%
\pgfusepath{stroke,fill}%
}%
\begin{pgfscope}%
\pgfsys@transformshift{0.625000in}{0.871031in}%
\pgfsys@useobject{currentmarker}{}%
\end{pgfscope}%
\end{pgfscope}%
\begin{pgfscope}%
\pgfsetbuttcap%
\pgfsetroundjoin%
\definecolor{currentfill}{rgb}{0.000000,0.000000,0.000000}%
\pgfsetfillcolor{currentfill}%
\pgfsetlinewidth{0.602250pt}%
\definecolor{currentstroke}{rgb}{0.000000,0.000000,0.000000}%
\pgfsetstrokecolor{currentstroke}%
\pgfsetdash{}{0pt}%
\pgfsys@defobject{currentmarker}{\pgfqpoint{-0.027778in}{0.000000in}}{\pgfqpoint{0.000000in}{0.000000in}}{%
\pgfpathmoveto{\pgfqpoint{0.000000in}{0.000000in}}%
\pgfpathlineto{\pgfqpoint{-0.027778in}{0.000000in}}%
\pgfusepath{stroke,fill}%
}%
\begin{pgfscope}%
\pgfsys@transformshift{0.625000in}{0.916364in}%
\pgfsys@useobject{currentmarker}{}%
\end{pgfscope}%
\end{pgfscope}%
\begin{pgfscope}%
\pgfsetbuttcap%
\pgfsetroundjoin%
\definecolor{currentfill}{rgb}{0.000000,0.000000,0.000000}%
\pgfsetfillcolor{currentfill}%
\pgfsetlinewidth{0.602250pt}%
\definecolor{currentstroke}{rgb}{0.000000,0.000000,0.000000}%
\pgfsetstrokecolor{currentstroke}%
\pgfsetdash{}{0pt}%
\pgfsys@defobject{currentmarker}{\pgfqpoint{-0.027778in}{0.000000in}}{\pgfqpoint{0.000000in}{0.000000in}}{%
\pgfpathmoveto{\pgfqpoint{0.000000in}{0.000000in}}%
\pgfpathlineto{\pgfqpoint{-0.027778in}{0.000000in}}%
\pgfusepath{stroke,fill}%
}%
\begin{pgfscope}%
\pgfsys@transformshift{0.625000in}{1.223697in}%
\pgfsys@useobject{currentmarker}{}%
\end{pgfscope}%
\end{pgfscope}%
\begin{pgfscope}%
\pgfsetbuttcap%
\pgfsetroundjoin%
\definecolor{currentfill}{rgb}{0.000000,0.000000,0.000000}%
\pgfsetfillcolor{currentfill}%
\pgfsetlinewidth{0.602250pt}%
\definecolor{currentstroke}{rgb}{0.000000,0.000000,0.000000}%
\pgfsetstrokecolor{currentstroke}%
\pgfsetdash{}{0pt}%
\pgfsys@defobject{currentmarker}{\pgfqpoint{-0.027778in}{0.000000in}}{\pgfqpoint{0.000000in}{0.000000in}}{%
\pgfpathmoveto{\pgfqpoint{0.000000in}{0.000000in}}%
\pgfpathlineto{\pgfqpoint{-0.027778in}{0.000000in}}%
\pgfusepath{stroke,fill}%
}%
\begin{pgfscope}%
\pgfsys@transformshift{0.625000in}{1.379755in}%
\pgfsys@useobject{currentmarker}{}%
\end{pgfscope}%
\end{pgfscope}%
\begin{pgfscope}%
\pgfsetbuttcap%
\pgfsetroundjoin%
\definecolor{currentfill}{rgb}{0.000000,0.000000,0.000000}%
\pgfsetfillcolor{currentfill}%
\pgfsetlinewidth{0.602250pt}%
\definecolor{currentstroke}{rgb}{0.000000,0.000000,0.000000}%
\pgfsetstrokecolor{currentstroke}%
\pgfsetdash{}{0pt}%
\pgfsys@defobject{currentmarker}{\pgfqpoint{-0.027778in}{0.000000in}}{\pgfqpoint{0.000000in}{0.000000in}}{%
\pgfpathmoveto{\pgfqpoint{0.000000in}{0.000000in}}%
\pgfpathlineto{\pgfqpoint{-0.027778in}{0.000000in}}%
\pgfusepath{stroke,fill}%
}%
\begin{pgfscope}%
\pgfsys@transformshift{0.625000in}{1.490480in}%
\pgfsys@useobject{currentmarker}{}%
\end{pgfscope}%
\end{pgfscope}%
\begin{pgfscope}%
\pgfsetbuttcap%
\pgfsetroundjoin%
\definecolor{currentfill}{rgb}{0.000000,0.000000,0.000000}%
\pgfsetfillcolor{currentfill}%
\pgfsetlinewidth{0.602250pt}%
\definecolor{currentstroke}{rgb}{0.000000,0.000000,0.000000}%
\pgfsetstrokecolor{currentstroke}%
\pgfsetdash{}{0pt}%
\pgfsys@defobject{currentmarker}{\pgfqpoint{-0.027778in}{0.000000in}}{\pgfqpoint{0.000000in}{0.000000in}}{%
\pgfpathmoveto{\pgfqpoint{0.000000in}{0.000000in}}%
\pgfpathlineto{\pgfqpoint{-0.027778in}{0.000000in}}%
\pgfusepath{stroke,fill}%
}%
\begin{pgfscope}%
\pgfsys@transformshift{0.625000in}{1.576364in}%
\pgfsys@useobject{currentmarker}{}%
\end{pgfscope}%
\end{pgfscope}%
\begin{pgfscope}%
\pgfsetbuttcap%
\pgfsetroundjoin%
\definecolor{currentfill}{rgb}{0.000000,0.000000,0.000000}%
\pgfsetfillcolor{currentfill}%
\pgfsetlinewidth{0.602250pt}%
\definecolor{currentstroke}{rgb}{0.000000,0.000000,0.000000}%
\pgfsetstrokecolor{currentstroke}%
\pgfsetdash{}{0pt}%
\pgfsys@defobject{currentmarker}{\pgfqpoint{-0.027778in}{0.000000in}}{\pgfqpoint{0.000000in}{0.000000in}}{%
\pgfpathmoveto{\pgfqpoint{0.000000in}{0.000000in}}%
\pgfpathlineto{\pgfqpoint{-0.027778in}{0.000000in}}%
\pgfusepath{stroke,fill}%
}%
\begin{pgfscope}%
\pgfsys@transformshift{0.625000in}{1.646537in}%
\pgfsys@useobject{currentmarker}{}%
\end{pgfscope}%
\end{pgfscope}%
\begin{pgfscope}%
\pgfsetbuttcap%
\pgfsetroundjoin%
\definecolor{currentfill}{rgb}{0.000000,0.000000,0.000000}%
\pgfsetfillcolor{currentfill}%
\pgfsetlinewidth{0.602250pt}%
\definecolor{currentstroke}{rgb}{0.000000,0.000000,0.000000}%
\pgfsetstrokecolor{currentstroke}%
\pgfsetdash{}{0pt}%
\pgfsys@defobject{currentmarker}{\pgfqpoint{-0.027778in}{0.000000in}}{\pgfqpoint{0.000000in}{0.000000in}}{%
\pgfpathmoveto{\pgfqpoint{0.000000in}{0.000000in}}%
\pgfpathlineto{\pgfqpoint{-0.027778in}{0.000000in}}%
\pgfusepath{stroke,fill}%
}%
\begin{pgfscope}%
\pgfsys@transformshift{0.625000in}{1.705867in}%
\pgfsys@useobject{currentmarker}{}%
\end{pgfscope}%
\end{pgfscope}%
\begin{pgfscope}%
\pgfsetbuttcap%
\pgfsetroundjoin%
\definecolor{currentfill}{rgb}{0.000000,0.000000,0.000000}%
\pgfsetfillcolor{currentfill}%
\pgfsetlinewidth{0.602250pt}%
\definecolor{currentstroke}{rgb}{0.000000,0.000000,0.000000}%
\pgfsetstrokecolor{currentstroke}%
\pgfsetdash{}{0pt}%
\pgfsys@defobject{currentmarker}{\pgfqpoint{-0.027778in}{0.000000in}}{\pgfqpoint{0.000000in}{0.000000in}}{%
\pgfpathmoveto{\pgfqpoint{0.000000in}{0.000000in}}%
\pgfpathlineto{\pgfqpoint{-0.027778in}{0.000000in}}%
\pgfusepath{stroke,fill}%
}%
\begin{pgfscope}%
\pgfsys@transformshift{0.625000in}{1.757262in}%
\pgfsys@useobject{currentmarker}{}%
\end{pgfscope}%
\end{pgfscope}%
\begin{pgfscope}%
\pgfsetbuttcap%
\pgfsetroundjoin%
\definecolor{currentfill}{rgb}{0.000000,0.000000,0.000000}%
\pgfsetfillcolor{currentfill}%
\pgfsetlinewidth{0.602250pt}%
\definecolor{currentstroke}{rgb}{0.000000,0.000000,0.000000}%
\pgfsetstrokecolor{currentstroke}%
\pgfsetdash{}{0pt}%
\pgfsys@defobject{currentmarker}{\pgfqpoint{-0.027778in}{0.000000in}}{\pgfqpoint{0.000000in}{0.000000in}}{%
\pgfpathmoveto{\pgfqpoint{0.000000in}{0.000000in}}%
\pgfpathlineto{\pgfqpoint{-0.027778in}{0.000000in}}%
\pgfusepath{stroke,fill}%
}%
\begin{pgfscope}%
\pgfsys@transformshift{0.625000in}{1.802595in}%
\pgfsys@useobject{currentmarker}{}%
\end{pgfscope}%
\end{pgfscope}%
\begin{pgfscope}%
\pgfsetbuttcap%
\pgfsetroundjoin%
\definecolor{currentfill}{rgb}{0.000000,0.000000,0.000000}%
\pgfsetfillcolor{currentfill}%
\pgfsetlinewidth{0.602250pt}%
\definecolor{currentstroke}{rgb}{0.000000,0.000000,0.000000}%
\pgfsetstrokecolor{currentstroke}%
\pgfsetdash{}{0pt}%
\pgfsys@defobject{currentmarker}{\pgfqpoint{-0.027778in}{0.000000in}}{\pgfqpoint{0.000000in}{0.000000in}}{%
\pgfpathmoveto{\pgfqpoint{0.000000in}{0.000000in}}%
\pgfpathlineto{\pgfqpoint{-0.027778in}{0.000000in}}%
\pgfusepath{stroke,fill}%
}%
\begin{pgfscope}%
\pgfsys@transformshift{0.625000in}{2.109929in}%
\pgfsys@useobject{currentmarker}{}%
\end{pgfscope}%
\end{pgfscope}%
\begin{pgfscope}%
\pgfsetbuttcap%
\pgfsetroundjoin%
\definecolor{currentfill}{rgb}{0.000000,0.000000,0.000000}%
\pgfsetfillcolor{currentfill}%
\pgfsetlinewidth{0.602250pt}%
\definecolor{currentstroke}{rgb}{0.000000,0.000000,0.000000}%
\pgfsetstrokecolor{currentstroke}%
\pgfsetdash{}{0pt}%
\pgfsys@defobject{currentmarker}{\pgfqpoint{-0.027778in}{0.000000in}}{\pgfqpoint{0.000000in}{0.000000in}}{%
\pgfpathmoveto{\pgfqpoint{0.000000in}{0.000000in}}%
\pgfpathlineto{\pgfqpoint{-0.027778in}{0.000000in}}%
\pgfusepath{stroke,fill}%
}%
\begin{pgfscope}%
\pgfsys@transformshift{0.625000in}{2.265986in}%
\pgfsys@useobject{currentmarker}{}%
\end{pgfscope}%
\end{pgfscope}%
\begin{pgfscope}%
\pgfsetbuttcap%
\pgfsetroundjoin%
\definecolor{currentfill}{rgb}{0.000000,0.000000,0.000000}%
\pgfsetfillcolor{currentfill}%
\pgfsetlinewidth{0.602250pt}%
\definecolor{currentstroke}{rgb}{0.000000,0.000000,0.000000}%
\pgfsetstrokecolor{currentstroke}%
\pgfsetdash{}{0pt}%
\pgfsys@defobject{currentmarker}{\pgfqpoint{-0.027778in}{0.000000in}}{\pgfqpoint{0.000000in}{0.000000in}}{%
\pgfpathmoveto{\pgfqpoint{0.000000in}{0.000000in}}%
\pgfpathlineto{\pgfqpoint{-0.027778in}{0.000000in}}%
\pgfusepath{stroke,fill}%
}%
\begin{pgfscope}%
\pgfsys@transformshift{0.625000in}{2.376711in}%
\pgfsys@useobject{currentmarker}{}%
\end{pgfscope}%
\end{pgfscope}%
\begin{pgfscope}%
\pgfsetbuttcap%
\pgfsetroundjoin%
\definecolor{currentfill}{rgb}{0.000000,0.000000,0.000000}%
\pgfsetfillcolor{currentfill}%
\pgfsetlinewidth{0.602250pt}%
\definecolor{currentstroke}{rgb}{0.000000,0.000000,0.000000}%
\pgfsetstrokecolor{currentstroke}%
\pgfsetdash{}{0pt}%
\pgfsys@defobject{currentmarker}{\pgfqpoint{-0.027778in}{0.000000in}}{\pgfqpoint{0.000000in}{0.000000in}}{%
\pgfpathmoveto{\pgfqpoint{0.000000in}{0.000000in}}%
\pgfpathlineto{\pgfqpoint{-0.027778in}{0.000000in}}%
\pgfusepath{stroke,fill}%
}%
\begin{pgfscope}%
\pgfsys@transformshift{0.625000in}{2.462595in}%
\pgfsys@useobject{currentmarker}{}%
\end{pgfscope}%
\end{pgfscope}%
\begin{pgfscope}%
\pgfsetbuttcap%
\pgfsetroundjoin%
\definecolor{currentfill}{rgb}{0.000000,0.000000,0.000000}%
\pgfsetfillcolor{currentfill}%
\pgfsetlinewidth{0.602250pt}%
\definecolor{currentstroke}{rgb}{0.000000,0.000000,0.000000}%
\pgfsetstrokecolor{currentstroke}%
\pgfsetdash{}{0pt}%
\pgfsys@defobject{currentmarker}{\pgfqpoint{-0.027778in}{0.000000in}}{\pgfqpoint{0.000000in}{0.000000in}}{%
\pgfpathmoveto{\pgfqpoint{0.000000in}{0.000000in}}%
\pgfpathlineto{\pgfqpoint{-0.027778in}{0.000000in}}%
\pgfusepath{stroke,fill}%
}%
\begin{pgfscope}%
\pgfsys@transformshift{0.625000in}{2.532768in}%
\pgfsys@useobject{currentmarker}{}%
\end{pgfscope}%
\end{pgfscope}%
\begin{pgfscope}%
\pgfsetbuttcap%
\pgfsetroundjoin%
\definecolor{currentfill}{rgb}{0.000000,0.000000,0.000000}%
\pgfsetfillcolor{currentfill}%
\pgfsetlinewidth{0.602250pt}%
\definecolor{currentstroke}{rgb}{0.000000,0.000000,0.000000}%
\pgfsetstrokecolor{currentstroke}%
\pgfsetdash{}{0pt}%
\pgfsys@defobject{currentmarker}{\pgfqpoint{-0.027778in}{0.000000in}}{\pgfqpoint{0.000000in}{0.000000in}}{%
\pgfpathmoveto{\pgfqpoint{0.000000in}{0.000000in}}%
\pgfpathlineto{\pgfqpoint{-0.027778in}{0.000000in}}%
\pgfusepath{stroke,fill}%
}%
\begin{pgfscope}%
\pgfsys@transformshift{0.625000in}{2.592099in}%
\pgfsys@useobject{currentmarker}{}%
\end{pgfscope}%
\end{pgfscope}%
\begin{pgfscope}%
\pgfsetbuttcap%
\pgfsetroundjoin%
\definecolor{currentfill}{rgb}{0.000000,0.000000,0.000000}%
\pgfsetfillcolor{currentfill}%
\pgfsetlinewidth{0.602250pt}%
\definecolor{currentstroke}{rgb}{0.000000,0.000000,0.000000}%
\pgfsetstrokecolor{currentstroke}%
\pgfsetdash{}{0pt}%
\pgfsys@defobject{currentmarker}{\pgfqpoint{-0.027778in}{0.000000in}}{\pgfqpoint{0.000000in}{0.000000in}}{%
\pgfpathmoveto{\pgfqpoint{0.000000in}{0.000000in}}%
\pgfpathlineto{\pgfqpoint{-0.027778in}{0.000000in}}%
\pgfusepath{stroke,fill}%
}%
\begin{pgfscope}%
\pgfsys@transformshift{0.625000in}{2.643493in}%
\pgfsys@useobject{currentmarker}{}%
\end{pgfscope}%
\end{pgfscope}%
\begin{pgfscope}%
\pgfsetbuttcap%
\pgfsetroundjoin%
\definecolor{currentfill}{rgb}{0.000000,0.000000,0.000000}%
\pgfsetfillcolor{currentfill}%
\pgfsetlinewidth{0.602250pt}%
\definecolor{currentstroke}{rgb}{0.000000,0.000000,0.000000}%
\pgfsetstrokecolor{currentstroke}%
\pgfsetdash{}{0pt}%
\pgfsys@defobject{currentmarker}{\pgfqpoint{-0.027778in}{0.000000in}}{\pgfqpoint{0.000000in}{0.000000in}}{%
\pgfpathmoveto{\pgfqpoint{0.000000in}{0.000000in}}%
\pgfpathlineto{\pgfqpoint{-0.027778in}{0.000000in}}%
\pgfusepath{stroke,fill}%
}%
\begin{pgfscope}%
\pgfsys@transformshift{0.625000in}{2.688826in}%
\pgfsys@useobject{currentmarker}{}%
\end{pgfscope}%
\end{pgfscope}%
\begin{pgfscope}%
\pgfsetbuttcap%
\pgfsetroundjoin%
\definecolor{currentfill}{rgb}{0.000000,0.000000,0.000000}%
\pgfsetfillcolor{currentfill}%
\pgfsetlinewidth{0.602250pt}%
\definecolor{currentstroke}{rgb}{0.000000,0.000000,0.000000}%
\pgfsetstrokecolor{currentstroke}%
\pgfsetdash{}{0pt}%
\pgfsys@defobject{currentmarker}{\pgfqpoint{-0.027778in}{0.000000in}}{\pgfqpoint{0.000000in}{0.000000in}}{%
\pgfpathmoveto{\pgfqpoint{0.000000in}{0.000000in}}%
\pgfpathlineto{\pgfqpoint{-0.027778in}{0.000000in}}%
\pgfusepath{stroke,fill}%
}%
\begin{pgfscope}%
\pgfsys@transformshift{0.625000in}{2.996160in}%
\pgfsys@useobject{currentmarker}{}%
\end{pgfscope}%
\end{pgfscope}%
\begin{pgfscope}%
\pgfsetbuttcap%
\pgfsetroundjoin%
\definecolor{currentfill}{rgb}{0.000000,0.000000,0.000000}%
\pgfsetfillcolor{currentfill}%
\pgfsetlinewidth{0.602250pt}%
\definecolor{currentstroke}{rgb}{0.000000,0.000000,0.000000}%
\pgfsetstrokecolor{currentstroke}%
\pgfsetdash{}{0pt}%
\pgfsys@defobject{currentmarker}{\pgfqpoint{-0.027778in}{0.000000in}}{\pgfqpoint{0.000000in}{0.000000in}}{%
\pgfpathmoveto{\pgfqpoint{0.000000in}{0.000000in}}%
\pgfpathlineto{\pgfqpoint{-0.027778in}{0.000000in}}%
\pgfusepath{stroke,fill}%
}%
\begin{pgfscope}%
\pgfsys@transformshift{0.625000in}{3.152217in}%
\pgfsys@useobject{currentmarker}{}%
\end{pgfscope}%
\end{pgfscope}%
\begin{pgfscope}%
\pgfsetbuttcap%
\pgfsetroundjoin%
\definecolor{currentfill}{rgb}{0.000000,0.000000,0.000000}%
\pgfsetfillcolor{currentfill}%
\pgfsetlinewidth{0.602250pt}%
\definecolor{currentstroke}{rgb}{0.000000,0.000000,0.000000}%
\pgfsetstrokecolor{currentstroke}%
\pgfsetdash{}{0pt}%
\pgfsys@defobject{currentmarker}{\pgfqpoint{-0.027778in}{0.000000in}}{\pgfqpoint{0.000000in}{0.000000in}}{%
\pgfpathmoveto{\pgfqpoint{0.000000in}{0.000000in}}%
\pgfpathlineto{\pgfqpoint{-0.027778in}{0.000000in}}%
\pgfusepath{stroke,fill}%
}%
\begin{pgfscope}%
\pgfsys@transformshift{0.625000in}{3.262942in}%
\pgfsys@useobject{currentmarker}{}%
\end{pgfscope}%
\end{pgfscope}%
\begin{pgfscope}%
\pgfsetbuttcap%
\pgfsetroundjoin%
\definecolor{currentfill}{rgb}{0.000000,0.000000,0.000000}%
\pgfsetfillcolor{currentfill}%
\pgfsetlinewidth{0.602250pt}%
\definecolor{currentstroke}{rgb}{0.000000,0.000000,0.000000}%
\pgfsetstrokecolor{currentstroke}%
\pgfsetdash{}{0pt}%
\pgfsys@defobject{currentmarker}{\pgfqpoint{-0.027778in}{0.000000in}}{\pgfqpoint{0.000000in}{0.000000in}}{%
\pgfpathmoveto{\pgfqpoint{0.000000in}{0.000000in}}%
\pgfpathlineto{\pgfqpoint{-0.027778in}{0.000000in}}%
\pgfusepath{stroke,fill}%
}%
\begin{pgfscope}%
\pgfsys@transformshift{0.625000in}{3.348827in}%
\pgfsys@useobject{currentmarker}{}%
\end{pgfscope}%
\end{pgfscope}%
\begin{pgfscope}%
\pgfsetbuttcap%
\pgfsetroundjoin%
\definecolor{currentfill}{rgb}{0.000000,0.000000,0.000000}%
\pgfsetfillcolor{currentfill}%
\pgfsetlinewidth{0.602250pt}%
\definecolor{currentstroke}{rgb}{0.000000,0.000000,0.000000}%
\pgfsetstrokecolor{currentstroke}%
\pgfsetdash{}{0pt}%
\pgfsys@defobject{currentmarker}{\pgfqpoint{-0.027778in}{0.000000in}}{\pgfqpoint{0.000000in}{0.000000in}}{%
\pgfpathmoveto{\pgfqpoint{0.000000in}{0.000000in}}%
\pgfpathlineto{\pgfqpoint{-0.027778in}{0.000000in}}%
\pgfusepath{stroke,fill}%
}%
\begin{pgfscope}%
\pgfsys@transformshift{0.625000in}{3.419000in}%
\pgfsys@useobject{currentmarker}{}%
\end{pgfscope}%
\end{pgfscope}%
\begin{pgfscope}%
\pgfsetbuttcap%
\pgfsetroundjoin%
\definecolor{currentfill}{rgb}{0.000000,0.000000,0.000000}%
\pgfsetfillcolor{currentfill}%
\pgfsetlinewidth{0.602250pt}%
\definecolor{currentstroke}{rgb}{0.000000,0.000000,0.000000}%
\pgfsetstrokecolor{currentstroke}%
\pgfsetdash{}{0pt}%
\pgfsys@defobject{currentmarker}{\pgfqpoint{-0.027778in}{0.000000in}}{\pgfqpoint{0.000000in}{0.000000in}}{%
\pgfpathmoveto{\pgfqpoint{0.000000in}{0.000000in}}%
\pgfpathlineto{\pgfqpoint{-0.027778in}{0.000000in}}%
\pgfusepath{stroke,fill}%
}%
\begin{pgfscope}%
\pgfsys@transformshift{0.625000in}{3.478330in}%
\pgfsys@useobject{currentmarker}{}%
\end{pgfscope}%
\end{pgfscope}%
\begin{pgfscope}%
\definecolor{textcolor}{rgb}{0.000000,0.000000,0.000000}%
\pgfsetstrokecolor{textcolor}%
\pgfsetfillcolor{textcolor}%
\pgftext[x=0.184220in,y=1.980000in,,bottom,rotate=90.000000]{\color{textcolor}\sffamily\fontsize{10.000000}{12.000000}\selectfont QMC Error}%
\end{pgfscope}%
\begin{pgfscope}%
\pgfpathrectangle{\pgfqpoint{0.625000in}{0.440000in}}{\pgfqpoint{3.875000in}{3.080000in}}%
\pgfusepath{clip}%
\pgfsetbuttcap%
\pgfsetroundjoin%
\definecolor{currentfill}{rgb}{0.000000,0.000000,0.000000}%
\pgfsetfillcolor{currentfill}%
\pgfsetlinewidth{1.003750pt}%
\definecolor{currentstroke}{rgb}{0.000000,0.000000,0.000000}%
\pgfsetstrokecolor{currentstroke}%
\pgfsetdash{}{0pt}%
\pgfsys@defobject{currentmarker}{\pgfqpoint{-0.041667in}{-0.041667in}}{\pgfqpoint{0.041667in}{0.041667in}}{%
\pgfpathmoveto{\pgfqpoint{0.000000in}{-0.041667in}}%
\pgfpathcurveto{\pgfqpoint{0.011050in}{-0.041667in}}{\pgfqpoint{0.021649in}{-0.037276in}}{\pgfqpoint{0.029463in}{-0.029463in}}%
\pgfpathcurveto{\pgfqpoint{0.037276in}{-0.021649in}}{\pgfqpoint{0.041667in}{-0.011050in}}{\pgfqpoint{0.041667in}{0.000000in}}%
\pgfpathcurveto{\pgfqpoint{0.041667in}{0.011050in}}{\pgfqpoint{0.037276in}{0.021649in}}{\pgfqpoint{0.029463in}{0.029463in}}%
\pgfpathcurveto{\pgfqpoint{0.021649in}{0.037276in}}{\pgfqpoint{0.011050in}{0.041667in}}{\pgfqpoint{0.000000in}{0.041667in}}%
\pgfpathcurveto{\pgfqpoint{-0.011050in}{0.041667in}}{\pgfqpoint{-0.021649in}{0.037276in}}{\pgfqpoint{-0.029463in}{0.029463in}}%
\pgfpathcurveto{\pgfqpoint{-0.037276in}{0.021649in}}{\pgfqpoint{-0.041667in}{0.011050in}}{\pgfqpoint{-0.041667in}{0.000000in}}%
\pgfpathcurveto{\pgfqpoint{-0.041667in}{-0.011050in}}{\pgfqpoint{-0.037276in}{-0.021649in}}{\pgfqpoint{-0.029463in}{-0.029463in}}%
\pgfpathcurveto{\pgfqpoint{-0.021649in}{-0.037276in}}{\pgfqpoint{-0.011050in}{-0.041667in}}{\pgfqpoint{0.000000in}{-0.041667in}}%
\pgfpathclose%
\pgfusepath{stroke,fill}%
}%
\begin{pgfscope}%
\pgfsys@transformshift{0.801136in}{3.323263in}%
\pgfsys@useobject{currentmarker}{}%
\end{pgfscope}%
\begin{pgfscope}%
\pgfsys@transformshift{1.121384in}{3.233313in}%
\pgfsys@useobject{currentmarker}{}%
\end{pgfscope}%
\begin{pgfscope}%
\pgfsys@transformshift{1.441632in}{2.750199in}%
\pgfsys@useobject{currentmarker}{}%
\end{pgfscope}%
\begin{pgfscope}%
\pgfsys@transformshift{1.761880in}{2.594848in}%
\pgfsys@useobject{currentmarker}{}%
\end{pgfscope}%
\begin{pgfscope}%
\pgfsys@transformshift{2.082128in}{2.381956in}%
\pgfsys@useobject{currentmarker}{}%
\end{pgfscope}%
\begin{pgfscope}%
\pgfsys@transformshift{2.402376in}{2.219327in}%
\pgfsys@useobject{currentmarker}{}%
\end{pgfscope}%
\begin{pgfscope}%
\pgfsys@transformshift{2.722624in}{2.132874in}%
\pgfsys@useobject{currentmarker}{}%
\end{pgfscope}%
\begin{pgfscope}%
\pgfsys@transformshift{3.042872in}{2.028116in}%
\pgfsys@useobject{currentmarker}{}%
\end{pgfscope}%
\begin{pgfscope}%
\pgfsys@transformshift{3.363120in}{1.481470in}%
\pgfsys@useobject{currentmarker}{}%
\end{pgfscope}%
\begin{pgfscope}%
\pgfsys@transformshift{3.683368in}{1.230665in}%
\pgfsys@useobject{currentmarker}{}%
\end{pgfscope}%
\begin{pgfscope}%
\pgfsys@transformshift{4.003616in}{1.002166in}%
\pgfsys@useobject{currentmarker}{}%
\end{pgfscope}%
\begin{pgfscope}%
\pgfsys@transformshift{4.323864in}{0.580000in}%
\pgfsys@useobject{currentmarker}{}%
\end{pgfscope}%
\end{pgfscope}%
\begin{pgfscope}%
\pgfpathrectangle{\pgfqpoint{0.625000in}{0.440000in}}{\pgfqpoint{3.875000in}{3.080000in}}%
\pgfusepath{clip}%
\pgfsetbuttcap%
\pgfsetroundjoin%
\pgfsetlinewidth{1.505625pt}%
\definecolor{currentstroke}{rgb}{0.000000,0.000000,0.000000}%
\pgfsetstrokecolor{currentstroke}%
\pgfsetdash{{5.550000pt}{2.400000pt}}{0.000000pt}%
\pgfpathmoveto{\pgfqpoint{0.801136in}{3.380000in}}%
\pgfpathlineto{\pgfqpoint{4.323864in}{0.779699in}}%
\pgfusepath{stroke}%
\end{pgfscope}%
\begin{pgfscope}%
\pgfsetrectcap%
\pgfsetmiterjoin%
\pgfsetlinewidth{0.803000pt}%
\definecolor{currentstroke}{rgb}{0.000000,0.000000,0.000000}%
\pgfsetstrokecolor{currentstroke}%
\pgfsetdash{}{0pt}%
\pgfpathmoveto{\pgfqpoint{0.625000in}{0.440000in}}%
\pgfpathlineto{\pgfqpoint{0.625000in}{3.520000in}}%
\pgfusepath{stroke}%
\end{pgfscope}%
\begin{pgfscope}%
\pgfsetrectcap%
\pgfsetmiterjoin%
\pgfsetlinewidth{0.000000pt}%
\definecolor{currentstroke}{rgb}{0.000000,0.000000,0.000000}%
\pgfsetstrokecolor{currentstroke}%
\pgfsetstrokeopacity{0.000000}%
\pgfsetdash{}{0pt}%
\pgfpathmoveto{\pgfqpoint{4.500000in}{0.440000in}}%
\pgfpathlineto{\pgfqpoint{4.500000in}{3.520000in}}%
\pgfusepath{}%
\end{pgfscope}%
\begin{pgfscope}%
\pgfsetrectcap%
\pgfsetmiterjoin%
\pgfsetlinewidth{0.803000pt}%
\definecolor{currentstroke}{rgb}{0.000000,0.000000,0.000000}%
\pgfsetstrokecolor{currentstroke}%
\pgfsetdash{}{0pt}%
\pgfpathmoveto{\pgfqpoint{0.625000in}{0.440000in}}%
\pgfpathlineto{\pgfqpoint{4.500000in}{0.440000in}}%
\pgfusepath{stroke}%
\end{pgfscope}%
\begin{pgfscope}%
\pgfsetrectcap%
\pgfsetmiterjoin%
\pgfsetlinewidth{0.000000pt}%
\definecolor{currentstroke}{rgb}{0.000000,0.000000,0.000000}%
\pgfsetstrokecolor{currentstroke}%
\pgfsetstrokeopacity{0.000000}%
\pgfsetdash{}{0pt}%
\pgfpathmoveto{\pgfqpoint{0.625000in}{3.520000in}}%
\pgfpathlineto{\pgfqpoint{4.500000in}{3.520000in}}%
\pgfusepath{}%
\end{pgfscope}%
\begin{pgfscope}%
\pgfsetbuttcap%
\pgfsetmiterjoin%
\definecolor{currentfill}{rgb}{1.000000,1.000000,1.000000}%
\pgfsetfillcolor{currentfill}%
\pgfsetfillopacity{0.800000}%
\pgfsetlinewidth{1.003750pt}%
\definecolor{currentstroke}{rgb}{0.800000,0.800000,0.800000}%
\pgfsetstrokecolor{currentstroke}%
\pgfsetstrokeopacity{0.800000}%
\pgfsetdash{}{0pt}%
\pgfpathmoveto{\pgfqpoint{3.042243in}{3.155267in}}%
\pgfpathlineto{\pgfqpoint{4.402778in}{3.155267in}}%
\pgfpathquadraticcurveto{\pgfqpoint{4.430556in}{3.155267in}}{\pgfqpoint{4.430556in}{3.183045in}}%
\pgfpathlineto{\pgfqpoint{4.430556in}{3.422778in}}%
\pgfpathquadraticcurveto{\pgfqpoint{4.430556in}{3.450556in}}{\pgfqpoint{4.402778in}{3.450556in}}%
\pgfpathlineto{\pgfqpoint{3.042243in}{3.450556in}}%
\pgfpathquadraticcurveto{\pgfqpoint{3.014465in}{3.450556in}}{\pgfqpoint{3.014465in}{3.422778in}}%
\pgfpathlineto{\pgfqpoint{3.014465in}{3.183045in}}%
\pgfpathquadraticcurveto{\pgfqpoint{3.014465in}{3.155267in}}{\pgfqpoint{3.042243in}{3.155267in}}%
\pgfpathclose%
\pgfusepath{stroke,fill}%
\end{pgfscope}%
\begin{pgfscope}%
\pgfsetbuttcap%
\pgfsetroundjoin%
\pgfsetlinewidth{1.505625pt}%
\definecolor{currentstroke}{rgb}{0.000000,0.000000,0.000000}%
\pgfsetstrokecolor{currentstroke}%
\pgfsetdash{{5.550000pt}{2.400000pt}}{0.000000pt}%
\pgfpathmoveto{\pgfqpoint{3.070021in}{3.321543in}}%
\pgfpathlineto{\pgfqpoint{3.347799in}{3.321543in}}%
\pgfusepath{stroke}%
\end{pgfscope}%
\begin{pgfscope}%
\definecolor{textcolor}{rgb}{0.000000,0.000000,0.000000}%
\pgfsetstrokecolor{textcolor}%
\pgfsetfillcolor{textcolor}%
\pgftext[x=3.458910in,y=3.272932in,left,base]{\color{textcolor}\sffamily\fontsize{10.000000}{12.000000}\selectfont \(\displaystyle 5.4218N_{\mathrm{QMC}}^{-0.8860}\)}%
\end{pgfscope}%
\end{pgfpicture}%
\makeatother%
\endgroup%

    \caption[Quasi-Monte-Carlo error, for $Q=\grad u((1,1))$ and $k=30$.]{Quasi-Monte-Carlo error with increasing number of Quasi-Monte-Carlo points, for $Q=\grad u((1,1))$ and $k=30$. \label{fig:qmcgradient_top_right30}}
  \end{figure}
\begin{figure}[h]
  \centering
  %% Creator: Matplotlib, PGF backend
%%
%% To include the figure in your LaTeX document, write
%%   \input{<filename>.pgf}
%%
%% Make sure the required packages are loaded in your preamble
%%   \usepackage{pgf}
%%
%% Figures using additional raster images can only be included by \input if
%% they are in the same directory as the main LaTeX file. For loading figures
%% from other directories you can use the `import` package
%%   \usepackage{import}
%% and then include the figures with
%%   \import{<path to file>}{<filename>.pgf}
%%
%% Matplotlib used the following preamble
%%   \usepackage{fontspec}
%%   \setmainfont{DejaVuSerif.ttf}[Path=/home/owen/progs/firedrake-complex/firedrake/lib/python3.5/site-packages/matplotlib/mpl-data/fonts/ttf/]
%%   \setsansfont{DejaVuSans.ttf}[Path=/home/owen/progs/firedrake-complex/firedrake/lib/python3.5/site-packages/matplotlib/mpl-data/fonts/ttf/]
%%   \setmonofont{DejaVuSansMono.ttf}[Path=/home/owen/progs/firedrake-complex/firedrake/lib/python3.5/site-packages/matplotlib/mpl-data/fonts/ttf/]
%%
\begingroup%
\makeatletter%
\begin{pgfpicture}%
\pgfpathrectangle{\pgfpointorigin}{\pgfqpoint{5.000000in}{4.000000in}}%
\pgfusepath{use as bounding box, clip}%
\begin{pgfscope}%
\pgfsetbuttcap%
\pgfsetmiterjoin%
\definecolor{currentfill}{rgb}{1.000000,1.000000,1.000000}%
\pgfsetfillcolor{currentfill}%
\pgfsetlinewidth{0.000000pt}%
\definecolor{currentstroke}{rgb}{1.000000,1.000000,1.000000}%
\pgfsetstrokecolor{currentstroke}%
\pgfsetdash{}{0pt}%
\pgfpathmoveto{\pgfqpoint{0.000000in}{0.000000in}}%
\pgfpathlineto{\pgfqpoint{5.000000in}{0.000000in}}%
\pgfpathlineto{\pgfqpoint{5.000000in}{4.000000in}}%
\pgfpathlineto{\pgfqpoint{0.000000in}{4.000000in}}%
\pgfpathclose%
\pgfusepath{fill}%
\end{pgfscope}%
\begin{pgfscope}%
\pgfsetbuttcap%
\pgfsetmiterjoin%
\definecolor{currentfill}{rgb}{1.000000,1.000000,1.000000}%
\pgfsetfillcolor{currentfill}%
\pgfsetlinewidth{0.000000pt}%
\definecolor{currentstroke}{rgb}{0.000000,0.000000,0.000000}%
\pgfsetstrokecolor{currentstroke}%
\pgfsetstrokeopacity{0.000000}%
\pgfsetdash{}{0pt}%
\pgfpathmoveto{\pgfqpoint{0.625000in}{0.440000in}}%
\pgfpathlineto{\pgfqpoint{4.500000in}{0.440000in}}%
\pgfpathlineto{\pgfqpoint{4.500000in}{3.520000in}}%
\pgfpathlineto{\pgfqpoint{0.625000in}{3.520000in}}%
\pgfpathclose%
\pgfusepath{fill}%
\end{pgfscope}%
\begin{pgfscope}%
\pgfsetbuttcap%
\pgfsetroundjoin%
\definecolor{currentfill}{rgb}{0.000000,0.000000,0.000000}%
\pgfsetfillcolor{currentfill}%
\pgfsetlinewidth{0.803000pt}%
\definecolor{currentstroke}{rgb}{0.000000,0.000000,0.000000}%
\pgfsetstrokecolor{currentstroke}%
\pgfsetdash{}{0pt}%
\pgfsys@defobject{currentmarker}{\pgfqpoint{0.000000in}{-0.048611in}}{\pgfqpoint{0.000000in}{0.000000in}}{%
\pgfpathmoveto{\pgfqpoint{0.000000in}{0.000000in}}%
\pgfpathlineto{\pgfqpoint{0.000000in}{-0.048611in}}%
\pgfusepath{stroke,fill}%
}%
\begin{pgfscope}%
\pgfsys@transformshift{0.801136in}{0.440000in}%
\pgfsys@useobject{currentmarker}{}%
\end{pgfscope}%
\end{pgfscope}%
\begin{pgfscope}%
\definecolor{textcolor}{rgb}{0.000000,0.000000,0.000000}%
\pgfsetstrokecolor{textcolor}%
\pgfsetfillcolor{textcolor}%
\pgftext[x=0.801136in,y=0.342778in,,top]{\color{textcolor}\sffamily\fontsize{10.000000}{12.000000}\selectfont \(\displaystyle 10^{0}\)}%
\end{pgfscope}%
\begin{pgfscope}%
\pgfsetbuttcap%
\pgfsetroundjoin%
\definecolor{currentfill}{rgb}{0.000000,0.000000,0.000000}%
\pgfsetfillcolor{currentfill}%
\pgfsetlinewidth{0.803000pt}%
\definecolor{currentstroke}{rgb}{0.000000,0.000000,0.000000}%
\pgfsetstrokecolor{currentstroke}%
\pgfsetdash{}{0pt}%
\pgfsys@defobject{currentmarker}{\pgfqpoint{0.000000in}{-0.048611in}}{\pgfqpoint{0.000000in}{0.000000in}}{%
\pgfpathmoveto{\pgfqpoint{0.000000in}{0.000000in}}%
\pgfpathlineto{\pgfqpoint{0.000000in}{-0.048611in}}%
\pgfusepath{stroke,fill}%
}%
\begin{pgfscope}%
\pgfsys@transformshift{1.864977in}{0.440000in}%
\pgfsys@useobject{currentmarker}{}%
\end{pgfscope}%
\end{pgfscope}%
\begin{pgfscope}%
\definecolor{textcolor}{rgb}{0.000000,0.000000,0.000000}%
\pgfsetstrokecolor{textcolor}%
\pgfsetfillcolor{textcolor}%
\pgftext[x=1.864977in,y=0.342778in,,top]{\color{textcolor}\sffamily\fontsize{10.000000}{12.000000}\selectfont \(\displaystyle 10^{1}\)}%
\end{pgfscope}%
\begin{pgfscope}%
\pgfsetbuttcap%
\pgfsetroundjoin%
\definecolor{currentfill}{rgb}{0.000000,0.000000,0.000000}%
\pgfsetfillcolor{currentfill}%
\pgfsetlinewidth{0.803000pt}%
\definecolor{currentstroke}{rgb}{0.000000,0.000000,0.000000}%
\pgfsetstrokecolor{currentstroke}%
\pgfsetdash{}{0pt}%
\pgfsys@defobject{currentmarker}{\pgfqpoint{0.000000in}{-0.048611in}}{\pgfqpoint{0.000000in}{0.000000in}}{%
\pgfpathmoveto{\pgfqpoint{0.000000in}{0.000000in}}%
\pgfpathlineto{\pgfqpoint{0.000000in}{-0.048611in}}%
\pgfusepath{stroke,fill}%
}%
\begin{pgfscope}%
\pgfsys@transformshift{2.928818in}{0.440000in}%
\pgfsys@useobject{currentmarker}{}%
\end{pgfscope}%
\end{pgfscope}%
\begin{pgfscope}%
\definecolor{textcolor}{rgb}{0.000000,0.000000,0.000000}%
\pgfsetstrokecolor{textcolor}%
\pgfsetfillcolor{textcolor}%
\pgftext[x=2.928818in,y=0.342778in,,top]{\color{textcolor}\sffamily\fontsize{10.000000}{12.000000}\selectfont \(\displaystyle 10^{2}\)}%
\end{pgfscope}%
\begin{pgfscope}%
\pgfsetbuttcap%
\pgfsetroundjoin%
\definecolor{currentfill}{rgb}{0.000000,0.000000,0.000000}%
\pgfsetfillcolor{currentfill}%
\pgfsetlinewidth{0.803000pt}%
\definecolor{currentstroke}{rgb}{0.000000,0.000000,0.000000}%
\pgfsetstrokecolor{currentstroke}%
\pgfsetdash{}{0pt}%
\pgfsys@defobject{currentmarker}{\pgfqpoint{0.000000in}{-0.048611in}}{\pgfqpoint{0.000000in}{0.000000in}}{%
\pgfpathmoveto{\pgfqpoint{0.000000in}{0.000000in}}%
\pgfpathlineto{\pgfqpoint{0.000000in}{-0.048611in}}%
\pgfusepath{stroke,fill}%
}%
\begin{pgfscope}%
\pgfsys@transformshift{3.992658in}{0.440000in}%
\pgfsys@useobject{currentmarker}{}%
\end{pgfscope}%
\end{pgfscope}%
\begin{pgfscope}%
\definecolor{textcolor}{rgb}{0.000000,0.000000,0.000000}%
\pgfsetstrokecolor{textcolor}%
\pgfsetfillcolor{textcolor}%
\pgftext[x=3.992658in,y=0.342778in,,top]{\color{textcolor}\sffamily\fontsize{10.000000}{12.000000}\selectfont \(\displaystyle 10^{3}\)}%
\end{pgfscope}%
\begin{pgfscope}%
\pgfsetbuttcap%
\pgfsetroundjoin%
\definecolor{currentfill}{rgb}{0.000000,0.000000,0.000000}%
\pgfsetfillcolor{currentfill}%
\pgfsetlinewidth{0.602250pt}%
\definecolor{currentstroke}{rgb}{0.000000,0.000000,0.000000}%
\pgfsetstrokecolor{currentstroke}%
\pgfsetdash{}{0pt}%
\pgfsys@defobject{currentmarker}{\pgfqpoint{0.000000in}{-0.027778in}}{\pgfqpoint{0.000000in}{0.000000in}}{%
\pgfpathmoveto{\pgfqpoint{0.000000in}{0.000000in}}%
\pgfpathlineto{\pgfqpoint{0.000000in}{-0.027778in}}%
\pgfusepath{stroke,fill}%
}%
\begin{pgfscope}%
\pgfsys@transformshift{0.636345in}{0.440000in}%
\pgfsys@useobject{currentmarker}{}%
\end{pgfscope}%
\end{pgfscope}%
\begin{pgfscope}%
\pgfsetbuttcap%
\pgfsetroundjoin%
\definecolor{currentfill}{rgb}{0.000000,0.000000,0.000000}%
\pgfsetfillcolor{currentfill}%
\pgfsetlinewidth{0.602250pt}%
\definecolor{currentstroke}{rgb}{0.000000,0.000000,0.000000}%
\pgfsetstrokecolor{currentstroke}%
\pgfsetdash{}{0pt}%
\pgfsys@defobject{currentmarker}{\pgfqpoint{0.000000in}{-0.027778in}}{\pgfqpoint{0.000000in}{0.000000in}}{%
\pgfpathmoveto{\pgfqpoint{0.000000in}{0.000000in}}%
\pgfpathlineto{\pgfqpoint{0.000000in}{-0.027778in}}%
\pgfusepath{stroke,fill}%
}%
\begin{pgfscope}%
\pgfsys@transformshift{0.698040in}{0.440000in}%
\pgfsys@useobject{currentmarker}{}%
\end{pgfscope}%
\end{pgfscope}%
\begin{pgfscope}%
\pgfsetbuttcap%
\pgfsetroundjoin%
\definecolor{currentfill}{rgb}{0.000000,0.000000,0.000000}%
\pgfsetfillcolor{currentfill}%
\pgfsetlinewidth{0.602250pt}%
\definecolor{currentstroke}{rgb}{0.000000,0.000000,0.000000}%
\pgfsetstrokecolor{currentstroke}%
\pgfsetdash{}{0pt}%
\pgfsys@defobject{currentmarker}{\pgfqpoint{0.000000in}{-0.027778in}}{\pgfqpoint{0.000000in}{0.000000in}}{%
\pgfpathmoveto{\pgfqpoint{0.000000in}{0.000000in}}%
\pgfpathlineto{\pgfqpoint{0.000000in}{-0.027778in}}%
\pgfusepath{stroke,fill}%
}%
\begin{pgfscope}%
\pgfsys@transformshift{0.752458in}{0.440000in}%
\pgfsys@useobject{currentmarker}{}%
\end{pgfscope}%
\end{pgfscope}%
\begin{pgfscope}%
\pgfsetbuttcap%
\pgfsetroundjoin%
\definecolor{currentfill}{rgb}{0.000000,0.000000,0.000000}%
\pgfsetfillcolor{currentfill}%
\pgfsetlinewidth{0.602250pt}%
\definecolor{currentstroke}{rgb}{0.000000,0.000000,0.000000}%
\pgfsetstrokecolor{currentstroke}%
\pgfsetdash{}{0pt}%
\pgfsys@defobject{currentmarker}{\pgfqpoint{0.000000in}{-0.027778in}}{\pgfqpoint{0.000000in}{0.000000in}}{%
\pgfpathmoveto{\pgfqpoint{0.000000in}{0.000000in}}%
\pgfpathlineto{\pgfqpoint{0.000000in}{-0.027778in}}%
\pgfusepath{stroke,fill}%
}%
\begin{pgfscope}%
\pgfsys@transformshift{1.121384in}{0.440000in}%
\pgfsys@useobject{currentmarker}{}%
\end{pgfscope}%
\end{pgfscope}%
\begin{pgfscope}%
\pgfsetbuttcap%
\pgfsetroundjoin%
\definecolor{currentfill}{rgb}{0.000000,0.000000,0.000000}%
\pgfsetfillcolor{currentfill}%
\pgfsetlinewidth{0.602250pt}%
\definecolor{currentstroke}{rgb}{0.000000,0.000000,0.000000}%
\pgfsetstrokecolor{currentstroke}%
\pgfsetdash{}{0pt}%
\pgfsys@defobject{currentmarker}{\pgfqpoint{0.000000in}{-0.027778in}}{\pgfqpoint{0.000000in}{0.000000in}}{%
\pgfpathmoveto{\pgfqpoint{0.000000in}{0.000000in}}%
\pgfpathlineto{\pgfqpoint{0.000000in}{-0.027778in}}%
\pgfusepath{stroke,fill}%
}%
\begin{pgfscope}%
\pgfsys@transformshift{1.308717in}{0.440000in}%
\pgfsys@useobject{currentmarker}{}%
\end{pgfscope}%
\end{pgfscope}%
\begin{pgfscope}%
\pgfsetbuttcap%
\pgfsetroundjoin%
\definecolor{currentfill}{rgb}{0.000000,0.000000,0.000000}%
\pgfsetfillcolor{currentfill}%
\pgfsetlinewidth{0.602250pt}%
\definecolor{currentstroke}{rgb}{0.000000,0.000000,0.000000}%
\pgfsetstrokecolor{currentstroke}%
\pgfsetdash{}{0pt}%
\pgfsys@defobject{currentmarker}{\pgfqpoint{0.000000in}{-0.027778in}}{\pgfqpoint{0.000000in}{0.000000in}}{%
\pgfpathmoveto{\pgfqpoint{0.000000in}{0.000000in}}%
\pgfpathlineto{\pgfqpoint{0.000000in}{-0.027778in}}%
\pgfusepath{stroke,fill}%
}%
\begin{pgfscope}%
\pgfsys@transformshift{1.441632in}{0.440000in}%
\pgfsys@useobject{currentmarker}{}%
\end{pgfscope}%
\end{pgfscope}%
\begin{pgfscope}%
\pgfsetbuttcap%
\pgfsetroundjoin%
\definecolor{currentfill}{rgb}{0.000000,0.000000,0.000000}%
\pgfsetfillcolor{currentfill}%
\pgfsetlinewidth{0.602250pt}%
\definecolor{currentstroke}{rgb}{0.000000,0.000000,0.000000}%
\pgfsetstrokecolor{currentstroke}%
\pgfsetdash{}{0pt}%
\pgfsys@defobject{currentmarker}{\pgfqpoint{0.000000in}{-0.027778in}}{\pgfqpoint{0.000000in}{0.000000in}}{%
\pgfpathmoveto{\pgfqpoint{0.000000in}{0.000000in}}%
\pgfpathlineto{\pgfqpoint{0.000000in}{-0.027778in}}%
\pgfusepath{stroke,fill}%
}%
\begin{pgfscope}%
\pgfsys@transformshift{1.544729in}{0.440000in}%
\pgfsys@useobject{currentmarker}{}%
\end{pgfscope}%
\end{pgfscope}%
\begin{pgfscope}%
\pgfsetbuttcap%
\pgfsetroundjoin%
\definecolor{currentfill}{rgb}{0.000000,0.000000,0.000000}%
\pgfsetfillcolor{currentfill}%
\pgfsetlinewidth{0.602250pt}%
\definecolor{currentstroke}{rgb}{0.000000,0.000000,0.000000}%
\pgfsetstrokecolor{currentstroke}%
\pgfsetdash{}{0pt}%
\pgfsys@defobject{currentmarker}{\pgfqpoint{0.000000in}{-0.027778in}}{\pgfqpoint{0.000000in}{0.000000in}}{%
\pgfpathmoveto{\pgfqpoint{0.000000in}{0.000000in}}%
\pgfpathlineto{\pgfqpoint{0.000000in}{-0.027778in}}%
\pgfusepath{stroke,fill}%
}%
\begin{pgfscope}%
\pgfsys@transformshift{1.628965in}{0.440000in}%
\pgfsys@useobject{currentmarker}{}%
\end{pgfscope}%
\end{pgfscope}%
\begin{pgfscope}%
\pgfsetbuttcap%
\pgfsetroundjoin%
\definecolor{currentfill}{rgb}{0.000000,0.000000,0.000000}%
\pgfsetfillcolor{currentfill}%
\pgfsetlinewidth{0.602250pt}%
\definecolor{currentstroke}{rgb}{0.000000,0.000000,0.000000}%
\pgfsetstrokecolor{currentstroke}%
\pgfsetdash{}{0pt}%
\pgfsys@defobject{currentmarker}{\pgfqpoint{0.000000in}{-0.027778in}}{\pgfqpoint{0.000000in}{0.000000in}}{%
\pgfpathmoveto{\pgfqpoint{0.000000in}{0.000000in}}%
\pgfpathlineto{\pgfqpoint{0.000000in}{-0.027778in}}%
\pgfusepath{stroke,fill}%
}%
\begin{pgfscope}%
\pgfsys@transformshift{1.700186in}{0.440000in}%
\pgfsys@useobject{currentmarker}{}%
\end{pgfscope}%
\end{pgfscope}%
\begin{pgfscope}%
\pgfsetbuttcap%
\pgfsetroundjoin%
\definecolor{currentfill}{rgb}{0.000000,0.000000,0.000000}%
\pgfsetfillcolor{currentfill}%
\pgfsetlinewidth{0.602250pt}%
\definecolor{currentstroke}{rgb}{0.000000,0.000000,0.000000}%
\pgfsetstrokecolor{currentstroke}%
\pgfsetdash{}{0pt}%
\pgfsys@defobject{currentmarker}{\pgfqpoint{0.000000in}{-0.027778in}}{\pgfqpoint{0.000000in}{0.000000in}}{%
\pgfpathmoveto{\pgfqpoint{0.000000in}{0.000000in}}%
\pgfpathlineto{\pgfqpoint{0.000000in}{-0.027778in}}%
\pgfusepath{stroke,fill}%
}%
\begin{pgfscope}%
\pgfsys@transformshift{1.761880in}{0.440000in}%
\pgfsys@useobject{currentmarker}{}%
\end{pgfscope}%
\end{pgfscope}%
\begin{pgfscope}%
\pgfsetbuttcap%
\pgfsetroundjoin%
\definecolor{currentfill}{rgb}{0.000000,0.000000,0.000000}%
\pgfsetfillcolor{currentfill}%
\pgfsetlinewidth{0.602250pt}%
\definecolor{currentstroke}{rgb}{0.000000,0.000000,0.000000}%
\pgfsetstrokecolor{currentstroke}%
\pgfsetdash{}{0pt}%
\pgfsys@defobject{currentmarker}{\pgfqpoint{0.000000in}{-0.027778in}}{\pgfqpoint{0.000000in}{0.000000in}}{%
\pgfpathmoveto{\pgfqpoint{0.000000in}{0.000000in}}%
\pgfpathlineto{\pgfqpoint{0.000000in}{-0.027778in}}%
\pgfusepath{stroke,fill}%
}%
\begin{pgfscope}%
\pgfsys@transformshift{1.816298in}{0.440000in}%
\pgfsys@useobject{currentmarker}{}%
\end{pgfscope}%
\end{pgfscope}%
\begin{pgfscope}%
\pgfsetbuttcap%
\pgfsetroundjoin%
\definecolor{currentfill}{rgb}{0.000000,0.000000,0.000000}%
\pgfsetfillcolor{currentfill}%
\pgfsetlinewidth{0.602250pt}%
\definecolor{currentstroke}{rgb}{0.000000,0.000000,0.000000}%
\pgfsetstrokecolor{currentstroke}%
\pgfsetdash{}{0pt}%
\pgfsys@defobject{currentmarker}{\pgfqpoint{0.000000in}{-0.027778in}}{\pgfqpoint{0.000000in}{0.000000in}}{%
\pgfpathmoveto{\pgfqpoint{0.000000in}{0.000000in}}%
\pgfpathlineto{\pgfqpoint{0.000000in}{-0.027778in}}%
\pgfusepath{stroke,fill}%
}%
\begin{pgfscope}%
\pgfsys@transformshift{2.185225in}{0.440000in}%
\pgfsys@useobject{currentmarker}{}%
\end{pgfscope}%
\end{pgfscope}%
\begin{pgfscope}%
\pgfsetbuttcap%
\pgfsetroundjoin%
\definecolor{currentfill}{rgb}{0.000000,0.000000,0.000000}%
\pgfsetfillcolor{currentfill}%
\pgfsetlinewidth{0.602250pt}%
\definecolor{currentstroke}{rgb}{0.000000,0.000000,0.000000}%
\pgfsetstrokecolor{currentstroke}%
\pgfsetdash{}{0pt}%
\pgfsys@defobject{currentmarker}{\pgfqpoint{0.000000in}{-0.027778in}}{\pgfqpoint{0.000000in}{0.000000in}}{%
\pgfpathmoveto{\pgfqpoint{0.000000in}{0.000000in}}%
\pgfpathlineto{\pgfqpoint{0.000000in}{-0.027778in}}%
\pgfusepath{stroke,fill}%
}%
\begin{pgfscope}%
\pgfsys@transformshift{2.372558in}{0.440000in}%
\pgfsys@useobject{currentmarker}{}%
\end{pgfscope}%
\end{pgfscope}%
\begin{pgfscope}%
\pgfsetbuttcap%
\pgfsetroundjoin%
\definecolor{currentfill}{rgb}{0.000000,0.000000,0.000000}%
\pgfsetfillcolor{currentfill}%
\pgfsetlinewidth{0.602250pt}%
\definecolor{currentstroke}{rgb}{0.000000,0.000000,0.000000}%
\pgfsetstrokecolor{currentstroke}%
\pgfsetdash{}{0pt}%
\pgfsys@defobject{currentmarker}{\pgfqpoint{0.000000in}{-0.027778in}}{\pgfqpoint{0.000000in}{0.000000in}}{%
\pgfpathmoveto{\pgfqpoint{0.000000in}{0.000000in}}%
\pgfpathlineto{\pgfqpoint{0.000000in}{-0.027778in}}%
\pgfusepath{stroke,fill}%
}%
\begin{pgfscope}%
\pgfsys@transformshift{2.505473in}{0.440000in}%
\pgfsys@useobject{currentmarker}{}%
\end{pgfscope}%
\end{pgfscope}%
\begin{pgfscope}%
\pgfsetbuttcap%
\pgfsetroundjoin%
\definecolor{currentfill}{rgb}{0.000000,0.000000,0.000000}%
\pgfsetfillcolor{currentfill}%
\pgfsetlinewidth{0.602250pt}%
\definecolor{currentstroke}{rgb}{0.000000,0.000000,0.000000}%
\pgfsetstrokecolor{currentstroke}%
\pgfsetdash{}{0pt}%
\pgfsys@defobject{currentmarker}{\pgfqpoint{0.000000in}{-0.027778in}}{\pgfqpoint{0.000000in}{0.000000in}}{%
\pgfpathmoveto{\pgfqpoint{0.000000in}{0.000000in}}%
\pgfpathlineto{\pgfqpoint{0.000000in}{-0.027778in}}%
\pgfusepath{stroke,fill}%
}%
\begin{pgfscope}%
\pgfsys@transformshift{2.608570in}{0.440000in}%
\pgfsys@useobject{currentmarker}{}%
\end{pgfscope}%
\end{pgfscope}%
\begin{pgfscope}%
\pgfsetbuttcap%
\pgfsetroundjoin%
\definecolor{currentfill}{rgb}{0.000000,0.000000,0.000000}%
\pgfsetfillcolor{currentfill}%
\pgfsetlinewidth{0.602250pt}%
\definecolor{currentstroke}{rgb}{0.000000,0.000000,0.000000}%
\pgfsetstrokecolor{currentstroke}%
\pgfsetdash{}{0pt}%
\pgfsys@defobject{currentmarker}{\pgfqpoint{0.000000in}{-0.027778in}}{\pgfqpoint{0.000000in}{0.000000in}}{%
\pgfpathmoveto{\pgfqpoint{0.000000in}{0.000000in}}%
\pgfpathlineto{\pgfqpoint{0.000000in}{-0.027778in}}%
\pgfusepath{stroke,fill}%
}%
\begin{pgfscope}%
\pgfsys@transformshift{2.692806in}{0.440000in}%
\pgfsys@useobject{currentmarker}{}%
\end{pgfscope}%
\end{pgfscope}%
\begin{pgfscope}%
\pgfsetbuttcap%
\pgfsetroundjoin%
\definecolor{currentfill}{rgb}{0.000000,0.000000,0.000000}%
\pgfsetfillcolor{currentfill}%
\pgfsetlinewidth{0.602250pt}%
\definecolor{currentstroke}{rgb}{0.000000,0.000000,0.000000}%
\pgfsetstrokecolor{currentstroke}%
\pgfsetdash{}{0pt}%
\pgfsys@defobject{currentmarker}{\pgfqpoint{0.000000in}{-0.027778in}}{\pgfqpoint{0.000000in}{0.000000in}}{%
\pgfpathmoveto{\pgfqpoint{0.000000in}{0.000000in}}%
\pgfpathlineto{\pgfqpoint{0.000000in}{-0.027778in}}%
\pgfusepath{stroke,fill}%
}%
\begin{pgfscope}%
\pgfsys@transformshift{2.764027in}{0.440000in}%
\pgfsys@useobject{currentmarker}{}%
\end{pgfscope}%
\end{pgfscope}%
\begin{pgfscope}%
\pgfsetbuttcap%
\pgfsetroundjoin%
\definecolor{currentfill}{rgb}{0.000000,0.000000,0.000000}%
\pgfsetfillcolor{currentfill}%
\pgfsetlinewidth{0.602250pt}%
\definecolor{currentstroke}{rgb}{0.000000,0.000000,0.000000}%
\pgfsetstrokecolor{currentstroke}%
\pgfsetdash{}{0pt}%
\pgfsys@defobject{currentmarker}{\pgfqpoint{0.000000in}{-0.027778in}}{\pgfqpoint{0.000000in}{0.000000in}}{%
\pgfpathmoveto{\pgfqpoint{0.000000in}{0.000000in}}%
\pgfpathlineto{\pgfqpoint{0.000000in}{-0.027778in}}%
\pgfusepath{stroke,fill}%
}%
\begin{pgfscope}%
\pgfsys@transformshift{2.825721in}{0.440000in}%
\pgfsys@useobject{currentmarker}{}%
\end{pgfscope}%
\end{pgfscope}%
\begin{pgfscope}%
\pgfsetbuttcap%
\pgfsetroundjoin%
\definecolor{currentfill}{rgb}{0.000000,0.000000,0.000000}%
\pgfsetfillcolor{currentfill}%
\pgfsetlinewidth{0.602250pt}%
\definecolor{currentstroke}{rgb}{0.000000,0.000000,0.000000}%
\pgfsetstrokecolor{currentstroke}%
\pgfsetdash{}{0pt}%
\pgfsys@defobject{currentmarker}{\pgfqpoint{0.000000in}{-0.027778in}}{\pgfqpoint{0.000000in}{0.000000in}}{%
\pgfpathmoveto{\pgfqpoint{0.000000in}{0.000000in}}%
\pgfpathlineto{\pgfqpoint{0.000000in}{-0.027778in}}%
\pgfusepath{stroke,fill}%
}%
\begin{pgfscope}%
\pgfsys@transformshift{2.880139in}{0.440000in}%
\pgfsys@useobject{currentmarker}{}%
\end{pgfscope}%
\end{pgfscope}%
\begin{pgfscope}%
\pgfsetbuttcap%
\pgfsetroundjoin%
\definecolor{currentfill}{rgb}{0.000000,0.000000,0.000000}%
\pgfsetfillcolor{currentfill}%
\pgfsetlinewidth{0.602250pt}%
\definecolor{currentstroke}{rgb}{0.000000,0.000000,0.000000}%
\pgfsetstrokecolor{currentstroke}%
\pgfsetdash{}{0pt}%
\pgfsys@defobject{currentmarker}{\pgfqpoint{0.000000in}{-0.027778in}}{\pgfqpoint{0.000000in}{0.000000in}}{%
\pgfpathmoveto{\pgfqpoint{0.000000in}{0.000000in}}%
\pgfpathlineto{\pgfqpoint{0.000000in}{-0.027778in}}%
\pgfusepath{stroke,fill}%
}%
\begin{pgfscope}%
\pgfsys@transformshift{3.249066in}{0.440000in}%
\pgfsys@useobject{currentmarker}{}%
\end{pgfscope}%
\end{pgfscope}%
\begin{pgfscope}%
\pgfsetbuttcap%
\pgfsetroundjoin%
\definecolor{currentfill}{rgb}{0.000000,0.000000,0.000000}%
\pgfsetfillcolor{currentfill}%
\pgfsetlinewidth{0.602250pt}%
\definecolor{currentstroke}{rgb}{0.000000,0.000000,0.000000}%
\pgfsetstrokecolor{currentstroke}%
\pgfsetdash{}{0pt}%
\pgfsys@defobject{currentmarker}{\pgfqpoint{0.000000in}{-0.027778in}}{\pgfqpoint{0.000000in}{0.000000in}}{%
\pgfpathmoveto{\pgfqpoint{0.000000in}{0.000000in}}%
\pgfpathlineto{\pgfqpoint{0.000000in}{-0.027778in}}%
\pgfusepath{stroke,fill}%
}%
\begin{pgfscope}%
\pgfsys@transformshift{3.436399in}{0.440000in}%
\pgfsys@useobject{currentmarker}{}%
\end{pgfscope}%
\end{pgfscope}%
\begin{pgfscope}%
\pgfsetbuttcap%
\pgfsetroundjoin%
\definecolor{currentfill}{rgb}{0.000000,0.000000,0.000000}%
\pgfsetfillcolor{currentfill}%
\pgfsetlinewidth{0.602250pt}%
\definecolor{currentstroke}{rgb}{0.000000,0.000000,0.000000}%
\pgfsetstrokecolor{currentstroke}%
\pgfsetdash{}{0pt}%
\pgfsys@defobject{currentmarker}{\pgfqpoint{0.000000in}{-0.027778in}}{\pgfqpoint{0.000000in}{0.000000in}}{%
\pgfpathmoveto{\pgfqpoint{0.000000in}{0.000000in}}%
\pgfpathlineto{\pgfqpoint{0.000000in}{-0.027778in}}%
\pgfusepath{stroke,fill}%
}%
\begin{pgfscope}%
\pgfsys@transformshift{3.569313in}{0.440000in}%
\pgfsys@useobject{currentmarker}{}%
\end{pgfscope}%
\end{pgfscope}%
\begin{pgfscope}%
\pgfsetbuttcap%
\pgfsetroundjoin%
\definecolor{currentfill}{rgb}{0.000000,0.000000,0.000000}%
\pgfsetfillcolor{currentfill}%
\pgfsetlinewidth{0.602250pt}%
\definecolor{currentstroke}{rgb}{0.000000,0.000000,0.000000}%
\pgfsetstrokecolor{currentstroke}%
\pgfsetdash{}{0pt}%
\pgfsys@defobject{currentmarker}{\pgfqpoint{0.000000in}{-0.027778in}}{\pgfqpoint{0.000000in}{0.000000in}}{%
\pgfpathmoveto{\pgfqpoint{0.000000in}{0.000000in}}%
\pgfpathlineto{\pgfqpoint{0.000000in}{-0.027778in}}%
\pgfusepath{stroke,fill}%
}%
\begin{pgfscope}%
\pgfsys@transformshift{3.672410in}{0.440000in}%
\pgfsys@useobject{currentmarker}{}%
\end{pgfscope}%
\end{pgfscope}%
\begin{pgfscope}%
\pgfsetbuttcap%
\pgfsetroundjoin%
\definecolor{currentfill}{rgb}{0.000000,0.000000,0.000000}%
\pgfsetfillcolor{currentfill}%
\pgfsetlinewidth{0.602250pt}%
\definecolor{currentstroke}{rgb}{0.000000,0.000000,0.000000}%
\pgfsetstrokecolor{currentstroke}%
\pgfsetdash{}{0pt}%
\pgfsys@defobject{currentmarker}{\pgfqpoint{0.000000in}{-0.027778in}}{\pgfqpoint{0.000000in}{0.000000in}}{%
\pgfpathmoveto{\pgfqpoint{0.000000in}{0.000000in}}%
\pgfpathlineto{\pgfqpoint{0.000000in}{-0.027778in}}%
\pgfusepath{stroke,fill}%
}%
\begin{pgfscope}%
\pgfsys@transformshift{3.756646in}{0.440000in}%
\pgfsys@useobject{currentmarker}{}%
\end{pgfscope}%
\end{pgfscope}%
\begin{pgfscope}%
\pgfsetbuttcap%
\pgfsetroundjoin%
\definecolor{currentfill}{rgb}{0.000000,0.000000,0.000000}%
\pgfsetfillcolor{currentfill}%
\pgfsetlinewidth{0.602250pt}%
\definecolor{currentstroke}{rgb}{0.000000,0.000000,0.000000}%
\pgfsetstrokecolor{currentstroke}%
\pgfsetdash{}{0pt}%
\pgfsys@defobject{currentmarker}{\pgfqpoint{0.000000in}{-0.027778in}}{\pgfqpoint{0.000000in}{0.000000in}}{%
\pgfpathmoveto{\pgfqpoint{0.000000in}{0.000000in}}%
\pgfpathlineto{\pgfqpoint{0.000000in}{-0.027778in}}%
\pgfusepath{stroke,fill}%
}%
\begin{pgfscope}%
\pgfsys@transformshift{3.827867in}{0.440000in}%
\pgfsys@useobject{currentmarker}{}%
\end{pgfscope}%
\end{pgfscope}%
\begin{pgfscope}%
\pgfsetbuttcap%
\pgfsetroundjoin%
\definecolor{currentfill}{rgb}{0.000000,0.000000,0.000000}%
\pgfsetfillcolor{currentfill}%
\pgfsetlinewidth{0.602250pt}%
\definecolor{currentstroke}{rgb}{0.000000,0.000000,0.000000}%
\pgfsetstrokecolor{currentstroke}%
\pgfsetdash{}{0pt}%
\pgfsys@defobject{currentmarker}{\pgfqpoint{0.000000in}{-0.027778in}}{\pgfqpoint{0.000000in}{0.000000in}}{%
\pgfpathmoveto{\pgfqpoint{0.000000in}{0.000000in}}%
\pgfpathlineto{\pgfqpoint{0.000000in}{-0.027778in}}%
\pgfusepath{stroke,fill}%
}%
\begin{pgfscope}%
\pgfsys@transformshift{3.889561in}{0.440000in}%
\pgfsys@useobject{currentmarker}{}%
\end{pgfscope}%
\end{pgfscope}%
\begin{pgfscope}%
\pgfsetbuttcap%
\pgfsetroundjoin%
\definecolor{currentfill}{rgb}{0.000000,0.000000,0.000000}%
\pgfsetfillcolor{currentfill}%
\pgfsetlinewidth{0.602250pt}%
\definecolor{currentstroke}{rgb}{0.000000,0.000000,0.000000}%
\pgfsetstrokecolor{currentstroke}%
\pgfsetdash{}{0pt}%
\pgfsys@defobject{currentmarker}{\pgfqpoint{0.000000in}{-0.027778in}}{\pgfqpoint{0.000000in}{0.000000in}}{%
\pgfpathmoveto{\pgfqpoint{0.000000in}{0.000000in}}%
\pgfpathlineto{\pgfqpoint{0.000000in}{-0.027778in}}%
\pgfusepath{stroke,fill}%
}%
\begin{pgfscope}%
\pgfsys@transformshift{3.943980in}{0.440000in}%
\pgfsys@useobject{currentmarker}{}%
\end{pgfscope}%
\end{pgfscope}%
\begin{pgfscope}%
\pgfsetbuttcap%
\pgfsetroundjoin%
\definecolor{currentfill}{rgb}{0.000000,0.000000,0.000000}%
\pgfsetfillcolor{currentfill}%
\pgfsetlinewidth{0.602250pt}%
\definecolor{currentstroke}{rgb}{0.000000,0.000000,0.000000}%
\pgfsetstrokecolor{currentstroke}%
\pgfsetdash{}{0pt}%
\pgfsys@defobject{currentmarker}{\pgfqpoint{0.000000in}{-0.027778in}}{\pgfqpoint{0.000000in}{0.000000in}}{%
\pgfpathmoveto{\pgfqpoint{0.000000in}{0.000000in}}%
\pgfpathlineto{\pgfqpoint{0.000000in}{-0.027778in}}%
\pgfusepath{stroke,fill}%
}%
\begin{pgfscope}%
\pgfsys@transformshift{4.312906in}{0.440000in}%
\pgfsys@useobject{currentmarker}{}%
\end{pgfscope}%
\end{pgfscope}%
\begin{pgfscope}%
\pgfsetbuttcap%
\pgfsetroundjoin%
\definecolor{currentfill}{rgb}{0.000000,0.000000,0.000000}%
\pgfsetfillcolor{currentfill}%
\pgfsetlinewidth{0.602250pt}%
\definecolor{currentstroke}{rgb}{0.000000,0.000000,0.000000}%
\pgfsetstrokecolor{currentstroke}%
\pgfsetdash{}{0pt}%
\pgfsys@defobject{currentmarker}{\pgfqpoint{0.000000in}{-0.027778in}}{\pgfqpoint{0.000000in}{0.000000in}}{%
\pgfpathmoveto{\pgfqpoint{0.000000in}{0.000000in}}%
\pgfpathlineto{\pgfqpoint{0.000000in}{-0.027778in}}%
\pgfusepath{stroke,fill}%
}%
\begin{pgfscope}%
\pgfsys@transformshift{4.500239in}{0.440000in}%
\pgfsys@useobject{currentmarker}{}%
\end{pgfscope}%
\end{pgfscope}%
\begin{pgfscope}%
\definecolor{textcolor}{rgb}{0.000000,0.000000,0.000000}%
\pgfsetstrokecolor{textcolor}%
\pgfsetfillcolor{textcolor}%
\pgftext[x=2.562500in,y=0.152809in,,top]{\color{textcolor}\sffamily\fontsize{10.000000}{12.000000}\selectfont \(\displaystyle N_\mathrm{QMC}\)}%
\end{pgfscope}%
\begin{pgfscope}%
\pgfsetbuttcap%
\pgfsetroundjoin%
\definecolor{currentfill}{rgb}{0.000000,0.000000,0.000000}%
\pgfsetfillcolor{currentfill}%
\pgfsetlinewidth{0.803000pt}%
\definecolor{currentstroke}{rgb}{0.000000,0.000000,0.000000}%
\pgfsetstrokecolor{currentstroke}%
\pgfsetdash{}{0pt}%
\pgfsys@defobject{currentmarker}{\pgfqpoint{-0.048611in}{0.000000in}}{\pgfqpoint{0.000000in}{0.000000in}}{%
\pgfpathmoveto{\pgfqpoint{0.000000in}{0.000000in}}%
\pgfpathlineto{\pgfqpoint{-0.048611in}{0.000000in}}%
\pgfusepath{stroke,fill}%
}%
\begin{pgfscope}%
\pgfsys@transformshift{0.625000in}{1.372026in}%
\pgfsys@useobject{currentmarker}{}%
\end{pgfscope}%
\end{pgfscope}%
\begin{pgfscope}%
\definecolor{textcolor}{rgb}{0.000000,0.000000,0.000000}%
\pgfsetstrokecolor{textcolor}%
\pgfsetfillcolor{textcolor}%
\pgftext[x=0.239775in,y=1.319264in,left,base]{\color{textcolor}\sffamily\fontsize{10.000000}{12.000000}\selectfont \(\displaystyle 10^{-1}\)}%
\end{pgfscope}%
\begin{pgfscope}%
\pgfsetbuttcap%
\pgfsetroundjoin%
\definecolor{currentfill}{rgb}{0.000000,0.000000,0.000000}%
\pgfsetfillcolor{currentfill}%
\pgfsetlinewidth{0.803000pt}%
\definecolor{currentstroke}{rgb}{0.000000,0.000000,0.000000}%
\pgfsetstrokecolor{currentstroke}%
\pgfsetdash{}{0pt}%
\pgfsys@defobject{currentmarker}{\pgfqpoint{-0.048611in}{0.000000in}}{\pgfqpoint{0.000000in}{0.000000in}}{%
\pgfpathmoveto{\pgfqpoint{0.000000in}{0.000000in}}%
\pgfpathlineto{\pgfqpoint{-0.048611in}{0.000000in}}%
\pgfusepath{stroke,fill}%
}%
\begin{pgfscope}%
\pgfsys@transformshift{0.625000in}{2.479879in}%
\pgfsys@useobject{currentmarker}{}%
\end{pgfscope}%
\end{pgfscope}%
\begin{pgfscope}%
\definecolor{textcolor}{rgb}{0.000000,0.000000,0.000000}%
\pgfsetstrokecolor{textcolor}%
\pgfsetfillcolor{textcolor}%
\pgftext[x=0.326581in,y=2.427118in,left,base]{\color{textcolor}\sffamily\fontsize{10.000000}{12.000000}\selectfont \(\displaystyle 10^{0}\)}%
\end{pgfscope}%
\begin{pgfscope}%
\pgfsetbuttcap%
\pgfsetroundjoin%
\definecolor{currentfill}{rgb}{0.000000,0.000000,0.000000}%
\pgfsetfillcolor{currentfill}%
\pgfsetlinewidth{0.602250pt}%
\definecolor{currentstroke}{rgb}{0.000000,0.000000,0.000000}%
\pgfsetstrokecolor{currentstroke}%
\pgfsetdash{}{0pt}%
\pgfsys@defobject{currentmarker}{\pgfqpoint{-0.027778in}{0.000000in}}{\pgfqpoint{0.000000in}{0.000000in}}{%
\pgfpathmoveto{\pgfqpoint{0.000000in}{0.000000in}}%
\pgfpathlineto{\pgfqpoint{-0.027778in}{0.000000in}}%
\pgfusepath{stroke,fill}%
}%
\begin{pgfscope}%
\pgfsys@transformshift{0.625000in}{0.597670in}%
\pgfsys@useobject{currentmarker}{}%
\end{pgfscope}%
\end{pgfscope}%
\begin{pgfscope}%
\pgfsetbuttcap%
\pgfsetroundjoin%
\definecolor{currentfill}{rgb}{0.000000,0.000000,0.000000}%
\pgfsetfillcolor{currentfill}%
\pgfsetlinewidth{0.602250pt}%
\definecolor{currentstroke}{rgb}{0.000000,0.000000,0.000000}%
\pgfsetstrokecolor{currentstroke}%
\pgfsetdash{}{0pt}%
\pgfsys@defobject{currentmarker}{\pgfqpoint{-0.027778in}{0.000000in}}{\pgfqpoint{0.000000in}{0.000000in}}{%
\pgfpathmoveto{\pgfqpoint{0.000000in}{0.000000in}}%
\pgfpathlineto{\pgfqpoint{-0.027778in}{0.000000in}}%
\pgfusepath{stroke,fill}%
}%
\begin{pgfscope}%
\pgfsys@transformshift{0.625000in}{0.792753in}%
\pgfsys@useobject{currentmarker}{}%
\end{pgfscope}%
\end{pgfscope}%
\begin{pgfscope}%
\pgfsetbuttcap%
\pgfsetroundjoin%
\definecolor{currentfill}{rgb}{0.000000,0.000000,0.000000}%
\pgfsetfillcolor{currentfill}%
\pgfsetlinewidth{0.602250pt}%
\definecolor{currentstroke}{rgb}{0.000000,0.000000,0.000000}%
\pgfsetstrokecolor{currentstroke}%
\pgfsetdash{}{0pt}%
\pgfsys@defobject{currentmarker}{\pgfqpoint{-0.027778in}{0.000000in}}{\pgfqpoint{0.000000in}{0.000000in}}{%
\pgfpathmoveto{\pgfqpoint{0.000000in}{0.000000in}}%
\pgfpathlineto{\pgfqpoint{-0.027778in}{0.000000in}}%
\pgfusepath{stroke,fill}%
}%
\begin{pgfscope}%
\pgfsys@transformshift{0.625000in}{0.931167in}%
\pgfsys@useobject{currentmarker}{}%
\end{pgfscope}%
\end{pgfscope}%
\begin{pgfscope}%
\pgfsetbuttcap%
\pgfsetroundjoin%
\definecolor{currentfill}{rgb}{0.000000,0.000000,0.000000}%
\pgfsetfillcolor{currentfill}%
\pgfsetlinewidth{0.602250pt}%
\definecolor{currentstroke}{rgb}{0.000000,0.000000,0.000000}%
\pgfsetstrokecolor{currentstroke}%
\pgfsetdash{}{0pt}%
\pgfsys@defobject{currentmarker}{\pgfqpoint{-0.027778in}{0.000000in}}{\pgfqpoint{0.000000in}{0.000000in}}{%
\pgfpathmoveto{\pgfqpoint{0.000000in}{0.000000in}}%
\pgfpathlineto{\pgfqpoint{-0.027778in}{0.000000in}}%
\pgfusepath{stroke,fill}%
}%
\begin{pgfscope}%
\pgfsys@transformshift{0.625000in}{1.038529in}%
\pgfsys@useobject{currentmarker}{}%
\end{pgfscope}%
\end{pgfscope}%
\begin{pgfscope}%
\pgfsetbuttcap%
\pgfsetroundjoin%
\definecolor{currentfill}{rgb}{0.000000,0.000000,0.000000}%
\pgfsetfillcolor{currentfill}%
\pgfsetlinewidth{0.602250pt}%
\definecolor{currentstroke}{rgb}{0.000000,0.000000,0.000000}%
\pgfsetstrokecolor{currentstroke}%
\pgfsetdash{}{0pt}%
\pgfsys@defobject{currentmarker}{\pgfqpoint{-0.027778in}{0.000000in}}{\pgfqpoint{0.000000in}{0.000000in}}{%
\pgfpathmoveto{\pgfqpoint{0.000000in}{0.000000in}}%
\pgfpathlineto{\pgfqpoint{-0.027778in}{0.000000in}}%
\pgfusepath{stroke,fill}%
}%
\begin{pgfscope}%
\pgfsys@transformshift{0.625000in}{1.126250in}%
\pgfsys@useobject{currentmarker}{}%
\end{pgfscope}%
\end{pgfscope}%
\begin{pgfscope}%
\pgfsetbuttcap%
\pgfsetroundjoin%
\definecolor{currentfill}{rgb}{0.000000,0.000000,0.000000}%
\pgfsetfillcolor{currentfill}%
\pgfsetlinewidth{0.602250pt}%
\definecolor{currentstroke}{rgb}{0.000000,0.000000,0.000000}%
\pgfsetstrokecolor{currentstroke}%
\pgfsetdash{}{0pt}%
\pgfsys@defobject{currentmarker}{\pgfqpoint{-0.027778in}{0.000000in}}{\pgfqpoint{0.000000in}{0.000000in}}{%
\pgfpathmoveto{\pgfqpoint{0.000000in}{0.000000in}}%
\pgfpathlineto{\pgfqpoint{-0.027778in}{0.000000in}}%
\pgfusepath{stroke,fill}%
}%
\begin{pgfscope}%
\pgfsys@transformshift{0.625000in}{1.200417in}%
\pgfsys@useobject{currentmarker}{}%
\end{pgfscope}%
\end{pgfscope}%
\begin{pgfscope}%
\pgfsetbuttcap%
\pgfsetroundjoin%
\definecolor{currentfill}{rgb}{0.000000,0.000000,0.000000}%
\pgfsetfillcolor{currentfill}%
\pgfsetlinewidth{0.602250pt}%
\definecolor{currentstroke}{rgb}{0.000000,0.000000,0.000000}%
\pgfsetstrokecolor{currentstroke}%
\pgfsetdash{}{0pt}%
\pgfsys@defobject{currentmarker}{\pgfqpoint{-0.027778in}{0.000000in}}{\pgfqpoint{0.000000in}{0.000000in}}{%
\pgfpathmoveto{\pgfqpoint{0.000000in}{0.000000in}}%
\pgfpathlineto{\pgfqpoint{-0.027778in}{0.000000in}}%
\pgfusepath{stroke,fill}%
}%
\begin{pgfscope}%
\pgfsys@transformshift{0.625000in}{1.264664in}%
\pgfsys@useobject{currentmarker}{}%
\end{pgfscope}%
\end{pgfscope}%
\begin{pgfscope}%
\pgfsetbuttcap%
\pgfsetroundjoin%
\definecolor{currentfill}{rgb}{0.000000,0.000000,0.000000}%
\pgfsetfillcolor{currentfill}%
\pgfsetlinewidth{0.602250pt}%
\definecolor{currentstroke}{rgb}{0.000000,0.000000,0.000000}%
\pgfsetstrokecolor{currentstroke}%
\pgfsetdash{}{0pt}%
\pgfsys@defobject{currentmarker}{\pgfqpoint{-0.027778in}{0.000000in}}{\pgfqpoint{0.000000in}{0.000000in}}{%
\pgfpathmoveto{\pgfqpoint{0.000000in}{0.000000in}}%
\pgfpathlineto{\pgfqpoint{-0.027778in}{0.000000in}}%
\pgfusepath{stroke,fill}%
}%
\begin{pgfscope}%
\pgfsys@transformshift{0.625000in}{1.321333in}%
\pgfsys@useobject{currentmarker}{}%
\end{pgfscope}%
\end{pgfscope}%
\begin{pgfscope}%
\pgfsetbuttcap%
\pgfsetroundjoin%
\definecolor{currentfill}{rgb}{0.000000,0.000000,0.000000}%
\pgfsetfillcolor{currentfill}%
\pgfsetlinewidth{0.602250pt}%
\definecolor{currentstroke}{rgb}{0.000000,0.000000,0.000000}%
\pgfsetstrokecolor{currentstroke}%
\pgfsetdash{}{0pt}%
\pgfsys@defobject{currentmarker}{\pgfqpoint{-0.027778in}{0.000000in}}{\pgfqpoint{0.000000in}{0.000000in}}{%
\pgfpathmoveto{\pgfqpoint{0.000000in}{0.000000in}}%
\pgfpathlineto{\pgfqpoint{-0.027778in}{0.000000in}}%
\pgfusepath{stroke,fill}%
}%
\begin{pgfscope}%
\pgfsys@transformshift{0.625000in}{1.705523in}%
\pgfsys@useobject{currentmarker}{}%
\end{pgfscope}%
\end{pgfscope}%
\begin{pgfscope}%
\pgfsetbuttcap%
\pgfsetroundjoin%
\definecolor{currentfill}{rgb}{0.000000,0.000000,0.000000}%
\pgfsetfillcolor{currentfill}%
\pgfsetlinewidth{0.602250pt}%
\definecolor{currentstroke}{rgb}{0.000000,0.000000,0.000000}%
\pgfsetstrokecolor{currentstroke}%
\pgfsetdash{}{0pt}%
\pgfsys@defobject{currentmarker}{\pgfqpoint{-0.027778in}{0.000000in}}{\pgfqpoint{0.000000in}{0.000000in}}{%
\pgfpathmoveto{\pgfqpoint{0.000000in}{0.000000in}}%
\pgfpathlineto{\pgfqpoint{-0.027778in}{0.000000in}}%
\pgfusepath{stroke,fill}%
}%
\begin{pgfscope}%
\pgfsys@transformshift{0.625000in}{1.900606in}%
\pgfsys@useobject{currentmarker}{}%
\end{pgfscope}%
\end{pgfscope}%
\begin{pgfscope}%
\pgfsetbuttcap%
\pgfsetroundjoin%
\definecolor{currentfill}{rgb}{0.000000,0.000000,0.000000}%
\pgfsetfillcolor{currentfill}%
\pgfsetlinewidth{0.602250pt}%
\definecolor{currentstroke}{rgb}{0.000000,0.000000,0.000000}%
\pgfsetstrokecolor{currentstroke}%
\pgfsetdash{}{0pt}%
\pgfsys@defobject{currentmarker}{\pgfqpoint{-0.027778in}{0.000000in}}{\pgfqpoint{0.000000in}{0.000000in}}{%
\pgfpathmoveto{\pgfqpoint{0.000000in}{0.000000in}}%
\pgfpathlineto{\pgfqpoint{-0.027778in}{0.000000in}}%
\pgfusepath{stroke,fill}%
}%
\begin{pgfscope}%
\pgfsys@transformshift{0.625000in}{2.039020in}%
\pgfsys@useobject{currentmarker}{}%
\end{pgfscope}%
\end{pgfscope}%
\begin{pgfscope}%
\pgfsetbuttcap%
\pgfsetroundjoin%
\definecolor{currentfill}{rgb}{0.000000,0.000000,0.000000}%
\pgfsetfillcolor{currentfill}%
\pgfsetlinewidth{0.602250pt}%
\definecolor{currentstroke}{rgb}{0.000000,0.000000,0.000000}%
\pgfsetstrokecolor{currentstroke}%
\pgfsetdash{}{0pt}%
\pgfsys@defobject{currentmarker}{\pgfqpoint{-0.027778in}{0.000000in}}{\pgfqpoint{0.000000in}{0.000000in}}{%
\pgfpathmoveto{\pgfqpoint{0.000000in}{0.000000in}}%
\pgfpathlineto{\pgfqpoint{-0.027778in}{0.000000in}}%
\pgfusepath{stroke,fill}%
}%
\begin{pgfscope}%
\pgfsys@transformshift{0.625000in}{2.146382in}%
\pgfsys@useobject{currentmarker}{}%
\end{pgfscope}%
\end{pgfscope}%
\begin{pgfscope}%
\pgfsetbuttcap%
\pgfsetroundjoin%
\definecolor{currentfill}{rgb}{0.000000,0.000000,0.000000}%
\pgfsetfillcolor{currentfill}%
\pgfsetlinewidth{0.602250pt}%
\definecolor{currentstroke}{rgb}{0.000000,0.000000,0.000000}%
\pgfsetstrokecolor{currentstroke}%
\pgfsetdash{}{0pt}%
\pgfsys@defobject{currentmarker}{\pgfqpoint{-0.027778in}{0.000000in}}{\pgfqpoint{0.000000in}{0.000000in}}{%
\pgfpathmoveto{\pgfqpoint{0.000000in}{0.000000in}}%
\pgfpathlineto{\pgfqpoint{-0.027778in}{0.000000in}}%
\pgfusepath{stroke,fill}%
}%
\begin{pgfscope}%
\pgfsys@transformshift{0.625000in}{2.234103in}%
\pgfsys@useobject{currentmarker}{}%
\end{pgfscope}%
\end{pgfscope}%
\begin{pgfscope}%
\pgfsetbuttcap%
\pgfsetroundjoin%
\definecolor{currentfill}{rgb}{0.000000,0.000000,0.000000}%
\pgfsetfillcolor{currentfill}%
\pgfsetlinewidth{0.602250pt}%
\definecolor{currentstroke}{rgb}{0.000000,0.000000,0.000000}%
\pgfsetstrokecolor{currentstroke}%
\pgfsetdash{}{0pt}%
\pgfsys@defobject{currentmarker}{\pgfqpoint{-0.027778in}{0.000000in}}{\pgfqpoint{0.000000in}{0.000000in}}{%
\pgfpathmoveto{\pgfqpoint{0.000000in}{0.000000in}}%
\pgfpathlineto{\pgfqpoint{-0.027778in}{0.000000in}}%
\pgfusepath{stroke,fill}%
}%
\begin{pgfscope}%
\pgfsys@transformshift{0.625000in}{2.308271in}%
\pgfsys@useobject{currentmarker}{}%
\end{pgfscope}%
\end{pgfscope}%
\begin{pgfscope}%
\pgfsetbuttcap%
\pgfsetroundjoin%
\definecolor{currentfill}{rgb}{0.000000,0.000000,0.000000}%
\pgfsetfillcolor{currentfill}%
\pgfsetlinewidth{0.602250pt}%
\definecolor{currentstroke}{rgb}{0.000000,0.000000,0.000000}%
\pgfsetstrokecolor{currentstroke}%
\pgfsetdash{}{0pt}%
\pgfsys@defobject{currentmarker}{\pgfqpoint{-0.027778in}{0.000000in}}{\pgfqpoint{0.000000in}{0.000000in}}{%
\pgfpathmoveto{\pgfqpoint{0.000000in}{0.000000in}}%
\pgfpathlineto{\pgfqpoint{-0.027778in}{0.000000in}}%
\pgfusepath{stroke,fill}%
}%
\begin{pgfscope}%
\pgfsys@transformshift{0.625000in}{2.372517in}%
\pgfsys@useobject{currentmarker}{}%
\end{pgfscope}%
\end{pgfscope}%
\begin{pgfscope}%
\pgfsetbuttcap%
\pgfsetroundjoin%
\definecolor{currentfill}{rgb}{0.000000,0.000000,0.000000}%
\pgfsetfillcolor{currentfill}%
\pgfsetlinewidth{0.602250pt}%
\definecolor{currentstroke}{rgb}{0.000000,0.000000,0.000000}%
\pgfsetstrokecolor{currentstroke}%
\pgfsetdash{}{0pt}%
\pgfsys@defobject{currentmarker}{\pgfqpoint{-0.027778in}{0.000000in}}{\pgfqpoint{0.000000in}{0.000000in}}{%
\pgfpathmoveto{\pgfqpoint{0.000000in}{0.000000in}}%
\pgfpathlineto{\pgfqpoint{-0.027778in}{0.000000in}}%
\pgfusepath{stroke,fill}%
}%
\begin{pgfscope}%
\pgfsys@transformshift{0.625000in}{2.429187in}%
\pgfsys@useobject{currentmarker}{}%
\end{pgfscope}%
\end{pgfscope}%
\begin{pgfscope}%
\pgfsetbuttcap%
\pgfsetroundjoin%
\definecolor{currentfill}{rgb}{0.000000,0.000000,0.000000}%
\pgfsetfillcolor{currentfill}%
\pgfsetlinewidth{0.602250pt}%
\definecolor{currentstroke}{rgb}{0.000000,0.000000,0.000000}%
\pgfsetstrokecolor{currentstroke}%
\pgfsetdash{}{0pt}%
\pgfsys@defobject{currentmarker}{\pgfqpoint{-0.027778in}{0.000000in}}{\pgfqpoint{0.000000in}{0.000000in}}{%
\pgfpathmoveto{\pgfqpoint{0.000000in}{0.000000in}}%
\pgfpathlineto{\pgfqpoint{-0.027778in}{0.000000in}}%
\pgfusepath{stroke,fill}%
}%
\begin{pgfscope}%
\pgfsys@transformshift{0.625000in}{2.813376in}%
\pgfsys@useobject{currentmarker}{}%
\end{pgfscope}%
\end{pgfscope}%
\begin{pgfscope}%
\pgfsetbuttcap%
\pgfsetroundjoin%
\definecolor{currentfill}{rgb}{0.000000,0.000000,0.000000}%
\pgfsetfillcolor{currentfill}%
\pgfsetlinewidth{0.602250pt}%
\definecolor{currentstroke}{rgb}{0.000000,0.000000,0.000000}%
\pgfsetstrokecolor{currentstroke}%
\pgfsetdash{}{0pt}%
\pgfsys@defobject{currentmarker}{\pgfqpoint{-0.027778in}{0.000000in}}{\pgfqpoint{0.000000in}{0.000000in}}{%
\pgfpathmoveto{\pgfqpoint{0.000000in}{0.000000in}}%
\pgfpathlineto{\pgfqpoint{-0.027778in}{0.000000in}}%
\pgfusepath{stroke,fill}%
}%
\begin{pgfscope}%
\pgfsys@transformshift{0.625000in}{3.008460in}%
\pgfsys@useobject{currentmarker}{}%
\end{pgfscope}%
\end{pgfscope}%
\begin{pgfscope}%
\pgfsetbuttcap%
\pgfsetroundjoin%
\definecolor{currentfill}{rgb}{0.000000,0.000000,0.000000}%
\pgfsetfillcolor{currentfill}%
\pgfsetlinewidth{0.602250pt}%
\definecolor{currentstroke}{rgb}{0.000000,0.000000,0.000000}%
\pgfsetstrokecolor{currentstroke}%
\pgfsetdash{}{0pt}%
\pgfsys@defobject{currentmarker}{\pgfqpoint{-0.027778in}{0.000000in}}{\pgfqpoint{0.000000in}{0.000000in}}{%
\pgfpathmoveto{\pgfqpoint{0.000000in}{0.000000in}}%
\pgfpathlineto{\pgfqpoint{-0.027778in}{0.000000in}}%
\pgfusepath{stroke,fill}%
}%
\begin{pgfscope}%
\pgfsys@transformshift{0.625000in}{3.146874in}%
\pgfsys@useobject{currentmarker}{}%
\end{pgfscope}%
\end{pgfscope}%
\begin{pgfscope}%
\pgfsetbuttcap%
\pgfsetroundjoin%
\definecolor{currentfill}{rgb}{0.000000,0.000000,0.000000}%
\pgfsetfillcolor{currentfill}%
\pgfsetlinewidth{0.602250pt}%
\definecolor{currentstroke}{rgb}{0.000000,0.000000,0.000000}%
\pgfsetstrokecolor{currentstroke}%
\pgfsetdash{}{0pt}%
\pgfsys@defobject{currentmarker}{\pgfqpoint{-0.027778in}{0.000000in}}{\pgfqpoint{0.000000in}{0.000000in}}{%
\pgfpathmoveto{\pgfqpoint{0.000000in}{0.000000in}}%
\pgfpathlineto{\pgfqpoint{-0.027778in}{0.000000in}}%
\pgfusepath{stroke,fill}%
}%
\begin{pgfscope}%
\pgfsys@transformshift{0.625000in}{3.254236in}%
\pgfsys@useobject{currentmarker}{}%
\end{pgfscope}%
\end{pgfscope}%
\begin{pgfscope}%
\pgfsetbuttcap%
\pgfsetroundjoin%
\definecolor{currentfill}{rgb}{0.000000,0.000000,0.000000}%
\pgfsetfillcolor{currentfill}%
\pgfsetlinewidth{0.602250pt}%
\definecolor{currentstroke}{rgb}{0.000000,0.000000,0.000000}%
\pgfsetstrokecolor{currentstroke}%
\pgfsetdash{}{0pt}%
\pgfsys@defobject{currentmarker}{\pgfqpoint{-0.027778in}{0.000000in}}{\pgfqpoint{0.000000in}{0.000000in}}{%
\pgfpathmoveto{\pgfqpoint{0.000000in}{0.000000in}}%
\pgfpathlineto{\pgfqpoint{-0.027778in}{0.000000in}}%
\pgfusepath{stroke,fill}%
}%
\begin{pgfscope}%
\pgfsys@transformshift{0.625000in}{3.341957in}%
\pgfsys@useobject{currentmarker}{}%
\end{pgfscope}%
\end{pgfscope}%
\begin{pgfscope}%
\pgfsetbuttcap%
\pgfsetroundjoin%
\definecolor{currentfill}{rgb}{0.000000,0.000000,0.000000}%
\pgfsetfillcolor{currentfill}%
\pgfsetlinewidth{0.602250pt}%
\definecolor{currentstroke}{rgb}{0.000000,0.000000,0.000000}%
\pgfsetstrokecolor{currentstroke}%
\pgfsetdash{}{0pt}%
\pgfsys@defobject{currentmarker}{\pgfqpoint{-0.027778in}{0.000000in}}{\pgfqpoint{0.000000in}{0.000000in}}{%
\pgfpathmoveto{\pgfqpoint{0.000000in}{0.000000in}}%
\pgfpathlineto{\pgfqpoint{-0.027778in}{0.000000in}}%
\pgfusepath{stroke,fill}%
}%
\begin{pgfscope}%
\pgfsys@transformshift{0.625000in}{3.416124in}%
\pgfsys@useobject{currentmarker}{}%
\end{pgfscope}%
\end{pgfscope}%
\begin{pgfscope}%
\pgfsetbuttcap%
\pgfsetroundjoin%
\definecolor{currentfill}{rgb}{0.000000,0.000000,0.000000}%
\pgfsetfillcolor{currentfill}%
\pgfsetlinewidth{0.602250pt}%
\definecolor{currentstroke}{rgb}{0.000000,0.000000,0.000000}%
\pgfsetstrokecolor{currentstroke}%
\pgfsetdash{}{0pt}%
\pgfsys@defobject{currentmarker}{\pgfqpoint{-0.027778in}{0.000000in}}{\pgfqpoint{0.000000in}{0.000000in}}{%
\pgfpathmoveto{\pgfqpoint{0.000000in}{0.000000in}}%
\pgfpathlineto{\pgfqpoint{-0.027778in}{0.000000in}}%
\pgfusepath{stroke,fill}%
}%
\begin{pgfscope}%
\pgfsys@transformshift{0.625000in}{3.480371in}%
\pgfsys@useobject{currentmarker}{}%
\end{pgfscope}%
\end{pgfscope}%
\begin{pgfscope}%
\definecolor{textcolor}{rgb}{0.000000,0.000000,0.000000}%
\pgfsetstrokecolor{textcolor}%
\pgfsetfillcolor{textcolor}%
\pgftext[x=0.184220in,y=1.980000in,,bottom,rotate=90.000000]{\color{textcolor}\sffamily\fontsize{10.000000}{12.000000}\selectfont QMC Error}%
\end{pgfscope}%
\begin{pgfscope}%
\pgfpathrectangle{\pgfqpoint{0.625000in}{0.440000in}}{\pgfqpoint{3.875000in}{3.080000in}}%
\pgfusepath{clip}%
\pgfsetbuttcap%
\pgfsetroundjoin%
\definecolor{currentfill}{rgb}{0.000000,0.000000,0.000000}%
\pgfsetfillcolor{currentfill}%
\pgfsetlinewidth{1.003750pt}%
\definecolor{currentstroke}{rgb}{0.000000,0.000000,0.000000}%
\pgfsetstrokecolor{currentstroke}%
\pgfsetdash{}{0pt}%
\pgfsys@defobject{currentmarker}{\pgfqpoint{-0.041667in}{-0.041667in}}{\pgfqpoint{0.041667in}{0.041667in}}{%
\pgfpathmoveto{\pgfqpoint{0.000000in}{-0.041667in}}%
\pgfpathcurveto{\pgfqpoint{0.011050in}{-0.041667in}}{\pgfqpoint{0.021649in}{-0.037276in}}{\pgfqpoint{0.029463in}{-0.029463in}}%
\pgfpathcurveto{\pgfqpoint{0.037276in}{-0.021649in}}{\pgfqpoint{0.041667in}{-0.011050in}}{\pgfqpoint{0.041667in}{0.000000in}}%
\pgfpathcurveto{\pgfqpoint{0.041667in}{0.011050in}}{\pgfqpoint{0.037276in}{0.021649in}}{\pgfqpoint{0.029463in}{0.029463in}}%
\pgfpathcurveto{\pgfqpoint{0.021649in}{0.037276in}}{\pgfqpoint{0.011050in}{0.041667in}}{\pgfqpoint{0.000000in}{0.041667in}}%
\pgfpathcurveto{\pgfqpoint{-0.011050in}{0.041667in}}{\pgfqpoint{-0.021649in}{0.037276in}}{\pgfqpoint{-0.029463in}{0.029463in}}%
\pgfpathcurveto{\pgfqpoint{-0.037276in}{0.021649in}}{\pgfqpoint{-0.041667in}{0.011050in}}{\pgfqpoint{-0.041667in}{0.000000in}}%
\pgfpathcurveto{\pgfqpoint{-0.041667in}{-0.011050in}}{\pgfqpoint{-0.037276in}{-0.021649in}}{\pgfqpoint{-0.029463in}{-0.029463in}}%
\pgfpathcurveto{\pgfqpoint{-0.021649in}{-0.037276in}}{\pgfqpoint{-0.011050in}{-0.041667in}}{\pgfqpoint{0.000000in}{-0.041667in}}%
\pgfpathclose%
\pgfusepath{stroke,fill}%
}%
\begin{pgfscope}%
\pgfsys@transformshift{0.801136in}{3.380000in}%
\pgfsys@useobject{currentmarker}{}%
\end{pgfscope}%
\begin{pgfscope}%
\pgfsys@transformshift{1.121384in}{3.251035in}%
\pgfsys@useobject{currentmarker}{}%
\end{pgfscope}%
\begin{pgfscope}%
\pgfsys@transformshift{1.441632in}{2.693429in}%
\pgfsys@useobject{currentmarker}{}%
\end{pgfscope}%
\begin{pgfscope}%
\pgfsys@transformshift{1.761880in}{2.565332in}%
\pgfsys@useobject{currentmarker}{}%
\end{pgfscope}%
\begin{pgfscope}%
\pgfsys@transformshift{2.082128in}{2.292991in}%
\pgfsys@useobject{currentmarker}{}%
\end{pgfscope}%
\begin{pgfscope}%
\pgfsys@transformshift{2.402376in}{2.056129in}%
\pgfsys@useobject{currentmarker}{}%
\end{pgfscope}%
\begin{pgfscope}%
\pgfsys@transformshift{2.722624in}{2.014465in}%
\pgfsys@useobject{currentmarker}{}%
\end{pgfscope}%
\begin{pgfscope}%
\pgfsys@transformshift{3.042872in}{1.963158in}%
\pgfsys@useobject{currentmarker}{}%
\end{pgfscope}%
\begin{pgfscope}%
\pgfsys@transformshift{3.363120in}{1.315901in}%
\pgfsys@useobject{currentmarker}{}%
\end{pgfscope}%
\begin{pgfscope}%
\pgfsys@transformshift{3.683368in}{1.009466in}%
\pgfsys@useobject{currentmarker}{}%
\end{pgfscope}%
\begin{pgfscope}%
\pgfsys@transformshift{4.003616in}{0.812725in}%
\pgfsys@useobject{currentmarker}{}%
\end{pgfscope}%
\begin{pgfscope}%
\pgfsys@transformshift{4.323864in}{0.580000in}%
\pgfsys@useobject{currentmarker}{}%
\end{pgfscope}%
\end{pgfscope}%
\begin{pgfscope}%
\pgfpathrectangle{\pgfqpoint{0.625000in}{0.440000in}}{\pgfqpoint{3.875000in}{3.080000in}}%
\pgfusepath{clip}%
\pgfsetbuttcap%
\pgfsetroundjoin%
\pgfsetlinewidth{1.505625pt}%
\definecolor{currentstroke}{rgb}{0.000000,0.000000,0.000000}%
\pgfsetstrokecolor{currentstroke}%
\pgfsetdash{{5.550000pt}{2.400000pt}}{0.000000pt}%
\pgfpathmoveto{\pgfqpoint{0.801136in}{3.375531in}}%
\pgfpathlineto{\pgfqpoint{4.323864in}{0.613575in}}%
\pgfusepath{stroke}%
\end{pgfscope}%
\begin{pgfscope}%
\pgfsetrectcap%
\pgfsetmiterjoin%
\pgfsetlinewidth{0.803000pt}%
\definecolor{currentstroke}{rgb}{0.000000,0.000000,0.000000}%
\pgfsetstrokecolor{currentstroke}%
\pgfsetdash{}{0pt}%
\pgfpathmoveto{\pgfqpoint{0.625000in}{0.440000in}}%
\pgfpathlineto{\pgfqpoint{0.625000in}{3.520000in}}%
\pgfusepath{stroke}%
\end{pgfscope}%
\begin{pgfscope}%
\pgfsetrectcap%
\pgfsetmiterjoin%
\pgfsetlinewidth{0.803000pt}%
\definecolor{currentstroke}{rgb}{0.000000,0.000000,0.000000}%
\pgfsetstrokecolor{currentstroke}%
\pgfsetdash{}{0pt}%
\pgfpathmoveto{\pgfqpoint{4.500000in}{0.440000in}}%
\pgfpathlineto{\pgfqpoint{4.500000in}{3.520000in}}%
\pgfusepath{stroke}%
\end{pgfscope}%
\begin{pgfscope}%
\pgfsetrectcap%
\pgfsetmiterjoin%
\pgfsetlinewidth{0.803000pt}%
\definecolor{currentstroke}{rgb}{0.000000,0.000000,0.000000}%
\pgfsetstrokecolor{currentstroke}%
\pgfsetdash{}{0pt}%
\pgfpathmoveto{\pgfqpoint{0.625000in}{0.440000in}}%
\pgfpathlineto{\pgfqpoint{4.500000in}{0.440000in}}%
\pgfusepath{stroke}%
\end{pgfscope}%
\begin{pgfscope}%
\pgfsetrectcap%
\pgfsetmiterjoin%
\pgfsetlinewidth{0.803000pt}%
\definecolor{currentstroke}{rgb}{0.000000,0.000000,0.000000}%
\pgfsetstrokecolor{currentstroke}%
\pgfsetdash{}{0pt}%
\pgfpathmoveto{\pgfqpoint{0.625000in}{3.520000in}}%
\pgfpathlineto{\pgfqpoint{4.500000in}{3.520000in}}%
\pgfusepath{stroke}%
\end{pgfscope}%
\begin{pgfscope}%
\pgfsetbuttcap%
\pgfsetmiterjoin%
\definecolor{currentfill}{rgb}{1.000000,1.000000,1.000000}%
\pgfsetfillcolor{currentfill}%
\pgfsetfillopacity{0.800000}%
\pgfsetlinewidth{1.003750pt}%
\definecolor{currentstroke}{rgb}{0.800000,0.800000,0.800000}%
\pgfsetstrokecolor{currentstroke}%
\pgfsetstrokeopacity{0.800000}%
\pgfsetdash{}{0pt}%
\pgfpathmoveto{\pgfqpoint{3.042243in}{3.155267in}}%
\pgfpathlineto{\pgfqpoint{4.402778in}{3.155267in}}%
\pgfpathquadraticcurveto{\pgfqpoint{4.430556in}{3.155267in}}{\pgfqpoint{4.430556in}{3.183045in}}%
\pgfpathlineto{\pgfqpoint{4.430556in}{3.422778in}}%
\pgfpathquadraticcurveto{\pgfqpoint{4.430556in}{3.450556in}}{\pgfqpoint{4.402778in}{3.450556in}}%
\pgfpathlineto{\pgfqpoint{3.042243in}{3.450556in}}%
\pgfpathquadraticcurveto{\pgfqpoint{3.014465in}{3.450556in}}{\pgfqpoint{3.014465in}{3.422778in}}%
\pgfpathlineto{\pgfqpoint{3.014465in}{3.183045in}}%
\pgfpathquadraticcurveto{\pgfqpoint{3.014465in}{3.155267in}}{\pgfqpoint{3.042243in}{3.155267in}}%
\pgfpathclose%
\pgfusepath{stroke,fill}%
\end{pgfscope}%
\begin{pgfscope}%
\pgfsetbuttcap%
\pgfsetroundjoin%
\pgfsetlinewidth{1.505625pt}%
\definecolor{currentstroke}{rgb}{0.000000,0.000000,0.000000}%
\pgfsetstrokecolor{currentstroke}%
\pgfsetdash{{5.550000pt}{2.400000pt}}{0.000000pt}%
\pgfpathmoveto{\pgfqpoint{3.070021in}{3.321543in}}%
\pgfpathlineto{\pgfqpoint{3.347799in}{3.321543in}}%
\pgfusepath{stroke}%
\end{pgfscope}%
\begin{pgfscope}%
\definecolor{textcolor}{rgb}{0.000000,0.000000,0.000000}%
\pgfsetstrokecolor{textcolor}%
\pgfsetfillcolor{textcolor}%
\pgftext[x=3.458910in,y=3.272932in,left,base]{\color{textcolor}\sffamily\fontsize{10.000000}{12.000000}\selectfont \(\displaystyle 6.4336N_{\mathrm{QMC}}^{-0.7528}\)}%
\end{pgfscope}%
\end{pgfpicture}%
\makeatother%
\endgroup%

\caption[Quasi-Monte-Carlo error, for $Q=\grad u((1,1))$ and $k=40$.]{Quasi-Monte-Carlo error with increasing number of Quasi-Monte-Carlo points, for $Q=\grad u((1,1))$ and $k=40$. \label{fig:qmcgradient_top_right0}}
\end{figure}
\begin{figure}[h]
    \centering
    %% Creator: Matplotlib, PGF backend
%%
%% To include the figure in your LaTeX document, write
%%   \input{<filename>.pgf}
%%
%% Make sure the required packages are loaded in your preamble
%%   \usepackage{pgf}
%%
%% Figures using additional raster images can only be included by \input if
%% they are in the same directory as the main LaTeX file. For loading figures
%% from other directories you can use the `import` package
%%   \usepackage{import}
%% and then include the figures with
%%   \import{<path to file>}{<filename>.pgf}
%%
%% Matplotlib used the following preamble
%%   \usepackage{fontspec}
%%   \setmainfont{DejaVuSerif.ttf}[Path=/home/owen/progs/firedrake-complex/firedrake/lib/python3.5/site-packages/matplotlib/mpl-data/fonts/ttf/]
%%   \setsansfont{DejaVuSans.ttf}[Path=/home/owen/progs/firedrake-complex/firedrake/lib/python3.5/site-packages/matplotlib/mpl-data/fonts/ttf/]
%%   \setmonofont{DejaVuSansMono.ttf}[Path=/home/owen/progs/firedrake-complex/firedrake/lib/python3.5/site-packages/matplotlib/mpl-data/fonts/ttf/]
%%
\begingroup%
\makeatletter%
\begin{pgfpicture}%
\pgfpathrectangle{\pgfpointorigin}{\pgfqpoint{5.000000in}{4.000000in}}%
\pgfusepath{use as bounding box, clip}%
\begin{pgfscope}%
\pgfsetbuttcap%
\pgfsetmiterjoin%
\definecolor{currentfill}{rgb}{1.000000,1.000000,1.000000}%
\pgfsetfillcolor{currentfill}%
\pgfsetlinewidth{0.000000pt}%
\definecolor{currentstroke}{rgb}{1.000000,1.000000,1.000000}%
\pgfsetstrokecolor{currentstroke}%
\pgfsetdash{}{0pt}%
\pgfpathmoveto{\pgfqpoint{0.000000in}{0.000000in}}%
\pgfpathlineto{\pgfqpoint{5.000000in}{0.000000in}}%
\pgfpathlineto{\pgfqpoint{5.000000in}{4.000000in}}%
\pgfpathlineto{\pgfqpoint{0.000000in}{4.000000in}}%
\pgfpathclose%
\pgfusepath{fill}%
\end{pgfscope}%
\begin{pgfscope}%
\pgfsetbuttcap%
\pgfsetmiterjoin%
\definecolor{currentfill}{rgb}{1.000000,1.000000,1.000000}%
\pgfsetfillcolor{currentfill}%
\pgfsetlinewidth{0.000000pt}%
\definecolor{currentstroke}{rgb}{0.000000,0.000000,0.000000}%
\pgfsetstrokecolor{currentstroke}%
\pgfsetstrokeopacity{0.000000}%
\pgfsetdash{}{0pt}%
\pgfpathmoveto{\pgfqpoint{0.625000in}{0.440000in}}%
\pgfpathlineto{\pgfqpoint{4.500000in}{0.440000in}}%
\pgfpathlineto{\pgfqpoint{4.500000in}{3.520000in}}%
\pgfpathlineto{\pgfqpoint{0.625000in}{3.520000in}}%
\pgfpathclose%
\pgfusepath{fill}%
\end{pgfscope}%
\begin{pgfscope}%
\pgfsetbuttcap%
\pgfsetroundjoin%
\definecolor{currentfill}{rgb}{0.000000,0.000000,0.000000}%
\pgfsetfillcolor{currentfill}%
\pgfsetlinewidth{0.803000pt}%
\definecolor{currentstroke}{rgb}{0.000000,0.000000,0.000000}%
\pgfsetstrokecolor{currentstroke}%
\pgfsetdash{}{0pt}%
\pgfsys@defobject{currentmarker}{\pgfqpoint{0.000000in}{-0.048611in}}{\pgfqpoint{0.000000in}{0.000000in}}{%
\pgfpathmoveto{\pgfqpoint{0.000000in}{0.000000in}}%
\pgfpathlineto{\pgfqpoint{0.000000in}{-0.048611in}}%
\pgfusepath{stroke,fill}%
}%
\begin{pgfscope}%
\pgfsys@transformshift{0.801136in}{0.440000in}%
\pgfsys@useobject{currentmarker}{}%
\end{pgfscope}%
\end{pgfscope}%
\begin{pgfscope}%
\definecolor{textcolor}{rgb}{0.000000,0.000000,0.000000}%
\pgfsetstrokecolor{textcolor}%
\pgfsetfillcolor{textcolor}%
\pgftext[x=0.801136in,y=0.342778in,,top]{\color{textcolor}\sffamily\fontsize{11.000000}{13.200000}\selectfont \(\displaystyle 10^{0}\)}%
\end{pgfscope}%
\begin{pgfscope}%
\pgfsetbuttcap%
\pgfsetroundjoin%
\definecolor{currentfill}{rgb}{0.000000,0.000000,0.000000}%
\pgfsetfillcolor{currentfill}%
\pgfsetlinewidth{0.803000pt}%
\definecolor{currentstroke}{rgb}{0.000000,0.000000,0.000000}%
\pgfsetstrokecolor{currentstroke}%
\pgfsetdash{}{0pt}%
\pgfsys@defobject{currentmarker}{\pgfqpoint{0.000000in}{-0.048611in}}{\pgfqpoint{0.000000in}{0.000000in}}{%
\pgfpathmoveto{\pgfqpoint{0.000000in}{0.000000in}}%
\pgfpathlineto{\pgfqpoint{0.000000in}{-0.048611in}}%
\pgfusepath{stroke,fill}%
}%
\begin{pgfscope}%
\pgfsys@transformshift{1.864977in}{0.440000in}%
\pgfsys@useobject{currentmarker}{}%
\end{pgfscope}%
\end{pgfscope}%
\begin{pgfscope}%
\definecolor{textcolor}{rgb}{0.000000,0.000000,0.000000}%
\pgfsetstrokecolor{textcolor}%
\pgfsetfillcolor{textcolor}%
\pgftext[x=1.864977in,y=0.342778in,,top]{\color{textcolor}\sffamily\fontsize{11.000000}{13.200000}\selectfont \(\displaystyle 10^{1}\)}%
\end{pgfscope}%
\begin{pgfscope}%
\pgfsetbuttcap%
\pgfsetroundjoin%
\definecolor{currentfill}{rgb}{0.000000,0.000000,0.000000}%
\pgfsetfillcolor{currentfill}%
\pgfsetlinewidth{0.803000pt}%
\definecolor{currentstroke}{rgb}{0.000000,0.000000,0.000000}%
\pgfsetstrokecolor{currentstroke}%
\pgfsetdash{}{0pt}%
\pgfsys@defobject{currentmarker}{\pgfqpoint{0.000000in}{-0.048611in}}{\pgfqpoint{0.000000in}{0.000000in}}{%
\pgfpathmoveto{\pgfqpoint{0.000000in}{0.000000in}}%
\pgfpathlineto{\pgfqpoint{0.000000in}{-0.048611in}}%
\pgfusepath{stroke,fill}%
}%
\begin{pgfscope}%
\pgfsys@transformshift{2.928818in}{0.440000in}%
\pgfsys@useobject{currentmarker}{}%
\end{pgfscope}%
\end{pgfscope}%
\begin{pgfscope}%
\definecolor{textcolor}{rgb}{0.000000,0.000000,0.000000}%
\pgfsetstrokecolor{textcolor}%
\pgfsetfillcolor{textcolor}%
\pgftext[x=2.928818in,y=0.342778in,,top]{\color{textcolor}\sffamily\fontsize{11.000000}{13.200000}\selectfont \(\displaystyle 10^{2}\)}%
\end{pgfscope}%
\begin{pgfscope}%
\pgfsetbuttcap%
\pgfsetroundjoin%
\definecolor{currentfill}{rgb}{0.000000,0.000000,0.000000}%
\pgfsetfillcolor{currentfill}%
\pgfsetlinewidth{0.803000pt}%
\definecolor{currentstroke}{rgb}{0.000000,0.000000,0.000000}%
\pgfsetstrokecolor{currentstroke}%
\pgfsetdash{}{0pt}%
\pgfsys@defobject{currentmarker}{\pgfqpoint{0.000000in}{-0.048611in}}{\pgfqpoint{0.000000in}{0.000000in}}{%
\pgfpathmoveto{\pgfqpoint{0.000000in}{0.000000in}}%
\pgfpathlineto{\pgfqpoint{0.000000in}{-0.048611in}}%
\pgfusepath{stroke,fill}%
}%
\begin{pgfscope}%
\pgfsys@transformshift{3.992658in}{0.440000in}%
\pgfsys@useobject{currentmarker}{}%
\end{pgfscope}%
\end{pgfscope}%
\begin{pgfscope}%
\definecolor{textcolor}{rgb}{0.000000,0.000000,0.000000}%
\pgfsetstrokecolor{textcolor}%
\pgfsetfillcolor{textcolor}%
\pgftext[x=3.992658in,y=0.342778in,,top]{\color{textcolor}\sffamily\fontsize{11.000000}{13.200000}\selectfont \(\displaystyle 10^{3}\)}%
\end{pgfscope}%
\begin{pgfscope}%
\pgfsetbuttcap%
\pgfsetroundjoin%
\definecolor{currentfill}{rgb}{0.000000,0.000000,0.000000}%
\pgfsetfillcolor{currentfill}%
\pgfsetlinewidth{0.602250pt}%
\definecolor{currentstroke}{rgb}{0.000000,0.000000,0.000000}%
\pgfsetstrokecolor{currentstroke}%
\pgfsetdash{}{0pt}%
\pgfsys@defobject{currentmarker}{\pgfqpoint{0.000000in}{-0.027778in}}{\pgfqpoint{0.000000in}{0.000000in}}{%
\pgfpathmoveto{\pgfqpoint{0.000000in}{0.000000in}}%
\pgfpathlineto{\pgfqpoint{0.000000in}{-0.027778in}}%
\pgfusepath{stroke,fill}%
}%
\begin{pgfscope}%
\pgfsys@transformshift{0.636345in}{0.440000in}%
\pgfsys@useobject{currentmarker}{}%
\end{pgfscope}%
\end{pgfscope}%
\begin{pgfscope}%
\pgfsetbuttcap%
\pgfsetroundjoin%
\definecolor{currentfill}{rgb}{0.000000,0.000000,0.000000}%
\pgfsetfillcolor{currentfill}%
\pgfsetlinewidth{0.602250pt}%
\definecolor{currentstroke}{rgb}{0.000000,0.000000,0.000000}%
\pgfsetstrokecolor{currentstroke}%
\pgfsetdash{}{0pt}%
\pgfsys@defobject{currentmarker}{\pgfqpoint{0.000000in}{-0.027778in}}{\pgfqpoint{0.000000in}{0.000000in}}{%
\pgfpathmoveto{\pgfqpoint{0.000000in}{0.000000in}}%
\pgfpathlineto{\pgfqpoint{0.000000in}{-0.027778in}}%
\pgfusepath{stroke,fill}%
}%
\begin{pgfscope}%
\pgfsys@transformshift{0.698040in}{0.440000in}%
\pgfsys@useobject{currentmarker}{}%
\end{pgfscope}%
\end{pgfscope}%
\begin{pgfscope}%
\pgfsetbuttcap%
\pgfsetroundjoin%
\definecolor{currentfill}{rgb}{0.000000,0.000000,0.000000}%
\pgfsetfillcolor{currentfill}%
\pgfsetlinewidth{0.602250pt}%
\definecolor{currentstroke}{rgb}{0.000000,0.000000,0.000000}%
\pgfsetstrokecolor{currentstroke}%
\pgfsetdash{}{0pt}%
\pgfsys@defobject{currentmarker}{\pgfqpoint{0.000000in}{-0.027778in}}{\pgfqpoint{0.000000in}{0.000000in}}{%
\pgfpathmoveto{\pgfqpoint{0.000000in}{0.000000in}}%
\pgfpathlineto{\pgfqpoint{0.000000in}{-0.027778in}}%
\pgfusepath{stroke,fill}%
}%
\begin{pgfscope}%
\pgfsys@transformshift{0.752458in}{0.440000in}%
\pgfsys@useobject{currentmarker}{}%
\end{pgfscope}%
\end{pgfscope}%
\begin{pgfscope}%
\pgfsetbuttcap%
\pgfsetroundjoin%
\definecolor{currentfill}{rgb}{0.000000,0.000000,0.000000}%
\pgfsetfillcolor{currentfill}%
\pgfsetlinewidth{0.602250pt}%
\definecolor{currentstroke}{rgb}{0.000000,0.000000,0.000000}%
\pgfsetstrokecolor{currentstroke}%
\pgfsetdash{}{0pt}%
\pgfsys@defobject{currentmarker}{\pgfqpoint{0.000000in}{-0.027778in}}{\pgfqpoint{0.000000in}{0.000000in}}{%
\pgfpathmoveto{\pgfqpoint{0.000000in}{0.000000in}}%
\pgfpathlineto{\pgfqpoint{0.000000in}{-0.027778in}}%
\pgfusepath{stroke,fill}%
}%
\begin{pgfscope}%
\pgfsys@transformshift{1.121384in}{0.440000in}%
\pgfsys@useobject{currentmarker}{}%
\end{pgfscope}%
\end{pgfscope}%
\begin{pgfscope}%
\pgfsetbuttcap%
\pgfsetroundjoin%
\definecolor{currentfill}{rgb}{0.000000,0.000000,0.000000}%
\pgfsetfillcolor{currentfill}%
\pgfsetlinewidth{0.602250pt}%
\definecolor{currentstroke}{rgb}{0.000000,0.000000,0.000000}%
\pgfsetstrokecolor{currentstroke}%
\pgfsetdash{}{0pt}%
\pgfsys@defobject{currentmarker}{\pgfqpoint{0.000000in}{-0.027778in}}{\pgfqpoint{0.000000in}{0.000000in}}{%
\pgfpathmoveto{\pgfqpoint{0.000000in}{0.000000in}}%
\pgfpathlineto{\pgfqpoint{0.000000in}{-0.027778in}}%
\pgfusepath{stroke,fill}%
}%
\begin{pgfscope}%
\pgfsys@transformshift{1.308717in}{0.440000in}%
\pgfsys@useobject{currentmarker}{}%
\end{pgfscope}%
\end{pgfscope}%
\begin{pgfscope}%
\pgfsetbuttcap%
\pgfsetroundjoin%
\definecolor{currentfill}{rgb}{0.000000,0.000000,0.000000}%
\pgfsetfillcolor{currentfill}%
\pgfsetlinewidth{0.602250pt}%
\definecolor{currentstroke}{rgb}{0.000000,0.000000,0.000000}%
\pgfsetstrokecolor{currentstroke}%
\pgfsetdash{}{0pt}%
\pgfsys@defobject{currentmarker}{\pgfqpoint{0.000000in}{-0.027778in}}{\pgfqpoint{0.000000in}{0.000000in}}{%
\pgfpathmoveto{\pgfqpoint{0.000000in}{0.000000in}}%
\pgfpathlineto{\pgfqpoint{0.000000in}{-0.027778in}}%
\pgfusepath{stroke,fill}%
}%
\begin{pgfscope}%
\pgfsys@transformshift{1.441632in}{0.440000in}%
\pgfsys@useobject{currentmarker}{}%
\end{pgfscope}%
\end{pgfscope}%
\begin{pgfscope}%
\pgfsetbuttcap%
\pgfsetroundjoin%
\definecolor{currentfill}{rgb}{0.000000,0.000000,0.000000}%
\pgfsetfillcolor{currentfill}%
\pgfsetlinewidth{0.602250pt}%
\definecolor{currentstroke}{rgb}{0.000000,0.000000,0.000000}%
\pgfsetstrokecolor{currentstroke}%
\pgfsetdash{}{0pt}%
\pgfsys@defobject{currentmarker}{\pgfqpoint{0.000000in}{-0.027778in}}{\pgfqpoint{0.000000in}{0.000000in}}{%
\pgfpathmoveto{\pgfqpoint{0.000000in}{0.000000in}}%
\pgfpathlineto{\pgfqpoint{0.000000in}{-0.027778in}}%
\pgfusepath{stroke,fill}%
}%
\begin{pgfscope}%
\pgfsys@transformshift{1.544729in}{0.440000in}%
\pgfsys@useobject{currentmarker}{}%
\end{pgfscope}%
\end{pgfscope}%
\begin{pgfscope}%
\pgfsetbuttcap%
\pgfsetroundjoin%
\definecolor{currentfill}{rgb}{0.000000,0.000000,0.000000}%
\pgfsetfillcolor{currentfill}%
\pgfsetlinewidth{0.602250pt}%
\definecolor{currentstroke}{rgb}{0.000000,0.000000,0.000000}%
\pgfsetstrokecolor{currentstroke}%
\pgfsetdash{}{0pt}%
\pgfsys@defobject{currentmarker}{\pgfqpoint{0.000000in}{-0.027778in}}{\pgfqpoint{0.000000in}{0.000000in}}{%
\pgfpathmoveto{\pgfqpoint{0.000000in}{0.000000in}}%
\pgfpathlineto{\pgfqpoint{0.000000in}{-0.027778in}}%
\pgfusepath{stroke,fill}%
}%
\begin{pgfscope}%
\pgfsys@transformshift{1.628965in}{0.440000in}%
\pgfsys@useobject{currentmarker}{}%
\end{pgfscope}%
\end{pgfscope}%
\begin{pgfscope}%
\pgfsetbuttcap%
\pgfsetroundjoin%
\definecolor{currentfill}{rgb}{0.000000,0.000000,0.000000}%
\pgfsetfillcolor{currentfill}%
\pgfsetlinewidth{0.602250pt}%
\definecolor{currentstroke}{rgb}{0.000000,0.000000,0.000000}%
\pgfsetstrokecolor{currentstroke}%
\pgfsetdash{}{0pt}%
\pgfsys@defobject{currentmarker}{\pgfqpoint{0.000000in}{-0.027778in}}{\pgfqpoint{0.000000in}{0.000000in}}{%
\pgfpathmoveto{\pgfqpoint{0.000000in}{0.000000in}}%
\pgfpathlineto{\pgfqpoint{0.000000in}{-0.027778in}}%
\pgfusepath{stroke,fill}%
}%
\begin{pgfscope}%
\pgfsys@transformshift{1.700186in}{0.440000in}%
\pgfsys@useobject{currentmarker}{}%
\end{pgfscope}%
\end{pgfscope}%
\begin{pgfscope}%
\pgfsetbuttcap%
\pgfsetroundjoin%
\definecolor{currentfill}{rgb}{0.000000,0.000000,0.000000}%
\pgfsetfillcolor{currentfill}%
\pgfsetlinewidth{0.602250pt}%
\definecolor{currentstroke}{rgb}{0.000000,0.000000,0.000000}%
\pgfsetstrokecolor{currentstroke}%
\pgfsetdash{}{0pt}%
\pgfsys@defobject{currentmarker}{\pgfqpoint{0.000000in}{-0.027778in}}{\pgfqpoint{0.000000in}{0.000000in}}{%
\pgfpathmoveto{\pgfqpoint{0.000000in}{0.000000in}}%
\pgfpathlineto{\pgfqpoint{0.000000in}{-0.027778in}}%
\pgfusepath{stroke,fill}%
}%
\begin{pgfscope}%
\pgfsys@transformshift{1.761880in}{0.440000in}%
\pgfsys@useobject{currentmarker}{}%
\end{pgfscope}%
\end{pgfscope}%
\begin{pgfscope}%
\pgfsetbuttcap%
\pgfsetroundjoin%
\definecolor{currentfill}{rgb}{0.000000,0.000000,0.000000}%
\pgfsetfillcolor{currentfill}%
\pgfsetlinewidth{0.602250pt}%
\definecolor{currentstroke}{rgb}{0.000000,0.000000,0.000000}%
\pgfsetstrokecolor{currentstroke}%
\pgfsetdash{}{0pt}%
\pgfsys@defobject{currentmarker}{\pgfqpoint{0.000000in}{-0.027778in}}{\pgfqpoint{0.000000in}{0.000000in}}{%
\pgfpathmoveto{\pgfqpoint{0.000000in}{0.000000in}}%
\pgfpathlineto{\pgfqpoint{0.000000in}{-0.027778in}}%
\pgfusepath{stroke,fill}%
}%
\begin{pgfscope}%
\pgfsys@transformshift{1.816298in}{0.440000in}%
\pgfsys@useobject{currentmarker}{}%
\end{pgfscope}%
\end{pgfscope}%
\begin{pgfscope}%
\pgfsetbuttcap%
\pgfsetroundjoin%
\definecolor{currentfill}{rgb}{0.000000,0.000000,0.000000}%
\pgfsetfillcolor{currentfill}%
\pgfsetlinewidth{0.602250pt}%
\definecolor{currentstroke}{rgb}{0.000000,0.000000,0.000000}%
\pgfsetstrokecolor{currentstroke}%
\pgfsetdash{}{0pt}%
\pgfsys@defobject{currentmarker}{\pgfqpoint{0.000000in}{-0.027778in}}{\pgfqpoint{0.000000in}{0.000000in}}{%
\pgfpathmoveto{\pgfqpoint{0.000000in}{0.000000in}}%
\pgfpathlineto{\pgfqpoint{0.000000in}{-0.027778in}}%
\pgfusepath{stroke,fill}%
}%
\begin{pgfscope}%
\pgfsys@transformshift{2.185225in}{0.440000in}%
\pgfsys@useobject{currentmarker}{}%
\end{pgfscope}%
\end{pgfscope}%
\begin{pgfscope}%
\pgfsetbuttcap%
\pgfsetroundjoin%
\definecolor{currentfill}{rgb}{0.000000,0.000000,0.000000}%
\pgfsetfillcolor{currentfill}%
\pgfsetlinewidth{0.602250pt}%
\definecolor{currentstroke}{rgb}{0.000000,0.000000,0.000000}%
\pgfsetstrokecolor{currentstroke}%
\pgfsetdash{}{0pt}%
\pgfsys@defobject{currentmarker}{\pgfqpoint{0.000000in}{-0.027778in}}{\pgfqpoint{0.000000in}{0.000000in}}{%
\pgfpathmoveto{\pgfqpoint{0.000000in}{0.000000in}}%
\pgfpathlineto{\pgfqpoint{0.000000in}{-0.027778in}}%
\pgfusepath{stroke,fill}%
}%
\begin{pgfscope}%
\pgfsys@transformshift{2.372558in}{0.440000in}%
\pgfsys@useobject{currentmarker}{}%
\end{pgfscope}%
\end{pgfscope}%
\begin{pgfscope}%
\pgfsetbuttcap%
\pgfsetroundjoin%
\definecolor{currentfill}{rgb}{0.000000,0.000000,0.000000}%
\pgfsetfillcolor{currentfill}%
\pgfsetlinewidth{0.602250pt}%
\definecolor{currentstroke}{rgb}{0.000000,0.000000,0.000000}%
\pgfsetstrokecolor{currentstroke}%
\pgfsetdash{}{0pt}%
\pgfsys@defobject{currentmarker}{\pgfqpoint{0.000000in}{-0.027778in}}{\pgfqpoint{0.000000in}{0.000000in}}{%
\pgfpathmoveto{\pgfqpoint{0.000000in}{0.000000in}}%
\pgfpathlineto{\pgfqpoint{0.000000in}{-0.027778in}}%
\pgfusepath{stroke,fill}%
}%
\begin{pgfscope}%
\pgfsys@transformshift{2.505473in}{0.440000in}%
\pgfsys@useobject{currentmarker}{}%
\end{pgfscope}%
\end{pgfscope}%
\begin{pgfscope}%
\pgfsetbuttcap%
\pgfsetroundjoin%
\definecolor{currentfill}{rgb}{0.000000,0.000000,0.000000}%
\pgfsetfillcolor{currentfill}%
\pgfsetlinewidth{0.602250pt}%
\definecolor{currentstroke}{rgb}{0.000000,0.000000,0.000000}%
\pgfsetstrokecolor{currentstroke}%
\pgfsetdash{}{0pt}%
\pgfsys@defobject{currentmarker}{\pgfqpoint{0.000000in}{-0.027778in}}{\pgfqpoint{0.000000in}{0.000000in}}{%
\pgfpathmoveto{\pgfqpoint{0.000000in}{0.000000in}}%
\pgfpathlineto{\pgfqpoint{0.000000in}{-0.027778in}}%
\pgfusepath{stroke,fill}%
}%
\begin{pgfscope}%
\pgfsys@transformshift{2.608570in}{0.440000in}%
\pgfsys@useobject{currentmarker}{}%
\end{pgfscope}%
\end{pgfscope}%
\begin{pgfscope}%
\pgfsetbuttcap%
\pgfsetroundjoin%
\definecolor{currentfill}{rgb}{0.000000,0.000000,0.000000}%
\pgfsetfillcolor{currentfill}%
\pgfsetlinewidth{0.602250pt}%
\definecolor{currentstroke}{rgb}{0.000000,0.000000,0.000000}%
\pgfsetstrokecolor{currentstroke}%
\pgfsetdash{}{0pt}%
\pgfsys@defobject{currentmarker}{\pgfqpoint{0.000000in}{-0.027778in}}{\pgfqpoint{0.000000in}{0.000000in}}{%
\pgfpathmoveto{\pgfqpoint{0.000000in}{0.000000in}}%
\pgfpathlineto{\pgfqpoint{0.000000in}{-0.027778in}}%
\pgfusepath{stroke,fill}%
}%
\begin{pgfscope}%
\pgfsys@transformshift{2.692806in}{0.440000in}%
\pgfsys@useobject{currentmarker}{}%
\end{pgfscope}%
\end{pgfscope}%
\begin{pgfscope}%
\pgfsetbuttcap%
\pgfsetroundjoin%
\definecolor{currentfill}{rgb}{0.000000,0.000000,0.000000}%
\pgfsetfillcolor{currentfill}%
\pgfsetlinewidth{0.602250pt}%
\definecolor{currentstroke}{rgb}{0.000000,0.000000,0.000000}%
\pgfsetstrokecolor{currentstroke}%
\pgfsetdash{}{0pt}%
\pgfsys@defobject{currentmarker}{\pgfqpoint{0.000000in}{-0.027778in}}{\pgfqpoint{0.000000in}{0.000000in}}{%
\pgfpathmoveto{\pgfqpoint{0.000000in}{0.000000in}}%
\pgfpathlineto{\pgfqpoint{0.000000in}{-0.027778in}}%
\pgfusepath{stroke,fill}%
}%
\begin{pgfscope}%
\pgfsys@transformshift{2.764027in}{0.440000in}%
\pgfsys@useobject{currentmarker}{}%
\end{pgfscope}%
\end{pgfscope}%
\begin{pgfscope}%
\pgfsetbuttcap%
\pgfsetroundjoin%
\definecolor{currentfill}{rgb}{0.000000,0.000000,0.000000}%
\pgfsetfillcolor{currentfill}%
\pgfsetlinewidth{0.602250pt}%
\definecolor{currentstroke}{rgb}{0.000000,0.000000,0.000000}%
\pgfsetstrokecolor{currentstroke}%
\pgfsetdash{}{0pt}%
\pgfsys@defobject{currentmarker}{\pgfqpoint{0.000000in}{-0.027778in}}{\pgfqpoint{0.000000in}{0.000000in}}{%
\pgfpathmoveto{\pgfqpoint{0.000000in}{0.000000in}}%
\pgfpathlineto{\pgfqpoint{0.000000in}{-0.027778in}}%
\pgfusepath{stroke,fill}%
}%
\begin{pgfscope}%
\pgfsys@transformshift{2.825721in}{0.440000in}%
\pgfsys@useobject{currentmarker}{}%
\end{pgfscope}%
\end{pgfscope}%
\begin{pgfscope}%
\pgfsetbuttcap%
\pgfsetroundjoin%
\definecolor{currentfill}{rgb}{0.000000,0.000000,0.000000}%
\pgfsetfillcolor{currentfill}%
\pgfsetlinewidth{0.602250pt}%
\definecolor{currentstroke}{rgb}{0.000000,0.000000,0.000000}%
\pgfsetstrokecolor{currentstroke}%
\pgfsetdash{}{0pt}%
\pgfsys@defobject{currentmarker}{\pgfqpoint{0.000000in}{-0.027778in}}{\pgfqpoint{0.000000in}{0.000000in}}{%
\pgfpathmoveto{\pgfqpoint{0.000000in}{0.000000in}}%
\pgfpathlineto{\pgfqpoint{0.000000in}{-0.027778in}}%
\pgfusepath{stroke,fill}%
}%
\begin{pgfscope}%
\pgfsys@transformshift{2.880139in}{0.440000in}%
\pgfsys@useobject{currentmarker}{}%
\end{pgfscope}%
\end{pgfscope}%
\begin{pgfscope}%
\pgfsetbuttcap%
\pgfsetroundjoin%
\definecolor{currentfill}{rgb}{0.000000,0.000000,0.000000}%
\pgfsetfillcolor{currentfill}%
\pgfsetlinewidth{0.602250pt}%
\definecolor{currentstroke}{rgb}{0.000000,0.000000,0.000000}%
\pgfsetstrokecolor{currentstroke}%
\pgfsetdash{}{0pt}%
\pgfsys@defobject{currentmarker}{\pgfqpoint{0.000000in}{-0.027778in}}{\pgfqpoint{0.000000in}{0.000000in}}{%
\pgfpathmoveto{\pgfqpoint{0.000000in}{0.000000in}}%
\pgfpathlineto{\pgfqpoint{0.000000in}{-0.027778in}}%
\pgfusepath{stroke,fill}%
}%
\begin{pgfscope}%
\pgfsys@transformshift{3.249066in}{0.440000in}%
\pgfsys@useobject{currentmarker}{}%
\end{pgfscope}%
\end{pgfscope}%
\begin{pgfscope}%
\pgfsetbuttcap%
\pgfsetroundjoin%
\definecolor{currentfill}{rgb}{0.000000,0.000000,0.000000}%
\pgfsetfillcolor{currentfill}%
\pgfsetlinewidth{0.602250pt}%
\definecolor{currentstroke}{rgb}{0.000000,0.000000,0.000000}%
\pgfsetstrokecolor{currentstroke}%
\pgfsetdash{}{0pt}%
\pgfsys@defobject{currentmarker}{\pgfqpoint{0.000000in}{-0.027778in}}{\pgfqpoint{0.000000in}{0.000000in}}{%
\pgfpathmoveto{\pgfqpoint{0.000000in}{0.000000in}}%
\pgfpathlineto{\pgfqpoint{0.000000in}{-0.027778in}}%
\pgfusepath{stroke,fill}%
}%
\begin{pgfscope}%
\pgfsys@transformshift{3.436399in}{0.440000in}%
\pgfsys@useobject{currentmarker}{}%
\end{pgfscope}%
\end{pgfscope}%
\begin{pgfscope}%
\pgfsetbuttcap%
\pgfsetroundjoin%
\definecolor{currentfill}{rgb}{0.000000,0.000000,0.000000}%
\pgfsetfillcolor{currentfill}%
\pgfsetlinewidth{0.602250pt}%
\definecolor{currentstroke}{rgb}{0.000000,0.000000,0.000000}%
\pgfsetstrokecolor{currentstroke}%
\pgfsetdash{}{0pt}%
\pgfsys@defobject{currentmarker}{\pgfqpoint{0.000000in}{-0.027778in}}{\pgfqpoint{0.000000in}{0.000000in}}{%
\pgfpathmoveto{\pgfqpoint{0.000000in}{0.000000in}}%
\pgfpathlineto{\pgfqpoint{0.000000in}{-0.027778in}}%
\pgfusepath{stroke,fill}%
}%
\begin{pgfscope}%
\pgfsys@transformshift{3.569313in}{0.440000in}%
\pgfsys@useobject{currentmarker}{}%
\end{pgfscope}%
\end{pgfscope}%
\begin{pgfscope}%
\pgfsetbuttcap%
\pgfsetroundjoin%
\definecolor{currentfill}{rgb}{0.000000,0.000000,0.000000}%
\pgfsetfillcolor{currentfill}%
\pgfsetlinewidth{0.602250pt}%
\definecolor{currentstroke}{rgb}{0.000000,0.000000,0.000000}%
\pgfsetstrokecolor{currentstroke}%
\pgfsetdash{}{0pt}%
\pgfsys@defobject{currentmarker}{\pgfqpoint{0.000000in}{-0.027778in}}{\pgfqpoint{0.000000in}{0.000000in}}{%
\pgfpathmoveto{\pgfqpoint{0.000000in}{0.000000in}}%
\pgfpathlineto{\pgfqpoint{0.000000in}{-0.027778in}}%
\pgfusepath{stroke,fill}%
}%
\begin{pgfscope}%
\pgfsys@transformshift{3.672410in}{0.440000in}%
\pgfsys@useobject{currentmarker}{}%
\end{pgfscope}%
\end{pgfscope}%
\begin{pgfscope}%
\pgfsetbuttcap%
\pgfsetroundjoin%
\definecolor{currentfill}{rgb}{0.000000,0.000000,0.000000}%
\pgfsetfillcolor{currentfill}%
\pgfsetlinewidth{0.602250pt}%
\definecolor{currentstroke}{rgb}{0.000000,0.000000,0.000000}%
\pgfsetstrokecolor{currentstroke}%
\pgfsetdash{}{0pt}%
\pgfsys@defobject{currentmarker}{\pgfqpoint{0.000000in}{-0.027778in}}{\pgfqpoint{0.000000in}{0.000000in}}{%
\pgfpathmoveto{\pgfqpoint{0.000000in}{0.000000in}}%
\pgfpathlineto{\pgfqpoint{0.000000in}{-0.027778in}}%
\pgfusepath{stroke,fill}%
}%
\begin{pgfscope}%
\pgfsys@transformshift{3.756646in}{0.440000in}%
\pgfsys@useobject{currentmarker}{}%
\end{pgfscope}%
\end{pgfscope}%
\begin{pgfscope}%
\pgfsetbuttcap%
\pgfsetroundjoin%
\definecolor{currentfill}{rgb}{0.000000,0.000000,0.000000}%
\pgfsetfillcolor{currentfill}%
\pgfsetlinewidth{0.602250pt}%
\definecolor{currentstroke}{rgb}{0.000000,0.000000,0.000000}%
\pgfsetstrokecolor{currentstroke}%
\pgfsetdash{}{0pt}%
\pgfsys@defobject{currentmarker}{\pgfqpoint{0.000000in}{-0.027778in}}{\pgfqpoint{0.000000in}{0.000000in}}{%
\pgfpathmoveto{\pgfqpoint{0.000000in}{0.000000in}}%
\pgfpathlineto{\pgfqpoint{0.000000in}{-0.027778in}}%
\pgfusepath{stroke,fill}%
}%
\begin{pgfscope}%
\pgfsys@transformshift{3.827867in}{0.440000in}%
\pgfsys@useobject{currentmarker}{}%
\end{pgfscope}%
\end{pgfscope}%
\begin{pgfscope}%
\pgfsetbuttcap%
\pgfsetroundjoin%
\definecolor{currentfill}{rgb}{0.000000,0.000000,0.000000}%
\pgfsetfillcolor{currentfill}%
\pgfsetlinewidth{0.602250pt}%
\definecolor{currentstroke}{rgb}{0.000000,0.000000,0.000000}%
\pgfsetstrokecolor{currentstroke}%
\pgfsetdash{}{0pt}%
\pgfsys@defobject{currentmarker}{\pgfqpoint{0.000000in}{-0.027778in}}{\pgfqpoint{0.000000in}{0.000000in}}{%
\pgfpathmoveto{\pgfqpoint{0.000000in}{0.000000in}}%
\pgfpathlineto{\pgfqpoint{0.000000in}{-0.027778in}}%
\pgfusepath{stroke,fill}%
}%
\begin{pgfscope}%
\pgfsys@transformshift{3.889561in}{0.440000in}%
\pgfsys@useobject{currentmarker}{}%
\end{pgfscope}%
\end{pgfscope}%
\begin{pgfscope}%
\pgfsetbuttcap%
\pgfsetroundjoin%
\definecolor{currentfill}{rgb}{0.000000,0.000000,0.000000}%
\pgfsetfillcolor{currentfill}%
\pgfsetlinewidth{0.602250pt}%
\definecolor{currentstroke}{rgb}{0.000000,0.000000,0.000000}%
\pgfsetstrokecolor{currentstroke}%
\pgfsetdash{}{0pt}%
\pgfsys@defobject{currentmarker}{\pgfqpoint{0.000000in}{-0.027778in}}{\pgfqpoint{0.000000in}{0.000000in}}{%
\pgfpathmoveto{\pgfqpoint{0.000000in}{0.000000in}}%
\pgfpathlineto{\pgfqpoint{0.000000in}{-0.027778in}}%
\pgfusepath{stroke,fill}%
}%
\begin{pgfscope}%
\pgfsys@transformshift{3.943980in}{0.440000in}%
\pgfsys@useobject{currentmarker}{}%
\end{pgfscope}%
\end{pgfscope}%
\begin{pgfscope}%
\pgfsetbuttcap%
\pgfsetroundjoin%
\definecolor{currentfill}{rgb}{0.000000,0.000000,0.000000}%
\pgfsetfillcolor{currentfill}%
\pgfsetlinewidth{0.602250pt}%
\definecolor{currentstroke}{rgb}{0.000000,0.000000,0.000000}%
\pgfsetstrokecolor{currentstroke}%
\pgfsetdash{}{0pt}%
\pgfsys@defobject{currentmarker}{\pgfqpoint{0.000000in}{-0.027778in}}{\pgfqpoint{0.000000in}{0.000000in}}{%
\pgfpathmoveto{\pgfqpoint{0.000000in}{0.000000in}}%
\pgfpathlineto{\pgfqpoint{0.000000in}{-0.027778in}}%
\pgfusepath{stroke,fill}%
}%
\begin{pgfscope}%
\pgfsys@transformshift{4.312906in}{0.440000in}%
\pgfsys@useobject{currentmarker}{}%
\end{pgfscope}%
\end{pgfscope}%
\begin{pgfscope}%
\pgfsetbuttcap%
\pgfsetroundjoin%
\definecolor{currentfill}{rgb}{0.000000,0.000000,0.000000}%
\pgfsetfillcolor{currentfill}%
\pgfsetlinewidth{0.602250pt}%
\definecolor{currentstroke}{rgb}{0.000000,0.000000,0.000000}%
\pgfsetstrokecolor{currentstroke}%
\pgfsetdash{}{0pt}%
\pgfsys@defobject{currentmarker}{\pgfqpoint{0.000000in}{-0.027778in}}{\pgfqpoint{0.000000in}{0.000000in}}{%
\pgfpathmoveto{\pgfqpoint{0.000000in}{0.000000in}}%
\pgfpathlineto{\pgfqpoint{0.000000in}{-0.027778in}}%
\pgfusepath{stroke,fill}%
}%
\begin{pgfscope}%
\pgfsys@transformshift{4.500239in}{0.440000in}%
\pgfsys@useobject{currentmarker}{}%
\end{pgfscope}%
\end{pgfscope}%
\begin{pgfscope}%
\definecolor{textcolor}{rgb}{0.000000,0.000000,0.000000}%
\pgfsetstrokecolor{textcolor}%
\pgfsetfillcolor{textcolor}%
\pgftext[x=2.562500in,y=0.139368in,,top]{\color{textcolor}\sffamily\fontsize{11.000000}{13.200000}\selectfont \(\displaystyle N_\mathrm{QMC}\)}%
\end{pgfscope}%
\begin{pgfscope}%
\pgfsetbuttcap%
\pgfsetroundjoin%
\definecolor{currentfill}{rgb}{0.000000,0.000000,0.000000}%
\pgfsetfillcolor{currentfill}%
\pgfsetlinewidth{0.803000pt}%
\definecolor{currentstroke}{rgb}{0.000000,0.000000,0.000000}%
\pgfsetstrokecolor{currentstroke}%
\pgfsetdash{}{0pt}%
\pgfsys@defobject{currentmarker}{\pgfqpoint{-0.048611in}{0.000000in}}{\pgfqpoint{0.000000in}{0.000000in}}{%
\pgfpathmoveto{\pgfqpoint{0.000000in}{0.000000in}}%
\pgfpathlineto{\pgfqpoint{-0.048611in}{0.000000in}}%
\pgfusepath{stroke,fill}%
}%
\begin{pgfscope}%
\pgfsys@transformshift{0.625000in}{1.186875in}%
\pgfsys@useobject{currentmarker}{}%
\end{pgfscope}%
\end{pgfscope}%
\begin{pgfscope}%
\definecolor{textcolor}{rgb}{0.000000,0.000000,0.000000}%
\pgfsetstrokecolor{textcolor}%
\pgfsetfillcolor{textcolor}%
\pgftext[x=0.217899in,y=1.128838in,left,base]{\color{textcolor}\sffamily\fontsize{11.000000}{13.200000}\selectfont \(\displaystyle 10^{-1}\)}%
\end{pgfscope}%
\begin{pgfscope}%
\pgfsetbuttcap%
\pgfsetroundjoin%
\definecolor{currentfill}{rgb}{0.000000,0.000000,0.000000}%
\pgfsetfillcolor{currentfill}%
\pgfsetlinewidth{0.803000pt}%
\definecolor{currentstroke}{rgb}{0.000000,0.000000,0.000000}%
\pgfsetstrokecolor{currentstroke}%
\pgfsetdash{}{0pt}%
\pgfsys@defobject{currentmarker}{\pgfqpoint{-0.048611in}{0.000000in}}{\pgfqpoint{0.000000in}{0.000000in}}{%
\pgfpathmoveto{\pgfqpoint{0.000000in}{0.000000in}}%
\pgfpathlineto{\pgfqpoint{-0.048611in}{0.000000in}}%
\pgfusepath{stroke,fill}%
}%
\begin{pgfscope}%
\pgfsys@transformshift{0.625000in}{2.298646in}%
\pgfsys@useobject{currentmarker}{}%
\end{pgfscope}%
\end{pgfscope}%
\begin{pgfscope}%
\definecolor{textcolor}{rgb}{0.000000,0.000000,0.000000}%
\pgfsetstrokecolor{textcolor}%
\pgfsetfillcolor{textcolor}%
\pgftext[x=0.309721in,y=2.240608in,left,base]{\color{textcolor}\sffamily\fontsize{11.000000}{13.200000}\selectfont \(\displaystyle 10^{0}\)}%
\end{pgfscope}%
\begin{pgfscope}%
\pgfsetbuttcap%
\pgfsetroundjoin%
\definecolor{currentfill}{rgb}{0.000000,0.000000,0.000000}%
\pgfsetfillcolor{currentfill}%
\pgfsetlinewidth{0.803000pt}%
\definecolor{currentstroke}{rgb}{0.000000,0.000000,0.000000}%
\pgfsetstrokecolor{currentstroke}%
\pgfsetdash{}{0pt}%
\pgfsys@defobject{currentmarker}{\pgfqpoint{-0.048611in}{0.000000in}}{\pgfqpoint{0.000000in}{0.000000in}}{%
\pgfpathmoveto{\pgfqpoint{0.000000in}{0.000000in}}%
\pgfpathlineto{\pgfqpoint{-0.048611in}{0.000000in}}%
\pgfusepath{stroke,fill}%
}%
\begin{pgfscope}%
\pgfsys@transformshift{0.625000in}{3.410416in}%
\pgfsys@useobject{currentmarker}{}%
\end{pgfscope}%
\end{pgfscope}%
\begin{pgfscope}%
\definecolor{textcolor}{rgb}{0.000000,0.000000,0.000000}%
\pgfsetstrokecolor{textcolor}%
\pgfsetfillcolor{textcolor}%
\pgftext[x=0.309721in,y=3.352379in,left,base]{\color{textcolor}\sffamily\fontsize{11.000000}{13.200000}\selectfont \(\displaystyle 10^{1}\)}%
\end{pgfscope}%
\begin{pgfscope}%
\pgfsetbuttcap%
\pgfsetroundjoin%
\definecolor{currentfill}{rgb}{0.000000,0.000000,0.000000}%
\pgfsetfillcolor{currentfill}%
\pgfsetlinewidth{0.602250pt}%
\definecolor{currentstroke}{rgb}{0.000000,0.000000,0.000000}%
\pgfsetstrokecolor{currentstroke}%
\pgfsetdash{}{0pt}%
\pgfsys@defobject{currentmarker}{\pgfqpoint{-0.027778in}{0.000000in}}{\pgfqpoint{0.000000in}{0.000000in}}{%
\pgfpathmoveto{\pgfqpoint{0.000000in}{0.000000in}}%
\pgfpathlineto{\pgfqpoint{-0.027778in}{0.000000in}}%
\pgfusepath{stroke,fill}%
}%
\begin{pgfscope}%
\pgfsys@transformshift{0.625000in}{0.605554in}%
\pgfsys@useobject{currentmarker}{}%
\end{pgfscope}%
\end{pgfscope}%
\begin{pgfscope}%
\pgfsetbuttcap%
\pgfsetroundjoin%
\definecolor{currentfill}{rgb}{0.000000,0.000000,0.000000}%
\pgfsetfillcolor{currentfill}%
\pgfsetlinewidth{0.602250pt}%
\definecolor{currentstroke}{rgb}{0.000000,0.000000,0.000000}%
\pgfsetstrokecolor{currentstroke}%
\pgfsetdash{}{0pt}%
\pgfsys@defobject{currentmarker}{\pgfqpoint{-0.027778in}{0.000000in}}{\pgfqpoint{0.000000in}{0.000000in}}{%
\pgfpathmoveto{\pgfqpoint{0.000000in}{0.000000in}}%
\pgfpathlineto{\pgfqpoint{-0.027778in}{0.000000in}}%
\pgfusepath{stroke,fill}%
}%
\begin{pgfscope}%
\pgfsys@transformshift{0.625000in}{0.744457in}%
\pgfsys@useobject{currentmarker}{}%
\end{pgfscope}%
\end{pgfscope}%
\begin{pgfscope}%
\pgfsetbuttcap%
\pgfsetroundjoin%
\definecolor{currentfill}{rgb}{0.000000,0.000000,0.000000}%
\pgfsetfillcolor{currentfill}%
\pgfsetlinewidth{0.602250pt}%
\definecolor{currentstroke}{rgb}{0.000000,0.000000,0.000000}%
\pgfsetstrokecolor{currentstroke}%
\pgfsetdash{}{0pt}%
\pgfsys@defobject{currentmarker}{\pgfqpoint{-0.027778in}{0.000000in}}{\pgfqpoint{0.000000in}{0.000000in}}{%
\pgfpathmoveto{\pgfqpoint{0.000000in}{0.000000in}}%
\pgfpathlineto{\pgfqpoint{-0.027778in}{0.000000in}}%
\pgfusepath{stroke,fill}%
}%
\begin{pgfscope}%
\pgfsys@transformshift{0.625000in}{0.852199in}%
\pgfsys@useobject{currentmarker}{}%
\end{pgfscope}%
\end{pgfscope}%
\begin{pgfscope}%
\pgfsetbuttcap%
\pgfsetroundjoin%
\definecolor{currentfill}{rgb}{0.000000,0.000000,0.000000}%
\pgfsetfillcolor{currentfill}%
\pgfsetlinewidth{0.602250pt}%
\definecolor{currentstroke}{rgb}{0.000000,0.000000,0.000000}%
\pgfsetstrokecolor{currentstroke}%
\pgfsetdash{}{0pt}%
\pgfsys@defobject{currentmarker}{\pgfqpoint{-0.027778in}{0.000000in}}{\pgfqpoint{0.000000in}{0.000000in}}{%
\pgfpathmoveto{\pgfqpoint{0.000000in}{0.000000in}}%
\pgfpathlineto{\pgfqpoint{-0.027778in}{0.000000in}}%
\pgfusepath{stroke,fill}%
}%
\begin{pgfscope}%
\pgfsys@transformshift{0.625000in}{0.940230in}%
\pgfsys@useobject{currentmarker}{}%
\end{pgfscope}%
\end{pgfscope}%
\begin{pgfscope}%
\pgfsetbuttcap%
\pgfsetroundjoin%
\definecolor{currentfill}{rgb}{0.000000,0.000000,0.000000}%
\pgfsetfillcolor{currentfill}%
\pgfsetlinewidth{0.602250pt}%
\definecolor{currentstroke}{rgb}{0.000000,0.000000,0.000000}%
\pgfsetstrokecolor{currentstroke}%
\pgfsetdash{}{0pt}%
\pgfsys@defobject{currentmarker}{\pgfqpoint{-0.027778in}{0.000000in}}{\pgfqpoint{0.000000in}{0.000000in}}{%
\pgfpathmoveto{\pgfqpoint{0.000000in}{0.000000in}}%
\pgfpathlineto{\pgfqpoint{-0.027778in}{0.000000in}}%
\pgfusepath{stroke,fill}%
}%
\begin{pgfscope}%
\pgfsys@transformshift{0.625000in}{1.014660in}%
\pgfsys@useobject{currentmarker}{}%
\end{pgfscope}%
\end{pgfscope}%
\begin{pgfscope}%
\pgfsetbuttcap%
\pgfsetroundjoin%
\definecolor{currentfill}{rgb}{0.000000,0.000000,0.000000}%
\pgfsetfillcolor{currentfill}%
\pgfsetlinewidth{0.602250pt}%
\definecolor{currentstroke}{rgb}{0.000000,0.000000,0.000000}%
\pgfsetstrokecolor{currentstroke}%
\pgfsetdash{}{0pt}%
\pgfsys@defobject{currentmarker}{\pgfqpoint{-0.027778in}{0.000000in}}{\pgfqpoint{0.000000in}{0.000000in}}{%
\pgfpathmoveto{\pgfqpoint{0.000000in}{0.000000in}}%
\pgfpathlineto{\pgfqpoint{-0.027778in}{0.000000in}}%
\pgfusepath{stroke,fill}%
}%
\begin{pgfscope}%
\pgfsys@transformshift{0.625000in}{1.079134in}%
\pgfsys@useobject{currentmarker}{}%
\end{pgfscope}%
\end{pgfscope}%
\begin{pgfscope}%
\pgfsetbuttcap%
\pgfsetroundjoin%
\definecolor{currentfill}{rgb}{0.000000,0.000000,0.000000}%
\pgfsetfillcolor{currentfill}%
\pgfsetlinewidth{0.602250pt}%
\definecolor{currentstroke}{rgb}{0.000000,0.000000,0.000000}%
\pgfsetstrokecolor{currentstroke}%
\pgfsetdash{}{0pt}%
\pgfsys@defobject{currentmarker}{\pgfqpoint{-0.027778in}{0.000000in}}{\pgfqpoint{0.000000in}{0.000000in}}{%
\pgfpathmoveto{\pgfqpoint{0.000000in}{0.000000in}}%
\pgfpathlineto{\pgfqpoint{-0.027778in}{0.000000in}}%
\pgfusepath{stroke,fill}%
}%
\begin{pgfscope}%
\pgfsys@transformshift{0.625000in}{1.136004in}%
\pgfsys@useobject{currentmarker}{}%
\end{pgfscope}%
\end{pgfscope}%
\begin{pgfscope}%
\pgfsetbuttcap%
\pgfsetroundjoin%
\definecolor{currentfill}{rgb}{0.000000,0.000000,0.000000}%
\pgfsetfillcolor{currentfill}%
\pgfsetlinewidth{0.602250pt}%
\definecolor{currentstroke}{rgb}{0.000000,0.000000,0.000000}%
\pgfsetstrokecolor{currentstroke}%
\pgfsetdash{}{0pt}%
\pgfsys@defobject{currentmarker}{\pgfqpoint{-0.027778in}{0.000000in}}{\pgfqpoint{0.000000in}{0.000000in}}{%
\pgfpathmoveto{\pgfqpoint{0.000000in}{0.000000in}}%
\pgfpathlineto{\pgfqpoint{-0.027778in}{0.000000in}}%
\pgfusepath{stroke,fill}%
}%
\begin{pgfscope}%
\pgfsys@transformshift{0.625000in}{1.521552in}%
\pgfsys@useobject{currentmarker}{}%
\end{pgfscope}%
\end{pgfscope}%
\begin{pgfscope}%
\pgfsetbuttcap%
\pgfsetroundjoin%
\definecolor{currentfill}{rgb}{0.000000,0.000000,0.000000}%
\pgfsetfillcolor{currentfill}%
\pgfsetlinewidth{0.602250pt}%
\definecolor{currentstroke}{rgb}{0.000000,0.000000,0.000000}%
\pgfsetstrokecolor{currentstroke}%
\pgfsetdash{}{0pt}%
\pgfsys@defobject{currentmarker}{\pgfqpoint{-0.027778in}{0.000000in}}{\pgfqpoint{0.000000in}{0.000000in}}{%
\pgfpathmoveto{\pgfqpoint{0.000000in}{0.000000in}}%
\pgfpathlineto{\pgfqpoint{-0.027778in}{0.000000in}}%
\pgfusepath{stroke,fill}%
}%
\begin{pgfscope}%
\pgfsys@transformshift{0.625000in}{1.717325in}%
\pgfsys@useobject{currentmarker}{}%
\end{pgfscope}%
\end{pgfscope}%
\begin{pgfscope}%
\pgfsetbuttcap%
\pgfsetroundjoin%
\definecolor{currentfill}{rgb}{0.000000,0.000000,0.000000}%
\pgfsetfillcolor{currentfill}%
\pgfsetlinewidth{0.602250pt}%
\definecolor{currentstroke}{rgb}{0.000000,0.000000,0.000000}%
\pgfsetstrokecolor{currentstroke}%
\pgfsetdash{}{0pt}%
\pgfsys@defobject{currentmarker}{\pgfqpoint{-0.027778in}{0.000000in}}{\pgfqpoint{0.000000in}{0.000000in}}{%
\pgfpathmoveto{\pgfqpoint{0.000000in}{0.000000in}}%
\pgfpathlineto{\pgfqpoint{-0.027778in}{0.000000in}}%
\pgfusepath{stroke,fill}%
}%
\begin{pgfscope}%
\pgfsys@transformshift{0.625000in}{1.856228in}%
\pgfsys@useobject{currentmarker}{}%
\end{pgfscope}%
\end{pgfscope}%
\begin{pgfscope}%
\pgfsetbuttcap%
\pgfsetroundjoin%
\definecolor{currentfill}{rgb}{0.000000,0.000000,0.000000}%
\pgfsetfillcolor{currentfill}%
\pgfsetlinewidth{0.602250pt}%
\definecolor{currentstroke}{rgb}{0.000000,0.000000,0.000000}%
\pgfsetstrokecolor{currentstroke}%
\pgfsetdash{}{0pt}%
\pgfsys@defobject{currentmarker}{\pgfqpoint{-0.027778in}{0.000000in}}{\pgfqpoint{0.000000in}{0.000000in}}{%
\pgfpathmoveto{\pgfqpoint{0.000000in}{0.000000in}}%
\pgfpathlineto{\pgfqpoint{-0.027778in}{0.000000in}}%
\pgfusepath{stroke,fill}%
}%
\begin{pgfscope}%
\pgfsys@transformshift{0.625000in}{1.963970in}%
\pgfsys@useobject{currentmarker}{}%
\end{pgfscope}%
\end{pgfscope}%
\begin{pgfscope}%
\pgfsetbuttcap%
\pgfsetroundjoin%
\definecolor{currentfill}{rgb}{0.000000,0.000000,0.000000}%
\pgfsetfillcolor{currentfill}%
\pgfsetlinewidth{0.602250pt}%
\definecolor{currentstroke}{rgb}{0.000000,0.000000,0.000000}%
\pgfsetstrokecolor{currentstroke}%
\pgfsetdash{}{0pt}%
\pgfsys@defobject{currentmarker}{\pgfqpoint{-0.027778in}{0.000000in}}{\pgfqpoint{0.000000in}{0.000000in}}{%
\pgfpathmoveto{\pgfqpoint{0.000000in}{0.000000in}}%
\pgfpathlineto{\pgfqpoint{-0.027778in}{0.000000in}}%
\pgfusepath{stroke,fill}%
}%
\begin{pgfscope}%
\pgfsys@transformshift{0.625000in}{2.052001in}%
\pgfsys@useobject{currentmarker}{}%
\end{pgfscope}%
\end{pgfscope}%
\begin{pgfscope}%
\pgfsetbuttcap%
\pgfsetroundjoin%
\definecolor{currentfill}{rgb}{0.000000,0.000000,0.000000}%
\pgfsetfillcolor{currentfill}%
\pgfsetlinewidth{0.602250pt}%
\definecolor{currentstroke}{rgb}{0.000000,0.000000,0.000000}%
\pgfsetstrokecolor{currentstroke}%
\pgfsetdash{}{0pt}%
\pgfsys@defobject{currentmarker}{\pgfqpoint{-0.027778in}{0.000000in}}{\pgfqpoint{0.000000in}{0.000000in}}{%
\pgfpathmoveto{\pgfqpoint{0.000000in}{0.000000in}}%
\pgfpathlineto{\pgfqpoint{-0.027778in}{0.000000in}}%
\pgfusepath{stroke,fill}%
}%
\begin{pgfscope}%
\pgfsys@transformshift{0.625000in}{2.126430in}%
\pgfsys@useobject{currentmarker}{}%
\end{pgfscope}%
\end{pgfscope}%
\begin{pgfscope}%
\pgfsetbuttcap%
\pgfsetroundjoin%
\definecolor{currentfill}{rgb}{0.000000,0.000000,0.000000}%
\pgfsetfillcolor{currentfill}%
\pgfsetlinewidth{0.602250pt}%
\definecolor{currentstroke}{rgb}{0.000000,0.000000,0.000000}%
\pgfsetstrokecolor{currentstroke}%
\pgfsetdash{}{0pt}%
\pgfsys@defobject{currentmarker}{\pgfqpoint{-0.027778in}{0.000000in}}{\pgfqpoint{0.000000in}{0.000000in}}{%
\pgfpathmoveto{\pgfqpoint{0.000000in}{0.000000in}}%
\pgfpathlineto{\pgfqpoint{-0.027778in}{0.000000in}}%
\pgfusepath{stroke,fill}%
}%
\begin{pgfscope}%
\pgfsys@transformshift{0.625000in}{2.190904in}%
\pgfsys@useobject{currentmarker}{}%
\end{pgfscope}%
\end{pgfscope}%
\begin{pgfscope}%
\pgfsetbuttcap%
\pgfsetroundjoin%
\definecolor{currentfill}{rgb}{0.000000,0.000000,0.000000}%
\pgfsetfillcolor{currentfill}%
\pgfsetlinewidth{0.602250pt}%
\definecolor{currentstroke}{rgb}{0.000000,0.000000,0.000000}%
\pgfsetstrokecolor{currentstroke}%
\pgfsetdash{}{0pt}%
\pgfsys@defobject{currentmarker}{\pgfqpoint{-0.027778in}{0.000000in}}{\pgfqpoint{0.000000in}{0.000000in}}{%
\pgfpathmoveto{\pgfqpoint{0.000000in}{0.000000in}}%
\pgfpathlineto{\pgfqpoint{-0.027778in}{0.000000in}}%
\pgfusepath{stroke,fill}%
}%
\begin{pgfscope}%
\pgfsys@transformshift{0.625000in}{2.247774in}%
\pgfsys@useobject{currentmarker}{}%
\end{pgfscope}%
\end{pgfscope}%
\begin{pgfscope}%
\pgfsetbuttcap%
\pgfsetroundjoin%
\definecolor{currentfill}{rgb}{0.000000,0.000000,0.000000}%
\pgfsetfillcolor{currentfill}%
\pgfsetlinewidth{0.602250pt}%
\definecolor{currentstroke}{rgb}{0.000000,0.000000,0.000000}%
\pgfsetstrokecolor{currentstroke}%
\pgfsetdash{}{0pt}%
\pgfsys@defobject{currentmarker}{\pgfqpoint{-0.027778in}{0.000000in}}{\pgfqpoint{0.000000in}{0.000000in}}{%
\pgfpathmoveto{\pgfqpoint{0.000000in}{0.000000in}}%
\pgfpathlineto{\pgfqpoint{-0.027778in}{0.000000in}}%
\pgfusepath{stroke,fill}%
}%
\begin{pgfscope}%
\pgfsys@transformshift{0.625000in}{2.633322in}%
\pgfsys@useobject{currentmarker}{}%
\end{pgfscope}%
\end{pgfscope}%
\begin{pgfscope}%
\pgfsetbuttcap%
\pgfsetroundjoin%
\definecolor{currentfill}{rgb}{0.000000,0.000000,0.000000}%
\pgfsetfillcolor{currentfill}%
\pgfsetlinewidth{0.602250pt}%
\definecolor{currentstroke}{rgb}{0.000000,0.000000,0.000000}%
\pgfsetstrokecolor{currentstroke}%
\pgfsetdash{}{0pt}%
\pgfsys@defobject{currentmarker}{\pgfqpoint{-0.027778in}{0.000000in}}{\pgfqpoint{0.000000in}{0.000000in}}{%
\pgfpathmoveto{\pgfqpoint{0.000000in}{0.000000in}}%
\pgfpathlineto{\pgfqpoint{-0.027778in}{0.000000in}}%
\pgfusepath{stroke,fill}%
}%
\begin{pgfscope}%
\pgfsys@transformshift{0.625000in}{2.829095in}%
\pgfsys@useobject{currentmarker}{}%
\end{pgfscope}%
\end{pgfscope}%
\begin{pgfscope}%
\pgfsetbuttcap%
\pgfsetroundjoin%
\definecolor{currentfill}{rgb}{0.000000,0.000000,0.000000}%
\pgfsetfillcolor{currentfill}%
\pgfsetlinewidth{0.602250pt}%
\definecolor{currentstroke}{rgb}{0.000000,0.000000,0.000000}%
\pgfsetstrokecolor{currentstroke}%
\pgfsetdash{}{0pt}%
\pgfsys@defobject{currentmarker}{\pgfqpoint{-0.027778in}{0.000000in}}{\pgfqpoint{0.000000in}{0.000000in}}{%
\pgfpathmoveto{\pgfqpoint{0.000000in}{0.000000in}}%
\pgfpathlineto{\pgfqpoint{-0.027778in}{0.000000in}}%
\pgfusepath{stroke,fill}%
}%
\begin{pgfscope}%
\pgfsys@transformshift{0.625000in}{2.967998in}%
\pgfsys@useobject{currentmarker}{}%
\end{pgfscope}%
\end{pgfscope}%
\begin{pgfscope}%
\pgfsetbuttcap%
\pgfsetroundjoin%
\definecolor{currentfill}{rgb}{0.000000,0.000000,0.000000}%
\pgfsetfillcolor{currentfill}%
\pgfsetlinewidth{0.602250pt}%
\definecolor{currentstroke}{rgb}{0.000000,0.000000,0.000000}%
\pgfsetstrokecolor{currentstroke}%
\pgfsetdash{}{0pt}%
\pgfsys@defobject{currentmarker}{\pgfqpoint{-0.027778in}{0.000000in}}{\pgfqpoint{0.000000in}{0.000000in}}{%
\pgfpathmoveto{\pgfqpoint{0.000000in}{0.000000in}}%
\pgfpathlineto{\pgfqpoint{-0.027778in}{0.000000in}}%
\pgfusepath{stroke,fill}%
}%
\begin{pgfscope}%
\pgfsys@transformshift{0.625000in}{3.075740in}%
\pgfsys@useobject{currentmarker}{}%
\end{pgfscope}%
\end{pgfscope}%
\begin{pgfscope}%
\pgfsetbuttcap%
\pgfsetroundjoin%
\definecolor{currentfill}{rgb}{0.000000,0.000000,0.000000}%
\pgfsetfillcolor{currentfill}%
\pgfsetlinewidth{0.602250pt}%
\definecolor{currentstroke}{rgb}{0.000000,0.000000,0.000000}%
\pgfsetstrokecolor{currentstroke}%
\pgfsetdash{}{0pt}%
\pgfsys@defobject{currentmarker}{\pgfqpoint{-0.027778in}{0.000000in}}{\pgfqpoint{0.000000in}{0.000000in}}{%
\pgfpathmoveto{\pgfqpoint{0.000000in}{0.000000in}}%
\pgfpathlineto{\pgfqpoint{-0.027778in}{0.000000in}}%
\pgfusepath{stroke,fill}%
}%
\begin{pgfscope}%
\pgfsys@transformshift{0.625000in}{3.163771in}%
\pgfsys@useobject{currentmarker}{}%
\end{pgfscope}%
\end{pgfscope}%
\begin{pgfscope}%
\pgfsetbuttcap%
\pgfsetroundjoin%
\definecolor{currentfill}{rgb}{0.000000,0.000000,0.000000}%
\pgfsetfillcolor{currentfill}%
\pgfsetlinewidth{0.602250pt}%
\definecolor{currentstroke}{rgb}{0.000000,0.000000,0.000000}%
\pgfsetstrokecolor{currentstroke}%
\pgfsetdash{}{0pt}%
\pgfsys@defobject{currentmarker}{\pgfqpoint{-0.027778in}{0.000000in}}{\pgfqpoint{0.000000in}{0.000000in}}{%
\pgfpathmoveto{\pgfqpoint{0.000000in}{0.000000in}}%
\pgfpathlineto{\pgfqpoint{-0.027778in}{0.000000in}}%
\pgfusepath{stroke,fill}%
}%
\begin{pgfscope}%
\pgfsys@transformshift{0.625000in}{3.238201in}%
\pgfsys@useobject{currentmarker}{}%
\end{pgfscope}%
\end{pgfscope}%
\begin{pgfscope}%
\pgfsetbuttcap%
\pgfsetroundjoin%
\definecolor{currentfill}{rgb}{0.000000,0.000000,0.000000}%
\pgfsetfillcolor{currentfill}%
\pgfsetlinewidth{0.602250pt}%
\definecolor{currentstroke}{rgb}{0.000000,0.000000,0.000000}%
\pgfsetstrokecolor{currentstroke}%
\pgfsetdash{}{0pt}%
\pgfsys@defobject{currentmarker}{\pgfqpoint{-0.027778in}{0.000000in}}{\pgfqpoint{0.000000in}{0.000000in}}{%
\pgfpathmoveto{\pgfqpoint{0.000000in}{0.000000in}}%
\pgfpathlineto{\pgfqpoint{-0.027778in}{0.000000in}}%
\pgfusepath{stroke,fill}%
}%
\begin{pgfscope}%
\pgfsys@transformshift{0.625000in}{3.302675in}%
\pgfsys@useobject{currentmarker}{}%
\end{pgfscope}%
\end{pgfscope}%
\begin{pgfscope}%
\pgfsetbuttcap%
\pgfsetroundjoin%
\definecolor{currentfill}{rgb}{0.000000,0.000000,0.000000}%
\pgfsetfillcolor{currentfill}%
\pgfsetlinewidth{0.602250pt}%
\definecolor{currentstroke}{rgb}{0.000000,0.000000,0.000000}%
\pgfsetstrokecolor{currentstroke}%
\pgfsetdash{}{0pt}%
\pgfsys@defobject{currentmarker}{\pgfqpoint{-0.027778in}{0.000000in}}{\pgfqpoint{0.000000in}{0.000000in}}{%
\pgfpathmoveto{\pgfqpoint{0.000000in}{0.000000in}}%
\pgfpathlineto{\pgfqpoint{-0.027778in}{0.000000in}}%
\pgfusepath{stroke,fill}%
}%
\begin{pgfscope}%
\pgfsys@transformshift{0.625000in}{3.359544in}%
\pgfsys@useobject{currentmarker}{}%
\end{pgfscope}%
\end{pgfscope}%
\begin{pgfscope}%
\definecolor{textcolor}{rgb}{0.000000,0.000000,0.000000}%
\pgfsetstrokecolor{textcolor}%
\pgfsetfillcolor{textcolor}%
\pgftext[x=0.162343in,y=1.980000in,,bottom,rotate=90.000000]{\color{textcolor}\sffamily\fontsize{11.000000}{13.200000}\selectfont \textrm{QMC Error}}%
\end{pgfscope}%
\begin{pgfscope}%
\pgfpathrectangle{\pgfqpoint{0.625000in}{0.440000in}}{\pgfqpoint{3.875000in}{3.080000in}}%
\pgfusepath{clip}%
\pgfsetbuttcap%
\pgfsetroundjoin%
\definecolor{currentfill}{rgb}{0.000000,0.000000,0.000000}%
\pgfsetfillcolor{currentfill}%
\pgfsetlinewidth{1.003750pt}%
\definecolor{currentstroke}{rgb}{0.000000,0.000000,0.000000}%
\pgfsetstrokecolor{currentstroke}%
\pgfsetdash{}{0pt}%
\pgfsys@defobject{currentmarker}{\pgfqpoint{-0.041667in}{-0.041667in}}{\pgfqpoint{0.041667in}{0.041667in}}{%
\pgfpathmoveto{\pgfqpoint{0.000000in}{-0.041667in}}%
\pgfpathcurveto{\pgfqpoint{0.011050in}{-0.041667in}}{\pgfqpoint{0.021649in}{-0.037276in}}{\pgfqpoint{0.029463in}{-0.029463in}}%
\pgfpathcurveto{\pgfqpoint{0.037276in}{-0.021649in}}{\pgfqpoint{0.041667in}{-0.011050in}}{\pgfqpoint{0.041667in}{0.000000in}}%
\pgfpathcurveto{\pgfqpoint{0.041667in}{0.011050in}}{\pgfqpoint{0.037276in}{0.021649in}}{\pgfqpoint{0.029463in}{0.029463in}}%
\pgfpathcurveto{\pgfqpoint{0.021649in}{0.037276in}}{\pgfqpoint{0.011050in}{0.041667in}}{\pgfqpoint{0.000000in}{0.041667in}}%
\pgfpathcurveto{\pgfqpoint{-0.011050in}{0.041667in}}{\pgfqpoint{-0.021649in}{0.037276in}}{\pgfqpoint{-0.029463in}{0.029463in}}%
\pgfpathcurveto{\pgfqpoint{-0.037276in}{0.021649in}}{\pgfqpoint{-0.041667in}{0.011050in}}{\pgfqpoint{-0.041667in}{0.000000in}}%
\pgfpathcurveto{\pgfqpoint{-0.041667in}{-0.011050in}}{\pgfqpoint{-0.037276in}{-0.021649in}}{\pgfqpoint{-0.029463in}{-0.029463in}}%
\pgfpathcurveto{\pgfqpoint{-0.021649in}{-0.037276in}}{\pgfqpoint{-0.011050in}{-0.041667in}}{\pgfqpoint{0.000000in}{-0.041667in}}%
\pgfpathclose%
\pgfusepath{stroke,fill}%
}%
\begin{pgfscope}%
\pgfsys@transformshift{0.801136in}{3.306050in}%
\pgfsys@useobject{currentmarker}{}%
\end{pgfscope}%
\begin{pgfscope}%
\pgfsys@transformshift{1.121384in}{3.096833in}%
\pgfsys@useobject{currentmarker}{}%
\end{pgfscope}%
\begin{pgfscope}%
\pgfsys@transformshift{1.441632in}{2.715208in}%
\pgfsys@useobject{currentmarker}{}%
\end{pgfscope}%
\begin{pgfscope}%
\pgfsys@transformshift{1.761880in}{2.586218in}%
\pgfsys@useobject{currentmarker}{}%
\end{pgfscope}%
\begin{pgfscope}%
\pgfsys@transformshift{2.082128in}{2.352320in}%
\pgfsys@useobject{currentmarker}{}%
\end{pgfscope}%
\begin{pgfscope}%
\pgfsys@transformshift{2.402376in}{2.228567in}%
\pgfsys@useobject{currentmarker}{}%
\end{pgfscope}%
\begin{pgfscope}%
\pgfsys@transformshift{2.722624in}{2.200525in}%
\pgfsys@useobject{currentmarker}{}%
\end{pgfscope}%
\begin{pgfscope}%
\pgfsys@transformshift{3.042872in}{2.181541in}%
\pgfsys@useobject{currentmarker}{}%
\end{pgfscope}%
\begin{pgfscope}%
\pgfsys@transformshift{3.363120in}{1.332946in}%
\pgfsys@useobject{currentmarker}{}%
\end{pgfscope}%
\begin{pgfscope}%
\pgfsys@transformshift{3.683368in}{0.938845in}%
\pgfsys@useobject{currentmarker}{}%
\end{pgfscope}%
\begin{pgfscope}%
\pgfsys@transformshift{4.003616in}{0.697550in}%
\pgfsys@useobject{currentmarker}{}%
\end{pgfscope}%
\begin{pgfscope}%
\pgfsys@transformshift{4.323864in}{0.580000in}%
\pgfsys@useobject{currentmarker}{}%
\end{pgfscope}%
\end{pgfscope}%
\begin{pgfscope}%
\pgfpathrectangle{\pgfqpoint{0.625000in}{0.440000in}}{\pgfqpoint{3.875000in}{3.080000in}}%
\pgfusepath{clip}%
\pgfsetbuttcap%
\pgfsetroundjoin%
\pgfsetlinewidth{1.505625pt}%
\definecolor{currentstroke}{rgb}{0.000000,0.000000,0.000000}%
\pgfsetstrokecolor{currentstroke}%
\pgfsetdash{{5.550000pt}{2.400000pt}}{0.000000pt}%
\pgfpathmoveto{\pgfqpoint{0.801136in}{3.380000in}}%
\pgfpathlineto{\pgfqpoint{4.323864in}{0.656101in}}%
\pgfusepath{stroke}%
\end{pgfscope}%
\begin{pgfscope}%
\pgfsetrectcap%
\pgfsetmiterjoin%
\pgfsetlinewidth{0.803000pt}%
\definecolor{currentstroke}{rgb}{0.000000,0.000000,0.000000}%
\pgfsetstrokecolor{currentstroke}%
\pgfsetdash{}{0pt}%
\pgfpathmoveto{\pgfqpoint{0.625000in}{0.440000in}}%
\pgfpathlineto{\pgfqpoint{0.625000in}{3.520000in}}%
\pgfusepath{stroke}%
\end{pgfscope}%
\begin{pgfscope}%
\pgfsetrectcap%
\pgfsetmiterjoin%
\pgfsetlinewidth{0.000000pt}%
\definecolor{currentstroke}{rgb}{0.000000,0.000000,0.000000}%
\pgfsetstrokecolor{currentstroke}%
\pgfsetstrokeopacity{0.000000}%
\pgfsetdash{}{0pt}%
\pgfpathmoveto{\pgfqpoint{4.500000in}{0.440000in}}%
\pgfpathlineto{\pgfqpoint{4.500000in}{3.520000in}}%
\pgfusepath{}%
\end{pgfscope}%
\begin{pgfscope}%
\pgfsetrectcap%
\pgfsetmiterjoin%
\pgfsetlinewidth{0.803000pt}%
\definecolor{currentstroke}{rgb}{0.000000,0.000000,0.000000}%
\pgfsetstrokecolor{currentstroke}%
\pgfsetdash{}{0pt}%
\pgfpathmoveto{\pgfqpoint{0.625000in}{0.440000in}}%
\pgfpathlineto{\pgfqpoint{4.500000in}{0.440000in}}%
\pgfusepath{stroke}%
\end{pgfscope}%
\begin{pgfscope}%
\pgfsetrectcap%
\pgfsetmiterjoin%
\pgfsetlinewidth{0.000000pt}%
\definecolor{currentstroke}{rgb}{0.000000,0.000000,0.000000}%
\pgfsetstrokecolor{currentstroke}%
\pgfsetstrokeopacity{0.000000}%
\pgfsetdash{}{0pt}%
\pgfpathmoveto{\pgfqpoint{0.625000in}{3.520000in}}%
\pgfpathlineto{\pgfqpoint{4.500000in}{3.520000in}}%
\pgfusepath{}%
\end{pgfscope}%
\begin{pgfscope}%
\pgfsetbuttcap%
\pgfsetmiterjoin%
\definecolor{currentfill}{rgb}{1.000000,1.000000,1.000000}%
\pgfsetfillcolor{currentfill}%
\pgfsetfillopacity{0.800000}%
\pgfsetlinewidth{1.003750pt}%
\definecolor{currentstroke}{rgb}{0.800000,0.800000,0.800000}%
\pgfsetstrokecolor{currentstroke}%
\pgfsetstrokeopacity{0.800000}%
\pgfsetdash{}{0pt}%
\pgfpathmoveto{\pgfqpoint{2.916233in}{3.119488in}}%
\pgfpathlineto{\pgfqpoint{4.393056in}{3.119488in}}%
\pgfpathquadraticcurveto{\pgfqpoint{4.423611in}{3.119488in}}{\pgfqpoint{4.423611in}{3.150043in}}%
\pgfpathlineto{\pgfqpoint{4.423611in}{3.413056in}}%
\pgfpathquadraticcurveto{\pgfqpoint{4.423611in}{3.443611in}}{\pgfqpoint{4.393056in}{3.443611in}}%
\pgfpathlineto{\pgfqpoint{2.916233in}{3.443611in}}%
\pgfpathquadraticcurveto{\pgfqpoint{2.885678in}{3.443611in}}{\pgfqpoint{2.885678in}{3.413056in}}%
\pgfpathlineto{\pgfqpoint{2.885678in}{3.150043in}}%
\pgfpathquadraticcurveto{\pgfqpoint{2.885678in}{3.119488in}}{\pgfqpoint{2.916233in}{3.119488in}}%
\pgfpathclose%
\pgfusepath{stroke,fill}%
\end{pgfscope}%
\begin{pgfscope}%
\pgfsetbuttcap%
\pgfsetroundjoin%
\pgfsetlinewidth{1.505625pt}%
\definecolor{currentstroke}{rgb}{0.000000,0.000000,0.000000}%
\pgfsetstrokecolor{currentstroke}%
\pgfsetdash{{5.550000pt}{2.400000pt}}{0.000000pt}%
\pgfpathmoveto{\pgfqpoint{2.946789in}{3.300408in}}%
\pgfpathlineto{\pgfqpoint{3.252344in}{3.300408in}}%
\pgfusepath{stroke}%
\end{pgfscope}%
\begin{pgfscope}%
\definecolor{textcolor}{rgb}{0.000000,0.000000,0.000000}%
\pgfsetstrokecolor{textcolor}%
\pgfsetfillcolor{textcolor}%
\pgftext[x=3.374567in,y=3.246936in,left,base]{\color{textcolor}\sffamily\fontsize{11.000000}{13.200000}\selectfont \(\displaystyle 9.3894N_{\mathrm{QMC}}^{-0.7399}\)}%
\end{pgfscope}%
\end{pgfpicture}%
\makeatother%
\endgroup%

    \caption[Quasi-Monte-Carlo error, for $Q=\grad u((1,1))$ and $k=50$.]{Quasi-Monte-Carlo error with increasing number of Quasi-Monte-Carlo points, for $Q=\grad u((1,1))$ and $k=50$. \label{fig:qmcgradient_top_right50}}
  \end{figure}
\begin{figure}[h]
  \centering
%% Creator: Matplotlib, PGF backend
%%
%% To include the figure in your LaTeX document, write
%%   \input{<filename>.pgf}
%%
%% Make sure the required packages are loaded in your preamble
%%   \usepackage{pgf}
%%
%% Figures using additional raster images can only be included by \input if
%% they are in the same directory as the main LaTeX file. For loading figures
%% from other directories you can use the `import` package
%%   \usepackage{import}
%% and then include the figures with
%%   \import{<path to file>}{<filename>.pgf}
%%
%% Matplotlib used the following preamble
%%   \usepackage{fontspec}
%%   \setmainfont{DejaVuSerif.ttf}[Path=/home/owen/progs/firedrake-complex/firedrake/lib/python3.5/site-packages/matplotlib/mpl-data/fonts/ttf/]
%%   \setsansfont{DejaVuSans.ttf}[Path=/home/owen/progs/firedrake-complex/firedrake/lib/python3.5/site-packages/matplotlib/mpl-data/fonts/ttf/]
%%   \setmonofont{DejaVuSansMono.ttf}[Path=/home/owen/progs/firedrake-complex/firedrake/lib/python3.5/site-packages/matplotlib/mpl-data/fonts/ttf/]
%%
\begingroup%
\makeatletter%
\begin{pgfpicture}%
\pgfpathrectangle{\pgfpointorigin}{\pgfqpoint{5.000000in}{5.000000in}}%
\pgfusepath{use as bounding box, clip}%
\begin{pgfscope}%
\pgfsetbuttcap%
\pgfsetmiterjoin%
\definecolor{currentfill}{rgb}{1.000000,1.000000,1.000000}%
\pgfsetfillcolor{currentfill}%
\pgfsetlinewidth{0.000000pt}%
\definecolor{currentstroke}{rgb}{1.000000,1.000000,1.000000}%
\pgfsetstrokecolor{currentstroke}%
\pgfsetdash{}{0pt}%
\pgfpathmoveto{\pgfqpoint{0.000000in}{0.000000in}}%
\pgfpathlineto{\pgfqpoint{5.000000in}{0.000000in}}%
\pgfpathlineto{\pgfqpoint{5.000000in}{5.000000in}}%
\pgfpathlineto{\pgfqpoint{0.000000in}{5.000000in}}%
\pgfpathclose%
\pgfusepath{fill}%
\end{pgfscope}%
\begin{pgfscope}%
\pgfsetbuttcap%
\pgfsetmiterjoin%
\definecolor{currentfill}{rgb}{1.000000,1.000000,1.000000}%
\pgfsetfillcolor{currentfill}%
\pgfsetlinewidth{0.000000pt}%
\definecolor{currentstroke}{rgb}{0.000000,0.000000,0.000000}%
\pgfsetstrokecolor{currentstroke}%
\pgfsetstrokeopacity{0.000000}%
\pgfsetdash{}{0pt}%
\pgfpathmoveto{\pgfqpoint{0.625000in}{0.550000in}}%
\pgfpathlineto{\pgfqpoint{4.500000in}{0.550000in}}%
\pgfpathlineto{\pgfqpoint{4.500000in}{4.400000in}}%
\pgfpathlineto{\pgfqpoint{0.625000in}{4.400000in}}%
\pgfpathclose%
\pgfusepath{fill}%
\end{pgfscope}%
\begin{pgfscope}%
\pgfsetbuttcap%
\pgfsetroundjoin%
\definecolor{currentfill}{rgb}{0.000000,0.000000,0.000000}%
\pgfsetfillcolor{currentfill}%
\pgfsetlinewidth{0.803000pt}%
\definecolor{currentstroke}{rgb}{0.000000,0.000000,0.000000}%
\pgfsetstrokecolor{currentstroke}%
\pgfsetdash{}{0pt}%
\pgfsys@defobject{currentmarker}{\pgfqpoint{0.000000in}{-0.048611in}}{\pgfqpoint{0.000000in}{0.000000in}}{%
\pgfpathmoveto{\pgfqpoint{0.000000in}{0.000000in}}%
\pgfpathlineto{\pgfqpoint{0.000000in}{-0.048611in}}%
\pgfusepath{stroke,fill}%
}%
\begin{pgfscope}%
\pgfsys@transformshift{0.801136in}{0.550000in}%
\pgfsys@useobject{currentmarker}{}%
\end{pgfscope}%
\end{pgfscope}%
\begin{pgfscope}%
\definecolor{textcolor}{rgb}{0.000000,0.000000,0.000000}%
\pgfsetstrokecolor{textcolor}%
\pgfsetfillcolor{textcolor}%
\pgftext[x=0.801136in,y=0.452778in,,top]{\color{textcolor}\sffamily\fontsize{11.000000}{13.200000}\selectfont \(\displaystyle 10^{0}\)}%
\end{pgfscope}%
\begin{pgfscope}%
\pgfsetbuttcap%
\pgfsetroundjoin%
\definecolor{currentfill}{rgb}{0.000000,0.000000,0.000000}%
\pgfsetfillcolor{currentfill}%
\pgfsetlinewidth{0.803000pt}%
\definecolor{currentstroke}{rgb}{0.000000,0.000000,0.000000}%
\pgfsetstrokecolor{currentstroke}%
\pgfsetdash{}{0pt}%
\pgfsys@defobject{currentmarker}{\pgfqpoint{0.000000in}{-0.048611in}}{\pgfqpoint{0.000000in}{0.000000in}}{%
\pgfpathmoveto{\pgfqpoint{0.000000in}{0.000000in}}%
\pgfpathlineto{\pgfqpoint{0.000000in}{-0.048611in}}%
\pgfusepath{stroke,fill}%
}%
\begin{pgfscope}%
\pgfsys@transformshift{1.864977in}{0.550000in}%
\pgfsys@useobject{currentmarker}{}%
\end{pgfscope}%
\end{pgfscope}%
\begin{pgfscope}%
\definecolor{textcolor}{rgb}{0.000000,0.000000,0.000000}%
\pgfsetstrokecolor{textcolor}%
\pgfsetfillcolor{textcolor}%
\pgftext[x=1.864977in,y=0.452778in,,top]{\color{textcolor}\sffamily\fontsize{11.000000}{13.200000}\selectfont \(\displaystyle 10^{1}\)}%
\end{pgfscope}%
\begin{pgfscope}%
\pgfsetbuttcap%
\pgfsetroundjoin%
\definecolor{currentfill}{rgb}{0.000000,0.000000,0.000000}%
\pgfsetfillcolor{currentfill}%
\pgfsetlinewidth{0.803000pt}%
\definecolor{currentstroke}{rgb}{0.000000,0.000000,0.000000}%
\pgfsetstrokecolor{currentstroke}%
\pgfsetdash{}{0pt}%
\pgfsys@defobject{currentmarker}{\pgfqpoint{0.000000in}{-0.048611in}}{\pgfqpoint{0.000000in}{0.000000in}}{%
\pgfpathmoveto{\pgfqpoint{0.000000in}{0.000000in}}%
\pgfpathlineto{\pgfqpoint{0.000000in}{-0.048611in}}%
\pgfusepath{stroke,fill}%
}%
\begin{pgfscope}%
\pgfsys@transformshift{2.928818in}{0.550000in}%
\pgfsys@useobject{currentmarker}{}%
\end{pgfscope}%
\end{pgfscope}%
\begin{pgfscope}%
\definecolor{textcolor}{rgb}{0.000000,0.000000,0.000000}%
\pgfsetstrokecolor{textcolor}%
\pgfsetfillcolor{textcolor}%
\pgftext[x=2.928818in,y=0.452778in,,top]{\color{textcolor}\sffamily\fontsize{11.000000}{13.200000}\selectfont \(\displaystyle 10^{2}\)}%
\end{pgfscope}%
\begin{pgfscope}%
\pgfsetbuttcap%
\pgfsetroundjoin%
\definecolor{currentfill}{rgb}{0.000000,0.000000,0.000000}%
\pgfsetfillcolor{currentfill}%
\pgfsetlinewidth{0.803000pt}%
\definecolor{currentstroke}{rgb}{0.000000,0.000000,0.000000}%
\pgfsetstrokecolor{currentstroke}%
\pgfsetdash{}{0pt}%
\pgfsys@defobject{currentmarker}{\pgfqpoint{0.000000in}{-0.048611in}}{\pgfqpoint{0.000000in}{0.000000in}}{%
\pgfpathmoveto{\pgfqpoint{0.000000in}{0.000000in}}%
\pgfpathlineto{\pgfqpoint{0.000000in}{-0.048611in}}%
\pgfusepath{stroke,fill}%
}%
\begin{pgfscope}%
\pgfsys@transformshift{3.992658in}{0.550000in}%
\pgfsys@useobject{currentmarker}{}%
\end{pgfscope}%
\end{pgfscope}%
\begin{pgfscope}%
\definecolor{textcolor}{rgb}{0.000000,0.000000,0.000000}%
\pgfsetstrokecolor{textcolor}%
\pgfsetfillcolor{textcolor}%
\pgftext[x=3.992658in,y=0.452778in,,top]{\color{textcolor}\sffamily\fontsize{11.000000}{13.200000}\selectfont \(\displaystyle 10^{3}\)}%
\end{pgfscope}%
\begin{pgfscope}%
\pgfsetbuttcap%
\pgfsetroundjoin%
\definecolor{currentfill}{rgb}{0.000000,0.000000,0.000000}%
\pgfsetfillcolor{currentfill}%
\pgfsetlinewidth{0.602250pt}%
\definecolor{currentstroke}{rgb}{0.000000,0.000000,0.000000}%
\pgfsetstrokecolor{currentstroke}%
\pgfsetdash{}{0pt}%
\pgfsys@defobject{currentmarker}{\pgfqpoint{0.000000in}{-0.027778in}}{\pgfqpoint{0.000000in}{0.000000in}}{%
\pgfpathmoveto{\pgfqpoint{0.000000in}{0.000000in}}%
\pgfpathlineto{\pgfqpoint{0.000000in}{-0.027778in}}%
\pgfusepath{stroke,fill}%
}%
\begin{pgfscope}%
\pgfsys@transformshift{0.636345in}{0.550000in}%
\pgfsys@useobject{currentmarker}{}%
\end{pgfscope}%
\end{pgfscope}%
\begin{pgfscope}%
\pgfsetbuttcap%
\pgfsetroundjoin%
\definecolor{currentfill}{rgb}{0.000000,0.000000,0.000000}%
\pgfsetfillcolor{currentfill}%
\pgfsetlinewidth{0.602250pt}%
\definecolor{currentstroke}{rgb}{0.000000,0.000000,0.000000}%
\pgfsetstrokecolor{currentstroke}%
\pgfsetdash{}{0pt}%
\pgfsys@defobject{currentmarker}{\pgfqpoint{0.000000in}{-0.027778in}}{\pgfqpoint{0.000000in}{0.000000in}}{%
\pgfpathmoveto{\pgfqpoint{0.000000in}{0.000000in}}%
\pgfpathlineto{\pgfqpoint{0.000000in}{-0.027778in}}%
\pgfusepath{stroke,fill}%
}%
\begin{pgfscope}%
\pgfsys@transformshift{0.698040in}{0.550000in}%
\pgfsys@useobject{currentmarker}{}%
\end{pgfscope}%
\end{pgfscope}%
\begin{pgfscope}%
\pgfsetbuttcap%
\pgfsetroundjoin%
\definecolor{currentfill}{rgb}{0.000000,0.000000,0.000000}%
\pgfsetfillcolor{currentfill}%
\pgfsetlinewidth{0.602250pt}%
\definecolor{currentstroke}{rgb}{0.000000,0.000000,0.000000}%
\pgfsetstrokecolor{currentstroke}%
\pgfsetdash{}{0pt}%
\pgfsys@defobject{currentmarker}{\pgfqpoint{0.000000in}{-0.027778in}}{\pgfqpoint{0.000000in}{0.000000in}}{%
\pgfpathmoveto{\pgfqpoint{0.000000in}{0.000000in}}%
\pgfpathlineto{\pgfqpoint{0.000000in}{-0.027778in}}%
\pgfusepath{stroke,fill}%
}%
\begin{pgfscope}%
\pgfsys@transformshift{0.752458in}{0.550000in}%
\pgfsys@useobject{currentmarker}{}%
\end{pgfscope}%
\end{pgfscope}%
\begin{pgfscope}%
\pgfsetbuttcap%
\pgfsetroundjoin%
\definecolor{currentfill}{rgb}{0.000000,0.000000,0.000000}%
\pgfsetfillcolor{currentfill}%
\pgfsetlinewidth{0.602250pt}%
\definecolor{currentstroke}{rgb}{0.000000,0.000000,0.000000}%
\pgfsetstrokecolor{currentstroke}%
\pgfsetdash{}{0pt}%
\pgfsys@defobject{currentmarker}{\pgfqpoint{0.000000in}{-0.027778in}}{\pgfqpoint{0.000000in}{0.000000in}}{%
\pgfpathmoveto{\pgfqpoint{0.000000in}{0.000000in}}%
\pgfpathlineto{\pgfqpoint{0.000000in}{-0.027778in}}%
\pgfusepath{stroke,fill}%
}%
\begin{pgfscope}%
\pgfsys@transformshift{1.121384in}{0.550000in}%
\pgfsys@useobject{currentmarker}{}%
\end{pgfscope}%
\end{pgfscope}%
\begin{pgfscope}%
\pgfsetbuttcap%
\pgfsetroundjoin%
\definecolor{currentfill}{rgb}{0.000000,0.000000,0.000000}%
\pgfsetfillcolor{currentfill}%
\pgfsetlinewidth{0.602250pt}%
\definecolor{currentstroke}{rgb}{0.000000,0.000000,0.000000}%
\pgfsetstrokecolor{currentstroke}%
\pgfsetdash{}{0pt}%
\pgfsys@defobject{currentmarker}{\pgfqpoint{0.000000in}{-0.027778in}}{\pgfqpoint{0.000000in}{0.000000in}}{%
\pgfpathmoveto{\pgfqpoint{0.000000in}{0.000000in}}%
\pgfpathlineto{\pgfqpoint{0.000000in}{-0.027778in}}%
\pgfusepath{stroke,fill}%
}%
\begin{pgfscope}%
\pgfsys@transformshift{1.308717in}{0.550000in}%
\pgfsys@useobject{currentmarker}{}%
\end{pgfscope}%
\end{pgfscope}%
\begin{pgfscope}%
\pgfsetbuttcap%
\pgfsetroundjoin%
\definecolor{currentfill}{rgb}{0.000000,0.000000,0.000000}%
\pgfsetfillcolor{currentfill}%
\pgfsetlinewidth{0.602250pt}%
\definecolor{currentstroke}{rgb}{0.000000,0.000000,0.000000}%
\pgfsetstrokecolor{currentstroke}%
\pgfsetdash{}{0pt}%
\pgfsys@defobject{currentmarker}{\pgfqpoint{0.000000in}{-0.027778in}}{\pgfqpoint{0.000000in}{0.000000in}}{%
\pgfpathmoveto{\pgfqpoint{0.000000in}{0.000000in}}%
\pgfpathlineto{\pgfqpoint{0.000000in}{-0.027778in}}%
\pgfusepath{stroke,fill}%
}%
\begin{pgfscope}%
\pgfsys@transformshift{1.441632in}{0.550000in}%
\pgfsys@useobject{currentmarker}{}%
\end{pgfscope}%
\end{pgfscope}%
\begin{pgfscope}%
\pgfsetbuttcap%
\pgfsetroundjoin%
\definecolor{currentfill}{rgb}{0.000000,0.000000,0.000000}%
\pgfsetfillcolor{currentfill}%
\pgfsetlinewidth{0.602250pt}%
\definecolor{currentstroke}{rgb}{0.000000,0.000000,0.000000}%
\pgfsetstrokecolor{currentstroke}%
\pgfsetdash{}{0pt}%
\pgfsys@defobject{currentmarker}{\pgfqpoint{0.000000in}{-0.027778in}}{\pgfqpoint{0.000000in}{0.000000in}}{%
\pgfpathmoveto{\pgfqpoint{0.000000in}{0.000000in}}%
\pgfpathlineto{\pgfqpoint{0.000000in}{-0.027778in}}%
\pgfusepath{stroke,fill}%
}%
\begin{pgfscope}%
\pgfsys@transformshift{1.544729in}{0.550000in}%
\pgfsys@useobject{currentmarker}{}%
\end{pgfscope}%
\end{pgfscope}%
\begin{pgfscope}%
\pgfsetbuttcap%
\pgfsetroundjoin%
\definecolor{currentfill}{rgb}{0.000000,0.000000,0.000000}%
\pgfsetfillcolor{currentfill}%
\pgfsetlinewidth{0.602250pt}%
\definecolor{currentstroke}{rgb}{0.000000,0.000000,0.000000}%
\pgfsetstrokecolor{currentstroke}%
\pgfsetdash{}{0pt}%
\pgfsys@defobject{currentmarker}{\pgfqpoint{0.000000in}{-0.027778in}}{\pgfqpoint{0.000000in}{0.000000in}}{%
\pgfpathmoveto{\pgfqpoint{0.000000in}{0.000000in}}%
\pgfpathlineto{\pgfqpoint{0.000000in}{-0.027778in}}%
\pgfusepath{stroke,fill}%
}%
\begin{pgfscope}%
\pgfsys@transformshift{1.628965in}{0.550000in}%
\pgfsys@useobject{currentmarker}{}%
\end{pgfscope}%
\end{pgfscope}%
\begin{pgfscope}%
\pgfsetbuttcap%
\pgfsetroundjoin%
\definecolor{currentfill}{rgb}{0.000000,0.000000,0.000000}%
\pgfsetfillcolor{currentfill}%
\pgfsetlinewidth{0.602250pt}%
\definecolor{currentstroke}{rgb}{0.000000,0.000000,0.000000}%
\pgfsetstrokecolor{currentstroke}%
\pgfsetdash{}{0pt}%
\pgfsys@defobject{currentmarker}{\pgfqpoint{0.000000in}{-0.027778in}}{\pgfqpoint{0.000000in}{0.000000in}}{%
\pgfpathmoveto{\pgfqpoint{0.000000in}{0.000000in}}%
\pgfpathlineto{\pgfqpoint{0.000000in}{-0.027778in}}%
\pgfusepath{stroke,fill}%
}%
\begin{pgfscope}%
\pgfsys@transformshift{1.700186in}{0.550000in}%
\pgfsys@useobject{currentmarker}{}%
\end{pgfscope}%
\end{pgfscope}%
\begin{pgfscope}%
\pgfsetbuttcap%
\pgfsetroundjoin%
\definecolor{currentfill}{rgb}{0.000000,0.000000,0.000000}%
\pgfsetfillcolor{currentfill}%
\pgfsetlinewidth{0.602250pt}%
\definecolor{currentstroke}{rgb}{0.000000,0.000000,0.000000}%
\pgfsetstrokecolor{currentstroke}%
\pgfsetdash{}{0pt}%
\pgfsys@defobject{currentmarker}{\pgfqpoint{0.000000in}{-0.027778in}}{\pgfqpoint{0.000000in}{0.000000in}}{%
\pgfpathmoveto{\pgfqpoint{0.000000in}{0.000000in}}%
\pgfpathlineto{\pgfqpoint{0.000000in}{-0.027778in}}%
\pgfusepath{stroke,fill}%
}%
\begin{pgfscope}%
\pgfsys@transformshift{1.761880in}{0.550000in}%
\pgfsys@useobject{currentmarker}{}%
\end{pgfscope}%
\end{pgfscope}%
\begin{pgfscope}%
\pgfsetbuttcap%
\pgfsetroundjoin%
\definecolor{currentfill}{rgb}{0.000000,0.000000,0.000000}%
\pgfsetfillcolor{currentfill}%
\pgfsetlinewidth{0.602250pt}%
\definecolor{currentstroke}{rgb}{0.000000,0.000000,0.000000}%
\pgfsetstrokecolor{currentstroke}%
\pgfsetdash{}{0pt}%
\pgfsys@defobject{currentmarker}{\pgfqpoint{0.000000in}{-0.027778in}}{\pgfqpoint{0.000000in}{0.000000in}}{%
\pgfpathmoveto{\pgfqpoint{0.000000in}{0.000000in}}%
\pgfpathlineto{\pgfqpoint{0.000000in}{-0.027778in}}%
\pgfusepath{stroke,fill}%
}%
\begin{pgfscope}%
\pgfsys@transformshift{1.816298in}{0.550000in}%
\pgfsys@useobject{currentmarker}{}%
\end{pgfscope}%
\end{pgfscope}%
\begin{pgfscope}%
\pgfsetbuttcap%
\pgfsetroundjoin%
\definecolor{currentfill}{rgb}{0.000000,0.000000,0.000000}%
\pgfsetfillcolor{currentfill}%
\pgfsetlinewidth{0.602250pt}%
\definecolor{currentstroke}{rgb}{0.000000,0.000000,0.000000}%
\pgfsetstrokecolor{currentstroke}%
\pgfsetdash{}{0pt}%
\pgfsys@defobject{currentmarker}{\pgfqpoint{0.000000in}{-0.027778in}}{\pgfqpoint{0.000000in}{0.000000in}}{%
\pgfpathmoveto{\pgfqpoint{0.000000in}{0.000000in}}%
\pgfpathlineto{\pgfqpoint{0.000000in}{-0.027778in}}%
\pgfusepath{stroke,fill}%
}%
\begin{pgfscope}%
\pgfsys@transformshift{2.185225in}{0.550000in}%
\pgfsys@useobject{currentmarker}{}%
\end{pgfscope}%
\end{pgfscope}%
\begin{pgfscope}%
\pgfsetbuttcap%
\pgfsetroundjoin%
\definecolor{currentfill}{rgb}{0.000000,0.000000,0.000000}%
\pgfsetfillcolor{currentfill}%
\pgfsetlinewidth{0.602250pt}%
\definecolor{currentstroke}{rgb}{0.000000,0.000000,0.000000}%
\pgfsetstrokecolor{currentstroke}%
\pgfsetdash{}{0pt}%
\pgfsys@defobject{currentmarker}{\pgfqpoint{0.000000in}{-0.027778in}}{\pgfqpoint{0.000000in}{0.000000in}}{%
\pgfpathmoveto{\pgfqpoint{0.000000in}{0.000000in}}%
\pgfpathlineto{\pgfqpoint{0.000000in}{-0.027778in}}%
\pgfusepath{stroke,fill}%
}%
\begin{pgfscope}%
\pgfsys@transformshift{2.372558in}{0.550000in}%
\pgfsys@useobject{currentmarker}{}%
\end{pgfscope}%
\end{pgfscope}%
\begin{pgfscope}%
\pgfsetbuttcap%
\pgfsetroundjoin%
\definecolor{currentfill}{rgb}{0.000000,0.000000,0.000000}%
\pgfsetfillcolor{currentfill}%
\pgfsetlinewidth{0.602250pt}%
\definecolor{currentstroke}{rgb}{0.000000,0.000000,0.000000}%
\pgfsetstrokecolor{currentstroke}%
\pgfsetdash{}{0pt}%
\pgfsys@defobject{currentmarker}{\pgfqpoint{0.000000in}{-0.027778in}}{\pgfqpoint{0.000000in}{0.000000in}}{%
\pgfpathmoveto{\pgfqpoint{0.000000in}{0.000000in}}%
\pgfpathlineto{\pgfqpoint{0.000000in}{-0.027778in}}%
\pgfusepath{stroke,fill}%
}%
\begin{pgfscope}%
\pgfsys@transformshift{2.505473in}{0.550000in}%
\pgfsys@useobject{currentmarker}{}%
\end{pgfscope}%
\end{pgfscope}%
\begin{pgfscope}%
\pgfsetbuttcap%
\pgfsetroundjoin%
\definecolor{currentfill}{rgb}{0.000000,0.000000,0.000000}%
\pgfsetfillcolor{currentfill}%
\pgfsetlinewidth{0.602250pt}%
\definecolor{currentstroke}{rgb}{0.000000,0.000000,0.000000}%
\pgfsetstrokecolor{currentstroke}%
\pgfsetdash{}{0pt}%
\pgfsys@defobject{currentmarker}{\pgfqpoint{0.000000in}{-0.027778in}}{\pgfqpoint{0.000000in}{0.000000in}}{%
\pgfpathmoveto{\pgfqpoint{0.000000in}{0.000000in}}%
\pgfpathlineto{\pgfqpoint{0.000000in}{-0.027778in}}%
\pgfusepath{stroke,fill}%
}%
\begin{pgfscope}%
\pgfsys@transformshift{2.608570in}{0.550000in}%
\pgfsys@useobject{currentmarker}{}%
\end{pgfscope}%
\end{pgfscope}%
\begin{pgfscope}%
\pgfsetbuttcap%
\pgfsetroundjoin%
\definecolor{currentfill}{rgb}{0.000000,0.000000,0.000000}%
\pgfsetfillcolor{currentfill}%
\pgfsetlinewidth{0.602250pt}%
\definecolor{currentstroke}{rgb}{0.000000,0.000000,0.000000}%
\pgfsetstrokecolor{currentstroke}%
\pgfsetdash{}{0pt}%
\pgfsys@defobject{currentmarker}{\pgfqpoint{0.000000in}{-0.027778in}}{\pgfqpoint{0.000000in}{0.000000in}}{%
\pgfpathmoveto{\pgfqpoint{0.000000in}{0.000000in}}%
\pgfpathlineto{\pgfqpoint{0.000000in}{-0.027778in}}%
\pgfusepath{stroke,fill}%
}%
\begin{pgfscope}%
\pgfsys@transformshift{2.692806in}{0.550000in}%
\pgfsys@useobject{currentmarker}{}%
\end{pgfscope}%
\end{pgfscope}%
\begin{pgfscope}%
\pgfsetbuttcap%
\pgfsetroundjoin%
\definecolor{currentfill}{rgb}{0.000000,0.000000,0.000000}%
\pgfsetfillcolor{currentfill}%
\pgfsetlinewidth{0.602250pt}%
\definecolor{currentstroke}{rgb}{0.000000,0.000000,0.000000}%
\pgfsetstrokecolor{currentstroke}%
\pgfsetdash{}{0pt}%
\pgfsys@defobject{currentmarker}{\pgfqpoint{0.000000in}{-0.027778in}}{\pgfqpoint{0.000000in}{0.000000in}}{%
\pgfpathmoveto{\pgfqpoint{0.000000in}{0.000000in}}%
\pgfpathlineto{\pgfqpoint{0.000000in}{-0.027778in}}%
\pgfusepath{stroke,fill}%
}%
\begin{pgfscope}%
\pgfsys@transformshift{2.764027in}{0.550000in}%
\pgfsys@useobject{currentmarker}{}%
\end{pgfscope}%
\end{pgfscope}%
\begin{pgfscope}%
\pgfsetbuttcap%
\pgfsetroundjoin%
\definecolor{currentfill}{rgb}{0.000000,0.000000,0.000000}%
\pgfsetfillcolor{currentfill}%
\pgfsetlinewidth{0.602250pt}%
\definecolor{currentstroke}{rgb}{0.000000,0.000000,0.000000}%
\pgfsetstrokecolor{currentstroke}%
\pgfsetdash{}{0pt}%
\pgfsys@defobject{currentmarker}{\pgfqpoint{0.000000in}{-0.027778in}}{\pgfqpoint{0.000000in}{0.000000in}}{%
\pgfpathmoveto{\pgfqpoint{0.000000in}{0.000000in}}%
\pgfpathlineto{\pgfqpoint{0.000000in}{-0.027778in}}%
\pgfusepath{stroke,fill}%
}%
\begin{pgfscope}%
\pgfsys@transformshift{2.825721in}{0.550000in}%
\pgfsys@useobject{currentmarker}{}%
\end{pgfscope}%
\end{pgfscope}%
\begin{pgfscope}%
\pgfsetbuttcap%
\pgfsetroundjoin%
\definecolor{currentfill}{rgb}{0.000000,0.000000,0.000000}%
\pgfsetfillcolor{currentfill}%
\pgfsetlinewidth{0.602250pt}%
\definecolor{currentstroke}{rgb}{0.000000,0.000000,0.000000}%
\pgfsetstrokecolor{currentstroke}%
\pgfsetdash{}{0pt}%
\pgfsys@defobject{currentmarker}{\pgfqpoint{0.000000in}{-0.027778in}}{\pgfqpoint{0.000000in}{0.000000in}}{%
\pgfpathmoveto{\pgfqpoint{0.000000in}{0.000000in}}%
\pgfpathlineto{\pgfqpoint{0.000000in}{-0.027778in}}%
\pgfusepath{stroke,fill}%
}%
\begin{pgfscope}%
\pgfsys@transformshift{2.880139in}{0.550000in}%
\pgfsys@useobject{currentmarker}{}%
\end{pgfscope}%
\end{pgfscope}%
\begin{pgfscope}%
\pgfsetbuttcap%
\pgfsetroundjoin%
\definecolor{currentfill}{rgb}{0.000000,0.000000,0.000000}%
\pgfsetfillcolor{currentfill}%
\pgfsetlinewidth{0.602250pt}%
\definecolor{currentstroke}{rgb}{0.000000,0.000000,0.000000}%
\pgfsetstrokecolor{currentstroke}%
\pgfsetdash{}{0pt}%
\pgfsys@defobject{currentmarker}{\pgfqpoint{0.000000in}{-0.027778in}}{\pgfqpoint{0.000000in}{0.000000in}}{%
\pgfpathmoveto{\pgfqpoint{0.000000in}{0.000000in}}%
\pgfpathlineto{\pgfqpoint{0.000000in}{-0.027778in}}%
\pgfusepath{stroke,fill}%
}%
\begin{pgfscope}%
\pgfsys@transformshift{3.249066in}{0.550000in}%
\pgfsys@useobject{currentmarker}{}%
\end{pgfscope}%
\end{pgfscope}%
\begin{pgfscope}%
\pgfsetbuttcap%
\pgfsetroundjoin%
\definecolor{currentfill}{rgb}{0.000000,0.000000,0.000000}%
\pgfsetfillcolor{currentfill}%
\pgfsetlinewidth{0.602250pt}%
\definecolor{currentstroke}{rgb}{0.000000,0.000000,0.000000}%
\pgfsetstrokecolor{currentstroke}%
\pgfsetdash{}{0pt}%
\pgfsys@defobject{currentmarker}{\pgfqpoint{0.000000in}{-0.027778in}}{\pgfqpoint{0.000000in}{0.000000in}}{%
\pgfpathmoveto{\pgfqpoint{0.000000in}{0.000000in}}%
\pgfpathlineto{\pgfqpoint{0.000000in}{-0.027778in}}%
\pgfusepath{stroke,fill}%
}%
\begin{pgfscope}%
\pgfsys@transformshift{3.436399in}{0.550000in}%
\pgfsys@useobject{currentmarker}{}%
\end{pgfscope}%
\end{pgfscope}%
\begin{pgfscope}%
\pgfsetbuttcap%
\pgfsetroundjoin%
\definecolor{currentfill}{rgb}{0.000000,0.000000,0.000000}%
\pgfsetfillcolor{currentfill}%
\pgfsetlinewidth{0.602250pt}%
\definecolor{currentstroke}{rgb}{0.000000,0.000000,0.000000}%
\pgfsetstrokecolor{currentstroke}%
\pgfsetdash{}{0pt}%
\pgfsys@defobject{currentmarker}{\pgfqpoint{0.000000in}{-0.027778in}}{\pgfqpoint{0.000000in}{0.000000in}}{%
\pgfpathmoveto{\pgfqpoint{0.000000in}{0.000000in}}%
\pgfpathlineto{\pgfqpoint{0.000000in}{-0.027778in}}%
\pgfusepath{stroke,fill}%
}%
\begin{pgfscope}%
\pgfsys@transformshift{3.569313in}{0.550000in}%
\pgfsys@useobject{currentmarker}{}%
\end{pgfscope}%
\end{pgfscope}%
\begin{pgfscope}%
\pgfsetbuttcap%
\pgfsetroundjoin%
\definecolor{currentfill}{rgb}{0.000000,0.000000,0.000000}%
\pgfsetfillcolor{currentfill}%
\pgfsetlinewidth{0.602250pt}%
\definecolor{currentstroke}{rgb}{0.000000,0.000000,0.000000}%
\pgfsetstrokecolor{currentstroke}%
\pgfsetdash{}{0pt}%
\pgfsys@defobject{currentmarker}{\pgfqpoint{0.000000in}{-0.027778in}}{\pgfqpoint{0.000000in}{0.000000in}}{%
\pgfpathmoveto{\pgfqpoint{0.000000in}{0.000000in}}%
\pgfpathlineto{\pgfqpoint{0.000000in}{-0.027778in}}%
\pgfusepath{stroke,fill}%
}%
\begin{pgfscope}%
\pgfsys@transformshift{3.672410in}{0.550000in}%
\pgfsys@useobject{currentmarker}{}%
\end{pgfscope}%
\end{pgfscope}%
\begin{pgfscope}%
\pgfsetbuttcap%
\pgfsetroundjoin%
\definecolor{currentfill}{rgb}{0.000000,0.000000,0.000000}%
\pgfsetfillcolor{currentfill}%
\pgfsetlinewidth{0.602250pt}%
\definecolor{currentstroke}{rgb}{0.000000,0.000000,0.000000}%
\pgfsetstrokecolor{currentstroke}%
\pgfsetdash{}{0pt}%
\pgfsys@defobject{currentmarker}{\pgfqpoint{0.000000in}{-0.027778in}}{\pgfqpoint{0.000000in}{0.000000in}}{%
\pgfpathmoveto{\pgfqpoint{0.000000in}{0.000000in}}%
\pgfpathlineto{\pgfqpoint{0.000000in}{-0.027778in}}%
\pgfusepath{stroke,fill}%
}%
\begin{pgfscope}%
\pgfsys@transformshift{3.756646in}{0.550000in}%
\pgfsys@useobject{currentmarker}{}%
\end{pgfscope}%
\end{pgfscope}%
\begin{pgfscope}%
\pgfsetbuttcap%
\pgfsetroundjoin%
\definecolor{currentfill}{rgb}{0.000000,0.000000,0.000000}%
\pgfsetfillcolor{currentfill}%
\pgfsetlinewidth{0.602250pt}%
\definecolor{currentstroke}{rgb}{0.000000,0.000000,0.000000}%
\pgfsetstrokecolor{currentstroke}%
\pgfsetdash{}{0pt}%
\pgfsys@defobject{currentmarker}{\pgfqpoint{0.000000in}{-0.027778in}}{\pgfqpoint{0.000000in}{0.000000in}}{%
\pgfpathmoveto{\pgfqpoint{0.000000in}{0.000000in}}%
\pgfpathlineto{\pgfqpoint{0.000000in}{-0.027778in}}%
\pgfusepath{stroke,fill}%
}%
\begin{pgfscope}%
\pgfsys@transformshift{3.827867in}{0.550000in}%
\pgfsys@useobject{currentmarker}{}%
\end{pgfscope}%
\end{pgfscope}%
\begin{pgfscope}%
\pgfsetbuttcap%
\pgfsetroundjoin%
\definecolor{currentfill}{rgb}{0.000000,0.000000,0.000000}%
\pgfsetfillcolor{currentfill}%
\pgfsetlinewidth{0.602250pt}%
\definecolor{currentstroke}{rgb}{0.000000,0.000000,0.000000}%
\pgfsetstrokecolor{currentstroke}%
\pgfsetdash{}{0pt}%
\pgfsys@defobject{currentmarker}{\pgfqpoint{0.000000in}{-0.027778in}}{\pgfqpoint{0.000000in}{0.000000in}}{%
\pgfpathmoveto{\pgfqpoint{0.000000in}{0.000000in}}%
\pgfpathlineto{\pgfqpoint{0.000000in}{-0.027778in}}%
\pgfusepath{stroke,fill}%
}%
\begin{pgfscope}%
\pgfsys@transformshift{3.889561in}{0.550000in}%
\pgfsys@useobject{currentmarker}{}%
\end{pgfscope}%
\end{pgfscope}%
\begin{pgfscope}%
\pgfsetbuttcap%
\pgfsetroundjoin%
\definecolor{currentfill}{rgb}{0.000000,0.000000,0.000000}%
\pgfsetfillcolor{currentfill}%
\pgfsetlinewidth{0.602250pt}%
\definecolor{currentstroke}{rgb}{0.000000,0.000000,0.000000}%
\pgfsetstrokecolor{currentstroke}%
\pgfsetdash{}{0pt}%
\pgfsys@defobject{currentmarker}{\pgfqpoint{0.000000in}{-0.027778in}}{\pgfqpoint{0.000000in}{0.000000in}}{%
\pgfpathmoveto{\pgfqpoint{0.000000in}{0.000000in}}%
\pgfpathlineto{\pgfqpoint{0.000000in}{-0.027778in}}%
\pgfusepath{stroke,fill}%
}%
\begin{pgfscope}%
\pgfsys@transformshift{3.943980in}{0.550000in}%
\pgfsys@useobject{currentmarker}{}%
\end{pgfscope}%
\end{pgfscope}%
\begin{pgfscope}%
\pgfsetbuttcap%
\pgfsetroundjoin%
\definecolor{currentfill}{rgb}{0.000000,0.000000,0.000000}%
\pgfsetfillcolor{currentfill}%
\pgfsetlinewidth{0.602250pt}%
\definecolor{currentstroke}{rgb}{0.000000,0.000000,0.000000}%
\pgfsetstrokecolor{currentstroke}%
\pgfsetdash{}{0pt}%
\pgfsys@defobject{currentmarker}{\pgfqpoint{0.000000in}{-0.027778in}}{\pgfqpoint{0.000000in}{0.000000in}}{%
\pgfpathmoveto{\pgfqpoint{0.000000in}{0.000000in}}%
\pgfpathlineto{\pgfqpoint{0.000000in}{-0.027778in}}%
\pgfusepath{stroke,fill}%
}%
\begin{pgfscope}%
\pgfsys@transformshift{4.312906in}{0.550000in}%
\pgfsys@useobject{currentmarker}{}%
\end{pgfscope}%
\end{pgfscope}%
\begin{pgfscope}%
\pgfsetbuttcap%
\pgfsetroundjoin%
\definecolor{currentfill}{rgb}{0.000000,0.000000,0.000000}%
\pgfsetfillcolor{currentfill}%
\pgfsetlinewidth{0.602250pt}%
\definecolor{currentstroke}{rgb}{0.000000,0.000000,0.000000}%
\pgfsetstrokecolor{currentstroke}%
\pgfsetdash{}{0pt}%
\pgfsys@defobject{currentmarker}{\pgfqpoint{0.000000in}{-0.027778in}}{\pgfqpoint{0.000000in}{0.000000in}}{%
\pgfpathmoveto{\pgfqpoint{0.000000in}{0.000000in}}%
\pgfpathlineto{\pgfqpoint{0.000000in}{-0.027778in}}%
\pgfusepath{stroke,fill}%
}%
\begin{pgfscope}%
\pgfsys@transformshift{4.500239in}{0.550000in}%
\pgfsys@useobject{currentmarker}{}%
\end{pgfscope}%
\end{pgfscope}%
\begin{pgfscope}%
\definecolor{textcolor}{rgb}{0.000000,0.000000,0.000000}%
\pgfsetstrokecolor{textcolor}%
\pgfsetfillcolor{textcolor}%
\pgftext[x=2.562500in,y=0.249368in,,top]{\color{textcolor}\sffamily\fontsize{11.000000}{13.200000}\selectfont \(\displaystyle N_\mathrm{QMC}\)}%
\end{pgfscope}%
\begin{pgfscope}%
\pgfsetbuttcap%
\pgfsetroundjoin%
\definecolor{currentfill}{rgb}{0.000000,0.000000,0.000000}%
\pgfsetfillcolor{currentfill}%
\pgfsetlinewidth{0.803000pt}%
\definecolor{currentstroke}{rgb}{0.000000,0.000000,0.000000}%
\pgfsetstrokecolor{currentstroke}%
\pgfsetdash{}{0pt}%
\pgfsys@defobject{currentmarker}{\pgfqpoint{-0.048611in}{0.000000in}}{\pgfqpoint{0.000000in}{0.000000in}}{%
\pgfpathmoveto{\pgfqpoint{0.000000in}{0.000000in}}%
\pgfpathlineto{\pgfqpoint{-0.048611in}{0.000000in}}%
\pgfusepath{stroke,fill}%
}%
\begin{pgfscope}%
\pgfsys@transformshift{0.625000in}{0.652304in}%
\pgfsys@useobject{currentmarker}{}%
\end{pgfscope}%
\end{pgfscope}%
\begin{pgfscope}%
\definecolor{textcolor}{rgb}{0.000000,0.000000,0.000000}%
\pgfsetstrokecolor{textcolor}%
\pgfsetfillcolor{textcolor}%
\pgftext[x=0.217899in,y=0.594267in,left,base]{\color{textcolor}\sffamily\fontsize{11.000000}{13.200000}\selectfont \(\displaystyle 10^{-1}\)}%
\end{pgfscope}%
\begin{pgfscope}%
\pgfsetbuttcap%
\pgfsetroundjoin%
\definecolor{currentfill}{rgb}{0.000000,0.000000,0.000000}%
\pgfsetfillcolor{currentfill}%
\pgfsetlinewidth{0.803000pt}%
\definecolor{currentstroke}{rgb}{0.000000,0.000000,0.000000}%
\pgfsetstrokecolor{currentstroke}%
\pgfsetdash{}{0pt}%
\pgfsys@defobject{currentmarker}{\pgfqpoint{-0.048611in}{0.000000in}}{\pgfqpoint{0.000000in}{0.000000in}}{%
\pgfpathmoveto{\pgfqpoint{0.000000in}{0.000000in}}%
\pgfpathlineto{\pgfqpoint{-0.048611in}{0.000000in}}%
\pgfusepath{stroke,fill}%
}%
\begin{pgfscope}%
\pgfsys@transformshift{0.625000in}{2.459711in}%
\pgfsys@useobject{currentmarker}{}%
\end{pgfscope}%
\end{pgfscope}%
\begin{pgfscope}%
\definecolor{textcolor}{rgb}{0.000000,0.000000,0.000000}%
\pgfsetstrokecolor{textcolor}%
\pgfsetfillcolor{textcolor}%
\pgftext[x=0.309721in,y=2.401673in,left,base]{\color{textcolor}\sffamily\fontsize{11.000000}{13.200000}\selectfont \(\displaystyle 10^{0}\)}%
\end{pgfscope}%
\begin{pgfscope}%
\pgfsetbuttcap%
\pgfsetroundjoin%
\definecolor{currentfill}{rgb}{0.000000,0.000000,0.000000}%
\pgfsetfillcolor{currentfill}%
\pgfsetlinewidth{0.803000pt}%
\definecolor{currentstroke}{rgb}{0.000000,0.000000,0.000000}%
\pgfsetstrokecolor{currentstroke}%
\pgfsetdash{}{0pt}%
\pgfsys@defobject{currentmarker}{\pgfqpoint{-0.048611in}{0.000000in}}{\pgfqpoint{0.000000in}{0.000000in}}{%
\pgfpathmoveto{\pgfqpoint{0.000000in}{0.000000in}}%
\pgfpathlineto{\pgfqpoint{-0.048611in}{0.000000in}}%
\pgfusepath{stroke,fill}%
}%
\begin{pgfscope}%
\pgfsys@transformshift{0.625000in}{4.267117in}%
\pgfsys@useobject{currentmarker}{}%
\end{pgfscope}%
\end{pgfscope}%
\begin{pgfscope}%
\definecolor{textcolor}{rgb}{0.000000,0.000000,0.000000}%
\pgfsetstrokecolor{textcolor}%
\pgfsetfillcolor{textcolor}%
\pgftext[x=0.309721in,y=4.209080in,left,base]{\color{textcolor}\sffamily\fontsize{11.000000}{13.200000}\selectfont \(\displaystyle 10^{1}\)}%
\end{pgfscope}%
\begin{pgfscope}%
\pgfsetbuttcap%
\pgfsetroundjoin%
\definecolor{currentfill}{rgb}{0.000000,0.000000,0.000000}%
\pgfsetfillcolor{currentfill}%
\pgfsetlinewidth{0.602250pt}%
\definecolor{currentstroke}{rgb}{0.000000,0.000000,0.000000}%
\pgfsetstrokecolor{currentstroke}%
\pgfsetdash{}{0pt}%
\pgfsys@defobject{currentmarker}{\pgfqpoint{-0.027778in}{0.000000in}}{\pgfqpoint{0.000000in}{0.000000in}}{%
\pgfpathmoveto{\pgfqpoint{0.000000in}{0.000000in}}%
\pgfpathlineto{\pgfqpoint{-0.027778in}{0.000000in}}%
\pgfusepath{stroke,fill}%
}%
\begin{pgfscope}%
\pgfsys@transformshift{0.625000in}{0.569602in}%
\pgfsys@useobject{currentmarker}{}%
\end{pgfscope}%
\end{pgfscope}%
\begin{pgfscope}%
\pgfsetbuttcap%
\pgfsetroundjoin%
\definecolor{currentfill}{rgb}{0.000000,0.000000,0.000000}%
\pgfsetfillcolor{currentfill}%
\pgfsetlinewidth{0.602250pt}%
\definecolor{currentstroke}{rgb}{0.000000,0.000000,0.000000}%
\pgfsetstrokecolor{currentstroke}%
\pgfsetdash{}{0pt}%
\pgfsys@defobject{currentmarker}{\pgfqpoint{-0.027778in}{0.000000in}}{\pgfqpoint{0.000000in}{0.000000in}}{%
\pgfpathmoveto{\pgfqpoint{0.000000in}{0.000000in}}%
\pgfpathlineto{\pgfqpoint{-0.027778in}{0.000000in}}%
\pgfusepath{stroke,fill}%
}%
\begin{pgfscope}%
\pgfsys@transformshift{0.625000in}{1.196388in}%
\pgfsys@useobject{currentmarker}{}%
\end{pgfscope}%
\end{pgfscope}%
\begin{pgfscope}%
\pgfsetbuttcap%
\pgfsetroundjoin%
\definecolor{currentfill}{rgb}{0.000000,0.000000,0.000000}%
\pgfsetfillcolor{currentfill}%
\pgfsetlinewidth{0.602250pt}%
\definecolor{currentstroke}{rgb}{0.000000,0.000000,0.000000}%
\pgfsetstrokecolor{currentstroke}%
\pgfsetdash{}{0pt}%
\pgfsys@defobject{currentmarker}{\pgfqpoint{-0.027778in}{0.000000in}}{\pgfqpoint{0.000000in}{0.000000in}}{%
\pgfpathmoveto{\pgfqpoint{0.000000in}{0.000000in}}%
\pgfpathlineto{\pgfqpoint{-0.027778in}{0.000000in}}%
\pgfusepath{stroke,fill}%
}%
\begin{pgfscope}%
\pgfsys@transformshift{0.625000in}{1.514656in}%
\pgfsys@useobject{currentmarker}{}%
\end{pgfscope}%
\end{pgfscope}%
\begin{pgfscope}%
\pgfsetbuttcap%
\pgfsetroundjoin%
\definecolor{currentfill}{rgb}{0.000000,0.000000,0.000000}%
\pgfsetfillcolor{currentfill}%
\pgfsetlinewidth{0.602250pt}%
\definecolor{currentstroke}{rgb}{0.000000,0.000000,0.000000}%
\pgfsetstrokecolor{currentstroke}%
\pgfsetdash{}{0pt}%
\pgfsys@defobject{currentmarker}{\pgfqpoint{-0.027778in}{0.000000in}}{\pgfqpoint{0.000000in}{0.000000in}}{%
\pgfpathmoveto{\pgfqpoint{0.000000in}{0.000000in}}%
\pgfpathlineto{\pgfqpoint{-0.027778in}{0.000000in}}%
\pgfusepath{stroke,fill}%
}%
\begin{pgfscope}%
\pgfsys@transformshift{0.625000in}{1.740472in}%
\pgfsys@useobject{currentmarker}{}%
\end{pgfscope}%
\end{pgfscope}%
\begin{pgfscope}%
\pgfsetbuttcap%
\pgfsetroundjoin%
\definecolor{currentfill}{rgb}{0.000000,0.000000,0.000000}%
\pgfsetfillcolor{currentfill}%
\pgfsetlinewidth{0.602250pt}%
\definecolor{currentstroke}{rgb}{0.000000,0.000000,0.000000}%
\pgfsetstrokecolor{currentstroke}%
\pgfsetdash{}{0pt}%
\pgfsys@defobject{currentmarker}{\pgfqpoint{-0.027778in}{0.000000in}}{\pgfqpoint{0.000000in}{0.000000in}}{%
\pgfpathmoveto{\pgfqpoint{0.000000in}{0.000000in}}%
\pgfpathlineto{\pgfqpoint{-0.027778in}{0.000000in}}%
\pgfusepath{stroke,fill}%
}%
\begin{pgfscope}%
\pgfsys@transformshift{0.625000in}{1.915627in}%
\pgfsys@useobject{currentmarker}{}%
\end{pgfscope}%
\end{pgfscope}%
\begin{pgfscope}%
\pgfsetbuttcap%
\pgfsetroundjoin%
\definecolor{currentfill}{rgb}{0.000000,0.000000,0.000000}%
\pgfsetfillcolor{currentfill}%
\pgfsetlinewidth{0.602250pt}%
\definecolor{currentstroke}{rgb}{0.000000,0.000000,0.000000}%
\pgfsetstrokecolor{currentstroke}%
\pgfsetdash{}{0pt}%
\pgfsys@defobject{currentmarker}{\pgfqpoint{-0.027778in}{0.000000in}}{\pgfqpoint{0.000000in}{0.000000in}}{%
\pgfpathmoveto{\pgfqpoint{0.000000in}{0.000000in}}%
\pgfpathlineto{\pgfqpoint{-0.027778in}{0.000000in}}%
\pgfusepath{stroke,fill}%
}%
\begin{pgfscope}%
\pgfsys@transformshift{0.625000in}{2.058740in}%
\pgfsys@useobject{currentmarker}{}%
\end{pgfscope}%
\end{pgfscope}%
\begin{pgfscope}%
\pgfsetbuttcap%
\pgfsetroundjoin%
\definecolor{currentfill}{rgb}{0.000000,0.000000,0.000000}%
\pgfsetfillcolor{currentfill}%
\pgfsetlinewidth{0.602250pt}%
\definecolor{currentstroke}{rgb}{0.000000,0.000000,0.000000}%
\pgfsetstrokecolor{currentstroke}%
\pgfsetdash{}{0pt}%
\pgfsys@defobject{currentmarker}{\pgfqpoint{-0.027778in}{0.000000in}}{\pgfqpoint{0.000000in}{0.000000in}}{%
\pgfpathmoveto{\pgfqpoint{0.000000in}{0.000000in}}%
\pgfpathlineto{\pgfqpoint{-0.027778in}{0.000000in}}%
\pgfusepath{stroke,fill}%
}%
\begin{pgfscope}%
\pgfsys@transformshift{0.625000in}{2.179740in}%
\pgfsys@useobject{currentmarker}{}%
\end{pgfscope}%
\end{pgfscope}%
\begin{pgfscope}%
\pgfsetbuttcap%
\pgfsetroundjoin%
\definecolor{currentfill}{rgb}{0.000000,0.000000,0.000000}%
\pgfsetfillcolor{currentfill}%
\pgfsetlinewidth{0.602250pt}%
\definecolor{currentstroke}{rgb}{0.000000,0.000000,0.000000}%
\pgfsetstrokecolor{currentstroke}%
\pgfsetdash{}{0pt}%
\pgfsys@defobject{currentmarker}{\pgfqpoint{-0.027778in}{0.000000in}}{\pgfqpoint{0.000000in}{0.000000in}}{%
\pgfpathmoveto{\pgfqpoint{0.000000in}{0.000000in}}%
\pgfpathlineto{\pgfqpoint{-0.027778in}{0.000000in}}%
\pgfusepath{stroke,fill}%
}%
\begin{pgfscope}%
\pgfsys@transformshift{0.625000in}{2.284555in}%
\pgfsys@useobject{currentmarker}{}%
\end{pgfscope}%
\end{pgfscope}%
\begin{pgfscope}%
\pgfsetbuttcap%
\pgfsetroundjoin%
\definecolor{currentfill}{rgb}{0.000000,0.000000,0.000000}%
\pgfsetfillcolor{currentfill}%
\pgfsetlinewidth{0.602250pt}%
\definecolor{currentstroke}{rgb}{0.000000,0.000000,0.000000}%
\pgfsetstrokecolor{currentstroke}%
\pgfsetdash{}{0pt}%
\pgfsys@defobject{currentmarker}{\pgfqpoint{-0.027778in}{0.000000in}}{\pgfqpoint{0.000000in}{0.000000in}}{%
\pgfpathmoveto{\pgfqpoint{0.000000in}{0.000000in}}%
\pgfpathlineto{\pgfqpoint{-0.027778in}{0.000000in}}%
\pgfusepath{stroke,fill}%
}%
\begin{pgfscope}%
\pgfsys@transformshift{0.625000in}{2.377009in}%
\pgfsys@useobject{currentmarker}{}%
\end{pgfscope}%
\end{pgfscope}%
\begin{pgfscope}%
\pgfsetbuttcap%
\pgfsetroundjoin%
\definecolor{currentfill}{rgb}{0.000000,0.000000,0.000000}%
\pgfsetfillcolor{currentfill}%
\pgfsetlinewidth{0.602250pt}%
\definecolor{currentstroke}{rgb}{0.000000,0.000000,0.000000}%
\pgfsetstrokecolor{currentstroke}%
\pgfsetdash{}{0pt}%
\pgfsys@defobject{currentmarker}{\pgfqpoint{-0.027778in}{0.000000in}}{\pgfqpoint{0.000000in}{0.000000in}}{%
\pgfpathmoveto{\pgfqpoint{0.000000in}{0.000000in}}%
\pgfpathlineto{\pgfqpoint{-0.027778in}{0.000000in}}%
\pgfusepath{stroke,fill}%
}%
\begin{pgfscope}%
\pgfsys@transformshift{0.625000in}{3.003795in}%
\pgfsys@useobject{currentmarker}{}%
\end{pgfscope}%
\end{pgfscope}%
\begin{pgfscope}%
\pgfsetbuttcap%
\pgfsetroundjoin%
\definecolor{currentfill}{rgb}{0.000000,0.000000,0.000000}%
\pgfsetfillcolor{currentfill}%
\pgfsetlinewidth{0.602250pt}%
\definecolor{currentstroke}{rgb}{0.000000,0.000000,0.000000}%
\pgfsetstrokecolor{currentstroke}%
\pgfsetdash{}{0pt}%
\pgfsys@defobject{currentmarker}{\pgfqpoint{-0.027778in}{0.000000in}}{\pgfqpoint{0.000000in}{0.000000in}}{%
\pgfpathmoveto{\pgfqpoint{0.000000in}{0.000000in}}%
\pgfpathlineto{\pgfqpoint{-0.027778in}{0.000000in}}%
\pgfusepath{stroke,fill}%
}%
\begin{pgfscope}%
\pgfsys@transformshift{0.625000in}{3.322063in}%
\pgfsys@useobject{currentmarker}{}%
\end{pgfscope}%
\end{pgfscope}%
\begin{pgfscope}%
\pgfsetbuttcap%
\pgfsetroundjoin%
\definecolor{currentfill}{rgb}{0.000000,0.000000,0.000000}%
\pgfsetfillcolor{currentfill}%
\pgfsetlinewidth{0.602250pt}%
\definecolor{currentstroke}{rgb}{0.000000,0.000000,0.000000}%
\pgfsetstrokecolor{currentstroke}%
\pgfsetdash{}{0pt}%
\pgfsys@defobject{currentmarker}{\pgfqpoint{-0.027778in}{0.000000in}}{\pgfqpoint{0.000000in}{0.000000in}}{%
\pgfpathmoveto{\pgfqpoint{0.000000in}{0.000000in}}%
\pgfpathlineto{\pgfqpoint{-0.027778in}{0.000000in}}%
\pgfusepath{stroke,fill}%
}%
\begin{pgfscope}%
\pgfsys@transformshift{0.625000in}{3.547878in}%
\pgfsys@useobject{currentmarker}{}%
\end{pgfscope}%
\end{pgfscope}%
\begin{pgfscope}%
\pgfsetbuttcap%
\pgfsetroundjoin%
\definecolor{currentfill}{rgb}{0.000000,0.000000,0.000000}%
\pgfsetfillcolor{currentfill}%
\pgfsetlinewidth{0.602250pt}%
\definecolor{currentstroke}{rgb}{0.000000,0.000000,0.000000}%
\pgfsetstrokecolor{currentstroke}%
\pgfsetdash{}{0pt}%
\pgfsys@defobject{currentmarker}{\pgfqpoint{-0.027778in}{0.000000in}}{\pgfqpoint{0.000000in}{0.000000in}}{%
\pgfpathmoveto{\pgfqpoint{0.000000in}{0.000000in}}%
\pgfpathlineto{\pgfqpoint{-0.027778in}{0.000000in}}%
\pgfusepath{stroke,fill}%
}%
\begin{pgfscope}%
\pgfsys@transformshift{0.625000in}{3.723034in}%
\pgfsys@useobject{currentmarker}{}%
\end{pgfscope}%
\end{pgfscope}%
\begin{pgfscope}%
\pgfsetbuttcap%
\pgfsetroundjoin%
\definecolor{currentfill}{rgb}{0.000000,0.000000,0.000000}%
\pgfsetfillcolor{currentfill}%
\pgfsetlinewidth{0.602250pt}%
\definecolor{currentstroke}{rgb}{0.000000,0.000000,0.000000}%
\pgfsetstrokecolor{currentstroke}%
\pgfsetdash{}{0pt}%
\pgfsys@defobject{currentmarker}{\pgfqpoint{-0.027778in}{0.000000in}}{\pgfqpoint{0.000000in}{0.000000in}}{%
\pgfpathmoveto{\pgfqpoint{0.000000in}{0.000000in}}%
\pgfpathlineto{\pgfqpoint{-0.027778in}{0.000000in}}%
\pgfusepath{stroke,fill}%
}%
\begin{pgfscope}%
\pgfsys@transformshift{0.625000in}{3.866147in}%
\pgfsys@useobject{currentmarker}{}%
\end{pgfscope}%
\end{pgfscope}%
\begin{pgfscope}%
\pgfsetbuttcap%
\pgfsetroundjoin%
\definecolor{currentfill}{rgb}{0.000000,0.000000,0.000000}%
\pgfsetfillcolor{currentfill}%
\pgfsetlinewidth{0.602250pt}%
\definecolor{currentstroke}{rgb}{0.000000,0.000000,0.000000}%
\pgfsetstrokecolor{currentstroke}%
\pgfsetdash{}{0pt}%
\pgfsys@defobject{currentmarker}{\pgfqpoint{-0.027778in}{0.000000in}}{\pgfqpoint{0.000000in}{0.000000in}}{%
\pgfpathmoveto{\pgfqpoint{0.000000in}{0.000000in}}%
\pgfpathlineto{\pgfqpoint{-0.027778in}{0.000000in}}%
\pgfusepath{stroke,fill}%
}%
\begin{pgfscope}%
\pgfsys@transformshift{0.625000in}{3.987147in}%
\pgfsys@useobject{currentmarker}{}%
\end{pgfscope}%
\end{pgfscope}%
\begin{pgfscope}%
\pgfsetbuttcap%
\pgfsetroundjoin%
\definecolor{currentfill}{rgb}{0.000000,0.000000,0.000000}%
\pgfsetfillcolor{currentfill}%
\pgfsetlinewidth{0.602250pt}%
\definecolor{currentstroke}{rgb}{0.000000,0.000000,0.000000}%
\pgfsetstrokecolor{currentstroke}%
\pgfsetdash{}{0pt}%
\pgfsys@defobject{currentmarker}{\pgfqpoint{-0.027778in}{0.000000in}}{\pgfqpoint{0.000000in}{0.000000in}}{%
\pgfpathmoveto{\pgfqpoint{0.000000in}{0.000000in}}%
\pgfpathlineto{\pgfqpoint{-0.027778in}{0.000000in}}%
\pgfusepath{stroke,fill}%
}%
\begin{pgfscope}%
\pgfsys@transformshift{0.625000in}{4.091962in}%
\pgfsys@useobject{currentmarker}{}%
\end{pgfscope}%
\end{pgfscope}%
\begin{pgfscope}%
\pgfsetbuttcap%
\pgfsetroundjoin%
\definecolor{currentfill}{rgb}{0.000000,0.000000,0.000000}%
\pgfsetfillcolor{currentfill}%
\pgfsetlinewidth{0.602250pt}%
\definecolor{currentstroke}{rgb}{0.000000,0.000000,0.000000}%
\pgfsetstrokecolor{currentstroke}%
\pgfsetdash{}{0pt}%
\pgfsys@defobject{currentmarker}{\pgfqpoint{-0.027778in}{0.000000in}}{\pgfqpoint{0.000000in}{0.000000in}}{%
\pgfpathmoveto{\pgfqpoint{0.000000in}{0.000000in}}%
\pgfpathlineto{\pgfqpoint{-0.027778in}{0.000000in}}%
\pgfusepath{stroke,fill}%
}%
\begin{pgfscope}%
\pgfsys@transformshift{0.625000in}{4.184415in}%
\pgfsys@useobject{currentmarker}{}%
\end{pgfscope}%
\end{pgfscope}%
\begin{pgfscope}%
\definecolor{textcolor}{rgb}{0.000000,0.000000,0.000000}%
\pgfsetstrokecolor{textcolor}%
\pgfsetfillcolor{textcolor}%
\pgftext[x=0.162343in,y=2.475000in,,bottom,rotate=90.000000]{\color{textcolor}\sffamily\fontsize{11.000000}{13.200000}\selectfont \textrm{QMC Error}}%
\end{pgfscope}%
\begin{pgfscope}%
\pgfpathrectangle{\pgfqpoint{0.625000in}{0.550000in}}{\pgfqpoint{3.875000in}{3.850000in}}%
\pgfusepath{clip}%
\pgfsetbuttcap%
\pgfsetroundjoin%
\definecolor{currentfill}{rgb}{0.000000,0.000000,0.000000}%
\pgfsetfillcolor{currentfill}%
\pgfsetlinewidth{1.003750pt}%
\definecolor{currentstroke}{rgb}{0.000000,0.000000,0.000000}%
\pgfsetstrokecolor{currentstroke}%
\pgfsetdash{}{0pt}%
\pgfsys@defobject{currentmarker}{\pgfqpoint{-0.041667in}{-0.041667in}}{\pgfqpoint{0.041667in}{0.041667in}}{%
\pgfpathmoveto{\pgfqpoint{0.000000in}{-0.041667in}}%
\pgfpathcurveto{\pgfqpoint{0.011050in}{-0.041667in}}{\pgfqpoint{0.021649in}{-0.037276in}}{\pgfqpoint{0.029463in}{-0.029463in}}%
\pgfpathcurveto{\pgfqpoint{0.037276in}{-0.021649in}}{\pgfqpoint{0.041667in}{-0.011050in}}{\pgfqpoint{0.041667in}{0.000000in}}%
\pgfpathcurveto{\pgfqpoint{0.041667in}{0.011050in}}{\pgfqpoint{0.037276in}{0.021649in}}{\pgfqpoint{0.029463in}{0.029463in}}%
\pgfpathcurveto{\pgfqpoint{0.021649in}{0.037276in}}{\pgfqpoint{0.011050in}{0.041667in}}{\pgfqpoint{0.000000in}{0.041667in}}%
\pgfpathcurveto{\pgfqpoint{-0.011050in}{0.041667in}}{\pgfqpoint{-0.021649in}{0.037276in}}{\pgfqpoint{-0.029463in}{0.029463in}}%
\pgfpathcurveto{\pgfqpoint{-0.037276in}{0.021649in}}{\pgfqpoint{-0.041667in}{0.011050in}}{\pgfqpoint{-0.041667in}{0.000000in}}%
\pgfpathcurveto{\pgfqpoint{-0.041667in}{-0.011050in}}{\pgfqpoint{-0.037276in}{-0.021649in}}{\pgfqpoint{-0.029463in}{-0.029463in}}%
\pgfpathcurveto{\pgfqpoint{-0.021649in}{-0.037276in}}{\pgfqpoint{-0.011050in}{-0.041667in}}{\pgfqpoint{0.000000in}{-0.041667in}}%
\pgfpathclose%
\pgfusepath{stroke,fill}%
}%
\begin{pgfscope}%
\pgfsys@transformshift{0.801136in}{4.225000in}%
\pgfsys@useobject{currentmarker}{}%
\end{pgfscope}%
\begin{pgfscope}%
\pgfsys@transformshift{1.121384in}{3.926983in}%
\pgfsys@useobject{currentmarker}{}%
\end{pgfscope}%
\begin{pgfscope}%
\pgfsys@transformshift{1.441632in}{3.569945in}%
\pgfsys@useobject{currentmarker}{}%
\end{pgfscope}%
\begin{pgfscope}%
\pgfsys@transformshift{1.761880in}{3.091560in}%
\pgfsys@useobject{currentmarker}{}%
\end{pgfscope}%
\begin{pgfscope}%
\pgfsys@transformshift{2.082128in}{2.813014in}%
\pgfsys@useobject{currentmarker}{}%
\end{pgfscope}%
\begin{pgfscope}%
\pgfsys@transformshift{2.402376in}{2.462297in}%
\pgfsys@useobject{currentmarker}{}%
\end{pgfscope}%
\begin{pgfscope}%
\pgfsys@transformshift{2.722624in}{2.348152in}%
\pgfsys@useobject{currentmarker}{}%
\end{pgfscope}%
\begin{pgfscope}%
\pgfsys@transformshift{3.042872in}{2.335524in}%
\pgfsys@useobject{currentmarker}{}%
\end{pgfscope}%
\begin{pgfscope}%
\pgfsys@transformshift{3.363120in}{1.319511in}%
\pgfsys@useobject{currentmarker}{}%
\end{pgfscope}%
\begin{pgfscope}%
\pgfsys@transformshift{3.683368in}{1.044364in}%
\pgfsys@useobject{currentmarker}{}%
\end{pgfscope}%
\begin{pgfscope}%
\pgfsys@transformshift{4.003616in}{1.067431in}%
\pgfsys@useobject{currentmarker}{}%
\end{pgfscope}%
\begin{pgfscope}%
\pgfsys@transformshift{4.323864in}{1.029164in}%
\pgfsys@useobject{currentmarker}{}%
\end{pgfscope}%
\end{pgfscope}%
\begin{pgfscope}%
\pgfpathrectangle{\pgfqpoint{0.625000in}{0.550000in}}{\pgfqpoint{3.875000in}{3.850000in}}%
\pgfusepath{clip}%
\pgfsetbuttcap%
\pgfsetroundjoin%
\pgfsetlinewidth{1.505625pt}%
\definecolor{currentstroke}{rgb}{0.000000,0.000000,0.000000}%
\pgfsetstrokecolor{currentstroke}%
\pgfsetdash{{5.550000pt}{2.400000pt}}{0.000000pt}%
\pgfpathmoveto{\pgfqpoint{0.801136in}{4.147158in}}%
\pgfpathlineto{\pgfqpoint{4.323864in}{0.725000in}}%
\pgfusepath{stroke}%
\end{pgfscope}%
\begin{pgfscope}%
\pgfsetrectcap%
\pgfsetmiterjoin%
\pgfsetlinewidth{0.803000pt}%
\definecolor{currentstroke}{rgb}{0.000000,0.000000,0.000000}%
\pgfsetstrokecolor{currentstroke}%
\pgfsetdash{}{0pt}%
\pgfpathmoveto{\pgfqpoint{0.625000in}{0.550000in}}%
\pgfpathlineto{\pgfqpoint{0.625000in}{4.400000in}}%
\pgfusepath{stroke}%
\end{pgfscope}%
\begin{pgfscope}%
\pgfsetrectcap%
\pgfsetmiterjoin%
\pgfsetlinewidth{0.000000pt}%
\definecolor{currentstroke}{rgb}{0.000000,0.000000,0.000000}%
\pgfsetstrokecolor{currentstroke}%
\pgfsetstrokeopacity{0.000000}%
\pgfsetdash{}{0pt}%
\pgfpathmoveto{\pgfqpoint{4.500000in}{0.550000in}}%
\pgfpathlineto{\pgfqpoint{4.500000in}{4.400000in}}%
\pgfusepath{}%
\end{pgfscope}%
\begin{pgfscope}%
\pgfsetrectcap%
\pgfsetmiterjoin%
\pgfsetlinewidth{0.803000pt}%
\definecolor{currentstroke}{rgb}{0.000000,0.000000,0.000000}%
\pgfsetstrokecolor{currentstroke}%
\pgfsetdash{}{0pt}%
\pgfpathmoveto{\pgfqpoint{0.625000in}{0.550000in}}%
\pgfpathlineto{\pgfqpoint{4.500000in}{0.550000in}}%
\pgfusepath{stroke}%
\end{pgfscope}%
\begin{pgfscope}%
\pgfsetrectcap%
\pgfsetmiterjoin%
\pgfsetlinewidth{0.000000pt}%
\definecolor{currentstroke}{rgb}{0.000000,0.000000,0.000000}%
\pgfsetstrokecolor{currentstroke}%
\pgfsetstrokeopacity{0.000000}%
\pgfsetdash{}{0pt}%
\pgfpathmoveto{\pgfqpoint{0.625000in}{4.400000in}}%
\pgfpathlineto{\pgfqpoint{4.500000in}{4.400000in}}%
\pgfusepath{}%
\end{pgfscope}%
\begin{pgfscope}%
\pgfsetbuttcap%
\pgfsetmiterjoin%
\definecolor{currentfill}{rgb}{1.000000,1.000000,1.000000}%
\pgfsetfillcolor{currentfill}%
\pgfsetfillopacity{0.800000}%
\pgfsetlinewidth{1.003750pt}%
\definecolor{currentstroke}{rgb}{0.800000,0.800000,0.800000}%
\pgfsetstrokecolor{currentstroke}%
\pgfsetstrokeopacity{0.800000}%
\pgfsetdash{}{0pt}%
\pgfpathmoveto{\pgfqpoint{2.916233in}{3.999488in}}%
\pgfpathlineto{\pgfqpoint{4.393056in}{3.999488in}}%
\pgfpathquadraticcurveto{\pgfqpoint{4.423611in}{3.999488in}}{\pgfqpoint{4.423611in}{4.030043in}}%
\pgfpathlineto{\pgfqpoint{4.423611in}{4.293056in}}%
\pgfpathquadraticcurveto{\pgfqpoint{4.423611in}{4.323611in}}{\pgfqpoint{4.393056in}{4.323611in}}%
\pgfpathlineto{\pgfqpoint{2.916233in}{4.323611in}}%
\pgfpathquadraticcurveto{\pgfqpoint{2.885678in}{4.323611in}}{\pgfqpoint{2.885678in}{4.293056in}}%
\pgfpathlineto{\pgfqpoint{2.885678in}{4.030043in}}%
\pgfpathquadraticcurveto{\pgfqpoint{2.885678in}{3.999488in}}{\pgfqpoint{2.916233in}{3.999488in}}%
\pgfpathclose%
\pgfusepath{stroke,fill}%
\end{pgfscope}%
\begin{pgfscope}%
\pgfsetbuttcap%
\pgfsetroundjoin%
\pgfsetlinewidth{1.505625pt}%
\definecolor{currentstroke}{rgb}{0.000000,0.000000,0.000000}%
\pgfsetstrokecolor{currentstroke}%
\pgfsetdash{{5.550000pt}{2.400000pt}}{0.000000pt}%
\pgfpathmoveto{\pgfqpoint{2.946789in}{4.180408in}}%
\pgfpathlineto{\pgfqpoint{3.252344in}{4.180408in}}%
\pgfusepath{stroke}%
\end{pgfscope}%
\begin{pgfscope}%
\definecolor{textcolor}{rgb}{0.000000,0.000000,0.000000}%
\pgfsetstrokecolor{textcolor}%
\pgfsetfillcolor{textcolor}%
\pgftext[x=3.374567in,y=4.126936in,left,base]{\color{textcolor}\sffamily\fontsize{11.000000}{13.200000}\selectfont \(\displaystyle 8.5827N_{\mathrm{QMC}}^{-0.5717}\)}%
\end{pgfscope}%
\end{pgfpicture}%
\makeatother%
\endgroup%

\caption[Quasi-Monte-Carlo error, for $Q=\grad u((1,1))$ and $k=60$.]{Quasi-Monte-Carlo error with increasing number of Quasi-Monte-Carlo points, for $Q=\grad u((1,1))$ and $k=60$. \label{fig:qmcgradient_top_right60}}
\end{figure}



\chapter{Computational set-up}\label{app:compsetup}
All of the computations in this thesis were performed with the following computational setup, unless otherwise stated.

\paragraph{PDE/FEM:}
The PDE we solve is the Interior Impedance Problem, i.e., \cref{prob:vtedp} with $\Dm = \emptyset,$ posed on the 2-d unit square; $D = [0,1]^2.$ We use first-order continuous finite elements, with $h = k^{-3/2}$. We use regular grids, see \cref{fig:grid} for an example grid. Where we needed to calculate a preconditioner $\AmatoI,$ we calculated the exact $LU$ decomposition of $\Amato$.

\begin{figure}[h]
  \centering
  %% Creator: Matplotlib, PGF backend
%%
%% To include the figure in your LaTeX document, write
%%   \input{<filename>.pgf}
%%
%% Make sure the required packages are loaded in your preamble
%%   \usepackage{pgf}
%%
%% Figures using additional raster images can only be included by \input if
%% they are in the same directory as the main LaTeX file. For loading figures
%% from other directories you can use the `import` package
%%   \usepackage{import}
%% and then include the figures with
%%   \import{<path to file>}{<filename>.pgf}
%%
%% Matplotlib used the following preamble
%%   \usepackage{fontspec}
%%   \setmainfont{DejaVuSerif.ttf}[Path=/home/owen/progs/firedrake-complex/firedrake/lib/python3.5/site-packages/matplotlib/mpl-data/fonts/ttf/]
%%   \setsansfont{DejaVuSans.ttf}[Path=/home/owen/progs/firedrake-complex/firedrake/lib/python3.5/site-packages/matplotlib/mpl-data/fonts/ttf/]
%%   \setmonofont{DejaVuSansMono.ttf}[Path=/home/owen/progs/firedrake-complex/firedrake/lib/python3.5/site-packages/matplotlib/mpl-data/fonts/ttf/]
%%
\begingroup%
\makeatletter%
\begin{pgfpicture}%
\pgfpathrectangle{\pgfpointorigin}{\pgfqpoint{6.400000in}{4.800000in}}%
\pgfusepath{use as bounding box, clip}%
\begin{pgfscope}%
\pgfsetbuttcap%
\pgfsetmiterjoin%
\definecolor{currentfill}{rgb}{1.000000,1.000000,1.000000}%
\pgfsetfillcolor{currentfill}%
\pgfsetlinewidth{0.000000pt}%
\definecolor{currentstroke}{rgb}{1.000000,1.000000,1.000000}%
\pgfsetstrokecolor{currentstroke}%
\pgfsetdash{}{0pt}%
\pgfpathmoveto{\pgfqpoint{0.000000in}{0.000000in}}%
\pgfpathlineto{\pgfqpoint{6.400000in}{0.000000in}}%
\pgfpathlineto{\pgfqpoint{6.400000in}{4.800000in}}%
\pgfpathlineto{\pgfqpoint{0.000000in}{4.800000in}}%
\pgfpathclose%
\pgfusepath{fill}%
\end{pgfscope}%
\begin{pgfscope}%
\pgfsetbuttcap%
\pgfsetmiterjoin%
\definecolor{currentfill}{rgb}{1.000000,1.000000,1.000000}%
\pgfsetfillcolor{currentfill}%
\pgfsetlinewidth{0.000000pt}%
\definecolor{currentstroke}{rgb}{0.000000,0.000000,0.000000}%
\pgfsetstrokecolor{currentstroke}%
\pgfsetstrokeopacity{0.000000}%
\pgfsetdash{}{0pt}%
\pgfpathmoveto{\pgfqpoint{1.432000in}{0.528000in}}%
\pgfpathlineto{\pgfqpoint{5.128000in}{0.528000in}}%
\pgfpathlineto{\pgfqpoint{5.128000in}{4.224000in}}%
\pgfpathlineto{\pgfqpoint{1.432000in}{4.224000in}}%
\pgfpathclose%
\pgfusepath{fill}%
\end{pgfscope}%
\begin{pgfscope}%
\pgfpathrectangle{\pgfqpoint{1.432000in}{0.528000in}}{\pgfqpoint{3.696000in}{3.696000in}}%
\pgfusepath{clip}%
\pgfsetbuttcap%
\pgfsetroundjoin%
\definecolor{currentfill}{rgb}{0.000000,0.000000,0.000000}%
\pgfsetfillcolor{currentfill}%
\pgfsetlinewidth{1.003750pt}%
\definecolor{currentstroke}{rgb}{0.000000,0.000000,0.000000}%
\pgfsetstrokecolor{currentstroke}%
\pgfsetdash{}{0pt}%
\pgfpathmoveto{\pgfqpoint{1.600000in}{0.671220in}}%
\pgfpathcurveto{\pgfqpoint{1.606572in}{0.671220in}}{\pgfqpoint{1.612875in}{0.673831in}}{\pgfqpoint{1.617522in}{0.678478in}}%
\pgfpathcurveto{\pgfqpoint{1.622169in}{0.683125in}}{\pgfqpoint{1.624780in}{0.689428in}}{\pgfqpoint{1.624780in}{0.696000in}}%
\pgfpathcurveto{\pgfqpoint{1.624780in}{0.702572in}}{\pgfqpoint{1.622169in}{0.708875in}}{\pgfqpoint{1.617522in}{0.713522in}}%
\pgfpathcurveto{\pgfqpoint{1.612875in}{0.718169in}}{\pgfqpoint{1.606572in}{0.720780in}}{\pgfqpoint{1.600000in}{0.720780in}}%
\pgfpathcurveto{\pgfqpoint{1.593428in}{0.720780in}}{\pgfqpoint{1.587125in}{0.718169in}}{\pgfqpoint{1.582478in}{0.713522in}}%
\pgfpathcurveto{\pgfqpoint{1.577831in}{0.708875in}}{\pgfqpoint{1.575220in}{0.702572in}}{\pgfqpoint{1.575220in}{0.696000in}}%
\pgfpathcurveto{\pgfqpoint{1.575220in}{0.689428in}}{\pgfqpoint{1.577831in}{0.683125in}}{\pgfqpoint{1.582478in}{0.678478in}}%
\pgfpathcurveto{\pgfqpoint{1.587125in}{0.673831in}}{\pgfqpoint{1.593428in}{0.671220in}}{\pgfqpoint{1.600000in}{0.671220in}}%
\pgfpathclose%
\pgfusepath{stroke,fill}%
\end{pgfscope}%
\begin{pgfscope}%
\pgfpathrectangle{\pgfqpoint{1.432000in}{0.528000in}}{\pgfqpoint{3.696000in}{3.696000in}}%
\pgfusepath{clip}%
\pgfsetbuttcap%
\pgfsetroundjoin%
\definecolor{currentfill}{rgb}{0.000000,0.000000,0.000000}%
\pgfsetfillcolor{currentfill}%
\pgfsetlinewidth{1.003750pt}%
\definecolor{currentstroke}{rgb}{0.000000,0.000000,0.000000}%
\pgfsetstrokecolor{currentstroke}%
\pgfsetdash{}{0pt}%
\pgfpathmoveto{\pgfqpoint{1.600000in}{1.007220in}}%
\pgfpathcurveto{\pgfqpoint{1.606572in}{1.007220in}}{\pgfqpoint{1.612875in}{1.009831in}}{\pgfqpoint{1.617522in}{1.014478in}}%
\pgfpathcurveto{\pgfqpoint{1.622169in}{1.019125in}}{\pgfqpoint{1.624780in}{1.025428in}}{\pgfqpoint{1.624780in}{1.032000in}}%
\pgfpathcurveto{\pgfqpoint{1.624780in}{1.038572in}}{\pgfqpoint{1.622169in}{1.044875in}}{\pgfqpoint{1.617522in}{1.049522in}}%
\pgfpathcurveto{\pgfqpoint{1.612875in}{1.054169in}}{\pgfqpoint{1.606572in}{1.056780in}}{\pgfqpoint{1.600000in}{1.056780in}}%
\pgfpathcurveto{\pgfqpoint{1.593428in}{1.056780in}}{\pgfqpoint{1.587125in}{1.054169in}}{\pgfqpoint{1.582478in}{1.049522in}}%
\pgfpathcurveto{\pgfqpoint{1.577831in}{1.044875in}}{\pgfqpoint{1.575220in}{1.038572in}}{\pgfqpoint{1.575220in}{1.032000in}}%
\pgfpathcurveto{\pgfqpoint{1.575220in}{1.025428in}}{\pgfqpoint{1.577831in}{1.019125in}}{\pgfqpoint{1.582478in}{1.014478in}}%
\pgfpathcurveto{\pgfqpoint{1.587125in}{1.009831in}}{\pgfqpoint{1.593428in}{1.007220in}}{\pgfqpoint{1.600000in}{1.007220in}}%
\pgfpathclose%
\pgfusepath{stroke,fill}%
\end{pgfscope}%
\begin{pgfscope}%
\pgfpathrectangle{\pgfqpoint{1.432000in}{0.528000in}}{\pgfqpoint{3.696000in}{3.696000in}}%
\pgfusepath{clip}%
\pgfsetbuttcap%
\pgfsetroundjoin%
\definecolor{currentfill}{rgb}{0.000000,0.000000,0.000000}%
\pgfsetfillcolor{currentfill}%
\pgfsetlinewidth{1.003750pt}%
\definecolor{currentstroke}{rgb}{0.000000,0.000000,0.000000}%
\pgfsetstrokecolor{currentstroke}%
\pgfsetdash{}{0pt}%
\pgfpathmoveto{\pgfqpoint{1.600000in}{1.343220in}}%
\pgfpathcurveto{\pgfqpoint{1.606572in}{1.343220in}}{\pgfqpoint{1.612875in}{1.345831in}}{\pgfqpoint{1.617522in}{1.350478in}}%
\pgfpathcurveto{\pgfqpoint{1.622169in}{1.355125in}}{\pgfqpoint{1.624780in}{1.361428in}}{\pgfqpoint{1.624780in}{1.368000in}}%
\pgfpathcurveto{\pgfqpoint{1.624780in}{1.374572in}}{\pgfqpoint{1.622169in}{1.380875in}}{\pgfqpoint{1.617522in}{1.385522in}}%
\pgfpathcurveto{\pgfqpoint{1.612875in}{1.390169in}}{\pgfqpoint{1.606572in}{1.392780in}}{\pgfqpoint{1.600000in}{1.392780in}}%
\pgfpathcurveto{\pgfqpoint{1.593428in}{1.392780in}}{\pgfqpoint{1.587125in}{1.390169in}}{\pgfqpoint{1.582478in}{1.385522in}}%
\pgfpathcurveto{\pgfqpoint{1.577831in}{1.380875in}}{\pgfqpoint{1.575220in}{1.374572in}}{\pgfqpoint{1.575220in}{1.368000in}}%
\pgfpathcurveto{\pgfqpoint{1.575220in}{1.361428in}}{\pgfqpoint{1.577831in}{1.355125in}}{\pgfqpoint{1.582478in}{1.350478in}}%
\pgfpathcurveto{\pgfqpoint{1.587125in}{1.345831in}}{\pgfqpoint{1.593428in}{1.343220in}}{\pgfqpoint{1.600000in}{1.343220in}}%
\pgfpathclose%
\pgfusepath{stroke,fill}%
\end{pgfscope}%
\begin{pgfscope}%
\pgfpathrectangle{\pgfqpoint{1.432000in}{0.528000in}}{\pgfqpoint{3.696000in}{3.696000in}}%
\pgfusepath{clip}%
\pgfsetbuttcap%
\pgfsetroundjoin%
\definecolor{currentfill}{rgb}{0.000000,0.000000,0.000000}%
\pgfsetfillcolor{currentfill}%
\pgfsetlinewidth{1.003750pt}%
\definecolor{currentstroke}{rgb}{0.000000,0.000000,0.000000}%
\pgfsetstrokecolor{currentstroke}%
\pgfsetdash{}{0pt}%
\pgfpathmoveto{\pgfqpoint{1.600000in}{1.679220in}}%
\pgfpathcurveto{\pgfqpoint{1.606572in}{1.679220in}}{\pgfqpoint{1.612875in}{1.681831in}}{\pgfqpoint{1.617522in}{1.686478in}}%
\pgfpathcurveto{\pgfqpoint{1.622169in}{1.691125in}}{\pgfqpoint{1.624780in}{1.697428in}}{\pgfqpoint{1.624780in}{1.704000in}}%
\pgfpathcurveto{\pgfqpoint{1.624780in}{1.710572in}}{\pgfqpoint{1.622169in}{1.716875in}}{\pgfqpoint{1.617522in}{1.721522in}}%
\pgfpathcurveto{\pgfqpoint{1.612875in}{1.726169in}}{\pgfqpoint{1.606572in}{1.728780in}}{\pgfqpoint{1.600000in}{1.728780in}}%
\pgfpathcurveto{\pgfqpoint{1.593428in}{1.728780in}}{\pgfqpoint{1.587125in}{1.726169in}}{\pgfqpoint{1.582478in}{1.721522in}}%
\pgfpathcurveto{\pgfqpoint{1.577831in}{1.716875in}}{\pgfqpoint{1.575220in}{1.710572in}}{\pgfqpoint{1.575220in}{1.704000in}}%
\pgfpathcurveto{\pgfqpoint{1.575220in}{1.697428in}}{\pgfqpoint{1.577831in}{1.691125in}}{\pgfqpoint{1.582478in}{1.686478in}}%
\pgfpathcurveto{\pgfqpoint{1.587125in}{1.681831in}}{\pgfqpoint{1.593428in}{1.679220in}}{\pgfqpoint{1.600000in}{1.679220in}}%
\pgfpathclose%
\pgfusepath{stroke,fill}%
\end{pgfscope}%
\begin{pgfscope}%
\pgfpathrectangle{\pgfqpoint{1.432000in}{0.528000in}}{\pgfqpoint{3.696000in}{3.696000in}}%
\pgfusepath{clip}%
\pgfsetbuttcap%
\pgfsetroundjoin%
\definecolor{currentfill}{rgb}{0.000000,0.000000,0.000000}%
\pgfsetfillcolor{currentfill}%
\pgfsetlinewidth{1.003750pt}%
\definecolor{currentstroke}{rgb}{0.000000,0.000000,0.000000}%
\pgfsetstrokecolor{currentstroke}%
\pgfsetdash{}{0pt}%
\pgfpathmoveto{\pgfqpoint{1.600000in}{2.015220in}}%
\pgfpathcurveto{\pgfqpoint{1.606572in}{2.015220in}}{\pgfqpoint{1.612875in}{2.017831in}}{\pgfqpoint{1.617522in}{2.022478in}}%
\pgfpathcurveto{\pgfqpoint{1.622169in}{2.027125in}}{\pgfqpoint{1.624780in}{2.033428in}}{\pgfqpoint{1.624780in}{2.040000in}}%
\pgfpathcurveto{\pgfqpoint{1.624780in}{2.046572in}}{\pgfqpoint{1.622169in}{2.052875in}}{\pgfqpoint{1.617522in}{2.057522in}}%
\pgfpathcurveto{\pgfqpoint{1.612875in}{2.062169in}}{\pgfqpoint{1.606572in}{2.064780in}}{\pgfqpoint{1.600000in}{2.064780in}}%
\pgfpathcurveto{\pgfqpoint{1.593428in}{2.064780in}}{\pgfqpoint{1.587125in}{2.062169in}}{\pgfqpoint{1.582478in}{2.057522in}}%
\pgfpathcurveto{\pgfqpoint{1.577831in}{2.052875in}}{\pgfqpoint{1.575220in}{2.046572in}}{\pgfqpoint{1.575220in}{2.040000in}}%
\pgfpathcurveto{\pgfqpoint{1.575220in}{2.033428in}}{\pgfqpoint{1.577831in}{2.027125in}}{\pgfqpoint{1.582478in}{2.022478in}}%
\pgfpathcurveto{\pgfqpoint{1.587125in}{2.017831in}}{\pgfqpoint{1.593428in}{2.015220in}}{\pgfqpoint{1.600000in}{2.015220in}}%
\pgfpathclose%
\pgfusepath{stroke,fill}%
\end{pgfscope}%
\begin{pgfscope}%
\pgfpathrectangle{\pgfqpoint{1.432000in}{0.528000in}}{\pgfqpoint{3.696000in}{3.696000in}}%
\pgfusepath{clip}%
\pgfsetbuttcap%
\pgfsetroundjoin%
\definecolor{currentfill}{rgb}{0.000000,0.000000,0.000000}%
\pgfsetfillcolor{currentfill}%
\pgfsetlinewidth{1.003750pt}%
\definecolor{currentstroke}{rgb}{0.000000,0.000000,0.000000}%
\pgfsetstrokecolor{currentstroke}%
\pgfsetdash{}{0pt}%
\pgfpathmoveto{\pgfqpoint{1.600000in}{2.351220in}}%
\pgfpathcurveto{\pgfqpoint{1.606572in}{2.351220in}}{\pgfqpoint{1.612875in}{2.353831in}}{\pgfqpoint{1.617522in}{2.358478in}}%
\pgfpathcurveto{\pgfqpoint{1.622169in}{2.363125in}}{\pgfqpoint{1.624780in}{2.369428in}}{\pgfqpoint{1.624780in}{2.376000in}}%
\pgfpathcurveto{\pgfqpoint{1.624780in}{2.382572in}}{\pgfqpoint{1.622169in}{2.388875in}}{\pgfqpoint{1.617522in}{2.393522in}}%
\pgfpathcurveto{\pgfqpoint{1.612875in}{2.398169in}}{\pgfqpoint{1.606572in}{2.400780in}}{\pgfqpoint{1.600000in}{2.400780in}}%
\pgfpathcurveto{\pgfqpoint{1.593428in}{2.400780in}}{\pgfqpoint{1.587125in}{2.398169in}}{\pgfqpoint{1.582478in}{2.393522in}}%
\pgfpathcurveto{\pgfqpoint{1.577831in}{2.388875in}}{\pgfqpoint{1.575220in}{2.382572in}}{\pgfqpoint{1.575220in}{2.376000in}}%
\pgfpathcurveto{\pgfqpoint{1.575220in}{2.369428in}}{\pgfqpoint{1.577831in}{2.363125in}}{\pgfqpoint{1.582478in}{2.358478in}}%
\pgfpathcurveto{\pgfqpoint{1.587125in}{2.353831in}}{\pgfqpoint{1.593428in}{2.351220in}}{\pgfqpoint{1.600000in}{2.351220in}}%
\pgfpathclose%
\pgfusepath{stroke,fill}%
\end{pgfscope}%
\begin{pgfscope}%
\pgfpathrectangle{\pgfqpoint{1.432000in}{0.528000in}}{\pgfqpoint{3.696000in}{3.696000in}}%
\pgfusepath{clip}%
\pgfsetbuttcap%
\pgfsetroundjoin%
\definecolor{currentfill}{rgb}{0.000000,0.000000,0.000000}%
\pgfsetfillcolor{currentfill}%
\pgfsetlinewidth{1.003750pt}%
\definecolor{currentstroke}{rgb}{0.000000,0.000000,0.000000}%
\pgfsetstrokecolor{currentstroke}%
\pgfsetdash{}{0pt}%
\pgfpathmoveto{\pgfqpoint{1.600000in}{2.687220in}}%
\pgfpathcurveto{\pgfqpoint{1.606572in}{2.687220in}}{\pgfqpoint{1.612875in}{2.689831in}}{\pgfqpoint{1.617522in}{2.694478in}}%
\pgfpathcurveto{\pgfqpoint{1.622169in}{2.699125in}}{\pgfqpoint{1.624780in}{2.705428in}}{\pgfqpoint{1.624780in}{2.712000in}}%
\pgfpathcurveto{\pgfqpoint{1.624780in}{2.718572in}}{\pgfqpoint{1.622169in}{2.724875in}}{\pgfqpoint{1.617522in}{2.729522in}}%
\pgfpathcurveto{\pgfqpoint{1.612875in}{2.734169in}}{\pgfqpoint{1.606572in}{2.736780in}}{\pgfqpoint{1.600000in}{2.736780in}}%
\pgfpathcurveto{\pgfqpoint{1.593428in}{2.736780in}}{\pgfqpoint{1.587125in}{2.734169in}}{\pgfqpoint{1.582478in}{2.729522in}}%
\pgfpathcurveto{\pgfqpoint{1.577831in}{2.724875in}}{\pgfqpoint{1.575220in}{2.718572in}}{\pgfqpoint{1.575220in}{2.712000in}}%
\pgfpathcurveto{\pgfqpoint{1.575220in}{2.705428in}}{\pgfqpoint{1.577831in}{2.699125in}}{\pgfqpoint{1.582478in}{2.694478in}}%
\pgfpathcurveto{\pgfqpoint{1.587125in}{2.689831in}}{\pgfqpoint{1.593428in}{2.687220in}}{\pgfqpoint{1.600000in}{2.687220in}}%
\pgfpathclose%
\pgfusepath{stroke,fill}%
\end{pgfscope}%
\begin{pgfscope}%
\pgfpathrectangle{\pgfqpoint{1.432000in}{0.528000in}}{\pgfqpoint{3.696000in}{3.696000in}}%
\pgfusepath{clip}%
\pgfsetbuttcap%
\pgfsetroundjoin%
\definecolor{currentfill}{rgb}{0.000000,0.000000,0.000000}%
\pgfsetfillcolor{currentfill}%
\pgfsetlinewidth{1.003750pt}%
\definecolor{currentstroke}{rgb}{0.000000,0.000000,0.000000}%
\pgfsetstrokecolor{currentstroke}%
\pgfsetdash{}{0pt}%
\pgfpathmoveto{\pgfqpoint{1.600000in}{3.023220in}}%
\pgfpathcurveto{\pgfqpoint{1.606572in}{3.023220in}}{\pgfqpoint{1.612875in}{3.025831in}}{\pgfqpoint{1.617522in}{3.030478in}}%
\pgfpathcurveto{\pgfqpoint{1.622169in}{3.035125in}}{\pgfqpoint{1.624780in}{3.041428in}}{\pgfqpoint{1.624780in}{3.048000in}}%
\pgfpathcurveto{\pgfqpoint{1.624780in}{3.054572in}}{\pgfqpoint{1.622169in}{3.060875in}}{\pgfqpoint{1.617522in}{3.065522in}}%
\pgfpathcurveto{\pgfqpoint{1.612875in}{3.070169in}}{\pgfqpoint{1.606572in}{3.072780in}}{\pgfqpoint{1.600000in}{3.072780in}}%
\pgfpathcurveto{\pgfqpoint{1.593428in}{3.072780in}}{\pgfqpoint{1.587125in}{3.070169in}}{\pgfqpoint{1.582478in}{3.065522in}}%
\pgfpathcurveto{\pgfqpoint{1.577831in}{3.060875in}}{\pgfqpoint{1.575220in}{3.054572in}}{\pgfqpoint{1.575220in}{3.048000in}}%
\pgfpathcurveto{\pgfqpoint{1.575220in}{3.041428in}}{\pgfqpoint{1.577831in}{3.035125in}}{\pgfqpoint{1.582478in}{3.030478in}}%
\pgfpathcurveto{\pgfqpoint{1.587125in}{3.025831in}}{\pgfqpoint{1.593428in}{3.023220in}}{\pgfqpoint{1.600000in}{3.023220in}}%
\pgfpathclose%
\pgfusepath{stroke,fill}%
\end{pgfscope}%
\begin{pgfscope}%
\pgfpathrectangle{\pgfqpoint{1.432000in}{0.528000in}}{\pgfqpoint{3.696000in}{3.696000in}}%
\pgfusepath{clip}%
\pgfsetbuttcap%
\pgfsetroundjoin%
\definecolor{currentfill}{rgb}{0.000000,0.000000,0.000000}%
\pgfsetfillcolor{currentfill}%
\pgfsetlinewidth{1.003750pt}%
\definecolor{currentstroke}{rgb}{0.000000,0.000000,0.000000}%
\pgfsetstrokecolor{currentstroke}%
\pgfsetdash{}{0pt}%
\pgfpathmoveto{\pgfqpoint{1.600000in}{3.359220in}}%
\pgfpathcurveto{\pgfqpoint{1.606572in}{3.359220in}}{\pgfqpoint{1.612875in}{3.361831in}}{\pgfqpoint{1.617522in}{3.366478in}}%
\pgfpathcurveto{\pgfqpoint{1.622169in}{3.371125in}}{\pgfqpoint{1.624780in}{3.377428in}}{\pgfqpoint{1.624780in}{3.384000in}}%
\pgfpathcurveto{\pgfqpoint{1.624780in}{3.390572in}}{\pgfqpoint{1.622169in}{3.396875in}}{\pgfqpoint{1.617522in}{3.401522in}}%
\pgfpathcurveto{\pgfqpoint{1.612875in}{3.406169in}}{\pgfqpoint{1.606572in}{3.408780in}}{\pgfqpoint{1.600000in}{3.408780in}}%
\pgfpathcurveto{\pgfqpoint{1.593428in}{3.408780in}}{\pgfqpoint{1.587125in}{3.406169in}}{\pgfqpoint{1.582478in}{3.401522in}}%
\pgfpathcurveto{\pgfqpoint{1.577831in}{3.396875in}}{\pgfqpoint{1.575220in}{3.390572in}}{\pgfqpoint{1.575220in}{3.384000in}}%
\pgfpathcurveto{\pgfqpoint{1.575220in}{3.377428in}}{\pgfqpoint{1.577831in}{3.371125in}}{\pgfqpoint{1.582478in}{3.366478in}}%
\pgfpathcurveto{\pgfqpoint{1.587125in}{3.361831in}}{\pgfqpoint{1.593428in}{3.359220in}}{\pgfqpoint{1.600000in}{3.359220in}}%
\pgfpathclose%
\pgfusepath{stroke,fill}%
\end{pgfscope}%
\begin{pgfscope}%
\pgfpathrectangle{\pgfqpoint{1.432000in}{0.528000in}}{\pgfqpoint{3.696000in}{3.696000in}}%
\pgfusepath{clip}%
\pgfsetbuttcap%
\pgfsetroundjoin%
\definecolor{currentfill}{rgb}{0.000000,0.000000,0.000000}%
\pgfsetfillcolor{currentfill}%
\pgfsetlinewidth{1.003750pt}%
\definecolor{currentstroke}{rgb}{0.000000,0.000000,0.000000}%
\pgfsetstrokecolor{currentstroke}%
\pgfsetdash{}{0pt}%
\pgfpathmoveto{\pgfqpoint{1.600000in}{3.695220in}}%
\pgfpathcurveto{\pgfqpoint{1.606572in}{3.695220in}}{\pgfqpoint{1.612875in}{3.697831in}}{\pgfqpoint{1.617522in}{3.702478in}}%
\pgfpathcurveto{\pgfqpoint{1.622169in}{3.707125in}}{\pgfqpoint{1.624780in}{3.713428in}}{\pgfqpoint{1.624780in}{3.720000in}}%
\pgfpathcurveto{\pgfqpoint{1.624780in}{3.726572in}}{\pgfqpoint{1.622169in}{3.732875in}}{\pgfqpoint{1.617522in}{3.737522in}}%
\pgfpathcurveto{\pgfqpoint{1.612875in}{3.742169in}}{\pgfqpoint{1.606572in}{3.744780in}}{\pgfqpoint{1.600000in}{3.744780in}}%
\pgfpathcurveto{\pgfqpoint{1.593428in}{3.744780in}}{\pgfqpoint{1.587125in}{3.742169in}}{\pgfqpoint{1.582478in}{3.737522in}}%
\pgfpathcurveto{\pgfqpoint{1.577831in}{3.732875in}}{\pgfqpoint{1.575220in}{3.726572in}}{\pgfqpoint{1.575220in}{3.720000in}}%
\pgfpathcurveto{\pgfqpoint{1.575220in}{3.713428in}}{\pgfqpoint{1.577831in}{3.707125in}}{\pgfqpoint{1.582478in}{3.702478in}}%
\pgfpathcurveto{\pgfqpoint{1.587125in}{3.697831in}}{\pgfqpoint{1.593428in}{3.695220in}}{\pgfqpoint{1.600000in}{3.695220in}}%
\pgfpathclose%
\pgfusepath{stroke,fill}%
\end{pgfscope}%
\begin{pgfscope}%
\pgfpathrectangle{\pgfqpoint{1.432000in}{0.528000in}}{\pgfqpoint{3.696000in}{3.696000in}}%
\pgfusepath{clip}%
\pgfsetbuttcap%
\pgfsetroundjoin%
\definecolor{currentfill}{rgb}{0.000000,0.000000,0.000000}%
\pgfsetfillcolor{currentfill}%
\pgfsetlinewidth{1.003750pt}%
\definecolor{currentstroke}{rgb}{0.000000,0.000000,0.000000}%
\pgfsetstrokecolor{currentstroke}%
\pgfsetdash{}{0pt}%
\pgfpathmoveto{\pgfqpoint{1.600000in}{4.031220in}}%
\pgfpathcurveto{\pgfqpoint{1.606572in}{4.031220in}}{\pgfqpoint{1.612875in}{4.033831in}}{\pgfqpoint{1.617522in}{4.038478in}}%
\pgfpathcurveto{\pgfqpoint{1.622169in}{4.043125in}}{\pgfqpoint{1.624780in}{4.049428in}}{\pgfqpoint{1.624780in}{4.056000in}}%
\pgfpathcurveto{\pgfqpoint{1.624780in}{4.062572in}}{\pgfqpoint{1.622169in}{4.068875in}}{\pgfqpoint{1.617522in}{4.073522in}}%
\pgfpathcurveto{\pgfqpoint{1.612875in}{4.078169in}}{\pgfqpoint{1.606572in}{4.080780in}}{\pgfqpoint{1.600000in}{4.080780in}}%
\pgfpathcurveto{\pgfqpoint{1.593428in}{4.080780in}}{\pgfqpoint{1.587125in}{4.078169in}}{\pgfqpoint{1.582478in}{4.073522in}}%
\pgfpathcurveto{\pgfqpoint{1.577831in}{4.068875in}}{\pgfqpoint{1.575220in}{4.062572in}}{\pgfqpoint{1.575220in}{4.056000in}}%
\pgfpathcurveto{\pgfqpoint{1.575220in}{4.049428in}}{\pgfqpoint{1.577831in}{4.043125in}}{\pgfqpoint{1.582478in}{4.038478in}}%
\pgfpathcurveto{\pgfqpoint{1.587125in}{4.033831in}}{\pgfqpoint{1.593428in}{4.031220in}}{\pgfqpoint{1.600000in}{4.031220in}}%
\pgfpathclose%
\pgfusepath{stroke,fill}%
\end{pgfscope}%
\begin{pgfscope}%
\pgfpathrectangle{\pgfqpoint{1.432000in}{0.528000in}}{\pgfqpoint{3.696000in}{3.696000in}}%
\pgfusepath{clip}%
\pgfsetbuttcap%
\pgfsetroundjoin%
\definecolor{currentfill}{rgb}{0.000000,0.000000,0.000000}%
\pgfsetfillcolor{currentfill}%
\pgfsetlinewidth{1.003750pt}%
\definecolor{currentstroke}{rgb}{0.000000,0.000000,0.000000}%
\pgfsetstrokecolor{currentstroke}%
\pgfsetdash{}{0pt}%
\pgfpathmoveto{\pgfqpoint{1.936000in}{0.671220in}}%
\pgfpathcurveto{\pgfqpoint{1.942572in}{0.671220in}}{\pgfqpoint{1.948875in}{0.673831in}}{\pgfqpoint{1.953522in}{0.678478in}}%
\pgfpathcurveto{\pgfqpoint{1.958169in}{0.683125in}}{\pgfqpoint{1.960780in}{0.689428in}}{\pgfqpoint{1.960780in}{0.696000in}}%
\pgfpathcurveto{\pgfqpoint{1.960780in}{0.702572in}}{\pgfqpoint{1.958169in}{0.708875in}}{\pgfqpoint{1.953522in}{0.713522in}}%
\pgfpathcurveto{\pgfqpoint{1.948875in}{0.718169in}}{\pgfqpoint{1.942572in}{0.720780in}}{\pgfqpoint{1.936000in}{0.720780in}}%
\pgfpathcurveto{\pgfqpoint{1.929428in}{0.720780in}}{\pgfqpoint{1.923125in}{0.718169in}}{\pgfqpoint{1.918478in}{0.713522in}}%
\pgfpathcurveto{\pgfqpoint{1.913831in}{0.708875in}}{\pgfqpoint{1.911220in}{0.702572in}}{\pgfqpoint{1.911220in}{0.696000in}}%
\pgfpathcurveto{\pgfqpoint{1.911220in}{0.689428in}}{\pgfqpoint{1.913831in}{0.683125in}}{\pgfqpoint{1.918478in}{0.678478in}}%
\pgfpathcurveto{\pgfqpoint{1.923125in}{0.673831in}}{\pgfqpoint{1.929428in}{0.671220in}}{\pgfqpoint{1.936000in}{0.671220in}}%
\pgfpathclose%
\pgfusepath{stroke,fill}%
\end{pgfscope}%
\begin{pgfscope}%
\pgfpathrectangle{\pgfqpoint{1.432000in}{0.528000in}}{\pgfqpoint{3.696000in}{3.696000in}}%
\pgfusepath{clip}%
\pgfsetbuttcap%
\pgfsetroundjoin%
\definecolor{currentfill}{rgb}{0.000000,0.000000,0.000000}%
\pgfsetfillcolor{currentfill}%
\pgfsetlinewidth{1.003750pt}%
\definecolor{currentstroke}{rgb}{0.000000,0.000000,0.000000}%
\pgfsetstrokecolor{currentstroke}%
\pgfsetdash{}{0pt}%
\pgfpathmoveto{\pgfqpoint{1.936000in}{1.007220in}}%
\pgfpathcurveto{\pgfqpoint{1.942572in}{1.007220in}}{\pgfqpoint{1.948875in}{1.009831in}}{\pgfqpoint{1.953522in}{1.014478in}}%
\pgfpathcurveto{\pgfqpoint{1.958169in}{1.019125in}}{\pgfqpoint{1.960780in}{1.025428in}}{\pgfqpoint{1.960780in}{1.032000in}}%
\pgfpathcurveto{\pgfqpoint{1.960780in}{1.038572in}}{\pgfqpoint{1.958169in}{1.044875in}}{\pgfqpoint{1.953522in}{1.049522in}}%
\pgfpathcurveto{\pgfqpoint{1.948875in}{1.054169in}}{\pgfqpoint{1.942572in}{1.056780in}}{\pgfqpoint{1.936000in}{1.056780in}}%
\pgfpathcurveto{\pgfqpoint{1.929428in}{1.056780in}}{\pgfqpoint{1.923125in}{1.054169in}}{\pgfqpoint{1.918478in}{1.049522in}}%
\pgfpathcurveto{\pgfqpoint{1.913831in}{1.044875in}}{\pgfqpoint{1.911220in}{1.038572in}}{\pgfqpoint{1.911220in}{1.032000in}}%
\pgfpathcurveto{\pgfqpoint{1.911220in}{1.025428in}}{\pgfqpoint{1.913831in}{1.019125in}}{\pgfqpoint{1.918478in}{1.014478in}}%
\pgfpathcurveto{\pgfqpoint{1.923125in}{1.009831in}}{\pgfqpoint{1.929428in}{1.007220in}}{\pgfqpoint{1.936000in}{1.007220in}}%
\pgfpathclose%
\pgfusepath{stroke,fill}%
\end{pgfscope}%
\begin{pgfscope}%
\pgfpathrectangle{\pgfqpoint{1.432000in}{0.528000in}}{\pgfqpoint{3.696000in}{3.696000in}}%
\pgfusepath{clip}%
\pgfsetbuttcap%
\pgfsetroundjoin%
\definecolor{currentfill}{rgb}{0.000000,0.000000,0.000000}%
\pgfsetfillcolor{currentfill}%
\pgfsetlinewidth{1.003750pt}%
\definecolor{currentstroke}{rgb}{0.000000,0.000000,0.000000}%
\pgfsetstrokecolor{currentstroke}%
\pgfsetdash{}{0pt}%
\pgfpathmoveto{\pgfqpoint{1.936000in}{1.343220in}}%
\pgfpathcurveto{\pgfqpoint{1.942572in}{1.343220in}}{\pgfqpoint{1.948875in}{1.345831in}}{\pgfqpoint{1.953522in}{1.350478in}}%
\pgfpathcurveto{\pgfqpoint{1.958169in}{1.355125in}}{\pgfqpoint{1.960780in}{1.361428in}}{\pgfqpoint{1.960780in}{1.368000in}}%
\pgfpathcurveto{\pgfqpoint{1.960780in}{1.374572in}}{\pgfqpoint{1.958169in}{1.380875in}}{\pgfqpoint{1.953522in}{1.385522in}}%
\pgfpathcurveto{\pgfqpoint{1.948875in}{1.390169in}}{\pgfqpoint{1.942572in}{1.392780in}}{\pgfqpoint{1.936000in}{1.392780in}}%
\pgfpathcurveto{\pgfqpoint{1.929428in}{1.392780in}}{\pgfqpoint{1.923125in}{1.390169in}}{\pgfqpoint{1.918478in}{1.385522in}}%
\pgfpathcurveto{\pgfqpoint{1.913831in}{1.380875in}}{\pgfqpoint{1.911220in}{1.374572in}}{\pgfqpoint{1.911220in}{1.368000in}}%
\pgfpathcurveto{\pgfqpoint{1.911220in}{1.361428in}}{\pgfqpoint{1.913831in}{1.355125in}}{\pgfqpoint{1.918478in}{1.350478in}}%
\pgfpathcurveto{\pgfqpoint{1.923125in}{1.345831in}}{\pgfqpoint{1.929428in}{1.343220in}}{\pgfqpoint{1.936000in}{1.343220in}}%
\pgfpathclose%
\pgfusepath{stroke,fill}%
\end{pgfscope}%
\begin{pgfscope}%
\pgfpathrectangle{\pgfqpoint{1.432000in}{0.528000in}}{\pgfqpoint{3.696000in}{3.696000in}}%
\pgfusepath{clip}%
\pgfsetbuttcap%
\pgfsetroundjoin%
\definecolor{currentfill}{rgb}{0.000000,0.000000,0.000000}%
\pgfsetfillcolor{currentfill}%
\pgfsetlinewidth{1.003750pt}%
\definecolor{currentstroke}{rgb}{0.000000,0.000000,0.000000}%
\pgfsetstrokecolor{currentstroke}%
\pgfsetdash{}{0pt}%
\pgfpathmoveto{\pgfqpoint{1.936000in}{1.679220in}}%
\pgfpathcurveto{\pgfqpoint{1.942572in}{1.679220in}}{\pgfqpoint{1.948875in}{1.681831in}}{\pgfqpoint{1.953522in}{1.686478in}}%
\pgfpathcurveto{\pgfqpoint{1.958169in}{1.691125in}}{\pgfqpoint{1.960780in}{1.697428in}}{\pgfqpoint{1.960780in}{1.704000in}}%
\pgfpathcurveto{\pgfqpoint{1.960780in}{1.710572in}}{\pgfqpoint{1.958169in}{1.716875in}}{\pgfqpoint{1.953522in}{1.721522in}}%
\pgfpathcurveto{\pgfqpoint{1.948875in}{1.726169in}}{\pgfqpoint{1.942572in}{1.728780in}}{\pgfqpoint{1.936000in}{1.728780in}}%
\pgfpathcurveto{\pgfqpoint{1.929428in}{1.728780in}}{\pgfqpoint{1.923125in}{1.726169in}}{\pgfqpoint{1.918478in}{1.721522in}}%
\pgfpathcurveto{\pgfqpoint{1.913831in}{1.716875in}}{\pgfqpoint{1.911220in}{1.710572in}}{\pgfqpoint{1.911220in}{1.704000in}}%
\pgfpathcurveto{\pgfqpoint{1.911220in}{1.697428in}}{\pgfqpoint{1.913831in}{1.691125in}}{\pgfqpoint{1.918478in}{1.686478in}}%
\pgfpathcurveto{\pgfqpoint{1.923125in}{1.681831in}}{\pgfqpoint{1.929428in}{1.679220in}}{\pgfqpoint{1.936000in}{1.679220in}}%
\pgfpathclose%
\pgfusepath{stroke,fill}%
\end{pgfscope}%
\begin{pgfscope}%
\pgfpathrectangle{\pgfqpoint{1.432000in}{0.528000in}}{\pgfqpoint{3.696000in}{3.696000in}}%
\pgfusepath{clip}%
\pgfsetbuttcap%
\pgfsetroundjoin%
\definecolor{currentfill}{rgb}{0.000000,0.000000,0.000000}%
\pgfsetfillcolor{currentfill}%
\pgfsetlinewidth{1.003750pt}%
\definecolor{currentstroke}{rgb}{0.000000,0.000000,0.000000}%
\pgfsetstrokecolor{currentstroke}%
\pgfsetdash{}{0pt}%
\pgfpathmoveto{\pgfqpoint{1.936000in}{2.015220in}}%
\pgfpathcurveto{\pgfqpoint{1.942572in}{2.015220in}}{\pgfqpoint{1.948875in}{2.017831in}}{\pgfqpoint{1.953522in}{2.022478in}}%
\pgfpathcurveto{\pgfqpoint{1.958169in}{2.027125in}}{\pgfqpoint{1.960780in}{2.033428in}}{\pgfqpoint{1.960780in}{2.040000in}}%
\pgfpathcurveto{\pgfqpoint{1.960780in}{2.046572in}}{\pgfqpoint{1.958169in}{2.052875in}}{\pgfqpoint{1.953522in}{2.057522in}}%
\pgfpathcurveto{\pgfqpoint{1.948875in}{2.062169in}}{\pgfqpoint{1.942572in}{2.064780in}}{\pgfqpoint{1.936000in}{2.064780in}}%
\pgfpathcurveto{\pgfqpoint{1.929428in}{2.064780in}}{\pgfqpoint{1.923125in}{2.062169in}}{\pgfqpoint{1.918478in}{2.057522in}}%
\pgfpathcurveto{\pgfqpoint{1.913831in}{2.052875in}}{\pgfqpoint{1.911220in}{2.046572in}}{\pgfqpoint{1.911220in}{2.040000in}}%
\pgfpathcurveto{\pgfqpoint{1.911220in}{2.033428in}}{\pgfqpoint{1.913831in}{2.027125in}}{\pgfqpoint{1.918478in}{2.022478in}}%
\pgfpathcurveto{\pgfqpoint{1.923125in}{2.017831in}}{\pgfqpoint{1.929428in}{2.015220in}}{\pgfqpoint{1.936000in}{2.015220in}}%
\pgfpathclose%
\pgfusepath{stroke,fill}%
\end{pgfscope}%
\begin{pgfscope}%
\pgfpathrectangle{\pgfqpoint{1.432000in}{0.528000in}}{\pgfqpoint{3.696000in}{3.696000in}}%
\pgfusepath{clip}%
\pgfsetbuttcap%
\pgfsetroundjoin%
\definecolor{currentfill}{rgb}{0.000000,0.000000,0.000000}%
\pgfsetfillcolor{currentfill}%
\pgfsetlinewidth{1.003750pt}%
\definecolor{currentstroke}{rgb}{0.000000,0.000000,0.000000}%
\pgfsetstrokecolor{currentstroke}%
\pgfsetdash{}{0pt}%
\pgfpathmoveto{\pgfqpoint{1.936000in}{2.351220in}}%
\pgfpathcurveto{\pgfqpoint{1.942572in}{2.351220in}}{\pgfqpoint{1.948875in}{2.353831in}}{\pgfqpoint{1.953522in}{2.358478in}}%
\pgfpathcurveto{\pgfqpoint{1.958169in}{2.363125in}}{\pgfqpoint{1.960780in}{2.369428in}}{\pgfqpoint{1.960780in}{2.376000in}}%
\pgfpathcurveto{\pgfqpoint{1.960780in}{2.382572in}}{\pgfqpoint{1.958169in}{2.388875in}}{\pgfqpoint{1.953522in}{2.393522in}}%
\pgfpathcurveto{\pgfqpoint{1.948875in}{2.398169in}}{\pgfqpoint{1.942572in}{2.400780in}}{\pgfqpoint{1.936000in}{2.400780in}}%
\pgfpathcurveto{\pgfqpoint{1.929428in}{2.400780in}}{\pgfqpoint{1.923125in}{2.398169in}}{\pgfqpoint{1.918478in}{2.393522in}}%
\pgfpathcurveto{\pgfqpoint{1.913831in}{2.388875in}}{\pgfqpoint{1.911220in}{2.382572in}}{\pgfqpoint{1.911220in}{2.376000in}}%
\pgfpathcurveto{\pgfqpoint{1.911220in}{2.369428in}}{\pgfqpoint{1.913831in}{2.363125in}}{\pgfqpoint{1.918478in}{2.358478in}}%
\pgfpathcurveto{\pgfqpoint{1.923125in}{2.353831in}}{\pgfqpoint{1.929428in}{2.351220in}}{\pgfqpoint{1.936000in}{2.351220in}}%
\pgfpathclose%
\pgfusepath{stroke,fill}%
\end{pgfscope}%
\begin{pgfscope}%
\pgfpathrectangle{\pgfqpoint{1.432000in}{0.528000in}}{\pgfqpoint{3.696000in}{3.696000in}}%
\pgfusepath{clip}%
\pgfsetbuttcap%
\pgfsetroundjoin%
\definecolor{currentfill}{rgb}{0.000000,0.000000,0.000000}%
\pgfsetfillcolor{currentfill}%
\pgfsetlinewidth{1.003750pt}%
\definecolor{currentstroke}{rgb}{0.000000,0.000000,0.000000}%
\pgfsetstrokecolor{currentstroke}%
\pgfsetdash{}{0pt}%
\pgfpathmoveto{\pgfqpoint{1.936000in}{2.687220in}}%
\pgfpathcurveto{\pgfqpoint{1.942572in}{2.687220in}}{\pgfqpoint{1.948875in}{2.689831in}}{\pgfqpoint{1.953522in}{2.694478in}}%
\pgfpathcurveto{\pgfqpoint{1.958169in}{2.699125in}}{\pgfqpoint{1.960780in}{2.705428in}}{\pgfqpoint{1.960780in}{2.712000in}}%
\pgfpathcurveto{\pgfqpoint{1.960780in}{2.718572in}}{\pgfqpoint{1.958169in}{2.724875in}}{\pgfqpoint{1.953522in}{2.729522in}}%
\pgfpathcurveto{\pgfqpoint{1.948875in}{2.734169in}}{\pgfqpoint{1.942572in}{2.736780in}}{\pgfqpoint{1.936000in}{2.736780in}}%
\pgfpathcurveto{\pgfqpoint{1.929428in}{2.736780in}}{\pgfqpoint{1.923125in}{2.734169in}}{\pgfqpoint{1.918478in}{2.729522in}}%
\pgfpathcurveto{\pgfqpoint{1.913831in}{2.724875in}}{\pgfqpoint{1.911220in}{2.718572in}}{\pgfqpoint{1.911220in}{2.712000in}}%
\pgfpathcurveto{\pgfqpoint{1.911220in}{2.705428in}}{\pgfqpoint{1.913831in}{2.699125in}}{\pgfqpoint{1.918478in}{2.694478in}}%
\pgfpathcurveto{\pgfqpoint{1.923125in}{2.689831in}}{\pgfqpoint{1.929428in}{2.687220in}}{\pgfqpoint{1.936000in}{2.687220in}}%
\pgfpathclose%
\pgfusepath{stroke,fill}%
\end{pgfscope}%
\begin{pgfscope}%
\pgfpathrectangle{\pgfqpoint{1.432000in}{0.528000in}}{\pgfqpoint{3.696000in}{3.696000in}}%
\pgfusepath{clip}%
\pgfsetbuttcap%
\pgfsetroundjoin%
\definecolor{currentfill}{rgb}{0.000000,0.000000,0.000000}%
\pgfsetfillcolor{currentfill}%
\pgfsetlinewidth{1.003750pt}%
\definecolor{currentstroke}{rgb}{0.000000,0.000000,0.000000}%
\pgfsetstrokecolor{currentstroke}%
\pgfsetdash{}{0pt}%
\pgfpathmoveto{\pgfqpoint{1.936000in}{3.023220in}}%
\pgfpathcurveto{\pgfqpoint{1.942572in}{3.023220in}}{\pgfqpoint{1.948875in}{3.025831in}}{\pgfqpoint{1.953522in}{3.030478in}}%
\pgfpathcurveto{\pgfqpoint{1.958169in}{3.035125in}}{\pgfqpoint{1.960780in}{3.041428in}}{\pgfqpoint{1.960780in}{3.048000in}}%
\pgfpathcurveto{\pgfqpoint{1.960780in}{3.054572in}}{\pgfqpoint{1.958169in}{3.060875in}}{\pgfqpoint{1.953522in}{3.065522in}}%
\pgfpathcurveto{\pgfqpoint{1.948875in}{3.070169in}}{\pgfqpoint{1.942572in}{3.072780in}}{\pgfqpoint{1.936000in}{3.072780in}}%
\pgfpathcurveto{\pgfqpoint{1.929428in}{3.072780in}}{\pgfqpoint{1.923125in}{3.070169in}}{\pgfqpoint{1.918478in}{3.065522in}}%
\pgfpathcurveto{\pgfqpoint{1.913831in}{3.060875in}}{\pgfqpoint{1.911220in}{3.054572in}}{\pgfqpoint{1.911220in}{3.048000in}}%
\pgfpathcurveto{\pgfqpoint{1.911220in}{3.041428in}}{\pgfqpoint{1.913831in}{3.035125in}}{\pgfqpoint{1.918478in}{3.030478in}}%
\pgfpathcurveto{\pgfqpoint{1.923125in}{3.025831in}}{\pgfqpoint{1.929428in}{3.023220in}}{\pgfqpoint{1.936000in}{3.023220in}}%
\pgfpathclose%
\pgfusepath{stroke,fill}%
\end{pgfscope}%
\begin{pgfscope}%
\pgfpathrectangle{\pgfqpoint{1.432000in}{0.528000in}}{\pgfqpoint{3.696000in}{3.696000in}}%
\pgfusepath{clip}%
\pgfsetbuttcap%
\pgfsetroundjoin%
\definecolor{currentfill}{rgb}{0.000000,0.000000,0.000000}%
\pgfsetfillcolor{currentfill}%
\pgfsetlinewidth{1.003750pt}%
\definecolor{currentstroke}{rgb}{0.000000,0.000000,0.000000}%
\pgfsetstrokecolor{currentstroke}%
\pgfsetdash{}{0pt}%
\pgfpathmoveto{\pgfqpoint{1.936000in}{3.359220in}}%
\pgfpathcurveto{\pgfqpoint{1.942572in}{3.359220in}}{\pgfqpoint{1.948875in}{3.361831in}}{\pgfqpoint{1.953522in}{3.366478in}}%
\pgfpathcurveto{\pgfqpoint{1.958169in}{3.371125in}}{\pgfqpoint{1.960780in}{3.377428in}}{\pgfqpoint{1.960780in}{3.384000in}}%
\pgfpathcurveto{\pgfqpoint{1.960780in}{3.390572in}}{\pgfqpoint{1.958169in}{3.396875in}}{\pgfqpoint{1.953522in}{3.401522in}}%
\pgfpathcurveto{\pgfqpoint{1.948875in}{3.406169in}}{\pgfqpoint{1.942572in}{3.408780in}}{\pgfqpoint{1.936000in}{3.408780in}}%
\pgfpathcurveto{\pgfqpoint{1.929428in}{3.408780in}}{\pgfqpoint{1.923125in}{3.406169in}}{\pgfqpoint{1.918478in}{3.401522in}}%
\pgfpathcurveto{\pgfqpoint{1.913831in}{3.396875in}}{\pgfqpoint{1.911220in}{3.390572in}}{\pgfqpoint{1.911220in}{3.384000in}}%
\pgfpathcurveto{\pgfqpoint{1.911220in}{3.377428in}}{\pgfqpoint{1.913831in}{3.371125in}}{\pgfqpoint{1.918478in}{3.366478in}}%
\pgfpathcurveto{\pgfqpoint{1.923125in}{3.361831in}}{\pgfqpoint{1.929428in}{3.359220in}}{\pgfqpoint{1.936000in}{3.359220in}}%
\pgfpathclose%
\pgfusepath{stroke,fill}%
\end{pgfscope}%
\begin{pgfscope}%
\pgfpathrectangle{\pgfqpoint{1.432000in}{0.528000in}}{\pgfqpoint{3.696000in}{3.696000in}}%
\pgfusepath{clip}%
\pgfsetbuttcap%
\pgfsetroundjoin%
\definecolor{currentfill}{rgb}{0.000000,0.000000,0.000000}%
\pgfsetfillcolor{currentfill}%
\pgfsetlinewidth{1.003750pt}%
\definecolor{currentstroke}{rgb}{0.000000,0.000000,0.000000}%
\pgfsetstrokecolor{currentstroke}%
\pgfsetdash{}{0pt}%
\pgfpathmoveto{\pgfqpoint{1.936000in}{3.695220in}}%
\pgfpathcurveto{\pgfqpoint{1.942572in}{3.695220in}}{\pgfqpoint{1.948875in}{3.697831in}}{\pgfqpoint{1.953522in}{3.702478in}}%
\pgfpathcurveto{\pgfqpoint{1.958169in}{3.707125in}}{\pgfqpoint{1.960780in}{3.713428in}}{\pgfqpoint{1.960780in}{3.720000in}}%
\pgfpathcurveto{\pgfqpoint{1.960780in}{3.726572in}}{\pgfqpoint{1.958169in}{3.732875in}}{\pgfqpoint{1.953522in}{3.737522in}}%
\pgfpathcurveto{\pgfqpoint{1.948875in}{3.742169in}}{\pgfqpoint{1.942572in}{3.744780in}}{\pgfqpoint{1.936000in}{3.744780in}}%
\pgfpathcurveto{\pgfqpoint{1.929428in}{3.744780in}}{\pgfqpoint{1.923125in}{3.742169in}}{\pgfqpoint{1.918478in}{3.737522in}}%
\pgfpathcurveto{\pgfqpoint{1.913831in}{3.732875in}}{\pgfqpoint{1.911220in}{3.726572in}}{\pgfqpoint{1.911220in}{3.720000in}}%
\pgfpathcurveto{\pgfqpoint{1.911220in}{3.713428in}}{\pgfqpoint{1.913831in}{3.707125in}}{\pgfqpoint{1.918478in}{3.702478in}}%
\pgfpathcurveto{\pgfqpoint{1.923125in}{3.697831in}}{\pgfqpoint{1.929428in}{3.695220in}}{\pgfqpoint{1.936000in}{3.695220in}}%
\pgfpathclose%
\pgfusepath{stroke,fill}%
\end{pgfscope}%
\begin{pgfscope}%
\pgfpathrectangle{\pgfqpoint{1.432000in}{0.528000in}}{\pgfqpoint{3.696000in}{3.696000in}}%
\pgfusepath{clip}%
\pgfsetbuttcap%
\pgfsetroundjoin%
\definecolor{currentfill}{rgb}{0.000000,0.000000,0.000000}%
\pgfsetfillcolor{currentfill}%
\pgfsetlinewidth{1.003750pt}%
\definecolor{currentstroke}{rgb}{0.000000,0.000000,0.000000}%
\pgfsetstrokecolor{currentstroke}%
\pgfsetdash{}{0pt}%
\pgfpathmoveto{\pgfqpoint{1.936000in}{4.031220in}}%
\pgfpathcurveto{\pgfqpoint{1.942572in}{4.031220in}}{\pgfqpoint{1.948875in}{4.033831in}}{\pgfqpoint{1.953522in}{4.038478in}}%
\pgfpathcurveto{\pgfqpoint{1.958169in}{4.043125in}}{\pgfqpoint{1.960780in}{4.049428in}}{\pgfqpoint{1.960780in}{4.056000in}}%
\pgfpathcurveto{\pgfqpoint{1.960780in}{4.062572in}}{\pgfqpoint{1.958169in}{4.068875in}}{\pgfqpoint{1.953522in}{4.073522in}}%
\pgfpathcurveto{\pgfqpoint{1.948875in}{4.078169in}}{\pgfqpoint{1.942572in}{4.080780in}}{\pgfqpoint{1.936000in}{4.080780in}}%
\pgfpathcurveto{\pgfqpoint{1.929428in}{4.080780in}}{\pgfqpoint{1.923125in}{4.078169in}}{\pgfqpoint{1.918478in}{4.073522in}}%
\pgfpathcurveto{\pgfqpoint{1.913831in}{4.068875in}}{\pgfqpoint{1.911220in}{4.062572in}}{\pgfqpoint{1.911220in}{4.056000in}}%
\pgfpathcurveto{\pgfqpoint{1.911220in}{4.049428in}}{\pgfqpoint{1.913831in}{4.043125in}}{\pgfqpoint{1.918478in}{4.038478in}}%
\pgfpathcurveto{\pgfqpoint{1.923125in}{4.033831in}}{\pgfqpoint{1.929428in}{4.031220in}}{\pgfqpoint{1.936000in}{4.031220in}}%
\pgfpathclose%
\pgfusepath{stroke,fill}%
\end{pgfscope}%
\begin{pgfscope}%
\pgfpathrectangle{\pgfqpoint{1.432000in}{0.528000in}}{\pgfqpoint{3.696000in}{3.696000in}}%
\pgfusepath{clip}%
\pgfsetbuttcap%
\pgfsetroundjoin%
\definecolor{currentfill}{rgb}{0.000000,0.000000,0.000000}%
\pgfsetfillcolor{currentfill}%
\pgfsetlinewidth{1.003750pt}%
\definecolor{currentstroke}{rgb}{0.000000,0.000000,0.000000}%
\pgfsetstrokecolor{currentstroke}%
\pgfsetdash{}{0pt}%
\pgfpathmoveto{\pgfqpoint{2.272000in}{0.671220in}}%
\pgfpathcurveto{\pgfqpoint{2.278572in}{0.671220in}}{\pgfqpoint{2.284875in}{0.673831in}}{\pgfqpoint{2.289522in}{0.678478in}}%
\pgfpathcurveto{\pgfqpoint{2.294169in}{0.683125in}}{\pgfqpoint{2.296780in}{0.689428in}}{\pgfqpoint{2.296780in}{0.696000in}}%
\pgfpathcurveto{\pgfqpoint{2.296780in}{0.702572in}}{\pgfqpoint{2.294169in}{0.708875in}}{\pgfqpoint{2.289522in}{0.713522in}}%
\pgfpathcurveto{\pgfqpoint{2.284875in}{0.718169in}}{\pgfqpoint{2.278572in}{0.720780in}}{\pgfqpoint{2.272000in}{0.720780in}}%
\pgfpathcurveto{\pgfqpoint{2.265428in}{0.720780in}}{\pgfqpoint{2.259125in}{0.718169in}}{\pgfqpoint{2.254478in}{0.713522in}}%
\pgfpathcurveto{\pgfqpoint{2.249831in}{0.708875in}}{\pgfqpoint{2.247220in}{0.702572in}}{\pgfqpoint{2.247220in}{0.696000in}}%
\pgfpathcurveto{\pgfqpoint{2.247220in}{0.689428in}}{\pgfqpoint{2.249831in}{0.683125in}}{\pgfqpoint{2.254478in}{0.678478in}}%
\pgfpathcurveto{\pgfqpoint{2.259125in}{0.673831in}}{\pgfqpoint{2.265428in}{0.671220in}}{\pgfqpoint{2.272000in}{0.671220in}}%
\pgfpathclose%
\pgfusepath{stroke,fill}%
\end{pgfscope}%
\begin{pgfscope}%
\pgfpathrectangle{\pgfqpoint{1.432000in}{0.528000in}}{\pgfqpoint{3.696000in}{3.696000in}}%
\pgfusepath{clip}%
\pgfsetbuttcap%
\pgfsetroundjoin%
\definecolor{currentfill}{rgb}{0.000000,0.000000,0.000000}%
\pgfsetfillcolor{currentfill}%
\pgfsetlinewidth{1.003750pt}%
\definecolor{currentstroke}{rgb}{0.000000,0.000000,0.000000}%
\pgfsetstrokecolor{currentstroke}%
\pgfsetdash{}{0pt}%
\pgfpathmoveto{\pgfqpoint{2.272000in}{1.007220in}}%
\pgfpathcurveto{\pgfqpoint{2.278572in}{1.007220in}}{\pgfqpoint{2.284875in}{1.009831in}}{\pgfqpoint{2.289522in}{1.014478in}}%
\pgfpathcurveto{\pgfqpoint{2.294169in}{1.019125in}}{\pgfqpoint{2.296780in}{1.025428in}}{\pgfqpoint{2.296780in}{1.032000in}}%
\pgfpathcurveto{\pgfqpoint{2.296780in}{1.038572in}}{\pgfqpoint{2.294169in}{1.044875in}}{\pgfqpoint{2.289522in}{1.049522in}}%
\pgfpathcurveto{\pgfqpoint{2.284875in}{1.054169in}}{\pgfqpoint{2.278572in}{1.056780in}}{\pgfqpoint{2.272000in}{1.056780in}}%
\pgfpathcurveto{\pgfqpoint{2.265428in}{1.056780in}}{\pgfqpoint{2.259125in}{1.054169in}}{\pgfqpoint{2.254478in}{1.049522in}}%
\pgfpathcurveto{\pgfqpoint{2.249831in}{1.044875in}}{\pgfqpoint{2.247220in}{1.038572in}}{\pgfqpoint{2.247220in}{1.032000in}}%
\pgfpathcurveto{\pgfqpoint{2.247220in}{1.025428in}}{\pgfqpoint{2.249831in}{1.019125in}}{\pgfqpoint{2.254478in}{1.014478in}}%
\pgfpathcurveto{\pgfqpoint{2.259125in}{1.009831in}}{\pgfqpoint{2.265428in}{1.007220in}}{\pgfqpoint{2.272000in}{1.007220in}}%
\pgfpathclose%
\pgfusepath{stroke,fill}%
\end{pgfscope}%
\begin{pgfscope}%
\pgfpathrectangle{\pgfqpoint{1.432000in}{0.528000in}}{\pgfqpoint{3.696000in}{3.696000in}}%
\pgfusepath{clip}%
\pgfsetbuttcap%
\pgfsetroundjoin%
\definecolor{currentfill}{rgb}{0.000000,0.000000,0.000000}%
\pgfsetfillcolor{currentfill}%
\pgfsetlinewidth{1.003750pt}%
\definecolor{currentstroke}{rgb}{0.000000,0.000000,0.000000}%
\pgfsetstrokecolor{currentstroke}%
\pgfsetdash{}{0pt}%
\pgfpathmoveto{\pgfqpoint{2.272000in}{1.343220in}}%
\pgfpathcurveto{\pgfqpoint{2.278572in}{1.343220in}}{\pgfqpoint{2.284875in}{1.345831in}}{\pgfqpoint{2.289522in}{1.350478in}}%
\pgfpathcurveto{\pgfqpoint{2.294169in}{1.355125in}}{\pgfqpoint{2.296780in}{1.361428in}}{\pgfqpoint{2.296780in}{1.368000in}}%
\pgfpathcurveto{\pgfqpoint{2.296780in}{1.374572in}}{\pgfqpoint{2.294169in}{1.380875in}}{\pgfqpoint{2.289522in}{1.385522in}}%
\pgfpathcurveto{\pgfqpoint{2.284875in}{1.390169in}}{\pgfqpoint{2.278572in}{1.392780in}}{\pgfqpoint{2.272000in}{1.392780in}}%
\pgfpathcurveto{\pgfqpoint{2.265428in}{1.392780in}}{\pgfqpoint{2.259125in}{1.390169in}}{\pgfqpoint{2.254478in}{1.385522in}}%
\pgfpathcurveto{\pgfqpoint{2.249831in}{1.380875in}}{\pgfqpoint{2.247220in}{1.374572in}}{\pgfqpoint{2.247220in}{1.368000in}}%
\pgfpathcurveto{\pgfqpoint{2.247220in}{1.361428in}}{\pgfqpoint{2.249831in}{1.355125in}}{\pgfqpoint{2.254478in}{1.350478in}}%
\pgfpathcurveto{\pgfqpoint{2.259125in}{1.345831in}}{\pgfqpoint{2.265428in}{1.343220in}}{\pgfqpoint{2.272000in}{1.343220in}}%
\pgfpathclose%
\pgfusepath{stroke,fill}%
\end{pgfscope}%
\begin{pgfscope}%
\pgfpathrectangle{\pgfqpoint{1.432000in}{0.528000in}}{\pgfqpoint{3.696000in}{3.696000in}}%
\pgfusepath{clip}%
\pgfsetbuttcap%
\pgfsetroundjoin%
\definecolor{currentfill}{rgb}{0.000000,0.000000,0.000000}%
\pgfsetfillcolor{currentfill}%
\pgfsetlinewidth{1.003750pt}%
\definecolor{currentstroke}{rgb}{0.000000,0.000000,0.000000}%
\pgfsetstrokecolor{currentstroke}%
\pgfsetdash{}{0pt}%
\pgfpathmoveto{\pgfqpoint{2.272000in}{1.679220in}}%
\pgfpathcurveto{\pgfqpoint{2.278572in}{1.679220in}}{\pgfqpoint{2.284875in}{1.681831in}}{\pgfqpoint{2.289522in}{1.686478in}}%
\pgfpathcurveto{\pgfqpoint{2.294169in}{1.691125in}}{\pgfqpoint{2.296780in}{1.697428in}}{\pgfqpoint{2.296780in}{1.704000in}}%
\pgfpathcurveto{\pgfqpoint{2.296780in}{1.710572in}}{\pgfqpoint{2.294169in}{1.716875in}}{\pgfqpoint{2.289522in}{1.721522in}}%
\pgfpathcurveto{\pgfqpoint{2.284875in}{1.726169in}}{\pgfqpoint{2.278572in}{1.728780in}}{\pgfqpoint{2.272000in}{1.728780in}}%
\pgfpathcurveto{\pgfqpoint{2.265428in}{1.728780in}}{\pgfqpoint{2.259125in}{1.726169in}}{\pgfqpoint{2.254478in}{1.721522in}}%
\pgfpathcurveto{\pgfqpoint{2.249831in}{1.716875in}}{\pgfqpoint{2.247220in}{1.710572in}}{\pgfqpoint{2.247220in}{1.704000in}}%
\pgfpathcurveto{\pgfqpoint{2.247220in}{1.697428in}}{\pgfqpoint{2.249831in}{1.691125in}}{\pgfqpoint{2.254478in}{1.686478in}}%
\pgfpathcurveto{\pgfqpoint{2.259125in}{1.681831in}}{\pgfqpoint{2.265428in}{1.679220in}}{\pgfqpoint{2.272000in}{1.679220in}}%
\pgfpathclose%
\pgfusepath{stroke,fill}%
\end{pgfscope}%
\begin{pgfscope}%
\pgfpathrectangle{\pgfqpoint{1.432000in}{0.528000in}}{\pgfqpoint{3.696000in}{3.696000in}}%
\pgfusepath{clip}%
\pgfsetbuttcap%
\pgfsetroundjoin%
\definecolor{currentfill}{rgb}{0.000000,0.000000,0.000000}%
\pgfsetfillcolor{currentfill}%
\pgfsetlinewidth{1.003750pt}%
\definecolor{currentstroke}{rgb}{0.000000,0.000000,0.000000}%
\pgfsetstrokecolor{currentstroke}%
\pgfsetdash{}{0pt}%
\pgfpathmoveto{\pgfqpoint{2.272000in}{2.015220in}}%
\pgfpathcurveto{\pgfqpoint{2.278572in}{2.015220in}}{\pgfqpoint{2.284875in}{2.017831in}}{\pgfqpoint{2.289522in}{2.022478in}}%
\pgfpathcurveto{\pgfqpoint{2.294169in}{2.027125in}}{\pgfqpoint{2.296780in}{2.033428in}}{\pgfqpoint{2.296780in}{2.040000in}}%
\pgfpathcurveto{\pgfqpoint{2.296780in}{2.046572in}}{\pgfqpoint{2.294169in}{2.052875in}}{\pgfqpoint{2.289522in}{2.057522in}}%
\pgfpathcurveto{\pgfqpoint{2.284875in}{2.062169in}}{\pgfqpoint{2.278572in}{2.064780in}}{\pgfqpoint{2.272000in}{2.064780in}}%
\pgfpathcurveto{\pgfqpoint{2.265428in}{2.064780in}}{\pgfqpoint{2.259125in}{2.062169in}}{\pgfqpoint{2.254478in}{2.057522in}}%
\pgfpathcurveto{\pgfqpoint{2.249831in}{2.052875in}}{\pgfqpoint{2.247220in}{2.046572in}}{\pgfqpoint{2.247220in}{2.040000in}}%
\pgfpathcurveto{\pgfqpoint{2.247220in}{2.033428in}}{\pgfqpoint{2.249831in}{2.027125in}}{\pgfqpoint{2.254478in}{2.022478in}}%
\pgfpathcurveto{\pgfqpoint{2.259125in}{2.017831in}}{\pgfqpoint{2.265428in}{2.015220in}}{\pgfqpoint{2.272000in}{2.015220in}}%
\pgfpathclose%
\pgfusepath{stroke,fill}%
\end{pgfscope}%
\begin{pgfscope}%
\pgfpathrectangle{\pgfqpoint{1.432000in}{0.528000in}}{\pgfqpoint{3.696000in}{3.696000in}}%
\pgfusepath{clip}%
\pgfsetbuttcap%
\pgfsetroundjoin%
\definecolor{currentfill}{rgb}{0.000000,0.000000,0.000000}%
\pgfsetfillcolor{currentfill}%
\pgfsetlinewidth{1.003750pt}%
\definecolor{currentstroke}{rgb}{0.000000,0.000000,0.000000}%
\pgfsetstrokecolor{currentstroke}%
\pgfsetdash{}{0pt}%
\pgfpathmoveto{\pgfqpoint{2.272000in}{2.351220in}}%
\pgfpathcurveto{\pgfqpoint{2.278572in}{2.351220in}}{\pgfqpoint{2.284875in}{2.353831in}}{\pgfqpoint{2.289522in}{2.358478in}}%
\pgfpathcurveto{\pgfqpoint{2.294169in}{2.363125in}}{\pgfqpoint{2.296780in}{2.369428in}}{\pgfqpoint{2.296780in}{2.376000in}}%
\pgfpathcurveto{\pgfqpoint{2.296780in}{2.382572in}}{\pgfqpoint{2.294169in}{2.388875in}}{\pgfqpoint{2.289522in}{2.393522in}}%
\pgfpathcurveto{\pgfqpoint{2.284875in}{2.398169in}}{\pgfqpoint{2.278572in}{2.400780in}}{\pgfqpoint{2.272000in}{2.400780in}}%
\pgfpathcurveto{\pgfqpoint{2.265428in}{2.400780in}}{\pgfqpoint{2.259125in}{2.398169in}}{\pgfqpoint{2.254478in}{2.393522in}}%
\pgfpathcurveto{\pgfqpoint{2.249831in}{2.388875in}}{\pgfqpoint{2.247220in}{2.382572in}}{\pgfqpoint{2.247220in}{2.376000in}}%
\pgfpathcurveto{\pgfqpoint{2.247220in}{2.369428in}}{\pgfqpoint{2.249831in}{2.363125in}}{\pgfqpoint{2.254478in}{2.358478in}}%
\pgfpathcurveto{\pgfqpoint{2.259125in}{2.353831in}}{\pgfqpoint{2.265428in}{2.351220in}}{\pgfqpoint{2.272000in}{2.351220in}}%
\pgfpathclose%
\pgfusepath{stroke,fill}%
\end{pgfscope}%
\begin{pgfscope}%
\pgfpathrectangle{\pgfqpoint{1.432000in}{0.528000in}}{\pgfqpoint{3.696000in}{3.696000in}}%
\pgfusepath{clip}%
\pgfsetbuttcap%
\pgfsetroundjoin%
\definecolor{currentfill}{rgb}{0.000000,0.000000,0.000000}%
\pgfsetfillcolor{currentfill}%
\pgfsetlinewidth{1.003750pt}%
\definecolor{currentstroke}{rgb}{0.000000,0.000000,0.000000}%
\pgfsetstrokecolor{currentstroke}%
\pgfsetdash{}{0pt}%
\pgfpathmoveto{\pgfqpoint{2.272000in}{2.687220in}}%
\pgfpathcurveto{\pgfqpoint{2.278572in}{2.687220in}}{\pgfqpoint{2.284875in}{2.689831in}}{\pgfqpoint{2.289522in}{2.694478in}}%
\pgfpathcurveto{\pgfqpoint{2.294169in}{2.699125in}}{\pgfqpoint{2.296780in}{2.705428in}}{\pgfqpoint{2.296780in}{2.712000in}}%
\pgfpathcurveto{\pgfqpoint{2.296780in}{2.718572in}}{\pgfqpoint{2.294169in}{2.724875in}}{\pgfqpoint{2.289522in}{2.729522in}}%
\pgfpathcurveto{\pgfqpoint{2.284875in}{2.734169in}}{\pgfqpoint{2.278572in}{2.736780in}}{\pgfqpoint{2.272000in}{2.736780in}}%
\pgfpathcurveto{\pgfqpoint{2.265428in}{2.736780in}}{\pgfqpoint{2.259125in}{2.734169in}}{\pgfqpoint{2.254478in}{2.729522in}}%
\pgfpathcurveto{\pgfqpoint{2.249831in}{2.724875in}}{\pgfqpoint{2.247220in}{2.718572in}}{\pgfqpoint{2.247220in}{2.712000in}}%
\pgfpathcurveto{\pgfqpoint{2.247220in}{2.705428in}}{\pgfqpoint{2.249831in}{2.699125in}}{\pgfqpoint{2.254478in}{2.694478in}}%
\pgfpathcurveto{\pgfqpoint{2.259125in}{2.689831in}}{\pgfqpoint{2.265428in}{2.687220in}}{\pgfqpoint{2.272000in}{2.687220in}}%
\pgfpathclose%
\pgfusepath{stroke,fill}%
\end{pgfscope}%
\begin{pgfscope}%
\pgfpathrectangle{\pgfqpoint{1.432000in}{0.528000in}}{\pgfqpoint{3.696000in}{3.696000in}}%
\pgfusepath{clip}%
\pgfsetbuttcap%
\pgfsetroundjoin%
\definecolor{currentfill}{rgb}{0.000000,0.000000,0.000000}%
\pgfsetfillcolor{currentfill}%
\pgfsetlinewidth{1.003750pt}%
\definecolor{currentstroke}{rgb}{0.000000,0.000000,0.000000}%
\pgfsetstrokecolor{currentstroke}%
\pgfsetdash{}{0pt}%
\pgfpathmoveto{\pgfqpoint{2.272000in}{3.023220in}}%
\pgfpathcurveto{\pgfqpoint{2.278572in}{3.023220in}}{\pgfqpoint{2.284875in}{3.025831in}}{\pgfqpoint{2.289522in}{3.030478in}}%
\pgfpathcurveto{\pgfqpoint{2.294169in}{3.035125in}}{\pgfqpoint{2.296780in}{3.041428in}}{\pgfqpoint{2.296780in}{3.048000in}}%
\pgfpathcurveto{\pgfqpoint{2.296780in}{3.054572in}}{\pgfqpoint{2.294169in}{3.060875in}}{\pgfqpoint{2.289522in}{3.065522in}}%
\pgfpathcurveto{\pgfqpoint{2.284875in}{3.070169in}}{\pgfqpoint{2.278572in}{3.072780in}}{\pgfqpoint{2.272000in}{3.072780in}}%
\pgfpathcurveto{\pgfqpoint{2.265428in}{3.072780in}}{\pgfqpoint{2.259125in}{3.070169in}}{\pgfqpoint{2.254478in}{3.065522in}}%
\pgfpathcurveto{\pgfqpoint{2.249831in}{3.060875in}}{\pgfqpoint{2.247220in}{3.054572in}}{\pgfqpoint{2.247220in}{3.048000in}}%
\pgfpathcurveto{\pgfqpoint{2.247220in}{3.041428in}}{\pgfqpoint{2.249831in}{3.035125in}}{\pgfqpoint{2.254478in}{3.030478in}}%
\pgfpathcurveto{\pgfqpoint{2.259125in}{3.025831in}}{\pgfqpoint{2.265428in}{3.023220in}}{\pgfqpoint{2.272000in}{3.023220in}}%
\pgfpathclose%
\pgfusepath{stroke,fill}%
\end{pgfscope}%
\begin{pgfscope}%
\pgfpathrectangle{\pgfqpoint{1.432000in}{0.528000in}}{\pgfqpoint{3.696000in}{3.696000in}}%
\pgfusepath{clip}%
\pgfsetbuttcap%
\pgfsetroundjoin%
\definecolor{currentfill}{rgb}{0.000000,0.000000,0.000000}%
\pgfsetfillcolor{currentfill}%
\pgfsetlinewidth{1.003750pt}%
\definecolor{currentstroke}{rgb}{0.000000,0.000000,0.000000}%
\pgfsetstrokecolor{currentstroke}%
\pgfsetdash{}{0pt}%
\pgfpathmoveto{\pgfqpoint{2.272000in}{3.359220in}}%
\pgfpathcurveto{\pgfqpoint{2.278572in}{3.359220in}}{\pgfqpoint{2.284875in}{3.361831in}}{\pgfqpoint{2.289522in}{3.366478in}}%
\pgfpathcurveto{\pgfqpoint{2.294169in}{3.371125in}}{\pgfqpoint{2.296780in}{3.377428in}}{\pgfqpoint{2.296780in}{3.384000in}}%
\pgfpathcurveto{\pgfqpoint{2.296780in}{3.390572in}}{\pgfqpoint{2.294169in}{3.396875in}}{\pgfqpoint{2.289522in}{3.401522in}}%
\pgfpathcurveto{\pgfqpoint{2.284875in}{3.406169in}}{\pgfqpoint{2.278572in}{3.408780in}}{\pgfqpoint{2.272000in}{3.408780in}}%
\pgfpathcurveto{\pgfqpoint{2.265428in}{3.408780in}}{\pgfqpoint{2.259125in}{3.406169in}}{\pgfqpoint{2.254478in}{3.401522in}}%
\pgfpathcurveto{\pgfqpoint{2.249831in}{3.396875in}}{\pgfqpoint{2.247220in}{3.390572in}}{\pgfqpoint{2.247220in}{3.384000in}}%
\pgfpathcurveto{\pgfqpoint{2.247220in}{3.377428in}}{\pgfqpoint{2.249831in}{3.371125in}}{\pgfqpoint{2.254478in}{3.366478in}}%
\pgfpathcurveto{\pgfqpoint{2.259125in}{3.361831in}}{\pgfqpoint{2.265428in}{3.359220in}}{\pgfqpoint{2.272000in}{3.359220in}}%
\pgfpathclose%
\pgfusepath{stroke,fill}%
\end{pgfscope}%
\begin{pgfscope}%
\pgfpathrectangle{\pgfqpoint{1.432000in}{0.528000in}}{\pgfqpoint{3.696000in}{3.696000in}}%
\pgfusepath{clip}%
\pgfsetbuttcap%
\pgfsetroundjoin%
\definecolor{currentfill}{rgb}{0.000000,0.000000,0.000000}%
\pgfsetfillcolor{currentfill}%
\pgfsetlinewidth{1.003750pt}%
\definecolor{currentstroke}{rgb}{0.000000,0.000000,0.000000}%
\pgfsetstrokecolor{currentstroke}%
\pgfsetdash{}{0pt}%
\pgfpathmoveto{\pgfqpoint{2.272000in}{3.695220in}}%
\pgfpathcurveto{\pgfqpoint{2.278572in}{3.695220in}}{\pgfqpoint{2.284875in}{3.697831in}}{\pgfqpoint{2.289522in}{3.702478in}}%
\pgfpathcurveto{\pgfqpoint{2.294169in}{3.707125in}}{\pgfqpoint{2.296780in}{3.713428in}}{\pgfqpoint{2.296780in}{3.720000in}}%
\pgfpathcurveto{\pgfqpoint{2.296780in}{3.726572in}}{\pgfqpoint{2.294169in}{3.732875in}}{\pgfqpoint{2.289522in}{3.737522in}}%
\pgfpathcurveto{\pgfqpoint{2.284875in}{3.742169in}}{\pgfqpoint{2.278572in}{3.744780in}}{\pgfqpoint{2.272000in}{3.744780in}}%
\pgfpathcurveto{\pgfqpoint{2.265428in}{3.744780in}}{\pgfqpoint{2.259125in}{3.742169in}}{\pgfqpoint{2.254478in}{3.737522in}}%
\pgfpathcurveto{\pgfqpoint{2.249831in}{3.732875in}}{\pgfqpoint{2.247220in}{3.726572in}}{\pgfqpoint{2.247220in}{3.720000in}}%
\pgfpathcurveto{\pgfqpoint{2.247220in}{3.713428in}}{\pgfqpoint{2.249831in}{3.707125in}}{\pgfqpoint{2.254478in}{3.702478in}}%
\pgfpathcurveto{\pgfqpoint{2.259125in}{3.697831in}}{\pgfqpoint{2.265428in}{3.695220in}}{\pgfqpoint{2.272000in}{3.695220in}}%
\pgfpathclose%
\pgfusepath{stroke,fill}%
\end{pgfscope}%
\begin{pgfscope}%
\pgfpathrectangle{\pgfqpoint{1.432000in}{0.528000in}}{\pgfqpoint{3.696000in}{3.696000in}}%
\pgfusepath{clip}%
\pgfsetbuttcap%
\pgfsetroundjoin%
\definecolor{currentfill}{rgb}{0.000000,0.000000,0.000000}%
\pgfsetfillcolor{currentfill}%
\pgfsetlinewidth{1.003750pt}%
\definecolor{currentstroke}{rgb}{0.000000,0.000000,0.000000}%
\pgfsetstrokecolor{currentstroke}%
\pgfsetdash{}{0pt}%
\pgfpathmoveto{\pgfqpoint{2.272000in}{4.031220in}}%
\pgfpathcurveto{\pgfqpoint{2.278572in}{4.031220in}}{\pgfqpoint{2.284875in}{4.033831in}}{\pgfqpoint{2.289522in}{4.038478in}}%
\pgfpathcurveto{\pgfqpoint{2.294169in}{4.043125in}}{\pgfqpoint{2.296780in}{4.049428in}}{\pgfqpoint{2.296780in}{4.056000in}}%
\pgfpathcurveto{\pgfqpoint{2.296780in}{4.062572in}}{\pgfqpoint{2.294169in}{4.068875in}}{\pgfqpoint{2.289522in}{4.073522in}}%
\pgfpathcurveto{\pgfqpoint{2.284875in}{4.078169in}}{\pgfqpoint{2.278572in}{4.080780in}}{\pgfqpoint{2.272000in}{4.080780in}}%
\pgfpathcurveto{\pgfqpoint{2.265428in}{4.080780in}}{\pgfqpoint{2.259125in}{4.078169in}}{\pgfqpoint{2.254478in}{4.073522in}}%
\pgfpathcurveto{\pgfqpoint{2.249831in}{4.068875in}}{\pgfqpoint{2.247220in}{4.062572in}}{\pgfqpoint{2.247220in}{4.056000in}}%
\pgfpathcurveto{\pgfqpoint{2.247220in}{4.049428in}}{\pgfqpoint{2.249831in}{4.043125in}}{\pgfqpoint{2.254478in}{4.038478in}}%
\pgfpathcurveto{\pgfqpoint{2.259125in}{4.033831in}}{\pgfqpoint{2.265428in}{4.031220in}}{\pgfqpoint{2.272000in}{4.031220in}}%
\pgfpathclose%
\pgfusepath{stroke,fill}%
\end{pgfscope}%
\begin{pgfscope}%
\pgfpathrectangle{\pgfqpoint{1.432000in}{0.528000in}}{\pgfqpoint{3.696000in}{3.696000in}}%
\pgfusepath{clip}%
\pgfsetbuttcap%
\pgfsetroundjoin%
\definecolor{currentfill}{rgb}{0.000000,0.000000,0.000000}%
\pgfsetfillcolor{currentfill}%
\pgfsetlinewidth{1.003750pt}%
\definecolor{currentstroke}{rgb}{0.000000,0.000000,0.000000}%
\pgfsetstrokecolor{currentstroke}%
\pgfsetdash{}{0pt}%
\pgfpathmoveto{\pgfqpoint{2.608000in}{0.671220in}}%
\pgfpathcurveto{\pgfqpoint{2.614572in}{0.671220in}}{\pgfqpoint{2.620875in}{0.673831in}}{\pgfqpoint{2.625522in}{0.678478in}}%
\pgfpathcurveto{\pgfqpoint{2.630169in}{0.683125in}}{\pgfqpoint{2.632780in}{0.689428in}}{\pgfqpoint{2.632780in}{0.696000in}}%
\pgfpathcurveto{\pgfqpoint{2.632780in}{0.702572in}}{\pgfqpoint{2.630169in}{0.708875in}}{\pgfqpoint{2.625522in}{0.713522in}}%
\pgfpathcurveto{\pgfqpoint{2.620875in}{0.718169in}}{\pgfqpoint{2.614572in}{0.720780in}}{\pgfqpoint{2.608000in}{0.720780in}}%
\pgfpathcurveto{\pgfqpoint{2.601428in}{0.720780in}}{\pgfqpoint{2.595125in}{0.718169in}}{\pgfqpoint{2.590478in}{0.713522in}}%
\pgfpathcurveto{\pgfqpoint{2.585831in}{0.708875in}}{\pgfqpoint{2.583220in}{0.702572in}}{\pgfqpoint{2.583220in}{0.696000in}}%
\pgfpathcurveto{\pgfqpoint{2.583220in}{0.689428in}}{\pgfqpoint{2.585831in}{0.683125in}}{\pgfqpoint{2.590478in}{0.678478in}}%
\pgfpathcurveto{\pgfqpoint{2.595125in}{0.673831in}}{\pgfqpoint{2.601428in}{0.671220in}}{\pgfqpoint{2.608000in}{0.671220in}}%
\pgfpathclose%
\pgfusepath{stroke,fill}%
\end{pgfscope}%
\begin{pgfscope}%
\pgfpathrectangle{\pgfqpoint{1.432000in}{0.528000in}}{\pgfqpoint{3.696000in}{3.696000in}}%
\pgfusepath{clip}%
\pgfsetbuttcap%
\pgfsetroundjoin%
\definecolor{currentfill}{rgb}{0.000000,0.000000,0.000000}%
\pgfsetfillcolor{currentfill}%
\pgfsetlinewidth{1.003750pt}%
\definecolor{currentstroke}{rgb}{0.000000,0.000000,0.000000}%
\pgfsetstrokecolor{currentstroke}%
\pgfsetdash{}{0pt}%
\pgfpathmoveto{\pgfqpoint{2.608000in}{1.007220in}}%
\pgfpathcurveto{\pgfqpoint{2.614572in}{1.007220in}}{\pgfqpoint{2.620875in}{1.009831in}}{\pgfqpoint{2.625522in}{1.014478in}}%
\pgfpathcurveto{\pgfqpoint{2.630169in}{1.019125in}}{\pgfqpoint{2.632780in}{1.025428in}}{\pgfqpoint{2.632780in}{1.032000in}}%
\pgfpathcurveto{\pgfqpoint{2.632780in}{1.038572in}}{\pgfqpoint{2.630169in}{1.044875in}}{\pgfqpoint{2.625522in}{1.049522in}}%
\pgfpathcurveto{\pgfqpoint{2.620875in}{1.054169in}}{\pgfqpoint{2.614572in}{1.056780in}}{\pgfqpoint{2.608000in}{1.056780in}}%
\pgfpathcurveto{\pgfqpoint{2.601428in}{1.056780in}}{\pgfqpoint{2.595125in}{1.054169in}}{\pgfqpoint{2.590478in}{1.049522in}}%
\pgfpathcurveto{\pgfqpoint{2.585831in}{1.044875in}}{\pgfqpoint{2.583220in}{1.038572in}}{\pgfqpoint{2.583220in}{1.032000in}}%
\pgfpathcurveto{\pgfqpoint{2.583220in}{1.025428in}}{\pgfqpoint{2.585831in}{1.019125in}}{\pgfqpoint{2.590478in}{1.014478in}}%
\pgfpathcurveto{\pgfqpoint{2.595125in}{1.009831in}}{\pgfqpoint{2.601428in}{1.007220in}}{\pgfqpoint{2.608000in}{1.007220in}}%
\pgfpathclose%
\pgfusepath{stroke,fill}%
\end{pgfscope}%
\begin{pgfscope}%
\pgfpathrectangle{\pgfqpoint{1.432000in}{0.528000in}}{\pgfqpoint{3.696000in}{3.696000in}}%
\pgfusepath{clip}%
\pgfsetbuttcap%
\pgfsetroundjoin%
\definecolor{currentfill}{rgb}{0.000000,0.000000,0.000000}%
\pgfsetfillcolor{currentfill}%
\pgfsetlinewidth{1.003750pt}%
\definecolor{currentstroke}{rgb}{0.000000,0.000000,0.000000}%
\pgfsetstrokecolor{currentstroke}%
\pgfsetdash{}{0pt}%
\pgfpathmoveto{\pgfqpoint{2.608000in}{1.343220in}}%
\pgfpathcurveto{\pgfqpoint{2.614572in}{1.343220in}}{\pgfqpoint{2.620875in}{1.345831in}}{\pgfqpoint{2.625522in}{1.350478in}}%
\pgfpathcurveto{\pgfqpoint{2.630169in}{1.355125in}}{\pgfqpoint{2.632780in}{1.361428in}}{\pgfqpoint{2.632780in}{1.368000in}}%
\pgfpathcurveto{\pgfqpoint{2.632780in}{1.374572in}}{\pgfqpoint{2.630169in}{1.380875in}}{\pgfqpoint{2.625522in}{1.385522in}}%
\pgfpathcurveto{\pgfqpoint{2.620875in}{1.390169in}}{\pgfqpoint{2.614572in}{1.392780in}}{\pgfqpoint{2.608000in}{1.392780in}}%
\pgfpathcurveto{\pgfqpoint{2.601428in}{1.392780in}}{\pgfqpoint{2.595125in}{1.390169in}}{\pgfqpoint{2.590478in}{1.385522in}}%
\pgfpathcurveto{\pgfqpoint{2.585831in}{1.380875in}}{\pgfqpoint{2.583220in}{1.374572in}}{\pgfqpoint{2.583220in}{1.368000in}}%
\pgfpathcurveto{\pgfqpoint{2.583220in}{1.361428in}}{\pgfqpoint{2.585831in}{1.355125in}}{\pgfqpoint{2.590478in}{1.350478in}}%
\pgfpathcurveto{\pgfqpoint{2.595125in}{1.345831in}}{\pgfqpoint{2.601428in}{1.343220in}}{\pgfqpoint{2.608000in}{1.343220in}}%
\pgfpathclose%
\pgfusepath{stroke,fill}%
\end{pgfscope}%
\begin{pgfscope}%
\pgfpathrectangle{\pgfqpoint{1.432000in}{0.528000in}}{\pgfqpoint{3.696000in}{3.696000in}}%
\pgfusepath{clip}%
\pgfsetbuttcap%
\pgfsetroundjoin%
\definecolor{currentfill}{rgb}{0.000000,0.000000,0.000000}%
\pgfsetfillcolor{currentfill}%
\pgfsetlinewidth{1.003750pt}%
\definecolor{currentstroke}{rgb}{0.000000,0.000000,0.000000}%
\pgfsetstrokecolor{currentstroke}%
\pgfsetdash{}{0pt}%
\pgfpathmoveto{\pgfqpoint{2.608000in}{1.679220in}}%
\pgfpathcurveto{\pgfqpoint{2.614572in}{1.679220in}}{\pgfqpoint{2.620875in}{1.681831in}}{\pgfqpoint{2.625522in}{1.686478in}}%
\pgfpathcurveto{\pgfqpoint{2.630169in}{1.691125in}}{\pgfqpoint{2.632780in}{1.697428in}}{\pgfqpoint{2.632780in}{1.704000in}}%
\pgfpathcurveto{\pgfqpoint{2.632780in}{1.710572in}}{\pgfqpoint{2.630169in}{1.716875in}}{\pgfqpoint{2.625522in}{1.721522in}}%
\pgfpathcurveto{\pgfqpoint{2.620875in}{1.726169in}}{\pgfqpoint{2.614572in}{1.728780in}}{\pgfqpoint{2.608000in}{1.728780in}}%
\pgfpathcurveto{\pgfqpoint{2.601428in}{1.728780in}}{\pgfqpoint{2.595125in}{1.726169in}}{\pgfqpoint{2.590478in}{1.721522in}}%
\pgfpathcurveto{\pgfqpoint{2.585831in}{1.716875in}}{\pgfqpoint{2.583220in}{1.710572in}}{\pgfqpoint{2.583220in}{1.704000in}}%
\pgfpathcurveto{\pgfqpoint{2.583220in}{1.697428in}}{\pgfqpoint{2.585831in}{1.691125in}}{\pgfqpoint{2.590478in}{1.686478in}}%
\pgfpathcurveto{\pgfqpoint{2.595125in}{1.681831in}}{\pgfqpoint{2.601428in}{1.679220in}}{\pgfqpoint{2.608000in}{1.679220in}}%
\pgfpathclose%
\pgfusepath{stroke,fill}%
\end{pgfscope}%
\begin{pgfscope}%
\pgfpathrectangle{\pgfqpoint{1.432000in}{0.528000in}}{\pgfqpoint{3.696000in}{3.696000in}}%
\pgfusepath{clip}%
\pgfsetbuttcap%
\pgfsetroundjoin%
\definecolor{currentfill}{rgb}{0.000000,0.000000,0.000000}%
\pgfsetfillcolor{currentfill}%
\pgfsetlinewidth{1.003750pt}%
\definecolor{currentstroke}{rgb}{0.000000,0.000000,0.000000}%
\pgfsetstrokecolor{currentstroke}%
\pgfsetdash{}{0pt}%
\pgfpathmoveto{\pgfqpoint{2.608000in}{2.015220in}}%
\pgfpathcurveto{\pgfqpoint{2.614572in}{2.015220in}}{\pgfqpoint{2.620875in}{2.017831in}}{\pgfqpoint{2.625522in}{2.022478in}}%
\pgfpathcurveto{\pgfqpoint{2.630169in}{2.027125in}}{\pgfqpoint{2.632780in}{2.033428in}}{\pgfqpoint{2.632780in}{2.040000in}}%
\pgfpathcurveto{\pgfqpoint{2.632780in}{2.046572in}}{\pgfqpoint{2.630169in}{2.052875in}}{\pgfqpoint{2.625522in}{2.057522in}}%
\pgfpathcurveto{\pgfqpoint{2.620875in}{2.062169in}}{\pgfqpoint{2.614572in}{2.064780in}}{\pgfqpoint{2.608000in}{2.064780in}}%
\pgfpathcurveto{\pgfqpoint{2.601428in}{2.064780in}}{\pgfqpoint{2.595125in}{2.062169in}}{\pgfqpoint{2.590478in}{2.057522in}}%
\pgfpathcurveto{\pgfqpoint{2.585831in}{2.052875in}}{\pgfqpoint{2.583220in}{2.046572in}}{\pgfqpoint{2.583220in}{2.040000in}}%
\pgfpathcurveto{\pgfqpoint{2.583220in}{2.033428in}}{\pgfqpoint{2.585831in}{2.027125in}}{\pgfqpoint{2.590478in}{2.022478in}}%
\pgfpathcurveto{\pgfqpoint{2.595125in}{2.017831in}}{\pgfqpoint{2.601428in}{2.015220in}}{\pgfqpoint{2.608000in}{2.015220in}}%
\pgfpathclose%
\pgfusepath{stroke,fill}%
\end{pgfscope}%
\begin{pgfscope}%
\pgfpathrectangle{\pgfqpoint{1.432000in}{0.528000in}}{\pgfqpoint{3.696000in}{3.696000in}}%
\pgfusepath{clip}%
\pgfsetbuttcap%
\pgfsetroundjoin%
\definecolor{currentfill}{rgb}{0.000000,0.000000,0.000000}%
\pgfsetfillcolor{currentfill}%
\pgfsetlinewidth{1.003750pt}%
\definecolor{currentstroke}{rgb}{0.000000,0.000000,0.000000}%
\pgfsetstrokecolor{currentstroke}%
\pgfsetdash{}{0pt}%
\pgfpathmoveto{\pgfqpoint{2.608000in}{2.351220in}}%
\pgfpathcurveto{\pgfqpoint{2.614572in}{2.351220in}}{\pgfqpoint{2.620875in}{2.353831in}}{\pgfqpoint{2.625522in}{2.358478in}}%
\pgfpathcurveto{\pgfqpoint{2.630169in}{2.363125in}}{\pgfqpoint{2.632780in}{2.369428in}}{\pgfqpoint{2.632780in}{2.376000in}}%
\pgfpathcurveto{\pgfqpoint{2.632780in}{2.382572in}}{\pgfqpoint{2.630169in}{2.388875in}}{\pgfqpoint{2.625522in}{2.393522in}}%
\pgfpathcurveto{\pgfqpoint{2.620875in}{2.398169in}}{\pgfqpoint{2.614572in}{2.400780in}}{\pgfqpoint{2.608000in}{2.400780in}}%
\pgfpathcurveto{\pgfqpoint{2.601428in}{2.400780in}}{\pgfqpoint{2.595125in}{2.398169in}}{\pgfqpoint{2.590478in}{2.393522in}}%
\pgfpathcurveto{\pgfqpoint{2.585831in}{2.388875in}}{\pgfqpoint{2.583220in}{2.382572in}}{\pgfqpoint{2.583220in}{2.376000in}}%
\pgfpathcurveto{\pgfqpoint{2.583220in}{2.369428in}}{\pgfqpoint{2.585831in}{2.363125in}}{\pgfqpoint{2.590478in}{2.358478in}}%
\pgfpathcurveto{\pgfqpoint{2.595125in}{2.353831in}}{\pgfqpoint{2.601428in}{2.351220in}}{\pgfqpoint{2.608000in}{2.351220in}}%
\pgfpathclose%
\pgfusepath{stroke,fill}%
\end{pgfscope}%
\begin{pgfscope}%
\pgfpathrectangle{\pgfqpoint{1.432000in}{0.528000in}}{\pgfqpoint{3.696000in}{3.696000in}}%
\pgfusepath{clip}%
\pgfsetbuttcap%
\pgfsetroundjoin%
\definecolor{currentfill}{rgb}{0.000000,0.000000,0.000000}%
\pgfsetfillcolor{currentfill}%
\pgfsetlinewidth{1.003750pt}%
\definecolor{currentstroke}{rgb}{0.000000,0.000000,0.000000}%
\pgfsetstrokecolor{currentstroke}%
\pgfsetdash{}{0pt}%
\pgfpathmoveto{\pgfqpoint{2.608000in}{2.687220in}}%
\pgfpathcurveto{\pgfqpoint{2.614572in}{2.687220in}}{\pgfqpoint{2.620875in}{2.689831in}}{\pgfqpoint{2.625522in}{2.694478in}}%
\pgfpathcurveto{\pgfqpoint{2.630169in}{2.699125in}}{\pgfqpoint{2.632780in}{2.705428in}}{\pgfqpoint{2.632780in}{2.712000in}}%
\pgfpathcurveto{\pgfqpoint{2.632780in}{2.718572in}}{\pgfqpoint{2.630169in}{2.724875in}}{\pgfqpoint{2.625522in}{2.729522in}}%
\pgfpathcurveto{\pgfqpoint{2.620875in}{2.734169in}}{\pgfqpoint{2.614572in}{2.736780in}}{\pgfqpoint{2.608000in}{2.736780in}}%
\pgfpathcurveto{\pgfqpoint{2.601428in}{2.736780in}}{\pgfqpoint{2.595125in}{2.734169in}}{\pgfqpoint{2.590478in}{2.729522in}}%
\pgfpathcurveto{\pgfqpoint{2.585831in}{2.724875in}}{\pgfqpoint{2.583220in}{2.718572in}}{\pgfqpoint{2.583220in}{2.712000in}}%
\pgfpathcurveto{\pgfqpoint{2.583220in}{2.705428in}}{\pgfqpoint{2.585831in}{2.699125in}}{\pgfqpoint{2.590478in}{2.694478in}}%
\pgfpathcurveto{\pgfqpoint{2.595125in}{2.689831in}}{\pgfqpoint{2.601428in}{2.687220in}}{\pgfqpoint{2.608000in}{2.687220in}}%
\pgfpathclose%
\pgfusepath{stroke,fill}%
\end{pgfscope}%
\begin{pgfscope}%
\pgfpathrectangle{\pgfqpoint{1.432000in}{0.528000in}}{\pgfqpoint{3.696000in}{3.696000in}}%
\pgfusepath{clip}%
\pgfsetbuttcap%
\pgfsetroundjoin%
\definecolor{currentfill}{rgb}{0.000000,0.000000,0.000000}%
\pgfsetfillcolor{currentfill}%
\pgfsetlinewidth{1.003750pt}%
\definecolor{currentstroke}{rgb}{0.000000,0.000000,0.000000}%
\pgfsetstrokecolor{currentstroke}%
\pgfsetdash{}{0pt}%
\pgfpathmoveto{\pgfqpoint{2.608000in}{3.023220in}}%
\pgfpathcurveto{\pgfqpoint{2.614572in}{3.023220in}}{\pgfqpoint{2.620875in}{3.025831in}}{\pgfqpoint{2.625522in}{3.030478in}}%
\pgfpathcurveto{\pgfqpoint{2.630169in}{3.035125in}}{\pgfqpoint{2.632780in}{3.041428in}}{\pgfqpoint{2.632780in}{3.048000in}}%
\pgfpathcurveto{\pgfqpoint{2.632780in}{3.054572in}}{\pgfqpoint{2.630169in}{3.060875in}}{\pgfqpoint{2.625522in}{3.065522in}}%
\pgfpathcurveto{\pgfqpoint{2.620875in}{3.070169in}}{\pgfqpoint{2.614572in}{3.072780in}}{\pgfqpoint{2.608000in}{3.072780in}}%
\pgfpathcurveto{\pgfqpoint{2.601428in}{3.072780in}}{\pgfqpoint{2.595125in}{3.070169in}}{\pgfqpoint{2.590478in}{3.065522in}}%
\pgfpathcurveto{\pgfqpoint{2.585831in}{3.060875in}}{\pgfqpoint{2.583220in}{3.054572in}}{\pgfqpoint{2.583220in}{3.048000in}}%
\pgfpathcurveto{\pgfqpoint{2.583220in}{3.041428in}}{\pgfqpoint{2.585831in}{3.035125in}}{\pgfqpoint{2.590478in}{3.030478in}}%
\pgfpathcurveto{\pgfqpoint{2.595125in}{3.025831in}}{\pgfqpoint{2.601428in}{3.023220in}}{\pgfqpoint{2.608000in}{3.023220in}}%
\pgfpathclose%
\pgfusepath{stroke,fill}%
\end{pgfscope}%
\begin{pgfscope}%
\pgfpathrectangle{\pgfqpoint{1.432000in}{0.528000in}}{\pgfqpoint{3.696000in}{3.696000in}}%
\pgfusepath{clip}%
\pgfsetbuttcap%
\pgfsetroundjoin%
\definecolor{currentfill}{rgb}{0.000000,0.000000,0.000000}%
\pgfsetfillcolor{currentfill}%
\pgfsetlinewidth{1.003750pt}%
\definecolor{currentstroke}{rgb}{0.000000,0.000000,0.000000}%
\pgfsetstrokecolor{currentstroke}%
\pgfsetdash{}{0pt}%
\pgfpathmoveto{\pgfqpoint{2.608000in}{3.359220in}}%
\pgfpathcurveto{\pgfqpoint{2.614572in}{3.359220in}}{\pgfqpoint{2.620875in}{3.361831in}}{\pgfqpoint{2.625522in}{3.366478in}}%
\pgfpathcurveto{\pgfqpoint{2.630169in}{3.371125in}}{\pgfqpoint{2.632780in}{3.377428in}}{\pgfqpoint{2.632780in}{3.384000in}}%
\pgfpathcurveto{\pgfqpoint{2.632780in}{3.390572in}}{\pgfqpoint{2.630169in}{3.396875in}}{\pgfqpoint{2.625522in}{3.401522in}}%
\pgfpathcurveto{\pgfqpoint{2.620875in}{3.406169in}}{\pgfqpoint{2.614572in}{3.408780in}}{\pgfqpoint{2.608000in}{3.408780in}}%
\pgfpathcurveto{\pgfqpoint{2.601428in}{3.408780in}}{\pgfqpoint{2.595125in}{3.406169in}}{\pgfqpoint{2.590478in}{3.401522in}}%
\pgfpathcurveto{\pgfqpoint{2.585831in}{3.396875in}}{\pgfqpoint{2.583220in}{3.390572in}}{\pgfqpoint{2.583220in}{3.384000in}}%
\pgfpathcurveto{\pgfqpoint{2.583220in}{3.377428in}}{\pgfqpoint{2.585831in}{3.371125in}}{\pgfqpoint{2.590478in}{3.366478in}}%
\pgfpathcurveto{\pgfqpoint{2.595125in}{3.361831in}}{\pgfqpoint{2.601428in}{3.359220in}}{\pgfqpoint{2.608000in}{3.359220in}}%
\pgfpathclose%
\pgfusepath{stroke,fill}%
\end{pgfscope}%
\begin{pgfscope}%
\pgfpathrectangle{\pgfqpoint{1.432000in}{0.528000in}}{\pgfqpoint{3.696000in}{3.696000in}}%
\pgfusepath{clip}%
\pgfsetbuttcap%
\pgfsetroundjoin%
\definecolor{currentfill}{rgb}{0.000000,0.000000,0.000000}%
\pgfsetfillcolor{currentfill}%
\pgfsetlinewidth{1.003750pt}%
\definecolor{currentstroke}{rgb}{0.000000,0.000000,0.000000}%
\pgfsetstrokecolor{currentstroke}%
\pgfsetdash{}{0pt}%
\pgfpathmoveto{\pgfqpoint{2.608000in}{3.695220in}}%
\pgfpathcurveto{\pgfqpoint{2.614572in}{3.695220in}}{\pgfqpoint{2.620875in}{3.697831in}}{\pgfqpoint{2.625522in}{3.702478in}}%
\pgfpathcurveto{\pgfqpoint{2.630169in}{3.707125in}}{\pgfqpoint{2.632780in}{3.713428in}}{\pgfqpoint{2.632780in}{3.720000in}}%
\pgfpathcurveto{\pgfqpoint{2.632780in}{3.726572in}}{\pgfqpoint{2.630169in}{3.732875in}}{\pgfqpoint{2.625522in}{3.737522in}}%
\pgfpathcurveto{\pgfqpoint{2.620875in}{3.742169in}}{\pgfqpoint{2.614572in}{3.744780in}}{\pgfqpoint{2.608000in}{3.744780in}}%
\pgfpathcurveto{\pgfqpoint{2.601428in}{3.744780in}}{\pgfqpoint{2.595125in}{3.742169in}}{\pgfqpoint{2.590478in}{3.737522in}}%
\pgfpathcurveto{\pgfqpoint{2.585831in}{3.732875in}}{\pgfqpoint{2.583220in}{3.726572in}}{\pgfqpoint{2.583220in}{3.720000in}}%
\pgfpathcurveto{\pgfqpoint{2.583220in}{3.713428in}}{\pgfqpoint{2.585831in}{3.707125in}}{\pgfqpoint{2.590478in}{3.702478in}}%
\pgfpathcurveto{\pgfqpoint{2.595125in}{3.697831in}}{\pgfqpoint{2.601428in}{3.695220in}}{\pgfqpoint{2.608000in}{3.695220in}}%
\pgfpathclose%
\pgfusepath{stroke,fill}%
\end{pgfscope}%
\begin{pgfscope}%
\pgfpathrectangle{\pgfqpoint{1.432000in}{0.528000in}}{\pgfqpoint{3.696000in}{3.696000in}}%
\pgfusepath{clip}%
\pgfsetbuttcap%
\pgfsetroundjoin%
\definecolor{currentfill}{rgb}{0.000000,0.000000,0.000000}%
\pgfsetfillcolor{currentfill}%
\pgfsetlinewidth{1.003750pt}%
\definecolor{currentstroke}{rgb}{0.000000,0.000000,0.000000}%
\pgfsetstrokecolor{currentstroke}%
\pgfsetdash{}{0pt}%
\pgfpathmoveto{\pgfqpoint{2.608000in}{4.031220in}}%
\pgfpathcurveto{\pgfqpoint{2.614572in}{4.031220in}}{\pgfqpoint{2.620875in}{4.033831in}}{\pgfqpoint{2.625522in}{4.038478in}}%
\pgfpathcurveto{\pgfqpoint{2.630169in}{4.043125in}}{\pgfqpoint{2.632780in}{4.049428in}}{\pgfqpoint{2.632780in}{4.056000in}}%
\pgfpathcurveto{\pgfqpoint{2.632780in}{4.062572in}}{\pgfqpoint{2.630169in}{4.068875in}}{\pgfqpoint{2.625522in}{4.073522in}}%
\pgfpathcurveto{\pgfqpoint{2.620875in}{4.078169in}}{\pgfqpoint{2.614572in}{4.080780in}}{\pgfqpoint{2.608000in}{4.080780in}}%
\pgfpathcurveto{\pgfqpoint{2.601428in}{4.080780in}}{\pgfqpoint{2.595125in}{4.078169in}}{\pgfqpoint{2.590478in}{4.073522in}}%
\pgfpathcurveto{\pgfqpoint{2.585831in}{4.068875in}}{\pgfqpoint{2.583220in}{4.062572in}}{\pgfqpoint{2.583220in}{4.056000in}}%
\pgfpathcurveto{\pgfqpoint{2.583220in}{4.049428in}}{\pgfqpoint{2.585831in}{4.043125in}}{\pgfqpoint{2.590478in}{4.038478in}}%
\pgfpathcurveto{\pgfqpoint{2.595125in}{4.033831in}}{\pgfqpoint{2.601428in}{4.031220in}}{\pgfqpoint{2.608000in}{4.031220in}}%
\pgfpathclose%
\pgfusepath{stroke,fill}%
\end{pgfscope}%
\begin{pgfscope}%
\pgfpathrectangle{\pgfqpoint{1.432000in}{0.528000in}}{\pgfqpoint{3.696000in}{3.696000in}}%
\pgfusepath{clip}%
\pgfsetbuttcap%
\pgfsetroundjoin%
\definecolor{currentfill}{rgb}{0.000000,0.000000,0.000000}%
\pgfsetfillcolor{currentfill}%
\pgfsetlinewidth{1.003750pt}%
\definecolor{currentstroke}{rgb}{0.000000,0.000000,0.000000}%
\pgfsetstrokecolor{currentstroke}%
\pgfsetdash{}{0pt}%
\pgfpathmoveto{\pgfqpoint{2.944000in}{0.671220in}}%
\pgfpathcurveto{\pgfqpoint{2.950572in}{0.671220in}}{\pgfqpoint{2.956875in}{0.673831in}}{\pgfqpoint{2.961522in}{0.678478in}}%
\pgfpathcurveto{\pgfqpoint{2.966169in}{0.683125in}}{\pgfqpoint{2.968780in}{0.689428in}}{\pgfqpoint{2.968780in}{0.696000in}}%
\pgfpathcurveto{\pgfqpoint{2.968780in}{0.702572in}}{\pgfqpoint{2.966169in}{0.708875in}}{\pgfqpoint{2.961522in}{0.713522in}}%
\pgfpathcurveto{\pgfqpoint{2.956875in}{0.718169in}}{\pgfqpoint{2.950572in}{0.720780in}}{\pgfqpoint{2.944000in}{0.720780in}}%
\pgfpathcurveto{\pgfqpoint{2.937428in}{0.720780in}}{\pgfqpoint{2.931125in}{0.718169in}}{\pgfqpoint{2.926478in}{0.713522in}}%
\pgfpathcurveto{\pgfqpoint{2.921831in}{0.708875in}}{\pgfqpoint{2.919220in}{0.702572in}}{\pgfqpoint{2.919220in}{0.696000in}}%
\pgfpathcurveto{\pgfqpoint{2.919220in}{0.689428in}}{\pgfqpoint{2.921831in}{0.683125in}}{\pgfqpoint{2.926478in}{0.678478in}}%
\pgfpathcurveto{\pgfqpoint{2.931125in}{0.673831in}}{\pgfqpoint{2.937428in}{0.671220in}}{\pgfqpoint{2.944000in}{0.671220in}}%
\pgfpathclose%
\pgfusepath{stroke,fill}%
\end{pgfscope}%
\begin{pgfscope}%
\pgfpathrectangle{\pgfqpoint{1.432000in}{0.528000in}}{\pgfqpoint{3.696000in}{3.696000in}}%
\pgfusepath{clip}%
\pgfsetbuttcap%
\pgfsetroundjoin%
\definecolor{currentfill}{rgb}{0.000000,0.000000,0.000000}%
\pgfsetfillcolor{currentfill}%
\pgfsetlinewidth{1.003750pt}%
\definecolor{currentstroke}{rgb}{0.000000,0.000000,0.000000}%
\pgfsetstrokecolor{currentstroke}%
\pgfsetdash{}{0pt}%
\pgfpathmoveto{\pgfqpoint{2.944000in}{1.007220in}}%
\pgfpathcurveto{\pgfqpoint{2.950572in}{1.007220in}}{\pgfqpoint{2.956875in}{1.009831in}}{\pgfqpoint{2.961522in}{1.014478in}}%
\pgfpathcurveto{\pgfqpoint{2.966169in}{1.019125in}}{\pgfqpoint{2.968780in}{1.025428in}}{\pgfqpoint{2.968780in}{1.032000in}}%
\pgfpathcurveto{\pgfqpoint{2.968780in}{1.038572in}}{\pgfqpoint{2.966169in}{1.044875in}}{\pgfqpoint{2.961522in}{1.049522in}}%
\pgfpathcurveto{\pgfqpoint{2.956875in}{1.054169in}}{\pgfqpoint{2.950572in}{1.056780in}}{\pgfqpoint{2.944000in}{1.056780in}}%
\pgfpathcurveto{\pgfqpoint{2.937428in}{1.056780in}}{\pgfqpoint{2.931125in}{1.054169in}}{\pgfqpoint{2.926478in}{1.049522in}}%
\pgfpathcurveto{\pgfqpoint{2.921831in}{1.044875in}}{\pgfqpoint{2.919220in}{1.038572in}}{\pgfqpoint{2.919220in}{1.032000in}}%
\pgfpathcurveto{\pgfqpoint{2.919220in}{1.025428in}}{\pgfqpoint{2.921831in}{1.019125in}}{\pgfqpoint{2.926478in}{1.014478in}}%
\pgfpathcurveto{\pgfqpoint{2.931125in}{1.009831in}}{\pgfqpoint{2.937428in}{1.007220in}}{\pgfqpoint{2.944000in}{1.007220in}}%
\pgfpathclose%
\pgfusepath{stroke,fill}%
\end{pgfscope}%
\begin{pgfscope}%
\pgfpathrectangle{\pgfqpoint{1.432000in}{0.528000in}}{\pgfqpoint{3.696000in}{3.696000in}}%
\pgfusepath{clip}%
\pgfsetbuttcap%
\pgfsetroundjoin%
\definecolor{currentfill}{rgb}{0.000000,0.000000,0.000000}%
\pgfsetfillcolor{currentfill}%
\pgfsetlinewidth{1.003750pt}%
\definecolor{currentstroke}{rgb}{0.000000,0.000000,0.000000}%
\pgfsetstrokecolor{currentstroke}%
\pgfsetdash{}{0pt}%
\pgfpathmoveto{\pgfqpoint{2.944000in}{1.343220in}}%
\pgfpathcurveto{\pgfqpoint{2.950572in}{1.343220in}}{\pgfqpoint{2.956875in}{1.345831in}}{\pgfqpoint{2.961522in}{1.350478in}}%
\pgfpathcurveto{\pgfqpoint{2.966169in}{1.355125in}}{\pgfqpoint{2.968780in}{1.361428in}}{\pgfqpoint{2.968780in}{1.368000in}}%
\pgfpathcurveto{\pgfqpoint{2.968780in}{1.374572in}}{\pgfqpoint{2.966169in}{1.380875in}}{\pgfqpoint{2.961522in}{1.385522in}}%
\pgfpathcurveto{\pgfqpoint{2.956875in}{1.390169in}}{\pgfqpoint{2.950572in}{1.392780in}}{\pgfqpoint{2.944000in}{1.392780in}}%
\pgfpathcurveto{\pgfqpoint{2.937428in}{1.392780in}}{\pgfqpoint{2.931125in}{1.390169in}}{\pgfqpoint{2.926478in}{1.385522in}}%
\pgfpathcurveto{\pgfqpoint{2.921831in}{1.380875in}}{\pgfqpoint{2.919220in}{1.374572in}}{\pgfqpoint{2.919220in}{1.368000in}}%
\pgfpathcurveto{\pgfqpoint{2.919220in}{1.361428in}}{\pgfqpoint{2.921831in}{1.355125in}}{\pgfqpoint{2.926478in}{1.350478in}}%
\pgfpathcurveto{\pgfqpoint{2.931125in}{1.345831in}}{\pgfqpoint{2.937428in}{1.343220in}}{\pgfqpoint{2.944000in}{1.343220in}}%
\pgfpathclose%
\pgfusepath{stroke,fill}%
\end{pgfscope}%
\begin{pgfscope}%
\pgfpathrectangle{\pgfqpoint{1.432000in}{0.528000in}}{\pgfqpoint{3.696000in}{3.696000in}}%
\pgfusepath{clip}%
\pgfsetbuttcap%
\pgfsetroundjoin%
\definecolor{currentfill}{rgb}{0.000000,0.000000,0.000000}%
\pgfsetfillcolor{currentfill}%
\pgfsetlinewidth{1.003750pt}%
\definecolor{currentstroke}{rgb}{0.000000,0.000000,0.000000}%
\pgfsetstrokecolor{currentstroke}%
\pgfsetdash{}{0pt}%
\pgfpathmoveto{\pgfqpoint{2.944000in}{1.679220in}}%
\pgfpathcurveto{\pgfqpoint{2.950572in}{1.679220in}}{\pgfqpoint{2.956875in}{1.681831in}}{\pgfqpoint{2.961522in}{1.686478in}}%
\pgfpathcurveto{\pgfqpoint{2.966169in}{1.691125in}}{\pgfqpoint{2.968780in}{1.697428in}}{\pgfqpoint{2.968780in}{1.704000in}}%
\pgfpathcurveto{\pgfqpoint{2.968780in}{1.710572in}}{\pgfqpoint{2.966169in}{1.716875in}}{\pgfqpoint{2.961522in}{1.721522in}}%
\pgfpathcurveto{\pgfqpoint{2.956875in}{1.726169in}}{\pgfqpoint{2.950572in}{1.728780in}}{\pgfqpoint{2.944000in}{1.728780in}}%
\pgfpathcurveto{\pgfqpoint{2.937428in}{1.728780in}}{\pgfqpoint{2.931125in}{1.726169in}}{\pgfqpoint{2.926478in}{1.721522in}}%
\pgfpathcurveto{\pgfqpoint{2.921831in}{1.716875in}}{\pgfqpoint{2.919220in}{1.710572in}}{\pgfqpoint{2.919220in}{1.704000in}}%
\pgfpathcurveto{\pgfqpoint{2.919220in}{1.697428in}}{\pgfqpoint{2.921831in}{1.691125in}}{\pgfqpoint{2.926478in}{1.686478in}}%
\pgfpathcurveto{\pgfqpoint{2.931125in}{1.681831in}}{\pgfqpoint{2.937428in}{1.679220in}}{\pgfqpoint{2.944000in}{1.679220in}}%
\pgfpathclose%
\pgfusepath{stroke,fill}%
\end{pgfscope}%
\begin{pgfscope}%
\pgfpathrectangle{\pgfqpoint{1.432000in}{0.528000in}}{\pgfqpoint{3.696000in}{3.696000in}}%
\pgfusepath{clip}%
\pgfsetbuttcap%
\pgfsetroundjoin%
\definecolor{currentfill}{rgb}{0.000000,0.000000,0.000000}%
\pgfsetfillcolor{currentfill}%
\pgfsetlinewidth{1.003750pt}%
\definecolor{currentstroke}{rgb}{0.000000,0.000000,0.000000}%
\pgfsetstrokecolor{currentstroke}%
\pgfsetdash{}{0pt}%
\pgfpathmoveto{\pgfqpoint{2.944000in}{2.015220in}}%
\pgfpathcurveto{\pgfqpoint{2.950572in}{2.015220in}}{\pgfqpoint{2.956875in}{2.017831in}}{\pgfqpoint{2.961522in}{2.022478in}}%
\pgfpathcurveto{\pgfqpoint{2.966169in}{2.027125in}}{\pgfqpoint{2.968780in}{2.033428in}}{\pgfqpoint{2.968780in}{2.040000in}}%
\pgfpathcurveto{\pgfqpoint{2.968780in}{2.046572in}}{\pgfqpoint{2.966169in}{2.052875in}}{\pgfqpoint{2.961522in}{2.057522in}}%
\pgfpathcurveto{\pgfqpoint{2.956875in}{2.062169in}}{\pgfqpoint{2.950572in}{2.064780in}}{\pgfqpoint{2.944000in}{2.064780in}}%
\pgfpathcurveto{\pgfqpoint{2.937428in}{2.064780in}}{\pgfqpoint{2.931125in}{2.062169in}}{\pgfqpoint{2.926478in}{2.057522in}}%
\pgfpathcurveto{\pgfqpoint{2.921831in}{2.052875in}}{\pgfqpoint{2.919220in}{2.046572in}}{\pgfqpoint{2.919220in}{2.040000in}}%
\pgfpathcurveto{\pgfqpoint{2.919220in}{2.033428in}}{\pgfqpoint{2.921831in}{2.027125in}}{\pgfqpoint{2.926478in}{2.022478in}}%
\pgfpathcurveto{\pgfqpoint{2.931125in}{2.017831in}}{\pgfqpoint{2.937428in}{2.015220in}}{\pgfqpoint{2.944000in}{2.015220in}}%
\pgfpathclose%
\pgfusepath{stroke,fill}%
\end{pgfscope}%
\begin{pgfscope}%
\pgfpathrectangle{\pgfqpoint{1.432000in}{0.528000in}}{\pgfqpoint{3.696000in}{3.696000in}}%
\pgfusepath{clip}%
\pgfsetbuttcap%
\pgfsetroundjoin%
\definecolor{currentfill}{rgb}{0.000000,0.000000,0.000000}%
\pgfsetfillcolor{currentfill}%
\pgfsetlinewidth{1.003750pt}%
\definecolor{currentstroke}{rgb}{0.000000,0.000000,0.000000}%
\pgfsetstrokecolor{currentstroke}%
\pgfsetdash{}{0pt}%
\pgfpathmoveto{\pgfqpoint{2.944000in}{2.351220in}}%
\pgfpathcurveto{\pgfqpoint{2.950572in}{2.351220in}}{\pgfqpoint{2.956875in}{2.353831in}}{\pgfqpoint{2.961522in}{2.358478in}}%
\pgfpathcurveto{\pgfqpoint{2.966169in}{2.363125in}}{\pgfqpoint{2.968780in}{2.369428in}}{\pgfqpoint{2.968780in}{2.376000in}}%
\pgfpathcurveto{\pgfqpoint{2.968780in}{2.382572in}}{\pgfqpoint{2.966169in}{2.388875in}}{\pgfqpoint{2.961522in}{2.393522in}}%
\pgfpathcurveto{\pgfqpoint{2.956875in}{2.398169in}}{\pgfqpoint{2.950572in}{2.400780in}}{\pgfqpoint{2.944000in}{2.400780in}}%
\pgfpathcurveto{\pgfqpoint{2.937428in}{2.400780in}}{\pgfqpoint{2.931125in}{2.398169in}}{\pgfqpoint{2.926478in}{2.393522in}}%
\pgfpathcurveto{\pgfqpoint{2.921831in}{2.388875in}}{\pgfqpoint{2.919220in}{2.382572in}}{\pgfqpoint{2.919220in}{2.376000in}}%
\pgfpathcurveto{\pgfqpoint{2.919220in}{2.369428in}}{\pgfqpoint{2.921831in}{2.363125in}}{\pgfqpoint{2.926478in}{2.358478in}}%
\pgfpathcurveto{\pgfqpoint{2.931125in}{2.353831in}}{\pgfqpoint{2.937428in}{2.351220in}}{\pgfqpoint{2.944000in}{2.351220in}}%
\pgfpathclose%
\pgfusepath{stroke,fill}%
\end{pgfscope}%
\begin{pgfscope}%
\pgfpathrectangle{\pgfqpoint{1.432000in}{0.528000in}}{\pgfqpoint{3.696000in}{3.696000in}}%
\pgfusepath{clip}%
\pgfsetbuttcap%
\pgfsetroundjoin%
\definecolor{currentfill}{rgb}{0.000000,0.000000,0.000000}%
\pgfsetfillcolor{currentfill}%
\pgfsetlinewidth{1.003750pt}%
\definecolor{currentstroke}{rgb}{0.000000,0.000000,0.000000}%
\pgfsetstrokecolor{currentstroke}%
\pgfsetdash{}{0pt}%
\pgfpathmoveto{\pgfqpoint{2.944000in}{2.687220in}}%
\pgfpathcurveto{\pgfqpoint{2.950572in}{2.687220in}}{\pgfqpoint{2.956875in}{2.689831in}}{\pgfqpoint{2.961522in}{2.694478in}}%
\pgfpathcurveto{\pgfqpoint{2.966169in}{2.699125in}}{\pgfqpoint{2.968780in}{2.705428in}}{\pgfqpoint{2.968780in}{2.712000in}}%
\pgfpathcurveto{\pgfqpoint{2.968780in}{2.718572in}}{\pgfqpoint{2.966169in}{2.724875in}}{\pgfqpoint{2.961522in}{2.729522in}}%
\pgfpathcurveto{\pgfqpoint{2.956875in}{2.734169in}}{\pgfqpoint{2.950572in}{2.736780in}}{\pgfqpoint{2.944000in}{2.736780in}}%
\pgfpathcurveto{\pgfqpoint{2.937428in}{2.736780in}}{\pgfqpoint{2.931125in}{2.734169in}}{\pgfqpoint{2.926478in}{2.729522in}}%
\pgfpathcurveto{\pgfqpoint{2.921831in}{2.724875in}}{\pgfqpoint{2.919220in}{2.718572in}}{\pgfqpoint{2.919220in}{2.712000in}}%
\pgfpathcurveto{\pgfqpoint{2.919220in}{2.705428in}}{\pgfqpoint{2.921831in}{2.699125in}}{\pgfqpoint{2.926478in}{2.694478in}}%
\pgfpathcurveto{\pgfqpoint{2.931125in}{2.689831in}}{\pgfqpoint{2.937428in}{2.687220in}}{\pgfqpoint{2.944000in}{2.687220in}}%
\pgfpathclose%
\pgfusepath{stroke,fill}%
\end{pgfscope}%
\begin{pgfscope}%
\pgfpathrectangle{\pgfqpoint{1.432000in}{0.528000in}}{\pgfqpoint{3.696000in}{3.696000in}}%
\pgfusepath{clip}%
\pgfsetbuttcap%
\pgfsetroundjoin%
\definecolor{currentfill}{rgb}{0.000000,0.000000,0.000000}%
\pgfsetfillcolor{currentfill}%
\pgfsetlinewidth{1.003750pt}%
\definecolor{currentstroke}{rgb}{0.000000,0.000000,0.000000}%
\pgfsetstrokecolor{currentstroke}%
\pgfsetdash{}{0pt}%
\pgfpathmoveto{\pgfqpoint{2.944000in}{3.023220in}}%
\pgfpathcurveto{\pgfqpoint{2.950572in}{3.023220in}}{\pgfqpoint{2.956875in}{3.025831in}}{\pgfqpoint{2.961522in}{3.030478in}}%
\pgfpathcurveto{\pgfqpoint{2.966169in}{3.035125in}}{\pgfqpoint{2.968780in}{3.041428in}}{\pgfqpoint{2.968780in}{3.048000in}}%
\pgfpathcurveto{\pgfqpoint{2.968780in}{3.054572in}}{\pgfqpoint{2.966169in}{3.060875in}}{\pgfqpoint{2.961522in}{3.065522in}}%
\pgfpathcurveto{\pgfqpoint{2.956875in}{3.070169in}}{\pgfqpoint{2.950572in}{3.072780in}}{\pgfqpoint{2.944000in}{3.072780in}}%
\pgfpathcurveto{\pgfqpoint{2.937428in}{3.072780in}}{\pgfqpoint{2.931125in}{3.070169in}}{\pgfqpoint{2.926478in}{3.065522in}}%
\pgfpathcurveto{\pgfqpoint{2.921831in}{3.060875in}}{\pgfqpoint{2.919220in}{3.054572in}}{\pgfqpoint{2.919220in}{3.048000in}}%
\pgfpathcurveto{\pgfqpoint{2.919220in}{3.041428in}}{\pgfqpoint{2.921831in}{3.035125in}}{\pgfqpoint{2.926478in}{3.030478in}}%
\pgfpathcurveto{\pgfqpoint{2.931125in}{3.025831in}}{\pgfqpoint{2.937428in}{3.023220in}}{\pgfqpoint{2.944000in}{3.023220in}}%
\pgfpathclose%
\pgfusepath{stroke,fill}%
\end{pgfscope}%
\begin{pgfscope}%
\pgfpathrectangle{\pgfqpoint{1.432000in}{0.528000in}}{\pgfqpoint{3.696000in}{3.696000in}}%
\pgfusepath{clip}%
\pgfsetbuttcap%
\pgfsetroundjoin%
\definecolor{currentfill}{rgb}{0.000000,0.000000,0.000000}%
\pgfsetfillcolor{currentfill}%
\pgfsetlinewidth{1.003750pt}%
\definecolor{currentstroke}{rgb}{0.000000,0.000000,0.000000}%
\pgfsetstrokecolor{currentstroke}%
\pgfsetdash{}{0pt}%
\pgfpathmoveto{\pgfqpoint{2.944000in}{3.359220in}}%
\pgfpathcurveto{\pgfqpoint{2.950572in}{3.359220in}}{\pgfqpoint{2.956875in}{3.361831in}}{\pgfqpoint{2.961522in}{3.366478in}}%
\pgfpathcurveto{\pgfqpoint{2.966169in}{3.371125in}}{\pgfqpoint{2.968780in}{3.377428in}}{\pgfqpoint{2.968780in}{3.384000in}}%
\pgfpathcurveto{\pgfqpoint{2.968780in}{3.390572in}}{\pgfqpoint{2.966169in}{3.396875in}}{\pgfqpoint{2.961522in}{3.401522in}}%
\pgfpathcurveto{\pgfqpoint{2.956875in}{3.406169in}}{\pgfqpoint{2.950572in}{3.408780in}}{\pgfqpoint{2.944000in}{3.408780in}}%
\pgfpathcurveto{\pgfqpoint{2.937428in}{3.408780in}}{\pgfqpoint{2.931125in}{3.406169in}}{\pgfqpoint{2.926478in}{3.401522in}}%
\pgfpathcurveto{\pgfqpoint{2.921831in}{3.396875in}}{\pgfqpoint{2.919220in}{3.390572in}}{\pgfqpoint{2.919220in}{3.384000in}}%
\pgfpathcurveto{\pgfqpoint{2.919220in}{3.377428in}}{\pgfqpoint{2.921831in}{3.371125in}}{\pgfqpoint{2.926478in}{3.366478in}}%
\pgfpathcurveto{\pgfqpoint{2.931125in}{3.361831in}}{\pgfqpoint{2.937428in}{3.359220in}}{\pgfqpoint{2.944000in}{3.359220in}}%
\pgfpathclose%
\pgfusepath{stroke,fill}%
\end{pgfscope}%
\begin{pgfscope}%
\pgfpathrectangle{\pgfqpoint{1.432000in}{0.528000in}}{\pgfqpoint{3.696000in}{3.696000in}}%
\pgfusepath{clip}%
\pgfsetbuttcap%
\pgfsetroundjoin%
\definecolor{currentfill}{rgb}{0.000000,0.000000,0.000000}%
\pgfsetfillcolor{currentfill}%
\pgfsetlinewidth{1.003750pt}%
\definecolor{currentstroke}{rgb}{0.000000,0.000000,0.000000}%
\pgfsetstrokecolor{currentstroke}%
\pgfsetdash{}{0pt}%
\pgfpathmoveto{\pgfqpoint{2.944000in}{3.695220in}}%
\pgfpathcurveto{\pgfqpoint{2.950572in}{3.695220in}}{\pgfqpoint{2.956875in}{3.697831in}}{\pgfqpoint{2.961522in}{3.702478in}}%
\pgfpathcurveto{\pgfqpoint{2.966169in}{3.707125in}}{\pgfqpoint{2.968780in}{3.713428in}}{\pgfqpoint{2.968780in}{3.720000in}}%
\pgfpathcurveto{\pgfqpoint{2.968780in}{3.726572in}}{\pgfqpoint{2.966169in}{3.732875in}}{\pgfqpoint{2.961522in}{3.737522in}}%
\pgfpathcurveto{\pgfqpoint{2.956875in}{3.742169in}}{\pgfqpoint{2.950572in}{3.744780in}}{\pgfqpoint{2.944000in}{3.744780in}}%
\pgfpathcurveto{\pgfqpoint{2.937428in}{3.744780in}}{\pgfqpoint{2.931125in}{3.742169in}}{\pgfqpoint{2.926478in}{3.737522in}}%
\pgfpathcurveto{\pgfqpoint{2.921831in}{3.732875in}}{\pgfqpoint{2.919220in}{3.726572in}}{\pgfqpoint{2.919220in}{3.720000in}}%
\pgfpathcurveto{\pgfqpoint{2.919220in}{3.713428in}}{\pgfqpoint{2.921831in}{3.707125in}}{\pgfqpoint{2.926478in}{3.702478in}}%
\pgfpathcurveto{\pgfqpoint{2.931125in}{3.697831in}}{\pgfqpoint{2.937428in}{3.695220in}}{\pgfqpoint{2.944000in}{3.695220in}}%
\pgfpathclose%
\pgfusepath{stroke,fill}%
\end{pgfscope}%
\begin{pgfscope}%
\pgfpathrectangle{\pgfqpoint{1.432000in}{0.528000in}}{\pgfqpoint{3.696000in}{3.696000in}}%
\pgfusepath{clip}%
\pgfsetbuttcap%
\pgfsetroundjoin%
\definecolor{currentfill}{rgb}{0.000000,0.000000,0.000000}%
\pgfsetfillcolor{currentfill}%
\pgfsetlinewidth{1.003750pt}%
\definecolor{currentstroke}{rgb}{0.000000,0.000000,0.000000}%
\pgfsetstrokecolor{currentstroke}%
\pgfsetdash{}{0pt}%
\pgfpathmoveto{\pgfqpoint{2.944000in}{4.031220in}}%
\pgfpathcurveto{\pgfqpoint{2.950572in}{4.031220in}}{\pgfqpoint{2.956875in}{4.033831in}}{\pgfqpoint{2.961522in}{4.038478in}}%
\pgfpathcurveto{\pgfqpoint{2.966169in}{4.043125in}}{\pgfqpoint{2.968780in}{4.049428in}}{\pgfqpoint{2.968780in}{4.056000in}}%
\pgfpathcurveto{\pgfqpoint{2.968780in}{4.062572in}}{\pgfqpoint{2.966169in}{4.068875in}}{\pgfqpoint{2.961522in}{4.073522in}}%
\pgfpathcurveto{\pgfqpoint{2.956875in}{4.078169in}}{\pgfqpoint{2.950572in}{4.080780in}}{\pgfqpoint{2.944000in}{4.080780in}}%
\pgfpathcurveto{\pgfqpoint{2.937428in}{4.080780in}}{\pgfqpoint{2.931125in}{4.078169in}}{\pgfqpoint{2.926478in}{4.073522in}}%
\pgfpathcurveto{\pgfqpoint{2.921831in}{4.068875in}}{\pgfqpoint{2.919220in}{4.062572in}}{\pgfqpoint{2.919220in}{4.056000in}}%
\pgfpathcurveto{\pgfqpoint{2.919220in}{4.049428in}}{\pgfqpoint{2.921831in}{4.043125in}}{\pgfqpoint{2.926478in}{4.038478in}}%
\pgfpathcurveto{\pgfqpoint{2.931125in}{4.033831in}}{\pgfqpoint{2.937428in}{4.031220in}}{\pgfqpoint{2.944000in}{4.031220in}}%
\pgfpathclose%
\pgfusepath{stroke,fill}%
\end{pgfscope}%
\begin{pgfscope}%
\pgfpathrectangle{\pgfqpoint{1.432000in}{0.528000in}}{\pgfqpoint{3.696000in}{3.696000in}}%
\pgfusepath{clip}%
\pgfsetbuttcap%
\pgfsetroundjoin%
\definecolor{currentfill}{rgb}{0.000000,0.000000,0.000000}%
\pgfsetfillcolor{currentfill}%
\pgfsetlinewidth{1.003750pt}%
\definecolor{currentstroke}{rgb}{0.000000,0.000000,0.000000}%
\pgfsetstrokecolor{currentstroke}%
\pgfsetdash{}{0pt}%
\pgfpathmoveto{\pgfqpoint{3.280000in}{0.671220in}}%
\pgfpathcurveto{\pgfqpoint{3.286572in}{0.671220in}}{\pgfqpoint{3.292875in}{0.673831in}}{\pgfqpoint{3.297522in}{0.678478in}}%
\pgfpathcurveto{\pgfqpoint{3.302169in}{0.683125in}}{\pgfqpoint{3.304780in}{0.689428in}}{\pgfqpoint{3.304780in}{0.696000in}}%
\pgfpathcurveto{\pgfqpoint{3.304780in}{0.702572in}}{\pgfqpoint{3.302169in}{0.708875in}}{\pgfqpoint{3.297522in}{0.713522in}}%
\pgfpathcurveto{\pgfqpoint{3.292875in}{0.718169in}}{\pgfqpoint{3.286572in}{0.720780in}}{\pgfqpoint{3.280000in}{0.720780in}}%
\pgfpathcurveto{\pgfqpoint{3.273428in}{0.720780in}}{\pgfqpoint{3.267125in}{0.718169in}}{\pgfqpoint{3.262478in}{0.713522in}}%
\pgfpathcurveto{\pgfqpoint{3.257831in}{0.708875in}}{\pgfqpoint{3.255220in}{0.702572in}}{\pgfqpoint{3.255220in}{0.696000in}}%
\pgfpathcurveto{\pgfqpoint{3.255220in}{0.689428in}}{\pgfqpoint{3.257831in}{0.683125in}}{\pgfqpoint{3.262478in}{0.678478in}}%
\pgfpathcurveto{\pgfqpoint{3.267125in}{0.673831in}}{\pgfqpoint{3.273428in}{0.671220in}}{\pgfqpoint{3.280000in}{0.671220in}}%
\pgfpathclose%
\pgfusepath{stroke,fill}%
\end{pgfscope}%
\begin{pgfscope}%
\pgfpathrectangle{\pgfqpoint{1.432000in}{0.528000in}}{\pgfqpoint{3.696000in}{3.696000in}}%
\pgfusepath{clip}%
\pgfsetbuttcap%
\pgfsetroundjoin%
\definecolor{currentfill}{rgb}{0.000000,0.000000,0.000000}%
\pgfsetfillcolor{currentfill}%
\pgfsetlinewidth{1.003750pt}%
\definecolor{currentstroke}{rgb}{0.000000,0.000000,0.000000}%
\pgfsetstrokecolor{currentstroke}%
\pgfsetdash{}{0pt}%
\pgfpathmoveto{\pgfqpoint{3.280000in}{1.007220in}}%
\pgfpathcurveto{\pgfqpoint{3.286572in}{1.007220in}}{\pgfqpoint{3.292875in}{1.009831in}}{\pgfqpoint{3.297522in}{1.014478in}}%
\pgfpathcurveto{\pgfqpoint{3.302169in}{1.019125in}}{\pgfqpoint{3.304780in}{1.025428in}}{\pgfqpoint{3.304780in}{1.032000in}}%
\pgfpathcurveto{\pgfqpoint{3.304780in}{1.038572in}}{\pgfqpoint{3.302169in}{1.044875in}}{\pgfqpoint{3.297522in}{1.049522in}}%
\pgfpathcurveto{\pgfqpoint{3.292875in}{1.054169in}}{\pgfqpoint{3.286572in}{1.056780in}}{\pgfqpoint{3.280000in}{1.056780in}}%
\pgfpathcurveto{\pgfqpoint{3.273428in}{1.056780in}}{\pgfqpoint{3.267125in}{1.054169in}}{\pgfqpoint{3.262478in}{1.049522in}}%
\pgfpathcurveto{\pgfqpoint{3.257831in}{1.044875in}}{\pgfqpoint{3.255220in}{1.038572in}}{\pgfqpoint{3.255220in}{1.032000in}}%
\pgfpathcurveto{\pgfqpoint{3.255220in}{1.025428in}}{\pgfqpoint{3.257831in}{1.019125in}}{\pgfqpoint{3.262478in}{1.014478in}}%
\pgfpathcurveto{\pgfqpoint{3.267125in}{1.009831in}}{\pgfqpoint{3.273428in}{1.007220in}}{\pgfqpoint{3.280000in}{1.007220in}}%
\pgfpathclose%
\pgfusepath{stroke,fill}%
\end{pgfscope}%
\begin{pgfscope}%
\pgfpathrectangle{\pgfqpoint{1.432000in}{0.528000in}}{\pgfqpoint{3.696000in}{3.696000in}}%
\pgfusepath{clip}%
\pgfsetbuttcap%
\pgfsetroundjoin%
\definecolor{currentfill}{rgb}{0.000000,0.000000,0.000000}%
\pgfsetfillcolor{currentfill}%
\pgfsetlinewidth{1.003750pt}%
\definecolor{currentstroke}{rgb}{0.000000,0.000000,0.000000}%
\pgfsetstrokecolor{currentstroke}%
\pgfsetdash{}{0pt}%
\pgfpathmoveto{\pgfqpoint{3.280000in}{1.343220in}}%
\pgfpathcurveto{\pgfqpoint{3.286572in}{1.343220in}}{\pgfqpoint{3.292875in}{1.345831in}}{\pgfqpoint{3.297522in}{1.350478in}}%
\pgfpathcurveto{\pgfqpoint{3.302169in}{1.355125in}}{\pgfqpoint{3.304780in}{1.361428in}}{\pgfqpoint{3.304780in}{1.368000in}}%
\pgfpathcurveto{\pgfqpoint{3.304780in}{1.374572in}}{\pgfqpoint{3.302169in}{1.380875in}}{\pgfqpoint{3.297522in}{1.385522in}}%
\pgfpathcurveto{\pgfqpoint{3.292875in}{1.390169in}}{\pgfqpoint{3.286572in}{1.392780in}}{\pgfqpoint{3.280000in}{1.392780in}}%
\pgfpathcurveto{\pgfqpoint{3.273428in}{1.392780in}}{\pgfqpoint{3.267125in}{1.390169in}}{\pgfqpoint{3.262478in}{1.385522in}}%
\pgfpathcurveto{\pgfqpoint{3.257831in}{1.380875in}}{\pgfqpoint{3.255220in}{1.374572in}}{\pgfqpoint{3.255220in}{1.368000in}}%
\pgfpathcurveto{\pgfqpoint{3.255220in}{1.361428in}}{\pgfqpoint{3.257831in}{1.355125in}}{\pgfqpoint{3.262478in}{1.350478in}}%
\pgfpathcurveto{\pgfqpoint{3.267125in}{1.345831in}}{\pgfqpoint{3.273428in}{1.343220in}}{\pgfqpoint{3.280000in}{1.343220in}}%
\pgfpathclose%
\pgfusepath{stroke,fill}%
\end{pgfscope}%
\begin{pgfscope}%
\pgfpathrectangle{\pgfqpoint{1.432000in}{0.528000in}}{\pgfqpoint{3.696000in}{3.696000in}}%
\pgfusepath{clip}%
\pgfsetbuttcap%
\pgfsetroundjoin%
\definecolor{currentfill}{rgb}{0.000000,0.000000,0.000000}%
\pgfsetfillcolor{currentfill}%
\pgfsetlinewidth{1.003750pt}%
\definecolor{currentstroke}{rgb}{0.000000,0.000000,0.000000}%
\pgfsetstrokecolor{currentstroke}%
\pgfsetdash{}{0pt}%
\pgfpathmoveto{\pgfqpoint{3.280000in}{1.679220in}}%
\pgfpathcurveto{\pgfqpoint{3.286572in}{1.679220in}}{\pgfqpoint{3.292875in}{1.681831in}}{\pgfqpoint{3.297522in}{1.686478in}}%
\pgfpathcurveto{\pgfqpoint{3.302169in}{1.691125in}}{\pgfqpoint{3.304780in}{1.697428in}}{\pgfqpoint{3.304780in}{1.704000in}}%
\pgfpathcurveto{\pgfqpoint{3.304780in}{1.710572in}}{\pgfqpoint{3.302169in}{1.716875in}}{\pgfqpoint{3.297522in}{1.721522in}}%
\pgfpathcurveto{\pgfqpoint{3.292875in}{1.726169in}}{\pgfqpoint{3.286572in}{1.728780in}}{\pgfqpoint{3.280000in}{1.728780in}}%
\pgfpathcurveto{\pgfqpoint{3.273428in}{1.728780in}}{\pgfqpoint{3.267125in}{1.726169in}}{\pgfqpoint{3.262478in}{1.721522in}}%
\pgfpathcurveto{\pgfqpoint{3.257831in}{1.716875in}}{\pgfqpoint{3.255220in}{1.710572in}}{\pgfqpoint{3.255220in}{1.704000in}}%
\pgfpathcurveto{\pgfqpoint{3.255220in}{1.697428in}}{\pgfqpoint{3.257831in}{1.691125in}}{\pgfqpoint{3.262478in}{1.686478in}}%
\pgfpathcurveto{\pgfqpoint{3.267125in}{1.681831in}}{\pgfqpoint{3.273428in}{1.679220in}}{\pgfqpoint{3.280000in}{1.679220in}}%
\pgfpathclose%
\pgfusepath{stroke,fill}%
\end{pgfscope}%
\begin{pgfscope}%
\pgfpathrectangle{\pgfqpoint{1.432000in}{0.528000in}}{\pgfqpoint{3.696000in}{3.696000in}}%
\pgfusepath{clip}%
\pgfsetbuttcap%
\pgfsetroundjoin%
\definecolor{currentfill}{rgb}{0.000000,0.000000,0.000000}%
\pgfsetfillcolor{currentfill}%
\pgfsetlinewidth{1.003750pt}%
\definecolor{currentstroke}{rgb}{0.000000,0.000000,0.000000}%
\pgfsetstrokecolor{currentstroke}%
\pgfsetdash{}{0pt}%
\pgfpathmoveto{\pgfqpoint{3.280000in}{2.015220in}}%
\pgfpathcurveto{\pgfqpoint{3.286572in}{2.015220in}}{\pgfqpoint{3.292875in}{2.017831in}}{\pgfqpoint{3.297522in}{2.022478in}}%
\pgfpathcurveto{\pgfqpoint{3.302169in}{2.027125in}}{\pgfqpoint{3.304780in}{2.033428in}}{\pgfqpoint{3.304780in}{2.040000in}}%
\pgfpathcurveto{\pgfqpoint{3.304780in}{2.046572in}}{\pgfqpoint{3.302169in}{2.052875in}}{\pgfqpoint{3.297522in}{2.057522in}}%
\pgfpathcurveto{\pgfqpoint{3.292875in}{2.062169in}}{\pgfqpoint{3.286572in}{2.064780in}}{\pgfqpoint{3.280000in}{2.064780in}}%
\pgfpathcurveto{\pgfqpoint{3.273428in}{2.064780in}}{\pgfqpoint{3.267125in}{2.062169in}}{\pgfqpoint{3.262478in}{2.057522in}}%
\pgfpathcurveto{\pgfqpoint{3.257831in}{2.052875in}}{\pgfqpoint{3.255220in}{2.046572in}}{\pgfqpoint{3.255220in}{2.040000in}}%
\pgfpathcurveto{\pgfqpoint{3.255220in}{2.033428in}}{\pgfqpoint{3.257831in}{2.027125in}}{\pgfqpoint{3.262478in}{2.022478in}}%
\pgfpathcurveto{\pgfqpoint{3.267125in}{2.017831in}}{\pgfqpoint{3.273428in}{2.015220in}}{\pgfqpoint{3.280000in}{2.015220in}}%
\pgfpathclose%
\pgfusepath{stroke,fill}%
\end{pgfscope}%
\begin{pgfscope}%
\pgfpathrectangle{\pgfqpoint{1.432000in}{0.528000in}}{\pgfqpoint{3.696000in}{3.696000in}}%
\pgfusepath{clip}%
\pgfsetbuttcap%
\pgfsetroundjoin%
\definecolor{currentfill}{rgb}{0.000000,0.000000,0.000000}%
\pgfsetfillcolor{currentfill}%
\pgfsetlinewidth{1.003750pt}%
\definecolor{currentstroke}{rgb}{0.000000,0.000000,0.000000}%
\pgfsetstrokecolor{currentstroke}%
\pgfsetdash{}{0pt}%
\pgfpathmoveto{\pgfqpoint{3.280000in}{2.351220in}}%
\pgfpathcurveto{\pgfqpoint{3.286572in}{2.351220in}}{\pgfqpoint{3.292875in}{2.353831in}}{\pgfqpoint{3.297522in}{2.358478in}}%
\pgfpathcurveto{\pgfqpoint{3.302169in}{2.363125in}}{\pgfqpoint{3.304780in}{2.369428in}}{\pgfqpoint{3.304780in}{2.376000in}}%
\pgfpathcurveto{\pgfqpoint{3.304780in}{2.382572in}}{\pgfqpoint{3.302169in}{2.388875in}}{\pgfqpoint{3.297522in}{2.393522in}}%
\pgfpathcurveto{\pgfqpoint{3.292875in}{2.398169in}}{\pgfqpoint{3.286572in}{2.400780in}}{\pgfqpoint{3.280000in}{2.400780in}}%
\pgfpathcurveto{\pgfqpoint{3.273428in}{2.400780in}}{\pgfqpoint{3.267125in}{2.398169in}}{\pgfqpoint{3.262478in}{2.393522in}}%
\pgfpathcurveto{\pgfqpoint{3.257831in}{2.388875in}}{\pgfqpoint{3.255220in}{2.382572in}}{\pgfqpoint{3.255220in}{2.376000in}}%
\pgfpathcurveto{\pgfqpoint{3.255220in}{2.369428in}}{\pgfqpoint{3.257831in}{2.363125in}}{\pgfqpoint{3.262478in}{2.358478in}}%
\pgfpathcurveto{\pgfqpoint{3.267125in}{2.353831in}}{\pgfqpoint{3.273428in}{2.351220in}}{\pgfqpoint{3.280000in}{2.351220in}}%
\pgfpathclose%
\pgfusepath{stroke,fill}%
\end{pgfscope}%
\begin{pgfscope}%
\pgfpathrectangle{\pgfqpoint{1.432000in}{0.528000in}}{\pgfqpoint{3.696000in}{3.696000in}}%
\pgfusepath{clip}%
\pgfsetbuttcap%
\pgfsetroundjoin%
\definecolor{currentfill}{rgb}{0.000000,0.000000,0.000000}%
\pgfsetfillcolor{currentfill}%
\pgfsetlinewidth{1.003750pt}%
\definecolor{currentstroke}{rgb}{0.000000,0.000000,0.000000}%
\pgfsetstrokecolor{currentstroke}%
\pgfsetdash{}{0pt}%
\pgfpathmoveto{\pgfqpoint{3.280000in}{2.687220in}}%
\pgfpathcurveto{\pgfqpoint{3.286572in}{2.687220in}}{\pgfqpoint{3.292875in}{2.689831in}}{\pgfqpoint{3.297522in}{2.694478in}}%
\pgfpathcurveto{\pgfqpoint{3.302169in}{2.699125in}}{\pgfqpoint{3.304780in}{2.705428in}}{\pgfqpoint{3.304780in}{2.712000in}}%
\pgfpathcurveto{\pgfqpoint{3.304780in}{2.718572in}}{\pgfqpoint{3.302169in}{2.724875in}}{\pgfqpoint{3.297522in}{2.729522in}}%
\pgfpathcurveto{\pgfqpoint{3.292875in}{2.734169in}}{\pgfqpoint{3.286572in}{2.736780in}}{\pgfqpoint{3.280000in}{2.736780in}}%
\pgfpathcurveto{\pgfqpoint{3.273428in}{2.736780in}}{\pgfqpoint{3.267125in}{2.734169in}}{\pgfqpoint{3.262478in}{2.729522in}}%
\pgfpathcurveto{\pgfqpoint{3.257831in}{2.724875in}}{\pgfqpoint{3.255220in}{2.718572in}}{\pgfqpoint{3.255220in}{2.712000in}}%
\pgfpathcurveto{\pgfqpoint{3.255220in}{2.705428in}}{\pgfqpoint{3.257831in}{2.699125in}}{\pgfqpoint{3.262478in}{2.694478in}}%
\pgfpathcurveto{\pgfqpoint{3.267125in}{2.689831in}}{\pgfqpoint{3.273428in}{2.687220in}}{\pgfqpoint{3.280000in}{2.687220in}}%
\pgfpathclose%
\pgfusepath{stroke,fill}%
\end{pgfscope}%
\begin{pgfscope}%
\pgfpathrectangle{\pgfqpoint{1.432000in}{0.528000in}}{\pgfqpoint{3.696000in}{3.696000in}}%
\pgfusepath{clip}%
\pgfsetbuttcap%
\pgfsetroundjoin%
\definecolor{currentfill}{rgb}{0.000000,0.000000,0.000000}%
\pgfsetfillcolor{currentfill}%
\pgfsetlinewidth{1.003750pt}%
\definecolor{currentstroke}{rgb}{0.000000,0.000000,0.000000}%
\pgfsetstrokecolor{currentstroke}%
\pgfsetdash{}{0pt}%
\pgfpathmoveto{\pgfqpoint{3.280000in}{3.023220in}}%
\pgfpathcurveto{\pgfqpoint{3.286572in}{3.023220in}}{\pgfqpoint{3.292875in}{3.025831in}}{\pgfqpoint{3.297522in}{3.030478in}}%
\pgfpathcurveto{\pgfqpoint{3.302169in}{3.035125in}}{\pgfqpoint{3.304780in}{3.041428in}}{\pgfqpoint{3.304780in}{3.048000in}}%
\pgfpathcurveto{\pgfqpoint{3.304780in}{3.054572in}}{\pgfqpoint{3.302169in}{3.060875in}}{\pgfqpoint{3.297522in}{3.065522in}}%
\pgfpathcurveto{\pgfqpoint{3.292875in}{3.070169in}}{\pgfqpoint{3.286572in}{3.072780in}}{\pgfqpoint{3.280000in}{3.072780in}}%
\pgfpathcurveto{\pgfqpoint{3.273428in}{3.072780in}}{\pgfqpoint{3.267125in}{3.070169in}}{\pgfqpoint{3.262478in}{3.065522in}}%
\pgfpathcurveto{\pgfqpoint{3.257831in}{3.060875in}}{\pgfqpoint{3.255220in}{3.054572in}}{\pgfqpoint{3.255220in}{3.048000in}}%
\pgfpathcurveto{\pgfqpoint{3.255220in}{3.041428in}}{\pgfqpoint{3.257831in}{3.035125in}}{\pgfqpoint{3.262478in}{3.030478in}}%
\pgfpathcurveto{\pgfqpoint{3.267125in}{3.025831in}}{\pgfqpoint{3.273428in}{3.023220in}}{\pgfqpoint{3.280000in}{3.023220in}}%
\pgfpathclose%
\pgfusepath{stroke,fill}%
\end{pgfscope}%
\begin{pgfscope}%
\pgfpathrectangle{\pgfqpoint{1.432000in}{0.528000in}}{\pgfqpoint{3.696000in}{3.696000in}}%
\pgfusepath{clip}%
\pgfsetbuttcap%
\pgfsetroundjoin%
\definecolor{currentfill}{rgb}{0.000000,0.000000,0.000000}%
\pgfsetfillcolor{currentfill}%
\pgfsetlinewidth{1.003750pt}%
\definecolor{currentstroke}{rgb}{0.000000,0.000000,0.000000}%
\pgfsetstrokecolor{currentstroke}%
\pgfsetdash{}{0pt}%
\pgfpathmoveto{\pgfqpoint{3.280000in}{3.359220in}}%
\pgfpathcurveto{\pgfqpoint{3.286572in}{3.359220in}}{\pgfqpoint{3.292875in}{3.361831in}}{\pgfqpoint{3.297522in}{3.366478in}}%
\pgfpathcurveto{\pgfqpoint{3.302169in}{3.371125in}}{\pgfqpoint{3.304780in}{3.377428in}}{\pgfqpoint{3.304780in}{3.384000in}}%
\pgfpathcurveto{\pgfqpoint{3.304780in}{3.390572in}}{\pgfqpoint{3.302169in}{3.396875in}}{\pgfqpoint{3.297522in}{3.401522in}}%
\pgfpathcurveto{\pgfqpoint{3.292875in}{3.406169in}}{\pgfqpoint{3.286572in}{3.408780in}}{\pgfqpoint{3.280000in}{3.408780in}}%
\pgfpathcurveto{\pgfqpoint{3.273428in}{3.408780in}}{\pgfqpoint{3.267125in}{3.406169in}}{\pgfqpoint{3.262478in}{3.401522in}}%
\pgfpathcurveto{\pgfqpoint{3.257831in}{3.396875in}}{\pgfqpoint{3.255220in}{3.390572in}}{\pgfqpoint{3.255220in}{3.384000in}}%
\pgfpathcurveto{\pgfqpoint{3.255220in}{3.377428in}}{\pgfqpoint{3.257831in}{3.371125in}}{\pgfqpoint{3.262478in}{3.366478in}}%
\pgfpathcurveto{\pgfqpoint{3.267125in}{3.361831in}}{\pgfqpoint{3.273428in}{3.359220in}}{\pgfqpoint{3.280000in}{3.359220in}}%
\pgfpathclose%
\pgfusepath{stroke,fill}%
\end{pgfscope}%
\begin{pgfscope}%
\pgfpathrectangle{\pgfqpoint{1.432000in}{0.528000in}}{\pgfqpoint{3.696000in}{3.696000in}}%
\pgfusepath{clip}%
\pgfsetbuttcap%
\pgfsetroundjoin%
\definecolor{currentfill}{rgb}{0.000000,0.000000,0.000000}%
\pgfsetfillcolor{currentfill}%
\pgfsetlinewidth{1.003750pt}%
\definecolor{currentstroke}{rgb}{0.000000,0.000000,0.000000}%
\pgfsetstrokecolor{currentstroke}%
\pgfsetdash{}{0pt}%
\pgfpathmoveto{\pgfqpoint{3.280000in}{3.695220in}}%
\pgfpathcurveto{\pgfqpoint{3.286572in}{3.695220in}}{\pgfqpoint{3.292875in}{3.697831in}}{\pgfqpoint{3.297522in}{3.702478in}}%
\pgfpathcurveto{\pgfqpoint{3.302169in}{3.707125in}}{\pgfqpoint{3.304780in}{3.713428in}}{\pgfqpoint{3.304780in}{3.720000in}}%
\pgfpathcurveto{\pgfqpoint{3.304780in}{3.726572in}}{\pgfqpoint{3.302169in}{3.732875in}}{\pgfqpoint{3.297522in}{3.737522in}}%
\pgfpathcurveto{\pgfqpoint{3.292875in}{3.742169in}}{\pgfqpoint{3.286572in}{3.744780in}}{\pgfqpoint{3.280000in}{3.744780in}}%
\pgfpathcurveto{\pgfqpoint{3.273428in}{3.744780in}}{\pgfqpoint{3.267125in}{3.742169in}}{\pgfqpoint{3.262478in}{3.737522in}}%
\pgfpathcurveto{\pgfqpoint{3.257831in}{3.732875in}}{\pgfqpoint{3.255220in}{3.726572in}}{\pgfqpoint{3.255220in}{3.720000in}}%
\pgfpathcurveto{\pgfqpoint{3.255220in}{3.713428in}}{\pgfqpoint{3.257831in}{3.707125in}}{\pgfqpoint{3.262478in}{3.702478in}}%
\pgfpathcurveto{\pgfqpoint{3.267125in}{3.697831in}}{\pgfqpoint{3.273428in}{3.695220in}}{\pgfqpoint{3.280000in}{3.695220in}}%
\pgfpathclose%
\pgfusepath{stroke,fill}%
\end{pgfscope}%
\begin{pgfscope}%
\pgfpathrectangle{\pgfqpoint{1.432000in}{0.528000in}}{\pgfqpoint{3.696000in}{3.696000in}}%
\pgfusepath{clip}%
\pgfsetbuttcap%
\pgfsetroundjoin%
\definecolor{currentfill}{rgb}{0.000000,0.000000,0.000000}%
\pgfsetfillcolor{currentfill}%
\pgfsetlinewidth{1.003750pt}%
\definecolor{currentstroke}{rgb}{0.000000,0.000000,0.000000}%
\pgfsetstrokecolor{currentstroke}%
\pgfsetdash{}{0pt}%
\pgfpathmoveto{\pgfqpoint{3.280000in}{4.031220in}}%
\pgfpathcurveto{\pgfqpoint{3.286572in}{4.031220in}}{\pgfqpoint{3.292875in}{4.033831in}}{\pgfqpoint{3.297522in}{4.038478in}}%
\pgfpathcurveto{\pgfqpoint{3.302169in}{4.043125in}}{\pgfqpoint{3.304780in}{4.049428in}}{\pgfqpoint{3.304780in}{4.056000in}}%
\pgfpathcurveto{\pgfqpoint{3.304780in}{4.062572in}}{\pgfqpoint{3.302169in}{4.068875in}}{\pgfqpoint{3.297522in}{4.073522in}}%
\pgfpathcurveto{\pgfqpoint{3.292875in}{4.078169in}}{\pgfqpoint{3.286572in}{4.080780in}}{\pgfqpoint{3.280000in}{4.080780in}}%
\pgfpathcurveto{\pgfqpoint{3.273428in}{4.080780in}}{\pgfqpoint{3.267125in}{4.078169in}}{\pgfqpoint{3.262478in}{4.073522in}}%
\pgfpathcurveto{\pgfqpoint{3.257831in}{4.068875in}}{\pgfqpoint{3.255220in}{4.062572in}}{\pgfqpoint{3.255220in}{4.056000in}}%
\pgfpathcurveto{\pgfqpoint{3.255220in}{4.049428in}}{\pgfqpoint{3.257831in}{4.043125in}}{\pgfqpoint{3.262478in}{4.038478in}}%
\pgfpathcurveto{\pgfqpoint{3.267125in}{4.033831in}}{\pgfqpoint{3.273428in}{4.031220in}}{\pgfqpoint{3.280000in}{4.031220in}}%
\pgfpathclose%
\pgfusepath{stroke,fill}%
\end{pgfscope}%
\begin{pgfscope}%
\pgfpathrectangle{\pgfqpoint{1.432000in}{0.528000in}}{\pgfqpoint{3.696000in}{3.696000in}}%
\pgfusepath{clip}%
\pgfsetbuttcap%
\pgfsetroundjoin%
\definecolor{currentfill}{rgb}{0.000000,0.000000,0.000000}%
\pgfsetfillcolor{currentfill}%
\pgfsetlinewidth{1.003750pt}%
\definecolor{currentstroke}{rgb}{0.000000,0.000000,0.000000}%
\pgfsetstrokecolor{currentstroke}%
\pgfsetdash{}{0pt}%
\pgfpathmoveto{\pgfqpoint{3.616000in}{0.671220in}}%
\pgfpathcurveto{\pgfqpoint{3.622572in}{0.671220in}}{\pgfqpoint{3.628875in}{0.673831in}}{\pgfqpoint{3.633522in}{0.678478in}}%
\pgfpathcurveto{\pgfqpoint{3.638169in}{0.683125in}}{\pgfqpoint{3.640780in}{0.689428in}}{\pgfqpoint{3.640780in}{0.696000in}}%
\pgfpathcurveto{\pgfqpoint{3.640780in}{0.702572in}}{\pgfqpoint{3.638169in}{0.708875in}}{\pgfqpoint{3.633522in}{0.713522in}}%
\pgfpathcurveto{\pgfqpoint{3.628875in}{0.718169in}}{\pgfqpoint{3.622572in}{0.720780in}}{\pgfqpoint{3.616000in}{0.720780in}}%
\pgfpathcurveto{\pgfqpoint{3.609428in}{0.720780in}}{\pgfqpoint{3.603125in}{0.718169in}}{\pgfqpoint{3.598478in}{0.713522in}}%
\pgfpathcurveto{\pgfqpoint{3.593831in}{0.708875in}}{\pgfqpoint{3.591220in}{0.702572in}}{\pgfqpoint{3.591220in}{0.696000in}}%
\pgfpathcurveto{\pgfqpoint{3.591220in}{0.689428in}}{\pgfqpoint{3.593831in}{0.683125in}}{\pgfqpoint{3.598478in}{0.678478in}}%
\pgfpathcurveto{\pgfqpoint{3.603125in}{0.673831in}}{\pgfqpoint{3.609428in}{0.671220in}}{\pgfqpoint{3.616000in}{0.671220in}}%
\pgfpathclose%
\pgfusepath{stroke,fill}%
\end{pgfscope}%
\begin{pgfscope}%
\pgfpathrectangle{\pgfqpoint{1.432000in}{0.528000in}}{\pgfqpoint{3.696000in}{3.696000in}}%
\pgfusepath{clip}%
\pgfsetbuttcap%
\pgfsetroundjoin%
\definecolor{currentfill}{rgb}{0.000000,0.000000,0.000000}%
\pgfsetfillcolor{currentfill}%
\pgfsetlinewidth{1.003750pt}%
\definecolor{currentstroke}{rgb}{0.000000,0.000000,0.000000}%
\pgfsetstrokecolor{currentstroke}%
\pgfsetdash{}{0pt}%
\pgfpathmoveto{\pgfqpoint{3.616000in}{1.007220in}}%
\pgfpathcurveto{\pgfqpoint{3.622572in}{1.007220in}}{\pgfqpoint{3.628875in}{1.009831in}}{\pgfqpoint{3.633522in}{1.014478in}}%
\pgfpathcurveto{\pgfqpoint{3.638169in}{1.019125in}}{\pgfqpoint{3.640780in}{1.025428in}}{\pgfqpoint{3.640780in}{1.032000in}}%
\pgfpathcurveto{\pgfqpoint{3.640780in}{1.038572in}}{\pgfqpoint{3.638169in}{1.044875in}}{\pgfqpoint{3.633522in}{1.049522in}}%
\pgfpathcurveto{\pgfqpoint{3.628875in}{1.054169in}}{\pgfqpoint{3.622572in}{1.056780in}}{\pgfqpoint{3.616000in}{1.056780in}}%
\pgfpathcurveto{\pgfqpoint{3.609428in}{1.056780in}}{\pgfqpoint{3.603125in}{1.054169in}}{\pgfqpoint{3.598478in}{1.049522in}}%
\pgfpathcurveto{\pgfqpoint{3.593831in}{1.044875in}}{\pgfqpoint{3.591220in}{1.038572in}}{\pgfqpoint{3.591220in}{1.032000in}}%
\pgfpathcurveto{\pgfqpoint{3.591220in}{1.025428in}}{\pgfqpoint{3.593831in}{1.019125in}}{\pgfqpoint{3.598478in}{1.014478in}}%
\pgfpathcurveto{\pgfqpoint{3.603125in}{1.009831in}}{\pgfqpoint{3.609428in}{1.007220in}}{\pgfqpoint{3.616000in}{1.007220in}}%
\pgfpathclose%
\pgfusepath{stroke,fill}%
\end{pgfscope}%
\begin{pgfscope}%
\pgfpathrectangle{\pgfqpoint{1.432000in}{0.528000in}}{\pgfqpoint{3.696000in}{3.696000in}}%
\pgfusepath{clip}%
\pgfsetbuttcap%
\pgfsetroundjoin%
\definecolor{currentfill}{rgb}{0.000000,0.000000,0.000000}%
\pgfsetfillcolor{currentfill}%
\pgfsetlinewidth{1.003750pt}%
\definecolor{currentstroke}{rgb}{0.000000,0.000000,0.000000}%
\pgfsetstrokecolor{currentstroke}%
\pgfsetdash{}{0pt}%
\pgfpathmoveto{\pgfqpoint{3.616000in}{1.343220in}}%
\pgfpathcurveto{\pgfqpoint{3.622572in}{1.343220in}}{\pgfqpoint{3.628875in}{1.345831in}}{\pgfqpoint{3.633522in}{1.350478in}}%
\pgfpathcurveto{\pgfqpoint{3.638169in}{1.355125in}}{\pgfqpoint{3.640780in}{1.361428in}}{\pgfqpoint{3.640780in}{1.368000in}}%
\pgfpathcurveto{\pgfqpoint{3.640780in}{1.374572in}}{\pgfqpoint{3.638169in}{1.380875in}}{\pgfqpoint{3.633522in}{1.385522in}}%
\pgfpathcurveto{\pgfqpoint{3.628875in}{1.390169in}}{\pgfqpoint{3.622572in}{1.392780in}}{\pgfqpoint{3.616000in}{1.392780in}}%
\pgfpathcurveto{\pgfqpoint{3.609428in}{1.392780in}}{\pgfqpoint{3.603125in}{1.390169in}}{\pgfqpoint{3.598478in}{1.385522in}}%
\pgfpathcurveto{\pgfqpoint{3.593831in}{1.380875in}}{\pgfqpoint{3.591220in}{1.374572in}}{\pgfqpoint{3.591220in}{1.368000in}}%
\pgfpathcurveto{\pgfqpoint{3.591220in}{1.361428in}}{\pgfqpoint{3.593831in}{1.355125in}}{\pgfqpoint{3.598478in}{1.350478in}}%
\pgfpathcurveto{\pgfqpoint{3.603125in}{1.345831in}}{\pgfqpoint{3.609428in}{1.343220in}}{\pgfqpoint{3.616000in}{1.343220in}}%
\pgfpathclose%
\pgfusepath{stroke,fill}%
\end{pgfscope}%
\begin{pgfscope}%
\pgfpathrectangle{\pgfqpoint{1.432000in}{0.528000in}}{\pgfqpoint{3.696000in}{3.696000in}}%
\pgfusepath{clip}%
\pgfsetbuttcap%
\pgfsetroundjoin%
\definecolor{currentfill}{rgb}{0.000000,0.000000,0.000000}%
\pgfsetfillcolor{currentfill}%
\pgfsetlinewidth{1.003750pt}%
\definecolor{currentstroke}{rgb}{0.000000,0.000000,0.000000}%
\pgfsetstrokecolor{currentstroke}%
\pgfsetdash{}{0pt}%
\pgfpathmoveto{\pgfqpoint{3.616000in}{1.679220in}}%
\pgfpathcurveto{\pgfqpoint{3.622572in}{1.679220in}}{\pgfqpoint{3.628875in}{1.681831in}}{\pgfqpoint{3.633522in}{1.686478in}}%
\pgfpathcurveto{\pgfqpoint{3.638169in}{1.691125in}}{\pgfqpoint{3.640780in}{1.697428in}}{\pgfqpoint{3.640780in}{1.704000in}}%
\pgfpathcurveto{\pgfqpoint{3.640780in}{1.710572in}}{\pgfqpoint{3.638169in}{1.716875in}}{\pgfqpoint{3.633522in}{1.721522in}}%
\pgfpathcurveto{\pgfqpoint{3.628875in}{1.726169in}}{\pgfqpoint{3.622572in}{1.728780in}}{\pgfqpoint{3.616000in}{1.728780in}}%
\pgfpathcurveto{\pgfqpoint{3.609428in}{1.728780in}}{\pgfqpoint{3.603125in}{1.726169in}}{\pgfqpoint{3.598478in}{1.721522in}}%
\pgfpathcurveto{\pgfqpoint{3.593831in}{1.716875in}}{\pgfqpoint{3.591220in}{1.710572in}}{\pgfqpoint{3.591220in}{1.704000in}}%
\pgfpathcurveto{\pgfqpoint{3.591220in}{1.697428in}}{\pgfqpoint{3.593831in}{1.691125in}}{\pgfqpoint{3.598478in}{1.686478in}}%
\pgfpathcurveto{\pgfqpoint{3.603125in}{1.681831in}}{\pgfqpoint{3.609428in}{1.679220in}}{\pgfqpoint{3.616000in}{1.679220in}}%
\pgfpathclose%
\pgfusepath{stroke,fill}%
\end{pgfscope}%
\begin{pgfscope}%
\pgfpathrectangle{\pgfqpoint{1.432000in}{0.528000in}}{\pgfqpoint{3.696000in}{3.696000in}}%
\pgfusepath{clip}%
\pgfsetbuttcap%
\pgfsetroundjoin%
\definecolor{currentfill}{rgb}{0.000000,0.000000,0.000000}%
\pgfsetfillcolor{currentfill}%
\pgfsetlinewidth{1.003750pt}%
\definecolor{currentstroke}{rgb}{0.000000,0.000000,0.000000}%
\pgfsetstrokecolor{currentstroke}%
\pgfsetdash{}{0pt}%
\pgfpathmoveto{\pgfqpoint{3.616000in}{2.015220in}}%
\pgfpathcurveto{\pgfqpoint{3.622572in}{2.015220in}}{\pgfqpoint{3.628875in}{2.017831in}}{\pgfqpoint{3.633522in}{2.022478in}}%
\pgfpathcurveto{\pgfqpoint{3.638169in}{2.027125in}}{\pgfqpoint{3.640780in}{2.033428in}}{\pgfqpoint{3.640780in}{2.040000in}}%
\pgfpathcurveto{\pgfqpoint{3.640780in}{2.046572in}}{\pgfqpoint{3.638169in}{2.052875in}}{\pgfqpoint{3.633522in}{2.057522in}}%
\pgfpathcurveto{\pgfqpoint{3.628875in}{2.062169in}}{\pgfqpoint{3.622572in}{2.064780in}}{\pgfqpoint{3.616000in}{2.064780in}}%
\pgfpathcurveto{\pgfqpoint{3.609428in}{2.064780in}}{\pgfqpoint{3.603125in}{2.062169in}}{\pgfqpoint{3.598478in}{2.057522in}}%
\pgfpathcurveto{\pgfqpoint{3.593831in}{2.052875in}}{\pgfqpoint{3.591220in}{2.046572in}}{\pgfqpoint{3.591220in}{2.040000in}}%
\pgfpathcurveto{\pgfqpoint{3.591220in}{2.033428in}}{\pgfqpoint{3.593831in}{2.027125in}}{\pgfqpoint{3.598478in}{2.022478in}}%
\pgfpathcurveto{\pgfqpoint{3.603125in}{2.017831in}}{\pgfqpoint{3.609428in}{2.015220in}}{\pgfqpoint{3.616000in}{2.015220in}}%
\pgfpathclose%
\pgfusepath{stroke,fill}%
\end{pgfscope}%
\begin{pgfscope}%
\pgfpathrectangle{\pgfqpoint{1.432000in}{0.528000in}}{\pgfqpoint{3.696000in}{3.696000in}}%
\pgfusepath{clip}%
\pgfsetbuttcap%
\pgfsetroundjoin%
\definecolor{currentfill}{rgb}{0.000000,0.000000,0.000000}%
\pgfsetfillcolor{currentfill}%
\pgfsetlinewidth{1.003750pt}%
\definecolor{currentstroke}{rgb}{0.000000,0.000000,0.000000}%
\pgfsetstrokecolor{currentstroke}%
\pgfsetdash{}{0pt}%
\pgfpathmoveto{\pgfqpoint{3.616000in}{2.351220in}}%
\pgfpathcurveto{\pgfqpoint{3.622572in}{2.351220in}}{\pgfqpoint{3.628875in}{2.353831in}}{\pgfqpoint{3.633522in}{2.358478in}}%
\pgfpathcurveto{\pgfqpoint{3.638169in}{2.363125in}}{\pgfqpoint{3.640780in}{2.369428in}}{\pgfqpoint{3.640780in}{2.376000in}}%
\pgfpathcurveto{\pgfqpoint{3.640780in}{2.382572in}}{\pgfqpoint{3.638169in}{2.388875in}}{\pgfqpoint{3.633522in}{2.393522in}}%
\pgfpathcurveto{\pgfqpoint{3.628875in}{2.398169in}}{\pgfqpoint{3.622572in}{2.400780in}}{\pgfqpoint{3.616000in}{2.400780in}}%
\pgfpathcurveto{\pgfqpoint{3.609428in}{2.400780in}}{\pgfqpoint{3.603125in}{2.398169in}}{\pgfqpoint{3.598478in}{2.393522in}}%
\pgfpathcurveto{\pgfqpoint{3.593831in}{2.388875in}}{\pgfqpoint{3.591220in}{2.382572in}}{\pgfqpoint{3.591220in}{2.376000in}}%
\pgfpathcurveto{\pgfqpoint{3.591220in}{2.369428in}}{\pgfqpoint{3.593831in}{2.363125in}}{\pgfqpoint{3.598478in}{2.358478in}}%
\pgfpathcurveto{\pgfqpoint{3.603125in}{2.353831in}}{\pgfqpoint{3.609428in}{2.351220in}}{\pgfqpoint{3.616000in}{2.351220in}}%
\pgfpathclose%
\pgfusepath{stroke,fill}%
\end{pgfscope}%
\begin{pgfscope}%
\pgfpathrectangle{\pgfqpoint{1.432000in}{0.528000in}}{\pgfqpoint{3.696000in}{3.696000in}}%
\pgfusepath{clip}%
\pgfsetbuttcap%
\pgfsetroundjoin%
\definecolor{currentfill}{rgb}{0.000000,0.000000,0.000000}%
\pgfsetfillcolor{currentfill}%
\pgfsetlinewidth{1.003750pt}%
\definecolor{currentstroke}{rgb}{0.000000,0.000000,0.000000}%
\pgfsetstrokecolor{currentstroke}%
\pgfsetdash{}{0pt}%
\pgfpathmoveto{\pgfqpoint{3.616000in}{2.687220in}}%
\pgfpathcurveto{\pgfqpoint{3.622572in}{2.687220in}}{\pgfqpoint{3.628875in}{2.689831in}}{\pgfqpoint{3.633522in}{2.694478in}}%
\pgfpathcurveto{\pgfqpoint{3.638169in}{2.699125in}}{\pgfqpoint{3.640780in}{2.705428in}}{\pgfqpoint{3.640780in}{2.712000in}}%
\pgfpathcurveto{\pgfqpoint{3.640780in}{2.718572in}}{\pgfqpoint{3.638169in}{2.724875in}}{\pgfqpoint{3.633522in}{2.729522in}}%
\pgfpathcurveto{\pgfqpoint{3.628875in}{2.734169in}}{\pgfqpoint{3.622572in}{2.736780in}}{\pgfqpoint{3.616000in}{2.736780in}}%
\pgfpathcurveto{\pgfqpoint{3.609428in}{2.736780in}}{\pgfqpoint{3.603125in}{2.734169in}}{\pgfqpoint{3.598478in}{2.729522in}}%
\pgfpathcurveto{\pgfqpoint{3.593831in}{2.724875in}}{\pgfqpoint{3.591220in}{2.718572in}}{\pgfqpoint{3.591220in}{2.712000in}}%
\pgfpathcurveto{\pgfqpoint{3.591220in}{2.705428in}}{\pgfqpoint{3.593831in}{2.699125in}}{\pgfqpoint{3.598478in}{2.694478in}}%
\pgfpathcurveto{\pgfqpoint{3.603125in}{2.689831in}}{\pgfqpoint{3.609428in}{2.687220in}}{\pgfqpoint{3.616000in}{2.687220in}}%
\pgfpathclose%
\pgfusepath{stroke,fill}%
\end{pgfscope}%
\begin{pgfscope}%
\pgfpathrectangle{\pgfqpoint{1.432000in}{0.528000in}}{\pgfqpoint{3.696000in}{3.696000in}}%
\pgfusepath{clip}%
\pgfsetbuttcap%
\pgfsetroundjoin%
\definecolor{currentfill}{rgb}{0.000000,0.000000,0.000000}%
\pgfsetfillcolor{currentfill}%
\pgfsetlinewidth{1.003750pt}%
\definecolor{currentstroke}{rgb}{0.000000,0.000000,0.000000}%
\pgfsetstrokecolor{currentstroke}%
\pgfsetdash{}{0pt}%
\pgfpathmoveto{\pgfqpoint{3.616000in}{3.023220in}}%
\pgfpathcurveto{\pgfqpoint{3.622572in}{3.023220in}}{\pgfqpoint{3.628875in}{3.025831in}}{\pgfqpoint{3.633522in}{3.030478in}}%
\pgfpathcurveto{\pgfqpoint{3.638169in}{3.035125in}}{\pgfqpoint{3.640780in}{3.041428in}}{\pgfqpoint{3.640780in}{3.048000in}}%
\pgfpathcurveto{\pgfqpoint{3.640780in}{3.054572in}}{\pgfqpoint{3.638169in}{3.060875in}}{\pgfqpoint{3.633522in}{3.065522in}}%
\pgfpathcurveto{\pgfqpoint{3.628875in}{3.070169in}}{\pgfqpoint{3.622572in}{3.072780in}}{\pgfqpoint{3.616000in}{3.072780in}}%
\pgfpathcurveto{\pgfqpoint{3.609428in}{3.072780in}}{\pgfqpoint{3.603125in}{3.070169in}}{\pgfqpoint{3.598478in}{3.065522in}}%
\pgfpathcurveto{\pgfqpoint{3.593831in}{3.060875in}}{\pgfqpoint{3.591220in}{3.054572in}}{\pgfqpoint{3.591220in}{3.048000in}}%
\pgfpathcurveto{\pgfqpoint{3.591220in}{3.041428in}}{\pgfqpoint{3.593831in}{3.035125in}}{\pgfqpoint{3.598478in}{3.030478in}}%
\pgfpathcurveto{\pgfqpoint{3.603125in}{3.025831in}}{\pgfqpoint{3.609428in}{3.023220in}}{\pgfqpoint{3.616000in}{3.023220in}}%
\pgfpathclose%
\pgfusepath{stroke,fill}%
\end{pgfscope}%
\begin{pgfscope}%
\pgfpathrectangle{\pgfqpoint{1.432000in}{0.528000in}}{\pgfqpoint{3.696000in}{3.696000in}}%
\pgfusepath{clip}%
\pgfsetbuttcap%
\pgfsetroundjoin%
\definecolor{currentfill}{rgb}{0.000000,0.000000,0.000000}%
\pgfsetfillcolor{currentfill}%
\pgfsetlinewidth{1.003750pt}%
\definecolor{currentstroke}{rgb}{0.000000,0.000000,0.000000}%
\pgfsetstrokecolor{currentstroke}%
\pgfsetdash{}{0pt}%
\pgfpathmoveto{\pgfqpoint{3.616000in}{3.359220in}}%
\pgfpathcurveto{\pgfqpoint{3.622572in}{3.359220in}}{\pgfqpoint{3.628875in}{3.361831in}}{\pgfqpoint{3.633522in}{3.366478in}}%
\pgfpathcurveto{\pgfqpoint{3.638169in}{3.371125in}}{\pgfqpoint{3.640780in}{3.377428in}}{\pgfqpoint{3.640780in}{3.384000in}}%
\pgfpathcurveto{\pgfqpoint{3.640780in}{3.390572in}}{\pgfqpoint{3.638169in}{3.396875in}}{\pgfqpoint{3.633522in}{3.401522in}}%
\pgfpathcurveto{\pgfqpoint{3.628875in}{3.406169in}}{\pgfqpoint{3.622572in}{3.408780in}}{\pgfqpoint{3.616000in}{3.408780in}}%
\pgfpathcurveto{\pgfqpoint{3.609428in}{3.408780in}}{\pgfqpoint{3.603125in}{3.406169in}}{\pgfqpoint{3.598478in}{3.401522in}}%
\pgfpathcurveto{\pgfqpoint{3.593831in}{3.396875in}}{\pgfqpoint{3.591220in}{3.390572in}}{\pgfqpoint{3.591220in}{3.384000in}}%
\pgfpathcurveto{\pgfqpoint{3.591220in}{3.377428in}}{\pgfqpoint{3.593831in}{3.371125in}}{\pgfqpoint{3.598478in}{3.366478in}}%
\pgfpathcurveto{\pgfqpoint{3.603125in}{3.361831in}}{\pgfqpoint{3.609428in}{3.359220in}}{\pgfqpoint{3.616000in}{3.359220in}}%
\pgfpathclose%
\pgfusepath{stroke,fill}%
\end{pgfscope}%
\begin{pgfscope}%
\pgfpathrectangle{\pgfqpoint{1.432000in}{0.528000in}}{\pgfqpoint{3.696000in}{3.696000in}}%
\pgfusepath{clip}%
\pgfsetbuttcap%
\pgfsetroundjoin%
\definecolor{currentfill}{rgb}{0.000000,0.000000,0.000000}%
\pgfsetfillcolor{currentfill}%
\pgfsetlinewidth{1.003750pt}%
\definecolor{currentstroke}{rgb}{0.000000,0.000000,0.000000}%
\pgfsetstrokecolor{currentstroke}%
\pgfsetdash{}{0pt}%
\pgfpathmoveto{\pgfqpoint{3.616000in}{3.695220in}}%
\pgfpathcurveto{\pgfqpoint{3.622572in}{3.695220in}}{\pgfqpoint{3.628875in}{3.697831in}}{\pgfqpoint{3.633522in}{3.702478in}}%
\pgfpathcurveto{\pgfqpoint{3.638169in}{3.707125in}}{\pgfqpoint{3.640780in}{3.713428in}}{\pgfqpoint{3.640780in}{3.720000in}}%
\pgfpathcurveto{\pgfqpoint{3.640780in}{3.726572in}}{\pgfqpoint{3.638169in}{3.732875in}}{\pgfqpoint{3.633522in}{3.737522in}}%
\pgfpathcurveto{\pgfqpoint{3.628875in}{3.742169in}}{\pgfqpoint{3.622572in}{3.744780in}}{\pgfqpoint{3.616000in}{3.744780in}}%
\pgfpathcurveto{\pgfqpoint{3.609428in}{3.744780in}}{\pgfqpoint{3.603125in}{3.742169in}}{\pgfqpoint{3.598478in}{3.737522in}}%
\pgfpathcurveto{\pgfqpoint{3.593831in}{3.732875in}}{\pgfqpoint{3.591220in}{3.726572in}}{\pgfqpoint{3.591220in}{3.720000in}}%
\pgfpathcurveto{\pgfqpoint{3.591220in}{3.713428in}}{\pgfqpoint{3.593831in}{3.707125in}}{\pgfqpoint{3.598478in}{3.702478in}}%
\pgfpathcurveto{\pgfqpoint{3.603125in}{3.697831in}}{\pgfqpoint{3.609428in}{3.695220in}}{\pgfqpoint{3.616000in}{3.695220in}}%
\pgfpathclose%
\pgfusepath{stroke,fill}%
\end{pgfscope}%
\begin{pgfscope}%
\pgfpathrectangle{\pgfqpoint{1.432000in}{0.528000in}}{\pgfqpoint{3.696000in}{3.696000in}}%
\pgfusepath{clip}%
\pgfsetbuttcap%
\pgfsetroundjoin%
\definecolor{currentfill}{rgb}{0.000000,0.000000,0.000000}%
\pgfsetfillcolor{currentfill}%
\pgfsetlinewidth{1.003750pt}%
\definecolor{currentstroke}{rgb}{0.000000,0.000000,0.000000}%
\pgfsetstrokecolor{currentstroke}%
\pgfsetdash{}{0pt}%
\pgfpathmoveto{\pgfqpoint{3.616000in}{4.031220in}}%
\pgfpathcurveto{\pgfqpoint{3.622572in}{4.031220in}}{\pgfqpoint{3.628875in}{4.033831in}}{\pgfqpoint{3.633522in}{4.038478in}}%
\pgfpathcurveto{\pgfqpoint{3.638169in}{4.043125in}}{\pgfqpoint{3.640780in}{4.049428in}}{\pgfqpoint{3.640780in}{4.056000in}}%
\pgfpathcurveto{\pgfqpoint{3.640780in}{4.062572in}}{\pgfqpoint{3.638169in}{4.068875in}}{\pgfqpoint{3.633522in}{4.073522in}}%
\pgfpathcurveto{\pgfqpoint{3.628875in}{4.078169in}}{\pgfqpoint{3.622572in}{4.080780in}}{\pgfqpoint{3.616000in}{4.080780in}}%
\pgfpathcurveto{\pgfqpoint{3.609428in}{4.080780in}}{\pgfqpoint{3.603125in}{4.078169in}}{\pgfqpoint{3.598478in}{4.073522in}}%
\pgfpathcurveto{\pgfqpoint{3.593831in}{4.068875in}}{\pgfqpoint{3.591220in}{4.062572in}}{\pgfqpoint{3.591220in}{4.056000in}}%
\pgfpathcurveto{\pgfqpoint{3.591220in}{4.049428in}}{\pgfqpoint{3.593831in}{4.043125in}}{\pgfqpoint{3.598478in}{4.038478in}}%
\pgfpathcurveto{\pgfqpoint{3.603125in}{4.033831in}}{\pgfqpoint{3.609428in}{4.031220in}}{\pgfqpoint{3.616000in}{4.031220in}}%
\pgfpathclose%
\pgfusepath{stroke,fill}%
\end{pgfscope}%
\begin{pgfscope}%
\pgfpathrectangle{\pgfqpoint{1.432000in}{0.528000in}}{\pgfqpoint{3.696000in}{3.696000in}}%
\pgfusepath{clip}%
\pgfsetbuttcap%
\pgfsetroundjoin%
\definecolor{currentfill}{rgb}{0.000000,0.000000,0.000000}%
\pgfsetfillcolor{currentfill}%
\pgfsetlinewidth{1.003750pt}%
\definecolor{currentstroke}{rgb}{0.000000,0.000000,0.000000}%
\pgfsetstrokecolor{currentstroke}%
\pgfsetdash{}{0pt}%
\pgfpathmoveto{\pgfqpoint{3.952000in}{0.671220in}}%
\pgfpathcurveto{\pgfqpoint{3.958572in}{0.671220in}}{\pgfqpoint{3.964875in}{0.673831in}}{\pgfqpoint{3.969522in}{0.678478in}}%
\pgfpathcurveto{\pgfqpoint{3.974169in}{0.683125in}}{\pgfqpoint{3.976780in}{0.689428in}}{\pgfqpoint{3.976780in}{0.696000in}}%
\pgfpathcurveto{\pgfqpoint{3.976780in}{0.702572in}}{\pgfqpoint{3.974169in}{0.708875in}}{\pgfqpoint{3.969522in}{0.713522in}}%
\pgfpathcurveto{\pgfqpoint{3.964875in}{0.718169in}}{\pgfqpoint{3.958572in}{0.720780in}}{\pgfqpoint{3.952000in}{0.720780in}}%
\pgfpathcurveto{\pgfqpoint{3.945428in}{0.720780in}}{\pgfqpoint{3.939125in}{0.718169in}}{\pgfqpoint{3.934478in}{0.713522in}}%
\pgfpathcurveto{\pgfqpoint{3.929831in}{0.708875in}}{\pgfqpoint{3.927220in}{0.702572in}}{\pgfqpoint{3.927220in}{0.696000in}}%
\pgfpathcurveto{\pgfqpoint{3.927220in}{0.689428in}}{\pgfqpoint{3.929831in}{0.683125in}}{\pgfqpoint{3.934478in}{0.678478in}}%
\pgfpathcurveto{\pgfqpoint{3.939125in}{0.673831in}}{\pgfqpoint{3.945428in}{0.671220in}}{\pgfqpoint{3.952000in}{0.671220in}}%
\pgfpathclose%
\pgfusepath{stroke,fill}%
\end{pgfscope}%
\begin{pgfscope}%
\pgfpathrectangle{\pgfqpoint{1.432000in}{0.528000in}}{\pgfqpoint{3.696000in}{3.696000in}}%
\pgfusepath{clip}%
\pgfsetbuttcap%
\pgfsetroundjoin%
\definecolor{currentfill}{rgb}{0.000000,0.000000,0.000000}%
\pgfsetfillcolor{currentfill}%
\pgfsetlinewidth{1.003750pt}%
\definecolor{currentstroke}{rgb}{0.000000,0.000000,0.000000}%
\pgfsetstrokecolor{currentstroke}%
\pgfsetdash{}{0pt}%
\pgfpathmoveto{\pgfqpoint{3.952000in}{1.007220in}}%
\pgfpathcurveto{\pgfqpoint{3.958572in}{1.007220in}}{\pgfqpoint{3.964875in}{1.009831in}}{\pgfqpoint{3.969522in}{1.014478in}}%
\pgfpathcurveto{\pgfqpoint{3.974169in}{1.019125in}}{\pgfqpoint{3.976780in}{1.025428in}}{\pgfqpoint{3.976780in}{1.032000in}}%
\pgfpathcurveto{\pgfqpoint{3.976780in}{1.038572in}}{\pgfqpoint{3.974169in}{1.044875in}}{\pgfqpoint{3.969522in}{1.049522in}}%
\pgfpathcurveto{\pgfqpoint{3.964875in}{1.054169in}}{\pgfqpoint{3.958572in}{1.056780in}}{\pgfqpoint{3.952000in}{1.056780in}}%
\pgfpathcurveto{\pgfqpoint{3.945428in}{1.056780in}}{\pgfqpoint{3.939125in}{1.054169in}}{\pgfqpoint{3.934478in}{1.049522in}}%
\pgfpathcurveto{\pgfqpoint{3.929831in}{1.044875in}}{\pgfqpoint{3.927220in}{1.038572in}}{\pgfqpoint{3.927220in}{1.032000in}}%
\pgfpathcurveto{\pgfqpoint{3.927220in}{1.025428in}}{\pgfqpoint{3.929831in}{1.019125in}}{\pgfqpoint{3.934478in}{1.014478in}}%
\pgfpathcurveto{\pgfqpoint{3.939125in}{1.009831in}}{\pgfqpoint{3.945428in}{1.007220in}}{\pgfqpoint{3.952000in}{1.007220in}}%
\pgfpathclose%
\pgfusepath{stroke,fill}%
\end{pgfscope}%
\begin{pgfscope}%
\pgfpathrectangle{\pgfqpoint{1.432000in}{0.528000in}}{\pgfqpoint{3.696000in}{3.696000in}}%
\pgfusepath{clip}%
\pgfsetbuttcap%
\pgfsetroundjoin%
\definecolor{currentfill}{rgb}{0.000000,0.000000,0.000000}%
\pgfsetfillcolor{currentfill}%
\pgfsetlinewidth{1.003750pt}%
\definecolor{currentstroke}{rgb}{0.000000,0.000000,0.000000}%
\pgfsetstrokecolor{currentstroke}%
\pgfsetdash{}{0pt}%
\pgfpathmoveto{\pgfqpoint{3.952000in}{1.343220in}}%
\pgfpathcurveto{\pgfqpoint{3.958572in}{1.343220in}}{\pgfqpoint{3.964875in}{1.345831in}}{\pgfqpoint{3.969522in}{1.350478in}}%
\pgfpathcurveto{\pgfqpoint{3.974169in}{1.355125in}}{\pgfqpoint{3.976780in}{1.361428in}}{\pgfqpoint{3.976780in}{1.368000in}}%
\pgfpathcurveto{\pgfqpoint{3.976780in}{1.374572in}}{\pgfqpoint{3.974169in}{1.380875in}}{\pgfqpoint{3.969522in}{1.385522in}}%
\pgfpathcurveto{\pgfqpoint{3.964875in}{1.390169in}}{\pgfqpoint{3.958572in}{1.392780in}}{\pgfqpoint{3.952000in}{1.392780in}}%
\pgfpathcurveto{\pgfqpoint{3.945428in}{1.392780in}}{\pgfqpoint{3.939125in}{1.390169in}}{\pgfqpoint{3.934478in}{1.385522in}}%
\pgfpathcurveto{\pgfqpoint{3.929831in}{1.380875in}}{\pgfqpoint{3.927220in}{1.374572in}}{\pgfqpoint{3.927220in}{1.368000in}}%
\pgfpathcurveto{\pgfqpoint{3.927220in}{1.361428in}}{\pgfqpoint{3.929831in}{1.355125in}}{\pgfqpoint{3.934478in}{1.350478in}}%
\pgfpathcurveto{\pgfqpoint{3.939125in}{1.345831in}}{\pgfqpoint{3.945428in}{1.343220in}}{\pgfqpoint{3.952000in}{1.343220in}}%
\pgfpathclose%
\pgfusepath{stroke,fill}%
\end{pgfscope}%
\begin{pgfscope}%
\pgfpathrectangle{\pgfqpoint{1.432000in}{0.528000in}}{\pgfqpoint{3.696000in}{3.696000in}}%
\pgfusepath{clip}%
\pgfsetbuttcap%
\pgfsetroundjoin%
\definecolor{currentfill}{rgb}{0.000000,0.000000,0.000000}%
\pgfsetfillcolor{currentfill}%
\pgfsetlinewidth{1.003750pt}%
\definecolor{currentstroke}{rgb}{0.000000,0.000000,0.000000}%
\pgfsetstrokecolor{currentstroke}%
\pgfsetdash{}{0pt}%
\pgfpathmoveto{\pgfqpoint{3.952000in}{1.679220in}}%
\pgfpathcurveto{\pgfqpoint{3.958572in}{1.679220in}}{\pgfqpoint{3.964875in}{1.681831in}}{\pgfqpoint{3.969522in}{1.686478in}}%
\pgfpathcurveto{\pgfqpoint{3.974169in}{1.691125in}}{\pgfqpoint{3.976780in}{1.697428in}}{\pgfqpoint{3.976780in}{1.704000in}}%
\pgfpathcurveto{\pgfqpoint{3.976780in}{1.710572in}}{\pgfqpoint{3.974169in}{1.716875in}}{\pgfqpoint{3.969522in}{1.721522in}}%
\pgfpathcurveto{\pgfqpoint{3.964875in}{1.726169in}}{\pgfqpoint{3.958572in}{1.728780in}}{\pgfqpoint{3.952000in}{1.728780in}}%
\pgfpathcurveto{\pgfqpoint{3.945428in}{1.728780in}}{\pgfqpoint{3.939125in}{1.726169in}}{\pgfqpoint{3.934478in}{1.721522in}}%
\pgfpathcurveto{\pgfqpoint{3.929831in}{1.716875in}}{\pgfqpoint{3.927220in}{1.710572in}}{\pgfqpoint{3.927220in}{1.704000in}}%
\pgfpathcurveto{\pgfqpoint{3.927220in}{1.697428in}}{\pgfqpoint{3.929831in}{1.691125in}}{\pgfqpoint{3.934478in}{1.686478in}}%
\pgfpathcurveto{\pgfqpoint{3.939125in}{1.681831in}}{\pgfqpoint{3.945428in}{1.679220in}}{\pgfqpoint{3.952000in}{1.679220in}}%
\pgfpathclose%
\pgfusepath{stroke,fill}%
\end{pgfscope}%
\begin{pgfscope}%
\pgfpathrectangle{\pgfqpoint{1.432000in}{0.528000in}}{\pgfqpoint{3.696000in}{3.696000in}}%
\pgfusepath{clip}%
\pgfsetbuttcap%
\pgfsetroundjoin%
\definecolor{currentfill}{rgb}{0.000000,0.000000,0.000000}%
\pgfsetfillcolor{currentfill}%
\pgfsetlinewidth{1.003750pt}%
\definecolor{currentstroke}{rgb}{0.000000,0.000000,0.000000}%
\pgfsetstrokecolor{currentstroke}%
\pgfsetdash{}{0pt}%
\pgfpathmoveto{\pgfqpoint{3.952000in}{2.015220in}}%
\pgfpathcurveto{\pgfqpoint{3.958572in}{2.015220in}}{\pgfqpoint{3.964875in}{2.017831in}}{\pgfqpoint{3.969522in}{2.022478in}}%
\pgfpathcurveto{\pgfqpoint{3.974169in}{2.027125in}}{\pgfqpoint{3.976780in}{2.033428in}}{\pgfqpoint{3.976780in}{2.040000in}}%
\pgfpathcurveto{\pgfqpoint{3.976780in}{2.046572in}}{\pgfqpoint{3.974169in}{2.052875in}}{\pgfqpoint{3.969522in}{2.057522in}}%
\pgfpathcurveto{\pgfqpoint{3.964875in}{2.062169in}}{\pgfqpoint{3.958572in}{2.064780in}}{\pgfqpoint{3.952000in}{2.064780in}}%
\pgfpathcurveto{\pgfqpoint{3.945428in}{2.064780in}}{\pgfqpoint{3.939125in}{2.062169in}}{\pgfqpoint{3.934478in}{2.057522in}}%
\pgfpathcurveto{\pgfqpoint{3.929831in}{2.052875in}}{\pgfqpoint{3.927220in}{2.046572in}}{\pgfqpoint{3.927220in}{2.040000in}}%
\pgfpathcurveto{\pgfqpoint{3.927220in}{2.033428in}}{\pgfqpoint{3.929831in}{2.027125in}}{\pgfqpoint{3.934478in}{2.022478in}}%
\pgfpathcurveto{\pgfqpoint{3.939125in}{2.017831in}}{\pgfqpoint{3.945428in}{2.015220in}}{\pgfqpoint{3.952000in}{2.015220in}}%
\pgfpathclose%
\pgfusepath{stroke,fill}%
\end{pgfscope}%
\begin{pgfscope}%
\pgfpathrectangle{\pgfqpoint{1.432000in}{0.528000in}}{\pgfqpoint{3.696000in}{3.696000in}}%
\pgfusepath{clip}%
\pgfsetbuttcap%
\pgfsetroundjoin%
\definecolor{currentfill}{rgb}{0.000000,0.000000,0.000000}%
\pgfsetfillcolor{currentfill}%
\pgfsetlinewidth{1.003750pt}%
\definecolor{currentstroke}{rgb}{0.000000,0.000000,0.000000}%
\pgfsetstrokecolor{currentstroke}%
\pgfsetdash{}{0pt}%
\pgfpathmoveto{\pgfqpoint{3.952000in}{2.351220in}}%
\pgfpathcurveto{\pgfqpoint{3.958572in}{2.351220in}}{\pgfqpoint{3.964875in}{2.353831in}}{\pgfqpoint{3.969522in}{2.358478in}}%
\pgfpathcurveto{\pgfqpoint{3.974169in}{2.363125in}}{\pgfqpoint{3.976780in}{2.369428in}}{\pgfqpoint{3.976780in}{2.376000in}}%
\pgfpathcurveto{\pgfqpoint{3.976780in}{2.382572in}}{\pgfqpoint{3.974169in}{2.388875in}}{\pgfqpoint{3.969522in}{2.393522in}}%
\pgfpathcurveto{\pgfqpoint{3.964875in}{2.398169in}}{\pgfqpoint{3.958572in}{2.400780in}}{\pgfqpoint{3.952000in}{2.400780in}}%
\pgfpathcurveto{\pgfqpoint{3.945428in}{2.400780in}}{\pgfqpoint{3.939125in}{2.398169in}}{\pgfqpoint{3.934478in}{2.393522in}}%
\pgfpathcurveto{\pgfqpoint{3.929831in}{2.388875in}}{\pgfqpoint{3.927220in}{2.382572in}}{\pgfqpoint{3.927220in}{2.376000in}}%
\pgfpathcurveto{\pgfqpoint{3.927220in}{2.369428in}}{\pgfqpoint{3.929831in}{2.363125in}}{\pgfqpoint{3.934478in}{2.358478in}}%
\pgfpathcurveto{\pgfqpoint{3.939125in}{2.353831in}}{\pgfqpoint{3.945428in}{2.351220in}}{\pgfqpoint{3.952000in}{2.351220in}}%
\pgfpathclose%
\pgfusepath{stroke,fill}%
\end{pgfscope}%
\begin{pgfscope}%
\pgfpathrectangle{\pgfqpoint{1.432000in}{0.528000in}}{\pgfqpoint{3.696000in}{3.696000in}}%
\pgfusepath{clip}%
\pgfsetbuttcap%
\pgfsetroundjoin%
\definecolor{currentfill}{rgb}{0.000000,0.000000,0.000000}%
\pgfsetfillcolor{currentfill}%
\pgfsetlinewidth{1.003750pt}%
\definecolor{currentstroke}{rgb}{0.000000,0.000000,0.000000}%
\pgfsetstrokecolor{currentstroke}%
\pgfsetdash{}{0pt}%
\pgfpathmoveto{\pgfqpoint{3.952000in}{2.687220in}}%
\pgfpathcurveto{\pgfqpoint{3.958572in}{2.687220in}}{\pgfqpoint{3.964875in}{2.689831in}}{\pgfqpoint{3.969522in}{2.694478in}}%
\pgfpathcurveto{\pgfqpoint{3.974169in}{2.699125in}}{\pgfqpoint{3.976780in}{2.705428in}}{\pgfqpoint{3.976780in}{2.712000in}}%
\pgfpathcurveto{\pgfqpoint{3.976780in}{2.718572in}}{\pgfqpoint{3.974169in}{2.724875in}}{\pgfqpoint{3.969522in}{2.729522in}}%
\pgfpathcurveto{\pgfqpoint{3.964875in}{2.734169in}}{\pgfqpoint{3.958572in}{2.736780in}}{\pgfqpoint{3.952000in}{2.736780in}}%
\pgfpathcurveto{\pgfqpoint{3.945428in}{2.736780in}}{\pgfqpoint{3.939125in}{2.734169in}}{\pgfqpoint{3.934478in}{2.729522in}}%
\pgfpathcurveto{\pgfqpoint{3.929831in}{2.724875in}}{\pgfqpoint{3.927220in}{2.718572in}}{\pgfqpoint{3.927220in}{2.712000in}}%
\pgfpathcurveto{\pgfqpoint{3.927220in}{2.705428in}}{\pgfqpoint{3.929831in}{2.699125in}}{\pgfqpoint{3.934478in}{2.694478in}}%
\pgfpathcurveto{\pgfqpoint{3.939125in}{2.689831in}}{\pgfqpoint{3.945428in}{2.687220in}}{\pgfqpoint{3.952000in}{2.687220in}}%
\pgfpathclose%
\pgfusepath{stroke,fill}%
\end{pgfscope}%
\begin{pgfscope}%
\pgfpathrectangle{\pgfqpoint{1.432000in}{0.528000in}}{\pgfqpoint{3.696000in}{3.696000in}}%
\pgfusepath{clip}%
\pgfsetbuttcap%
\pgfsetroundjoin%
\definecolor{currentfill}{rgb}{0.000000,0.000000,0.000000}%
\pgfsetfillcolor{currentfill}%
\pgfsetlinewidth{1.003750pt}%
\definecolor{currentstroke}{rgb}{0.000000,0.000000,0.000000}%
\pgfsetstrokecolor{currentstroke}%
\pgfsetdash{}{0pt}%
\pgfpathmoveto{\pgfqpoint{3.952000in}{3.023220in}}%
\pgfpathcurveto{\pgfqpoint{3.958572in}{3.023220in}}{\pgfqpoint{3.964875in}{3.025831in}}{\pgfqpoint{3.969522in}{3.030478in}}%
\pgfpathcurveto{\pgfqpoint{3.974169in}{3.035125in}}{\pgfqpoint{3.976780in}{3.041428in}}{\pgfqpoint{3.976780in}{3.048000in}}%
\pgfpathcurveto{\pgfqpoint{3.976780in}{3.054572in}}{\pgfqpoint{3.974169in}{3.060875in}}{\pgfqpoint{3.969522in}{3.065522in}}%
\pgfpathcurveto{\pgfqpoint{3.964875in}{3.070169in}}{\pgfqpoint{3.958572in}{3.072780in}}{\pgfqpoint{3.952000in}{3.072780in}}%
\pgfpathcurveto{\pgfqpoint{3.945428in}{3.072780in}}{\pgfqpoint{3.939125in}{3.070169in}}{\pgfqpoint{3.934478in}{3.065522in}}%
\pgfpathcurveto{\pgfqpoint{3.929831in}{3.060875in}}{\pgfqpoint{3.927220in}{3.054572in}}{\pgfqpoint{3.927220in}{3.048000in}}%
\pgfpathcurveto{\pgfqpoint{3.927220in}{3.041428in}}{\pgfqpoint{3.929831in}{3.035125in}}{\pgfqpoint{3.934478in}{3.030478in}}%
\pgfpathcurveto{\pgfqpoint{3.939125in}{3.025831in}}{\pgfqpoint{3.945428in}{3.023220in}}{\pgfqpoint{3.952000in}{3.023220in}}%
\pgfpathclose%
\pgfusepath{stroke,fill}%
\end{pgfscope}%
\begin{pgfscope}%
\pgfpathrectangle{\pgfqpoint{1.432000in}{0.528000in}}{\pgfqpoint{3.696000in}{3.696000in}}%
\pgfusepath{clip}%
\pgfsetbuttcap%
\pgfsetroundjoin%
\definecolor{currentfill}{rgb}{0.000000,0.000000,0.000000}%
\pgfsetfillcolor{currentfill}%
\pgfsetlinewidth{1.003750pt}%
\definecolor{currentstroke}{rgb}{0.000000,0.000000,0.000000}%
\pgfsetstrokecolor{currentstroke}%
\pgfsetdash{}{0pt}%
\pgfpathmoveto{\pgfqpoint{3.952000in}{3.359220in}}%
\pgfpathcurveto{\pgfqpoint{3.958572in}{3.359220in}}{\pgfqpoint{3.964875in}{3.361831in}}{\pgfqpoint{3.969522in}{3.366478in}}%
\pgfpathcurveto{\pgfqpoint{3.974169in}{3.371125in}}{\pgfqpoint{3.976780in}{3.377428in}}{\pgfqpoint{3.976780in}{3.384000in}}%
\pgfpathcurveto{\pgfqpoint{3.976780in}{3.390572in}}{\pgfqpoint{3.974169in}{3.396875in}}{\pgfqpoint{3.969522in}{3.401522in}}%
\pgfpathcurveto{\pgfqpoint{3.964875in}{3.406169in}}{\pgfqpoint{3.958572in}{3.408780in}}{\pgfqpoint{3.952000in}{3.408780in}}%
\pgfpathcurveto{\pgfqpoint{3.945428in}{3.408780in}}{\pgfqpoint{3.939125in}{3.406169in}}{\pgfqpoint{3.934478in}{3.401522in}}%
\pgfpathcurveto{\pgfqpoint{3.929831in}{3.396875in}}{\pgfqpoint{3.927220in}{3.390572in}}{\pgfqpoint{3.927220in}{3.384000in}}%
\pgfpathcurveto{\pgfqpoint{3.927220in}{3.377428in}}{\pgfqpoint{3.929831in}{3.371125in}}{\pgfqpoint{3.934478in}{3.366478in}}%
\pgfpathcurveto{\pgfqpoint{3.939125in}{3.361831in}}{\pgfqpoint{3.945428in}{3.359220in}}{\pgfqpoint{3.952000in}{3.359220in}}%
\pgfpathclose%
\pgfusepath{stroke,fill}%
\end{pgfscope}%
\begin{pgfscope}%
\pgfpathrectangle{\pgfqpoint{1.432000in}{0.528000in}}{\pgfqpoint{3.696000in}{3.696000in}}%
\pgfusepath{clip}%
\pgfsetbuttcap%
\pgfsetroundjoin%
\definecolor{currentfill}{rgb}{0.000000,0.000000,0.000000}%
\pgfsetfillcolor{currentfill}%
\pgfsetlinewidth{1.003750pt}%
\definecolor{currentstroke}{rgb}{0.000000,0.000000,0.000000}%
\pgfsetstrokecolor{currentstroke}%
\pgfsetdash{}{0pt}%
\pgfpathmoveto{\pgfqpoint{3.952000in}{3.695220in}}%
\pgfpathcurveto{\pgfqpoint{3.958572in}{3.695220in}}{\pgfqpoint{3.964875in}{3.697831in}}{\pgfqpoint{3.969522in}{3.702478in}}%
\pgfpathcurveto{\pgfqpoint{3.974169in}{3.707125in}}{\pgfqpoint{3.976780in}{3.713428in}}{\pgfqpoint{3.976780in}{3.720000in}}%
\pgfpathcurveto{\pgfqpoint{3.976780in}{3.726572in}}{\pgfqpoint{3.974169in}{3.732875in}}{\pgfqpoint{3.969522in}{3.737522in}}%
\pgfpathcurveto{\pgfqpoint{3.964875in}{3.742169in}}{\pgfqpoint{3.958572in}{3.744780in}}{\pgfqpoint{3.952000in}{3.744780in}}%
\pgfpathcurveto{\pgfqpoint{3.945428in}{3.744780in}}{\pgfqpoint{3.939125in}{3.742169in}}{\pgfqpoint{3.934478in}{3.737522in}}%
\pgfpathcurveto{\pgfqpoint{3.929831in}{3.732875in}}{\pgfqpoint{3.927220in}{3.726572in}}{\pgfqpoint{3.927220in}{3.720000in}}%
\pgfpathcurveto{\pgfqpoint{3.927220in}{3.713428in}}{\pgfqpoint{3.929831in}{3.707125in}}{\pgfqpoint{3.934478in}{3.702478in}}%
\pgfpathcurveto{\pgfqpoint{3.939125in}{3.697831in}}{\pgfqpoint{3.945428in}{3.695220in}}{\pgfqpoint{3.952000in}{3.695220in}}%
\pgfpathclose%
\pgfusepath{stroke,fill}%
\end{pgfscope}%
\begin{pgfscope}%
\pgfpathrectangle{\pgfqpoint{1.432000in}{0.528000in}}{\pgfqpoint{3.696000in}{3.696000in}}%
\pgfusepath{clip}%
\pgfsetbuttcap%
\pgfsetroundjoin%
\definecolor{currentfill}{rgb}{0.000000,0.000000,0.000000}%
\pgfsetfillcolor{currentfill}%
\pgfsetlinewidth{1.003750pt}%
\definecolor{currentstroke}{rgb}{0.000000,0.000000,0.000000}%
\pgfsetstrokecolor{currentstroke}%
\pgfsetdash{}{0pt}%
\pgfpathmoveto{\pgfqpoint{3.952000in}{4.031220in}}%
\pgfpathcurveto{\pgfqpoint{3.958572in}{4.031220in}}{\pgfqpoint{3.964875in}{4.033831in}}{\pgfqpoint{3.969522in}{4.038478in}}%
\pgfpathcurveto{\pgfqpoint{3.974169in}{4.043125in}}{\pgfqpoint{3.976780in}{4.049428in}}{\pgfqpoint{3.976780in}{4.056000in}}%
\pgfpathcurveto{\pgfqpoint{3.976780in}{4.062572in}}{\pgfqpoint{3.974169in}{4.068875in}}{\pgfqpoint{3.969522in}{4.073522in}}%
\pgfpathcurveto{\pgfqpoint{3.964875in}{4.078169in}}{\pgfqpoint{3.958572in}{4.080780in}}{\pgfqpoint{3.952000in}{4.080780in}}%
\pgfpathcurveto{\pgfqpoint{3.945428in}{4.080780in}}{\pgfqpoint{3.939125in}{4.078169in}}{\pgfqpoint{3.934478in}{4.073522in}}%
\pgfpathcurveto{\pgfqpoint{3.929831in}{4.068875in}}{\pgfqpoint{3.927220in}{4.062572in}}{\pgfqpoint{3.927220in}{4.056000in}}%
\pgfpathcurveto{\pgfqpoint{3.927220in}{4.049428in}}{\pgfqpoint{3.929831in}{4.043125in}}{\pgfqpoint{3.934478in}{4.038478in}}%
\pgfpathcurveto{\pgfqpoint{3.939125in}{4.033831in}}{\pgfqpoint{3.945428in}{4.031220in}}{\pgfqpoint{3.952000in}{4.031220in}}%
\pgfpathclose%
\pgfusepath{stroke,fill}%
\end{pgfscope}%
\begin{pgfscope}%
\pgfpathrectangle{\pgfqpoint{1.432000in}{0.528000in}}{\pgfqpoint{3.696000in}{3.696000in}}%
\pgfusepath{clip}%
\pgfsetbuttcap%
\pgfsetroundjoin%
\definecolor{currentfill}{rgb}{0.000000,0.000000,0.000000}%
\pgfsetfillcolor{currentfill}%
\pgfsetlinewidth{1.003750pt}%
\definecolor{currentstroke}{rgb}{0.000000,0.000000,0.000000}%
\pgfsetstrokecolor{currentstroke}%
\pgfsetdash{}{0pt}%
\pgfpathmoveto{\pgfqpoint{4.288000in}{0.671220in}}%
\pgfpathcurveto{\pgfqpoint{4.294572in}{0.671220in}}{\pgfqpoint{4.300875in}{0.673831in}}{\pgfqpoint{4.305522in}{0.678478in}}%
\pgfpathcurveto{\pgfqpoint{4.310169in}{0.683125in}}{\pgfqpoint{4.312780in}{0.689428in}}{\pgfqpoint{4.312780in}{0.696000in}}%
\pgfpathcurveto{\pgfqpoint{4.312780in}{0.702572in}}{\pgfqpoint{4.310169in}{0.708875in}}{\pgfqpoint{4.305522in}{0.713522in}}%
\pgfpathcurveto{\pgfqpoint{4.300875in}{0.718169in}}{\pgfqpoint{4.294572in}{0.720780in}}{\pgfqpoint{4.288000in}{0.720780in}}%
\pgfpathcurveto{\pgfqpoint{4.281428in}{0.720780in}}{\pgfqpoint{4.275125in}{0.718169in}}{\pgfqpoint{4.270478in}{0.713522in}}%
\pgfpathcurveto{\pgfqpoint{4.265831in}{0.708875in}}{\pgfqpoint{4.263220in}{0.702572in}}{\pgfqpoint{4.263220in}{0.696000in}}%
\pgfpathcurveto{\pgfqpoint{4.263220in}{0.689428in}}{\pgfqpoint{4.265831in}{0.683125in}}{\pgfqpoint{4.270478in}{0.678478in}}%
\pgfpathcurveto{\pgfqpoint{4.275125in}{0.673831in}}{\pgfqpoint{4.281428in}{0.671220in}}{\pgfqpoint{4.288000in}{0.671220in}}%
\pgfpathclose%
\pgfusepath{stroke,fill}%
\end{pgfscope}%
\begin{pgfscope}%
\pgfpathrectangle{\pgfqpoint{1.432000in}{0.528000in}}{\pgfqpoint{3.696000in}{3.696000in}}%
\pgfusepath{clip}%
\pgfsetbuttcap%
\pgfsetroundjoin%
\definecolor{currentfill}{rgb}{0.000000,0.000000,0.000000}%
\pgfsetfillcolor{currentfill}%
\pgfsetlinewidth{1.003750pt}%
\definecolor{currentstroke}{rgb}{0.000000,0.000000,0.000000}%
\pgfsetstrokecolor{currentstroke}%
\pgfsetdash{}{0pt}%
\pgfpathmoveto{\pgfqpoint{4.288000in}{1.007220in}}%
\pgfpathcurveto{\pgfqpoint{4.294572in}{1.007220in}}{\pgfqpoint{4.300875in}{1.009831in}}{\pgfqpoint{4.305522in}{1.014478in}}%
\pgfpathcurveto{\pgfqpoint{4.310169in}{1.019125in}}{\pgfqpoint{4.312780in}{1.025428in}}{\pgfqpoint{4.312780in}{1.032000in}}%
\pgfpathcurveto{\pgfqpoint{4.312780in}{1.038572in}}{\pgfqpoint{4.310169in}{1.044875in}}{\pgfqpoint{4.305522in}{1.049522in}}%
\pgfpathcurveto{\pgfqpoint{4.300875in}{1.054169in}}{\pgfqpoint{4.294572in}{1.056780in}}{\pgfqpoint{4.288000in}{1.056780in}}%
\pgfpathcurveto{\pgfqpoint{4.281428in}{1.056780in}}{\pgfqpoint{4.275125in}{1.054169in}}{\pgfqpoint{4.270478in}{1.049522in}}%
\pgfpathcurveto{\pgfqpoint{4.265831in}{1.044875in}}{\pgfqpoint{4.263220in}{1.038572in}}{\pgfqpoint{4.263220in}{1.032000in}}%
\pgfpathcurveto{\pgfqpoint{4.263220in}{1.025428in}}{\pgfqpoint{4.265831in}{1.019125in}}{\pgfqpoint{4.270478in}{1.014478in}}%
\pgfpathcurveto{\pgfqpoint{4.275125in}{1.009831in}}{\pgfqpoint{4.281428in}{1.007220in}}{\pgfqpoint{4.288000in}{1.007220in}}%
\pgfpathclose%
\pgfusepath{stroke,fill}%
\end{pgfscope}%
\begin{pgfscope}%
\pgfpathrectangle{\pgfqpoint{1.432000in}{0.528000in}}{\pgfqpoint{3.696000in}{3.696000in}}%
\pgfusepath{clip}%
\pgfsetbuttcap%
\pgfsetroundjoin%
\definecolor{currentfill}{rgb}{0.000000,0.000000,0.000000}%
\pgfsetfillcolor{currentfill}%
\pgfsetlinewidth{1.003750pt}%
\definecolor{currentstroke}{rgb}{0.000000,0.000000,0.000000}%
\pgfsetstrokecolor{currentstroke}%
\pgfsetdash{}{0pt}%
\pgfpathmoveto{\pgfqpoint{4.288000in}{1.343220in}}%
\pgfpathcurveto{\pgfqpoint{4.294572in}{1.343220in}}{\pgfqpoint{4.300875in}{1.345831in}}{\pgfqpoint{4.305522in}{1.350478in}}%
\pgfpathcurveto{\pgfqpoint{4.310169in}{1.355125in}}{\pgfqpoint{4.312780in}{1.361428in}}{\pgfqpoint{4.312780in}{1.368000in}}%
\pgfpathcurveto{\pgfqpoint{4.312780in}{1.374572in}}{\pgfqpoint{4.310169in}{1.380875in}}{\pgfqpoint{4.305522in}{1.385522in}}%
\pgfpathcurveto{\pgfqpoint{4.300875in}{1.390169in}}{\pgfqpoint{4.294572in}{1.392780in}}{\pgfqpoint{4.288000in}{1.392780in}}%
\pgfpathcurveto{\pgfqpoint{4.281428in}{1.392780in}}{\pgfqpoint{4.275125in}{1.390169in}}{\pgfqpoint{4.270478in}{1.385522in}}%
\pgfpathcurveto{\pgfqpoint{4.265831in}{1.380875in}}{\pgfqpoint{4.263220in}{1.374572in}}{\pgfqpoint{4.263220in}{1.368000in}}%
\pgfpathcurveto{\pgfqpoint{4.263220in}{1.361428in}}{\pgfqpoint{4.265831in}{1.355125in}}{\pgfqpoint{4.270478in}{1.350478in}}%
\pgfpathcurveto{\pgfqpoint{4.275125in}{1.345831in}}{\pgfqpoint{4.281428in}{1.343220in}}{\pgfqpoint{4.288000in}{1.343220in}}%
\pgfpathclose%
\pgfusepath{stroke,fill}%
\end{pgfscope}%
\begin{pgfscope}%
\pgfpathrectangle{\pgfqpoint{1.432000in}{0.528000in}}{\pgfqpoint{3.696000in}{3.696000in}}%
\pgfusepath{clip}%
\pgfsetbuttcap%
\pgfsetroundjoin%
\definecolor{currentfill}{rgb}{0.000000,0.000000,0.000000}%
\pgfsetfillcolor{currentfill}%
\pgfsetlinewidth{1.003750pt}%
\definecolor{currentstroke}{rgb}{0.000000,0.000000,0.000000}%
\pgfsetstrokecolor{currentstroke}%
\pgfsetdash{}{0pt}%
\pgfpathmoveto{\pgfqpoint{4.288000in}{1.679220in}}%
\pgfpathcurveto{\pgfqpoint{4.294572in}{1.679220in}}{\pgfqpoint{4.300875in}{1.681831in}}{\pgfqpoint{4.305522in}{1.686478in}}%
\pgfpathcurveto{\pgfqpoint{4.310169in}{1.691125in}}{\pgfqpoint{4.312780in}{1.697428in}}{\pgfqpoint{4.312780in}{1.704000in}}%
\pgfpathcurveto{\pgfqpoint{4.312780in}{1.710572in}}{\pgfqpoint{4.310169in}{1.716875in}}{\pgfqpoint{4.305522in}{1.721522in}}%
\pgfpathcurveto{\pgfqpoint{4.300875in}{1.726169in}}{\pgfqpoint{4.294572in}{1.728780in}}{\pgfqpoint{4.288000in}{1.728780in}}%
\pgfpathcurveto{\pgfqpoint{4.281428in}{1.728780in}}{\pgfqpoint{4.275125in}{1.726169in}}{\pgfqpoint{4.270478in}{1.721522in}}%
\pgfpathcurveto{\pgfqpoint{4.265831in}{1.716875in}}{\pgfqpoint{4.263220in}{1.710572in}}{\pgfqpoint{4.263220in}{1.704000in}}%
\pgfpathcurveto{\pgfqpoint{4.263220in}{1.697428in}}{\pgfqpoint{4.265831in}{1.691125in}}{\pgfqpoint{4.270478in}{1.686478in}}%
\pgfpathcurveto{\pgfqpoint{4.275125in}{1.681831in}}{\pgfqpoint{4.281428in}{1.679220in}}{\pgfqpoint{4.288000in}{1.679220in}}%
\pgfpathclose%
\pgfusepath{stroke,fill}%
\end{pgfscope}%
\begin{pgfscope}%
\pgfpathrectangle{\pgfqpoint{1.432000in}{0.528000in}}{\pgfqpoint{3.696000in}{3.696000in}}%
\pgfusepath{clip}%
\pgfsetbuttcap%
\pgfsetroundjoin%
\definecolor{currentfill}{rgb}{0.000000,0.000000,0.000000}%
\pgfsetfillcolor{currentfill}%
\pgfsetlinewidth{1.003750pt}%
\definecolor{currentstroke}{rgb}{0.000000,0.000000,0.000000}%
\pgfsetstrokecolor{currentstroke}%
\pgfsetdash{}{0pt}%
\pgfpathmoveto{\pgfqpoint{4.288000in}{2.015220in}}%
\pgfpathcurveto{\pgfqpoint{4.294572in}{2.015220in}}{\pgfqpoint{4.300875in}{2.017831in}}{\pgfqpoint{4.305522in}{2.022478in}}%
\pgfpathcurveto{\pgfqpoint{4.310169in}{2.027125in}}{\pgfqpoint{4.312780in}{2.033428in}}{\pgfqpoint{4.312780in}{2.040000in}}%
\pgfpathcurveto{\pgfqpoint{4.312780in}{2.046572in}}{\pgfqpoint{4.310169in}{2.052875in}}{\pgfqpoint{4.305522in}{2.057522in}}%
\pgfpathcurveto{\pgfqpoint{4.300875in}{2.062169in}}{\pgfqpoint{4.294572in}{2.064780in}}{\pgfqpoint{4.288000in}{2.064780in}}%
\pgfpathcurveto{\pgfqpoint{4.281428in}{2.064780in}}{\pgfqpoint{4.275125in}{2.062169in}}{\pgfqpoint{4.270478in}{2.057522in}}%
\pgfpathcurveto{\pgfqpoint{4.265831in}{2.052875in}}{\pgfqpoint{4.263220in}{2.046572in}}{\pgfqpoint{4.263220in}{2.040000in}}%
\pgfpathcurveto{\pgfqpoint{4.263220in}{2.033428in}}{\pgfqpoint{4.265831in}{2.027125in}}{\pgfqpoint{4.270478in}{2.022478in}}%
\pgfpathcurveto{\pgfqpoint{4.275125in}{2.017831in}}{\pgfqpoint{4.281428in}{2.015220in}}{\pgfqpoint{4.288000in}{2.015220in}}%
\pgfpathclose%
\pgfusepath{stroke,fill}%
\end{pgfscope}%
\begin{pgfscope}%
\pgfpathrectangle{\pgfqpoint{1.432000in}{0.528000in}}{\pgfqpoint{3.696000in}{3.696000in}}%
\pgfusepath{clip}%
\pgfsetbuttcap%
\pgfsetroundjoin%
\definecolor{currentfill}{rgb}{0.000000,0.000000,0.000000}%
\pgfsetfillcolor{currentfill}%
\pgfsetlinewidth{1.003750pt}%
\definecolor{currentstroke}{rgb}{0.000000,0.000000,0.000000}%
\pgfsetstrokecolor{currentstroke}%
\pgfsetdash{}{0pt}%
\pgfpathmoveto{\pgfqpoint{4.288000in}{2.351220in}}%
\pgfpathcurveto{\pgfqpoint{4.294572in}{2.351220in}}{\pgfqpoint{4.300875in}{2.353831in}}{\pgfqpoint{4.305522in}{2.358478in}}%
\pgfpathcurveto{\pgfqpoint{4.310169in}{2.363125in}}{\pgfqpoint{4.312780in}{2.369428in}}{\pgfqpoint{4.312780in}{2.376000in}}%
\pgfpathcurveto{\pgfqpoint{4.312780in}{2.382572in}}{\pgfqpoint{4.310169in}{2.388875in}}{\pgfqpoint{4.305522in}{2.393522in}}%
\pgfpathcurveto{\pgfqpoint{4.300875in}{2.398169in}}{\pgfqpoint{4.294572in}{2.400780in}}{\pgfqpoint{4.288000in}{2.400780in}}%
\pgfpathcurveto{\pgfqpoint{4.281428in}{2.400780in}}{\pgfqpoint{4.275125in}{2.398169in}}{\pgfqpoint{4.270478in}{2.393522in}}%
\pgfpathcurveto{\pgfqpoint{4.265831in}{2.388875in}}{\pgfqpoint{4.263220in}{2.382572in}}{\pgfqpoint{4.263220in}{2.376000in}}%
\pgfpathcurveto{\pgfqpoint{4.263220in}{2.369428in}}{\pgfqpoint{4.265831in}{2.363125in}}{\pgfqpoint{4.270478in}{2.358478in}}%
\pgfpathcurveto{\pgfqpoint{4.275125in}{2.353831in}}{\pgfqpoint{4.281428in}{2.351220in}}{\pgfqpoint{4.288000in}{2.351220in}}%
\pgfpathclose%
\pgfusepath{stroke,fill}%
\end{pgfscope}%
\begin{pgfscope}%
\pgfpathrectangle{\pgfqpoint{1.432000in}{0.528000in}}{\pgfqpoint{3.696000in}{3.696000in}}%
\pgfusepath{clip}%
\pgfsetbuttcap%
\pgfsetroundjoin%
\definecolor{currentfill}{rgb}{0.000000,0.000000,0.000000}%
\pgfsetfillcolor{currentfill}%
\pgfsetlinewidth{1.003750pt}%
\definecolor{currentstroke}{rgb}{0.000000,0.000000,0.000000}%
\pgfsetstrokecolor{currentstroke}%
\pgfsetdash{}{0pt}%
\pgfpathmoveto{\pgfqpoint{4.288000in}{2.687220in}}%
\pgfpathcurveto{\pgfqpoint{4.294572in}{2.687220in}}{\pgfqpoint{4.300875in}{2.689831in}}{\pgfqpoint{4.305522in}{2.694478in}}%
\pgfpathcurveto{\pgfqpoint{4.310169in}{2.699125in}}{\pgfqpoint{4.312780in}{2.705428in}}{\pgfqpoint{4.312780in}{2.712000in}}%
\pgfpathcurveto{\pgfqpoint{4.312780in}{2.718572in}}{\pgfqpoint{4.310169in}{2.724875in}}{\pgfqpoint{4.305522in}{2.729522in}}%
\pgfpathcurveto{\pgfqpoint{4.300875in}{2.734169in}}{\pgfqpoint{4.294572in}{2.736780in}}{\pgfqpoint{4.288000in}{2.736780in}}%
\pgfpathcurveto{\pgfqpoint{4.281428in}{2.736780in}}{\pgfqpoint{4.275125in}{2.734169in}}{\pgfqpoint{4.270478in}{2.729522in}}%
\pgfpathcurveto{\pgfqpoint{4.265831in}{2.724875in}}{\pgfqpoint{4.263220in}{2.718572in}}{\pgfqpoint{4.263220in}{2.712000in}}%
\pgfpathcurveto{\pgfqpoint{4.263220in}{2.705428in}}{\pgfqpoint{4.265831in}{2.699125in}}{\pgfqpoint{4.270478in}{2.694478in}}%
\pgfpathcurveto{\pgfqpoint{4.275125in}{2.689831in}}{\pgfqpoint{4.281428in}{2.687220in}}{\pgfqpoint{4.288000in}{2.687220in}}%
\pgfpathclose%
\pgfusepath{stroke,fill}%
\end{pgfscope}%
\begin{pgfscope}%
\pgfpathrectangle{\pgfqpoint{1.432000in}{0.528000in}}{\pgfqpoint{3.696000in}{3.696000in}}%
\pgfusepath{clip}%
\pgfsetbuttcap%
\pgfsetroundjoin%
\definecolor{currentfill}{rgb}{0.000000,0.000000,0.000000}%
\pgfsetfillcolor{currentfill}%
\pgfsetlinewidth{1.003750pt}%
\definecolor{currentstroke}{rgb}{0.000000,0.000000,0.000000}%
\pgfsetstrokecolor{currentstroke}%
\pgfsetdash{}{0pt}%
\pgfpathmoveto{\pgfqpoint{4.288000in}{3.023220in}}%
\pgfpathcurveto{\pgfqpoint{4.294572in}{3.023220in}}{\pgfqpoint{4.300875in}{3.025831in}}{\pgfqpoint{4.305522in}{3.030478in}}%
\pgfpathcurveto{\pgfqpoint{4.310169in}{3.035125in}}{\pgfqpoint{4.312780in}{3.041428in}}{\pgfqpoint{4.312780in}{3.048000in}}%
\pgfpathcurveto{\pgfqpoint{4.312780in}{3.054572in}}{\pgfqpoint{4.310169in}{3.060875in}}{\pgfqpoint{4.305522in}{3.065522in}}%
\pgfpathcurveto{\pgfqpoint{4.300875in}{3.070169in}}{\pgfqpoint{4.294572in}{3.072780in}}{\pgfqpoint{4.288000in}{3.072780in}}%
\pgfpathcurveto{\pgfqpoint{4.281428in}{3.072780in}}{\pgfqpoint{4.275125in}{3.070169in}}{\pgfqpoint{4.270478in}{3.065522in}}%
\pgfpathcurveto{\pgfqpoint{4.265831in}{3.060875in}}{\pgfqpoint{4.263220in}{3.054572in}}{\pgfqpoint{4.263220in}{3.048000in}}%
\pgfpathcurveto{\pgfqpoint{4.263220in}{3.041428in}}{\pgfqpoint{4.265831in}{3.035125in}}{\pgfqpoint{4.270478in}{3.030478in}}%
\pgfpathcurveto{\pgfqpoint{4.275125in}{3.025831in}}{\pgfqpoint{4.281428in}{3.023220in}}{\pgfqpoint{4.288000in}{3.023220in}}%
\pgfpathclose%
\pgfusepath{stroke,fill}%
\end{pgfscope}%
\begin{pgfscope}%
\pgfpathrectangle{\pgfqpoint{1.432000in}{0.528000in}}{\pgfqpoint{3.696000in}{3.696000in}}%
\pgfusepath{clip}%
\pgfsetbuttcap%
\pgfsetroundjoin%
\definecolor{currentfill}{rgb}{0.000000,0.000000,0.000000}%
\pgfsetfillcolor{currentfill}%
\pgfsetlinewidth{1.003750pt}%
\definecolor{currentstroke}{rgb}{0.000000,0.000000,0.000000}%
\pgfsetstrokecolor{currentstroke}%
\pgfsetdash{}{0pt}%
\pgfpathmoveto{\pgfqpoint{4.288000in}{3.359220in}}%
\pgfpathcurveto{\pgfqpoint{4.294572in}{3.359220in}}{\pgfqpoint{4.300875in}{3.361831in}}{\pgfqpoint{4.305522in}{3.366478in}}%
\pgfpathcurveto{\pgfqpoint{4.310169in}{3.371125in}}{\pgfqpoint{4.312780in}{3.377428in}}{\pgfqpoint{4.312780in}{3.384000in}}%
\pgfpathcurveto{\pgfqpoint{4.312780in}{3.390572in}}{\pgfqpoint{4.310169in}{3.396875in}}{\pgfqpoint{4.305522in}{3.401522in}}%
\pgfpathcurveto{\pgfqpoint{4.300875in}{3.406169in}}{\pgfqpoint{4.294572in}{3.408780in}}{\pgfqpoint{4.288000in}{3.408780in}}%
\pgfpathcurveto{\pgfqpoint{4.281428in}{3.408780in}}{\pgfqpoint{4.275125in}{3.406169in}}{\pgfqpoint{4.270478in}{3.401522in}}%
\pgfpathcurveto{\pgfqpoint{4.265831in}{3.396875in}}{\pgfqpoint{4.263220in}{3.390572in}}{\pgfqpoint{4.263220in}{3.384000in}}%
\pgfpathcurveto{\pgfqpoint{4.263220in}{3.377428in}}{\pgfqpoint{4.265831in}{3.371125in}}{\pgfqpoint{4.270478in}{3.366478in}}%
\pgfpathcurveto{\pgfqpoint{4.275125in}{3.361831in}}{\pgfqpoint{4.281428in}{3.359220in}}{\pgfqpoint{4.288000in}{3.359220in}}%
\pgfpathclose%
\pgfusepath{stroke,fill}%
\end{pgfscope}%
\begin{pgfscope}%
\pgfpathrectangle{\pgfqpoint{1.432000in}{0.528000in}}{\pgfqpoint{3.696000in}{3.696000in}}%
\pgfusepath{clip}%
\pgfsetbuttcap%
\pgfsetroundjoin%
\definecolor{currentfill}{rgb}{0.000000,0.000000,0.000000}%
\pgfsetfillcolor{currentfill}%
\pgfsetlinewidth{1.003750pt}%
\definecolor{currentstroke}{rgb}{0.000000,0.000000,0.000000}%
\pgfsetstrokecolor{currentstroke}%
\pgfsetdash{}{0pt}%
\pgfpathmoveto{\pgfqpoint{4.288000in}{3.695220in}}%
\pgfpathcurveto{\pgfqpoint{4.294572in}{3.695220in}}{\pgfqpoint{4.300875in}{3.697831in}}{\pgfqpoint{4.305522in}{3.702478in}}%
\pgfpathcurveto{\pgfqpoint{4.310169in}{3.707125in}}{\pgfqpoint{4.312780in}{3.713428in}}{\pgfqpoint{4.312780in}{3.720000in}}%
\pgfpathcurveto{\pgfqpoint{4.312780in}{3.726572in}}{\pgfqpoint{4.310169in}{3.732875in}}{\pgfqpoint{4.305522in}{3.737522in}}%
\pgfpathcurveto{\pgfqpoint{4.300875in}{3.742169in}}{\pgfqpoint{4.294572in}{3.744780in}}{\pgfqpoint{4.288000in}{3.744780in}}%
\pgfpathcurveto{\pgfqpoint{4.281428in}{3.744780in}}{\pgfqpoint{4.275125in}{3.742169in}}{\pgfqpoint{4.270478in}{3.737522in}}%
\pgfpathcurveto{\pgfqpoint{4.265831in}{3.732875in}}{\pgfqpoint{4.263220in}{3.726572in}}{\pgfqpoint{4.263220in}{3.720000in}}%
\pgfpathcurveto{\pgfqpoint{4.263220in}{3.713428in}}{\pgfqpoint{4.265831in}{3.707125in}}{\pgfqpoint{4.270478in}{3.702478in}}%
\pgfpathcurveto{\pgfqpoint{4.275125in}{3.697831in}}{\pgfqpoint{4.281428in}{3.695220in}}{\pgfqpoint{4.288000in}{3.695220in}}%
\pgfpathclose%
\pgfusepath{stroke,fill}%
\end{pgfscope}%
\begin{pgfscope}%
\pgfpathrectangle{\pgfqpoint{1.432000in}{0.528000in}}{\pgfqpoint{3.696000in}{3.696000in}}%
\pgfusepath{clip}%
\pgfsetbuttcap%
\pgfsetroundjoin%
\definecolor{currentfill}{rgb}{0.000000,0.000000,0.000000}%
\pgfsetfillcolor{currentfill}%
\pgfsetlinewidth{1.003750pt}%
\definecolor{currentstroke}{rgb}{0.000000,0.000000,0.000000}%
\pgfsetstrokecolor{currentstroke}%
\pgfsetdash{}{0pt}%
\pgfpathmoveto{\pgfqpoint{4.288000in}{4.031220in}}%
\pgfpathcurveto{\pgfqpoint{4.294572in}{4.031220in}}{\pgfqpoint{4.300875in}{4.033831in}}{\pgfqpoint{4.305522in}{4.038478in}}%
\pgfpathcurveto{\pgfqpoint{4.310169in}{4.043125in}}{\pgfqpoint{4.312780in}{4.049428in}}{\pgfqpoint{4.312780in}{4.056000in}}%
\pgfpathcurveto{\pgfqpoint{4.312780in}{4.062572in}}{\pgfqpoint{4.310169in}{4.068875in}}{\pgfqpoint{4.305522in}{4.073522in}}%
\pgfpathcurveto{\pgfqpoint{4.300875in}{4.078169in}}{\pgfqpoint{4.294572in}{4.080780in}}{\pgfqpoint{4.288000in}{4.080780in}}%
\pgfpathcurveto{\pgfqpoint{4.281428in}{4.080780in}}{\pgfqpoint{4.275125in}{4.078169in}}{\pgfqpoint{4.270478in}{4.073522in}}%
\pgfpathcurveto{\pgfqpoint{4.265831in}{4.068875in}}{\pgfqpoint{4.263220in}{4.062572in}}{\pgfqpoint{4.263220in}{4.056000in}}%
\pgfpathcurveto{\pgfqpoint{4.263220in}{4.049428in}}{\pgfqpoint{4.265831in}{4.043125in}}{\pgfqpoint{4.270478in}{4.038478in}}%
\pgfpathcurveto{\pgfqpoint{4.275125in}{4.033831in}}{\pgfqpoint{4.281428in}{4.031220in}}{\pgfqpoint{4.288000in}{4.031220in}}%
\pgfpathclose%
\pgfusepath{stroke,fill}%
\end{pgfscope}%
\begin{pgfscope}%
\pgfpathrectangle{\pgfqpoint{1.432000in}{0.528000in}}{\pgfqpoint{3.696000in}{3.696000in}}%
\pgfusepath{clip}%
\pgfsetbuttcap%
\pgfsetroundjoin%
\definecolor{currentfill}{rgb}{0.000000,0.000000,0.000000}%
\pgfsetfillcolor{currentfill}%
\pgfsetlinewidth{1.003750pt}%
\definecolor{currentstroke}{rgb}{0.000000,0.000000,0.000000}%
\pgfsetstrokecolor{currentstroke}%
\pgfsetdash{}{0pt}%
\pgfpathmoveto{\pgfqpoint{4.624000in}{0.671220in}}%
\pgfpathcurveto{\pgfqpoint{4.630572in}{0.671220in}}{\pgfqpoint{4.636875in}{0.673831in}}{\pgfqpoint{4.641522in}{0.678478in}}%
\pgfpathcurveto{\pgfqpoint{4.646169in}{0.683125in}}{\pgfqpoint{4.648780in}{0.689428in}}{\pgfqpoint{4.648780in}{0.696000in}}%
\pgfpathcurveto{\pgfqpoint{4.648780in}{0.702572in}}{\pgfqpoint{4.646169in}{0.708875in}}{\pgfqpoint{4.641522in}{0.713522in}}%
\pgfpathcurveto{\pgfqpoint{4.636875in}{0.718169in}}{\pgfqpoint{4.630572in}{0.720780in}}{\pgfqpoint{4.624000in}{0.720780in}}%
\pgfpathcurveto{\pgfqpoint{4.617428in}{0.720780in}}{\pgfqpoint{4.611125in}{0.718169in}}{\pgfqpoint{4.606478in}{0.713522in}}%
\pgfpathcurveto{\pgfqpoint{4.601831in}{0.708875in}}{\pgfqpoint{4.599220in}{0.702572in}}{\pgfqpoint{4.599220in}{0.696000in}}%
\pgfpathcurveto{\pgfqpoint{4.599220in}{0.689428in}}{\pgfqpoint{4.601831in}{0.683125in}}{\pgfqpoint{4.606478in}{0.678478in}}%
\pgfpathcurveto{\pgfqpoint{4.611125in}{0.673831in}}{\pgfqpoint{4.617428in}{0.671220in}}{\pgfqpoint{4.624000in}{0.671220in}}%
\pgfpathclose%
\pgfusepath{stroke,fill}%
\end{pgfscope}%
\begin{pgfscope}%
\pgfpathrectangle{\pgfqpoint{1.432000in}{0.528000in}}{\pgfqpoint{3.696000in}{3.696000in}}%
\pgfusepath{clip}%
\pgfsetbuttcap%
\pgfsetroundjoin%
\definecolor{currentfill}{rgb}{0.000000,0.000000,0.000000}%
\pgfsetfillcolor{currentfill}%
\pgfsetlinewidth{1.003750pt}%
\definecolor{currentstroke}{rgb}{0.000000,0.000000,0.000000}%
\pgfsetstrokecolor{currentstroke}%
\pgfsetdash{}{0pt}%
\pgfpathmoveto{\pgfqpoint{4.624000in}{1.007220in}}%
\pgfpathcurveto{\pgfqpoint{4.630572in}{1.007220in}}{\pgfqpoint{4.636875in}{1.009831in}}{\pgfqpoint{4.641522in}{1.014478in}}%
\pgfpathcurveto{\pgfqpoint{4.646169in}{1.019125in}}{\pgfqpoint{4.648780in}{1.025428in}}{\pgfqpoint{4.648780in}{1.032000in}}%
\pgfpathcurveto{\pgfqpoint{4.648780in}{1.038572in}}{\pgfqpoint{4.646169in}{1.044875in}}{\pgfqpoint{4.641522in}{1.049522in}}%
\pgfpathcurveto{\pgfqpoint{4.636875in}{1.054169in}}{\pgfqpoint{4.630572in}{1.056780in}}{\pgfqpoint{4.624000in}{1.056780in}}%
\pgfpathcurveto{\pgfqpoint{4.617428in}{1.056780in}}{\pgfqpoint{4.611125in}{1.054169in}}{\pgfqpoint{4.606478in}{1.049522in}}%
\pgfpathcurveto{\pgfqpoint{4.601831in}{1.044875in}}{\pgfqpoint{4.599220in}{1.038572in}}{\pgfqpoint{4.599220in}{1.032000in}}%
\pgfpathcurveto{\pgfqpoint{4.599220in}{1.025428in}}{\pgfqpoint{4.601831in}{1.019125in}}{\pgfqpoint{4.606478in}{1.014478in}}%
\pgfpathcurveto{\pgfqpoint{4.611125in}{1.009831in}}{\pgfqpoint{4.617428in}{1.007220in}}{\pgfqpoint{4.624000in}{1.007220in}}%
\pgfpathclose%
\pgfusepath{stroke,fill}%
\end{pgfscope}%
\begin{pgfscope}%
\pgfpathrectangle{\pgfqpoint{1.432000in}{0.528000in}}{\pgfqpoint{3.696000in}{3.696000in}}%
\pgfusepath{clip}%
\pgfsetbuttcap%
\pgfsetroundjoin%
\definecolor{currentfill}{rgb}{0.000000,0.000000,0.000000}%
\pgfsetfillcolor{currentfill}%
\pgfsetlinewidth{1.003750pt}%
\definecolor{currentstroke}{rgb}{0.000000,0.000000,0.000000}%
\pgfsetstrokecolor{currentstroke}%
\pgfsetdash{}{0pt}%
\pgfpathmoveto{\pgfqpoint{4.624000in}{1.343220in}}%
\pgfpathcurveto{\pgfqpoint{4.630572in}{1.343220in}}{\pgfqpoint{4.636875in}{1.345831in}}{\pgfqpoint{4.641522in}{1.350478in}}%
\pgfpathcurveto{\pgfqpoint{4.646169in}{1.355125in}}{\pgfqpoint{4.648780in}{1.361428in}}{\pgfqpoint{4.648780in}{1.368000in}}%
\pgfpathcurveto{\pgfqpoint{4.648780in}{1.374572in}}{\pgfqpoint{4.646169in}{1.380875in}}{\pgfqpoint{4.641522in}{1.385522in}}%
\pgfpathcurveto{\pgfqpoint{4.636875in}{1.390169in}}{\pgfqpoint{4.630572in}{1.392780in}}{\pgfqpoint{4.624000in}{1.392780in}}%
\pgfpathcurveto{\pgfqpoint{4.617428in}{1.392780in}}{\pgfqpoint{4.611125in}{1.390169in}}{\pgfqpoint{4.606478in}{1.385522in}}%
\pgfpathcurveto{\pgfqpoint{4.601831in}{1.380875in}}{\pgfqpoint{4.599220in}{1.374572in}}{\pgfqpoint{4.599220in}{1.368000in}}%
\pgfpathcurveto{\pgfqpoint{4.599220in}{1.361428in}}{\pgfqpoint{4.601831in}{1.355125in}}{\pgfqpoint{4.606478in}{1.350478in}}%
\pgfpathcurveto{\pgfqpoint{4.611125in}{1.345831in}}{\pgfqpoint{4.617428in}{1.343220in}}{\pgfqpoint{4.624000in}{1.343220in}}%
\pgfpathclose%
\pgfusepath{stroke,fill}%
\end{pgfscope}%
\begin{pgfscope}%
\pgfpathrectangle{\pgfqpoint{1.432000in}{0.528000in}}{\pgfqpoint{3.696000in}{3.696000in}}%
\pgfusepath{clip}%
\pgfsetbuttcap%
\pgfsetroundjoin%
\definecolor{currentfill}{rgb}{0.000000,0.000000,0.000000}%
\pgfsetfillcolor{currentfill}%
\pgfsetlinewidth{1.003750pt}%
\definecolor{currentstroke}{rgb}{0.000000,0.000000,0.000000}%
\pgfsetstrokecolor{currentstroke}%
\pgfsetdash{}{0pt}%
\pgfpathmoveto{\pgfqpoint{4.624000in}{1.679220in}}%
\pgfpathcurveto{\pgfqpoint{4.630572in}{1.679220in}}{\pgfqpoint{4.636875in}{1.681831in}}{\pgfqpoint{4.641522in}{1.686478in}}%
\pgfpathcurveto{\pgfqpoint{4.646169in}{1.691125in}}{\pgfqpoint{4.648780in}{1.697428in}}{\pgfqpoint{4.648780in}{1.704000in}}%
\pgfpathcurveto{\pgfqpoint{4.648780in}{1.710572in}}{\pgfqpoint{4.646169in}{1.716875in}}{\pgfqpoint{4.641522in}{1.721522in}}%
\pgfpathcurveto{\pgfqpoint{4.636875in}{1.726169in}}{\pgfqpoint{4.630572in}{1.728780in}}{\pgfqpoint{4.624000in}{1.728780in}}%
\pgfpathcurveto{\pgfqpoint{4.617428in}{1.728780in}}{\pgfqpoint{4.611125in}{1.726169in}}{\pgfqpoint{4.606478in}{1.721522in}}%
\pgfpathcurveto{\pgfqpoint{4.601831in}{1.716875in}}{\pgfqpoint{4.599220in}{1.710572in}}{\pgfqpoint{4.599220in}{1.704000in}}%
\pgfpathcurveto{\pgfqpoint{4.599220in}{1.697428in}}{\pgfqpoint{4.601831in}{1.691125in}}{\pgfqpoint{4.606478in}{1.686478in}}%
\pgfpathcurveto{\pgfqpoint{4.611125in}{1.681831in}}{\pgfqpoint{4.617428in}{1.679220in}}{\pgfqpoint{4.624000in}{1.679220in}}%
\pgfpathclose%
\pgfusepath{stroke,fill}%
\end{pgfscope}%
\begin{pgfscope}%
\pgfpathrectangle{\pgfqpoint{1.432000in}{0.528000in}}{\pgfqpoint{3.696000in}{3.696000in}}%
\pgfusepath{clip}%
\pgfsetbuttcap%
\pgfsetroundjoin%
\definecolor{currentfill}{rgb}{0.000000,0.000000,0.000000}%
\pgfsetfillcolor{currentfill}%
\pgfsetlinewidth{1.003750pt}%
\definecolor{currentstroke}{rgb}{0.000000,0.000000,0.000000}%
\pgfsetstrokecolor{currentstroke}%
\pgfsetdash{}{0pt}%
\pgfpathmoveto{\pgfqpoint{4.624000in}{2.015220in}}%
\pgfpathcurveto{\pgfqpoint{4.630572in}{2.015220in}}{\pgfqpoint{4.636875in}{2.017831in}}{\pgfqpoint{4.641522in}{2.022478in}}%
\pgfpathcurveto{\pgfqpoint{4.646169in}{2.027125in}}{\pgfqpoint{4.648780in}{2.033428in}}{\pgfqpoint{4.648780in}{2.040000in}}%
\pgfpathcurveto{\pgfqpoint{4.648780in}{2.046572in}}{\pgfqpoint{4.646169in}{2.052875in}}{\pgfqpoint{4.641522in}{2.057522in}}%
\pgfpathcurveto{\pgfqpoint{4.636875in}{2.062169in}}{\pgfqpoint{4.630572in}{2.064780in}}{\pgfqpoint{4.624000in}{2.064780in}}%
\pgfpathcurveto{\pgfqpoint{4.617428in}{2.064780in}}{\pgfqpoint{4.611125in}{2.062169in}}{\pgfqpoint{4.606478in}{2.057522in}}%
\pgfpathcurveto{\pgfqpoint{4.601831in}{2.052875in}}{\pgfqpoint{4.599220in}{2.046572in}}{\pgfqpoint{4.599220in}{2.040000in}}%
\pgfpathcurveto{\pgfqpoint{4.599220in}{2.033428in}}{\pgfqpoint{4.601831in}{2.027125in}}{\pgfqpoint{4.606478in}{2.022478in}}%
\pgfpathcurveto{\pgfqpoint{4.611125in}{2.017831in}}{\pgfqpoint{4.617428in}{2.015220in}}{\pgfqpoint{4.624000in}{2.015220in}}%
\pgfpathclose%
\pgfusepath{stroke,fill}%
\end{pgfscope}%
\begin{pgfscope}%
\pgfpathrectangle{\pgfqpoint{1.432000in}{0.528000in}}{\pgfqpoint{3.696000in}{3.696000in}}%
\pgfusepath{clip}%
\pgfsetbuttcap%
\pgfsetroundjoin%
\definecolor{currentfill}{rgb}{0.000000,0.000000,0.000000}%
\pgfsetfillcolor{currentfill}%
\pgfsetlinewidth{1.003750pt}%
\definecolor{currentstroke}{rgb}{0.000000,0.000000,0.000000}%
\pgfsetstrokecolor{currentstroke}%
\pgfsetdash{}{0pt}%
\pgfpathmoveto{\pgfqpoint{4.624000in}{2.351220in}}%
\pgfpathcurveto{\pgfqpoint{4.630572in}{2.351220in}}{\pgfqpoint{4.636875in}{2.353831in}}{\pgfqpoint{4.641522in}{2.358478in}}%
\pgfpathcurveto{\pgfqpoint{4.646169in}{2.363125in}}{\pgfqpoint{4.648780in}{2.369428in}}{\pgfqpoint{4.648780in}{2.376000in}}%
\pgfpathcurveto{\pgfqpoint{4.648780in}{2.382572in}}{\pgfqpoint{4.646169in}{2.388875in}}{\pgfqpoint{4.641522in}{2.393522in}}%
\pgfpathcurveto{\pgfqpoint{4.636875in}{2.398169in}}{\pgfqpoint{4.630572in}{2.400780in}}{\pgfqpoint{4.624000in}{2.400780in}}%
\pgfpathcurveto{\pgfqpoint{4.617428in}{2.400780in}}{\pgfqpoint{4.611125in}{2.398169in}}{\pgfqpoint{4.606478in}{2.393522in}}%
\pgfpathcurveto{\pgfqpoint{4.601831in}{2.388875in}}{\pgfqpoint{4.599220in}{2.382572in}}{\pgfqpoint{4.599220in}{2.376000in}}%
\pgfpathcurveto{\pgfqpoint{4.599220in}{2.369428in}}{\pgfqpoint{4.601831in}{2.363125in}}{\pgfqpoint{4.606478in}{2.358478in}}%
\pgfpathcurveto{\pgfqpoint{4.611125in}{2.353831in}}{\pgfqpoint{4.617428in}{2.351220in}}{\pgfqpoint{4.624000in}{2.351220in}}%
\pgfpathclose%
\pgfusepath{stroke,fill}%
\end{pgfscope}%
\begin{pgfscope}%
\pgfpathrectangle{\pgfqpoint{1.432000in}{0.528000in}}{\pgfqpoint{3.696000in}{3.696000in}}%
\pgfusepath{clip}%
\pgfsetbuttcap%
\pgfsetroundjoin%
\definecolor{currentfill}{rgb}{0.000000,0.000000,0.000000}%
\pgfsetfillcolor{currentfill}%
\pgfsetlinewidth{1.003750pt}%
\definecolor{currentstroke}{rgb}{0.000000,0.000000,0.000000}%
\pgfsetstrokecolor{currentstroke}%
\pgfsetdash{}{0pt}%
\pgfpathmoveto{\pgfqpoint{4.624000in}{2.687220in}}%
\pgfpathcurveto{\pgfqpoint{4.630572in}{2.687220in}}{\pgfqpoint{4.636875in}{2.689831in}}{\pgfqpoint{4.641522in}{2.694478in}}%
\pgfpathcurveto{\pgfqpoint{4.646169in}{2.699125in}}{\pgfqpoint{4.648780in}{2.705428in}}{\pgfqpoint{4.648780in}{2.712000in}}%
\pgfpathcurveto{\pgfqpoint{4.648780in}{2.718572in}}{\pgfqpoint{4.646169in}{2.724875in}}{\pgfqpoint{4.641522in}{2.729522in}}%
\pgfpathcurveto{\pgfqpoint{4.636875in}{2.734169in}}{\pgfqpoint{4.630572in}{2.736780in}}{\pgfqpoint{4.624000in}{2.736780in}}%
\pgfpathcurveto{\pgfqpoint{4.617428in}{2.736780in}}{\pgfqpoint{4.611125in}{2.734169in}}{\pgfqpoint{4.606478in}{2.729522in}}%
\pgfpathcurveto{\pgfqpoint{4.601831in}{2.724875in}}{\pgfqpoint{4.599220in}{2.718572in}}{\pgfqpoint{4.599220in}{2.712000in}}%
\pgfpathcurveto{\pgfqpoint{4.599220in}{2.705428in}}{\pgfqpoint{4.601831in}{2.699125in}}{\pgfqpoint{4.606478in}{2.694478in}}%
\pgfpathcurveto{\pgfqpoint{4.611125in}{2.689831in}}{\pgfqpoint{4.617428in}{2.687220in}}{\pgfqpoint{4.624000in}{2.687220in}}%
\pgfpathclose%
\pgfusepath{stroke,fill}%
\end{pgfscope}%
\begin{pgfscope}%
\pgfpathrectangle{\pgfqpoint{1.432000in}{0.528000in}}{\pgfqpoint{3.696000in}{3.696000in}}%
\pgfusepath{clip}%
\pgfsetbuttcap%
\pgfsetroundjoin%
\definecolor{currentfill}{rgb}{0.000000,0.000000,0.000000}%
\pgfsetfillcolor{currentfill}%
\pgfsetlinewidth{1.003750pt}%
\definecolor{currentstroke}{rgb}{0.000000,0.000000,0.000000}%
\pgfsetstrokecolor{currentstroke}%
\pgfsetdash{}{0pt}%
\pgfpathmoveto{\pgfqpoint{4.624000in}{3.023220in}}%
\pgfpathcurveto{\pgfqpoint{4.630572in}{3.023220in}}{\pgfqpoint{4.636875in}{3.025831in}}{\pgfqpoint{4.641522in}{3.030478in}}%
\pgfpathcurveto{\pgfqpoint{4.646169in}{3.035125in}}{\pgfqpoint{4.648780in}{3.041428in}}{\pgfqpoint{4.648780in}{3.048000in}}%
\pgfpathcurveto{\pgfqpoint{4.648780in}{3.054572in}}{\pgfqpoint{4.646169in}{3.060875in}}{\pgfqpoint{4.641522in}{3.065522in}}%
\pgfpathcurveto{\pgfqpoint{4.636875in}{3.070169in}}{\pgfqpoint{4.630572in}{3.072780in}}{\pgfqpoint{4.624000in}{3.072780in}}%
\pgfpathcurveto{\pgfqpoint{4.617428in}{3.072780in}}{\pgfqpoint{4.611125in}{3.070169in}}{\pgfqpoint{4.606478in}{3.065522in}}%
\pgfpathcurveto{\pgfqpoint{4.601831in}{3.060875in}}{\pgfqpoint{4.599220in}{3.054572in}}{\pgfqpoint{4.599220in}{3.048000in}}%
\pgfpathcurveto{\pgfqpoint{4.599220in}{3.041428in}}{\pgfqpoint{4.601831in}{3.035125in}}{\pgfqpoint{4.606478in}{3.030478in}}%
\pgfpathcurveto{\pgfqpoint{4.611125in}{3.025831in}}{\pgfqpoint{4.617428in}{3.023220in}}{\pgfqpoint{4.624000in}{3.023220in}}%
\pgfpathclose%
\pgfusepath{stroke,fill}%
\end{pgfscope}%
\begin{pgfscope}%
\pgfpathrectangle{\pgfqpoint{1.432000in}{0.528000in}}{\pgfqpoint{3.696000in}{3.696000in}}%
\pgfusepath{clip}%
\pgfsetbuttcap%
\pgfsetroundjoin%
\definecolor{currentfill}{rgb}{0.000000,0.000000,0.000000}%
\pgfsetfillcolor{currentfill}%
\pgfsetlinewidth{1.003750pt}%
\definecolor{currentstroke}{rgb}{0.000000,0.000000,0.000000}%
\pgfsetstrokecolor{currentstroke}%
\pgfsetdash{}{0pt}%
\pgfpathmoveto{\pgfqpoint{4.624000in}{3.359220in}}%
\pgfpathcurveto{\pgfqpoint{4.630572in}{3.359220in}}{\pgfqpoint{4.636875in}{3.361831in}}{\pgfqpoint{4.641522in}{3.366478in}}%
\pgfpathcurveto{\pgfqpoint{4.646169in}{3.371125in}}{\pgfqpoint{4.648780in}{3.377428in}}{\pgfqpoint{4.648780in}{3.384000in}}%
\pgfpathcurveto{\pgfqpoint{4.648780in}{3.390572in}}{\pgfqpoint{4.646169in}{3.396875in}}{\pgfqpoint{4.641522in}{3.401522in}}%
\pgfpathcurveto{\pgfqpoint{4.636875in}{3.406169in}}{\pgfqpoint{4.630572in}{3.408780in}}{\pgfqpoint{4.624000in}{3.408780in}}%
\pgfpathcurveto{\pgfqpoint{4.617428in}{3.408780in}}{\pgfqpoint{4.611125in}{3.406169in}}{\pgfqpoint{4.606478in}{3.401522in}}%
\pgfpathcurveto{\pgfqpoint{4.601831in}{3.396875in}}{\pgfqpoint{4.599220in}{3.390572in}}{\pgfqpoint{4.599220in}{3.384000in}}%
\pgfpathcurveto{\pgfqpoint{4.599220in}{3.377428in}}{\pgfqpoint{4.601831in}{3.371125in}}{\pgfqpoint{4.606478in}{3.366478in}}%
\pgfpathcurveto{\pgfqpoint{4.611125in}{3.361831in}}{\pgfqpoint{4.617428in}{3.359220in}}{\pgfqpoint{4.624000in}{3.359220in}}%
\pgfpathclose%
\pgfusepath{stroke,fill}%
\end{pgfscope}%
\begin{pgfscope}%
\pgfpathrectangle{\pgfqpoint{1.432000in}{0.528000in}}{\pgfqpoint{3.696000in}{3.696000in}}%
\pgfusepath{clip}%
\pgfsetbuttcap%
\pgfsetroundjoin%
\definecolor{currentfill}{rgb}{0.000000,0.000000,0.000000}%
\pgfsetfillcolor{currentfill}%
\pgfsetlinewidth{1.003750pt}%
\definecolor{currentstroke}{rgb}{0.000000,0.000000,0.000000}%
\pgfsetstrokecolor{currentstroke}%
\pgfsetdash{}{0pt}%
\pgfpathmoveto{\pgfqpoint{4.624000in}{3.695220in}}%
\pgfpathcurveto{\pgfqpoint{4.630572in}{3.695220in}}{\pgfqpoint{4.636875in}{3.697831in}}{\pgfqpoint{4.641522in}{3.702478in}}%
\pgfpathcurveto{\pgfqpoint{4.646169in}{3.707125in}}{\pgfqpoint{4.648780in}{3.713428in}}{\pgfqpoint{4.648780in}{3.720000in}}%
\pgfpathcurveto{\pgfqpoint{4.648780in}{3.726572in}}{\pgfqpoint{4.646169in}{3.732875in}}{\pgfqpoint{4.641522in}{3.737522in}}%
\pgfpathcurveto{\pgfqpoint{4.636875in}{3.742169in}}{\pgfqpoint{4.630572in}{3.744780in}}{\pgfqpoint{4.624000in}{3.744780in}}%
\pgfpathcurveto{\pgfqpoint{4.617428in}{3.744780in}}{\pgfqpoint{4.611125in}{3.742169in}}{\pgfqpoint{4.606478in}{3.737522in}}%
\pgfpathcurveto{\pgfqpoint{4.601831in}{3.732875in}}{\pgfqpoint{4.599220in}{3.726572in}}{\pgfqpoint{4.599220in}{3.720000in}}%
\pgfpathcurveto{\pgfqpoint{4.599220in}{3.713428in}}{\pgfqpoint{4.601831in}{3.707125in}}{\pgfqpoint{4.606478in}{3.702478in}}%
\pgfpathcurveto{\pgfqpoint{4.611125in}{3.697831in}}{\pgfqpoint{4.617428in}{3.695220in}}{\pgfqpoint{4.624000in}{3.695220in}}%
\pgfpathclose%
\pgfusepath{stroke,fill}%
\end{pgfscope}%
\begin{pgfscope}%
\pgfpathrectangle{\pgfqpoint{1.432000in}{0.528000in}}{\pgfqpoint{3.696000in}{3.696000in}}%
\pgfusepath{clip}%
\pgfsetbuttcap%
\pgfsetroundjoin%
\definecolor{currentfill}{rgb}{0.000000,0.000000,0.000000}%
\pgfsetfillcolor{currentfill}%
\pgfsetlinewidth{1.003750pt}%
\definecolor{currentstroke}{rgb}{0.000000,0.000000,0.000000}%
\pgfsetstrokecolor{currentstroke}%
\pgfsetdash{}{0pt}%
\pgfpathmoveto{\pgfqpoint{4.624000in}{4.031220in}}%
\pgfpathcurveto{\pgfqpoint{4.630572in}{4.031220in}}{\pgfqpoint{4.636875in}{4.033831in}}{\pgfqpoint{4.641522in}{4.038478in}}%
\pgfpathcurveto{\pgfqpoint{4.646169in}{4.043125in}}{\pgfqpoint{4.648780in}{4.049428in}}{\pgfqpoint{4.648780in}{4.056000in}}%
\pgfpathcurveto{\pgfqpoint{4.648780in}{4.062572in}}{\pgfqpoint{4.646169in}{4.068875in}}{\pgfqpoint{4.641522in}{4.073522in}}%
\pgfpathcurveto{\pgfqpoint{4.636875in}{4.078169in}}{\pgfqpoint{4.630572in}{4.080780in}}{\pgfqpoint{4.624000in}{4.080780in}}%
\pgfpathcurveto{\pgfqpoint{4.617428in}{4.080780in}}{\pgfqpoint{4.611125in}{4.078169in}}{\pgfqpoint{4.606478in}{4.073522in}}%
\pgfpathcurveto{\pgfqpoint{4.601831in}{4.068875in}}{\pgfqpoint{4.599220in}{4.062572in}}{\pgfqpoint{4.599220in}{4.056000in}}%
\pgfpathcurveto{\pgfqpoint{4.599220in}{4.049428in}}{\pgfqpoint{4.601831in}{4.043125in}}{\pgfqpoint{4.606478in}{4.038478in}}%
\pgfpathcurveto{\pgfqpoint{4.611125in}{4.033831in}}{\pgfqpoint{4.617428in}{4.031220in}}{\pgfqpoint{4.624000in}{4.031220in}}%
\pgfpathclose%
\pgfusepath{stroke,fill}%
\end{pgfscope}%
\begin{pgfscope}%
\pgfpathrectangle{\pgfqpoint{1.432000in}{0.528000in}}{\pgfqpoint{3.696000in}{3.696000in}}%
\pgfusepath{clip}%
\pgfsetbuttcap%
\pgfsetroundjoin%
\definecolor{currentfill}{rgb}{0.000000,0.000000,0.000000}%
\pgfsetfillcolor{currentfill}%
\pgfsetlinewidth{1.003750pt}%
\definecolor{currentstroke}{rgb}{0.000000,0.000000,0.000000}%
\pgfsetstrokecolor{currentstroke}%
\pgfsetdash{}{0pt}%
\pgfpathmoveto{\pgfqpoint{4.960000in}{0.671220in}}%
\pgfpathcurveto{\pgfqpoint{4.966572in}{0.671220in}}{\pgfqpoint{4.972875in}{0.673831in}}{\pgfqpoint{4.977522in}{0.678478in}}%
\pgfpathcurveto{\pgfqpoint{4.982169in}{0.683125in}}{\pgfqpoint{4.984780in}{0.689428in}}{\pgfqpoint{4.984780in}{0.696000in}}%
\pgfpathcurveto{\pgfqpoint{4.984780in}{0.702572in}}{\pgfqpoint{4.982169in}{0.708875in}}{\pgfqpoint{4.977522in}{0.713522in}}%
\pgfpathcurveto{\pgfqpoint{4.972875in}{0.718169in}}{\pgfqpoint{4.966572in}{0.720780in}}{\pgfqpoint{4.960000in}{0.720780in}}%
\pgfpathcurveto{\pgfqpoint{4.953428in}{0.720780in}}{\pgfqpoint{4.947125in}{0.718169in}}{\pgfqpoint{4.942478in}{0.713522in}}%
\pgfpathcurveto{\pgfqpoint{4.937831in}{0.708875in}}{\pgfqpoint{4.935220in}{0.702572in}}{\pgfqpoint{4.935220in}{0.696000in}}%
\pgfpathcurveto{\pgfqpoint{4.935220in}{0.689428in}}{\pgfqpoint{4.937831in}{0.683125in}}{\pgfqpoint{4.942478in}{0.678478in}}%
\pgfpathcurveto{\pgfqpoint{4.947125in}{0.673831in}}{\pgfqpoint{4.953428in}{0.671220in}}{\pgfqpoint{4.960000in}{0.671220in}}%
\pgfpathclose%
\pgfusepath{stroke,fill}%
\end{pgfscope}%
\begin{pgfscope}%
\pgfpathrectangle{\pgfqpoint{1.432000in}{0.528000in}}{\pgfqpoint{3.696000in}{3.696000in}}%
\pgfusepath{clip}%
\pgfsetbuttcap%
\pgfsetroundjoin%
\definecolor{currentfill}{rgb}{0.000000,0.000000,0.000000}%
\pgfsetfillcolor{currentfill}%
\pgfsetlinewidth{1.003750pt}%
\definecolor{currentstroke}{rgb}{0.000000,0.000000,0.000000}%
\pgfsetstrokecolor{currentstroke}%
\pgfsetdash{}{0pt}%
\pgfpathmoveto{\pgfqpoint{4.960000in}{1.007220in}}%
\pgfpathcurveto{\pgfqpoint{4.966572in}{1.007220in}}{\pgfqpoint{4.972875in}{1.009831in}}{\pgfqpoint{4.977522in}{1.014478in}}%
\pgfpathcurveto{\pgfqpoint{4.982169in}{1.019125in}}{\pgfqpoint{4.984780in}{1.025428in}}{\pgfqpoint{4.984780in}{1.032000in}}%
\pgfpathcurveto{\pgfqpoint{4.984780in}{1.038572in}}{\pgfqpoint{4.982169in}{1.044875in}}{\pgfqpoint{4.977522in}{1.049522in}}%
\pgfpathcurveto{\pgfqpoint{4.972875in}{1.054169in}}{\pgfqpoint{4.966572in}{1.056780in}}{\pgfqpoint{4.960000in}{1.056780in}}%
\pgfpathcurveto{\pgfqpoint{4.953428in}{1.056780in}}{\pgfqpoint{4.947125in}{1.054169in}}{\pgfqpoint{4.942478in}{1.049522in}}%
\pgfpathcurveto{\pgfqpoint{4.937831in}{1.044875in}}{\pgfqpoint{4.935220in}{1.038572in}}{\pgfqpoint{4.935220in}{1.032000in}}%
\pgfpathcurveto{\pgfqpoint{4.935220in}{1.025428in}}{\pgfqpoint{4.937831in}{1.019125in}}{\pgfqpoint{4.942478in}{1.014478in}}%
\pgfpathcurveto{\pgfqpoint{4.947125in}{1.009831in}}{\pgfqpoint{4.953428in}{1.007220in}}{\pgfqpoint{4.960000in}{1.007220in}}%
\pgfpathclose%
\pgfusepath{stroke,fill}%
\end{pgfscope}%
\begin{pgfscope}%
\pgfpathrectangle{\pgfqpoint{1.432000in}{0.528000in}}{\pgfqpoint{3.696000in}{3.696000in}}%
\pgfusepath{clip}%
\pgfsetbuttcap%
\pgfsetroundjoin%
\definecolor{currentfill}{rgb}{0.000000,0.000000,0.000000}%
\pgfsetfillcolor{currentfill}%
\pgfsetlinewidth{1.003750pt}%
\definecolor{currentstroke}{rgb}{0.000000,0.000000,0.000000}%
\pgfsetstrokecolor{currentstroke}%
\pgfsetdash{}{0pt}%
\pgfpathmoveto{\pgfqpoint{4.960000in}{1.343220in}}%
\pgfpathcurveto{\pgfqpoint{4.966572in}{1.343220in}}{\pgfqpoint{4.972875in}{1.345831in}}{\pgfqpoint{4.977522in}{1.350478in}}%
\pgfpathcurveto{\pgfqpoint{4.982169in}{1.355125in}}{\pgfqpoint{4.984780in}{1.361428in}}{\pgfqpoint{4.984780in}{1.368000in}}%
\pgfpathcurveto{\pgfqpoint{4.984780in}{1.374572in}}{\pgfqpoint{4.982169in}{1.380875in}}{\pgfqpoint{4.977522in}{1.385522in}}%
\pgfpathcurveto{\pgfqpoint{4.972875in}{1.390169in}}{\pgfqpoint{4.966572in}{1.392780in}}{\pgfqpoint{4.960000in}{1.392780in}}%
\pgfpathcurveto{\pgfqpoint{4.953428in}{1.392780in}}{\pgfqpoint{4.947125in}{1.390169in}}{\pgfqpoint{4.942478in}{1.385522in}}%
\pgfpathcurveto{\pgfqpoint{4.937831in}{1.380875in}}{\pgfqpoint{4.935220in}{1.374572in}}{\pgfqpoint{4.935220in}{1.368000in}}%
\pgfpathcurveto{\pgfqpoint{4.935220in}{1.361428in}}{\pgfqpoint{4.937831in}{1.355125in}}{\pgfqpoint{4.942478in}{1.350478in}}%
\pgfpathcurveto{\pgfqpoint{4.947125in}{1.345831in}}{\pgfqpoint{4.953428in}{1.343220in}}{\pgfqpoint{4.960000in}{1.343220in}}%
\pgfpathclose%
\pgfusepath{stroke,fill}%
\end{pgfscope}%
\begin{pgfscope}%
\pgfpathrectangle{\pgfqpoint{1.432000in}{0.528000in}}{\pgfqpoint{3.696000in}{3.696000in}}%
\pgfusepath{clip}%
\pgfsetbuttcap%
\pgfsetroundjoin%
\definecolor{currentfill}{rgb}{0.000000,0.000000,0.000000}%
\pgfsetfillcolor{currentfill}%
\pgfsetlinewidth{1.003750pt}%
\definecolor{currentstroke}{rgb}{0.000000,0.000000,0.000000}%
\pgfsetstrokecolor{currentstroke}%
\pgfsetdash{}{0pt}%
\pgfpathmoveto{\pgfqpoint{4.960000in}{1.679220in}}%
\pgfpathcurveto{\pgfqpoint{4.966572in}{1.679220in}}{\pgfqpoint{4.972875in}{1.681831in}}{\pgfqpoint{4.977522in}{1.686478in}}%
\pgfpathcurveto{\pgfqpoint{4.982169in}{1.691125in}}{\pgfqpoint{4.984780in}{1.697428in}}{\pgfqpoint{4.984780in}{1.704000in}}%
\pgfpathcurveto{\pgfqpoint{4.984780in}{1.710572in}}{\pgfqpoint{4.982169in}{1.716875in}}{\pgfqpoint{4.977522in}{1.721522in}}%
\pgfpathcurveto{\pgfqpoint{4.972875in}{1.726169in}}{\pgfqpoint{4.966572in}{1.728780in}}{\pgfqpoint{4.960000in}{1.728780in}}%
\pgfpathcurveto{\pgfqpoint{4.953428in}{1.728780in}}{\pgfqpoint{4.947125in}{1.726169in}}{\pgfqpoint{4.942478in}{1.721522in}}%
\pgfpathcurveto{\pgfqpoint{4.937831in}{1.716875in}}{\pgfqpoint{4.935220in}{1.710572in}}{\pgfqpoint{4.935220in}{1.704000in}}%
\pgfpathcurveto{\pgfqpoint{4.935220in}{1.697428in}}{\pgfqpoint{4.937831in}{1.691125in}}{\pgfqpoint{4.942478in}{1.686478in}}%
\pgfpathcurveto{\pgfqpoint{4.947125in}{1.681831in}}{\pgfqpoint{4.953428in}{1.679220in}}{\pgfqpoint{4.960000in}{1.679220in}}%
\pgfpathclose%
\pgfusepath{stroke,fill}%
\end{pgfscope}%
\begin{pgfscope}%
\pgfpathrectangle{\pgfqpoint{1.432000in}{0.528000in}}{\pgfqpoint{3.696000in}{3.696000in}}%
\pgfusepath{clip}%
\pgfsetbuttcap%
\pgfsetroundjoin%
\definecolor{currentfill}{rgb}{0.000000,0.000000,0.000000}%
\pgfsetfillcolor{currentfill}%
\pgfsetlinewidth{1.003750pt}%
\definecolor{currentstroke}{rgb}{0.000000,0.000000,0.000000}%
\pgfsetstrokecolor{currentstroke}%
\pgfsetdash{}{0pt}%
\pgfpathmoveto{\pgfqpoint{4.960000in}{2.015220in}}%
\pgfpathcurveto{\pgfqpoint{4.966572in}{2.015220in}}{\pgfqpoint{4.972875in}{2.017831in}}{\pgfqpoint{4.977522in}{2.022478in}}%
\pgfpathcurveto{\pgfqpoint{4.982169in}{2.027125in}}{\pgfqpoint{4.984780in}{2.033428in}}{\pgfqpoint{4.984780in}{2.040000in}}%
\pgfpathcurveto{\pgfqpoint{4.984780in}{2.046572in}}{\pgfqpoint{4.982169in}{2.052875in}}{\pgfqpoint{4.977522in}{2.057522in}}%
\pgfpathcurveto{\pgfqpoint{4.972875in}{2.062169in}}{\pgfqpoint{4.966572in}{2.064780in}}{\pgfqpoint{4.960000in}{2.064780in}}%
\pgfpathcurveto{\pgfqpoint{4.953428in}{2.064780in}}{\pgfqpoint{4.947125in}{2.062169in}}{\pgfqpoint{4.942478in}{2.057522in}}%
\pgfpathcurveto{\pgfqpoint{4.937831in}{2.052875in}}{\pgfqpoint{4.935220in}{2.046572in}}{\pgfqpoint{4.935220in}{2.040000in}}%
\pgfpathcurveto{\pgfqpoint{4.935220in}{2.033428in}}{\pgfqpoint{4.937831in}{2.027125in}}{\pgfqpoint{4.942478in}{2.022478in}}%
\pgfpathcurveto{\pgfqpoint{4.947125in}{2.017831in}}{\pgfqpoint{4.953428in}{2.015220in}}{\pgfqpoint{4.960000in}{2.015220in}}%
\pgfpathclose%
\pgfusepath{stroke,fill}%
\end{pgfscope}%
\begin{pgfscope}%
\pgfpathrectangle{\pgfqpoint{1.432000in}{0.528000in}}{\pgfqpoint{3.696000in}{3.696000in}}%
\pgfusepath{clip}%
\pgfsetbuttcap%
\pgfsetroundjoin%
\definecolor{currentfill}{rgb}{0.000000,0.000000,0.000000}%
\pgfsetfillcolor{currentfill}%
\pgfsetlinewidth{1.003750pt}%
\definecolor{currentstroke}{rgb}{0.000000,0.000000,0.000000}%
\pgfsetstrokecolor{currentstroke}%
\pgfsetdash{}{0pt}%
\pgfpathmoveto{\pgfqpoint{4.960000in}{2.351220in}}%
\pgfpathcurveto{\pgfqpoint{4.966572in}{2.351220in}}{\pgfqpoint{4.972875in}{2.353831in}}{\pgfqpoint{4.977522in}{2.358478in}}%
\pgfpathcurveto{\pgfqpoint{4.982169in}{2.363125in}}{\pgfqpoint{4.984780in}{2.369428in}}{\pgfqpoint{4.984780in}{2.376000in}}%
\pgfpathcurveto{\pgfqpoint{4.984780in}{2.382572in}}{\pgfqpoint{4.982169in}{2.388875in}}{\pgfqpoint{4.977522in}{2.393522in}}%
\pgfpathcurveto{\pgfqpoint{4.972875in}{2.398169in}}{\pgfqpoint{4.966572in}{2.400780in}}{\pgfqpoint{4.960000in}{2.400780in}}%
\pgfpathcurveto{\pgfqpoint{4.953428in}{2.400780in}}{\pgfqpoint{4.947125in}{2.398169in}}{\pgfqpoint{4.942478in}{2.393522in}}%
\pgfpathcurveto{\pgfqpoint{4.937831in}{2.388875in}}{\pgfqpoint{4.935220in}{2.382572in}}{\pgfqpoint{4.935220in}{2.376000in}}%
\pgfpathcurveto{\pgfqpoint{4.935220in}{2.369428in}}{\pgfqpoint{4.937831in}{2.363125in}}{\pgfqpoint{4.942478in}{2.358478in}}%
\pgfpathcurveto{\pgfqpoint{4.947125in}{2.353831in}}{\pgfqpoint{4.953428in}{2.351220in}}{\pgfqpoint{4.960000in}{2.351220in}}%
\pgfpathclose%
\pgfusepath{stroke,fill}%
\end{pgfscope}%
\begin{pgfscope}%
\pgfpathrectangle{\pgfqpoint{1.432000in}{0.528000in}}{\pgfqpoint{3.696000in}{3.696000in}}%
\pgfusepath{clip}%
\pgfsetbuttcap%
\pgfsetroundjoin%
\definecolor{currentfill}{rgb}{0.000000,0.000000,0.000000}%
\pgfsetfillcolor{currentfill}%
\pgfsetlinewidth{1.003750pt}%
\definecolor{currentstroke}{rgb}{0.000000,0.000000,0.000000}%
\pgfsetstrokecolor{currentstroke}%
\pgfsetdash{}{0pt}%
\pgfpathmoveto{\pgfqpoint{4.960000in}{2.687220in}}%
\pgfpathcurveto{\pgfqpoint{4.966572in}{2.687220in}}{\pgfqpoint{4.972875in}{2.689831in}}{\pgfqpoint{4.977522in}{2.694478in}}%
\pgfpathcurveto{\pgfqpoint{4.982169in}{2.699125in}}{\pgfqpoint{4.984780in}{2.705428in}}{\pgfqpoint{4.984780in}{2.712000in}}%
\pgfpathcurveto{\pgfqpoint{4.984780in}{2.718572in}}{\pgfqpoint{4.982169in}{2.724875in}}{\pgfqpoint{4.977522in}{2.729522in}}%
\pgfpathcurveto{\pgfqpoint{4.972875in}{2.734169in}}{\pgfqpoint{4.966572in}{2.736780in}}{\pgfqpoint{4.960000in}{2.736780in}}%
\pgfpathcurveto{\pgfqpoint{4.953428in}{2.736780in}}{\pgfqpoint{4.947125in}{2.734169in}}{\pgfqpoint{4.942478in}{2.729522in}}%
\pgfpathcurveto{\pgfqpoint{4.937831in}{2.724875in}}{\pgfqpoint{4.935220in}{2.718572in}}{\pgfqpoint{4.935220in}{2.712000in}}%
\pgfpathcurveto{\pgfqpoint{4.935220in}{2.705428in}}{\pgfqpoint{4.937831in}{2.699125in}}{\pgfqpoint{4.942478in}{2.694478in}}%
\pgfpathcurveto{\pgfqpoint{4.947125in}{2.689831in}}{\pgfqpoint{4.953428in}{2.687220in}}{\pgfqpoint{4.960000in}{2.687220in}}%
\pgfpathclose%
\pgfusepath{stroke,fill}%
\end{pgfscope}%
\begin{pgfscope}%
\pgfpathrectangle{\pgfqpoint{1.432000in}{0.528000in}}{\pgfqpoint{3.696000in}{3.696000in}}%
\pgfusepath{clip}%
\pgfsetbuttcap%
\pgfsetroundjoin%
\definecolor{currentfill}{rgb}{0.000000,0.000000,0.000000}%
\pgfsetfillcolor{currentfill}%
\pgfsetlinewidth{1.003750pt}%
\definecolor{currentstroke}{rgb}{0.000000,0.000000,0.000000}%
\pgfsetstrokecolor{currentstroke}%
\pgfsetdash{}{0pt}%
\pgfpathmoveto{\pgfqpoint{4.960000in}{3.023220in}}%
\pgfpathcurveto{\pgfqpoint{4.966572in}{3.023220in}}{\pgfqpoint{4.972875in}{3.025831in}}{\pgfqpoint{4.977522in}{3.030478in}}%
\pgfpathcurveto{\pgfqpoint{4.982169in}{3.035125in}}{\pgfqpoint{4.984780in}{3.041428in}}{\pgfqpoint{4.984780in}{3.048000in}}%
\pgfpathcurveto{\pgfqpoint{4.984780in}{3.054572in}}{\pgfqpoint{4.982169in}{3.060875in}}{\pgfqpoint{4.977522in}{3.065522in}}%
\pgfpathcurveto{\pgfqpoint{4.972875in}{3.070169in}}{\pgfqpoint{4.966572in}{3.072780in}}{\pgfqpoint{4.960000in}{3.072780in}}%
\pgfpathcurveto{\pgfqpoint{4.953428in}{3.072780in}}{\pgfqpoint{4.947125in}{3.070169in}}{\pgfqpoint{4.942478in}{3.065522in}}%
\pgfpathcurveto{\pgfqpoint{4.937831in}{3.060875in}}{\pgfqpoint{4.935220in}{3.054572in}}{\pgfqpoint{4.935220in}{3.048000in}}%
\pgfpathcurveto{\pgfqpoint{4.935220in}{3.041428in}}{\pgfqpoint{4.937831in}{3.035125in}}{\pgfqpoint{4.942478in}{3.030478in}}%
\pgfpathcurveto{\pgfqpoint{4.947125in}{3.025831in}}{\pgfqpoint{4.953428in}{3.023220in}}{\pgfqpoint{4.960000in}{3.023220in}}%
\pgfpathclose%
\pgfusepath{stroke,fill}%
\end{pgfscope}%
\begin{pgfscope}%
\pgfpathrectangle{\pgfqpoint{1.432000in}{0.528000in}}{\pgfqpoint{3.696000in}{3.696000in}}%
\pgfusepath{clip}%
\pgfsetbuttcap%
\pgfsetroundjoin%
\definecolor{currentfill}{rgb}{0.000000,0.000000,0.000000}%
\pgfsetfillcolor{currentfill}%
\pgfsetlinewidth{1.003750pt}%
\definecolor{currentstroke}{rgb}{0.000000,0.000000,0.000000}%
\pgfsetstrokecolor{currentstroke}%
\pgfsetdash{}{0pt}%
\pgfpathmoveto{\pgfqpoint{4.960000in}{3.359220in}}%
\pgfpathcurveto{\pgfqpoint{4.966572in}{3.359220in}}{\pgfqpoint{4.972875in}{3.361831in}}{\pgfqpoint{4.977522in}{3.366478in}}%
\pgfpathcurveto{\pgfqpoint{4.982169in}{3.371125in}}{\pgfqpoint{4.984780in}{3.377428in}}{\pgfqpoint{4.984780in}{3.384000in}}%
\pgfpathcurveto{\pgfqpoint{4.984780in}{3.390572in}}{\pgfqpoint{4.982169in}{3.396875in}}{\pgfqpoint{4.977522in}{3.401522in}}%
\pgfpathcurveto{\pgfqpoint{4.972875in}{3.406169in}}{\pgfqpoint{4.966572in}{3.408780in}}{\pgfqpoint{4.960000in}{3.408780in}}%
\pgfpathcurveto{\pgfqpoint{4.953428in}{3.408780in}}{\pgfqpoint{4.947125in}{3.406169in}}{\pgfqpoint{4.942478in}{3.401522in}}%
\pgfpathcurveto{\pgfqpoint{4.937831in}{3.396875in}}{\pgfqpoint{4.935220in}{3.390572in}}{\pgfqpoint{4.935220in}{3.384000in}}%
\pgfpathcurveto{\pgfqpoint{4.935220in}{3.377428in}}{\pgfqpoint{4.937831in}{3.371125in}}{\pgfqpoint{4.942478in}{3.366478in}}%
\pgfpathcurveto{\pgfqpoint{4.947125in}{3.361831in}}{\pgfqpoint{4.953428in}{3.359220in}}{\pgfqpoint{4.960000in}{3.359220in}}%
\pgfpathclose%
\pgfusepath{stroke,fill}%
\end{pgfscope}%
\begin{pgfscope}%
\pgfpathrectangle{\pgfqpoint{1.432000in}{0.528000in}}{\pgfqpoint{3.696000in}{3.696000in}}%
\pgfusepath{clip}%
\pgfsetbuttcap%
\pgfsetroundjoin%
\definecolor{currentfill}{rgb}{0.000000,0.000000,0.000000}%
\pgfsetfillcolor{currentfill}%
\pgfsetlinewidth{1.003750pt}%
\definecolor{currentstroke}{rgb}{0.000000,0.000000,0.000000}%
\pgfsetstrokecolor{currentstroke}%
\pgfsetdash{}{0pt}%
\pgfpathmoveto{\pgfqpoint{4.960000in}{3.695220in}}%
\pgfpathcurveto{\pgfqpoint{4.966572in}{3.695220in}}{\pgfqpoint{4.972875in}{3.697831in}}{\pgfqpoint{4.977522in}{3.702478in}}%
\pgfpathcurveto{\pgfqpoint{4.982169in}{3.707125in}}{\pgfqpoint{4.984780in}{3.713428in}}{\pgfqpoint{4.984780in}{3.720000in}}%
\pgfpathcurveto{\pgfqpoint{4.984780in}{3.726572in}}{\pgfqpoint{4.982169in}{3.732875in}}{\pgfqpoint{4.977522in}{3.737522in}}%
\pgfpathcurveto{\pgfqpoint{4.972875in}{3.742169in}}{\pgfqpoint{4.966572in}{3.744780in}}{\pgfqpoint{4.960000in}{3.744780in}}%
\pgfpathcurveto{\pgfqpoint{4.953428in}{3.744780in}}{\pgfqpoint{4.947125in}{3.742169in}}{\pgfqpoint{4.942478in}{3.737522in}}%
\pgfpathcurveto{\pgfqpoint{4.937831in}{3.732875in}}{\pgfqpoint{4.935220in}{3.726572in}}{\pgfqpoint{4.935220in}{3.720000in}}%
\pgfpathcurveto{\pgfqpoint{4.935220in}{3.713428in}}{\pgfqpoint{4.937831in}{3.707125in}}{\pgfqpoint{4.942478in}{3.702478in}}%
\pgfpathcurveto{\pgfqpoint{4.947125in}{3.697831in}}{\pgfqpoint{4.953428in}{3.695220in}}{\pgfqpoint{4.960000in}{3.695220in}}%
\pgfpathclose%
\pgfusepath{stroke,fill}%
\end{pgfscope}%
\begin{pgfscope}%
\pgfpathrectangle{\pgfqpoint{1.432000in}{0.528000in}}{\pgfqpoint{3.696000in}{3.696000in}}%
\pgfusepath{clip}%
\pgfsetbuttcap%
\pgfsetroundjoin%
\definecolor{currentfill}{rgb}{0.000000,0.000000,0.000000}%
\pgfsetfillcolor{currentfill}%
\pgfsetlinewidth{1.003750pt}%
\definecolor{currentstroke}{rgb}{0.000000,0.000000,0.000000}%
\pgfsetstrokecolor{currentstroke}%
\pgfsetdash{}{0pt}%
\pgfpathmoveto{\pgfqpoint{4.960000in}{4.031220in}}%
\pgfpathcurveto{\pgfqpoint{4.966572in}{4.031220in}}{\pgfqpoint{4.972875in}{4.033831in}}{\pgfqpoint{4.977522in}{4.038478in}}%
\pgfpathcurveto{\pgfqpoint{4.982169in}{4.043125in}}{\pgfqpoint{4.984780in}{4.049428in}}{\pgfqpoint{4.984780in}{4.056000in}}%
\pgfpathcurveto{\pgfqpoint{4.984780in}{4.062572in}}{\pgfqpoint{4.982169in}{4.068875in}}{\pgfqpoint{4.977522in}{4.073522in}}%
\pgfpathcurveto{\pgfqpoint{4.972875in}{4.078169in}}{\pgfqpoint{4.966572in}{4.080780in}}{\pgfqpoint{4.960000in}{4.080780in}}%
\pgfpathcurveto{\pgfqpoint{4.953428in}{4.080780in}}{\pgfqpoint{4.947125in}{4.078169in}}{\pgfqpoint{4.942478in}{4.073522in}}%
\pgfpathcurveto{\pgfqpoint{4.937831in}{4.068875in}}{\pgfqpoint{4.935220in}{4.062572in}}{\pgfqpoint{4.935220in}{4.056000in}}%
\pgfpathcurveto{\pgfqpoint{4.935220in}{4.049428in}}{\pgfqpoint{4.937831in}{4.043125in}}{\pgfqpoint{4.942478in}{4.038478in}}%
\pgfpathcurveto{\pgfqpoint{4.947125in}{4.033831in}}{\pgfqpoint{4.953428in}{4.031220in}}{\pgfqpoint{4.960000in}{4.031220in}}%
\pgfpathclose%
\pgfusepath{stroke,fill}%
\end{pgfscope}%
\begin{pgfscope}%
\pgfpathrectangle{\pgfqpoint{1.432000in}{0.528000in}}{\pgfqpoint{3.696000in}{3.696000in}}%
\pgfusepath{clip}%
\pgfsetbuttcap%
\pgfsetroundjoin%
\pgfsetlinewidth{1.505625pt}%
\definecolor{currentstroke}{rgb}{0.000000,0.000000,0.000000}%
\pgfsetstrokecolor{currentstroke}%
\pgfsetdash{}{0pt}%
\pgfpathmoveto{\pgfqpoint{1.600000in}{0.696000in}}%
\pgfpathlineto{\pgfqpoint{1.600000in}{1.032000in}}%
\pgfpathlineto{\pgfqpoint{1.936000in}{0.696000in}}%
\pgfpathlineto{\pgfqpoint{1.600000in}{0.696000in}}%
\pgfusepath{stroke}%
\end{pgfscope}%
\begin{pgfscope}%
\pgfpathrectangle{\pgfqpoint{1.432000in}{0.528000in}}{\pgfqpoint{3.696000in}{3.696000in}}%
\pgfusepath{clip}%
\pgfsetbuttcap%
\pgfsetroundjoin%
\pgfsetlinewidth{1.505625pt}%
\definecolor{currentstroke}{rgb}{0.000000,0.000000,0.000000}%
\pgfsetstrokecolor{currentstroke}%
\pgfsetdash{}{0pt}%
\pgfpathmoveto{\pgfqpoint{1.600000in}{1.032000in}}%
\pgfpathlineto{\pgfqpoint{1.936000in}{0.696000in}}%
\pgfpathlineto{\pgfqpoint{1.936000in}{1.032000in}}%
\pgfpathlineto{\pgfqpoint{1.600000in}{1.032000in}}%
\pgfusepath{stroke}%
\end{pgfscope}%
\begin{pgfscope}%
\pgfpathrectangle{\pgfqpoint{1.432000in}{0.528000in}}{\pgfqpoint{3.696000in}{3.696000in}}%
\pgfusepath{clip}%
\pgfsetbuttcap%
\pgfsetroundjoin%
\pgfsetlinewidth{1.505625pt}%
\definecolor{currentstroke}{rgb}{0.000000,0.000000,0.000000}%
\pgfsetstrokecolor{currentstroke}%
\pgfsetdash{}{0pt}%
\pgfpathmoveto{\pgfqpoint{1.936000in}{0.696000in}}%
\pgfpathlineto{\pgfqpoint{1.936000in}{1.032000in}}%
\pgfpathlineto{\pgfqpoint{2.272000in}{0.696000in}}%
\pgfpathlineto{\pgfqpoint{1.936000in}{0.696000in}}%
\pgfusepath{stroke}%
\end{pgfscope}%
\begin{pgfscope}%
\pgfpathrectangle{\pgfqpoint{1.432000in}{0.528000in}}{\pgfqpoint{3.696000in}{3.696000in}}%
\pgfusepath{clip}%
\pgfsetbuttcap%
\pgfsetroundjoin%
\pgfsetlinewidth{1.505625pt}%
\definecolor{currentstroke}{rgb}{0.000000,0.000000,0.000000}%
\pgfsetstrokecolor{currentstroke}%
\pgfsetdash{}{0pt}%
\pgfpathmoveto{\pgfqpoint{1.600000in}{1.032000in}}%
\pgfpathlineto{\pgfqpoint{1.936000in}{1.032000in}}%
\pgfpathlineto{\pgfqpoint{1.600000in}{1.368000in}}%
\pgfpathlineto{\pgfqpoint{1.600000in}{1.032000in}}%
\pgfusepath{stroke}%
\end{pgfscope}%
\begin{pgfscope}%
\pgfpathrectangle{\pgfqpoint{1.432000in}{0.528000in}}{\pgfqpoint{3.696000in}{3.696000in}}%
\pgfusepath{clip}%
\pgfsetbuttcap%
\pgfsetroundjoin%
\pgfsetlinewidth{1.505625pt}%
\definecolor{currentstroke}{rgb}{0.000000,0.000000,0.000000}%
\pgfsetstrokecolor{currentstroke}%
\pgfsetdash{}{0pt}%
\pgfpathmoveto{\pgfqpoint{1.936000in}{1.032000in}}%
\pgfpathlineto{\pgfqpoint{2.272000in}{0.696000in}}%
\pgfpathlineto{\pgfqpoint{2.272000in}{1.032000in}}%
\pgfpathlineto{\pgfqpoint{1.936000in}{1.032000in}}%
\pgfusepath{stroke}%
\end{pgfscope}%
\begin{pgfscope}%
\pgfpathrectangle{\pgfqpoint{1.432000in}{0.528000in}}{\pgfqpoint{3.696000in}{3.696000in}}%
\pgfusepath{clip}%
\pgfsetbuttcap%
\pgfsetroundjoin%
\pgfsetlinewidth{1.505625pt}%
\definecolor{currentstroke}{rgb}{0.000000,0.000000,0.000000}%
\pgfsetstrokecolor{currentstroke}%
\pgfsetdash{}{0pt}%
\pgfpathmoveto{\pgfqpoint{1.936000in}{1.032000in}}%
\pgfpathlineto{\pgfqpoint{1.600000in}{1.368000in}}%
\pgfpathlineto{\pgfqpoint{1.936000in}{1.368000in}}%
\pgfpathlineto{\pgfqpoint{1.936000in}{1.032000in}}%
\pgfusepath{stroke}%
\end{pgfscope}%
\begin{pgfscope}%
\pgfpathrectangle{\pgfqpoint{1.432000in}{0.528000in}}{\pgfqpoint{3.696000in}{3.696000in}}%
\pgfusepath{clip}%
\pgfsetbuttcap%
\pgfsetroundjoin%
\pgfsetlinewidth{1.505625pt}%
\definecolor{currentstroke}{rgb}{0.000000,0.000000,0.000000}%
\pgfsetstrokecolor{currentstroke}%
\pgfsetdash{}{0pt}%
\pgfpathmoveto{\pgfqpoint{2.272000in}{0.696000in}}%
\pgfpathlineto{\pgfqpoint{2.272000in}{1.032000in}}%
\pgfpathlineto{\pgfqpoint{2.608000in}{0.696000in}}%
\pgfpathlineto{\pgfqpoint{2.272000in}{0.696000in}}%
\pgfusepath{stroke}%
\end{pgfscope}%
\begin{pgfscope}%
\pgfpathrectangle{\pgfqpoint{1.432000in}{0.528000in}}{\pgfqpoint{3.696000in}{3.696000in}}%
\pgfusepath{clip}%
\pgfsetbuttcap%
\pgfsetroundjoin%
\pgfsetlinewidth{1.505625pt}%
\definecolor{currentstroke}{rgb}{0.000000,0.000000,0.000000}%
\pgfsetstrokecolor{currentstroke}%
\pgfsetdash{}{0pt}%
\pgfpathmoveto{\pgfqpoint{1.936000in}{1.032000in}}%
\pgfpathlineto{\pgfqpoint{2.272000in}{1.032000in}}%
\pgfpathlineto{\pgfqpoint{1.936000in}{1.368000in}}%
\pgfpathlineto{\pgfqpoint{1.936000in}{1.032000in}}%
\pgfusepath{stroke}%
\end{pgfscope}%
\begin{pgfscope}%
\pgfpathrectangle{\pgfqpoint{1.432000in}{0.528000in}}{\pgfqpoint{3.696000in}{3.696000in}}%
\pgfusepath{clip}%
\pgfsetbuttcap%
\pgfsetroundjoin%
\pgfsetlinewidth{1.505625pt}%
\definecolor{currentstroke}{rgb}{0.000000,0.000000,0.000000}%
\pgfsetstrokecolor{currentstroke}%
\pgfsetdash{}{0pt}%
\pgfpathmoveto{\pgfqpoint{1.600000in}{1.368000in}}%
\pgfpathlineto{\pgfqpoint{1.936000in}{1.368000in}}%
\pgfpathlineto{\pgfqpoint{1.600000in}{1.704000in}}%
\pgfpathlineto{\pgfqpoint{1.600000in}{1.368000in}}%
\pgfusepath{stroke}%
\end{pgfscope}%
\begin{pgfscope}%
\pgfpathrectangle{\pgfqpoint{1.432000in}{0.528000in}}{\pgfqpoint{3.696000in}{3.696000in}}%
\pgfusepath{clip}%
\pgfsetbuttcap%
\pgfsetroundjoin%
\pgfsetlinewidth{1.505625pt}%
\definecolor{currentstroke}{rgb}{0.000000,0.000000,0.000000}%
\pgfsetstrokecolor{currentstroke}%
\pgfsetdash{}{0pt}%
\pgfpathmoveto{\pgfqpoint{2.272000in}{1.032000in}}%
\pgfpathlineto{\pgfqpoint{2.608000in}{0.696000in}}%
\pgfpathlineto{\pgfqpoint{2.608000in}{1.032000in}}%
\pgfpathlineto{\pgfqpoint{2.272000in}{1.032000in}}%
\pgfusepath{stroke}%
\end{pgfscope}%
\begin{pgfscope}%
\pgfpathrectangle{\pgfqpoint{1.432000in}{0.528000in}}{\pgfqpoint{3.696000in}{3.696000in}}%
\pgfusepath{clip}%
\pgfsetbuttcap%
\pgfsetroundjoin%
\pgfsetlinewidth{1.505625pt}%
\definecolor{currentstroke}{rgb}{0.000000,0.000000,0.000000}%
\pgfsetstrokecolor{currentstroke}%
\pgfsetdash{}{0pt}%
\pgfpathmoveto{\pgfqpoint{2.272000in}{1.032000in}}%
\pgfpathlineto{\pgfqpoint{1.936000in}{1.368000in}}%
\pgfpathlineto{\pgfqpoint{2.272000in}{1.368000in}}%
\pgfpathlineto{\pgfqpoint{2.272000in}{1.032000in}}%
\pgfusepath{stroke}%
\end{pgfscope}%
\begin{pgfscope}%
\pgfpathrectangle{\pgfqpoint{1.432000in}{0.528000in}}{\pgfqpoint{3.696000in}{3.696000in}}%
\pgfusepath{clip}%
\pgfsetbuttcap%
\pgfsetroundjoin%
\pgfsetlinewidth{1.505625pt}%
\definecolor{currentstroke}{rgb}{0.000000,0.000000,0.000000}%
\pgfsetstrokecolor{currentstroke}%
\pgfsetdash{}{0pt}%
\pgfpathmoveto{\pgfqpoint{1.936000in}{1.368000in}}%
\pgfpathlineto{\pgfqpoint{1.600000in}{1.704000in}}%
\pgfpathlineto{\pgfqpoint{1.936000in}{1.704000in}}%
\pgfpathlineto{\pgfqpoint{1.936000in}{1.368000in}}%
\pgfusepath{stroke}%
\end{pgfscope}%
\begin{pgfscope}%
\pgfpathrectangle{\pgfqpoint{1.432000in}{0.528000in}}{\pgfqpoint{3.696000in}{3.696000in}}%
\pgfusepath{clip}%
\pgfsetbuttcap%
\pgfsetroundjoin%
\pgfsetlinewidth{1.505625pt}%
\definecolor{currentstroke}{rgb}{0.000000,0.000000,0.000000}%
\pgfsetstrokecolor{currentstroke}%
\pgfsetdash{}{0pt}%
\pgfpathmoveto{\pgfqpoint{2.608000in}{0.696000in}}%
\pgfpathlineto{\pgfqpoint{2.608000in}{1.032000in}}%
\pgfpathlineto{\pgfqpoint{2.944000in}{0.696000in}}%
\pgfpathlineto{\pgfqpoint{2.608000in}{0.696000in}}%
\pgfusepath{stroke}%
\end{pgfscope}%
\begin{pgfscope}%
\pgfpathrectangle{\pgfqpoint{1.432000in}{0.528000in}}{\pgfqpoint{3.696000in}{3.696000in}}%
\pgfusepath{clip}%
\pgfsetbuttcap%
\pgfsetroundjoin%
\pgfsetlinewidth{1.505625pt}%
\definecolor{currentstroke}{rgb}{0.000000,0.000000,0.000000}%
\pgfsetstrokecolor{currentstroke}%
\pgfsetdash{}{0pt}%
\pgfpathmoveto{\pgfqpoint{2.272000in}{1.032000in}}%
\pgfpathlineto{\pgfqpoint{2.608000in}{1.032000in}}%
\pgfpathlineto{\pgfqpoint{2.272000in}{1.368000in}}%
\pgfpathlineto{\pgfqpoint{2.272000in}{1.032000in}}%
\pgfusepath{stroke}%
\end{pgfscope}%
\begin{pgfscope}%
\pgfpathrectangle{\pgfqpoint{1.432000in}{0.528000in}}{\pgfqpoint{3.696000in}{3.696000in}}%
\pgfusepath{clip}%
\pgfsetbuttcap%
\pgfsetroundjoin%
\pgfsetlinewidth{1.505625pt}%
\definecolor{currentstroke}{rgb}{0.000000,0.000000,0.000000}%
\pgfsetstrokecolor{currentstroke}%
\pgfsetdash{}{0pt}%
\pgfpathmoveto{\pgfqpoint{1.936000in}{1.368000in}}%
\pgfpathlineto{\pgfqpoint{2.272000in}{1.368000in}}%
\pgfpathlineto{\pgfqpoint{1.936000in}{1.704000in}}%
\pgfpathlineto{\pgfqpoint{1.936000in}{1.368000in}}%
\pgfusepath{stroke}%
\end{pgfscope}%
\begin{pgfscope}%
\pgfpathrectangle{\pgfqpoint{1.432000in}{0.528000in}}{\pgfqpoint{3.696000in}{3.696000in}}%
\pgfusepath{clip}%
\pgfsetbuttcap%
\pgfsetroundjoin%
\pgfsetlinewidth{1.505625pt}%
\definecolor{currentstroke}{rgb}{0.000000,0.000000,0.000000}%
\pgfsetstrokecolor{currentstroke}%
\pgfsetdash{}{0pt}%
\pgfpathmoveto{\pgfqpoint{1.600000in}{1.704000in}}%
\pgfpathlineto{\pgfqpoint{1.936000in}{1.704000in}}%
\pgfpathlineto{\pgfqpoint{1.600000in}{2.040000in}}%
\pgfpathlineto{\pgfqpoint{1.600000in}{1.704000in}}%
\pgfusepath{stroke}%
\end{pgfscope}%
\begin{pgfscope}%
\pgfpathrectangle{\pgfqpoint{1.432000in}{0.528000in}}{\pgfqpoint{3.696000in}{3.696000in}}%
\pgfusepath{clip}%
\pgfsetbuttcap%
\pgfsetroundjoin%
\pgfsetlinewidth{1.505625pt}%
\definecolor{currentstroke}{rgb}{0.000000,0.000000,0.000000}%
\pgfsetstrokecolor{currentstroke}%
\pgfsetdash{}{0pt}%
\pgfpathmoveto{\pgfqpoint{2.608000in}{1.032000in}}%
\pgfpathlineto{\pgfqpoint{2.944000in}{0.696000in}}%
\pgfpathlineto{\pgfqpoint{2.944000in}{1.032000in}}%
\pgfpathlineto{\pgfqpoint{2.608000in}{1.032000in}}%
\pgfusepath{stroke}%
\end{pgfscope}%
\begin{pgfscope}%
\pgfpathrectangle{\pgfqpoint{1.432000in}{0.528000in}}{\pgfqpoint{3.696000in}{3.696000in}}%
\pgfusepath{clip}%
\pgfsetbuttcap%
\pgfsetroundjoin%
\pgfsetlinewidth{1.505625pt}%
\definecolor{currentstroke}{rgb}{0.000000,0.000000,0.000000}%
\pgfsetstrokecolor{currentstroke}%
\pgfsetdash{}{0pt}%
\pgfpathmoveto{\pgfqpoint{2.608000in}{1.032000in}}%
\pgfpathlineto{\pgfqpoint{2.272000in}{1.368000in}}%
\pgfpathlineto{\pgfqpoint{2.608000in}{1.368000in}}%
\pgfpathlineto{\pgfqpoint{2.608000in}{1.032000in}}%
\pgfusepath{stroke}%
\end{pgfscope}%
\begin{pgfscope}%
\pgfpathrectangle{\pgfqpoint{1.432000in}{0.528000in}}{\pgfqpoint{3.696000in}{3.696000in}}%
\pgfusepath{clip}%
\pgfsetbuttcap%
\pgfsetroundjoin%
\pgfsetlinewidth{1.505625pt}%
\definecolor{currentstroke}{rgb}{0.000000,0.000000,0.000000}%
\pgfsetstrokecolor{currentstroke}%
\pgfsetdash{}{0pt}%
\pgfpathmoveto{\pgfqpoint{2.272000in}{1.368000in}}%
\pgfpathlineto{\pgfqpoint{1.936000in}{1.704000in}}%
\pgfpathlineto{\pgfqpoint{2.272000in}{1.704000in}}%
\pgfpathlineto{\pgfqpoint{2.272000in}{1.368000in}}%
\pgfusepath{stroke}%
\end{pgfscope}%
\begin{pgfscope}%
\pgfpathrectangle{\pgfqpoint{1.432000in}{0.528000in}}{\pgfqpoint{3.696000in}{3.696000in}}%
\pgfusepath{clip}%
\pgfsetbuttcap%
\pgfsetroundjoin%
\pgfsetlinewidth{1.505625pt}%
\definecolor{currentstroke}{rgb}{0.000000,0.000000,0.000000}%
\pgfsetstrokecolor{currentstroke}%
\pgfsetdash{}{0pt}%
\pgfpathmoveto{\pgfqpoint{1.936000in}{1.704000in}}%
\pgfpathlineto{\pgfqpoint{1.600000in}{2.040000in}}%
\pgfpathlineto{\pgfqpoint{1.936000in}{2.040000in}}%
\pgfpathlineto{\pgfqpoint{1.936000in}{1.704000in}}%
\pgfusepath{stroke}%
\end{pgfscope}%
\begin{pgfscope}%
\pgfpathrectangle{\pgfqpoint{1.432000in}{0.528000in}}{\pgfqpoint{3.696000in}{3.696000in}}%
\pgfusepath{clip}%
\pgfsetbuttcap%
\pgfsetroundjoin%
\pgfsetlinewidth{1.505625pt}%
\definecolor{currentstroke}{rgb}{0.000000,0.000000,0.000000}%
\pgfsetstrokecolor{currentstroke}%
\pgfsetdash{}{0pt}%
\pgfpathmoveto{\pgfqpoint{2.944000in}{0.696000in}}%
\pgfpathlineto{\pgfqpoint{2.944000in}{1.032000in}}%
\pgfpathlineto{\pgfqpoint{3.280000in}{0.696000in}}%
\pgfpathlineto{\pgfqpoint{2.944000in}{0.696000in}}%
\pgfusepath{stroke}%
\end{pgfscope}%
\begin{pgfscope}%
\pgfpathrectangle{\pgfqpoint{1.432000in}{0.528000in}}{\pgfqpoint{3.696000in}{3.696000in}}%
\pgfusepath{clip}%
\pgfsetbuttcap%
\pgfsetroundjoin%
\pgfsetlinewidth{1.505625pt}%
\definecolor{currentstroke}{rgb}{0.000000,0.000000,0.000000}%
\pgfsetstrokecolor{currentstroke}%
\pgfsetdash{}{0pt}%
\pgfpathmoveto{\pgfqpoint{2.608000in}{1.032000in}}%
\pgfpathlineto{\pgfqpoint{2.944000in}{1.032000in}}%
\pgfpathlineto{\pgfqpoint{2.608000in}{1.368000in}}%
\pgfpathlineto{\pgfqpoint{2.608000in}{1.032000in}}%
\pgfusepath{stroke}%
\end{pgfscope}%
\begin{pgfscope}%
\pgfpathrectangle{\pgfqpoint{1.432000in}{0.528000in}}{\pgfqpoint{3.696000in}{3.696000in}}%
\pgfusepath{clip}%
\pgfsetbuttcap%
\pgfsetroundjoin%
\pgfsetlinewidth{1.505625pt}%
\definecolor{currentstroke}{rgb}{0.000000,0.000000,0.000000}%
\pgfsetstrokecolor{currentstroke}%
\pgfsetdash{}{0pt}%
\pgfpathmoveto{\pgfqpoint{2.272000in}{1.368000in}}%
\pgfpathlineto{\pgfqpoint{2.608000in}{1.368000in}}%
\pgfpathlineto{\pgfqpoint{2.272000in}{1.704000in}}%
\pgfpathlineto{\pgfqpoint{2.272000in}{1.368000in}}%
\pgfusepath{stroke}%
\end{pgfscope}%
\begin{pgfscope}%
\pgfpathrectangle{\pgfqpoint{1.432000in}{0.528000in}}{\pgfqpoint{3.696000in}{3.696000in}}%
\pgfusepath{clip}%
\pgfsetbuttcap%
\pgfsetroundjoin%
\pgfsetlinewidth{1.505625pt}%
\definecolor{currentstroke}{rgb}{0.000000,0.000000,0.000000}%
\pgfsetstrokecolor{currentstroke}%
\pgfsetdash{}{0pt}%
\pgfpathmoveto{\pgfqpoint{1.936000in}{1.704000in}}%
\pgfpathlineto{\pgfqpoint{2.272000in}{1.704000in}}%
\pgfpathlineto{\pgfqpoint{1.936000in}{2.040000in}}%
\pgfpathlineto{\pgfqpoint{1.936000in}{1.704000in}}%
\pgfusepath{stroke}%
\end{pgfscope}%
\begin{pgfscope}%
\pgfpathrectangle{\pgfqpoint{1.432000in}{0.528000in}}{\pgfqpoint{3.696000in}{3.696000in}}%
\pgfusepath{clip}%
\pgfsetbuttcap%
\pgfsetroundjoin%
\pgfsetlinewidth{1.505625pt}%
\definecolor{currentstroke}{rgb}{0.000000,0.000000,0.000000}%
\pgfsetstrokecolor{currentstroke}%
\pgfsetdash{}{0pt}%
\pgfpathmoveto{\pgfqpoint{1.600000in}{2.040000in}}%
\pgfpathlineto{\pgfqpoint{1.936000in}{2.040000in}}%
\pgfpathlineto{\pgfqpoint{1.600000in}{2.376000in}}%
\pgfpathlineto{\pgfqpoint{1.600000in}{2.040000in}}%
\pgfusepath{stroke}%
\end{pgfscope}%
\begin{pgfscope}%
\pgfpathrectangle{\pgfqpoint{1.432000in}{0.528000in}}{\pgfqpoint{3.696000in}{3.696000in}}%
\pgfusepath{clip}%
\pgfsetbuttcap%
\pgfsetroundjoin%
\pgfsetlinewidth{1.505625pt}%
\definecolor{currentstroke}{rgb}{0.000000,0.000000,0.000000}%
\pgfsetstrokecolor{currentstroke}%
\pgfsetdash{}{0pt}%
\pgfpathmoveto{\pgfqpoint{2.944000in}{1.032000in}}%
\pgfpathlineto{\pgfqpoint{3.280000in}{0.696000in}}%
\pgfpathlineto{\pgfqpoint{3.280000in}{1.032000in}}%
\pgfpathlineto{\pgfqpoint{2.944000in}{1.032000in}}%
\pgfusepath{stroke}%
\end{pgfscope}%
\begin{pgfscope}%
\pgfpathrectangle{\pgfqpoint{1.432000in}{0.528000in}}{\pgfqpoint{3.696000in}{3.696000in}}%
\pgfusepath{clip}%
\pgfsetbuttcap%
\pgfsetroundjoin%
\pgfsetlinewidth{1.505625pt}%
\definecolor{currentstroke}{rgb}{0.000000,0.000000,0.000000}%
\pgfsetstrokecolor{currentstroke}%
\pgfsetdash{}{0pt}%
\pgfpathmoveto{\pgfqpoint{2.944000in}{1.032000in}}%
\pgfpathlineto{\pgfqpoint{2.608000in}{1.368000in}}%
\pgfpathlineto{\pgfqpoint{2.944000in}{1.368000in}}%
\pgfpathlineto{\pgfqpoint{2.944000in}{1.032000in}}%
\pgfusepath{stroke}%
\end{pgfscope}%
\begin{pgfscope}%
\pgfpathrectangle{\pgfqpoint{1.432000in}{0.528000in}}{\pgfqpoint{3.696000in}{3.696000in}}%
\pgfusepath{clip}%
\pgfsetbuttcap%
\pgfsetroundjoin%
\pgfsetlinewidth{1.505625pt}%
\definecolor{currentstroke}{rgb}{0.000000,0.000000,0.000000}%
\pgfsetstrokecolor{currentstroke}%
\pgfsetdash{}{0pt}%
\pgfpathmoveto{\pgfqpoint{2.608000in}{1.368000in}}%
\pgfpathlineto{\pgfqpoint{2.272000in}{1.704000in}}%
\pgfpathlineto{\pgfqpoint{2.608000in}{1.704000in}}%
\pgfpathlineto{\pgfqpoint{2.608000in}{1.368000in}}%
\pgfusepath{stroke}%
\end{pgfscope}%
\begin{pgfscope}%
\pgfpathrectangle{\pgfqpoint{1.432000in}{0.528000in}}{\pgfqpoint{3.696000in}{3.696000in}}%
\pgfusepath{clip}%
\pgfsetbuttcap%
\pgfsetroundjoin%
\pgfsetlinewidth{1.505625pt}%
\definecolor{currentstroke}{rgb}{0.000000,0.000000,0.000000}%
\pgfsetstrokecolor{currentstroke}%
\pgfsetdash{}{0pt}%
\pgfpathmoveto{\pgfqpoint{2.272000in}{1.704000in}}%
\pgfpathlineto{\pgfqpoint{1.936000in}{2.040000in}}%
\pgfpathlineto{\pgfqpoint{2.272000in}{2.040000in}}%
\pgfpathlineto{\pgfqpoint{2.272000in}{1.704000in}}%
\pgfusepath{stroke}%
\end{pgfscope}%
\begin{pgfscope}%
\pgfpathrectangle{\pgfqpoint{1.432000in}{0.528000in}}{\pgfqpoint{3.696000in}{3.696000in}}%
\pgfusepath{clip}%
\pgfsetbuttcap%
\pgfsetroundjoin%
\pgfsetlinewidth{1.505625pt}%
\definecolor{currentstroke}{rgb}{0.000000,0.000000,0.000000}%
\pgfsetstrokecolor{currentstroke}%
\pgfsetdash{}{0pt}%
\pgfpathmoveto{\pgfqpoint{1.936000in}{2.040000in}}%
\pgfpathlineto{\pgfqpoint{1.600000in}{2.376000in}}%
\pgfpathlineto{\pgfqpoint{1.936000in}{2.376000in}}%
\pgfpathlineto{\pgfqpoint{1.936000in}{2.040000in}}%
\pgfusepath{stroke}%
\end{pgfscope}%
\begin{pgfscope}%
\pgfpathrectangle{\pgfqpoint{1.432000in}{0.528000in}}{\pgfqpoint{3.696000in}{3.696000in}}%
\pgfusepath{clip}%
\pgfsetbuttcap%
\pgfsetroundjoin%
\pgfsetlinewidth{1.505625pt}%
\definecolor{currentstroke}{rgb}{0.000000,0.000000,0.000000}%
\pgfsetstrokecolor{currentstroke}%
\pgfsetdash{}{0pt}%
\pgfpathmoveto{\pgfqpoint{3.280000in}{0.696000in}}%
\pgfpathlineto{\pgfqpoint{3.280000in}{1.032000in}}%
\pgfpathlineto{\pgfqpoint{3.616000in}{0.696000in}}%
\pgfpathlineto{\pgfqpoint{3.280000in}{0.696000in}}%
\pgfusepath{stroke}%
\end{pgfscope}%
\begin{pgfscope}%
\pgfpathrectangle{\pgfqpoint{1.432000in}{0.528000in}}{\pgfqpoint{3.696000in}{3.696000in}}%
\pgfusepath{clip}%
\pgfsetbuttcap%
\pgfsetroundjoin%
\pgfsetlinewidth{1.505625pt}%
\definecolor{currentstroke}{rgb}{0.000000,0.000000,0.000000}%
\pgfsetstrokecolor{currentstroke}%
\pgfsetdash{}{0pt}%
\pgfpathmoveto{\pgfqpoint{2.944000in}{1.032000in}}%
\pgfpathlineto{\pgfqpoint{3.280000in}{1.032000in}}%
\pgfpathlineto{\pgfqpoint{2.944000in}{1.368000in}}%
\pgfpathlineto{\pgfqpoint{2.944000in}{1.032000in}}%
\pgfusepath{stroke}%
\end{pgfscope}%
\begin{pgfscope}%
\pgfpathrectangle{\pgfqpoint{1.432000in}{0.528000in}}{\pgfqpoint{3.696000in}{3.696000in}}%
\pgfusepath{clip}%
\pgfsetbuttcap%
\pgfsetroundjoin%
\pgfsetlinewidth{1.505625pt}%
\definecolor{currentstroke}{rgb}{0.000000,0.000000,0.000000}%
\pgfsetstrokecolor{currentstroke}%
\pgfsetdash{}{0pt}%
\pgfpathmoveto{\pgfqpoint{2.608000in}{1.368000in}}%
\pgfpathlineto{\pgfqpoint{2.944000in}{1.368000in}}%
\pgfpathlineto{\pgfqpoint{2.608000in}{1.704000in}}%
\pgfpathlineto{\pgfqpoint{2.608000in}{1.368000in}}%
\pgfusepath{stroke}%
\end{pgfscope}%
\begin{pgfscope}%
\pgfpathrectangle{\pgfqpoint{1.432000in}{0.528000in}}{\pgfqpoint{3.696000in}{3.696000in}}%
\pgfusepath{clip}%
\pgfsetbuttcap%
\pgfsetroundjoin%
\pgfsetlinewidth{1.505625pt}%
\definecolor{currentstroke}{rgb}{0.000000,0.000000,0.000000}%
\pgfsetstrokecolor{currentstroke}%
\pgfsetdash{}{0pt}%
\pgfpathmoveto{\pgfqpoint{2.272000in}{1.704000in}}%
\pgfpathlineto{\pgfqpoint{2.608000in}{1.704000in}}%
\pgfpathlineto{\pgfqpoint{2.272000in}{2.040000in}}%
\pgfpathlineto{\pgfqpoint{2.272000in}{1.704000in}}%
\pgfusepath{stroke}%
\end{pgfscope}%
\begin{pgfscope}%
\pgfpathrectangle{\pgfqpoint{1.432000in}{0.528000in}}{\pgfqpoint{3.696000in}{3.696000in}}%
\pgfusepath{clip}%
\pgfsetbuttcap%
\pgfsetroundjoin%
\pgfsetlinewidth{1.505625pt}%
\definecolor{currentstroke}{rgb}{0.000000,0.000000,0.000000}%
\pgfsetstrokecolor{currentstroke}%
\pgfsetdash{}{0pt}%
\pgfpathmoveto{\pgfqpoint{1.936000in}{2.040000in}}%
\pgfpathlineto{\pgfqpoint{2.272000in}{2.040000in}}%
\pgfpathlineto{\pgfqpoint{1.936000in}{2.376000in}}%
\pgfpathlineto{\pgfqpoint{1.936000in}{2.040000in}}%
\pgfusepath{stroke}%
\end{pgfscope}%
\begin{pgfscope}%
\pgfpathrectangle{\pgfqpoint{1.432000in}{0.528000in}}{\pgfqpoint{3.696000in}{3.696000in}}%
\pgfusepath{clip}%
\pgfsetbuttcap%
\pgfsetroundjoin%
\pgfsetlinewidth{1.505625pt}%
\definecolor{currentstroke}{rgb}{0.000000,0.000000,0.000000}%
\pgfsetstrokecolor{currentstroke}%
\pgfsetdash{}{0pt}%
\pgfpathmoveto{\pgfqpoint{1.600000in}{2.376000in}}%
\pgfpathlineto{\pgfqpoint{1.936000in}{2.376000in}}%
\pgfpathlineto{\pgfqpoint{1.600000in}{2.712000in}}%
\pgfpathlineto{\pgfqpoint{1.600000in}{2.376000in}}%
\pgfusepath{stroke}%
\end{pgfscope}%
\begin{pgfscope}%
\pgfpathrectangle{\pgfqpoint{1.432000in}{0.528000in}}{\pgfqpoint{3.696000in}{3.696000in}}%
\pgfusepath{clip}%
\pgfsetbuttcap%
\pgfsetroundjoin%
\pgfsetlinewidth{1.505625pt}%
\definecolor{currentstroke}{rgb}{0.000000,0.000000,0.000000}%
\pgfsetstrokecolor{currentstroke}%
\pgfsetdash{}{0pt}%
\pgfpathmoveto{\pgfqpoint{3.280000in}{1.032000in}}%
\pgfpathlineto{\pgfqpoint{3.616000in}{0.696000in}}%
\pgfpathlineto{\pgfqpoint{3.616000in}{1.032000in}}%
\pgfpathlineto{\pgfqpoint{3.280000in}{1.032000in}}%
\pgfusepath{stroke}%
\end{pgfscope}%
\begin{pgfscope}%
\pgfpathrectangle{\pgfqpoint{1.432000in}{0.528000in}}{\pgfqpoint{3.696000in}{3.696000in}}%
\pgfusepath{clip}%
\pgfsetbuttcap%
\pgfsetroundjoin%
\pgfsetlinewidth{1.505625pt}%
\definecolor{currentstroke}{rgb}{0.000000,0.000000,0.000000}%
\pgfsetstrokecolor{currentstroke}%
\pgfsetdash{}{0pt}%
\pgfpathmoveto{\pgfqpoint{3.280000in}{1.032000in}}%
\pgfpathlineto{\pgfqpoint{2.944000in}{1.368000in}}%
\pgfpathlineto{\pgfqpoint{3.280000in}{1.368000in}}%
\pgfpathlineto{\pgfqpoint{3.280000in}{1.032000in}}%
\pgfusepath{stroke}%
\end{pgfscope}%
\begin{pgfscope}%
\pgfpathrectangle{\pgfqpoint{1.432000in}{0.528000in}}{\pgfqpoint{3.696000in}{3.696000in}}%
\pgfusepath{clip}%
\pgfsetbuttcap%
\pgfsetroundjoin%
\pgfsetlinewidth{1.505625pt}%
\definecolor{currentstroke}{rgb}{0.000000,0.000000,0.000000}%
\pgfsetstrokecolor{currentstroke}%
\pgfsetdash{}{0pt}%
\pgfpathmoveto{\pgfqpoint{2.944000in}{1.368000in}}%
\pgfpathlineto{\pgfqpoint{2.608000in}{1.704000in}}%
\pgfpathlineto{\pgfqpoint{2.944000in}{1.704000in}}%
\pgfpathlineto{\pgfqpoint{2.944000in}{1.368000in}}%
\pgfusepath{stroke}%
\end{pgfscope}%
\begin{pgfscope}%
\pgfpathrectangle{\pgfqpoint{1.432000in}{0.528000in}}{\pgfqpoint{3.696000in}{3.696000in}}%
\pgfusepath{clip}%
\pgfsetbuttcap%
\pgfsetroundjoin%
\pgfsetlinewidth{1.505625pt}%
\definecolor{currentstroke}{rgb}{0.000000,0.000000,0.000000}%
\pgfsetstrokecolor{currentstroke}%
\pgfsetdash{}{0pt}%
\pgfpathmoveto{\pgfqpoint{2.608000in}{1.704000in}}%
\pgfpathlineto{\pgfqpoint{2.272000in}{2.040000in}}%
\pgfpathlineto{\pgfqpoint{2.608000in}{2.040000in}}%
\pgfpathlineto{\pgfqpoint{2.608000in}{1.704000in}}%
\pgfusepath{stroke}%
\end{pgfscope}%
\begin{pgfscope}%
\pgfpathrectangle{\pgfqpoint{1.432000in}{0.528000in}}{\pgfqpoint{3.696000in}{3.696000in}}%
\pgfusepath{clip}%
\pgfsetbuttcap%
\pgfsetroundjoin%
\pgfsetlinewidth{1.505625pt}%
\definecolor{currentstroke}{rgb}{0.000000,0.000000,0.000000}%
\pgfsetstrokecolor{currentstroke}%
\pgfsetdash{}{0pt}%
\pgfpathmoveto{\pgfqpoint{2.272000in}{2.040000in}}%
\pgfpathlineto{\pgfqpoint{1.936000in}{2.376000in}}%
\pgfpathlineto{\pgfqpoint{2.272000in}{2.376000in}}%
\pgfpathlineto{\pgfqpoint{2.272000in}{2.040000in}}%
\pgfusepath{stroke}%
\end{pgfscope}%
\begin{pgfscope}%
\pgfpathrectangle{\pgfqpoint{1.432000in}{0.528000in}}{\pgfqpoint{3.696000in}{3.696000in}}%
\pgfusepath{clip}%
\pgfsetbuttcap%
\pgfsetroundjoin%
\pgfsetlinewidth{1.505625pt}%
\definecolor{currentstroke}{rgb}{0.000000,0.000000,0.000000}%
\pgfsetstrokecolor{currentstroke}%
\pgfsetdash{}{0pt}%
\pgfpathmoveto{\pgfqpoint{1.936000in}{2.376000in}}%
\pgfpathlineto{\pgfqpoint{1.600000in}{2.712000in}}%
\pgfpathlineto{\pgfqpoint{1.936000in}{2.712000in}}%
\pgfpathlineto{\pgfqpoint{1.936000in}{2.376000in}}%
\pgfusepath{stroke}%
\end{pgfscope}%
\begin{pgfscope}%
\pgfpathrectangle{\pgfqpoint{1.432000in}{0.528000in}}{\pgfqpoint{3.696000in}{3.696000in}}%
\pgfusepath{clip}%
\pgfsetbuttcap%
\pgfsetroundjoin%
\pgfsetlinewidth{1.505625pt}%
\definecolor{currentstroke}{rgb}{0.000000,0.000000,0.000000}%
\pgfsetstrokecolor{currentstroke}%
\pgfsetdash{}{0pt}%
\pgfpathmoveto{\pgfqpoint{3.616000in}{0.696000in}}%
\pgfpathlineto{\pgfqpoint{3.616000in}{1.032000in}}%
\pgfpathlineto{\pgfqpoint{3.952000in}{0.696000in}}%
\pgfpathlineto{\pgfqpoint{3.616000in}{0.696000in}}%
\pgfusepath{stroke}%
\end{pgfscope}%
\begin{pgfscope}%
\pgfpathrectangle{\pgfqpoint{1.432000in}{0.528000in}}{\pgfqpoint{3.696000in}{3.696000in}}%
\pgfusepath{clip}%
\pgfsetbuttcap%
\pgfsetroundjoin%
\pgfsetlinewidth{1.505625pt}%
\definecolor{currentstroke}{rgb}{0.000000,0.000000,0.000000}%
\pgfsetstrokecolor{currentstroke}%
\pgfsetdash{}{0pt}%
\pgfpathmoveto{\pgfqpoint{3.280000in}{1.032000in}}%
\pgfpathlineto{\pgfqpoint{3.616000in}{1.032000in}}%
\pgfpathlineto{\pgfqpoint{3.280000in}{1.368000in}}%
\pgfpathlineto{\pgfqpoint{3.280000in}{1.032000in}}%
\pgfusepath{stroke}%
\end{pgfscope}%
\begin{pgfscope}%
\pgfpathrectangle{\pgfqpoint{1.432000in}{0.528000in}}{\pgfqpoint{3.696000in}{3.696000in}}%
\pgfusepath{clip}%
\pgfsetbuttcap%
\pgfsetroundjoin%
\pgfsetlinewidth{1.505625pt}%
\definecolor{currentstroke}{rgb}{0.000000,0.000000,0.000000}%
\pgfsetstrokecolor{currentstroke}%
\pgfsetdash{}{0pt}%
\pgfpathmoveto{\pgfqpoint{2.944000in}{1.368000in}}%
\pgfpathlineto{\pgfqpoint{3.280000in}{1.368000in}}%
\pgfpathlineto{\pgfqpoint{2.944000in}{1.704000in}}%
\pgfpathlineto{\pgfqpoint{2.944000in}{1.368000in}}%
\pgfusepath{stroke}%
\end{pgfscope}%
\begin{pgfscope}%
\pgfpathrectangle{\pgfqpoint{1.432000in}{0.528000in}}{\pgfqpoint{3.696000in}{3.696000in}}%
\pgfusepath{clip}%
\pgfsetbuttcap%
\pgfsetroundjoin%
\pgfsetlinewidth{1.505625pt}%
\definecolor{currentstroke}{rgb}{0.000000,0.000000,0.000000}%
\pgfsetstrokecolor{currentstroke}%
\pgfsetdash{}{0pt}%
\pgfpathmoveto{\pgfqpoint{2.608000in}{1.704000in}}%
\pgfpathlineto{\pgfqpoint{2.944000in}{1.704000in}}%
\pgfpathlineto{\pgfqpoint{2.608000in}{2.040000in}}%
\pgfpathlineto{\pgfqpoint{2.608000in}{1.704000in}}%
\pgfusepath{stroke}%
\end{pgfscope}%
\begin{pgfscope}%
\pgfpathrectangle{\pgfqpoint{1.432000in}{0.528000in}}{\pgfqpoint{3.696000in}{3.696000in}}%
\pgfusepath{clip}%
\pgfsetbuttcap%
\pgfsetroundjoin%
\pgfsetlinewidth{1.505625pt}%
\definecolor{currentstroke}{rgb}{0.000000,0.000000,0.000000}%
\pgfsetstrokecolor{currentstroke}%
\pgfsetdash{}{0pt}%
\pgfpathmoveto{\pgfqpoint{2.272000in}{2.040000in}}%
\pgfpathlineto{\pgfqpoint{2.608000in}{2.040000in}}%
\pgfpathlineto{\pgfqpoint{2.272000in}{2.376000in}}%
\pgfpathlineto{\pgfqpoint{2.272000in}{2.040000in}}%
\pgfusepath{stroke}%
\end{pgfscope}%
\begin{pgfscope}%
\pgfpathrectangle{\pgfqpoint{1.432000in}{0.528000in}}{\pgfqpoint{3.696000in}{3.696000in}}%
\pgfusepath{clip}%
\pgfsetbuttcap%
\pgfsetroundjoin%
\pgfsetlinewidth{1.505625pt}%
\definecolor{currentstroke}{rgb}{0.000000,0.000000,0.000000}%
\pgfsetstrokecolor{currentstroke}%
\pgfsetdash{}{0pt}%
\pgfpathmoveto{\pgfqpoint{1.936000in}{2.376000in}}%
\pgfpathlineto{\pgfqpoint{2.272000in}{2.376000in}}%
\pgfpathlineto{\pgfqpoint{1.936000in}{2.712000in}}%
\pgfpathlineto{\pgfqpoint{1.936000in}{2.376000in}}%
\pgfusepath{stroke}%
\end{pgfscope}%
\begin{pgfscope}%
\pgfpathrectangle{\pgfqpoint{1.432000in}{0.528000in}}{\pgfqpoint{3.696000in}{3.696000in}}%
\pgfusepath{clip}%
\pgfsetbuttcap%
\pgfsetroundjoin%
\pgfsetlinewidth{1.505625pt}%
\definecolor{currentstroke}{rgb}{0.000000,0.000000,0.000000}%
\pgfsetstrokecolor{currentstroke}%
\pgfsetdash{}{0pt}%
\pgfpathmoveto{\pgfqpoint{1.600000in}{2.712000in}}%
\pgfpathlineto{\pgfqpoint{1.936000in}{2.712000in}}%
\pgfpathlineto{\pgfqpoint{1.600000in}{3.048000in}}%
\pgfpathlineto{\pgfqpoint{1.600000in}{2.712000in}}%
\pgfusepath{stroke}%
\end{pgfscope}%
\begin{pgfscope}%
\pgfpathrectangle{\pgfqpoint{1.432000in}{0.528000in}}{\pgfqpoint{3.696000in}{3.696000in}}%
\pgfusepath{clip}%
\pgfsetbuttcap%
\pgfsetroundjoin%
\pgfsetlinewidth{1.505625pt}%
\definecolor{currentstroke}{rgb}{0.000000,0.000000,0.000000}%
\pgfsetstrokecolor{currentstroke}%
\pgfsetdash{}{0pt}%
\pgfpathmoveto{\pgfqpoint{3.616000in}{1.032000in}}%
\pgfpathlineto{\pgfqpoint{3.952000in}{0.696000in}}%
\pgfpathlineto{\pgfqpoint{3.952000in}{1.032000in}}%
\pgfpathlineto{\pgfqpoint{3.616000in}{1.032000in}}%
\pgfusepath{stroke}%
\end{pgfscope}%
\begin{pgfscope}%
\pgfpathrectangle{\pgfqpoint{1.432000in}{0.528000in}}{\pgfqpoint{3.696000in}{3.696000in}}%
\pgfusepath{clip}%
\pgfsetbuttcap%
\pgfsetroundjoin%
\pgfsetlinewidth{1.505625pt}%
\definecolor{currentstroke}{rgb}{0.000000,0.000000,0.000000}%
\pgfsetstrokecolor{currentstroke}%
\pgfsetdash{}{0pt}%
\pgfpathmoveto{\pgfqpoint{3.616000in}{1.032000in}}%
\pgfpathlineto{\pgfqpoint{3.280000in}{1.368000in}}%
\pgfpathlineto{\pgfqpoint{3.616000in}{1.368000in}}%
\pgfpathlineto{\pgfqpoint{3.616000in}{1.032000in}}%
\pgfusepath{stroke}%
\end{pgfscope}%
\begin{pgfscope}%
\pgfpathrectangle{\pgfqpoint{1.432000in}{0.528000in}}{\pgfqpoint{3.696000in}{3.696000in}}%
\pgfusepath{clip}%
\pgfsetbuttcap%
\pgfsetroundjoin%
\pgfsetlinewidth{1.505625pt}%
\definecolor{currentstroke}{rgb}{0.000000,0.000000,0.000000}%
\pgfsetstrokecolor{currentstroke}%
\pgfsetdash{}{0pt}%
\pgfpathmoveto{\pgfqpoint{3.280000in}{1.368000in}}%
\pgfpathlineto{\pgfqpoint{2.944000in}{1.704000in}}%
\pgfpathlineto{\pgfqpoint{3.280000in}{1.704000in}}%
\pgfpathlineto{\pgfqpoint{3.280000in}{1.368000in}}%
\pgfusepath{stroke}%
\end{pgfscope}%
\begin{pgfscope}%
\pgfpathrectangle{\pgfqpoint{1.432000in}{0.528000in}}{\pgfqpoint{3.696000in}{3.696000in}}%
\pgfusepath{clip}%
\pgfsetbuttcap%
\pgfsetroundjoin%
\pgfsetlinewidth{1.505625pt}%
\definecolor{currentstroke}{rgb}{0.000000,0.000000,0.000000}%
\pgfsetstrokecolor{currentstroke}%
\pgfsetdash{}{0pt}%
\pgfpathmoveto{\pgfqpoint{2.944000in}{1.704000in}}%
\pgfpathlineto{\pgfqpoint{2.608000in}{2.040000in}}%
\pgfpathlineto{\pgfqpoint{2.944000in}{2.040000in}}%
\pgfpathlineto{\pgfqpoint{2.944000in}{1.704000in}}%
\pgfusepath{stroke}%
\end{pgfscope}%
\begin{pgfscope}%
\pgfpathrectangle{\pgfqpoint{1.432000in}{0.528000in}}{\pgfqpoint{3.696000in}{3.696000in}}%
\pgfusepath{clip}%
\pgfsetbuttcap%
\pgfsetroundjoin%
\pgfsetlinewidth{1.505625pt}%
\definecolor{currentstroke}{rgb}{0.000000,0.000000,0.000000}%
\pgfsetstrokecolor{currentstroke}%
\pgfsetdash{}{0pt}%
\pgfpathmoveto{\pgfqpoint{2.608000in}{2.040000in}}%
\pgfpathlineto{\pgfqpoint{2.272000in}{2.376000in}}%
\pgfpathlineto{\pgfqpoint{2.608000in}{2.376000in}}%
\pgfpathlineto{\pgfqpoint{2.608000in}{2.040000in}}%
\pgfusepath{stroke}%
\end{pgfscope}%
\begin{pgfscope}%
\pgfpathrectangle{\pgfqpoint{1.432000in}{0.528000in}}{\pgfqpoint{3.696000in}{3.696000in}}%
\pgfusepath{clip}%
\pgfsetbuttcap%
\pgfsetroundjoin%
\pgfsetlinewidth{1.505625pt}%
\definecolor{currentstroke}{rgb}{0.000000,0.000000,0.000000}%
\pgfsetstrokecolor{currentstroke}%
\pgfsetdash{}{0pt}%
\pgfpathmoveto{\pgfqpoint{2.272000in}{2.376000in}}%
\pgfpathlineto{\pgfqpoint{1.936000in}{2.712000in}}%
\pgfpathlineto{\pgfqpoint{2.272000in}{2.712000in}}%
\pgfpathlineto{\pgfqpoint{2.272000in}{2.376000in}}%
\pgfusepath{stroke}%
\end{pgfscope}%
\begin{pgfscope}%
\pgfpathrectangle{\pgfqpoint{1.432000in}{0.528000in}}{\pgfqpoint{3.696000in}{3.696000in}}%
\pgfusepath{clip}%
\pgfsetbuttcap%
\pgfsetroundjoin%
\pgfsetlinewidth{1.505625pt}%
\definecolor{currentstroke}{rgb}{0.000000,0.000000,0.000000}%
\pgfsetstrokecolor{currentstroke}%
\pgfsetdash{}{0pt}%
\pgfpathmoveto{\pgfqpoint{1.936000in}{2.712000in}}%
\pgfpathlineto{\pgfqpoint{1.600000in}{3.048000in}}%
\pgfpathlineto{\pgfqpoint{1.936000in}{3.048000in}}%
\pgfpathlineto{\pgfqpoint{1.936000in}{2.712000in}}%
\pgfusepath{stroke}%
\end{pgfscope}%
\begin{pgfscope}%
\pgfpathrectangle{\pgfqpoint{1.432000in}{0.528000in}}{\pgfqpoint{3.696000in}{3.696000in}}%
\pgfusepath{clip}%
\pgfsetbuttcap%
\pgfsetroundjoin%
\pgfsetlinewidth{1.505625pt}%
\definecolor{currentstroke}{rgb}{0.000000,0.000000,0.000000}%
\pgfsetstrokecolor{currentstroke}%
\pgfsetdash{}{0pt}%
\pgfpathmoveto{\pgfqpoint{3.952000in}{0.696000in}}%
\pgfpathlineto{\pgfqpoint{3.952000in}{1.032000in}}%
\pgfpathlineto{\pgfqpoint{4.288000in}{0.696000in}}%
\pgfpathlineto{\pgfqpoint{3.952000in}{0.696000in}}%
\pgfusepath{stroke}%
\end{pgfscope}%
\begin{pgfscope}%
\pgfpathrectangle{\pgfqpoint{1.432000in}{0.528000in}}{\pgfqpoint{3.696000in}{3.696000in}}%
\pgfusepath{clip}%
\pgfsetbuttcap%
\pgfsetroundjoin%
\pgfsetlinewidth{1.505625pt}%
\definecolor{currentstroke}{rgb}{0.000000,0.000000,0.000000}%
\pgfsetstrokecolor{currentstroke}%
\pgfsetdash{}{0pt}%
\pgfpathmoveto{\pgfqpoint{3.616000in}{1.032000in}}%
\pgfpathlineto{\pgfqpoint{3.952000in}{1.032000in}}%
\pgfpathlineto{\pgfqpoint{3.616000in}{1.368000in}}%
\pgfpathlineto{\pgfqpoint{3.616000in}{1.032000in}}%
\pgfusepath{stroke}%
\end{pgfscope}%
\begin{pgfscope}%
\pgfpathrectangle{\pgfqpoint{1.432000in}{0.528000in}}{\pgfqpoint{3.696000in}{3.696000in}}%
\pgfusepath{clip}%
\pgfsetbuttcap%
\pgfsetroundjoin%
\pgfsetlinewidth{1.505625pt}%
\definecolor{currentstroke}{rgb}{0.000000,0.000000,0.000000}%
\pgfsetstrokecolor{currentstroke}%
\pgfsetdash{}{0pt}%
\pgfpathmoveto{\pgfqpoint{3.280000in}{1.368000in}}%
\pgfpathlineto{\pgfqpoint{3.616000in}{1.368000in}}%
\pgfpathlineto{\pgfqpoint{3.280000in}{1.704000in}}%
\pgfpathlineto{\pgfqpoint{3.280000in}{1.368000in}}%
\pgfusepath{stroke}%
\end{pgfscope}%
\begin{pgfscope}%
\pgfpathrectangle{\pgfqpoint{1.432000in}{0.528000in}}{\pgfqpoint{3.696000in}{3.696000in}}%
\pgfusepath{clip}%
\pgfsetbuttcap%
\pgfsetroundjoin%
\pgfsetlinewidth{1.505625pt}%
\definecolor{currentstroke}{rgb}{0.000000,0.000000,0.000000}%
\pgfsetstrokecolor{currentstroke}%
\pgfsetdash{}{0pt}%
\pgfpathmoveto{\pgfqpoint{2.944000in}{1.704000in}}%
\pgfpathlineto{\pgfqpoint{3.280000in}{1.704000in}}%
\pgfpathlineto{\pgfqpoint{2.944000in}{2.040000in}}%
\pgfpathlineto{\pgfqpoint{2.944000in}{1.704000in}}%
\pgfusepath{stroke}%
\end{pgfscope}%
\begin{pgfscope}%
\pgfpathrectangle{\pgfqpoint{1.432000in}{0.528000in}}{\pgfqpoint{3.696000in}{3.696000in}}%
\pgfusepath{clip}%
\pgfsetbuttcap%
\pgfsetroundjoin%
\pgfsetlinewidth{1.505625pt}%
\definecolor{currentstroke}{rgb}{0.000000,0.000000,0.000000}%
\pgfsetstrokecolor{currentstroke}%
\pgfsetdash{}{0pt}%
\pgfpathmoveto{\pgfqpoint{2.608000in}{2.040000in}}%
\pgfpathlineto{\pgfqpoint{2.944000in}{2.040000in}}%
\pgfpathlineto{\pgfqpoint{2.608000in}{2.376000in}}%
\pgfpathlineto{\pgfqpoint{2.608000in}{2.040000in}}%
\pgfusepath{stroke}%
\end{pgfscope}%
\begin{pgfscope}%
\pgfpathrectangle{\pgfqpoint{1.432000in}{0.528000in}}{\pgfqpoint{3.696000in}{3.696000in}}%
\pgfusepath{clip}%
\pgfsetbuttcap%
\pgfsetroundjoin%
\pgfsetlinewidth{1.505625pt}%
\definecolor{currentstroke}{rgb}{0.000000,0.000000,0.000000}%
\pgfsetstrokecolor{currentstroke}%
\pgfsetdash{}{0pt}%
\pgfpathmoveto{\pgfqpoint{2.272000in}{2.376000in}}%
\pgfpathlineto{\pgfqpoint{2.608000in}{2.376000in}}%
\pgfpathlineto{\pgfqpoint{2.272000in}{2.712000in}}%
\pgfpathlineto{\pgfqpoint{2.272000in}{2.376000in}}%
\pgfusepath{stroke}%
\end{pgfscope}%
\begin{pgfscope}%
\pgfpathrectangle{\pgfqpoint{1.432000in}{0.528000in}}{\pgfqpoint{3.696000in}{3.696000in}}%
\pgfusepath{clip}%
\pgfsetbuttcap%
\pgfsetroundjoin%
\pgfsetlinewidth{1.505625pt}%
\definecolor{currentstroke}{rgb}{0.000000,0.000000,0.000000}%
\pgfsetstrokecolor{currentstroke}%
\pgfsetdash{}{0pt}%
\pgfpathmoveto{\pgfqpoint{1.936000in}{2.712000in}}%
\pgfpathlineto{\pgfqpoint{2.272000in}{2.712000in}}%
\pgfpathlineto{\pgfqpoint{1.936000in}{3.048000in}}%
\pgfpathlineto{\pgfqpoint{1.936000in}{2.712000in}}%
\pgfusepath{stroke}%
\end{pgfscope}%
\begin{pgfscope}%
\pgfpathrectangle{\pgfqpoint{1.432000in}{0.528000in}}{\pgfqpoint{3.696000in}{3.696000in}}%
\pgfusepath{clip}%
\pgfsetbuttcap%
\pgfsetroundjoin%
\pgfsetlinewidth{1.505625pt}%
\definecolor{currentstroke}{rgb}{0.000000,0.000000,0.000000}%
\pgfsetstrokecolor{currentstroke}%
\pgfsetdash{}{0pt}%
\pgfpathmoveto{\pgfqpoint{1.600000in}{3.048000in}}%
\pgfpathlineto{\pgfqpoint{1.936000in}{3.048000in}}%
\pgfpathlineto{\pgfqpoint{1.600000in}{3.384000in}}%
\pgfpathlineto{\pgfqpoint{1.600000in}{3.048000in}}%
\pgfusepath{stroke}%
\end{pgfscope}%
\begin{pgfscope}%
\pgfpathrectangle{\pgfqpoint{1.432000in}{0.528000in}}{\pgfqpoint{3.696000in}{3.696000in}}%
\pgfusepath{clip}%
\pgfsetbuttcap%
\pgfsetroundjoin%
\pgfsetlinewidth{1.505625pt}%
\definecolor{currentstroke}{rgb}{0.000000,0.000000,0.000000}%
\pgfsetstrokecolor{currentstroke}%
\pgfsetdash{}{0pt}%
\pgfpathmoveto{\pgfqpoint{3.952000in}{1.032000in}}%
\pgfpathlineto{\pgfqpoint{4.288000in}{0.696000in}}%
\pgfpathlineto{\pgfqpoint{4.288000in}{1.032000in}}%
\pgfpathlineto{\pgfqpoint{3.952000in}{1.032000in}}%
\pgfusepath{stroke}%
\end{pgfscope}%
\begin{pgfscope}%
\pgfpathrectangle{\pgfqpoint{1.432000in}{0.528000in}}{\pgfqpoint{3.696000in}{3.696000in}}%
\pgfusepath{clip}%
\pgfsetbuttcap%
\pgfsetroundjoin%
\pgfsetlinewidth{1.505625pt}%
\definecolor{currentstroke}{rgb}{0.000000,0.000000,0.000000}%
\pgfsetstrokecolor{currentstroke}%
\pgfsetdash{}{0pt}%
\pgfpathmoveto{\pgfqpoint{3.952000in}{1.032000in}}%
\pgfpathlineto{\pgfqpoint{3.616000in}{1.368000in}}%
\pgfpathlineto{\pgfqpoint{3.952000in}{1.368000in}}%
\pgfpathlineto{\pgfqpoint{3.952000in}{1.032000in}}%
\pgfusepath{stroke}%
\end{pgfscope}%
\begin{pgfscope}%
\pgfpathrectangle{\pgfqpoint{1.432000in}{0.528000in}}{\pgfqpoint{3.696000in}{3.696000in}}%
\pgfusepath{clip}%
\pgfsetbuttcap%
\pgfsetroundjoin%
\pgfsetlinewidth{1.505625pt}%
\definecolor{currentstroke}{rgb}{0.000000,0.000000,0.000000}%
\pgfsetstrokecolor{currentstroke}%
\pgfsetdash{}{0pt}%
\pgfpathmoveto{\pgfqpoint{3.616000in}{1.368000in}}%
\pgfpathlineto{\pgfqpoint{3.280000in}{1.704000in}}%
\pgfpathlineto{\pgfqpoint{3.616000in}{1.704000in}}%
\pgfpathlineto{\pgfqpoint{3.616000in}{1.368000in}}%
\pgfusepath{stroke}%
\end{pgfscope}%
\begin{pgfscope}%
\pgfpathrectangle{\pgfqpoint{1.432000in}{0.528000in}}{\pgfqpoint{3.696000in}{3.696000in}}%
\pgfusepath{clip}%
\pgfsetbuttcap%
\pgfsetroundjoin%
\pgfsetlinewidth{1.505625pt}%
\definecolor{currentstroke}{rgb}{0.000000,0.000000,0.000000}%
\pgfsetstrokecolor{currentstroke}%
\pgfsetdash{}{0pt}%
\pgfpathmoveto{\pgfqpoint{3.280000in}{1.704000in}}%
\pgfpathlineto{\pgfqpoint{2.944000in}{2.040000in}}%
\pgfpathlineto{\pgfqpoint{3.280000in}{2.040000in}}%
\pgfpathlineto{\pgfqpoint{3.280000in}{1.704000in}}%
\pgfusepath{stroke}%
\end{pgfscope}%
\begin{pgfscope}%
\pgfpathrectangle{\pgfqpoint{1.432000in}{0.528000in}}{\pgfqpoint{3.696000in}{3.696000in}}%
\pgfusepath{clip}%
\pgfsetbuttcap%
\pgfsetroundjoin%
\pgfsetlinewidth{1.505625pt}%
\definecolor{currentstroke}{rgb}{0.000000,0.000000,0.000000}%
\pgfsetstrokecolor{currentstroke}%
\pgfsetdash{}{0pt}%
\pgfpathmoveto{\pgfqpoint{2.944000in}{2.040000in}}%
\pgfpathlineto{\pgfqpoint{2.608000in}{2.376000in}}%
\pgfpathlineto{\pgfqpoint{2.944000in}{2.376000in}}%
\pgfpathlineto{\pgfqpoint{2.944000in}{2.040000in}}%
\pgfusepath{stroke}%
\end{pgfscope}%
\begin{pgfscope}%
\pgfpathrectangle{\pgfqpoint{1.432000in}{0.528000in}}{\pgfqpoint{3.696000in}{3.696000in}}%
\pgfusepath{clip}%
\pgfsetbuttcap%
\pgfsetroundjoin%
\pgfsetlinewidth{1.505625pt}%
\definecolor{currentstroke}{rgb}{0.000000,0.000000,0.000000}%
\pgfsetstrokecolor{currentstroke}%
\pgfsetdash{}{0pt}%
\pgfpathmoveto{\pgfqpoint{2.608000in}{2.376000in}}%
\pgfpathlineto{\pgfqpoint{2.272000in}{2.712000in}}%
\pgfpathlineto{\pgfqpoint{2.608000in}{2.712000in}}%
\pgfpathlineto{\pgfqpoint{2.608000in}{2.376000in}}%
\pgfusepath{stroke}%
\end{pgfscope}%
\begin{pgfscope}%
\pgfpathrectangle{\pgfqpoint{1.432000in}{0.528000in}}{\pgfqpoint{3.696000in}{3.696000in}}%
\pgfusepath{clip}%
\pgfsetbuttcap%
\pgfsetroundjoin%
\pgfsetlinewidth{1.505625pt}%
\definecolor{currentstroke}{rgb}{0.000000,0.000000,0.000000}%
\pgfsetstrokecolor{currentstroke}%
\pgfsetdash{}{0pt}%
\pgfpathmoveto{\pgfqpoint{2.272000in}{2.712000in}}%
\pgfpathlineto{\pgfqpoint{1.936000in}{3.048000in}}%
\pgfpathlineto{\pgfqpoint{2.272000in}{3.048000in}}%
\pgfpathlineto{\pgfqpoint{2.272000in}{2.712000in}}%
\pgfusepath{stroke}%
\end{pgfscope}%
\begin{pgfscope}%
\pgfpathrectangle{\pgfqpoint{1.432000in}{0.528000in}}{\pgfqpoint{3.696000in}{3.696000in}}%
\pgfusepath{clip}%
\pgfsetbuttcap%
\pgfsetroundjoin%
\pgfsetlinewidth{1.505625pt}%
\definecolor{currentstroke}{rgb}{0.000000,0.000000,0.000000}%
\pgfsetstrokecolor{currentstroke}%
\pgfsetdash{}{0pt}%
\pgfpathmoveto{\pgfqpoint{1.936000in}{3.048000in}}%
\pgfpathlineto{\pgfqpoint{1.600000in}{3.384000in}}%
\pgfpathlineto{\pgfqpoint{1.936000in}{3.384000in}}%
\pgfpathlineto{\pgfqpoint{1.936000in}{3.048000in}}%
\pgfusepath{stroke}%
\end{pgfscope}%
\begin{pgfscope}%
\pgfpathrectangle{\pgfqpoint{1.432000in}{0.528000in}}{\pgfqpoint{3.696000in}{3.696000in}}%
\pgfusepath{clip}%
\pgfsetbuttcap%
\pgfsetroundjoin%
\pgfsetlinewidth{1.505625pt}%
\definecolor{currentstroke}{rgb}{0.000000,0.000000,0.000000}%
\pgfsetstrokecolor{currentstroke}%
\pgfsetdash{}{0pt}%
\pgfpathmoveto{\pgfqpoint{4.288000in}{0.696000in}}%
\pgfpathlineto{\pgfqpoint{4.288000in}{1.032000in}}%
\pgfpathlineto{\pgfqpoint{4.624000in}{0.696000in}}%
\pgfpathlineto{\pgfqpoint{4.288000in}{0.696000in}}%
\pgfusepath{stroke}%
\end{pgfscope}%
\begin{pgfscope}%
\pgfpathrectangle{\pgfqpoint{1.432000in}{0.528000in}}{\pgfqpoint{3.696000in}{3.696000in}}%
\pgfusepath{clip}%
\pgfsetbuttcap%
\pgfsetroundjoin%
\pgfsetlinewidth{1.505625pt}%
\definecolor{currentstroke}{rgb}{0.000000,0.000000,0.000000}%
\pgfsetstrokecolor{currentstroke}%
\pgfsetdash{}{0pt}%
\pgfpathmoveto{\pgfqpoint{3.952000in}{1.032000in}}%
\pgfpathlineto{\pgfqpoint{4.288000in}{1.032000in}}%
\pgfpathlineto{\pgfqpoint{3.952000in}{1.368000in}}%
\pgfpathlineto{\pgfqpoint{3.952000in}{1.032000in}}%
\pgfusepath{stroke}%
\end{pgfscope}%
\begin{pgfscope}%
\pgfpathrectangle{\pgfqpoint{1.432000in}{0.528000in}}{\pgfqpoint{3.696000in}{3.696000in}}%
\pgfusepath{clip}%
\pgfsetbuttcap%
\pgfsetroundjoin%
\pgfsetlinewidth{1.505625pt}%
\definecolor{currentstroke}{rgb}{0.000000,0.000000,0.000000}%
\pgfsetstrokecolor{currentstroke}%
\pgfsetdash{}{0pt}%
\pgfpathmoveto{\pgfqpoint{3.616000in}{1.368000in}}%
\pgfpathlineto{\pgfqpoint{3.952000in}{1.368000in}}%
\pgfpathlineto{\pgfqpoint{3.616000in}{1.704000in}}%
\pgfpathlineto{\pgfqpoint{3.616000in}{1.368000in}}%
\pgfusepath{stroke}%
\end{pgfscope}%
\begin{pgfscope}%
\pgfpathrectangle{\pgfqpoint{1.432000in}{0.528000in}}{\pgfqpoint{3.696000in}{3.696000in}}%
\pgfusepath{clip}%
\pgfsetbuttcap%
\pgfsetroundjoin%
\pgfsetlinewidth{1.505625pt}%
\definecolor{currentstroke}{rgb}{0.000000,0.000000,0.000000}%
\pgfsetstrokecolor{currentstroke}%
\pgfsetdash{}{0pt}%
\pgfpathmoveto{\pgfqpoint{3.280000in}{1.704000in}}%
\pgfpathlineto{\pgfqpoint{3.616000in}{1.704000in}}%
\pgfpathlineto{\pgfqpoint{3.280000in}{2.040000in}}%
\pgfpathlineto{\pgfqpoint{3.280000in}{1.704000in}}%
\pgfusepath{stroke}%
\end{pgfscope}%
\begin{pgfscope}%
\pgfpathrectangle{\pgfqpoint{1.432000in}{0.528000in}}{\pgfqpoint{3.696000in}{3.696000in}}%
\pgfusepath{clip}%
\pgfsetbuttcap%
\pgfsetroundjoin%
\pgfsetlinewidth{1.505625pt}%
\definecolor{currentstroke}{rgb}{0.000000,0.000000,0.000000}%
\pgfsetstrokecolor{currentstroke}%
\pgfsetdash{}{0pt}%
\pgfpathmoveto{\pgfqpoint{2.944000in}{2.040000in}}%
\pgfpathlineto{\pgfqpoint{3.280000in}{2.040000in}}%
\pgfpathlineto{\pgfqpoint{2.944000in}{2.376000in}}%
\pgfpathlineto{\pgfqpoint{2.944000in}{2.040000in}}%
\pgfusepath{stroke}%
\end{pgfscope}%
\begin{pgfscope}%
\pgfpathrectangle{\pgfqpoint{1.432000in}{0.528000in}}{\pgfqpoint{3.696000in}{3.696000in}}%
\pgfusepath{clip}%
\pgfsetbuttcap%
\pgfsetroundjoin%
\pgfsetlinewidth{1.505625pt}%
\definecolor{currentstroke}{rgb}{0.000000,0.000000,0.000000}%
\pgfsetstrokecolor{currentstroke}%
\pgfsetdash{}{0pt}%
\pgfpathmoveto{\pgfqpoint{2.608000in}{2.376000in}}%
\pgfpathlineto{\pgfqpoint{2.944000in}{2.376000in}}%
\pgfpathlineto{\pgfqpoint{2.608000in}{2.712000in}}%
\pgfpathlineto{\pgfqpoint{2.608000in}{2.376000in}}%
\pgfusepath{stroke}%
\end{pgfscope}%
\begin{pgfscope}%
\pgfpathrectangle{\pgfqpoint{1.432000in}{0.528000in}}{\pgfqpoint{3.696000in}{3.696000in}}%
\pgfusepath{clip}%
\pgfsetbuttcap%
\pgfsetroundjoin%
\pgfsetlinewidth{1.505625pt}%
\definecolor{currentstroke}{rgb}{0.000000,0.000000,0.000000}%
\pgfsetstrokecolor{currentstroke}%
\pgfsetdash{}{0pt}%
\pgfpathmoveto{\pgfqpoint{2.272000in}{2.712000in}}%
\pgfpathlineto{\pgfqpoint{2.608000in}{2.712000in}}%
\pgfpathlineto{\pgfqpoint{2.272000in}{3.048000in}}%
\pgfpathlineto{\pgfqpoint{2.272000in}{2.712000in}}%
\pgfusepath{stroke}%
\end{pgfscope}%
\begin{pgfscope}%
\pgfpathrectangle{\pgfqpoint{1.432000in}{0.528000in}}{\pgfqpoint{3.696000in}{3.696000in}}%
\pgfusepath{clip}%
\pgfsetbuttcap%
\pgfsetroundjoin%
\pgfsetlinewidth{1.505625pt}%
\definecolor{currentstroke}{rgb}{0.000000,0.000000,0.000000}%
\pgfsetstrokecolor{currentstroke}%
\pgfsetdash{}{0pt}%
\pgfpathmoveto{\pgfqpoint{1.936000in}{3.048000in}}%
\pgfpathlineto{\pgfqpoint{2.272000in}{3.048000in}}%
\pgfpathlineto{\pgfqpoint{1.936000in}{3.384000in}}%
\pgfpathlineto{\pgfqpoint{1.936000in}{3.048000in}}%
\pgfusepath{stroke}%
\end{pgfscope}%
\begin{pgfscope}%
\pgfpathrectangle{\pgfqpoint{1.432000in}{0.528000in}}{\pgfqpoint{3.696000in}{3.696000in}}%
\pgfusepath{clip}%
\pgfsetbuttcap%
\pgfsetroundjoin%
\pgfsetlinewidth{1.505625pt}%
\definecolor{currentstroke}{rgb}{0.000000,0.000000,0.000000}%
\pgfsetstrokecolor{currentstroke}%
\pgfsetdash{}{0pt}%
\pgfpathmoveto{\pgfqpoint{1.600000in}{3.384000in}}%
\pgfpathlineto{\pgfqpoint{1.936000in}{3.384000in}}%
\pgfpathlineto{\pgfqpoint{1.600000in}{3.720000in}}%
\pgfpathlineto{\pgfqpoint{1.600000in}{3.384000in}}%
\pgfusepath{stroke}%
\end{pgfscope}%
\begin{pgfscope}%
\pgfpathrectangle{\pgfqpoint{1.432000in}{0.528000in}}{\pgfqpoint{3.696000in}{3.696000in}}%
\pgfusepath{clip}%
\pgfsetbuttcap%
\pgfsetroundjoin%
\pgfsetlinewidth{1.505625pt}%
\definecolor{currentstroke}{rgb}{0.000000,0.000000,0.000000}%
\pgfsetstrokecolor{currentstroke}%
\pgfsetdash{}{0pt}%
\pgfpathmoveto{\pgfqpoint{4.288000in}{1.032000in}}%
\pgfpathlineto{\pgfqpoint{4.624000in}{0.696000in}}%
\pgfpathlineto{\pgfqpoint{4.624000in}{1.032000in}}%
\pgfpathlineto{\pgfqpoint{4.288000in}{1.032000in}}%
\pgfusepath{stroke}%
\end{pgfscope}%
\begin{pgfscope}%
\pgfpathrectangle{\pgfqpoint{1.432000in}{0.528000in}}{\pgfqpoint{3.696000in}{3.696000in}}%
\pgfusepath{clip}%
\pgfsetbuttcap%
\pgfsetroundjoin%
\pgfsetlinewidth{1.505625pt}%
\definecolor{currentstroke}{rgb}{0.000000,0.000000,0.000000}%
\pgfsetstrokecolor{currentstroke}%
\pgfsetdash{}{0pt}%
\pgfpathmoveto{\pgfqpoint{4.288000in}{1.032000in}}%
\pgfpathlineto{\pgfqpoint{3.952000in}{1.368000in}}%
\pgfpathlineto{\pgfqpoint{4.288000in}{1.368000in}}%
\pgfpathlineto{\pgfqpoint{4.288000in}{1.032000in}}%
\pgfusepath{stroke}%
\end{pgfscope}%
\begin{pgfscope}%
\pgfpathrectangle{\pgfqpoint{1.432000in}{0.528000in}}{\pgfqpoint{3.696000in}{3.696000in}}%
\pgfusepath{clip}%
\pgfsetbuttcap%
\pgfsetroundjoin%
\pgfsetlinewidth{1.505625pt}%
\definecolor{currentstroke}{rgb}{0.000000,0.000000,0.000000}%
\pgfsetstrokecolor{currentstroke}%
\pgfsetdash{}{0pt}%
\pgfpathmoveto{\pgfqpoint{3.952000in}{1.368000in}}%
\pgfpathlineto{\pgfqpoint{3.616000in}{1.704000in}}%
\pgfpathlineto{\pgfqpoint{3.952000in}{1.704000in}}%
\pgfpathlineto{\pgfqpoint{3.952000in}{1.368000in}}%
\pgfusepath{stroke}%
\end{pgfscope}%
\begin{pgfscope}%
\pgfpathrectangle{\pgfqpoint{1.432000in}{0.528000in}}{\pgfqpoint{3.696000in}{3.696000in}}%
\pgfusepath{clip}%
\pgfsetbuttcap%
\pgfsetroundjoin%
\pgfsetlinewidth{1.505625pt}%
\definecolor{currentstroke}{rgb}{0.000000,0.000000,0.000000}%
\pgfsetstrokecolor{currentstroke}%
\pgfsetdash{}{0pt}%
\pgfpathmoveto{\pgfqpoint{3.616000in}{1.704000in}}%
\pgfpathlineto{\pgfqpoint{3.280000in}{2.040000in}}%
\pgfpathlineto{\pgfqpoint{3.616000in}{2.040000in}}%
\pgfpathlineto{\pgfqpoint{3.616000in}{1.704000in}}%
\pgfusepath{stroke}%
\end{pgfscope}%
\begin{pgfscope}%
\pgfpathrectangle{\pgfqpoint{1.432000in}{0.528000in}}{\pgfqpoint{3.696000in}{3.696000in}}%
\pgfusepath{clip}%
\pgfsetbuttcap%
\pgfsetroundjoin%
\pgfsetlinewidth{1.505625pt}%
\definecolor{currentstroke}{rgb}{0.000000,0.000000,0.000000}%
\pgfsetstrokecolor{currentstroke}%
\pgfsetdash{}{0pt}%
\pgfpathmoveto{\pgfqpoint{3.280000in}{2.040000in}}%
\pgfpathlineto{\pgfqpoint{2.944000in}{2.376000in}}%
\pgfpathlineto{\pgfqpoint{3.280000in}{2.376000in}}%
\pgfpathlineto{\pgfqpoint{3.280000in}{2.040000in}}%
\pgfusepath{stroke}%
\end{pgfscope}%
\begin{pgfscope}%
\pgfpathrectangle{\pgfqpoint{1.432000in}{0.528000in}}{\pgfqpoint{3.696000in}{3.696000in}}%
\pgfusepath{clip}%
\pgfsetbuttcap%
\pgfsetroundjoin%
\pgfsetlinewidth{1.505625pt}%
\definecolor{currentstroke}{rgb}{0.000000,0.000000,0.000000}%
\pgfsetstrokecolor{currentstroke}%
\pgfsetdash{}{0pt}%
\pgfpathmoveto{\pgfqpoint{2.944000in}{2.376000in}}%
\pgfpathlineto{\pgfqpoint{2.608000in}{2.712000in}}%
\pgfpathlineto{\pgfqpoint{2.944000in}{2.712000in}}%
\pgfpathlineto{\pgfqpoint{2.944000in}{2.376000in}}%
\pgfusepath{stroke}%
\end{pgfscope}%
\begin{pgfscope}%
\pgfpathrectangle{\pgfqpoint{1.432000in}{0.528000in}}{\pgfqpoint{3.696000in}{3.696000in}}%
\pgfusepath{clip}%
\pgfsetbuttcap%
\pgfsetroundjoin%
\pgfsetlinewidth{1.505625pt}%
\definecolor{currentstroke}{rgb}{0.000000,0.000000,0.000000}%
\pgfsetstrokecolor{currentstroke}%
\pgfsetdash{}{0pt}%
\pgfpathmoveto{\pgfqpoint{2.608000in}{2.712000in}}%
\pgfpathlineto{\pgfqpoint{2.272000in}{3.048000in}}%
\pgfpathlineto{\pgfqpoint{2.608000in}{3.048000in}}%
\pgfpathlineto{\pgfqpoint{2.608000in}{2.712000in}}%
\pgfusepath{stroke}%
\end{pgfscope}%
\begin{pgfscope}%
\pgfpathrectangle{\pgfqpoint{1.432000in}{0.528000in}}{\pgfqpoint{3.696000in}{3.696000in}}%
\pgfusepath{clip}%
\pgfsetbuttcap%
\pgfsetroundjoin%
\pgfsetlinewidth{1.505625pt}%
\definecolor{currentstroke}{rgb}{0.000000,0.000000,0.000000}%
\pgfsetstrokecolor{currentstroke}%
\pgfsetdash{}{0pt}%
\pgfpathmoveto{\pgfqpoint{2.272000in}{3.048000in}}%
\pgfpathlineto{\pgfqpoint{1.936000in}{3.384000in}}%
\pgfpathlineto{\pgfqpoint{2.272000in}{3.384000in}}%
\pgfpathlineto{\pgfqpoint{2.272000in}{3.048000in}}%
\pgfusepath{stroke}%
\end{pgfscope}%
\begin{pgfscope}%
\pgfpathrectangle{\pgfqpoint{1.432000in}{0.528000in}}{\pgfqpoint{3.696000in}{3.696000in}}%
\pgfusepath{clip}%
\pgfsetbuttcap%
\pgfsetroundjoin%
\pgfsetlinewidth{1.505625pt}%
\definecolor{currentstroke}{rgb}{0.000000,0.000000,0.000000}%
\pgfsetstrokecolor{currentstroke}%
\pgfsetdash{}{0pt}%
\pgfpathmoveto{\pgfqpoint{1.936000in}{3.384000in}}%
\pgfpathlineto{\pgfqpoint{1.600000in}{3.720000in}}%
\pgfpathlineto{\pgfqpoint{1.936000in}{3.720000in}}%
\pgfpathlineto{\pgfqpoint{1.936000in}{3.384000in}}%
\pgfusepath{stroke}%
\end{pgfscope}%
\begin{pgfscope}%
\pgfpathrectangle{\pgfqpoint{1.432000in}{0.528000in}}{\pgfqpoint{3.696000in}{3.696000in}}%
\pgfusepath{clip}%
\pgfsetbuttcap%
\pgfsetroundjoin%
\pgfsetlinewidth{1.505625pt}%
\definecolor{currentstroke}{rgb}{0.000000,0.000000,0.000000}%
\pgfsetstrokecolor{currentstroke}%
\pgfsetdash{}{0pt}%
\pgfpathmoveto{\pgfqpoint{4.624000in}{0.696000in}}%
\pgfpathlineto{\pgfqpoint{4.624000in}{1.032000in}}%
\pgfpathlineto{\pgfqpoint{4.960000in}{0.696000in}}%
\pgfpathlineto{\pgfqpoint{4.624000in}{0.696000in}}%
\pgfusepath{stroke}%
\end{pgfscope}%
\begin{pgfscope}%
\pgfpathrectangle{\pgfqpoint{1.432000in}{0.528000in}}{\pgfqpoint{3.696000in}{3.696000in}}%
\pgfusepath{clip}%
\pgfsetbuttcap%
\pgfsetroundjoin%
\pgfsetlinewidth{1.505625pt}%
\definecolor{currentstroke}{rgb}{0.000000,0.000000,0.000000}%
\pgfsetstrokecolor{currentstroke}%
\pgfsetdash{}{0pt}%
\pgfpathmoveto{\pgfqpoint{4.288000in}{1.032000in}}%
\pgfpathlineto{\pgfqpoint{4.624000in}{1.032000in}}%
\pgfpathlineto{\pgfqpoint{4.288000in}{1.368000in}}%
\pgfpathlineto{\pgfqpoint{4.288000in}{1.032000in}}%
\pgfusepath{stroke}%
\end{pgfscope}%
\begin{pgfscope}%
\pgfpathrectangle{\pgfqpoint{1.432000in}{0.528000in}}{\pgfqpoint{3.696000in}{3.696000in}}%
\pgfusepath{clip}%
\pgfsetbuttcap%
\pgfsetroundjoin%
\pgfsetlinewidth{1.505625pt}%
\definecolor{currentstroke}{rgb}{0.000000,0.000000,0.000000}%
\pgfsetstrokecolor{currentstroke}%
\pgfsetdash{}{0pt}%
\pgfpathmoveto{\pgfqpoint{3.952000in}{1.368000in}}%
\pgfpathlineto{\pgfqpoint{4.288000in}{1.368000in}}%
\pgfpathlineto{\pgfqpoint{3.952000in}{1.704000in}}%
\pgfpathlineto{\pgfqpoint{3.952000in}{1.368000in}}%
\pgfusepath{stroke}%
\end{pgfscope}%
\begin{pgfscope}%
\pgfpathrectangle{\pgfqpoint{1.432000in}{0.528000in}}{\pgfqpoint{3.696000in}{3.696000in}}%
\pgfusepath{clip}%
\pgfsetbuttcap%
\pgfsetroundjoin%
\pgfsetlinewidth{1.505625pt}%
\definecolor{currentstroke}{rgb}{0.000000,0.000000,0.000000}%
\pgfsetstrokecolor{currentstroke}%
\pgfsetdash{}{0pt}%
\pgfpathmoveto{\pgfqpoint{3.616000in}{1.704000in}}%
\pgfpathlineto{\pgfqpoint{3.952000in}{1.704000in}}%
\pgfpathlineto{\pgfqpoint{3.616000in}{2.040000in}}%
\pgfpathlineto{\pgfqpoint{3.616000in}{1.704000in}}%
\pgfusepath{stroke}%
\end{pgfscope}%
\begin{pgfscope}%
\pgfpathrectangle{\pgfqpoint{1.432000in}{0.528000in}}{\pgfqpoint{3.696000in}{3.696000in}}%
\pgfusepath{clip}%
\pgfsetbuttcap%
\pgfsetroundjoin%
\pgfsetlinewidth{1.505625pt}%
\definecolor{currentstroke}{rgb}{0.000000,0.000000,0.000000}%
\pgfsetstrokecolor{currentstroke}%
\pgfsetdash{}{0pt}%
\pgfpathmoveto{\pgfqpoint{3.280000in}{2.040000in}}%
\pgfpathlineto{\pgfqpoint{3.616000in}{2.040000in}}%
\pgfpathlineto{\pgfqpoint{3.280000in}{2.376000in}}%
\pgfpathlineto{\pgfqpoint{3.280000in}{2.040000in}}%
\pgfusepath{stroke}%
\end{pgfscope}%
\begin{pgfscope}%
\pgfpathrectangle{\pgfqpoint{1.432000in}{0.528000in}}{\pgfqpoint{3.696000in}{3.696000in}}%
\pgfusepath{clip}%
\pgfsetbuttcap%
\pgfsetroundjoin%
\pgfsetlinewidth{1.505625pt}%
\definecolor{currentstroke}{rgb}{0.000000,0.000000,0.000000}%
\pgfsetstrokecolor{currentstroke}%
\pgfsetdash{}{0pt}%
\pgfpathmoveto{\pgfqpoint{2.944000in}{2.376000in}}%
\pgfpathlineto{\pgfqpoint{3.280000in}{2.376000in}}%
\pgfpathlineto{\pgfqpoint{2.944000in}{2.712000in}}%
\pgfpathlineto{\pgfqpoint{2.944000in}{2.376000in}}%
\pgfusepath{stroke}%
\end{pgfscope}%
\begin{pgfscope}%
\pgfpathrectangle{\pgfqpoint{1.432000in}{0.528000in}}{\pgfqpoint{3.696000in}{3.696000in}}%
\pgfusepath{clip}%
\pgfsetbuttcap%
\pgfsetroundjoin%
\pgfsetlinewidth{1.505625pt}%
\definecolor{currentstroke}{rgb}{0.000000,0.000000,0.000000}%
\pgfsetstrokecolor{currentstroke}%
\pgfsetdash{}{0pt}%
\pgfpathmoveto{\pgfqpoint{2.608000in}{2.712000in}}%
\pgfpathlineto{\pgfqpoint{2.944000in}{2.712000in}}%
\pgfpathlineto{\pgfqpoint{2.608000in}{3.048000in}}%
\pgfpathlineto{\pgfqpoint{2.608000in}{2.712000in}}%
\pgfusepath{stroke}%
\end{pgfscope}%
\begin{pgfscope}%
\pgfpathrectangle{\pgfqpoint{1.432000in}{0.528000in}}{\pgfqpoint{3.696000in}{3.696000in}}%
\pgfusepath{clip}%
\pgfsetbuttcap%
\pgfsetroundjoin%
\pgfsetlinewidth{1.505625pt}%
\definecolor{currentstroke}{rgb}{0.000000,0.000000,0.000000}%
\pgfsetstrokecolor{currentstroke}%
\pgfsetdash{}{0pt}%
\pgfpathmoveto{\pgfqpoint{2.272000in}{3.048000in}}%
\pgfpathlineto{\pgfqpoint{2.608000in}{3.048000in}}%
\pgfpathlineto{\pgfqpoint{2.272000in}{3.384000in}}%
\pgfpathlineto{\pgfqpoint{2.272000in}{3.048000in}}%
\pgfusepath{stroke}%
\end{pgfscope}%
\begin{pgfscope}%
\pgfpathrectangle{\pgfqpoint{1.432000in}{0.528000in}}{\pgfqpoint{3.696000in}{3.696000in}}%
\pgfusepath{clip}%
\pgfsetbuttcap%
\pgfsetroundjoin%
\pgfsetlinewidth{1.505625pt}%
\definecolor{currentstroke}{rgb}{0.000000,0.000000,0.000000}%
\pgfsetstrokecolor{currentstroke}%
\pgfsetdash{}{0pt}%
\pgfpathmoveto{\pgfqpoint{1.936000in}{3.384000in}}%
\pgfpathlineto{\pgfqpoint{2.272000in}{3.384000in}}%
\pgfpathlineto{\pgfqpoint{1.936000in}{3.720000in}}%
\pgfpathlineto{\pgfqpoint{1.936000in}{3.384000in}}%
\pgfusepath{stroke}%
\end{pgfscope}%
\begin{pgfscope}%
\pgfpathrectangle{\pgfqpoint{1.432000in}{0.528000in}}{\pgfqpoint{3.696000in}{3.696000in}}%
\pgfusepath{clip}%
\pgfsetbuttcap%
\pgfsetroundjoin%
\pgfsetlinewidth{1.505625pt}%
\definecolor{currentstroke}{rgb}{0.000000,0.000000,0.000000}%
\pgfsetstrokecolor{currentstroke}%
\pgfsetdash{}{0pt}%
\pgfpathmoveto{\pgfqpoint{1.600000in}{3.720000in}}%
\pgfpathlineto{\pgfqpoint{1.936000in}{3.720000in}}%
\pgfpathlineto{\pgfqpoint{1.600000in}{4.056000in}}%
\pgfpathlineto{\pgfqpoint{1.600000in}{3.720000in}}%
\pgfusepath{stroke}%
\end{pgfscope}%
\begin{pgfscope}%
\pgfpathrectangle{\pgfqpoint{1.432000in}{0.528000in}}{\pgfqpoint{3.696000in}{3.696000in}}%
\pgfusepath{clip}%
\pgfsetbuttcap%
\pgfsetroundjoin%
\pgfsetlinewidth{1.505625pt}%
\definecolor{currentstroke}{rgb}{0.000000,0.000000,0.000000}%
\pgfsetstrokecolor{currentstroke}%
\pgfsetdash{}{0pt}%
\pgfpathmoveto{\pgfqpoint{4.624000in}{1.032000in}}%
\pgfpathlineto{\pgfqpoint{4.960000in}{0.696000in}}%
\pgfpathlineto{\pgfqpoint{4.960000in}{1.032000in}}%
\pgfpathlineto{\pgfqpoint{4.624000in}{1.032000in}}%
\pgfusepath{stroke}%
\end{pgfscope}%
\begin{pgfscope}%
\pgfpathrectangle{\pgfqpoint{1.432000in}{0.528000in}}{\pgfqpoint{3.696000in}{3.696000in}}%
\pgfusepath{clip}%
\pgfsetbuttcap%
\pgfsetroundjoin%
\pgfsetlinewidth{1.505625pt}%
\definecolor{currentstroke}{rgb}{0.000000,0.000000,0.000000}%
\pgfsetstrokecolor{currentstroke}%
\pgfsetdash{}{0pt}%
\pgfpathmoveto{\pgfqpoint{4.624000in}{1.032000in}}%
\pgfpathlineto{\pgfqpoint{4.288000in}{1.368000in}}%
\pgfpathlineto{\pgfqpoint{4.624000in}{1.368000in}}%
\pgfpathlineto{\pgfqpoint{4.624000in}{1.032000in}}%
\pgfusepath{stroke}%
\end{pgfscope}%
\begin{pgfscope}%
\pgfpathrectangle{\pgfqpoint{1.432000in}{0.528000in}}{\pgfqpoint{3.696000in}{3.696000in}}%
\pgfusepath{clip}%
\pgfsetbuttcap%
\pgfsetroundjoin%
\pgfsetlinewidth{1.505625pt}%
\definecolor{currentstroke}{rgb}{0.000000,0.000000,0.000000}%
\pgfsetstrokecolor{currentstroke}%
\pgfsetdash{}{0pt}%
\pgfpathmoveto{\pgfqpoint{4.288000in}{1.368000in}}%
\pgfpathlineto{\pgfqpoint{3.952000in}{1.704000in}}%
\pgfpathlineto{\pgfqpoint{4.288000in}{1.704000in}}%
\pgfpathlineto{\pgfqpoint{4.288000in}{1.368000in}}%
\pgfusepath{stroke}%
\end{pgfscope}%
\begin{pgfscope}%
\pgfpathrectangle{\pgfqpoint{1.432000in}{0.528000in}}{\pgfqpoint{3.696000in}{3.696000in}}%
\pgfusepath{clip}%
\pgfsetbuttcap%
\pgfsetroundjoin%
\pgfsetlinewidth{1.505625pt}%
\definecolor{currentstroke}{rgb}{0.000000,0.000000,0.000000}%
\pgfsetstrokecolor{currentstroke}%
\pgfsetdash{}{0pt}%
\pgfpathmoveto{\pgfqpoint{3.952000in}{1.704000in}}%
\pgfpathlineto{\pgfqpoint{3.616000in}{2.040000in}}%
\pgfpathlineto{\pgfqpoint{3.952000in}{2.040000in}}%
\pgfpathlineto{\pgfqpoint{3.952000in}{1.704000in}}%
\pgfusepath{stroke}%
\end{pgfscope}%
\begin{pgfscope}%
\pgfpathrectangle{\pgfqpoint{1.432000in}{0.528000in}}{\pgfqpoint{3.696000in}{3.696000in}}%
\pgfusepath{clip}%
\pgfsetbuttcap%
\pgfsetroundjoin%
\pgfsetlinewidth{1.505625pt}%
\definecolor{currentstroke}{rgb}{0.000000,0.000000,0.000000}%
\pgfsetstrokecolor{currentstroke}%
\pgfsetdash{}{0pt}%
\pgfpathmoveto{\pgfqpoint{3.616000in}{2.040000in}}%
\pgfpathlineto{\pgfqpoint{3.280000in}{2.376000in}}%
\pgfpathlineto{\pgfqpoint{3.616000in}{2.376000in}}%
\pgfpathlineto{\pgfqpoint{3.616000in}{2.040000in}}%
\pgfusepath{stroke}%
\end{pgfscope}%
\begin{pgfscope}%
\pgfpathrectangle{\pgfqpoint{1.432000in}{0.528000in}}{\pgfqpoint{3.696000in}{3.696000in}}%
\pgfusepath{clip}%
\pgfsetbuttcap%
\pgfsetroundjoin%
\pgfsetlinewidth{1.505625pt}%
\definecolor{currentstroke}{rgb}{0.000000,0.000000,0.000000}%
\pgfsetstrokecolor{currentstroke}%
\pgfsetdash{}{0pt}%
\pgfpathmoveto{\pgfqpoint{3.280000in}{2.376000in}}%
\pgfpathlineto{\pgfqpoint{2.944000in}{2.712000in}}%
\pgfpathlineto{\pgfqpoint{3.280000in}{2.712000in}}%
\pgfpathlineto{\pgfqpoint{3.280000in}{2.376000in}}%
\pgfusepath{stroke}%
\end{pgfscope}%
\begin{pgfscope}%
\pgfpathrectangle{\pgfqpoint{1.432000in}{0.528000in}}{\pgfqpoint{3.696000in}{3.696000in}}%
\pgfusepath{clip}%
\pgfsetbuttcap%
\pgfsetroundjoin%
\pgfsetlinewidth{1.505625pt}%
\definecolor{currentstroke}{rgb}{0.000000,0.000000,0.000000}%
\pgfsetstrokecolor{currentstroke}%
\pgfsetdash{}{0pt}%
\pgfpathmoveto{\pgfqpoint{2.944000in}{2.712000in}}%
\pgfpathlineto{\pgfqpoint{2.608000in}{3.048000in}}%
\pgfpathlineto{\pgfqpoint{2.944000in}{3.048000in}}%
\pgfpathlineto{\pgfqpoint{2.944000in}{2.712000in}}%
\pgfusepath{stroke}%
\end{pgfscope}%
\begin{pgfscope}%
\pgfpathrectangle{\pgfqpoint{1.432000in}{0.528000in}}{\pgfqpoint{3.696000in}{3.696000in}}%
\pgfusepath{clip}%
\pgfsetbuttcap%
\pgfsetroundjoin%
\pgfsetlinewidth{1.505625pt}%
\definecolor{currentstroke}{rgb}{0.000000,0.000000,0.000000}%
\pgfsetstrokecolor{currentstroke}%
\pgfsetdash{}{0pt}%
\pgfpathmoveto{\pgfqpoint{2.608000in}{3.048000in}}%
\pgfpathlineto{\pgfqpoint{2.272000in}{3.384000in}}%
\pgfpathlineto{\pgfqpoint{2.608000in}{3.384000in}}%
\pgfpathlineto{\pgfqpoint{2.608000in}{3.048000in}}%
\pgfusepath{stroke}%
\end{pgfscope}%
\begin{pgfscope}%
\pgfpathrectangle{\pgfqpoint{1.432000in}{0.528000in}}{\pgfqpoint{3.696000in}{3.696000in}}%
\pgfusepath{clip}%
\pgfsetbuttcap%
\pgfsetroundjoin%
\pgfsetlinewidth{1.505625pt}%
\definecolor{currentstroke}{rgb}{0.000000,0.000000,0.000000}%
\pgfsetstrokecolor{currentstroke}%
\pgfsetdash{}{0pt}%
\pgfpathmoveto{\pgfqpoint{2.272000in}{3.384000in}}%
\pgfpathlineto{\pgfqpoint{1.936000in}{3.720000in}}%
\pgfpathlineto{\pgfqpoint{2.272000in}{3.720000in}}%
\pgfpathlineto{\pgfqpoint{2.272000in}{3.384000in}}%
\pgfusepath{stroke}%
\end{pgfscope}%
\begin{pgfscope}%
\pgfpathrectangle{\pgfqpoint{1.432000in}{0.528000in}}{\pgfqpoint{3.696000in}{3.696000in}}%
\pgfusepath{clip}%
\pgfsetbuttcap%
\pgfsetroundjoin%
\pgfsetlinewidth{1.505625pt}%
\definecolor{currentstroke}{rgb}{0.000000,0.000000,0.000000}%
\pgfsetstrokecolor{currentstroke}%
\pgfsetdash{}{0pt}%
\pgfpathmoveto{\pgfqpoint{1.936000in}{3.720000in}}%
\pgfpathlineto{\pgfqpoint{1.600000in}{4.056000in}}%
\pgfpathlineto{\pgfqpoint{1.936000in}{4.056000in}}%
\pgfpathlineto{\pgfqpoint{1.936000in}{3.720000in}}%
\pgfusepath{stroke}%
\end{pgfscope}%
\begin{pgfscope}%
\pgfpathrectangle{\pgfqpoint{1.432000in}{0.528000in}}{\pgfqpoint{3.696000in}{3.696000in}}%
\pgfusepath{clip}%
\pgfsetbuttcap%
\pgfsetroundjoin%
\pgfsetlinewidth{1.505625pt}%
\definecolor{currentstroke}{rgb}{0.000000,0.000000,0.000000}%
\pgfsetstrokecolor{currentstroke}%
\pgfsetdash{}{0pt}%
\pgfpathmoveto{\pgfqpoint{4.624000in}{1.032000in}}%
\pgfpathlineto{\pgfqpoint{4.960000in}{1.032000in}}%
\pgfpathlineto{\pgfqpoint{4.624000in}{1.368000in}}%
\pgfpathlineto{\pgfqpoint{4.624000in}{1.032000in}}%
\pgfusepath{stroke}%
\end{pgfscope}%
\begin{pgfscope}%
\pgfpathrectangle{\pgfqpoint{1.432000in}{0.528000in}}{\pgfqpoint{3.696000in}{3.696000in}}%
\pgfusepath{clip}%
\pgfsetbuttcap%
\pgfsetroundjoin%
\pgfsetlinewidth{1.505625pt}%
\definecolor{currentstroke}{rgb}{0.000000,0.000000,0.000000}%
\pgfsetstrokecolor{currentstroke}%
\pgfsetdash{}{0pt}%
\pgfpathmoveto{\pgfqpoint{4.288000in}{1.368000in}}%
\pgfpathlineto{\pgfqpoint{4.624000in}{1.368000in}}%
\pgfpathlineto{\pgfqpoint{4.288000in}{1.704000in}}%
\pgfpathlineto{\pgfqpoint{4.288000in}{1.368000in}}%
\pgfusepath{stroke}%
\end{pgfscope}%
\begin{pgfscope}%
\pgfpathrectangle{\pgfqpoint{1.432000in}{0.528000in}}{\pgfqpoint{3.696000in}{3.696000in}}%
\pgfusepath{clip}%
\pgfsetbuttcap%
\pgfsetroundjoin%
\pgfsetlinewidth{1.505625pt}%
\definecolor{currentstroke}{rgb}{0.000000,0.000000,0.000000}%
\pgfsetstrokecolor{currentstroke}%
\pgfsetdash{}{0pt}%
\pgfpathmoveto{\pgfqpoint{3.952000in}{1.704000in}}%
\pgfpathlineto{\pgfqpoint{4.288000in}{1.704000in}}%
\pgfpathlineto{\pgfqpoint{3.952000in}{2.040000in}}%
\pgfpathlineto{\pgfqpoint{3.952000in}{1.704000in}}%
\pgfusepath{stroke}%
\end{pgfscope}%
\begin{pgfscope}%
\pgfpathrectangle{\pgfqpoint{1.432000in}{0.528000in}}{\pgfqpoint{3.696000in}{3.696000in}}%
\pgfusepath{clip}%
\pgfsetbuttcap%
\pgfsetroundjoin%
\pgfsetlinewidth{1.505625pt}%
\definecolor{currentstroke}{rgb}{0.000000,0.000000,0.000000}%
\pgfsetstrokecolor{currentstroke}%
\pgfsetdash{}{0pt}%
\pgfpathmoveto{\pgfqpoint{3.616000in}{2.040000in}}%
\pgfpathlineto{\pgfqpoint{3.952000in}{2.040000in}}%
\pgfpathlineto{\pgfqpoint{3.616000in}{2.376000in}}%
\pgfpathlineto{\pgfqpoint{3.616000in}{2.040000in}}%
\pgfusepath{stroke}%
\end{pgfscope}%
\begin{pgfscope}%
\pgfpathrectangle{\pgfqpoint{1.432000in}{0.528000in}}{\pgfqpoint{3.696000in}{3.696000in}}%
\pgfusepath{clip}%
\pgfsetbuttcap%
\pgfsetroundjoin%
\pgfsetlinewidth{1.505625pt}%
\definecolor{currentstroke}{rgb}{0.000000,0.000000,0.000000}%
\pgfsetstrokecolor{currentstroke}%
\pgfsetdash{}{0pt}%
\pgfpathmoveto{\pgfqpoint{3.280000in}{2.376000in}}%
\pgfpathlineto{\pgfqpoint{3.616000in}{2.376000in}}%
\pgfpathlineto{\pgfqpoint{3.280000in}{2.712000in}}%
\pgfpathlineto{\pgfqpoint{3.280000in}{2.376000in}}%
\pgfusepath{stroke}%
\end{pgfscope}%
\begin{pgfscope}%
\pgfpathrectangle{\pgfqpoint{1.432000in}{0.528000in}}{\pgfqpoint{3.696000in}{3.696000in}}%
\pgfusepath{clip}%
\pgfsetbuttcap%
\pgfsetroundjoin%
\pgfsetlinewidth{1.505625pt}%
\definecolor{currentstroke}{rgb}{0.000000,0.000000,0.000000}%
\pgfsetstrokecolor{currentstroke}%
\pgfsetdash{}{0pt}%
\pgfpathmoveto{\pgfqpoint{2.944000in}{2.712000in}}%
\pgfpathlineto{\pgfqpoint{3.280000in}{2.712000in}}%
\pgfpathlineto{\pgfqpoint{2.944000in}{3.048000in}}%
\pgfpathlineto{\pgfqpoint{2.944000in}{2.712000in}}%
\pgfusepath{stroke}%
\end{pgfscope}%
\begin{pgfscope}%
\pgfpathrectangle{\pgfqpoint{1.432000in}{0.528000in}}{\pgfqpoint{3.696000in}{3.696000in}}%
\pgfusepath{clip}%
\pgfsetbuttcap%
\pgfsetroundjoin%
\pgfsetlinewidth{1.505625pt}%
\definecolor{currentstroke}{rgb}{0.000000,0.000000,0.000000}%
\pgfsetstrokecolor{currentstroke}%
\pgfsetdash{}{0pt}%
\pgfpathmoveto{\pgfqpoint{2.608000in}{3.048000in}}%
\pgfpathlineto{\pgfqpoint{2.944000in}{3.048000in}}%
\pgfpathlineto{\pgfqpoint{2.608000in}{3.384000in}}%
\pgfpathlineto{\pgfqpoint{2.608000in}{3.048000in}}%
\pgfusepath{stroke}%
\end{pgfscope}%
\begin{pgfscope}%
\pgfpathrectangle{\pgfqpoint{1.432000in}{0.528000in}}{\pgfqpoint{3.696000in}{3.696000in}}%
\pgfusepath{clip}%
\pgfsetbuttcap%
\pgfsetroundjoin%
\pgfsetlinewidth{1.505625pt}%
\definecolor{currentstroke}{rgb}{0.000000,0.000000,0.000000}%
\pgfsetstrokecolor{currentstroke}%
\pgfsetdash{}{0pt}%
\pgfpathmoveto{\pgfqpoint{2.272000in}{3.384000in}}%
\pgfpathlineto{\pgfqpoint{2.608000in}{3.384000in}}%
\pgfpathlineto{\pgfqpoint{2.272000in}{3.720000in}}%
\pgfpathlineto{\pgfqpoint{2.272000in}{3.384000in}}%
\pgfusepath{stroke}%
\end{pgfscope}%
\begin{pgfscope}%
\pgfpathrectangle{\pgfqpoint{1.432000in}{0.528000in}}{\pgfqpoint{3.696000in}{3.696000in}}%
\pgfusepath{clip}%
\pgfsetbuttcap%
\pgfsetroundjoin%
\pgfsetlinewidth{1.505625pt}%
\definecolor{currentstroke}{rgb}{0.000000,0.000000,0.000000}%
\pgfsetstrokecolor{currentstroke}%
\pgfsetdash{}{0pt}%
\pgfpathmoveto{\pgfqpoint{1.936000in}{3.720000in}}%
\pgfpathlineto{\pgfqpoint{2.272000in}{3.720000in}}%
\pgfpathlineto{\pgfqpoint{1.936000in}{4.056000in}}%
\pgfpathlineto{\pgfqpoint{1.936000in}{3.720000in}}%
\pgfusepath{stroke}%
\end{pgfscope}%
\begin{pgfscope}%
\pgfpathrectangle{\pgfqpoint{1.432000in}{0.528000in}}{\pgfqpoint{3.696000in}{3.696000in}}%
\pgfusepath{clip}%
\pgfsetbuttcap%
\pgfsetroundjoin%
\pgfsetlinewidth{1.505625pt}%
\definecolor{currentstroke}{rgb}{0.000000,0.000000,0.000000}%
\pgfsetstrokecolor{currentstroke}%
\pgfsetdash{}{0pt}%
\pgfpathmoveto{\pgfqpoint{4.960000in}{1.032000in}}%
\pgfpathlineto{\pgfqpoint{4.624000in}{1.368000in}}%
\pgfpathlineto{\pgfqpoint{4.960000in}{1.368000in}}%
\pgfpathlineto{\pgfqpoint{4.960000in}{1.032000in}}%
\pgfusepath{stroke}%
\end{pgfscope}%
\begin{pgfscope}%
\pgfpathrectangle{\pgfqpoint{1.432000in}{0.528000in}}{\pgfqpoint{3.696000in}{3.696000in}}%
\pgfusepath{clip}%
\pgfsetbuttcap%
\pgfsetroundjoin%
\pgfsetlinewidth{1.505625pt}%
\definecolor{currentstroke}{rgb}{0.000000,0.000000,0.000000}%
\pgfsetstrokecolor{currentstroke}%
\pgfsetdash{}{0pt}%
\pgfpathmoveto{\pgfqpoint{4.624000in}{1.368000in}}%
\pgfpathlineto{\pgfqpoint{4.288000in}{1.704000in}}%
\pgfpathlineto{\pgfqpoint{4.624000in}{1.704000in}}%
\pgfpathlineto{\pgfqpoint{4.624000in}{1.368000in}}%
\pgfusepath{stroke}%
\end{pgfscope}%
\begin{pgfscope}%
\pgfpathrectangle{\pgfqpoint{1.432000in}{0.528000in}}{\pgfqpoint{3.696000in}{3.696000in}}%
\pgfusepath{clip}%
\pgfsetbuttcap%
\pgfsetroundjoin%
\pgfsetlinewidth{1.505625pt}%
\definecolor{currentstroke}{rgb}{0.000000,0.000000,0.000000}%
\pgfsetstrokecolor{currentstroke}%
\pgfsetdash{}{0pt}%
\pgfpathmoveto{\pgfqpoint{4.288000in}{1.704000in}}%
\pgfpathlineto{\pgfqpoint{3.952000in}{2.040000in}}%
\pgfpathlineto{\pgfqpoint{4.288000in}{2.040000in}}%
\pgfpathlineto{\pgfqpoint{4.288000in}{1.704000in}}%
\pgfusepath{stroke}%
\end{pgfscope}%
\begin{pgfscope}%
\pgfpathrectangle{\pgfqpoint{1.432000in}{0.528000in}}{\pgfqpoint{3.696000in}{3.696000in}}%
\pgfusepath{clip}%
\pgfsetbuttcap%
\pgfsetroundjoin%
\pgfsetlinewidth{1.505625pt}%
\definecolor{currentstroke}{rgb}{0.000000,0.000000,0.000000}%
\pgfsetstrokecolor{currentstroke}%
\pgfsetdash{}{0pt}%
\pgfpathmoveto{\pgfqpoint{3.952000in}{2.040000in}}%
\pgfpathlineto{\pgfqpoint{3.616000in}{2.376000in}}%
\pgfpathlineto{\pgfqpoint{3.952000in}{2.376000in}}%
\pgfpathlineto{\pgfqpoint{3.952000in}{2.040000in}}%
\pgfusepath{stroke}%
\end{pgfscope}%
\begin{pgfscope}%
\pgfpathrectangle{\pgfqpoint{1.432000in}{0.528000in}}{\pgfqpoint{3.696000in}{3.696000in}}%
\pgfusepath{clip}%
\pgfsetbuttcap%
\pgfsetroundjoin%
\pgfsetlinewidth{1.505625pt}%
\definecolor{currentstroke}{rgb}{0.000000,0.000000,0.000000}%
\pgfsetstrokecolor{currentstroke}%
\pgfsetdash{}{0pt}%
\pgfpathmoveto{\pgfqpoint{3.616000in}{2.376000in}}%
\pgfpathlineto{\pgfqpoint{3.280000in}{2.712000in}}%
\pgfpathlineto{\pgfqpoint{3.616000in}{2.712000in}}%
\pgfpathlineto{\pgfqpoint{3.616000in}{2.376000in}}%
\pgfusepath{stroke}%
\end{pgfscope}%
\begin{pgfscope}%
\pgfpathrectangle{\pgfqpoint{1.432000in}{0.528000in}}{\pgfqpoint{3.696000in}{3.696000in}}%
\pgfusepath{clip}%
\pgfsetbuttcap%
\pgfsetroundjoin%
\pgfsetlinewidth{1.505625pt}%
\definecolor{currentstroke}{rgb}{0.000000,0.000000,0.000000}%
\pgfsetstrokecolor{currentstroke}%
\pgfsetdash{}{0pt}%
\pgfpathmoveto{\pgfqpoint{3.280000in}{2.712000in}}%
\pgfpathlineto{\pgfqpoint{2.944000in}{3.048000in}}%
\pgfpathlineto{\pgfqpoint{3.280000in}{3.048000in}}%
\pgfpathlineto{\pgfqpoint{3.280000in}{2.712000in}}%
\pgfusepath{stroke}%
\end{pgfscope}%
\begin{pgfscope}%
\pgfpathrectangle{\pgfqpoint{1.432000in}{0.528000in}}{\pgfqpoint{3.696000in}{3.696000in}}%
\pgfusepath{clip}%
\pgfsetbuttcap%
\pgfsetroundjoin%
\pgfsetlinewidth{1.505625pt}%
\definecolor{currentstroke}{rgb}{0.000000,0.000000,0.000000}%
\pgfsetstrokecolor{currentstroke}%
\pgfsetdash{}{0pt}%
\pgfpathmoveto{\pgfqpoint{2.944000in}{3.048000in}}%
\pgfpathlineto{\pgfqpoint{2.608000in}{3.384000in}}%
\pgfpathlineto{\pgfqpoint{2.944000in}{3.384000in}}%
\pgfpathlineto{\pgfqpoint{2.944000in}{3.048000in}}%
\pgfusepath{stroke}%
\end{pgfscope}%
\begin{pgfscope}%
\pgfpathrectangle{\pgfqpoint{1.432000in}{0.528000in}}{\pgfqpoint{3.696000in}{3.696000in}}%
\pgfusepath{clip}%
\pgfsetbuttcap%
\pgfsetroundjoin%
\pgfsetlinewidth{1.505625pt}%
\definecolor{currentstroke}{rgb}{0.000000,0.000000,0.000000}%
\pgfsetstrokecolor{currentstroke}%
\pgfsetdash{}{0pt}%
\pgfpathmoveto{\pgfqpoint{2.608000in}{3.384000in}}%
\pgfpathlineto{\pgfqpoint{2.272000in}{3.720000in}}%
\pgfpathlineto{\pgfqpoint{2.608000in}{3.720000in}}%
\pgfpathlineto{\pgfqpoint{2.608000in}{3.384000in}}%
\pgfusepath{stroke}%
\end{pgfscope}%
\begin{pgfscope}%
\pgfpathrectangle{\pgfqpoint{1.432000in}{0.528000in}}{\pgfqpoint{3.696000in}{3.696000in}}%
\pgfusepath{clip}%
\pgfsetbuttcap%
\pgfsetroundjoin%
\pgfsetlinewidth{1.505625pt}%
\definecolor{currentstroke}{rgb}{0.000000,0.000000,0.000000}%
\pgfsetstrokecolor{currentstroke}%
\pgfsetdash{}{0pt}%
\pgfpathmoveto{\pgfqpoint{2.272000in}{3.720000in}}%
\pgfpathlineto{\pgfqpoint{1.936000in}{4.056000in}}%
\pgfpathlineto{\pgfqpoint{2.272000in}{4.056000in}}%
\pgfpathlineto{\pgfqpoint{2.272000in}{3.720000in}}%
\pgfusepath{stroke}%
\end{pgfscope}%
\begin{pgfscope}%
\pgfpathrectangle{\pgfqpoint{1.432000in}{0.528000in}}{\pgfqpoint{3.696000in}{3.696000in}}%
\pgfusepath{clip}%
\pgfsetbuttcap%
\pgfsetroundjoin%
\pgfsetlinewidth{1.505625pt}%
\definecolor{currentstroke}{rgb}{0.000000,0.000000,0.000000}%
\pgfsetstrokecolor{currentstroke}%
\pgfsetdash{}{0pt}%
\pgfpathmoveto{\pgfqpoint{4.624000in}{1.368000in}}%
\pgfpathlineto{\pgfqpoint{4.960000in}{1.368000in}}%
\pgfpathlineto{\pgfqpoint{4.624000in}{1.704000in}}%
\pgfpathlineto{\pgfqpoint{4.624000in}{1.368000in}}%
\pgfusepath{stroke}%
\end{pgfscope}%
\begin{pgfscope}%
\pgfpathrectangle{\pgfqpoint{1.432000in}{0.528000in}}{\pgfqpoint{3.696000in}{3.696000in}}%
\pgfusepath{clip}%
\pgfsetbuttcap%
\pgfsetroundjoin%
\pgfsetlinewidth{1.505625pt}%
\definecolor{currentstroke}{rgb}{0.000000,0.000000,0.000000}%
\pgfsetstrokecolor{currentstroke}%
\pgfsetdash{}{0pt}%
\pgfpathmoveto{\pgfqpoint{4.288000in}{1.704000in}}%
\pgfpathlineto{\pgfqpoint{4.624000in}{1.704000in}}%
\pgfpathlineto{\pgfqpoint{4.288000in}{2.040000in}}%
\pgfpathlineto{\pgfqpoint{4.288000in}{1.704000in}}%
\pgfusepath{stroke}%
\end{pgfscope}%
\begin{pgfscope}%
\pgfpathrectangle{\pgfqpoint{1.432000in}{0.528000in}}{\pgfqpoint{3.696000in}{3.696000in}}%
\pgfusepath{clip}%
\pgfsetbuttcap%
\pgfsetroundjoin%
\pgfsetlinewidth{1.505625pt}%
\definecolor{currentstroke}{rgb}{0.000000,0.000000,0.000000}%
\pgfsetstrokecolor{currentstroke}%
\pgfsetdash{}{0pt}%
\pgfpathmoveto{\pgfqpoint{3.952000in}{2.040000in}}%
\pgfpathlineto{\pgfqpoint{4.288000in}{2.040000in}}%
\pgfpathlineto{\pgfqpoint{3.952000in}{2.376000in}}%
\pgfpathlineto{\pgfqpoint{3.952000in}{2.040000in}}%
\pgfusepath{stroke}%
\end{pgfscope}%
\begin{pgfscope}%
\pgfpathrectangle{\pgfqpoint{1.432000in}{0.528000in}}{\pgfqpoint{3.696000in}{3.696000in}}%
\pgfusepath{clip}%
\pgfsetbuttcap%
\pgfsetroundjoin%
\pgfsetlinewidth{1.505625pt}%
\definecolor{currentstroke}{rgb}{0.000000,0.000000,0.000000}%
\pgfsetstrokecolor{currentstroke}%
\pgfsetdash{}{0pt}%
\pgfpathmoveto{\pgfqpoint{3.616000in}{2.376000in}}%
\pgfpathlineto{\pgfqpoint{3.952000in}{2.376000in}}%
\pgfpathlineto{\pgfqpoint{3.616000in}{2.712000in}}%
\pgfpathlineto{\pgfqpoint{3.616000in}{2.376000in}}%
\pgfusepath{stroke}%
\end{pgfscope}%
\begin{pgfscope}%
\pgfpathrectangle{\pgfqpoint{1.432000in}{0.528000in}}{\pgfqpoint{3.696000in}{3.696000in}}%
\pgfusepath{clip}%
\pgfsetbuttcap%
\pgfsetroundjoin%
\pgfsetlinewidth{1.505625pt}%
\definecolor{currentstroke}{rgb}{0.000000,0.000000,0.000000}%
\pgfsetstrokecolor{currentstroke}%
\pgfsetdash{}{0pt}%
\pgfpathmoveto{\pgfqpoint{3.280000in}{2.712000in}}%
\pgfpathlineto{\pgfqpoint{3.616000in}{2.712000in}}%
\pgfpathlineto{\pgfqpoint{3.280000in}{3.048000in}}%
\pgfpathlineto{\pgfqpoint{3.280000in}{2.712000in}}%
\pgfusepath{stroke}%
\end{pgfscope}%
\begin{pgfscope}%
\pgfpathrectangle{\pgfqpoint{1.432000in}{0.528000in}}{\pgfqpoint{3.696000in}{3.696000in}}%
\pgfusepath{clip}%
\pgfsetbuttcap%
\pgfsetroundjoin%
\pgfsetlinewidth{1.505625pt}%
\definecolor{currentstroke}{rgb}{0.000000,0.000000,0.000000}%
\pgfsetstrokecolor{currentstroke}%
\pgfsetdash{}{0pt}%
\pgfpathmoveto{\pgfqpoint{2.944000in}{3.048000in}}%
\pgfpathlineto{\pgfqpoint{3.280000in}{3.048000in}}%
\pgfpathlineto{\pgfqpoint{2.944000in}{3.384000in}}%
\pgfpathlineto{\pgfqpoint{2.944000in}{3.048000in}}%
\pgfusepath{stroke}%
\end{pgfscope}%
\begin{pgfscope}%
\pgfpathrectangle{\pgfqpoint{1.432000in}{0.528000in}}{\pgfqpoint{3.696000in}{3.696000in}}%
\pgfusepath{clip}%
\pgfsetbuttcap%
\pgfsetroundjoin%
\pgfsetlinewidth{1.505625pt}%
\definecolor{currentstroke}{rgb}{0.000000,0.000000,0.000000}%
\pgfsetstrokecolor{currentstroke}%
\pgfsetdash{}{0pt}%
\pgfpathmoveto{\pgfqpoint{2.608000in}{3.384000in}}%
\pgfpathlineto{\pgfqpoint{2.944000in}{3.384000in}}%
\pgfpathlineto{\pgfqpoint{2.608000in}{3.720000in}}%
\pgfpathlineto{\pgfqpoint{2.608000in}{3.384000in}}%
\pgfusepath{stroke}%
\end{pgfscope}%
\begin{pgfscope}%
\pgfpathrectangle{\pgfqpoint{1.432000in}{0.528000in}}{\pgfqpoint{3.696000in}{3.696000in}}%
\pgfusepath{clip}%
\pgfsetbuttcap%
\pgfsetroundjoin%
\pgfsetlinewidth{1.505625pt}%
\definecolor{currentstroke}{rgb}{0.000000,0.000000,0.000000}%
\pgfsetstrokecolor{currentstroke}%
\pgfsetdash{}{0pt}%
\pgfpathmoveto{\pgfqpoint{2.272000in}{3.720000in}}%
\pgfpathlineto{\pgfqpoint{2.608000in}{3.720000in}}%
\pgfpathlineto{\pgfqpoint{2.272000in}{4.056000in}}%
\pgfpathlineto{\pgfqpoint{2.272000in}{3.720000in}}%
\pgfusepath{stroke}%
\end{pgfscope}%
\begin{pgfscope}%
\pgfpathrectangle{\pgfqpoint{1.432000in}{0.528000in}}{\pgfqpoint{3.696000in}{3.696000in}}%
\pgfusepath{clip}%
\pgfsetbuttcap%
\pgfsetroundjoin%
\pgfsetlinewidth{1.505625pt}%
\definecolor{currentstroke}{rgb}{0.000000,0.000000,0.000000}%
\pgfsetstrokecolor{currentstroke}%
\pgfsetdash{}{0pt}%
\pgfpathmoveto{\pgfqpoint{4.960000in}{1.368000in}}%
\pgfpathlineto{\pgfqpoint{4.624000in}{1.704000in}}%
\pgfpathlineto{\pgfqpoint{4.960000in}{1.704000in}}%
\pgfpathlineto{\pgfqpoint{4.960000in}{1.368000in}}%
\pgfusepath{stroke}%
\end{pgfscope}%
\begin{pgfscope}%
\pgfpathrectangle{\pgfqpoint{1.432000in}{0.528000in}}{\pgfqpoint{3.696000in}{3.696000in}}%
\pgfusepath{clip}%
\pgfsetbuttcap%
\pgfsetroundjoin%
\pgfsetlinewidth{1.505625pt}%
\definecolor{currentstroke}{rgb}{0.000000,0.000000,0.000000}%
\pgfsetstrokecolor{currentstroke}%
\pgfsetdash{}{0pt}%
\pgfpathmoveto{\pgfqpoint{4.624000in}{1.704000in}}%
\pgfpathlineto{\pgfqpoint{4.288000in}{2.040000in}}%
\pgfpathlineto{\pgfqpoint{4.624000in}{2.040000in}}%
\pgfpathlineto{\pgfqpoint{4.624000in}{1.704000in}}%
\pgfusepath{stroke}%
\end{pgfscope}%
\begin{pgfscope}%
\pgfpathrectangle{\pgfqpoint{1.432000in}{0.528000in}}{\pgfqpoint{3.696000in}{3.696000in}}%
\pgfusepath{clip}%
\pgfsetbuttcap%
\pgfsetroundjoin%
\pgfsetlinewidth{1.505625pt}%
\definecolor{currentstroke}{rgb}{0.000000,0.000000,0.000000}%
\pgfsetstrokecolor{currentstroke}%
\pgfsetdash{}{0pt}%
\pgfpathmoveto{\pgfqpoint{4.288000in}{2.040000in}}%
\pgfpathlineto{\pgfqpoint{3.952000in}{2.376000in}}%
\pgfpathlineto{\pgfqpoint{4.288000in}{2.376000in}}%
\pgfpathlineto{\pgfqpoint{4.288000in}{2.040000in}}%
\pgfusepath{stroke}%
\end{pgfscope}%
\begin{pgfscope}%
\pgfpathrectangle{\pgfqpoint{1.432000in}{0.528000in}}{\pgfqpoint{3.696000in}{3.696000in}}%
\pgfusepath{clip}%
\pgfsetbuttcap%
\pgfsetroundjoin%
\pgfsetlinewidth{1.505625pt}%
\definecolor{currentstroke}{rgb}{0.000000,0.000000,0.000000}%
\pgfsetstrokecolor{currentstroke}%
\pgfsetdash{}{0pt}%
\pgfpathmoveto{\pgfqpoint{3.952000in}{2.376000in}}%
\pgfpathlineto{\pgfqpoint{3.616000in}{2.712000in}}%
\pgfpathlineto{\pgfqpoint{3.952000in}{2.712000in}}%
\pgfpathlineto{\pgfqpoint{3.952000in}{2.376000in}}%
\pgfusepath{stroke}%
\end{pgfscope}%
\begin{pgfscope}%
\pgfpathrectangle{\pgfqpoint{1.432000in}{0.528000in}}{\pgfqpoint{3.696000in}{3.696000in}}%
\pgfusepath{clip}%
\pgfsetbuttcap%
\pgfsetroundjoin%
\pgfsetlinewidth{1.505625pt}%
\definecolor{currentstroke}{rgb}{0.000000,0.000000,0.000000}%
\pgfsetstrokecolor{currentstroke}%
\pgfsetdash{}{0pt}%
\pgfpathmoveto{\pgfqpoint{3.616000in}{2.712000in}}%
\pgfpathlineto{\pgfqpoint{3.280000in}{3.048000in}}%
\pgfpathlineto{\pgfqpoint{3.616000in}{3.048000in}}%
\pgfpathlineto{\pgfqpoint{3.616000in}{2.712000in}}%
\pgfusepath{stroke}%
\end{pgfscope}%
\begin{pgfscope}%
\pgfpathrectangle{\pgfqpoint{1.432000in}{0.528000in}}{\pgfqpoint{3.696000in}{3.696000in}}%
\pgfusepath{clip}%
\pgfsetbuttcap%
\pgfsetroundjoin%
\pgfsetlinewidth{1.505625pt}%
\definecolor{currentstroke}{rgb}{0.000000,0.000000,0.000000}%
\pgfsetstrokecolor{currentstroke}%
\pgfsetdash{}{0pt}%
\pgfpathmoveto{\pgfqpoint{3.280000in}{3.048000in}}%
\pgfpathlineto{\pgfqpoint{2.944000in}{3.384000in}}%
\pgfpathlineto{\pgfqpoint{3.280000in}{3.384000in}}%
\pgfpathlineto{\pgfqpoint{3.280000in}{3.048000in}}%
\pgfusepath{stroke}%
\end{pgfscope}%
\begin{pgfscope}%
\pgfpathrectangle{\pgfqpoint{1.432000in}{0.528000in}}{\pgfqpoint{3.696000in}{3.696000in}}%
\pgfusepath{clip}%
\pgfsetbuttcap%
\pgfsetroundjoin%
\pgfsetlinewidth{1.505625pt}%
\definecolor{currentstroke}{rgb}{0.000000,0.000000,0.000000}%
\pgfsetstrokecolor{currentstroke}%
\pgfsetdash{}{0pt}%
\pgfpathmoveto{\pgfqpoint{2.944000in}{3.384000in}}%
\pgfpathlineto{\pgfqpoint{2.608000in}{3.720000in}}%
\pgfpathlineto{\pgfqpoint{2.944000in}{3.720000in}}%
\pgfpathlineto{\pgfqpoint{2.944000in}{3.384000in}}%
\pgfusepath{stroke}%
\end{pgfscope}%
\begin{pgfscope}%
\pgfpathrectangle{\pgfqpoint{1.432000in}{0.528000in}}{\pgfqpoint{3.696000in}{3.696000in}}%
\pgfusepath{clip}%
\pgfsetbuttcap%
\pgfsetroundjoin%
\pgfsetlinewidth{1.505625pt}%
\definecolor{currentstroke}{rgb}{0.000000,0.000000,0.000000}%
\pgfsetstrokecolor{currentstroke}%
\pgfsetdash{}{0pt}%
\pgfpathmoveto{\pgfqpoint{2.608000in}{3.720000in}}%
\pgfpathlineto{\pgfqpoint{2.272000in}{4.056000in}}%
\pgfpathlineto{\pgfqpoint{2.608000in}{4.056000in}}%
\pgfpathlineto{\pgfqpoint{2.608000in}{3.720000in}}%
\pgfusepath{stroke}%
\end{pgfscope}%
\begin{pgfscope}%
\pgfpathrectangle{\pgfqpoint{1.432000in}{0.528000in}}{\pgfqpoint{3.696000in}{3.696000in}}%
\pgfusepath{clip}%
\pgfsetbuttcap%
\pgfsetroundjoin%
\pgfsetlinewidth{1.505625pt}%
\definecolor{currentstroke}{rgb}{0.000000,0.000000,0.000000}%
\pgfsetstrokecolor{currentstroke}%
\pgfsetdash{}{0pt}%
\pgfpathmoveto{\pgfqpoint{4.624000in}{1.704000in}}%
\pgfpathlineto{\pgfqpoint{4.960000in}{1.704000in}}%
\pgfpathlineto{\pgfqpoint{4.624000in}{2.040000in}}%
\pgfpathlineto{\pgfqpoint{4.624000in}{1.704000in}}%
\pgfusepath{stroke}%
\end{pgfscope}%
\begin{pgfscope}%
\pgfpathrectangle{\pgfqpoint{1.432000in}{0.528000in}}{\pgfqpoint{3.696000in}{3.696000in}}%
\pgfusepath{clip}%
\pgfsetbuttcap%
\pgfsetroundjoin%
\pgfsetlinewidth{1.505625pt}%
\definecolor{currentstroke}{rgb}{0.000000,0.000000,0.000000}%
\pgfsetstrokecolor{currentstroke}%
\pgfsetdash{}{0pt}%
\pgfpathmoveto{\pgfqpoint{4.288000in}{2.040000in}}%
\pgfpathlineto{\pgfqpoint{4.624000in}{2.040000in}}%
\pgfpathlineto{\pgfqpoint{4.288000in}{2.376000in}}%
\pgfpathlineto{\pgfqpoint{4.288000in}{2.040000in}}%
\pgfusepath{stroke}%
\end{pgfscope}%
\begin{pgfscope}%
\pgfpathrectangle{\pgfqpoint{1.432000in}{0.528000in}}{\pgfqpoint{3.696000in}{3.696000in}}%
\pgfusepath{clip}%
\pgfsetbuttcap%
\pgfsetroundjoin%
\pgfsetlinewidth{1.505625pt}%
\definecolor{currentstroke}{rgb}{0.000000,0.000000,0.000000}%
\pgfsetstrokecolor{currentstroke}%
\pgfsetdash{}{0pt}%
\pgfpathmoveto{\pgfqpoint{3.952000in}{2.376000in}}%
\pgfpathlineto{\pgfqpoint{4.288000in}{2.376000in}}%
\pgfpathlineto{\pgfqpoint{3.952000in}{2.712000in}}%
\pgfpathlineto{\pgfqpoint{3.952000in}{2.376000in}}%
\pgfusepath{stroke}%
\end{pgfscope}%
\begin{pgfscope}%
\pgfpathrectangle{\pgfqpoint{1.432000in}{0.528000in}}{\pgfqpoint{3.696000in}{3.696000in}}%
\pgfusepath{clip}%
\pgfsetbuttcap%
\pgfsetroundjoin%
\pgfsetlinewidth{1.505625pt}%
\definecolor{currentstroke}{rgb}{0.000000,0.000000,0.000000}%
\pgfsetstrokecolor{currentstroke}%
\pgfsetdash{}{0pt}%
\pgfpathmoveto{\pgfqpoint{3.616000in}{2.712000in}}%
\pgfpathlineto{\pgfqpoint{3.952000in}{2.712000in}}%
\pgfpathlineto{\pgfqpoint{3.616000in}{3.048000in}}%
\pgfpathlineto{\pgfqpoint{3.616000in}{2.712000in}}%
\pgfusepath{stroke}%
\end{pgfscope}%
\begin{pgfscope}%
\pgfpathrectangle{\pgfqpoint{1.432000in}{0.528000in}}{\pgfqpoint{3.696000in}{3.696000in}}%
\pgfusepath{clip}%
\pgfsetbuttcap%
\pgfsetroundjoin%
\pgfsetlinewidth{1.505625pt}%
\definecolor{currentstroke}{rgb}{0.000000,0.000000,0.000000}%
\pgfsetstrokecolor{currentstroke}%
\pgfsetdash{}{0pt}%
\pgfpathmoveto{\pgfqpoint{3.280000in}{3.048000in}}%
\pgfpathlineto{\pgfqpoint{3.616000in}{3.048000in}}%
\pgfpathlineto{\pgfqpoint{3.280000in}{3.384000in}}%
\pgfpathlineto{\pgfqpoint{3.280000in}{3.048000in}}%
\pgfusepath{stroke}%
\end{pgfscope}%
\begin{pgfscope}%
\pgfpathrectangle{\pgfqpoint{1.432000in}{0.528000in}}{\pgfqpoint{3.696000in}{3.696000in}}%
\pgfusepath{clip}%
\pgfsetbuttcap%
\pgfsetroundjoin%
\pgfsetlinewidth{1.505625pt}%
\definecolor{currentstroke}{rgb}{0.000000,0.000000,0.000000}%
\pgfsetstrokecolor{currentstroke}%
\pgfsetdash{}{0pt}%
\pgfpathmoveto{\pgfqpoint{2.944000in}{3.384000in}}%
\pgfpathlineto{\pgfqpoint{3.280000in}{3.384000in}}%
\pgfpathlineto{\pgfqpoint{2.944000in}{3.720000in}}%
\pgfpathlineto{\pgfqpoint{2.944000in}{3.384000in}}%
\pgfusepath{stroke}%
\end{pgfscope}%
\begin{pgfscope}%
\pgfpathrectangle{\pgfqpoint{1.432000in}{0.528000in}}{\pgfqpoint{3.696000in}{3.696000in}}%
\pgfusepath{clip}%
\pgfsetbuttcap%
\pgfsetroundjoin%
\pgfsetlinewidth{1.505625pt}%
\definecolor{currentstroke}{rgb}{0.000000,0.000000,0.000000}%
\pgfsetstrokecolor{currentstroke}%
\pgfsetdash{}{0pt}%
\pgfpathmoveto{\pgfqpoint{2.608000in}{3.720000in}}%
\pgfpathlineto{\pgfqpoint{2.944000in}{3.720000in}}%
\pgfpathlineto{\pgfqpoint{2.608000in}{4.056000in}}%
\pgfpathlineto{\pgfqpoint{2.608000in}{3.720000in}}%
\pgfusepath{stroke}%
\end{pgfscope}%
\begin{pgfscope}%
\pgfpathrectangle{\pgfqpoint{1.432000in}{0.528000in}}{\pgfqpoint{3.696000in}{3.696000in}}%
\pgfusepath{clip}%
\pgfsetbuttcap%
\pgfsetroundjoin%
\pgfsetlinewidth{1.505625pt}%
\definecolor{currentstroke}{rgb}{0.000000,0.000000,0.000000}%
\pgfsetstrokecolor{currentstroke}%
\pgfsetdash{}{0pt}%
\pgfpathmoveto{\pgfqpoint{4.960000in}{1.704000in}}%
\pgfpathlineto{\pgfqpoint{4.624000in}{2.040000in}}%
\pgfpathlineto{\pgfqpoint{4.960000in}{2.040000in}}%
\pgfpathlineto{\pgfqpoint{4.960000in}{1.704000in}}%
\pgfusepath{stroke}%
\end{pgfscope}%
\begin{pgfscope}%
\pgfpathrectangle{\pgfqpoint{1.432000in}{0.528000in}}{\pgfqpoint{3.696000in}{3.696000in}}%
\pgfusepath{clip}%
\pgfsetbuttcap%
\pgfsetroundjoin%
\pgfsetlinewidth{1.505625pt}%
\definecolor{currentstroke}{rgb}{0.000000,0.000000,0.000000}%
\pgfsetstrokecolor{currentstroke}%
\pgfsetdash{}{0pt}%
\pgfpathmoveto{\pgfqpoint{4.624000in}{2.040000in}}%
\pgfpathlineto{\pgfqpoint{4.288000in}{2.376000in}}%
\pgfpathlineto{\pgfqpoint{4.624000in}{2.376000in}}%
\pgfpathlineto{\pgfqpoint{4.624000in}{2.040000in}}%
\pgfusepath{stroke}%
\end{pgfscope}%
\begin{pgfscope}%
\pgfpathrectangle{\pgfqpoint{1.432000in}{0.528000in}}{\pgfqpoint{3.696000in}{3.696000in}}%
\pgfusepath{clip}%
\pgfsetbuttcap%
\pgfsetroundjoin%
\pgfsetlinewidth{1.505625pt}%
\definecolor{currentstroke}{rgb}{0.000000,0.000000,0.000000}%
\pgfsetstrokecolor{currentstroke}%
\pgfsetdash{}{0pt}%
\pgfpathmoveto{\pgfqpoint{4.288000in}{2.376000in}}%
\pgfpathlineto{\pgfqpoint{3.952000in}{2.712000in}}%
\pgfpathlineto{\pgfqpoint{4.288000in}{2.712000in}}%
\pgfpathlineto{\pgfqpoint{4.288000in}{2.376000in}}%
\pgfusepath{stroke}%
\end{pgfscope}%
\begin{pgfscope}%
\pgfpathrectangle{\pgfqpoint{1.432000in}{0.528000in}}{\pgfqpoint{3.696000in}{3.696000in}}%
\pgfusepath{clip}%
\pgfsetbuttcap%
\pgfsetroundjoin%
\pgfsetlinewidth{1.505625pt}%
\definecolor{currentstroke}{rgb}{0.000000,0.000000,0.000000}%
\pgfsetstrokecolor{currentstroke}%
\pgfsetdash{}{0pt}%
\pgfpathmoveto{\pgfqpoint{3.952000in}{2.712000in}}%
\pgfpathlineto{\pgfqpoint{3.616000in}{3.048000in}}%
\pgfpathlineto{\pgfqpoint{3.952000in}{3.048000in}}%
\pgfpathlineto{\pgfqpoint{3.952000in}{2.712000in}}%
\pgfusepath{stroke}%
\end{pgfscope}%
\begin{pgfscope}%
\pgfpathrectangle{\pgfqpoint{1.432000in}{0.528000in}}{\pgfqpoint{3.696000in}{3.696000in}}%
\pgfusepath{clip}%
\pgfsetbuttcap%
\pgfsetroundjoin%
\pgfsetlinewidth{1.505625pt}%
\definecolor{currentstroke}{rgb}{0.000000,0.000000,0.000000}%
\pgfsetstrokecolor{currentstroke}%
\pgfsetdash{}{0pt}%
\pgfpathmoveto{\pgfqpoint{3.616000in}{3.048000in}}%
\pgfpathlineto{\pgfqpoint{3.280000in}{3.384000in}}%
\pgfpathlineto{\pgfqpoint{3.616000in}{3.384000in}}%
\pgfpathlineto{\pgfqpoint{3.616000in}{3.048000in}}%
\pgfusepath{stroke}%
\end{pgfscope}%
\begin{pgfscope}%
\pgfpathrectangle{\pgfqpoint{1.432000in}{0.528000in}}{\pgfqpoint{3.696000in}{3.696000in}}%
\pgfusepath{clip}%
\pgfsetbuttcap%
\pgfsetroundjoin%
\pgfsetlinewidth{1.505625pt}%
\definecolor{currentstroke}{rgb}{0.000000,0.000000,0.000000}%
\pgfsetstrokecolor{currentstroke}%
\pgfsetdash{}{0pt}%
\pgfpathmoveto{\pgfqpoint{3.280000in}{3.384000in}}%
\pgfpathlineto{\pgfqpoint{2.944000in}{3.720000in}}%
\pgfpathlineto{\pgfqpoint{3.280000in}{3.720000in}}%
\pgfpathlineto{\pgfqpoint{3.280000in}{3.384000in}}%
\pgfusepath{stroke}%
\end{pgfscope}%
\begin{pgfscope}%
\pgfpathrectangle{\pgfqpoint{1.432000in}{0.528000in}}{\pgfqpoint{3.696000in}{3.696000in}}%
\pgfusepath{clip}%
\pgfsetbuttcap%
\pgfsetroundjoin%
\pgfsetlinewidth{1.505625pt}%
\definecolor{currentstroke}{rgb}{0.000000,0.000000,0.000000}%
\pgfsetstrokecolor{currentstroke}%
\pgfsetdash{}{0pt}%
\pgfpathmoveto{\pgfqpoint{2.944000in}{3.720000in}}%
\pgfpathlineto{\pgfqpoint{2.608000in}{4.056000in}}%
\pgfpathlineto{\pgfqpoint{2.944000in}{4.056000in}}%
\pgfpathlineto{\pgfqpoint{2.944000in}{3.720000in}}%
\pgfusepath{stroke}%
\end{pgfscope}%
\begin{pgfscope}%
\pgfpathrectangle{\pgfqpoint{1.432000in}{0.528000in}}{\pgfqpoint{3.696000in}{3.696000in}}%
\pgfusepath{clip}%
\pgfsetbuttcap%
\pgfsetroundjoin%
\pgfsetlinewidth{1.505625pt}%
\definecolor{currentstroke}{rgb}{0.000000,0.000000,0.000000}%
\pgfsetstrokecolor{currentstroke}%
\pgfsetdash{}{0pt}%
\pgfpathmoveto{\pgfqpoint{4.624000in}{2.040000in}}%
\pgfpathlineto{\pgfqpoint{4.960000in}{2.040000in}}%
\pgfpathlineto{\pgfqpoint{4.624000in}{2.376000in}}%
\pgfpathlineto{\pgfqpoint{4.624000in}{2.040000in}}%
\pgfusepath{stroke}%
\end{pgfscope}%
\begin{pgfscope}%
\pgfpathrectangle{\pgfqpoint{1.432000in}{0.528000in}}{\pgfqpoint{3.696000in}{3.696000in}}%
\pgfusepath{clip}%
\pgfsetbuttcap%
\pgfsetroundjoin%
\pgfsetlinewidth{1.505625pt}%
\definecolor{currentstroke}{rgb}{0.000000,0.000000,0.000000}%
\pgfsetstrokecolor{currentstroke}%
\pgfsetdash{}{0pt}%
\pgfpathmoveto{\pgfqpoint{4.288000in}{2.376000in}}%
\pgfpathlineto{\pgfqpoint{4.624000in}{2.376000in}}%
\pgfpathlineto{\pgfqpoint{4.288000in}{2.712000in}}%
\pgfpathlineto{\pgfqpoint{4.288000in}{2.376000in}}%
\pgfusepath{stroke}%
\end{pgfscope}%
\begin{pgfscope}%
\pgfpathrectangle{\pgfqpoint{1.432000in}{0.528000in}}{\pgfqpoint{3.696000in}{3.696000in}}%
\pgfusepath{clip}%
\pgfsetbuttcap%
\pgfsetroundjoin%
\pgfsetlinewidth{1.505625pt}%
\definecolor{currentstroke}{rgb}{0.000000,0.000000,0.000000}%
\pgfsetstrokecolor{currentstroke}%
\pgfsetdash{}{0pt}%
\pgfpathmoveto{\pgfqpoint{3.952000in}{2.712000in}}%
\pgfpathlineto{\pgfqpoint{4.288000in}{2.712000in}}%
\pgfpathlineto{\pgfqpoint{3.952000in}{3.048000in}}%
\pgfpathlineto{\pgfqpoint{3.952000in}{2.712000in}}%
\pgfusepath{stroke}%
\end{pgfscope}%
\begin{pgfscope}%
\pgfpathrectangle{\pgfqpoint{1.432000in}{0.528000in}}{\pgfqpoint{3.696000in}{3.696000in}}%
\pgfusepath{clip}%
\pgfsetbuttcap%
\pgfsetroundjoin%
\pgfsetlinewidth{1.505625pt}%
\definecolor{currentstroke}{rgb}{0.000000,0.000000,0.000000}%
\pgfsetstrokecolor{currentstroke}%
\pgfsetdash{}{0pt}%
\pgfpathmoveto{\pgfqpoint{3.616000in}{3.048000in}}%
\pgfpathlineto{\pgfqpoint{3.952000in}{3.048000in}}%
\pgfpathlineto{\pgfqpoint{3.616000in}{3.384000in}}%
\pgfpathlineto{\pgfqpoint{3.616000in}{3.048000in}}%
\pgfusepath{stroke}%
\end{pgfscope}%
\begin{pgfscope}%
\pgfpathrectangle{\pgfqpoint{1.432000in}{0.528000in}}{\pgfqpoint{3.696000in}{3.696000in}}%
\pgfusepath{clip}%
\pgfsetbuttcap%
\pgfsetroundjoin%
\pgfsetlinewidth{1.505625pt}%
\definecolor{currentstroke}{rgb}{0.000000,0.000000,0.000000}%
\pgfsetstrokecolor{currentstroke}%
\pgfsetdash{}{0pt}%
\pgfpathmoveto{\pgfqpoint{3.280000in}{3.384000in}}%
\pgfpathlineto{\pgfqpoint{3.616000in}{3.384000in}}%
\pgfpathlineto{\pgfqpoint{3.280000in}{3.720000in}}%
\pgfpathlineto{\pgfqpoint{3.280000in}{3.384000in}}%
\pgfusepath{stroke}%
\end{pgfscope}%
\begin{pgfscope}%
\pgfpathrectangle{\pgfqpoint{1.432000in}{0.528000in}}{\pgfqpoint{3.696000in}{3.696000in}}%
\pgfusepath{clip}%
\pgfsetbuttcap%
\pgfsetroundjoin%
\pgfsetlinewidth{1.505625pt}%
\definecolor{currentstroke}{rgb}{0.000000,0.000000,0.000000}%
\pgfsetstrokecolor{currentstroke}%
\pgfsetdash{}{0pt}%
\pgfpathmoveto{\pgfqpoint{2.944000in}{3.720000in}}%
\pgfpathlineto{\pgfqpoint{3.280000in}{3.720000in}}%
\pgfpathlineto{\pgfqpoint{2.944000in}{4.056000in}}%
\pgfpathlineto{\pgfqpoint{2.944000in}{3.720000in}}%
\pgfusepath{stroke}%
\end{pgfscope}%
\begin{pgfscope}%
\pgfpathrectangle{\pgfqpoint{1.432000in}{0.528000in}}{\pgfqpoint{3.696000in}{3.696000in}}%
\pgfusepath{clip}%
\pgfsetbuttcap%
\pgfsetroundjoin%
\pgfsetlinewidth{1.505625pt}%
\definecolor{currentstroke}{rgb}{0.000000,0.000000,0.000000}%
\pgfsetstrokecolor{currentstroke}%
\pgfsetdash{}{0pt}%
\pgfpathmoveto{\pgfqpoint{4.960000in}{2.040000in}}%
\pgfpathlineto{\pgfqpoint{4.624000in}{2.376000in}}%
\pgfpathlineto{\pgfqpoint{4.960000in}{2.376000in}}%
\pgfpathlineto{\pgfqpoint{4.960000in}{2.040000in}}%
\pgfusepath{stroke}%
\end{pgfscope}%
\begin{pgfscope}%
\pgfpathrectangle{\pgfqpoint{1.432000in}{0.528000in}}{\pgfqpoint{3.696000in}{3.696000in}}%
\pgfusepath{clip}%
\pgfsetbuttcap%
\pgfsetroundjoin%
\pgfsetlinewidth{1.505625pt}%
\definecolor{currentstroke}{rgb}{0.000000,0.000000,0.000000}%
\pgfsetstrokecolor{currentstroke}%
\pgfsetdash{}{0pt}%
\pgfpathmoveto{\pgfqpoint{4.624000in}{2.376000in}}%
\pgfpathlineto{\pgfqpoint{4.288000in}{2.712000in}}%
\pgfpathlineto{\pgfqpoint{4.624000in}{2.712000in}}%
\pgfpathlineto{\pgfqpoint{4.624000in}{2.376000in}}%
\pgfusepath{stroke}%
\end{pgfscope}%
\begin{pgfscope}%
\pgfpathrectangle{\pgfqpoint{1.432000in}{0.528000in}}{\pgfqpoint{3.696000in}{3.696000in}}%
\pgfusepath{clip}%
\pgfsetbuttcap%
\pgfsetroundjoin%
\pgfsetlinewidth{1.505625pt}%
\definecolor{currentstroke}{rgb}{0.000000,0.000000,0.000000}%
\pgfsetstrokecolor{currentstroke}%
\pgfsetdash{}{0pt}%
\pgfpathmoveto{\pgfqpoint{4.288000in}{2.712000in}}%
\pgfpathlineto{\pgfqpoint{3.952000in}{3.048000in}}%
\pgfpathlineto{\pgfqpoint{4.288000in}{3.048000in}}%
\pgfpathlineto{\pgfqpoint{4.288000in}{2.712000in}}%
\pgfusepath{stroke}%
\end{pgfscope}%
\begin{pgfscope}%
\pgfpathrectangle{\pgfqpoint{1.432000in}{0.528000in}}{\pgfqpoint{3.696000in}{3.696000in}}%
\pgfusepath{clip}%
\pgfsetbuttcap%
\pgfsetroundjoin%
\pgfsetlinewidth{1.505625pt}%
\definecolor{currentstroke}{rgb}{0.000000,0.000000,0.000000}%
\pgfsetstrokecolor{currentstroke}%
\pgfsetdash{}{0pt}%
\pgfpathmoveto{\pgfqpoint{3.952000in}{3.048000in}}%
\pgfpathlineto{\pgfqpoint{3.616000in}{3.384000in}}%
\pgfpathlineto{\pgfqpoint{3.952000in}{3.384000in}}%
\pgfpathlineto{\pgfqpoint{3.952000in}{3.048000in}}%
\pgfusepath{stroke}%
\end{pgfscope}%
\begin{pgfscope}%
\pgfpathrectangle{\pgfqpoint{1.432000in}{0.528000in}}{\pgfqpoint{3.696000in}{3.696000in}}%
\pgfusepath{clip}%
\pgfsetbuttcap%
\pgfsetroundjoin%
\pgfsetlinewidth{1.505625pt}%
\definecolor{currentstroke}{rgb}{0.000000,0.000000,0.000000}%
\pgfsetstrokecolor{currentstroke}%
\pgfsetdash{}{0pt}%
\pgfpathmoveto{\pgfqpoint{3.616000in}{3.384000in}}%
\pgfpathlineto{\pgfqpoint{3.280000in}{3.720000in}}%
\pgfpathlineto{\pgfqpoint{3.616000in}{3.720000in}}%
\pgfpathlineto{\pgfqpoint{3.616000in}{3.384000in}}%
\pgfusepath{stroke}%
\end{pgfscope}%
\begin{pgfscope}%
\pgfpathrectangle{\pgfqpoint{1.432000in}{0.528000in}}{\pgfqpoint{3.696000in}{3.696000in}}%
\pgfusepath{clip}%
\pgfsetbuttcap%
\pgfsetroundjoin%
\pgfsetlinewidth{1.505625pt}%
\definecolor{currentstroke}{rgb}{0.000000,0.000000,0.000000}%
\pgfsetstrokecolor{currentstroke}%
\pgfsetdash{}{0pt}%
\pgfpathmoveto{\pgfqpoint{3.280000in}{3.720000in}}%
\pgfpathlineto{\pgfqpoint{2.944000in}{4.056000in}}%
\pgfpathlineto{\pgfqpoint{3.280000in}{4.056000in}}%
\pgfpathlineto{\pgfqpoint{3.280000in}{3.720000in}}%
\pgfusepath{stroke}%
\end{pgfscope}%
\begin{pgfscope}%
\pgfpathrectangle{\pgfqpoint{1.432000in}{0.528000in}}{\pgfqpoint{3.696000in}{3.696000in}}%
\pgfusepath{clip}%
\pgfsetbuttcap%
\pgfsetroundjoin%
\pgfsetlinewidth{1.505625pt}%
\definecolor{currentstroke}{rgb}{0.000000,0.000000,0.000000}%
\pgfsetstrokecolor{currentstroke}%
\pgfsetdash{}{0pt}%
\pgfpathmoveto{\pgfqpoint{4.624000in}{2.376000in}}%
\pgfpathlineto{\pgfqpoint{4.960000in}{2.376000in}}%
\pgfpathlineto{\pgfqpoint{4.624000in}{2.712000in}}%
\pgfpathlineto{\pgfqpoint{4.624000in}{2.376000in}}%
\pgfusepath{stroke}%
\end{pgfscope}%
\begin{pgfscope}%
\pgfpathrectangle{\pgfqpoint{1.432000in}{0.528000in}}{\pgfqpoint{3.696000in}{3.696000in}}%
\pgfusepath{clip}%
\pgfsetbuttcap%
\pgfsetroundjoin%
\pgfsetlinewidth{1.505625pt}%
\definecolor{currentstroke}{rgb}{0.000000,0.000000,0.000000}%
\pgfsetstrokecolor{currentstroke}%
\pgfsetdash{}{0pt}%
\pgfpathmoveto{\pgfqpoint{4.288000in}{2.712000in}}%
\pgfpathlineto{\pgfqpoint{4.624000in}{2.712000in}}%
\pgfpathlineto{\pgfqpoint{4.288000in}{3.048000in}}%
\pgfpathlineto{\pgfqpoint{4.288000in}{2.712000in}}%
\pgfusepath{stroke}%
\end{pgfscope}%
\begin{pgfscope}%
\pgfpathrectangle{\pgfqpoint{1.432000in}{0.528000in}}{\pgfqpoint{3.696000in}{3.696000in}}%
\pgfusepath{clip}%
\pgfsetbuttcap%
\pgfsetroundjoin%
\pgfsetlinewidth{1.505625pt}%
\definecolor{currentstroke}{rgb}{0.000000,0.000000,0.000000}%
\pgfsetstrokecolor{currentstroke}%
\pgfsetdash{}{0pt}%
\pgfpathmoveto{\pgfqpoint{3.952000in}{3.048000in}}%
\pgfpathlineto{\pgfqpoint{4.288000in}{3.048000in}}%
\pgfpathlineto{\pgfqpoint{3.952000in}{3.384000in}}%
\pgfpathlineto{\pgfqpoint{3.952000in}{3.048000in}}%
\pgfusepath{stroke}%
\end{pgfscope}%
\begin{pgfscope}%
\pgfpathrectangle{\pgfqpoint{1.432000in}{0.528000in}}{\pgfqpoint{3.696000in}{3.696000in}}%
\pgfusepath{clip}%
\pgfsetbuttcap%
\pgfsetroundjoin%
\pgfsetlinewidth{1.505625pt}%
\definecolor{currentstroke}{rgb}{0.000000,0.000000,0.000000}%
\pgfsetstrokecolor{currentstroke}%
\pgfsetdash{}{0pt}%
\pgfpathmoveto{\pgfqpoint{3.616000in}{3.384000in}}%
\pgfpathlineto{\pgfqpoint{3.952000in}{3.384000in}}%
\pgfpathlineto{\pgfqpoint{3.616000in}{3.720000in}}%
\pgfpathlineto{\pgfqpoint{3.616000in}{3.384000in}}%
\pgfusepath{stroke}%
\end{pgfscope}%
\begin{pgfscope}%
\pgfpathrectangle{\pgfqpoint{1.432000in}{0.528000in}}{\pgfqpoint{3.696000in}{3.696000in}}%
\pgfusepath{clip}%
\pgfsetbuttcap%
\pgfsetroundjoin%
\pgfsetlinewidth{1.505625pt}%
\definecolor{currentstroke}{rgb}{0.000000,0.000000,0.000000}%
\pgfsetstrokecolor{currentstroke}%
\pgfsetdash{}{0pt}%
\pgfpathmoveto{\pgfqpoint{3.280000in}{3.720000in}}%
\pgfpathlineto{\pgfqpoint{3.616000in}{3.720000in}}%
\pgfpathlineto{\pgfqpoint{3.280000in}{4.056000in}}%
\pgfpathlineto{\pgfqpoint{3.280000in}{3.720000in}}%
\pgfusepath{stroke}%
\end{pgfscope}%
\begin{pgfscope}%
\pgfpathrectangle{\pgfqpoint{1.432000in}{0.528000in}}{\pgfqpoint{3.696000in}{3.696000in}}%
\pgfusepath{clip}%
\pgfsetbuttcap%
\pgfsetroundjoin%
\pgfsetlinewidth{1.505625pt}%
\definecolor{currentstroke}{rgb}{0.000000,0.000000,0.000000}%
\pgfsetstrokecolor{currentstroke}%
\pgfsetdash{}{0pt}%
\pgfpathmoveto{\pgfqpoint{4.960000in}{2.376000in}}%
\pgfpathlineto{\pgfqpoint{4.624000in}{2.712000in}}%
\pgfpathlineto{\pgfqpoint{4.960000in}{2.712000in}}%
\pgfpathlineto{\pgfqpoint{4.960000in}{2.376000in}}%
\pgfusepath{stroke}%
\end{pgfscope}%
\begin{pgfscope}%
\pgfpathrectangle{\pgfqpoint{1.432000in}{0.528000in}}{\pgfqpoint{3.696000in}{3.696000in}}%
\pgfusepath{clip}%
\pgfsetbuttcap%
\pgfsetroundjoin%
\pgfsetlinewidth{1.505625pt}%
\definecolor{currentstroke}{rgb}{0.000000,0.000000,0.000000}%
\pgfsetstrokecolor{currentstroke}%
\pgfsetdash{}{0pt}%
\pgfpathmoveto{\pgfqpoint{4.624000in}{2.712000in}}%
\pgfpathlineto{\pgfqpoint{4.288000in}{3.048000in}}%
\pgfpathlineto{\pgfqpoint{4.624000in}{3.048000in}}%
\pgfpathlineto{\pgfqpoint{4.624000in}{2.712000in}}%
\pgfusepath{stroke}%
\end{pgfscope}%
\begin{pgfscope}%
\pgfpathrectangle{\pgfqpoint{1.432000in}{0.528000in}}{\pgfqpoint{3.696000in}{3.696000in}}%
\pgfusepath{clip}%
\pgfsetbuttcap%
\pgfsetroundjoin%
\pgfsetlinewidth{1.505625pt}%
\definecolor{currentstroke}{rgb}{0.000000,0.000000,0.000000}%
\pgfsetstrokecolor{currentstroke}%
\pgfsetdash{}{0pt}%
\pgfpathmoveto{\pgfqpoint{4.288000in}{3.048000in}}%
\pgfpathlineto{\pgfqpoint{3.952000in}{3.384000in}}%
\pgfpathlineto{\pgfqpoint{4.288000in}{3.384000in}}%
\pgfpathlineto{\pgfqpoint{4.288000in}{3.048000in}}%
\pgfusepath{stroke}%
\end{pgfscope}%
\begin{pgfscope}%
\pgfpathrectangle{\pgfqpoint{1.432000in}{0.528000in}}{\pgfqpoint{3.696000in}{3.696000in}}%
\pgfusepath{clip}%
\pgfsetbuttcap%
\pgfsetroundjoin%
\pgfsetlinewidth{1.505625pt}%
\definecolor{currentstroke}{rgb}{0.000000,0.000000,0.000000}%
\pgfsetstrokecolor{currentstroke}%
\pgfsetdash{}{0pt}%
\pgfpathmoveto{\pgfqpoint{3.952000in}{3.384000in}}%
\pgfpathlineto{\pgfqpoint{3.616000in}{3.720000in}}%
\pgfpathlineto{\pgfqpoint{3.952000in}{3.720000in}}%
\pgfpathlineto{\pgfqpoint{3.952000in}{3.384000in}}%
\pgfusepath{stroke}%
\end{pgfscope}%
\begin{pgfscope}%
\pgfpathrectangle{\pgfqpoint{1.432000in}{0.528000in}}{\pgfqpoint{3.696000in}{3.696000in}}%
\pgfusepath{clip}%
\pgfsetbuttcap%
\pgfsetroundjoin%
\pgfsetlinewidth{1.505625pt}%
\definecolor{currentstroke}{rgb}{0.000000,0.000000,0.000000}%
\pgfsetstrokecolor{currentstroke}%
\pgfsetdash{}{0pt}%
\pgfpathmoveto{\pgfqpoint{3.616000in}{3.720000in}}%
\pgfpathlineto{\pgfqpoint{3.280000in}{4.056000in}}%
\pgfpathlineto{\pgfqpoint{3.616000in}{4.056000in}}%
\pgfpathlineto{\pgfqpoint{3.616000in}{3.720000in}}%
\pgfusepath{stroke}%
\end{pgfscope}%
\begin{pgfscope}%
\pgfpathrectangle{\pgfqpoint{1.432000in}{0.528000in}}{\pgfqpoint{3.696000in}{3.696000in}}%
\pgfusepath{clip}%
\pgfsetbuttcap%
\pgfsetroundjoin%
\pgfsetlinewidth{1.505625pt}%
\definecolor{currentstroke}{rgb}{0.000000,0.000000,0.000000}%
\pgfsetstrokecolor{currentstroke}%
\pgfsetdash{}{0pt}%
\pgfpathmoveto{\pgfqpoint{4.624000in}{2.712000in}}%
\pgfpathlineto{\pgfqpoint{4.960000in}{2.712000in}}%
\pgfpathlineto{\pgfqpoint{4.624000in}{3.048000in}}%
\pgfpathlineto{\pgfqpoint{4.624000in}{2.712000in}}%
\pgfusepath{stroke}%
\end{pgfscope}%
\begin{pgfscope}%
\pgfpathrectangle{\pgfqpoint{1.432000in}{0.528000in}}{\pgfqpoint{3.696000in}{3.696000in}}%
\pgfusepath{clip}%
\pgfsetbuttcap%
\pgfsetroundjoin%
\pgfsetlinewidth{1.505625pt}%
\definecolor{currentstroke}{rgb}{0.000000,0.000000,0.000000}%
\pgfsetstrokecolor{currentstroke}%
\pgfsetdash{}{0pt}%
\pgfpathmoveto{\pgfqpoint{4.288000in}{3.048000in}}%
\pgfpathlineto{\pgfqpoint{4.624000in}{3.048000in}}%
\pgfpathlineto{\pgfqpoint{4.288000in}{3.384000in}}%
\pgfpathlineto{\pgfqpoint{4.288000in}{3.048000in}}%
\pgfusepath{stroke}%
\end{pgfscope}%
\begin{pgfscope}%
\pgfpathrectangle{\pgfqpoint{1.432000in}{0.528000in}}{\pgfqpoint{3.696000in}{3.696000in}}%
\pgfusepath{clip}%
\pgfsetbuttcap%
\pgfsetroundjoin%
\pgfsetlinewidth{1.505625pt}%
\definecolor{currentstroke}{rgb}{0.000000,0.000000,0.000000}%
\pgfsetstrokecolor{currentstroke}%
\pgfsetdash{}{0pt}%
\pgfpathmoveto{\pgfqpoint{3.952000in}{3.384000in}}%
\pgfpathlineto{\pgfqpoint{4.288000in}{3.384000in}}%
\pgfpathlineto{\pgfqpoint{3.952000in}{3.720000in}}%
\pgfpathlineto{\pgfqpoint{3.952000in}{3.384000in}}%
\pgfusepath{stroke}%
\end{pgfscope}%
\begin{pgfscope}%
\pgfpathrectangle{\pgfqpoint{1.432000in}{0.528000in}}{\pgfqpoint{3.696000in}{3.696000in}}%
\pgfusepath{clip}%
\pgfsetbuttcap%
\pgfsetroundjoin%
\pgfsetlinewidth{1.505625pt}%
\definecolor{currentstroke}{rgb}{0.000000,0.000000,0.000000}%
\pgfsetstrokecolor{currentstroke}%
\pgfsetdash{}{0pt}%
\pgfpathmoveto{\pgfqpoint{3.616000in}{3.720000in}}%
\pgfpathlineto{\pgfqpoint{3.952000in}{3.720000in}}%
\pgfpathlineto{\pgfqpoint{3.616000in}{4.056000in}}%
\pgfpathlineto{\pgfqpoint{3.616000in}{3.720000in}}%
\pgfusepath{stroke}%
\end{pgfscope}%
\begin{pgfscope}%
\pgfpathrectangle{\pgfqpoint{1.432000in}{0.528000in}}{\pgfqpoint{3.696000in}{3.696000in}}%
\pgfusepath{clip}%
\pgfsetbuttcap%
\pgfsetroundjoin%
\pgfsetlinewidth{1.505625pt}%
\definecolor{currentstroke}{rgb}{0.000000,0.000000,0.000000}%
\pgfsetstrokecolor{currentstroke}%
\pgfsetdash{}{0pt}%
\pgfpathmoveto{\pgfqpoint{4.960000in}{2.712000in}}%
\pgfpathlineto{\pgfqpoint{4.624000in}{3.048000in}}%
\pgfpathlineto{\pgfqpoint{4.960000in}{3.048000in}}%
\pgfpathlineto{\pgfqpoint{4.960000in}{2.712000in}}%
\pgfusepath{stroke}%
\end{pgfscope}%
\begin{pgfscope}%
\pgfpathrectangle{\pgfqpoint{1.432000in}{0.528000in}}{\pgfqpoint{3.696000in}{3.696000in}}%
\pgfusepath{clip}%
\pgfsetbuttcap%
\pgfsetroundjoin%
\pgfsetlinewidth{1.505625pt}%
\definecolor{currentstroke}{rgb}{0.000000,0.000000,0.000000}%
\pgfsetstrokecolor{currentstroke}%
\pgfsetdash{}{0pt}%
\pgfpathmoveto{\pgfqpoint{4.624000in}{3.048000in}}%
\pgfpathlineto{\pgfqpoint{4.288000in}{3.384000in}}%
\pgfpathlineto{\pgfqpoint{4.624000in}{3.384000in}}%
\pgfpathlineto{\pgfqpoint{4.624000in}{3.048000in}}%
\pgfusepath{stroke}%
\end{pgfscope}%
\begin{pgfscope}%
\pgfpathrectangle{\pgfqpoint{1.432000in}{0.528000in}}{\pgfqpoint{3.696000in}{3.696000in}}%
\pgfusepath{clip}%
\pgfsetbuttcap%
\pgfsetroundjoin%
\pgfsetlinewidth{1.505625pt}%
\definecolor{currentstroke}{rgb}{0.000000,0.000000,0.000000}%
\pgfsetstrokecolor{currentstroke}%
\pgfsetdash{}{0pt}%
\pgfpathmoveto{\pgfqpoint{4.288000in}{3.384000in}}%
\pgfpathlineto{\pgfqpoint{3.952000in}{3.720000in}}%
\pgfpathlineto{\pgfqpoint{4.288000in}{3.720000in}}%
\pgfpathlineto{\pgfqpoint{4.288000in}{3.384000in}}%
\pgfusepath{stroke}%
\end{pgfscope}%
\begin{pgfscope}%
\pgfpathrectangle{\pgfqpoint{1.432000in}{0.528000in}}{\pgfqpoint{3.696000in}{3.696000in}}%
\pgfusepath{clip}%
\pgfsetbuttcap%
\pgfsetroundjoin%
\pgfsetlinewidth{1.505625pt}%
\definecolor{currentstroke}{rgb}{0.000000,0.000000,0.000000}%
\pgfsetstrokecolor{currentstroke}%
\pgfsetdash{}{0pt}%
\pgfpathmoveto{\pgfqpoint{3.952000in}{3.720000in}}%
\pgfpathlineto{\pgfqpoint{3.616000in}{4.056000in}}%
\pgfpathlineto{\pgfqpoint{3.952000in}{4.056000in}}%
\pgfpathlineto{\pgfqpoint{3.952000in}{3.720000in}}%
\pgfusepath{stroke}%
\end{pgfscope}%
\begin{pgfscope}%
\pgfpathrectangle{\pgfqpoint{1.432000in}{0.528000in}}{\pgfqpoint{3.696000in}{3.696000in}}%
\pgfusepath{clip}%
\pgfsetbuttcap%
\pgfsetroundjoin%
\pgfsetlinewidth{1.505625pt}%
\definecolor{currentstroke}{rgb}{0.000000,0.000000,0.000000}%
\pgfsetstrokecolor{currentstroke}%
\pgfsetdash{}{0pt}%
\pgfpathmoveto{\pgfqpoint{4.624000in}{3.048000in}}%
\pgfpathlineto{\pgfqpoint{4.960000in}{3.048000in}}%
\pgfpathlineto{\pgfqpoint{4.624000in}{3.384000in}}%
\pgfpathlineto{\pgfqpoint{4.624000in}{3.048000in}}%
\pgfusepath{stroke}%
\end{pgfscope}%
\begin{pgfscope}%
\pgfpathrectangle{\pgfqpoint{1.432000in}{0.528000in}}{\pgfqpoint{3.696000in}{3.696000in}}%
\pgfusepath{clip}%
\pgfsetbuttcap%
\pgfsetroundjoin%
\pgfsetlinewidth{1.505625pt}%
\definecolor{currentstroke}{rgb}{0.000000,0.000000,0.000000}%
\pgfsetstrokecolor{currentstroke}%
\pgfsetdash{}{0pt}%
\pgfpathmoveto{\pgfqpoint{4.288000in}{3.384000in}}%
\pgfpathlineto{\pgfqpoint{4.624000in}{3.384000in}}%
\pgfpathlineto{\pgfqpoint{4.288000in}{3.720000in}}%
\pgfpathlineto{\pgfqpoint{4.288000in}{3.384000in}}%
\pgfusepath{stroke}%
\end{pgfscope}%
\begin{pgfscope}%
\pgfpathrectangle{\pgfqpoint{1.432000in}{0.528000in}}{\pgfqpoint{3.696000in}{3.696000in}}%
\pgfusepath{clip}%
\pgfsetbuttcap%
\pgfsetroundjoin%
\pgfsetlinewidth{1.505625pt}%
\definecolor{currentstroke}{rgb}{0.000000,0.000000,0.000000}%
\pgfsetstrokecolor{currentstroke}%
\pgfsetdash{}{0pt}%
\pgfpathmoveto{\pgfqpoint{3.952000in}{3.720000in}}%
\pgfpathlineto{\pgfqpoint{4.288000in}{3.720000in}}%
\pgfpathlineto{\pgfqpoint{3.952000in}{4.056000in}}%
\pgfpathlineto{\pgfqpoint{3.952000in}{3.720000in}}%
\pgfusepath{stroke}%
\end{pgfscope}%
\begin{pgfscope}%
\pgfpathrectangle{\pgfqpoint{1.432000in}{0.528000in}}{\pgfqpoint{3.696000in}{3.696000in}}%
\pgfusepath{clip}%
\pgfsetbuttcap%
\pgfsetroundjoin%
\pgfsetlinewidth{1.505625pt}%
\definecolor{currentstroke}{rgb}{0.000000,0.000000,0.000000}%
\pgfsetstrokecolor{currentstroke}%
\pgfsetdash{}{0pt}%
\pgfpathmoveto{\pgfqpoint{4.960000in}{3.048000in}}%
\pgfpathlineto{\pgfqpoint{4.624000in}{3.384000in}}%
\pgfpathlineto{\pgfqpoint{4.960000in}{3.384000in}}%
\pgfpathlineto{\pgfqpoint{4.960000in}{3.048000in}}%
\pgfusepath{stroke}%
\end{pgfscope}%
\begin{pgfscope}%
\pgfpathrectangle{\pgfqpoint{1.432000in}{0.528000in}}{\pgfqpoint{3.696000in}{3.696000in}}%
\pgfusepath{clip}%
\pgfsetbuttcap%
\pgfsetroundjoin%
\pgfsetlinewidth{1.505625pt}%
\definecolor{currentstroke}{rgb}{0.000000,0.000000,0.000000}%
\pgfsetstrokecolor{currentstroke}%
\pgfsetdash{}{0pt}%
\pgfpathmoveto{\pgfqpoint{4.624000in}{3.384000in}}%
\pgfpathlineto{\pgfqpoint{4.288000in}{3.720000in}}%
\pgfpathlineto{\pgfqpoint{4.624000in}{3.720000in}}%
\pgfpathlineto{\pgfqpoint{4.624000in}{3.384000in}}%
\pgfusepath{stroke}%
\end{pgfscope}%
\begin{pgfscope}%
\pgfpathrectangle{\pgfqpoint{1.432000in}{0.528000in}}{\pgfqpoint{3.696000in}{3.696000in}}%
\pgfusepath{clip}%
\pgfsetbuttcap%
\pgfsetroundjoin%
\pgfsetlinewidth{1.505625pt}%
\definecolor{currentstroke}{rgb}{0.000000,0.000000,0.000000}%
\pgfsetstrokecolor{currentstroke}%
\pgfsetdash{}{0pt}%
\pgfpathmoveto{\pgfqpoint{4.288000in}{3.720000in}}%
\pgfpathlineto{\pgfqpoint{3.952000in}{4.056000in}}%
\pgfpathlineto{\pgfqpoint{4.288000in}{4.056000in}}%
\pgfpathlineto{\pgfqpoint{4.288000in}{3.720000in}}%
\pgfusepath{stroke}%
\end{pgfscope}%
\begin{pgfscope}%
\pgfpathrectangle{\pgfqpoint{1.432000in}{0.528000in}}{\pgfqpoint{3.696000in}{3.696000in}}%
\pgfusepath{clip}%
\pgfsetbuttcap%
\pgfsetroundjoin%
\pgfsetlinewidth{1.505625pt}%
\definecolor{currentstroke}{rgb}{0.000000,0.000000,0.000000}%
\pgfsetstrokecolor{currentstroke}%
\pgfsetdash{}{0pt}%
\pgfpathmoveto{\pgfqpoint{4.624000in}{3.384000in}}%
\pgfpathlineto{\pgfqpoint{4.960000in}{3.384000in}}%
\pgfpathlineto{\pgfqpoint{4.624000in}{3.720000in}}%
\pgfpathlineto{\pgfqpoint{4.624000in}{3.384000in}}%
\pgfusepath{stroke}%
\end{pgfscope}%
\begin{pgfscope}%
\pgfpathrectangle{\pgfqpoint{1.432000in}{0.528000in}}{\pgfqpoint{3.696000in}{3.696000in}}%
\pgfusepath{clip}%
\pgfsetbuttcap%
\pgfsetroundjoin%
\pgfsetlinewidth{1.505625pt}%
\definecolor{currentstroke}{rgb}{0.000000,0.000000,0.000000}%
\pgfsetstrokecolor{currentstroke}%
\pgfsetdash{}{0pt}%
\pgfpathmoveto{\pgfqpoint{4.288000in}{3.720000in}}%
\pgfpathlineto{\pgfqpoint{4.624000in}{3.720000in}}%
\pgfpathlineto{\pgfqpoint{4.288000in}{4.056000in}}%
\pgfpathlineto{\pgfqpoint{4.288000in}{3.720000in}}%
\pgfusepath{stroke}%
\end{pgfscope}%
\begin{pgfscope}%
\pgfpathrectangle{\pgfqpoint{1.432000in}{0.528000in}}{\pgfqpoint{3.696000in}{3.696000in}}%
\pgfusepath{clip}%
\pgfsetbuttcap%
\pgfsetroundjoin%
\pgfsetlinewidth{1.505625pt}%
\definecolor{currentstroke}{rgb}{0.000000,0.000000,0.000000}%
\pgfsetstrokecolor{currentstroke}%
\pgfsetdash{}{0pt}%
\pgfpathmoveto{\pgfqpoint{4.960000in}{3.384000in}}%
\pgfpathlineto{\pgfqpoint{4.624000in}{3.720000in}}%
\pgfpathlineto{\pgfqpoint{4.960000in}{3.720000in}}%
\pgfpathlineto{\pgfqpoint{4.960000in}{3.384000in}}%
\pgfusepath{stroke}%
\end{pgfscope}%
\begin{pgfscope}%
\pgfpathrectangle{\pgfqpoint{1.432000in}{0.528000in}}{\pgfqpoint{3.696000in}{3.696000in}}%
\pgfusepath{clip}%
\pgfsetbuttcap%
\pgfsetroundjoin%
\pgfsetlinewidth{1.505625pt}%
\definecolor{currentstroke}{rgb}{0.000000,0.000000,0.000000}%
\pgfsetstrokecolor{currentstroke}%
\pgfsetdash{}{0pt}%
\pgfpathmoveto{\pgfqpoint{4.624000in}{3.720000in}}%
\pgfpathlineto{\pgfqpoint{4.288000in}{4.056000in}}%
\pgfpathlineto{\pgfqpoint{4.624000in}{4.056000in}}%
\pgfpathlineto{\pgfqpoint{4.624000in}{3.720000in}}%
\pgfusepath{stroke}%
\end{pgfscope}%
\begin{pgfscope}%
\pgfpathrectangle{\pgfqpoint{1.432000in}{0.528000in}}{\pgfqpoint{3.696000in}{3.696000in}}%
\pgfusepath{clip}%
\pgfsetbuttcap%
\pgfsetroundjoin%
\pgfsetlinewidth{1.505625pt}%
\definecolor{currentstroke}{rgb}{0.000000,0.000000,0.000000}%
\pgfsetstrokecolor{currentstroke}%
\pgfsetdash{}{0pt}%
\pgfpathmoveto{\pgfqpoint{4.624000in}{3.720000in}}%
\pgfpathlineto{\pgfqpoint{4.960000in}{3.720000in}}%
\pgfpathlineto{\pgfqpoint{4.624000in}{4.056000in}}%
\pgfpathlineto{\pgfqpoint{4.624000in}{3.720000in}}%
\pgfusepath{stroke}%
\end{pgfscope}%
\begin{pgfscope}%
\pgfpathrectangle{\pgfqpoint{1.432000in}{0.528000in}}{\pgfqpoint{3.696000in}{3.696000in}}%
\pgfusepath{clip}%
\pgfsetbuttcap%
\pgfsetroundjoin%
\pgfsetlinewidth{1.505625pt}%
\definecolor{currentstroke}{rgb}{0.000000,0.000000,0.000000}%
\pgfsetstrokecolor{currentstroke}%
\pgfsetdash{}{0pt}%
\pgfpathmoveto{\pgfqpoint{4.960000in}{3.720000in}}%
\pgfpathlineto{\pgfqpoint{4.624000in}{4.056000in}}%
\pgfpathlineto{\pgfqpoint{4.960000in}{4.056000in}}%
\pgfpathlineto{\pgfqpoint{4.960000in}{3.720000in}}%
\pgfusepath{stroke}%
\end{pgfscope}%
\begin{pgfscope}%
\pgfsetrectcap%
\pgfsetmiterjoin%
\pgfsetlinewidth{0.000000pt}%
\definecolor{currentstroke}{rgb}{0.000000,0.000000,0.000000}%
\pgfsetstrokecolor{currentstroke}%
\pgfsetstrokeopacity{0.000000}%
\pgfsetdash{}{0pt}%
\pgfpathmoveto{\pgfqpoint{1.432000in}{0.528000in}}%
\pgfpathlineto{\pgfqpoint{1.432000in}{4.224000in}}%
\pgfusepath{}%
\end{pgfscope}%
\begin{pgfscope}%
\pgfsetrectcap%
\pgfsetmiterjoin%
\pgfsetlinewidth{0.000000pt}%
\definecolor{currentstroke}{rgb}{0.000000,0.000000,0.000000}%
\pgfsetstrokecolor{currentstroke}%
\pgfsetstrokeopacity{0.000000}%
\pgfsetdash{}{0pt}%
\pgfpathmoveto{\pgfqpoint{5.128000in}{0.528000in}}%
\pgfpathlineto{\pgfqpoint{5.128000in}{4.224000in}}%
\pgfusepath{}%
\end{pgfscope}%
\begin{pgfscope}%
\pgfsetrectcap%
\pgfsetmiterjoin%
\pgfsetlinewidth{0.000000pt}%
\definecolor{currentstroke}{rgb}{0.000000,0.000000,0.000000}%
\pgfsetstrokecolor{currentstroke}%
\pgfsetstrokeopacity{0.000000}%
\pgfsetdash{}{0pt}%
\pgfpathmoveto{\pgfqpoint{1.432000in}{0.528000in}}%
\pgfpathlineto{\pgfqpoint{5.128000in}{0.528000in}}%
\pgfusepath{}%
\end{pgfscope}%
\begin{pgfscope}%
\pgfsetrectcap%
\pgfsetmiterjoin%
\pgfsetlinewidth{0.000000pt}%
\definecolor{currentstroke}{rgb}{0.000000,0.000000,0.000000}%
\pgfsetstrokecolor{currentstroke}%
\pgfsetstrokeopacity{0.000000}%
\pgfsetdash{}{0pt}%
\pgfpathmoveto{\pgfqpoint{1.432000in}{4.224000in}}%
\pgfpathlineto{\pgfqpoint{5.128000in}{4.224000in}}%
\pgfusepath{}%
\end{pgfscope}%
\end{pgfpicture}%
\makeatother%
\endgroup%

  \caption{A sample mesh, similar to those used in all the computations in this thesis.\label{fig:grid}}
  \end{figure}

\paragraph{Numerical setup}
All finite-element calculations were carried out using the Firedrake software library \cite{RaHaMiLaLuMcBeMaKe:16,LuVaRaBeRaHaKe:15}, which uses PETSc to perform its linear solves \cite{BaAbAsBrBrBuDaEiGrKaKnMaMcMiMuRuSaSmZaZhZh:18,BaGrMcSm:97,DaPaKlCo:11,BaAbAdBrBrBuDaDeEiGrKaKaKnMaMcMiMuRuSaSmZaZh:19} and Chaco \cite{HeLe:95} to perform graph partitions. PETSc uses the MUMPS \cite{AmDuLEKo:01,AmGuLEPr:06} solver to perform $LU$ factorisations and direct solves. When GMRES was used, the stopping criterion was a relative error (relative to the 2-norm of the right-hand side) of $10^{-5}$ or an absolute error of $10^{-50}.$ To generated QMC points, we made use of Dirk Nuyen's `Magic Point Shop' code \cite{Nu,KuNu:16}, which uses a base-2 lattice sequence with generating vector from \cite{CoKuNu:06}. Many of the computations were carried out on the Balena High Performance Computing (HPC) Service at the University of Bath.


\bibliographystyle{plain}
\bibliography{biblio_thesis}

\end{document}
