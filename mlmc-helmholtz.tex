%%%%%%% Generic Stuff %%%%%%%%
\documentclass[12pt]{article}

%%%%%%% Page Setup %%%%%%%
\usepackage[a4paper]{geometry}
\geometry{left={2cm}, right={2cm}, top={2cm}, bottom={2cm}}


%%%%%%% Other Packages %%%%%%%
\usepackage{mathtools} 
\usepackage{thmtools} 
\usepackage{hyperref} 
%\usepackage{color} 
\usepackage{amssymb} 
\usepackage{amsthm}
%\usepackage{tikz} 
%\usepgflibrary{shapes.misc}
%\usetikzlibrary{positioning}
%\usetikzlibrary{matrix} 
%\usetikzlibrary{patterns}
%\usepackage{enumerate} % Allows letters in enumeration
\usepackage{mleftright} % Fixes occasional bugs with \left and \right introducing extra spacing
%\usepackage{bbm} % Allows blackboard bold 1
%\usepackage{enumitem}
\usepackage[capitalise]{cleveref}
\usepackage{showkeys}
\usepackage[show]{ed}
\usepackage{cases} % Gives numcases environment

\newcommand{\Ml}{M_{l}}
\newcommand{\Mlmo}{M_{l-1}}
\newcommand{\Qhltilde}{\widetilde{\QoI}_{\hl}}
\newcommand{\Qhlmotilde}{\widetilde{\QoI}_{\hlmo}}
\newcommand{\QMl}{CHANGE\QoI_{\hl}}
\newcommand{\QMlmo}{CHANGE\QoI_{\hlmo}}
\newcommand{\QoI}{Q}
\newcommand{\uhl}{u_{l}}
\newcommand{\hl}{h_{l}}
\newcommand{\Yl}{Y_{l}}
\newcommand{\Ym}{Y_{m}}
\newcommand{\Yz}{Y_{0}}
\newcommand{\Qhztilde}{\widetilde{\QoI}_{\hz}}
\newcommand{\QhLtilde}{\widetilde{\QoI}_{\hL}}
\newcommand{\QMz}{CHANGE\QoI_{\hz}}
\newcommand{\Mz}{M_{0}}
\newcommand{\Ylhat}{\hat{\Yl}}
\newcommand{\Nl}{N_{l}}
\newcommand{\Ylj}{\Yl^{(j)}}
\newcommand{\QhatMLML}{CHANGE\hat{\QoI}^{ML}_{\hL}}
\newcommand{\QhatMLhL}{\hat{\QoI}^{ML}_{\hL}}
\newcommand{\QhatMC}{\hat{\QoI}^{MC}_{h}}
\newcommand{\ML}{M_{L}}
%\newcommand{\co}{c_{1}}
%\newcommand{\ct}{c_{2}}
\newcommand{\cth}{c_{3}}
\newcommand{\Cl}{\cC_{l}}
\newcommand{\Vl}{V_{l}}
\newcommand{\err}[1]{\mathrm{Err}\mleft(#1\mright)}
\newcommand{\CMLML}{CHANGE\cC^{ML}_{M_{L}}}
\newcommand{\CMLhL}{\cC^{ML}_{\hL}}
\renewcommand{\ln}[1]{\mathrm{ln}\mleft(#1\mright)}
\newcommand{\Cppw}{\Ccoarse}
\newcommand{\CFEM}{C_{FEM}}
\newcommand{\uinfty}{u_{\infty}}
\newcommand{\uS}{u^{S}}
%\newcommand{\uI}{u^{I}}
\newcommand{\CMh}{C_{Mh}}
\newcommand{\hlmo}{h_{l-1}}
\newcommand{\uhlmo}{u_{l-1}}
\newcommand{\hz}{h_{0}}
\newcommand{\NMC}{N_{MC}}
\newcommand{\QMLMC}{Q_{\MLMC}}
\newcommand{\MLMC}{M_{\LMC}}
%\newcommand{\LMC}{L_{MC}}
\newcommand{\LMC}{\tilde{L}}
\newcommand{\V}{\cV}
\newcommand{\CMC}{\Cost{\QhatMC}}
\newcommand{\hL}{h_{L}}
\newcommand{\hMC}{h_{MC}}
%\newcommand{\ceil}[1]{\lceil#1\rceil}
\newcommand{\Vj}{V_{j}}
%\newcommand{\Cj}{\cC_{j}}
\newcommand{\errQhatMLhL}{\err{\QhatMLhL}}
\newcommand{\QhL}{Q_{\hL}}
\newcommand{\hLMC}{h_{LMC}}
\newcommand{\Ccoarse}{C_{\mathrm{coarse}}}
\newcommand{\coarseexp}{\mu}
\newcommand{\splitting}{\chi}
\newcommand{\func}{\eta}
%\newcommand{\Do}{D_{1}}
%\newcommand{\Dt}{D_{2}}
%\newcommand{\muo}{\mu_{1}}
%\newcommand{\mut}{\mu_{2}}
%\newcommand{\Az}{A_{0}}
%\newcommand{\nz}{n_{0}}
%\newcommand{\bxz}{\bx_{0}}
\newcommand{\LD}{L_{D}}
%\newcommand{\LI}{L_{I}}
\newcommand{\aD}{a_{D}}
\newcommand{\aI}{a_{I}}
\newcommand{\NLtOLtGD}[1]{\N{#1}_{\LtOLtGD}}
\newcommand{\LtOLtGD}{\Lt{\Omega;\LtGD}}
\newcommand{\Cth}{C_{3}}
\newcommand{\Cthtilde}{\widetilde{C}_{3}}
%\newcommand{\NLtOLtGI}[1]{\N{#1}_{\LtOLtGI}}
%\newcommand{\LtOLtGI}{\Lt{\Omega;\LtGI}}
\newcommand{\Cf}{C_{4}}
\newcommand{\Cfi}{C_{5}}
\newcommand{\hetbeta}{\zeta}
\newcommand{\Cotilde}{\widetilde{C}_{1}}
\newcommand{\nmaxGI}{n_{\max,\GI}}
\newcommand{\Coo}{C^{1,1}}
\newcommand{\epsbx}{\eps_{\bx}}
%\newcommand{\Ball}[2]{B_{#1}\mleft(#2\mright)}
%\newcommand{\bzero}{\mathbf{0}}
%\newcommand{\IPGI}[2]{\IP{#1}{#2}_{\GI}}
%\newcommand{\HmhGI}{\Hmh{\GI}}
%\newcommand{\Hmh}[1]{H^{-1/2}\mleft(#1\mright)}
%\newcommand{\dd}{\mathrm{d}}
%\newcommand{\Leb}{\lambda}
%\newcommand{\LiDp}{\Li{\Dp}}
%\newcommand{\LiDpCC}{\Li{\Dp;\CC}}
%\newcommand{\compcont}{\subset\subset}
%\newcommand{\LtDp}{\Lt{\Dp}}
%\newcommand{\LtDpRR}{\Lt{\Dp;\RR}}
%\newcommand{\LiDpRR}{\Li{\Dp;\RR}}
%% \newcommand{\vtb}{\overline{v}_{2}}
%% \newcommand{\LtOHozDD}{\Lt{\Omega;\HozDD}}
%% \newcommand{\LtOLtD}{\Lt{\Omega;\LtD}}
%% \newcommand{\LiOLiDRR}{\Li{\Omega;\LiDRR}}
%% \newcommand{\WoiDRR}{\Woi{D;\RR}}
%% \newcommand{\Woi}[1]{W^{1,\infty}\mleft(#1\mright)}
%% \newcommand{\LtO}{\Lt{\Omega}}
%% \newcommand{\LfO}{\Lf{\Omega}}
%% \newcommand{\Lf}[1]{L^{4}\mleft(#1\mright)}
%% \newcommand{\NLtOLtD}[1]{\N{#1}_{\LtOLtD}}
%% \newcommand{\NLoO}[1]{\N{#1}_{\LoO}}
%% \newcommand{\LoO}{\Lo{\Omega}}
%% \newcommand{\Lo}[1]{L^{1}\mleft(#1\mright)}
%% \newcommand{\Yj}{Y_{j}}
%% \newcommand{\psij}{\psi_{j}}
%% \newcommand{\Unif}{\mathrm{unif}}
%% \newcommand{\NTn}[1]{\mathrm{NT}_{n}\mleft(#1\mright)}
\newcommand{\Lconst}{C_{L}}
%\newcommand{\ud}{u_{d}}
%% \DeclareMathOperator{\esssup}{ess\,sup}
%\newcommand{\us}{u_{s}}
%\newcommand{\Htht}[1]{H^{3/2}\mleft(#1\mright)}
\newcommand{\HthtGD}{\Htht{\GD}}
\newcommand{\HthtD}{\Htht{D}}
\newcommand{\HozDDprime}{\HozDD'}
%\newcommand{\HhGD}{\Hh{\GD}}
%\newcommand{\WoiDRRdtd}{\Woi{D;\RRdtd}}
\newcommand{\appspace}{U} % Space used in the appendix
\newcommand{\WspU}{\Wsp{\appspace}}
\newcommand{\Wsp}[1]{W^{s,p}\mleft(#1\mright)}
\newcommand{\WsmoppdU}{\Wsmopp{\dU}}
\newcommand{\dU}{\partial \appspace}
\newcommand{\Wsmopp}[1]{W^{s-\frac1p,p}\mleft(#1\mright)}
\newcommand{\udtilde}{\widetilde{u}_{d}}
\newcommand{\WspRRdsmDm}{\Wsp{\RRd\setminus\Dm}}
\newcommand{\Qhtilde}{\widetilde{Q}_{h}}
\newcommand{\Cmesh}{C_{\mathrm{mesh}}}
\newcommand{\mesh}{HERE DAWG}
%\newcommand{\RRp}{\RR^{+}}
\newcommand{\hmaxomega}{h^{\max}_{\omega}}
\newcommand{\homega}{h_{\omega}}
\newcommand{\hzomega}{{h_0}_{\omega}}
\newcommand{\Qtildehomega}{\widetilde{Q}_{\homega}}
\newcommand{\Qh}{Q_{h}}
\newcommand{\Ch}{\cC_{h}}
\newcommand{\cthtilde}{\widetilde{c}_{3}}
\newcommand{\cotilde}{\widetilde{c}_{1}}
\newcommand{\Cost}[1]{\cC\mleft(#1\mright)}
\newcommand{\LpO}{\Lp{\Omega}}
%\newcommand{\Lp}[1]{L^{p}\mleft(#1\mright)}
\newcommand{\Lqgamma}[1]{L^{q\gamma}\mleft(#1\mright)}
\newcommand{\LqgammaO}{\Lqgamma{\Omega}}
\newcommand{\NLpO}[1]{\N{#1}_{\LpO}}
\newcommand{\NLqgammaO}[1]{\N{#1}_{\LqgammaO}}
\newcommand{\Qhomegatilde}{\widetilde{Q}_{\homega}}
%\newcommand{\ddPPomega}{\dd\PP(\omega)}
\newcommand{\csum}[1]{C_{\mathrm{sum},#1}}
\newcommand{\csumdelta}{\csum{\delta}}
\newcommand{\csumgammambetat}{\csum{\frac{\gamma-\beta}{2}}}
\newcommand{\Cgammagtrbeta}{C_{\gamma > \beta}}
\newcommand{\Cgammalessbeta}{C_{\gamma < \beta}}
\newcommand{\Qhtildesj}{\Qhtilde^{(j)}}
\newcommand{\Omegabad}{\Omega_{\mathrm{bad}}}
\newcommand{\omegasj}{\omega^{(j)}}
\newcommand{\ho}{h_{1}}
\newcommand{\uhL}{u_{\hL}}
\newcommand{\uhz}{u_{\hz}}
\newcommand{\Yzhat}{\Yhat_{0}}
\newcommand{\Nz}{N_{0}}
\newcommand{\NL}{N_{L}}
\newcommand{\kh}{k_{h}}
\newcommand{\Qhl}{Q_{\hl}}
\newcommand{\Qhlmo}{Q_{\hlmo}}
\newcommand{\Qhat}{\hat{Q}}
\newcommand{\Yhatz}{\Yhat_{0}}
\newcommand{\Qhz}{Q_{\hz}}
\newcommand{\Yhatl}{\Yhat_{l}}
\newcommand{\NHo}[1]{\N{#1}_{H^{1}}}
\newcommand{\omegaso}{\omega^{(1)}}
\newcommand{\omegast}{\omega^{(2)}}
\newcommand{\Cdata}{C_{\mathrm{data}}}
\newcommand{\Qhomega}{Q_{\homega}}
\newcommand{\Qhmaxomega}{Q_{\hmaxomega}}


\newcommand{\epsymb}{\star}
\newcommand{\api}{a_{\epsymb}}
\newcommand{\ftilde}{\tilde{f}}
\newcommand{\gtilde}{\tilde{g}}
\newcommand{\Ph}{\cP_{h}}
\newcommand{\Npi}[1]{\N{#1}_{\epsymb}}
\newcommand{\CHtell}{C_{H^{2},\epsymb}}
\newcommand{\CHoell}{C_{H^{1},\epsymb}}
\newcommand{\CHthh}{C_{H^2}}
\newcommand{\Nunsure}[1]{\N{#1}_{\HhGI}}
\newcommand{\Cmess}{\widetilde{C}}
\newcommand{\Cfemo}{C_{\mathrm{FEM},1}}
\newcommand{\Cfemt}{C_{\mathrm{FEM},2}}
\newcommand{\Cfemth}{C_{\mathrm{FEM},3}}
\newcommand{\aadj}{\widetilde{a}}
\newcommand{\CRes}{C_{\mathrm{Res}}}
\newcommand{\uz}{u_{0}}
\newcommand{\Fadj}{\widetilde{F}}
\newcommand{\rhs}{M\mleft(f,\gI,\gD\mright)}

%%%%%%%%%%%%%% All the below macros have been taken from Anton Geraschenko's files at stacky.net %%%%%%%%%%%%%%%

%	Bold Letters
\renewcommand{\AA}{\mathbb A}
\newcommand{\BB}{\mathbb B}
\newcommand{\CC}{\mathbb C}
\newcommand{\DD}{\mathbb D}
\newcommand{\EE}{\mathbb E}
\newcommand{\FF}{\mathbb F}
\newcommand{\GG}{\mathbb G}
\newcommand{\HH}{\mathbb H}
\newcommand{\II}{\mathbb I}
\newcommand{\JJ}{\mathbb J}
\newcommand{\KK}{\mathbb K}
\newcommand{\LL}{\mathbb L}
\newcommand{\MM}{\mathbb M}
\newcommand{\NN}{\mathbb N}
\newcommand{\OO}{\mathbb O}
\newcommand{\PP}{\mathbb P}
\newcommand{\QQ}{\mathbb Q}
\newcommand{\RR}{\mathbb R}
\renewcommand{\SS}{\mathbb S}
\newcommand{\TT}{\mathbb T}
\newcommand{\UU}{\mathbb U}
\newcommand{\VV}{\mathbb V}
\newcommand{\WW}{\mathbb W}
\newcommand{\XX}{\mathbb X}
\newcommand{\YY}{\mathbb Y}
\newcommand{\ZZ}{\mathbb Z}

%Cal Letters
\newcommand{\cA}{\mathcal A}
\newcommand{\cB}{\mathcal B}
\newcommand{\cC}{\mathcal C}
\newcommand{\cD}{\mathcal D}
\newcommand{\cE}{\mathcal E}
\newcommand{\cF}{\mathcal F}
\newcommand{\cG}{\mathcal G}
\newcommand{\cH}{\mathcal H}
\newcommand{\cI}{\mathcal I}
\newcommand{\cJ}{\mathcal J}
\newcommand{\cK}{\mathcal K}
\newcommand{\cL}{\mathcal L}
\newcommand{\cM}{\mathcal M}
\newcommand{\cN}{\mathcal N}
\newcommand{\cO}{\mathcal O}
\newcommand{\cP}{\mathcal P}
\newcommand{\cQ}{\mathcal Q}
\newcommand{\cR}{\mathcal R}
\newcommand{\cS}{\mathcal S}
\newcommand{\cT}{\mathcal T}
\newcommand{\cU}{\mathcal U}
\newcommand{\cV}{\mathcal V}
\newcommand{\cW}{\mathcal W}
\newcommand{\cX}{\mathcal X}
\newcommand{\cY}{\mathcal Y}
\newcommand{\cZ}{\mathcal Z}

% Bold
\newcommand{\ba}{\mathbf{a}}
\newcommand{\bb}{\mathbf{b}}
\newcommand{\bc}{\mathbf{c}}
\newcommand{\bd}{\mathbf{d}}
\newcommand{\be}{\mathbf{e}}
\newcommand{\bff}{\mathbf{f}}
\newcommand{\bg}{\mathbf{g}}
\newcommand{\bh}{\mathbf{h}}
\newcommand{\bi}{\mathbf{i}}
\newcommand{\bj}{\mathbf{j}}
\newcommand{\bk}{\mathbf{k}}
\newcommand{\bl}{\mathbf{l}}
\newcommand{\bm}{\mathbf{m}}
\newcommand{\bn}{\mathbf{n}}
\newcommand{\bp}{\mathbf{p}}
\newcommand{\bq}{\mathbf{q}}
\newcommand{\br}{\mathbf{r}}
\newcommand{\bs}{\mathbf{s}}
\newcommand{\bt}{\mathbf{t}}
\newcommand{\bu}{\mathbf{u}}
\newcommand{\bv}{\mathbf{v}}
\newcommand{\bw}{\mathbf{w}}
\newcommand{\bx}{\mathbf{x}}
\newcommand{\by}{\mathbf{y}}
\newcommand{\bz}{\mathbf{z}}


% Nicked from Euan Spence
\newcommand{\beq}{\begin{equation}}
\newcommand{\eeq}{\end{equation}}
\newcommand{\beqs}{\begin{equation*}}
\newcommand{\eeqs}{\end{equation*}}
\newcommand{\bit}{\begin{itemize}}
\newcommand{\eit}{\end{itemize}}
\newcommand{\ben}{\begin{enumerate}}
\newcommand{\een}{\end{enumerate}}
\newcommand{\bal}{\begin{align}}
\newcommand{\eal}{\end{align}}
\newcommand{\bals}{\begin{align*}}
\newcommand{\eals}{\end{align*}}
\newcommand{\bse}{\begin{subequations}}
\newcommand{\ese}{\end{subequations}}
\newcommand{\bpr}{\begin{proposition}}
\newcommand{\epr}{\end{proposition}}
\newcommand{\bre}{\begin{remark}}
\newcommand{\ere}{\end{remark}}
\newcommand{\bpf}{\begin{proof}}
\newcommand{\epf}{\end{proof}}
\newcommand{\ble}{\begin{lemma}}
\newcommand{\ele}{\end{lemma}}
\newcommand{\bco}{\begin{corollary}}
\newcommand{\eco}{\end{corollary}}
\newcommand{\bex}{\begin{example}}
\newcommand{\eex}{\end{example}}
\newcommand{\bth}{\begin{theorem}}
\newcommand{\enth}{\end{theorem}}
\newcommand{\bcon}{\begin{condition}}
\newcommand{\econ}{\end{condition}}
\newcommand{\bas}{\begin{assumption}}
\newcommand{\eas}{\end{assumption}}
\newcommand{\bde}{\begin{definition}}
\newcommand{\ede}{\end{definition}}



\newcommand{\ton}{\text{ on }}
\newcommand{\tin}{\text{ in }}
\newcommand{\tfa}{\text{ for all }}
\newcommand{\tfor}{\text{ for }}
\newcommand{\tas}{\text{ as }}
\newcommand{\tand}{\text{ and }}
\newcommand{\tst}{\text{ such that }}
\newcommand{\tif}{\text{ if }}
\newcommand{\tals}{\text{ almost surely}}

\newcommand{\EXP}[1]{\mathbb{E}\mleft[#1\mright]}
\newcommand{\set}[1]{\mleft\{#1\mright\}}
\newcommand{\de}{\coloneqq}
\newcommand{\abs}[1]{\mleft|#1\mright|}
\newcommand{\VAR}[1]{\mathbb{V}\mleft[#1\mright]}
\newcommand{\half}{\frac{1}{2}}
\newcommand{\eps}{\varepsilon}
\newcommand{\NW}[1]{\N{#1}_{1,k}}
\newcommand{\N}[1]{\mleft\|#1\mright\|}
\newcommand{\NLtD}[1]{\N{#1}_{\LtD}}
\newcommand{\LtD}{\Lt{D}}
\newcommand{\Lt}[1]{L^{2}\mleft(#1\mright)}
\newcommand{\NLtGI}[1]{\N{#1}_{\LtGI}}
\newcommand{\LtGI}{\Lt{\GI}}
\newcommand{\GI}{\Gamma_{I}}
\newcommand{\bx}{\mathbf{x}}
\newcommand{\grad}{\nabla}
\newcommand{\vbar}{\overline{v}}
\newcommand{\IP}[2]{\langle#1,#2\rangle}
\newcommand{\kz}{k_0}
\newcommand{\dn}{\partial_{\nu}}
\newcommand{\vh}{v_{h}}
\newcommand{\Vh}{V_{h}}
\newcommand{\uh}{u_{h}}
\newcommand{\IPLtD}[2]{\IP{#1}{#2}_{\LtD}}
\newcommand{\Ih}{\cI_{h}}
\newcommand{\vo}{v_{1}}
\newcommand{\vt}{v_{2}}
\newcommand{\vtbar}{\overline{v}_{2}}
\newcommand{\Ho}[1]{H^{1}\mleft(#1\mright)}
\newcommand{\HoD}{\Ho{D}}
\newcommand{\Ht}[1]{H^{2}\mleft(#1\mright)}
\newcommand{\HtD}{\Ht{D}}
\newcommand{\Hh}[1]{H^{1/2}\mleft(#1\mright)}
\newcommand{\HhGI}{\Hh{\GI}}
\newcommand{\NHhGI}[1]{\N{#1}_{\HhGI}}
\newcommand{\NHtD}[1]{\N{#1}_{\HtD}}
\newcommand{\NHoD}[1]{\N{#1}_{\HoD}}
\newcommand{\Dm}{D_{-}}
\newcommand{\Dp}{D_{+}}
\newcommand{\RRd}{\RR^{d}}
\newcommand{\clos}[1]{\overline{#1}}
\newcommand{\Dmclos}{\clos{\Dm}}
\newcommand{\Dtilde}{\widetilde{D}}
\newcommand{\GD}{\Gamma_{D}}
\newcommand{\gD}{g_{D}}
\newcommand{\HoGD}{\Ho{\GD}}
\newcommand{\gI}{g_{I}}
\newcommand{\LiGIRR}{\Li{\GI;\RR}}
\newcommand{\Li}[1]{L^{\infty}\mleft(#1\mright)}
\newcommand{\LiDRR}{\Li{D;\RR}}
\DeclareMathOperator{\dist}{dist}
\DeclareMathOperator{\supp}{supp}
\newcommand{\nmin}{n_{\min}}
\newcommand{\nmax}{n_{\max}}
\newcommand{\LiDRRdtd}{\Li{D;\RRdtd}}
\newcommand{\RRdtd}{\RR^{d\times d}}
\newcommand{\Amin}{A_{\min}}
\newcommand{\Amax}{A_{\max}}
\newcommand{\bxi}{\mathbf{\xi}}
\newcommand{\CCd}{\CC^{d}}
\newcommand{\ii}{\mathrm{i}}
\newcommand{\trGD}{\tr_{D}}
\newcommand{\HozD}[1]{H^{1}_{0,D}\mleft(#1\mright)}
\newcommand{\HozDD}{\HozD{D}}
\newcommand{\st}{:}
\newcommand{\trGI}{\tr_{I}}
\newcommand{\defn}[1]{\emph{#1}}
\newcommand{\tr}{\gamma}
\newcommand{\restrict}[1]{|_{#1}}
\newcommand{\dD}{\partial D}
\newcommand{\LtdD}{\Lt{\dD}}
\newcommand{\NLtdD}[1]{\N{#1}_{\LtdD}}
\newcommand{\NHhdD}[1]{\N{#1}_{\HhdD}}
\newcommand{\HhdD}{\Hh{\dD}}
\newcommand{\quarter}{\frac{1}{4}}
\newcommand{\NWs}[1]{\N{#1}_{1,k,*}}
\newcommand{\HozDDs}{H^{1}_{0,D}\mleft(D\mright)^{*}}
\newcommand{\CzoDRRdtd}{\Czo{D;\RRdtd}}
\newcommand{\Czo}[1]{C^{0,1}\mleft(#1\mright)}
\newcommand{\wh}{w_{h}}
\newcommand{\whtilde}{\widetilde{w}_{h}}
\newcommand{\NLiDRR}[1]{\N{#1}_{\LiDRR}}
\newcommand{\vb}{\overline{v}}
\newcommand{\OFP}{\mleft(\Omega,\cF,\PP\mright)}
\newcommand{\gradGD}{\grad_{\GD}}
\newcommand{\LtGD}{\Lt{\GD}}
\newcommand{\NLtGD}[1]{\N{#1}_{\LtGD}}

\ifdefined\supervisorversion
    \newcommand{\optodo}[1]{}
\else
    \newcommand{\optodo}[1]{\ednote{OPTODO: #1}}
    \fi
    % Based on https://tex.stackexchange.com/a/1495
\newcommand{\minispace}{\;\!}


%%% Number equations within section
\numberwithin{equation}{chapter}


%%%%%%%%These conflict with SIAM style
\newtheorem{theorem}{Theorem}[chapter]

% This is only here if amsthm is disabled
%\newenvironment{proof}[1][Proof]{\noindent\emph{#1}\,}{\hfill$\square$}
\newtheorem{corollary}[theorem]{Corollary}
\newtheorem{definition}[theorem]{Definition}
\newtheorem{lemma}[theorem]{Lemma}
\newtheorem{proposition}[theorem]{Proposition}
\newtheorem{assumption}[theorem]{Assumption}
\newtheorem{example}[theorem]{Example}
\newtheorem{remark}[theorem]{Remark}
\newtheorem{summary}[theorem]{Summary}
\newtheorem{criterion}[theorem]{Criterion}
\newtheorem{notation}[theorem]{Notation}
\newtheorem{algorithm}[theorem]{Algorithm}
\newtheorem{construction}[theorem]{Construction}
\newtheorem{condition}[theorem]{Condition}
\newtheorem{problem}[theorem]{Problem}


\crefname{assumption}{Assumption}{Assumptions}
\Crefname{assumption}{Assumption}{Assumptions}
\crefname{chapter}{Chapter}{Chapters}
\Crefname{chapter}{Chapter}{Chapters}
\crefname{listrequirement}{Requirement}{Requirements}
\Crefname{listrequirement}{Requirement}{Requirements}
\crefname{itemachievement}{Achievement}{Achievements}
\Crefname{itemachievement}{Achievement}{Achievements}
\crefname{itempart}{part}{parts}
\Crefname{itempart}{Part}{Parts}
\crefname{itemblank}{}{}
\Crefname{itemblank}{}{}
\crefname{problem}{Problem}{Problems}
\Crefname{problem}{Problem}{Problems}

% Making cleverref play nicely with newly defined environments
\crefname{convariable}{Condition}{Conditions}
\Crefname{convariable}{Condition}{Conditions}
\crefname{probvariable}{Problem}{Problems}
\Crefname{probvariable}{Problem}{Problems}
\crefname{myprobsec}{Problem}{Problems}
\Crefname{myprobsec}{Problem}{Problems}
\crefname{step}{Step}{Steps}
\Crefname{step}{Step}{Steps}
\crefname{point}{Point}{Points}
\Crefname{point}{Point}{Points}
\crefname{part}{Part}{Parts}
\Crefname{part}{Part}{Parts}
\crefname{condition}{Condition}{Conditions}
\Crefname{condition}{Condition}{Conditions}
\crefname{problem}{Problem}{Problems}
\Crefname{problem}{Problem}{Problems}

% Write out 'Figures' in full
\crefname{figure}{Figure}{Figures}
\Crefname{figure}{Figure}{Figures}

% Customising cleverref
\newcommand{\crefrangeconjunction}{--}

% Add a serial/Oxford comma by default.
\def\lastconj{, and~}
\newcommand{\creflastconjunction}{\lastconj}

% Making it do equations like eqref
\crefname{equation}{}{}
\Crefname{equation}{}{}


\title{Multi-level Monte Carlo for the Helmholtz equation}

\author{Owen Pembery}

\begin{document}

\maketitle

%%%%%%% Body Text %%%%%%%
%
%\begin{abstract}
%
%\end{abstract}

\section{$h \lesssim k^{-3/2}$} % THIS IS IMPORTANT

\subsection{Statement of Problem, Assumptions, and Main Result}

In this section, we prove that the finite-element approximation of the solution to the Helmholtz TEDP exists if $ h \lesssim k^{-3/2}.$ Moreover, we give an expression for the hidden constant that is completely explicit in $A$ and $n$, and we also prove a bound on the finite-element error, again completely explicit in $A$ and $n$. The argument in this section closely follows those in \cite{FeWu:11,ChNi:18} in its use of an elliptic projection argument to prove the required finite-element existence result and error bound. The paper \cite{FeWu:11} proved a similar result for the Helmholtz equation in homogeneous media, and \cite{ChNi:18} does so for the homogeneous Helmholtz equation with corner singularities.

Whilst we prove the results in this section for the TEDP, we expect that they can be extended to the Helmholtz Exterior Dirichlet Problem (EDP) where the infinite domain is truncated, and the Dirichlet-to-Neumann map is realised exactly on the truncated boundary. However, our proof below uses recently-proved bounds on the solution of a related problem to the TEDP from \cite{ChNiTo:18}; in order to extend our results to the EDP we would need analogues to the results in \cite{ChNiTo:18} for the EDP.

\paragraph{Problem Set-up} Let $\Dm$ be a bounded Lipschitz\ednote{We actually need this to be a $C^{k,\lambda}$ set, for $k+\lambda > 1.5,$ so that we can do the whole non-zero Dirichlet data thing. This is getting a bit complicated. I guess our options are (i) persevere, (ii) give up and just do the theory for zero Dirichlet data, or (iii) assume that we know $\ud,$ not just $\gD.$ Thoughts?} open set such that the open complement $\Dp\de \RRd\setminus \Dmclos$ is connected. Let $\Dtilde$ be a bounded connected Lipschitz open set such that $\Dmclos \subset\subset\Dtilde$. 
Let $D\de\Dtilde\setminus\Dm$, $\GD\de \partial \Dm$, and $\GI \de\partial \Dtilde$, so that $\partial D= \GD \cup \GI$ and $\GD\cap \GI = \emptyset$. Throughout $\tr$ will denote the trace onto the whole boundary $\dD,$ whereas $\trGI$ and $\trGD$ will denote the traces on $\GI$ and $\GD$ respectively. Throughout we assume there exists some $\kz > 0$ such that $k \geq \kz$. Let $\NW{v}$ denote the weighted $H^1$ norm on $\HoD$:
\beqs
\NW{v}^2 \de \NLtD{\grad v}^2 + k^2 \NLtD{v}^2.
\eeqs


Let
\bit
\item $f\in \LtD$ 
\item $\gD\in \HthtGD$,
\item $\gI\in \LtGI$
\item $n\in \LiDRR$ such that $\dist\mleft(\supp\mleft(1-n\mright),\GI\mright)>0$, satisfying
\beq
0<\nmin \leq n\mleft(\bx\mright)\leq\nmax<\infty\,\, \text{ for almost every } \bx \in D,
\eeq
\item $A \in \WoiDRRdtd$ such that $\dist\mleft(\supp\mleft(I -A\mright),\GI\mright)>0$, $A$ is symmetric, and there exist $0<\Amin\leq \Amax<\infty$ such that
\beq\label{eq:AellEDP}
 \Amin |\bxi|^2\leq\mleft(A\mleft(\bx\mright) \bxi\mright) \cdot \overline{ \bxi}  \leq \Amax|\bxi|^2 \quad\text{ for almost every }\bx \in D \text{ and for all } \bxi\in \CCd.
\eeq
\eit
%we say $u\in \HoD$ satisfies the Helmholtz Truncated Exterior Dirichlet Problem (TEDP) if 
%\beqs
%\grad\cdot\mleft(A \grad u \mright) + k^2 n u = -f \quad \tin D,
%\eeqs
%\beqs
%\trGD u =\gD \quad\ton \GD,
%\eeqs
%and 
%\beq\label{eq:TEDP3}
%\dn u - \ii k  \trGI u = \gI \ton \GI.
%\eeq
In order to study the TEDP with $\gD\neq0,$ we must, in essence reformulate to the TEDP with $\gD=0$ but a different right-hand side for the domain term.

Define the space
\beqs
\HozDD \de \set{v \in \HoD \st \trGD u = 0}.
\eeqs
%and the sesquilinear form and antilinear functional
%The variational formulation of the TEDP with $\gD = 0$ is%\optodo{Check exactly what's needed in hetero}
%
%\beq\label{eq:tedpz}
%\text{Find } u \in \HozDD\quad \tst\quad a(u,v) = F(v)\quad \tfa v \in \HozDD,
%\eeq
%
%where
%
%\beqs
%a(u,v) \de \int_D \mleft(A \grad u\mright)\cdot \grad \vb - k^2 n u\vb - ik \int_{\GI} \trGI u \trGI \vb\quad \tand\quad F(v) \de \int_D f\vb + \int_{\GI} \gI\trGI \vb.
%\eeqs
%
In order to deal with non-zero Dirichlet data $\gD,$ we let  $\ud \in \HtD$ be such that $\trGD \ud = \gD$, and $\esssup \ud \compcont D$. The proof that such a $\ud$ exists is in \cref{lem:ud}.
The variational formulation of the TEDP is then
\beq\label{eq:tedp}
\text{Find } u \in \HozDD\quad \tst\quad a(u,v) = F(v)\quad \tfa v \in \HozDD,
\eeq
where
\beqs
a(u,v) \de \int_D \mleft(A \grad u\mright)\cdot \grad \vb - k^2 n u\vb - ik \int_{\GI} \trGI u \trGI \vb
\eeqs
and
\beqs
F(v) \de  \int_D \mleft(f - \grad \cdot \mleft(A\grad \ud\mright) - k^2 n\ud\mright)\vb + \int_{\GI} \mleft(\gI-\dn\ud\mright)\trGI \vb.
\eeqs
The function $\us = u+ \ud$ is then the solution of the Helmholtz equation
\beqs
\grad\cdot\mleft(A \grad \us \mright) + k^2 n \us = -f \quad \tin D,
\eeqs
\beqs
\trGD \us =\gD \quad\ton \GD,
\eeqs
and 
\beq\label{eq:TEDP3}
\dn \us - \ii k  \trGI \us = \gI \ton \GI.
\eeq
\bre[Reducing the smoothness of $\gD$]
The assumption that $\gD \in \HthtGD$ is made so that the lifting $\ud$ of $\gD$ is in $\HtD$ (see \cref{app:ud}). As $\ud \in \HtD,$ the antilinear functional $F$ defined above is well-defined. We could reduce the smoothness of $\gD$ to $\HoGD$ (meaning $\ud \in \HthtD$) but this reduction in smoothness would then require us to reformulate the functional $F$ as
\beqs
F(v) = \int_D \mleft(A \grad \ud\mright)\cdot \grad \vb - k^2 n \ud \vb + f \vb + \int_{\GI} \mleft(\gI - \dn \ud\mright)\vb.
\eeqs\optodo{Put a proof of this somewhere, in 28/2/19 notes}
With this reformulation, $F \in \HozDDprime,$ but does not have a representative function in $\LtD$. Our proofs below will use results from \cite{ChNiTo:18}, which are stated for the TEDP with zero Dirichlet boundary condition and $L^2$ right-hand side. To avoid the complications stated above, and to allow us to use the results in \cite{ChNiTo:18}, we therefore impose the additional smoothness on $\gD.$ Also, in the case with $F$ only in $\HozDDprime$, proving a priori bounds on the solution of the TEDP is more complicated (c.f., e.g., \cite[Theorem 2.5]{GrPeSp:18} and \cite[Corollary 2.16]{GrPeSp:18} which consider the analogous EDP). For the same reason, we assume $A \in \WoiDRRdtd;$ if we only had $A \in \LiDRRdtd,$ we could reformulate $F$ as outlined above, but we would have the same complications as just described.

The fact that we cannot reduce the smoothness of $\gD$ further to $\HhGD$ is due to the Morawetz multiplier techniques used to obtain the a priori bounds in \cite{GrPeSp:18}, see \cite[(iii), p. 2874]{GrPeSp:18}.
\ere
\optodo{Probably need to prove this---I guess it'll go along similar lines as the proof in hetero}
Also, for later use we state the \defn{adjoint} problem.
\beq\label{eq:tedpadj}
\text{Find } u \in \HozDD\quad \tst\quad \aadj(u,v) = F(v)\quad \tfa v \in \HozDD,
\eeq
where
\beqs
\aadj(u,v) \de \int_D \mleft(A \grad u\mright)\cdot \grad \vb - k^2 n u\vb + ik \int_{\GI} \trGI u \trGI \vb.
\eeqs
%and
%\beqs
%\Fadj(v) \de \aadj(\uz,v) + \int_D f\vb + \int_{\GI} \gI\trGI \vb.
%\eeqs

The statement of the main result requires the following related sesquilinear form and \lcnamecref{lem:relatedwp}.

\bde[Related sesquilinear form]
For $\vo, \vt \in \HozDD$ we define
\beqs
\api(\vo,\vt) = \int_D \IP{A \grad \vo}{\vt} - ik\int_{\GI} \vo\vtbar.
\eeqs
\ede

\ble[Related PDE is well-posed and solution is in $H^2$]\label{lem:relatedwp}
If $A \in \CzoDRRdtd,$ then the solution $\psi \in \HozDD$ of the related PDE
\beq\label{eq:relpde}
\api(u,v) = \IPLtD{f}{v}\quad \tfa\quad v \in \HozDD
\eeq
exists, is unique, is in $\HtD,$ and satisfies the a priori bound
% \beqs\label{eq:relpdehobound}
% \NW{\psi} \lesssim \frac{\max\set{\Amin^{-1},1}}k,
% \eeqs
% and
% \beqs
% \NHoD{\psi} \lesssim \CHoell \NLtD{f}
% \eeqs
% and
\beqs
\NHtD{\psi} \lesssim \CHtell \NLtD{f}.
\eeqs
for some constant $\CHtell > 0$ depending on $A,$ but independent of $k.$
\ele

\bre[Proof of \cref{lem:relatedwp}]
\Cref{lem:relatedwp} is proved in \cite{ChNiTo:18}, although the dependence on $A$ is not made explicit.
\ere

%\paragraph{Finite-Element Set-up} Let $\Vh$ be the first-order linear finite-element space on some mesh on $D$ with mesh size $h.$

\bas[Existence, uniqueness, and an a priori bound]\label{ass:bound}
We assume that the coefficients $A$ and $n$ are such that for all $k \geq \kz$ the solutions of the TEDP \eqref{eq:tedp} and its adjoint \eqref{eq:tedpadj} exist, are unique, are in $\HtD,$ and satisfy the bound
\beq\label{eq:hhbound}
\NHtD{u} \lesssim \CHthh \,k \mleft(\NLtD{f} + \Nunsure{g} + \NLtGD{\gradGD \gD} + k \NLtGD{\gD}\mright),
\eeq
where $\gradGD$ is the surface gradient on $\GD,$ $u$ is the solution of the TEDP or its adjoint, and $\CHthh >0$ is a constant dependent on $A$, $n,$ and possibly $k.$
% \footnote{Determining the dependence of $\CHthh$ on $A$ and $n$ could be tricky. It was done in \cite{ChScTe:13} for a $C^2$ domain with scalar $A$ and homogeneous Dirichlet boundary conditions.}
 \eas

\bde[Finite-element approximation] 
 The finite-element approximation to \eqref{eq:tedp} is the following:
\beq\label{eq:tedpfe}
\text{Find } \uh \in \Vh\quad \tst\quad a(\uh,\vh) = F(\vh)\quad \tfa \vh \in \Vh,
\eeq
\ede

The main theorem we prove is the following:

\bth[Finite-element-error bound]\label{thm:febound}
If $A \in \CzoDRRdtd,$ $h \lesssim 1/k,$ \cref{ass:bound} holds, and
\beq\label{eq:hcond}
h \lesssim \mleft(\NLiDRR{n} \mleft(\Amax + \half\mright)\CHtell \CHthh\mright)^{-1/2}k^{-3/2}, % There should be a factor of a half in front of the right-hand side of this, as it makes things clearer what's going on in the proof. However, since we're doing everything with \lesssim, a factor of a half doesn't matter. We could replace the half with any \eps in (0,1), but then the constant hidden in the \lesssim in \eqref{eq:hherrltbound} has a factor 1/\eps.
\eeq
the finite-element solution $\uh$ to the problem \eqref{eq:tedpfe} exists, is unique, and satisfies the bounds
\beq\label{eq:hherrltbound}
\NLtD{u-\uh} \lesssim \Cfemo \mleft(hk\mright)^2 \mleft(\NLtD{f} + \Nunsure{\gI} + \NLtGD{\gradGD \gD} + k \NLtGD{\gD}\mright)
\eeq
and
\beq\label{eq:hherrwbound}
\NW{u-\uh} \lesssim \mleft(\Cfemt hk +  \Cfemth h^2k^3\mright)\mleft(\NLtD{f} + \Nunsure{\gI} + \NLtGD{\gradGD \gD} + k \NLtGD{\gD}\mright),
\eeq
where
\beqs
\Cfemo \de \mleft(\Amax + \half\mright)\CHthh^2,
\eeqs
\beqs
\Cfemt \de \frac{\Amax+\half}{\Amin} \CHthh,
\eeqs
\beqs
\Cfemth \de \frac{\mleft(\Amin+ \NLiDRR{n}\mright)^{1/2}}{\Amin^{1/2}}\Cfemo,
\eeqs
and $u$ is the solution of the TEDP \eqref{eq:tedp}.
\enth

\subsubsection{Properties of the Elliptic Projection, and a related PDE}

The proof technique we use below (adapted from \cite{FeWu:11,ChNi:18}) uses an `elliptic projection' of the solution of the TEDP using the related sesquilinear form $\api.$ We define the energy norm induced by the sesquilinear form $\api$:
\beqs
\Npi{\vo} = \sqrt{\abs{\api(\vo,\vo)}}.
\eeqs

\ble[Energy Norm is a norm]\label{lem:inducednorm}
The induced norm $\Npi{\cdot}$ is a norm on $\HoD.$
\ele

\bpf[Proof of \cref{lem:inducednorm}]
The main thing to check is that, for $v \in \HoD,$ $\Npi{v}=0 \implies v=0.$ By construction, if $\Npi{v}=0,$ then $\int_{D} \IP{A \grad v}{\grad v} =0$ and $\NLtGI{v}^2 = 0,$ as these are the real and imaginary parts of $\api(v,v).$ By \eqref{eq:AellEDP}, it follows that $\Amin \abs{\grad v}^2 \leq 0,$ and thus $v$ is constant. As $\NLtGI{v} = 0,$ it follows that $\trGI v =0,$ and hence by the trace theorem, as $v$ is constant, it follows that $v=0.$

Other properties of norms follow analagously as with any definition of an energy norm.
\epf
\ble[Energy norm is equivalent to weighted norm]\label{lem:normbound}
If $v \in \HoD,$ then
\beq\label{eq:boundew}
\Npi{v} \lesssim \sqrt{\Amax+\half}\NW{v}
\eeq
and
\beq\label{eq:boundwe}
\NW{v} \lesssim \max\set{\Amin^{-\half},1} \Npi{v}
\eeq
\ele

\bpf[Proof of Lemma \ref{lem:normbound}]
To show \eqref{eq:boundew}, for $ v \in \HoD$ we have
\begin{align*}
  \Npi{v}^2 &= \abs{\api(v,v)}\\
            &\lesssim \abs{\int_{D} \IP{A \grad v}{\grad v}} + k\NLtGI{v}^2 \\
            &\lesssim \abs{\int_{D} \IP{A \grad v}{\grad v}} + k\NLtD{v}\NHoD{v}, \text{ by the multiplicative trace inequality}\\
            &\lesssim \Amax \NLtD{\grad v}^2 + \half k^2 \NLtD{v}^2 + \half \NHoD{v}^2\\
  &\lesssim \mleft(\Amax+\half\mright)\NW{v}^2
\end{align*}
as required.

To show \eqref{eq:boundwe} we first show that, for $v \in \HoD,$ $\Npi{v} \gtrsim \min\set{\Amin^{\half},1} \mleft(\NLtD{\grad v} + k^{\half} \NLtGI{\trGI v}\mright)$:
\begin{align}
  \Npi{v} &= \mleft(\abs{\api(v,v)}\mright)^{\half}\nonumber\\
          &= \mleft(\mleft(\int_D \IP{A \grad v}{\grad v}\mright)^2 + k^2 \mleft(\int_{\GI}\abs{\trGI v}^2\mright)^2\mright)^{\quarter}\nonumber\\
  &\geq \mleft(\mleft(\int_D \Amin \abs{\grad v}\mright)^2 + k^2 \NLtGI{\trGI v}^4\mright)^{\quarter}\nonumber\\
          &= \mleft(\Amin^2 \NLtD{\grad v}^4 + k^2 \NLtGI{\trGI v}^4\mright)^{\quarter}\nonumber\\
          &\geq \min\set{\Amin^{\half},1}\mleft(\NLtD{\grad v}^4 + k^2 \NLtGI{\trGI v}^4\mright)^{\quarter}\nonumber\\
  &\gtrsim \min\set{\Amin^{\half},1} \mleft(\NLtD{\grad v} + k^{\half} \NLtGI{\trGI v}\mright), \text{ as } \mleft(x+y\mright)^4 \lesssim x^4 + y^4.\label{eq:Npifour}
\end{align}

We recall the fact that for $v \in \HoD,$
\beq\label{eq:poincarelike}
\NLtD{v} \lesssim \NLtD{\grad v} + \NLtGI{\trGI v},
\eeq
see, e.g., \cite[Equation (6.16)]{Sp:15}. We can then prove \eqref{eq:boundwe}:
\begin{align*}
   \NW{v} &\lesssim \NLtD{v}+ \NLtD{\grad v}\\
          &\lesssim \NLtGI{\trGI v}+ \NLtD{\grad v} + \NLtD{\grad v}\text{ by \eqref{eq:poincarelike}}\\
          &\lesssim k^{\half}\NLtGI{v} + \NLtD{\grad v}\\
  &\lesssim \max\set{\Amin^{-\half},1}\Npi{v}, \text{ by \eqref{eq:Npifour}.}
\end{align*}
\epf

% \ble[Bound on $L^2$ norm]\label{lem:ltbound}
% If $v \in \HoD$ then the bound
% \beqs
% \NLtD{v} \lesssim  \NLtGI{\trGI v} + \NLtD{\grad v}
% \eeqs
% holds.
% \ele

% \bpf[Proof of \cref{lem:ltbound}]
% \optodo{Look at proof in IbyPs article}
% If $v \in \HozDD,$ then by the Poincar\'e inequality, we have that $\NLtD{v} \lesssim \NLtD{\grad v}.$ Alternatively, if $\GD = \emptyset$ and $\trGI v$ is constant, then $v - \trGI v \in \HozDD,$ and thus (abusing notation, and letting $\trGI v$ denote the value of the constant, and also a constant function defined on $D$ taking that value everywhere)
% \begin{align*}
%   \NLtD{v} &\leq \NLtD{\trGI v} + \NLtD{v-\trGI v}\\
%   &= \NLtGI{\trGI v} + \NLtD{v-\trGI v}\\
%            &\lesssim \NLtGI{\trGI v} +  \NLtD{\grad \mleft(v-\trGI v\mright)}\\
%              &= \NLtGI{\trGI v} + \NLtD{\grad v},
% \end{align*}
% as required.
% \epf

% \ble[Bound on weighted norm by energy norm]\label{lem:othernormbound}
% If $v \in \HozDD,$ or if $\GD = \emptyset$ and $\trGI v$ is constant, the bound
% \beq\label{eq:boundwe}
% \NW{v} \lesssim \max\set{\Amin^{-\half},1} \Npi{v}
% \eeq
% holds.
% \ele

% \bpf[Proof of \cref{lem:othernormbound}]
% \epf

We now define the elliptic projection of a function in $\HoD.$% and also define a related PDE that will be used in proving the approximation properties of the elliptic projection.

\bde[Elliptic Projection]
For $w \in \HoD$ we define the \defn{elliptic projection} $\Ph w \in \Vh$ of $w$ by
\beq\label{eq:ellproj}
\api(\vh,\Ph w) = \api(\vh,w) \tfa \vh \in \Vh.
\eeq
\ede

% \bde[Related PDE]\label{lem:relpde}
% Given $f \in \LtD$ we define the related (adjoint) PDE; find $\psi \in \HoD$ such that for all $v \in \HoD$
% \beq\label{eq:relpde}
% \api(\psi,v) = \IPLtD{f}{v}.
% \eeq\ede

% \bpf[Proof of \cref{lem:relatedwp}]
% By \eqref{eq:boundwe} we have, for $v \in \HozDD$
% \beqs
% \min\set{\Amin,1}\NW{v}^2 \lesssim \abs{\api(v,v)},
% \eeqs
% and we also have that
% \beqs
% \NWs{\IP{f}{\cdot}} \leq \frac1k \NLtD{f},
% \eeqs
% where $\NWs{\cdot}$ denotes the norm on $\HozDDs$ induced by $\NW{\cdot}.$ By the Lax--Milgram Theorem, we can therefore conclude that $\psi$ exists, is unique, and satisfies the bound
% \beqs
% \NW{\psi} \lesssim \frac{\max\set{\Amin^{-1},1}}{k}\NLtD{f}.
% \eeqs
% Use Grisvard Magic to get $H^2.$\optodo{this}
% \epf

\ble[Properties of elliptic projection]\label{lem:ellprojbounds}
Let $A \in \CzoDRRdtd.$ If $w \in \HtD,$ then the elliptic projection $\Ph w$ exists, is unique, and the error satisfies the bounds
\beq\label{eq:ellprojenbound}
\Npi{w-\Ph w} \lesssim \sqrt{\Amax+\half}\,h\NHtD{w},
\eeq
and
\beq\label{eq:ellprojltbound}
\NLtD{w-\Ph w} \lesssim  \mleft(\Amax+\half\mright)\CHtell\,h^2\NHtD{w}.
\eeq
\ele

\bpf[Proof of \cref{lem:ellprojbounds}]
We first assume $\Ph w$ exists. To show \eqref{eq:ellprojenbound} we apply C\'{e}a's Lemma in $\Vh$ using the energy norm $\Npi{\cdot}$ to conclude
\beqs
\Npi{w-\Ph w} \leq \Npi{w-\Ih w}.
\eeqs
We then apply \cref{lem:normbound,lem:scottzhangbound} to conclude \eqref{eq:ellprojenbound}.

To prove \eqref{eq:ellprojltbound} we let $\psi$ solve the related PDE (\cref{lem:relpde}) with $f = w-\Ph w.$ By \cref{lem:relatedwp} $\psi \in \HtD$ and thus by  \cref{lem:normbound} and \cref{lem:scottzhangbound}
\beqs
\Npi{\psi - \Ih \psi} \lesssim \sqrt{\Amax + \half}\CHtell \,h\NLtD{w-\Ph w}.
\eeqs

If we now set $v = w-\Ph w$ in \eqref{eq:relpde}, then we obtain
\begin{align}
  \NLtD{w - \Ph w}^2 &= \api\mleft(\psi,w-\Ph w\mright)\nonumber\\
                     &= \api\mleft(\psi-\Ih \psi,w-\Ph w\mright) \text{ by Galerkin orthogonality for } w-\Ph w\nonumber\\
                     &\leq \Npi{\psi-\Ih \psi}\Npi{w-\Ph w}\nonumber\\
                       &\lesssim \sqrt{\Amax + \half}\CHtell \,h\NLtD{w-\Ph w}\Npi{w-\Ph w}\label{eq:epltfinal}.
\end{align}
By cancelling $\NLtD{w- \Ph w}$ from both sides of \eqref{eq:epltfinal} and using \eqref{eq:ellprojenbound} we obtain \eqref{eq:ellprojltbound}.

We have proved the bounds \eqref{eq:ellprojenbound} and \eqref{eq:ellprojltbound} under the assumption of existence. To show uniqueness, suppose $\wh, \whtilde$ both satisfy \eqref{eq:ellproj} (with $\Ph = \wh$ or $\whtilde$ respectively). Then by linearity, for all $\vh \in \Vh,$
\beqs
\api\mleft(\vh,\Ph\mleft(\wh-\whtilde\mright)\mright) = \IP{\vh}{w-w} = 0.
\eeqs
That is, the function $\wh - \whtilde$ is an elliptic projection of the zero function.

Therefore, by \eqref{eq:ellprojltbound} $\NLtD{0 - \mleft(\wh - \whtilde\mright)} \lesssim 0,$ i.e., $\wh = \whtilde.$ Therefore, if the elliptic projection $\Ph w$ exists, it is unique. As the space $\Vh$ is finite-dimensional, by the Rank--Nullity Theorem, the uniqueness of $\Ph w$ implies its existence; hence $\Ph w$ exists, and is unique, as required.
\epf


\subsubsection{Proof of Main Result}

We let $\Ih$ denote the Scott--Zhang quasi-interpolant in $\Vh$ (see \cite{ScZh:90}), and will use its following property.

\ble[Properties of Scott-Zhang interpolant]\label{lem:scottzhangbound}
let $h \lesssim 1/k.$ If $w \in \HtD,$ then
\beq\label{eq:scottzhangbound}
\NW{w - \Ih w} \lesssim h \NHtD{w}.
\eeq
\ele

\bpf[Proof of \cref{lem:scottzhangbound}]
The Scott-Zhang interpolant $\Ih w$ satisfies
\beq\label{eq:szlt}
\NLtD{w-\Ih w} \lesssim h^2 \NHtD{w}
\eeq
\and
\beq\label{eq:szho}
\NHoD{w-\Ih w} \lesssim h \NHtD{w}.
\eeq
Hence by the definition of $\NW{\cdot},$ by combining \eqref{eq:szlt} and \eqref{eq:szho} we have
\beqs
\NW{w-\Ih w} \lesssim h\mleft(1+hk\mright)\NHtD{w}.
\eeqs
As $h\lesssim 1/k,$ \eqref{eq:scottzhangbound} follows.
\epf

The following \lcnamecref{cor:hhszbound} follows from \cref{ass:bound}, and is used to prove \cref{thm:febound}.

\bco\label{cor:hhszbound}
If $u$ is the solution of the Helmholtz Interior Impedance Problem (or its adjoint) then the error in the Scott--Zhang quasi-interpolant satisfies
\beq\label{eq:hhszbound}
\NW{u-\Ih u} \lesssim \CHthh hk \mleft(\NLtD{f} + \Nunsure{g} + \NLtGD{\gradGD \gD} + k \NLtGD{\gD}\mright).
\eeq
\eco

The proof of the main theorem (\cref{thm:febound} below) also uses the fact that $a$ satisfies a G\r{a}rding inequality.
\ble[G\r{a}rding inequality]
If $v \in \HozDD,$ then
\beq\label{eq:garding}
\Re\mleft(a(v,v)\mright) \geq \Amin \NW{v}^2 - \mleft(\Amin + \NLiDRR{n}\mright)\NLtD{v}^2,
\eeq
where $\Re$ denotes the real part.
\ele

Finally, we recall \defn{Cauchy's inquality}: For all $a,b \in \RR$, and for all $\eps > 0,$
\beq\label{eq:cauchy}
ab \leq \frac{a^2}{2\eps} + \frac{\eps b^2}{2}.
\eeq

\bpf[Proof of \cref{cor:hhszbound}]
The proof follows from \cref{lem:scottzhangbound,ass:bound}.
\epf

We are now in a position to prove our main theorem.




\bpf[Proof of \cref{thm:febound}]
In this proof, for brevity we let
\beqs
\rhs = \NLtD{f} + \Nunsure{\gI} + \NLtGD{\gradGD \gD} + k \NLtGD{\gD}.
\eeqs
By \cref{ass:bound} the solution $u$ of the TEDP exists and is unique. Assume the finite-element solution $\uh$ exists. Let $\xi \in \HoD$ satisfy the adjoint TEDP \eqref{eq:tedpadj} with $f=u-\uh,$ $\gD=0,$ and $\gI=0.$ Taking complex conjugates, it follows that
\beq\label{eq:errordual}
a(v,\xi) = \IPLtD{v}{u-\uh} \tfa v \in \HoD.
\eeq
By \cref{ass:bound} $\xi$ exists, is unique and is in $\HtD.$. Setting $v = u-\uh$ in \eqref{eq:errordual} we obtain
\begin{align*}
  \NLtD{u-\uh}^2 &= a\mleft(u-\uh,\xi\mright)\\
                 &= a\mleft(u-\uh,\xi-\Ph \xi\mright) \quad\text{by Galerkin orthogonality for } u-\uh\\
                 &= \api\mleft(u-\uh,\xi-\Ph \xi\mright) - k^2 \IPLtD{n\mleft(u-\uh\mright)}{\xi-\Ph \xi}\\
                 &= \api\mleft(u-\Ih u,\xi-\Ph \xi\mright) - k^2 \IPLtD{n\mleft(u-\uh\mright)}{\xi-\Ph \xi}\\
  &\quad\quad\quad\text{by Galerkin orthogonality for }\xi  - \Ph \xi\\
                 &\leq \Npi{u-\Ih u}\Npi{\xi - \Ph \xi} + \NLiDRR{n} k^2 \NLtD{u-\uh}\NLtD{\xi-\Ph \xi}\\
                 &\lesssim \sqrt{\Amax + \half}\, \CHthh\, hk \rhs\Npi{\xi-\Ph \xi}\\
  &\quad\quad+  \NLiDRR{n} k^2 \NLtD{u-\uh}\NLtD{\xi-\Ph \xi}\quad \text{by \eqref{eq:boundew} and \eqref{eq:hhszbound}}\\
                 &\lesssim \sqrt{\Amax + \half}\, \CHthh\, hk\rhs\,\sqrt{\Amax + \half}\,h\NHtD{\xi}\\
                 &\quad\quad  + \NLiDRR{n} k^2 \NLtD{u-\uh}\NLtD{\xi-\Ph \xi}\quad \text{by \eqref{eq:ellprojenbound}}\\
                 &\lesssim \mleft(\Amax + \half\mright)\CHthh\, hk\rhs \, \CHthh \,hk \NLtD{u-\uh}\\
&\quad\quad  + \NLiDRR{n} k^2 \NLtD{u-\uh}\NLtD{\xi-\Ph \xi}\quad \text{by \eqref{eq:hhbound}}\\
\end{align*}
Cancelling a factor of $\NLtD{u-\uh}$ and rearranging terms we obtain
\beqs
\NLtD{u-\uh} \lesssim \mleft(\Amax + \half\mright)\CHthh^2 \mleft(hk\mright)^2\rhs + k^2 \NLiDRR{n} \NLtD{\xi - \Ph \xi}
\eeqs
and therefore
\begin{align*}
&  \NLtD{u-\uh} \lesssim \mleft(\Amax + \half\mright)\CHthh^2 \mleft(hk\mright)^2 \rhs\\
&\quad\quad  + h^2k^3 \NLiDRR{n} \mleft(\Amax + \half\mright) \CHtell \CHthh \NLtD{u-\uh}
\end{align*}
  using the definition of $\xi$, \eqref{eq:ellprojltbound}, and \eqref{eq:hhbound}Therefore if $h$ satisfies \eqref{eq:hcond} we obtain \eqref{eq:hherrltbound}.
% \beqs
% \half \NLtD{u-\uh} \lesssim \mleft(\Amax + \half\mright)\CHthh \mleft(hk\mright)^2 \mleft(\NLtD{f} + \Nunsure{g}\mright).
% \eeqs
% that is, if $\Cmess \de (\mleft(\NLiDRR{n} \mleft(\Amax + \half\mright) \CHthh \mright)^{-1/2},$ then
% \beqs
% \NLtD{u-\uh} \lesssim \mleft(\Amax + \half\mright)\CHthh \Cmess^2k^{-1} \mleft(\NLtD{f} + \Nunsure{g}\mright),
% \eeqs
To obtain the bound \eqref{eq:hherrwbound}, we use the G\r{a}rding inequality \eqref{eq:garding}:
\begin{align*}
  \Amin \NW{u-\uh}^2 &\leq \Re\mleft(a\mleft(u-\uh,u-\uh\mright)\mright) + k^2\mleft(\Amin+ \NLiDRR{n}\mright) \NLtD{u-\uh}^2\\
                     &= \Re\mleft(a\mleft(u-\uh,u-\Ih u\mright)\mright) + k^2\mleft(\Amin+ \NLiDRR{n}\mright) \NLtD{u-\uh}^2\\
  &\quad\quad\quad\text{by Galerkin orthogonality}\\
                     &\leq \mleft(\Amax+\half\mright) \NW{u-\uh}\NW{u-\Ih u} + k^2\mleft(\Amin+ \NLiDRR{n}\mright) \NLtD{u-\uh}^2\\
  &\quad\quad\quad\text{by \cref{lem:normbound}}\\
  &\leq \frac{\mleft(\Amax+\half\mright)^2}{2\Amin} \NW{u-\Ih u}^2 + \frac{\Amin}2 \NW{u-\uh}^2 + k^2\mleft(\Amin+ \NLiDRR{n}\mright) \NLtD{u-\uh}^2,\\
\end{align*}
by Cauchy's inequality \eqref{eq:cauchy} with $\eps = \Amin.$ Therefore,
\beqs
\NW{u-\uh}^2 \leq \frac2{\Amin} \mleft(\frac{\mleft(\Amax+\half\mright)^2}{2\Amin} \NW{u-\Ih u}^2+ k^2\mleft(\Amin+ \NLiDRR{n}\mright) \NLtD{u-\uh}^2\mright)
\eeqs
and hence
\beq\label{eq:hherrboundnearly}
\NW{u-\uh} \lesssim \frac1{\Amin^{1/2}} \mleft(\frac{\Amax+\half}{\Amin^{1/2}} \NW{u-\Ih u}+ k\mleft(\Amin+ \NLiDRR{n}\mright)^{1/2} \NLtD{u-\uh}\mright).
\eeq
By substituting \eqref{eq:hhszbound} and \eqref{eq:hherrltbound} into \eqref{eq:hherrboundnearly} we obtain \eqref{eq:hherrwbound}.

To show that $\uh$ exists, as in the proof of \cref{lem:ellprojbounds} we can use the error bound \eqref{eq:hherrltbound} to show that $\uh$ is unique, and we can then use the fact that $\Vh$ is finite-dimensional to show that $\uh$ exists.
\epf








% \section{Summary}


% BELOW HERE IS OLD STUFF


% The key step in \cite{ChNi:18} is that properties of the `elliptic projection' allow us to prove the bound
% \beqs
% k\NLtD{u-\uh} \lesssim h^2k^3 \NLtD{f},
% \eeqs
% which we then leverage to show the main result.
% Contrast this with the corresponding bound for the Scott-Zhang quasi-interpolant:
% \beq\label{eq:SZ}
% k\NLtD{u-\Ih u} \lesssim  \mleft(hk^2 + \mleft(hk\mright)^2\mright)\NLtD{f}.
% \eeq
% Any argument involving the quasi-interpolant will therefore include the condition $h \lesssim k^{-2},$ due to the $hk^2$ term in \eqref{eq:SZ}, whereas the bound \eqref{eq:Lt} only includes the term $h^2k^3,$ and so will only require the condition $h \lesssim k^{-3/2}.$


% \section{Details}

% We are interested in the error in the $h$-FEM for the following variational problem:

% Find $u \in V$ such that  for all $v \in V$
% \beqs
% a(u,v) \de \int_D \IP{A \grad u}{\grad v} - k^2 n u\vbar - ik\int_{\GI} u \vbar = \int_{D} f \vbar + \int_{\GI} g \vbar,
% \eeqs
% where all of the boundary terms are understood as traces, and we assume that $A$, $n$ and the domain $D$ (with impedance boundary $\GI$ are sufficiently well-behaved such that an a priori bound of the form
% \beqs
% \NW{u} \lesssim \NLtD{f}
% \eeqs
% for all $k \geq \kz,$ for some $\kz > 0$, holds, where the hidden constant may depend on $A$ and $n,$ but is independent of $k$. 

% We introduce the following `elliptic projection' sesquilinear form:

% \beqs
% \api(\vo,\vt) = \int_D \IP{A \grad \vo}{\vt} - ik\int_{\GI} \vo\vtbar.
% \eeqs
% %so that the equation $\api(u,v) = \int_D \ftilde\vbar + \int_{\GI} \gtilde \vbar$ corresponds to the PDE
% %\beqs
% %\grad \cdot \mleft(A \grad u\mright) = -\ftilde \tin D,
% %\eeqs
% %\beqs
% %\dn u - iku = \gtilde \ton \GI.
% %\eeqs

% We then define the elliptic projection $\Ph $ of $w \in \HoD$ by
% \beq\label{eq:ellproj}
% \api(\vh,\Ph w) = \api(\vh,w) \tfa \vh \in \Vh,
% \eeq
% where $\Vh$ is the finite-element space.We also define the corresponding energy norm
% \beqs
% \Npi{u} = \sqrt{\abs{\api(u,u)}}.
% \eeqs

% Using this elliptic projection, we are able to prove the following theorem\footnote{I've generalised this slightly from \cite{ChNi:18} as you can get a little bit more out of their argument, so that you \emph{nearly} get that the relative error is bounded.}

% \bth
% If $u$ solves the Helmholtz equation, $u \in \HtD,$ and $h \lesssim k^{-3/2},$ then
% \beqs
% \frac{\NW{u-\uh}}{\NW{u}} \lesssim hk + h^2k^3 + \mleft(h + h^2k^2\mright)\frac{\NLtD{f} + \NHhGI{g}}{\NW{u}}.
% \eeqs
% \enth

% \bre
% If we can show that
% \beqs
% \NW{u} \gtrsim \NLtD{f} + \NHhGI{g},
% \eeqs
% for all $k \geq \kz,$ then we will be able to prove that the relative error is bounded, under the assumption $h \lesssim k^{-3/2}.$
% \ere

% \bpf[Sketch Proof]
% The proof consists of four main steps:

% \paragraph{Step 1} If $w \in \HtD,$ then
% \beq\label{eq:step1}
% \Npi{w - \Ph w} \lesssim h\mleft(\NHtD{w} + k\NHoD{w}\mright).
% \eeq
% Proof - By Cea's Lemma.

% \paragraph{Step 2} If $w \in \HtD,$ then
% \beq\label{eq:step2}
% k \NLtD{w-\Ph w} \lesssim h^2 k\mleft(\NHtD{w} + k\NHoD{w}\mright).
% \eeq
% Proof - duality argument using \eqref{eq:step1}.

% \paragraph{Step 3} If $u$ solves the Helmholtz equation, $u \in \HtD,$ and $h \lesssim k^{-3/2},$ then
% \beq\label{eq:step3}
% \NLtD{u-\uh} \lesssim h^2 k\mleft(k \NHoD{u} + \NLtD{f} + \NHhGI{g}\mright).
% \eeq
% Proof - duality argument, splitting $a$ into an `$\api$' part and a `$k^2 \IPLtD{\cdot}{\cdot}$' part, and using \eqref{eq:step1}, \eqref{eq:step2}, and \eqref{eq:step3}.
% \epf

%We first prove the following lemma\optodo{this in cref}.
%
%\ble\label{lem:projbound}
%If $u$ exists, then the error in the elliptic projection satisfies the bounds
%\beq\label{eq:projboundenergy}
%\Npi{u - \Ph u} \lesssim hk \NLtD{f}
%\eeq
%\ele
%and
%\beq\label{eq:projboundlt}
%k \NLtD{u - \Ph u} \lesssim h^2 k^2 \NLtD{f}
%\eeq
%\bpf[Sketch Proof of \cref{lem:projbound}]
%We obtain  \eqref{eq:projboundenergy} using standard arguments, as $\api$ is bounded and coercive, so we apply Cea's Lemma (the factor $k$ arising from $u$ on the right-hand side of \eqref{eq:ellproj}).
%
%We then obtain  \eqref{eq:projboundlt} using \eqref{eq:projboundenergy} and a duality argument.
%\epf
%
%We can then prove the following theorem\optodo{This in cref?}:
%
%\bth\label{thm:bound}
%If $h \lesssim k^{-3/2}$ (where the hidden constant is dependent on $A$ and $n$) then
%\beq\label{eq:wbound}
%\NW{u - \uh} \lesssim \NLtD{f},
%\eeq
%where, again, the hidden constant depends on $A$ and $n.$
%\enth
%
%\bpf[Sketch Proof of \cref{thm:bound}]
%We first seek the bound 
%\beqs
%\NLtD{u-\uh} \lesssim (hk)^2 \NLtD{f}
%\eeqs
% via the following steps.
%\ben
%\item Study $\psi$ satisfying the dual problem 
%\beq\label{eq:dualplain}
%a(v,\psi) = \IPLtD{v}{u-\uh}.
%\eeq
%\item Use Galerkin orthogonality to obtain 
%\beq\label{eq:dual}
%\NLtD{u-\uh}^2 = a(u-\uh,\psi-\Ph \psi).
%\eeq
%\item Split $a$ into an`elliptic projection' part and a `$k^2 L^2$' part, i.e. 
%\beqs
%a(u-\uh,\psi-\Ph \psi) = \api(u-\uh,\psi-\Ph \psi) - k^2\IPLtD{u-\uh}{\psi-\Ph\psi}.
%\eeqs
%\item The `elliptic projection' part is bounded by $(hk)^2\NLtD{u-\uh}\NLtD{f}$ by \cref{lem:projbound}.The $k^2 L^2$ part is bounded by $h^2k^3\NLtD{u-\uh}^2$ (again using \cref{lem:projbound}) and hence can be absorbed into the left-hand side of \eqref{eq:dual} if $h^2k^3 \lesssim 1.$
%\een
%The bound on $\NW{u-\uh}$ is obtained using the $L^2$ result, the fact that $h \lesssim k^{-3/2}$, and the fact that $a$ satisfies a G\r{a}rding inequality.
%\epf
%
%\section{Comparison with Schatz' argument for quasi-optimality}
%
%The core of Schatz' argument (as in, for example, \cite[Theorem 5.21]{Sp:15}) is the same dual problem \eqref{eq:dualplain}.



%% \section{Introduction}

%% \paragraph{Structure}

The structure should be (each is a section)

\bit
\item Setup of final problem, main result.
\item Generic MLMC theory for non-path-dependent levels, but $k$-dependence everything.
\item $k^{-3/2}$ generic theory
\item Applying generic theory to HH IIP/EDP \emph{with $A=I$ and a scatterer} to obtain main result.
\bit
\item Specify exact setup (domain etc. as in $k^{3/2}$, stochasitc side, specify model as in stochastic (but give concrete example) and quote from stochastic about well-posedness, bounds, etc.)
\item Explicit proof for $H^2$ constants
\item Show everything (mehs condition, a priori bound) is ok and path independent
\item seriving values of $\alpha$ etc. for different scenarios
\item putting it all together to get mlmc results
\item compare to mc
\eit
\eit

\section{MLMC Complexity Theorem}\label{sec:comp}

%% In this \lcnamecref{sec:comp} we state and prove an abstract result on the convergence of multi-level Monte Carlo methods, laregly following the proof of \cite[Theorem 1]{ClGiScTe:11}. Our result is a generalisation of \cite[Theorem 1]{ClGiScTe:11} in the following three ways:
%% \ben
%% \item In \cite{ClGiScTe:11} it is assumed that the convergence of the approximate QoIs $\Qhl$, and the cost of producing samples of these QoIs, only depends on the parameter $\hl$ (where, in stochastic PDE applications, $\hl$ is the mesh size for the finite-element discretisation). However, in this work, we assume that the convergence and cost also depend on another parameter $k,$ and we make the dependence of the final computational cost of the MLMC method explicit in $k.$ In our application to the Helmholtz equation, $k$ will be the wavenumber of the problem.
%% \item In \cite{ClGiScTe:11} it is assumed that the approximating QoIS $\Qhl$ exist for all levels $l$. This corresponds to the finite-element solution of the PDE under investigation existing for all mesh sizes $h.$ Whilst this assumption is true for the stationary diffusion equation studied in \cite{ClGiScTe:11}, it is \emph{not} true for the Helmholtz equation that we study here. Therefore we make the additional assumption (\cref{ass:qoie} below) that $\Qhl$ only exists for sufficiently small $\hl.$
%% \item In \cite{ClGiScTe:11} the error $\eps$ incurred in the MLMC method is equally divided between the bias and the variance of the MLMC method (see the Proof of \cref{thm:mlmccomp3}). However, in this work we assume that there is a quantity $\splitting \in (0,1)$ (see \cref{ass:splittingbounds}), possibly dependent on $k$ that allows a vairable `split' of the error between the bias and the variance. Our main use of this is in\optodo{Insert refs once it's done}, where we use this variable splitting to compensate for the fact that to bound the (squared) bias error by $\eps^2/2$ would mean we take $\hL \lesssim k^{-1},$ but to ensure the finite-element solution $\uh$ exists, we must take $\hL \lesssim k^{-3/2}.$
%% \een
%% We now proceed to prove our abstract MLMC convergence result, comtaining the generalisations metioned above.

Let $\OFP$ be a probability space, and let $Q$ be a random variable\footnote{One can think of $Q$ as being $Q(u),$ where $u$ is the solution of some stochastic PDE.} on $\OFP$ such that $\EXP{Q} < \infty.$ We will refer to $Q$ as the \defn{quantity of interest} or QoI. In order to define the multi-level Monte Carlo (MLMC) method for estimating $\EXP{Q},$ we must also define the following quantities, following  \cite[Theorem 1]{ClGiScTe:11}. We assume there exist
\bit
\item A set of levels\footnote{One can think of $\hl$ as the mesh size associated with level $l$.} $\set{\hl}_{l=0}^L$ ($L$ to be chosen) such that $\hl =\frac{\hlmo}s$ for $l \geq 1.$
\item A set of random variables (that may or may not exist)\footnote{One can think of $\Qhtilde$ as $Q(\uh),$ where $\uhl$ is the finite-element solution of the PDE with mesh size $h  $.} $\set{\Qhtilde}_{h \in (0,1)}.$
  \eit

We denote the random variables $\Qhtilde,$ in order to simplify the notation for mesh dependence in what follows.

In order to do things for the Helmholtz equation, we use the following assumption:

\bas[Mesh conditions for existence and uniqueness]
There exists a measurable function $\Cmesh:\Omega\rightarrow \RRp$ and a function $\mesh:\RRp\rightarrow \RRp$ such that $\Qhtilde(\omega)$ exists (and satisfies the error bounds etc. below) if
\beqs
h \leq \Cmesh(\omega)\mesh(k).
\eeqs
Note that $\mesh(k)\rightarrow 0$ as $k\rightarrow \infty.$
\eas

Observe that for a given $k, \omega$ there is no guarantee that $\Qhtilde(\omega)$ exists. Therefore, we follow [Graham, Parkinson, Scheichl] and define
\beq\label{eq:hmaxomega}
\hmaxomega \de \Cmesh(\omega)\mesh(k).
\eeq
We then define
\beqs
\homega \de \max\set{h,\hmaxomega}
\eeqs
and subsequently define
\beqs
\Qh(\omega) \de \Qtildehomega.
\eeqs
That is, the random variable $\Qh$ is (thinking about things in terms of PDEs etc.) the QoI evaluated at the numerical solution, where that solution is taken on a mesh that is the finer of $h$ and $\hmaxomega$. This guarantees the QoI exists, and the error bounds below hold.

\bre[What is $\mesh(k)$?]
If nontrapping, $\mesh(k)=k^{-3/2}$. If trapping, more stringent, nothing proved in literature, but would expect to be similar to results for contant wavespeed.
\ere

With this setup in place, we define the following quantities.

We define the correction operators\optodo{You may be able to save some time computing these - if both $\hl$ and $\hlmo$ are larger that $\hmaxomega$, then the difference between them is zero.} between the levels by $\Yl \de \Qhl - \Qhlmo, l \geq 1,$ $\Yz = \Qhz.$ We let $\Ylhat$ be an unbiased estimator of $\Yl$, i.e., $\EXP{\Ylhat} = \EXP{\Yl}.$ In what follows $\Ylhat$ will be the Monte Carlo estimator
 \beqs
\Ylhat \de \frac1{\Nl}\sum_{i=1}^{\Nl} \Yli,
 \eeqs
 with $\Nl$ to be chosen, where $\Yli$ denotes independent samples of $\Yl$. Finally we are able to define the \defn{multi-level Monte Carlo estimator}
 \beqs
 \QhatMLhL \de \sum_{l=1}^L \Ylhat,
 \eeqs
 where the $\Ylhat$ are independent.

  The following assumptions
  % \lcnamecrefs{ass:coarse}
   will form the backbone of our analysis. They are a generalisation of the assumptions contained in \cite{ClGiScTe:11,ChScTe:13} for the MLMC method, the generalisation being that we assume that the quantities below depend not only on the levels $\hl$ but also on some additional parameter $k>1.$ When this theory is applied to the Helmholtz equation, $k$ will be the wavenumber of the Helmholtz equation.

%% The following assumption (which will be realised in a more concrete setting for the Helmholtz equation) concerns the existence of the approximating QoIs $\Qhl.$

%% \bas[Existence of $\Qhl$]\label{ass:qoie}
%% There exist $\Ccoarse,\coarseexp > 0$ with $\Ccoarse$ independent of $k$ such that if
%% \beqs
%% \hl \leq \Ccoarse k^{-\coarseexp},
%% \eeqs
%% then the QoI $\Qhl$ exists.
%% \eas

\bas[Convergence of numerical method]\label{ass:a}
There exist $\co, \alpha, \sigma> 0$, such that $\co$ is independent of $h$ and $k$, and
\beqs
\abs{\EXP{\Qh-Q}} \leq \co k^\sigma h^{\alpha}.
\eeqs
\eas

\bas[Variance of correction operators]\label{ass:b}
There exist $\ct, \beta, \tau > 0$, such that $\ct$ is independent of $h$ and $k,$ and
\beqs
\Vl \de \VAR{\Yl} \leq \ct k^\tau\hl^{\beta},
\eeqs  where $\VAR{\cdot}$ denotes variance.
\eas

\bas[Cost of one sample]\label{ass:costone}
There exist $\cthtilde, \gamma > 0$ such that $\cthtilde$ is independent of $h$ and $k$, and if $\Qhtilde(\omega)$ exists, then
\beqs
\Cost{\Qhtilde(\omega)} \leq \cthtilde(\omega) h^{-\gamma},
\eeqs
\eas

In order to obtain a nice expression for the cost of computing one sample of $\Qh,$ we require the following assumption on the coarse space:

\bas[Dependence of coarse space on $k$]\label{ass:coarse}
We let
\beqs
\hz = \Ccoarse \mesh(k).
\eeqs
for some chosen constant $\Ccoarse > 0.$
\eas

\ble[Expected cost of one sample]\label{lem:c}
If
\beq\label{eq:cass}
\cthtilde \in \LpO\text{ for some }p \geq 1 \tand 1/\Cmesh \in \LqgammaO \tfor q \text{ the conjugate exponent of } p,
\eeq
then
\beq\label{eq:singlecost}
\EXP{\Cost{\Qh}} \leq \cth h^{-\gamma},
\eeq
where $\cth = \NLoO{\cthtilde} + \Ccoarse^\gamma \NLpO{\cthtilde}\NLqgammaO{\Cmesh^{-1}}^\gamma.$
\ele

\bpf[Proof of \cref{lem:c}]
The proof follows closely that in \cite[Lemma 5.8]{GrScPa:19}.

We have
\begin{align}
\EXP{\Cost{\Qh}} &= \int_{\Omega} \Cost{\Qhomegatilde(\omega)} \ddPPomega\nonumber\\
&\leq \int_\Omega \cthtilde(\omega) \homega^{-\gamma} \ddPPomega \text{ by \eqref{eq:singlecost}}\nonumber\\
&\leq \int_{\Omega} \cthtilde(\omega) \mleft(h^{-\gamma} +  \mleft(\hmaxomega\mright)^{-\gamma} \mright) \ddPPomega \text{ as } \homega = \max\set{h,\hmaxomega} \leq h + \hmaxomega\nonumber\\
&=\int_{\Omega} \cthtilde(\omega) \mleft(h^{-\gamma} + \Cmesh^{-\gamma} \mesh(k)^{-\gamma}\mright)\ddPPomega \text{ by \eqref{eq:hmaxomega}}\nonumber\\
= h^{-\gamma} \NLoO{\cthtilde} + \mesh(k)^{-\gamma}\EXP{\cthtilde\Cmesh^{-\gamma}}\label{eq:cfinal}
\end{align}
As the assumptions \eqref{eq:cass} hold, the result follows.
\epf
 
% We write $\Vl$ for $\VAR{\Yl}.$
 
 We want to determine the choices of $L$ and $\Nl, l = 0,\ldots,L,$ such that the root-mean-squared eror (RMSE)
 \beqs
 \err{\QhatMLhL} \de \mleft(\EXP{\mleft(\QhatMLhL - \EXP{Q}\mright)^2}\mright)^{\half}
 \eeqs
 satisfies $\err{\QhatMLhL} \leq \eps,$ for some pre-defined $\eps > 0.$

The proof of the main \lcnamecref{thm:mlmccomp} will require the following \lcnamecref{lem:sumbound}.

\ble\label{lem:sumbound}
If $L$ is given by
\beq\label{eq:Ldef}
L = \ceil{\frac1\alpha\log_{s}\mleft(\sqrt{2}\co  \Ccoarse^\alpha k^{\sigma}\mesh(k)^\alpha \eps^{-1}\mright)},
\eeq
then, for $s>1$ and $\delta \in \RR,$ we have the bound
\beq\label{eq:sumbound}
\sum_{l=0}^{L} s^{\delta l} \leq
\begin{cases}
L+1 & \tif \delta = 0,\\
\frac{\mleft(\sqrt{2}\co\mright)^{\frac\delta\alpha}\Ccoarse^{\delta}s^{\delta}}{1-s^{-\delta}}k^{\frac{\delta\sigma}{\alpha}}\mesh(k)^\delta\eps^{-\frac\delta\alpha} &\tif \delta >0\\
\frac{\mleft(\sqrt{2}\co\mright)^{\frac\delta\alpha}\Ccoarse^{\delta}}{1-s^{-\delta}}k^{\frac{\delta\sigma}{\alpha}}\mesh(k)^\delta\eps^{-\frac\delta\alpha}&\tif \delta < 0
\end{cases}
\eeq
\ele

\bpf[Proof of \cref{lem:sumbound}]
The proof follows that in \cite{ClGiScTe:11}. We first observe that, since $L$ is given by \eqref{eq:Ldef}, it follows that
\beq\label{eq:Lbounds}
\frac1\alpha\log_s\mleft(\sqrt{2}\co\Ccoarse^\alpha k^{\sigma}\mesh(k)^\alpha \eps^{-1}\mright) \leq L < \frac1\alpha\log_s\mleft(\sqrt{2}\co\Ccoarse^\alpha k^{\sigma}\mesh(k)^\alpha \eps^{-1}\mright) + 1.
\eeq
Rearranging \eqref{eq:Lbounds}, we obtain the bounds
\beq\label{eq:saLbounds}
\sqrt{2}\co \Ccoarse^\alpha k^{\sigma}\mesh(k)^\alpha\eps^{-1} \leq s^{\alpha L} < \sqrt{2}\co \Ccoarse^\alpha k^{\sigma}\mesh(k)^\alpha\eps^{-1}s^\alpha.
\eeq
If $\delta > 0,$ then we use the right-hand bound in \eqref{eq:saLbounds} to obtain
\beq\label{eq:sdLpos}
s^{\delta L} < \mleft(\sqrt{2}\co\mright)^{\frac\delta\alpha}\Cppw^{\delta}k^{\frac{\delta\sigma}{\alpha}}\mesh(k)^\delta\eps^{-\frac\delta\alpha}s^{\delta},
\eeq
and if $\delta < 0,$ we use the left-hand bound in \eqref{eq:saLbounds} to obtain
\beq\label{eq:sdLneg}
s^{\delta L} \leq \mleft(\sqrt{2}\co\mright)^{\frac\delta\alpha}\Cppw^{\delta}k^{\frac{\delta\sigma}{\alpha}}\mesh(k)^\delta\eps^{-\frac\delta\alpha}.
\eeq
We now observe that, for $\delta \neq 0,$
\begin{align}
\sum_{l=0}^L s^{\delta l} &= \frac{s^{\delta\mleft(L+1\mright)} -1}{s^{\delta}-1}\nonumber\\
&= \frac{s^{\delta L} - s^{-\delta}}{1-s^{-\delta}}\nonumber\\
&\leq \frac{s^{\delta L}}{1-s^{-\delta}},\label{eq:ssumbound}
\end{align}
since $s^{-\delta} > 0,$ as $s >0.$ Combining \eqref{eq:ssumbound} with \eqref{eq:sdLpos} and \eqref{eq:sdLneg}, we obtain \eqref{eq:sumbound} in the cases $\delta \neq 0.$ The case $\delta=0$ is straightforward.
\epf


%
 \paragraph{The nice case, where $k^{-\sigma/\alpha} \lesssim k^{-\coarseexp}.$}
\optodo{Might need todo something with the constants, as we need $\hL$ (as calculated) $< \hz.$ ensuring the constants are monotone is probably sufficient, as it'll just mean `for $\eps$ sufficiently small'.}
 The following theorem describes the computational effort needed to obtain RMSE $\leq \eps$. It is exactly the same as \cite[Theorem 1]{ClGiScTe:11}, but with the dependence on all the parameters explicit.%, and with some additional cases enumerated. %\Cref{thm:mlmccomp} contains more cases than in \cite[Theorem 1]{ClGiScTe:11} because \cite[Theorem 1]{ClGiScTe:11} makes the assumption throughout that $\alpha \geq 1/2\min\set{\beta,\gamma}.$ This assumption does not always hold for the Helmholtz equation (see the cases of a direct solver in 3-D below), however, examining the proof of \cite[Theorem 1]{ClGiScTe:11}  shows that in any given case, one only needs the assumption $\alpha \geq \beta/2$ or the assumption $\alpha \geq \gamma2$, never both at the same time. Therefore, for convenience, we explicitly state when these conditions are needed, and for completeness, we give the results when these conditions are violated. 

  The next two \lcnamecref{ass:powersnice}\optodo{plural} means that the restriction on the coarse space in \cref{ass:coarse} do not come into play,

 \bas[Epsilon sufficiently small]\label{ass:constants}
 Assume
 \beqs
\eps \leq \sqrt{2} \co \Ccoarse^{\alpha}.
 \eeqs
 \eas

 \bas\label{ass:powersnice}
 Suppose
 \beqs
\frac{\sigma}{\alpha} \geq \coarseexp.
 \eeqs
 \eas
 
 \bth[MLMC Complexity Theorem]\label{thm:mlmccomp}
If \cref{ass:constants,ass:powersnice} hold, $L$ is given by
\beq\label{eq:Lcond}
L = \ceil{\frac1\alpha\log_{s}\mleft(\sqrt{2}\co  \Ccoarse^\alpha k^{\sigma-\coarseexp\alpha} \eps^{-1}\mright)},
\eeq
that is,
\beqs
\hL \leq \mleft(\frac{\eps}{\sqrt{2}\co k^\sigma}\mright)^{\frac1\alpha},
\eeqs
and the number of samples on each computational level is given by
\beqs
\Nl = \ceil{\frac2{\eps^{2}} \mleft(\frac{\Vl}{\Cl}\mright)^{\half}\sum_{j=0}^{L} \mleft(\Vj\Cj\mright)^{\half}},
\eeqs
then computational effort $\CMLhL(\eps)$ required to obtain $\err{\QhatMLhL} \leq \eps$ satisfies the bounds
 
 \begin{numcases}{ \CMLhL(\eps) \lesssim}
 k^{\tau + \rho+\coarseexp\mleft(\gamma - \beta\mright)}\eps^{-2}\mleft(\log_s\mleft(\sqrt{2}\co\Cppw^\alpha k^{\sigma-\coarseexp\alpha} \eps^{-1}\mright)+2\alpha\mright)^2 +  k^{\rho +  \frac{\gamma\sigma}\alpha}\eps^{-\frac\gamma\alpha}
 & if $\beta = \gamma$,\label{eq:mlmchheq}\\ 
k^{\tau + \rho+\mleft(\gamma-\beta\mright)\frac\sigma\alpha}\eps^{-2+\mleft(\frac{\beta-\gamma}{\alpha}\mright)}
 +  k^{\rho +  \frac{\gamma\sigma}\alpha}\eps^{-\frac\gamma\alpha} & otherwise.\label{eq:mlmchhoth}
\end{numcases}
 \enth
 \optodo{Need to say why the latter two cases are the same - in one case $\gamma/\alpha$ dominates, and in the other case the other term dominates? (At least in the Cliffe et. al. set up)}
 \bpf[Proof of \cref{thm:mlmccomp}]
 \ednote{This isn't all the details of the proof, but the bits I've skipped over are exactly the same as those in {\cite{ClGiScTe:11}}.}
 
We first decompose the (squared) mean-squared error into the bias error and the sampling error:

\beqs
\errQhatMLhL^2 = \mleft(\EXP{\QhatMLhL} - \EXP{Q}\mright)^2 + \underbrace{\EXP{\mleft(\QhatMLhL - \EXP{\QhatMLhL}\mright)^2}}_{V\de},
\eeqs
the first term is the \emph{bias}, and the second term is the \emph{variance} of the estimator $\QhatMLhL.$ We now proceed to choose the parameters $L$ and $\Nl, l = 0,\ldots,L$ such that we can bound both the bias and the variance by $\eps^2/2.$

We first bound the bias, to do this, we only need to choose $L.$ One can show\ednote{As in {\cite{ClGiScTe:11}}} that the bias is equal to $\abs{\EXP{\QhL - Q}}^2.$ By \cref{ass:constants,ass:powersnice}, we don't need to worry about the coarse mesh restriction\optodo{Make this proper speak}. Therefore a sufficient condition for the bias to be $\leq \eps^2/2$ is (by \cref{ass:a})
\beqs
\co k^\sigma \hL^\alpha \leq \frac{\eps}{\sqrt{2}},
\eeqs
that is
\beq\label{eq:hLcond}
\hL \leq \mleft(\frac{\eps}{\sqrt{2}\co k^\sigma}\mright)^{\frac1\alpha}.
\eeq
\ednote{Observe that if $Q$ is the weighted $H^1$ norm, then we assume (see below for details) $\alpha=2$ and $\sigma=3,$ so we require $\hL \lesssim k^{-\frac32}.$ If we take $Q$ to be the $L^2$ norm, and assume $\alpha=2$ and $\sigma=2,$ then we only require $\hL \lesssim k^{-1}.$}

As $\hL = \hz s^{-L},$ it follows from \eqref{eq:hLcond} that a sufficient condition for the bias to be $\leq \eps^2/2$ is
\beq\label{eq:Lcondpart}
L = \ceil{\frac1\alpha\log_s\mleft(\sqrt{2}\co k^\sigma \hz^\alpha \eps^{-1}\mright)}.
\eeq
As $\hz = \Ccoarse k^{-\coarseexp},$ we can simplify \eqref{eq:Lcondpart} to obtain \eqref{eq:Lcond}.
% \beqs
% L = \ceil{\frac1\alpha\log_s\mleft(\sqrt{2}\co\Ccoarse^\alpha k^{\sigma-\coarseexp\alpha} \eps^{-1}\mright)}.
% \eeqs

We now seek to bound the variance. One can show\ednote{Again, as in \cite{ClGiScTe:11}} the variance $V = \sum_{l=0}^L \Nl^{-1} \Vl,$ and the cost is $\cC = \sum_{l=0}^L \Nl \Cl.$

To find the optimal number of samples per level (the values of $\Nl, l=0,\ldots,L$) we formulate this as an optimisation problem to find $\Nl$ that minimise $\cC$, subject to $V=\eps/2.$ This can be solved using a Lagrange multiplier as in \cite{Gi:15}, and we obtain
\beq\label{eq:Nl}
\Nl = \ceil{\frac2{\eps^{2}} \mleft(\frac{\Vl}{\Cl}\mright)^{\half}\sum_{j=0}^L \mleft(\Vj\Cj\mright)^{\half}}.
\eeq
\optodo{Check this is correct, should it be divided by the sum?}
We now just need to infer the computational complexity for MLMC with $L$ given by \eqref{eq:Lcond} and the $\Nl$ given by \eqref{eq:Nl}.

The computational complexity $\cC$ is given by
\begin{align}
\cC &= \sum_{l=0}^{L} \Cl \Nl\nonumber\\
&\leq \sum_{l=0}^L \Cl \mleft(\frac2{\eps^{2}} \mleft(\frac{\Vl}{\Cl}\mright)^{\half}\sum_{j=0}^L \mleft(\Vj\Cj\mright)^{\half} + 1\mright) \text{ (by \eqref{eq:Nl})}\nonumber\\
&= 2\eps^{-2}\mleft(\sum_{l=0}^L\mleft(\Vl\Cl\mright)^{\half}\mright)^2 + \sum_{l=0}^L \Cl\nonumber\\
&\leq 2 \ct \cth k^{\tau + \rho}\eps^{-2} \mleft(\sum_{l=0}^L \hl^{\frac{\beta-\gamma}2}\mright)^2 + \cth k^\rho \sum_{l=0}^L \hl^{-\gamma} \text{ (by \cref{ass:b,ass:c})}\nonumber\\
&= 2 \ct\cth k^{\tau + \rho}\eps^{-2}\hz^{\beta-\gamma}\mleft(\sum_{l=0}^L s^{-l\mleft(\frac{\beta-\gamma}2\mright)}\mright)^2 + \cth k^\rho \hz^{-\gamma} \sum_{l=0}^L s^{\gamma l} \text{ (by definition of } \hl\text{ )}\nonumber\\
&=2 \ct\cth \Cppw^{\beta-\gamma}k^{\tau + \rho+\coarseexp\mleft(\gamma - \beta\mright)}\eps^{-2}\mleft(\sum_{l=0}^L s^{l\mleft(\frac{\gamma-\beta}2\mright)}\mright)^2 + \cth\Cppw^{-\gamma} k^{\rho + \gamma\coarseexp}  \sum_{l=0}^L s^{\gamma l} \text{ (by definition of } \hz\text{ )}\nonumber\\
&\leq2\ct\cth \Cppw^{\beta-\gamma}k^{\tau + \rho+\coarseexp\mleft(\gamma - \beta\mright)}\eps^{-2}\mleft(\sum_{l=0}^L s^{l\mleft(\frac{\gamma-\beta}2\mright)}\mright)^2 +  \frac{\mleft(\sqrt{2}\co\mright)^{\frac\gamma\alpha}\cth s^{\gamma}}{1-s^{-\gamma}}k^{\rho +  \frac{\gamma\sigma}\alpha}\eps^{-\frac\gamma\alpha} \text{ (since }\gamma>0,\text{ by \cref{lem:sumbound})}.\label{eq:complexitymidway}
\end{align}
To bound the sum in the first part of \eqref{eq:complexitymidway}, we must distinguish three cases based on $\gamma - \beta.$

If $\gamma=\beta,$ then \eqref{eq:complexitymidway} becomes (using \cref{lem:sumbound})
\begin{multline}
2 \ct\cth \Cppw^{\beta-\gamma}k^{\tau + \rho+\coarseexp\mleft(\gamma - \beta\mright)}\eps^{-2}\mleft(L+1\mright)^2 +  \frac{\mleft(\sqrt{2}\co\mright)^{\frac\gamma\alpha}\cth s^{\gamma}}{1-s^{-\gamma}}k^{\rho +  \frac{\gamma\sigma}\alpha}\eps^{-\frac\gamma\alpha}\\
\leq
2\ct\cth \Cppw^{\beta-\gamma}k^{\tau + \rho+\coarseexp\mleft(\gamma - \beta\mright)}\eps^{-2}\mleft(\frac1\alpha\log_s\mleft(\sqrt{2}\co\Cppw^\alpha k^{\sigma-\coarseexp\alpha} \eps^{-1}\mright)+2\mright)^2 +  \frac{\mleft(\sqrt{2}\co\mright)^{\frac\gamma\alpha}\cth s^{\gamma}}{1-s^{-\gamma}}k^{\rho +  \frac{\gamma\sigma}\alpha}\eps^{-\frac\gamma\alpha}\label{eq:gammaequal}
\end{multline}
by \eqref{eq:Lcond}.

If $\gamma > \beta$ then by \cref{lem:sumbound} \eqref{eq:complexitymidway} becomes
\beq
2 \ct\cth \Cppw^{\beta-\gamma}
\frac{\mleft(\sqrt{2}\co\mright)^{\mleft(\frac{\gamma-\beta}{\alpha}\mright)}\Cppw^{\mleft(\gamma-\beta\mright)}s^{\gamma-\beta}}{\mleft(1-s^{\mleft(\frac{\beta-\gamma}2\mright)}\mright)^{2}}k^{\tau + \rho+\mleft(\gamma-\beta\mright)\frac\sigma\alpha}\eps^{-2+\mleft(\frac{\beta-\gamma}{\alpha}\mright)}
 +  \frac{\mleft(\sqrt{2}\co\mright)^{\frac\gamma\alpha}\cth s^{\gamma}}{1-s^{-\gamma}}k^{\rho +  \frac{\gamma\sigma}\alpha}\eps^{-\frac\gamma\alpha}.\label{eq:gammagtr}
\eeq
If $\gamma < \beta,$ then by \cref{lem:sumbound} \eqref{eq:complexitymidway} becomes
\beq
2 \ct\cth \Cppw^{\beta-\gamma}
\frac{\mleft(\sqrt{2}\co\mright)^{\mleft(\frac{\gamma-\beta}{\alpha}\mright)}\Cppw^{\mleft(\gamma-\beta\mright)}}{\mleft(1-s^{\mleft(\frac{\beta-\gamma}2\mright)}\mright)^{2}}k^{\tau + \rho+\mleft(\gamma-\beta\mright)\frac\sigma\alpha}\eps^{-2+\mleft(\frac{\beta-\gamma}{\alpha}\mright)}
 +  \frac{\mleft(\sqrt{2}\co\mright)^{\frac\gamma\alpha}\cth s^{\gamma}}{1-s^{-\gamma}}k^{\rho +  \frac{\gamma\sigma}\alpha}\eps^{-\frac\gamma\alpha},\label{eq:gammaless}
\eeq
the only difference from \eqref{eq:gammagtr} being the loss of the $s^{\gamma-\beta}$ term.

Removing all the terms that are not of interest from \eqref{eq:gammaequal}, \eqref{eq:gammagtr}, and \eqref{eq:gammaless}, we obtain \eqref{eq:mlmchheq} and \eqref{eq:mlmchhoth}.
\epf


The following theorem describes the computational effort needed to obtain RMSE $\leq \eps$. It is exactly the same as \cite[Theorem 1]{ClGiScTe:11}, but with the dependence on all the parameters explicit.%, and with some additional cases enumerated. %\Cref{thm:mlmccomp} contains more cases than in \cite[Theorem 1]{ClGiScTe:11} because \cite[Theorem 1]{ClGiScTe:11} makes the assumption throughout that $\alpha \geq 1/2\min\set{\beta,\gamma}.$ This assumption does not always hold for the Helmholtz equation (see the cases of a direct solver in 3-D below), however, examining the proof of \cite[Theorem 1]{ClGiScTe:11}  shows that in any given case, one only needs the assumption $\alpha \geq \beta/2$ or the assumption $\alpha \geq \gamma2$, never both at the same time. Therefore, for convenience, we explicitly state when these conditions are needed, and for completeness, we give the results when these conditions are violated. 

%% The following \lcnamecref{ass:constants3} will ensure that \cref{ass:qoie} is satisfied.

%%  \bas[$\eps$ sufficiently small]\label{ass:constants3}
%%  Assume
%%  \beqs
%% \eps \leq \sqrt{2} \co \Ccoarse^{\alpha} k^{\sigma-\coarseexp\alpha}.
%%  \eeqs
%%  \eas

\paragraph{Notation}

$\Cl \de \cth\hl^{-\gamma}$

$\cC \de \EXP{\Cost{\QhatMLhL}}$

\bas[Assumptions on $\eps$ and $k$ to make things nicer]\label{ass:epsk}
\beqs
\eps \leq \min\set{\frac{\sqrt{2}\co\Ccoarse^\alpha}{s^{2\alpha}},\frac1{\sqrt{2}\co\Ccoarse^\alpha}},
\eeqs
and
\beqs
k^\sigma \mesh(k)^\alpha \geq 1.
\eeqs
\eas

\bth[MLMC Complexity Theorem]\label{thm:mlmccomp}
If \cref{ass:epsk} holds,  $L$ is given by \eqref{eq:Ldef}
that is,
\beq\label{eq:hLcond}
\hL \leq \mleft(\frac\eps{\sqrt{2}\co k^{\sigma}}\mright)^{\frac1\alpha},
\eeq
and the number of samples on each computational level is given by
\beqs
\Nl = \ceil{\frac2{\eps^{2}} \mleft(\frac{\Vl}{\Cl}\mright)^{\half}\sum_{j=0}^{L} \mleft(\Vj\Cj\mright)^{\half}},
\eeqs
then computational effort $\CMLhL(\eps)$ required to obtain $\err{\QhatMLhL} \leq \eps$ satisfies the bounds
 
 \begin{numcases}{ \CMLhL(\eps) \lesssim}
k^\tau \mleft(\log\mleft(\frac{k^\sigma \mesh(k)^\alpha}\eps\mright)\mright)^2 \eps^{-2} + k^{\frac{\gamma\sigma}\alpha} \eps^{-\frac\gamma\alpha}  & if $\beta = \gamma$,\label{eq:mlmchheq}\\ 
k^{\tau + \mleft(\gamma-\beta\mright)\frac\sigma\alpha} \eps^{-2-\frac{\gamma-\beta}{\alpha}} + k^{\frac{\gamma\sigma}{\alpha}}\eps^{-\frac\gamma\alpha} & otherwise.\label{eq:mlmchhoth}
\end{numcases}
 \enth
 \bpf[Proof of \cref{thm:mlmccomp}]
We first decompose the (squared) mean-squared error into the bias error and the sampling error:

\beqs
\errQhatMLhL^2 = \mleft(\EXP{\QhatMLhL} - \EXP{Q}\mright)^2 + \underbrace{\EXP{\mleft(\QhatMLhL - \EXP{\QhatMLhL}\mright)^2}}_{V\de},
\eeqs
the first term is the \emph{bias}, and the second term is the \emph{variance} of the estimator $\QhatMLhL.$ We now proceed to choose the parameters $L$ and $\Nl, l = 0,\ldots,L$ such that we can bound both the bias and the variance by $\eps^2/2.$

We first bound the bias, to do this, we only need to choose $L.$ One can show that the bias is equal to $\abs{\EXP{\QhL - Q}}^2.$ Therefore a sufficient condition for the bias to be $\leq \eps^2/2$ is (by \cref{ass:a})
\beqs
\co k^\sigma \hL^\alpha \leq \frac{\eps}{\sqrt{2}},
\eeqs
that is, \eqref{eq:hLcond}. As $\hL = \hz s^{-L},$ it follows from \eqref{eq:hLcond} that a sufficient condition for the bias to be $\leq \eps^2/2$ is
\beq\label{eq:Lcondpart}
L = \ceil{\frac1\alpha\log_s\mleft(\sqrt{2}\co k^\sigma \hz^\alpha \eps^{-1}\mright)}.
\eeq
As $\hz = \Ccoarse \mesh(k),$ we can simplify \eqref{eq:Lcondpart} to obtain \eqref{eq:Ldef}.
% \beqs
% L = \ceil{\frac1\alpha\log_s\mleft(\sqrt{2}\co\Ccoarse^\alpha k^{\sigma-\coarseexp\alpha} \eps^{-1}\mright)}.
% \eeqs

We now seek to bound the variance. One can show the variance $V = \sum_{l=0}^L \Nl^{-1} \Vl,$ and the cost is: (following \cite{GrPaSc:19})
\begin{align}
\cC &= \EXP{\Cost{\QhatMLhL}}\nonumber\\
&\leq \sum_{l=0}^L \EXP{\Cost{\Ylhat}}\text{less equal as could save if you've had to over-refine on both levels in $\Ylhat$ - difference will be zero}\nonumber\\
&= \sum_{l=0}^L \sum_{i=1}^{\Nl} \EXP{\Cost{\Yli}}\nonumber\\
&\leq \sum_{l=0}^L \sum_{i=0}^{\Nl} \mleft(\EXP{\Cost{\Qhl}} + \EXP{\Cost{\Qhlmo}}\mright)\nonumber\\
&\leq \sum_{l=0}^L \Nl \mleft(\cth \hl^{-\gamma} + \cth \hlmo^{-\gamma}\mright)\nonumber\\
&=\sum_{l=0}^L \Nl\mleft(1+s^{-\gamma}\mright) \cth \hl^{-\gamma}\nonumber\\
&= \mleft(1+s^{-\gamma}\mright) \sum_{l=0}^L \Nl\Cl\label{eq:Cboundformin}
\end{align}

To find the optimal number of samples per level (the values of $\Nl, l=0,\ldots,L$) we formulate this as an optimisation problem to find $\Nl$ that minimise \eqref{eq:Cboundformin}, subject to $V=\eps/2.$ This can be solved using a Lagrange multiplier as in \cite{Gi:15}, and we obtain
\beq\label{eq:Nl}
\Nl = \ceil{\frac2{\eps^{2}} \mleft(\frac{\Vl}{\Cl}\mright)^{\half}\sum_{j=0}^L \mleft(\Vj\Cj\mright)^{\half}}.
\eeq
We now just need to infer the computational complexity for MLMC with $L$ given by \eqref{eq:Ldef} and the $\Nl$ given by \eqref{eq:Nl}.

The computational complexity $\cC$ is given by
\begin{align}
\cC &\leq \mleft(1+s^{-\gamma}\mright)\sum_{l=0}^{L} \Cl \Nl\nonumber\\
&\leq \mleft(1+s^{-\gamma}\mright)\sum_{l=0}^L \Cl \mleft(\frac2{\eps^{2}} \mleft(\frac{\Vl}{\Cl}\mright)^{\half}\sum_{j=0}^L \mleft(\Vj\Cj\mright)^{\half} + 1\mright) \text{ (by \eqref{eq:Nl})}\nonumber\\
&= 2\eps^{-2}\mleft(1+s^{-\gamma}\mright)\mleft(\sum_{l=0}^L\mleft(\Vl\Cl\mright)^{\half}\mright)^2 + \mleft(1+s^{-\gamma}\mright)\sum_{l=0}^L \Cl\nonumber\\
&= 2 \ct \cth \mleft(1+s^{-\gamma}\mright)k^{\tau}\eps^{-2} \mleft(\sum_{l=0}^L \hl^{\frac{\beta-\gamma}2}\mright)^2 + \cth \mleft(1+s^{-\gamma}\mright) \sum_{l=0}^L \hl^{-\gamma} \text{ (by \cref{ass:b,ass:c})}\nonumber\\
&= 2 \ct\cth \mleft(1+s^{-\gamma}\mright)k^{\tau}\eps^{-2}\hz^{\beta-\gamma}\mleft(\sum_{l=0}^L s^{l\mleft(\frac{\gamma-\beta}2\mright)}\mright)^2 + \cth\mleft(1+s^{-\gamma}\mright) \hz^{-\gamma} \sum_{l=0}^L s^{\gamma l} \text{ (by definition of } \hl\text{ )}\label{eq:complexitymidway}\\
%% &=2 \ct\cth \Cppw^{\beta-\gamma}k^{\tau + \rho+\coarseexp\mleft(\gamma - \beta\mright)}\eps^{-2}\mleft(\sum_{l=0}^L s^{l\mleft(\frac{\gamma-\beta}2\mright)}\mright)^2 + \cth\Cppw^{-\gamma} k^{\rho + \gamma\coarseexp}  \sum_{l=0}^L s^{\gamma l} \text{ (by definition of } \hz\text{ )}\nonumber\\
%% &\leq2\ct\cth \Cppw^{\beta-\gamma}k^{\tau + \rho+\coarseexp\mleft(\gamma - \beta\mright)}\eps^{-2}\mleft(\sum_{l=0}^L s^{l\mleft(\frac{\gamma-\beta}2\mright)}\mright)^2 +  \frac{\mleft(\sqrt{2}\co\mright)^{\frac\gamma\alpha}\cth s^{\gamma}}{1-s^{-\gamma}}k^{\rho +  \frac{\gamma\sigma}\alpha}\eps^{-\frac\gamma\alpha} \text{ (since }\gamma>0,\text{ by \cref{lem:sumbound})}.\label{eq:complexitymidway}
\end{align}

Using \cref{lem:sumbound}, the second term in \eqref{eq:complexitymidway} can be bounded (as $\gamma > 0$) by %(letting \csumdelta \de \mleft(\sqrt{2}\co\mright)^{\frac\delta\alpha} \Ccoarse^\delta / \mleft(1-s^{-\delta}\mright)$)
\beq\label{eq:firstterm}
\frac{\mleft(1+s^{-\gamma}\mright) \cth \hz^{-\gamma} \mleft(\sqrt{2}\co\mright)^{\frac\gamma\alpha} s^\gamma \Ccoarse^\gamma}{1-s^{-\gamma}} k^{\frac{\gamma\sigma}\alpha} \mesh(k)^\gamma \eps^{-\frac\gamma\alpha}
= \frac{\mleft(1+s^{-\gamma}\mright)\cth \mleft(\sqrt{2}\co\mright)^{\frac\gamma\alpha} s^\gamma}{1-s^{-\gamma}} k^{\frac{\gamma\sigma}\alpha} \eps^{-\frac\gamma\alpha}
\eeq

To bound the sum in the first part of \eqref{eq:complexitymidway}, we must distinguish three cases based on $\gamma - \beta.$


If $\gamma=\beta,$ then the first part of \eqref{eq:complexitymidway} becomes (using \cref{lem:sumbound})
\beq
2 \ct\cth \mleft(1+s^{-\gamma}\mright)k^{\tau}\eps^{-2}\mleft(L+1\mright)^2 \leq 2 \ct\cth \mleft(1+s^{-\gamma}\mright)k^{\tau}\eps^{-2}\mleft(\frac1\alpha \log_s \mleft(\frac{\sqrt{2} \co \Ccoarse^\alpha k^\sigma \mesh(k)^\alpha}\eps\mright)+2\mright)^2,
\label{eq:gammaequal}
\eeq
by \eqref{eq:Ldef}. We wish to simplify \eqref{eq:gammaequal}, so that it is of the form $\mathrm{Constant} \times \text{Terms involving } \eps \text{ and } k.$ To achieve this simplification, we use \cref{ass:epsk}. As $k^\sigma \mesh(k)^\alpha \geq 1$ and $\eps \leq \mleft(\sqrt{2} \co \Ccoarse^{\alpha}\mright)/s^{2\alpha},$ it follows that
\beqs
2 \leq \frac1\alpha \log_s \mleft(\frac{\sqrt{2} \co \Ccoarse^\alpha k^\sigma \mesh(k)^\alpha}\eps\mright),
\eeqs
and thus \eqref{eq:gammaequal} can be bounded by
\beq\label{eq:gammaequalpart1}
8 \ct\cth \mleft(1+s^{-\gamma}\mright)k^{\tau}\eps^{-2}\mleft(\frac1\alpha \log_s \mleft(\frac{\sqrt{2} \co \Ccoarse^\alpha k^\sigma \mesh(k)^\alpha}\eps\mright)\mright)^2.
\eeq
As $k^\sigma \mesh(k)^\alpha \geq 1$ and $\eps \leq 1/\mleft(\sqrt{2}\co\Ccoarse^\alpha\mright),$ we can bound \eqref{eq:gammaequalpart1} by (including a change of base in the logarithm)
\beq\label{eq:gammaequalfinal}
\frac{32 \ct\cth \mleft(1+s^{-\gamma}\mright)}{\alpha^2 \mleft(\log(s)\mright)^2} k^\tau \mleft(\log\mleft(\frac{k^\sigma \mesh(k)^\alpha}\eps\mright)\mright)^2.
\eeq

For simplicity in what follows, we define
\beqs
\csumdelta \de \frac{\mleft(\sqrt{2}\co\mright)^{\frac\delta\alpha}\Ccoarse^{\delta}}{1-s^{-\delta}}.
\eeqs

If $\gamma > \beta$ then by \cref{lem:sumbound} the first term in \eqref{eq:complexitymidway} becomes\optodo{Check if the below is right - it seems to be saying the complexity is independent of the coarse mesh}
\beq
\eps^{-2}2\ct\cth \mleft(1+s^{-\gamma}\mright) k^\tau \hz^{\beta-\gamma}\mleft(\csumgammambetat s^{\frac{\gamma-\beta}2} k^{\frac{\gamma-\beta}2\frac\sigma\alpha} \mesh(k)^{\frac{\gamma-\beta}2} \eps^{-\frac{\gamma-\beta}{2\alpha}}\mright)^2 = \Cgammagtrbeta k^{\tau + \mleft(\gamma-\beta\mright)\frac\sigma\alpha} \eps^{-2-\frac{\gamma-\beta}{\alpha}},\label{eq:gammagtr}
\eeq
where
\beqs
\Cgammagtrbeta \de 2\ct\cth\mleft(1+s^{-\gamma}\mright)\csumgammambetat^2 s^{\gamma-\beta} \Ccoarse^{\beta-\gamma}.
\eeqs
If $\gamma < \beta,$ then analagously the first term in \eqref{eq:complexitymidway} is
\beqs
\Cgammalessbeta k^{\tau + \mleft(\gamma-\beta\mright)\frac\sigma\alpha} \eps^{-2-\frac{\gamma-\beta}{\alpha}},
\eeqs
where
\beqs
\Cgammalessbeta \de \frac{\Cgammagtrbeta}{s^{\gamma-\beta}}.
\eeqs

We now combine \eqref{eq:firstterm}, \eqref{eq:gammaequalfinal}, \eqref{eq:gammagtr}, and \eqref{eq:gammaless} and supress all the constants to obtain the result.
%Removing all the terms that are not of interest from \eqref{eq:gammaequal}, \eqref{eq:gammagtr}, and \eqref{eq:gammaless}, we obtain \eqref{eq:mlmchheq} and \eqref{eq:mlmchhoth}.
\epf


%
\paragraph{The nasty case, where $k^{-\sigma/\alpha} \gtrsim k^{-\coarseexp}.$}

 \bas\label{ass:powersnasty}
 Suppose
 \beqs
\frac{\sigma}{\alpha} \leq \coarseexp.
 \eeqs
 \eas

  \bas[Epsilon sufficiently small]\label{ass:constantsnasty}
 Assume
 \beqs
\eps \leq \co \Ccoarse^{\alpha}.
 \eeqs
 \eas

\bth[MLMC Complexity Theorem]\label{thm:mlmccomp2}
Assume \cref{ass:powersnasty,ass:constantsnasty} hold. Assume $k \geq 1.$ If $L$ is given by
\beq\label{eq:Lcond2}
L = \ceil{\frac1\alpha\log_{s}\mleft(\sqrt{2}\co  \Ccoarse^\alpha k^{\sigma-\coarseexp\alpha} \eps^{-1}\mright)},
\eeq
that is,
\beqs
\hL \leq \mleft(\frac{\eps}{\sqrt{2}\co k^\sigma}\mright)^{\frac1\alpha},
\eeqs
and the number of samples on each computational level is given by
\beqs
\Nl = \ceil{\frac2{\eps^{2}} \mleft(\frac{\Vl}{\Cl}\mright)^{\half}\sum_{j=0}^{L} \mleft(\Vj\Cj\mright)^{\half}},
\eeqs
then computational effort $\CMLhL(\eps)$ required to obtain $\err{\QhatMLhL} \leq \eps$ satisfies the bounds
 
 \begin{numcases}{ \CMLhL(\eps) \lesssim}
 k^{\tau + \rho+\coarseexp\mleft(\gamma - \beta\mright)}\eps^{-2}\mleft(\log_s\mleft(\sqrt{2}\co\Cppw^\alpha k^{\sigma-\coarseexp\alpha} \eps^{-1}\mright)+2\alpha\mright)^2 +  k^{\rho +  \frac{\gamma\sigma}\alpha}\eps^{-\frac\gamma\alpha}
 & if $\beta = \gamma$,\label{eq:mlmchheq2}\\ 
k^{\tau + \rho+\mleft(\gamma-\beta\mright)\frac\sigma\alpha}\eps^{-2+\mleft(\frac{\beta-\gamma}{\alpha}\mright)}
 +  k^{\rho +  \frac{\gamma\sigma}\alpha}\eps^{-\frac\gamma\alpha} & otherwise.\label{eq:mlmchhoth2}
\end{numcases}
 \enth
 \optodo{Need to say why the latter two cases are the same - in one case $\gamma/\alpha$ dominates, and in the other case the other term dominates? (At least in the Cliffe et. al. set up)}

 \bpf[Proof of \cref{thm:mlmccomp}]
 \ednote{This isn't all the details of the proof, but the bits I've skipped over are exactly the same as those in {\cite{ClGiScTe:11}}.}
Do the bias-variance decomposition.  
 We now proceed to choose the parameters $L$ and $\Nl, l = 0,\ldots,L$, however, because the dominant term in this case\optodo{show this} is the \emph{Coarse restriction} we tentatively\ednote{I have no idea at this stage whether this will work} We bound the bias by $\eps^2k^{\sigma-\alpha\coarseexp}$ and the variance by $\eps^2\mleft(1-k^{\sigma-\alpha\coarseexp}\mright).$\optodo{This is all fine by the assumptions}

We first bound the bias, to do this, we only need to choose $L.$ One can show\ednote{As in {\cite{ClGiScTe:11}}} that the bias is equal to $\abs{\EXP{\QhL - Q}}^2.$ Therefore a sufficient condition for the bias to be $\leq \eps^2k^{\sigma-\alpha\coarseexp}$ is (by \cref{ass:a})
\beqs
\co k^\sigma \hL^\alpha \leq \eps k^{\frac{\sigma-\alpha\coarseexp}2},
\eeqs
that is
\beq\label{eq:hLcond2}
\hL \leq \mleft(\frac{\eps}{\sqrt{2}\co k^\sigma}\mright)^{\frac1\alpha}.
\eeq
\ednote{Observe that if $Q$ is the weighted $H^1$ norm, then we assume (see below for details) $\alpha=2$ and $\sigma=3,$ so we require $\hL \lesssim k^{-\frac32}.$ If we take $Q$ to be the $L^2$ norm, and assume $\alpha=2$ and $\sigma=2,$ then we only require $\hL \lesssim k^{-1}.$}

As $\hL = \hz s^{-L},$ it follows from \eqref{eq:hLcond2} that a sufficient condition for the bias to be $\leq \eps^2/2$ is
\beq\label{eq:Lcondpart2}
L = \ceil{\frac1\alpha\log_s\mleft(\sqrt{2}\co k^\sigma \hz^\alpha \eps^{-1}\mright)}.
\eeq
As $\hz = \Ccoarse k^{-\coarseexp},$ we can simplify \eqref{eq:Lcondpart2} to obtain \eqref{eq:Lcond2}.
% \beqs
% L = \ceil{\frac1\alpha\log_s\mleft(\sqrt{2}\co\Ccoarse^\alpha k^{\sigma-\coarseexp\alpha} \eps^{-1}\mright)}.
% \eeqs

We now seek to bound the variance. One can show\ednote{Again, as in \cite{ClGiScTe:11}} the variance $V = \sum_{l=0}^L \Nl^{-1} \Vl,$ and the cost is $\cC = \sum_{l=0}^L \Nl \Cl.$

To find the optimal number of samples per level (the values of $\Nl, l=0,\ldots,L$) we formulate this as an optimisation problem to find $\Nl$ that minimise $\cC$, subject to $V=\eps/2.$ This can be solved using a Lagrange multiplier as in \cite{Gi:15}, and we obtain
\beq\label{eq:Nl2}
\Nl = \ceil{\frac2{\eps^{2}} \mleft(\frac{\Vl}{\Cl}\mright)^{\half}\sum_{j=0}^L \mleft(\Vj\Cj\mright)^{\half}}.
\eeq
\optodo{Check this is correct, should it be divided by the sum?}
We now just need to infer the computational complexity for MLMC with $L$ given by \eqref{eq:Lcond2} and the $\Nl$ given by \eqref{eq:Nl2}.

The computational complexity $\cC$ is given by
\begin{align}
\cC &= \sum_{l=0}^{L} \Cl \Nl\nonumber\\
&\leq \sum_{l=0}^L \Cl \mleft(\frac2{\eps^{2}} \mleft(\frac{\Vl}{\Cl}\mright)^{\half}\sum_{j=0}^L \mleft(\Vj\Cj\mright)^{\half} + 1\mright) \text{ (by \eqref{eq:Nl2})}\nonumber\\
&= 2\eps^{-2}\mleft(\sum_{l=0}^L\mleft(\Vl\Cl\mright)^{\half}\mright)^2 + \sum_{l=0}^L \Cl\nonumber\\
&\leq 2 \ct \cth k^{\tau + \rho}\eps^{-2} \mleft(\sum_{l=0}^L \hl^{\frac{\beta-\gamma}2}\mright)^2 + \cth k^\rho \sum_{l=0}^L \hl^{-\gamma} \text{ (by \cref{ass:b,ass:c})}\nonumber\\
&= 2 \ct\cth k^{\tau + \rho}\eps^{-2}\hz^{\beta-\gamma}\mleft(\sum_{l=0}^L s^{-l\mleft(\frac{\beta-\gamma}2\mright)}\mright)^2 + \cth k^\rho \hz^{-\gamma} \sum_{l=0}^L s^{\gamma l} \text{ (by definition of } \hl\text{ )}\nonumber\\
&=2 \ct\cth \Cppw^{\beta-\gamma}k^{\tau + \rho+\coarseexp\mleft(\gamma - \beta\mright)}\eps^{-2}\mleft(\sum_{l=0}^L s^{l\mleft(\frac{\gamma-\beta}2\mright)}\mright)^2 + \cth\Cppw^{-\gamma} k^{\rho + \gamma\coarseexp}  \sum_{l=0}^L s^{\gamma l} \text{ (by definition of } \hz\text{ )}\nonumber\\
&\leq2\ct\cth \Cppw^{\beta-\gamma}k^{\tau + \rho+\coarseexp\mleft(\gamma - \beta\mright)}\eps^{-2}\mleft(\sum_{l=0}^L s^{l\mleft(\frac{\gamma-\beta}2\mright)}\mright)^2 +  \frac{\mleft(\sqrt{2}\co\mright)^{\frac\gamma\alpha}\cth s^{\gamma}}{1-s^{-\gamma}}k^{\rho +  \frac{\gamma\sigma}\alpha}\eps^{-\frac\gamma\alpha} \text{ (since }\gamma>0,\text{ by \cref{lem:sumbound})}.\label{eq:complexitymidway2}
\end{align}
To bound the sum in the first part of \eqref{eq:complexitymidway2}, we must distinguish three cases based on $\gamma - \beta.$

If $\gamma=\beta,$ then \eqref{eq:complexitymidway2} becomes (using \cref{lem:sumbound})
\begin{multline}
2 \ct\cth \Cppw^{\beta-\gamma}k^{\tau + \rho+\coarseexp\mleft(\gamma - \beta\mright)}\eps^{-2}\mleft(L+1\mright)^2 +  \frac{\mleft(\sqrt{2}\co\mright)^{\frac\gamma\alpha}\cth s^{\gamma}}{1-s^{-\gamma}}k^{\rho +  \frac{\gamma\sigma}\alpha}\eps^{-\frac\gamma\alpha}\\
\leq
2\ct\cth \Cppw^{\beta-\gamma}k^{\tau + \rho+\coarseexp\mleft(\gamma - \beta\mright)}\eps^{-2}\mleft(\frac1\alpha\log_s\mleft(\sqrt{2}\co\Cppw^\alpha k^{\sigma-\coarseexp\alpha} \eps^{-1}\mright)+2\mright)^2 +  \frac{\mleft(\sqrt{2}\co\mright)^{\frac\gamma\alpha}\cth s^{\gamma}}{1-s^{-\gamma}}k^{\rho +  \frac{\gamma\sigma}\alpha}\eps^{-\frac\gamma\alpha}\label{eq:gammaequal2}
\end{multline}
by \eqref{eq:Lcond2}.

If $\gamma > \beta$ then by \cref{lem:sumbound} \eqref{eq:complexitymidway2} becomes
\beq
2 \ct\cth \Cppw^{\beta-\gamma}
\frac{\mleft(\sqrt{2}\co\mright)^{\mleft(\frac{\gamma-\beta}{\alpha}\mright)}\Cppw^{\mleft(\gamma-\beta\mright)}s^{\gamma-\beta}}{\mleft(1-s^{\mleft(\frac{\beta-\gamma}2\mright)}\mright)^{2}}k^{\tau + \rho+\mleft(\gamma-\beta\mright)\frac\sigma\alpha}\eps^{-2+\mleft(\frac{\beta-\gamma}{\alpha}\mright)}
 +  \frac{\mleft(\sqrt{2}\co\mright)^{\frac\gamma\alpha}\cth s^{\gamma}}{1-s^{-\gamma}}k^{\rho +  \frac{\gamma\sigma}\alpha}\eps^{-\frac\gamma\alpha}.\label{eq:gammagtr2}
\eeq
If $\gamma < \beta,$ then by \cref{lem:sumbound} \eqref{eq:complexitymidway2} becomes
\beq
2 \ct\cth \Cppw^{\beta-\gamma}
\frac{\mleft(\sqrt{2}\co\mright)^{\mleft(\frac{\gamma-\beta}{\alpha}\mright)}\Cppw^{\mleft(\gamma-\beta\mright)}}{\mleft(1-s^{\mleft(\frac{\beta-\gamma}2\mright)}\mright)^{2}}k^{\tau + \rho+\mleft(\gamma-\beta\mright)\frac\sigma\alpha}\eps^{-2+\mleft(\frac{\beta-\gamma}{\alpha}\mright)}
 +  \frac{\mleft(\sqrt{2}\co\mright)^{\frac\gamma\alpha}\cth s^{\gamma}}{1-s^{-\gamma}}k^{\rho +  \frac{\gamma\sigma}\alpha}\eps^{-\frac\gamma\alpha},\label{eq:gammaless2}
\eeq
the only difference from \eqref{eq:gammagtr2} being the loss of the $s^{\gamma-\beta}$ term.

Removing all the terms that are not of interest from \eqref{eq:gammaequal2}, \eqref{eq:gammagtr2}, and \eqref{eq:gammaless2}, we obtain \eqref{eq:mlmchheq2} and \eqref{eq:mlmchhoth2}.
\epf





%\section{MLMC for the Helmholtz equation}

%\subsection{Setup}\optodo{Remove $A$ from everything}
Let either (i) $\Dm \subset \RRd,$ $d=2,3,$ be a bounded Lipschitz open set such that $\bzero \in \Dm$ and the open complement $\Dp\de \RR^d\setminus \overline{\Dm}$ is connected or (ii) $\Dm = \emptyset.$ Let $\GD = \partial \Dm.$ 
 Define $\GI := \partial \Dtilde$ and $D \de \Dp \cap \Dtilde$ (see \cref{fig:domain}). Let $\gamma$ denote the trace operator from $D$ to $\partial D = \GD \cup \GI$ and define $\HozDD \de \set{v \in \HoD \st \gamma v = 0 \ton \GD}.$ 
 
Let $\trGI: H^{1/2}(\Gamma_R) \rightarrow H^{-1/2}(\Gamma_R)$ be the Dirichlet-to-Neumann map for the deterministic equation $\Delta u+k^2 u=0$ posed in the exterior of $\Dtilde$ with the Sommerfeld radiation condition 

\beq
\frac{\partial u}{\partial r}(\bx) - \ii ku(\bx) = o\mleft(\frac1{r^{(d-1)/2}}\mright) \text{ as } r\de\abs{\bx}\rightarrow \infty, \text{ uniformly in } \frac{\bx}{\abs{\bx}}.
\eeq
Let $\IPGI{\cdot}{\cdot}$ be the duality pairing on $\GI$ between $\HmhGI$ and $\HhGI$ and write $\dd\Leb$ for Lebesgue measure.

 Where the range of functions is $\CC$ we suppress the second argument in a function space, e.g.~we write $\LiDp$ for $\LiDpCC.$ We write $\Do \compcont \Dt$ if $\Do$ is a compact subset of the open set $\Dt.$ Let $\OFP$ be a complete probability space. Let
\bit
\item $f:\Omega\rightarrow\LtDp$ be such that $\supp f \compcont \Dtilde$ almost surely
\item $n:\Omega\rightarrow \LiDpRR$ be such that $\supp(1-n) \compcont \Dtilde$ almost surely and there exist $\nmin,\nmax :\Omega rightarrow \RR$ such that
\beqs
0 < \nmin(\omega)\leq n(\omega)(\bx) \leq \nmax(\omega)
\eeqs
for almost every $\bx \in \Dp$ almost surely, and
\eit

Define the sesquilinear form $a(\omega)$ on $\HozDD \times \HozDD$ by
\beq
\mleft[a(\omega)\mright]\mleft(\vo,\vt\mright)\de\int_{D_R}\Big( \mleft(\grad \vo\mright)\cdot \grad \vtb 
 - k^2 n(\omega)\, \vo\,\vtb \Big)\dd\Leb- \big\langle T_R \gamma \vo,\gamma \vt\big\rangle_{\Gamma_R},
 \eeq
 and the antilinear functional $L(\omega)$ on $\HozDD$ by
\beq
\mleft[L(\omega)\mright](\vt)\de \int_{D_R} f(\omega)\, \vtb\,\dd\Leb.
\eeq

\optodo{Problem}
Find $u\in\LtOHozDD$ such that
\beqs
\mleft[a(\omega)\mright]\mleft(u(\omega),v\mright) = \mleft[L(\omega)\mright](v) \tfa v \in \HozDD \text{ almost surely.}
\eeqs

\bcon[Regularity and stochastic regularity of $f,$ $A,$ and $n$]\label{con:reg}
The map $f \in \LtOLtD$  and the map $n \in \LiOLiDRR.$
\econ

\bcon[{$k$-independent nontrapping conditions on (random) $A$ and $n$}]\label{con:nt}
The map $n:\Omega\rightarrow \WoiDRR,$ and there exists $\mu:\Omega\rightarrow \RR$  such that $\mu(\omega) > 0$ almost surely and
\beq
2\nz(\omega,\bx) + \bx\cdot\grad \nz(\omega\bx) \geq \mu(\omega)
\eeq
for almost every $\bx \in D,$ almost surely.
\econ

\bde[Star-shaped]
The set $D \subseteq \RRd$ is \defn{star-shaped with respect to} the point $\bxz$ if for any $\bx \in D$ the line segment $\mleft[\bxz,\bx\mright] \subseteq D.$
\ede

\bas\label{ass:nmaxmu}
The functions $\mu \in \LtO, 1/\mu \in \LfO, \nmax \in \LfO,$ and $1/\nmax \in \LtO.$
\eas

\bth[]\optodo{Double check generic requirements}Let $\Dm$ be star-shaped with respect to the origin. Let $\LD \de \min_{\bx \in \GD} \abs{\bx},$ $\LI \de\max_{\bx \in \LI} \abs{\bx}$, $\aD\LD$ be the radius of the ball with resepct to which $\Dm$ is star-shaped, and let $\aI\LI$ be the radius of the ball with respect to which $\Dtilde$ is star-shaped. Under \cref{ass:nmaxmu} \optodo{Conditions}:
The solution $u \in \LtOHozDD$ of \optodo{problem} exists, is unique, and, given $\kz > 0,$ satisfies the bound
\begin{multline*}
  \NLtOLtD{\grad u}^2 + k^2\NLtOLtD{u}^2 + \frac{\aD\LD}{2} \NLtOLtGD{\dn u}^2\\
  \leq \NLoO{\Cth} \NLtOLtD{f}^2 + \NLoO{\Cthtilde} \NLtOLtGI{\gI}^2\\
  + \NLoO{\Cf}\NLtOLtGD{\gradGD \gD}^2 + k^2 \NLoO{\Cfi} \NLtOLtGD{\gD}^2
\end{multline*}\optodo{Lesssim or leq with a factor of 2 or something?}
for all $k\geq\kz$, where $\Cth,\Cthtilde,\Cf\Cfi:\Omega\rightarrow\RR$ are given by
\beqs
\Cth \de 2\max\set{1,\frac1\mu}\mleft(\frac{2}{\mu} \mleft(1 + \frac32 \nmax \mright)^2 \mleft(\LI^2 + \mleft(\hetbeta + \frac{d-2}{2\kz}\mright)^2\mright) + \frac{\mu}{2\kz^2 \nmax}\mright),
\eeqs
\beqs
\Cthtilde \de  2\max\set{1,\frac1\mu}\mleft(\mleft(1+\frac32 \nmax \mright)\Cotilde + \frac{2\mu^2}{\LI\kz^2}\mright),
\eeqs
\beqs
\Cf \de 2 \max\set{1,\frac1\mu}\mleft(1 + \frac32 \nmax\mright) \LD \mleft(1+\frac4{\aD}\mright),\tand
\eeqs
\beqs
\Cfi \de 2 \max\set{1,\frac1\mu}\mleft(\mleft(1+\frac32 \nmax\mright)\frac4{\aD\LD}\mleft(\hetbeta + \frac{d-2}{2\kz}\mright)^2 + \frac{2\mu^2}{\aD\LD}\mright),
\eeqs
where
\beqs
\Cotilde \de 2\mleft(2\mleft(1+\frac2\aI\mright) + \frac{\hetbeta}{\LI} + \frac{\mleft(d-1\mright)^2}{4}\mright)\LI
\eeqs
and
\beqs
\hetbeta \de \LI\mleft(2 + \frac1{\mleft(k\LI\mright)^2} + 2\mleft(1+\frac2{\aI}\mright)\mright)
\eeqs
\optodo{When I've copied these, mult all by the $1\mu$ bit, and $\nmaxGI$}
\enth
\optodo{Sketch proof/say you can apply the arguments in stochastic - check measurability etc.}

Let $m \in \NN$ and define
\beq\label{eq:nseriesml}
n(\omega,\bx) = \nz(\bx) + \sum_{j=1}^m\Yj(\omega) \psij(\bx)
\eeq
where:
\bit
\item $\supp\mleft(1-\nz\mright) \compcont \Dtilde,$
\item $\Yj \sim \Unif(-1/2,1/2)$ i.i.d.,
\item $\psij \in \WoiDRR$ with $\supp \psij \compcont \Dtilde$ for all $j = 1,\ldots,m,$ and
\eit
\optodo{Double check that all the generalisations work}

\ble[Series expansion of $n$ satisfies \cref{con:hh-hetero}]
Let $\mu > 0$ and $\delta \in \mleft(0,1\mright).$ If $\nz \in \NTn{\mu}$ and
\beq\label{eq:nseriescond}
\sum_{j=1}^m\NLiDRR{\psij(\bx) + \bx\cdot\grad\psij(\bx)} \leq 2\delta\mu,
\eeq
then $n \in \NTn{(1-\delta)\mu}.$\optodo{Maybe change this so that we get path dependent $\mu.$}
\ele

\subsection{$H^2$ bounds}

\bas[Assumptions needed for Chaumont bound]\label{ass:chaumont}
$\GD$ and $\GI$ are piecewise $\Coo$\ednote{\cite{ChNiTo:18} only proves the a priori bound for $\gD=0$ - is this an issue/what do we do?}, and for all $\bx \in \dD,$ there exists $\epsbx > 0$ such that either:
\ben
\item $\partial\mleft(\Ball{\epsbx}{\bx} \cap D\mright)$ is convex, or
\item $\Ball{\epsbx}{\bx} \cap \dD$ is $\Coo.$
  \een
    If $d=3,$ $ \bx \in \GI,$ and $\Ball{\epsbx}{\bx}$ is only convex, then $\Ball{\epsbx}{\bx} \cap \GI$ is piecewise flat.
    \eas

    \bth[Bound on related PDE {\cite[Theorem 5.1]{ChNiTo:18}}]
Under \cref{ass:chaumont} if $u \in \HozDD$ solves \eqref{eq:relpde} with $\gD=0,$ then $u \in \HtD$ and $\CHtell \sim 1.$
    \enth
%

\section{The lemma in generality}

\ble\label{lem:sumboundnew}
If $L$ is given by
\beq\label{eq:Ldefgen}
L = \ceil{\Lconst\log_{s}\mleft( \func \eps^{-1}\mright)},
\eeq
for some $\func > 0,$ then, for $s>1$ and $\delta \in \RR,$ we have the bound
\beq\label{eq:sumboundgen}
\sum_{l=0}^{L} s^{\delta l} \leq
\begin{cases}
L+1 & \tif \delta = 0,\\
\frac{s^{\delta}}{1-s^{-\delta}}\func^{\delta\Lconst}\eps^{-\delta\Lconst} &\tif \delta >0\\
\frac{1}{1-s^{-\delta}}\func^{\delta\Lconst}\eps^{-\delta\Lconst}&\tif \delta < 0
\end{cases}
\eeq
\beq\label{eq:sumboundLmo}
\sum_{l=0}^{L} s^{\delta l} \leq
\begin{cases}
L & \tif \delta = 0,\\
\frac{s^{\delta}}{1-s^{-\delta}}\func^{\delta\Lconst}\eps^{-\delta\Lconst} &\tif \delta >0\\
\frac{1}{1-s^{-\delta}}\func^{\delta\Lconst}\eps^{-\delta\Lconst}&\tif \delta < 0
\end{cases}
\eeq
\optodo{Tidy}
\ele

\bpf[Proof of \cref{lem:sumboundnew}]
The proof follows that in \cite{ClGiScTe:11}. We first observe that, since $L$ is given by \eqref{eq:Ldefgen}, it follows that
\beq\label{eq:Lboundsgen}
\Lconst\log_s\mleft(\func \eps^{-1}\mright) \leq L < \Lconst\log_s\mleft(\func \eps^{-1}\mright) + 1.
\eeq
Rearranging \eqref{eq:Lboundsgen}, we obtain the bounds
\beq\label{eq:saLbounds}
\mleft( \func\eps^{-1}\mright)^{\alpha \Lconst} \leq s^{\alpha L} < \mleft( \func\eps^{-1}\mright)^{\alpha \Lconst}s^\alpha.
\eeq
If $\delta > 0,$ then we use the right-hand bound in \eqref{eq:saLbounds} to obtain
\beq\label{eq:sdLpos}
s^{\delta L} < \func^{\delta\Lconst}\eps^{-\delta\Lconst}s^{\delta},
\eeq
and if $\delta < 0,$ we use the left-hand bound in \eqref{eq:saLbounds} to obtain
\beq\label{eq:sdLneg}
s^{\delta L} \leq \func^{\delta\Lconst}\eps^{-\delta\Lconst}.
\eeq
We now observe that, for $\delta \neq 0,$
\begin{align}
\sum_{l=0}^L s^{\delta l} &= \frac{s^{\delta\mleft(L+1\mright)} -1}{s^{\delta}-1}\nonumber\\
&= \frac{s^{\delta L} - s^{-\delta}}{1-s^{-\delta}}\nonumber\\
&\leq \frac{s^{\delta L}}{1-s^{-\delta}},\label{eq:ssumbound}
\end{align}
since $s^{-\delta} > 0,$ as $s >0.$ Combining \eqref{eq:ssumbound} with \eqref{eq:sdLpos} and \eqref{eq:sdLneg}, we obtain \eqref{eq:sumboundgen} in the cases $\delta \neq 0.$ The case $\delta=0$ is straightforward.



\paragraph{For sum up to $L-1$}
The proof follows that in \cite{ClGiScTe:11}. We first observe that, since $L$ is given by \eqref{eq:Ldefgen}, it follows that
\beq\label{eq:Lmobounds}
\Lconst\log_s\mleft(\func \eps^{-1}\mright)-1 \leq L-1 < \Lconst\log_s\mleft(\func \eps^{-1}\mright).
\eeq
Rearranging \eqref{eq:Lmobounds}, we obtain the bounds
\beq\label{eq:saLmobounds}
\mleft( \func\eps^{-1}\mright)^{\alpha \Lconst}s^{-\alpha} \leq s^{\alpha (L-1)} < \mleft( \func\eps^{-1}\mright)^{\alpha \Lconst}.
\eeq
If $\delta > 0,$ then we use the right-hand bound in \eqref{eq:saLmobounds} to obtain
\beq\label{eq:sdLmopos}
s^{\delta (L-1)} < \func^{\delta\Lconst}\eps^{-\delta\Lconst}
\eeq
and if $\delta < 0,$ we use the left-hand bound in \eqref{eq:saLmobounds} to obtain
\beq\label{eq:sdLmoneg}
s^{\delta (L-1)} \leq \func^{\delta\Lconst}\eps^{-\delta\Lconst}s^{-\delta}.
\eeq
We now observe that, for $\delta \neq 0,$
\begin{align}
\sum_{l=0}^{L-1} s^{\delta l} &= \frac{s^{\delta\mleft(L\mright)} -1}{s^{\delta}-1}\nonumber\\
&= \frac{s^{\delta (L-1)} - s^{-\delta}}{1-s^{-\delta}}\nonumber\\
&\leq \frac{s^{\delta (L-1)}}{1-s^{-\delta}},\label{eq:ssumboundLmo}
\end{align}
since $s^{-\delta} > 0,$ as $s >0.$ Combining \eqref{eq:ssumboundLmo} with \eqref{eq:sdLmopos} and \eqref{eq:sdLmoneg}, we obtain \eqref{eq:sumboundLmo} in the cases $\delta \neq 0.$ The case $\delta=0$ is straightforward.
\epf


%\optodo{REDO}
We now apply \cref{thm:mlmccomp} to the (heterogeneous) Helmholtz equation. We first need to ascertain what the quantities $\alpha, \beta, \gamma, \sigma, \tau$ and $\rho$ are.Throughout, $D$ will be the spatial domain.

%\bre[Relationship between $\Ml$ and $\hl$]
%Recall that $\Ml$ denotes the number of DoFs associated with the discretisation on level $l.$ Therefore if $\hl$ denotes the mesh size on level $l,$ we have 
%\beqs
%\Ml = \CMh^d \hl^{-d},
%\eeqs
%for some constant $\CMh$ (i.e., halving the mesh size increases the number of DoFs by a factor $2^d)$. It follows that 
%\beqs
%\hl = \CMh \Ml^{-1/d},
%\eeqs
%and thus $\hl = s^{1/d} \hlmo.$ I.e., if we perform uniform refinement, in 2-D, $s=4.$ For uniform refinement in 3-D, $s=8.$ We'll use the relationship between $\Ml$ and $\hl$ a lot, as it's more intutive to work with $\hl$ than $\Ml$. (Alternatively, we could recast \cref{thm:mlmccomp} in terms of $\hl.$)
%\ere
%
%Throughout, we'll use the following assumption:
%
%\bas[At least fixed points-per-wavelength, and error term]\label{ass:ppwfem}
%There exist constants $\Cppw$ and $\CFEM > 0$, independent of all the parameters of interest, such that
%\beqs
%\hl \leq \frac{\Cppw}k
%\eeqs
%and
%\beq\label{eq:femerror}
%\NW{u-\uhl} \leq \CFEM \hl^2 k^3
%\eeq
%for all $l \geq 0,$ where $u$ is the true solution of the Helmholtz equation, and $\uhl$ is the solution of the piecewise-linear continuous finite-element approximation of the Helmholtz equation on level $l.$
%\eas
%
%\bre[How to show \cref{ass:a,ass:b}]
%For the \emph{heterogeneous} Helmholtz equation, showing \eqref{eq:femerror} is currently an open question, although we might be able to do it using the argument in \cite{ChNi:18}. In general, $\CFEM$ will depend on any forcing functions in the Helmholtz equation, and on the coefficients, but we don't worry about that here.
%
%Of course, because we'll be dealing with the Helmholtz equation with random coefficients, $\CFEM$ will, in general, be a random variable. We won't worry about its precise properties for now, we only assume that $\CFEM$ is `nice enough' that we can take expectations of all the quantities we need to. (Nontrapping almost-surely would, I think, give a `nice enough' $\CFEM$.)
%\ere

We now show \cref{ass:a,ass:b,ass:c} hold, and we will calculate the values of $\alpha, \beta, \gamma, \sigma, \tau,$ and $\rho$.

We assume throughout that the FEM error for the weighted $H^1$ norm decays like $h^2k^3,$ and the the FEM error for the $L^2$ norm decays like $h^2k^2.$

%\subsection{Verifying the assumptions of MLMC}

%Throughout, we will take
%\beqs
%\hz = \frac{\Cppw}k,
%\eeqs
%i.e., our coarsest mesh corresponds to a fixed number of points per wavelength. This corresponds to
%\beqs
%\Mz = \frac{\CMh^d}{\Cppw^d} k^d,
%\eeqs
%that is, $\rho = d$ in \cref{cor:mlmccomphh}.

For now, we consider two different quantities of interest $Q$: $Q = \NW{u} $ and $Q = \NLtD{u}$.% and $Q = \NLtGI{\uinfty},$ where $\uinfty$ is an approximation on the impedance boundary $\GI$ of the far-field pattern. To elaborate on the construction of $\uinfty,$ we have that the scattered field $\uS$ satisfies
%\beqs
%\uS(\bx) = \frac{e^{ikr}}{r}\mleft(\uinfty + O\mleft(\frac1r\mright)\mright),
%\eeqs
%for the \emph{true} far-field pattern $\uinfty$ hence if we assume $r \approx 1$ on the impedance boundary, we form an approximation of the far-field pattern on the impedance boundary by taking
%\beqs
%\uinfty \approx e^{-ik} \uS = e^{-ik}\mleft(u-\uI\mright).
%\eeqs
%We'll see that $\uI$ doesn't come into the calculations we do below.

\subsection{Verifying \cref{ass:a,ass:b} for $Q = \NW{u}$}\label{sec:abweight}
To verify \cref{ass:a}, we have
\begin{align*}
\abs{\EXP{\Qhl(\omega) - Q}} &\leq \EXP{\abs{\NW{\uhl} - \NW{u}}}\\
&\leq \EXP{\NW{\uhl - u}}\\
&\leq \CFEM \hl^2 k^3
\end{align*}
so $\sigma = 3$ and $\alpha=2.$

To verify \cref{ass:b}, we have 
\begin{align*}
\VAR{\Yl} &= \EXP{\Yl^2} - \EXP{\Yl}^2\\
&\leq \EXP{\Yl^2}\\
&= \EXP{\abs{\Qhl-\Qhlmo}^2}\\
&= \EXP{\abs{\NW{\uhl}-\NW{\uhlmo}}^2}\\
&\leq \EXP{\NW{\uhl-\uhlmo}^2}\\
&\leq 2\EXP{\NW{\uhl-u}^2 + \NW{\uhlmo-u}^2}\\
&\leq 2\CFEM^2 k^6 \mleft(\hl^4 + \hlmo^4\mright)\\
&= 2\CFEM^2 k^6 \mleft(\hl^4 + s^{4d}\hl^4\mright)\\
&= 2\CFEM^2\mleft(1+s^{4d}\mright) k^6 \hl^4
\end{align*}
so $\tau = 6$ and $\beta = 4.$

\subsection{Verifying \cref{ass:a,ass:b} for $Q = \NLtD{u}$}\label{sec:abltwo}

This proceeds similarly to the calculations for $\NW{u},$ except we use the error estimate
\beqs
\NLtD{\uhl -u} \leq \mleft(\hl k\mright)^2.
\eeqs
Proceeding as above, we find $\sigma = 2,$ $\alpha = 2,$ $\tau = 4$, and $\beta = 4.$

%\subsection{Verifying Assumptions \ref{ass:a} and \ref{ass:b} for $Q = \NLtGI{\uinfty}$}cref!!!
%
%To come.

\subsection{Verifying \cref{ass:c} for the Helmholtz equation}

\Cref{ass:c} is asking what the cost is of solving one realisation of the Helmholtz equation on a mesh with size $\hl.$

My knowledge here is currently not good enough to give a complete answer, so in the analysis that follows, I'll look at two situations:

\ben[(i)]
\item\label{it:solverlu} $\cth$ independent of all parameters of interest, and $\gamma = 1.5d,$ and
\item\label{it:solverlin} $\cth$ independent of all parameters of interest, and $\gamma= d.$

Situation \ref{it:solverlu} is the case where for each realisation we use a sparse direct solver (I think I've got this right, although my knowledge of solver complexities isn't very deep - this is taken from \cite[Section 4]{ClGiScTe:11}).
Situation \ref{it:solverlin} is the (theoretical) case where for each realisation we use a $\cO(\Ml)$ solver for the Helmholtz equation (discretised with mesh size $\hl$)---i.e., we have an `$\cO(N)$' solver for the Helmholtz equation. This is included so that we can see what the `best case' is.
\een

\subsection{MLMC complexity for $Q=\NW{u}$}

We have $\rho=0,$ $\sigma = 3, \alpha = 2, \tau = 6,$ and $\beta = 4$.

\subsubsection{In 2-D}

\paragraph{Direct solver}

We have $\gamma = 3$. We check that  $\beta \neq \gamma.$ Therefore we are in case \eqref{eq:mlmchhoth}, and the computational complexity satisfies
\beqs
\CMLhL(\eps) \lesssim k^{4.5} \eps^{-1.5}.
\eeqs

\paragraph{`Ideal' solver}

We have $\gamma = 2$. We check that  $\beta \neq \gamma.$ Therefore we are in case \eqref{eq:mlmchhoth}, and the computational complexity satisfies
\beqs
\CMLhL(\eps) \lesssim k^3 \eps^{-1}.
\eeqs

\subsubsection{In 3-D}

\paragraph{Direct solver}

We have $\gamma = 4.5$. We check that  $\beta \neq \gamma.$ Therefore we are in case \eqref{eq:mlmchhoth}, and the computational complexity satisfies
\beqs
\CMLhL(\eps) \lesssim k^{6.75} \eps^{-2.25}.
\eeqs

\paragraph{`Ideal' solver}

We have $\gamma = 3$. We check that  $\beta \neq \gamma.$ Therefore we are in case \eqref{eq:mlmchhoth}, and the computational complexity satisfies
\beqs
\CMLhL(\eps) \lesssim k^{4.5} \eps^{-1.5}.
\eeqs

\subsection{MLMC complexity for $Q=\NLtD{u}$}

We have $\rho=0,$ $\sigma = 2, \alpha = 2, \tau = 4,$ and $\beta = 4$.

\subsubsection{In 2-D}

\paragraph{Direct solver}

We have $\gamma = 3$. We check that  $\beta \neq \gamma.$ Therefore we are in case \eqref{eq:mlmchhoth}, and the computational complexity satisfies
\beqs
\CMLhL(\eps) \lesssim k^3 \eps^{-1.5}.
\eeqs

\paragraph{`Ideal' solver}

We have $\gamma = 2$. We check that  $\beta \neq \gamma.$ Therefore we are in case \eqref{eq:mlmchhoth}, and the computational complexity satisfies
\beqs
\CMLhL(\eps) \lesssim k^2 \eps^{-1}.
\eeqs

\subsubsection{In 3-D}

\paragraph{Direct solver}

We have $\gamma = 4.5$. We check that  $\beta \neq \gamma.$ Therefore we are in case \eqref{eq:mlmchhoth}, and the computational complexity satisfies
\beqs
\CMLhL(\eps) \lesssim k^{4.5} \eps^{-2.25}.
\eeqs

\paragraph{`Ideal' solver}

We have $\gamma = 3$. We check that  $\beta \neq \gamma.$ Therefore we are in case \eqref{eq:mlmchhoth}, and the computational complexity satisfies
\beqs
\CMLhL(\eps) \lesssim k^3 \eps^{-1.5}.
\eeqs
%
%\section{Monte Carlo for the Helmholtz equation}
%
%

Here we calculate the complexity for `standard' Monte Carlo applied to the Helmholtz equation to achieve a RMSE $\leq \eps.$ This will enable us to see whether MLMC is `better' or not.

We let $\QhatMC$ denote the Monte Carlo estimator of $Q$, that is
\beqs
\QhatMC = \frac1{\NMC} \sum_{i=1}^{\NMC} \QMLMC,
\eeqs
where $\LMC$ is a level to be chosen, and $\NMC$ is the number of Monte Carlo samples, to be chosen.

In the context of SPDEs, $\MLMC$ represents the number of DoFs in our discretisation (observe that now all of our samples are taken at the same mesh size). We therefore need to choose the level $\LMC$ and the number of samples $\NMC$ to ensure the RMSE is $\leq \eps.$

One can repeat the computation in \cite[Section 2.1]{ClGiScTe:11} (i.e., the bias-variance decomposition in the proof of \cref{thm:mlmccomp}) to show that sufficient conditions for RMSE $\leq \eps$ are

\beq\label{eq:mch}
\hLMC = \mleft(\frac{\eps}{\sqrt{2}\co k^\sigma}\mright)^{\frac1\alpha}
\eeq
(c.f. \eqref{eq:hLcond} and
\beqs
\NMC = 2 \V \eps^{-2},
\eeqs
where $\V = \VAR{\Qhl,},$ is assumed independent of $\hl.$\footnote{This is assumed in \cite[Section 2.1]{ClGiScTe:11}. For the Helmholtz equation, if the QoIs are norms, as studied above, if we have a mesh-independent bound on the finite-element solution, then this assumption is satisfied-ish (the variance is bounded independently of $h$). Also see notes from 3/9/18.}

With these choices, and using Assumption \eqref{ass:c}  the computational complexity to achieve a RMSE $\leq \eps$ satisfies the bound
\beq\label{eq:mchhcomp1}
\CMC(\eps) \lesssim k^{\rho + \frac{\sigma\gamma}{\alpha}}\eps^{-\mleft(2+ \frac\gamma\alpha\mright)},
\eeq
as in \cite[Equation before Section 2.2]{ClGiScTe:11}.

\subsection{MC Complexity for $Q=\NW{u}$}

We have $\sigma = 3$ and $\alpha = 2.$

\subsubsection{In 2-D}

\paragraph{Direct solver}

We have $\rho=0$ and $\gamma = 3$. Therefore  the computational complexity satisfies
\beqs
\CMC(\eps) \lesssim k^{4.5} \eps^{-3.5}.
\eeqs

\paragraph{`Ideal' solver}

We have $\rho=0$ and  $\gamma = 2$. Therefore  the computational complexity satisfies
\beqs
\CMC(\eps) \lesssim k^3 \eps^{-3}.
\eeqs

\subsubsection{In 3-D}

\paragraph{Direct solver}

We have $\rho=0$ and $\gamma = 4.5$. Therefore  the computational complexity satisfies
\beqs
\CMC(\eps) \lesssim k^{6.75} \eps^{-4.25}.
\eeqs

\paragraph{`Ideal' solver}

We have $\rho=0$ and  $\gamma = 3$. Therefore  the computational complexity satisfies
\beqs
\CMC(\eps) \lesssim k^{4.5} \eps^{-3.5}.
\eeqs

\subsection{MC Complexity for $Q=\NLtD{u}$}

We have $\sigma = 2$ and $\alpha = 2.$

\subsubsection{In 2-D}

\paragraph{Direct solver}

We have $\gamma = 3$. Therefore  the computational complexity satisfies
\beqs
\CMC(\eps) \lesssim k^3 \eps^{-3.5}.
\eeqs

\paragraph{`Ideal' solver}

We have $\gamma = 2$. Therefore  the computational complexity satisfies
\beqs
\CMC(\eps) \lesssim k^2 \eps^{-3}.
\eeqs

\subsubsection{In 3-D}

\paragraph{Direct solver}

We have $\gamma = 4.5$. Therefore  the computational complexity satisfies
\beqs
\CMC(\eps) \lesssim k^{4.5} \eps^{-4.25}.
\eeqs

\paragraph{`Ideal' solver}

We have $\gamma = 3$. Therefore  the computational complexity satisfies
\beqs
\CMC(\eps) \lesssim k^3 \eps^{-3.5}.
\eeqs
%
%\section{Comparison of MLMC and MC for Helmholtz}

\appendix

\section{Proof of the existence of a lifting operator} % THIS IS IMPORTANT
The main ingredient of our proof is the following Theorem (slightly simplified from the original)
\bth[\cite{Ma:87}]\label{thm:marschall}
Let $U$ be a bounded or unbounded Lipschitz domain. Suppose that $1<p<\infty$ and $s > 1- 1/p'$, where $p'$ is the conjugate exponent to $p,$ i.e., $1/p + 1/p' = 1.$\ednote{I presume this is the definition of $p'$, \cite{Ma:87} doesn't actually define it} Suppose further that $D$ is a $C^{k,\lambda}$ set, for $k+\lambda > 1-1/p.$ Then the trace map
\beqs
\gamma : \WspU \rightarrow \WsmoppdU
\eeqs
is a surjection\ednote{This relies on the construction of the Besov space on the boundary in \cite{Ma:87} being the same as the Sobolev space. I'm really not sure this is the case, can someone help me please?}. In the case $s- 1/p$ is not an integer, it has a bounded linear right inverse.
\enth

\ble[Existence of $\ud$]\label{lem:ud}
Let $\gD \in \HthtGD.$ Then there exists $\ud \in \HtD$ such that $\trGD \ud = \gD$ in $\HthtGD.$
\ele
\bpf[Proof of \cref{lem:ud}]
The proof is immediate from \cref{thm:marschall}, as we can construct  $\udtilde \in \WspRRdsmDm$ such that $\trGD \udtilde = \gD,$ and then let $\ud = \udtilde\restrict{D}.$
\epf\label{app:ud}

%We now want to compare whether Multi-level Monte Carlo performs better than `standard' Monte Carlo. The results of the two previous sections (i.e., the computational complexity required to obtain a RMSE $\leq \eps$) are summarised for the different cases we considered in the following table:

\begin{tabular}{c|c|c}
Scenario (Type of norm in QoI, dimension, direct or `ideal' solver) & MLMC & MC \\
\hline
&&\\
weighted, 2d, direct & $k^{4.5} \eps^{-1.5}$ & $k^{4.5} \eps^{-3.5}$\\
weighted, 2d, ideal & $k^3 \eps^{-1}$ & $k^3 \eps^{-3}$\\
weighted, 3d, direct & $k^{6.75} \eps^{-2.25}$& $k^{6.75} \eps^{-4.25}$ \\
weighted, 3d, ideal & $k^{4.5} \eps^{-1.5}$& $k^{4.5} \eps^{-3.5}$ \\
$L^2$, 2d, direct & $k^3 \eps^{-1.5}$& $k^3 \eps^{-3.5}$ \\
$L^2$, 2d, ideal & $k^2 \eps^{-1}$& $k^2 \eps^{-3}$\\
$L^2$, 3d, direct & $k^{4.5} \eps^{-2.25}$& $k^{4.5} \eps^{-4.25}$ \\
$L^2$, 3d, ideal & $k^3 \eps^{-1.5}$ & $k^3 \eps^{-3.5}$ \\
\end{tabular}

To summarise, we see that in terms of the dependence on $\eps,$ MLMC is always better (which is no surprise, this is well-known). However, in terms of the dependence on $k,$ MLMC is always the same as MC.

As of yet, I don't have an intution for why this is true - given the MLMC calculations take account of the `constants' (i.e., terms independent of $\eps,$ which are dependent on $k$), it doesn't seem immediately obvious why there is no improvement w.r.t. $k.$ This requires further investigation.

%I've got no definite conclusion on why this is true, but have sketched out below are some thoughts on why this might be the case, and what we could possibly do about it.

%\subsection{Suggestions why the observed behaviour might occur}
%
%I haven't written explicitly above how we chose the finest level $L$ in the MLMC method, nor have I written how $\ML$ and $\MLMC$ depend on $k$ (that is, I haven't written out how the finest discretisation level for MLMC, or the fixed discretisation level in MC, depend on $k$). However, one can figure this out (although I won't write out all the details now, as I'm a little short on time) and one obtains that for MLMC the finest mesh size $\hL$ satisfies
%\beqs
%\hL \sim 
%\begin{cases}
%k^{-3}\eps^{0.5} & \text{ in 2-D if } Q = \NW{u}\\
%k^{-2}\eps^{0.5}& \text{ in 2-D if } Q = \NLtD{u}\\
%k^{-4.5}\eps^{0.5}& \text{ in 3-D if } Q = \NW{u}\\
%k^{-3}\eps^{0.5} & \text{ in 3-D if } Q = \NLtD{u}\\
%\end{cases}.
%\eeqs
%
%In contrast, for MC, the mesh size $\hMC$ satisfies
%\beqs
%\hMC \sim 
%\begin{cases}
%k^{-3/2}\eps^{0.5} & \text{ in 2-D if } Q = \NW{u}\\
%k^{-1}\eps^{0.5}& \text{ in 2-D if } Q = \NLtD{u}\\
%k^{-3/2}\eps^{0.5}& \text{ in 3-D if } Q = \NW{u}\\
%k^{-1}\eps^{0.5} & \text{ in 3-D if } Q = \NLtD{u}\\
%\end{cases}.\eeqs
%
%The results for Monte Carlo make sense - when the QoI the weighted $H^1$ norm of the true solution, the error term (in \cref{ass:a} is $h^2k^3,$ (see \S \ref{sec:abweight}) and so we refine to avoid this. Similarly, when the QoI is the $L^2$ norm of the true solution, the error term (see \S \ref{sec:abltwo}) is $h^2k^2$, and so we refine to avoid this.
%
%However, for MLMC, we appear to \emph{over-refine} in order to achieve a given error tolerance. Why this is happening isn't particularly clear to me, other than I can guess that we're trading better dependence on $\eps$ for worse dependence on $k$ (which intutively makes sense, as the standard MLMC complexity theorem is only interested in the dependence on $\eps,$ and doesn't `see' $k,$ as it's just hidden in the constants. 
%
%This needs more thought from me as to exactly why this is happening.
%
%\subsection{Remedies}
%
%I don't have any definite ideas, although the main drawback of the above analysis is that $k$ does not explicitly come into the choice of $\hL$ and $\hMC$ (or the choice of the tthe numbers of samples)---the choice of $\hL$ and $\hMC$ is motivated by $\eps$ alone, and then $k$ appears in the final analysis, since the constants in \cref{thm:mlmccomp} depend on $k$. Therefore, maybe there is scope for a more `$k$-explicit construction that isn't just `apply the standard MLMC theorem, and see how things depend on $k$'. But I haven't had any more time to think about this yet.

%%%%%% Bibliography %%%%%%%

\bibliographystyle{plain}
\bibliography{../master-writing/biblio-owen}

\end{document}
