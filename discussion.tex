\subsection{Discussion of the main results in the context of other work on UQ for time-harmonic wave equations}\label{sec:otherwork}

In this section we discuss existing results on well-posedness of \eqref{eq:hh-intro}, as well as analogous results for the elastic wave equation and the time-harmonic Maxwell's equations. The most closely-related work to the current paper is \cite{FeLiLo:15} (and its analogue for elastic waves \cite{FeLo:17}), in that a large component of \cite{FeLiLo:15} consists of attempting to prove well-posedness and a priori bounds for the stochastic variational formulation (i.e.~\cref{prob:svsedp}) of the Helmholtz Interior Impedance Problem; i.e., \eqref{eq:hh-intro} with $A=I$ and stochastic $n$ posed in a bounded domain with an impedance boundary condition $\partial u/\partial \nu - ik u = g$ (recall that this boundary condition is a simple approximation to the Dirichlet-to-Neumann map $\TrR$ defined above \eqref{eq:src}). Under the assumption of existence, \cite{FeLiLo:15} shows that for any $k>0$ the solution is unique and satisfies an a priori bound of the form \eqref{eq:Sbound1} (with different constant $\Co$), provided $n=1+\eta$ where the random field $\eta$ satisfies (almost surely) $\N{\eta}_{L^\infty} \leq C/k$ for some $C>0$ independent of $k$. \cite{FeLiLo:15} then invokes Fredholm theory to conclude existence, but this relies on an incorrect assumption about compact inclusion of Bochner spaces---see \cref{sec:federico} below. However, combining \cref{thm:hh-gen,rem:tedp,rem:kdep} with $A=I$ and $\nz=1+\eta$ (with $\eta$ as above) produces an analogous result to \cref{thm:hh-hetero}, and gives a correct proof of \cite[Theorem 2.5]{FeLiLo:15}. Therefore the analysis of the Monte Carlo interior penalty discontinuous Galerkin method in \cite{FeLiLo:15} can proceed under the assumptions of \cref{thm:hh-gen,rem:tedp,rem:kdep}.

The paper \cite{HiScScSc:15} considers the Helmholtz transmission problem with a stochastic interface, i.e.~\eqref{eq:hh-intro} posed in $\RRd$ with both $A$ and $n$ piecewise constant and jumping on a common, randomly-located interface. A component of this work is establishing well-posedness of \cref{prob:msedp} for this setup. To do this, the authors make the assumption that $k$ is small (to avoid problems with trapping mentioned above---see the comments after \cite[Theorem 4.3]{HiScScSc:15}); the sesquilinear form $a$ is then coercive and an a priori bound (in principle explicit in $A$ and $n$) follows \cite[Lemma 4.5]{HiScScSc:15}. By \cref{rem:jumps}, the results of this paper can be used to obtain the analogous well-posedness result for large $k$ in the case of nontrapping jumps.

The paper \cite{BuGh:14} studies the \defn{Bayesian inverse problem} associated to \eqref{eq:hh-intro} with $A=I$ and $n=1$  posed in the  exterior of a Dirichlet obstacle. That is, \cite{BuGh:14} analyses computing the posterior distribution of the shape of the obstacle given noisy observations of the acoustic field in the exterior of the obstacle. A component of the analysis in \cite{BuGh:14} is the well-posedness of the forward problem for an obstacle with a variable boundary \cite[Proposition 3.5]{BuGh:14}. Instead of mapping the problem to one with  a fixed domain and variable $A$ and $n,$ \cite{BuGh:14} instead works with the variability of the obstacle directly, using boundary-integral equations. The $k$-dependence of the solution operator is not considered, but would enter in \cite[Lemma 3.1]{BuGh:14}.

%The paper \cite{KaPo:18} considers the time-harmonic Maxwell's equations posed on a fixed domain, with the material coefficient $\sigma$ defined by a random field on a subset of the domain. Part of the analysis of this paper is proving the well-posedness of the analogue of \cref{prob:somsedp} for the time-harmonic Maxwell's equations. The authors achieve this well-posedness result by working in a low-frequency regime, so that the analogue of the sesquilinear form $a$ is bounded and coercive (analagous to the approach in \cite{HiScScSc:15}), and therefore the Lax--Milgram Theorem can be applied to the deterministic problem to obtain existence, uniqueness, and a deterministic a priori bound. This bound is then converted into a bound for the stochastic problem using similar techniques to those in this paper. 

The papers \cite{JeScZe:17} and \cite{JeSc:16} consider the time-harmonic Maxwell's equations with (i) the material coefficients $\eps,\mu$ constant in the exterior of a perfectly-conducting random obstacle and (ii) $\eps,\mu$ piecewise constant and jumping on a common randomly located interface; in both cases these problems are mapped to problems where the domain/interface is fixed and $\eps$ and $\mu$ are random and heterogeneous. The papers \cite{JeScZe:17} and \cite{JeSc:16} essentially consider the analogue of \cref{prob:msedp} for the time-harmonic Maxwell's equations, obtaining well-posedness from the corresponding results for the related deterministic problems.

\subsection{Outline of the paper} In \cref{sec:otherwork} we discuss our results in the context of related literature. In \cref{sec:general} we  state general results on a priori bounds and well-posedness for stochastic variational formulations. In \cref{sec:genproof} we prove the results in \cref{sec:general}. In \cref{sec:hhproof} we prove \cref{thm:hh-gen,thm:hh-hetero}. In \cref{sec:federico} we discuss the failure of Fredholm theory for the stochastic variational formulation of Helmholtz problems. In \cref{app:mtBs} we recap results from measure theory and the theory of Bochner spaces.