%%% Number equations within section
%\numberwithin{equation}{section}

%\newcounter{neverused}
%\newtheorem{theorem}{Theorem}[section]

%\newenvironment{proof}[1][Proof]{\noindent\emph{#1}\,}{\hfill$\square$}
%\newtheorem{corollary}[theorem]{Corollary}
%\newtheorem{definition}[theorem]{Definition}
%\newtheorem{lemma}[theorem]{Lemma}
%\newtheorem{proposition}[theorem]{Proposition}
%\newtheorem{assumption}{Assumption}
%\newtheorem{example}{Example}
%\newremark{remark}{Remark}
%\newsiamremark{summary}[theorem]{Summary}
%\newsiamremark{criterion}[theorem]{Criterion}
%\newtheorem{notation}[theorem]{Notation}
%\newtheorem{algorithm}[theorem]{Algorithm}
%\newtheorem{construction}[theorem]{Construction}
%\newtheorem{condition}{Condition}
%\newtheorem{problem}{Problem}
\newtheorem{neverused}{Labelled Problems and Theorems}

\newcounter{literallydoesntmatter}

% The following is inspired by, but, I feel, not copied from 
% https://tex.stackexchange.com/questions/53978/custom-theorem-numbering
%\newsiamthm{convariable}[neverused]{Condition}
\newtheorem{convariable}[literallydoesntmatter]{Condition}
\newcommand{\bconvar}[1]{\renewcommand{\theconvariable}{#1}\begin{convariable}}%\setcounter{theorem}{\value{theorem}-1}
%\newcommand{\bconvar}{\begin{convariable}}
\newcommand{\econvar}{\end{convariable}}
%\newsiamthm{probvariable}[neverused]{Problem}
\newtheorem{probvariable}[literallydoesntmatter]{Problem}
\newcommand{\bprobvar}[1]{\renewcommand{\theprobvariable}{#1}\begin{probvariable}}%\setcounter{theorem}{\value{theorem}-1}
\newcommand{\eprobvar}{\end{probvariable}}


\newcommand{\bconS}{\bconvar{S}}
\newcommand{\bconB}{\bconvar{B}}
\newcommand{\bconL}{\bconvar{L2}}
\newcommand{\bconAo}{\bconvar{A1}}
\newcommand{\bconLo}{\bconvar{L1}}
\newcommand{\bconA}{\bconvar{A2}}
\newcommand{\bconCo}{\bconvar{C1}}
\newcommand{\bconC}{\bconvar{C2}}
\newcommand{\bconAL}{\bconvar{AL}}
\newcommand{\bconK}{\bconvar{U}}

\newcommand{\bprobAE}{\bprobvar{AS}}
\newcommand{\bprobM}{\bprobvar{MAS}}
\newcommand{\bprobLT}{\bprobvar{SOAS}}
\newcommand{\bprobSVAR}{\bprobvar{SV}}


%\newtheorem{probo}[neverused]{Problem}
%\renewcommand{\theprobo}{{\ref{prob:sedp}}}
%
%\newtheorem{probt}[neverused]{Problem}
%\renewcommand{\theprobt}{{\ref{prob:msedp}}}
%
%\newtheorem{probth}[neverused]{Problem}
%\renewcommand{\theprobth}{{\ref{prob:somsedp}}}
%
%\newtheorem{probf}[neverused]{Problem}
%\renewcommand{\theprobf}{{\ref{prob:svsedp}}}
%










%%%%%%%%%%%Eike's new theorem version 1%%%%%%%%%%%%%%%%

%\newcounter{mycount}
%\newsiamthm{myprob}{Problem}
%\renewcommand{\themycount}{\Alph{mycount}}

%\addtoreset{mycount}{section}

%%%%%%%%%%%%Eike's new theorem version 2%%%%%%%%%%%5
%%%%%% give e.g Theorem 2.A 
%\newcounter{mycountsec}
%\newsiamthm{myprobsec}[mycountsec]{Problem}
%\newsiamthm{myprobsec}{Problem}
%\renewcommand{\themycountsec}{\thesection.\Alph{mycountsec}}

%\addtoreset{mycount}{section}
%%%%%% line above resets counter at beginning of section

%%%%%%%%%%%%%%%%%%%%%%%%%%%%%%%%%%


% Shortcuts for special theorems etc.
\newcommand{\bmps}{\begin{myprobsec}}
\newcommand{\emps}{\end{myprobsec}}
\newcommand{\bmp}{\begin{myprob}}
\newcommand{\emp}{\end{myprob}}
