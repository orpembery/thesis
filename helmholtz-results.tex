\subsection{Statement of main results}\label{sec:hh-results}

\paragraph{Notation and basic definitions}Let either (i) $\Dm \subset \RRd,$ $d=2,3,$ be a bounded Lipschitz open set such that $\bzero \in \Dm$ and the open complement $\Dp\de \RR^d\setminus \overline{\Dm}$ is connected, or (ii) $\Dm = \emptyset.$ Let $\GD = \partial \Dm.$ 
Fix $R>0$ and let $\BR$ be the ball of radius $R$ centred at the origin. Define $\GR := \partial \BR$ and $\DR \de \Dp \cap \BR$ (see \cref{fig:domain}). Let $\gamma$ denote the trace operator from $\DR$ to $\partial \DR = \GD \cup \GR$ and define $\HozDDR \de \set{v \in \HoDR \st \gamma v = 0 \ton \GD}.$ 
 
Let $\TrR: H^{1/2}(\Gamma_R) \rightarrow H^{-1/2}(\Gamma_R)$ be the Dirichlet-to-Neumann map for the deterministic equation $\Delta u+k^2 u=0$ posed in the exterior of $\BR$ with the Sommerfeld radiation condition 
\beq\label{eq:src}
\frac{\partial u}{\partial r}(\bx) - \ri ku(\bx) = o\mleft(\frac1{r^{(d-1)/2}}\mright) \text{ as } r\de\abs{\bx}\rightarrow \infty, \text{ uniformly in } \frac{\bx}{\abs{\bx}};
\eeq
see \cite[Section 2.6.3]{Ne:01} and \cite[Equations 3.5 and 3.6]{ChMo:08} for an explicit expression for $\TrR$  in terms of Hankel functions and Fourier series ($d=2$)/spherical harmonics ($d=3$). Let $\IPGR{\cdot}{\cdot}$ be the duality pairing on $\GR$ between $\HmhGR$ and $\HhGR$ and write $\dd\Leb$ for Lebesgue measure.

Let $\LiDpRRdtd$ be the set of all matrix-valued functions $A:\Dp\rightarrow\RRdtd$ such that $A_{i,j} \in \LiDpRR$ for all $i,j = 1,\ldots,d.$ Where the range of functions is $\CC$ we suppress the second argument in a function space, e.g.~we write $\LiDp$ for $\LiDpCC.$ We write $\Do \compcont \Dt$ if $\Do$ is a compact subset of the open set $\Dt.$ Let $\OFP$ be a complete probability space. Throughout this paper, unless stated otherwise we equip a topological space with its Borel $\sigma$-algebra. See \cref{app:mtBs} for a summary of the measure-theoretic concepts used in this paper. Let
\bit
\item $f:\Omega\rightarrow\LtDp$ be such that $\supp f \compcont \BR$ almost surely
\item $n:\Omega\rightarrow \LiDpRR$ be such that $\supp(1-n) \compcont \BR$ almost surely and there exist $\nmin, \nmax:\Omega\rightarrow \RR$ such that
%\beqs
$0 < \nmin(\omega) \leq n(\omega)(\bx) \leq \nmax(\omega)$
%\eeqs
for almost every $\bx \in \Dp$ almost surely, and
\item $A:\Omega\rightarrow\LiDpRRdtd$ be such that $\supp(I-A) \compcont \BR,\, A_{ij} = A_{ji}$ almost surely, and there exist $\Amin,\Amax:\Omega \rightarrow \RR$ such that $0 < \Amin(\omega) < \Amax(\omega)$ almost surely and
%\beqs
$\Amin(\omega)\abs{\bxi}^2 \leq \big(A(\omega)(\bx)\bxi\big)\cdot\bxi \leq \Amax(\omega)\abs{\bxi}^2$
%\eeqs
for almost every $\bx \in \Dp$ and for all $\bxi \in \CCd$ almost surely.
\eit
If $v:\Omega \rightarrow Z$ for some function space $Z$ of functions on $\RRd,$ we abuse notation slightly and write $v(\omega,\bx)$ instead of $v(\omega)(\bx).$


\def\domainscale{2}

\begin{tikzpicture}[even odd rule]
  % \draw (-2,2) .. controls (-1,3) and (-2,3) .. (-1,2.5);
  % \draw (-2,2) .. controls (-1,4) and (-2,3.5) .. (-1,2.5);
  % Sketch
  %\draw[color=gray] (-2,2) -- (-1,2.5) -- (0,2.5) -- (2,3) -- (3,2.5) -- (4,1.5) -- (4,-1.5) -- (3,-2.5) -- (2,-3) -- (0.5,-3) -- (-3,-1) -- (-4,0.5) -- cycle;

  % \draw[color=gray]
    % (4.5,0) --
    % (4.25,-2) --
    % (3,-3) --
    % (2.25,-3.5) --
    % (1.5,-3) --
    % (0.5,-2.5) --
    % (-0.75,-3) --
    % (-2,-3.5) --
    % (-3,-3) --
    % (-3.75,-1.5) --
    % (-5.25,-1) --
    % (-6,-0.75) --
    % (-6,0.25) --
    % (-4.5,1.5) --
    % (-3.5,2.25) --
    % (-3,3.5) --
    % (0,3) --
    % (0.75,3) --
    % (1.5,3.5) --
    % (3,3.5) --
    % (3.75,3) --
    % (4.25,1.75) --
    % cycle;
    
    % This will hopefully be some lines
    \pgfdeclarepatternformonly{owennortheast}
    {\pgfpointorigin}{\pgfpoint{1cm}{1cm}}
    {\pgfpoint{1cm}{1cm}}
    {
    \pgfpathmoveto{\pgfpointorigin}
    \pgfpathlineto{\pgfpoint{1cm}{1cm}}
    \pgfsetlinewidth{0.5\pgflinewidth}
    \pgfusepath{stroke}
    }
    

     \pgfdeclarepatternformonly{owennorthwest}
    {\pgfpointorigin}{\pgfpoint{1cm}{1cm}}
    {\pgfpoint{1cm}{1cm}}
    {
    \pgfpathmoveto{\pgfpoint{1cm}{0cm}}
    \pgfpathlineto{\pgfpoint{0cm}{1cm}}
   \pgfsetlinewidth{0.5\pgflinewidth}
    \pgfusepath{stroke}
    }

    \filldraw[pattern=owennorthwest,scale=\domainscale] % Truncation
    (3.5, 0.0) --
    (1.081559480312316, 3.3286978070330373) .. controls (-0.581559480312316, 2.3) ..
    (-2.8315594803123156, 2.0572483830236563) .. controls (-3.8315594803123156, 0.0572483830236563) and (-3.8315594803123165, -0.0572483830236554) ..
    (-2.8315594803123165, -2.0572483830236554) .. controls (-2.3315594803123165, -3.0572483830236554) and (-0.5815594803123152, -3.4286978070330377) ..
    (1.0815594803123152, -3.3286978070330377) --
    %% (3.0, 0.0) --
    %% (0.9270509831248424, 2.8531695488854605) --
    %% (-2.427050983124842, 1.7633557568774196) --
    %% (-2.427050983124843, -1.7633557568774192) --
    %% (0.9270509831248417, -2.853169548885461) --
    %% (4.0, 0.0) --
    %% (1.2360679774997898, 3.804226065180614) --
    %% (-3.2360679774997894, 2.351141009169893) --
    %% (-3.2360679774997902, -2.351141009169892) --
    %% (1.236067977499789, -3.8042260651806146) --
    cycle;

    %% \filldraw[pattern=owennorthwest] % Truncation

    %% cycle;
    
   %% [scale=2/3] (4.5,0) .. controls (4.5,-0.5) and (4.25,-2) .. %A
   %%  (3,-3) .. controls (2.875,-3.1) and (2.25,-3.5) ..%B
   %%  (1.5,-3) .. controls (0.75,-2.5) and (0.5,-2.5) ..%C
   %%  (-0.75,-3) .. controls (-2,-3.5) and (-2.8,-3.4)..%D
   %%  (-3,-3) .. controls (-3.75,-1.5) and (-4.95,-1.1) ..%E
   %%  (-5.25,-1) .. controls (-6,-0.75) and (-6,0.25) ..%F
   %%  (-4.5,1.5) .. controls (-4.35,1.625) and (-3.5,2.25) ..%G
   %%  (-3,3.5) .. controls (0,3) and (0.6,2.9) ..%H
   %%  (1.5,3.5) .. controls (2.25,4) and (3.5,3.5) ..%I
   %%  (3.75,3) .. controls (4,2.5) and (4.5,0.5)..%J
    %%  cycle;
    

    \filldraw[pattern=owennortheast,scale=\domainscale] % The obstacle
    (1.0, 0.0) .. controls (1.0,0.3) and (1.0071067811865476, 0.7071067811865475) ..
    (0.7071067811865476, 0.7071067811865475) .. controls (0.4071067811865476, 0.7071067811865475) and (0.2, 0.8) ..
    (0.0, 1.0) .. controls (-0.2,1.2) and (-0.5071067811865475, 0.9071067811865476) ..
    (-0.7071067811865475, 0.7071067811865476) .. controls (-0.9071067811865475, 0.5071067811865476) and (-1.2,0.2) ..
    (-1.0, 0.0) .. controls (-0.8,-0.2) and (-0.7071067811865477, -0.5071067811865475) ..
    (-0.7071067811865477, -0.7071067811865475) .. controls (-0.7071067811865477, -0.9071067811865475) and (-0.2,-0.8) ..
    (0.0, -1.0) .. controls (0.2,-0.6) and (0.4071067811865474, -0.6071067811865477) .. % The corner at the start of this line is the one where there's a 'proper' corner
    (0.7071067811865474, -0.7071067811865477) .. controls (1.0071067811865474, -0.8071067811865477) and (1.0,-0.2) ..
    cycle
    ;

    \draw[scale=\domainscale] (0,0) circle [radius=2.25]; % The ball
    
    %% (4.5,0) -- %controls (4.5,-0.5) and (4.25,-2) .. %A
    %% (3,-3) -- %controls (2.875,-3.1) and (2.25,-3.5) ..%B
    %% (1.5,-3) -- %controls (0.75,-2.5) and (0.5,-2.5) ..%C
    %% (-0.75,-3) -- %controls (-2,-3.5) and (-2.8,-3.4)..%D
    %% (-3,-3) -- %controls (-3.75,-1.5) and (-4.95,-1.1) ..%E
    %% (-5.25,-1) -- %controls (-6,-0.75) and (-6,0.25) ..%F
    %% (-4.5,1.5) -- %controls (-4.35,1.625) and (-3.5,2.25) ..%G
    %% (-3,3.5) -- %controls (0,3) and (0.6,2.9) ..%H
    %% (1.5,3.5) -- %controls (2.25,4) and (3.5,3.5) ..%I
    %% (3.75,3) -- %controls (4,2.5) and (4.5,0.5)..%J
    %% cycle;

% The labels

\draw (0,0) node[fill=white] {$\Dm$};
\draw (-3.5,-3.5) node[fill=white] {$D:=\Dtilde\setminus \clos{\Dm}$};

\draw (3.2,-2) node[fill=white] {$\GD$};

\draw (3.7,7) node {$\GI$};
    
  % \draw (-2,2) .. controls (-1,3) and (-2,1.5) .. (-1,2.5)
  % .. controls (0,1.5) and (-1,1) .. (0,2.5)
  % .. controls (1,3) and (1,4.5) .. (2,3)
  % .. controls (3,1.5) and (2,2) .. (3,2.5)
  % .. controls (4,3) and (4,2.5) .. (4,1.5)
  % .. controls (4,0.5) and (4.5,-0.5) .. (4,-1.5)
  % .. controls (3.5,-2.5) and (4,-3.5) .. (3,-2.5)
  % .. controls \hout and \iin .. (2,-3)
  % .. controls (1,-4) and (1.5,-4) .. (0.5,-3)
  % .. controls (-0.5,-2) and (-2,-1) .. (-3,-1)
  % .. controls (-4,-1) and (-5,-0.25) .. (-4,0.5)
  % .. controls (-3,1.25) and (-3,1) .. cycle;
\end{tikzpicture}


\paragraph{Variational Formulations} We consider three different formulations of the  \emph{Helmholtz stochastic exterior Dirichlet problem} (stochastic EDP); \cref{prob:msedp,prob:somsedp,prob:svsedp} below.

Define the sesquilinear form $a(\omega)$ on $\HozDDR \times \HozDDR$ by
\beq\label{eq:SEDPa}
\mleft[a(\omega)\mright]\mleft(\vo,\vt\mright)\de\int_{D_R}\Big( \mleft( A(\omega) \grad \vo\mright)\cdot \grad \vtb 
 - k^2 n(\omega)\, \vo\,\vtb \Big)\dd\Leb- \big\langle T_R \gamma \vo,\gamma \vt\big\rangle_{\Gamma_R},
 \eeq
 and the antilinear functional $L(\omega)$ on $\HozDDR$ by
\beq\label{eq:SEDPL}
\mleft[L(\omega)\mright](\vt)\de \int_{D_R} f(\omega)\, \vtb\,\dd\Leb.
\eeq
Define the sesquilinear form $\as$ on $L^2\big(\Omega;H_{0,D}^1(D_R)\big)\times L^2\big(\Omega;H_{0,D}^1(D_R)\big)$ and the antilinear functional $\Ls$ on $L^2\big(\Omega;H_{0,D}^1(D_R)\big)$ by 
\beq\label{eq:SEDPas}
\as\mleft(\vo,\vt\mright)\de \int_\Omega \mleft[a(\omega)\mright]\mleft(\vo(\omega),\vt(\omega)\mright)\dd\PP(\omega)
\quad\text{ and } \quad
%\eeq
%
%\beq
%\label{eq:SEDPLs}
\Ls(\vt)\de \int_\Omega \mleft[L(\omega)\mright]\mleft(\vt(\omega)\mright)\dd\PP(\omega).
\eeq
We consider the following three problems:

\bprobvar{1}[Measurable EDP almost surely]\label{prob:msedp}
Find a measurable $u:\Omega\rightarrow\HozDDR$ such that
\vspace{-2ex}
\beqs
\mleft[a(\omega)\mright]\mleft(u(\omega),v\mright) = \mleft[L(\omega)\mright](v) \tforall v \in \HozDDR \text{ almost surely.}
\eeqs
\eprobvar

\bprobvar{2}[Second-order EDP almost surely]\label{prob:somsedp}
Find $u\in L^2\big(\Omega;H_{0,D}^1(D_R)\big)$ such that
\beqs
\mleft[a(\omega)\mright]\mleft(u(\omega),v\mright) = \mleft[L(\omega)\mright](v) \tforall v \in \HozDDR \text{ almost surely.}
\eeqs
\eprobvar

\bprobvar{3}[Stochastic variational EDP]\label{prob:svsedp}
Find $u\in L^2\big(\Omega;H_{0,D}^1(D_R)\big)$ such that
\beqs
\as(u,v) = \Ls(v) \tforall v \in \LtOHozDDR.
\eeqs
\eprobvar

%\bre[The relationship between \cref{prob:msedp,prob:somsedp,prob:svsedp}]\label{rem:lit}

\Cref{prob:somsedp} is the foundation of sampling-based UQ methods, such as Monte-Carlo and Stochastic-Collocation methods; its analogue for the stationary diffusion equation is well-studied in, e.g., \cite{XiHe:05,BaNoTe:07,NoTeWe:08a,Ch:12,ChScTe:13,TeJaWeGu:15,KuNu:16,HeLaSc:18}. Similarly \cref{prob:svsedp} is the foundation of the Stochastic Galerkin method (a finite element method in $\Omega \times D,$ where $D$ is the spatial domain), and is studied for the Helmholtz Interior Impedance Problem in \cite{FeLiLo:15}, and its analogue for the stationary diffusion equation is considered in, e.g., \cite{BaTeZo:04,KhSc:11,BaScZo:11,GuWeZh:14}.

%The prevalance of \cref{prob:somsedp,prob:svsedp} in the literature therefore raises two questions:
%\ben
%\item\label{q:whyone} Why consider \cref{prob:msedp}?
%\item\label{q:wpe} When are the well-posedness and equivalence of \cref{prob:msedp,prob:somsedp,prob:svsedp} non-trivial to establish?
%\een
%
%We address the first question in the next remark. Regarding the second question, in the stationary diffusion case when the coefficient $\kappa$ in \eqref{eq:diffusion} is uniformly bounded above and below on $\Omega \times D,$ the bilinear form for the analogue of \cref{prob:svsedp} is coercive on $\Omega \times D,$ and so we can obtain well-posedness and an a priori bound on the solution using the Lax--Milgram theorem. As we show in the general framework below, one can then obtain similar results for \cref{prob:somsedp,prob:msedp} under mild regularity and integrability assumptions. However, when the coefficient $\kappa$ is not bounded above and below on $\Omega \times D$ (as in the well-studied case of a lognormal random field), one can only show well-posedness and an a priori bound for the analogue of \cref{prob:somsedp} using a pathwise approach (as in \cite{ChScTe:13}) and for the analogue of \cref{prob:svsedp} by modifiying the functions spaces used, as in \cite{Gi:10,MuSt:11}.
%
%In the Helmholtz case, even when the coefficients $A$ and $n$ are bounded uniformly above and below, the sesquilinear form $\as$ in \cref{prob:svsedp} is not coercive on $\Omega\times \DR$, so one cannot use the Lax--Milgram theorem, and Fredholm theory (which can be used in the deterministic case to conclude existence and uniqueness) is not applicable, because the Bochner space $\LtOHoDR$ is not compactly contained in $\LtOLtDR$---see \cref{sec:federico}. Therefore, currently the only way to obtain well-posedness results and a priori bounds on the solution of \cref{prob:svsedp} is to use the results in this paper. We use a pathwise approach, imposing conditions on $A$ and $n$ such that we can construct a solution to \cref{prob:msedp}, and then showing this solution is also a solution of \cref{prob:somsedp,prob:svsedp} with the necessary properties. 
%\ere

\bre[Why consider \cref{prob:msedp}?]\label{rem:whyone}

The difference between \cref{prob:msedp,prob:somsedp} is that \cref{prob:msedp} requires no integrability of $u$ over $\Omega$, whereas \cref{prob:somsedp} requires $u \in \LtOHozDDRnormal$. Since all the theory for sampling-based UQ methods assume some integrability of the solution, the natural question is: why consider \cref{prob:msedp} at all?

The main reason we consider \cref{prob:msedp} is that, given the existing PDE theory for the Helmholtz equation, we can prove existence of a solution to \cref{prob:msedp} under general conditions on $A$ and $n$, but there is no current prospect of proving existence of a solution to \cref{prob:somsedp} under general conditions on $A$ and $n$. The explanation for this consists of the following three points:
\ben
\item The only two known ways to obtain a solution to \cref{prob:somsedp} are: (i) obtain a deterministic a priori bound, explicit in all parameters, and integrate (followed, e.g., in \cite{ChScTe:13} for \eqref{eq:diffusion} with lognormal coefficients) and (ii) obtain a solution to \cref{prob:svsedp} and show this is a solution to \cref{prob:somsedp}. In the Helmholtz case, doing (ii) is difficult as neither the Lax--Milgram theorem nor Fredholm theory is applicable (as explained in the introduction), and so we follow the approach in (i).
\item The only known bounds on the solution of the Helmholtz equation explicit in all parameters are those recently obtained for nontrapping scenarios in \cite{GrPeSp:19,GaSpWu:18}.
\item Obtaining a bound explicit in all parameters for a general class of $A$ and $n$, e.g., $A \in \WoiDRRRdtd$ and $n \in \LiDRRR$ is well beyond current techniques. Indeed, a general class of $A$ and $n$ will include both trapping and nontrapping scenarios, and such a bound would need to capture the exponential blow-up in $k$ for trapping $A$ and $n$, the uniform boundedness in $k$ for nontrapping $A$ and $n$, and be explicit in $A$ and $n$.
\een
Given this fact that there is no current prospect of proving existence of a solution to \cref{prob:somsedp} under general conditions on $A$ and $n$ we keep \cref{prob:msedp} so that we prove an (albeit weaker) existence result for the Helmholtz equation with general coefficients.


%or Fredholm theory are not
%
% We can show that $u \in \LtOHozDDRnormal$ by obtaining an a priori bound in $\HozDDR$ on the solution of the corresponding deterministic problem, where the bound is explicit in all of the stochastic parameters, checking conditions under which this bound can be integrated over $\Omega,$ and then integrating this deterministic bound over $\Omega$ to obtain a bound in $\LtOHozDDRnormal.$ This is precisely the approach followed in \cite{ChScTe:13} to obtain well-posedness results and an a priori bound for the stationary diffusion equation with a lognormal coefficient.
%
%However, in the Helmholtz case, as mentioned in \cref{sec:intro}, establishing an a priori bound on the corresponding deterministic problem that is explicit in all of the parameters of interest (such as $A$ and $n$) is difficult---the only such bounds where $A$ and $n$ both vary are those recently obtained in \cite{GrPeSp:19}, where these bounds are obtained under certain conditions on $A$ and $n$. Therefore, one can only prove that a solution of \cref{prob:somsedp,prob:svsedp} exists for small classes of $A$ and $n$---see \cref{thm:hh-hetero} below for such a result. Hence, given all the existing PDE theory for the Helmholtz equation, one \emph{cannot} prove \emph{existence} results for \cref{prob:somsedp,prob:svsedp} for general coefficients, including so-called trapping coefficients; \cref{prob:msedp} is the \emph{only} problem for which one can prove existence and uniqueness---see \cref{thm:hh-gen} below for this result.
%
%Moreover, \cref{prob:msedp} (or rather, its more general analogue \cref{prob:meas} in \cref{sec:notdef} below) can be used as a stepping stone to proving well-posedness results and a priori bounds for the more general analogues of \cref{prob:somsedp,prob:svsedp} (\cref{prob:lt,prob:svar} below). If one can show that a solution of \cref{prob:meas} exists (e.g., by constructing a solution pathwise), and that \cref{con:coeffstoform,con:A,con:coeffstofunc,con:L,con:cborel,con:C,con:B,con:K} in \cref{sec:cons} below hold, then one can show well-posedness and equivalence of all of \cref{prob:meas,prob:lt,prob:svar} and prove an a priori bound on the solution. See \cref{fig:ladder} for a summary of these results.
\ere

\bre[Measurability of $u$ in \cref{prob:msedp}]
It is natural to construct  the solution of \cref{prob:msedp} pathwise; that is, one defines $u(\omega)$ to be the solution of the deterministic problem with coefficients $A(\omega)$ and $n(\omega).$ However, is it then not obvious that $u$ is measurable.
In the proof of \cref{thm:hh-gen} below, we show that the measurability of $u$ follows from
\ben
\item a natural condition on the measurability of the coefficients and data (\cref{con:cborel} below), and 
\item the continuity of the map taking the coefficients of the deterministic PDE to the solution of the deterministic PDE (see \cref{lem:solcont} below).
  \een
\ere


In \cref{thm:hh-gen,thm:hh-hetero} we prove results on the well-posedness of \cref{prob:msedp,prob:somsedp,prob:svsedp} under conditions on $A,$ $n,$ $f,$ and $\Dm.$ Although $A,n,$ and $f$ are defined on $\Dp,$ since $\supp(I-A),$ $\supp(1-n),$ and $\supp f$ are compactly contained in  $\DR$ we can consider $A,n,$ and $f$ as functions on $\DR.$

\bcon[Regularity and stochastic regularity of $f,$ $A,$ and $n$]\label{con:hh-fAn}
The random fields $f, A,$ and $n$ satisfy $f \in \LtOLtDR,$  $A:\Omega \rightarrow \WoiDRRRdtd$ with $A \in \LiOLiDRRRdtd,$ and $n \in \LiOLiDRRR.$
\econ

\bth[Equivalence of variational problems]\label{thm:hh-gen}
Under \cref{con:hh-fAn}:
\bit
\item The maps $\as$ and $\Ls$ (defined by \eqref{eq:SEDPas}) are well-defined.
\item $u \in %\LtOHozDDR$ 
L^2\big(\Omega;H_{0,D}^1(D_R)\big)$
solves \cref{prob:somsedp} if and only if $u$ solves \cref{prob:svsedp}.
\item If $u \in
L^2\big(\Omega;H_{0,D}^1(D_R)\big)$
%\in \LtOHozDDR$ 
solves \cref{prob:somsedp}, then any member of the equivalence class of $u$ solves \cref{prob:msedp}.
\item The solution of \cref{prob:msedp} exists and is unique up to modification on a set of measure zero in $\Omega.$
\item The solution of \cref{prob:somsedp,prob:svsedp} is unique in $L^2\big(\Omega;H_{0,D}^1(D_R)\big)$. %$\LtOHozDDR.$
\eit
\enth

Observe that the only relationship between formulations not proved in \cref{thm:hh-gen} is: if $u:\Omega\rightarrow \HozDDR$  solves \cref{prob:msedp} then $u \in 
L^2\big(\Omega;H_{0,D}^1(D_R)\big)$
%\LtOHozDDR$ 
and $u$ solves \cref{prob:somsedp}. \cref{thm:hh-hetero} below includes this relationship, but we need additional assumptions on $A,n,$ and $\Dm.$

\bde[A particular class of (deterministic) nontrapping coefficients]\label{def:hh-nontrapping}
Let $\muo,\mut >0,$ $\Az \in \WoiDRRRdtd$ with $\supp(I-\Az) \compcont \BR$, and $\nz \in \WoiDRRR$ with $\supp(1-\nz) \compcont \BR.$ We write $\Az \in \NTA{\muo}$ and $\nz \in \NTn{\mut}$ if
\beq\label{eq:hh-Acond}
\Az(\bx) - \mleft(\bx\cdot\grad\mright)\Az(\bx) \geq \muo \quad \tand \quad \nz(\bx) + \bx\cdot\grad \nz(\bx) \geq \mut
\eeq
for almost every $\bx \in \DR,$ where the first inequality holds in the sense of quadratic forms.
\ede

The significance of the class of coefficients in \cref{def:hh-nontrapping} is that \cite[Theorem 2.5]{GrPeSp:19} proves bounds on the solution of \eqref{eq:hh-intro} for such $A$ and $n,$ where the constant in the bound only depends on $\muo, \mut, k, R, $ and $d.$

\bcon[{$k$-independent nontrapping conditions on (random) $A$ and $n$}]\label{con:hh-hetero}
The random fields $A$ and $n$ satisfy $A:\Omega\rightarrow\WoiDRRRdtd$ and $n:\Omega\rightarrow \WoiDRRR.$ Furthermore, there exist $\muo, \mut:\Omega\rightarrow \RR,$ independent of $f,$ with $\muo(\omega),\mut(\omega) > 0$ almost surely and $1/\muo,1/\mut \in \LtORR$  such that $A(\omega) \in \NTA{\muo(\omega)}$ almost surely and $n(\omega) \in \NTn{\mut(\omega)}$ almost surely.
\econ


\bde[Star-shaped]
The set $D \subseteq \RRd$ is \defn{star-shaped with respect to} the point $\bxz$ if for any $\bx \in D$ the line segment $\mleft[\bxz,\bx\mright] \subseteq D.$
\ede

\bth[{Equivalence of variational problems in a nontrapping case}]\label{thm:hh-hetero}
Let $\Dm$ be star-shaped with respect to the origin. Under \cref{con:hh-fAn,con:hh-hetero}:
\bit
\item The maps $\as$ and $\Ls$ (defined by \eqref{eq:SEDPas}) are well-defined.
\item \cref{prob:msedp,prob:somsedp,prob:svsedp} are all equivalent.
\item The solution $u \in %\LtOHozDDR$ 
L^2\big(\Omega;H_{0,D}^1(D_R)\big)$
of these problems exists, is unique, and, given $\kz > 0,$ satisfies the bound
\beq\label{eq:Sbound1}
\NLtOLtDR{\grad u}^2 + k^2\NLtOLtDR{u}^2\leq \NLoO{\Co} \NLtOLtDR{f}^2
\eeq
for all $k\geq\kz$, where $\Co:\Omega\rightarrow\RR$ is given by
\beq\label{eq:C1}
\Co = \max\mleft\{\frac1{\mu_1},\frac1{\mu_2}\mright\}\mleft(\frac{R^2}{\mu_1} + \frac{2}{\mu_2}\mleft(R+ \frac{d-1}{2\kz}\mright)^2\mright).
\eeq
\eit
\enth

As highlighted above, \cref{thm:hh-hetero} is obtained from combining
deterministic a priori bounds from \cite{GrPeSp:19} with the general
arguments in \cref{sec:general} about well-posedness of variational
formulations of stochastic PDEs. \cref{thm:hh-hetero} uses the most basic a priori bound proved
in \cite{GrPeSp:19} (from \cite[Theorem 2.5]{GrPeSp:19}), but \cite{GrPeSp:19}
contains several extensions of this bound. \cref{rem:planewave,rem:ones,rem:tedp,rem:jumps,rem:kdep}
outline the implications that these (deterministic) extensions have for
the stochastic Helmholtz equation.

\bre[Dirichlet boundary conditions on $\GD$ and plane-wave incidence]\label{rem:planewave}
The formulations of the stochastic EDP above assume that $u=0$ on the boundary $\GD.$ An important scattering problem for which $u \neq 0$ on $\GD$ is when $u$ is the field scattered by an incident plane wave; in this case $\gamma u = -\gamma \uI,$ where $\uI$ is the incident plane wave \cite[p. 107]{ChGrLaSp:12}.

The results in this paper can be easily extended to the case when $u\neq0$ on $\pDm$ using \cite[Theorem 2.19(ii)]{GrPeSp:19}  which proves a priori (deterministic) bounds in this case. One subtlety, however, is that $f$ is then not necessarily independent of $\muo$ and $\mut.$ Indeed in this case
%\beqs%\label{eq:fplanewave}
$f = -\grad\cdot\mleft(A\grad \uI\mright) - k^2n\uI$.
%\eeqs
If $\muo$ depends on $A$ and $\mut$ depends on $n$ then %\eqref{eq:fplanewave} shows that 
$f$ may be not be independent of $\muo$ and $\mut.$ One can produce an analogue of \cref{thm:hh-hetero} in the case where $f,\muo,$ and $\mut$ are dependent, but one requires $1/\muo, 1/\mut \in \LfO$ and $f \in \LfOLtD;$ see \cref{rem:notindep} below.
\ere

\bre[The case when either $n=1$ or $A=I$]\label{rem:ones}
When either $n=1$ or $A=I,$ \cite[Theorem 2.19]{GrPeSp:19} gives deterministic bounds under weaker conditions on $A$ and $n$ respectively; the corresponding results for the stochastic case are that:

\bit
\item 

When $n=1$  almost surely, the condition $A(\omega) \in \NTA{\muo(\omega)}$ in \cref{con:hh-hetero} can be improved to
%\beqs
$2A(\omega) - \mleft(\bx \cdot \nabla \mright)A(\omega) \geq \muo(\omega)$
%\eeqs
for almost every $\bx\in \Dp,$ almost surely.

\item When $A=I$ almost surely, the condition $n(\omega) \in \NTn{\mut(\omega)}$ in \cref{con:hh-hetero} can be improved to:
\beq\label{eq:nimproved}
2n(\omega) + \bx \cdot \nabla n(\omega) \geq \mut(\omega) \,\text{ for almost every $\bx \in \Dp$, almost surely}.
\eeq
%for almost every $\bx\in \Dp,$ almost surely.
\eit
\ere

\bre[Geometric interpretation of the conditions on $A$ and $n$ in \cref{def:hh-nontrapping}]
Recall that the $k\tendi$ asymptotics of solutions of the Helmholtz
equation are governed by the behaviour of rays (see, e.g.,
\cite{BaBu:91}). Given (deterministic) $\Az$ and $\nz,$ the Helmholtz EDP is \defn{nontrapping} if all rays starting
in $\DR$ and evolving according to the
Hamiltonian flow defined by the symbol of $\grad\cdot\mleft(\Az\grad u\mright) + k^2\nz u = -\fz$  escape from $\DR$ after some uniform time (see, e.g., \cite[Definition 1.1]{Bu:02}); the EDP is \defn{trapping} otherwise.
The $k$-dependence of the solution operator depends strongly on
whether the problem is trapping, and the type of trapping present;
see, e.g., the overview discussions in \cite[Section 1]{GrPeSp:19},
\cite[Section 1.1]{ChSpGiSm:17}.

The conditions on $A$ and $n$ in \cref{con:hh-hetero} and the star-shapedness restriction
on $\Dm$ are sufficient for the Helmholtz stochastic EDP to be nontrapping almost surely. As noted in \cref{rem:ones}, when $A=I$ almost surely the condition on $n$  can be improved  from that in \eqref{eq:hh-Acond} to \eqref{eq:nimproved} using \cite[Theorem 2.19(ii)]{GrPeSp:19}.
The condition \eqref{eq:nimproved} is equivalent  to nontrapping when $n$ is radial, i.e. $n(\omega,\bx)=
n(\omega,\abs{\bx})$. Indeed, if $n$ is radial and $2n(\omega,\bx)+ \bx \cdot\nabla n(\omega,\bx)<0$ at a point $\bx \in \RRd$,
then the deterministic Helmholtz EDP given by $n(\omega,\bx)$ is trapping; see \cite{Ra:71} and \cite[Theorem 7.7]{GrPeSp:19}.
\ere

\bre[The Helmholtz stochastic truncated exterior Dirichlet problem]\label{rem:tedp}
When applying the Galerkin method to \cref{prob:msedp,prob:somsedp,prob:svsedp}, the Dirichlet-to-Neumann map $\TrR$ is expensive to compute. Therefore, it is
common to approximate the DtN map on $\GR$ by an `absorbing
boundary condition' (see, e.g., \cite[Section 3.3]{Ih:98} and the references
therein), the  simplest of which is the impedance boundary condition
$\partial u/\partial \nu - \ri k u=0$. We call the Helmholtz stochastic EDP posed in $\DR$ with
an impedance boundary condition on $\GR$ the stochastic \emph{truncated
exterior Dirichlet problem} (stochastic TEDP). In fact, since we no longer need to know
the DtN map explicitly on the truncation boundary, the truncation
boundary can be arbitrary (i.e. it does not have to be just a circle/sphere). Note that in the case when the obstacle is the empty set, the TEDP is just the Interior Impedance Problem.

The results in this paper also hold for the stochastic TEDP (with arbitrary Lipschitz truncation boundary) under an analogue of \cref{con:hh-hetero} based on the deterministic bounds in \cite[Theorem A.6(i)]{GrPeSp:19} instead of \cite[Theorem 2.5]{GrPeSp:19}.
\ere

\bre[Discontinuous $A$ and $n$]\label{rem:jumps}
The requirements on $A$ and $n$ in \cref{def:hh-nontrapping} require them to be continuous (since  $\WoiDR = \CzoDR$ as $\DR$ is Lipschitz; see, e.g., \cite[Section 4.2.3, Theorem 5]{EvGa:92}). In addition to proving deterministic a priori bounds for the class of $A$ and $n$ in \cref{def:hh-nontrapping}, the paper \cite{GrPeSp:19} proves deterministic bounds for discontinuous $A$ and $n$ satisfying \eqref{eq:hh-Acond} in a distributional sense; see \cite[Theorem 2.7]{GrPeSp:19}. In this case, when moving outward from the obstacle to infinity,  $A$ can jump downwards and $n$ can
jump upwards on interfaces that are star-shaped. (When the jumps are in the opposite direction, the problem is trapping; see \cite{PoVo:99a} and \cite[Section 6]{MoSp:19}). The
well-posedness results and a priori bounds in this paper can therefore be adapted to prove results about the stochastic Helmholtz equation
for a class of random $A$ and $n$ that allows nontrapping jumps on randomly-placed star-shaped interfaces.
\ere

\bre[$k$-dependent $A$ and $n$]\label{rem:kdep}
In this paper we focus on random fields $A$ and $n$ varying independently of $k;$ this corresponds to a fixed physical
medium, characterised by $A$ and $n$, with waves of frequency $k$ passing through.
In \cref{sec:generating} below we construct $A$ and $n$ as ($k$-independent) $W^{1,\infty}$ perturbations of random fields $\Az$ and $\nz$ satisfying \cref{con:hh-hetero}.
We note, however, that results for $A$ and $n$ being
\emph{$k$-dependent} $L^\infty$ perturbations (i.e. rougher, but $k$-dependent perturbations) of $\Az$
and $\nz$ satisfying \cref{con:hh-hetero} can easily be obtained.

The basis for these bounds is observing that \emph{deterministic} a priori bounds hold when
(a) $A\in \NTA{\mu_1}$, $n = \nz + \eta,$ where $\nz \in \NTn{\mu_2}$
and $k\NLiDRRR{\eta}$ is sufficiently small, and
(b) $A=\Az+B$, $n=\nz+\eta$, where $\Az\in \NTA{\mu_1}$, $\nz \in
\NTn{\mu_2}$, $k\NLiDRRR{\eta}$ and $k \NWoiDRRRdtd{B}$ are
both sufficiently small, and $A, n,$ and $D_-$ are such that $u\in
H^2(D_R)$ (see, e.g., \cite[Theorem 4.18(i)]{Mc:00} or \cite[Theorems
2.3.3.2 and 2.4.2.5]{Gr:85} for these latter requirements). Given these deterministic bounds, the general arguments in this paper
can then be used to prove well-posedness of the analogous stochastic
problems.

To understand why bounds hold in the case (a), observe that one can
write the PDE as
\beq\label{eq:pert}
\nabla \cdot(A\grad u) +k^2 n_0 u = -f - k^2 \eta u;
\eeq
if $k\NLiDRRR{\eta}$ is sufficiently small then the contribution from
the $k^2 \eta u$ term on the right-hand side of \eqref{eq:pert} can be
absorbed into the $k^2\|u\|^2_{L^2(D_R)}$ term appearing on the left-hand
side of the bound (the deterministic analogue of \eqref{eq:Sbound1}).
In the case $\nz=1,$ this is essentially the argument used to prove the
a priori bound in \cite[Theorem 2.4]{FeLiLo:15} (see \cite[Remark
2.15]{GrPeSp:19}).
The reason bounds hold in the case (b) is similar, except now we need
the $H^2$ norm of $u$ on the left-hand side of the bound (as well as
the $H^1$ norm) to absorb the contribution from the $\nabla \cdot
(B\grad u)$ term on the right-hand side.
\ere
