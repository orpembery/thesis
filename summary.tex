We study the Helmholtz equation with heterogeneous and random coefficients, developing rigorous theory for the Helmholtz equation with random coefficients and devising and studying algorithms to speed up the approximation of its solution. The Helmholtz equation is the simplest possible model of wave propagation, describing time-harmonic solutions of the wave equation.

We first study $h$-finite-element approximations of the (deterministic) heterogeneous Helmholtz equation. We provide a survey of all of the literature on error bounds for the $h$-finite element method for the Helmholtz equation, before comparing and contrasting the different proof techniques used to prove these error bounds. We then prove the first error bounds for the higher-order $h$-finite-element method for the Helmholtz equation in heterogeneous media.

We then move on to the stochastic Helmholtz equation, i.e., the Helmholtz equation with random field coefficients. We prove the first $k$-independent well-posedness results for the stochastic Helmholtz equation. (I.e. we prove existence and uniqueness of a solution, and we prove a $k$-explicit bound on the solution, under suitable $k$-independent assumptions.) To prove this well posedness, we develop general, abstract arguments that allow us to prove well-posedness for three different formulations of stochastic PDEs; these arguments can be applied more widely than just to the stochastic Helmholtz equation.

We introduce nearby preconditioning, a computational technique for speeding up solving many linear systems arising from discretisations of realisations of the stochastic Helmholtz equation. This speedup is achieved by reusing the preconditioner corresponding to one realisation for many other realisations. We prove rigorous results on when nearby preconditioning is effective, and investigate this effectiveness numerically in a range of situations.

We then combine nearby preconditioning with a Quasi-Monte-Carlo method (i.e., a high-dimensional integration rule) to compute quantities of interest of the stochastic Helmholtz equation. We see that nearby preconditioning offers a significant computational saving, with 98\% of linear-system solves being made with a previously-calculated preconditioner. We also provide some preliminary computational evidence that, when using QMC methods for the Helmholtz equation, the total number of realisations used must increase with $k.$

Finally, we study the Multi-Level Monte-Carlo method for the Helmholtz equation, where one reduces the variance in the UQ calculation  by solving the underlying PDE on a hierarchy of meshes. We prove rigorous frequency-explicit bounds on the computational cost of Monte Carlo and Multi-Level Monte-Carlo methods for the Helmholtz equation, and we see that the Multi-Level Monte-Carlo method gives theoretical speedup over the Monte-Carlo Method. This result requires that we generalise the standard Multi-Level Monte-Carlo convergence theory to the $k$-dependent case.
