We study the Helmholtz equation with heterogeneous and random coefficients, developing theory for the Helmholtz equation with random coefficients and devising and studying algorithms to speed up the approximation of its solution. The Helmholtz equation is the simplest possible model of wave propagation, describing time-harmonic solutions of the wave equation.

We study $h$-finite-element approximations of the (deterministic) heterogeneous Helmholtz equation. We provide a survey of all of the literature on error bounds for the $h$-finite element method for the Helmholtz equation, before comparing and contrasting the different proofs techniques used to prove these error bounds. We then prove the first error bounds for the higher-order $h$-finite-element method for the Helmholtz equation in heterogeneous media.

We then move on to the stochastic Helmholtz equation, i.e., the Helmholtz equation with random field coefficients. We prove the first $k$-independent well-posedness results for the stochastic Helmholtz equation. (I.e. we prove existence and uniqueness of a solution, and we prove a $k$-explicit bound on the solution, under suitable $k$-independent assumptions.) To prove this well posedness, we develop general, abstract arguments that allow us to prove well-posedness for three different formulations of stochastic PDEs; these arguments can be applied more widely than just to stochastic Helmholtz equation.

We introduce a technique (nearby preconditioning) for speeding up the solution of many linear systems arising from $h$-finite-element discretisations of realisations of the stochastic Helmholtz equation. This speedup is achieved by reusing the preconditioner corresponding for one realisation for many other realisations. We prove rigorous bounds on when this preconditioning technique is effective, and investigate this effectiveness numerically in a range of situations. We then apply nearby preconditioning with a Quasi-Monte-Carlo method to compute quantities of interest of the stochastic Helmholtz equation. We see than nearby preconditioning offers a significant computational saving, with 98\% of linear-system solves being made with a precviously-calculated preconditioner. We also provide some preliminary computational evidence that, when using QMC methods for the Helmholtz equation, one must take the total number of realisations of the Helmholtz equation to scale with $k.$
