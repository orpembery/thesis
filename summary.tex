\abstract{The Helmholtz equation is the simplest possible model of wave propagation, describing time-harmonic solutions of the wave equation. We study the Helmholtz equation with heterogeneous and random coefficients, corresponding to wave propagation through a medium with spatially variable and random physical characteristics.

In \cref{chap:background} we study $h$-finite-element approximations of the (deterministic) heterogeneous Helmholtz equation. We prove the first error bounds (explicit in the frequency) for the higher-order $h$-finite-element method for the Helmholtz equation in heterogeneous media.

In \cref{chap:stochastic} we move on to the stochastic Helmholtz equation, i.e., the Helmholtz equation with random field coefficients. We prove the first frequency-independent well-posedness results for the stochastic Helmholtz equation. (I.e. we prove existence and uniqueness of a solution, and we prove a frequency-explicit bound on the solution, under a suitable assumption on the coefficients that corresponds to the problem being `nontrapping' almost surely.) To prove this well posedness, we develop general, abstract arguments that allow us to prove well-posedness for three different formulations of stochastic PDEs; these arguments can be applied more widely than just to the stochastic Helmholtz equation.

In \cref{chap:nbpc} we study nearby preconditioning, a computational technique for speeding up solving many linear systems arising from discretisations of realisations of the stochastic Helm\-holtz equation. This speedup is achieved by reusing the preconditioner corresponding to one realisation for many other realisations. We prove rigorous results on when nearby preconditioning is effective, and investigate this effectiveness numerically in a range of situations.

Also in \cref{chap:nbpc}, we combine nearby preconditioning with a Quasi-Monte-Carlo method (i.e., a high-dimensional integration rule) to compute quantities of interest of the stochastic Helmholtz equation. We see that nearby preconditioning offers a significant computational saving, with 98\% of linear-system solves being made with a previously-calculated preconditioner. As a by-product of these results we also provide some preliminary computational evidence, of independent interest, that when using QMC methods for the Helmholtz equation the total number of realisations used must increase with the frequency.

Finally, in \cref{chap:mlmc}, we study the Multi-Level Monte-Carlo method for the Helmholtz equation, where one reduces the variance in an Uncertainty Quantification calculation  by solving the underlying PDE on a hierarchy of meshes. We generalise the standard Multi-Level Monte-Carlo convergence theory to the frequency-dependent case and prove rigorous frequency-explicit bounds on the computational cost of Monte Carlo and Multi-Level Monte-Carlo methods for the Helm\-holtz equation. We see that the Multi-Level Monte-Carlo method gives theoretical speedup over the Monte-Carlo Method.
}\optodo{edit here}
