\optodo{REDO}
We now apply \cref{thm:mlmccomp} to the (heterogeneous) Helmholtz equation. We first need to ascertain what the quantities $\alpha, \beta, \gamma, \sigma, \tau$ and $\rho$ are.Throughout, $D$ will be the spatial domain.

%\bre[Relationship between $\Ml$ and $\hl$]
%Recall that $\Ml$ denotes the number of DoFs associated with the discretisation on level $l.$ Therefore if $\hl$ denotes the mesh size on level $l,$ we have 
%\beqs
%\Ml = \CMh^d \hl^{-d},
%\eeqs
%for some constant $\CMh$ (i.e., halving the mesh size increases the number of DoFs by a factor $2^d)$. It follows that 
%\beqs
%\hl = \CMh \Ml^{-1/d},
%\eeqs
%and thus $\hl = s^{1/d} \hlmo.$ I.e., if we perform uniform refinement, in 2-D, $s=4.$ For uniform refinement in 3-D, $s=8.$ We'll use the relationship between $\Ml$ and $\hl$ a lot, as it's more intutive to work with $\hl$ than $\Ml$. (Alternatively, we could recast \cref{thm:mlmccomp} in terms of $\hl.$)
%\ere
%
%Throughout, we'll use the following assumption:
%
%\bas[At least fixed points-per-wavelength, and error term]\label{ass:ppwfem}
%There exist constants $\Cppw$ and $\CFEM > 0$, independent of all the parameters of interest, such that
%\beqs
%\hl \leq \frac{\Cppw}k
%\eeqs
%and
%\beq\label{eq:femerror}
%\NW{u-\uhl} \leq \CFEM \hl^2 k^3
%\eeq
%for all $l \geq 0,$ where $u$ is the true solution of the Helmholtz equation, and $\uhl$ is the solution of the piecewise-linear continuous finite-element approximation of the Helmholtz equation on level $l.$
%\eas
%
%\bre[How to show \cref{ass:a,ass:b}]
%For the \emph{heterogeneous} Helmholtz equation, showing \eqref{eq:femerror} is currently an open question, although we might be able to do it using the argument in \cite{ChNi:18}. In general, $\CFEM$ will depend on any forcing functions in the Helmholtz equation, and on the coefficients, but we don't worry about that here.
%
%Of course, because we'll be dealing with the Helmholtz equation with random coefficients, $\CFEM$ will, in general, be a random variable. We won't worry about its precise properties for now, we only assume that $\CFEM$ is `nice enough' that we can take expectations of all the quantities we need to. (Nontrapping almost-surely would, I think, give a `nice enough' $\CFEM$.)
%\ere

We now show \cref{ass:a,ass:b,ass:c} hold, and we will calculate the values of $\alpha, \beta, \gamma, \sigma, \tau,$ and $\rho$.

We assume throughout that the FEM error for the weighted $H^1$ norm decays like $h^2k^3,$ and the the FEM error for the $L^2$ norm decays like $h^2k^2.$

%\subsection{Verifying the assumptions of MLMC}

%Throughout, we will take
%\beqs
%\hz = \frac{\Cppw}k,
%\eeqs
%i.e., our coarsest mesh corresponds to a fixed number of points per wavelength. This corresponds to
%\beqs
%\Mz = \frac{\CMh^d}{\Cppw^d} k^d,
%\eeqs
%that is, $\rho = d$ in \cref{cor:mlmccomphh}.

For now, we consider two different quantities of interest $Q$: $Q = \NW{u} $ and $Q = \NLtD{u}$.% and $Q = \NLtGI{\uinfty},$ where $\uinfty$ is an approximation on the impedance boundary $\GI$ of the far-field pattern. To elaborate on the construction of $\uinfty,$ we have that the scattered field $\uS$ satisfies
%\beqs
%\uS(\bx) = \frac{e^{ikr}}{r}\mleft(\uinfty + O\mleft(\frac1r\mright)\mright),
%\eeqs
%for the \emph{true} far-field pattern $\uinfty$ hence if we assume $r \approx 1$ on the impedance boundary, we form an approximation of the far-field pattern on the impedance boundary by taking
%\beqs
%\uinfty \approx e^{-ik} \uS = e^{-ik}\mleft(u-\uI\mright).
%\eeqs
%We'll see that $\uI$ doesn't come into the calculations we do below.

\subsection{Verifying \cref{ass:a,ass:b} for $Q = \NW{u}$}\label{sec:abweight}
To verify \cref{ass:a}, we have
\begin{align*}
\abs{\EXP{\Qhl(\omega) - Q}} &\leq \EXP{\abs{\NW{\uhl} - \NW{u}}}\\
&\leq \EXP{\NW{\uhl - u}}\\
&\leq \CFEM \hl^2 k^3
\end{align*}
so $\sigma = 3$ and $\alpha=2.$

To verify \cref{ass:b}, we have 
\begin{align*}
\VAR{\Yl} &= \EXP{\Yl^2} - \EXP{\Yl}^2\\
&\leq \EXP{\Yl^2}\\
&= \EXP{\abs{\Qhl-\Qhlmo}^2}\\
&= \EXP{\abs{\NW{\uhl}-\NW{\uhlmo}}^2}\\
&\leq \EXP{\NW{\uhl-\uhlmo}^2}\\
&\leq 2\EXP{\NW{\uhl-u}^2 + \NW{\uhlmo-u}^2}\\
&\leq 2\CFEM^2 k^6 \mleft(\hl^4 + \hlmo^4\mright)\\
&= 2\CFEM^2 k^6 \mleft(\hl^4 + s^{4d}\hl^4\mright)\\
&= 2\CFEM^2\mleft(1+s^{4d}\mright) k^6 \hl^4
\end{align*}
so $\tau = 6$ and $\beta = 4.$

\subsection{Verifying \cref{ass:a,ass:b} for $Q = \NLtD{u}$}\label{sec:abltwo}

This proceeds similarly to the calculations for $\NW{u},$ except we use the error estimate
\beqs
\NLtD{\uhl -u} \leq \mleft(\hl k\mright)^2.
\eeqs
Proceeding as above, we find $\sigma = 2,$ $\alpha = 2,$ $\tau = 4$, and $\beta = 4.$

%\subsection{Verifying Assumptions \ref{ass:a} and \ref{ass:b} for $Q = \NLtGI{\uinfty}$}cref!!!
%
%To come.

\subsection{Verifying \cref{ass:c} for the Helmholtz equation}

\Cref{ass:c} is asking what the cost is of solving one realisation of the Helmholtz equation on a mesh with size $\hl.$

My knowledge here is currently not good enough to give a complete answer, so in the analysis that follows, I'll look at two situations:

\ben[(i)]
\item\label{it:solverlu} $\cth$ independent of all parameters of interest, and $\gamma = 1.5d,$ and
\item\label{it:solverlin} $\cth$ independent of all parameters of interest, and $\gamma= d.$

Situation \ref{it:solverlu} is the case where for each realisation we use a sparse direct solver (I think I've got this right, although my knowledge of solver complexities isn't very deep - this is taken from \cite[Section 4]{ClGiScTe:11}).
Situation \ref{it:solverlin} is the (theoretical) case where for each realisation we use a $\cO(\Ml)$ solver for the Helmholtz equation (discretised with mesh size $\hl$)---i.e., we have an `$\cO(N)$' solver for the Helmholtz equation. This is included so that we can see what the `best case' is.
\een

\subsection{MLMC complexity for $Q=\NW{u}$}

We have $\rho=0,$ $\sigma = 3, \alpha = 2, \tau = 6,$ and $\beta = 4$.

\subsubsection{In 2-D}

\paragraph{Direct solver}

We have $\gamma = 3$. We check that  $\beta \neq \gamma.$ Therefore we are in case \eqref{eq:mlmchhoth}, and the computational complexity satisfies
\beqs
\CMLhL(\eps) \lesssim k^{4.5} \eps^{-1.5}.
\eeqs

\paragraph{`Ideal' solver}

We have $\gamma = 2$. We check that  $\beta \neq \gamma.$ Therefore we are in case \eqref{eq:mlmchhoth}, and the computational complexity satisfies
\beqs
\CMLhL(\eps) \lesssim k^3 \eps^{-1}.
\eeqs

\subsubsection{In 3-D}

\paragraph{Direct solver}

We have $\gamma = 4.5$. We check that  $\beta \neq \gamma.$ Therefore we are in case \eqref{eq:mlmchhoth}, and the computational complexity satisfies
\beqs
\CMLhL(\eps) \lesssim k^{6.75} \eps^{-2.25}.
\eeqs

\paragraph{`Ideal' solver}

We have $\gamma = 3$. We check that  $\beta \neq \gamma.$ Therefore we are in case \eqref{eq:mlmchhoth}, and the computational complexity satisfies
\beqs
\CMLhL(\eps) \lesssim k^{4.5} \eps^{-1.5}.
\eeqs

\subsection{MLMC complexity for $Q=\NLtD{u}$}

We have $\rho=0,$ $\sigma = 2, \alpha = 2, \tau = 4,$ and $\beta = 4$.

\subsubsection{In 2-D}

\paragraph{Direct solver}

We have $\gamma = 3$. We check that  $\beta \neq \gamma.$ Therefore we are in case \eqref{eq:mlmchhoth}, and the computational complexity satisfies
\beqs
\CMLhL(\eps) \lesssim k^3 \eps^{-1.5}.
\eeqs

\paragraph{`Ideal' solver}

We have $\gamma = 2$. We check that  $\beta \neq \gamma.$ Therefore we are in case \eqref{eq:mlmchhoth}, and the computational complexity satisfies
\beqs
\CMLhL(\eps) \lesssim k^2 \eps^{-1}.
\eeqs

\subsubsection{In 3-D}

\paragraph{Direct solver}

We have $\gamma = 4.5$. We check that  $\beta \neq \gamma.$ Therefore we are in case \eqref{eq:mlmchhoth}, and the computational complexity satisfies
\beqs
\CMLhL(\eps) \lesssim k^{4.5} \eps^{-2.25}.
\eeqs

\paragraph{`Ideal' solver}

We have $\gamma = 3$. We check that  $\beta \neq \gamma.$ Therefore we are in case \eqref{eq:mlmchhoth}, and the computational complexity satisfies
\beqs
\CMLhL(\eps) \lesssim k^3 \eps^{-1.5}.
\eeqs