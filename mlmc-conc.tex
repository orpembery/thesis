\section{Summary and future work}

\subsection{Summary}
In this \lcnamecref{chap:mlmc} we analysed the computational cost of Monte-Carlo (MC) and Multi-Level Monte-Carlo (MLMC) methods for the Helmholtz equation. In particular:
\bit
\item In \cref{sec:mlmcsetup,sec:mc,sec:mlmcan} we adapted the standard Monte-Carlo and Multi-Level Monte-Carlo complexity theory to the $k$-dependent case.
  \item In \cref{sec:mlmcapplying} we the applied the adapted theory to two Quantities of interest (QoIs), under two different assumptions on the behaviour of the underlying finite-element method, and saw that MLMC is consistently cheaper than MC, with respect to the required tolerance $\eps.$
\eit

\subsection{Future work}
There are several immediate possibilities for building on the work in this \lcnamecref{chap:mlmc}:
\bit
\item Applying the adapted theory to other, more physically realistic QoIs, e.g., the far-field pattern of $u$ (see, e.g., \cite[Section 2.5]{CoKr:13}).
\item Performing numerical experiments to investigate if the predicted speedup of MLMC methods over MC methods is obtained in practice.
  \item Investigating extensions of MLMC methods, as has already been done for the stationary diffusion equation, e.g., Multi-Level Quasi-Monte-Carlo methods, see, e.g., \cite{KuScScSlUl:17} and Multi-Level Markov-Chain Monte-Carlo methods \cite{DoKeScTe:15}.
\eit
