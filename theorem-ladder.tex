\begin{figure}[h]
  \centering
  \scalebox{0.90}{
\begin{tikzpicture}

% Inspired by https://tex.stackexchange.com/questions/109102/anchor-arrow-start-position

\draw (0,1) node [rounded rectangle, fill=gray!45!white] (meas) {\Cref{prob:meas}};



\draw (0,-1) node [rounded rectangle, fill=gray!45!white] (lt) {\Cref{prob:lt}};


\draw (0,-3) node [rounded rectangle, fill=gray!45!white] (svar) {\Cref{prob:svar}};


\path node [left = \owenshift of meas.north] (meas top left) {};
\path node [right = \owenshift of meas.north] (meas top right) {};
\path node [left = \owenshift of meas.south] (meas bottom left) {};
\path node [right = \owenshift of meas.south] (meas bottom right) {};

\path node [left = \owenshift of lt.north] (lt top left) {};
\path node [right = \owenshift of lt.north] (lt top right) {};
\path node [left = \owenshift of lt.south] (lt bottom left) {};
\path node [right = \owenshift of lt.south] (lt bottom right) {};

\path node [left = \owenshift of svar.north] (svar top left) {};
\path node [right = \owenshift of svar.north] (svar top right) {};




% Arrows
\begin{scope}[->]


\draw
(meas bottom right)
--
node[right,align=center,text width=5cm] {Under \cref{con:B}, get stochastic a priori bound \eqref{eq:sbresult} (\Cref{thm:3})}
 (lt top right);

\draw
(lt top left)
--
node[left] {Immediate}
 (meas bottom left);

\draw
(lt bottom right)
--
node[right,align=center,text width=5cm] {Under \cref{con:coeffstofunc,,con:L,,con:cborel,,con:C},  (\Cref{thm:11})}
 (svar top right);

\draw
(svar top left)
--
node[left,align=center,text width=5cm] {If \cref{prob:svar} is well-defined (\cref{thm:12})}
(lt bottom left);
\end{scope}

\path node [align=center,below = \owentextshift of svar,text width=10cm] (svar wd) {Well-defined under  \cref{con:coeffstoform,,con:A,,con:coeffstofunc,,con:L,,con:cborel,,con:C} (\Cref{lem:svarwelldefined})};


\end{tikzpicture}
}
\caption[The relationship between the different variational formulations of stochastics PDEs]{The relationship between the variational formulations. An arrow from Problem P to Problem Q with Conditions R indicates `under Conditions R, the solution of Problem P is a solution of Problem Q'}\label{fig:ladder}
\end{figure}
