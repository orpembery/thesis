We now prove our new error bounds for the higher-order finite-element approximation of the solution of the Helmholtz equation in heterogeneous media. Our results are a generalisation of results proved by Du and Wu \cite{DuWu:15} for higher-order finite-element approximations of the Helmholtz equation in homogeneous media; our results and proofs broadly follow those in \cite{DuWu:15}, with the differences that: (i) we modify the proofs to cater for the heterogeneity of the coefficients, and (ii) the dependence of our results on all of the constants involved is explicit. In particular our results are (in principle) explicit in $A$  and $p$ and are explicit in $n$ and $k$.

The proofs of our results have many parts, and appear technical, largely due to the burden of explicitly keeping track of all of the constants involved. However, in essence, the proof consists of two ideas:
\ben
\item Decompose the error $u-\uh = (u - \Ph u) + (\Ph u - \uh)$, where $\Ph u$ is an `elliptic projection' of $u$, and
\item Bound the error $\Ph u - \uh$ in higher-order `discrete' norms.
\een

In order to use this notion of discrete norms, we develop a notion of discrete derivatives. Also, a number of different Galerkin projections (including the `elliptic projection' mentioned above) play a key role in the proofs, and so we prove error bounds for these projections. Our results also use best approximation results for the Helmholtz equation, and we prove these, following the techniques in \cite{ChNi:18a}, but ensuring that these results are explicit in all of the constants involved.
\optodo{Put in structure summary once it's settled down.}
As we are using higher-order finite elements (which we assume are of degree $p$), we will require extra smoothness assumptions on $A$, $n,$ and the boundaries $\GD$ and $\GI$. We also make simple assumptions on $k,$ $\nmin$, and $\NLiD{n}$ in order to simplify our calculations.
\optodo{Correct notation in variational formulation of TEDP}

\bas[Assumptions for higher-$p$ FEM bounds]\label{ass:highp}
Assume
\bit
\item $\GD$ and $\GI$ are $\Cpo$
\item $\Aij \in \CpmooDclos$ for all $i,j$, and
\item $n \in \LiD \cap \HmD \cap \CfloordtD.$
%\item $f \in \HpmoD$
%  \item $\gI \in \HpmhGI$
  \eit
\eas

Also, throughout this \lcnamecref{sec:fem} we assume we are working with the finite-element space defined in \cref{def:fespace}, and that consequently \cref{lem:scottzhang} holds.

We make the following \lcnamecref{ass:htwo} on the solution of \cref{prob:vtedp}.

\bas\label{ass:htwo}
\Cref{prob:vtedp} (or its adjoint) has a unique solution $u$ in $\HtD$, and there exists $\CAnk>0$ (possibly dependent on $A$, $n,$ and $k$) such that
\beq\label{eq:generalhtwo}
k \NLtD{u} + \SNHoD{u} + \frac1k \SNHtD{u} \leq \CAnk \Cfg,
\eeq
where $\Cfg \de \NLtD{f} + \NLtGI{\gI}$.
\eas

Finally, we make the following \lcnamecref{ass:convenient} to simplify the proofs in this section. \Cref{ass:convenient} is by no means necessary, but greatly simplifies the proofs.

\bas[Assumptions for convenience of proofs]\label{ass:convenient}
Assume: $k \geq 1$, $\NLiD{n} \geq 1,$ $\nmin \leq 1$, $hk \leq 1$, and there exists $\Ctildemin > 0$ independent of $k$ such that $\CAnk k \geq \Ctildemin.$
\eas
%Note that one could require $\NHhGI{\gI}$ on the right-hand side of \cref{eq:generalhtwo} (as $\gI \in \HhGI$); however, since the bound in \cref{thm:tedp} only has $\NLtGI{\gI}$, we use the form \cref{eq:generalhtwo} to include \cref{thm:tedp}.

We define the following quantities, that we use to simplify the expressions involving $n$:

\beqs
\nvar = \frac{\NLiD{n}}{\nmin},
\eeqs
and
\beqs
\En = \max\set{\frac{\NHsD{n}}{\nmin^2} \st s \in (d/2,p-1];\frac{\NWsiD{n}}{\nmin^2}\st s \in [2,d/2];1}\nvar.
\eeqs
The quantity $\En$ arises from bounds on the `elliptic projection' mentioned above in weighted norms, see \cref{lem:ellprojerrw} below.

Our main \lcnamecref{thm:fembound} and \lcnamecref{cor:fembound} are the following:
\optodo{Put somewhere that $\Condn(n) = \Pfn{p-2}\mleft(\NLiD{n}\mright)^{-\frac1{p+1}}\mleft(\mleft(\En\nvar\mright)^{\half(\floor{\frac{p-1}2}+1)}\NLiD{n}\nmin^{-\frac{p}2}\mright)^{-\frac{p-1}{2p}}$.}
\bth[Higher-order error bound for the heterogeneous Helmholtz equation]\label{thm:fembound}
Under \cref{ass:highp,ass:htwo}, let $u$ be the solution of \cref{prob:vtedp}. Under \cref{ass:highp,ass:htwo,ass:convenient}, there exist constants $\CFEMLt, \CFEMHo, \Chcond > 0$, independent of $h$, $k$ and $n$, and a function $\Condn(n) >0$ (independent of $h$ and $k$) such that if
\beq\label{eq:hfemcond}
h \leq \Chcond \Condn(n) \CAnk^{-\frac1{2p}}k^{-1-\frac1{2p}},
\eeq
then the finite-element solution $\uh$ exists, is unique, and satisfies the error bounds
\beq
\NLtD{u-\uh} \leq \CFEMLt\CLtn(n) \mleft(h +  \CAnk (hk)^{p}\mright)\inf_{\vh \in \Vhp} \NW{u-\vh},\label{eq:femltbound}
\eeq
and
\beq
\NW{u-\uh} \leq \CFEMHo \CHon(n)\mleft(1 + \CAnk k(kh)^p\mright) \inf_{\vh \in \Vhp} \NW{u-\vh}.\label{eq:femhobound}
\eeq
\enth
\optodo{Make a Corollary about QO, and say this is just what Theo and Serge have, but with explicitness in the coefficients.}
\optodo{$\CLtn = \nvar^6\nmin^{-(p+1)} \mleft(\En \nvar\mright)^{\floor{\frac{p-1}2}+1}\Pfn{p-2}\mleft(\NLiD{n}\mright)$}
\optodo{$\CHon = \mleft(1+\mleft(\En\nvar\mright)^{\half(\floor{\frac{p-1}2}+1)}\NLiD{n}\nmin^{-\frac{p}2}\mright)\CLtn$}
\bco[Higher-order-finite-element bound]\label{cor:fembound}
Under the assumptions of \cref{thm:fembound},
\beqs
\NLtD{u-\uh} \leq \CcorLt \CLtn(n)\mleft(h^2 + \CAnk^2 (hk)^{2p}\mright)\Cfg\tand
\eeqs
\beqs
\NW{u-\uh} \leq \CcorHo \CHon(n)\mleft(h + \CAnk (hk)^p + \CAnk^2k (hk)^{2p}\mright)\Cfg.
\eeqs
\eco



Whilst the calculations in this \lcnamecref{sec:fem} are explicit in all the constants involved, these dependencies are complex and, to a certain extent, unnecessary to understand the arguments. Therefore, the definition of all the constants (which are many-layered and interdependent) are relegated to \cref{app:constants}; that is, any constant introduced or defined in this \lcnamecref{sec:fem} will be listed in \cref{app:constants}.

\subsection{Decomposition of solution and best approximation bound}

For the first part of the proof of \cref{thm:fembound}, we prove a best approximation bound in $\Vhp$ for the solution of the Helmholtz equation; following the presentation in \cite{ChNi:18a} (although we explicitly keep track of the constants involved at each point). In order to obtain bounds for high $p$, we require the following shift \lcnamecref{thm:shift}:

\bth[Shift theorem]\label{thm:shift}
Under \cref{ass:highp}, for all integers $l \in \mleft[0,p-1\mright]$ there exists a constant $\CAl>0$ (depending on $A$) such that if $\ftilde \in \HlD$ and $\gItilde \in \HlphGI$, then there exists a unique $\utilde \in \HlptD$ such that $\utilde$ solves
\beq\label{eq:shifteq}
\grad \cdot \mleft(A \grad \utilde\mright) = -\ftilde,
\eeq
\beq\label{eq:shiftdbc}
\dn \utilde = \gItilde, \tand
\eeq
\beq\label{eq:shiftnbc}
\trD \utilde = 0
\eeq
and $\utilde$ satisfies the bound
\beq\label{eq:shift}
\NHlptD{\utilde} \leq \CAl \mleft(\NHlD{\ftilde} + \NHlphGI{\gItilde}\mright).
\eeq
\enth

\bpf[Proof of \cref{thm:shift}]
The uniqueness and existence of $\utilde$ (in $\HozDD$) follows from the Lax--Milgram theorem, as the variational formulation of \cref{eq:shifteq,eq:shiftdbc,eq:shiftnbc} is bounded and coercive. The proof for the higher regularity bounds uses standard elliptic regularity estimates in the interior and near the boundaries $\GD$ and $\GI$, and the work of the proof is combining these estimates. As a reference for these estimates we use \cite[pp. 137-138]{Mc:00}; \cref{ass:highp} means we can apply these results, as we have higher regularity of the boundaries $\GD$ and $\GI$ and the coefficient $A$.

To deal with the interior regularity and regularity near the boundary separately, we define the following subsets of $D$: $\Dint,\Dinttilde,\Dscat,$ and $\Dtrunc$\optodo{Put this in a picture - sketch is in Research 17, notes for 17th May} with the following properties:
\bit
\item $\Dint \compcont \Dinttilde \compcont D$,
\item $\GD \subset \Dscatclos$
\item $\dist(\Dscat,\GI) > 0 $
  \item $\GI \subset \Dtruncclos$
\item $\dist(\Dtrunc,\GD) > 0 $
    \eit
    First applying interior regularity \cite[Theorem 4.16]{Mc:00} in $\Dinttilde,$ we obtain the bound
    \beq\label{eq:shiftint}
\NHlptDint{\utilde} \leq \CintAl \mleft(\NHoDinttilde{\utilde} + \NHlDinttilde{\ftilde}\mright).
\eeq
Applying regularity up to the boundary for Dirichlet data \cite[Theorem 4.18 (i)]{Mc:00} in $\Dscat,$ we obtain (as $\trD \utilde = 0$)
\beq\label{eq:shiftscat}
\NHlptDscat{\utilde} \leq \CscatAl \mleft(\NHoD{\utilde} + \NHlD{\ftilde}\mright)
\eeq
and similarly\optodo{Spell checker} for Neumann data \cite[Theorem 4.18 (ii)]{Mc:00} in $\Dtrunc,$ we obtain
\beq\label{eq:shifttrunc}
\NHlptDtrunc{\utilde} \leq \CtruncAl \mleft(\NHoD{\utilde} + \NHlphGI{\dn \utilde} + \NHlD{\ftilde}\mright).
\eeq
Combining \cref{eq:shiftint,eq:shiftscat,eq:shifttrunc}, we obtain \cref{eq:shift}.
\epf




%% The following trace theorem is standard, see, e.g., \cite[Theorem 3.37]{Mc:00}.


The following \lcnamecref{lem:domainshift} follows from \cref{thm:shift}.

\bco\label{lem:domainshift}
Under \cref{ass:highp}, let $\ftilde \in \HlD$ and $\gtilde \in \HlpoD,$ for $0 \leq l \leq p-1$. If $\utilde \in \HlptD$ solves
\beqs
\grad \cdot \mleft(A\grad \utilde\mright) = -\ftilde,
\eeqs
\beqs \trGD u = 0,
\eeqs
and
\beqs
\dn u = \gtilde,
\eeqs
then
\beqs
\NHlptD{\utilde} \leq \CAl\mleft(1+\CTrlpo\mright)\mleft(\NHlD{\ftilde} + \NHlpoD{\gtilde}\mright).
\eeqs
\eco

The proof of \cref{lem:domainshift} requires the Trace theorem.

\bth[Trace Theorem]\label{thm:trace}
If $v \in \HsD,$ for $1/2 < s \leq p+1$, then there exists $\CTrs > 0$ independent of $v$ such that
\beqs
\NHsmhGI{\trI v} \leq \CTrs \NHsD{v}.
\eeqs
\enth


\bpf[Proof of \cref{lem:domainshift}]
By \cref{thm:shift,thm:trace}
\beqs
\NHlptD{\utilde} \leq \CAl \mleft(\NHlD{\ftilde} + \NHlphGI{\gtilde}\mright) \leq \CAl \mleft(\NHlD{\ftilde} + \CTrs\NHlpoD{\gtilde}\mright),
\eeqs
and the result follows.
\epf

We are now able to prove the following \lcnamecref{thm:expansion} giving a decomposition of the solution $u$ of \cref{prob:vtedp} into lower-order, less oscillatory parts and a smoother, more-oscillatory part. 

\bth[Expansion of the solution of the Helmholtz equation]\label{thm:expansion}
Under \cref{ass:highp,ass:htwo} there exists $\uosc \in \HppoD$ and a sequence $\usj \in \HjptD,$ $j = 0,\ldots,p-2$ such that if $u$ is the solution of \cref{prob:tedp} or its adjoint, then
\beq\label{eq:expansionuj}
\NHjptD{\usj} \leq \Cej \Pj(\NLiD{n}) k^j \Cfg, \text{ for }\Cj > 0,
\eeq
\beq\label{eq:expansionuosc}
\NHppoD{\uosc} \leq \Cosc \CAnk k^p \Cfg, \text{ for some } \Cosc > 0,
\eeq
and
\beq\label{eq:expansionid}
u = \uosc + \sum_{j=0}^{p-2} \usj.
\eeq
Where
\beq\label{eq:p}
\Pj(x) =
\begin{dcases}
1 & j = 0,1\\
x^{\floor{j/2}}& j \geq 2
\end{dcases}
\eeq.
%=\sum_{m=0}^{j-2}\pfn{j,m}x^m $
%% are polynomials of degree
%% \beq\label{eq:polydegree}
%% \begin{dcases}
%% \frac{j}2 & \tif  j \text{ is even}\\
%% \frac{j-1}2 \tif j  \text{ is odd}.
%% \end{dcases}
%% \eeq
%% (except for $j=0,1,$ where $\Pj$ is a polynomial of degree 0) given by the recurrence relation
%% \beq\label{eq:pjdef}
%% \Pj(x) = \CAj \mleft(1+\CTrjpo\mright)\mleft(x\Pfn{j-2}(x) + \Pfn{j-1}(x)\mright).
%% \eeq
% No idea if this coefficient stuf is right
%% \begin{align}
%% \label{eq:p1}\pfn{0,0}&= \CAz\\
%% \label{eq:p2}\pfn{1,0}&  = \CAz\CAo\mleft(1+\CTrt\mright)\\
%% \label{eq:p3}\pfn{j,m} &=
%% \begin{dcases}
%%  0, &\tif m \leq \floor{\frac{j-2}2}\\
%% \CAj\mleft(1+\CTrjpo\mright) \mleft(\pfn{j-2,m-1} + \pfn{j-1,m-1}\mright),&\tif \floor{\frac{j-2}2} < m \leq j-3
%% \end{dcases}\\
%% \label{eq:p4}\pfn{j,j-2}& = \CAj\mleft(1+\CTrjpo\mright) \pfn{j-1,j-3},
%% \end{align}
%% and $\Cosc$ is given by the recurrence relations
%% \begin{align}
%% \label{eq:osc1}\Coscfn{1} &= \CAo\mleft(1+\CTrt\mright),\\
%% \label{eq:osc2}\Coscfn{2} &= \CAt\mleft(1+\CTrth\mright)\mleft(1 + \CAo\mleft(1+\CTrt\mright)\mright)\tand\\
%% \label{eq:osc3}\Coscfn{j} &= \CAfn{j}\mleft(1+\CTrfn{j+1}\mright)\mleft(\Coscfn{j-1} + \Coscfn{j-2}\mright),
%% \end{align}
%% and
%% \beq\label{eq:cosc}
%% \Cosc = \frac{\Cefn{p-1}}{\max\set{1,\CAz}}.
%% \eeq
\enth

Recall that a higher power of $k$ appearing in an a priori bound indicates that a function is more oscillatory. In \cref{eq:expansionuj} below, the $(j+2)$th-order norm of $\usj$ is of order $k^j$ (i.e., the power of $k$ is two orders of magnitude less than the order of the norm) whereas in \cref{eq:expansionuosc} the $(p+1)$st-order norm of $\uosc$ is of order $k^p$ (the power of $k$ is one order of mangitude less that the order of the norm). Therefore, in this sense, $\uosc$ is `more oscillatory' than $\usj.$

\Cref{thm:expansion} is essentially just \cite[Theorem 1]{ChNi:18a} in the particular case of a Helmholtz problem, but with the dependence on all the constants kept track of. The results in \cite{ChNi:18a} are stated for a much wider class of problems, but the dependence on all of the constants is not made explicit.

We also point out that \cref{thm:expansion} is reminiscent of results by Melenk and Sauter in \cite{MeSa:10,MeSa:11}who show, for the homogeneous Helmholtz equation that the solution $u$ can be decomposed as $u = \uHt + \uA,$ where $\uHt \in \HtD$ but is not oscillatory ($\NHtD{\uHt} \lesssim 1$), and $\uA \in C^\infty(D),$ but $\uA$ is oscillatory ($\NHsD{\uA} \lesssim k^{s-2}$ for all $s$)\footnote{These results are proved with no obstacle and $f$ given by a Dirac delta function in \cite[Lemma 3.5]{MeSa:10} and for: (i) the IIP with a bounded Lipschitz boundary that is either a 2-D polygon or analytic, or (ii) the EDP with an analytic scatterer, in \cite[Theorems 4.10, 4.20]{MeSa:11} respectively, under an assumption of a polynomial growth of the a priori bound---i.e., $\CAnk$ is a polynomial in $k$.}. This decomposition is used in \cite{MeSa:10,MeSa:11} to prove convergence results for $hp$-finite element methods for the Helmholtz equation. We note that the results in this \lcnamecref{sec:fem} could, in principle, be used in the analysis of $hp$-methods, as it is, in principle, possible to track the dependence of the constants on the polynomial degree $p.$

\bpf[Proof of \cref{thm:expansion}]
The idea of the proof is as follows. We write $u$ as a formal series expansion
\beq\label{eq:formalseries}
u = \sum_{j=0}^\infty \usj,
\eeq
and then substitute this series into the PDE \cref{eq:tedp} and the boundary condition \eqref{eq:ibc}. Equating powers of $k$, we derive a recursive sequence of stationary diffusion equations for the functions $\usj,$ with right-hand sides dependent on $\ujmo$ and $\ujmt$. We use this recursive sequence and \cref{lem:domainshift} to prove the a priori bounds \cref{eq:expansionuj}. We then define the $l$th remainder $\rl = u - \sum_{j=0}^{l-1} \usj,$ and by applying the operator $\grad\cdot\mleft(A\grad \cdot\mright)$ with Neumann boundary conditions to $\rl$, we obtain a recursive sequence for the remainders $\rl,$ and can similarly prove a priori bounds for the $\rl$s. The oscillatory function $\uosc$ is then just $\rpmo.$ The format of this proof is identical to that in \cite[Theorem 1]{ChNi:18a}, except we keep track of all of the constants involved.

For the purposes of the proof, it is more convenient to define $\vj = \usj/k^j,$ so that the series expansion \cref{eq:formalseries} becomes
\beq\label{eq:formalseriesv}
u = \sum_{j=0}^\infty k^j\vj
\eeq
as in \cite{ChNi:18a}. Also, in this proof, all the boundary-value problems involved included a zero Dirichlet condition on the scatterer $\GD;$ we omit this throughout the proof for brevity.

By applying the Helmholtz operator to the formal series \eqref{eq:formalseriesv} we obtain the following problems for $\vj \in \HjptD, j \geq 1$:
\beqs
\grad \cdot \mleft(A\grad \vz\mright) = -f \quad\tand\quad \dn \vz = \gI,
\eeqs
\beqs
\grad \cdot \mleft(A\grad \vo\mright) = 0\quad\tand\quad\dn \vo = i\vz,
\eeqs
and, for $j \in \mleft[2,p-2\mright]$
\beq\label{eq:vj}
\grad \cdot \mleft(A\grad \vj\mright) = - n\vjmt\quad\tand\quad\dn \vz = i\vjmo.
\eeq

By \cref{thm:shift} we immediately conclude the bound
\beq\label{eq:expuz}
\NHtD{\vz} \leq \CAz\Cfg \leq \Cefn{0}\Cfg,
\eeq
and by \cref{lem:domainshift,eq:expuz} we can conclude the bound
\beqs
\NHthD{\vo} \leq \CAo \mleft(1+\CTrt\mright)\NHtD{i\vz} \leq \max\set{1,\CAo}\mleft(1+\CTrfn{2}\mright)\Cefn{0} \Cfg = \Cefn{1}\Cfg.
\eeqs

We prove the bound \cref{eq:expansionuj} by induction; suppose \cref{eq:expansionuj} holds for all $s \in [0,j-1].$ Using \cref{lem:domainshift}, we conclude that (using the observation that $\Cefn{j-1} \geq \Cefn{j-2}$)
\begin{align*}
\NHfn{j+2}{D}{\vj} &\leq \CAj \mleft(1+\CTrjpo\mright)\mleft(\NLiD{n} \NHfn{j}{D}{\vfn{j-2}} + \NHfn{j+1}{D}{\vfn{j-1}}\mright)\\
&\leq \max\set{1,\CAj} \mleft(1+\CTrjpo\mright)\mleft(\NLiD{n} \Cefn{j-2} \Pfn{j-2}\mleft(\NLiD{n}\mright) + \Cefn{j-1} \Pfn{j-1}\mleft(\NLiD{n}\mright)\mright)\Cfg\\
&\leq 2\max\set{1,\CAj} \mleft(1+\CTrjpo\mright)\Cefn{j-1} \Pfn{j}(\NLiD{n})\Cfg\\
&= \Cefn{j} \Pfn{j}(\NLiD{n})\Cfg.
\end{align*}
%% Therefore the definition of the polynomials $\Pj$ \cref{eq:pjdef} holds (and it is straightforward to see that $\deg\mleft(\Pj\mright) = \deg\mleft(\Pfn{j-2}\mright)+1$, where $\deg$ denotes polynomial degree, and therefore since $\deg\mleft(\Pfn{0}\mright) = \deg\mleft(\Pfn{1}\mright)=0,$ \cref{eq:polydegree} holds. Hence, by the relationship between $\usj$ and $\vj$, we have the bound \cref{eq:expansionuj}.

We will now define the remainders $\rl$, and proceed similarly.
Let $\ro \in \HthD$ solve
\beqs
\grad \cdot \mleft(A\grad \ro\mright) = -k^2 u
\eeqs
\beqs
\dn \ro = iku.
\eeqs
Then by \cref{lem:domainshift}
\beqs%gin{align*}
\NHthD{\ro} \leq \CAo\mleft(1+\CTrt\mright)\mleft(k^2\NHoD{u} + k\NHtD{u}\mright)\leq \CAo\mleft(1+\CTrt\mright)k^2\CAnk\Cfg\leq \frac{\Cefn{1}}{\max\set{1,\CAo}}.
\eeqs%nd{align*}
Let $\rt \in \HfD$ solve
\beqs
\grad \cdot \mleft(A\grad \rt\mright) = -k^2 u
\eeqs
\beqs
\dn \rt = ik\ro.
\eeqs
Then by \cref{lem:domainshift}
\begin{align*}
\NHfD{\rt} &\leq \CAt\mleft(1+\CTrth\mright)\mleft(k^2\NHtD{u} + k\NHthD{\ro}\mright)\\
&\leq \CAt\mleft(1+\CTrth\mright)\mleft(1 + \frac{\Cefn{1}}{\max\set{1,\CAo}}\mright)\CAnk k^3\Cfg\leq \frac{\Cefn{2}}{\max\set{1,\CAo}}.
\end{align*}
Then for $j \geq 3,$ let $\rj \in \HjptD$ solve
\beqs
\grad \cdot \mleft(A\grad \rt\mright) = -k^2 \rjmt
\eeqs
\beqs
\dn \rj = ik\rjmo.
\eeqs
And by induction and \cref{lem:domainshift} again, letting $\uosc = \rfn{p-1},$ we have \cref{eq:expansionuosc}. It is straightforward to see that $\rfn{p-1} + \sum_{j=1}^{p-2} \uj$ solves \cref{prob:tedp}, and therefore \cref{eq:expansionid} holds, as $u$ is unique.
\epf
%\{Just a note while I think about it - could you do DtN boundary conditions in this proof by testing with a different function when wanting to bound the $L^2$ norm on the boundary? I wonder if testing with the NtD of $u$ would mean you end up with a $\LtGI{u}^2$ term, and then you could use properties of the NtD map to bound other bits. Might foil $k$-dependency though.}
Using the expansion in \cref{thm:expansion}, we can prove the following error bound for the best approximation of $u$ in $\Vhp$:

%% We also have the following best-approximation error in higer-order Sobolev spaces:
%% \ble[Best approximation in $\HsD$]\label{lem:bestapproxhigh}
%% For integer $s$ in $[1,p+1]$, there exists $\Cinterps>0$ such that for every $v \in \HsD$ there exists $\vhhat \in \Vhp$ such that
%% \beqs
%% \NLtD{v - \vhhat} + h\NHoD{v-\vhhat} \leq \Cinterps h^s \SNHsD{v}.
%% \eeqs
%% \ele

\ble[Best approximation error bound]\label{lem:bestapprox}
Under \cref{ass:highp,ass:htwo}, there exist constants $\CFEMo, \CFEMt > 0$ independent of $k$ and $n$ (although dependent on $A$ and $p$) such that if $u$ solves \cref{prob:vtedp} or its adjoint, then there exists $\uhhat \in \Vhp$ such that
\beq\label{eq:bestapproxL2}
\NLtD{u-\uhhat} \leq \Pfn{p-2}\mleft(\NLiD{n}\mright)\mleft(\CFEMo  h^2 + \CFEMt \CAnk h\mleft(hk\mright)^p \mright)\Cfg,
\eeq
%% \beq\label{eq:bestapproxH1}
%% \NHoD{u-\uhhat} \leq \mleft(\CFEMo h + \CFEMt \CAnk \mleft(hk\mright)^p \mright)\Cfg \tand
%% \eeq
\beq\label{eq:bestapproxW}
\NW{u-\uhhat} \leq 2\Pfn{p-2}\mleft(\NLiD{n}\mright)\mleft(\CFEMo  h + \CFEMt \CAnk \mleft(hk\mright)^p \mright)\Cfg.
\eeq
\ele

We write bounds for the standard and weighted $H^1$ norms separately, as we will use each individual bound in differents of proofs in \cref{sec:fembound}.

\bpf[Proof of \cref{lem:bestapprox}]
We apply \cref{lem:scottzhang} to all the $\usj$ and $\uosc$ in \cref{thm:expansion}, and obtain that there exist $\ujh,$ $j=0,\ldots,p-2$ and $\uosch$ in $\Vhp$ such that 
\beqs
\NLtD{\usj - \ujh} + h\NHoD{\usj - \ujh} \leq \Cinterpfn{j+2} \Cefn{j} \Pj\mleft(\NLiD{n}\mright) h^{j+2}k^j \Cfg
\eeqs
and
\beqs
\NLtD{\uosc - \uosch} + h\NHoD{\uosc - \uosch} \leq \Cinterpfn{p+1} \Cosc\CAnk h^{p+1}k^p \Cfg.
\eeqs
Therefore, by letting $\uhhat = \uosch + \sum_{j=0}^{p-2} \ujh,$ we have \cref{eq:bestapproxL2,eq:bestapproxW} (using the fact that $hk \leq 1$ and, as $\NLiD{n} \geq 1$, $\Pfn{p-2}(\NLiD{n}) \geq \Pfn{j}(\NLiD{n}) \geq 1$ for all $j \leq p-2$).
\epf

\subsection{Routine analysis results}\label{sec:anbackground}
In this \lcnamecref{sec:anbackground} we collect together routine results that we use throughout \cref{sec:fem}.

We first note the following inequality: for $N \in \NN$ and $a_1,\ldots,a_N > 0$
\beq\label{eq:simple}
\sqrt{\sum_{j=1}^N a_j^2} \leq \sum_{j=1}^N a_j.
\eeq



\bth[Multiplicative Trace Inequality]\label{thm:multiplicativetrace}%BS THm 1.6.6
There exists $\CMT > 0$ such that for all $v \in \HoD$
\beqs
\NLtdD{v} \leq \CMT \NLtD{v}^\half \NHoD{D}^\half.
\eeqs
\enth

\ble[Poincar\'e Inequality]\label{lem:poincare}
There exists $\CP > 0$ such that for all $v \in \HozDD$
\beqs
\NLtD{v} \leq \CP \SNHoD{v}.
\eeqs
\ele

\bth[Multiplication in $\HmD$]\label{thm:banachalg}
\ben
\item\label[itempart]{it:ban1}If $m > d/2,$ then for all $\vo, \vt \in \HsD$, $\vo\vt \in \HmD$ and there exists a constant $\CBanfn{s} > 0$ independent of $\vo$ and $\vt$ such that
\beqs
\NHmD{\vo\vt} \leq \CBanfn{m} \NHmD{\vo}\NHmD{\vt}.
\eeqs
\item\label[itempart]{it:ban2}For $m \in \NN,$ if $\vo \in \CmD$, $\supp\mleft(\vo-1\mright) \compcont D,$ and  $\vt \in \HmD,$ then $\vo \in \WmiD,$ $\vo\vt \in \HmD$, and there exists a constant $\Cprod{m} > 0$ independent of $\vo$ and $\vt$ such that%If $\vo \in \CrcompD$ for some $r \in \NN, r \geq m,$ and
\beq\label{eq:ban2}
\NHmD{\vo\vt} \leq \Cprod{m} \mleft(1+\NWmiD{\vo}\mright)\NHmD{\vt}.
\eeq
\een
\enth

\bpf[Proof of \cref{thm:banachalg}]
\Cref{it:ban1} is given in \cite[Section 1.8.1, Theorem]{Ma:11}. For \cref{it:ban2}, observe that as $\vo-1$ has compact support, $\vo \in \WmiD,$ and $\vo-1 \in \CmcompD$. By \cite[Theorem 3.20]{Mc:00}, there exists $\Cmclean{m} > 0$ such that\footnote{In \cite[Theorem 3.20]{Mc:00} $\Cmclean{m}$ is denoted $C_{m}$.}
\beqs
\NHmD{(\vo-1)\vt} \leq \Cmclean{m} \NWmiD{\vo-1}\NHmD{\vt}.
\eeqs
Therefore as $\vo = \vo -1 + 1,$ we have \cref{eq:ban2}.
\epf



\subsection{Error bounds for simple Galerkin projections}\label{sec:errgalerkin}
In this \lcnamecref{sec:errgalerkin} we state a sequence of error bounds in negative Sobolev norms for three different projection operators. The proofs of these error bounds are all simple modifications of the standard proofs of finite-element errors in negative Sobolev norms, as in, e.g., \cite[Theorem 5.8.3]{BrSc:08},  and so we only prove \cref{lem:wltdprojerr,lem:ellprojerrw} below, as these are the proofs that deviate most from the standard proof.
%\ednote{Both-should I write out at least one of these modified proofs?}. % Note to self, I have these proofs sketched out, but just use the argument from Brenner and Scott and use an L^2 inner product on the function, not their gradients, for the elliptic projection.
We first define the projections we use.

Define the elliptic projection $\Ph:\HozDD\rightarrow\Vhp$ by, for $w \in \HozDD$
\beqs
\IPLtD{A\grad\vh}{\grad\Ph w} = \IPLtD{A\grad\vh}{\grad w} \tforall \vh \in \Vhp.
\eeqs
%% and define the weighted elliptic projection $\Phn:\HozDD\rightarrow\Vhp$ by, for $w \in \HozDD$
%% \beqs
%% \IPLtD{A\grad\vh}{\grad\Phn w} = \IPLtDn{A\grad\vh}{\grad w} \tforall \vh \in \Vhp.
%% \eeqs

Observe that $\Ph w$ is the finite-element solution of a stationary diffusion problem with `diffusion coefficient' $A$ and a right-hand side in $\HozDD'.$

We define the $\LtD$-projection $\Qh:\HozDD\rightarrow \Vhp$ by, for $w \in \HozDD$
\beqs
\IPLtD{\Qh w}{\vh} = \IPLtD{w}{\vh} \tforall \vh \in \Vhp.
\eeqs

We also need to define the $\LtD$ projection in a norm weighted by $n$

For $v,w \in \LtD,$ define the $n$-weighted inner product
\beqs
\IPLtDn{v}{w} = \int_{D} n v \wbar,
\eeqs
and the corresponding $n$-weighted $\LtD$ norm $\NLtDn{v} = \sqrt{\int_{D} n \abs{v}^2}$.

For $s \in \NN$, we define the $n$-weighted $\HsD$ norms $\NHsDn{v}^2 = \sum_{\alpha \st \abs{\alpha} \leq s}\NLtDn{D^\alpha v}$, and the negative weighted Sobolev norms by
\beq\label{eq:negweightnorm}
\NHnfn{-s}{D}{v} = \sup_{w \in \HsD} \frac{\IPLtDn{v}{w}}{\NHsDn{w}}.
\eeq
Observe that, for $v \in \HsD,$
\beq\label{eq:nconv}
\nmin\NHsD{v} \leq \NHsDn{v} \leq \NLiD{n} \NHsD{v}.
\eeq
%We put the \emph{non-weighted} $H^s$-norm in the denominator of \cref{eq:negweightnorm}, but the \emph{weighted} inner product in the numerator for ease of manipulation in some of the following proofs---we could place the weighted $H^s$ norm in the denominator, and this would be equivalent to the current definition, up to a factor involving $n$.
%% so that we have, for all $v \in \LtD$
%% \beq\label{eq:neqweightid}
%% \NHmsD{n v} = \NHmsDn{v}.
%% \eeq
%% \beq\label{eq:negativeweightbounds}
%% \frac{\NHmsD{v}}{\NLiD{n}} \leq \NHmsDn{v} \leq \frac{NHmsD{v}}{\nmin}.
%% \eeq

Define the $\LtD$ projection in the $n$-weighted norm $\Qhn:\HozDD\rightarrow \Vhp$ by, for $w \in \HozDD$
\beqs
\IPLtDn{\Qhn w}{\vh} = \IPLtDn{w}{\vh} \tforall \vh \in \Vhp.
\eeqs

We now state and prove our error bounds. These can all be obtained by modifications of the proof in \cite[Theorem 5.8.3]{BrSc:08} because all the projections defined above are Galerkin projections given by coercive and bounded sesquilinear forms on $\HoD$ (for $\Ph$) or $\LtD$ (for $\Qh$ and $\Qhn$).\optodo{Check if $\Qh$ is used.}

The elliptic projection obeys the following error bounds:
\ble[Error bounds for elliptic projection]\label{lem:ellprojerr}
Under \cref{ass:highp}, for any integer $s \in [-1,p-1],$ there exists a constant $\Cmso >0$ such that for all $w \in \HozDD$
\beq\label{eq:ellprojerr}
\NHmsD{w-\Ph w} \leq \Cmso h^{s+1} \BAHoD{w}{\wh}.
\eeq
\ele
\optodo{Correct language to talk about $n$-weighted}


\ble[Error bounds for elliptic projection in $n$-weighted norms]\label{lem:ellprojerrw}
Let $m \in \NNz.$ Under \cref{ass:highp}, there exists a constant $\Cwmm >0$ such that for all $w \in \HozDD$
\beqs
\NHnfn{-m}{D}{w-\Ph w} \leq \Cwmm \errn{m} h^{m+1} \BAHoDn{w}{\wh},
\eeqs
where
\beqs
\errn{m} =
\begin{dcases}
\frac{\NHsD{n}}{\nmin^2}\nvar &\tif m \in \mleft(\frac{d}2,p-1\mright)\\
\frac{\NWsiD{n}}{\nmin^2}\nvar &\tif m \in \mleft[1,\frac{d}2\mright]\\
\nvar &\tif m = -1,0.
\end{dcases}
\eeqs
%% \ben
%% \item\label[itempart]{it:wep1} If $s \in (d/2,p-1],$ then
%% \beq\label{eq:wep1}%\label{eq:ellprojerr}
%% \NHnfn{-s}{D}{w-\Ph w} \leq \Cwmso \frac{\NHsD{n}}{\nmin^2}\nvar h^{s+1} \BAHoDn{w}{\wh}.
%% \eeq
%% \item\label[itempart]{it:wep2} If $s \in [2,d/2]$ then
%% \beq\label{eq:wep2}
%% \NHnfn{-s}{D}{w-\Ph w} \leq \Cwmso \frac{\NWsiD{n}}{\nmin^2}\nvar h^{s+1} \BAHoDn{w}{\wh}.
%% \eeq
%% \item\label[itempart]{it:wep4} For $s = 0$
%% \beq\label{eq:wep4}
%% \NLtDn{w-\Ph w} \leq \Cwz \nvar h \BAHoDn{w}{\wh}.
%% \eeq
%% \item\label[itempart]{it:wep3}Also, we have
%% \beq\label{eq:wep3}
%% \NHnfn{1}{D}{w-\Ph w} \leq \Cwo\nvar \BAHoDn{w}{\wh}.
%% \eeq
%% \een
\ele

\bpf[Proof of \cref{lem:ellprojerrw}]
For \cref{it:wep1,it:wep2}, let $\phi \in \HsD,$ and observe that by \cref{thm:banachalg} $n\phi\in \HsD$ also. Let $ \vtilde = \sdsol(\phi),$ observe\ednote{$\Sn$ is defined in \cref{sec:discsob}---I haven't yet decided where to put the definition.} that by \cref{ass:highp,thm:shift} $\vtilde \in \Hfn{}{s+2}{D}$. For all $v \in \HozDD,$ we have
\beqs
\IPLtD{A \grad \vtilde}{\grad v} = \IPLtD{n\phi}{v} = \IPLtDn{\phi}{v}.
\eeqs
If we take $v = w-\Ph w,$ then we can compute
\begin{align}
\IPLtDn{\phi}{v} &= \IPLtD{A\grad\mleft(\vtilde - \vh\mright)}{\grad\mleft(w-\Ph w\mright)} \text{ for } \vh \text{ the quasi-interpolant of } v\text{, by Galerkin orthogonality for } \Ph\nonumber\\
&\leq \CSZfn{s+2}\NLiDop{A} \NHfn{s+2}{D}{\vtilde} h^{s+1} \SNHoD{w-\Ph w} \text{ by \cref{lem:scottzhang}}\nonumber\\
&\leq \CSZfn{s+2} \CAfn{s} \NLiDop{A} \NHfn{s}{D}{n\phi} h^{s+1}\SNHoD{w-\Ph w}\text{ by \cref{thm:shift}}\label{eq:ellprojwpart}
\end{align}

For \cref{it:wep1}, $s > d/2,$ and so by \cref{thm:banachalg,eq:nconv}, \cref{eq:ellprojwpart} is bounded above by
\beq\label{eq:wep1pt0}
\CSZfn{s+2} \CAfn{s} \CBanfn{s} \NLiDop{A} \frac{\NHfn{s}{D}{n}}{\nmin}\NHnfn{s}{D}{\phi} h^{s+1}\SNHoD{w-\Ph w}.
\eeq


Therefore we have, by definition of $\NHnfn{-s}{D}{\cdot}$ and $\NHoDn{\cdot}$,
\beq\label{eq:wep1pt1}
\NHnfn{-s}{D}{w-\Ph w} \leq \CSZfn{s+2}\CAfn{s} \CBanfn{s} \NLiDop{A} \frac{\NHmD{n}}{\nmin^2} h^{s+1} \NHoDn{w - \Ph w}.
\eeq
Applying C\'ea's Lemma to $\Ph$ in the $n$-weighted $H^1$ norm, we find
\beq\label{eq:wepcea}
\NHoDn{w-\Ph w} \leq \frac{2\NLiDop{A}}{\min\set{1,1/\CP^2}\Amin}\nvar\BAHoDn{w}{\wh},
\eeq
as $\Ph$ corresponds to a sesquilinear form that is bounded with constant $\NLiDop{A}/\nmin$ and coercive with constant $\mleft(\Amin\min\set{1,1/\CP}\mright)/\mleft(2\NLiD{n}\mright)$ in the $n$-weighted $H^1$ norm.
and then combining \cref{eq:wep1pt1,eq:wepcea}, we obtain \cref{eq:wep1}.

For \cref{it:wep2}, the application of \cref{thm:banachalg} yields, instead of \cref{eq:wep1pt0},
\beq\label{eq:wep2pt1}
2\CSZfn{s+2} \CAfn{s} \Cprod{s} \NLiDop{A} \frac{\NWsiD{n}}{\nmin}\NHnfn{s}{D}{\phi} h^{s+1}\SNHoD{w-\Ph w},
\eeq
(since $1 + \NWsiD{n} \leq 2\NWsiD{n},$ as $\NLiD{n} \geq 1$) and by a similar reasoning to that before, we obtain \cref{eq:wep2}.

For \cref{it:wep4}, using \cref{lem:ellprojerr} and then converting to the $n$-weighted $L^2$ norm yields \cref{eq:wep4}. For \cref{it:wep3}, \cref{eq:wepcea} immediately gives \cref{eq:wep3}. 
\epf

%% We also give the following error bounds for the weighted elliptic projection, and give the proof, even though it is conceptually similar to the proof of \cref{lem:ellprojw}.\{Check the original one is still needed.}

%% \ble[Error bounds for weighted elliptic projection in $n$-weighted norms]\label{lem:wellprojerrw}
%% Let $s \in \ZZ.$ There exists a constant $\Cwmso >0$ such that for all $w \in \HozDD$
%% \ben
%% \item\label[itempart]{it:wwep1} If $s \in (d/2,p-1],$ and $n \in \HsD$ then
%% \beq\label{eq:wwep1}%\label{eq:ellprojerr}
%% \NHnfn{-s}{D}{w-\Phn w} \leq \Cwmso \frac{\NHsD{n}\NLiD{n}^2}{\nmin^3}h^{s+1} \BAHoDn{w}{\wh}.
%% \eeq
%% \item\label[itempart]{it:wwep2} If $s \in [1,d/2]$ and $n \in \CsD$, then
%% \beq\label{eq:wwep2}
%% \NHnfn{-s}{D}{w-\Phn w} \leq \Cwmso \frac{\NWsiD{n}\NLiD{n}^2}{\nmin^3} h^{s+1} \BAHoDn{w}{\wh}.
%% \eeq
%% \item\label[itempart]{it:wwep4} For $s = 0$
%% \beq\label{eq:wwep4}
%% \NLtDn{w-\Phn w} \leq \Cwfn{0,1} \frac{\NLiD{n}^3}{\nmin^3}h \BAHoDn{w}{\wh}
%% \eeq
%% \item\label[itempart]{it:wwep3} For $s = 1$
%% \beq\label{eq:it:wwep3}
%% \NHnfn{1}{D}{w-\Phn w} \leq \Cwfn{1,1}\frac{\NLiD{n}^2}{\nmin^2} \BAHoDn{w}{\wh}.
%% \eeq
%% \een
%% \ele

%% \bpf[Proof of \cref{lem:wellprojerrw}]
%% For \cref{it:wep1,it:wep2}, the proof follows the proof in \cite[Theorem 5.8.3]{BrSc:08}. Let $\phi \in \HsD,$ and observe that by \cref{thm:banachalg} $n\phi\in \HsD$ also. Let $ \vtilde = \sdsol(\phi),$ observe that by \cref{thm:shift} $\vtilde \in \Hfn{}{s+2}{D}$. For all $v \in \HozDD,$ we have
%% \beqs
%% \IPLtD{A \grad \vtilde}{\grad v} = \IPLtD{n\phi}{v} = \IPLtDn{\phi}{v}.
%% \eeqs
%% If we take $v = w-\Phn w,$ then we can compute
%% \begin{align}
%% \IPLtDn{\phi}{v} &= \IPLtD{A\grad\mleft(\vtilde - \vh\mright)}{\grad\mleft(w-\Phn w\mright)} \text{ for } \vh \text{ as in \cref{lem:scottzhang}, by Galerkin orthogonality for } \Phn\nonumber\\
%% &\leq \CSZfn{s+2}\NLiDop{A} \NHfn{s+2}{D}{\vtilde} h^{s+1} \SNHoD{w-\Phn w} \text{ by \cref{lem:scottzhang}}\nonumber\\
%% &\leq \CSZfn{s+2} \CAfn{s} \NLiDop{A} \NHfn{s}{D}{n\phi} h^{s+1}\SNHoD{w-\Phn w}\text{ by \cref{thm:shift}}\label{eq:ellprojwpart}
%% \end{align}

%% For \cref{it:wep1}, $s > d/2,$ and so by \cref{thm:banachalg}, \cref{eq:ellprojwpart} is bounded above by
%% \beq\label{eq:wep1pt0}
%% \CSZfn{s+2} \CAfn{s} \CBanfn{s} \NLiDop{A} \frac{\NHfn{s}{D}{n}}{\nmin}\NHnfn{s}{D}{\phi} h^{s+1}\SNHoD{w-\Phn w}
%% \eeq
%% by \cref{thm:banachalg,eq:nconv}.

%% Therefore we have, by definition of $\NHnfn{-s}{D}{\cdot}$ and $\NHoDn{\cdot}$,
%% \beq\label{eq:wep1pt1}
%% \NHnfn{-s}{D}{w-\Phn w} \leq \CSZfn{s+2}\CAfn{s} \CBanfn{s} \NLiDop{A} \frac{\NHmD{n}}{\nmin^2} h^{s+1} \NHoDn{w - \Phn w}.
%% \eeq
%% Applying C\'ea's Lemma to $\Phn$ in the weighted $H^1$ norm, we find
%% \beq\label{eq:wepcea}
%% \NHoDn{w-\Phn w} \leq \frac{2\NLiDop{A}\NLiD{n}^2}{\min\set{1,1/\CP^2}\Amin\nmin^2}\BAHoDn{w}{\wh},
%% \eeq
%% and then combining \cref{eq:wep1pt1,eq:wepcea}, we obtain \cref{eq:wep1}.

%% For \cref{it:wep2}, the application of \cref{thm:banachalg} yields, instead of \cref{eq:wep1pt0},
%% \beq\label{eq:wep2pt1}
%% 2\CSZfn{s+2} \CAfn{s} \Cprod{s} \NLiDop{A} \frac{\NWsiD{n}}{\nmin}\NHnfn{s}{D}{\phi} h^{s+1}\SNHoD{w-\Phn w},
%% \eeq
%% since $1 + \NWsiD{n} \leq 2\NWsiD{n},$ as $\NLiD{n} \geq 1,$ and by a similar reasoning to that before, we obtain \cref{eq:wep2}.

%% For \cref{it:wep3}, \cref{eq:wepcea} immediately gives the result. For \cref{it:wep4}, performing a standard duality argument in \emph{non-weighted} norms (and then using \cref{eq:it:wep3}, yields \cref{eq:wep4}.
%% \epf


The $\LtD$ projection satisfies the following error bound.
\ble[Error bounds for $\LtD$ projection]\label{lem:ltdprojerr}
Under \cref{ass:highp}, for any integer $s \in [0,p-1],$ for all $w \in \LtD$
\beqs
\NHmsD{w-\Qh w} \leq \Cmsz h^{s} \BALtD{w}{\wh}.
\eeqs
\ele

Similarly, the weighted $\LtD$ projection satisfies the following error bound.
The $n$-weighted $\LtD$ projection satisfies the following error bound:
\ble[Error bounds for weighted $\LtD$ projection]\label{lem:wltdprojerr}
Under \cref{ass:highp}, for any integer $s \in [0,p-1],$ for all $w \in \HozDD$
\beq\label{eq:wltdprojerr}
\NHmsDn{w-\Qhn w} \leq \CSZfn{s} \frac{\NLiD{n}}{\nmin} h^{s} \BALtDn{w}{\wh}.
\eeq
\ele

\bpf[Proof of \cref{lem:wltdprojerr}]
Fix $\vtilde \in \HsD$, trivially $\vtilde$ solves the adjoint problem
\beqs
\IPLtDn{v}{\vtilde} = \IPLtDn{v}{\vtilde} \tforall v \in \LtD.
\eeqs
Then letting $v=w-\Qhn w$ and using Galerkin orthogonality, for $\vhptilde$ as in \cref{lem:scottzhang} we have
\beqs%gin{align*}
\IPLtDn{w-\Qhn w}{\vtilde} \leq \NLtDn{w-\Qhn w}\NLtDn{\vtilde-\vhptilde}\leq \CSZfn{s} \frac{\NLiD{n}}{\nmin}\NLtDn{w-\Qhn w} h^s \NHsDn{\vtilde}.
\eeqs%nd{align*}
Taking the supremum over $\vtilde,$ we have
\beqs
\NHmsDn{w-\Qhn w} \leq \CSZfn{s} \frac{\NLiD{n}}{\nmin} h^s \NLtDn{w-\Qhn w},
\eeqs
and hence by C\'ea's Lemma (as the inner product $\IPLtDn{\cdot}{\cdot}$ is clearly bounded and coercive in the weighted $L^2$-norm $\NLtDn{\cdot}$) the result follows.
\epf

\subsection{Discrete Sobolev spaces}\label{sec:discsob}
In order to perform our analysis for high-order FEM, we will need to measure higher-order norms of functions in the finite-element space $\Vhp.$ However, as these functions do not have higher-order weak derivatives, we must first develop some theory of so-called discrete Sobolev spaces; we follow the presentation in \cite{DuWu:15}, albeit working in the heterogeneous case, and with some changes of notation.

We let $\DeltaAI:\LtD\rightarrow\HtD$ denote the solution operator for the stationary diffusion equation: given $\ftilde \in \LtD$ find $\utilde \in \HtD$ such that
\beq\label{eq:sdeq}
\grad \cdot \mleft(A\grad \utilde\mright) = -n\ftilde \text{ in } D
\eeq
\beq\label{eq:sddbc}
\trD \utilde = 0
\eeq
\beq\label{eq:sdnbc}
\dn \utilde = 0.
\eeq
%% $A$-weighted Laplacian; that is $\DeltaA w = \grad\cdot\mleft(\grad w\mright),$ and give it domain $\DomainDeltaA = \set{w \in \HtD \st \trD w = 0, \dn w = 0};$
%% hence $\DeltaA:\DomainDeltaA \rightarrow \LtD.$ Observe that, for any $f \in \LtD$ there exists $\wf \in \DomainDeltaA$ such that $\DeltaA \wf = -f.$
Observe that $\sdsol$ is well-defined by \cref{lem:domainshift}. Also, observe that $\sdsol^{m}$ is defined for any $m \in \NN,$ as $\HtD \subseteq \LtD,$ and so one can place $\sdsol \ftilde$ on the right-hand side. 
Observe that for any $\ftilde \in \LtD$ and for any $v \in \HozDD,$ we have, by Green's identity,
\beqs
\int_D \mleft(A \grad \mleft(\DeltaAI\ftilde\mright)\mright)\cdot \grad \vb = \int_D n\ftilde \vb,
\eeqs
i.e.,
\beq\label{eq:deltaagreen}
\IPLtD{A\grad \mleft(\sdsol \ftilde\mright)}{\grad v} = \IPLtDn{\ftilde}{v}.
\eeq


\bde[Discrete derivative operator]
Define the \defn{$A$-weighted discrete second derivative operator} $\Deltah:\Vhp\rightarrow\Vhp$ for $\wh \in \Vhp$ by
\beq\label{eq:discderdef}
\IPLtDn{\Deltah \wh}{\vh} = \IPLtD{A \grad \wh}{\grad \vh} \tforall \vh \in \Vhp.
\eeq
\ede

\bre[Why the factor $n$ on the right-hand side of \cref{eq:sdeq}?]
The reason for defining $\Sn$ with a factor $n$ on the right-hand side of \cref{eq:sdeq} is somewhat buried in the proof of \cref{thm:fembound} and its associated lemmas. However, we give an overview of the reason here.

All the bounds in the proofs of \cref{lem:boundarybound,lem:higherbound,lem:ltthetahbound} are in weighted discrete norms, because we can only prove bounds on the weighted $L^2$ projection $\Qhn$ in weighted higher-order discrete norms (as in \cref{lem:wltdprojerr}), not in non-weighted norms. Because we only work in weighted norms, we need \cref{lem:intoip} below to hold in the weighted inner product, which necessitates that we use the weighted inner product on the left-hand side of \cref{eq:discderdef}. This final use of the weighted inner product means we must have the weighted inner product on the right-hand side of \cref{eq:deltaagreen} so that we can show (at the beginning of the proof of \cref{lem:negdiscsum}) that if $\wh \in \Vhp,$ $\zh = \Deltah^{-1} \wh$, and $z = \Sn \wh,$ then $\zh = \Ph z;$ we need this fact in the remainder of the proof of \cref{lem:negdiscsum}.

We do not, however, put a factor $n$ on the left-hand side of \cref{eq:sdeq}. If we did, when we apply \cref{thm:shift} to $\Sn \ftilde$ (as we do in \cref{lem:shiftnegativew}), the resulting bounds would not be fully explicit in $n$ (because in \cref{thm:shift} we do not know explicitly how the constants depend on the `diffusion coefficient'). Not putting a factor $n$ on the left-hand side of \cref{eq:sdeq} is the reason why there is a \emph{non-weighted} inner product on the right-hand side of \cref{eq:discderdef} (so that the beginning of the proof of \cref{lem:negdiscsum}, as mentioned above, works).
\ere

\ble[Discrete derivative operator is well-defined]\label{lem:ddwd}
For any $\wh \in \Vhp,$ $\Deltah \wh$ exists and is unique.
\ele

\bpf[Proof of \cref{lem:ddwd}]
If one chooses an orthonormal (in the $n$-weighted inner product) basis  $\phij$ for $\Vhp,$ writes $\Delta\wh = \sum_j \wj \phij$, and takes $\vh = \phij$ for each $j,$ then we see \cref{eq:discderdef} is equivalent to the linear system $\Imat \bw = \bb,$ where $\bb_{j} = \IPLtDn{A \grad \wh}{\grad \phij}.$ The solution of this linear system clearly exists and is unique.
%% We equip $\Vhp$ with the $H^1$-norm. Observe that $\Deltah \wh$ satisfies the variational problem: Find $\vhtilde  \in \Vhp$ such that $\add(\uh,\vh) = \Ldd(\vh)$ for all $\vh \in \Vhp,$ where $\add(\uh,\vh) = \IPLtD{\uh}{\vh}$ and $\Ldd(\vh) = \IPLtD{A \grad \wh}{\grad \vh}.$ Observe that $\Ldd$ is bounded in $\Vhp$, as $\Ldd(\vh) \leq \NLiDop{A} \SNHoD{\wh}\SNHoD{\vh} \leq \NLiDop{A} \SNHoD{\wh}\NHoD{\vh},$ and $\add$ is coercive on $\Vhp,$ as, for $\vh \in \Vhp$, $\add(\vh,\vh) = \NLtD{\vh}^2 \geq \CinvVhp^2 \NHoD{\vh}^2$ by the standard inverse estimate\optodo{Add in?}. Therefore, by the Lax--Milgram Theorem applied in $\Vhp$ (as $\Vhp$ is a finite-dimensional inner product space over a complete field, it is a Hilbert space), $\Deltah \wh$ exists and is unique.
\epf

%% By\optodo{McLean Thm 4.12 - double check and maybe write out in more detail---exactly what operator are we talking about, especially if we want weak derivatives?}, there exists a sequence of eigenfunctions $\phio,\phit,\ldots \in \HoD$ of $\DeltaA$\optodo{What exactly does McLean mean here, if they don't have second-order derivatives?} and corresponding eigenvalues $0 < \lambdao<\lambdat < \cdots \rightarrow\infty$ such that the eigenfunctions form a complete orthonormal system in $\LtD.$\optodo{Maybe define this}

Since $A$ is real and symmetric, it is self-adjoint. Hence it follows that $\Deltah$ is self-adjoint, as
\beqs
\IPLtDn{\Deltah \wh}{\vh} = \IPLtD{A \grad \wh}{\grad \vh} = \IPLtD{\grad \wh}{A\grad \vh} = \overline{\IPLtDn{\Deltah \vh}{\wh}} = \IPLtDn{\wh}{\Deltah \vh}.
\eeqs
Therefore $\Deltah$ is diagonalisable, i.e., there exists a set of eigenfunctions $\phioh,\ldots,\phidimVhph$  with corresponding real eigenvalues\ednote{Both - is it common enough knowledge that a symmetric matrix has real eigenvalues, that I don't need to reference this?} $\lambdaoh, \ldots, \lambdadimVhph$ such that the $\phimh$ form an orthonormal (in the $n$-weighted inner product) basis of $\Vhp$.

\bde[Higher-order discrete derivative operators]
For $\vh \in \Vhp$, if $\vh = \sum_{m=1}^{\dimVhp} \am \phimh,$ then for $j \in \RR$ define
\beqs
\Deltah^j \vh = \sum_{m=1}^{\dimVhp} \lambdamh^j \am \phimh.
\eeqs


\ede
One can think of $\DeltahI$ as being a `discrete solution operator', i.e., a discrete counterpart to $\sdsol.$
%% Similarly, for $v \in \LtD,$ if $v = \sum_{m=1}^\infty \am \phim,$ then for $j \in \RR$ define
%% \beq\label{eq:deltaaseries}
%% \DeltaA^j v = \sum_{m=1}^\infty \lambdam^j \am \phim,
%% \eeq
%% if this series exists in $\LtD.$
%We let $\Domain{\DeltaA^j}$ denote the subset of $\LtD$ on which $\DeltaA^j$ is defined.
%% \bre[Negative powers of $\DeltaA$]
%% Observe that for \emph{every} $j \leq 0$, $\DeltaA^j v$ is defined for \emph{any} $v \in \LtD$ (i.e., $\Domain{\DeltaA^{j}} = \LtD$ for $j \leq 0$). For $\lambdam \geq 1,$ $\lambdam^j < \lambdam$, and only finitely many $\lambdam$ are in the interval $(0,1)$; therefore the series \cref{eq:deltaaseries} can be decomposed as a finite sum (for $\lambdam < 1$) and a convergent series (for $\lambdam \geq 1$).
%% \ere
%% \bre[Consistent Notation]
%% Observe that the notation $\DeltaA^j$ is consistent, i.e., $\DeltaA^0 v = v$, for $j \in \NN,$ $\DeltaA^j$ is equal to the $j$-fold application of $\DeltaA,$ and $\DeltaA^{-1}$ is the inverse of $\DeltaA.$\optodo{Maybe just double-check this is watertight.}\optodo{This needs work - need $A$ to be smooth enough to define proper higher-order derivatives.}
%% \ere
We can use the higher-order derivative operators to define discrete higher-order norms:

\bde[Discrete higher-order norm]
For $\vh \in \Vhp$ and $m \in \RR$, if define
\beqs
\Nshn{\vh} = \NLtDn{\Deltah^{s/2} v}.
\eeqs
\ede

%% \bde[$A$-weighted higher-order continuous norm]
%% For $v \in \LtD$ and $m \in \RR$, if $\DeltaA^{m/2} v$ exists, define
%% \beqs
%% \NmA{v} = \NLtD{\DeltaA^{m/2} v}.
%% \eeqs
%% \ede

%% In order to prove a relationship between the $A$-weighted higher order norm and the standard $H^m$ norms, we first must prove the following \lcnamecref{lem:normrelationshiptech}\optodo{THIS NEEDS PROVING AND I DON'T KNOW HOW.}
%% \ble[Relationship between $\DeltaA^m$ and standard higher-order derivatives]\label{lem:normrelationshiptech}
%% For $m \in \NN,$ $\Domain{\DeltaA^{m/2}} \subseteq \HmD,$ and there exists a constant $\Cma > 0$ such that for all $v \in \Domain{\DeltaA^{m}}$
%% \beqs
%% \NHmD{v} \leq \Cma \NLtD{\DeltaA^{m/2}v}.
%% \eeqs\optodo{The latter bit will probably use something shift-theorem-like, but you need to be careful because powers of the differential operator aren't defined standardly.}
%% \ele


%% \ble[Relationship between $A$-weighted and standard higher-order continuous norms]\label{lem:normrelationship}
%% For all $m \in \NNz,$ for all $v \in \HmmD,$
%% \beqs
%% \NHmmD{v} \geq \frac1{\Cma} \NmmA{v}.
%% \eeqs
%% \ele

%% \bpf[Proof of \cref{lem:normrelationship}]
%% For $v \in \HmmD,$ we have
%% \begin{align*}
%% \NHmmD{v} &= \sup_{w \in \HmD} \frac{\IPLtD{v}{w}}{\NHmD{w}}\\
%% &\geq \frac1{\Cma} \sup_{w \in \Domain{\DeltaA^{m/2}}} \frac{\IPLtD{v}{w}}{\NmA{w}} \text{ by \cref{lem:normrelationshiptech}}\\
%% & = \frac1{\Cma} \sup_{w \in \Domain{\DeltaA^{m/2}}} \frac{\IPLtD{\DeltaA^{-m/2}v}{\DeltaA^{m/2}w}}{\NmA{w}} \text{ by \cref{lem:intoip}}\\
%% &= \frac1{\Cma} \NmmA{v} \text{ the supremum is achieved when } \IPLtD{\DeltaA^{-m/2}v,\DeltaA^{m/2}w} = \NLtD{\DeltaA^{-m/2}v}\NLtD{\DeltaA^{m/2}w}
%% \end{align*}
%% \epf

We will use the following \lcnamecref{lem:intoip} to bound the inner product of two discrete functions by their negative- and positive-higher-order discrete norms, or to transfer discrete derivatives from one argument of the inner product to the other.

\ble[Introduction of derivatives into inner product]\label{lem:intoip}
%% For $m \in \RR,$ $v \in \LtD$, and $w \in \LtD\cap\Domain{\DeltaA^{m/2}}$ we have
%% \beqs
%% \IPLtD{\DeltaAmmt v}{\DeltaAmt w} = \IPLtD{v}{w}.
%% \eeqs
%% Similarly, f
For $\vh, \wh \in \Vhp,$ and $s \in \RR$ we have
\beq\label{eq:feipsplit}
\IPLtDn{\Deltahmst \vh}{\Deltahst \wh} = \IPLtDn{\vh}{\wh}
\eeq
and
\beq\label{eq:feiptrans}
\IPLtDn{\Deltahst \vh}{\Deltahst \vh} =  \IPLtDn{\Deltah^s \vh}{\vh}.
\eeq
\ele
\bpf[Proof of \cref{lem:intoip}]
We only prove \cref{eq:feipsplit}, as the proof of \cref{eq:feiptrans} is analagous. Since $\vh,\wh \in \Vhp,$ there exist sequences $(\aj)_{j =1,\ldots,\dimVhp}$ and $(\bsl)_{l =1,\ldots,\dimVhp}$ such that $\vh = \sum_{j=1}^{\dimVhp} \aj\phij$ and $\wh = \sum_{l=1}^{\dimVhp} \bsl \phil.$ Then we have
\begin{align*}
\IPLtDn{\Deltahmst \vh}{\Deltahst \wh} &= \int_D n\mleft(\sum_{j=1}^{\dimVhp}\lambdaj^{-m/2} \aj\phij\mright)\overline{\mleft(\sum_{l=1}^{\dimVhp} \lambdal^{m/2}\bsl \phil\mright)}\\
&= \sum_{j,l=1}^{\dimVhp} \lambdaj^{-m/2} \lambdal^{m/2} \aj \bsl \int_D n \phij \philbar \text{ as the } \lambdaj \text{ are real}\\
& =\sum_{j}^{\dimVhp} \aj \bsj \int_D n\abs{\phij}^2 \text{ as the } \phij \text{ are orthonormal}\\
&= \IPLtDn{\vh}{\wh}
\end{align*}
by repeating the above process in reverse, without the factors $\lambdaj^{-m/2}$ and $\lambdal^{m/2}$.
\epf
The next \lcnamecref{cor:ipdiscbound} follows from \cref{lem:intoip} and the Cauchy--Schwarz inequality.

\bco[Inner product bounded by discrete norms]\label{cor:ipdiscbound}
If $\vh, \wh \in \Vhp,$ then for all $j \in \RR$
\beqs
\IPLtDn{\vh}{\wh} \leq \Njh{\vh}\Nmjh{\wh}.
\eeqs
\eco

We recall the standard inverse inequality for finite-element functions, so that we can prove an analagous inverse inequality for discrete norms.

\ble[Standard inverse inequality]\label{lem:inverseinequality}
There exists $\CinvVhp > 0$ such that for all $\vh \in \Vhp$
\beqs
\NHoD{\vh} \leq \CinvVhp h^{-1} \NLtD{\vh}.
\eeqs
\ele



\ble[Inverse inequality for discrete norms]\label{lem:inversediscrete}
For all $j \in \ZZ$, for all $\vh \in \Vhp$
\beqs
\Njhn{\vh} \leq \Chinv \frac{1}{\nmin} h^{-1} \Njmohn{\vh}.
\eeqs
\ele

\bpf[Proof of \cref{lem:inversediscrete}]
by the definition of $\Njhn{\cdot}$, and the fact that $\Deltah^{j/2} = \Deltah^{1/2}\Deltah^{(j-1)/2},$ it suffices to prove the result for $j=1$, as one can then perform induction on $j$. We have
\begin{align*}
\Nfn{1,h,n}{\vh}^2 &= \IPLtDn{\Deltahh \vh}{\Deltahh \vh}\\
&= \IPLtDn{\Deltah \vh}{\vh} \text{ by \cref{eq:feiptrans}}\\
&= \IPLtD{A \grad \vh}{\grad \vh} \text{ by definition of } \Deltah\\
&\leq \NLiDop{A} \CinvVhp h^{-2}\frac{1}{\nmin^2} \NLtDn{\vh}^2
\end{align*}
by the standard inverse estimate, and the result follows as $\Nzhn{\cdot} = \NLtDn{\cdot}$.
\epf


\ble[Relationship between standard and discrete $H^1$ norms]\label{lem:h1contdisc}
Let $\vh \in \Vhp$. Then
\beqs
\SNHoD{\vh} \leq \Amin^{-\half} \Nohn{\vh}.
\eeqs
\ele

\bpf[Proof of \cref{lem:h1contdisc}]
We have, using \cref{eq:feiptrans}, $\Nohn{\vh}^2 = \IPLtDn{\Deltahh \vh}{\Deltahh \vh} = \IPLtDn{\Deltah \vh}{\vh}= \IPLtD{A \grad \vh}{\grad \vh} \geq \Amin \NLtD{\grad \vh}^2$, and the result follows.
\epf

To prove \cref{lem:negdiscsum} below on the relationship between discrete and continuous negative Sobolev normns, we require the following \lcnamecref{lem:shiftnegativew} giving the shift theorem in negative weighted norms.

\ble[Shift theorem in negative weighted norms]\label{lem:shiftnegativew}
Under \cref{ass:highp}, let $\ftilde \in \LtD.$ For integer $m \in [0,p-1],$ we have
\beq\label{eq:shiftnegativew}
\NHnfn{-m}{D}{\sdsol\ftilde} \leq \Cshiftfn{m} \errn{m} \NHnfn{-(m+2)}{D}{\ftilde},
\eeq
where
\beqs
\errn{m} \de 
\begin{dcases}
 \NHmD{n}\nvar &\tif m > \frac{d}2\\
\NWmiD{n}\nvar  &\tif m \in \mleft[1,\frac{d}2\mright]\\
\NLiD{n}\nvar &\tif m = 0\\
\NLiD{n}^2 &\tif m = -1.
\end{dcases}
\eeqs
\ele

\bpf[Proof of \cref{lem:shiftnegativew}]
We first observe that the operator $\sdsolo$ is self-adjoint on $\LtD$: Let $\DeltaA:\HtD\rightarrow\LtD$ denote the stationary diffusion operator $\grad\cdot\mleft(A\grad\cdot\mright).$ Then for any $\vo \in \LtD,$ $\DeltaA \circ \sdsolo \vo = \vo.$ Moreover, by Green's Theorem, $\DeltaA$ is self-adjoint on the set $\set{v \in \HtD \st v \text{ satisfies }\cref{eq:sddbc,eq:sdnbc}},$ this set is contained in the image of $\sdsolo.$ Therefore, for any $\vo,\vt \in \LtD,$ we have
\beqs
\IPLtD{\sdsolo \vo}{\vt} =\IPLtD{\sdsolo \vo}{\DeltaA \circ \sdsolo \vt} = \IPLtD{\DeltaA \circ \sdsolo \vo}{\sdsolo \vt} = \IPLtD{\vo}{\sdsolo \vt},
\eeqs
i.e., $\sdsolo$ is self-adjoint on $\LtD.$

Observe that by \cref{thm:banachalg}, if $v \in \HmD$ then $nv \in \HmD,$ and therefore by \cref{thm:shift} $\sdsolo(nv) \in \HmptD.$ With these facts in place we can compute
\begin{align*}
\NHnfn{-m}{D}{\sdsol \ftilde} &= \sup_{v \in \HmD} \frac{\IPLtD{\sdsol \ftilde}{nv}}{\NHmDn{v}}\\
&= \sup_{v \in \HmD} \frac{\IPLtD{\sdsolo\mleft(n \ftilde\mright)}{nv}}{\NHmDn{v}}\\
&= \sup_{v \in \HmD} \frac{\IPLtDn{\ftilde}{\sdsolo\mleft(n v\mright)}}{\NHmDn{v}}\tas \sdsolo \text{ is self-adjoint}\\
&\leq \sup_{v \in \HmD} \frac{\NHnfn{-(m+2)}{D}{\ftilde}\NHnfn{m+2}{D}{\sdsolo\mleft(nv\mright)} }{\NHmDn{v}}\\
&\leq \sup_{v \in \HmD}  \frac{\CAfn{m} \NLiD{n} \NHmD{nv} \NHnfn{-(m+2)}{D}{\ftilde}}{\NHmDn{v}}
%% & = \sup_{v \in \HmD} \frac{\IPLtD{ n\ftilde}{\sdsol(v)}}{\NHmD{v}}\text{ by definition of } \sdsol, \text{ and as } \sdsolo \text{ is self-adjoint}\\
%% &= \sup_{v \in \HmD} \frac{\IPLtDn{\ftilde}{\frac{\sdsol(nv)}n}}{\NHmD{v}}\\
%% &\leq \CBanfn{m+2}\sup_{v \in \HmD} \frac{\NHfn{-(m+2)}{D}{\ftilde}\NHfn{m+2}{D}{\sdsol(nv)}\NHfn{m+2}{D}{\frac1n}}{\NHmD{v}} \text{ by \cref{thm:banachalg}}\\
%% &\leq \CBanfn{m+2}\CAfn{m}\sup_{v \in \HmD} \frac{\NHfn{-(m+2)}{D}{\ftilde}\NHfn{m}{D}{nv}\NHfn{m+2}{D}{\frac1n}}{\NHmD{v}} \text{ by \cref{thm:shift}}\\
\end{align*}
and by applying \cref{thm:banachalg} to the term $\NHfn{m}{D}{nv}$ (or observing that $\NLtD{nv} \leq \NLiD{n}\NLtD{v}$, in the case $m=0$), the result follows, except for $m=-1.$

For $m=-1,$ we have, by the Lax--Milgram Theorem in non-weighted norms, $\NHoD{\Sn \ftilde} \leq \NHmoD{n\ftilde}/\Amin.$ One can straightforwardly show that $\NHnfn{1}{D}{\Sn \ftilde} \leq \NLiD{n} \NHoD{\Sn \ftilde}$ and $\NHmoD{n\ftilde} \leq \NLiD{n} \NHnfn{-1}{D}{\ftilde},$ and so the result follows.
\epf

%% \ble[Shift theorem in negative norms]\label{lem:shiftnegative}
%% For $m \in \NN,$ and $\ftilde \in \LtD,$ we have
%% \beqs
%% \NHmmD{\DeltaAI\ftilde} \leq \CAm \NHmmmtD{\ftilde}.
%% \eeqs
%% \ele

%% \bpf[Proof of \cref{lem:shiftnegative}]
%% Throughout the proof we let $\utilde$ denote $\DeltaAI\ftilde.$ As $\ftilde \in \LtD,$ it follows that $\utilde \in \HtD$ by \cref{thm:shift}, and so in particular $\utilde \in \HmmD.$ We will use the fact that $\sdsol$ is self-adjoint; this follows from the fact that the boundary-value problem \cref{eq:sdeq,eq:sddbc,eq:sdnbc} is self-adjoint\ednote{Both - do I need to show this? It's straightforward.}.% Note to self, have (Rv,w), where R is resolvent, for v in L^2, w in H^m. Have PR = Id (but not necessarily the other way round) so w = PRw. Then (Rv,w) = (Rv,PRw) = (PRv,Rw) (P self-adjoint) = (v,Rw). Denote differential operator (on the correct space) by P.
%% We can then compute
%% \begin{align*}
%% \NHmmD{\utilde} &=  \sup_{0 \neq \vtilde \in \HmD} \frac{\IPLtD{\ftilde}{\DeltaAI\vtilde}}{\NHmD{\vtilde}}\text{ as } \DeltaAI \text{ is self-adjoint}\\
%% &\leq \sup_{0 \neq \vtilde \in \HmD} \frac{\NHmmmtD{\ftilde}\CAm\NHmD{\vtilde}}{\NHmD{\vtilde}} \text{ by \cref{thm:shift}}\\
%% &= \CAm \NHmmmtD{\ftilde}
%% \end{align*}
%% as required. \epf

We can now prove the following \lcnamecref{lem:negdiscsum} on the relationship between the negative-order discrete norms and the negative-order continuous norms.
\ble[Relationship between discrete and continuous negative-order norms]\label{lem:negdiscsum}
Under \cref{ass:highp}, for any integer $j \in [0,p+1],$ there exists a constant $\Csumj > 0$ such that for all $\vh \in \Vhp,$
\beqs
\Nfn{-j,h,n}{\vh} \leq \mleft(\En\nvar\mright)^{\floor{\frac{j}2}}\NLiD{n}\Csumj \sum_{m=0}^j h^{m} \NHnfn{-(j-m)}{D}{\vh}.
\eeqs
\ele

\bpf[Proof of \cref{lem:negdiscsum}]
Let $\wh \in \Vhp,$ and define $\zh = \DeltahI \wh,$ $z = \DeltaAI \wh$ (observe $z$ is well-defined as $\Vhp \subseteq \LtD$). Then, for all $\vh \in \Vhp$, we have

\begin{align*}
\IPLtD{A \grad z}{\grad \vh} = \IPLtD{A \grad \mleft(\sdsol \wh\mright)}{\grad \vh} = \IPLtDn{\wh}{\vh} \text{, and}\\
\IPLtD{A \grad \zh}{\grad \vh} = \IPLtDn{\Deltah \zh}{\vh} = \IPLtDn{\wh}{\vh},
\end{align*}
where the equalities in the first line follow from the definition of $z$ and \cref{eq:deltaagreen}, and the equalities in the second line follows from \cref{eq:discderdef} and the definition of $\zh.$  Therefore (using the fact that $A$ is symmetric), for all $\vh \in \Vhp,$ $\IPLtD{A\grad \vh}{\grad z} = \IPLtD{A\grad \vh}{\grad \zh},$ i.e., $\zh = \Ph z.$

We now have, for $m \in [-1,p-1]$ (using $\errn{m}$ to denote the terms depending on $n$ in \cref{eq:wep1,eq:wep2,eq:wep3,eq:wep4})
\begin{align}
\NHnfn{-m}{D}{\DeltahI \wh} &\leq \NHnfn{-m}{D}{z} + \NHnfn{-m}{D}{z-\zh}\nonumber\\
&\leq \NHnfn{-m}{D}{z} + \Cwfn{-m}  \CAfn{0} \CSZfn{2} \errn{m} \NLiD{n} h^{m+2} \NLtD{\wh}\nonumber\\
&\quad\quad\text{ by \cref{lem:ellprojerrw,lem:scottzhang,thm:shift}, as }\zh = \Ph z\nonumber\\
&= \Cshiftfn{m} \shiftn{m} \NHnfn{-(m+2)}{D}{\wh} + \Cm \errn{m} \frac{\NLiD{n}}{\nmin} h^{m+2} \NLtDn{\wh}\label{eq:sumforrecursion}
\end{align}
by \cref{lem:shiftnegativew}, where we use $\shiftn{m}$ to denote the terms depending on $n$ in \cref{eq:shiftnegativew}.
From \cref{eq:sumforrecursion}, we can conclude that, for $l \in \NN$ and $\vh \in \Vhp$, writing $\wh = \Deltahmlpo \vh$,
\beq\label{eq:lrecursion}
\NHnfn{-m}{D}{\Deltahml \vh} \leq \Cshiftfn{m}\shiftn{m} \NHnfn{-(m+2)}{D}{\Deltahmlpo \vh} + \Cm \errn{m} \frac{\NLiD{n}}{\nmin} h^{m+2} \NLtDn{\Deltahmlpo \vh}
\eeq
as $\Deltahml = \DeltahI \Deltahmlpo$. We now use \cref{eq:lrecursion} recursively to bound $\Nfn{j,h,n}{\vh}$, and adopt the notation $\Rn = \max\set{\shiftn{m}\st m = -1,\ldots,p-1},$ $\En\max\set{\errn{m}\st m = -1,\ldots,p-1},$ and $\nvar = \NLiD{n}/\nmin.$

If $j = 2l,$ then one can show inductively using \cref{eq:lrecursion} that for any integer $t \in [0,l]$ that
\beq\label{eq:evenrecursivesum}
\Nfn{-2l,h,n}{\vh} \leq \mleft(\En\nvar\mright)^t\sum_{m=0}^t \Efn{m,t} h^{2m} \NHDfn{-2(t-m)}{\vh} ,
\eeq
\optodo{Fix creflabelformat for equations in subscripts}
where we define the $\Efn{m,t}$ inductively by
\begin{align}
\label{eq:Emt1}\Efn{0,0} &=1\\
\label{eq:Emt2}\Efn{m,t} &= \Cshiftfn{2(t-1-m)} \Efn{m,t-1} \tfor 0 \leq m \leq t-1 \quad\tand\\
\label{eq:Emt3}\Efn{t,t} &= \sum_{m=0}^{t-1} \Csumrecfn{2(t-1-m)} \Efn{m,t-1}.
\end{align}
To see this recurrence, we prove the inductive step: suppose
\beqs
\Nfn{-2l,h,n}{\vh} \leq \mleft(\En\nvar\mright)^{t-1}\sum_{m=0}^{t-1} \Efn{m,t-1} h^{2m} \NHDfn{-2(t-1-m)}{\vh}.
\eeqs
Then using \cref{eq:lrecursion}, we have
\begin{align*}
\Nfn{-2l,h,n}{\vh} &\leq \mleft(\En\nvar\mright)^{t-1}\sum_{m=0}^{t-1} \Efn{m,t-1} h^{2m} \mleft(\Cshiftfn{2(t-1-m)}\shiftn{2(t-1-m)} \NHnfn{-(2(t-1-m)+2}{D}{\Deltah^{-l+t} \vh}\mright.\\
&\quad\quad\mleft.+ \Csumrecfn{2(t-1-m)} \errn{2(t-1-m)} \nvar h^{2(t-1-m)+2} \NLtDn{\Deltah^{-l+t} \vh}\mright),
\end{align*}
which upon rearranging, and using the fact that $\shiftn{2(t-1-m)} \leq \errn{2(t-1-m)}$ and $\nvar \geq 1,$ yields \cref{eq:evenrecursivesum}, with the recurrence \cref{eq:Emt1,eq:Emt2,eq:Emt3}.

If $j=2l+1,$ then we first reduce $\Nfn{-j,h}{\vh}$ to a point analagous to the even case, and then proceed as before. Let $\wh$ and $\zh$ be as at the beginning of the proof, and let $z$ solve the variational formulation\footnote{We use the variational formulation here, as we will need to bound the $H^1$-norm of $z$ by the $H^{-1}$-norm of $\wh$, which is immediate from the Lax--Milgram theorem.}  of \cref{eq:sdeq,eq:sddbc,eq:sdnbc} (with $\ftilde = \wh$). Observe that we still have $\zh = \Ph z,$ and

\beq\label{eq:LMHmo}
\NHfn{1}{D}{z} \leq \frac1{\Amin} \NHfn{-1}{D}{n\wh}
\eeq
by the Lax--Milgram Theorem. We have
\begin{align}
\NLtDn{\Deltah^{-1/2}\wh} &= \NLtDn{\Deltah^{1/2} \zh}\nonumber\\
&= \IPLtDn{\Deltah \zh}{\zh} \text{ by \cref{eq:feiptrans}}\\
&= \IPLtD{A \grad \zh}{\grad \zh}\nonumber\\
&= \NLiDop{A} \NHoD{\Ph z}\nonumber\text{ by the Cauchy--Schwartz inequality and the definition of } \zh\\
&\leq \NLiDop{A}\mleft(\NHoD{z} + \Cprojfn{1}\NHoD{0 - z}\mright) \text{ by \cref{lem:ellprojerr}}\nonumber\\
&\leq \frac{\mleft(1+\Cprojfn{1}\mright)\NLiDop{A}}{\Amin}  \NHmoD{n\wh}\text{ by \cref{eq:LMHmo}}\label{eq:deltahhalf}\\
&\leq \frac{\mleft(1+\Cprojfn{1}\mright)\NLiDop{A}}{\Amin}\NLiD{n}  \NHnfn{-1}{D}{\wh}\text{ as in the proof of \cref{lem:shiftnegativew}}\nonumber
\end{align}
We now return to
\begin{align*}
\Nfn{j,h,n}{\vh} &= \NLtDn{\Deltah^{-l-1/2} \vh} = \NLtDn{\Deltah^{-1/2} \Deltah^{-l} \vh}\\
&\leq \frac{\mleft(1+\Cprojfn{1}\mright)\NLiDop{A}}{\Amin}\NLiD{n}\NHnfn{-1}{D}{\Deltah^{-l}\vh}
\end{align*}
by \cref{eq:deltahhalf}.

Similarly to \cref{eq:evenrecursivesum}, one can use \cref{eq:lrecursion} recursively to show that, for any integer $t \in [0,l]$
\beq\label{eq:oddrecursive}
\NHnfn{-1}{D}{\Deltah^{-l}\vh} \leq \mleft(\En \nvar\mright)^t\mleft(\Etildefn{0,t} \NHnfn{-(2t+1)}{D}{\Deltah^{-l+t}\vh} + \sum_{m=0}^t \Etildefn{m,t} h^{2m+1}  \NHnfn{-2(t-m)}{D}{\Deltah^{-l+t}\vh}\mright),
\eeq
where  we define the $\Etildefn{m,t}$ inductively, for $t \in [0,l]$ by
\begin{align}
\label{eq:Etilde1}\Etildefn{0,0} &= 1\\
%\label{eq:Etilde2}\Etildefn{1,1} &= \Csumrecfn{1}\\
\label{eq:Etilde3}\Etildefn{0,t} &= \Etildefn{0,t-1}\Cshiftfn{2(t-1)+1)}\\
\label{eq:Etilde4}\Etildefn{m,t} &= \Etildefn{m,t-1}\Cshiftfn{2(t-1-m)}\tfor m = 1,\ldots,t-1\\
\label{eq:Etilde5}\Etildefn{t,t} &= \Csumrecfn{2(t-1)+1} + \sum_{m=0}^{t-1}\Etildefn{m,t-1}\Csumrecfn{2(t-1-m)}.\\
\end{align}
To show \cref{eq:oddrecursive,eq:Etilde1,eq:Etilde3,eq:Etilde4,eq:Etilde5}, observe that \cref{eq:Etilde1} is immediate; we now show \cref{eq:oddrecursive,eq:Etilde3,eq:Etilde4,eq:Etilde5} by induction: suppose %\cref{eq:oddrecursive} holds for $t-1,$ then
\beqs
\NHnfn{-1}{D}{\Deltah^{-l}\vh} \leq \mleft(\En \nvar\mright)^{t-1}\mleft(\Etildefn{0,t-1} \NHnfn{-(2(t-1)+1)}{D}{\Deltah^{-l+t-1}\vh} + \sum_{m=0}^{t-1} \Etildefn{m,t-1} h^{2m+1}  \NHnfn{-2(t-1-m)}{D}{\Deltah^{-l+t-1}\vh}\mright).
\eeqs
Then
\begin{align*}
\NHnfn{m}{D}{\Deltah^{-l}\vh} \leq& \mleft(\En \nvar\mright)^{t-1} \Etildefn{0,t-1} \mleft(\Cshiftfn{2(t-1)+1} \shiftn{2(t-1)+1}\NHnfn{-(2(t-1)+1)}{D}{\Deltah^{-l + t}\vh}\mright.\\
&\hphantom{\mleft(\En \nvar\mright)^{t-1} \Etildefn{0,t-1}}\quad\quad\mleft.+\Csumrecfn{2(t-1)t+1} \errn{2(t-1)+1}\nvar h^{2t+1} \NLtDn{\Deltah^{-l + t}\vh}\mright)\\
&\quad\quad+ \mleft(\En \nvar\mright)^{t-1}\sum_{m=1}^{t-1} \Etildefn{m,t-1} h^{2m+1} \mleft(\Cshiftfn{2(t-1-m)} \shiftn{2(t-1-m)}\NHnfn{-2(t-1-m)}{D}{\Deltah^{-l + t}\vh}\mright.\\
&\mleft.\hphantom{\quad\quad+ \mleft(\En \nvar\mright)^{t-1}\sum_{m=1}^{t-1} \Etildefn{m,t-1} h^{2m+1}}\quad\quad+ h^{2(t-m)} \Csumrecfn{2(t-1-m)}\errn{2(t-1-m)}\nvar \NLtDn{\Deltah^{-l + t}\vh}\mright)\\
%% &= \Etildefn{0,t-1} \CAfn{2t-1} \NHfn{-(2(t-1)+1)}{D}{\Deltah^{-l + t}\vh}\\
%% &\quad\quad+ \mleft(\Csumrecfn{2t-1}\sum_{m=1}^{t-1} \Etildefn{m,t-1}\CAfn{2(t-1-m)}\mright) h^{2t+1}\NLtD{\Deltah^{-l + t}\vh}\\
%% &\quad\quad+ \sum_{m=1}^{t-1} \Etildefn{m,t-1} h^{2m+1} \CAfn{2(t-1-m)}\NHfn{-2(t-m)}{D}{\Deltah^{-l + t}\vh},
\end{align*}
and rearranging, and using the facts that $\shiftn{m} \leq \errn{m}$ for all $m$ and $\nvar \geq 1,$ we obtain \cref{eq:oddrecursive} with the constants $\Etildefn{m,t}$ given by \cref{eq:Etilde1,eq:Etilde3,eq:Etilde4,eq:Etilde5}. Therefore, we conclude that if $j = 2l+1$
\begin{align}
\Nfn{-j,h,n}{\vh} \leq& \frac{\mleft(1+\Cprojfn{1}\mright)\NLiDop{A}}{\Amin}\NLiD{n}\mleft(\En \nvar\mright)^{l}\mleft(\Etildefn{0,l} \NHnfn{-(2l+1)}{D}{\vh}\mright.\nonumber\\
&\mleft.\hphantom{\frac{\mleft(1+\Cprojfn{1}\mright)\NLiDop{A}}{\Amin}\NLiD{n}\mleft(\En \nvar\mright)^{l}}
\quad\quad+ \sum_{m=0}^{l} \Etildefn{m,l} h^{2m+1}  \NHnfn{-2(l-m)}{D}{\vh}\mright)\label{eq:oddfinal}
\end{align}
and combinining \cref{eq:evenrecursivesum} (with $t=l$) and \cref{eq:oddfinal}, and letting $\Csumj$ be as in \cref{app:constants}, the result follows.%\optodo{Maybe put recursion in, but it's all in notes.}
\epf

\subsection{Main finite-element-error bound}\label{sec:fembound}

Having established the need results on discrete Sobolev spaces, we are now in a position to prove our main theorem, \cref{thm:fembound}, which we do via a series of lemmas.

Two quantities that are key in the proof are
\beqs
\rho \de u - \Ph u, \tand
\eeqs
\beqs
\thetah \de \Ph u - \uh.
\eeqs
The main idea of the proof is to decompose the error $u - \uh = \rho + \thetah,$ use bounds on the elliptic projection operator to bound $\rho,$ and then bound $\thetah$ (in higher-order discrete norms) in terms of $\rho.$

The following \lcnamecref{lem:simpleform} sets up the argument in the lemmas that follow.
\ble[Expression for $a(\thetah,\vh)$]\label{lem:simpleform}
For any $\vh \in \Vhp,$
\beq\label{eq:thetaform}
a(\thetah,\vh) = k^2\IPLtDn{\Qhn\rho}{\vh} + ik \IPLtGI{\rho}{\vh}.
\eeq
\ele

\bpf[Proof of \cref{lem:simpleform}]
Let $\vh \in \Vhp.$ Then $a(\thetah,\vh) = a(u-\uh,\vh) - a(\rho,\vh) = -a(\rho,\vh)$ by Galerkin orthogonality. By definition of $a,$ we have $-a(\rho,\vh) = -\IPLtD{A\grad\mleft(u-\Ph u\mright)}{\vh} + k^2 \IPLtD{n\rho}{\vh} + ik\IPLtGI{\rho}{\vh}.$ By Galerkin orthogonality for the elliptic projection, $\IPLtD{A\grad\mleft(u-\Ph u\mright)}{\vh} = 0$, and so by the definition of the $n$-weighted $L^2$ inner product, and the $n$-weighted $L^2$-projection $\Qhn,$ the result follows.
\epf

\ble[Bound on $\NLtGI{\thetah}$ by $\Npmoh{\thetah}$]\label{lem:boundarybound}
Under the assumptions of \cref{thm:fembound}, we have
\beq\label{eq:boundarybound}
\NLtGI{\thetah}^2 \leq \Cboundaryo \mleft(\En\nvar\mright)^{2\mleft(\floor{\frac{p-1}2}+1\mright)}\NLiD{n}^4 k^2 h^{2p-1} \Nfn{p-1,h,n}{\thetah}^2 + \Cboundaryt h \NW{\rho}^2,
\eeq
\ele

\bpf[Proof of \cref{lem:boundarybound}]
In \cref{eq:thetaform}, let $\vh = \thetah,$ and take the imaginary part to obtain
\beqs
-k \NLtGI{\thetah}^2 \leq \Im k^2 \IPLtDn{\Qhn \rho}{\thetah} + \Re k \IPLtGI{\rho}{\thetah},
\eeqs
and therefore by \cref{cor:ipdiscbound}
\beq
\NLtGI{\thetah}^2 \leq  k \Nfn{1-p,h,n}{\Qhn \rho}\Nfn{p-1,h,n}{\thetah} + \NLtGI{\rho}\NLtGI{\thetah}.\label{eq:thetaboundarypart}
%% \nonumber\\
%% &\leq  k \NLiD{n}^2 \Nfn{1-p,h}{\Qhn \rho}\Nfn{p-1,h}{\thetah} + \NLtGI{\rho}\NLtGI{\thetah}
\eeq
We first bound the negative norm $\Nfn{1-p,h}{\Qhn\rho}$, to do this we use \cref{lem:negdiscsum}; however, we therefore need to estimate negative Sobolev norms of $\Qhn\rho$; for integers $m \in [0,p-1]$ we have (observing that $\Qhn \Ph u = \Ph u$ as $\Ph u \in \Vhp.$
\begin{align}
\NHnfn{-(p-1-m)}{D}{\Qhn\rho} &\leq \NHnfn{-(p-1-m)}{D}{\Qhn u - u} + \NHnfn{-(p-1-m)}{D}{u - \Ph u}\nonumber\\
&\leq \CSZfn{(p-1-m)} \nvar h^{(p-1-m)} \NLtDn{u-\Ph u} + \Cwfn{-(p-1-m)}\errn{(p-1-m)} h^{p-m} \NHoDn{u - \Ph u}\nonumber\\
&\quad\quad\text{ by \cref{lem:ellprojerrw,lem:wltdprojerr}, taking } \wh = \Ph u \text{ in \cref{lem:ellprojerrw,eq:wltdprojerr}}\nonumber\\
&\leq \mleft(\CSZfn{(p-1-m)} \Cwz + \Cwfn{-(p-1-m)}\mright)\En\nvar\NLiD{n} h^{p-m} \NW{\rho} \text{ by \cref{lem:ellprojerrw}}\label{eq:Qhnrhoneg}
\end{align}
By \cref{lem:negdiscsum,eq:Qhnrhoneg} we have
\begin{align}
\Nfn{1-p,h,n}{\Qhn \rho}&\leq \mleft(\En\nvar\mright)^{\floor{\frac{p-1}2}}\NLiD{n} \Csumfn{p-1} \sum_{m=0}^{p-1} h^{m} \NHfn{-(p-1-m)}{D}{\Qhn \rho} \nonumber\\
&\leq \Cmess \mleft(\En\nvar\mright)^{\floor{\frac{p-1}2}+1}\NLiD{n}^2 h^p \NW{\rho}.\label{eq:Qhnrhosum}
\end{align}
To deal with the second term on the right-hand side of \cref{eq:thetaboundarypart} we use \cref{thm:multiplicativetrace,lem:ellprojerr}, taking $\wh = \Ph u$ in \cref{eq:ellprojerr} and the fact that $\NHoD{\cdot} \leq \NW{\cdot}$ to obtain
\beq\label{eq:rhomtbound}
\NLtGI{\rho} \leq \CMT\NHoD{\rho}^{1/2}\NLtD{\rho}^{1/2} \leq \CMT \Cprojfn{0}^{\half} h^\half \NHoD{\rho}\leq \CMT \Cprojfn{0}^{\half} h^{\half} \NW{\rho}.
\eeq
Therefore by \cref{eq:rhomtbound} and Young's inequality, we obtain
\beq\label{eq:rhothetamt}
\NLtGI{\rho}\NLtGI{\thetah} \leq \half \CMT^2 \Cprojfn{0} h \NW{\rho}^2 + \half \NLtGI{\thetah}^2.
\eeq
By combining \cref{eq:thetaboundarypart,eq:Qhnrhosum,eq:rhothetamt} we have
\beq\label{eq:thetahboundnear}
\NLtGI{\thetah}^2 \leq k \Cmess \mleft(\En\nvar\mright)^{\floor{\frac{p-1}2}+1}\NLiD{n}^2 h^p \NW{\rho}\Nfn{p-1,h,n}{\thetah} + \half \CMT^2 \Cprojfn{0} h\NW{\rho}^2 + \half \NLtGI{\thetah}^2
\eeq
By using Young's inequality on the first term in \cref{eq:thetahboundnear}, and moving the $\NLtGI{\thetah}^2$ term onto the left-hand side, we obtain \cref{eq:boundarybound}.
\epf

\ble[Bound on higher-order discrete norms of $\thetah$ by $\NLtD{\thetah}$]\label{lem:higherbound}
Under the assumptions of \cref{thm:fembound}, for integer $m \in [1,p]$ there exist constants $\Chighmo,$ $\Chighmt > 0$ such that
\begin{align}
\Nfn{m,h,n}{\thetah} &\leq \Chighmo \mleft(\mleft(\En\nvar\mright)^{\half(\floor{\frac{p-1}2}+1)}\NLiD{n}\nmin^{-\frac{p}2}\mright)^mk^m \NLtD{\thetah}\nonumber\\
&\quad\quad+ \Chighmt \nvar^3\NLiD{n}\nmin^{1-m} h^{1-m} \NW{\rho}.\label{eq:chigh}
\end{align}
\ele
\bpf[Proof of \cref{lem:higherbound}]
By inserting the definitions of $a$ and $\Deltah$ in \cref{eq:thetaform} and rearranging, we have for any $\vh \in \Vhp$
\beqs
\IPLtDn{\Deltah \thetah}{\vh} = k^2 \IPLtDn{\thetah}{\vh} + k^2\IPLtDn{\Qhn \rho}{\vh} + ik \IPLtGI{\thetah}{\vh} + ik \IPLtGI{\rho}{\vh}.
\eeqs
Therefore, if we take $\vh = \Deltah^{m-1}\thetah$, by \cref{lem:intoip} we have
\beq\label{eq:deltahm}
\Nfn{m,h,n}{\thetah}^2 = k^2 \Nfn{m-1,h,n}{\thetah}^2 + k^2 \IPLtDn{\Deltah^{\frac{m-1}2} \Qhn \rho}{\Deltah^{\frac{m-1}2}\thetah} + ik\IPLtGI{\thetah}{\Deltah^{m-1} \thetah} + ik \IPLtGI{\rho}{\Deltah^{m-1} \thetah}.
\eeq
We now proceed to bound the two terms in \cref{eq:deltahm} defined on the truncation boundary $\GI.$ For the first term, we have
\begin{align}
\IPLtGI{\thetah}{\Deltah^{m-1}\thetah} &\leq \NLtGI{\thetah}\NLtGI{\Deltah^{m-1}\thetah}\nonumber\\
&\leq \CMT \CinvVhp^{1/2} \NLtGI{\thetah} h^{-\half} \NLtD{\Deltah^{m-1} \thetah}\nonumber\\
&\quad\quad\text{ by \cref{thm:multiplicativetrace,lem:inverseinequality}}\nonumber\\
&= \CMT \CinvVhp^{1/2} h^{-\half}\NLtGI{\thetah}\nmin^{-1}\Nfn{2m-2,h,n}{\thetah}\text{ by the definition of }\Nfn{2m-2,h,n}{\cdot}\nonumber\\
&\leq \CMT \CinvVhp^{1/2} \Chinv^{m-1} h^{-m+\half} \nmin^{-m} \NLtGI{\thetah}\Nfn{m-1,h,n}{\thetah} \text{ by \cref{lem:inversediscrete} applied } m-1 \text{ times}\label{eq:useitagain}\\
&\leq \CMT \CinvVhp^{1/2} \Chinv^{m-1} h^{-m+\half} \nmin^{-m}\nonumber\\
&\quad\quad\mleft(\Cboundaryo^{\half} \mleft(\En\nvar\mright)^{\floor{\frac{p-1}2}+1}\NLiD{n}^2 k h^{p-\half} \Nfn{p-1,h,n}{\thetah} + \Cboundaryt^{\half} h^{\half} \NW{\rho}\mright)\Nfn{m-1,h,n}{\thetah}\nonumber\\
&\quad\quad\text{ by \cref{lem:boundarybound} and \cref{eq:simple}}\nonumber\\
&\leq \mleft(\CBo \mleft(\En\nvar\mright)^{\floor{\frac{p-1}2}+1}\NLiD{n}^2\nmin^{-p}k \Nfn{m-1,h,n}{\thetah} + \CBt \nmin^{-m}h^{1-m} \NW{\rho}\mright)\Nfn{m-1,h,n}{\thetah}\label{eq:firstboundary}
\end{align}
by \cref{lem:inversediscrete} applied $p-m$ times.
To bound the second boundary term in \cref{eq:deltahm}, we have
\beq\label{eq:secondboundarytemp}
\IPLtGI{\rho}{\Deltah^{m-1} \thetah} \leq \CMT \CinvVhp^{1/2} \Chinv^{m-1}\nmin^{-m}h^{\half-m}\Nfn{m-1,h}{\thetah}\NLtGI{\rho}
\eeq
using the same reasoning as we used to obtain \cref{eq:useitagain} above. By \cref{thm:multiplicativetrace,lem:ellprojerr} (with $\wh = \Ph u$) we have
\beq\label{eq:secondboundarytemp2}
\NLtGI{\rho} \leq \CMT \Cprojfn{0}^{\half} h^{\half} \NW{\rho}.
\eeq

Inserting \cref{eq:secondboundarytemp2} into \cref{eq:secondboundarytemp} we obtain
\beq\label{eq:secondboundary}
\IPLtD{\rho}{\Deltah^{m-1}\thetah} \leq \CBth \nmin^{-m}h^{1-m} \NW{\rho} \Nfn{m-1,h}{\thetah}.
\eeq

Therefore, from \cref{eq:deltahm,eq:firstboundary,eq:secondboundary} and the Cauchy--Schwartz inequality, we have
\begin{align*}
\Nfn{m,h}{\thetah}^2 &\leq k^2 \Nfn{m-1,h}{\thetah}^2 + k^2 \Nfn{m-1,h,n}{\Qhn \rho}\Nfn{m-1,h,n}{\thetah}\nonumber\\
&\quad\quad+ k\mleft(\CBo \mleft(\En\nvar\mright)^{\floor{\frac{p-1}2}+1}\NLiD{n}^2\nmin^{-p}k \Nfn{m-1,h,n}{\thetah} + \CBt \nmin^{-m}h^{1-m} \NW{\rho}\mright)\Nfn{m-1,h,n}{\thetah}\nonumber\\
&\quad\quad+ k\CBth \nmin^{-m}h^{1-m} \NW{\rho} \Nfn{m-1,h}{\thetah}%\label{eq:thetahighnearlynearly}.
\end{align*}
Therefore using Young's inequality we have
\begin{align*}
\Nfn{m,h}{\thetah}^2 &\leq k^2\mleft(\frac32 + \CBo \mleft(\En\nvar\mright)^{\floor{\frac{p-1}2}+1}\NLiD{n}^2\nmin^{-p} + \half \CBt \nmin^{-m} + \half\CBth \nmin^{-m}\mright)\Nfn{m-1,h,n}{\thetah}^2\\
&\quad\quad+ \frac{k^2}2 \Nfn{m-1,h,n}{\Qhn \rho}^2 +  h^{2(1-m)} \NW{\rho}^2,
\end{align*}
and therefore by \cref{eq:simple}
\begin{align}
\Nfn{m,h}{\thetah} &\leq k\mleft(\sqrt{\frac32} + \CBo^{\half} \mleft(\En\nvar\mright)^{\half(\floor{\frac{p-1}2}+1)}\NLiD{n}\nmin^{-\frac{p}2} + \frac1{\sqrt{2}} \CBt^{\half} \nmin^{-\frac{m}2} + \frac1{\sqrt{2}}\CBth^{\half} \nmin^{-\frac{m}2}\mright)\Nfn{m-1,h,n}{\thetah}\nonumber\\
&\quad\quad+ \frac{k}{\sqrt{2}} \Nfn{m-1,h,n}{\Qhn \rho} +  h^{1-m} \NW{\rho},\label{eq:thetahighnearly}
\end{align}

We now proceed to bound $\Nfn{m-1,h,n}{\Qhn \rho}$: By \cref{lem:inversediscrete} and the fact that $\Qhn \Ph u = \Ph u$ (as $\Ph u \in \Vhp$), we have
\beq\label{eq:qhnmmo}
\Nfn{m-1,h,n}{\Qhn \rho} \leq \Chinv^{m-1} \nmin^{1-m}h^{1-m} \mleft(\NLtDn{\Qhn u - u} + \NLtDn{u-\Ph u}\mright).
\eeq

Using \cref{lem:wltdprojerr} we can bound the first of these terms by the second, $\NLtDn{\Qhn u - u} \leq \CSZfn{0} \nvar \NLtDn{u-\Ph u}.$ We can also bound $\NLtDn{u - \Ph u} \leq \Cwfn{0} \nvar^3h \NHnfn{1}{D}{\rho} \leq \Cwfn{0} \nvar^3\NLiD{n} h \NW{\rho}$ by \cref{lem:ellprojerrw}. Therefore by \cref{eq:qhnmmo} we have (as $kh \leq 1$ and $\nvar \geq 1$)
\beq\label{eq:qhnnearly}
k \Nfn{m-1,h}{\Qhn \rho} \leq \Chinv^{m-1}\Cwz \mleft(1+\CSZfn{0}\mright)\nvar^3\NLiD{n}\nmin^{1-m} h^{1-m}\NW{\rho}.
\eeq
Therefore using \cref{eq:thetahighnearly,eq:qhnnearly} (and the fact that $\NLiD{n}, \nmin^{-1} \geq 1,$ and so $\En, \nvar \geq1$)  we have

\beq\label{eq:readytorecurse}
\Nfn{m,h}{\thetah} \leq \CRecofn{m} \mleft(1+\mleft(\En\nvar\mright)^{\half(\floor{\frac{p-1}2}+1)}\NLiD{n}\nmin^{-\frac{p}2}\mright)k \Nfn{m-1,h}{\thetah} + \CRect \nvar^3\NLiD{n}\nmin^{1-m} h^{1-m} \NW{\rho}.
\eeq
Using \cref{eq:readytorecurse} recursively, and the facts that $\nmin^{-1} \geq 1$ and $hk \leq 1,$ we obtain \cref{eq:chigh}.
\epf

The following \lcnamecref{lem:continuity} is straightforward to prove, and used in the proof of \cref{lem:ltthetahbound} below
\ble[Continuity of $a$]\label{lem:continuity}
For any $\vo, \vt \in \HozDD,$
\beqs
\abs{a(\vo,\vt)} \leq \Cc \NLiD{n} \NW{\vo}\NW{\vt}.
\eeqs
\ele

\ble[Bound $\NLtDn{\thetah}$ by $\Nfn{p-1,h,n}{\thetah}$]\label{lem:ltthetahbound}
Under the assumptions of \cref{thm:fembound}, there exist constants $\Cfirst, \Csecond > 0$ such that
\begin{align}
\NLtD{\thetah}&\leq \mleft(\Cfirst\NW{\rho} +\Csecond  k^2h^p\Nfn{p-1,h,n}{\thetah}\mright)\mleft(\En \nvar\mright)^{\floor{\frac{p-1}2}+1}\NLiD{n}^2\nonumber\\
&\quad\quad\Pfn{p-2}\mleft(\NLiD{n}\mright)\mleft(\CFEMotilde h + \CFEMttilde \CAnk (hk)^p\mright)\label{eq:ltthetahbound}
\end{align}
\ele

\bpf[Proof of \cref{lem:ltthetahbound}]
The proof initially uses a standard duality technique, but then becomes more complex than standard proofs, as we are bounding $\thetah$ by its higher-order discrete norms, rather than the error $\eh$ by its $H^1$ norm.

Consider the adjoint variational problem: Find $w \in \HozDD$ such that for all $v \in \HozDD$
\beq\label{eq:adjointtheta}
a(v,w) = \IPLtDn{v}{\thetah}
\eeq
(i.e., $w$ solves the adjoint problem with right-hand side given by $n\thetah$). Let $\eh \de u -\uh$ be the finite-element error, and put $v = \eh$ in \cref{eq:adjointtheta}. By Galerkin orthogonality for $\eh$ and $u-\Ph u,$ we have (recalling $\eh = \rho + \thetah$)
\begin{align}
\IPLtDn{\eh}{\thetah} = a(\eh,w-\Ph w) &= \IPLtD{A \grad \rho}{\grad(w-\Ph w)}  -k^2 \IPLtDn{\eh}{w-\Ph w} -ik \IPLtGI{\eh}{w-\Ph w},\nonumber\\
&=a(\rho,w-\Ph w) -k^2 \IPLtDn{\thetah}{w-\Ph w} -ik \IPLtGI{\thetah}{w-\Ph w}\label{eq:doublego}
\end{align}

Therefore as $\eh = \rho + \thetah$ we can rearrange \cref{eq:doublego} and use Cauchy-Schwartz to obtain
\begin{align}
\NLtDn{\thetah}^2 &\leq \Cc\NLiD{n} \NW{\rho} \NW{w- \Ph w} + k^2 \abs{\IPLtDn{\thetah}{w- \Ph w}}\nonumber\\
&+ k \abs{\IPLtGI{\thetah}{w-\Ph w}} + \NLtDn{\rho}\NLtDn{\thetah}\label{eq:boundingLtwo}
\end{align}
By combining \cref{lem:bestapprox,lem:ellprojerr}, we can show (as $w$ satisfies an adjoint Helmholtz problem with right-hand side $\thetah$)
\beq\label{eq:wlt}
\NLtD{w - \Ph w} \leq \Cprojfn{0} \Pfn{p-2}\mleft(\NLiD{n}\mright)\mleft(\CFEMotilde h^2 + \CFEMttilde \CAnk h(hk)^p\mright)\NLtD{\thetah} \tand
\eeq
\beq\label{eq:who}
\NHoD{w - \Ph w} \leq 2\Cprojfn{-1} \Pfn{p-2}\mleft(\NLiD{n}\mright)\mleft(\CFEMotilde h + \CFEMttilde \CAnk (hk)^p\mright)\NLtD{\thetah}.
\eeq
We will be able to use \cref{eq:wlt,eq:who} to bound terms involving $w - \Ph w$ in \cref{eq:boundingLtwo}. We first estimate the inner product terms in \cref{eq:boundingLtwo}:
\begin{align}
\abs{\IPLtDn{\thetah}{w- \Ph w}} &= \abs{\IPLtDn{\thetah}{\Qhn w - \Ph w}}\nonumber\\
&\leq \Nfn{p-1,h,n}{\thetah}\Nfn{1-p,h,n}{\Qhn w - \Ph w}\text{ by \cref{lem:intoip}}\nonumber\\
&\leq \Nfn{p-1,h,n}{\thetah} \Csumfn{p-1} \mleft(\En \nvar\mright)^{\floor{\frac{p-1}2}}\NLiD{n}\nonumber\\
&\quad\quad\sum_{m=0}^{p-1} h^{m}\mleft(\NHnfn{-(p-1-m)}{D}{\Qhn w - w}+ \NHnfn{-(p-1-m)}{D}{w - \Ph w}\mright) \text{ by \cref{lem:negdiscsum}}\nonumber\\
&\leq \Nfn{p-1,h,n}{\thetah} \Csumfn{p-1}\mleft(\En \nvar\mright)^{\floor{\frac{p-1}2}}\NLiD{n} h^{p-1}\nonumber\\
&\quad\quad\sum_{m=0}^{p-1} \mleft(\CSZfn{p-1-m}\nvar \NLtDn{w - \Ph w} + \Cwfn{p-1-m} \errn{p-1-m} h \NHoDn{w - \Ph w}\mright)\nonumber\\
&\quad\quad\text{ by \cref{lem:wltdprojerr,lem:ellprojerrw}, taking $\wh = \Ph u$ in \cref{eq:wltdprojerr,lem:ellprojerrw}}\nonumber\\
&\leq \Nfn{p-1,h,n}{\thetah} \Csumfn{p-1}\sum_{m=0}^{p-1} \mleft(\CSZfn{p-1-m}\Cprojfn{0} + \Cwfn{p-1-m} \Cprojfn{-1}\mright)                         \mleft(\En \nvar\mright)^{\floor{\frac{p-1}2}+1}\NLiD{n}^2 \nonumber\\
&\quad\quad h^p \Pfn{p-2}\mleft(\NLiD{n}\mright)\mleft(\CFEMotilde h + \CFEMttilde \CAnk (hk)^p\mright) \NLtD{\thetah} \text{ by \cref{eq:wlt,eq:who}}.\nonumber\\
&=\Nfn{p-1,h,n}{\thetah}  \Cfourteen \mleft(\En \nvar\mright)^{\floor{\frac{p-1}2}+1}\NLiD{n}^2 h^p \Pfn{p-2}\mleft(\NLiD{n}\mright)\nonumber\\
&\quad\quad\mleft(\CFEMotilde h + \CFEMttilde \CAnk (hk)^p\mright) \NLtD{\thetah} \label{eq:innerprod1}.
\end{align}

We now estimate the other inner product term
\begin{align}
\abs{\IPLtGI{\thetah}{w - \Ph w}} &\leq \CMT\CinvVhp\mleft(\Cboundaryo^{\half} \mleft(\En\nvar\mright)^{\floor{\frac{p-1}2}+1}\NLiD{n} k h^{p-\half} \Nfn{p-1,h,n}{\thetah}\mright.\nonumber\\
&\mleft.\quad\quad+ \Cboundaryt^{\half} h^{\half} \NW{\rho}\mright) h^{-\half}\NLtD{w - \Ph w}\nonumber\\
&\quad\quad\text{ by \cref{lem:boundarybound,eq:simple,thm:multiplicativetrace,lem:inverseinequality}}\nonumber\\
&\leq \CMT \CinvVhp \Cprojfn{0}\mleft(\Cboundaryo^{\half} \mleft(\En\nvar\mright)^{\floor{\frac{p-1}2}+1}\NLiD{n} k h^{p} \Nfn{p-1,h,n}{\thetah} + \Cboundaryt^{\half} h \NW{\rho}\mright)\nonumber\\
&\quad\quad \Pfn{p-2}\mleft(\NLiD{n}\mright)\mleft(\CFEMotilde h + \CFEMttilde \CAnk (hk)^p\mright)\NLtD{\thetah}\label{eq:innerprod2}
\end{align}
 by \cref{eq:wlt}.

Now insert \cref{eq:wlt,eq:who,eq:innerprod1,eq:innerprod2} into \cref{eq:boundingLtwo}:
\begin{align*}
\NLtD{\thetah}^2 &\leq \Bigg[\Cc\NLiD{n} \NW{\rho} \mleft(\Cprojfn{0} + 2\Cprojfn{1}\mright)\\
&+k^2\Nfn{p-1,h,n}{\thetah}  \Cfourteen \mleft(\En \nvar\mright)^{\floor{\frac{p-1}2}+1}\NLiD{n}^2 h^p \\
&+k\CMT \CinvVhp \Cprojfn{0} \mleft(\Cboundaryo^{\half} \mleft(\En\nvar\mright)^{\floor{\frac{p-1}2}+1}\NLiD{n} k h^{p} \Nfn{p-1,h,n}{\thetah} + \Cboundaryt^{\half} h \NW{\rho}\mright)\Bigg]\nonumber\\
&\quad\quad \Pfn{p-2}\mleft(\NLiD{n}\mright)\mleft(\CFEMotilde h + \CFEMttilde \CAnk (hk)^p\mright)\NLtD{\thetah}\\
&+\NLtDn{\rho}\NLtDn{\thetah}\nonumber\\
%newline
&\leq \Bigg[\mleft(\Cc\mleft(\Cprojfn{0} + 2\Cprojfn{1}\mright)\NLiD{n} + \CMT \CinvVhp \Cprojfn{0} \Cboundaryt^{\half} + \Cprojfn{0}\nvar  \mright)\NW{\rho} \\
&+\mleft(  \Cfourteen \mleft(\En \nvar\mright)^{\floor{\frac{p-1}2}+1}\NLiD{n}^2 + \CMT \CinvVhp \Cprojfn{0}\Cboundaryo^{\half} \mleft(\En\nvar\mright)^{\floor{\frac{p-1}2}+1}\NLiD{n}\mright)k^2h^p\Nfn{p-1,h,n}{\thetah}\Bigg]\nonumber\\
&\quad\quad \Pfn{p-2}\mleft(\NLiD{n}\mright)\mleft(\CFEMotilde h + \CFEMttilde \CAnk (hk)^p\mright)\NLtD{\thetah}\nonumber\\
&\text{ rearranging and using \cref{lem:ellprojerr} and the fact that $hk \leq 1$}\nonumber\\
&\leq \Bigg[\mleft(\Cc\mleft(\Cprojfn{0} + 2\Cprojfn{1}\mright) + \CMT \CinvVhp \Cprojfn{0} \Cboundaryt^{\half} + \Cprojfn{0}  \mright)\NW{\rho} \nonumber\\
&+\mleft(  \Cfourteen  + \CMT \CinvVhp \Cprojfn{0}\Cboundaryo^{\half} \mright)k^2h^p\Nfn{p-1,h,n}{\thetah}\Bigg]\nonumber\\
&\quad\quad \mleft(\En \nvar\mright)^{\floor{\frac{p-1}2}+1}\NLiD{n}^2\Pfn{p-2}\mleft(\NLiD{n}\mright)\mleft(\CFEMotilde h + \CFEMttilde \CAnk (hk)^p\mright)\NLtD{\thetah}\nonumber\\
\end{align*}
and therefore using Young's inequality to separate out the $\NLtD{\thetah}$ term on the right-hand side, and then move it to the left hand side, followed by \cref{eq:simple}, we obtain
\begin{align}
\NLtD{\thetah}&\leq \Bigg[\mleft(\Cc\mleft(\Cprojfn{0} + 2\Cprojfn{1}\mright) + \CMT \CinvVhp \Cprojfn{0} \Cboundaryt^{\half} + \Cprojfn{0}  \mright)\NW{\rho}\nonumber \\
&+\mleft(  \Cfourteen  + \CMT \CinvVhp \Cprojfn{0}\Cboundaryo^{\half} \mright)k^2h^p\Nfn{p-1,h,n}{\thetah}\Bigg]\nonumber\\
&\quad\quad \mleft(\En \nvar\mright)^{\floor{\frac{p-1}2}+1}\NLiD{n}^2\Pfn{p-2}\mleft(\NLiD{n}\mright)\mleft(\CFEMotilde h + \CFEMttilde \CAnk (hk)^p\mright)\label{eq:part3penultimate},
\end{align}
and on rearranging \cref{eq:part3penultimate} we obtain \cref{eq:ltthetahbound}.

\epf
With all our technical lemmas proved, we can now prove our main \lcnamecref{thm:fembound}.

\bpf[Proof of \cref{thm:fembound}]
By using \cref{lem:higherbound} (with $m=p-1$) in \cref{eq:ltthetahbound} and the fact that $\NLiD{n}, \En, \nvar \geq 1$, we have
\begin{align}
\NLtD{\thetah}&\leq \mleft(\Cfirst + \Csecond\Chighfn{p-1,2} \nvar^3\NLiD{n}\nmin^{2-p} \mright)\mleft(\En \nvar\mright)^{\floor{\frac{p-1}2}+1}\NLiD{n}^2\nonumber\\
&\quad\quad\Pfn{p-2}\mleft(\NLiD{n}\mright)\mleft(\CFEMotilde h + \CFEMttilde \CAnk (hk)^p\mright)\NW{\rho}\nonumber\\
&+\Csecond  \Chighfn{p-1,1} \mleft(\mleft(\En\nvar\mright)^{\half(\floor{\frac{p-1}2}+1)}\NLiD{n}\nmin^{-\frac{p}2}\mright)^{p-1}\mleft(\En \nvar\mright)^{\floor{\frac{p-1}2}+1}\NLiD{n}^2\nonumber\\
&\quad\quad\Pfn{p-2}\mleft(\NLiD{n}\mright)\mleft(\CFEMotilde (hk )^{p+1}+ \CFEMttilde \CAnk h^{2p}k^{2p+1}\mright)\NLtD{\thetah}\label{eq:doublehkdep}
\end{align}
Choosing $h$ according to \cref{eq:hfemcond}, \cref{eq:doublehkdep} simplifies to
\begin{align*}
\NLtD{\thetah} &\leq \mleft(\Cfirst + \Csecond\Chighfn{p-1,2} \nvar^4\nmin^{1-p} \mright)\mleft(\En \nvar\mright)^{\floor{\frac{p-1}2}+1}\NLiD{n}^2\\
&\quad\quad\Pfn{p-2}\mleft(\NLiD{n}\mright)\mleft(\CFEMotilde h + \CFEMttilde \CAnk (hk)^p\mright)\NW{\rho} + \half \NLtD{\thetah},
\end{align*}

and therefore it follows that
\beq\label{eq:ltboundwithrho}
\NLtD{\thetah} \leq \nvar^6\nmin^{-(p+1)} \mleft(\En \nvar\mright)^{\floor{\frac{p-1}2}+1} \Pfn{p-2}\mleft(\NLiD{n}\mright)\mleft(\CLtboundo h + \CLtboundt  \CAnk (hk)^p\mright)\NW{\rho}.
\eeq

It only now remains to bound the weighted $H^1$ norm of the error. By \cref{lem:h1contdisc,lem:higherbound} (with $m=1$) we have
\begin{align}
\SNHoD{\thetah} &\leq \Amin^{\half} \mleft[\Chighfn{1,1} \mleft(\mleft(\En\nvar\mright)^{\half(\floor{\frac{p-1}2}+1)}\NLiD{n}\nmin^{-\frac{p}2}\mright)k \NLtD{\thetah}\mright.\nonumber\\
&\mleft.\quad\quad+ \Chighmt \nvar^3\NLiD{n} \NW{\rho}\mright],\label{eq:sntheta}
\end{align}

and by combining \cref{eq:sntheta,eq:ltboundwithrho} we obtain (as $hk \leq 1$ and $\nvar,\NLiD{n} \geq 1$)
\begin{align}
\SNHoD{\thetah} &\leq
\mleft(\mleft(\En\nvar\mright)^{\half(\floor{\frac{p-1}2}+1)}\NLiD{n}\nmin^{-\frac{p}2}\mright)\nvar^6\nmin^{-(p+1)} \mleft(\En \nvar\mright)^{\floor{\frac{p-1}2}+1}\Pfn{p-2}\mleft(\NLiD{n}\mright)\nonumber\\
&\quad\quad\mleft(\CHoboundo  +\CHoboundt \CAnk k(hk)^p\mright)\NW{\rho}.\label{eq:hoboundwithrho}
\end{align}
Therefore by combining \cref{eq:ltboundwithrho,eq:hoboundwithrho}, using \cref{eq:ellprojerr} to bound $\NW{\rho}$, and using the fact that $hk \leq 1,$ we obtain \cref{eq:femltbound,eq:femhobound}.
\epf


