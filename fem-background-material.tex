\section{Theory of the Finite-Element Discretisation of the Helmholtz Equation}\label{sec:helmfe}

We now shift our attention to the numerical analysis of the Helmholtz equation in heterogeneous media; we  study the finite-element method for the Helmholtz equation. We first provide the variational formulations of the Helmholtz equation, define the finite element method, and recall results on the approximation properties of finite-element spaces, before proving our main result, a new error bound for the finite-element method for the Helmholtz equation in heterogeneous media.

  \subsection{Variational Formulations for the Helmholtz equation}\label{sec:varform}
  The finite element method is based on the variational formulation of the Helmholtz equation; for simplicity of exposition, we state the variational formulation of \cref{prob:edp,prob:tedp} in the case $\gD = 0,$ although these can be generalised to the case $\gD \neq 0.$
  
\bprob[Variational formulation of EDP when $\gD = 0$]\label{prob:vedp}
Let $\Dp, A, n,$ and $f$ be as in \cref{prob:edp}. Choose $R>0$ such that $\supp f,$ $\supp(I-A),$ $\supp(1-n) \compcont \BR,$ and define $\DR \de \Dp \cap \BR.$

We say $u \in \HozDDR$ satisfies the \defn{variational formulation of the exterior Dirichlet problem} with $\gD = 0$ if
\beqs
\aE(u,v) = \FE(v) \tfa v \in \HozDDR,
\eeqs
where
\beqs
\aE(w,v) \de \int_{\DR} \mleft(\IPRRd{A \grad w}{\grad v} - k^2 n\minispace w \vbar\mright) - \DPGR{\DtN \trGR w}{\trGR v}
\eeqs
and
\beqs
\FE(v) \de \int_{\DR} f\minispace\vbar,
\eeqs
where $\DtN:\HhGR\rightarrow \HmhGR$ is the Dirichlet-to-Neumann map for the homogeneous Helmholtz equation $\Lap u + k^2 u = 0$ combined with the Sommerfeld radiation condition in the exterior of $\BR$; and $\DPGR{\cdot}{\cdot}$ is the duality pairing on $\GR.$
\eprob

\ble[Equivalence of formulations for the EDP]\label{lem:edpform}
\Cref{prob:edp,prob:vedp} are equivalent. If $u \in \HolocDp$ solves \cref{prob:edp} then $u\restrict_{\DR} \in \HozDDR$ and $u\restrict_{\DR}$ solves \cref{prob:vedp}  (for $R$ as in \cref{prob:vedp}). Conversely, if $u \in \HozDDR$ solves \cref{prob:vedp}, then extending $u$ to $\HolocDp$ by the solution of the exterior Dirichlet problem for the homogeneous Helmholtz equation with the Sommerfeld radiation condition in the exterior of $\DR$ (with Dirichlet data $\tr u$ on $\partial \BR$), $u$ solves \cref{prob:edp}.
\ele

For a proof of \cref{lem:edpform}, see \cite[Lemma 3.3]{GrPeSp:19}.

\bprob[Variational formulation of TEDP when $\gD = 0$]\label{prob:vtedp}
Let $D, A, n, f,$ and $\gI$ be as in \cref{prob:tedp}. We say $u \in \HozDDR$ satisfies the \defn{variational formulation of the truncated exterior Dirichlet problem} with $\gD = 0$ if
\beqs
\aT(u,v) = \FT(v) \tfa v \in \HozDDR,
\eeqs
where
\beqs
\aT(w,v) \de \int_{\DR} \mleft(\IPRRd{A \grad w}{\grad v} - k^2 n\minispace w \vbar\mright) - ik\int_{\GI} \trGI w\minispace\trGI \vbar
\eeqs
and
\beqs
\FT(v) \de \int_{\DR} f\minispace\vbar + \int_{\GI} \gI \minispace\trGI \vbar
\eeqs
\eprob
\optodo{Double check with Euan about Proof of $H^m$ regularity when $f$ etc. is smooth enough - Grisvard/McClean}

\ble[Equivalence of formulations for the TEDP]\label{lem:tedpform}
\Cref{prob:tedp,prob:vtedp} are equivalent, i.e., $u \in \HozDDR$ solves \cref{prob:tedp} if, and only if, $u$ solves \cref{prob:vtedp}.
\ele

For a proof of \cref{lem:tedpform}, see \cite[Lemma A.7]{GrPeSp:19}.
  
\subsection{Background Concepts in Finite-Element Theory}\label{sec:fetheory}

We now give a brief summary of elementary concepts in finite-element theory. We focus only on those concepts that we be needed to
prove the new error bound for finite-element discretisations of the Helmholtz equation in \cref{sec:errbound} below.

Throughout this thesis, we use the terms `mesh` and `triangulation' to refer to a triangulation of simplices in the sense of Ciarlet \cite[(FEM1) p. 61, ($\cT_{h}$5) p. 71]{Ci:93}. Similarly, we use $h$ to denote the mesh size of a mesh\footnote{Frequently in this thesis we will consider families of meshes indexed by their mesh size $h$.}, and we use $\Vhp$ to denote the set of piecewise-polynomials of degree $p$ on a mesh with mesh size $h$.
  %%   triangulation of the domain. (Note that in this thesis we use the words `mesh' and `triangulation' interchangably; strictly speaking, the term `triangulation' only makes sense in 2-d, but we ignore this technicality.

  %%   \bde[Triangulation]
  %%   A triangulation of a polygonal domain $D$ is a finite collection of sets $\Ki \subseteq \Dclos$ such that
  %%   \ben
  %% \item $\interior{\Ki} \cap \interior{\Kj} = \emptyset$ for $ i \neq j,$
  %% \item $\bigcup_i \Ki = \Dclos,$ and
  %%   \item Something about triangles that works in 3-D too\optodo{Find ref - Brezzi/Johnson? Make sure to use simplical meshes.}.
  %%   \een
  %%   \ede\optodo{Change to definition in Ciarlet, Handbook of numerical analysis}

%% \bde[Mesh Size]
%% Given a mesh $\cT = \set{\Ki}$ on a domain $D$, the \defn{mesh size} of $\cT$ is defined to be
%% \beqs
%% h = \max_{\Ki} \diam{\Ki}.
%% \eeqs
%% \ede

%%     We can now define finite-element spaces associated with a triangulation; we first define the space of polynomials on a set.

%% \bde[Set of polynomials]
%% For $K \subseteq \RRd,$ $\polyp{K}$ is the set of polynomials defined on $K$ with total degree at most $p.$
%% \ede
%%     \bde[Finite element space of degree $p$]\label{def:fespace}
%%     Given a triangulation $\set{\Ki}$ with mesh size $h$ of a domain $D$, the \defn{(continuous) piecewise-polynomial finite-element space of degree $p$} associated with $\set{\Ki}$ is
%%     \beqs
%% \Vhp \de \set{\vh : D \rightarrow \CC \st \vh \in \CzD \tand \vh\restrict_{\Ki} \in \polyp{\Kiclos} \tforall \Ki }.
%%     \eeqs
%%     \ede

Throughout this thesis, we only consider the $h$-finite-element method, i.e., the degree $p$ of the polynomials associated with the space is assumed fixed, and we consider refining $h.$ This is in contrast to the $p$-FEM, where $h$ is fixed and $p$ is increased, and the $hp$-FEM, where both $h$ and $p$ are increased according to some rule. $hp$-FEM methods for the homogeneous Helmholtz equation were analysed by Melenk and Sauter in \cite{MeSa:10,MeSa:11}; however, their analysis relies on a splitting of the solution of the Helmholtz equation that is, to our knowledge, not possible for the heterogeneous problem.
%\ednote{I couldn't immediately see in these results that one obtains exponential convergence with respect to the number of DOFs---have I missed something, or is this result contained somewhere else?}, where they showed one can obtain an exponential rate of convergence with respect to the number of degrees of freedom.
\optodo{Get Schwab's hp book and look this up}

    The following \lcnamecref{lem:scottzhang} shows how the approximation properties of the space $\Vhp$ depend on $h$ and $p$:
\optodo{Figure out exactly why you need a simplical mesh (is it to ensure you get $C^0$ elements?) and then put this assumption in.}
    \ble[Existence of best approximation]\label{lem:scottzhang}
    Let $v \in \HmD$, $s \geq 1,$ and $p \in \set{1,\ldots,s}$. Then there exist a constant $\CSZm>0$ independent of $v$ and $\vhptilde \in \Vhp$ such that

    \beqs
\NLtD{v - \vhptilde} \leq \CSZm h^m \NHmD{v}
    \eeqs
     and
    \beqs
\NHoD{v - \vhptilde} \leq \CSZm h^{m-1} \NHmD{v} \tfor m \geq 1.
    \eeqs
    \ele

%    We adopt the terminology that $\vhptilde$ is the \defn{quasi-interpolant} of $v.$ For a proof of \cref{lem:scottzhang} see, e.g., \cite[Corollary 4.4.24, Remark 4.4.27]{BrSc:08}.

%%     \bre[The function $\vhptilde$]
%% The function $\vhptilde$ in \cref{lem:scottzhang} can be constructed using `averaged Taylor polynomials', see, e.g., \cite{ScZh:90},\cite[Section 4.4]{BrSc:08} for details\footnote{Observe that in \cite{BrSc:08}, the authors use different notation to us; they use $m$ to denote the regularity of $v$ and $p$ to denote the integrability of $v,$ i.e., in \cite{BrSc:08}, $v \in \WmpD.$}.
%%     \ere

    With the concept of a finite-element space in place, we can now define the finite-element approximation to the variational problems \cref{prob:vedp,prob:vtedp}.

    \bprob[Finite-element approximation of \cref{prob:vtedp}]\label{prob:fevtedp}
We say that $\uh \in \Vhp$ is the finite-element approximation of $u$ (the solution to \cref{prob:vtedp}) if
    \beqs
    a(\uh,\vh) = L(\vh) \tforall \vh \in \Vhp.
    \eeqs
    The finite-element approximation of \cref{prob:vedp} is defined analagously.
    \eprob

    \bre[Not considering variational crimes]
    Observe that \cref{prob:fevtedp} requires $\Vhp \subset \HozDDR.$ This inclusion is true if $\DR$ can be triangulated; otherwise, we must modify the definition of \cref{prob:fevtedp} and commit a variational crime by approximating the boundary of $\DR$ by a polygon, or introducing mesh elements with curved boundaries using interpolated boundary conditions or isoparametric finite elements. See, e.g., \cite[Chapter 10]{BrSc:08} for an overview of the additional errors introduced by variational crimes (although not in the context of the Helmholtz equation). In this thesis we  ignore such variational crimes, and the additional errors they induce; such analysis is standard. Therefore we assume that \cref{lem:scottzhang} holds, even if we have not triangulated the domain $D$ exactly.
    \ere
    

%%     \subsection{Error bound for the heterogeneous IIP}\label{sec:errbound}

%%     We now move on to present new work---bounds on the error $\uh-u$ between the finite-element approximation and the true solution of \cref{prob:vtedp}. These bounds, proven here for the Helmholtz equation in \emph{heterogeneous} media are generalisations of results already in the literature that the finite-element error in the $L^2$- and $H^1$-norms is bounded (as $k\rightarrow \infty$) provided $h \lesssim k^{-(2p+1)/2p}$. These results are crucial for our analysis of the multi-level Monte Carlo method for the Helmholtz equation in \cref{chap:mlmc}.

%% The proof of these results uses a so-called `elliptic projection' technique, introduced by Feng and Wu in \cite{FeWu:09} for Discontinuous Galerkin methods for the Helmholtz equation and modified by Du and Wu \cite{DuWu:15}, where the variational formulation of a PDE related to \cref{prob:vtedp} is used as part of the proof. We  only prove results for \cref{prob:fevtedp}, and not for the finite-element approximation of \cref{prob:vedp}, as our proof uses properties of the related PDE and these properties have only been proven with impedance boundary conditions (in the recent preprint \cite{ChNiTo:18}) and not with an exact Dirichlet-to-Neumann map on the truncation boundary. However, we imagine the results proven for the related PDE with an impedance boundary condition also hold for an exact Dirichlet-to-Neumann boundary condition, and so we anticipate that the results we prove here also hold for finite-element approximations of \cref{prob:vedp}.\optodo{This paragraph needs rewriting---acknowledge Du \& Wu}

\subsection{Discussion of FEM for the Helmholtz equation}\label{sec:helmfedisc}
\optodo{This section needs a major rewrite, I think, to incorporate the overview of different proof techniques.}
We now discuss error bounds for finite-element methods for the Helmholtz equation. We  give some intuition behind these bounds, provide a brief history of their development, and briefly constrast them with error bounds for the stationary diffusion equation.

\paragraph{Intuition for fixed number of points per wavelength} Recall from \cref{sec:numsolve} that if one takes the mesh size in the finite element method $h \sim 1/k$ (for first-order finite elements), then the interpolation (or best approximation) error is bounded uniformly in $k$. This restriction is motivated by observing that solutions of the Helmholtz equation typically\footnote{If $f$ and $\gD$ are independent of $k$,  combining \cref{thm:edp} and standard elliptic regularity results, e.g., \cite[Theorems 4.16 and 4.18]{Mc:00} gives the fact that $\NHtD{u} \lesssim k$.} have $\NHtD{u} \sim k,$ and so one can bound the $H^1$-norm of the interpolation error if $h \sim 1/k$ using \cref{lem:scottzhang}.  As explained in \cref{sec:numsolve}, this restriction ensures there are a fixed number of discretisation points per wavelength of the solution.

An alternative motivation for taking $h \sim 1/k$ (for first-order elements) is the Nyquist--Shannon sampling theorem\footnote{Proved by Shannon in his seminal paper in information theory \cite[Theorem 1]{Sh:49}.} (see, e.g., \cite[\S 5.21]{BaNaBe:00}) that states that any function $v$ (in 1-d) whose Fourier transform lies inside $[-\lambda,\lambda]$ (for some $\lambda >0$) is completely determined (via its Fourier series) by the point values $v(0)$, $v(\pm \mu),$ $v(\pm2\mu),  \ldots$, for any $\mu < 1/\mleft(2\lambda\mright).$

The solution of the one-dimensional Helmholtz equation with constant coefficients is
\beq\label{eq:hh-1d}
u(x) = A \sin(kx) + B \cos(kx),
\eeq
(for some constants $A$ and $B$), and $\lambda = k/2\pi$ as $u$ has Fourier Transform
\beqs
\uhat(\xi) = \frac{A}{2i} \mleft(\delta\mleft(\xi-\frac{k}{2\pi}\mright) - \delta\mleft(\xi+\frac{k}{2\pi}\mright)\mright) + \frac{B}{2} \mleft(\delta\mleft(\xi-\frac{k}{2\pi}\mright) + \delta\mleft(\xi+\frac{k}{2\pi}\mright)\mright).
\eeqs
\optodo{Find reference}
Therefore, if $u$ is sampled at regularly-spaced points that are strictly closer together than $\pi/k = 1/(2\lambda)$, then $u$ can be reconstructed perfectly from these samples.\optodo{Consider writing something about mutliple dimensions} In conclusion, one expects the be able to interpolate the solution of the Helmholtz equation with uniform error in $k$ if $h \sim 1/k.$

However, as was stated in \cref{sec:numsolve}, the finite-element method for the Helmholtz equation suffers from pollution; and $h \sim 1/k$ is not sufficient to keep the finite-element error bounded as $k\rightarrow \infty.$ Therefore we  now provide an overview of previous work giving mesh conditions under which the finite-element error $\NW{u-\uh}$ is bounded as $k\rightarrow \infty$, as well as mesh conditions under which the finite-element method is quasi-optimal. Concerning errors, it would be more natural to consider the relative error $\NW{u-\uh}/\NW{u};$ and to investigate the conditions on $h$ which enable the relative error to be bounded as $k\rightarrow \infty.$ However, current technology only allows us to prove results for the error, not the relative error, and so we investigate the error instead.

%We will now briefly review the literature that seeks to answer the question `What mesh condition will keep the relative error bounded as $k \rightarrow \infty$?'


%% \bde[Bounded finite-element error]
%% Let $(\Th)_{h \in (0,1)}$ be a shape-regular family of triangulations of $\DR$, where we assume $h$ is a function of $k$ (e.g., $h(k) = 1/k$). We say that \defn{the error in the finite-element method is bounded uniformly in $k$} if for all $A,n$ in some class there exists $C(A,n)$ (but independent of $k$ and $h$) such that
%% \beqs
%% \NW{u-\uh} \leq C\mleft(\NLtD{f} + \NLtGD{\gD} + \NLtGI{\gI}\mright).
%% \eeqs
%% \ede



 %% In $d=2$ or $3$ with $D$ a convex polygon or polyhedron,  \cite[Theorem 6.1]{Wu:14} showed that the error (but not the relative error) in both the $H^1$ seminorm and the $L^2$ norm is bounded uniformly in $k$ if $h \lesssim k^{-3/2}$. The proof of \cite[Theorem 6.1]{Wu:14} showed an analagous result for a continuous interior-penalty method, and then took the limit as the penalty parameter tends to 0 to obtain the result for the continuous piecewise-linear finite-element method. In summary, the mesh condition $h \lesssim k^{-3/2}$ seems to be sufficient for the (relative) error to be bounded uniformly in $k.$

\paragraph{Previous results on the finite-element error} Bayliss, Goldstein, and Turkel \cite[\S 3]{BaGoTu:85} performed computations for $d=2$ and first-order finite elements with $D$ a square with Neumann boundary conditions on three sides and an impedance-like boundary condition on the other side. The results of the computations suggested the mesh condition $h \lesssim k^{-3/2}$ is sufficient to bound the relative error in the $L^2$ norm uniformly in $k.$ Ihlenberg and Babuska \cite[Theorem 5, \S 3.4]{IhBa:95} proved that the error in the $H^1$ seminorm for a 1-dimensional problem on an interval with a zero Dirichlet boundary condition at one end, an impedance boundary condition at the other end and a uniform mesh is bounded independently of $k$ provided $h^2k^2$ is sufficiently small, and then concluded that the relative error is similarly bounded. However, the proof in \cite{IhBa:95} used the fact that the solution of the Helmholtz equation in 1-d is given by \eqref{eq:hh-1d} and so has not been generalised to higher dimensions.

Subsequent work on proving error bounds for the Helmholtz equation focused on first proving quasi-optimality for the finite-element method, and then concluding bounds on the error; we discuss results on quasi-optimality below. However, recent efforts on proving error bounds have used so-called elliptic projection ideas; these ideas are at the heart of our results in \cref{sec:heterr} below.

This idea of using elliptic projections was introduced to prove error bounds for discontinuous Galerkin methods for the TEDP by Feng and Wu \cite{FeWu:09,FeWu:11}. Wu \cite{Wu:14} then used elliptic projctions to prove that the error in the first-order finite-element method for the IIP is of the order $ h^2k^3$ if $h \sim k^{-3/2}$ (and therefore the error is bounded); in \cite{Wu:14} this result was shown for a discontinuous-Galerkin first-order method for the Helmholtz equation, and the results for the standard finite-element method were obtained as a special case. Du and Wu \cite{DuWu:15} extended the results in \cite{Wu:14} to higher-order finite-elements for the IIP, showing that the error $\sim h^{2p}k^{2p+1}.$ Chaumont-Frelet and Nicaise \cite{ChNi:18} obtained similar results for first-order finite elements for the TEDP when the scatterer induces corner singularities (and therefore \cite{ChNi:18} includes additional contraints on the mesh arising from the corner singularities that we do not mention here). Wu and Zou \cite[Lemma 3.3]{WuZo:18} obtained the first results for heterogeneous media; they showed the error is bounded if $h^2k^3$ is sufficiently small for first-order finite elements for the IIP. The results in \cite{WuZo:18} we obtained for a special class of heterogeneous media as part of an argument proving similar results for a nonlinear Helmholtz equation.

Observe that the results proved above for higher-order finite elements (the error is bounded if $h^{2p}k^{2p+1}$ is sufficiently small) become less stringent as $p$ increases ($h^{2p}k^{2p+1}$ is bounded if $h \sim k^{-1-1/2p}$). In this section we present similar results for general classes of heterogeneous media; that is, we show that the error in the finite-element method is bounded if $h^{2p}k^{2p+1}$ is sufficiently small and the underlying PDE is nontrapping.

\paragraph{Previous results on quasi-optimality} As mentioned above, one way to prove error bounds for the Helmholtz equation is to first prove the finite-element method is quasi-optimal, and then conclude that the error is bounded. However, as we  show below, the mesh conditions required for the finite-element method to be quasi-optimal are \emph{more} restrictive than those required for the error to be bounded. Recall that the finite-element method is quasi-optimal if there exists $C>0$, independent of $h$ such that
\beqs
\N{u-\uh} \leq C \inf_{\vh \in \Vhp} \N{u-\vh};
\eeqs
i.e., up to a constant, $\uh$ is the best approximation to $u$ in the space $\Vhp$ in the norm $\N{\cdot}.$

We first contrast the Helmholtz equation with the stationary diffusion equation \eqref{eq:stdiff}. For the stationary diffusion equation, one immediately obtains quasi-optimality for \emph{any} mesh by C\'ea's Lemma (and then obtains that the relative error is bounded by properties of particular members of the finite-element space as in, e.g., \cref{lem:scottzhang}. We emphasise again that this result holds for any mesh, with no restriction on $h.$

However, proving quasi-optimality for the Helmholtz equation is more tricky; C\'ea's Lemma relies on the coercivity of the sesquilinear form, and the sesquilinear forms arising from standard discretisations of the Helmholtz equation are not coercive for large $k$. Therefore, to prove quasi-optimality, one instead uses the Aubin--Nitsche duality argument. It was first introduced by Aubin \cite{Au:67} and Nitsche \cite{Ni:68} for coercive problems, and applied to problems satisfying a G\r{a}rding inequality by Schatz \cite{Sc:74}.

The Aubin--Nitsche arument was first used for Helmholtz problems by Melenk in his PhD thesis \cite[Proposition 8.2.7]{Me:95}, he proved that the finite-element method for the Helmholtz equation is quasi-optimal under the very restrictive mesh condition\footnote{Observe that for large values of $k,$ the mesh condition $hk^2$ is sufficiently small is prohibitive---it would result in linear systems of size, e.g., $~10^{12}$, for the Helmholtz equation with $k=100$ in 3-D.} $hk^2$ is sufficiently small. Graham and Sauter \cite[Remark 4.4 b.]{GrSa:18} and Galkowski, Spence, and Wunsch \cite[Theorem 3]{GaSpWu:18} have obtained analogous conditions for first-order finite elements for the IIP \cite{GrSa:18} and the EDP \cite{GaSpWu:18} respectively. A modification of the Aubin--Nitsche argument is used by Chaumont-Frelet and Nicaise \cite[Lemmas 1 and 2]{ChNi:18a} to obtain quasi-optimality results for higher-order finite elements for general wave propagation problems in heterogeneous and homogeneous media.

In conclusion, the error for the finite-element method for Helmholtz problems is bounded if $h^{2p}k^{2p+1}$ is sufficiently small, and the finite-element method is quasi-optimal if $hk^2$ is sufficiently small (for first-order finite elements). We  now prove analagous results on the boundedness of the error for heterogeneous Helmholtz problems.\optodo{Check what Theo did, and whether anyone has done qo for higher-order elements.}

\subsection{Extended discussion of proof techniques for finite-element errors for the Helmholtz equation}
We now discuss in some detail proof techniques used to show that the finite-element method applied to the Helmholtz equation (i) is quasi-optimal in the weighted $H^1$ norm, or (ii) has bounded error in the weighted $H^1$ norm. We discuss these techniques (and the places where they appear in the literature) because the proof techniques are related but different, and we hope that discussing the details, and comparing and contrasting these techniques will prove helpful to the research community, and will increase understanding of proofs that are, at times, a little technical. However, we also discuss these results to set the coefficient-explicit bounds in \cref{sec:fem} below in context.

In all of the cases we discuss below, extra restrictions on the mesh size $h$ will be required to ensure that the pollution error in the finite element method is suppressed.

For simplicity's sake, our exposition below will assume that we are treating the homogeneous Helmholz TEDP (i.e., $A=I$ and $n=1$) with $\gI = 0,$ and that the problem is nontrapping\footnote{Recall that we say that the problem is nontrapping if we have a $k$-independent a priori bound.}. Also, we suppress all of the constants involved, instead opting to use $\lesssim$ notation, where $a \lesssim b$ if $a \leq C b,$ with $C$ independent of $k,$ $h$, and $p.$ The new results we present in \cref{sec:fem} below relax all of these assumptions, and the price of being more technical to state.

\subsubsection{Duality arguments}
The first class of arguments we consider are (modified) duality arguments. In these arguments, one uses a G\r{a}rding inequality satisfied by the Helmholtz equation to show that the error in the weighted $H^1$ norm is bounded by the error in the $L^2$ norm. One then uses a modification of the standard Aubin--Nitsche duality argument to bound the $L^2$ norm. To show the error in the weighted $H^1$ norm is bounded by the error in the $L^2$ norm, observe
\begin{align}
\NW{u-\uh}^2 &\leq \Re{a(u-\uh,u-\uh)} + k^2 \NLtD{u-\uh}^2\nonumber\\
&= \Re{a(u-\uh,u-\vh)} + k^2 \NLtD{u-\uh}^2\nonumber\\
&\leq \NW{u-\uh}\NW{u-\vh} + k^2 \NLtD{u-\uh}^2,\label{eq:gardingerror}
\end{align}
for any $\vh \in \Vhp.$

Recall that the Schatz argument, outlined below, (showing quasi-optimality) then proceeds to show
\beq\label{eq:schatzbound}
k\NLtD{u-\uh} \lesssim hk^2 \NW{u-\uh},
\eeq
and combining \cref{eq:gardingerror,eq:schatzbound}, one can show quasi-optimality by taking the second term in \cref{eq:gardingerror} onto the left-hand side if $hk^2$ is sufficiently small and cancelling $\NW{u-\uh},$ to obtain
\beqs
\NW{u-\uh} \lesssim \inf_{\vh \in \Vhp} \NW{u-\uh}.
\eeqs

In contrast, so-called \defn{elliptic projection} arguments, also outlined below, show (in the first-order case) that the finite-element error is bounded by first showing
\beq\label{eq:epbound}
k\NLtD{u-\uh} \lesssim h^2k^3 \NLtD{f},
\eeq
and then combining \cref{eq:gardingerror,eq:epbound} to show
\begin{align}
\NW{u-\uh}^2 &\lesssim \NW{u-\uh}\NW{u-\vh} + h^2k^3 \NLtD{f}\nonumber\\
&\leq \eps \NW{u-\uh}^2 + \frac1\eps \NW{u-\vh}^2 +\mleft(h^2 k^3 \NLtD{f}\mright)^2\label{eq:epboundpart1}
\end{align}
Taking $\eps$ sufficiently small, and moving the first term in \cref{eq:epboundpart1} to the left-hand side, taking $\vh$ to be the best approximation of $u$ in $\Vhp,$ and using \cref{lem:scottzhang} and the fact that\footnote{For sufficiently smooth $\GD$ and $\GI$---see \cite[Remark 2.14]{GrPeSp:19}} we can show $\NHtD{u} \lesssim k\NLtD{f}$
\beqs
\NW{u-\uh} \lesssim \mleft(hk + h^2k^3\mright)\NLtD{f}.
\eeqs

We will now look over the `duality argument' part of the Schatz argument in more detail; we will see presently that the elliptic projection arguments mentioned above further modify the Schatz argument, and so understanding it to begin with is instructive.

We first establish some notation that will enable us to discuss best approximation errors for solutions of Helmholtz problems. We let $\solfem:\LtD\rightarrow \HoD$ denote the solution operator for the Helmholtz TEDP with zero impedance boundary condition, and we let $\solfems$ denote the solution operator for the corresponding adjoint problem. That is, for any $f \in \LtD$ and for all $v \in \HozDD,$
\beqs
a(\solfem(f),v) = \IPLtD{f}{v}
\eeqs
and
\beqs
a(v,\solfems(f) = \IPLtD{v}{f}.
\eeqs
There exist obvious generalisations of these operators to the case with non-zero impedance boundary conditions. We next define the approximability constants:
\beqs
\wba \de \sup_{f \in \LtD}\inf_{\vh \in \Vhp} \frac{\NW{\solfem(f) - \vh}}{\NLtD{f}}
\eeqs
and
\beqs
\wbaadj \de \sup_{f \in \LtD}\inf_{\vh \in \Vhp} \frac{\NW{\solfems(f) - \vh}}{\NLtD{f}}.
\eeqs
We note that we are changing notation slightly from that normally prevelant in the literature; notation for $\wbaadj$ was first introduced by Sauter in \cite[Section 2.2]{Sa:06}, but $\wbaadj$ was instead denoted $\tilde{\eta}$ in \cite{Sa:06} and denoted $\eta$ in \cite[Equation (4.5)]{MeSa:10}.  When dealing purely with the Schatz argument for quasi-optimality (as in \cite[Section 2.2]{Sa:06}) one only needs consider the approximation of adjoint problems in the duality-argument step, and hence one only needs notation for $\wbaadj$. However, in our exposition of elliptic-projection-based arguments below, we will also need to consider approximation of non-adjoint\ednote{Both---Is there a word for this? `Standard'?} Helmholtz problems, and hence we adopt the notation above.

\paragraph{The Schatz argument} We now recap the so-called Schatz argument for proving quasi-optimality for an indefinite sesquilinear form (first introduced by Schatz in \cite{Sc:74}. This recapitulation will allow us to see where $\wbaadj$ enters the argument, and will also allow us to compare and contrast this argument with the arguments used to obtain a bounded finite-element error using an elliptic projection.

We let $\xi \in \HozDD$ solve the adjoint Helmholtz problem
\beqs
a(v,\xi) = \IPLtD{v}{u-\uh} \tforall v \in \HozDD.
\eeqs
Then, taking $v = u-\uh,$ we have
\begin{align}
  \NLtD{u-\uh}^2 &= a(u-\uh,\xi)\label{eq:schatz0}\\
  &a(u-\uh,\xi - \vh) \text{ by Galerkin orthogonality for } u-\uh, \text{ for any } \vh \in \Vhp\nonumber\\
  &\lesssim \NW{u-\uh}\NW{\xi-\vh}\nonumber\\
  &\lesssim \NW{u-\uh} \wbaadj \NLtD{u-\uh}\nonumber,
\end{align}
and cancelling a factor of $\NLtD{u-\uh},$ we obtain
\beq\label{eq:schatz1}
\NLtD{u-\uh} \lesssim \wbaadj \NW{u-\uh}.
\eeq
Combining \cref{eq:gardingerror,eq:schatz1}, we have
\beqs
\NW{u-\uh}^2 \lesssim \NW{u-\uh}\NW{u-\vh} + \mleft(k\wbaadj\mright) \NW{u-\uh}^2,
\eeqs
and hence
\beqs
\NW{u-\uh} \lesssim \inf_{\vh \in \Vhp} \NW{u-\vh} \tif k\wbaadj \text{ is sufficiently small}.
\eeqs
All results showing quasi-optimality for the Helmholtz equation (for different finite-element spaces and different domains) can then be seen as simply obtaining estimates on $\eta$ in these different scenarios; this literature is summarised in\optodo{Insert ref} below.

\paragraph{Elliptic-projection arguments} We now move on to compare and contrast the Schatz argument given above with so-called elliptic-projection arguments.\optodo{Put defn here, briefly?} The main difference in the result of these arguments, compared to the Schatz argument given above, is that elliptic projection arguments only produce a \emph{bound} on the finite-element error, rather than quasi-optimality. However, these bounds are obtained under mesh restrictions that are less restrictive than those required for quasi-optimality.

The elliptic projection of a function $ w \in \HozDD$ is the finite-element function $\Ph w$ that has the same action as $w$ on the finite-element space $\Vhp$ under some elliptic operator; i.e., $\Ph w$ is defined by
\beqs
\aep(\Ph w,\vh) = \aep(w,\vh) \tforall \vh \in \Vhp
\eeqs
For some sesquilinear form $\aep$, such that $\Ph$ is well-defined. For Helmholtz problems (where for ease we simply consider problems with an impedance boundary condition), choices for $\aep(\vo,\vt)$ are either
\beqs
\aep(\vo,\vt) = \IPLtD{\grad \vo}{\grad \vt} \tor \aep(\vo,\vt) = \IPLtD{\grad \vo}{\grad \vt} - ik\IPLtGI{\vo}{\vt}.
\eeqs
These elliptic projections correspond to finding finite-element approximations of the solution of the PDEs
\begin{align*}
  \Delta w &= F \tin D\\
  w &= 0 \ton \GD\\
  \dn w &= 0 \ton \GI
\end{align*}
and
\begin{align*}
  \Delta w &= F \tin D\\
  w &= 0 \ton \GD\\
  \dn w -ikw &= 0 \ton \GI,
\end{align*}
where $F$ is an appropriately chosen function. In the following exposition, we will assume the second choice.
Since $\Ph$ is a Galerkin projection, one can show that in its energy norm $\Nep{\cdot}$ (equivalent to the $k$-weighted $H^1$ norm $\NW{\cdot}$, with equivalence constants independent of $k$) $\Ph$ is coercive and continuous, and hence is quasi-optimal:
\beq\label{eq:epho}
\NW{w-\Ph w} \lesssim \inf_{\vh \in \Vhp} \NW{w-\vh}
\eeq
and by an Aubin--Nitsche duality argument,
\beq\label{eq:eplt}
\NLtD{w-\Ph w} \lesssim h \NW{w-\Ph w}.
\eeq

Elliptic projection arguments are a modification of the Schatz argument above; starting from \cref{eq:schatz0} instead of introducing an arbitrary $\vh \in \Vhp$ into the second argument, we instead introduce the elliptic projection $\Ph \xi$:
\begin{align}
  \NLtD{u-\uh}^2 &= a(u-\uh,\xi-\Ph\xi)\nonumber\\
  &= \aep(u-\uh,\xi-\Ph\xi) - k^2\IPLtD{u-\uh}{\xi-\Ph\xi}\nonumber\\
  &= \aep(u-\Ih u,\xi-\Ph\xi) - k^2\IPLtD{u-\uh}{\xi-\Ph\xi} \text{ by Galerkin orthogonality for }\xi-\Ph\xi\nonumber\\
  &\lesssim \wba \NLtD{f} \wba \NLtD{u-\uh} + k^2 \NLtD{u-\uh} \NLtD{\xi-\Ph\xi}\text{ by \cref{eq:epho}}\nonumber\\
  &\lesssim \wba^2 \NLtD{f}\NLtD{u-\uh} + k^2 \NLtD{u-\uh} h\wba \NLtD{u-\uh}\text{ by \cref{eq:epho,eq:eplt}.}\label{eq:schatz2}
\end{align}
Therefore if $hk^2\wba$ is sufficiently small, the second term in \cref{eq:schatz2} can be absorbed into the left-hand side, and cancelling a factor of $\NLtD{u-\uh},$ we obtain (multiplying by $k$, so as to compare with the quasi-optimality results above, as the $L^2$ term in $\NW{\cdot}$ is also multiplied by a factor $k$)
\beq\label{eq:ep1}
k \NLtD{u-\uh} \lesssim k \wba^2 \NLtD{f} \tif hk^2\wba \text{ is sufficiently small}.
\eeq
To obtain a bound on the error in the weighted $H^1$ norm, we take $\vh = \Ih u$ in \cref{eq:gardingerror}, where $\Ih$ denotes the interpolant or quasi-interpolant, so that
\beqs
\NW{u - \Ih u} \lesssim \inf_{\vh \in \Vhp} \NW{u - \vh}.
\eeqs
Analagous to \cref{eq:epboundpart1}, we obtain for any $\eps > 0$
\beq\label{eq:ep2}
\NW{u-\uh}^2 \lesssim \eps \NW{u-\uh}^2 + \frac1\eps\NW{u-\Ih u}^2 + \mleft(k\wba^2\mright)^2 \NLtD{f}^2.
\eeq
Taking $\eps$ sufficiently small, and moving the first term on the right-hand side of \cref{eq:ep2} to the left-hand side, and taking a square root, we obtain
\beqs
\NW{u-\uh} \lesssim \NW{u-\Ih u} + k\wba^2 \NLtD{f},
\eeqs
and thus
\beq\label{eq:ep3}
\NW{u-\uh} \lesssim \mleft(\wba + k\wba^2\mright) \NLtD{f} \tif hk^2\wba\text{ is sufficiently small}.
\eeq

The term $\wba$ on the right-hand side of \cref{eq:ep3} is the best-approximation error (see\optodo{ref} above), and the term $k\wba^2$ is the \defn{pollution} term resulting from the numerical method.\optodo{Somewhere, not necessarily here, discuss all the pollution stuff.} As for the quasi-optimality results above, we remark that different bounds on the finite-element error (for different finite-element space $\Vhp$ and different domains) can be thought of as proving bounds on $\eta$ in these different situations.
\ednote{Both---I don't yet have a clear explanation of \emph{why} this works. Currently, it seems just like a trick to me.}
\optodo{Need to put in above this a definition of the elliptic projection, and the properties it needs. Will record properties here for now. (For now) Quasi-optimality in weighted norm with $k$-independent constant. Aubin-Nitsche bound showing $L^2$ error converges like $h$ times $H^1$ error. Need to define $\Ih$/acknowledge that it achieves best approx error}\optodo{Put, perhaps in a footnote, that the definition of the elliptic projection is slightly problem-dependent, and the norms one works in are also problem-dependent, but the structure I'll present is constant.}
\optodo{Put more detailed literature review somewhere}
\optodo{Put somewhere that literature is a bit dense, because arguments first used in DG context, where there's lots of extra notation, and forms are discretisation dependent to control jumps. Here we will try and give essence of arguments, but in a non-DG setting, so all forms arestandard.}
\subsubsection{Error-splitting arguments}.
The second class of argument used in the literature is an error-splitting argument, where the finite-element error is split using the elliptic projection of the solution $u$. To begin, we make the trivial observation that
\beqs\label{eq:split1}
u-\uh = \mleft(u-\Ph u\mright) + \mleft(\Ph u - \uh\mright).
\eeqs
Proving an error bound for $u-\uh$ therefore reduces to proving an error bound for the elliptic projection error $u- \Ph u$ and proving a bound on the difference $\Ph u - \uh.$ The former can either be accomplished by showing quasi-optimality of the elliptic projection (as in \optodo{Insert ref once done} above) or by proving such an error bound directly. The former approach is taken in \cite{DuWu:15}, where the authors do not include an impedance boundary term in their elliptic projection. The latter approach is taken in \cite[Lemma 5.2]{FeWu:09} (althought the proof is only contained in \cite[Lemma 5.2]{FeWu:08}), \cite[Lemma 4.3]{FeWu:11}, and \cite[Lemma 4.2]{Wu:14}, where an impedance boundary term is included in the elliptic projection. In \cite{FeWu:09,FeWu:11,DuWu:15} the bound on the elliptic projection error is proved by observing that the sesquilinear form $\IPLtD{\grad \vo}{\grad \vt}$ is coercive on $\HozDD$, and then also controlling the additional term arising from the impedance boundary condition.

To bound the difference $\Ph u - \uh,$ one first shows that it solves a deterministic Helmholtz problem:
By Galerkin orthogonality for $u-\uh,$ for any $\vh \in \Vhp$ we have $a(u-\uh,\vh) = 0$, and by Galerkin orthogonality\optodo{Need to define EP in first argument here} for $u-\Ph u,$ $\aep(u-\Ph u,\vh) = 0$. Therefore
\begin{align*}
  a(\Ph u - \uh,\vh) &= a(\Ph u - u,\vh) + a(u-\uh,\vh)\\
  &= a(\Ph u - u,\vh)\\
  &= \aep(\Ph u - u,\vh) - k^2\IPLtD{\Ph u - u}{\vh}\\
  &= - k^2\IPLtD{\Ph u - u}{\vh},
\end{align*}
that is, $\Ph u - \uh$ solves the discrete Helmholtz problem
\beqs
a(\Ph u - \uh,\vh) = \IPLtD{\ftilde}{\vh} \tforall \vh \in \Vhp,
\eeqs
where $\ftilde = k^2\mleft(u-\Ph u\mright).$ One then uses the fact that the difference satisfies a discrete Helmholtz problem to prove a bound on the difference directly.

In \cite{FeWu:09,FeWu:11,Wu:14}\optodo{Check these} a discrete multiplier argument is used, reminiscent of the multiplier arguments used to prove a priori bounds on Helmholtz problems in \cref{sec:pdetheory}. In \cite{DuWu:15} an argument using higher-order (discrete) norms of are used in an argument conceptually similar to the modified duality arguments above. The $L^2$-norm of $\Ph u - \uh$ is bounded by $\NW{u-\uh}$ and a higher-order norm of $\Ph u - \uh$ (with both terms multiplied by powers of $k$ and $h$) using a modified duality argument. Also, the higher-order norm of $\Ph u - \uh$ is bounded by terms involvinh $k,$ $h$, $\NW{u-\uh}$ and the $L^2$ norm of $\Ph u - \uh$. These two norm bounds are then combined to produce a bound on the $L^2$ norm of $\Ph u - \uh$, provided $h^{2p}k^{2p+1}$ is sufficiently small.
\optodo{For this section; for the first set you only need a bound on the elliptic projection error (with the right powers of $h$ and $k$). For the second one, We use quasi-optimality in $H^1$, the lower-order bounds in terms of the higher-order one, and a bound on the best approximation error. Maybe you could combine the first and third and get away with a bound on the elliptic projection error with the correct powers?}

Observe that when using the ellptic projection in such a splitting argument, there are two differences compared to the use of an elliptic projection in modified duality arguments:
\ben
\item One does not need the elliptic projection to be quasi-optimal \cref{eq:epho}, rather, one only need to bound the error $\NW{u-\Ph u}$, where $u$ solves a Helmholtz problem, in terms of $\NLtD{f}$.
\item The elliptic projection should be defined in the first argument, not the second, i.e.
  \beqs
a(\Ph u,\vh) = a(u,\vh) \tforall \vh \in \Vhp.
  \eeqs
\een

\subsubsection{Merits and drawbacks of each argument}
We now briefly discuss the positive and negative points for each of the two classes of argument we have outlined above; duality arguments and splitting arguments.

The positive points for duality arguments are their simplicity---we have stated the arguments almost in their entirety above (the only part left unstated is how to bound $\wba$ and $\wbaadj$), and they are closely related with familiar duality arguments used in the finite-element-error analysis for more familiar problems, such as the stationary diffusion equation. In addition, it is straightforward to see that for higher-order finite elements (or $hp$-finite-elements) if one knows the dependence of $\wba$ and $\wbaadj$ on $h$, $p$, and $k$, then one can easily conclude bounds under conditions, both of which are completely $h$-, $p$-, and $k$-explicit. Such an analysis was performed for quasi-optimality of $hp$-finite-element methods for the Helmholtz equation in homogeneous media by Melenk and Sauter \cite{MeSa:10,MeSa:11}, where $\wbaadj$ was bounded using a profound splitting of the solution of the Helmholtz equation.

However, the main drawback of duality arguments is their lack of sharpness in the conditions imposed on $h$. For example, if we assume $p$ is fixed and $\wba = h + \mleft(hk\mright)^p$ (as is shown, for example, by Chaumont-Frelet and Nicaise in \cite[Lemma 1]{ChNi:18a}), then the condition $hk^2\eta$ is sufficiently small in \cref{eq:ep3} above translates to requiring\optodo{This isn't sharp...} $h^2k^2 + h^{p+1}k^{p+2}$ sufficiently small; this final condition is ensured if $h^{p+1}k^{p+2}$ is sufficiently small. However, the pollution term in \cref{eq:ep3} is of the form $k\wba^2 \lesssim h^2k + h^{2p}k^{2p+1} \lesssim h + h^{2p}k^{2p+1}$ (as $hk \lesssim 1$). That is, the mesh restriction under which we can prove a finite-element error bound is more restrictive than the mesh restriction needed to bound the pollution error uniformly in $k.$
\optodo{Mention somewhere the asymptotic/preasymptotic terminology}

In contrast, the positive point for splitting arguments is that they can give mesh conditions that are sharp in their $p$-dependence; in \cite{DuWu:15}, Du and Wu prove that the pollution term in the finite-element error is of the order $h^{2p}k^{2p+1},$ and they do so under the restriction that $h^{2p}k^{2p+1}$ is sufficiently small, i.e., the mesh restriction under which we can prove a finite-element error bound is `the same' as the mesh restriction needed to bound the pollution error uniformly in $k.$

However, the drawback of splitting arguments is their complexity---obtaining a priori bounds on the difference $\Ph u - \uh$ involves proving bounds on the solution of discrete Helmholtz problems; such bounds are complicated to prove (especially in the higher-order cases, as in \cite{DuWu:15} and in \cref{sec:fem} below), and the constants involved depend on the polynomial degree $p$ in highly complicated ways.

\subsubsection{Complete technical overview of the literature}
We now give a brief overview of the literature on quasi-optimality/error bounds for finite-element discretisations of the Helmholtz equation. We record this information tersely, having given more insight into the various techniques above. In what follows, if a quantity $p$ is mentioned, with no restriction on $p,$ then one can assume the results hold for any \emph{fixed} polynomial degree $p$. If no quantity $p$ is mentioned, then the results only hold for first-order finite elements. In the columns marked `mesh condition' the quantity listed must be sufficiently small, e.g., if `mesh condition' reads `$hk^2$' then the condition is `$hk^2$ sufficiently small'.
%% Work
%% Mesh condition
%% Error bound
%% Notes
%\paragraph{Quasi-optimality}
\begin{table}[h]
\begin{tabu}{ccc}
  \toprule
& Mesh Condition & Notes\\
  \midrule
  \cite{Me:95} & $hk^2$ & $d=2,$ Interior Impedance Problem (IIP)\\
  \cite{MeSa:10,MeSa:11} & $h^{p}k^{p+1}$ or $\log(k)/p$ and $kh/p$ ($hp$) & Full-space problem, IIP, and EDP\\
  \cite{ChNi:19} & $h^pk^{p+\alpha+1}$ ($p$ fixed) & Arbitrary problem, a priori bound grows at rate $k^\alpha$\\
  \cite{ChNi:18} & $hk^2$ & TEDP with re-entrant corners\\
  \cite{ChGaNiTo:19} & $h^{p}k^{p+1}$ ($p$ fixed) & Truncated PML\\
\bottomrule
\end{tabu}
\caption{Summary of the literature on quasi-optimality}
\end{table}

\begin{table}[h]
\begin{tabu}{cccc}
  \toprule
  & Mesh Condition & Error bound & Notes\\
      \midrule
  \cite{ZhWu:13} & $\displaystyle \frac{kh}p \mleft(\frac{k}p\mright)^{\frac1{p+1}}$ ($hp$) or $h^{p+1}k^{p+2}$ ($p$ fixed) & $\displaystyle \frac{h}p + \frac1{p}\mleft(\frac{kh}{\sigma p}\mright)^p + \frac{k}{p^2} \mleft(\frac{kh}{\sigma p}\mright)^{2p}$ & IIP\\
  \cite{LiWu:18} & $h^2k^3$ & $h^2k^3$ & Truncated PML\\
  \cite{ChGaNiTo:18} & $h^{p+1}k^{p+2}$ & $h^{2p}k^{2p+1}$ & Truncated PML\\
    \cite{ChNi:18} & $h^{1+\alpha}k^3$ &$hk + h^2 k^3 + h^\alpha k^{\alpha-1/2}$ & TEDP with re-entrant corners, $\alpha$ corresponds to strength of singularities\\
\bottomrule
\end{tabu}
\caption{Summary of the literature on error bounds using a modified duality argument}
\end{table}

\begin{table}[h]
\begin{tabu}{cccc}
  \toprule
  & Mesh Condition & Error bound & Notes\\
  \midrule
  \cite{FeWu:09} & $kh$ & $h^{4/3} k^{8/3}$ & TEDP, Interior-Penalty Discontinuous Galerkin (IPDG) method\\
  \cite{FeWu:11} & $\displaystyle \frac{h^2 k^3}{p^2}$ & $\displaystyle \mleft(\frac{h}p+ \frac{h^2 k^2}{p^2}\mright)$ & TEDP, $hp$-IPDG method\\
  \cite{Wu:14} & $h^2 k^3$ & $hk + h^2 k^3$ & IIP\\
  \cite{DuWu:15} &  $h^{2p}k^{2p+1}$ & $\mleft(hk\mright)^p + h^{2p} k^{2p+1}$ &  IIP\\
\bottomrule
\end{tabu}
\caption{Summary of the literature on error bounds using a splitting argument}
\end{table}

%\optodo{Would it be better to put bounds on $\eta$ here?}


%%5 Can put the below somewhere %%

%% To keep the finite-element error bounded when solving \eqref{eq:introdet}, one must over-refine the numerical grid. That is, rather than using a fixed number of points per wavelength, one must increase the number of points per wavelength as $k$ increases. To achieve bounded finite-element error, one must refine the finite-element mesh size $h$ like $k^{-3/2}$. Whilst this result has been known numerically for some time, it was proven for \eqref{eq:introdet} only with constant coefficients (on various domains and for various finite-element spaces) in \cite{IhBa:95a,Wu:14,DuWu:15,ChNi:18}, and the first proof (to our knowledge) for \eqref{eq:introdet} with heterogeneous coefficients  is contained in \cref{chap:background}. Choosing $h \sim k^{-3/2}$ means \eqref{eq:intromat} is a linear system of size $k^{3d/2}$, larger than if one merely wants the interpolation error to be bounded. Hence, requiring a bounded finite-element error gives rise to very large linear systems.

%% More briefly, if one wants the finite-element solution to be quasi-optimal (that is, up to a constant, the finite-element solution is the best approximation in the finite-element space), then one must over-refine even more, and take $h \sim k^{-2}$. This mesh condition will give rise to linear systems with $k^{2d}$ degres of freedom. See \cref{chap:background} for further details on the necessity of this mesh condition, and further discussion of all the mesh conditions discussed above.\optodo{EDIT THE ABOVE TO REMOVE `OVER-REFINE'}


%% \subsection{New error bounds for the Helmholtz equation in heterogeneous media}\label{sec:heterr}
%% In this section, we prove that the finite-element approximation of the solution to the Helmholtz TEDP exists if $ h \lesssim k^{-3/2}.$ Moreover, we give an expression for the hidden constant that is completely explicit in $A$ and $n$, and we also prove a bound on the finite-element error, again completely explicit in $A$ and $n$. The argument in this section closely follows those in \cite{FeWu:11,ChNi:18} in its use of an elliptic projection argument to prove the required finite-element existence result and error bound. The paper \cite{FeWu:11} proved a similar result for the Helmholtz equation in homogeneous media, and \cite{ChNi:18} does so for the homogeneous Helmholtz equation with corner singularities.

%% Whilst we prove the results in this section for the TEDP, we expect that they can be extended to the Helmholtz Exterior Dirichlet Problem (EDP) where the infinite domain is truncated, and the Dirichlet-to-Neumann map is realised exactly on the truncated boundary. However, our proof below uses recently-proved bounds on the solution of a related problem to the TEDP from \cite{ChNiTo:18}; in order to extend our results to the EDP we would need analogues to the results in \cite{ChNiTo:18} for the EDP.

%% %% \paragraph{Problem Set-up} Let $\Dm$ be a bounded Lipschitz\ednote{We actually need this to be a $C^{k,\lambda}$ set, for $k+\lambda > 1.5,$ so that we can do the whole non-zero Dirichlet data thing. This is getting a bit complicated. I guess our options are (i) persevere, (ii) give up and just do the theory for zero Dirichlet data, or (iii) assume that we know $\ud,$ not just $\gD.$ Thoughts?} open set such that the open complement $\Dp\de \RRd\setminus \Dmclos$ is connected. Let $\Dtilde$ be a bounded connected Lipschitz open set such that $\Dmclos \subset\subset\Dtilde$. 
%% %% Let $D\de\Dtilde\setminus\Dm$, $\GD\de \partial \Dm$, and $\GI \de\partial \Dtilde$, so that $\partial D= \GD \cup \GI$ and $\GD\cap \GI = \emptyset$. Throughout $\tr$ will denote the trace onto the whole boundary $\dD,$ whereas $\trGI$ and $\trGD$ will denote the traces on $\GI$ and $\GD$ respectively. Throughout we assume there exists some $\kz > 0$ such that $k \geq \kz$. Let $\NW{v}$ denote the weighted $H^1$ norm on $\HoD$:
%% %% \beqs
%% %% \NW{v}^2 \de \NLtD{\grad v}^2 + k^2 \NLtD{v}^2.
%% %% \eeqs


%% %% Let
%% %% \bit
%% %% \item $f\in \LtD$ 
%% %% \item $\gD\in \HthtGD$,
%% %% \item $\gI\in \LtGI$
%% %% \item $n\in \LiDRR$ such that $\dist\mleft(\supp\mleft(1-n\mright),\GI\mright)>0$, satisfying
%% %% \beq
%% %% 0<\nmin \leq n\mleft(\bx\mright)\leq\nmax<\infty\,\, \text{ for almost every } \bx \in D,
%% %% \eeq
%% %% \item $A \in \WoiDRRdtd$ such that $\dist\mleft(\supp\mleft(I -A\mright),\GI\mright)>0$, $A$ is symmetric, and there exist $0<\Amin\leq \Amax<\infty$ such that
%% %% \beq\label{eq:AellEDP}
%% %%  \Amin |\bxi|^2\leq\mleft(A\mleft(\bx\mright) \bxi\mright) \cdot \overline{ \bxi}  \leq \Amax|\bxi|^2 \quad\text{ for almost every }\bx \in D \text{ and for all } \bxi\in \CCd.
%% %% \eeq
%% %% \eit
%% %we say $u\in \HoD$ satisfies the Helmholtz Truncated Exterior Dirichlet Problem (TEDP) if 
%% %\beqs
%% %\grad\cdot\mleft(A \grad u \mright) + k^2 n u = -f \quad \tin D,
%% %\eeqs
%% %\beqs
%% %\trGD u =\gD \quad\ton \GD,
%% %\eeqs
%% %and 
%% %\beq\label{eq:TEDP3}
%% %\dn u - \ii k  \trGI u = \gI \ton \GI.
%% %\eeq
%% In order to study the TEDP with $\gD\neq0,$ we must, in essence reformulate to the TEDP with $\gD=0$ but a different right-hand side for the domain term.

%% %% Define the space
%% %% \beqs
%% %% \HozDD \de \set{v \in \HoD \st \trGD u = 0}.
%% %% \eeqs
%% %and the sesquilinear form and antilinear functional
%% %The variational formulation of the TEDP with $\gD = 0$ is%\optodo{Check exactly what's needed in hetero}
%% %
%% %\beq\label{eq:tedpz}
%% %\text{Find } u \in \HozDD\quad \tst\quad a(u,v) = F(v)\quad \tfa v \in \HozDD,
%% %\eeq
%% %
%% %where
%% %
%% %\beqs
%% %a(u,v) \de \int_D \mleft(A \grad u\mright)\cdot \grad \vb - k^2 n u\vb - ik \int_{\GI} \trGI u \trGI \vb\quad \tand\quad F(v) \de \int_D f\vb + \int_{\GI} \gI\trGI \vb.
%% %\eeqs
%% %
%% %% In order to deal with non-zero Dirichlet data $\gD,$ we let  $\ud \in \HtD$ be such that $\trGD \ud = \gD$, and $\esssup \ud \compcont D$. The proof that such a $\ud$ exists is in \cref{lem:ud}.
%% %% The variational formulation of the TEDP is then
%% %% \beq\label{eq:tedp}
%% %% \text{Find } u \in \HozDD\quad \tst\quad a(u,v) = F(v)\quad \tfa v \in \HozDD,
%% %% \eeq
%% %% where
%% %% \beqs
%% %% a(u,v) \de \int_D \mleft(A \grad u\mright)\cdot \grad \vb - k^2 n u\vb - ik \int_{\GI} \trGI u \trGI \vb
%% %% \eeqs
%% %% and
%% %% \beqs
%% %% F(v) \de  \int_D \mleft(f - \grad \cdot \mleft(A\grad \ud\mright) - k^2 n\ud\mright)\vb + \int_{\GI} \mleft(\gI-\dn\ud\mright)\trGI \vb.
%% %% \eeqs
%% %% The function $\us = u+ \ud$ is then the solution of the Helmholtz equation
%% %% \beqs
%% %% \grad\cdot\mleft(A \grad \us \mright) + k^2 n \us = -f \quad \tin D,
%% %% \eeqs
%% %% \beqs
%% %% \trGD \us =\gD \quad\ton \GD,
%% %% \eeqs
%% %% and 
%% %% \beq\label{eq:TEDP3}
%% %% \dn \us - \ii k  \trGI \us = \gI \ton \GI.
%% %% \eeq
%% %% \bre[Reducing the smoothness of $\gD$]
%% %% The assumption that $\gD \in \HthtGD$ is made so that the lifting $\ud$ of $\gD$ is in $\HtD$ (see \cref{app:ud}). As $\ud \in \HtD,$ the antilinear functional $F$ defined above is well-defined. We could reduce the smoothness of $\gD$ to $\HoGD$ (meaning $\ud \in \HthtD$) but this reduction in smoothness would then require us to reformulate the functional $F$ as
%% %% \beqs
%% %% F(v) = \int_D \mleft(A \grad \ud\mright)\cdot \grad \vb - k^2 n \ud \vb + f \vb + \int_{\GI} \mleft(\gI - \dn \ud\mright)\vb.
%% %% \eeqs\optodo{Put a proof of this somewhere, in 28/2/19 notes}
%% %% With this reformulation, $F \in \HozDDprime,$ but does not have a representative function in $\LtD$. Our proofs below will use results from \cite{ChNiTo:18}, which are stated for the TEDP with zero Dirichlet boundary condition and $L^2$ right-hand side. To avoid the complications stated above, and to allow us to use the results in \cite{ChNiTo:18}, we therefore impose the additional smoothness on $\gD.$ Also, in the case with $F$ only in $\HozDDprime$, proving a priori bounds on the solution of the TEDP is more complicated (c.f., e.g., \cite[Theorem 2.5]{GrPeSp:19} and \cite[Corollary 2.16]{GrPeSp:19} which consider the analogous EDP). For the same reason, we assume $A \in \WoiDRRdtd;$ if we only had $A \in \LiDRRdtd,$ we could reformulate $F$ as outlined above, but we would have the same complications as just described.

%% We  restrict our meshes to the following class:
%% \bde[Shape-regular]
%% A family $(\Th)_{h \in (0,1)}$ of meshes of $\DR$ is said to be \defn{shape-regular} if there exists $\rho > 0$ such that for all $T \in \Th$ and for all $h \in (0,1]$
%%   \beqs
%% \diam B(T) \geq \rho \diam T,
%% \eeqs
%% where $B(T)$ is the largest ball contained in $T$ such that $T$ is star-shaped with respect to $B(T)$.
%% \ede

%% The fact that we cannot reduce the smoothness of $\gD$ further to $\HhGD$ is due to the Morawetz multiplier techniques used to obtain the a priori bounds in \cite{GrPeSp:19}, see \cite[(iii), p. 2874]{GrPeSp:19}.
%% %% \ere
%% Also, for later use we state the \defn{adjoint} problem.
%% \beq\label{eq:tedpadj}
%% \text{Find } u \in \HozDDR\quad \tst\quad \aadj(u,v) = F(v)\quad \tfa v \in \HozDDR,
%% \eeq
%% where
%% \beqs
%% \aadj(u,v) \de \int_D \mleft(A \grad u\mright)\cdot \grad \vb - k^2 n u\vb + ik \int_{\GI} \trGI u \trGI \vb.
%% \eeqs
%% %and
%% %\beqs
%% %\Fadj(v) \de \aadj(\uz,v) + \int_D f\vb + \int_{\GI} \gI\trGI \vb.
%% %\eeqs

%% The statement of the main result requires the following related sesquilinear form and \lcnamecref{lem:relatedwp}.

%% \bde[Related sesquilinear form]
%% For $\vo, \vt \in \HozDDR$ we define
%% \beqs
%% \api(\vo,\vt) = \int_D \IP{A \grad \vo}{\vt} - ik\int_{\GI} \vo\vtbar.
%% \eeqs
%% \ede

%% \ble[Related PDE is well-posed and solution is in $H^2$]\label{lem:relatedwp}
%% If $A \in \CzoDRRRdtd,$ then the solution $\psi \in \HozDDR$ of the related PDE
%% \beq\label{eq:relpde}
%% \api(u,v) = \IPLtDR{f}{v}\quad \tfa\quad v \in \HozDDR
%% \eeq
%% exists, is unique, is in $\HtDR,$ and satisfies the a priori bound
%% % \beqs\label{eq:relpdehobound}
%% % \NW{\psi} \lesssim \frac{\max\set{\Amin^{-1},1}}k,
%% % \eeqs
%% % and
%% % \beqs
%% % \NHoD{\psi} \lesssim \CHoell \NLtD{f}
%% % \eeqs
%% % and
%% \beqs
%% \NHtDR{\psi} \lesssim \CHtell \NLtDR{f}.
%% \eeqs
%% for some constant $\CHtell > 0$ depending on $A,$ but independent of $k.$
%% \ele

%% \bre[Proof of \cref{lem:relatedwp}]
%% \Cref{lem:relatedwp} is proved in \cite{ChNiTo:18}, although the dependence on $A$ is not made explicit.
%% \ere

%% %\paragraph{Finite-Element Set-up} Let $\Vh$ be the first-order linear finite-element space on some mesh on $D$ with mesh size $h.$

%% \bas[Existence, uniqueness, and an a priori bound]\label{ass:bound}
%% We assume that the coefficients $A$ and $n$ are such that for all $k \geq \kz$ the solutions of the \cref{prob:vtedp} and its adjoint \eqref{eq:tedpadj} exist, are unique, are in $\HtDR,$ and satisfy the bound
%% \beq\label{eq:hhbound}
%% \NHtDR{u} \lesssim \CHthh \,k \mleft(\NLtDR{f} + \Nunsure{g} + \NLtGD{\gradGD \gD} + k \NLtGD{\gD}\mright),
%% \eeq
%% where $\gradGD$ is the surface gradient on $\GD,$ $u$ is the solution of the TEDP or its adjoint, and $\CHthh >0$ is a constant dependent on $A$, $n,$ and possibly $k.$
%% % \footnote{Determining the dependence of $\CHthh$ on $A$ and $n$ could be tricky. It was done in \cite{ChScTe:13} for a $C^2$ domain with scalar $A$ and homogeneous Dirichlet boundary conditions.}
%%  \eas

%% %% \bde[Finite-element approximation] 
%% %%  The finite-element approximation to \eqref{eq:tedp} is the following:
%% %% \beq\label{eq:tedpfe}
%% %% \text{Find } \uh \in \Vh\quad \tst\quad a(\uh,\vh) = F(\vh)\quad \tfa \vh \in \Vh,
%% %% \eeq
%% %% \ede

%% The main theorem we prove is the following:

%% \bth[Finite-element-error bound]\label{thm:febound}
%% If $A \in \CzoDRRRdtd,$ $h \lesssim 1/k,$ \cref{ass:bound} holds, and
%% \beq\label{eq:hcond}
%% h \lesssim \mleft(\NLiDRRR{n} \mleft(\Amax + \half\mright)\CHtell \CHthh\mright)^{-1/2}k^{-3/2}, % There should be a factor of a half in front of the right-hand side of this, as it makes things clearer what's going on in the proof. However, since we're doing everything with \lesssim, a factor of a half doesn't matter. We could replace the half with any \eps in (0,1), but then the constant hidden in the \lesssim in \eqref{eq:hherrltbound} has a factor 1/\eps.
%% \eeq
%% the finite-element solution $\uh$ to the \cref{prob:fevtedp} exists, is unique, and satisfies the bounds
%% \beq\label{eq:hherrltbound}
%% \NLtD{u-\uh} \lesssim \Cfemo \mleft(hk\mright)^2 \mleft(\NLtD{f} + \Nunsure{\gI} + \NLtGD{\gradGD \gD} + k \NLtGD{\gD}\mright)
%% \eeq
%% and
%% \beq\label{eq:hherrwbound}
%% \NW{u-\uh} \lesssim \mleft(\Cfemt hk +  \Cfemth h^2k^3\mright)\mleft(\NLtD{f} + \Nunsure{\gI} + \NLtGD{\gradGD \gD} + k \NLtGD{\gD}\mright),
%% \eeq
%% where
%% \beqs
%% \Cfemo \de \mleft(\Amax + \half\mright)\CHthh^2,
%% \eeqs
%% \beqs
%% \Cfemt \de \frac{\Amax+\half}{\Amin} \CHthh,
%% \eeqs
%% \beqs
%% \Cfemth \de \frac{\mleft(\Amin+ \NLiDRRR{n}\mright)^{1/2}}{\Amin^{1/2}}\Cfemo,
%% \eeqs
%% and $u$ is the solution of \cref{prob:vtedp}.
%% \enth

%% \subsubsection{Properties of the Elliptic Projection, and a related PDE}

%% The proof technique we use below (adapted from \cite{FeWu:11,ChNi:18}) uses an `elliptic projection' of the solution of the TEDP using the related sesquilinear form $\api.$ We define the energy norm induced by the sesquilinear form $\api$:
%% \beqs
%% \Npi{\vo} = \sqrt{\abs{\api(\vo,\vo)}}.
%% \eeqs

%% \ble[Energy Norm is a norm]\label{lem:inducednorm}
%% The induced norm $\Npi{\cdot}$ is a norm on $\HoD.$
%% \ele

%% \bpf[Proof of \cref{lem:inducednorm}]
%% The main thing to check is that, for $v \in \HoD,$ $\Npi{v}=0 \implies v=0.$ By construction, if $\Npi{v}=0,$ then $\int_{D} \IP{A \grad v}{\grad v} =0$ and $\NLtGI{v}^2 = 0,$ as these are the real and imaginary parts of $\api(v,v).$ By \eqref{eq:AellEDP}, it follows that $\Amin \abs{\grad v}^2 \leq 0,$ and thus $v$ is constant. As $\NLtGI{v} = 0,$ it follows that $\trGI v =0,$ and hence by the trace theorem, as $v$ is constant, it follows that $v=0.$

%% Other properties of norms follow analagously as with any definition of an energy norm.
%% \epf
%% \ble[Energy norm is equivalent to weighted norm]\label{lem:normbound}
%% If $v \in \HoD,$ then
%% \beq\label{eq:boundew}
%% \Npi{v} \lesssim \sqrt{\Amax+\half}\NW{v}
%% \eeq
%% and
%% \beq\label{eq:boundwe}
%% \NW{v} \lesssim \max\set{\Amin^{-\half},1} \Npi{v}
%% \eeq
%% \ele

%% \bpf[Proof of Lemma \ref{lem:normbound}]
%% To show \eqref{eq:boundew}, for $ v \in \HoD$ we have
%% \begin{align*}
%%   \Npi{v}^2 &= \abs{\api(v,v)}\\
%%             &\lesssim \abs{\int_{D} \IP{A \grad v}{\grad v}} + k\NLtGI{v}^2 \\
%%             &\lesssim \abs{\int_{D} \IP{A \grad v}{\grad v}} + k\NLtD{v}\NHoD{v}, \text{ by the multiplicative trace inequality}\\
%%             &\lesssim \Amax \NLtD{\grad v}^2 + \half k^2 \NLtD{v}^2 + \half \NHoD{v}^2\\
%%   &\lesssim \mleft(\Amax+\half\mright)\NW{v}^2
%% \end{align*}
%% as required.

%% To show \eqref{eq:boundwe} we first show that, for $v \in \HoD,$ $\Npi{v} \gtrsim \min\set{\Amin^{\half},1} \mleft(\NLtD{\grad v} + k^{\half} \NLtGI{\trGI v}\mright)$:
%% \begin{align}
%%   \Npi{v} &= \mleft(\abs{\api(v,v)}\mright)^{\half}\nonumber\\
%%           &= \mleft(\mleft(\int_D \IP{A \grad v}{\grad v}\mright)^2 + k^2 \mleft(\int_{\GI}\abs{\trGI v}^2\mright)^2\mright)^{\quarter}\nonumber\\
%%   &\geq \mleft(\mleft(\int_D \Amin \abs{\grad v}\mright)^2 + k^2 \NLtGI{\trGI v}^4\mright)^{\quarter}\nonumber\\
%%           &= \mleft(\Amin^2 \NLtD{\grad v}^4 + k^2 \NLtGI{\trGI v}^4\mright)^{\quarter}\nonumber\\
%%           &\geq \min\set{\Amin^{\half},1}\mleft(\NLtD{\grad v}^4 + k^2 \NLtGI{\trGI v}^4\mright)^{\quarter}\nonumber\\
%%   &\gtrsim \min\set{\Amin^{\half},1} \mleft(\NLtD{\grad v} + k^{\half} \NLtGI{\trGI v}\mright), \text{ as } \mleft(x+y\mright)^4 \lesssim x^4 + y^4.\label{eq:Npifour}
%% \end{align}

%% We recall the fact that for $v \in \HoD,$
%% \beq\label{eq:poincarelike}
%% \NLtD{v} \lesssim \NLtD{\grad v} + \NLtGI{\trGI v},
%% \eeq
%% see, e.g., \cite[Equation (6.16)]{Sp:15}. We can then prove \eqref{eq:boundwe}:
%% \begin{align*}
%%    \NW{v} &\lesssim \NLtD{v}+ \NLtD{\grad v}\\
%%           &\lesssim \NLtGI{\trGI v}+ \NLtD{\grad v} + \NLtD{\grad v}\text{ by \eqref{eq:poincarelike}}\\
%%           &\lesssim k^{\half}\NLtGI{v} + \NLtD{\grad v}\\
%%   &\lesssim \max\set{\Amin^{-\half},1}\Npi{v}, \text{ by \eqref{eq:Npifour}.}
%% \end{align*}
%% \epf

%% % \ble[Bound on $L^2$ norm]\label{lem:ltbound}
%% % If $v \in \HoD$ then the bound
%% % \beqs
%% % \NLtD{v} \lesssim  \NLtGI{\trGI v} + \NLtD{\grad v}
%% % \eeqs
%% % holds.
%% % \ele

%% % \bpf[Proof of \cref{lem:ltbound}]
%% % \optodo{Look at proof in IbyPs article}
%% % If $v \in \HozDD,$ then by the Poincar\'e inequality, we have that $\NLtD{v} \lesssim \NLtD{\grad v}.$ Alternatively, if $\GD = \emptyset$ and $\trGI v$ is constant, then $v - \trGI v \in \HozDD,$ and thus (abusing notation, and letting $\trGI v$ denote the value of the constant, and also a constant function defined on $D$ taking that value everywhere)
%% % \begin{align*}
%% %   \NLtD{v} &\leq \NLtD{\trGI v} + \NLtD{v-\trGI v}\\
%% %   &= \NLtGI{\trGI v} + \NLtD{v-\trGI v}\\
%% %            &\lesssim \NLtGI{\trGI v} +  \NLtD{\grad \mleft(v-\trGI v\mright)}\\
%% %              &= \NLtGI{\trGI v} + \NLtD{\grad v},
%% % \end{align*}
%% % as required.
%% % \epf

%% % \ble[Bound on weighted norm by energy norm]\label{lem:othernormbound}
%% % If $v \in \HozDD,$ or if $\GD = \emptyset$ and $\trGI v$ is constant, the bound
%% % \beq\label{eq:boundwe}
%% % \NW{v} \lesssim \max\set{\Amin^{-\half},1} \Npi{v}
%% % \eeq
%% % holds.
%% % \ele

%% % \bpf[Proof of \cref{lem:othernormbound}]
%% % \epf

%% We now define the elliptic projection of a function in $\HoD.$% and also define a related PDE that will be used in proving the approximation properties of the elliptic projection.

%% \bde[Elliptic Projection]
%% For $w \in \HoD$ we define the \defn{elliptic projection} $\Ph w \in \Vhp$ of $w$ by
%% \beq\label{eq:ellproj}
%% \api(\vh,\Ph w) = \api(\vh,w) \tfa \vh \in \Vhp.
%% \eeq
%% \ede

%% % \bde[Related PDE]\label{lem:relpde}
%% % Given $f \in \LtD$ we define the related (adjoint) PDE; find $\psi \in \HoD$ such that for all $v \in \HoD$
%% % \beq\label{eq:relpde}
%% % \api(\psi,v) = \IPLtD{f}{v}.
%% % \eeq\ede

%% % \bpf[Proof of \cref{lem:relatedwp}]
%% % By \eqref{eq:boundwe} we have, for $v \in \HozDD$
%% % \beqs
%% % \min\set{\Amin,1}\NW{v}^2 \lesssim \abs{\api(v,v)},
%% % \eeqs
%% % and we also have that
%% % \beqs
%% % \NWs{\IP{f}{\cdot}} \leq \frac1k \NLtD{f},
%% % \eeqs
%% % where $\NWs{\cdot}$ denotes the norm on $\HozDDs$ induced by $\NW{\cdot}.$ By the Lax--Milgram Theorem, we can therefore conclude that $\psi$ exists, is unique, and satisfies the bound
%% % \beqs
%% % \NW{\psi} \lesssim \frac{\max\set{\Amin^{-1},1}}{k}\NLtD{f}.
%% % \eeqs
%% % Use Grisvard Magic to get $H^2.$\optodo{this}
%% % \epf

%% \ble[Properties of elliptic projection]\label{lem:ellprojbounds}
%% Let $A \in \CzoDRRRdtd.$ If $w \in \HtDR,$ then the elliptic projection $\Ph w$ exists, is unique, and the error satisfies the bounds
%% \beq\label{eq:ellprojenbound}
%% \Npi{w-\Ph w} \lesssim \sqrt{\Amax+\half}\,h\NHtDR{w},
%% \eeq
%% and
%% \beq\label{eq:ellprojltbound}
%% \NLtDR{w-\Ph w} \lesssim  \mleft(\Amax+\half\mright)\CHtell\,h^2\NHtDR{w}.
%% \eeq
%% \ele

%% \bpf[Proof of \cref{lem:ellprojbounds}]
%% We first assume $\Ph w$ exists. To show \eqref{eq:ellprojenbound} we apply C\'{e}a's Lemma in $\Vhp$ using the energy norm $\Npi{\cdot}$ to conclude
%% \beqs
%% \Npi{w-\Ph w} \leq \Npi{w-\Ih w}.
%% \eeqs
%% We then apply \cref{lem:normbound,lem:scottzhangbound} to conclude \eqref{eq:ellprojenbound}.

%% To prove \eqref{eq:ellprojltbound} we let $\psi$ solve the related PDE \eqref{eq:relpde} with $f = w-\Ph w.$ By \cref{lem:relatedwp} $\psi \in \HtDR$ and thus by  \cref{lem:normbound} and \cref{lem:scottzhangbound}
%% \beqs
%% \Npi{\psi - \Ih \psi} \lesssim \sqrt{\Amax + \half}\CHtell \,h\NLtDR{w-\Ph w}.
%% \eeqs

%% If we now set $v = w-\Ph w$ in \eqref{eq:relpde}, then we obtain
%% \begin{align}
%%   \NLtDR{w - \Ph w}^2 &= \api\mleft(\psi,w-\Ph w\mright)\nonumber\\
%%                      &= \api\mleft(\psi-\Ih \psi,w-\Ph w\mright) \text{ by Galerkin orthogonality for } w-\Ph w\nonumber\\
%%                      &\leq \Npi{\psi-\Ih \psi}\Npi{w-\Ph w}\nonumber\\
%%                        &\lesssim \sqrt{\Amax + \half}\CHtell \,h\NLtDR{w-\Ph w}\Npi{w-\Ph w}\label{eq:epltfinal}.
%% \end{align}
%% By cancelling $\NLtDR{w- \Ph w}$ from both sides of \eqref{eq:epltfinal} and using \eqref{eq:ellprojenbound} we obtain \eqref{eq:ellprojltbound}.

%% We have proved the bounds \eqref{eq:ellprojenbound} and \eqref{eq:ellprojltbound} under the assumption of existence. To show uniqueness, suppose $\wh, \whtilde$ both satisfy \eqref{eq:ellproj} (with $\Ph = \wh$ or $\whtilde$ respectively). Then by linearity, for all $\vh \in \Vhp,$
%% \beqs
%% \api\mleft(\vh,\Ph\mleft(\wh-\whtilde\mright)\mright) = \IP{\vh}{w-w} = 0.
%% \eeqs
%% That is, the function $\wh - \whtilde$ is an elliptic projection of the zero function.

%% Therefore, by \eqref{eq:ellprojltbound} $\NLtDR{0 - \mleft(\wh - \whtilde\mright)} \lesssim 0,$ i.e., $\wh = \whtilde.$ Therefore, if the elliptic projection $\Ph w$ exists, it is unique. As the space $\Vhp$ is finite-dimensional, by the Rank--Nullity Theorem, the uniqueness of $\Ph w$ implies its existence; hence $\Ph w$ exists, and is unique, as required.
%% \epf


%% \subsubsection{Proof of Main Result}

%% We let $\Ih$ denote the Scott--Zhang quasi-interpolant in $\Vhp$ (see \cite{ScZh:90}), and  use its following property.
%% \ble[Properties of Scott-Zhang interpolant]\label{lem:scottzhangbound}
%% let $h \lesssim 1/k.$ If $w \in \HtDR,$ then
%% \beq\label{eq:scottzhangbound}
%% \NW{w - \Ih w} \lesssim h \NHtDR{w}.
%% \eeq
%% \ele

%% \bpf[Proof of \cref{lem:scottzhangbound}]
%% The Scott-Zhang interpolant $\Ih w$ satisfies
%% \beq\label{eq:szlt}
%% \NLtDR{w-\Ih w} \lesssim h^2 \NHtDR{w}
%% \eeq
%% \and
%% \beq\label{eq:szho}
%% \NHoDR{w-\Ih w} \lesssim h \NHtDR{w}.
%% \eeq
%% Hence by the definition of $\NW{\cdot},$ by combining \eqref{eq:szlt} and \eqref{eq:szho} we have
%% \beqs
%% \NW{w-\Ih w} \lesssim h\mleft(1+hk\mright)\NHtDR{w}.
%% \eeqs
%% As $h\lesssim 1/k,$ \eqref{eq:scottzhangbound} follows.
%% \epf

%% The following \lcnamecref{cor:hhszbound} follows from \cref{ass:bound}, and is used to prove \cref{thm:febound}.

%% \bco\label{cor:hhszbound}
%% If $u$ is the solution of the Helmholtz Interior Impedance Problem (or its adjoint) then the error in the Scott--Zhang quasi-interpolant satisfies
%% \beq\label{eq:hhszbound}
%% \NW{u-\Ih u} \lesssim \CHthh hk \mleft(\NLtDR{f} + \Nunsure{g} + \NLtGD{\gradGD \gD} + k \NLtGD{\gD}\mright).
%% \eeq
%% \eco

%% The proof of the main theorem (\cref{thm:febound} below) also uses the fact that $a$ satisfies a G\r{a}rding inequality.
%% \ble[G\r{a}rding inequality]
%% If $v \in \HozDDR,$ then
%% \beq\label{eq:garding}
%% \Re\mleft(a(v,v)\mright) \geq \Amin \NW{v}^2 - k^2\mleft(\Amin + \NLiDRRR{n}\mright)\NLtDR{v}^2,
%% \eeq
%% where $\Re$ denotes the real part.
%% \ele

%% Finally, we recall \defn{Cauchy's inquality}: For all $a,b \in \RR$, and for all $\eps > 0,$
%% \beq\label{eq:cauchy}
%% ab \leq \frac{a^2}{2\eps} + \frac{\eps b^2}{2}.
%% \eeq

%% \bpf[Proof of \cref{cor:hhszbound}]
%% The proof follows from \cref{lem:scottzhangbound,ass:bound}.
%% \epf

%% We are now in a position to prove our main theorem.




%% \bpf[Proof of \cref{thm:febound}]
%% In this proof, for brevity we let
%% \beqs
%% \Mfg = \NLtDR{f} + \Nunsure{\gI} + \NLtGD{\gradGD \gD} + k \NLtGD{\gD}.
%% \eeqs
%% By \cref{ass:bound} the solution $u$ of the TEDP exists and is unique. Assume the finite-element solution $\uh$ exists. Let $\xi \in \HoD$ satisfy the adjoint TEDP \eqref{eq:tedpadj} with $f=u-\uh,$ $\gD=0,$ and $\gI=0.$ Taking complex conjugates, it follows that
%% \beq\label{eq:errordual}
%% a(v,\xi) = \IPLtDR{v}{u-\uh} \tfa v \in \HoDR.
%% \eeq
%% By \cref{ass:bound} $\xi$ exists, is unique and is in $\HtDR.$. Setting $v = u-\uh$ in \eqref{eq:errordual} we obtain
%% \begin{align*}
%%   \NLtDR{u-\uh}^2 &= a\mleft(u-\uh,\xi\mright)\\
%%                  &= a\mleft(u-\uh,\xi-\Ph \xi\mright) \quad\text{by Galerkin orthogonality for } u-\uh\\
%%                  &= \api\mleft(u-\uh,\xi-\Ph \xi\mright) - k^2 \IPLtDR{n\mleft(u-\uh\mright)}{\xi-\Ph \xi}\\
%%                  &= \api\mleft(u-\Ih u,\xi-\Ph \xi\mright) - k^2 \IPLtDR{n\mleft(u-\uh\mright)}{\xi-\Ph \xi}\\
%%   &\quad\quad\quad\text{by Galerkin orthogonality for }\xi  - \Ph \xi\\
%%                  &\leq \Npi{u-\Ih u}\Npi{\xi - \Ph \xi} + \NLiDRRR{n} k^2 \NLtDR{u-\uh}\NLtDR{\xi-\Ph \xi}\\
%%                  &\lesssim \sqrt{\Amax + \half}\, \CHthh\, hk \Mfg\Npi{\xi-\Ph \xi}\\
%%   &\quad\quad+  \NLiDRRR{n} k^2 \NLtDR{u-\uh}\NLtDR{\xi-\Ph \xi}\quad \text{by \eqref{eq:boundew} and \eqref{eq:hhszbound}}\\
%%                  &\lesssim \sqrt{\Amax + \half}\, \CHthh\, hk\Mfg\,\sqrt{\Amax + \half}\,h\NHtDR{\xi}\\
%%                  &\quad\quad  + \NLiDRRR{n} k^2 \NLtDR{u-\uh}\NLtDR{\xi-\Ph \xi}\quad \text{by \eqref{eq:ellprojenbound}}\\
%%                  &\lesssim \mleft(\Amax + \half\mright)\CHthh\, hk\Mfg \, \CHthh \,hk \NLtDR{u-\uh}\\
%% &\quad\quad  + \NLiDRRR{n} k^2 \NLtDR{u-\uh}\NLtDR{\xi-\Ph \xi}\quad \text{by \eqref{eq:hhbound}}\\
%% \end{align*}
%% Cancelling a factor of $\NLtDR{u-\uh}$ and rearranging terms we obtain
%% \beqs
%% \NLtDR{u-\uh} \lesssim \mleft(\Amax + \half\mright)\CHthh^2 \mleft(hk\mright)^2\Mfg + k^2 \NLiDRRR{n} \NLtDR{\xi - \Ph \xi}
%% \eeqs
%% and therefore
%% \begin{align*}
%% &  \NLtDR{u-\uh} \lesssim \mleft(\Amax + \half\mright)\CHthh^2 \mleft(hk\mright)^2 \Mfg\\
%% &\quad\quad  + h^2k^3 \NLiDRRR{n} \mleft(\Amax + \half\mright) \CHtell \CHthh \NLtDR{u-\uh}
%% \end{align*}
%%   using the definition of $\xi$, \eqref{eq:ellprojltbound}, and \eqref{eq:hhbound}Therefore if $h$ satisfies \eqref{eq:hcond} we obtain \eqref{eq:hherrltbound}.
%% % \beqs
%% % \half \NLtD{u-\uh} \lesssim \mleft(\Amax + \half\mright)\CHthh \mleft(hk\mright)^2 \mleft(\NLtD{f} + \Nunsure{g}\mright).
%% % \eeqs
%% % that is, if $\Cmess \de (\mleft(\NLiDRR{n} \mleft(\Amax + \half\mright) \CHthh \mright)^{-1/2},$ then
%% % \beqs
%% % \NLtD{u-\uh} \lesssim \mleft(\Amax + \half\mright)\CHthh \Cmess^2k^{-1} \mleft(\NLtD{f} + \Nunsure{g}\mright),
%% % \eeqs
%% To obtain the bound \eqref{eq:hherrwbound}, we use the G\r{a}rding inequality \eqref{eq:garding}:
%% \begin{align*}
%%   \Amin \NW{u-\uh}^2 &\leq \Re\mleft(a\mleft(u-\uh,u-\uh\mright)\mright) + k^2\mleft(\Amin+ \NLiDRRR{n}\mright) \NLtDR{u-\uh}^2\\
%%                      &= \Re\mleft(a\mleft(u-\uh,u-\Ih u\mright)\mright) + k^2\mleft(\Amin+ \NLiDRRR{n}\mright) \NLtDR{u-\uh}^2\\
%%   &\quad\quad\quad\text{by Galerkin orthogonality}\\
%%                      &\leq \mleft(\Amax+\half\mright) \NW{u-\uh}\NW{u-\Ih u} + k^2\mleft(\Amin+ \NLiDRRR{n}\mright) \NLtDR{u-\uh}^2\\
%%   &\quad\quad\quad\text{by \cref{lem:normbound}}\\
%%   &\leq \frac{\mleft(\Amax+\half\mright)^2}{2\Amin} \NW{u-\Ih u}^2 + \frac{\Amin}2 \NW{u-\uh}^2 + k^2\mleft(\Amin+ \NLiDRRR{n}\mright) \NLtDR{u-\uh}^2,\\
%% \end{align*}
%% by Cauchy's inequality \eqref{eq:cauchy} with $\eps = \Amin.$ Therefore,
%% \beqs
%% \NW{u-\uh}^2 \leq \frac2{\Amin} \mleft(\frac{\mleft(\Amax+\half\mright)^2}{2\Amin} \NW{u-\Ih u}^2+ k^2\mleft(\Amin+ \NLiDRRR{n}\mright) \NLtDR{u-\uh}^2\mright)
%% \eeqs
%% and hence
%% \beq\label{eq:hherrboundnearly}
%% \NW{u-\uh} \lesssim \frac1{\Amin^{1/2}} \mleft(\frac{\Amax+\half}{\Amin^{1/2}} \NW{u-\Ih u}+ k\mleft(\Amin+ \NLiDRRR{n}\mright)^{1/2} \NLtDR{u-\uh}\mright).
%% \eeq
%% By substituting \eqref{eq:hhszbound} and \eqref{eq:hherrltbound} into \eqref{eq:hherrboundnearly} we obtain \eqref{eq:hherrwbound}.

%% To show that $\uh$ exists, as in the proof of \cref{lem:ellprojbounds} we can use the error bound \eqref{eq:hherrltbound} to show that $\uh$ is unique, and we can then use the fact that $\Vhp$ is finite-dimensional to show that $\uh$ exists.
%% \epf

\section{New Finite-Element-Error bounds for the Heterogeneous Helmholtz Equation}\label{sec:fem}

We now prove our new error bounds for the higher-order finite-element approximation of the solution of the Helmholtz equation in heterogeneous media. Our results are a generalisation of results proved by Du and Wu \cite{DuWu:15} for higher-order finite-element approximations of the Helmholtz equation in homogeneous media; our results and proofs broadly follow those in \cite{DuWu:15}, with the differences that: (i) we modify the proofs to cater for the heterogeneity of the coefficients, and (ii) the dependence of our results on all of the constants involved is explicit. In particular our results are (in principle) explicit in $A$  and $p$ and are explicit in $n$ and $k$.

The proofs of our results have many parts, and appear technical, largely due to the burden of explicitly keeping track of all of the constants involved. However, in essence, the proof consists of two ideas:
\ben
\item Decompose the error $u-\uh = (u - \Ph u) + (\Ph u - \uh)$, where $\Ph u$ is an `elliptic projection' of $u$, and
\item Bound the error $u - \Ph u$ using the fact that $\Ph u$ is a Galerkin projection.
\item Bound the error $\Ph u - \uh$ in higher-order `discrete' norms.
\een

In order to use this notion of discrete norms, we develop a notion of discrete derivatives. Also, a number of different Galerkin projections (including the `elliptic projection' mentioned above) play a key role in the proofs, and so we prove error bounds for these projections. Our results also use best approximation results for the Helmholtz equation, and we prove these, following the techniques in \cite{ChNi:18a}, but ensuring that these results are explicit in all of the constants involved.
\optodo{Put in structure summary once it's settled down.}
As we are using higher-order finite elements (which we assume are of degree $p$), we will require extra smoothness assumptions on $A$, $n,$ and the boundaries $\GD$ and $\GI$. We also make simple assumptions on $k,$ $\nmin$, and $\NLiD{n}$ in order to simplify our calculations.
\optodo{Correct notation in variational formulation of TEDP}

\subsection{Finite-element-error bounds}

\bas[Assumptions for higher-$p$ FEM bounds]\label{ass:highp}
Assume
\bit
\item $\GD$ and $\GI$ are $\Cpo$
\item $\Aij \in \CpmooDclos$ for all $i,j$, and
\item $n \in \LiD \cap \HmD \cap \CfloordtD.$
%\item $f \in \HpmoD$
%  \item $\gI \in \HpmhGI$
  \eit
\eas

Also, throughout this \lcnamecref{sec:fem} we assume we are working with the finite-element space defined in \cref{def:fespace}, and that consequently \cref{lem:scottzhang} holds.

We make the following \lcnamecref{ass:htwo} on the solution of \cref{prob:vtedp}.

\bas\label{ass:htwo}
\Cref{prob:vtedp} (and its adjoint) has a unique solution $u$ in $\HtD$, and there exists $\CAnk>0$ (possibly dependent on $A$, $n,$ and $k$) such that
\beq\label{eq:generalhtwo}
k \NLtD{u} + \SNHoD{u} + \frac1k \SNHtD{u} \leq \CAnk \Cfg,
\eeq
where $\Cfg \de \NLtD{f} + \NLtGI{\gI}$.
\eas

Finally, we make the following \lcnamecref{ass:convenient} to simplify the proofs in this section. \Cref{ass:convenient} is by no means necessary, but greatly simplifies the proofs.

\bas[Assumptions for convenience of proofs]\label{ass:convenient}
Assume: $k \geq 1$, $\NLiD{n} \geq 1,$ $\nmin \leq 1$, $hk \leq 1$, and there exists $\Ctildemin > 0$ independent of $k$ such that $\CAnk k \geq \Ctildemin.$
\eas
%Note that one could require $\NHhGI{\gI}$ on the right-hand side of \cref{eq:generalhtwo} (as $\gI \in \HhGI$); however, since the bound in \cref{thm:tedp} only has $\NLtGI{\gI}$, we use the form \cref{eq:generalhtwo} to include \cref{thm:tedp}.

Throughout this \lcnamecref{sec:fem}, we adopt the following piece of notation.
\beqs
\nvar = \frac{\NLiD{n}}{\nmin},
\eeqs
%% and
%% \beqs
%% \En = \max\set{\frac{\NHsD{n}}{\nmin^2} \st s \in (d/2,p-1];\frac{\NWsiD{n}}{\nmin^2}\st s \in [2,d/2];1}\nvar.
%% \eeqs
%The quantity $\En$ arises from bounds on the `elliptic projection' mentioned above in weighted norms, see \cref{lem:ellprojerrw} below.

Our main \lcnamecref{thm:fembound} is the following \lcnamecref{thm:fembound}.
\bth[Higher-order error bound for the heterogeneous Helmholtz equation]\label{thm:fembound}
Under \cref{ass:highp,ass:htwo}, let $u$ be the solution of \cref{prob:vtedp}. Under \cref{ass:highp,ass:htwo,ass:convenient}, there exist constants $\CFEMLt, \CFEMHo, \Chcond > 0$, independent of $h$, $k$ and $n$, and a function $\Condn(n) >0$ (independent of $h$ and $k$) such that if
\beq\label{eq:hfemcond}
h \leq \Chcond \Condn(n) \CAnk^{-\frac1{2p}}k^{-1-\frac1{2p}},
\eeq
then the finite-element solution $\uh$ exists, is unique, and satisfies the error bounds
\beq
\NLtD{u-\uh} \leq \CcorLt \CLtn(n)\mleft(h^2 + \CAnk^2 (hk)^{2p}\mright)\Cfg\tand\label{eq:femltbound}
\eeq
\beq
\NW{u-\uh} \leq \CcorHo \CHon(n)\mleft(h + \CAnk (hk)^p + \CAnk^2k (hk)^{2p}\mright)\Cfg,\label{eq:femhobound}
\eeq
where
\beqs
\Condn(n) = \Pfn{p-2}\mleft(\NLiD{n}\mright)^{-\frac1{p+1}}\mleft(\mleft(\En\nvar\mright)^{\half(\floor{\frac{p-1}2}+1)}\NLiD{n}\nmin^{-\frac{p}2}\mright)^{-\frac{p-1}{2p}},
\eeqs
\beqs
\CLtn(n) = \nvar^6\nmin^{-(p+1)} \mleft(\En \nvar\mright)^{\floor{\frac{p-1}2}+1}\Pfn{p-2}\mleft(\NLiD{n}\mright)m\tand
\eeqs
\beqs
\CHon(n) = \mleft(1+\mleft(\En\nvar\mright)^{\half(\floor{\frac{p-1}2}+1)}\NLiD{n}\nmin^{-\frac{p}2}\mright)\CLtn,
\eeqs
where the polynomial $\Pfn(x)$ is defined in \cref{eq:p} below, and the $n$-dependent constant $\En$ is defined in \cref{eq:en} below.
\enth
%\bco[Higher-order-finite-element bound]\label{cor:fembound}
%Under the assumptions of \cref{thm:fembound},
%\eco



Whilst the calculations in this \lcnamecref{sec:fem} are explicit in all the constants involved, these dependencies are complex and, to a large extent, unnecessary to understand the arguments. Therefore, for ease of reference, the definition of all the constants (which are many-layered and interdependent) are given in \cref{app:constants}; i.e., any constant introduced or defined in this \lcnamecref{sec:fem} will be listed in \cref{app:constants}.

\subsubsection{Discussion of new finite-element-error bounds}

\bre[Relationship of new bounds to the work of Du and Wu]
In \cite{DuWu:15} Du and Wu proved that the error for high-order finite-element methods for the \emph{homogeneous} Helmholtz interior impedance problem is bounded provided $h^{2p}k^{2p+1}$ is sufficiently small. Our proof follows theirs, and achieves analagous results---observe that the condition \cref{eq:hfemcond} requires $h^{2p}k^{2p+1}$ to be sufficiently small, where the `sufficient smallness' condition depends on $A$ and $n$. We note that our results and proof have the following modifications to those of Du and Wu (the modifications listed in order of their impact upon the proof).
\ben
\item We prove bounds for \emph{heterogeneous} $A$ and $n$; in particular, in several places in the proof we must work with $n$-weighted inner products and norms; see \cref{rem:why} for more details on why $n$-weighted inner products and norms are required.
\item We explicitly track all of the constants involved in the proof---out results are completely explicit in $n$, and are, in theory, explicit in $A$ (see \cref{rem:explicita} below for information on the explicitness in $A$.
\item We allow the possibility that the Helmholtz problem is trapping---the constant $\CAnk$ appearing in \cref{eq:hfemcond} may depend on $k$ (as well as $A$ and $n$). If the Helmholtz problem was nontrapping, $\CAnk$ would be independent of $k$.
  \item We assume the existence of a Dirichlet scatterer $\Dm,$ as opposed to only considering the interior impedance problem, where $\Dm=\emptyset.$
\een
\ere

\bre[Why are the results not fully explicit in $A$?]\label{rem:explicita}
The condition \cref{eq:hfemcond} and the bounds \cref{eq:femlt,femho} are not fully explicit in $A$, i.e. the constants $\Ccond, \CcorLt,$ and $\CcorHo$ may depend on $A$. This dependence is because the constants in the shift theorem for the stationary diffusion equation (\cref{thm:shift}) below are not explicit in their $A$-dependence.\optodo{Check there's no other places this comes up.} In principle one can determine this dependence (it is determined for a right-hand side in $\LtD$ and a solution in $\HtD$ in \cite[Appendix A]{ChScTe:13}).
\ere

\bre[Why the appearance of $\nvar$?]\label{rem:nvar}
The quantity $\nvar = \NLiD{n}/\nmin$ appears in multiple places in the definition of the $n$-dependent constants $\Condn(n), \CLtn(n),$ and $\CHon(n).$ This appearance is mainly due to the fact that in multiple places in the proof of \cref{thm:fembound}, we must convert from working in an $n$-weighted norm to a standard norm, and then later convert back to an $n$-weighted norm again. The necessity for this conversion is frequently because certain results (such as \cref{thm:shift,lem:poincare,thm:trace,thm:multiplicativetrace}) are only available in the literature in standard (non-$n$-weighted) norms, and so to apply these results we must first transfer to non-$n$-weighted norms, apply the results, and then tranfer back. If one could prove these results for $n$-weighted norms (under sufficient smoothness conditions on $n$) with constants that were completely explicit in $n$, then many of the instances of $\nvar$ could be removed.
\ere

\bre[Bounds not sharp in $n$]\label{rem:nsharp}
The $n$-dependence of the constants $\Condn(n), \CLtn(n),$ and $\CHon(n)$ is almost certainly not sharp, because
\ben
\item The proof of \cref{thm:fembound} is complex, and involves recursively applying bounds on finite-element functions (as in, e.g., the proof of \cref{lem:negdiscsum}, see \cref{eq:evenrecursivesum,eq:oddrecursive}). These arguments may well result in non-sharp $n$-dependence.
  \item As described in \cref{rem:nvar} above, many of the appearances of $\nvar$ in the constants $\Condn(n), \CLtn(n),$ and $\CHon(n)$ is purely due to changing norms, then changing back, and so it is possible that the dependence of these constants on $\nvar$ (and therefore $n$) is not sharp.
\een
\ere

\bre[$p$-dependence]
if one wants to prove error bounds for $hp$-finite-element methods, then one must know explicitly how the error bounds \cref{eq:femltbound,eq:femhobound} depend on $p$ as well as $h$ and $k$; such an analysis was carried out for the Helmholtz equation in homogeneous media by Melenk and Sauter in \cite{MeSa:10,MeSa:11} and for the related time-harmonic Maxwell's equation in homogeneous media in \cite{MeSa:18}. In principle (as our results are explicit in all of the constants involved) one could calculate the $p$-dependence, however, such dependence is likely to be highly complicated, and may very well not be sharp, for similar reasons that the $n$-dependence of our results may not be sharp, as outlined in \cref{rem:nsharp} above.
\ere

\bre[Special cases of $n$]
In order to understand the constants $\Condn(n), \CLtn(n),$ and $\CHon(n)$ it may be instructive to note their behaviour in the following three cases:
\bit
\item If $n=1,$ then $\Condn(n)=1, \CLtn(n)=1,$ and $\CHon(n)=2$.
\item If $\nmin$ is fixed, and $\NLiD{n} \rightarrow \infty,$ then $\Condn(n)\rightarrow 0,$ (i.e., the condition \cref{eq:hfemcond} becomes more restrictive), $\CLtn(n)\rightarrow \infty,$ and $\CHon(n)\rightarrow \infty$.
\item If $\NLiD{n}$ is fixed, and $\nmin \rightarrow 0,$ then $\Condn(n)\rightarrow 0,$ $\CLtn(n)\rightarrow \infty,$ and $\CHon(n)\rightarrow \infty$.
  \eit
  One can easily check each of these behaviours using the definition of $\nvar$, the definition of the constants $\Condn(n),$ $\CLtn(n)$, and $\CHon(n)$, and the definition of the polynomial $\Pfn(x)$ in \cref{eq:p} below, and the defintion of $\En$ in \cref{eq:en} below.
  \ere

  \bre[Extension to different boundary conditions on $\GI$]
  As in \cite[Remark 5.4(e)]{DuWu:15}, we remark that it is not obvious how to extend the proof of \cref{thm:fembound} to a Helmholtz problem with an exact Dirichlet-to-Neumann boundary condition on $\GI.$ In \cref{lem:higherbound,lem:thetahbound} one must bound terms involving $\thetah$ on the truncation boundary $\GI$; these bounds are achieved using \cref{lem:boundarybound}, which bounds $\NLtGI{\thetah}$ by higher-order discrete norms of $\thetah$ and the weighted $H^1$-norm of $\rho.$ However, a crucial part of the proof of \cref{lem:boundarybound} is the fact that one has the bound $k \NLtGI{\thetah}^2 \lesssim \IPLtGI{\DtNapprox \thetah}{\thetah}$, where $\DtNapprox$ is the impedance approximation to the DtN map $\DtN,$ i.e. $\DtNapprox = ik.$

  To replicate the proof of \cref{lem:boundarybound} for an exact DtN boundary condition, one would need to show that $k \NLtGI{\thetah}^2 \lesssim \IPLtGI{\DtN \thetah}{\theta}$ for the \emph{exact} DtN map $\DtN.$ As mentioned in \cite[Remark 5.4(e)]{DuWu:15}, whether such a bound holds in general is an open question.\optodo{Melenk and Sauter '10 equation 3.4b, unlikely, but not that interesting numerically.}

\optodo{Understand PML papers, then maybe write: However, there may be more promise in extending the results in this \namecref{sec:fem} to \cref{prob:edp} truncated with a PML. The formulation of the PML problem (as in, e.g., \cite[Equations (2.5)--(2.7)]{LiWu:19}) involves ...}
  \ere

\subsection{Decomposition of solution and best approximation bound}

For the first part of the proof of \cref{thm:fembound}, we prove a best approximation bound in $\Vhp$ for the solution of the Helmholtz equation; following the presentation in \cite{ChNi:18a} (although we explicitly keep track of the constants involved at each point). In order to obtain bounds for high $p$, we require the following shift \lcnamecref{thm:shift}:

\bth[Shift theorem]\label{thm:shift}
Under \cref{ass:highp}, for all integers $l \in \mleft[0,p-1\mright]$ there exists a constant $\CAl>0$ (depending on $A$) such that if $\ftilde \in \HlD$ and $\gItilde \in \HlphGI$, then there exists a unique $\utilde \in \HlptD$ such that $\utilde$ solves
\beq\label{eq:shifteq}
\grad \cdot \mleft(A \grad \utilde\mright) = -\ftilde,
\eeq
\beq\label{eq:shiftdbc}
\dn \utilde = \gItilde, \tand
\eeq
\beq\label{eq:shiftnbc}
\trD \utilde = 0
\eeq
and $\utilde$ satisfies the bound
\beq\label{eq:shift}
\NHlptD{\utilde} \leq \CAl \mleft(\NHlD{\ftilde} + \NHlphGI{\gItilde}\mright).
\eeq
\enth

\bpf[Proof of \cref{thm:shift}]
The uniqueness and existence of $\utilde$ (in $\HozDD$) follows from the Lax--Milgram theorem, as the variational formulation of \cref{eq:shifteq,eq:shiftdbc,eq:shiftnbc} is bounded and coercive. The proof for the higher regularity bounds uses standard elliptic regularity estimates in the interior and near the boundaries $\GD$ and $\GI$, and the work of the proof is combining these estimates. As a reference for these estimates we use \cite[pp. 137-138]{Mc:00}; \cref{ass:highp} means we can apply these results, as we have higher regularity of the boundaries $\GD$ and $\GI$ and the coefficient $A$.

To deal with the interior regularity and regularity near the boundary separately, we define the following subsets of $D$: $\Dint,\Dinttilde,\Dscat,$ and $\Dtrunc$\optodo{Put this in a picture - sketch is in Research 17, notes for 17th May} with the following properties:
\bit
\item $\Dint \compcont \Dinttilde \compcont D$,
\item $\GD \subset \Dscatclos$
\item $\dist(\Dscat,\GI) > 0 $
  \item $\GI \subset \Dtruncclos$
\item $\dist(\Dtrunc,\GD) > 0 $
    \eit
    First applying interior regularity \cite[Theorem 4.16]{Mc:00} in $\Dinttilde,$ we obtain the bound
    \beq\label{eq:shiftint}
\NHlptDint{\utilde} \leq \CintAl \mleft(\NHoDinttilde{\utilde} + \NHlDinttilde{\ftilde}\mright).
\eeq
Applying regularity up to the boundary for Dirichlet data \cite[Theorem 4.18 (i)]{Mc:00} in $\Dscat,$ we obtain (as $\trD \utilde = 0$)
\beq\label{eq:shiftscat}
\NHlptDscat{\utilde} \leq \CscatAl \mleft(\NHoD{\utilde} + \NHlD{\ftilde}\mright)
\eeq
and similarly\optodo{Spell checker} for Neumann data \cite[Theorem 4.18 (ii)]{Mc:00} in $\Dtrunc,$ we obtain
\beq\label{eq:shifttrunc}
\NHlptDtrunc{\utilde} \leq \CtruncAl \mleft(\NHoD{\utilde} + \NHlphGI{\dn \utilde} + \NHlD{\ftilde}\mright).
\eeq
Combining \cref{eq:shiftint,eq:shiftscat,eq:shifttrunc}, we obtain \cref{eq:shift}.
\epf




%% The following trace theorem is standard, see, e.g., \cite[Theorem 3.37]{Mc:00}.


The following \lcnamecref{lem:domainshift} follows from \cref{thm:shift}.

\bco\label{lem:domainshift}
Under \cref{ass:highp}, let $\ftilde \in \HlD$ and $\gtilde \in \HlpoD,$ for $0 \leq l \leq p-1$. If $\utilde \in \HlptD$ solves
\beqs
\grad \cdot \mleft(A\grad \utilde\mright) = -\ftilde,
\eeqs
\beqs \trGD u = 0,
\eeqs
and
\beqs
\dn u = \gtilde,
\eeqs
then
\beqs
\NHlptD{\utilde} \leq \CAl\mleft(1+\CTrlpo\mright)\mleft(\NHlD{\ftilde} + \NHlpoD{\gtilde}\mright).
\eeqs
\eco

The proof of \cref{lem:domainshift} requires the Trace theorem.

\bth[Trace Theorem]\label{thm:trace}
If $v \in \HmD,$ for $1/2 < m \leq p+1$, then there exists $\CTrm > 0$ independent of $v$ such that
\beqs
\NHmmhGI{\trI v} \leq \CTrm \NHmD{v}.
\eeqs
\enth


\bpf[Proof of \cref{lem:domainshift}]
By \cref{thm:shift,thm:trace}
\beqs
\NHlptD{\utilde} \leq \CAl \mleft(\NHlD{\ftilde} + \NHlphGI{\gtilde}\mright) \leq \CAl \mleft(\NHlD{\ftilde} + \CTrfn{l}\NHlpoD{\gtilde}\mright),
\eeqs
and the result follows.
\epf

We are now able to prove the following \lcnamecref{thm:expansion} giving a decomposition of the solution $u$ of \cref{prob:vtedp} into lower-order, less oscillatory parts and a smoother, more-oscillatory part. 

\bth[Expansion of the solution of the Helmholtz equation]\label{thm:expansion}
Under \cref{ass:highp,ass:htwo} there exists $\uosc \in \HppoD$ and a sequence $\usj \in \HjptD,$ $j = 0,\ldots,p-2$ such that if $u$ is the solution of \cref{prob:tedp} or its adjoint, then
\beq\label{eq:expansionid}
u = \uosc + \sum_{j=0}^{p-2} \usj.
\eeq
Furthermore,
\beq\label{eq:expansionuj}
\NHjptD{\usj} \leq \Cej \Pj(\NLiD{n}) k^j \Cfg,
\eeq
and
\beq\label{eq:expansionuosc}
\NHppoD{\uosc} \leq \Cosc \CAnk k^p \Cfg,
\eeq
where
\beq\label{eq:p}
\Pj(x) =
\begin{dcases}
1 & j = 0,1\\
x^{\floor{j/2}}& j \geq 2
\end{dcases}
\eeq.
%=\sum_{m=0}^{j-2}\pfn{j,m}x^m $
%% are polynomials of degree
%% \beq\label{eq:polydegree}
%% \begin{dcases}
%% \frac{j}2 & \tif  j \text{ is even}\\
%% \frac{j-1}2 \tif j  \text{ is odd}.
%% \end{dcases}
%% \eeq
%% (except for $j=0,1,$ where $\Pj$ is a polynomial of degree 0) given by the recurrence relation
%% \beq\label{eq:pjdef}
%% \Pj(x) = \CAj \mleft(1+\CTrjpo\mright)\mleft(x\Pfn{j-2}(x) + \Pfn{j-1}(x)\mright).
%% \eeq
% No idea if this coefficient stuf is right
%% \begin{align}
%% \label{eq:p1}\pfn{0,0}&= \CAz\\
%% \label{eq:p2}\pfn{1,0}&  = \CAz\CAo\mleft(1+\CTrt\mright)\\
%% \label{eq:p3}\pfn{j,m} &=
%% \begin{dcases}
%%  0, &\tif m \leq \floor{\frac{j-2}2}\\
%% \CAj\mleft(1+\CTrjpo\mright) \mleft(\pfn{j-2,m-1} + \pfn{j-1,m-1}\mright),&\tif \floor{\frac{j-2}2} < m \leq j-3
%% \end{dcases}\\
%% \label{eq:p4}\pfn{j,j-2}& = \CAj\mleft(1+\CTrjpo\mright) \pfn{j-1,j-3},
%% \end{align}
%% and $\Cosc$ is given by the recurrence relations
%% \begin{align}
%% \label{eq:osc1}\Coscfn{1} &= \CAo\mleft(1+\CTrt\mright),\\
%% \label{eq:osc2}\Coscfn{2} &= \CAt\mleft(1+\CTrth\mright)\mleft(1 + \CAo\mleft(1+\CTrt\mright)\mright)\tand\\
%% \label{eq:osc3}\Coscfn{j} &= \CAfn{j}\mleft(1+\CTrfn{j+1}\mright)\mleft(\Coscfn{j-1} + \Coscfn{j-2}\mright),
%% \end{align}
%% and
%% \beq\label{eq:cosc}
%% \Cosc = \frac{\Cefn{p-1}}{\max\set{1,\CAz}}.
%% \eeq
\enth

\bre[How oscillatory are the functions in \cref{thm:expansion}?]
Recall that a higher power of $k$ appearing in an a priori bound indicates that a function is more oscillatory. In \cref{eq:expansionuj} below, the $(j+2)$th-order norm of $\usj$ is of order $k^j$ (i.e., the power of $k$ is two orders of magnitude less than the order of the norm) whereas in \cref{eq:expansionuosc} the $(p+1)$st-order norm of $\uosc$ is of order $k^p$ (the power of $k$ is one order of mangitude less that the order of the norm). Therefore, in this sense, $\uosc$ is `more oscillatory' than $\usj.$

\Cref{thm:expansion} is essentially just \cite[Theorem 1]{ChNi:18a} in the particular case of a Helmholtz problem, but with the dependence on all the constants kept track of. The results in \cite{ChNi:18a} are stated for a much wider class of problems, but the dependence on all of the constants is not made explicit.

We also point out that \cref{thm:expansion} is reminiscent of results by Melenk and Sauter in \cite{MeSa:10,MeSa:11}who show, for the homogeneous Helmholtz equation that the solution $u$ can be decomposed as $u = \uHt + \uA,$ where $\uHt \in \HtD$ but is not oscillatory ($\NHtD{\uHt} \lesssim 1$), and $\uA \in C^\infty(D),$ but $\uA$ is oscillatory ($\NHmD{\uA} \lesssim k^{s-2}$ for all $s$)\footnote{These results are proved with no obstacle and $f$ given by a Dirac delta function in \cite[Lemma 3.5]{MeSa:10} and for: (i) the IIP with a bounded Lipschitz boundary that is either a 2-D polygon or analytic, or (ii) the EDP with an analytic scatterer, in \cite[Theorems 4.10, 4.20]{MeSa:11} respectively, under an assumption of a polynomial growth of the a priori bound---i.e., $\CAnk$ is a polynomial in $k$.}. This decomposition is used in \cite{MeSa:10,MeSa:11} to prove convergence results for $hp$-finite element methods for the Helmholtz equation. %We note that the results in this \lcnamecref{sec:fem} could, in principle, be used in the analysis of $hp$-methods, as it is, in principle, possible to track the dependence of the constants on the polynomial degree $p.$
\ere

\bpf[Proof of \cref{thm:expansion}]
The idea of the proof is as follows. We write $u$ as a formal series expansion
\beq\label{eq:formalseries}
u = \sum_{j=0}^\infty \usj,
\eeq
and then substitute this series into the PDE \cref{eq:tedp} and the boundary condition \eqref{eq:ibc}. Equating powers of $k$, we derive a recursive sequence of stationary diffusion equations for the functions $\usj,$ with right-hand sides dependent on $\ujmo$ and $\ujmt$. We use this recursive sequence and \cref{lem:domainshift} to prove the a priori bounds \cref{eq:expansionuj}. We then define the $l$th remainder $\rl = u - \sum_{j=0}^{l-1} \usj,$ and by applying the operator $\grad\cdot\mleft(A\grad \cdot\mright)$ with Neumann boundary conditions to $\rl$, we obtain a recursive sequence for the remainders $\rl,$ and can similarly prove a priori bounds for the $\rl$s. The oscillatory function $\uosc$ is then just $\rpmo.$ The format of this proof is identical to that in \cite[Theorem 1]{ChNi:18a}, except we keep track of all of the constants involved.

For the purposes of the proof, it is more convenient to define $\vj = \usj/k^j,$ so that the series expansion \cref{eq:formalseries} becomes
\beq\label{eq:formalseriesv}
u = \sum_{j=0}^\infty k^j\vj
\eeq
as in \cite{ChNi:18a}. Also, in this proof, all the boundary-value problems involved included a zero Dirichlet condition on the scatterer $\GD;$ we omit this throughout the proof for brevity.

By applying the Helmholtz operator to the formal series \eqref{eq:formalseriesv} we obtain the following problems for $\vj \in \HjptD, j \geq 1$:
\beqs
\grad \cdot \mleft(A\grad \vz\mright) = -f \quad\tand\quad \dn \vz = \gI,
\eeqs
\beqs
\grad \cdot \mleft(A\grad \vo\mright) = 0\quad\tand\quad\dn \vo = i\vz,
\eeqs
and, for $j \in \mleft[2,p-2\mright]$
\beq\label{eq:vj}
\grad \cdot \mleft(A\grad \vj\mright) = - n\vjmt\quad\tand\quad\dn \vz = i\vjmo.
\eeq

By \cref{thm:shift} we immediately conclude the bound
\beq\label{eq:expuz}
\NHtD{\vz} \leq \CAz\Cfg \leq \Cefn{0}\Cfg,
\eeq
and by \cref{lem:domainshift,eq:expuz} we can conclude the bound
\beqs
\NHthD{\vo} \leq \CAo \mleft(1+\CTrt\mright)\NHtD{i\vz} \leq \max\set{1,\CAo}\mleft(1+\CTrfn{2}\mright)\Cefn{0} \Cfg = \Cefn{1}\Cfg.
\eeqs

We prove the bound \cref{eq:expansionuj} by induction; suppose \cref{eq:expansionuj} holds for all $s \in [0,j-1].$ Using \cref{lem:domainshift}, we conclude that (using the observation that $\Cefn{j-1} \geq \Cefn{j-2}$)
\begin{align*}
\NHfn{j+2}{D}{\vj} &\leq \CAj \mleft(1+\CTrjpo\mright)\mleft(\NLiD{n} \NHfn{j}{D}{\vfn{j-2}} + \NHfn{j+1}{D}{\vfn{j-1}}\mright)\\
&\leq \max\set{1,\CAj} \mleft(1+\CTrjpo\mright)\mleft(\NLiD{n} \Cefn{j-2} \Pfn{j-2}\mleft(\NLiD{n}\mright) + \Cefn{j-1} \Pfn{j-1}\mleft(\NLiD{n}\mright)\mright)\Cfg\\
&\leq 2\max\set{1,\CAj} \mleft(1+\CTrjpo\mright)\Cefn{j-1} \Pfn{j}(\NLiD{n})\Cfg\\
&= \Cefn{j} \Pfn{j}(\NLiD{n})\Cfg.
\end{align*}
%% Therefore the definition of the polynomials $\Pj$ \cref{eq:pjdef} holds (and it is straightforward to see that $\deg\mleft(\Pj\mright) = \deg\mleft(\Pfn{j-2}\mright)+1$, where $\deg$ denotes polynomial degree, and therefore since $\deg\mleft(\Pfn{0}\mright) = \deg\mleft(\Pfn{1}\mright)=0,$ \cref{eq:polydegree} holds. Hence, by the relationship between $\usj$ and $\vj$, we have the bound \cref{eq:expansionuj}.

We will now define the remainders $\rl$, and proceed similarly.
Let $\ro \in \HthD$ solve
\beqs
\grad \cdot \mleft(A\grad \ro\mright) = -k^2 u
\eeqs
\beqs
\dn \ro = iku.
\eeqs
Then by \cref{lem:domainshift}
\beqs%gin{align*}
\NHthD{\ro} \leq \CAo\mleft(1+\CTrt\mright)\mleft(k^2\NHoD{u} + k\NHtD{u}\mright)\leq \CAo\mleft(1+\CTrt\mright)k^2\CAnk\Cfg\leq \frac{\Cefn{1}}{\max\set{1,\CAo}}.
\eeqs%nd{align*}
Let $\rt \in \HfD$ solve
\beqs
\grad \cdot \mleft(A\grad \rt\mright) = -k^2 u
\eeqs
\beqs
\dn \rt = ik\ro.
\eeqs
Then by \cref{lem:domainshift}
\begin{align*}
\NHfD{\rt} &\leq \CAt\mleft(1+\CTrth\mright)\mleft(k^2\NHtD{u} + k\NHthD{\ro}\mright)\\
&\leq \CAt\mleft(1+\CTrth\mright)\mleft(1 + \frac{\Cefn{1}}{\max\set{1,\CAo}}\mright)\CAnk k^3\Cfg\leq \frac{\Cefn{2}}{\max\set{1,\CAo}}.
\end{align*}
Then for $j \geq 3,$ let $\rj \in \HjptD$ solve
\beqs
\grad \cdot \mleft(A\grad \rt\mright) = -k^2 \rjmt
\eeqs
\beqs
\dn \rj = ik\rjmo.
\eeqs
And by induction and \cref{lem:domainshift} again, letting $\uosc = \rfn{p-1},$ we have \cref{eq:expansionuosc}. It is straightforward to see that $\rfn{p-1} + \sum_{j=1}^{p-2} \uj$ solves \cref{prob:tedp}, and therefore \cref{eq:expansionid} holds, as $u$ is unique.
\epf
%\{Just a note while I think about it - could you do DtN boundary conditions in this proof by testing with a different function when wanting to bound the $L^2$ norm on the boundary? I wonder if testing with the NtD of $u$ would mean you end up with a $\LtGI{u}^2$ term, and then you could use properties of the NtD map to bound other bits. Might foil $k$-dependency though.}
Using the expansion in \cref{thm:expansion}, we can prove the following error bound for the best approximation of $u$ in $\Vhp$:

%% We also have the following best-approximation error in higer-order Sobolev spaces:
%% \ble[Best approximation in $\HmD$]\label{lem:bestapproxhigh}
%% For integer $s$ in $[1,p+1]$, there exists $\Cinterps>0$ such that for every $v \in \HmD$ there exists $\vhhat \in \Vhp$ such that
%% \beqs
%% \NLtD{v - \vhhat} + h\NHoD{v-\vhhat} \leq \Cinterps h^s \SNHmD{v}.
%% \eeqs
%% \ele

\ble[Best approximation error bound]\label{lem:bestapprox}
Under \cref{ass:highp,ass:htwo}, there exist constants $\CFEMo, \CFEMt > 0$ independent of $k$ and $n$ (although dependent on $A$ and $p$) such that if $u$ solves \cref{prob:vtedp} or its adjoint, then there exists $\uhhat \in \Vhp$ such that
\beq\label{eq:bestapproxL2}
\NLtD{u-\uhhat} \leq \Pfn{p-2}\mleft(\NLiD{n}\mright)\mleft(\CFEMo  h^2 + \CFEMt \CAnk h\mleft(hk\mright)^p \mright)\Cfg,
\eeq
%% \beq\label{eq:bestapproxH1}
%% \NHoD{u-\uhhat} \leq \mleft(\CFEMo h + \CFEMt \CAnk \mleft(hk\mright)^p \mright)\Cfg \tand
%% \eeq
\beq\label{eq:bestapproxW}
\NW{u-\uhhat} \leq 2\Pfn{p-2}\mleft(\NLiD{n}\mright)\mleft(\CFEMo  h + \CFEMt \CAnk \mleft(hk\mright)^p \mright)\Cfg.
\eeq
\ele

%We write bounds for the standard and weighted $H^1$ norms separately, as we will use each individual bound in differents of proofs in \cref{sec:fembound}.

\bpf[Proof of \cref{lem:bestapprox}]
We apply \cref{lem:scottzhang} to all the $\usj$ and $\uosc$ in \cref{thm:expansion}, and obtain that there exist $\ujh,$ $j=0,\ldots,p-2$ and $\uosch$ in $\Vhp$ such that 
\beqs
\NLtD{\usj - \ujh} + h\NHoD{\usj - \ujh} \leq \Cinterpfn{j+2} \Cefn{j} \Pj\mleft(\NLiD{n}\mright) h^{j+2}k^j \Cfg
\eeqs
and
\beqs
\NLtD{\uosc - \uosch} + h\NHoD{\uosc - \uosch} \leq \Cinterpfn{p+1} \Cosc\CAnk h^{p+1}k^p \Cfg.
\eeqs
Therefore, by letting $\uhhat = \uosch + \sum_{j=0}^{p-2} \ujh,$ we have \cref{eq:bestapproxL2,eq:bestapproxW} (using the fact that $hk \leq 1$ and, as $\NLiD{n} \geq 1$, $\Pfn{p-2}(\NLiD{n}) \geq \Pfn{j}(\NLiD{n}) \geq 1$ for all $j \leq p-2$).
\epf

\subsection{Routine analysis results}\label{sec:anbackground}
In this \lcnamecref{sec:anbackground} we collect together routine results that we use throughout \cref{sec:fem}.

We first note the following inequality: for $N \in \NN$ and $a_1,\ldots,a_N > 0$
\beq\label{eq:simple}
\sqrt{\sum_{j=1}^N a_j^2} \leq \sum_{j=1}^N a_j.
\eeq



\bth[Multiplicative Trace Inequality]\label{thm:multiplicativetrace}%BS THm 1.6.6
There exists $\CMT > 0$ such that for all $v \in \HoD$
\beqs
\NLtdD{v} \leq \CMT \NLtD{v}^\half \NHoD{D}^\half.
\eeqs
\enth

\ble[Poincar\'e Inequality]\label{lem:poincare}
There exists $\CP > 0$ such that for all $v \in \HozDD$
\beqs
\NLtD{v} \leq \CP \SNHoD{v}.
\eeqs
\ele

\bth[Multiplication in $\HmD$]\label{thm:banachalg}
\ben
\item\label[itempart]{it:ban1}If $m > d/2,$ then for all $\vo, \vt \in \HmD$, $\vo\vt \in \HmD$ and there exists a constant $\CBanfn{m} > 0$ independent of $\vo$ and $\vt$ such that
\beqs
\NHmD{\vo\vt} \leq \CBanfn{m} \NHmD{\vo}\NHmD{\vt}.
\eeqs
\item\label[itempart]{it:ban2}For $m \in \NN,$ if $\vo \in \CmD$, $\supp\mleft(\vo-1\mright) \compcont D,$ and  $\vt \in \HmD,$ then $\vo \in \WmiD,$ $\vo\vt \in \HmD$, and there exists a constant $\Cprod{m} > 0$ independent of $\vo$ and $\vt$ such that%If $\vo \in \CrcompD$ for some $r \in \NN, r \geq m,$ and
\beq\label{eq:ban2}
\NHmD{\vo\vt} \leq \Cprod{m} \mleft(1+\NWmiD{\vo}\mright)\NHmD{\vt}.
\eeq
\een
\enth

\bpf[Proof of \cref{thm:banachalg}]
\Cref{it:ban1} is given in \cite[Section 1.8.1, Theorem]{Ma:11}. For \cref{it:ban2}, observe that as $\vo-1$ has compact support, $\vo \in \WmiD,$ and $\vo-1 \in \CmcompD$. By \cite[Theorem 3.20]{Mc:00}, there exists $\Cmclean{m} > 0$ such that\footnote{In \cite[Theorem 3.20]{Mc:00} $\Cmclean{m}$ is denoted $C_{m}$.}
\beqs
\NHmD{(\vo-1)\vt} \leq \Cmclean{m} \NWmiD{\vo-1}\NHmD{\vt}.
\eeqs
Therefore as $\vo = \vo -1 + 1,$ we have \cref{eq:ban2}.
\epf



\subsection{Error bounds for simple Galerkin projections}\label{sec:errgalerkin}
In this \lcnamecref{sec:errgalerkin} we state a sequence of error bounds in negative Sobolev norms for three different projection operators. The proofs of these error bounds are all simple modifications of the standard proofs of finite-element errors in negative Sobolev norms, as in, e.g., \cite[Theorem 5.8.3]{BrSc:08},  and so we only prove \cref{lem:wltdprojerr,lem:ellprojerrw} below, as these are the proofs that deviate most from the standard proof.
%\ednote{Both-should I write out at least one of these modified proofs?}. % Note to self, I have these proofs sketched out, but just use the argument from Brenner and Scott and use an L^2 inner product on the function, not their gradients, for the elliptic projection.
We first define the projections we use.

Define the elliptic projection $\Ph:\HozDD\rightarrow\Vhp$ by, for $w \in \HozDD$
\beqs
\IPLtD{A\grad\vh}{\grad\Ph w} = \IPLtD{A\grad\vh}{\grad w} \tforall \vh \in \Vhp.
\eeqs
%% and define the weighted elliptic projection $\Phn:\HozDD\rightarrow\Vhp$ by, for $w \in \HozDD$
%% \beqs
%% \IPLtD{A\grad\vh}{\grad\Phn w} = \IPLtDn{A\grad\vh}{\grad w} \tforall \vh \in \Vhp.
%% \eeqs

Observe that $\Ph w$ is the finite-element solution of a stationary diffusion problem with `diffusion coefficient' $A$ and a right-hand side in $\HozDD'.$

We define the $\LtD$-projection $\Qh:\HozDD\rightarrow \Vhp$ by, for $w \in \HozDD$
\beqs
\IPLtD{\Qh w}{\vh} = \IPLtD{w}{\vh} \tforall \vh \in \Vhp.
\eeqs

We also need to define the $\LtD$ projection in a norm weighted by $n$

For $v,w \in \LtD,$ define the $n$-weighted inner product
\beqs
\IPLtDn{v}{w} = \int_{D} n v \wbar,
\eeqs
and the corresponding $n$-weighted $\LtD$ norm $\NLtDn{v} = \sqrt{\int_{D} n \abs{v}^2}$.

For $s \in \NN$, we define the $n$-weighted $\HmD$ norms $\NHmDn{v}^2 = \sum_{\alpha \st \abs{\alpha} \leq s}\NLtDn{D^\alpha v}$, and the negative weighted Sobolev norms by
\beq\label{eq:negweightnorm}
\NHnfn{-m}{D}{v} = \sup_{w \in \HmD} \frac{\IPLtDn{v}{w}}{\NHmDn{w}}.
\eeq
Observe that, for $v \in \HmD,$
\beq\label{eq:nconv}
\nmin\NHmD{v} \leq \NHmDn{v} \leq \NLiD{n} \NHmD{v}.
\eeq
%We put the \emph{non-weighted} $H^s$-norm in the denominator of \cref{eq:negweightnorm}, but the \emph{weighted} inner product in the numerator for ease of manipulation in some of the following proofs---we could place the weighted $H^s$ norm in the denominator, and this would be equivalent to the current definition, up to a factor involving $n$.
%% so that we have, for all $v \in \LtD$
%% \beq\label{eq:neqweightid}
%% \NHmsD{n v} = \NHmsDn{v}.
%% \eeq
%% \beq\label{eq:negativeweightbounds}
%% \frac{\NHmsD{v}}{\NLiD{n}} \leq \NHmsDn{v} \leq \frac{NHmsD{v}}{\nmin}.
%% \eeq

Define the $\LtD$ projection in the $n$-weighted norm $\Qhn:\HozDD\rightarrow \Vhp$ by, for $w \in \HozDD$
\beqs
\IPLtDn{\Qhn w}{\vh} = \IPLtDn{w}{\vh} \tforall \vh \in \Vhp.
\eeqs

We now state and prove our error bounds. These can all be obtained by modifications of the proof in \cite[Theorem 5.8.3]{BrSc:08} because all the projections defined above are Galerkin projections given by coercive and bounded sesquilinear forms on $\HoD$ (for $\Ph$) or $\LtD$ (for $\Qh$ and $\Qhn$).\optodo{Check if $\Qh$ is used.}

The elliptic projection obeys the following error bounds:
\ble[Error bounds for elliptic projection]\label{lem:ellprojerr}
Under \cref{ass:highp}, for any integer $m \in [-1,p-1],$ there exists a constant $\Cmmo >0$ such that for all $w \in \HozDD$
\beq\label{eq:ellprojerr}
\NHmmD{w-\Ph w} \leq \Cmmo h^{s+1} \BAHoD{w}{\wh}.
\eeq
\ele
\optodo{Correct language to talk about $n$-weighted}


\ble[Error bounds for elliptic projection in $n$-weighted norms]\label{lem:ellprojerrw}
Let $m \in \NNz.$ Under \cref{ass:highp}, there exists a constant $\Cwmm >0$ such that for all $w \in \HozDD$
\beqs
\NHnfn{-m}{D}{w-\Ph w} \leq \Cwmm \errn{m} h^{m+1} \BAHoDn{w}{\wh},
\eeqs
where
\beq\label{eq:errn}
\errn{m} =
\begin{dcases}
\frac{\NHmD{n}}{\nmin^2}\nvar &\tif m \in \mleft(\frac{d}2,p-1\mright)\\
\frac{\NWsiD{n}}{\nmin^2}\nvar &\tif m \in \mleft[1,\frac{d}2\mright]\\
\nvar &\tif m = -1,0.
\end{dcases}
\eeq
We use the notation
\beq\label{eq:en}
\En=\max\set{\errn{m}\st m = -1,\ldots,p-1}.
\eeq
%% \ben
%% \item\label[itempart]{it:wep1} If $s \in (d/2,p-1],$ then
%% \beq\label{eq:wep1}%\label{eq:ellprojerr}
%% \NHnfn{-s}{D}{w-\Ph w} \leq \Cwmso \frac{\NHmD{n}}{\nmin^2}\nvar h^{s+1} \BAHoDn{w}{\wh}.
%% \eeq
%% \item\label[itempart]{it:wep2} If $s \in [2,d/2]$ then
%% \beq\label{eq:wep2}
%% \NHnfn{-s}{D}{w-\Ph w} \leq \Cwmso \frac{\NWsiD{n}}{\nmin^2}\nvar h^{s+1} \BAHoDn{w}{\wh}.
%% \eeq
%% \item\label[itempart]{it:wep4} For $s = 0$
%% \beq\label{eq:wep4}
%% \NLtDn{w-\Ph w} \leq \Cwz \nvar h \BAHoDn{w}{\wh}.
%% \eeq
%% \item\label[itempart]{it:wep3}Also, we have
%% \beq\label{eq:wep3}
%% \NHnfn{1}{D}{w-\Ph w} \leq \Cwo\nvar \BAHoDn{w}{\wh}.
%% \eeq
%% \een
\ele

\bpf[Proof of \cref{lem:ellprojerrw}]
For the case $m \geq 1$, let $\phi \in \HmD,$ and observe that by \cref{thm:banachalg} $n\phi\in \HmD$ also. Let $ \vtilde = \sdsol(\phi),$ observe\ednote{$\Sn$ is defined in \cref{sec:discsob}---I haven't yet decided where to put the definition.} that by \cref{ass:highp,thm:shift} $\vtilde \in \Hfn{}{s+2}{D}$. For all $v \in \HozDD,$ we have
\beqs
\IPLtD{A \grad \vtilde}{\grad v} = \IPLtD{n\phi}{v} = \IPLtDn{\phi}{v}.
\eeqs
If we take $v = w-\Ph w,$ then we can compute
\begin{align}
\IPLtDn{\phi}{v} &= \IPLtD{A\grad\mleft(\vtilde - \vh\mright)}{\grad\mleft(w-\Ph w\mright)} \text{ for } \vh \text{ the quasi-interpolant of } v\text{, by Galerkin orthogonality for } \Ph\nonumber\\
&\leq \CSZfn{s+2}\NLiDop{A} \NHfn{s+2}{D}{\vtilde} h^{s+1} \SNHoD{w-\Ph w} \text{ by \cref{lem:scottzhang}}\nonumber\\
&\leq \CSZfn{s+2} \CAfn{s} \NLiDop{A} \NHfn{s}{D}{n\phi} h^{s+1}\SNHoD{w-\Ph w}\text{ by \cref{thm:shift}}\label{eq:ellprojwpart}
\end{align}

For $m > d/2,$ and so by \cref{thm:banachalg,eq:nconv}, \cref{eq:ellprojwpart} is bounded above by
\beq\label{eq:wep1pt0}
\CSZfn{s+2} \CAfn{s} \CBanfn{s} \NLiDop{A} \frac{\NHfn{s}{D}{n}}{\nmin}\NHnfn{s}{D}{\phi} h^{s+1}\SNHoD{w-\Ph w}.
\eeq


Therefore we have, by definition of $\NHnfn{-s}{D}{\cdot}$ and $\NHoDn{\cdot}$,
\beq\label{eq:wep1pt1}
\NHnfn{-s}{D}{w-\Ph w} \leq \CSZfn{s+2}\CAfn{s} \CBanfn{s} \NLiDop{A} \frac{\NHmD{n}}{\nmin^2} h^{s+1} \NHoDn{w - \Ph w}.
\eeq
Applying C\'ea's Lemma to $\Ph$ in the $n$-weighted $H^1$ norm, we find
\beq\label{eq:wepcea}
\NHoDn{w-\Ph w} \leq \frac{2\NLiDop{A}}{\min\set{1,1/\CP^2}\Amin}\nvar\BAHoDn{w}{\wh},
\eeq
as $\Ph$ corresponds to a sesquilinear form that is bounded with constant $\NLiDop{A}/\nmin$ and coercive with constant $\mleft(\Amin\min\set{1,1/\CP}\mright)/\mleft(2\NLiD{n}\mright)$ in the $n$-weighted $H^1$ norm.
and then combining \cref{eq:wep1pt1,eq:wepcea}, we obtain \cref{eq:errn} in the case $m > d/2$.

For $s \in [1,d/2],$ the application of \cref{thm:banachalg} yields, instead of \cref{eq:wep1pt0},
\beq\label{eq:wep2pt1}
2\CSZfn{s+2} \CAfn{s} \Cprod{s} \NLiDop{A} \frac{\NWsiD{n}}{\nmin}\NHnfn{s}{D}{\phi} h^{s+1}\SNHoD{w-\Ph w},
\eeq
(since $1 + \NWsiD{n} \leq 2\NWsiD{n},$ as $\NLiD{n} \geq 1$) and by a similar reasoning to that before, we obtain \cref{eq:errn}, in the case $m \in [1,d/2]$.

For the case $m=0,$, using \cref{lem:ellprojerr} and then converting to the $n$-weighted $L^2$ norm yields \cref{eq:errn}. For $m=-1,$, \cref{eq:wepcea} immediately gives \cref{eq:errn}. 
\epf

%% We also give the following error bounds for the weighted elliptic projection, and give the proof, even though it is conceptually similar to the proof of \cref{lem:ellprojw}.\{Check the original one is still needed.}

%% \ble[Error bounds for weighted elliptic projection in $n$-weighted norms]\label{lem:wellprojerrw}
%% Let $s \in \ZZ.$ There exists a constant $\Cwmso >0$ such that for all $w \in \HozDD$
%% \ben
%% \item\label[itempart]{it:wwep1} If $s \in (d/2,p-1],$ and $n \in \HmD$ then
%% \beq\label{eq:wwep1}%\label{eq:ellprojerr}
%% \NHnfn{-s}{D}{w-\Phn w} \leq \Cwmso \frac{\NHmD{n}\NLiD{n}^2}{\nmin^3}h^{s+1} \BAHoDn{w}{\wh}.
%% \eeq
%% \item\label[itempart]{it:wwep2} If $s \in [1,d/2]$ and $n \in \CsD$, then
%% \beq\label{eq:wwep2}
%% \NHnfn{-s}{D}{w-\Phn w} \leq \Cwmso \frac{\NWsiD{n}\NLiD{n}^2}{\nmin^3} h^{s+1} \BAHoDn{w}{\wh}.
%% \eeq
%% \item\label[itempart]{it:wwep4} For $s = 0$
%% \beq\label{eq:wwep4}
%% \NLtDn{w-\Phn w} \leq \Cwfn{0,1} \frac{\NLiD{n}^3}{\nmin^3}h \BAHoDn{w}{\wh}
%% \eeq
%% \item\label[itempart]{it:wwep3} For $s = 1$
%% \beq\label{eq:it:wwep3}
%% \NHnfn{1}{D}{w-\Phn w} \leq \Cwfn{1,1}\frac{\NLiD{n}^2}{\nmin^2} \BAHoDn{w}{\wh}.
%% \eeq
%% \een
%% \ele

%% \bpf[Proof of \cref{lem:wellprojerrw}]
%% For \cref{it:wep1,it:wep2}, the proof follows the proof in \cite[Theorem 5.8.3]{BrSc:08}. Let $\phi \in \HmD,$ and observe that by \cref{thm:banachalg} $n\phi\in \HmD$ also. Let $ \vtilde = \sdsol(\phi),$ observe that by \cref{thm:shift} $\vtilde \in \Hfn{}{s+2}{D}$. For all $v \in \HozDD,$ we have
%% \beqs
%% \IPLtD{A \grad \vtilde}{\grad v} = \IPLtD{n\phi}{v} = \IPLtDn{\phi}{v}.
%% \eeqs
%% If we take $v = w-\Phn w,$ then we can compute
%% \begin{align}
%% \IPLtDn{\phi}{v} &= \IPLtD{A\grad\mleft(\vtilde - \vh\mright)}{\grad\mleft(w-\Phn w\mright)} \text{ for } \vh \text{ as in \cref{lem:scottzhang}, by Galerkin orthogonality for } \Phn\nonumber\\
%% &\leq \CSZfn{s+2}\NLiDop{A} \NHfn{s+2}{D}{\vtilde} h^{s+1} \SNHoD{w-\Phn w} \text{ by \cref{lem:scottzhang}}\nonumber\\
%% &\leq \CSZfn{s+2} \CAfn{s} \NLiDop{A} \NHfn{s}{D}{n\phi} h^{s+1}\SNHoD{w-\Phn w}\text{ by \cref{thm:shift}}\label{eq:ellprojwpart}
%% \end{align}

%% For \cref{it:wep1}, $s > d/2,$ and so by \cref{thm:banachalg}, \cref{eq:ellprojwpart} is bounded above by
%% \beq\label{eq:wep1pt0}
%% \CSZfn{s+2} \CAfn{s} \CBanfn{s} \NLiDop{A} \frac{\NHfn{s}{D}{n}}{\nmin}\NHnfn{s}{D}{\phi} h^{s+1}\SNHoD{w-\Phn w}
%% \eeq
%% by \cref{thm:banachalg,eq:nconv}.

%% Therefore we have, by definition of $\NHnfn{-s}{D}{\cdot}$ and $\NHoDn{\cdot}$,
%% \beq\label{eq:wep1pt1}
%% \NHnfn{-s}{D}{w-\Phn w} \leq \CSZfn{s+2}\CAfn{s} \CBanfn{s} \NLiDop{A} \frac{\NHmD{n}}{\nmin^2} h^{s+1} \NHoDn{w - \Phn w}.
%% \eeq
%% Applying C\'ea's Lemma to $\Phn$ in the weighted $H^1$ norm, we find
%% \beq\label{eq:wepcea}
%% \NHoDn{w-\Phn w} \leq \frac{2\NLiDop{A}\NLiD{n}^2}{\min\set{1,1/\CP^2}\Amin\nmin^2}\BAHoDn{w}{\wh},
%% \eeq
%% and then combining \cref{eq:wep1pt1,eq:wepcea}, we obtain \cref{eq:wep1}.

%% For \cref{it:wep2}, the application of \cref{thm:banachalg} yields, instead of \cref{eq:wep1pt0},
%% \beq\label{eq:wep2pt1}
%% 2\CSZfn{s+2} \CAfn{s} \Cprod{s} \NLiDop{A} \frac{\NWsiD{n}}{\nmin}\NHnfn{s}{D}{\phi} h^{s+1}\SNHoD{w-\Phn w},
%% \eeq
%% since $1 + \NWsiD{n} \leq 2\NWsiD{n},$ as $\NLiD{n} \geq 1,$ and by a similar reasoning to that before, we obtain \cref{eq:wep2}.

%% For \cref{it:wep3}, \cref{eq:wepcea} immediately gives the result. For \cref{it:wep4}, performing a standard duality argument in \emph{non-weighted} norms (and then using \cref{eq:it:wep3}, yields \cref{eq:wep4}.
%% \epf


The $\LtD$ projection satisfies the following error bound.
\ble[Error bounds for $\LtD$ projection]\label{lem:ltdprojerr}
Under \cref{ass:highp}, for any integer $m \in [0,p-1],$ for all $w \in \LtD$
\beqs
\NHmmD{w-\Qh w} \leq \Cmmz h^{m} \BALtD{w}{\wh}.
\eeqs
\ele

Similarly, the weighted $\LtD$ projection satisfies the following error bound.
The $n$-weighted $\LtD$ projection satisfies the following error bound:
\ble[Error bounds for weighted $\LtD$ projection]\label{lem:wltdprojerr}
Under \cref{ass:highp}, for any integer $m \in [0,p-1],$ for all $w \in \HozDD$
\beq\label{eq:wltdprojerr}
\NHmmDn{w-\Qhn w} \leq \CSZfn{m} \frac{\NLiD{n}}{\nmin} h^{m} \BALtDn{w}{\wh}.
\eeq
\ele

\bpf[Proof of \cref{lem:wltdprojerr}]
Fix $\vtilde \in \HmD$, trivially $\vtilde$ solves the adjoint problem
\beqs
\IPLtDn{v}{\vtilde} = \IPLtDn{v}{\vtilde} \tforall v \in \LtD.
\eeqs
Then letting $v=w-\Qhn w$ and using Galerkin orthogonality, for $\vhptilde$ as in \cref{lem:scottzhang} we have
\beqs%gin{align*}
\IPLtDn{w-\Qhn w}{\vtilde} \leq \NLtDn{w-\Qhn w}\NLtDn{\vtilde-\vhptilde}\leq \CSZfn{m} \frac{\NLiD{n}}{\nmin}\NLtDn{w-\Qhn w} h^m \NHmDn{\vtilde}.
\eeqs%nd{align*}
Taking the supremum over $\vtilde,$ we have
\beqs
\NHmmDn{w-\Qhn w} \leq \CSZfn{m} \frac{\NLiD{n}}{\nmin} h^m \NLtDn{w-\Qhn w},
\eeqs
and hence by C\'ea's Lemma (as the inner product $\IPLtDn{\cdot}{\cdot}$ is clearly bounded and coercive in the weighted $L^2$-norm $\NLtDn{\cdot}$) the result follows.
\epf

\subsection{Discrete Sobolev spaces}\label{sec:discsob}
In order to perform our analysis for high-order FEM, we will need to measure higher-order norms of functions in the finite-element space $\Vhp.$ However, as these functions do not have higher-order weak derivatives, we must first develop some theory of so-called discrete Sobolev spaces; we follow the presentation in \cite{DuWu:15}, albeit working in the heterogeneous case, and with some changes of notation.

We let $\DeltaAI:\LtD\rightarrow\HtD$ denote the solution operator for the stationary diffusion equation: given $\ftilde \in \LtD$ find $\utilde \in \HtD$ such that
\beq\label{eq:sdeq}
\grad \cdot \mleft(A\grad \utilde\mright) = -n\ftilde \text{ in } D
\eeq
\beq\label{eq:sddbc}
\trD \utilde = 0
\eeq
\beq\label{eq:sdnbc}
\dn \utilde = 0.
\eeq
%% $A$-weighted Laplacian; that is $\DeltaA w = \grad\cdot\mleft(\grad w\mright),$ and give it domain $\DomainDeltaA = \set{w \in \HtD \st \trD w = 0, \dn w = 0};$
%% hence $\DeltaA:\DomainDeltaA \rightarrow \LtD.$ Observe that, for any $f \in \LtD$ there exists $\wf \in \DomainDeltaA$ such that $\DeltaA \wf = -f.$
Observe that $\sdsol$ is well-defined by \cref{lem:domainshift}. Also, observe that $\sdsol^{m}$ is defined\optodo{more here} for any $m \in \NN,$ as $\HtD \subseteq \LtD,$ and so one can place $\sdsol \ftilde$ on the right-hand side. 
Observe that for any $\ftilde \in \LtD$ and for any $v \in \HozDD,$ we have, by Green's identity,
\beqs
\int_D \mleft(A \grad \mleft(\DeltaAI\ftilde\mright)\mright)\cdot \grad \vb = \int_D n\ftilde \vb,
\eeqs
i.e.,
\beq\label{eq:deltaagreen}
\IPLtD{A\grad \mleft(\sdsol \ftilde\mright)}{\grad v} = \IPLtDn{\ftilde}{v}.
\eeq


\bde[Discrete derivative operator]
Define the \defn{$A$-weighted discrete second derivative operator} $\Deltah:\Vhp\rightarrow\Vhp$ for $\wh \in \Vhp$ by
\beq\label{eq:discderdef}
\IPLtDn{\Deltah \wh}{\vh} = \IPLtD{A \grad \wh}{\grad \vh} \tforall \vh \in \Vhp.
\eeq
\ede

\bre[Why the factor $n$ on the right-hand side of \cref{eq:sdeq}?]\label{rem:why}
The reason for defining $\Sn$ with a factor $n$ on the right-hand side of \cref{eq:sdeq} is somewhat buried in the proof of \cref{thm:fembound} and its associated lemmas. However, we give an overview of the reason here.

All the bounds in the proofs of \cref{lem:boundarybound,lem:higherbound,lem:ltthetahbound} are in weighted discrete norms, because we can only prove bounds on the weighted $L^2$ projection $\Qhn$ in weighted higher-order discrete norms (as in \cref{lem:wltdprojerr}), not in non-weighted norms. Because we only work in weighted norms, we need \cref{lem:intoip} below to hold in the weighted inner product, which necessitates that we use the weighted inner product on the left-hand side of \cref{eq:discderdef}. This final use of the weighted inner product means we must have the weighted inner product on the right-hand side of \cref{eq:deltaagreen} so that we can show (at the beginning of the proof of \cref{lem:negdiscsum}) that if $\wh \in \Vhp,$ $\zh = \Deltah^{-1} \wh$, and $z = \Sn \wh,$ then $\zh = \Ph z;$ we need this fact in the remainder of the proof of \cref{lem:negdiscsum}.

We do not, however, put a factor $n$ on the left-hand side of \cref{eq:sdeq}. If we did, when we apply \cref{thm:shift} to $\Sn \ftilde$ (as we do in \cref{lem:shiftnegativew}), the resulting bounds would not be fully explicit in $n$ (because in \cref{thm:shift} we do not know explicitly how the constants depend on the `diffusion coefficient'). Not putting a factor $n$ on the left-hand side of \cref{eq:sdeq} is the reason why there is a \emph{non-weighted} inner product on the right-hand side of \cref{eq:discderdef} (so that the beginning of the proof of \cref{lem:negdiscsum}, as mentioned above, works).
\ere

\ble[Discrete derivative operator is well-defined]\label{lem:ddwd}
For any $\wh \in \Vhp,$ $\Deltah \wh$ exists and is unique.
\ele

\bpf[Proof of \cref{lem:ddwd}]
If one chooses an orthonormal (in the $n$-weighted inner product) basis  $\phij$ for $\Vhp,$ writes $\Delta\wh = \sum_j \wj \phij$, and takes $\vh = \phij$ for each $j,$ then we see \cref{eq:discderdef} is equivalent to the linear system $\Imat \bw = \bb,$ where $\bb_{j} = \IPLtDn{A \grad \wh}{\grad \phij}.$ The solution of this linear system clearly exists and is unique.
%% We equip $\Vhp$ with the $H^1$-norm. Observe that $\Deltah \wh$ satisfies the variational problem: Find $\vhtilde  \in \Vhp$ such that $\add(\uh,\vh) = \Ldd(\vh)$ for all $\vh \in \Vhp,$ where $\add(\uh,\vh) = \IPLtD{\uh}{\vh}$ and $\Ldd(\vh) = \IPLtD{A \grad \wh}{\grad \vh}.$ Observe that $\Ldd$ is bounded in $\Vhp$, as $\Ldd(\vh) \leq \NLiDop{A} \SNHoD{\wh}\SNHoD{\vh} \leq \NLiDop{A} \SNHoD{\wh}\NHoD{\vh},$ and $\add$ is coercive on $\Vhp,$ as, for $\vh \in \Vhp$, $\add(\vh,\vh) = \NLtD{\vh}^2 \geq \CinvVhp^2 \NHoD{\vh}^2$ by the standard inverse estimate\optodo{Add in?}. Therefore, by the Lax--Milgram Theorem applied in $\Vhp$ (as $\Vhp$ is a finite-dimensional inner product space over a complete field, it is a Hilbert space), $\Deltah \wh$ exists and is unique.
\epf

%% By\optodo{McLean Thm 4.12 - double check and maybe write out in more detail---exactly what operator are we talking about, especially if we want weak derivatives?}, there exists a sequence of eigenfunctions $\phio,\phit,\ldots \in \HoD$ of $\DeltaA$\optodo{What exactly does McLean mean here, if they don't have second-order derivatives?} and corresponding eigenvalues $0 < \lambdao<\lambdat < \cdots \rightarrow\infty$ such that the eigenfunctions form a complete orthonormal system in $\LtD.$\optodo{Maybe define this}

Since $A$ is real and symmetric, it is self-adjoint. Hence it follows that $\Deltah$ is self-adjoint, as
\beqs
\IPLtDn{\Deltah \wh}{\vh} = \IPLtD{A \grad \wh}{\grad \vh} = \IPLtD{\grad \wh}{A\grad \vh} = \overline{\IPLtDn{\Deltah \vh}{\wh}} = \IPLtDn{\wh}{\Deltah \vh}.
\eeqs
Therefore $\Deltah$ is diagonalisable, i.e., there exists a set of eigenfunctions $\phioh,\ldots,\phidimVhph$  with corresponding real eigenvalues\ednote{Both - is it common enough knowledge that a symmetric matrix has real eigenvalues, that I don't need to reference this?} $\lambdaoh, \ldots, \lambdadimVhph$ such that the $\phimh$ form an orthonormal (in the $n$-weighted inner product) basis of $\Vhp$.

\bde[Higher-order discrete derivative operators]
For $\vh \in \Vhp$, if $\vh = \sum_{m=1}^{\dimVhp} \am \phimh,$ then for $j \in \RR$ define
\beqs
\Deltah^j \vh = \sum_{m=1}^{\dimVhp} \lambdamh^j \am \phimh.
\eeqs


\ede
One can think of $\DeltahI$ as being a `discrete solution operator', i.e., a discrete counterpart to $\sdsol.$
%% Similarly, for $v \in \LtD,$ if $v = \sum_{m=1}^\infty \am \phim,$ then for $j \in \RR$ define
%% \beq\label{eq:deltaaseries}
%% \DeltaA^j v = \sum_{m=1}^\infty \lambdam^j \am \phim,
%% \eeq
%% if this series exists in $\LtD.$
%We let $\Domain{\DeltaA^j}$ denote the subset of $\LtD$ on which $\DeltaA^j$ is defined.
%% \bre[Negative powers of $\DeltaA$]
%% Observe that for \emph{every} $j \leq 0$, $\DeltaA^j v$ is defined for \emph{any} $v \in \LtD$ (i.e., $\Domain{\DeltaA^{j}} = \LtD$ for $j \leq 0$). For $\lambdam \geq 1,$ $\lambdam^j < \lambdam$, and only finitely many $\lambdam$ are in the interval $(0,1)$; therefore the series \cref{eq:deltaaseries} can be decomposed as a finite sum (for $\lambdam < 1$) and a convergent series (for $\lambdam \geq 1$).
%% \ere
%% \bre[Consistent Notation]
%% Observe that the notation $\DeltaA^j$ is consistent, i.e., $\DeltaA^0 v = v$, for $j \in \NN,$ $\DeltaA^j$ is equal to the $j$-fold application of $\DeltaA,$ and $\DeltaA^{-1}$ is the inverse of $\DeltaA.$\optodo{Maybe just double-check this is watertight.}\optodo{This needs work - need $A$ to be smooth enough to define proper higher-order derivatives.}
%% \ere
We can use the higher-order derivative operators to define discrete higher-order norms:

\bde[Discrete higher-order norm]
For $\vh \in \Vhp$ and $m \in \RR$, if define
\beqs
\Nshn{\vh} = \NLtDn{\Deltah^{s/2} v}.
\eeqs
\ede

%% \bde[$A$-weighted higher-order continuous norm]
%% For $v \in \LtD$ and $m \in \RR$, if $\DeltaA^{m/2} v$ exists, define
%% \beqs
%% \NmA{v} = \NLtD{\DeltaA^{m/2} v}.
%% \eeqs
%% \ede

%% In order to prove a relationship between the $A$-weighted higher order norm and the standard $H^m$ norms, we first must prove the following \lcnamecref{lem:normrelationshiptech}\optodo{THIS NEEDS PROVING AND I DON'T KNOW HOW.}
%% \ble[Relationship between $\DeltaA^m$ and standard higher-order derivatives]\label{lem:normrelationshiptech}
%% For $m \in \NN,$ $\Domain{\DeltaA^{m/2}} \subseteq \HmD,$ and there exists a constant $\Cma > 0$ such that for all $v \in \Domain{\DeltaA^{m}}$
%% \beqs
%% \NHmD{v} \leq \Cma \NLtD{\DeltaA^{m/2}v}.
%% \eeqs\optodo{The latter bit will probably use something shift-theorem-like, but you need to be careful because powers of the differential operator aren't defined standardly.}
%% \ele


%% \ble[Relationship between $A$-weighted and standard higher-order continuous norms]\label{lem:normrelationship}
%% For all $m \in \NNz,$ for all $v \in \HmmD,$
%% \beqs
%% \NHmmD{v} \geq \frac1{\Cma} \NmmA{v}.
%% \eeqs
%% \ele

%% \bpf[Proof of \cref{lem:normrelationship}]
%% For $v \in \HmmD,$ we have
%% \begin{align*}
%% \NHmmD{v} &= \sup_{w \in \HmD} \frac{\IPLtD{v}{w}}{\NHmD{w}}\\
%% &\geq \frac1{\Cma} \sup_{w \in \Domain{\DeltaA^{m/2}}} \frac{\IPLtD{v}{w}}{\NmA{w}} \text{ by \cref{lem:normrelationshiptech}}\\
%% & = \frac1{\Cma} \sup_{w \in \Domain{\DeltaA^{m/2}}} \frac{\IPLtD{\DeltaA^{-m/2}v}{\DeltaA^{m/2}w}}{\NmA{w}} \text{ by \cref{lem:intoip}}\\
%% &= \frac1{\Cma} \NmmA{v} \text{ the supremum is achieved when } \IPLtD{\DeltaA^{-m/2}v,\DeltaA^{m/2}w} = \NLtD{\DeltaA^{-m/2}v}\NLtD{\DeltaA^{m/2}w}
%% \end{align*}
%% \epf

We will use the following \lcnamecref{lem:intoip} to bound the inner product of two discrete functions by their negative- and positive-higher-order discrete norms, or to transfer discrete derivatives from one argument of the inner product to the other.

\ble[Introduction of derivatives into inner product]\label{lem:intoip}
%% For $m \in \RR,$ $v \in \LtD$, and $w \in \LtD\cap\Domain{\DeltaA^{m/2}}$ we have
%% \beqs
%% \IPLtD{\DeltaAmmt v}{\DeltaAmt w} = \IPLtD{v}{w}.
%% \eeqs
%% Similarly, f
For $\vh, \wh \in \Vhp,$ and $s \in \RR$ we have
\beq\label{eq:feipsplit}
\IPLtDn{\Deltahmst \vh}{\Deltahst \wh} = \IPLtDn{\vh}{\wh}
\eeq
and
\beq\label{eq:feiptrans}
\IPLtDn{\Deltahst \vh}{\Deltahst \vh} =  \IPLtDn{\Deltah^s \vh}{\vh}.
\eeq
\ele
\bpf[Proof of \cref{lem:intoip}]
We only prove \cref{eq:feipsplit}, as the proof of \cref{eq:feiptrans} is analagous. Since $\vh,\wh \in \Vhp,$ there exist sequences $(\aj)_{j =1,\ldots,\dimVhp}$ and $(\bsl)_{l =1,\ldots,\dimVhp}$ such that $\vh = \sum_{j=1}^{\dimVhp} \aj\phij$ and $\wh = \sum_{l=1}^{\dimVhp} \bsl \phil.$ Then we have
\begin{align*}
\IPLtDn{\Deltahmst \vh}{\Deltahst \wh} &= \int_D n\mleft(\sum_{j=1}^{\dimVhp}\lambdaj^{-m/2} \aj\phij\mright)\overline{\mleft(\sum_{l=1}^{\dimVhp} \lambdal^{m/2}\bsl \phil\mright)}\\
&= \sum_{j,l=1}^{\dimVhp} \lambdaj^{-m/2} \lambdal^{m/2} \aj \bsl \int_D n \phij \philbar \text{ as the } \lambdaj \text{ are real}\\
& =\sum_{j}^{\dimVhp} \aj \bsj \int_D n\abs{\phij}^2 \text{ as the } \phij \text{ are orthonormal}\\
&= \IPLtDn{\vh}{\wh}
\end{align*}
by repeating the above process in reverse, without the factors $\lambdaj^{-m/2}$ and $\lambdal^{m/2}$.
\epf
The next \lcnamecref{cor:ipdiscbound} follows from \cref{lem:intoip} and the Cauchy--Schwarz inequality.

\bco[Inner product bounded by discrete norms]\label{cor:ipdiscbound}
If $\vh, \wh \in \Vhp,$ then for all $j \in \RR$
\beqs
\IPLtDn{\vh}{\wh} \leq \Njh{\vh}\Nmjh{\wh}.
\eeqs
\eco

We recall the standard inverse inequality for finite-element functions, so that we can prove an analagous inverse inequality for discrete norms.

\ble[Standard inverse inequality]\label{lem:inverseinequality}
There exists $\CinvVhp > 0$ such that for all $\vh \in \Vhp$
\beqs
\NHoD{\vh} \leq \CinvVhp h^{-1} \NLtD{\vh}.
\eeqs
\ele



\ble[Inverse inequality for discrete norms]\label{lem:inversediscrete}
For all $j \in \ZZ$, for all $\vh \in \Vhp$
\beqs
\Njhn{\vh} \leq \Chinv \frac{1}{\nmin} h^{-1} \Njmohn{\vh}.
\eeqs
\ele

\bpf[Proof of \cref{lem:inversediscrete}]
by the definition of $\Njhn{\cdot}$, and the fact that $\Deltah^{j/2} = \Deltah^{1/2}\Deltah^{(j-1)/2},$ it suffices to prove the result for $j=1$, as one can then perform induction on $j$. We have
\begin{align*}
\Nfn{1,h,n}{\vh}^2 &= \IPLtDn{\Deltahh \vh}{\Deltahh \vh}\\
&= \IPLtDn{\Deltah \vh}{\vh} \text{ by \cref{eq:feiptrans}}\\
&= \IPLtD{A \grad \vh}{\grad \vh} \text{ by definition of } \Deltah\\
&\leq \NLiDop{A} \CinvVhp h^{-2}\frac{1}{\nmin^2} \NLtDn{\vh}^2
\end{align*}
by the standard inverse estimate, and the result follows as $\Nzhn{\cdot} = \NLtDn{\cdot}$.
\epf


\ble[Relationship between standard and discrete $H^1$ norms]\label{lem:h1contdisc}
Let $\vh \in \Vhp$. Then
\beqs
\SNHoD{\vh} \leq \Amin^{-\half} \Nohn{\vh}.
\eeqs
\ele

\bpf[Proof of \cref{lem:h1contdisc}]
We have, using \cref{eq:feiptrans}, $\Nohn{\vh}^2 = \IPLtDn{\Deltahh \vh}{\Deltahh \vh} = \IPLtDn{\Deltah \vh}{\vh}= \IPLtD{A \grad \vh}{\grad \vh} \geq \Amin \NLtD{\grad \vh}^2$, and the result follows.
\epf

To prove \cref{lem:negdiscsum} below on the relationship between discrete and continuous negative Sobolev normns, we require the following \lcnamecref{lem:shiftnegativew} giving the shift theorem in negative weighted norms.

\ble[Shift theorem in negative weighted norms]\label{lem:shiftnegativew}
Under \cref{ass:highp}, let $\ftilde \in \LtD.$ For integer $m \in [0,p-1],$ we have
\beq\label{eq:shiftnegativew}
\NHnfn{-m}{D}{\sdsol\ftilde} \leq \Cshiftfn{m} \shiftn{m} \NHnfn{-(m+2)}{D}{\ftilde},
\eeq
where
\beqs
\shiftn{m} \de 
\begin{dcases}
 \NHmD{n}\nvar &\tif m > \frac{d}2\\
\NWmiD{n}\nvar  &\tif m \in \mleft[1,\frac{d}2\mright]\\
\NLiD{n}\nvar &\tif m = 0\\
\NLiD{n}^2 &\tif m = -1.
\end{dcases}
\eeqs
We use the notation $\Rn = \max\set{\shiftn{m}\st m = -1,\ldots,p-1}.$
\ele

\bpf[Proof of \cref{lem:shiftnegativew}]
We first observe that the operator $\sdsolo$ is self-adjoint on $\LtD$: Let $\DeltaA:\HtD\rightarrow\LtD$ denote the stationary diffusion operator $\grad\cdot\mleft(A\grad\cdot\mright).$ Then for any $\vo \in \LtD,$ $\DeltaA \circ \sdsolo \vo = \vo.$ Moreover, by Green's Theorem, $\DeltaA$ is self-adjoint on the set $\set{v \in \HtD \st v \text{ satisfies }\cref{eq:sddbc,eq:sdnbc}},$ this set is contained in the image of $\sdsolo.$ Therefore, for any $\vo,\vt \in \LtD,$ we have
\beqs
\IPLtD{\sdsolo \vo}{\vt} =\IPLtD{\sdsolo \vo}{\DeltaA \circ \sdsolo \vt} = \IPLtD{\DeltaA \circ \sdsolo \vo}{\sdsolo \vt} = \IPLtD{\vo}{\sdsolo \vt},
\eeqs
i.e., $\sdsolo$ is self-adjoint on $\LtD.$

Observe that by \cref{thm:banachalg}, if $v \in \HmD$ then $nv \in \HmD,$ and therefore by \cref{thm:shift} $\sdsolo(nv) \in \HmptD.$ With these facts in place we can compute
\begin{align*}
\NHnfn{-m}{D}{\sdsol \ftilde} &= \sup_{v \in \HmD} \frac{\IPLtD{\sdsol \ftilde}{nv}}{\NHmDn{v}}\\
&= \sup_{v \in \HmD} \frac{\IPLtD{\sdsolo\mleft(n \ftilde\mright)}{nv}}{\NHmDn{v}}\\
&= \sup_{v \in \HmD} \frac{\IPLtDn{\ftilde}{\sdsolo\mleft(n v\mright)}}{\NHmDn{v}}\tas \sdsolo \text{ is self-adjoint}\\
&\leq \sup_{v \in \HmD} \frac{\NHnfn{-(m+2)}{D}{\ftilde}\NHnfn{m+2}{D}{\sdsolo\mleft(nv\mright)} }{\NHmDn{v}}\\
&\leq \sup_{v \in \HmD}  \frac{\CAfn{m} \NLiD{n} \NHmD{nv} \NHnfn{-(m+2)}{D}{\ftilde}}{\NHmDn{v}}
%% & = \sup_{v \in \HmD} \frac{\IPLtD{ n\ftilde}{\sdsol(v)}}{\NHmD{v}}\text{ by definition of } \sdsol, \text{ and as } \sdsolo \text{ is self-adjoint}\\
%% &= \sup_{v \in \HmD} \frac{\IPLtDn{\ftilde}{\frac{\sdsol(nv)}n}}{\NHmD{v}}\\
%% &\leq \CBanfn{m+2}\sup_{v \in \HmD} \frac{\NHfn{-(m+2)}{D}{\ftilde}\NHfn{m+2}{D}{\sdsol(nv)}\NHfn{m+2}{D}{\frac1n}}{\NHmD{v}} \text{ by \cref{thm:banachalg}}\\
%% &\leq \CBanfn{m+2}\CAfn{m}\sup_{v \in \HmD} \frac{\NHfn{-(m+2)}{D}{\ftilde}\NHfn{m}{D}{nv}\NHfn{m+2}{D}{\frac1n}}{\NHmD{v}} \text{ by \cref{thm:shift}}\\
\end{align*}
and by applying \cref{thm:banachalg} to the term $\NHfn{m}{D}{nv}$ (or observing that $\NLtD{nv} \leq \NLiD{n}\NLtD{v}$, in the case $m=0$), the result follows, except for $m=-1.$

For $m=-1,$ we have, by the Lax--Milgram Theorem in non-weighted norms, $\NHoD{\Sn \ftilde} \leq \NHmoD{n\ftilde}/\Amin.$ One can straightforwardly show that $\NHnfn{1}{D}{\Sn \ftilde} \leq \NLiD{n} \NHoD{\Sn \ftilde}$ and $\NHmoD{n\ftilde} \leq \NLiD{n} \NHnfn{-1}{D}{\ftilde},$ and so the result follows.
\epf

%% \ble[Shift theorem in negative norms]\label{lem:shiftnegative}
%% For $m \in \NN,$ and $\ftilde \in \LtD,$ we have
%% \beqs
%% \NHmmD{\DeltaAI\ftilde} \leq \CAm \NHmmmtD{\ftilde}.
%% \eeqs
%% \ele

%% \bpf[Proof of \cref{lem:shiftnegative}]
%% Throughout the proof we let $\utilde$ denote $\DeltaAI\ftilde.$ As $\ftilde \in \LtD,$ it follows that $\utilde \in \HtD$ by \cref{thm:shift}, and so in particular $\utilde \in \HmmD.$ We will use the fact that $\sdsol$ is self-adjoint; this follows from the fact that the boundary-value problem \cref{eq:sdeq,eq:sddbc,eq:sdnbc} is self-adjoint\ednote{Both - do I need to show this? It's straightforward.}.% Note to self, have (Rv,w), where R is resolvent, for v in L^2, w in H^m. Have PR = Id (but not necessarily the other way round) so w = PRw. Then (Rv,w) = (Rv,PRw) = (PRv,Rw) (P self-adjoint) = (v,Rw). Denote differential operator (on the correct space) by P.
%% We can then compute
%% \begin{align*}
%% \NHmmD{\utilde} &=  \sup_{0 \neq \vtilde \in \HmD} \frac{\IPLtD{\ftilde}{\DeltaAI\vtilde}}{\NHmD{\vtilde}}\text{ as } \DeltaAI \text{ is self-adjoint}\\
%% &\leq \sup_{0 \neq \vtilde \in \HmD} \frac{\NHmmmtD{\ftilde}\CAm\NHmD{\vtilde}}{\NHmD{\vtilde}} \text{ by \cref{thm:shift}}\\
%% &= \CAm \NHmmmtD{\ftilde}
%% \end{align*}
%% as required. \epf

We can now prove the following \lcnamecref{lem:negdiscsum} on the relationship between the negative-order discrete norms and the negative-order continuous norms.
\ble[Relationship between discrete and continuous negative-order norms]\label{lem:negdiscsum}
Under \cref{ass:highp}, for any integer $j \in [0,p+1],$ there exists a constant $\Csumj > 0$ such that for all $\vh \in \Vhp,$
\beq\label{eq:negdiscsum}
\Nfn{-j,h,n}{\vh} \leq \Csumj\mleft(\En\nvar\mright)^{\floor{\frac{j}2}}\NLiD{n} \sum_{m=0}^j h^{m} \NHnfn{-(j-m)}{D}{\vh}.
\eeq
\ele

\bpf[Proof of \cref{lem:negdiscsum}]
Let $\wh \in \Vhp,$ and define $\zh = \DeltahI \wh,$ $z = \DeltaAI \wh$ (observe $z$ is well-defined as $\Vhp \subseteq \LtD$). Then, for all $\vh \in \Vhp$, we have

\begin{align*}
\IPLtD{A \grad z}{\grad \vh} = \IPLtD{A \grad \mleft(\sdsol \wh\mright)}{\grad \vh} = \IPLtDn{\wh}{\vh} \text{, and}\\
\IPLtD{A \grad \zh}{\grad \vh} = \IPLtDn{\Deltah \zh}{\vh} = \IPLtDn{\wh}{\vh},
\end{align*}
where the equalities in the first line follow from the definition of $z$ and \cref{eq:deltaagreen}, and the equalities in the second line follows from \cref{eq:discderdef} and the definition of $\zh.$  Therefore (using the fact that $A$ is symmetric), for all $\vh \in \Vhp,$ $\IPLtD{A\grad \vh}{\grad z} = \IPLtD{A\grad \vh}{\grad \zh},$ i.e., $\zh = \Ph z.$

We now have, for $m \in [-1,p-1]$
\begin{align}
\NHnfn{-m}{D}{\DeltahI \wh} &\leq \NHnfn{-m}{D}{z} + \NHnfn{-m}{D}{z-\zh}\nonumber\\
&\leq \NHnfn{-m}{D}{z} + \Cwfn{-m}  \CAfn{0} \CSZfn{2} \errn{m} \NLiD{n} h^{m+2} \NLtD{\wh}\nonumber\\
&\quad\quad\text{ by \cref{lem:ellprojerrw,lem:scottzhang,thm:shift}, as }\zh = \Ph z\nonumber\\
&= \Cshiftfn{m} \shiftn{m} \NHnfn{-(m+2)}{D}{\wh} + \Cm \errn{m} \frac{\NLiD{n}}{\nmin} h^{m+2} \NLtDn{\wh}\label{eq:sumforrecursion}
\end{align}
by \cref{lem:shiftnegativew}, where we use $\shiftn{m}$ to denote the terms depending on $n$ in \cref{eq:shiftnegativew}.
From \cref{eq:sumforrecursion}, we can conclude that, for $l \in \NN$ and $\vh \in \Vhp$, writing $\wh = \Deltahmlpo \vh$,
\beq\label{eq:lrecursion}
\NHnfn{-m}{D}{\Deltahml \vh} \leq \Cshiftfn{m}\shiftn{m} \NHnfn{-(m+2)}{D}{\Deltahmlpo \vh} + \Cm \errn{m} \frac{\NLiD{n}}{\nmin} h^{m+2} \NLtDn{\Deltahmlpo \vh}
\eeq
as $\Deltahml = \DeltahI \Deltahmlpo$. We now use \cref{eq:lrecursion} recursively to bound $\Nfn{j,h,n}{\vh}$.

If $j = 2l,$ then one can show inductively using \cref{eq:lrecursion} that for any integer $t \in [0,l]$ that
\beq\label{eq:evenrecursivesum}
\Nfn{-2l,h,n}{\vh} \leq \mleft(\En\nvar\mright)^t\sum_{m=0}^t \Efn{m,t} h^{2m} \NHDfn{-2(t-m)}{\vh} ,
\eeq
\optodo{Fix creflabelformat for equations in subscripts}
where we define the $\Efn{m,t}$ inductively by
\begin{align}
\label{eq:Emt1}\Efn{0,0} &=1\\
\label{eq:Emt2}\Efn{m,t} &= \Cshiftfn{2(t-1-m)} \Efn{m,t-1} \tfor 0 \leq m \leq t-1 \quad\tand\\
\label{eq:Emt3}\Efn{t,t} &= \sum_{m=0}^{t-1} \Csumrecfn{2(t-1-m)} \Efn{m,t-1}.
\end{align}
To see this recurrence, we prove the inductive step: suppose
\beqs
\Nfn{-2l,h,n}{\vh} \leq \mleft(\En\nvar\mright)^{t-1}\sum_{m=0}^{t-1} \Efn{m,t-1} h^{2m} \NHDfn{-2(t-1-m)}{\vh}.
\eeqs
Then using \cref{eq:lrecursion}, we have
\begin{align*}
\Nfn{-2l,h,n}{\vh} &\leq \mleft(\En\nvar\mright)^{t-1}\sum_{m=0}^{t-1} \Efn{m,t-1} h^{2m} \mleft(\Cshiftfn{2(t-1-m)}\shiftn{2(t-1-m)} \NHnfn{-(2(t-1-m)+2}{D}{\Deltah^{-l+t} \vh}\mright.\\
&\quad\quad\mleft.+ \Csumrecfn{2(t-1-m)} \errn{2(t-1-m)} \nvar h^{2(t-1-m)+2} \NLtDn{\Deltah^{-l+t} \vh}\mright),
\end{align*}
which upon rearranging, and using the fact that $\shiftn{2(t-1-m)} \leq \errn{2(t-1-m)}$ and $\nvar \geq 1,$ yields \cref{eq:evenrecursivesum}, with the recurrence \cref{eq:Emt1,eq:Emt2,eq:Emt3}.

If $j=2l+1,$ then we first reduce $\Nfn{-j,h}{\vh}$ to a point analagous to the even case, and then proceed as before. Let $\wh$ and $\zh$ be as at the beginning of the proof, and let $z$ solve the variational formulation\footnote{We use the variational formulation here, as we will need to bound the $H^1$-norm of $z$ by the $H^{-1}$-norm of $\wh$, which is immediate from the Lax--Milgram theorem.}  of \cref{eq:sdeq,eq:sddbc,eq:sdnbc} (with $\ftilde = \wh$). Observe that we still have $\zh = \Ph z,$ and

\beq\label{eq:LMHmo}
\NHfn{1}{D}{z} \leq \frac1{\Amin} \NHfn{-1}{D}{n\wh}
\eeq
by the Lax--Milgram Theorem. We have
\begin{align}
\NLtDn{\Deltah^{-1/2}\wh} &= \NLtDn{\Deltah^{1/2} \zh}\nonumber\\
&= \IPLtDn{\Deltah \zh}{\zh} \text{ by \cref{eq:feiptrans}}\\
&= \IPLtD{A \grad \zh}{\grad \zh}\nonumber\\
&= \NLiDop{A} \NHoD{\Ph z}\nonumber\text{ by the Cauchy--Schwartz inequality and the definition of } \zh\\
&\leq \NLiDop{A}\mleft(\NHoD{z} + \Cprojfn{1}\NHoD{0 - z}\mright) \text{ by \cref{lem:ellprojerr}}\nonumber\\
&\leq \frac{\mleft(1+\Cprojfn{1}\mright)\NLiDop{A}}{\Amin}  \NHmoD{n\wh}\text{ by \cref{eq:LMHmo}}\label{eq:deltahhalf}\\
&\leq \frac{\mleft(1+\Cprojfn{1}\mright)\NLiDop{A}}{\Amin}\NLiD{n}  \NHnfn{-1}{D}{\wh}\text{ as in the proof of \cref{lem:shiftnegativew}}\nonumber
\end{align}
We now return to
\begin{align*}
\Nfn{j,h,n}{\vh} &= \NLtDn{\Deltah^{-l-1/2} \vh} = \NLtDn{\Deltah^{-1/2} \Deltah^{-l} \vh}\\
&\leq \frac{\mleft(1+\Cprojfn{1}\mright)\NLiDop{A}}{\Amin}\NLiD{n}\NHnfn{-1}{D}{\Deltah^{-l}\vh}
\end{align*}
by \cref{eq:deltahhalf}.

Similarly to \cref{eq:evenrecursivesum}, one can use \cref{eq:lrecursion} recursively to show that, for any integer $t \in [0,l]$
\beq\label{eq:oddrecursive}
\NHnfn{-1}{D}{\Deltah^{-l}\vh} \leq \mleft(\En \nvar\mright)^t\mleft(\Etildefn{0,t} \NHnfn{-(2t+1)}{D}{\Deltah^{-l+t}\vh} + \sum_{m=0}^t \Etildefn{m,t} h^{2m+1}  \NHnfn{-2(t-m)}{D}{\Deltah^{-l+t}\vh}\mright),
\eeq
where  we define the $\Etildefn{m,t}$ inductively, for $t \in [0,l]$ by
\begin{align}
\label{eq:Etilde1}\Etildefn{0,0} &= 1\\
%\label{eq:Etilde2}\Etildefn{1,1} &= \Csumrecfn{1}\\
\label{eq:Etilde3}\Etildefn{0,t} &= \Etildefn{0,t-1}\Cshiftfn{2(t-1)+1)}\\
\label{eq:Etilde4}\Etildefn{m,t} &= \Etildefn{m,t-1}\Cshiftfn{2(t-1-m)}\tfor m = 1,\ldots,t-1\\
\label{eq:Etilde5}\Etildefn{t,t} &= \Csumrecfn{2(t-1)+1} + \sum_{m=0}^{t-1}\Etildefn{m,t-1}\Csumrecfn{2(t-1-m)}.
\end{align}
To show \cref{eq:oddrecursive,eq:Etilde1,eq:Etilde3,eq:Etilde4,eq:Etilde5}, observe that \cref{eq:Etilde1} is immediate; we now show \cref{eq:oddrecursive,eq:Etilde3,eq:Etilde4,eq:Etilde5} by induction: suppose %\cref{eq:oddrecursive} holds for $t-1,$ then
\beqs
\NHnfn{-1}{D}{\Deltah^{-l}\vh} \leq \mleft(\En \nvar\mright)^{t-1}\mleft(\Etildefn{0,t-1} \NHnfn{-(2(t-1)+1)}{D}{\Deltah^{-l+t-1}\vh} + \sum_{m=0}^{t-1} \Etildefn{m,t-1} h^{2m+1}  \NHnfn{-2(t-1-m)}{D}{\Deltah^{-l+t-1}\vh}\mright).
\eeqs
Then
\begin{align*}
\NHnfn{m}{D}{\Deltah^{-l}\vh} \leq& \mleft(\En \nvar\mright)^{t-1} \Etildefn{0,t-1} \mleft(\Cshiftfn{2(t-1)+1} \shiftn{2(t-1)+1}\NHnfn{-(2(t-1)+1)}{D}{\Deltah^{-l + t}\vh}\mright.\\
&\hphantom{\mleft(\En \nvar\mright)^{t-1} \Etildefn{0,t-1}}\quad\quad\mleft.+\Csumrecfn{2(t-1)t+1} \errn{2(t-1)+1}\nvar h^{2t+1} \NLtDn{\Deltah^{-l + t}\vh}\mright)\\
&\quad\quad+ \mleft(\En \nvar\mright)^{t-1}\sum_{m=1}^{t-1} \Etildefn{m,t-1} h^{2m+1} \mleft(\Cshiftfn{2(t-1-m)} \shiftn{2(t-1-m)}\NHnfn{-2(t-1-m)}{D}{\Deltah^{-l + t}\vh}\mright.\\
&\mleft.\hphantom{\quad\quad+ \mleft(\En \nvar\mright)^{t-1}\sum_{m=1}^{t-1} \Etildefn{m,t-1} h^{2m+1}}\quad\quad+ h^{2(t-m)} \Csumrecfn{2(t-1-m)}\errn{2(t-1-m)}\nvar \NLtDn{\Deltah^{-l + t}\vh}\mright)\\
%% &= \Etildefn{0,t-1} \CAfn{2t-1} \NHfn{-(2(t-1)+1)}{D}{\Deltah^{-l + t}\vh}\\
%% &\quad\quad+ \mleft(\Csumrecfn{2t-1}\sum_{m=1}^{t-1} \Etildefn{m,t-1}\CAfn{2(t-1-m)}\mright) h^{2t+1}\NLtD{\Deltah^{-l + t}\vh}\\
%% &\quad\quad+ \sum_{m=1}^{t-1} \Etildefn{m,t-1} h^{2m+1} \CAfn{2(t-1-m)}\NHfn{-2(t-m)}{D}{\Deltah^{-l + t}\vh},
\end{align*}
and rearranging, and using the facts that $\shiftn{m} \leq \errn{m}$ for all $m$ and $\nvar \geq 1,$ we obtain \cref{eq:oddrecursive} with the constants $\Etildefn{m,t}$ given by \cref{eq:Etilde1,eq:Etilde3,eq:Etilde4,eq:Etilde5}. Therefore, we conclude that if $j = 2l+1$
\begin{align}
\Nfn{-j,h,n}{\vh} \leq& \frac{\mleft(1+\Cprojfn{1}\mright)\NLiDop{A}}{\Amin}\NLiD{n}\mleft(\En \nvar\mright)^{l}\mleft(\Etildefn{0,l} \NHnfn{-(2l+1)}{D}{\vh}\mright.\nonumber\\
&\mleft.\hphantom{\frac{\mleft(1+\Cprojfn{1}\mright)\NLiDop{A}}{\Amin}\NLiD{n}\mleft(\En \nvar\mright)^{l}}
\quad\quad+ \sum_{m=0}^{l} \Etildefn{m,l} h^{2m+1}  \NHnfn{-2(l-m)}{D}{\vh}\mright)\label{eq:oddfinal}
\end{align}
and combinining \cref{eq:evenrecursivesum} (with $t=l$) and \cref{eq:oddfinal}, and letting $\Csumj$ be as in \cref{app:constants}, the result follows.%\optodo{Maybe put recursion in, but it's all in notes.}
\epf

\subsection{Main finite-element-error bound}\label{sec:fembound}

Having established the need results on discrete Sobolev spaces, we are now in a position to prove our main theorem, \cref{thm:fembound}, which we do via a series of lemmas.

Two quantities that are key in the proof are
\beqs
\rho \de u - \Ph u, \tand
\eeqs
\beqs
\thetah \de \Ph u - \uh.
\eeqs
The main idea of the proof is to decompose the error $u - \uh = \rho + \thetah,$ use bounds on the elliptic projection operator to bound $\rho,$ and then bound $\thetah$ (in higher-order discrete norms) in terms of $\rho.$

The following \lcnamecref{lem:simpleform} sets up the argument in the lemmas that follow.
\ble[Expression for $a(\thetah,\vh)$]\label{lem:simpleform}
For any $\vh \in \Vhp,$
\beq\label{eq:thetaform}
a(\thetah,\vh) = k^2\IPLtDn{\Qhn\rho}{\vh} + ik \IPLtGI{\rho}{\vh}.
\eeq
\ele

\bpf[Proof of \cref{lem:simpleform}]
Let $\vh \in \Vhp.$ Then $a(\thetah,\vh) = a(u-\uh,\vh) - a(\rho,\vh) = -a(\rho,\vh)$ by Galerkin orthogonality. By definition of $a,$ we have $-a(\rho,\vh) = -\IPLtD{A\grad\mleft(u-\Ph u\mright)}{\vh} + k^2 \IPLtD{n\rho}{\vh} + ik\IPLtGI{\rho}{\vh}.$ By Galerkin orthogonality for the elliptic projection, $\IPLtD{A\grad\mleft(u-\Ph u\mright)}{\vh} = 0$, and so by the definition of the $n$-weighted $L^2$ inner product, and the $n$-weighted $L^2$-projection $\Qhn,$ the result follows.
\epf

\ble[Bound on $\NLtGI{\thetah}$ by $\Npmoh{\thetah}$]\label{lem:boundarybound}
Under the assumptions of \cref{thm:fembound}, we have
\beq\label{eq:boundarybound}
\NLtGI{\thetah}^2 \leq \Cboundaryo \mleft(\En\nvar\mright)^{2\mleft(\floor{\frac{p-1}2}+1\mright)}\NLiD{n}^4 k^2 h^{2p-1} \Nfn{p-1,h,n}{\thetah}^2 + \Cboundaryt h \NW{\rho}^2,
\eeq
\ele

\bpf[Proof of \cref{lem:boundarybound}]
In \cref{eq:thetaform}, let $\vh = \thetah,$ and take the imaginary part to obtain
\beq\label{eq:boundboundarybelow}
-k \NLtGI{\thetah}^2 \leq \Im k^2 \IPLtDn{\Qhn \rho}{\thetah} + \Re k \IPLtGI{\rho}{\thetah},
\eeq
and therefore by \cref{cor:ipdiscbound}
\beq
\NLtGI{\thetah}^2 \leq  k \Nfn{1-p,h,n}{\Qhn \rho}\Nfn{p-1,h,n}{\thetah} + \NLtGI{\rho}\NLtGI{\thetah}.\label{eq:thetaboundarypart}
%% \nonumber\\
%% &\leq  k \NLiD{n}^2 \Nfn{1-p,h}{\Qhn \rho}\Nfn{p-1,h}{\thetah} + \NLtGI{\rho}\NLtGI{\thetah}
\eeq
We first bound the negative norm $\Nfn{1-p,h}{\Qhn\rho}$, to do this we use \cref{lem:negdiscsum}; however, we therefore need to estimate negative Sobolev norms of $\Qhn\rho$; for integers $m \in [0,p-1]$ we have (observing that $\Qhn \Ph u = \Ph u$ as $\Ph u \in \Vhp.$
\begin{align}
\NHnfn{-(p-1-m)}{D}{\Qhn\rho} &\leq \NHnfn{-(p-1-m)}{D}{\Qhn u - u} + \NHnfn{-(p-1-m)}{D}{u - \Ph u}\nonumber\\
&\leq \CSZfn{(p-1-m)} \nvar h^{(p-1-m)} \NLtDn{u-\Ph u} + \Cwfn{-(p-1-m)}\errn{(p-1-m)} h^{p-m} \NHoDn{u - \Ph u}\nonumber\\
&\quad\quad\text{ by \cref{lem:ellprojerrw,lem:wltdprojerr}, taking } \wh = \Ph u \text{ in \cref{lem:ellprojerrw,eq:wltdprojerr}}\nonumber\\
&\leq \mleft(\CSZfn{(p-1-m)} \Cwz + \Cwfn{-(p-1-m)}\mright)\En\nvar\NLiD{n} h^{p-m} \NW{\rho} \text{ by \cref{lem:ellprojerrw}}\label{eq:Qhnrhoneg}
\end{align}
By \cref{lem:negdiscsum,eq:Qhnrhoneg} we have
\begin{align}
\Nfn{1-p,h,n}{\Qhn \rho}&\leq \mleft(\En\nvar\mright)^{\floor{\frac{p-1}2}}\NLiD{n} \Csumfn{p-1} \sum_{m=0}^{p-1} h^{m} \NHfn{-(p-1-m)}{D}{\Qhn \rho} \nonumber\\
&\leq \Cmess \mleft(\En\nvar\mright)^{\floor{\frac{p-1}2}+1}\NLiD{n}^2 h^p \NW{\rho}.\label{eq:Qhnrhosum}
\end{align}
To deal with the second term on the right-hand side of \cref{eq:thetaboundarypart} we use \cref{thm:multiplicativetrace,lem:ellprojerr}, taking $\wh = \Ph u$ in \cref{eq:ellprojerr} and the fact that $\NHoD{\cdot} \leq \NW{\cdot}$ to obtain
\beq\label{eq:rhomtbound}
\NLtGI{\rho} \leq \CMT\NHoD{\rho}^{1/2}\NLtD{\rho}^{1/2} \leq \CMT \Cprojfn{0}^{\half} h^\half \NHoD{\rho}\leq \CMT \Cprojfn{0}^{\half} h^{\half} \NW{\rho}.
\eeq
Therefore by \cref{eq:rhomtbound} and Young's inequality, we obtain
\beq\label{eq:rhothetamt}
\NLtGI{\rho}\NLtGI{\thetah} \leq \half \CMT^2 \Cprojfn{0} h \NW{\rho}^2 + \half \NLtGI{\thetah}^2.
\eeq
By combining \cref{eq:thetaboundarypart,eq:Qhnrhosum,eq:rhothetamt} we have
\beq\label{eq:thetahboundnear}
\NLtGI{\thetah}^2 \leq k \Cmess \mleft(\En\nvar\mright)^{\floor{\frac{p-1}2}+1}\NLiD{n}^2 h^p \NW{\rho}\Nfn{p-1,h,n}{\thetah} + \half \CMT^2 \Cprojfn{0} h\NW{\rho}^2 + \half \NLtGI{\thetah}^2
\eeq
By using Young's inequality on the first term in \cref{eq:thetahboundnear}, and moving the $\NLtGI{\thetah}^2$ term onto the left-hand side, we obtain \cref{eq:boundarybound}.
\epf

\ble[Bound on higher-order discrete norms of $\thetah$ by $\NLtD{\thetah}$]\label{lem:higherbound}
Under the assumptions of \cref{thm:fembound}, for integer $m \in [1,p]$ there exist constants $\Chighmo,$ $\Chighmt > 0$ such that
\begin{align}
\Nfn{m,h,n}{\thetah} &\leq \Chighmo \mleft(\mleft(\En\nvar\mright)^{\half(\floor{\frac{p-1}2}+1)}\NLiD{n}\nmin^{-\frac{p}2}\mright)^mk^m \NLtD{\thetah}\nonumber\\
&\quad\quad+ \Chighmt \nvar^3\NLiD{n}\nmin^{1-m} h^{1-m} \NW{\rho}.\label{eq:chigh}
\end{align}
\ele
\bpf[Proof of \cref{lem:higherbound}]
By inserting the definitions of $a$ and $\Deltah$ in \cref{eq:thetaform} and rearranging, we have for any $\vh \in \Vhp$
\beqs
\IPLtDn{\Deltah \thetah}{\vh} = k^2 \IPLtDn{\thetah}{\vh} + k^2\IPLtDn{\Qhn \rho}{\vh} + ik \IPLtGI{\thetah}{\vh} + ik \IPLtGI{\rho}{\vh}.
\eeqs
Therefore, if we take $\vh = \Deltah^{m-1}\thetah$, by \cref{lem:intoip} we have
\beq\label{eq:deltahm}
\Nfn{m,h,n}{\thetah}^2 = k^2 \Nfn{m-1,h,n}{\thetah}^2 + k^2 \IPLtDn{\Deltah^{\frac{m-1}2} \Qhn \rho}{\Deltah^{\frac{m-1}2}\thetah} + ik\IPLtGI{\thetah}{\Deltah^{m-1} \thetah} + ik \IPLtGI{\rho}{\Deltah^{m-1} \thetah}.
\eeq
We now proceed to bound the two terms in \cref{eq:deltahm} defined on the truncation boundary $\GI.$ For the first term, we have
\begin{align}
\IPLtGI{\thetah}{\Deltah^{m-1}\thetah} &\leq \NLtGI{\thetah}\NLtGI{\Deltah^{m-1}\thetah}\nonumber\\
&\leq \CMT \CinvVhp^{1/2} \NLtGI{\thetah} h^{-\half} \NLtD{\Deltah^{m-1} \thetah}\nonumber\\
&\quad\quad\text{ by \cref{thm:multiplicativetrace,lem:inverseinequality}}\nonumber\\
&= \CMT \CinvVhp^{1/2} h^{-\half}\NLtGI{\thetah}\nmin^{-1}\Nfn{2m-2,h,n}{\thetah}\text{ by the definition of }\Nfn{2m-2,h,n}{\cdot}\nonumber\\
&\leq \CMT \CinvVhp^{1/2} \Chinv^{m-1} h^{-m+\half} \nmin^{-m} \NLtGI{\thetah}\Nfn{m-1,h,n}{\thetah} \text{ by \cref{lem:inversediscrete} applied } m-1 \text{ times}\label{eq:useitagain}\\
&\leq \CMT \CinvVhp^{1/2} \Chinv^{m-1} h^{-m+\half} \nmin^{-m}\nonumber\\
&\quad\quad\mleft(\Cboundaryo^{\half} \mleft(\En\nvar\mright)^{\floor{\frac{p-1}2}+1}\NLiD{n}^2 k h^{p-\half} \Nfn{p-1,h,n}{\thetah} + \Cboundaryt^{\half} h^{\half} \NW{\rho}\mright)\Nfn{m-1,h,n}{\thetah}\nonumber\\
&\quad\quad\text{ by \cref{lem:boundarybound} and \cref{eq:simple}}\nonumber\\
&\leq \mleft(\CBo \mleft(\En\nvar\mright)^{\floor{\frac{p-1}2}+1}\NLiD{n}^2\nmin^{-p}k \Nfn{m-1,h,n}{\thetah} + \CBt \nmin^{-m}h^{1-m} \NW{\rho}\mright)\Nfn{m-1,h,n}{\thetah}\label{eq:firstboundary}
\end{align}
by \cref{lem:inversediscrete} applied $p-m$ times.
To bound the second boundary term in \cref{eq:deltahm}, we have
\beq\label{eq:secondboundarytemp}
\IPLtGI{\rho}{\Deltah^{m-1} \thetah} \leq \CMT \CinvVhp^{1/2} \Chinv^{m-1}\nmin^{-m}h^{\half-m}\Nfn{m-1,h}{\thetah}\NLtGI{\rho}
\eeq
using the same reasoning as we used to obtain \cref{eq:useitagain} above. By \cref{thm:multiplicativetrace,lem:ellprojerr} (with $\wh = \Ph u$) we have
\beq\label{eq:secondboundarytemp2}
\NLtGI{\rho} \leq \CMT \Cprojfn{0}^{\half} h^{\half} \NW{\rho}.
\eeq

Inserting \cref{eq:secondboundarytemp2} into \cref{eq:secondboundarytemp} we obtain
\beq\label{eq:secondboundary}
\IPLtD{\rho}{\Deltah^{m-1}\thetah} \leq \CBth \nmin^{-m}h^{1-m} \NW{\rho} \Nfn{m-1,h}{\thetah}.
\eeq

Therefore, from \cref{eq:deltahm,eq:firstboundary,eq:secondboundary} and the Cauchy--Schwartz inequality, we have
\begin{align*}
\Nfn{m,h}{\thetah}^2 &\leq k^2 \Nfn{m-1,h}{\thetah}^2 + k^2 \Nfn{m-1,h,n}{\Qhn \rho}\Nfn{m-1,h,n}{\thetah}\nonumber\\
&\quad\quad+ k\mleft(\CBo \mleft(\En\nvar\mright)^{\floor{\frac{p-1}2}+1}\NLiD{n}^2\nmin^{-p}k \Nfn{m-1,h,n}{\thetah} + \CBt \nmin^{-m}h^{1-m} \NW{\rho}\mright)\Nfn{m-1,h,n}{\thetah}\nonumber\\
&\quad\quad+ k\CBth \nmin^{-m}h^{1-m} \NW{\rho} \Nfn{m-1,h}{\thetah}%\label{eq:thetahighnearlynearly}.
\end{align*}
Therefore using Young's inequality we have
\begin{align*}
\Nfn{m,h}{\thetah}^2 &\leq k^2\mleft(\frac32 + \CBo \mleft(\En\nvar\mright)^{\floor{\frac{p-1}2}+1}\NLiD{n}^2\nmin^{-p} + \half \CBt \nmin^{-m} + \half\CBth \nmin^{-m}\mright)\Nfn{m-1,h,n}{\thetah}^2\\
&\quad\quad+ \frac{k^2}2 \Nfn{m-1,h,n}{\Qhn \rho}^2 +  h^{2(1-m)} \NW{\rho}^2,
\end{align*}
and therefore by \cref{eq:simple}
\begin{align}
\Nfn{m,h}{\thetah} &\leq k\mleft(\sqrt{\frac32} + \CBo^{\half} \mleft(\En\nvar\mright)^{\half(\floor{\frac{p-1}2}+1)}\NLiD{n}\nmin^{-\frac{p}2} + \frac1{\sqrt{2}} \CBt^{\half} \nmin^{-\frac{m}2} + \frac1{\sqrt{2}}\CBth^{\half} \nmin^{-\frac{m}2}\mright)\Nfn{m-1,h,n}{\thetah}\nonumber\\
&\quad\quad+ \frac{k}{\sqrt{2}} \Nfn{m-1,h,n}{\Qhn \rho} +  h^{1-m} \NW{\rho},\label{eq:thetahighnearly}
\end{align}

We now proceed to bound $\Nfn{m-1,h,n}{\Qhn \rho}$: By \cref{lem:inversediscrete} and the fact that $\Qhn \Ph u = \Ph u$ (as $\Ph u \in \Vhp$), we have
\beq\label{eq:qhnmmo}
\Nfn{m-1,h,n}{\Qhn \rho} \leq \Chinv^{m-1} \nmin^{1-m}h^{1-m} \mleft(\NLtDn{\Qhn u - u} + \NLtDn{u-\Ph u}\mright).
\eeq

Using \cref{lem:wltdprojerr} we can bound the first of these terms by the second, $\NLtDn{\Qhn u - u} \leq \CSZfn{0} \nvar \NLtDn{u-\Ph u}.$ We can also bound $\NLtDn{u - \Ph u} \leq \Cwfn{0} \nvar^3h \NHnfn{1}{D}{\rho} \leq \Cwfn{0} \nvar^3\NLiD{n} h \NW{\rho}$ by \cref{lem:ellprojerrw}. Therefore by \cref{eq:qhnmmo} we have (as $kh \leq 1$ and $\nvar \geq 1$)
\beq\label{eq:qhnnearly}
k \Nfn{m-1,h}{\Qhn \rho} \leq \Chinv^{m-1}\Cwz \mleft(1+\CSZfn{0}\mright)\nvar^3\NLiD{n}\nmin^{1-m} h^{1-m}\NW{\rho}.
\eeq
Therefore using \cref{eq:thetahighnearly,eq:qhnnearly} (and the fact that $\NLiD{n}, \nmin^{-1} \geq 1,$ and so $\En, \nvar \geq1$)  we have

\beq\label{eq:readytorecurse}
\Nfn{m,h}{\thetah} \leq \CRecofn{m} \mleft(1+\mleft(\En\nvar\mright)^{\half(\floor{\frac{p-1}2}+1)}\NLiD{n}\nmin^{-\frac{p}2}\mright)k \Nfn{m-1,h}{\thetah} + \CRect \nvar^3\NLiD{n}\nmin^{1-m} h^{1-m} \NW{\rho}.
\eeq
Using \cref{eq:readytorecurse} recursively, and the facts that $\nmin^{-1} \geq 1$ and $hk \leq 1,$ we obtain \cref{eq:chigh}.
\epf

The following \lcnamecref{lem:continuity} is straightforward to prove, and used in the proof of \cref{lem:ltthetahbound} below
\ble[Continuity of $a$]\label{lem:continuity}
For any $\vo, \vt \in \HozDD,$
\beqs
\abs{a(\vo,\vt)} \leq \Cc \NLiD{n} \NW{\vo}\NW{\vt}.
\eeqs
\ele

\ble[Bound $\NLtDn{\thetah}$ by $\Nfn{p-1,h,n}{\thetah}$]\label{lem:ltthetahbound}
Under the assumptions of \cref{thm:fembound},%there exist constants $\Cfirst, \Csecond > 0$ such that
\begin{align}
\NLtD{\thetah}&\leq \mleft(\Cfirst\NW{\rho} +\Csecond  k^2h^p\Nfn{p-1,h,n}{\thetah}\mright)\mleft(\En \nvar\mright)^{\floor{\frac{p-1}2}+1}\NLiD{n}^2\nonumber\\
&\quad\quad\Pfn{p-2}\mleft(\NLiD{n}\mright)\mleft(\CFEMotilde h + \CFEMttilde \CAnk (hk)^p\mright)\label{eq:ltthetahbound}
\end{align}
\ele

\bpf[Proof of \cref{lem:ltthetahbound}]
The proof initially uses a standard duality technique, but then becomes more complex than standard proofs, as we are bounding $\thetah$ by its higher-order discrete norms, rather than the error $\eh$ by its $H^1$ norm.

Consider the adjoint variational problem: Find $w \in \HozDD$ such that for all $v \in \HozDD$
\beq\label{eq:adjointtheta}
a(v,w) = \IPLtDn{v}{\thetah}
\eeq
(i.e., $w$ solves the adjoint problem with right-hand side given by $n\thetah$). Let $\eh \de u -\uh$ be the finite-element error, and put $v = \eh$ in \cref{eq:adjointtheta}. By Galerkin orthogonality for $\eh$ and $u-\Ph u,$ we have (recalling $\eh = \rho + \thetah$)
\begin{align}
\IPLtDn{\eh}{\thetah} = a(\eh,w-\Ph w) &= \IPLtD{A \grad \rho}{\grad(w-\Ph w)}  -k^2 \IPLtDn{\eh}{w-\Ph w} -ik \IPLtGI{\eh}{w-\Ph w},\nonumber\\
&=a(\rho,w-\Ph w) -k^2 \IPLtDn{\thetah}{w-\Ph w} -ik \IPLtGI{\thetah}{w-\Ph w}\label{eq:doublego}
\end{align}

Therefore as $\eh = \rho + \thetah$ we can rearrange \cref{eq:doublego} and use Cauchy-Schwartz to obtain
\begin{align}
\NLtDn{\thetah}^2 &\leq \Cc\NLiD{n} \NW{\rho} \NW{w- \Ph w} + k^2 \abs{\IPLtDn{\thetah}{w- \Ph w}}\nonumber\\
&+ k \abs{\IPLtGI{\thetah}{w-\Ph w}} + \NLtDn{\rho}\NLtDn{\thetah}\label{eq:boundingLtwo}
\end{align}
By combining \cref{lem:bestapprox,lem:ellprojerr}, we can show (as $w$ satisfies an adjoint Helmholtz problem with right-hand side $\thetah$)
\beq\label{eq:wlt}
\NLtD{w - \Ph w} \leq \Cprojfn{0} \Pfn{p-2}\mleft(\NLiD{n}\mright)\mleft(\CFEMotilde h^2 + \CFEMttilde \CAnk h(hk)^p\mright)\NLtD{\thetah} \tand
\eeq
\beq\label{eq:who}
\NHoD{w - \Ph w} \leq 2\Cprojfn{-1} \Pfn{p-2}\mleft(\NLiD{n}\mright)\mleft(\CFEMotilde h + \CFEMttilde \CAnk (hk)^p\mright)\NLtD{\thetah}.
\eeq
We will be able to use \cref{eq:wlt,eq:who} to bound terms involving $w - \Ph w$ in \cref{eq:boundingLtwo}. We first estimate the inner product terms in \cref{eq:boundingLtwo}:
\begin{align}
\abs{\IPLtDn{\thetah}{w- \Ph w}} &= \abs{\IPLtDn{\thetah}{\Qhn w - \Ph w}}\nonumber\\
&\leq \Nfn{p-1,h,n}{\thetah}\Nfn{1-p,h,n}{\Qhn w - \Ph w}\text{ by \cref{lem:intoip}}\nonumber\\
&\leq \Nfn{p-1,h,n}{\thetah} \Csumfn{p-1} \mleft(\En \nvar\mright)^{\floor{\frac{p-1}2}}\NLiD{n}\nonumber\\
&\quad\quad\sum_{m=0}^{p-1} h^{m}\mleft(\NHnfn{-(p-1-m)}{D}{\Qhn w - w}+ \NHnfn{-(p-1-m)}{D}{w - \Ph w}\mright) \text{ by \cref{lem:negdiscsum}}\nonumber\\
&\leq \Nfn{p-1,h,n}{\thetah} \Csumfn{p-1}\mleft(\En \nvar\mright)^{\floor{\frac{p-1}2}}\NLiD{n} h^{p-1}\nonumber\\
&\quad\quad\sum_{m=0}^{p-1} \mleft(\CSZfn{p-1-m}\nvar \NLtDn{w - \Ph w} + \Cwfn{p-1-m} \errn{p-1-m} h \NHoDn{w - \Ph w}\mright)\nonumber\\
&\quad\quad\text{ by \cref{lem:wltdprojerr,lem:ellprojerrw}, taking $\wh = \Ph u$ in \cref{eq:wltdprojerr,lem:ellprojerrw}}\nonumber\\
&\leq \Nfn{p-1,h,n}{\thetah} \Csumfn{p-1}\sum_{m=0}^{p-1} \mleft(\CSZfn{p-1-m}\Cprojfn{0} + \Cwfn{p-1-m} \Cprojfn{-1}\mright)                         \mleft(\En \nvar\mright)^{\floor{\frac{p-1}2}+1}\NLiD{n}^2 \nonumber\\
&\quad\quad h^p \Pfn{p-2}\mleft(\NLiD{n}\mright)\mleft(\CFEMotilde h + \CFEMttilde \CAnk (hk)^p\mright) \NLtD{\thetah} \text{ by \cref{eq:wlt,eq:who}}.\nonumber\\
&=\Nfn{p-1,h,n}{\thetah}  \Cfourteen \mleft(\En \nvar\mright)^{\floor{\frac{p-1}2}+1}\NLiD{n}^2 h^p \Pfn{p-2}\mleft(\NLiD{n}\mright)\nonumber\\
&\quad\quad\mleft(\CFEMotilde h + \CFEMttilde \CAnk (hk)^p\mright) \NLtD{\thetah} \label{eq:innerprod1}.
\end{align}

We now estimate the other inner product term
\begin{align}
\abs{\IPLtGI{\thetah}{w - \Ph w}} &\leq \CMT\CinvVhp\mleft(\Cboundaryo^{\half} \mleft(\En\nvar\mright)^{\floor{\frac{p-1}2}+1}\NLiD{n} k h^{p-\half} \Nfn{p-1,h,n}{\thetah}\mright.\nonumber\\
&\mleft.\quad\quad+ \Cboundaryt^{\half} h^{\half} \NW{\rho}\mright) h^{-\half}\NLtD{w - \Ph w}\nonumber\\
&\quad\quad\text{ by \cref{lem:boundarybound,eq:simple,thm:multiplicativetrace,lem:inverseinequality}}\nonumber\\
&\leq \CMT \CinvVhp \Cprojfn{0}\mleft(\Cboundaryo^{\half} \mleft(\En\nvar\mright)^{\floor{\frac{p-1}2}+1}\NLiD{n} k h^{p} \Nfn{p-1,h,n}{\thetah} + \Cboundaryt^{\half} h \NW{\rho}\mright)\nonumber\\
&\quad\quad \Pfn{p-2}\mleft(\NLiD{n}\mright)\mleft(\CFEMotilde h + \CFEMttilde \CAnk (hk)^p\mright)\NLtD{\thetah}\label{eq:innerprod2}
\end{align}
 by \cref{eq:wlt}.

Now insert \cref{eq:wlt,eq:who,eq:innerprod1,eq:innerprod2} into \cref{eq:boundingLtwo}:
\begin{align*}
\NLtD{\thetah}^2 &\leq \Bigg[\Cc\NLiD{n} \NW{\rho} \mleft(\Cprojfn{0} + 2\Cprojfn{1}\mright)\\
&+k^2\Nfn{p-1,h,n}{\thetah}  \Cfourteen \mleft(\En \nvar\mright)^{\floor{\frac{p-1}2}+1}\NLiD{n}^2 h^p \\
&+k\CMT \CinvVhp \Cprojfn{0} \mleft(\Cboundaryo^{\half} \mleft(\En\nvar\mright)^{\floor{\frac{p-1}2}+1}\NLiD{n} k h^{p} \Nfn{p-1,h,n}{\thetah} + \Cboundaryt^{\half} h \NW{\rho}\mright)\Bigg]\nonumber\\
&\quad\quad \Pfn{p-2}\mleft(\NLiD{n}\mright)\mleft(\CFEMotilde h + \CFEMttilde \CAnk (hk)^p\mright)\NLtD{\thetah}\\
&+\NLtDn{\rho}\NLtDn{\thetah}\nonumber\\
%newline
&\leq \Bigg[\mleft(\Cc\mleft(\Cprojfn{0} + 2\Cprojfn{1}\mright)\NLiD{n} + \CMT \CinvVhp \Cprojfn{0} \Cboundaryt^{\half} + \Cprojfn{0}\nvar  \mright)\NW{\rho} \\
&+\mleft(  \Cfourteen \mleft(\En \nvar\mright)^{\floor{\frac{p-1}2}+1}\NLiD{n}^2 + \CMT \CinvVhp \Cprojfn{0}\Cboundaryo^{\half} \mleft(\En\nvar\mright)^{\floor{\frac{p-1}2}+1}\NLiD{n}\mright)k^2h^p\Nfn{p-1,h,n}{\thetah}\Bigg]\nonumber\\
&\quad\quad \Pfn{p-2}\mleft(\NLiD{n}\mright)\mleft(\CFEMotilde h + \CFEMttilde \CAnk (hk)^p\mright)\NLtD{\thetah}\nonumber\\
&\text{ rearranging and using \cref{lem:ellprojerr} and the fact that $hk \leq 1$}\nonumber\\
&\leq \Bigg[\mleft(\Cc\mleft(\Cprojfn{0} + 2\Cprojfn{1}\mright) + \CMT \CinvVhp \Cprojfn{0} \Cboundaryt^{\half} + \Cprojfn{0}  \mright)\NW{\rho} \nonumber\\
&+\mleft(  \Cfourteen  + \CMT \CinvVhp \Cprojfn{0}\Cboundaryo^{\half} \mright)k^2h^p\Nfn{p-1,h,n}{\thetah}\Bigg]\nonumber\\
&\quad\quad \mleft(\En \nvar\mright)^{\floor{\frac{p-1}2}+1}\NLiD{n}^2\Pfn{p-2}\mleft(\NLiD{n}\mright)\mleft(\CFEMotilde h + \CFEMttilde \CAnk (hk)^p\mright)\NLtD{\thetah}\nonumber\\
\end{align*}
and therefore using Young's inequality to separate out the $\NLtD{\thetah}$ term on the right-hand side, and then move it to the left hand side, followed by \cref{eq:simple}, we obtain
\begin{align}
\NLtD{\thetah}&\leq \Bigg[\mleft(\Cc\mleft(\Cprojfn{0} + 2\Cprojfn{1}\mright) + \CMT \CinvVhp \Cprojfn{0} \Cboundaryt^{\half} + \Cprojfn{0}  \mright)\NW{\rho}\nonumber \\
&+\mleft(  \Cfourteen  + \CMT \CinvVhp \Cprojfn{0}\Cboundaryo^{\half} \mright)k^2h^p\Nfn{p-1,h,n}{\thetah}\Bigg]\nonumber\\
&\quad\quad \mleft(\En \nvar\mright)^{\floor{\frac{p-1}2}+1}\NLiD{n}^2\Pfn{p-2}\mleft(\NLiD{n}\mright)\mleft(\CFEMotilde h + \CFEMttilde \CAnk (hk)^p\mright)\label{eq:part3penultimate},
\end{align}
and on rearranging \cref{eq:part3penultimate} we obtain \cref{eq:ltthetahbound}.

\epf
With all our technical lemmas proved, we can now prove our main \lcnamecref{thm:fembound}.

\bpf[Proof of \cref{thm:fembound}]
By using \cref{lem:higherbound} (with $m=p-1$) in \cref{eq:ltthetahbound} and the fact that $\NLiD{n}, \En, \nvar \geq 1$, we have
\begin{align}
\NLtD{\thetah}&\leq \mleft(\Cfirst + \Csecond\Chighfn{p-1,2} \nvar^3\NLiD{n}\nmin^{2-p} \mright)\mleft(\En \nvar\mright)^{\floor{\frac{p-1}2}+1}\NLiD{n}^2\nonumber\\
&\quad\quad\Pfn{p-2}\mleft(\NLiD{n}\mright)\mleft(\CFEMotilde h + \CFEMttilde \CAnk (hk)^p\mright)\NW{\rho}\nonumber\\
&+\Csecond  \Chighfn{p-1,1} \mleft(\mleft(\En\nvar\mright)^{\half(\floor{\frac{p-1}2}+1)}\NLiD{n}\nmin^{-\frac{p}2}\mright)^{p-1}\mleft(\En \nvar\mright)^{\floor{\frac{p-1}2}+1}\NLiD{n}^2\nonumber\\
&\quad\quad\Pfn{p-2}\mleft(\NLiD{n}\mright)\mleft(\CFEMotilde (hk )^{p+1}+ \CFEMttilde \CAnk h^{2p}k^{2p+1}\mright)\NLtD{\thetah}\label{eq:doublehkdep}
\end{align}
Choosing $h$ according to \cref{eq:hfemcond}, \cref{eq:doublehkdep} simplifies to
\begin{align*}
\NLtD{\thetah} &\leq \mleft(\Cfirst + \Csecond\Chighfn{p-1,2} \nvar^4\nmin^{1-p} \mright)\mleft(\En \nvar\mright)^{\floor{\frac{p-1}2}+1}\NLiD{n}^2\\
&\quad\quad\Pfn{p-2}\mleft(\NLiD{n}\mright)\mleft(\CFEMotilde h + \CFEMttilde \CAnk (hk)^p\mright)\NW{\rho} + \half \NLtD{\thetah},
\end{align*}

and therefore it follows that
\beq\label{eq:ltboundwithrho}
\NLtD{\thetah} \leq \nvar^6\nmin^{-(p+1)} \mleft(\En \nvar\mright)^{\floor{\frac{p-1}2}+1} \Pfn{p-2}\mleft(\NLiD{n}\mright)\mleft(\CLtboundo h + \CLtboundt  \CAnk (hk)^p\mright)\NW{\rho}.
\eeq

It only now remains to bound the weighted $H^1$ norm of the error. By \cref{lem:h1contdisc,lem:higherbound} (with $m=1$) we have
\begin{align}
\SNHoD{\thetah} &\leq \Amin^{\half} \mleft[\Chighfn{1,1} \mleft(\mleft(\En\nvar\mright)^{\half(\floor{\frac{p-1}2}+1)}\NLiD{n}\nmin^{-\frac{p}2}\mright)k \NLtD{\thetah}\mright.\nonumber\\
&\mleft.\quad\quad+ \Chighmt \nvar^3\NLiD{n} \NW{\rho}\mright],\label{eq:sntheta}
\end{align}

and by combining \cref{eq:sntheta,eq:ltboundwithrho} we obtain (as $hk \leq 1$ and $\nvar,\NLiD{n} \geq 1$)
\begin{align}
\SNHoD{\thetah} &\leq
\mleft(\mleft(\En\nvar\mright)^{\half(\floor{\frac{p-1}2}+1)}\NLiD{n}\nmin^{-\frac{p}2}\mright)\nvar^6\nmin^{-(p+1)} \mleft(\En \nvar\mright)^{\floor{\frac{p-1}2}+1}\Pfn{p-2}\mleft(\NLiD{n}\mright)\nonumber\\
&\quad\quad\mleft(\CHoboundo  +\CHoboundt \CAnk k(hk)^p\mright)\NW{\rho}.\label{eq:hoboundwithrho}
\end{align}
Therefore by combining \cref{eq:ltboundwithrho,eq:hoboundwithrho}, using \cref{eq:ellprojerr,lem:scottzhang} to bound $\NW{\rho}$, and using the fact that $hk \leq 1,$ we obtain \cref{eq:femltbound,eq:femhobound}.
\epf


