The results in this \namecref{sec:fem} are a generalisation of the results of \cite{DuWu:15} to the case of hetergenenous media; we also ensure our results are (in principle) explicit in $A$ and are explicit in $n,$ $k$, and $p.$

We first recap a collection of results that will be needed for the error bounds that follow. All of these results are already available elsewhere, but we name the constants explicitly, so that what follows is, in principle, explicit in $k,$ $A$, and $n.$

We assume throughout this section that were are working with degree-$p$ finite elements, for some integer $p \geq 1.$ Also, we work with a general impedance boundary (satisfying certain assumptions, listed below) and so our domain is $D$ rather than $\DR$.

Throughout this section, we make the following additional \namecrefs{ass:highp}:
\bas[Assumptions for higher-$p$ FEM bounds]\label{ass:highp}
\bit
\item $k \geq 1$
\item $\Aij \in \CpmooDclos$ for all $i,j$
\item $\GD$ and $\GI$ are $\Cpo$
\item $\NLiD{n} \geq 1.$
%\item $f \in \HpmoD$
%  \item $\gI \in \HpmhGI$
  \eit
\eas

Whilst the calculations in this \namecref{sec:fem} are explicit in all the constants involved, these dependencies are complex and, to a certain extent, unnecessary to understand the arguments. Therefore, the definition of all the constants (which are many-layered and interdependent) are relegated to \cref{app:constants}; that is, any constant introduced or defined in this \namecref{sec:fem} will be listed in \cref{app:constants}.

\subsection{Decomposition of solution and best approximation bound}

We first prove a best approximation bound in $\Vhp$ for the solution of the Helmholtz equation; following the presentation in \cite{ChNi:18a} (although we explicitly keep track of the constants involved at each point). In order to obtain bounds for higher $p$, we require the following shift \namecref{thm:shift}:

\bth[Shift theorem]\label{thm:shift}
Under \cref{ass:highp}, For all integers $l \in \mleft[0,p-1\mright]$ there exists a constant $\CAl>0$ (depending on $A$) such that, if $\ftilde \in \HlD$ and $\gItilde \in \HlphGI$, then there exists a unique $\utilde \in \HlptD$ such that $\utilde$ solves
\beqs
\grad \cdot \mleft(A \grad \utilde\mright) = -\ftilde,
\eeqs
\beqs
\dn \utilde = \gItilde, \tand
\eeqs
\beqs
\trD \utilde = 0
\eeqs
and $\utilde$ satisfies the bound
\beq\label{eq:shift}
\NHlptD{\utilde} \leq \CAl \mleft(\NHlD{\ftilde} + \NHlphGI{\gItilde}\mright).
\eeq
\enth

\bpf[Proof of \cref{thm:shift}]
The uniqueness and existence of $\utilde$ (in $\HozDD$) follows from the Lax--Milgram theorem, as the variational formulation of is bounded and coercive. The proof for the higher regularity bounds uses standard elliptic regularity estimates in the interior, and near the boundaries $\GD$ and $\GI$, as a reference for these we use \cite[pp. 137-138]{Mc:00}; \cref{ass:highp} means we can apply these results.

To deal with the interior regularity and regularity near the boundary separately, we define the following subsets of $D$: $\Dint,\Dinttilde,\Dscat,$ and $\Dtrunc$\optodo{Put this in a picture - sketch is in Research 17, notes for 17th May} with the following properties:
\bit
\item $\Dint \compcont \Dinttilde \compcont D$,
\item $\GD \subset \Dscatclos$
\item $\dist(\Dscat,\GI) > 0 $
  \item $\GI \subset \Dtruncclos$
\item $\dist(\Dtrunc,\GD) > 0 $
    \eit
    First applying interior regularity \cite[Theorem 4.16]{Mc:00} in $\Dinttilde,$ we obtain the bound
    \beq\label{eq:shiftint}
\NHlptDint{\utilde} \leq \CintAl \mleft(\NHoDinttilde{\utilde} + \NHlDinttilde{\ftilde}\mright).
\eeq
Applying regularity up to the boundary for Dirichlet data \cite[Theorem 4.18 (i)]{Mc:00} in $\Dscat,$ we obtain (as $\trD \utilde = 0$)
\beq\label{eq:shiftscat}
\NHlptDscat{\utilde} \leq \CscatAl \mleft(\NHoD{\utilde} + \NHlD{\ftilde}\mright)
\eeq
and similarly\optodo{Spell checker} for Neumann data \cite[Theorem 4.18 (ii)]{Mc:00} in $\Dtrunc,$ we obtain
\beq\label{eq:shifttrunc}
\NHlptDtrunc{\utilde} \leq \CtruncAl \mleft(\NHoD{\utilde} + \NHlphGI{\dn \utilde} + \NHlD{\ftilde}\mright).
\eeq
Combining \cref{eq:shiftint,eq:shiftscat,eq:shifttrunc}, we obtain \cref{eq:shift}.
\epf

We make the following \namecref{ass:htwo} on the solution of \cref{prob:vtedp}, and adopt notation for an a priori bound on its solution:

\bas\label{ass:htwo}
\Cref{prob:vtedp} (or its adjoint) has a unique solution $u$ in $\HtD$, and there exists $\CAnk>0$ (possibly dependent on $A$, $n,$ and $k$) such that
\beq\label{eq:generalhtwo}
k \NLtD{u} + \SNHoD{u} + \frac1k \SNHtD{u} \leq \CAnk \Cfg,
\eeq
where $\Cfg \de \NLtD{f} + \NLtGI{\gI}$.
\eas
Note that one could require $\NHhGI{\gI}$ on the right-hand side of \cref{eq:generalhtwo} (as $\gI \in \HhGI$); however, since the bound in \cref{thm:tedp} only has $\NLtGI{\gI}$, we use the form \cref{eq:generalhtwo} to include \cref{thm:tedp}.

We are now able to prove the following theorem giving a decomposition of the solution $u$ of \cref{prob:vtedp} into lower-order, less oscillatory parts (meaning the power of $k$ in the a priori bound is lower) and a smoother, more-oscilatory part. This result is essentially \cite[Theorem 1]{ChNi:18a}, in the particular case of a Helmholtz problem, but with the dependence on all the constants kept track of.

The following trace theorem is standard, see, e.g., \cite[Theorem 3.37]{Mc:00}.
\bth[Trace Theorem]\label{thm:trace}
If $v \in \HsD,$ for $1/2 < s \leq p+1$, then there exists $\CTrs > 0$ independent of $v$ such that
\beqs
\NHsmhGI{\trI v} \leq \CTrs \NHsD{v}.
\eeqs
\enth

Also, we will require the following result on the multiplication of functions.

\bth[Multiplication in $\HmD$]\label{thm:banachalg}
\ben
\item\label[itempart]{it:ban1}If $m > d/2,$ then for all $\vo, \vt \in \HsD$, $\vo\vt \in \HmD$ and there exists a constant $\CBanfn{s} > 0$ independent of $\vo$ and $\vt$ such that
\beqs
\NHmD{\vo\vt} \leq \CBanfn{m} \NHmD{\vo}\NHmD{\vt}.
\eeqs
\item\label[itempart]{it:ban2}For $m \in \ZZ, m \geq 1,$ if $\vo \in \CmD$, $\supp\mleft(\vo-1\mright) \compcont D,$ and  $\vt \in \HmD,$ then $\vo \in \WmiD,$ $\vo\vt \in \HmD$, and there exists a constant $\Cprod{m} > 0$ independent of $\vo$ and $\vt$ such that%If $\vo \in \CrcompD$ for some $r \in \NN, r \geq m,$ and
\beq\label{eq:ban2}
\NHmD{\vo\vt} \leq \Cprod{m} \mleft(1+\NWmiD{\vo}\mright)\NHmD{\vt}.
\eeq
\een
\enth

\bpf[Proof of \cref{thm:banachalg}]
\Cref{it:ban1} is given in\optodo{Find ref}. For \cref{it:ban2}, observe that as $\vo-1$ has compact support, $\vo \in \WmiD,$ and $\vo-1 \in \CmcompD$. By \cite[Theorem 3.20]{Mc:00}, there exists $\Cmclean{m} > 0$ such that\footnote{In \cite[Theorem 3.20]{Mc:00} $\Cmclean{m}$ is denoted $C_{m}$.}
\beqs
\NHmD{(\vo-1)\vt} \leq \Cmclean{m} \NWmiD{\vo-1}\NHmD{\vt}.
\eeqs
Therefore as $\vo = \vo -1 + 1,$ we have \cref{eq:ban2}.
\epf

We first have this simple \namecref{lem:domainshift}:

\ble\label{lem:domainshift}
Let $\ftilde \in \HlD$ and $\gtilde \in \HlpoD,$ for $0 \leq l \leq p-1$. If $\utilde \in \HlptD$ solves
\beqs
\grad \cdot \mleft(A\grad \utilde\mright) = -\ftilde,
\eeqs
\beqs \trGD u = 0,
\eeqs
and
\beqs
\dn u = \gtilde,
\eeqs
then
\beqs
\NHlptD{\utilde} \leq \CAl\mleft(1+\CTrlpo\mright)\mleft(\NHlD{\ftilde} + \NHlpoD{\gtilde}\mright).
\eeqs
\ele
\bpf[Proof of \cref{lem:domainshift}]
By \cref{thm:shift,lem:domainshift}
\beqs
\NHlptD{\utilde} \leq \CAl \mleft(\NHlD{\ftilde} + \NHlphGI{\gtilde}\mright) \leq \CAl \mleft(\NHlD{\ftilde} + \CTrs\NHlpoD{\gtilde}\mright),
\eeqs
and the result follows.
\epf

We can now prove our expansion of the solution of \cref{prob:vtedp} or its adjoint.

\bth[Expansion of the solution of the Helmholtz equation]\label{thm:expansion}
Under \cref{ass:highp,ass:htwo} there exists $\uosc \in \HppoD$ and a sequence $\usj \in \HjptD,$ $j = 0,\ldots,p-2$ such that
\beq\label{eq:expansionuj}
\NHjptD{\usj} \leq \Cej \Pj(\NLiD{n}) k^j \Cfg, \text{ for }\Cj > 0,
\eeq
\beq\label{eq:expansionuosc}
\NHppoD{\uosc} \leq \Cosc \CAnk k^p \Cfg, \text{ for some } \Cosc > 0,
\eeq
and
\beq\label{eq:expansionid}
u = \uosc + \sum_{j=0}^{p-2} \usj.
\eeq
Where
\beq\label{eq:p}
\Pj(x) =
\begin{dcases}
1 & j = 0,1\\
x^{\floor{j/2}}& j \geq 2
\end{dcases}
\eeq.
%=\sum_{m=0}^{j-2}\pfn{j,m}x^m $
%% are polynomials of degree
%% \beq\label{eq:polydegree}
%% \begin{dcases}
%% \frac{j}2 & \tif  j \text{ is even}\\
%% \frac{j-1}2 \tif j  \text{ is odd}.
%% \end{dcases}
%% \eeq
%% (except for $j=0,1,$ where $\Pj$ is a polynomial of degree 0) given by the recurrence relation
%% \beq\label{eq:pjdef}
%% \Pj(x) = \CAj \mleft(1+\CTrjpo\mright)\mleft(x\Pfn{j-2}(x) + \Pfn{j-1}(x)\mright).
%% \eeq
% No idea if this coefficient stuf is right
%% \begin{align}
%% \label{eq:p1}\pfn{0,0}&= \CAz\\
%% \label{eq:p2}\pfn{1,0}&  = \CAz\CAo\mleft(1+\CTrt\mright)\\
%% \label{eq:p3}\pfn{j,m} &=
%% \begin{dcases}
%%  0, &\tif m \leq \floor{\frac{j-2}2}\\
%% \CAj\mleft(1+\CTrjpo\mright) \mleft(\pfn{j-2,m-1} + \pfn{j-1,m-1}\mright),&\tif \floor{\frac{j-2}2} < m \leq j-3
%% \end{dcases}\\
%% \label{eq:p4}\pfn{j,j-2}& = \CAj\mleft(1+\CTrjpo\mright) \pfn{j-1,j-3},
%% \end{align}
%% and $\Cosc$ is given by the recurrence relations
%% \begin{align}
%% \label{eq:osc1}\Coscfn{1} &= \CAo\mleft(1+\CTrt\mright),\\
%% \label{eq:osc2}\Coscfn{2} &= \CAt\mleft(1+\CTrth\mright)\mleft(1 + \CAo\mleft(1+\CTrt\mright)\mright)\tand\\
%% \label{eq:osc3}\Coscfn{j} &= \CAfn{j}\mleft(1+\CTrfn{j+1}\mright)\mleft(\Coscfn{j-1} + \Coscfn{j-2}\mright),
%% \end{align}
%% and
%% \beq\label{eq:cosc}
%% \Cosc = \frac{\Cefn{p-1}}{\max\set{1,\CAz}}.
%% \eeq

\enth

\bit
\item Powers of $k$ in non-oscillatory parts one smaller
\item Just copying Theo and Serge
\item Make contact with Melenk and Sauter
\item In principle exponential growth of constant with $p$
\eit

\bpf[Proof of \cref{thm:expansion}]
The idea of the proof is as follows. We write $u$ as a formal series expansion
\beq\label{eq:formalseries}
u = \sum_{j=0}^\infty \usj,
\eeq
and then substitute this series into the PDE \cref{eq:tedp} and the boundary condition \eqref{eq:ibc}. Equating powers of $k$, we derive a recursive sequence of stationary diffusion equations for the functions $\usj,$ with right-hand sides dependent on $\ujmo$ and $\ujmt$. We use this recursive sequence and \cref{lem:domainshift} to prove the a priori bounds \cref{eq:expansionuj}. We then define the $l$th remainder $\rl = u - \sum_{j=0}^{l-1} \usj,$ and by applying the operator $\grad\cdot\mleft(A\grad \cdot\mright)$ with Neumann boundary conditions to $\rl$, we obtain a recursive sequence for the remainders $\rl,$ and can similarly prove a priori bounds for the $\rl$s. The oscillatory function $\uosc$ is then just $\rpmo.$ The format of this proof is identical to that in \cite[Theorem 1]{ChNi:18a}, except we keep track of all of the constants involved.

For the purposes of the proof, it is more convenient to define $\vj = \usj/k^j,$ so that the series expansion \cref{eq:formalseries} becomes
\beq\label{eq:formalseriesv}
u = \sum_{j=0}^\infty k^j\vj
\eeq
as in \cite{ChNi:18a}. Also, in this proof, all the boundary-value problems involved included a zero Dirichlet condition on the scatterer $\GD;$ we omit this throughout the proof for brevity.

By applying the Helmholtz operator to the formal series \eqref{eq:formalseriesv} we obtain the following problems for $\vj \in \HjptD, j \geq 1$:
\beqs
\grad \cdot \mleft(A\grad \vz\mright) = -f \quad\tand\quad \dn \vz = \gI,
\eeqs
\beqs
\grad \cdot \mleft(A\grad \vo\mright) = 0\quad\tand\quad\dn \vo = i\vz,
\eeqs
and, for $j \in \mleft[2,p-2\mright]$
\beq\label{eq:vj}
\grad \cdot \mleft(A\grad \vj\mright) = - n\vjmt\quad\tand\quad\dn \vz = i\vjmo.
\eeq

By \cref{thm:shift} we immediately conclude the bound
\beq\label{eq:expuz}
\NHtD{\vz} \leq \CAz\Cfg \leq \Cefn{0}\Cfg,
\eeq
and by \cref{lem:domainshift,eq:expuz} we can conclude the bound
\beqs
\NHthD{\vo} \leq \CAo \mleft(1+\CTrt\mright)\NHtD{i\vz} \leq \max\set{1,\CAo}\mleft(1+\CTrfn{2}\mright)\Cefn{0} \Cfg = \Cefn{1}\Cfg.
\eeqs

We prove the bound \cref{eq:expansionuj} by induction; suppose \cref{eq:expansionuj} holds for all $s \in [0,j-1].$ Using \cref{lem:domainshift}, we conclude that (using the observation that $\Cefn{j-1} \geq \Cefn{j-2}$)
\begin{align*}
\NHfn{j+2}{D}{\vj} &\leq \CAj \mleft(1+\CTrjpo\mright)\mleft(\NLiD{n} \NHfn{j}{D}{\vfn{j-2}} + \NHfn{j+1}{D}{\vfn{j-1}}\mright)\\
&\leq \max\set{1,\CAj} \mleft(1+\CTrjpo\mright)\mleft(\NLiD{n} \Cefn{j-2} \Pfn{j-2}\mleft(\NLiD{n}\mright) + \Cefn{j-1} \Pfn{j-1}\mleft(\NLiD{n}\mright)\mright)\Cfg\\
&\leq 2\max\set{1,\CAj} \mleft(1+\CTrjpo\mright)\Cefn{j-1} \Pfn{j}(\NLiD{n})\Cfg\\
&= \Cefn{j} \Pfn{j}(\NLiD{n})\Cfg.
\end{align*}
%% Therefore the definition of the polynomials $\Pj$ \cref{eq:pjdef} holds (and it is straightforward to see that $\deg\mleft(\Pj\mright) = \deg\mleft(\Pfn{j-2}\mright)+1$, where $\deg$ denotes polynomial degree, and therefore since $\deg\mleft(\Pfn{0}\mright) = \deg\mleft(\Pfn{1}\mright)=0,$ \cref{eq:polydegree} holds. Hence, by the relationship between $\usj$ and $\vj$, we have the bound \cref{eq:expansionuj}.

We will now define the remainders $\rl$, and proceed similarly.
Let $\ro \in \HthD$ solve
\beqs
\grad \cdot \mleft(A\grad \ro\mright) = -k^2 u
\eeqs
\beqs
\dn \ro = iku.
\eeqs
Then by \cref{lem:domainshift}
\begin{align*}
\NHthD{\ro} &\leq \CAo\mleft(1+\CTrt\mright)\mleft(k^2\NHoD{u} + k\NHtD{u}\mright)\\
&\leq \CAo\mleft(1+\CTrt\mright)k^2\CAnk\Cfg\\
&\leq \frac{\Cefn{1}}{\max\set{1,\CAo}}.
\end{align*}
Let $\rt \in \HfD$ solve
\beqs
\grad \cdot \mleft(A\grad \rt\mright) = -k^2 u
\eeqs
\beqs
\dn \rt = ik\ro.
\eeqs
Then by \cref{lem:domainshift}
\begin{align*}
\NHfD{\rt} &\leq \CAt\mleft(1+\CTrth\mright)\mleft(k^2\NHtD{u} + k\NHthD{\ro}\mright)\\
&\leq \CAt\mleft(1+\CTrth\mright)\mleft(1 + \frac{\Cefn{1}}{\max\set{1,\CAo}}\mright)\CAnk k^3\Cfg\\
&\leq \frac{\Cefn{2}}{\max\set{1,\CAo}}
\end{align*}
Then for $j \geq 3,$ let $\rj \in \HjptD$ solve
\beqs
\grad \cdot \mleft(A\grad \rt\mright) = -k^2 \rjmt
\eeqs
\beqs
\dn \rj = ik\rjmo.
\eeqs
And by induction and \cref{lem:domainshift} again, letting $\uosc = \rfn{p-1},$ we have \cref{eq:expansionuosc}. It is straightforward to see that $\rfn{p-1} + \sum_{j=1}^{p-2} \uj$ solves \cref{prob:tedp}, and therefore \cref{eq:expansionid} holds, as $u$ is unique.
\epf
%\optodo{Just a note while I think about it - could you do DtN boundary conditions in this proof by testing with a different function when wanting to bound the $L^2$ norm on the boundary? I wonder if testing with the NtD of $u$ would mean you end up with a $\LtGI{u}^2$ term, and then you could use properties of the NtD map to bound other bits. Might foil $k$-dependency though.}
Using the expansion in \cref{thm:expansion}, we can prove the following error bound for the best approximation of $u$ in $\Vhp$:

%% We also have the following best-approximation error in higer-order Sobolev spaces:
%% \ble[Best approximation in $\HsD$]\label{lem:bestapproxhigh}
%% For integer $s$ in $[1,p+1]$, there exists $\Cinterps>0$ such that for every $v \in \HsD$ there exists $\vhhat \in \Vhp$ such that
%% \beqs
%% \NLtD{v - \vhhat} + h\NHoD{v-\vhhat} \leq \Cinterps h^s \SNHsD{v}.
%% \eeqs
%% \ele

\ble[Best approximation error bound]\label{lem:bestapprox}
Under \cref{ass:highp,ass:htwo}, There exist constants $\CFEMo, \CFEMt > 0$ and independent of $k$ (although dependent on $A$ and $n$) such that if $u$ solves \cref{prob:vtedp} or its adjoint, then there exists $\uhhat \in \Vhp$ such that
\beq\label{eq:bestapproxL2}
\NLtD{u-\uhhat} \leq \mleft(\CFEMo \Pfn{p-2}\mleft(\NLiD{n}\mright) h^2 + \CFEMt \CAnk h\mleft(hk\mright)^p \mright)\Cfg,
\eeq
%% \beq\label{eq:bestapproxH1}
%% \NHoD{u-\uhhat} \leq \mleft(\CFEMo h + \CFEMt \CAnk \mleft(hk\mright)^p \mright)\Cfg \tand
%% \eeq
\beq\label{eq:bestapproxW}
\NW{u-\uhhat} \leq 2\mleft(\CFEMo \Pfn{p-2}\mleft(\NLiD{n}\mright) h + \CFEMt \CAnk \mleft(hk\mright)^p \mright)\Cfg.
\eeq
\ele

We write bounds for the standard and weighted $H^1$ norms separately, as we will use each individual bound in differents of proofs in \cref{sec:fembound}.

\bpf[Proof of \cref{lem:bestapprox}]
We apply \cref{lem:scottzhang} to all the $\usj$ and $\uosc$ in \cref{thm:expansion}, and obtain that there exists $\ujh,$ $j=0,\ldots,p-2$ and $\uosch$ in $\Vhp$ such that 
\beqs
\NLtD{\usj - \ujh} + h\NHoD{\usj - \ujh} \leq \Cinterpfn{j+2} \Cefn{j} \Pj\mleft(\NLiD{n}\mright) h^{j+2}k^j \Cfg
\eeqs
and
\beqs
\NLtD{\uosc - \uosch} + h\NHoD{\uosc - \uosch} \leq \Cinterpfn{p+1} \Cosc\CAnk h^{p+1}k^p \Cfg.
\eeqs
Therefore, by letting $\uhhat = \uosch + \sum_{j=0}^{p-2} \ujh,$ we have (using the fact that $hk \leq 1$ and, as $\NLiD{n} \geq 1$, $\Pfn{p-2}(\NLiD{n}) \geq \Pfn{j}(\NLiD{n})$ for all $j \leq p-2$) \cref{eq:bestapproxL2,eq:bestapproxW}.
\epf

\subsection{Error bounds for simple Galerkin projections}\label{sec:errgalerkin}
In this \namecref{sec:errgalerkin} we state a sequence of error bounds for various projections, in various norms. The proofs of these error bounds are all simple modifications of the proofs of \cite[Theorem 5.8.3]{BrSc:08},  and so we only sketch \cref{lem:wltdprojerr} below; as this result uses the $n$-weighted norms and inner product.\ednote{Both-should I write out at least one of these modified proofs?}. % Note to self, I have these proofs sketched out, but just use the argument from Brenner and Scott and use an L^2 inner product on the function, not their gradients, for the elliptic projection.
We first define the projections we use.

Define the elliptic projection $\Ph:\HozDD\rightarrow\Vhp$ by, for $w \in \HozDD$
\beqs
\IPLtD{A\grad\vh}{\grad\Ph w} = \IPLtD{A\grad\vh}{\grad w} \tforall \vh \in \Vhp
\eeqs

Observe that $\Ph$ is simply the finite-element solution of a stationary diffusion problem with a particular right-hand side in $\HozDD'.$

We define the $\LtD$-projection $\Qh:\HozDD\rightarrow \Vhp$ by, for $w \in \HozDD$
\beqs
\IPLtD{\Qh w}{\vh} = \IPLtD{w}{\vh} \tforall \vh \in \Vhp.
\eeqs

We also need to define the $\LtD$ projection in a norm weighted by $n$

For $v,w \in \LtD,$ define the $n$-weighted inner product
\beqs
\IPLtDn{v}{w} = \int_{D} n v \wbar,
\eeqs
and the corresponding $n$-weighted $\LtD$ norm
\beqs
\NLtDn{v} = \sqrt{\int_{D} n \abs{v}^2}.
\eeqs
The $n$-weighted $\HsD$ norms, for $s \in \NN$, are then defined by $\NHsDn{v}^2 = \sum_{\alpha \st \abs{\alpha} \leq s}\NLtDn{D^\alpha v}$, and the negative weighted Sobolev norms are defined by, for $s \in \NN$
\beq\label{eq:negweightnorm}
\NHnfn{-s}{D}{v} = \sup_{w \in \HsD} \frac{\IPLtDn{v}{w}}{\NHsDn{w}}.
\eeq
Observe that, for $v \in \HsD,$
\beq\label{eq:nconv}
\nmin\NHsD{v} \leq \NHsDn{v} \leq \NLiD{n} \NHsD{v}.
\eeq
%We put the \emph{non-weighted} $H^s$-norm in the denominator of \cref{eq:negweightnorm}, but the \emph{weighted} inner product in the numerator for ease of manipulation in some of the following proofs---we could place the weighted $H^s$ norm in the denominator, and this would be equivalent to the current definition, up to a factor involving $n$.
%% so that we have, for all $v \in \LtD$
%% \beq\label{eq:neqweightid}
%% \NHmsD{n v} = \NHmsDn{v}.
%% \eeq
%% \beq\label{eq:negativeweightbounds}
%% \frac{\NHmsD{v}}{\NLiD{n}} \leq \NHmsDn{v} \leq \frac{NHmsD{v}}{\nmin}.
%% \eeq

Define the $\LtD$ projection in the $n$-weighted norm $\Qhn:\HozDD\rightarrow \Vhp$ by, for $w \in \HozDD$
\beqs
\IPLtDn{\Qhn w}{\vh} = \IPLtDn{w}{\vh} \tforall \vh \in \Vhp.
\eeqs


The elliptic projection obeys the following error bounds:
\ble[Error bounds for elliptic projection]\label{lem:ellprojerr}
\optodo{Check if this is still needed}
For any integer $s \in [-1,p-1],$ there exists a constant $\Cmso >0$ such that for all $w \in \HozDD$
\beq\label{eq:ellprojerr}
\NHmsD{w-\Ph w} \leq \Cmso h^{s+1} \BAHoD{w}{\wh}.
\eeq
\ele

We state the Poincar\'e inequality:
\ble[Poincar\'e Inequality]\label{lem:poincare}
There exists $\CP > 0$ such that for all $v \in \HozDD$
\beqs
\NLtD{v} \leq \CP \SNHoD{v}.
\eeqs
\ele

\ble[Error bounds for elliptic projection in $n$-weighted norms]\label{lem:ellprojerrw}
Let $s \in \ZZ.$ There exists a constant $\Cwmso >0$ such that for all $w \in \HozDD$
\ben
\item\label[itempart]{it:wep1} If $s \in (d/2,p-1],$ and $n \in \HsD$ then
\beq\label{eq:wep1}%\label{eq:ellprojerr}
\NHnfn{-s}{D}{w-\Ph w} \leq \Cwmso \frac{\NHsD{n}\NLiD{n}^2}{\nmin^3}h^{s+1} \BAHoDn{w}{\wh}.
\eeq
\item\label[itempart]{it:wep2} If $s \in [1,d/2]$ and $n \in \CsD$, then
\beq\label{eq:wep2}
\NHnfn{-s}{D}{w-\Ph w} \leq \Cwmso \frac{\NWsiD{n}\NLiD{n}^2}{\nmin^3} h^{s+1} \BAHoDn{w}{\wh}.
\eeq
\item\label[itempart]{it:wep4} For $s = 0$
\beq\label{eq:wep4}
\NLtDn{w-\Ph w} \leq \Cwfn{0,1} \frac{\NLiD{n}^3}{\nmin^3}h \BAHoDn{w}{\wh}
\eeq
\item\label[itempart]{it:wep3} For $s = 1$
\beq\label{eq:it:wep3}
\NHnfn{1}{D}{w-\Ph w} \leq \Cwfn{1,1}\frac{\NLiD{n}^2}{\nmin^2} \BAHoDn{w}{\wh}.
\eeq
\een
\ele

\bpf[Proof of \cref{lem:ellprojerrw}]
For \cref{it:wep1,it:wep2}, the proof follows the proof in \cite[Theorem 5.8.3]{BrSc:08}. Let $\phi \in \HsD,$ and observe that by \cref{thm:banachalg} $n\phi\in \HsD$ also. Let $ \vtilde = \sdsol(n\phi),$ observe that by \cref{thm:shift} $\vtilde \in \Hfn{}{s+2}{D}$. For all $v \in \HozDD,$ we have
\beqs
\IPLtD{A \grad \vtilde}{\grad v} = \IPLtD{n\phi}{v} = \IPLtDn{\phi}{v}.
\eeqs
If we take $v = w-\Ph w,$ then we can compute
\begin{align}
\IPLtDn{\phi}{v} &= \IPLtD{A\grad\mleft(\vtilde - \vh\mright)}{\grad\mleft(w-\Ph w\mright)} \text{ for } \vh \text{ as in \cref{lem:scottzhang}, by Galerkin orthogonality for } \Ph\nonumber\\
&\leq \CSZfn{s+2}\NLiDop{A} \NHfn{s+2}{D}{\vtilde} h^{s+1} \SNHoD{w-\Ph w} \text{ by \cref{lem:scottzhang}}\nonumber\\
&\leq \CSZfn{s+2} \CAfn{s} \NLiDop{A} \NHfn{s}{D}{n\phi} h^{s+1}\SNHoD{w-\Ph w}\text{ by \cref{thm:shift}}\label{eq:ellprojwpart}
\end{align}

For \cref{it:wep1}, $s > d/2,$ and so by \cref{thm:banachalg}, \cref{eq:ellprojwpart} is bounded above by
\beq\label{eq:wep1pt0}
\CSZfn{s+2} \CAfn{s} \CBanfn{s} \NLiDop{A} \frac{\NHfn{s}{D}{n}}{\nmin}\NHnfn{s}{D}{\phi} h^{s+1}\SNHoD{w-\Ph w}
\eeq
by \cref{thm:banachalg,eq:nconv}.

Therefore we have, by definition of $\NHnfn{-s}{D}{\cdot}$ and $\NHoDn{\cdot}$,
\beq\label{eq:wep1pt1}
\NHnfn{-s}{D}{w-\Ph w} \leq \CSZfn{s+2}\CAfn{s} \CBanfn{s} \NLiDop{A} \frac{\NHmD{n}}{\nmin^2} h^{s+1} \NHoDn{w - \Ph w}.
\eeq
Applying C\'ea's Lemma to $\Ph$ in the weighted $H^1$ norm, we find
\beq\label{eq:wepcea}
\NHoDn{w-\Ph w} \leq \frac{2\NLiDop{A}\NLiD{n}^2}{\min\set{1,1/\CP^2}\Amin\nmin^2}\BAHoDn{w}{\wh},
\eeq
and then combining \cref{eq:wep1pt1,eq:wepcea}, we obtain \cref{eq:wep1}.

For \cref{it:wep2}, the application of \cref{thm:banachalg} yields, instead of \cref{eq:wep1pt0},
\beq\label{eq:wep2pt1}
2\CSZfn{s+2} \CAfn{s} \Cprod{s} \NLiDop{A} \frac{\NWsiD{n}}{\nmin}\NHnfn{s}{D}{\phi} h^{s+1}\SNHoD{w-\Ph w},
\eeq
since $1 + \NWsiD{n} \leq 2\NWsiD{n},$ as $\NLiD{n} \geq 1,$ and by a similar reasoning to that before, we obtain \cref{eq:wep2}.

For \cref{it:wep3}, \cref{eq:wepcea} immediately gives the result. For \cref{it:wep4}, performing a standard duality argument in \emph{non-weighted} norms (and then using \cref{eq:it:wep3}, yields \cref{eq:wep4}.
\epf


The $\LtD$ projection satisfies the following error bound:
\ble[Error bounds for $\LtD$ projection]\label{lem:ltdprojerr}
For any integer $s \in [0,p-1],$ for all $w \in \LtD$
\beqs
\NHmsD{w-\Qh w} \leq \Cmsz h^{s} \BALtD{w}{\wh}.
\eeqs
\ele

The $n$-weighted $\LtD$ projection satisfies the following error bound:
\ble[Error bounds for weighted $\LtD$ projection]\label{lem:wltdprojerr}
For any integer $s \in [0,p-1],$ for all $w \in \HozDD$
\beq\label{eq:wltdprojerr}
\NHmsDn{w-\Qhn w} \leq \CSZfn{s} \frac{\NLiD{n}}{\nmin} h^{s} \BALtDn{w}{\wh}.
\eeq
\ele

\bpf[Proof of \cref{lem:wltdprojerr}]
The following proof is a slight modification of the proof in \cite[Theorem 5.8.3]{BrSc:08}. Fix $\vtilde \in \HsD$, trivially $\vtilde$ solves the adjoint problem
\beqs
\IPLtDn{v}{\vtilde} = \IPLtDn{v}{\vtilde} \tforall v \in \LtD.
\eeqs
Then letting $v=w-\Qhn w$ and using Galerkin orthogonality, for $\vhptilde$ as in \cref{lem:scottzhang} we have
\begin{align*}
\IPLtDn{w-\Qhn w}{\vtilde} &\leq \NLtDn{w-\Qhn w}\NLtDn{\vtilde-\vhptilde}\\
&\leq \CSZfn{s} \frac{\NLiD{n}}{\nmin}\NLtDn{w-\Qhn w} h^s \NHsDn{\vtilde}.
\end{align*}
Taking the supremum over $\vtilde,$ we have
\beqs
\NHmsDn{w-\Qhn w} \leq \CSZfn{s} \frac{\NLiD{n}}{\nmin} h^s \NLtDn{w-\Qhn w},
\eeqs
and hence by C\'ea's Lemma (as the inner product $\IPLtDn{\cdot}{\cdot}$ is clearly bounded and coercive in the weighted $L^2$-norm $\NLtDn{\cdot}$) the result follows.
\epf

\subsection{Discrete Sobolev spaces}\label{sec:discsob}
In order to perform our analysis for high-order FEM, we will need to measure higher-order norms of functions in the finite-element space $\Vhp.$ However, as these functions do not have higher-order weak derivates, we must first develop some theory of so-called discrete Sobolev spaces; we follow the presentation in \cite{DuWu:15}, albeit working in the heterogeneous case, and with some changes of notation.

We let $\DeltaAI:\LtD\rightarrow\HtD$ denote the solution operator for the stationary diffusion equation: given $\ftilde \in \LtD$ find $\utilde \in \HtD$ such that
\beq\label{eq:sdeq}
\grad \cdot \mleft(nA\grad \utilde\mright) = -n\ftilde \text{ in } D
\eeq
\beq\label{eq:sddbc}
\trD \utilde = 0
\eeq
\beq\label{eq:sdnbc}
\dn \utilde = 0.
\eeq
%% $A$-weighted Laplacian; that is $\DeltaA w = \grad\cdot\mleft(\grad w\mright),$ and give it domain $\DomainDeltaA = \set{w \in \HtD \st \trD w = 0, \dn w = 0};$
%% hence $\DeltaA:\DomainDeltaA \rightarrow \LtD.$ Observe that, for any $f \in \LtD$ there exists $\wf \in \DomainDeltaA$ such that $\DeltaA \wf = -f.$
Observe that $\sdsol$ is well-defined by \cref{lem:domainshift}. Also, observe that $\sdsol^{m}$ is defined for any $m \in \NN,$ as $\HtD \subseteq \LtD,$ and so one can place $\sdsol \ftilde$ on the right-hand side. 
Observe that for any $\ftilde \in \LtD$ and for any $v \in \HozDD,$ we have, by Green's identity,
\beqs\label{eq:deltaagreen}
\int_D n\mleft(A \grad \mleft(\DeltaAI\ftilde\mright)\mright)\cdot \grad \vb = \int_D n\ftilde \vb,
\eeqs
i.e.,
\beqs
\IPLtDn{A\grad \mleft(\sdsol \ftilde\mright)}{\grad v} = \IPLtDn{\ftilde}{v}.
\eeqs
\optodo{Chat}

\bde[Discrete derivative operator]
Define the \defn{$A$-weighted discrete second derivative operator} $\Deltah:\Vhp\rightarrow\Vhp$ for $\wh \in \Vhp$ by
\beq\label{eq:discderdef}
\IPLtDn{\Deltah \wh}{\vh} = \IPLtDn{A \grad \wh}{\grad \vh} \tforall \vh \in \Vhp.
\eeq
\ede

\ble[Discrete derivative operator is well-defined]\label{lem:ddwd}
For any $\wh \in \Vhp,$ $\Deltah \wh$ exists and is unique.
\ele

\bpf[Proof of \cref{lem:ddwd}]
If one chooses an orthonormal (in the $n$-weighted inner product) basis  $\phij$ for $\Vhp,$ writes $\Delta\wh = \sum_j \wj \phij$, and takes $\vh = \phij$ for each $j,$ then we see \cref{eq:discderdef} is equivalent to the linear system $\Imat \bw = \bb,$ where $\bb_{j} = \IPLtDn{A \grad \wh}{\grad \phij}.$ The solution of this linear system clearly exists and is unique.
%% We equip $\Vhp$ with the $H^1$-norm. Observe that $\Deltah \wh$ satisfies the variational problem: Find $\vhtilde  \in \Vhp$ such that $\add(\uh,\vh) = \Ldd(\vh)$ for all $\vh \in \Vhp,$ where $\add(\uh,\vh) = \IPLtD{\uh}{\vh}$ and $\Ldd(\vh) = \IPLtD{A \grad \wh}{\grad \vh}.$ Observe that $\Ldd$ is bounded in $\Vhp$, as $\Ldd(\vh) \leq \NLiDop{A} \SNHoD{\wh}\SNHoD{\vh} \leq \NLiDop{A} \SNHoD{\wh}\NHoD{\vh},$ and $\add$ is coercive on $\Vhp,$ as, for $\vh \in \Vhp$, $\add(\vh,\vh) = \NLtD{\vh}^2 \geq \CinvVhp^2 \NHoD{\vh}^2$ by the standard inverse estimate\optodo{Add in?}. Therefore, by the Lax--Milgram Theorem applied in $\Vhp$ (as $\Vhp$ is a finite-dimensional inner product space over a complete field, it is a Hilbert space), $\Deltah \wh$ exists and is unique.
\epf

%% By\optodo{McLean Thm 4.12 - double check and maybe write out in more detail---exactly what operator are we talking about, especially if we want weak derivatives?}, there exists a sequence of eigenfunctions $\phio,\phit,\ldots \in \HoD$ of $\DeltaA$\optodo{What exactly does McLean mean here, if they don't have second-order derivatives?} and corresponding eigenvalues $0 < \lambdao<\lambdat < \cdots \rightarrow\infty$ such that the eigenfunctions form a complete orthonormal system in $\LtD.$\optodo{Maybe define this}

Since $A$ is real and symmetric, it is self-adjoint. Hence it follows that $\Deltah$ is self-adjoint, as
\beqs
\IPLtDn{\Deltah \wh}{\vh} = \IPLtDn{A \grad \wh}{\grad \vh} = \IPLtDn{\grad \wh}{A\grad \vh} = \overline{\IPLtDn{\Deltah \vh}{\wh}} = \IPLtDn{\wh}{\Deltah \vh}.
\eeqs
Therefore $\Deltah$ is diagonalisable, i.e., there exists a set of eigenfunctions $\phioh,\ldots,\phidimVhph$  with corresponding real eigenvalues\ednote{Both - is it common enough knowledge that a symmetric matrix has real eigenvalues, that I don't need to reference this?} $\lambdaoh, \ldots, \lambdadimVhph$ such that the $\phimh$ form an orthonormal (in the $n$-weighted inner product basis of $\Vhp$.

\bde[Higher-order discrete derivative operators]
For $\vh \in \Vhp$, if $\vh = \sum_{m=1}^{\dimVhp} \am \phimh,$ then for $j \in \RR$ define
\beqs
\Deltah^j \vh = \sum_{m=1}^{\dimVhp} \lambdamh^j \am \phimh.
\eeqs


\ede
In particular, one can think of $\DeltahI$ as being a `discrete solution operator', i.e., a discrete counterpart to $\sdsol.$

%% Similarly, for $v \in \LtD,$ if $v = \sum_{m=1}^\infty \am \phim,$ then for $j \in \RR$ define
%% \beq\label{eq:deltaaseries}
%% \DeltaA^j v = \sum_{m=1}^\infty \lambdam^j \am \phim,
%% \eeq
%% if this series exists in $\LtD.$


%We let $\Domain{\DeltaA^j}$ denote the subset of $\LtD$ on which $\DeltaA^j$ is defined.

%% \bre[Negative powers of $\DeltaA$]
%% Observe that for \emph{every} $j \leq 0$, $\DeltaA^j v$ is defined for \emph{any} $v \in \LtD$ (i.e., $\Domain{\DeltaA^{j}} = \LtD$ for $j \leq 0$). For $\lambdam \geq 1,$ $\lambdam^j < \lambdam$, and only finitely many $\lambdam$ are in the interval $(0,1)$; therefore the series \cref{eq:deltaaseries} can be decomposed as a finite sum (for $\lambdam < 1$) and a convergent series (for $\lambdam \geq 1$).
%% \ere

%% \bre[Consistent Notation]
%% Observe that the notation $\DeltaA^j$ is consistent, i.e., $\DeltaA^0 v = v$, for $j \in \NN,$ $\DeltaA^j$ is equal to the $j$-fold application of $\DeltaA,$ and $\DeltaA^{-1}$ is the inverse of $\DeltaA.$\optodo{Maybe just double-check this is watertight.}\optodo{This needs work - need $A$ to be smooth enough to define proper higher-order derivatives.}
%% \ere

We can use the higher-order derivative operators to define discrete higher-order norms:

\bde[Discrete higher-order norm]
For $\vh \in \Vhp$ and $m \in \RR$, if define
\beqs
\Nshn{\vh} = \NLtDn{\Deltah^{s/2} v}.
\eeqs
\ede

%% \bde[$A$-weighted higher-order continuous norm]
%% For $v \in \LtD$ and $m \in \RR$, if $\DeltaA^{m/2} v$ exists, define
%% \beqs
%% \NmA{v} = \NLtD{\DeltaA^{m/2} v}.
%% \eeqs
%% \ede

%% In order to prove a relationship between the $A$-weighted higher order norm and the standard $H^m$ norms, we first must prove the following \namecref{lem:normrelationshiptech}\optodo{THIS NEEDS PROVING AND I DON'T KNOW HOW.}
%% \ble[Relationship between $\DeltaA^m$ and standard higher-order derivatives]\label{lem:normrelationshiptech}
%% For $m \in \NN,$ $\Domain{\DeltaA^{m/2}} \subseteq \HmD,$ and there exists a constant $\Cma > 0$ such that for all $v \in \Domain{\DeltaA^{m}}$
%% \beqs
%% \NHmD{v} \leq \Cma \NLtD{\DeltaA^{m/2}v}.
%% \eeqs\optodo{The latter bit will probably use something shift-theorem-like, but you need to be careful because powers of the differential operator aren't defined standardly.}
%% \ele


%% \ble[Relationship between $A$-weighted and standard higher-order continuous norms]\label{lem:normrelationship}
%% For all $m \in \NNz,$ for all $v \in \HmmD,$
%% \beqs
%% \NHmmD{v} \geq \frac1{\Cma} \NmmA{v}.
%% \eeqs
%% \ele

%% \bpf[Proof of \cref{lem:normrelationship}]
%% For $v \in \HmmD,$ we have
%% \begin{align*}
%% \NHmmD{v} &= \sup_{w \in \HmD} \frac{\IPLtD{v}{w}}{\NHmD{w}}\\
%% &\geq \frac1{\Cma} \sup_{w \in \Domain{\DeltaA^{m/2}}} \frac{\IPLtD{v}{w}}{\NmA{w}} \text{ by \cref{lem:normrelationshiptech}}\\
%% & = \frac1{\Cma} \sup_{w \in \Domain{\DeltaA^{m/2}}} \frac{\IPLtD{\DeltaA^{-m/2}v}{\DeltaA^{m/2}w}}{\NmA{w}} \text{ by \cref{lem:intoip}}\\
%% &= \frac1{\Cma} \NmmA{v} \text{ the supremum is achieved when } \IPLtD{\DeltaA^{-m/2}v,\DeltaA^{m/2}w} = \NLtD{\DeltaA^{-m/2}v}\NLtD{\DeltaA^{m/2}w}
%% \end{align*}
%% \epf

\optodo{Chat about lemma}

\ble[Introduction of derivatives into inner product]\label{lem:intoip}
%% For $m \in \RR,$ $v \in \LtD$, and $w \in \LtD\cap\Domain{\DeltaA^{m/2}}$ we have
%% \beqs
%% \IPLtD{\DeltaAmmt v}{\DeltaAmt w} = \IPLtD{v}{w}.
%% \eeqs
%% Similarly, f
For $\vh, \wh \in \Vhp,$ and $s \in \RR$ we have
\beq\label{eq:feipsplit}
\IPLtDn{\Deltahmst \vh}{\Deltahst \wh} = \IPLtDn{\vh}{\wh}
\eeq
and
\beq\label{eq:feiptrans}
\IPLtDn{\Deltahst \vh}{\Deltahst \vh} =  \IPLtDn{\Deltah^s \vh}{\vh}.
\eeq
\ele
\bpf[Proof of \cref{lem:intoip}]
We only prove \cref{eq:feipsplit}, as the proof of \cref{eq:feiptrans} is analagous. Since $\vh,\wh \in \Vhp,$ there exist sequences $(\aj)_{j =1,\ldots,\dimVhp}$ and $(\bsl)_{l =1,\ldots,\dimVhp}$ such that $\vh = \sum_{j=1}^{\dimVhp} \aj\phij$ and $\wh = \sum_{l=1}^{\dimVhp} \bsl \phil.$ Then we have
\begin{align*}
\IPLtDn{\Deltahmst \vh}{\Deltahst \wh} &= \int_D n\mleft(\sum_{j=1}^{\dimVhp}\lambdaj^{-m/2} \aj\phij\mright)\overline{\mleft(\sum_{l=1}^{\dimVhp} \lambdal^{m/2}\bsl \phil\mright)}\\
&= \sum_{j,l=1}^{\dimVhp} \lambdaj^{-m/2} \lambdal^{m/2} \aj \bsl \int_D n \phij \philbar \text{ as the } \lambdaj \text{ are real}\\
& =\sum_{j}^{\dimVhp} \aj \bsj \int_D n\abs{\phij}^2 \text{ as the } \phij \text{ are orthonormal}\\
&= \IPLtDn{\vh}{\wh}
\end{align*}
by repeating the above process in reverse, without the factors $\lambdaj^{-m/2}$ and $\lambdal^{m/2}$.
\epf

The next \namecref{cor:ipdiscbound} follows from \cref{lem:intoip} and the Cauchy--Schwarz inequality.

\bco[Inner product bounded by discrete norms]\label{cor:ipdiscbound}
If $vh, \wh \in \Vhp,$ then for all $j \in \RR$
\beqs
\IPLtDn{\vh}{\wh} \leq \Njh{\vh}\Nmjh{\wh}.
\eeqs
\eco

We recall the standard inverse inequality for finite-element functions, so that we can prove an analagous inverse inequality for discrete norms.

\ble[Standard inverse inequality]\label{lem:inverseinequality}
There exists $\CinvVhp > 0$ such that for all $\vh \in \Vhp$
\beqs
\NHoD{\vh} \leq \CinvVhp h^{-1} \NLtD{\vh}.
\eeqs
\ele



\ble[Inverse inequality for discrete norms]\label{lem:inversediscrete}
For all $j \in \ZZ$, for all $\vh \in \Vhp$
\beqs
\Njhn{\vh} \leq \Chinv \frac{\NLiD{n}}{\nmin} h^{-1} \Njmohn{\vh},
\eeqs
where
\beqs
\Chinv = \CinvVhp \NLiDop{A}^{1/2}.
\eeqs
\ele

\bpf[Proof of \cref{lem:inversediscrete}]
by the definition of $\Njhn{\cdot}$, and the fact that $\Deltah^{j/2} = \Deltah^{1/2}\Deltah^{(j-1)/2},$ it suffices to prove the result for $j=1$, as one can then perform induction on $j$. We have
\begin{align*}
\Nfn{1,h,n}{\vh}^2 &= \IPLtDn{\Deltahh \vh}{\Deltahh \vh}\\
&= \IPLtDn{\Deltah \vh}{\vh} \text{ by \cref{eq:feiptrans}}\\
&= \IPLtDn{A \grad \vh}{\grad \vh} \text{ by definition of } \Deltah\\
&\leq \NLiDop{A} \CinvVhp h^{-2}\frac{\NLiD{n}^2}{\nmin^2} \NLtDn{\vh}^2
\end{align*}
by the standard inverse estimate, and the result follows as $\Nzhn{\cdot} = \NLtDn{\cdot}$.
\epf
\ble[Relationship between standard and discrete $H^1$ norms]\label{lem:h1contdisc}
Let $\vh \in \Vhp$. Then
\beqs
\SNHoDn{\vh} \leq \Amin^{-\half} \Nohn{\vh}.
\eeqs
\ele

\bpf[Proof of \cref{lem:h1contdisc}]
We have, using \cref{eq:feiptrans},???? $\Nohn{\vh}^2 = \IPLtDn{\Deltahh \vh}{\Deltahh \vh} = \IPLtDn{\Deltah \vh}{\vh}= \IPLtDn{A \grad \vh}{\grad \vh} \geq \Amin \NLtDn{\grad \vh}^2$, and the result follows.
\epf



To prove \cref{lem:negdiscsum} on the relationship between discrete and continuous negative Sobolev normns, we require the following \namecref{lem:shiftnegativew} giving the shift theorem in negative weighted norms.

\tohere
\ble[Shift theorem in negative weighted norms]\label{lem:shiftnegativew}
Forand $\ftilde \in \LtD,$ if $m > d/2,$ $1/n \in \Hfn{}{m+2}{D}$, and $n \in \HmD,$ then
\beqs
????\NHnfn{-m}{D}{\sdsol\ftilde} \leq \Cshiftfn{m} \NHfn{m}{D}{n}\NHfn{m+2}{D}{\frac1n}\NHnfn{-(m+2)}{D}{\ftilde}.
\eeqs
\ele

\bpf[Proof of \cref{lem:shiftnegativew}]
Observe that by \cref{thm:banachalg}, if $v \in \HmD$ then $nv \in \HmD,$ and therefore by \cref{thm:shift} $\sdsol(nv) \in \HmptD.$ Hence, by \cref{thm:banachalg} again, $\sdsol(nv)/n \in \HmptD.$ We these facts in place we can compute
\begin{align*}
\NHnfn{-m}{D}{\sdsol \ftilde} &= \sup_{v \in \HmD} \frac{\IPLtD{\sdsol \ftilde}{nv}}{\NHmD{v}}\text{ by definition of the weighted negative norm}\\
& = \sup_{v \in \HmD} \frac{\IPLtD{ \ftilde}{\sdsol(nv)}}{\NHmD{v}}\text{ as } \sdsol \text{ is self-adjoint}\\
&= \sup_{v \in \HmD} \frac{\IPLtDn{\ftilde}{\frac{\sdsol(nv)}n}}{\NHmD{v}}\\
&\leq \CBanfn{m+2}\sup_{v \in \HmD} \frac{\NHfn{-(m+2)}{D}{\ftilde}\NHfn{m+2}{D}{\sdsol(nv)}\NHfn{m+2}{D}{\frac1n}}{\NHmD{v}} \text{ by \cref{thm:banachalg}}\\
&\leq \CBanfn{m+2}\CAfn{m}\sup_{v \in \HmD} \frac{\NHfn{-(m+2)}{D}{\ftilde}\NHfn{m}{D}{nv}\NHfn{m+2}{D}{\frac1n}}{\NHmD{v}} \text{ by \cref{thm:shift}}\\
\end{align*}
and by applying \cref{thm:banachalg} to the term $\NHfn{m}{D}{nv}$, the result follows.
\epf

%% \ble[Shift theorem in negative norms]\label{lem:shiftnegative}
%% For $m \in \NN,$ and $\ftilde \in \LtD,$ we have
%% \beqs
%% \NHmmD{\DeltaAI\ftilde} \leq \CAm \NHmmmtD{\ftilde}.
%% \eeqs
%% \ele

%% \bpf[Proof of \cref{lem:shiftnegative}]
%% Throughout the proof we let $\utilde$ denote $\DeltaAI\ftilde.$ As $\ftilde \in \LtD,$ it follows that $\utilde \in \HtD$ by \cref{thm:shift}, and so in particular $\utilde \in \HmmD.$ We will use the fact that $\sdsol$ is self-adjoint; this follows from the fact that the boundary-value problem \cref{eq:sdeq,eq:sddbc,eq:sdnbc} is self-adjoint\ednote{Both - do I need to show this? It's straightforward.}.% Note to self, have (Rv,w), where R is resolvent, for v in L^2, w in H^m. Have PR = Id (but not necessarily the other way round) so w = PRw. Then (Rv,w) = (Rv,PRw) = (PRv,Rw) (P self-adjoint) = (v,Rw). Denote differential operator (on the correct space) by P.
%% We can then compute
%% \begin{align*}
%% \NHmmD{\utilde} &=  \sup_{0 \neq \vtilde \in \HmD} \frac{\IPLtD{\ftilde}{\DeltaAI\vtilde}}{\NHmD{\vtilde}}\text{ as } \DeltaAI \text{ is self-adjoint}\\
%% &\leq \sup_{0 \neq \vtilde \in \HmD} \frac{\NHmmmtD{\ftilde}\CAm\NHmD{\vtilde}}{\NHmD{\vtilde}} \text{ by \cref{thm:shift}}\\
%% &= \CAm \NHmmmtD{\ftilde}
%% \end{align*}
%% as required. \epf

We can now prove the following \namecref{lem:negdiscsum} on the relationship between the negative-order discrete norms and the negative-order continuous norms.
\ble[Relationship between discrete and continuous negative-order norms]\label{lem:negdiscsum}
For any integer $j \in [0,p+1],$ there exists a constant $\Csumj > 0$ such that for all $\vh \in \Vhp,$
\beqs
\Nmjh{\vh} \leq \Csumj \sum_{m=0}^j h^{m} \NHfn{-(j-m)}{D}{\vh}.
\eeqs
\ele

\bpf[Proof of \cref{lem:negdiscsum}]
Let $\wh \in \Vhp,$ and define $\zh = \DeltahI \wh,$ $z = \DeltaAI \wh$ (observe $z$ is well-defined as $\Vhp \subseteq \LtD$). Then, for all $\yh \in \Vhp$, we have
\begin{align*}
\IPLtD{\wh}{\yh} &= \IPLtD{A \grad z}{\grad \yh} = \IPLtD{\grad z}{A\grad \yh} = \overline{\IPLtD{A\grad \yh}{\grad z}} \text{, and}\\
\IPLtD{\wh}{\yh} &= \IPLtD{A \grad \zh}{\grad \yh} = \IPLtD{\grad \zh}{A\grad \yh} = \overline{\IPLtD{A\grad \yh}{\grad \zh}}
\end{align*}
where the first equalities in each line follow from \cref{eq:deltaagreen,eq:discderdef}, respectively, and the second inequalities in each line follow from the fact that $A$ is symmetric. That is, for all $\yh \in \Vhp,$ $\IPLtD{A\grad \yh}{\grad z} = \IPLtD{A\grad \yh}{\grad \zh},$ i.e., $\zh = \Ph z.$

We now have, for $m \in [-1,p-1]$
\begin{align}
\NHmmD{\DeltahI \wh} &\leq \NHmmD{z} + \NHmmD{z-\zh}\nonumber\\
&\leq \NHmmD{z} + \Cmmo \Cinterpt \CAt h^{m+2} \NLtD{\wh}\nonumber\\
&\quad\quad\text{ by \cref{lem:ellprojerr,lem:scottzhang,thm:shift}, as }\zh = \Ph z\nonumber\\
&= \CAm \NHmmmtD{\wh} + \Cm h^{m+2} \NLtD{\wh}\label{eq:sumforrecursion}
\end{align}
by \cref{lem:shiftnegative}, where $\Cm \de \Cmmo \Cinterpt \CAt.$

From \cref{eq:sumforrecursion}, we can conclude that, for $l \in \NN$ and $\vh \in \Vhp$, writing $\wh = \Deltahmlpo \vh$,
\beq\label{eq:lrecursion}
\NHmmD{\Deltahml \vh} \leq \CAm \NHmmmtD{\Deltahmlpo \vh} + \Cm h^{m+2} \NLtD{\Deltahmlpo \vh}
\eeq
as $\Deltahml = \DeltahI \Deltahmlpo$. We now use \cref{eq:lrecursion} recursively to bound $\Nmjh{\vh}$:
If $j = 2l,$ then one can show inductively using \cref{eq:lrecursion} that for any integer $t \in [0,l]$ that
\beq\label{eq:evenrecursivesum}
\Nfn{-2l,h}{\vh} \leq \sum_{m=0}^l \Efn{m,l} h^{2m} \NHDfn{-2(l-m)}{\vh} ,
\eeq
where the $\Efn{m,t}$ are defined inductively by
\begin{align}
\label{eq:Emt1}\Efn{0,0} &=1\\
\label{eq:Emt2}\Efn{m,t} &= \CAfn{2(t-1-m)} \Efn{m,t-1} \tfor 0 \leq m \leq t-1 \quad\tand\\
\label{eq:Emt3}\Efn{t,t} &= \sum_{m=0}^{t-1} \Cfn{2(t-1-m)} \Efn{m,t-1}.
\end{align}

If $j=2l+1,$ then we first reduce $\Nfn{\vh}{-j,h}$ to a point analagous to the even case, and then proceed as before. Let $\wh$ and $\zh$ be as at the beginning of the proof, and let $z$ solve the variational formulation\footnote{We use the variational formulation here, as we will need to bound the $H^1$-norm of $z$ by the $H^{-1}$-norm of $\wh$, which is immediate from the Lax--Milgram theorem.}  of \cref{eq:sdeq,eq:sddbc,eq:sdnbc} (with $\ftilde = \wh$). Observe that we still have $\zh = \Ph z,$ and
\beq\label{eq:LMHmo}
\NHoD{z} \leq \frac{\NHmoD{\wh}}{\Amin}
\eeq
by the Lax--Milgram Theorem. We have
\begin{align}
\NLtD{\Deltah^{-1/2}\wh} &= \NLtD{\Deltah^{1/2} \zh}\nonumber\\
&= \IPLtD{\Deltah \zh}{\zh} \text{ by \cref{eq:feiptrans}}\\
&= \IPLtD{A \grad \zh}{\grad \zh}\nonumber\\
&= \NLiDop{A} \NHoD{\Ph z}\nonumber\text{ by the Cauchy--Schwartz inequality and the definition of } \zh\\
&\leq \NLiDop{A}\mleft(\NHoD{z} + \Cfn{1,1}\NHoD{0 - z}\mright) \text{ by \cref{lem:ellprojerr}}\nonumber\\
&\leq 2\frac{\Cfn{1,1}\NLiDop{A}}{\Amin}  \NHmoD{\wh}\text{ by \cref{eq:LMHmo}}\label{eq:deltahhalf}
\end{align}
We now return to $\Nmjh{\vh} = \NLtD{\Deltah^{-l-1/2} \vh} = \NLtD{\Deltah^{-1/2} \Deltah^{-l} \vh} \leq 2\Cfn{1,1}\NLiDop{A}\NHmoD{\Deltah^{-l} \vh}/\Amin$ by \cref{eq:deltahhalf}.

Similarly to \cref{eq:evenrecursivesum}, one can use \cref{eq:lrecursion} recursively to show that, for any integer $t \in [0,l]$
\beq\label{eq:oddrecursive}
\NHmoD{\Deltah^{-l}\vh} \leq \Etildefn{0,t} \NHDfn{-(2t+1)}{\Deltah^{-l+t}\vh} + \sum_{m=0}^t \Etildefn{m,t} h^{2m+1}  \NHDfn{-2(t-m)}{\Deltah^{-l+t}\vh},
\eeq
where  the $\Etildefn{m,t}$ are given, for $t \in [0,l]$ by
\begin{align}
\label{eq:Etilde1}\Etildefn{0,1} &= \CAfn{1}\\
\label{eq:Etilde2}\Etildefn{1,1} &= \Cfn{1}\\
\label{eq:Etilde3}\Etildefn{0,t} &= \Etildefn{0,t-1}\CAfn{2t-1}\\
\label{eq:Etilde4}\Etildefn{m,t} &= \Etildefn{m,t-1}\CAfn{2(t-1-m)}\tfor m = 1,\ldots,t-1\\
\label{eq:Etilde5}\Etildefn{t,t} &= \Cfn{2t-1} + \sum_{m=0}^{t-1}\Etildefn{m,t-1}\CAfn{2(t-1-m)}.\\
\end{align}
To show \cref{eq:oddrecursive,eq:Etilde1,eq:Etilde2,eq:Etilde3,eq:Etilde4,eq:Etilde5}, observe that \cref{eq:Etilde1,eq:Etilde2} follow immediately from \cref{eq:lrecursion}. We now show \cref{eq:oddrecursive,eq:Etilde3,eq:Etilde4,eq:Etilde5} by induction: suppose \cref{eq:oddrecursive} holds for $t-1,$ then
\begin{align*}
\NHmoD{\Deltah^{-l}} &\leq \Etildefn{0,t-1} \mleft(\CAfn{2t-1} \NHfn{-(2(t-1)+1)}{D}{\Deltah^{-l + t}\vh} +\Cfn{2t-1} h^{2t+1} \NLtD{\Deltah^{-l + t}\vh}\mright)\\
&\quad\quad+ \sum_{m=1}^{t-1} \Etildefn{m,t-1} h^{2m+1} \mleft(\CAfn{2(t-1-m)} \NHfn{-2(t-m)}{D}{\Deltah^{-l + t}\vh} + h^{2(t-m)} \Cfn{2(t-1-m)} \NLtD{\Deltah^{-l + t}\vh}\mright)\\
&= \Etildefn{0,t-1} \CAfn{2t-1} \NHfn{-(2(t-1)+1)}{D}{\Deltah^{-l + t}\vh}\\
&\quad\quad+ \mleft(\Cfn{2t-1}\sum_{m=1}^{t-1} \Etildefn{m,t-1}\CAfn{2(t-1-m)}\mright) h^{2t+1}\NLtD{\Deltah^{-l + t}\vh}\\
&\quad\quad+ \sum_{m=1}^{t-1} \Etildefn{m,t-1} h^{2m+1} \CAfn{2(t-1-m)}\NHfn{-2(t-m)}{D}{\Deltah^{-l + t}\vh},
\end{align*}
which is of the form \cref{eq:oddrecursive} with the constants $\Etildefn{m,t}$ given by \cref{eq:Etilde1,eq:Etilde2,eq:Etilde3,eq:Etilde4,eq:Etilde5}. Therefore, we conclude that if $j = 2l+1$
\beq\label{eq:oddfinal}
\Nfn{-j,h}{\vh} \leq \frac{2\Cfn{1,1}\NLiDop{A}}{\Amin}\mleft(\Etildefn{0,l} \NHDfn{-(2l+1)}{\vh} + \sum_{m=0}^l \Etildefn{m,l} h^{2m+1}  \NHDfn{-2(l-m)}{\vh}\mright),
\eeq
and combinining \cref{eq:evenrecursivesum,eq:oddfinal}, and letting $\Csumj$ be as in \cref{app:constants}, the result follows.%\optodo{Maybe put recursion in, but it's all in notes.}
\epf

\subsection{Main finite-element-error bound}\label{sec:fembound}

Having established the need results on discrete Sobolev spaces, we are now in a position to prove our main theorem, \cref{thm:fembound} below, which we do via a series of lemmas.

\bth[Higher-order error bound for the heterogeneous Helmholtz equation]\label{thm:fembound}
Let $u$ be the solution of \cref{prob:vtedp}. Under \cref{ass:highp,ass:htwo}, if $hk \leq 1,$ and if there exists $Ctildemin > 0$ independent of $k$ such that $\CAnk k \geq \Ctildemin,$ then there exist constants $\CFEMLt, \CFEMHo, \Chcond > 0$, independent of $h$ and $k$ such that if
\beq\label{eq:hfemcond}
h \leq \Chcond k^{-1-\frac1{2p}} \CAnk^{-\frac1{2p}},
\eeq
then the finite-element solution $\uh$ exists, is unique, and satisfies the error bounds
\beq\label{eq:femltbound}
\NLtD{u-\uh} \leq \CFEMLt \mleft(h \CAnk (hk)^{p}\mright)\inf_{\vh \in \Vhp} \NW{u-\vh}, \tand
\eeq
\beq
\NW{u-\uh} \leq \CFEMHo \mleft(1 + \CAnk k(kh)^p\mright) \inf_{\vh \in \Vhp} \NW{u-\vh},
\eeq\label{eq:femhobound}
where $\CFEMLt$, $\CFEMHo,$ and $\Chcond$ are given in \cref{app:constants}.
\enth

From \cref{thm:fembound,lem:scottzhang} we have the following corollary, with $\CcorLt = \CFEMLt\mleft(\CFEMo+\CFEMt\mright)$ and $\CcorHo = \CFEMHo\mleft(\CFEMotilde+\CFEMttilde\mright)$

\bco[Completely explicit FEM bound]\label{cor:fembound}
Under the assumptions on \cref{thm:fembound},
\beqs
\NLtD{u-\uh} \leq \CcorLt \mleft(h^3 + \CAnk h^2 (hk)^p + \CAnk^2 (hk)^{2p}\mright)\Cfg\tand
\eeqs
\beqs
\NW{u-\uh} \leq \CcorHo \mleft(h + \CAnk (hk)^p + \CAnk^2 (hk)^{2p}\mright)\Cfg.
\eeqs
\eco\optodo{I'm suspicious about two orders of magnitude...}


Two quantiies that are key in the proof are
\beqs
\rho \de u - \Ph u, \tand
\eeqs
\beqs
\thetah \de \Ph u - \uh.
\eeqs
The main idea of the proof is to decompose the error $u - \uh = \rho + \thetah,$ use bounds on the elliptic projection operator to bound $\rho,$ and the bound $\thetah$ (in higher-order discrete norms) in terms of $\rho.$

The following simple \namecref{lem:simpleform} sets up the argument in the lemmas that follow.
\optodo{Either just put the assumptions of the main theorem everywhere, or work through which ones are actually needed at each point. Not sure what to do....}
\ble[Expression for $a(\thetah,\vh)$]\label{lem:simpleform}
For any $\vh \in \Vhp,$
\beq\label{eq:thetaform}
a(\thetah,\vh) = k^2\IPLtDn{\Qhn\rho}{\vh} + ik \IPLtGI{\rho}{\vh}.
\eeq
\ele

\bpf[Proof of \cref{lem:simpleform}]
Let $\vh \in \Vhp.$ Then $a(\thetah,\vh) = a(u-\uh,\vh) - a(\rho,\vh) = -a(\rho,\vh)$ by Galerkin orthogonality. By definition of $a,$ we have $-a(\rho,\vh) = -\IPLtD{A\grad\mleft(u-\Ph u\mright)}{\vh} + k^2 \IPLtD{n\rho}{\vh} + ik\IPLtGI{\rho}{\vh}.$ By Galerkin orthogonality for the elliptic projection, $\IPLtD{A\grad\mleft(u-\Ph u\mright)}{\vh} = 0$, and so by the definition of the $n$-weighted $L^2$ inner product, and the $n$-weighted $L^2$-projection $\Qhn,$ the result follows.
\epf

\ble[Bound on $\NLtGI{\thetah}$ by $\Npmoh{\thetah}$]\label{lem:boundarybound}
\beq\label{eq:boundarybound}
\NLtGI{\thetah}^2 \leq \Cboundaryo \NLiD{n}^6 k^2 h^{2p-1} \Nfn{p-1,h}{\thetah}^2 + \Cboundaryt h \NW{\rho}^2,
\eeq
\ele

\bpf[Proof of \cref{lem:boundarybound}]
In \cref{eq:thetaform}, let $\vh = \thetah,$ and take the imaginary part to obtain
\beqs
-k \NLtGI{\thetah}^2 \leq \Im k^2 \IPLtDn{\Qhn \rho}{\thetah} + \Re k \IPLtGI{\rho}{\thetah},
\eeqs
and therefore by \cref{lem:intoip} and the Cauchy--Schwartz inequality
\begin{align}
\NLtGI{\thetah}^2 &\leq  k \Nfn{1-p,h,n}{\Qhn \rho}\Nfn{p-1,h,n}{\thetah} + \NLtGI{\rho}\NLtGI{\thetah}\nonumber\\
&\leq  k \NLiD{n}^2 \Nfn{1-p,h}{\Qhn \rho}\Nfn{p-1,h}{\thetah} + \NLtGI{\rho}\NLtGI{\thetah}.\label{eq:thetaboundarypart}
\end{align}
We first bound the negative norm $\Nfn{1-p,h}{\Qhn\rho}$, to do this we use \cref{lem:negdiscsum}; however, we therefore need to estimate negative Sobolev norms of $\Qhn\rho$; for integers $m \in [0,p-1]$ we have (observing that $\Qhn \Ph u = \Ph u$ as $\Ph u \in \Vhp.$)
\begin{align}
\NHmmD{\Qhn\rho} &\leq \NHmmD{\Qhn u - u} + \NHmmD{u - \Ph u}\nonumber\\
&\leq \CSZfn{m} \NLiD{n} h^m \NLtD{u-\Ph u} + \Cmmo h^{m+1} \NHoD{u - \Ph u}\\
&\quad\quad\text{ by \cref{lem:ellprojerr,lem:wltdprojerr}, taking } \wh = \Ph u \text{ in \cref{eq:ellprojerr,eq:wltdprojerr}}\nonumber\\
&\leq \mleft(\CSZfn{m} \NLiD{n}\Cfn{0,1} + \Cfn{-m,1}\mright) h^{m+1} \NW{\rho} \text{ by \cref{lem:ellprojerr}}\label{eq:Qhnrhoneg}
\end{align}
By \cref{lem:negdiscsum,eq:Qhnrhoneg} and the fact that $\NLiD{n} \geq 1,$ we have
\beq\label{eq:Qhnrhosum}
\Nfn{1-p,h}{\Qhn \rho}\leq \Csumfn{p-1} \sum_{m=0}^{p-1} h^{m} \NHfn{-(p-1-m)}{D}{\vh} \leq \Cmess \NLiD{n}h^p \NW{\rho}.
\eeq
To deal with the second term on the right-hand side of \cref{eq:thetaboundarypart} we use \cref{thm:multiplicativetrace,lem:ellprojerr}, taking $\wh = \Ph u$ in \cref{eq:ellprojerr} and the fact that $\NHoD{\cdot} \leq \NW{\cdot}$ to obtain
\beq\label{eq:rhomtbound}
\NLtGI{\rho} \leq \CMT\NHoD{\rho}^{1/2}\NLtD{\rho}^{1/2} \leq \CMT \Cfn{0,1}^{\half} h^\half \NHoD{\rho}\leq \CMT \Cfn{0,1}^{\half} h^{\half} \NW{\rho}.
\eeq
Therefore by \cref{eq:rhomtbound} and Young's inequality, we obtain
\beq\label{eq:rhothetamt}
\NLtGI{\rho}\NLtGI{\thetah} \leq \half \CMT^2 \Cfn{0,1} h \NW{\rho}^2 + \half \NLtGI{\thetah}^2.
\eeq
By combining \cref{eq:thetaboundarypart,eq:Qhnrhosum,eq:rhothetamt}
\beq\label{eq:thetahboundnear}
\NLtGI{\thetah}^2 \leq k \Cmess h^p \NLiD{n}^3\NW{\rho} \Nfn{p-1,h}{\thetah} + \half \CMT^2 \Cfn{0,1} h\NW{\rho}^2 + \half \NLtGI{\thetah}^2
\eeq
By using Young's inequality on the first term in \cref{eq:thetahboundnear}, and moving the $\NLtGI{\thetah}^2$ term onto the left-hand side, we obtain \cref{eq:boundarybound}.
\epf

\ble[Bound on higher-order discrete norms of $\thetah$ by $\NLtD{\thetah}$]\label{lem:higherbound}
For integer $m \in [1,p]$ there exist constants $\Chighmo,$ $\Chighmt > 0$ such that
\begin{align}
\Nmh{\thetah} &\leq \Chighmo \mleft(1+\NLiD{n}\mright)^m k^m \NLtD{\thetah}\nonumber\\
&\quad\quad+ \Chighmt h^{1-m}\mleft(1+\NLiD{n} + \frac{\NLiD{n}^3}{\nmin}\mright)\mleft(1+\NLiD{n}\mright)^{m-1} \NW{\rho}.\label{eq:chigh}
\end{align}
\ele

\bpf[Proof of \cref{lem:higherbound}]
By inserting the definition of $\Deltah$ in \cref{eq:thetaform} and rearranging, we have for any $\vh \in \Vhp$
\beqs
\IPLtD{\Deltah \thetah}{\vh} = k^2 \IPLtDn{\thetah}{\vh} + k^2\IPLtD{\Qhn \rho}{\vh} + ik \IPLtGI{\thetah}{\vh} + ik \IPLtGI{\rho}{\vh}.
\eeqs
Therefore, if we take $\vh = \Deltah^{m-1}\thetah,$, by \cref{lem:intoip} we have
\beq\label{eq:deltahm}
\Nmh{\thetah}^2 = k^2 \Nfn{m-1,h,n}{\thetah}^2 + k^2 \IPLtDn{\Deltah^{\frac{m-1}2} \Qhn \rho}{\Deltah^{\frac{m-1}2}\thetah} + ik\IPLtGI{\thetah}{\Deltah^{m-1} \thetah} + ik \IPLtGI{\rho}{\Deltah^{m-1} \thetah}.
\eeq
We now proceed to bound the two terms in \cref{eq:deltahm} defined on the truncation boundary $\GI.$ For the first term, we have
\begin{align}
\IPLtGI{\thetah}{\Deltah^{m-1}\thetah} &\leq \NLtGI{\thetah}\NLtGI{\Deltah^{m-1}\thetah}\nonumber\\
&\leq \CMT \CinvVhp^{1/2} \NLtGI{\thetah} h^{-\half} \NLtD{\Deltah^{m-1} \thetah}\\
&\quad\quad\text{ by \cref{thm:multiplicativetrace,lem:inverseinequality} and the definition of }\Nfn{2m-2,h}{\cdot}\nonumber\\
&= \CMT \CinvVhp^{1/2} h^{-\half}\NLtGI{\thetah}\Nfn{2m-2,h}{\thetah}\nonumber\\
&\leq \CMT \CinvVhp^{1/2} \Chinv^{m-1} h^{-m+\half} \NLtGI{\thetah}\Nfn{m-1,h}{\thetah} \text{ by \cref{lem:inversediscrete} applied } m-1 \text{ times}\nonumber\\
&\leq \CMT \CinvVhp^{1/2} \Chinv^{m-1}\nonumber\\
&\quad\quad\mleft(\Cboundaryo^{\half} \NLiD{n}^2 k h^{p-m} \Nfn{p-1,h}{\thetah} + \Cboundaryt^{\half} h^{1-m} \NW{\rho}\mright)\Nfn{m-1,h}{\thetah}\\
&\quad\quad\text{ by \cref{lem:boundarybound} and \cref{eq:simple}}\nonumber\\
&\leq \mleft(\CBo \NLiD{n}^2k \Nfn{m-1,h}{\thetah} + \CBt h^{1-m} \NW{\rho}\mright)\Nfn{m-1,h}{\thetah} \text{ by \cref{lem:inversediscrete},},\label{eq:firstboundary}.
\end{align}

To bound the second boundary term in \cref{eq:deltahm}, we have
\beq\label{eq:secondboundarytemp}
\IPLtGI{\rho}{\Deltah^{m-1} \thetah} \leq \CMT \CinvVhp^{1/2} \Chinv^{m-1}h^{-\half-m}\Nfn{m-1,h}{\thetah}\NLtGI{\rho}
\eeq
using \cref{thm:multiplicativetrace,lem:inverseinequality} as above. By \cref{thm:multiplicativetrace,lem:ellprojerr} (with $\wh = \Ph u$) we have
\beq\label{eq:secondboundarytemp2}
\NLtGI{\rho} \leq \CMT \Cfn{0,1}^{\half} h^{\half} \NW{\rho}.
\eeq
Inseting \cref{eq:secondboundarytemp2} into \cref{eq:secondboundarytemp} we obtain
\beq\label{eq:secondboundary}
\IPLtD{\rho}{\Deltah^{m-1}\thetah} \leq \CBth h^{1-m} \NW{\rho} \Nfn{m-1,h}{\thetah}.
\eeq

Therefore, from \cref{eq:deltahm,eq:firstboundary,eq:secondboundary} and the Cauchy--Schwartz inequality, we have
\begin{align*}
\Nfn{m,h}{\thetah}^2 &\leq k^2 \Nfn{m-1,h}{\thetah}^2 + k^2 \NLiD{n}^2 \Nfn{m-1,h}{\Qhn \rho}\Nfn{m-1,h}{\thetah}\nonumber\\
&\quad\quad+ k\mleft(\CBo \NLiD{n}^2 k\Nfn{m-1,h}{\thetah}+ \CBt h^{1-m} \NW{\rho}\mright)\Nfn{m-1,h}{\thetah}\nonumber\\
&\quad\quad+ k \CBth h^{1-m} \NW{\rho} \Nfn{m-1,h}{\thetah}%\label{eq:thetahighnearlynearly}.
\end{align*}
Therefore using Young's inequality and \cref{eq:simple}, we have
\begin{align}
\Nfn{m,h}{\thetah} &\leq\sqrt{1+\frac{\CBt^2}2 + \frac{\CBth^2}2 + \mleft(\half+\CBo\mright)\NLiD{n}^2}\Nfn{m-1,h}{\thetah}\\
&\quad\quad+ \frac{\NLiD{n}}{\sqrt{2}} k \Nfn{m-1,h}{\Qhn \rho} +  h^{1-m} \NW{\rho}.\label{eq:thetahighnearly}
\end{align}
We now proceed to bound $\Nfn{m-1,h}{\Qhn \rho}$: By \cref{lem:inversediscrete} and the fact that $\Qhn \Ph u = \Ph u$ (as $\Ph u \in \Vhp$), we have
\beq\label{eq:qhnmmo}
\Nfn{m-1,h}{\Qhn \rho} \leq \Chinv^{m-1} h^{1-m} \mleft(\NLtD{\Qhn u - u} + \NLtD{u-\Ph u}\mright).
\eeq
We can bound the first of these terms by the second:
\begin{align*}
\NLtD{\Qhn u - u} &\leq \frac{\NLtDn{\Qhn u - u}}{\nmin}\\
&\leq \frac{\CSZfn{0} \NLiD{n}}{\nmin} \NLtDn{u - \Ph u}\\
&\leq \frac{\CSZfn{0}\NLiD{n}^2}{\nmin} \NLtD{u-\Ph u},
\end{align*}
using \cref{lem:wltdprojerr} and properties of the $n$-weighted $L^2$ norm. We can also bound $\NLtD{u - \Ph u} \leq \Cfn{0,1} h \NW{\rho}.$ by \cref{lem:ellprojerr}. Therefore by \cref{eq:qhnmmo} we have (as $kh \leq 1$)
\beq\label{eq:qhnnearly}
k \Nfn{m-1,h}{\Qhn \rho} \leq \Chinv^{m-1} \mleft(\frac{\CSZfn{0}\NLiD{n}^2}{\nmin} + 1\mright) \Cfn{0,1} h^{1-m} \NW{\rho}.
\eeq
Therefore using \cref{eq:thetahighnearly,eq:qhnnearly,eq:simple} we have
\beq\label{eq:readytorecurse}
\Nfn{m,h}{\thetah} \leq \CReco \mleft(1+\NLiD{n}\mright)k \Nfn{m-1,h}{\thetah} + \CRect h^{1-m} \mleft(1+\NLiD{n} + \frac{\NLiD{n}^3}{\nmin}\mright)\NW{\rho}.
\eeq
Using \cref{eq:readytorecurse} and the fact that $hk \leq 1,$ we obtain \cref{eq:chigh}.
\epf

\ble[Bound $\NLtD{\thetah}$ by $\Nfn{p-1,h}{\thetah}$]\label{lem:ltthetahbound}
There exist constants $\Cfirst, \Csecond > 0$ such that
\beq\label{eq:ltthetahbound}
\NLtD{\thetah} \leq \Cfirst \mleft(\CFEMotilde h + \CFEMttilde \CAnk (hk)^p\mright) \NW{\rho} + \Csecond k^2h^p \mleft(\CFEMotilde h + \CFEMttilde \CAnk (hk)^p\mright) \Nfn{p-1,h}{\thetah}.
\eeq
\ele

\bpf[Proof of \cref{lem:ltthetahbound}]
The proof initially uses a standard duality technique, but then becomes more complex than standard proofs, as we are bounding $\thetah$ by its higher-order discrete norms, rather than the error $\eh$ by its $H^1$ norm.
Consider the adjoint variational problem: Find $w \in \HozDD$ such that for all $v \in \HozDD$
\beq\label{eq:adjointtheta}
a(v,w) = \IPLtD{v}{\thetah}
\eeq
(i.e., $w$ solves the adjoint problem with right-hand side given by $\thetah$). Let $\eh \de u -\uh$ be the finite-element error, and put $v = \eh$ in \cref{eq:adjointtheta}. By Galerkin orthogonality for $\eh,$ we have
\begin{align}
\IPLtD{\eh}{\thetah} = a(\eh,w-\Ph w) &= \IPLtD{A \grad \rho}{\grad(w-\Ph w)}  -k^2 \IPLtDn{\eh}{w-\Ph w} -ik \IPLtGI{\eh}{w-\Ph w},\nonumber\\
&=a(\rho,w-\Ph w) -k^2 \IPLtDn{\thetah}{w-\Ph w} -ik \IPLtGI{\thetah}{w-\Ph w}\label{eq:doublego}
\end{align}
using the fact that $\eh = \rho + \thetah$ and Galerkin orthogonality for $w - \Ph w.$ Therefore we can rearrange \cref{eq:doublego} and use Cauchy-Schwartz to obtain
\begin{align}
\NLtD{\thetah}^2 &\leq \Cc \NW{\rho} \NW{w- \Ph w} + k^2 \abs{\IPLtDn{\thetah}{w- \Ph w}}\nonumber\\
&+ k \abs{\IPLtGI{\thetah}{w-\Ph w}} + \NLtD{\rho}\NLtD{\thetah}\label{eq:boundingLtwo}
\end{align}
By combining \cref{lem:bestapprox,lem:ellprojerr}, we can show (as $w$ satisfies an adjoint Helmholtz problem with right-hand side $\thetah$)
\beq\label{eq:wlt}
\NLtD{w - \Ph w} \leq \Cfn{0,1} \mleft(\CFEMotilde \Pfn{p-2}\mleft(\NLiD{n}\mright)h^2 + \CFEMttilde \CAnk h(hk)^p\mright)\NLtD{\thetah} \tand
\eeq
\beq\label{eq:who}
\NHoD{w - \Ph w} \leq 2\Cfn{-1,1} \mleft(\CFEMotilde \Pfn{p-2}\mleft(\NLiD{n}\mright)h + \CFEMttilde \CAnk (hk)^p\mright)\NLtD{\thetah}.
\eeq
We will be able to use \cref{eq:wlt,eq:who} to bound terms involving $w - \Ph w$ in \cref{eq:boundingLtwo}. We first estimate the inner product terms in \cref{eq:boundingLtwo}:
\begin{align}
\abs{\IPLtDn{\thetah}{w- \Ph w}} &= \abs{\IPLtDn{\thetah}{\Qhn w - \Ph w}}\nonumber\\
&\leq \Nfn{p-1,h,n}{\thetah}\Nfn{1-p,h,n}{\Qhn w - \Ph w}\text{ by \cref{lem:intoip}}\nonumber\\
&\leq \Nfn{p-1,h,n}{\thetah} \Csumfn{p-1} \NLiD{n} \sum_{m=0}^{p-1} h^{m}\mleft(\NHfn{-(p-1-m)}{D}{\Qhn w - w} + \NHfn{-(p-1-m)}{D}{w - \Ph w}\mright) \text{ by \cref{lem:negdiscsum}}\nonumber\\
&\leq \Nfn{p-1,h}{\thetah} \Csumfn{p-1} \NLiD{n}^2 h^{p-1} \sum_{m=0}^{p-1} \mleft(\frac{\CSZfn{m} \NLiD{n}}{\nmin} \NLtD{w - \Ph w} + \Cfn{-m,1} h \NHoD{w - \Ph w}\mright)\\
&\quad\quad\text{ by \cref{lem:wltdprojerr,lem:ellprojerr}, taking $\wh = \Ph u$ in \cref{eq:wltdprojerr,eq:ellprojerr}}\nonumber\\
&\leq \Nfn{p-1,h}{\thetah} \Csumfn{p-1}\mleft(\sum_{m=0}^{p-1} \CSZfn{m} \NLiD{n}\Cfn{0,1} + \Cfn{-m,1}\Cfn{-1,1}\mright) \frac{\NLiD{n}^2}{\nmin}\\
&\quad\quad h^p \mleft(\CFEMotilde h + \CFEMttilde \CAnk (hk)^p\mright) \NLtD{\thetah} \text{ by \cref{eq:wlt,eq:who}}.\nonumber\\
&=\Nfn{p-1,h}{\thetah}  \Cfourteen \frac{\NLiD{n}^2}{\nmin} h^p \mleft(\CFEMotilde h + \CFEMttilde \CAnk (hk)^p\mright) \NLtD{\thetah} \label{eq:innerprod1},
\end{align}
where $\Cfourteen$ is defined in \cref{app:constants}. We now estimate the other inner product term
\begin{align}
\abs{\IPLtGI{\thetah}{w - \Ph w}} &\leq \mleft(\sqrt{2} \Cmess\NLiD{n}^2 k h^{p-\half} \Nfn{p-1,h}{\thetah} + \sqrt{\half + \half \CMT^2 \Cfn{0,1}} h^{\half} \NW{\rho}\mright) \NLtGI{w - \Ph w}\\
&\quad\quad\text{ by \cref{lem:boundarybound,eq:simple}}\nonumber\\
&\leq \CMT \Cfn{-1,}^{\half}\Cfn{-1,1}^{\half} \mleft(\sqrt{2} \Cmess \NLiD{n}^2 kh^p \Nfn{p-1,h}{\thetah} + \sqrt{\half + \half \CMT^2 \Cfn{0,1}} h \NW{\rho}\mright)\nonumber\\
&\quad\quad\mleft(\CFEMotilde h + \CFEMttilde \CAnk (hk)^p\mright) \NLtD{\thetah}\label{eq:innerprod2}
\end{align}
 by the \cref{thm:multiplicativetrace,eq:wlt,eq:who}.

Now insert \cref{eq:wlt,eq:who,eq:innerprod1,eq:innerprod2} into \cref{eq:boundingLtwo}:
\begin{align*}
\NLtD{\thetah}^2 &\leq \Big(\Cc \NW{\rho} \mleft(\Cfn{0,1} + \Cfn{1,1}\mright) + k^2 \Cfourteen \Nfn{p-1,h} h^p +\\
&k\mleft(\sqrt{2} \Cmess \NLiD{n}^2 kh^p \Nfn{p-1,h}{\thetah} + \sqrt{\half + \half\CMT^2 \Cfn{0,1}}h \NW{\rho}\mright)\CMT\Cfn{0,1}^{\half}\Cfn{-1,1}^{\half}\Big)\\
&\mleft(\CFEMotilde h + \CFEMttilde \CAnk (hk)^p\mright) \NLtD{\thetah}
\end{align*}
and therefore using Young's inequality to separate out the $\NLtD{\thetah}$ term on the right-hand side, and then move it to the left hand side, followed by \cref{eq:simple}, we obtain
\begin{align}\label{eq:part3penultimate}
\NLtD{\thetah} &\leq \mleft(\Cc \NW{\rho} \mleft(\Cfn{0,1} + \Cfn{1,1}\mright) + k^2 \Cfourteen \Nfn{p-1,h} h^p\mright.\\
&\quad\quad\mleft.+ k\mleft(\sqrt{2} \Cmess \NLiD{n}^2 kh^p \Nfn{p-1,h}{\thetah} + \sqrt{\half + \half\CMT^2 \Cfn{0,1}}h \NW{\rho}\mright)\CMT\Cfn{0,1}^{\half}\Cfn{-1,1}^{\half}\mright)\\& \mleft(\CFEMotilde h + \CFEMttilde \CAnk (hk)^p\mright).
\end{align}
As $hk \leq 1,$ we simplify \cref{eq:part3penultimate} to obtain \cref{eq:ltthetahbound}.
\epf

With all our technical lemmas proved, we can now prove our main \namecref{thm:fembound}.

\bpf[Proof of \cref{thm:fembound}]
By using \cref{lem:higherbound} (with $m=p-1$) in \cref{eq:ltthetahbound}, we have
\begin{align}\label{eq:doublehkdep}
\NLtD{\thetah} &\leq \mleft(\Cfirst+ \Csecond \mleft(\sum_{j=0}^{p-1} \CReco^j \CRect^j\mright)\mright) \mleft(\CFEMotilde h + \CFEMttilde \CAnk (hk)^p\mright)\NW{\rho}\nonumber\\
&\quad\quad+ \Csecond \CReco^{p-1} \NLtD{\thetah}\mleft(\CFEMotilde (hk)^{p+1} + \CFEMttilde \CAnk k(hk)^{2p}\mright).
\end{align}
Choosing $h$ according to \cref{eq:hfemcond}, \cref{eq:doublehkdep} simplifies to
\beqs
\NLtD{\thetah} \leq \mleft(\Cfirst+ \Csecond \mleft(\sum_{j=0}^{p-1} \CReco^j \CRect^j\mright)\mright) \mleft(\CFEMotilde h + \CFEMttilde \CAnk (hk)^p\mright)\NW{\rho} + \half \NLtD{\thetah},
\eeqs
and therefore it follows that
\beq\label{eq:ltboundwithrho}
\NLtD{\thetah} \leq \mleft(\CLtboundo h + \CLtboundt \CAnk (hk)^p\mright)\NW{\rho},
\eeq
where
\beqs
\CLtboundo = \mleft(\Cfirst+ \Csecond \mleft(\sum_{j=0}^{p-1} \CReco^j \CRect^j\mright)\mright) \CFEMotilde \tand
\eeqs
\beqs
\CLtboundt = \mleft(\Cfirst+ \Csecond \mleft(\sum_{j=0}^{p-1} \CReco^j \CRect^j\mright)\mright)\CFEMttilde.
\eeqs
It only now remains to bound the weighted $H^1$ norm of the error. By \cref{lem:h1contdisc,lem:higherbound} (with $m=1$) we have
\beq\label{eq:sntheta}
\SNHoD{\thetah} \leq \Amin^{\half} \mleft(\Chighfn{1,1} k \NLtD{\thetah} + \Chighfn{1,2} \NW{\rho}\mright),
\eeq
and by combining \cref{eq:sntheta,eq:ltboundwithrho} we obtain (as $hk \leq 1$)
\beqs
\SNHoD{\thetah} \leq \mleft(\CHoboundo + \CHoboundt \CAnk k(hk)^p\mright)\NW{\rho},
\eeqs
where
\beqs
\CHoboundo = \Amin^{\half} \mleft(\Chighfn{1,1}\CLtboundo + \Chighfn{1,2}\mright)\tand
\eeqs
\beqs
\CHoboundt = \Amin^{\half} \mleft(\Chighfn{1,1}\CLtboundt\mright).
\eeqs
Therefore, by \cref{lem:ellprojerr,lem:bestapprox}, and the fact that $hk \leq 1,$ we obtain \cref{eq:femltbound,eq:femhobound}, where
\beqs
\CFEMLt = \Cfn{-1,1} \max\set{\CLtboundo,\CLtboundt}\tand
\eeqs
\beqs
\CFEMHo = \Cfn{-1,1}\max\set{\CHoboundo,\CHoboundt}.
\eeqs
\epf


Put the following somewhere:
For $a,b > 0$\optodo{Extend to many}
\beq\label{eq:simple}
\sqrt{a^2 + b^2} \leq a + b.
\eeq

And this:
\bth[Multiplicative Trace Inequality]\label{thm:multiplicativetrace}%BS THm 1.6.6
There exists $\CMT > 0$ such that for all $v \in \HoD$
\beqs
\NLtdD{v} \leq \CMT \NLtD{v}^\half \NHoD{D}^\half.
\eeqs
\enth
