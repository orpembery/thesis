Recall that here, and in the rest of this thesis, $\OFP$ is a complete probability space.

\section{Recap of measure theory results}

We first recall some results from measure theory, with our main reference \cite{Bo:07}. Even though \cite{Bo:07} mainly considers maps with image $\RR,$ the results we quote for more general images are straightforward generalisations of the results in \cite{Bo:07}.

\begin{definition}[Measurable map]\label{def:meas}
If $(\M,\FM)$ and $(N,\FN)$ are measurable spaces, we say that $f:\M\rightarrow N$ is measurable (with respect to $(\FM,\FN)$) if $f^{-1}(E) \in \FM$ for all $E \in \FN.$
\end{definition}



\bde[Borel {$\sigma$}-algebra]\label{def:borelsigma}
If $(S,\TS)$ is a topological space, the \defn{Borel $\sigma$-algebra} $\Borel{S}$ on $S$ is the $\sigma$-algebra generated by $\TS.$
\ede


If $V$ is any topological space (including a Hilbert, Banach, metric, or normed vector space) then we will take always the Borel $\sigma$-algebra on $V$ unless stated otherwise.


\ble[Continuous maps are measurable { \cite[Theorem 2.1.2]{Bo:07}}] \label{lem:contmeas}
Any continuous function between two topological spaces is measurable.
\ele





\ble[The composition of a measurable and a continuous map is measurable {\cite[Text at top of p. 146]{Bo:07}}]\label{lem:contplusmeas}
Let $(\M,\FM)$ be a measurable space and let $\mleft(S,\TS\mright)$ and $\mleft(T,\TopT\mright)$ be topological spaces. Let $f:M \rightarrow S$ be measurable and let $h : S \rightarrow T$ be continuous. Then $h \circ f$ is measurable.
\ele

\bde[Product $\sigma$-algebra {\cite[Section IV.11]{Do:94}}]\label{def:prodsigma}
Let $\mleft(\M_1,\FM_1\mright),\ldots,\mleft(\M_m,\FM_m\mright)$ be measurable spaces. The \emph{product $\sigma$-algebra} $\M_1\otimes\cdots\otimes\M_m$ is defined as the $\sigma$-algebra generated by the set of measurable rectangles
$%\label{eq:measrectdef}
\set{R_{1} \times \cdots \times R_m  \st R_{1} \in \FM_1, \ldots, R_m \in \FM_m}.
$
\ede

\ble[Measurability of the Cartesian product of measurable functions]\label{lem:measprod}

Let $\mleft(\M_1,\FM_1\mright),\ldots,\mleft(\M_m,\FM_m\mright)$ be measurable spaces and $h_j:\Omega \rightarrow \M_j,\, j = 1,\ldots,m$ be measurable functions. Then the product map $\Prodf:\Omega \rightarrow \M_1 \times \cdots \times \M_m$ given by
%\beqs
$\Prodf(\omega) \de \mleft(h_1(\omega),\ldots,h_m(\omega)\mright)$
%\eeqs
is measurable with respect to $\mleft(\cF,\FM_1 \otimes \cdots\otimes\FM_m\mright).$
\ele

\bpf[Sketch proof of \cref{lem:measprod}]

Let $\measrectmany{\FM_1,\ldots,\FM_m}$ denote the set of measurable rectangles, as in \cref{def:prodsigma}. Let %Define the set $\cP$ by
%\beqs
$\cP \de \set{C \subseteq \M_1 \times \cdots \times \M_m \st \Prodf^{-1}\mleft(C\mright) \in \cF}.$
%\eeqs
The proof of the lemma consists of the following straightforward steps, whose proofs are omitted:
%\ben
%\item 
(i) Show 

\noindent $\measrectmany{\FM_1,\ldots,\FM_m} \subseteq \cP.$
%\item 
(ii) Show $\cP$ is a $\sigma$-algebra.
%\item 
(iii) Deduce $\FM_1 \otimes \cdots \otimes \FM_m \subseteq \cP$ (since $\FM_1 \otimes \cdots \otimes \FM_m$ is generated by measurable rectangles).
%\item 
(iv) Conclude $\Prodf$ is measurable with respect to $\mleft(\cF,\FM_1\otimes\cdots\otimes\FM_m\mright).$
%\een
\epf

\ble[Product of Borel $\sigma$-algebras is Borel $\sigma$-algebra of the product {\cite[Lemma 6.2.1 (i)]{Bo:07}}]\label{lem:bogachev}
Let $H_1,H_2$ be Hausdorff spaces and let $H_2$ have a countable base (e.g.~$H_2$ could be a separable metric space). Then $\Borel{H_1\times H_2} = \Borel{H_1} \otimes \Borel{H_2},$ where $\Borel{H_1\times H_2}$ is the Borel $\sigma$-algebra of the product topology on $H_1\times H_2.$
\ele

\section{Recap of results on Bochner spaces}

We now recap the theory of Bochner spaces, using \cite{DiUh:77} as our main reference. In what follows the space $V$ is always a Banach space.

\bde[Simple function]
A function $v:\Omega \rightarrow V$ is \defn{simple} if there exist $v_1,\ldots,v_m \in V$ and $E_1,\ldots,E_m \in \cF$ such that
%\beqs
$v = \sum_{i=1}^m v_i \chi_{E_{i}},$
%\eeqs
where $\chi_{E_{i}}$ is the indicator function on $E_{i}.$
\ede

\bde[Strongly measurable]\label{def:strongmeas}
A function $v:\Omega \rightarrow V$ is \defn{strongly measurable}
\footnote{In \cite{DiUh:77} the authors use the term \defn{$\mu$-measurable} instead of \defn{strongly measurable} (where $\mu$ is the measure on the domain of the functions under consideration).} 
if there exists a sequence of simple functions $(\vn)_{n \in \NN}$ such that
%\beqs
$\lim_{n \rightarrow \infty} \NV{\vn - v} = 0$, %\quad
$\PP$-almost everywhere.
%\eeqs
\ede

%Note that \cite{DiUh:77} uses the term \defn{$\mu$-measurable} instead of \defn{strongly measurable} (where $\mu$ is the measure on the domain of the functions under consideration).


\bde[Bochner integrable {\cite[p. 49]{DiUh:77}}]
A strongly measurable function $v:\Omega \rightarrow V$ is called \defn{Bochner integrable} if there exists a sequence of simple functions $(\vn)_{n \in \NN}$ such that
%\beqs
$\lim_{n \rightarrow \infty} \int_{\Omega} \NV{\vn(\omega) - v(\omega)} \dd\PP(\omega) = 0.$
%\eeqs
\ede

\bth[Condition for Bochner integrability {\cite[Theorem II.2.2]{DiUh:77}}]\label{thm:bochnercond}
A strongly measurable function $v:\Omega \rightarrow V$ is Bochner integrable if and only if $\int_\Omega \NV{v} \dd\PP < \infty.$
\enth

\bco[Sufficient condition for Bochner integrability]\label{cor:bochnersimple}
Let $p \geq 1.$ If a strongly measurable function $v:\Omega \rightarrow V$ has $\int_\Omega \NV{v}^p \dd\PP < \infty,$ then $v$ is Bochner integrable.
\eco



\bde[Bochner norm]\label{def:bochnernorm}
For a Bochner integrable function $v:\Omega\rightarrow V,$ let
\beqs
\NLpOV{v} \de \mleft(\int_\Omega \NV{v(\omega)}^p \dd\PP(\omega)\mright)^{1/p}, \,1 \leq p < \infty, \,\,\text{and}\,\,\NLiOV{v} \de \esssup_{\omega \in \Omega} \NV{v(\omega)}.
\eeqs
\ede

\bde[Bochner space]\label{def:bochnerspace}
Let $1\leq p \leq \infty.$ Then
\beqs
\LpOV \de \set{v:\Omega\rightarrow V \st v \text{ is Bochner integrable,}\,\NLpOV{v} < \infty}.
\eeqs
\ede

\bde[Complete probability space]
A probability space $\OFP$ is complete if for every $\Eo \in \cF$ with $\PP(\Eo)=0,$ the inclusion $\Et \subseteq \Eo$ implies that $\Et \in \cF.$
\ede

\bde[Separable space]
A topological space is \defn{separable} if it contains a countable, dense subset.
\ede

\bde[$\sigma$-finite]
A probability space $\OFP$ is \defn{$\sigma$-finite} if there exist $E_1,E_2,\ldots \in \cF$ with $\PP(E_m) < \infty$ for all $m \in \NN$ such that $\Omega = \cup_{m=1}^\infty E_m.$
\ede

\bth[Pettis measurability theorem {\cite[Proposition 2.15]{Ry:02}}]\label{thm:pettis}
Let $\OFP$ be a complete $\sigma$-finite measure space. The following are equivalent for a function $v:\Omega \rightarrow V$:
%\ben
%\item 
(i) $v$ is  strongly measurable,
%\item 
(ii) $v$ is measurable and $\PP$-essentially separably valued.
%\een
\enth

\bco[Equivalence of measurable and strongly measurable when the image is separable]\label{cor:pettis}
Let $\OFP$ be a $\sigma$-finite measure space. If $V$ is a separable Banach space, then a function $v:\Omega\rightarrow V$ is strongly measurable if, and only if, it is measurable.
\eco

\ble[The composition of a continuous map and a {$\PP$}-essentially separably valued map]\label{lem:esssep}
Let $\mleft(S,\TS\mright)$ and $\mleft(T,\TopT\mright)$ be topological spaces. If $f_1:\Omega \rightarrow S$ and $f_2:S\rightarrow T$ are such that $f_1$ is $\PP$-essentially separably valued and $f_2$ is continuous, then $f_2\circ f_1$ is $\PP$-essentially separably valued.
\ele

\bpf[Proof of \cref{lem:esssep}]
As $f_1$ is $\PP$-essentially separably valued, there exists $E \in \cF$ such that $\PP(E) = 1$ and $f_1(E) \subseteq G \subseteq S,$ where $G$ is separable. As $\ft$ is continuous, $\ft(G)$ is separable \cite[Theorem 16.4(a)]{Wi:70}. Therefore, since $\mleft(\ft \circ\fo\mright)(E) \subseteq \ft(G),$ it follows that $\ft\circ\fo$ is $\PP$-essentially separably valued.
\epf



\ble[The composition of a continuous map and a strongly measurable map]\label{lem:contplusstrong}
If $\Bo$ and $\Bt$ are Banach spaces and there exist $f_1:\Omega \rightarrow \Bo$ and $f_2:\Bo\rightarrow \Bt$ such that $f_1$ is strongly measurable and $f_2$ is continuous, then $f_2\circ f_1$ is strongly measurable.
\ele

\bpf[Proof of \cref{lem:contplusstrong}]
By \cref{thm:pettis}, $\fo$ is both measurable and $\PP$-essentially separably valued. Therefore we can apply \cref{lem:contplusmeas,lem:esssep} to conclude $\ft \circ \fo$ is both measurable and $\PP$-essentially separably valued. Hence by \cref{thm:pettis} $\ft\circ\fo$ is strongly measurable.
\epf

\ble[Zero in all integrals implies zero almost everywhere {\cite[Corollary II.2.5]{DiUh:77}}]\label{lem:gotoae}
If $\alpha$ is  Bochner integrable and  $\int_E \alpha(\omega) \dd\PP(\omega) = 0 $ for each $E \in \cF$ then $\alpha=0$ $\PP$-almost everywhere.
\ele

\ble[Cartesian product of $\PP$-essentially separably valued maps]\label{lem:prodsep}
Let 

\noindent $\mleft(\cCo,\Top{\cCo}\mright),\ldots,\mleft(\cCm,\Top{\cCm}\mright)$ be topological spaces, and let $s_j:\Omega\rightarrow\cCj,\,j=1,\ldots,m$ be $\PP$-essentially separably valued. Define $\cC \de \cCo \times \cdots \times \cCm$ and equip $\cC$ with the product topology. Then the map $f:\Omega\rightarrow \cC$ given by $s(\omega) \de \mleft(s_1(\omega),\ldots,s_m(\omega)\mright) $ is $\PP$-essentially separably valued.
\ele

The proof of \cref{lem:prodsep} is straightforward and omitted.
