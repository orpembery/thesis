\section{Introduction}
This chapter has two main foci:
%    \setlist[enumerate]{restart} ????
\optodo{Sort numbering}
\ben
\item Recapping theory for the deterministic Helmholtz equation in heterogeneous media, focussing especially on well-posedness and a priori bounds on the solution, and
\item Recapping and extending theory for the finite-element method for the deterministic Helmholtz equation, especially error bounds.
  \een

  In this chapter we will first provide an overview of well-posedness results for the deterministic Helmholtz equation, and a priori bounds on its solution.\optodo{Put somewhere below that Fredholm gives a bound without dependence on $k$} Such results are more complicated than analagous results for the simpler stationary diffusion equation
  \beqs
\grad \cdot \mleft( A \grad u \mright) = -f,
\eeqs
as results for the Helmholtz equation depend on whether the medium described by the coefficients $A$ and $n$ is `trapping' or `nontrapping' (these terms will be discussed in more precision below). We will then move on to discussing the finite-element approximation of \eqref{eq:introdet}, focussing especially on error bounds and quasi-optimality, and the $k$-dependent mesh conditions under which such properties can be proven.

In \cref{sec:varform} we will define the deterministic Helmholtz problems we will study in this chapter (these are the deterministic analogues of the stochastic Helmholtz problems we will study), prove a straightforward lemma on the regularity of the solution of these problems, and show that these problems satisfy a so-called G\r{a}rding inequality. In \cref{sec:wpbounds} we will then state in some detail the well-posedness results and a priori bounds from \cite{GrPeSp:19}, these results will be crucial for our analysis of stochastic Helmholtz problems in \cref{chap:stochastic}. In \cref{sec:wpdisc} we will review recent and historical research efforts in this area. We then move on to the finite-element method for \eqref{eq:introdet}; in \cref{sec:fetheory} we recap basic concepts of the finite-element method, before proving new results on error bounds for the finite-element method for \eqref{eq:introdet} in \cref{sec:errbound}. We then give an overview of the literatureon error bounds and quasi-optimality for \eqref{eq:introdet} in \cref{sec:helmfedisc}.


\section{PDE Theory of the Deterministic Helmholtz Equation}
  

  We will begin by defining the two deterministic Helmholtz problems that we consider in this thesis (and the stochastic analogues of which we also consider). We first state these problems in `strong form' (that is, where derivatives are understood as distributional derivatives), in \cref{sec:varform}, when considering finite-element approximations of the problems, we will consider the variational formulations of these problems.\optodo{Somewhere explain convention on function spaces, and define matrix space} In this section we largely follow the presentation in \cite{GrPeSp:19}

  \bprob[Exterior Dirichlet Problem]\label{prob:edp}
  Let $\Dm$ be a bounded Lipschitz\optodo{Needed for statement?} open set such that the open complement $\Dp \de \RRd \setminus \Dmclos$ is connected. Let $\GD \de \partial \Dm.$ Given
  \bit
  \item $k > 0,$
\item $f \in \LtDp$ with compact support,
\item $\gD \in \HhGD,$
\item $n \in \LiDpRR$ such that $1-n$ has compact support and there exist $0 < \nmin < \nmax < \infty$ such that
  \beqs
\nmin \leq n(\bx) < \nmax \tfae \bx \in \Dp,
  \eeqs
\item $A \in \LiDpRRdtd$ such that $I-A$ has compact support, $A$ is symmetric, and there exist $0 < \Amin < \Amax < \infty$ such that
  \beqs
\Amin \abs{\bxi}^2 \leq \mleft(A(\bx) \bxi \mright) \cdot \bxibar < \Amax \abs{\bxi}^2 \tfa \bxi \in \CCd \tfae \bx \in \Dp,
  \eeqs
  \eit
  we say $u \in \HolocDp$ satisfies the \defn{exterior Dirichlet problem} if
  \beqs
\grad \cdot \mleft(A \grad u\mright) + k^2 n u = -f \tin \Dp,
\eeqs
\beq\label{eq:dbc}
\trGD u = \gD,
\eeq
and $u$ satisfies the Sommerfeld radiation condition\optodo{Make sure you understand this}
\beq\label{eq:sommerfeld}
\dudr(\bx) - iku(\bx) = o\mleft(\frac1{r^{(d-1)/2}}\mright)
\eeq
as $r \de \abs{\bx} \rightarrow \infty,$ uniformly in $\bxhat \de \bx/\abs{\bx}.$
\eprob

To interpret \cref{prob:edp} physically, we can think of $u$ as being the acoustic pressure field caused by the scattering of an incoming wave $\ui$ by the scatterer $\Dm.$\footnote{In the literature the scattered field is sometimes denotes $us,$ and $u$ denotes the total field $\ui+ \us.$} The Dirichlet boundary condition \eqref{eq:dbc} means $\Dm$ corresponds to a sound-soft scatterer, that is, one on which the total field $u + \ui$ vanishes, with $\gD = - \trGD \ui.$ The function $f$ represents other pressume sources in the domain, although if one is solely interested in scattering, $f=0.$

The Sommerfeld radiation condition \eqref{eq:sommerfeld} ensures that the solutions of \cref{prob:edp} correspond to physically `outgoing' waves\optodo{Find reference}, and also guarantees the uniqueness of solution to \cref{prob:edp}\optodo{Ref for this}. However, \cref{prob:edp} is posed on an infinite domain, which cannot be fully represented when discretising with the FEM. Therefore, one must truncated the infinite domain $\Dp$ and impose an artificial boundary condition on the external boundary of the truncated domain. Options for the truncated boundary condition include a perfectly matched layer which\optodo{check} mimics the whole of the external domain, or so-called FEM-BEM coupling\optodo{Check}, where a boundary element method is used to approximate the solution in the exterior of the truncated domain. However, in this thesis, we will use a simpler approach, imposing an \defn{impedance boundary condition}
\beq\label{eq:imptext}
\dn u - iku = \gI
\eeq
on the truncated boundary. If $\gI = 0,$ then \eqref{eq:imptext} can be seen as a first-order approximation to \eqref{eq:sommerfeld}. Truncating with an impedance boundary condition gives rise to the following alternative deterministic Helmholtz problem\optodo{Make a picture explaining EDP and TEDP on one pic}
\bprob[Truncated Exterior Dirichlet Problem]\label{prob:tedp}
 Let $\Dm$ be a bounded Lipschitz\optodo{Needed for statement?} open set such that the open complement $\Dp \de \RRd \setminus \Dmclos$ is connected. Let $\Dtilde$ be a bounded connected Lipschitz open set such that $\Dmclos \compcont \Dtilde.$ Let $D \de \Dtilde \setminus \Dmclos$, $\GD \de \partial \Dm,$ and $\GI \de \partial \Dtilde.$ Given
  \bit
  \item $k > 0,$
\item $f \in \LtD$
\item $\gD \in \HhGD,$
  \item $\gI \in \LtGI$\optodo{Understand why these are the function space requirements}
\item $n \in \LiDRR$ such that $\supp\mleft(1-n\mright) \compcont D$ and there exist $0 < \nmin < \nmax < \infty$ such that
  \beqs
\nmin \leq n(\bx) < \nmax \tfae \bx \in D,
  \eeqs
\item $A \in \LiDRRdtd$ such that $\supp\mleft(I-A\mright)\compcont D$,\ednote{Euan---in hetero, this requirement is instead phrased `$\dist(\supp(I-A),\GI) > 0'$. Is there any reason for phrasing it that way?} $A$ is symmetric, and there exist $0 < \Amin < \Amax < \infty$ such that
  \beqs
\Amin \abs{\bxi}^2 \leq \mleft(A(\bx) \bxi \mright) \cdot \bxibar < \Amax \abs{\bxi}^2 \tfa \bxi \in \CCd \tfae \bx \in D,
  \eeqs
  \eit
  we say $u \in \HoD$ satisfies the \defn{truncated exterior Dirichlet problem} if
  \beqs
\grad \cdot \mleft(A \grad u\mright) + k^2 n u = -f \tin D,
\eeqs
\beqs
\trGD u = \gD, \tand
\eeqs
\beq\label{eq:ibc}
\trGI \dn u - ik\trGI u = \gI.
\eeq
\eprob
Observe that, by construction, $\partial D = \GI \cup \GI$ and $\GD \cap \GI = \emptyset.$

Whilst the impedance boundary condition \eqref{eq:ibc} does not exactly mimic the Sommerfeld radiation condition \eqref{eq:sommerfeld}, the solutions of \cref{prob:tedp} are still `wave-like', and we will see below that solutions of \cref{prob:tedp} posses many of the same properties as solutions of \cref{prob:edp}. We also note that a common Helmholtz model problem in the numerical-analysis community is the \defn{interior impedance problem}, which is simply \cref{prob:tedp} in the case $\Dm = \emptyset.$

In order to approximation \cref{prob:edp,prob:tedp} by the finite-element method, we must instead work with the following 'variational forms' or 'weak forms' of \cref{prob:edp,prob:tedp}. For simplicity of exposition, we state both variational forms in the case $\gD = 0,$ although these can be generalised to the case $\gD \neq 0.$
\optodo{Insert ball notation somewhere}
\optodo{Define DtN}
\optodo{Define duality pairing}

\optodo{Fix cleverref referring to Problems as Theorems}


  \subsection{Well-posedness and bounds}\label{sec:wpbounds}

  We will now recap the well-posedness results and a priori bounds for \cref{prob:edp,prob:tedp} from \cite{GrPeSp:19}; these results will, in particular, be crucial for proving well-posedness results and a priori bounds for the stochastic analogues of \cref{prob:edp,prob:tedp} in \cref{chap:stochastic}. The novelty of these results is that they hold independently of $k,$ and that the a priori bounds we prove are explicit in $A$ and $n$. The explicitness in $A$ and $n$ is necessary in order to prove a priori bounds for stochastic $A$ and $n$, and we prove the results under conditions on $A$ and $n$ that are, in some sense `nontrapping'. Informally, a medium is `nontrapping' if all rays travelling through the medium escape in a uniform time; this definition, and the sense in which our conditions are `nontrapping', is discussed in \cref{sec:wpdisc} below.

  We first define the classes of $A$ and $n$ for which we will prove well-posedness results and a priori bounds.% Since we will need to consider the classes on both finite and infinite domains (for \cref{prob:tedp,prob:edp} respectively), we first define the classes for finite domains, before using this definition to define the classes for infinite domains.

  \optodo{Define star-shaped w.r.t. a point, ball}

%%   \bde[Class of nontrapping media on a finite domain]\label{def:NTfinite}
%% Let $\Dm$ be star-shaped with respect to the origin, and let $D$ be as in \cref{prob:tedp}. Let $A \in \CzoDclosRRdtd, n \in \CzoDclosRR$ and $\muo, \mut > 0.$ We say that 
%%   \ede
  
\bde[Class of nontrapping media]\label{def:NT}
Let $A \in \CzoDpclosRRdtd, n \in \CzoDpclosRR,$ and $\muo, \mut > 0.$ We say that $A \in \NTADp{\muo}$ if
\beqs
A(\bx) - \mleft(\bx \cdot \grad\mright)A(\bx) \geq \muo
\eeqs
in the sense of quadratic forms for almost every $\bx \in \Dp$. We say that $n \in \NTnDp{\mut}$ if
\beqs
n(\bx) + \bx \cdot \grad n(\bx) \geq \mut
\eeqs
for almost every $\bx \in \Dp.$

If $D$ is as in \cref{prob:tedp}, then we define $\NTAD{\muo}$ and $\NTnD{\mut}$ analagously.
\ede
\ednote{Both---Can you think of better notation for these conditions? I'm not overly keen to write $\mathrm{NT}_{\mathrm{A}}$ etc., as there is potentially confusion between then function $A$ and the $\mathrm{A}$ is the subscript. Or am I being overcautious?}

We can now prove well-posedness results and a priori bounds for the Helmholtz equation in the class of heterogeneous media we have just defined.

\bth[Well-posedness and bound for the EDP]\label{thm:edp}
If $\Dm, A, n,$ and $f$ satisfy the requirements in \cref{prob:edp}, $\Dm$ is star-shaped with respect to the origin, there exists $\muo, \mut > 0$ such that $A \in \NTADp{\muo}$ and $n \in \NTnDp{\mut}$, and $\gD = 0,$ then the solution of \cref{prob:edp} exists and is unique. Furthermore, given $R>0$ such that $\supp\mleft(I-A\mright),$ $\supp(1-n),$ and $\supp f$ are compactly contained in $\DR,$ then
\beqs
\muo \NLtDR{u}^2 + \mut k^2 \NLtDR{\grad u}^2 \leq \Co \NLtDR{f}^2,
\eeqs
for all $k>0,$ where
\beqs
\Co \de 4\mleft(\frac{R^2}{\muo} +\frac1{\mut}\mleft(R + \frac{d-1}{2k}\mright)^2\mright).
\eeqs
\enth

For the proof of \cref{thm:edp}, see \cite[Theorem 2.5]{GrPeSp:19}.

One can prove an analagous result to \cref{thm:edp} for \cref{prob:tedp} using the same techniques as in the proof of \cref{thm:edp}. However, the statement of the theorem is slightly more complicated due to the presence of the impedance boundary $\GI$, and its effect on the solution.

\bth[Well-posedness and bound for the TEDP]\label{thm:tedp}
If $\Dm, A, n, f,$ and $\gI$ satisfy the requirements in \cref{prob:tedp}, $\Dm$ is star-shaped with respect to the origin, $\Dtilde$, is star-shaped with respect to a ball, there exists $\muo, \mut > 0$ such that $A \in \NTAD{\muo}$ and $n \in \NTnD{\mut}$, and $\gD = 0,$ then the solution of \cref{prob:tedp} exists and is unique. Let:
\bit
\item $\LI \de \max_{\bx \in \GI} \abs{\bx}$ and
\item $a\LI$ be the radius of the ball with respect to which $\Dtilde$ is star-shaped.
    \eit
Then
\begin{multline*}
  \muo \NLtD{u}^2 + \mut k^2 \NLtD{\grad u}^2 + a\LI\NLtGI{\gradGI \trGI u}^2 + 2\LI k^2 \NLtGI{\trGI u}^2\\
  \leq \Ct \NLtDR{f}^2 + \Cttilde \NLtGI{\gI}^2
\end{multline*}
\optodo{Check that hetero grad boundary notation isn't hiding anything}
for all $k>0,$ where $\gradGI$ is the surface gradient on $\GI,$
\beqs
\Ct \de 4\mleft(\frac{\LI^2}{\muo} + \frac1{\mut}\mleft(\beta + \frac{d-1}{2k}\mright)^2\mright),
\eeqs
\beqs
\Cttilde \de 2\mleft(2\mleft(1+\frac2a\mright) + \frac\beta{\LI} + \frac{\mleft(d-1\mright)^2}4\mright)\LI,
\eeqs
and
\beqs
\beta \de \LI \mleft(2+\frac1{\mleft(k\LI\mright)^2} + 2\mleft(1+\frac2a\mright)\mright).
\eeqs
\enth

Observe that the above results are stated only in the case that $\gD = 0$. Whilst there is no mathematical difficulty in proving analagous results in the case $\gD \neq 0,$ the calculations in this case are more involved, as one must consider the surface gradient on the Dirichlet boundary, and this surface gradient depends on $A.$ In the case $A=I,$ these calculations are significantly simplified, and so in the case $A=I$ and $\gD \neq 0$ analagous results to \cref{thm:edp,thm:tedp} are proved in \cite[Theorem 2.19(ii)]{GrPeSp:19} (for \cref{prob:edp}) and \cite[Theorem A.6(iv)]{GrPeSp:19} (for \cref{prob:tedp}).

We highlight that these results are significant for the following two reasons.
\bit
\item These are the first $k$-explicit bounds on the solution of the Helmholtz equation in the case where both $A$ and $n$ are heterogeneous. As will discussed in more detail in \cref{sec:wpdisc} below, previous results were either not $k$-explicit, or did not have $A$ \emph{and} $n$ varying. The $k$-explicitness of these results is crucial for understanding how the solution of the Helmholtz equation (and numerical methods for its approximation) behave for large $k.$
  \item These are the first bounds explicit in $A$ and $n$ where the bound and the restrictions on $A$ and $n$ are independent of $k.$ Such bounds are crucial for the rigorous analysis of the analagous stochastic problems, and previous results in the literature only proved such bounds by imposing conditions on $A$ and $n$ that became more stringent as $k \rightarrow \infty;$ again, this will be more fully discussed in \cref{sec:wpdisc} below.
\eit

We remark that \cref{thm:edp,thm:tedp} are extended to wider classes of heterogeneous $A$ and $n$ and to the case $\gD \neq 0$ in \cite{GrPeSp:19}. As stated above, the case $\gD \neq 0$ (with $A=I$) is treated in \cite[Theorem 2.19(ii)]{GrPeSp:19} (for \cref{prob:edp}) and \cite[Theorem A.6(iv)]{GrPeSp:19} (for \cref{prob:tedp}), and the case $n=1$ is covered in \cite[Theorem 2.19(i)]{GrPeSp:19} (for \cref{prob:edp}) and \cite[Theorem A.6(ii)]{GrPeSp:19}. We highlight that when $A=I$ or $n=1$ the conditions on the other coefficient can be slightly weakened from those in \cref{def:NT}. When $A$ and $n$ are discontinuous, \cite[Condition 2.6]{GrPeSp:19} gives analogues of the conditions in \cref{def:NT}, and then the result corresponding to \cref{thm:edp} is proved in \cite[Theorem 2.7]{GrPeSp:19}. Letting $A$ and $n$ be $L^\infty$-perturbations of nontrapping media is discussed in \cite[Remark 2.15]{GrPeSp:19}, and relaxing the Lipschitz assumption on $\GD$ is outlined in \cite[Remark 2.13]{GrPeSp:19}, with the caveat that when $\GD$ is non-Lipschitz, we instead formulate \cref{prob:edp} as a variational problem, which is discussed in \cref{sec:varform} below. The above extensions and generalisations all can also be applied to \cref{prob:tedp}, as mentioned in \cite[p. 2916]{GrPeSp:19}.

\subsection{Discussion of results on well-posedness and a priori bounds for the Helmholtz equation}\label{sec:wpdisc}
\bit
\item Will review history of well-posedness and a priori bounds for Helmholtz
\item Will contrast with stationary diffusion (becuase of UQ)
  \eit

  \bit
\item Well-posedness $=$ existence, uniqueness, a priori bound (continuous dependence)
\item Want bound explicit in $k$ and parameters
  \item Also $k$- and parameter-explicit bounds needed for $k$- and $parameter$-explicit analysis of numerical methods
  \item And to understand how medium affects wave propagation
    \item A lot more subtle than stationary diffusion
  \eit

Recall for stationary diffusion\optodo{PUT SOMEWHERE ELSE?}:
\bit
\item Variational formulation has bounded and coercive form
\item Lax--Milgram gives existence, uniqueness, a priori bound explicit in coeffs
\eit
  
Helmholtz:
\bit
\item Above approach doesn't work---not coercive for large $k$
\item But do have G\r{a}rding - indentity $+$ compact
\item So use Fredholm alternative to show uniqueness $\implies$ existence and bound (bound not explicit in $k$)
\item Show uniqueness via
\bit
\item UCP - but don't get a priori bound explicit in $k,$ parameters\optodo{Why does SRc give uniqueness for homogeneous problems?}
\item Proving a priori bound directly - get uniqueness
\eit
\eit
\bit
\item Will now talk about bounds
\item Focus on $k$-dependence (and later) parameter-dependence
\item Long history of proving bounds on Helmholtz from 'analysis' community \cite{Bl:73,Va:75,BlKa:77,Bu:98,PeVe:99,PoVo:99a,PoVo:99b,CaPoVo:99,Bu:02,Be:03,Ca:12,CaLePa:12,NgVo:12,Sh:18}\optodo{Check each and every reference}\optodo{Check about Morawetz} and recent interest from NA \cite{Me:95,CuFe:06,He:07,EsMe:12,Sp:14,FeLiLo:15,Ch:15,BaSpWu:16,BrGaPe:17,BaChGo:17,SaTo:18,OhVe:18,GrSa:18,GaSpWu:18,GrPeSp:19,MoSp:19}
\item Worst-case scenario - bounds grows exponentially in $k$
\bit
\item If everything $C^\infty$, shown for $n=1$ in \cite{Bu:02} and if $A=1$ and $n$ Lipschitz in \cite{Sh:18}. For both jumping across same $C^\infty$ interface in \cite{Bu:98}.
\item Graham-Sauter \cite{GrSa:18} show exponential blow up in (properties of) coefficients, not $k$ in 1-D\ednote{Ivan, can we chat about your work with Stefan to check I've understood it correctly?}
\item \cite{MoSp:19} - numerical evidence for super-algebraic growith with trapping jumps - proved in \cite{PoVo:99a,PoVo:99b,CaPoVo:99}.
\eit
\item But only get exponential blow up when problem is trapping
\item When everything $C^\infty,$ can define rays (for penetrable and impenetrable obstacle) and thus define nontrapping
\item Nontrapping gives $1/k$ cut-off resolvent bound
\item Will get cases later where things aren't so smooth, but still get cut-off resolvent bound
  \item Call these nontrapping
\item Trapping of rays (angle of incidence = angle of reflection) stay for arbitrarily long time
\item Can get trapping if there are jumps in one direction, but not in other direction - \cite{MoSp:19}
\item Actually, trapping only happens at very few frequencies-\cite{MoSp:19,LaSpWu:19}
\item Related to idea of `resonant frequency'
\item Can be caused by obstacles or variations in speed - c.f. total internal reflection
\item If problem non-trapping (i.e. rays escape) then get bound independent of $k$ - see hetero\optodo{Find a reference for this}\optodo{iff?}
\item Therefore results proving $k$-indep bounds under some conditions can be seen as providing sufficient conditions for problem to be nontrapping.
\item Such results use either:
\bit
\item Multiplier techniques going back to Morawetz (and Rellich) \optodo{List results, demarcating homo and hetero} which give $k$-explicit bound directly, or
\item more technical microlocal tools, which give nontrapping, from which you can conclude bound\optodo{Look at Vainberg/Melrose--Sj\"ostrand}
\item Recent work on bounds explicit in parameters---motivated in part by UQ - \cite{FeLiLo:15,GrPeSp:19,PeSp:18} prove bounds under conditions that ensure nontrapping, \cite{GaSpWu:18} assume problem is nontrapping and get `length of longest ray'.
\eit
\item In conclusion, subtle how things depend on $k,$ coefficients, and more difficult than stationary diffusion.
\eit

\bit
\item Now shift attention to NA of Helmholtz, specifically, FEM.
  \item Will prove new result on error bounds for Helmholtz in hetero media, generalises homo media result.
  \eit

\section{Theory of the Discretisation of the Helmholtz Equation}\label{sec:helmfe}

  \subsection{Variational Formulations for the Helmholtz equation}\label{sec:varform}

\optodo{Chat}
  
\bprob[Variational formulation of EDP when $\gD = 0$]\label{prob:vedp}
Let $\Dp, A, n,$ and $f$ be as in \cref{prob:edp}. Choose $R>0$ such that $\supp f, \supp(I-A), \supp(1-n) \compcont \BR,$ and define $\DR \de \Dp \cap \BR.$

We say $u \in \HozDDR$ satisfies the \defn{variational formulation of the exterior Dirichlet problem} with $\gD = 0$ if
\beqs
\aE(u,v) = \FE(v) \tfa v \in \HozDDR,
\eeqs
where
\beqs
\aE(w,v) \de \int_{\DR} \mleft(\IPRRd{A \grad w}{\grad v} - k^2 n w \vbar\mright) - \DPGR{\TR \trGR w}{\trGR v}
\eeqs
and
\beqs
\FE(v) \de \int_{\DR} f\vbar
\eeqs
\eprob

\ble[Equivalence of formulations for the EDP]\label{lem:edpform}
\Cref{prob:edp,prob:vedp} are equivalent, i.e., if $u \in \HolocDp$ solves \cref{prob:edp} then $u\restrict_{\DR} \in \HozDDR$ (for $R$ as in \cref{prob:vedp}) and $u\restrict_{\DR}$ solves \cref{prob:vedp}.\optodo{Sort restriction notation}
\ele

For a proof of \cref{lem:edpform}, see \cite[Lemma 3.3]{GrPeSp:19}.

\bprob[Variational formulation of TEDP when $\gD = 0$]\label{prob:vtedp}
Let $D, A, n, f,$ and $\gI$ be as in \cref{prob:tedp}. We say $u \in \HozDDR$ satisfies the \defn{variational formulation of the truncated exterior Dirichlet problem} with $\gD = 0$ if
\beqs
\aT(u,v) = \FT(v) \tfa v \in \HozDDR,
\eeqs
where
\beqs
\aT(w,v) \de \int_{\DR} \mleft(\IPRRd{A \grad w}{\grad v} - k^2 n w \vbar\mright) - ik\int_{\GI} \trGI w\trGI \vbar
\eeqs
and
\beqs
\FT(v) \de \int_{\DR} f\vbar + \int_{\GI} \gI \trGI \vbar
\eeqs
\eprob
\ednote{Both---can we discuss whether it's worth stating the variational formulation of the EDP/TEDP with non-zero Dirichlet data? On the one hand, it's more technical, but on the other hand, if we want (e.g. for MLMC) to consider the far-field as a quantity of interest, it makes most sense to consider a `real' scattering problem, i.e., one with non-zero Dirichlet data.}

\ble[Equivalence of formulations for the TEDP]\label{lem:tedpform}
\Cref{prob:tedp,prob:vtedp} are equivalent, i.e., $u \in \HozDDR$ solves \cref{prob:tedp} if, and only if, $u$ solves \cref{prob:vtedp}.
\ele

For a proof of \cref{lem:tedpform}, see \cite[Lemma A.7]{GrPeSp:19}.
  
\subsection{Finite-element theory}\label{sec:fetheory}

Now give requisite background finite-element theory

\bit
\item Definition - Finite element/Finite element space
\item Definition - Triangulation
\item Definition - Mesh size
\item Definition - Nodal Finite element
\item Definition - Fe space (of order $p$) associated with a triangulation
\item Proof of $H^m$ regularity when $f$ etc. is smooth enough\optodo{Why did you think you needed more than $H^2$?}
\item Definition - Scott--Zhang quasi-interpolant
\item Lemma - approximation properties of Scott--Zhang quasi-interpolant
\item Definition - Finite-element problem
\item Remark - won't be considering variational crimes, like approximating geometry\optodo{At least understand this}
\eit

\subsection{Error bound for the heterogeneous IIP}\label{sec:errbound}
\bit
\item This section contains new work
\item Prove heterogeneous analogue of results on bounded error if $h \lesssim k^{-(2p+1)/2p}$
\item Proof technique uses `elliptic projection' idea - see Feng \& Wu, Wu, Du \& Wu, Chaumont-Frelet \& Nicaise, WU \& Zou.
\item Prove for IIP only, as need to use results on a related PDE from Chaumont-Frelet and Nicaise; only proved with impedance boundary condition - expect can extend to, e.g., exact DtN.
\item These bounds crucial for MLMC analysis in \cref{chap:mlmc}
\item Will prove for non-zero Dirichlet boundary condition, as motivation for MLMC is scattering
  \eit
  Include:
  \bit
\item G\r{a}rding inequality
  \eit

\subsection{Discussion of FEM for Helmholtz}\label{sec:helmfedisc}
\bit
\item Recall from intro $h\sim k^{-1}$ (for first order; fixed ppw as num. points $\sim 1/h$) means interpolation error bounded, in general fixed ppw, but condition varies.
\bit
\item Can see by interpolation bounds from, e.g., Scott-Zhang (see below)
\item In 1-D\optodo{Find out if it works in higher D} motivate also by Shannon--Nyquist theorem\optodo{Find Shannon Proc I. R. E., 31:10-21, '49 Communication in the prescence of noise}
\item Both make precise intuition that we have bounded errors under fixed points per wavelength
\eit
\item Also recall Helmholtz suffers from pollution - fixed ppw not enough to keep FEM error bounded
\item This section will summarise conditions under which error bounded, and related property of quasi-optimality holds
\item Plan:
\bit
\item Introduce concepts of bounded error and quasi-optimality
%\item Show (for contrast) that they are straightforward for stationary diffusion
\item Give overview of results on bounded error and Quasi-optimality for Helmholtz
\eit
\eit
\bit
\item Definition: error bounded
\item Would really like relative error bounded, however, not possible with current technology
\item Definition: Quasi-optimal
\item Natural question: what mesh size keeps error bounded?
\item Short history:
\bit
\item Bayliss, Goldstein, Turkel - computations showing $h\sim k^{-3/2}$ sufficient to keep error bounded
\item Ihlenburg, Babu\v{s}ka - prove error $\sim h^2k^3$ in 1-D if $h \sim k^{-1}$
\item Wu - error $\sim h^2k^3$ if $h \sim k^{-3/2}$ for 1st-order FEM
\item Du and Wu - error $\sim h^{2p}k^{2p+1}$ for higher-order FEM
\item Chaumont-Frelet \& Nicaise - similar result with corner singularities
  \item WU \& Zou - special case in heterogeneous media
\item Here in \cref{sec:errbound} first proof that $h\sim k^{-(2p+1)/2p}$ sufficient to keep relative error bounded in heterogeneous problems
\eit
\item For stationary diffusion, things are straightforward
\item Prove quasi optimality using C\'ea
\item Get FEM error bounded by using properties of interpolant, e.g., via Scott-Zhang.
\item No condition on $h$ needed, unlike Helmholtz
\item For Helmholtz Cea unavailable as not coercive
\item Proving Quasi-optimality more tricky
\item Instead use Aubin--Nitsche duality argument
\bit
\item Introduced by Nitsche\optodo{FOr what?}
\item Used by Aubin\optodo{For what}
\item Applied by Schatz to problems satisfying G\r{a}rding inequality
\item First applied to Helmholtz by Melenk
\item Obtain quasi-optimality under nontrapping assumptions and restrictive mesh condition $h \sim k^{-2}$---see \cite[Proposition 8.2.7]{Me:95} for homogeneous, \cite{GrSa:18} and \cite[Theorem 3]{GaSpWu:18} for heterogeneous, related bound in \cite[Section 1.4]{ChSpGiSm:17} sketched for parabolic trapping cases
\item Mesh condition is prohibitive for large $k$
\eit
\eit
Conclusion:
\bit
\item Rigorous NA of Helmholtz more demanding than, e.g., stationary diffusion
\item Mesh conditions require decreasing mesh size with $k$; esp. for quasi-optimality, very restrictive.
\eit


\optodo{Put size of linear systems in}
\optodo{Put in poor performance of GMRES?}
\optodo{Put in need for preconditioners?}


%%5 Can put the below somewhere %%

%% To keep the finite-element error bounded when solving \eqref{eq:introdet}, one must over-refine the numerical grid. That is, rather than using a fixed number of points per wavelength, one must increase the number of points per wavelength as $k$ increases. To achieve bounded finite-element error, one must refine the finite-element mesh size $h$ like $k^{-3/2}$. Whilst this result has been known numerically for some time, it was proven for \eqref{eq:introdet} only with constant coefficients (on various domains and for various finite-element spaces) in \cite{IhBa:95a,Wu:14,DuWu:15,ChNi:18}, and the first proof (to our knowledge) for \eqref{eq:introdet} with heterogeneous coefficients  is contained in \cref{chap:background}. Choosing $h \sim k^{-3/2}$ means \eqref{eq:intromat} is a linear system of size $k^{3d/2}$, larger than if one merely wants the interpolation error to be bounded. Hence, requiring a bounded finite-element error gives rise to very large linear systems.

%% More briefly, if one wants the finite-element solution to be quasi-optimal (that is, up to a constant, the finite-element solution is the best approximation in the finite-element space), then one must over-refine even more, and take $h \sim k^{-2}$. This mesh condition will give rise to linear systems with $k^{2d}$ degres of freedom. See \cref{chap:background} for further details on the necessity of this mesh condition, and further discussion of all the mesh conditions discussed above.\optodo{EDIT THE ABOVE TO REMOVE `OVER-REFINE'}

