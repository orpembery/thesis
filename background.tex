\section{Introduction}
This chapter has two main foci:
%    \setlist[enumerate]{restart} ????
\optodo{Sort numbering}
\ben
\item Recapping theory for the deterministic Helmholtz equation in heterogeneous media, focussing especially on well-posedness and a priori bounds on the solution, and
\item Recapping and extending theory for the finite-element method for the deterministic Helmholtz equation, especially error bounds.
  \een

  In this chapter we will first provide an overview of well-posedness results for the deterministic Helmholtz equation, and a priori bounds on its solution.\optodo{Put somewhere below that Fredholm gives a bound without dependence on $k$} Such results are more complicated than analagous results for the simpler stationary diffusion equation
  \beq\label{eq:stdiff}
\grad \cdot \mleft( A \grad u \mright) = -f,
\eeq
as results for the Helmholtz equation depend on whether the medium described by the coefficients $A$ and $n$ is `trapping' or `nontrapping' (these terms will be discussed in more precision below). We will then move on to discussing the finite-element approximation of \eqref{eq:introdet}, focussing especially on error bounds and quasi-optimality, and the $k$-dependent mesh conditions under which such properties can be proven.

In \cref{sec:varform} we will define the deterministic Helmholtz problems we will study in this chapter (these are the deterministic analogues of the stochastic Helmholtz problems we will study), prove a straightforward lemma on the regularity of the solution of these problems, and show that these problems satisfy a so-called G\r{a}rding inequality. In \cref{sec:wpbounds} we will then state in some detail the well-posedness results and a priori bounds from \cite{GrPeSp:19}, these results will be crucial for our analysis of stochastic Helmholtz problems in \cref{chap:stochastic}. In \cref{sec:wpdisc} we will review recent and historical research efforts in this area. We then move on to the finite-element method for \eqref{eq:introdet}; in \cref{sec:fetheory} we recap basic concepts of the finite-element method, before proving new results on error bounds for the finite-element method for \eqref{eq:introdet} in \cref{sec:errbound}. We then give an overview of the literatureon error bounds and quasi-optimality for \eqref{eq:introdet} in \cref{sec:helmfedisc}.


\section{PDE Theory of the Deterministic Helmholtz Equation}
  

  We will begin by defining the two deterministic Helmholtz problems that we consider in this thesis (and the stochastic analogues of which we also consider). We first state these problems in `strong form' (that is, where derivatives are understood as distributional derivatives), in \cref{sec:varform}, when considering finite-element approximations of the problems, we will consider the variational formulations of these problems.\optodo{Somewhere explain convention on function spaces, and define matrix space} In this section we largely follow the presentation in \cite{GrPeSp:19}

  \bprob[Exterior Dirichlet Problem]\label{prob:edp}
  Let $\Dm$ be a bounded Lipschitz\optodo{Needed for statement?} open set such that the open complement $\Dp \de \RRd \setminus \Dmclos$ is connected. Let $\GD \de \partial \Dm.$ Given
  \bit
  \item $k > 0,$
\item $f \in \LtDp$ with compact support,
\item $\gD \in \HhGD,$
\item $n \in \LiDpRR$ such that $1-n$ has compact support and there exist $0 < \nmin < \nmax < \infty$ such that
  \beqs
\nmin \leq n(\bx) < \nmax \tfae \bx \in \Dp,
  \eeqs
\item $A \in \LiDpRRdtd$ such that $I-A$ has compact support, $A$ is symmetric, and there exist $0 < \Amin < \Amax < \infty$ such that
  \beqs
\Amin \abs{\bxi}^2 \leq \mleft(A(\bx) \bxi \mright) \cdot \bxibar < \Amax \abs{\bxi}^2 \tfa \bxi \in \CCd \tfae \bx \in \Dp,
  \eeqs
  \eit
  we say $u \in \HolocDp$ satisfies the \defn{exterior Dirichlet problem} if
  \beqs
\grad \cdot \mleft(A \grad u\mright) + k^2 n u = -f \tin \Dp,
\eeqs
\beq\label{eq:dbc}
\trGD u = \gD,
\eeq
and $u$ satisfies the Sommerfeld radiation condition\optodo{Make sure you understand this}
\beq\label{eq:sommerfeld}
\dudr(\bx) - iku(\bx) = o\mleft(\frac1{r^{(d-1)/2}}\mright)
\eeq
as $r \de \abs{\bx} \rightarrow \infty,$ uniformly in $\bxhat \de \bx/\abs{\bx}.$
\eprob

To interpret \cref{prob:edp} physically, we can think of $u$ as being the acoustic pressure field caused by the scattering of an incoming wave $\ui$ by the scatterer $\Dm.$\footnote{In the literature the scattered field is sometimes denotes $us,$ and $u$ denotes the total field $\ui+ \us.$} The Dirichlet boundary condition \eqref{eq:dbc} means $\Dm$ corresponds to a sound-soft scatterer, that is, one on which the total field $u + \ui$ vanishes, with $\gD = - \trGD \ui.$ The function $f$ represents other pressume sources in the domain, although if one is solely interested in scattering, $f=0.$

The Sommerfeld radiation condition \eqref{eq:sommerfeld} ensures that the solutions of \cref{prob:edp} correspond to physically `outgoing' waves\optodo{Find reference}, and also guarantees the uniqueness of solution to \cref{prob:edp}\optodo{Ref for this}. However, \cref{prob:edp} is posed on an infinite domain, which cannot be fully represented when discretising with the FEM. Therefore, one must truncated the infinite domain $\Dp$ and impose an artificial boundary condition on the external boundary of the truncated domain. Options for the truncated boundary condition include a perfectly matched layer which\optodo{check} mimics the whole of the external domain, or so-called FEM-BEM coupling\optodo{Check}, where a boundary element method is used to approximate the solution in the exterior of the truncated domain. However, in this thesis, we will use a simpler approach, imposing an \defn{impedance boundary condition}
\beq\label{eq:imptext}
\dn u - iku = \gI
\eeq
on the truncated boundary. If $\gI = 0,$ then \eqref{eq:imptext} can be seen as a first-order approximation to \eqref{eq:sommerfeld}. Truncating with an impedance boundary condition gives rise to the following alternative deterministic Helmholtz problem\optodo{Make a picture explaining EDP and TEDP on one pic}
\bprob[Truncated Exterior Dirichlet Problem]\label{prob:tedp}
 Let $\Dm$ be a bounded Lipschitz\optodo{Needed for statement?} open set such that the open complement $\Dp \de \RRd \setminus \Dmclos$ is connected. Let $\Dtilde$ be a bounded connected Lipschitz open set such that $\Dmclos \compcont \Dtilde.$ Let $D \de \Dtilde \setminus \Dmclos$, $\GD \de \partial \Dm,$ and $\GI \de \partial \Dtilde.$ Given
  \bit
  \item $k > 0,$
\item $f \in \LtD$
\item $\gD \in \HhGD,$
  \item $\gI \in \LtGI$\optodo{Understand why these are the function space requirements}
\item $n \in \LiDRR$ such that $\supp\mleft(1-n\mright) \compcont D$ and there exist $0 < \nmin < \nmax < \infty$ such that
  \beqs
\nmin \leq n(\bx) < \nmax \tfae \bx \in D,
  \eeqs
\item $A \in \LiDRRdtd$ such that $\supp\mleft(I-A\mright)\compcont D$,\ednote{Euan---in hetero, this requirement is instead phrased `$\dist(\supp(I-A),\GI) > 0'$. Is there any reason for phrasing it that way?} $A$ is symmetric, and there exist $0 < \Amin < \Amax < \infty$ such that
  \beqs
\Amin \abs{\bxi}^2 \leq \mleft(A(\bx) \bxi \mright) \cdot \bxibar < \Amax \abs{\bxi}^2 \tfa \bxi \in \CCd \tfae \bx \in D,
  \eeqs
  \eit
  we say $u \in \HoD$ satisfies the \defn{truncated exterior Dirichlet problem} if
  \beqs
\grad \cdot \mleft(A \grad u\mright) + k^2 n u = -f \tin D,
\eeqs
\beqs
\trGD u = \gD, \tand
\eeqs
\beq\label{eq:ibc}
\trGI \dn u - ik\trGI u = \gI.
\eeq
\eprob
Observe that, by construction, $\partial D = \GI \cup \GI$ and $\GD \cap \GI = \emptyset.$

Whilst the impedance boundary condition \eqref{eq:ibc} does not exactly mimic the Sommerfeld radiation condition \eqref{eq:sommerfeld}, the solutions of \cref{prob:tedp} are still `wave-like', and we will see below that solutions of \cref{prob:tedp} posses many of the same properties as solutions of \cref{prob:edp}. We also note that a common Helmholtz model problem in the numerical-analysis community is the \defn{interior impedance problem}, which is simply \cref{prob:tedp} in the case $\Dm = \emptyset.$

In order to approximation \cref{prob:edp,prob:tedp} by the finite-element method, we must instead work with the following 'variational forms' or 'weak forms' of \cref{prob:edp,prob:tedp}. For simplicity of exposition, we state both variational forms in the case $\gD = 0,$ although these can be generalised to the case $\gD \neq 0.$
\optodo{Insert ball notation somewhere}
\optodo{Define DtN}
\optodo{Define duality pairing}

\optodo{Fix cleverref referring to Problems as Theorems}


  \subsection{Well-posedness and bounds}\label{sec:wpbounds}

  We will now recap the well-posedness results and a priori bounds for \cref{prob:edp,prob:tedp} from \cite{GrPeSp:19}; these results will, in particular, be crucial for proving well-posedness results and a priori bounds for the stochastic analogues of \cref{prob:edp,prob:tedp} in \cref{chap:stochastic}. The novelty of these results is that they hold independently of $k,$ and that the a priori bounds we prove are explicit in $A$ and $n$. The explicitness in $A$ and $n$ is necessary in order to prove a priori bounds for stochastic $A$ and $n$, and we prove the results under conditions on $A$ and $n$ that are, in some sense `nontrapping'. Informally, a medium is `nontrapping' if all rays travelling through the medium escape in a uniform time; this definition, and the sense in which our conditions are `nontrapping', is discussed in \cref{sec:wpdisc} below.

  We first define the classes of $A$ and $n$ for which we will prove well-posedness results and a priori bounds.% Since we will need to consider the classes on both finite and infinite domains (for \cref{prob:tedp,prob:edp} respectively), we first define the classes for finite domains, before using this definition to define the classes for infinite domains.

  \optodo{Define star-shaped w.r.t. a point, ball}

%%   \bde[Class of nontrapping media on a finite domain]\label{def:NTfinite}
%% Let $\Dm$ be star-shaped with respect to the origin, and let $D$ be as in \cref{prob:tedp}. Let $A \in \CzoDclosRRdtd, n \in \CzoDclosRR$ and $\muo, \mut > 0.$ We say that 
%%   \ede
  
\bde[Class of nontrapping media]\label{def:NT}
Let $A \in \CzoDpclosRRdtd, n \in \CzoDpclosRR,$ and $\muo, \mut > 0.$ We say that $A \in \NTADp{\muo}$ if
\beqs
A(\bx) - \mleft(\bx \cdot \grad\mright)A(\bx) \geq \muo
\eeqs
in the sense of quadratic forms for almost every $\bx \in \Dp$. We say that $n \in \NTnDp{\mut}$ if
\beqs
n(\bx) + \bx \cdot \grad n(\bx) \geq \mut
\eeqs
for almost every $\bx \in \Dp.$

If $D$ is as in \cref{prob:tedp}, then we define $\NTAD{\muo}$ and $\NTnD{\mut}$ analagously.
\ede
\ednote{Both---Can you think of better notation for these conditions? I'm not overly keen to write $\mathrm{NT}_{\mathrm{A}}$ etc., as there is potentially confusion between then function $A$ and the $\mathrm{A}$ is the subscript. Or am I being overcautious?}

We can now prove well-posedness results and a priori bounds for the Helmholtz equation in the class of heterogeneous media we have just defined.

\bth[Well-posedness and bound for the EDP]\label{thm:edp}
If $\Dm, A, n,$ and $f$ satisfy the requirements in \cref{prob:edp}, $\Dm$ is star-shaped with respect to the origin, there exists $\muo, \mut > 0$ such that $A \in \NTADp{\muo}$ and $n \in \NTnDp{\mut}$, and $\gD = 0,$ then the solution of \cref{prob:edp} exists and is unique. Furthermore, given $R>0$ such that $\supp\mleft(I-A\mright),$ $\supp(1-n),$ and $\supp f$ are compactly contained in $\DR,$ then
\beqs
\muo \NLtDR{u}^2 + \mut k^2 \NLtDR{\grad u}^2 \leq \Co \NLtDR{f}^2,
\eeqs
for all $k>0,$ where
\beqs
\Co \de 4\mleft(\frac{R^2}{\muo} +\frac1{\mut}\mleft(R + \frac{d-1}{2k}\mright)^2\mright).
\eeqs
\enth

For the proof of \cref{thm:edp}, see \cite[Theorem 2.5]{GrPeSp:19}.

One can prove an analagous result to \cref{thm:edp} for \cref{prob:tedp} using the same techniques as in the proof of \cref{thm:edp}. However, the statement of the theorem is slightly more complicated due to the presence of the impedance boundary $\GI$, and its effect on the solution.

\bth[Well-posedness and bound for the TEDP]\label{thm:tedp}
If $\Dm, A, n, f,$ and $\gI$ satisfy the requirements in \cref{prob:tedp}, $\Dm$ is star-shaped with respect to the origin, $\Dtilde$, is star-shaped with respect to a ball, there exists $\muo, \mut > 0$ such that $A \in \NTAD{\muo}$ and $n \in \NTnD{\mut}$, and $\gD = 0,$ then the solution of \cref{prob:tedp} exists and is unique. Let:
\bit
\item $\LI \de \max_{\bx \in \GI} \abs{\bx}$ and
\item $a\LI$ be the radius of the ball with respect to which $\Dtilde$ is star-shaped.
    \eit
Then
\begin{multline*}
  \muo \NLtD{u}^2 + \mut k^2 \NLtD{\grad u}^2 + a\LI\NLtGI{\gradGI \trGI u}^2 + 2\LI k^2 \NLtGI{\trGI u}^2\\
  \leq \Ct \NLtDR{f}^2 + \Cttilde \NLtGI{\gI}^2
\end{multline*}
\optodo{Check that hetero grad boundary notation isn't hiding anything}
for all $k>0,$ where $\gradGI$ is the surface gradient on $\GI,$
\beqs
\Ct \de 4\mleft(\frac{\LI^2}{\muo} + \frac1{\mut}\mleft(\beta + \frac{d-1}{2k}\mright)^2\mright),
\eeqs
\beqs
\Cttilde \de 2\mleft(2\mleft(1+\frac2a\mright) + \frac\beta{\LI} + \frac{\mleft(d-1\mright)^2}4\mright)\LI,
\eeqs
and
\beqs
\beta \de \LI \mleft(2+\frac1{\mleft(k\LI\mright)^2} + 2\mleft(1+\frac2a\mright)\mright).
\eeqs
\enth

Observe that the above results are stated only in the case that $\gD = 0$. Whilst there is no mathematical difficulty in proving analagous results in the case $\gD \neq 0,$ the calculations in this case are more involved, as one must consider the surface gradient on the Dirichlet boundary, and this surface gradient depends on $A.$ In the case $A=I,$ these calculations are significantly simplified, and so in the case $A=I$ and $\gD \neq 0$ analagous results to \cref{thm:edp,thm:tedp} are proved in \cite[Theorem 2.19(ii)]{GrPeSp:19} (for \cref{prob:edp}) and \cite[Theorem A.6(iv)]{GrPeSp:19} (for \cref{prob:tedp}).

We highlight that these results are significant for the following two reasons.
\bit
\item These are the first $k$-explicit bounds on the solution of the Helmholtz equation in the case where both $A$ and $n$ are heterogeneous. As will discussed in more detail in \cref{sec:wpdisc} below, previous results were either not $k$-explicit, or did not have $A$ \emph{and} $n$ varying. The $k$-explicitness of these results is crucial for understanding how the solution of the Helmholtz equation (and numerical methods for its approximation) behave for large $k.$
  \item These are the first bounds explicit in $A$ and $n$ where the bound and the restrictions on $A$ and $n$ are independent of $k.$ Such bounds are crucial for the rigorous analysis of the analagous stochastic problems, and previous results in the literature only proved such bounds by imposing conditions on $A$ and $n$ that became more stringent as $k \rightarrow \infty;$ again, this will be more fully discussed in \cref{sec:wpdisc} below.
\eit

We remark that \cref{thm:edp,thm:tedp} are extended to wider classes of heterogeneous $A$ and $n$ and to the case $\gD \neq 0$ in \cite{GrPeSp:19}. As stated above, the case $\gD \neq 0$ (with $A=I$) is treated in \cite[Theorem 2.19(ii)]{GrPeSp:19} (for \cref{prob:edp}) and \cite[Theorem A.6(iv)]{GrPeSp:19} (for \cref{prob:tedp}), and the case $n=1$ is covered in \cite[Theorem 2.19(i)]{GrPeSp:19} (for \cref{prob:edp}) and \cite[Theorem A.6(ii)]{GrPeSp:19}. We highlight that when $A=I$ or $n=1$ the conditions on the other coefficient can be slightly weakened from those in \cref{def:NT}. When $A$ and $n$ are discontinuous, \cite[Condition 2.6]{GrPeSp:19} gives analogues of the conditions in \cref{def:NT}, and then the result corresponding to \cref{thm:edp} is proved in \cite[Theorem 2.7]{GrPeSp:19}. Letting $A$ and $n$ be $L^\infty$-perturbations of nontrapping media is discussed in \cite[Remark 2.15]{GrPeSp:19}, and relaxing the Lipschitz assumption on $\GD$ is outlined in \cite[Remark 2.13]{GrPeSp:19}, with the caveat that when $\GD$ is non-Lipschitz, we instead formulate \cref{prob:edp} as a variational problem, which is discussed in \cref{sec:varform} below. The above extensions and generalisations all can also be applied to \cref{prob:tedp}, as mentioned in \cite[p. 2916]{GrPeSp:19}.

\subsection{Discussion of results on well-posedness and a priori bounds for the Helmholtz equation}\label{sec:wpdisc}

We will now review the historical development of well-posedness results, and a priori bounds for the Helmholtz equation. Recall that by `well-posedness', we mean that a solution of the Helmholtz equation exists, is unique, and continuously depends on the data $f$ and $\gD$ (and $\gI$ for \cref{prob:tedp}). Observe that an a-priori bound of the form
\beq\label{eq:bgbound}
\N{u} \leq C \mleft(\N{f} + \N{\gD}\mright)
\eeq
(where $C$ is some constant, possibly dependent on $k, A, n,$ and the domain) immediately allows us to conclude uniqueness (as the solution of the Helmholtz equation with zero data must be the zero function) and continuous dependence (because small perturbations in $f$ and $\gD$ can only result in small perturbations in $u$). However, we will particularly focus our attention on a priori bounds that are explicit in their dependence both on $k$, and on the material coefficients $A$ and $n$. This is because knowing such dependence enables us to perform $k$- and parameter-explicit analysis of numerical methods for the heterogeneous (and stochastic) Helmholtz equation and also to understand in more detail how the medium (and the interplay between the medium and the frequency) affects the propagation of the wave.

We note that proving well-posedness results and a priori bounds for the Helmholtz equation is much more involved than proving such results for the stationary diffusion equation
\beq\label{eq:stdiff}
\grad \cdot \mleft(A \grad u\mright) = -f \tin D.
\eeq
In \eqref{eq:stdiff} (where $A$ is heterogeneous but not stochastic), if $A$ is bounded above and below, then the associated bilinear form is bounded and coercive, and so the Lax--Milgram Theorem applies, and one immediately obtains well-posedness and an a priori bound (in $\HoD$) that is explicit in $A$.\optodo{Want to make a remark here about things being more subtle, but check multi-scale stuff for st. diff. first, so that I'm not misrepresenting}

In contrast, the associated sesquilinear form for the Helmholtz equation is not coercive, and so Laz--Milgram theory does not apply. However, the sesquilinear form $a$ satisfies a G\r{a}rding inequality
\beq\label{eq:gardingbrief}
\Re{a(v,v)} + 2 k^2 \nmax \NLtD{v}^2 \geq \NW{v}^2,
\eeq
, where $\NW{v} \de \mleft(\NLtD{\grad v} + k^2 \NLtD{v}\mright)^{1/2}$ is the weighted $H^1$ norm used frequently when studying Helmholtz problems, because solutions of the Helmholtz equation typically are of order $1/k$ and have first-order derivatives of order $k$. I.e., the G\r{a}rding inequality means $\Re{a(v,v)}$ is `coercive' if an appropriate multiple of the $L^2$-norm is added. In this case, if one can show that the solution of the Helmholtz equation is unique, then existence and an a priori bound on the solution follow from Fredholm Theory (see, e.g. \cite[Theorems 5.10 and 5.18]{Sp:15}). However, the a priori bound is \emph{not} explicit in $k,$ $A$ or $n$. Therefore, the main effort in showing well-posedness for the Helmholtz equation is reduced to:
\ben
\item Showing that a solution of the Helmholtz equation is unique, and
  \item Proving a bound on the solution that is explicit in the parameters of interest, such as $k,$ $A$, and $n$.
    \een
    Observe that, as mentioned above, an a priori bound immediately gives uniqueness, but uniqueness also follows from the Sommerfeld radiation condition (for homogeneous problems)\optodo{Why?} or the Unique Continutation Principle (UCP), under some smoothness assumptions on $A$ and $n$ (see, e.g., \cite[p. 2871]{GrPeSp:19} for a discussion of the UCP and \cite[Section 2]{GrSa:18} for the application of the UCP to show uniqueness for hetergeneous Helmholtz problems). Therefore, in our following discussion, we will focus on a priori bounds \eqref{eq:bgbound} for the solution of the Helmholtz equation, and the dependence of these bounds on $k$, $A$, and $n.$ All these bounds will, unless otherwise stated, be for the weighted $H^1$ norm $\NW{\cdot}.$

    There are two main classes of techniques for proving a priori bounds on the Helmholtz equation. The first uses techniques from semiclassical analysis (a subset of microlocal analysis), and studies the behaviour of rays through the medium. In order for this approach to be used $\Dm,$, $A$, and $n$ must all be smooth so that, e.g., the notion of reflections from the scatterer $\Dm$ can be defined (the notion of a reflection is not well-defined if the scatterer has a corner).\ednote{Euan---I want to add some chat here about when these methods were first introduced, but I'm struggling to figure it out? Is it with Lax \& Phillips?} The second class of techniques is multiplier techniques, where the PDE\optodo{ref?} is multiplied by carefully chosen multiples of $u$ and $\bx\cdot\grad u$, and the resulting expression is then integrated by parts, and rearranged. Whilst simple, these methods allow one to prove bounds in situations that are inaccesible to semiclassical analysis, e.g., when the scatterer or coefficients are not smooth. These techniques were first introduced by Morawetz in the 60s for for studying how energy decays in the wave equation---see \cite{GaGrPaSaSoTa:18} for an overview of this, and other aspects of Morawetz's work and\optodo{Look up Vodev 99} for the connection between energy decay for the wave equation and a priori bounds on the Helmholtz equation. However, multiplier techniques typically require more severe restrictions on the geometry of the scatterer (and truncation boundary, in the case of \cref{prob:tedp}) than semiclassical analysis techniques\footnote{One can choose more complicated multipliers to mitigate some of these restrictions, as in \cite{MoRaSt:77}, but most of the works we will discuss below do not.}

When one uses microlocal analysis results to prove a \emph{$k$-independent} bound, one typically proceeds by showing that the problem is nontrapping (in the sense of rays, defined above\optodo{Move}), and then using further technical tools\footnote{Either: (i) Vainberg's paramatrix argument from \cite{Va:75} and Melrose and Sj\"ostran's results propagation of singularities \cite{MeSj:82}, Lax--Phillips theorem \cite{LaPh:??} and propagation of singularities, or Burq's defect-measure argument \cite{Bu:02}.} to conclude a $k$-independent bound. We will see below that one can prove $k$-independent bounds even when rays cannot be defined (typically when $\Dm,$ $A$, and $n$ are not smooth), and so, in an abuse of terminology, we call such situations `nontrapping'. We also note that trapping behaviour can be caused either by $\Dm$ (for example, if $\Dm$ contains a cavity in which rays can be `trapped', see, e.g.,\optodo{Make figure}, or by variations in $A$ and $n$ (in the case where $A$ and $n$ are discontinuous, these discontinuities can cause rays to be trapped in a manner analagous to the concept of total internal reflection.

We now summarise the state of the field regarding bounds \eqref{eq:bgbound}, and especially the dependence of the constant $C$ on $k,$ $A$, $n$, and $\Dm.$ Some of these results are proved in the context of energy decay results for the time-domain wave equation; for simplicity's sake, we do not distinguish in our comments between these results, and those proving bouinds \eqref{eq:bgbound} directly.

    In the worst case, the constant $C$ can depend exponentially on $k$; in \cite{Bu:98}\ednote{Euan, is \cite[Theorem 2]{Bu:98} saying that the pole-free region is exponentially (as $k\rightarrow \infty$) close to the real axis, and in this region, the bound is exponential? My French isn't that great....}, \cite{Sh:18}, and \cite{Be:02} it was shown that $C = \Co \exp(k\Ct),$ for some constants $\Co$ and $\Ct$ depending on $\Dm, $ $A$, and $n$ when $n=1$ and $A$ are $C^\infty$ or $\Dm = \emptyset,$ $A=I$, and $n$ is a Lipschitz perturbation of $1$, or $A$ and $n$ are piecewise constant, jumping acros a common $C^\infty$ interface\optodo{Check}, respectively. Carleman estimates\optodo{Find out what these are/what went on} were used in \cite{Bu:98,Be:02}, and semiclassical analysis techniques (to prove energy decay for the in \cite{Sh:18}. In \cite{PoVo:99a}, it was shown that the same bound holds for the transmission problem (i.e., when $\Dm=\emptyset$ and $A$ and $n$ jump on a common interface) when $A=I$ and $n$ jumps downwards across an interface.\ednote{Euan---can you briefly recap for me how one moves from nontrapping/results on location of resonances to an a priori bound?} The papers \cite{Ca:12,CaLePa:13} study the transmission problem across a sphere and consider arbitrary-order Sobolev norms on spheres outside the transmission boundary. Using an expansion of the solution in terms of Bessel functions they show that the constant $C$ can grow super-algebraically in $k$ when $n$ jumps downwards, but is bounded when $n$ jump upwards. They also show that, even in the case of $n$ jumping downwards, if the jump is sufficiently small, $k$ is sufficiently small, or the norm is measured sufficiently far away (where all the definitions of `sufficient' depend on the size of the jump and the size of the sphere) $C$ is bounded independently of $k.$

    For \cref{prob:edp}, results showing that $C$ is bounded independently of $k$ (subject to restrictions on the geometry of the scatterer) are aplenty: \cite{Mo:61,MoLu:68,Mo:75}\ednote{Euan, I'm having a hard time seeing what's different in each of these references} proves such for constant $A$ and $n$ in the exterior of a star-shaped Dirichlet scatterer (\cite{Mo:75} also proves a $k$-independent bound in the exterior of a convex Neumann scatterer). This approach is generalised in \cite{MoRaSt:77} to more complicated scatterers (but still constant $A$ and $n$) that possess a so-called `escape function', which can be thought of as analogous to ensuring rays escape from the scatterer. The works \cite{CaPoVo:99,PoVo:99b} use semiclassical analysis techniques to obtain $k$-independent $C$ for the transmission problem when piecewise-constant $n$ jumps up across a strictly convex, smooth boundary.

%    In the case that $A$ or $n$ vary in the exterior of a Dirichlet obstacle, \cite{BlKa:77} proves $C$ is independent of $k$ when $A=I$, provided $\Dm$ can be `illuminated from the exterior', which, again, is an analogue to ensuring no rays are trapped.

% Mention Bloom & Bloom Kazarinoff? As they make things go to zero, and Perthame-Vega?
    
    \optodo{Say somewhere that we're only considering (essentially) comnpact stuff}
    \optodo{Check that you've mentioned a definition of nontrapping somewhere}
%    to show that the domain (i.e., the combination of the scatterer $\Dm$ and the coefficients $A$ and $n$) is \defn{nontrapping}. In order to be nontrapping, all rays that start within any bounded region must have exited this region after a time that is uniform over all the rays (i.e., rays do not get trapped for an arbitrarily long time). In order to use this approach, $\Dm,$ $A$, and $n$ must all be smooth
\optodo{Double check have mentioned that e.g. Burq are looking at hetero problems}
Analagous bounds for \cref{prob:tedp} (or, more commonly, the \defn{interior impedance problem} (IIP), that is, \cref{prob:tedp} with no scatterer) have mostly been studied by the numerical analysis community, due to \cref{prob:tedp}'s usage as a model problem for numerical methods (the truncation boundary means that the problem is posed on a finite domain, enabling domain-discretisation methods, such as finite elements, to be used). The vast majority of these results use, in essence, the multiplier techniques introduced by Morawetz. The works \cite{Me:95,CuFe:06} prove that $C$ is $k$-independent for the IIP in a domain that is either convex or star-shaped and smooth, and this result is generalised to a mixture of impedance, Dirichlet, and Neumann boundary conditions in \cite{He:07}. In \cite{EsMe:12}, the restriction on the domain is loosened to only a Lipschitz condition, but can only prove that $C$ increases with $k$ at the rate $k^{5/2}.$ (The work \cite{EsMe:12} does not use multiplier techniques, but rather uses properties of the Green's function for the Helmholtz equation, which follow from \cite{MeSa:12}.) The rate of growth of $C$ for a general (Lipschitz) domain was improved to order $k$ in \cite{Sp:14} (using multiplier techniques) and finally eliminated for a general $C^\infty$ domain (using microlocal analysis) in \cite{BaSpWu:16}.

Given the (largely) complete picture of bounds for the Helmholtz equation in \emph{homogeneous} media, recent research effort (especially in the numerical-analysis community) has shifted towards bounds for the Helmholtz equation in \emph{heterogeneous} media, of which the work \cite{GrPeSp:19} is a part; this shift in focus is motivated by the development of algorithms to solve the Helmholtz equation in heterogeneous or random media; the development of uncertainty quantification algorithms for random media also necessitates knowing explicitly how the constant $C$ in \eqref{eq:bgbound} depends on $A$ and $n$. Bounds where $C$ is independent of $k$ are given in \cite{FeLiLo:15} for random media, under the assumption that $A=I$ and $n = 1 + \eta,$ with $\eta$ a random field and $\NLiD{\eta} \lesssim 1/k.$ Results with $k$-independent $C$ are also proved in \cite{BrGaPe:17}, under related conditions to those in \cite{GrPeSp:19}, although the bounds in \cite{GrGaPe:17} are not explicit in $A$ and $n$. The case where $A$ is scalar-valued (i.e. $A = \eps I$) and given by many small inclusions is studied in \cite{OhVe:18}\optodo{Check this}, motivated by the analysis of multiscale methods for the Helmholtz equation; however \cite{OhVe:18} is only able to prove that $C$ is proportional to $k^3$ in this case, and the bound is not explicit in the parameters. The work \cite{GrSa:18} takes a very similar approach to \cite{GrPeSp:19}, proving results for heterogeneous media when $A=I$ under conditions on $n$ that are analogous to those in \cite{GrPeSp:19}; however, the bounds in \cite{GrSa:18} are not explicit in $n.$ In \cite{BaChGo:17} a $k$-independent bound is proven for 2-D piecwise-constant media, under suitable restrictions on $n.$ The work \cite{MoSp:19} proves $k$-independent bounds for the transmission problem with varying $A$ and $n$, where the bounds are fully explicit in all the parameters, under assumptions on the jumps in $A$ and $n.$ Finally, \cite{GaSpWu:18} uses microlocal analysis techniques to prove that, when all the parameters of interest are smooth, that the constant $C$ in a $k$-independent bound is given by the length of the longest ray.

In related results, for the 1-dimensional Helmholtz equation in heterogeneous media, \cite{Ch:15} proves a $k$-independent bound for piecewise constant media, under assumptions on the media that limit the number of `pieces', whereas \cite{SaTo:18} proves a bound for arbitrarily many `pieces', but with the consequence that $C$ grows algebraically in $k$, using properties of the 1-dimensional Green's function. Also, \cite{LiWu:18} proves bounds for \cref{prob:edp} truncated with a perfectly matched layer (see, e.g., \optodo{Find a ref}), using properties of Bessel functions.

Observe that many of the results above that prove a $k$-independent bound are of the form `place restrictions on $\Dm$, $A$, and $n$ to ensure nontrapping, then conclude result'. However, the recent works \cite{MoSp:19,LaSpWu:19} have shown that even in the worst case of exponential growth in $k$ of the constant $C$, this behaviour is realised at very few frequencies; \cite{MoSp:19} provided numerical evidence for the transmission problem through a sphere that showed the realisation of super-algebraic growth is very sensitive to the value of $k$ (in some cases, changing $k$ at the 13th significant figure changed the behaviour completely. More rigorously, \cite{LaSpWu:19} used microlocal analysis techniques to show that one can exclude a set of $k$ of arbitrarily small measure, and then obtain merely algebraic growth in $C$.

We note in passing that the discussions above have been mainly concerned with the effect inhomogeneities in $A$ and $n$ have on the behaviour of the solution $u$, and thus on the constant $C$. When one considers the trapping caused by the scatterer $\Dm,$ a range of behaviours can occur; $C$ can grow logarithmically, polynomially, or exponentially in $k$ depending on the scatterer. We refer the reader to \cite[Seciotns 1.1 and 1.3]{ChGiSpSm:17} for a recent overview of the effect of obstacle scattering.
   




\section{Theory of the Discretisation of the Helmholtz Equation}\label{sec:helmfe}

We will now shift our attention to the numerical analysis of the Helmholtz equation in heterogeneous media; more specifically, we will study the finite-element method for the Helmholtz equation. We will first provide the necessary formulations of the Helmholtz equation and definition of the finite element method, and related results, before proving our main result, a new error bound for the finite-element method for the Helmholtz equation in heterogeneous media.

  \subsection{Variational Formulations for the Helmholtz equation}\label{sec:varform}

In order to apply the finite-element method to the Helmholtz equation, we must first consider its variational formulation, which we define below for both \cref{prob:edp,prob:tedp}. With the variational formulation in place, we can then go on to define the finite-element method, before proving our error bounds.
  
\bprob[Variational formulation of EDP when $\gD = 0$]\label{prob:vedp}
Let $\Dp, A, n,$ and $f$ be as in \cref{prob:edp}. Choose $R>0$ such that $\supp f, \supp(I-A), \supp(1-n) \compcont \BR,$ and define $\DR \de \Dp \cap \BR.$

We say $u \in \HozDDR$ satisfies the \defn{variational formulation of the exterior Dirichlet problem} with $\gD = 0$ if
\beqs
\aE(u,v) = \FE(v) \tfa v \in \HozDDR,
\eeqs
where
\beqs
\aE(w,v) \de \int_{\DR} \mleft(\IPRRd{A \grad w}{\grad v} - k^2 n w \vbar\mright) - \DPGR{\TR \trGR w}{\trGR v}
\eeqs
and
\beqs
\FE(v) \de \int_{\DR} f\vbar
\eeqs
\eprob

\ble[Equivalence of formulations for the EDP]\label{lem:edpform}
\Cref{prob:edp,prob:vedp} are equivalent, i.e., if $u \in \HolocDp$ solves \cref{prob:edp} then $u\restrict_{\DR} \in \HozDDR$ (for $R$ as in \cref{prob:vedp}) and $u\restrict_{\DR}$ solves \cref{prob:vedp}.\optodo{Sort restriction notation}
\ele

For a proof of \cref{lem:edpform}, see \cite[Lemma 3.3]{GrPeSp:19}.

\bprob[Variational formulation of TEDP when $\gD = 0$]\label{prob:vtedp}
Let $D, A, n, f,$ and $\gI$ be as in \cref{prob:tedp}. We say $u \in \HozDDR$ satisfies the \defn{variational formulation of the truncated exterior Dirichlet problem} with $\gD = 0$ if
\beqs
\aT(u,v) = \FT(v) \tfa v \in \HozDDR,
\eeqs
where
\beqs
\aT(w,v) \de \int_{\DR} \mleft(\IPRRd{A \grad w}{\grad v} - k^2 n w \vbar\mright) - ik\int_{\GI} \trGI w\trGI \vbar
\eeqs
and
\beqs
\FT(v) \de \int_{\DR} f\vbar + \int_{\GI} \gI \trGI \vbar
\eeqs
\eprob
\optodo{Proof of $H^m$ regularity when $f$ etc. is smooth enough}

\ble[Equivalence of formulations for the TEDP]\label{lem:tedpform}
\Cref{prob:tedp,prob:vtedp} are equivalent, i.e., $u \in \HozDDR$ solves \cref{prob:tedp} if, and only if, $u$ solves \cref{prob:vtedp}.
\ele

For a proof of \cref{lem:tedpform}, see \cite[Lemma A.7]{GrPeSp:19}.
  
\subsection{Finite-element theory}\label{sec:fetheory}

We now give a brief summary of elementary concepts in finite-element theory, so that we can:
\ben
\item Prove the new error bound for finite-elements discretisations of the Helmholtz equation in \cref{sec:errbound} below, and
  \item Develop and study a nearby preconditioning technique for finite-element discretisations of the Helmholtz equation in \cref{chap:nbpc} below.
\een
    \bde[Triangulation]
    A triangulation of a \optodo{Polygonal?} domain $D$ is a finite collection of sets $\Ki \subseteq \Dclos$ such that
    \ben
  \item $\interior{\Ki} \cap \interior{\Kj} = \emptyset$ for $ i \neq j,$
  \item $\bigcup_i \Ki = \Dclos,$ and
    \item Something about triangles that works in 3-D too\optodo{Find ref - Brezzi/Johnson?}.
    \een
    \ede


    A crucial quantity in our analysis of finite-element methods will be the mesh size of the triangulation.\optodo{Say somewhere that `triangulation' and`'mesh' are interchangable}

\bde[Mesh Size]
Given a triangulation $\cT = \set{\Ki}$ of a domain $D$, the \defn{mesh size} of $\cT$ is defined to be
\beqs
h = \max_{\Ki} \diam{\Ki}.
\eeqs
\ede\optodo{Adapted from B\&S 4.4.13}

    We can now define finite-element spaces associated with a trinagulation. Throughout this thesis, we only consider the $h$-finite-element method, i.e., the degree $p$ of the polynomials associated with the space is assumed fixed, and we consider refining $h.$
    \bde[Finite element space of degree $p$]
    Given a triangulation $\set{\Ki}$ of a domain $D$, the \defn{(continuous) piecewise-polynomial finite-element space of degree $p$} associated with $\set{\Ki}$ is
    \beqs
\Vhp \de \set{\vh : D \rightarrow \CC \st \vh \restrict{\Ki} \in \polyp{\Kiclos} \tforall \Ki \tand \vh \in \CzD}
    \eeqs\optodo{Check whether we want more than continuity for higher order}
    \ede

    The following \namecref{lem:fespaceapprox} shows how the approximation properties of the space $\Vhp$ depend on $h$ and $p$:

    \ble[Approximation properties of $\Vhp.$]\label{lem:fespaceapprox}
    \optodo{Find what this actually is, and reference it}
    There exists $C>0$ and $\vhptilde \in \Vhp$ such that for all $m \leq p+1$ if $v \in \HmD,$ then for all $s \leq m$\optodo{Do I mean integers here?}
    \beqs
\NLtD{v - \vhptilde} \leq C h^s \NHsD{v}
    \eeqs
    and
    \beqs
\NHoD{v - \vhptilde} \leq C h^s \NHsD{v}.
    \eeqs
    \ele

    \bre[The function $\vhptilde$]
The function $\vhptilde$ in \cref{lem:scottzhang} is constructed using `averaged Taylor polynomials', see, e.g., \cite{ScZh:90},\cite[Section 4.4]{BrSc:08} for details.
    \ere

    With the concept of a finite-element space in place, we can now define the finite-element approximation to the variational problems \cref{prob:vedp,prob:vtedp}.

    \bprob[Finite-element approximation of \cref{prob:vtedp}]\label{prob:fevtedp}
    Let $\Vhp \subset \HozDDR$ be a finite-element space. We say that $\uh \in \Vhp$ is the finite-element approximation of $u$ (the solution to \cref{prob:vedp} if
    \beqs
    a(\uh,\vh) = L(\vh) \tforall \vh \in \Vhp.
    The finite-element approximation of \cref{prob:vedp} is defined analagously.
    \eeqs
    \eprob

    \bre[Not considering variational crimes]
    Observe that \cref{prob:fevtedp} requires $\Vhp \subseteq \HozDDR.$ This inclusion is true if $\DR$ can be triangulated; otherwise, we must modify the definition of \cref{prob:fevtedp} and commit a variational crime by approximating the boundary of $\DR$ by a polygon, or introducing mesh elements with curved boundaries; see, e.g.\optodo{Find refs for Helmholtz, otherwise Brenner and Scott has some of this}. In this thesis we will ignore such variational crimes, and the additional errors they induce; such analysis is standard, and orthogonal to the analysis in this thesis.
    \ere
    

    \subsection{Error bound for the heterogeneous IIP}\label{sec:errbound}

    We now move on to present new work---bounds on the error $\uh-u$ between the finite-element approximation and the true solution of \cref{prob:vtedp}. These bounds, proven for the Helmholtz equation in \emph{heterogeneous} media are generalisations of results already in the literature that the finite-element error is bounded (as $k\rightarrow \infty$) provided $h \lesssim k^{-(2p+1)/2p}$. They will also be crucial for our analysis of the multi-level Monte Carlo method for the Helmholtz equation in \cref{chap:mlmc}. The proof uses a so-called `elliptic projection' technique, where the variational; formulation of a PDE related to \cref{pron:vtedp} is used as part of the proof. 

We will only prove results for \cref{prob:fevtedp}, as our proof uses properties of the related PDE used in the elliptic projection; the required results for this PDE have only been proven with impedance boundary conditions, in the recent preprint \cite{ChNiTo:18}, and not with an exact Dirichlet-to-Neumann map on the truncation boundary. However, whilst we only prove results with impedance boundary conditions, we imagine our results could be extended to an exact Dirichlet-to-Neumann boundary condition, if the required results on the related PDE were available.

\subsection{Discussion of FEM for Helmholtz}\label{sec:helmfedisc}

We will first provide a discussion of error bounds for finite-element methods for the Helmholtz equation. We will give some intuition behind them, provide a brief history of their development, and constrast them with error bounds for the stationary diffusion equation.

Recall from \cref{sec:numsolve} that if one takes the mesh size in the finite element method $h \sim 1/k$ (for first-order finite elements), then the interpolation (or best approximation) error is bounded uniformly in $k$; see \optodo{Add this in when it's done} below for more detail. As explained in \cref{sec:numsolve}, this restriction ensures there are a fixed number of discretisation points per wavelength of the solutiuon. This restriction can also be motivated by the Nyquist--Shannon sampling theorem\optodo{Find Shannon Proc I. R. E., 31:10-21, '49 Communication in the prescence of noise} (see, e.g., \cite[\S 5.21]{BaNaBe:00}) that states that any function $v$ (in 1-d) whose Fourier transform lies inside $[-\lambda,\lambda]$ (for some $\lambda >0$) is completely determined (via its Fourier series) by the point values $v(0)$, $v(\pm \mu),$ $v(\pm2\mu),  \ldots$, for any $\mu < 1/2\lambda.$

Therefore, since the solution of the one-dimensional Helmholtz equation with constant coefficients is
\beq\label{eq:hh-1d}
\uz(x) = A \sin(kx) + B \cos(kx),
\eeq
(for some constants $A$ and $B$), if $\uz(x)$ is sampled at regularly-spaced points that are strictly closer together than $\pi/k$ (as $\lambda = 2\pi/k$ here), then $\uz$ can be reconstructed perfectly from these samples. In conclusion, one expects the be able to interpolate the solution of the Helmholtz equation with uniform error in $k$ if $h \sim 1/k.$

However, as was stated in \cref{sec:numsolve}, the finite-element method for the Helmholtz equation suffers from pollution; and $h \sim 1/k$ is not a sufficient condition to keep the finite-element error bounded uniformly as $k\rightarrow \infty.$ Therefore we will now provide an overview of previous work giving conditions under which the finite-element error is bounded as $k\rightarrow \infty$, as well as conditions under which the finite-element method is quasi-optimal.

We will first define the properties of the finite-element method that we wish to investigate.
\bde
Let $(\Th)_{h \in (0,1)}$ be a \optodo{Double check this is the right terminology, and define it somewhere if it is}shape-regular family of tranigulations of $\DR$, where we assume $h$ is a function of $k$ (e.g., $h(k) = 1/k$). We say that \defn{the error in the finite-element method is bounded uniformly in $k$} if for all $A,n$ in some class there exists $C(A,n)$ (but independent of $k$ and $h$) such that
\beqs
\NW{u-\uh} \leq C\NLtD{f}.
\eeqs
\optodo{Include impedance data?}
\optodo{Does it need to be this formal?}
\ede

A more natural quantity to consider would be the relative error $\NW{u-\uh}/\NW{u};$ and to investigate the conditions on $h$ which enable the relative error to be bounded as $k\rightarrow \infty.$ However, current technology only allows us to prove results for the error, not the relative error, and so we investigate the error instead.

A natural question to ask is: `What mesh condition will keep the relative error bounded as $k \rightarrow \infty'?$

We first provide a brief overview of the literature around this question. In \cite{BaGoTu:85}, Bayliss, Golstein, and Turkel showed empirically that taking $h \sim k^{-3/2}$ with first-order finite elements kept the error bounded. In \cite{IhBa}\optodo{Which one?}\optodo{What about the other one/the book?} Ihlenburg and Babuska proved for first-order elements if $h \sim k^{-1},$\optodo{lesssim?} then the error itself $\sim h^2 k^3$ (a quantity which is bounded if $h \lesssim k^{-3/2}$). However, their proof used special properties\optodo{what exactly?} of the 1-dimensional Green's function for the Helmholtz equation, and so has not been generalised to higher dimensions. Recent work proving results on bounded finite-element error (without first recoursing to proving quasi-optimality of the finite-element method, which will be discussed below) have used the idea of an elliptic projection; that is, using a related PDE as part of the proof process, similar to duality arguments for proving finite-element error bounds in the more well-known case of the stationary diffusion equation.

This technique was used for the Helmholtz equation in homogeneous media in dimensions 2 and 3 in \optodo{Feng Wu?}\cite{Wu:14}, where it was proved that the error in the first-order finite-element meothd $\sim h^2k^3$ if $h \sim k^{-3/2}$ (and therefore the error is bounded); this results was shown for a discontinuous-Galerkin first-order method for the Helmholtz equation, and the results for the standard finite-element method were obtained as a special case; the results in \cite{Wu:14} were extended in \cite{DuWu:15} to higher-order finite-elements, showing that the error $\sim h^{2p}k^{2p+1}.$ Similar results were obtained in \cite{ChNi:18} for Helmholtz problems with corner singularities (and therefore there were additional contraints on the mesh arising from the corner singularities, that we do not mention here). The first results for heterogeneous media were obtained for first-order finite elements in \cite[Lemma 3.3]{WuZo:18} for a special case of heterogeneous media, showing the error is bounded if $h \lesssim k^{-3/2}$; these results were used as part of an argument proving similar results for a nonlinear Helmholtz equation.

Observe that the results proved above for higher-order finite elements (error is bounded if $h^{2p}k^{2p+1}$ is bounded) become less stringent as $p$ increases (the condition translates to requring $h \lesssim k^{-1-1/2p}$). In this section we present the first such results for general classes of heterogeneous media; that is, that the error in the finite-element method is bounded if $h \lesssim k^{-1-1/2p}$.

Before presenting these results, we briefly outline results concerning the related property of quasi-optimality. Recall that the finite-element method is quasi-optimal if there exists $C>0$, independent of $h$ such that
\beqs
\N{u-\uh} \leq C \inf_{\vh \in \Vhp} \N{u-\vh};
\eeqs
i.e., up to a constant, $\uh$ is the best approximation to $u$ in the space $\Vhp$ in the norm $\N{\cdot}.$

In order to present results for the Helmholtz equation, we contrast with the stationary diffusion equation \eqref{eq:stdiff}. For the stationary diffusion equation, one immediately obtains quasi-optimality for emph{any} mesh by C\'ea's Lemma\optodo{ref} (and then obtains that the relative error is bounded by properties of particular members of the finite-element space as in, e.g., \cref{lem:fespaceapprox}. We emphaise again that this holds for any mesh, with no restriction on $h.$

For the Helmholtz equation, proving quasi-optimality is more tricky; C\'ea's Lemma relies on the coercivity of the sesquilinear form in question; in general the sesquilinear forms arising from standard discretisations of the Helmholtz equation are not coercive. Therefore, one instead uses the so-called Aubin--Nitsche duality argument to prove quasi-optimality. This duality argument was first introduced by Nitsche \cite{Ni:67} and Aubin \cite{Au:68} for coercive problems, and applied to problems satisfying a G\r{a}rding inequality by Schatz \cite{Sc:74}. This argument was first used for Helmholtz problems by Melenk in his PhD thesis \cite[Proposition 8.2.7]{Me:95}.

Using this so-called Aubin--Nitsche argument, one obtains quais-optimality for the Helmholtz equation under the incredbily restrictive mesh condition $h \lesssim k^{-2}.$ (Observe that for large values of $k,$ this mesh condition is prohibitive---it would result in linear systems that are too large to solve, even on modern high-performance computers. The derivation of analagous conditions for quasi-optimality for heterogeneous problems (under the assumption that the problem is nontrapping) have been conducted in \cite[Remark 4.4 b.]{GrSa:18}, \cite[Theorem 3]GaSpWu:18}.

In conclusion, the rigorous numerical analysis of the Helmholtz equation is more theoretically demanding than for the stationary diffusion equation. Furthermore, the restrictions one obtains on usable mesh sizes mean that finite-element methods for high-frequency Helmholtz problems are very computationally demanding.

\optodo{Put size of linear systems in}
\optodo{Put in poor performance of GMRES?}
\optodo{Put in need for preconditioners?}


%%5 Can put the below somewhere %%

%% To keep the finite-element error bounded when solving \eqref{eq:introdet}, one must over-refine the numerical grid. That is, rather than using a fixed number of points per wavelength, one must increase the number of points per wavelength as $k$ increases. To achieve bounded finite-element error, one must refine the finite-element mesh size $h$ like $k^{-3/2}$. Whilst this result has been known numerically for some time, it was proven for \eqref{eq:introdet} only with constant coefficients (on various domains and for various finite-element spaces) in \cite{IhBa:95a,Wu:14,DuWu:15,ChNi:18}, and the first proof (to our knowledge) for \eqref{eq:introdet} with heterogeneous coefficients  is contained in \cref{chap:background}. Choosing $h \sim k^{-3/2}$ means \eqref{eq:intromat} is a linear system of size $k^{3d/2}$, larger than if one merely wants the interpolation error to be bounded. Hence, requiring a bounded finite-element error gives rise to very large linear systems.

%% More briefly, if one wants the finite-element solution to be quasi-optimal (that is, up to a constant, the finite-element solution is the best approximation in the finite-element space), then one must over-refine even more, and take $h \sim k^{-2}$. This mesh condition will give rise to linear systems with $k^{2d}$ degres of freedom. See \cref{chap:background} for further details on the necessity of this mesh condition, and further discussion of all the mesh conditions discussed above.\optodo{EDIT THE ABOVE TO REMOVE `OVER-REFINE'}

\subsection{New error bounds for the Helmholtz equation in heterogeneous media}\label{sec:heterr}
\optodo{Garding ineq}
In this section, we prove that the finite-element approximation of the solution to the Helmholtz TEDP exists if $ h \lesssim k^{-3/2}.$ Moreover, we give an expression for the hidden constant that is completely explicit in $A$ and $n$, and we also prove a bound on the finite-element error, again completely explicit in $A$ and $n$. The argument in this section closely follows those in \cite{FeWu:11,ChNi:18} in its use of an elliptic projection argument to prove the required finite-element existence result and error bound. The paper \cite{FeWu:11} proved a similar result for the Helmholtz equation in homogeneous media, and \cite{ChNi:18} does so for the homogeneous Helmholtz equation with corner singularities.

Whilst we prove the results in this section for the TEDP, we expect that they can be extended to the Helmholtz Exterior Dirichlet Problem (EDP) where the infinite domain is truncated, and the Dirichlet-to-Neumann map is realised exactly on the truncated boundary. However, our proof below uses recently-proved bounds on the solution of a related problem to the TEDP from \cite{ChNiTo:18}; in order to extend our results to the EDP we would need analogues to the results in \cite{ChNiTo:18} for the EDP.

\paragraph{Problem Set-up} Let $\Dm$ be a bounded Lipschitz\ednote{We actually need this to be a $C^{k,\lambda}$ set, for $k+\lambda > 1.5,$ so that we can do the whole non-zero Dirichlet data thing. This is getting a bit complicated. I guess our options are (i) persevere, (ii) give up and just do the theory for zero Dirichlet data, or (iii) assume that we know $\ud,$ not just $\gD.$ Thoughts?} open set such that the open complement $\Dp\de \RRd\setminus \Dmclos$ is connected. Let $\Dtilde$ be a bounded connected Lipschitz open set such that $\Dmclos \subset\subset\Dtilde$. 
Let $D\de\Dtilde\setminus\Dm$, $\GD\de \partial \Dm$, and $\GI \de\partial \Dtilde$, so that $\partial D= \GD \cup \GI$ and $\GD\cap \GI = \emptyset$. Throughout $\tr$ will denote the trace onto the whole boundary $\dD,$ whereas $\trGI$ and $\trGD$ will denote the traces on $\GI$ and $\GD$ respectively. Throughout we assume there exists some $\kz > 0$ such that $k \geq \kz$. Let $\NW{v}$ denote the weighted $H^1$ norm on $\HoD$:
\beqs
\NW{v}^2 \de \NLtD{\grad v}^2 + k^2 \NLtD{v}^2.
\eeqs


Let
\bit
\item $f\in \LtD$ 
\item $\gD\in \HthtGD$,
\item $\gI\in \LtGI$
\item $n\in \LiDRR$ such that $\dist\mleft(\supp\mleft(1-n\mright),\GI\mright)>0$, satisfying
\beq
0<\nmin \leq n\mleft(\bx\mright)\leq\nmax<\infty\,\, \text{ for almost every } \bx \in D,
\eeq
\item $A \in \WoiDRRdtd$ such that $\dist\mleft(\supp\mleft(I -A\mright),\GI\mright)>0$, $A$ is symmetric, and there exist $0<\Amin\leq \Amax<\infty$ such that
\beq\label{eq:AellEDP}
 \Amin |\bxi|^2\leq\mleft(A\mleft(\bx\mright) \bxi\mright) \cdot \overline{ \bxi}  \leq \Amax|\bxi|^2 \quad\text{ for almost every }\bx \in D \text{ and for all } \bxi\in \CCd.
\eeq
\eit
%we say $u\in \HoD$ satisfies the Helmholtz Truncated Exterior Dirichlet Problem (TEDP) if 
%\beqs
%\grad\cdot\mleft(A \grad u \mright) + k^2 n u = -f \quad \tin D,
%\eeqs
%\beqs
%\trGD u =\gD \quad\ton \GD,
%\eeqs
%and 
%\beq\label{eq:TEDP3}
%\dn u - \ii k  \trGI u = \gI \ton \GI.
%\eeq
In order to study the TEDP with $\gD\neq0,$ we must, in essence reformulate to the TEDP with $\gD=0$ but a different right-hand side for the domain term.

Define the space
\beqs
\HozDD \de \set{v \in \HoD \st \trGD u = 0}.
\eeqs
%and the sesquilinear form and antilinear functional
%The variational formulation of the TEDP with $\gD = 0$ is%\optodo{Check exactly what's needed in hetero}
%
%\beq\label{eq:tedpz}
%\text{Find } u \in \HozDD\quad \tst\quad a(u,v) = F(v)\quad \tfa v \in \HozDD,
%\eeq
%
%where
%
%\beqs
%a(u,v) \de \int_D \mleft(A \grad u\mright)\cdot \grad \vb - k^2 n u\vb - ik \int_{\GI} \trGI u \trGI \vb\quad \tand\quad F(v) \de \int_D f\vb + \int_{\GI} \gI\trGI \vb.
%\eeqs
%
In order to deal with non-zero Dirichlet data $\gD,$ we let  $\ud \in \HtD$ be such that $\trGD \ud = \gD$, and $\esssup \ud \compcont D$. The proof that such a $\ud$ exists is in \cref{lem:ud}.
The variational formulation of the TEDP is then
\beq\label{eq:tedp}
\text{Find } u \in \HozDD\quad \tst\quad a(u,v) = F(v)\quad \tfa v \in \HozDD,
\eeq
where
\beqs
a(u,v) \de \int_D \mleft(A \grad u\mright)\cdot \grad \vb - k^2 n u\vb - ik \int_{\GI} \trGI u \trGI \vb
\eeqs
and
\beqs
F(v) \de  \int_D \mleft(f - \grad \cdot \mleft(A\grad \ud\mright) - k^2 n\ud\mright)\vb + \int_{\GI} \mleft(\gI-\dn\ud\mright)\trGI \vb.
\eeqs
The function $\us = u+ \ud$ is then the solution of the Helmholtz equation
\beqs
\grad\cdot\mleft(A \grad \us \mright) + k^2 n \us = -f \quad \tin D,
\eeqs
\beqs
\trGD \us =\gD \quad\ton \GD,
\eeqs
and 
\beq\label{eq:TEDP3}
\dn \us - \ii k  \trGI \us = \gI \ton \GI.
\eeq
\bre[Reducing the smoothness of $\gD$]
The assumption that $\gD \in \HthtGD$ is made so that the lifting $\ud$ of $\gD$ is in $\HtD$ (see \cref{app:ud}). As $\ud \in \HtD,$ the antilinear functional $F$ defined above is well-defined. We could reduce the smoothness of $\gD$ to $\HoGD$ (meaning $\ud \in \HthtD$) but this reduction in smoothness would then require us to reformulate the functional $F$ as
\beqs
F(v) = \int_D \mleft(A \grad \ud\mright)\cdot \grad \vb - k^2 n \ud \vb + f \vb + \int_{\GI} \mleft(\gI - \dn \ud\mright)\vb.
\eeqs\optodo{Put a proof of this somewhere, in 28/2/19 notes}
With this reformulation, $F \in \HozDDprime,$ but does not have a representative function in $\LtD$. Our proofs below will use results from \cite{ChNiTo:18}, which are stated for the TEDP with zero Dirichlet boundary condition and $L^2$ right-hand side. To avoid the complications stated above, and to allow us to use the results in \cite{ChNiTo:18}, we therefore impose the additional smoothness on $\gD.$ Also, in the case with $F$ only in $\HozDDprime$, proving a priori bounds on the solution of the TEDP is more complicated (c.f., e.g., \cite[Theorem 2.5]{GrPeSp:18} and \cite[Corollary 2.16]{GrPeSp:18} which consider the analogous EDP). For the same reason, we assume $A \in \WoiDRRdtd;$ if we only had $A \in \LiDRRdtd,$ we could reformulate $F$ as outlined above, but we would have the same complications as just described.

The fact that we cannot reduce the smoothness of $\gD$ further to $\HhGD$ is due to the Morawetz multiplier techniques used to obtain the a priori bounds in \cite{GrPeSp:18}, see \cite[(iii), p. 2874]{GrPeSp:18}.
\ere
\optodo{Probably need to prove this---I guess it'll go along similar lines as the proof in hetero}
Also, for later use we state the \defn{adjoint} problem.
\beq\label{eq:tedpadj}
\text{Find } u \in \HozDD\quad \tst\quad \aadj(u,v) = F(v)\quad \tfa v \in \HozDD,
\eeq
where
\beqs
\aadj(u,v) \de \int_D \mleft(A \grad u\mright)\cdot \grad \vb - k^2 n u\vb + ik \int_{\GI} \trGI u \trGI \vb.
\eeqs
%and
%\beqs
%\Fadj(v) \de \aadj(\uz,v) + \int_D f\vb + \int_{\GI} \gI\trGI \vb.
%\eeqs

The statement of the main result requires the following related sesquilinear form and \lcnamecref{lem:relatedwp}.

\bde[Related sesquilinear form]
For $\vo, \vt \in \HozDD$ we define
\beqs
\api(\vo,\vt) = \int_D \IP{A \grad \vo}{\vt} - ik\int_{\GI} \vo\vtbar.
\eeqs
\ede

\ble[Related PDE is well-posed and solution is in $H^2$]\label{lem:relatedwp}
If $A \in \CzoDRRdtd,$ then the solution $\psi \in \HozDD$ of the related PDE
\beq\label{eq:relpde}
\api(u,v) = \IPLtD{f}{v}\quad \tfa\quad v \in \HozDD
\eeq
exists, is unique, is in $\HtD,$ and satisfies the a priori bound
% \beqs\label{eq:relpdehobound}
% \NW{\psi} \lesssim \frac{\max\set{\Amin^{-1},1}}k,
% \eeqs
% and
% \beqs
% \NHoD{\psi} \lesssim \CHoell \NLtD{f}
% \eeqs
% and
\beqs
\NHtD{\psi} \lesssim \CHtell \NLtD{f}.
\eeqs
for some constant $\CHtell > 0$ depending on $A,$ but independent of $k.$
\ele

\bre[Proof of \cref{lem:relatedwp}]
\Cref{lem:relatedwp} is proved in \cite{ChNiTo:18}, although the dependence on $A$ is not made explicit.
\ere

%\paragraph{Finite-Element Set-up} Let $\Vh$ be the first-order linear finite-element space on some mesh on $D$ with mesh size $h.$

\bas[Existence, uniqueness, and an a priori bound]\label{ass:bound}
We assume that the coefficients $A$ and $n$ are such that for all $k \geq \kz$ the solutions of the TEDP \eqref{eq:tedp} and its adjoint \eqref{eq:tedpadj} exist, are unique, are in $\HtD,$ and satisfy the bound
\beq\label{eq:hhbound}
\NHtD{u} \lesssim \CHthh \,k \mleft(\NLtD{f} + \Nunsure{g} + \NLtGD{\gradGD \gD} + k \NLtGD{\gD}\mright),
\eeq
where $\gradGD$ is the surface gradient on $\GD,$ $u$ is the solution of the TEDP or its adjoint, and $\CHthh >0$ is a constant dependent on $A$, $n,$ and possibly $k.$
% \footnote{Determining the dependence of $\CHthh$ on $A$ and $n$ could be tricky. It was done in \cite{ChScTe:13} for a $C^2$ domain with scalar $A$ and homogeneous Dirichlet boundary conditions.}
 \eas

\bde[Finite-element approximation] 
 The finite-element approximation to \eqref{eq:tedp} is the following:
\beq\label{eq:tedpfe}
\text{Find } \uh \in \Vh\quad \tst\quad a(\uh,\vh) = F(\vh)\quad \tfa \vh \in \Vh,
\eeq
\ede

The main theorem we prove is the following:

\bth[Finite-element-error bound]\label{thm:febound}
If $A \in \CzoDRRdtd,$ $h \lesssim 1/k,$ \cref{ass:bound} holds, and
\beq\label{eq:hcond}
h \lesssim \mleft(\NLiDRR{n} \mleft(\Amax + \half\mright)\CHtell \CHthh\mright)^{-1/2}k^{-3/2}, % There should be a factor of a half in front of the right-hand side of this, as it makes things clearer what's going on in the proof. However, since we're doing everything with \lesssim, a factor of a half doesn't matter. We could replace the half with any \eps in (0,1), but then the constant hidden in the \lesssim in \eqref{eq:hherrltbound} has a factor 1/\eps.
\eeq
the finite-element solution $\uh$ to the problem \eqref{eq:tedpfe} exists, is unique, and satisfies the bounds
\beq\label{eq:hherrltbound}
\NLtD{u-\uh} \lesssim \Cfemo \mleft(hk\mright)^2 \mleft(\NLtD{f} + \Nunsure{\gI} + \NLtGD{\gradGD \gD} + k \NLtGD{\gD}\mright)
\eeq
and
\beq\label{eq:hherrwbound}
\NW{u-\uh} \lesssim \mleft(\Cfemt hk +  \Cfemth h^2k^3\mright)\mleft(\NLtD{f} + \Nunsure{\gI} + \NLtGD{\gradGD \gD} + k \NLtGD{\gD}\mright),
\eeq
where
\beqs
\Cfemo \de \mleft(\Amax + \half\mright)\CHthh^2,
\eeqs
\beqs
\Cfemt \de \frac{\Amax+\half}{\Amin} \CHthh,
\eeqs
\beqs
\Cfemth \de \frac{\mleft(\Amin+ \NLiDRR{n}\mright)^{1/2}}{\Amin^{1/2}}\Cfemo,
\eeqs
and $u$ is the solution of the TEDP \eqref{eq:tedp}.
\enth

\subsubsection{Properties of the Elliptic Projection, and a related PDE}

The proof technique we use below (adapted from \cite{FeWu:11,ChNi:18}) uses an `elliptic projection' of the solution of the TEDP using the related sesquilinear form $\api.$ We define the energy norm induced by the sesquilinear form $\api$:
\beqs
\Npi{\vo} = \sqrt{\abs{\api(\vo,\vo)}}.
\eeqs

\ble[Energy Norm is a norm]\label{lem:inducednorm}
The induced norm $\Npi{\cdot}$ is a norm on $\HoD.$
\ele

\bpf[Proof of \cref{lem:inducednorm}]
The main thing to check is that, for $v \in \HoD,$ $\Npi{v}=0 \implies v=0.$ By construction, if $\Npi{v}=0,$ then $\int_{D} \IP{A \grad v}{\grad v} =0$ and $\NLtGI{v}^2 = 0,$ as these are the real and imaginary parts of $\api(v,v).$ By \eqref{eq:AellEDP}, it follows that $\Amin \abs{\grad v}^2 \leq 0,$ and thus $v$ is constant. As $\NLtGI{v} = 0,$ it follows that $\trGI v =0,$ and hence by the trace theorem, as $v$ is constant, it follows that $v=0.$

Other properties of norms follow analagously as with any definition of an energy norm.
\epf
\ble[Energy norm is equivalent to weighted norm]\label{lem:normbound}
If $v \in \HoD,$ then
\beq\label{eq:boundew}
\Npi{v} \lesssim \sqrt{\Amax+\half}\NW{v}
\eeq
and
\beq\label{eq:boundwe}
\NW{v} \lesssim \max\set{\Amin^{-\half},1} \Npi{v}
\eeq
\ele

\bpf[Proof of Lemma \ref{lem:normbound}]
To show \eqref{eq:boundew}, for $ v \in \HoD$ we have
\begin{align*}
  \Npi{v}^2 &= \abs{\api(v,v)}\\
            &\lesssim \abs{\int_{D} \IP{A \grad v}{\grad v}} + k\NLtGI{v}^2 \\
            &\lesssim \abs{\int_{D} \IP{A \grad v}{\grad v}} + k\NLtD{v}\NHoD{v}, \text{ by the multiplicative trace inequality}\\
            &\lesssim \Amax \NLtD{\grad v}^2 + \half k^2 \NLtD{v}^2 + \half \NHoD{v}^2\\
  &\lesssim \mleft(\Amax+\half\mright)\NW{v}^2
\end{align*}
as required.

To show \eqref{eq:boundwe} we first show that, for $v \in \HoD,$ $\Npi{v} \gtrsim \min\set{\Amin^{\half},1} \mleft(\NLtD{\grad v} + k^{\half} \NLtGI{\trGI v}\mright)$:
\begin{align}
  \Npi{v} &= \mleft(\abs{\api(v,v)}\mright)^{\half}\nonumber\\
          &= \mleft(\mleft(\int_D \IP{A \grad v}{\grad v}\mright)^2 + k^2 \mleft(\int_{\GI}\abs{\trGI v}^2\mright)^2\mright)^{\quarter}\nonumber\\
  &\geq \mleft(\mleft(\int_D \Amin \abs{\grad v}\mright)^2 + k^2 \NLtGI{\trGI v}^4\mright)^{\quarter}\nonumber\\
          &= \mleft(\Amin^2 \NLtD{\grad v}^4 + k^2 \NLtGI{\trGI v}^4\mright)^{\quarter}\nonumber\\
          &\geq \min\set{\Amin^{\half},1}\mleft(\NLtD{\grad v}^4 + k^2 \NLtGI{\trGI v}^4\mright)^{\quarter}\nonumber\\
  &\gtrsim \min\set{\Amin^{\half},1} \mleft(\NLtD{\grad v} + k^{\half} \NLtGI{\trGI v}\mright), \text{ as } \mleft(x+y\mright)^4 \lesssim x^4 + y^4.\label{eq:Npifour}
\end{align}

We recall the fact that for $v \in \HoD,$
\beq\label{eq:poincarelike}
\NLtD{v} \lesssim \NLtD{\grad v} + \NLtGI{\trGI v},
\eeq
see, e.g., \cite[Equation (6.16)]{Sp:15}. We can then prove \eqref{eq:boundwe}:
\begin{align*}
   \NW{v} &\lesssim \NLtD{v}+ \NLtD{\grad v}\\
          &\lesssim \NLtGI{\trGI v}+ \NLtD{\grad v} + \NLtD{\grad v}\text{ by \eqref{eq:poincarelike}}\\
          &\lesssim k^{\half}\NLtGI{v} + \NLtD{\grad v}\\
  &\lesssim \max\set{\Amin^{-\half},1}\Npi{v}, \text{ by \eqref{eq:Npifour}.}
\end{align*}
\epf

% \ble[Bound on $L^2$ norm]\label{lem:ltbound}
% If $v \in \HoD$ then the bound
% \beqs
% \NLtD{v} \lesssim  \NLtGI{\trGI v} + \NLtD{\grad v}
% \eeqs
% holds.
% \ele

% \bpf[Proof of \cref{lem:ltbound}]
% \optodo{Look at proof in IbyPs article}
% If $v \in \HozDD,$ then by the Poincar\'e inequality, we have that $\NLtD{v} \lesssim \NLtD{\grad v}.$ Alternatively, if $\GD = \emptyset$ and $\trGI v$ is constant, then $v - \trGI v \in \HozDD,$ and thus (abusing notation, and letting $\trGI v$ denote the value of the constant, and also a constant function defined on $D$ taking that value everywhere)
% \begin{align*}
%   \NLtD{v} &\leq \NLtD{\trGI v} + \NLtD{v-\trGI v}\\
%   &= \NLtGI{\trGI v} + \NLtD{v-\trGI v}\\
%            &\lesssim \NLtGI{\trGI v} +  \NLtD{\grad \mleft(v-\trGI v\mright)}\\
%              &= \NLtGI{\trGI v} + \NLtD{\grad v},
% \end{align*}
% as required.
% \epf

% \ble[Bound on weighted norm by energy norm]\label{lem:othernormbound}
% If $v \in \HozDD,$ or if $\GD = \emptyset$ and $\trGI v$ is constant, the bound
% \beq\label{eq:boundwe}
% \NW{v} \lesssim \max\set{\Amin^{-\half},1} \Npi{v}
% \eeq
% holds.
% \ele

% \bpf[Proof of \cref{lem:othernormbound}]
% \epf

We now define the elliptic projection of a function in $\HoD.$% and also define a related PDE that will be used in proving the approximation properties of the elliptic projection.

\bde[Elliptic Projection]
For $w \in \HoD$ we define the \defn{elliptic projection} $\Ph w \in \Vh$ of $w$ by
\beq\label{eq:ellproj}
\api(\vh,\Ph w) = \api(\vh,w) \tfa \vh \in \Vh.
\eeq
\ede

% \bde[Related PDE]\label{lem:relpde}
% Given $f \in \LtD$ we define the related (adjoint) PDE; find $\psi \in \HoD$ such that for all $v \in \HoD$
% \beq\label{eq:relpde}
% \api(\psi,v) = \IPLtD{f}{v}.
% \eeq\ede

% \bpf[Proof of \cref{lem:relatedwp}]
% By \eqref{eq:boundwe} we have, for $v \in \HozDD$
% \beqs
% \min\set{\Amin,1}\NW{v}^2 \lesssim \abs{\api(v,v)},
% \eeqs
% and we also have that
% \beqs
% \NWs{\IP{f}{\cdot}} \leq \frac1k \NLtD{f},
% \eeqs
% where $\NWs{\cdot}$ denotes the norm on $\HozDDs$ induced by $\NW{\cdot}.$ By the Lax--Milgram Theorem, we can therefore conclude that $\psi$ exists, is unique, and satisfies the bound
% \beqs
% \NW{\psi} \lesssim \frac{\max\set{\Amin^{-1},1}}{k}\NLtD{f}.
% \eeqs
% Use Grisvard Magic to get $H^2.$\optodo{this}
% \epf

\ble[Properties of elliptic projection]\label{lem:ellprojbounds}
Let $A \in \CzoDRRdtd.$ If $w \in \HtD,$ then the elliptic projection $\Ph w$ exists, is unique, and the error satisfies the bounds
\beq\label{eq:ellprojenbound}
\Npi{w-\Ph w} \lesssim \sqrt{\Amax+\half}\,h\NHtD{w},
\eeq
and
\beq\label{eq:ellprojltbound}
\NLtD{w-\Ph w} \lesssim  \mleft(\Amax+\half\mright)\CHtell\,h^2\NHtD{w}.
\eeq
\ele

\bpf[Proof of \cref{lem:ellprojbounds}]
We first assume $\Ph w$ exists. To show \eqref{eq:ellprojenbound} we apply C\'{e}a's Lemma in $\Vh$ using the energy norm $\Npi{\cdot}$ to conclude
\beqs
\Npi{w-\Ph w} \leq \Npi{w-\Ih w}.
\eeqs
We then apply \cref{lem:normbound,lem:scottzhangbound} to conclude \eqref{eq:ellprojenbound}.

To prove \eqref{eq:ellprojltbound} we let $\psi$ solve the related PDE (\cref{lem:relpde}) with $f = w-\Ph w.$ By \cref{lem:relatedwp} $\psi \in \HtD$ and thus by  \cref{lem:normbound} and \cref{lem:scottzhangbound}
\beqs
\Npi{\psi - \Ih \psi} \lesssim \sqrt{\Amax + \half}\CHtell \,h\NLtD{w-\Ph w}.
\eeqs

If we now set $v = w-\Ph w$ in \eqref{eq:relpde}, then we obtain
\begin{align}
  \NLtD{w - \Ph w}^2 &= \api\mleft(\psi,w-\Ph w\mright)\nonumber\\
                     &= \api\mleft(\psi-\Ih \psi,w-\Ph w\mright) \text{ by Galerkin orthogonality for } w-\Ph w\nonumber\\
                     &\leq \Npi{\psi-\Ih \psi}\Npi{w-\Ph w}\nonumber\\
                       &\lesssim \sqrt{\Amax + \half}\CHtell \,h\NLtD{w-\Ph w}\Npi{w-\Ph w}\label{eq:epltfinal}.
\end{align}
By cancelling $\NLtD{w- \Ph w}$ from both sides of \eqref{eq:epltfinal} and using \eqref{eq:ellprojenbound} we obtain \eqref{eq:ellprojltbound}.

We have proved the bounds \eqref{eq:ellprojenbound} and \eqref{eq:ellprojltbound} under the assumption of existence. To show uniqueness, suppose $\wh, \whtilde$ both satisfy \eqref{eq:ellproj} (with $\Ph = \wh$ or $\whtilde$ respectively). Then by linearity, for all $\vh \in \Vh,$
\beqs
\api\mleft(\vh,\Ph\mleft(\wh-\whtilde\mright)\mright) = \IP{\vh}{w-w} = 0.
\eeqs
That is, the function $\wh - \whtilde$ is an elliptic projection of the zero function.

Therefore, by \eqref{eq:ellprojltbound} $\NLtD{0 - \mleft(\wh - \whtilde\mright)} \lesssim 0,$ i.e., $\wh = \whtilde.$ Therefore, if the elliptic projection $\Ph w$ exists, it is unique. As the space $\Vh$ is finite-dimensional, by the Rank--Nullity Theorem, the uniqueness of $\Ph w$ implies its existence; hence $\Ph w$ exists, and is unique, as required.
\epf


\subsubsection{Proof of Main Result}

We let $\Ih$ denote the Scott--Zhang quasi-interpolant in $\Vh$ (see \cite{ScZh:90}), and will use its following property.

\ble[Properties of Scott-Zhang interpolant]\label{lem:scottzhangbound}
let $h \lesssim 1/k.$ If $w \in \HtD,$ then
\beq\label{eq:scottzhangbound}
\NW{w - \Ih w} \lesssim h \NHtD{w}.
\eeq
\ele

\bpf[Proof of \cref{lem:scottzhangbound}]
The Scott-Zhang interpolant $\Ih w$ satisfies
\beq\label{eq:szlt}
\NLtD{w-\Ih w} \lesssim h^2 \NHtD{w}
\eeq
\and
\beq\label{eq:szho}
\NHoD{w-\Ih w} \lesssim h \NHtD{w}.
\eeq
Hence by the definition of $\NW{\cdot},$ by combining \eqref{eq:szlt} and \eqref{eq:szho} we have
\beqs
\NW{w-\Ih w} \lesssim h\mleft(1+hk\mright)\NHtD{w}.
\eeqs
As $h\lesssim 1/k,$ \eqref{eq:scottzhangbound} follows.
\epf

The following \lcnamecref{cor:hhszbound} follows from \cref{ass:bound}, and is used to prove \cref{thm:febound}.

\bco\label{cor:hhszbound}
If $u$ is the solution of the Helmholtz Interior Impedance Problem (or its adjoint) then the error in the Scott--Zhang quasi-interpolant satisfies
\beq\label{eq:hhszbound}
\NW{u-\Ih u} \lesssim \CHthh hk \mleft(\NLtD{f} + \Nunsure{g} + \NLtGD{\gradGD \gD} + k \NLtGD{\gD}\mright).
\eeq
\eco

The proof of the main theorem (\cref{thm:febound} below) also uses the fact that $a$ satisfies a G\r{a}rding inequality.
\ble[G\r{a}rding inequality]
If $v \in \HozDD,$ then
\beq\label{eq:garding}
\Re\mleft(a(v,v)\mright) \geq \Amin \NW{v}^2 - \mleft(\Amin + \NLiDRR{n}\mright)\NLtD{v}^2,
\eeq
where $\Re$ denotes the real part.
\ele

Finally, we recall \defn{Cauchy's inquality}: For all $a,b \in \RR$, and for all $\eps > 0,$
\beq\label{eq:cauchy}
ab \leq \frac{a^2}{2\eps} + \frac{\eps b^2}{2}.
\eeq

\bpf[Proof of \cref{cor:hhszbound}]
The proof follows from \cref{lem:scottzhangbound,ass:bound}.
\epf

We are now in a position to prove our main theorem.




\bpf[Proof of \cref{thm:febound}]
In this proof, for brevity we let
\beqs
\rhs = \NLtD{f} + \Nunsure{\gI} + \NLtGD{\gradGD \gD} + k \NLtGD{\gD}.
\eeqs
By \cref{ass:bound} the solution $u$ of the TEDP exists and is unique. Assume the finite-element solution $\uh$ exists. Let $\xi \in \HoD$ satisfy the adjoint TEDP \eqref{eq:tedpadj} with $f=u-\uh,$ $\gD=0,$ and $\gI=0.$ Taking complex conjugates, it follows that
\beq\label{eq:errordual}
a(v,\xi) = \IPLtD{v}{u-\uh} \tfa v \in \HoD.
\eeq
By \cref{ass:bound} $\xi$ exists, is unique and is in $\HtD.$. Setting $v = u-\uh$ in \eqref{eq:errordual} we obtain
\begin{align*}
  \NLtD{u-\uh}^2 &= a\mleft(u-\uh,\xi\mright)\\
                 &= a\mleft(u-\uh,\xi-\Ph \xi\mright) \quad\text{by Galerkin orthogonality for } u-\uh\\
                 &= \api\mleft(u-\uh,\xi-\Ph \xi\mright) - k^2 \IPLtD{n\mleft(u-\uh\mright)}{\xi-\Ph \xi}\\
                 &= \api\mleft(u-\Ih u,\xi-\Ph \xi\mright) - k^2 \IPLtD{n\mleft(u-\uh\mright)}{\xi-\Ph \xi}\\
  &\quad\quad\quad\text{by Galerkin orthogonality for }\xi  - \Ph \xi\\
                 &\leq \Npi{u-\Ih u}\Npi{\xi - \Ph \xi} + \NLiDRR{n} k^2 \NLtD{u-\uh}\NLtD{\xi-\Ph \xi}\\
                 &\lesssim \sqrt{\Amax + \half}\, \CHthh\, hk \rhs\Npi{\xi-\Ph \xi}\\
  &\quad\quad+  \NLiDRR{n} k^2 \NLtD{u-\uh}\NLtD{\xi-\Ph \xi}\quad \text{by \eqref{eq:boundew} and \eqref{eq:hhszbound}}\\
                 &\lesssim \sqrt{\Amax + \half}\, \CHthh\, hk\rhs\,\sqrt{\Amax + \half}\,h\NHtD{\xi}\\
                 &\quad\quad  + \NLiDRR{n} k^2 \NLtD{u-\uh}\NLtD{\xi-\Ph \xi}\quad \text{by \eqref{eq:ellprojenbound}}\\
                 &\lesssim \mleft(\Amax + \half\mright)\CHthh\, hk\rhs \, \CHthh \,hk \NLtD{u-\uh}\\
&\quad\quad  + \NLiDRR{n} k^2 \NLtD{u-\uh}\NLtD{\xi-\Ph \xi}\quad \text{by \eqref{eq:hhbound}}\\
\end{align*}
Cancelling a factor of $\NLtD{u-\uh}$ and rearranging terms we obtain
\beqs
\NLtD{u-\uh} \lesssim \mleft(\Amax + \half\mright)\CHthh^2 \mleft(hk\mright)^2\rhs + k^2 \NLiDRR{n} \NLtD{\xi - \Ph \xi}
\eeqs
and therefore
\begin{align*}
&  \NLtD{u-\uh} \lesssim \mleft(\Amax + \half\mright)\CHthh^2 \mleft(hk\mright)^2 \rhs\\
&\quad\quad  + h^2k^3 \NLiDRR{n} \mleft(\Amax + \half\mright) \CHtell \CHthh \NLtD{u-\uh}
\end{align*}
  using the definition of $\xi$, \eqref{eq:ellprojltbound}, and \eqref{eq:hhbound}Therefore if $h$ satisfies \eqref{eq:hcond} we obtain \eqref{eq:hherrltbound}.
% \beqs
% \half \NLtD{u-\uh} \lesssim \mleft(\Amax + \half\mright)\CHthh \mleft(hk\mright)^2 \mleft(\NLtD{f} + \Nunsure{g}\mright).
% \eeqs
% that is, if $\Cmess \de (\mleft(\NLiDRR{n} \mleft(\Amax + \half\mright) \CHthh \mright)^{-1/2},$ then
% \beqs
% \NLtD{u-\uh} \lesssim \mleft(\Amax + \half\mright)\CHthh \Cmess^2k^{-1} \mleft(\NLtD{f} + \Nunsure{g}\mright),
% \eeqs
To obtain the bound \eqref{eq:hherrwbound}, we use the G\r{a}rding inequality \eqref{eq:garding}:
\begin{align*}
  \Amin \NW{u-\uh}^2 &\leq \Re\mleft(a\mleft(u-\uh,u-\uh\mright)\mright) + k^2\mleft(\Amin+ \NLiDRR{n}\mright) \NLtD{u-\uh}^2\\
                     &= \Re\mleft(a\mleft(u-\uh,u-\Ih u\mright)\mright) + k^2\mleft(\Amin+ \NLiDRR{n}\mright) \NLtD{u-\uh}^2\\
  &\quad\quad\quad\text{by Galerkin orthogonality}\\
                     &\leq \mleft(\Amax+\half\mright) \NW{u-\uh}\NW{u-\Ih u} + k^2\mleft(\Amin+ \NLiDRR{n}\mright) \NLtD{u-\uh}^2\\
  &\quad\quad\quad\text{by \cref{lem:normbound}}\\
  &\leq \frac{\mleft(\Amax+\half\mright)^2}{2\Amin} \NW{u-\Ih u}^2 + \frac{\Amin}2 \NW{u-\uh}^2 + k^2\mleft(\Amin+ \NLiDRR{n}\mright) \NLtD{u-\uh}^2,\\
\end{align*}
by Cauchy's inequality \eqref{eq:cauchy} with $\eps = \Amin.$ Therefore,
\beqs
\NW{u-\uh}^2 \leq \frac2{\Amin} \mleft(\frac{\mleft(\Amax+\half\mright)^2}{2\Amin} \NW{u-\Ih u}^2+ k^2\mleft(\Amin+ \NLiDRR{n}\mright) \NLtD{u-\uh}^2\mright)
\eeqs
and hence
\beq\label{eq:hherrboundnearly}
\NW{u-\uh} \lesssim \frac1{\Amin^{1/2}} \mleft(\frac{\Amax+\half}{\Amin^{1/2}} \NW{u-\Ih u}+ k\mleft(\Amin+ \NLiDRR{n}\mright)^{1/2} \NLtD{u-\uh}\mright).
\eeq
By substituting \eqref{eq:hhszbound} and \eqref{eq:hherrltbound} into \eqref{eq:hherrboundnearly} we obtain \eqref{eq:hherrwbound}.

To show that $\uh$ exists, as in the proof of \cref{lem:ellprojbounds} we can use the error bound \eqref{eq:hherrltbound} to show that $\uh$ is unique, and we can then use the fact that $\Vh$ is finite-dimensional to show that $\uh$ exists.
\epf



