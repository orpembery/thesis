%% In this \lcnamecref{sec:comp} we state and prove an abstract result on the convergence of multi-level Monte Carlo methods, laregly following the proof of \cite[Theorem 1]{ClGiScTe:11}. Our result is a generalisation of \cite[Theorem 1]{ClGiScTe:11} in the following three ways:
%% \ben
%% \item In \cite{ClGiScTe:11} it is assumed that the convergence of the approximate QoIs $\Qhl$, and the cost of producing samples of these QoIs, only depends on the parameter $\hl$ (where, in stochastic PDE applications, $\hl$ is the mesh size for the finite-element discretisation). However, in this work, we assume that the convergence and cost also depend on another parameter $k,$ and we make the dependence of the final computational cost of the MLMC method explicit in $k.$ In our application to the Helmholtz equation, $k$ will be the wavenumber of the problem.
%% \item In \cite{ClGiScTe:11} it is assumed that the approximating QoIS $\Qhl$ exist for all levels $l$. This corresponds to the finite-element solution of the PDE under investigation existing for all mesh sizes $h.$ Whilst this assumption is true for the stationary diffusion equation studied in \cite{ClGiScTe:11}, it is \emph{not} true for the Helmholtz equation that we study here. Therefore we make the additional assumption (\cref{ass:qoie} below) that $\Qhl$ only exists for sufficiently small $\hl.$
%% \item In \cite{ClGiScTe:11} the error $\eps$ incurred in the MLMC method is equally divided between the bias and the variance of the MLMC method (see the Proof of \cref{thm:mlmccomp3}). However, in this work we assume that there is a quantity $\splitting \in (0,1)$ (see \cref{ass:splittingbounds}), possibly dependent on $k$ that allows a vairable `split' of the error between the bias and the variance. Our main use of this is in\optodo{Insert refs once it's done}, where we use this variable splitting to compensate for the fact that to bound the (squared) bias error by $\eps^2/2$ would mean we take $\hL \lesssim k^{-1},$ but to ensure the finite-element solution $\uh$ exists, we must take $\hL \lesssim k^{-3/2}.$
%% \een
%% We now proceed to prove our abstract MLMC convergence result, comtaining the generalisations metioned above.

\section{Set up}

We work in the framework of \cref{chap:stochastic}. For $h>0,$ define the random field\optodo{Assume or show it is one} $\uh:\Omega \rightarrow \Vhp$ by letting $\uh(\omega)$ solve \cref{prob:fevedp} with coefficients $A(\omega)$ and $n(\omega)$ (if it exists). Let $Q:\HozDDR\rightarrow\RR$ be a (measurable) \defn{quantity of interest} (QoI) of the solution $u$ (so that $Q(u)$ is a random variable). As an abuse of notation, we also use $Q$ to denote $Q \circ u,$ where the context means this notation is unambiguous. If $\uh(\omega)$ exists, we let $\Qh$ denote $Q \circ \uh.$

As hnited at in the preceeding paragraph, the finite-element solution $\uh(\omega)$ may not always exist. Moreover, the existence (or not) of $\uh(\omega)$ is due to the fact that the constants involved in the definitions of accuracy and data-accuracy of a finite-element method (\cref{def:hkacc,def:hkdatacc} above) are dependent on the coefficients $A$ and $n$. Therefore, when $A$ and $n$ are random fields, the associated existence and uniqueness criterion (and a priori bounds) are all path-dependent. To define this path-dependence precisely, we use the following \lcnamecref{def:probdataacc}.

\bas[Probabilistic version of data-accuracy]\label{def:probdataacc}
There exist random variables $\Co$ and $\cotilde$ such that if
\beq\label{eq:probdataacc}
hk^a < \Co(\omega),
\eeq
then $\uh(\omega)$ exists, is unique, and $Q$ and $\Qh$ satisfy
\beqs
\abs{Q(\omega)-\Qh(\omega)} \leq \cotilde(\omega) h^\alpha k^\sigma \Cfg.
\eeqs
\eas

In order to define a finite-element approximation of $u$ (and therefore a random variable $\Qh$) that exists almost surely, we borrow a technique from \cite{GrPaSc:19}. Informally, for a given $h>0,$ for any realisations $\omega \in \Omega$ such that \cref{eq:probdataacc} is \emph{not} satisfied, we compute on a finer mesh (that does satisfy \cref{eq:probdataacc}). For fixed $h>0$ we define
\beq\label{eq:hmaxomega}
\hmaxomega \de \Co(\omega)k^{-a}.
\eeq
We then define
\beq\label{eq:homega}
\homega \de \min\set{h,\hmaxomega}.
\eeq
Observe that $\homega$ \emph{always} satisfies \cref{eq:probdataacc}. We then define
\beqs
\uhtilde(\omega) = u_{\homega}(\omega),
\eeqs
i.e., $\uhtilde$ is the finite-element approximation of $u(\omega)$ on the mesh with mesh size $\homega.$ We then define
\beq\label{eq:Qhtilde}
\Qhtilde = Q \circ \uhtilde.
\eeq
\optodo{Assume or prove that $\uhtilde$ is a random field}

Given $\uhtilde$ has a random mesh size, the cost of computing $\uhtilde$ (or $\Qhtilde$) will be a random variable\optodo{Prove?}. Therefore we make the following \lcnamecref{ass:costone} on the cost of computing samples.

\bas[Cost of one sample]\label{ass:costone}
There exist $\cthtilde, \gamma > 0$ such that $\cthtilde$ is independent of $h$ and $k$, and if $\Qhtilde(\omega)$ exists, then
\beqs
\Cost{\Qhtilde(\omega)} \leq \cthtilde(\omega) \homega^{-\gamma},
\eeqs
\eas

The following \lcnamecref{lem:c} shows how, provided the set $\Omegabad$\optodo{Chat about this earlier/here} has small probability, the expected cost of computing a sample of $\Qhtilde$ is governed solely by the specified mesh size $h$ (and not by any over-refinement). The assumption \cref{eq:cass} is the technical version of the statement `$\Omegabad$ has small probability'.

\bas[Assumptions on $\Omegabad$]\label{ass:omegabad}
Assume:
\bit
\item $hk^a \leq 1$
\item \beq\label{eq:cass}
\cth \de 2\EXP{\cthtilde\mleft(1+\Co^{-\gamma}\mright)} < \infty
\eeq
\eit
\eas

\ble[Expected cost of one sample]\label{lem:c}
If \cref{ass:omegabad} holds, then 
\beq\label{eq:singlecost}
\EXP{\Cost{\Qh}} \leq \cth h^{-\gamma}.
\eeq
\ele
\optodo{Show/assume cost is RV}
\bpf[Proof of \cref{lem:c}]
The proof follows closely that in \cite[Lemma 5.8]{GrPaSc:19}.
We have
\beq\label{eq:costpf1}
\Cost{\uhtilde(\omega)} \leq \cthtilde(\omega)\homega^{-\gamma} \leq \cthtilde(\omega) \mleft(h^{-\gamma} + \mleft(\hmaxomega\mright)^{-\gamma}\mright)
\eeq
by \cref{ass:costone,eq:homega}. Then using \cref{eq:hmaxomega}, the definition of $\hmaxomega,$ \cref{eq:costpf1} is bounded by
\beq\label{eq:costpf2}
\cthtilde(\omega)h^{-\gamma} + \mleft(\cthtilde\Co\mright)(\omega) k^{a\gamma},
\eeq
and since $hk^a \leq 1$ and \cref{eq:cass} holds, we obtain \cref{eq:singlecost}.
\epf

\ble[Convergence of numerical method]\label{ass:a}
Under \cref{def:probdataacc}, there exist constant $\co, \alpha, \sigma> 0$, such that $\co$ is independent of $h$ and $k$, and
\beqs
\abs{\EXP{\Qhtilde-Q}} \leq \co k^\sigma h^{\alpha}.
\eeqs
\ele

The proof of \cref{ass:a} is immediate (with $\co = \EXP{\cotilde}$) from the definition of $\Qhtilde$ \cref{eq:Qhtilde}, \cref{def:probdataacc} and the fact that $\homega \leq h$ (by \cref{eq:homega}.
\optodo{Define RMSE}
\section{Monte-Carlo methods}

We now define the Monte-Carlo estimator of $Q$,
\beqs
\QhatMC \de \frac1{\NMC} \sum_{j=1}^{\NMC} \Qhtildesj,
\eeqs
where the $\Qhtildesj$ are independently and identically distributed samples of $\Qhtilde.$

We can prove the following \lcnamecref{thm:hhmc} on the computational complexity of the Monte-Carlo estimator $\QhatMC$

\bth[Computational complexity of Monte-Carlo]\label{thm:hhmc}
Assume $\VAR{\Qhtilde}$ is independent of $h$ and the assumptions of \cref{lem:c,ass:a} hold. Given $\eps \in (0,1),$ if
\beq\label{eq:NMC}
\NMC  \sim 2\VAR{\Qhtilde}\eps^{-2}
\eeq
and
\beq\label{eq:hMC}
h \sim \mleft(\sqrt{2}\co\mright)^{-\frac1{\alpha}}k^{-\frac\sigma\alpha}\eps^{\frac1{\alpha}},
\eeq
then $\err{\QhatMC} \leq \eps$ and the computational complexity of $\QhatMC$ satisfies
\beqs
\EXP{\CMC} \sim 2^{1+\frac\gamma{2\alpha}}\VAR{\Qhtilde}\co^{\frac\gamma{\alpha}}\eps^{-2-\frac{\gamma}{\alpha}}k^{\frac{\gamma\sigma}\alpha}.
\eeqs
\enth

\bpf[Proof of \cref{thm:hhmc}]
The proof is standard, see, e.g., \cite[Section 2.1]{ClGiScTe:11}. We have
\begin{align}
\err{\QhatMC}^2 &= \EXP{\mleft(\QhatMC - \EXP{\QhatMC} + \EXP{\QhatMC} - \EXP{Q}\mright)^2}\nonumber\\
&= \EXP{\mleft(\QhatMC - \EXP{\QhatMC}\mright)^2} + \mleft(\EXP{\QhatMC} - \EXP{Q}\mright)^2\nonumber\\
&= \VAR{\QhatMC} + \mleft(\EXP{\QhatMC} - \EXP{Q}\mright)^2,\label{eq:mccomp1}
\end{align}
where the second line follows from the fact that $\EXP{\QhatMC - \EXP{\QhatMC}} = 0$, and the third line follows from the fact that $\QhatMC$ is an unbiased estimator.

By definition of $\QhatMC$, and the fact that the samples $\Qhtildesj$ are independent, we have
\beq\label{eq:mccomp2}
\VAR{\QhatMC} = \frac1{\NMC^2}\sum_{j=1}^{\NMC}\VAR{\Qhtildesj} = \frac1{\NMC} \VAR{\Qhtilde}.
\eeq
Therefore we can conclude from \cref{eq:mccomp1,eq:mccomp2} that the root-mean-squared-error satisfies
\beq\label{eq:mccomp3}
\err{\QhatMC}^2 = \frac1{\NMC}\VAR{\Qhtilde} + \NLoO{\Qhtilde-Q}^2.
\eeq
By \cref{eq:NMC,eq:hMC}, each of the terms in \cref{eq:mccomp3} is bounded by  $\eps^2/2,$ and therefore $\err{\QhatMC} \leq \eps.$

All that remains is to estimate the (expected) computational complexity. We have
\beqs
\EXP{\Cost{\QhatMC}} = \NMC \EXP{\Cost{\uhtilde}} \leq \NMC \cth h^{-\gamma} \sim 2^{1+\frac\gamma{2\alpha}}\VAR{\Qhtilde}\co^{\frac{\gamma}{\alpha}}\eps^{-2-\frac{\gamma}{\alpha}}k^{\frac{\gamma\sigma}\alpha}
\eeqs
as required.
\epf

\section{Multi-level Monte-Carlo}


TO HERE

Let $\OFP$ be a probability space, and let $Q$ be a random variable\footnote{One can think of $Q$ as being $Q(u),$ where $u$ is the solution of some stochastic PDE.} on $\OFP$ such that $\EXP{Q} < \infty.$ We will refer to $Q$ as the \defn{quantity of interest} or QoI. In order to define the multi-level Monte Carlo (MLMC) method for estimating $\EXP{Q},$ we must also define the following quantities, following  \cite[Theorem 1]{ClGiScTe:11}. We assume there exist
\bit
\item A set of levels\footnote{One can think of $\hl$ as the mesh size associated with level $l$.} $\set{\hl}_{l=0}^L$ ($L$ to be chosen) such that $\hl =\frac{\hlmo}s$ for $l \geq 1.$
\item A set of random variables (that may or may not exist)\footnote{One can think of $\Qhtilde$ as $Q(\uh),$ where $\uhl$ is the finite-element solution of the PDE with mesh size $h  $.} $\set{\Qhtilde}_{h \in (0,1)}.$
  \eit

We denote the random variables $\Qhtilde,$ in order to simplify the notation for mesh dependence in what follows.

In order to do things for the Helmholtz equation, we use the following assumption:

%% \bas[Mesh conditions for existence and uniqueness]
%% There exists a measurable function $\Cmesh:\Omega\rightarrow \RRp$ and a function $\mesh:\RRp\rightarrow \RRp$ such that $\Qhtilde(\omega)$ exists (and satisfies the error bounds etc. below) if
%% \beqs
%% h \leq \Cmesh(\omega)\mesh(k).
%% \eeqs
%% Note that $\mesh(k)\rightarrow 0$ as $k\rightarrow \infty.$
%% \eas


Observe that for a given $k, \omega$ there is no guarantee that $\Qhtilde(\omega)$ exists. Therefore, we follow [Graham, Parkinson, Scheichl] and 
That is, the random variable $\Qh$ is (thinking about things in terms of PDEs etc.) the QoI evaluated at the numerical solution, where that solution is taken on a mesh that is the finer of $h$ and $\hmaxomega$. This guarantees the QoI exists, and the error bounds below hold.

%% \bre[What is $\mesh(k)$?]
%% If nontrapping, $\mesh(k)=k^{-3/2}$. If trapping, more stringent, nothing proved in literature, but would expect to be similar to results for contant wavespeed.
%% \ere

With this setup in place, we define the following quantities.

We define the correction operators\optodo{You may be able to save some time computing these - if both $\hl$ and $\hlmo$ are larger that $\hmaxomega$, then the difference between them is zero.} between the levels by $\Yl \de \Qhl - \Qhlmo, l \geq 1,$ $\Yz = \Qhz.$ We let $\Ylhat$ be an unbiased estimator of $\Yl$, i.e., $\EXP{\Ylhat} = \EXP{\Yl}.$ In what follows $\Ylhat$ will be the Monte Carlo estimator
 \beqs
\Ylhat \de \frac1{\Nl}\sum_{i=1}^{\Nl} \Yli,
 \eeqs
 with $\Nl$ to be chosen, where $\Yli$ denotes independent samples of $\Yl$. Finally we are able to define the \defn{multi-level Monte Carlo estimator}
 \beqs
 \QhatMLhL \de \sum_{l=1}^L \Ylhat,
 \eeqs
 where the $\Ylhat$ are independent.

  The following assumptions
  % \lcnamecrefs{ass:coarse}
   will form the backbone of our analysis. They are a generalisation of the assumptions contained in \cite{ClGiScTe:11,ChScTe:13} for the MLMC method, the generalisation being that we assume that the quantities below depend not only on the levels $\hl$ but also on some additional parameter $k>1.$ When this theory is applied to the Helmholtz equation, $k$ will be the wavenumber of the Helmholtz equation.

%% The following assumption (which will be realised in a more concrete setting for the Helmholtz equation) concerns the existence of the approximating QoIs $\Qhl.$

%% \bas[Existence of $\Qhl$]\label{ass:qoie}
%% There exist $\Ccoarse,\coarseexp > 0$ with $\Ccoarse$ independent of $k$ such that if
%% \beqs
%% \hl \leq \Ccoarse k^{-\coarseexp},
%% \eeqs
%% then the QoI $\Qhl$ exists.
%% \eas

\bas[Variance of correction operators]\label{ass:b}
There exist $\ct, \beta, \tau > 0$, such that $\ct$ is independent of $h$ and $k,$ and
\beqs
\Vl \de \VAR{\Yl} \leq \ct k^\tau\hl^{\beta},
\eeqs  where $\VAR{\cdot}$ denotes variance.
\eas

In order to obtain a nice expression for the cost of computing one sample of $\Qh,$ we require the following assumption on the coarse space:

\bas[Dependence of coarse space on $k$]\label{ass:coarse}
We let
\beqs
\hz = \Ccoarse \mesh(k).
\eeqs
for some chosen constant $\Ccoarse > 0.$
\eas

 
% We write $\Vl$ for $\VAR{\Yl}.$
 
 We want to determine the choices of $L$ and $\Nl, l = 0,\ldots,L,$ such that the root-mean-squared eror (RMSE)
 \beqs
 \err{\QhatMLhL} \de \mleft(\EXP{\mleft(\QhatMLhL - \EXP{Q}\mright)^2}\mright)^{\half}
 \eeqs
 satisfies $\err{\QhatMLhL} \leq \eps,$ for some pre-defined $\eps > 0.$



%
 \paragraph{The nice case, where $k^{-\sigma/\alpha} \lesssim k^{-\coarseexp}.$}
\optodo{Might need todo something with the constants, as we need $\hL$ (as calculated) $< \hz.$ ensuring the constants are monotone is probably sufficient, as it'll just mean `for $\eps$ sufficiently small'.}
 The following theorem describes the computational effort needed to obtain RMSE $\leq \eps$. It is exactly the same as \cite[Theorem 1]{ClGiScTe:11}, but with the dependence on all the parameters explicit.%, and with some additional cases enumerated. %\Cref{thm:mlmccomp} contains more cases than in \cite[Theorem 1]{ClGiScTe:11} because \cite[Theorem 1]{ClGiScTe:11} makes the assumption throughout that $\alpha \geq 1/2\min\set{\beta,\gamma}.$ This assumption does not always hold for the Helmholtz equation (see the cases of a direct solver in 3-D below), however, examining the proof of \cite[Theorem 1]{ClGiScTe:11}  shows that in any given case, one only needs the assumption $\alpha \geq \beta/2$ or the assumption $\alpha \geq \gamma2$, never both at the same time. Therefore, for convenience, we explicitly state when these conditions are needed, and for completeness, we give the results when these conditions are violated. 

  The next two \lcnamecref{ass:powersnice}\optodo{plural} means that the restriction on the coarse space in \cref{ass:coarse} do not come into play,

 \bas[Epsilon sufficiently small]\label{ass:constants}
 Assume
 \beqs
\eps \leq \sqrt{2} \co \Ccoarse^{\alpha}.
 \eeqs
 \eas

 \bas\label{ass:powersnice}
 Suppose
 \beqs
\frac{\sigma}{\alpha} \geq \coarseexp.
 \eeqs
 \eas
 
 \bth[MLMC Complexity Theorem]\label{thm:mlmccomp}
If \cref{ass:constants,ass:powersnice} hold, $L$ is given by
\beq\label{eq:Lcond}
L = \ceil{\frac1\alpha\log_{s}\mleft(\sqrt{2}\co  \Ccoarse^\alpha k^{\sigma-\coarseexp\alpha} \eps^{-1}\mright)},
\eeq
that is,
\beqs
\hL \leq \mleft(\frac{\eps}{\sqrt{2}\co k^\sigma}\mright)^{\frac1\alpha},
\eeqs
and the number of samples on each computational level is given by
\beqs
\Nl = \ceil{\frac2{\eps^{2}} \mleft(\frac{\Vl}{\Cl}\mright)^{\half}\sum_{j=0}^{L} \mleft(\Vj\Cj\mright)^{\half}},
\eeqs
then computational effort $\CMLhL(\eps)$ required to obtain $\err{\QhatMLhL} \leq \eps$ satisfies the bounds
 
 \begin{numcases}{ \CMLhL(\eps) \lesssim}
 k^{\tau + \rho+\coarseexp\mleft(\gamma - \beta\mright)}\eps^{-2}\mleft(\log_s\mleft(\sqrt{2}\co\Cppw^\alpha k^{\sigma-\coarseexp\alpha} \eps^{-1}\mright)+2\alpha\mright)^2 +  k^{\rho +  \frac{\gamma\sigma}\alpha}\eps^{-\frac\gamma\alpha}
 & if $\beta = \gamma$,\label{eq:mlmchheq}\\ 
k^{\tau + \rho+\mleft(\gamma-\beta\mright)\frac\sigma\alpha}\eps^{-2+\mleft(\frac{\beta-\gamma}{\alpha}\mright)}
 +  k^{\rho +  \frac{\gamma\sigma}\alpha}\eps^{-\frac\gamma\alpha} & otherwise.\label{eq:mlmchhoth}
\end{numcases}
 \enth
 \optodo{Need to say why the latter two cases are the same - in one case $\gamma/\alpha$ dominates, and in the other case the other term dominates? (At least in the Cliffe et. al. set up)}
 \bpf[Proof of \cref{thm:mlmccomp}]
 \ednote{This isn't all the details of the proof, but the bits I've skipped over are exactly the same as those in {\cite{ClGiScTe:11}}.}
 
We first decompose the (squared) mean-squared error into the bias error and the sampling error:

\beqs
\errQhatMLhL^2 = \mleft(\EXP{\QhatMLhL} - \EXP{Q}\mright)^2 + \underbrace{\EXP{\mleft(\QhatMLhL - \EXP{\QhatMLhL}\mright)^2}}_{V\de},
\eeqs
the first term is the \emph{bias}, and the second term is the \emph{variance} of the estimator $\QhatMLhL.$ We now proceed to choose the parameters $L$ and $\Nl, l = 0,\ldots,L$ such that we can bound both the bias and the variance by $\eps^2/2.$

We first bound the bias, to do this, we only need to choose $L.$ One can show\ednote{As in {\cite{ClGiScTe:11}}} that the bias is equal to $\abs{\EXP{\QhL - Q}}^2.$ By \cref{ass:constants,ass:powersnice}, we don't need to worry about the coarse mesh restriction\optodo{Make this proper speak}. Therefore a sufficient condition for the bias to be $\leq \eps^2/2$ is (by \cref{ass:a})
\beqs
\co k^\sigma \hL^\alpha \leq \frac{\eps}{\sqrt{2}},
\eeqs
that is
\beq\label{eq:hLcond}
\hL \leq \mleft(\frac{\eps}{\sqrt{2}\co k^\sigma}\mright)^{\frac1\alpha}.
\eeq
\ednote{Observe that if $Q$ is the weighted $H^1$ norm, then we assume (see below for details) $\alpha=2$ and $\sigma=3,$ so we require $\hL \lesssim k^{-\frac32}.$ If we take $Q$ to be the $L^2$ norm, and assume $\alpha=2$ and $\sigma=2,$ then we only require $\hL \lesssim k^{-1}.$}

As $\hL = \hz s^{-L},$ it follows from \eqref{eq:hLcond} that a sufficient condition for the bias to be $\leq \eps^2/2$ is
\beq\label{eq:Lcondpart}
L = \ceil{\frac1\alpha\log_s\mleft(\sqrt{2}\co k^\sigma \hz^\alpha \eps^{-1}\mright)}.
\eeq
As $\hz = \Ccoarse k^{-\coarseexp},$ we can simplify \eqref{eq:Lcondpart} to obtain \eqref{eq:Lcond}.
% \beqs
% L = \ceil{\frac1\alpha\log_s\mleft(\sqrt{2}\co\Ccoarse^\alpha k^{\sigma-\coarseexp\alpha} \eps^{-1}\mright)}.
% \eeqs

We now seek to bound the variance. One can show\ednote{Again, as in \cite{ClGiScTe:11}} the variance $V = \sum_{l=0}^L \Nl^{-1} \Vl,$ and the cost is $\cC = \sum_{l=0}^L \Nl \Cl.$

To find the optimal number of samples per level (the values of $\Nl, l=0,\ldots,L$) we formulate this as an optimisation problem to find $\Nl$ that minimise $\cC$, subject to $V=\eps/2.$ This can be solved using a Lagrange multiplier as in \cite{Gi:15}, and we obtain
\beq\label{eq:Nl}
\Nl = \ceil{\frac2{\eps^{2}} \mleft(\frac{\Vl}{\Cl}\mright)^{\half}\sum_{j=0}^L \mleft(\Vj\Cj\mright)^{\half}}.
\eeq
\optodo{Check this is correct, should it be divided by the sum?}
We now just need to infer the computational complexity for MLMC with $L$ given by \eqref{eq:Lcond} and the $\Nl$ given by \eqref{eq:Nl}.

The computational complexity $\cC$ is given by
\begin{align}
\cC &= \sum_{l=0}^{L} \Cl \Nl\nonumber\\
&\leq \sum_{l=0}^L \Cl \mleft(\frac2{\eps^{2}} \mleft(\frac{\Vl}{\Cl}\mright)^{\half}\sum_{j=0}^L \mleft(\Vj\Cj\mright)^{\half} + 1\mright) \text{ (by \eqref{eq:Nl})}\nonumber\\
&= 2\eps^{-2}\mleft(\sum_{l=0}^L\mleft(\Vl\Cl\mright)^{\half}\mright)^2 + \sum_{l=0}^L \Cl\nonumber\\
&\leq 2 \ct \cth k^{\tau + \rho}\eps^{-2} \mleft(\sum_{l=0}^L \hl^{\frac{\beta-\gamma}2}\mright)^2 + \cth k^\rho \sum_{l=0}^L \hl^{-\gamma} \text{ (by \cref{ass:b,ass:c})}\nonumber\\
&= 2 \ct\cth k^{\tau + \rho}\eps^{-2}\hz^{\beta-\gamma}\mleft(\sum_{l=0}^L s^{-l\mleft(\frac{\beta-\gamma}2\mright)}\mright)^2 + \cth k^\rho \hz^{-\gamma} \sum_{l=0}^L s^{\gamma l} \text{ (by definition of } \hl\text{ )}\nonumber\\
&=2 \ct\cth \Cppw^{\beta-\gamma}k^{\tau + \rho+\coarseexp\mleft(\gamma - \beta\mright)}\eps^{-2}\mleft(\sum_{l=0}^L s^{l\mleft(\frac{\gamma-\beta}2\mright)}\mright)^2 + \cth\Cppw^{-\gamma} k^{\rho + \gamma\coarseexp}  \sum_{l=0}^L s^{\gamma l} \text{ (by definition of } \hz\text{ )}\nonumber\\
&\leq2\ct\cth \Cppw^{\beta-\gamma}k^{\tau + \rho+\coarseexp\mleft(\gamma - \beta\mright)}\eps^{-2}\mleft(\sum_{l=0}^L s^{l\mleft(\frac{\gamma-\beta}2\mright)}\mright)^2 +  \frac{\mleft(\sqrt{2}\co\mright)^{\frac\gamma\alpha}\cth s^{\gamma}}{1-s^{-\gamma}}k^{\rho +  \frac{\gamma\sigma}\alpha}\eps^{-\frac\gamma\alpha} \text{ (since }\gamma>0,\text{ by \cref{lem:sumbound})}.\label{eq:complexitymidway}
\end{align}
To bound the sum in the first part of \eqref{eq:complexitymidway}, we must distinguish three cases based on $\gamma - \beta.$

If $\gamma=\beta,$ then \eqref{eq:complexitymidway} becomes (using \cref{lem:sumbound})
\begin{multline}
2 \ct\cth \Cppw^{\beta-\gamma}k^{\tau + \rho+\coarseexp\mleft(\gamma - \beta\mright)}\eps^{-2}\mleft(L+1\mright)^2 +  \frac{\mleft(\sqrt{2}\co\mright)^{\frac\gamma\alpha}\cth s^{\gamma}}{1-s^{-\gamma}}k^{\rho +  \frac{\gamma\sigma}\alpha}\eps^{-\frac\gamma\alpha}\\
\leq
2\ct\cth \Cppw^{\beta-\gamma}k^{\tau + \rho+\coarseexp\mleft(\gamma - \beta\mright)}\eps^{-2}\mleft(\frac1\alpha\log_s\mleft(\sqrt{2}\co\Cppw^\alpha k^{\sigma-\coarseexp\alpha} \eps^{-1}\mright)+2\mright)^2 +  \frac{\mleft(\sqrt{2}\co\mright)^{\frac\gamma\alpha}\cth s^{\gamma}}{1-s^{-\gamma}}k^{\rho +  \frac{\gamma\sigma}\alpha}\eps^{-\frac\gamma\alpha}\label{eq:gammaequal}
\end{multline}
by \eqref{eq:Lcond}.

If $\gamma > \beta$ then by \cref{lem:sumbound} \eqref{eq:complexitymidway} becomes
\beq
2 \ct\cth \Cppw^{\beta-\gamma}
\frac{\mleft(\sqrt{2}\co\mright)^{\mleft(\frac{\gamma-\beta}{\alpha}\mright)}\Cppw^{\mleft(\gamma-\beta\mright)}s^{\gamma-\beta}}{\mleft(1-s^{\mleft(\frac{\beta-\gamma}2\mright)}\mright)^{2}}k^{\tau + \rho+\mleft(\gamma-\beta\mright)\frac\sigma\alpha}\eps^{-2+\mleft(\frac{\beta-\gamma}{\alpha}\mright)}
 +  \frac{\mleft(\sqrt{2}\co\mright)^{\frac\gamma\alpha}\cth s^{\gamma}}{1-s^{-\gamma}}k^{\rho +  \frac{\gamma\sigma}\alpha}\eps^{-\frac\gamma\alpha}.\label{eq:gammagtr}
\eeq
If $\gamma < \beta,$ then by \cref{lem:sumbound} \eqref{eq:complexitymidway} becomes
\beq
2 \ct\cth \Cppw^{\beta-\gamma}
\frac{\mleft(\sqrt{2}\co\mright)^{\mleft(\frac{\gamma-\beta}{\alpha}\mright)}\Cppw^{\mleft(\gamma-\beta\mright)}}{\mleft(1-s^{\mleft(\frac{\beta-\gamma}2\mright)}\mright)^{2}}k^{\tau + \rho+\mleft(\gamma-\beta\mright)\frac\sigma\alpha}\eps^{-2+\mleft(\frac{\beta-\gamma}{\alpha}\mright)}
 +  \frac{\mleft(\sqrt{2}\co\mright)^{\frac\gamma\alpha}\cth s^{\gamma}}{1-s^{-\gamma}}k^{\rho +  \frac{\gamma\sigma}\alpha}\eps^{-\frac\gamma\alpha},\label{eq:gammaless}
\eeq
the only difference from \eqref{eq:gammagtr} being the loss of the $s^{\gamma-\beta}$ term.

Removing all the terms that are not of interest from \eqref{eq:gammaequal}, \eqref{eq:gammagtr}, and \eqref{eq:gammaless}, we obtain \eqref{eq:mlmchheq} and \eqref{eq:mlmchhoth}.
\epf




\subsection{Lemma}

The proof of the main \lcnamecref{thm:mlmccomp} will require the following \lcnamecref{lem:sumbound}.

\ble\label{lem:sumbound}
If $L$ is given by
\beq\label{eq:Ldef}
L = \ceil{\frac1\alpha\log_{s}\mleft(\sqrt{2}\co  \Ccoarse^\alpha k^{\sigma}\mesh(k)^\alpha \eps^{-1}\mright)},
\eeq
then, for $s>1$ and $\delta \in \RR,$ we have the bound
\beq\label{eq:sumbound}
\sum_{l=0}^{L} s^{\delta l} \leq
\begin{cases}
L+1 & \tif \delta = 0,\\
\frac{\mleft(\sqrt{2}\co\mright)^{\frac\delta\alpha}\Ccoarse^{\delta}s^{\delta}}{1-s^{-\delta}}k^{\frac{\delta\sigma}{\alpha}}\mesh(k)^\delta\eps^{-\frac\delta\alpha} &\tif \delta >0\\
\frac{\mleft(\sqrt{2}\co\mright)^{\frac\delta\alpha}\Ccoarse^{\delta}}{1-s^{-\delta}}k^{\frac{\delta\sigma}{\alpha}}\mesh(k)^\delta\eps^{-\frac\delta\alpha}&\tif \delta < 0
\end{cases}
\eeq
\ele

\bpf[Proof of \cref{lem:sumbound}]
The proof follows that in \cite{ClGiScTe:11}. We first observe that, since $L$ is given by \eqref{eq:Ldef}, it follows that
\beq\label{eq:Lbounds}
\frac1\alpha\log_s\mleft(\sqrt{2}\co\Ccoarse^\alpha k^{\sigma}\mesh(k)^\alpha \eps^{-1}\mright) \leq L < \frac1\alpha\log_s\mleft(\sqrt{2}\co\Ccoarse^\alpha k^{\sigma}\mesh(k)^\alpha \eps^{-1}\mright) + 1.
\eeq
Rearranging \eqref{eq:Lbounds}, we obtain the bounds
\beq\label{eq:saLbounds}
\sqrt{2}\co \Ccoarse^\alpha k^{\sigma}\mesh(k)^\alpha\eps^{-1} \leq s^{\alpha L} < \sqrt{2}\co \Ccoarse^\alpha k^{\sigma}\mesh(k)^\alpha\eps^{-1}s^\alpha.
\eeq
If $\delta > 0,$ then we use the right-hand bound in \eqref{eq:saLbounds} to obtain
\beq\label{eq:sdLpos}
s^{\delta L} < \mleft(\sqrt{2}\co\mright)^{\frac\delta\alpha}\Cppw^{\delta}k^{\frac{\delta\sigma}{\alpha}}\mesh(k)^\delta\eps^{-\frac\delta\alpha}s^{\delta},
\eeq
and if $\delta < 0,$ we use the left-hand bound in \eqref{eq:saLbounds} to obtain
\beq\label{eq:sdLneg}
s^{\delta L} \leq \mleft(\sqrt{2}\co\mright)^{\frac\delta\alpha}\Cppw^{\delta}k^{\frac{\delta\sigma}{\alpha}}\mesh(k)^\delta\eps^{-\frac\delta\alpha}.
\eeq
We now observe that, for $\delta \neq 0,$
\begin{align}
\sum_{l=0}^L s^{\delta l} &= \frac{s^{\delta\mleft(L+1\mright)} -1}{s^{\delta}-1}\nonumber\\
&= \frac{s^{\delta L} - s^{-\delta}}{1-s^{-\delta}}\nonumber\\
&\leq \frac{s^{\delta L}}{1-s^{-\delta}},\label{eq:ssumbound}
\end{align}
since $s^{-\delta} > 0,$ as $s >0.$ Combining \eqref{eq:ssumbound} with \eqref{eq:sdLpos} and \eqref{eq:sdLneg}, we obtain \eqref{eq:sumbound} in the cases $\delta \neq 0.$ The case $\delta=0$ is straightforward.
\epf


The following theorem describes the computational effort needed to obtain RMSE $\leq \eps$. It is exactly the same as \cite[Theorem 1]{ClGiScTe:11}, but with the dependence on all the parameters explicit.%, and with some additional cases enumerated. %\Cref{thm:mlmccomp} contains more cases than in \cite[Theorem 1]{ClGiScTe:11} because \cite[Theorem 1]{ClGiScTe:11} makes the assumption throughout that $\alpha \geq 1/2\min\set{\beta,\gamma}.$ This assumption does not always hold for the Helmholtz equation (see the cases of a direct solver in 3-D below), however, examining the proof of \cite[Theorem 1]{ClGiScTe:11}  shows that in any given case, one only needs the assumption $\alpha \geq \beta/2$ or the assumption $\alpha \geq \gamma2$, never both at the same time. Therefore, for convenience, we explicitly state when these conditions are needed, and for completeness, we give the results when these conditions are violated. 

%% The following \lcnamecref{ass:constants3} will ensure that \cref{ass:qoie} is satisfied.

%%  \bas[$\eps$ sufficiently small]\label{ass:constants3}
%%  Assume
%%  \beqs
%% \eps \leq \sqrt{2} \co \Ccoarse^{\alpha} k^{\sigma-\coarseexp\alpha}.
%%  \eeqs
%%  \eas

\paragraph{Notation}

$\Cl \de \cth\hl^{-\gamma}$

$\cC \de \EXP{\Cost{\QhatMLhL}}$

\bas[Assumptions on $\eps$ and $k$ to make things nicer]\label{ass:epsk}
\beqs
\eps \leq \min\set{\frac{\sqrt{2}\co\Ccoarse^\alpha}{s^{2\alpha}},\frac1{\sqrt{2}\co\Ccoarse^\alpha}},
\eeqs
and
\beqs
k^\sigma \mesh(k)^\alpha \geq 1.
\eeqs
\eas

\bth[MLMC Complexity Theorem]\label{thm:mlmccomp}
If \cref{ass:epsk} holds,  $L$ is given by \eqref{eq:Ldef}
that is,
\beq\label{eq:hLcond}
\hL \leq \mleft(\frac\eps{\sqrt{2}\co k^{\sigma}}\mright)^{\frac1\alpha},
\eeq
and the number of samples on each computational level is given by
\beqs
\Nl = \ceil{\frac2{\eps^{2}} \mleft(\frac{\Vl}{\Cl}\mright)^{\half}\sum_{j=0}^{L} \mleft(\Vj\Cj\mright)^{\half}},
\eeqs
then computational effort $\CMLhL(\eps)$ required to obtain $\err{\QhatMLhL} \leq \eps$ satisfies the bounds
 
 \begin{numcases}{ \CMLhL(\eps) \lesssim}
k^\tau \mleft(\log\mleft(\frac{k^\sigma \mesh(k)^\alpha}\eps\mright)\mright)^2 \eps^{-2} + k^{\frac{\gamma\sigma}\alpha} \eps^{-\frac\gamma\alpha}  & if $\beta = \gamma$,\label{eq:mlmchheq}\\ 
k^{\tau + \mleft(\gamma-\beta\mright)\frac\sigma\alpha} \eps^{-2-\frac{\gamma-\beta}{\alpha}} + k^{\frac{\gamma\sigma}{\alpha}}\eps^{-\frac\gamma\alpha} & otherwise.\label{eq:mlmchhoth}
\end{numcases}
 \enth
 \bpf[Proof of \cref{thm:mlmccomp}]
We first decompose the (squared) mean-squared error into the bias error and the sampling error:

\beqs
\errQhatMLhL^2 = \mleft(\EXP{\QhatMLhL} - \EXP{Q}\mright)^2 + \underbrace{\EXP{\mleft(\QhatMLhL - \EXP{\QhatMLhL}\mright)^2}}_{V\de},
\eeqs
the first term is the \emph{bias}, and the second term is the \emph{variance} of the estimator $\QhatMLhL.$ We now proceed to choose the parameters $L$ and $\Nl, l = 0,\ldots,L$ such that we can bound both the bias and the variance by $\eps^2/2.$

We first bound the bias, to do this, we only need to choose $L.$ One can show that the bias is equal to $\abs{\EXP{\QhL - Q}}^2.$ Therefore a sufficient condition for the bias to be $\leq \eps^2/2$ is (by \cref{ass:a})
\beqs
\co k^\sigma \hL^\alpha \leq \frac{\eps}{\sqrt{2}},
\eeqs
that is, \eqref{eq:hLcond}. As $\hL = \hz s^{-L},$ it follows from \eqref{eq:hLcond} that a sufficient condition for the bias to be $\leq \eps^2/2$ is
\beq\label{eq:Lcondpart}
L = \ceil{\frac1\alpha\log_s\mleft(\sqrt{2}\co k^\sigma \hz^\alpha \eps^{-1}\mright)}.
\eeq
As $\hz = \Ccoarse \mesh(k),$ we can simplify \eqref{eq:Lcondpart} to obtain \eqref{eq:Ldef}.
% \beqs
% L = \ceil{\frac1\alpha\log_s\mleft(\sqrt{2}\co\Ccoarse^\alpha k^{\sigma-\coarseexp\alpha} \eps^{-1}\mright)}.
% \eeqs

We now seek to bound the variance. One can show the variance $V = \sum_{l=0}^L \Nl^{-1} \Vl,$ and the cost is: (following \cite{GrPaSc:19})
\begin{align}
\cC &= \EXP{\Cost{\QhatMLhL}}\nonumber\\
&\leq \sum_{l=0}^L \EXP{\Cost{\Ylhat}}\text{less equal as could save if you've had to over-refine on both levels in $\Ylhat$ - difference will be zero}\nonumber\\
&= \sum_{l=0}^L \sum_{i=1}^{\Nl} \EXP{\Cost{\Yli}}\nonumber\\
&\leq \sum_{l=0}^L \sum_{i=0}^{\Nl} \mleft(\EXP{\Cost{\Qhl}} + \EXP{\Cost{\Qhlmo}}\mright)\nonumber\\
&\leq \sum_{l=0}^L \Nl \mleft(\cth \hl^{-\gamma} + \cth \hlmo^{-\gamma}\mright)\nonumber\\
&=\sum_{l=0}^L \Nl\mleft(1+s^{-\gamma}\mright) \cth \hl^{-\gamma}\nonumber\\
&= \mleft(1+s^{-\gamma}\mright) \sum_{l=0}^L \Nl\Cl\label{eq:Cboundformin}
\end{align}

To find the optimal number of samples per level (the values of $\Nl, l=0,\ldots,L$) we formulate this as an optimisation problem to find $\Nl$ that minimise \eqref{eq:Cboundformin}, subject to $V=\eps/2.$ This can be solved using a Lagrange multiplier as in \cite{Gi:15}, and we obtain
\beq\label{eq:Nl}
\Nl = \ceil{\frac2{\eps^{2}} \mleft(\frac{\Vl}{\Cl}\mright)^{\half}\sum_{j=0}^L \mleft(\Vj\Cj\mright)^{\half}}.
\eeq
We now just need to infer the computational complexity for MLMC with $L$ given by \eqref{eq:Ldef} and the $\Nl$ given by \eqref{eq:Nl}.

The computational complexity $\cC$ is given by
\begin{align}
\cC &\leq \mleft(1+s^{-\gamma}\mright)\sum_{l=0}^{L} \Cl \Nl\nonumber\\
&\leq \mleft(1+s^{-\gamma}\mright)\sum_{l=0}^L \Cl \mleft(\frac2{\eps^{2}} \mleft(\frac{\Vl}{\Cl}\mright)^{\half}\sum_{j=0}^L \mleft(\Vj\Cj\mright)^{\half} + 1\mright) \text{ (by \eqref{eq:Nl})}\nonumber\\
&= 2\eps^{-2}\mleft(1+s^{-\gamma}\mright)\mleft(\sum_{l=0}^L\mleft(\Vl\Cl\mright)^{\half}\mright)^2 + \mleft(1+s^{-\gamma}\mright)\sum_{l=0}^L \Cl\nonumber\\
&= 2 \ct \cth \mleft(1+s^{-\gamma}\mright)k^{\tau}\eps^{-2} \mleft(\sum_{l=0}^L \hl^{\frac{\beta-\gamma}2}\mright)^2 + \cth \mleft(1+s^{-\gamma}\mright) \sum_{l=0}^L \hl^{-\gamma} \text{ (by \cref{ass:b,ass:c})}\nonumber\\
&= 2 \ct\cth \mleft(1+s^{-\gamma}\mright)k^{\tau}\eps^{-2}\hz^{\beta-\gamma}\mleft(\sum_{l=0}^L s^{l\mleft(\frac{\gamma-\beta}2\mright)}\mright)^2 + \cth\mleft(1+s^{-\gamma}\mright) \hz^{-\gamma} \sum_{l=0}^L s^{\gamma l} \text{ (by definition of } \hl\text{ )}\label{eq:complexitymidway}\\
%% &=2 \ct\cth \Cppw^{\beta-\gamma}k^{\tau + \rho+\coarseexp\mleft(\gamma - \beta\mright)}\eps^{-2}\mleft(\sum_{l=0}^L s^{l\mleft(\frac{\gamma-\beta}2\mright)}\mright)^2 + \cth\Cppw^{-\gamma} k^{\rho + \gamma\coarseexp}  \sum_{l=0}^L s^{\gamma l} \text{ (by definition of } \hz\text{ )}\nonumber\\
%% &\leq2\ct\cth \Cppw^{\beta-\gamma}k^{\tau + \rho+\coarseexp\mleft(\gamma - \beta\mright)}\eps^{-2}\mleft(\sum_{l=0}^L s^{l\mleft(\frac{\gamma-\beta}2\mright)}\mright)^2 +  \frac{\mleft(\sqrt{2}\co\mright)^{\frac\gamma\alpha}\cth s^{\gamma}}{1-s^{-\gamma}}k^{\rho +  \frac{\gamma\sigma}\alpha}\eps^{-\frac\gamma\alpha} \text{ (since }\gamma>0,\text{ by \cref{lem:sumbound})}.\label{eq:complexitymidway}
\end{align}

Using \cref{lem:sumbound}, the second term in \eqref{eq:complexitymidway} can be bounded (as $\gamma > 0$) by %(letting \csumdelta \de \mleft(\sqrt{2}\co\mright)^{\frac\delta\alpha} \Ccoarse^\delta / \mleft(1-s^{-\delta}\mright)$)
\beq\label{eq:firstterm}
\frac{\mleft(1+s^{-\gamma}\mright) \cth \hz^{-\gamma} \mleft(\sqrt{2}\co\mright)^{\frac\gamma\alpha} s^\gamma \Ccoarse^\gamma}{1-s^{-\gamma}} k^{\frac{\gamma\sigma}\alpha} \mesh(k)^\gamma \eps^{-\frac\gamma\alpha}
= \frac{\mleft(1+s^{-\gamma}\mright)\cth \mleft(\sqrt{2}\co\mright)^{\frac\gamma\alpha} s^\gamma}{1-s^{-\gamma}} k^{\frac{\gamma\sigma}\alpha} \eps^{-\frac\gamma\alpha}
\eeq

To bound the sum in the first part of \eqref{eq:complexitymidway}, we must distinguish three cases based on $\gamma - \beta.$


If $\gamma=\beta,$ then the first part of \eqref{eq:complexitymidway} becomes (using \cref{lem:sumbound})
\beq
2 \ct\cth \mleft(1+s^{-\gamma}\mright)k^{\tau}\eps^{-2}\mleft(L+1\mright)^2 \leq 2 \ct\cth \mleft(1+s^{-\gamma}\mright)k^{\tau}\eps^{-2}\mleft(\frac1\alpha \log_s \mleft(\frac{\sqrt{2} \co \Ccoarse^\alpha k^\sigma \mesh(k)^\alpha}\eps\mright)+2\mright)^2,
\label{eq:gammaequal}
\eeq
by \eqref{eq:Ldef}. We wish to simplify \eqref{eq:gammaequal}, so that it is of the form $\mathrm{Constant} \times \text{Terms involving } \eps \text{ and } k.$ To achieve this simplification, we use \cref{ass:epsk}. As $k^\sigma \mesh(k)^\alpha \geq 1$ and $\eps \leq \mleft(\sqrt{2} \co \Ccoarse^{\alpha}\mright)/s^{2\alpha},$ it follows that
\beqs
2 \leq \frac1\alpha \log_s \mleft(\frac{\sqrt{2} \co \Ccoarse^\alpha k^\sigma \mesh(k)^\alpha}\eps\mright),
\eeqs
and thus \eqref{eq:gammaequal} can be bounded by
\beq\label{eq:gammaequalpart1}
8 \ct\cth \mleft(1+s^{-\gamma}\mright)k^{\tau}\eps^{-2}\mleft(\frac1\alpha \log_s \mleft(\frac{\sqrt{2} \co \Ccoarse^\alpha k^\sigma \mesh(k)^\alpha}\eps\mright)\mright)^2.
\eeq
As $k^\sigma \mesh(k)^\alpha \geq 1$ and $\eps \leq 1/\mleft(\sqrt{2}\co\Ccoarse^\alpha\mright),$ we can bound \eqref{eq:gammaequalpart1} by (including a change of base in the logarithm)
\beq\label{eq:gammaequalfinal}
\frac{32 \ct\cth \mleft(1+s^{-\gamma}\mright)}{\alpha^2 \mleft(\log(s)\mright)^2} k^\tau \mleft(\log\mleft(\frac{k^\sigma \mesh(k)^\alpha}\eps\mright)\mright)^2.
\eeq

For simplicity in what follows, we define
\beqs
\csumdelta \de \frac{\mleft(\sqrt{2}\co\mright)^{\frac\delta\alpha}\Ccoarse^{\delta}}{1-s^{-\delta}}.
\eeqs

If $\gamma > \beta$ then by \cref{lem:sumbound} the first term in \eqref{eq:complexitymidway} becomes\optodo{Check if the below is right - it seems to be saying the complexity is independent of the coarse mesh}
\beq
\eps^{-2}2\ct\cth \mleft(1+s^{-\gamma}\mright) k^\tau \hz^{\beta-\gamma}\mleft(\csumgammambetat s^{\frac{\gamma-\beta}2} k^{\frac{\gamma-\beta}2\frac\sigma\alpha} \mesh(k)^{\frac{\gamma-\beta}2} \eps^{-\frac{\gamma-\beta}{2\alpha}}\mright)^2 = \Cgammagtrbeta k^{\tau + \mleft(\gamma-\beta\mright)\frac\sigma\alpha} \eps^{-2-\frac{\gamma-\beta}{\alpha}},\label{eq:gammagtr}
\eeq
where
\beqs
\Cgammagtrbeta \de 2\ct\cth\mleft(1+s^{-\gamma}\mright)\csumgammambetat^2 s^{\gamma-\beta} \Ccoarse^{\beta-\gamma}.
\eeqs
If $\gamma < \beta,$ then analagously the first term in \eqref{eq:complexitymidway} is
\beqs
\Cgammalessbeta k^{\tau + \mleft(\gamma-\beta\mright)\frac\sigma\alpha} \eps^{-2-\frac{\gamma-\beta}{\alpha}},
\eeqs
where
\beqs
\Cgammalessbeta \de \frac{\Cgammagtrbeta}{s^{\gamma-\beta}}.
\eeqs

We now combine \eqref{eq:firstterm}, \eqref{eq:gammaequalfinal}, \eqref{eq:gammagtr}, and \eqref{eq:gammaless} and supress all the constants to obtain the result.
%Removing all the terms that are not of interest from \eqref{eq:gammaequal}, \eqref{eq:gammagtr}, and \eqref{eq:gammaless}, we obtain \eqref{eq:mlmchheq} and \eqref{eq:mlmchhoth}.
\epf


%
\paragraph{The nasty case, where $k^{-\sigma/\alpha} \gtrsim k^{-\coarseexp}.$}

 \bas\label{ass:powersnasty}
 Suppose
 \beqs
\frac{\sigma}{\alpha} \leq \coarseexp.
 \eeqs
 \eas

  \bas[Epsilon sufficiently small]\label{ass:constantsnasty}
 Assume
 \beqs
\eps \leq \co \Ccoarse^{\alpha}.
 \eeqs
 \eas

\bth[MLMC Complexity Theorem]\label{thm:mlmccomp2}
Assume \cref{ass:powersnasty,ass:constantsnasty} hold. Assume $k \geq 1.$ If $L$ is given by
\beq\label{eq:Lcond2}
L = \ceil{\frac1\alpha\log_{s}\mleft(\sqrt{2}\co  \Ccoarse^\alpha k^{\sigma-\coarseexp\alpha} \eps^{-1}\mright)},
\eeq
that is,
\beqs
\hL \leq \mleft(\frac{\eps}{\sqrt{2}\co k^\sigma}\mright)^{\frac1\alpha},
\eeqs
and the number of samples on each computational level is given by
\beqs
\Nl = \ceil{\frac2{\eps^{2}} \mleft(\frac{\Vl}{\Cl}\mright)^{\half}\sum_{j=0}^{L} \mleft(\Vj\Cj\mright)^{\half}},
\eeqs
then computational effort $\CMLhL(\eps)$ required to obtain $\err{\QhatMLhL} \leq \eps$ satisfies the bounds
 
 \begin{numcases}{ \CMLhL(\eps) \lesssim}
 k^{\tau + \rho+\coarseexp\mleft(\gamma - \beta\mright)}\eps^{-2}\mleft(\log_s\mleft(\sqrt{2}\co\Cppw^\alpha k^{\sigma-\coarseexp\alpha} \eps^{-1}\mright)+2\alpha\mright)^2 +  k^{\rho +  \frac{\gamma\sigma}\alpha}\eps^{-\frac\gamma\alpha}
 & if $\beta = \gamma$,\label{eq:mlmchheq2}\\ 
k^{\tau + \rho+\mleft(\gamma-\beta\mright)\frac\sigma\alpha}\eps^{-2+\mleft(\frac{\beta-\gamma}{\alpha}\mright)}
 +  k^{\rho +  \frac{\gamma\sigma}\alpha}\eps^{-\frac\gamma\alpha} & otherwise.\label{eq:mlmchhoth2}
\end{numcases}
 \enth
 \optodo{Need to say why the latter two cases are the same - in one case $\gamma/\alpha$ dominates, and in the other case the other term dominates? (At least in the Cliffe et. al. set up)}

 \bpf[Proof of \cref{thm:mlmccomp}]
 \ednote{This isn't all the details of the proof, but the bits I've skipped over are exactly the same as those in {\cite{ClGiScTe:11}}.}
Do the bias-variance decomposition.  
 We now proceed to choose the parameters $L$ and $\Nl, l = 0,\ldots,L$, however, because the dominant term in this case\optodo{show this} is the \emph{Coarse restriction} we tentatively\ednote{I have no idea at this stage whether this will work} We bound the bias by $\eps^2k^{\sigma-\alpha\coarseexp}$ and the variance by $\eps^2\mleft(1-k^{\sigma-\alpha\coarseexp}\mright).$\optodo{This is all fine by the assumptions}

We first bound the bias, to do this, we only need to choose $L.$ One can show\ednote{As in {\cite{ClGiScTe:11}}} that the bias is equal to $\abs{\EXP{\QhL - Q}}^2.$ Therefore a sufficient condition for the bias to be $\leq \eps^2k^{\sigma-\alpha\coarseexp}$ is (by \cref{ass:a})
\beqs
\co k^\sigma \hL^\alpha \leq \eps k^{\frac{\sigma-\alpha\coarseexp}2},
\eeqs
that is
\beq\label{eq:hLcond2}
\hL \leq \mleft(\frac{\eps}{\sqrt{2}\co k^\sigma}\mright)^{\frac1\alpha}.
\eeq
\ednote{Observe that if $Q$ is the weighted $H^1$ norm, then we assume (see below for details) $\alpha=2$ and $\sigma=3,$ so we require $\hL \lesssim k^{-\frac32}.$ If we take $Q$ to be the $L^2$ norm, and assume $\alpha=2$ and $\sigma=2,$ then we only require $\hL \lesssim k^{-1}.$}

As $\hL = \hz s^{-L},$ it follows from \eqref{eq:hLcond2} that a sufficient condition for the bias to be $\leq \eps^2/2$ is
\beq\label{eq:Lcondpart2}
L = \ceil{\frac1\alpha\log_s\mleft(\sqrt{2}\co k^\sigma \hz^\alpha \eps^{-1}\mright)}.
\eeq
As $\hz = \Ccoarse k^{-\coarseexp},$ we can simplify \eqref{eq:Lcondpart2} to obtain \eqref{eq:Lcond2}.
% \beqs
% L = \ceil{\frac1\alpha\log_s\mleft(\sqrt{2}\co\Ccoarse^\alpha k^{\sigma-\coarseexp\alpha} \eps^{-1}\mright)}.
% \eeqs

We now seek to bound the variance. One can show\ednote{Again, as in \cite{ClGiScTe:11}} the variance $V = \sum_{l=0}^L \Nl^{-1} \Vl,$ and the cost is $\cC = \sum_{l=0}^L \Nl \Cl.$

To find the optimal number of samples per level (the values of $\Nl, l=0,\ldots,L$) we formulate this as an optimisation problem to find $\Nl$ that minimise $\cC$, subject to $V=\eps/2.$ This can be solved using a Lagrange multiplier as in \cite{Gi:15}, and we obtain
\beq\label{eq:Nl2}
\Nl = \ceil{\frac2{\eps^{2}} \mleft(\frac{\Vl}{\Cl}\mright)^{\half}\sum_{j=0}^L \mleft(\Vj\Cj\mright)^{\half}}.
\eeq
\optodo{Check this is correct, should it be divided by the sum?}
We now just need to infer the computational complexity for MLMC with $L$ given by \eqref{eq:Lcond2} and the $\Nl$ given by \eqref{eq:Nl2}.

The computational complexity $\cC$ is given by
\begin{align}
\cC &= \sum_{l=0}^{L} \Cl \Nl\nonumber\\
&\leq \sum_{l=0}^L \Cl \mleft(\frac2{\eps^{2}} \mleft(\frac{\Vl}{\Cl}\mright)^{\half}\sum_{j=0}^L \mleft(\Vj\Cj\mright)^{\half} + 1\mright) \text{ (by \eqref{eq:Nl2})}\nonumber\\
&= 2\eps^{-2}\mleft(\sum_{l=0}^L\mleft(\Vl\Cl\mright)^{\half}\mright)^2 + \sum_{l=0}^L \Cl\nonumber\\
&\leq 2 \ct \cth k^{\tau + \rho}\eps^{-2} \mleft(\sum_{l=0}^L \hl^{\frac{\beta-\gamma}2}\mright)^2 + \cth k^\rho \sum_{l=0}^L \hl^{-\gamma} \text{ (by \cref{ass:b,ass:c})}\nonumber\\
&= 2 \ct\cth k^{\tau + \rho}\eps^{-2}\hz^{\beta-\gamma}\mleft(\sum_{l=0}^L s^{-l\mleft(\frac{\beta-\gamma}2\mright)}\mright)^2 + \cth k^\rho \hz^{-\gamma} \sum_{l=0}^L s^{\gamma l} \text{ (by definition of } \hl\text{ )}\nonumber\\
&=2 \ct\cth \Cppw^{\beta-\gamma}k^{\tau + \rho+\coarseexp\mleft(\gamma - \beta\mright)}\eps^{-2}\mleft(\sum_{l=0}^L s^{l\mleft(\frac{\gamma-\beta}2\mright)}\mright)^2 + \cth\Cppw^{-\gamma} k^{\rho + \gamma\coarseexp}  \sum_{l=0}^L s^{\gamma l} \text{ (by definition of } \hz\text{ )}\nonumber\\
&\leq2\ct\cth \Cppw^{\beta-\gamma}k^{\tau + \rho+\coarseexp\mleft(\gamma - \beta\mright)}\eps^{-2}\mleft(\sum_{l=0}^L s^{l\mleft(\frac{\gamma-\beta}2\mright)}\mright)^2 +  \frac{\mleft(\sqrt{2}\co\mright)^{\frac\gamma\alpha}\cth s^{\gamma}}{1-s^{-\gamma}}k^{\rho +  \frac{\gamma\sigma}\alpha}\eps^{-\frac\gamma\alpha} \text{ (since }\gamma>0,\text{ by \cref{lem:sumbound})}.\label{eq:complexitymidway2}
\end{align}
To bound the sum in the first part of \eqref{eq:complexitymidway2}, we must distinguish three cases based on $\gamma - \beta.$

If $\gamma=\beta,$ then \eqref{eq:complexitymidway2} becomes (using \cref{lem:sumbound})
\begin{multline}
2 \ct\cth \Cppw^{\beta-\gamma}k^{\tau + \rho+\coarseexp\mleft(\gamma - \beta\mright)}\eps^{-2}\mleft(L+1\mright)^2 +  \frac{\mleft(\sqrt{2}\co\mright)^{\frac\gamma\alpha}\cth s^{\gamma}}{1-s^{-\gamma}}k^{\rho +  \frac{\gamma\sigma}\alpha}\eps^{-\frac\gamma\alpha}\\
\leq
2\ct\cth \Cppw^{\beta-\gamma}k^{\tau + \rho+\coarseexp\mleft(\gamma - \beta\mright)}\eps^{-2}\mleft(\frac1\alpha\log_s\mleft(\sqrt{2}\co\Cppw^\alpha k^{\sigma-\coarseexp\alpha} \eps^{-1}\mright)+2\mright)^2 +  \frac{\mleft(\sqrt{2}\co\mright)^{\frac\gamma\alpha}\cth s^{\gamma}}{1-s^{-\gamma}}k^{\rho +  \frac{\gamma\sigma}\alpha}\eps^{-\frac\gamma\alpha}\label{eq:gammaequal2}
\end{multline}
by \eqref{eq:Lcond2}.

If $\gamma > \beta$ then by \cref{lem:sumbound} \eqref{eq:complexitymidway2} becomes
\beq
2 \ct\cth \Cppw^{\beta-\gamma}
\frac{\mleft(\sqrt{2}\co\mright)^{\mleft(\frac{\gamma-\beta}{\alpha}\mright)}\Cppw^{\mleft(\gamma-\beta\mright)}s^{\gamma-\beta}}{\mleft(1-s^{\mleft(\frac{\beta-\gamma}2\mright)}\mright)^{2}}k^{\tau + \rho+\mleft(\gamma-\beta\mright)\frac\sigma\alpha}\eps^{-2+\mleft(\frac{\beta-\gamma}{\alpha}\mright)}
 +  \frac{\mleft(\sqrt{2}\co\mright)^{\frac\gamma\alpha}\cth s^{\gamma}}{1-s^{-\gamma}}k^{\rho +  \frac{\gamma\sigma}\alpha}\eps^{-\frac\gamma\alpha}.\label{eq:gammagtr2}
\eeq
If $\gamma < \beta,$ then by \cref{lem:sumbound} \eqref{eq:complexitymidway2} becomes
\beq
2 \ct\cth \Cppw^{\beta-\gamma}
\frac{\mleft(\sqrt{2}\co\mright)^{\mleft(\frac{\gamma-\beta}{\alpha}\mright)}\Cppw^{\mleft(\gamma-\beta\mright)}}{\mleft(1-s^{\mleft(\frac{\beta-\gamma}2\mright)}\mright)^{2}}k^{\tau + \rho+\mleft(\gamma-\beta\mright)\frac\sigma\alpha}\eps^{-2+\mleft(\frac{\beta-\gamma}{\alpha}\mright)}
 +  \frac{\mleft(\sqrt{2}\co\mright)^{\frac\gamma\alpha}\cth s^{\gamma}}{1-s^{-\gamma}}k^{\rho +  \frac{\gamma\sigma}\alpha}\eps^{-\frac\gamma\alpha},\label{eq:gammaless2}
\eeq
the only difference from \eqref{eq:gammagtr2} being the loss of the $s^{\gamma-\beta}$ term.

Removing all the terms that are not of interest from \eqref{eq:gammaequal2}, \eqref{eq:gammagtr2}, and \eqref{eq:gammaless2}, we obtain \eqref{eq:mlmchheq2} and \eqref{eq:mlmchhoth2}.
\epf


\subsection{The lemma in generality}

\ble\label{lem:sumboundnew}
If $L$ is given by
\beq\label{eq:Ldefgen}
L = \ceil{\Lconst\log_{s}\mleft( \func \eps^{-1}\mright)},
\eeq
for some $\func > 0,$ then, for $s>1$ and $\delta \in \RR,$ we have the bound
\beq\label{eq:sumboundgen}
\sum_{l=0}^{L} s^{\delta l} \leq
\begin{cases}
L+1 & \tif \delta = 0,\\
\frac{s^{\delta}}{1-s^{-\delta}}\func^{\delta\Lconst}\eps^{-\delta\Lconst} &\tif \delta >0\\
\frac{1}{1-s^{-\delta}}\func^{\delta\Lconst}\eps^{-\delta\Lconst}&\tif \delta < 0
\end{cases}
\eeq
\beq\label{eq:sumboundLmo}
\sum_{l=0}^{L} s^{\delta l} \leq
\begin{cases}
L & \tif \delta = 0,\\
\frac{s^{\delta}}{1-s^{-\delta}}\func^{\delta\Lconst}\eps^{-\delta\Lconst} &\tif \delta >0\\
\frac{1}{1-s^{-\delta}}\func^{\delta\Lconst}\eps^{-\delta\Lconst}&\tif \delta < 0
\end{cases}
\eeq
\optodo{Tidy}
\ele

\bpf[Proof of \cref{lem:sumboundnew}]
The proof follows that in \cite{ClGiScTe:11}. We first observe that, since $L$ is given by \eqref{eq:Ldefgen}, it follows that
\beq\label{eq:Lboundsgen}
\Lconst\log_s\mleft(\func \eps^{-1}\mright) \leq L < \Lconst\log_s\mleft(\func \eps^{-1}\mright) + 1.
\eeq
Rearranging \eqref{eq:Lboundsgen}, we obtain the bounds
\beq\label{eq:saLboundsgen}
\mleft( \func\eps^{-1}\mright)^{\alpha \Lconst} \leq s^{\alpha L} < \mleft( \func\eps^{-1}\mright)^{\alpha \Lconst}s^\alpha.
\eeq
If $\delta > 0,$ then we use the right-hand bound in \eqref{eq:saLboundsgen} to obtain
\beq\label{eq:sdLposgen}
s^{\delta L} < \func^{\delta\Lconst}\eps^{-\delta\Lconst}s^{\delta},
\eeq
and if $\delta < 0,$ we use the left-hand bound in \eqref{eq:saLboundsgen} to obtain
\beq\label{eq:sdLneggen}
s^{\delta L} \leq \func^{\delta\Lconst}\eps^{-\delta\Lconst}.
\eeq
We now observe that, for $\delta \neq 0,$
\begin{align}
\sum_{l=0}^L s^{\delta l} &= \frac{s^{\delta\mleft(L+1\mright)} -1}{s^{\delta}-1}\nonumber\\
&= \frac{s^{\delta L} - s^{-\delta}}{1-s^{-\delta}}\nonumber\\
&\leq \frac{s^{\delta L}}{1-s^{-\delta}},\label{eq:ssumboundgen}
\end{align}
since $s^{-\delta} > 0,$ as $s >0.$ Combining \eqref{eq:ssumboundgen} with \eqref{eq:sdLposgen} and \eqref{eq:sdLneggen}, we obtain \eqref{eq:sumboundgen} in the cases $\delta \neq 0.$ The case $\delta=0$ is straightforward.



\paragraph{For sum up to $L-1$}
The proof follows that in \cite{ClGiScTe:11}. We first observe that, since $L$ is given by \eqref{eq:Ldefgen}, it follows that
\beq\label{eq:Lmobounds}
\Lconst\log_s\mleft(\func \eps^{-1}\mright)-1 \leq L-1 < \Lconst\log_s\mleft(\func \eps^{-1}\mright).
\eeq
Rearranging \eqref{eq:Lmobounds}, we obtain the bounds
\beq\label{eq:saLmobounds}
\mleft( \func\eps^{-1}\mright)^{\alpha \Lconst}s^{-\alpha} \leq s^{\alpha (L-1)} < \mleft( \func\eps^{-1}\mright)^{\alpha \Lconst}.
\eeq
If $\delta > 0,$ then we use the right-hand bound in \eqref{eq:saLmobounds} to obtain
\beq\label{eq:sdLmopos}
s^{\delta (L-1)} < \func^{\delta\Lconst}\eps^{-\delta\Lconst}
\eeq
and if $\delta < 0,$ we use the left-hand bound in \eqref{eq:saLmobounds} to obtain
\beq\label{eq:sdLmoneg}
s^{\delta (L-1)} \leq \func^{\delta\Lconst}\eps^{-\delta\Lconst}s^{-\delta}.
\eeq
We now observe that, for $\delta \neq 0,$
\begin{align}
\sum_{l=0}^{L-1} s^{\delta l} &= \frac{s^{\delta\mleft(L\mright)} -1}{s^{\delta}-1}\nonumber\\
&= \frac{s^{\delta (L-1)} - s^{-\delta}}{1-s^{-\delta}}\nonumber\\
&\leq \frac{s^{\delta (L-1)}}{1-s^{-\delta}},\label{eq:ssumboundLmo}
\end{align}
since $s^{-\delta} > 0,$ as $s >0.$ Combining \eqref{eq:ssumboundLmo} with \eqref{eq:sdLmopos} and \eqref{eq:sdLmoneg}, we obtain \eqref{eq:sumboundLmo} in the cases $\delta \neq 0.$ The case $\delta=0$ is straightforward.
\epf
